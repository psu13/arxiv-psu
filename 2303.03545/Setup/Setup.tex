\section{Experimental setup}\label{sec:setup}
The core of the setup is a type-I superconducting trap with a magnetic particle levitated therein, as shown in Fig.~\ref{fig1:setup}b.
The trap is made of tantalum with a critical temperature of $T_c = \SI{4.48}{K}$. We perform the experiment at temperatures below \SI{100}{mK}.
The trap has an elliptical shape (\SI{4.5}{mm}$~\times$~\SI{3.5}{mm}, with the height from bottom of the trap to the coil being \SI{4.7}{mm}) to confine the modes of the levitated magnetic particle to the axial system of the trap.
The particle is composed of a set of three $\SI{0.25}{mm} \times \SI{0.25}{mm} \times \SI{0.25}{mm}$ Nd$_{2}$Fe$_{14}$B magnets that are magnetically attached North-to-South as also shown in Fig.~\ref{fig1:setup}b, and a spherical glass bead, \SI{0.25}{mm} radius, that is attached to the middle magnet using stycast. This bead is added to break the rotational symmetry of the zeppelin around the x-axis (angle $\gamma$).
Typical remnant magnetisation of Nd$_{2}$Fe$_{14}$B magnets is in the order of \SI{1.4}{T}.
The estimated mass of the full particle, as depicted in Fig.~\ref{fig1:setup}d, is \SI{0.43}{mg}.
Using the infinite plane approximation of Vinante et al. \cite{vinante2020}, we calculate an expected z-mode frequency of \SI{27}{Hz} in this geometry.

The motion of the particle results in a change of flux through a loop at the top of the trap (the pick-up loop), which is detected using a two-stage biased SQUID coupled inductively to the pick-up loop. The loop is positioned off-center so that the symmetry is broken, and all modes couple to the loop.
A third loop positioned halfway between the SQUID input loop and the pick-up loop is coupled inductively to a calibration loop. This transformer is used to calibrate the energy coupling $\beta^2$ between the detection circuit and the degrees of motion of the zeppelin, providing calibrated motion of the zeppelin from the measured flux signal.
This procedure is further described in App.~\ref{app:calibration}.

The set-up is hung by springs from a multi-stage mass spring system to shield the experiment from external vibrations, both vertical and lateral.
The bottom three masses (one aluminum, two copper) are similar in weight to the experimental setup, with a lowest resonance frequency at \SI{0.9}{Hz}.
Above that is a millikelvin mass spring system with a lowest resonance frequency of \SI{4.8}{Hz}.
We refer to Ref.~\cite{wit2019} for more details on a near identical mass-spring system and its performance in a similar dilution refrigerator.
This combination is hung from the 1K plate by a long spring. 
Thermalisation of the experiment is provided by a flattened silver wire, which is mechanically soft while providing a good thermal link.
This entire system is depicted in Fig.~\ref{fig1:setup}a. 

The cryostat as a whole is rigidly attached to a 25 metric ton concrete block, which is again placed on pneumatic dampers to limit vibrations coupling in from the building. The pulse tube cooler and the vacuum pumps for the circulation of the mixture are rigidly attached to the building through a second frame, and attached to the cryostat only by edge welded bellows and soft copper braiding to further limit external excitations from reaching the particle.

To demonstrate the force sensitivity of the system and as a proof of concept for gravitational coupling in levitated magnetic systems, we utilized an electrically driven wheel with a set of three \SI{2.45}{kg} brass masses, placed equally spaced along the outer rim.
This wheel was used to create a time dependent gravitational gradient at the resonant frequency of a selected mode of the zeppelin, in an effort to drive the motion gravitationally.
The frequency of the masses was read out optically using a laser and photodiode, in which the masses act as a mechanical shutter.

\begin{figure}[ht]
    \centering
    \includegraphics[width=0.8\textwidth]{Setup/SetupSchematic.pdf}%,bb=0 0 6000 6000]
    \caption{Schematic depiction of the experimental setup. \textbf{\emph{A}}: Multi-stage mass spring system to isolate from external vibrations, as discussed in the text. Electromagnetic shielding of the trap is discussed in appendix~\ref{app:photo}.
    \textbf{\emph{B}}: Conventions for degrees of freedom adopted from Vinante et al.~\cite{vinante2020}. Detection by SQUID as discussed in the text.  Calibration loop as discussed in the text and App.~\ref{app:calibration}. 
    \textbf{\emph{C}}: An image of: the dilution refrigerator used for the experiments, including the multi-stage mass spring system. 
    \textbf{\emph{D}}: The magnetic particle, composed of three $\SI{0.25}{mm} \times \SI{0.25}{mm} \times \SI{0.25}{mm}$ Nd$_{2}$Fe$_{14}$B magnets (SuperMagnetMan, C0005-10) magnetically attached end-to-end and a single spherical glass bead with a \SI{0.25}{mm} radius attached using Stycast to the middle of the magnets, which is used to break the symmetry of the $\gamma$ mode.
    \textbf{\emph{E}}: The trap, as placed in the aluminium holder without the shielding cylinder. The aluminium foil envelope provides additional electro-magnetic shielding between the calibration transformer and the pick-up loop.
    Further details and images of the setup are shown in Appendix~\ref{app:photo}.
    }\label{fig1:setup}
\end{figure}