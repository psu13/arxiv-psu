\section{Conclusions}\label{sec:conclusions}
We demonstrate the detection of a \SI{30}{aN} gravitational signal at \SI{27}{Hz} and a damping linewidth as low as $\gamma/2\pi = \SI{2.9}{\mu Hz}$, with a \SI{0.43}{mg} test mass, paving the way for future experiments in which both source and test mass are in this regime. This work could be used to derive a more stringent bound on dissipative collapse models. Furthermore it provides a promising platform to test for possible deviations from inverse square force laws and fifth force models~\cite{Blakemore2021, smullin2005}, theories of modified Newtonian dynamics~\cite{milgrom1983, bekenstein2004} and other extensions of the standard model~\cite{carney2021}.

By ensuring that the pick-up loop is placed off-centre with respect to the trap and by breaking the rotational symmetry in \textgamma\ we demonstrate detection of all six mechanical modes, in comparison to earlier work. As we will discuss in future work, this is critical to the stability of the mode under test due to non-linear mixing between the different modes.

With a mode temperature of 3 kelvin compared to an operating temperature of 30 millikelvin, we are currently not yet thermally limited.
It would require another 20 dB of vibration isolation to reach thermal motion.

By using a second particle in a different trap as source mass, or a similar construction, this work paves the way towards easily scaleable measurements of gravitational coupling in the hertz regime and with source masses at Planck mass level, ultimately allowing for testing gravity in a yet unexplored low-mass regime and pushing into the quantum controlled domain. Coupling of the detection SQUIDs in this scheme to an superconducting LC circuit would provide a means of inserting single microwave photons, providing access to the toolbox of quantum state manipulation. This would further extend this work towards truly macroscopic superposition measurements and gravitationally-induced entanglement.

