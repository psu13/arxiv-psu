Einstein’s theory of general relativity (GR), our widely accepted theory of gravity, has seen different experimental confirmations~\cite{Einstein1916,walsh1979} by observing massive astronomical objects and their dynamics, most recently by the direct observation of gravitational waves from the merger of two black holes~\cite{abbott2016} and the imaging of a black hole by the event horizon telescope~\cite{akiyama2019}, as well as dedicated satellite missions for testing the basic principle of GR --- the equivalence principle~\cite{touboul2017} and frame dragging effects~\cite{everitt2011}. Laboratory experiments have been continuously increasing the sensitivity of gravity phenomena, including general relativistic effects in atom clocks and atom interferometers~\cite{bothwell2022, asenbaum2017}, tests of the equivalence principle~\cite{rosi2017, asenbaum2020}, precision measurements of Newton's constant~\cite{rosi2014, quinn2013} and tests of the validity of Newton's law at micrometre-scale distances~\cite{geraci2008, tan2020}. 

However, gravity has never been tested for small masses and on the level of the Planck mass. Measurements of gravity from classical sources in laboratory table-top settings is contrasted by an increasing interest to study gravitational phenomena originating from quantum states of source masses, for example, in the form of the gravitational field generated by a quantum superposition state~\cite{bronstein2012republication, rickles2011role, bose2017, marletto2017, al2018optomechanical}. The effort ultimately aims at directly probing the interplay between quantum mechanics and general relativity in table-top experiments. Because quantum coherence is easily lost for increasing system size, it is important to isolate gravity as a coupling force for as small objects as possible, which in turn means to measure gravitational forces and interactions extremely precisely. 

At the same time, massive quantum sensors are especially suited for tests in a regime with appreciable gravitational influences, which is favourable in probing physical models of the wave function collapse~\cite{diosi2015, vinante2016,bassi2013}, namely those that feature the system mass explicitly, such as the continuous spontaneous localization (CSL) model~\cite{ghirardi1990} and the Di\'{o}si-Penrose model of gravitationally-induced collapse~\cite{diosi1987,penrose1996,oosterkamp2013}.


An emerging technology for ultra-sensitive sensing is based on levitated mechanical systems. These can be used for the mechanical sensing of very weak forces and to probe quantum physics at increasing scales of mass (and space). In optical levitation schemes, the heating from trapping lasers is the most prominent source of noise. Worse, in any quantum experiment, they provide a significant source of decoherence, greatly increasing the difficulty of creating macroscopic quantum states. In magnetically levitated systems, this pathway of decoherence is largely removed~\cite{Romero2021}.


The extremely low damping of magnetic systems, combined with their relatively high mass and operation in low noise cryogenic environments, makes them well suited for mesoscopic probes of quantum mechanics and could provide a test to possible limits of the applicability of quantum mechanics to the macroscopic world~\cite{leggett2002,arndt2014}.
Van Waarde et al.~\cite{Waarde2016} and Vinante et al.~\cite{vinante2020} have previously realised such magnetic levitation of sub-milligram particles, in which the motional state of the particle is read out by means of superconducting quantum interference device (SQUID) detection.

%The extremely soft coupling and resulting high isolation from the environment are critical to any quantum mechanical experiment at low frequency, since even at millikelvin temperatures the bath temperature will otherwise dominate the decoherence time of any quantum state. 


%Their motion can be detected using SQUID detection of the change in flux in a nearby coil due to the particle’s motion at frequencies in the range of \SIrange{1}{100}{Hz}.
%Because the particles are levitated above a type-I superconductor, the experiments are performed at cryogenic temperatures. Together with the low damping this results in a very low force noise. These superconducting traps also provide excellent shielding from magnetic and electric forces, further shielding the probe from non-gravitational sources of noise. This, then, seems to be a promising route for combining millimeter sized test masses with detection in the hertz regime. 

In a recent publication, Westphal et al. have demonstrated gravitational coupling between two \SI{90}{mg}, \SI{1}{mm} radius, gold spheres, achieved off resonance at millihertz frequencies in a torsion balance-type geometry~\cite{aspelmeyer2021}. Recent work by Brack et al. ~\cite{brack2022} has shown the dynamical detection of gravitational coupling between two parallel beams of a  meter in size in the hertz regime. In this paper we present work with a \SI{2.4}{kg} source mass and a magnetically levitated sub-milligram test mass, giving a coupling of \SIrange{10}{30}{aN} with a force noise of $\SI{0.5}{fN/\sqrt{Hz}}$. This work provides an intermediate step towards an experiment where a small test mass senses the gravity sourced by a small source mass.