\section{Background}
\label{section:background}



Interferometry is the practice of using a two or more element radio telescope array to observe astronomical sources.  A single telescope can reconstruct an image with  the angular resolution

\begin{equation}\theta = \frac{\lambda}{D}\end{equation}

\noindent{where $\lambda$ is the wavelength of light emitting from the source we are trying to observe and $D$ is the diameter of the telescope. Thus, to increase resolution we must increase the diameter of our telescope. Building telescopes with diameters large enough to image distant celestial objects is not only difficult, but it often impossible. However, by collecting data from an array of telescopes it is possible to approximate a single telescope with an effective diameter the size of the largest distance between telescopes in the array. These methods are referred to as very long baseline imaging (VLBI). }

In this section we will give a brief introduction to VLBI as well as an overview of current methods used to reconstruct images from VLBI data. 

\subsection{Brief Introduction to VLBI}

A single antenna cannot distinguish between the signals originating from different point sources in the sky, as it is measuring a combined voltage from all sources. However, by using two or more antennas, it is possible to separate out the signals. Suppose we are given two antennas that are separated by a baseline of length $B$. Both antennas are pointed at a source in the sky in the direction of the unit vector $\hat{s}$ and receive the same signal from the source. However, because the antennas are separated by a distance $B$ they will not receive the signals at the same time. Since we know that the source is very far away from each antenna we can assume that the signal received by the antennas are plane waves. Thus, we can approximate the difference in the distance traveled to the antennas to be 

\begin{figure}[h!]
\centering
\subfigure[]{\includegraphics[width=0.35\linewidth]
		{images/background/interferometry/multiplesetup}}
		\qquad
\subfigure[]{\includegraphics[width=0.4\linewidth]
		{images/eccv/earthRotation.eps}}
                    \caption{}
 \label{fig:introinterferometry}
\end{figure}


\begin{equation} \Delta d  = B \cdot \hat{s} \label{eq:chdist} \end{equation}

\noindent{When there are multiple sources of radio waves in the sky, the time delay between the antennas for each point source is different (Figure~\ref{fig:introinterferometry}a). Using this geometry, along with the speed of light, we can calculate the time delay between a signal reaching the antennas, which in turn can then be used to reconstruct an image of the distribution of light in the sky!}

The \textit{Van Cittert-Zernike Theorem}  states that  the time-averaged cross-correlation of the measured signals from the two antennas, $i$ and $j$, seperated by baseline, $B$, for a single wavelength of coherent light, $\lambda$, can be approximated by the equation 

\begin{equation} \Gamma_{i,j} \approx \int_\ell{\int_{m} {e^{-i 2 \pi  (u\ell + vm) }} I_{\lambda}(\ell,m) dl} dm  \label{eq:visibility} \end{equation} 

\noindent{where $ I_{\lambda}(\ell,m)$ is the light of wavelength $\lambda$ emitting from the celestial sphere from the direction  $\hat{s} = (\ell, m, \sqrt{1 - \ell^2 - m^2} )$ in the sky.}

\begin{equation} \frac{B}{\lambda} = (u, v) \end{equation}

\noindent{However, Equation~\ref{eq:visibility} is just the Fouler transform of $I_{\lambda}(\ell,m)$! Thus, the time-averaged cross-correlation of the measured signals from the two antennas gives a single Fourier pair in the frequency domain at position $(u,v)$ on the 2D frequency plane.  By inspecting Equation~\ref{eq:visibility} it is apparent that as you increase the baseline between the antennas, thus increasing the values of $u$ and $v$, the frequency corresponding to the Fourier pair increases. This means that the resolution of the features that you can pick up on the sky is dependent on how far apart you place the two antennas. }

\paragraph{Earth Rotation Synthesis}

For every pair of telescopes, for every wavelength $\lambda$, at a given time we can take a single measurement in the $(u,v)$ plane. However, as the Earth rotates, the angle at which the telescopes must point towards the source, $\hat{s} = (\ell, m, \sqrt{1 - \ell^2 - m^2} )$, changes. Therefore, although the baseline $B=(u,v)$ between the antennas remains the same, the result of Equation~\ref{eq:visibility} is different at different points in time. If you assume your image has not changed very much over the course of the day, this corresponds to taking different samples from the $(u,v)$ frequency plane. These samples carve out elliptical paths in the $(u,v)$ plane such as can be seen in Figure~\ref{fig:introinterferometry}b.  

\noindent{ {\bf Phase Closure} In all equations and derivations above it was assumed that light was moving through a vacuum. However, in reality,  inhomogeneity in the atmosphere causes the light to move at different velocities towards each antenna. This propagation delay has a huge effect on the phase of measurements, and renders the phase unusable for image reconstructions. }

Although absolute phase measurements cannot be used, a property called phase closure allows us to still obtain some information from these phases. It can be shown that the phase error for a single visibility measurement, $\Gamma_{i,j}$, is 

\begin{equation} \Gamma_{i,j}^{\mbox{\tiny{measured}}} = e^{i(\phi_i - \phi_j)}\Gamma_{i,j}^{\mbox{\tiny{true}}} \end{equation}

\noindent{where $\phi_i$ and $\phi_j$ are the phase delays to telescopes $i$ and $j$ respectively. Thus, by multiplying the visibilities from 3 different telescopes, we are able to get a term that is invariant to the atmosphere (Equation~\ref{eq:phaseclosure}). We refer to this property in the remainder of the paper as phase closure.  }

\begin{align}  
\notag \Gamma^{\mbox{\tiny{measured}}}_{i,j}\Gamma^{\mbox{\tiny{measured}}}_{j,k}\Gamma^{\mbox{\tiny{measured}}}_{k,i} = e^{i(\phi_i-\phi_j)}\Gamma^{\mbox{\tiny{true}}}_{i,j}e^{i(\phi_j-\phi_k)}\Gamma^{\mbox{\tiny{true}}}_{j,k}e^{i(\phi_k-\phi_i)}\Gamma^{\mbox{\tiny{true}}}_{k,i} \\
= e^{i(\phi_i-\phi_j+\phi_j-\phi_k + \phi_k-\phi_i)}\Gamma^{\mbox{\tiny{true}}}_{i,j}\Gamma^{\mbox{\tiny{true}}}_{j,k}\Gamma^{\mbox{\tiny{true}}}_{k,i} =\Gamma^{\mbox{\tiny{true}}}_{i,j}\Gamma^{\mbox{\tiny{true}}}_{j,k}\Gamma^{\mbox{\tiny{true}}}_{k,i} 
\label{eq:phaseclosure}
 \end{align}




\subsection{Previous Work}