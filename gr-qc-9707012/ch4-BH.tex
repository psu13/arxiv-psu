
\chapter{Rotating Black Holes}
%\section{Rotating Black Holes}

\section{Uniqueness Theorems}\index{uniqueness theorems}

\subsection{Spacetime Symmetries}

\paragraph{Definition}  An asymptotically flat spacetime is 
\emphin{stationary} if and only if there exists a Killing vector field, $k$,
that is timelike near $\infty$ (where we may normalize it s.t. $k^2\to -1$).

i.e. outside a possible horizon, $k=\partial/\partial t$ where $t$ is a 
time coordinate.  The general stationary metric in these coordinates is 
therefore
\be
ds^2= g_{00}(\vec{x})dt^2 + 2g_{0i}(\vec{x})dt\,dx^i+g_{ij}(\vec{x})dx^i\,dx^j
\ee
A stationary spacetime is \emphin{static} at least near $\infty$ if it 
is also \emph{invariant under time-reversal}. This requires $g_{0i}=0$, so the
general static metric can be written as
\be
ds^2= g_{00}(\vec{x})dt^2+g_{ij}(\vec{x})dx^i\,dx^j
\ee
for a static spacetime outside a possible horizon.

\paragraph{Definition}  An asymptotically flat spacetime is 
\emphin{axisymmetric} if there exists a Killing vector field $m$ (an
`axial' Killing vector field) that is spacelike near $\infty$ and for which
\emph{all orbits are closed}.  

We can choose coordinates such that
\be
m=\pd{}{\phi}
\ee
where $\phi$ is a coordinate \emph{identified modulo $2\pi$}, such that
$m^2/r^2\rightarrow 1$ as $r\rightarrow\infty$. Thus, as for $k$, there is
a natural choice of normalization for an axial Killing vector field in an
asymptotically flat spacetime.

%\subsection{Uniqueness Theorems}\index{uniqueness theorems}

\paragraph{Birkhoff's theorem}\index{Birkhoff's theorem} says that any spherically
symmetric vacuum solution is static, which effectively implies that it must be
Schwarzschild.  A generalization of this theorem to the Einstein-Maxwell
system shows that the only spherically symmetric solution is RN. \\

But suppose we know only that the metric exterior to a star is static.  
Unfortunately static $\not\Rightarrow$ spherical symmetry.  However, if the
`star' is actually a black hole we have:

\paragraph{Israel's theorem}\index{Israel's theorem} If $(M,g)$ is an 
asymptotically-flat, \emph{static}, \emph{vacuum} spacetime that is non-singular
on and outside an event horizon, then $(M,g)$ is Schwarzschild. \\

Even more remarkable is the:

\paragraph{Carter-Robinson theorem}\index{Carter-Robinson theorem} If 
$(M,g)$ is an asymptotically-flat \emph{stationary} and \emph{axi-symmetric}
vacuum spacetime that is non-singular on and outside an event horizon, then
$(M,g)$ is a member of the two-parameter Kerr family (given later).  The
parameters are the mass $M$ an the angular momentum $J$. \\

The assumption of axi-symmetry has since been shown to be unnecessary, i.e. 
\emph{for black holes, stationarity $\Rightarrow$ axisymmetry} (Hawking, Wald).
\\

Stationarity $\Leftrightarrow$ equilibrium, so we expect the final state of 
gravitational collapse to be a stationary spacetime.  The uniqueness theorems
say that if the collapse is to a black hole then this spacetime is uniquely
determined by its mass and angular momentum (cf. state of matter in thermal
equilibrium).  Thus, \emph{all multipole moments of the gravitational field are
radiated away} in the collapse to a black hole, except the monopole and dipole
moments (which can't be radiated away because the graviton\index{graviton} has
spin 2). \\

These theorems can be generalized to `vacuum' Einstein-Maxwell equations.  
The result is that a stationary black hole spacetimes must belong to the
3-parameter \emphin{Kerr-Newman family}.  In \emphin{Boyer-Linquist
coordinates}  the KN metric is
\bebox{
\begin{array}{rcl}
ds^2 & = & -\bgfrac{ \left(\Delta -a^2\sin^2\theta\right)}{\Sigma}dt^2 - 2 a 
\sin^2\theta \bgfrac{ \left(r^2+a^2-\Delta\right)}{\Sigma}dt\,d\phi \\
 & & +\left( \bgfrac{ \left(r^2+a^2\right)^2-\Delta a^2\sin^2\theta}{\Sigma}
\right)\sin^2\theta d\phi^2 +\bgfrac{\Sigma}{\Delta}dr^2+\Sigma d\theta^2 
\end{array} }
where
\bebox{
\begin{array}{rcl}
\Sigma & = & r^2+a^2\cos^2\theta \\
\Delta & = & r^2-2Mr+a^2+e^2 \end{array}}
The three parameters are $M$, $a$, and $e$.  It can be shown that
\be
a=\frac{J}{M}
\ee
where $J$ is the total angular momentum, while
\be
e = \sqrt{ Q^2+P^2}
\ee
where $Q$ and $P$ are the electric and magnetic (monopole) charges, 
respectively.  The Maxwell 1-form of the KN solution is 
\bebox{
A= \bgfrac{ Qr\left(dt-a\sin^2\theta d\phi\right)-P\cos\theta
\left[a dt-\left(r^2+a^2\right)d\phi\right] }{\Sigma} }
\subsubsection{Remarks}
\newcounter{KNremarks}
\begin{list}{(\roman{KNremarks})}
{\usecounter{KNremarks}}
\item When $a=0$ the KN solution reduces to the RN solution.
\item Taking $\phi \to -\phi$ effectively changes the sign of $a$, so we 
may choose $a \ge 0$ without loss of generality.
\item  The KN solution has the discrete isometry
\be
t\to -t, \quad \phi \to -\phi
\ee
\end{list}

\section{The Kerr Solution}

This is obtained from KN by setting $e=0$.  Then
\bea
\Delta & = & r^2-2Mr+a^2 \\
(\Sigma & = & r^2+a^2\cos^2\theta ) 
\eea
The Kerr metric\index{Kerr metric} is important astrophysically since it 
is a good \emph{approximation} to the metric of a rotating star at large
distances where all multipole moments except $l=0$ and $l=1$ are unimportant. 
The only known solution of Einstein's equations for which Kerr is \emph{exact}
for $r>R$ is the Kerr solution itself (for which $T_{\mu\nu}=0$), i.e. it has
not been matched to any known non-vacuum solution that could represent the
interior of a star, in contrast to the Schwarzschild solution which is
guaranteed by Birkhoff's theorem to be the exact exterior spacetime that
matches on to the interior solution for any spherically symmetric star. \\

The Kerr metric in BL coordinates has \emph{coordinate} singularities at
\newcounter{Kerrsing}
\begin{list}{(\alph{Kerrsing})}
{\usecounter{Kerrsing}}
\item $\theta=0$ (i.e on axis of symmetry)
\item $\Delta =0$
\end{list}
Write
\be
\Delta=\left(r-r_+\right)\left(r-r_-\right) 
\ee
where
\be
r_{\pm}=M\pm \sqrt{M^2-a^2}
\ee
There are 3 cases to consider
\newcounter{Kerrcases}
\begin{list}{(\roman{Kerrcases})}
{\usecounter{Kerrcases}}
\item \ul{$M^2<a^2$}:  $r_{\pm}$ are complex, so $\Delta$ has no real zeroes, 
and there are no coordinate singularities there.  The metric still has a
coordinate singularity at $\theta=0$.  More significantly, it has a
\emph{curvature singularity} at $\Sigma=0$, i.e.
\be
r=0,\quad \theta = \pi/2
\ee
The nature of this singularity is best seen in Kerr-Schild 
coordinates\index{Kerr-Schild coordinates} $(\tilde{t},x,y,z)$ (which also
removes the coordinate singularity at $\theta=0$).  These are defined by
\bea
x+iy & = & (r+ia)\sin\theta \exp \left[i\int\left(d\phi+
\frac{a}{\Delta}dr\right)\right] \\
z & = & r\cos\theta \\
\tilde{t} & = & \int \left(dt+\frac{r^2+a^2}{\Delta}dr\right)-r
\eea
which implies that $r=r(x,y,z)$ is given implicitly by
\be
r^4-\left(x^2+y^2+z^2-a^2\right)r^2-a^2z^2=0
\ee
In these coordinates the metric is
\bea
 ds^2 &  = & -d\tilde{t}^2+dx^2+dy^2+dz^2  \\
 &  + & \frac{2Mr^3}{r^4+a^2z^2} \left[ \frac{ r(x\,dx+y\,dy)-
a(x\,dy-y\,dx) }{r^2+a^2} + \frac{zdz}{r}+d\tilde{t}\right]^2  \nn
\eea
which shows that the spacetime is flat (Minkowski) when $M=0$.  \\

The surfaces of constant $\tilde{t},r$ are confocal ellipsoids which 
degenerate at $r=0$ to the disc $z=0,\;x^2+y^2\le a^2$.
\begin{center}\input{p83-1.pictex}\end{center}
$\theta=\pi/2$ corresponds to the boundary of the disc at $x^2+y^2=a^2$ 
so the curvature singularity occurs on the boundary of the disc, i.e. on
the `ring'
\be
x^2+y^2=a^2,\quad z=0
\ee
There is no reason to restrict $r$ to be positive.  The spacetime can be 
analytically continued through the disc to another asymptotically flat region
with $r<0$.

%\end{list} %(Kerrcases)

\subsubsection{Causal structure}  

Because we now have only axial symmetry we really need a 3-dim spacetime 
diagram to encode the causal structure, but the $\theta=0,\pi/2$ submanifolds
are \emphin{totally-geodesic}, i.e. a geodesic that is initially tangent to the
submanifold remains tangent to it, so we can draw 2-dim CP diagrams for them.
\begin{center}\input{p84-1.pictex}\end{center}
For $\theta=\pi/2$ each point in the diagram represents a circle 
$(0\le\phi< 2\pi)$.  Each ingoing radial geodesic hits the ring singularity at
$r=0$, which is clearly \emph{naked}.  For $\theta=0$ we are considering only
geodesics on the axis of symmetry.  Ingoing radial null geodesics pass through
the disc at $r=0$ into the other region with $r<0$.  We can summarize both
diagrams by the single one.
\begin{center}\input{p84-2.pictex}\end{center}
The spacetime is unphysical for another reason.  Consider the norm of the 
Killing vector field $m=\partial/\partial\phi$:
\be
m^2= g_{\phi\phi} =a^2\sin^2\theta\left(1+\frac{r^2}{a^2}\right)+
\frac{Ma^2}{r}\left( \frac{2\sin^4\theta}{1+\frac{a^2}{r^2}\cos^2\theta} \right)
\ee
Let $r/a=\delta$ (small) and consider $\theta=\pi/2+\delta$.  Then
\bea
m^2 & = & a^2+\frac{Ma}{\delta}+\mathcal{O}(\delta), \quad \mbox{for 
$\delta \ll 1$}  \\
 & < & 0 \quad \mbox{for sufficiently small negative $\delta$} \nn
\eea
So \emph{$m$ becomes timelike near the ring-singularity on the $r<0$ branch}.  
But the orbits of $m$ are \emph{closed}, so the spacetime admits closed timelike
curves (CTCs).  This constitute a \emphin{global violation of causality}. \\

Moreover because of the absence of a horizon these CTCs may be deformed to 
pass through \emph{any point} of the spacetime (Carter).  They also miss the
singularity by a distance $\sim M$, for $M\sim a$, and $M$ can be arbitrarily
large.  Since the ring singularity would be naked for $M^2<a^2$, then even if
the white hole region is replaced by a collapsing star, we can invoke cosmic
censorship to rule out $M^2<a^2$.

\item \ul{$M^2>a^2$}.  We still have a ring-singularity but now the metric 
(in BL coordinates) is singular at $r=r_+$ and $r=r_-$.  These are coordinate
singularities.  To see this we define new coordinates $v$ and $\chi$ by
\bea
dv & = & dt+\frac{\left(r^2-a^2\right)}{\Delta}dr \\
d\chi & = & d\phi+\frac{a}{\Delta} dr 
\eea
This yields the Kerr solution in Kerr coordinates $(v,r,\theta,\chi)$ which 
are analogous to ingoing EF for Schwarzschild:
\bebox{
\begin{array}{rcl}
ds^2 & = & -\bgfrac{ \left(\Delta-a^2\sin^2\theta\right)}{\Sigma}dv^2 +
2dv\,dr-\bgfrac{2a\sin^2\theta\left(r^2+a^2-\Delta\right)}{\Sigma} dv\,d\chi \\
 & & -2a\sin^2\theta d\chi\,dr+\bgfrac{ \left[\left(r^2+a^2\right)^2-
\Delta a^2\sin^2\theta \right]}{\Sigma} \sin^2\theta d\chi^2+\Sigma d\theta^2 
\end{array}
}
This metric is non singular when $\Delta = 0$, i.e. when $r=r_+$ or $r=r_-$.

\paragraph{Proposition}  The hypersurfaces $r=r_{\pm}$ are Killing horizons of 
the Killing vector fields
\be
\xi_{\pm}=k+\left(\frac{a}{r_{\pm}^2+a^2}\right)m
\ee
with surface gravities
\be
\kappa_{\pm} = \frac{ r_{\pm}-r_{\mp} }{2\left(r_{\pm}^2+a^2\right)}
\ee 

\paragraph{Proof}  Let $\mcN_{\pm}$ be the hypersurfaces $r=r_{\pm}$.  The 
normals are
\bea
l_{\pm} & = & \left. f_{\pm} g^{\mu r}\right|_{\mcN_{\pm}}\partial_{\mu}, 
\quad \mbox{for some non-zero functions $f_{\pm}$} \\
 & = & -\left(\frac{r_{\pm}^2+a^2}{r_{\pm}^2+a^2\cos^2\theta}\right)f_{\pm} 
\underbrace{ \left( \pd{}{v}+\frac{a}{r_{\pm}^2+a^2}\pd{}{\chi} \right)
}_{\displaystyle \xi_{\pm}} \quad \mbox{(Exercise)}
\eea
First
\be
l_{\pm}^2 \propto \left. \left(g_{vv}+\frac{2a}{r^2+a^2}g_{v\chi}+
\frac{a^2}{\left(r^2+a^2\right)^2}g_{\chi\chi}\right) \right|_{\Delta=0} = 0 
\ee
so $\mcN_{\pm}$ are null hypersurfaces.  Since 
$\left.\xi_{\pm}\right|_{\mcN_{\pm}}\propto l_{\pm}$, they are Killing horizons
of $\xi_{\pm}$.  It remains to compute $\xi_{\pm}D\xi_{\pm}^{\mu}$.  This gives
the result for $\kappa_{\pm}$ (Exercise). \\

This result can be used to find KS type coordinates that cover 4 
regions around a BK axis of each Killing horizon, and the $\theta=0$ and
$\theta=\pi/2$ CP diagram of the maximal analytic extension of 
$M^2 > a^2$ Kerr can be found. Note that the diagram can be extended infinitely
in both time directions.
\begin{center}\input{p87-1.pictex}\end{center}

\subsection{Angular Velocity of the Horizon}

The event horizon is a Killing horizon of
\be
\xi=k+\Omega_Hm
\ee
where
\be
\Omega_H = \frac{a}{r_{+}^2+a^2}= \frac{J}{2M\left[M^2+\sqrt{M^4-J^2}\right]}
\ee
In coordinates for which $k=\partial/\partial t$ and 
$m=\partial/\partial\phi$ we have that
\be
\xi^{\mu}\partial_{\mu}\left(\phi-\Omega_H t\right)=0
\ee
i.e. $\phi=\Omega_H t+\mbox{constant}$, on orbits of $\xi$, whereas $\phi$ 
is constant on orbits of $k$.  Note that $k$ is \emph{unique}.  Consider 
\be
(k+\alpha m)^2=g_{tt}+2\alpha g_{t\phi}+\alpha^2g_{\phi\phi}
\ee
As long as $g_{t\phi}$ is finite and $g_{\phi\phi}\sim r^2$ as $r\to\infty$, 
we have $(k+\alpha m)^2\sim \alpha^2r^2>0$ (if $\alpha\neq 0$) as $r\to\infty$. 
So there can be only \emph{one} Killing vector $k$ that is timelike at $\infty$
and normalized s.t. $k^2\to -1$ as $r\to \infty$.

Thus particles on orbits of $\xi$ rotate with angular velocity $\Omega_H$ 
relative to static particles, those on orbits of $k$, and hence relative to a
stationary frame at $\infty$.  Since the null geodesic generators of the horizon
follow orbits of $\xi$ the black hole is rotating with angular velocity
$\Omega_H$.

\paragraph{Lemma} $\xi\cdot k=0$ on  a Killing horizon, $\mcN$, of $\xi$.

\paragraph{Proof} 
\bea
\left. \xi\cdot k\right|_{\mcN} & = &  \left.\xi^2\right|_{\mcN}-\left. 
\Omega_H \xi \cdot m\right|_{\mcN}  \\
 & = & -\left.\Omega_H\xi\cdot m\right|_{\mcN} \quad \mbox{ (since 
$\xi^2=0$ on $\mcN$)}
\eea
Now, $\mcN$ is a fixed point set of $m$, since $m$ is Killing  (Choose 
coordinates s.t. $m=\partial/\partial\phi$.  The metric is $\phi$ independent,
so the position of the horizon is independent of $\phi$).  So $m$ must be
tangent to $\mcN$ or $l\cdot m=0$ where $l$ is normal to $\mcN$.  But
$\xi\propto l$ on $\mcN$, so $\left.\xi\cdot m\right|_{\mcN}=0$.  Hence result.

\subsubsection{Consistency checks (See Question III.3)}

$\xi^2=0$ implies that
\be
k^2+2\Omega_H m\cdot k-m^2\Omega_H = 0, \quad \mbox{on $\mcN$}
\ee
But $\xi\cdot k=0$ implies that
\be
k^2+\Omega_H m\cdot k=0, \quad \mbox{on $\mcN$}
\ee
Consistency requires 
\be
\left.D\equiv (k\cdot m)^2-k^2m^2\right|_{\mcN}=0
\ee
For Kerr, $D=\Delta\sin^2\theta =0$ on $\mcN$ $\Box$.

Also
\bea
\Omega_H & = & -\frac{k^2}{m\cdot k} = \left. -
\frac{g_{tt}}{g_{t\phi}}\right|_{\mcN} \quad \mbox{in BL coordinates} \\
 & = & \frac{-a^2\sin^2\theta}{-2a\sin^2\theta\left(r^2_++a^2\right)} \\
 & = & \frac{a}{r_+^2+a^2} \quad \Box.
\eea

\item \ul{$M^2=a^2$ Extreme Kerr}

In this case we have a \emph{degenerate} ($\kappa=0$) Killing horizon 
at $r=M$ of the Killing vector field
\be
\xi = k+\Omega_H m, \quad \Omega_H=\frac{a}{2M}
\ee 
The CP diagram is
\begin{center}\input{p90-1.pictex}\end{center}

%\paragraph{Uniqueness of $k$} 
%\be
%(k+\alpha m)^2=g_{tt}+2\alpha g_{t\phi}+\alpha^2g_{\phi\phi}
%\ee
%As long as $g_{t\phi}$ is finite and $g_{\phi\phi}\sim r^2$ as $r\to\infty$, 
%we have $(k+\alpha m)^2\sim \alpha^2r^2>0$ (if $\alpha\neq 0$) as $r\to\infty$. 
So there can be only \emph{one} Killing vector $k$ for which $k\cdot k\to -1$ as
$r\to \infty$.

\end{list} %(Kerrcases)

N.B. If you change the sign of $r$ in the Kerr metric this effectively 
changes the sign of $M$.

\section{The Ergosphere}\index{ergosphere}

Although $k$ is timelike at $\infty$ it need not be timelike everywhere 
outside the horizon.  For Kerr,
\be
k^2=g_{tt}=-\frac{ \left(\Delta-a^2\sin^2\theta\right) }
{\Sigma}=-\left( 1-\frac{ 2Mr }{r^2+a^2\cos^2\theta }\right)
\ee
so $k$ is timelike provided that
\be
r^2+a^2\cos^2\theta -2Mr > 0 
\ee
For $M^2 \gg a^2$ this implies that
\be
r>M+\sqrt{M^2-a^2\cos^2\theta}
\ee
(or $r< M-\sqrt{M^2-a^2\cos^2\theta}$, but this is not physically relevant).

The boundary of this region, i.e. the hypersurface 
\be
r=M+\sqrt{M^2-a^2\cos^2\theta}
\ee
is the \emph{ergosphere}.  The ergosphere intersects the event horizon at 
$\theta=0,\pi$, but it lies \emph{outside} the horizon for other values of
$\theta$.  Thus, \emph{$k$ can become spacelike in a region outside the event
horizon}.  This is called the \emphin{ergoregion}.
\begin{center}\input{p91-1.pictex}\end{center}

\section{The Penrose Process}\index{Penrose!process}

Suppose that a particle approaches a Kerr black hole along a geodesic.  If 
$p$ is its 4-momentum we can identify the constant of the motion
\be
E=-p\cdot k
\ee
as its energy (since $E=p^0$ at $\infty$).  Now suppose that the particle 
decays into two others, one of which falls into the hole while the other escapes
to $\infty$.
\begin{center}\input{p92-1.pictex}\end{center}
By conservation of energy
\be
E_2=E-E_1
\ee
Normally $E_1>0$ so $E_2 < E$, but in this case
\be
E_1=-p_1\cdot k
\ee
which is \emph{not necessarily positive in the ergoregion} since $k$ may 
be spacelike there.  Thus, if the decay takes place in the ergoregion we may
have $E_2>E$, so \emph{energy has been extracted from the black hole}.

\subsection{Limits to Energy Extraction}

For particles passing through the horizon at $r=r_+$ we have
\be
-p\cdot \xi \ge 0
\ee 
Since $\xi$ is future-directed null on the horizon and $p$ is 
future-directed timelike or null.  Since $\xi = k+\Omega_H m$,
\be
E-\Omega_H L \ge 0 
\ee
where $L=p\cdot m$ is the component of the particle's angular momentum in 
the direction defined by $m$ (only this component is a constant of the motion). 
Thus
\be
L \le \frac{E}{\Omega_H}
\ee
If $E$ is negative, as it is for particle \bold{1} in the Penrose process 
then $L$ is also negative, so the hole's angular momentum is reduced.  We end up
with a hole of mass $M+\delta M$ and angular momentum $J+\delta J$ where $\delta
M=E$ and $\delta J=L$ so
\be
\delta J \le \frac{\delta M}{\Omega_H} = 
\frac{ 2M\left( M^2+\sqrt{ M^4-J^2}\right) }{J} \delta M
\ee 
from formula for $\Omega_H$.  This is equivalent to (Exercise)
\be
\delta\left( M^2+\sqrt{ M^4-J^2}\right) \ge 0 
\ee
(This quantity must increase in the Penrose process).

\paragraph{Lemma}  $A=8\pi\left[M^2+\sqrt{M^4-J^2}\right]$ is the `area of 
the event horizon', of a Kerr black hole (i.e. area of intersection of $\mcH^+$
with partial Cauchy surface, e.g. area of $v=$ constant, $r=r_+$ in Kerr
coordinates (See Question III.5).

\paragraph{Corollary}  Energy extraction by Penrose process is limited by 
the requirement that $\delta A \ge 0$.  This is a special case of the second 
law of black hole mechanics.

\subsection{Super-radiance}\index{super-radiance}

The Penrose process has a close analogue in the scattering of radiation by a 
Kerr black hole.  For simplicity, consider a massless scalar field $\Phi$.  Its
stress tensor is
\be
T_{\mu\nu} = \partial_{\mu}\Phi\partial_{\nu}\Phi-
\half g_{\mu\nu}(\partial\Phi)^2
\ee
Since $D_{\mu}T^{\mu}_{\I\I \nu}=0$ we have
\be
D_{\mu}\left(T^{\mu}_{\I\I\nu}k^{\nu}\right) = T^{\mu\nu}D_{\mu}k_{\nu}=0
\ee
so we can consider
\be
j^{\mu}=-T^{\mu}_{\I\I\nu}k^{\nu} = -\partial^{\mu}\Phi k\cdot 
\partial\Phi+\half k^{\mu}(\partial\Phi)^2
\ee
as the future directed ($-k\cdot J>0$) energy flux 4-vector of $\Phi$. 
Now consider the following region, $S$, of spacetime with a null hypersurface
$\mcN \subset \mcH^+$ as one boundary.
\begin{center}\input{p94-1.pictex}\end{center}
Assume that $\partial\Phi=0$ at $i_0$.  Since $D_{\mu}j^{\mu}=0$ we have
\bea
0 & = & \int_S \dx{4}{x} \sqrt{-g}D_{\mu}j^{\mu}=\int_{\partial S} 
dS_{\mu}\,j^{\mu} \\
 & = & \int_{\Sigma_2} dS_{\mu}\,j^{\mu}-\int_{\Sigma_1} 
dS_{\mu}\,j^{\mu} -\int_{\mcN} dS_{\mu}\,j^{\mu} \\
 & = & E_2 -E_1 - \int_{\mcN} dS_{\mu}\,j^{\mu}
\eea
where $E_i$ is the energy of the scalar field on the spacelike 
hypersurface $\Sigma_i$.  The energy going through the horizon is therefore
\bea
\Delta E = E_1-E_2 & = & -\int_{\mcN}dS_{\mu}\, j^{\mu} \\
 & = & -\int dA\,dv\,\xi_{\mu}j^{\mu}, \quad \mbox{($v$ is Kerr coordinate)}
\eea
The energy flux lost/unit time (power) is therefore
\bea
P & = & -\int dA\, \xi_{\mu}j^{\mu}=\int dA\, 
(\xi\cdot\partial\Phi)(k\cdot D\Phi) \\
\lefteqn{ \mbox{(since $\xi\cdot k=0$ on horizon by previous Lemma)} } 
\nn \\
 & = & \int dA\, \left(\pd{}{v}\Phi+\Omega_H \pd{}{\chi}\Phi\right)
\left(\pd{\Phi}{v}\right)
\eea
For a wave-mode of angular-frequency $\omega$
\be
\Phi=\Phi_0 \cos\left(\omega v- \nu\chi\right), \quad \nu\in\Z \quad 
\mbox{(angular quantum no.)}
\ee
The time average power lost across the horizon is
\be
P=\half \Phi_0^2 A\omega(\omega-\nu\Omega)
\ee
where $A$ is the area of the horizon. 

$P$ is positive for most values of $\omega$, but for $\omega$ in the range
\be
0 < \omega < \nu\Omega_H
\ee
it is negative, i.e. a wave-mode with $\omega,\nu$ satisfying the 
inequality is \emph{amplified} by the black hole.

\subsubsection{Remarks}
\newcounter{superradrem}
\begin{list}{\roman{superradrem})}
{\usecounter{superradrem}}
\item Process is positive only for $\nu\neq 0$ because the amplified field 
must also take away angular momentum from the hole.

\item Process is similar to stimulated emission in atomic physics, which
suggests the possibility of a spontaneous emission effect. This can be shown to
occur in the quantum theory so any black with an ergoregion cannot be stable
quantum mechanically.

\item  We have neglected the back-reaction of $\Phi$ on the metric.  When 
corrected for back-reaction the metric can be stationary only if
$\partial\Phi/\partial\phi=0$, but then $j^{\mu}=0$ and the black hole energy
doesn't change, i.e. strictly speaking super-radiance is incompatible with
stationarity.

\end{list}



