\chapter{Example Sheets}

%% TJWP removed plain tex commands
%\magnification\magstephalf
%\font\bb=msbm10 scaled 1200
%\font\small=cmr8
%\font\little=cmmi7
%%%%%%%%%%%%%%%%%%%%%%%

%% TJWP include page numbers
%\nopagenumbers
\def \pr{\partial}
\def\pmb#1{\setbox0=\hbox{#1}%
 \kern-.025em\copy0\kern-\wd0
 \kern.05em\copy0\kern-\wd0
 \kern-.025em\raise.0433em\box0 }
\def \bE{{\pmb {${\cal E}$}}}
\noindent

%\centerline{{\bf Example Sheet 1}}
%\vskip 10pt
\section{Example Sheet 1}

\noindent
{\bf 1.} Explain why 
\vskip 0.3cm
(i) GR effects are important for neutron stars but not for white
dwarfs
\vskip 0.3cm
(ii) inverse beta-decay becomes energetically favourable for
densities higher than those in white dwarfs.
\vskip 10 pt
\noindent
{\bf 2.} Use Newtonian theory to derive the Newtonian pressure
support equation
$$
P'(r) \equiv {dP\over dr} = -{Gm\rho\over r^2}\ ,
$$
where
$$
m=4\pi \int_0^r \tilde r^2 \rho(\tilde r) d\tilde r\ ,
$$
for a spherically-symmetric and static star with pressure $P(r)$ and 
density $\rho(r)$. Show that
%% TJWP no eqalign in latex  \eqalign{
\bean
\int_0^r P(\tilde r)\tilde r^3 d\tilde r & = & 
{P(r)r^4\over 4} -{1\over 4}\int_0^r P'(\tilde r) \tilde r^4 d\tilde 
r \\
 & = & {Gm^2(r)\over 32\pi} + {P(r)r^4\over 4}\ .
\eean
Assuming that $P'\le 0$, with $P=0$ at the star's surface, show that
$$
{d\over dr}\left[\left(\int_0^r P(\tilde r)\tilde r^3 d\tilde
r\right)^{3/4}\right] \le {3\sqrt{2}\over 4} P^{3/4} r^2\ .
$$
Assuming the bound
$$
P\ {\buildrel <\over\sim}\ (\hbar c) n_e^{4/3}\ ,
$$
where $n_e(r)$ is the electron number density, show that the total
mass, $M$, of the star satisfies
$$
M\ {\buildrel <\over\sim}\ \left({hc\over
G}\right)^{3/2}\left({\mu_e\over m_N}\right)^2 
$$
where $m_N$ is the nucleon mass and $\mu_e$ is the number of
electrons per nucleon. Why is it reasonable to bound the pressure as
you have done? Compare your bound with Chandresekhar's limit.

\vskip 10 pt
\noindent
{\bf 3.} A particle orbits a Schwarzschild black hole with
non-zero angular momentum per unit mass $h$. Given that $\sigma=0$ for
a massless particle and $\sigma=1$ for a massive particle, show that
the orbit satisfies  
$$ 
{d^2 u\over d\phi^2} + u = {M\sigma \over
h^2} + 3Mu^2 
$$
where $u=1/r$ and $\phi$ is the azimuthal angle. Verify that this
equation is solved by
$$
u= {1\over 6M} + {2\omega^2\over 3M} -{2\omega^2\over M\cosh ^2
(\omega\phi)}\ ,
$$
where $\omega$ is given by
$$
4\omega^2 = \pm\sqrt{\left(1 - {12M^2\sigma\over h^2}\right)}\ .
$$
where $\sigma=1$ for a massive particle and $\sigma=0$ for a massless
particle. Interpret these orbits in terms of the effective potential.
Comment on the cases $\omega^2=1/4$, $\omega^2=1/8$ and $\omega^2=0$.

\vskip 10 pt
\noindent
{\bf 4.} A photon is emitted outward from a point P outside
a Schwarzschild black hole with radial coordinate r in the range
$2M<r<3M$. Show that if the photon is to reach infinity the angle its
initial direction makes with the radial direction (as determined by 
a stationary observer at P) cannot exceed
$$
{\rm arcsin} \sqrt{{27M^2\over r^2}\left( 1-{2M\over r}\right)}\ .
$$

\vskip 10 pt
\noindent
{\bf 5.} Show that in region II of the Kruskal manifold one may
regard $r$ as a time coordinate and introduce a new spatial
coordinate $x$ such that
$$
ds^2 = -{dr^2\over \left({2M\over r} -1\right)} +\left({2M\over
r}-1\right)dx^2 + r^2d\Omega^2\ .
$$
Hence show that {\it every} timelike
curve in region II intersects the singularity at $r=0$ within a proper
time no greater than $\pi M$. For what curves is this bound attained?
Compare your result with the time taken for the collapse of a ball of
pressure free matter of the same gravitational mass $M$. Calculate
the binding energy of such a ball of dust as a fraction of its
(conserved) rest mass.
\vskip 10 pt
\noindent
{\bf 6.} Using the map
$$
(t,x,y,z) \mapsto X= \pmatrix{t+z & x+iy\cr x-iy & t-z}\ ,
$$
show that Minkowski spacetime may be identified with the space of
Hermitian $2\times 2$ matrices $X$ with metric
$$
ds^2 = -\det (dX)\ .
$$
Using the Cayley map $X\mapsto U={1+iX\over1-iX}$, show further that
Minkowski spacetime may be identified with the space of unitary
$2\times 2$ matrices $U$ for which $\det (1+U)\ne0$. Now show that any
$2\times 2$ unitary matrix $U$ may be expressed uniquely in terms of
a real number $\tau$ and two complex numbers $\alpha$, $\beta$, as 
$$
U=e^{i\tau}\pmatrix{\alpha & \beta\cr -\bar\beta & \bar\alpha}
$$
where the parameters $(\tau,\alpha,\beta)$ satisfy $|\alpha|^2
+|\beta|^2 =1$, and are subject to the identification 
$$
(\tau,\alpha,\beta) \sim (\tau +\pi, -\alpha, -\beta)\ .
$$
Using the relation
$$
(1+U)dX = -2i dU(1+U)^{-1}\ ,
$$
deduce that 
$$
ds^2 = {1\over (\cos \tau + {\cal R}e\,\alpha)^2}\left(-d\tau^2
+|d\alpha|^2 + |d\beta|^2\right) 
$$
is the metric on Minkowski spacetime and hence conclude that the
conformal compactification of Minkowski spacetime may be identified
with the space of unitary $2\times 2$ matrices, i.e the group
$U(2)$. Explain how $U(2)$ may be identified with a portion of the
Einstein static universe $S^3\times {\bb R}$.

\vfill\eject

%\centerline {{\bf Example Sheet 2}}
%\vskip 10 pt
\section{Example Sheet 2}

\noindent
{\bf 1.} Let $\zeta$ be a Killing vector field. Prove that
$$
D_\sigma D_\mu \zeta_\nu = R_{\nu\mu\sigma}{}^\lambda \zeta_\lambda\ ,
$$
where $R_{\nu\mu\sigma\lambda}$ is the Riemann tensor, defined by
$[D_\mu,D_\nu]\, v_\rho = R_{\mu\nu\rho}{}^\sigma v_\sigma$ for
arbitrary vector field $v$.

\vskip 10 pt
\noindent
{\bf 2.}  A conformal Killing vector is one for which  
$$
({\cal L}_\xi g)_{\mu\nu} = \Omega^2 g_{\mu\nu}\ .
$$
for some non-zero function $\Omega$.
Given that $\xi$ is a Killing vector of $ds^2$, show that it is a
conformal Killing vector of the conformally-equivalent
metric $\Lambda^2 ds^2$ for arbitrary (non-vanishing) conformal factor
$\Lambda$.
 
Show that the action for a {\it massless} particle,
$$ S[x,e]={1\over2}\int d\lambda\, e^{-1}\dot x^\mu\dot x^\nu
g_{\mu\nu}(x)\ , $$ is invariant, to first order in the constant
$\alpha$, under the transformation 
$$
x^\mu \rightarrow x^\mu + \alpha \xi^\mu(x) \qquad \qquad
e \rightarrow e + {1\over4}\alpha\, e g^{\mu\nu}({\cal L}_\xi
g)_{\mu\nu}
$$
if $\xi =\xi^\mu \partial_\mu$ is a conformal Killing vector. Show
that $\xi$ is the operator corresponding to the conserved charge
implied by Noether's theorem.



\vskip 10 pt
\noindent
{\bf 3.} Show that the extreme RN metric in isotropic coordinates is
$$
ds^2 = -\left(1+{M\over\rho}\right)^{-2}dt^2 + 
\left(1+{M\over\rho}\right)^{2}\left(d\rho^2 + \rho^2 d\Omega^2\right)
\qquad (\dagger)
$$
Verify that $\rho=0$ is at infinite proper distance from any finite
$\rho$ along any curve of constant $t$. Verify also that
$|t|\rightarrow \infty$ as $\rho\rightarrow 0$ along any timelike or
null curve but that a timelike or null ingoing radial geodesic reaches
$\rho=0$ for {\it finite} affine parameter. By introducing a null
coordinate to replace $\rho$ show that $\rho=0$ is merely a coordinate
singularity and hence that the metric ($\dagger$) is geodesically
incomplete. What happens to the particles that reach $\rho=0$?
Illustrate your answers using a Penrose diagram.

\vskip 10 pt
 \noindent
{\bf 4.} The action for a particle of mass $m$ and charge $q$ is
$$
S[x,e] =\int d\lambda\,\left[{1\over2} e^{-1}\dot x^\mu\dot x^\nu
g_{\mu\nu}(x) -{1\over2}m^2 e -q\, \dot x^\mu A_\mu(x)\right]\qquad
\qquad (*) 
$$
where $A_\mu$ is the electromagnetic 4-potential. Show
that if
$$
({\cal L}_\xi A)_\mu \equiv \xi^\nu\partial_\nu A_\mu +
(\partial_\mu \xi^\nu) A_\nu =0
$$
for Killing vector $\xi$, then $S$ is invariant, to first-order in
$\xi$, under the transformation $x^\mu\rightarrow x^\mu
+\alpha\xi^\mu(x)$. Verify that the corresponding Noether charge
$$
-\xi^\mu \left(m u_\mu -q A_\mu\right)\ ,
$$ 
where $u^\mu$ is the particle's 4-velocity, is a constant of the 
motion. Verify for the Reissner-Nordstrom solution of the vacuum
Einstein-Maxwell equations, with mass $M$ and charge $Q$, that ${\cal
L}_k A =0$ for $k={\partial\over \partial t}$ and hence deduce, for
$m\ne 0$, that  
$$
\left(1-{2M\over r} + {Q^2\over r^2}\right) {dt\over d\tau} =
\varepsilon -{q\over m}{Q\over r}\ ,
$$
where $\tau$ is the particle's proper time and $\varepsilon$ is the
energy per unit mass. Show that the trajectories $r(t)$ of
massive particles with zero angular momentum satisfy
$$
({dr\over d\tau})^2 = (\varepsilon^2 -1) + \left(1-\varepsilon
{qQ\over mM}\right){2M\over r} +\left(\left({q\over m}\right)^2
-1\right){Q^2\over r^2}\ .
$$
Give a physical interpretation of this result for the special case
for which $q^2=m^2$, $qQ=mM$, and $\varepsilon=1$.


\vskip 10 pt
\noindent
{\bf 5.} Show that the action
$$
S[p,x,e]=\int d\lambda \big\{ p_\mu \dot x^\mu
-{1\over2}e\, \big[g^{\mu\nu}(x)p_\mu p_\nu + m^2\big]\big\}
$$
for a point particle of mass $m$ is equivalent, for $q=0$, to the action
of Q.4. Show that $S$ is invariant to
first order in $\alpha$ under the transformation
$$
\delta x^\mu =\alpha K^{\mu\nu}p_\nu \qquad \delta p_\mu
=-{1\over2}\alpha\, p_\rho p_\sigma \partial_\mu K^{\rho\sigma}
$$
for any symmetric tensor $K_{\mu\nu}$ obeying the {\it Killing
tensor} condition
$$
D_{(\rho} K_{\mu\nu)}=0\ .
$$
Show that the corresponding Noether charge is proportional to
$K^{\mu\nu}p_\mu p_\nu$ and verify that it is a constant of the
motion. A trivial example is $K_{\mu\nu}=g_{\mu\nu}$; what is
the corresponding constant of the motion? Show that
$\xi_\mu\xi_\nu$ is a Killing tensor if $\xi$ is a Killing vector. [A
Killing tensor that cannot be constructed from the metric and Killing
vectors is said to be irreducible. In a general axisymmetric
metric there are no such tensors, and so only three constants of the
motion, but for geodesics of the Kerr-Newman metric there is a 
`fourth constant' of the motion corresponding to an
irreducible Killing tensor.]
%$$
%\eqalign{
%K_{\mu\nu}dx^\mu dx^\nu = - &{\Sigma a^2\cos^2\theta\over \Delta}dr^2
%+ {\Delta a^2 \cos^2\theta\over \Sigma}(dt -a\sin^2\theta\,
%d\phi)^2\cr & +{r^2\sin^2\theta\over\Sigma}[adt - (r^2 + a^2)d\phi]^2
%+ r^2\Sigma
% d\theta^2\ .}
%$$

\vskip 10 pt 
\noindent 
{\bf 6.} By replacing the time coordinate $t$
by one of the radial null coordinates
$$
u= t+ {M\over \lambda} \qquad v= t- {M\over \lambda}
$$
show that the singularity at $\lambda=0$ of the Robinson-Bertotti (RB)
metric
$$
ds^2 = -\lambda^2 dt^2 + M^2 \left({d\lambda\over \lambda}\right)^2 +
M^2 d\Omega^2 
$$
is merely a coordinate singularity.
Show also that $\lambda=0$ is a degenerate Killing Horizon with respect
to $\partial\over \partial t$. By
introducing the new coordinates $(U,V)$, defined by
$$
u= \tan \left({U\over 2}\right)\qquad v=-\cot \left({V\over2}\right)
$$
obtain the maximal analytic extension of the RB metric and deduce its
Penrose diagram.

\vfill\eject

%\centerline {{\bf Example Sheet 3}}
%\vskip 10pt
\section{Example Sheet 3}

\noindent 
{\bf 1.} Let $\varepsilon$ and $h$ be the energy and and angular
momentum per unit mass of a zero charge particle in free fall
within the equatorial plane, i.e on a timelike ($\sigma=1$) or null
($\sigma=0$) geodesic with $\theta=\pi/2$, of a Kerr-Newman black hole.
Show that the particle's Boyer-Lindquist radial coordinate $r$
satisfies   
$$ 
\left({dr\over d\lambda}\right)^2 =\varepsilon^2 - V_{eff}(r)\ ,
$$ 
where $\lambda$ is an affine parameter, and the effective potential
$V_{eff}$ is given by $$ 
V_{eff} =
\left(1-{2M\over r}+{e^2\over r^2}\right)\left(\sigma + {h^2\over
r^2}\right) + {2a\varepsilon h\over r^3}\left( 2M -{e^2\over r}\right)
+{a^2\over r^2}\left[\sigma-\varepsilon^2\left(1+{2M\over r}-{e^2\over
r^2}\right)\right] \ .
$$

\vskip 10 pt 
\noindent 
{\bf 2.} Show that the surface gravity of the event horizon of a Kerr
black hole of mass $M$ and angular momentum $J$ is given by
$$
\kappa = {\sqrt{M^4 -J^2}\over 2M( M^2 + \sqrt{M^4 -J^2})}\ .
$$

\vskip 10 pt 
\noindent 
{\bf 3.} A particle at fixed $r$ and $\theta$ in a stationary
spacetime, with metric $ds^2= g_{\mu\nu}(r,\theta)dx^\mu dx^\nu$, has
angular velocity $\Omega= {d\phi\over dt}$ with respect to infinity.
Show that $\Omega(r,\theta)$ must satisfy
$$
g_{tt} + 2g_{t\phi}\Omega + g_{\phi\phi}\Omega^2 \le 0
$$
and hence deduce that
$$
{\cal D}\equiv g_{t\phi}^2 - g_{tt}g_{\phi\phi} \ge 0
$$
Show that ${\cal D}=\Delta (r) \sin^2\theta$ for the Kerr-Newman 
metric in Boyer-Lindquist coordinates, where $\Delta= r^2-2Mr+a^2+e^2$.
What happens if $(r,\theta)$ are such that ${\cal D}<0$? For what
values of $(r,\theta)$ can $\Omega$ vanish? Given that $r_{\pm}$ are the 
roots of $\Delta$, show that when ${\cal D}=0$  
$$ 
\Omega = {a\over r_{\pm}^2 +a^2} \ .
$$
\vskip 10pt
\noindent
{\bf 4.} Show that the area of the event horizon of a Kerr-Newman
black hole is
$$
A= 8\pi\big[ M^2 - {e^2\over2} + \sqrt{ M^4 -e^2 M^2 -J^2 }\,\big]\ .
$$

\vskip 10 pt
\noindent
{\bf 5.} A perfect fluid has stress tensor
$$
T_{\mu\nu} = (\rho + P)u_\mu u_\nu + P g_{\mu\nu}\ ,
$$
where $\rho$ is the density and $P(\rho)$ the pressure. State the
dominant energy condition for $T_{\mu\nu}$ and show that
for a perfect fluid in Minkowski spacetime this condition
is equivalent to  
$$ 
\rho\ge |P|\ .
$$
Show that the same condition arises from the requirement of
causality, i.e. that the speed of sound, $\sqrt{|dP/d\rho|}$, not
exceed that of light, together with the fact that the pressure 
vanishes in the vacuum.
\vskip 10pt
\noindent
{\bf 6.} The vacuum Einstein-Maxwell equations are
$$
G_{\mu\nu}= 8\pi T_{\mu\nu}(F) \qquad D_\mu F^{\mu\nu}=0
$$
where $F_{\mu\nu}= \partial_{\mu}A_{\nu}- \partial_\nu A_\mu$, and
$$
T_{\mu\nu}(F)= {1\over 4\pi}\big(F_{\mu}{}^\lambda F_{\nu\lambda}
-{1\over4}g_{\mu\nu}F^{\alpha\beta}F_{\alpha\beta}\big)\ .
$$
Asymptotically-flat solutions are stationary and axisymmetric if
the metric admits Killing vectors $k$ and $m$ that can be taken to be
$k={\partial\over\partial t}$ and $m={\partial\over\partial \phi}$ near
infinity, and if (for some choice of electromagnetic gauge)
$$
{\cal L}_k A={\cal L}_m A=0\ ,
$$
where the Lie derivative of $A$ with respect to a vector $\xi$, 
${\cal L}_\xi A$, is as defined in Q.4 of Example Sheet 2. The event
horizon of such a solution is necessarily a Killing horizon of $\xi =
k+\Omega_H m$, for some constant $\Omega_H$. What is the physical
interpretation of $\Omega_H$? What is its value for the Kerr-Newman
solution? The co-rotating electric potential is defined by 
$$ 
\Phi = \xi^\mu A_\mu \ .
$$
Use the fact that $R_{\mu\nu}\xi^\mu\xi^\nu=0$ on a Killing horizon
to show that $\Phi$ is constant on the horizon. In particular, show
that for a choice of the electromagnetic gauge for which $\Phi=0$ at
infinity, 
$$
\Phi_H= {Qr_+\over r_+^2 +a^2}
$$
for a charged rotating black hole, where $r_+= M+\sqrt{M^2-Q^2-a^2}$.

\vskip 10pt
\noindent
{\bf 7.} Let $({\cal M},g,A)$ be an asymptotically flat, stationary,
axisymmetric, solution of the Einstein-Maxwell equations of Q.6 and
let $\Sigma$ be a spacelike hypersurface with one boundary at spatial
infinity and an internal boundary, $H$, at the event horizon of a black
hole of charge $Q$. Show that    
$$ 
-2\int_\Sigma dS_\mu T^\mu{}_\nu(F)\xi^\nu = \Phi_H Q  
$$
where $\Phi_H$ is the co-rotating electric potential on the horizon. 
Use this result to deduce that the mass $M$ of a charged rotating black
hole is given by 
$$
M= {\kappa A\over 4\pi} + 2\Omega_H J + \Phi_H Q\ .
$$
where $J$ is the total angular momentum.
Use this formula for $M$ to deduce the first law of black hole
mechanics for charged rotating black holes: 
$$
dM= {\kappa \over 8\pi}dA + \Omega_H dJ + \Phi_H dQ\ .
$$
[Hint: ${\cal L}_\xi (F^{\mu\nu}A_\nu)=0$ ]

\vfill\eject


%\centerline{{\bf Example Sheet 4}}
%\vskip 10pt
\section{Example Sheet 4}

\noindent
{\bf 1.} Use the Komar integral,
$$
J= {1\over 16\pi G}\oint_\infty dS_{\mu\nu}D^\mu m^\nu\ ,
$$
for the total angular momentum of an asymptotically-flat axisymmetric
spacetime (with Killing vector $m$) to verify that $J=Ma$ for the
Kerr-Newman solution with parameter $a$.

\vskip 10pt
\noindent
{\bf 2.} Let $l$ and $n$ be two linearly independent vectors and
$\hat B$ a second rank tensor such that
$$
\hat B_\mu{}^\nu l_\nu =\hat B_\mu{}^\nu n_\nu =0\ .
$$
Given that $\eta^{(i)}$ $(i=1,2)$ are two further linearly
independent vectors, show that
$$
\varepsilon^{\mu\nu\rho\sigma}l_\mu n_\nu \hat B_\rho{}^\lambda
\big(\eta^{(1)}_\lambda\eta^{(2)}_\sigma - 
\eta^{(1)}_\sigma\eta^{(2)}_\lambda\big) =  \theta\, 
\varepsilon^{\mu\nu\rho\sigma} l_\mu n_\nu
\eta_\rho^{(1)}\eta_\sigma^{(2)}\ .
$$
where $\theta= \hat B_\alpha{}^\alpha$.

\vskip 10pt
\noindent
{\bf 3.} Let ${\cal N}$ be a Killing horizon of a Killing vector field
$\xi$, with surface gravity $\kappa$. Explain why, for any third-rank
totally-antisymmetric tensor $A$, the scalar 
$\Psi = A^{\mu\nu\rho}(\xi_\mu D_\nu\xi_\rho)$ vanishes on ${\cal N}$.
Use this to show that
$$
(\xi_{[\rho}D_{\sigma]} \xi_\nu)(D^\nu\xi^\mu) =\kappa
\xi_{[\rho}D_{\sigma]}\xi^\mu \qquad ({\rm on}\ {\cal N})\ ,\qquad (*)
$$
where the square brackets indicate antisymmetrization on the enclosed
indices.

>From the fact that $\Psi$ vanishes on ${\cal N}$ it follows that its
derivative on ${\cal N}$ is normal to ${\cal N}$, and hence that
$\xi_{[\mu}\partial_{\nu]}\Psi=0$ on ${\cal N}$. Use this fact and the
Killing vector lemma of Q.II.1 to deduce that, on ${\cal N}$,
$$
(\xi_\nu R_{\sigma\rho[\beta}{}^\lambda\xi_{\alpha]}
+\xi_\rho R_{\nu\sigma[\beta}{}^\lambda\xi_{\alpha]}
+\xi_\sigma R_{\rho\nu[\beta}{}^\lambda\xi_{\alpha]})\xi_\lambda\ .
$$
Contract on $\rho$ and $\alpha$ and use the fact that $\xi^2=0$ on
${\cal N}$ to show that
$$
\xi^\nu\xi_{[\rho}R_{\sigma]\nu\mu}{}^\lambda\xi_\lambda =
-\xi_\mu\xi_{[\rho}R_{\sigma]}{}^\lambda\xi_\lambda
\qquad ({\rm on}\ {\cal N})\ ,\qquad (\dagger)
$$
where $R_{\mu\nu}$ is the Ricci tensor.

For any vector $v$ the scalar $\Phi=(\xi\cdot D\xi -\kappa\xi)\cdot v$
vanishes on ${\cal N}$. It follows that
$\xi_{[\mu}\partial_{\nu]}\Phi|_{\cal N} =0$. Show that this fact, the
result (*) derived above and the Killing vector lemma imply
that, on ${\cal N}$, 
%% $$ \eqalign{
\bean
\xi^\mu\xi_{[\rho}\partial_{\sigma]}\kappa & = & \xi^\nu
R_{\mu\nu[\sigma}{}^\lambda\xi_{\rho]}\xi_\lambda \\
& = &\xi^\nu\xi_{[\rho}R_{\sigma]\nu\mu}{}^\lambda\xi_\lambda\ ,
\eean
%% } $$
where the second line is a consequence of the cyclic identity
satisfied by the Riemann tensor. Now use $(\dagger)$ to show that, on
${\cal N}$,
%$$ \eqalign{
\bea
\xi^\mu\xi_{[\rho}\partial_{\sigma]}\kappa & = &
\xi_{[\sigma}R_{\rho]}{}^\lambda\xi_\lambda \\
&= & 8\pi G\, \xi_{[\sigma}T_{\rho]}{}^\lambda\xi_\lambda\ ,
\eea
%} $$
where the second line follows on using the Einstein equations. Hence
deduce the zeroth law of black hole mechanics: that, provided the
matter stress tensor satisfies the dominant energy condition, the
surface gravity of any Killing vector field $\xi$ is constant on
each connected component of its Killing horizon (in particular, on the
event horizon of a stationary spacetime). 
% TJWP removed \vfill\eject

\vskip 10pt
\noindent
{\bf 4.} A scalar field $\Phi$ in the Kruskal spacetime satisfies
the Klein-Gordon equation
$$
D^2\Phi -m^2\Phi =0\ .
$$
Given that, in static Schwarzshild coordinates, $\Phi$ takes the form
$$
\Phi = R_\ell(r) e^{-i\omega t} Y_{\ell}(\theta,\phi)
$$
where $Y_{\ell\, m}$ is a spherical harmonic, find the radial equation
satisfied by $R_\ell(r)$. Show that near the horizon at $r=2M$,
$\Phi\sim e^{\pm i\omega r^*}$, where $r^*$ is the Regge-Wheeler radial
coordinate. Verify that ingoing waves are analytic, in Kruskal
coordinates, on the future horizon, ${\cal H}^+$, but not, in general,
on the past horizon, ${\cal H}^-$, and conversely for outgoing waves.

Given that both $m$ and $\omega$ vanish, show that
$$
R_\ell = A_\ell P_\ell(z) + B_\ell Q_\ell(z)
$$
for constants $A_\ell,\, B_\ell$, where $z=(r-M)/M$, $P_\ell(z)$ is a
Legendre Polynomial and $Q_\ell(z)$ a linearly-independent solution.
Hence show that there are no {\it non-constant} solutions that are both
regular on the horizon, ${\cal H}= {\cal H}^+ \cup {\cal H}^-$, and
bounded at infinity.

\vskip 10 pt
\noindent
{\bf 5.} Use the fact that a Schwarzschild black hole radiates at the
Hawking temperature
$$
T_H ={1\over 8\pi M}
$$
(in units for which $\hbar$, $G$, $c$, and Bolzmann's constant all
equal $1$) to show that the thermal equilibrium of a black hole with an
infinite reservoir of radiation at temperature $T_H$ is unstable.

A finite reservoir of radiation of volume $V$ at temperature $T$ has
an energy, $E_{res}$ and entropy, $S_{res}$ given by
$$
E_{res} = \sigma VT^4 \qquad S_{res} ={4\over3}\sigma VT^3
$$
where $\sigma$ is a constant. A Schwarzschild black hole of mass $M$ is
placed in the reservoir. Assuming that the black hole has entropy
$$
S_{BH} =4\pi M^2\ ,
$$
show that the total entropy $S= S_{BH}+S_{res}$ is extremized 
for fixed total energy $E= M+E_{res}$, when $T=T_H$, Show that the
extremum is a maximum if and only if $V<V_c$, where the critical value
of $V$ is $$
V_c = {2^{20}\pi^4E^5\over 5^5\sigma }
$$
What happens as $V$ passes from $V<V_c$ to $V>V_c$, or
vice-versa?


\vskip 10pt
\noindent
{\bf 6.} The specific heat of a charged black hole of mass $M$, at
fixed charge $Q$, is 
$$
C\equiv T_H {\partial S_{BH}\over \partial
T_H}\bigg|_Q \ ,
$$
where $T_H$ is its Hawking temperature and $S_{BH}$ its
entropy. Assuming that the entropy of a black hole is given by $S_{BH}=
{1\over4}A$, where $A$ is the area of the event horizon, show that the
specific heat of a Reissner-Nordstrom black hole is
$$
C= {2S_{BH}\sqrt{M^2-Q^2}\over (M-2\sqrt{M^2-Q^2})}\ .
$$
Hence show that $C^{-1}$ changes sign when $M$ passes through
${2|Q|\over\sqrt{3}}$. 

Repeat Q.5 for a Reissner-Nordstrom black hole.
Specifically, show that the critical reservoir volume, $V_c$, is
infinite for $|Q|\le M \le {2|Q|\over\sqrt{3}}$. Why is this result
to be expected from your previous result for $C$? 



%\end









