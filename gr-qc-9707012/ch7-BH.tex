\chapter{Hawking Radiation}

\section{Quantization of the Free Scalar Field}

Let $\Phi(x)$ be a real scalar field satisfying the Klein-Gordon 
equation\index{Klein-Gordon equation}.
\be
\left(D^{\mu}\partial_{\mu}-m^2\right)\Phi(x)=0
\ee
Let $\left\{\phi_{\alpha}\right\}$ span the space ${\cal S}$ of solutions. We
shall assume that the spacetime is globally hyperbolic, i.e. that $\exists$ a
Cauchy surface $\Sigma$. A point in the space ${\cal S}$ then corresponds to a
choice of initial data on $\Sigma$. The space ${\cal S}$ has a natural
\emphin{symplectic structure}.
\be
\phi_{\alpha}\wedge \phi_{\beta}=\int_{\Sigma}dS_{\mu}
\phi_{\alpha}\stackrel{\leftrightarrow}{\partial}^{\mu}\phi_{\beta},\quad
\left(=-\phi_{\beta}\wedge \phi_{\alpha}\right)
\ee
where $\lrpd{}$ is defined by
\be
f \lrpd{} g = f \partial g -g \partial f
\ee
`Natural' means that $\wedge$ \emph{does not depend on the choice of $\Sigma$}.
\be
\left(\phi_{\alpha}\wedge\phi_{\beta}\right)_{\Sigma}-\left(\phi_{\alpha}
\wedge\phi_{\beta}\right)_{\Sigma'} = \int_S \dx{4}{x}
\sqrt{-g}D_{\mu}\left(\phi_{\alpha}\lrpd{\mu}\phi_{\beta}\right) 
\ee
\begin{center}\input{p125-1.pictex}\end{center}
But
\bea
D_{\mu}\left(\phi_{\alpha}\lrpd{\mu}\phi_{\beta}\right) & = & 
\phi_{\alpha}\left(D_{\mu}\partial^{\mu}\phi_{\beta}\right)-
\left(D_{\mu}\partial^{\mu}\phi_{\alpha}\right)\phi_{\beta}
\\ 
 & = & \phi_{\alpha}\left(m^2\phi_{\beta}\right)-
\left(m^2\phi_{\alpha}\right)\phi_{\beta} = 0 \ , 
\eea
using the Klein-Gordon equation in the last step.

The antisymmetric form $\phi_{\alpha}\wedge\phi_{\beta}$ can be brought 
to a canonical block diagonal form, with $2\times 2$ blocks
$\matrixtwo{0}{1}{-1}{0}$, by a change of basis (Darboux's theorem). Thus,
real solutions of the Klein-Gordon equation can be grouped in pairs
$(\phi,\phi')$ with $\phi\wedge\phi'=1$. It then follows that the complex
solution $\psi=\left(\phi-i\phi'\right)/\sqrt{2}$ has unit norm if we define its norm $||\psi||$ by
$||\psi||^2=\phi\wedge\phi'$ or, equivalently,
\be
||\psi||^2 = i\int_\Sigma dS_{\mu}\psi^*\lrpd{\mu}\psi\, .
\ee
More generally, we can introduce a complex basis $\{\psi_i\}$ of solutions
of the Klein-Gordon equation with hermitian inner product defined by
\be
(\psi_i, \psi_j) = i\int dS_{\mu}\, \psi_i^*\lrpd{\mu}\psi_j\ ,
\ee
and we can choose this basis such that $(\psi_i, \psi_j)=\delta_{ij}$. This
inner product is not positive definite, however, because 
$||\psi^*||^2 = -||\psi||^2$. In fact, we can choose the basis $\{\psi_i\}$ such
that
\be
\left( \begin{array}{rclcrcl}
\left(\psi_i,\psi_j\right) & = & \delta_{ij} & 
\quad & \left(\psi_i,\psi_j^*\right) & = & 0 \\ \\
\left(\psi_i^*,\psi_j\right) & = & 0 & \quad & 
\left(\psi_i^*,\psi_j^*\right) & = & -\delta_{ij} \end{array} \right)
\label{eq:inner_prod}
\ee

We could interpret the complex solution $\Psi=\sum_i a_i\psi_i$ as 
the wavefunction of a free particle since $(\;,\;)$ is positive-definite when
restricted to such solutions, but this cannot work when interactions are
present.  It is also inapplicable for \emph{real} scalar fields.  A real
solution $\Phi$ of the K-G equation can be written as
\be
\Phi(x)=\sum\left[ a_i\psi_i(x)+a_i^*\psi_i^*(x)\right]
\ee
To quantize we pass to the \emph{quantum} field
\be
\Phi(x)=\sum\left[a_i\psi_i(x)+a_i^{\dagger}\psi_i^*(x)\right]
\ee
where $\left\{a_i\right\}$ are now operators in a Hilbert space $\mcH$ with 
Hermitian conjugates $a_i^{\dagger}$ satisfying the commutation relations
\be
\left[a_i,a_j\right]=0, \quad \left[a_i,a_j^{\dagger}\right]=\delta_{ij} 
\quad (\hbar=1)
\ee
We choose the Hilbert space to be the Fock space built from a `vacuum' 
state $\ket{\mbox{vac}}$ satisfying
\bea
a_i\vac & = & 0 \quad \forall i \\
\left<\mbox{vac}|\mbox{vac}\right> & = & 1 
\eea
i.e. $\mcH$ has the basis
\bdm
\left\{\vac,a_i^{\dagger}\vac, a_i^{\dagger}a_j^{\dagger}\vac, \ldots \right\}
\edm
$\left<\;|\;\right>$ is a positive-definite inner product on this space.

This basis for $\mcH$ is determined by the choice of $\vac$, but this 
depends on the choice of complex basis $\left\{\psi_i\right\}$ of solutions of
the K-G equation satisfying (\ref{eq:inner_prod}).  There are \emph{many} such
bases. \\

Consider $\left\{\psi_i'\right\}$ where
\be
\psi_i'=\sum_j\left(A_{ij}\psi_j+B_{ij}\psi_j^*\right)
\label{eq:psi_i_prime}
\ee
This has the same inner product matrix (\ref{eq:inner_prod}) provided that
\bebox{
\begin{array}{rcl} AA^{\dagger}-BB^{\dagger} & = & \mathbf{1} \\ \\
\transpose{AB}-\transpose{BA} & = & 0  \end{array} \label{eq:ip_dagger}
}
Inversion of (\ref{eq:psi_i_prime}) leads to
\be
\psi_j = \sum_k A_{jk}'\psi_k'+B_{jk}'{\psi_k'}^*
\ee
where
\be
A'=A^{\dagger}, \quad B'=-\transpose{B}
\ee
\ul{Check}
\bea
\psi' & = & A\left(A'\psi'+B'{\psi'}^*\right)+B\left({A'}^*{\psi'}^*+{B'}^*\psi'\right) \\
 & = & \left(AA'+B{B'}^*\right)\psi'+\left(AB'+B{A'}^*\right){\psi'}^* \\
 & = & \left(AA^{\dagger}-BB^{\dagger}\right)\psi'-\left(A\transpose{B}-B\transpose{A}\right)\psi' \\
 & = & \psi'
\eea
But $A'$ and $B'$ must satisfy the same conditions as $A$ and $B$, i.e.
\bea
A'{A'}^{\dagger}-B'{B'}^{\dagger} & = & \mathbf{1} \\
A'\transpose{B'}-B'\transpose{A'} & = & 0 
\eea
Equivalently,
\bebox{
\begin{array}{rcl}
A^{\dagger}A-\transpose{B}B^* & = & \mathbf{1} \\
A^{\dagger}B-\transpose{B}A^* & = & 0 \end{array}
\label{eq:relations_star}
}
These conditions are not implied by (\ref{eq:ip_dagger}); the additional information contained in them is the invertibility of the change of basis.

In a general spacetime there is no `preferred' choice of basis satisfying 
(\ref{eq:inner_prod}) and so no preferred choice of vacuum.  In a stationary
spacetime, however, we can choose the basis $\left\{u_i\right\}$ of
\emph{positive frequency} eigenfunctions of $k$, i.e.
\be
k^{\mu}\partial_{\mu}u_i = -i\omega_i u_i, \quad \omega_i \ge 0 
\ee
Notes
\newcounter{pfenotes}
\begin{list}{(\arabic{pfenotes})}
{\usecounter{pfenotes}}
\item Since $k$ is Killing it maps solutions of the Klein-Gordon equation to 
solutions (\bold{Proof: Exercise}). 

\item $k$ is anti-hermitian, so it can be diagonalized with pure-imaginary 
eigenvalues.

\item Eigenfunctions with distinct eigenvalues are orthogonal so
\be
\left(u_i,u_j^*\right)=0
\ee
We can normalize $\left\{u_i\right\}$ s.t. $\left(u_i,u_j\right)=\delta_{ij}$, 
so the basis $\left\{u_i\right\}$ can be chosen s.t. (\ref{eq:inner_prod}) is
satisfied.  

\item We exclude functions with $\omega=0$.

\end{list}

For this choice of basis the vacuum state $\vac$ is actually the state of 
lowest energy.  The states $a_i^{\dagger}\vac$ are one-particle states,
$a_i^{\dagger}a_j^{\dagger}\vac$ two-particle states, etc., and
\be
N=\sum_i a_i^{\dagger}a_i
\ee
is the particle number operator\index{particle number operator}.

\section{Particle Production in Non-Stationary Spacetimes}

Consider a `sandwich' spacetime\index{sandwich spacetime} 
$M=M_- \cup M_0 \cup M_+$
\begin{center}\input{p128-1.pictex}\end{center}
In $M_-$ we can choose to expand a scalar field solution of the 
Klein-Gordon equation as
\be
\Phi(x)=\sum_i \left[a_i u_i(x)+a_i^{\dagger}u_i^*(x)\right] \qquad 
\mbox{in $M_-$}
\ee
The functions $u_i(x)$ solve the KG equation in $M_-$ but \ul{not} in $M$, 
so its continuation through $M_0$ will lead to some new function $\psi_i(x)$ in
$M_+$, so
\be
\Phi(x)=\sum_i \left[a_i \psi_i(x)+a_i^{\dagger}\psi_i^*(x)\right] \qquad 
\mbox{in $M_+$}
\ee
Because the inner product $(\;,\;)$ was independent of the hypersurface 
$\Sigma$, the matrix of inner products will still be as before, i.e. as in
(\ref{eq:inner_prod}).  But, as we have seen this implies only that
\be
\psi_i = \sum_j \left(A_{ij}u_j+B_{ij}u_j^*\right)
\ee
for some matrices $A$ and $B$ satisfying (\ref{eq:ip_dagger}).  Thus, in $M_+$
\bea
\Phi(x) & = & \sum_i\left(a_i\psi_i+a_i^{\dagger}\psi_i^* \right) \\
 & = & \sum_i\left[a_i \sum_j\left(A_{ij}u_j+B_{ij}u_j^*\right)+
a_i^{\dagger}\sum_j\left(A_{ij}^*u_j^*+B_{ij}^*u_j\right)\right] \\
 & = & \sum_i \left[a_i' u_i(x)+{a_i'}^{\dagger}u_i^*(x)\right] 
\eea
where
\bebox{
a_j' = \sum_i\left(a_iA_{ij}+a_i^{\dagger}B_{ij}^*\right)
}
This is called a \emphin{Bogoliubov transformation}.  $A$ and $B$ are 
the \emph{Bogoliubov coefficients}.   \\

Note that (\bold{Exercise})
\be
\left.\begin{array}{rcl}
\left[a_i',a_j'\right] & = & 0 \\ \\ \left[a_i',{a_j'}^{\dagger}\right] 
& = & \delta_{ij} \end{array}\right\} \Leftrightarrow \mbox{ relations
(\ref{eq:relations_star}) satisfied by $A$ \& $B$}
\ee
If $B=0$ then (\ref{eq:ip_dagger}) and (\ref{eq:relations_star}) imply $A^{\dagger}A=AA^{\dagger}=1$, i.e. the 
change of basis from $\left\{u_i\right\}$ to $\left\{\psi_i\right\}$ is just a
unitary transformation which permutes the annihilation operators but does not
change the definition of the vacuum.

The particle number operator for the $i^{\subtext{th}}$ mode of $k$ is 
\be
\begin{array}{rclcr} N_i & = & a_i^{\dagger}a_i & \quad & \mbox{in $M_-$} \\ \\
N_i' & = & {a_i'}^{\dagger}a_i' & \quad & \mbox{in $M_+$} 
\end{array} 
\ee
The state with no particles in $M_-$ is $\vac$ s.t. $a_i\vac=0\; \forall i$.  
The expected number of particles in the $i^{\subtext{th}}$ mode in $M_+$ is then
\bea
\left<N_i'\right> & \equiv & \braket{\mbox{vac}}{N_i'}{\mbox{vac}} = 
\braket{\mbox{vac}}{{a_i'}^{\dagger}a_i'}{\mbox{vac}} \\
 & = & \sum_{j,k}\braket{\mbox{vac}}{\left(a_k B_{ki}\right)
\left(a_j^{\dagger}B_{ji}^*\right)}{\mbox{vac}} \\ 
 & = & \sum_{j,k}\underbrace{ \braket{\mbox{vac}}{a_ka_j^{\dagger}}
{\mbox{vac}}}_{\delta_{kj}} B_{ki}B_{ij}^{\dagger} \\
 & = & \left(B^{\dagger}B\right)_{ii}
\eea
The expected total number of particles is therefore 
$\tr\left(B^{\dagger}B\right)$.  Since $B^{\dagger}B$ is positive semi-definite,
this vanishes iff $B=0$. 

\section{Hawking Radiation}

The spacetime associated to gravitational collapse to a black hole cannot be 
everywhere stationary so we expect particle creation.  But the exterior
spacetime is stationary at late times, so we might expect particle creation to
be just a transient phenomenon determined by details of the collapse.

But the \emph{infinite time dilation} at the horizon of a black hole means 
that particles created in the collapse can take arbitrarily long to escape -
suggests a possible flux of particles at late times that is due to the existence
of the horizon and \emph{independent of the details of the collapse}.  There is
such a particle flux, and it turns out to be thermal - this is \emph{Hawking
radiation}\index{Hawking!radiation}

We shall consider only a massless scalar field $\Phi$ in a Schwarzschild black 
hole spacetime.  From Question IV.4 we learn that the positive frequency
outgoing modes of $\Phi$ have the behaviour
\be
\Phi_{\omega}\sim e^{-i\omega u}
\ee
near $\scri^+$.  Consider a geometric optics approximation in which a 
particle's worldline is a null ray, $\gamma$, of constant phase $u$, and trace
this ray backwards in time from $\scri^+$.  The later it reaches $\scri^+$ the
closer it must approach $\mcH^+$ in the exterior spacetime before entering the
star.
\begin{center}\input{p133-1.pictex}\end{center}
The ray $\gamma$ is one of a family of rays whose limit as $t\to\infty$ is a 
null geodesic generator, $\gamma_H$, of $\mcH^+$.  We can specify $\gamma$ by
giving its affine distance from $\gamma_H$ along an ingoing null geodesic
through $\mcH^+$
\begin{center}\input{p133-2.pictex}\end{center}
The affine parameter on this ingoing null geodesic is $U$, so $U=-\epsilon$.  
Equivalently
\be
u=-\frac{1}{\kappa}\log \epsilon \qquad \mbox{(on $\gamma$ near $\mcH^+$)}
\ee
so
\be
\Phi_{\omega}\sim \exp\left(\frac{i\omega}{\kappa}\log \epsilon\right) 
\qquad \mbox{near $\mcH^+$}
\ee
This oscillates increasingly rapidly as $\epsilon\to 0$, so 
\emph{the geometric optics approximation is justified at late times}.

We need to match $\Phi_{\omega}$ onto a solution of the K-G equation 
near $\scri^-$.  In the geometric optics approximation we just
parallely-transport $n$ and $l$ back to $\scri^-$ along the continuation of
$\gamma_H$.  Let this continuation meet $\scri^-$ at $v=0$.  The continuation of
the ray $\gamma$ back to $\scri^-$ will now meet $\scri^-$ at an affine distance
$\epsilon$ along an outgoing null geodesic on $\scri^-$
\begin{center}\input{p134-1.pictex}\end{center}
The affine parameter on outgoing null geodesics in $\scri^-$ is $v$ 
(since $ds^2=du\,dv+r^2d\Omega^2$ on $\scri^-$), so $v=-\epsilon$ on 
$\gamma$ so
\be
\Phi_{\omega}\sim \exp\left\{\frac{i\omega}{\kappa}\log(-v)\right\}
\ee
This is for $v<0$. For $v>0$ an ingoing null ray from $\scri^-$ passes 
through $\mcH^+$ and doesn't reach $\scri^+$, so
$\Phi_{\omega}=\Phi_{\omega}(v)$ on $\scri^-$, where
\be
\Phi_{\omega}(v) = \left\{ \begin{array}{ccc} 0 & \quad & v>0 \\
 \exp\left(\frac{i\omega}{\kappa}\log(-v)\right) & \quad & v<0 
\end{array}\right.
\ee

Take the Fourier transform,
\bea
\tilde{\Phi}_{\omega} & = & \int^{\infty}_{-\infty} 
e^{i\omega' v}\Phi_{\omega}(v)dv \\
 & = & \int^0_{-\infty}\exp\left\{i\omega'v +
\frac{i\omega}{\kappa}\log(-v)\right\}dv
\eea
\paragraph{Lemma} 
\bebox{
\tilde{\Phi}_{\omega}(-\omega')=-\exp\left(-\frac{\pi\omega}
{\kappa}\right)\tilde{\Phi}_{\omega}(\omega') \qquad \mbox{for $\omega'>0$}
}

\paragraph{Proof}  Choose branch cut in complex $v$-plane to lie along 
the real axis
\begin{center}\input{p135-1.pictex}\end{center}
For $\omega'>0$ rotate contour to the positive imaginary axis and then 
set $v=ix$ to get
\bea
\tilde{\Phi}_{\omega}(\omega') & = & -i\int^{\infty}_0 \exp\left
\{-\omega'x+\frac{i\omega}{\kappa}\log\left(xe^{-i\pi/2}\right)\right\}dx \\
 & = & -\exp\left(\frac{\pi\omega}{2\kappa}\right) \int_0^{\infty} 
\exp\left\{-\omega'x+\frac{i\omega}{\kappa}\log(x)\right\}dx
\eea
Since $\omega'>0$ the integral converges.  When $\omega'<0$ we rotate 
the contour to the negative imaginary axis and then set $v=-ix$ to get
\bea
\tilde{\Phi}_{\omega}(\omega') & = & i\int_0^{\infty} \exp\left
\{ \omega'x+\frac{i\omega}{\kappa}\log\left(xe^{i\pi/2}\right)\right\}dx \\
 & = & \exp\left(-\frac{\pi\omega}{2\kappa}\right)\int^{\infty}_0 
\exp\left\{ \omega'x+\frac{i\omega}{\kappa}\log(x)\right\} dx 
\eea
Hence the result. 

\paragraph{Corollary}  A mode of \emph{positive} frequency $\omega$ on 
$\scri^+$, \emph{at late times}, matches onto \emph{mixed positive and negative}
modes on $\scri^-$.  We can identify (for positive $\omega'$)
\bea
A_{\omega\omega'} & = & \tilde{\Phi}_{\omega}(\omega') \\
B_{\omega\omega'} & = & \tilde{\Phi}_{\omega}(-\omega') = 
-e^{-\pi\omega/\kappa}\tilde{\Phi}_{\omega}(\omega')
\eea
as the Bogoliubov coefficients.  We see that
\bebox{
B_{ij}=-e^{-\pi \omega_i/\kappa}A_{ij}
}
But the matrices $A$ and $B$ must satisfy the Bogoliubov relations, e.g.
\bea
\delta_{ij} & = & \left(AA^{\dagger}-BB^{\dagger}\right)_{ij} \\
 & = & \sum_k A_{ik}A_{jk}^*-B_{ik}B^*_{jk} \\
 & = & \left[ e^{\pi\left(\omega_i+\omega_j\right)/\kappa}-1\right]\sum_k B_{ik}B_{jk}^* 
\eea
Take $i=j$ to get
\be
\left(BB^{\dagger}\right)_{ii} = \frac{1}{e^{2\pi\omega_i/\kappa}-1}
\ee
Now, what we actually need are the \emph{inverse} Bogoliubov coefficients corresponding to a positive frequency mode on $\scri^-$ matching onto mixed positive and negative frequency modes on $\scri^+$.  As we saw earlier, the inverse $B$ coefficient is
\be
B'=-\transpose{B}
\ee
The late time particle flux through $\scri^+$ given a vacuum on $\scri^-$ is
\be
\left<N_i\right>_{\scri^+} = \left(\left(B'\right)^{\dagger}B'\right)_{ii} = \left(B^*\transpose{B}\right)_{ii} = \left(B\transpose{B}\right)^*_{ii}
\ee
But $\left(B\transpose{B}\right)_{ii}$ is real so
\bebox{
\left<N_i\right>_{\scri^+} = \frac{1}{e^{2\pi\omega_i/\kappa}-1}
}
This is the Planck distribution\index{Planck distribution} for black body 
radiation\index{black body radiation} at the Hawking
temperature\index{Hawking!temperature}
\be
T_H=\frac{\hbar\kappa}{2\pi}
\ee

We conclude that at late times the black hole radiates away its energy at 
this temperature.  From Stephan's law\index{Stephan's law}
\be
\frac{dE}{dt}\simeq -\sigma AT_H^4, \qquad 
\left(\sigma=\frac{\pi^2k_B^4}{60\hbar^3c^2}\right)
\ee
where $A$ is the black hole area.  Since
\be
E=Mc^2,\quad A=\left(\frac{MG}{c^2}\right)^2, \quad k_BT_H \sim 
\frac{\hbar c^3}{GM}
\ee
we have
\be
\frac{dM}{dt} \sim \frac{\hbar c^4}{G^2M^2} 
\ee
which gives a lifetime 
\be
\tau \sim \left(\frac{G^2}{\hbar c^4}\right)M^3 
\ee
%which is larger than the age of the universe for $M\sim M_{\odot}$.

\paragraph{Note} The calculation of Hawking radiation assumed no 
backreaction, i.e. $M$ was taken to be constant.  This is a good approximation
when $dM/dt\ll M$, but fails in the final stages of evaporation.\section{Black
Holes and Thermodynamics}

Since $T=\frac{\hbar\kappa}{2\pi}$ is the black hole temperature, 
we can now rewrite the $1^{\subtext{st}}$ law of black hole mechanics as
\be
dM=TdS_{\subtext{BH}}+\Omega_H dJ +\Phi_H dQ, \qquad \mbox{($\Omega_H,\Phi_H$ 
intensive, $J,Q$ extensive)}
\ee
where
\bebox{
S_{\subtext{BH}}=\frac{A}{4\hbar}
}
is the black hole (or Beckenstein-Hawking) 
entropy\index{black hole!entropy}\index{Beckenstein-Hawking entropy}.  

Clearly, black hole evaporation via Hawking radiation will 
cause $S_{\subtext{BH}}$ to \emph{decrease} in violation of the
$2^{\subtext{nd}}$ law of black hole mechanics (derived on the assumption of
classical physics).  But the entropy is 
\be
S=S_{\subtext{BH}}+S_{\subtext{ext}}
\ee
where $S_{\subtext{ext}}$ is the entropy of matter in exterior spacetime.  
But because the Hawking radiation is \emph{thermal}, $S_{\subtext{ext}}$
increases with the result that $S$ is a non-decreasing function of time.  This
suggests:

\subsubsection{Generalized $2^{\subtext{nd}}$ Law of Thermodynamics}

$S=S_{\subtext{BH}}+S_{\subtext{ext}}$ is always a non-decreasing function 
of time (in any process). \\

This was first suggested by Beckenstein (without knowledge of the precise 
form of $S_{\subtext{BH}}$) on the grounds that the entropy in the exterior
spacetime could be decreased by throwing matter into a black hole.  This would
violate the $2^{\subtext{nd}}$ law of thermodynamics unless the black hole is
assigned an entropy.

\subsection{The Information Problem}

Taking Hawking radiation into account, a black hole that forms from 
gravitational collapse will eventually evaporate, after which the spacetime has
no event horizon.  This is depicted by the following CP diagram:
\begin{center}\input{p140-1.pictex}\end{center}
$\Sigma_1$ is a Cauchy surface for this spacetime, but $\Sigma_2$ is not 
because its past domain of dependence $D^-\left(\Sigma_2\right)$ does not
include the black hole region.  Information from $\Sigma_1$ can propagate into
the black hole region instead of to $\Sigma_2$.  Thus it appears that
information is `lost' into the black hole.  This would imply a
\emph{non-unitary} evolution from $\Sigma_1$ to $\Sigma_2$, and hence put QFT in
curved spacetime in conflict with a basic principle of Q.M.  However, from the
point of view of a static external observer, nothing actually passes through
$\mcH^+$, so maybe the information is not really lost.  A complete calculation
including all back-reaction effects might resolve the issue, but even this is
controversial since some authors claim that the resolution requires an
understanding of the Planck scale physics.  The point is that whereas QFT in
curved spacetime predicts $T_{\subtext{loc}}\to\infty$ on the horizon of a black
hole, this should not be believed when $kT$ reaches the Planck energy
$\left(\hbar c/G\right)^{1/2}c^2$ because i) Quantum Gravity\index{quantum
gravity} effects cannot then be ignored and ii) this temperature is then of the order
maximum (Hagedorn) temperature in string theory\index{string theory}. \\

