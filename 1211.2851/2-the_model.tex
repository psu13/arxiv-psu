\section{The Simplicial Model} \label{section:the-model}

In this section, we will apply the techniques of Section \ref{section:models-from-universes} to construct a model of type theory in the category $\sSets$. As mentioned in the Introduction, type dependency is interpreted using Kan fibrations and in particular the closed type will be Kan complexes. To this end, we construct (for any regular cardinal $\alpha$) a Kan fibration $p_\alpha \colon \UUt \to \UU$, weakly universal among Kan fibrations with $\alpha$-small fibers, and investigate the key properties of $\UU$ and $p_\alpha$. We then show that $\UU$ is a Kan complex, and (when $\alpha$ is inaccessible) carries the various logical structures defined in Section~\ref{subsec:logical-structure-on-universes}.  Together, these yield our first main goal: a model of type theory in $\sSets$, with an internal universe.

\subsection{A universe of Kan complexes} \label{subsec:representability-of-fibs}

In constructing a universe $\UU$ intended to represent $\alpha$-small Kan fibrations, one might expect (by the Yoneda lemma) to simply define $(\UU)_n$ as the set of $\alpha$-small fibrations over $\Delta[n]$.  This definition has two problems: firstly, it gives not sets, but proper classes; and secondly, it is not strictly functorial, since pullback is functorial only up to isomorphism.

Some extra technical device is therefore needed to resolve these issues.  Several possible solutions exist\footnote{Other possible approaches include ones based on the general results of \cite{hofmann:on-the-interpretation} and \cite{lumsdaine-warren:local-universes}, or taking $\UU_n$ as $[(\int\! \Delta[n])^\op, \Sets_{<\alpha}]$ as in \cite{hofmann-streicher:lifting-grothendieck-universes}.}; we take the approach of passing to isomorphism classes, having first added well-orderings to the mix so that fibrations have no non-trivial automorphisms (without which the crucial Lemmas~\ref{lemma:w-preserves-lims}, \ref{lemma:w-representable} would fail).  We emphasise, however, that this is the sole reason for introducing the well-orderings: they are of no intrinsic interest or significance. % (and are indeed occasionally something of an inconvenience). % inconvenience = pain in the ass

\begin{definition}
A \emph{well-ordered morphism} of simplicial sets consists of an ordinary map of simplicial sets $f \colon Y \to X$, together with a function assigning to each simplex $x \in X_n$ a well-ordering on the fiber $Y_x := f^{-1}(x) \subseteq Y_n$.

If $f \colon Y \to X$, $f' \colon Y' \to X$ are well-ordered morphisms into a common base $X$, an \emph{isomorphism} of well-ordered morphisms from $f$ to $f'$ is an isomorphism $Y \iso Y'$ over $X$ preserving the well-orderings on the fibers.
\end{definition}

\begin{proposition} \label{prop:well-orderings-rigid}
Given two well-ordered sets, there is at most one isomorphism between them.  Given two well-ordered morphisms over a common base, there is at most one isomorphism between them.
\end{proposition}

\begin{proof}
The first statement is classical (and immediate by induction); the second follows from the first, applied in each fiber.
\end{proof}

\begin{definition}
Fix (for the remainder of this and the following section) a regular cardinal $\alpha$.  Say a map of simplicial sets $f \colon Y \to X$ is \emph{$\alpha$-small} if each of its fibers $Y_x$ has cardinality $< \alpha$.
\end{definition}

Given a simplicial set $X$, define $\W (X)$ to be the set of isomorphism classes\footnote{We use \emph{isomorphism classes} in the sense of “Scott’s trick” \cite{scott:scotts-trick} for constructing proper class quotients.  The class of \emph{all} well-ordered morphisms isomorphic to a given one is a proper class, so one instead uses the subclass of such morphisms \emph{of minimal rank}, which is a set.} of $\alpha$-small well-ordered morphisms $Y \to X$; together with the pullback action $\W(f) := f^* \colon \W(X) \to \W(X')$, for $f \colon X' \to X$, this gives a contravariant functor $\W \colon \sSets^{\op} \to \Sets$.

\begin{lemma} \label{lemma:w-preserves-lims}
$\W$ preserves all limits: $\W(\colim_i X_i) \iso \lim_i \W(X_i)$. 
\end{lemma}

\begin{proof}
Suppose $F \colon \I \to \sSets$ is some diagram, and $X = \colim_\I F$ is its colimit, with injections $\nu_i \colon F(i) \to X$.  We need to show that the canonical map $\W(X) \to \lim_\I \W(F(i))$ is an isomorphism.

To see that it is surjective, suppose we are given $[f_i \colon Y_i \to F(i)] \in \lim_\I \W(F(i))$.  For each $x \in X_n$, choose some $i$ and $\bar{x} \in F(i)$ with $\nu(\bar{x}) = x$, and set $Y_x := (Y_i)_{\bar{x}}$.  By Proposition~\ref{prop:well-orderings-rigid}, this is well-defined up to \emph{canonical} isomorphism, independent of the choices of representatives $i$, $\bar{x}$, $Y_i$, $f_i$.  The total space of these fibers then defines a well-ordered morphism $f \colon Y \to X$, with fibers of size $<\alpha$, and with pullbacks isomorphic to $f_i$ as required.

For injectivity, suppose $f, f'$ are well-ordered morphisms over $X$, and $\nu_i^* f \iso \nu_i^* f'$ for each $i$.  By Proposition~\ref{prop:well-orderings-rigid}, these isomorphisms must agree on each fiber, so together give an isomorphism $f \iso f'$.
\end{proof}

Define the simplicial set $\WW$ by
\[\WW := \W \cdot \y^{\op} \colon \Delta^{\op} \to \Sets,\]
where $\y$ denotes the Yoneda embedding $\Delta \to \sSets$.

\begin{lemma} \label{lemma:w-representable}
The functor $\W$ is representable, represented by $\WW$.
\end{lemma}

\begin{proof}
The functors $\W$ and $\Hom(-,\WW)$ agree up to isomorphism on the standard simplices (by the Yoneda lemma), and send colimits in $\sSets$ to limits; but every simplicial set is canonically a colimit of standard simplices.
\end{proof}

\begin{notation}
Given an $\alpha$-small well-ordered map $f \colon Y \to X$, the corresponding map $X \to \WW$ will be denoted by $\name{f}$.
\end{notation}

Applying the natural isomorphism above to the identity map $\WW \to \WW$ yields a universal $\alpha$-small well-ordered simplicial set $\WWt \to \WW$.  Explicitly, $n$-simplices of $\WWt$ are classes of pairs
\[(f \colon Y \to \Delta [n], s \in f^{-1}(1_{[n]}))\]
i.e.\ the fiber of $\WWt$ over an $n$-simplex $\name{f} \in \WW$ is exactly (an isomorphic copy of) the main fiber of $f$.  So, by construction:

\begin{proposition}
The canonical projection $\WWt \to \WW$ is strictly universal for $\alpha$-small well-ordered morphisms; that is, any such morphism can be expressed uniquely as a pullback of this projection. \qed
\end{proposition}

\begin{corollary}
The canonical projection $\WWt \to \WW$ is weakly universal for $\alpha$-small morphisms of simplicial sets: any such morphism can be given, not necessarily uniquely, as a pullback of this projection.
\end{corollary}

\begin{proof}
By the well-ordering principle and the axiom of choice, one can well-order the fibers, and then use the universal property of $\WW$.
\end{proof}

\begin{definition}
 Let $\U \subseteq \W$ (respectively, $\UU \subseteq \WW$) be the subobject consisting of (isomorphism classes of) $\alpha$-small well-ordered fibrations\footnote{Here and throughout, by ``fibration'' we always mean ``Kan fibration''.}; and define $p_\alpha \colon \UUt \to \UU$ as the pullback:
 \[\xymatrix{ \UUt \ar[r] \ar[d]_{p_\alpha}  \pb & \WWt \ar[d] \\
  \UU \ar@{^{(}->}[r] & \WW
 }\]
\end{definition}

\begin{lemma}\label{U:Kan_fib}
 The map $p_\alpha \colon \UUt \to \UU$ is a fibration.
\end{lemma}

\begin{proof}
 Consider a horn to be filled
  \[\xymatrix{ \Lambda^k [n] \ar[r] \ar@{^{(}->}[d] & \UUt \ar[d]^{p_\alpha} \\
  \Delta [n] \ar[r]^{\name{x}} & \UU
 }\]
 for some $0 \leq k \leq n$.  It factors through the pullback
   \[\xymatrix{ \Lambda^k [n] \ar[r] \ar@{^{(}->}[d] & \bullet \ar[r] \ar[d]^{x} \pb & \UUt \ar[d]^{p_\alpha} \\
  \Delta [n] \ar@{=}[r] & \Delta [n] \ar[r]^{\name{x}} & \UU
 }\]
 where by the definition of $\UU$ and $\UUt$, $x$ is a fibration. Thus the left square admits a diagonal filler, and hence so does the outer rectangle.
\end{proof}

\begin{lemma}
 An $\alpha$-small well-ordered morphism $f \colon Y \to X \in \W (X)$ is a fibration if and only if $\name{f} \colon X \to \WW$ factors through $\UU$.
\end{lemma}

\begin{proof}
 For `$\Rightarrow$', assume that $f \colon Y \to X$ is a fibration. Then the pullback of $f$ to any representable is certainly a fibration:
 \[\xymatrix{ \bullet \ar[r] \ar[d]_{x^*f} \pb & Y \ar[d]^{f} \\
 \Delta [n] \ar[r]^x & X.
 }\]
 so $\name{f}(x) =\name{ x^*f} \in \UU$, and hence $\name{f}$ factors through $\UU$.

 Conversely, suppose $\name{f}$ factors through $\UU$. Then we obtain:
 \[\xymatrix{ Y \ar[r] \ar[d]_{f} \pb & \UUt \ar[r] \ar[d]^{p_\alpha} \pb & \WWt \ar[d] \\
 X \ar[r] & \UU \ar@{^{(}->}[r] & \WW,
 }\]
 where the lower composite is $\name{f}$, and the outer rectangle and the right square are by construction pullbacks.  Hence so is the left square; so by Lemma~\ref{U:Kan_fib} $f$ is a fibration.
\end{proof}

\begin{corollary} \label{cor:U_classifies}
The functor $\U$ is representable, represented by $\UU$; so ${p_\alpha} \colon \UUt \to \UU$ is strictly universal for $\alpha$-small well-ordered fibrations, and weakly universal for $\alpha$-small fibrations. \qed
\end{corollary}

In Section~\ref{subsec:pullback-reps}, we will strengthen this universal property, showing that while the representation of a fibration as a pullback of $p_\alpha$ may not be strictly unique, it is unique up to homotopy: precisely, the space of such representations is contractible.

\subsection{Kan fibrancy of the universe}  \label{subsec:fibrancy-of-u}

The previous section provides the main ingredients needed to use $\UU$ as a universe in the sense of Section~\ref{section:models-from-universes}, and hence to give a model of the core type theory.  However, to give additionally a type-theoretic universe within that model, we need to show that each $\UU$ itself can be seen as a \emph{type} of the model; in other words, that it is Kan.  The main goal of this section is therefore to prove the following theorem:

\begin{theorem}\label{U:KanCpx}
 The simplicial set $\UU$ is a Kan complex.
\end{theorem}

Before proceeding with the proof we will gather four useful lemmas.  The first two concern \emph{minimal fibrations}, which for the present purposes are a technical device whose details, beyond these two lemmas, are unimportant.

\begin{lemma}[Quillen's Lemma, {\cite{quillen:minimal}}] \label{Quillen_lemma}
Any fibration $f \colon Y \to X$ may be factored as $f = pg$, where $p$ is a minimal fibration and $g$ is a trivial fibration.
\end{lemma}

\begin{lemma}[{\cite[III.5.6]{barratt-gugenheim-moore}}; see also {\cite[Cor.~11.7]{may:simplicial-book}}] \label{lemma:may-lemma}
Suppose $X$ is contractible, with $x_0 \in X$, and $p \colon Y \to X$ is a minimal fibration with fiber $F := Y_{x_0}$. Then there is an isomorphism over $X$:
\[\xymatrix@C=0.5cm{
  Y \ar[rr]^<>(0.5)g \ar[rd]_p & & F \times X \ar[ld]^{\pi_2} \\
  & X. &
}\]
\end{lemma}

For Lemma~\ref{lemma:joyal-lemma}, the proof we give is due to Andr\'e Joyal; we include details here since the original \cite{joyal:kan} is not currently publicly available.  For this, and again for Theorem~\ref{thm:univalence} below, we make crucial use of exponentiation along cofibrations; so we pause first to establish some facts about this.

\begin{lemma}[Cf.~{\cite[Lemma~0.2]{joyal:kan}}] \label{lemma:exp_along_cofib}
For any map $i \colon A \to B$,
\begin{enumerate}[1.]
\item $\toposPi_i \colon \sSets/A \to \sSets/B$ preserves trivial fibrations;
\end{enumerate}
\noindent and if moreover $i$ is a cofibration, then: 
\begin{enumerate}[1.] \setcounter{enumi}{1}
\item the counit $i^* \toposPi_i \to 1_{\sSets/A}$ is an isomorphism;
\item if $p \colon E \to A$ is $\alpha$-small, then so is $\toposPi_i p$. \end{enumerate}
\end{lemma}

\begin{proof} \
\begin{enumerate}[1.]
\item By adjunction, since $i^*$ preserves cofibrations.

\item Since $i$ is mono, $i^* \toposSigma_i \iso 1_{\sSets/A}$; so by adjointness, $i^*\toposPi_i \iso 1_{\sSets/A}$.

\item For any $n$-simplex $x \colon \Delta[n] \to B$, we have $(\toposPi_i p)_x \iso \Hom_{\sSets/B}(x,\toposPi_i p) \iso \Hom_{\sSets/B}(i^*x,p)$.  As a subobject of $\Delta[n]$, $i^*x$ has only finitely many non-degenerate simplices, so $(\toposPi_i p)_x$ injects into a finite product of fibers of $p$ and is thus of size $< \alpha$. \qedhere
\end{enumerate} \end{proof}

\begin{lemma}[{\cite[Lemma~0.2]{joyal:kan}}] \label{lemma:joyal-lemma}
Trivial fibrations extend along cofibrations.  That is, if $t \colon Y \to X$ is a trivial fibration and $j \colon X \to X'$ is a cofibration, then there exists a trivial fibration $t' \colon Y' \to X'$ and a pullback square of the form:
 \[\xymatrix{ Y \ar@{.>}[r] \ar[d]_t \pb & Y' \ar@{.>}[d]^{t'} \\
 X \ar@{^{(}->}[r]^j & X'.
 }\]

Moreover, if $t$ is $\alpha$-small, then $t'$ may be chosen to also be.
\end{lemma}

\begin{proof}
Take $t' := \toposPi_j t$.  By part 1 of Lemma~\ref{lemma:exp_along_cofib}, this is a trivial fibration; by part 2, $j^*Y' \iso Y$; and by part 3, it is $\alpha$-small.
\end{proof}

We are now ready to prove that $\UU$ is a Kan complex.

\begin{proof}[Proof of Theorem~\ref{U:KanCpx}]
We need to show that we can extend any horn in $\UU$ to a simplex:
 \[\xymatrix{ \Lambda^k[n] \ar[r] \ar@{^{(}->}[d] & \UU \\
  \Delta [n] \ar@{.>}[ur] & \\
 }\]
By Corollary~\ref{cor:U_classifies}, any such horn $\name{q}$ corresponds to an $\alpha$-small well-ordered fibration $q \colon Y \to \Lambda^k [n]$.  To extend $\name{q}$ to a simplex, we just need to construct an $\alpha$-small fibration $Y'$ over $\Delta[n]$ which restricts on the horn to $Y$:
\[\xymatrix{
  Y \ar@{.>}[r] \ar[d]_q \pb & Y' \ar@{.>}[d]^{q'} \\
 \Lambda^k[n] \ar@{^{(}->}[r] & \Delta[n].
}\]
By the axiom of choice one can then extend the well-ordering of $q$ to $q'$, so the map $\name{q'} \colon \Delta [n] \to \UU$ gives the desired simplex.

By Quillen's Lemma, we can factor $q$ as
\[\xymatrix{ Y \ar[r]^{q_t} & Y_0 \ar[r]^{q_m} & \Lambda^k[n],}\]
where $q_t$ is a trivial fibration and $q_m$ is a minimal fibration.  Both are still $\alpha$-small: each fiber of $q_t$ is a subset of a fiber of $q$, and since a trivial fibration is onto, each fiber of $q_m$ is a quotient of a fiber of $q$.

By Lemma~\ref{lemma:may-lemma}, we have an isomorphism $Y_0 \cong F \times \Lambda^k[n]$, and hence a pullback diagram:
 \[\xymatrix{Y_0 \ar@{^{(}->}[r] \ar[d] \pb & F \times \Delta[n] \ar[d] \\
 \Lambda^k[n] \ar@{^{(}->}[r] & \Delta[n]
 }\]

By Lemma~\ref{lemma:joyal-lemma}, we can then complete the upper square in the following diagram, with both right-hand vertical maps $\alpha$-small fibrations:
 \[\xymatrix{ Y \ar[d]_{q_t} \ar[r] \pb & Y' \ar[d] \\
  Y_0 \ar@{^{(}->}[r] \ar[d]_{q_m} \pb & F \times \Delta[n] \ar[d] \\
 \Lambda^k[n] \ar@{^{(}->}[r] & \Delta[n]
 }\]

  Since $\alpha$ is regular, the composite of the right-hand side is again $\alpha$-small; so we are done.
\end{proof}


\subsection{Modelling type theory in simplicial sets} \label{subsec:model-in-ssets}

To prove that $\UU$ carries the structure to model type theory, we will need a couple of further lemmas; firstly, that taking dependent products preserves fibrations:

\begin{lemma} \label{lemma:dep-prod-of-fibs}
Suppose $Z \to^q Y \to^p X$ are fibrations.  Then the dependent product $\toposPi_p q$ is a fibration over $X$.
\end{lemma}

\begin{proof}
The pullback functor $p^* \colon \sSets/X \to \sSets/Y$ preserves trivial cofibrations (since $\sSets$ is right proper and cofibrations are monomorphisms); so its right adjoint $\toposPi_p$ preserves fibrant objects.
\end{proof}

Secondly, to model $\synId$-types, we will require well-behaved fibered path objects.  The construction below may be found in \cite[Thm.~2.25]{warren:thesis}; we recall it in more elementary terms, which will be useful to us later.

\begin{definition}
Given a fibration $p \colon E \to B$, define the \emph{fibered path object} $\paths_B(E)$ as the pullback
\[\xymatrix{
  \paths_B(E) \ar[r] \ar[d] \pb & E^{\Delta[1]} \ar[d]^-{p^{\Delta[1]}} \\
  B \ar[r]^-{c} & B^{\Delta[1]},
}\]
the object of paths in $E$ that are constant in $B$.

The “constant path” map $c \colon E \to E^{\Delta[1]}$ factors through $\paths_B(E)$; call the resulting map $r_p \colon E \to \paths_B(E)$.  There are also evident source and target maps $s_p,t_p \colon \paths_B(E) \to E$.  (On all of these maps, we will omit the subscripts when they are clear from context.)
\end{definition}

\begin{proposition} \label{prop:stable-fibered-path-obs}
For any fibration $p \colon E \to B$, the maps 
\[E \to^r \paths_B(E) \to^{(s,t)} E \times_B E\]
give a factorisation of the diagonal map $\Delta_p \colon E \to E \times_B E$ over $B$ as a (trivial cofibration, fibration); and this is \emph{stable over $B$} in that the pullback along any $B' \to B$ is again such a factorisation. 
\end{proposition}

\begin{proof}
It is clear that these maps give a factorisation of $\Delta_p$ over $B$.  To see that they are a trivial cofibration and a fibration respectively, consider the pullback construction of $\paths_B(E)$ via two intermediate stages:
\[\xymatrix{
  \paths_B(E) \ar[d]_-{(s,t)} \ar[r] \pb & E^{\Delta[1]} \ar[d]^-{(s,p^{\Delta[1]},t)} \\
  E \times_B E \ar[d]_-{\pi_1} \ar[r] \pb & E \times_B B^{\Delta[1]} \times_B E \ar[d]^-{(\pi_1,\pi_2)} \\
  E \ar[d] \ar[d] \ar[r] \pb & E \times_B B^{\Delta[1]} \ar[d] \\
  B \ar[r]^{c} & B^{\Delta[1]} 
}\]

Now $(s,t)$ is certainly a fibration, since it is a pullback of the map $E^{\Delta[1]} \to E \times_B B^{\Delta[1]} \times_B E \iso E^{1+1} \times_{B^{1+1}} B^{\Delta[1]}$, which is a fibration by the monoidal model category axioms \cite[Lemma~4.2.2(3)]{hovey:book}, applied to the cofibration $1+1 \to \Delta[1]$ and the fibration $p$.

Similarly, the source map $s \colon \paths_B(E) \to E$ is a trivial fibration, since it is a pullback of $E^{\Delta[1]} \to E^1 \times_{B^1} B^{\Delta[1]}$, which is one by the monoidal model category axioms.  But $s$ is a retraction of $r$, so $r$ is a weak equivalence (by 2-out-of-3) and a monomorphism, so is a trivial cofibration as desired.

Finally, stability of these properties under pullback follows immediately from the stability (up to isomorphism) of the construction itself: for any $f \colon B' \to B$, there is a canonical isomorphism $\paths_{B'} (f^*E) \iso f^* \paths_B(E)$, commuting with the maps $r,s,t$.
\end{proof}

We are now fully equipped for the main result of the present section:
\begin{theorem} \label{thm:the-model-in-ssets}
Let $\alpha$ be an inaccessible cardinal\footnote{I.e.\ infinite, regular, and strong limit; \emph{strongly inaccessible} in some literature.}. Then $\UU$ carries $\PiStrux$-, $\SigmaStrux$-, $\IdStrux$-, $\WStrux$-, $\oneStrux$-, $\zeroStrux$-, and $\plusStrux$-structures.  

Moreover, if $\beta < \alpha$ is also inaccessible, then $\UU[\beta]$ gives an internal universe in $\UU[\alpha]$ closed under all these constructors.
\end{theorem}

\begin{proof}
($\PiStrux$-structure): Given a pair of $\alpha$-small fibrations $Z \to^q Y \to^p X$, the dependent product $\toposPi_p q$ in $\sSets/X$ is again a fibration, by Lemma~\ref{lemma:dep-prod-of-fibs}; it is also $\alpha$-small, since $\alpha$ is inaccessible.

Hence by Corollary~\ref{cor:U_classifies}, the universal dependent product over $\UU^{\synPi\text{-\form}}$ is representable as the pullback of $\UUt$ along some map $\PiStrux \colon \UU^{\synPi\text{-\form}} \to \UU$, giving the desired $\PiStrux$-structure.

($\SigmaStrux$-structure): Similarly, given $\alpha$-small fibrations $Z \to^q Y \to^p X$, the composite $p \cdot q$ is again an $\alpha$-small fibration.  So the universal dependent sum over $\UU^{\synSigma\text{-\form}}$ is representable by some map $\SigmaStrux \colon \UU^{\synSigma\text{-\form}} \to \UU$.

($\IdStrux$-structure): Given any $\alpha$-small fibration $p \colon Y \to X$, consider the factorisation of the diagonal $\Delta_p$ as $Y \to^{r} \paths_X(Y) \to^{(s,t)} Y \times_X Y$.  The fibration $(s,t)$ is easily seen to be $\alpha$-small; and by Proposition~\ref{prop:stable-fibered-path-obs}, $r$ is stably orthogonal to $(s,t)$ over $X$.

Applying this construction to $p_\alpha \colon \UUt \to \UU$ itself yields, via Proposition~\ref{prop:lifting-op-iff-stably-orthog}, the desired $\IdStrux$-structure on $\UU$.

($\WStrux$-structure): Given $\alpha$-small fibrations $Z \to^q Y \to^p X$, the initial algebra $W_q \to X$ for the induced polynomial endofunctor on $\sSets/X$ may be obtained as a transfinite colimit of iterations of the endofunctor; it can be shown from this description that it is again an $\alpha$-small fibration (cf.~\cite[Thm.~3.4]{moerdijk-van-den-berg:w-types-in-hott} and \cite{moerdijk-van-den-berg:w-types-in-hott-corr}).

($\zeroStrux$-structure), ($\oneStrux$-structure), ($\plusStrux$-structure): straightforward.

(Internal universe.)  Since $\beta < \alpha$, $\UU[\beta]$ is itself $\alpha$-small; and by Theorem~\ref{U:KanCpx}, it is Kan.  So $\UU[\beta]$ is representable as the pullback of $\UUt$ along some $u_\beta \colon 1 \to \UU$.  Moreover, there is a natural inclusion $i \colon \UU[\beta] \to \UU$, with $\UUt[\beta] \iso i^* \UUt$ by construction.  Together these give the desired internal universe $(u_\beta, i)$.

Finally, to see that $(u_\beta,i)$ is closed under the appropriate constructors in $i$, note that for each of $\PiStrux$, $\SigmaStrux$, and $\IdStrux$ as constructed above, the image of the composite with $i$ lies again in $\UU[\beta]$, and hence factors through $i$; for instance, in the case of $\PiStrux$,
\[ \xymatrix{ \UU[\beta]^{\synPi\text{-\form}} \ar[r]^{i^{\synPi\text{-\form}}} \ar@{.>}[d]^{\PiStrux} & \UU^{\synPi\text{-\form}} \ar[d]^\PiStrux \\
              \UU[\beta] \ar[r]^i & \UU } \]

(Note that while we do already have a $\PiStrux$-structure (and so on) on $\UU[\beta]$ as constructed in the first parts of this theorem, those choices of the structure do not automatically commute with $i$.)  
\end{proof}

\begin{corollary} \label{cor:simplicial-model}
Let $\beta < \alpha$ be inaccessible cardinals.  Then there is a model of dependent type theory in $\sSets_{\UU}$ with all the logical constructors of Section~\ref{subsec:logical-rules}, and a universe (given by $\UU[\beta]$) closed under these constructors. \qed
\end{corollary}

Assuming initiality (Conjecture~\ref{conj:initiality}), this implies the existence of a morphism $\CC(\TT) \to \sSets_{\UU}$, interpreting the syntax of Martin-Löf type theory in simplicial sets.  Even without assuming initiality, it gives us the operations with which to heuristically interpret individual judgements of the syntax by hand.  We therefore freely make notational use of the interpretation, writing $\interp{ \mathcal{J} }$ for the interpretation of a judgement $\mathcal{J}$.

In doing so, we will make several systematic abuses of notation.  Firstly, referring in the syntax to fibrations, we will write $E$ rather than $\name{E}$, and so on, whenever some choice of name $\name{E} \colon B \to \UU$ for the fibration is understood; and conversely, referring to the interpretation of a type $\Gamma \types T\ \type$, we use $\interp{T}$ to refer to the fibration over $\interp{\Gamma}$ given by pulling back $\UUt$ along the literal interpretation $\interp{\Gamma \types T\ \type} \colon \interp{\Gamma} \to \UU$.

As a first characteristic of the model, we note that both of the extra principles on equality of functions hold.

\begin{proposition} \label{prop:eta-and-funext}
The $\eta$-rule and functional extensionality rules of Section~\ref{subsec:optional-rules} hold in the simplicial model.
\end{proposition}

\begin{proof}
The $\eta$-rule follows immediately from our use of categorical exponentials to interpret $\synPi$-types, by the uniqueness in the categorical universal property.

For functional extensionality, Garner \cite[Sec.~5]{garner:on-the-strength} shows that it holds just if each product of identity types, 
\[ f, g \oftype \synPi_{x \oftype A} B(x) \types \synPi_{x \oftype A} \synId_{B(x)} (\syn{app}(f,x),\syn{app}(g,x))\ \type\]
admits the structure given by the rules for the identity type on the corresponding product types,
\[f,g \oftype \synPi_{x \oftype A} B(x) \types \synId_{\synPi_{x \oftype A} B(x)} (f,g)\ \type . \]

So it is enough to show that for any pair of ($\alpha$-small, well-ordered) fibrations $Z \to^q Y \to^p X$, given by names
\[ \name{Y} \colon X \to \UU, \qquad \name{Z} \colon Y \to \UU,\]
the interpretation of the product of identity types
\[ \interp{\synPi_{x \oftype Y} \synId_{Z(x)} (\syn{app}(f,x),\syn{app}(g,x)) } \iso \toposPi_p (\paths_Y Z),\]
 gives a suitably stable path object for the interpretation of the product types, 
\[ \interp{\synId_{\synPi_{x \oftype Y} Z(x)} (f,g)} \iso \toposPi_p Z.\]

For this, it is clear that $\toposPi_p (s,t) \colon \toposPi_p (\paths_Y Z) \to \toposPi_p (Z \times_Y Z) \iso \toposPi_p Z \times_X \toposPi_p Z$ is a fibration, since $\toposPi_p$ preserves fibrations (Lemma~\ref{lemma:dep-prod-of-fibs}). Similarly, $\toposPi_p r_q \colon \toposPi_p Z \to \toposPi_p (\paths_Y Z)$ is a cofibration since $\toposPi_p$ preserves monomorphisms; and it is a weak equivalence, since $\toposPi_p$ preserves trivial fibrations (Lemma~\ref{lemma:joyal-lemma}), and so the retraction $\toposPi_p s_q \colon \toposPi_p (\paths_Y Z) \to \toposPi_p Z$ is again a trivial fibration.  Finally, by the Beck-Chevalley condition in an LCCC, the entire construction is stable under pullback in $X$, as required.
\end{proof}

It now remains only to show that the Univalence Axiom holds in this model.

%%% Local Variables: 
%%% mode: latex
%%% TeX-master: "simplicial-model"
%%% End: 
