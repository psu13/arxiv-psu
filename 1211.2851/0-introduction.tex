\section*{Introduction}

The Univalent Foundations programme is a new proposed approach to foundations of mathematics, originally suggested by Vladimir Voevodsky in \cite{voevodsky:homotopy-lambda-calculus}, building on the systems of dependent type theory developed by Martin-Löf and others.

A major motivation for earlier work with such logical systems has been their well-suitedness to computer implementation.  One notable example is the Coq proof assistant, based on the Calculus of Inductive Constructions (a closely related dependent type theory), which has shown itself feasible for large-scale formal verification of mathematics, with developments including formal proofs of the Four-Colour Theorem \cite{gonthier:four-color} and the Feit-Thompson (Odd Order) Theorem \cite{gonthier:feit-thompson}.

One feature of dependent type theory which has previously remained comparatively unexploited, however, is its richer treatment of equality.  In traditional foundations, equality carries no information beyond its truth-value: if two things are equal, they are equal in at most one way.  This is fine for equality between elements of discrete sets; but it is unnatural for objects of categories (or higher-dimensional categories), or points of spaces.  In particular, it is at odds with the informal mathematical practice of treating isomorphic (and sometimes more weakly equivalent) objects as equal; which is why this usage must be so often disclaimed as an abuse of language, and kept rigorously away from formal statements, even though it is so appealing.

In dependent type theory, equalities can carry information: two things may be equal in multiple ways.  So the basic objects—the \emph{types}—may behave not just like discrete sets, but more generally like higher groupoids (with equalities being morphisms in the groupoid), or spaces (with equalities being paths in the space).  And, crucially, this is the \emph{only} equality one can talk about within the logical system: one cannot ask whether elements of a type are “equal on the nose”, in the classical sense.\footnote{There is a strict equality, called \emph{judgemental} or \emph{definitional}, but there is no \emph{type/proposition} in the system expressing it, just as with e.g.\ literal syntactic equality traditional foundations.}  The logical language only allows one to talk about properties and constructions which respect its equality.

The \emph{Univalence Axiom}, introduced by Voevodsky, strengthens this characteristic.  In classical foundations one has sets of sets, or classes of sets, and uses these to quantify over classes of structures.  Similarly, in type theory, types of types—\emph{universes}—are a key feature of the language.  The Univalence Axiom states that equality between types, as elements of a universe, is the same as equivalence between them, as types.  It formalises the practice of treating equivalent structures as completely interchangeable; it ensures that one can only talk about properties of types, or more general structures, that respect such equivalence.  In sum, it helps solidify the idea of types as some kind of spaces, in the homotopy-theoretic sense; and more practically—its original motivation—it provides for free many theorems (transfer along equivalences, naturality with respect to these, and so on) which must otherwise be re-proved by hand for each new construction.

The main goal of this paper is to justify the intuition outlined above, of types as spaces.
%
To this end, we focus on the Quillen model category $\sSets$ of simplicial sets, a well-studied model for topological spaces in homotopy theory; we construct a model of type theory in $\sSets$, and show that it satisfies the Univalence Axiom.
%
The fibrations of this model structure, called Kan fibrations, will serve as an interpretation of type dependency.
% 
In particular the closed types will be interpreted as Kan complexes, which also serve as a model for $\infty$-groupoids, for instance in Joyal and Lurie's approach to higher category theory.

It follows from this model that Martin-Löf type theory plus the Univalence Axiom (presented in terms of contextual categories) is consistent, provided that the classical foundations we use are---precisely, ZFC together with the existence of two strongly inaccessible cardinals, or equivalently two Grothendieck universes.

As hinted above, there is one important technical caveat regarding our treatment of type theory: we state the model and consistency results in terms of contextual categories, not syntax, so as to avoid reliance on initiality results.

This paper therefore includes a mixture of logical and homotopy-theoretic ingredients; however, we have aimed to separate the two wherever possible.  Good background references for the logical parts include \cite{n-p-s:programming}, a general introduction to the type theory; \cite{hofmann:syntax-and-semantics}, for the categorical semantics; and \cite{martin-lof:bibliopolis}, the \emph{locus classicus} for the logical rules.  For the homotopy-theoretic aspects, \cite{goerss-jardine} and \cite{hovey:book} are both excellent and sufficient references.  Finally, for the category-theoretic language used throughout, \cite{mac-lane:cwm} is canonical.

\paragraph{Organisation} In Section~\ref{section:models-from-universes} we consider general techniques for constructing models of type theory. After setting out (in Section~\ref{subsec:the-type-theory}) the specific type theory that we will consider, we review (Section~\ref{subsec:contextual-cats}) some fundamental facts about its intended semantics in contextual categories, following \cite{streicher:book}.  In Section~\ref{subsec:contextualization}, we use universes to construct contextual categories, representing the structural core of type theory; and in Section~\ref{subsec:logical-structure-on-universes}, we use categorical constructions on the universe to model the logical constructions of type theory.  Together, these present a new solution to the \emph{coherence problem} for modelling type theory (cf.\ \cite{hofmann:on-the-interpretation}).

In Section~\ref{section:the-model}, we turn towards constructing a model in the category of simplicial sets. Sections~\ref{subsec:representability-of-fibs} and \ref{subsec:fibrancy-of-u} are dedicated to the construction and investigation of a (weakly) universal Kan fibration (a “universe of Kan complexes”); in Section~\ref{subsec:model-in-ssets} we use this universe to apply the techniques of Section~\ref{section:models-from-universes}, giving a model of the full type theory in simplicial sets.

Section~\ref{section:univalence} is devoted to the Univalence Axiom.   We formulate univalence first in type theory (Section~\ref{subsec:type-theoretic-univalence}), then directly in homotopy-theoretic terms (Section~\ref{subsec:simplicial-univalence}), and show that these definitions correspond under the simplicial model (Section~\ref{subsec:univalence-equivalence}).  In Section~\ref{subsec:univalence-of-uu}, we show that the universal Kan fibration is univalent, and hence that the Univalence Axiom holds in the simplicial model.  Finally, in Section~\ref{subsec:pullback-reps} we discuss an alternative formulation of univalence, shedding further light on the universal property of the universe.

We include also two appendices, setting out in full the type theory under consideration: first a conventional syntactic presentation in Appendix~\ref{app:type-theory}, and then in Appendix~\ref{app:cxl-structure} its translation into algebraic structure on contextual categories.

\paragraph*{History of the paper}
This paper started life as notes by the current authors based on Vladimir Voevodsky's lectures at the 2011 Oberwolfach workshop \cite{oberwolfach-report} along with his associated manuscript \cite{voevodsky:notes-on-type-systems}.
It was subsequently expanded with Voevodsky’s collaboration into the present full exposition of the simplicial model, and appeared as a preprint in 2012 with Voevodsky included as co-author.

In 2016, due to his dissatisfaction with the existing literature on type theory, which this paper took as background, Voevodsky asked us to remove him as co-author and delay publication until he had finished developing his own treatment of semantics of type theory (cf.~\cite{voevodsky:c-system-of-module} and sequels) and presentation of the simplicial model in that framework.

Tragically, Voevodsky passed away in September 2017, before completing that project.  This paper therefore remains the only account of Voevodsky's construction of the simplicial model, so with the support of Daniel Grayson, Voevodsky’s academic executor, we have prepared it again for publication.  We have made several changes to accommodate Voevodsky's reservations regarding the treatment of semantics; most importantly, we present the initiality of syntax as a conjecture rather than a theorem (Conjecture \ref{conj:initiality}), and give all main results in terms of contextual categories rather than syntax.  Otherwise, the paper remains substantially unchanged from the original 2012 version.

The main results of the paper are due to Voevodsky, including Theorems~\ref{thm:structure-on-U-to-CU}, \ref{thm:the-model-in-ssets}, \ref{thm:simplicial-model-univalent} and \ref{thm:univalent-characterization}. Mathematical contributions of Kapulkin and Lumsdaine include all of Section~\ref{subsec:univalence-equivalence}, along with streamlining various parts of the main constructions and completing portions omitted in \cite{voevodsky:notes-on-type-systems}.

Out of respect for Voevodsky’s stated wishes, and following discussion with his executor, he remains absent as an author of the final version of this paper. 
%
However, we wish to leave no doubt regarding the share of the credit that is his due.
%
We mourn the loss of an exceptional mathematician and mentor, and dedicate this paper to his memory.

\paragraph*{Related work}
While the present paper discusses just models of type theory with the univalence axiom, the major motivation for this is the actual development of mathematics within these foundations. Introductions to various aspects of this are given in \cite{grayson:intro-to-univalent-foundations}, \cite{voevodsky:experimental-library}, \cite{pelayo-warren:univalent-foundations-paper}, and \cite{hott:book},
while large computer-formalised developments include the UniMath\footnote{\url{https://www.github.com/UniMath/UniMath}} and HoTT\footnote{\url{https://www.github.com/HoTT/HoTT}} libraries, presented in \cite{voevodsky-ahrens-grayson:unimath} and \cite{bauer-et-al:hott-library}.

Earlier work on homotopy-theoretic models of type theory can be found in \cite{hofmann-streicher}, \cite{awodey-warren}, \cite{warren:strict-w-groupoid-model}.  Other current and recent work on such models includes \cite{garner-van-den-berg}, \cite{arndt-kapulkin}, and \cite{shulman:inverse-diagrams}.  Other general coherence theorems, for comparison with the results of Section~\ref{section:models-from-universes}, can be found in \cite{hofmann:on-the-interpretation} and \cite{lumsdaine-warren:local-universes}.  Univalence in homotopy-theoretic settings is also considered in %\cite{moerdijk:univalence} and 
\cite{gepner-kock:univalence}.  (These references are, of course, far from exhaustive.)

\paragraph*{Acknowledgements} 
First and foremost we would like to thank Vladimir Voevodsky: the creator of Univalent Foundations, a mentor to both the authors, and whose insight and ingenuity underlie not only the present paper but much subsequent work in the field.
We are particularly indebted also to Michael Warren, whose illuminating seminars and discussions heavily influenced our understanding and presentation of the simplicial model.
We also thank Daniel Grayson, Ieke Moerdijk, Mike Shulman, Raffael Stenzel, and Karol Szumi{\l}o, for helpful correspondence, conversations, and corrections to drafts and earlier versions, and Steve Awodey, for support and encouragement.

The first-named author was financially supported during this work by the NSF, Grant DMS-1001191 (P.I.~Steve Awodey), and by a grant from the Benter Foundation (P.I.~Thomas C.~Hales); the second-named author, by an AARMS postdoctoral fellowship at Dalhousie University, and grants from NSERC (P.I.’s~Peter Selinger, Robert Dawson, and Dorette Pronk).

%%% Local Variables:
%%% mode: latex
%%% TeX-master: "simplicial-model"
%%% End: 