% Note: standard ordering of rules we currently follow throughout is:
% Pi; Sigma; Id; W; 0; 1; +; universes.

\section{Syntax of Martin-L\"of Type Theory} \label{app:type-theory}

A full introduction to Martin-Löf type theory is beyond the scope of this paper.
%
For the reader new to type theory, we recommend \cite{martin-lof:bibliopolis} or \cite{n-p-s:programming} as a general introduction, and \cite{hofmann:syntax-and-semantics} for a more detailed presentation of the syntax.

However, there are many variant presentations of the theory; in this appendix, we lay out the one we have in mind for the present paper, and which is intended to correspond to the contextual categories described in Appendix~\ref{app:cxl-structure}.

We consider the syntax as constructed in two stages: first the \emph{raw} or \emph{untyped} syntax of the theory---the set of expressions that are at least parseable, but not necessarily meaningful---and then the \emph{derivable judgements}, certain inductively-generated predicates picking out the genuinely meaningful contexts, types, and terms.

The raw syntax may be constructed as certain strings of symbols, or alternatively, certain labelled trees. On this, one then defines \emph{$alpha$-equivalence} (i.e.\ syntactic identity modulo renaming of bound variables), and the operation of \emph{(capture-free) substitution}.  This step is well-standardised in the literature.

At the second stage, one defines on the raw syntax several multi-place relations, picking out the \emph{derivable judgements} of the theory.  For instance, “$\Gamma \types a : A$” will be a relation on triples $(\Gamma, a, A)$ of a raw context, term, and type expression respectively, to be read as “$a$ is a term of type $A$, in context $\Gamma$”.  These relations are defined by mutual induction, as the smallest family of relations closed under a bevy of specified closure conditions, the \emph{inference rules} of the theory.

Details of the judgements and inference rules used vary somewhat; we therefore set our choice out here in full.  For the structural rules, our presentation is based largely on \cite{hofmann:syntax-and-semantics}; our selection of logical rules, and in particular our treatment of the universe, follows \cite{martin-lof:bibliopolis}.

We take as basic four judgement forms:
\begin{mathpar}
  \Gamma \types A\ \type
\and
  \Gamma \types A = A'\ \type
\and
  \Gamma \types a : A
\and
  \Gamma \types a = a' : A.
\end{mathpar}
We take the context judgement as defined from these: that is, if $\Gamma$ is a list $(x_i \oftype A_i)_{i < n}$, with $x_i$ distinct variables and $A_i$ raw type expressions, then $\types \Gamma\ \cxt$ is an abbreviation for the statement that for each $i < n$, $(x_j \oftype A_j)_{j < i} \types A_i\ \type$.

\emph{Derivability}, for these four basic judgements, is then defined as the smallest family of relations closed under the closure conditions specified by the inference rules below.
%
These are given in “rule-notation”, so for instance
\[
  \inferrule*{\Gamma \types a : A \\ \Gamma \types A = B \ \type}{\Gamma \types a : B}
\]
expresses the closure condition “for all suitable raw expressions $\Gamma$, $a$, $A$, $B$, if the judgements $\Gamma \types a : A$ and $\Gamma \types A = B$ are derivable, then so is $\Gamma \types a : B$”.

The inference rules fall into two groups: the \emph{structural rules}, which we assume are always included, and the \emph{logical rules}, which different type theories may include different subsets of.

\subsection{Structural Rules} \label{subsec:structural-rules}
The structural rules of the type theory are (where $\mathcal{J}$ may be the conclusion of any of the judgement forms):

\begin{mathpar}
  \inferrule*[right=$\Vble$]{\types \Gamma,\ x \oftype A,\ \Delta\ \cxt}{\Gamma,\ x \oftype A,\ \Delta \types x : A}
\and
  \inferrule*[right=$\Subst$]{\Gamma \types a : A \\ \Gamma,\ x \oftype A,\ \Delta \types \mathcal{J}}{\Gamma,\ \Delta[a/x] \types \mathcal{J}[a/x]}
\and
  \inferrule*[right=$\Weak$]{\Gamma \types A\ \type \\ \Gamma,\ \Delta \types \mathcal{J}}{\Gamma,\ x \oftype A,\ \Delta \types \mathcal{J}}
\end{mathpar}

Definitional equality (also known as syntactic or judgemental equality):
\begin{mathparpagebreakable}
  \inferrule*{\Gamma \types A\ \type}{\Gamma \types A = A\ \type} 
\and
  \inferrule*{\Gamma \types A=B\ \type}{\Gamma \types B = A\ \type}
\and
  \inferrule*{\Gamma \types A=B\ \type \\ \Gamma \types B=C \ \type}{\Gamma \types A = C\ \type}
\and
  \inferrule*{\Gamma \types a : A}{\Gamma \types a = a : A}
\and
  \inferrule*{\Gamma \types a=b : A}{\Gamma \types b=a : A}
\and
  \inferrule*{\Gamma \types a=b : A \\ \Gamma \types b=c : A}{\Gamma \types a=c : A}
\and
  \inferrule*{\Gamma \types a : A \\ \Gamma \types A = B \ \type}{\Gamma \types a : B}
\and
  \inferrule*{\Gamma \types a = b : A \\ \Gamma \types A = B \ \type}{\Gamma \types a = b : B}
\end{mathparpagebreakable}

\subsection{Logical Constructors} \label{subsec:logical-rules}

In this and subsequent sections, we present rules introducing various type- and term-constructors.  For each such constructor, we assume (besides the explicitly stated rules introducing and governing it) a \emph{congruence} rule stating that it preserves definitional equality in each of its arguments; for instance, along with the $\synPi$-\intro\ rule introducing the constructor $\lambda$, we assume the rule
\[\inferrule*[right=$\lambda$-eq]{\Gamma \types A = A'\ \type \\ \Gamma,\ x \oftype A \types B(x) = B'(x)\ \type \\ \Gamma,\ x \oftype A \types b(x) = b'(x) : B(x)}{\Gamma \types \lambda x \oftype A.b(x) = \lambda x \oftype A'.b'(x) : \synPi_{x \oftype A} B(x)}\]

The rules fall naturally into groups according to the various logical constructors.  Many of the constructors considered ($\synSigma$-, $\synId$-, $\synW$-, $\synZero$-, $\synOne$-, and $+$-types) follow a common pattern, as \emph{inductive types/families} with constructors and an elimination principle; here, as in \cite{martin-lof:bibliopolis}, this pattern is purely heuristic, but in approaches such as the Calculus of Inductive Constructions \cite{werner:thesis} they are unified formally as instances of a single scheme.
 
\paragraph{$\synPi$-types} (Dependent products; dependent function types).

\begin{mathparpagebreakable}
  \inferrule*[right=$\synPi$-\form]{\Gamma,\ x \oftype A \types B(x)\ \type}{\Gamma \types \synPi_{x \oftype A} B(x)\ \type}
\and
  \inferrule*[right=$\synPi$-\intro]{\Gamma,\ x \oftype A \types B(x)\ \type \\ \Gamma,\ x \oftype A \types b(x) : B(x)}{\Gamma \types \lambda x \oftype A.b(x) : \synPi_{x \oftype A} B(x)}
\and
  \inferrule*[right=$\synPi$-\appRule]{\Gamma \types f \oftype \synPi_{x \oftype A} B(x) \\ \Gamma \types a : A}{\Gamma \types \app (f, a) : B(a)}
\and
  \inferrule*[right=$\synPi$-\comp]{\Gamma,\ x \oftype A \types B(x)\ \type \\ \Gamma,\ x \oftype A \types b(x) : B(x) \\ \Gamma \types a : A}{\Gamma \types \app(\lambda x \oftype A .b(x), a)=b(a) : B(a)}
\end{mathparpagebreakable}

As a special case of this, when $B$ does not depend on $x$, we obtain the ordinary function type $[A, B] := \synPi_{x \oftype A} B$. \\

\paragraph*{$\synSigma$-types}  (Dependent sums; type-indexed disjoint sums.)

\begin{mathparpagebreakable}
  \inferrule*[right=$\synSigma$-\form]{\Gamma \types A\ \type \\ \Gamma,\ x \oftype A \types B(x)\ \type}{\Gamma \types \synSigma_{x \oftype A} B(x)\ \type}
\and
  \inferrule*[right=$\synSigma$-\intro]{\Gamma \types A\ \type \\ \Gamma,\ x \oftype A \types B(x)\ \type}{\Gamma,\ x \oftype A,\ y \oftype B(x) \types \pair (x, y) : \synSigma_{x \oftype A} B(x)}
\and
  \inferrule*[right=$\synSigma$-\elim]{\Gamma,\ z \oftype \synSigma_{x \oftype A} B(x) \types C(z)\ \type \\ \Gamma,\ x \oftype A,\ y \oftype B(x) \types d(x,y) : C(\pair(x, y))}{\Gamma,\ z \oftype \synSigma_{x \oftype A} B(x) \types \synsplit_d (z) : C(z)}
\and
  \inferrule*[right=$\synSigma$-\comp]{\Gamma,\ z \oftype \synSigma_{x \oftype A} B(x) \types C(z)\ \type \\ \Gamma,\ x \oftype A,\ y \oftype B(x) \types d(x,y) : C(\pair(x, y))}{\Gamma,\ x \oftype A,\ y \oftype B(x) \types \synsplit_d (\pair (x,y))=d(x,y) : C(\pair (x, y))}
\end{mathparpagebreakable}

Again,  the special case where $B$ does not depend on $x$ is of particular interest: this gives the cartesian product $A \times B := \synSigma_{x \oftype A} B$. \\

\paragraph*{$\synId$-types.}  (Identity types, equality types.)

\begin{mathparpagebreakable}
  \inferrule*[right=$\synId$-\form]{\Gamma \types A\ \type}{\Gamma,\ x,y\oftype A \types \synId_A(x,y)\ \type}
\and
  \inferrule*[right=$\synId$-\intro]{\Gamma \types A\ \type}{\Gamma,\ x\oftype A \types \refl_A(x):\synId_A(x,x)}
\and
  \inferrule*[right=$\synId$-\elim]{\Gamma,\ x,y\oftype A,\ u\oftype \synId_A(x,y) \types C(x,y,u)\ \type \\ \Gamma,\ z\oftype A \types d(z):C(z,z,\refl_A(z))}{\Gamma,\ x,y\oftype A,\ u\oftype \synId_A(x,y) \types \synJ_{z. d}(x,y,u) : C(x,y,u)}
\and
  \inferrule*[right=$\synId$-\comp]{\Gamma,\ x,y\oftype A,\ u\oftype \synId_A(x,y) \types C(x,y,u)\ \type \\ \Gamma,\ z\oftype A \types d(z):C(z,z,r(z))}{\Gamma,\ x\oftype A \types \synJ_{z.d}(x,x,\refl_A(x)) = d(x) : C(x,x,\refl_A(x))}
\end{mathparpagebreakable}

\paragraph*{$\synW$-types.}  (Types of well-founded trees; free term algebras.)

\begin{mathparpagebreakable}
  \inferrule*[right=$\synW$-\form]{\Gamma,\ x \oftype A \types B(x)\ \type}{\Gamma \types \synW_{x \oftype A} B(x)\ \type}
\and
  \inferrule*[right=$\synW$-\intro]{\Gamma,\ x \oftype A \types B(x)\ \type}{\Gamma, \ x \oftype A, \ y \oftype [B(x), \synW_{u \oftype A} B(u)] \types \synsup(x, y) : \synW_{u \oftype A} B(u)}
\\
  \mathclap{\inferrule*[right=$\synW$-\elim]
    {\Gamma, \ w \oftype \synW_{x \oftype A} B(x) \types C(w) \ \type \\
     \Gamma,\ x \oftype A,\ y \oftype [B(x), \synW_{u \oftype A} B(u)],\ z \oftype \synPi_{u \oftype B(x)} C(\app(y, u)) \hspace{1.55cm} \\
     \hspace{6cm} \types d(x, y, z) : C(\synsup(x, y))}
  {\Gamma,\ w \oftype \synW_{x \oftype A} B(x) \types \wrec_{d} (w) : C(w)}}
\\
  \mathclap{\inferrule*[right=$\synW$-\comp]
  {\Gamma, \ w \oftype \synW_{x \oftype A} B(x) \types C(w) \ \type \\
   \Gamma,\ x \oftype A,\ y \oftype [B(x), \synW_{u \oftype A} B(u)],\ z \oftype \synPi_{u \oftype B(x)} C(\app(y, u)) \hspace{2cm} \\
    \hspace{6.45cm} \types d(x, y, z) : C(\synsup(x, y))}
  {\Gamma, \ x \oftype A, \ y \oftype [B(x), \synW_{u \oftype A} B(u)] \types \wrec_{d} (\synsup(x, y)) \hspace{2.8cm} \\
    \hspace{2.4cm} = d(x, y, \lambda u \oftype B(x). \wrec_d(\app(y, u))): C(\synsup(x, y))}}
\end{mathparpagebreakable}

\paragraph*{$\synZero$}  (Empty type.)

\begin{mathparpagebreakable}
  \inferrule*[right=$\synZero$-\form]{\types \Gamma\ \cxt}{\Gamma \types \synZero\ \type}
\and
  \text{(No introduction rules.)}
\\
  \inferrule*[right=$\synZero$-\elim]{\Gamma, \ x \oftype \synZero \types C(x)\ \type}{\Gamma, \ x \oftype \synZero \ \types \syncase (x) : C(x)}
\and
  \text{(No computation rules.)}
\end{mathparpagebreakable}

\paragraph*{$\synOne$}  (Unit type, singleton type.)

\begin{mathparpagebreakable}
  \inferrule*[right=$\synOne$-\form]{\types \Gamma\ \cxt }{\Gamma \types \synOne \ \type}
\and
  \inferrule*[right=$\synOne$-\intro]{\types \Gamma\ \cxt }{\Gamma \types * : \synOne}
\and
  \inferrule*[right=$\synOne$-\elim]{\Gamma,\ x \oftype \synOne \types C(x)\ \type \\ \Gamma \types d : C(*)}{\Gamma,\ x \oftype \synOne \types \synrec_d (x) : C(x)}
\and
  \inferrule*[right=$\synOne$-\comp]{\Gamma,\ x \oftype \synOne \types C(x)\ \type \\ \Gamma \types d : C(*)}{\Gamma \types \synrec_d (*) = d : C(*)}
\end{mathparpagebreakable}
% \todo{Improve naming conventions of eliminators?}

\paragraph*{$+$-types} (Binary disjoint sums.)

\begin{mathparpagebreakable}
  \inferrule*[right=$+$-\form]{\Gamma \types A\ \type \\ \Gamma \types B\ \type}{\Gamma \types A + B \ \type}
\and
  \inferrule*[right=$+$-\intro\ 1.]{\Gamma \types A\ \type \\ \Gamma \types B\ \type}{\Gamma,\ x \oftype A \types \inl(x) : A+ B}
\and
  \inferrule*[right=$+$-\intro\ 2.]{\Gamma \types A\ \type \\ \Gamma \types B\ \type}{\Gamma,\ y \oftype B \types \inr(y) : A +B}
\and
  \inferrule*[right=$+$-\elim]{\Gamma,\ z \oftype A + B \types C(z)\ \type \\ \Gamma,\ x \oftype A \types d_l(x) : C(\inl (x)) \\ \Gamma,\ y \oftype B \types d_r(y) : C(\inr (y))}{\Gamma,\ z \oftype A + B \types \syncase_{d_l,d_r}(z) : C(z)}
\and
  \inferrule*[right=$+$-\comp\ 1.]{\Gamma,\ z \oftype A + B \types C(z)\ \type \\ \Gamma,\ x \oftype A \types d_l(x) : C(\inl (x)) \\ \Gamma,\ y \oftype B \types d_r(y) : C(\inr (y))}{\Gamma,\ x \oftype A \types \syncase_{d_l, d_r}(\inl(x))=d_l(x) : C(\inl(x))}
\and
  \inferrule*[right=$+$-\comp\ 2.]{\Gamma,\ z \oftype A + B \types C(z)\ \type \\ \Gamma,\ x \oftype A \types d_l(x) : C(\inl (x)) \\ \Gamma,\ y \oftype B \types d_r(y) : C(\inr (y))}{\Gamma,\ y \oftype B \types \syncase_{d_l, d_r}(\inr(y))=d_r(y) : C(\inr(y))}
\end{mathparpagebreakable}

\subsection{Universes} \label{subsec:universe-rules}

A universe within the theory may be closed under some or all of the logical constructors of the theory; we include below the rules corresponding to closure under all of the constructors given above.

\begin{mathparpagebreakable}
  \inferrule{ }{\types \synU\ \type}
\and
  \inferrule{ }{x \oftype \synU \types \el (x) \ \type}
\\
  \inferrule{\Gamma \types a : \synU \\ \Gamma,\ x \oftype \el(a) \types b(x) : \synU}{\Gamma \types \synpi (a, x. b(x)) : \synU}
\and
  \inferrule{\Gamma \types a : \synU \\ \Gamma,\ x \oftype \el(a) \types b(x) : \synU}{\Gamma \types \el (\synpi (a, x. b(x)) = \synPi_{x : \el(a)} \el(b(x)) \ \type}
\\
  \inferrule{\Gamma \types a : \synU \\ \Gamma,\ x \oftype \el(a) \types b(x) : \synU}{ \Gamma \types \synsigma (a, x. b(x)) : \synU}
\and
  \inferrule{ \Gamma \types a : \synU \\ \Gamma,\ x \oftype \el(a) \types b(x) : \synU}{ \Gamma \types \el (\synsigma (a, x. b(x)) = \synSigma_{x : \el(a)} \el(b(x)) \ \type}
\\
  \inferrule{ \Gamma \types a : \synU \\ \Gamma \types b, c : \el (a)}{ \Gamma \types \synid_A (b,c) : \synU}
\and
  \inferrule{ \Gamma \types a : \synU \\  \Gamma \types b, c : \el (a)}{ \Gamma \types \el (\synid_a(b,c)) = \synId_{\el (a)}(b, c) \ \type}
\\
  \inferrule{ }{\types \synz : \synU}
\and
  \inferrule{ }{\types \el (\synz) = \synZero \ \type}
\and
  \inferrule{ }{\types \syno : \synU}
\and
  \inferrule{ }{\types \el (\syno) = \synOne \ \type}
\\
  \inferrule{\Gamma \types a, b : \synU}{ \Gamma \types a \smallplus b : \synU}
\and
  \inferrule{ \Gamma \types a, b : \synU}{ \Gamma \types \el (a \smallplus b) = \el (a) + \el (b) \ \type}
\\
  \inferrule{ \Gamma \types a : \synU \\  \Gamma ,\ x \oftype \el(a) \types b(x) : \synU}{ \Gamma \types \synw (a, x. b(x)) : \synU}
\and
  \inferrule{ \Gamma \types a : \synU \\  \Gamma ,\ x \oftype \el(a) \types b(x) : \synU}{ \Gamma \types \el (\synw (a, x. b(x)) = \synW_{x : \el(a)} \el(b(x)) \ \type}
\end{mathparpagebreakable}

\subsection{Further rules} \label{subsec:optional-rules}

The rules above are somewhat weak in their implications for equality of functions.  To this end, some further rules are often adopted: the \emph{$\eta$-rule} for $\synPi$-types, and the \emph{functional extensionality} rule(s).  Our formulation of the latter is taken from \cite{garner:on-the-strength}; see also \cite{hofmann:thesis}.

\begin{mathparpagebreakable}
  \inferrule*[right=$\synPi$-$\eta$]
    {\Gamma \types f : \synPi_{x \oftype A} B(x)}
    {\Gamma \types f = \lambda x \oftype A. \app(f,x) : \synPi_{x \oftype A} B(x) }
\and
  \inferrule*[right=$\synPi$-ext]
    {\Gamma \types f, g : \synPi_{x \oftype A} B(x) \\
     \Gamma \types h : \synPi_{x \oftype A} \synId_{B(x)}(\app(f,x),\app(g,x)) }
    {\Gamma \types \ext(f,g,h) : \synId_{\synPi_{x \oftype A} B(x)}(f,g) }
\and
  \inferrule*[right=$\synPi$-ext-comp-prop]
    {\Gamma, x \oftype A \types b : B(x) }
    {\Gamma \types \extcomp(x.b): \synId_{\synPi_{x \oftype A} B(x)} \hspace{4.4cm}  \\
     \hspace{2cm} (\ext(\lambda x \oftype A. b,\lambda x \oftype A. b,\lambda x \oftype A. \refl b), \refl (\lambda x \oftype A. b)) }
\end{mathparpagebreakable}

%%% Local Variables: 
%%% mode: latex
%%% TeX-master: "simplicial-model"
%%% End: 