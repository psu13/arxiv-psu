\documentclass[12pt]{article}%
\usepackage{enumitem}
\setcounter{secnumdepth}{0}

%\usepackage{amsmath,amsthm,amscd,amssymb}
\usepackage[full]{textcomp}
\usepackage{XCharter}% lining figures in math, osf in text
\usepackage[scaled=1.04,varqu,varl]{inconsolata}% inconsolata typewriter
\usepackage[xcharter,vvarbb,scaled=1.03,smallerops,bigdelims]{newtxmath}
\usepackage[cal=cm,scr=esstix]{mathalpha}
\usepackage[papersize={7in, 10.0in}, left=.6in, right=.6in, top=.6in, bottom=.9in]{geometry}
\linespread{1.05}
\sloppy
\raggedbottom
\pagestyle{plain}

% these include amsmath and that can cause trouble in older docs.
\input{/Users/psu/arxiv-psu/helpers/cmrsum}
\makeatletter

\DeclareFontFamily{OMX}{MnSymbolE}{}
\DeclareSymbolFont{largesymbolsX}{OMX}{MnSymbolE}{m}{n}
\DeclareFontShape{OMX}{MnSymbolE}{m}{n}{
    <-6>  MnSymbolE5
   <6-7>  MnSymbolE6
   <7-8>  MnSymbolE7
   <8-9>  MnSymbolE8
   <9-10> MnSymbolE9
  <10-12> MnSymbolE10
  <12->   MnSymbolE12}{}

\DeclareMathSymbol{\downbrace}    {\mathord}{largesymbolsX}{'251}
\DeclareMathSymbol{\downbraceg}   {\mathord}{largesymbolsX}{'252}
\DeclareMathSymbol{\downbracegg}  {\mathord}{largesymbolsX}{'253}
\DeclareMathSymbol{\downbraceggg} {\mathord}{largesymbolsX}{'254}
\DeclareMathSymbol{\downbracegggg}{\mathord}{largesymbolsX}{'255}
\DeclareMathSymbol{\upbrace}      {\mathord}{largesymbolsX}{'256}
\DeclareMathSymbol{\upbraceg}     {\mathord}{largesymbolsX}{'257}
\DeclareMathSymbol{\upbracegg}    {\mathord}{largesymbolsX}{'260}
\DeclareMathSymbol{\upbraceggg}   {\mathord}{largesymbolsX}{'261}
\DeclareMathSymbol{\upbracegggg}  {\mathord}{largesymbolsX}{'262}
\DeclareMathSymbol{\braceld}      {\mathord}{largesymbolsX}{'263}
\DeclareMathSymbol{\bracelu}      {\mathord}{largesymbolsX}{'264}
\DeclareMathSymbol{\bracerd}      {\mathord}{largesymbolsX}{'265}
\DeclareMathSymbol{\braceru}      {\mathord}{largesymbolsX}{'266}
\DeclareMathSymbol{\bracemd}      {\mathord}{largesymbolsX}{'267}
\DeclareMathSymbol{\bracemu}      {\mathord}{largesymbolsX}{'270}
\DeclareMathSymbol{\bracemid}     {\mathord}{largesymbolsX}{'271}

\def\horiz@expandable#1#2#3#4#5#6#7#8{%
  \@mathmeasure\z@#7{#8}%
  \@tempdima=\wd\z@
  \@mathmeasure\z@#7{#1}%
  \ifdim\noexpand\wd\z@>\@tempdima
    $\m@th#7#1$%
  \else
    \@mathmeasure\z@#7{#2}%
    \ifdim\noexpand\wd\z@>\@tempdima
      $\m@th#7#2$%
    \else
      \@mathmeasure\z@#7{#3}%
      \ifdim\noexpand\wd\z@>\@tempdima
        $\m@th#7#3$%
      \else
        \@mathmeasure\z@#7{#4}%
        \ifdim\noexpand\wd\z@>\@tempdima
          $\m@th#7#4$%
        \else
          \@mathmeasure\z@#7{#5}%
          \ifdim\noexpand\wd\z@>\@tempdima
            $\m@th#7#5$%
          \else
           #6#7%
          \fi
        \fi
      \fi
    \fi
  \fi}

\def\overbrace@expandable#1#2#3{\vbox{\m@th\ialign{##\crcr
  #1#2{#3}\crcr\noalign{\kern2\p@\nointerlineskip}%
  $\m@th\hfil#2#3\hfil$\crcr}}}
\def\underbrace@expandable#1#2#3{\vtop{\m@th\ialign{##\crcr
  $\m@th\hfil#2#3\hfil$\crcr
  \noalign{\kern2\p@\nointerlineskip}%
  #1#2{#3}\crcr}}}

\def\overbrace@#1#2#3{\vbox{\m@th\ialign{##\crcr
  #1#2\crcr\noalign{\kern2\p@\nointerlineskip}%
  $\m@th\hfil#2#3\hfil$\crcr}}}
\def\underbrace@#1#2#3{\vtop{\m@th\ialign{##\crcr
  $\m@th\hfil#2#3\hfil$\crcr
  \noalign{\kern2\p@\nointerlineskip}%
  #1#2\crcr}}}

\def\bracefill@#1#2#3#4#5{$\m@th#5#1\leaders\hbox{$#4$}\hfill#2\leaders\hbox{$#4$}\hfill#3$}

\def\downbracefill@{\bracefill@\braceld\bracemd\bracerd\bracemid}
\def\upbracefill@{\bracefill@\bracelu\bracemu\braceru\bracemid}

\DeclareRobustCommand{\downbracefill}{\downbracefill@\textstyle}
\DeclareRobustCommand{\upbracefill}{\upbracefill@\textstyle}

\def\upbrace@expandable{%
  \horiz@expandable
    \upbrace
    \upbraceg
    \upbracegg
    \upbraceggg
    \upbracegggg
    \upbracefill@}
\def\downbrace@expandable{%
  \horiz@expandable
    \downbrace
    \downbraceg
    \downbracegg
    \downbraceggg
    \downbracegggg
    \downbracefill@}

\DeclareRobustCommand{\overbrace}[1]{\mathop{\mathpalette{\overbrace@expandable\downbrace@expandable}{#1}}\limits}
\DeclareRobustCommand{\underbrace}[1]{\mathop{\mathpalette{\underbrace@expandable\upbrace@expandable}{#1}}\limits}

\makeatother


\usepackage[small]{titlesec}
\titlelabel{\thetitle.\quad}

\usepackage{cite}
\usepackage{microtype}

% hyperref last because otherwise some things go wrong.
\usepackage{hyperref}
\hypersetup{colorlinks=true
,breaklinks=true
,urlcolor=blue
,anchorcolor=blue
,citecolor=blue
,filecolor=blue
,linkcolor=blue
,menucolor=blue
,linktocpage=true}
\hypersetup{
bookmarksopen=true,
bookmarksnumbered=true,
bookmarksopenlevel=10,
pdfencoding=auto, psdextra
}

% avoid weird spacing after the period here.
\def\No#1{No.\@ #1}

% make sure there is enough TOC for reasonable pdf bookmarks.
\setcounter{tocdepth}{3}

%\usepackage[dotinlabels]{titletoc}
%\titlelabel{{\thetitle}.\quad}
%\titleformat{\section}[block]
  {\fillast\medskip}
  {{\thesection. }}
  {1ex minus .1ex}
  {\scshape}
 
\titleformat*{\subsection}{\itshape}
\titleformat*{\subsubsection}{\itshape}

\setcounter{tocdepth}{2}

\titlecontents{section}
              [2.3em] 
              {\bigskip}
              {{\contentslabel{2.3em}}\large\scshape}
              {\hspace*{-2.3em}}
              {\titlerule*[1pc]{}\contentspage}
              
\titlecontents{subsection}
              [4.7em] 
              {}
              {{\contentslabel{2.3em}}}
              {\hspace*{-2.3em}}
              {\titlerule*[.5pc]{}\contentspage}

% hopefully not used.           
\titlecontents{subsubsection}
              [7.9em]
              {}
              {{\contentslabel{3.3em}}}
              {\hspace*{-3.3em}}
              {\titlerule*[.5pc]{}\contentspage}
%\makeatletter
\renewcommand\tableofcontents{%
    \section*{\contentsname
        \@mkboth{%
           \MakeLowercase\contentsname}{\MakeLowercase\contentsname}}%
    \@starttoc{toc}%
    }
\def\@oddhead{{\scshape\rightmark}\hfil{\small\scshape\thepage}}%
\def\sectionmark#1{%
      \markright{\MakeLowercase{%
        \ifnum \c@secnumdepth >\m@ne
          \thesection\quad
        \fi
        #1}}}
        
\makeatother



%\makeatletter

 \def\small{%
  \@setfontsize\small\@xipt{13pt}%
  \abovedisplayskip 8\p@ \@plus3\p@ \@minus6\p@
  \belowdisplayskip \abovedisplayskip
  \abovedisplayshortskip \z@ \@plus3\p@
  \belowdisplayshortskip 6.5\p@ \@plus3.5\p@ \@minus3\p@
  \def\@listi{%
    \leftmargin\leftmargini
    \topsep 9\p@ \@plus3\p@ \@minus5\p@
    \parsep 4.5\p@ \@plus2\p@ \@minus\p@
    \itemsep \parsep
  }%
}%
 \def\footnotesize{%
  \@setfontsize\footnotesize\@xpt{12pt}%
  \abovedisplayskip 10\p@ \@plus2\p@ \@minus5\p@
  \belowdisplayskip \abovedisplayskip
  \abovedisplayshortskip \z@ \@plus3\p@
  \belowdisplayshortskip 6\p@ \@plus3\p@ \@minus3\p@
  \def\@listi{%
    \leftmargin\leftmargini
    \topsep 6\p@ \@plus2\p@ \@minus2\p@
    \parsep 3\p@ \@plus2\p@ \@minus\p@
    \itemsep \parsep
  }%
}%
\def\open@column@one#1{%
 \ltxgrid@info@sw{\class@info{\string\open@column@one\string#1}}{}%
 \unvbox\pagesofar
  \gdef\thepagegrid{one}%
 \global\pagegrid@col#1%
 \global\pagegrid@cur\@ne
 \global\count\footins\@m
 \set@column@hsize\pagegrid@col
 \set@colht
}%

\def\frontmatter@abstractheading{%
\bigskip
 \begingroup
  \centering\large
  \abstractname
  \par\bigskip
 \endgroup
}%

\makeatother

%\DeclareSymbolFont{CMlargesymbols}{OMX}{cmex}{m}{n}
%\DeclareMathSymbol{\sum}{\mathop}{CMlargesymbols}{"50}
%\pdfbookmark[1]{Introduction}{Introduction}

\begin{document}

\title{Tchaikovsky Overview 2001}
\date{}
\maketitle

\tableofcontents

\newpage

\addcontentsline{toc}{section}{\protect\textbf{Introduction}}

\noindent
Tchaikovsky was an extremely sensitive and loving person, and that comes across in his music, which has  special appeal to people who themselves are sensitive. His music is easy to like and enjoy, and that means musical snobs rather sneer at him, but he is certainly one of the great composers.
Part of his greatness is that he can appeal to listeners on all sorts of levels.

His popularity also means a crowded catalog. Tchaikovsky sells, and even unsympathetic conductors try to cash in on that. A great many recordings sound like hasty run-throughs. His music relies more heavily than usual on the melodic line. His melodies need time to be heard, space to flower in, air to sound in. It is not good to hurry them along, to drive them too hard, to make them march rather than flow. This is a catalog of shortcomings that plague recorded versions, even from celebrated interpreters. In the last 20 years the race to record Tchaikovsky has faded a great deal, and his CD listings have shrunk every year as companies delete older recordings.

The hasty run-through syndrome applies even to the early symphonies. Almost nobody records them except as part of a complete cycle---and complete cycles are a fairly recent phenomenon. So for years it was hard to get symphonies 1, 2, 3 and {\it Manfred}. Before the era of stereo Beecham recorded 2 and 3, Toscanini {\it Manfred}; there wasn't much else. When I was a student I finally found the horrendous Swarowskys on Urania. Then came Dorati's boring set on Mercury. (On CD, \No2 is the best, \No1 is perverse, and the strings sound really scrawny in \No3.) The first widely available stereo recordings worth having were Bernstein's in the early 70s. That seemed to break the dam; and Muti, Mehta, Rostropovich, Ormandy, and Karajan soon followed---all designed to complete a cycle. But many of them never conducted the early symphonies before they recorded them, and it shows.

I wonder how we ever lived without those wonderful early symphonies.

\No1 ({\it Winter Daydreams}) may be the most atmospheric thing Tchaikovsky ever wrote and he knew it and never lost his fondness for it. 2 is the {\it Little Russian} (Ukrainian), so named for the use of folktunes he picked up around Kiev. 3 (the English call it the {\it Polish}---a worthless nickname, with no basis in the music) is just packed with melody: what a fertile composer Tchaikovsky was! This was written the same time as {\it Swan Lake} and sounds it.

With the last three symphonies Tchaikovsky wins his place in history; he was one of the few composers able to impose their own personality on the symphony and thus bring new life to the form. In 4 he combines ballet music (considered out of place in a symphony then), folk music (ditto), Russian atmosphere, a sense of Fate, and a triumph over the darker emotions. 5 is much the same recipe, but the final triumph starts to sound hollow. The composer himself later spoke of its insincerity. But it has always been his most popular symphony---it is just so lovable, so assertive, so thrilling. \No6, the {\it Pathetique}, is ``permeated with subjective feeling, and quite often composing it in my mind I wept copiously''. Seldom had he been able to write a symphony down so fast (\No3 took less than a month, though) and seldom did his work affect him so deeply. In light of the composer's heavy emotional investment in this music, we become positively angry with surface skimmings and brisk tempos. Tchaikovsky without emotion is pretty pointless. This is not facile music; when in the last movement he submits to his fate, he knows it may mean death.

There have been at least 50 recordings of the late symphonies, and they seem to fall into five general categories:

\begin{itemize}[label={}]
\item
{\it Indulgent}: Bernstein, Rostropovich

\item
{it Balance and Grandeur}: Ormandy, Stokowski, Wit

\item
{\it Moderate}: Litton, Slatkin, Mehta, Rozhdestvensky

\item
{\it Fast and Businesslike}: Szell, Monteux, Jansons, Muti, Haitink, Markevitch, Mravinsky 

\item
{\it Tame and Boring}: Masur, Abbado, Dutoit, Previn, Abravanel, Dorati, Fedoseyev
\end{itemize}
Boring interpretations can be fast or slow, and the examples listed under fast can also be exciting (but some are boring---many of Haitink's are). They are businesslike in that they allow little emotional indulgence, but the musicality of the conductor can still save the performance.

Note that Ormandy recorded 5 and 6 many times, but as he got older he got slower. You may prefer his slower recordings (RCA, Delos), but to most of us they sound weaker, less exciting.

Our musical tastes often boil down to how comfortable we are with our emotions and how much we indulge them. If we are very uncomfortable with the emotions, we should avoid music altogether. It is about the emotions! If we are utterly self-indulgent, then classical music is far too stuffy for us. Emotional slobs listen to rock; the next level up is jazz. But classical conductors (and solo musicians) also run a gamut from ``indulgent'' (insofar as classical can be that---think of Rostropovich and Bernstein) to stiff and unyielding (the Szells and Dohnanyis and Zinmans). I am very emotional but tend to prefer the quieter to the red-hot emotions; a moderate amount of control is good. So I prefer pretty direct interpretations without either flamboyance or extreme heart-on-sleeve. But the emotions must be there and given full value---without distorting the musical line and its natural flow. Impatient Tchaikovsky is a travesty. Eugene Ormandy may be his greatest interpreter.

\section{Sets}

The only sets in the current catalog are by Jansons, Karajan, Marriner, and Pletnev. None of them are worth having. In the past there were other complete sets, and you may be able to find some of them.

Rostropovich: excellent sound---sweet and succulent, from Kingsway Hall in 1976. The interpretations are emotionally generous. The Third Symphony seems weakest, especially its outer movements. The last three symphonies have the strongest emotional charge, and the conductor really identifies with the music. This set makes Haitink sound very tame, Markevitch and Muti rather stiff, and it's better recorded and more consistent than Karajan.
Some of our reviewers consider it a first choice---but it certainly is not for the early symphonies.

Ricardo Muti was exciting in an Italian hot-blooded way, but his recordings lack atmosphere and Russian feeling and a feeling for orchestral tone. Romantic longing is slighted, and things often feel driven. But sometimes he seems refreshingly direct and to-the-point. That is especially so next to Temirkanov, who pays so much attention to details that the overall sweep is lost. He plays around with tempos and makes a thousand little points. The overall result seems tentative, and one suspects the (Royal Philharmonic) orchestra and conductor were not on the same wavelength.

One of the major disasters of phonographic history was the Abbado cycle in Chicago (Sony), Nothing is right about it. (One of our writers actually called it ``sickening''.) Nor do most of Abbado's other Tchaikovsky recordings make our list. There is something bland and soulless about his conducting that seems almost insulting to Tchaikovsky.

The Pletnev set is a disaster. It doesn't sound Russian; in fact, it has no tradition, no conviction, no affection. The orchestra is very agile but has no character and sound the same all the way through, The brass are genteel, and everything is too well-mannered to amount to much of a statement.  
Controlled, calculated, cold.

The Svetlanovs, from the late 1960s are already ``historic'' performances, because the style of conducting and playing are rapidly disappearing. He is quite powerful and original in the early symphonies, and the sound is good, but BMG seems to have dropped them at least in the USA.

\section{Individually}

\section{Symphony 1}

It is rather unusual to find comeplete
agreement between conductors, but, Ormandy and Rozhdestvensky had exactly same timings and tempos. Both were very fine.
Very close to them in interpretation Michael Tilson Thomas on DG, who got everything right---and the Boston Symphony played so beautifully. Andrew Litton on Virgin is quite similar (Mar/ Apr 1991) and had outstanding sound.

Jansons is fast but dull and vacuous. He pushes and pulls the music, stop-and-start. Flow and atmosphere are never quite right the emotions are not engaged. In IV, in same overall time as Litton, he manages to drag out the Andante introduction and then fly thru the rest (Allegro maestoso!). The Oslo strings fail to lift the big tunes above the rest of the orchestra, and the brass are sometimes raucous.

The Seattle/Schwarz seems tired and short on atmosphere. It does have good sound. So does the Slatkin, but Mr Slatkin is his usual prosaic self, and there's not much passion or poetry or mystery. It's worth hearing but not on the Michael Tilson Thomas level.

Nobody here can stand the Chicago Abbado:\@ perfunctory, flat, lacking direction---and coarse brass. A few of us like the Karajan, but to me he sounds bored with the music, as if he's recording it only to make a complete set. I'd say the same for Muti. Many of us like the Bernstein: it's exaggerated a bit, naturally, but it works. It has heaps of magical atmosphere. gorgeous playing, and powerful sound.

Markevitch is on the fast side, direct and well conducted but not strong on atmosphere. The Leaper on Naxos is excellent and even has more atmosphere than Markevitch, but in iv he makes the pauses too long in the introduction. Certainly if you don't know the symphony it's a best buy; and it will stand up to of the full-priced versions.

\bigskip
\begin{tabular}{ l r }
Thomas&	DG NA\\
Ormandy&	RCA NA\\
Litton&	Virgin NA\\
Bernstein&	Sony 47631\\
Leaper&	Naxos 550517
\end{tabular}

\section{Symphony 2}

The best ever of this symphony is Markevitch, available in a two-disc set (with 1 and 3 and {\it Francesca}), It is wildly exciting but never too fast; the full value of the music is there, in perfect tempos and great sound. Since the two discs cost no more than one full-priced disc, you should buy it while you can.

The Bernstein is slap-dash, and you'll get it if you buy his \No1. He obviously did not respond to this work and simply rushed through it to get it over with. Even Muti, who also destroyed it, didn't murder it at quite that speed. But Previn, Karajan, and Svetlanov so take the main part of I too fast. Maazel, despite a similar timing for the movement, seems much less rushed. Among those who got this first movement just right are Beecham, Ormandy, Rozhdestvensky, Markevitch, Jansons, and Litton. All were close to 12 minutes, and none made too strong a contrast between the long introduction and the movement proper.

Some will like Svetlanov for emotional extremes, Maazel for smoothness (and smooth Telarc sound), Rozhdestvensky and the DG Abbado ({\it not} CBS) for healthy balance, and if it ever returns, the Ormandy for naturalness and majesty. Geoffrey Simon's Chandos recording is not the same music; it's a fine performance of the inferior earlier version.

Svetlanov gets just the right contrast in the tempos of IV. Too much contrast sounds artificial. Some conductors build very effectively from the slow opening to the faster main body (Litton especially). Some make very little contrast; with Karajan the whole movement sounds noisy and aggressive. Jansons whips up a lot of excitement here.
Slatkin is heavy and massive but still cheerful and attractive. Giulini is often too fast but very exciting---a very individual interpretation in gorgeous 1956 sound (but there are cuts in IV). Kurt Masur recorded it twice; the earlier Dresden one (Berlin) is fast and wild, while the later Leipzig (Teldec) seems dull and lumbering. Adrian Leaper does a fine job with this for Naxos (coupled with 4); his tempos are as perfectly judged as Markevitch's. Markevitch and Litton seem slightly preferable---playing, sound, and atmosphere (feel for the music).

\bigskip
\begin{tabular}{ l r }
Markevitch& Philips 446148 [2CD]\\
Litton& Virgin NA\\
Leaper& Naxos 550488\\
Jansons& Chandos 8460\\
Ormandy& RCA NA
\end{tabular}

\section{Symphony 3}

The introduction is labelled {\it Moderato assai\/} and ``Funeral March Tempo''.
That's not very definite, and conductors make what they will of it. Obviously it is a ``slow'' introduction to the main {\it Allegro brillante}, the question is how slow. In some recordings there's little contrast with the main {\it Allegro}; in others it's radical. Jansons is among the former; Gilbert Levine is an example of the latter.

Karajan is very strange: I, II, and V are very slow, but III is much too fast to be an Elegy (8 minutes versus 10 minutes for most of the other recordings). You can hear the hand of a master conductor shaping the phrases and making the music speak Karajanese. It's an imperious performance! It's also a travesty.

Muti manages to sound rushed most of the time. The actual timings are often reasonable; but even in II, where the timing is perfectly normal, he never relaxes. First and last movements are terribly driven; he never stops for breath, never seems to enjoy a good tune.

Bernstein is one of the slowest, as usual---except the introduction. I is joyous and buoyant, with a balletic lift to the rhythms. His Elegy is 11 minutes. He milks the music for all its emotion (and Tchaikovsky was a very emotional man). The big problem is the raw sound of the New York Philharmonic violins: like chalk on a blackboard. They are sour and tinny and ugly; they hurt the ears, especially in I. I can't listen past them to the great conducting. Yet this may be the best interpretation of the music.

Telarc gives us wonderful sound, with atmosphere and subtlety to match Mr Levine's conducting. He does so much more, phrase by phrase, than any other conductor, The Royal Philharmonic plays beautifully. Note that the Introduction is slowed way down and almost whispered. Levine likes contrasts.

Ormandy and Rozhdestvensky were fine, with their usual healthy, moderate tempos; but Ormandy's orchestra was vastly superior to the Russian one and made them sound scrawny. The catalog lists three really bad ones besides the questionable Karajan: Abbado, Abravanel, Masur. The Masur is totally insensitive---without tenderness and yearning, without grandeur.

The Jansons is exciting, with clear and pleasant sound. The Oslo brass are nicely balanced, the strings sweetly metallic, the wind solos prominent and well done. Jansons does not indulge emotional wallowers; he never slows the pace when a Big Tune comes along. Still, the central Elegy is very beautiful.

It is the central Elegy (this is a five movement work) that does in the Slatkin. He takes a stultifying 12:24 for it---10 minutes is normal---and it clashes badly with rather vigorous outer movements. A flaccid Capriccio Italien fills up the disc.

Litton brought out adjectives like ``flaccid'' and ``shapeless'' from our reviewers, and one even called it deadly dull and tired. It is rather smoothed out, and the phrasing does seem weak; but it has beautiful moments, and the sound is. Some of us find it among the better versions, but the strings are a bit raw in I, as if copying the New York Philharmonic (passages that are hard to play smoothly and with good tone and adequate vibrato). Certainly Jansons has more vigor, but Litton is warmer (sound is further back) and seems to have more affection for the music. (Jansons is simply not an
``affectionate'' conductor!) And excitement catches up to him, even in I. Perhaps he chose to build it gradually, while Jansons ``lays it on'' practically from the beginning. True, when it's over Wit and Jansons leave you exhilarated, Litton somehow exhausted.

Once again we may hope for the return of Ormandy; but Jansons will remain more exciting. The Markevitch comes with his 1 and 2. The introduction is way too slow---it withers on the vine. The Elegy is very elegiac---almost 12 minutes to Karajan's wretched 8. The other gorgeous elegies are Wit and Levine. Overall, the Markevitch doesn't measure up to these or to Jansons.

The Wit on Naxos begins (like Levine on Telarc) with a very slow and quiet introduction that will take some getting used to, but after that it is one of the greatest recordings this symphony has had---far better than Jansons or Bernstein or Litton. In fact, though Levine seems more romantic, and though Naxos sound is not Telarc sound, this Naxos recording is better than anything else on the market.

\bigskip
\begin{tabular}{ l r }
Wit& Naxos 550518\\
Jansons& Chandos 8463
\end{tabular}

\section{Symphony 4}

The Abbado recording in Chicago is very, very dull. If you want Chicago---domineering brass and dry strings and all---Solti is the one: intense, impetuous, restless, volatile, muscular, powerful, dramatic (he was certainly not a dull conductor). Barenboim is not as dull as Abbado, but his Fourth is labored and heavy-handed, with no intensity or passion; and the Chicago strings are thin and hard.
Barenboim's New York Fourth was better.

The Ormandy Fourth in Philadelphia is heady and majestic. The Sony CD reissue is gorgeous. The excitement never lets up, but Ormandy never gets carried away, either.
There is no angst---no wild dramatic surges, no quirks. It is solid and satisfying---very lyrical, very romantic, but never wayward or indulgent. Ormandy had an unerring feel for the shape of a melody. The woodwind solos are gorgeous. It all adds up to a ``this is the way it should be'' performance, and when you compare it to 30 or 40 other recordings, they all seem lacking.

Bernstein's second recording (1975) is also a ``perfect'' interpretation, but the sound isn't as immediate and powerful as the Ormandy. I is back-in-the-hall sound---unusual for Columbia. Most of us dislike the turgid third Bernstein (DG; the New York Philharmonic was a its best), but this one has searing drama and tremendous insight and plenty of ``soul'' a real improvement on his first recording from some 15 years before.

Stokowski has been reissued on Vanguard but it is mannered and eccentric---especially the stop-and-go effects in I---and often faster than most other performances. There's plenty of emotion and intensity. Wind solos are weak the strings don't have the bloom of Ormandy's. It's the American Symphony Orchestra.

Inbal on Denon has some odd moments (the distended ending of l) but sounds spectacular. Trumpets over-blow, but in general the brass is powerful. Strings are creamy. The playing can seem rather detached and emotionless and the interpretation lacking urgency. It may not end up on our ``5 or 6'' best list, but it's pretty good.

The Svetlanov has very crude and blairing brass, and they sound distorted, too (Russ
1967); the bass is weak (would you belie tenor timpani?). But the strings are very good and Svetlanov does whip up a lot of excitment. His last movement is much too fast, but some like it hot. I and IV are the same Mravinsky; II and III are faster than Mravinsky. Speed in itself is not exciting; and, indeed, the speedy Mravinsky leaves us cold.

Rozhdestvensky's best recording is the Erato. The sound is terrific, and the playing very refined. His interpretation involves ma subtleties and nuances, with exciting climax Tempos are on the slow side. The Rozhdestvensky's IMP set of 4, 5, and 6 is not worth the money---generic in every way.
Yuri Temirkanov is a remarkable conductor, but there is no reason to own his set of 4,5, and 6 either (RCA). Our reviewer found the sound and playing uneven and the interpretations often slack and languorous. The Beecham was reissued on CD and may still be available in England. It is not stereo but true high fidelity---it sounds superb. No other recording is better loved by all of us. Kubelik recorded it with Beecham's orchestra---also EMI---and that may be around (68223). It comes pretty close to the Beecham but is a little more relaxed. The Beecham is not relaxed! It is heady, even manic---but in a thrilling way. The playing on the Barbirolli is very ragged and coarse.

Mehta was wonderful, and the LA strings sounded like Philadelphia's. The Ashkenazy (London too---and deleted too) was almost the same, even in sound, but the orchestra was slightly cleaner and the conducting more sensitive. Karl Böhm was very free with tempos; he took his time and gave the woodwinds considerable latitude (delightful---and deleted). Haitink was solid, warm, and balanced, with clean, beautiful orchestral playing He does not indulge in flashy or emotional effects.
Maazel does not seem very involved, but the Cleveland Orchestra sounded really great on Telarc (deleted). The Szell was fast and business-like and seemed utterly insensitive to many listeners, but the orchestra and sound were superb, and some of us like it. The Litton was slack, sodden, ill-shaped, lacking in energy---old-man's Tchaikovsky. Slatkin was rather sluggish and marred by too much tampering with the flow.

The Boston Symphony/Monteux series was reissued by RCA as a set of 4, 5, and 6---all beautifully played. Monteux generated a great deal of heat. But in this symphony, the weakest of his three, he certainly rushes ll unconscionably. Those who like it call it balletic and joyful. Very few of us like Dutoit as a conductor, and his Fourth is deleted anyway. Jansons is passionate and electrifying---great if you are not moved by grandeur and want Tchaikovsky fast, full of passion and fury.

Naxos brings us Adrian Leaper with the
Polish Radio Orchestra. As Naxos fans have come to know, that's a very fine orchestra, and their sound is enhanced by the hall they record in and by the Naxos engineering. Naxos couples symphonies 2 and 4---a very full disc and an incredible bargain. Leaper's 4 doesn't have the drama and emotion of a Bernstein; it's not overwrought. It's gentler and lighter than many---very sensible. It is not a great reading, but it's very pleasant. Don't buy it to replace Beecham or Solti, but as a start or a supplement.

\bigskip
\begin{tabular}{ l r }
Ormandy & Sony 46334\\
Beecham& EMI NA\\
Bernstein& Sony 47633
\end{tabular}

\section{Symphony 5}

One of the most thrilling recordings is Ormandy's first one in stereo (1959), which Sony has reissued in the budget Essential Classics series. It is essential. Ormandy in his prime was a fast conductor, but next to the 3 Ms---Markevitch, Mravinsky, and Monteux---he sounds positively moderate. As usual, grandeur is the word---and ``soulful'', even though Ormandy is a very direct interpreter. He brings out the full emotional value of the music without throwing it in your face. The many wind solos are superb, the brass healthy but not overwhelming, the strings glorious, of course. 1959 means a little hiss, soon submerged in the Philadelphia Sound.

Leonard Bernstein tried to match Ormandy for the same label. That may be the most exciting Tchaikovsky Fifth ever put on record. It also has great brass and plenty of headiness as well as feeling. With the same orchestra on DG, Bernstein presents a fierce romantic struggle (both singly and with 6 in a two-disc set reviewed this issue). The dramatic moments are just as strong, but the brooding sections are more somber and foreboding. It is a reading of extremes, very revealing. it will stand for a long time as a great recording, but for frequent listening the Sonys (Bernstein and Ormandy) are better.

Another great recording was the Stokowski on London (Decca). The other Stokowskis in the catalog all sound terrible (from 78s).

In listing great Fifths we cannot ignore Herbert von Karajan, who really seemed to take to this work. Again we must stress that his earliest DG recordings are better than the cold, hard digital wonders of the 1980s, and the deleted EMI may be his best Tchaikovsky Fifth. The sound is super-technicolor, with very powerful bass-something DG never had. The strings have some real flesh on them, unlike the DGs. The perspective varies from section to section; the engineering is somewhat eccentric but very impressive. The interpretation belongs with Ormandy, Bernstein, Stokowski, and Monteux at the very top. No eccentricities there. The flow is natural; there's not even a big slowing for the grand finale: it just seems to happen by itself. For some reason this EMI recording has far more drama, abandon, and sheer emotion than any of the DGs. The big, bold gestures seem almost un-Teutonic---uncharacteristic.

Rudolf Kempe may have been a greater conductor than Karajan. His 1960 EMI Fifth has been reissued by Testament (1100). It's with the Berlin Philharmonic: their playing and sound are simply thrilling. The conducting is powerful---strong and solid and Germanic.
Tchaikovsky sounds almost like Beethoven.
Kempe's interpretation is different; it most resembles Karajan, but Kempe seems to have more flair.

Monteux is perfectly balanced, and the Boston Symphony manages always to sound beautiful and refined.

Böhm had the slowest finale---almost Brucknerian in its grandeur and nobility. For slow and steady it was a good choice, with terrific sound. The Horenstein seems odd and somewhat out of shape. Dorati is rather plain and slightly mechanical---normal for him.
Only a few of us like the Mercury sound.

Mravinsky is favored by a couple reviewers, but not by me, though I saw him conduct this with the Leningrad Philharmonic. That was an exciting concert, but on record much of it seems too fast. Indeed, only the first movement is not the fastest known to me. The last movement, at 11 minutes, is simply outrageous. There are six listings of Mravinsky performances in the current catalog; who is going to buy them?

Rozhdestvensky has what we called bizarre but sometimes stimulating peculiarities---along with some wooden and boring stretches.

Naxos really hit the top with this one: Wit's conducting is simply outstanding. Everything is just right: tempos, phrasing, the brass theme in IV. As a long-time fan of Bernstein and Ormandy I didn't expect to be swept off my feet by this wonderful recording!

We really dislike the deadly, desolate Polyansky: sludge. The Dmitriev is weak in feeling and nothing special. So was the Leinsdorf---a heavy, rather serious interpretation that missed much of Tchaikovsky's personality.

The Solti is impressive (either one: he did it Twice in Chicago): brilliant playing, dominant brass, brisk tempos, great sound---but no Russian soul. Ashkenazy and the DG Abbado (Vienna, not the inferior Chicago remake) are both appealing. Ashkenazy's sound had more depth, but Abbado does more with the ending.
Ozawa on DG was too bright and artificial sounding. RCA should reissue his Chicago recording. The latest Chicago one is led by Barenboim, and we found it ``designed to death''---that is, too calculated and unspontaneous---``computer-generated''.

\bigskip
\begin{tabular}{ l r }
Bernstein& Sony 47634\\
Ormandy& Sony 46538\\
Wit& Naxos 550716\\
Bernstein& DG 429234 or 469214 (2CD]
\end{tabular}

\section{Symphony 6}

There's so much emotion in this music that it seems unnecessary to underline it
very heavily. Bernstein (DG) has Tchaikovsky writhing on the floor in agony and moaning with self-pity. Ormandy's Tchaikovsky is more dignified but still depressed. That is enough for most of us, I suspect. What is not acceptable is a businesslike run-through.

The DG Bernstein is the slowest ever---by far. Tension and anxiety pervade the music, even in an hour-long performance. No one has ever plumbed its depths like Bernstein. Every emotion is intensified; it drips with sorrow and self-pity. Everyone should hear it, but some will not prefer it---and some will only feel up to that kind of intensity once in a while.

If the DG is bathed in black despair from beginning to end, the emotional range of the Sony is broader. II flows effortlessly, and III almost jaunty. It is still a powerful interpretation. 

If that level of intensity is too much and feels like an invasion of Tchaikovsky's privacy, there is Ormandy, who treats it as great and beautiful music. Tchaikovsky maintains his dignity, but you get a pretty strong feel for the music just the same. This recording alone would place Ormandy among the great conductors and the Philadelphia among the very greatest orchestras. Again, this is the first stereo version, from 1960.

The Koizumi is nicely phrased, flexible, and graceful. Litton had plenty of passion and turmoil, and the dark parts were appropriately gloomy. Virgin's sound was excellent. The Soli is well played and recorded, and his dramatic approach will appeal to some who are tired of another kind of emotion. Some of us are convinced that he misses the point. The Muti was disturbing and unsettling, with more bite and bitterness than many. The reissued 30-year old Giulini is satisfying, with plenty of grace and intense feeling, a budget price, and great sound.

We originally reviewed Dohnanyi and Ozawa together, rejecting the brusque, boring metronomic Dohnanyi and raving about Ozawa. The Ozawa was one of the most beautiful in sound, and the Boston strings were still among the sweetest to be heard. Ozawa was not as expressive as Ormandy or Bernstein. He was more like Monteux: the tragedy is there, but self-pity is not allowed. It's a beautiful tragedy but not ours; we see it from a slight distance. Moderation and balance rule the orchestra
Erato issued it in 1987, but it's gone now.

In 1993 Leonard Slatkin displaced it. As a conductor he identifies more with the music you can feel the depression much more powerfully, and in IV it becomes almost overwhelming. It is not a writhing kind of suffering as in Bernstein, but the listless depression that accompanies hopeless resignation to Fate. The strings are glorious. The whole orchestra is thrilling. Ormandy had more passion and was better at certain details, but the Columbia recording isn't quite as good as the Slatkin.
And Slatkin had plenty of feeling.

Polyansky seems fairly routine and dull and slow. Svetlanov was also fairly routine and dry but rather fast, with thin string tone and raucous brass---not as sloppy as the 5th but still unacceptable. Bychkov was certainly ``the full treatment'', warm-sounding and warmly emotional. The Inbal was deadpan, bland, and bloodless---very boring and long (49 minutes).
Marriner is fairly close to Monteux but minus some indefinable quality that makes the Monteux more soulful (analog sound? the Boston string players?).

There were a number of Mravinskys, but the finest in both sound and interpretation was the Erato from 1982: it breathes just a little easier than the 1960 DG, though it's still fast.
Mitropoulos blasted his way thru this music.
He did the last movement in half the time Bernstein took (on DG). Gibson is utterly ordinary, and the Dutoit had not one shred of vitality.

The Kempe, with the Philharmonia, is simply gorgeous. He does not indulge the mood of Nas much as many; he is like Monteux there.
III has tremendous force and thrust. Unlike the 5th, Kempe's 6th is monaural; but the sound is so beautiful that you have to concentrate to hear that it is not stereo. (It was 1958.)

Monteux is fast, but IV has plenty of feeling. Stokowski is deeply emotional; his first movement is the most moving of them all (except Bernstein). Stokowski also gets wonderful sounds from the musicians: in that respect he joins Ormandy and Ozawa at the top.

There's room at the top for Wit, who again runs in a truly great recording for Naxos that has everything. It is coupled with a Francesca of passion and flair. When Naxos does so well, there's no sense in anyone issuing new full. priced recordings.

Karajan recorded this six times. Look for recording dates and choose one from the 1960s or 1970s---recent enough to sound good, but not DDD (they sound terrible). His best Patetique is probably the one on EMI, but it may be hard to find. Karajan's Tchaikovsky 5 and 6 are among the best still.

What about Reiner? It's a tightly-controlled performance, as you would expect, but with plenty of passion and intensity. The bleakness of the final pages seems more effective after the energy of the two inner movements. Slightly hissy sound.

The newest Giulini is dull and lumpish. That was not true of his EMI, now reissued.
There was once a very sensitive Vienna Philharmonic/Martinon; perhaps this conductor is no longer ``commercially viable'', but that record belonged in the top 10 or 12. Ashkenazy (same label---London or Decca) was also very fine.

There were two by Rostropovich: EMI (set) and Sony. The EMI was deeply felt, but the Sony revealed greater conductorial proficiency. It is with the National in Washington. I is not desperate---this conductor is an optimist.
II is beautifully wistful, III fierce and defiant. IV is not as black as Bernstein, but the tragedy is there. The Sony was one of the dozen best.

Günter Wand's is a somewhat detached view---objective, like a psychiatrist discussing a case. It has plenty of drive and energy, like Toscanini or Reiner. It's also tight and controlled and not emotionally involving.

\bigskip
\begin{tabular}{ l r }
Stokowski& RCA NA\\
Ormandy& Sony 47657\\
Bernstein& DG 419604 or 469214[2CD]\\
Slatkin& RCA NA\\
Monteux &RCA 61901 [2CD]\\
Kempe &Testament 1104\\
Wit &Naxos 550782\\
Giulini &EMI 67789
\end{tabular}

\section{Historical Note}

More than 50 years after they were etched into wax, Koussevitzky's classic Tchaikovsky recordings still pack a tremendous wallop.
There's a sense of discovery and wonder at every turn---not to mention a raw emotional power that's paradoxically coupled with patrician nobility and refinement. His tempos are all over the road, yet they've been so carefully chosen that the music flows naturally and with inexorable logic. Climaxes are titanic, explosive.
The Pathetique was recorded first. There's still a hint of portamento in the strings, making the music even more poignant and touching.
The black darkness of the basses at the end of
IV is devastating. Symphonies 4 and 5 are bracing, dramatic, and almost overwhelming in their intensity, and the orchestral execution is staggering even by today's standards. Sound is clear and rich---above average for the era (1930-44). BSO Classics has the cleanest transfer of 5, except in lI, which Biddulph took from a near-pristine set of 45 RPM discs (same performance). Avoid the RCA Pathetique; an unapproved take of one side of IV was used.
The outstanding 1949 Koussevitzky 4th is still locked up in RCA's vaults (the one on Biddulph is earlier).

\medskip
\noindent
\hfill{\sc Godell}

\section{Manfred}


It was Balakirev who kept pestering Tchaikovsky to do a Manfred Symphony. When he finally yielded to the pressure, Tchaikovsky produced a work of great power and originality. It is mature Tchaikovsky (after the Fourth Symphony) and is full of passionate melody. It is also one of his most moving works.
It is not really a symphony, though Tchaikovsky called it that (but didn't give it a number). In fact, it's quite similar to Harold in Italy, which Berlioz also called a symphony. We might call this a four-movement symphonic poem. A wealth of melodic material is beautifully orchestrated but loosely structured, without real symphonic architecture.

As for recordings, some of us remember the smooth Previn, the rich-sounding Maazel, the brilliant Ashkenazy, and decent performances by Ormandy and Rostropovich (all gone, except Previn). Some remember further back to the Toscanini, recorded in 1946. It sounded better on CD than one would expect for its age. Certainly it was a powerful and gripping performance---but also, as usual with Toscanini, rather driven and relentless.

Closest to Toscanini of any in modern sound was the Jansons. It was taut and brilliant, but had no soul (and the organ sounded like an overgrown accordion). The Simonov is sprawling and turgid, with hollow sound.

The Chailly was spectacular, the sound rich and unstrained, with the gorgeous Concertgebouw ambience. The Michael Tilson Thomas sounded boxy and confined, with glassy high strings---a shame, because MTT is probably the best living conductor of this piece, and Chailly is very ordinary. If only Thomas had had the recording company and engineers Chailly had!

Raymond Leppard seems too English to conduct Tchaikovsky well, and he hasn't recorded much, but his Manfred is pretty good. It's never indulgent or flaccid; there's a
``classical'' alertness to it. It's a bracing performance with some very convincing phrasing and plenty of forward momentum. Jansons is too fast; Leppard is also fast, but he knows how to phrase and let the music breathe naturally---and Jansons did not. Leppard does not conduct for sensualists who like to wallow in the music.

The Abravanel was dismal and tired and unromantic. Koizumi gets soggy at times and does not sustain its great moments. The strings seem weak. Litton seemed rather bland and underplayed, but still polished and nicely paced.

Masur's Leipzig recording for Teldec was very good, with plenty of poetry, imposing sweep, and a grand stride in the finale---quite remarkable since his other Tchaikovsky symphonies are utterly plodding, colorless, and uninspired. Masur should steer clear of the Russians.

The Goossens on Everest is a major achievement, very colorful and expressive, with overwhelming climaxes. (You can hear why Koussevitzky wanted Goossens to succeed him in Boston.) It was recorded in 1959 but sounds very good, and the London Symphony plays radiantly. Its only serious drawback is massive cutting: whole sections of the music have disappeared.

Most of us dislike Pletnev.
On Naxos Leaper lacks passion and
sweep---is in fact rather boring.
Muti's Manfred was absolutely glorious
and electrifying and sounded fantastic. He even had a great organ. Every movement has a moment or two when you can't believe how beautiful and thrilling it is. It should always be available---and not just in a set. The Previn comes in a two-disc set with Rachmaninoff; he shapes things beautifully, and the organ
sounds good.
Svetlanov is very exciting and very Russian, We who know it love it. It is the best you can get on a single disc right now.

\bigskip
\begin{tabular}{ l r }
Muti &EMI NA\\
Svetlanov & Melodiya NA\\
Previn & EMI 69776 [2CD]
\end{tabular}

\section{Piano Concerto 1}

This is the very essence of a warhorse, and though there are 66 recordings in the current catalog the total made over the years must be at least double.

Earl Wild has the best-balanced interpretation in superb sound: airy, cushioned strings and plenty of warmth and ambience. And the orchestra was Beecham's Royal Philharmonic---one of the greatest at the time. Rubinstein and Leinsdorf treat the music with respect; they don't just take the old warhorse for another ride. Sometimes the coordination seems choppy, but that is partly Leinsdorfs style. There is something too earthbound about that recording, and conductor and pianist do seem mismatched. Arrau did it many years later with the same orchestra (Boston): he was very poetic and they sounded gorgeous (Philips). That was a low-key recording but more inspired than the Rubinstein.

Most of us think the Cliburn overrated, but it is not a shallow, splashy recording; it has some depth. Majesty and breadth were emphasized in the Ousset/Masur (EMI, deleted) at the expense of excitement. The BIS is rather routine. Gavrilov with Muti was extroverted and imposing in the grand manner, but Gavrilov redid it with Ashkenazy, and that has more dignity and lyricism.

Ashkenazy hasn't played the work in many years, but his recording with Maazel from the early 60s had dark, rich sound. Entremont/Bernstein was just right---beautiful team-work---but has never made it to CD. The Bolet was stodgy.

Barry Douglas gave us good solid playing without much individuality. He is not a very interesting pianist, but Slatkin's fine conducting and beautiful sound helped. Mr Douglas plowed thru it with bright tone and good articulation but few new ideas. Still, a blunt and straightforward reading can be refreshing, and I liked it very much. The coupling, Tchaikovsky's Concert Fantasy, is usually boring. This was one of its few exciting recordings. Only the Glenser on Naxos comes close.

Pogorelich is the opposite: eccentric and very slow; he preens before the microphones for his fans. Berman/Karajan was interesting He tries to avoid any hint of virtuoso display, but it is smooth and consistent, and the teamwork is great. Berman gets Karajan moving; no one else does. Certainly Richter doesn't: massive, turgid, incoherent. Richter's own playing is powerful, understanding, and authoritative; but most of us do not like the accompaniment.
Richter's most exciting recording is with Ancerl, but the sound is terrible. Graffman/ Szell is a harsh-sounding recording of a tense, tonally unalluring, businesslike performance with no particular insights. Only Sell's most loyal fans could stand it.

Jon-Kimura Parker on Telarc was rather short on temperament but quite pleasant. The other Telarc is Horacio Gutierrez, who thumps
and pounds mercilessly. And David Zinman is one of the most unromantic conductors around. He is fast and bland, and the sound is about the worst you can hear on Telarc: both thin and muddy!

Gilels recorded this with Reiner, Maazel, Mehta, and Mravinsky. We are happy with the Maazel, but the Mehta is lumpy and lackluster; the conductor does make a difference, and Maazel is bracing and energetic. The main theme is majestic, but there's plenty of passion and drama as the music moves along. The recording is perfectly balanced. RCA has reissued the one with Reiner (1955). It sounds much better on CD than on LP, but one would hardly call it perfect sound. Clarity and vigor mark the interpretation, but some of us prefer other characteristics.

Engerer is relaxed and introspective, a bit lacking in energy, but a different way to hear the work (Denon). Most will find it slow and labored. Watts on Sony is stiff and earth-bound---too bad, because Bernstein conducts, and his feel for Tchaikovsky is palpable. The Telarc Watts is much better played: fluent and melodious, with many special touches. II is especially tender.

Pennario took a light approach: Tchaikovsky as Mendelssohn. Some found it refreshing. Paik and Zander also approach it anew and make it seem fresh. Naxos has two recordings of this. The older one, by Joseph Banowitz, has some coordination problems in III, but I is powerful and poetic and II light and graceful.
The newer one, by Bernd Glemser, is fresh and vigorous and altogether appealing, with excellent conducting.

Martha Argerich has recorded it three times---with Dutoit, Kondrashin, and Abbado.
All are breathtaking and sometimes breathless,
with her usual sweep, command, and charisma. The first version is more relaxed and has better proportions than the other two. The Kondrashin is exciting but not as finicky as the Abbado. Over the years she has become fussier, and sometimes she just plays with the music. Some of her phrasing has become rather nervous, too. If you must have Argerich, we prefer the first, though the latest one does have much better sound.

Peter Rösel with Masur is very solid: great technique, personal charm, gorgeous sound.
It's not very unusual or striking---just a really good, straightforward, satisfying performance.
Sultanov is rather routine, Lowenthal drab.
Pletnev: no majesty or sense of occasion, inadequate orchestra, very ordinary piano sound.
Feltsman: rather cold, except for the conducting. Berezovsky: decent playing, weak orchestra, nothing memorable. Nakamura: pompous, mannered, sluggish. Rudy: sluggish. Ogdon: wooden and tentative. Eresko: fresh, alert, spacious, warmly romantic---and deleted. Derek Han (deleted) was stuck with substandard orchestral playing. Historic: Horowitz/Toscanini---great fire and intensity (1943 Carnegie Hall).

\bigskip
\begin{tabular}{l r}
Wild & Chesky 13\\
Ashkenazy& London NA\\
Glemser& Naxos 550819\\
Rösel& Berlin 9308\\
Gilels & EMI NA
\end{tabular}


\section{Piano Concerto 2}

The Second Concerto was given its world premiere by the New York Philharmonic in 1881. They played it again in 1903 and 1923, but have they played it since? No one does! It was written around the time of the Fourth Symphony and Swan Lake and it resembles both. All three are optimistic works, and that was unusual with Tchaikovsky. ARG Editor James Lyons once offered an explanation for its neglect: ``There are many for whom Tchaikovsky without tears is not Tchaikovsky''. But there are no tears in the Fourth either. The melodies are as attractive as one expects from Tchaikovsky. It's as well crafted as the First Concerto, and it seems even more difficult for the pianist. That may be the real reason it is seldom played. True, \No1 is dreamier, more rhapsodic. This work is brighter, more alert-sounding.

When Tchaikovsky wrote this he got carried away in II and wrote long solos for violin and cello, as if he forgot it was a piano concerto. It becomes a triple concerto or perhaps even a piano trio, because it is such intimate music. His friends and advisors pointed out that II was swollen way out of proportion.

They were right, but he refused to change it as long as he lived. It is inspired music---very, very beautiful. After he died, the concerto was published with huge cuts and alterations (II was cut in half by Alexander Siloti and a preface that claimed the composer's sanction. Nonsense! Tchaikovsky wanted ONLY his original version to be performed and to his death refused to allow Siloti's version to be published. It may be a better concerto with the cuts, but the music is so beautiful that we want to hear it---and the composer wanted us to! Sometimes heavenly rambling is just the thing.

In 1965 Columbia ended a long drought for lovers of this work and issued it with Gary Grafiman and Ormandy. It was a joy, even if II was cut (the traditional Siloti cuts). About 10 years later the Spivia Kersenbaum/Martinon came to the USA. If Martinon's I seemed too fast after Ormandy, his III made Ormandy seem too fast. The Martinon was uncut. The pianist was quite wonderful; the orchestra sounded rather distant, as in a cavernous hall. It has been reissued in Europe by EMI.
Michael Ponti does a great job with the piano part, but the orchestra and recording are just not acceptable. Barry Douglas did not hare that problem: Slatkin and the English orchestra were excellent, the pianist straightforward but articulate. The uncut II should have been more sensitive; it was much too matter-of-fact.

Hans Werner Haas with Inbal was rather passionless and the orchestra rather distant.
The 1965 Magaloff/Davis was very well conducted, and the orchestra sounded good. But the pianist himself defeated the recording: tentative, unsure.

Melodiya owns a Zhukov recording---one of the few Russian ones not cut. Zhukov is exciting but sometimes brutal, and the orchestra is a little distant, though Rozhdestvensky is the fine conductor.

EMI has deleted Donohoe, a recording few of us liked, despite its fine sound and orchestra. Giles did all three concertos with Maazel for EMI. II was cut, but what was left of the movement was operatic in quality. I was hairraising in its virtuosity---and also its depth. III was peasant-like in its roughness and relent-lessness-not up to the Cherkassky level.

Leonskaya and Masur in Leipzig were uninspired.

Derek Han destroys III, and his orchestra is bad. Pommier was slack and languid---no fire. Shura Cherkassky was a barn-storming virtuoso, and poetic as well. His Cincinnati recording used the cut version of II, but the reading was warm and committed and passionate, with seamless phrasing and plenty of energy. Here is one pianist who can really bring it off!

The Boriskin on Newport was a solid run-through. We don't like the Pletnev: no way passion, beauty, choppy conducting of a tu small orchestra. David Bar-Illan had beautiful tone and varied his dynamic levels to come the poetry of the music. The orchestra was superb and sounded good, too. The only problem was the traditional cuts.

Lowenthal restores that cut material, but his piano tone is so hard that it even seems clattery at times. And the whole thing sounded under-rehearsed (Concerto 3, too)---pianist (fistfuls of wrong notes) and orchestra---the sound is congested and one-dimensional. Another one full of wrong notes was Jablonsk (with Dutoit): not very satisfying in any respect. Postnikova/Rozhdestvensky was made in Vienna. It is 50 minutes long (45 or 46 is more usual); and the pianist characterizes
everything.

Naxos again comes thru: their recording is inferior to none. In every respect it is outstanding: sound, piano, pianist, orchestra, conducting. It is so good that we almost decided to forego the preceding discussion (the competition is really weak). The Concert Fantasy is the coupling---a work that usually bores the pants off us. Only Barry Douglas brought it to life as vividly as Bernd Glemser does.

\bigskip
\begin{tabular}{l r}
Glemser &Naxos 550820
\end{tabular}

\section{Piano Concerto 3}

This started life as a symphony. Tchaikovsky decided to make the three movements he had written into a piano concerto and had orchestrated the first movement by the time he died. Taneyev, a former student and close friend orchestrated the other two movements. That's probably why many recordings of this are only one movement---including the best performance ever, Graffman/Ormandy (not on CD). (Other one-movement recordings are by Gavrilov, Pletnev, Feltsman, Eresco, Lowen. thal, Tozer, and Douglas.)

The music doesn't have the depth and mastery of Tchaikovsky at his best, but its occasional appearance on a concert program would be more welcome than yet another tired, stale traversal of the much-abused Concerto 1. All three movements are well crafted and tighter and more compact than either of the earlier concertos. Taneyev's orchestration is a pretty good imitation of Tchaikovsky's, perhaps not as fresh and original.

The Feltsman is well played, but Rostropovich's conducting is what makes it worth hearing. Gavrilov with Ashkenazy was good. So was Eresko. Slatkin's conducting helped Barry Douglas.

Ponti was the first widely-available recording, and it's on CD; but it's an old-sounding recording by an inferior orchestra. In fact, the Ponti has always reminded us that Tchaikovsky didn't finish the concerto because he felt the material was inferior.

The latest recording is Glemser on Naxos, and it will make you feel better about the music. The first movement is not as crisp as Ponti or Graffman, but II is lyrical and glowing (it parallels the same movement of Concerto 2, the piano duetting with the cello quite a bit). III is played with bravura and elan. The coupling is a very good Concerto 1.

\bigskip
\begin{tabular}{l r}
Glemser & Naxos 550819
\end{tabular}

\section{Violin Concerto}

Heifetz is the most exciting and virtuosic. Oistrakh has more Slavic soul, and Ormandy is warmer than Reiner but less fiery. II is nicely melancholy. Munch was in a rare reflective mood when he conducted for Szeryng: that is the most poetic reading of all. After these three---all old recordings that sound great everything else is a PS. (Szeryng may be hard to find.)

Kremer is well balanced, Mullova passionate. Kantorow was very individual---different from any other recording. Takezawa was gentle. The Suwanai is marred by audience noises. Isaac Stern is overripe, full of slides and scoops, and too far in front of the orchestra. Rachlin is polished but seems unexpressive compared to many others.

\bigskip
\begin{tabular}{l r}
Heifetz/Reiner &RCA 5933, 61743, 61495\\
Oistrakh/Ormandy & Sony 46339\\
Szeryng/Munch & RCA NA
\end{tabular}

\section{Suites for Orchestra}

Some of Tchaikovsky's freshest and most winsome music can be heard in these suites---a form Tchaikovsky almost invented. A lot of it sounds like his ballet music. Perhaps he felt freer in this form than with the symphony.

Neeme Jarvi is very straightforward in Suite
3: no charm or magic. It can never be compared to Svetlanov or Kondrashin or Michael Tilson Thomas. At least it's not as heavy as the Maazel was. Jarvi did much better in Suite 2, but none of his performances (he's done all four in Detroit) are anything to get excited about.

The Rozhdestvensky \No3 on Erato was special, but it is deleted. Complete sets are by Dorati, Belohlavek, and Sanderling. The Doratis are on Philips, and they were done with the Philharmonia, so you get Hugh Bean as violin soloist in 3. Dorati also has the most pungent accordions in 2:III. The polonaise in 3 is very grand, and the waltzes in 1:II and 2:II have a to Svetlanov.

Supraphon gives Belohlavek detailed sonics, and he brings passion and intensity. There are missed opportunities in some details, but overall the playing is marvelous, and the interpretations do not have the excesses of Svetlanov (which some of us really like). Sanderling has a good feel for the music and gives each suite plenty of character. \No2 is especially dramatic. In fact, after the great disc of 1 and 2, Suite 3 seems lackluster. Partly it's tempos, partly the strings and even the violin soloist. At least Naxos sells the discs separately, pairing 1 and 2 and 3 and 4, as BMG did for Svetlanov. \No4, Mozartiana, is far better at Sanderling's reasonable tempos than rushed by Bélohlavek. Sanderling is the golden mean; Svetlanov may seem too slow to some. Supraphon's bright sound emphasizes winds and horns; the darker Naxos sound has a much more satisfying bass and richer strings.

Michael Tilson Thomas recorded all but \No1, and we await a Sony CD reissue. (There was a CBS of 2 + 4.) They were quite wonderful: buoyant without slighting the great melodies. Don't forget the classic Svetlanovs: lovely readings, deeply felt and aptly paced and phrased. Everything feels just right. The climaxes seem natural and inevitable, and Svetlanov seems to speak with the composer's own voice. The strings are utterly gorgeous, especially in \No3; and in that same suite Svetlanov gives the waltz an intoxicating lilt. In fact, Svetlanov's 3 is the best by far.

\bigskip
\begin{tabular}{l r}
Svetlanov& Melodiya NA\\
Sanderling&Naxos 550644 (1+2)\\
Belohlavek& Supraphon 0969 [2CD]\\
Thomas & Sony NA\\
Dorati & Phi 454253 [2CD]
\end{tabular}


\section{Serenade for Strings}

There have been four great recordings: Karajan, Ormandy, Bernstein, and Svetlanov. The Philips Stokowski (not listed in current catalogs) wasn't all it should have been. Perhaps Philips engineering was just wrong for Stokowski (though it was pretty good for Colin Davis).

The DDD Karajan doesn't sound quite as good as the earlier ADD Galleria. In this music, if it's not lush, it's not right. Tchaikovsky wanted masses of strings---``the more the better''--- and that rules out chamber orchestra versions (most of the catalog). We were not taken with Kantorow or Rachlevsky. Note, however, that we praised two chamber versions: Australian Chamber Orchestra on Omega---``sounds larger than it is...sounds gorgeous''---and Norwegian Chamber Orchestra on Simax. The sound must be plush and full and opulent. The Bashmet on RCA was creamy and smooth as silk, 
but even with a full-sized string section the Litton sounded undernourished.

Charles Munch conducted this with warmth and fluid grace, his tempos crisp but still natural in flow. The slow movement sings.
The Boston strings sound lighter than many, but there is still plenty of sensitivity. A number of our writers speak highly of the Barbirolli (intermittently available). Karajan demonstrates again how richly he deserves to be called a Great Conductor. He refuses to merely play it through. He feels it, and he makes us feel it. Bernstein's New York Philharmonic recording was very stylish and had the most elegiac Elegy. But Svetlanov is definitely the most emotional and expressive. The USSR strings are as good as any anywhere. No matter what version you have, we urge you to try Svetlanov to hear what this music can be.

Ormandy is fastest---and quickest: 22 minutes versus Svetlanov's 33. (He avoids repeated material and drops the three-minute development in IV.) But what a heady sound those Philadelphia strings make---strong and majestic. The Ormandy was 1960, the Svetlanov 1970; both sound simply wonderful on CD---better than almost anything else.

We should mention Slatkin on Telarc: beautiful sound but a rather ordinary interpretation.

\bigskip
\begin{tabular}{l r}
Svetlanov &Melodiya 37878\\
Ormandy & Sony NA\\
\end{tabular}

\section{Rococo Variations}

Everyone has a favorite cellist, so we may be wasting our breath covering this. We don't want to talk you out of Rostropovich, but let us put in a good word for Walerska (gorgeous singing tone and not as dramatic or flamboyant as Rostropovich) and Haroy (expressive, with a real individual profile and a more tenor-like tone, not as deep as Walerska's). The Walerska was with a wonderful Dvorak Concerto; the Haroy was all Tchaikorsky cello pieces. And the Yo-Yo Ma (comes with Prokofief) has to be one of the most beautifully played.

\bigskip
\begin{tabular}{l r}
Walevska& Philips or IMP NA\\
Hamoy& RCA NA\\
Ma& Sony 48382
\end{tabular}

\section{Swan Lake}

The three Tchaikovsky ballets are the essence of ballet and the pillars of the repertory. Swan Lake was the first, and it remains the most beautiful of all ballets---and one of the few worth listening to from beginning to end.
Since the current catalog has about 85 Swan Lake listings, it is impossible to discuss every recording.

The London set from Dutoit has lustrous string tone, but the conducting is so neutral it says nothing at all and leaves you cold. It is never ardent or glowing, passionate or sensitive never noble or exhilarating. It just is. At least his rhythms are not as rigid and Germanic as Sawallisch's. Sawallisch doesn't convey joy or sweep, and his climaxes are heavy-handed, but at least the Philadelphia orchestra sounded really good. Ozawa was much lighter than Sawallisch, quite balletic, but often prosaic (DG). The Fedotov was billed as the ``original version'', but when you read the fine print you find it is not Tchaikovsky's original but Drigo's. We prefer Tchaikovsky's score, but the Fedotov was exciting on the fast side, with powerful brass.

Michael Tilson Thomas conducted a ``big moments'' performance, marking time between massive outbursts of brass and percussion. As with his more recent Prokofieff Romeo and Juliet, it proves wearisome. Slatkin's RCA recording was similar in that only the big moments made much of an impression; the rest sounded like sight-reading. Mark Ermler's Conifer recording has clean phrasing, good integration, sensible pacing, and superb sonics. Svetlanov (Melodiya) was in top form, perfectly attuned to the drama and tragedy of this great score.

There have always been great single discs of the best selections from the ballet, and the best of those are by Arthur Fiedler (RCA 7879) and Eugene Ormandy (Sony 46341). Do not settle for a mere suite from this glorious ballet; get at least a full disc.

\section{Sleeping Beauty}

We were quite pleased when Naxos brought the Mogrelia. It made many older recordings (Dorati, for example) sound pedestrian and uninspired. And its sound is better than the more expensive Previn. Previn's tempos were pretty easy-going for home listening, even if they might be ideal for dancing. The RCA Slatkin was the opposite: great sound but ridiculously fast tempos, Gergiev is also rather slow for home listening, the string tone of his orchestra is terrible next to Slatkin or Previn, and his trumpet is strident. The conducting b stiff, wooden, stilted, and perfunctory. His disc of excerpts is not in the order of the ballet.
For this ballet, one can be content with a single disc of excerpts. Again, there are many listings, but watch that you get a full disc of Sleeping Beauty and not a mere suite. The excellent Ormandy is Sony 46340, and some d us wish the Monteux and the Stokowski were around.

\section{Nutcracker}

Many more Nutcrackers remain in the catalog because the companies expect seasonal sale every year. Thus DG has kept the Ozawa around. It is well-mannered, well played decently recorded, and utterly generic and ordinary. Ashkenazy seems limp and shapeless---no magic. The Telarc Mackerras Is pretty impressive, but some tempos are unduly brisk (``Waltz of the Flowers''), Previn is slower, and that allows for more vibrant string tone; but some scenes are rather tepid. Svetlanov is dramatic often too much so---and the trumpets are shrill. The Jansons is favored by some of us for the way it ``captures the magical world of childhood'', and some of us consider the Rodzinski a classic of magic and poetic intimacy. As for selections, again we beg you not to settle for the suite. There is so much more! It's too bad the whole ballet won't fit on one disc.
Actually, Dorati takes 79 minutes for it, but it's on two discs! Most performances are more like 85 minutes.) Michael Tilson Thomas gives us
71 minutes (Sony 62675); we wish he had recorded the rest, because his touch is so perfect for this music. We also wish the Philharmonia had recorded this with more strings.
There was once a full disc from Ormandy---we need it now.

\section{1812 Overture}

Many an 1812 has sold well on the basis of its cannon shots, but cannons are noise, not music. Still, Tchaikovsky described it as ``loud and noisy'', so why not cannons? If gunfire is your main criterion, the Dorati on Mercury makes a strong impression (though the Mercury sound is tinny, so I prefer his Detroit recording on London). Others are Bernstein on Sony and Gibson on Chesky (the latter coupled with a tepid Pathetique). There was an even bigger sounding one conducted by Eduardo Diazmunoz---half the Mexican army must have been involved, along with a big organ.
Not subtle, but this isn't subtle music---overkill is impossible. Reiner merely beats time and cuts a big chunk out of the middle. The Chicago Abbado is boring, Svetlanov hectic.

Recordings with cannon shots are marked
CA. The composer would also have loved the extra brass (EB) and big bells (BB) some conductors add to the climax: what a glorious noise they make! A chorus (CH) doing the opening hymn can be very moving. Literalists sniff at all this ``adding to the score'', but anyone with ears knows that it really helps the piece. Here is a list of the best recordings, sound included.

\bigskip
\begin{tabular}{l r}
Haitink, EB & Phi NA\\
Gould BB,CA & RCA NA\\
Fiedler BB,EB, CA & DG NA\\
Ormandy BB,EB,CA,CH& Sony 46334\\
Dorati BB, CA & Lon 443003\\
Karajan BB, CA,CH & DG NA\\
Buketoff BB, CH, & RCA 7731\\
Barenboim CA& DG 45 523\\
Davis CH & Philips NA\\
Litton CH & Delos 3196\\
Kunzel CA & Telarc 80041\\
Solti &London 430446\\
Bernstein &Sony 47634
\end{tabular}
\bigskip

\noindent
Haitink leads a solid, very musical performance with a wonderful orchestra and sound.
How perfectly it builds up to the thrilling entry of the extra brass: the Netherlands Royal Military Band. It was available at budget price, as was the Buketoff and the Ormandy. Gould is big and splashy and sounds good enough for the digital age. The Fiedler has fantastic sound---Symphony Hall sound---and is beautifully conducted. (He wore earplugs to block out the shotguns, which sound like Revolutionary War muskets.) Colin Davis also had Symphony Hall sound, along with chorus and organ.

Ormandy just feels so right. He has the Mormon Tabernacle Choir in English. (The dimmer-sounding RCA Ormandy has a different chorus, they sing in Russian, and they come back at the climax, which overloads the climax and destroys its effect. Stokowski did the same thing. You really can't have BB, EB, CA, and CH all in the same bars.) Dorati manages quite a majestic conclusion, and Lon-don's sound is excellent. Karajan starts off with the unaccompanied Don Cossack Choir; a very impressive, somewhat Prussian (military) performance follows. The only flaw is strident, piercing trumpets at the climax. Buketoff has stood up well over the years; his chorus and bells are among the best. The newest choral 1812 is the Dallas Delos, conducted by Andrew Litton. He has a big 200-voice chorus (they really sound wonderful!) and they sing all three sections based on hymns. The climax is congested---not the sound, but the texture: it's just too much. But many listeners will respond well to Litton's 1812. Barenboim has brilliant sound-even the strings have a bright sheen. The cannons shift back and forth between the speakers. The ones listed below are quite good but not on the level of our top choices. Kunzel's cannons are famous, but De Preist's are better (authentic 1812 models, too). So is the frank and natural Delos sound. But the performance is slow and tepid. Avoid: Sargent, Reiner, Markevitch, Jansons, Previn, Mehta, Maazel, and most that we haven't mentioned.

\section{Capriccio Italien}

After the trumpet introduction this popular piece is based on three main themes that may derive from Italian street songs. The second theme---the sweet and romantic one that turns up about five minutes in---first occurs after an Andante section and has the tempo marking Pochissimo piu mosso, or very slightly faster.
The tempo changes to Allegro moderato for its second statement, not much later; and that's the marking also for its return at the end. In most recordings it sounds the same all three times. In fact, if you never looked at the score, Herbert von Karajan's Capriccio Italien seems Strange, even perverse. He takes the theme very slowly at first, then a bit faster the second time, then faster yet at the end. So that theme becomes the focus of his performance. It is exaggerated, but what a striking and exciting impression it makes! Litton tried to do what Karajan did, but it didn't come off, partly because the return of theme drowns in tub-thumping (percussion). Likewise Ashkenazy: we sit amid the cymbals and drums, straining to hear the strings. 
Is this engineering or conducting?

Ormandy (Sony) as usual hardly interprets at all---or so it seems, because it all flows so naturally and tastefully. He balances so the strings dominate, and that pays off. That second theme never sounded more romantic. All is primary colors, but what flowing, vibrant colors they are! The RCA Ormandy is less vivid.
RCA does offer what one critic called ``one of the most thrilling Tchaikovsky recordings ever made''---Eduardo Mata's Capriccio Italien. The sound, including the strings, is excellent; and the tempos are fast, but that only adds vitality---it doesn't seem rushed. Still, after the Karajan and the Ormandy, we can't enthuse as wildly as our brother critic did. Some of us are fond of the Bernstein (though it is shamelessly indulgent); others don't respond so well to the frequent tempo shifts. Smoothness is missing, but it has passion.
Nice, but nothing special: Kunzel (sweet and simple), Barenboim, Jansons, Dorati, Stokowski (yes! It was for Philips and seemed rather pale for old Stoky.) Boring, superficial: Gergiev, Mehta, Ozawa.

\bigskip
\begin{tabular}{l r}
Ormandy &Sony 47657\\
Karajan & DG NA\\
Mata &RCA 63586\\
Bernstein & Sony NA\\
Ancerl& Supraphon 1943
\end{tabular}

\section{Romeo and Juliet}

Such passionate music! A good performance must capture both the hotheaded conflict between the clans and the soaring rhapsody of the young lovers. You'd be surprised how very few manage that. You might think Solti could, but his R\&J is very tame, and the sound of his trumpets and strings is hard to take. Surely Muti? He was bland both times. No English conductor seems to have any passion at all:
Sargent, Boult, Davis, even Previn (English orchestra---but his EMI at least has gorgeous sound, and that counts). Ashkenazy, with a London orchestra, didn't clash, didn't soar, and missed the hushed stillness after the first love scene. Chailly conducted extremely well: he opens with dark foreboding, and the love theme soars. But the strings are grainy and close-up. Munch conducted just as well---he certainly had passion and flair. His Boston recording, not the terrible Paris one, is missing only some of the rhapsody from the love music. (It's also missing from the catalog.)

Next level down but still very good: Litton.

Done in by inferior sound: Shaw and Barenboim---thin, piercing, tinny. Sound alone would place the Maazel miles ahead of most, had Telarc not deleted it. The strings are richer than any Vienna recording, but they do not swoon with passion the way Ormandy gets his strings to. The Vienna Karajan (on Decca) also has rich, warm strings. (Side by side with Berlin, you'd have to be deaf not to choose Vienna.) Ozawa was fast, but orchestra and sound were very good. The Jansons has a fiery battle scene, but neither the performance nor the disc inspired us when we reviewed it Deleted but notable: Abbado in Boston (not Chicago---that's just a run-through). Giulini is less passionate than serene, but so beautiful. The emotional Rostropovich is pretty good but hard to find. We described the Bernstein as follows: The passion of the star-crossed lovers and the hammer blows of sword on sword, taken at a hell-for-leather tempo with a few extra cymbal crashes thrown in for good measure, make this wonderful music sound as fresh and virile as when the composer first set pen to paper. (The DG remake was also marvelous.)

\bigskip
\begin{tabular}{l r}
Ormandy & Sony 63281\\
Bernstein & Sony 47632\\
Maazel & Telarc NA
\end{tabular}




\section{Francesca da Rimini}

Only in the last ten years or so have recording become plentiful; it used to be hard to find. still doesn't seem to sell enough to keep great recordings in the catalog. The Philips Stokonski would be a top recommendation. The Dell'Arte Stokowski was an exciting performance Perhaps more exciting than the Philips was but quite a bit older and sounds it. The Evers transfer (9037) is better than the others.

Bernstein on DG or Sony is very exciting; the Sony was more indulgent, but the DG is better played and has much better sound. The Munch is fast, exciting, visceral; and Ormandy was quite stirring (1976, but it sounded far better than the 1985 digital Chailly). The Haitink, like the Ormandy, was beautifully played but slightly less passionate than Munch or Stokowski or Bernstein. The Giulini (on Seraphim?) has plenty of fire in the outer sections and plenty of poetry in the middle. The Rostropovich had all the emotion needed. There was a rather good Erato by Rozhdestvensky, and the Temirkanov was passionate. Eschenbach was dull. David Zinman ought to give up chaikovsky; he never gets it right. His Francesca is all wrong. Gergiev had furious tempos in some sections and ponderous ones in others- near-catatonic. Masur bored us with both his Francesca and Romeo.


\section{Tone Poems}

This category includes {\it Fatum}, {\it The Storm}, {\it The
Tempest}, {\it Hamlet}, and {\it The Voyevoda}. {\it The Storm}---really an overture---is rare, but its best recording comes with Wit's Tchaikovsky Fifth Symphony on Naxos. The Jarvi is as good, but it resides on a disc where nothing else is on that level, including {\it Fatum}. That one is even rarer, and you almost have to buy one of the tone poem collections (two discs) to get a good performance. There is no better recording of {\it The Voyevoda} than Andrew Litton's (Delos 3196), with his 1812. Both {\it Fatum} and {\it The Voyevoda} were beautifully recorded by Slatkin and issued with his Tchaikovsky Fourth on RCA (deleted). Neither is top-drawer Tchaikovsky---well, none of these are, but the best is probably  {\it Hamlet}. And its most famous and most moving recording is the Stokowski (Everest 9037, with Francesca). Similar in its dramatic impact is the Bernstein (Sony 47635, with his 6th). The DePreist {\it Hamlet} and {\it Tempest} (with his 1812 on Delos) did not impress us: slow, lacking in ardor and atmosphere.
Leonard Slatkin offered the best-sounding
Hamlet with his Pathetique (RCA NA).
The Fedoseyev {\it Tempest} is a let-down, and his {\it Voyevoda} is not the tone poem but a totally unrelated overture. Other {\it Tempest} let-downs: Abbado, Butt, Litton, Maga, Pletnev, Slatkin.

Maybe the best way to get to know it is on Naxos 550518 with Antoni Wit's Symphony 3.
The two sets of these are on Philips and London, two discs each. The London is led by Dorati, the Philips by Inbal and others. Neither
brings you the best of anything, but both include all of these, except that Dorati leaves out {\it The Storm}. The Philips collection seems better to us---more satisfying conducting.

\section{Souvenir de Florence}

Anything bigger than a quintet and you might as well go all the way. Tchaikovsky wrote this for six strings, but it is often recorded by larger groups. 35 or 40 of them seem to play in the Australian Chamber Orchestra recording (Omega 1010), and they sound rich and full and inviting. I Musici di Montreal are only about 15 players, and their tone is darker and more Russian---quite soulful, even if they sound less full (Chandos 8547). The Georgian Chamber Orchestra is not as dark and a little fuller, with classical restraint and clean lines.
The Zagreb Soloists recorded it with 11 players---probably not enough to please those of us who like a lot of strings. Their reading of II is gorgeous. The Kremlin Chamber Orchestra doesn't have the drive of the Zagreb group, but their sound is voluptuous and their lyricism warm. The Marriner recording seems rushed to us. Nor are we impressed with the Vienna Chamber Orchestra under Entremont, even though it does sound fresh and spontaneous.
That Naxos release pairs the Florence piece with the Serenade, and there simply aren't enough strings to make the Serenade work.
Watch for that: others are so coupled.
The Philharmonia Virtuosi use only six players for this, but we are not impressed.
They sound like they needed a conductor and a lot more strings; that is, they don't sound like a coherent chamber ensemble. We don't like the Vermeer Quartet in this: too tight and energetic; not enough degrees of intensity. The Schubert Quartet phrases better and has longer lines, but IV becomes plodding. The Boston Chamber Music Society (Northeastern
249) has serenity, transparency, nostalgia, melancholy, and variety of color and touch.
Great playing. The two Borodin Quartet recordings make it very personal and very Russian. Both come with the quartets (EMI and Teldec).

\section{Trio}

Our reviewers have rejected most recordings.
Beaux Arts is unromantic, prosaic, square; the Yuval insipid, the Delos recording uninspired. The Ashkenazy-Perlman-Harrell is cold, choppy, and insensitive---mostly the pianist's fault. The Chung Trio is mannered and heartless; the Rembrandt Trio on Dorian is utterly unromantic and just plain boring. The Suk trio is refined but weak in passion. The Borodin also lacks that; it's very serious, even depressed. But depression is not foreign to Tchaikovsky or to this music, and some of us like the second Borodin recording, coupled with Alabiev. Lin, Hoffman, and Bronfman on Sony left us pretty neutral---but Bronfman's playing often does.
The Haydn Trio of Vienna gave a vigorous and intelligent reading of wistful, not oppressive, melancholy---warm and sentimental, with very musical flow. The affect seemed just right. EMI had a fine performance by Cecil Licad, Nadia Salerno-Sonnenberg and Antonia Meneses- full of passion and romantic heat Golub, Kaplan, and Cart start out a little fast but then settle down to a beautiful, coherent performance with especially attractive piano tone. It is not as emotionally indulgent as the Naxos. Some will prefer its slight reserve, but it's hard not to love the Naxos. The Ashkenazy in the Natos is not Vladimir but his son, and he has ten times the passion and involvement. The Naxos is the best recording of this you can buy right now, at any price.

\bigskip
\begin{tabular}{l r}
Ashkenazy & Naxos 550467\\
Haydn & Teldec NA\\
Golub & Arabesque 6661\\
Licad& EMI NA
\end{tabular}

\section{Quartets}

Tchaikovsky's three quartets are seldom heard and were very rare before the CD era. Since the only recording most of us heard was by the Borodin Quartet, they have influenced the way we hear them to this day. Quartets 2 and 3 have never drawn much interest, but when I think of Quartet 1 I can hum the theme, and it's always the way the Borodins played it. They had passion, abandon, and ecstatic lyricism, and the way they played it makes groups like Moscow Quartet and St Petersburg Quartet sound shallow and ``wrong''. They leave us cold. And they are Russian. One Russian group we like is the Glinka Quartet on Sonora 22575---very Russian: they really dig in. Of non-Russians we like only the Emerson (very American---almost blunt, but emotional and beautifully played; DG 427616) and the Talich (full-toned; Calliope 9202).

Since the Borodin recorded this more than once, we should note that the early Melodiya (issued here on Angel and Columbia) had the freshest performance. The EMI (NA) is also good, but the sound is a little less satisfying.
The best sound is on Teldec 90422, but the players have lost some naturalness and finesse.

By the way, the famous Andante Cantabile is from the First Quartet.

\section{Piano Music}

Even people who love Tchaikovsky find it hard to love his two piano sonatas. The one in C-sharp minor is an early work, and parts of it turn up in symphonies. The one in G starts ou quite nice---even gets majestic---but most of it is just noisy and irritating. But he also wrote collections of short piano pieces that are often attractive---or at least have some nice pieces.
The most recorded is {\it The Seasons}, which should really be called {\it The Months}, because there are 12 pieces, one for each month. Somehow Tchaikovsky got roped into writing them, one at a time, for a monthly magazine. They are perfect little character pieces, but they also make an attractive set. They only last about 45 minutes, and unfortunately no one has come up with decent fill. Whatever recording of them you buy, you are buying for the {\it Months} alone.
Starkman on Pope is clattery, insensitive, and noisy. Kubalek on Dorian is superficial and small-scale. There was an excellent set from Katin on Olympia. He had the best technique of them all: smoother, evener touch, intelligent pedalling, better articulation---the voices are cleaner and better balanced. But he is not a Russian, and Pletnev, Postnikova, and Bronfman all get more out of the melancholy and romantic pieces. The Pletnev to have is the
1985 MK recording, not the dull Virgin remake.
Pletnev wrecks April, smashing the poor snowdrops into the mud; but at least six of these pieces are played better by him than by anyone. Postnikova comes to grief in the Troika (November---they already have snow in Russia!), and Pletnev gets it just right.

If you can find the MK Pletnev, get it by all means. If not, the Bronfman will do. This pianist, whose work I usually dislike, manage to get most of these pieces right and only makes a mess of March---and that's not one of the better pieces. The famous Barcarolle (une) is better with Pletnev, Postnikova, or Katin; bui most of the other pieces are quite satisfying with Bronfman. He never rushes thru them; he always lets us savor their melodies. His technique is not on Katin's level, but he is level---that is, he never seems eccentric, never seems inept, never uneven in his touch. Nor is he ever as delicate as Katin or Postnikova. And Pletnev is often more atmospheric.
This has been orchestrated, and the best recording of the orchestral version is by Gin on Ondine. Orchestras can sustain as pianos cannot, and {December} is one piece that sounds much better this way---as if it got left out of {\it Nutcracker} by mistake.

\bigskip
\begin{tabular}{l r}
Pletnev& MK 418008\\
Katin & Olympia 192\\
Bronfman & Sony 60689\\
Grin (orch) & Ondine 782
\end{tabular}

\end{document}
