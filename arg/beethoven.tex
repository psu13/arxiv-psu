\documentclass[12pt]{article}%
\usepackage{enumitem}
\setcounter{secnumdepth}{0}

%\usepackage{amsmath,amsthm,amscd,amssymb}
\usepackage[full]{textcomp}
\usepackage{XCharter}% lining figures in math, osf in text
\usepackage[scaled=1.04,varqu,varl]{inconsolata}% inconsolata typewriter
\usepackage[xcharter,vvarbb,scaled=1.03,smallerops,bigdelims]{newtxmath}
\usepackage[cal=cm,scr=esstix]{mathalpha}
\usepackage[papersize={7in, 10.0in}, left=.6in, right=.6in, top=.6in, bottom=.9in]{geometry}
\linespread{1.05}
\sloppy
\raggedbottom
\pagestyle{plain}

% these include amsmath and that can cause trouble in older docs.
\input{/Users/psu/arxiv-psu/helpers/cmrsum}
\makeatletter

\DeclareFontFamily{OMX}{MnSymbolE}{}
\DeclareSymbolFont{largesymbolsX}{OMX}{MnSymbolE}{m}{n}
\DeclareFontShape{OMX}{MnSymbolE}{m}{n}{
    <-6>  MnSymbolE5
   <6-7>  MnSymbolE6
   <7-8>  MnSymbolE7
   <8-9>  MnSymbolE8
   <9-10> MnSymbolE9
  <10-12> MnSymbolE10
  <12->   MnSymbolE12}{}

\DeclareMathSymbol{\downbrace}    {\mathord}{largesymbolsX}{'251}
\DeclareMathSymbol{\downbraceg}   {\mathord}{largesymbolsX}{'252}
\DeclareMathSymbol{\downbracegg}  {\mathord}{largesymbolsX}{'253}
\DeclareMathSymbol{\downbraceggg} {\mathord}{largesymbolsX}{'254}
\DeclareMathSymbol{\downbracegggg}{\mathord}{largesymbolsX}{'255}
\DeclareMathSymbol{\upbrace}      {\mathord}{largesymbolsX}{'256}
\DeclareMathSymbol{\upbraceg}     {\mathord}{largesymbolsX}{'257}
\DeclareMathSymbol{\upbracegg}    {\mathord}{largesymbolsX}{'260}
\DeclareMathSymbol{\upbraceggg}   {\mathord}{largesymbolsX}{'261}
\DeclareMathSymbol{\upbracegggg}  {\mathord}{largesymbolsX}{'262}
\DeclareMathSymbol{\braceld}      {\mathord}{largesymbolsX}{'263}
\DeclareMathSymbol{\bracelu}      {\mathord}{largesymbolsX}{'264}
\DeclareMathSymbol{\bracerd}      {\mathord}{largesymbolsX}{'265}
\DeclareMathSymbol{\braceru}      {\mathord}{largesymbolsX}{'266}
\DeclareMathSymbol{\bracemd}      {\mathord}{largesymbolsX}{'267}
\DeclareMathSymbol{\bracemu}      {\mathord}{largesymbolsX}{'270}
\DeclareMathSymbol{\bracemid}     {\mathord}{largesymbolsX}{'271}

\def\horiz@expandable#1#2#3#4#5#6#7#8{%
  \@mathmeasure\z@#7{#8}%
  \@tempdima=\wd\z@
  \@mathmeasure\z@#7{#1}%
  \ifdim\noexpand\wd\z@>\@tempdima
    $\m@th#7#1$%
  \else
    \@mathmeasure\z@#7{#2}%
    \ifdim\noexpand\wd\z@>\@tempdima
      $\m@th#7#2$%
    \else
      \@mathmeasure\z@#7{#3}%
      \ifdim\noexpand\wd\z@>\@tempdima
        $\m@th#7#3$%
      \else
        \@mathmeasure\z@#7{#4}%
        \ifdim\noexpand\wd\z@>\@tempdima
          $\m@th#7#4$%
        \else
          \@mathmeasure\z@#7{#5}%
          \ifdim\noexpand\wd\z@>\@tempdima
            $\m@th#7#5$%
          \else
           #6#7%
          \fi
        \fi
      \fi
    \fi
  \fi}

\def\overbrace@expandable#1#2#3{\vbox{\m@th\ialign{##\crcr
  #1#2{#3}\crcr\noalign{\kern2\p@\nointerlineskip}%
  $\m@th\hfil#2#3\hfil$\crcr}}}
\def\underbrace@expandable#1#2#3{\vtop{\m@th\ialign{##\crcr
  $\m@th\hfil#2#3\hfil$\crcr
  \noalign{\kern2\p@\nointerlineskip}%
  #1#2{#3}\crcr}}}

\def\overbrace@#1#2#3{\vbox{\m@th\ialign{##\crcr
  #1#2\crcr\noalign{\kern2\p@\nointerlineskip}%
  $\m@th\hfil#2#3\hfil$\crcr}}}
\def\underbrace@#1#2#3{\vtop{\m@th\ialign{##\crcr
  $\m@th\hfil#2#3\hfil$\crcr
  \noalign{\kern2\p@\nointerlineskip}%
  #1#2\crcr}}}

\def\bracefill@#1#2#3#4#5{$\m@th#5#1\leaders\hbox{$#4$}\hfill#2\leaders\hbox{$#4$}\hfill#3$}

\def\downbracefill@{\bracefill@\braceld\bracemd\bracerd\bracemid}
\def\upbracefill@{\bracefill@\bracelu\bracemu\braceru\bracemid}

\DeclareRobustCommand{\downbracefill}{\downbracefill@\textstyle}
\DeclareRobustCommand{\upbracefill}{\upbracefill@\textstyle}

\def\upbrace@expandable{%
  \horiz@expandable
    \upbrace
    \upbraceg
    \upbracegg
    \upbraceggg
    \upbracegggg
    \upbracefill@}
\def\downbrace@expandable{%
  \horiz@expandable
    \downbrace
    \downbraceg
    \downbracegg
    \downbraceggg
    \downbracegggg
    \downbracefill@}

\DeclareRobustCommand{\overbrace}[1]{\mathop{\mathpalette{\overbrace@expandable\downbrace@expandable}{#1}}\limits}
\DeclareRobustCommand{\underbrace}[1]{\mathop{\mathpalette{\underbrace@expandable\upbrace@expandable}{#1}}\limits}

\makeatother


\usepackage[small]{titlesec}
\titlelabel{\thetitle.\quad}

\usepackage{cite}
\usepackage{microtype}

\def\startrectable{\bigskip
\begin{tabular}{ l r }}
\def\endrectable{\end{tabular}}

% hyperref last because otherwise some things go wrong.
\usepackage{hyperref}
\hypersetup{colorlinks=true
,breaklinks=true
,urlcolor=blue
,anchorcolor=blue
,citecolor=blue
,filecolor=blue
,linkcolor=blue
,menucolor=blue
,linktocpage=true}
\hypersetup{
bookmarksopen=true,
bookmarksnumbered=true,
bookmarksopenlevel=10,
pdfencoding=auto, psdextra
}

% avoid weird spacing after the period here.
\def\No#1{No.\@ #1}

% make sure there is enough TOC for reasonable pdf bookmarks.
\setcounter{tocdepth}{3}

%\usepackage[dotinlabels]{titletoc}
%\titlelabel{{\thetitle}.\quad}
%\titleformat{\section}[block]
  {\fillast\medskip}
  {{\thesection. }}
  {1ex minus .1ex}
  {\scshape}
 
\titleformat*{\subsection}{\itshape}
\titleformat*{\subsubsection}{\itshape}

\setcounter{tocdepth}{2}

\titlecontents{section}
              [2.3em] 
              {\bigskip}
              {{\contentslabel{2.3em}}\large\scshape}
              {\hspace*{-2.3em}}
              {\titlerule*[1pc]{}\contentspage}
              
\titlecontents{subsection}
              [4.7em] 
              {}
              {{\contentslabel{2.3em}}}
              {\hspace*{-2.3em}}
              {\titlerule*[.5pc]{}\contentspage}

% hopefully not used.           
\titlecontents{subsubsection}
              [7.9em]
              {}
              {{\contentslabel{3.3em}}}
              {\hspace*{-3.3em}}
              {\titlerule*[.5pc]{}\contentspage}
%\makeatletter
\renewcommand\tableofcontents{%
    \section*{\contentsname
        \@mkboth{%
           \MakeLowercase\contentsname}{\MakeLowercase\contentsname}}%
    \@starttoc{toc}%
    }
\def\@oddhead{{\scshape\rightmark}\hfil{\small\scshape\thepage}}%
\def\sectionmark#1{%
      \markright{\MakeLowercase{%
        \ifnum \c@secnumdepth >\m@ne
          \thesection\quad
        \fi
        #1}}}
        
\makeatother



%\makeatletter

 \def\small{%
  \@setfontsize\small\@xipt{13pt}%
  \abovedisplayskip 8\p@ \@plus3\p@ \@minus6\p@
  \belowdisplayskip \abovedisplayskip
  \abovedisplayshortskip \z@ \@plus3\p@
  \belowdisplayshortskip 6.5\p@ \@plus3.5\p@ \@minus3\p@
  \def\@listi{%
    \leftmargin\leftmargini
    \topsep 9\p@ \@plus3\p@ \@minus5\p@
    \parsep 4.5\p@ \@plus2\p@ \@minus\p@
    \itemsep \parsep
  }%
}%
 \def\footnotesize{%
  \@setfontsize\footnotesize\@xpt{12pt}%
  \abovedisplayskip 10\p@ \@plus2\p@ \@minus5\p@
  \belowdisplayskip \abovedisplayskip
  \abovedisplayshortskip \z@ \@plus3\p@
  \belowdisplayshortskip 6\p@ \@plus3\p@ \@minus3\p@
  \def\@listi{%
    \leftmargin\leftmargini
    \topsep 6\p@ \@plus2\p@ \@minus2\p@
    \parsep 3\p@ \@plus2\p@ \@minus\p@
    \itemsep \parsep
  }%
}%
\def\open@column@one#1{%
 \ltxgrid@info@sw{\class@info{\string\open@column@one\string#1}}{}%
 \unvbox\pagesofar
  \gdef\thepagegrid{one}%
 \global\pagegrid@col#1%
 \global\pagegrid@cur\@ne
 \global\count\footins\@m
 \set@column@hsize\pagegrid@col
 \set@colht
}%

\def\frontmatter@abstractheading{%
\bigskip
 \begingroup
  \centering\large
  \abstractname
  \par\bigskip
 \endgroup
}%

\makeatother

%\DeclareSymbolFont{CMlargesymbols}{OMX}{cmex}{m}{n}
%\DeclareMathSymbol{\sum}{\mathop}{CMlargesymbols}{"50}
%\pdfbookmark[1]{Introduction}{Introduction}

\begin{document}

\title{Beethoven Overview 2003}
\date{}
\maketitle

\tableofcontents

\newpage

\addcontentsline{toc}{section}{\protect\textbf{Introduction}}
\noindent
Beethoven is the cornerstone of the symphonic repertoire. His music is the summation of everything that came before him and the progenitor of nearly everything that has come since. Who better combined the intellectual rigor of Bach, the classical balance of Mozart and Haydn, and the expressive sensibilities of the romantics? None of his successors were able to emulate him, but most of them are indebted to him in one way or another. The vocal symphonies of Mahler, Shostakovich, Vaughan Williams, and even Havergal Brian would have been unthinkable without the precedent of Beethoven's 9th.

Whether Beethoven was a classicist or a romantic has always been a subject of debate.
His profound emphasis on organic form suggests the perfection of high classicism, and so it is. But the inspiration to reach those heights of structure and balance was a highly romantic emotional sensibility. It's a synthesis of both;
Beethoven really only sounds like himself.
However, this duality of heart and mind may explain why both classical and romantic interpretive approaches work.

The music is full of great intellectual struggles invested with almost superhuman emotional intensity. Beethoven wrestled and fought and finally subdued each work into an immutable whole so complete that when we hear it we can't imagine how it could be different. No wonder he's so popular in our willful age: the triumph of will over nature. Yet he genuinely loved nature and often took long walks in the Vienna woods-listen to the disingenuous bird calls at the end of the Pastoral Symphony's slow movement. He also had a rough, hearty, unsubtle, but genuine sense of humor. The inner movements of the Eighth could not have been written by a man who didn't know laughter. Very few performers are able to encompass the multiple-one might even say contradictory-facets of Beethoven's emotional spectrum.

Despite its emotional range, Beethoven can sound angry and domineering. Many conductors overemphasize the music's unrelenting side and make it harsh, overbearing, emotionally monochromatic. As our Editor said in our last Beethoven Overview, ``That may be the very reason many Americans enjoy it...This is hard-hitting music; it snarls at us the way we snarl at each other...Beethoven wasn't angry 30 or 35 [40 or 45, now] years ago. He even had a certain charm. Our social life has changed the way we play Beethoven. We are not genial people, and neither is our Beethoven.'' Although most Beethoven recordings are made in Europe, the world is becoming increasingly Americanized. As much they may resent American cultural hegemony, Europeans are getting more like us in order to compete with us. Many of the recordings that ARG critics prefer are older, more genial ones.
Maybe that's why few American conductors figure in the recommendations below.

Conductors who were active in the early part of the 20th Century were products of a romantic sensibility that also valued depth of expression, warmth, tonal beauty, and plasticity in phrasing. So many of our preferred recordings are by conductors who were quite elderly when they made them. Fortunately
they lived long enough to commit their interpretations to tape in full-range, high-fidelity stereo sound (Furtwängler and Toscanini ai notable exceptions). Youth generally is superficial, which is why there should probably be minimum age limit for committing Beethoven to disc (say 45 or 50?). Young conductors have given us good Beethoven recordings, but few great ones. Great Beethoven interpretations require maturity and practical experience, as well as a sense for the music's deeper levels. A conductor has to live with Beethoven---and the ups and downs of his own life---to grasp, music's inner meaning. Some never get it. This is true with many composers, but more so with Beethoven, perhaps owing to his position as first among equals in the Composer's Pantheon. Young conductors like to show they're hotshots, so they slam through the music demonstrate how they can master it and mi handle it. Some of them are also trying to compete with Toscanini, whose performance became harder and faster as he grew old:
Toscanini also may have been trying to prove he wasn't getting old and losing his ``edge'', another obsession of contemporary culture.

Another modern obsession: ``historical authenticity''. Don't be gulled by the historical reconstructionist (``period instrument'') faction's claims that Beethoven shouldn't be
``interpreted'' (another Toscanini legacy). Over the years, we've reviewed many performances that were mechanical, expressionless, and devoid of feeling (some of them breathlessly trying to observe Beethoven's impossible metronome markings). They're mostly historical curiosities-not uninteresting in themselves but, like going to a Civil War battle reenactment or Colonial Williamsburg, no place to live permanently.

It's been eight years since our last Beethoven overview, but his part of the catalog seems to have been pretty stable. Most of the classic recordings are from the pre-digital era, about 25 years old or more, and most have made it to CD. We've tended to rate even the best of what has been done since ``almost as good as'' rather than ``trample your mother to get at one of these''.
That's hardly the case with Mahler or Bruckner, Shostakovich, Tchaikovsky, Rachmaninoff. Maybe we're just living through a fallow period in Beethoven interpretation.

Even so, there are so many recordings that I had to set some ground rules. I decided to limit the recommendations to releases with high-quality stereo sound; it's painful, but I'm leaving out classic but monophonic or pre-high fidelity stuff. The goal here is to recommend a manageable group of all-around first-rate recordings that can form the cornerstone of a beginning collector's Beethoven library.
That also rules out unusual, eccentric, or otherwise "different" performances. We're looking for strong, solid, compelling interpretations that will yield as much reward on the 50th or 500th listening as the first.

I've also leaned toward recordings that are available (or likely to be again) at a local record store or on-line retailer. This does leave out some good-sounding but obscure-label items.
Releases come and go from the catalog with such bewildering swiftness, and it's impossible o tell you exactly what's ``available'' at any given moment. If a record company is not currently offering a release, plenty of may still be in the pipeline, on dealers' shelves
(sometimes the cut-out bin), or available through Internet retailers. Used CDs are also a
reasonable option, since the format isn't prey
to the degradation that ordinary use inflicts on
LPs; a used CD will be as good as a new one,
unless it's been abused.
Much of the following information is
gleaned from reviews that have appeared in
ARG since the last overview. Where our 1995 recommendations are still valid, I've kept them. Where possible, I've tried to concentrate on recordings where there was some consensus among ARG reviewers. Consider this a collaborative effort, but I'm responsible for any errors and omissions. Some reviewers may think a given recording is better than others of us do. Some of your favorite recordings may have been excluded under the ground rules. If it's not mentioned here, that doesn't mean it's not a worthy performance.

\section{Symphonies: Sets}

No conductor does all nine symphonies equally well, so every set has at least one or two weak links. Conductors who are good at nailing down the aggressive Beethoven are not usually very good at the other Beethoven.
Those who do well with the Fifth or Seventh tend to give us a nervous Sixth. And, unless you'd just as soon not have to put up with all that aggression, the ones who give us the perfect Pastorals seem to let us down in the works that surround it. But sets are convenient: if you know you like Beethoven and you know you want all of the symphonies in your collec-tion, it's so much easier to buy a 5 or 6 CD set and be done with it. If you want the best of each symphony, you'll have to collect them piecemeal. The corollary to this is that, in some cases, outstanding individual performances are trapped in otherwise lackluster sets that cannot be recommended as a whole.
It is always a great help when record companies issue sets on separately available single discs.

Eight years ago, we were still waiting for some important recordings to make it to CD; now, just about everything worth having is—or has been—available. The modern ``compleatist'' fetish, which seems to have been fueled by the advent of the CD, means that most conductors who record the symphonies now do them all, rather than the ones they like and understand best. It's a waste of plastic, so be wary.

Bernstein's Vienna Philharmonic set (DG) still seems pretty much the best all around. No orchestra knows Beethoven better, and they play beautifully for a conductor they always had a special rapport with. Bernstein's interpretations are energetic and vivacious, serious and imposing, lyrical and expansive whatever the music calls for at the moment, he has it.
Bernstein's earlier New York set (Sony) has good moments, but overall it's more superficial and mannered than the Vienna.

In contrast, Abbado's Berlin accounts (DG) are literal, unyielding, metronomic, and often monotonous. Why did DG go to the effort and expense to produce it? Despite fine moments here and there, Abbado does not make us feel
that this is some of the greatest music ever written and here's what it means. The even-numbered symphonies come off better than the odd, but the playing is not particularly beautiful and the recorded sound is only so-so.
Barenboim (Teldec) is also balanced and not too aggressive, but there are not many new ideas either. They are sober, balanced, and beautifully recorded-but, like Abbado, full price. I was not at all impressed with the Ninth, but Philip Haldeman liked the complete set very much. One person's un-fussy performance is another's boring one.

No such reservations apply to Böhm (DG), who, like Bernstein, has the benefit of the Vienna Philharmonic. Böhm is not as impulsive as Bernstein, but his performances have a structural integrity and eloquence that put them in the first rank. The orchestra plays as well for him as they do for Bernstein. The Cluytens (EMI Seraphim) is something of a sleeper. He is often associated with French repertoire, but he had a great affinity for German music. One suspects that he was somewhat overshadowed by the many Austro-German masters still active. John McKelvey pronounced the performances clean, unrushed, and sensitive to rhythmic pulse and found this set competitive with Walter, Klemperer, Szell, and Böhm. Weaknesses: uninspired 3 and 9; strengths: great 5 and 7. Good news: it's available as separate, rock-bottom priced Seraphim discs.

Drahos is also inexpensive, but we're trying to figure out who'd want it. The Nicolaus Esterhazy Sinfonia is a cut-down chamber-sized ensemble, so the orchestral sound is way too scrawny. The no-nonsense accounts are small-scale, stiff, sometimes hasty, inexpressive. They're period-instrument performances for people who don't like period instruments.

Ferencsik's set costs about the same and is worlds apart. The Hungarian State Orchestra may not be the Vienna Philharmonic, but it plays with a mellow old-world tone. A contemporary of Böhm, Ferencsik was a conductor in a similar, though perhaps less strict, mold. His Beethoven is still appropriately forceful and, to be honest, sometimes gruff. 7 and 9 are among best on disc, and 1, 2, 4, and 8 are excellent; the "set problem" strikes in the form of the routine 5 and 6.

Giulini recorded 3, 5, and 6 in Los Angeles (DG)---good middle-of-the road, slightly introspective accounts. His later set with the La Scala orchestra (Sony) is interesting but slow to the point of deconstructionism. Listening can be simultaneously exasperating and rewarding. Also exasperating for some is Harnoncourt (Teldec), who elbows his way through the music in a uniformly angry-aggressive manner. The Chamber Orchestra of Europe keeps up, but nobody seems to be enjoying themselves. Another ``period'' performance without the period instruments.

Karajan made four sets, and the 1962 Berlin set (DG) belongs in the same top-recommendation league as Bernstein and Böhm The performances have power and intensity they are speedy but not forced, and the orchestra plays beautifully. They play even more beautifully in the harder-to-find 1972 set, and Karajan's interpretations have deepened, but the softer sonics may not appeal to all. Karajan's early EMI set with the Philharmonia is scrappy, hasty, and incompletely thought out!
The final Berlin one from the 1980s (DG) shows a great artist in sad decline.

The Klemperer (EMI) is a landmark from the early stereo era. The performances are slow, controlled, and granitic, with a tremen: dous sense of architecture, like building® cathedral or Greek temple very slowly, on stone at a time. The style fits some of the sy phonies better than others (4 and 6 work be 7 and 9 fail to cohere). Konwitschny (Berling reminiscent of Klemperer and, to a lesse extent, Böhm in the tempos of his stel! implacable, muscular, unyielding, but pole fully effective, classically structured interpret tions. His inflexible Germanic rhythmic pi may not appeal to all.

The refurbished CD sound of the Krips (Everest) is very fine, but the orchestral tone nothing special. The interpretations are sober and steady but somewhat superficial. John McKelvey finds them smoother and less angular than Ferencsik, while Editor Vroon feels the tempos are similar to Szell but the more relaxed rhythms create softer edges. This was a reasonable contender when it came out in the early 1960s and there was less competition; approach with caution. Masur (Philips) offers straightforward, detailed, well-played renderings with sensibly chosen tempos, but it's hard to warm up to it. Böhm has the same ineluctable, monolithic quality but more imaginative interpretations and slightly better sound.

Pierre Monteux recorded 1-8 for Decca (9, for some reason, for Westminster) in the late 1950s with the London Symphony and Vienna Philharmonic. Both orchestras play gorgeous-ly, and these are wonderfully polished, firm, musical interpretations. So far, only 1, 3, 6, and 8 seem to have made it to CD. Less genial, the Leibowitz (Chesky) is fast and propulsive-some would call it ``driven''--—in the Toscanini mold, but with fine stereo sound. They are exciting, but not particularly perceptive. Leinsdorf (RCA) is often humorless, too. His set was overshadowed when it was
stands up well now, but it often seems to skim the surface.
The Boston Symphony plays well,
but the performances are often merely facile and rarely moving.
The Muti (EMI), made in Philadelphia in the 1980s, is a fine all-around group of interpretations. Some of them have been reissued on ultra-cheap Seraphim discs. The performances range from an exciting but bland Eroica to a memorable Ninth. Toscanini---Leibowitz speed and drive are there, but Muti also knows when the music has to slow down. The orchestra plays with a much leaner (but still burnished) sound than it did for Ormandy.
Ormandy (Sony) is in a class by himself---Beethoven for sound sensualists. The orchestra is in peak form, and Ormandy makes Beethoven less angry, pushy, or aggressive than just about anybody. Our Editor loves 1, 2, 4, and 9 and says 5 and 8 may be ``the best ever''; tempos are moderate but not draggy.

Reiner (RCA) never recorded an integral set, but many of us would not want to do without the Chicago recordings he left us (1, 5, 6, 7, and 9 in stereo, a monaural 3): precision with power and fire. 5, 6, and 7 are among the best on disc. Solti recorded the symphonies twice (Decca) with the same orchestra. The earlier analog group from the 1970s is better, but on records at least Solti never "got" the 5th or the 9th. The performances are impulsive, pungent, and dramatic-hardly "rock solid" and "gran-
His late 1950s Vienna recordings of 3, 5 and 7 are less punchy and driven. Szell's (Sony) set is a perennial in the catalog, despite a weak Fifth,
and the performances are energetic and exciting. If you like the more relaxed approaches of Walter, Böhm, or Ormandy, Szell's approach may seem too punchy and driven. He could be more relaxed with other orchestras (maybe he felt under pressure to
``play it straight'' for posterity in Cleveland), but 1, 3, 4, and 9 are hard to beat.

Bruno Walter (Sony) recorded his stereo set with a pickup orchestra late in life, after age and infirmity had taken their toll. Neverthe-less, it is one of the warmest and most genial of sets. There are slack moments, and the Ninth is almost a complete loss, but the Sixth is one of the best ever, and our Editor dearly loves Walter's Eroica. You can cover the poles of Beethoven interpretation by having both Szell and Walter.

Wand was seems cold and uninviting, though there are no significant disqualifying characteristics, but should Beethoven be this unimaginative and humorless? Speaking of unimaginative, Zinman's (Arte Nova) came out on separate discs, but its chier selling point is the new Barenreitner Urtext scores its uses (since recorded by others, including Abbado) and the conductor's attempt to adhere strictly to Beethoven's metronome markings. The speed can be breathtaking, but the performances are often stiff, charmless, and mechanical-and they prove that those metronome markings, however Beethoven arrived at them, simply don't work.
The new Simon Rattle set (with the Vienna Philharmonic) arrived just past our deadline; we will have a review in the next issue.

There have been period-instrument sets by Goodman (Nimbus), Hogwood (Decca), and Norrington (EMI), but after a flurry of interest when they first came out, they seem not to have worn well. Hogwood has the best sound and is more straightforward and less eccentric than Norrington. It's probably the best bet in this category, unless eccentricity and capriciousness appeal to you.

\newpage
Recommended and currently available:

\startrectable
Bernstein &DG 423481\\
Karajan &DG 453701\\
Walter &Sony 48099\\
Böhm &DG 439681 (1,2,4,5)\\
&437368 (3+9)\\
&437928 (6,7,8)
\endrectable

\section{Symphony 1}

Beethoven's first symphony appeared in 1801—--Symphony No. 1 in the first year of a new century. As Leonard Bernstein pointed out, it wears its Haydnish clothes until the third movement, which is an amusingly speeded-up minuet. It's a light, classical piece, but clearly the work of a new voice. It requires a lighter touch than the later works; the aggressive approach does not serve it well. Conductors who do well in it (as well as 2, 4, and 8) are not necessarily the best in the heavier symphonies.

Bernstein pretty much swept the field with his Vienna recording. It's full of high spirits and fizzy energy, with slightly weightier touches at the right times-the young Beethoven tweaking the nose of the Establishment. (One of our reviewers said, ``It hits you like a ton of happy bricks''.) However, Reiner, Karajan 1962, Szell, and the analog Solti (recommend-ed last time around) are almost as rewarding;
Monteux and Jochum are more poised. We found Dohnanyi and Leibowitz earthbound.
Ormandy gives it more body and substance than usual. Sawallisch has the Concertgebouw Orchestra, but his lumpen approach doesn't say much with it or the music. This work is also the least strain for the period groups, so Hogwood, Norrington, and Brüggen all turn in credible but rather mechanical performances.

\startrectable
Bernstein &DG \\
Szell& Sony 89838 (+6)\\
Monteux& Decca 440627 (+3,6,8)
\endrectable

\section{Symphony 2}

The First and Second are often lumped together as early Haydnish works, but \No2 really is more expansive and more ambitious; it bridges the gap between the classical First and the truly Beethovenian Third. Bernstein (Vienna) approaches the work as a foretaste of the Eroica and whips up lots of confident youthful energy tempered with lyricism.
Beecham is special: charming, witty, warm---Beethoven as Schubert. It's just the sort of thing he excelled at-finding wonderful things in a work that many other conductors over-looked. Walter is genial and charming, too; and Böhm is exciting without rigid tempos or effects-mongering. Ormandy had plenty of energy and "snap"; the orchestra is amazing.
Karajan 1962 is fluent and almost relaxed sometimes-an underrated performance.
Wand (coupled with 1) has almost as much energy and enthusiasm as Toscanini, and the playing is alert, but we wonder if Wand isn't a little too hard for this music. Leibowitz (Chesky) has the energy and great clarity plus even better sound.

\startrectable
Bernstein & DG\\
Beecham & EMI\\
Walter & Sony 64460 (+1)\\
Böhm & DG 439681 (+1,4,5)\\
Leibowitz & Chesky 17 (+5)
\endrectable

\section{Symphony 3}

The Third is the first fully mature Beethoven symphony. Much is made of its length being nearly twice that of the average Haydn symphony—an exaggeration if you consider that Haydn's London symphonies run 25-30 minutes and the average Eroica comes in around 45. If you take every single possible repeat, Mozart's last two symphonies can be bloated up to about the same length, too. The difference in the Eroica is the length of the musical ideas and the complexity of their working-out.

Szell's Third is one of the best-bracing and swift but not rushed. Walter is expansive and Apollonian—a high point of his set. Bernstein takes the Funeral March dangerously slow, but it works, and the rest of the performance has a grandeur and warmth that makes Klemperer and Karajan sound cold in compar-ison. You'll get the Vienna Philharmonic with Böhm, too, and it's one of the best examples of a "mainstream" approach. Böhm also recorded it in Berlin: less stark, more blended sound (also DG). On paper, Sawallisch should have been as good, but it turns out to be a plodding, high-cholesterol trudge. Giulini's earlier DG traversal is long-breathed and expansive, but the Los Angeles orchestra is a bit too generic sounding. His later Sony lasts nearly a full hour---too long---and no matter how beautifully the La Scala orchestra plays they can't sustain the tempos.

Peter Tiboris (Elysium) uses Mahler's retouchings of the orchestration to make it more suitable for large halls; interesting, witha cataclysmic Funeral March, but not vital.
Stokowski (RCA) is rather "straight" but inconsistent: a vigorous I, followed by a II lacking drive, but a propulsive finale. The recently reissued Kempe (Testament) has a massive, powerful, broadlypaced I. Like Böhm and Szell, he keeps a steady hand on the tempos, and the rest of the performance has incisive playing and firm rhythm. Boult takes a similar approach (Vanguard, with 5, 6, 7, overtures) with clean, organized lines and polished phrasing; the essential drama is never lacking, and tempos feel right, like Monteux and Reiner. The Dohnanyi on Telarc is gorgeously played and recorded, moves right along but can also be big and brash where it matters.

\startrectable
Bernstein & DG\\
Böhm & DG 437368 (+9)\\
Szell & Sony 89832 or 46328\\
Walter & ..Sony 64461 (+8)\\
Dohnanyi &Telarc 80090\\
Kempe & Testament 1270
\endrectable

\section{Symphony 4}

The Fourth seems light-a throwback to
Hayan after the weighty Eroica. 
course, it's not a retrograde move, but the guffawing high spirits of its fast movement require a lighter touch, like the First, and the sighing lyricism of its slow movement benefit from a light, warm approach. Our reviewers have never agreed on recommendations.

Bernstein's New York recording is vibrant witty, and gorgeous, but of course the Vienna is more relaxed and consistent-and better recorded. The 1972 Karajan is the next runner-up, but the 1962 is nearly as good. Szell played Haydn brilliantly, so it's no wonder that he excels in this symphony. If you want a weightier approach that emphasizes kinship with 3 and 5, Carlos Kleiber (Orfeo) and Masur may be your choices. Haitink (Philips), Krips and Ormandy hew closer to that approach, while Steinberg and Schmidt-Isserstedt are crisper and faster. Marriner's lovely account (Philips) was the high point of his otherwise pedestrian
Beethoven set, and many of us love the Walter.
The Editor adds Leinsdorf to the list.

\startrectable
Bernstein& DG 463468 (+6,9)\\
Karajan & DG\\
Kleiber & Orfeo 100841\\
Walter & Sony 64462 (+6)\\
Krips & Everest 9102 (+7)
\endrectable

\section{Symphony 5}

The Fifth is so familiar a conductor has to
be able to make us listen to it with fresh ears. When one does, we realize it breaks different ground from the Eroica. Drive and focus are what a conductor needs here; the piece has almost no melodies. It consists of short themes that are rigorously, even obsessively worked The uncompromising, exciting Kleiber
(DG) seems to be the top of the heap, but Philip Haldeman speaks for a few of us when he says Reiner's Fifth is the best ever recorded, with nearly super-human playing, propulsion of tempo, and energy without short-changing expressive detail. Szell's hectic Cleveland recording is not a high point of the set, but his later one with the Amsterdam Concertgebouw (Philips or MHS) is treasurable; it has focus
and energy but also grandeur. Walter is noble and songful, with delicious details but weak in drive and energy. Ormandy is moderate and direct: the music pure and simple. We all feel Solti misses the mark, but his best recording of this was his last one, in Vienna.

I have to break my rule about banishing monaural releases to recommend the Vienna Furtwängler. It is simply the most powerful, noblest, most majestic Fifth ever, and, owing to the unusually broad dynamic range of EMI's recording, it doesn't sound faded at all. Rattle
(EMI) is a high-voltage affair with lots of stress
and anger, heavy on clarity and rhythmic
strength. The Vienna Philharmonic is pushed
and the
to sound much too aggressive, and even the
slow movement lacks geniality. (Rattle has
redone this with the same orchestra---review
next issue.) Thielemann (DG) is weighted
vibrant, down by heavy-handed pomp in slow sections,
though his other tempos are better judged.
Our reviewer liked it, but with lots of quibbles.
Zander (Telarc) has lickety-split tempos-an-
other conductor trying to observe Beethoven's
metronome markings. It's good for listeners
with short attention spans.

\startrectable
Kleiber & DG 447400 (+7)\\
Reiner & RCA 68976\\
Szell (Amsterdam)&  Philips 464682 or MHS
\endrectable

\section{Symphony 6}

After the rigors of the Fifth, another surprise: a Pastoral, nature symphony. It's not really program music, because it doesn't tell any sort of narrative story, but it is a remarkable accomplishment of painting nature in sound. Unlike the driven, melody-less Fifth, it is full of warm, flowing tunes.
Walter is just about the best all around---warm, glowing, flowing---so right that you could almost get by with his recording alone.
Monteux comes strikingly close, as do Böhm and Klemperer. Reiner is also warm and beautiful quite a surprise, given his affinity for the
Fifth and Seventh. Gerald Fox found Bernstein's New York recording ``wonderful'', with pastoral woodwinds, sweet strings, and no interpretive distortions; but the Vienna version seems, again, even finer---especially the Brucknerian grandeur of the finale. This is the performance that taught me to really love this work.

\startrectable
Walter & Sony 64462 (+4)\\
Reiner & RCA\\
Monteux  & Decca\\
Bernstein & DG 431025\\
Böhm &DG 437928 (+7,8) or 447433
\endrectable

\section{Symphony 7}

Wagner did this symphony no favors by calling it the ``Apotheosis of the Dance'' (yes, ballet companies have staged it). It is actually a return to the rigors of the Fifth, tempered by good humor, despite another funeral march (though not so labeled) in the second movement.

Again, Carlos Kleiber is nearly definitive; at least it's a performance that many different kinds of listeners seem to enjoy. It is fast. Karajan's Vienna recording (Decca) is probably his best (it breathes), though the 1970s DG comes close (1962 DG is fine, too). Reiner's 7th is almost as exciting as his 5th; the finale seems as near to orchestral perfection as mortals can get (the interplay between the first and second violins is thrilling at that speed). Steinberg (EMI) has a following; our reviewer thought it a pleasure, with opening chords that have a real ``crunch'' leading to a first movement that is
``so inevitable you can't imagine the piece going any other way''. The Editor still responds strongly to Leinsdorf's Viennese reading on RCA. Beecham seems less driven and happier than most, but he is among the fastest and gets through the whole thing in 26 minutes. Böhm is still the antidote for those who don't like breathless Sevenths, but Walter seems a better compromise between openness and expeditiousness, and it's one of the high points of his set. Colin Davis is the choice if you want to get through the work as quickly as possible. Muti offered some Reiner-like precision and inevitability in an underrated performance.

\startrectable
Kleiber & DG 447400 (+5)\\
Reiner & RCA\\
Karajan & DG\\
Walter & Sony 64463 (+5)\\
Beecham & EMI
\endrectable

\section{Symphony 8}

The pattern of rigor and relaxation continues, with the Eighth a relaxing, rustic interlude between the striving Seventh and the monumental Ninth. It's anything but a throwback to Haydn, though, or even the Beethoven of the First. There was little consensus in our last overview, and I found no more guidance in later reviews.
For an all-around balance of polish and high spirits, Monteux (Decca) may be our man---but to the Editor he is too energetic and doesn't ``float''---that is, he is not genial enough for this genial music.
Some older critics like Scherchen, but he is so fast! (exciting, too). Most of us hate the Dohnanyi. Ormandy is less pushy and beautifully played, but I've got to stick with my preferences for Munch (RCA) and the large-scale but un-cumbersome analog Solti (Decca).
Karajan, Bernstein, Haitink, Szell, and Walter are all fine, too; and again Steinberg finds favor in some quarters, but only the most recent EMI remastering.

\startrectable
Monteux & Decca\\
Solti  & Decca 417765\\
Ormandy & Sony 63266 (+5,7)
\endrectable

\section{Symphony 9}

After completing eight symphonies, Beethoven reached the limits of expression for instrumental music and resorted to the human voice in the finale of his symphony.
Schiller's text is, frankly, padded with truisms, but Beethoven affixed it to one of the world's greatest musical utterances. I've never heard of a conductor of any experience who has not performed it, but I suppose there have been some. Almost any conductor will jump at a chance to record it.

There are many the recordings, but none I've heard completely lives up to the music's potential, with one exception ("no-mono" rule about to be broken again): Furtwängler's 1954 Lucerne Festival account on Tahra. As in his Vienna Fifth, Furtwängler seems to get closer to the work's inner meaning than any other, and Tahra's clear, full-range sound makes it preferable even to his 1951 Bayreuth performance (EMI) that is on everybody's top Ninths list. Paul Althouse observed that in the Bayreuth account, Furtwängler ``leads you though the Dark Valley and makes you think''. Entirely true, but III and IV have too many slack moments, which are entirely absent from the Lucerne. Now let's return to our regular stereo listings.

The 1962 Karajan is first rate, with a great solo quartet. I'm probably quibbling, but again, I prefer the later 1970s release for its greater introspection and even more polished orchestral execution. Böhm recorded the Ninth in Vienna and Berlin; the problem is he's just a bit too slow. The forward thrusting but flexible Abbado (Sony) has good soloists and a textual clarity that equals the best of the period-instrument performances, but is that necessarily a virtue? Herreweghe's period affair (Harmonia Mundi) is way too fast, beyond the pale,
just driven too hard---and inexpressive.
One of our writers says "People have been shot for less than what Norrington did to this piece''. You have to be very, very tired of Beethoven's romanticism to find Norrington bracing. If you want period instruments and practices, Hogwood (Decca) is the most palatable.

Leinsdorf (RCA) is vigorous and straight-forward, with no odd mannerisms, but not much emotional depth. The orchestral tone isn't very lush, either; our reviewer called it a
``clinical dissection'' of the score. Muti (EMI) is noble, powerful, expressive, and satisfying with full-bodied orchestral playing and sens: ble tempos. Ormandy (Sony), on the other hand, offers up something much different with the same orchestra: rich, smooth, thick, almost fuzzy orchestral tone. It's an antidote to angry  Previn (RCA) a straightforward, cogent, mainstream accoun with great sound. Paul Althouse said the cho rus in the Szell ``sounds like a thousand wild animals who have gone too long without a meal. They grab the piece and sing the bejesus out of it.'' The singing is exciting, and the performance is a classic example of a swift forceful Ninth---one of the high points o Szell's set. Reiner (RCA) is very similar, though the Chicago Symphony is a little richer in tone. If you want something a little juicier and warmer but still in the Toscanini-precision mold, Munch (RCA) is your man. Another conductor who avoids the bloating this work often suffers from is Dohnanyi. He is always moderate; he moves right along. That helps in IV, but some of us feel the Adagio is too fast. Gorgeous sound and incredible playing.

Giulini (EMI) never really seems to get
started; Walter (Sony) and Klemperer (EMI)
seem endless-bloated, cumbersome traversals. The physical demands of the work were
beyond what they could handle at such late
stages in their careers. Stokowski (Decca) is
also caught too late in his career, in an uncharacteristically detached performance. Jochum
(EMI) offers a pleasantly polished account
with emphasis in all the right places. So does Ferencsik, though in common with Szell, hes
too reserved in the slow movement. Macal
(Koss) is tidy, with beautiful playing from the
Milwaukee Symphony, but he insists on mal find
ing the music stiff and rigid.

\startrectable
Furtwängler & Tahra 1003\\
Karajan & DG 447401\\
Dohnanyi &Telarc 80120\\
Muti & Seraphim 73284\\
Reiner & RCA \\
Szell & Sony 60987
\endrectable

\end{document}
