\documentclass[12pt]{article}%
\usepackage{enumitem}
\setcounter{secnumdepth}{0}

%\usepackage{amsmath,amsthm,amscd,amssymb}
\usepackage[full]{textcomp}
\usepackage{XCharter}% lining figures in math, osf in text
\usepackage[scaled=1.04,varqu,varl]{inconsolata}% inconsolata typewriter
\usepackage[xcharter,vvarbb,scaled=1.03,smallerops,bigdelims]{newtxmath}
\usepackage[cal=cm,scr=esstix]{mathalpha}
\usepackage[papersize={7in, 10.0in}, left=.6in, right=.6in, top=.6in, bottom=.9in]{geometry}
\linespread{1.05}
\sloppy
\raggedbottom
\pagestyle{plain}

% these include amsmath and that can cause trouble in older docs.
\input{/Users/psu/arxiv-psu/helpers/cmrsum}
\makeatletter

\DeclareFontFamily{OMX}{MnSymbolE}{}
\DeclareSymbolFont{largesymbolsX}{OMX}{MnSymbolE}{m}{n}
\DeclareFontShape{OMX}{MnSymbolE}{m}{n}{
    <-6>  MnSymbolE5
   <6-7>  MnSymbolE6
   <7-8>  MnSymbolE7
   <8-9>  MnSymbolE8
   <9-10> MnSymbolE9
  <10-12> MnSymbolE10
  <12->   MnSymbolE12}{}

\DeclareMathSymbol{\downbrace}    {\mathord}{largesymbolsX}{'251}
\DeclareMathSymbol{\downbraceg}   {\mathord}{largesymbolsX}{'252}
\DeclareMathSymbol{\downbracegg}  {\mathord}{largesymbolsX}{'253}
\DeclareMathSymbol{\downbraceggg} {\mathord}{largesymbolsX}{'254}
\DeclareMathSymbol{\downbracegggg}{\mathord}{largesymbolsX}{'255}
\DeclareMathSymbol{\upbrace}      {\mathord}{largesymbolsX}{'256}
\DeclareMathSymbol{\upbraceg}     {\mathord}{largesymbolsX}{'257}
\DeclareMathSymbol{\upbracegg}    {\mathord}{largesymbolsX}{'260}
\DeclareMathSymbol{\upbraceggg}   {\mathord}{largesymbolsX}{'261}
\DeclareMathSymbol{\upbracegggg}  {\mathord}{largesymbolsX}{'262}
\DeclareMathSymbol{\braceld}      {\mathord}{largesymbolsX}{'263}
\DeclareMathSymbol{\bracelu}      {\mathord}{largesymbolsX}{'264}
\DeclareMathSymbol{\bracerd}      {\mathord}{largesymbolsX}{'265}
\DeclareMathSymbol{\braceru}      {\mathord}{largesymbolsX}{'266}
\DeclareMathSymbol{\bracemd}      {\mathord}{largesymbolsX}{'267}
\DeclareMathSymbol{\bracemu}      {\mathord}{largesymbolsX}{'270}
\DeclareMathSymbol{\bracemid}     {\mathord}{largesymbolsX}{'271}

\def\horiz@expandable#1#2#3#4#5#6#7#8{%
  \@mathmeasure\z@#7{#8}%
  \@tempdima=\wd\z@
  \@mathmeasure\z@#7{#1}%
  \ifdim\noexpand\wd\z@>\@tempdima
    $\m@th#7#1$%
  \else
    \@mathmeasure\z@#7{#2}%
    \ifdim\noexpand\wd\z@>\@tempdima
      $\m@th#7#2$%
    \else
      \@mathmeasure\z@#7{#3}%
      \ifdim\noexpand\wd\z@>\@tempdima
        $\m@th#7#3$%
      \else
        \@mathmeasure\z@#7{#4}%
        \ifdim\noexpand\wd\z@>\@tempdima
          $\m@th#7#4$%
        \else
          \@mathmeasure\z@#7{#5}%
          \ifdim\noexpand\wd\z@>\@tempdima
            $\m@th#7#5$%
          \else
           #6#7%
          \fi
        \fi
      \fi
    \fi
  \fi}

\def\overbrace@expandable#1#2#3{\vbox{\m@th\ialign{##\crcr
  #1#2{#3}\crcr\noalign{\kern2\p@\nointerlineskip}%
  $\m@th\hfil#2#3\hfil$\crcr}}}
\def\underbrace@expandable#1#2#3{\vtop{\m@th\ialign{##\crcr
  $\m@th\hfil#2#3\hfil$\crcr
  \noalign{\kern2\p@\nointerlineskip}%
  #1#2{#3}\crcr}}}

\def\overbrace@#1#2#3{\vbox{\m@th\ialign{##\crcr
  #1#2\crcr\noalign{\kern2\p@\nointerlineskip}%
  $\m@th\hfil#2#3\hfil$\crcr}}}
\def\underbrace@#1#2#3{\vtop{\m@th\ialign{##\crcr
  $\m@th\hfil#2#3\hfil$\crcr
  \noalign{\kern2\p@\nointerlineskip}%
  #1#2\crcr}}}

\def\bracefill@#1#2#3#4#5{$\m@th#5#1\leaders\hbox{$#4$}\hfill#2\leaders\hbox{$#4$}\hfill#3$}

\def\downbracefill@{\bracefill@\braceld\bracemd\bracerd\bracemid}
\def\upbracefill@{\bracefill@\bracelu\bracemu\braceru\bracemid}

\DeclareRobustCommand{\downbracefill}{\downbracefill@\textstyle}
\DeclareRobustCommand{\upbracefill}{\upbracefill@\textstyle}

\def\upbrace@expandable{%
  \horiz@expandable
    \upbrace
    \upbraceg
    \upbracegg
    \upbraceggg
    \upbracegggg
    \upbracefill@}
\def\downbrace@expandable{%
  \horiz@expandable
    \downbrace
    \downbraceg
    \downbracegg
    \downbraceggg
    \downbracegggg
    \downbracefill@}

\DeclareRobustCommand{\overbrace}[1]{\mathop{\mathpalette{\overbrace@expandable\downbrace@expandable}{#1}}\limits}
\DeclareRobustCommand{\underbrace}[1]{\mathop{\mathpalette{\underbrace@expandable\upbrace@expandable}{#1}}\limits}

\makeatother


\usepackage[small]{titlesec}
\titlelabel{\thetitle.\quad}

\usepackage{cite}
\usepackage{microtype}

% hyperref last because otherwise some things go wrong.
\usepackage{hyperref}
\hypersetup{colorlinks=true
,breaklinks=true
,urlcolor=blue
,anchorcolor=blue
,citecolor=blue
,filecolor=blue
,linkcolor=blue
,menucolor=blue
,linktocpage=true}
\hypersetup{
bookmarksopen=true,
bookmarksnumbered=true,
bookmarksopenlevel=10,
pdfencoding=auto, psdextra
}

% avoid weird spacing after the period here.
\def\No#1{No.\@ #1}

% make sure there is enough TOC for reasonable pdf bookmarks.
\setcounter{tocdepth}{3}

%\usepackage[dotinlabels]{titletoc}
%\titlelabel{{\thetitle}.\quad}
%\titleformat{\section}[block]
  {\fillast\medskip}
  {{\thesection. }}
  {1ex minus .1ex}
  {\scshape}
 
\titleformat*{\subsection}{\itshape}
\titleformat*{\subsubsection}{\itshape}

\setcounter{tocdepth}{2}

\titlecontents{section}
              [2.3em] 
              {\bigskip}
              {{\contentslabel{2.3em}}\large\scshape}
              {\hspace*{-2.3em}}
              {\titlerule*[1pc]{}\contentspage}
              
\titlecontents{subsection}
              [4.7em] 
              {}
              {{\contentslabel{2.3em}}}
              {\hspace*{-2.3em}}
              {\titlerule*[.5pc]{}\contentspage}

% hopefully not used.           
\titlecontents{subsubsection}
              [7.9em]
              {}
              {{\contentslabel{3.3em}}}
              {\hspace*{-3.3em}}
              {\titlerule*[.5pc]{}\contentspage}
%\makeatletter
\renewcommand\tableofcontents{%
    \section*{\contentsname
        \@mkboth{%
           \MakeLowercase\contentsname}{\MakeLowercase\contentsname}}%
    \@starttoc{toc}%
    }
\def\@oddhead{{\scshape\rightmark}\hfil{\small\scshape\thepage}}%
\def\sectionmark#1{%
      \markright{\MakeLowercase{%
        \ifnum \c@secnumdepth >\m@ne
          \thesection\quad
        \fi
        #1}}}
        
\makeatother



%\makeatletter

 \def\small{%
  \@setfontsize\small\@xipt{13pt}%
  \abovedisplayskip 8\p@ \@plus3\p@ \@minus6\p@
  \belowdisplayskip \abovedisplayskip
  \abovedisplayshortskip \z@ \@plus3\p@
  \belowdisplayshortskip 6.5\p@ \@plus3.5\p@ \@minus3\p@
  \def\@listi{%
    \leftmargin\leftmargini
    \topsep 9\p@ \@plus3\p@ \@minus5\p@
    \parsep 4.5\p@ \@plus2\p@ \@minus\p@
    \itemsep \parsep
  }%
}%
 \def\footnotesize{%
  \@setfontsize\footnotesize\@xpt{12pt}%
  \abovedisplayskip 10\p@ \@plus2\p@ \@minus5\p@
  \belowdisplayskip \abovedisplayskip
  \abovedisplayshortskip \z@ \@plus3\p@
  \belowdisplayshortskip 6\p@ \@plus3\p@ \@minus3\p@
  \def\@listi{%
    \leftmargin\leftmargini
    \topsep 6\p@ \@plus2\p@ \@minus2\p@
    \parsep 3\p@ \@plus2\p@ \@minus\p@
    \itemsep \parsep
  }%
}%
\def\open@column@one#1{%
 \ltxgrid@info@sw{\class@info{\string\open@column@one\string#1}}{}%
 \unvbox\pagesofar
  \gdef\thepagegrid{one}%
 \global\pagegrid@col#1%
 \global\pagegrid@cur\@ne
 \global\count\footins\@m
 \set@column@hsize\pagegrid@col
 \set@colht
}%

\def\frontmatter@abstractheading{%
\bigskip
 \begingroup
  \centering\large
  \abstractname
  \par\bigskip
 \endgroup
}%

\makeatother

%\DeclareSymbolFont{CMlargesymbols}{OMX}{cmex}{m}{n}
%\DeclareMathSymbol{\sum}{\mathop}{CMlargesymbols}{"50}
%\pdfbookmark[1]{Introduction}{Introduction}

\begin{document}

\title{Beethoven Overview 2003}
\date{}
\maketitle

\tableofcontents

\newpage

\addcontentsline{toc}{section}{\protect\textbf{Introduction}}
\noindent
Beethoven is the cornerstone of the symphonic repertoire. His music is the summation of everything that came before him and the progenitor of nearly everything that has come since. Who better combined the intellectual rigor of Bach, the classical balance of Mozart and Haydn, and the expressive sensibilities of the romantics? None of his successors were able to emulate him, but most of them are indebted to him in one way or another. The vocal symphonies of Mahler, Shostakovich, Vaughan Williams, and even Havergal Brian would have been unthinkable without the precedent of Beethoven's 9th.

Whether Beethoven was a classicist or a romantic has always been a subject of debate.
His profound emphasis on organic form suggests the perfection of high classicism, and so it is. But the inspiration to reach those heights of structure and balance was a highly romantic emotional sensibility. It's a synthesis of both;
Beethoven really only sounds like himself.
However, this duality of heart and mind may explain why both classical and romantic interpretive approaches work.

The music is full of great intellectual struggles invested with almost superhuman emotional intensity. Beethoven wrestled and fought and finally subdued each work into an immutable whole so complete that when we hear it we can't imagine how it could be different. No wonder he's so popular in our willful age: the triumph of will over nature. Yet he genuinely loved nature and often took long walks in the Vienna woods-listen to the disingenuous bird calls at the end of the Pastoral Symphony's slow movement. He also had a rough, hearty, unsubtle, but genuine sense of humor. The inner movements of the Eighth could not have been written by a man who didn't know laughter. Very few performers are able to encompass the multiple-one might even say contradictory-facets of Beethoven's emotional spectrum.

Despite its emotional range, Beethoven can sound angry and domineering. Many conductors overemphasize the music's unrelenting side and make it harsh, overbearing, emotionally monochromatic. As our Editor said in our last Beethoven Overview, ``That may be the very reason many Americans enjoy it...This is hard-hitting music; it snarls at us the way we snarl at each other...Beethoven wasn't angry 30 or 35 [40 or 45, now] years ago. He even had a certain charm. Our social life has changed the way we play Beethoven. We are not genial people, and neither is our Beethoven.'' Although most Beethoven recordings are made in Europe, the world is becoming increasingly Americanized. As much they may resent American cultural hegemony, Europeans are getting more like us in order to compete with us. Many of the recordings that ARG critics prefer are older, more genial ones.
Maybe that's why few American conductors figure in the recommendations below.

Conductors who were active in the early part of the 20th Century were products of a romantic sensibility that also valued depth of expression, warmth, tonal beauty, and plasticity in phrasing. So many of our preferred recordings are by conductors who were quit elderly when they made them. Fortunately
they lived long enough to commit their interpretations to tape in full-range, high-fidelity stereo sound (Furtwängler and Toscanini ai notable exceptions). Youth generally is superficial, which is why there should probably be minimum age limit for committing Beethoven to disc (say 45 or 50?). Young conductors have given us good Beethoven recordings, but few great ones. Great Beethoven interpretations require maturity and practical experience, as well as a sense for the music's deeper levels. A conductor has to live with Beethoven---and the ups and downs of his own life---to grasp, music's inner meaning. Some never get it. This is true with many composers, but more so with Beethoven, perhaps owing to his position as first among equals in the Composer's Pantheon. Young conductors like to show they're hotshots, so they slam through the music demonstrate how they can master it and mi handle it. Some of them are also trying to compete with Toscanini, whose performance became harder and faster as he grew old:
Toscanini also may have been trying to prove he wasn't getting old and losing his ``edge'', another obsession of contemporary culture.

Another modern obsession: ``historical authenticity''. Don't be gulled by the historical reconstructionist (``period instrument'') faction's claims that Beethoven shouldn't be
``interpreted'' (another Toscanini legacy). Over the years, we've reviewed many performances that were mechanical, expressionless, and devoid of feeling (some of them breathlessly trying to observe Beethoven's impossible metronome markings). They're mostly historical curiosities-not uninteresting in themselves but, like going to a Civil War battle reenactment or Colonial Williamsburg, no place to live permanently.

It's been eight years since our last Beethoven overview, but his part of the catalog seems to have been pretty stable. Most of the classic recordings are from the pre-digital era, about 25 years old or more, and most have made it to CD. We've tended to rate even the best of what has been done since ``almost as good as'' rather than ``trample your mother to get at one of these''.
That's hardly the case with Mahler or Bruckner, Shostakovich, Tchaikovsky, Rachmaninoff. Maybe we're just living through a fallow period in Beethoven interpretation.

Even so, there are so many recordings that I had to set some ground rules. I decided to limit the recommendations to releases with high-quality stereo sound; it's painful, but I'm leaving out classic but monophonic or pre-high fidelity stuff. The goal here is to recommend a manageable group of all-around first-rate recordings that can form the cornerstone of a beginning collector's Beethoven library.
That also rules out unusual, eccentric, or otherwise "different" performances. We're looking for strong, solid, compelling interpretations that will yield as much reward on the 50th or 500th listening as the first.

I've also leaned toward recordings that are available (or likely to be again) at a local record store or on-line retailer. This does leave out some good-sounding but obscure-label items.
Releases come and go from the catalog with such bewildering swiftness, and it's impossible o tell you exactly what's ``available'' at any given moment. If a record company is not currently offering a release, plenty of may still be in the pipeline, on dealers' shelves
(sometimes the cut-out bin), or available through Internet retailers. Used CDs are also a
reasonable option, since the format isn't prey
to the degradation that ordinary use inflicts on
LPs; a used CD will be as good as a new one,
unless it's been abused.
Much of the following information is
gleaned from reviews that have appeared in
ARG since the last overview. Where our 1995 recommendations are still valid, I've kept them. Where possible, I've tried to concentrate on recordings where there was some consensus among ARG reviewers. Consider this a collaborative effort, but I'm responsible for any errors and omissions. Some reviewers may think a given recording is better than others of us do. Some of your favorite recordings may have been excluded under the ground rules. If it's not mentioned here, that doesn't mean it's not a worthy performance.

\section{Symphonies: Sets}

No conductor does all nine symphonies equally well, so every set has at least one or two weak links. Conductors who are good at nailing down the aggressive Beethoven are not usually very good at the other Beethoven.
Those who do well with the Fifth or Seventh tend to give us a nervous Sixth. And, unless you'd just as soon not have to put up with all that aggression, the ones who give us the perfect Pastorals seem to let us down in the works that surround it. But sets are convenient: if you know you like Beethoven and you know you want all of the symphonies in your collec-tion, it's so much easier to buy a 5 or 6 CD set and be done with it. If you want the best of each symphony, you'll have to collect them piecemeal. The corollary to this is that, in some cases, outstanding individual performances are trapped in otherwise lackluster sets that cannot be recommended as a whole.
It is always a great help when record companies issue sets on separately available single discs.

Eight years ago, we were still waiting for some important recordings to make it to CD; now, just about everything worth having is—or has been—available. The modern ``compleatist'' fetish, which seems to have been fueled by the advent of the CD, means that most conductors who record the symphonies now do them all, rather than the ones they like and understand best. It's a waste of plastic, so be wary.

Bernstein's Vienna Philharmonic set (DG) still seems pretty much the best all around. No orchestra knows Beethoven better, and they play beautifully for a conductor they always had a special rapport with. Bernstein's interpretations are energetic and vivacious, serious and imposing, lyrical and expansive whatever the music calls for at the moment, he has it.
Bernstein's earlier New York set (Sony) has good moments, but overall it's more superficial and mannered than the Vienna.

In contrast, Abbado's Berlin accounts (DG) are literal, unyielding, metronomic, and often monotonous. Why did DG go to the effort and expense to produce it? Despite fine moments here and there, Abbado does not make us feel
that this is some of the greatest music ever written and here's what it means. The even-numbered symphonies come off better than the odd, but the playing is not particularly beautiful and the recorded sound is only so-so.
Barenboim (Teldec) is also balanced and not too aggressive, but there are not many new ideas either. They are sober, balanced, and beautifully recorded-but, like Abbado, full price. I was not at all impressed with the Ninth, but Philip Haldeman liked the complete set very much. One person's un-fussy performance is another's boring one.

No such reservations apply to Böhm (DG), who, like Bernstein, has the benefit of the Vienna Philharmonic. Böhm is not as impulsive as Bernstein, but his performances have a structural integrity and eloquence that put them in the first rank. The orchestra plays as well for him as they do for Bernstein. The Cluytens (EMI Seraphim) is something of a sleeper. He is often associated with French repertoire, but he had a great affinity for German music. One suspects that he was somewhat overshadowed by the many Austro-German masters still active. John McKelvey pronounced the performances clean, unrushed, and sensitive to rhythmic pulse and found this set competitive with Walter, Klemperer, Szell, and Böhm. Weaknesses: uninspired 3 and 9; strengths: great 5 and 7. Good news: it's available as separate, rock-bottom priced Seraphim discs.

Drahos is also inexpensive, but we're trying to figure out who'd want it. The Nicolaus Esterhazy Sinfonia is a cut-down chamber-sized ensemble, so the orchestral sound is way too scrawny. The no-nonsense accounts are small-scale, stiff, sometimes hasty, inexpressive. They're period-instrument performances for people who don't like period instruments.

Ferencsik's set costs about the same and is worlds apart. The Hungarian State Orchestra may not be the Vienna Philharmonic, but it plays with a mellow old-world tone. A contemporary of Böhm, Ferencsik was a conductor in a similar, though perhaps less strict, mold. His Beethoven is still appropriately forceful and, to be honest, sometimes gruff. 7 and 9 are among best on disc, and 1, 2, 4, and 8 are excellent; the "set problem" strikes in the form of the routine 5 and 6.

Giulini recorded 3, 5, and 6 in Los Angeles (DG)---good middle-of-the road, slightly introspective accounts. His later set with the La Scala orchestra (Sony) is interesting but slow to the point of deconstructionism. Listening can be simultaneously exasperating and rewarding. Also exasperating for some is Harnoncourt (Teldec), who elbows his way through the music in a uniformly angry-aggressive manner. The Chamber Orchestra of Europe keeps up, but nobody seems to be enjoying themselves. Another ``period'' performance without the period instruments.

Karajan made four sets, and the 1962 Berlin set (DG) belongs in the same top-recommendation league as Bernstein and Böhm The performances have power and intensity they are speedy but not forced, and the orchestra plays beautifully. They play even more beautifully in the harder-to-find 1972 set, and Karajan's interpretations have deepened, but the softer sonics may not appeal to all. Karajan's early EMI set with the Philharmonia is scrappy, hasty, and incompletely thought out!
The final Berlin one from the 1980s (DG) shows a great artist in sad decline.

The Klemperer (EMI) is a landmark from the early stereo era. The performances are slow, controlled, and granitic, with a tremen: dous sense of architecture, like building® cathedral or Greek temple very slowly, on stone at a time. The style fits some of the sy phonies better than others (4 and 6 work be 7 and 9 fail to cohere). Konwitschny (Berling reminiscent of Klemperer and, to a lesse extent, Böhm in the tempos of his stel! implacable, muscular, unyielding, but pole fully effective, classically structured interpret tions. His inflexible Germanic rhythmic pi may not appeal to all.

The refurbished CD sound of the Krips (Everest) is very fine, but the orchestral tone nothing special. The interpretations are sober and steady but somewhat superficial. John McKelvey finds them smoother and less angular than Ferencsik, while Editor Vroon feels the tempos are similar to Szell but the more relaxed rhythms create softer edges. This was a reasonable contender when it came out in the early 1960s and there was less competition; approach with caution. Masur (Philips) offers straightforward, detailed, well-played renderings with sensibly chosen tempos, but it's hard to warm up to it. Böhm has the same ineluctable, monolithic quality but more imaginative interpretations and slightly better sound.

Pierre Monteux recorded 1-8 for Decca (9, for some reason, for Westminster) in the late 1950s with the London Symphony and Vienna Philharmonic. Both orchestras play gorgeous-ly, and these are wonderfully polished, firm, musical interpretations. So far, only 1, 3, 6, and 8 seem to have made it to CD. Less genial, the Leibowitz (Chesky) is fast and propulsive-some would call it ``driven''--—in the Toscanini mold, but with fine stereo sound. They are exciting, but not particularly perceptive. Leinsdorf (RCA) is often humorless, too. His set was overshadowed when it was
stands up well now, but it often seems to skim the surface.

\end{document}
