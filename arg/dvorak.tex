\documentclass[12pt]{article}%
\input setup

\begin{document}

\title{Dvorak Overview 1998}
\date{}
\maketitle

\addcontentsline{toc}{section}{\protect\textbf{Introduction}}
\noindent
Dvorak was an inventive and spontaneous composer---a ``natural''.
His music always flows, never sags, never seems abstract or sterile. 
There are no tensions, no great yearnings.
Even at his least inspired he is genial and pleasant and relaxed-without guile-and at his most inspired
he is a melodic genius, a second Schubert, who produced songful sounds of endless
beauty. His harmonies are rich and
sensuous, his themes fresh and joyful, his rhythms strong and elemental. He 
learned orchestration by playing in an orchestra, though he credited 
the birds for much of his use of woodwinds (you can hear that). 
His favorite score marking is grandioso, and when a conductor 
is on his wavelength you hear plenty of grandeur and majesty. 
As you are well aware, these elements increasingly elude living conductors, 
and more and more we are living in a world with no sense of grandeur or majesty. Dvorak can be made exciting
and will always be beautiful-without grandeur, but something will also be missing.
Antonin Dvorak was born September 8, 1841 in Bohemia, the eldest of eight children 
of essentially peasant stock---his father ran the local pub. He played the violin as a child. At 16
he went to Prague to study music. His earliest models were Beethoven and Schubert;
he spoke of Beethoven with awe but of Schubert with love.
From our vantage point he seems very much like another Schubert;
they both seemed to emit music the way trees emit oxygen.
He was also influenced by Wagner. He was very successful in Prague.
Starting in 1884 he made nine trips to England, where he was welcomed as a second Mendelssohn.
From 1892-94 he was head of the National Conservatory 
in New York and spent the summers traveling in the USA. 
In 1897 he was supposed to return to the USA but begged off---some think because
he was so upset over the death of his friend Brahms. He died in Prague in May 1904.

To discover a new Dvorak work is to enlarge one's list of favorites. Everyone knows the 9th Symphony---the New World---and when you go back from there you find one sparkling treasure after another. Many critics consider 7 and 8 greater symphonies than 9. Wherever it is performed, audiences still greet 6 with spontaneous enthusiasm. \No5 is exciting, and \No4 has two powerful Dvorak hythms. The Third is sumptuous in its orchesation, and \No1 is very winsome. Only \No2 can safely be ignored.

\section{Symphonies}
The late publication dates of many Dvorak symphonies account for the numbering problem.
Only five were published by Simrock, and they were numbered 1-5,
naturally. When the others were published in the 1950s and 60s the numbering
was revised to reflect the order they were written. Here's a chart explaining that.

\begin{enumerate}
\setlength{\itemsep}{0pt}
\item C minor, Bells of Zlonice
\item B-flat
\item E-flat
\item D minor
\item F, was No. 3
\item D, was No. 1
\item D minor, was No. 2
\item G, was No. 4
\item E minor, New World, was No. 5
\end{enumerate}
Some of the important books about Dvorak were published before the early symphonies were published. For example, Alec Robertson's book has a chapter on the symphonies that acknowledges right off that ``I can only speak of the first three symphonies at second hand'', and he proceeds to say almost nothing---and nothing worthwhile---about 1 and 2.

\subsection*{Symphony 1}

People who write about Dvorak's symphonies are very condescending about \No1. Karl Schumann's notes for the Kubelik set begin with something that does seem true:
\begin{quote}
Dvorak, with the emotional directness of his Slavic nature, open-hearted and filled with the urge to pour out his feelings in music, was too naive to experience the doubts that afflicted other composers when it came to tackling the problems of the symphony.
\end{quote}
Having said that, he points out that Dvorak wrote No. 1 when he was only 23, "although he had not mastered the problems which the form posed". He goes on to say that the work "shows the hand of a beginner". And that's about it: he says almost nothing further about it.

I cannot dismiss it like that. Sometime around 1970 I bought the Kertesz LP and fell in love with the first movement, with its gorgeous spinning-wheel theme for lower strings and brass. I have listened to every recording of it to come out since, and I've reached the stage where the only movement that doesn't excite me wildly is II, the Adagio. It is bland and leaves no impression; I think it is there for contrast (repose), and it was not intended to make a strong impression. But the Scherzo is a work of genius, reaching a peak in much the same manner as the scherzo in Beethoven's Fifth.

The finale is typical of Dvorak; it even has a great example of ``Dvorak rhythm''.
And it builds to a thrilling conclusion. Every time I listen to this symphony I love it more. It certainly has a bit of Schumann in it---and of Mendelssohn---but it is truly Dvorak: full of his special genius and the splendor of his orchestral sounds. Yet Dvorak never heard it, and it wasn't published until 1961. The Kertesz was the first recording (1965, I think).

Recordings have followed by Rowicki, Kubelik, Neumann, Suitner, Järvi, Gunzenhauser, and Macal. Having heard these, I have decided to discuss the four that I think are the strongest contenders for your dollars: Macal, Neumann, Kubelik, and Kertesz. First, the timings:

\begin{center}
\begin{tabular}{l c c c c}
&Macal & Neumann & Kubelik & Kertesz\\
I & 10:43 & 14:16 & 13:30 &18:50\\
II & 13:43 & 14:21 & 11:08 & 13:10\\
III & 8:20 &  9:20 & 9:35 & 8:40 \\
IV & 11:18 & 13:30 & 13:36 & 14:40\\
\end{tabular}
\end{center}

{\bf I}. Neumann's tempo is broader than Macal's and allows for more majesty. Neumann makes me love the music, as Kertesz did on LP. (1 don't know what accounts for Kertesz's timing---it doesn't seem outrageously slow, though it does emphasize the gloomy side of the music. It's not as ``heady'' as Neumann.) The main theme is quite wonderful and will remind you of a spinning wheel (a golden one or perhaps Omphale's).
Koss has the best sound of all---fairly close to the Kertesz LP, but I think the Kertesz on CD is not as smooth.
Supraphon is closer to the strings, and that makes that bass spinning-wheel
theme all the more imposing; but when the violins go at it full blast, they are so bright 
that you blink Still, What is Dvorak without those thrilling Czech violins? 
They play so briliantly, with a sort of frenzied elan. 
It's an effortless brilliance, like confident peasant dancing.
And the Czech Philharmonic certainly sounds more Czech--no surprise--especially the violins.
In Milwaukee they have to strain for whatever brilliance they attain.
In London they seem to lose body in the brilliant passages.

Kubelik has the Berlin Philharmonic, and they sound just great---especially the violins.
Dvorak was inclined to work his violins hard; they carry much of the music. 
DG (or MHS same recording) is inclined, as you know, to go light on the bass---to aim
for a ``clean'' sound with a lot of definition and detail. 
But the Kubelik is less that way than the Neumann; the bass is actually
pretty good, if not as good as London or Koss. It's certainly better than in
the average Karajan recording. And it was recorded at Jesus-Christus
Church---a great place to record. One trademark of the Berlin Philharmonic
sound is very strong timpani. 
In this first movement Kubelik is exciting and briliant, but he is missing Neumann's majesty.

{\bf II}. Neumann's conducting and orchestra
sound are very idiomatic. Supraphon's crisp recorded sound is attractive, 
but one can barely beat the warm, rich sound of the Koss. But Neumann shapes and phrases
more (or better) and makes the Macal and the Kertesz sound shapeless.
In a slow movement that is a liability. Macal sounds slower, though his tempo is perfectly
normal and even a little faster than Neumann's. Neumann just varies the pulse more. 
Kubelik has a very nasal ``Central European'' oboe---actually rather nice, and pretty rare these days.
Again Kubelik is exciting, and again he misses something---but here what he misses is very obvious:
the movement is labelled Molto Adagio. That means ``very slow''
and no one in his right mind would call what Kubelik does here slow---let alone very slow.
He may have decided that the movement needed help---and he may be right---but
the result is not what the composer called for. In this movement Neumann's conducting sweeps
the others aside. Again the Milwaukee winds are a joy---as good as any in any recording
(and Kertesz's and Kubelik's are delightful, too).

{\bf III}. London and Milwaukee are a little too refined here: the Czechs do some serious stomping. Their frenzied abandon carries the day, even at a slower tempo. 
There's nothing like it in any other recording. But, stomping
aside, the best scherzo musically comes from Berlin (Kubelik)---I just want to play it over and over.
Listen to that orchestra: the fullness of sound and the precise togetherness!

{\bf IV}. Again one marvels at the Berlin Philharmonic. They are simply great,
and they are part of what makes Kubelik's the most exciting performance.
Milwaukee is very noble and refined. One must admire their beautiful sound.
But Neumann is wild and unbuttoned and gets more exciting as it goes on---all
the way to the final glorious measures (which, by the way, fall rather flat
in the Kertesz but are better with Macal). The Suitner seems rather slow and dull
next to our choices, and the Rowicki is not well played or balanced (too much brass).
Järvi just isn't as revealing or exciting as the ones we list below.

For bright clarity and excitement and the quintessential Czech sound, turn to Neumann.
Moving toward more natural sound, Kubelik has every bit as much excitement---and that
glorious orchestra. Kertesz had very fine sound, if you can find it---but the LP was smoother.
The most beautiful sound of all is on the latest
recording: Macal on Koss---warm, natural concert-hall sound.

\st
Berlin/Kubelik & DG or MHS set\\
Milwaukee/Macal & Koss 1024\\
Czech Phil/Neumann & Sup 1003 [2CD]\\
London Sym/Kertesz & London NA
\et

\subsection*{Symphony 2}
When this symphony became available on LP (Kertesz, 1967) Ray Minshull wrote glowing notes,
praising the work to the skies (``outstanding achievement...magnificent road sweep...an abundance
of vintage-type Dvorak themes...remind one of Tchaikovsky''). 
Now it's true that in those days prospective yers could read the liner notes
on the back the album, and maybe Mr Minshull was ``laying it on''
to sell an utterly unknown work. But those of us who fell for it were
sorely disappointed. It is a dull and boring symphony without any 
``broad sweep'' or ``Dvorak-type themes'' at all---and Tchaikovsky is the last composer
one would be reminded of. All of us agree that the Second is the weakest Dvorak symphony,
and most of us could happily live without it. It starts out rather nicely but
never goes anywhere; it rambles and drifts, and there are no memorable melodies
and none of the youthful vigor of \No1. 55 minutes is far too long for its meager material.

It was written in 1865, same year as the First; but this one he did hear
performed---many years later, in 1888---and by that time he had revised it and
shortened it (but it's still too long). It was not published until 1959.

Carl Bauman likes it better than the rest of us, and he recommends the loving account of Kertesz
and the superior Kubelik. He calls the Neumann "flaccid" and "a disaster" , but both
he and Steven Haller find the Järvi a decent third choice.
The Gunzenhauser (Naxos) is in that same category. 
Kurt Moses reviewed the Pesek (July/Aug 1996) and seemed quite bored by it. The sound is terrific.

\st
Berlin/Kubelik & DG set or MHS 564673\\
London Sym/Kertesz & London NA
\et


\section*{Symphony 3}


This one came eight years after the first two, and it was performed: Simetana led
the Prague Philharmonic. It's the only Dvorak symphony with only three movements. 
The middle one is his longest slow movement. It's basically a theme and variations,
but the theme gives birth to themelets. In the middle of it a catchy march-like
section in pure Dvorak rhythm (dum, dum, dum-da-DUM-dum) acts as a scherzo.
It builds up, then sub-sides; but it returns slower and quieter in the coda
of the movement. Dvorak still hasn't written a good slow movement, 
but there is a sort of genius in the way this is organized-and you'll never forget that march.

The opening movement has wonderful flow and warmth; you are hooked right from the start. And the finale has an excitement and drive that will remind you of Schubert's Ninth.
Dvorak has Schubert's light-hearted playful-ness, too. Still, the orchestration of this symphony is sumptuous, almost Wagnerian. But, Schubert and Wagner aside, Dvorak always sounds only like Dvorak---he could never be mistaken for anyone else. And he loved this symphony all his life. Brahms loved it, too, and thought it worthy of the Austrian State Prize.
(The critic Hanslick was also on the committee, and Dvorak got a five-year government grant out of it.)

Järvi is much smoother than Kubelik. He has a natural flow and is both confident and serene. Kubelik is frantic, raucous, blasty---out to prove something---except in II. Järvi is faster, except in III. 
In II Kubelik constantly manipulates the tempos, fusses all the time, 
and loses the pulse of the march (it's deadly) and the spirit of the piece. 
Järvi lets the music carry him along. The Adagio is bland and has no character, until the march breaks in.
Kubelik tries to give it character but fails. Järvi plows thru it almost two minutes
faster to get to the march—because that's where the movement comes alive.
Kubelik takes 5 minutes for the march, playing around with its tempos to get more
expression out of it. Again it sounds manipulated. Järvi takes almost two minutes less,
never letting up on its forward motion.
I'm with Järvi. Once the march is essentially
over, the movement never quite falls back to its bland beginning; the march has given life to the other themes too! The Berlin Philharmonic saves the day again and again. The strings especially are way ahead of any other orchestra that has recorded this. But up to this point Kubelik has not won our hearts the way Dvorak should.

Kubelik drives things on to an exciting conclusion, but it still sounds manipulated.
Symphony 3 is Kubelik's worst recording. Järvi is again utterly natural, the music finding a more comfortable pace but building almost inexorably to a thrilling conclusion.
Those two are the contrasting recordings.
Kertesz is much closer to Järvi than Kubelik and could also be considered close to ideal.
Pesek on Virgin is a little on the slow side but very natural. The relaxed, leisurely pace allows for some lovely turns of phrase; and the sound is warm, creamy, and rich. Neumann is middle-of-the-road---a little rough in I---but has harsh sound. The Naxos sound for Gunzen-hauser is the opposite: too mellow and blended. The music needs an edge somewhere, and we should hear the woodwinds.

The Koss (Macal) is also rather mellow; but it's closer-up than the Järvi and without the reverberation. I rather like the atmospheric Chandos sound, but if you dislike echo you'll prefer Macal. In some climaxes everything gets jumbled together---but I like that; one often hears it like that in real life. The conducting is good in both cases, but Järvi is inclined to make more of the climaxes and more inclined to majesty. Macal takes the slow movement much slower and the last movement faster.
That makes it seem less gentle and attractive a little brusque, I think. But Milwaukee is a better orchestra than the Scots in every department (both could use more strings), and the Koss engineers supply more warmth and clarity and less echo. For my money no one comes near Järvi in 3. Rowicki is rather general and vague, Suitner even more so.

\st
Scots/Järvi & Chandos 8575\\
Milwaukee/Macal & Koss 1019\\
London Sym/Kertesz & London set (NA)
\et

\section*{Symphony 4}

This is less smooth and flowing than \No3. It has two striking Dvorak rhythms
and two gorgeous flowing melodies. You get one of each in I; the rhythm is TA-ta-tum-te-DUM. II is a long theme with variations-very Wagnerian; it is not very original, but it's his best slow movement so far. Ill is a swaggering scherzo-village music. Then comes the repetitious IV, where the first theme is another Dvorak rhythm---rum-tum, RUM-ti-tum---heard 17 times—and the second one of his trademark flowing melodies. The overall
\end{document}
