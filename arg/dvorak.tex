\documentclass[12pt]{article}%
\input setup

\begin{document}

\title{Dvorak Overview 1998}
\date{}
\maketitle

\addcontentsline{toc}{section}{\protect\textbf{Introduction}}
\noindent
Dvorak was an inventive and spontaneous composer---a ``natural''.
His music always flows, never sags, never seems abstract or sterile. 
There are no tensions, no great yearnings.
Even at his least inspired he is genial and pleasant and relaxed-without guile-and at his most inspired
he is a melodic genius, a second Schubert, who produced songful sounds of endless
beauty. His harmonies are rich and
sensuous, his themes fresh and joyful, his rhythms strong and elemental. He 
learned orchestration by playing in an orchestra, though he credited 
the birds for much of his use of woodwinds (you can hear that). 
His favorite score marking is grandioso, and when a conductor 
is on his wavelength you hear plenty of grandeur and majesty. 
As you are well aware, these elements increasingly elude living conductors, 
and more and more we are living in a world with no sense of grandeur or majesty. Dvorak can be made exciting
and will always be beautiful-without grandeur, but something will also be missing.
Antonin Dvorak was born September 8, 1841 in Bohemia, the eldest of eight children 
of essentially peasant stock---his father ran the local pub. He played the violin as a child. At 16
he went to Prague to study music. His earliest models were Beethoven and Schubert;
he spoke of Beethoven with awe but of Schubert with love.
From our vantage point he seems very much like another Schubert;
they both seemed to emit music the way trees emit oxygen.
He was also influenced by Wagner. He was very successful in Prague.
Starting in 1884 he made nine trips to England, where he was welcomed as a second Mendelssohn.
From 1892-94 he was head of the National Conservatory 
in New York and spent the summers traveling in the USA. 
In 1897 he was supposed to return to the USA but begged off---some think because
he was so upset over the death of his friend Brahms. He died in Prague in May 1904.

To discover a new Dvorak work is to enlarge one's list of favorites. Everyone knows the 9th Symphony---the New World---and when you go back from there you find one sparkling treasure after another. Many critics consider 7 and 8 greater symphonies than 9. Wherever it is performed, audiences still greet 6 with spontaneous enthusiasm. \No5 is exciting, and \No4 has two powerful Dvorak hythms. The Third is sumptuous in its orchesation, and \No1 is very winsome. Only \No2 can safely be ignored.

\section{Symphonies}
The late publication dates of many Dvorak symphonies account for the numbering problem.
Only five were published by Simrock, and they were numbered 1-5,
naturally. When the others were published in the 1950s and 60s the numbering
was revised to reflect the order they were written. Here's a chart explaining that.

\begin{enumerate}
\setlength{\itemsep}{0pt}
\item C minor, Bells of Zlonice
\item B-flat
\item E-flat
\item D minor
\item F, was No. 3
\item D, was No. 1
\item D minor, was No. 2
\item G, was No. 4
\item E minor, New World, was No. 5
\end{enumerate}
Some of the important books about Dvorak were published before the early symphonies were published. For example, Alec Robertson's book has a chapter on the symphonies that acknowledges right off that ``I can only speak of the first three symphonies at second hand'', and he proceeds to say almost nothing---and nothing worthwhile---about 1 and 2.

\subsection*{Symphony 1}

People who write about Dvorak's symphonies are very condescending about \No1. Karl Schumann's notes for the Kubelik set begin with something that does seem true:
\begin{quote}
Dvorak, with the emotional directness of his Slavic nature, open-hearted and filled with the urge to pour out his feelings in music, was too naive to experience the doubts that afflicted other composers when it came to tackling the problems of the symphony.
\end{quote}
Having said that, he points out that Dvorak wrote No. 1 when he was only 23, "although he had not mastered the problems which the form posed". He goes on to say that the work "shows the hand of a beginner". And that's about it: he says almost nothing further about it.

I cannot dismiss it like that. Sometime around 1970 I bought the Kertesz LP and fell in love with the first movement, with its gorgeous spinning-wheel theme for lower strings and brass. I have listened to every recording of it to come out since, and I've reached the stage where the only movement that doesn't excite me wildly is II, the Adagio. It is bland and leaves no impression; I think it is there for contrast (repose), and it was not intended to make a strong impression. But the Scherzo is a work of genius, reaching a peak in much the same manner as the scherzo in Beethoven's Fifth.

The finale is typical of Dvorak; it even has a great example of ``Dvorak rhythm''.
And it builds to a thrilling conclusion. Every time I listen to this symphony I love it more. It certainly has a bit of Schumann in it---and of Mendelssohn---but it is truly Dvorak: full of his special genius and the splendor of his orchestral sounds. Yet Dvorak never heard it, and it wasn't published until 1961. The Kertesz was the first recording (1965, I think).

Recordings have followed by Rowicki, Kubelik, Neumann, Suitner, Järvi, Gunzenhauser, and Macal. Having heard these, I have decided to discuss the four that I think are the strongest contenders for your dollars: Macal, Neumann, Kubelik, and Kertesz. First, the timings:

\begin{center}
\begin{tabular}{l c c c c}
&Macal & Neumann & Kubelik & Kertesz\\
I & 10:43 & 14:16 & 13:30 &18:50\\
II & 13:43 & 14:21 & 11:08 & 13:10\\
III & 8:20 &  9:20 & 9:35 & 8:40 \\
IV & 11:18 & 13:30 & 13:36 & 14:40\\
\end{tabular}
\end{center}

{\bf I}. Neumann's tempo is broader than Macal's and allows for more majesty. Neumann makes me love the music, as Kertesz did on LP. (1 don't know what accounts for Kertesz's timing---it doesn't seem outrageously slow, though it does emphasize the gloomy side of the music. It's not as ``heady'' as Neumann.) The main theme is quite wonderful and will remind you of a spinning wheel (a golden one or perhaps Omphale's).
Koss has the best sound of all---fairly close to the Kertesz LP, but I think the Kertesz on CD is not as smooth.
Supraphon is closer to the strings, and that makes that bass spinning-wheel
theme all the more imposing; but when the violins go at it full blast, they are so bright 
that you blink Still, What is Dvorak without those thrilling Czech violins? 
They play so briliantly, with a sort of frenzied elan. 
It's an effortless brilliance, like confident peasant dancing.
And the Czech Philharmonic certainly sounds more Czech--no surprise--especially the violins.
In Milwaukee they have to strain for whatever brilliance they attain.
In London they seem to lose body in the brilliant passages.

Kubelik has the Berlin Philharmonic, and they sound just great---especially the violins.
Dvorak was inclined to work his violins hard; they carry much of the music. 
DG (or MHS same recording) is inclined, as you know, to go light on the bass---to aim
for a ``clean'' sound with a lot of definition and detail. 
But the Kubelik is less that way than the Neumann; the bass is actually
pretty good, if not as good as London or Koss. It's certainly better than in
the average Karajan recording. And it was recorded at Jesus-Christus
Church---a great place to record. One trademark of the Berlin Philharmonic
sound is very strong timpani. 
In this first movement Kubelik is exciting and briliant, but he is missing Neumann's majesty.

{\bf II}. Neumann's conducting and orchestra
sound are very idiomatic. Supraphon's crisp recorded sound is attractive, 
but one can barely beat the warm, rich sound of the Koss. But Neumann shapes and phrases
more (or better) and makes the Macal and the Kertesz sound shapeless.
In a slow movement that is a liability. Macal sounds slower, though his tempo is perfectly
normal and even a little faster than Neumann's. Neumann just varies the pulse more. 
Kubelik has a very nasal "Central European" oboe-actually rather nice, and pretty rare these days.
Again Kubelik is exciting, and again he misses something but here what he misses is very obvious:
the movement is labelled Molto Adagio. That means ``very slow''
and no one in his right mind would call what Kubelik does here slow---let alone very slow.
He may have decided that the movement needed help---and he may be right---but
the result is not what the composer called for. In this movement Neumann's conducting sweeps
the others aside. Again the Milwaukee winds are a joy---as good as any in any recording
(and Kertesz's and Kubelik's are delightful, too).

{\bf III}. London and Milwaukee are a little too refined here: the Czechs do some serious stomping. Their frenzied abandon carries the day, even at a slower tempo. There's nothing
\end{document}
