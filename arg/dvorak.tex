\documentclass[12pt]{article}%
\input setup

\begin{document}

\title{Dvořák Overview 1998}
\date{}
\maketitle

\addcontentsline{toc}{section}{\protect\textbf{Introduction}}
\noindent
Dvořák was an inventive and spontaneous composer---a ``natural''.
His music always flows, never sags, never seems abstract or sterile. 
There are no tensions, no great yearnings.
Even at his least inspired he is genial and pleasant and relaxed-without guile-and at his most inspired
he is a melodic genius, a second Schubert, who produced songful sounds of endless
beauty. His harmonies are rich and
sensuous, his themes fresh and joyful, his rhythms strong and elemental. He 
learned orchestration by playing in an orchestra, though he credited 
the birds for much of his use of woodwinds (you can hear that). 
His favorite score marking is grandioso, and when a conductor 
is on his wavelength you hear plenty of grandeur and majesty. 
As you are well aware, these elements increasingly elude living conductors, 
and more and more we are living in a world with no sense of grandeur or majesty. Dvořák can be made exciting
and will always be beautiful-without grandeur, but something will also be missing.
Antonin Dvořák was born September 8, 1841 in Bohemia, the eldest of eight children 
of essentially peasant stock---his father ran the local pub. He played the violin as a child. At 16
he went to Prague to study music. His earliest models were Beethoven and Schubert;
he spoke of Beethoven with awe but of Schubert with love.
From our vantage point he seems very much like another Schubert;
they both seemed to emit music the way trees emit oxygen.
He was also influenced by Wagner. He was very successful in Prague.
Starting in 1884 he made nine trips to England, where he was welcomed as a second Mendelssohn.
From 1892-94 he was head of the National Conservatory 
in New York and spent the summers traveling in the USA. 
In 1897 he was supposed to return to the USA but begged off---some think because
he was so upset over the death of his friend Brahms. He died in Prague in May 1904.

To discover a new Dvořák work is to enlarge one's list of favorites. Everyone knows the 9th Symphony---the New World---and when you go back from there you find one sparkling treasure after another. Many critics consider 7 and 8 greater symphonies than 9. Wherever it is performed, audiences still greet 6 with spontaneous enthusiasm. \No5 is exciting, and \No4 has two powerful Dvořák hythms. The Third is sumptuous in its orchestration, and \No1 is very winsome. Only \No2 can safely be ignored.

\section{Symphonies}
The late publication dates of many Dvořák symphonies account for the numbering problem.
Only five were published by Simrock, and they were numbered 1-5,
naturally. When the others were published in the 1950s and 60s the numbering
was revised to reflect the order they were written. Here's a chart explaining that.

\begin{enumerate}
\setlength{\itemsep}{0pt}
\item C minor, Bells of Zlonice
\item B-flat
\item E-flat
\item D minor
\item F, was No. 3
\item D, was No. 1
\item D minor, was No. 2
\item G, was No. 4
\item E minor, New World, was No. 5
\end{enumerate}
Some of the important books about Dvořák were published before the early symphonies were published. For example, Alec Robertson's book has a chapter on the symphonies that acknowledges right off that ``I can only speak of the first three symphonies at second hand'', and he proceeds to say almost nothing---and nothing worthwhile---about 1 and 2.

\subsection*{Symphony 1}

People who write about Dvořák's symphonies are very condescending about \No1. Karl Schumann's notes for the Kubelik set begin with something that does seem true:
\begin{quote}
Dvořák, with the emotional directness of his Slavic nature, open-hearted and filled with the urge to pour out his feelings in music, was too naive to experience the doubts that afflicted other composers when it came to tackling the problems of the symphony.
\end{quote}
Having said that, he points out that Dvořák wrote No. 1 when he was only 23, "although he had not mastered the problems which the form posed". He goes on to say that the work "shows the hand of a beginner". And that's about it: he says almost nothing further about it.

I cannot dismiss it like that. Sometime around 1970 I bought the Kertesz LP and fell in love with the first movement, with its gorgeous spinning-wheel theme for lower strings and brass. I have listened to every recording of it to come out since, and I've reached the stage where the only movement that doesn't excite me wildly is II, the Adagio. It is bland and leaves no impression; I think it is there for contrast (repose), and it was not intended to make a strong impression. But the Scherzo is a work of genius, reaching a peak in much the same manner as the scherzo in Beethoven's Fifth.

The finale is typical of Dvořák; it even has a great example of ``Dvořák rhythm''.
And it builds to a thrilling conclusion. Every time I listen to this symphony I love it more. It certainly has a bit of Schumann in it---and of Mendelssohn---but it is truly Dvořák: full of his special genius and the splendor of his orchestral sounds. Yet Dvořák never heard it, and it wasn't published until 1961. The Kertesz was the first recording (1965, I think).

Recordings have followed by Rowicki, Kubelik, Neumann, Suitner, Järvi, Gunzenhauser, and Macal. Having heard these, I have decided to discuss the four that I think are the strongest contenders for your dollars: Macal, Neumann, Kubelik, and Kertesz. First, the timings:

\begin{center}
\begin{tabular}{l c c c c}
&Macal & Neumann & Kubelik & Kertesz\\
I & 10:43 & 14:16 & 13:30 &18:50\\
II & 13:43 & 14:21 & 11:08 & 13:10\\
III & 8:20 &  9:20 & 9:35 & 8:40 \\
IV & 11:18 & 13:30 & 13:36 & 14:40\\
\end{tabular}
\end{center}

{\bf I}. Neumann's tempo is broader than Macal's and allows for more majesty. Neumann makes me love the music, as Kertesz did on LP. (1 don't know what accounts for Kertesz's timing---it doesn't seem outrageously slow, though it does emphasize the gloomy side of the music. It's not as ``heady'' as Neumann.) The main theme is quite wonderful and will remind you of a spinning wheel (a golden one or perhaps Omphale's).
Koss has the best sound of all---fairly close to the Kertesz LP, but I think the Kertesz on CD is not as smooth.
Supraphon is closer to the strings, and that makes that bass spinning-wheel
theme all the more imposing; but when the violins go at it full blast, they are so bright 
that you blink Still, What is Dvořák without those thrilling Czech violins? 
They play so briliantly, with a sort of frenzied elan. 
It's an effortless brilliance, like confident peasant dancing.
And the Czech Philharmonic certainly sounds more Czech--no surprise--especially the violins.
In Milwaukee they have to strain for whatever brilliance they attain.
In London they seem to lose body in the brilliant passages.

Kubelik has the Berlin Philharmonic, and they sound just great---especially the violins.
Dvořák was inclined to work his violins hard; they carry much of the music. 
DG (or MHS same recording) is inclined, as you know, to go light on the bass---to aim
for a ``clean'' sound with a lot of definition and detail. 
But the Kubelik is less that way than the Neumann; the bass is actually
pretty good, if not as good as London or Koss. It's certainly better than in
the average Karajan recording. And it was recorded at Jesus-Christus
Church---a great place to record. One trademark of the Berlin Philharmonic
sound is very strong timpani. 
In this first movement Kubelik is exciting and briliant, but he is missing Neumann's majesty.

{\bf II}. Neumann's conducting and orchestra
sound are very idiomatic. Supraphon's crisp recorded sound is attractive, 
but one can barely beat the warm, rich sound of the Koss. But Neumann shapes and phrases
more (or better) and makes the Macal and the Kertesz sound shapeless.
In a slow movement that is a liability. Macal sounds slower, though his tempo is perfectly
normal and even a little faster than Neumann's. Neumann just varies the pulse more. 
Kubelik has a very nasal ``Central European'' oboe---actually rather nice, and pretty rare these days.
Again Kubelik is exciting, and again he misses something---but here what he misses is very obvious:
the movement is labelled Molto Adagio. That means ``very slow''
and no one in his right mind would call what Kubelik does here slow---let alone very slow.
He may have decided that the movement needed help---and he may be right---but
the result is not what the composer called for. In this movement Neumann's conducting sweeps
the others aside. Again the Milwaukee winds are a joy---as good as any in any recording
(and Kertesz's and Kubelik's are delightful, too).

{\bf III}. London and Milwaukee are a little too refined here: the Czechs do some serious stomping. Their frenzied abandon carries the day, even at a slower tempo. 
There's nothing like it in any other recording. But, stomping
aside, the best scherzo musically comes from Berlin (Kubelik)---I just want to play it over and over.
Listen to that orchestra: the fullness of sound and the precise togetherness!

{\bf IV}. Again one marvels at the Berlin Philharmonic. They are simply great,
and they are part of what makes Kubelik's the most exciting performance.
Milwaukee is very noble and refined. One must admire their beautiful sound.
But Neumann is wild and unbuttoned and gets more exciting as it goes on---all
the way to the final glorious measures (which, by the way, fall rather flat
in the Kertesz but are better with Macal). The Suitner seems rather slow and dull
next to our choices, and the Rowicki is not well played or balanced (too much brass).
Järvi just isn't as revealing or exciting as the ones we list below.

For bright clarity and excitement and the quintessential Czech sound, turn to Neumann.
Moving toward more natural sound, Kubelik has every bit as much excitement---and that
glorious orchestra. Kertesz had very fine sound, if you can find it---but the LP was smoother.
The most beautiful sound of all is on the latest
recording: Macal on Koss---warm, natural concert-hall sound.

\st
Berlin/Kubelik & DG or MHS set\\
Milwaukee/Macal & Koss 1024\\
Czech Phil/Neumann & Sup 1003 [2CD]\\
London Sym/Kertesz & London NA
\et

\subsection*{Symphony 2}
When this symphony became available on LP (Kertesz, 1967) Ray Minshull wrote glowing notes,
praising the work to the skies (``outstanding achievement...magnificent road sweep...an abundance
of vintage-type Dvořák themes...remind one of Tchaikovsky''). 
Now it's true that in those days prospective yers could read the liner notes
on the back the album, and maybe Mr Minshull was ``laying it on''
to sell an utterly unknown work. But those of us who fell for it were
sorely disappointed. It is a dull and boring symphony without any 
``broad sweep'' or ``Dvořák-type themes'' at all---and Tchaikovsky is the last composer
one would be reminded of. All of us agree that the Second is the weakest Dvořák symphony,
and most of us could happily live without it. It starts out rather nicely but
never goes anywhere; it rambles and drifts, and there are no memorable melodies
and none of the youthful vigor of \No1. 55 minutes is far too long for its meager material.

It was written in 1865, same year as the First; but this one he did hear
performed---many years later, in 1888---and by that time he had revised it and
shortened it (but it's still too long). It was not published until 1959.

Carl Bauman likes it better than the rest of us, and he recommends the loving account of Kertesz
and the superior Kubelik. He calls the Neumann "flaccid" and "a disaster" , but both
he and Steven Haller find the Järvi a decent third choice.
The Gunzenhauser (Naxos) is in that same category. 
Kurt Moses reviewed the Pe\v{s}ek (July/Aug 1996) and seemed quite bored by it. The sound is terrific.

\st
Berlin/Kubelik & DG set or MHS 564673\\
London Sym/Kertesz & London NA
\et


\section*{Symphony 3}


This one came eight years after the first two, and it was performed: Simetana led
the Prague Philharmonic. It's the only Dvořák symphony with only three movements. 
The middle one is his longest slow movement. It's basically a theme and variations,
but the theme gives birth to themelets. In the middle of it a catchy march-like
section in pure Dvořák rhythm (dum, dum, dum-da-DUM-dum) acts as a scherzo.
It builds up, then sub-sides; but it returns slower and quieter in the coda
of the movement. Dvořák still hasn't written a good slow movement, 
but there is a sort of genius in the way this is organized-and you'll never forget that march.

The opening movement has wonderful flow and warmth; you are hooked right from the start. And the finale has an excitement and drive that will remind you of Schubert's Ninth.
Dvořák has Schubert's light-hearted playful-ness, too. Still, the orchestration of this symphony is sumptuous, almost Wagnerian. But, Schubert and Wagner aside, Dvořák always sounds only like Dvořák---he could never be mistaken for anyone else. And he loved this symphony all his life. Brahms loved it, too, and thought it worthy of the Austrian State Prize.
(The critic Hanslick was also on the committee, and Dvořák got a five-year government grant out of it.)

Järvi is much smoother than Kubelik. He has a natural flow and is both confident and serene. Kubelik is frantic, raucous, blasty---out to prove something---except in II. Järvi is faster, except in III. 
In II Kubelik constantly manipulates the tempos, fusses all the time, 
and loses the pulse of the march (it's deadly) and the spirit of the piece. 
Järvi lets the music carry him along. The Adagio is bland and has no character, until the march breaks in.
Kubelik tries to give it character but fails. Järvi plows thru it almost two minutes
faster to get to the march—because that's where the movement comes alive.
Kubelik takes 5 minutes for the march, playing around with its tempos to get more
expression out of it. Again it sounds manipulated. Järvi takes almost two minutes less,
never letting up on its forward motion.
I'm with Järvi. Once the march is essentially
over, the movement never quite falls back to its bland beginning; the march has given life to the other themes too! The Berlin Philharmonic saves the day again and again. The strings especially are way ahead of any other orchestra that has recorded this. But up to this point Kubelik has not won our hearts the way Dvořák should.

Kubelik drives things on to an exciting conclusion, but it still sounds manipulated.
Symphony 3 is Kubelik's worst recording. Järvi is again utterly natural, the music finding a more comfortable pace but building almost inexorably to a thrilling conclusion.
Those two are the contrasting recordings.
Kertesz is much closer to Järvi than Kubelik and could also be considered close to ideal.
Pe\v{s}ek on Virgin is a little on the slow side but very natural. The relaxed, leisurely pace allows for some lovely turns of phrase; and the sound is warm, creamy, and rich. Neumann is middle-of-the-road---a little rough in I---but has harsh sound. The Naxos sound for Gunzen-hauser is the opposite: too mellow and blended. The music needs an edge somewhere, and we should hear the woodwinds.

The Koss (Macal) is also rather mellow; but it's closer-up than the Järvi and without the reverberation. I rather like the atmospheric Chandos sound, but if you dislike echo you'll prefer Macal. In some climaxes everything gets jumbled together---but I like that; one often hears it like that in real life. The conducting is good in both cases, but Järvi is inclined to make more of the climaxes and more inclined to majesty. Macal takes the slow movement much slower and the last movement faster.
That makes it seem less gentle and attractive a little brusque, I think. But Milwaukee is a better orchestra than the Scots in every department (both could use more strings), and the Koss engineers supply more warmth and clarity and less echo. For my money no one comes near Järvi in 3. Rowicki is rather general and vague, Suitner even more so.

\st
Scots/Järvi & Chandos 8575\\
Milwaukee/Macal & Koss 1019\\
London Sym/Kertesz & London set (NA)
\et

\section*{Symphony 4}

This is less smooth and flowing than \No3. It has two striking Dvořák rhythms
and two gorgeous flowing melodies. You get one of each in I; the rhythm is TA-ta-tum-te-DUM. II is a long theme with variations-very Wagnerian; it is not very original, but it's his best slow movement so far. Ill is a swaggering scherzo-village music. Then comes the repetitious IV, where the first theme is another Dvořák rhythm---rum-tum, RUM-ti-tum---heard 17 times—and the second one of his trademark flowing melodies. The overall
impression when you've heard it all is of exuberant grandeur-and maybe that's a good description of Dvořák's music in general.

The Kertesz stood alone for years as the only really good recording. It's quite bracing and will always be worth hearing. At this stage Kertesz's scherzo is still the most exciting. The Kubelik was no particular threat: Carl Bauman calls it turgid; I call it jumpy: the scherzo seems especially lumpy, and the second theme of the finale is especially jittery---has no repose. There is more bounce and surge and bump than flow in Kubelik's Fourth. The Järvi is no threat either---soggy and boring.

But after 25 years at the top of the list the Kertesz was finally surpassed by the Pe\v{s}ek. The gorgeous, velvety sound---especially the strings---is a large part of its superiority. The orchestra is the Czech Philharmonic, in their home hall, and their playing and the place add a great deal to the picture. They respond to Pe\v{s}ek with much richer tone and much more fire than they were giving Neumann the last few years of his tenure. The approach is smoother and tempos are broader than Kertesz, but I is not as slow as Järvi. Musical opulence and majesty! When you hear a recording this good, you thank God for digital.

A year or two later Koss brought out the Macal, and now there are three great Fourths---Kertesz, Macal, and Pe\v{s}ek---but the greatest of these is Pe\v{s}ek. Macal is much smoother and warmer sounding than the rather stark and edgy Kertesz (which vas miked pretty close-up) though the tempos are about the same in I and III.
A slower II makes it seem more Wagnerian than the Kertesz does, and a slower IV makes little or no difference.
The Milwaukee players of today are as good as the London Symphony people of the late 1960s, and the Koss recording is more natural---more ``concert-hall''.

\st
Czech Phil/Pe\v{s}ek& Virgin 59016\\
Milwaukee/Macal&Koss 1015\\
London Sym/Kertesz & London NA
\et

\section*{Symphony 5}

Dvořák seems to have simplified his style
a bit for this one. The opening is very pastoral----the woods, the fields, the birds. It
could easily be by Smetana (Bohemia's Forests and Meadows). The opening clarinet theme is light and cheery. The theme of innocent enjoyment of nature seems to carry thru the whole of I, including an ingenious and serene coda. II has a lovely cello melody. III is bursting with song and wit---a typical Dvořák scherzo. IV involves a change of pace and key, but a good conductor can make it seem a natural outgrowth of what has come before. It's quite stirring, and the whole symphony seems to have been building up to it, when you look back. It's
no longer a pastoral symphony by the time you reach the last movement.

The Kubelik is very exciting, and part of it is certainly the Berlin Philharmonic: listen to those strings! But I prefer the Scottish oboe (Järvi). There is no difference in tempos between Järvi and Kubelik---a matter of a few seconds here and there-and the Kertesz is pretty close, too. Three terrific conductors in great form. The Järvi is recorded further back in the hall; I like that; Tom Godell likes it less, and Steven Haller complains bitterly of echo-chamber sound. The Kubelik and Kertesz are closer up. Sometimes Kubelik startles you, because an orchestral outburst can seem so close at hand. Kertesz sounded smoother on LP than on CD. But I have to say that it's impossible to choose among these three. I have even come to accept Kubelik's I, which I objected to in the 1990 overview. The clarinet theme is so winsome, and the movement gets so exciting that I can't figure out what I once objected to—probably just that the other two conductors are a little smoother, more pastoral
and flowing.

The Neumann is rather dull. The sound is crisp, and the orchestra plays well; but you will miss the intensity and alertness, the contrasts and the coherence of a conductor like Järvi. In fact, Järvi has it down perfectly: he seems more Czech than Neumann. Pe\v{s}ek again comes thru here, with the same virtues as his 4th: glorious sound, great orchestra, natural Czech interpretation. Macal has the virtues and drawbacks you would expect. The Milwaukee Symphony plays very beautifully, but the strings are not as thrilling as the Czech or Berlin ones. The sound is as good as it gets (if you like it back-in-the-hall; Steven Haller prefers more of an edge and finds it muffled---review this issue).
Macal is choppier and more energetic than Järvi---or almost anyone else---but I prefer Dvořák smoother and more relaxed. IV is about as good as it gets anywhere. Koss's notes are not well written.

The old Rowicki is pretty good---one of his better Dvořáks. Mr Haldeman calls the ansons fresh and spontaneous, but others don't know it. Steven Haller liked the Naxos Gunzenhauser, but it is coupled
with a less-than-competitive 7th. Still, if you don't know the 5th, Naxos gives you a cheap way to hear a good performance of it.

\st
Scots/Järvi & Chandos 8552\\
Berlin/Kubelik & DG or MHS set\\
London Sym/Kertesz & London set (NA)\\
Czech Phil/Pe\v{s}ek  & Virgin 59522\\
Milwaukee/Macal & Koss 1026
\et

\section*{Symphony 6}

This one was written for Hans Richter and the Vienna Phiharmonic in 1880, but the first performance took place in Prague, and it took the Vienna Philharmonic a long time to get around to it (though Richter conducted it often in other cities). It is in the same key as the Brahms Second, but its opening resembles the Brahms Fourth---and it's another brilliant opening that ``hooks'' the listener right away. II is sweet and gentle, and near the end we hear a hint of the quiet violin solo near the end of Also Sprach Zarathustra. The scherzo has the usual Dvořák rhythmic vigor, and the trio is serene.
The final movement is happy and untroubled.

For some reason, Kubelik takes longer over that last movement than anyone else. His tempos are varied more; some parts are faster than most conductors. The Berlin Philharmonic is simply unbeatable. The Kubelik remains a very exciting and dramatic recording-almost a lesson in great conducting: how to make the details count while building a whole structure that makes sense and reaches all the peaks. He has a lot of fire, but there are also a lot of drastic tempo shifts.

Some of us prefer the music less manipulated, with a more natural and dependable flow. For that it is impossible to beat B\v{e}lohlavek. He enriches the music with majesty and subtlety (listen to those strings!).
He also has the best sound of any-absolutely and without doubt. Kertesz is more emphatic than B\v{e}lohlavek---and brassier and brasher---but still softer and gentler than Kubelik. He also has better sound than Kubelik, but it is closer-up sound, and that may irritate some listeners. But I am very unhappy with his exposition repeat: four minutes into I he starts all over---and rather dashes thru it all the second time around. Why bother? Dvořák himself never took that repeat and never wanted it.

Libor Pe\v{s}ek, who impressed us so much in a number of the other symphonies, leaves us cold in this one. (Why is it that no one seems to do them all well?) The sound is a little distant this time, and the conductor seems to float thru the music—misses a lot of points that Kubelik and Kertesz facilitate. Those two conductors are more cumulative-know how to build a climax. They are also more coherent, more stirring, more heady-more of a total experience (I include B\v{e}lohlavek in this description). It's the same Czech Philharmonic that plays for B\v{e}lohlavek, and they sound very good, but B\v{e}lohlavek has figured this symphony out and Pe\v{s}ek has not.

Another Czech Philharmonic recording---with Ancerl---promises a lot but lets us down.
For some reason the conducting is brusque and routine, with very little shaping.
Neumann's 1982 recording---the one readily available right now---is quite beautiful.
The sound is not as appealing as the B\v{e}lohlavek,
unless you like it cleaner (as opposed to warmer), but those Czech Philharmonic violins are something special. It, like the rest of his series, is in a box with two other sym-phonies—4 and 5.

Macal is pretty straightforward, with little tempo manipulation. I is strong and lite, tough but with sensitive detail---very appealing. II is very lovely and makes the Järvi sound insensitive. IV is as exciting as it gets. The Milwaukee Symphony again plays beautifully, and the sound is demonstration-class Jan/ Feb 1990). The Cleveland Orchestra also sounds wonderful for Dohnanyi, and his II is just right. We decided that the conductor and the engineering overdramatize the music (Nov/ Dec 1991). The Rowicki---along with his 5---is excellent, but the sound is 1960s and not up to Kertesz, let alone B\v{e}lohlavek. The Suitner was reviewed as ``energetic but ordinary''. Slatkin has the slowest II and comes in an expensive box set available only from the St Louis Symphony. Tom Godell considers it the best 6th ever, and I remember it as one of the best. Another specific criticism of the Järvi is that his tempo in IV is cranked up too far—the orchestra has trouble with it, too Jan/Feb 1988—Haller). Beyond that, it just seems generally inferior to the ones we recommend.

\st
Czech Phil/Belohlavek & Chandos 9170\\
Berlin/Kubelik  & DG or MHS set\\
London Sym/Kertesz & London NA\\
Milwaukee/Macal & Koss 1001\\
Czech Phil/NeumannSupraphon & 1005 [2CD]
\et

\section*{Symphony 7}

The Seventh was written for the London Philharmonic in 1885, and  Dvořák conducted the first performance. 
It was an immediate success, and wherever it has been played ever since it has won many friends. Sir Donald Tovey placed it alongside the Brahms four as the greatest symphonies since Beethoven.
Other eminent critics have called it profusely inspired and a great symphony by any standard. Many think it is Dvorak's greatest.

It is also his most Brahmsian. It took him only three months to write it—not surprising, considering the level of inspiration. Some people hear hints of two great themes in it: the second theme in I resembles the beautiful cello solo in the Second Brahms Piano Concerto, and in II one spot sounds like the love duet from Tristan and Isolde. (To me it all sounds like Brahms, not Wagner.) The Scherzo trips along lightly and cheerfully; conductors are inclined to over-inflect it, but it's really a prime example of the natural flow of Dvorak's music. Yet there is passion in this symphony, and that can be slighted as well. It's not an easy work to get right.
Suitner's 7 is forceful, with sharp attacks


\end{document}
