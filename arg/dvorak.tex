\documentclass[12pt]{article}%
\input setup

\begin{document}

\title{Dvořák Overview 1998}
\date{}
\maketitle

\addcontentsline{toc}{section}{\protect\textbf{Introduction}}
\noindent
Dvořák was an inventive and spontaneous composer---a ``natural''.
His music always flows, never sags, never seems abstract or sterile. 
There are no tensions, no great yearnings.
Even at his least inspired he is genial and pleasant and relaxed---without guile---and at his most inspired
he is a melodic genius, a second Schubert, who produced songful sounds of endless
beauty. His harmonies are rich and
sensuous, his themes fresh and joyful, his rhythms strong and elemental. He 
learned orchestration by playing in an orchestra, though he credited 
the birds for much of his use of woodwinds (you can hear that). 
His favorite score marking is grandioso, and when a conductor 
is on his wavelength you hear plenty of grandeur and majesty. 
As you are well aware, these elements increasingly elude living conductors, 
and more and more we are living in a world with no sense of grandeur or majesty. Dvořák can be made exciting
and will always be beautiful---without grandeur, but something will also be missing.
Antonin Dvořák was born September 8, 1841 in Bohemia, the eldest of eight children 
of essentially peasant stock---his father ran the local pub. He played the violin as a child. At 16
he went to Prague to study music. His earliest models were Beethoven and Schubert;
he spoke of Beethoven with awe but of Schubert with love.
From our vantage point he seems very much like another Schubert;
they both seemed to emit music the way trees emit oxygen.
He was also influenced by Wagner. He was very successful in Prague.
Starting in 1884 he made nine trips to England, where he was welcomed as a second Mendelssohn.
From 1892-94 he was head of the National Conservatory 
in New York and spent the summers traveling in the USA. 
In 1897 he was supposed to return to the USA but begged off---some think because
he was so upset over the death of his friend Brahms. He died in Prague in May 1904.

To discover a new Dvořák work is to enlarge one's list of favorites. Everyone knows the 9th Symphony---the New World---and when you go back from there you find one sparkling treasure after another. Many critics consider 7 and 8 greater symphonies than 9. Wherever it is performed, audiences still greet 6 with spontaneous enthusiasm. \No5 is exciting, and \No4 has two powerful Dvořák hythms. The Third is sumptuous in its orchestration, and \No1 is very winsome. Only \No2 can safely be ignored.

\section{Symphonies}
The late publication dates of many Dvořák symphonies account for the numbering problem.
Only five were published by Simrock, and they were numbered 1-5,
naturally. When the others were published in the 1950s and 60s the numbering
was revised to reflect the order they were written. Here's a chart explaining that.

\begin{enumerate}
\setlength{\itemsep}{0pt}
\item C minor, Bells of Zlonice
\item B-flat
\item E-flat
\item D minor
\item F, was No. 3
\item D, was No. 1
\item D minor, was No. 2
\item G, was No. 4
\item E minor, New World, was No. 5
\end{enumerate}
Some of the important books about Dvořák were published before the early symphonies were published. For example, Alec Robertson's book has a chapter on the symphonies that acknowledges right off that ``I can only speak of the first three symphonies at second hand'', and he proceeds to say almost nothing---and nothing worthwhile---about 1 and 2.

\section{Symphony 1}

People who write about Dvořák's symphonies are very condescending about \No1. Karl Schumann's notes for the Kubelik set begin with something that does seem true:
\begin{quote}
Dvořák, with the emotional directness of his Slavic nature, open-hearted and filled with the urge to pour out his feelings in music, was too naive to experience the doubts that afflicted other composers when it came to tackling the problems of the symphony.
\end{quote}
Having said that, he points out that Dvořák wrote No. 1 when he was only 23, ``although he had not mastered the problems which the form posed''. He goes on to say that the work ``shows the hand of a beginner''. And that's about it: he says almost nothing further about it.

I cannot dismiss it like that. Sometime around 1970 I bought the Kertesz LP and fell in love with the first movement, with its gorgeous spinning-wheel theme for lower strings and brass. I have listened to every recording of it to come out since, and I've reached the stage where the only movement that doesn't excite me wildly is II, the Adagio. It is bland and leaves no impression; I think it is there for contrast (repose), and it was not intended to make a strong impression. But the Scherzo is a work of genius, reaching a peak in much the same manner as the scherzo in Beethoven's Fifth.

The finale is typical of Dvořák; it even has a great example of ``Dvořák rhythm''.
And it builds to a thrilling conclusion. Every time I listen to this symphony I love it more. It certainly has a bit of Schumann in it---and of Mendelssohn---but it is truly Dvořák: full of his special genius and the splendor of his orchestral sounds. Yet Dvořák never heard it, and it wasn't published until 1961. The Kertesz was the first recording (1965, I think).

Recordings have followed by Rowicki, Kubelik, Neumann, Suitner, Järvi, Gunzenhauser, and Macal. Having heard these, I have decided to discuss the four that I think are the strongest contenders for your dollars: Macal, Neumann, Kubelik, and Kertesz. First, the timings:\footnote{Kertesz really takes 12:40-ish for the 4th movement here. The 14:40 seems to come from the original LP, but none of the CD reissues have that time. All the other timings here are about right. The review of the Macal in the ARG issue where this appears explains why the timings on the CD are completely different than what is listed. The CD book lists the contents of the disk in the wrong order.}

\begin{center}
\begin{tabular}{l c c c c}
&Macal & Neumann & Kubelik & Kertesz\\
I & 10:43 & 14:16 & 13:30 &18:50\\
II & 13:43 & 14:21 & 11:08 & 13:10\\
III & 8:20 &  9:20 & 9:35 & 8:40 \\
IV & 11:18 & 13:30 & 13:36 & 14:40\\
\end{tabular}
\end{center}

{\bf I}. Neumann's tempo is broader than Macal's and allows for more majesty. Neumann makes me love the music, as Kertesz did on LP. (I don't know what accounts for Kertesz's timing---it doesn't seem outrageously slow, though it does emphasize the gloomy side of the music. It's not as ``heady'' as Neumann.) The main theme is quite wonderful and will remind you of a spinning wheel (a golden one or perhaps Omphale's).
Koss has the best sound of all---fairly close to the Kertesz LP, but I think the Kertesz on CD is not as smooth.
Supraphon is closer to the strings, and that makes that bass spinning-wheel
theme all the more imposing; but when the violins go at it full blast, they are so bright 
that you blink Still, What is Dvořák without those thrilling Czech violins? 
They play so brilliantly, with a sort of frenzied elan. 
It's an effortless brilliance, like confident peasant dancing.
And the Czech Philharmonic certainly sounds more Czech---no surprise--especially the violins.
In Milwaukee they have to strain for whatever brilliance they attain.
In London they seem to lose body in the brilliant passages.

Kubelik has the Berlin Philharmonic, and they sound just great---especially the violins.
Dvořák was inclined to work his violins hard; they carry much of the music. 
DG (or MHS same recording) is inclined, as you know, to go light on the bass---to aim
for a ``clean'' sound with a lot of definition and detail. 
But the Kubelik is less that way than the Neumann; the bass is actually
pretty good, if not as good as London or Koss. It's certainly better than in
the average Karajan recording. And it was recorded at Jesus-Christus
Church---a great place to record. One trademark of the Berlin Philharmonic
sound is very strong timpani. 
In this first movement Kubelik is exciting and briliant, but he is missing Neumann's majesty.

{\bf II}. Neumann's conducting and orchestra
sound are very idiomatic. Supraphon's crisp recorded sound is attractive, 
but one can barely beat the warm, rich sound of the Koss. But Neumann shapes and phrases
more (or better) and makes the Macal and the Kertesz sound shapeless.
In a slow movement that is a liability. Macal sounds slower, though his tempo is perfectly
normal and even a little faster than Neumann's. Neumann just varies the pulse more. 
Kubelik has a very nasal ``Central European'' oboe---actually rather nice, and pretty rare these days.
Again Kubelik is exciting, and again he misses something---but here what he misses is very obvious:
the movement is labelled Molto Adagio. That means ``very slow''
and no one in his right mind would call what Kubelik does here slow---let alone very slow.
He may have decided that the movement needed help---and he may be right---but
the result is not what the composer called for. In this movement Neumann's conducting sweeps
the others aside. Again the Milwaukee winds are a joy---as good as any in any recording
(and Kertesz's and Kubelik's are delightful, too).

{\bf III}. London and Milwaukee are a little too refined here: the Czechs do some serious stomping. Their frenzied abandon carries the day, even at a slower tempo. 
There's nothing like it in any other recording. But, stomping
aside, the best scherzo musically comes from Berlin (Kubelik)---I just want to play it over and over.
Listen to that orchestra: the fullness of sound and the precise togetherness!

{\bf IV}. Again one marvels at the Berlin Philharmonic. They are simply great,
and they are part of what makes Kubelik's the most exciting performance.
Milwaukee is very noble and refined. One must admire their beautiful sound.
But Neumann is wild and unbuttoned and gets more exciting as it goes on---all
the way to the final glorious measures (which, by the way, fall rather flat
in the Kertesz but are better with Macal). The Suitner seems rather slow and dull
next to our choices, and the Rowicki is not well played or balanced (too much brass).
Järvi just isn't as revealing or exciting as the ones we list below.

For bright clarity and excitement and the quintessential Czech sound, turn to Neumann.
Moving toward more natural sound, Kubelik has every bit as much excitement---and that
glorious orchestra. Kertesz had very fine sound, if you can find it---but the LP was smoother.
The most beautiful sound of all is on the latest
recording: Macal on Koss---warm, natural concert-hall sound.

\st
Berlin/Kubelik & DG or MHS set\\
Milwaukee/Macal & Koss 1024\\
Czech Phil/Neumann & Sup 1003 [2CD]\\
London Sym/Kertesz & London NA
\et

\section{Symphony 2}
When this symphony became available on LP (Kertesz, 1967) Ray Minshull wrote glowing notes,
praising the work to the skies (``outstanding achievement...magnificent road sweep...an abundance
of vintage-type Dvořák themes...remind one of Tchaikovsky''). 
Now it's true that in those days prospective yers could read the liner notes
on the back the album, and maybe Mr Minshull was ``laying it on''
to sell an utterly unknown work. But those of us who fell for it were
sorely disappointed. It is a dull and boring symphony without any 
``broad sweep'' or ``Dvořák-type themes'' at all---and Tchaikovsky is the last composer
one would be reminded of. All of us agree that the Second is the weakest Dvořák symphony,
and most of us could happily live without it. It starts out rather nicely but
never goes anywhere; it rambles and drifts, and there are no memorable melodies
and none of the youthful vigor of \No1. 55 minutes is far too long for its meager material.

It was written in 1865, same year as the First; but this one he did hear
performed---many years later, in 1888---and by that time he had revised it and
shortened it (but it's still too long). It was not published until 1959.

Carl Bauman likes it better than the rest of us, and he recommends the loving account of Kertesz
and the superior Kubelik. He calls the Neumann ``flaccid'' and ``a disaster'' , but both
he and Steven Haller find the Järvi a decent third choice.
The Gunzenhauser (Naxos) is in that same category. 
Kurt Moses reviewed the Pešek (July/Aug 1996) and seemed quite bored by it. The sound is terrific.

\st
Berlin/Kubelik & DG set or MHS 564673\\
London Sym/Kertesz & London NA
\et


\section{Symphony 3}

This one came eight years after the first two, and it was performed: Simetana led
the Prague Philharmonic. It's the only Dvořák symphony with only three movements. 
The middle one is his longest slow movement. It's basically a theme and variations,
but the theme gives birth to themelets. In the middle of it a catchy march-like
section in pure Dvořák rhythm (dum, dum, dum-da-DUM-dum) acts as a scherzo.
It builds up, then sub-sides; but it returns slower and quieter in the coda
of the movement. Dvořák still hasn't written a good slow movement, 
but there is a sort of genius in the way this is organized-and you'll never forget that march.

The opening movement has wonderful flow and warmth; you are hooked right from the start. And the finale has an excitement and drive that will remind you of Schubert's Ninth.
Dvořák has Schubert's light-hearted playful-ness, too. Still, the orchestration of this symphony is sumptuous, almost Wagnerian. But, Schubert and Wagner aside, Dvořák always sounds only like Dvořák---he could never be mistaken for anyone else. And he loved this symphony all his life. Brahms loved it, too, and thought it worthy of the Austrian State Prize.
(The critic Hanslick was also on the committee, and Dvořák got a five-year government grant out of it.)

Järvi is much smoother than Kubelik. He has a natural flow and is both confident and serene. Kubelik is frantic, raucous, blasty---out to prove something---except in II. Järvi is faster, except in III. 
In II Kubelik constantly manipulates the tempos, fusses all the time, 
and loses the pulse of the march (it's deadly) and the spirit of the piece. 
Järvi lets the music carry him along. The Adagio is bland and has no character, until the march breaks in.
Kubelik tries to give it character but fails. Järvi plows thru it almost two minutes
faster to get to the march---because that's where the movement comes alive.
Kubelik takes 5 minutes for the march, playing around with its tempos to get more
expression out of it. Again it sounds manipulated. Järvi takes almost two minutes less,
never letting up on its forward motion.
I'm with Järvi. Once the march is essentially
over, the movement never quite falls back to its bland beginning; the march has given life to the other themes too! The Berlin Philharmonic saves the day again and again. The strings especially are way ahead of any other orchestra that has recorded this. But up to this point Kubelik has not won our hearts the way Dvořák should.

Kubelik drives things on to an exciting conclusion, but it still sounds manipulated.
Symphony 3 is Kubelik's worst recording. Järvi is again utterly natural, the music finding a more comfortable pace but building almost inexorably to a thrilling conclusion.
Those two are the contrasting recordings.
Kertesz is much closer to Järvi than Kubelik and could also be considered close to ideal.
Pešek on Virgin is a little on the slow side but very natural. The relaxed, leisurely pace allows for some lovely turns of phrase; and the sound is warm, creamy, and rich. Neumann is middle-of-the-road---a little rough in I---but has harsh sound. The Naxos sound for Gunzenhauser is the opposite: too mellow and blended. The music needs an edge somewhere, and we should hear the woodwinds.

The Koss (Macal) is also rather mellow; but it's closer-up than the Järvi and without the reverberation. I rather like the atmospheric Chandos sound, but if you dislike echo you'll prefer Macal. In some climaxes everything gets jumbled together---but I like that; one often hears it like that in real life. The conducting is good in both cases, but Järvi is inclined to make more of the climaxes and more inclined to majesty. Macal takes the slow movement much slower and the last movement faster.
That makes it seem less gentle and attractive a little brusque, I think. But Milwaukee is a better orchestra than the Scots in every department (both could use more strings), and the Koss engineers supply more warmth and clarity and less echo. For my money no one comes near Järvi in 3. Rowicki is rather general and vague, Suitner even more so.

\st
Scots/Järvi & Chandos 8575\\
Milwaukee/Macal & Koss 1019\\
London Sym/Kertesz & London set (NA)
\et

\section{Symphony 4}

This is less smooth and flowing than \No3. It has two striking Dvořák rhythms
and two gorgeous flowing melodies. You get one of each in I; the rhythm is TA-ta-tum-te-DUM. II is a long theme with variations---very Wagnerian; it is not very original, but it's his best slow movement so far. Ill is a swaggering scherzo-village music. Then comes the repetitious IV, where the first theme is another Dvořák rhythm---rum-tum, RUM-ti-tum---heard 17 times---and the second one of his trademark flowing melodies. The overall
impression when you've heard it all is of exuberant grandeur---and maybe that's a good description of Dvořák's music in general.

The Kertesz stood alone for years as the only really good recording. It's quite bracing and will always be worth hearing. At this stage Kertesz's scherzo is still the most exciting. The Kubelik was no particular threat: Carl Bauman calls it turgid; I call it jumpy: the scherzo seems especially lumpy, and the second theme of the finale is especially jittery---has no repose. There is more bounce and surge and bump than flow in Kubelik's Fourth. The Järvi is no threat either---soggy and boring.

But after 25 years at the top of the list the Kertesz was finally surpassed by the Pešek. The gorgeous, velvety sound---especially the strings---is a large part of its superiority. The orchestra is the Czech Philharmonic, in their home hall, and their playing and the place add a great deal to the picture. They respond to Pešek with much richer tone and much more fire than they were giving Neumann the last few years of his tenure. The approach is smoother and tempos are broader than Kertesz, but I is not as slow as Järvi. Musical opulence and majesty! When you hear a recording this good, you thank God for digital.

A year or two later Koss brought out the Macal, and now there are three great Fourths---Kertesz, Macal, and Pešek---but the greatest of these is Pešek. Macal is much smoother and warmer sounding than the rather stark and edgy Kertesz (which vas miked pretty close-up) though the tempos are about the same in I and III.
A slower II makes it seem more Wagnerian than the Kertesz does, and a slower IV makes little or no difference.
The Milwaukee players of today are as good as the London Symphony people of the late 1960s, and the Koss recording is more natural---more ``concert-hall''.

\st
Czech Phil/Pešek& Virgin 59016\\
Milwaukee/Macal&Koss 1015\\
London Sym/Kertesz & London NA
\et

\section{Symphony 5}

Dvořák seems to have simplified his style
a bit for this one. The opening is very pastoral---the woods, the fields, the birds. It
could easily be by Smetana (Bohemia's Forests and Meadows). The opening clarinet theme is light and cheery. The theme of innocent enjoyment of nature seems to carry thru the whole of I, including an ingenious and serene coda. II has a lovely cello melody. III is bursting with song and wit---a typical Dvořák scherzo. IV involves a change of pace and key, but a good conductor can make it seem a natural outgrowth of what has come before. It's quite stirring, and the whole symphony seems to have been building up to it, when you look back. It's
no longer a pastoral symphony by the time you reach the last movement.

The Kubelik is very exciting, and part of it is certainly the Berlin Philharmonic: listen to those strings! But I prefer the Scottish oboe (Järvi). There is no difference in tempos between Järvi and Kubelik---a matter of a few seconds here and there---and the Kertesz is pretty close, too. Three terrific conductors in great form. The Järvi is recorded further back in the hall; I like that; Tom Godell likes it less, and Steven Haller complains bitterly of echo-chamber sound. The Kubelik and Kertesz are closer up. Sometimes Kubelik startles you, because an orchestral outburst can seem so close at hand. Kertesz sounded smoother on LP than on CD. But I have to say that it's impossible to choose among these three. I have even come to accept Kubelik's I, which I objected to in the 1990 overview. The clarinet theme is so winsome, and the movement gets so exciting that I can't figure out what I once objected to---probably just that the other two conductors are a little smoother, more pastoral
and flowing.

The Neumann is rather dull. The sound is crisp, and the orchestra plays well; but you will miss the intensity and alertness, the contrasts and the coherence of a conductor like Järvi. In fact, Järvi has it down perfectly: he seems more Czech than Neumann. Pešek again comes thru here, with the same virtues as his 4th: glorious sound, great orchestra, natural Czech interpretation. Macal has the virtues and drawbacks you would expect. The Milwaukee Symphony plays very beautifully, but the strings are not as thrilling as the Czech or Berlin ones. The sound is as good as it gets (if you like it back-in-the-hall; Steven Haller prefers more of an edge and finds it muffled---review this issue).
Macal is choppier and more energetic than Järvi---or almost anyone else---but I prefer Dvořák smoother and more relaxed. IV is about as good as it gets anywhere. Koss's notes are not well written.

The old Rowicki is pretty good---one of his better Dvořáks. Mr Haldeman calls the ansons fresh and spontaneous, but others don't know it. Steven Haller liked the Naxos Gunzenhauser, but it is coupled
with a less-than-competitive 7th. Still, if you don't know the 5th, Naxos gives you a cheap way to hear a good performance of it.

\st
Scots/Järvi & Chandos 8552\\
Berlin/Kubelik & DG or MHS set\\
London Sym/Kertesz & London set (NA)\\
Czech Phil/Pešek  & Virgin 59522\\
Milwaukee/Macal & Koss 1026
\et

\section{Symphony 6}

This one was written for Hans Richter and the Vienna Phiharmonic in 1880, but the first performance took place in Prague, and it took the Vienna Philharmonic a long time to get around to it (though Richter conducted it often in other cities). It is in the same key as the Brahms Second, but its opening resembles the Brahms Fourth---and it's another brilliant opening that ``hooks'' the listener right away. II is sweet and gentle, and near the end we hear a hint of the quiet violin solo near the end of Also Sprach Zarathustra. The scherzo has the usual Dvořák rhythmic vigor, and the trio is serene.
The final movement is happy and untroubled.

For some reason, Kubelik takes longer over that last movement than anyone else. His tempos are varied more; some parts are faster than most conductors. The Berlin Philharmonic is simply unbeatable. The Kubelik remains a very exciting and dramatic recording---almost a lesson in great conducting: how to make the details count while building a whole structure that makes sense and reaches all the peaks. He has a lot of fire, but there are also a lot of drastic tempo shifts.

Some of us prefer the music less manipulated, with a more natural and dependable flow. For that it is impossible to beat Bělohlávek. He enriches the music with majesty and subtlety (listen to those strings!).
He also has the best sound of any-absolutely and without doubt. Kertesz is more emphatic than Bělohlávek---and brassier and brasher---but still softer and gentler than Kubelik. He also has better sound than Kubelik, but it is closer-up sound, and that may irritate some listeners. But I am very unhappy with his exposition repeat: four minutes into I he starts all over---and rather dashes thru it all the second time around. Why bother? Dvořák himself never took that repeat and never wanted it.

Libor Pešek, who impressed us so much in a number of the other symphonies, leaves us cold in this one. (Why is it that no one seems to do them all well?) The sound is a little distant this time, and the conductor seems to float thru the music---misses a lot of points that Kubelik and Kertesz facilitate. Those two conductors are more cumulative-know how to build a climax. They are also more coherent, more stirring, more heady-more of a total experience (I include Bělohlávek in this description). It's the same Czech Philharmonic that plays for Bělohlávek, and they sound very good, but Bělohlávek has figured this symphony out and Pešek has not.

Another Czech Philharmonic recording---with Ancerl---promises a lot but lets us down.
For some reason the conducting is brusque and routine, with very little shaping.
Neumann's 1982 recording---the one readily available right now---is quite beautiful.
The sound is not as appealing as the Bělohlávek,
unless you like it cleaner (as opposed to warmer), but those Czech Philharmonic violins are something special. It, like the rest of his series, is in a box with two other symphonies: 4 and 5.

Macal is pretty straightforward, with little tempo manipulation. I is strong and lite, tough but with sensitive detail---very appealing. II is very lovely and makes the Järvi sound insensitive. IV is as exciting as it gets. The Milwaukee Symphony again plays beautifully, and the sound is demonstration-class Jan/Feb 1990). The Cleveland Orchestra also sounds wonderful for Dohnanyi, and his II is just right. We decided that the conductor and the engineering overdramatize the music (Nov/ Dec 1991). The Rowicki---along with his 5---is excellent, but the sound is 1960s and not up to Kertesz, let alone Bělohlávek. The Suitner was reviewed as ``energetic but ordinary''. Slatkin has the slowest II and comes in an expensive box set available only from the St Louis Symphony. Tom Godell considers it the best 6th ever, and I remember it as one of the best. Another specific criticism of the Järvi is that his tempo in IV is cranked up too far---the orchestra has trouble with it, too Jan/Feb 1988---Haller). Beyond that, it just seems generally inferior to the ones we recommend.

\st
Czech Phil/Bělohlávek & Chandos 9170\\
Berlin/Kubelik  & DG or MHS set\\
London Sym/Kertesz & London NA\\
Milwaukee/Macal & Koss 1001\\
Czech Phil/Neumann & Supraphon 1005 [2CD]
\et

\section{Symphony 7}

The Seventh was written for the London Philharmonic in 1885, and  Dvořák conducted the first performance. 
It was an immediate success, and wherever it has been played ever since it has won many friends. Sir Donald Tovey placed it alongside the Brahms four as the greatest symphonies since Beethoven.
Other eminent critics have called it profusely inspired and a great symphony by any standard. Many think it is Dvořák's greatest.

It is also his most Brahmsian. It took him only three months to write it---not surprising, considering the level of inspiration. Some people hear hints of two great themes in it: the second theme in I resembles the beautiful cello solo in the Second Brahms Piano Concerto, and in II one spot sounds like the love duet from Tristan and Isolde. (To me it all sounds like Brahms, not Wagner.) The Scherzo trips along lightly and cheerfully; conductors are inclined to over-inflect it, but it's really a prime example of the natural flow of Dvořák's music. Yet there is passion in this symphony, and that can be slighted as well. It's not an easy work to get right.
Suitner's 7 is forceful, with sharp attacks
and little charm-sinewy and intense rather than graceful. Giulini on EMI is almost the other extreme: only mildly engaging, a run-through. (The Chicago Symphony has made available a far better Giulini 7th; see Nov/Dec 1996 and July/Aug 1994.) Giulini is a maddeningly inconsistent conductor. There was a Giulini on DG. It was better conducted than the EMI or Sony, but the sound was not great. The Sony Giulini (Concertgebouw) is very slow and ponderous, with no energy or pulse. Jansons has plenty of that, but he doesn't slow down for the appealing parts-goes flying right thru them---and III sounds half-hearted. Colin Davis is rather cold emotionally and gets rather cold sound, too. He achieves tremendous drive and clarity, but he refuses to linger over anything---and Dvořák gives us plenty to linger over.

The Marriner seems perverse: he rushes thru all the big tunes and slows down where there is no reason to. II is breathless, and the Minnesota Orchestra is decidedly mediocre.
Minnesota Orchestra is decidedly mediocre.
The Maazel was odd and not very well shaped.
John McKelvey was very impressed by the
Mata 7th (M/J 1988)---a Germanic approach,
with good flow, plenty of warmth, and a good recording. The rest of us don't know it.
Another one that impressed Mr McKelvey was the Inbal on Teldec---very well planned,
conducted, and recorded; very refined playing, with a dark Central European
sound (Sept/Oct 1992---available now only with 8 and 9). The sound is warm and 
appealing and the approach quite lyrical and songful. Good timpani.
Järvi's 7th was the first in his Dvořák series (1987), and Steven Haller raved about it.
It still stands up very well next to everything that's been 
done since, but for sheer beauty of sound it is not quite in the very top group.
Gunzenhauser on Naxos is warm and satisfying but somewhat low-key, a decent introduction
to the work if you are on a budget.
The Dorati is actually quite good, but the Mercury sound is somewhat tinny. Tempos are close to ideal. From the same period there was a Barbirolli that I find unbearable: the conductor is at a total loss in Dvořák. He is blatant, has no subtlety: everything is on the same obvious level. There is no flow, no cumulative effect. Everything is heavy-handed. The orchestra is terrible. It's a shame that EMI brought this piece of dreck to CD: the orchestra simply couldn't cope with the music.
Dozens of American orchestras play it better.
But EMI's recording will win no prizes either: hard, crude sounds too close-up---a
flute louder than a whole string section. The Kosler is also pretty old,
but it's a genial, well-inflected account.
Zdenek Macal again delivers excitement---in spades---and Koss delivers gorgeous sound.
II and IlI are the fastest around, and they sound driven. The music has been deprived of its natural flow, and this is an eccentric version--for people who like things fast. Another conductor who drives things too hard is Chung on BIS. The Scherzo sounds more like Beethoven than Dvořák, and the orchestra is nowhere near as good as Milwaukee. James Levine's 7th also seems hard-driven: excitement at the expense of the music's emotional range. What Monteux had was both---along with a very nice sound from the London Symphony. There is no particular need for the Monteux with so many fine---and better-sounding---recordings on the market, but Steven Haller is not the only ARG reviewer who reveres it. Nor do we really need the 1963 Bernstein, just reissued. It is exciting and passionate, but so are many far better---played and---recorded ones (actually, the sound is vastly improved on CD). The Ormandy was better recorded (RCA), but Ormandy was not good at Dvořák.

Andrew Davis does nothing with the music, and the sound is distant, cloudy, and screechy (all!). Another bad recording job is the EMI Sawallisch---but part of the problem is certainly the decline of the Philadelphia Orchestra: the violins are scrappy, scrawny, and tinny. There is no blend or warmth.

Among the best-sounding 7ths---along with Macal and Inbal---must be counted the Previn on Telarc. Has the Los Angeles Philharmonic ever sounded more beautiful? Dark, rich sound; strings are gorgeous, winds charming, brass great, and the bass sounds are all full and strong. Tempos are moderate---Brahmsian,
one would have to say---but the scherzo is fast (modelled after Monteux rather than Szell). On the whole, in Mr Haldeman's words, it is both majestic and sympathetic. Telarc seems to have deleted both issues of the Previn
(one was midprice), but some day they will bring this back.

Another one that could win a prize for sound is Bělohlávek. The Czech Philharmonic sounds a little more idiomatic than any other orchestra (strings and winds), and Chandos puts us back a bit in the hall compared to Virgin (Pešek) or Telarc. Bělohlávek is a more exciting conductor than Previn or Pešek: he makes them sound tame in this symphony.
But some of us need a gentler recording as well as an exciting one. The Neumann, with the same orchestra, has some charm but little vitality. The Pešek is actually with his Liverpool orchestra-not the Czech Philharmonic-but the strings are utterly gorgeous, the flow is perfect, and the rhapsodic flavor makes it feel like Rachmaninoff. (That's a real plus to me---romantic music should sound romantic---but others may find it too indulgent.)

Kubelik whips up far more excitement than Pešek or Previn---right from the first notes. Do you prefer emphatic, exciting conducting or smooth, serene conducting? Kubelik's DG sound is also a bit harsh, and certainly all our recommendations have far better sound.
The earliest recording worth having is the Szell (1960). You might not expect it, but it is very graceful. But Szell is also wild, heady, ecstatic---still the most exciting interpreter.
The sound is hissy, but the strings have an icy brilliance that suits Dvořák---and that few besides Szell were able to manage. The horns are very good, and fairly close miking helps many instruments stand out. Naturally, it is very well played. Dohnanyi with the same orchestra years later has none of Szell's charm; it's Germanic and square, with some drama, but flow and grandeur are foreign to this conductor. It might as well be Beethoven.

\st
Czech Phil/Bělohlávek & Chandos 9391\\
Liverpool Phil/Pešek & Virgin 59516\\
LA/Previn & Telarc NA\\
Cleveland/Szell & Sony 63151 [2CD]\\
Philharmonia/Inbal & Teldec 95497 [2CD]
\et

\section{Symphony 8}

After the Seventh, the Eighth represents a
simplification. It is looser formally---more rhapsodic---and it sounds almost improvised.
Melodies flow in profusion; there are almost
too many of them, and they are all delightful.
The plain and direct score explains why it communicates so readily. II is elegiac; III a flowing, lilting, waltz-like piece, serene, gracious, and buoyant---brimming with freshness and joy and ecstasy. IV is a series of variations in many different moods, but it's not tightly organized and seems more instinctive than planned (almost naive). You might say that it has all of Dvořák's Czech virtues without academic complications. Even Beecham was charmed by it---and he knew what charm was!

The Sawallisch is terrible (see 7 or July/Aug
1990), as is the Andrew Davis (see 7 or Jan/Feb
1991). The other Davis---Colin---is mediocre; even those among us who like his 7th don't like the 8th. Jansons fails to do anything with III, and that throws off his whole performance.
A brilliant finale cannot compensate. The same goes for Previn: nothing goes anywhere until the finale; it all just sits there. I is totally misconceived (more on Previn below).

Giulini has given us three 8ths. The one on DG was an abomination; the engineering was positively ugly---one of the ugliest discs I've ever heard. (Gramophone gave it an engineering prize---which goes to show how much their
It was the Chicago Symphony, and the violins sounded the way they did in Orchestra Hall before last year's renovation: all scratch and no tone. The EMI is much better,
but the London Philharmonic could use some of the drive the Chicago Symphony had. They are often too refined, and their brass are too reticent. Still, for a mellow, bucolic, laid-back 8th, Giulini is worth investigating. Even the Sony is that way, but Giulini plays around with the tempo more than any other conductor---he wipes out the natural flow of the music. The Sony has very nice sound and the Concertgebouw Orchestra (Nov/Dec 1991).

The genial Bruno Walter doesn't stand up well in this company. It lacks grandeur. II is tender enough, but it doesn't accumulate; it just plays itself out. There is no flow---and none of the panoramic quality Szell produces.
III comes off pretty drab, and IV is joyless.
Brasses blast and rhythms plod. Neumann (Mar/Apr 1987) is also pretty drab---III and IV especially could use more energy---and the Supraphon sound is somewhat raw.

Macal has again some of the nicest sound you'll ever hear---warm sound, with perspective and a sense of place. He is a little choppy in I---that is, it's not smooth, and the flow doesn't seem natural. He doesn't have the musicians hold on to notes long enough, and he tends to be too emphatic. But II is much
smoother. Tempos in I and III are the same as Szell. IV builds up considerable momentum and excitement. What makes this disc worth buying, though, is the superb sound (much richer bass than Szell) and the best Czech Suite on records as filler.

The Kubelik is almost the opposite of Walter. It has his usual tempo manipulations---you know by now whether you like them or they drive you crazy. One thing is sure: they allow no majestic unfolding; they disrupt the flow. But they can be very exciting, and Kubelik's 8 must be considered one of the two or three most exciting around---certainly on the level of Szell or Beecham. (The Beecham is not worth looking for if you have the Szell or Kubelik.) I wish DG had better sound; I certainly don't like the metallic violins in some passages (especially near the beginning), though their playing is ecstatic. Generally there's too much close-up glare. When the trumpet opens the last movement it's so harsh you jump out of your chair. The catalog lists two DG issues of the same recording, one with 9 and a two-disc set with 7, 9, and Smetana.
MHS has the complete set of 9.

Herbert von Karajan recorded this in Vienna (for London) and twice in Berlin (for
DG and EMI). Dvořák was never a Karajan specialty, and there are many better recordings.
The latest one was on DG, and it has no flow---it seems always agitated---and Karajan treats it like Beethoven. As you move back in time they get better. The EMI is around right now, and
Karajan fans should try it. Most of us find Abbado an empty conductor,
but Gerald Fox liked his 8th on Sony (May/June 1995).
Klaus Peter Flor delivered a faceless reading---never majestic, never graceful, never sensitive, often slow and deadly dull. Another placid or flaccid reading is the Previn on Telarc: we are dragged drugged thru three movements until things suddenly come alive at the end. Nothing before the last movement shows signs of thought or expression. Telarc gave the LA Philharmonic wonderful sound, but Previn blew the opportunity (deleted).
Ozawa is also pretty dull and pedestrian. Like Previn (in this work) he has no feel for grandeur; well, then he shouldn't be conducting Dvořák---or most of the romantics. There's no flow, no spontaneity, no romance.

The Bělohlávek is not as good as his other recordings. He seems to be driving the orchestra too hard, and there is little bucolic atmosphere. The sound is glassy and harsh---not like Chandos---and there is little blend.
Myung-Whun Chung on BIS is better in the 8th than in the 7th, and III is relaxed and quite beautiful; but the Gothenburg Symphony just doesn't measure up to other orchestras that have recorded this. Steven Haller liked the Marriner 8th (still available as Capriccio 10354) but felt that the brass held back at the final climax. Only one of us has heard the Suitner, but he liked it: Tom McClain compared it to Walter. The sound is rich and warm, the orchestral tone quite beautiful (Sept/Oct 1997). Another one reported on by only one of us is Levine (in Dresden): Philip Haldeman thinks it belongs up there with the Szell: dramatic and intense outer movements and smooth, affectionate inner ones (May/June 1997). John McKelvey thinks the same of the Barbirolli, but he admits that the sound is a little fierce and hard-edged (Nov/Dec 1992). In our recent listening sessions, we decided that the huge contrasts in tempo and dynamics impede the flow. He rushes I, eliminating grandeur. The strings are terrible in III, and the interpretation seems heartless. Next to Szell (our choice for comparison), Barbirolli is a non-starter.

Mr McKelvey is also an advocate for two fine Teldec recordings: Inbal and Masur. The Inbal is nicely recorded but not as good as his 7th in any respect. The sound is not as full, the cellos not as rich. Inbal doesn't have as many good ideas about the music, and he sounds weak next to Szell. The Masur sounds more beautiful and is better played and more eloquent (Sept/Oct 1994); but it is not as smooth and majestic as the Szell, and the tempos seem artificial, with unnecessary pauses, presumably for dramatic effect (especially in IV).
Masur applies an all-purpose sauce to it that doesn't work; he totally misses Dvořák's wonderful flow. II is slow and stop-and-start. III is a little too fast in the main part, but at 2 minutes in the New York strings sound really terrific. The sound is gorgeous in general, but the woodwinds are much closer to the microphones than the strings are, and that can really become disconcerting. The Previn has not been surpassed for sheer sound.

Steven Haller liked the Paita 8th (Sept/Oct 1990, deleted), but most of us found it much too fast. Antal Dorati's 8th is also too fast---especially the graceless Allegretto grazioso (III). Dorati was a choppy conductor, and Dvořák must flow. Some people like the Mercury sound; many of us do not. Järvi disrupts the flow to play around with details. His strings are inadequate---especially the all-important violins. They need to sound full and rich in so many places. One orchestra that did was the Ljubljana Symphony under Nanut.
Their terrific 8th may still be around on one of the really cheap labels. Warm, sometimes fiery strings, stunning and refined playing, and a perfect Dvořák sound. Only in I does Nanut 
seem to drive things too hard. The rest sounds just like Szell.

George Szell got the 8th exactly right---as he did the 7th. Once you've heard his way with these symphonies, you start comparing other conductors to him, usually to their detriment.
The Sony 8th sounds better than the 7th that comes with it, even though the 8th is older. It is hissy but still powerful in its impact. Szell's I is fire and ice---brilliance, passion, and flair.
Szell had a passion for clarity, so that no detail gets lost, nothing gets smudged or falls out of the overall design. The Cleveland strings sound bright and healthy---almost Czech---and this is string-dominated music. II is perfectly paced, and III carries us into ecstasy with the celestial beauty of the strings. No other recording has quite this much ecstasy, but Kubelik comes close (in inferior sound). John McKelvey is irritated by Szell's tempo for the coda of III, but according to the Eulenberg score it should be almost doubled. Also, Szell may seem faster than others because he slows down, per score, leading into the coda.
Kubelik, for one, does not. In IV Szell fires things up to a brilliant climax---as electrifying as Beecham but in far better sound. And Szell has built his way to that climax---prepared us for it---and that adds to its effect. There are two Szell recordings---Sony and EMI. The Sony has been around on Odyssey and is now reissued in a set with 7 and 9. The EMI always had warmer sound, but the tempos are all a little slower and the strings are less ecstatic. The difference does not seem vast, but we generally prefer the Sony on the grounds of more alert conducting. Both are great recordings. Note, however, that the EMI is part of a two-disc set with the Beethoven Emperor Concerto. All the Szells are mid-price or better, and we must urge you to be sure you have one of them in your library.

\st
Cleveland/Szell & Sony 63151 [2CD]\\
Cleveland/Szell & EMI 69509 [2CD] \\
Ljubljana/Nanut & ?\\
Berlin/Kubelik & DG or MHS set
\et

\section{Symphony 9}

Not many pieces of music have been recorded as often as this one. And there
are a lot of good recordings, too. After all, any full-time orchestra has 
played it a lot and knows it well---and that is not true of the other symphonies.
There are over 100 recordings of it available right now---many in numerous incarnations.
There are eight listings for Neumann and six for Nanut, though they seem to represent
five actual Neumann recordings and only one by Nanut. 
Needless to say, a lot of recordings are out of the catalog, too.
To help us narrow things down, we let the sound of the Toscanini, Keilberth,
Tennstedt, and Fricsay rule them out. The Minnesota Orchestra has no tonal allure,
so in spite of rather good conducting we cannot recommend Marriner.
We expected more from Karajan and Horenstein: they are not very special.
Mr Chakwin likes Horenstein's fire and energy,
but most of us hear nothing that distinguishes his from other recordings. Chesky's reproduction is superb.
Järvi is sodden and lead-footed. Dorati is not competitive anymore (May/June 1997).
None of the Kubeliks are very good. The Denon (Czech Philharmonic) is sloppily played and badly balanced. The Berlin one is full of his usual tempo manipulations (Carl Bauman calls it ``flexibility'') and is played much better, but the sound reeks of artificial reverberation, and there's no warmth at all.
The strings sound slightly metallic and have no body. Kubelik's approach is genial and pleasant but not stirring or moving.

There are terrible 9ths by Ozawa (but Philip Haldeman likes it) and Eschenbach, among others. There are also a host of perfectly reasonable but unremarkable and uncompetitive readings, and we needn't discuss all of them. In this issue we review the Frühbeck de Burgos. It is beautifully conceived, and this man proves again that he is a far greater conductor than today's big names. It is utterly romantic---the opposite of Toscanini or Paray---and much more sensitive than Fiedler
or Kertesz or Szell. The main reason this doesn't make our top ten is that it's a concert performance with distant miking and audience coughing. The playing is very beautiful.

The Ančerl is a fine recording. 
So is the Kempe (with Beecham's Royal Philharmonic); it has come and gone on a number of labels.
It's fresh and straightforward and much better recorded than the Szell. The Szell is a bit tight and inflexible---as Szell could be. There was another fine Kempe on London (LSO). Barbirolli's 9th is much better played than his 7th or 8th (both dreadful) but still too English: blunt and businesslike, with no subtleties or refinements, no dreamy or heart-stopping moments. The strings are never beautiful, and the English horn solo in II is very dull.
James Levine's 9th is coupled with a good 8th (DG, May/June 1997). It's a pretty straightforward interpretation in a rather cavernous acoustic. His older one (RCA, deleted) was more aggressive, emphatic, high-intensity; and the sound was pretty good.

Arthur Fiedler's (with the Boston Symphony) is a brilliant performance that really crackles in the last two movements, where he is one of the two or three fastest. The very good sound is missing only the deep bass we can hear in some recordings. The Boston wind soloists are wonderful. Sawallisch has a great orchestra,
too---Philadelphia---and the sound is warm, dark, almost tangible. Nice ambience, too. A smooth, sumptuous performance gorgeously recorded.

Kertesz, by contrast, is much more vivid and exciting, with powerful sound that comes with a slight hiss. The English horn soloist in Vienna doesn't have the tone of the one in Philadelphia. His sound is quavery, Oriental, folk-like. But Mr Chakwin calls this recording wonderful, and we agree. Listen to the timpani! The later LSO Kertesz is not as good, and you hear it right away.

Kondrashin had the same orchestra---Vienna---in 1979 instead of 1967. (It was issued in 1980.) The 1967 English horn player has either improved or been replaced. The sound is similar to the Sawallisch, with loads of ambience but stronger bass. The Sofienssaal has to be the world's greatest hall for recording. This all-digital disc is very ``classy'' in sound, in playing, and in interpretation. Kondrashin has his own very expressive (and impressive) approach; he's especially good at building tension, and he is never slow. Some may find him weak on sweetness and warmth---especially next to Walter or Masur. Another gorgeous-sounding recording is the Previn on Telarc (Mar/Apr 1991). It's a pretty conventional interpretation---not boring or bland, but rather good---but people will love it for its sound. 

Bruno Walter knows how to relax and bask
in the warmth, but he doesn't miss much of
the drama and excitement (maybe a little in the final climax)---a perfect specimen of balanced, healthy, brilliant, moving conducting that serves the music ideally and never calls attention to itself. There's also plenty of weight and majesty. His III is the slowest we know, but it works, and you hear so much more detail. The sound shows no signs of age except that 1959 was a vintage year. Vibrant.

Lawrence Hansen called the Pešek dubious and episodic---meandering and unfocussed---but had some kind words for Neumann, whom Mr Bauman also calls first-rate. The hushed beauty of II is a high point, but the whole performance is powerful, lovingly phrased, and tonally rich. This seems to apply to both the Supraphon and Denon performances (abour 10 years apart)---see Jan/Feb 1995. (Steven Haller didn't like the Neumann in Mar/Apr
1987.) Another fine Czech conductor is Bělohlávek, and his thoughtful 9th belongs among the top 10 (also Jan/Feb 1995).

Macal is very Czech (how did he get the staid old London Philharmonic to sound like that?) and downright thrilling in both sound and interpretation (Jan/Feb 1989---it was licensed to MHS at the time, but EMI has since brought it out [62006] and that may be available in some places). The sound walks a magic line between vivid and too close-up. The strings are not as rich as some. His later recording on Koss is not played quite as well and does not sound quite as good July/Aug 1991).
Another Czech recording is the Albrecht on Canyon (he, of course, is German). It is one of the fine recordings that don't quite make it into the very top tier (Nov/Dec 1997), and the label costs much more than full price. Karl Böhm is also Germanic---or Austrian, or better yet, Brahmsian---with very little temperament.
Other perfectly adequate recordings were by Ormandy, Muti, Klemperer, Giulini, and Mehta. The Slatkin is beautifully played and
recorded but barely scratches the emotional
surface of the music.

We never reviewed the Inbal on Telec, but we listened for this overview. Teldec engineering often seems to favor back-in-the-hall mikes for the strings and close-up ones for winds and brass and section solos (sometimes even violins or cellos, if they have the leading theme.
That's what you get in the Inbal 9th. The first entry of the cellos is thrilling; the brass are loud and raucous. The violins---usually off in the distance---don't sound very good when the spotlight is on them. It's exciting sound until you compare it to the other Teldec---Masur in New York. Every soloist is better in New York. Every section is better, too. As for conducting Inbal is exciting---very dramatic---but the central part of II is definitely too slow, and III is
rather ordinary. If there weren't the Masur to
compare it to, the Inbal would come across as one of the better recordings.

Kurt Masur's New York recording was reviewed in July/August 1992 by Tom Godell, who called it ``a New World for the ages that instantly takes its place as the best available.'' When an editor reads words like that,
he is suspicious. Doesn't Mr Godell know the Stokowski (not available at the time)? And has he heard our other Overview recommendations (many of the same in 1990 as here)? So I had to get the Masur for myself. The orchestra is quite something, and the soloists are unbeatable.
Teldec's rich, full sound lets you hear how warm and polished the strings are. Tempos are relaxed but not too slow, and the colors, clarity of texture, and balances are ideal. II is unsurpassed in its serene loveliness and poetic power. Tom Godell told the truth; this goes on the select list of the best ever. It may very well be the best---best played, best conducted, best
recorded.

The CBS Bernstein with the same orchestra certainly has temperament and emotion, energy and enthusiasm; it still sounds good, too, if you can find it. Its biggest fault was a rushed
scherzo. The DG Bernstein will seem distorted to most listeners; II is quite distended at over 18 minutes (normal is more like 12). It's one thing to milk the music; it's something else to turn it to cheese. Bernstein plays around with every phrase can't leave anything alone. The first movement is so slow it never seems to begin. Another slow performance---downright sluggish---is the newest Giulini (Sony; Nov/ Dec 1994). Giulini doesn't produce quite the abomination Bernstein does, but Steven Haller describes it as a study in torpor.

Mr Haller still thinks the Reiner ``supreme'' along with the Paray. Many of us can't stand the cold Mercury sound of the Paray; it's brass-heavy, and the strings are squeaky. It is airless---never expansive. The timpani is strong. To some extent you come to accept Paray's tempos, but they are the fastest I know. II is very nice, even though it's two minutes faster than anybody else. The English horn soloist is good. But in the final analysis the Paray is a ``so-what'' performance: it's just not moving, and it cannot be at those tempos. Carl Bauman thinks a lot of the Reiner, too---along with the Szell. Szell has the same timings as Barbirolli, but the Cleveland Orchestra makes those Halle chaps sound like rank amateurs. Still, Szell is very little interested in beautiful sounds, and his strings can be pretty blunt. Szell doesn't give us a beautiful performance---he is not satisfying to the senses---but he does make it exciting. II is nice; III takes a minute longer than Paray's and is far more digestible. IV is exciting, but here as elsewhere there's no
``float'' to the music: Szell is too earthbound.
Reiner can be fiery, but he is utterly insensitive here. Lawrence Hansen says I am being too hard on Reiner: it is tight, powerful, dramatic, even driven, but not nervous and spastic like Toscanini (none of us can stand the Toscanini). Nor does it have the ``ferocious clarity'' of Szell. The Editor finds all three too fast and too cold---and the Editor is not alone. If you like the music tight and driven, one of these three may be for you---and Szell is probably the best of the type.

For years I (the Editor) have thought Stokowski the best interpreter of this symphony.
His 1973 Philharmonia recording is only available right now in a boxed set, but surely RCA will reissue it by itself. It has rich sound, loving phrasing, and gobs of emotion. Yet it moves right along. It's the full romantic treatment without any distortion. Stokowski was a master of orchestral colors, and you will hear many remarkable details no one else brings out. Yet the blend is perfect, and the strings dominate with gorgeous tone. A great recording.

So we recommend ten great recordings that make up the top tier. I have seven of them in my library. To hold on to so many is very unusual for me, but I really cannot choose: I can't think of giving up a single one. Each one makes the music live anew for me---I remark to myself what wonderful music it is---and each approaches it slightly differently. I can live without Sawallisch because a couple of the others are similar, but if I didn't have them, I'd have to have him. And I'd buy the Neumann and Bělohlávek if I didn't already have seven recordings!
Here are the best recordings of each decade.
All are stereo; all have excellent sound:

\begin{center}
\begin{tabular}{l l >{\raggedright\arraybackslash}p{2in}}
1950s: & Walter & Sony 64484\\
1960s:& Vienna Phil/Kertesz & London 417678\\
1970s:& Philharmonia/Stokowski &RCA 62601\\
1980s:& Vienna Phil/Kondrashin & London 430702\\
1990s: &  New York/Masur & Teldec 73244
\end{tabular}
\end{center}

\noindent
To those we must add:

\begin{center}
\begin{tabular}{l l >{\raggedright\arraybackslash}p{2in}}
1970s: &  Boston/Fiedler & RCA 6530\\
1980s: &  London Phil/Macal & EMI NA\\
1980s: & Czech Phil/Neumann & Supraphon (many)\\
&& or Denon 75968\\
1990s: &  Philadelphia/Sawallisch & EMI 69804\\
1990s: & Czech Phil/Bělohlávek & Supraphon 1987
\end{tabular}
\end{center}

\section{Historic Recordings}

Tom Godell dealt with the Talich recordings of symphonies 6-8 in the Historic Conductors Overview in May/June 1998. His are among the most important Dvorak performances in the
catalog. After all, his connection with Dvorak went back to Dvorak's time. (Dvorak recommended him for admission to the Prague Conservatory in 1897.) I agree with his comments on the Talich 6 and 7 and 8 on Koch.
These are indispensable for the collector of older material. I also wouldn't be without the wonderfully propulsive Talich 8th on Supraphon.

Another Czech conductor just as important as Talich was Karel Sejna. His slightly more recent-and thus better recorded-Supraphon disc of Symphony 5, coupled with one of the finest recordings of the three Slavonic Rhapsodies, is nearly indispensable---as are his versions of 6 and 7 in the same series. Tempos are never rushed, and the ebb and flow of the music is wonderful.
I hesitate to mention it because the 1944 and 1946 Prague Radio transcriptions sound pretty bad, but there was a Multisonic disc of true historic importance. Rafael Kubelik conducts Symphony 8 and the Piano Concerto with Rudolf Firkusny and the Czech Philharmonic---young musicians in music they loved all their lives. The Czech Philharmonic was in top form.

The {\it New World} is over-recorded, but the Talich is one of the most important. It is a vital, typically Czech interpretation. I cannot resist mentioning the pre-World War II recording by George Szell with the Czech Philharmonic. It has much of the tension and drama of the conductor's later Cleveland recordings but also a
I cannot resist mentioning the pre-World War II recording by George Szell with the Czech Philharmonic. It has much of the tension and drama of the conductor's later Cleveland recordings but also a youthful freshness. Dutton has managed a superb transfer, and the coupling is one of the truly legendary recordings: Szell with Casals in the Cello Concerto (it has never been out of the catalogs in over 60 years). EMI and Pearl also currently list it, but the Dutton transfer and discmate are better.

Not many people know that Sir Thomas Beecham and the London Philharmonic recorded the Fifth Symphony, long before it was commonly heard. It surely must have led many to realize just how wonderful early Dvorak is.

One other disc deserves mention, taped at the Prague Spring Festival in 1951 and 1956.
Andre Navarra is the cellist in the Cello Concerto with the much under-appreciated Frantisek Stupka conducting. Also included is a marvelous performance by the Czech violinist Vasa Prihoda with Jaroslav Krombholc in the Violin Concerto. The sound isn't particularly good,
but the performances are worth having. The Prague Radio Symphony accompanies (was on Multisonic---a label that seems to have vanished).

The {\it Slavonic Dances} have also been over-recorded. At least one Czech conductor has recorded them each decade for almost 70 years.
Talich recorded them twice. His 1935 set is on Music and Arts, and his 1950 edition is on Supraphon. That is probably the most outstanding recording of them. It glows with a passion and nobility that few others have. Also worthy of mention is a just-reissued recording by Otakar Jeremias on Dante-Lys. It was made in 1943 during the wartime occupation of Czechoslovakia by Germany. Jeremias was the permanent conductor of the Prague Radio Or-chestra. Rarely have I heard this music conducted with such high voltage. Tempos are almost rushed, but there is a defiance and excitement in the music that makes one wonder it this wasn't a case of the Czech musicians using their national music to thumb their collective noses at the Nazis. I couldn't recommend it as an only recording, but it is definitely unique.

\section{Sets}

Complete sets are a marketing ploy. There is no reason on earth why any musician should record all of any composer. Each conductor should record the symphonies he loves. If the industry operated by that rule we would all have an easier time deciding what to buy. I have never met a conductor who felt an affinity for all nine Dvoraks---or even all nine Beethovens.
Nor is it easy to find one who can resist temptation when asked to record them all!
We do not recommend that you buy a complete set, but if you insist, you must also buy a few individual discs to make up for the weaknesses in the set. Kubelik's set dates from 1966-73 (No. 8 is the oldest, No. 1 the most recent). No. 3 is terrible, and No. 4 is noi among the best-and, naturally, there are better New Worlds and 7ths and 8ths (though those two are pretty good). As for the Kertesz set, there are better recordings of 1,7,8, and 9.
Only a few of Järvi and Neumann rank among the best.
[While I was working on the symphonies, Lawrence Hansen and Tom Godell spent a few days in Cincinnati, and we listened together.
Many of the above observations are the results of our discussions.-Ed]

\section{Symphonic Poems}

Of the five symphonic poems, four are based on the lurid tales of Karel Jaromir Erben, which are even more gruesome than the Brothers Grimm, reveling in morbid accounts of butchery, body mutilation, and other grisly goings-on. In {\it The Midday Witch} a mother threatens her mischievous child with tales of a malevolent hobgoblin, who promptly appears before her horrified eyes to claim the boy; the father returns home to find the child dead---crushed beneath the mother, who has collapsed in fright. In {\it The Wood Dove} we have the all-too-easily-consoled widow who despite her lamentations is quite prepared to marry again until the plaintive sound of the Wood Dove fills her heart with remorse, since it was she who did her husband in. {\it The Water Goblin} is a particularly ominous goblin who seizes a young maiden and takes her with him to the bottom of the lake; when she tricks him into letting her see her mother one last time, he relents on the condition that she leave their baby behind; but when she reneges on the deal and refuses to return, the {\it Watersprite} has his revenge-the girl answers a knock on the door to find the baby's headless body on the porch. Only {\it The Golden Spinning Wheel} might be said to have a happy ending: an old woman and her daughter hack up the beautiful stepdaughter and pull the old switcheroo once they learn the king seeks her hand in marriage, but the Spinning Wheel---apparently equipped with a sound card---tells the king what happened, and with the help of a mysterious old man the girl is put back together and marries the king after all.

No such morbid content informs Dvorak's last tone poem, {\it Heldenlied} (Hero's Song); though vaguely programmatic, contrasting the Hero's initial despair (or is it the composer's?) with his eventual victory and triumph, we are not given a blow-by-blow analysis as with Richard Strauss's {\it Heldenleben} written a couple
years later.

If you want all five tone poems together, your best bet is the Chandos set with Järvi
(8798), pulling together the fillers from the symphonies; in general they're more sumptuously recorded than the symphonies, and it's good to have them together in one box (coupled with the overture {\it My Home}). While Lawrence Hansen finds Järvi's readings ``a little too hard and slick and homogenized at times'', Mr Haller feels the combination of Järvi's compelling narrative, gorgeous sound, and the convenient boxed set makes this the one to buy---especially if you want {\it Hero's} Song too.

Supraphon brought out all five tone poems with the Czech Philharmonic under Bohumil
Gregor (2196); but the {\it Hero's Song} was the only reason to buy the set, since Gregor's slick, superficial readings rarely delved beneath the bright melodic exterior of the moody, sinister
Erben fables. The same recording was on Supraphon 0378, coupled with the {\it Slavonic Rhapsodies}, {\it My Home}, {\it Symphonic Variations}, and {\it Scherzo Capriccioso}. Don Vroon liked Gregor's {\it Hero's Song} a lot more than Järvi's, which he called unidiomatic. Macal seems too fast.

Rafael Kubelik never recorded {\it Hero's Song}; but his account of the other four pieces for DG (435 074) is hard to beat, extracting the very essence of the composer's dramatic (and sometimes long-winded) narrative in sonics that---if not as upfront and vital as Chandos's---are still clear and detailed to a fault. On Naxos 550598, Stephen Gunzenhauser offers terse, well-characterized portraits of the {\it Wood Dove}, {\it Midday Witch} and {\it Spinning Wheel}; sonics are warm and enveloping, and performances are first rate. Lawrence Hansen likes Vernon Handley's recording of the {\it Wood Dove}, {\it Watersprite} and {\it Spinning Wheel} better than Järvi's: Handley seems more sincere and expressive and allows the music to flow more naturally---though occasional shifts in sonic perspective prove distracting.

Vaclav Neumann on Supraphon 0199 can be recommended as a quick-fix way of obtaining all four Erben-inspired pieces on one well-filled disc; while much of the atmosphere of the music still manages to come across, the recording is cramped and cavernous. Compared to Järvi and Kubelik, Zdenek Chalabala's vintage recordings are rather raw sounding---especially the winds, with plangent (occasionally shrill) brasses; but even such sonic shortcomings do not detract measurably from glorious performances that demonstrate a singing quality and flexibility of phrasing perhaps possible only with a conductor and ensemble (the Czech Philharmonic, of course) thoroughly familiar with the music. Chalabala was a wonderful interpreter of Dvorak and Smetana, and these
are marvelously idiomatic performances. They were originally issued on Urania but are now on Supraphon 3056. The classic Talich recordings from 1949-51 are available on Supraphon---not the same as the 1954 concert performances of {\it Watersprite}, {\it Midday Witch} and {\it Wood Dove} on Panton (both good).
Istvan Kertesz's superb London recordings of three of the tone poems have long been available only as Decca Ovation imports, the {\it Spinning Wheel} and {\it Watersprite} coupled with the {\it Scherzo Capriccioso} and {\it Othello Overture} on 425 060 and the {\it Midday Witch} coupled with the {\it Symphonic Variations}, {\it Hussites} and {\it Serenade for Winds} on 425 061. These have now been put together on a "Double Decca", replacing the {\it Wind Serenade} with three overtures: {\it Carnival}, {\it Nature's Realm}, and {\it My Home}.
Watch for a US issue on London.

There are many individual versions of the symphonic poems, usually doled out as fillers with the symphonies. Chief among these are the Järvi: Hero's Song is with 1, {\it Watersprite} with 5, {\it Midday Witch} with 6, {\it Spinning Wheel} with 7, and {\it Wood Dove} with 8. Among the best with Macal's Milwaukee series of symphonies are {\it The Noon Witch} (with 4) and {\it The Golden Spinning Wheel} (with 5, reviewed this issue).
Both are absorbing and have excellent sound.
Inbal includes a strong and joyous {\it Wood Dove} with his Teldec set of symphonies 7,8, and 9.
{\it Midday Witch} was included with the Eighth Symphony by Abbado on Sony (64303); Gerald Fox thought it was as good a performance as any he had heard, emphasizing the musical qualities over the gruesome story line. Seiji Ozawa's Witch is with 8 on a not very well filled Philips disc (434 990), but it is atmospheric, and sound is clear and detailed.

Carl Bauman was disappointed with Belohlavek's {\it Spinning Wheel} on Chandos
(9048), coupled with 8; he found it lacking in elegance next to Beecham, Talich, Kubelik, or Chalabala---the recording glassy and harsh.

Mr Haller likes Peter Tiboris's account of {\it Watersprite} very much---at once broadly expressive and sharply characterized---though the remainder of the disc is a rather mixed bag (Elysium 701). The Editor considers both Belohlavek's Watersprite and the Seventh Symphony coupled with it on Chandos 9391 as good as they get: excellent performances in superb sound. Koss (1010) has straitjacketed Zdenek Macal's Water Goblin in diffuse, gray sonics (but other reviewers like it). Naxos had Antoni Wit do it in combination with the Piano Concerto (550896)---no doubt as complement to Stephen Gunzenhauser's account of the other three noted above—and he did it well enough to get top marks from Vroon, who considers it the best there is at any price, gloriously played and recorded.

Mr Vroon had high praise for Belohlavek's Wood Dove, as well as the accompanying Sixth Symphony (Chandos 9170); indeed he liked it even more than the Kubelik, praising Belohlavek's expansive treatment of the main theme as well as the thrilling sound. Mr Vroon finds the Macal recording for Koss (1009) too hard-driven---``smooth but tense''---despite splendid sound.

Haller had high praise for the Naxos CD of {\it Hero's Song} conducted by Antoni Wit, coupled with the Czech Suite, Hussites, and Festival March (553005); it's a gloriously festive performance with richly textured sound to match.

\section{Symphonic Overtures}

The {\it Carnival} Overture, one of Dvorak's best-loved works, is the third of three overtures grouped together under the title ``Nature, Life, and Love''; the other two are {\it In Nature's Realm} and {\it Othello}. These are almost more akin to tone poems than overtures; indeed, one writer has called them ``a symphonic cycle consisting of a pastorale, scherzo, and finale''. It is gratifying that so many conductors have recorded the entire cycle, because the familiar leitmotif introduced by the clarinet in all three overtures---described by Karl Schumann as ``a symbol of nature as the basis of all life''---can best be appreciated when it is heard as the unifying theme of all three, not just a pleasant melody encountered along the way.

Gerd Albrecht recorded the triptych for Supraphon, coupled with the Brahms First
(1830); Paul Althouse called {\it Carnival} in particular a wild listening experience. Karel Ančerl's account for Supraphon, coupled with {\it My Home} and {\it Hussites} (0605), was praised by Haller as idiomatic and compelling, the sonics crisp and clean. Bohumil Gregor is even drabber in the overtures than in the tone poems.
On Naxos 550600, Stephen Gunzenhauser couples all three with {\it My Home} and the even rarer {\it Vanda}; they are well worth hearing, despite marked echo, and demonstrate Gunzenhauser's love for the music. Vernon Handley on Chandos 8453 seems more at home in the less familiar {\it Nature's Realm} and {\it Othello} than in the far better known {\it Carnival}; he seems to hold back, and the cavernous sonics don't help matters any. In contrast, on ASV 794 John Farrer has {\it Carnival } fairly explode into your living room, and {\it In Nature's Realm} is one of the better ones as well, but his {\it Othello} doesn't capture the somber mood as well as some others. Kubelik on DG 435 074, coupled with the tone poems, is hard to beat in this repertory, and his hyper {\it Hussites} even outdoes Kertesz for sheer visceral excitement. He offers a fine {\it My Home} as well.

{\it Carnival} turns up pretty often on its own; there are far too many to list, but pride of place would surely go to Bernstein (Sony), Szell (Sony-just reissued with Symphonies 7-9), Dorati (Mercury), and Reiner (RCA). In addi-Carl Bauman would recommend Belohlavek (Supraphon 1987, with 9) and Pešek on Virgin (with 3). Vroon praises Järvi (Chandos, with 3). He finds no particular advantage to sound or performance in Libor Pešek's Virgin CD of {\it In Nature's Realm} (with
6), though it's impressive enough if you don't know Kertesz or Kubelik. Mr Haller recommends Charles Gerhardt on Chesky 108 (part of "Light Classics", Volume 2. See tone poems for information on a possible Kertesz reissue---it would top them all.)

\section{Slavonic Dances}

Dvorak's publisher requested these, hoping to build on the tremendous sales he had had of the Brahms Hungarian Dances, played at the time by every pianist in Europe. The first set of eight (Op. 46) were mostly Bohemian (only \No2 was not: it was of Serbian origin). In the second set (Op. 72) Dvorak became more cosmopolitan, with dances from Poland and Russia and Slovakia. But perhaps by then the inspiration was wearing thin: the first set is more appealing.

Starting with A, we called the Altrichter inflexible and perfunctory. His lumbering tread changes several numbers into marches (Nov/Dec 1997). Neumann found more warmth and humanity in these pieces, even with faster tempos. And Neumann leads with delightful nuances and tempo fluctuations.
The latest Neumann is on Canyon-pretty expensive-but it's four minutes slower overall than the 1985 Supraphon, and Carl Bauman says it's worth the extra money (Nov/Dec
1994). The rest of us are happy enough with the Supraphon, and Mr Bauman agrees that
it's excellent.

Karel Sejna was a generation earlier than
Neumann and recorded these in 1959; the bass is weak, but the overall sound is pretty good.
Carl Bauman freely mentions Sejna in the same breath with Talich and Ancerl. And Lawrence Hansen calls the Sejna recording colorful, idiomatic, and exciting. Another Czech conductor who found this music natural was Zdenek Kosler. He holds his own against the heady Neumann, the more elegant Kubelik (just reissued by DG), and the manic but tightly-focussed Szell. As Lawrence Hansen told us, these are rip-snorting performances full of gusto and flair, with a big, gutsy orchestral sound. They have a lot of energy, vitality, joy, and freedom.

The Segerstam on BIS has good sound---especially bass---and some original ideas. Fasts seem faster than usual and slows slower. Still, the orchestra isn't world-class. We found the
Maazel boring (July/Aug 1989 covers that and four others). There is plenty of vigor in the recording Dorati made in his 80th year (Vox-Mar/Apr 1998), and readers who like that conductor will respond to his two earlier recordings as well; but most of us wouldn't list Dorati among the top recommendations.
The last two listings below are the piano duo original, before Dvorak orchestrated them.
Thorson and Thurber relish every piece and every mood; they enjoy the rhythms and the melodies, and they help us to enjoy them more than any other team does. The other great piano recording was by the incomparable Duo Crommelynck on Claves (Nov/Dec 1991).

\st
Neumann & Supraphon 1959\\
Neumann & Canyon 3615\\
Kosler & Naxos 550143\\
Sejna & Supraphon 1916\\
Thorson \& Thurber & Olympia 362\\
Duo Crommelynck & Claves 910
\et

\section{Legends}

The 10 Legends of Opus 59 (March 1881) were Dvorak's first attempt to capitalize on the great success of the first set of Slavonic Dances. In fact, 1, 3, 5, and 9 are Slavonic dances in all but name. They were written for piano duet, but Dvorak himself orchestrated them right away.
Today they sound better orchestrated, though there are fine piano versions-including the listing below (elegant, graceful, relaxed).
Kubelik and Sejna are old classics; the Leppard (Philips) was for some time in the LP era the best available. Järvi is pretty good. But if you can afford it, the Canyon recording is gorgeous. Naxos needs to issue the excellent Gunzenhauser on a single disc (it is spread over two as fill).

\st
Czech Phil/Albrecht & Canyon 293\\
Thorson and Thurber & Olympia 363
\et

\section{Slavonic Rhapsodies}

These are among the many Dvorak works looked down upon by the musicologists. Only No. 3 has been considered respectable. But all three are delicious, and the Editor's favorite is No. 1. Its best recording was by Neumann, but that hasn't been around in years. You should enjoy all three, and the Naxos is the one to have, unless you can find the Gregor on Supraphon---either deleted or not imported any more, it would seem. The Centaur (Burke, with American Suite) is tamer than the Naxos but still quite Czech.

\st
Kosler & Naxos 550610
\et

\section{American Suite}

This started out as piano music in 1894 (Opus
98), and it was composed purely for pleasure.

\end{document}
