% !TEX root = quantum-black-holes.tex
\usepackage[papersize={6.6in, 10.0in}, left=.5in, right=.5in, top=.6in, bottom=.9in]{geometry}
\linespread{1.05}
%\sloppy
\raggedbottom
\pagestyle{plain}
\usepackage{mathpartir}
\usepackage{stmaryrd}
\usepackage{mathtools}
\usepackage{tikz-cd}
\usepackage{microtype}
%\usepackage{fdsymbol}

% these include amsmath and that can cause trouble in older docs.
\makeatletter
\@ifpackageloaded{amsmath}{}{\RequirePackage{amsmath}}

\DeclareFontFamily{U}  {cmex}{}
\DeclareSymbolFont{Csymbols}       {U}  {cmex}{m}{n}
\DeclareFontShape{U}{cmex}{m}{n}{
    <-6>  cmex5
   <6-7>  cmex6
   <7-8>  cmex6
   <8-9>  cmex7
   <9-10> cmex8
  <10-12> cmex9
  <12->   cmex10}{}

\def\Set@Mn@Sym#1{\@tempcnta #1\relax}
\def\Next@Mn@Sym{\advance\@tempcnta 1\relax}
\def\Prev@Mn@Sym{\advance\@tempcnta-1\relax}
\def\@Decl@Mn@Sym#1#2#3#4{\DeclareMathSymbol{#2}{#3}{#4}{#1}}
\def\Decl@Mn@Sym#1#2#3{%
  \if\relax\noexpand#1%
    \let#1\undefined
  \fi
  \expandafter\@Decl@Mn@Sym\expandafter{\the\@tempcnta}{#1}{#3}{#2}%
  \Next@Mn@Sym}
\def\Decl@Mn@Alias#1#2#3{\Prev@Mn@Sym\Decl@Mn@Sym{#1}{#2}{#3}}
\let\Decl@Mn@Char\Decl@Mn@Sym
\def\Decl@Mn@Op#1#2#3{\def#1{\DOTSB#3\slimits@}}
\def\Decl@Mn@Int#1#2#3{\def#1{\DOTSI#3\ilimits@}}

\let\sum\undefined
\DeclareMathSymbol{\tsum}{\mathop}{Csymbols}{"50}
\DeclareMathSymbol{\dsum}{\mathop}{Csymbols}{"51}

\Decl@Mn@Op\sum\dsum\tsum

\makeatother

\makeatletter
\@ifpackageloaded{amsmath}{}{\RequirePackage{amsmath}}

\DeclareFontFamily{OMX}{MnSymbolE}{}
\DeclareSymbolFont{largesymbolsX}{OMX}{MnSymbolE}{m}{n}
\DeclareFontShape{OMX}{MnSymbolE}{m}{n}{
    <-6>  MnSymbolE5
   <6-7>  MnSymbolE6
   <7-8>  MnSymbolE7
   <8-9>  MnSymbolE8
   <9-10> MnSymbolE9
  <10-12> MnSymbolE10
  <12->   MnSymbolE12}{}

\DeclareMathSymbol{\downbrace}    {\mathord}{largesymbolsX}{'251}
\DeclareMathSymbol{\downbraceg}   {\mathord}{largesymbolsX}{'252}
\DeclareMathSymbol{\downbracegg}  {\mathord}{largesymbolsX}{'253}
\DeclareMathSymbol{\downbraceggg} {\mathord}{largesymbolsX}{'254}
\DeclareMathSymbol{\downbracegggg}{\mathord}{largesymbolsX}{'255}
\DeclareMathSymbol{\upbrace}      {\mathord}{largesymbolsX}{'256}
\DeclareMathSymbol{\upbraceg}     {\mathord}{largesymbolsX}{'257}
\DeclareMathSymbol{\upbracegg}    {\mathord}{largesymbolsX}{'260}
\DeclareMathSymbol{\upbraceggg}   {\mathord}{largesymbolsX}{'261}
\DeclareMathSymbol{\upbracegggg}  {\mathord}{largesymbolsX}{'262}
\DeclareMathSymbol{\braceld}      {\mathord}{largesymbolsX}{'263}
\DeclareMathSymbol{\bracelu}      {\mathord}{largesymbolsX}{'264}
\DeclareMathSymbol{\bracerd}      {\mathord}{largesymbolsX}{'265}
\DeclareMathSymbol{\braceru}      {\mathord}{largesymbolsX}{'266}
\DeclareMathSymbol{\bracemd}      {\mathord}{largesymbolsX}{'267}
\DeclareMathSymbol{\bracemu}      {\mathord}{largesymbolsX}{'270}
\DeclareMathSymbol{\bracemid}     {\mathord}{largesymbolsX}{'271}

\def\horiz@expandable#1#2#3#4#5#6#7#8{%
  \@mathmeasure\z@#7{#8}%
  \@tempdima=\wd\z@
  \@mathmeasure\z@#7{#1}%
  \ifdim\noexpand\wd\z@>\@tempdima
    $\m@th#7#1$%
  \else
    \@mathmeasure\z@#7{#2}%
    \ifdim\noexpand\wd\z@>\@tempdima
      $\m@th#7#2$%
    \else
      \@mathmeasure\z@#7{#3}%
      \ifdim\noexpand\wd\z@>\@tempdima
        $\m@th#7#3$%
      \else
        \@mathmeasure\z@#7{#4}%
        \ifdim\noexpand\wd\z@>\@tempdima
          $\m@th#7#4$%
        \else
          \@mathmeasure\z@#7{#5}%
          \ifdim\noexpand\wd\z@>\@tempdima
            $\m@th#7#5$%
          \else
           #6#7%
          \fi
        \fi
      \fi
    \fi
  \fi}

\def\overbrace@expandable#1#2#3{\vbox{\m@th\ialign{##\crcr
  #1#2{#3}\crcr\noalign{\kern2\p@\nointerlineskip}%
  $\m@th\hfil#2#3\hfil$\crcr}}}
\def\underbrace@expandable#1#2#3{\vtop{\m@th\ialign{##\crcr
  $\m@th\hfil#2#3\hfil$\crcr
  \noalign{\kern2\p@\nointerlineskip}%
  #1#2{#3}\crcr}}}

\def\overbrace@#1#2#3{\vbox{\m@th\ialign{##\crcr
  #1#2\crcr\noalign{\kern2\p@\nointerlineskip}%
  $\m@th\hfil#2#3\hfil$\crcr}}}
\def\underbrace@#1#2#3{\vtop{\m@th\ialign{##\crcr
  $\m@th\hfil#2#3\hfil$\crcr
  \noalign{\kern2\p@\nointerlineskip}%
  #1#2\crcr}}}

\def\bracefill@#1#2#3#4#5{$\m@th#5#1\leaders\hbox{$#4$}\hfill#2\leaders\hbox{$#4$}\hfill#3$}

\def\downbracefill@{\bracefill@\braceld\bracemd\bracerd\bracemid}
\def\upbracefill@{\bracefill@\bracelu\bracemu\braceru\bracemid}

\DeclareRobustCommand{\downbracefill}{\downbracefill@\textstyle}
\DeclareRobustCommand{\upbracefill}{\upbracefill@\textstyle}

\def\upbrace@expandable{%
  \horiz@expandable
    \upbrace
    \upbraceg
    \upbracegg
    \upbraceggg
    \upbracegggg
    \upbracefill@}
\def\downbrace@expandable{%
  \horiz@expandable
    \downbrace
    \downbraceg
    \downbracegg
    \downbraceggg
    \downbracegggg
    \downbracefill@}

\DeclareRobustCommand{\overbrace}[1]{\mathop{\mathpalette{\overbrace@expandable\downbrace@expandable}{#1}}\limits}
\DeclareRobustCommand{\underbrace}[1]{\mathop{\mathpalette{\underbrace@expandable\upbrace@expandable}{#1}}\limits}

\makeatother

\makeatletter
\DeclareFontFamily{U}{matha}{\hyphenchar\font45}
\DeclareFontShape{U}{matha}{m}{n}{
      <5> <6> <7> <8> <9> <10> gen * matha
      <10.95> matha10 <12> <14.4> <17.28> <20.74> <24.88> matha12
      }{}
\DeclareSymbolFont{matha}{U}{matha}{m}{n}
\DeclareFontSubstitution{U}{matha}{m}{n}

\def\mathabx@aliases#1#2{\@mathabx@aliases#1#2?\@end}
\def\@mathabx@aliases#1#2#3\@end{\ifx#2?\else
	\let#2=#1\@mathabx@aliases#1#3\@end\fi}%
\DeclareMathSymbol{\leftarrow}             {3}{matha}{"D0}
	\mathabx@aliases\leftarrow\gets
\DeclareMathSymbol{\rightarrow}            {3}{matha}{"D1}
	\mathabx@aliases\rightarrow\to
\DeclareMathSymbol{\wedge}         {2}{matha}{"5E}
	\mathabx@aliases\wedge\land
\DeclareMathSymbol{\vee}           {2}{matha}{"5F}
	\mathabx@aliases\vee\lor
\DeclareMathSymbol{\vdash}         {3}{matha}{"24}
\DeclareMathSymbol{\dashv}         {3}{matha}{"25}
\DeclareMathSymbol{\nvdash}        {3}{matha}{"26}
\DeclareMathSymbol{\ndashv}        {3}{matha}{"27}
\DeclareMathSymbol{\vDash}         {3}{matha}{"28}
\DeclareMathSymbol{\Dashv}         {3}{matha}{"29}
\DeclareMathSymbol{\nvDash}        {3}{matha}{"2A}
\DeclareMathSymbol{\nDashv}        {3}{matha}{"2B}
\DeclareMathSymbol{\Vdash}         {3}{matha}{"2C}
\DeclareMathSymbol{\dashV}         {3}{matha}{"2D}
\DeclareMathSymbol{\nVdash}        {3}{matha}{"2E}
\DeclareMathSymbol{\ndashV}        {3}{matha}{"2F}
\makeatother

\usepackage[small]{titlesec}
\usepackage{cite}
\usepackage{upgreek}

\def\Phi{\Upphi}

\title{\large Quantum Processes Near Black Holes%
\footnote{
This is a remake of the paper {\it Quantum Processes Near Black Holes} originally published in the first
Marcel Grossmann Meeting on the Recent Progress of the Fundamentals of General Relativity, 1975, pages 449-458.
The new version was typed out by Pete Su on November 18, 2024.
}}

\author{\normalsize G. W. Gibbons\\
\normalsize University of Cambridge, D.A.M.T.P.\\
\normalsize Silver Street, Cambridge, England}

\date{}

\def\ni{\noindent}
\def\be{\begin{equation}}
\def\ee{\end{equation}}
\def\dt{\mathop{dt}}
\def\dphi{\mathop{d\phi}}
\def\gtt{g_{tt}}

\NewDocumentCommand{\grad}{e{_^}}{%
  \mathop{}\!% \mathop for good spacing before \nabla
  \nabla
  \IfValueT{#1}{_{\!#1}}% tuck in the subscript
  \IfValueT{#2}{^{#2}}% possible superscript
}

\begin{document}

\maketitle
\thispagestyle{empty}

\begin{abstract}
A general review is given of quantum processes near black holes with
a special emphasis on the Hawking Thermodynamic Emission Process. Astrophysical
applications are not discussed.
\end{abstract}

\bigskip

\ni
I wish in this talk to summarize recent work on quantum effects near
black holes. In doing so I wish to confine myself to giving an outline of what
principles go into the calculations and the results. I shall not discuss any
astrophysical applications
(for which see e.g. [40]). I have tried to put the
various results in some sort of perspective and I hope that in doing so I have not given
insufficient weight to anyone's contribution or incorrectly judged it.
If I have done so I apologize in advance. In my talk I hope to indicate what parts of the
theory look satisfactory and which require more work and also I
shall try to indicate parallels which other parts of physics --- especially
the theory of quantum processes in strong external electromagnetic fields.

The first indication of potentially interesting effects arose when Penrose pointed out 
the existence of what has come to be known as the ``Penrose Effect'' [1]. 
This arises because of the existence of negative energy orbits in-side the ``ergosphere''
of a rotating black hole (the region where the Killing vector which is timelike at
infinity becomes spacelike). Given a region of negative energy orbits it is possible to
extract energy --- in this case the rotational energy of the black hole. 
One simply drops in a particle with positive energy $E_1$, and lets it split (inside the region) into 
2 particles one with positive energy
$E_2$, which emerges and the other with negative energy $E_3$ which remains inside.
Since $E_1 = E_2 + E_3$ we have $E_2 > E_1$. This situation also occurs in electromagnetism
near a point charge, in special relativity or indeed in any deep enough potential well [2].
It also occurs near charged black holes [3].

In fact in any electro-magnetic background which is stationary, axisymmetric and
invariant under simultaneous inversion of time and angle coordinates one finds that the
energy $E$ and angular momentum $L$ of a particle of mass $m$ and charge $e$ must satisfy
\be
(E\, \dt + L\, \dphi + eA)^2 > m^2
\ee
where $A$ is the vector potential which falls to zero at infinity.
This expression (or a simple generalization of it if there is a third constant of the motion)
determines two surfaces $E^{\pm} (r,\theta)$ in the $(E, r, \theta)$ space between which
a classical particle cannot exist. If the surface $E^+$ can fall below $- m$ we have just the
required situation referred to sometimes as ``level crossing''. The region $r$,
is referred to as a ``generalized ergosphere'' for the mode in question.
It is easy to check the existence of such a region in ``superheavy'' atoms. If
\be
A = \Phi\, \dt + B\, \dphi; \quad \Omega = g_{\phi t}/(\gtt)
\ee
the rate of rotation of inertial frames and $\sigma^2 = (g_{\phi t})^2 - g_{\phi\phi}\gtt$ we have
\be
E^\pm = e\Phi + (L + eB)\Omega \pm \sigma^2 \sqrt{m^2 + (L + eB)^2} .
\ee
\ni
On a horizon $\sigma\rightarrow 0$ and $\Omega \rightarrow \Omega_H$, $\Phi + \Omega B \rightarrow \Omega_H$, and
$E^\pm \rightarrow e \Phi_H + L \Omega_H = \mu_H$
which may be thought of as a chemical potential for the mode in question.

From the duality between waves and particles one expects a similar phenomenon to occur for waves and indeed this turns out to be the case (Misner [4] and Zeldovich [5]), and one has here the phenomenon of ``super radiance.'' For a classical scalar field this arises because the conserved flux vector
\be
J_\mu = {(\bar\phi\grad_{\!\mu}\phi-\phi\grad_{\!\mu}\bar\phi) \over 2i}
\ee
need not necessarily be future directed timelike. An incident wave carrying positive flux can send negative flux down the hole and the reflected positive flux can be greater than the incident flux. All of this is very reminiscent of the well known ``Klein Paradox'' situation [6] and indeed in the most general case of a charged, rotating black hole we have a rather close analogy to the Klein Paradox. 

In our previous notation we find the $\phi$ can be written as $\phi = e^{iEt}e^{iL\phi}\chi$ and $\chi$ obeys
\be
{1 \over \sigma}(\grad_A \, \sigma \, \grad^A \chi) + {[(E \dt + L \dphi + eA)^2 - m^2]}\chi = 0
\ee
where $\grad_A$ denotes covariant differentiation in the $r, \theta$ plane. The conserved flux is
\be
J = (E\dt + L\dphi + A) |\chi|^2 + {(\bar\chi d\chi - \chi d\bar\chi) \over 2i}
\ee
The null generator of the horizon is $\ell = \partial / \partial t - \Omega_H \partial/\partial\phi$.
The flux through the horizon is $\propto \langle J, \ell \rangle \propto E - \mu_H$. Thus $E < \mu_H$
but $E^2 < m^2$ so we have superradiance. This is of course just the previous criterion.

For classical spin $\small{1\over 2}$
fields the situation is different the conserved flux vector
\be
J = \bar\psi \gamma_\mu \psi
\ee
is always future directed timelike and so simple super radiance is not possible [7].
However it is still possible for negative energy to fall down the hole since the stress
tensor of a spin ${1\over 2}$ field does not obey the positive energy condition. Note that in both
these cases a "hole" is necessary. Super radiance cannot occur unless a particle or energy
can be trapped inside a certain region. Having seen how super radiance is possible, the analogy
with ``stimulated emission'' is very close. On rather general grounds --- Dirac [8], Feynman [9],
Einstein [10] one expects --- at least for bosons a related ``spontaneous emission.''
Further each mode should be emitted with a coefficient just given by the super radiant
coefficient (Starobinsky [11]). Note that while these physical arguments seem quite
compelling one possible objection is that they seem to imply that a black hole can be some sort
of thermal equilibrium with a surrounding heat bath. This as we shall see will
turn out to be the case but at the early stages of this subject this seemed rather
puzzling. Before I go on, it seems worthwhile here to point out that interesting as these
speculations seem, the motivation for following them up would have been
rather low had it not beenin one's mind that rather small black holes
(${\rm masses} > {\hbox{\rm planck mass}} \sim 10^{-5}$g)
had been postulated earlier by Hawking [12] as possibly arising in the early stages of
a chaotic big bang universe, although the idea of black holes smaller than the Chandrasekhar
limit had been suggested earlier by Zeldovich [13]. In this connection these early
speculations brought to light an amusing coincidence Starobinsky [11] pointed out
that the order of magnitude for the time for spontaneous loss of all of its angular
momentum by a black hole of mass must be (in units such that $G = c = h = k = 1$)
\be
t \sim M^3
\ee
Thus a hole would lose all of its angular momentum in less than $10^{10}$
years if its mass were less than $10^{-13}\,{\rm cm}$ -- a number not without
significance in other contexts.

\end{document}
