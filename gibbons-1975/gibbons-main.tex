% !TEX root = gibbons-1975.tex
\usepackage[papersize={6.6in, 10.0in}, left=.5in, right=.5in, top=.6in, bottom=.9in]{geometry}
\linespread{1.05}
%\sloppy
\raggedbottom
\pagestyle{plain}
\usepackage{mathpartir}
\usepackage{stmaryrd}
\usepackage{mathtools}
\usepackage{tikz-cd}
\usepackage{microtype}
%\usepackage{fdsymbol}

% these include amsmath and that can cause trouble in older docs.
\makeatletter
\@ifpackageloaded{amsmath}{}{\RequirePackage{amsmath}}

\DeclareFontFamily{U}  {cmex}{}
\DeclareSymbolFont{Csymbols}       {U}  {cmex}{m}{n}
\DeclareFontShape{U}{cmex}{m}{n}{
    <-6>  cmex5
   <6-7>  cmex6
   <7-8>  cmex6
   <8-9>  cmex7
   <9-10> cmex8
  <10-12> cmex9
  <12->   cmex10}{}

\def\Set@Mn@Sym#1{\@tempcnta #1\relax}
\def\Next@Mn@Sym{\advance\@tempcnta 1\relax}
\def\Prev@Mn@Sym{\advance\@tempcnta-1\relax}
\def\@Decl@Mn@Sym#1#2#3#4{\DeclareMathSymbol{#2}{#3}{#4}{#1}}
\def\Decl@Mn@Sym#1#2#3{%
  \if\relax\noexpand#1%
    \let#1\undefined
  \fi
  \expandafter\@Decl@Mn@Sym\expandafter{\the\@tempcnta}{#1}{#3}{#2}%
  \Next@Mn@Sym}
\def\Decl@Mn@Alias#1#2#3{\Prev@Mn@Sym\Decl@Mn@Sym{#1}{#2}{#3}}
\let\Decl@Mn@Char\Decl@Mn@Sym
\def\Decl@Mn@Op#1#2#3{\def#1{\DOTSB#3\slimits@}}
\def\Decl@Mn@Int#1#2#3{\def#1{\DOTSI#3\ilimits@}}

\let\sum\undefined
\DeclareMathSymbol{\tsum}{\mathop}{Csymbols}{"50}
\DeclareMathSymbol{\dsum}{\mathop}{Csymbols}{"51}

\Decl@Mn@Op\sum\dsum\tsum

\makeatother

\makeatletter
\@ifpackageloaded{amsmath}{}{\RequirePackage{amsmath}}

\DeclareFontFamily{OMX}{MnSymbolE}{}
\DeclareSymbolFont{largesymbolsX}{OMX}{MnSymbolE}{m}{n}
\DeclareFontShape{OMX}{MnSymbolE}{m}{n}{
    <-6>  MnSymbolE5
   <6-7>  MnSymbolE6
   <7-8>  MnSymbolE7
   <8-9>  MnSymbolE8
   <9-10> MnSymbolE9
  <10-12> MnSymbolE10
  <12->   MnSymbolE12}{}

\DeclareMathSymbol{\downbrace}    {\mathord}{largesymbolsX}{'251}
\DeclareMathSymbol{\downbraceg}   {\mathord}{largesymbolsX}{'252}
\DeclareMathSymbol{\downbracegg}  {\mathord}{largesymbolsX}{'253}
\DeclareMathSymbol{\downbraceggg} {\mathord}{largesymbolsX}{'254}
\DeclareMathSymbol{\downbracegggg}{\mathord}{largesymbolsX}{'255}
\DeclareMathSymbol{\upbrace}      {\mathord}{largesymbolsX}{'256}
\DeclareMathSymbol{\upbraceg}     {\mathord}{largesymbolsX}{'257}
\DeclareMathSymbol{\upbracegg}    {\mathord}{largesymbolsX}{'260}
\DeclareMathSymbol{\upbraceggg}   {\mathord}{largesymbolsX}{'261}
\DeclareMathSymbol{\upbracegggg}  {\mathord}{largesymbolsX}{'262}
\DeclareMathSymbol{\braceld}      {\mathord}{largesymbolsX}{'263}
\DeclareMathSymbol{\bracelu}      {\mathord}{largesymbolsX}{'264}
\DeclareMathSymbol{\bracerd}      {\mathord}{largesymbolsX}{'265}
\DeclareMathSymbol{\braceru}      {\mathord}{largesymbolsX}{'266}
\DeclareMathSymbol{\bracemd}      {\mathord}{largesymbolsX}{'267}
\DeclareMathSymbol{\bracemu}      {\mathord}{largesymbolsX}{'270}
\DeclareMathSymbol{\bracemid}     {\mathord}{largesymbolsX}{'271}

\def\horiz@expandable#1#2#3#4#5#6#7#8{%
  \@mathmeasure\z@#7{#8}%
  \@tempdima=\wd\z@
  \@mathmeasure\z@#7{#1}%
  \ifdim\noexpand\wd\z@>\@tempdima
    $\m@th#7#1$%
  \else
    \@mathmeasure\z@#7{#2}%
    \ifdim\noexpand\wd\z@>\@tempdima
      $\m@th#7#2$%
    \else
      \@mathmeasure\z@#7{#3}%
      \ifdim\noexpand\wd\z@>\@tempdima
        $\m@th#7#3$%
      \else
        \@mathmeasure\z@#7{#4}%
        \ifdim\noexpand\wd\z@>\@tempdima
          $\m@th#7#4$%
        \else
          \@mathmeasure\z@#7{#5}%
          \ifdim\noexpand\wd\z@>\@tempdima
            $\m@th#7#5$%
          \else
           #6#7%
          \fi
        \fi
      \fi
    \fi
  \fi}

\def\overbrace@expandable#1#2#3{\vbox{\m@th\ialign{##\crcr
  #1#2{#3}\crcr\noalign{\kern2\p@\nointerlineskip}%
  $\m@th\hfil#2#3\hfil$\crcr}}}
\def\underbrace@expandable#1#2#3{\vtop{\m@th\ialign{##\crcr
  $\m@th\hfil#2#3\hfil$\crcr
  \noalign{\kern2\p@\nointerlineskip}%
  #1#2{#3}\crcr}}}

\def\overbrace@#1#2#3{\vbox{\m@th\ialign{##\crcr
  #1#2\crcr\noalign{\kern2\p@\nointerlineskip}%
  $\m@th\hfil#2#3\hfil$\crcr}}}
\def\underbrace@#1#2#3{\vtop{\m@th\ialign{##\crcr
  $\m@th\hfil#2#3\hfil$\crcr
  \noalign{\kern2\p@\nointerlineskip}%
  #1#2\crcr}}}

\def\bracefill@#1#2#3#4#5{$\m@th#5#1\leaders\hbox{$#4$}\hfill#2\leaders\hbox{$#4$}\hfill#3$}

\def\downbracefill@{\bracefill@\braceld\bracemd\bracerd\bracemid}
\def\upbracefill@{\bracefill@\bracelu\bracemu\braceru\bracemid}

\DeclareRobustCommand{\downbracefill}{\downbracefill@\textstyle}
\DeclareRobustCommand{\upbracefill}{\upbracefill@\textstyle}

\def\upbrace@expandable{%
  \horiz@expandable
    \upbrace
    \upbraceg
    \upbracegg
    \upbraceggg
    \upbracegggg
    \upbracefill@}
\def\downbrace@expandable{%
  \horiz@expandable
    \downbrace
    \downbraceg
    \downbracegg
    \downbraceggg
    \downbracegggg
    \downbracefill@}

\DeclareRobustCommand{\overbrace}[1]{\mathop{\mathpalette{\overbrace@expandable\downbrace@expandable}{#1}}\limits}
\DeclareRobustCommand{\underbrace}[1]{\mathop{\mathpalette{\underbrace@expandable\upbrace@expandable}{#1}}\limits}

\makeatother

\makeatletter
\DeclareFontFamily{U}{matha}{\hyphenchar\font45}
\DeclareFontShape{U}{matha}{m}{n}{
      <5> <6> <7> <8> <9> <10> gen * matha
      <10.95> matha10 <12> <14.4> <17.28> <20.74> <24.88> matha12
      }{}
\DeclareSymbolFont{matha}{U}{matha}{m}{n}
\DeclareFontSubstitution{U}{matha}{m}{n}

\def\mathabx@aliases#1#2{\@mathabx@aliases#1#2?\@end}
\def\@mathabx@aliases#1#2#3\@end{\ifx#2?\else
	\let#2=#1\@mathabx@aliases#1#3\@end\fi}%
\DeclareMathSymbol{\leftarrow}             {3}{matha}{"D0}
	\mathabx@aliases\leftarrow\gets
\DeclareMathSymbol{\rightarrow}            {3}{matha}{"D1}
	\mathabx@aliases\rightarrow\to
\DeclareMathSymbol{\wedge}         {2}{matha}{"5E}
	\mathabx@aliases\wedge\land
\DeclareMathSymbol{\vee}           {2}{matha}{"5F}
	\mathabx@aliases\vee\lor
\DeclareMathSymbol{\vdash}         {3}{matha}{"24}
\DeclareMathSymbol{\dashv}         {3}{matha}{"25}
\DeclareMathSymbol{\nvdash}        {3}{matha}{"26}
\DeclareMathSymbol{\ndashv}        {3}{matha}{"27}
\DeclareMathSymbol{\vDash}         {3}{matha}{"28}
\DeclareMathSymbol{\Dashv}         {3}{matha}{"29}
\DeclareMathSymbol{\nvDash}        {3}{matha}{"2A}
\DeclareMathSymbol{\nDashv}        {3}{matha}{"2B}
\DeclareMathSymbol{\Vdash}         {3}{matha}{"2C}
\DeclareMathSymbol{\dashV}         {3}{matha}{"2D}
\DeclareMathSymbol{\nVdash}        {3}{matha}{"2E}
\DeclareMathSymbol{\ndashV}        {3}{matha}{"2F}
\makeatother

\usepackage[small]{titlesec}
\usepackage{cite}
\usepackage{upgreek}

\def\Phi{\Upphi}

\title{\large Quantum Processes Near Black Holes%
\footnote{
This is a remake of the paper {\it Quantum Processes Near Black Holes} originally published in the first
Marcel Grossmann Meeting on the Recent Progress of the Fundamentals of General Relativity, 1975, pages 449-458.
The new version was typed out by Pete Su on November 18, 2024.
}}

\author{\normalsize G. W. Gibbons\\
\normalsize University of Cambridge, D.A.M.T.P.\\
\normalsize Silver Street, Cambridge, England}

\date{}

\def\ni{\noindent}
\def\be{\begin{equation}}
\def\ee{\end{equation}}

\def\dt{\mathop{dt}}
\def\dphi{\mathop{d\phi}}
\def\gtt{g_{tt}}
\def\Tuv{\langle T_{\mu\nu} \rangle}
\def\TOO{\langle T_{00} \rangle}
\def\TOr{\langle T_{0r} \rangle}
\def\out{\text{\it out}}
\def\k{\kappa}
\def\OH{\Omega_H}
\def\PH{\Phi_{\!H}}
\def\AH{A_H}
\def\half{\small{\frac{1}{2}}}

\NewDocumentCommand{\grad}{e{_^}}{%
  \mathop{}\!% \mathop for good spacing before \nabla
  \nabla
  \IfValueT{#1}{_{\!#1}}% tuck in the subscript
  \IfValueT{#2}{^{#2}}% possible superscript
}

\NewDocumentCommand{\GG}{e{_^}}{%
  \mathop{}\!% \mathop for good spacing before \nabla
  \Gamma
  \IfValueT{#1}{_{\!#1}}% tuck in the subscript
  \IfValueT{#2}{^{#2}}% possible superscript
}


\begin{document}

\maketitle
\thispagestyle{empty}

\begin{abstract}
A general review is given of quantum processes near black holes with
a special emphasis on the Hawking Thermodynamic Emission Process. Astrophysical
applications are not discussed.
\end{abstract}

\bigskip

\ni
I wish in this talk to summarize recent work on quantum effects near
black holes. In doing so I wish to confine myself to giving an outline of what
principles go into the calculations and the results. I shall not discuss any
astrophysical applications
(for which see e.g. [40]). I have tried to put the
various results in some sort of perspective and I hope that in doing so I have not given
insufficient weight to anyone's contribution or incorrectly judged it.
If I have done so I apologize in advance. In my talk I hope to indicate what parts of the
theory look satisfactory and which require more work and also I
shall try to indicate parallels which other parts of physics --- especially
the theory of quantum processes in strong external electromagnetic fields.

The first indication of potentially interesting effects arose when Penrose pointed out 
the existence of what has come to be known as the ``Penrose Effect'' [1]. 
This arises because of the existence of negative energy orbits in-side the ``ergosphere''
of a rotating black hole (the region where the Killing vector which is timelike at
infinity becomes spacelike). Given a region of negative energy orbits it is possible to
extract energy --- in this case the rotational energy of the black hole. 
One simply drops in a particle with positive energy $E_1$, and lets it split (inside the region) into 
2 particles one with positive energy
$E_2$, which emerges and the other with negative energy $E_3$ which remains inside.
Since $E_1 = E_2 + E_3$ we have $E_2 > E_1$. This situation also occurs in electromagnetism
near a point charge, in special relativity or indeed in any deep enough potential well [2].
It also occurs near charged black holes [3].

In fact in any electro-magnetic background which is stationary, axisymmetric and
invariant under simultaneous inversion of time and angle coordinates one finds that the
energy $E$ and angular momentum $L$ of a particle of mass $m$ and charge $e$ must satisfy
\be
(E\, \dt + L\, \dphi + eA)^2 > m^2
\ee
where $A$ is the vector potential which falls to zero at infinity.
This expression (or a simple generalization of it if there is a third constant of the motion)
determines two surfaces $E^{\pm} (r,\theta)$ in the $(E, r, \theta)$ space between which
a classical particle cannot exist. If the surface $E^+$ can fall below $- m$ we have just the
required situation referred to sometimes as ``level crossing''. The region $r$,
is referred to as a ``generalized ergosphere'' for the mode in question.
It is easy to check the existence of such a region in ``superheavy'' atoms. If
\be
A = \Phi\, \dt + B\, \dphi; \quad \Omega = g_{\phi t}/(\gtt)
\ee
the rate of rotation of inertial frames and $\sigma^2 = (g_{\phi t})^2 - g_{\phi\phi}\gtt$ we have
\be
E^\pm = e\Phi + (L + eB)\Omega \pm \sigma^2 \sqrt{m^2 + (L + eB)^2} .
\ee
\ni
On a horizon $\sigma\rightarrow 0$ and $\Omega \rightarrow \OH$, $\Phi + \Omega B \rightarrow \OH$, and
$E^\pm \rightarrow e \PH + L \OH = \mu_H$
which may be thought of as a chemical potential for the mode in question.

From the duality between waves and particles one expects a similar phenomenon to occur for waves and indeed this turns out to be the case (Misner [4] and Zeldovich [5]), and one has here the phenomenon of ``super radiance.'' For a classical scalar field this arises because the conserved flux vector
\be
J_\mu = {(\bar\phi\grad_{\!\mu}\phi-\phi\grad_{\!\mu}\bar\phi) \over 2i}
\ee
need not necessarily be future directed timelike. An incident wave carrying positive flux can send negative flux down the hole and the reflected positive flux can be greater than the incident flux. All of this is very reminiscent of the well known ``Klein Paradox'' situation [6] and indeed in the most general case of a charged, rotating black hole we have a rather close analogy to the Klein Paradox. 

In our previous notation we find the $\phi$ can be written as $\phi = e^{iEt}e^{iL\phi}\chi$ and $\chi$ obeys
\be
{1 \over \sigma}(\grad_A \, \sigma \, \grad^A \chi) + {[(E \dt + L \dphi + eA)^2 - m^2]}\chi = 0
\ee
where $\grad_A$ denotes covariant differentiation in the $r, \theta$ plane. The conserved flux is
\be
J = (E\dt + L\dphi + A) |\chi|^2 + {(\bar\chi d\chi - \chi d\bar\chi) \over 2i}
\ee
The null generator of the horizon is $\ell = \partial / \partial t - \OH \partial/\partial\phi$.
The flux through the horizon is $\propto \langle J, \ell \rangle \propto E - \mu_H$. Thus $E < \mu_H$
but $E^2 < m^2$ so we have superradiance. This is of course just the previous criterion.

For classical spin $\half$
fields the situation is different the conserved flux vector
\be
J = \bar\psi \GG_\mu \psi
\ee
is always future directed timelike and so simple super radiance is not possible [7].
However it is still possible for negative energy to fall down the hole since the stress
tensor of a spin $\half$ field does not obey the positive energy condition. Note that in both
these cases a "hole" is necessary. Super radiance cannot occur unless a particle or energy
can be trapped inside a certain region. Having seen how super radiance is possible, the analogy
with ``stimulated emission'' is very close. On rather general grounds --- Dirac [8], Feynman [9],
Einstein [10] one expects --- at least for bosons a related ``spontaneous emission.''
Further each mode should be emitted with a coefficient just given by the super radiant
coefficient (Starobinsky [11]). Note that while these physical arguments seem quite
compelling one possible objection is that they seem to imply that a black hole can be some sort
of thermal equilibrium with a surrounding heat bath. This as we shall see will
turn out to be the case but at the early stages of this subject this seemed rather
puzzling. Before I go on, it seems worthwhile here to point out that interesting as these
speculations seem, the motivation for following them up would have been
rather low had it not beenin one's mind that rather small black holes
(${\rm masses} > {\hbox{\rm planck mass}} \sim 10^{-5}$g)
had been postulated earlier by Hawking [12] as possibly arising in the early stages of
a chaotic big bang universe, although the idea of black holes smaller than the Chandrasekhar
limit had been suggested earlier by Zeldovich [13]. In this connection these early
speculations brought to light an amusing coincidence Starobinsky [11] pointed out
that the order of magnitude for the time for spontaneous loss of all of its angular
momentum by a black hole of mass must be (in units such that $G = c = h = k = 1$)
\be
t \sim M^3
\ee
Thus a hole would lose all of its angular momentum in less than $10^{10}$
years if its mass were less than $10^{-13}\,{\rm cm}$ -- a number not without
significance in other contexts. The extension of these ideas to the charge
of a black hole [14] showed similarly that unless the hole had a mass
of this order, $e^2/m_e$, it would be energetically favorable for it to discharge
itself even if it possessed a single electron charge. The rate was expected
to depend on the field strength and in a Schwinger [15] type way. Thus unless the
electric field is less than the critical field mile the rate is very fast.
This implied that to have a charge comparable with its mass the black hole mass
must exceed $e^2/m_e$ (which is coincidentally the least mass of a ``classical geon'' [16].
Essentially the same ideas seem to have occurred to Zaumen [17] independently.
The story has also been taken up by Ruffini and Deruelle [19]and Ruffini and Damour [20].
These estimates made it very unlikely that mini black holes possessed charge.

Having seen the physical ideas which enter it remained to give them a
more rigorous expression. The first person to tackle this problem was
Unruh [21]. Since there is at present no well worked out candidate for a quantum theory
of gravity Unruh adopted an approach in which the gravitational field was treated as
a classical background--the so called external field approach. Thus one takes
the equations describing a free quantum field in flat space and minimally
couples them to the external field by the replacement $\partial_\mu \to \grad_\mu - ieA_\mu$.
This does not always yield a sensible theory [41] but in the case of spin 0, $\half$ and $1$ a workable
theory results.

The next problem one encounters is the definition of particle states or may
be summarized as follows: the basic strategy of the quanvacuum state. This
quantum theory of fields is to resolve a field into normal modes. The coefficients
of these normal modes obey the familiar bose einstein/ fermi-dirac commutation/anticommutation
relations. This gives field commutation/anticommutation relations which are independent
of the choice of normal modes--provided they are properly normalized with the
natural sesquilinear form available:
\begin{align}
\frac{1}{2i} \int (\bar\phi\grad_\mu\phi - \phi\grad_\mu\phi) d\Sigma^\mu \quad & \text{for spin 0}\\
\int \bar\psi\GG^\mu\psi d\Sigma^\mu \quad & \text{for spin \small$\half$}
\end{align}
What is {\it not} independent of the choice is the vacuum state. Any transformation
of the normal modes (Bogoliubov transformation) which mixes up particle and
antiparticle modes (or positive and negative frequencies to use a conventional expression)
will give an inequivalent definition of the vacuum state. Indeed, as
seems to occur in most practical examples, the number of ``created particles'' diverges and
the two different state vectors may not even be connected by a unitary transformation [22].
Unruh made a particular choice --- essentially that the particle modes be positive
frequency with respect to the Killing vector $\partial/\partial t$ in the Kerr solution.
He then computed the stress tensor expectation value $\Tuv$ in this state and found
that $\TOO$ was infinite, $\TOr$ was finite and corresponded to an outward flux of
super radiant modes at the expected rate. Similar results were subsequently found by Ford [23].
It should be mentioned that the gravitational background used was the maximally
extended Kerr solution. We see that in general we meet three generic types of problem:
\begin{enumerate}
\item choice of vacuum state
\item infinities in $T$
\item breakdown for higher spins.
\end{enumerate}
All of these problems occur and are familiar in the corresponding electromagnetic
case. The next advance came with the work of Hawking [24]. He realized that
\begin{enumerate}
\item One can only satisfactorily define particle states at infinity

\item One must for a satisfactory treatment take collapse
into account.
\end{enumerate}
To take the first point; Hawking decided in the spirit of the matrix
approach to define two vacua, the in vacuum and the out vacuum $\vert 0_{-} \rangle$, $\vert 0_+ \rangle$.
Provided past infinity constitutes a Cauchy surface (thus excluding the mixed white
hole/black hole situation considered by Unruh) one may define an initial no
particle state by the usual prescription of associating positive frequencies
with particle states and conversely negative frequencies with anti-particle states.
Since the idea of positive frequency is invariant under the asymptotic symmetry group, the B.M.S.
group. This remains valid even in the presence of gravitational radiation. 
Indeed one can show that the gravitational field of a plane wave (like a plane electromagnetic wave)
is incapable of producing particles [25]. Since at infinity any ingoing radiation will be effectively plane it is clear that as far
as past infinity is concerned we have a reasonable definition of what it means to
say that there are initially no ingoing particles. Similarly we can identify outgoing particles
at future infinity. The task of identifying particle states in the interaction region is much less clear.
This is perhaps not unreasonable. Physically a particle is really a certain sort
of normal mode with high symmetry. Mathematically it is connected with irreducible
representations of the Poincar\'e Group [26]. Neither of these concepts is applicable near the black hole.
These remarks do {\it not} apply to the strong electromagnetic field around a large black hole since
here the typical wavelength of the created particles is much smaller than the horizon size.

Now to turn to the next point. The problem is to count the number of out-going particles in
state which initially contained no ingoing particles. We are working in the Heisenberg picture.
\be
N_i^{\out} = \langle 0_{-} \vert (a_i^\out)^\dagger(a_i^\out)\vert 0_{-}\rangle
\ee
Taking the collapse into account and the very high redshifts associated with the formation
of the event horizon Hawking was able to show that this number diverges which he also
showed corresponded to a steady emission at a rate
\be
R_i = \GG_i \,\biggl(\exp \biggl ( \frac{E_i - N_i\OH - e_i\PH}{T}\biggr) \mp 1\biggr)^{-1} \quad
\begin{array}{ll}
\small - & \text{\small for bosons}\\
\small + & \text{\small for fermions}
\end{array}
\ee
where $E_i$ is the energy of the outgoing particles; $N_i$ the angular momemtum; $\OH$
the angular velocity of the hole; $e_i$ the charge of the outgoing particle; $\PH$ the electrical
potential of the hole. $\Gamma_i$ is the absorbtion coefficient of the hole for
classical waves of energy $E_i$, etc, and $T$ is related to a constant $\k$ (``the surface gravity'')
which plays an important role in black hole physics. For a black hole of mass $M$, charge $Q$ these
constants are 
\begin{align*}
\OH &= J/M^2\\
\k &= r_+ - r_-/(2r^2)\\
r_0^2 &= r_+^2+\frac{J^2}{M^2}\\
\PH &= Qr_+/r_0^2\\
r_{\pm} &= M \pm \sqrt{M^2 - \frac{J^2}{M^2} - Q^2}\\
T &= \k/2\pi.
\end{align*}
In terms of an ``irreducible mass'' $M_0 = r_0$ and the event horizon area $\AH$ we have
\begin{align}
A_H & = 16\pi M_0^2 = 4\pi r_0^2\\
M^2 &= \biggl(M_0 + \frac{Q^2}{4M_0^2}\biggr)^2 + \frac{J^2}{4M_0^2}
\end{align}
There are several remarkable features of this result. Firstly, observe that while
it was necessary to include the collapse in order to give meaning to the
calculation the details of the collapse do not enter at all into the answer.
If one does not consider the collapse then the most natural way of doing the
calculations (i.e. in the fully extended Kruskal manifold) give no production [27].
It should be pointed out that the result one gets in the Kruskal manifold depends
crucially on what boundary conditions one sets on the past horizon. A suitable choice
could in principle yield any desired result. Unruh has pointed out that one may obtain
Hawking's result by chosing as part of one's set of normal modes a set entering the
external region of the black hole from the past horizon and behaving like a complex
exponential of the affine parameter onthe post horizon (see his article).
Secondly in order to obtain the result it is necessary to take into account normal
modes of arbitrarily high energy(way above the Planck frequency) essentially because
of the red shifting effect of the horizon. Indeed, if one imposes a cutoff frequency
of $\omega_c$ the emission will stop after a time $t \sim (1/\kappa) \log(w_c/\kappa)$.
This in fact is likely to be true in any calculation since it comes from considering
how close to the horizon (in terms of an affine parameter) you have to be in
order that you can send signals which reach infinity at a retarded time.

Thirdly we have here just the thermal emission which was lacking in our previous physical
arguments --- although the analogy between atomic levels and a black hole is not
absolutely precise. It should be mentioned here that the idea that a black hole
has associated with it a temperature and an entropy (which is of course
what these results imply) had been suggested previously by Beckenstein [28]
on the basis of certain analogies between black hole physics and thermodynamics (cf. [29]) --- 
the non-decreasing event horizon area playing the part of entropy. The precise
relation is now seen to be $S_H = \AH/4$.

One could of course spend a great deal of time describing the implications
of these spectacular results. In what follows I shall bring out just a few points.
Before proceeding, however, I should note that Hawking's work encompasses
all the previous results of Unruh and Starobinsky as limiting cases. The way in which
this comes about for charged black holes and the relation of all this to the well
known formulae due to Schwinger [15] for particle creation in uniform electric
fields has been described by myself [30] cf also [20]. There are of course obvious
analogies with thermionic emission from metal surfaces, $e\PH$ acting as a
``work function''.

It is sometimes said that the super-radiant processes like particle creation
in a uniform static electric field do not require a time dependent field in
contradistinction to theHawking process. This is not really correct. The stationary
(and in general non physical) backgrounds can only be used in conjunction
with boundary conditions. These boundary conditions relate to what particles
enter the space from the past horizon or from the infinite past respectively.
This decision is made (via the choice of boundary conditions for Feynman propa-
gators or a choice of a complete set of normal modes) in a way appropriate for a
situation in which the interaction is ``turned off'' in the infinite past. From
this point of view the ``method of level crossing'' amounts to:
\begin{enumerate}
\item a particular choice of normal modes (proportional to $e^{i \omega t}$).
\item a method of deciding whether the ingoing antiparticles overlap with
outgoing particles (i.e. whether the Bogoliubov coefficients vanish)
or not
\item an elegant representation of how large the effects will be by computing 
transmission coefficients in a certain effective potential. This
is explained in more detail by Dr. Damour in his article and in [30]
but the essential point is that what is a particle or anti-particle
is determined by the flux it carries through a Cauchy surface.
As I explained earlier this depends, for horizon modes, on $E-\mu_H$.
Thus roughly speaking, a wave can appear as antiparticle near the horizon
($E-\mu_H < 0$) but as a particle at infinity ($E > 0$).
\end{enumerate}
From what has been said above it is clear that for all its spectacular
successes, Hawking's derivation of his result is not entirely satisfactory For
an alternative attempt see [31], for skeptical remarks [32], [33].
This it seems to me not due to any fault on Hawking's part but due to the intrinsic problems
with the whole external field theory method. For instance we still have
the divergences in $\Tuv$ (and in the total number of particles) which need to be
taken into account before we begin to feed back this expectation value into a
Hartree-Fock version of Einstein's equations, if indeed that is appropriate. Some
of these infinities can be absorbed as infinite renormalizations of the cosmological
constant and Newton's constant $G$. Some, however, are of an essentially new nature, 
as has been pointed out by DeWitt [34]. In terms of an effective Lagrangian 
they correspond to terms of the form $R_{ab}R^{ab}$ and $(R_{ab}g^{ab})^2$.
No detailed calculations of these to have be been given in the black case but there
seems little ground for doubting their existence. Thus even the external field case
external field problem is unrenormalizable which is closely related to the fact
that the only quantum theory of gravity which is amenable to calculations suffers
from the same defect [35]. One might also worry that particle interactions have not been
taken into account. This is presumably valid for large holes but for holes of size
$10^{-13}\text{\small cm}$ this considerable flux of particles in such a small volume should surely
require the use of interacting field theory.

All of these problems are deep and difficult, what I want to do in the
final part of the talk is to cut through the Gordian Knot as it were, disregard
the original field theory derivation and hang everything on the thermodynamic
idea.

Before doing so it is perhaps worthwhile trying to see why a black hole should give a thermal
spectrum at all. The most natural interpretation is that the Hawking process consists of very
many individual events consisting of the creation of a virtual pair near the hole and the
tunnelling of a member (presumably carrying negative energy) through the horizon, the other reaching infinity.

What is seen is a statistical ensemble of such independent events each happening
with a probability proportional to the phase space available. The factor $\GG_i$ arises
because the particles are created (in some sense) near the horizon and have to tunnel
out through a combined curvature and centrifugal barrier. In all such calculations we
need a constraint and in fact if we constrain the rate of dissipation of entropy or irreducible
mass we obtain the Hawking formulae --- with of course the temperature undetermined.
Thus a black hole emits thermally because it is the most likely thing for it to do. 
Support for the idea comes from the work of Wald [22] and Parker [36]
who have shown that the statistics of outgoing particles are those of a black body.

Given that we have a small body whose temperature decreases the more
energy it loses. This is, of course, a manifestation of the well known fact that
gravitating systems can have negative specific heats. The evolution of such
a system will be towards a hotter and hotter state ---the black hole presumably dis-
appearing altogether, and presumably with it the baryons that make it up. To my
knowledge now one has mechanism whereby black holes can emit more baryons than
antibaryon without making use of long range fields carrying baryonic charge [37].
If we adopt the thermodynamic viewpoint even when particle interactions
aretaken into account, then presumably a black hole will emit precisely what it
would accrete from a heat bath at the same temperature. In terms of our simple
arguments before the probability of emission is now no longer simply proportional
to the available phase space. Carter, Lin and Perry and myself [38] have recently
made some rough calculations on the basis. We represented a heat bath of strongly
interacting particles as a perfect fluid with an equation of state. The accretion
problem is straight forward and the corresponding emission is described by
a sort of stellar wind. The interesting thing about our calculation is that
provided the high density gas has a reasonably hard equation of state the wind is
essentially transparent. This indicates that interactions might not be very
important. It is perhaps worth reemphasizing here that for higher spin
($S > {\small \frac{3}{2}}$) the external field theory approach breaks down.

In this thermodynamic vein it is amusing to note that in this light the
old theory of black body radiation requires revision. Consider a cavity of fixed
volume $V$. They fill it with more and more energy.

At first the cavity will contain an ordinary black body gas and its
temperature will rise. As it does, 
however, a qualitatively different behavior sets
in. We have to maximize the entropy, $S$, of the configuration subject to the
energy, $E$, being held constant.
\begin{align}
S &= 4\pi M^2 + \frac{4}{3}a VT^3\\
E &= M + aVT^4
\end{align}
where $a$ is Stefan's constant.

If $x = M/E$ and $y = (aV/E^5)/3\pi$ this amounts to maxmimizing $F = x^2 - (1-x)^{3/4}$
on $[0,1]$. For $y > {2^5}{3^{-1}} 5^{-5/4} = 1.4266$ has no turning points and its greatest
value is attained at $x=0$ (pure radiation). For $1.4266 > y > 1.01440$ there is a
local minimum at $x < \frac{4}{5}$ and a local maximum at $x < \frac{4}{5}$ but 
that $x = 0$ is still a global maximum. For $y< 1.01440$ the local maximum at $x > 0.97702$
is also a global maximum.
Thus for a box with sufficient energy a black hole of
mass $M = \frac{1}{8\pi T}$ will condense out.
As the energy of the enclosure is further increased the temperature will drop.
Thus for any volume $V$ there is a maximum temperature 
\begin{align*}
T_m &= \frac{1}{8\pi}x_c^{3/4}aV^{-1/5}(3\pi y_c)^{4/5}\\
& = x_c^{-1/4} T_c \\
&  \,\,\quad x_c = 0.97702, \,\, y_c = 1.01440.
\end{align*}
The point of the example is to indicate how basic Hawking's discovery is
in relation to our views of fundamental physics. 
What part the sort of speculations contained in [38] on the deeper
role that black holes have to play in the
scheme of things remains to be seen. With that comment I shall close this review
--- in doing so is appropriate for me to thank my many colleagues in this field
but most especially S. W. Hawking for many discussions.


\bigskip
\centerline{\sc Addendum}
\bigskip

After completing the review I received the following preprints. The
of which is most pertinent to the final section:

\bigskip

S. W. Hawking, "Black Holesand Thermodynamics"

T. Damour "On the Correspondence between Classical and Quantum Energy
States in Stationary Geometries"'

N. Deruelle, R. Ruffini "Klein Paradox in a Kerr Geometry"

W. Unruh "Notes on Black Hole Evaporation"

\newpage

\renewcommand{\section}[2]{}%

\bigskip
\bigskip
\centerline{\sc References}

\begin{thebibliography}{100}

\small
\raggedright

\bibitem[1]{1} R. Penrose. Riv. Nuovo Cimento 1 252 (1969)

\bibitem[2]{2} G. Denardo \& A. Treves. Lett. al. Nuovo Cimento \& 295 (1973)

\bibitem[3]{3} G. Denardo \& R. Ruffini. Phys. Lett. 45B 259 (1973); G. Denardo, L. Hively
\& R. Ruffini. Phys. Lett. 50B 270 (1974); cf. also Ya. B. Zeldovich \&
V. S. Popov. Uspekhi 14 673 (1972); L. I. Schiff, H. Snyder \& J. Weinberg,
Phys. Rev. 57 315 (1940)

\bibitem[4]{4} C. Misner. Bull. Amer. Phys. Soc. 17 472 (1972)

\bibitem[5]{} Ya. Zeldovich J.E.T.P. 35 (1972)

\bibitem[6]{} O. Klein, Zeit. fur Phys. 53 157 (1929)

\bibitem[7]{7} W. Unruh. Phys. Rev. Lett. 31 1265 (1973)

\bibitem[8]{8} P. A. M. Dirac. Quantum Mechanics O.U.P.

\bibitem[9]{} R. D. Feynman. Lectures on Physics, Vol. 111.

\bibitem[10]{} A. Einstein. Phys. Zeit 18 121 (1917)

\bibitem[11]{} A. Starobinsky. J.E.T.P. 37 28 (1973)

\bibitem[12]{} S. W. Hawking.
M.N.R.A.S. 152 75 (1971); B. J. Carr \& S. W. Hawking,
M.N.R.A.S. 168 399 (1974)

\bibitem[13]{13} Ya. Zeldovich. J.E.T.P. 446 (1962) 

\bibitem[14]{14} G. W. Gibbons \& S. W. Hawking. Work reported at Warsaw Conference 1973
see ``Gravitational Radiations and Gravitational Collapse'' C. deWitt (ed.) Reidel (1974)

\bibitem[15]{} J. Schwinger. Phys. Rev. 82 664 (1951)

\bibitem[16]{} J. Wheeler. ``Geometrodynamics'' Academic Press (1972)

\bibitem[17]{} W. T. Zaumen. Nature 247 530 (1974)

\bibitem[19]{} N. Deruelle \& R. Ruffini. Phys. Lett. 52B 437 (1975)

\bibitem[20]{} T. Damour \& R. Ruffini. ``Quantum electrodynamic effects in Kerr-Newman
Geometries'' Princeton Preprint. Dec. 1974

\bibitem[21]{} W. Unruh. Phys. Rev. D 10 3194 (1974) .

\bibitem[22]{} Cf. M. Castagnino, A. Verbeuri \& R. A. Weder,
Nuovo Cimento 26B 396 (1975), Phys. Lett. 48A 99 (1974);
R. Wald. Commun. Math. Phys. 46 (1975)

\bibitem[23]{} L. Ford ``Quantization of a Scalar Field in the Kerr Spacetime'' Milwaukee Preprint UWM-4867-74-17 (1974)

\bibitem[24]{} S. W. Hawking. Nature 248 30 (1974); S. W. Hawking. Commun. Math. Phys.
43 199 (1975)

\bibitem[25]{} G. W. Gibbons. Commun. M. Phys. 45 191 (1975)

\bibitem[26]{} E. Wigner. Ann. of Math. 40 (1939)

\bibitem[27]{} D. Boulware. Phys. Rev. D 11 1406 (1975); D. Boulware. ``Spin 1/2 Quantum
Field Theory in Schwarzschild Space'' Seattle Preprint RL-1388-689 (1975).

\bibitem[28]{} J. D. Beckenstein. Phys. Rev. D7 2333 (1972)

\bibitem[29]{} J. Bardeen, B. Carter \& S. W. Hawking, Commun. Math. Phys. 31 162 (1973)

\bibitem[30]{} G. W. Gibbons, Commun. Math. Phys. 44 245 (1975)

\bibitem[31]{} U. Gerlach. ``Mechanism of Black Body Radiation from an incipient Black
Hole'' Ohio Preprint.

\bibitem[32]{} J. G. Taylor \& P. Davies. Nature 250 37 (1974) .

\bibitem[33]{} P. W. Davies. J. Phys. A 8 609 (1975).

\bibitem[34]{} B. deWitt. Physics Reports 29C. 295 (1975).

\bibitem[35]{} E.g. S. Deser and P. van Nieuwenhuizen. Phys. Rev. d 10 411 (1974) .

\bibitem[36]{} L. Parker Phys. Rev. D 12 (1975).

\bibitem[37]{} B. Carter. Phys. Rev. Lett. 33 558 (1974).

\bibitem[38]{} B. Carter, G. W. Gibbons, D. Lin \& M. Perry ``The Black Hole emission
process in the high energy limit'' in preparation.

\bibitem[39]{} J. Sarfatt. Nature 240 101 (1972) .

\bibitem[40]{} B. J. Carr. ``The Primordial Mass Spectrum'' Caltech Preprint (1975).

\bibitem[41]{} G. W. Gibbons. J. Phys. A 9 145 (1976).

\end{thebibliography}
\end{document}
