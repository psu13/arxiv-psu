\documentclass[12pt]{article}
\usepackage[full]{textcomp}
\usepackage[osf,semibold,scaled=1.05]{ebgaramond}

%\usepackage[sups,osf]{garamondx} % osf (or tosf) for text, not math
%\usepackage[sups,osf,p]{fbb} % osf (or tosf) for text, not math
%\usepackage[osf,sups]{Baskervaldx}
%\usepackage[sb,osf,p]{libertine}
%\usepackage[sc,osf]{mathpazo}

\usepackage[perpage]{footmisc}
\renewcommand{\thefootnote}{\fnsymbol{footnote}}

\usepackage[colorlinks=true
,urlcolor=blue
,anchorcolor=blue
,citecolor=blue
,filecolor=blue
,linkcolor=blue
,menucolor=blue
,linktocpage=true]{hyperref}
\hypersetup{
bookmarksopen=true,
bookmarksnumbered=true,
bookmarksopenlevel=10
}
\usepackage[papersize={6.9in, 10.0in}, left=.5in, right=.5in, top=1in, bottom=.9in]{geometry}
\tolerance=5000
\hbadness=2000
%\frenchspacing

\interfootnotelinepenalty=10000
\raggedbottom

\usepackage[small]{titlesec}
\titleformat{\section}[block]
  {\fillast\medskip}
  {{\bf\thesection. }}
  {.5ex minus .1ex}
  {\bf}
 
\titleformat*{\subsection}{\itshape}
\titleformat*{\subsubsection}{\itshape}

\begin{document}

\title{Genius and Biographers: The Fictionalization of Evariste Galois}
\author{Tony Rothman}
\date{}
\maketitle

\section{Introduction}

In Paris, on the obscure morning of May 30, 1832, near a pond not far from the pension Sieur Faultrier, Evariste Galois confronted an adversary in a duel to be fought with pistols, and was shot through the stomach. Hours later, lying wounded and alone, Galois was found by a passing peasant. He was taken to the Hospital Cochin where he died the following day in the arms of his brother Alfred, after having refused the services of a priest. Had Galois lived another five months, until October 25, he would have attained the age of twenty-one. 

The legend of Evariste Galois, one of the creators of group theory, has fired the imagination of generations of mathematics students. Many of us have experienced the excitement of Freeman Dyson who writes:

\begin{quote}
In those days, my head was full of the romantic prose of E.T. Bell's \emph{Men of Mathematics}, a collection of biographies of the great mathematicians. This is a splendid book for a young boy to read (unfortunately, there is not much in it to inspire a girl, with Sonya Kovalevsky allotted only half a chapter), and it has awakened many people of my generation to the beauties of mathematics. The most memorable chapter is called ``Genius and Stupidity'' and describes the life and death of the French mathematician Galois, who was killed in a duel at the age of twenty \cite{1}
\end{quote}
Dyson goes on to quote Bell's famous description of Galois's last night before the duel:

\begin{quote}
All night long he had spent the fleeting hours feverishly dashing off his scientific last will and testament, writing against time to glean a few of the great things in his teeming mind before the death he saw could overtake him. Time after time he broke off to scribble in the margin ``I have not time; I have not time,'' and passed on to the next frantically scrawled outline. What he wrote in those last desperate hours before the dawn will keep generations of mathematicians busy for hundreds of years. He had found, once and for all, the true solution of a riddle which had tormented mathematicians for centuries: underwhat conditions can an equation be solved? \cite{2}
\end{quote}
This extract is likely the very paragraph which has given the greatest impetus to the Galois legend. As with all legends the truth has become one of many threads in the embroidery. E.T. Bell has embroidered more than most but he is not alone. James R. Newman, writing in {\it The World of Mathematics}, notes: ``The term {\it group} was first used in a technical sense by the French mathematician Evariste Galois in 1830. He wrote his brilliant paper on the subject at the age of twenty, the night before he was killed in a stupid duel \cite{3}.'' 

From the prospectus of the famed Bullitt archives of mathematics issued by the University of Louisville library, we learn: ``Goaded by a `mignonne' and two of her slattern confederates into a `duel of honor,' Galois was shot and killed at the age of twenty \cite{4}.'' 

In John Sommerfield's novel \emph{The Adversaries}, based on Bell's account, Galois is given the name Roger Constant and ``much fuss is made about the woman \cite{5}.''  Leopold Infeld, in his biography of Galois \cite{6}, invokes a conspiracy theory to explain Galois's death: Galois was considered one of the most dangerous republicans in Paris; the government wanted to get rid of him; a female agent provocateur set him up for the duel et cetera. Fred Hoyle, in his \emph{Ten Faces of the Universe} \cite{7}, attempts a partial inversion of the argument: Galois's ability to carry on complex calculations entirely in his head made him appear distant to others; personal animosities arose with republican friends; they began to think he was not entirely for the cause; Galois in their eyes was the agent provocateur; et cetera. All three authors, Bell, Hoyle and Infeld, invoke a political cause for the duel, with a mysterious coquette just off center.

This article is an attempt to sift some of the facts of Galois's life from the embroidery. It will not be an entirely complete account and will assume the reader is familiar with the story, presumably through Bell's version. Because these authors have emphasized the end of Galois's life, I will do so here. As will become apparent, many of the statements just cited are at at worst nonsensical, or at best have no basis in the known facts.

Although a number of the documents presented here are, I believe, translated into English for the first time, it should be emphasized that they are not new, just ignored. There is more known about Galois than recent authors admit. In the first version of this article, which appeared in the American Mathematical Monthly, I expressed the hope that some ambitious historian would find the requisite letter in an attic trunk or a newspaper clipping in the Paris archives to unravel the remaining mysteries. As it turned out, unbeknownst to me, the clipping had already been found a number of years ago. Remaining mysteries are now very few. Those of you who prefer to bask in the warm orange glow of the unknown should stop now; the rest of you can carry on.

\section{Sources}

It is not difficult to trace the story of Galois's brief life through its increasingly embellished incarnations. The primary source of information, containing eyewitness accounts and many relevant documents, is the original study of Paul Dupuy, which appeared in 1896 \cite{8}. Dupuy was a historian and the Surveillant General of the Ecole Normale. Bell, Hoyle and Infeld all cite it as an important reference but never once explicitly quote it. Indeed, Bell acknowledges that his account is based on Dupuy \cite{9} and the documents in Tannery \cite{10} (see below), but it remains unclear how much of his forebearer Bell has read, for while numerous passages are lifted bodily from Dupuy, other important information contained in the Surveillant General's biography is strangely absent. Dupuy's study itself is lacking a number of important letters and documents. Whether Dupuy was unaware of their existence or chose not to publish them I do not know. He also makes a number of minor errors in chronology. In any case, the first lesson is already learned: those who use Dupuy as their sole source of information must makes mistakes. Nevertheless, this original biography is much more complete and accurate than subsequent dilutions and contains more information than a reading of Bell, Hoyle or Infeld would even suggest. A translation of Dupuy into English should be undertaken.

Some of the documents not found in Dupuy are contained in Tannery's 1908 edition of Galois's papers \cite{10}. All can be found in the definitive 1962 edition of Bourgne and Azra \cite{11}. This volume contains every scrap of paper known to have been written by Galois, an accurate chronology, facsimiles of some of his original manuscripts and a number of relevant letters by others. When quoting Galois, I have worked exclusively from this edition. The memoirs of Alexandre Dumas \cite{12} contain a pertinent chapter, and the \emph{Lettres sur les Prisons du Paris} by Francois Vincent Raspail \cite{13} are the primary source on Galois's months in prison. Some of these letters are quoted by Dupuy and Infeld. Other references will be cited as they appear.

\section{Early Life and Louis-le-Grand}

I will not dwell at length on the first sixteen years of Galois's life, for they are reported with fair accuracy by Bell. This is not surprising; his account approaches that of a somewhat abridged translation of Dupuy. The divergences will set in later. Hence this section and the next may be taken as a rather condensed review and criticism of Bell. Infeld and Hoyle, who concentrate most of their energies on the duel, will be dealt with at the appropriate time.

Evariste Galois was born on October 25, 1811, not far from Paris in the town of Bourg-la-Reine, France. His father was Nicholas-Gabriel Galois, who was then thirty-six years of age, and his mother was Adelaide-Marie Demante. Both parents were highly intelligent and well educated in the subjects considered important at the time: philosophy, classical literature and religion. Bell points out that there is no record of mathematical talent on either side of the family. A more neutral statement should perhaps be made: no record exists in favor of or against any such talent. M. Galois did possess the gift for composing rhymed couplets with which he would amuse neighbors. This harmless activity, as Bell notes, would later cause his undoing. Evariste seems to have inherited some of his ability, participating in the fun at house parties. For the first twelve years of his life, Evariste's mother served as his sole teacher, giving him a solid background in Greek and Latin, as well as passing on her own skepticism toward religion.

In 1815, during The One Hundred Days, M. Galois was elected mayor of Bourg-la-Reine. He had been a supporter of Napoleon and, in fact, had been elected chief of the town's liberal party during Napoleon's first exile. After Waterloo, he had planned to relinquish his post to his predecessor, but the latter had left the country. Galois demanded to be either confirmed or replaced, and in the confusion managed to keep his office. He served the new King faithfully, but from his point on he met increasing resistance from the conservative elements of his town. It is probably safe to say that the younger Galois inherited his liberal ideas from his parents.

On October 6, 1823, Evariste was enrolled in the Lycee of Louis-le Grand, a famous preparatory school (which still exists) in Paris.\footnote{Note: At the time the school was called the College de Louis-le-Grand.} Both Robespierre and Victor Hugo had studied there. Louis-le-Grand is where Evariste's troubles began, where Infeld's account of his life essentially opens and where Bell introduces his theme of ``Genius and Stupidity,'' taking on the tone of a blanket condemnation of almost everyone and everything that surrounded Galois. ``Galois was no `ineffectual angel,' '' Bell writes in his introduction, ``but even his magnificent powers were shattered before the massed stupidity aligned against him, and he beat out his life fighting one unconquerable fool after another \cite{14}.'' I believe we will see the problems ran much deeper than that.

Bell's first liberties with Dupuy are minor. Bell describes Louis-le-Grand as a ``dismal horror'' and goes on to say ``the place looked like a prison and was \cite{15}.'' Admittedly, Dupuy writes that Louis-le-Grand happened to look like a jail because of its grills, but he then goes on to describe the underlying ``passions of work, academic triumph, passions of liberal ideals, passions of memories of the Revolution and Empire, contempt and hate for the legitimist reaction \cite{16}.'' Bell, by cutting Dupuy's sentence in half, has begun the slant to the negative.

At this particular time, there were undeniable problems. During Galois's first term, the students, who suspected the new provisor of planning to return the conservative Jesuits to the school, protested by staging a minor rebellion. When required to sing at a chapel service they refused. When required to recite in class, they refused. When required to toast Louis XVIII at an official school banquet, they refused. The provisor summarily expelled the forty students whom he suspected of leading the insurrection. Galois was not among those expelled, nor is it known if he was even among the rebels, but we may guess that the arbitrariness of the provisor and the general severity of the school's regime made a deep impression on him.

Nevertheless, Galois's first two years at Louis-le-Grand were marked by a number of successes. He received the first prize for Latin verse and three honorable mentions, as well as a mention in Greek for the General Concourse. At this point we witness the first of Bell's distortions of chronology to give the impression that Galois was misunderstood and persecuted. Galois was asked to repeat his third year because of his poor work in rhetoric Bell writes, ``His mathematical genius was already stirring,'' and ``He was forced to lick up the stale leavings which his genius had rejected \cite{17}.'' I cannot say with any certainty whether Galois's mathematical genius was already stirring, but it is known that Evariste did not enroll in his first mathematics course until after he had been demoted.

During his mathematics course, which he began in February 1827, Galois discovered Legendre's text on geometry, soon followed by Lagrange's original memoirs: \emph{Resolution of Numerical Equations\footnote{Here and elsewhere, read ``algebraic equations'' for ``numerical equations.''}: Theory of Analytic Functions and Lessons on the Calculus of Functions.} Doubtlessly, Galois received his initial ideas on the theory of equations from Lagrange. I do not understand why Bell claims Galois's classwork was mediocre; his instructor M. Vernier constantly writes such accolades as ``zeal and success,'' ``zeal and progress very marked \cite{18}.''

With his discovery of mathematics, Galois became absorbed and neglected his other courses. Before rolling in M. Vernier's class, typical comments about him had been:

\medskip
\begin{tabular}{ll}
Religious Duties--Good &Work--Sustained\\
Conduct--Good & Progress--Marked\\
Disposition--Happy & Character--Good, but singular
\end{tabular}
\medskip

\noindent
After a trimester in M. Vernier's class, the comments were:

\medskip
\begin{tabular}{ll}
Religious Duties--Good & Work--Inconstant \\
Conduct--Passable & Progress--Not very satisfactory \\
Disposition--Happy & Character--Withdrawn and original\\
\end{tabular}
\medskip


The words ``singular,'' ``bizarre,'' ``original'' and ``withdrawn'' would appear more and more frequently during the course of Galois's career at Louis-le-Grand. His own family began to think him strange. His rhetoric teachers would term him ``dissipated.'' Bell discusses these remarks at some length and his use of the indefinite pronoun ``they'' gives the impression that the entire faculty was aligned against Galois. A perusal of Dupuy's appendix, however, shows the negative remarks were penned, by and large, by Galois's two rhetoric teachers. Until this point in his life, I believe it is fair to say that Galois was somewhat misunderstood by his teachers in the humanities, but not that he was persecuted.

Slightly more serious problems were soon to arise. His mathematics teacher, M. Vernier, constantly implored Galois to work more systematically. His remark on one of Galois's trimester reports makes this clear: ``Intelligence, marked progress, but not enough method \cite{20}.'' Galois did not take the advice; he took the entrance examination to 'Ecole Polytechnique a year early, without the usual special course in mathematics, and failed. To Evariste, his failure was a complete denial of justice. This and subsequent rejections embittered him for life. When we examine some of his later writings, I think it will be evident that he developed not a little paranoia.

Galois did not yet give up. The same year, 1828, saw him enroll in the course of Louis-Paul-Emile Richard, a distinguished instructor of mathematics. Richard encouraged Galois immensely, even proclaimed that he should be admitted to the Polytechnique without examination. The results of such encouragement were spectacular. In april of 1929, Galois published his first small paper, ``Proof of a Theorem on Periodic Continued Fractions.'' It appeared in the {\it Annales de Gergonne.}

This paper was a minor aside. Galois had also been working on the theory of equations (``Galois theory''). On May 25 and June 1, 1829, while still only 17, he submitted to the Academy his first researches on the solubility of equations of prime degree. Cauchy was appointed referee.

We now encounter a major myth which evidently had its origins in the very first writings on Galois and which had been perpetuated by virtually all writers since. This myth is the assertion that Cauchy either forgot or lost the papers (Dupuy, Bell \cite{21}) or intentionally threw them out (Infeld \cite{22}). Recently, however, Rene Taton has discovered a letter of Cauchy in the Academy archives which conclusively proves that he did not lose Galois's memoirs but had planned to present them to the Academy in January 1830 \cite{23}. There is even some evidence that Cauchy encouraged Galois. I will discuss this letter and related events in more detail below; for now I note only that to hold Cauchy responsible for ``one of the major disasters in the history of mathematics,'' to paraphrase Bell \cite{24},is simply incorrect, and to add neglect by the Academy to the list of Galois's difficulties during this period appears entirely unwarranted.

A truly tragic blow came within a month of the submission: on the second of July, 1829, Evariste's father committed suicide. The reactionary priest of Bourg-la-Reine had signed Mayor Galois's name to a number of maliciously forged epigrams directed at Galois's own relatives. A scandal erupted. M. Galois's good nature could not stand such an attack and he asphixiated himself in his Paris apartment ``not two steps from Louis-le-Grand.'' During the funeral, when the same clergyman attempted to participate, a small riot erupted. The loss of his father may explain much of Galois's future behavior. We must wait a few years, until Evariste's second prison term, to see this. In any case, he loved his father dearly, and if an iron link had not already been forged between the Bourbon government and the Jesuits, it had now.\footnote{Probably the clearest picture of the relationship between the Jesuits and the Bourbons, one which contains episodes paralelling that of M. Galois's misfortunes, is Stendhal's famous novel \emph{The Red} and \emph{The Black.}}

But Galois's troubles were not yet over. A few days later, he failed his examination to l'Ecole Polytechnique for the second and final time. Legend has it that Galois, who worked almost entirely in his head and who was poor at presenting his ideas verbally, became so enraged at the stupidity of his examiner that he hurled an eraser at him. Bell records this as a fact \cite{25} but according to the little-known study of Joseph Bertrand \cite{26} the tradition is false. Bertrand, who appears to have detailed information about the event, records that Galois, while expounding on the properties of logarithmic series, refused to prove his statements to the examiner M. Dinet and, in response to Dinet's questions, replied merely that the answer was completely obvious. So was the result.

The examination failures, as well as the misunderstanding of his humanities teachers, left Galois irrevocably embittered. Bell quotes him as writing, ``Genius is condemned by a malicious social organization to an eternal denial of justice in favor of fawning mediocrity \cite{27}.'' I believe Bell constructed this quotation from a passage of Dupuy's \cite{28} but Galois did express similar sentiments in his fragmentary essay ``Sciences, Hierarchie: Ecoles'' and in ``Sur l'Enseignment des Sciences'' (``Hierarchy is a means for the inferior \cite{29}.''). In Bell's diatribe against this famous examination, as well as in other accounts of it, the suggestion that the death of Galois's father several days before may have had something to do with the outcome never arises. It is a simple matter for Bell to lay the fault squarely with the examiner's stupidity because Bell has placed the examination before M. Galois's unfortunate suicide. In this instance, Bell is not fully to blame; Dupuy does not date the examination \cite{30}. I do not wish to suggest Galois should have been failed. I only wish to point out that the examination must have been held under the worst possible conditions.

Thus, Galois's secondary school career ended in a string of minor setbacks and two major disasters. Evariste had not planned to take the Baccalaureate examinations, because the Ecole Polytechnique did not require them. Now, having failed the Polytechnique's entrance examination and having decided to enter the less prestigious Ecole Normale\footnote{Then called the l'Ecole Preparatoire. The faculty of the Polytechnique before 1830 included, among others, Lagrange, Laplace, Fourier, Ampere, Cauchy, Coriolis, Poisson and Gay-Lussac. It is not surprising that young scientists wished to enroll.}, he was forced to reconsider. ``Still persecuted and maliciously misunderstood by his preceptors,'' in Bell's words, ``Galois prepared himself for the final examinations \cite{31}.'' Despite such malice, Galois did very well in mathematics and physics, though less well in literature. He received both a Bachelor of Letters and a Bachelor of Science on the twenty-ninth of December, 1829.

It is interesting to note that, although he has continued to play the role of muckraker of malice, Bell fails to mention M. Richard's distinct cooling toward Galois, on whom he had previously bestowed high encomia. After the first trimester of the 1828-1829 academic year, Richard wrote: ``The student is markedly superior to all his classmates.'' After the second: ``This student works only in the highest realms of mathematics.'' After the third: ``Conduct good, work satisfactory.''

Because I do not have an accurate date for this report, I cannot propose a specific event as the cause of this obvious change in attitude. Presumably it occurred in the spring of 1829, shortly before or after Galois's time of troubles began. One could of course argue that M. Richard had simply become bored with Galois. Otherwise it does serve to show that Bell's black-and-white presentation of Galois's preceptors is an oversimplification.

\section{L'Ecole Normale}

The early months of 1830, which saw Galois officially enrolled as a student at l'Ecole Normale, also witnessed an interesting series of transactions with the Academy. Recall that Galois had submitted his first researches to the Academy on May 25 and June 1 of 1829. On January 18, 1830, Cauchy wrote the previously mentioned letter discovered by Taton \cite{32}:

\begin{quote}
I was supposed to present today to the Academy first a report on the work of the young Galois, and second a memoir on the analytic determination of primitive roots in which I show how one can reduce this determination to the solution of numerical equations of which all roots are positive integers. Am indisposed at home. I regret not being able to attend today's session, and I would like you to schedule me for the following session for the two indicated subjects. Please accept my homage...  \hfill A.-L. Cauchy
\end{quote}
This letter makes it clear that, six months after their receipt, Cauchy was still in possession of Galois's manuscripts, had read them and was very likely aware of their importance. At the following session on 25 January, however, Cauchy, while presenting his own memoir, did not present Galois's work. Taton hypothesizes that between January 18 and January 25, Cauchy persuaded Galois to combine his researches into a single memoir to be submitted for the Grand Prize in Mathematics, for which the deadline was March 1. Whether or not Cauchy actually made the suggestion cannot yet be proved, but in February Galois did submit such an entry to Fourier in his capacity of perpetual secretary of mathematics and physics for the Academy. In any event, there is an additional piece of evidence which attests to Cauchy's appreciation of Galois's work. This is an article which appeared the following year on 15 June 1831 in the Saint-Simonian journal {\em Le Globe.} The occasion was an appeal for Galois's acquittal after his arrest following the celebrated banquet at the \emph{Vendanges des Bourgogne}:

\begin{quote}
Last year before March 1, M. Galois gave to the secretary of the Institute a memoir on the solution of numerical equations. This memoir should have been entered in the competition for the Grand Prize in Mathematics. It deserved the prize, for it could resolve some difficulties that Lagrange had failed to do. \emph{M. Cauchy had conferred the highest praise on the author about this subject.} And what happened? The memoir is lost and the prize is given without the participation of the young savant. [Taton's Italics]\footnote{Note: My own interpretation of this article is slightly different than that of Taton. Taton writes that the journalist evidently had first-hand information. But note the date: 15 June 1831. In the aftermath of the July revolution, Cauchy fled France in september, almost nine months prior to the article's publication. It is difficult to see when the journalist would have spoken to Cauchy. However, the article appeared in a Saint-Simonian journal. Galois's best friend, Auguste Chevalier, was one of the most active Saint-Simonians. My own suspicion, which I cannot prove, is that the journalist was Chevalier and the information was coming directly from Galois. If this hypothesis is correct, Galois himself is admitting Cauchy's encouragement.}
\end{quote}

The misfortune referred to above was the death of Fourier on May 16, 1830. Galois's entry could not be found among Fourier's papers and in Galois's eyes this could not be an accident. ``The loss of my memoir is a very simple matter,'' he wrote. ``It was with M. Fourier, who was supposed to have read it and, at the death of this savant, the memoir was lost \cite{34}.'' It was an unfortunate coincidence; however, it was not Fourier's sole responsibility to read the manuscript, for the committee appointed to judge the Grand Prize consisted also of Lacroix, Poisson, Lengendre and Poinsot \cite{35}. I mention this because a number of sources give the impression that somehow Fourier either intentionally ``lost'' the paper or could not understand it \cite{36}. In spite of the setback caused by this loss of his manuscipt, April saw the publication of Galois's paper ``An analysis of a Memoir on the Algebraic Resolution of Equations'' in the \emph{Bulletin de Ferussac.} In June he published ``Notes on the Resolution of Numerical Equations'' and the important article ``On the Theory of Numbers.'' \cite{37}

In addition to propagating the legend that Cauchy lost the manuscripts, Bell, curiously, does not mention Fourier by name in the preceding misadventure, although Dupuy is explicit on the identity of the Academy's Perpetual Secretary. Perhaps Bell felt it a little too much to ``expose'' Cauchy, Fourier and later Poisson as incompetents. Bell also does not make it clear that the papers listed above (plus a later memoir) constitute what is now called Galois theory. If this point had been clarified, the claim that Galois had written down the theory on the eve of the duel would be difficult to substantiate or even to suggest.

From this point onward, the scenario of Galois as a passive victim to negligence, misunderstanding and bad luck begins to break down---if it has not already. More and more he participated in the creation of his own disasters. But this picture does not fit Bell's plan. Therefore chronology is rearranged, events are omitted and others invented in increasing quantity, until the end of his account is largely fantasy. The wholesale reordering of events will be especially evident in what follows.

Most important, Bell gives an extremely late start to Galois's political activities. He remarks that had Evariste's teachers at Louis-le-Grand allowed him to study only mathematics, he might have lived to be eighty \cite{38}. Unlikely. According to Dupuy, one of the reasons Galois had hoped to attend the Polytechnique was to participate in political activities. At l'Ecole Normale he became a ``polytechnician in exile.'' The July revolution of 1830 reared its head. The Director of l'Ecole Normale, M. Guigniault, locked the students in so that they would not be able to fight on the streets. Galois was so incensed at the decision that he tried to escape by scaling the walls. He failed, and in doing so missed the revolution. Afterwards, the Director put the students in the service of the provisional government. Charles X had fled France. He would be followed in September by Cauchy. Louis-Philippe was the new King.

The events of July Bell chronicles accurately. He does fail to point out that Galois probably joined the Society of Friends of the People, one of the most extreme republican secret societies, within the next month, certainly before December \cite{39}. The importance of this omission will be explained after I fill in the remaining gaps of the narrative as they took place historically. In December of that year, M. Guigniault was engaging in polemics against students in the pages of several newspapers. Galois saw his chance for attack and jumped into the squabble with a blistering letter to the \emph{Gazette des Ecoles.} It read in part:

\begin{quotation}
\noindent
Gentlemen:
\medskip

The letter which M. Guignault inserted yesterday in the Lycee on the occasion of one of the articles in your journal has seemed to me very inappropriate. I had thought that you would welcome with eagerness every means to expose this man.

Here are the facts which can be verified by forty-six students.

On the morning of July 28, when many of the students wished to leave the school and fight, M. Guigniault told them on two occasions that he would call the police to reestablish order wihin the school. The police on the 28th of July!

On the same day, M. Guigniault told us with his usual pedantry: ``There are many brave men fighting on both sides. If I were a soldier I would not know what to decide--to sacrifice liberty of LEGITIMACY?''

Here is the man who the next day covered his hat with an immense tricolor cockade. Here are our liberal doctrines \cite{40}!
\end{quotation}
Galois continues. According to Dupuy, every statement in the letter is accurate. Nevertheless, the result is what might have been anticipated: Galois was expelled. The action was to become official on January 4, but Galois quit school immediately and joined the Artillery of the National Guard, a branch of the militia which was almost entirely composed of republicans. It is interesting to note that the forty-six students referred to in the letter actually published a reply against Galois, but this seems to have been at the ``prompting'' of M. Guigniault \cite{41}.

December was a turbulent month for other reasons. After the Bourbons had fled France, four of their ex-ministers were to be tried for treason. Popular sentiment called for their execution. The decision to execute or imprison for life was to be announced on December 21. That day, the Artillery of the National Guard was stationed in the quadrangle at the Louvre. Galois was certainly there. The atmosphere was very tense. If the ministers were given a life sentence, the artillerymen had planned to revolt. But the Louvre was soon surrounded by the full National Guard and troops of the line, more trustworthy arms of the military. A distant cannon shot was heard. It signaled the end of the trial and that the ministers had indeed been given imprisonment over execution. The artillerymen and the National Guard readied themselves for bloodshed, but with the arrival at the Louvre of thousands of Parisians, the fighting did not erupt. Over the next few days, the situation in Paris grew calmer with the appearance of Lafayette, who called for peace, and daily proclamations calling for order. On December 31, 1830, the Artillery of the National Guard was abolished by a royal decree in fear of its threat to the throne \cite{42}.

In January 1831 Galois, no longer a student, attempted to organize a private class in algebra. At the first meeting, about forty students appeared \cite{43} but the endeavor did not last long, evidently because of Galois's political activities. On the 17h of that month, upon the invitation of Poisson, Galois submitted a third version of his memoir to the Academy. Later, in July, Poisson would reject the manuscript. This rejection will be discussed at the proper time, but we should note that by that time Galois would have already been arrested.

If we now leave history to return to Bell's account, we find a totally distorted chain of events: The months after July are missing; Galois still has not joined the Society of Friends of the People. He leaves school in December but has not joined the artillery. The events at the Louvre, which will turn out to have critical importance for the remainder of the story, never take place. Galois attempts to organize his private course in mathematics and, in Bell's words, ``Here he was at nineteen, a creative mathematician of the first rank, peddling to no takers.... Finding no students, Galois temporarily abandoned mathematics and joined the artillery of the National Guard....'' \cite{44} According to Bell, Galois submits his paper to Poisson, it is rejected; and this being the ``last straw,'' Galois decides to devote ``all his energy to revolutionary politics.''

The chronology presented by Bell is thus completely backwards. The impression given by this rearrangement of events is once again that of a misunderstood and persecuted Galois who, surrounded on all sides by idiots, finally gives up and goes into radical politics. By writing that Galois found no students, Bell of course strengthens this impression. A more balanced account clearly requires what is lacking in Bell: a Galois of volition. We may get a better indication of his character and behavior during the spring of 1831 from a letter written on April 18 by the mathematician Sophie Germain to her colleague Libri \cite{45}:

\begin{quote}
...Decidedly there is a misfortune concerning all that touches upon mathematics. Your preoccupation, that of Cauchy, the death of M. Fourier, have been the final blow for this student Galois who, in spite of his impertinence showed signs of a clever disposition. All this has done so much that he has been expelled from l'Ecole Normale. He is without money and his mother has very little also. Having returned home, he continued his habit of insult, a sample of which he gave you after your best lecture at the Academy. The poor woman fled her house, leaving just enough for her son to live on, and has been forced to place herself as a companion in order to make ends meet. They say he will go completely mad and I fear this is true.
\end{quote}
Unfortunately, as Bell observes, Galois was no ineffectual angel.

Before continuing, another historical detail should be mentioned. As an aftermath of the December events at the Louvre and the dissolution of the Artillery of the National Guard, nineteen officers were arrested, having beeen suspected of planning to deliver their cannons to the people. The charge was conspiracy to overthrow the government. In April, all nineteen were acquitted.

Until now my criticism has been devoted almost entirely to Bell. Partly this has been because his account is by far the most famous but there are other reasons as well. Hoyle's short essay, as already mentioned, is purely concerned with Galois's death and thus has little to say concerning the foregoing events. Infeld's account, on the other hand, is of book length. In a single article it would be difficult to debate all salient points. Nonetheless, Infeld has also stated \cite{46} that he is primarily concerned with the events surrounding the duel. It is then reasonable to devote attention here to that aspect of the book. Infeld's work is actually something of a curiosity. The bulk of it is a fictionalized biography, interspersed with real documents and eyewitness accounts. All dates, names and places are respected. The second part of the biography consists of a lengthy Afterword in which Infeld details exactly what he has invented, what he has not and what he believes to be true. He also includes a fairly comprehensive bibliography. In my criticisms of Infeld to follow, I only take issue with what he claims not to have invented. The reader may get the flavor of the author's intent by noting that at Galois's private algebra class, spoken of earlier, Infeld has stationed two police spies \cite{47}.

It might also be noted that, according to James R. Newman's brief remark quoted in the Introduction, Galois at this point in the narrative would be dead.

\section{Arrest and Prison}

And thus we arrive on May 9, 1831. The occasion was the republican banquet held at the restaurant \emph{Vendanges des Bourgogne}, where approximately two hundred republicans were gathered to celebrate the acquittal of the nineteen republicans on conspiracy charges. As Dumas says in his memoirs, ``It would be difficult to find in all Paris, two hundred persons more hostile to the government than those to be found reunited at five o'clock in the afternoon in the long hall on the ground floor above the garden \cite{48}.'' It is worth quoting Bell's description of this event:

\begin{quote}
The ninth of May, 1831, marked the beginning of the end. About two hundred young republicans held a banquet to protest against the royal order disbanding the artillery which Galois had joined. Toasts were drunk to the Revolutions of 1789 and 1793, to Robespierre, and to the Revolution of 1830. The whole atmosphere of the gathering was revolutionary and defiant. Galois rose to propose a toast, his glass in one hand, his open pocket knife in the other. ``To Louis-Philippe--the King.'' His companions misunderstood the purpose of the toast and whistled him down. Then they saw the open knife. Interpreting this as a threat against the life of the King, they howled their approval. A friend of Galois, seeing the great Alexander Dumas and other notables passing by the open windows, implored Galois to sit down, but the uproar continued. Galois was the hero of the moment, and the artillerists adjourned to the street to celebrate their exhuberance by dancing all night. The following day Galois was arrested at his mother's house and thrown into prison at Sainte-Pelagie \cite{49}.
\end{quote}
Dumas himself describes this event at length in his memoirs. Here is a portion:

\begin{quotation}
Suddenly, in the midst of a private conversation which I was carrying on with the person on my left, the name Louis-Philippe, followed by five or six whistles, caught my ear. I turned around. One of the most animated scenes was taking place fifteen or twenty seats from me.

A young man who had raised his glass and held an open dagger\footnote{Literally, poignard} in the same hand was trying to make himself heard. He was Evariste Galois, since killed by Pescheux d'Herbinville, a charming young man who made silk-paper cartridges which he would tie up with silk ribbons.\footnote{This is a literal translation of Dumas. We have not been able to discover exactly to what this occupation refers but it is a plausible guess that d'Herbinville made what the British call ``crackers'' (French, diablotins), party favors that pop when the ribbons are pulled and contain inspirational messages. They seem to have been invented at about this time.}

Evariste Galois was scarcely 23 or 24 at the time. He was one of the most ardent republicans. The noise was such that the very reason for this noise had become incomprehensible. 

All I could perceive was that there was a threat and that the name of Louis-Philippe had been mentioned: the intention was made clear by the open knife. 

This went way beyond the limits of my republican opinions. I yielded to the pressure from my neighbor on the left who, as one of the King's comedians, didn't care to be compromised, and we jumped from the window sill into the garden. 

I went home somewhat worried. It was clear this episode would have its consequences. Indeed, two or three days later, Evariste Galois was arrested. \cite{50}
\end{quotation}

The amusing discrepancies between the two accounts are not entirely difficult to explain. Bell has taken his description almost word for word from Dupuy, who in turn has based his account on Dumas and the report in the Gazette des Ecoles \cite{51}. The toasts Bell mentions as well as the description of the general atmosphere are found in Dupuy. But Bell has mistranslated: Dupuy writes that ``... \emph{Dumas et quelques autre passaient par le fenetre dans le jardin pour ne pas se compromettre.}..'' \cite{52} which, in this context, means ``...Dumas and several others jumped through the window in order not to be compromised.'' It does not here mean, ``Dumas and several others passed by the open window in order not to be compromised.'' Did Bell ask himself why Dumas should be passing by open windows in order not to be compromised? I do not know. He has also distorted the reason for the banquet. Dupuy clearly states \cite{53} that it was a celebration for the acquittal of the nineteen conspirators. But Bell has not mentioned the trial. For consistency's sake (such as there remains), he must therefore emphasize the obviously revolutionary character of the gathering.

The issue of accuracy becomes more important when we question the most glaring omission in Bell's account: the absence of any mention of Pescheux d'Herbinville. The single sentence in Dumas is the only extant evidence that d'Herbinville was the man who eventually shot Galois. Apparently, Bell has not read Dumas (otherwise he might have seen fit to close the discrepancies in the banquet accounts--in order not to be compromised) but he does claim to have read Dupuy who explicitly names d'Herbinville as Galois's adversary \cite{54}. Hoyle is guilty of the same charge; listing Dupuy as a main reference, he relegates d'Herbinville to the ranks of anonymous assassins. Infeld, who does identify d'Herbinville, attempts to prove he was a police agent.

For mathematics of course it is not important to know exactly who killed Galois; for historical accuracy it is. In light of the plethora of theories which have arisen to explain the cause of the celebrated duel, most of which involve police spies, agents provocateurs and political overtones, the identity of d'Herbinville might be a key piece of information. There are, in fact, two main candidates for the role of Galois's adversary and all evidence indicates that neither were police agents. Quite to the contrary. But for now let us postpone conspiracy theories and return to Galois.

Evariste was arrested at his mother's house the day following the banquet, which does indicate that police or informers were at the dinner, although the celebration was open to any subscriber. Galois was held in detention at Sainte-Pelagie prison until June 15, when he was tried for threatening the King's life. Bell's description of this event is highly oversimplified. Indeed, the defense lawyer did claim that Galois had actually said, ``To Louis-Philippe, {\it if he betrays},'' but that the noise had been such to drown out the qualifying clause. Nevertheless, the matter took on a less facetious aspect when the prosecutor asked Galois if he really intended to kill the King. Galois replied, ``Yes, if he betrays.'' The prosecutor goes on to ask how Galois ``can believe this abandonment of legality on the part of the King,'' and Galois answers, ``Everything makes us believe he will soon turn traitor if he has not done so already.'' Galois is asked to clarify his remarks and basically repeats what he has already said: ``I will say that the trend in government can make one suppose that Louis-Philippe will betray one day if he hasn't already.''

As Dumas aptly remarks, ``One understands that with such lucidity in the questions and answers, the discussion did not last long.'' Apparently moved by Galois's youth, the jury acquitted him within moments. Dumas goes on to say, ``I repeat that this is a rude generation, perhaps a bit foolish, but you will recall Beranger's song \emph{Les Fous} [``The Fools'' or ``The Madmen''].'' \cite{55}

Shortly after the event, the Academy rejected Galois's memoir on the resolution of equations, this time with Poisson as referee. The rejection was written on July 4, although according to Infeld \cite{56} Galois did not receive the letter until October, when he was in prison again.\footnote{I have found no other source which either corroborates or contradicts Infeld's claim that the rejected manuscript was not received until October, three months after the actual rejection.} By this time, about eight months had passed since he had submitted the paper at Poisson's request. As we will see, Galois did not take the rejection lightly.

The cause for Galois's second arrest was preventative: On Bastille Day, July 14, 1831, he and his republican friend Duchatelet were apprehended dressed in Artillery Guard uniforms and heavily armed. Because the Artillery Guard had been disbanded on the last day of 1830 in fear of its becoming an instrument of the republicans, to wear the uniform was an outright gesture of defiance. It was also illegal. This was the charge brought against Galois, but not until the late date of October 23; he was sentenced to six months in prison. The sentence was confirmed by the court of appeals on December 3. In the meantime, Galois had been languishing in Sainte-Pelagie prison since his arrest in July.

In Bell's fierce diatribe against this arrest he does not seem to comprehend that this was not the Paris of our day but Paris one year after a revolution, when street riots were rampant, assassination attempts not uncommon and republican activity dangerous \cite{57}. The ``celebration'' Bell mentions was a republican demonstration on Bastille Day. Today such a demonstration would be considered patriotic; then it was seditious. This is exactly what the police chief decided when he went on record opposing the demonstration \cite{58}. Bell concedes, ``True, Galois was armed to the teeth when arrested, but he had not resisted arrest \cite{59}.''  More precisely, Galois was carrying a loaded rifle, several pistols and his dagger, a punishable offense even in our more moderate times. To say that he had not resisted arrest may also be inaccurate. The police came to Galois's house to detain him, but Evariste had already decamped.

Galois's predicament was not helped by his friend Duchatelet, who drew on the wall of his cell a picture of the King's head lying next to a guillotine with the inscription, ``Philippe will carry his head to your altar, O Liberty! \cite{60}'' Part of the delay in bringing Galois to trial was the fact that Duchatelet was tried first. I am not arguing here for or against the justice of Galois's arrest. I am only trying to point out that he was behaving dangerously in a dangerous time. Two forces were clearly at work here: the government's intention to deal harshly with him after his threat of regicide and his own inability to keep out of trouble.

During his stay in prison, a number of events occurred which throw further light on Galois's personality. These incidents were recorded by the republican Francois Vincent Raspail. Raspail was an early botanist, one of the first to advocate the use of the microscope to examine cell structure in plants. He also had his troubles with the Academy and was sitting next to Dumas at the May 9 banquet. An ardent republican, he refused to receive the Cross of the Legion of Honor from Louis-Philippe and during the years 1830-1836 spent a total of twenty-seven months in prison \cite{61}. Later in life, Raspail became a famous statesman. He lived to be about eighty and is now remembered by a boulevard and metro stop in Paris. One of his many arrests occurred at about the same time Galois was taken. Raspail recorded the following incidents in several of his letters. Infeld quotes him several times at great length but never explains who he was.

In a letter dated July 25, 1831 but conceivably revised for publication after Galois's death, Raspail wrote that his fellow prisoners had taunted Galois into drinking some liquor, a pastime at which he was apparently a novice:

\begin{quote}
To refuse the challenge would be an act of cowardice. And our poor Bacchus had so much courage in his frail body that he would give his life for the hundredth part of the smallest good deed. He grasps the little glass like Socrates courageously taking the hemlock; he swallows it at one gulp, not without blinking and making a wry face. A second glass is not harder to empty than the first, and then the third. The beginner loses his equilibrium. Triumph! Homage to the Bacchus of the jail! You have intoxicated an ingenous soul, who holds wine in horror \cite{62}.
\end{quote}

The scene repeats itself. This time Galois empties a bottle of brandy in a single draught. Galois, drunk, pours out his soul to Raspail in what is either a retrospective invention or haunting prophesy:

\begin{quotation}
How I like you, at this moment more than ever. You do not get drunk, you are serious and a friend of the poor. But what is happening to my body? I have two men inside me, and unfortunately, I can guess which is going to overcome the other. I am too impatient to get to the goal. The passions of my age are all imbued with impatience. Even virtue has that vice with us. See here! I do not like liquor. At a word i drink it, holding my nose, and get drunk. I do not like women and it seems to me that I could only love a Tarpeia or a Graccha.\footnote{The \emph{Encylopedia Britannica} states that Tarpeia, according to Roman legend, was the daughter of the Roman commander in charge of defending the capital against the Sabines. She offered to betray the citadel in exchange for what the Sabines wore on their left arms, i.e., their bracelets. Taking her at her word, the Sabines crushed her beneath their shields. Graccha refers to Cornelia Graccha, the mother of Tiberius and Gaius, who is remembered as their educator as well as an accomplished author in her own right. Although hostile propaganda later suggested she encouraged her sons' more revolutionary policies, she seems rather to have restrained them.} And I tell you, I will die in a duel on the occasion of some {\it coquette de bas etage}. Why? Because she will invite me to avenge her honor which another has compromised. 

Do you know what I lack, my friend? I confide it only to you: it is someone whom I can love and love only in spirit. I have lost my father and no one has ever replaced him, do you hear me...? \cite{63}

\end{quotation}

The aftermath of the episode is neither heartwarming nor pleasant; Galois in a delerium attempts suicide:

\begin{quotation}
We laid him out on one of our beds. But the fever of intoxication tormented our unhappy friend.... He would fall senseless only to raise himself with new exaltation, and he foretold sublime things which a certain reserve often rendered ridiculous. 

``You despise me, you who are my friend! You are right, but I who committed such a crime must kill myself!'' 

And he would have done it if we had not flung ourselves on him, for he had a weapon in his hands... \cite{64}
\end{quotation}
These are crucial paragraphs for the Galois legend and several points need to be made about them. Bell, in his account, says only ``Goaded beyond endurance, Galois seized a bottle of brandy, not knowing or caring what it was, and drank it down. A decent fellow prisoner took care of him until he recovered \cite{65}.'' Thus the really important parts of the episode, which tell us something about Galois's character and which bear on future events, are omitted altogether.

Later, in attempting to understand the cause of Galois's death, Dupuy remarks, ``If I credit an allusion of Raspail, Galois lost his virgin heart to {\it quelque coquette de bas etage} \cite{66}.'' Bell writes, ``Some worthless girl [``\emph{quelque coquette de bas etage}''] initiated him \cite{67}.'' Here, Bell is taking a conjecture of Dupuy based on a letter of Raspail published seven years after Galois's death purporting to record an utterance of Galois spoken in a delerium a year before the duel as a characterization of real events. This can only be termed fabrication. And it is very likely that this piece of fabrication is responsible for the widespread belief that a prostitute was the cause of Galois's death.

Infeld, in his version of the prison scene, quotes the letters far more fully than Dupuy, but jumps from ``Tarpeia and Graccha'' to ``Do you know what I lack, my friend?'' In other words he omits Galois's apparent prophecy that he will die in a duel. He also makes no comment whatsoever on Galois's suicide attempt. This selective presentation and slanting of evidence is characteristic of Infeld's book. He publishes any document or any portion of a document which does not interfere with his stated hypothesis that Galois was killed by the secret police. I will present more obvious examples later when I discuss the actual circumstances surrounding the duel.

On August 2, Raspail chronicles an interesting series of events which took place after his previous letters. On July 27, the prisoners were invited to attend a mass in memory of those killed during the July revolution a year earlier. Because many of the prisoners were political, the atmosphere was tense and an open riot was expected to erupt at any moment. A few prudent prison leaders defused the situation and two days passed without violence. At lock-up time on the 29th, a shot was heard throughout the prison, followed by cries of ``Help, murder!'' The next day, the mystery was clarified. Raspail, quoting the conversation of another prisoner with the prison superintendent, writes:

\begin{quotation}
``Here are the facts. I am one of those in the attic room of the bathing pavilion. We were quietly going to bed. The man whose bed is between two casements had his face toward the window while undressing and he was humming a tune.''

``At that moment a shot was fired from the garret opposite. We thought our comrade was dead, but he was only unconscious. Not knowing where the shot had come from, nor how serious the wound was, we called for help. For in such a room, open in all directions through six windows, a better-aimed shot would have struck down its man.'' \cite{68}
\end{quotation}

The shot, it turned out, came from a garret across the street where one of the prison guards lived. Galois was not the man who was at the window and wounded. However, he was in the same room and later thrown into the dungeon, evidently because he had insulted the superintendent, probably accusing him of having intentionally arranged the shooting. Raspail continues to record the conversation. The prisoner already quoted is talking:

\begin{quotation}

``What? You have no order to seize the guilty man [the man who fired the shot]? But you have one to throw into the dungeon both the victim of this shameful trap and the witnesses of it? It may sound insolent to say that the administration pays turnkeys to murder prisoners. But what if this insolent statement is true? And I bear witness that no other insolence has come from those who were thrown into the dungeon. This young Galois doesn't raise his voice, as you well know; he remains as cold as his mathematics when he talks to you.''

``Galois in the dungeon!'' repeats the crowd. ``Oh, the bastards! They have a grudge against our little scholar.''

``Of course they have a grudge against him. They trick him like vipers. They entice him into every imaginable trap. And then too, they want an uprising.'' \cite{69}

\end{quotation}

An uprising they got. This oblique conversation ends with the superintendent taking to his heels as the prisoners take control of the prison. The situation remains stalemated until late tht night when the infantry is called in. The prisoners surrender without violence and remarkably no one is hurt.

I have tried to present this episode in as neutral a tone as possible. Infeld interprets the shot as an assassination attempt on Galois's life and later cites it in his Afterword as his first piece of evidence that Galois was murdered by the government \cite{70}. It is agreed that the moderate government of Louis-Philippe would have liked to have been rid of all political extremists. But a conspiracy theory presumes that there exists a reason to single out a particular victim. Why Galois over Raspail? A shot was fired in a prison full of political prisoners on the verge of a riot, at night (``lock-up time''), into a room containing an unknown number of men, evidently ``aimed'' at someone else. Yes, it could have been an attempt to kill Galois. I do not find the evidence compelling. More compelling is the evidence for the absolute hatred Galois had developed for the Academy, which I feel can only be termed paranoid. And, as is not uncommon with paranoiacs, there was a kernel of justification for the behavior. At some point in October, according to Infeld, Galois was notified of Poisson's rejection of his latest manuscript on the theory of equations. Of this rejection Bertrand writes \cite{71}:

\begin{quotation}
Poisson decided to study the memoir; three months later he drew up a report that to [Galois] was a much too severe reproach. ``We have made ever effort,'' says Poisson, ``to comprehend M. Galois's proof. His arguments are neither sufficiently clear nor developed for us to judge their rigor, and we are not in a position to even give an idea of them in this report.''

In declaring that, despite all efforts, he could not succeed in comprehending [Galois's work], Poisson's sincerity is very evident, and a reading of the memoir, printed twice since then, gives a sufficient explanation [to understand Poisson's failure]. The report ends with this benevolent remark:

``The author claims that the proposition which is the subject of his memoir is part of a general theory rich in application. Often, different parts of a theory are mutually clarifying, and it is easier to understand them together than in isolation. One should rather wait for the author to publish his work in entirety before forming a definite opinion.'' Poisson refused to approve the proof, but he did not condemn it. In all fairness, he is irreproachable. He did as much as he could and was obliged to do.
\end{quotation}

Bell, elaborating from Dupuy, states that Poisson found the manuscript ``incomprehensible'' but ``did not state how long it had taken him to reach this remarkable conclusion \cite{72}.'' I believe this is an unfair characterization of Poisson's comments. This is the rejection that Bell has occurring before Galois's arrest.

In light of previous events and in light of his character, it is not terribly surprising that Galois reacted violently to what might nowadays be considered an encouraging rejection letter. He gave up all plans to publish his papers through the Academy and decided to publish them privately with the help of his friend Auguste Chevalier. Galois collected his manuscripts and in December, while still in Sainte-Pelagie, penned what must surely be one of the most remarkable documents in the history of mathematics, his Preface. The entire Preface runs about five pages. Infeld, to his credit, prints some of it, although he alters and omits some parts of it at will. I here quote only the first page. The full text can be found in Bourgne and Azra:

\begin{quote}
Firstly, you will notice the second page of this work is not encumbered by surnames, Christian names or titles. Absent are eulogies to some prince whose purse would have opened at the smoke of incense, threatening to close once the incense holder was empty. Neither will you see, in characters three times as high as those in the text, homage respectfully paid to some high-ranking official in science, or to some savant-protector, a thing thought to be indispensable (I should say inevitable) for someone wishing to write at twenty. I tell no one that I owe anything of value in my work to his advice or encouragement. I do not say so because it would be a lie. If I addressed anything to the important men of science or the world (and I grant the distinction between the two at times is imperceptible) I swear it would not be thanks. I owe to important men the fact that the first of these papers is appearing so late. I owe to other important men that the whole thing was written in prison, a place, you will agree, hardly suited for meditation, and where I have been dumbfounded at my own listlessness in keeping my mouth shut at my stupid, spiteful critics: and I think that I can say ``spiteful critics'' in all modesty because my adversaries are so low in my esteem. The whys and wherefores of my stay in prison have nothing to do with the subject at hand; but I must tell you how manuscripts go astray in the portfolios of the members of the Institute, although I cannot in truth conceive of such carelessness on the part of those who already have the death of Abel on their consciences. I do not want to compare myself with that illustrious mathematician but, suffice to say, I sent my memoir on the theory of equations to the Academy in February of 1830 (in a less complete form in 1829) and it has been impossible to find them or get them back. There are other anecdotes in this genre but I would be ungracious to recount them because, other than the loss of my manuscripts, those incidents do not concern me. Happy voyager, only my poor countenance saved me from the jaws of wolves. Perhaps I have already said too much for the reader to understand why, as much as I would have liked otherwise, it is absolutely impossible for me to embellish or disfigure this work with a dedication \cite{73}.
\end{quote}

The remainder of the Preface continues in much the same tone (``And thus it is knowingly that I expose myself to the laughter of fools''). Other of his writings are not dissimilar \cite{74}. Among his papers is the picture of a bizarre, torsoless figure, captioned by Bourgne and Azra ``Riquet a la Houppe \cite{75}'' The picture must have been drawn shortly before his death. It may be significant that Riquet a la Houppe was in French folklore a character, short, ugly, disdained by all, but nonetheless very clever.

\section{The Duel and Theories Surrounding It}

We are almost to the end of this short story. Galois remained in Sainte-Pelagie without further recorded incident until March 16, 1832, when he was transferred to the pension Sieur Faultrier. Ironically enough, this was to prevent the prisoners from being exposed to the cholera epidemic then sweeping Paris. Galois was due to be given his freedom on April 29. From this point on, the historical record is very scanty. On May 25, Galois writes to his friend Chevalier and clearly alludes to a broken love affair:

\begin{quotation}

My dear friend, there is a pleasure in being sad if one can hope for consolation; one is happy to suffer if one has friends. Your letter, full of apostolitic unction, has given me a little calm. But how can I remove the trace of such violent emotions as I have felt? 

How can I console myself when in one month I have exhausted the greatest source of happiness a man can have, when I have exhausted it without happiness, without hope, when I am certain it is drained for life \cite{76}?
\end{quotation}

The letter continues in similar tones. Galois goes on to say that he is disgusted with the world: ``I am disenchanted with everything, even the love of glory. How can a world I detest soil me?'' \cite{77}

The next few days are a complete blank. On the morning of May 30, the famous duel took place. The previous evening, Galois wrote several well-known letters to his republican friends:

\begin{quotation}
I beg patriots, my friends, not to reproach me for dying otherwise than for my country. 

I die the victim of an infamous coquette and her two dupes. It is in a miserable piece of slander that I end my life. 

Oh! Why die for something so little, so contemptible? 

I call on heaven to witness that only under compulsion and force have I yielded to a provocation which I have tried to avert by every means. I repent in having told the hateful truth to those who could not listen to it with dispassion. But to the end I told the truth. I go to the grave with a conscience free from patriots' blood. 

I would like to have given my life for the public good. 

Forgive those who kill me for they are of good faith \cite{78}.
\end{quotation}

Galois also writes another, similar letter to two republican friends, Napoleon Lebon and V. Delauney:

\begin{quotation}
\noindent
My good friends, 
\medskip

I have been provoked by two patriots...It is impossible for me to refuse. 

I beg your forgiveness for not having told you. 

But my adversaries have put me on my work of honor not to inform any patriot. 

Your task is simple: prove that I am fighting against my will, having exhausted all possible means of reconciliation; say whether I am capable of lying in even the most trivial matters. 

Please remember me since fate did not give me enough of my life to be remembered by my country. I die your friend \cite{79}.
\end{quotation}

I will return to Galois's activities during the last night later. Now I want to dispose of some of the many theories which purport to explain the cause of this celebrated duel, beginning in each case with circumstances and eventually rising to facts. If one is of a distrustful frame of mind, there is perhaps enough in the above two letters to raise suspicions of foul play. The attempts to make Galois the victim of royalists, a female agent provacateur, a prostitute or a government conspiracy doubtlessly stem from these letters for there is no other direct evidence in existence. Thus we have the origin of Bell's assertion:

\begin{quote}
What happened on May 29th is not definitely known. Extracts from two letters suggest what is usually accepted as the truth: Galois had run afoul of political enemies immediately after his release \cite{80}.
\end{quote}

The first statement is accurate, the second is not. Dupuy certainly believes the exact opposite, as will become clear below. He does mention that Alfred Galois, unjustifiably in his view, did maintain that his older brother was murdered. Because Bell ``followed'' Dupuy exclusively, one can only conclude that he took Alfred's position and termed it widely accepted or that he simply invented the whole thing.

Although Bell may have invented the theory, he is not its chief advocate. Infeld goes further. He assumes that the ``infamous coquette'' was a female agent provacateur who set up Galois for the duel with a police agent. Infeld's evidence is by admission circumstantial. In addition to the bullet episode at Sainte-Pelagie, it consists of the following \cite{81}: the police were known to have used spies; the police broke up a meeting of the Society of Friends of the People the night before Galois' funeral; Police Chief M. Gisquet wrote in 1840 that Galois ``had been killed by a friend''; police spies were unmasked in 1848, at which time a claim appeared in a journal that Galois ``had been murdered in a so-called duel of honor''; Galois's brother Alfred always maintained that Evariste had been murdered; Galois was abandoned by his adversaries and his seconds and found by a peasant.

Let me first counter circumstance by circumstance. Infeld's evidence is indeed consistent and does not contradict known facts. However, necessity does not follow from consistency. The bullet episode has already been discussed. It is true that the police used spies and that they were unmasked in 1848. I will return to this point shortly. Infeld does not mention that the newspapers announced Galois's funeral {\it before} the fact and explicitly named him as a member of both the Artillery of the National Guard and the Friends of the People. In any case, his membership in these organizations must have been widely known. One must weigh for oneself whether it is remarkable that police knew of republican meetings. Infeld finds it suspicious that the police chief, eight years later, knew that Galois had been ``killed by a friend.'' He does not find it suspicious that Dumas apparently knew more--precisely who that friend was. Dupuy feels that Alfred's position was the result of justifiable anger over his brother's death and points out some unlikely details Alfred attributed to the duel, such as stating that Evariste would have fired into the air. The assertion that Galois was abandoned to die, another of Alfred's claims, is also open to dispute. Dupuy notes that one of the witnesses went to Galois's mother the following day to explain what had happened \cite{82}. He considers it more likely, then, that the witnesses were searching for a doctor when the peasant happened along. This explanation may be weak; nevertheless Infeld fails to point out that Mme. Galois was informed.

Let us now move from circumstance to harder evidence. As we have seen, Dumas's choice for Galois's opponent--which Infeld accepts--was a ``delightful young man'' named Pescheux d'Herbinville. More is known about him than his anonymity. He was, in fact, one of the nineteen republicans who were acquitted on charges of conspiring to overthrow the government. Is there any reason to suspect d'Herbinville was a police agent? The historian Louis Blanc, in his exhaustive \emph{History of Ten Years} writes:

\begin{quote}
The trial gave rise to highly interesting scenes. In the sittings of the 7th of April, the president having reproached M. Pescheux d'Herbinville, one of the accused, for having [borne arms] and having distributed them, ``Yes,'' replied the prisoner, ``I have borne arms, a great many arms, and I will tell you how I came by them.'' Then, relating the part he had taken in the three days, he told how, followed by his comrades, he had disarmed posts, and sustained glorious conflicts: and how, though not wealthy, he had equipped national guards at his own cost. There still burned in the hearts of the people some of the fire kindled by the revolution of July; such recitals as this fanned the embers. The young man himself, as he concluded his brief defense, wore a face radiant with enthusiasm and his eyes filled with tears \cite{83}.
\end{quote}
In addition, Blanc mentions the appearance of General Layfayette during the trial:

\begin{quote}
The old general came to give his testimony in favor of the accused, almost all of whom he knew, and all saluted him from their places with looks and gestures of regard \cite{84}.
\end{quote}
D'Herbinville, it seems, was one of the heros of the hour. After the acquittal, the crowd pulled his coach through the streets of Paris, ``amid shouts of rapturous applause.''

Bell (and Hoyle below), by not mentioning d'Herbinville at all, relieve themselves of the difficulty of explaining why Galois should be killed in a political duel with a fellow republican or why d'Herbinville should be considered a political enemy. Infeld is in a more difficult position. Having acknowledged d'Herbinville's existence, he must explain why neither Dumas nor Blanc, both republicans, nor evidently the extremely liberal Lafayette\footnote{Lafayette had been considered republican enough to see his post of Commander of the National Guard dissolved after the events of December 1830.} (assuming he knew d'Herbinville personally), nor, one would gather from Blanc's account, any republican in Paris, ever held any suspicions that d'Herbinville was an agent. Infeld talks at length about the 1848 unmasking of the police spies but he does not mention the following extract from Dupuy:

\begin{quote}
Pescheux was certainly not a ``false-brother'': all the men who acted as police agents during the reign of Louis-Philippe were revealed in 1848 when Caussidiere became chief of police, as witness Lucien de la Hodde.\footnote{Hodde was a ``republican'' who was unmasked as a spy in 1848.} If Pescheux were suspect, he would certainly not have been nominated as curator of the palace of Fontainebleau. It is absolutely necessary to discard the idea of police intervention and a framed assassination \cite{85}.
\end{quote}
We see that there are, even at this level, some serious difficulties with the political enemies scenario. Infeld gets around theses obstacles in characteristic fashion: in his bibliography he cites both Blanc and Dupuy as primary sources but {\it quotes neither.} In his afterward he goes so far as to admit, ``There is no reason to believe Pescheux d'Herbinville was a police agent.'' But then he goes on to say: ``I believe there is enough circumstantial evidence to prove that the intervention of the secret police sealed Galois's fate. I do not believe it is possible to fit all the known facts without assuming Galois was murdered \cite{86}.''

While the reader is forming a rebuttal to this statement, we turn to events according to Hoyle where we find an amusing inversion of Infeld's theory. Hoyle writes:

\begin{quote}
Such are the bare bones of the story of the life and death of Evariste Galois. The classical biography of Galois [he then references Dupuy], in an attempt to add flesh to these bones, suggests that he was done to death by royalist enemies, as does E.T. Bell in his book\emph{ Men of Mathematics}. There are dark hints that the release from prison was but a device for encompassing his death, a necessary preliminary to his being matched against a highly skilled assailant in royalist pay. But why should Galois feel it critical to his honor that he should accept the challenge of a right-wing agent, especially if the agent were a known marksman? Gallic logic suggests on account of a girl...[87]
\end{quote}
We first note the complete misrepresentation of Dupuy's position. Undeterred, Hoyle then goes on to dispose of the ``infamous coquette'' and propose his own theory:

\begin{quotation}
It is possible that the ``infamous coquette'' was the source of a purely personal quarrel, but it is the normal biological rule among mammals that sexual quarrels between two males cease as soon as one side seeks ``accomodation.'' It is the normal rule that either party to such fights can simply walk away, which is just what Galois seems to have attempted to do. 

The more likely possibility is that Galois's habit of working mathematical problems in his head, his ability to think in parallel, caused serious animosities, and perhaps suspicions, to develop during his six months of imprisonment. There may have been suspicions that Galois was not wholly for the ``cause,'' or even that he was an {\it agent provocateur}...[88]
\end{quotation}
I am divided between anger and hilarity. To suggest as Hoyle does that any republican in Paris suspected Galois after his expulsion from l'Ecole Normale, his Artillery activities, his threat to the King, his arrests, trials, sentencings, resentencings and prison activities borders on the fantastical. This is in addition to the fact that two or three thousand republicans later attended the funeral of this supposed agent provocateur. One might equally well claim Lenin had been suspected of being a Menshevik.

As to Hoyle's bio-sociological theories, he is contradicted by the (circumstantial) historical record. The greatest Russian poet, Aleksandr Pushkin, was killed in 1837 at the age of 37 in a duel over his wife. England's Lord Camelford was killed in a duel over a prostitute. As late as 1838 members of the American legislature were engaging in similar duels. Toward the end of the eighteenth century, during election season, approximately 23 duels \emph{per day }were fought in Ireland {\it alone}, unlikely just for political reasons. In the last decades of the nineteeth century, Paris newspapers carried notices of daily duels and their terms. These practices continued until World War I. The causes of such ``affairs of honor'' ranged from geese, to insults, to politics, to women \cite{89}. Dupuy himself mentions that nothing was more common at the time in question than duels between republicans, and I think one can safely infer from his remarks that no one paid the slightest attention to them \cite{90}.

But this rebuttal is as circumstantial as Hoyle's argument and not very satisfactory so let us again rise to more concrete facts. The first of these is the existence of two fragmentary letters written to Galois by one Mademoiselle Stephanie D., who is none other than the ``infamous coquette'' over whom the duel was fought. A prostitute? An agent provocateur? Most authors have assumed her identity to be an absolute mystery and that she, like Galois's opponent, is an anyonymous casualty of history. Dupuy was apparently unaware of the letters or chose not to publish them. Bell and Hoyle never mention her name. Infeld calls her Eve Sorel (perhaps inspried by Stendhal). This is a strange state of affairs, for the letters are contained in Tannery's 1908 edition of Galois's papers. Tannery does not affix a name to the author of these fragments; it is left for the 1962 edition of Bourgne and Azra to attempt an identification. One can understand why Bell and Infeld did not mention her name since Tannery did not provide it. Hoyle does not have such an excuse, his book being published in 1977. One cannot understand why these letters are never mentioned by anyone, especially by Bell and Infeld who cite Tannery as a major source for Galois's manuscripts.

The letters, as they exist, are copies made by Galois himself on the back of one of his papers \cite{91}. The copies contain gaps, which may indicate he had previously torn up the originals and could not completely reconstruct them. More likely, Galois purposely omitted any incriminating or personally distasteful segments. I say this because some words in the French versions are broken in half; one generally does not remember only half a word. Galois has certainly obliterated Stephanie's last name in a fit of anger. Due to the fragmentary nature of these letters their translation has proved difficult and may be uncertain in places. Where impossible to translate we have allowed the original French to stand.

\medskip\noindent
Letter I: 

\begin{quote}
Please let us break up this affair. I do not have the wit to follow...a correspondence of this nature...but I will try to have enough to converse with you as I did before anything happened. Here is... Mr.... the.... en a... qui... doit vous... qu'a... me and do not think about those things which did not exist and which never would have existed. 

\medskip
\hfill Mademoiselle Stephanie D 14 May 183--
\end{quote}
Letter II:

\begin{quote}
I have followed your advice and I have thought over what... has... happened... on whichever denomination it may have happened between us. In any case, Sir, be assured there never would have been more. You're assuming wrongly and your regrets have no foundation. True friendship exists nearly only between people of the same sex, particularly... of... friends ... full in the vacuum that... the absence of all feeling of this kind...my trust...but it has been very wounded...you have seen me sad... you have asked the reason: I answered you that I had sorrows that one had inflicted upon me. I had thought that you would take this as anyone in front of whom one drops a word for these one is not The calm of my thoughts leaves me to judge the persons that I usually see without much reflection: this is the reason that I rarely regret having been wrong in my judgement of a person. I am not of your opinion ...les sen...plus que...les a...exiger ...ni se... thank you sincerely for all those who you would bring down in my favor.
\end{quote}
These are highly tantalizing morsels, but is there anything else known about the author? Indeed there is. C.A. Infantozzi has examined the original of the first letter \cite{92}. With the help of a magnifying glass and ``appropriate lighting'' he is able to discern Stephanie's full signature under Galois's erasures: Stephanie Dumotel. Further archival investigations by Infantozzi shows she was Stephanie-Felicie Poterin du Motel, daughter of Jean-Louis Auguste Poterin du Motel, a resident physician at the Sieur Faultrier, where Galois stayed the last months of his life. In 1840 Stephanie married Oscar-Theodore Barrieu, a language professor. Any presumption that she was a prostitute must at this point be discarded as a complete figment of Bell's imagination.

From Stephanie's second letter it is not difficult to infer that Galois took some song of sorrows on her part too seriously and himself provoked the duel. On face value she certainly does seem to have been an unwitting participant in whatever transpired. 
Unfortunately, the establishment of Stephanie's identity does not conclusively establish exactly what happened. But for those who still insist that agents provocateurs and right-wing assassins are not absolutely ruled out, I offer the following passage from Andre Dalmas's biography of Galois in which he reprints an article from the June 4, 1832 issue of the Lyon ``journal constitutionnel'' le Precurseur \cite{93}:

\begin{quote}
Paris, 1 June--A deplorable duel yesterday has deprived the exact sciences of a young man who gave the highest expectations, but whose celebrated precocity was lately overshadowed by his political activities. The young Evariste Galois, condemned for a year as a result of a toast proposed at a banquet at the {\it Vendanges des Bourgogne} was fighting with one of his old friends, a young man like himself, like himself a member of the Society of Friends of the People, and who was known to have figured equally in a political trial. It is said that love was the cause of the combat. The pistol was the chosen weapon of the adversaries, but because of their old friendship they could not bear to look at one another and left the decision to blind fate. At point-blank range they were each armed with a pistol and fired. Only one pistol was charged. Galois was pierced through and through by a ball from his opponent; he was taken to the hospital Cochin where he died in about two hours. His age was 22. L.D., his adversary, is a bit younger.
\end{quote}
It is frustrating that a report written only several days after the event is incorrect on the date of the duel, the date of Galois's death and his age; this in itself should serve as a good lesson to armchair historians. And L.D.? The frustrations do not end. The man who fits the above description most accurately is not d'Herbinville at all but Galois's old friend Duchatelet who was arrested with him on Bastille Day. On the other hand, Bourgne and Azra give Duchatelet's first name as Ernest and d'Herbinville's surname also begins with a D, bearing in mind the variable orthographic procedures of the time (e.g., du Motel vs. Dumotel), not to mention the fact that he also participated in a political trial (though not with Galois). While I cast my vote for Duchatelet, the final distinction strikes me as unimportant. With Stephanie's letters and the newspaper article we arrive at a very consistent and believable picture of two old friends falling in love with the same girl and deciding the outcome by a gruesome version of Russian roulette. This is my fairy tale. It has the virtues of simplicity and psychological truth. By comparison the tales of Bell, Hoyle and Infeld are baroque, if not byzantine, inventions. Those who want to puruse the matter further are welcome to do so. I suggest a trip to Paris to check whether Duchatelet was boarding at the Sieur Faultrier during April 1832.

\section{The Last Night}

We saw in the introduction how Bell all but states outright that Galois committed his theory of equations to paper the night before he was shot. James R. Newman repeats this as an assertion and the vision of the doomed boy, sitting by candlelight, feverishly bringing group theory into the world seems to be the major myth which most scientists harbor concerning Galois. This is again due to Bell's embellishment of Dupuy who, in this instance, is sufficiently romantic of his own accord. But as has already been detailed at great length, Galois had been submitting papers on the subject since the age of 17. The term ``group,'' used in the sense of ``group of permutations'' is found in all of them. During the night before the duel, in addition to the letters already quoted, Galois wrote a long letter to his friend Chevalier \cite{94,95,96,97}. He begins:

\begin{quotation}
\noindent
My Dear Friend,
\medskip

I have made some new discoveries in analysis. 

The first concern the theory of equations, the others integral functions. 

In the theory of equations I have researched the conditions for the solvability of equations by radicals; this has given me the occasion to deepen this theory and describe all the transformations possible on an equation even though it is not solvable by radicals. 

All this will be found here in three memoirs.
\end{quotation}

Galois then goes on to describe and elucidate the contents of the memoir which was rejected by Poisson, as well as subsequent work. Galois had indeed helped to create a field which would keep mathematicians busy for hundreds of years but not ``in those last desperate hours before the dawn.'' During the course of the night he annotated and made corrections to some of his papers. He comes across a note that Poisson had left in the margin of his rejected memoir:

\begin{quote}
The proof of this lemma is not sufficient. But it is true according to Lagrange's paper, No. 100, Berlin 1775.
\end{quote}
Galois writes directly beneath it:

\begin{quote}
This proof is a textual transcription of that which we gave for this lemma in a memoir presented in 1830. We leave as an historic document the above note which M. Poisson felt obliged to insert. (Author's note.)
\end{quote}
A few pages later, Galois scrawls next to a theorem:

\begin{quote}
There are a few things left to be completed in this proof. I have not the time. (Author's note.)
\end{quote}
This famous inscription appears only once in Bourgne and Azra. It is unfortunate that Galois tarnished some of the romance by including his parenthetical ``author's note.'' Galois ends his letter to Chevalier with the following request:

\begin{quotation}

In my life I have often dared to advance propositions about which I was not sure. But all I have written down her has been clear in my head for over a year, and it would not be in my interest to leave myself open to the suscpicion that I announce theorems of which I do not have complete proof. 

Make a public request of Jacobi or Gauss to give their opinions not as to the truth but as to the importance of these theorems. 

After that, I hope some men will find it profitable to sort out this mess. 
I embrace you with effusion. {\it E. Galois}
\end{quotation}
And that was the end. The funeral was to be held on June 2. During the previous evening, the police broke up a meeting of the Society of Friends of the People on the pretext that the republicans wree planning a demonstration for Galois's funeral. Thirty of those present were arrested. The next day two or three thousand people were present at the services. Galois's body was interred in a common burial ground of which no trace remains today.

Later, Evariste's brother Alfred and his devoted friend Chevalier would laboriously recopy the mathematical papers and submit them to Gauss, Jacobi and others. By 1843 the manuscripts had found their way to Liouville who, after spending several months in an attempt to understand them, became convinced of their importance. He published the papers in 1846. There exist many fragments which indicate Galois carried on his mathematical researches, not only while in prison, but right up to the time of this death. The fact that he could work through such a turbulent life is testimony to the extraordinary fertility of his imagination. There is no question that Galois was a great mathematician who developed one of the most original idea in the history of mathematics. The invention of legends does not make him any greater.

\section{Harsher Words}

The account of Galois's life given here has not been entirely complete. There are more documents, letters and events. No doubt I will shortly be exposed for having selectively presented evidence. The purpose of this paper, however, has not been one of completeness, nor entirely one of biography. No, the purpose has been to show that something is curiously out of sync. Two highly respected physicists and an equally well-known mathematician, members of the professions which most loudly proclaim their devotion to Truth, have invented history.

Bell's account, by far the most famous, is also the most fictitious. It is a a myth devoid of such complications as a protagonist who is faulted as well as gifted. It is a myth based on the stereotype of the misunderstood genius whom the conservative hierarchy is out to conquer. As if the befuddled hierarchy is generally organized well enough for persecution. It is a myth based on a misunderstanding of the method by which a scientist works: as if a great theory could be written down coherently in a single night.

As an inventor of fairy tales, one can enjoy Bell; as a biographer it is unclear how far one can forgive him. Surely all his mistakes did not result from a poor knowledge of French. No, I believe Bell saw his opportunity to create a legend. The details which are absent in his account, such as Dumas at the banquet, such as d'Herbinville and the suicide attempt and Raspail, are those details which lend a concreteness and a humanness to Galois's life which a legend (at least a bad one) must not have. Unfortunately, if this was Bell's intent, he succeeded. After hearing of my investigation, physicists and mathematicians all open conversations with the same question: ``Did Galois really invent group theory the night before he was killed?'' No, he didn't.

Infeld presents far more details. He is not interested in making Galois a legend. He does intend to make Galois a hero of the people. Politics is the guiding principle for Infeld. His book might be termed the proletarian interpretation of Galois; certainly parts of it read like the local Workers' Party publication. Infeld is very good at covering his tracks. To delete a phrase here, a paragraph there, a counterargument in between, is all that is necessary to create conspiracy from chaos. As to Hoyle's motives, we can only take him at his word: He describes at length how as a child he was taught arithmetic by his mother, how he became proficient at mathematics and how school for him became an excruciating bore. Hoyle was forced to learn to ``think in parallel'' in order to fool the teacher into believing he paid attention in class. He then tells us, ``I mention these personal details because I believe they cast some light on the mysterious death of the French mathematician Galois.'' Further comment seems unnecessary.

Dupuy appears to have much less of a vested interest. I assume he included all the documents known to him at the time. If not, then he too should be scrutinized more carefully. He does seem unwilling to accept a conspiracy theory. At the very least, the three twentieth-century authors are guilty of distorting Dupuy's account and even falsifying it. In each case the story of Galois has been used to put a stamp of approval on the author's personal theories. Indeed, all history is interpretive. But if we do not approve, we understand the liberty: Galois, like Einstein, has passed into the public domain. No act or anecdote attributed to him is too outrageous to be given consideration. There is a closer analogy from farther afield. The Russian composer Reinhold Gliere once wrote a symphony, his third, which ran well over an hour. Stokowski--the story goes--worked with Gliere to edit the score down to manageable length. Since then every conductor presents his own edition. I do not know if I have ever heard the original.

The investigations of Galois discussed here have told us less about the man than about his biographers. The misfortune is that the biographers have been scientists. Because they appreciate his genius a century after its undisputed establishment, anyone who did not recognize it at the time is condemned. ``In all the history of science,'' writes Bell, ``there is no completer example of the triumph of crass stupidity over untamable genius.'' 

``Is it possible to avoid the obvious conclusion,'' asks Infeld, ``that the regime of Louis-Philippe was responsible for the early death of one of the greatest scientists who ever lived?'' The underlying assumption is apparent: Galois was persecuted because he was a genius and all scientists, to a greater or lesser degree, understand that genius is not tolerated by mediocrity. A genius must be recognized as such even when standing drunk at a banquet table with a dagger in his hand. Anyone who does not recognize him becomes a fool, an assassin or a prostitute. This is a presumption of the highest arrogance. Scientists should not be so enamored of themselves.

\section*{Acknowledgements}

I would like to thank Leonard Gillman for suggesting that I consolidate my researches on Galois into this article. Thanks to Cecile DeWitt-Morette for the gracious gift of her time in translating, and most of all to Marc Henneaux for his translations, patience, discussions and enthusiasm.

For this version I am grateful to Rene Taton for sending me some new information, which has changed a few of the facts, though not the conclusions, of the original American Mathematical Montly article.

When the Original version appeared, Monthly editor Ralph Boas would not allow me to thank him for his great help in supplying material and checking facts. I hope he will not mind if I thank him now. 

\renewcommand{\refname}{Notes}

\begin{thebibliography}{9999}
\small

\bibitem{1} Freeman Dyson, \emph{Disturbing the Universe}(New York: Harper and Row, 1979), p. 14. 

\bibitem{2} E.T. Bell, {\it Men of Mathematics} (New York: Simon and Schuster, 1937), p. 375. 

\bibitem{3} James R. Newman, {\it The World of Mathematics} (New York: Simon and Schuster, 1956), vol. 3, p. 1534. 

\bibitem{4} \emph{Checklist of the Bullitt Collection of Mathematics} (University of Louisville, 1979). 

\bibitem{5} John Sommerfield, \emph{The Adversaries} (London:Heinemann, 1952). I thank H. Schwerdtfeger for calling my attention to
this book. 

\bibitem{6} Leopold Infeld, \emph{Whom the Gods Love: The Story of Evariste Galois} (New York: Whittlesey House, 1948). 

\bibitem{7} Fred Hoyle, \emph{Ten Faces of the Universe} (San Fransico: W.H. Freeman, 1977), Chap. 1. 

\bibitem{8} Paul Dupuy, \emph{La Vie d'Evariste Galois}, Annales de l'Ecole Normale , 13, 197-266 (1896). 

\bibitem{9} Bell, p. vii. 

\bibitem{10} Jules Tannery, ed., \emph{Manuscrits d'Evariste Galois} (Paris: Gauthier-Villars, 1908). 

\bibitem{11} Robert Bourgne and J.P. Azra, eds., \emph{Ecrits et Memoires Mathematique d'Evariste Galois: Edition Critique Integr
ale de ses Manuscrits et Publications} (Paris: Gauthier-Villars, 1962). 

\bibitem{12} Alexandre Dumas, \emph{Mes Memoirs} (Paris: Editions Gallimard, 1967), vol. 4., Chap. 204. 

\bibitem{13} Francois Vincent Raspail, \emph{Lettres sur les Prisons de Paris} (Paris: 1839), vol. 2. 

\bibitem{14} Bell, p. 362. 

\bibitem{15} Bell, p. 363. 

\bibitem{16} Dupuy, p. 203. 

\bibitem{17} Bell, p. 364. Compare with Dupuy, p. 205. 

\bibitem{18} Dupuy, pp. 255-256. 

\bibitem{19} Dupuy, pp. 254-255. 

\bibitem{20} Dupuy, p. 256. 

\bibitem{21} Dupuy, p. 209; Bell, p. 368. 

\bibitem{22} Infeld, p. 306. 

\bibitem{23} Rene Taton, ``Sur les relations scientifiques d'Augustin Cauchy et d'Evariste Galois,'' \emph{Revue d'Histoire des
Sciences} 24, 123 (1971). 

\bibitem{24} Bell, p. 368. 

\bibitem{25} Bell, p. 369. 

\bibitem{26} Joseph Bertrand, ``La Vie d'Evariste Galois par Paul Dupuy,'' \emph{Eloges Academique, nouv. serie}, Paris, 329-345 (1902). 

\bibitem{27} Bell, p. 371. 

\bibitem{28} Dupuy, p. 217. 

\bibitem{29} Bourgne and Azra, p. 27. See also pp. 21-25. 

\bibitem{30} I assume here, as elsewhere, the chronology given by Bourgne and Azra, pp. XXVII-XXXI. 

\bibitem{31} Bell, p. 370. 

\bibitem{32} Taton, p. 134. 

\bibitem{33} Taton, p. 139. 

\bibitem{34} Bourgne and Azra, p. XXVIII. 

\bibitem{35} Taton, p. 138. 

\bibitem{36} See, for example, Lilian Lieber, \emph{Galois and the Theory of Groups} (1932).

\bibitem{37} Bourgne and Azra, p. XXVIII. 

\bibitem{38} Bell, p. 366. 

\bibitem{39} Dupuy, p. 221, and Bourgne and Azra, p. XXIX.

\bibitem{40} Bourgne and Azra, p. 462. Also Dupuy, p. 225. Translated in part by Infeld, p. 155. 

\bibitem{41} Dupuy, pp. 227-228. 

\bibitem{42} This account is based on information from Louis Blanc, \emph{History of Ten Years} (London: Chapman and Hall, 1844). A
consistent account, also based on Blanc, is given by Infeld, Chap. 5. 

\bibitem{43} Dupuy, p. 234. 

\bibitem{44} Bell, p. 372. 

\bibitem{45} C. Henry, ``Manuscrits de Sophie Germain,'' \emph{Revue Philosophique}, 631 (1879). 

\bibitem{46} See, for example, his Afterword. 

\bibitem{47} Infeld, p. 169. 

\bibitem{48} Dumas, p. 331. 

\bibitem{49} Bell, p. 372. 

\bibitem{50} Dumas, pp. 332-333. 

\bibitem{51} Dupuy, pp. 234-235. 

\bibitem{52} Dupuy, p. 235. 

\bibitem{53} Dupuy, p. 234. 

\bibitem{54} Dupuy, p. 247. 

\bibitem{55} Because of a change of libraries, this account of the trial is based on a different edition of Dumas's memiors
: Alexandre Dumas, {\it Mes Memoirs} (Paris: Union Generale d'Editions), vol. 2, chap. 37. (No copyright date given.) 

\bibitem{56} Infeld, p. 230. 

\bibitem{57} See, for example, T.E.B. Howarth, \emph{Citizen King: The Life of Louis-Philippe} (London: Eyre \&
Spottiswoode, 1961). 

\bibitem{58} Dupuy, p. 238. 

\bibitem{59} Bell, p. 378. 

\bibitem{60} Dupuy, p. 238. 

\bibitem{61} Dora B. Weiner, \emph{Raspail: Scientist and Reformer} (New York: Columbia University Press, 1968). 

\bibitem{62} Raspail, p. 84. 

\bibitem{63} Raspail, p. 89. 

\bibitem{64} Raspail, p. 90. 

\bibitem{65} Bell, p. 374. 

\bibitem{66} Dupuy, p. 245. 

\bibitem{67} Bell, p. 374. 

\bibitem{68} Raspail, pp. 117-118 

\bibitem{69} Raspail, p. 118. Also discussed by Dupuy, p. 243. 

\bibitem{70} Infeld, p. 308. 

\bibitem{71} Bertrand, pp. 340-341. (He does not give a reference for the source of Poisson's letter.) 

\bibitem{72} Bell, p. 371. 

\bibitem{73} Bourgne and Azra, pp. 3-11. 

\bibitem{74} See again Bourgne and Azra, pp. 21-27. 

\bibitem{75} Bourgne and Azra, facsimiles. 

\bibitem{76} Bourgne and Azra, pp. 468-469. 

\bibitem{77} Bourgne and Azra, p. 469. 

\bibitem{78} Bourgne and Azra, p. 470. 

\bibitem{79} Bourgne and Azra, p. 471. 

\bibitem{80} Bell, p. 375. 

\bibitem{81} Infeld, pp. 308-311. 

\bibitem{82} Dupuy, pp. 247-248. 

\bibitem{83} Blanc, p. 431. 

\bibitem{84} Blanc, p. 431. 

\bibitem{85} Dupuy, p. 247. 

\bibitem{86} Infeld, p. 310. 

\bibitem{87} Hoyle, p. 14. 

\bibitem{88} Hoyle, p. 15. 

\bibitem{89} See, for example, Charles Mackay, \emph{Extraordinary Popular Delusions and the Madness of Crowds} (USA: Noonday Press, 1932), Chap. ``Duels and Ordeals''; and Roger Shattuck, \emph{The Banquet Years} (New York: Vintage Books, 1968), Chap. 1. 

\bibitem{90} Dupuy, p. 247. 

\bibitem{91} The letters and the descriptions of them are in Bourgne and Azra, pp. 489-491. 

\bibitem{92} C.A. Infantozzi, ``Sur l'a mort d'Evariste Galois,'' {\it Revue d'Histoire des Sciences}, 2, 157 (1968). 

\bibitem{93} Andre Dalmas, Evariste Galois, {\it Revolutionnaire et Geometre} (Paris: Fasquelle, 1956), pp. 77-78. 

\bibitem{94} Bourgne and Azra, p. 173. 

\bibitem{95} Bourgne and Azra, p. 48 and facsimiles. 

\bibitem{96} Bourgne and Azra, p. 54 and facsimiles. 

\bibitem{97} Bourgne and Azra. p. 185.  
\end{thebibliography}

\end{document}
