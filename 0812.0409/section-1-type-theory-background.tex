\section{Type-theoretic setting}\label{sec:type-theory-background}

\subsection{The type theories \texorpdfstring{$\MLfrag$}{ML\_Id}, \texorpdfstring{$\MLfrag[X]$}{ML\_Id[X]}}

\noindent Our main theories of interest are the various versions of Intensional Martin-L\"of Type Theory, usually given with identity types ($\Id$-types), dependent sums and products ($\sum$- and $\prod$-types), units ($\mathbf{1}$-types), and possibly more base types (natural numbers, Booleans$\ldots$).  To cover all these in the main theorem, and for a self-contained presentation, we will work throughout this paper in the fragment $\MLfrag$ with only $\Id$-types, and construct our operad from this.

Some care is thus required in our choice of presentation; presentations which are equivalent in the presence of $\sum$- or $\prod$-types may not be so in their absence.  The presentation we use is taken, up to notation, from that of Jacobs \cite{jacobs:categorical-logic}; we list in Table \ref{table:all-rules} the rules assumed, referring to \cite{jacobs:categorical-logic} for their statements, except for the $\Id$-rules, given in full in Table \ref{table:Id-rules}.  A few more of the rules will later be given explicitly, as their precise statements are required.

\begin{table}[hbp]
\begin{center}\begin{tabular}{|@{\ }c@{\ \ \ }c@{\ }|}
\hline
\multicolumn{2}{|c|}{Basic judgement forms} \\
\hline
$ \Gamma \types A\ \type $ & $ \Gamma \types A = B\ \type $ \\
$ \Gamma \types a:A $ & $ \Gamma \types a = b : A $ \\
\hline
\end{tabular} \\
\begin{tabular}{c@{\ \ \ \ }c}
\\ 
\begin{tabular}{|c|}
\hline
Structural groups \\
\hline
Variables ($\Vble$) \\
Substitution ($\Subst$) \\
Weakening ($\Weak$) \\
Exchange ($\Exch$) \\
Equality ($=$) \\
\hline
\end{tabular} &
\begin{tabular}{|c|}
\hline
$\Id$-rules \\
\hline
$\Id$-$\form$ \\
$\Id$-$\intro$ \\
$\Id$-$\elim$ \\
$\Id$-$\comp$ (``$\beta$'' in \cite{jacobs:categorical-logic})\\
compatibility with \\
substitution and $=$ \\
\hline
\end{tabular}
\end{tabular}\end{center}
\caption{The type theory $\MLfrag$} \label{table:all-rules}
\end{table}

\noindent The only features perhaps needing comment are the explicit inclusion of exchange rules, and of the extra dependent context $\Delta$ in the $\Id$-rules; these are each natural rules, but often omitted since they are derivable in the presence of $\Pi$-types (as discussed on e.g.\ p.587 of \cite{jacobs:categorical-logic}).

Note that from $\Exch$ and this $\Id$-\elim\ rule, we can derive a still slightly more general elimination rule $\Id$-$\elim^+$, as $\Id$-\elim\ but with context
\[\Gamma,\ x:A,\ \Delta,\ y:A,\ \Delta',\ p:\Id_A(x,y),\ \Delta''.\]

To simplify notation when referring to iterated identity types, we introduce the notation (following Warren \cite{warren:thesis}) $\uA[n]$ for the $n$th iterated identity type of a type $A$; that is, if $\Gamma \types A\ \type$, then $\Gamma \types \uA[0] := A\ \type$, and inductively
\begin{samepage}
\[\Gamma,\ x_0,y_0:\uA[0],\ x_1,y_1:\uA[1](x_0,y_0),\ \ldots,\
x_n,y_n:\uA[n](x_0,y_0;\ldots ;x_{n-1},y_{n-1}) \qquad \quad
\]
\[ \qquad \qquad \qquad \qquad \qquad \types \uA[n+1](x_0,y_0; \ldots
;x_n, y_n) := \Id_{\uA[n](x_0,\ldots)}(x_n,y_n)\ \type .
\]
\end{samepage}%
We will often omit the superscripts on these when unambiguous.  As usual, we will also be inconsistent in suppression of free variables, writing usually e.g.\ $\y : \Gamma \types A(\y)\ \type$ for clarity in simple cases, but sometimes $\Gamma \types A\ \type$ to avoid unmanageable proliferations of variables.

Finally, for a finite partial order $I = \{i_1 < \ldots < i_n\}$, we will write $\bigwedge_{i \in I} x_i:A_i$ (or just $\bigwedge_{i \in I} A_i$) to denote the context $x_{i_1}:A_{i_1}, \ldots, x_{i_n}:A_{i_n}$.

\begin{table} 
$$\inferrule*[right=$\Id$-\form]{\Gamma \types A\ \type}{\Gamma,\ x,y:A \types \Id_A(x,y)\ \type}$$
$$\inferrule*[right=$\Id$-\intro]{\Gamma \types A\ \type}{\Gamma,\ x:A \types r(x):\Id_A(x,x)}$$
$$\inferrule*[right=$\Id$-\elim]{\Gamma,\ x,y:A,\ p:\Id_A(x,y),\ \Delta(x,y,p) \types C(x,y,p)\ \type \\ \Gamma,\ z:A,\ \Delta(z,z,r(z)) \types d(z):C(z,z,r(z))}{\Gamma,\ x,y:A,\ p:\Id_A(x,y),\ \Delta(x,y,p) \types \idelim{z}{d}{x}{y}{p} : C(x,y,p)}$$
$$\inferrule*[right=$\Id$-\comp]{\textit{(premises as for $\Id$-$\elim$)}}{\Gamma,\ x:A,\ \Delta(x,x,r(x)) \types \idelim{z}{d}{x}{x}{r(x)} = d(x) : C(x,x,r(x))}$$
\caption{Rules for $\Id$-types} \label{table:Id-rules}
\end{table}

\subsection{Translations and syntactic categories}

For reference on this section (including proofs not given here), see Cartmell \cite{cartmell:generalised-algebraic-theories} and Jacobs \cite{jacobs:categorical-logic}.

From here on, we will consider type theories extending $\MLfrag$; formally, by a \emph{type theory} we will mean a \emph{generalised algebraic theory} in the sense of Cartmell \cite{cartmell:generalised-algebraic-theories}, together with an interpretation of the $\Id$-rules in $\T$.

Recall that a \emph{translation} $F$ from such a type theory $\T$ into a type theory $\S$ consists of suitable mappings of types, terms, and derivable judgements, taking each judgement $\Gamma \types A\ \type$ in $\T$ to a judgement $F(\Gamma) \types F(A)\ \type$ in $\S$, and so on, preserving $\Id$-types and their term-constructors, considered up to definitional equality.  (In other words, it is a morphism of generalised algebraic theories, preserving the interpretation of the $\Id$-rules.)

Given $\T$, we write $\T[X]$ for the result of adjoining to $\T$ a fresh base type $X$
\[\inferrule*[right=$X$-\form]{\ }{ \types X\ \type}\]
with no term formation rules.  For any $\S$, a translation $F \colon \T[X] \to \S$ then consists of a translation $F'\colon \T \to \S$ together with a closed type of $\S$. Stating this universal property precisely, in the particular case that we will need:

\begin{prop} \label{prop:universal-property}If $\S$ is any type theory extending $\MLfrag$, and $A$ any closed type of $\S$, then there is a unique translation $F_{S,A} \colon \MLfrag[X] \to \S$ preserving $\Id$-types and their term-constructors and with $F_{S,A}(X) = A$.
\end{prop}

For any type theory $\T$, there is a \emph{syntactic category} $\C(\T)$, having as objects the closed contexts $\Gamma$ of $\T$, and as arrows $f\colon \Gamma \to \Delta$ suitable strings of terms in context $\Gamma$ (\emph{context maps}), all up to definitional equality.  Moreover, a translation $F \colon \T \to \S$ induces a functor $\C(F)\colon\C(\T) \to \C(\S)$; in other words, we have a functor $\C(-)\colon\Th \to \Cat$.

Context maps are sometimes known as \emph{substitutions}, since substitution along them is an important derived rule: if $\y : \Gamma \types A(\y)\ \type$ and $f: \x:\Gamma \to \Delta$ is a context map, then $\x : \Delta \types A(f(\x)\ \type$.  When suppressing free variables, we will write this as $\Delta \types f^*(A)\ \type$.

We will need a simple proposition on limits in syntactic categories:

\begin{prop} \label{prop:dependent-projections-give-limits}
Suppose $\Gamma = \bigwedge_{i \in I} x_i:A_i$ is a context in $\T$, and $\F \subseteq \mathcal{P}(I)$ a set of subsets of $I$, closed under binary intersection and with $\bigcup \F = I$, such that for each $J \in \F$, $\Gamma_J = \bigwedge_{i \in J} x_i : A_i$ is also a well-formed context.

 Then the $\Gamma_J\!$'s and dependent projections between them give a diagram 
\[\Gamma_{-} \colon (\F, \subseteq)^\op \to \C(\T),\]
and the dependent projections $\Gamma \to \Gamma_J$ express $\Gamma$ as its limit:
\[\Gamma = \lim \! {}_{J \in \F}\ \Gamma_J .\]
Moreover, for a translation $F\colon \T \to \T'$, the functor $\C(F)$ preserves such limits. \qed
\end{prop} 

Here \emph{dependent projections} are the obvious context morphisms from a context to any well-formed subcontext, constructed from the $\Vble$, $\Weak$ and $\Exch$ rules.

A familiar special case asserts that if $\Gamma \types A\ \type$ and $\Gamma \types B\ \type$, then the following square of projections is a pullback:
\[\xymatrix{ \Gamma, x:A, y:B \ar[r] \ar[d] & \Gamma ,y:B \ar[d] \\
\Gamma , x: A \ar[r] & \Gamma } 
\]
The proof of the general proposition is essentially the same.

To relativise the constructions of this section to dependent types and contexts over a (closed) context $\Gamma = \bigwedge_{0 \leq i < n} x_i:A_i$ of $\T$, we can consider the \emph{slice type theory} $\T/\Gamma$, given by adjoining to $\T$ a ``generic term of type $\Gamma$'', i.e.\ $n$ new constant symbols $c_i$ and axioms $\types c_i : A_i(c_0,\ldots,c_{i-1})$.  Closed types (resp.\ terms, contexts) of $\T/\Gamma$ then correspond precisely to types (terms, contexts) of $\T$ in context $\Gamma$.

