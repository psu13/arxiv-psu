\section{The contractible globular operad \texorpdfstring{$\P$}{P\_ML\_Id}} \label{sec:the-operad}

\noindent In this section, we construct the promised operad $\P$ of all definable composition laws; we then show that it is contractible, and describe (in the main theorem) how it acts to give the desired weak-$\omega$-category structures on types.

\subsection{Construction of \texorpdfstring{$\P$}{P\_ML\_Id}}

We saw above that for a type $A$ in a type theory $\T$ extending $\MLfrag$, the contexts
\[x_0,y_0:A,\ x_1,y_1:\uA[1](x_0,y_0)\ldots,\ z:\uA[n](x_0,y_0\ldots,
x_{n-1},y_{n-1}),
\]
and the dependent projections between them form a globular context $\A \colon  \G^\op \to \C(\T)$.  In particular, the generic type $X$ gives a globular context $\X$ in $\C(\MLfrag[X])$.  Using the machinery of the previous section, it is now easy to describe $\P$: it will be $\End_{\C(\MLfrag[X])}(\X)$.  However, since $\C(\T)$ does not have all finite limits in general, to use the description of $\End_{\C(\T)}(\A)$ provided by Proposition \ref{prop:endo-operad} we must construct contexts $\Gamma_\pi$ exhibiting the objects $A_\pi$.

Accordingly, suppose we are given $\pi \in T1_n$, with associated globular set $\hat{\pi}$.  There are various ways of putting a total order on the $i$-cells of $\hat{\pi}$ for each $i \leq n$; pick any such.

(There is in fact a canonical choice of such orderings, using the representation of pasting diagrams as Batanin trees (\cite{batanin:natural-environment}, \cite[8.1]{leinster:book}).  This choice has some good compatibility between the orderings on different pasting diagrams, which will later spare us some use of $\Exch$ rules, so for simplicity we will assume it is the ordering chosen; however, since this is purely cosmetic, we will not go into the details here.)

Then take $\Gamma_\pi$ to be the context
\[\bigwedge_{c \in \hat{\pi}_0} x_c\! :\! A,\ \bigwedge_{c \in
\hat{\pi}_1} x_c\! :\! \uA[1](x_{s(c)},x_{t(c)}),\ \ldots\
\bigwedge_{c \in \hat{\pi}_n} x_c\! :\! \uA[n](x_{s^n(c)},x_{t^n(c)};
\ldots;x_{s(c)},x_{t(c)}).
\]
For instance, $\Gamma_{(\bullet \to \bullet \to \bullet)}$ is the context
\[x,y,z:A,\ p:\Id_A(x,y),\ q:\Id_A(y,z)\]
which we met back in the introduction.

Note that we also have projections $\src,\tgt \colon \Gamma_\pi \to \Gamma_{\d \pi}$.

\begin{lem}The context $\x: \Gamma_\pi$, together with the obvious dependent projections, is the object $(\A)_\pi$ of Definition \ref{def:a-pi}; that is, $\Gamma_\pi = \lim_{c \in \int \pi} A_{\dim c}$.  Moreover, if $F \colon \T \to \S$ is a translation of type theories, then $\C(F) \colon \C(\T) \to \C(\S)$ preserves this limit.
\end{lem}
\begin{proof}
Immediate by Proposition \ref{prop:dependent-projections-give-limits} 
\end{proof}

Thus, by Proposition~\ref{prop:endo-operad}, we have:
\begin{thm}The globular object $\A$ in $\C(\T)$ has an endomorphism operad $\End_{\C(\T)}(\A)$, as described in Proposition \ref{prop:endo-operad}, and if $F \colon \T \to \S$ is a translation of type theories, there is an induced map of operads $\End_{\C(\T)}(\A) \to \End_{\C(\S)}(F\A)$. \qed
\end{thm}

Let us unfold what this operad $P := \End_{\C(\T)}(\A)$ actually looks like.  For $\pi \in T1_n$, an element of $P(\pi)$ consists of a map $\rho \colon \Gamma_\pi\to A_n$ in $\C(\T)$, and for $0 \leq k < n$, maps $\sigma_k \colon \Gamma_{\d^{n-k}(\pi)} \to A_k$ and $\tau_k \colon \Gamma_{\d^{n-k}(\pi)} \to A_n$, commuting with the dependent projections.

So, concretely, an element of $P(\pi)$ (a \emph{composition law for $\pi$}) is a sequence of terms $\vec \rho = ((\sigma_i, \tau_i)_{0 \leq i < n}; \rho)$, such that
\begin{eqnarray*}
\x : \Gamma_{\d^n(\pi)} & \types & \sigma_0(\x) : A \\
\x : \Gamma_{\d^n(\pi)} & \types & \tau_0(\x) : A \\
& \vdots & \\
\x : \Gamma_{\d^{n-k}(\pi)} & \types & \sigma_k(\x): \Id (\sigma_{k-1}(\src\ \x),\tau_{k-1}(\tgt\ \x)),\\
\x : \Gamma_{\d^{n-k}(\pi)} & \types & \tau_k(\x): \Id (\sigma_{k-1}(\src\ \x)),\tau_{k-1}(\tgt\ \x)),\\
& \vdots & \\
\x : \Gamma_\pi & \types & \rho(\x) : \Id(\sigma_{n-1}(\src\ \x),\tau_{n-1}(\tgt\ \x)).
\end{eqnarray*} 

The source of this is then the composition law $(\sigma_0,\tau_0,\ldots, \sigma_{n-1},\tau_{n-1};\sigma_n) \in P(\d(\pi))$, and its target is $(\sigma_0,\tau_0,\ldots, \sigma_{n-1},\tau_{n-1};\tau_n) \in P(\d(\pi))$.

We make no attempt to give a formal syntactic description of composition within this operad, but in specific cases it is ``exactly what you would expect'', and is essentially just substitution.  %%  (With a little help from the other structural rules, of course...)

\begin{defi} \label{defn:operad-p}As as special case of the above construction, we take 
\[\P := \End_{\C(\MLfrag[X])}(\X),\]
 the operad of all definable composition laws on the generic type. 
\end{defi}

For general $\T,A$, we cannot expect $\End_{\C(\T)}(\A)$ to be contractible: contractibility implies (at least) that any two elements of $\End_{\C(\T)}(\A)(\bullet)$ are connected by an element of $\End_{\C(\T)}(\A)(\bullet \to \bullet)$, or in other words that any two terms $x:A \types \tau(x), \tau'(x) : A$ are propositionally equal, which clearly may fail.  However, in the specific case of $\P$, we do wish to show contractibility, since this is the operad which naturally acts on any type.

What precisely does contractibility mean, here?  For every pasting diagram $\pi$ and every parallel pair of composition laws $\vec \sigma, \vec \tau  \in \P(\d(\pi))$, we need to find some filler $\vec \rho \in \P(\pi)$, with $s(\vec \rho) = \vec \sigma$, $t(\vec \rho) = \vec \tau$.

Given $\pi$, such a parallel pair amounts to terms $(\sigma_i,\tau_i)_{0 \leq i < n}$ as in the definition of a composition law for $\pi$, and a filler is a term $\rho$ completing the definition; that is, we seek to derive a judgment
\[\x : \Gamma_\pi \types \rho (\x) : \Id ( \sigma_{n-1}(\src\ \x),
\tau_{n-1} (\tgt\ \x) ).
\]

Playing with small examples (the reader is strongly encouraged to try this---to derive, for instance, the composition and associativity terms mentioned in the introduction) suggests that we should be able to do this by applying $\Id$-\elim\ (possibly repeatedly, working bottom-up as usual) to the variables of identity types in $\Gamma_\pi$.  $\Id$-\elim\ says that to obtain $\rho$, it's enough to obtain it in the case where one of the variables is of the form $r(-)$, and its source and target variables are equal; and by repeated application, it's enough to obtain $\rho$ in the case where multiple higher cells have had identities plugged in in this way.

Now, since the terms $\sigma_i,\tau_i$ have themselves been built up from just the $\Id$-rules, as we plug $r(-)$ terms into them and identify the lower variables, they should sooner or later collapse by $\Id$-\comp\ to be of the form $r^i(x)$ themselves.  In particular, after applying $\Id$-\elim\ as far as possible, plugging in reflexivity terms for the higher variables and contracting all variables of type $X$ to a single $x:X$, the $\sigma_i, \tau_i$ should \emph{all} reduce to reflexivity terms, and in particular $\sigma_{n-1} = \tau_{n-1} = r^{n-1}(x)$, so we can take the desired filler to be
\[x:X \types r^n(x) : \Id(r^{n-1}(x),r^{n-1}(x)).\]
Below, we formalise this argument.  The crucial lemma is that the context $x:X$ is an initial object in $\C(\MLfrag[X])$: that is, since any context $\Gamma$ in $\MLfrag[X]$ is built up just from $X$ and its higher identity types, there is always a unique way to substitute $x$ and its reflexivity terms $r^i(x)$ for all variables of $\Gamma$, and when we subsitute these in to any context morphism $\sigma \colon \Gamma \to \Gamma'$, the result must again reduce to terms of this form.

\subsection{\texorpdfstring{$X$}{X} is initial in \texorpdfstring{$\MLfrag[X]$}{ML\_Id[X]}} \label{subsec:initiality}

\begin{lem} \label{lemma:initiality} The context $x:X$ is an initial object in $\C(\MLfrag[X])$; that is, for any closed context $\Gamma$ there is a unique context map $\ r^\Gamma \colon (x:X) \to \Gamma$. 
\end{lem}

\begin{rem}This lemma does not generally hold in extensions of $\MLfrag[X]$; in $\ML[X]$, for instance, it is easily seen to be false, since for instance there is no term $x:X \types \tau : \Pi_{y:X} \Id(x,y)$.
\end{rem}

\proof We work by structural induction (as, essentially, we must, since this is a property of the theory $\MLfrag[X]$ which can fail in extensions).

So, given any derivation $\delta$ of a judgement $J$ in $\MLfrag[X]$, we recursively derive various terms and/or judgments, depending on the form of $J$, assuming that we have already done so for all sub-derivations of $\delta$.  The form of the terms and judgements we derive will depend on the form of J as follows:\medskip
\[\begin{array}{|c|c|c|c}
\cline{1-3} \rule[-1ex]{0ex}{3.1ex}
J & \textrm{term} & \textrm{judgement} & \\ 

\cline{1-3}  \rule[-1ex]{0ex}{3.5ex} 
\y:\Gamma \types A(\y)\ \type & r^{\Gamma \,\types\, A}(x) & x:X \types r^{\Gamma \,\types\, A}(x) : A(\r^\Gamma (x)) & \\ 

\cline{1-3}  \rule[-1ex]{0ex}{3.5ex} 
\y:\Gamma \types A(\y) = A'(\y) \ \type & - & x:X \types r^{\Gamma \,\types\, A}(x) = r^{\Gamma \,\types\, A'}(x) : A(\r^\Gamma (x)) & (*) \\

\cline{1-3}  \rule[-1ex]{0ex}{3.5ex} 
\y:\Gamma \types \tau(\y) : A(\y) & - & x:X \types \tau(r^\Gamma (x)) = r^{\Gamma \,\types\, A} (x) : A(r^\Gamma (x)) & (**) \\ 

\cline{1-3}  \rule[-1ex]{0ex}{3.5ex} 
\y:\Gamma \types \tau(\y) = \tau'(\y) : A(\y) & - & - & \\ 

\cline{1-3} \end{array}\]\medskip
Here, for a context $\Gamma\ =\ y_0:A_0, \ldots, y_n : A_n(\y_{< n})$, we write $\r^\Gamma$ for the context map $(x:X) \to \Gamma$ consisting of the terms $r^{\types\,A_0}(x) : A_0$, $r^{A_0\, \types\, A_1}(x) : A_1(r^{\types\,A_0}(x))$, \ldots %, $r^{\Gamma_{<n}\, \types\, A_n(\y_{<n})}(x) : A_n(\r^{\Gamma_{<n}}(x))$.

Moreover, applying (*) and (**) above to this definition shows that the maps $\r^\Gamma$ respect definitional equality in $\Gamma$, and are preserved by context maps in that for any $f \colon \Delta \to \Gamma$, we have $f(\r^\Delta (x)) = \r^\Gamma (x)$.

Finally, once the induction is complete, applying this last fact together with the definition $r^{(x:X)}(x) := x$ will show that for any other context map $f \colon (x:X) \to \Gamma$, we have $f(x) = f(r^{(x:X)}) = \r^\Gamma (x)$, and so $\r^\Gamma$ is the unique such map, as originally desired.

(This last step is an instance of the general categorical fact that given an object $X$ in a category $\C$ and natural maps $!_Y \colon X \to Y$ to every other object, such that $!_X = 1_X$, it follows that $X$ is initial.)

As usual, the induction proceeds by cases on the last rule used in the derivation of $J$.  Most cases are routine; we include here $X$-\form\ and $\Weak$-\sctype\ as examples of these, together with the less straightforward cases of the $\Id$-rules and $\Subst$-\sctype .

Our definitions for the $\Subst$-\sctype\ and $\Weak$-\sctype\ cases ensure, as usual, that the terms constructed do not depend on the derivation of the judgement used.  As warned earlier, we will vary for readability between showing dependent variables and leaving them implicit, and hence also between the notations $A(f(\x))$ and $f^*A$ for substitution.\medskip

% $$\textrm{Given }\left\{ \begin{array}{l} \y:\Gamma \types A(\y)\ \type \\ \y:\Gamma \types A(\y) = A'(\y) \ \type \\ \y:\Gamma \types \tau(\y) : A(\y) \end{array} \right. \textrm{ we derive } \left\{ \begin{array}{l} 
% x:X \types r^{\Gamma \,\types\, A} : A(r^\Gamma) \\
% x:X \types r^{\Gamma \,\types\, A} = r^{\Gamma \,\types\, A'} : A(r^\Gamma) \\ 
% x:X \types \tau(r^\Gamma) = r^{\Gamma \,\types\, A} : A(r^\Gamma) \end{array} \right. $$
% %% }} (to match the ones matched by \right.)
% 
% The context morphism $r^\Gamma \colon (x:X) \to \Gamma$ is built up inductively, by
% $$r^{\Gamma, y: A} = r^\Gamma, r^{\Gamma \,\types\, A}.$$
% The above judgments then ensure that this is the unique context from $(x:X)$ to $\Gamma$, by induction on the length of $\Gamma$.
% 
% The induction is essentially routine.  As ever, given a judgment, we work by cases, depending on its last rule. We give here the cases for the $\Weak$-\sctype, $\Id$- and $\Subst$-\sctype\ rules.\\ 

\noindent($X$-\form):\ in the easiest case, our derivation consists of just the axiom $X$-form
\[\inferrule*[right=$X$-\form]{\ }{ \types X\ \type}\]
and so defining $r^{\types\,X}(X) := x$, we have $x:X \types x: X\ \type$ as needed. \miniqed

\noindent($\Weak$-\sctype):\ Given a derivation ending
\[\inferrule*[right=$\Weak$-\sctype]{\Gamma \types A\ \type \\ \Gamma
\types B\ \type}{\Gamma,\ y:A \types B\ \type}
\]
we inductively already have 
$x:X \types r^{\Gamma \,\types\, B} : (r^\Gamma)^*B $,
and by the $\Subst$ rules,
$ x:X \types (r^\Gamma)^*B = (r^{\Gamma, y:A})^*B\ \type$,
so by equality rules we conclude
$ x:X \types r^{\Gamma \,\types\, B} : (r^{\Gamma,y:A})^*B $
and hence, by $\Weak$-\scterm{}, can set
\[r^{\Gamma,y:A \,\types\, B} := r^{\Gamma \,\types\, B}.\eqno{\Diamond}\]
\noindent($\Id$-\form):\ Given a derivation ending
\[\inferrule*[right=$\Id$-\form]{\Gamma \types A\ \type}{\Gamma,\
y,y':A \types \Id_A(y,y')\ \type}
\]
we need to find a term
\[x:X \types r^{\Gamma,y,y':A \,\types\, \Id_A(y,y')} :
\Id_{(r^\Gamma)^*A}(r^{\Gamma \,\types\, A},r^{\Gamma,y:A \,\types\,
A})
\]
But $\Gamma, y:A \types A\ \type$ may be derived using weakening, and so by our construction for $\Weak$-\sctype\ above, $r^{\Gamma,y:A \,\types\, A} = r^{\Gamma \,\types\, A}$, so we have
\[x:X \types r(r^{\Gamma \,\types\, A}) :
\Id_{(r^\Gamma)^*A}(r^{\Gamma \,\types\, A},r^{\Gamma,y:A \,\types\,
A})
\]
and so can set
\[r^{\Gamma, y,y':A \,\types\, \Id_A(y,y')} := r(r^{\Gamma \,\types\,
A}). \eqno{\Diamond}
\]
\noindent($\Id$-\intro):\ Now we are given a derivation with last step
\[\inferrule*[right=$\Id$-\intro]{\Gamma \types A\ \type}{\Gamma,\ y:A
\types r(y):\Id_A(y,y)}
\]
and wish to show
\[x:X \types r(r^{\Gamma \,\types\, A}) = r^{\Gamma, y,y:A \,\types\,
\Id_A(y,y)} : (r^\Gamma)^* A.
\]
But by our construction of $r^{\Gamma, y,y':A \,\types\, \Id_A(y,y')}$ above (our $\Id$-\form\ case), and of $r^{\Gamma, y:A \,\types\, \Id_A(y,y)}$ from it (our $\Contr$-\sctype\ case), this is just the definition of $r^{\Gamma, y,y:A \,\types\, \Id_A(y,y)}$. \miniqed \medskip

% \ %phantom paragraph commented as there's a page break

\noindent($\Id$-\elim):\ Here, we are given a derivation ending
\[\inferrule*[right=$\Id$-\elim]{\Gamma,\ y,y':A,\ p:\Id_A(y,y'),\
\Delta(y,y',p) \types C(y,y',p)\ \type \\ \Gamma,\ z:A,\
\Delta(z,z,r(z)) \types d(z):C(z,z,r(z))}{\Gamma,\ y,y':A,\
p:\Id_A(y,y'),\ \Delta(y,y',p) \types \idelim{z}{d}{y}{y'}{p} :
C(y,y',p)};
\]
for readability, we assume $\Delta$ is empty.  We want to derive the judgement
\begin{eqnarray*} x:X \types (r^{\Gamma, y,y':A, p:\Id(y,y')})^* (\idelim{z}{d}{y}{y'}{p}) & = & r^{\Gamma, y,y':A, p:\Id(y,y') \,\types\, C(y,y',p)} \\
& & \qquad : (r^{\Gamma, y,y':A, p:\Id(y,y')})^*C.
\end{eqnarray*}
Unwrapping the former term, we have (all in context $(x:X)$):
\[\begin{tabular}{llr}
\multicolumn{3}{l}{$\displaystyle (r^{\Gamma, y,y':A, p:\Id(y,y')})^* (\idelim{z}{d}{y}{y'}{p})$} \\
$\quad$ & $\displaystyle = \idelim{z}{(r^\Gamma)^* d}{r^{\Gamma \,\types\, A}}{r^{\Gamma,y:A \,\types\, A}}{r^{\Gamma,y,y':A \,\types\, \Id(y,y')}}$& \\
& $\displaystyle = \idelim{z}{(r^\Gamma)^* d}{r^{\Gamma \,\types\, A}}{r^{\Gamma \,\types\, A}}{r(r^{\Gamma \,\types\, A})}$ & \\
& $\displaystyle = (r^\Gamma)^* d(r^{\Gamma \,\types\, A})$ & (by $\Id$-\comp)\\
& $\displaystyle = (r^{\Gamma, z:A})^* d$ & \\
& $\displaystyle = r^{\Gamma, z:A \,\types\, C(z,z,r(z))}$ & (by induction) \\
& $\displaystyle = r^{\Gamma, y,y':A, p:\Id(y,y') \,\types\, C(y,y',p)}$ \\
\multicolumn{3}{r}{(by the definition of $\displaystyle r^{\Gamma, z:A \,\types\, C(z,z,r(z))}$} \\
\multicolumn{3}{r}{using our $\Weak$-\sctype\ and $\Id$-\elim\ cases.)}
\end{tabular}
\]
If $\Delta$ in the application of $\Id$-\elim\ is non-empty, we have a few more lines, relying inductively on our $\Subst$-rules cases. \miniqed\medskip

\noindent($\Subst$-\sctype):\ For this case we will need one more piece of notation, generalising the context maps $\r^\Gamma$: for a dependent context $\Delta = \bigwedge_i A_i$ over $\Gamma$, we write $r^{\Gamma \,\types\, \Delta} \colon (x:X) \to (r^\Gamma)^* \Delta$ for the map built up from terms $r^{\Gamma, \Delta_{< i} \,\types\, A_{i}}$ in the obvious way.

So, we are given a derivation ending with the rule
\[\inferrule*[right=$\Subst$-\sctype]{\Gamma,\ y:A,\ \Delta \types B\
\type \\ \Gamma \types f:A}{\Gamma,\ f^*\Delta \types f^*B\ \type}
\]
and we wish to derive a judgement
\[x:X \types r^{\Gamma,f^*\Delta \,\types\, f^*B} :
(f^{\Gamma,f^*\Delta})^*(f^*B).
\]
Unfolding the definition of the desired type, we have 
\[\begin{tabular}{ll}
\multicolumn{2}{l}{$\displaystyle (f^{\Gamma,f^*\Delta})^*(f^*B)$} \\
$\quad$ & $\displaystyle = (r^{\Gamma \,\types\, f^*\Delta})^*(r^\Gamma)^*(f^*B)$ \\
 & $\displaystyle = (r^{\Gamma \,\types\, f^*\Delta})^*((r^\Gamma)^*f)^*(r^\Gamma)^*B$ \\
 & $\displaystyle = (r^{\Gamma \,\types\, f^*\Delta})^*(r^{\Gamma \,\types\, A})^*(r^\Gamma)^*B \qquad$ \hfill (by induction) \\
 & $\displaystyle = (r^{\Gamma \,\types\, f^*\Delta})^*(r^{\Gamma,y:A})^*B$ \\
 & $\displaystyle = (r^{\Gamma,y:A \,\types\, \Delta})^*(r^{\Gamma,y:A})^*B \qquad \qquad \quad$ \hfill (by def'n of $r^{\Gamma \,\types\, f^*\Delta}$, i.e.\ by previous\\
 & \hfill applications of \emph{this} case)  \\
 & $\displaystyle = (r^{\Gamma,y:A,\Delta})^*B$ \\
\end{tabular}
\]
so since by induction $\displaystyle x:X \types r^{\Gamma,y:A,\Delta \,\types\, B}:(r^{\Gamma,y:A,\Delta})^*B$, we take $r^{\Gamma,f^*\Delta \,\types\, f^*B} := r^{\Gamma,y:A,\Delta \,\types\, B}$.

The cases for the other structural rules and $X$-\form\ are straightforward, similar to the $\Weak$-\sctype\ case above. \doubleqed

\subsection{Contractibility of \texorpdfstring{$\P$}{P\_ML\_Id}}

We are now ready to show that $\P$ is contractible, arguing along the lines sketched above.

\begin{thm}\label{theorem:p-is-contractible}The operad $\P$ is contractible.
\end{thm}

\begin{proof}As described above, this amounts to the statement: for every $n \in \N$ and pasting diagram $\pi \in T1_n$, and every sequence $(\sigma_i,\tau_i)_{i<n}$ of terms such that
\[\x : \Gamma_{\d^{n-i}(\pi)} \types \sigma_i(\x) : \uX
(\sigma_0(\src^i\ \x),\ldots,\tau_{i-1}(\tgt\ \x))
\]
\[\x : \Gamma_{\d^{n-i}(\pi)} \types \tau_i(\x) : \uX
(\sigma_0(\src^i\ \x),\ldots,\tau_{i-1}(\tgt\ \x))
\]
($i < n$) are derivable in $\MLfrag[X]$, we can find a ``filler'', i.e.\ a term $\rho$ with
\[\x : \Gamma_\pi \types \rho(\x) : \uX (\sigma_0(\src^n\
\x),\ldots,\tau_{n-1}(\tgt\ \x))
\]
We show this by induction on the number of cells in $\pi$.

Suppose $\pi$ has more than one cell.  Then it must have some cells in dimension $> 1$.  Let $k$ be the highest dimension in which $\pi$ has cells, and $c$ be some $k$-cell of $\hat{\pi}$.  Now take $\pi^{-c} \in T1_n$ to be the pasting diagram whose globular set is obtained (up to isomorphism) from that of $\pi$ by removing $c$ and identifying $s(c)$ and $t(c)$.

Now $\Gamma_{\pi^{-c}}$ is exactly (up to renaming of variables, and possibly re-ordering if we do not assume that we chose compatible orderings of the cells of pasting diagrams) the context obtained from $\Gamma_\pi$ by removing the variables $x^k_c$ and $x^{k-1}_{t(c)}$, and replacing any occurrences of the latter in subsequent types by $x^{k-1}_{s(c)}$, and we have a natural context map $h \colon \Gamma_{\pi^{-c}} \to \Gamma_\pi$ given by plugging in $x^{k-1}_{s(c)}$ for $x^{k-1}_{t(c)}$ and $r(x^{k-1}_{s(c)})$ for $x^k_c$; and these are exactly right for
\[\inferrule*{\x:\Gamma_{\pi^{-c}} \types \rho^{-c}(\x) : \uX
(\sigma_0(\src^n\ h(\x)),\ldots,\tau_{n-1}(\tgt\ h(\x)))}{\x :
\Gamma_\pi \types
\idelim{x^{k-1}_{s(c)}}{\rho^{-c}}{x^{k-1}_{s(c)}}{x^{k-1}_{t(c)}}{x^k_c}
: \uX (\sigma_0(\src^n\ \x),\ldots,\tau_{n-1}(\tgt\ \x))}
\]
to be an instance of $\Id$-$\elim^+$.  So to give the desired filler $\rho$, it is enough to give $\rho^{-c}$ with
\[\x:\Gamma_{\pi^{-c}} \types \rho^{-c}(\x) : \uX (\sigma_0(\src^n\
h(\x)),\ldots,\tau_{n-1}(\tgt\ h(\x))).
\]

But now note that
\[\d^{n-i}(\pi^{-c}) = \left\{ \begin{array}{ll} \d^{n-i}(\pi) &
\textrm{for $n-i < k$} \\ (\d^{n-i}(\pi))^{-c} & \textrm{for $n-i \geq
k$} \end{array} \right. ;
\] 
moreover, we can construct context maps
\[h^s_i, h^t_i \colon \Gamma_{\d^{n-i}(\pi^{-c})} \to
\Gamma_{\d^{n-i}(\pi)}
\]
(analogous to $h$ if $i \geq k$, and just the identity otherwise), and these commute with the maps $\src$ and $\tgt$.  So for each $i < n$, we have
\[\x : \Gamma_{\d^{n-i}(\pi^{-c})} \types \sigma_i(h(\x)) : \uX
(\sigma_0(h(\src^i\ \x)),\ldots,\tau_{i-1}(h(\tgt\ \x))),
\]
\[\x : \Gamma_{\d^{n-i}(\pi^{-c})} \types \tau_i(h(\x)) : \uX
(\sigma_0(h(\src^i\ \x)),\ldots,\tau_{i-1}(h(\tgt\ \x))),
\]
i.e.\ the sequence of terms $(h^*(\sigma_i),h^*(\tau_i))_{i<n}$ are a parallel pair for $\pi^{-c}$.  So by induction (since $\pi^{-c}$ has fewer cells than $\pi$), these terms have a filler; but this filler is exactly the desired term $\rho^{-c}$.

Thus it is enough to show the existence of fillers in the case where $\pi$ has just one cell, i.e.\ where $\pi = ( \bullet )$.  But in this case, $\Gamma_\pi = \Gamma_{\d^i(\pi)} = \Gamma_{\d^i(\pi)} = (x:X)$ for each $i < n$, and so by the initiality of $(x:X)$ we must have $\sigma_i(x) = \tau_i(x) = r^i(x)$ for each $i$; so now $\rho := r^n(x)$ gives the filler, and we are done.
\end{proof}

Unwinding this induction, we can see that it exactly formalises the process described at the start of Subsection \ref{subsec:initiality}, of repeatedly plugging in higher reflexivity terms for all variables, knowing that the given composites will themselves eventually compute down to higher reflexivity terms.

Note that Lemma \ref{lemma:initiality} was applied only at the base case of the induction, and only to show that terms $x:X \types \sigma: \Id(r^n(x),r^n(x))$ must be equal to $r^{n+1}(x)$.  A sufficiently strong normalisation result would also imply this, resting on showing that these are the only appropriate normal forms; this could then extend also to the operad $\End_{\ML[X]}(\X)$ of all composition laws of the \emph{full} type theory, which cannot be shown contractible by the present method.  However, working with the fragment $\MLfrag$ seems more economical, showing that $\Id$-types are the only structure required.

\subsection{Types as weak \texorpdfstring{$\omega$}{omega}-categories} \label{subsec:payoff}

Putting the above results together, we obtain our main goal:

\begin{thm}Let $\T$ be any type theory extending the fragment $\MLfrag$, $\Gamma$ any closed context of $\T$, $A$ a dependent type over $\Gamma$.  Then the globular context $\A$ carries the structure of a $\P$-algebra in $\C(\T/\Gamma)$.
\end{thm}

\proof By Proposition \ref{prop:universal-property}, there is a unique translation $F_{\T/\Gamma,A} \colon  \MLfrag[X] \to \T/\Gamma$ taking $X$ to $A$, and hence taking $\X$ to $\A$.  By Proposition \ref{prop:endo-operad}, this induces an action of $\P$ on $\A$, and so, since by Theorem \ref{theorem:p-is-contractible} $\P$ admits a contraction, an action of $L$ (the initial operad-with-contraction) on $\A$, as desired. \qed

\begin{cor}Let $\T$, $\Gamma$, $A$ be as above, and $\Delta$ a dependent context over $\Gamma$.  Then the globular set of terms of types $A$, $\Id_A$, $\Id_{\Id_A}$, $\ldots$ in context $\Gamma, \Delta$ carries the structure of a $\P$-algebra, and hence of a weak $\omega$-category.
\end{cor}

\proof This is just the globular set of $\C(\T/\Gamma)(\Delta,\A)$ of context maps
\[f\colon  \Gamma, \Delta \to \Gamma, x_0,y_0:A, \ldots ,\
x_{n-1},y_{n-1}:\uA(x_0,\ldots,y_{n-2}),\ z:\uA(x_0,\ldots,y_{n-1})
\]
and so inherits a $\P$-action, and hence an $L$-action, from the actions on $\A$.
\qed

\begin{rem}[Functoriality]  The construction of the $\P$-algebra $\C(\T/\Gamma)(\Delta,\A)$ should be covariantly functorial in $\T$, and contravariantly in $\Gamma$ and $\Delta$.  That is, translations $\T \to \T'$ and context maps $\Gamma' \to \Gamma$, $\Delta' \to \Delta$ should induce \emph{strict} maps of $\P$-algebras, composing appropriately.  A proof of this should be fairly straightforward, by an extension of the methods of the current paper; essentially, the missing ingredient is a treatment of maps of internal operad algebras.

More subtly, it should be functorial in $A$, but only to \emph{weak} maps:  a map of types $A \to A'$ should induce \emph{weak} maps of $\P$- or $L$-algebras---that is, weak $\omega$-functors.  This seems an altogether trickier question, due partly but not only to the lack, until fairly recently (\cite{garner:homomorphisms}), of a suitable definition of weak $\omega$-functor.
\end{rem}

\begin{rem}[Comparison with \cite{benno-richard}]  As mentioned in the Introduction, Richard Garner and Benno van den Berg have independently given (\cite{benno-richard}) a proof of essentially the same result.  The core of their approach is the same as that given here: the $\omega$-category action is induced via contractible operads constructed from endomorphism operads of the globular contexts of identity types.  The main differences between construction of the present paper and that of \cite{benno-richard} are, roughly, as follows:

\begin{enumerate}[(1)]
\item Garner and van den Berg use Batanin's presentation \cite{batanin:natural-environment} of globular operads and higher categories, while I have used the later presentation of Leinster \cite{leinster:book}.  This is essentially a superficial difference; the two presentations  are intertranslatable.
\item  Garner and van den Berg work from the categorical structure on syntactic categories given by the identity types, rather than from the identity types in the syntax directly.
\item  Where I have used the single operad $\P$ of definable composition laws on the generic type, Garner and van den Berg use, for each type, a tailor-made operad of composition laws \emph{on that type}, constructed from the endomorphism operad over it in the syntactic category of the particular theory in question.
\item  As remarked after Definition \ref{defn:operad-p}, this entire endomorphism operad will not in general be contractible; consequently, Garner and van den Berg pass to a sub-operad of ``point-preserving'' operations, which is always contractible.  From this point of view, Subsection \ref{subsec:initiality} (the initiality of $(x:X)$ in $\MLfrag[X]$) may be seen as showing that over the generic type in $\MLfrag[X]$, \emph{all} composition laws are point-preserving.
\item  Finally, Garner and van den Berg show moreover that the weak $\omega$-categories produced are in fact weak $\omega$-\emph{groupoids}, according to the criterion of Cheng \cite{cheng:duals-give-inverses}.
\end{enumerate}
\end{rem}

