
\section{Globular operads and weak \texorpdfstring{$\omega$}{omega}-categories}\label{sec:operads-background}

\noindent As described in the introduction, we want to describe ``the globular operad of composition laws''.  Accordingly, we recall briefly in this section what a globular operad is, and how it formalises the intuition of a set of composition laws for pasting diagrams with structure specifying how these laws themselves compose.  For a slightly (resp.\ much) fuller treatment, and background on strict higher categories, see Leinster \cite{leinster:survey} (resp.\ \cite{leinster:book}).

\subsection{Globular sets and operads}

A \emph{globular set} $\A$ is a presheaf on the category $\G$ generated by arrows
\[0 \two^{s_0}_{t_0} 1 \two^{s_1}_{t_1} 2 \two \ldots \]
subject to the equations $ss = ts$, $st = tt$ (omitting subscripts on the arrows, as usual).  We thus have the category $\GSets := \longGSets$ of globular sets and natural transformations between them.  More generally, a \emph{globular object} in a category $\C$ is a functor $\A \colon  \G^\op \to \C$.  

Explicitly, a globular set $\A$ has a set $A_n$ of ``$n$-cells'' for each $n \in \N$, and each $(n+1)$-cell $x$ has parallel source and target $n$-cells $s(x)$, $t(x)$, as illustrated in the first line of Fig.~\ref{figure:assoc-laws}.  (Cells $x,y$ of dimension $>0$ are \emph{parallel} if $s(x) = s(y)$ and $t(x) = t(y)$; all $0$-cells are considered parallel.)  For parallel $x,y \in A_n$, we write $A(x,y) := \{ z \in A_{n+1}\ |\ s(z) = x, t(z) = y\}$, the set of $(n+1)$-cells from $x$ to $y$.

Our notation will vary: we will typically call globular objects $\A, \B, \ldots$ when emphasising the point of view of the category $\C$, or $A, B, \ldots$ when working more in $[\G^\op, \C]$.  %% Query: take this out??  And if so, make notation more consistent, or is the variation fine without comment??

\begin{exa} \label{ex:pi-omega}For any topological space $X$, there is a globular set $\Pi_\omega(X)$ in which 0-cells are points of $X$, 1-cells are paths between points, 2-cells are homotopies between paths keeping endpoints fixed, and in general, $n$-cells are suitable maps $H\colon [0,1]^n \to X$, viewed as homotopies between $(n-1)$-cells.
\end{exa}

\begin{exa}For any type $A$ in a type theory $\T$, the contexts 
\[x_0,y_0:A,\ x_1,y_1:\uA[1](x_0,y_0)\ldots,\ z:\uA[n](x_0,y_0\ldots,
x_{n-1},y_{n-1}),
\]
along with their dependent projections, form a globular object $\A$ in $\C(\T)$.
\end{exa}

Any strict $\omega$-category (as sketched in the introduction) has an evident underlying globular set, and in fact there is an adjunction (moreover monadic) $F: \GSets \two/->`<-/ \strwCat : U$, giving rise to the ``free strict $\omega$-category'' monad $(T,\mu,\eta)$ on $\GSets$.  Cells of $T\A$ are free (strictly associative) pastings-together of cells from $\A$, including degenerate pastings from the identity cells of $F(\A)$ (as shown in figure \ref{figure:pastings}).

\begin{figure}[htbp]
\[\begin{array}{c@{\ \ \ }c}
\begin{array}{c}
\ 
\xy
(0,-35)*{\bullet};
(0,35)*{\scriptstyle a};
(-70,85)*{}="tl";
(70,85)*{}="tr";
(-70,-85)*{}="bl";
(70,-85)*{}="br";
"tl";"tr" **\dir{.};
"tl";"bl" **\dir{.};
"tr";"br" **\dir{.};
"bl";"br" **\dir{.};
\endxy \ \ 
a \in TA_0 \qquad
\xy
(0,-35)*{\bullet};
(0,35)*{\scriptstyle a};
(-70,85)*{}="tl";
(70,85)*{}="tr";
(-70,-85)*{}="bl";
(70,-85)*{}="br";
"tl";"tr" **\dir{.};
"tl";"bl" **\dir{.};
"tr";"br" **\dir{.};
"bl";"br" **\dir{.};
\endxy \ \ 
1_a \in TA_1\ \\ \\
\xy
(0,0)*+{\bullet}="a";
(0,75)*{\scriptstyle a};
(400,0)*+{\bullet}="b";
(400,75)*{\scriptstyle b};
(800,0)*+{\bullet}="c";
(800,75)*{\scriptstyle c};
{\ar^h "a";"b"};
{\ar^g "b";"c"};
(-70,125)="tl";
(870,125)="tr";
(-70,-50)="bl";
(870,-50)="br";
"tl";"tr" **\dir{.};
"tl";"bl" **\dir{.};
"tr";"br" **\dir{.};
"bl";"br" **\dir{.};
\endxy \\
g \cdot_0 h \in TA_1
\end{array}
&
\begin{array}{c}
\xy
(0,0)*+{\bullet}="a";
(0,75)*{\scriptstyle a};
(450,0)*+{\bullet}="b";
(450,75)*{\scriptstyle b};
(900,0)*+{\bullet}="c";
(900,75)*{\scriptstyle c};
(1350,0)*+{\bullet}="d";
(1350,75)*{\scriptstyle d};
{\ar@/^1.75pc/^{h''} "a";"b"};
{\ar|{h'} "a";"b"};
{\ar@/_1.75pc/_h "a";"b"};
{\ar@/^1pc/^{g'} "b";"c"};
{\ar@/_1pc/_g "b";"c"};
{\ar^f "c";"d"};
{\ar@{=>}^<<<{\beta'} (210,180);(210,40)};
{\ar@{=>}^<<<{\beta} (210,-40);(210,-180)};
{\ar@{=>}^<<<{\alpha} (675,75);(675,-75)};
(-70,320)="tl";
(1420,320)="tr";
(-70,-320)="bl";
(1420,-320)="br";
"tl";"tr" **\dir{.};
"tl";"bl" **\dir{.};
"tr";"br" **\dir{.};
"bl";"br" **\dir{.};
(675,-350)*{\ }="strut";
\endxy \\
1_f \cdot_0 \alpha \cdot_0 (\beta \cdot_1 \beta') \in TA_2
\end{array}
\end{array}
\]
\[(\mbox{for }a,b,\ldots \in A_0,\ f,g,\ldots \in A_1,\ \alpha,\beta,\ldots \in A_2)\]
\caption{Some labelled pasting diagrams, elements of a free strict $\omega$-category $T\A$. \label{figure:pastings} } 
\end{figure}

\noindent In particular, $T1$ (where $1$ denotes the terminal globular set, with just one cell of each dimension) consists informally of pastings of this sort, but without labels on the cells.  This is the crucial globular set of \emph{pasting diagrams}.  A peculiarity of $T1$ is that the source and target of any pasting diagram are equal; for this ambivalent operation we write $\d \pi := s(\pi) = t(\pi)$. 

Every pasting diagram $\pi \in T1_n$ has an associated globular set $\hat{\pi}$---intuitively, the set of cells appearing in $\pi$, as shown in our pictures of pasting diagrams throughout.  We then have maps of globular sets $\hat{s}^k, \hat{t}^k\colon  \widehat{\d^k \pi} \to \hat \pi$, embedding $\widehat{\d^k \pi}$ as the $(n-k)$-dimensional source or target of  $\hat \pi$.

Taking categories of elements then gives categories $\int \pi := \int_\G \hat{\pi}$, with objects the cells of $\hat{\pi}$ and arrows into each cell $c$ from its sources and targets $s^k(c)$, $t^k(c)$, and with a functor $\dim \colon  \int \pi \to \G$ giving the dimension of each cell; $\int \pi$ may be seen as the shape of the canonical diagram of basic cells whose colimit in $\GSets$ gives $\hat{\pi}$.

For more discussion of these and other various ways of looking at a pasting diagram, see Street \cite{street:petit-topos}.

% The following diagram (from earlier versions of this paper, is no longer relevant, but it's so pretty I can't bring myself to delete it:

% \begin{figure}[htbp] \label{fig:exploded-diagram}
% \[
% \xy
% (-875,25)*{}="tlleft"; % "left": 0-source
% (-725,25)*{}="trleft";
% (-875,-125)*{}="blleft";
% (-725,-125)*{}="brleft";
% "tlleft";"trleft" **\dir{.};
% "tlleft";"blleft" **\dir{.};
% "trleft";"brleft" **\dir{.};
% "blleft";"brleft" **\dir{.};
% (-800,-50)*+{\bullet}="aleft";
% {\ar@{.>} (-755,25)*{};(-665,195)*{}};
% {\ar@{.>} (-725,-115)*{};(-505,-240)*{}};
% (-675,575)*{}="tltop"; % "top": 1-source"
% (475,575)*{}="trtop";
% (-675,225)*{}="bltop";
% (475,225)*{}="brtop";
% "tltop";"trtop" **\dir{.};
% "tltop";"bltop" **\dir{.};
% "trtop";"brtop" **\dir{.};
% "bltop";"brtop" **\dir{.};
% (-600,300)*+{\bullet}="atop";
% (-100,300)*+{\bullet}="btop";
% (400,300)*+{\bullet}="ctop";
% {\ar@/^1.75pc/ "atop";"btop"};
% {\ar@/^1pc/ "btop";"ctop"};
% (-575,275)*{}="tl"; % "": \pi itself
% (575,275)*{}="tr";
% (-575,-275)*{}="bl";
% (575,-275)*{}="br";
% "tl";"tr" **\dir{.};
% "tl";"bl" **\dir{.};
% "tr";"br" **\dir{.};
% "bl";"br" **\dir{.};
% (-500,0)*+{\bullet}="a";
% (0,0)*+{\bullet}="b";
% (500,0)*+{\bullet}="c";
% {\ar "a";"b"};
% {\ar@/^1.75pc/ "a";"b"};
% {\ar@/_1.75pc/ "a";"b"};
% {\ar@{=>} (-250,180)*{};(-250,32)*{}} ;
% {\ar@{=>} (-250,-32)*{};(-250,-180)*{}} ;
% {\ar@/^1pc/ "b";"c"};
% {\ar@/_1pc/ "b";"c"};
% {\ar@{=>} (250,100)*{};(250,-100)*{}} ;
% {\ar@{.>} (-75,225)*{};(-25,75)*{}};
% {\ar@{.>} (75,-225)*{};(25,-75)*{}};
% (-475,-225)*{}="tlbot"; % "bot": 1-target
% (675,-225)*{}="trbot";
% (-475,-575)*{}="blbot";
% (675,-575)*{}="brbot";
% "tlbot";"trbot" **\dir{.};
% "tlbot";"blbot" **\dir{.};
% "trbot";"brbot" **\dir{.};
% "blbot";"brbot" **\dir{.};
% (-400,-300)*+{\bullet}="abot";
% (100,-300)*+{\bullet}="bbot";
% (600,-300)*+{\bullet}="cbot";
% {\ar@/_1.75pc/ "abot";"bbot"};
% {\ar@/_1pc/ "bbot";"cbot"};
% (875,-25)*{}="tlright"; % "right": 0-target
% (725,-25)*{}="trright";
% (875,125)*{}="blright";
% (725,125)*{}="brright";
% "tlright";"trright" **\dir{.};
% "tlright";"blright" **\dir{.};
% "trright";"brright" **\dir{.};
% "blright";"brright" **\dir{.};
% (800,50)*+{\bullet}="cright";
% {\ar@{.>} (755,-25)*{};(665,-195)*{}};
% {\ar@{.>} (725,115)*{};(505,240)*{}};
% \endxy
% \]
% \caption{An exploded diagram $e(\pi): \G/2 \rightarrow \pd$ \label{figure:exploded-diagram}}
% \end{figure}

A \emph{globular operad} is a globular set $P$ with maps $a\colon P \to T1$ (``arity''), $e\colon  1 \to P$ (``units''), $m \colon  TP \times_{T1} P \to P$ (``composition''), such that
\[\bfig
\node 1(-250,-150)[1]
\node P(250,-150)[P]
\node T1(0,-650)[T1]
\arrow[1`P;e]
\arrow|l|[1`T1;\eta]
\arrow|r|[P`T1;a]
\node TPxP(1000,0)[TP \times_{T1} P]
\node P'(1750,0)[P]
\node TP(750,-250)[TP]
\node P''(1250,-250)[P]
\node T1'(1000,-500)[T1]
\node TT1(850,-650)[T^2 1]
\node T1''(1300,-800)[T1]
\arrow[TPxP`P';m]
\arrow[TPxP`TP;]
\arrow[TPxP`P'';]
\arrow|a|[TP`T1';T!]
\arrow|a|[P''`T1';a]
\arrow|l|/{@{>}@/_5pt/}/[TP`TT1;Ta]
\arrow|l|/{@{>}@/_5pt/}/[TT1`T1'';\mu]
\arrow|r|[P'`T1'';a]
\efig
\]
commute (i.e.\ $e$ and $m$ are maps over $T1$), satisfying the axioms 
\[m \cdot ( \eta \cdot e \times 1_P) = 1_P = m \cdot (\eta \times e)
\colon  P \to P,
\]
\[m \cdot (\mu \times m) = m \cdot (Tm \times 1_P) : T^2 P \times_{T^2
1} TP \times_{T1} P \to P.
\]
Considering the fibers of $a$, we may view $P$ as a family of sets $P(\pi)$ of ``$\pi$-ary operations'' for each $\pi \in T1$: an element $p$ of $P(\pi)$ is seen as a formal operation symbol, taking $\pi$-shaped labelled pasting diagrams as input and returning $n$-cells as output.  The map $e$ then gives us an $n$-cell ``identity'' operation for each $n$, while $m$ allows us to compose operations appropriately.

For readers not familiar with this definition, it may be helpful to first contemplate the simpler case of \emph{plain operads}, defined by diagrams as above but with $\E = \Sets$ and $T$ the ``free monoid'' monad.   These thus have arities valued in $T1 \iso \N$, and present certain finitary, single-sorted equational theories \cite[2.2]{leinster:book}.  However, ``operad'' from here on will always mean ``globular operad''; we will not deal further with any other kind.

A map $f\colon P \to Q$ of globular operads is a map of underlying globular sets commuting with $a, e$ and $m$.

An \emph{action} of a globular operad $P$ on a globular set $A$ is a composition map $c\colon  TA \times_{T1} P \to A$, satisfying 
\[ c \cdot (\eta \times e) = 1_A \colon  A \to A,\]
\[c \cdot (\mu \times m) = c \cdot (Tc \times 1_P) : T^2 A \times_{T^2
1} TP \times_{T1} P \to A.
\]

Informally, this implements the ``formal operations'' in $P$ as actual composition operations on $A$.  An element of $TA \times_{T1} P$ over some $\pi \in T1_n$ is a $\pi$-shaped diagram $\vec x$ with labels from $A$, together with a $\pi$-ary operation $p$ of $P$; $c$ tells us how to apply $p$ to $\vec x$, yielding a single $n$-cell $c(\vec x, p)$ of $A$.

A \emph{$P$-algebra} is a globular set $A$ together with an action of $P$ on $A$. A map $f\colon A \to B$ of $P$-algebras is a map of globular sets commuting with the $P$-actions.  We denote the resulting category by $P$-$\Alg$.

\begin{exa}The globular set $T1$ is itself trivially an operad (indeed, the terminal one), with $a = 1_{T1}$, i.e.\ $T1(\pi) = 1$ for every $\pi$; a $T1$-algebra is then exactly a strict $\omega$-category.  This fits with our description above of a strict $\omega$-category having a unique composition for each pasting diagram.  
\end{exa}

Weak $\omega$-categories will also be described as algebras for a certain globular operad; to find a suitable operad, we need to specify a little extra structure.

A \emph{contraction} on a map $d\colon  A \to B$ of globular sets is a choice of liftings for fillers of parallel pairs: that is, for each parallel pair $x,x' \in A$ (with the convention that all $0$-cells are parallel), a map $\chi_{x,x'}\colon  B(dx,dx') \to A(x,x')$, such that $d \cdot \chi = 1_B$. A \emph{globular operad with contraction} is a globular operad $P$ with a contraction on the map $a\colon  P \to T1$; this ensures both that enough composition operations exist in $P$, and that the operations will be associative up to cells of the next dimension, themselves satisfying appropriate coherence laws up to yet higher cells, and so on.

 It is shown in \cite{leinster:book} that the category of globular operads with contraction has an initial object $L$; this gives the key definition:

\begin{defi} A \emph{weak $\omega$-category} is an $L$-algebra, where $L$ is the initial operad-with-contraction.
\end{defi}

A map $O \to P$ of operads induces a ``restriction of scalars'' functor $P$-$\Alg \to O$-$\Alg$; so if we have an algebra $A$ for any operad $P$ with contraction, restriction along the unique operad-with-contraction map $L \to P$ endows $A$ with the structure of a weak $\omega$-category.

\begin{exa}
The terminal operad $T1$ has a trivial contraction, giving a canonical functor $\strwCat \to \wkwCat$.
\end{exa}

\begin{exa}
For any space $X$, the set $\Pi_\omega(X)$ of Example \ref{ex:pi-omega} may be naturally made into a weak $\omega$-category, the \emph{fundamental weak $\omega$-groupoid} of $X$. \cite[9.2.7]{leinster:book}
\end{exa}

\subsection{Endomorphism operads and more general actions}

\begin{defi} \label{def:a-pi}
For a globular object $\A$ in a category $\C$, and a pasting diagram $\pi \in T1_n$, we define
\[A_\pi := \textstyle \lim_{c \in \int \pi} A_{\dim c},\]
``the object of diagrams of shape $\pi$ in $\A$'', whenever this limit exists in $\C$.  The maps $\hat{s}^k, \hat{t}^k\colon  \widehat{\d^k pi} \to \hat{\pi}$ induce evident projections $s^k, t^k\colon  A_\pi \to A_{\d^k \pi}$.
\end{defi}

An illustration may be useful here: the definition of $A_\pi$ says, for instance, that if $\pi = (\xymatrix{ \bullet \rtwocell & \bullet \rtwocell & \bullet})$, then
\begin{eqnarray*} A_\pi & := & \lim \left( 
\bfig
\node A0l(0,0)[A_0]
\node A0m(700,0)[A_0]
\node A0r(1400,0)[A_0]
\node A1tl(350,300)[A_1]
\node A1tr(1050,300)[A_1]
\node A1bl(350,-300)[A_1]
\node A1br(1050,-300)[A_1]
\node A2l(350,0)[A_2]
\node A2r(1050,0)[A_2]
\arrow[A1tl`A0l;s]
\arrow[A1tl`A0m;t]
\arrow[A1tr`A0m;s]
\arrow[A1tr`A0r;t]
\arrow[A2l`A1tl;s]
\arrow[A2r`A1tr;s]
\arrow[A2l`A1bl;t]
\arrow[A2r`A1br;t]
\arrow[A1bl`A0l;s]
\arrow[A1bl`A0m;t]
\arrow[A1br`A0m;s]
\arrow[A1br`A0r;t]
\efig
\right) \\
& \iso & A_2 \times_{A_0} A_2,
\end{eqnarray*}
giving the object of 0-composable pairs of 2-cells in $A$.  Similarly, if $\pi$ is the basic $n$-cell, then $A_\pi = A_n$.
 
In the case $\C = \Sets$, the sets $A_\pi$ are precisely the fibers of the map $T! \colon  TA \to T1$, by the description of $T$ as a familially representable functor (\cite[8.1]{leinster:book}).

%% (More abstractly, whenever $\C$ has all limits, the objects $A_\pi$ may be seen to arise from the right Kan extension of $\A$ along $\yon^\op : \G^\op \to {\GSets}^\op$.) %% Query: include this as a footnote, or remove it?  Verdict: it's a nice point, but it's not relevant to this paper!  LEAVE IT OUT; save it for my thesis :-)

\begin{prop}\label{prop:endo-operad}
If $\A$ is a globular object in a category $\C$, and the objects $A_\pi$ exist, then there is an operad $\End_\C(\A)$, the \emph{endomorphism globular operad of $\A$}, in which (for $\pi \in T1_n$) an element of $\End_\C(\A)(\pi)$ is a sequence of maps $(\sigma_0, \tau_0;\sigma_1,\tau_1;\ldots ,\tau_{n-1}; \rho)$,
\[\rho\colon A_\pi \to A_n, \qquad \sigma_i, \tau_i\colon
A_{\d^{n-i}\pi} \to A_i,
\]
commuting appropriately with the source and target maps, in the sense that
\[s \cdot \sigma_i = s \cdot \tau_i = \sigma_{i-1} \cdot s, \qquad s
\cdot \rho = \sigma_{n-1} \cdot s,
\]
\[t \cdot \sigma_i = t \cdot \tau_i = \tau_{i-1} \cdot t, \qquad t
\cdot \rho = \tau_{n-1} \cdot t.
\]
Moreover, if $F \colon  \C \to \D$ is a functor preserving the limits $A_\pi$, we can also construct $\End(FA)$, and there is a natural map of operads $\End(A) \to \End(FA)$.
\end{prop}

In other words, an element of $\End(\A)(\pi)$ is a map $\rho$ composing diagrams of shape $\pi$ in $\A$ to basic $n$-cells of $A$, extending maps $(\sigma_i,\tau_i)$ composing their sources and targets in each lower dimension.

Such a diagram of maps may be more abstractly seen as a natural transformation $\vec{\rho}\colon  A_{\leq \pi} \to A_{\leq n}$ between two evident functors $A_{\leq \pi}, A_{\leq n}\colon  (\G/n)^\op \to \C$.

\proof
This construction of the endomorphism operad is a straightforward generalisation of the topological case given in \cite[9.2.7]{leinster:book}.  The proof requires more technical background on globular operads from \cite{leinster:book} than can be recalled here; readers unfamiliar with this are encouraged to ``black-box'' this proof and skip to the last few paragraphs of the section.

Recall from \cite[6.4]{leinster:book} that if $S$ is a cartesian monad on a locally cartesian closed category $\E$, then any object $A$ of $\E$ has an \emph{endomorphism $S$-operad} $\End_S(A)$ given by the exponential in $\E/S1$ of the objects $SA \to S1$, $S1 \times A \to S1$; in the internal language of $\E$ this may be written as the dependent sum of exponentials:
\[\End_S(A) = \textstyle \sum_{\pi : S1} [SA_\pi, A].\]

Now, in the case of $(\GSets,T)$, this gives for any globular set $\A$ an endomorphism globular operad $\End_T(\A)$.  For $\pi \in T1_n$, an operad element $p \in \End_T(\A)(\pi)$ then corresponds to a commutative triangle
\[\bfig
\node yn(-300,0)[\yon(n)]
\node EndA(300,250)[\End_T(\A)]
\node T1(300,-250)[T 1]
\arrow[yn`EndA;p]
\arrow|l|[yn`T1;\pi]
\arrow|r|[EndA`T1;]
\efig
\]
and hence, by the definition of $\End_T(\A)$ as a dependent sum of exponentials, to a map from the pullback
\[\bfig
\node TyApi(0,250)[\yon(n) \times_\pi T(\A)]
\node yn(-250,0)[\yon(n)]
\node TyA(250,0)[T(\A)]
\node T1(0,-250)[T1]
\arrow[TyApi`yn;]
\arrow[TyApi`TyA;]
\arrow[yn`T1;\pi]
\arrow[TyA`T1;]
\efig
\] % Query: add pullback symbol?
into $\A$.  But this in turn corresponds to a map $\yon(n) \times_\pi T(\A) \to \yon(n) \times \A$ in $\GSets/\yon(n)$ and hence, via the equivalence $\GSets/\yon(n) \equiv [(\G/n)^\op,\Sets]$, to a map $\vec{\rho}\colon  A_{\leq \pi} \to A_{\leq n}$ as described above.

Now, given any category $\C$, consider the category $\E = [\C^\op, \GSets] \iso [(\C \times \G)^\op,\Sets] \iso [\G^\op, \widehat{\C}]$.   Composition with $T$ induces a cartesian monad $T^*$ on $\E$.  Since $\E$ is a presheaf category, it is locally cartesian closed; so any object $Y = Y(-)_\bullet$ of $\E$ has an endomorphism $T^*$-operad $\End_{T^*}(Y(-))$.

Moreover, there is an adjunction 
\[\Delta: \E \two/->`<-/ \GSets : \Gamma,\]
where $\Delta$ is the ``$\C$-constant functor'' functor given by $\Delta(\A)(C)_n = A_n$, and $\Gamma$ is ``$\G$-global sections'': $\Gamma(F)_n = \E(\Delta(\yon(n)),F)$.

Using the familial representability of $T$ and the fact that $\Gamma$ preserves limits, we have a cartesian lax map of cartesian monads $(\Gamma,\kappa)\colon  (\E,T^*) \to (\GSets,T)$, and hence an induced functor $\Gamma_* \colon  T^*$-$\Operads_\E \to T$-$\Operads_{\GSets}$.

\begin{defi}Now, any globular object $\A\colon  \G^\op \to \C$ gives an object $\yon(\A)$ of $\E$; we define
\[\End_\C(\A) := \Gamma_* \End_{T^*}(\yon(\A)).\]
\end{defi}

As in the case $\E = \GSets$, we wish to show that this agrees with the explicit description of $\End_\C(\A)$ given in Proposition \ref{prop:endo-operad}.  Again, an element $p \in \End_\C(\A)(\pi)$ is by definition a triangle in $\GSets$:
\[\bfig
\node yn(-300,0)[\yon(n)]
\node EndA(300,250)[\Gamma_* \End_{T^*}(\yon(\A))]
\node T1(300,-250)[\Gamma_* T^* 1]
\arrow[yn`EndA;p]
\arrow|l|[yn`T1;\pi]
\arrow|r|[EndA`T1;]
\efig
\]
which corresponds, by the adjunction $\Delta \dashv \Gamma$ and the definition of $\End_{T^*}(C)$, to a map
$\Delta(\yon(n)) \times_{\Delta(\pi)} T^*(\yon(\A)) \to \Delta(\yon(n)) \times \yon(\A)$

Since all limits and colimits are pointwise, this corresponds to a
family of maps 
\[\yon(n) \times_{\pi} T (\C(C,\A)) \to \yon(n) \times T^*(\C(C,\A))\]
natural in $C$, so (as before) to a natural family of maps
\[\vec \rho_C\colon  \C(C,\A)_{\leq \pi} \to \C(C,\A)_{\leq n},\]
i.e.\ to a map
\[\vec \rho_{(-)}\colon  \yon(\A)_{\leq \pi} \to \yon(\A)_{\leq n}.\]

But since $\yon$ is full and faithful and preserves all existing limits, if the objects $A_\pi$ exist in $\C$ then this corresponds in turn to a map $\vec \rho\colon  A_{\leq \pi} \to A_{\leq n}$, as desired. \qed

%% Yerks...  I've been bad, I've not argued the functoriality.  It's clear if $F$ is flat, since in this case, it lifts to $Lan_K(F)$ between the presheaf categories, preserving all limits and colimits and hence commuting with everything we did.  But in our case, F may not preserve _all_ limits... is it _flat_?  Probably, but don't want to open that extra can of worms --- would add a bunch more dense undergrowth for poor readers to wade through!  In any case, know from my earlier uglier proof [in first arXiv version of paper] that the functoriality works as long as $F$ preserves the $X_\pi$, and hence in this case.  But how can we see that straightforwardly with this argument??

We can now extend the definitions of the previous subsection.  An \emph{action} of an operad $P$ on $\A$ is a map of operads $P \to \End(\A)$. (If $\C = \Sets$ then this agrees with our earlier notions of an action on a globular set, by \cite[6.4]{leinster:book}).  A \emph{$P$-algebra} in a category $\C$ is a globular object in $\C$ together with a $P$-action; an \emph{internal weak $\omega$-category} in $\C$ is an $L$-algebra in $\C$.

Moreover, an action of $P$ on $\A$ induces an action of $P$ on the globular set $\C(Y,\A)$ for any $Y \in \C$, since $\C(Y, - ) \colon  \C \to \Sets$ preserves all limits, and hence we have maps $P \to \End_\C(\A) \to \End_\Sets(\C(Y,\A))$.

%% From another point of view, we may see $P$-algebras in $\GSets$ as models of a certain essentially algebraic theory in $\Sets$; as one would hope, models of this theory in other categories $\C$ with enough limits are exactly $P$-algebras as defined here.  Unfortunately, the proof of this will not fit in this comment...
