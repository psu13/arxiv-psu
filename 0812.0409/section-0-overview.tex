\section{Introduction}

\subsection{Overview}

\noindent Starting with the Hofmann-Streicher groupoid model \cite{hofmann-streicher}, higher categories have emerged as a natural approach to the semantics of intensional Martin-L\"of type theory.  In the globular approach to higher categories, a higher category has objects (``0-cells''), arrows (``1-cells'') between objects, 2-cells between 1-cells, and so on, with various composition operations and laws depending on the kind of category in question (strict or weak, $n$- or $\omega$-, $\ldots$).  The paradigm for semantics of type theory is then (very roughly!) that types (or contexts) are thought of as objects $\lscott A \rscott$, terms $x:A \types \tau(x):B$ as arrows $\lscott \tau \rscott \colon \lscott A \rscott \to \lscott B \rscott$, terms of identity type $\rho : \Id_B (\tau, \tau')$ as 2-cells $\lscott \rho \rscott \colon \lscott \tau \rscott \to/=>/ \lscott \tau' \rscott$, terms $\chi : \Id (\rho,\rho')$ as 3-cells, and so on.

This idea has recently been explored by various authors in various directions: see for instance \cite{gambino-garner}, \cite{garner:2-d-models}, \cite{awodey-warren}.  One such direction is investigating the structures formed by the syntax of type theory.  In particular, it has been suggested that (terms of) any type, considered together with its higher identity types, should carry the structure of a weak $\omega$-category or -groupoid.  We will show that this is indeed the case, using the definition of weak $\omega$-category given by Tom Leinster in \cite{leinster:book} following the approach of Michael Batanin \cite{batanin:natural-environment}.

(Note that for this construction, based on a specific type $A$, the dimensions of cells are always one lower than described above: $0$-cells will be terms $\tau:A$, $1$-cells will be terms $\rho:\Id_A(\tau,\tau')$, and so on.  This comes from the general rule that if $X$, $A$ are objects of an $n$-category $\bigC$, then $\bigC(X,A)$ forms an $(n-1)$-category whose 0-cells are 1-cells of $\bigC$, and so on.)

While writing this paper, I found that Benno van den Berg had independently discovered a similar proof (proposed in 2006 and completed in unpublished work \cite{benno:talk}); a development of this is forthcoming in joint work of van den Berg and Richard Garner \cite{benno-richard}. \\


\noindent {\bf Acknowledgements.} I would like to thank Steve Awodey, Pierre-Louis Curien, Richard Garner, Chris Kapulkin, Benno van den Berg, Michael Warren, and the anonymous referees for helpful conversations, comments, and support in preparing this paper.

\subsection{Outline of the construction} \label{subsec:outline}

(We assume throughout some general familiarity with the concepts of higher category theory, but not with the particular definition of weak $\omega$-category used, which we will recall in detail in later sections, and similarly for the type theory.)

In the globular approach, an $\omega$-category $\bigC$ has a set $C_n$ of ``$n$-cells'' for each $n > 0$.  The $0$- and $1$-cells correspond to the objects and arrows of an ordinary category: each arrow $f$ has source and target objects $a = s(f)$, $b = t(f)$.  Similarly, the source and target of a 2-cell $\alpha$ are a parallel pair of 1-cells $f,g: a \two b$, and generally the source and target of an $(n+1)$-cell are a parallel pair of $n$-cells.

Cells of each dimension can be composed along a common boundary in any lower dimension, and in a \emph{strict} $\omega$-category, the composition satisfies various associativity, unit, and interchange laws, captured by the generalised associativity law: each labelled pasting diagram has a unique composite. (See illustrations in Fig.\ \ref{figure:assoc-laws}).

\begin{figure}
\[
\begin{array}{c}
\begin{array}{cccc}
\ \xy
(0,0)*{\bullet};
(0,80)*{a};
\endxy \quad
&
\ \xy
(0,0)*{\bullet}="a";
(0,80)*{\scriptstyle a};
(400,0)*{\bullet}="b";
(400,80)*{\scriptstyle b};
{\ar "a";"b"};
(200,80)*{f};
\endxy \ 
&
\ \xy
(0,0)*+{\bullet}="a";
(0,80)*{\scriptstyle a};
(450,0)*+{\bullet}="b";
(450,80)*{\scriptstyle b};
{\ar@/^1pc/^{f} "a";"b"};
{\ar@/_1pc/_{g} "a";"b"};
{\ar@{=>} (210,85)*{};(210,-85)*{}};
(280,0)*{\alpha};
\endxy \ 
&
\ \xy
(0,0)*+{\bullet}="a";
(600,0)*+{\bullet}="b";
{\ar@/^1.75pc/^{f} "a";"b"};
{\ar@/_1.75pc/_{g} "a";"b"};
{\ar@2{->}@/_0.5pc/|{\alpha} (220,140);(220,-140)} ;
{\ar@2{->}@/^0.5pc/|{\beta} (380,140);(380,-140)} ;
{\ar@3{->} (225,-20);(375,-20)};
(300,60)*{\Theta};
\endxy \ 
\end{array} \\
\begin{array}{ccc}
\ \xy(0,0)*{\bullet}="a";
(0,80)*{\scriptstyle a};
(300,0)*{\bullet}="b";
(300,80)*{\scriptstyle b};
(600,0)*{\bullet}="c";
(600,80)*{\scriptstyle c};
{\ar^f "a";"b"};
{\ar^g "b";"c"};
\endxy \ 
&
\ \xy
(0,0)*+{\bullet}="a";
(0,80)*{\scriptstyle a};
(500,0)*+{\bullet}="b";
(500,80)*{\scriptstyle b};
{\ar@/^1.75pc/|f "a";"b"};
{\ar|{f'} "a";"b"};
{\ar@/_1.75pc/|{f''} "a";"b"};
{\ar@{=>}^{\alpha} (250,160)*{};(250,50)*{}} ;
{\ar@{=>}^{\gamma} (250,-50)*{};(250,-160)*{}} ;
(0,-250)*{\ };
\endxy \ 
&
\ \xy
(0,0)*+{\bullet}="a";
(0,80)*{\scriptstyle a};
(400,0)*+{\bullet}="b";
(400,80)*{\scriptstyle b};
(800,0)*+{\bullet}="c";
(800,80)*{\scriptstyle c};
{\ar@/^1.1pc/|f "a";"b"};
{\ar@/_1.1pc/|{f'} "a";"b"};
{\ar@/^1.1pc/|g "b";"c"};
{\ar@/_1.1pc/|{g'} "b";"c"};
{\ar@{=>}^{\alpha} (200,80)*{};(200,-80)*{}} ;
{\ar@{=>}^{\beta} (600,80)*{};(600,-80)*{}} ;
\endxy \ \\
g \cdot_0 f &
\gamma \cdot_1 \alpha &
\beta \cdot_0 \alpha
\end{array}
\\
\begin{array}{cc}
\ \xy(0,0)*{\bullet}="a";
%(0,80)*{\scriptstyle a};
(300,0)*{\bullet}="b";
%(300,80)*{\scriptstyle b};
(600,0)*{\bullet}="c";
%(600,80)*{\scriptstyle c};
(900,0)*{\bullet}="d";
%(900,80)*{\scriptstyle d};
{\ar^f "a";"b"};
{\ar^g "b";"c"};
{\ar^h "c";"d"};
\endxy \ &
\ \xy
(0,0)*+{\bullet}="a";
(400,0)*+{\bullet}="b";
{\ar@/^1.5pc/ "a";"b"};
{\ar "a";"b"};
{\ar@/_1.5pc/ "a";"b"};
{\ar@{=>}^{\alpha} (200,150)*{};(200,25)*{}} ;
{\ar@{=>}^{\gamma} (200,-25)*{};(200,-150)*{}} ;
(800,0)*+{\bullet}="c";
{\ar@/^1.5pc/ "b";"c"};
{\ar "b";"c"};
{\ar@/_1.5pc/ "b";"c"};
{\ar@{=>}^{\beta} (600,150)*{};(600,25)*{}} ;
{\ar@{=>}^{\delta} (600,-25)*{};(600,-150)*{}};
(0,250)*{\ };
(0,-220)*{\ };
\endxy \ \\
\begin{array}{c} h \cdot_0 (g \cdot_0 f) =  \\ (h \cdot_0 g) \cdot_0 f \end{array} &
\begin{array}{c}(\delta \cdot_0 \gamma) \cdot_1 (\beta \cdot_0 \alpha) = \\
(\gamma \cdot_1 \alpha) \cdot_0 (\delta \cdot_1 \beta)\end{array}
\end{array}
\end{array}
\]
\caption{Some cells, composites, and associativities in a strict $\omega$-category \label{figure:assoc-laws}} 
\end{figure}
\noindent In a weak $\omega$-category, we do not expect strict associativity, so may have multiple composites for a given pasting diagram, but we do demand that these composites agree up to cells of the next dimension (``up to homotopy''), and that these associativity cells satisfy certain coherence laws of their own, again up to cells of higher dimension, and so on.

This is exactly the situation we find in intensional type theory.  For instance, even in constructing a term witnessing the transitivity of identity---that is, a composition law for the pasting diagram $(\xymatrix{ \bullet \ar[r] & \bullet \ar[r] & \bullet })$, or explicitly a term $c$ such that 
\[x,y,z:X,\ p:\Id(x,y),\ q:\Id(y,z) \types c(q,p): \Id(x,z)\]
---one finds that there is no single canonical candidate: most obvious are the two equally natural terms $c_l$, $c_r$ obtained by applying ($\Id$-\elim) to $p$ or to $q$ respectively.  These are not definitionally equal, but are propositionally equal, i.e.\ equal up to a 2-cell: there is a term $e$ with
\[x,y,z:X,\ p:\Id(x,y),\ q:\Id(y,z) \types e(q,p): \Id(c_l(q,p),c_r(q,p)).\]

In Leinster's definition \cite{leinster:book}, a system of composition laws of this sort is wrapped up in the algebraic structure of a \emph{globular operad with contraction}, and a weak $\omega$-category is given by a globular set equipped with an \emph{action} of such an operad.  We generalise this slightly, to define an \emph{internal weak $\omega$-category} in any suitable category $\C$.

Accordingly, we would like to find an operad-with-contraction $\operadP$ of all such type-theoretically definable composition laws, acting on terms of any type and its identity types.  In fact, rather than using the full type theory for this, it is more convenient to consider the composition laws definable using just the $\Id$- rules, hence also obtaining the construction for a wider class of theories.

The heart of the paper is Sect.\ \ref{sec:the-operad}, where we formalise this idea.  We consider $\MLfrag[X]$, the fragment of intensional Martin-L\"of type theory generated just by the structural and $\Id$-rules plus a single generic base type $X$. The operad $\operadP$ of definable composition therein laws may then be formally constructed as an endomorphism operad in its syntactic category $\C(\MLfrag [X])$; and by some analysis of the fragment $\MLfrag[X]$, we show that $\operadP$ is contractible.

Since $X$ is generic, $\operadP$ acts on all other types, giving our main theorem:

\begin{thmstar}Let $\T$ be any type theory extending $\MLfrag$, and $A$ any type of $\T$.  Then the system of types $(A, Id_A, Id_{Id_A}, \ldots)$ is equipped naturally with a $\operadP$-action, and hence with the structure of an internal weak $\omega$-category in $\C(\T)$.
\end{thmstar}

To prepare for this, we first lay out in Sect.\ \ref{sec:type-theory-background} our presentation of the type theory $\MLfrag$, and in Sect.\ \ref{sec:operads-background} the relevant background on globular operads and their algebras.

