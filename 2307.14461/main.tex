\documentclass[submission]{eptcs}
\providecommand{\event}{ACT 2023}

\usepackage[british]{babel}
\usepackage[utf8]{inputenc}
\usepackage[T1]{fontenc}

% Bibliography

%\usepackage[hyphen]{url}
\hypersetup{colorlinks, urlcolor=RoyalBlue, linkcolor=Red!80, citecolor=Green} % Needed to display links
\usepackage[hyphenbreaks]{breakurl}

%  Graphics
\usepackage{graphicx} % Needed to include images
%\usepackage{subcaption} % Needed to define subfigures
\usepackage[dvipsnames]{xcolor} % Needed to use color names in hyperref options

% Tikz
  \usepackage{tikz} % TikZ
  \usetikzlibrary{
    cd, % to make easy commutative diagrams
    petri, % To draw petri nets
    backgrounds, % To define image backgrounds
    arrows, % To use and define further arrow tips
    positioning, % To use expressions like "right = 1 of 1"
    decorations.markings, % Needed to define oriented wiring diagrams
    decorations.pathmorphing,
    calc,  % Needed to define oriented wiring diagrams
    fit, % Needed to compose wiring diagrams
    shapes.multipart
  }

% A TikZ style for curved arrows of a fixed height, due to AndréC.
\tikzset{curve/.style={settings={#1},to path={(\tikztostart)
    .. controls ($(\tikztostart)!\pv{pos}!(\tikztotarget)!\pv{height}!270:(\tikztotarget)$)
    and ($(\tikztostart)!1-\pv{pos}!(\tikztotarget)!\pv{height}!270:(\tikztotarget)$)
    .. (\tikztotarget)\tikztonodes}},
    settings/.code={\tikzset{quiver/.cd,#1}
        \def\pv##1{\pgfkeysvalueof{/tikz/quiver/##1}}},
    quiver/.cd,pos/.initial=0.35,height/.initial=0}

% TikZ arrowhead/tail styles.
\tikzset{tail reversed/.code={\pgfsetarrowsstart{tikzcd to}}}
\tikzset{2tail/.code={\pgfsetarrowsstart{Implies[reversed]}}}
\tikzset{2tail reversed/.code={\pgfsetarrowsstart{Implies}}}
% TikZ arrow styles.
\tikzset{no body/.style={/tikz/dash pattern=on 0 off 1mm}}

% Maths
\usepackage{mathtools} % Basic math capabilities
\usepackage{amssymb} % Extra mathematical symbols (loads amsfonts)
\usepackage{dsfont}
\usepackage{mathrsfs}
\let\proof\relax
\let\endproof\relax
\usepackage{amsthm} % Theorem environments
\usepackage{stmaryrd} % Some maths symbols for logic and computer science

%% WRITING %%
\usepackage{relsize} % Additional relative sizes for fonts
\usepackage{microtype} % Improves appearance of writing
\usepackage{multicol} % Multi-column environments
\usepackage{csquotes} % Environments for quotes
\usepackage{xspace}
\usepackage{enumitem}
\usepackage{fontawesome} % Popular fontawesome symbols

% List of Symbols
\usepackage{tabularx, cellspace} % Needed to typeset the tables for listing symbols
\setlength\cellspacetoplimit{3pt}
\setlength\cellspacebottomlimit{3pt}
% David's
\usepackage[capitalize]{cleveref}

\usepackage{datetime}

% Theorem Environments
%
% Theorem Counters
%
\newcounter{theoremUnified} % Unified coutner for all theorem environments
\def\thetheoremUnified{\arabic{section}} % Needed to have counters going with sections
\numberwithin{theoremUnified}{section} % Numbering within sections
\numberwithin{theoremUnified}{section} % Equations are also numbered within sections

% Theorem Styles
%
\newtheoremstyle{plainStyle} % Plain theorem style
	{2mm} % Space above
	{2mm} % Space below
	{} % Body font
	{} % Indent amount
	{\bfseries} % Theorem head font
	{.} % Punctuation after theorem head
	{.5em} % Space after theorem head
	{} % Theorem head spec (can be left empty, meaning `normal')

\newtheoremstyle{italicStyle} % Italic theorem style
	{2mm} % Space above
	{2mm} % Space below
	{\itshape} % Body font
	{} % Indent amount
	{\bfseries} % Theorem head font
	{.} % Punctuation after theorem head
	{.5em} % Space after theorem head
	{} % Theorem head spec (can be left empty, meaning `normal')

% Environments
\theoremstyle{plainStyle}
	\newtheorem{notation}{Notation}{\rmfamily}{\rmfamily}
	\newtheorem{definition}{Definition}{\rmfamily}{\rmfamily}
	\newtheorem{remark}{Remark}{\rmfamily}{\rmfamily}
	\newtheorem{example}{Example}{\rmfamily}{\rmfamily}
	\newtheorem{counterexample}{Counterexample}{\rmfamily}{\rmfamily}

\theoremstyle{italicStyle}
	\newtheorem{proposition}{Proposition}{\rmfamily}{\rmfamily}
	\newtheorem{theorem}{Theorem}{\rmfamily}{\rmfamily}
	\newtheorem{corollary}{Corollary}{\rmfamily}{\rmfamily}

\relpenalty=10000
\binoppenalty=10000
\setlength\parindent{1em}

% Autoref styling
% Capitalize references
    \renewcommand{\chapterautorefname}{Chapter}
    \renewcommand{\sectionautorefname}{Section}
    \renewcommand{\theoremautorefname}{Theorem}
    \renewcommand{\chapterautorefname}{Chapter}
    \renewcommand{\sectionautorefname}{Section}
% Refer to sub- and subsubsections as 'section'
    \let\subsectionautorefname\sectionautorefname
    \let\subsubsectionautorefname\sectionautorefname
% Give a name to new environments
    \newcommand{\notationautorefname}{Notation}
    \newcommand{\definitionautorefname}{Definition}
    \newcommand{\propositionautorefname}{Proposition}
    \newcommand{\remarkautorefname}{Remark}
    \newcommand{\exampleautorefname}{Example}
    \newcommand{\counterexampleautorefname}{Counterexample}
    \newcommand{\corollaryautorefname}{Corollary}
%%%% MACROS FOR NOTATION %%%%
% Use these for any notation where there are multiple options.

%%% Notes and exercise sections
\makeatletter
\newcommand{\sectionNotes}{\phantomsection\section*{Notes}\addcontentsline{toc}{section}{Notes}\markright{\textsc{\@chapapp{} \thechapter{} Notes}}}
\newcommand{\sectionExercises}[1]{\ifdef{\OPTexerciseperpage}{\newpage}{}\phantomsection\section*{Exercises}\addcontentsline{toc}{section}{Exercises}\markright{\textsc{\@chapapp{} \thechapter{} Exercises}}}
\makeatother

%%% Definitional equality (used infix) %%%
\newcommand{\jdeq}{\equiv}      % An equality judgment
\let\judgeq\jdeq
%\newcommand{\defeq}{\coloneqq}  % An equality currently being defined
\newcommand{\defeq}{\vcentcolon\equiv}  % A judgmental equality currently being defined

%%% Term being defined
\newcommand{\define}[1]{\textbf{#1}}

%%% Vec (for example)

\newcommand{\Vect}{\ensuremath{\mathsf{Vec}}}
\newcommand{\Fin}{\ensuremath{\mathsf{Fin}}}
\newcommand{\fmax}{\ensuremath{\mathsf{fmax}}}
\newcommand{\seq}[1]{\langle #1\rangle}

%%% Dependent products %%%
\def\prdsym{\textstyle\prod}
%% Call the macro like \prd{x,y:A}{p:x=y} with any number of
%% arguments.  Make sure that whatever comes *after* the call doesn't
%% begin with an open-brace, or it will be parsed as another argument.
\makeatletter
% Currently the macro is configured to produce
%     {\textstyle\prod}(x:A) \; {\textstyle\prod}(y:B),{\ }
% in display-math mode, and
%     \prod_{(x:A)} \prod_{y:B}
% in text-math mode.
% \def\prd#1{\@ifnextchar\bgroup{\prd@parens{#1}}{%
%     \@ifnextchar\sm{\prd@parens{#1}\@eatsm}{%
%         \prd@noparens{#1}}}}
\def\prd#1{\@ifnextchar\bgroup{\prd@parens{#1}}{%
    \@ifnextchar\sm{\prd@parens{#1}\@eatsm}{%
    \@ifnextchar\prd{\prd@parens{#1}\@eatprd}{%
    \@ifnextchar\;{\prd@parens{#1}\@eatsemicolonspace}{%
    \@ifnextchar\\{\prd@parens{#1}\@eatlinebreak}{%
    \@ifnextchar\narrowbreak{\prd@parens{#1}\@eatnarrowbreak}{%
      \prd@noparens{#1}}}}}}}}
\def\prd@parens#1{\@ifnextchar\bgroup%
  {\mathchoice{\@dprd{#1}}{\@tprd{#1}}{\@tprd{#1}}{\@tprd{#1}}\prd@parens}%
  {\@ifnextchar\sm%
    {\mathchoice{\@dprd{#1}}{\@tprd{#1}}{\@tprd{#1}}{\@tprd{#1}}\@eatsm}%
    {\mathchoice{\@dprd{#1}}{\@tprd{#1}}{\@tprd{#1}}{\@tprd{#1}}}}}
\def\@eatsm\sm{\sm@parens}
\def\prd@noparens#1{\mathchoice{\@dprd@noparens{#1}}{\@tprd{#1}}{\@tprd{#1}}{\@tprd{#1}}}
% Helper macros for three styles
\def\lprd#1{\@ifnextchar\bgroup{\@lprd{#1}\lprd}{\@@lprd{#1}}}
\def\@lprd#1{\mathchoice{{\textstyle\prod}}{\prod}{\prod}{\prod}({\textstyle #1})\;}
\def\@@lprd#1{\mathchoice{{\textstyle\prod}}{\prod}{\prod}{\prod}({\textstyle #1}),\ }
\def\tprd#1{\@tprd{#1}\@ifnextchar\bgroup{\tprd}{}}
\def\@tprd#1{\mathchoice{{\textstyle\prod_{(#1)}}}{\prod_{(#1)}}{\prod_{(#1)}}{\prod_{(#1)}}}
\def\dprd#1{\@dprd{#1}\@ifnextchar\bgroup{\dprd}{}}
\def\@dprd#1{\prod_{(#1)}\,}
\def\@dprd@noparens#1{\prod_{#1}\,}

% Look through spaces and linebreaks
\def\@eatnarrowbreak\narrowbreak{%
  \@ifnextchar\prd{\narrowbreak\@eatprd}{%
    \@ifnextchar\sm{\narrowbreak\@eatsm}{%
      \narrowbreak}}}
\def\@eatlinebreak\\{%
  \@ifnextchar\prd{\\\@eatprd}{%
    \@ifnextchar\sm{\\\@eatsm}{%
      \\}}}
\def\@eatsemicolonspace\;{%
  \@ifnextchar\prd{\;\@eatprd}{%
    \@ifnextchar\sm{\;\@eatsm}{%
      \;}}}

%%% Lambda abstractions.
% Each variable being abstracted over is a separate argument.  If
% there is more than one such argument, they *must* be enclosed in
% braces.  Arguments can be untyped, as in \lam{x}{y}, or typed with a
% colon, as in \lam{x:A}{y:B}. In the latter case, the colons are
% automatically noticed and (with current implementation) the space
% around the colon is reduced.  You can even give more than one variable
% the same type, as in \lam{x,y:A}.
\def\lam#1{{\lambda}\@lamarg#1:\@endlamarg\@ifnextchar\bgroup{.\,\lam}{.\,}}
\def\@lamarg#1:#2\@endlamarg{\if\relax\detokenize{#2}\relax #1\else\@lamvar{\@lameatcolon#2},#1\@endlamvar\fi}
\def\@lamvar#1,#2\@endlamvar{(#2\,{:}\,#1)}
% \def\@lamvar#1,#2{{#2}^{#1}\@ifnextchar,{.\,{\lambda}\@lamvar{#1}}{\let\@endlamvar\relax}}
\def\@lameatcolon#1:{#1}
\let\lamt\lam
% This version silently eats any typing annotation.
\def\lamu#1{{\lambda}\@lamuarg#1:\@endlamuarg\@ifnextchar\bgroup{.\,\lamu}{.\,}}
\def\@lamuarg#1:#2\@endlamuarg{#1}

%%% Dependent products written with \forall, in the same style
\def\fall#1{\forall (#1)\@ifnextchar\bgroup{.\,\fall}{.\,}}

%%% Existential quantifier %%%
\def\exis#1{\exists (#1)\@ifnextchar\bgroup{.\,\exis}{.\,}}

%%% Dependent sums %%%
\def\smsym{\textstyle\sum}
% Use in the same way as \prd
\def\sm#1{\@ifnextchar\bgroup{\sm@parens{#1}}{%
    \@ifnextchar\prd{\sm@parens{#1}\@eatprd}{%
    \@ifnextchar\sm{\sm@parens{#1}\@eatsm}{%
    \@ifnextchar\;{\sm@parens{#1}\@eatsemicolonspace}{%
    \@ifnextchar\\{\sm@parens{#1}\@eatlinebreak}{%
    \@ifnextchar\narrowbreak{\sm@parens{#1}\@eatnarrowbreak}{%
        \sm@noparens{#1}}}}}}}}
\def\sm@parens#1{\@ifnextchar\bgroup%
  {\mathchoice{\@dsm{#1}}{\@tsm{#1}}{\@tsm{#1}}{\@tsm{#1}}\sm@parens}%
  {\@ifnextchar\prd%
    {\mathchoice{\@dsm{#1}}{\@tsm{#1}}{\@tsm{#1}}{\@tsm{#1}}\@eatprd}%
    {\mathchoice{\@dsm{#1}}{\@tsm{#1}}{\@tsm{#1}}{\@tsm{#1}}}}}
\def\@eatprd\prd{\prd@parens}
\def\sm@noparens#1{\mathchoice{\@dsm@noparens{#1}}{\@tsm{#1}}{\@tsm{#1}}{\@tsm{#1}}}
\def\lsm#1{\@ifnextchar\bgroup{\@lsm{#1}\lsm}{\@@lsm{#1}}}
\def\@lsm#1{\mathchoice{{\textstyle\sum}}{\sum}{\sum}{\sum}({\textstyle #1})\;}
\def\@@lsm#1{\mathchoice{{\textstyle\sum}}{\sum}{\sum}{\sum}({\textstyle #1}),\ }
\def\tsm#1{\@tsm{#1}\@ifnextchar\bgroup{\tsm}{}}
\def\@tsm#1{\mathchoice{{\textstyle\sum_{(#1)}}}{\sum_{(#1)}}{\sum_{(#1)}}{\sum_{(#1)}}}
\def\dsm#1{\@dsm{#1}\@ifnextchar\bgroup{\dsm}{}}
\def\@dsm#1{\sum_{(#1)}\,}
\def\@dsm@noparens#1{\sum_{#1}\,}

%%% W-types
\def\wtypesym{{\mathsf{W}}}
\def\wtype#1{\@ifnextchar\bgroup%
  {\mathchoice{\@twtype{#1}}{\@twtype{#1}}{\@twtype{#1}}{\@twtype{#1}}\wtype}%
  {\mathchoice{\@twtype{#1}}{\@twtype{#1}}{\@twtype{#1}}{\@twtype{#1}}}}
\def\lwtype#1{\@ifnextchar\bgroup{\@lwtype{#1}\lwtype}{\@@lwtype{#1}}}
\def\@lwtype#1{\mathchoice{{\textstyle\mathsf{W}}}{\mathsf{W}}{\mathsf{W}}{\mathsf{W}}({\textstyle #1})\;}
\def\@@lwtype#1{\mathchoice{{\textstyle\mathsf{W}}}{\mathsf{W}}{\mathsf{W}}{\mathsf{W}}({\textstyle #1}),\ }
\def\twtype#1{\@twtype{#1}\@ifnextchar\bgroup{\twtype}{}}
\def\@twtype#1{\mathchoice{{\textstyle\mathsf{W}_{(#1)}}}{\mathsf{W}_{(#1)}}{\mathsf{W}_{(#1)}}{\mathsf{W}_{(#1)}}}
\def\dwtype#1{\@dwtype{#1}\@ifnextchar\bgroup{\dwtype}{}}
\def\@dwtype#1{\mathsf{W}_{(#1)}\,}

\newcommand{\suppsym}{{\mathsf{sup}}}
\newcommand{\supp}{\ensuremath\suppsym\xspace}

\def\wtypeh#1{\@ifnextchar\bgroup%
  {\mathchoice{\@lwtypeh{#1}}{\@twtypeh{#1}}{\@twtypeh{#1}}{\@twtypeh{#1}}\wtypeh}%
  {\mathchoice{\@@lwtypeh{#1}}{\@twtypeh{#1}}{\@twtypeh{#1}}{\@twtypeh{#1}}}}
\def\lwtypeh#1{\@ifnextchar\bgroup{\@lwtypeh{#1}\lwtypeh}{\@@lwtypeh{#1}}}
\def\@lwtypeh#1{\mathchoice{{\textstyle\mathsf{W}^h}}{\mathsf{W}^h}{\mathsf{W}^h}{\mathsf{W}^h}({\textstyle #1})\;}
\def\@@lwtypeh#1{\mathchoice{{\textstyle\mathsf{W}^h}}{\mathsf{W}^h}{\mathsf{W}^h}{\mathsf{W}^h}({\textstyle #1}),\ }
\def\twtypeh#1{\@twtypeh{#1}\@ifnextchar\bgroup{\twtypeh}{}}
\def\@twtypeh#1{\mathchoice{{\textstyle\mathsf{W}^h_{(#1)}}}{\mathsf{W}^h_{(#1)}}{\mathsf{W}^h_{(#1)}}{\mathsf{W}^h_{(#1)}}}
\def\dwtypeh#1{\@dwtypeh{#1}\@ifnextchar\bgroup{\dwtypeh}{}}
\def\@dwtypeh#1{\mathsf{W}^h_{(#1)}\,}


\makeatother

% Other notations related to dependent sums
\let\setof\Set    % from package 'braket', write \setof{ x:A | P(x) }.
\newcommand{\pair}{\ensuremath{\mathsf{pair}}\xspace}
\newcommand{\tup}[2]{(#1,#2)}
\newcommand{\proj}[1]{\ensuremath{\mathsf{pr}_{#1}}\xspace}
\newcommand{\fst}{\ensuremath{\proj1}\xspace}
\newcommand{\snd}{\ensuremath{\proj2}\xspace}
\newcommand{\ac}{\ensuremath{\mathsf{ac}}\xspace} % not needed in symbol index

%%% recursor and induction
\newcommand{\rec}[1]{\mathsf{rec}_{#1}}
\newcommand{\ind}[1]{\mathsf{ind}_{#1}}
\newcommand{\indid}[1]{\ind{=_{#1}}} % (Martin-Lof) path induction principle for identity types
\newcommand{\indidb}[1]{\ind{=_{#1}}'} % (Paulin-Mohring) based path induction principle for identity types

%%% Uniqueness principles
\newcommand{\uniq}[1]{\mathsf{uniq}_{#1}}

% Paths in pairs
\newcommand{\pairpath}{\ensuremath{\mathsf{pair}^{\mathord{=}}}\xspace}
% \newcommand{\projpath}[1]{\proj{#1}^{\mathord{=}}}
\newcommand{\projpath}[1]{\ensuremath{\apfunc{\proj{#1}}}\xspace}
\newcommand{\pairct}{\ensuremath{\mathsf{pair}^{\mathord{\ct}}}\xspace}

%%% For quotients %%%
%\newcommand{\pairr}[1]{{\langle #1\rangle}}
\newcommand{\pairr}[1]{{\mathopen{}(#1)\mathclose{}}}
\newcommand{\Pairr}[1]{{\mathopen{}\left(#1\right)\mathclose{}}}

% \newcommand{\type}{\ensuremath{\mathsf{Type}}} % this command is overridden below, so it's commented out
\newcommand{\im}{\ensuremath{\mathsf{im}}} % the image

%%% 2D path operations
\newcommand{\leftwhisker}{\mathbin{{\ct}_{\mathsf{l}}}}  % was \ell
\newcommand{\rightwhisker}{\mathbin{{\ct}_{\mathsf{r}}}} % was r
\newcommand{\hct}{\star}

%%% modalities %%%
\newcommand{\modal}{\ensuremath{\ocircle}}
\let\reflect\modal
\newcommand{\modaltype}{\ensuremath{\type_\modal}}
% \newcommand{\ism}[1]{\ensuremath{\mathsf{is}_{#1}}}
% \newcommand{\ismodal}{\ism{\modal}}
% \newcommand{\existsmodal}{\ensuremath{{\exists}_{\modal}}}
% \newcommand{\existsmodalunique}{\ensuremath{{\exists!}_{\modal}}}
% \newcommand{\modalfunc}{\textsf{\modal-fun}}
% \newcommand{\Ecirc}{\ensuremath{\mathsf{E}_\modal}}
% \newcommand{\Mcirc}{\ensuremath{\mathsf{M}_\modal}}
\newcommand{\mreturn}{\ensuremath{\eta}}
\let\project\mreturn
%\newcommand{\mbind}[1]{\ensuremath{\hat{#1}}}
\newcommand{\ext}{\mathsf{ext}}
%\newcommand{\mmap}[1]{\ensuremath{\bar{#1}}}
%\newcommand{\mjoin}{\ensuremath{\mreturn^{-1}}}
% Subuniverse
\renewcommand{\P}{\ensuremath{\type_{P}}\xspace}

%%% Localizations
% \newcommand{\islocal}[1]{\ensuremath{\mathsf{islocal}_{#1}}\xspace}
% \newcommand{\loc}[1]{\ensuremath{\mathcal{L}_{#1}}\xspace}

%%% Identity types %%%
\newcommand{\idsym}{{=}}
\newcommand{\id}[3][]{\ensuremath{#2 =_{#1} #3}\xspace}
\newcommand{\idtype}[3][]{\ensuremath{\mathsf{Id}_{#1}(#2,#3)}\xspace}
\newcommand{\idtypevar}[1]{\ensuremath{\mathsf{Id}_{#1}}\xspace}
% A propositional equality currently being defined
\newcommand{\defid}{\coloneqq}

%%% Dependent paths
\newcommand{\dpath}[4]{#3 =^{#1}_{#2} #4}

%%% singleton
% \newcommand{\sgl}{\ensuremath{\mathsf{sgl}}\xspace}
% \newcommand{\sctr}{\ensuremath{\mathsf{sctr}}\xspace}

%%% Reflexivity terms %%%
% \newcommand{\reflsym}{{\mathsf{refl}}}
\newcommand{\refl}[1]{\ensuremath{\mathsf{refl}_{#1}}\xspace}

%%% Path concatenation (used infix, in diagrammatic order) %%%
\newcommand{\ct}{%
  \mathchoice{\mathbin{\raisebox{0.5ex}{$\displaystyle\centerdot$}}}%
             {\mathbin{\raisebox{0.5ex}{$\centerdot$}}}%
             {\mathbin{\raisebox{0.25ex}{$\scriptstyle\,\centerdot\,$}}}%
             {\mathbin{\raisebox{0.1ex}{$\scriptscriptstyle\,\centerdot\,$}}}
}

%%% Path reversal %%%
\newcommand{\opp}[1]{\mathord{{#1}^{-1}}}
\let\rev\opp

%%% Coherence paths %%%
\newcommand{\ctassoc}{\mathsf{assoc}} % associativity law

%%% Transport (covariant) %%%
\newcommand{\trans}[2]{\ensuremath{{#1}_{*}\mathopen{}\left({#2}\right)\mathclose{}}\xspace}
\let\Trans\trans
%\newcommand{\Trans}[2]{\ensuremath{{#1}_{*}\left({#2}\right)}\xspace}
\newcommand{\transf}[1]{\ensuremath{{#1}_{*}}\xspace} % Without argument
%\newcommand{\transport}[2]{\ensuremath{\mathsf{transport}_{*} \: {#2}\xspace}}
\newcommand{\transfib}[3]{\ensuremath{\mathsf{transport}^{#1}(#2,#3)\xspace}}
\newcommand{\Transfib}[3]{\ensuremath{\mathsf{transport}^{#1}\Big(#2,\, #3\Big)\xspace}}
\newcommand{\transfibf}[1]{\ensuremath{\mathsf{transport}^{#1}\xspace}}

%%% 2D transport
\newcommand{\transtwo}[2]{\ensuremath{\mathsf{transport}^2\mathopen{}\left({#1},{#2}\right)\mathclose{}}\xspace}

%%% Constant transport
\newcommand{\transconst}[3]{\ensuremath{\mathsf{transportconst}}^{#1}_{#2}(#3)\xspace}
\newcommand{\transconstf}{\ensuremath{\mathsf{transportconst}}\xspace}

%%% Map on paths %%%
\newcommand{\mapfunc}[1]{\ensuremath{\mathsf{ap}_{#1}}\xspace} % Without argument
\newcommand{\map}[2]{\ensuremath{{#1}\mathopen{}\left({#2}\right)\mathclose{}}\xspace}
\let\Ap\map
%\newcommand{\Ap}[2]{\ensuremath{{#1}\left({#2}\right)}\xspace}
\newcommand{\mapdepfunc}[1]{\ensuremath{\mathsf{apd}_{#1}}\xspace} % Without argument
% \newcommand{\mapdep}[2]{\ensuremath{{#1}\llparenthesis{#2}\rrparenthesis}\xspace}
\newcommand{\mapdep}[2]{\ensuremath{\mapdepfunc{#1}\mathopen{}\left(#2\right)\mathclose{}}\xspace}
\let\apfunc\mapfunc
\let\ap\map
\let\apdfunc\mapdepfunc
\let\apd\mapdep

%%% 2D map on paths
\newcommand{\aptwofunc}[1]{\ensuremath{\mathsf{ap}^2_{#1}}\xspace}
\newcommand{\aptwo}[2]{\ensuremath{\aptwofunc{#1}\mathopen{}\left({#2}\right)\mathclose{}}\xspace}
\newcommand{\apdtwofunc}[1]{\ensuremath{\mathsf{apd}^2_{#1}}\xspace}
\newcommand{\apdtwo}[2]{\ensuremath{\apdtwofunc{#1}\mathopen{}\left(#2\right)\mathclose{}}\xspace}

%%% Identity functions %%%
\newcommand{\idfunc}[1][]{\ensuremath{\mathsf{id}_{#1}}\xspace}

%%% Homotopies (written infix) %%%
\newcommand{\htpy}{\sim}

%%% Other meanings of \sim
\newcommand{\bisim}{\sim}       % bisimulation
\newcommand{\eqr}{\sim}         % an equivalence relation

%%% Equivalence types %%%
\newcommand{\eqv}[2]{\ensuremath{#1 \simeq #2}\xspace}
\newcommand{\eqvspaced}[2]{\ensuremath{#1 \;\simeq\; #2}\xspace}
\newcommand{\eqvsym}{\simeq}    % infix symbol
\newcommand{\texteqv}[2]{\ensuremath{\mathsf{Equiv}(#1,#2)}\xspace}
\newcommand{\isequiv}{\ensuremath{\mathsf{isequiv}}}
\newcommand{\qinv}{\ensuremath{\mathsf{qinv}}}
\newcommand{\ishae}{\ensuremath{\mathsf{ishae}}}
\newcommand{\linv}{\ensuremath{\mathsf{linv}}}
\newcommand{\rinv}{\ensuremath{\mathsf{rinv}}}
\newcommand{\biinv}{\ensuremath{\mathsf{biinv}}}
\newcommand{\lcoh}[3]{\mathsf{lcoh}_{#1}(#2,#3)}
\newcommand{\rcoh}[3]{\mathsf{rcoh}_{#1}(#2,#3)}
\newcommand{\hfib}[2]{{\mathsf{fib}}_{#1}(#2)}

%%% Map on total spaces %%%
\newcommand{\total}[1]{\ensuremath{\mathsf{total}(#1)}}

%%% Universe types %%%
%\newcommand{\type}{\ensuremath{\mathsf{Type}}\xspace}
\newcommand{\UU}{\ensuremath{\mathcal{U}}\xspace}
\let\bbU\UU
\let\type\UU
% Universes of truncated types
\newcommand{\typele}[1]{\ensuremath{{#1}\text-\mathsf{Type}}\xspace}
\newcommand{\typeleU}[1]{\ensuremath{{#1}\text-\mathsf{Type}_\UU}\xspace}
\newcommand{\typelep}[1]{\ensuremath{{(#1)}\text-\mathsf{Type}}\xspace}
\newcommand{\typelepU}[1]{\ensuremath{{(#1)}\text-\mathsf{Type}_\UU}\xspace}
\let\ntype\typele
\let\ntypeU\typeleU
\let\ntypep\typelep
\let\ntypepU\typelepU
\renewcommand{\set}{\ensuremath{\mathsf{Set}}\xspace}
\newcommand{\setU}{\ensuremath{\mathsf{Set}_\UU}\xspace}
\newcommand{\prop}{\ensuremath{\mathsf{Prop}}\xspace}
\newcommand{\propU}{\ensuremath{\mathsf{Prop}_\UU}\xspace}
%Pointed types
\newcommand{\pointed}[1]{\ensuremath{#1_\bullet}}

%%% Ordinals and cardinals
\newcommand{\card}{\ensuremath{\mathsf{Card}}\xspace}
\newcommand{\ord}{\ensuremath{\mathsf{Ord}}\xspace}
\newcommand{\ordsl}[2]{{#1}_{/#2}}

%%% Univalence
\newcommand{\ua}{\ensuremath{\mathsf{ua}}\xspace} % the inverse of idtoeqv
\newcommand{\idtoeqv}{\ensuremath{\mathsf{idtoeqv}}\xspace}
\newcommand{\univalence}{\ensuremath{\mathsf{univalence}}\xspace} % the full axiom

%%% Truncation levels
\newcommand{\iscontr}{\ensuremath{\mathsf{isContr}}}
\newcommand{\contr}{\ensuremath{\mathsf{contr}}} % The path to the center of contraction
\newcommand{\isset}{\ensuremath{\mathsf{isSet}}}
\newcommand{\isprop}{\ensuremath{\mathsf{isProp}}}
% h-propositions
% \newcommand{\anhprop}{a mere proposition\xspace}
% \newcommand{\hprops}{mere propositions\xspace}

%%% Homotopy fibers %%%
%\newcommand{\hfiber}[2]{\ensuremath{\mathsf{hFiber}(#1,#2)}\xspace}
\let\hfiber\hfib

%%% Bracket/squash/truncation types %%%
% \newcommand{\brck}[1]{\textsf{mere}(#1)}
% \newcommand{\Brck}[1]{\textsf{mere}\Big(#1\Big)}
% \newcommand{\trunc}[2]{\tau_{#1}(#2)}
% \newcommand{\Trunc}[2]{\tau_{#1}\Big(#2\Big)}
% \newcommand{\truncf}[1]{\tau_{#1}}
%\newcommand{\trunc}[2]{\Vert #2\Vert_{#1}}
\newcommand{\trunc}[2]{\mathopen{}\left\Vert #2\right\Vert_{#1}\mathclose{}}
\newcommand{\ttrunc}[2]{\bigl\Vert #2\bigr\Vert_{#1}}
\newcommand{\Trunc}[2]{\Bigl\Vert #2\Bigr\Vert_{#1}}
\newcommand{\truncf}[1]{\Vert \blank \Vert_{#1}}
\newcommand{\tproj}[3][]{\mathopen{}\left|#3\right|_{#2}^{#1}\mathclose{}}
\newcommand{\tprojf}[2][]{|\blank|_{#2}^{#1}}
\def\pizero{\trunc0}
%\newcommand{\brck}[1]{\trunc{-1}{#1}}
%\newcommand{\Brck}[1]{\Trunc{-1}{#1}}
%\newcommand{\bproj}[1]{\tproj{-1}{#1}}
%\newcommand{\bprojf}{\tprojf{-1}}

\newcommand{\brck}[1]{\trunc{}{#1}}
\newcommand{\bbrck}[1]{\ttrunc{}{#1}}
\newcommand{\Brck}[1]{\Trunc{}{#1}}
\newcommand{\bproj}[1]{\tproj{}{#1}}
\newcommand{\bprojf}{\tprojf{}}

% Big parentheses
\newcommand{\Parens}[1]{\Bigl(#1\Bigr)}

% Projection and extension for truncations
\let\extendsmb\ext
\newcommand{\extend}[1]{\extendsmb(#1)}

%
%%% The empty type
\newcommand{\emptyt}{\ensuremath{\mathbf{0}}\xspace}

%%% The unit type
\newcommand{\unit}{\ensuremath{\mathbf{1}}\xspace}
\newcommand{\ttt}{\ensuremath{\star}\xspace}

%%% The two-element type
\newcommand{\bool}{\ensuremath{\mathbf{2}}\xspace}
\newcommand{\btrue}{{1_{\bool}}}
\newcommand{\bfalse}{{0_{\bool}}}

%%% Injections into binary sums and pushouts
\newcommand{\inlsym}{{\mathsf{inl}}}
\newcommand{\inrsym}{{\mathsf{inr}}}
\newcommand{\inl}{\ensuremath\inlsym\xspace}
\newcommand{\inr}{\ensuremath\inrsym\xspace}

%%% The segment of the interval
\newcommand{\seg}{\ensuremath{\mathsf{seg}}\xspace}

%%% Free groups
\newcommand{\freegroup}[1]{F(#1)}
\newcommand{\freegroupx}[1]{F'(#1)} % the "other" free group

%%% Glue of a pushout
\newcommand{\glue}{\mathsf{glue}}

%%% Colimits
\newcommand{\colim}{\mathsf{colim}}
\newcommand{\inc}{\mathsf{inc}}
\newcommand{\cmp}{\mathsf{cmp}}

%%% Circles and spheres
\newcommand{\Sn}{\mathbb{S}}
\newcommand{\base}{\ensuremath{\mathsf{base}}\xspace}
\newcommand{\lloop}{\ensuremath{\mathsf{loop}}\xspace}
\newcommand{\surf}{\ensuremath{\mathsf{surf}}\xspace}

%%% Suspension
\newcommand{\susp}{\Sigma}
\newcommand{\north}{\mathsf{N}}
\newcommand{\south}{\mathsf{S}}
\newcommand{\merid}{\mathsf{merid}}

%%% Blanks (shorthand for lambda abstractions)
\newcommand{\blank}{\mathord{\hspace{1pt}\text{--}\hspace{1pt}}}

%%% Nameless objects
\newcommand{\nameless}{\mathord{\hspace{1pt}\underline{\hspace{1ex}}\hspace{1pt}}}

%%% Some decorations
%\newcommand{\bbU}{\ensuremath{\mathbb{U}}\xspace}
% \newcommand{\bbB}{\ensuremath{\mathbb{B}}\xspace}
\newcommand{\bbP}{\ensuremath{\mathbb{P}}\xspace}

%%% Some categories
\newcommand{\uset}{\ensuremath{\mathcal{S}et}\xspace}
\newcommand{\ucat}{\ensuremath{{\mathcal{C}at}}\xspace}
\newcommand{\urel}{\ensuremath{\mathcal{R}el}\xspace}
\newcommand{\uhilb}{\ensuremath{\mathcal{H}ilb}\xspace}
\newcommand{\utype}{\ensuremath{\mathcal{T}\!ype}\xspace}

% Pullback corner
\newbox\pbbox
\setbox\pbbox=\hbox{\xy \POS(65,0)\ar@{-} (0,0) \ar@{-} (65,65)\endxy}
\def\pb{\save[]+<3.5mm,-3.5mm>*{\copy\pbbox} \restore}

% Macros for the categories chapter
\newcommand{\inv}[1]{{#1}^{-1}}
\newcommand{\idtoiso}{\ensuremath{\mathsf{idtoiso}}\xspace}
\newcommand{\isotoid}{\ensuremath{\mathsf{isotoid}}\xspace}
\newcommand{\op}{^{\mathrm{op}}}
\newcommand{\y}{\ensuremath{\mathbf{y}}\xspace}
\newcommand{\dgr}[1]{{#1}^{\dagger}}
\newcommand{\unitaryiso}{\mathrel{\cong^\dagger}}
\newcommand{\cteqv}[2]{\ensuremath{#1 \simeq #2}\xspace}
\newcommand{\cteqvsym}{\simeq}     % Symbol for equivalence of categories

%%% Natural numbers
\newcommand{\N}{\ensuremath{\mathbb{N}}\xspace}
%\newcommand{\N}{\textbf{N}}
\let\nat\N
\newcommand{\natp}{\ensuremath{\nat'}\xspace} % alternative nat in induction chapter

\newcommand{\zerop}{\ensuremath{0'}\xspace}   % alternative zero in induction chapter
\newcommand{\suc}{\mathsf{succ}}
\newcommand{\sucp}{\ensuremath{\suc'}\xspace} % alternative suc in induction chapter
\newcommand{\add}{\mathsf{add}}
\newcommand{\ack}{\mathsf{ack}}
\newcommand{\ite}{\mathsf{iter}}
\newcommand{\assoc}{\mathsf{assoc}}
\newcommand{\dbl}{\ensuremath{\mathsf{double}}}
\newcommand{\dblp}{\ensuremath{\dbl'}\xspace} % alternative double in induction chapter


%%% Lists
\newcommand{\lst}[1]{\mathsf{List}(#1)}
\newcommand{\nil}{\mathsf{nil}}
\newcommand{\cons}{\mathsf{cons}}
\newcommand{\lost}[1]{\mathsf{Lost}(#1)}

%%% Vectors of given length, used in induction chapter
\newcommand{\vect}[2]{\ensuremath{\mathsf{Vec}_{#1}(#2)}\xspace}

%%% Integers
\newcommand{\Z}{\ensuremath{\mathbb{Z}}\xspace}
\newcommand{\Zsuc}{\mathsf{succ}}
\newcommand{\Zpred}{\mathsf{pred}}

%%% Rationals
\newcommand{\Q}{\ensuremath{\mathbb{Q}}\xspace}

%%% Function extensionality
\newcommand{\funext}{\mathsf{funext}}
\newcommand{\happly}{\mathsf{happly}}

%%% A naturality lemma
\newcommand{\com}[3]{\mathsf{swap}_{#1,#2}(#3)}

%%% Code/encode/decode
\newcommand{\code}{\ensuremath{\mathsf{code}}\xspace}
\newcommand{\encode}{\ensuremath{\mathsf{encode}}\xspace}
\newcommand{\decode}{\ensuremath{\mathsf{decode}}\xspace}

% Function definition with domain and codomain
\newcommand{\function}[4]{\left\{\begin{array}{rcl}#1 &
      \longrightarrow & #2 \\ #3 & \longmapsto & #4 \end{array}\right.}

%%% Cones and cocones
\newcommand{\cone}[2]{\mathsf{cone}_{#1}(#2)}
\newcommand{\cocone}[2]{\mathsf{cocone}_{#1}(#2)}
% Apply a function to a cocone
\newcommand{\composecocone}[2]{#1\circ#2}
\newcommand{\composecone}[2]{#2\circ#1}
%%% Diagrams
\newcommand{\Ddiag}{\mathscr{D}}

%%% (pointed) mapping spaces
\newcommand{\Map}{\mathsf{Map}}

%%% The interval
\newcommand{\interval}{\ensuremath{I}\xspace}
\newcommand{\izero}{\ensuremath{0_{\interval}}\xspace}
\newcommand{\ione}{\ensuremath{1_{\interval}}\xspace}

%%% Arrows
\newcommand{\epi}{\ensuremath{\twoheadrightarrow}}
\newcommand{\mono}{\ensuremath{\rightarrowtail}}

%%% Sets
\newcommand{\bin}{\ensuremath{\mathrel{\widetilde{\in}}}}

%%% Semigroup structure
\newcommand{\semigroupstrsym}{\ensuremath{\mathsf{SemigroupStr}}}
\newcommand{\semigroupstr}[1]{\ensuremath{\mathsf{SemigroupStr}}(#1)}
\newcommand{\semigroup}[0]{\ensuremath{\mathsf{Semigroup}}}

%%% Macros for the formal type theory
\newcommand{\emptyctx}{\ensuremath{\cdot}}
\newcommand{\production}{\vcentcolon\vcentcolon=}
\newcommand{\conv}{\downarrow}
\newcommand{\ctx}{\ensuremath{\mathsf{ctx}}}
\newcommand{\wfctx}[1]{#1\ \ctx}
\newcommand{\oftp}[3]{#1 \vdash #2 : #3}
\newcommand{\jdeqtp}[4]{#1 \vdash #2 \jdeq #3 : #4}
\newcommand{\judg}[2]{#1 \vdash #2}
\newcommand{\tmtp}[2]{#1 \mathord{:} #2}

% rule names
\newcommand{\rform}{\textsc{form}}
\newcommand{\rintro}{\textsc{intro}}
\newcommand{\relim}{\textsc{elim}}
\newcommand{\rcomp}{\textsc{comp}}
\newcommand{\runiq}{\textsc{uniq}}
\newcommand{\Weak}{\mathsf{Wkg}}
\newcommand{\Vble}{\mathsf{Vble}}
\newcommand{\Exch}{\mathsf{Exch}}
\newcommand{\Subst}{\mathsf{Subst}}

%%% Macros for HITs
\newcommand{\cc}{\mathsf{c}}
\newcommand{\pp}{\mathsf{p}}
\newcommand{\cct}{\widetilde{\mathsf{c}}}
\newcommand{\ppt}{\widetilde{\mathsf{p}}}
\newcommand{\Wtil}{\ensuremath{\widetilde{W}}\xspace}

%%% Macros for n-types
\newcommand{\istype}[1]{\mathsf{is}\mbox{-}{#1}\mbox{-}\mathsf{type}}
\newcommand{\nplusone}{\ensuremath{(n+1)}}
\newcommand{\nminusone}{\ensuremath{(n-1)}}
\newcommand{\fact}{\mathsf{fact}}

%%% Macros for homotopy
\newcommand{\kbar}{\overline{k}} % Used in van Kampen's theorem

%%% Macros for induction
\newcommand{\natw}{\ensuremath{\mathbf{N^w}}\xspace}
\newcommand{\zerow}{\ensuremath{0^\mathbf{w}}\xspace}
\newcommand{\sucw}{\ensuremath{\mathsf{succ}^{\mathbf{w}}}\xspace}
\newcommand{\nalg}{\nat\mathsf{Alg}}
\newcommand{\nhom}{\nat\mathsf{Hom}}
\newcommand{\ishinitw}{\mathsf{isHinit}_{\mathsf{W}}}
\newcommand{\ishinitn}{\mathsf{isHinit}_\nat}
\newcommand{\w}{\mathsf{W}}
\newcommand{\walg}{\w\mathsf{Alg}}
\newcommand{\whom}{\w\mathsf{Hom}}

%%% Macros for real numbers
\newcommand{\RC}{\ensuremath{\mathbb{R}_\mathsf{c}}\xspace} % Cauchy
\newcommand{\RD}{\ensuremath{\mathbb{R}_\mathsf{d}}\xspace} % Dedekind
\newcommand{\R}{\ensuremath{\mathbb{R}}\xspace}           % Either
\newcommand{\barRD}{\ensuremath{\bar{\mathbb{R}}_\mathsf{d}}\xspace} % Dedekind completion of Dedekind

\newcommand{\close}[1]{\sim_{#1}} % Relation of closeness
\newcommand{\closesym}{\mathord\sim}
\newcommand{\rclim}{\mathsf{lim}} % HIT constructor for Cauchy reals
\newcommand{\rcrat}{\mathsf{rat}} % Embedding of rationals into Cauchy reals
\newcommand{\rceq}{\mathsf{eq}_{\RC}} % HIT path constructor
\newcommand{\CAP}{\mathcal{C}}    % The type of Cauchy approximations
\newcommand{\Qp}{\Q_{+}}
\newcommand{\apart}{\mathrel{\#}}  % apartness
\newcommand{\dcut}{\mathsf{isCut}}  % Dedekind cut
\newcommand{\cover}{\triangleleft} % inductive cover
\newcommand{\intfam}[3]{(#2, \lam{#1} #3)} % family of rational intervals

% Macros for the Cauchy reals construction
\newcommand{\bsim}{\frown}
\newcommand{\bbsim}{\smile}

\newcommand{\hapx}{\diamondsuit\approx}
\newcommand{\hapname}{\diamondsuit}
\newcommand{\hapxb}{\heartsuit\approx}
\newcommand{\hapbname}{\heartsuit}
\newcommand{\tap}[1]{\bullet\approx_{#1}\triangle}
\newcommand{\tapname}{\triangle}
\newcommand{\tapb}[1]{\bullet\approx_{#1}\square}
\newcommand{\tapbname}{\square}

%%% Macros for surreals
\newcommand{\NO}{\ensuremath{\mathsf{No}}\xspace}
\newcommand{\surr}[2]{\{\,#1\,\big|\,#2\,\}}
\newcommand{\LL}{\mathcal{L}}
\newcommand{\RR}{\mathcal{R}}
\newcommand{\noeq}{\mathsf{eq}_{\NO}} % HIT path constructor

\newcommand{\ble}{\trianglelefteqslant}
\newcommand{\blt}{\vartriangleleft}
\newcommand{\bble}{\sqsubseteq}
\newcommand{\bblt}{\sqsubset}

\newcommand{\hle}{\diamondsuit\preceq}
\newcommand{\hlt}{\diamondsuit\prec}
\newcommand{\hlname}{\diamondsuit}
\newcommand{\hleb}{\heartsuit\preceq}
\newcommand{\hltb}{\heartsuit\prec}
\newcommand{\hlbname}{\heartsuit}
% \newcommand{\tle}{(\bullet\preceq\triangle)}
% \newcommand{\tlt}{(\bullet\prec\triangle)}
\newcommand{\tle}{\triangle\preceq}
\newcommand{\tlt}{\triangle\prec}
\newcommand{\tlname}{\triangle}
% \newcommand{\tleb}{(\bullet\preceq\square)}
% \newcommand{\tltb}{(\bullet\prec\square)}
\newcommand{\tleb}{\square\preceq}
\newcommand{\tltb}{\square\prec}
\newcommand{\tlbname}{\square}

%%% Macros for set theory
\newcommand{\vset}{\mathsf{set}}  % point constructor for cummulative hierarchy V
\def\cd{\tproj0}
\newcommand{\inj}{\ensuremath{\mathsf{inj}}} % type of injections
\newcommand{\acc}{\ensuremath{\mathsf{acc}}} % accessibility

\newcommand{\atMostOne}{\mathsf{atMostOne}}

\newcommand{\power}[1]{\mathcal{P}(#1)} % power set
\newcommand{\powerp}[1]{\mathcal{P}_+(#1)} % inhabited power set

%%%% THEOREM ENVIRONMENTS %%%%

% The cleveref package provides \cref{...} which is like \ref{...}
% except that it automatically inserts the type of the thing you're
% referring to, e.g. it produces "Theorem 3.8" instead of just "3.8"
% (and hyperref makes the whole thing a hyperlink).  This saves a slight amount
% of typing, but more importantly it means that if you decide later on
% that 3.8 should be a Lemma or a Definition instead of a Theorem, you
% don't have to change the name in all the places you referred to it.

% The following hack improves on this by using the same counter for
% all theorem-type environments, so that after Theorem 1.1 comes
% Corollary 1.2 rather than Corollary 1.1.  This makes it much easier
% for the reader to find a particular theorem when flipping through
% the document.
\makeatletter
\def\defthm#1#2#3{%
  %% Ensure all theorem types are numbered with the same counter
  \newaliascnt{#1}{thm}
  \newtheorem{#1}[#1]{#2}
  \aliascntresetthe{#1}
  %% This command tells cleveref's \cref what to call things
  \crefname{#1}{#2}{#3}% following brace must be on separate line to support poorman cleveref sed file
}

% Now define a bunch of theorem-type environments.
\newtheorem{thm}{Theorem}[section]
\crefname{thm}{Theorem}{Theorems}
%\defthm{prop}{Proposition}   % Probably we shouldn't use "Proposition" in this way
\defthm{cor}{Corollary}{Corollaries}
\defthm{lem}{Lemma}{Lemmas}
\defthm{axiom}{Axiom}{Axioms}
% Since definitions and theorems in type theory are synonymous, should
% we actually use the same theoremstyle for them?
\theoremstyle{definition}
\defthm{defn}{Definition}{Definitions}
\theoremstyle{remark}
\defthm{rmk}{Remark}{Remarks}
\defthm{eg}{Example}{Examples}
\defthm{egs}{Examples}{Examples}
\defthm{notes}{Notes}{Notes}
% Number exercises within chapters, with their own counter.
\newtheorem{ex}{Exercise}[chapter]
\crefname{ex}{Exercise}{Exercises}

% Display format for sections
\crefformat{section}{\S#2#1#3}
\Crefformat{section}{Section~#2#1#3}
\crefrangeformat{section}{\S\S#3#1#4--#5#2#6}
\Crefrangeformat{section}{Sections~#3#1#4--#5#2#6}
\crefmultiformat{section}{\S\S#2#1#3}{ and~#2#1#3}{, #2#1#3}{ and~#2#1#3}
\Crefmultiformat{section}{Sections~#2#1#3}{ and~#2#1#3}{, #2#1#3}{ and~#2#1#3}
\crefrangemultiformat{section}{\S\S#3#1#4--#5#2#6}{ and~#3#1#4--#5#2#6}{, #3#1#4--#5#2#6}{ and~#3#1#4--#5#2#6}
\Crefrangemultiformat{section}{Sections~#3#1#4--#5#2#6}{ and~#3#1#4--#5#2#6}{, #3#1#4--#5#2#6}{ and~#3#1#4--#5#2#6}

% Display format for appendices
\crefformat{appendix}{Appendix~#2#1#3}
\Crefformat{appendix}{Appendix~#2#1#3}
\crefrangeformat{appendix}{Appendices~#3#1#4--#5#2#6}
\Crefrangeformat{appendix}{Appendices~#3#1#4--#5#2#6}
\crefmultiformat{appendix}{Appendices~#2#1#3}{ and~#2#1#3}{, #2#1#3}{ and~#2#1#3}
\Crefmultiformat{appendix}{Appendices~#2#1#3}{ and~#2#1#3}{, #2#1#3}{ and~#2#1#3}
\crefrangemultiformat{appendix}{Appendices~#3#1#4--#5#2#6}{ and~#3#1#4--#5#2#6}{, #3#1#4--#5#2#6}{ and~#3#1#4--#5#2#6}
\Crefrangemultiformat{appendix}{Appendices~#3#1#4--#5#2#6}{ and~#3#1#4--#5#2#6}{, #3#1#4--#5#2#6}{ and~#3#1#4--#5#2#6}

\crefname{part}{Part}{Parts}

% Number subsubsections
\setcounter{secnumdepth}{5}

% Display format for figures
\crefname{figure}{Figure}{Figures}

%%%% EQUATION NUMBERING %%%%

% The following hack uses the single theorem counter to number
% equations as well, so that we don't have both Theorem 1.1 and
% equation (1.1).
\let\c@equation\c@thm
\numberwithin{equation}{section}


%%%% ENUMERATE NUMBERING %%%%

% Number the first level of enumerates as (i), (ii), ...
\renewcommand{\theenumi}{(\roman{enumi})}
\renewcommand{\labelenumi}{\theenumi}


%%%% MARGINS %%%%

% This is a matter of personal preference, but I think the left
% margins on enumerates and itemizes are too wide.
\setitemize[1]{leftmargin=2em}
\setenumerate[1]{leftmargin=*}

% Likewise that they are too spaced out.
\setitemize[1]{itemsep=-0.2em}
\setenumerate[1]{itemsep=-0.2em}

%%% Notes %%%
\def\noteson{%
\gdef\note##1{\mbox{}\marginpar{\color{blue}\textasteriskcentered\ ##1}}}
\gdef\notesoff{\gdef\note##1{\null}}
\noteson

\newcommand{\Coq}{\textsc{Coq}\xspace}
\newcommand{\Agda}{\textsc{Agda}\xspace}
\newcommand{\NuPRL}{\textsc{NuPRL}\xspace}

%%%% CITATIONS %%%%

% \let \cite \citep

%%%% INDEX %%%%

\newcommand{\footstyle}[1]{{\hyperpage{#1}}n} % If you index something that is in a footnote
\newcommand{\defstyle}[1]{\textbf{\hyperpage{#1}}}  % Style for pageref to a definition

\newcommand{\indexdef}[1]{\index{#1|defstyle}}   % Index a definition
\newcommand{\indexfoot}[1]{\index{#1|footstyle}} % Index a term in a footnote
\newcommand{\indexsee}[2]{\index{#1|see{#2}}}    % Index "see also"


%%%% Standard phrasing or spelling of common phrases %%%%

\newcommand{\ZF}{Zermelo--Fraenkel}
\newcommand{\CZF}{Constructive \ZF{} Set Theory}

\newcommand{\LEM}[1]{\ensuremath{\mathsf{LEM}_{#1}}\xspace}
\newcommand{\choice}[1]{\ensuremath{\mathsf{AC}_{#1}}\xspace}

%%%% MISC %%%%

\newcommand{\mentalpause}{\medskip} % Use for "mental" pause, instead of \smallskip or \medskip

%% Use \symlabel instead of \label to mark a pageref that you need in the index of symbols
\newcounter{symindex}
\newcommand{\symlabel}[1]{\refstepcounter{symindex}\label{#1}}

% Local Variables:
% mode: latex
% TeX-master: "hott-online"
% End:

% Ubiquitous set names
%
\newcommand{\Naturals}{\mathbb{N}} % Set of natural numbers
\newcommand{\Integers}{\mathbb{Z}} % Set of interer numbers
\newcommand{\Complex}{\mathbb{C}}  % Set of complex numbers

% Basic Definitions
%
\newcommand{\Cp}{\fatsemi} % Morphism composition in diagrammatic order

\DeclareMathOperator{\Obj}{Obj}%[1]{\operatorname{Obj} \, #1} % Set of objects of category #1
\DeclareMathOperator{\Mor}{Mor}%[1]{\operatorname{Mor} \, #1} % Set of objects of category #1
\DeclareMathOperator{\GObj}{GObj}%[1]{\operatorname{GenObj} \, #1} % Set of objects of category #1
\DeclareMathOperator{\GMor}{GMor}%[1]{\operatorname{GenMor} \, #1} % Set of objects of category #1

\newcommand{\Homtotal}[1]{\operatorname{Hom}_{\,#1}} % Set of morphisms of category #1
\newcommand{\Hom}[3]{\operatorname{Hom}_{\,#1}\left[#2,#3\right]} % Set of morphisms of category #1 from object #2 to object #3
\newcommand{\Source}[1]{\operatorname{s}(#1)} % Domain of function/morphism #1
\newcommand{\Target}[1]{\operatorname{t}(#1)} % Domain of function/morphism #1
\newcommand{\Id}[1]{\text{id}_{#1}} % Identity morphism of object #1
\newcommand{\Op}[1]{{#1}^{\text{op}}} % op functor

% Generic names for categories
%
\newcommand{\CategoryA}{\mathit{A}}
\newcommand{\CategoryB}{\mathit{B}}
\newcommand{\CategoryC}{\mathit{C}}
\newcommand{\CategoryD}{\mathit{D}}
\newcommand{\CategoryE}{\mathit{E}}

\newcommand{\WTerm}{\mathds{1}}
\newcommand{\Term}{\mathbf{1}}

% Category names    
%
\newcommand{\Set}{\mathbf{Set}} % Category of sets and functions
\newcommand{\SetS}{{\Set_{*}}} % Category of sets and functions
\newcommand{\Hask}{\mathbf{Hask}} % Category of data types and Haskell functions
\newcommand{\Group}{\mathbf{Group}} % Category of groups and homomorphisms
\newcommand{\Top}{\mathbf{Top}} % Category of topological spaces and continuous functions
\newcommand{\Rel}{\mathbf{Rel}} % Category of sets and relations
\newcommand{\Span}{\mathbf{Span}} % Category of sets and spans
\newcommand{\hTop}{\mathbf{hTop}} % Category of topological spaces and homotopy classes of continuous functions
\newcommand{\Cat}{\mathbf{Cat}} % Category of categories

% Monoidal Categories
%
\newcommand{\Tensor}{\otimes} % Monoidal tensor
\newcommand{\TensorUnit}{I} % Monoidal tensor unit


% Logic
%
\newcommand{\Suchthat}[2]{\left\{#1 \: \middle\vert \: #2\right\}} % Set of elements #1 such that condition #2 holds 

\setcounter{tocdepth}{2}
%
%
\begin{document}
    \title{Obstructions to Compositionality}

    \author{        
        Caterina Puca
            %\email{OrcID here}
            \institute{Quantinuum\textsuperscript{*}}
            \email{caterina.puca@quantinuum.com}
        \and
        Amar Hadzihasanovic
            %\email{OrcID here}
            \institute{$^1$ Quantinuum\textsuperscript{*} \\$^2$ Tallinn University of Technology}%\footref{note: Quantinuum Address}}
            \email{amar.hadzihasanovic@quantinuum.com}
        \and
        Fabrizio Genovese
            %\email{OrcID here}
            \institute{20squares}
            \email{noreply@20squares.xyz}
        \and
        Bob Coecke
            %\email{OrcID here}
            \institute{Quantinuum\footnote{17 Beaumont Street, Oxford OX1 2NA, United Kingdom}}
            \email{bob.coecke@quantinuum.com}
    }

    \def\titlerunning{Obstructions to Compositionality}
    \def\authorrunning{Puca, Hadzihasanovic, Genovese, Coecke}

    \maketitle
    \begin{abstract}

Compositionality is at the heart of computer science and several other areas of applied category theory such as computational linguistics, categorical quantum mechanics, interpretable AI, dynamical systems, compositional game theory, and Petri nets. 
However, the meaning of the term seems to vary across the many different applications.
This work contributes to understanding, and in particular qualifying, different kinds of compositionality.

Formally, we introduce invariants of categories that we call zeroth and first homotopy posets, generalising in a precise sense the $\pi_0$ and $\pi_1$ of a groupoid.
These posets can be used to obtain a qualitative description of how far an object is from being terminal and a morphism is from being iso.
In the context of applied category theory, this formal machinery gives us a way to qualitatively describe the ``failures of compositionality'', seen as failures of certain (op)lax functors to be strong, by classifying obstructions to the (op)laxators being isomorphisms. 

Failure of compositionality, for example for the interpretation of a categorical syntax in a semantic universe, can both be a bad thing and a good thing, which we illustrate by respective examples in graph theory and quantum theory.

    \end{abstract}
    %
    \paragraph*{\bf Acknowledgements}
    %
        A.H.\ was supported by the ESF funded Estonian IT Academy research measure (project 2014-2020.4.05.19-0001) and by the Estonian Research Council grant PSG764.
        We thank Sean Tull and Robin Lorenz for helpful comments on an earlier draft.

    \chapter*{Introduction}
\markboth{\textsc{Introduction}}{}
\addcontentsline{toc}{chapter}{Introduction}
\setcounter{page}{1}
\pagenumbering{arabic}


\emph{Homotopy type theory} is a new branch of mathematics that combines aspects of several different fields in a surprising way. It is based on a recently discovered connection between \emph{homotopy theory} and \emph{type theory}.
Homotopy theory is an outgrowth of algebraic topology and homological algebra, with relationships to higher category theory; while type theory is a branch of mathematical logic and theoretical computer science.
Although the connections between the two are currently the focus of intense investigation, it is increasingly clear that they are just the beginning of a subject that will take more time and more hard work to fully understand.
It touches on topics as seemingly distant as the homotopy groups of spheres, the algorithms for type checking, and the definition of weak $\infty$-groupoids.

Homotopy type theory also brings new ideas into the very foundation of mathematics.
\index{foundations, univalent}%
On the one hand, there is Voevodsky's subtle and beautiful \emph{univalence axiom}. 
\index{univalence axiom}%
The univalence axiom implies, in particular, that isomorphic structures can be identified, a principle that mathematicians have been happily using on workdays, despite its incompatibility with the ``official'' doctrines of conventional foundations.
On the other hand, we have \emph{higher inductive types}, which provide direct, logical descriptions of some of the basic spaces and constructions of homotopy theory: spheres, cylinders, truncations, localizations, etc.
Both ideas are impossible to capture directly in classical set-theoretic foundations, but when combined in homotopy type theory, they permit an entirely new kind of ``logic of homotopy types''.
\index{foundations}%

This suggests a new conception of foundations of mathematics, with intrinsic homotopical content, an ``invariant'' conception of the objects of mathematics --- and convenient machine implementations, which can serve as a practical aid to the working mathematician.
This is the \emph{Univalent Foundations} program.
The present book is intended as a first systematic exposition of the basics of univalent foundations, and a collection of examples of this new style of reasoning --- but without requiring the reader to know or learn any formal logic, or to use any computer proof assistant.

% This enlarges the page by one line in letter format. Use sparringly.
\OPTwidow

We emphasize that homotopy type theory is a young field, and univalent foundations is very much a work in progress. 
This book should be regarded as a ``snapshot'' of just one portion of the field, taken at the time it was written, rather than a polished exposition of a completed edifice. 
As we will discuss briefly later, there are many aspects of homotopy type theory that are not yet fully understood --- and some that are not even touched upon here. 
The ultimate theory will almost certainly not look exactly like the one described in this book, but it will surely be \emph{at least} as capable and powerful; we therefore believe that univalent foundations will eventually become a viable alternative to set theory as the ``implicit foundation'' for the unformalized mathematics done by most mathematicians.

\subsection*{Type theory}

Type theory was originally invented by Bertrand Russell \cite{Russell:1908},\index{Russell, Bertrand} as a device for blocking the paradoxes in the logical foundations of mathematics  that were under investigation at the time.
It was developed further by many people over the next few decades, particularly Church~\cite{Church:1940tu,Church:1941tc} who combined it with his \textit{$\lambda$-calculus}.
Although it is not generally regarded as the foundation for classical mathematics, set theory being more customary, type theory still has numerous applications, especially in computer science and the theory of programming languages~\cite{Pierce-TAPL}.
\index{programming}%
\index{type theory}%
\index{lambda-calculus@$\lambda$-calculus}%
Per Martin-L\"{o}f \cite{Martin-Lof-1972,Martin-Lof-1973,Martin-Lof-1979,martin-lof:bibliopolis}, among others,
developed a ``predicative'' modification of Church's type system, which is now usually called dependent, constructive, intuitionistic, or simply \emph{Martin\--L\"of type theory}. This is the basis of the system that we consider here; it was originally intended as a rigorous framework for the formalization of constructive mathematics.  In what follows, we will often use ``type theory'' to refer specifically to this system and similar ones, although type theory as a subject is much broader (see \cite{somma,kamar} for the history of type theory).

In type theory, unlike set theory, objects are classified using a primitive notion of \emph{type}, similar to the data-types used in programming languages.  These elaborately structured types can be used to express detailed specifications of the objects classified, giving rise to principles of reasoning about these objects.  To take a very simple example, the objects of a product type $A\times B$ are known to be of the form $\pairr{a,b}$, and so one automatically knows how to construct them and how to decompose them. Similarly, an object of function type $A\to B$ can be acquired from an object of type $B$ parametrized by objects of type $A$, and can be evaluated at an argument of type $A$.  This rigidly predictable behavior of all objects (as opposed to set theory's more liberal formation principles, allowing inhomogeneous sets) is one aspect of type theory that has led to its extensive use in verifying the correctness of computer programs.  The clear reasoning principles associated with the construction of types also form the basis of modern \emph{computer proof assistants},%
\index{proof!assistant}%
\indexsee{computer proof assistant}{proof assistant}
\index{mathematics!formalized}%
which are used for formalizing mathematics and verifying the correctness of formalized proofs.  We return to this aspect of type theory below.  

One problem in understanding type theory from a mathematical point of view, however, has always been that the basic concept of \emph{type} is unlike that of \emph{set} in ways that have been hard to make precise.  We believe that the new idea of regarding types, not as strange sets (perhaps constructed without using classical logic), but as spaces, viewed from the perspective of homotopy theory, is a significant step forward.  In particular, it solves the problem of understanding how the notion of equality of elements of a type differs from that of elements of a set.

In homotopy theory one is concerned with spaces
\index{topological!space}%
and continuous mappings between them, 
\index{function!continuous!in classical homotopy theory}%
up to homotopy.  A \emph{homotopy}
\index{homotopy!topological}%
between a pair of continuous maps $f : X \to Y$
and  $g : X\to Y$ is 
a continuous map $H : X \times [0, 1] \to Y$ satisfying
$H(x, 0) = f (x)$  and $H(x, 1) = g(x)$. The homotopy $H$ may be thought of as a ``continuous deformation'' of $f$ into $g$. The spaces $X$ and $Y$ are said to be \emph{homotopy equivalent},
\index{homotopy!equivalence!topological}%
$\eqv X Y$, if there are continuous maps going back and forth, the composites of which are homotopical to the respective identity mappings, i.e., if they are isomorphic ``up to homotopy''.  Homotopy equivalent spaces have the same algebraic invariants (e.g., homology, or the fundamental group), and are said to have the same \emph{homotopy type}.

\subsection*{Homotopy type theory}

Homotopy type theory (HoTT) interprets type theory from a homotopical perspective.
In homotopy type theory, we regard the types as ``spaces'' (as studied in homotopy theory) or higher groupoids, and the logical constructions (such as the product $A\times B$) as homotopy-invariant constructions on these spaces.
In this way, we are able to manipulate spaces directly without first having to develop point-set topology (or any combinatorial replacement for it, such as the theory of simplicial sets).
To briefly explain this perspective, consider first the basic concept of type theory, namely that
the \emph{term} $a$ is of \emph{type} $A$, which is written:
\[ a:A. \]
This expression is traditionally thought of as akin to:
\begin{center}
``$a$ is an element of the set $A$''.
\end{center}
However, in homotopy type theory we think of it instead as:
\begin{center}
``$a$ is a point of the space $A$''.
\end{center}
\index{continuity of functions in type theory@``continuity'' of functions in type theory}%
Similarly, every function $f : A\to B$ in type theory is regarded as a continuous map from the space $A$ to the space $B$.

We should stress that these ``spaces'' are treated purely homotopically, not topologically.
For instance, there is no notion of ``open subset'' of a type or of ``convergence'' of a sequence of elements of a type.
We only have ``homotopical'' notions, such as paths between points and homotopies between paths, which also make sense in other models of homotopy theory (such as simplicial sets).
Thus, it would be more accurate to say that we treat types as \emph{$\infty$-groupoids}\index{.infinity-groupoid@$\infty$-groupoid}; this is a name for the ``invariant objects'' of homotopy theory which can be presented by topological spaces,
\index{topological!space}%
simplicial sets, or any other model for homotopy theory.
However, it is convenient to sometimes use topological words such as ``space'' and ``path'', as long as we remember that other topological concepts are not applicable.

(It is tempting to also use the phrase \emph{homotopy type}
\index{homotopy!type}%
for these objects, suggesting the dual interpretation of ``a type (as in type theory) viewed homotopically'' and ``a space considered from the point of view of homotopy theory''.
The latter is a bit different from the classical meaning of ``homotopy type'' as an \emph{equivalence class} of spaces modulo homotopy equivalence, although it does preserve the meaning of phrases such as ``these two spaces have the same homotopy type''.)

The idea of interpreting types as structured objects, rather than sets, has a long pedigree, and is known to clarify various mysterious aspects of type theory.
For instance, interpreting types as sheaves helps explain the intuitionistic nature of type-theoretic logic, while interpreting them as partial equivalence relations or ``domains'' helps explain its computational aspects.
It also implies that we can use type-theoretic reasoning to study the structured objects, leading to the rich field of categorical logic.
The homotopical interpretation fits this same pattern: it clarifies the nature of \emph{identity} (or equality) in type theory, and allows us to use type-theoretic reasoning in the study of homotopy theory.

The key new idea of the homotopy interpretation is that the logical notion of identity $a = b$ of two objects $a, b: A$ of the same type $A$ can be understood as the existence of a path $p : a \leadsto b$ from point $a$ to point $b$ in the space $A$.
This also means that two functions $f, g: A\to B$ can be identified if they are homotopic, since a homotopy is just a (continuous) family of paths $p_x: f(x) \leadsto g(x)$ in $B$, one for each $x:A$.
In type theory, for every type $A$ there is a (formerly somewhat mysterious) type $\idtypevar{A}$ of identifications of two objects of $A$; in homotopy type theory, this is just the \emph{path space} $A^I$ of all continuous maps $I\to A$ from the unit interval.
\index{unit!interval}%
\index{interval!topological unit}%
\index{path!topological}%
\index{topological!path}%
In this way, a term $p : \idtype[A]{a}{b}$ represents a path $p : a \leadsto b$ in $A$. 

The idea of homotopy type theory arose around 2006 in independent work by Awodey and Warren~\cite{AW} and Voevodsky~\cite{VV}, but it was inspired by 
Hofmann and Streicher's earlier groupoid interpretation~\cite{hs:gpd-typethy}.
Indeed, higher-dimensional category theory (particularly the theory of weak $\infty$-groupoids) is now known to be intimately connected to homotopy theory, as proposed by Grothendieck and now being studied intensely by mathematicians of both sorts.
The original semantic models of Awodey--Warren and Voevodsky use well-known notions and techniques from homotopy theory which are now also in use in higher category theory, such as  Quillen model categories and Kan\index{Kan complex} simplicial sets\index{simplicial!sets}.
\index{Quillen model category}%
\index{model category}%

In particular, Voevodsky constructed an interpretation of type theory in Kan simplicial sets, and recognized that this interpretation satisfied a further crucial property which he dubbed \emph{univalence}.
This had not previously been considered in type theory (although Church's principle of extensionality for propositions turns out to be a very special case of it, and Hofmann and Streicher had considered another special case under the name ``universe extensionality'').
Adding univalence to type theory in the form of a new axiom has far-reaching consequences, many of which are natural, simplifying and compelling.
The univalence axiom also further strengthens the homotopical view of type theory, since it holds in the simplicial model and other related models, while failing under the view of types as sets.

\subsection*{Univalent foundations}

Very briefly, the basic idea of the univalence axiom can be explained as follows.
In type theory, one can have a universe $\UU$, the terms of which are themselves types, $A : \UU$, etc.
Those types that are terms of $\UU$ are commonly called \emph{small} types.
\index{type!small}%
\index{small!type}%
Like any type, $\UU$ has an identity type $\idtypevar{\UU}$, which expresses the identity relation $A = B$ between small types.
Thinking of types as spaces, $\UU$ is a space, the points of which are spaces; to understand its identity type, we must ask, what is a path $p : A \leadsto B$ between spaces in $\UU$?
The univalence axiom says that such paths correspond to homotopy equivalences $\eqv A B$, (roughly) as explained above.
A bit more precisely, given any (small) types $A$ and $B$, in addition to the primitive type $\idtype[\UU]AB$ of identifications of $A$ with $B$, there is the defined type $\texteqv AB$ of equivalences from $A$ to $B$.
Since the identity map on any object is an equivalence, there is a canonical map,
\[\idtype[\UU]AB\to\texteqv AB.\]
The univalence axiom states that this map is itself an equivalence.
At the risk of oversimplifying, we can state this succinctly as follows:

\begin{description}\index{univalence axiom}%
\item[Univalence Axiom:]  $\eqvspaced{(A = B)}{(\eqv A B)}$.
\end{description}
%
In other words, identity is equivalent to equivalence. \index{identity}% 
In particular, one may say that ``equivalent types are identical''.
However, this phrase is somewhat misleading, since it may sound like a sort of ``skeletality'' condition which \emph{collapses} the notion of equivalence to coincide with identity, whereas in fact univalence is about \emph{expanding} the notion of identity so as to coincide with the (unchanged) notion of equivalence.

From the homotopical point of view, univalence implies that spaces of the same homotopy type are connected by a path in the universe $\UU$, in accord with the intuition of a classifying space for (small) spaces.
From the logical point of view, however, it is a radically new idea: it says that isomorphic things can be identified!  Mathematicians are of course used to identifying isomorphic structures in practice, but they generally do so by ``abuse of notation''\index{abuse!of notation}, or some other informal device, knowing that the objects involved are not ``really'' identical.  But in this new foundational scheme, such structures can be formally identified, in the logical sense that every property or construction involving one also applies to the other. Indeed, the identification is now made explicit, and properties and constructions can be systematically transported along it.  Moreover, the different ways in which such identifications may be made themselves form a structure that one can (and should!)\ take into account.

Thus in sum, for points $A$ and $B$ of the universe $\UU$ (i.e., small types), the univalence axiom identifies the following three notions:
\begin{itemize}
\item (logical) an identification $p:A=B$ of $A$ and $B$
\item (topological) a path $p:A \leadsto B$ from $A$ to $B$ in $\UU$
\item (homotopical) an equivalence $p:\eqv A B$ between $A$ and $B$.
\end{itemize}

\subsection*{Higher inductive types}\index{type!higher inductive}%

One of the classical advantages of type theory is its simple and effective techniques for working with inductively defined structures.
The simplest nontrivial inductively defined structure is the natural numbers, which is inductively generated by zero and the successor function.
From this statement one can algorithmically\index{algorithm} extract the principle of mathematical induction, which characterizes the natural numbers.
More general inductive definitions encompass lists and well-founded trees of all sorts, each of which is characterized by a corresponding ``induction principle''.
This includes most data structures used in certain programming languages; hence the usefulness of type theory in formal reasoning about the latter.
If conceived in a very general sense, inductive definitions also include examples such as a disjoint union $A+B$, which may be regarded as ``inductively'' generated by the two injections $A\to A+B$ and $B\to A+B$.
The ``induction principle'' in this case is ``proof by case analysis'', which characterizes the disjoint union.

In homotopy theory, it is natural to consider also ``inductively defined spaces'' which are generated not merely by a collection of \emph{points}, but also by collections of \emph{paths} and higher paths.
Classically, such spaces are called \emph{CW complexes}.
\index{CW complex}%
For instance, the circle $S^1$ is generated by a single point and a single path from that point to itself.
Similarly, the 2-sphere $S^2$ is generated by a single point $b$ and a single two-dimensional path from the constant path at $b$ to itself, while the torus $T^2$ is generated by a single point, two paths $p$ and $q$ from that point to itself, and a two-dimensional path from $p\ct q$ to $q\ct p$.

By using the identification of paths with identities in homotopy type theory, these sort of ``inductively defined spaces'' can be characterized in type theory by ``induction principles'', entirely analogously to classical examples such as the natural numbers and the disjoint union.
The resulting \emph{higher inductive types}
\index{type!higher inductive}%
give a direct ``logical'' way to reason about familiar spaces such as spheres, which (in combination with univalence) can be used to perform familiar arguments from homotopy theory, such as calculating homotopy groups of spheres, in a purely formal way.
The resulting proofs are a marriage of classical homotopy-theoretic ideas with classical type-theoretic ones, yielding new insight into both disciplines.

Moreover, this is only the tip of the iceberg: many abstract constructions from homotopy theory, such as homotopy colimits, suspensions, Postnikov towers, localization, completion, and spectrification, can also be expressed as higher inductive types.
Many of these are classically constructed using Quillen's ``small object argument'', which can be regarded as a finite way of algorithmically describing an infinite CW complex presentation\index{presentation!of a space as a CW complex} of a space, just as ``zero and successor'' is a finite algorithmic\index{algorithm} description of the infinite set of natural numbers.
Spaces produced by the small object argument are infamously complicated and difficult to understand; the type-theoretic approach is potentially much simpler, bypassing the need for any explicit construction by giving direct access to the appropriate ``induction principle''.
Thus, the combination of univalence and higher inductive types suggests the possibility of a revolution, of sorts, in the practice of homotopy theory.


\subsection*{Sets in univalent foundations}

\index{set|(}%

We have claimed that univalent foundations can eventually serve as a foundation for ``all'' of mathematics, but so far we have discussed 
only homotopy theory.  Of course, there are many specific examples of the use of type theory without the new homotopy type theory features to formalize mathematics,
\index{mathematics!formalized}%
\index{theorem!Feit--Thompson}%
\index{theorem!odd-order}%
\index{Feit--Thompson theorem}%
\index{odd-order theorem}%
such as the recent formalization of the Feit--Thompson odd-order theorem in \Coq~\cite{gonthier}.

But the traditional view is that mathematics is founded on set theory, in the sense that all mathematical objects and constructions can be coded into a theory such as Zermelo--Fraenkel set theory (ZF).
\index{set theory!Zermelo--Fraenkel}%
\indexsee{Zermelo-Fraenkel set theory}{set theory}%
\indexsee{ZF}{set theory}%
\indexsee{ZFC}{set theory}%
However, it is well-established by now that for most mathematics outside of set theory proper, the intricate hierarchical membership structure of sets in ZF is really unnecessary: a more ``structural'' theory, such as Lawvere's\index{Lawvere} Elementary Theory of the Category of Sets~\cite{lawvere:etcs-long}, suffices.
\index{Elementary Theory of the Category of Sets}%

In univalent foundations, the basic objects are ``homotopy types'' rather than sets, but we can \emph{define} a class of types which behave like sets.
Homotopically, these can be thought of as spaces in which every connected component is contractible, i.e.\ those which are homotopy equivalent to a discrete space.
\index{discrete!space}%
It is a theorem  that the category of such ``sets'' satisfies Lawvere's\index{Lawvere} axioms (or related ones, depending on the details of the theory).
Thus, any sort of mathematics that can be represented in an ETCS-like theory (which, experience suggests, is essentially all of mathematics) can equally well be represented in univalent foundations.  

This supports the claim that univalent foundations is at least as good as existing foundations of mathematics.
A mathematician working in univalent foundations can build structures out of sets in a familiar way, with more general homotopy types waiting in the foundational background until there is need of them.
For this reason, most of the applications in this book have been chosen to be areas where univalent foundations has something \emph{new} to contribute that distinguishes it from existing foundational systems.

Unsurprisingly, homotopy theory and category theory are two of these, but perhaps less obvious is that univalent foundations has something new and interesting to offer even in subjects such as set theory and real analysis.
For instance, the univalence axiom allows us to identify isomorphic structures, while higher inductive types allow direct descriptions of objects by their universal properties.
Thus we can generally avoid resorting to arbitrarily chosen representatives or transfinite iterative constructions.
In fact, even the objects of study in ZF set theory can be characterized, inside the sets of univalent foundations, by such an inductive universal property.

\index{set|)}%


\subsection*{Informal type theory}

\index{mathematics!formalized|(defstyle}%
\index{informal type theory|(defstyle}%
\index{type theory!informal|(defstyle}%
\index{type theory!formal|(}%
One difficulty often encountered by the classical mathematician when faced with learning about type theory is that it is usually presented as a fully or partially formalized deductive system.
This style, which is very useful for proof-theoretic investigations, is not particularly convenient for use in applied, informal reasoning.
Nor is it even familiar to most working mathematicians, even those who might be interested in foundations of mathematics.
One objective of the present work is to develop an informal style of doing mathematics in univalent foundations that is at once rigorous and precise, but is also closer to the language and style of presentation of everyday mathematics.

In present-day mathematics, one usually constructs and reasons about mathematical objects in a way that could in principle, one presumes, be formalized in a system of elementary set theory, such as ZFC --- at least given enough ingenuity and patience.
For the most part, one does not even need to be aware of this possibility, since it largely coincides with the condition that a proof be ``fully rigorous'' (in the sense that all mathematicians have come to understand intuitively through education and experience).
But one does need to learn to be careful about a few aspects of ``informal set theory'': the use of collections too large or inchoate to be sets; the axiom of choice and its equivalents; even (for undergraduates) the method of proof by contradiction; and so on.
Adopting a new foundational system such as homotopy type theory as the \emph{implicit formal basis} of informal reasoning will require adjusting some of one's instincts and practices.
The present text is intended to serve as an example of this ``new kind of mathematics'', which is still informal, but could now in principle be formalized in homotopy type theory, rather than ZFC, again given enough ingenuity and patience.

It is worth emphasizing that, in this new system, such formalization can have real practical benefits.
The formal system of type theory is suited to computer systems and has been implemented in existing proof assistants.
\index{proof!assistant}%
A proof assistant is a computer program which guides the user in construction of a fully formal proof, only allowing valid steps of reasoning.
It also provides some degree of automation, can search libraries for existing theorems, and can even extract numerical algorithms\index{algorithm} \index{extraction of algorithms} from the resulting (constructive) proofs.

We believe that this aspect of the univalent foundations program distinguishes it from other approaches to foundations, potentially providing a new practical utility for the working mathematician.
Indeed, proof assistants based on older type theories have already been used to formalize substantial mathematical proofs, such as the four-color theorem\index{theorem!four-color} \index{four-color theorem} and the Feit--Thompson theorem.
Computer implementations of univalent foundations are presently works in progress (like the theory itself).
\index{proof!assistant}%
However, even its currently available implementations (which are mostly small modifications to existing proof assistants such as \Coq and 
\Agda) have already demonstrated their worth, not only in the formalization of known proofs, but in the discovery of new ones.
Indeed, many of the proofs described in this book were actually \emph{first} done in a fully formalized form in a proof assistant, and are only now being ``unformalized'' for the first time --- a reversal of the usual relation between formal and informal mathematics.

One can imagine a not-too-distant future when it will be possible for mathematicians to verify the correctness of their own papers by working within the system of univalent foundations, formalized in a proof assistant, and that doing so will become as natural as typesetting their own papers in \TeX.
%(Whether this proves to be the publishers' dream or their nightmare remains to be seen.) 
In principle, this could be equally true for any other foundational system, but we believe it to be more practically attainable using univalent foundations, as witnessed by the present work and its formal counterpart.

\index{type theory!formal|)}%
\index{informal type theory|)}%
\index{type theory!informal|)}%
\index{mathematics!formalized|)}%

\subsection*{Constructivity} 

\index{mathematics!constructive|(}%

One of the most striking differences between classical\index{mathematics!classical} foundations and type theory is the idea of \emph{proof relevance}, according to which mathematical statements, and even their proofs, become first-class mathematical objects.
In type theory, we represent mathematical statements by types, which can be regarded simultaneously as both mathematical constructions and mathematical assertions, a conception also known as \emph{propositions as types}.
\index{proposition!as types}%
Accordingly, we can regard a term $a : A$ as both an element of the type $A$ (or in homotopy type theory, a point of the space $A$), and at the same time, a proof of the proposition $A$.
To take an example, suppose we have sets $A$ and $B$ (discrete spaces),
\index{discrete!space}%
and consider the statement ``$A$ is isomorphic to $B$''.
In type theory, this can be rendered as:
\begin{narrowmultline*}
  \mathsf{Iso}(A,B) \defeq \narrowbreak
  \sm{f : A\to B}{g : B\to A}\Big(\big(\tprd{x:A} g(f(x)) = x\big) \times \big(\tprd{y:B}\, f(g(y)) = y\big)\Big).
\end{narrowmultline*}
%
Reading the type constructors $\Sigma, \Pi, \times$  here  as ``there exists'', ``for all'', and ``and'' respectively yields the usual formulation of ``$A$ and $B$ are isomorphic''; on the other hand, reading them as sums and products yields the \emph{type of all isomorphisms} between $A$ and $B$!  To prove that $A$ and $B$ are isomorphic, one  constructs a proof $p : \mathsf{Iso}(A,B)$, which is therefore the same  as constructing an isomorphism between $A$ and $B$, i.e., exhibiting a pair of functions $f, g$ together with \emph{proofs} that their composites are the respective identity maps.  The latter proofs, in turn, are nothing but homotopies of the appropriate sorts.  In this way, \emph{proving a proposition is the same as constructing an element of some particular type.}
In particular, to prove a statement of the form ``$A$ and $B$'' is just to prove $A$ and to prove $B$, i.e., to give an element of the type $A\times B$.
And to prove that $A$ implies $B$ is just to find an element of $A\to B$, i.e.\ a function from $A$ to $B$ (determining a mapping of proofs of $A$ to proofs of $B$).

The logic of propositions-as-types is flexible and supports many variations, such as using only a subclass of types to represent propositions.
In homotopy type theory, there are natural such subclasses arising from the fact that the system of all types, just like spaces in classical homotopy theory, is ``stratified'' according to the dimensions in which their higher homotopy structure exists or collapses.
In particular, Voevodsky has found a purely type-theoretic definition of \emph{homotopy $n$-types}, corresponding to spaces with no nontrivial homotopy information above dimension $n$.
(The $0$-types are the ``sets'' mentioned previously as satisfying Lawvere's axioms\index{Lawvere}.)
Moreover, with higher inductive types, we can universally ``truncate'' a type into an $n$-type; in classical homotopy theory this would be its $n^{\mathrm{th}}$ Postnikov\index{Postnikov tower} section.\index{n-type@$n$-type}
Particularly important for logic is the case of homotopy $(-1)$-types, which we call \emph{mere propositions}.
Classically, every $(-1)$-type is empty or contractible; we interpret these possibilities as the truth values ``false'' and ``true'' respectively.

Using all types as propositions yields a very ``constructive'' conception of logic; for more on this, see~\cite{kolmogorov,TroelstraI,TroelstraII}.
For instance, every proof that something exists carries with it enough information to actually find such an object; and every proof that ``$A$ or $B$'' holds is either a proof that $A$ holds or a proof that $B$ holds.
Thus, from every proof we can automatically extract an algorithm;\index{algorithm} \index{extraction of algorithms} this can be very useful in applications to computer programming.

On the other hand, however, this logic does diverge from the traditional understanding of existence proofs in mathematics.
In particular, it does not faithfully represent certain important classical principles of reasoning, such as the axiom of choice and the law of excluded middle.
For these we need to use the ``$(-1)$-truncated'' logic, in which only the homotopy $(-1)$-types represent propositions.

\index{axiom!of choice}%
More specifically, consider on one hand the \emph{axiom of choice}: ``if for every $x: A$ there exists a $y:B$ such that $R(x,y)$, there is a function $f : A\to B$ such that for all $x:A$ we have $R(x, f(x))$.''
The pure propositions-as-types notion of ``there exists'' is strong enough to make this statement simply provable --- yet it does not have all the consequences of the usual axiom of choice.
However, in $(-1)$-truncated logic, this statement is not automatically true, but is a strong assumption with the same sorts of consequences as its counterpart in classical\index{mathematics!classical} set theory.

\index{excluded middle}%
\index{univalence axiom}%
On the other hand, consider the \emph{law of excluded middle}: ``for all $A$, either $A$ or not $A$.''
Interpreting this in the pure propositions-as-types logic yields a statement that is inconsistent with the univalence axiom.
For since proving ``$A$'' means exhibiting an element of it, this assumption would give a uniform way of selecting an element from every nonempty type --- a sort of Hilbertian choice operator.
Univalence implies that the element of $A$ selected by such a choice operator must be invariant under all self-equivalences of $A$, since these are identified with self-identities and every operation must respect identity; but clearly some types have automorphisms with no fixed points, e.g.\ we can swap the elements of a two-element type.
\index{automorphism!fixed-point-free}%
However, the ``$(-1)$-truncated law of excluded middle'', though also not automatically true, may consistently be assumed with most of the same consequences as in classical mathematics.

In other words, while the pure propositions-as-types logic is ``constructive'' in the strong algorithmic sense mentioned above, the default $(-1)$-truncated logic is ``constructive'' in a different sense (namely, that of the logic formalized by Heyting under the name ``intuitionistic''); and to the latter we may freely add the axioms of choice and excluded middle to obtain a logic that may be called ``classical''.
Thus, homotopy type theory is compatible with both constructive and classical conceptions of logic, and many more besides.
\index{logic!constructive vs classical}%
Indeed, the homotopical perspective reveals that classical and constructive logic can coexist, as endpoints of a spectrum of different systems, with an infinite number of possibilities in between (the homotopy $n$-types for $-1 < n < \infty$).
We may speak of ``\LEM{n}'' and ``\choice{n}'', with $\choice{\infty}$ being provable and \LEM{\infty} inconsistent with univalence, while $\choice{-1}$ and $\LEM{-1}$ are the versions familiar to classical mathematicians (hence in most cases it is appropriate to assume the subscript $(-1)$ when none is given).  Indeed, one can even have useful systems in which only \emph{certain} types satisfy such further ``classical'' principles, while types in general remain ``constructive''.\index{excluded middle}\index{axiom!of choice}%%

It is worth emphasizing that univalent foundations does not \emph{require} the use of constructive or intuitionistic logic.\index{logic!intuitionistic}\index{logic!constructive} %
Most of classical mathematics which depends on the law of excluded middle and the axiom of choice can be performed in univalent foundations, simply by assuming that these two principles hold (in their proper, $(-1)$-truncated, form).
However, type theory does encourage avoiding these principles when they are unnecessary, for several reasons.

First of all, every mathematician knows that a theorem is more powerful when proven using fewer assumptions, since it applies to more examples.
The situation with \choice{} and \LEM{} is no different:
type theory admits many interesting ``nonstandard'' models, such as in sheaf toposes,\index{topos} where classicality principles such as \choice{} and \LEM{} tend to fail.
Homotopy type theory admits similar models in higher toposes, such as are studied in~\cite{ToenVezzosi02,Rezk05,lurie:higher-topoi}.
Thus, if we avoid using these principles, the theorems we prove will be valid internally to all such models.

Secondly, one of the additional virtues of type theory is its computable character.
In addition to being a foundation for mathematics, type theory is a formal theory of computation, and can be treated as a powerful programming language.
\index{programming}%
From this perspective, the rules of the system cannot be chosen arbitrarily the way set-theoretic axioms can: there must be a harmony between them which allows all proofs to be ``executed'' as programs.
We do not yet fully understand the new principles introduced by homotopy type theory, such as univalence and higher inductive types, from
this point of view, but the basic outlines are emerging; see, for example,~\cite{lh:canonicity}.
It has been known for a long time, however, that principles such as \choice{} and \LEM{} are fundamentally antithetical to computability, since they assert baldly that certain things exist without giving any way to compute them.
Thus, avoiding them is necessary to maintain the character of type theory as a theory of computation.

Fortunately, constructive reasoning is not as hard as it may seem.
In some cases, simply by rephrasing some definitions, a theorem can be made constructive and its proof more elegant.
Moreover, in univalent foundations this seems to happen more often.
For instance:
\begin{enumerate}
\item In set-theoretic foundations, at various points in homotopy theory and category theory one needs the axiom of choice to perform transfinite constructions.
  But with higher inductive types, we can encode these constructions directly and constructively.
  In particular, none of the ``synthetic'' homotopy theory in \cref{cha:homotopy} requires \LEM{} or \choice{}.
\item In set-theoretic foundations, the statement ``every fully faithful and essentially surjective functor is an equivalence of categories'' is equiv\-a\-lent to the axiom of choice.
  But with the univalence axiom, it is just \emph{true}; see \cref{cha:category-theory}.
\item In set theory, various circumlocutions are required to obtain notions of ``cardinal number'' and ``ordinal number'' which canonically represent isomorphism classes of sets and well-ordered sets, respectively --- possibly involving the axiom of choice or the axiom of foundation.
  But with univalence and higher inductive types, we can obtain such representatives directly by truncating the universe; see \cref{cha:set-math}.
\item In set-theoretic foundations, the definition of the real numbers as equivalence classes of Cauchy sequences requires either the law of excluded middle or the axiom of (countable) choice to be well-behaved.
  But with higher inductive types, we can give a version of this definition which is well-behaved and avoids any choice principles; see \cref{cha:real-numbers}.
\end{enumerate}
Of course, these simplifications could as well be taken as evidence that the new methods will not, ultimately, prove to be really constructive.  However, we emphasize again that the reader does not have to care, or worry, about constructivity in order to read this book.  The point is that in all of the above examples, the version of the theory we give has independent advantages, whether or not \LEM{} and \choice{} are assumed to be available.  Constructivity, if attained, will be an added bonus.\index{constructivity}%

Given this discussion of adding new principles such as univalence, higher inductive types, \choice{}, and \LEM{}, one may wonder whether the resulting system remains consistent.
(One of the original virtues of type theory, relative to set theory, was that it can be seen to be consistent by proof-theoretic means).
As with any foundational system, consistency\index{consistency} is a relative question: ``consistent with respect to what?''
The short answer is that all of the constructions and axioms considered in this book have a model in the category of Kan\index{Kan complex} complexes, due to Voevodsky~\cite{klv:ssetmodel} (see~\cite{ls:hits} for higher inductive types).
Thus, they are known to be consistent relative to ZFC (with as many inaccessible cardinals
\index{inaccessible cardinal}\index{consistency}%
as we need nested univalent universes).
Giving a more traditionally type-theoretic account of this consistency is work in progress (see,
e.g.,~\cite{lh:canonicity,coquand2012constructive}).

We summarize the different points of view of the type-theoretic operations in \cref{tab:pov}.

\begin{table}[htb]
  \centering
  \OPTsmalltable
 \begin{tabular}{lllll}
    \toprule
       Types && Logic & Sets & Homotopy\\ \addlinespace[2pt]
    \midrule
       $A$ && proposition & set & space\\ \addlinespace[2pt]
       $a:A$ && proof & element & point \\ \addlinespace[2pt]
       $B(x)$ && predicate & family of sets & fibration \\ \addlinespace[2pt]
       $b(x) : B(x)$ && conditional proof & family of elements & section\\ \addlinespace[2pt]
       $\emptyt, \unit$ && $\bot, \top$ & $\emptyset, \{ \emptyset \}$ & $\emptyset, *$\\ \addlinespace[2pt]
       $A + B$ && $A\vee B$ & disjoint union & coproduct\\ \addlinespace[2pt]
       $A\times B$ && $A\wedge B$ & set of pairs & product space\\ \addlinespace[2pt]
       $A\to B$ && $A\Rightarrow B$ & set of functions & function space\\ \addlinespace[2pt]
       $\sm{x:A}B(x)$ &&  $\exists_{x:A}B(x)$ & disjoint sum & total space\\ \addlinespace[2pt]
       $\prd{x:A}B(x)$ &&  $\forall_{x:A}B(x)$ & product & space of sections\\ \addlinespace[2pt]
       $\mathsf{Id}_{A}$ && equality $=$ & $\setof{\pairr{x,x} | x\in A}$ & path space $A^I$ \\ \addlinespace[2pt]
    \bottomrule
  \end{tabular}
  \caption{Comparing points of view on type-theoretic operations}\label{tab:pov}
\end{table}

\index{mathematics!constructive|)}%

\subsection*{Open problems} 

\index{open!problem|(}%

For those interested in contributing to this new branch of mathematics, it may be encouraging to know that there are many interesting open questions.

\index{univalence axiom!constructivity of}%
Perhaps the most pressing of them is the ``constructivity'' of the Univalence Axiom, posed by Voevodsky in \cite{Universe-poly}.
The basic system of type theory follows the structure of Gentzen's natural deduction. Logical connectives are defined by their introduction rules, and have elimination rules justified by computation rules. Following this pattern, and using Tait's computability method, originally designed to analyse G\"odel's Dialectica interpretation, one can show the property of \emph{normalization} for type theory. This in turn implies important properties such as decidability of type-checking (a crucial property since type-checking corresponds to proof-checking, and one can argue that we should be able to ``recognize a proof when we see one''), and the so-called ``canonicity\index{canonicity} property'' that any closed term of the type of natural numbers reduces to a numeral. This last property, and the uniform structure of introduction/elimination rules, are lost when one extends type theory with an axiom, such as the axiom of function extensionality, or the univalence axiom. Voevodsky has formulated a precise mathematical conjecture connected to this question of canonicity for type theory extended with the axiom of Univalence: given a closed term of the type of natural numbers, is it always possible to find a numeral and a proof that this term is equal to this numeral, where this proof of equality may itself use the univalence axiom? More generally, an important issue is whether it is possible to provide a constructive justification of the univalence axiom.
What about if one adds other homotopically motivated constructions, like higher inductive types?
These questions remain open at the present time, although methods are currently being developed to try to find answers.

Another basic issue is the difficulty of working with types, such as the natural numbers, that are essentially sets (i.e., discrete spaces),
\index{discrete!space}%
containing only trivial paths.
At present, homotopy type theory can really only characterize spaces up to homotopy equivalence, which means that these ``discrete spaces'' may only be \emph{homotopy equivalent} to discrete spaces.
Type-theoretically, this means there are many paths that are equal to reflexivity, but not \emph{judgmentally} equal to it (see \cref{sec:types-vs-sets} for the meaning of ``judgmentally'').
While this homotopy-invariance has advantages, these ``meaningless'' identity terms do introduce needless complications into arguments and constructions, so it would be convenient to have a systematic way of eliminating or collapsing them.
% In some cases, the proliferation of such superfluous identity terms makes it very difficult or impossible to formulate what should be a straightforward concept, such as the definition of a (semi-)simplicial type.

A more specialized, but no less important, problem is the relation between homotopy type theory and the research on \emph{higher toposes}%
\index{.infinity1-topos@$(\infty,1)$-topos}
currently happening at the intersection of higher category theory and homotopy theory.
There is a growing conviction among those familiar with both subjects that they are intimately connected.
For instance, the notion of a univalent universe should coincide with that of an object classifier, while higher inductive types should be an ``elementary'' reflection of local presentability.
More generally, homotopy type theory should be the ``internal language'' of $(\infty,1)$-toposes, just as intuitionistic higher-order logic is the internal language of ordinary 1-toposes.
Despite this general consensus, however, details remain to be worked out --- in particular, questions of coherence and strictness remain to be addressed  --- and doing so will undoubtedly lead to further insights into both concepts.

\index{mathematics!formalized}%
But by far the largest field of work to be done is in the ongoing formalization of everyday mathematics in this new system.
Recent successes in formalizing some facts from basic homotopy theory and category theory have been encouraging; some of these are described in \cref{cha:homotopy,cha:category-theory}.
Obviously, however, much work remains to be done.

\index{open!problem|)}%

The homotopy type theory community maintains a web site and group blog at \url{http://homotopytypetheory.org}, as well as a discussion email list.
Newcomers are always welcome!


\subsection*{How to read this book}

This book is divided into two parts.
\cref{part:foundations}, ``Foundations'', develops the fundamental concepts of homotopy type theory.
This is the mathematical foundation on which the development of specific subjects is built, and which is required for the understanding of the univalent foundations approach. To a programmer, this is ``library code''.
Since univalent foundations is a new and different kind of mathematics, its basic notions take some getting used to; thus \cref{part:foundations} is fairly extensive.

\cref{part:mathematics}, ``Mathematics'', consists of four chapters that build on the basic notions of \cref{part:foundations} to exhibit some of the new things we can do with univalent foundations in four different areas of mathematics: homotopy theory (\cref{cha:homotopy}), category theory (\cref{cha:category-theory}), set theory (\cref{cha:set-math}), and real analysis (\cref{cha:real-numbers}).
The chapters in \cref{part:mathematics} are more or less independent of each other, although occasionally one will use a lemma proven in another.

A reader who wants to seriously understand univalent foundations, and be able to work in it, will eventually have to read and understand most of \cref{part:foundations}.
However, a reader who just wants to get a taste of univalent foundations and what it can do may understandably balk at having to work through over 200 pages before getting to the ``meat'' in \cref{part:mathematics}.
Fortunately, not all of \cref{part:foundations} is necessary in order to read the chapters in \cref{part:mathematics}.
Each chapter in \cref{part:mathematics} begins with a brief overview of its subject, what univalent foundations has to contribute to it, and the necessary background from \cref{part:foundations}, so the courageous reader can turn immediately to the appropriate chapter for their favorite subject.
For those who want to understand one or more chapters in \cref{part:mathematics} more deeply than this, but are not ready to read all of \cref{part:foundations}, we provide here a brief summary of \cref{part:foundations}, with remarks about which parts are necessary for which chapters in \cref{part:mathematics}.

\cref{cha:typetheory} is about the basic notions of type theory, prior to any homotopical interpretation.
A reader who is familiar with Martin-L\"of type theory can quickly skim it to pick up the particulars of the theory we are using.
However, readers without experience in type theory will need to read \cref{cha:typetheory}, as there are many subtle differences between type theory and other foundations such as set theory.

\cref{cha:basics} introduces the homotopical viewpoint on type theory, along with the basic notions supporting this view, and describes the homotopical behavior of each component of the type theory from \cref{cha:typetheory}.
It also introduces the \emph{univalence axiom} (\cref{sec:compute-universe}) --- the first of the two basic innovations of homotopy type theory.
Thus, it is quite basic and we encourage everyone to read it, especially \crefrange{sec:equality}{sec:basics-equivalences}.

\cref{cha:logic} describes how we represent logic in homotopy type theory, and its connection to classical logic as well as to constructive and intuitionistic logic.
Here we define the law of excluded middle, the axiom of choice, and the axiom of propositional resizing (although, for the most part, we do not need to assume any of these in the rest of the book), as well as the \emph{propositional truncation} which is essential for representing traditional logic.
This chapter is essential background for \cref{cha:set-math,cha:real-numbers}, less important for \cref{cha:category-theory}, and not so necessary for \cref{cha:homotopy}.

\cref{cha:equivalences,cha:induction} study two special topics in detail: equivalences (and related notions) and generalized inductive definitions.
While these are important subjects in their own rights and provide a deeper understanding of homotopy type theory, for the most part they are not necessary for \cref{part:mathematics}.
Only a few lemmas from \cref{cha:equivalences} are used here and there, while the general discussions in \cref{sec:bool-nat,sec:strictly-positive,sec:generalizations} are helpful for providing the intuition required for \cref{cha:hits}.
The generalized sorts of inductive definition discussed in \cref{sec:generalizations} are also used in a few places in \cref{cha:set-math,cha:real-numbers}.

\cref{cha:hits} introduces the second basic innovation of homotopy type theory --- \emph{higher inductive types} --- with many examples.
Higher inductive types are the primary object of study in \cref{cha:homotopy}, and some particular ones play important roles in \cref{cha:set-math,cha:real-numbers}.
They are not so necessary for \cref{cha:category-theory}, although one example is used in \cref{sec:rezk}.

Finally, \cref{cha:hlevels} discusses homotopy $n$-types and related notions such as $n$-connected types.
These notions are important for \cref{cha:homotopy}, but not so important in the rest of \cref{part:mathematics}, although the case $n=-1$ of some of the lemmas are used in \cref{sec:piw-pretopos}.

This completes \cref{part:foundations}.
As mentioned above, \cref{part:mathematics} consists of four largely unrelated chapters, each describing what univalent foundations has to offer to a particular subject.

Of the chapters in \cref{part:mathematics}, \cref{cha:homotopy} (Homotopy theory) is perhaps the most radical.
Univalent foundations has a very different ``synthetic'' approach to homotopy theory in which homotopy types are the basic objects (namely, the types) rather than being constructed using topological spaces or some other set-theoretic model.
This enables new styles of proof for classical theorems in algebraic topology, of which we present a sampling, from $\pi_1(\Sn^1)=\Z$ to the Freudenthal suspension theorem.

In \cref{cha:category-theory} (Category theory), we develop some basic (1-)category theory, adhering to the principle of the univalence axiom that \emph{equality is isomorphism}.
This has the pleasant effect of ensuring that all definitions and constructions are automatically invariant under equivalence of categories: indeed, equivalent categories are equal just as equivalent types are equal.
(It also has connections to higher category theory and higher topos theory.)

\cref{cha:set-math} (Set theory) studies sets in univalent foundations.
The category of sets has its usual properties, hence provides a foundation for any mathematics that doesn't need homotopical or higher-categorical structures.
We also observe that univalence makes cardinal and ordinal numbers a bit more pleasant, and that higher inductive types yield a cumulative hierarchy satisfying the usual axioms of Zermelo--Fraenkel set theory.

In \cref{cha:real-numbers} (Real numbers), we summarize the construction of Dedekind real numbers, and then observe that higher inductive types allow a definition of Cauchy real numbers that avoids some associated problems in constructive mathematics.
Then we sketch a similar approach to Conway's surreal numbers.

Each chapter in this book ends with a Notes section, which collects historical comments, references to the literature, and attributions of results, to the extent possible.
We have also included Exercises at the end of each chapter, to assist the reader in gaining familiarity with doing mathematics in univalent foundations.

Finally, recall that this book was written as a massively collaborative effort by a large number of people.
We have done our best to achieve consistency in terminology and notation, and to put the mathematics in a linear sequence that flows logically, but it is very likely that some imperfections remain.
We ask the reader's forgiveness for any such infelicities, and welcome suggestions for improvement of the next edition.


% Local Variables:
% TeX-master: "hott-online"
% End:

    \section{Homotopy posets}\label{sec: homotopy poset}
%
To begin, we focus on obstructions to weak terminality.
Having fixed a category $\CategoryC$, we interpret objects of a category $\CategoryC$ as points, and morphisms between them as paths. 
From this point of view, a weak terminal object is an object that is always reachable from any generic object $x$ in $\CategoryC$.

Intuitively, we can fix a ``weak terminal object candidate''\footnote{
    In this paper, we will use $\WTerm$ to denote ``terminal object candidates'', that is, objects for which we want to investigate how far they are from being terminal. For an object that we know or presume to be terminal, we will instead use the notation $\Term$.
} $\WTerm$ and consider any object $x$ such that there is \emph{no} morphism $x \to \WTerm$ as an \emph{obstruction to weak terminality}.
Moreover:
%
%
\begin{itemize}
    \item If $x, y$ are obstructions for $\WTerm$, and there are morphisms $x \to y$ and $y \to x$, we regard them as equivalent: if there were a morphism $x \to \WTerm$ there would be a morphism $y \to \WTerm$, and vice versa.
    \item If $x,y$ are obstructions for $\WTerm$ and there is a morphism $x \to y$, then we regard $x$ as a ``more fundamental obstruction than $y$''. 
    This is because, if there were a morphism $y \to \WTerm$, we would automatically obtain a morphism $x \to \WTerm$ by composition (one can ``go from $x$ to $y$ and then to $\WTerm$''), while the opposite is not true.
\end{itemize}
%
We will devote this section to making this intuition formal.
%
%
\begin{definition}[Poset reflection] \label{def: poset reflection}
    Let $\catpos$ be the large\footnote{We will denote categories in \textit{italics} and large categories in \textbf{bold}. Note that in our constructions, what matters is only the \emph{relative} size: a construction which associates a poset to a category can be applied to a large category, producing a large poset.}
    category of posets and order\nbd preserving maps.
    There is a full and faithful functor $\imath\colon \catpos \incl \catcat$, whose image consists of the categories that are
    \begin{itemize}
        \item \emph{thin} (each hom\nbd set contains at most one morphism), and
        \item \emph{skeletal} (every isomorphism is an automorphism).
    \end{itemize}
    The \emph{poset reflection} $\posref{\CategoryC}$ of a category $\CategoryC$ is its image under the left adjoint $\posref{-}\colon \catcat \to \catpos$ to $\imath$:
    \begin{itemize}
        \item the elements of $\posref{\CategoryC}$ are equivalence classes $\posref{x}$ of objects $x$ of $\CategoryC$, where $\posref{x} = \posref{y}$ if and only if there exist morphisms $x \to y$ and $y \to x$ in $\CategoryC$, and
        \item $\posref{x} \leq \posref{y}$ if and only if there exists a morphism $x \to y$ in $\CategoryC$.
    \end{itemize}
\end{definition}
%
%
\begin{proposition}\label{prop: weak terminal is greatest in posref}
    Let $\CategoryC$ be a category and $\WTerm$ an object in $\CategoryC$.
    The following are equivalent:
    \begin{enumerate}[label=(\alph*)]
        \item $\WTerm$ is a weak terminal (respectively, initial) object in $\CategoryC$;
        \item $\posref{\WTerm}$ is the greatest (respectively, least) element of $\posref{\CategoryC}$.
    \end{enumerate}
\end{proposition}
%
%
%
\begin{definition}[Arrow category]\label{def: arrow category}
    Let $\wkarr$ be the ``walking arrow'' category, that is, the free category on the graph
    \[\begin{tikzcd}[sep=scriptsize]
        0 && 1
        \arrow["a", from=1-1, to=1-3]
    \end{tikzcd}.\]
    The \emph{arrow category} of a category $\CategoryC$ is the functor category $\CategoryC^\wkarr$.
    Explicitly, the objects of $\CategoryC^\wkarr$ are morphisms of $\CategoryC$, while morphisms of $\CategoryC^\wkarr$ are commutative squares in $\CategoryC$.
    There are functors $\mathrm{dom}$, $\mathrm{cod}\colon \CategoryC^\wkarr \to \CategoryC$ which, given a morphism $(h_0, h_1)$, return $h_0$, respectively, $h_1$.
\end{definition}
%
%    
\begin{definition}[Category of pointed objects]\label{def: pointed objects category}
    Let $\CategoryC$ be a category with a chosen terminal object $\Term$.
    A \emph{pointed object} $(x, v)$ of $\CategoryC$ is an object $x$ of $\CategoryC$ together with a morphism $v\colon \Term \to x$, called its \emph{basepoint}.
    The \emph{category of pointed objects} of $\CategoryC$ --- denoted by $\pointed{\CategoryC}$ --- is the coslice category $\slice{\Term}{\CategoryC}$.
\end{definition}
%
%
\begin{proposition}[Functoriality of arrow and pointed objects categories]\label{prop: functoriality of arrow and pointed cats}
    Let $\fun{F}\colon \CategoryC \to \CategoryD$ be a functor.
    Then $\fun{F}$ lifts to a functor $\fun{F}^\wkarr\colon \CategoryC^\wkarr \to \CategoryD^\wkarr$
    using the pointwise action of $\fun{F}$ on $\CategoryC$. 
    
    If moreover $\CategoryC$ and $\CategoryD$ have a chosen terminal object, and if $\fun{F}$ preserves it, then it also lifts to a functor $\pointed{\fun{F}}\colon \pointed{\CategoryC} \to \pointed{\CategoryD}$ sending a pointed object $(x, v)$ of $\CategoryC$ to $(\fun{F}x, \fun{F}v)$, a pointed object of $\CategoryD$.
\end{proposition}
%
%
\begin{definition}[Quotient of an object by a morphism]\label{def: quotient by a morphism}
    Let $\CategoryC$ be a category with chosen pushouts and a terminal object $\Term$.
    Given a morphism $f\colon x \to y$, the \emph{quotient of $y$ by $f$} is the pushout
    \begin{equation*}
        \begin{tikzcd}[sep=scriptsize]
            x && \Term \\
            \\
            y && {y\sslash f}
            \arrow["{!}", from=1-1, to=1-3]
            \arrow["f"', from=1-1, to=3-1]
            \arrow[from=3-1, to=3-3]
            \arrow["{[x]}", from=1-3, to=3-3]
            \arrow["\lrcorner"{anchor=center, pos=0.125, rotate=180}, draw=none, from=3-3, to=1-1]
        \end{tikzcd}
    \end{equation*}
    where $!\colon x \to \Term$ is the unique morphism from $x$ to the terminal object.
\end{definition}
%
%    
\begin{proposition}[Functoriality of the quotient]\label{prop: functoriality of the quotient}
    If $\CategoryC$ has chosen pushouts and a terminal object $\Term$, then for each morphism $f\colon x \to y$ in $\CategoryC$ \autoref{def: quotient by a morphism} determines a pointed object $\fun{Q}(f) \eqdef (y \sslash f, [x])$ of $\CategoryC$. This extends to a functor $\fun{Q}\colon \CategoryC^\wkarr \to \pointed{\CategoryC}$.
    If both $\CategoryC$ and $\CategoryD$ have chosen pushouts and a chosen terminal object $\Term$, and if $\fun{F}$ preserves them, then $\fun{F}$ induces a commutative square of functors
    %
    %
    \begin{equation*}
        \begin{tikzcd}[sep=scriptsize]
            \CategoryC^\wkarr && \pointed{\CategoryC} \\
            \\
            \CategoryD^\wkarr && \pointed{\CategoryD}.
            \arrow["\fun{Q}", from=1-1, to=1-3]
            \arrow["\fun{F}^\wkarr"', from=1-1, to=3-1]
            \arrow["\fun{Q}"', from=3-1, to=3-3]
            \arrow["\pointed{\fun{F}}", from=1-3, to=3-3]
        \end{tikzcd}
    \end{equation*}
\end{proposition}
%
% 
The categories $\catcat$ and $\catpos$ have all limits and colimits, so in particular they have pushouts and a terminal object. The poset reflection functor $\posref{-}\colon \catcat \to \catpos$ sends the terminal category to the terminal poset, and preserves pushouts, since it is a left adjoint.
The preservation can be made strict with respect to a choice on both sides.
We are in the conditions of \autoref{prop: functoriality of the quotient}: there is a commutative square
\begin{equation} \label{eq: quotient and posref}
    \begin{tikzcd}[sep=scriptsize]
        \catcat^\wkarr && \pointed{\catcat} \\
        \\
        \catpos^\wkarr && \pointed{\catpos}.
        \arrow["\fun{Q}", from=1-1, to=1-3]
        \arrow["\posref{-}^\wkarr"', from=1-1, to=3-1]
        \arrow["\fun{Q}", from=3-1, to=3-3]
        \arrow["\pointed{\posref{-}}", from=1-3, to=3-3]
    \end{tikzcd}
\end{equation}
%
We are now ready to define the object of interest of this section.
%
%
\begin{definition}[Zeroth homotopy poset] \label{def: 0th-directed homotopy poset}
    Let $\CategoryC$ be a category and $x$ an object in $\CategoryC$.
    The \emph{zeroth homotopy poset of $\CategoryC$ over $x$} is the pointed poset
    \begin{equation*}
        (\dhom{0}{\CategoryC}{x}, \; [x])
    \end{equation*}
    obtained by applying the functor $\catcat^\wkarr \to \pointed{\catpos}$ from \autoref{eq: quotient and posref} to the slice projection functor
    \begin{equation*}
        \mathrm{dom}\colon \slice{\CategoryC}{x} \to \CategoryC.
    \end{equation*}
\end{definition}
%
%
Let us unravel the definition of $\dhom{0}{\cat{C}}{x}$ to a more explicit form.
We start from the projection functor $\mathrm{dom}\colon \slice{\cat{C}}{x} \to \cat{C}$.
    To this we may either apply $\fun{Q}$ or $\posref{-}^\wkarr$.
    Since quotients in $\catpos$ are simpler to compute than quotients in $\catcat$, we apply poset reflection first, which gives us an order-preserving map
    %
    %
    \begin{equation*}
        \posref{\mathrm{dom}}\colon \posref{\slice{\cat{C}}{x}} \to \posref{\cat{C}}.
    \end{equation*}
    %
    Unravelling the explicit definition of poset reflection for $\slice{\cat{C}}{x}$, we see that:
    %
    %
    \begin{itemize}
        \item an element of $\posref{\slice{\cat{C}}{x}}$ is an equivalence class $\posref{f\colon y \to x}$ of morphisms of $\cat{C}$ with codomain $x$, where $\posref{f} = \posref{g}$ if and only if $f$ factors through $g$ and $g$ factors through $f$, and
        \item $\posref{f} \leq \posref{g}$ if and only if $f$ factors through $g$.
    \end{itemize}
    %
    The map $\posref{\mathrm{dom}}$ sends $\posref{f}$ to $\posref{\mathrm{dom}\, f}$.
    The image of $\posref{\mathrm{dom}}$ is then the set
    \begin{equation*}
        \{ \posref{y} \mid \text{there exists a morphism $f\colon y \to x$ in $\cat{C}$} \},
    \end{equation*}
    which is, equivalently, the lower set of $\posref{x}$ in $\posref{\cat{C}}$.

    Applying $\fun{Q}\colon \catpos^\wkarr \to \pointed{\catpos}$ to this map produces the quotient of $\posref{\cat{C}}$ with all elements of this set identified, pointed with the element resulting from their identification, which we denote by $[x]$.
    Hence, an element of $\dhom{0}{\CategoryC}{x}$ is either $[x]$, or it is $\posref{y}$ for some object $y$ such that there exists no morphism $f\colon y \to x$ in $\CategoryC$.
    The order relation is defined as follows, by case distinction:
    \begin{itemize}
        \item $[x] \leq [x]$ trivially;
        \item $[x] \leq \posref{y}$ if and only if there exists a span $(x \xleftarrow{f} z \xrightarrow{g} y)$ in $\CategoryC$;
        \item it is never the case that $\posref{y} \leq [x]$;
        \item $\posref{y} \leq \posref{z}$ if and only if there exists a morphism $f\colon y \to z$ in $\CategoryC$.
    \end{itemize}
    Notice that $[x]$ is always minimal in $\dhom{0}{\CategoryC}{x}$.

The partial order on $\dhom{0}{\CategoryC}{x}$ ranks obstructions to weak terminality by ``size'': if we removed an obstruction $\posref{y}$, adding a morphism $y \to x$, we would also have to remove all the ``smaller'' obstructions $\posref{z} \leq \posref{y}$.
The minimal element $[x]$ represents the ``non-obstructions'':
\begin{proposition} \label{prop:dhom0_trivial_when_weak_terminal}
Let $\cat{C}$ be a category and $x$ an object in $\cat{C}$.
The following are equivalent:
\begin{enumerate}[label=(\alph*)]
    \item $\dhom{0}{\cat{C}}{x} = \{[x]\}$;
    \item $x$ is a weak terminal object in $\cat{C}$.
\end{enumerate}
\end{proposition}

The notation and terminology is suggestive of the $\pi_0$ of a pointed topological space or groupoid, that is, its set of connected components, pointed with the connected component of the basepoint. 
The following result shows that, indeed, the notions coincide when $\cat{C}$ happens to be a groupoid.

\begin{proposition}[$\dhom{0}{\cat{G}}{x}$ for a groupoid]\label{prop: dhom0 is pi0 for groupoids}
    Let $\cat{G}$ be a groupoid and $x$ an object in $\cat{G}$.
    Then
    \begin{enumerate}
        \item $\dhom{0}{\cat{G}}{x}$ is a ``set'', that is, a discrete poset, and
        \item as a pointed set, it is isomorphic to the set $\pi_0(\cat{G})$ of connected components of $\cat{G}$, pointed with the connected component of $x$.
    \end{enumerate}
\end{proposition}

    Now, we investigate obstructions to \emph{subterminality}.
Our main strategy will be to recast subterminality in a way that allows us to leverage \autoref{def: 0th-directed homotopy poset}.
We know that an object $\WTerm$ fails to be subterminal when, for an object $x$, the arrow $x \to \WTerm$ is not unique.
As such, we will describe obstructions to subterminality as pairs of parallel, unequal arrows.
%
%
\begin{definition}[Category of parallel arrows over an object]\label{def: category of parallel arrows}
    Let $\CategoryC$ be a category and $x$ an object in $\CategoryC$.
    The \emph{category of parallel arrows in $\CategoryC$ over $x$} is the category $\pararr{\CategoryC}{x}$ where:
    %
    %
    \begin{itemize}
        \item Objects are pairs of morphisms $(f_0, f_1\colon y \to x)$ with codomain $x$.
        \item A morphism from $(f_0, f_1\colon y \to x)$ to $(g_0, g_1\colon z \to x)$ is a morphism $h\colon y \to z$ such that $f_0 = h\Cp g_0$ and $f_1 = h\Cp g_1$.
    \end{itemize}
    This comes with a projection functor $\mathrm{dom}\colon \pararr{\CategoryC}{x} \to \CategoryC$ sending a parallel pair to its domain.
\end{definition}
%
%    
\begin{proposition}\label{prop: subterminal as weak terminal parallel arrow}
    Let $\CategoryC$ be a category and $\WTerm$ an object in $\CategoryC$.
    The following are equivalent:
    %
    %
    \begin{enumerate}[label=(\alph*)]
        \item $\WTerm$ is subterminal in $\CategoryC$;
        \item $(\idd{\WTerm}, \idd{\WTerm})$ is a terminal object in $\pararr{\CategoryC}{\WTerm}$;
        \item $(\idd{\WTerm}, \idd{\WTerm})$ is a weak terminal object in $\pararr{\CategoryC}{\WTerm}$.
    \end{enumerate}
\end{proposition}
%
%
\autoref{prop: subterminal as weak terminal parallel arrow} allows us to reduce the study of obstructions to subterminality of an object $\WTerm$ in $\CategoryC$ to the study of obstructions to weak terminality of $(\idd{\WTerm}, \idd{\WTerm})$ in $\pararr{\CategoryC}{\WTerm}$. 
%
%
\begin{definition}[First homotopy poset] \label{def: 1st-directed homotopy poset}
    Let $\CategoryC$ be a category and $x$ an object in $\CategoryC$.
    The \emph{first homotopy poset of $\CategoryC$ over $x$} is the pointed poset
    %
    %
    \begin{equation*}
        (\dhom{1}{\CategoryC}{x}, \, [x]) \eqdef \left(\dhom{0}{\pararr{\CategoryC}{x}}{(\idd{x}, \idd{x})}, \, [(\idd{x}, \idd{x})]\right).
    \end{equation*}
\end{definition}
%
%
Putting together the description of the 0th homotopy poset, the definition of $\pararr{\CategoryC}{x}$ in \autoref{def: category of parallel arrows}, and \autoref{prop: subterminal as weak terminal parallel arrow}, we see that an element of $\dhom{1}{\CategoryC}{x}$ is either $[x]$, or $\posref{(f, g)}$ for some parallel pair of morphisms $f, g\colon y \to x$ in $\CategoryC$ with $f \neq g$.
    %
    The order relation is defined as follows:
    \begin{itemize}
        \item $[x] \leq [x]$ trivially;
        \item $[x] \leq \posref{(f, g\colon y \to x)}$ if and only if there exists a morphism $h\colon z \to y$ in $\CategoryC$ equalising $(f, g)$, that is, satisfying $h\Cp f = h\Cp g$;
        \item it is never the case that $\posref{(f, g)} \leq [x]$;
        \item $\posref{(f, g\colon y \to x)} \leq \posref{(f', g'\colon y' \to x)}$ if and only if there exists a morphism $h\colon y \to y'$ such that $f = h\Cp f'$ and $g = h\Cp g'$ in $\CategoryC$.
    \end{itemize}
%
%
\begin{proposition} \label{prop:dhom1_trivial_when_subterminal}
Let $\cat{C}$ be a category and $x$ an object in $\cat{C}$.
The following are equivalent:
\begin{enumerate}[label=(\alph*)]
    \item $\dhom{1}{\cat{C}}{x} = \{[x]\}$;
    \item $x$ is subterminal in $\cat{C}$.
\end{enumerate}
\end{proposition}

\begin{corollary} \label{prop:dhoms_trivial_when_terminal}
Let $\cat{C}$ be a category and $x$ an object in $\cat{C}$.
The following are equivalent:
\begin{enumerate}[label=(\alph*)]
    \item $\dhom{0}{\cat{C}}{x} = \{[x]\}$ and $\dhom{1}{\cat{C}}{x} = \{[x]\}$,
    \item $x$ is a terminal object in $\cat{C}$.
\end{enumerate}
\end{corollary}

\begin{remark}\label{rem: pi1 of a groupoid}
    Recall that the (underlying set of the) fundamental group of a pointed topological space $(X, x)$ is defined by
    \begin{equation*}
        \pi_1(X, x) \eqdef \pi_0(\Omega(X, x), c_x)
    \end{equation*}
    %
    where $\Omega(X, x)$ is the space of loops in $X$ based at $x$, and $c_x$ is the constant path at $x$.
    For a pointed groupoid, which may be seen as the fundamental groupoid of a pointed space, this reduces to the set of automorphisms of the object $x$, pointed with the identity automorphism.
    
    The definition of $\dhom{1}{\CategoryC}{x}$ is made in analogy with this, letting the category of parallel arrows over $x$ replace the space of loops based at $x$, and a pair of identity morphisms replace the constant path.
    The following result proves that, just like the zeroth homotopy poset, the first homotopy poset is a generalisation of its groupoidal analogue.
\end{remark}
%
%
\begin{proposition}[$\dhom{1}{\cat{G}}{x}$ for a groupoid]\label{prop: dhom1 is pi1 for groupoids}
    Let $\cat{G}$ be a groupoid and $x$ an object in $\cat{G}$.
    Then:
    \begin{enumerate}
        \item $\dhom{1}{\cat{G}}{x}$ is a ``set'', that is, a discrete poset, and
        \item as a pointed set, it is isomorphic to the underlying pointed set of the group $\pi_1(\cat{G}, x) = \homset{\cat{G}}{x}{x}$.
    \end{enumerate}
\end{proposition}

\begin{remark}
    We mention here that the field of \emph{directed algebraic topology} \cite{grandis2009directed, fajstrup2016directed} has also produced ``non-invertible'' versions of $\pi_1$, namely, the fundamental \emph{category} and \emph{monoids}, that apply to directed spaces.
    If applied to a category, these pick out ``tautologically'' the category itself and its monoids of endomorphisms.
    To our knowledge, there is no strong relation to our line of research.
\end{remark}

%The elements of $\dhom{0}{\CategoryC}{x}$ and $\dhom{1}{\CategoryC}{x}$ that are minimal in the complement of $\{ [x] \}$ are often of particular importance, for the following reason.
%
%
%\begin{proposition}[Existence of joins]\label{prop: If C has coproducts and slice has weak initials, then dhom0 has joins}
%Let $\cat{C}$ be a category, $x$ an object of $\cat{C}$, and $\kappa$ a cardinal.
%If $\cat{C}$ has $\kappa$\nbd small coproducts, then $\dhom{0}{\cat{C}}{x}$ and $\dhom{1}{\cat{C}}{x}$ have $\kappa$\nbd small joins.
%\end{proposition}
%
%This has the consequence that, in many cases, elements of the homotopy posets are describable as joins of smaller elements; in particular, minimal elements in the complement of $\{ [x] \}$.
%We will call these \emph{minimal obstructions}.

To conclude this section, we show in what way the homotopy posets are functorial in the pair $(\cat{C}, x)$ of a category and an object.

\begin{proposition}[Functoriality of the homotopy posets] \label{prop: Homotopy posets are functorial}
Let $\cat{C}$ be a category, $i \in \{0, 1\}$.
Then:
\begin{enumerate}
    \item the assignment $x \mapsto \dhom{i}{\cat{C}}{x}$ extends to a functor
        $\dhom{i}{\cat{C}}{-}\colon \cat{C} \to \pointed{\catpos}$;
    \item a functor $\fun{F}\colon \cat{C} \to \cat{D}$ induces a natural transformation 
        $\pi_i(\fun{F})\colon \dhom{i}{\cat{C}}{-} \Rightarrow \dhom{i}{\cat{D}}{\fun{F}-}.$
\end{enumerate}
Given another functor $\fun{G}\colon \cat{D} \to \cat{E}$,  this assignment satisfies
\begin{equation*}
    \pi_i(\fun{F}\Cp \fun{G}) = \pi_i(\fun{F}) \Cp \pi_i(\fun{G}), \quad \quad \pi_i(\idd{C}) = \idd{\dhom{i}{\cat{C}}{-}}.
\end{equation*}
\end{proposition}

A concise way of packaging this information is to say that $\pi_i$ defines a functor from $\catcat$ to the \emph{lax slice} $\laxslice{\lcatcat}{\pointed{\catpos}}$, where $\lcatcat$ is the ``huge'' category of possibly large categories.
The objects of the lax slice are pairs of a possibly large category $\lcat{C}$ and a functor $\lcat{C} \to \pointed{\catpos}$, and the morphisms are triangles of functors commuting up to a natural transformation.
Indeed, given $\fun{F}\colon \cat{C} \to \cat{D}$, we have a triangle
\begin{equation*}
\begin{tikzcd}
	{\cat{C}} &&& \pointed{\catpos} \\
	\\
	{\cat{D}}
	\arrow["{\fun{F}}"', from=1-1, to=3-1]
	\arrow["{\dhom{i}{\cat{D}}{-}}"', from=3-1, to=1-4]
	\arrow[""{name=0, anchor=center, inner sep=0}, "{\dhom{i}{\cat{C}}{-}}", from=1-1, to=1-4]
	\arrow["{\pi_i(\fun{F})}"', shorten <=17pt, shorten >=26pt, Rightarrow, from=0, to=3-1]
\end{tikzcd}
\end{equation*}
commuting up to the natural transformation $\pi_i(\fun{F})$.
%
%
\begin{remark}[Dual invariants] \label{rmk: Dual invariants}
As usual, all the constructions can be dualised to $\opp{\cat{C}}$.
This will replace the slice over an object and its domain opfibration with the slice under an object and its codomain fibration, producing invariants classifying obstructions to \emph{initiality} of the object.
\end{remark}

    \section{Obstructions to a morphism being iso} \label{sec: obstructions}


As remarked in the Introduction, one of our main motivations for introducing homotopy posets was measuring how far a generic morphism is from being iso.
Just as we could separate obstructions to terminality into obstructions to weak terminality and subterminality, we can separate obstructions to a morphism being iso into obstructions to a morphism being split epi and mono, respectively.
%
%
\begin{proposition}\label{prop: spit epi weak term mono subterm}
    Let $f\colon X \to Y$ be a morphism in a category $\CategoryC$. Then:
    %
    %
    \begin{itemize}
        \item $f$ is split epi in $\CategoryC$ if and only if $f$ is weak terminal in $\slice{\CategoryC}{Y}$,
        \item $f$ is mono in $\CategoryC$ if and only if $f$ is subterminal in $\slice{\CategoryC}{Y}$.
    \end{itemize}
    %
\end{proposition}
%
%
\begin{corollary}\label{cor: spit epi iff dhom0 trivial mono iff dhom1 trivial}
    Let $f: X \to Y$ be a morphism in a category $\CategoryC$. Then:
    \begin{itemize}
        \item $f$ is split epi if and only if $\dhom{0}{(\slice{\cat{C}}{Y})}{f}$ is trivial;
        \item $f$ is mono if and only if $\dhom{1}{(\slice{\cat{C}}{Y})}{f}$ is trivial, and:
        \item $f$ is iso if and only if both $\dhom{0}{(\slice{\cat{C}}{Y})}{f}$ and $\dhom{1}{(\slice{\cat{C}}{Y})}{f}$ are trivial.
    \end{itemize}
\end{corollary}
%
Furthermore, when the homotopy posets associated to a morphism $f$ are not trivial, they give us precise information about why $f$ fails to be split epi and mono.

To make this more concrete, let us spell out precisely how to compute the invariants associated to a function between sets, where split epi (assuming choice) means \emph{surjective} and mono means \emph{injective}.
This amounts to calculating $\dhom{0}{(\slice{\Set}{Y})}{f}$ and $\dhom{1}{(\slice{\Set}{Y})}{f}$ for some function $f\colon X \to Y$. 
%
%
\begin{proposition}\label{prop: dhom0 for set/y}
    Let $f\colon X \to Y$ be a function between sets. $\posref{\slice{\catset}{Y}}$ is isomorphic, as a poset, to the power set $\powerset{Y}$, via the assignment $(S \subseteq Y) \mapsto \posref{\imath_S}$, where $\imath_S$ is the injective function including $S$ into $Y$.
    Through this bijection, $\posref{f}$ corresponds to the image $f(X)$ of $f$.
\end{proposition}
%
Using this correspondence and quotienting by the lower set of $f(X)$, which contains in particular $\varnothing$, we may identify $\dhom{0}{(\slice{\catset}{Y})}{f}$ with the subposet of $\powerset{Y}$ whose elements are either $\varnothing$ or subsets of $Y$ that contain at least one element $y \notin f(X)$.
The ``minimal obstructions'', that is, the minimal elements in the complement of the basepoint, are the singletons $\{y\}$ with $y \in Y \setminus f(X)$.
This poset is trivial if and only if $f(X) = Y$, that is, iff $f$ is surjective.
%
%
\begin{example}
    Let $f\colon \{0,1\} \to \{0,1,2,3\}$ be the function mapping $0 \mapsto 0$ and $1 \mapsto 1$. 
    The homotopy poset $\dhom{0}{(\slice{\catset}{\{0,1,2,3\}})}{f}$ has the following structure:
    %
    %
    \begin{equation*}
    \def\interval{1.75}
        \scalebox{0.75}{
        \begin{tikzpicture}
            \node(4) at (0,4*\interval) {$\{0,1,2,3\}$};
            \node (3a) at (-6,3*\interval) {$\{0,1,2\}$};
            \node (3b) at (-2,3*\interval) {$\{0,2,3\}$};
            \node (3c) at (2,3*\interval) {$\{1,2,3\}$};
            \node (3d) at (6,3*\interval) {$\{0,1,3\}$};

            \node (2a) at (-8,2*\interval) {$\{0,2\}$};
            \node (2b) at (-4,2*\interval) {$\{1,2\}$};
            \node (2c) at (0,2*\interval) {$\{2,3\}$};
            \node (2d) at (4,2*\interval) {$\{0,3\}$};
            \node (2e) at (8,2*\interval) {$\{1,3\}$};

            \node (1a) at (-4,1*\interval) {$\{2\}$};
            \node (1b) at (4,1*\interval) {$\{3\}$};

        \node (0) at (0,0) {$\varnothing$};
            
            \draw[thick] (3a) -- (4);
            \draw[thick] (3b) -- (4);
            \draw[thick] (3c) -- (4);
            \draw[thick] (3d) -- (4);

            \draw[thick] (2a) -- (3a);
            \draw[thick] (2a) -- (3b);

            \draw[thick] (2b) -- (3a);
            \draw[thick] (2b) -- (3c);

            \draw[thick] (2c) -- (3b);
            \draw[thick] (2c) -- (3c);

            \draw[thick] (2d) -- (3b);
            \draw[thick] (2d) -- (3d);

            \draw[thick] (2e) -- (3c);
            \draw[thick] (2e) -- (3d);

            \draw[thick] (1a) -- (2a);
            \draw[thick] (1a) -- (2b);
            \draw[thick] (1a) -- (2c);

            \draw[thick] (1b) -- (2c);
            \draw[thick] (1b) -- (2d);
            \draw[thick] (1b) -- (2e);

            \draw[thick] (0) -- (1a);
            \draw[thick] (0) -- (1b);
        \end{tikzpicture}
        }
    \end{equation*}
    The minimal obstructions $\{2\}$ and $\{3\}$ are in bijection with the elements not in the image of $f$.
\end{example}
%
%
\begin{proposition}\label{prop: dhom1 for set/y}
    Let $X \times_f X$ be the pullback of $f$ along itself --- that is, the set $\{(x_0, x_1) \mid f(x_0) = f(x_1)\}$ --- and let $p_f\colon X \times_f X \to Y$ be the function $(x_0, x_1) \mapsto f(x_0) = f(x_1)$. Then:
    \begin{enumerate}
        \item $\posref{\pararr{(\slice{\catset}{Y})}{f}}$ is isomorphic to $\powerset{(X \times_f X)}$ via the assignment $(S \subseteq X \times_f X) \mapsto \posref{(\restr{p_0}{S}, \restr{p_1}{S})}$, where $\restr{p_i}{S}$ are the projections $X \times_f X \to Y$, restricted to $S$, seen as morphisms $\restr{p_f}{S} \to f$ in $\posref{\pararr{(\slice{\catset}{Y})}{f}}$;
        \item through this bijection, $\posref{(\idd{f}, \idd{f})}$ is identified with the diagonal $\Delta X$.
    \end{enumerate}
\end{proposition}
%
%
Using this correspondence, we may identify $\dhom{1}{\catset}{X}$ with the subposet of $\powerset{(X \times_f X)}$ whose elements are either $\varnothing$, or contain at least one pair $(x_0, x_1)$ such that $x_0 \neq x_1$.
This poset is trivial if and only if $f$ is injective. 
Notice that the minimal obstructions to injectiveness of $f$ are in bijection with pairs $(x_0, x_1)$ where $x_0 \neq x_1$ but $f(x_0) = f(x_1)$.
%
%
\begin{example}
    Let $f: \{0,1\} \to \{*\}$ be the function mapping $0 \mapsto *$, $1 \mapsto *$. Then $\{0,1\} \times_f \{0,1\}$ is the set \{(0,0),(0,1),(1,0),(1,1)\}, and $\dhom{1}{(\slice{\catset}{\{*\}})}{f}$ has the following structure:
    %
    %
    \begin{equation*}
    \def\interval{1.75}
        \scalebox{0.75}{
        \begin{tikzpicture}
            \node (4a) at (0,4*\interval) {$\{(0,0),(0,1),(1,0),(1,1)\}$};

            \node (3a) at (-6,3*\interval) {$\{(0,0),(0,1),(1,1)\}$};
            \node (3b) at (-2,3*\interval) {$\{(0,1),(1,0),(1,1)\}$};
            \node (3c) at (2,3*\interval) {$\{(0,0),(0,1),(1,0)\}$};
            \node (3d) at (6,3*\interval) {$\{(0,0),(1,0),(1,1)\}$};

            \node (2a) at (-8,2*\interval) {$\{(1,1),(0,1)\}$};
            \node (2b) at (-4,2*\interval) {$\{(0,0),(0,1)\}$};
            \node (2c) at (0,2*\interval) {$\{(0,1),(1,0)\}$};
            \node (2d) at (4,2*\interval) {$\{(1,1),(1,0)\}$};
            \node (2e) at (8,2*\interval) {$\{(0,0),(1,0)\}$};

            \node (1a) at (-4,1*\interval) {$\{(0,1)\}$};
            \node (1b) at (4,1*\interval) {$\{(1,0)\}$};

            \node (0) at (0,0) {$\varnothing$};

            \draw[thick] (3a) -- (4a);
            \draw[thick] (3b) -- (4a);
            \draw[thick] (3c) -- (4a);
            \draw[thick] (3d) -- (4a);
            
            \draw[thick] (2a) -- (3a);
            \draw[thick] (2a) -- (3b);
            \draw[thick] (2b) -- (3a);
            \draw[thick] (2b) -- (3c);
            \draw[thick] (2c) -- (3b);
            \draw[thick] (2c) -- (3c);
            \draw[thick] (2d) -- (3b);
            \draw[thick] (2d) -- (3d);
            \draw[thick] (2e) -- (3d);
            \draw[thick] (2e) -- (3c);

            \draw[thick] (1a) -- (2a);
            \draw[thick] (1a) -- (2b);
            \draw[thick] (1a) -- (2c);
            \draw[thick] (1b) -- (2c);
            \draw[thick] (1b) -- (2d);
            \draw[thick] (1b) -- (2e);

            \draw[thick] (0) -- (1a);
            \draw[thick] (0) -- (1b);
        \end{tikzpicture}
        }
    \end{equation*}
Notice that, via the isomorphism $\Set \simeq \slice{\Set}{\{*\}}$, this is isomorphic to $\dhom{1}{\Set}{\{0, 1\}}$.
\end{example}
%
%
To conclude, suppose that two morphisms are both components of the same natural transformation.
Is there a relation between the associated invariants?
The following result answers this question in the affirmative.
%
%
\begin{proposition}[Covariance over the domain of a natural transformation] \label{prop: covariance natural transformation}
Let $\fun{F}, \fun{G}\colon \cat{C} \to \cat{D}$ be two functors and let $\alpha\colon \fun{F} \Rightarrow \fun{G}$ be a natural transformation.
For all $i \in \{ 0, 1\}$, the assignment
\begin{equation*}
    x \; \mapsto \; \dhom{i}{(\slice{\cat{D}}{\fun{G}{x}})}{\alpha_x}
\end{equation*}
extends to a functor $\cat{C} \to \pointed{\catpos}$.
\end{proposition}
%
%
Notice that this is \emph{not} simply a consequence of \autoref{prop: Homotopy posets are functorial}, that is, it does not arise from the general functoriality result by pre-composition with another functor.\footnote{There is a unifying perspective on the two functoriality results, involving the theory of fibrations and cofibrations of categories; this will be discussed in an extended technical paper.}
It implies that we can naturally map obstructions for $\alpha_x$ to obstructions for $\alpha_y$ along a morphism $f\colon x \to y$ in $\cat{C}$; we can think of morphisms in $\cat{C}$ as inducing a ``flow'' of obstructions to the components of $\alpha$, under which a non-trivial obstruction may be trivialised, but it can never be the case that a non-obstruction is ``un-trivialised''.

    \section{Qualifying compositionality} \label{sec: qualifying}

Now let $\fun{P}\colon \cat{C} \to \cat{D}$ be a \emph{lax} functor of \emph{bicategories}.
This means that, for all triples of objects $X, Y, Z$ in $\cat{C}$, we have two functors
\begin{equation*}
    (\fun{P}-) \Cp (\fun{P}-), \; \fun{P}(- \Cp -)\colon \homset{\cat{C}}{X}{Y} \times \homset{\cat{C}}{Y}{Z} \to \homset{\cat{D}}{\fun{P}X}{\fun{P}Z}
\end{equation*}
connected by a natural transformation, the \emph{laxator} $\varphi\colon (\fun{P}-) \Cp (\fun{P}-) \Rightarrow \fun{P}(- \Cp -)$.\footnote{Technically, the laxators are a family of natural transformations indexed by $X, Y, Z$, but we will leave the indexing implicit.}
As a special case, when $\cat{C}$ and $\cat{D}$ are monoidal categories seen as one-object bicategories, $\fun{P}$ is a lax monoidal functor, and the laxator is a natural transformation $(\fun{P}-) \otimes (\fun{P}-) \Rightarrow \fun{P}(- \otimes -)$.

By Proposition \ref{prop: covariance natural transformation}, we obtain functors $\homset{\cat{C}}{X}{Y} \times \homset{\cat{C}}{Y}{Z} \to \pointed{\catpos}$
sending a pair of morphisms $(f\colon X \to Y, g\colon Y \to Z)$ to the homotopy posets
\begin{equation*}
    \dhom{i}
    {(\slice{\homset{\cat{D}}{\fun{P}X}{\fun{P}Z}}{\fun{P}(f\Cp g)})}
    {\varphi_{f,g}}
\end{equation*}
associated to the component $\varphi_{f,g}$ of the laxator.

In the scenario sketched in the Introduction, the failure of $\varphi_{f,g}$ to be iso is a failure of the ``semantic'' functor $\fun{P}$ to be ``fully compositional'' with respect to the composition $f \Cp g$.
Thus the elements of these homotopy posets may be seen as local \emph{obstructions to compositionality} of $\fun{P}$.
Most interestingly, these obstructions are covariant with respect to the 2-morphisms of $\cat{C}$; thus we can think of ``modifying $f$ and $g$'' by acting on them with a 2-morphism, and see how that affects the obstructions.

\subsection{Open Graphs}\label{subsec: open graphs}
%
%
We apply our framework to a couple of tangible examples.
Open graphs, defined in~\cite{Fong2015}, can be thought of as \emph{graphs with interfaces}. Formally, open graphs are (isomorphism classes of) decorated cospans with decorations in the category $\catgrph$ of graphs and homomorphisms. Intuitively, they are depicted as in the examples below, with \emph{input} vertices on the left and \emph{output} vertices on the right:
%
%
\begin{equation*}
    \scalebox{0.75}{
    \begin{tikzpicture}
        \begin{scope}[xshift=-2cm]
            \node[circle, fill, minimum size=5pt, inner sep=0pt, label=left:{$1$}] (al1) at (-2,0) {};
            \node[circle, fill, minimum size=5pt, inner sep=0pt, label=right:{$1$}] (ar1) at (0,0) {};
            \node[circle, fill, minimum size=5pt, inner sep=0pt, label=right:{$2$}] (ar2) at (0,-1) {};
            \node[circle, fill, minimum size=5pt, inner sep=0pt, label=right:{$3$}] (ar3) at (0,-2) {};
                \draw[thick] (al1) to (ar1);
                \draw[thick, out=180, in=180, looseness=2] (ar2) to (ar3);
        \end{scope}
        \begin{scope}[xshift=2cm]
            \node[circle, fill, minimum size=5pt, inner sep=0pt, label=right:{$1$}] (br3) at (2,-2) {};
            \node[circle, fill, minimum size=5pt, inner sep=0pt, label=left:{$1$}] (bl1) at (0,0) {};
            \node[circle, fill, minimum size=5pt, inner sep=0pt, label=left:{$2$}] (bl2) at (0,-1) {};
            \node[circle, fill, minimum size=5pt, inner sep=0pt, label=left:{$3$}] (bl3) at (0,-2) {};
                \draw[thick, out=0, in=0, looseness=2] (bl1) to (bl2);
                \draw[thick] (bl3) to (br3);
        \end{scope}
        \begin{scope}[xshift=8cm]
            \begin{scope}[xshift=-0cm]
                \node[circle, fill, minimum size=5pt, inner sep=0pt,label=left:{$1$}] (al1) at (-2,0) {};
                \node[circle, fill, minimum size=5pt, inner sep=0pt] (ar1) at (0,0) {};
                \node[circle, fill, minimum size=5pt, inner sep=0pt] (ar2) at (0,-1) {};
                \node[circle, fill, minimum size=5pt, inner sep=0pt] (ar3) at (0,-2) {};
                    \draw[thick] (al1) to (ar1);
                    \draw[thick, out=180, in=180, looseness=2] (ar2) to (ar3);
            \end{scope}
            \begin{scope}[xshift=0cm]
                \node[circle, fill, minimum size=5pt, inner sep=0pt, label=right:{$1$}] (br3) at (2,-2) {};
                \node[circle, fill, minimum size=5pt, inner sep=0pt] (bl1) at (0,0) {};
                \node[circle, fill, minimum size=5pt, inner sep=0pt] (bl2) at (0,-1) {};
                \node[circle, fill, minimum size=5pt, inner sep=0pt] (bl3) at (0,-2) {};
                    \draw[thick, out=0, in=0, looseness=2] (bl1) to (bl2);
                    \draw[thick] (bl3) to (br3);
            \end{scope}
        \end{scope}
    \end{tikzpicture}}
\end{equation*}
%
Indeed, there is a bicategory $\catopengrph$ that has sets as objects, open graphs as morphisms, and interface-preserving graph homomorphisms as 2-morphisms.
For instance, the first and second open graphs above correspond to morphisms $G\colon \{1\} \to \{1,2,3\}$ and $H\colon \{1,2,3\} \to \{1\}$. 
These morphisms can be composed, resulting in the morphism $G \Cp H\colon \{1\} \to \{1\}$ corresponding to the third open graph in the picture above.

Every graph can be mapped to its \emph{reachability relation}\footnote{Cfr. \cite{lorenz2023causal}, for the similar example of open causal models and causal influence.}: this is a relation on the vertexes of the graph, where two vertexes are considered related iff there is a path between them.
Reachability can be recast as a lax functor $\catopengrph \to \catrel$ to the bicategory of sets, relations, and inclusions of relations, which maps an open graph $G\colon X \to Y$ to the relation $\fun{R}G\colon X \to Y$ defined by
\begin{equation*}
    \text{$\fun{R}G(x, y)$ if and only if there is a path between the input vertex $x$ and the output vertex $y$.}
\end{equation*}
Because $\catrel$ is locally posetal, to define $\fun{R}$ on 2-morphisms it suffices to verify that, if $f\colon G \to G'$ is a graph homomorphism, then $\fun{R}G \subseteq \fun{R}G'$.
The laxators are also uniquely defined.

We can see that this functor is not strong.
In the example above we have that $\fun{R}G \subseteq \{1\} \times \{1,2,3\}$ only contains the pair $(1,1)$, since there are no paths from $1$ to $2$ and from $1$ to $3$ in $G$.
Similarly, $\fun{R}H \subseteq \{1,2,3\} \times \{1\}$ only contains the pair $(3,1)$.
It follows that $\fun{R}G \Cp \fun{R}H\colon \{1\} \to \{1\}$ is the empty relation, but $\fun{R}(G \Cp H)\colon \{1\} \to \{1\}$ is total, so $\fun{R}G \Cp \fun{R}H \subsetneq \fun{R}(G \Cp H)$.

The result is that, if we want to compute the reachability relation of $G \Cp H$ by looking at the reachability relations of $G$ and $H$ separately, we are going to miss something.
This ``compositionality gap'' is tracked by the $\pi_0$ associated to the laxator components $\varphi_{G, H}\colon \fun{R}G \Cp \fun{R}H \subseteq \fun{R}(G \Cp H)$ (because these are all injective, the $\pi_1$ will always be trivial).

In our example, $\dhom{0}{(\slice{\homset{\catrel}{\{1\}}{\{1\}}}{\fun{R}(G\Cp H)})}{\varphi_{G,H}}$ is isomorphic to the poset $(\varnothing < \{(1, 1)\})$ pointed with $\varnothing$, so there is exactly one non-trivial obstruction.
Using covariance, we can think of ``removing the obstruction'' by modifying one or both of the parts $G$ or $H$ with a 2-morphism, that is, with a graph homomorphism.
For example, we can act on $G$ with the homomorphism which identifies the output vertices $1$ and $3$.
The resulting graph $G'$ has $\fun{R}G' = \{(1, 1), (1, 3)\}$, so $\fun{R}G' \Cp \fun{R}H = \fun{R}(G' \Cp H) = \{(1, 1)\}$; correspondingly, we obtain a map of pointed posets from the $\pi_0$ associated to $\varphi_{G, H}$ to the $\pi_0$ associated to $\varphi_{G', H}$, which ``trivialises all obstructions''.
%
%
\subsection{Schr\"odinger Compositionality}\label{subsec: schrodinger compositionality}

The name \emph{Schr\"odinger compositionality} was introduced in \cite{coecke2021compositionality} to refer to the form of compositionality that exists in quantum mechanics, where \emph{non-separable states} are present, to disambiguate it from others.
\footnote{For the purposes of this work, we are leaving out of the present analysis the aspects of Schr\"odinger compositionality regarding the  ``ontological interpretation", originally presented in  \cite{coecke2021compositionality}.}
%One key implication of Schr\"odinger compositionality is that ``a state can be more than its parts''.
In the following, we will focus on the special case of a state that can be ``more than its parts''.
This is arguably what makes composition interesting in quantum mechanics: it makes entanglement possible, which Schr\"odinger described as ``the characteristic trait of quantum mechanics'' \cite{Schrodinger_1935}.
In contrast with the example of open graphs, where the ``compositionality gap'' represents an obstacle to a computation strategy, here it can be seen as a positive feature.
Our approach can be used in both contexts; we will focus on the case study of non-separable states, recasting it as the failure of a lax functor to be strong.

In the context of monoidal categories,
%\footnote{Technically, thoughout the section we implicitly assume that our monoidal categories are strict. The example can easily be reworked for general monoidal categories by introducing unitors where it is suitable.}
a \emph{state} is a morphism $\TensorUnit \to A$, where $\TensorUnit$ is the monoidal unit.
We say that a state $\psi\colon \TensorUnit \to A \otimes B$ is \emph{separable} if there exist states $\psi_A\colon \TensorUnit \to A$ and $\psi_B\colon \TensorUnit \to B$ such that $\psi = \psi_A \otimes \psi_B$.%\footnote{Note that we are referring here to the notion of separability of pure states.}, or, graphically:
%
%
%\begin{equation*}
    %\begin{tikzpicture}
     %   \node[draw,thick,minimum width=2cm, minimum height=0.75cm] (phiAB) at (0,0) {$\psi$};
      %      \draw[thick] ($(phiAB.north) - (0.5,0)$) -- +(0,0.75);
       %     \draw[thick] ($(phiAB.north) + (0.5,0)$) -- +(0,0.75);
       % \node (equals) at (1.75,0) {$=$};
       % \node[draw,thick,minimum width=1cm, minimum height=2] (phiA) at (3,0) {$\psi_{A}$};
       %     \draw[thick] (phiA.north) -- +(0,0.75);
       % \node[draw,thick,minimum width=1cm, minimum height=2] (phiB) at (4.5,0) {$\psi_{B}$};
       %     \draw[thick] (phiB.north) -- +(0,0.75);
    %\end{tikzpicture}
%\end{equation*}
%

%
%
\begin{definition}
    Let $(\cat{C}, \otimes, \TensorUnit)$ be a monoidal category.
    The \emph{state functor} of $\cat{C}$ is the representable functor $\homset{\cat{C}}{\TensorUnit}{-}\colon \cat{C} \to \catset$. 
\end{definition}
%
\begin{proposition}[Laxity of the state functor]\label{prop: state functor lax}
    The state functor lifts to a lax monoidal functor from $(\cat{C}, \otimes, \TensorUnit)$ to $(\catset, \times, \{*\})$, with laxator components
    \begin{align*}
        \varphi_{A,B}\colon \homset{\CategoryC}{\TensorUnit}{A} \times \homset{\CategoryC}{\TensorUnit}{B}
            &\rightarrow 
            \homset{\CategoryC}{\TensorUnit}{A \otimes B}\\
        (\psi_A, \psi_B) 
            &\mapsto
            \psi_A \otimes \psi_B.
    \end{align*}
\end{proposition}
%
%
Recall that a monoidal category is \emph{semicartesian} if its monoidal unit is terminal.
The following result is a consequence of the general fact that a functor from a semicartesian to a cartesian monoidal category has a canonical oplax monoidal structure.
%
%
\begin{proposition}[Oplaxity of the state functor]\label{prop: state functor oplax}
    Let $(\cat{C}, \otimes, \Term)$ be a semicartesian category.
    Then the state functor lifts to an oplax monoidal functor from $(\cat{C}, \otimes, \Term)$ to $(\catset, \times, \{*\})$.
\end{proposition}
%
Clearly, there are cases where the state functor is not just lax or oplax, but strong.
The following result captures the well-known fact that in a cartesian monoidal category every state is separable.
%
\begin{proposition}[Strongness of the state functor]\label{prop: state functor strong}
    If $(\CategoryC, \times, \Term)$ is cartesian, then the state functor is strong monoidal.
\end{proposition}

Having turned Schr\"odinger compositionality into a question about (op)laxity of a functor, we can put our framework to good work.
By \autoref{prop: covariance natural transformation}, we have functors $\cat{C} \times \cat{C} \to \pointed{\catpos}$ sending pairs of objects $(A, B)$ of $\cat{C}$ to the homotopy posets 
\begin{equation} \label{eq: state_dhom }
    \dhom{i}{(\slice{\Set}{\homset{\cat{C}}{I}{A \otimes B}})}{\varphi_{A, B}}, \quad i \in \{ 0, 1 \}.
\end{equation}
Using the description of homotopy posets for slices of $\Set$ from \autoref{sec: obstructions}, we see that
\begin{itemize}
    \item minimal obstructions in $\pi_0$ are in bijection with non-separable states of $A \otimes B$,
    \item minimal obstructions in $\pi_1$ are in bijection with pairs of pairs of states $((\psi_A, \psi_B), (\chi_A, \chi_B))$ such that $\psi_A \otimes \psi_B = \chi_A \otimes \chi_B$.
\end{itemize}
For example, in $(\lcat{Vect}_\mathbb{C}, \otimes, \mathbb{C})$, the monoidal category of complex vector spaces with their tensor product, whenever $A$ and $B$ are at least 2-dimensional, we have instances of both:
\begin{itemize}
    \item the state $1 \mapsto \begin{pmatrix}1 \\ 0\end{pmatrix} \otimes \begin{pmatrix}1 \\ 0\end{pmatrix} + \begin{pmatrix}0 \\ 1\end{pmatrix} \otimes \begin{pmatrix}0 \\ 1\end{pmatrix}$ of $\mathbb{C}^2 \otimes \mathbb{C}^2$ is non-separable,
    \item given any pair of states $(\psi_A, \psi_B)$ and any non-zero $\lambda \in \mathbb{C}$, the pair $(\chi_A, \chi_B) \eqdef (\lambda \psi_A, \invrs{\lambda} \psi_B)$ satisfies $\psi_A \otimes \psi_B = \chi_A \otimes \chi_B$.
\end{itemize}
We can derive a few simple, immediate consequences from the covariance of (\ref{eq: state_dhom }) in the pair $(A, B)$.
\begin{enumerate}
    \item Given morphisms $f\colon A \to A'$, $g\colon B \to B'$, the induced maps of posets preserve the basepoint, that is, map ``non-obstructions'' to ``non-obstructions''.
    In this case, this implies that \emph{it is not possible to entangle a separable state by local actions}, that is, by applying morphisms on $A$ and $B$ separately.
    \item On the other hand, it is, in principle, possible for the induced maps to send non-trivial obstructions to the basepoint.
    For example, in complex vector spaces, acting on $A$ or $B$ with a rank-1 linear map always has a separating effect.
\end{enumerate}

    \section{Directions and Conclusions}
\label{sec:conclusions}

We have presented our initial exploration of \DV\ across a wide range of tasks and domains, providing supporting evidence to the claim that \DV's abilities are comparable to human-level for many of them. This conclusion is consistent with the findings by OpenAI presented in \cite{gpt4}. A primary goal of our experiments is to give a preliminary assessment of \DV's {\em intelligence}, which is an arduous task given the lack of formal definition for this concept, especially for artificial systems. We hope that our exploration provides a useful and necessary first step to appreciate the remarkable capabilities and challenges of {\DV}, and that it opens up new opportunities for developing more formal and comprehensive methods for testing and analyzing future AI systems with such broad intelligence. The capabilities of the model, which have been demonstrated above, both in terms of depth and generality, suggest that the machine learning community needs to move beyond classical benchmarking via structured datasets and tasks, and that the evaluation of the capabilities and cognitive abilities of those new models have become much closer in essence to the task of evaluating those of a human rather than those of a narrow AI model. We hope our investigation stimulates further research on {\DV} and similar systems, both in terms of exploring new applications and domains, and in terms of understanding the mechanisms and principles that underlie their intelligence.
\newline

The central claim of our work is that \DV\ attains a form of \emph{general} intelligence, indeed showing {\em sparks of artificial general intelligence}. This is demonstrated by its core mental capabilities (such as reasoning, creativity, and deduction), its range of topics on which it has gained expertise (such as literature, medicine, and coding), and the variety of tasks it is able to perform (e.g., playing games, using tools, explaining itself, ...). A lot remains to be done to create a system that could qualify as a complete AGI. We conclude this paper by discussing several immediate next steps, regarding defining AGI itself, building some of missing components in LLMs for AGI, as well as gaining better understanding into the origin of the intelligence displayed by the recent LLMs.

\subsection{Definitions of intelligence, AI, and AGI} \label{sec:otherdefinitions}
In this paper, we have used the 1994 definition of intelligence by a group of psychologists \cite{gottfredson1997mainstream} as a guiding framework to explore \DV's artificial intelligence. This definition captures some important aspects of intelligence, such as reasoning, problem-solving, and abstraction, but it is also vague and incomplete. It does not specify how to measure or compare these abilities. Moreover, it may not reflect the specific challenges and opportunities of artificial systems, which may have different goals and constraints than natural ones. Therefore, we acknowledge that this definition is not the final word on intelligence, but rather a useful starting point for our investigation. There is a rich and ongoing literature that attempts to propose more formal and comprehensive definitions of intelligence, artificial intelligence, and artificial general intelligence \cite{goertzel2014artificial, chollet2019measure}, but none of them is without problems or controversies.
For instance, Legg and Hutter \cite{legg2008machine} propose a goal-oriented definition of artificial general intelligence: Intelligence measures an agent’s ability to achieve goals in a wide range of environments. However, this definition does not necessarily capture the full spectrum of intelligence, as it excludes passive or reactive systems that can perform complex tasks or answer questions without any intrinsic motivation or goal. One could imagine as an artificial general intelligence, a brilliant oracle, for example, that has no agency or preferences, but can provide accurate and useful information on any topic or domain. Moreover, the definition around achieving goals in a wide range of environments also implies a certain degree of universality or optimality, which may not be realistic (certainly human intelligence is in no way universal or optimal). The need to recognize the importance of priors (as opposed to {\em universality}) was emphasized in the definition put forward by Chollet in \cite{chollet2019measure} which centers intelligence around skill-acquisition efficiency, or in other words puts the emphasis on a single component of the 1994 definition: learning from experience (which also happens to be one of the key weaknesses of LLMs). Another candidate definition of artificial general intelligence from Legg and Hutter \cite{legg2007universal} is: a system that can do anything a human can do. However, this definition is also problematic, as it assumes that there is a single standard or measure of human intelligence or ability, which is clearly not the case. Humans have different skills, talents, preferences, and limitations, and there is no human that can do everything that any other human can do. Furthermore, this definition also implies a certain anthropocentric bias, which may not be appropriate or relevant for artificial systems. While we do not adopt any of those definitions in the paper, we recognize that they provide important angles on intelligence. For example, whether intelligence can be achieved without any agency or intrinsic motivation is an important philosophical question. Equipping LLMs with agency and intrinsic motivation is a fascinating and important direction for future work. With this direction of work, great care would have to be taken on alignment and safety per a system's abilities to take autonomous actions in the world and to perform autonomous self-improvement via cycles of learning. We discuss a few other crucial missing components of LLMs next.

\subsection{On the path to more general artificial intelligence}
%We have provided evidence supporting that claim that {\DV} performance on a wide range of tasks is comparable to human-level abilities. We have argued that the model attains a form of \emph{general} intelligence in terms of core mental capabilities (such as reasoning, creativity, and deduction), in terms of the range of topics on which is has gained expertise (such as literature, medicine, and coding), and in terms of the variety of tasks it is able to perform (e.g., playing games, using tools, explaining itself, ...). We have also shown that {\DV} can generate and understand content that combines different topics, skills, and modalities, demonstrating its flexibility and creativity and that, despite being trained purely on text, it demonstrates remarkable capabilities in a variety of modalities such as vision. We have compared {\DV}'s performance to those of previous large language models (LLMs), most notably ChatGPT \cite{gpt3}, and we have found that {\DV} is far superior in terms of generality, creativity, and closeness to human-level intelligence. 

%As we allude to in the title of the paper, this work explores a ``first contact" with {\DV} and its potential descendants, rather than a comprehensive evaluation of the model's intelligence. We hope that our exploration provides a useful and necessary first step to appreciate the remarkable capabilities and challenges of {\DV}, and that it opens up new opportunities for developing more formal and comprehensive methods for testing and analyzing future AGI systems. The capabilities of the model, which have been demonstrated above, both in terms of depth and generality, suggest that the machine learning community needs to move beyond classical benchmarking via structured datasets and tasks, and that the evaluation of the capabilities and cognitive abilities of those new models have become much closer in essence to the task of evaluating those of a human rather than those of a narrow AI model. We hope our investigation stimulates further research on {\DV} and similar systems, both in terms of exploring new applications and domains, and in terms of understanding the mechanisms and principles that underlie their intelligence.

%We have also identified some of the main drawbacks of \DV, and we have discussed how they might be addressed in future work. These drawbacks include:
Some of the areas where \DV\ (and LLMs more generally) should be improved to achieve more general intelligence include (note that many of them are interconnected):
\begin{itemize}
    \item \textbf{Confidence calibration:} The model has trouble knowing when it should be confident and when it is just guessing. It both makes up facts that have not appeared in its training data, and also exhibits inconsistencies between the generated content and the prompt, which we referred to as {\em open-domain} and {\em closed-domain} hallucination in Figure \ref{fig:hallucination}. These hallucinations can be stated in a confident and persuasive manner that can be difficult to detect. Thus, such generations can lead to errors, and also to confusion and mistrust. While hallucination is a good thing when generating creative content, reliance on factual claims made by a model with hallucinations can be costly, especially for uses in high-stakes domains such as healthcare. There are several complementary ways to attempt to address hallucinations. One way is to improve the calibration of the model (either via prompting or fine-tuning) so that it either abstains from answering when it is unlikely to be correct or provides some other indicator of confidence that can be used downstream. Another approach, that is suitable for mitigating open-domain hallucination, is to insert information that the model lacks into the prompt, for example by allowing the model to make calls to external sources of information, such as a search engine as in Section \ref{sec:affordances}. For closed-domain hallucination the use of additional model computation through post-hoc checks is also promising, see Figure \ref{fig:hallucination} for an example. Finally, building the user experience of an application with the possibility of hallucinations in mind can also be part of an effective mitigation strategy. %Other directions include developing and refining mechanisms that endow systems with well-calibrated likelihoods that its generations are grounded, or, more directly, the likelihood that it is hallucinating versus relying upon and communicating content that it has learned from its training data. 
    \item \textbf{Long-term memory:} The model's context is very limited (currently 8000 tokens, but not scalable in terms of computation), it operates in a ``stateless" fashion and there is no obvious way to teach the model new facts. In fact, it is not even clear whether the model is able to perform tasks which require an evolving memory and context, such as reading a book, with the task of following the plot and understanding references to prior chapters over the course of reading.
    \item \textbf{Continual learning:} The model lacks the ability to update itself or adapt to a changing environment. The model is fixed once it is trained, and there is no mechanism for incorporating new information or feedback from the user or the world. One can fine-tune the model on new data, but this can cause degradation of performance or overfitting. Given the potential lag between cycles of training, the system will often be out of date when it comes to events, information, and knowledge that came into being after the latest cycle of training.
    \item \textbf{Personalization:} Some of the applications require the model to be tailored to a specific organization or end user. The system may need to acquire knowledge about the workings of an organization or the preferences of an individual. And in many cases, the system would need to adapt in a personalized manner over periods of time with specific changes linked to the dynamics of people and organizations. For example, in an educational setting, there would be an expectation of the need for the system to understand particular learning styles as well as to adapt over time to a student's progress with comprehension and prowess. The model does not have any way to incorporate such personalized information into its responses, except by using meta-prompts, which are both limited and inefficient. 
    \item \textbf{Planning and conceptual leaps:} As suggested by the examples in Section \ref{sec:limitations}, the model exhibits difficulties in performing tasks that require planning ahead or that require a ``Eureka idea" constituting a discontinuous conceptual leap in the progress towards completing a task. In other words, the model does not perform well on tasks that require the sort of conceptual leaps of the form that often typifies human genius.  
    \item \textbf{Transparency, interpretability and consistency:} Not only does the model hallucinate, make up facts and produce inconsistent content, but it seems that the model has no way of verifying whether or not the content that it produces is consistent with the training data, or whether it's self-consistent. While the model is often able to provide high-quality post-hoc explanations for its decisions (as demonstrated in Section \ref{sec:explainability}), using explanations to verify the process that led to a certain decision or conclusion only works when that process is accurately modeled and a sufficiently powerful explanation process is also accurately modeled (Section \ref{sec:explainability}). Both of these conditions are hard to verify, and when they fail there are inconsistencies between the model's decisions and its explanations. Since the model does not have a clear sense of its own limitations it makes it hard to establish trust or collaboration with the user without extensive experimentation in a narrow domain.
    \item \textbf{Cognitive fallacies and irrationality:} The model seems to exhibit some of some of the limitations of human knowledge and reasoning, such as cognitive biases and irrationality (such as biases of confirmation, anchoring, and base-rate neglect) and statistical fallacies. The model may inherit some of the biases, prejudices, or errors that are present in its training data, which may reflect the distribution of opinions or perspectives linked to subsets of the population or  larger common views and assessments. 
     \item \textbf{Challenges with sensitivity to inputs:} The model's responses can be very sensitive to details of the framing or wording of prompts and their sequencing in a session. Such non-robustness suggests that significant effort and experimentation is often required with engineering prompts and their sequencing and that uses in the absence of such investments of time and effort by people can lead to suboptimal and non-aligned inferences and results. 
\end{itemize}

A limitation of our exploration is the absence of a clear distinction between drawbacks founded in the way that the reinforcement learning step (RLHF) was carried out, versus drawbacks which are fundamentally inherent in the larger architecture and methodology. For example, it is not clear to what extent the hallucination problem can be addressed via a refined reinforcement learning step or via a focused effort to introduce new forms of calibration about the likelihoods of the veracity of alternative inferences that the system can compute and consider in its generations (see also \cite{gpt4} for more discussion on this). To draw an analogy to humans, cognitive biases and irrational thinking may be based in artifacts of our culture as well as to limitations in our cognitive capabilities. Pursuing better understandings of the sources and potential solutions to challenges of hallucination in \DV, will benefit from studies that compare several versions of the RL stage over the same architecture.
\newline

A broader question on the identified limitations is: which of the aforementioned drawbacks can be mitigated within the scope of next word prediction? Is it simply the case that a bigger model and more data will fix those issues, or does the architecture need to be modified, extended, or reformulated? Potential extensions to next word prediction include the following:

\begin{itemize}
    \item External calls by the model to components and tools such as a calculator, a database search or code execution, as suggested in Section \ref{sec:affordances}. 
    \item A richer, more complex ``slow-thinking" deeper mechanism that oversees the ``fast-thinking" mechanism of next word prediction. Such an approach could allow the model to perform long-term planning, exploration, or verification, and to maintain a working memory or a plan of action. The slow-thinking mechanism would use the next word prediction model as a subroutine, but it would also have access to external sources of information or feedback, and it would be able to revise or correct the outputs of the fast-thinking mechanism.
    \item Integration of long-term memory as an inherent part of the architecture, perhaps in the sense that both the input and output of the model will include, in addition to the tokens representing the text, a vector which represents the context.
    \item Going beyond single-word prediction: Replacing the sequence of tokens by a hierarchical structure, where higher-level parts of the text such as sentences, paragraphs or ideas are represented in the embedding and where the content is generated in a top-down manner. It is unclear whether richer predictions about the sequencing and interdependency of such higher-level concepts might emerge from large-scale compute and data centered on a next-word--prediction paradigm.
\end{itemize}

%In conclusion, we have worked to demonstrate that {\DV} has remarkable capabilities that challenge many of the recent assumptions and expectations within the AI community. We have also shown that {\DV} is by no means a perfect or complete AGI system, and that it has many limitations and biases that need to be addressed and understood. We hope that our exploration will inspire and inform further research on {\DV} and similar systems, both in terms of exploring their potential applications and domains, and in terms of understanding foundational mechanisms and potentially with identifying principles of intelligence. We believe that {\DV} represents a paradigm shift in the field of computer science and beyond, and that the model and its capabilities frame new questions, possibilities, and horizons for the field and for the advancement of human capabilities and well-being.

\subsection{What is actually happening?} \label{sec:whatsgoingon}
Our study of {\DV} is entirely phenomenological: We have focused on the surprising things that {\DV} can do, but we do not address the fundamental questions of why and how it achieves such remarkable intelligence. How does it reason, plan, and create? Why does it exhibit such general and flexible intelligence when it is at its core merely the combination of simple algorithmic components---gradient descent and large-scale transformers with extremely large amounts of data? These questions are part of the mystery and fascination of LLMs, which challenge our understanding of learning and cognition, fuel our curiosity, and motivate deeper research. Key directions include ongoing research on the phenomenon of emergence in LLMs (see \cite{wei2022emergent} for a recent survey). Yet, despite intense interest in questions about the capabilities of LLMs, progress to date has been quite limited with only toy models where some phenomenon of emergence is proved \cite{barak2022hidden, ahn2022learning,jelassi2022vision}. One general hypothesis \cite{olah2020zoom} is that the large amount of data (especially the  diversity of the content) forces neural networks to learn generic and useful ``neural circuits'', such as the ones discovered in \cite{olsson2022context, zhang2022unveiling, liu2022transformers}, while the large size of models provide enough redundancy and diversity for the neural circuits to specialize and fine-tune to specific tasks. 
%The Mixture of Experts (MoE) layers in modern LLMs can also contribute to the generality of the model~\cite{chen2022towards}. 
Proving these hypotheses for large-scale models remains a challenge, and, moreover, it is all but certain that the conjecture is only part of the answer. On another direction of thinking, the huge size of the model could have several other benefits, such as making gradient descent more effective by connecting different minima \cite{venturi2019spurious} or by simply enabling smooth fitting of high-dimensional data \cite{pmlr-v49-eldan16, NEURIPS2021_f197002b}. Overall, elucidating the nature and mechanisms of AI systems such as {\DV} is a formidable challenge that has suddenly become important and urgent.
\newline

\paragraph{Acknowledgements.} We thank OpenAI for creating such a marvelous tool and giving us early access to experience it. We also thank Miles Brundage at OpenAI, and the numerous people at Microsoft, who have provided thoughtful feedback on this work.

	\nocite{*}
	\bibliographystyle{eptcs}
	\bibliography{main}

    \begin{partbacktext}
\part{Appendices}
\end{partbacktext}

\appendix

\chapter{Collections}
\label{appendix:collections}

\abstract{This appendix gives a description of the vector collections used in experiments
throughout this monograph. These collections demonstrate different operating points in
a typical use-case. For example, some consist of dense vectors, others of sparse vectors;
some have few dimensions and others are in much higher dimensions; some are relatively small
while others contain a large number of points.}

\bigskip

Table~\ref{table:appendix:collections:dense} gives a description of the dense vector collections
used throughout this monograph and summarizes their key statistics.

\begin{table*}[ht]
\caption{Dense collections used in this monograph along with select statistics.}
\scriptsize
\label{table:appendix:collections:dense}
\begin{center}
\begin{sc}
\begin{tabular}{p{5cm}|ccc}
\toprule
Collection & Vector Count & Query Count & Dimensions \\
\midrule
\textsc{GloVe}-$25$~\citep{pennington-etal-2014-glove} & $1.18$M & $10{,}000$ & $25$ \\
\textsc{GloVe}-$50$ & $1.18$M & $10{,}000$ & $50$ \\
\textsc{GloVe}-$100$ & $1.18$M & $10{,}000$ & $100$ \\
\textsc{GloVe}-$200$ & $1.18$M & $10{,}000$ & $200$ \\
\textsc{Deep1b}~\citep{deep1b} & $9.99$M & $10{,}000$ & $96$ \\
\textsc{MS Turing}~\citep{msturingDataset} & $10$M & $100{,}000$ & $100$ \\
\textsc{Sift}~\citep{Lowe2004DistinctiveIF} & $1$M & $10{,}000$ & $128$ \\
\textsc{Gist}~\citep{Oliva2001ModelingTS} & $1$M & $1{,}000$ & $960$ \\
\bottomrule
\end{tabular}
\end{sc}
\end{center}
\end{table*}

In addition to the vector collections above, we convert a few text collections
into vectors using various embedding models. These collections are described in
Table~\ref{table:appendix:collections:text}. Please see~\citep{nguyen2016msmarco} for
a complete description of the MS MARCO v1 collection and~\citep{thakur2021beir} for the others.

\begin{table*}[ht]
\caption{Text collections along with key statistics.
The rightmost two columns report the average number of non-zero
entries in data points and, in parentheses, queries for sparse vector
representations of the collections.}
\scriptsize
\label{table:appendix:collections:text}
\begin{center}
\begin{sc}
\begin{tabular}{c|cc|cc}
\toprule
Collection & Vector Count & Query Count & \splade{} & \esplade{}\\
\midrule
\textsc{MS Marco} Passage& $8.8$M & $6{,}980$ & 127 (49) & 185 (5.9) \\
NQ & $2.68$M & $3{,}452$ & 153 (51) & 212 (8) \\
\textsc{Quora} & $523$K & $10{,}000$ & 68 (65) & 68 (8.9) \\
\textsc{HotpotQA} & $5.23$M & $7{,}405$ & 131 (59) & 125 (13) \\
\textsc{Fever} & $5.42$M & $6{,}666$ & 145 (67) & 140 (8.6) \\
\textsc{DBPedia} & $4.63$M & $400$ & 134 (49) & 131 (5.9) \\
\bottomrule
\end{tabular}
\end{sc}
\end{center}
\end{table*}

When transforming the text collections of Table~\ref{table:appendix:collections:text}
into vectors, we use the following embedding models:
\begin{itemize}
    \item \textsc{AllMiniLM-l6-v2}:\footnote{Available at \url{https://huggingface.co/sentence-transformers/all-MiniLM-L6-v2}}
    Projects text documents into $384$-dimensional dense vectors for retrieval with angular distance.

    \item \textsc{Tas-B}~\citep{tas-b}: A bi-encoder model that was trained using supervision from a cross-encoder and a ColBERT~\citep{colbert2020khattab} model,
    and produces $768$-dimensional dense vectors that are meant for MIPS.
    The checkpoint used in this work is available on HuggingFace.\footnote{Available at \url{https://huggingface.co/sentence-transformers/msmarco-distilbert-base-tas-b}}

    \item \splade{}~\citep{formal2022splade}:\footnote{Pre-trained checkpoint from HuggingFace available at \url{https://huggingface.co/naver/splade-cocondenser-ensembledistil}}
    Produces sparse representations for text.
    The vectors have roughly $30{,}000$ dimensions, where each dimension corresponds
    to a term in the BERT~\citep{devlin2019bert} WordPiece~\citep{wordpiece} vocabulary.
    Non-zero entries in a vector reflect learnt term importance weights.

    \item \esplade{}~\citep{lassance2022sigir}:\footnote{Pre-trained checkpoints for document and
    query encoders were obtained from \url{https://huggingface.co/naver/efficient-splade-V-large-doc} and \url{https://huggingface.co/naver/efficient-splade-V-large-query},
    respectively.}
    This model produces queries that have far fewer non-zero entries than the original
    \splade{} model, but documents that may have a larger number of non-zero entries.
\end{itemize}

\bibliographystyle{abbrvnat}
\bibliography{biblio}


\chapter{Probability Review}
\label{appendix:probability}

\abstract{We briefly review key concepts in probability in this appendix.}

\section{Probability}
We identify a \emph{probability space} denoted by $(\Omega, \mathcal{F}, \probability)$
with an \emph{outcome space}, an \emph{events} set, and a \emph{probability measure}.
The outcome space, $\Omega$, is the set of all
possible outcomes. For example, when flipping a two-sided coin, the outcome
space is simply $\{0, 1\}$. When rolling a six-sided die, it is instead
the set $[6] = \{ 1, 2, \ldots, 6\}$.

The events set $\mathcal{F}$ is a set of subsets of $\Omega$ that
includes $\Omega$ as a member and is closed under complementation and
countable unions. That is, if $E \in \mathcal{F}$,
then we must have that $E^\complement \mathcal{F}$.
Furthermore, the union of countably many events $E_i$'s
in $\mathcal{F}$ is itself in $\mathcal{F}$: $\cup_i E_i \in \mathcal{F}$.
A set $\mathcal{F}$ that satisfies these properties is called a $\sigma$-algebra.

Finally, a function $\probability: \mathcal{F} \rightarrow \mathbb{R}$ is
a probability measure if it satisfies the following conditions: $\probability[\Omega] = 1$;
$\probability[E] \geq 0$ for any event $E \in \mathcal{F}$;
$\probability[E^\complement] = 1 - \probability[E]$; and, finally,
for countably many disjoint events $E_i$'s:
$\probability[\cup_i E_i] = \sum_i \probability[E_i]$.

We should note that, $\probability$ is also known as a ``probability distribution''
or simply a ``distribution.'' The pair $(\Omega, \mathcal{F})$ is called
a \emph{measurable space}, and the elements of $\mathcal{F}$ are
known as a \emph{measurable sets}. The reason they are called ``measurable''
is because they can be ``measured'' with $\probability$: The function
$\probability$ assigns values to them.

In many of the discussions throughout this monograph, we omit the outcome space
and events set because that information is generally clear from context.
However, a more formal treatment of our arguments requires a complete
definition of the probability space.

\section{Random Variables}
A random variable on a measurable space $(\Omega, \mathcal{F})$ is
a measurable function $X: \Omega \rightarrow \mathbb{R}$.
It is measurable in the sense that the \emph{preimage} of any Borel set $B \in \mathcal{B}$
is an event: $X^{-1}(B) = \{ \omega \in \Omega \;|\; X(\omega) \in B \} \in \mathcal{F}$.

A random variable $X$ generates a $\sigma$-algebra that comprises of the preimage
of all Borel sets. It is denoted by $\sigma(X)$
and formally defined as $\sigma(X) = \{ X^{-1}(B) \;|\; B \in \mathcal{B} \}$.

\bigskip

Random variables are typically categorized as discrete or continuous.
$X$ is \emph{discrete} when it maps $\Omega$ to a discrete set.
In that case, its \emph{probability mass function} is defined as $\probability[X = x]$
for some $x$ in its range.
A \emph{continuous} random variable is often associated with a
probability \emph{density} function, $f_X$, such that:
\begin{equation*}
    \probability[a \leq X \leq b] = \int_a^b f_X(x) dx.
\end{equation*}

Consider, for instance, the following probability density function over the real line for
parameters $\mu \in \mathbb{R}$ and $\sigma > 0$:
\begin{equation*}
    f(x) = \frac{1}{\sqrt{2 \pi \sigma^2}} e^{- \frac{(x - \mu)^2}{2\sigma^2}}.
\end{equation*}
A random variable with the density function above is said to follow a Gaussian
distribution with mean $\mu$ and variance $\sigma^2$, denoted by $X \sim \mathcal{N}(\mu, \sigma^2)$.
When $\mu = 0$ and $\sigma^2 = 1$, the resulting distribution is called the standard
Normal distribution.

Gaussian random variables have attractive properties.
For example, the sum of two independent Gaussian random variables is itself a Gaussian variable.
Concretely, $X_1 \sim \mathcal{N}(\mu_1, \sigma_1^2)$ and $X_2 \sim \mathcal{N}(\mu_2, \sigma_2^2)$,
then $X_1 + X_2 \sim \mathcal{N}(\mu_1 + \mu_2, \sigma_1^2 + \sigma_2^2)$.
The sum of the squares of $m$ independent Gaussian random variables, on the other hand,
follows a $\chi^2$-distribution with $m$ degrees of freedom.

\section{Conditional Probability}
Conditional probabilities give us a way to model how the probability of an event changes
in the presence of extra information, such as partial knowledge about a random outcome.
Concretely, if $(\Omega, \mathcal{F}, \probability)$ is a probability space and
$A, B \in \mathcal{F}$ such that $\probability[B] > 0$, then the \emph{conditional
probability} of $A$ given the event $B$ is denoted by $\probability[A \;\lvert\; B]$ and
defined as follows:
\begin{equation*}
    \probability[A \;\lvert\; B] = \frac{\probability[A \cap B]}{\probability[B]}.
\end{equation*}

We use a number of helpful results concerning conditional probabilities
in proofs throughout the monograph. One particularly useful inequality
is what is known as the \emph{union bound} and is stated as follows:
\begin{equation*}
    \probability[\cup_i A_i] \leq \sum_i \probability[A_i].
\end{equation*}

Another fundamental property is the law of total probability.
It states that, for mutually disjoint events $A_i$'s such that
$\Omega = \cup A_i$, the probability of any event $B$ can be expanded
as follows:
\begin{equation*}
    \probability[B] = \sum_i \probability[B \;\lvert\; A_i] \probability[A_i].
\end{equation*}
This is easy to verify: the summand is by definition equal to $\probability[B \cap A_i]$
and, considering the events $(B \cap A_i)$'s are mutually disjoint, their sum
is equal to $\probability[B \cap (\cup A_i)] = \probability[B]$.


\section{Independence}
Another tool that reflects the effect (or lack thereof) of additional knowledge on probabilities
is the concept of \emph{independence}. Two events $A$ and $B$ are said to be
\emph{independent} if $\probability[A \cap B] = \probability[A] \times \probability[B]$.
Equivalently, we say that $A$ is independent of $B$ if and only if
$\probability[A \;\lvert\; B] = \probability[A]$ when $\probability[B] > 0$.

\bigskip

Independence between two random variables is defined similarly but requires a bit more care.
If $X$ and $Y$ are two random variables and $\sigma(X)$ and $\sigma(Y)$ denote
the $\sigma$-algebras generated by them, then $X$ is independent of $Y$ if
all events $A \in \sigma(X)$ and $B \in \sigma(Y)$ are independent.

When a sequence of random variables are \emph{mutually} independent and are drawn
from the same distribution (i.e., have the same probability density function),
we say the random variables are drawn \emph{iid}: independent and identically-distributed.
We stress that \emph{mutual} independence is a stronger restriction than
\emph{pairwise} independence: $m$ events $\{ E_i \}_{i=1}^m$ are mutually independent if
$\probability[\cap_i E_i] = \prod_i \probability[E_i]$.

We typically assume that data and query points are drawn \emph{iid} from some
(unknown) distribution. This is a standard and often necessary assumption
that eases analysis.

\section{Expectation and Variance}

The \emph{expected value} of a discrete random variable $X$ is denoted by $\ev[X]$
and defined as follows:
\begin{equation*}
    \ev[X] = \sum_x x \probability[X = x].
\end{equation*}
When $X$ is continuous, its expected value is based on the following Lebesgue integral:
\begin{equation*}
    \ev[X] = \int_{\Omega} X d \probability.
\end{equation*}
So when a random variable has probability density function $f_X$, its expected value
becomes:
\begin{equation*}
    \ev[X] = \int x f_X(x) dx.
\end{equation*}

For a \emph{nonnegative} random variable $X$, it is sometimes more convenient to
unpack $\ev{X}$ as follows instead:
\begin{equation*}
    \ev[X] = \int_0^\infty \probability[X > x] dx.
\end{equation*}

A fundamental property of expectation is that it is a linear operator.
Formally, $\ev[X + Y] = \ev[X] + \ev[Y]$ for two random variables $X$ and $Y$.
We use this property often in proofs.

We state another important property for independent random variables
that is easy to prove.
If $X$ and $Y$ are independent, then $\ev[XY] = \ev[X]\ev[Y]$.

\bigskip

The \emph{variance} of a random variable is defined as follows:
\begin{equation*}
    \var[X] = \ev\Big[ (X - \ev[X])^2 \Big] = \ev[X]^2 - \ev[X^2].
\end{equation*}
Unlike expectation, variance is not linear unless the random variables involved
are independent. It is also easy to see that $\var[aX] = a^2 \var[X]$ for a
constant $a$.

\section{Central Limit Theorem}
The result known as the Central Limit Theorem is one of the most
useful tools in probability. Informally, it states that the average of \emph{iid}
random variables with finite mean and variance converges to a Gaussian distribution.
There are several variants of this result that extend the claim to, for example,
independent but not identically distributed variables. Below we repeat the formal
result for the \emph{iid} case.

\begin{theorem}
    Let $X_i$'s be a sequence of $n$ \emph{iid} random variables with finite mean $\mu$
    and variance $\sigma^2$. Then, for any $x \in \mathbb{R}$:
    \begin{equation*}
        \lim_{n \rightarrow \infty} \probability \Big[
            \underbrace{\frac{(1/n \sum_{i=1}^n X_i) - \mu}{\sigma^2/n}}_Z \leq x
        \Big] = \int_{-\infty}^x \frac{1}{\sqrt{2 \pi}} e^{-\frac{t^2}{2}} dt,
    \end{equation*}
    implying that $Z \sim \mathcal{N}(0, 1)$.
\end{theorem}

\chapter{Concentration of Measure}
\label{appendix:measure}

\abstract{
By the strong law of large numbers, we know that the average of a sequence
of $m$ \emph{iid} random variables with mean $\mu$ converges to $\mu$ with
probability $1$ as $m$ tends to infinity. But how far is that average from
$\mu$ when $m$ is finite? Concentration inequalities helps us answer that question
quantitatively. This appendix reviews important inequalities that are used
in the proofs and arguments throughout this monograph.
}

\section{Markov's Inequality}

\begin{lemma}
    \label{lemma:appendix:concentration:markov}
    For a nonnegative random variable $X$ and a nonnegative constant $a \geq 0$:
    \begin{equation*}
        \probability[X \geq a] \leq \frac{\ev[X]}{a}.
    \end{equation*}
\end{lemma}
\begin{proof}
    Recall that the expectation of a nonnegative random variable $X$ can be written
    as:
    \begin{equation*}
        \ev[X] = \int_0^\infty \probability[X \geq x] dx.
    \end{equation*}
    Because $\probability[X \geq x]$ is monotonically nonincreasing, we can expand
    the above as follows to complete the proof:
    \begin{equation*}
        \ev[X] \geq \int_0^a \probability[X \geq x] dx \geq \int_0^a \probability[X \geq a] dx = a \probability[X \geq a].
    \end{equation*}
\end{proof}

\section{Chebyshev's Inequality}

\begin{lemma}
    \label{lemma:appendix:concentration:chebyshev}
    For a random variable $X$ and a constant $a > 0$:
    \begin{equation*}
        \probability \Big[ \big\lvert X - \ev[X] \big\rvert \geq a \Big] \leq \frac{\var[X]}{a^2}.
    \end{equation*}
\end{lemma}
\begin{proof}
    \begin{equation*}
        \probability \Big[ \big\lvert X - \ev[X] \big\rvert \geq a \Big] =
        \probability \Big[ \big( X - \ev[X] \big)^2 \geq a^2 \Big] \leq \frac{\var[X]}{a^2},
    \end{equation*}
    where the last step follows by the application of Markov's inequality.
\end{proof}

\begin{lemma}
    Let $\{ X_i \}_{i=1}^n$ be a sequence of iid random variables
    with mean $\mu < \infty$ and variance $\sigma^2 < \infty$. For $\delta \in (0, 1)$,
    with probability $1 - \delta$:
    \begin{equation*}
        \Big\lvert \frac{1}{n} \sum_{i = 1}^n X_i - \mu \Big\rvert \leq \sqrt{\frac{\sigma^2}{\delta n}}.
    \end{equation*}
\end{lemma}
\begin{proof}
    By Lemma~\ref{lemma:appendix:concentration:chebyshev}, for any $a > 0$:
    \begin{equation*}
        \probability \Bigg[ \Big\lvert \frac{1}{n}\sum_{i=1}^n X_i - \mu \Big\rvert \geq a \Bigg]
        \leq \frac{\sigma^2/n}{a^2}.
    \end{equation*}
    Setting the right-hand-side to $\delta$, we obtain:
    \begin{equation*}
        \frac{\sigma^2}{n a^2} = \delta \implies a = \sqrt{\frac{\sigma^2}{\delta n}},
    \end{equation*}
    which completes the proof.
\end{proof}

\section{Chernoff Bounds}

\begin{lemma}
    Let $\{ X_i \}_{i=1}^n$ be independent Bernoulli variables with success probability $p_i$.
    Define $X = \sum_i X_i$ and $\mu = \ev[X] = \sum_i p_i$. Then:
    \begin{equation*}
        \probability \Big[ X > (1 + \delta) \mu \Big] \leq e^{-h(\delta) \mu},
    \end{equation*}
    where,
    \begin{equation*}
        h(t) = (1 + t) \log(1 + t) - t.
    \end{equation*}
\end{lemma}
\begin{proof}
    Using Markov's inequality of Lemma~\ref{lemma:appendix:concentration:markov}
    we can write the following for any $t > 0$:
    \begin{equation*}
        \probability\Big[ X > (1 + \delta)\mu \Big] =
            \probability\Big[ e^{tX} > e^{t(1 + \delta)\mu} \Big] \leq
            \frac{\ev\big[ e^{tX} \big]}{e^{t (1 + \delta) \mu}}.
    \end{equation*}
    Expanding the expectation, we obtain:
    \begin{align*}
        \ev\big[e^{tX}\big] &= \ev\Big[ e^{t \sum_i X_i} \Big] = \ev\Big[ \prod_i e^{tX_i} \Big]
        = \prod_i \ev[e^{tX_i}] \\
        &= \prod_i \Big( p_i e^t + (1 - p_i) \Big) \\
        &= \prod_i \big( 1 + p_i (e^t - 1) \big) \\
        &\leq \prod_i e^{p_i(e^t - 1)} = e^{(e^t - 1)\mu}. && \text{by $(1 + t \leq e^t)$} \\
    \end{align*}
    Putting all this together gives us:
    \begin{equation}
        \label{equation:appendix:concentration:chernoff:proof}
        \probability\Big[ X > (1 + \delta)\mu \Big] \leq 
        \frac{e^{(e^t - 1) \mu}}{e^{t (1 + \delta) \mu}}.
    \end{equation}
    This bound holds for any value $t > 0$, and in particular a value of $t$ that
    minimizes the right-hand-side. To find such a $t$, we may differentiate
    the right-hand-side, set it to $0$, and solve for $t$ to obtain:
    \begin{align*}
        \frac{\mu e^t e^{(e^t - 1) \mu}}{e^{t (1 + \delta) \mu}} &-
        \mu ( 1 + \delta ) \frac{e^{(e^t - 1) \mu}}{e^{t (1 + \delta) \mu}} = 0 \\
        &\implies \mu e^t = \mu (1 + \delta) \\
        &\implies t = \log(1 + \delta).
    \end{align*}
    Substituting $t$ into Equation~(\ref{equation:appendix:concentration:chernoff:proof})
    gives the desired result.
\end{proof}

\section{Hoeffding's Inequality}

We need the following result, known as Hoeffding's Lemma, to present
Hoeffding's inequality.

\begin{lemma}
    \label{lemma:appendix:concentration:hoeffding-lemma}
    Let $X$ be a zero-mean random variable that takes values in $[a, b]$.
    For any $t > 0$:
    \begin{equation*}
        \ev\big[ e^{tX} \big] \leq \exp\Big( \frac{t^2 (b - a)^2}{8} \Big).
    \end{equation*}
\end{lemma}
\begin{proof}
    By convexity of $e^{tx}$ and given $x \in [a, b]$ we have that:
    \begin{equation*}
        e^{tx} \leq \frac{b - x}{b - a} e^{ta} +
            \frac{x - a}{b - a} e^{tb}.
    \end{equation*}
    Taking the expectation of both sides, we arrive at:
    \begin{equation*}
        \ev\Big[e^{tx}\Big] \leq
            \frac{b}{b - a} e^{ta} - \frac{a}{b - a} e^{tb}.
    \end{equation*}
    To conclude the proof, we first write the right-hand-side as
    $\exp(h(t(b - a)))$ where:
    \begin{equation*}
        h(x) = \frac{a}{b - a} x + \log \Big( \frac{b}{b - a} - \frac{a}{b - a} e^x \Big).
    \end{equation*}
    By expanding $h(x)$ using Taylor's theorem, it can be shown that
    $h(x) \leq x^2/8$. That completes the proof.
\end{proof}

We are ready to present Hoeffding's inequality.

\begin{lemma}
    Let $\{ X_i \}_{i=1}^n$ be a sequence of iid random variables
    with finite mean $\mu$ and suppose $X_i \in [a, b]$ almost surely.
    For all $\epsilon > 0$:
    \begin{equation*}
        \probability\Bigg[ \Big\lvert \frac{1}{n} \sum_{i=1}^n X_i - \mu \Big\rvert > \epsilon \Bigg] \leq 2 \exp\Big({-\frac{2n \epsilon^2}{(b - a)^2}}\Big).
    \end{equation*}
\end{lemma}
\begin{proof}
    Let $X = 1/n \sum_i X_i - \mu$. Observe by Markov's inequality that:
    \begin{equation*}
        \probability[X \geq \epsilon] = \probability\Big[ e^{tX} \geq e^{t\epsilon} \Big]
        \leq e^{-t\epsilon} \ev[e^{tX}].
    \end{equation*}
    By independence of $X_i$'s and
    the application of Lemma~\ref{lemma:appendix:concentration:hoeffding-lemma}:
    \begin{align*}
        \ev[e^{tX}] &= \ev \Bigg[ \prod_i e^\frac{t(X_i - \mu)}{n} \Bigg] \\
        &= \prod_i \ev \Big[ e^{\frac{t(X_i-\mu)}{n}} \Big] \\
        &\leq \prod_i \exp\Big( \frac{t^2 (b - a)^2}{8 n^2} \Big) \\
        &= \exp\Big( \frac{t^2 (b - a)^2}{8 n} \Big).
    \end{align*}
    We have shown that:
    \begin{equation*}
        \probability[X \geq \epsilon] \leq \exp\Big( -t \epsilon + \frac{t^2 (b - a)^2}{8 n} \Big).
    \end{equation*}
    That statement holds for all values of $t$ and in particular one that minimizes
    the right-hand-side. Solving for that value of $t$ gives us
    $t = 4n\epsilon / (b - a^2)$, which implies:
    \begin{equation*}
        \probability[X \geq \epsilon] \leq e^{-\frac{2n \epsilon^2}{(b - a)^2}}.
    \end{equation*}
    By a symmetric argument we can bound $\probability[X \leq -\epsilon]$. The claim
    follows by the union bound over the two cases.
\end{proof}

\section{Bennet's Inequality}

\begin{lemma}
    Let $\{ X_i \}_{i=1}^n$ be a sequence of independent random variables with zero mean
    and finite variance $\sigma_i^2$. Assume that $\lvert X_i \rvert \leq a$ almost surely for all $i$. Then:
    \begin{equation*}
        \probability\Big[\sum_i X_i \geq t \Big] \leq 
        \exp \Bigg( -\frac{\sigma^2}{a^2} h\Big( \frac{a t}{\sigma^2} \Big) \Bigg),
    \end{equation*}
    where $h(x) = (1 + x) \log(1 + x) - x$ and $\sigma^2 = \sum_i \sigma_i^2$.
\end{lemma}
\begin{proof}
    As usual, we take advantage of Markov's inequality to write:
    \begin{align*}
        \probability\Big[\sum_i X_i \geq t \Big] &\leq
            e^{-\lambda t} \ev \Big[ e^{\lambda \sum_i X_i} \Big] \\
        &= e^{-\lambda t} \ev \Big[ \prod_i e^{\lambda X_i} \Big] \\
        &= e^{-\lambda t} \prod_i \ev \Big[ e^{\lambda X_i} \Big] \\
    \end{align*}
    Using the Taylor expansion of $e^x$, we obtain:
    \begin{align*}
        \ev \Big[ e^{\lambda X_i} \Big] &= \ev \Big[ \sum_{k=0}^\infty \frac{\lambda^k X_i^k}{k!} \Big] \\
        &= 1 + \sum_{k=2}^\infty \frac{\lambda^k \ev[X_i^2 X_i^{k - 2}]}{k!} \\
        &\leq 1 + \sum_{k=2}^\infty \frac{\lambda^k \sigma_i^2 a^{k-2}}{k!} \\
        &= 1 + \frac{\sigma_i^2}{a^2} \sum_{k=2}^\infty \frac{\lambda^k a^k}{k!} \\
        &= 1 + \frac{\sigma_i^2}{a^2} \big( e^{\lambda a} - 1 - \lambda a \big) \\
        &\leq \exp\Big( \frac{\sigma_i^2}{a^2} \big( e^{\lambda a} - 1 - \lambda a \big) \Big).
    \end{align*}
    Putting it all together:
    \begin{align*}
        \probability\Big[\sum_i X_i \geq t \Big] &\leq
            e^{-\lambda t} \prod_i \exp\Big( \frac{\sigma_i^2}{a^2} \big( e^{\lambda a} - 1 - \lambda a \big) \Big) \\
        &= e^{-\lambda t} \exp\Big( \frac{\sigma^2}{a^2} \big( e^{\lambda a} - 1 - \lambda a \big) \Big).
    \end{align*}
    This inequality holds for all values of $\lambda$, and in particular one that minimizes the
    right-hand-side. Setting the derivative of the right-hand-side to $0$ and solving for $\lambda$
    leads to the desired result.
\end{proof}

\chapter{Linear Algebra Review}
\label{appendix:linear-algebra}

\abstract{
This appendix reviews basic concepts from Linear Algebra that are useful
in digesting the material in this monograph.
}

\section{Inner Product}

Denote by $\mathbb{H}$ a vector space.
An inner product $\langle \cdot, \cdot \rangle: \mathbb{H} \times \mathbb{H} \rightarrow \mathbb{R}$
is a function with the following properties:
\begin{itemize}
    \item $\forall \; u \in \mathbb{H},\; \langle u, u \rangle \geq 0$;
    \item $\forall \; u \in \mathbb{H},\; \langle u, u \rangle = 0 \Leftrightarrow u = 0$;
    \item $\forall \; u, v \in \mathbb{H},\; \langle u, v \rangle = \langle v, u \rangle$; and,
    \item $\forall \; u, v, w \in \mathbb{H}, \textit{ and } \alpha, \beta \in \mathbb{R},\; 
    \langle \alpha u + \beta v, w \rangle = \alpha \langle u, w \rangle + \beta \langle v, w \rangle$.
\end{itemize}

We call $\mathbb{H}$ together with the inner product $\langle \cdot, \cdot \rangle$
an \emph{inner product space}.
As an example, when $\mathbb{H} = \mathbb{R}^d$, given two vectors
$u = \sum_{i=1}^d u_i e_i$ and $v = \sum_{i=1}^d v_i e_i$, where $e_i$'s
are the standard basis vectors, the following is an inner product:
\begin{equation*}
    \langle u, v \rangle = \sum_{i = 1}^d u_i v_i.
\end{equation*}

We say two vectors $u$ and $v$ in an inner product space are \emph{orthogonal}
if their inner product is $0$: $\langle u, v \rangle = 0$.

\section{Norms}

A function $\Phi: \mathbb{H} \rightarrow \mathbb{R}_+$ is a norm on
$\mathbb{H}$ if it has the following properties:
\begin{itemize}
    \item Definiteness: For all $u \in \mathbb{H}$, $\Phi(u) = 0 \Leftrightarrow u = 0$;
    \item Homogeneity: For all $u \in \mathbb{H}$ and $\alpha \in \mathbb{R}$,
        $\Phi(\alpha u) = \lvert \alpha \rvert \Phi(u)$; and,
    \item Triangle inequality: $\forall \; u, v \in \mathbb{H}, \; \Phi(u + v) \leq \Phi(u) + \Phi(v)$.
\end{itemize}

Examples include the absolute value on $\mathbb{R}$,
and the $L_p$ norm (for $p \geq 1$) on $\mathbb{R}^d$ denoted by $\lVert \cdot \rVert_p$
and defined as:
\begin{equation*}
    \lVert u \rVert_p = \Big( \sum_{i=1}^d \lvert u_i \rvert^p \Big)^{\frac{1}{p}}.
\end{equation*}
Instances of $L_p$ include the commonly used $L_1$, $L_2$ (Euclidean),
and $L_\infty$ norms, where $\lVert u \rVert_\infty = \max_i \lvert u_i \rvert$.

Note that, when $\mathbb{H}$ is an inner product space, then
the function $\lVert u \rVert = \sqrt{\langle u, u \rangle}$ is a norm.

\section{Distance}
A norm on a vector space induces a notion of distance between two vectors.
Concretely, if $\mathbb{H}$ is a normed space equipped with $\lVert \cdot \rVert$,
then we define the distance between two vectors $u, v \in \mathbb{H}$ as follows:
\begin{equation*}
    \delta(u, v) = \lVert u - v \rVert.
\end{equation*}

\section{Orthogonal Projection}

\begin{lemma}
    Let $\mathbb{H}$ be an inner product space and suppose $u \in \mathbb{H}$ and $u \neq 0$.
    Any vector $v \in \mathbb{H}$ can be uniquely decomposed along $u$ as:
    \begin{equation*}
        v = v_{\perp} + v_{\parallel},
    \end{equation*}
    such that $\langle v_\perp, v_\parallel \rangle = 0$. Additionally:
    \begin{equation*}
        v_\parallel = \frac{\langle u, v \rangle}{\langle u, u \rangle} u,
    \end{equation*}
    and $v_\perp = v - v_\parallel$.
\end{lemma}
\begin{proof}
    Let $v_\parallel = \alpha u$ and $v_\perp = v - v_\parallel$.
    Because $v_\parallel$ and $v_\perp$ are orthogonal, we deduce that:
    \begin{align*}
        \langle v_\parallel, v_\perp \rangle = 0 \implies
            \langle \alpha u, v_\perp \rangle = 0 \implies
            \langle u, v_\perp \rangle = 0.
    \end{align*}
    That implies:
    \begin{align*}
        \langle v, u \rangle = \alpha \langle u, u \rangle \implies
        \alpha = \frac{\langle u, v \rangle}{\langle u, u \rangle},
    \end{align*}
    so that:
    \begin{equation*}
        v_\parallel = \frac{\langle u, v \rangle}{\langle u, u \rangle} u.
    \end{equation*}

    We prove the uniqueness of the decomposition by contradiction.
    Suppose there exists another decomposition of $v$ to $v_\parallel^\prime + v_\perp^\prime$.
    Then:
    \begin{align*}
        v_\parallel + v_\perp = v_\parallel^\prime + v_\perp^\prime &\implies
        \langle u, v_\parallel + v_\perp \rangle = \langle u,  v_\parallel^\prime + v_\perp^\prime\rangle \\
        &\implies \langle u, v_\parallel \rangle = \langle u,  v_\parallel^\prime \rangle \\
        &\implies \langle u, \alpha u \rangle = \langle u, \beta u \rangle \\
        &\implies \alpha = \beta.
    \end{align*}
    We must therefore also have that $v_\perp = v_\perp^\prime$.
\end{proof}

\end{document}
