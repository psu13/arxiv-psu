\section{Introduction}\label{sec: introduction}
%
%
\emph{Compositionality} is probably the most relevant principle in applied category theory (ACT) research.
While there is no unified definition \cite{Werning_Hinzen_Machery_2012, ghani2018compositional, coecke2021compositionality}, it refers, broadly speaking, to certain forms of relation between properties, behaviours, or observations of a composite system on one hand, and those of its components on the other.
A common concern, in this context, is whether it is possible to derive properties of the whole from properties of its parts, and vice versa.
In some cases, both directions are viable and inverse to each other, in which case a property is ``fully compositional''.
More frequently, only one direction is viable.

The need to formally quantify and/or qualify compositionality has been widely discussed in the ACT community at least since 2018 \cite{Genovese2018mod}, as researchers became increasingly aware of various ``failures of compositionality'', and wished to classify them beyond a simple yes-or-no statement.

Let us be more precise.
Much research in ACT has been devoted to the study of \emph{open systems}, that is, entities with open interfaces that can be composed with other entities of the same kind.
This approach has been pervasive, and has been applied in the study of \emph{categorical quantum mechanics} \cite{abramsky2009categorical}, \emph{natural language} \cite{coecke2021mathematics},  \emph{dynamical systems} \cite{Fong_Spivak_2019}, \emph{Petri nets} \cite{baez2021categories}, \emph{game theory} \cite{ghani2018compositional} and many other subjects.
When studying open systems, it is not rare to define functors mapping a ``theory of boxes''  --- in the form of a monoidal category or bicategory --- where the composition rules of the systems are defined, to a certain ``semantic universe'' of properties or behaviours of the systems.
The properties of these functors reflect how well the information that they capture adheres to the composition rules: a \emph{lax} functor $\fun{P}$, with structural \emph{laxator} morphisms in the direction $\fun{P}f \Cp \fun{P}g \to \fun{P}(f \Cp g)$, means that one can derive information on the whole system from information on its components; an \emph{oplax} functor, with structural morphisms in the direction $\fun{P}(f \Cp g) \to \fun{P}f \Cp \fun{P}g$, means that one can derive information on the components from information on the whole; while a \emph{strong} functor means that the information on components and the information on the whole completely determine each other.

For example, the functor sending \emph{open graphs} to their \emph{reachability relation} (see \autoref{subsec: open graphs}) is lax, which tells us that the reachability relation of a composition of open graphs can be strictly bigger than the composition of the reachability relations defined on its parts.
This is considered undesirable from a computational viewpoint, as it means that one cannot reconstruct the reachability of a graph by separately computing the reachability of its components.

On the other hand, in ``Schr\"odinger compositionality'' (covered in \autoref{subsec: schrodinger compositionality}), quantum-mechanical behaviour arises from the laxity of the functor mapping each object to its set of states.
This laxity implies that not all quantum states are separable, which is desirable, as it unlocks the use of \emph{entanglement} as a resource unavailable in classical mechanics.

In both cases, laxity represents a ``failure of compositionality'' which has both practical and foundational importance: the ``gap'' between a lax and a strong functor represents the gap between what we can compute compositionally with a ``divide-and-conquer'' strategy and what we cannot, or the gap between a classical and non-classical theory of processes.
In this light, the question: \emph{how can we qualify (failures of) compositionality?} becomes the question: \emph{how far is a lax functor from being strong?}\footnote{We will focus on lax functors in our discussion, but everything can be dualised to oplax functors.}
In this paper, we attempt to give a structured answer to the question.
Our chain of reasoning is the following.
%
%
\begin{definition}
    A lax functor is strong when all the components of its laxators are isomorphisms.
\end{definition}
%
Thus, we can think of reducing our question to the more general one: \emph{how far is a morphism from being an isomorphism?}\footnote{This approach, and the fact that it could be investigated with homotopical methods, was first suggested to us by Jules Hedges.}
Let us use the following, well-known characterisation of isomorphisms.
%
%
\begin{proposition}\label{prop: iso if terminal in slice}
    A morphism $f\colon X \to Y$ in a category $\CategoryC$ is an isomorphism if and only if it is terminal as an object of the slice category $\slice{\CategoryC}{Y}$.
\end{proposition}
%
This allows us to reduce further to the question: \emph{how far is an object from being terminal?}
Terminality can be split into the following pair of properties.
%
%
\begin{definition}\label{def: weak and subterminal}
    An object $\WTerm$ in a category $\CategoryC$ is
    \begin{itemize}
        \item \emph{weak terminal} if, for all objects $X$ of $\CategoryC$, there exists a morphism $X \to \WTerm$;
        \item \emph{subterminal} if, for all parallel pairs of morphisms $f, g\colon X \to \WTerm$, we have $f = g$.
    \end{itemize}
\end{definition}
%
Hence, to describe how far $\WTerm$ is from being terminal, we can separately describe how far $\WTerm$ is from being weak terminal and subterminal, respectively.

Following this chain of reasoning, we focus on classifying \emph{obstructions to weak terminality and subterminality} for objects in arbitrary categories.
Surprisingly, it turns out that there exists a natural way of associating certain \emph{pointed posets} to a pointed category (category with a chosen object), which we call the \emph{zeroth} and \emph{first homotopy poset}, because in a precise sense they generalise the $\pi_0$ and $\pi_1$ of a pointed groupoid seen as a homotopy 1-type.
This opens up the possibility of an \emph{invariant-based} approach to the formal study of compositionality: the homotopy posets contain no information that is not already in the functors and categories, but put it in a form which may be more tractable and intelligible.

In \autoref{sec: homotopy poset}, we give the definitions of homotopy posets and state their basic properties, demonstrating in which sense they answer our question about terminal objects.
In \autoref{sec: obstructions}, going backwards in our chain of reasoning, we apply them to the study of obstructions to morphisms being iso.
Finally, in \autoref{sec: qualifying}, we sketch through a couple of simple examples how our framework can be applied to the study of failures of compositionality, seen as failures of certain (op)lax functors to be strong.