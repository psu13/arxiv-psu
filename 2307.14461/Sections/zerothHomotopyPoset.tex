\section{Homotopy posets}\label{sec: homotopy poset}
%
To begin, we focus on obstructions to weak terminality.
Having fixed a category $\CategoryC$, we interpret objects of a category $\CategoryC$ as points, and morphisms between them as paths. 
From this point of view, a weak terminal object is an object that is always reachable from any generic object $x$ in $\CategoryC$.

Intuitively, we can fix a ``weak terminal object candidate''\footnote{
    In this paper, we will use $\WTerm$ to denote ``terminal object candidates'', that is, objects for which we want to investigate how far they are from being terminal. For an object that we know or presume to be terminal, we will instead use the notation $\Term$.
} $\WTerm$ and consider any object $x$ such that there is \emph{no} morphism $x \to \WTerm$ as an \emph{obstruction to weak terminality}.
Moreover:
%
%
\begin{itemize}
    \item If $x, y$ are obstructions for $\WTerm$, and there are morphisms $x \to y$ and $y \to x$, we regard them as equivalent: if there were a morphism $x \to \WTerm$ there would be a morphism $y \to \WTerm$, and vice versa.
    \item If $x,y$ are obstructions for $\WTerm$ and there is a morphism $x \to y$, then we regard $x$ as a ``more fundamental obstruction than $y$''. 
    This is because, if there were a morphism $y \to \WTerm$, we would automatically obtain a morphism $x \to \WTerm$ by composition (one can ``go from $x$ to $y$ and then to $\WTerm$''), while the opposite is not true.
\end{itemize}
%
We will devote this section to making this intuition formal.
%
%
\begin{definition}[Poset reflection] \label{def: poset reflection}
    Let $\catpos$ be the large\footnote{We will denote categories in \textit{italics} and large categories in \textbf{bold}. Note that in our constructions, what matters is only the \emph{relative} size: a construction which associates a poset to a category can be applied to a large category, producing a large poset.}
    category of posets and order\nbd preserving maps.
    There is a full and faithful functor $\imath\colon \catpos \incl \catcat$, whose image consists of the categories that are
    \begin{itemize}
        \item \emph{thin} (each hom\nbd set contains at most one morphism), and
        \item \emph{skeletal} (every isomorphism is an automorphism).
    \end{itemize}
    The \emph{poset reflection} $\posref{\CategoryC}$ of a category $\CategoryC$ is its image under the left adjoint $\posref{-}\colon \catcat \to \catpos$ to $\imath$:
    \begin{itemize}
        \item the elements of $\posref{\CategoryC}$ are equivalence classes $\posref{x}$ of objects $x$ of $\CategoryC$, where $\posref{x} = \posref{y}$ if and only if there exist morphisms $x \to y$ and $y \to x$ in $\CategoryC$, and
        \item $\posref{x} \leq \posref{y}$ if and only if there exists a morphism $x \to y$ in $\CategoryC$.
    \end{itemize}
\end{definition}
%
%
\begin{proposition}\label{prop: weak terminal is greatest in posref}
    Let $\CategoryC$ be a category and $\WTerm$ an object in $\CategoryC$.
    The following are equivalent:
    \begin{enumerate}[label=(\alph*)]
        \item $\WTerm$ is a weak terminal (respectively, initial) object in $\CategoryC$;
        \item $\posref{\WTerm}$ is the greatest (respectively, least) element of $\posref{\CategoryC}$.
    \end{enumerate}
\end{proposition}
%
%
%
\begin{definition}[Arrow category]\label{def: arrow category}
    Let $\wkarr$ be the ``walking arrow'' category, that is, the free category on the graph
    \[\begin{tikzcd}[sep=scriptsize]
        0 && 1
        \arrow["a", from=1-1, to=1-3]
    \end{tikzcd}.\]
    The \emph{arrow category} of a category $\CategoryC$ is the functor category $\CategoryC^\wkarr$.
    Explicitly, the objects of $\CategoryC^\wkarr$ are morphisms of $\CategoryC$, while morphisms of $\CategoryC^\wkarr$ are commutative squares in $\CategoryC$.
    There are functors $\mathrm{dom}$, $\mathrm{cod}\colon \CategoryC^\wkarr \to \CategoryC$ which, given a morphism $(h_0, h_1)$, return $h_0$, respectively, $h_1$.
\end{definition}
%
%    
\begin{definition}[Category of pointed objects]\label{def: pointed objects category}
    Let $\CategoryC$ be a category with a chosen terminal object $\Term$.
    A \emph{pointed object} $(x, v)$ of $\CategoryC$ is an object $x$ of $\CategoryC$ together with a morphism $v\colon \Term \to x$, called its \emph{basepoint}.
    The \emph{category of pointed objects} of $\CategoryC$ --- denoted by $\pointed{\CategoryC}$ --- is the coslice category $\slice{\Term}{\CategoryC}$.
\end{definition}
%
%
\begin{proposition}[Functoriality of arrow and pointed objects categories]\label{prop: functoriality of arrow and pointed cats}
    Let $\fun{F}\colon \CategoryC \to \CategoryD$ be a functor.
    Then $\fun{F}$ lifts to a functor $\fun{F}^\wkarr\colon \CategoryC^\wkarr \to \CategoryD^\wkarr$
    using the pointwise action of $\fun{F}$ on $\CategoryC$. 
    
    If moreover $\CategoryC$ and $\CategoryD$ have a chosen terminal object, and if $\fun{F}$ preserves it, then it also lifts to a functor $\pointed{\fun{F}}\colon \pointed{\CategoryC} \to \pointed{\CategoryD}$ sending a pointed object $(x, v)$ of $\CategoryC$ to $(\fun{F}x, \fun{F}v)$, a pointed object of $\CategoryD$.
\end{proposition}
%
%
\begin{definition}[Quotient of an object by a morphism]\label{def: quotient by a morphism}
    Let $\CategoryC$ be a category with chosen pushouts and a terminal object $\Term$.
    Given a morphism $f\colon x \to y$, the \emph{quotient of $y$ by $f$} is the pushout
    \begin{equation*}
        \begin{tikzcd}[sep=scriptsize]
            x && \Term \\
            \\
            y && {y\sslash f}
            \arrow["{!}", from=1-1, to=1-3]
            \arrow["f"', from=1-1, to=3-1]
            \arrow[from=3-1, to=3-3]
            \arrow["{[x]}", from=1-3, to=3-3]
            \arrow["\lrcorner"{anchor=center, pos=0.125, rotate=180}, draw=none, from=3-3, to=1-1]
        \end{tikzcd}
    \end{equation*}
    where $!\colon x \to \Term$ is the unique morphism from $x$ to the terminal object.
\end{definition}
%
%    
\begin{proposition}[Functoriality of the quotient]\label{prop: functoriality of the quotient}
    If $\CategoryC$ has chosen pushouts and a terminal object $\Term$, then for each morphism $f\colon x \to y$ in $\CategoryC$ \autoref{def: quotient by a morphism} determines a pointed object $\fun{Q}(f) \eqdef (y \sslash f, [x])$ of $\CategoryC$. This extends to a functor $\fun{Q}\colon \CategoryC^\wkarr \to \pointed{\CategoryC}$.
    If both $\CategoryC$ and $\CategoryD$ have chosen pushouts and a chosen terminal object $\Term$, and if $\fun{F}$ preserves them, then $\fun{F}$ induces a commutative square of functors
    %
    %
    \begin{equation*}
        \begin{tikzcd}[sep=scriptsize]
            \CategoryC^\wkarr && \pointed{\CategoryC} \\
            \\
            \CategoryD^\wkarr && \pointed{\CategoryD}.
            \arrow["\fun{Q}", from=1-1, to=1-3]
            \arrow["\fun{F}^\wkarr"', from=1-1, to=3-1]
            \arrow["\fun{Q}"', from=3-1, to=3-3]
            \arrow["\pointed{\fun{F}}", from=1-3, to=3-3]
        \end{tikzcd}
    \end{equation*}
\end{proposition}
%
% 
The categories $\catcat$ and $\catpos$ have all limits and colimits, so in particular they have pushouts and a terminal object. The poset reflection functor $\posref{-}\colon \catcat \to \catpos$ sends the terminal category to the terminal poset, and preserves pushouts, since it is a left adjoint.
The preservation can be made strict with respect to a choice on both sides.
We are in the conditions of \autoref{prop: functoriality of the quotient}: there is a commutative square
\begin{equation} \label{eq: quotient and posref}
    \begin{tikzcd}[sep=scriptsize]
        \catcat^\wkarr && \pointed{\catcat} \\
        \\
        \catpos^\wkarr && \pointed{\catpos}.
        \arrow["\fun{Q}", from=1-1, to=1-3]
        \arrow["\posref{-}^\wkarr"', from=1-1, to=3-1]
        \arrow["\fun{Q}", from=3-1, to=3-3]
        \arrow["\pointed{\posref{-}}", from=1-3, to=3-3]
    \end{tikzcd}
\end{equation}
%
We are now ready to define the object of interest of this section.
%
%
\begin{definition}[Zeroth homotopy poset] \label{def: 0th-directed homotopy poset}
    Let $\CategoryC$ be a category and $x$ an object in $\CategoryC$.
    The \emph{zeroth homotopy poset of $\CategoryC$ over $x$} is the pointed poset
    \begin{equation*}
        (\dhom{0}{\CategoryC}{x}, \; [x])
    \end{equation*}
    obtained by applying the functor $\catcat^\wkarr \to \pointed{\catpos}$ from \autoref{eq: quotient and posref} to the slice projection functor
    \begin{equation*}
        \mathrm{dom}\colon \slice{\CategoryC}{x} \to \CategoryC.
    \end{equation*}
\end{definition}
%
%
Let us unravel the definition of $\dhom{0}{\cat{C}}{x}$ to a more explicit form.
We start from the projection functor $\mathrm{dom}\colon \slice{\cat{C}}{x} \to \cat{C}$.
    To this we may either apply $\fun{Q}$ or $\posref{-}^\wkarr$.
    Since quotients in $\catpos$ are simpler to compute than quotients in $\catcat$, we apply poset reflection first, which gives us an order-preserving map
    %
    %
    \begin{equation*}
        \posref{\mathrm{dom}}\colon \posref{\slice{\cat{C}}{x}} \to \posref{\cat{C}}.
    \end{equation*}
    %
    Unravelling the explicit definition of poset reflection for $\slice{\cat{C}}{x}$, we see that:
    %
    %
    \begin{itemize}
        \item an element of $\posref{\slice{\cat{C}}{x}}$ is an equivalence class $\posref{f\colon y \to x}$ of morphisms of $\cat{C}$ with codomain $x$, where $\posref{f} = \posref{g}$ if and only if $f$ factors through $g$ and $g$ factors through $f$, and
        \item $\posref{f} \leq \posref{g}$ if and only if $f$ factors through $g$.
    \end{itemize}
    %
    The map $\posref{\mathrm{dom}}$ sends $\posref{f}$ to $\posref{\mathrm{dom}\, f}$.
    The image of $\posref{\mathrm{dom}}$ is then the set
    \begin{equation*}
        \{ \posref{y} \mid \text{there exists a morphism $f\colon y \to x$ in $\cat{C}$} \},
    \end{equation*}
    which is, equivalently, the lower set of $\posref{x}$ in $\posref{\cat{C}}$.

    Applying $\fun{Q}\colon \catpos^\wkarr \to \pointed{\catpos}$ to this map produces the quotient of $\posref{\cat{C}}$ with all elements of this set identified, pointed with the element resulting from their identification, which we denote by $[x]$.
    Hence, an element of $\dhom{0}{\CategoryC}{x}$ is either $[x]$, or it is $\posref{y}$ for some object $y$ such that there exists no morphism $f\colon y \to x$ in $\CategoryC$.
    The order relation is defined as follows, by case distinction:
    \begin{itemize}
        \item $[x] \leq [x]$ trivially;
        \item $[x] \leq \posref{y}$ if and only if there exists a span $(x \xleftarrow{f} z \xrightarrow{g} y)$ in $\CategoryC$;
        \item it is never the case that $\posref{y} \leq [x]$;
        \item $\posref{y} \leq \posref{z}$ if and only if there exists a morphism $f\colon y \to z$ in $\CategoryC$.
    \end{itemize}
    Notice that $[x]$ is always minimal in $\dhom{0}{\CategoryC}{x}$.

The partial order on $\dhom{0}{\CategoryC}{x}$ ranks obstructions to weak terminality by ``size'': if we removed an obstruction $\posref{y}$, adding a morphism $y \to x$, we would also have to remove all the ``smaller'' obstructions $\posref{z} \leq \posref{y}$.
The minimal element $[x]$ represents the ``non-obstructions'':
\begin{proposition} \label{prop:dhom0_trivial_when_weak_terminal}
Let $\cat{C}$ be a category and $x$ an object in $\cat{C}$.
The following are equivalent:
\begin{enumerate}[label=(\alph*)]
    \item $\dhom{0}{\cat{C}}{x} = \{[x]\}$;
    \item $x$ is a weak terminal object in $\cat{C}$.
\end{enumerate}
\end{proposition}

The notation and terminology is suggestive of the $\pi_0$ of a pointed topological space or groupoid, that is, its set of connected components, pointed with the connected component of the basepoint. 
The following result shows that, indeed, the notions coincide when $\cat{C}$ happens to be a groupoid.

\begin{proposition}[$\dhom{0}{\cat{G}}{x}$ for a groupoid]\label{prop: dhom0 is pi0 for groupoids}
    Let $\cat{G}$ be a groupoid and $x$ an object in $\cat{G}$.
    Then
    \begin{enumerate}
        \item $\dhom{0}{\cat{G}}{x}$ is a ``set'', that is, a discrete poset, and
        \item as a pointed set, it is isomorphic to the set $\pi_0(\cat{G})$ of connected components of $\cat{G}$, pointed with the connected component of $x$.
    \end{enumerate}
\end{proposition}
