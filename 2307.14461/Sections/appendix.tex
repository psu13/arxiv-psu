\newpage
\section{Proofs}

\subsection{Introduction}

    \begin{definition}[Slice category]\label{dfn:slice_category}
        Let $\CategoryC$ be a category and $X$ an object in $\CategoryC$.
        The \emph{slice of $\CategoryC$ over $X$} is the category $\slice{\CategoryC}{X}$ whose
        \begin{itemize}
            \item objects are morphisms $(f\colon Y \to X)$ in $\CategoryC$ with codomain $X$, and
            \item a morphism from $(f\colon Y \to X)$ to $(g\colon Z \to X)$ is a factorisation of $f$ through $g$, that is, a morphism $h\colon Y \to Z$ such that $f = h\Cp g$.
        \end{itemize}
        This comes with a projection functor $\mathrm{dom}\colon \slice{\CategoryC}{X} \to \CategoryC$.
    \end{definition}

    \begingroup
    \def\theproposition{\ref{prop: iso if terminal in slice}}
    \begin{proposition}
        A morphism $f:X \to Y$ in a category $\CategoryC$ is iso iff it is terminal in the slice category $\slice{\CategoryC}{Y}$.
    \end{proposition}
    \addtocounter{proposition}{-1}
    \endgroup
    \begin{proof}
    Follows from \autoref{prop: spit epi weak term mono subterm}, remembering that a morphism is split epi and mono iff it is iso.
    \end{proof}


\subsection{Homotopy posets}

    \begingroup
    \def\thedefinition{\ref{def: poset reflection}}
    \begin{definition}[Poset reflection]
        Let $\catpos$ be the large category of posets and order\nbd preserving maps.
        There is a full and faithful functor $\imath\colon \catpos \incl \catcat$, whose image consists of the categories that are
        \begin{itemize}
            \item \emph{thin} (each hom\nbd set contains at most one morphism), and
            \item \emph{skeletal} (every isomorphism is an automorphism).
        \end{itemize}

        The \emph{poset reflection} $\posref{\CategoryC}$ of a category $\CategoryC$ is its image under the left adjoint $\posref{-}\colon \catcat \to \catpos$ to $\imath$:
        \begin{itemize}
            \item the elements of $\posref{\CategoryC}$ are equivalence classes $\posref{x}$ of objects $x$ of $\CategoryC$, where $\posref{x} = \posref{y}$ if and only if there exist morphisms $x \to y$ and $y \to x$ in $\CategoryC$, and
            \item $\posref{x} \leq \posref{y}$ if and only if there exists a morphism $x \to y$ in $\CategoryC$.
        \end{itemize}
    \end{definition}
    \addtocounter{definition}{-1}
    \endgroup
    \begin{proof}
    Let $\cat{C}$ be a category and $P$ a poset; we will identify $P$ with the thin, skeletal category $\imath P$.
    Given a functor $\fun{F}\colon \cat{C} \to \imath P$, we define an order-preserving map $f\colon \posref{\cat{C}} \to P$ by letting $f(\posref{x}) \eqdef \fun{F}x$ for all objects $x$ of $\cat{C}$.
    This is well-defined as a function, since $\posref{x} = \posref{y}$ implies that there are morphisms $x \to y$ and $y \to x$, hence $\fun{F}x \leq \fun{F}y$ and $\fun{F}y \leq \fun{F}x$, so $\fun{F}x = \fun{F}y$ by antisymmetry of the order on $P$.
    It is also order-preserving: if $\posref{x} \leq \posref{y}$ then there exists a morphism $x \to y$, so $\fun{F}x \leq \fun{F}y$.

    Conversely, given an order-preserving map $f\colon \posref{\cat{C}} \to P$, we define a functor $\fun{F}\colon \cat{C} \to \imath P$ by letting $\fun{F}x \eqdef f(\posref{x})$.
    The two assignments are clearly inverse to each other, and naturality in $\cat{C}$ and $P$ is a straightforward check.
    This proves that $\posref{-}\colon \catcat \to \catpos$ is left adjoint to $\imath\colon \catpos \to \catcat$.
    \end{proof}

    \begingroup
    \def\theproposition{\ref{prop: weak terminal is greatest in posref}}
    \begin{proposition}
        Let $\CategoryC$ be a category and $\WTerm$ an object in $\CategoryC$.
        The following are equivalent:
        \begin{enumerate}[label=(\alph*)]
            \item $\WTerm$ is a weak terminal (respectively, initial) object in $\CategoryC$;
            \item $\posref{\WTerm}$ is the greatest (respectively, least) element of $\posref{\CategoryC}$.
        \end{enumerate}
    \end{proposition}
    \addtocounter{proposition}{-1}
    \endgroup
    \begin{proof}
        If $x$ is any object of $\posref{\CategoryC}$, then $x$ is of the form $\posref{\tilde{x}}$ for some $\tilde{x}$ in $\CategoryC$. 
        If $\WTerm$ is weak terminal, then there is at least one morphism $\tilde{x} \to \WTerm$ in $\CategoryC$, and so $x = \posref{\tilde{x}} \leq \posref{\WTerm}$ in $\posref{\CategoryC}$. 
        
        The other way around, taking any $x$ in $\CategoryC$, $\posref{\WTerm}$ being the greatest element of $\posref{\CategoryC}$ gives us $\posref{x} \leq \posref{\WTerm}$ in $\posref{\CategoryC}$; hence there is at least one morphism $x \to \WTerm$ in $\CategoryC$, proving that $\WTerm$ is weak terminal.
    \end{proof}

    \begingroup
    \def\thedefinition{\ref{def: arrow category}}
    \begin{definition}[Arrow category]
        Let $\wkarr$ be the ``walking arrow'' category, that is, the free category on the graph
        \[\begin{tikzcd}
            0 && 1
            \arrow["a", from=1-1, to=1-3]
        \end{tikzcd}.\]
        The \emph{arrow category} of a category $\CategoryC$ is the functor category $\CategoryC^\wkarr$.
        Explicitly, the objects of $\CategoryC^\wkarr$ are morphisms of $\CategoryC$, while morphisms of $\CategoryC^\wkarr$ are commutative squares in $\CategoryC$.
        There are functors $\mathrm{dom}$, $\mathrm{cod}\colon \CategoryC^\wkarr \to \CategoryC$ which, given a morphism $(h_0, h_1)$, return $h_0$, respectively, $h_1$.
    \end{definition}
    \addtocounter{definition}{-1}
    \endgroup
    \begin{proof}
    The proof that the arrow category is a category follows directly from the fact that it is defined as a functor category. Explicitly, composition law is induced point-wise by the composition in $\CategoryC$. The fact that composition is induced point-wise makes it also obvious to prove that $\mathrm{dom}$, $\mathrm{cod}$ are indeed functors. 
    \end{proof}

    %\begingroup
    %\def\thedefinition{\ref{def: pointed objects category}}
    %\begin{definition}[Category of pointed objects]
        %Let $\CategoryC$ be a category with a chosen terminal object $\Term$.
        %A \emph{pointed object} $(x, v)$ of $\CategoryC$ is an object $x$ of $\CategoryC$ together with a morphism $v\colon \Term \to x$, called its \emph{basepoint}.
        %The \emph{category of pointed objects} of $\CategoryC$ --- denoted by $\pointed{\CategoryC}$ --- is the coslice category $\slice{\Term}{\CategoryC}$.
    %\end{definition}
    %\addtocounter{definition}{-1}
    %\endgroup
    %\begin{proof}
    %Verifying that identity and composition laws hold for $\pointed{\CategoryC}$ follows from the fact that they hold for any coslice category, which is a well-known fact.
    %\end{proof}

    \begingroup
    \def\theproposition{\ref{prop: functoriality of arrow and pointed cats}}
    \begin{proposition}[Functoriality of arrow and pointed objects categories]
        Let $\fun{F}\colon \CategoryC \to \CategoryD$ be a functor.
        Then $\fun{F}$ lifts to a functor $\fun{F}^\wkarr\colon \CategoryC^\wkarr \to \CategoryD^\wkarr$
        using the pointwise action of $\fun{F}$ on $\CategoryC$. 
        
        If moreover $\CategoryC$ and $\CategoryD$ have a chosen terminal object, and if $\fun{F}$ preserves it, then it also lifts to a functor $\pointed{\fun{F}}\colon \pointed{\CategoryC} \to \pointed{\CategoryD}$ sending a pointed object $(x, v)$ of $\CategoryC$ to $(\fun{F}x, \fun{F}v)$, a pointed object of $\CategoryD$.
    \end{proposition}
    \addtocounter{proposition}{-1}
    \endgroup
    \begin{proof}
    The functor $\fun{F}^\wkarr$ is defined such that it acts by post-composition with $\fun{F}$ on objects of $\CategoryC^\wkarr$ and by right whiskering with $\fun{F}$ on morphisms. Identity and composition preservation easily follow from the definition. 

    By definition of category of pointed objects, it is enough to prove that $\fun{F}\colon \CategoryC \to \CategoryD$ induces a functor $\slice{x}{\CategoryC} \to \slice{Fx}{\CategoryD}$ on the respective coslice categories. In order to do that, it is enough to send an object $f:x \to y$ of $\slice{x}{\CategoryC}$ to $Ff: Fx \to Fy$ and a morphism $h$ of $\slice{x}{\CategoryC}$ to $Fh$. The functor $F$ preserves commutative diagrams, therefore $Fh$ is indeed a morphism in $\slice{Fx}{\CategoryD}$. Identity and composition preservation easily follow from the fact that $F$ is a functor. 
    \end{proof}

    \begingroup
    \def\theproposition{\ref{prop: functoriality of the quotient}}
    \begin{proposition}[Functoriality of the quotient]
        If $\CategoryC$ has chosen pushouts and a terminal object $\Term$, then for each morphism $f\colon x \to y$ in $\CategoryC$, \autoref{def: quotient by a morphism} determines a pointed object $\fun{Q}(f) \eqdef (y \sslash f, [x])$ of $\CategoryC$. This extends to a functor $\fun{Q}\colon \CategoryC^\wkarr \to \pointed{\CategoryC}$.
    
        Lastly, if both $\CategoryC$ and $\CategoryD$ have chosen pushouts and a chosen terminal object $\Term$, and if $\fun{F}$ preserves them, then $\fun{F}$ induces a commutative square of functors
        %
        %
        \begin{equation*}
            \begin{tikzcd}[sep=scriptsize]
                \CategoryC^\wkarr && \pointed{\CategoryC} \\
                \\
                \CategoryD^\wkarr && \pointed{\CategoryD}.
                \arrow["\fun{Q}", from=1-1, to=1-3]
                \arrow["\fun{F}^\wkarr"', from=1-1, to=3-1]
                \arrow["\fun{Q}"', from=3-1, to=3-3]
                \arrow["\pointed{\fun{F}}", from=1-3, to=3-3]
            \end{tikzcd}
        \end{equation*}
    \end{proposition}
    \addtocounter{proposition}{-1}
    \endgroup
    \begin{proof}
    First of all, let us build the functor $\fun{Q}$. \autoref{def: quotient by a morphism} already provides a definition on objects. On morphisms, given $(h_0,h_1)\colon f \to g$ in $\CategoryC^\wkarr$, corresponding to a commutative square
        \[\begin{tikzcd}[sep=scriptsize]
            & {y_0} && \\
            {x_0} && \\
            & {y_1} && \\
            {x_1} && 
            \arrow["f"', from=2-1, to=4-1]
            \arrow["g"'{pos=0.7}, from=1-2, to=3-2]
            \arrow["{h_0}", from=2-1, to=1-2]
            \arrow["{h_1}", from=4-1, to=3-2]
        \end{tikzcd}\]
        we have a diagram
        \[\begin{tikzcd}[sep=scriptsize]
            & {y_0} && \Term \\
            {x_0} && \Term \\
            & {y_1} && {y_1 \sslash g} \\
            {x_1} && {x_1 \sslash f}
            \arrow["{!}", from=2-1, to=2-3]
            \arrow["f"', from=2-1, to=4-1]
            \arrow[from=4-1, to=4-3]
            \arrow["{[x_0]}"{pos=0.7}, from=2-3, to=4-3]
            \arrow["\lrcorner"{anchor=center, pos=0.125, rotate=180}, draw=none, from=4-3, to=2-1]
            \arrow["g"'{pos=0.7}, from=1-2, to=3-2]
            \arrow["{!}", from=1-2, to=1-4]
            \arrow[from=3-2, to=3-4]
            \arrow["{[y_0]}"{pos=0.7}, from=1-4, to=3-4]
            \arrow["{h_0}", from=2-1, to=1-2]
            \arrow[Rightarrow, no head, from=2-3, to=1-4]
            \arrow["{h_1}", from=4-1, to=3-2]
            \arrow["\lrcorner"{anchor=center, pos=0.125, rotate=180}, draw=none, from=3-4, to=1-2]
        \end{tikzcd}\]
        where the left side commutes by assumption, the front and back by construction, and the top because $1$ is terminal.
        Then the universal property of the front pushout square induces a unique morphism
        \begin{equation*}
            \fun{Q}(h_0, h_1)\colon x_1 \sslash f \to y_1 \sslash g
        \end{equation*}
        which completes the diagram to a cube whose sides are all commutative. 
        
        In particular, $[x_0]\Cp  \fun{Q}(h_0, h_1) = [y_0]$, that is, $\fun{Q}(h_0, h_1)$ determines a morphism of pointed objects from $(x_1 \sslash f, [x_0])$ to $(y_1 \sslash g, [y_0])$.
        It is straightforward to check that this assignment respects composition and identities.

        Now, let $\fun{F}: \CategoryC \to \CategoryD$ preserve pushouts and terminal objects.
        First of all, notice that for each $f: x \to y$ in $\CategoryC$ we have:
        \begin{equation*}
            \pointed{\fun{F}}(\fun{Q}(f)) = \pointed{\fun{F}}(y \sslash f, [x]) = (\fun{F}(y \sslash f), \fun{F}[x]) = (\fun{F}y \sslash \fun{F}f, [\fun{F}x]) = \fun{Q}(\fun{F}^\wkarr(f))
        \end{equation*}
        verifying the commutative square of functors on objects. As for morphisms, consider the following cubes:
        \[
            \begin{tikzpicture}[baseline= (a).base]
                \node[scale=.7] (a) at (0,0){   
                    \begin{tikzcd}
                        & {\fun{F}y_0} && \fun{F}\Term \\
                        {\fun{F}x_0} && \fun{F}\Term \\
                        & {\fun{F}y_1} && {\fun{F}(y_1 \sslash g)} \\
                        {\fun{F}x_1} && {\fun{F}(x_1 \sslash f)}
                        \arrow["\fun{F}{!}", from=2-1, to=2-3]
                        \arrow["\fun{F}f"', from=2-1, to=4-1]
                        \arrow[from=4-1, to=4-3]
                        \arrow["{\fun{F}[x_0]}"{pos=0.7}, from=2-3, to=4-3]
                        \arrow["\lrcorner"{anchor=center, pos=0.125, rotate=180}, draw=none, from=4-3, to=2-1]
                        \arrow["\fun{F}g"'{pos=0.7}, from=1-2, to=3-2]
                        \arrow["\fun{F}{!}", from=1-2, to=1-4]
                        \arrow[from=3-2, to=3-4]
                        \arrow["{\fun{F}[y_0]}"{pos=0.7}, from=1-4, to=3-4]
                        \arrow["\fun{F}{h_0}", from=2-1, to=1-2]
                        \arrow[Rightarrow, no head, from=2-3, to=1-4]
                        \arrow["\fun{F}{h_1}", from=4-1, to=3-2]
                        \arrow["\lrcorner"{anchor=center, pos=0.125, rotate=180}, draw=none, from=3-4, to=1-2]
                    \end{tikzcd}
                };
            \end{tikzpicture}
        \qquad
        \begin{tikzpicture}[baseline= (a).base]
            \node[scale=.7] (a) at (0,0){  
                \begin{tikzcd}
                    & {\fun{F}y_0} && \Term \\
                    {\fun{F}x_0} && \Term \\
                    & {\fun{F}y_1} && {\fun{F}y_1 \sslash \fun{F}g} \\
                    {\fun{F}x_1} && {\fun{F}x_1 \sslash \fun{F}f}
                    \arrow["{!}", from=2-1, to=2-3]
                    \arrow["\fun{F}f"', from=2-1, to=4-1]
                    \arrow[from=4-1, to=4-3]
                    \arrow["{[\fun{F}x_0]}"{pos=0.7}, from=2-3, to=4-3]
                    \arrow["\lrcorner"{anchor=center, pos=0.125, rotate=180}, draw=none, from=4-3, to=2-1]
                    \arrow["\fun{F}g"'{pos=0.7}, from=1-2, to=3-2]
                    \arrow["{!}", from=1-2, to=1-4]
                    \arrow[from=3-2, to=3-4]
                    \arrow["{[\fun{F}y_0]}"{pos=0.7}, from=1-4, to=3-4]
                    \arrow["\fun{F}{h_0}", from=2-1, to=1-2]
                    \arrow[Rightarrow, no head, from=2-3, to=1-4]
                    \arrow["\fun{F}{h_1}", from=4-1, to=3-2]
                    \arrow["\lrcorner"{anchor=center, pos=0.125, rotate=180}, draw=none, from=3-4, to=1-2]
                \end{tikzcd}
                };
            \end{tikzpicture}
        \]
        The morphism $\fun{F}\fun{Q}(h_0,h_1)$ makes the first cube commute, while the universal property of the pushout provides $\fun{Q}(\fun{F}h_0,\fun{F}h_1)$ to close the second cube.
        But since $\fun{F}$ preserves the chosen pushouts and terminal objects, the two cubes are the same, and the universal property of the pushout implies the equation $\fun{F}\fun{Q}(h_0,h_1) = \fun{Q}(\fun{F}h_0,\fun{F}h_1)$, verifying the commutative square of functors on morphisms.
    \end{proof}

%    \begingroup
 %   \def\theproposition{\ref{prop: If C has coproducts and slice has weak initials, then dhom0 has joins}}
    %\begin{proposition}[Existence of joins]
     %   Let $\CategoryC$ be a category, $x$ an object of $\CategoryC$. If $\CategoryC$ has coproducts and $\slice{\CategoryC}{x}$ has a weak initial object, then $\dhom{0}{\CategoryC}{x}$ has joins.
    %\end{proposition}
    %\addtocounter{proposition}{-1}
    %\endgroup
    %\begin{proof}
       % First of all, notice that:
        %\begin{itemize}
         %   \item If $\CategoryC$ has weak coproducts, then its poset reflection $\posref{\CategoryC}$ has joins: Indeed, the coproduct of $a,b$ in $\CategoryC$ is an object $a \sqcup b$ such that there are morphisms $a \to a \sqcup b$ and $b \to a \sqcup b$, and, moreover, for any couple of morphisms $a \to c$ and $b \to c$, these morphisms factor uniquely throuch an universal morphism $a \sqcup b \to c$. Taking the poset reflection of all this, we see that $a \to a \sqcup b$ implies $\posref{a} \leq \posref{a \sqcup b}$, $b \to a \sqcup b$ implies $\posref{b} \leq \posref{a \sqcup b}$, and the universal property implies that for any $\posref{c}$ such that $\posref{a} \leq \posref{c}$ and $\posref{b} \leq \posref{c}$, we have $\posref{a \sqcup b} \leq \posref{c}$. This is the definition of a join in a poset, so $\posref{a \sqcup b} = \posref{a} \vee \posref{b}$.
          %  \item If $\slice{\CategoryC}{x}$ has a weak initial object, then then downward closure of $\posref{x}$ in $\posref{\CategoryC}$ has a least element: a weak initial object in $\slice{\CategoryC}{x}$ is a moprhism $\varepsilon_x \to x$ such that, for any other morphism $y \to x$, there is a morphism $\varepsilon_x \to y$ making the obvious triangle commute. 
           % Taking the poset reflection of $\CategoryC$, this means that $\posref{\varepsilon_x} \leq \posref{x}$, and that $\posref{y} \leq \posref{x}$ implies $\posref{\varepsilon_x} \leq \posref{y}$. Hence $\posref{\varepsilon_x}$ is the least element of the downward closure of $\posref{x}$.
        %\end{itemize}
        %Given these, it is now sufficient to prove that $\posref{\CategoryC}$ having joins and $\posref{x}$ having a least element imply that $\dhom{0}{\CategoryC}{x}$ has joins. 
        
        %\medskip
        %So, we now want to prove that for any $a,b$ in $\dhom{0}{\CategoryC}{x}$ we can build $a \vee b$ such that $a,b \leq a \vee b$, and moreover $a,b \leq c$ implies $c \leq a \vee b$.

        %Remember that a generic element of $\dhom{0}{\CategoryC}{x}$ is either $[x]$ or $\posref{y}$ for some $y$ in $\CategoryC$. Given $a,b$ in $\dhom{0}{\CategoryC}{x}$, define $\posref{\tilde{a}}$ in $\posref{\CategoryC}$ as follows: 
        %\begin{equation*}
            %\posref{\tilde{a}} \eqdef
            %\begin{cases}
            %a & \text{if $a = \posref{\tilde{a}}$,} \\
            %\posref{\varepsilon_x} & \text{if $a = [x]$,}
            %\end{cases}
        %\end{equation*}
        %where $\posref{\varepsilon_x}$ is the least element under $\posref{x}$ in $\posref{\cat{C}}$. Similarly define $\posref{\tilde{b}}$.

        %Let $\posref{\tilde{a}} \vee \posref{\tilde{b}}$ be the join of $\posref{\tilde{a}}, \posref{\tilde{b}}$ in $\posref{\CategoryC}$. By definition of poset reflection, the object $\posref{\tilde{a}} \vee \posref{\tilde{b}}$ is equal to $\posref{y}$ for some $y$ in $\CategoryC$ . We set the image of $\posref{y}$ under the quotient map $\posref{C} \surj \dhom{0}{\cat{C}}{x}$ to be the join $a \vee b$ of $a,b$ in $\dhom{0}{\CategoryC}{x}$.

        %First let us prove that $a,b \leq a \vee b$. If $a = \posref{\tilde{a}}$, then $\posref{\tilde{a}} \leq \posref{\tilde{a}} \vee \posref{\tilde{b}} = \posref{y}$ in $\posref{\cat{C}}$, and applying the quotient map, which is order-preserving, $a \leq a \vee b$ in $\dhom{0}{\cat{C}}{x}$.
        %If $a = [x]$, then by definition $\posref{\tilde{a}} = \posref{\varepsilon_x}$; since $\posref{\varepsilon_x}$ is the least element under $\posref{x}$ in $\posref{\CategoryC}$ it is $\posref{\varepsilon_x} \leq \posref{x}$, and so by definition of poset reflection there is a morphism $\varepsilon_x \to x$ in $\CategoryC$. On the other hand, again by definition of poset reflection,
        %\begin{equation*}
         %   \posref{\varepsilon_x} = \posref{\tilde{a}} \leq \posref{\tilde{a}} \vee \posref{\tilde{b}} = \posref{y}
        %\end{equation*}    
        %in $\posref{\CategoryC}$ implies the existence of a morphism $\varepsilon_x \to y$ in $\CategoryC$. So we have a span $x \leftarrow \varepsilon_x \rightarrow y$ in $\CategoryC$, and by definition of the order structure in $\dhom{0}{\CategoryC}{x}$ this implies $a = [x] \leq a \vee b$ in $\dhom{0}{\cat{C}}{x}$.
        %A similar argument proves that $b \leq a \vee b$ in $\dhom{0}{\cat{C}}{x}$.
        %\medskip
        
        %Now let $c$ be an element of $\dhom{0}{\cat{C}}{x}$, and suppose that $a,b \leq c$. We need to prove that $a \vee b \leq c$.
        %If $c = [x]$, then necessarily $a,b = [x]$ since $[x]$ is minimal in $\dhom{0}{\CategoryC}{x}$, 
        %so $\posref{\tilde{a}}, \posref{\tilde{b}} = \posref{\varepsilon_x}$. In turn, this means that 
        %
        %\begin{equation*}
            %\posref{y} = \posref{\tilde{a}} \vee \posref{\tilde{b}} = %\posref{\varepsilon_x} \vee \posref{\varepsilon_x} = %\posref{\varepsilon_x}
        %\end{equation*}
        %hence $\posref{y} = \posref{\varepsilon_x}$ is sent to $[x]$ by the quotient, and we have
        %$a \vee b = [x] = c$.

        %Suppose instead $c = \posref{z}$. By definition of the order structure in $\dhom{0}{\CategoryC}{x}$, $a \leq c$ implies that if $a = \posref{\tilde{a}}$, then $\posref{\tilde{a}} \leq \posref{z}$ in $\posref{\cat{C}}$; if instead $a = [x]$, then there exists a span of morphisms $x \leftarrow w \rightarrow z$ in $\cat{C}$, which implies that $\posref{w}$ is in the downward closure of $\posref{x}$ in $\posref{\cat{C}}$, and by definition of least element,
        %$\posref{\tilde{a}} = \posref{\varepsilon_x} \leq \posref{w} \leq \posref{z}$.
        %In any case, $\posref{\tilde{a}} \leq \posref{z}$, and a similar reasoning holds for $\posref{\tilde{b}}$; as $\posref{y}$ is defined as their join in $\posref{\CategoryC}$, its universal property implies $\posref{y} \leq \posref{z}$ in $\posref{\CategoryC}$, and applying the quotient map one gets $a \vee b \leq c$.
    %\end{proof}
    \begingroup
    \def\theproposition{\ref{prop:dhom0_trivial_when_weak_terminal}}
    \begin{proposition}
        Let $\cat{C}$ be a category and $x$ an object in $\cat{C}$.
        The following are equivalent:
        \begin{enumerate}[label=(\alph*)]
            \item $\dhom{0}{\cat{C}}{x} = \{[x]\}$;
            \item $x$ is a weak terminal object in $\cat{C}$.
        \end{enumerate}
    \end{proposition}
    \addtocounter{proposition}{-1}
    \endgroup
    \begin{proof}
        Follows from \autoref{def: weak and subterminal} and the explicit description given in \autoref{def: 0th-directed homotopy poset}.
    \end{proof}


    \begingroup
    \def\theproposition{\ref{prop: dhom0 is pi0 for groupoids}}
    \begin{proposition}[$\dhom{0}{\cat{G}}{x}$ for a groupoid]
        Let $\cat{G}$ be a groupoid and $x$ an object in $\cat{G}$.
        Then
        \begin{enumerate}
            \item $\dhom{0}{\cat{G}}{x}$ is a `set', that is, a discrete poset.
            \item As a pointed set, it is isomorphic to the set $\pi_0(\cat{G})$ of connected components of $\cat{G}$, pointed with the connected component of $x$.
        \end{enumerate}
    \end{proposition}
    \addtocounter{proposition}{-1}
    \endgroup
    \begin{proof}
        In the poset reflection of $\cat{G}$, $\posref{y} \leq \posref{z}$ if and only if there exists a morphism $f\colon y \to z$ in $\cat{G}$, and since any such morphism is invertible, then $\posref{z} \leq \posref{y}$, so $\posref{y} = \posref{z}$.
        It follows that $\posref{\cat{G}}$ is already a discrete poset, whose elements are in bijection with the connected components of $\cat{G}$, that is, with $\pi_0(\cat{G})$.
        
        The identity on $x$ factors through every morphism in $\slice{\cat{G}}{x}$, so $\posref{\slice{\cat{G}}{x}}$ is the poset with a single element, that is, the terminal poset.
        We know that the map $\posref{\mathrm{dom}}\colon \posref{\slice{\cat{G}}{x}} \to \posref{\cat{G}}$ selects the lower set of $\posref{x}$ in $\posref{\cat{G}}$, which in this case only contains $\posref{x}$ itself. 
        Hence $\posref{\mathrm{dom}}$ already exhibits $\posref{\cat{G}}$ as a pointed poset with basepoint $\posref{x}$, which the quotient leaves unaffected.
    \end{proof}

    \begingroup
    \def\thedefinition{\ref{def: category of parallel arrows}}
    \begin{definition}[Category of parallel arrows over an object]
        Let $\CategoryC$ be a category and $x$ an object in $\CategoryC$.
        The \emph{category of parallel arrows in $\CategoryC$ over $x$} is the category $\pararr{\CategoryC}{x}$ where:
        %
        %
        \begin{itemize}
            \item Objects are pairs of morphisms $(f_0, f_1\colon y \to x)$ with codomain $x$.
            \item A morphism from $(f_0, f_1\colon y \to x)$ to $(g_0, g_1\colon z \to x)$ is a morphism $h\colon y \to z$ such that $f_0 = h\Cp g_0$ and $f_1 = h\Cp g_1$.
        \end{itemize}
        This comes with a projection functor $\mathrm{dom}\colon \pararr{\CategoryC}{x} \to \CategoryC$ sending a parallel pair of morphisms to its domain.
    \end{definition}
    \addtocounter{definition}{-1}
    \endgroup
    \begin{proof}
        Compositions and identities in $\pararr{\CategoryC}{x}$ are as in $\CategoryC$. Associativity and identity laws are also inherited from $\CategoryC$, proving that $\pararr{\CategoryC}{x}$ is a category. Proving the functoriality of $\mathrm{dom}$ is equally trivial.
    \end{proof}

    \begingroup
    \def\theproposition{\ref{prop: subterminal as weak terminal parallel arrow}}
    \begin{proposition}
        Let $\CategoryC$ be a category and $\WTerm$ an object in $\CategoryC$.
        The following are equivalent:
        %
        %
        \begin{enumerate}[label=(\alph*)]
            \item $\WTerm$ is subterminal in $\CategoryC$.
            \item $(\idd{\WTerm}, \idd{\WTerm})$ is a terminal object in $\pararr{\CategoryC}{\WTerm}$;
            \item $(\idd{\WTerm}, \idd{\WTerm})$ is a weak terminal object in $\pararr{\CategoryC}{\WTerm}$.
        \end{enumerate}
    \end{proposition}
    \addtocounter{proposition}{-1}
    \endgroup
    \begin{proof}
        Let us begin by proving that the first and second point are equivalent. For any couple of morphisms $f,g: x \to \WTerm$, consider the following diagram, corresponding to the pair of objects $(f, g)$ and $(\idd{\WTerm}, \idd{\WTerm})$ in $\pararr{\CategoryC}{\WTerm}$:
        %
        %
        \[\begin{tikzcd}[sep=scriptsize]
            x && \WTerm \\
            \\
            & \WTerm
            \arrow["f"', curve={height=6pt}, from=1-1, to=3-2]
            \arrow["g", curve={height=-6pt}, from=1-1, to=3-2]
            \arrow["{\idd{\WTerm}}"', curve={height=6pt}, from=1-3, to=3-2]
            \arrow["{\idd{\WTerm}}", curve={height=-6pt}, from=1-3, to=3-2]
        \end{tikzcd}\]
        %
        If $\WTerm$ is subterminal in $\CategoryC$, then $f = g$, so $f$ determines a morphism $(f, g) \to (\idd{\WTerm}, \idd{\WTerm})$ in $\pararr{\CategoryC}{\WTerm}$, proving weak terminality of $(\idd{\WTerm}, \idd{\WTerm})$:
        \[\begin{tikzcd}[sep=scriptsize]
            x && \WTerm \\
            \\
            & \WTerm
            \arrow["f"', curve={height=6pt}, from=1-1, to=3-2]
            \arrow["f", curve={height=-6pt}, from=1-1, to=3-2]
            \arrow["{\idd{\WTerm}}"', curve={height=6pt}, from=1-3, to=3-2]
            \arrow["{\idd{\WTerm}}", curve={height=-6pt}, from=1-3, to=3-2]
            \arrow["f", from=1-1, to=1-3]
        \end{tikzcd}\]
        Conversely, weak terminality in $\pararr{\CategoryC}{\WTerm}$ implies that there is always some $h\colon x \to \WTerm$ such that 
        \[\begin{tikzcd}[sep=scriptsize]
            x && \WTerm \\
            \\
            & \WTerm
            \arrow["f"', curve={height=6pt}, from=1-1, to=3-2]
            \arrow["g", curve={height=-6pt}, from=1-1, to=3-2]
            \arrow["{\idd{\WTerm}}"', curve={height=6pt}, from=1-3, to=3-2]
            \arrow["{\idd{\WTerm}}", curve={height=-6pt}, from=1-3, to=3-2]
            \arrow["h", from=1-1, to=1-3]
        \end{tikzcd}\]
        commutes, implying
        \begin{equation*}
            f = h \Cp \Id{\WTerm} = g
        \end{equation*}
        %
        hence subterminality of $\WTerm$ in $\CategoryC$.

        Now we prove that the last two points are equivalent. One direction is trivial. As for the other one, suppose $(\idd{\WTerm}, \idd{\WTerm})$ is weak terminal in $\pararr{\CategoryC}{\WTerm}$, and suppose that both $h, k\colon x \to \WTerm$ make the diagram above commute. Then we have
        %
        %
        \begin{equation*}
            f = h \Cp \Id{\WTerm} = h \quad f = k \Cp \Id{\WTerm} = k
        \end{equation*}
        %
        hence $h=k$, completing the proof.
    \end{proof}
    
    \begingroup
    \def\theproposition{\ref{prop:dhom1_trivial_when_subterminal}}
    \begin{proposition}
        Let $\cat{C}$ be a category and $x$ an object in $\cat{C}$.
        The following are equivalent:
        \begin{enumerate}[label=(\alph*)]
            \item $\dhom{1}{\cat{C}}{x} = \{[x]\}$;
            \item $x$ is subterminal in $\cat{C}$.
        \end{enumerate}
    \end{proposition}
    \addtocounter{proposition}{-1}
    \endgroup
    \begin{proof}
        Follows from \autoref{def: weak and subterminal} and the explicit description given in \autoref{def: 1st-directed homotopy poset}.
    \end{proof}

    \begingroup
    \def\thecorollary{\ref{prop:dhoms_trivial_when_terminal}}
    \begin{corollary}
        Let $\cat{C}$ be a category and $x$ an object in $\cat{C}$.
        The following are equivalent:
        \begin{enumerate}[label=(\alph*)]
            \item $\dhom{0}{\cat{C}}{x} = \{[x]\}$ and $\dhom{1}{\cat{C}}{x} = \{[x]\}$,
            \item $x$ is a terminal object in $\cat{C}$.
        \end{enumerate}
    \end{corollary}
    \addtocounter{corollary}{-1}
    \endgroup
    \begin{proof}
        $x$ is a terminal object in $\cat{C}$ if and only if it is weak terminal and subterminal. Combine this with \autoref{prop:dhom0_trivial_when_weak_terminal} and \autoref{prop:dhom1_trivial_when_subterminal}.
    \end{proof}


    \begingroup
    \def\theproposition{\ref{prop: dhom1 is pi1 for groupoids}}
    \begin{proposition}[$\dhom{1}{\cat{G}}{x}$ for a groupoid]
        Let $\cat{G}$ be a groupoid and $x$ an object in $\cat{G}$.
        Then:
        \begin{enumerate}
            \item $\dhom{1}{\cat{G}}{x}$ is a `set', that is, a discrete poset.
            \item As a pointed set, it is isomorphic to the underlying pointed set of the group $\pi_1(\cat{G}, x) = \homset{\cat{G}}{x}{x}$.
        \end{enumerate}
    \end{proposition}
    \addtocounter{proposition}{-1}
    \endgroup
    \begin{proof}
        First of all, when $\cat{G}$ is a groupoid, so is $\pararr{\cat{G}}{x}$, so the fact that $\dhom{1}{\cat{G}}{x}$ is discrete follows from the first point of \autoref{prop: dhom0 is pi0 for groupoids}.

        As for the second point, we define a function $\dhom{1}{\cat{G}}{x} \to \homset{\cat{G}}{x}{x}$ by
        \begin{equation*}
            [x] \mapsto \idd{x}, \quad \quad \posref{(f, g)} \mapsto \invrs{g}\Cp f.
        \end{equation*}
        First of all, this is well-defined, in the sense that it is independent of the representative of $\posref{(f, g)}$: if $\posref{(f, g)} = \posref{(f', g')}$, then there exists $h$ such that $f = h\Cp f'$ and $g = h\Cp g'$, hence
        \begin{equation*}
            \invrs{g}\Cp f = \invrs{(h\Cp g')}\Cp (h\Cp f') = \invrs{g'}\Cp (\invrs{h}\Cp h)\Cp f' = \invrs{g'}\Cp f'.
        \end{equation*}
        Moreover, it is by construction a map of pointed sets.
        It is surjective: $\idd{x}$ is the image of $[x]$, and any $f\colon x \to x$ such that $f \neq \idd{x}$ is the image of $\posref{(\idd{x}, f)}$.
        
        To conclude, we need to prove that the function is also injective. Suppose that $(f, g\colon y \to x)$ and $(f', g'\colon z \to x)$ are two parallel pairs such that $\invrs{g}\Cp f = \invrs{g'}\Cp f'$.
        Then let $h\colon y \to z$ be defined by
        \begin{equation*}
            h \eqdef g\Cp \invrs{g'} = g \Cp \invrs{g'} \Cp \Id{z} =   g\Cp \invrs{g'}\Cp f'\Cp \invrs{f'} = g\Cp \invrs{g}\Cp f\Cp \invrs{f'} = f\Cp \invrs{f'}.
        \end{equation*}
        We have that $h\Cp  f' = f\Cp \invrs{f'}\Cp f' = f$ and $h\Cp  g' = g\Cp \invrs{g'}\Cp g' = g$, so $h$ is a morphism from $(f, g)$ to $(f', g')$ in $\pararr{\cat{G}}{x}$. Since $\cat{G}$ is a groupoid, $h$ is invertible, and $\posref{(f, g)} = \posref{(f', g')}$.
        This proves that the function is injective.
    \end{proof}
    %
    %
\subsubsection{Proof of Proposition 10}

    Proving~\autoref{prop: Homotopy posets are functorial} requires to build a hefty amount of theory, which is why we reserved a subsection of the Appendix only to this.

    \begin{definition}[Past extension] \label{dfn:past_extension}
    Let $\cat{A}$ be a category.
    A \emph{past extension of $\cat{A}$} is a functor $\imath\colon \cat{A} \incl \cat{B}$ with the following property: there exists a functor $\indic{\cat{A}}\colon \cat{B} \to \wkarr$ such that
    \begin{equation} \label{eq:past_extension}
    \begin{tikzcd}[sep=scriptsize]
	\cat{A} && \Term \\
	\\
	\cat{B} && \wkarr
	\arrow["{!}", from=1-1, to=1-3]
	\arrow[hook, "\imath", from=1-1, to=3-1]
	\arrow["\indic{\cat{A}}", from=3-1, to=3-3]
	\arrow[hook, "{1}", from=1-3, to=3-3]
	\arrow["\lrcorner"{anchor=center, pos=0.125}, draw=none, from=1-1, to=3-3]
    \end{tikzcd}
    \end{equation}
    is a pullback in $\catcat$.
    \end{definition}
    %
    %
    \begin{remark} \label{rmk:past extension collage}
    The following is an equivalent characterisation of past extensions: there exist a category $\cat{\bar{A}}$ and a profunctor $\fun{H}\colon \opp{\cat{\bar{A}}} \times \cat{A} \to \catset$ such that
    \begin{enumerate}
        \item $\cat{B}$ is isomorphic to the \emph{collage}, also known as \emph{cograph}, of $\fun{H}$, and
        \item $\imath$ is, up to isomorphism, the inclusion of $\cat{A}$ into the collage.
    \end{enumerate}
    A technical name for a functor satisfying the condition on $\imath$ is \emph{codiscrete coopfibration}; it is one leg of a two-sided codiscrete cofibration of categories.

    The idea is that $\imath$ embeds $\cat{A}$ into a larger category, whose objects outside of the image of $\cat{A}$ only have morphisms pointing \emph{towards} $\cat{A}$, hence are ``in the past'' of $\cat{A}$ if we interpret the direction of morphisms as a time direction.
    Notice that the fact that (\ref{eq:past_extension}) is a pullback implies that $\imath$ is injective on objects and morphisms, using their representation as functors from $\Term$ and $\wkarr$, respectively.
    
    The following picture illustrates the bipartition of $\cat{B}$ induced by $\indic{A}$, with the fibre $\cat{\bar{A}}$ of 0 ``in the past'' of the fibre $\cat{A}$ of 1:
    \[\begin{tikzcd}[sep=scriptsize]
	{\cat{B}} & {\blue{\cat{\bar{A}}}} & {\blue{\bullet}} &&& {\red{\bullet}} & {\red{\cat{A}}} \\
	& {\blue{\bullet}} &&&&& {\red{\bullet}} \\
	&& {\blue{\bullet}} &&& {\red{\bullet}} \\
	\wkarr && {\blue{0}} &&& {\red{1}}
	\arrow[color={rgb,255:red,92;green,92;blue,214}, curve={height=-6pt}, from=2-2, to=1-3]
	\arrow[color={rgb,255:red,92;green,92;blue,214}, curve={height=6pt}, from=3-3, to=1-3]
	\arrow[color={rgb,255:red,92;green,92;blue,214}, curve={height=-6pt}, from=2-2, to=3-3]
	\arrow[curve={height=-6pt}, from=1-3, to=1-6]
	\arrow[color={rgb,255:red,214;green,92;blue,92}, curve={height=-6pt}, from=1-6, to=2-7]
	\arrow["{\indic{A}}", from=1-1, to=4-1]
	\arrow[color={rgb,255:red,214;green,92;blue,92}, curve={height=-6pt}, from=1-6, to=3-6]
	\arrow[curve={height=-6pt}, from=3-3, to=3-6]
	\arrow[curve={height=-18pt}, from=1-3, to=3-6]
	\arrow["a", from=4-3, to=4-6]
    \end{tikzcd}\]
    \end{remark}
    %
    %
    \begin{definition}[Category of past extensions] \label{dfn:category_of_past_extensions}
    Let $\cat{A}$ be a category.
    The \emph{category of past extensions of $\cat{A}$} is the large category $\pastext{\cat{A}}$ whose
    \begin{itemize}
        \item objects are past extensions $\imath\colon A \incl B$, and
        \item a morphism from $(\imath\colon A \incl B)$ to $(j\colon A \incl B')$ is a factorisation of $j$ through $\imath$, that is, a functor $\fun{K}\colon \cat{B} \to \cat{B'}$ such that $j = \imath\Cp \fun{K}$.
    \end{itemize}
    \end{definition}
    %
    %
    \begin{proposition}[The indexed category of past extensions of functors]
    Let $\cat{A}$ and $\cat{C}$ be categories.
    Then there exists a functor
    \begin{equation*}
        \extfun{\cat{A}}{\cat{C}}\colon \opp{\pastext{\cat{A}}} \times \cat{C}^\cat{A} \to \catcat
    \end{equation*}
    whose object part is defined as follows: 
    given a past extension $\imath\colon \cat{A} \incl \cat{B}$ and a functor $\fun{F}\colon \cat{A} \to \cat{C}$, the category $\extfun{\cat{A}}{\cat{C}}(\imath, \fun{F})$ is the subcategory of $\cat{C}^\cat{B}$ whose
    \begin{itemize}
        \item objects are (strict) extensions of $\fun{F}$ along $\imath$, that is, functors $\fun{\tilde{F}}\colon \cat{B} \to \cat{C}$ such that
        \[\begin{tikzcd}[sep=scriptsize]
	   {\cat{A}} && {\cat{C}} \\
	   \\
	   {\cat{B}}
	   \arrow["\imath", hook, from=1-1, to=3-1]
	   \arrow["{\fun{F}}", from=1-1, to=1-3]
	   \arrow["{\fun{\tilde{F}}}"', from=3-1, to=1-3]
    \end{tikzcd}\]
    strictly commutes, and
    \item morphisms from $\fun{\tilde{F}_1}$ to $\fun{\tilde{F}_2}$ are natural transformations $\tau\colon \fun{\tilde{F}_1} \Rightarrow \fun{\tilde{F}_2}$ that restrict along $\imath$ to the identity natural transformation on $\fun{F}$.
    \end{itemize}
    \end{proposition}
    %
    \begin{remark}
    Before giving a detailed proof, we give a more high-level explanation of where $\extfun{\cat{A}}{\cat{C}}$ comes from.
    When $\imath\colon \cat{A} \incl \cat{B}$ is a coopfibration of categories and $\cat{C}$ a category, exponentiating $\imath$ always produces an opfibration $\imath^*\colon \cat{C}^\cat{B} \to \cat{C}^\cat{A}$. 
    In particular, when $\imath$ is a past extension, $\imath^*$ can be endowed with a canonical splitting, defining a contravariant functor
    \begin{equation*}
        \opp{\pastext{\cat{A}}} \to \lcat{SpOpfib}(\cat{C}^\cat{A})
    \end{equation*}
    to the large category of split opfibrations and split cocartesian morphisms over $\cat{C}^\cat{A}$.
    Via the Grothendieck construction, the latter is equivalent to the large category of functors $\catcat^{\cat{C}^\cat{A}}$.
    Composing with the inverse of the Grothendieck construction, we thus obtain a functor
    \begin{equation*}
        \opp{\pastext{\cat{A}}} \to \catcat^{\cat{C}^\cat{A}},
    \end{equation*}
    whose uncurried version is, up to natural isomorphism, our $\extfun{\cat{A}}{\cat{C}}$.
    \end{remark}
    %
    \begin{proof}
        Given a morphism $\fun{K}\colon (\imath\colon \cat{A} \incl \cat{B}) \to (j\colon \cat{A} \incl \cat{B'})$ in $\pastext{\cat{A}}$,
\begin{equation*}
    \fun{K}^* \eqdef \extfun{\cat{A}}{\cat{C}}(\fun{K},\fun{F})\colon \extfun{\cat{A}}{\cat{C}}(j, \fun{F}) \to \extfun{\cat{A}}{\cat{C}}(\imath, \fun{F})
\end{equation*}
is the functor that acts by precomposition, sending
\begin{itemize}
    \item $\fun{\tilde{F}}\colon \cat{B'} \to \cat{C}$ to $\fun{K}\Cp\fun{\tilde{F}}\colon \cat{B} \to \cat{C}$, and
    \item $\tau\colon \fun{\tilde{F}_1} \Rightarrow \fun{\tilde{F}_2}$ to $\fun{K}\Cp\tau\colon \fun{K}\Cp\fun{\tilde{F}_1} \Rightarrow \fun{K}\Cp\fun{\tilde{F}_2}$.
\end{itemize}
This is well-defined as
\begin{equation*}
    \imath\Cp\fun{K}\Cp\fun{\tilde{F}} = j\Cp\fun{\tilde{F}} = \fun{F}, \quad \quad
    \imath\Cp\fun{K}\Cp\tau = j\Cp\tau = \idd{\fun{F}}.
\end{equation*}
Moreover, it is straightforward to check that
\begin{equation*}
    (\idd{\imath})^* = \idd{\extfun{\cat{A}}{\cat{C}}(\imath, \fun{F})}, \quad \quad 
    (\fun{K}\Cp\fun{L})^* = \fun{L}^*\Cp \fun{K}^*
\end{equation*}
for any composable pair $\fun{K}, \fun{L}$ of morphisms in $\pastext{\cat{A}}$.

Given a natural transformation $\alpha\colon \fun{F} \Rightarrow \fun{G}$ between functors $\fun{F}, \fun{G}\colon \cat{A} \to \cat{C}$, the functor
\begin{equation*}
    \alpha_* \eqdef \extfun{\cat{A}}{\cat{C}}(\imath, \alpha)\colon \extfun{\cat{A}}{\cat{C}}(\imath, \fun{F}) \to \extfun{\cat{A}}{\cat{C}}(\imath, \fun{G})
\end{equation*}
is defined as follows.
Given an object $\fun{\tilde{F}}\colon \cat{B} \to \cat{C}$ of $\extfun{\cat{A}}{\cat{C}}(\imath, \fun{F})$, the functor $\alpha_*\fun{\tilde{F}}\colon \cat{B} \to \cat{C}$ is defined, on each morphism $f\colon x \to y$ in $\cat{B}$, by
\begin{equation*}
    \alpha_*\fun{\tilde{F}}(f) \eqdef
    \begin{cases}
        \fun{G}(f')
            & \text{if $\indic{A}(f) = 1$ and $f = \imath(f')$}, \\
        \fun{\tilde{F}}(f)\Cp\alpha_{y'}
            & \text{if $\indic{A}(f) = a$ and $y = \imath(y')$}, \\
        \fun{\tilde{F}}(f)
            & \text{if $\indic{A}(f) = 0$}.
    \end{cases}
\end{equation*}
By construction $\imath\Cp \alpha_*\fun{\tilde{F}} = \fun{G}$.
The following picture illustrates the definition.
\[\begin{tikzcd}[sep=scriptsize]
	&&&&& {\magenta{\fun{G}y}} & {\magenta{\fun{G}\cat{A}}} \\
	&&&&&& {\magenta{\bullet}} \\
	{\blue{\fun{\tilde{F}}\cat{\bar{A}} = \alpha_*\fun{\tilde{F}}\cat{\bar{A}}}} & {\blue{\fun{\tilde{F}}x}} &&& {\red{\fun{F}y}} & {\magenta{\bullet}} \\
	{\blue{\bullet}} &&&&& {\red{\bullet}} \\
	& {\blue{\bullet}} &&& {\red{\bullet}} & {\red{\fun{F}\cat{A}}}
	\arrow[color={rgb,255:red,92;green,92;blue,214}, curve={height=-6pt}, from=4-1, to=3-2]
	\arrow[color={rgb,255:red,92;green,92;blue,214}, curve={height=6pt}, from=5-2, to=3-2]
	\arrow[color={rgb,255:red,92;green,92;blue,214}, curve={height=-6pt}, from=4-1, to=5-2]
	\arrow["{\fun{\tilde{F}}f}"', curve={height=-6pt}, from=3-2, to=3-5]
	\arrow[color={rgb,255:red,214;green,92;blue,92}, curve={height=-6pt}, from=3-5, to=4-6]
	\arrow[color={rgb,255:red,214;green,92;blue,92}, curve={height=-6pt}, from=3-5, to=5-5]
	\arrow[curve={height=-6pt}, from=5-2, to=5-5]
	\arrow[curve={height=-18pt}, from=3-2, to=5-5]
	\arrow[color={rgb,255:red,214;green,92;blue,214}, curve={height=-6pt}, from=1-6, to=2-7]
	\arrow[color={rgb,255:red,214;green,92;blue,214}, curve={height=-6pt}, from=1-6, to=3-6]
	\arrow["{\alpha_y}"', color={rgb,255:red,36;green,143;blue,36}, from=3-5, to=1-6]
	\arrow[color={rgb,255:red,36;green,143;blue,36}, from=4-6, to=2-7]
	\arrow[color={rgb,255:red,36;green,143;blue,36}, from=5-5, to=3-6]
	\arrow["{\alpha_*\fun{\tilde{F}}f}", curve={height=-12pt}, dashed, from=3-2, to=1-6]
	\arrow[curve={height=-18pt}, dashed, from=3-2, to=3-6]
	\arrow[curve={height=-12pt}, dashed, from=5-2, to=3-6]
\end{tikzcd}\]
Let us show that $\alpha_*\fun{\tilde{F}}$ is well-defined as a functor.
\begin{enumerate}
    \item Given an identity $\idd{x}$ in $\cat{B}$, necessarily $\indic{A}(\idd{x}) = 0$, in which case 
    \[ \alpha_*\fun{\tilde{F}}(\idd{x}) = \fun{\tilde{F}}(\idd{x}) = \idd{\fun{\tilde{F}}(x)}, \] 
    or $\indic{A}(\idd{x}) = 1$, in which case
    \[ \alpha_*\fun{\tilde{F}}(\idd{x}) = \fun{G}(\idd{x'}) = \idd{\fun{G}(x')}, \]
    where $x'$ is the unique lift of $x$ to $\cat{A}$.
    Thus $\alpha_*\fun{\tilde{F}}$ preserves identities.

    \item Given a composable pair $f\colon x \to y$, $g\colon y \to z$, we have the following cases.
    \begin{itemize}
        \item If $\indic{A}(f) = \indic{A}(g) = 1$, then $\indic{A}(f\Cp g) = 1$, and
        \[
            \alpha_*\fun{\tilde{F}}(f)\Cp \alpha_*\fun{\tilde{F}}(g) = \fun{G}(f')\Cp\fun{G}(g') = \fun{G}(f'\Cp g') = \alpha_*\fun{\tilde{F}}(f\Cp g),
        \]
        where $f', g'$ are the unique lifts of $f, g$ to $\cat{A}$.
        \item If $\indic{A}(f) = \indic{A}(g) = 0$, then $\indic{A}(f\Cp g) = 0$, and
        \[
            \alpha_*\fun{\tilde{F}}(f)\Cp  \alpha_*\fun{\tilde{F}}(g) = \fun{\tilde{F}}(f)\Cp \fun{\tilde{F}}(g) = \fun{\tilde{F}}(f\Cp g) = \alpha_*\fun{\tilde{F}}(f\Cp g).
        \]
        \item If $\indic{A}(f) = 0$ and $\indic{A}(g) = a$, then $\indic{A}(f\Cp g) = a$, and
        \[
            \alpha_*\fun{\tilde{F}}(f)\Cp  \alpha_*\fun{\tilde{F}}(g) = \fun{\tilde{F}}(f)\Cp \fun{\tilde{F}}(g)\Cp \alpha_{z'} = \fun{\tilde{F}}(f\Cp g)\Cp \alpha_{z'} = \alpha_*\fun{\tilde{F}}(f\Cp g),
        \]
        where $z'$ is the unique lift of $z$ to $\cat{A}$.
        \item If $\indic{A}(f) = a$ and $\indic{A}(g) = 1$, then $\indic{A}(f\Cp g) = a$, and
        \[
            \alpha_*\fun{\tilde{F}}(f)\Cp  \alpha_*\fun{\tilde{F}}(g) = \fun{\tilde{F}}(f)\Cp \alpha_{y'}\Cp \fun{G}(g') = \fun{\tilde{F}}(f)\Cp \fun{F}(g')\Cp \alpha_{z'},
        \]
        where $g'\colon y' \to z'$ is the unique lift of $g$ to $\cat{A}$, and we used naturality of $\alpha$.
        
        Since $\fun{F}(g') = \tilde{\fun{F}}(\imath(g')) = \tilde{\fun{F}}(g)$, this is equal to 
        \[ \fun{\tilde{F}}(f)\Cp \fun{\tilde{F}}(g)\Cp \alpha_{z'} = \alpha_*\fun{\tilde{F}}(f\Cp g). \]
    \end{itemize}
    No other cases are possible.
\end{enumerate}
This proves that $\alpha_*\fun{\tilde{F}}$ is well-defined.

Given a morphism $\tau\colon \fun{\tilde{F}_1} \Rightarrow \fun{\tilde{F}_2}$ of $\extfun{\cat{A}}{\cat{C}}(\imath, \fun{F})$, the natural transformation $\alpha_*\tau\colon \alpha_*\fun{\tilde{F}_1} \Rightarrow \alpha_*\fun{\tilde{F}_2}$ is defined, on each object $x$ in $\cat{B}$, by
\begin{equation*}
    (\alpha_*\tau)_x \eqdef
    \begin{cases}
        \idd{\fun{G}(x')} & \text{if $\indic{A}(x) = 1$ and $x = \imath(x')$}, \\
        \tau_x & \text{if $\indic{A}(x) = 0$}.
    \end{cases}
\end{equation*}
To show that this is well-defined as a natural transformation, consider a morphism $f\colon x \to y$ in $\cat{B}$.
\begin{itemize}
    \item If $\indic{A}(f) = 1$ and $f'\colon x' \to y'$ is the unique lift of $f$ to $\cat{A}$, then
    \[
        \alpha_*\fun{\tilde{F}_1}(f)\Cp  (\alpha_*\tau)_y =
        \fun{G}(f')\Cp  \idd{\fun{G}(y')} = \idd{\fun{G}(x')}\Cp  \fun{G}(f') = (\alpha_*\tau)_x\Cp  \alpha_*\fun{\tilde{F}_2}(f).
    \]
    \item If $\indic{A}(f) = a$ and $y'$ is the unique lift of $y$ to $\cat{A}$, then
    \[
        \alpha_*\fun{\tilde{F}_1}(f)\Cp  (\alpha_*\tau)_y =
        \fun{\tilde{F}_1}(f)\Cp  \alpha_{y'}\Cp  \idd{\fun{G}(y')} = 
        \fun{\tilde{F}_1}(f)\Cp  \tau_{y}\Cp  \alpha_{y'}
    \]
    since $\tau_y = \tau_{\imath(y')} = \idd{\fun{F}(y')}$.
    By naturality of $\tau$, this is equal to
    \[
        \tau_x\Cp  \fun{\tilde{F}_2}(f)\Cp  \alpha_{y'} =
        (\alpha_*\tau)_x\Cp  \alpha_*\fun{\tilde{F}_2}(f).
    \]
    \item If $\indic{A}(f) = 0$, then
    \[
        \alpha_*\fun{\tilde{F}_1}(f)\Cp  (\alpha_*\tau)_y =
        \fun{\tilde{F}_1}(f)\Cp  \tau_{y} = \tau_x\Cp  \fun{\tilde{F}_2}(f) = (\alpha_*\tau)_x\Cp  \alpha_*\fun{\tilde{F}_2}(f).
    \]
\end{itemize}
This concludes the definition of $\alpha_*$.
It is straightforward to check that
\begin{equation*}
    (\idd{\fun{F}})_* = \idd{\extfun{\cat{A}}{\cat{C}}(\imath, \fun{F})}, \quad \quad 
    (\alpha\Cp \beta)_* = \alpha_* \Cp  \beta_*
\end{equation*}
for all pairs of natural transformations $\alpha, \beta$ composable as morphisms in $\cat{C}^\cat{A}$.
Finally, one can verify that, for all morphisms $\fun{K}\colon \imath \to j$ in $\pastext{\cat{A}}$ and $\alpha\colon \fun{F} \to \fun{G}$ in $\cat{C}^\cat{A}$, the diagram of functors
\[
\begin{tikzcd}[sep=scriptsize]
	\extfun{\cat{A}}{\cat{C}}(j, \fun{F}) && \extfun{\cat{A}}{\cat{C}}(\imath, \fun{F}) \\
	\\
	\extfun{\cat{A}}{\cat{C}}(j, \fun{G}) && \extfun{\cat{A}}{\cat{C}}(\imath, \fun{G})
	\arrow["\fun{K}^*", from=1-1, to=1-3]
	\arrow["\alpha_*", from=1-1, to=3-1]
	\arrow["\fun{K}^*", from=3-1, to=3-3]
	\arrow["\alpha_*", from=1-3, to=3-3]
\end{tikzcd}
\]
commutes in $\catcat$.
Thus we can define $\extfun{\cat{A}}{\cat{C}}(\fun{K}, \alpha)$ as either path in the commutative diagram, and conclude that $\extfun{\cat{A}}{\cat{C}}$ is well-defined as a functor.
    \end{proof}

    \begin{proposition}[Covariance of the $\extfun{\cat{A}}{\cat{C}}$]
    The assignment $\cat{C} \mapsto \extfun{\cat{A}}{\cat{C}}$ extends to a functor
    \begin{equation*}
        \extfun{\cat{A}}{}\colon \catcat \to \laxslice{\lcatcat}{\catcat}.
    \end{equation*}
    \end{proposition}
    \begin{proof}
    Given a functor $\fun{P}\colon \cat{C} \to \cat{D}$, post-composition with $\fun{P}$ defines a functor $\fun{P}_*\colon \cat{C}^\cat{A} \to \cat{D}^\cat{A}$.
    Then there is a natural transformation
    \begin{equation} \label{eq:naturality_extfun}
    \begin{tikzcd}
	{\opp{\pastext{\cat{A}}} \times \cat{C}^\cat{A}} &&& \catcat \\
	\\
	{\opp{\pastext{\cat{A}}} \times \cat{D}^\cat{A}}
	\arrow["{\idd{} \times \fun{P}_*}"', from=1-1, to=3-1]
	\arrow["{\extfun{\cat{A}}{\cat{D}}}"', from=3-1, to=1-4]
	\arrow[""{name=0, anchor=center, inner sep=0}, "{\extfun{\cat{A}}{\cat{C}}}", from=1-1, to=1-4]
	\arrow["{\extfun{\cat{A}}{\fun{P}}}"', shorten <=17pt, shorten >=26pt, Rightarrow, from=0, to=3-1]
    \end{tikzcd}
    \end{equation}
    defined as follows: given a past extension $\imath\colon \cat{A} \incl \cat{B}$ and a functor $\fun{F}\colon \cat{A} \to \cat{C}$, the functor
    \begin{equation*}
        \extfun{\cat{A}}{\fun{P}}(\imath, \fun{F})\colon \extfun{\cat{A}}{\cat{C}}(\imath, \fun{F}) \to \extfun{\cat{A}}{\cat{D}}(\imath, \fun{F}\Cp \fun{P})
    \end{equation*}
    acts both on objects and on morphisms by post-composition with $\fun{P}$.
    It is straightforward to check that the assignment $\fun{P} \mapsto \extfun{\cat{A}}{\fun{P}}$ respects identities and composition in $\catcat$.
    \end{proof}

    \begin{remark}[General functoriality pattern] \label{rmk: functoriality pattern}
    A fixed morphism $\fun{K}$ in $\pastext{\cat{A}}$ is classified by a functor $\wkarr \to \pastext{\cat{A}}$.
    Evaluating $\extfun{\cat{A}}{\cat{C}}$ at $\fun{K}$ thus determines a functor
    \begin{equation*}
        \extfun{\cat{A}}{\cat{C}}(\fun{K}, -)\colon \wkarr \times \cat{C}^\cat{A} \to \catcat,
    \end{equation*}
    which we can curry to obtain a functor
    \begin{equation} \label{eq:general_pattern_cat}
        \Lambda.\extfun{\cat{A}}{\cat{C}}(\fun{K}, -)\colon \cat{C}^\cat{A} \to \catcat^\wkarr.
    \end{equation}
    Given a functor $\fun{P}\colon \cat{C} \to \cat{D}$, we can also ``curry the natural transformation'' in (\ref{eq:naturality_extfun}) to obtain a diagram
    \begin{equation} \label{eq:general_functor_covariance}
    \begin{tikzcd}[sep=large]
	{\cat{C}^\cat{A}} &&& \catcat^\wkarr \\
	\\
	{\cat{D}^\cat{A}}
	\arrow["{\fun{P}_*}"', from=1-1, to=3-1]
	\arrow["{\Lambda.\extfun{\cat{A}}{\cat{D}}(\fun{K},-)}"', from=3-1, to=1-4]
	\arrow[""{name=0, anchor=center, inner sep=0}, "{\Lambda.\extfun{\cat{A}}{\cat{C}}(\fun{K},-)}", from=1-1, to=1-4]
	\arrow["{\Lambda.\extfun{\cat{A}}{\fun{P}}(\fun{K},-)}"{description}, shorten <=17pt, shorten >=26pt, Rightarrow, from=0, to=3-1]
    \end{tikzcd}
    \end{equation}
    which is part of a functor $\catcat \to \laxslice{\lcatcat}{\catcat^\wkarr}$.

    Post-composing with the functor $\catcat^\wkarr \to \pointed{\catpos}$ from (\ref{eq: quotient and posref}) we obtain a covariant family of functors $\cat{C}^\cat{A} \to \pointed{\catpos}$.

    We will show that, for suitable choices of $\cat{A}$ and $\fun{K}$, the image of these functors is included in the subcategory of $\pointed{\catpos}$ on the zeroth and first homotopy posets of $\cat{C}$ or categories associated with $\cat{C}$, exhibiting various kinds of functorial dependence of homotopy posets.
    \end{remark}
    
    \begingroup
    \def\theproposition{\ref{prop: Homotopy posets are functorial}}
    \begin{proposition}[Functoriality of the homotopy posets]
        Let $\cat{C}$ be a category, $i \in \{0, 1\}$.
        Then:
        \begin{enumerate}
            \item the assignment $x \mapsto \dhom{i}{\cat{C}}{x}$ extends to a functor
                $\dhom{i}{\cat{C}}{-}\colon \cat{C} \to \pointed{\catpos}$;
            \item a functor $\fun{F}\colon \cat{C} \to \cat{D}$ induces a natural transformation 
                $\pi_i(\fun{F})\colon \dhom{i}{\cat{C}}{-} \Rightarrow \dhom{i}{\cat{D}}{\fun{F}-}.$
        \end{enumerate}
        Given another functor $\fun{G}\colon \cat{D} \to \cat{E}$,  this assignment satisfies
        \begin{equation*}
            \pi_i(\fun{F}\Cp \fun{G}) = \pi_i(\fun{F}) \Cp \pi_i(\fun{G}), \quad \quad \pi_i(\idd{C}) = \idd{\dhom{i}{\cat{C}}{-}}.
        \end{equation*}
    \end{proposition}
    \addtocounter{proposition}{-1}
    \endgroup
    \begin{proof}
    We will derive the results for both $i \in \{0, 1\}$ from the general functoriality pattern of \autoref{rmk: functoriality pattern}.
    
    First we consider the case $i = 0$.
    Let $\Term$ be the terminal category.
    The inclusion $\fun{K_0}$ of the endpoints of the walking arrow induces a morphism in $\pastext{\Term}$, depicted as follows:
    \begin{equation*}
    \begin{tikzcd}[sep=small]
	{\blue{\bullet}} & {} & {\red{\bullet}} &&& {\blue{\bullet}} & {} & {\red{\bullet}}
	\arrow[from=1-6, to=1-8]
	\arrow[color={rgb,255:red,102;green,102;blue,102}, curve={height=-24pt}, shorten <=15pt, shorten >=15pt, dashed, hook, from=1-2, to=1-7]
    \end{tikzcd} \quad \quad
    \begin{tikzcd}[sep=scriptsize]
	& \Term \\
	\\
	{\Term+\Term} && \wkarr
	\arrow["{\fun{K_0}}"', hook, from=3-1, to=3-3]
	\arrow["{\imath_1}"', hook, from=1-2, to=3-1]
	\arrow["1", hook, from=1-2, to=3-3]
    \end{tikzcd}\end{equation*}
    We claim that, up to isomorphism of categories,
    \begin{equation*}
        \Lambda.\extfun{\Term}{\cat{C}}(\fun{K_0}, -)\colon \cat{C}^1 \to \catcat^\wkarr
    \end{equation*}
    sends an object $x$ of $\cat{C}^\Term$ --- which is, equivalently, an object of $\cat{C}$ --- to the slice projection functor
    \begin{equation*}
        \mathrm{dom}\colon \slice{\cat{C}}{x} \to \cat{C}.
    \end{equation*}
    The domain of $\Lambda.\extfun{\Term}{\cat{C}}(\fun{K_0}, x)$ is the category $\extfun{\Term}{\cat{C}}(1, x)$ whose
    \begin{itemize}
        \item objects are functors $f\colon \wkarr \to \cat{C}$ such that 
    \[\begin{tikzcd}[sep=scriptsize]
	   {\Term} && {\cat{C}} \\
	   \\
	   {\wkarr}
	   \arrow["1", hook, from=1-1, to=3-1]
	   \arrow["x", from=1-1, to=1-3]
	   \arrow["f"', from=3-1, to=1-3]
    \end{tikzcd}\]
        commutes, which are in bijection with morphisms $f$ of $\cat{C}$ whose codomain is $x$, and
        \item morphisms from $f$ to $g$ are natural transformations $h\colon f \Rightarrow g$ --- which are in bijection with commutative squares
        \[\begin{tikzcd}[sep=scriptsize]
	   y && z \\
	   \\
	   x && x
	   \arrow["f"', from=1-1, to=3-1]
	   \arrow["{h_1}"', from=3-1, to=3-3]
	   \arrow["{h_0}", from=1-1, to=1-3]
	   \arrow["g", from=1-3, to=3-3]
        \end{tikzcd}\]
        in $\cat{C}$ --- that restrict to the identity along $1\colon \Term \incl \wkarr$, that is, are such that $h_1 = \idd{x}$.
        These are in bijection with factorisations of $f$ through $g$.
    \end{itemize}
    This establishes an isomorphism between $\extfun{\Term}{\cat{C}}(1, x)$ and $\slice{\cat{C}}{x}$.
    %
    The codomain of $\Lambda.\extfun{\Term}{\cat{C}}(\fun{K_0}, x)$ is the category $\extfun{\Term}{\cat{C}}(\imath_1, x)$ whose
    \begin{itemize}
        \item objects are functors $(y_0, y_1)\colon \Term + \Term \to \cat{C}$ such that 
    \[\begin{tikzcd}[sep=scriptsize]
	   {\Term} && {\cat{C}} \\
	   \\
	   {\Term + \Term}
	   \arrow["{\imath_1}", hook, from=1-1, to=3-1]
	   \arrow["x", from=1-1, to=1-3]
	   \arrow["{(y_0, y_1)}"', from=3-1, to=1-3]
    \end{tikzcd}\]
        commutes, which are in bijection with pairs of objects $(y_0, y_1)$ of $\cat{C}$ such that $y_1 = x$, which are in bijection with objects of $\cat{C}$, and
        \item morphisms from $(y, x)$ to $(z, x)$ are in bijection with pairs of morphisms
    \[\begin{tikzcd}[sep=scriptsize]
	y && z \\
	x && x
	\arrow["{h_1}"', from=2-1, to=2-3]
	\arrow["{h_0}", from=1-1, to=1-3]
    \end{tikzcd}\]
        in $\cat{C}$ that restrict to the identity along $\imath_1$, that is, are such that $h_1 = \idd{x}$.
        These are in bijection with morphisms $y \to z$.
    \end{itemize}
    This establishes an isomorphism between $\extfun{\Term}{\cat{C}}(\imath_1, x)$ and $\cat{C}$.
    The functor $\extfun{\Term}{\cat{C}}(\fun{K_0}, x)$ acts by restriction of $f\colon \wkarr \to \cat{C}$ along $\fun{K_0}\colon \Term+\Term \incl \wkarr$; through the isomorphisms, this acts by mapping $f\colon y \to x$ to its domain $y$.
    This is, by inspection, the same as the action of $\mathrm{dom}$.

    We define
    \begin{equation*}
        \dhom{0}{\cat{C}}{-}\colon \cat{C} \to \pointed{\catpos}
    \end{equation*}
    to be the post-composition of $\Lambda.\extfun{\Term}{\cat{C}}(\fun{K_0}, -)$ with the functor of \autoref{eq: quotient and posref}.
    It follows from our argument that, up to isomorphism, this sends $x$ to the homotopy poset $\dhom{0}{\cat{C}}{x}$.
    The covariance in $\cat{C}$ then follows as an instance of \autoref{eq:general_functor_covariance}: given a functor $\fun{F}\colon \cat{C} \to \cat{D}$, we whisker the natural transformation $\Lambda.\extfun{\Term}{\fun{F}}(\fun{K_0}, -)$ with the functor of (\ref{eq: quotient and posref}) to obtain $\pi_0(\fun{F})\colon \dhom{0}{\cat{C}}{-} \Rightarrow \dhom{0}{\cat{D}}{\fun{F}-}$.

    Now, let us focus on the first homotopy poset.
    The functor $\fun{K_1}$ identifying two parallel arrows also induces a morphism in $\pastext{\Term}$, depicted as follows:
    \begin{equation*}
    \begin{tikzcd}[sep=small]
	{\blue{\bullet}} & {} & {\red{\bullet}} &&& {\blue{\bullet}} & {} & {\red{\bullet}}
	\arrow[curve={height=-6pt}, from=1-1, to=1-3]
	\arrow[curve={height=6pt}, from=1-1, to=1-3]
	\arrow[from=1-6, to=1-8]
	\arrow[color={rgb,255:red,102;green,102;blue,102}, curve={height=-24pt}, shorten <=15pt, shorten >=15pt, dashed, two heads, from=1-2, to=1-7]
    \end{tikzcd}
    \quad \quad
    \begin{tikzcd}[sep=scriptsize]
	& \Term \\
	\\
	  \mathrm{Par} && \wkarr
	\arrow["{\fun{K_1}}"', hook, from=3-1, to=3-3]
	\arrow["c"', hook, from=1-2, to=3-1]
	\arrow["1", hook, from=1-2, to=3-3]
    \end{tikzcd}
    \end{equation*}
    Here, $\mathrm{Par}$ denotes the ``walking parallel pair of arrows''.
    We claim that, up to isomorphism of categories,
    \begin{equation*}
        \Lambda.\extfun{\Term}{\cat{C}}(\fun{K_1}, -)\colon \cat{C} \to \catcat^\wkarr
    \end{equation*}
    sends an object $x$ of $\cat{C}$ to the slice projection functor
    \begin{equation*}
        \mathrm{dom}\colon \slice{\pararr{\cat{C}}{x}}{(\idd{x}, \idd{x})} \to \pararr{\cat{C}}{x}.
    \end{equation*}
    We have already established that the domain of $\Lambda.\extfun{\Term}{\cat{C}}(\fun{K_1}, x)$, which is the category $\extfun{\Term}{\cat{C}}(1, x)$, is isomorphic to $\slice{\cat{C}}{x}$, which can be shown to be isomorphic to $\slice{\pararr{\cat{C}}{x}}{(\idd{x}, \idd{x})}$ using Proposition \ref{prop: subterminal as weak terminal parallel arrow}.

    The codomain of $\Lambda.\extfun{\Term}{\cat{C}}(\fun{K_1}, x)$ is the category $\extfun{\Term}{\cat{C}}(c, x)$ whose
    \begin{itemize}
        \item objects are functors $(f_0, f_1)\colon \mathrm{Par} \to \cat{C}$ such that 
    \[\begin{tikzcd}[sep=scriptsize]
	{\Term} && {\cat{C}} \\
	\\
	{{\mathrm{Par}}}
	\arrow["c", hook, from=1-1, to=3-1]
	\arrow["x", from=1-1, to=1-3]
	\arrow["{(f_0, f_1)}"', from=3-1, to=1-3]
    \end{tikzcd}\]
    commutes, which are in bijection with pairs of morphisms $(f_0, f_1)$ of $\cat{C}$ whose codomain is $x$, and
    \item morphisms from the pair $(f_0, f_1)$ to $(g_0, g_1)$ are natural transformations $h\colon (f_0, f_1) \Rightarrow (g_0, g_1)$ that restrict to the identity along $c$, which are in bijection with morphisms $h$ such that $f_0 = h;g_0$ and $f_1 = h;g_1$.
    \end{itemize}
    This establishes an isomorphism between $\extfun{\Term}{\cat{C}}(c, x)$ and $\pararr{\cat{C}}{x}$.

    The functor $\extfun{\Term}{\cat{C}}(\fun{K_1}, x)$ acts by precomposing $f\colon \wkarr \to \cat{C}$ with $\fun{K_1}\colon \mathrm{Par} \to \wkarr$, which through the isomorphisms sends a pair $(f, f)$ with its unique morphism to $(\idd{x}, \idd{x})$ to the pair $(f, f)$ on its own.
    This is, by inspection, the same as the action of $\mathrm{dom}$.

    We define
    \begin{equation*}
        \dhom{1}{\cat{C}}{-}\colon \cat{C} \to \pointed{\catpos}
    \end{equation*}
    to be the post-composition of $\Lambda.\extfun{\Term}{\cat{C}}(\fun{K_1}, -)$ with the functor of \autoref{eq: quotient and posref}.
    It follows from our argument that, up to isomorphism, this sends $x$ to the homotopy poset $\dhom{1}{\cat{C}}{x}$.
    Again, we obtain covariance in $\cat{C}$ by whiskering instances of \autoref{eq:general_functor_covariance}.
    This completes the proof.
    \end{proof}

\subsection{Obstructions to a morphism being iso}

    \begingroup
    \def\theproposition{\ref{prop: spit epi weak term mono subterm}}
    \begin{proposition}
        Let $f\colon X \to Y$ be a morphism in a category $\CategoryC$. Then:
        %
        %
        \begin{itemize}
            \item $f$ is split epi in $\CategoryC$ if and only if $f$ is weak terminal in $\slice{\CategoryC}{Y}$,
            \item $f$ is mono in $\CategoryC$ if and only if $f$ is subterminal in $\slice{\CategoryC}{Y}$.
        \end{itemize}
        %
    \end{proposition}
    \addtocounter{proposition}{-1}
    \endgroup
    \begin{proof}
        As for the first point, $f\colon X \to Y$ being split epi in $\CategoryC$ means that there is a morphism $e\colon Y \to X$ in $\CategoryC$ such that $e \Cp f = \Id{Y}$. This means that for every morphism $g\colon Z \to Y$ the following triangle commutes:
        %
        %
        \[\begin{tikzcd}[sep=scriptsize]
	Z && Y && X \\
	\\
	&& Y
	\arrow[Rightarrow, no head, from=1-3, to=3-3]
	\arrow["g"', from=1-1, to=3-3]
	\arrow["g", from=1-1, to=1-3]
	\arrow["e", from=1-3, to=1-5]
	\arrow["f", from=1-5, to=3-3]
\end{tikzcd}\]
        implying subterminality of $f$ in $\slice{\CategoryC}{Y}$.

        The other way around, $f$ weak terminal in $\slice{\CategoryC}{Y}$ means that for each $g\colon Z \to Y$ there exists a morphism $e\colon Z \to X$ making the following triangle commute:
        %
        %
        \[\begin{tikzcd}[sep=scriptsize]
	Z && X \\
	\\
	& Y
	\arrow["g"', from=1-1, to=3-2]
	\arrow["f", from=1-3, to=3-2]
	\arrow["e", from=1-1, to=1-3]
        \end{tikzcd}\]
        %
        In particular, for $g \eqdef \Id{Y}$, we get $e\colon Y \to X$ such that $e \Cp f = \Id{Y}$, proving that $f$ is split epi in $\CategoryC$.

        Next, consider $f$ being mono. 
        For each parallel pair $g_1, g_2\colon Z \to X$, the fact that $g \eqdef g_1 \Cp f = g_2 \Cp f$ is equivalent to commutativity of the following diagram:
        %
        %
        \[\begin{tikzcd}[sep=scriptsize]
	Z && X \\
	\\
	& Y
	\arrow["g"', from=1-1, to=3-2]
	\arrow["f", from=1-3, to=3-2]
	\arrow["{g_1}", curve={height=-6pt}, from=1-1, to=1-3]
	\arrow["{g_2}"', curve={height=6pt}, from=1-1, to=1-3]
\end{tikzcd}\]
        From this, it follows immediately that $f$ being mono in $\CategoryC$ implies $g_1 = g_2$, hence subterminality in $\slice{\CategoryC}{Y}$, and vice-versa.
    \end{proof}

    \begingroup
    \def\thecorollary{\ref{cor: spit epi iff dhom0 trivial mono iff dhom1 trivial}}
    \begin{corollary}
        Let $f\colon X \to Y$ be a morphism in a category $\CategoryC$. Then:
        \begin{itemize}
            \item $f$ is split epi if and only if $\dhom{0}{(\slice{\cat{C}}{Y})}{f}$ is trivial;
            \item $f$ is mono if and only if $\dhom{1}{(\slice{\cat{C}}{Y})}{f}$ is trivial, and:
            \item $f$ is iso if and only if both $\dhom{0}{(\slice{\cat{C}}{Y})}{f}$ and $\dhom{1}{(\slice{\cat{C}}{Y})}{f}$ are trivial.
        \end{itemize}
    \end{corollary}
    \addtocounter{corollary}{-1}
    \endgroup
    \begin{proof}
        First notice that a morphism is iso if and only if it is split epi and mono, so the third condition is a consequence of the first two. \autoref{prop: spit epi weak term mono subterm} reduces the problem to proving:
        \begin{itemize}
            \item $\WTerm$ is weak terminal in $\CategoryC$ iff $\pi_0(\slice{\CategoryC}{\WTerm}, [\WTerm])$ is trivial;
            \item $\WTerm$ is subterminal in $\CategoryC$ iff $\pi_1(\slice{\CategoryC}{\WTerm}, [\WTerm])$ is trivial;
        \end{itemize}
        %
        For the first point, $\WTerm$ is weak terminal in $\CategoryC$ iff $\posref{\WTerm}$ is the greatest element in $\posref{\CategoryC}$ (\autoref{prop: weak terminal is greatest in posref}), which in turn is equivalent to say that the downward closure of $\posref{\WTerm}$ contains all $\posref{\CategoryC}$. Since the quotient functor $\fun{Q}$ collapses the whole downward closure of $\posref{\WTerm}$ to a single point, this is equivalent to say that $\pi_0(\slice{\CategoryC}{\WTerm}, [\WTerm])$ is trivial.

        As for the second point, the proof is the same by remembering that by \autoref{def: 1st-directed homotopy poset}
        %
        %
        \begin{equation*}
            (\dhom{1}{\CategoryC}{\WTerm}, \, [\WTerm]) \eqdef \left(\dhom{0}{\pararr{\CategoryC}{\WTerm}}{(\idd{\WTerm}, \idd{\WTerm})}, \, [(\idd{\WTerm}, \idd{\WTerm})]\right)
        \end{equation*}
        %
        and that $\WTerm$ is subterminal in $\CategoryC$ iff $(\Id{\WTerm}, \Id{\WTerm})$ is weak terminal in $\pararr{\CategoryC}{\WTerm}$ (\autoref{prop: subterminal as weak terminal parallel arrow}).
    \end{proof}


    \begingroup
    \def\theproposition{\ref{prop: dhom0 for set/y}}
    \begin{proposition}
        Let $f\colon X \to Y$ be a function between sets. $\posref{\slice{\catset}{Y}}$ is isomorphic, as a poset, to the power set $\powerset{Y}$, via the assignment $(S \subseteq Y) \mapsto \posref{\imath_S}$, where $\imath_S$ is the injective function including $S$ into $Y$. Through this bijection, $\posref{f}$ corresponds to the image $f(X)$ of $f$.
    \end{proposition}
    \addtocounter{proposition}{-1}
    \endgroup
    \begin{proof}
        Fix $S$ a subset of $\powerset{Y}$. The injection $\imath_S: S \incl Y$ is a function of sets with codomain $Y$, hence is an object of $\slice{\catset}{Y}$, and, as a consequence, part of the equivalence class $\posref{\imath\colon S \incl Y}$ in $\posref{\slice{\catset}{Y}}$.
        So the assignment $S \mapsto \posref{\imath\colon S \incl Y}$ makes sense. We now have to prove that this assignment is order preserving, injective and surjective.
        
        If $S \subseteq T$, then letting $j\colon S \incl T$ be the corresponding inclusion of sets, we have
        %
        %
        \[\begin{tikzcd}[sep=scriptsize]
	S && T \\
	\\
	& Y
	\arrow["{\imath_S}"', hook, from=1-1, to=3-2]
	\arrow["{\imath_T}", hook, from=1-3, to=3-2]
	\arrow["j", hook, from=1-1, to=1-3]
        \end{tikzcd}\]
        %
        So $j$ defines a morphism $\imath_S \to \imath_T$ in $\slice{\catset}{Y}$, so by definition
        $\posref{\imath_S\colon S \incl Y} \leq \posref{\imath_T\colon T \incl Y}$, and the order is preserved.

        As for injectivity, suppose that $S \neq T$, and suppose without loss of generality that there is an element $s \in S \setminus T$. Then there is no function $g\colon T \to S$ such that the following triangle commutes:
        %
        %
        \[\begin{tikzcd}[sep=scriptsize]
	S && T \\
	\\
	& Y
	\arrow["{\imath_S}"', hook, from=1-1, to=3-2]
	\arrow["{\imath_T}", hook, from=1-3, to=3-2]
	\arrow["g", from=1-1, to=1-3]
        \end{tikzcd}\]
        %
        since for all $g$ we have $\imath_S(s) = s \neq f(s) = \imath_T(f(s))$. Then $\posref{\imath_S\colon S \incl Y} \not\leq \posref{\imath_T\colon T \incl Y}$, and injectivity is proved.

        Finally, let $\posref{f\colon X \to Y}$ be any element of $\posref{\slice{\catset}{Y}}$, and recall that
        %
        %
        \begin{equation*}
            f(X) \eqdef \{y \in Y \mid \text{there exists $x \in X$ such that $y = f(x)$}\}.
        \end{equation*}
        %
        Then we have a surjective function $\tilde{f}\colon X \to f(X)$, defined as $f$ with codomain restricted to its image, which assuming the axiom of choice has a section $s\colon f(X) \to X$ so that
        %
        %
        \[\begin{tikzcd}[sep=scriptsize]
	X && {f(X)} \\
	\\
	& Y
	\arrow["f"', from=1-1, to=3-2]
	\arrow["{\imath_{f(X)}}", hook, from=1-3, to=3-2]
	\arrow["{\tilde{f}}", two heads, from=1-1, to=1-3]
\end{tikzcd}
        \quad \quad 
        \begin{tikzcd}[sep=scriptsize]
	{f(X)} && X \\
	\\
	& Y
	\arrow["{\imath_{f(X)}}"', hook, from=1-1, to=3-2]
	\arrow["f", from=1-3, to=3-2]
	\arrow["s", two heads, from=1-1, to=1-3]
\end{tikzcd}\]
        %
        commute. 
        It follows that $\posref{\imath_{f(X)}} \geq \posref{f}$ and $\posref{\imath_{f(X)}} \leq \posref{f}$, so we can conclude that $\posref{f} = \posref{\imath_{f(X)}}$.
        This proves both that the assignment $S \mapsto \posref{\imath_S}$ is surjective and that $f(X) \mapsto \posref{f}$, concluding the proof.
    \end{proof}

    \begingroup
    \def\theproposition{\ref{prop: dhom1 for set/y}}
    \begin{proposition}
        Let $X \times_f X$ be the pullback of $f$ along itself --- that is, the set $\{(x_0, x_1) \mid f(x_0) = f(x_1)\}$ --- and let $p_f\colon X \times_f X \to Y$ be the function $(x_0, x_1) \mapsto f(x_0) = f(x_1)$. Then:
            \begin{enumerate}
                \item $\posref{\pararr{(\slice{\catset}{Y})}{f}}$ is isomorphic to $\powerset{(X \times_f X)}$ via the assignment $(S \subseteq X \times_f X) \mapsto \posref{(\restr{p_0}{S}, \restr{p_1}{S})}$, where $\restr{p_i}{S}$ are the projections $X \times_f X \to Y$, restricted to $S$, seen as morphisms $\restr{p_f}{S} \to f$ in $\posref{\pararr{(\slice{\catset}{Y})}{f}}$;
                \item through this bijection, $\posref{(\idd{f}, \idd{f})}$ is identified with the diagonal $\Delta X$.
            \end{enumerate}
    \end{proposition}
    \addtocounter{proposition}{-1}
    \endgroup
        %Let $X \times_f X$ be the pullback of $f$ along itself -- that is, the set $\{(x_0, x_1) \mid f(x_0) = f(x_1)\}$ -- and let $p_f\colon X \times_f X \to Y$ be the function $(x_0, x_1) \mapsto f(x_0) = f(x_1)$. $\posref{\pararr{(\slice{\catset}{Y})}{f}}$ is isomorphic, as a poset, to $\powerset{(X \times_f X)}$ via the assignment $(S \subseteq X \times_f X) \mapsto \posref{(\restr{p_0}{S}, \restr{p_1}{S})}$, where $\restr{p_i}{S}$ are the projections $X \times_f X \to Y$, restricted to $S$, seen as morphisms $\restr{p_f}{S} \to f$ in $\posref{\pararr{(\slice{\catset}{Y})}{f}}$. Through this bijection, $\posref{(\idd{f}, \idd{f})}$ is identified with the diagonal $\Delta X$, which is always included in $X \times_f X$.
    %\end{proposition}
    \begin{proof}
        To understand how the assignment works, look at the following diagram. The blue arrow $S \to Y$ is just $\restr{p_f}{S}$. 
        The black diagram commutes because of the property of pullbacks. The two sides of the square are both $f$, hence can be identified. 
        The parallel pair of arrows $(\restr{p_0}{S}, \restr{p_1}{S})$ are hence a couple of morphisms from $\restr{p_f}{S}$ to $f$ in $\slice{\catset}{Y}$, hence by definition an object in $\pararr{(\slice{\catset}{Y})}{f}$. 
        Then it makes sense to send the subset $S$ to $\posref{(\restr{p_0}{S}, \restr{p_1}{S})}$ in $\posref{\pararr{(\slice{\catset}{Y})}{f}}$.
        %
        \[\begin{tikzcd}
	S &&&&& \textcolor{rgb,255:red,92;green,92;blue,214}{X} \\
	\\
	&& {X \times_f X} && X \\
	\\
	&&& X && Y
	\arrow["{p_1}"', from=3-3, to=5-4]
	\arrow["{p_0}", from=3-3, to=3-5]
	\arrow["f", from=3-5, to=5-6]
	\arrow["f"', from=5-4, to=5-6]
	\arrow["f", color={rgb,255:red,92;green,92;blue,214}, from=1-6, to=5-6]
	\arrow["{p_f}"{description}, color={rgb,255:red,92;green,92;blue,214}, from=3-3, to=5-6]
	\arrow[color={rgb,255:red,92;green,92;blue,214}, hook, from=1-1, to=3-3]
	\arrow["{\restr{p_0}{S}}", color={rgb,255:red,92;green,92;blue,214}, curve={height=-6pt}, from=1-1, to=1-6]
	\arrow["{\restr{p_1}{S}}"', color={rgb,255:red,92;green,92;blue,214}, curve={height=6pt}, from=1-1, to=1-6]
	\arrow["{\restr{p_1}{S}}"{description}, curve={height=24pt}, from=1-1, to=5-4]
	\arrow["{\restr{p_0}{S}}"{description}, curve={height=-6pt}, from=1-1, to=3-5]
\end{tikzcd}\]
        %
        Verifying that this is a order-preserving bijection is tedious, but not difficult, as it mimicks the proof of \autoref{prop: dhom0 for set/y}.
        From the picture it should also be clear that when $S$ is $X$ and $S \to X \times_f X$ is the diagonal morphism, the blue arrow $S \to Y$ becomes $f$, and $\restr{p_0}{S}$ and $\restr{p_1}{S}$ are just $\idd{X}$ in $\CategoryC$. But $\idd{X}$ can also be seen as a morphism from $f$ to itself in $\slice{\catset}{Y}$, hence $\idd{X}$ in $\CategoryC$ is also $\idd{f}$ in $\slice{\catset}{Y}$. So the diagonal $\Delta X$ corresponds to $\posref{(\idd{f}, \idd{f})}$ in $\posref{\pararr{(\slice{\catset}{Y})}{f}}$.
    \end{proof}

    \begin{definition}[Join of categories] \label{dfn:join_of_categories}
    Let $\cat{C}, \cat{D}$ be categories.
    The \emph{join of $\cat{C}$ and $\cat{D}$} is the collage $\cat{C} \join \cat{D}$ of the unique profunctor $\opp{\cat{C}} \times \cat{D} \to \catset$ sending every pair of objects to the one-element set.

    More explicitly, $\cat{C} \join \cat{D}$ is obtained from the disjoint union of $\cat{C}$ and $\cat{D}$ by adding a unique morphism from each object $x$ in $\cat{C}$ to each object $y$ in $\cat{D}$.
    With the empty category as unit, the join determines a non-symmetric, semicocartesian monoidal structure on $\catcat$.
    We refer to \cite{joyal2008theory} for more details.
    \end{definition}

    \begin{proposition} \label{prop:right_join_preserves_past_extensions}
    Let $\imath\colon \cat{A} \incl \cat{B}$ be a past extension, and $\cat{X}$ a category.
    Then $\imath \join \idd{\cat{X}}\colon \cat{A} \join \cat{X} \incl \cat{B} \join \cat{X}$ is a past extension.
    \end{proposition}
    \begin{proof}
    We use the characterisation of past extensions with collages of profunctors, given in \autoref{rmk:past extension collage}.
    Suppose $\imath$ is induced by the profunctor $\fun{H}\colon \opp{\cat{\bar{A}}} \times \cat{A} \to \catset$.
    We define a profunctor $\fun{\tilde{H}}\colon \opp{\cat{\bar{A}}} \times (\cat{A}\join\cat{X}) \to \catset$ by
    \begin{equation*}
        \fun{\tilde{H}}(x, y) \eqdef \begin{cases}
            \fun{H}(x, y) & \text{if $y$ is an object of $\cat{A}$}, \\
            1 & \text{if $y$ is an object of $\cat{X}$},
        \end{cases}
    \end{equation*}
    with the only action of morphisms compatible with $\fun{H}$.
    Then the collage of $\fun{\tilde{H}}$ is isomorphic to $\cat{B} \join \cat{X}$, and the inclusion of $\cat{A}\join\cat{X}$ into the collage is isomorphic to $\imath \join \idd{\cat{X}}$.
    \end{proof}

    \begin{remark} \label{rmk:join_functor_of_past}
    It follows from functoriality of joins and Proposition \ref{prop:right_join_preserves_past_extensions} that $- \join \idd{X}$ determines a functor $\pastext{A} \to \pastext{A \join X}$ for each pair of categories $A, X$.
    \end{remark}

    \begingroup
    \def\theproposition{\ref{prop: covariance natural transformation}}
    \begin{proposition}[Covariance over the domain of a natural transformation]
        Let $\fun{F}, \fun{G}\colon \cat{C} \to \cat{D}$ be two functors and let $\alpha\colon \fun{F} \Rightarrow \fun{G}$ be a natural transformation.
        For all $i \in \{ 0, 1\}$, the assignment
        \begin{equation*}
            x \; \mapsto \; \dhom{i}{(\slice{\cat{D}}{\fun{G}{x}})}{\alpha_x}
        \end{equation*}
        extends to a functor $\cat{C} \to \pointed{\catpos}$.
    \end{proposition}
    \addtocounter{proposition}{-1}
    \endgroup
    \begin{proof}
    Since $\wkarr$ is isomorphic to $\Term \join \Term$, there is a functor
    \begin{equation*}
        - \join \idd{\Term}\colon \pastext{\Term} \to \pastext{\wkarr}
    \end{equation*}
    as described in \autoref{rmk:join_functor_of_past}, and we can take
    \begin{equation*}
        \fun{K_0} \join \idd{\Term}, \quad \quad \fun{K_1} \join \idd{\Term}
    \end{equation*}
    for the morphisms we considered in the proof of \autoref{prop: Homotopy posets are functorial}, to get morphisms in $\pastext{\wkarr}$.
    These can be pictured as follows:
    \begin{equation*}
    \begin{tikzcd}[column sep=scriptsize]
	{\blue{\bullet}} & {} & {\red{\bullet}} && {\blue{\bullet}} & {} & {\red{\bullet}} && {\blue{\bullet}} & {} & {\red{\bullet}} \\
	&& {\red{\bullet}} &&&& {\red{\bullet}} &&&& {\red{\bullet}}
	\arrow[from=1-5, to=1-7]
        \arrow[curve={height=6pt}, from=1-5, to=2-7]
	\arrow[color={rgb,255:red,214;green,92;blue,92}, from=1-7, to=2-7]
	\arrow[color={rgb,255:red,214;green,92;blue,92}, from=1-3, to=2-3]
	\arrow[curve={height=6pt}, from=1-1, to=2-3]
	\arrow[curve={height=-6pt}, from=1-9, to=1-11]
	\arrow[curve={height=6pt}, from=1-9, to=1-11]
	\arrow[color={rgb,255:red,214;green,92;blue,92}, from=1-11, to=2-11]
	\arrow[curve={height=6pt}, from=1-9, to=2-11]
        \arrow[color={rgb,255:red,102;green,102;blue,102}, curve={height=24pt}, shorten <=12pt, shorten >=12pt, dashed, hook, from=1-2, to=1-6]
	\arrow[color={rgb,255:red,102;green,102;blue,102}, curve={height=-24pt}, shorten <=12pt, shorten >=12pt, dashed, two heads, from=1-10, to=1-6]
    \end{tikzcd}
    \end{equation*}
    These induce functors
    \begin{align*}
        \Lambda.\extfun{\wkarr}{\cat{C}}(\fun{K_0} \join \idd{\Term}, -) & \colon \cat{C}^\wkarr \to \catcat^\wkarr, \\
        \Lambda.\extfun{\wkarr}{\cat{C}}(\fun{K_1} \join \idd{\Term}, -) & \colon \cat{C}^\wkarr \to \catcat^\wkarr.
    \end{align*}
    according to the general functoriality pattern of \autoref{rmk: functoriality pattern}.
    We claim that, up to isomorphism of categories, these functors send an object of $\cat{C}^\wkarr$, that is, a morphism $f\colon x \to y$ in $\cat{C}$, to the slice projection functors
    \begin{align*}
        \mathrm{dom} & \colon \slice{(\slice{\cat{C}}{y})}{f} \to \slice{\cat{C}}{y}, \\
        \mathrm{dom} & \colon \slice{\pararr{(\slice{\cat{C}}{y})}{f}}{(\idd{f}, \idd{f})} \to \pararr{(\slice{\cat{C}}{y})}{f},
    \end{align*}
    hence their composites with the functor of \autoref{eq: quotient and posref} act as
    \begin{align}
        (f\colon x \to y) \; & \mapsto \; \dhom{0}{(\slice{\cat{C}}{y})}{f}, \label{eq:dhom0_morphism}
        \\
        (f\colon x \to y) \; & \mapsto \; \dhom{1}{(\slice{\cat{C}}{y})}{f}.
        \label{eq:dhom1_morphism}
    \end{align}
    This can be verified as in the proof of \autoref{prop: Homotopy posets are functorial} using the general fact that commutative diagrams of the form
    \[\begin{tikzcd}[sep=scriptsize]
	\Term \\
	{\cat{A} \join \Term} && {\cat{C}} \\
	{\cat{B} \join \Term}
	\arrow["{\fun{F} \join \idd{\Term}}"', from=2-1, to=3-1]
	\arrow["{\varnothing \join \idd{\Term}}"', from=1-1, to=2-1]
	\arrow["x", from=1-1, to=2-3]
	\arrow[from=2-1, to=2-3]
	\arrow[from=3-1, to=2-3]
    \end{tikzcd}\]
    are in natural bijection with commutative diagrams of the form
    \[\begin{tikzcd}[sep=scriptsize]
	{\cat{A}} && {\slice{\cat{C}}{x}} \\
	\\
	{\cat{B}}
	\arrow["{\fun{F}}"', from=1-1, to=3-1]
	\arrow[from=1-1, to=1-3]
	\arrow[from=3-1, to=1-3]
    \end{tikzcd}\]
    because the functor
    \begin{align*}
    - \join \Term \colon \catcat & \to \slice{1}{\catcat}, \\
    \cat{A} & \mapsto (\varnothing \join \idd{\Term}\colon \Term \to \cat{A} \join \Term)
    \end{align*}
    is left adjoint to the slicing functor $(x\colon \Term \to \cat{C}) \mapsto \slice{\cat{C}}{x}$.
    
    Now, a natural transformation $\alpha\colon \fun{F} \Rightarrow \fun{G}$ between functors $\fun{F}, \fun{G}\colon \cat{C} \to \cat{D}$ can be seen as a functor
    \begin{equation*}
        \alpha\colon \cat{C} \to \cat{D}^\wkarr
    \end{equation*}
    sending $x$ to $(\alpha_x\colon \fun{F}x \to \fun{G}x)$, which post-composed with (\ref{eq:dhom0_morphism}) and (\ref{eq:dhom1_morphism}) defines functors $\cat{C} \to \pointed{\catpos}$ acting on objects as
    \begin{equation*}
        x \; \mapsto \; \dhom{i}{(\slice{\cat{D}}{\fun{G}{x}})}{\alpha_x}, \quad i \in \{ 0, 1 \}.
    \end{equation*}
    This completes the proof.
    \end{proof}


\subsection{Qualifying compositionality}

    %\begin{proposition}[Laxity of the state functor]
        %For any monoidal category $(\CategoryC, \otimes, \TensorUnit)$, the state functor is lax.
    %\end{proposition}
    \begingroup
    \def\theproposition{\ref{prop: state functor lax}}
    \begin{proposition}[Laxity of the state functor]
        The state functor lifts to a lax monoidal functor from $(\cat{C}, \otimes, \TensorUnit)$ to $(\catset, \times, \{*\})$.
    \end{proposition}
    \addtocounter{proposition}{-1}
    \endgroup
    \begin{proof}
        We need to define a natural transformation and a morphism
        \begin{gather*}
            \Phi\colon \homset{\CategoryC}{\TensorUnit}{-} \times \homset{\CategoryC}{\TensorUnit}{-} \Rightarrow \homset{\CategoryC}{\TensorUnit}{- \otimes -}\\
            \varphi : \{*\} \to \homset{\CategoryC}{\TensorUnit}{\TensorUnit}
        \end{gather*}
        Switching to components, we need to define functions
        \begin{equation*}
            \Phi_{A,B} : \{\TensorUnit \xrightarrow{\psi_A} A \} \times \{\TensorUnit \xrightarrow{\psi_B} B \} \rightarrow 
            \{\TensorUnit \xrightarrow{\psi_{A,B}} A \otimes B \}
        \end{equation*}
        whereas $\varphi$ is just picking an element $\TensorUnit \xrightarrow{\phi_\TensorUnit} \TensorUnit$ of $\homset{\CategoryC}{\TensorUnit}{\TensorUnit}$.
         We define them by setting:
         \begin{gather*}
           \Phi_{A,B}(\psi_{A},\psi_B) = \rho^{-1}_\TensorUnit \Cp  (\psi_A \otimes \psi_B)\\
           \varphi(*) = id_\TensorUnit
         \end{gather*}
        where $\rho$ is the right unitor of $\CategoryC$, and its component at $\TensorUnit$ is $\rho: \TensorUnit \otimes \TensorUnit \to \TensorUnit$.

        Verifying that $\Phi$ is a natural transformation and that the coherence diagrams for $\Phi$ and $\varphi$ commute is a long, but simple exercise.
    \end{proof}


    \begingroup
    \def\theproposition{\ref{prop: state functor oplax}}
    \begin{proposition}[Oplaxity of the state functor]
        Let $(\cat{C}, \otimes, \Term)$ be a semicartesian category.
        Then the state functor lifts to an oplax monoidal functor from $(\cat{C}, \otimes, \Term)$ to $(\catset, \times, \{*\})$.
    \end{proposition}
        %For any monoidal category $(\CategoryC, \otimes, \TensorUnit)$ with \emph{delete}, that is, such that $\TensorUnit$ is terminal, the state functor is oplax.
    %\end{proposition}
    \addtocounter{proposition}{-1}
    \endgroup
    \begin{proof}
        If $\Term$ is terminal, then for each object $A$ there is a unique morphism $A \xrightarrow{!_A} \Term$.

        We need to define a natural transformation and a morphism:
        \begin{gather*}
            \Phi : \homset{\CategoryC}{\Term}{- \otimes -} \Rightarrow \homset{\CategoryC}{\Term}{-} \times \homset{\CategoryC}{\Term}{-}\\
            \varphi : \homset{\CategoryC}{\Term}{\Term} \to \Term_{\catset}
        \end{gather*}
        Switching to components, we need to define functions
        \begin{equation*}
            \Phi_{A,B} :
            \{\Term \xrightarrow{\psi_{A,B}} A \otimes B \} \rightarrow \{\Term \xrightarrow{\psi_A} A \} \times \{\Term \xrightarrow{\psi_B} B \}
        \end{equation*}
        Whereas $\varphi$ is just a function $\{\Term \xrightarrow{\phi_\Term} \Term \} \rightarrow \{*\}$, which is uniquely determined since $\{*\}$ is terminal in $\catset$.

        We define $\Phi_{A,B}$ by letting
        \begin{gather*}
            \Phi_{A,B}(\psi_{A,B}) =\left(\psi_{A,B} \Cp  (id_A \otimes !_B) \Cp  \rho_A ,\psi_{A,B} \Cp  (!_A \otimes id_B) \Cp  \lambda_B\right).
        \end{gather*}
        As in the previous proposition, verifying that all the required diagrams commute is a straightforward exercise.
    \end{proof}

    \begingroup
    \def\theproposition{\ref{prop: state functor strong}}
    \begin{proposition}[Strongness of the state functor]
        If $(\CategoryC, \times, \Term)$ is cartesian, then the state functor is strong monoidal.
    \end{proposition}
    \addtocounter{proposition}{-1}
    \endgroup
    \begin{proof}
        The result follows from the fact that $\homset{\cat{C}}{\Term}{-}$ preserves limits, which by definition gives:
        \begin{gather*}
            \homset{\CategoryC}{\Term}{A \times B} \simeq \homset{\CategoryC}{\Term}{A} \times \homset{\CategoryC}{\Term}{B}\\
            \homset{\CategoryC}{\Term}{\Term} \simeq \{*\}
        \end{gather*}
        naturally in $A$ and $B$.
    \end{proof}

    % \begingroup
    % \def\theproposition{\ref{}}

    % \addtocounter{proposition}{-1}
    % \endgroup
    % \begin{proof}
    % TODO
    % \end{proof}













 

%
%



% \begin{proposition}
%         For any monoidal category $(\CategoryC, \otimes, \TensorUnit)$, the $\homset{\CategoryC}{-}{-}$ functor is lax.
%     \end{proposition}


%    \begin{proof}
%         First of all, we need to define a natural transformation:
%         \begin{equation*}
%             \begin{tikzpicture}
%                 \node (dom) at (0,0) {$(\opp{\CategoryC} \times \CategoryC) \times (\opp{\CategoryC} \times \CategoryC)$};
%                 \node (cod) at (7,0) {$\catset$};
%                 \draw[
%                     -{Stealth[length=2.5mm]},
%                     thick,
%                     bend left
%                 ] (dom) to node[midway, fill=white] (x) {$\homset{\CategoryC}{-}{-} \times \homset{\CategoryC-}{-}$} (cod);
%                 \draw[
%                     -{Stealth[length=2.5mm]},
%                     thick,
%                     bend right
%                 ] (dom) to node[midway,fill=white] (y) {$\homset{\CategoryC}{- \otimes -}{- \otimes -}$} (cod);
%                 \draw[
%                     -{Stealth[length=2.5mm]},
%                     line width=1pt,
%                     double distance=1pt,
%                     thick
%                 ] (x.south) to node[midway, fill=white] {$\Phi$} (y.north);
%             \end{tikzpicture}
%         \end{equation*}
%         The component 
%         \begin{equation*}
%             \homset{\CategoryC}{A_0}{A_1} \times \homset{\CategoryC}{B_0}{B_1} \xrightarrow{\Phi_{A_0,A_1,B_0,B_1}} \homset{\CategoryC}{A_0 \otimes B_0}{A_1 \otimes B_1}
%         \end{equation*}
%         simply sends a couple of morphisms $f: A_0 \to A_1$, $g: B_0 \to B_1$ to their tensor product $f \otimes g$. Naturality of $\phi$ is easily deduced from functoriality of $\otimes$. \TensorUnitndeed, the following diagram 

%         \begin{center}
%         \begin{tikzcd}
% {Hom_C(A_0, A_1) \times Hom_C(B_0, B_1) } \arrow[rrr, "{\phi_{A_0, A_1, B_0, B_1}}"] \arrow[dd, "{Hom_C(h^{op}, k) \times Hom(\tilde{h}^{op}, \tilde{k})}"'] &  &  & {Hom_C(A_0 \otimes B_0, A_1 \otimes B_1)} \arrow[dd, "{Hom_C(h^{op} \otimes \tilde{h}^{op}, k \otimes \tilde{k})}"] \\
%                                                                                                                                                              &  &  &                                                                                                                     \\
% {Hom_C(C_0, C_1) \times Hom_C(D_0, D_1) } \arrow[rrr, "{\phi_{C_0, C_1, D_0, D_1}}"']                                                                        &  &  & {Hom_C(C_0 \otimes D_0, C_1 \otimes D_1)}                                                                          
% \end{tikzcd}

% \end{center}

% commutes because if $f \in Hom_C(A_0, A_1), g \in Hom_C(B_0, B_1)$, we have that

% \begin{equation*}
% (h ; f ; k) \otimes (\tilde{h} ; g ; \tilde{k})= (h \otimes \tilde{h}) ; (f \otimes g) ; (k \otimes \tilde{k})
% \end{equation*}
% due to functoriality of $\otimes$.


%         As for the units, we define $\varphi: \{*\} \to \homset{\CategoryC}{I}{I}$ as the function that picks $\idd{I}$.
%      \begin{center}
% \begin{tikzcd}
% {Hom_C(A_0, A_1) \times \left\{*\right\} } \arrow[dd] \arrow[rrr, "1 \times \varphi"] &  &  & {Hom_C(A_0, A_1) \times Hom_C(I,I)} \arrow[dd, "\phi_{A_0, A_1, I, I}"]    \\
%                                                                                     &  &  &                                                   \\
% {Hom_C(A_0, A_1)}                                                                   &  &  & {Hom_C(A_0 \otimes I, A_1 \otimes I)} \arrow[lll]
% \end{tikzcd}
% \end{center}
       
        
%         Let $f \in Hom_C(A_0, A_1)$. In order to prove unitality for the right unitor, it is enough to show that $\rho^{-1}_{A_0} ; (f \otimes id_I) ; \rho_{A_1} = f$. This is true by naturality of $\rho$. Indeed, the following diagram commutes: 
% \begin{center}
    

%         \begin{tikzcd}
% A_0  \arrow[r, "\rho^{-1}_{A_0}"] & A_0 \otimes I  \arrow[rr, "\rho_{A_0}"] \arrow[dd, "f \otimes id_I"'] &  & A_0 \arrow[dd, "f"] \\
%                                   &                                                                       &  &                     \\
%                                   & A_1 \otimes I  \arrow[rr, "\rho_{A_1}"']                              &  & A_1                
% \end{tikzcd}

% \end{center}
%         A totally symmetric argument can be used to show unitality for the left unitor as well as associativity.
        
%         %Associativity easily follows from the fact that $\alpha$, is a natural isomorphisms and its corresponding morphisms in $\opp{\cat{C}}$ is equal to its inverse.

%         \end{proof}



%      \begin{proposition}\label{prop: hom functor is oplax}
%         For any monoidal category $(\CategoryC, \otimes, \TensorUnit)$ such that $\TensorUnit$ is both initial and terminal, the $\homset{\CategoryC}{-}{-}$ functor is oplax.
%     \end{proposition}

%     \begin{proof}
%         To prove this, we define $\Phi$ such that its component
%         \begin{equation*}
%             \homset{\CategoryC}{A_0 \otimes B_0}{A_1 \otimes B_1}
%             \xrightarrow{\Phi_{A_0,A_1,B_0,B_1}} 
%             \homset{\CategoryC}{A_0}{A_1} \times \homset{\CategoryC}{B_0}{B_1} 
%         \end{equation*}
%         Sends a morphism $f:A_0 \otimes B_0 \to A_1 \otimes B_1$ to the couple whose components are:
%         \begin{gather*}
%             A_0 \xrightarrow{\rho^{-1}_{A_0}} A_0 \otimes I \xrightarrow{\idd{A_0} \otimes !^{B_0}} A_0 \otimes B_0 \xrightarrow{f} A_1 \otimes B_1 \xrightarrow{\idd{A} \otimes !_{B_1}} A_1 \otimes I \xrightarrow{\rho_{A_1}} A_1\\
%             B_0 \xrightarrow{\lambda^{-1}_{B_0}} I \otimes B_0 \xrightarrow{!^{A_0} \otimes \idd{B_0}} A_0 \otimes B_0 \xrightarrow{f} A_1 \otimes B_1 \xrightarrow{!_{A_1} \otimes \idd{B_1}} I \otimes B_1 \xrightarrow{\lambda_{B_1}} B_1
%         \end{gather*}

%         The above defined $\Phi$ is a natural transformation because of naturality of $\rho$ and universal property of initial and terminal objects.
%         As for the units, $\varphi: \homset{\CategoryC}{I}{I} \to \{*\}$ is just the unique function to the singleton set.
%         Unitality easily follows from functoriality of $\otimes$. \textbf{Associativity TODO}
%     \end{proof}
