\section{Obstructions to a morphism being iso} \label{sec: obstructions}


As remarked in the Introduction, one of our main motivations for introducing homotopy posets was measuring how far a generic morphism is from being iso.
Just as we could separate obstructions to terminality into obstructions to weak terminality and subterminality, we can separate obstructions to a morphism being iso into obstructions to a morphism being split epi and mono, respectively.
%
%
\begin{proposition}\label{prop: spit epi weak term mono subterm}
    Let $f\colon X \to Y$ be a morphism in a category $\CategoryC$. Then:
    %
    %
    \begin{itemize}
        \item $f$ is split epi in $\CategoryC$ if and only if $f$ is weak terminal in $\slice{\CategoryC}{Y}$,
        \item $f$ is mono in $\CategoryC$ if and only if $f$ is subterminal in $\slice{\CategoryC}{Y}$.
    \end{itemize}
    %
\end{proposition}
%
%
\begin{corollary}\label{cor: spit epi iff dhom0 trivial mono iff dhom1 trivial}
    Let $f: X \to Y$ be a morphism in a category $\CategoryC$. Then:
    \begin{itemize}
        \item $f$ is split epi if and only if $\dhom{0}{(\slice{\cat{C}}{Y})}{f}$ is trivial;
        \item $f$ is mono if and only if $\dhom{1}{(\slice{\cat{C}}{Y})}{f}$ is trivial, and:
        \item $f$ is iso if and only if both $\dhom{0}{(\slice{\cat{C}}{Y})}{f}$ and $\dhom{1}{(\slice{\cat{C}}{Y})}{f}$ are trivial.
    \end{itemize}
\end{corollary}
%
Furthermore, when the homotopy posets associated to a morphism $f$ are not trivial, they give us precise information about why $f$ fails to be split epi and mono.

To make this more concrete, let us spell out precisely how to compute the invariants associated to a function between sets, where split epi (assuming choice) means \emph{surjective} and mono means \emph{injective}.
This amounts to calculating $\dhom{0}{(\slice{\Set}{Y})}{f}$ and $\dhom{1}{(\slice{\Set}{Y})}{f}$ for some function $f\colon X \to Y$. 
%
%
\begin{proposition}\label{prop: dhom0 for set/y}
    Let $f\colon X \to Y$ be a function between sets. $\posref{\slice{\catset}{Y}}$ is isomorphic, as a poset, to the power set $\powerset{Y}$, via the assignment $(S \subseteq Y) \mapsto \posref{\imath_S}$, where $\imath_S$ is the injective function including $S$ into $Y$.
    Through this bijection, $\posref{f}$ corresponds to the image $f(X)$ of $f$.
\end{proposition}
%
Using this correspondence and quotienting by the lower set of $f(X)$, which contains in particular $\varnothing$, we may identify $\dhom{0}{(\slice{\catset}{Y})}{f}$ with the subposet of $\powerset{Y}$ whose elements are either $\varnothing$ or subsets of $Y$ that contain at least one element $y \notin f(X)$.
The ``minimal obstructions'', that is, the minimal elements in the complement of the basepoint, are the singletons $\{y\}$ with $y \in Y \setminus f(X)$.
This poset is trivial if and only if $f(X) = Y$, that is, iff $f$ is surjective.
%
%
\begin{example}
    Let $f\colon \{0,1\} \to \{0,1,2,3\}$ be the function mapping $0 \mapsto 0$ and $1 \mapsto 1$. 
    The homotopy poset $\dhom{0}{(\slice{\catset}{\{0,1,2,3\}})}{f}$ has the following structure:
    %
    %
    \begin{equation*}
    \def\interval{1.75}
        \scalebox{0.75}{
        \begin{tikzpicture}
            \node(4) at (0,4*\interval) {$\{0,1,2,3\}$};
            \node (3a) at (-6,3*\interval) {$\{0,1,2\}$};
            \node (3b) at (-2,3*\interval) {$\{0,2,3\}$};
            \node (3c) at (2,3*\interval) {$\{1,2,3\}$};
            \node (3d) at (6,3*\interval) {$\{0,1,3\}$};

            \node (2a) at (-8,2*\interval) {$\{0,2\}$};
            \node (2b) at (-4,2*\interval) {$\{1,2\}$};
            \node (2c) at (0,2*\interval) {$\{2,3\}$};
            \node (2d) at (4,2*\interval) {$\{0,3\}$};
            \node (2e) at (8,2*\interval) {$\{1,3\}$};

            \node (1a) at (-4,1*\interval) {$\{2\}$};
            \node (1b) at (4,1*\interval) {$\{3\}$};

        \node (0) at (0,0) {$\varnothing$};
            
            \draw[thick] (3a) -- (4);
            \draw[thick] (3b) -- (4);
            \draw[thick] (3c) -- (4);
            \draw[thick] (3d) -- (4);

            \draw[thick] (2a) -- (3a);
            \draw[thick] (2a) -- (3b);

            \draw[thick] (2b) -- (3a);
            \draw[thick] (2b) -- (3c);

            \draw[thick] (2c) -- (3b);
            \draw[thick] (2c) -- (3c);

            \draw[thick] (2d) -- (3b);
            \draw[thick] (2d) -- (3d);

            \draw[thick] (2e) -- (3c);
            \draw[thick] (2e) -- (3d);

            \draw[thick] (1a) -- (2a);
            \draw[thick] (1a) -- (2b);
            \draw[thick] (1a) -- (2c);

            \draw[thick] (1b) -- (2c);
            \draw[thick] (1b) -- (2d);
            \draw[thick] (1b) -- (2e);

            \draw[thick] (0) -- (1a);
            \draw[thick] (0) -- (1b);
        \end{tikzpicture}
        }
    \end{equation*}
    The minimal obstructions $\{2\}$ and $\{3\}$ are in bijection with the elements not in the image of $f$.
\end{example}
%
%
\begin{proposition}\label{prop: dhom1 for set/y}
    Let $X \times_f X$ be the pullback of $f$ along itself --- that is, the set $\{(x_0, x_1) \mid f(x_0) = f(x_1)\}$ --- and let $p_f\colon X \times_f X \to Y$ be the function $(x_0, x_1) \mapsto f(x_0) = f(x_1)$. Then:
    \begin{enumerate}
        \item $\posref{\pararr{(\slice{\catset}{Y})}{f}}$ is isomorphic to $\powerset{(X \times_f X)}$ via the assignment $(S \subseteq X \times_f X) \mapsto \posref{(\restr{p_0}{S}, \restr{p_1}{S})}$, where $\restr{p_i}{S}$ are the projections $X \times_f X \to Y$, restricted to $S$, seen as morphisms $\restr{p_f}{S} \to f$ in $\posref{\pararr{(\slice{\catset}{Y})}{f}}$;
        \item through this bijection, $\posref{(\idd{f}, \idd{f})}$ is identified with the diagonal $\Delta X$.
    \end{enumerate}
\end{proposition}
%
%
Using this correspondence, we may identify $\dhom{1}{\catset}{X}$ with the subposet of $\powerset{(X \times_f X)}$ whose elements are either $\varnothing$, or contain at least one pair $(x_0, x_1)$ such that $x_0 \neq x_1$.
This poset is trivial if and only if $f$ is injective. 
Notice that the minimal obstructions to injectiveness of $f$ are in bijection with pairs $(x_0, x_1)$ where $x_0 \neq x_1$ but $f(x_0) = f(x_1)$.
%
%
\begin{example}
    Let $f: \{0,1\} \to \{*\}$ be the function mapping $0 \mapsto *$, $1 \mapsto *$. Then $\{0,1\} \times_f \{0,1\}$ is the set \{(0,0),(0,1),(1,0),(1,1)\}, and $\dhom{1}{(\slice{\catset}{\{*\}})}{f}$ has the following structure:
    %
    %
    \begin{equation*}
    \def\interval{1.75}
        \scalebox{0.75}{
        \begin{tikzpicture}
            \node (4a) at (0,4*\interval) {$\{(0,0),(0,1),(1,0),(1,1)\}$};

            \node (3a) at (-6,3*\interval) {$\{(0,0),(0,1),(1,1)\}$};
            \node (3b) at (-2,3*\interval) {$\{(0,1),(1,0),(1,1)\}$};
            \node (3c) at (2,3*\interval) {$\{(0,0),(0,1),(1,0)\}$};
            \node (3d) at (6,3*\interval) {$\{(0,0),(1,0),(1,1)\}$};

            \node (2a) at (-8,2*\interval) {$\{(1,1),(0,1)\}$};
            \node (2b) at (-4,2*\interval) {$\{(0,0),(0,1)\}$};
            \node (2c) at (0,2*\interval) {$\{(0,1),(1,0)\}$};
            \node (2d) at (4,2*\interval) {$\{(1,1),(1,0)\}$};
            \node (2e) at (8,2*\interval) {$\{(0,0),(1,0)\}$};

            \node (1a) at (-4,1*\interval) {$\{(0,1)\}$};
            \node (1b) at (4,1*\interval) {$\{(1,0)\}$};

            \node (0) at (0,0) {$\varnothing$};

            \draw[thick] (3a) -- (4a);
            \draw[thick] (3b) -- (4a);
            \draw[thick] (3c) -- (4a);
            \draw[thick] (3d) -- (4a);
            
            \draw[thick] (2a) -- (3a);
            \draw[thick] (2a) -- (3b);
            \draw[thick] (2b) -- (3a);
            \draw[thick] (2b) -- (3c);
            \draw[thick] (2c) -- (3b);
            \draw[thick] (2c) -- (3c);
            \draw[thick] (2d) -- (3b);
            \draw[thick] (2d) -- (3d);
            \draw[thick] (2e) -- (3d);
            \draw[thick] (2e) -- (3c);

            \draw[thick] (1a) -- (2a);
            \draw[thick] (1a) -- (2b);
            \draw[thick] (1a) -- (2c);
            \draw[thick] (1b) -- (2c);
            \draw[thick] (1b) -- (2d);
            \draw[thick] (1b) -- (2e);

            \draw[thick] (0) -- (1a);
            \draw[thick] (0) -- (1b);
        \end{tikzpicture}
        }
    \end{equation*}
Notice that, via the isomorphism $\Set \simeq \slice{\Set}{\{*\}}$, this is isomorphic to $\dhom{1}{\Set}{\{0, 1\}}$.
\end{example}
%
%
To conclude, suppose that two morphisms are both components of the same natural transformation.
Is there a relation between the associated invariants?
The following result answers this question in the affirmative.
%
%
\begin{proposition}[Covariance over the domain of a natural transformation] \label{prop: covariance natural transformation}
Let $\fun{F}, \fun{G}\colon \cat{C} \to \cat{D}$ be two functors and let $\alpha\colon \fun{F} \Rightarrow \fun{G}$ be a natural transformation.
For all $i \in \{ 0, 1\}$, the assignment
\begin{equation*}
    x \; \mapsto \; \dhom{i}{(\slice{\cat{D}}{\fun{G}{x}})}{\alpha_x}
\end{equation*}
extends to a functor $\cat{C} \to \pointed{\catpos}$.
\end{proposition}
%
%
Notice that this is \emph{not} simply a consequence of \autoref{prop: Homotopy posets are functorial}, that is, it does not arise from the general functoriality result by pre-composition with another functor.\footnote{There is a unifying perspective on the two functoriality results, involving the theory of fibrations and cofibrations of categories; this will be discussed in an extended technical paper.}
It implies that we can naturally map obstructions for $\alpha_x$ to obstructions for $\alpha_y$ along a morphism $f\colon x \to y$ in $\cat{C}$; we can think of morphisms in $\cat{C}$ as inducing a ``flow'' of obstructions to the components of $\alpha$, under which a non-trivial obstruction may be trivialised, but it can never be the case that a non-obstruction is ``un-trivialised''.
