\section*{Conclusion}

We have introduced our new invariants of categories and stated their fundamental properties, before sketching, through a couple of simple examples, how they may be used to obtain a more fine-grained analysis of ``failures of compositionality'' than a simple yes-or-no judgement.
In an extended technical paper, we will study their formal aspects more in depth, including criteria for the existence of joins and meets, induced monoidal structures, and finer aspects of functoriality.

Most importantly, we hope to have opened a new avenue in ``formal compositionality theory''.
The greatest challenge will be to graduate from proof-of-concept examples to ones that reveal more interesting structure, perhaps in non-$\Set$-like categories where a split epi or mono is not simply a surjective or injective map.
We have been looking at case studies of this sort, which nevertheless have manageable combinatorics permitting an exhaustive study of their homotopy posets, and we hope to discuss them in future work.