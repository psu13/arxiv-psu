Now, we investigate obstructions to \emph{subterminality}.
Our main strategy will be to recast subterminality in a way that allows us to leverage \autoref{def: 0th-directed homotopy poset}.
We know that an object $\WTerm$ fails to be subterminal when, for an object $x$, the arrow $x \to \WTerm$ is not unique.
As such, we will describe obstructions to subterminality as pairs of parallel, unequal arrows.
%
%
\begin{definition}[Category of parallel arrows over an object]\label{def: category of parallel arrows}
    Let $\CategoryC$ be a category and $x$ an object in $\CategoryC$.
    The \emph{category of parallel arrows in $\CategoryC$ over $x$} is the category $\pararr{\CategoryC}{x}$ where:
    %
    %
    \begin{itemize}
        \item Objects are pairs of morphisms $(f_0, f_1\colon y \to x)$ with codomain $x$.
        \item A morphism from $(f_0, f_1\colon y \to x)$ to $(g_0, g_1\colon z \to x)$ is a morphism $h\colon y \to z$ such that $f_0 = h\Cp g_0$ and $f_1 = h\Cp g_1$.
    \end{itemize}
    This comes with a projection functor $\mathrm{dom}\colon \pararr{\CategoryC}{x} \to \CategoryC$ sending a parallel pair to its domain.
\end{definition}
%
%    
\begin{proposition}\label{prop: subterminal as weak terminal parallel arrow}
    Let $\CategoryC$ be a category and $\WTerm$ an object in $\CategoryC$.
    The following are equivalent:
    %
    %
    \begin{enumerate}[label=(\alph*)]
        \item $\WTerm$ is subterminal in $\CategoryC$;
        \item $(\idd{\WTerm}, \idd{\WTerm})$ is a terminal object in $\pararr{\CategoryC}{\WTerm}$;
        \item $(\idd{\WTerm}, \idd{\WTerm})$ is a weak terminal object in $\pararr{\CategoryC}{\WTerm}$.
    \end{enumerate}
\end{proposition}
%
%
\autoref{prop: subterminal as weak terminal parallel arrow} allows us to reduce the study of obstructions to subterminality of an object $\WTerm$ in $\CategoryC$ to the study of obstructions to weak terminality of $(\idd{\WTerm}, \idd{\WTerm})$ in $\pararr{\CategoryC}{\WTerm}$. 
%
%
\begin{definition}[First homotopy poset] \label{def: 1st-directed homotopy poset}
    Let $\CategoryC$ be a category and $x$ an object in $\CategoryC$.
    The \emph{first homotopy poset of $\CategoryC$ over $x$} is the pointed poset
    %
    %
    \begin{equation*}
        (\dhom{1}{\CategoryC}{x}, \, [x]) \eqdef \left(\dhom{0}{\pararr{\CategoryC}{x}}{(\idd{x}, \idd{x})}, \, [(\idd{x}, \idd{x})]\right).
    \end{equation*}
\end{definition}
%
%
Putting together the description of the 0th homotopy poset, the definition of $\pararr{\CategoryC}{x}$ in \autoref{def: category of parallel arrows}, and \autoref{prop: subterminal as weak terminal parallel arrow}, we see that an element of $\dhom{1}{\CategoryC}{x}$ is either $[x]$, or $\posref{(f, g)}$ for some parallel pair of morphisms $f, g\colon y \to x$ in $\CategoryC$ with $f \neq g$.
    %
    The order relation is defined as follows:
    \begin{itemize}
        \item $[x] \leq [x]$ trivially;
        \item $[x] \leq \posref{(f, g\colon y \to x)}$ if and only if there exists a morphism $h\colon z \to y$ in $\CategoryC$ equalising $(f, g)$, that is, satisfying $h\Cp f = h\Cp g$;
        \item it is never the case that $\posref{(f, g)} \leq [x]$;
        \item $\posref{(f, g\colon y \to x)} \leq \posref{(f', g'\colon y' \to x)}$ if and only if there exists a morphism $h\colon y \to y'$ such that $f = h\Cp f'$ and $g = h\Cp g'$ in $\CategoryC$.
    \end{itemize}
%
%
\begin{proposition} \label{prop:dhom1_trivial_when_subterminal}
Let $\cat{C}$ be a category and $x$ an object in $\cat{C}$.
The following are equivalent:
\begin{enumerate}[label=(\alph*)]
    \item $\dhom{1}{\cat{C}}{x} = \{[x]\}$;
    \item $x$ is subterminal in $\cat{C}$.
\end{enumerate}
\end{proposition}

\begin{corollary} \label{prop:dhoms_trivial_when_terminal}
Let $\cat{C}$ be a category and $x$ an object in $\cat{C}$.
The following are equivalent:
\begin{enumerate}[label=(\alph*)]
    \item $\dhom{0}{\cat{C}}{x} = \{[x]\}$ and $\dhom{1}{\cat{C}}{x} = \{[x]\}$,
    \item $x$ is a terminal object in $\cat{C}$.
\end{enumerate}
\end{corollary}

\begin{remark}\label{rem: pi1 of a groupoid}
    Recall that the (underlying set of the) fundamental group of a pointed topological space $(X, x)$ is defined by
    \begin{equation*}
        \pi_1(X, x) \eqdef \pi_0(\Omega(X, x), c_x)
    \end{equation*}
    %
    where $\Omega(X, x)$ is the space of loops in $X$ based at $x$, and $c_x$ is the constant path at $x$.
    For a pointed groupoid, which may be seen as the fundamental groupoid of a pointed space, this reduces to the set of automorphisms of the object $x$, pointed with the identity automorphism.
    
    The definition of $\dhom{1}{\CategoryC}{x}$ is made in analogy with this, letting the category of parallel arrows over $x$ replace the space of loops based at $x$, and a pair of identity morphisms replace the constant path.
    The following result proves that, just like the zeroth homotopy poset, the first homotopy poset is a generalisation of its groupoidal analogue.
\end{remark}
%
%
\begin{proposition}[$\dhom{1}{\cat{G}}{x}$ for a groupoid]\label{prop: dhom1 is pi1 for groupoids}
    Let $\cat{G}$ be a groupoid and $x$ an object in $\cat{G}$.
    Then:
    \begin{enumerate}
        \item $\dhom{1}{\cat{G}}{x}$ is a ``set'', that is, a discrete poset, and
        \item as a pointed set, it is isomorphic to the underlying pointed set of the group $\pi_1(\cat{G}, x) = \homset{\cat{G}}{x}{x}$.
    \end{enumerate}
\end{proposition}

\begin{remark}
    We mention here that the field of \emph{directed algebraic topology} \cite{grandis2009directed, fajstrup2016directed} has also produced ``non-invertible'' versions of $\pi_1$, namely, the fundamental \emph{category} and \emph{monoids}, that apply to directed spaces.
    If applied to a category, these pick out ``tautologically'' the category itself and its monoids of endomorphisms.
    To our knowledge, there is no strong relation to our line of research.
\end{remark}

%The elements of $\dhom{0}{\CategoryC}{x}$ and $\dhom{1}{\CategoryC}{x}$ that are minimal in the complement of $\{ [x] \}$ are often of particular importance, for the following reason.
%
%
%\begin{proposition}[Existence of joins]\label{prop: If C has coproducts and slice has weak initials, then dhom0 has joins}
%Let $\cat{C}$ be a category, $x$ an object of $\cat{C}$, and $\kappa$ a cardinal.
%If $\cat{C}$ has $\kappa$\nbd small coproducts, then $\dhom{0}{\cat{C}}{x}$ and $\dhom{1}{\cat{C}}{x}$ have $\kappa$\nbd small joins.
%\end{proposition}
%
%This has the consequence that, in many cases, elements of the homotopy posets are describable as joins of smaller elements; in particular, minimal elements in the complement of $\{ [x] \}$.
%We will call these \emph{minimal obstructions}.

To conclude this section, we show in what way the homotopy posets are functorial in the pair $(\cat{C}, x)$ of a category and an object.

\begin{proposition}[Functoriality of the homotopy posets] \label{prop: Homotopy posets are functorial}
Let $\cat{C}$ be a category, $i \in \{0, 1\}$.
Then:
\begin{enumerate}
    \item the assignment $x \mapsto \dhom{i}{\cat{C}}{x}$ extends to a functor
        $\dhom{i}{\cat{C}}{-}\colon \cat{C} \to \pointed{\catpos}$;
    \item a functor $\fun{F}\colon \cat{C} \to \cat{D}$ induces a natural transformation 
        $\pi_i(\fun{F})\colon \dhom{i}{\cat{C}}{-} \Rightarrow \dhom{i}{\cat{D}}{\fun{F}-}.$
\end{enumerate}
Given another functor $\fun{G}\colon \cat{D} \to \cat{E}$,  this assignment satisfies
\begin{equation*}
    \pi_i(\fun{F}\Cp \fun{G}) = \pi_i(\fun{F}) \Cp \pi_i(\fun{G}), \quad \quad \pi_i(\idd{C}) = \idd{\dhom{i}{\cat{C}}{-}}.
\end{equation*}
\end{proposition}

A concise way of packaging this information is to say that $\pi_i$ defines a functor from $\catcat$ to the \emph{lax slice} $\laxslice{\lcatcat}{\pointed{\catpos}}$, where $\lcatcat$ is the ``huge'' category of possibly large categories.
The objects of the lax slice are pairs of a possibly large category $\lcat{C}$ and a functor $\lcat{C} \to \pointed{\catpos}$, and the morphisms are triangles of functors commuting up to a natural transformation.
Indeed, given $\fun{F}\colon \cat{C} \to \cat{D}$, we have a triangle
\begin{equation*}
\begin{tikzcd}
	{\cat{C}} &&& \pointed{\catpos} \\
	\\
	{\cat{D}}
	\arrow["{\fun{F}}"', from=1-1, to=3-1]
	\arrow["{\dhom{i}{\cat{D}}{-}}"', from=3-1, to=1-4]
	\arrow[""{name=0, anchor=center, inner sep=0}, "{\dhom{i}{\cat{C}}{-}}", from=1-1, to=1-4]
	\arrow["{\pi_i(\fun{F})}"', shorten <=17pt, shorten >=26pt, Rightarrow, from=0, to=3-1]
\end{tikzcd}
\end{equation*}
commuting up to the natural transformation $\pi_i(\fun{F})$.
%
%
\begin{remark}[Dual invariants] \label{rmk: Dual invariants}
As usual, all the constructions can be dualised to $\opp{\cat{C}}$.
This will replace the slice over an object and its domain opfibration with the slice under an object and its codomain fibration, producing invariants classifying obstructions to \emph{initiality} of the object.
\end{remark}
