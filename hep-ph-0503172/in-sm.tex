\setcounter{section}{0} 
\renewcommand{\thesection}{}

\section{\hspace*{-.5cm} Pr\'eambule}

\subsubsection*{A short praise of the Standard Model}

The end of the last millennium witnessed the triumph of the Standard Model (SM)
of the electroweak and strong interactions of elementary particles
\cite{GSW,QCD}.  The electroweak theory, proposed by Glashow, Salam and
Weinberg \cite{GSW} to describe the electromagnetic \cite{QED} and weak
\cite{WEAK} interactions between quarks and leptons, is based on the gauge
symmetry group ${\rm SU(2)_L \times U(1)_Y}$ of weak left--handed isospin and
hypercharge. Combined with Quantum Chromo--Dynamics (QCD) \cite{QCD}, the
theory of the strong interactions between the colored quarks based on the 
symmetry group ${\rm SU(3)_C}$, the model provides a unified framework to
describe these three forces of Nature. The theory is perturbative at
sufficiently high energies \cite{QCD} and renormalizable \cite{RENORM}, and
thus describes these interactions at the quantum level.\s

A cornerstone of the SM is the mechanism of spontaneous electroweak symmetry
breaking (EWSB) proposed forty years ago by Higgs, Brout, Englert, Guralnik,
Hagen and Kibble \cite{Higgs} to generate the weak vector boson masses in a way
that is minimal and, as was shown later, respects the requirements of
renormalizability \cite{RENORM} and unitarity \cite{UNITARITY}.  An SU(2)
doublet of complex scalar fields is introduced and its neutral component
develops a non--zero vacuum expectation value. As a consequence, the
electroweak ${\rm SU(2)_L \times U(1)_Y}$ symmetry is spontaneously broken to
the electromagnetic ${\rm U(1)_Q}$ symmetry.  Three of the four degrees of
freedom of the doublet scalar field are absorbed by the $W^\pm$ and $Z$ weak
vector bosons to form their longitudinal polarizations and to acquire masses. 
The fermion masses are generated through a Yukawa interaction with the same
scalar field and its conjugate field.  The remaining degree of freedom
corresponds to a scalar particle, the Higgs boson. The discovery of this new
type of matter particle is unanimously considered as being of profound
importance.\s

The high--precision measurements of the last decade \cite{High-Precision,PDG}
carried out at LEP, SLC, Tevatron and elsewhere have provided a decisive test
of the Standard Model and firmly established that it provides the correct
effective description of the strong and electroweak interactions at present
energies. These tests, performed at the per mille level accuracy, have probed
the quantum corrections and the structure of the ${\rm SU(3)_C\times SU(2)_L
\times U(1)_Y}$ local symmetry.  The couplings of quarks and leptons to the
electroweak gauge bosons have been measured precisely and agree with those
predicted by the model. The trilinear couplings among electroweak vector bosons
have also been measured and agree with those dictated by the ${\rm
SU(2)_L\times U(1)_Y}$ gauge symmetry. The ${\rm SU(3)_C}$ gauge symmetric
description of the strong interactions has also been thoroughly tested at LEP
and elsewhere.  The only sector of the model which has not yet been probed in a
satisfactory way is the scalar sector. The missing and most important
ingredient of the model, the Higgs particle, has not been observed
\cite{PDG,LEP2-Higgs-exp} and only indirect constraints on its mass have
been inferred from the high--precision data \cite{High-Precision}.\s

\subsubsection*{Probing electroweak symmetry breaking: a brief survey of recent
developments} 

The SM of the electroweak interactions, including the EWSB mechanism for
generating particle masses, had been proposed in the mid--sixties; however, it
was only in the mid--seventies, most probably after the proof by 't Hooft and
Veltman that it is indeed a renormalizable theory \cite{RENORM} and the
discovery of the weak neutral current in the Gargamelle experiment
\cite{Gargamelle}, that all its facets began to be investigated thoroughly. 
After the discovery of the $W^\pm$ and $Z$ bosons at CERN \cite{WZ-Discovery},
probing the electroweak symmetry breaking  mechanism became a dominant theme of
elementary particle physics. The relic of this mechanism, the Higgs particle,
became the Holy Grail of high--energy collider physics and {\it l'objet de tous
nos d\'esirs}. Finding this particle and studying its fundamental properties
will be the major goal of the next generation of high--energy machines [and of
the upgraded Tevatron, if enough lumino\-sity is collected]: the CERN Large
Hadron Collider (LHC), which will start operation in a few years, and the next
high--energy and high--luminosity electron--positron linear collider, which
hopefully will allow very detailed studies of the EWSB mechanism in a decade.\s
 
In the seventies and eighties, an impressive amount of theoretical knowledge
was amassed on EWSB and on the expected properties of the Higgs boson(s), both
within the framework of the SM and  of its [supersymmetric and non 
supersymmetric]
extensions.  At the end of the eighties, the basic properties of the Higgs
particles had been discussed and their principal decay modes and main
production mechanisms at hadron and lepton colliders explored.  This
monumental endeavor was nicely and extensively reviewed in a celebrated book,
{\it The Higgs Hunter's Guide} \cite{HHG} by Gunion, Haber, Kane and
Dawson.  The constraints from the experimental data available at that time and
the prospects for discovering the Higgs particle(s) at the upcoming
high--energy experiments, the LEP, the SLC, the late SSC and the LHC, as well
as possible higher energy $\ee$ colliders, were analyzed and summarized. The
review indeed guided theoretical and phenomenological studies as well as
experimental searches performed over the last fifteen years.\s 

Meanwhile, several major developments took place. The LEP experiment, for which
the search for the Higgs boson was a central objective, was completed with mixed
results. On the one hand,  LEP played a key role in establishing the SM as the
effective theory of the strong and electroweak forces at presently accessible
energies. On the other hand, it unfortunately failed to find the Higgs particle
or any other new particle which could play a similar role.  Nevertheless, this
negative search led to a very strong limit on the mass of a SM--like Higgs
boson, $M_H \gsim 114.4$ GeV \cite{LEP2-Higgs-exp}.  This unambiguously ruled
out a broad low Higgs mass region, and in particular the range $M_H \lsim 5$
GeV, which was rather difficult to explore\footnote{This is mainly due to the
hadronic uncertainties which occur for such a small Higgs mass. Almost an
entire chapter of Ref.~\cite{HHG} was devoted to this mass range; see
pp.~32--56 and 94--129.}  before the advent of LEP1 and its very clean
experimental environment.  The mass range $M_H \lsim 100$ GeV would have been
extremely difficult to probe at very high--energy hadron colliders such as the
LHC.  At approximately the same period, the top quark was at last discovered at
the Tevatron \cite{Top-Discovery}. The determination of its mass entailed
that all the parameters of the Standard Model, except the Higgs boson mass, 
were then known\footnote{Another important outcome is due to the heaviness of 
the top quark \cite{Mt-Tevatron}: the search of the Higgs boson would have been
extremely more difficult at hadron colliders if the top quark mass were smaller
than $M_W$, a possibility for which many analyses were devoted in the past and
which is now ruled out. As a by--product of the large $m_t$ value, the cross
sections for some Higgs production channels at both hadron and $\ee$ machines
became rather large, thus increasing the chances for the discovery and/or 
study of the particle.}, implying that the profile of the Higgs boson will be
uniquely determined once its mass is fixed. \s 

Other major developments occurred in the planning and design of the
high--energy colliders. The project of the Superconducting Super Collider has
been unfortunately terminated and the energy and luminosity parameters of the
LHC became firmly established\footnote{The SSC was a project for a hadron
machine with a center of mass energy of $\sqrt{s}=40$ TeV and a yearly
integrated luminosity of 10 fb$^{-1}$ on which most of the emphasis for Higgs
searches at hadron colliders was put in Ref.~\cite{HHG}. Of course these
studies can be and actually have been adapted to the case of the LHC. Note
that in the late eighties, the c.m.  energy and the luminosity of the LHC were
expected to be $\sqrt{s}=17$ TeV and ${\cal L}=10^{33}$ cm$^{-2}$s$^{-1}$,
respectively, and the discovery range for the SM Higgs boson was considered to
be rather limited, $2M_W \lsim M_H \lsim 300$ GeV \cite{HHG}.}.  Furthermore,
the option of upgrading the Tevatron by raising the c.m.~energy and, more
importantly, the luminosity to a value which allows for Higgs searches in the
mass range $M_H \lsim 2M_Z$ was not yet considered. In addition, the path
toward future high--energy electron--positron colliders became more precise. 
The feasibility of the next generation machines, that is, $\ee$ linear
colliders operating in the energy range from $M_Z$ up to 1 TeV with very high
luminosities has been demonstrated [as in the case of the TESLA machine] and a
consensus on the technology of the future International Linear Collider (ILC)
has recently emerged. The designs for the next generation machines running at
energies in the multi--TeV range [such as the CLIC machine at CERN] also made
rapid developments.  Added to this, the option of turning future linear
colliders into high--energy and high--luminosity $\gamma \gamma$ colliders by
using Compton back--scattering of laser light off the high--energy electron
beams and the possibility of high--energy muon colliders have been seriously
discussed only in the last decade. \s

In parallel to these experimental and technological developments, a huge amount
of effort has been devoted to the detailed study of the decay and production
properties of the Higgs particle at these colliders. On the theoretical side,
advances in computer technology allowed one to perform almost automatically
very complicated calculations for loop diagrams and multi--particle processes
and enabled extremely precise predictions.  In particular, the
next--to--leading order radiative corrections to Higgs production in all the
important processes at hadron and $\ee$ colliders were calculated\footnote{This
started in the very late eighties and early nineties, when the one--loop QCD
corrections to associated Higgs production with $W/Z$ bosons and the $WW/ZZ$
and gluon--gluon fusion mechanisms at hadron colliders and the electroweak
corrections to the Higgs--strahlung production mechanism at $\ee$ colliders
have been derived, and continued until very recently when the QCD corrections
to associated Higgs production with heavy quarks at hadron colliders and the
electroweak corrections to all the remaining important Higgs production
processes at lepton colliders have been completed.}.  The radiative corrections
to the cross sections for some production processes, such as Higgs--strahlung
and gluon--gluon fusion at hadron colliders, have been calculated up to
next--to--next--to--leading order accuracy for the strong interaction part and
at next--to--leading order for the electroweak part, a development which
occurred only over the last few years. A vast literature on the higher order
effects in Higgs boson decays has also appeared in the last fifteen years and
some decay modes have been also investigated to next--to--next--to--leading
order accuracy and, in some cases, even beyond.  Moreover, thorough theoretical
studies of the various distributions in Higgs production and decays and new
techniques for the determination of the fundamental properties of the Higgs
particle [a vast subject which was only very briefly touched upon in
Ref.~\cite{HHG} for instance] have been recently carried out.\s

Finally, a plethora of analyses of the various Higgs signals and backgrounds,
many detailed parton--level analyses and Monte--Carlo simulations taking into
account the experimental environment [which is now more or less established, at
least for the Tevatron and the LHC and possibly for the first stage of the
$\ee$ linear collider, the ILC] have been performed to assess to what extent 
the Higgs particle can be observed and its properties studied in given 
processes at the various machines.

\subsubsection*{Objectives and limitations of the review}

On the experimental front, with the LEP experiment completed, we await the
accumulation of sufficient data from the upgraded Tevatron and the launch of
the LHC which will start operation in 2007. At this point, we believe that it
is useful to collect and summarize the large amount of work carried out over
the last fifteen years in preparation for the challenges ahead. This review is
an attempt to respond to this need. The review is structured in three parts. 
In this first part, we will concentrate on the Higgs boson of the Standard
Model, summarize the present experimental and theoretical information on the
Higgs sector, analyze the decay modes of the Higgs bosons including all the
relevant and important higher order effects, and discuss the production
properties of the Higgs boson and its detection strategies at the various
hadron and lepton machines presently under discussion. We will try to be 
as extensive and comprehensive as possible.\s 

However, because the subject is vast and the number of studies related to it is
huge\footnote{Simply by typing ``find title Higgs" in the search field of the
Spires database, one obtains more than 6.700 entries. Since this number does
not include all the articles dealing with the EWSB mechanism and not explicitly
mentioning the name of Prof. Higgs in the title, the total number of articles
written on the EWSB mechanism in the SM and its various extensions may, thus,
well exceed the level of 10.000.}, it is almost an impossible task to review
all its aspects. In addition, one needs to cover many different topics and each
of them could have [and, actually, often does have] its own review.  Therefore,
in many instances, one will have to face [sometimes Cornelian...] choices. The
ones made in this review  will be, of course, largely determined by the taste
of the author, his specialization and his own prejudice. I therefore apologize
in advance if some important aspects are overlooked and/or some injustice to
possibly relevant analyses is made.  Complementary material on the foundations
of the SM and the Higgs mechanism, which will only be briefly sketched here,
can be found in standard textbooks \cite{BOOKS} or in general reviews
\cite{SM-REVIEWS,Reviews-Higgs} and an account of the various calculations,
theoretical studies and phenomenological analyses mentioned above can be found
in many specialized reviews; see
Refs.~\cite{RCreviewEW,Reviews-AD,Review-Michael,RCreviewQCD,Review-CH} for
some examples. For the physics of the Higgs particle at the various colliders,
in particular for the discussion of the Higgs signals and their respective
backgrounds, as well as for the detection techniques, we  will simply summarize
the progress so far.  For this very important issue, we refer for additional
and more detailed informations to specialized reviews and, above all, to the
proceedings which describe the huge collective efforts at the various workshops
devoted to the subject.  Many of these studies and reviews will be referenced
in due time.  

\subsubsection*{Synopsis of the review}

The first part of this review (Tome I) on the electroweak symmetry breaking
mechanism is exclusively devoted to the SM Higgs particle. The discussion
of the Higgs sector of the Minimal Supersymmetric extension of the SM is 
given in an accompanying report \cite{Tome2}, while the EWSB mechanism in other
supersymmetric and non--supersymmetric extensions of the SM will be discussed
in a forthcoming report \cite{Tome3}. In our view, the SM incorporates an
elementary Higgs boson with a mass below 1 TeV and, thus, the very heavy or the
no--Higgs scenarios will not be discussed here and postponed to
Ref.~\cite{Tome3}.\s 

The first chapter is devoted to the description of the Higgs sector of the SM. 
After briefly recalling the basic ingredients of the model and its input
parameters, including an introduction to the electroweak symmetry breaking
mechanism and to the basic properties of the Higgs boson, we discuss the
high--precision tests of the SM and introduce the formalism which allows a
description of the radiative corrections which involve the contribution of the
only unknown parameter of the theory, the Higgs boson mass $M_H$ or,
alternatively, its self--coupling. This formalism will be needed when we
discuss the radiative corrections to Higgs decay and production modes. We then
summarize the indirect experimental constraints on $M_H$ from the
high--precision measurements and the constraints derived from direct Higgs
searches at past and present colliders. We close this chapter by discussing
some interesting constraints on the Higgs mass that can be derived from
theoretical considerations on the energy range in which the SM is valid before
perturbation theory breaks down and new phenomena should emerge.  The bounds on
$M_H$ from unitarity in scattering amplitudes, perturbativity of the Higgs
self--coupling, stability of the electroweak vacuum and fine--tuning in the
radiative corrections in the Higgs sector, are analyzed. \s

In the second chapter, we explore the decays of the SM Higgs particle. We
consider all decay modes which lead to potentially observable branching 
fractions: decays into quarks and leptons, decays into weak
massive vector bosons and loop induced decays into gluons and photons. We
discuss not only the dominant two--body decays, but also higher order decays,
which can be very important in some cases. We pay particular attention to the
radiative corrections and, especially, to the next--to--leading order QCD
corrections to the hadronic Higgs decays which turn out to be quite large. 
The higher order QCD corrections [beyond NLO] and the important electroweak
radiative corrections to all decay modes are briefly summarized. The expected
branching ratios of the Higgs particle, including the uncertainties which
affect them, are given.  Whenever possible, we compare the various decay
properties of the SM Higgs boson, with its distinctive spin and parity $J^{\rm
PC}=0^{++}$ quantum numbers, to those of hypothetical pseudoscalar Higgs bosons
with $J^{\rm PC}=0^{+-}$ which are predicted in many extensions of the SM Higgs
sector. This will highlight the unique prediction for the properties of the SM
Higgs particle [the more general case of anomalous Higgs couplings will be
discussed in the third part of this review]. \s

The third chapter is devoted to the production of the Higgs particle at hadron
machines. We consider both the $p\bar p$ Tevatron collider with a center of
mass energy of $\sqrt{s}=1.96$ TeV and the $pp$ Large Hadron Collider (LHC)
with a center of mass energy of $\sqrt{s}=14$ TeV. All the dominant production
processes, namely the associated production with $W/Z$ bosons, the weak vector
boson fusion processes, the gluon--gluon fusion mechanism and the associated
Higgs production with heavy top and bottom quarks, are discussed in detail.  In
particular, we analyze not only the total production cross sections, but also
the differential distributions and we pay  special attention to three 
important aspects: the QCD radiative corrections or the $K$--factors [and the
electroweak corrections when important] which are large in many cases, their
dependence on the renormalization and factorization scales which measures the
reliability of the theoretical predictions, and the choice of different sets of
parton distribution functions. We also discuss other production processes such
as Higgs pair production, production with a single top quark, production in
association with two gauge bosons or with one gauge boson and two quarks as
well as diffractive Higgs production. These channels are not considered as
Higgs discovery modes, but they might provide additional interesting
information.  We then summarize the main Higgs signals in the various detection
channels at the Tevatron and the LHC and the expectations for observing them 
experimentally.  At the end of this chapter, we briefly discuss the possible
ways of determining some of the properties of the Higgs particle at
the LHC: its mass and total decay width, its spin and parity quantum numbers
and its couplings to fermions and gauge bosons.  A brief summary of the
benefits that one can expect from raising the luminosity and energy
of hadron colliders is given.\s 

In the fourth chapter, we explore the production of the SM Higgs boson at
future lepton colliders. We mostly focus on future $\ee$ colliders in the
energy range $\sqrt{s}=350$--1000 GeV as planed for the ILC but we also discuss
the physics of EWSB at multi--TeV machines [such as CLIC] or by revisiting the
$Z$ boson pole [the GigaZ option], as well as at the $\gamma \gamma$ option of
the linear collider and at future muon colliders.  In the case of $\ee$
machines, we analyze in detail the main production mechanisms, the
Higgs--strahlung and the $WW$ boson fusion processes, as well as some
``subleading" but extremely important processes for determining the profile of
the Higgs boson such as associated production with top quark pairs and  Higgs
pair production.  Since $\ee$ colliders are known to be high--precision
machines, the theoretical predictions need to be rather accurate and we
summarize the work done on the radiative corrections to these processes [which
have been completed only recently] and to various distributions which allow to
test the fundamental nature of the Higgs particle.  The expectation for Higgs
production at the various possible center of mass energies and the potential of
these machines to probe the electroweak symmetry breaking mechanism in all its
facets and to check the SM predictions for the fundamental Higgs properties
such as the total width, the spin and  parity quantum numbers, the couplings to
the other SM particles [in particular, the important coupling to the top quark]
and the Higgs self--coupling [which allows the reconstruction of the scalar
potential which generates EWSB] are summarized. Higgs production at $\gamma
\gamma$ and at muon colliders are discussed in the two last sections, with some
emphasis on two points which are rather difficult to explore in $\ee$
collisions, namely, the determination of the Higgs spin--parity quantum numbers
and the total decay width.\s 

Since the primary goal of this review is to provide the necessary material to
discuss Higgs decays and production at present and future colliders, we present
the analytical expressions of the partial decay widths, the production cross
sections and some important distributions, including the higher order
corrections or effects, when they are simple enough to be displayed.  We
analyze in detail the main Higgs decay and production channels and also discuss
some channels which are not yet established but which can be useful and with
further effort might prove experimentally accessible. We also present summary
and updated plots as well as illustrative numerical examples [which can be used
as a normalization in future phenomenological and experimental studies] for the
total Higgs decay width and branching ratios, as well as for the cross sections
of the main production mechanisms at the Tevatron, the LHC and future $\ee$
colliders at various center of mass energies. In these updated analyses, we have
endeavored to include all currently available information. For collider Higgs
phenomenology, in particular for the discussion of the Higgs signals and
backgrounds, we  simply summarize, as previously mentioned, the main points and
refer to the literature for additional details and complementary discussions.  

\subsubsection*{Acknowledgments}

I would like first to thank the many collaborators with whom I shared the
pleasure to investigate various aspects of the theme discussed in this review. 
They are too numerous to be all listed here, but I would like at least to
mention Peter Zerwas with whom I started to work on the subject in an intensive
way.\s

I would also like to thank the many colleagues and friends who helped me during
the writing of this review and who made important remarks on the preliminary
versions of the manuscript and suggestions for improvements: Fawzi Boudjema,
Albert de Roeck, Klaus Desch, Michael Dittmar, Manuel Drees, Rohini Godbole,
Robert Harlander, Wolfgang Hollik, Karl Jakobs, Sasha Nikitenko, Giacomo
Polesello, Francois Richard, Pietro Slavich and Michael Spira.  Special thanks
go to Manuel Drees and Pietro Slavich for their very careful reading of large
parts of the manuscript and for their efficiency in hunting the many typos,
errors and awkwardnesses contained in the preliminary versions and for their
attempt at improving my poor English and fighting against my anarchic way of
``distributing commas". Additional help with the English by Martin Bucher and,
for the submission of this review to the archives, by Marco Picco are also
acknowledged. \s

I also thank the Djouadi {\it smala}, my sisters and brothers and their
children [at least one of them, Yanis, has already caught the virus of particle
physics and I hope that one day he will read this review], who bore my not
always joyful mood in the last two years. Their support was crucial for the
completion of this review. Finally, thanks to the team of {\it La Bonne
Franquette}, where in fact part of this work has been done, for their good
couscous and the nice atmosphere, as well as to Madjid Belkacem for sharing the
drinks with me.\s 

The writing of this review  started when I was at CERN as a scientific
associate, continued during the six months I spent at the LPTHE of Jussieu, 
and ended at the LPT d'Orsay. I thank all these institutions for their kind
hospitality. 

\newpage

\setcounter{section}{0} 
\renewcommand{\thesection}{\arabic{section}}
