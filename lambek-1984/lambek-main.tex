%!TEX root = lambek-1984-pal.tex
%\usepackage[dotinlabels]{titletoc}
%\titlelabel{{\thetitle}.\quad}
%\usepackage{titletoc}
\usepackage[small]{titlesec}

\titleformat{\section}[block]
  {\fillast\medskip}
  {\bfseries{\thesection. }}
  {1ex minus .1ex}
  {\bfseries}
 
\titleformat*{\subsection}{\itshape}
\titleformat*{\subsubsection}{\itshape}

\setcounter{tocdepth}{2}

\titlecontents{section}
              [2.3em] 
              {\bigskip}
              {{\contentslabel{2.3em}}}
              {\hspace*{-2.3em}}
              {\titlerule*[1pc]{}\contentspage}
              
\titlecontents{subsection}
              [4.7em] 
              {}
              {{\contentslabel{2.3em}}}
              {\hspace*{-2.3em}}
              {\titlerule*[.5pc]{}\contentspage}

% hopefully not used.           
\titlecontents{subsubsection}
              [7.9em]
              {}
              {{\contentslabel{3.3em}}}
              {\hspace*{-3.3em}}
              {\titlerule*[.5pc]{}\contentspage}
%\makeatletter
\renewcommand\tableofcontents{%
    \section*{\contentsname
        \@mkboth{%
           \MakeLowercase\contentsname}{\MakeLowercase\contentsname}}%
    \@starttoc{toc}%
    }
\def\@oddhead{{\scshape\rightmark}\hfil{\small\scshape\thepage}}%
\def\sectionmark#1{%
      \markright{\MakeLowercase{%
        \ifnum \c@secnumdepth >\m@ne
          \thesection\quad
        \fi
        #1}}}
        
\makeatother

%\makeatletter

 \def\small{%
  \@setfontsize\small\@xipt{13pt}%
  \abovedisplayskip 8\p@ \@plus3\p@ \@minus6\p@
  \belowdisplayskip \abovedisplayskip
  \abovedisplayshortskip \z@ \@plus3\p@
  \belowdisplayshortskip 6.5\p@ \@plus3.5\p@ \@minus3\p@
  \def\@listi{%
    \leftmargin\leftmargini
    \topsep 9\p@ \@plus3\p@ \@minus5\p@
    \parsep 4.5\p@ \@plus2\p@ \@minus\p@
    \itemsep \parsep
  }%
}%
 \def\footnotesize{%
  \@setfontsize\footnotesize\@xpt{12pt}%
  \abovedisplayskip 10\p@ \@plus2\p@ \@minus5\p@
  \belowdisplayskip \abovedisplayskip
  \abovedisplayshortskip \z@ \@plus3\p@
  \belowdisplayshortskip 6\p@ \@plus3\p@ \@minus3\p@
  \def\@listi{%
    \leftmargin\leftmargini
    \topsep 6\p@ \@plus2\p@ \@minus2\p@
    \parsep 3\p@ \@plus2\p@ \@minus\p@
    \itemsep \parsep
  }%
}%
\def\open@column@one#1{%
 \ltxgrid@info@sw{\class@info{\string\open@column@one\string#1}}{}%
 \unvbox\pagesofar
 \@ifvoid{\footsofar}{}{%
  \insert\footins\bgroup\unvbox\footsofar\egroup
  \penalty\z@
 }%
 \gdef\thepagegrid{one}%
 \global\pagegrid@col#1%
 \global\pagegrid@cur\@ne
 \global\count\footins\@m
 \set@column@hsize\pagegrid@col
 \set@colht
}%

\def\frontmatter@abstractheading{%
\bigskip
 \begingroup
  \centering\large
  \abstractname
  \par\bigskip
 \endgroup
}%

\makeatother

%\DeclareSymbolFont{CMlargesymbols}{OMX}{cmex}{m}{n}
%\DeclareMathSymbol{\sum}{\mathop}{CMlargesymbols}{"50}

\usepackage[papersize={6.6in, 10.0in}, left=.5in, right=.5in, top=.6in, bottom=.9in]{geometry}
\linespread{1.05}
\sloppy
\raggedbottom

\pagestyle{plain}
\usepackage{mathpartir}
\usepackage{stmaryrd}
\usepackage{mathtools}
\usepackage{tikz-cd}
\usepackage{microtype}
\usepackage{enumitem}
\usepackage{nccmath}
\usetikzlibrary {arrows.meta,bending,positioning}
\usepackage{bussproofs}

\newcommand{\mprime}{\ensuremath{^\prime}}

%\usepackage{fdsymbol}

% these include amsmath and that can cause trouble in older docs.
\makeatletter
\@ifpackageloaded{amsmath}{}{\RequirePackage{amsmath}}

\DeclareFontFamily{U}  {cmex}{}
\DeclareSymbolFont{Csymbols}       {U}  {cmex}{m}{n}
\DeclareFontShape{U}{cmex}{m}{n}{
    <-6>  cmex5
   <6-7>  cmex6
   <7-8>  cmex6
   <8-9>  cmex7
   <9-10> cmex8
  <10-12> cmex9
  <12->   cmex10}{}

\def\Set@Mn@Sym#1{\@tempcnta #1\relax}
\def\Next@Mn@Sym{\advance\@tempcnta 1\relax}
\def\Prev@Mn@Sym{\advance\@tempcnta-1\relax}
\def\@Decl@Mn@Sym#1#2#3#4{\DeclareMathSymbol{#2}{#3}{#4}{#1}}
\def\Decl@Mn@Sym#1#2#3{%
  \if\relax\noexpand#1%
    \let#1\undefined
  \fi
  \expandafter\@Decl@Mn@Sym\expandafter{\the\@tempcnta}{#1}{#3}{#2}%
  \Next@Mn@Sym}
\def\Decl@Mn@Alias#1#2#3{\Prev@Mn@Sym\Decl@Mn@Sym{#1}{#2}{#3}}
\let\Decl@Mn@Char\Decl@Mn@Sym
\def\Decl@Mn@Op#1#2#3{\def#1{\DOTSB#3\slimits@}}
\def\Decl@Mn@Int#1#2#3{\def#1{\DOTSI#3\ilimits@}}

\let\sum\undefined
\DeclareMathSymbol{\tsum}{\mathop}{Csymbols}{"50}
\DeclareMathSymbol{\dsum}{\mathop}{Csymbols}{"51}

\Decl@Mn@Op\sum\dsum\tsum

\makeatother

\makeatletter
\@ifpackageloaded{amsmath}{}{\RequirePackage{amsmath}}

\DeclareFontFamily{OMX}{MnSymbolE}{}
\DeclareSymbolFont{largesymbolsX}{OMX}{MnSymbolE}{m}{n}
\DeclareFontShape{OMX}{MnSymbolE}{m}{n}{
    <-6>  MnSymbolE5
   <6-7>  MnSymbolE6
   <7-8>  MnSymbolE7
   <8-9>  MnSymbolE8
   <9-10> MnSymbolE9
  <10-12> MnSymbolE10
  <12->   MnSymbolE12}{}

\DeclareMathSymbol{\downbrace}    {\mathord}{largesymbolsX}{'251}
\DeclareMathSymbol{\downbraceg}   {\mathord}{largesymbolsX}{'252}
\DeclareMathSymbol{\downbracegg}  {\mathord}{largesymbolsX}{'253}
\DeclareMathSymbol{\downbraceggg} {\mathord}{largesymbolsX}{'254}
\DeclareMathSymbol{\downbracegggg}{\mathord}{largesymbolsX}{'255}
\DeclareMathSymbol{\upbrace}      {\mathord}{largesymbolsX}{'256}
\DeclareMathSymbol{\upbraceg}     {\mathord}{largesymbolsX}{'257}
\DeclareMathSymbol{\upbracegg}    {\mathord}{largesymbolsX}{'260}
\DeclareMathSymbol{\upbraceggg}   {\mathord}{largesymbolsX}{'261}
\DeclareMathSymbol{\upbracegggg}  {\mathord}{largesymbolsX}{'262}
\DeclareMathSymbol{\braceld}      {\mathord}{largesymbolsX}{'263}
\DeclareMathSymbol{\bracelu}      {\mathord}{largesymbolsX}{'264}
\DeclareMathSymbol{\bracerd}      {\mathord}{largesymbolsX}{'265}
\DeclareMathSymbol{\braceru}      {\mathord}{largesymbolsX}{'266}
\DeclareMathSymbol{\bracemd}      {\mathord}{largesymbolsX}{'267}
\DeclareMathSymbol{\bracemu}      {\mathord}{largesymbolsX}{'270}
\DeclareMathSymbol{\bracemid}     {\mathord}{largesymbolsX}{'271}

\def\horiz@expandable#1#2#3#4#5#6#7#8{%
  \@mathmeasure\z@#7{#8}%
  \@tempdima=\wd\z@
  \@mathmeasure\z@#7{#1}%
  \ifdim\noexpand\wd\z@>\@tempdima
    $\m@th#7#1$%
  \else
    \@mathmeasure\z@#7{#2}%
    \ifdim\noexpand\wd\z@>\@tempdima
      $\m@th#7#2$%
    \else
      \@mathmeasure\z@#7{#3}%
      \ifdim\noexpand\wd\z@>\@tempdima
        $\m@th#7#3$%
      \else
        \@mathmeasure\z@#7{#4}%
        \ifdim\noexpand\wd\z@>\@tempdima
          $\m@th#7#4$%
        \else
          \@mathmeasure\z@#7{#5}%
          \ifdim\noexpand\wd\z@>\@tempdima
            $\m@th#7#5$%
          \else
           #6#7%
          \fi
        \fi
      \fi
    \fi
  \fi}

\def\overbrace@expandable#1#2#3{\vbox{\m@th\ialign{##\crcr
  #1#2{#3}\crcr\noalign{\kern2\p@\nointerlineskip}%
  $\m@th\hfil#2#3\hfil$\crcr}}}
\def\underbrace@expandable#1#2#3{\vtop{\m@th\ialign{##\crcr
  $\m@th\hfil#2#3\hfil$\crcr
  \noalign{\kern2\p@\nointerlineskip}%
  #1#2{#3}\crcr}}}

\def\overbrace@#1#2#3{\vbox{\m@th\ialign{##\crcr
  #1#2\crcr\noalign{\kern2\p@\nointerlineskip}%
  $\m@th\hfil#2#3\hfil$\crcr}}}
\def\underbrace@#1#2#3{\vtop{\m@th\ialign{##\crcr
  $\m@th\hfil#2#3\hfil$\crcr
  \noalign{\kern2\p@\nointerlineskip}%
  #1#2\crcr}}}

\def\bracefill@#1#2#3#4#5{$\m@th#5#1\leaders\hbox{$#4$}\hfill#2\leaders\hbox{$#4$}\hfill#3$}

\def\downbracefill@{\bracefill@\braceld\bracemd\bracerd\bracemid}
\def\upbracefill@{\bracefill@\bracelu\bracemu\braceru\bracemid}

\DeclareRobustCommand{\downbracefill}{\downbracefill@\textstyle}
\DeclareRobustCommand{\upbracefill}{\upbracefill@\textstyle}

\def\upbrace@expandable{%
  \horiz@expandable
    \upbrace
    \upbraceg
    \upbracegg
    \upbraceggg
    \upbracegggg
    \upbracefill@}
\def\downbrace@expandable{%
  \horiz@expandable
    \downbrace
    \downbraceg
    \downbracegg
    \downbraceggg
    \downbracegggg
    \downbracefill@}

\DeclareRobustCommand{\overbrace}[1]{\mathop{\mathpalette{\overbrace@expandable\downbrace@expandable}{#1}}\limits}
\DeclareRobustCommand{\underbrace}[1]{\mathop{\mathpalette{\underbrace@expandable\upbrace@expandable}{#1}}\limits}

\makeatother


% some nicer symbols
\makeatletter
\DeclareFontFamily{U}{matha}{\hyphenchar\font45}
\DeclareFontShape{U}{matha}{m}{n}{
      <5> <6> <7> <8> <9> <10> gen * matha
      <10.95> matha10 <12> <14.4> <17.28> <20.74> <24.88> matha12
      }{}
\DeclareSymbolFont{matha}{U}{matha}{m}{n}
\DeclareFontSubstitution{U}{matha}{m}{n}

\def\mathabx@aliases#1#2{\@mathabx@aliases#1#2?\@end}
\def\@mathabx@aliases#1#2#3\@end{\ifx#2?\else
	\let#2=#1\@mathabx@aliases#1#3\@end\fi}%
\DeclareMathSymbol{\leftarrow}             {3}{matha}{"D0}
	\mathabx@aliases\leftarrow\gets
\DeclareMathSymbol{\rightarrow}            {3}{matha}{"D1}
	\mathabx@aliases\rightarrow\to
\DeclareMathSymbol{\wedge}         {2}{matha}{"5E}
	\mathabx@aliases\wedge\land
\DeclareMathSymbol{\vee}           {2}{matha}{"5F}
	\mathabx@aliases\vee\lor
\DeclareMathSymbol{\vdash}         {3}{matha}{"24}
\DeclareMathSymbol{\dashv}         {3}{matha}{"25}
\DeclareMathSymbol{\nvdash}        {3}{matha}{"26}
\DeclareMathSymbol{\ndashv}        {3}{matha}{"27}
\DeclareMathSymbol{\vDash}         {3}{matha}{"28}
\DeclareMathSymbol{\Dashv}         {3}{matha}{"29}
\DeclareMathSymbol{\nvDash}        {3}{matha}{"2A}
\DeclareMathSymbol{\nDashv}        {3}{matha}{"2B}
\DeclareMathSymbol{\Vdash}         {3}{matha}{"2C}
\DeclareMathSymbol{\dashV}         {3}{matha}{"2D}
\DeclareMathSymbol{\nVdash}        {3}{matha}{"2E}
\DeclareMathSymbol{\ndashV}        {3}{matha}{"2F}
\makeatother

\usepackage[tiny]{titlesec}
\titleformat{\section}
  {\normalfont\bfseries}
  {\thesection.}
  {.5em}
  {}

\usepackage{cite}

% make sure there is enough TOC for reasonable pdf bookmarks.
\setcounter{tocdepth}{3}
\usepackage{amsthm}

\theoremstyle{definition}


\usepackage[colorlinks=true
,breaklinks=true
,urlcolor=blue
,anchorcolor=blue
,citecolor=blue
,filecolor=blue
,linkcolor=blue
,menucolor=blue
,linktocpage=true]{hyperref}
\hypersetup{
bookmarksopen=true,
bookmarksnumbered=true,
bookmarksopenlevel=10,
pdfencoding=auto, psdextra
}

\date{}
\def\deqd{\mathrel{\cdot\!\!=\!\!\cdot}}
\def\trans{\Phi}
\def\mm{\Vdash}
\def\kxa{\kappa_{x \in A}}
\def\prAB{\pi_{A,B}}
\def\prpAB{\pi'_{A,B}}
\def\pr#1{\pi_{#1}}
\def\prp#1{\pi'_{#1}}
\def\to{\longrightarrow}
\def\xto#1{\xrightarrow{\kern.6em #1 \kern.6em}}
\def\sxto#1{\xrightarrow{\kern.3em #1 \kern.3em}}
\def\ent{\vdash}
\def\entt{\Vdash}
\def\imp{\shortrightarrow}
\def\iff{\leftrightarrow}
\def\from{\Leftarrow}
\def\impl{\Rightarrow}
\def\union{\cup}
\def\fexp#1#2{#1^{#2}}
\def\fexpBA{\fexp{B}{A}}
\def\inc{\subseteq}
\def\Sub{\mathop{\rm sub}}
\def\dom{\mathop{\rm dom}}
\def\cod{\mathop{\rm cod}}
\def\ker{\mathop{\rm ker}}
\def\Ker{\mathop{\rm Ker}}
\def\cha{\mathop{\rm char}}
\def\id{{\mathrm 1}}
\def\res{\!\upharpoonleft\!}
\def\ffam{\varphi}
\def\comp{\circ}
\def\bbone{\mathbb 1}
\def\one{1}
\def\zeromap{0}
\def\bbzero{{\mathbb O}}
\def\ccc{{c.c.c.}}
\def\ev{\varepsilon}
\def\ebc{\varepsilon_{BC}}
\def\evBA{\varepsilon_{B,A}}
\def\L{\Lambda}
\def\OM{\Omega}
\def\LC{{\mathcal L}}
\def\MC{{\mathcal M}}
\def\TC{{\mathcal T}}
\def\FC{{\mathcal F}}
\def\l{\lambda}
\def\lamC{\l\text{-{\bf{\kern 0.2pt}calc}}}
\def\lx{\lambda_x}
\def\ly{\lambda_y}
\def\lu{\lambda_u}
\def\lv{\lambda_v}
\def\lz{\lambda_z}
\def\lxa{\l_{x \in A}}
\def\subX{\ensuremath{_\text{X}}}
\def\lm#1.#2{\lambda#1.\, #2}
\def\br#1{[\, #1 \, ]}
\def\V{V}
\def\U{U}
\def\D{D}
\def\cart{\text{\bf Cart}}
\def\C{\mathcal C}
\def\S{\mathcal S}
\def\lxy{\l x\, \l y . \,}
\def\lmm#1#2.#3{\l #1\, \l #2 . \, #3}
\def\sss{(*\!*\!*)}
\def\ss{(**)}
\def\ssn{(**_n)}
\def\scop{\S^{\C^{op}}}
\def\sland{\wedge}
\def\PU{\mathcal P U}
\def\P{\mathcal P}
\def\UU{(U\to U)}
\def\BA{B \to A}
\def\AB{A \to B}
\def\calA{{\cal A}}
\def\calB{{\cal B}}
\def\cI{{I}}
\def\cS{{S}}
\def\cK{{K}}
\def\cIA{\cI_{A}}
\def\cKAB{\cK_{A,B}}
\def\cSABC{\cS_{A,B,C}}
\def\la{\langle}
\def\ra{\rangle}
\def\bracket#1{\la #1 \ra}
\def\app{\mathop{{}^\wr}\kern-.8pt}
\def\schon{Sch\"onfinkel}
\def\nat{\mathbb N}
\def\pf{\varphi}

\newcommand{\be}{\begin{equation}}
\newcommand{\ee}{\end{equation}}
\newcommand{\bes}{\begin{equation*}}
\newcommand{\ees}{\end{equation*}}

\newcommand{\fcat}[1]{{\mathbf {#1}}} 
\newcommand{\Set}{\fcat{Sets}}
\newcommand{\Top}{\fcat{Top}}
\newcommand{\Lang}{\fcat{Lang}}
\newcommand{\grph}{\fcat{Grph}}
\newcommand{\Spec}{\fcat{Spec}}
\DeclareMathOperator{\Hom}{{Hom}}

\newcommand{\iso}{\cong}                % Isomorphism
\newcommand{\eqv}{\simeq}               % Equivalence
\newcommand{\sub}{\subseteq}            % Subset (possibly not proper)

\usepackage{amsmath,amsthm}

\theoremstyle{definition}

\newtheorem{thm}{Theorem}[section]
\newtheorem{lemma}[thm]{Lemma}
\newtheorem{prop}[thm]{Proposition}
\newtheorem{cor}[thm]{Corollary}
\newtheorem{defn}[thm]{Definition}
\newtheorem{example}[thm]{Example}
\newtheorem{remark}[thm]{Remark}
\newtheorem{note}[thm]{Note}

% makes "=" with "x" under it
\makeatletter
\DeclareRobustCommand{\eqx}{\mathrel{\mathpalette\eq@{X}}}
\DeclareRobustCommand{\eqlx}{\mathrel{\mathpalette\eq@{x}}}
\DeclareRobustCommand{\eqy}{\mathrel{\mathpalette\eq@{Y}}}
\DeclareRobustCommand{\eqxx}{\mathrel{\mathpalette\eq@{X \union \{x\}}}}
\DeclareRobustCommand{\eqtx}{\mathrel{\mathpalette\eq@{\trans(X)}}}
\newcommand{\eq@}[2]{%
  \vtop{\offinterlineskip
    \ialign{\hfil##\hfil\cr
      $\m@th#1=$\cr % top
      \noalign{\sbox\z@{$\m@th#1\mkern0mu$}\kern-\wd\z@}
      $\m@th\alexey@demote{#1}#2$\cr
    }%
  }%
}
\DeclareRobustCommand{\eqdX}{\mathrel{\mathpalette\eqd@{X}}}
\DeclareRobustCommand{\eqdx}{\mathrel{\mathpalette\eqd@{x}}}
\newcommand{\eqd@}[2]{%
  \vtop{\offinterlineskip
    \ialign{\hfil##\hfil\cr
      $\m@th#1\deqd$\cr % top
      \noalign{\sbox\z@{$\m@th#1\mkern0mu$}\kern-\wd\z@}
      $\m@th\alexey@demote{#1}#2$\cr
    }%
  }%
}
\newcommand{\alexey@demote}[1]{%
  \ifx#1\displaystyle\scriptstyle\else
  \ifx#1\textstyle\scriptstyle\else
  \scriptscriptstyle\fi\fi
}

\makeatother

% footnote tricks
\usepackage{footmisc}
%\renewcommand{\thefootnote}{\fnsymbol{footnote}}

\makeatletter
\let\original@footnotemark\footnotemark
\newcommand{\align@footnotemark}{%
  \ifmeasuring@
    \chardef\@tempfn=\value{footnote}%
    \original@footnotemark
    \setcounter{footnote}{\@tempfn}%
  \else
    \iffirstchoice@
      \original@footnotemark
    \fi
  \fi}
\pretocmd{\start@align}{\let\footnotemark\align@footnotemark}{}{}
\makeatother

\makeatletter
\newcommand*\dotop{\mathpalette\bigcdot@{.6}}
\newcommand*\bigcdot@[2]{\mathbin{\vcenter{\hbox{\scalebox{#2}{$\m@th#1\bullet$}}}}}
\makeatother

\title{\large Aspects of Higher Order Categorical Logic\footnote{The authors belong to the ``Groupe interuniversitaire en etudes categoriques'' in Montreal. They acknowledge support from the Natural Sciences and
Engineering Research Council of Canada and from the Quebec Department of
Education. They wish to thank Denis Higgs and the referee for their careful
reading of the manuscript.}}
\author{\normalsize J. Lambek and P. J. Scott%
\footnote{This is a remake of the paper {\em Aspects of Higher Order Categorical Logic}
originally published in Mathematical Applications of Category Theory, AMS, 1984. This
version of the paper was typed out in 2025 by Pete Su.} 
}

\setcounter{section}{-1}
\begin{document}

\maketitle

\section{Introduction}%
It has become clear for some time that categorists and logicians have been
doing the same thing under different names. This situation is strikingly
illustrated in higher order logic, where the connections are particularly
illuminating. In this article we briefly survey portions of our forthcoming
book [LS6], to which the reader is referred for more details.

Systems of higher order logic have been studied for a long time. We have
in mind logical systems in which variables and quantifiers range over functions
(as in the $\l$-calculus) or elements and subsets of given sets (as in type
theory), not just over individuals (as in first order logic). Higher order
concepts naturally occur in mathematics: algebraists quantify over ideals,
topologists over open sets and analysts over functions. Yet, in spite of its
expressive advantage, higher order logic never achieved the popularity of first
order logic. Perhaps its proof theory and model theory were deemed to be too
hard.

Thus, after the monumental work in type theory by Russell and Whitehead
(Principia Mathematica, 1908), there are only sporadic important contributions
to the literature. G\"{o}del (1931) established his incompleteness theorem for
type theory and related systems, Church (1940) combined $\l$-calculus with type
theory, Henkin (1949) discussed models for type theory and, more recently,
Abraham Robinson (1964) used type theory in his book ``Nonstandard Analysis'',
a lead not continued by his followers. Let us also mention that type theories
have recently become important in computer science.

Theories of functionality, namely $\l$-calculi, have fared somewhat better.
After pioneering work by Sch\"onfinkel, Curry, Church and Rosser in the 1920's
and 30's, these systems had a small but devoted following. In the late 60's,
an explosion of activity surrounding the models of Dana Scott (1971) and concomitant
work in computer science revived interest in them.

Meanwhile, the categorical side saw rapid development. The ready accept-
ance of category theory as a working language in many areas of mathematics
foreshadowed its introduction into logic and foundations. Nevertheless, it
came as a surprise to many people when, in the 1960's, Lawvere (1969, 1970) and
one of the present authors [L1] pointed out some remarkable connections between
category theory and logic. Indeed, after the discovery of elementary toposes
by Lawvere and Tierney (1970), the field of ``categorical logic'' underwent
intense development. The reader is referred to the survey article by Kock and
Reyes (1977) and the monograph by Makkai and Reyes (1977).

The situation in higher order logic, as far as it is treated in our book,
is summarized as follows:

\medskip
\renewcommand{\arraystretch}{1.5}

\begin{center}
\begin{tabular}{ l | l }
 \hline
\multicolumn{1}{c|}{Logic} & \multicolumn{1}{c}{Algebra} \\
 \hline
 untyped $\l$-calculus & C-monoids \\
 typed $\l$-calculus & cartesian closed categories \\
 type theory & toposes \\
 \hline
\end{tabular}
\end{center}
\medskip

\noindent
Indeed, we shall see that each side, when suitably formulated, gives rise to a
category. Moreover, the comparison between the two sides is mediated by
functors which set up an equivalence or adjointness. In the meantime, Robert
Seely, has found another such comparison:

\begin{center}
\begin{tabular}{ c | c }
 \hline
{Logic} & {Algebra} \\
 \hline
Martin-L\"of type theory  & locally cartesian closed categories \\
 \hline
\end{tabular}
\end{center}
\medskip

\noindent
Undoubtedly, there are many other such situations. For example, it would be
interesting to pinpoint the logical equivalent of cartesian closed categories
with equalizers.

\section{Cartesian closed categories and typed $\l$-calculi}

Cartesian closed categories are becoming increasingly important in many
branches of mathematics. Indeed, as we shall see, in logic they play a fundamental
role.

A {\em cartesian closed category} is a category $\C$ with (canonical) finite
products and exponentiation. This means that $\C$ has a terminal object $\one$ and
objects $A \times B$ and $\fexp{B}{A}$, for all $A,B$ in $\C$, together
with natural isomorphisms:

\bes
\Hom(C ,A) \times \Hom(C ,B)  \iso \Hom(C ,A \times B) ,
\tag{a}
\ees
\bes
\Hom(C \times A,B) \iso \Hom(C,\fexp{B}{A}).
\tag{b}
\ees

Cartesian closed categories abound in mathematics. For example, all
functor categories $\Set^{\C}$ ($\C$ small) and all toposes (see below) are cartesian
closed. So are the categories of Kelley spaces and Kuratowski limit spaces
(see MacLane, 1971).

A useful alternative presentation considers both products and exponentials
as adjoint functors, hence equationally definable (e.g., see MacLane, 1971).

\def\deq{\mathop{\cdot\mkern-5mu=\mkern-5mu\cdot}}

\begin{defn}
A {\em cartesian closed category} has the following objects,
arrows and equations:

\paragraph{Objects:}
\begin{enumerate}
\item[(i)] $\one$ is an object.
\item[(ii)] If $A$ and $B$ are objects, so are $A \times B$ and $\fexp{B}{A}$
\end{enumerate}

\paragraph{Arrows:}
\begin{enumerate}
\item[(i)] $1_A: A \to A$, $0_A: A \to \one$, $\prAB: A \times B \to A$, 
$\prpAB: A \times B \to B$,
and $\evBA: \fexpBA \times A \to B$ are arrows.

\item[(ii)] The following rules generate new arrows from old:

\[
\inferrule {f: A \to B \quad g: B \to C}{gf: A \to C},\,\,
\inferrule {f: C \to A \quad\! g: C \to B}{\la f, g \ra: C \to A \times B}, \,\,
\inferrule {f: C \times A \to B}{f^*: C \to \fexpBA}
\]

\end{enumerate}

\paragraph{Equations:}
\begin{enumerate}
\item[(1)] $f1_A \deqd f$, $1_Bf \deqd f$, and $(hg)f \deqd g(hf)$,
for all $f:A \to B$, $g:B \to C$, $h: C\to D$;
\item[(2)] $f \deqd 0_A$, for all $f: A \to 1$;
$\prAB\bracket{f,g} \deqd f$, $\prpAB\bracket{f,g} \deqd g$, and
$\la\prAB h, \prpAB h\ra \deqd h$,
for all $f:C \to A$, $g:C \to B$, and $h: C\to A\sland B$;
\item[(3)] $\evBA\bracket{f^*\pr{C,A}, \prp{C,A}} \deqd f$, and
$(\evBA\bracket{g\,\pr{C,A}, \prp{C,A}})^* \deqd g$,
for all $f: C\times A \to B$, and $g: C \to \fexpBA$.
\end{enumerate}

\end{defn}
\noindent
It is understood that $\deqd$ is a congruence relation%
on $\Hom$-sets satisfying the
above. There may be other objects, arrows and equations than follow from the
above.

The reader may easily verify that the equations (1) to (3) contain not
only the equations of a category with a terminal object, but yield also the
natural isomorphisms (a) and (b). These equations may be viewed as presenting
a multi-sorted partial algebra structure on the $\Hom$-sets, with nullary, unary
and binary operations satisfying appropriate identities. Alternatively, they
may be looked upon as describing a ``graphical algebra'' in the spirit of
Burroni (1981). To see this, the reader may first wish to replace the rule of
arrow formation introducing $f^*$ by another basic arrow 
$\eta_{C,A}: C \to \fexp{(C \times A)}{A}$.

Another view of cartesian closed categories is to consider them as
{\em deductive systems} [L1]. Write the objects $\one$, $A \times B$, and $\fexpBA$
as $\top$, $A \sland B$, and $A \impl B$ respectively. One may then think of an arrow
$f: A \to B$ as a {\em deduction} of $B$ from $A$ or as a {\em proof}
of the {\em entailment} $A \ent B$. Proofs may
be writ ten in ``tree form''. For example:

\[
\inferrule {A \sland B \xto{\pi'} B \quad A \sland B \xto{\pi} A}{A \sland B \xto{\la \pi', \pi \ra}  B \sland A}
\]
proves the commutative law for conjunction, while
\[
\inferrule{ }{(A \impl B) \sland A \xto{\evBA} B}
\]
may be regarded as an ``axiom'' (at least in a freely generated deductive system).

Logicians might say that we have presented a system of natural deduction
for the positive intuitionistic propositional calculus, but with an additional
twist: the equations of a cartesian closed category impose an {\em equality
relation} between proofs (or proof trees). For example:
\def\extraVskip{3pt}
\begin{prooftree}
\AxiomC{$C \xto{f} A$}
\AxiomC{$C \xto{g} B$}
\BinaryInfC{$C \xto{\la f,g \ra} A \sland B$}
\AxiomC{$A \sland B \xto{\pi} A$}
\RightLabel{\quad \vbox to 2em{\hbox{\textrm{equals } $C \xto{f} A$}}}
\BinaryInfC{$C \xto{\pi\la f,g \ra} A$}
\end{prooftree}

This notion of ``equivalence of proofs'' is essentially the same as that
introduced by logicians (Prawitz, 1971), as pointed out by Mann (1975).

There is still another connection of cartesian closed categories to logic.
This is via the language of {\em typed $\l$-calculus}. Although older than category
theory, $\l$-calculus may also be regarded as an equational theory of functions
(or functional processes) in which composition is mirrored by substitution.
As we shall see, some problems in cartesian closed categories are efficiently
handled using typed $\l$-calculus.

\begin{defn}
A {\em typed $\l$-calculus} is a formal theory as follows. It
consists of types, terms and equations (between terms of the same type).

\medskip
\noindent
(a) {\bf Types}:
\begin{enumerate}[label=(a\theenumi)]
\item $\one$ is a type.
\item If $A$ and $B$ are types so are $A \times B$ and $\fexpBA$
\end{enumerate}

\medskip
\noindent
(b) {\bf Terms}: (We write ``$t \in A''$ to say that $t$ is a term of type $A$.)
\begin{enumerate}[label=(b\theenumi)]
\item There are countably many variables of each type,
say $x^{A}_{i} \in A$ if $i \in \nat$.
\item $\star \in \one$.
\item If $a \in A$, $b \in B$ and $c \in A \times B$, then 
$\la a,b \ra \in A \times B$, $\prAB(c) \in A$ and $\prpAB(c) \in B$.
\item If $f \in \fexpBA$ and $a \in A$ then $\evBA(f,a) \in B$.
\item If $x$ is a variable of type $A$ and $\phi(x) \in B$
then $\lxa{\varphi(x)} \in \fexpBA$.
\end{enumerate}
\noindent
We often abbreviate $\evBA(f,a)$ as $f \app a$ when the type subscripts are clear
from the context.%
\footnote{Editor's note: Lambek uses this squiggle $\app$ to denote a binary operation for
function application. It appears here and in his later book about this same subject. I
made a best guess about how to translate this into \LaTeX.}


\def\lxa{\l_{x\in A}}

\medskip
\noindent
(c) {\bf Equations}: All equations have the form $a \eqx a'$, where $a$ and $a'$ have
the same type and $X$ is a set of variables containing all variables occurring
freely in $a$ and $a'$.
\begin{enumerate}[label=(c\theenumi)]
\item $\eqx$ is reflexive, symmetric and transitive. Moreover, if $X \inc Y$ we have
\[
\inferrule{a \eqx b}{a \eqy b}
\]
(that is, from $a \eqx b$ we may infer $a \eqy b$).
\item We have the substitution rules:
\[
\inferrule{c \eqx c'}{\pf(c) \eqx \pf(c)}, \quad 
\inferrule{a \eqx a', \quad b \eqx b'}{\psi(a,b), \eqx \psi(a',b')}
\]
if $\pf(z) \equiv \pi(z)$ or $\pi'(z)$ and $\psi(x,y) \equiv \la x,y \ra$ or $x \app y$ and also
\[
\inferrule{\pf(x) \eqxx \psi(x)}{\lxa\pf(x) \eqx \lxa\psi(x)}
\]
if $x \notin X$.
\item The following identities hold:
\begin{fleqn}
\[a \eqx \star, \hbox{ for all } a \in \one,\]

\[\pi(\la a,b \ra) \eqx a, \textrm{ for all } a \in A, b \in B,\]
\[\pi'(\la a,b \ra) \eqx b, \textrm{ for all } a \in A, b \in B,\]
\[\la \pi(c), \pi'(c) \ra \eqx c, \textrm{ for all } c \in A \times B \]
\[\lxa \pf(x)\app a \eqx \pf(a), \textrm{ for all } a \in A \hbox{ substitutable for } x,\]
\[\lxa (f \app x) \eqx f, \textrm{ for all } f \in \fexpBA \hbox{, provided } x \notin X,\]
\[\lxa \pf(x)\eqx \l_y \varphi(y)\footnotemark\]
\end{fleqn}
\end{enumerate}
\end{defn}
There may be types, terms and equations other than those following from
(a), (b) and (c) above.
\footnotetext{In their 1986 book on this subject Lambek and Scott write this equation like this: ``$\lxa \pf(x)\eqx \l_{x' \in A} \varphi(x')$ if $x'$ is substitutable for $x$ in $\pf(x)$ and x' is not free in $\pf(x)$''. And I think that's what they mean here too.}%
Some comments are in order. The intuitive meaning of the term forming
operations should be clear; for example, $\evBA$ means evaluation,
$\la a,b\ra$ is pairing, and $\lxa \pf(x)$ denotes the function $x \longmapsto \pf(x)$.
$\l$ acts like a quantifier, so the variable $x$ in $\lxa \pf(x)$ is {\em bound}, 
as in $\forall_x \pf(x)$ or $\int_a^b f(x) dx$.
We have the usual conventions for free and bound variables and when
a it is permitted to substitute a term for a variable.

The reader may wonder why we write the subscript $X$ on $\eqx$. The reason
is that there may be ``empty'' types, that is, there may not exist any closed
terms of certain types. This situation arises naturally when a $\l$-calculus
is the internal language of certain categories (see Section 2).

\begin{prop}
If $\pf(x) \eqx \psi(x)$ with $x$ of type $A$ and if $a$ is a
term of type $A$ such that $X$ contains all free variables occurring 
in $a$ (but not $x$), then $\pf(a) \eqx \psi(x)$. In particular if $f$
and $g$ do not contain $x$, $f \eqxx g$ provided there is at least one term of type
$A$ with free variables in $X$.
\end{prop}
It follows from this result that, if there are closed terms of each type
in a typed $\l$-calculus, then the subscript $X$ on $\eqx$ is redundant.

The reader may feel a bit uneasy about the lack of examples so far. Let
us rectify this at once. We recall that a graph consists of two classes, and
two mappings between them:

\tikzcdset{arrow style=tikz, diagrams={>=latex}}
\begin{center}
\begin{tikzcd}[column sep=large]
\text{ arrows  } \arrow[thick, r, yshift=1ex,"\text{ source }"] \arrow[thick, r, yshift=-1ex,"","\text{ target }"',""]
& \text{  objects}
\end{tikzcd}
\end{center}

Graph theorists would call the arrows ``oriented edges'' and the objects
``vertices'' or ``nodes''. We write $f: A \to B$ for ``$\mathop\text{source}(f) = A$'' and
``$\mathop{\text{target}(f)} = B$''.

\begin{example}
Given a graph $\cal G$ the $\l$-calculus $\L(\cal G)$ {\em generated} by $\cal G$
is defined as follows. Its types are generated inductively by the type forming
operations $(-)\times(-)$ and $\fexp{(-)}{(-)}$ from the basic type $\one$
and the vertices of $\cal G$ (which now count as basic types).
Its terms are generated inductively
from the basic terms $x_i^A$ and $\star$ by the term forming operations
$\la -,-\ra$, $\pi(-)$, $\pi'(-)$, $\ev(-,-)$, and $\lxa(-)$,
together with the new term forming operations:
\[
\inferrule{a \in A}{f a \in B}
\]
for each arrow $f: A \to B$ of $\cal G$. Finally, its equations are precisely those
which follow from (c1) to (c3) and no others.

In this example there are plenty of ``empty'' types; for instance, all the
nodes of $\cal G$.
\end{example}
In the above example, as well as in the next section, we allow our
languages to be proper classes in the sense of G\"odel-Bernays. If necessary,
we work in a set theory with universes, in which ``classes'' are replaced by
``sets in a sufficiently large universe''.


\section{The equivalence between $\cart$ and $\lamC$}

In this section we shall establish an equivalence between cartesian closed
categories and typed $\l$-calculi. In order to state this properly, we define
two categories:

$\cart$ is the category whose objects are cartesian closed categories and
whose arrows are those functors that preserve the structure on the nose.

$\lamC$ is the category whose objects are $\l$-calculi and whose arrows are
{translations}. A {\em translation} $\trans: \LC \to \LC'$ does the following:

\begin{enumerate}
\renewcommand\labelenumi{(\theenumi)}
\item $\trans$ sends types of $\LC$ to types of $\LC'$.
\item $\trans$ sends terms of $\LC$, say of type $A$, to terms of $\LC'$ of type
$\trans(A)$, in particular, variables to variables.
\item $\trans$ preserves everything on the nose, for example: $\trans(\one) = \one$, $\trans(A \times B) = \trans(A) \times \trans(B)$, $\trans(\star) = \star$, $\trans(\la a,b \ra) = \la \trans(a), \trans(b) \ra$, and so on.
\item $\trans$ preserves equality: if $a \eqx b$ then $\trans(a) \eqtx \trans(b)$

\end{enumerate}

Our next aim is to obtain a pair of functors

\begin{center}
\begin{tikzcd}[column sep=large,arrow style=tikz, diagrams={>=latex}]
{\cart\text{  }} \arrow[thick, r, yshift=1ex,"L"]  
& \lamC
 \arrow[thick, l, yshift=-1ex,"","C",""]
\end{tikzcd}
\end{center}

In particular, we wish to show that each cartesian closed category $\C$ has an
``internal language'' $L(\C)$ which is a $\l$-calculus. But to describe this
language, we first need to understand ``variables''.

One can adjoin ``variables'' or ``indeterminates'' to cartesian closed
categories much as one does to any universal algebra. Given a cartesian closed
category $\C$ with objects $D$ and $A$, we adjoin an indeterminate arrow
$x: D \to A$ to form the polynomial category $\C[x]$ as follows:

Objects of $\C[x]$ are the same as objects of $\C$. Arrows of $\C[x]$ are
generated from those in $\C$ and the new basic arrow $x: D \to A$ using the arrow
forming operations: composition, $\la -,- \ra$, $(-)^*$. Equality $\eqdX$ in $\C[x]$ 
is the smallest congruence relation on hom-sets of $\C[x]$ which respects equality $\deqd$
$\C$ and assures that $\C[x]$ is a cartesian closed category.

The category $\C[x]$ satisfies the appropriate universal property in $\cart$.
For the committed categorist, $\C[x]$ may also be constructed directly as the
Kleisli category of a certain co-monad (co-triple) on $\C$. For our purposes, the
most interesting result concerns the normal forms of polynomials in $\C[x]$.

\begin{prop}
({\em functional completeness} of cartesian closed categories).
For every polynomial $\pf(x): B \to C$
in $\C[x]$ in an indeterminate arrow
$x: D \to A$ there is a unique arrow $f: \fexp{A}{D} \times B \to C$ in $\C$
such that
\[
f\la (x\pi'_{B,D})^*, \one_B\ra \eqdx \pf(x)
\]
\end{prop}

An interesting alternative interpretation of this result is to think of
$\C[x]$ as a deductive system with the indeterminate $x: D \to A$
considered as a {\em new assumption}. Then functional completeness is
simply a form of the deduction theorem in logic,
with something extra added at the end: to every proof
$\pf(x): B \to C$ under the assumption $x: D \to A$ there is a unique
proof $f: (D \impl A) \sland B \to C$ such that the proof
\def\extraVskip{4pt}
\def\defaultHypSeparation{\hskip .1in}
\begin{prooftree}
\AxiomC{$B \sland D \to D$}
\AxiomC{$D \sxto{x} A$}
\BinaryInfC{$B \sland D \to A$}
\UnaryInfC{$B \to (D \impl A)$}
\AxiomC{$B \to B$}
\BinaryInfC{$B \to (D \impl A) \sland B$}
\AxiomC{$(D \impl A) \sland B \sxto{f} C$}
\BinaryInfC{$B \to C$}
\end{prooftree}
equals $B \sxto{\pf(x)} C$.

\begin{cor}
If we adjoin an indeterminate arrow $x: \one \to A$ to a
cartesian closed category $\C$ and if $\pf(x) : \one \to B$ is a polynomial in $\C[x]$
there exists a unique arrow $f: A \to B$ in $\C$ such that $fx \eqdx\pf(x)$.
Equivalently, under the same hypothesis, there is a unique arrow $g: \one \to \fexpBA$ such
that $g\app x \eqdx \pf(x)$.
\end{cor}

We remark that one may also adjoin several indeterminates simultaneously
to a cartesian closed category, for example, one may adjoin $x :\one \to A$
and $y : \one \to B$ to obtain a cartesian closed category $\C[x,y]$.
It is not difficult to see that $\C[x,y]\iso \C[x][y]$
and also $\C[x,y]\iso \C[z]$ where $z: \one \to A \times B$.

We now introduce another example of a typed $\l$-calculus.

\begin{example}
The {\em internal language} $L(\C)$ of a cartesian closed
category $\C$ is defined as follows.
Its types are the objects of $\C$. Variables of type
$A$ are indeterminate arrows $\one \to A$ over $\C$
and terms of type $B$ with free
variables $x_1, \dots, x_n$ are polynomials $\one \to B$ in
$\C[x_1, \dots, x_n]$. It is not difficult to show that
$L(\C)$ is a typed $\l$-calculus. In particular, for any term $\pf(x)$
of type $B$, that is, an arrow $\pf(x): \one \to B$ in $\C[x]$,
$\lxa \pf(x) \in \fexpBA$ is the unique arrow $g: \one \to \fexpBA$ such that
$g \app x \eqdx \pf(x)$ according to Corollary 2.2.
\end{example}

The object function $L$ described in Example 2.3 may easily be extended
to a functor $L: \cart \to \lamC$. There is also a functor $C: \lamC \to \cart$ in
the opposite direction. Its action on objects will be described in the
following example of a cartesian closed category.

\begin{example}
The cartesian closed category $\C(\LC)$ generated by a
$\l$-calculus $\LC$ is defined as follows.
Its objects are types of $\LC$. The arrows $A \to B$ in $C(\LC)$ are
pairs $(x, \pf(x))$ where where x is a variable of type $A$ and $\pf(x)$ a term
of type $B$ with no free variables but $x$ (Think of this as the
function $x \mapsto \pf(x)$. We agree to identify $(x,\pf(x))$ with $(x', \psi(x'))$
if $\pf(x) \eqlx \psi(x)$ in $\LC$. The identity arrow $A \to A$ is of course $(x,x)$,
and composition of arrows is given by substitution of polynomials. It is easily
seen that $C(\LC)$ is a cartesian closed category. In particular, the rule
\[
\inferrule{C \times A \to B}{C \to \fexpBA}
\]
assigns to the upper arrow $(y,\psi(y))$ of type $C \times A$, and the lower arrow
$(z, \lxa \psi(z,x))$, $z$ of type $C$.
\end{example}

\begin{thm}
The functors $L$ and $C$ yield an equivalence of categories
between $\cart$ and $\lamC$, that is, $LC \iso id$ and $CL \iso id$.
\end{thm}

The proof of this theorem depends on the useful observation that
$C(\LC)[x] \iso C(\LC(x))$, where $\LC(x)$ is the $\l$-calculus
obtained from $\LC$ by adjoining a ``parameter'' $x$ of type $A$.
In other words, closed terms of
$\LC(x)$ are terms of $\LC$ with at most the free variable $x$.

The equivalence between $\cart$ and $\lamC$ may be put to good use. For
example, one easily constructs a functor $\L$ from the category
$\grph \equiv \Set^{\cdot\rightrightarrows}$
of graphs to $\lamC$, whose values on objects $\mathcal G$ of $\grph$
is given in Example 1.4, and $\L$ is left adjoint to the forgetful functor
$\lamC \to \grph$. Hence $C\L$ is left adjoint to the forgetful functor
$\cart \to \grph$ and $C\L(\mathcal G)$ is the {\em free} 
cartesian closed category generated by the graph $\mathcal G$.

Additional equational data may easily be added to a cartesian closed
category. For example, one may introduce finite co-products. More interesting
to us is a {\em weak natural numbers object}, namely a diagram 
$\one \xto{0} N \xto{S} N$ such that for every diagram
$\one \xto{a} A \xto{f} A$ there is an arrow $g \equiv J(a,f) : N \to A$ such that
the following diagram commutes:

%\tikzcdset{arrows={-Computer Modern Rightarrow}}
\begin{center}
\begin{tikzcd}
\one  \arrow[r, "0"] \arrow[thick,dash,d,xshift=-1pt]
 & N \arrow[r, "S"] \arrow[d, dotted, "g"] & N \arrow[d, dotted, "g"] \\
\one \arrow[thick, dash,u,xshift=1pt] \arrow[r, "a"] & A \arrow[r, "f"] & A
\end{tikzcd}
\end{center}

Cartesian closed categories with a weak natural numbers object were introduced
by M.F. Thibault (1977, 1982) under the name ``prerecursive categories''. If
one also insists on the uniqueness of $g$, one obtains a natural numbers object
in the sense of Lawvere; but it is not obvious that the uniqueness of $g$ can
be expressed equationally.

Linguistically, a weak natural numbers object amounts to introducing into
typed $\l$-calculus the following data: a type $N$, a term $0 \in N$ and two term
forming operations (successor and iterator):
\[
\inferrule{n \in N}{Sn \in N} \qquad 
\inferrule{a \in N\\ h \in \fexp{A}{A}\\ n\in N}{I(a, h, n) \in A}
\]
satisfying the ``recursion' equations:
\[
I(a,h,0) \eqx a, \quad I(a,h,Sn) = h \app I(a,h,n) .
\]

The equivalence in Theorem 2.5 extends to one between
$\cart_N$ and $\lamC_N$.
where the subscript $N$ refers to a weak natural numbers object.
Moreover, the above mentioned functor $\L$ also extends to a functor
$\grph \to \lamC_N$. Applying this to the empty graph $\emptyset$
we obtain an initial object $\L(\emptyset)$ in $\lamC_N$, the {\em pure}
typed $\l$-calculus with a weak natural numbers object.
Hence $C\L(\emptyset)$ is an initial object in $\cart_N$.

A question of interest to categorists is that of {\em coherence}: when do
certain diagrams commute? In particular, one wants to decide when two arrows
$A \rightrightarrows B$ in $C\L(\emptyset)$ are equal.
This is equivalent to the decision problem for
equality in $\L(\emptyset)$, which has a positive solution. This solution depends on a
Church-Rosser theorem, dear to combinatory logicians, and a normalizability
theorem following Tait (1967). In the presence of surjective pairing such a
theorem was first obtained by de Vrijer (1982) using methods of Troelstra
(1970). A coherence theorem for cartesian closed categories has also been
obtained by Szabo (1974, 1978).

As a final historical remark, let us point out that the pure $\l$-calculus
$\L(\emptyset)$ corresponds to G\"odel's primitive recursive functionals of finite type
(G\"odel, 1958; Tait, 1967; Troelstra, 1970). G\"odel used such a system to give
an elementary proof of the consistency of arithmetic, his so-called {\em Dialectica
interpretation}. In a sense, the present paper may be considered as a
categorical examination of principles implicit in G\"odel's Dialectica
interpretation (see also [S]).

\section{Untyped $\l$-calculus and C-monoids}

Suppose we have a cartesian closed category with only two non-isomorphic
objects $\one$ and $U$, that is to say, every object is isomorphic to $\one$ or $U$,
and that
\[
\fexp{U}{U} \iso U \iso U \times U
\]
so functions, individuals and pairs are all on the same level. This allows,
for example, functions to apply to themselves as arguments.

Such a category is then completely determined by the monoid $\MC \equiv \Hom(U,U)$.
The equations specifying the cartesian closed structure when specialized to
elements of $\MC$ reduce drastically: ignoring the terminal object, we simply
erase all subscripts on the equations of a cartesian closed category.

\begin{defn}
A {\em $C$-monoid} $\MC$ is a monoid with extra structure
$(\pi, \pi', \ev, (-)^*, \la -,- \ra)$ where $\pi$, $\pi'$, and $\ev$ are
elements of $\MC$ (i.e. nullary operations), $(-)^*$ a unary and $\la -,- \ra$ is
a binary operation satisfying the following identities:
\begin{enumerate}[label=C\theenumi.]
\item $\pi\la a,b\ra = a$
\item $\pi'\la a,b\ra = b$
\item $\la \pi c, \pi' c \ra = c$
\item $\ev \la h^*\pi, \pi' \ra = h$
\item $(\ev \la k\pi, \pi' \ra)^* = k$
\end{enumerate}
\end{defn}

$\C$-monoids are the objects of a category, whose arrows are monoid homo-
morphisms preserving the additional structure. We may apply the usual
techniques of universal algebra to the variety of $\C$-monoids. For example, the
{\em polynomial} $\C$-monoid $\MC[x]$ has as elements polynomials, that is, words in $x$
modulo the smallest congruence relation satisfying Cl to C5. Evidently,
$\MC[x]$ has the appropriate universal property.

$\C$-monoids, like cartesian closed categories, satisfy a version of
{\em functional completeness}.

\begin{thm}
If $\pf(x)$ is a polynomial in the indeterminate $x$ over a
$\C$-monoid $\MC$, there exists a unique constant $f\in \MC$ such that
\[
f\la (x\pi')^*, 1 \ra = \pf(x)
\]
in $\MC[x]$.
\end{thm}
We may carry over various definitions from cartesian closed categories
to $\C$-monoids. For example, we write
\[
g\app a \equiv \ev \la g(a \pi')^*,1\ra,
\]
\[
\lx \pf(x) \equiv f^*,
\]
where $f$ is the constant from Theorem 3.2.%
\footnote{In the paper this is mis-labeled as 4.2}
It follows that
\[
\lx \pf(x) \app a = \pf(a).
\]
An amusing feature of the possibility of self-application is the
following fixed point theorem for $\C$-monoids, which is also behind Russell's
paradox.

\begin{prop}
For every polynomial $\pf(x)$ in $\MC[x]$ there exists an
element $a \in \MC$ such that $\pf(a) = a$.
\end{prop}

\begin{proof}
Put $b = \lx \pf(x \app x)$ and let $a \equiv b \app b$.
\end{proof}

To make the connection between $\C$-monoids and cartesian closed categories
precise, we recall the following.

\begin{defn}
The {\em Karoubi envelope} (or {\em idempotent splitting envelope})
$K(\mathcal A)$ of a category $\mathcal A$ is a category whose objects
are the idempotent arrows of $\mathcal A$ and whose arrows $f \to g$, where
$f^2 = f: A \to A$ and $g^2 = g: B \to B$, are triplets
$(f, \pf, g)$ such that $g\pf f = \pf$.
\end{defn}

The following is due to Dana Scott (1980).

\begin{thm}
Let $\MC$ be a $\C$-monoid.
\begin{enumerate}[label=(\roman*)]
\item $K(\MC)$ is a cartesian closed category  with terminal object
$T \equiv (\pi')^*$.
\item If $\mathcal A$ is any cartesian closed category with an object $U$
such that $\fexp{U}{U} \iso U \iso U \times U$, then $\text{End}_{\mathcal A}(U)$
is a $\C$-monoid. In particular if ${\mathcal A} = K(\MC)$
and $U$ is the object of $\MC$
regarded as a one-object category, then $\text{End}_{\mathcal A} \iso \MC$.
\end{enumerate}
\end{thm}

We remark that, if $K_0(\MC)$ is the full subcategory of $K(\MC)$ consisting
of all objects isomorphic to $T$ or $U$, then $K_0(\MC)$ is a cartesian closed
category with (at most) two non-isomorphic objects.

Examples of $\C$-monoids or, equivalently, of cartesian closed categories
with an object $U$ such that $\fexp{U}{U} \iso U \iso U \times U$
have been constructed by Dana Scott (1972), using the category of continuous
lattices, to mention only one such example.

$\C$-monoids are also closely related to an extended version of the usual
untyped $\l$-calculus.

\begin{defn}
An (untyped) $\l$-calculus is a formal language consisting
of terms and equations as follows. Among the terms there are countably many
variables. Moreover, if $a$, $b$ and $c$ are terms, then so are
$\pi(c)$, $\pi'(c)$, $(a,b)$, and $b \app a$.
Finally, if $\pf(x)$ is a term, possibly with a free occurrence
of the variable $x$, then $\lx\pf(x)$ is also a term (in which all occurrences of
$x$ are bound). Equality is an equivalence relation between terms which
satisfies the usual rules allowing substitution of equals for equals, including
the rule
\[
\inferrule{\pf(x) = \psi(x)}{\lx \pf(x) = \lx \psi(x)},
\]
which furthermore satisfies the substitution rule
\[
\inferrule{\pf(x) = \psi(x)}{ \pf(a) = \psi(a)},
\]
where it is assumed that a is substitutable for x, and which finally
satisfies the following equations:
\begin{enumerate}[label=L\theenumi.]
\item $\lx \pf(x)\app x = \pf(x)$.
\item $\lx (f \app x) = f$, if $x$ is not free in $f$.
\item $\pi((a,b)) = a$.
\item $\pi'((a,b)) = b$.
\item $(\pi(c), \pi' (c)) = c$.
\end{enumerate}
\end{defn}

The usual untyped $\l$-calculus omits the term forming operations $\pi(-)$,
$\pi'(-)$, and $(-,-)$, together with the equations L3, L4 and L5. Unfortunately,
the presence of L5 (``surjective pairing'') adds an unexpected complication:
the Church-Rosser theorem then fails (Barendregt, 1981, Exercise 15.4.4).

The (untyped) $\l$-calculi are the objects of a category whose arrows are
translations. A {\em translation} is a mapping from terms to terms which sends
variables to variables and preserves the term forming operations.

\begin{thm}
The category of $\C$-monoids is isomorphic to the category of
(untyped) $\l$-calculi.
\end{thm}

\begin{proof}
(Sketch). Starting with a $\C$-monoid $\MC$, we form the $\l$-calculus
$L(\MC)$ whose terms in the variables $x_1, ... , x_n$ are elements of $\MC[x_1,...,x_n]$
with term forming operations defined as follows:
\begin{align*}
\pi(c) & \equiv \pi \app c,\\
\pi' (c) & \equiv \pi' \app c,\\
(a,b) & \equiv \la \lx a, \lx b\ra \app 1,\\
f \app a & \equiv \ev \la f(a \pi')^*,1 \ra,\\
\lx \pf(x) &\equiv f^* \,\, \text{by functional completeness.%
\footnotemark}
\end{align*}
\footnotetext{This is a guess based on the discussion after Theorem 3.2}

Conversely, starting with a $\l$-calculus $\LC$, we form the $\C$-monoid $M(\LC)$
whose elements are the closed terms of $\LC$ with operations defined as follows:
\begin{align*}
1 &\equiv \lx x,\\
gf &\equiv \lx (g\app (f\app x)),\\
\pi & \equiv \lx \pi(x),\\
\pi' & \equiv \lx \pi'(x),\\
\la f,g \ra &\equiv \lx (f\app x, g \app x),\\
\ev & \equiv \l_z (\pi(z) \app \pi'(z)),\\
h^* & \equiv \lx\ly(h \app (x,y)).
\end{align*}
It is easily checked that
\[
ML(\MC) = \MC, \,\, LM(\LC) = \LC.
\]
It can also be shown that $M$ and $L$ extend to functors inverse to one
another.
\end{proof}

\section{Toposes and \intui\ type theory}

One of the most important concepts in category theory is that of a topos.
For our purposes, it will be convenient to insist that a topos contains a
natural numbers object in the sense of Lawvere.

\begin{defn}
A {\em topos} is a cartesian closed category with a subobject
classifier and a natural numbers object.

A subobject classifier is an object $\OM$ together with a natural
isomorphism $\Sub \cong \Hom(-, \OM)$ where $\Sub$ is the 
contravariant functor which
assigns to each object $A$ the set of its subobjects. There are several ways
of making this more precise; we prefer the following: there is given an arrow
$\top: 1 \to \OM$ such that
\begin{enumerate}
\item[(i)] For every arrow $h: A \to \OM$ an equalizer of $h$ and $\top 0_A: A \to 1 \to \OM$
exists, denoted by $\ker h: \Ker h \to A$ (the {\em kernel} of $h$),%
\footnote{This definition is very mysterious. First the arrow mentioned is not unique,
so it should be {\em a} kernel, not {\em the} kernel. Second, we never see a definition
of what $\Ker h$ is, although one guesses that it's the collection of things that $h$ maps
to some initial object in the category
In their book Lambek and Scott use such
a definition in a later proof related to a specific topos. 
But it's never made more clear in this definition.}
\item[(ii)] For every monomorphism $m: B \to A$ there is a unique characteristic
morphism $\cha m: A \to \OM$ such that
\[
\cha(\ker h) \deqd h, \quad \ker(\cha m) \cong m.
\]
\end{enumerate}
\end{defn}
Examples of toposes are the category $\Set$ (with $\OM \equiv \{\bot, \top\}$),
functor categories $\fexp{\Set}{\C}$ (including $\MC$-sets when $\C$ is 
a monoid $\MC$, the category of sheaves on a topological space and, 
more generally, any Grothendieck topos (Johnstone, 1977).

One simplification in Definition 4.1, first observed by Mikkelsen, is of
importance to us. Instead of requiring arbitrary exponents $\fexpBA$  one need only
require $\PA \cong \fexp{\OM}{A}$ and its associated structure.
More precisely, we postulate a power ~ structure $(P,\ev,\star)$: 
for each object $A$ there is an object $\PA$, an arrow $\ev_A : \PA \times A\to \OM$
and an arrow forming operation
\[
\inferrule{h: B \times A \to \OM}{h^*: B \to \PA}
\]
such that (subscripts being omitted)
\[
\ev\la h^* \pi,\pi'\ra \deqd h, (\ev\la k \pi,\pi'\ra)^* \deqd k 
\]
for all $h: B\times A \to \OM$ and $k: B \to \PA$.

While in the category $\Set$ the subsets of a given set form a Boolean
algebra, in an arbitrary topos the subobjects of a given object $A$ form a
Heyting algebra; hence so does the set $\Hom(A,\OM)$. This permits an 
interpretation of intuitionistic propositional logic in any topos, which can be
extended to higher order intuitionistic logic, which may be regarded as a
variant of set theory.

\begin{defn}
An {\em intuitionistic type theory} is a formal theory consisting of
types, terms and an entailment relation as follows.
\begin{enumerate}
\item[(a)] {\bf Types}. $1$, $N$ and $0$ are types and, if $A$ and $B$ are types, so
are $A \times B$ and $\PA$.
\item[(b)] {\bf Terms}. There are countably many variables of each type; other basic
terms and term forming operations are specified by the following chart, in
which terms are listed under their types:
\begin{center}
\begin{tabular}{ c c c c c }
$1$ & $N$ & $\OM$ & $A \times B$ & $\PA$\\
\hline
$\star$ & $0$ & $a = a'$ & $\la a,b \ra$& $\{x \in A \mid \pf (x)\}$\\
&$S n$&$a \in \alpha$\\
\end{tabular}
\end{center}
\medskip
Here $n$ is a term of type $N$, $a$ and $a'$ are terms of type $A$, $b$ is a term
of type $B$, $\alpha$ is a term of type $\PA$ and $\pf(x)$ is a term of type $\OM$
possibly containing a free variable $x$ of type $A$.

Terms of type $\OM$ are usually called {\em formulas}. The usual logical symbols%
\footnote{Here they mean $\top$, $\sland$, $\impl$ and $\forall$. This is clarified in the book
so I brought it over.}
are introduced by definitions:
\begin{align*}
\top &\equiv \star = \star,\\
p \sland q &\equiv \la p,q \ra = \la \top, \top\ra,\\
p \impl q &\equiv p \sland q = p,\\
\forall_{x \in A} \pf(x) &\equiv \{ x \in A \mid \pf(x)\} = \{x \in A \mid \top\}\\
\intertext{The symbols $\bot$, $\vee$ and $\exists$ may be defined in terms of the above (Prawitz,
1965); also}%
\neg p &\equiv p \impl \bot .
\end{align*}
\item[(b)] {\bf Entailment}. We introduce an entailment relation $p_1, \dots, p_n \ent_X q$
between formulas whose free variables are all contained in $X = \{x_1, \dots, x_n\}$.
This must satisfy the obvious structural rules, sufficiently many logical
rules to yield the usual intuitionistic predicate calculus, together with
comprehension, extensionality, products and Peano's axioms (see [LS1,LS2]).

The reason for the subscript $X$ on $\ent_X$, as in typed $\l$-calculus, is the
possibility of ``empty'' types. For example, we may infer:
\[
\inferrule
	{\forall_{x \in A} \pf(x) \ent_X \pf(x) \\ \pf(x) \ent_X \exists_{x\in A}\pf(x)}
	{\forall_{x \in A} \pf(x)  \ent_X \exists_{x\in A}\pf(x)}
\]
The subscript in the conclusion may only be erased if there is a closed term
of type $A$. Otherwise we would be able to infer from ``all unicorns are horned''
that ``some unicorns are horned''.
\end{enumerate}
\end{defn}
\noindent 
There may be types, terms and entailments other than those following from
(a), (b) and (c) above.

We shall discuss two examples of type theories in this section.

\begin{example}
{\em Pure type theory} is the type theory in which there are no
types, terms or entailments other than those that follow inductively from
Definition 4.2. Its philosophical interest derives from the claim that it
represents much of the reasoning acceptable to a moderate intuitionist or
constructivist.
\end{example}

\begin{example}
The internal language $L(\TC)$ of a topos $\TC$ is defined as
follows. Its types are the objects of $\TC$; in particular the
type $\OM$ is the object $\OM$, etc. Terms of type $A$ with free variables
$x_1, \dots, x_n$ are arrows $1 \to A$ in $\TC[x_1, \dots, x_n]$,
the cartesian closed category obtained by adjoining
indeterminates to $\TC$ (warning: $\TC[x]$ is not in general a topos, just as
$F[x]$ is not in general a field, even if $F$ is). In particular, we have the
following dictionary:
\begin{align*}
\text{variables of type $A$} & \equiv  \text{indeterminates } 1 \to A\\
\star &\equiv 1 \to 1\\
0 &\equiv 1 \xto{0} N\\
S_n &\equiv 1 \xto{n} N \xto{S} N\\
\la a,b \ra &\equiv 1 \xto{\la a,b \ra} A \times B\\
a = a' &\equiv  1 \xto{\la a,a' \ra} A \times A \xto{\delta_A} \OM\\
& \qquad \text{ where } \delta_A = \cha\la 1_A,1_A\ra\\
a\in \alpha &\equiv 1 \xto{\la \alpha,a\ra} \PA \times A \xto{\ev_A} \OM\\
\{x \in A \mid \pf(x)\} &\equiv \lxa \pf(x)
\end{align*}
which, by functional completeness of cartesian closed categories, is the
unique arrow $f: 1\to \PA$ in $\TC[x]$ such that $x \in f \eqxx \pf(x)$
in $\TC[X,x]$. Here $X$ comprises all variables other than $x$ which happen to be free in
$\pf(x)$.

Finally, entailment $p_1,\dots ,p_m \ent_X q$ in $L(\TC)$ is defined as follows. To
simplify matters, take $m=1$ and $X = \{x\}$, $x$ a variable of type $A$. Then
$\pf(x) \ent_X \psi(x)$ means:
\begin{quote}
For all objects $C$ and all arrows $a: C\to A$, if $fa \deqd \top 0_C$
then $ga \deqd \top 0_C$.
\end{quote}
Here we have used functional completeness to write $\pf(x) \eqdx fx$ and
$\psi(x) \eqdx gx$ with $f,g: A \to \OM$.
\end{example}

The description of $L(\TC)$ is now complete.
The definition of $\ent_X$ in $L(\TC)$ is the crux of the
so-called Beth-Kripke-Joyal semantics of a topos, which
views arrows $a: C \to A$ as ``elements of $A$ at stage $C$.''
We shall not pursue this theme
any further here. Instead, we point out that the internal
language of a topos allows us to work in $\TC$ as we would in $\Set$, provided
we use only intuitionistic logic. The following dictionary shows how a few
external statements about $\TC$ are translated into the internal language.
If $p$ is a closed formula in $L(\TC)$, that is, an arrow $p: 1 \to \OM$
in $\TC$, it is customary to write $p \deqd \top$ in $\TC$ as
$\TC \entt p$ and read this as ``$\TC$ satisfies $p$'' or
``$p$ holds in $\TC$.''

\begin{center}
\begin{tabular}{ l | l }
\multicolumn{1}{c}{In $\TC$} & \multicolumn{1}{c}{In $L(\TC)$} \\
 \hline
 $f \deqd g : A \rightrightarrows B$ & $\TC \entt \forall_{x\in A}\, fx = gx$ \\
 $f: A \to B$ is mono & $\TC \entt \forall_{x\in A}\, \forall_{x'\in A}\, fx = fx' \impl x = x'$\\
 $f: A \to B$ is epi & $\TC \entt \forall_{y\in B}\, \exists_{x\in A}\, fx = y$\\
 $P$ is projective & \begin{minipage}[t]{50mm}
for all formulas $\pf(z,x)$, if $\TC \entt \forall_{z\in P}\,\exists_{x \in A}\, \pf(z,x)$
then $\TC \entt \forall_{z\in P}\, \pf(z,fz)$
for some $f : P \to A$.
\end{minipage}\\[3.5em]
$\OM = 1 + 1$ & $\TC \entt \forall_{t\in \OM} (t \vee \neg t)$\\
 \hline
\end{tabular}
\end{center}
\medskip
Some historical remarks are in order. The internal logic of a topos is
implicit in Lawvere's definition of the logical connectives in any topos
(1972). As a formal language it first appears explicitly in the article by
Mitchell (1972), having also served as a working language in the Paris seminar
of B\'enabou (see. Coste, 1972-4). It was probably discovered independently by
many people; the first detailed exposition appears in a paper by Osius (1975),
see also Fourman (1974) and Boileau (1975).

Finally, topos semantics includes as special cases Beth and Kripke models,
forcing in model theory and set theory and some versions of realizability.

\section{Adjointness between toposes and languages}

Toposes are the objects of a category $\Top$, whose arrows are
{\em strict logical functors}, that is, functors which preserve the topos
structure on the nose (with exponentiation replaced by power set structure). Type theories are
the objects of a category $\Lang$, whose arrows are translations, that is,
mappings from types to types and from terms to terms sending variables to
variables, and which preserve type assignments, type forming operations, term
forming operations and entailments.

The construction of the internal language in Section 4 induces a functor
$L: \Top \to \Lang$. We shall try to construct a functor
$T$ in the opposite directions which is left adjoint to $L$.

To any type theory $\LC$ we associate the topos $T(\LC)$ generated by $\LC$ as
follows. Its objects are (names of) {\em sets} in $\LC$, that is, closed terms of type
$\PA$ in $\LC$ for some type $A$. If $\alpha$ of type $\PA$ and $\beta$ of type $\PB$ are two
objects of $T(\LC)$, an arrow $\alpha \to \beta$ is a
triplet $(\alpha,\rho, \beta)$ where $\rho$
is (the name of) a {\em provably functional relation},
that is, a closed term of type
$P(A \times B)$ such that in $\LC$ we can prove:
\begin{enumerate}
\item[(1)] $\ent {\!\rho} \subseteq \alpha \times \beta$
\item[(2)] $\ent\! \forall_{x \in A} (x \in \alpha \impl \exists!_{y \in B} \la x,y \ra \in \rho)$
\end{enumerate}
So far we have only got a category $T(\LC)$, but it is easily endowed with the
structure of a topos by imitating familiar constructions in the category of
sets. Furthermore, one can show that $T$ is a functor.

The following result supports the philosophical position called
nominalism: every topos is equivalent to one constructed linguistically.

\begin{prop}
Every topos $\TC$ is equivalent to $TL(\TC)$.
\end{prop}

While $T$ looks very much like a left adjoint to $L$, this is not strictly
speaking correct and cannot be asserted without handwaving. To make the
adjointness precise we need another idea.

Linguistic toposes $T(\LC)$ have an important property: they have {\em canonical
subobjects}. What this means is that every monomorphism is isomorphic to a
unique canonical one, just as in $\Set$ every monomorphism is isomorphic to
the inclusion mapping of its image. Indeed, a reasonable axiomatic
description of ``canonical subobjects'' leads to the conclusion that all the
toposes that have been mentioned as examples so far have them. In any case,
Proposition 5.1 assures that every topos is equivalent to one with canonical
subobjects.

Let $\Top_0$ be the subcategory of $\Top$ whose objects are toposes with
canonical subobjects and whose arrows are strict logical functors preserving
canonical subobjects. $\Top_0$ is a reflective (but not full) subcategory of
$\Top$.

\begin{thm}
$T: \Lang \to \Top_0$ is left adjoint to $L: \Top_0 \to \Lang$.
\end{thm}

This result, with some handwaving, is due to Volger (1975) and, with
canonical subobjects, to one of the authors [L4]. In both these articles type
theories were disguised as cartesian categories with additional equational
structure.

How close are we to having an equivalence between $\Lang$ and $\Top_0$? Even
if we ignore the technical point that Proposition 5.1 yields an equivalence
rather than an isomorphism, the best we can say about the adjunction
$\LC \to LT(\LC)$ for an arbitrary type theory $\LC$ is that it is a conservative
extension. One might assure that it is an isomorphism either by making the
definition of type theory more restrictive or by relaxing the notion of
translation. However, it appears that the adjointness asserted by Theorem 5.2
suffices for many useful applications, some of which we shall now discuss.

\section*{Application 5.3.}

If $\LC_0$ is pure type theory (see Example 4.3), it
follows from Theorem 5.2 that $\FC \equiv T(\LC_0)$ is an initial object
in $\Top_0$ , which has come to be known as the {\em free topos}· 
One could argue that $\FC$ is the universe of mathematics for a moderate
intuitionist or constructivist. The terminal object of $\FC$ has the important 
algebraic properties of being indecomposable and projective. As first observed 
by Peter Freyd (1978), these translate into the following meta-theorems
about pure type theory, which are basic to the philosophy of intuitionism:
\begin{enumerate}
\item[(i)] if $\ent p \vee q$ then either $\ent p$ or $\ent q$
\item[(ii)] if $\ent \exists_{x\in A} \pf(x)$ then $\ent \pf(x)$ 
for some closed term $a$ of type $A$.
\end{enumerate}
These results may be proved either algebraically, as Freyd would have it,
or linguistically [LS1,LS5].

\section*{Application 5.4.}

Composing adjoint functors as follows:
\[
\Top_0 \leftrightarrows \Lang \rightleftarrows \grph \equiv \Set^{\cdot\rightrightarrows}
\]

we see that the forgetful functor $\Top_0 \to \grph$ has a left adjoint. Much more
is true, $\Top_0$ is tripleable (monadic) over $\grph$ (or {\bf Cat}), even algebraic
in a sense recently propounded by Burroni (1981, see also [L5]).

\section*{Application 5.5.}

We saw earlier how to adjoin an indeterminate arrow
$x: 1 \to A$ to a topos to obtain a cartesian closed category $\TC[x]$. What if we
want to obtain a topos $\TC(x)$? This can be done in several ways, but we favour
\[
\TC(x) = T(L(\TC) (x)).
\]
We recall that a parameter $x$ may be adjoined to any language $\LC$ (see the
discussion of $\LC(x)$ following Theorem 2.5 above), in particular, to $L(\TC)$,
We note that $\TC(x)$ is equivalent (but not isomorphic) to $\TC/A$, which had been
proposed as a candidate for $\TC(x)$ by Joyal.

Of special interest is $\FC(x)$ when $\FC$ is the free topos and
$: 1 \to A = \PB$ for some object $B$ of $\FC$. In $\FC(x)$ too the terminal object
is projective, which fact is equivalent to the projectivity of $A$ in $\FC$ and
which gives rise in $\LC_0$ to Troelstra's uniformity principle (1970):
\[
\text{if } \ent\forall_{x \in \PB}\exists_{y \in N} \pf(x,n) \text{ then } 
\exists_{y \in N}\forall_{x \in \PB} \pf(x,y) .
\]
Similarly, for certain arrows $\underline{p} : 1 \to \OM$ in $\FC$ coming
from closed formulas $p$ of $\LC_0$, $\Ker \underline{p}$ is projective in $\FC$,
and this translates into what logicians call ``independence of premises''.

\section*{Application 5.6.}

If $F$ is a filter of closed formulas in a type theory
$\LC$, one may form a new type theory $\LC/F$ in which the formulas of $F$ are
assumed or postulated. In particular, if $F$ is a filter of arrows $1 \to \OM$
in a topos $\TC$ we may define
\[
\TC\!/F = T(L(\TC)/F) .
\]
Dividing a topos by a filter is very much like dividing a ring by an ideal.

Since the intersection of prime filters $P$	is $\{\top\}$, it follows that
every topos is a subdirect product of toposes of the form $\TC/P$. This is the
crux of Henkin's completeness theorem for higher order logic with choice (see
below). Actually, much more is true: the prime filters of the Heyting
algebra $\Hom(1, \OM)$ are the points of a topological 
space $\mathop{\Spec}\TC$, the toposes $\TC/P$ are the stalks of a 
sheaf on $\mathop{\Spec}\TC$ and the topos $\TC$ may be recaptured
from the global sections of this sheaf [LM]. Not surprisingly, a crucial step
in the proof of this result is the following theorem of intuitionistic type
theory called {\em definition by cases}: if $p_1, \dots, p_n$ are closed formulas
and $a_1, \dots, a_n$ are closed terms of type $A$, then
\[
\bigvee_{i=1}^n p_i,\, \bigwedge_{i,j=1}^n ((p_i \sland p_j) \impl a_i = a_j) \ent
\exists!_{x \in A} \bigwedge_{i=1}^n (p_i \impl x = a_i).
\]

\section*{Application 5.7.} 

By an {\em interpretation} of a type theory $\LC$ we mean a
translation $\tau:\LC \to L(\TC)$ where $\TC$ is a topos, which we may as well assume to
have canonical subobjects. By Theorem 5.2, this corresponds to a morphism
$T(\LC) \to \TC$ in $\Top_0$. The same theorem tells us that the category of inter-
pretations of $\LC$ has an initial object $\LC \to LT(\LC)$, the adjunction pertaining
to the pair of adjoint functors $(T,L)$. We shall now concentrate on toposes
which resemble the topos of sets.

We call an interpretation $\tau:\LC \to L(\TC)$ a {\em model} of $\LC$
if the terminal object of $\LC$ is an indecomposable projective. The completeness
theorem for higher order logic asserts that every type theory has enough models. This
means, for every closed formula $p$ of $\LC$, if $\TC \ent \tau(p)$ for all
models $\tau:\LC \to L(\TC)$, then $\ent p$ in $\LC$. For classical type theory 
this is due to Henkin (1950), while for intuitionistic theories we learned 
it from Aczel (1969).
If $\LC$ satisfies the rule of choice, the completeness theorem is nothing else
than the subdirect product representation of $T(\LC)$ mentioned in
Application 5.6 and one finds that the models are Henkin's nonstandard models.

This ends our discussion of higher order categorical logic. We regret
that space does not permit introducing further topics, e.g., recursive
functions. We feel that many connections between category theory and logic
still wait to be discovered.

\section*{References} 

\pdfbookmark{References}{bib}
\def\bb#1{\bibitem[#1]{#1}}

\begin{enumerate}[leftmargin=*, widest=8888, align=left]

\item[] P. H. Aczel, Saturated intuitionistic theories, in: H. A. Schmidt, K. Schutte,
H.-J. Thiele, {\em Contributions to mathematical logic}, (North-Holland,
Amsterdam, 1969), pp. 1-11.

\item[] H. P. Barendregt, {\em The lambda calculus, its syntax and semantics}, Studies in
Logic, 103 (North-Holland, Amsterdam, 1981).

\item[] A. Boileau, {\em Types vs. topos}. Thesis (Universit\'e de Montr\'eal, 1975).

\item[] A. Boileau et A. Joyal, {\em La logique des topos}, J. Symbolic Logic 46 (1981),
6-16.

\item[] A. Burroni, {\em Alg\`ebres graphiques}, 3\`eme colloque sur les cat\'egories, Cahiers de
Topologie et G\'eom\'etrie Differentielle 23 (1981), 249-265.

\item[] A. Church,{\em A formulation of the simple theory of types}, J. Symbolic Logic 5
(1940)' 56-68.

\item[] M. Coste, {\em Langage interne d'un topes}, S\'eminaire de J. B\'enabou 1972-73.

\item[] M. Coste, {\em Logique d'ordre superieur dans les topos elementaires}. S\'eminaire
de J. B\'enabou 1974.

\item[]M. P. Fourman, {\em Connections between category theory and logic}, Thesis
(Oxford University, 1974).

\item[]M. P. Fourman, {\em The logic of topoi}, in: J. Barwise, Handbook of Mathematical
Logic (North-Holland, Amsterdam, 1977).

\item[] P. Freyd, {\em On proving that 1 is an indecomposable projective in various free
categories}, Manuscript 1978.

\item[] K. G\"odel, {\em \"Uber formal unentscheidbare Satze der Principia Mathematica und
verwandter Systeme I}, Monatshefte f\"ur Math. und Physik 38 (1931, 173-198,
English translation in: J. van Heijenoort, {\em From Frege to G\"odel}
(Harvard University Press, Cambridge, MA, 1967).

\item[] K. G\"odel, {\em \"Uber eine bisher noch nicht benutzte Erweiterung des finiten
Standpunktes}, Dialectica 12 (1958)

\item[] L. Henkin, {\em Completeness in the theory of types}, J. Symbolic Logic 15 (1950),
81-91, reprinted in: J. Hintikka, The Philosophy of Mathematics
(Oxford University Press, Oxford, 1969).

\item[] J. R. Hindley, B. Lercher and J. P. Seldin, {\em Combinatory Logic} (Cambridge
University Press, Cambridge, 1972).

\item[] P. T. Johnstone, {\em Topos theory} (Academic Press, London, 1977).

\item[] A. Kock and G. E. Reyes, {\em Doctrines in categorical logic}, in: J. Barwise,
{\em Handbook of Mathematical Logic} (North-Holland, Amsterdam, 1977).

\item[{[L1]}] J. Lambek, {\em Deductive systems and categories, I, II, III}, J. Math.
Systems Theory 2 (1968), 287-318; Springer Lecture Notes in
Mathematics 86 (1969), 76-122; Springer Lecture Notes in
Mathematics 274 (1972), 57-82.

\item[{[L2]}] J. Lambek, {\em Functional completeness of cartesian categories}, Ann. Math.
Logic 6 (1974), 259-292.

\item[{[L3]}] J. Lambek, {\em From $\l$-calculus to Cartesian closed categories}, in: J. P.
Seldin and J. R. Hindley, {\em To H. B. Curry, Essays in combinatory
logic, lambda calculus and formalism} (Academic Press, London,
1980), pp. 375-402.

\item[{[L4]}] J. Lambek, {\em From types to sets}, Advances in Mathematics 36 (1980), 113-163.

\item[{[L5]}] J. Lambek, {\em Toposes are monadic over categories}, Springer Lecture Notes in
~thematics 962 (1982), 153-166.

\item[{[LM]}] J. Lambek and I. Moerdijk, {\em Two sheaf representations of elementary
toposes}, in: A. S. Troelstra and D. van Dalen, The L.E.J.
Brouwer Centenary Symposium (North-Holland, Amsterdam, 1982),
pp. 275-295.

\item[{[LS1]}] J. Lambek and P. J. Scott, {\em Intuitionist type theory and the free topos},
J. Pure and Applied Algebra 19 (1980), 576-619.

\item[{[LS2]}] J. Lambek and P. J. Scott, {\em Intuitionist type theory and foundations}, J. Philosophical Logic
--7-(1981)' 101-115.

\item[{[LS3]}] J. Lambek and P. J. Scott, {\em Algebraic aspects of topos theory}, Cahiers de Topologie et
G\'eom\'etrie Diff\'erentielle 22 (1981), 129-140.

\item[{[LS4]}] J. Lambek and P. J. Scott, {\em Independence of premisses and the free topos}, Proc. Symp.
Constructive Mathematics, Springer Lecture Notes in Math. 873
(1981)' 191-207.

\item[{[LS5]}] J. Lambek and P. J. Scott, {\em New proofs of some intuitionistic principles}, Zeitschrift
~r mathematische Logik und Grundlagen der Math. 29 (1983), 493~504.

\item[{[LS6]}] J. Lambek and P. J. Scott, {\em Introduction to higher order categorical logic}, Cambridge
University Press.

\item[] F. W. Lawvere, {\em An elementary theory of the category of sets}, Proc. Nat.
Acad. Sci, USA 52 (1964), 1506-1511.

\item[] F. W. Lawvere, {\em Adjointness in foundations}, Dialectica 23 (1969), 281-296.

\item[] F. W. Lawvere, {\em Quantifiers and sheaves}, in: Actes du Congr\`es International des
Mathematici\'ens, Nice 1970, vol. 1 (Gauthier-Villars, Paris, 1971),
pp. 329-334.

\item[] F. W. Lawvere, {\em Equality in hyperdoctrines and comprehension schema as an adjoint
functor}, Proc. Amer. Math. Soc, Applications of categorical algebra
(1970)' 1-14.

\item[] F. W. Lawvere, {\em Introduction}, in: Toposes, Algebraic Geometry and Logic,
Springer Lecture Notes in Math. 274 (1972), 1-12.

\item[] S. MacLane, {\em Categories for the working mathematician}, (Springer, New York,
1971).

\item[] M. Makkai and G. E. Reyes, {\em First order categorical logic}, Springer Lecture
Notes in Math. 611 (1977).

\item[] C. R. Mann, {\em The connection between equivalence of proofs and cartesian closed
categories}, Proc. London Math. Soc, 31 (1975), 289-310.

\item[] P. Martin-L\"of, {\em An intuitionistic theory of types: predicative part}, in:
H. E. Rose and J. C. Shepherdson, Logic Colloquium 1 73 (North-Holland,
Amsterdam, 1974), pp. 73-118.

\item[] W. Mitchell, {\em Boolean topoi and the theory of sets}, J. Pure and Applied Algebra
2 (1972), 261-274.

\item[] G. Osius, {\em Logical and set theoretical tools in elementary topoi}, Springer
Lecture Notes in Math. 445 (1975), 297-346.

\item[] D. Prawitz, {\em Natural deduction} (Almqvist and Wiskell, Stockholm, 1965).

\item[] D. Prawitz, {\em Ideas and results in proof theory}, in: J. E. Fenstad, Proc, Second
Scandinavian Logic Symposium (North-Holland, Amsterdam, 1971),
pp. 235-307.

\item[] A. Robinson, {\em Non-standard analysis} (North-Holland, Amsterdam, 1966).

\item[] B. Russell and A. N. Whitehead, {\em Principia Mathematica I-III} (Cambridge
University Press, Cambridge, 1910-1913).

\item[] D. S. Scott, {\em Continuous lattices}, Springer Lecture Notes in Math. 274 (1972),
97-136.

\item[] D. S. Scott, {\em Data types as lattices}, SIAM J. Computing 5 (1976), 522-587.

\item[] D. S. Scott, {\em Relating theories of the $\l$-calculus}, in: J. P. Seldin and J. R.
Hindley, {\em To H. B. Curry, Eassays in combinatory logic, lambda calculus
and foundations} (Academic Press, London, 1980), pp. 403-450.

\item[{[S]}] P. J. Scott, {\em The ``Dialectica'' intepretation and categories}, Zeitschr.
f. math. Logik and Grundlagen d. Math. 24 (1978), 553-575.

\item[] R. A. G. Seely. {\em Locally cartesian closed categories and type theory}. Mathematical
Proceedings of the Cambridge Philosophical Society, 95(1):33, 1984.

\item[] M. E. Szabo, {\em A categorical equivalence of proofs}, Notre Dame Journal of
Formal Logic 15 (1974), 177-191.

\item[] M. E. Szabo, {\em Algebra of proofs}, Studies in Logic and the foundations of
Mathematica 88 (North-Holland, Amsterdam, 1978).

\item[] W. W. Tait, {\em Intentional interpretation of functionals of finite type I, J}.
Symbolic Logic 32 (1967), 198-212.

\item[] M.F. Thibault, {\em Repr\'esentations des fonctions r\'ecursives dans les cat\'egories},
Thesis (McGill University, Montreal, 1977).

\item[] M.F. Thibault, {\em Prerecursive categories}, J. Pure and Applied Algebra 24 (1982), 79-93.

\item[] A. S. Troelstra, {\em Metamathematical investigation of intuitionistic arithmetic
and analysis}, Springer Lecture Notes in Math. 344 (1970).

\item[] H. Volger, {\em Logical and semantical categories and topoi}, Springer Lecture
Notes in Math. 445 (1975), 87-100.

\item[] R. C. de Vrijer, {\em Strong normalization in $N - \HA_{p}^{\omega}$}, Manuscript 1982.
\end{enumerate}


