%!TEX root = lambek-1984-pal.tex
%\usepackage[dotinlabels]{titletoc}
%\titlelabel{{\thetitle}.\quad}
%\usepackage{titletoc}
\usepackage[small]{titlesec}

\titleformat{\section}[block]
  {\fillast\medskip}
  {\bfseries{\thesection. }}
  {1ex minus .1ex}
  {\bfseries}
 
\titleformat*{\subsection}{\itshape}
\titleformat*{\subsubsection}{\itshape}

\setcounter{tocdepth}{2}

\titlecontents{section}
              [2.3em] 
              {\bigskip}
              {{\contentslabel{2.3em}}}
              {\hspace*{-2.3em}}
              {\titlerule*[1pc]{}\contentspage}
              
\titlecontents{subsection}
              [4.7em] 
              {}
              {{\contentslabel{2.3em}}}
              {\hspace*{-2.3em}}
              {\titlerule*[.5pc]{}\contentspage}

% hopefully not used.           
\titlecontents{subsubsection}
              [7.9em]
              {}
              {{\contentslabel{3.3em}}}
              {\hspace*{-3.3em}}
              {\titlerule*[.5pc]{}\contentspage}
%\makeatletter
\renewcommand\tableofcontents{%
    \section*{\contentsname
        \@mkboth{%
           \MakeLowercase\contentsname}{\MakeLowercase\contentsname}}%
    \@starttoc{toc}%
    }
\def\@oddhead{{\scshape\rightmark}\hfil{\small\scshape\thepage}}%
\def\sectionmark#1{%
      \markright{\MakeLowercase{%
        \ifnum \c@secnumdepth >\m@ne
          \thesection\quad
        \fi
        #1}}}
        
\makeatother

%\makeatletter

 \def\small{%
  \@setfontsize\small\@xipt{13pt}%
  \abovedisplayskip 8\p@ \@plus3\p@ \@minus6\p@
  \belowdisplayskip \abovedisplayskip
  \abovedisplayshortskip \z@ \@plus3\p@
  \belowdisplayshortskip 6.5\p@ \@plus3.5\p@ \@minus3\p@
  \def\@listi{%
    \leftmargin\leftmargini
    \topsep 9\p@ \@plus3\p@ \@minus5\p@
    \parsep 4.5\p@ \@plus2\p@ \@minus\p@
    \itemsep \parsep
  }%
}%
 \def\footnotesize{%
  \@setfontsize\footnotesize\@xpt{12pt}%
  \abovedisplayskip 10\p@ \@plus2\p@ \@minus5\p@
  \belowdisplayskip \abovedisplayskip
  \abovedisplayshortskip \z@ \@plus3\p@
  \belowdisplayshortskip 6\p@ \@plus3\p@ \@minus3\p@
  \def\@listi{%
    \leftmargin\leftmargini
    \topsep 6\p@ \@plus2\p@ \@minus2\p@
    \parsep 3\p@ \@plus2\p@ \@minus\p@
    \itemsep \parsep
  }%
}%
\def\open@column@one#1{%
 \ltxgrid@info@sw{\class@info{\string\open@column@one\string#1}}{}%
 \unvbox\pagesofar
 \@ifvoid{\footsofar}{}{%
  \insert\footins\bgroup\unvbox\footsofar\egroup
  \penalty\z@
 }%
 \gdef\thepagegrid{one}%
 \global\pagegrid@col#1%
 \global\pagegrid@cur\@ne
 \global\count\footins\@m
 \set@column@hsize\pagegrid@col
 \set@colht
}%

\def\frontmatter@abstractheading{%
\bigskip
 \begingroup
  \centering\large
  \abstractname
  \par\bigskip
 \endgroup
}%

\makeatother

%\DeclareSymbolFont{CMlargesymbols}{OMX}{cmex}{m}{n}
%\DeclareMathSymbol{\sum}{\mathop}{CMlargesymbols}{"50}

\usepackage[papersize={6.6in, 10.0in}, left=.5in, right=.5in, top=.6in, bottom=.9in]{geometry}
\linespread{1.05}
\sloppy
\raggedbottom
\usepackage[leqno]{amsmath}

\pagestyle{plain}
\usepackage{mathpartir}
\usepackage{stmaryrd}
\usepackage{mathtools}
\usepackage{tikz-cd}
\usepackage{microtype}
\usepackage{amssymb}
\usepackage{enumitem}

\newcommand{\mprime}{\ensuremath{^\prime}}

%\usepackage{fdsymbol}

% these include amsmath and that can cause trouble in older docs.
\makeatletter
\@ifpackageloaded{amsmath}{}{\RequirePackage{amsmath}}

\DeclareFontFamily{U}  {cmex}{}
\DeclareSymbolFont{Csymbols}       {U}  {cmex}{m}{n}
\DeclareFontShape{U}{cmex}{m}{n}{
    <-6>  cmex5
   <6-7>  cmex6
   <7-8>  cmex6
   <8-9>  cmex7
   <9-10> cmex8
  <10-12> cmex9
  <12->   cmex10}{}

\def\Set@Mn@Sym#1{\@tempcnta #1\relax}
\def\Next@Mn@Sym{\advance\@tempcnta 1\relax}
\def\Prev@Mn@Sym{\advance\@tempcnta-1\relax}
\def\@Decl@Mn@Sym#1#2#3#4{\DeclareMathSymbol{#2}{#3}{#4}{#1}}
\def\Decl@Mn@Sym#1#2#3{%
  \if\relax\noexpand#1%
    \let#1\undefined
  \fi
  \expandafter\@Decl@Mn@Sym\expandafter{\the\@tempcnta}{#1}{#3}{#2}%
  \Next@Mn@Sym}
\def\Decl@Mn@Alias#1#2#3{\Prev@Mn@Sym\Decl@Mn@Sym{#1}{#2}{#3}}
\let\Decl@Mn@Char\Decl@Mn@Sym
\def\Decl@Mn@Op#1#2#3{\def#1{\DOTSB#3\slimits@}}
\def\Decl@Mn@Int#1#2#3{\def#1{\DOTSI#3\ilimits@}}

\let\sum\undefined
\DeclareMathSymbol{\tsum}{\mathop}{Csymbols}{"50}
\DeclareMathSymbol{\dsum}{\mathop}{Csymbols}{"51}

\Decl@Mn@Op\sum\dsum\tsum

\makeatother

\makeatletter
\@ifpackageloaded{amsmath}{}{\RequirePackage{amsmath}}

\DeclareFontFamily{OMX}{MnSymbolE}{}
\DeclareSymbolFont{largesymbolsX}{OMX}{MnSymbolE}{m}{n}
\DeclareFontShape{OMX}{MnSymbolE}{m}{n}{
    <-6>  MnSymbolE5
   <6-7>  MnSymbolE6
   <7-8>  MnSymbolE7
   <8-9>  MnSymbolE8
   <9-10> MnSymbolE9
  <10-12> MnSymbolE10
  <12->   MnSymbolE12}{}

\DeclareMathSymbol{\downbrace}    {\mathord}{largesymbolsX}{'251}
\DeclareMathSymbol{\downbraceg}   {\mathord}{largesymbolsX}{'252}
\DeclareMathSymbol{\downbracegg}  {\mathord}{largesymbolsX}{'253}
\DeclareMathSymbol{\downbraceggg} {\mathord}{largesymbolsX}{'254}
\DeclareMathSymbol{\downbracegggg}{\mathord}{largesymbolsX}{'255}
\DeclareMathSymbol{\upbrace}      {\mathord}{largesymbolsX}{'256}
\DeclareMathSymbol{\upbraceg}     {\mathord}{largesymbolsX}{'257}
\DeclareMathSymbol{\upbracegg}    {\mathord}{largesymbolsX}{'260}
\DeclareMathSymbol{\upbraceggg}   {\mathord}{largesymbolsX}{'261}
\DeclareMathSymbol{\upbracegggg}  {\mathord}{largesymbolsX}{'262}
\DeclareMathSymbol{\braceld}      {\mathord}{largesymbolsX}{'263}
\DeclareMathSymbol{\bracelu}      {\mathord}{largesymbolsX}{'264}
\DeclareMathSymbol{\bracerd}      {\mathord}{largesymbolsX}{'265}
\DeclareMathSymbol{\braceru}      {\mathord}{largesymbolsX}{'266}
\DeclareMathSymbol{\bracemd}      {\mathord}{largesymbolsX}{'267}
\DeclareMathSymbol{\bracemu}      {\mathord}{largesymbolsX}{'270}
\DeclareMathSymbol{\bracemid}     {\mathord}{largesymbolsX}{'271}

\def\horiz@expandable#1#2#3#4#5#6#7#8{%
  \@mathmeasure\z@#7{#8}%
  \@tempdima=\wd\z@
  \@mathmeasure\z@#7{#1}%
  \ifdim\noexpand\wd\z@>\@tempdima
    $\m@th#7#1$%
  \else
    \@mathmeasure\z@#7{#2}%
    \ifdim\noexpand\wd\z@>\@tempdima
      $\m@th#7#2$%
    \else
      \@mathmeasure\z@#7{#3}%
      \ifdim\noexpand\wd\z@>\@tempdima
        $\m@th#7#3$%
      \else
        \@mathmeasure\z@#7{#4}%
        \ifdim\noexpand\wd\z@>\@tempdima
          $\m@th#7#4$%
        \else
          \@mathmeasure\z@#7{#5}%
          \ifdim\noexpand\wd\z@>\@tempdima
            $\m@th#7#5$%
          \else
           #6#7%
          \fi
        \fi
      \fi
    \fi
  \fi}

\def\overbrace@expandable#1#2#3{\vbox{\m@th\ialign{##\crcr
  #1#2{#3}\crcr\noalign{\kern2\p@\nointerlineskip}%
  $\m@th\hfil#2#3\hfil$\crcr}}}
\def\underbrace@expandable#1#2#3{\vtop{\m@th\ialign{##\crcr
  $\m@th\hfil#2#3\hfil$\crcr
  \noalign{\kern2\p@\nointerlineskip}%
  #1#2{#3}\crcr}}}

\def\overbrace@#1#2#3{\vbox{\m@th\ialign{##\crcr
  #1#2\crcr\noalign{\kern2\p@\nointerlineskip}%
  $\m@th\hfil#2#3\hfil$\crcr}}}
\def\underbrace@#1#2#3{\vtop{\m@th\ialign{##\crcr
  $\m@th\hfil#2#3\hfil$\crcr
  \noalign{\kern2\p@\nointerlineskip}%
  #1#2\crcr}}}

\def\bracefill@#1#2#3#4#5{$\m@th#5#1\leaders\hbox{$#4$}\hfill#2\leaders\hbox{$#4$}\hfill#3$}

\def\downbracefill@{\bracefill@\braceld\bracemd\bracerd\bracemid}
\def\upbracefill@{\bracefill@\bracelu\bracemu\braceru\bracemid}

\DeclareRobustCommand{\downbracefill}{\downbracefill@\textstyle}
\DeclareRobustCommand{\upbracefill}{\upbracefill@\textstyle}

\def\upbrace@expandable{%
  \horiz@expandable
    \upbrace
    \upbraceg
    \upbracegg
    \upbraceggg
    \upbracegggg
    \upbracefill@}
\def\downbrace@expandable{%
  \horiz@expandable
    \downbrace
    \downbraceg
    \downbracegg
    \downbraceggg
    \downbracegggg
    \downbracefill@}

\DeclareRobustCommand{\overbrace}[1]{\mathop{\mathpalette{\overbrace@expandable\downbrace@expandable}{#1}}\limits}
\DeclareRobustCommand{\underbrace}[1]{\mathop{\mathpalette{\underbrace@expandable\upbrace@expandable}{#1}}\limits}

\makeatother


% some nicer symbols
\makeatletter
\DeclareFontFamily{U}{matha}{\hyphenchar\font45}
\DeclareFontShape{U}{matha}{m}{n}{
      <5> <6> <7> <8> <9> <10> gen * matha
      <10.95> matha10 <12> <14.4> <17.28> <20.74> <24.88> matha12
      }{}
\DeclareSymbolFont{matha}{U}{matha}{m}{n}
\DeclareFontSubstitution{U}{matha}{m}{n}

\def\mathabx@aliases#1#2{\@mathabx@aliases#1#2?\@end}
\def\@mathabx@aliases#1#2#3\@end{\ifx#2?\else
	\let#2=#1\@mathabx@aliases#1#3\@end\fi}%
\DeclareMathSymbol{\leftarrow}             {3}{matha}{"D0}
	\mathabx@aliases\leftarrow\gets
\DeclareMathSymbol{\rightarrow}            {3}{matha}{"D1}
	\mathabx@aliases\rightarrow\to
\DeclareMathSymbol{\wedge}         {2}{matha}{"5E}
	\mathabx@aliases\wedge\land
\DeclareMathSymbol{\vee}           {2}{matha}{"5F}
	\mathabx@aliases\vee\lor
\DeclareMathSymbol{\vdash}         {3}{matha}{"24}
\DeclareMathSymbol{\dashv}         {3}{matha}{"25}
\DeclareMathSymbol{\nvdash}        {3}{matha}{"26}
\DeclareMathSymbol{\ndashv}        {3}{matha}{"27}
\DeclareMathSymbol{\vDash}         {3}{matha}{"28}
\DeclareMathSymbol{\Dashv}         {3}{matha}{"29}
\DeclareMathSymbol{\nvDash}        {3}{matha}{"2A}
\DeclareMathSymbol{\nDashv}        {3}{matha}{"2B}
\DeclareMathSymbol{\Vdash}         {3}{matha}{"2C}
\DeclareMathSymbol{\dashV}         {3}{matha}{"2D}
\DeclareMathSymbol{\nVdash}        {3}{matha}{"2E}
\DeclareMathSymbol{\ndashV}        {3}{matha}{"2F}
\makeatother

\usepackage[tiny]{titlesec}
\titleformat{\section}
  {\normalfont\bfseries}
  {\thesection.}
  {.5em}
  {}

\usepackage{cite}

% make sure there is enough TOC for reasonable pdf bookmarks.
\setcounter{tocdepth}{3}
\usepackage{amsthm}

\theoremstyle{definition}


\usepackage[colorlinks=true
,breaklinks=true
,urlcolor=blue
,anchorcolor=blue
,citecolor=blue
,filecolor=blue
,linkcolor=blue
,menucolor=blue
,linktocpage=true]{hyperref}
\hypersetup{
bookmarksopen=true,
bookmarksnumbered=true,
bookmarksopenlevel=10,
}

\date{}
\def\mm{\Vdash}
\def\kxa{\kappa_{x \in A}}
\def\prAB{\pi_{A,B}}
\def\prpAB{\pi'_{A,B}}
\def\pr#1{\pi_{#1}}
\def\prp#1{\pi'_{#1}}
\def\to{\longrightarrow}
\def\xto#1{\xrightarrow{\kern.6em #1 \kern.6em}}
\def\ent{\vdash}
\def\imp{\shortrightarrow}
\def\iff{\leftrightarrow}
\def\from{\Leftarrow}
\def\impl{\Rightarrow}
\def\union{\cup}
\def\fexp#1#2{#1^{#2}}
\def\fexpBA{\fexp{B}{A}}
\def\inc{\subseteq}
\def\dom{\mathop{\rm dom}}
\def\cod{\mathop{\rm cod}}
\def\id{{\mathrm 1}}
\def\res{\!\upharpoonleft\!}
\def\ffam{\varphi}
\def\comp{\circ}
\def\bbone{\mathbb 1}
\def\one{\mathbf 1}
\def\zeromap{0}
\def\bbzero{{\mathbb O}}
\def\ccc{{c.c.c.}}
\def\ev{\varepsilon}
\def\ebc{\varepsilon_{BC}}
\def\evBA{\varepsilon_{B,A}}
\def\L{\Lambda}
\def\l{\lambda}
\def\lx{\lambda_x}
\def\ly{\lambda_y}
\def\lu{\lambda_u}
\def\lv{\lambda_v}
\def\lz{\lambda_z}
\def\subX{\ensuremath{_\text{X}}}
\def\lm#1.#2{\lambda#1.\, #2}
\def\br#1{[\, #1 \, ]}
\def\V{V}
\def\U{U}
\def\D{D}
\def\C{\mathcal C}
\def\S{\mathcal S}
\def\lxy{\l x\, \l y . \,}
\def\lmm#1#2.#3{\l #1\, \l #2 . \, #3}
\def\sss{(*\!*\!*)}
\def\ss{(**)}
\def\ssn{(**_n)}
\def\scop{\S^{\C^{op}}}
\def\sland{\wedge}
\def\PU{\mathcal P U}
\def\P{\mathcal P}
\def\UU{(U\to U)}
\def\BA{B \to A}
\def\AB{A \to B}
\def\calA{{\cal A}}
\def\calB{{\cal B}}
\def\cI{{I}}
\def\cS{{S}}
\def\cK{{K}}
\def\cIA{\cI_{A}}
\def\cKAB{\cK_{A,B}}
\def\cSABC{\cS_{A,B,C}}
\def\la{\langle}
\def\ra{\rangle}
\def\bracket#1{\la #1 \ra}
\def\app{\mathop{{}^\wr}\kern-.8pt}
\def\schon{Sch\"onfinkel}
\def\nat{\mathbb N}
\def\pf{\varphi}

\newcommand{\be}{\begin{equation}}
\newcommand{\ee}{\end{equation}}
\newcommand{\bes}{\begin{equation*}}
\newcommand{\ees}{\end{equation*}}

\newcommand{\fcat}[1]{{\mathbf {#1}}} 
\newcommand{\Set}{\fcat{Sets}}

\DeclareMathOperator{\Hom}{{Hom}}

\newcommand{\iso}{\cong}                % Isomorphism
\newcommand{\eqv}{\simeq}               % Equivalence
\newcommand{\sub}{\subseteq}            % Subset (possibly not proper)

\usepackage{amsmath,amsthm}

\theoremstyle{definition}

\newtheorem{thm}{Theorem}[section]
\newtheorem{lemma}[thm]{Lemma}
\newtheorem{cor}[thm]{Corollary}

\theoremstyle{definition}
\newtheorem{defn}{Definition}[section]
\newtheorem{example}{Example}[section]

\theoremstyle{definition}
\newtheorem{remark}{Remark}[section]
\newtheorem{note}{Note}[section]

% makes "=" with "x" under it
\makeatletter
\DeclareRobustCommand{\eqx}{\mathrel{\mathpalette\eq@{X}}}
\DeclareRobustCommand{\eqy}{\mathrel{\mathpalette\eq@{Y}}}
\DeclareRobustCommand{\eqxx}{\mathrel{\mathpalette\eq@{X \union x}}}
\newcommand{\eq@}[2]{%
  \vtop{\offinterlineskip
    \ialign{\hfil##\hfil\cr
      $\m@th#1=$\cr % top
      \noalign{\sbox\z@{$\m@th#1\mkern0mu$}\kern-\wd\z@}
      $\m@th\alexey@demote{#1}#2$\cr
    }%
  }%
}
\newcommand{\alexey@demote}[1]{%
  \ifx#1\displaystyle\scriptstyle\else
  \ifx#1\textstyle\scriptstyle\else
  \scriptscriptstyle\fi\fi
}

\makeatother

% footnote tricks
\usepackage[symbol]{footmisc}
\renewcommand{\thefootnote}{\fnsymbol{footnote}}

\makeatletter
\let\original@footnotemark\footnotemark
\newcommand{\align@footnotemark}{%
  \ifmeasuring@
    \chardef\@tempfn=\value{footnote}%
    \original@footnotemark
    \setcounter{footnote}{\@tempfn}%
  \else
    \iffirstchoice@
      \original@footnotemark
    \fi
  \fi}
\pretocmd{\start@align}{\let\footnotemark\align@footnotemark}{}{}
\makeatother

\makeatletter
\newcommand*\dotop{\mathpalette\bigcdot@{.6}}
\newcommand*\bigcdot@[2]{\mathbin{\vcenter{\hbox{\scalebox{#2}{$\m@th#1\bullet$}}}}}
\makeatother

\title{\large Aspects of Higher Order Categorical Logic\footnote{The authors belong to the ``Groupe interuniversitaire en etudes categoriques'' in Montreal. They acknowledge support from the Natural Sciences and
Engineering Research Council of Canada and from the Quebec Department of
Education. They wish to thank Denis Higgs and the referee for their careful
reading of the manuscript.}}
\author{\normalsize J. Lambek and P. J. Scott}

\setcounter{section}{-1}
\begin{document}

\maketitle

\section{Introduction}

It has become clear for some time that categorists and logicians have been
doing the same thing under different names. This situation is strikingly
illustrated in higher order logic, where the connections are particularly
illuminating. In this article we briefly survey portions of our forthcoming
book [LS6], to which the reader is referred for more details.

Systems of higher order logic have been studied for a long time. We have
in mind logical systems in which variables and quantifiers range over functions
(as in the $\l$-calculus) or elements and subsets of given sets (as in type
theory), not just over individuals (as in first order logic). Higher order
concepts naturally occur in mathematics: algebraists quantify over ideals,
topologists over open sets and analysts over functions. Yet, in spite of its
expressive advantage, higher order logic never achieved the popularity of first
order logic. Perhaps its proof theory and model theory were deemed to be too
hard.

Thus, after the monumental work in type theory by Russell and Whitehead
(Principia Mathematica, 1908), there are only sporadic important contributions
to the literature. G\"{o}del (1931) established his incompleteness theorem for
type theory and related systems, Church (1940) combined $\l$-calculus with type
theory, Henkin (1949) discussed models for type theory and, more recently,
Abraham Robinson (1964) used type theory in his book ``Nonstandard Analysis'',
a lead not continued by his followers. Let us also mention that type theories
have recently become important in computer science.

Theories of functionality, namely $\l$-calculi, have fared somewhat better.
After pioneering work by Sch\"onfinkel, Curry, Church and Rosser in the 1920's
and 30's, these systems had a small but devoted following. In the late 60's,
an explosion of activity surrounding the models of Dana Scott (1971) and concomitant
work in computer science revived interest in them.

Meanwhile, the categorical side saw rapid development. The ready accept-
ance of category theory as a working language in many areas of mathematics
foreshadowed its introduction into logic and foundations. Nevertheless, it
came as a surprise to many people when, in the 1960's, Lawvere (1969, 1970) and
one of the present authors [L1] pointed out some remarkable connections between
category theory and logic. Indeed, after -the discovery of elementary toposes
by Lawvere and Tierney (1970), the field of "categorical logic" underwent
intense development. The reader is referred to the survey article by Kock and
Reyes (1977) and the monograph by Makkai and Reyes (1977).

The situation in higher order logic, as far as it is treated in our book,
is summarized as follows:

\medskip
\renewcommand{\arraystretch}{1.5}

\begin{center}
\begin{tabular}{ c | c }
 \hline
Logic & Algebra \\
 \hline
 untyped $\l$-calculus & C-monoids \\
 typed $\l$-calculus & cartesian closed categories \\
 type theory & toposes \\
 \hline
\end{tabular}
\end{center}
\medskip

\noindent
Indeed, we shall see that each side, when suitably formulated, gives rise to a
category. Moreover, the comparison between the two sides is mediated by
functors which set up an equivalence or adjointness. In the meantime, Robert
Seely, has found another such comparison:

\begin{center}
\begin{tabular}{ c | c }
 \hline
Logic & Algebra \\
 \hline
Martin-L\"of type theory  & locally cartesian closed categories \\
 \hline
\end{tabular}
\end{center}
\medskip

\noindent
Undoubtedly, there are many other such situations. For example, it would be
interesting to pinpoint the logical equivalent of cartesian closed categories
with equalizers.

\section{Cartesian closed categories and typed $\l$-calculi}

Cartesian closed categories are becoming increasingly important in many
branches of mathematics. Indeed, as we shall see, in logic they play a fundamental
role.

A {\em cartesian closed category} is a category $\C$ with (canonical) finite
products and exponentiation. This means that $\C$ has a terminal object $\one$ and
objects $A \times B$ and $\fexp{B}{A}$, for all $A,B$ in $\C$, together
with natural isomorphisms:

\bes
\Hom(C ,A) \times \Hom(C ,B)  \iso \Hom(C ,A \times B) ,
\tag{a}
\ees
\bes
\Hom(C \times A,B) \iso \Hom(C,\fexp{B}{A}).
\tag{b}
\ees

Cartesian closed categories abound in mathematics. For example, all
functor categories $\Set^{\C}$ ($\C$ small) and all toposes (see below) are cartesian
closed. So are the categories of Kelley spaces and Kuratowski limit spaces
(see MacLane, 1971).

A useful alternative presentation considers both products and exponentials
as adjoint functors, hence equationally definable (e.g., see MacLane, 1971).

\def\deq{\mathop{\cdot\mkern-5mu=\mkern-5mu\cdot}}

\begin{defn}
A {\em cartesian closed category} has the following objects,
arrows and equations:

\paragraph{Objects:}
\begin{enumerate}
\item[(i)] $\one$ is an object.
\item[(ii)] If $A$ and $B$ are objects, so are $A \times B$ and $\fexp{B}{A}$
\end{enumerate}

\paragraph{Arrows:}
\begin{enumerate}
\item[(i)] $1_A: A \to A$, $O_A: A \to \one$, $\prAB: A \times B \to A$, $\prpAB: A \times B \to B$, and $\evBA: \fexpBA \to A \times B$ are arrows.

\item[(ii)] The following rules generate new arrows from old:

\[
\inferrule {f: A \to B \quad g: B \to C}{gf: A \to C},\quad
\inferrule {f: C \to A \quad\! g: C \to B}{\la f, g \ra: C \to A \times B}, \quad
\inferrule {f: C \times A \to B}{f^*: C \to \fexpBA}
\]

\end{enumerate}

\paragraph{Equations:}
\begin{enumerate}
\item[(1)] $f1_A \doteq f$, $1_Bf \doteq f$, and $(hg)f \doteq g(hf)$,
for all $f:A \to B$, $g:B \to C$, $h: C\to D$;
\item[(2)] $f \doteq 0_A$, for all $f: A \to 1$;
$\prAB\bracket{f,g} \doteq f$, $\prpAB\bracket{f,g} \doteq g$, and
$\la\prAB h, \prpAB h\ra \doteq h$,
for all $f:C \to A$, $g:C \to B$, and $h: C\to A\sland B$;
\item[(3)] $\evBA\bracket{f^*\pr{C,A}, \prp{C,A}} \doteq f$, and
$(\evBA\bracket{g\,\pr{C,A}, \prp{C,A}})^* \doteq g$,
for all $f: C\times A \to B$, and $g: C \to \fexpBA$.
\end{enumerate}

\end{defn}
\noindent
It is understood that $\doteq$ is a congruence relation%
\footnote{Editor's note: the original paper used a symbol more like $\deq$ for this equivalence relation. I changed it to $\doteq$ because to really do it right I'd need to make a new glyph in some font somewhere and I don't want to do that.}
on $\Hom$-sets satisfying the
above. There may be other objects, arrows and equations than follow from the
above.

The reader may easily verify that the equations (1) to (3) contain not
only the equations of a category with a terminal object, but yield also the
natural isomorphisms (a) and (b). These equations may be viewed as presenting
a multi-sorted partial algebra structure on the $\Hom$-sets, with nullary, unary
and binary operations satisfying appropriate identities. Alternatively, they
may be looked upon as describing a ``graphical algebra'' in the spirit of
Burroni (1981). To see this, the reader may first wish to replace the rule of
arrow formation introducing $f^*$ by another basic arrow 
$\eta_{C,A}: C \to \fexp{(C \times A)}{A}$.

Another view of cartesian closed categories is to consider them as
{\em deductive systems} [Ll]. Write the objects $\one$, $A \times B$, and $\fexpBA$
as $\top$, $A \sland B$, and $A \impl B$ respectively. One may then think of an arrow
$f: A \to B$ as a {\em deduction} of $B$ from $A$ or as a {\em proof}
of the {\em entailment} $A \ent B$. Proofs may
be writ ten in ``tree form''. For example:

\[
\inferrule {A \sland B \xto{\pi'} B \quad A \sland B \xto{\pi} A}{A \sland B \xto{\la \pi', \pi \ra}  B \sland A}
\]
proves the commutative law for conjunction, while
\[
\inferrule{ }{(A \impl B) \sland A \xto{\evBA} B}
\]
may be regarded as an "axiom" (at least in a freely generated deductive system).

Logicians might say that we have presented a system of natural deduction
for the positive intuitionistic propositional calculus, but with an additional
twist: the equations of a cartesian closed category impose an {\em equality
relation} between proofs (or proof trees). For example:
\[
\begin{aligned}
\inferrule{C \xto{f} A \quad C \xto{g} B}{
\inferrule{C \xto{\la f,g \ra} A \sland B \quad A \sland B \xto{\pi} A}{C \xto{\pi\la f,g \ra} A }}
\end{aligned} \qquad \hbox{\textrm{equals}} \quad C \xto{f} A.
\]
This notion of ``equivalence of proofs'' is essentially the same as that intro-
duced by logicians (Prawitz, 1971), as pointed out by Mann (1975).

There is still another connection of cartesian closed categories to logic.
This is via the language of {\em typed $\l$-calculus}. Although older than category
theory, $\l$-calculus may also be regarded as an equational theory of functions
(or functional processes) in which composition is mirrored by substitution.
As we shall see, some problems in cartesian closed categories are efficiently
handled using typed $\l$-calculus.

\begin{defn}
A {\em typed $\l$-calculus} is a formal theory as follows. It
consists of types, terms and equations (between terms of the same type).

\medskip
\noindent
(a) {\bf Types}:
\begin{enumerate}
\item[(a1)] $\one$ is a type.
\item[(a2)] If $A$ and $B$ are types so are $A \times B$ and $\fexpBA$
\end{enumerate}

\medskip
\noindent
(b) {\bf Terms}: (We write ``$t \in A''$ to say that $t$ is a term of type $A$.)
\begin{enumerate}
\item[(b1)] There are countably many variables of each type, say $x^{A}_{i} \in A$ if $i \in \nat$.
\item[(b2)] $\star \in \one$.
\item[(b3)] If $a \in A$, $b \in B$ and $c \in A \times B$, then $\la a,b \ra \in A \times B$, $\prAB(c) \in A$ and $\prpAB(c) \in B$.
\item[(b4)] If $f \in \fexpBA$ and $a \in A$ then $\evBA(f,a) \in B$.
\item[(b5)] If $x$ is a variable of type $A$ and $\phi(x) \in B$ then $\l_{x \in A}{\varphi(x)} \in \fexpBA$.
\end{enumerate}
\noindent
We often abbreviate $\evBA(f,a)$ as $f \app a$ when the type subscripts are clear
from the context.

\medskip
\noindent
(c) {\bf Equations}: All equations have the form $a \eqx a'$, where $a$ and $a'$ have
the same type and $X$ is a set of variables containing all variables occurring
freely in $a$ and $a'$.
\begin{enumerate}
\item[(c1)] $\eqx$ is reflexive, symmetric and transitive. Moreover, if $X \inc Y$ we have
\[
\inferrule{a \eqx b}{a \eqy b}
\]
(that is, from $a \eqx b$ we may infer $a \eqy b$).
\item[(c2)] We have the substitution rules:
\[
\inferrule{c \eqx c'}{\pf(c) \eqx \pf(c)}, \quad 
\inferrule{a \eqx a', \quad b \eqx b'}{\psi(a,b), \eqx \psi(a',b')}
\]
if $\pf(z) \equiv \pi(z)$ or $\pi'(z)$ and $\psi(x,y) = \la x,y \ra$ or $x \app y$ and also
\[
\inferrule{\pf(x) \eqxx \psi(x)}{\l_{x\in A}\pf(x) \eqx \l_{x \in A}\psi(x)}
\]
if $x \notin X$.

\end{enumerate}
\end{defn}
