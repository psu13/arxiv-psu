\begin{thebibliography}{99}
\bibitem{B52} D. Bohm,  {\it A suggested interpretation of the quantum theory in terms of ``hidden'' variables. I},  Phys. Rev. {\bf 85}, 166-179; D. Bohm, {\it A suggested interpretation of the quantum theory in terms of ``hidden'' variables. II}, Phys. Rev.  {\bf 85}, 180-193 (1952)
\bibitem{DEBR60} L. de Broglie \textit{Une interpretation causale et non lindaire de la mechanique ondulatoire: la theorie de la double solution} (Gauthier-Villars, Paris, 1956). [English translation: \textit{ Non-linear wave mechanics: A causal Interpretation} (Elsevier, Amsterdam, 1960)]; L. de Broglie, The reinterpretation of wave mechanics. Found. Phys. {\bf 1}, 5-15 (1970)
\bibitem{DEWD83}  C. Dewdney, PhD Thesis, University of London, 1983
\bibitem{DEWD79}  C. Phillipides, C. Dewdney, B. J. Hiley, {\it Quantum interference and the quantum potential}, Nuovo Cimento B {\bf 52}, 15-28 (1979)
\bibitem{DEWD82} C. Dewdney,  B. J. Hiley, {\it A quantum potential description of one-dimensional time-dependent scattering from square barriers and square wells}, {\bf 12} 1, 27-28 (1982)
\bibitem{DEWD85} C. Dewdney,  {\it Particle trajectories and interference in a time-dependent model of neutron single crystal interferometry}, Phys. Lett. A {\bf 109}, 377-383 (1985);
\bibitem{DEWD86} C. Dewdney, P. R. Holland, A. Kyprianidis, {\it What happens in a spin measurement}, Phys. Lett. A, {\bf 119}, 259-267 (1986); 
\bibitem{DEWD87} C. Dewdney, P. R. Holland, A. Kyprianidis, {\it A quantum potential approach to spin superposition in neutron interferometry}, Phys. Lett. A {\bf 121}, 105-110 (1987)
\bibitem{DEWDEPR87} C. Dewdney, P. R. Holland,  A. Kyprianidis, {\it A causal account of non-local Einstein-Podolsky-Rosen correlations}, J. Phys. A: Math. Gen. {\bf 20}, 4717-4732 (1987)
\bibitem{DEWD88} C. Dewdney, P. R. Holland, A. Kyprianidis, J. P. Vigier, {\it Spin and  non-locality in quantum mechanics}, Nature {\bf 336}, 536-544 (1988)
\bibitem{BST55} D. Bohm, R. Schiller,  and J. Tiomno, {\it A causal interpretation of the Pauli equation (A)}, Nuoevo Cimento Supplement {\bf 1}, series X, No. 1, 48-66 (1955); D. Bohm, R. Schiller, {\it A causal interpretation of the Pauli equation (B)}, Nuoevo Cimento Supplement {\bf I}, series X, No. 1, 67-91 (1955)
\bibitem{K89} D. Home, P. N. Kaloyerou, {\it New twists to Einstein's two-slit experiment: Complementarity vis-a-vis the causal interpretation}, Journal of Physics A {\bf 22}, 3253-3266 (1989)
\bibitem{BR28} (a) N. Bohr, {\it The quantum postulate and the recent development of atomic theory} at Atti del Congresso Internazionale dei Fisici, Como, 11-20 September 1927 (Zanichelli, Bologna, 1928), vol 2   pp. 565-588; (b) substance of the Como lecture is reprinted in Nature {\bf 121}, 580-590 (1928); N. Bohr \textit{Atomic Theory and the Description of Nature} (Cambridge University Press, Cambridge, 1934, reprinted 1961); N. Bohr  \textit{ Atomic Physics and Human Knowledge} (Science Editions, New York, 1961); N. Bohr,  {\it Discussion with Einstein on epistemological problems in quantum mechanics} in  Albert Einstein, Philosopher-Scientist, ed. by P. A. Schilpp (Evansten, IL: Library of Living Philosophers, 1949) pp. 201-241;  reprint: (Open Court, La salle,  Illinois, third edition, 1982) pp. 201-241;  M. Jammer \textit{ The Philosophy of Quantum Mechanics: The Interpretations of Quantum Mechanics in Historical Perspective} (John Wiley \& Sons, New York, 1974) 
\bibitem{B51} D. Bohm,  {\it Quantum Theory} (Prentice-Hall Inc, New Jersey, 1951)
\bibitem{jon61} Von C. J\"{o}nsson, {\it Elektroneninterferenzen an mehreren k\"{u}nstlich hergestellten Feinspalten}, Z. Phys. {\bf 161}, 454-474 (1961).
\bibitem{K85} P. N. Kaloyerou, PhD Thesis, {\it Investigation of the Quantum Potential in the Relativistic Domain}, University of London (1985); D. Bohm, B. J. Hiley, P. N. Kaloyerou, {\it An ontological basis for the quantum theory: A causal interpretation of  quantum fields}, Phys. Rep. {\bf 144}, 349-375 (1987);  P. N. Kaloyerou, {\it The causal interpretation of the electromagnetic field}, Phys. Rep.  {\bf 244}, 287-358 (1994); P. N. Kaloyerou, {\it A field theoretic causal model  of the Mach-Zehnder Wheeler delayed-choice experiment}, Physica A {\bf 355},  297-318 (2005); P. N. Kaloyerou, {\it The GRA beam-splitter experiments and particle-wave duality of light}, J. Phys. A: Math. Gen.  {\bf 39}, 11541-11566 (2006).
\bibitem{BF89} R. L. Burden and J. D Faires, {\it Numerical Analysis}, (PWS Kent Publishing Company, Boston, 1989), p282, \S 5.9
\bibitem{WZ79} W. K. Wootters and W. H. Zurek, Complementarity in the double-slit experiment: quantum nonseparability and a quantitative statement of Bohr's principle. Phys. Rev. D {\bf 19}, 473 - 484 (1979)
\bibitem{K92} P. N. Kaloyerou,  The Wootters-Zurek development of Einstein's two-slit  experiment. Found. Phys.  {\bf 22}, 1345-1377 (1992)
\bibitem{K2016} P. N. Kaloyerou, {\it Critique of quantum optical experimental refutations of Bohr's principle of complementarity, of the Wootters-Zurek principle of complementarity, and of the particle-wave duality relation}, Found. Phys., {\bf 46}, 138-175 (2016).
\bibitem{K2017} P. N. Kaloyerou, {\it A brief overview and some comments on the weak measurement protocol}, Phys Astron Int J {\bf 5}, 1-7 (2017).
\bibitem{KOCSIS2011} S. Kocsis, B. Braverman, S. Ravets, M. J. Stevens, R. P. Mirin, L. K. Shalm, and A. M. Steinberg, {\it Observing the trajectories of a single photon using
weak measurement}, Science {\bf 332}, 1170 (2011).
\end{thebibliography}
