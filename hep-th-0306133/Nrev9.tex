%%%%%%%% Nrev9 as of Feb. 28 2003

\def\lfrac#1#2{{\displaystyle{#1\over#2}}}
\section Weakly interacting Bose gas and large $N$ techniques


%\beginbib

\nref\LY{T.D.  Lee and C.N.  Yang, {\it Phys.  Rev.}   112 (1957)
1419.}

\nref\huang{K. Huang, in {\it Stud.  Stat.  Mech}, {\bf II}, J. de
Boer and G.E.  Uhlenbeck, eds.  (North Holland Publ., Amsterdam, 1964), 1.}

\nref\toyoda{T.~Toyoda, {\it Ann. Phys.} (NY)   141  (1982)  154.}

\nref\stoof{H.T.C.  Stoof, {\it Phys.  Rev.}  A45 (1992)  8398.}
\nref\rBijl{M. Bijlsma and H.T.C.  Stoof, {\it Phys.  Rev.}  A54
(1996) 5085.}

\nref\GCL{P. Gr\"uter, D. Ceperley, and F. Lalo\"e, {\it Phys.  Rev.
Lett.} 79 (1997) 3549.}

\nref\laloe{M. Holzmann, P. Gr\"uter, and F. Lalo\"e, cond-mat/9809356, {\it
Euro.  Phys.  J.} B  10 (1999) 739.}

\nref\club{G. Baym, J.-P.  Blaizot, M. Holzmann, F. Lalo\"e, and D.
Vautherin, {\it Phys.  Rev.  Lett.} 83 (1999) 1703, cond-mat/9905430.}
\nref\rBBZJ{G. Baym, J.-P.  Blaizot and J. Zinn-Justin, {\it Euro. Phys. Lett.}
49 (2000) 150.}

\nref\rPA{P. Arnold, B. Tomasik,
{\it Phys. Rev.} A62 (2000) 063604, cond-mat/0005197.}
%\nref\zinn{J.~Zinn-Justin, {\it Quantum Field Theory and Critical
%Phenomena}, Clarendon Press (Oxford 1989, third ed.~1996).}
%\nref\rlargeN{H.E.  Stanley, {\it Phys.  Rev.} 176 (1968) 718; R. Abe,
%Prog.  Theor.  Phys. 48 (1972) 1414; {\it ibidem}\/ 49 (1973) 113; S.K.  Ma,
%{\it Phys.  Rev.  Lett.} 29 (1972) 1311, {\it Phys.  Rev.} A7 (1973) 2172; M.
%Suzuki, {\it Phys.  Lett.} 42A (1972) 5, {\it Prog.  Theor.  Phys.} 49 (1973)
%424; K.G.  Wilson, {\it Phys.  Rev.} D7 (1973) 2911.}
%\nref\rDoGriv{See also the contributions of S.K.  Ma and E. Br\'ezin,
%J.C.  Le Guillou and J. Zinn-Justin to {\it Phase Transitions and Critical
%Phenomena} vol. 6, C. Domb and M.S. Green eds.  (Academic Press, London
%1976).}
%\nref\rZJTai{For a recent review see J.~Zinn-Justin, {\it Vector
%models in the large $N$ limit:  a few applications}, lecture notes of the
%11$^{\rm th}$ Taiwan Spring School, Taipei 1997, Saclay preprint SPhT/97-018,
% hep-th/9810198.}

%\nref\aharo{R.A.  Ferrel and D.J.  Scalapino, {\it Phys.  Rev.  Lett.}
%29 (1972) 413; A. Aharony, {\it Phys.  Rev.} B10 (1974) 2834.}

%\nref\dimreg{J. Ashmore, {\it Lett.  Nuovo Cimento} 4 (1972) 289;
%G.~'t~Hooft and M.~Veltman, {\it Nucl.  Phys.} B44 (1972) 189; C.G.~Bollini
%and J.J.~Giambiaggi, {\it Phys.  Lett.} 40B (1972) 566, {\it Nuovo Cimento}
%12B (1972).}

\def\cite{\refs}

\nref\rarnold{
%P. Arnold, B. Tomasik, {\it Phys. Rev.} A64, 053609
%(2001), cond-mat/0105147;
P. Arnold, G.D. Moore, {\it Phys. Rev. Lett.} 87 (2001) 120401,
cond-mat/0103228.}\nref\rarnoldB{
% {\it Phys. Rev.} E64 (2001) 066113, cond-mat/0103227;
 P. Arnold, G.D. Moore, B. Tomasik, {\it
Phys. Rev.} A65 (2002) 013606, cond-mat/0107124.}
\nref\rKPS{V.A.
Kashurnikov, N. Prokof'ev and B. Svistunov, {\it Phys.  Rev. Lett.}
87 (2001) 120402,
 cond-mat/0103149.}
 %{ N. Prokof'ev, and B. Svistunov, cond-mat/0103146.}
%\nref\HoKr{M. Holzmann and W. Krauth, {\it Phys.  Rev. Lett.} 83
%(1999) 2687, cond-mat/9905198.}
\nref\rGBHV{G. Baym, J.-P. Blaizot, M. Holzmann, F. Laloe, D.
Vautherin; cond-mat/0107129, to appear in EJP B.}\nref\rHolz{ M.
Holzmann, G. Baym, J.-P. Blaizot, F. Laloe,
 {\it Phys. Rev. Lett.} 87 (2001)  120403, cond-mat/0103595.}

%\nref\rHBL{M. Holzmann, G. Baym, J.-P. Blaizot and F. Lalo\"e,
 %{\it Phys.  Rev. Lett.}
%87 (2001) 120403,  .}
 \nref\rKPR{J.-L. Kneur, M. B. Pinto and R. O. Ramos,
cond-mat/0207295.}

%\endbib

%
%\subsection Introduction

As it is well known a free Bose gas undergoes at a low but finite temperature a transition called Bose--Einstein condensation. The condensed low temperature phase is characterized by a macroscopic occupation of the one-particle ground state. \sslbl\scBEC \par
The effect of a weak repulsive two-body interaction on the transition temperature of a dilute Bose gas at fixed density has been controversial for a long time
%\cite
\refs{\LY
%,\huang,\toyoda,\stoof,\GCL,
-\laloe}. In ref.~\refs{\club} it has first been argued theoretically
 that the transition temperature $T_c$ increases linearly with the strength of the interaction, parametrized in terms of the scattering length $a$. However, the coefficient cannot be obtained from perturbation theory.  A simple self-consistent approximation was thus used to derive an explicit estimate. \par
A better understanding of the physics of the weakly interacting Bose gas came from the recognition that the universal properties of the system under study, like the helium superfluid transition, can be described by a particular example of the general $N$ vector model, for $N$ =2 \refs{\rBBZJ}.
Renormalization group (RG) arguments then allowed to prove the existence, besides the universal IR behaviour common with the superfluid transition, of a universal large momentum (UV) behaviour peculiar to systems with a very weak two-body interaction. The long wavelength properties of the weakly interacting Bose gas can be described by the three-dimensional, super-renormalizable, euclidean  $(\phib^2)^2$ field theory. A more direct and general RG derivation of the linear behaviour of the shift of the critical temperature and the universality of the coefficient followed.
\par
However, the calculation of the coefficient remained a non-perturbative problem. A possible method to calculate the temperature shift is to generalize the problem to arbitrary $N$.  This generalization makes new tools available; in particular, the coefficient of $\Delta T_c/T_c$ can be calculated by carrying out an expansion in $1/N$.  The leading contribution to $\Delta T_c/T_c$
already requires  a $1/N$ calculation \refs{\rBBZJ}, as we show explicitly below.
The result happens to be independent of $N$, for non-trivial reasons.  The
calculation involves subtle technical points, which are most easily dealt with by dimensional regularization.  More recently, a $1/N^2$ calculation
has been performed that yields the $1/N$ correction to $ \Delta
T_c/T_c$ \refs{\rPA}. The relative correction for $N=2$ is about
$26\%$, a  correction which is smaller than what is typically found in the calculation of critical exponents. \par
%We lay out the basics of the
%problem.  Then in Sec.~III we present the general $N$ vector model and
%analyze the behavior of the temperature shift by renormalization group
%arguments.  Finally in Sec.~IV we calculate the leading order contribution.
%
\subsection{Quantum field theory and Bose--Einstein condensation}

We consider a system of identical non-relativistic bosons of mass $m$,
at temperature $T=1/\beta $ close to the critical temperature $T_c$.  When the two-body potential $V_2$ is
short-range  and one is interested only in long wavelength
 phenomena, one can approximate the potential by a $\delta$-function
pseudo-potential,
$$V_2(x-y)=G\, \delta ^{(d)}(x-y), $$
regularized at short distance because such a potential is singular for $d>1$.
\par
 The partition function \sslbl\ssQSDPBdel
$${\cal Z}(\beta ,\mu )=\tr\e^{-\beta (H-\mu N)},$$
where $H$ is the hamiltonian in Fock space, $N$ the
particle number operator (which commutes with $H$), and $\mu$ the chemical potential, can then be expressed as a functional integral
$${\cal Z}=\int[\d\varphi\d\bar\varphi]\exp[-{\cal S}(\varphi,\bar\varphi)],
\eqnd\eZBE $$
where ${\cal S}(\varphi,\bar\varphi)$ is a  non-relativistic {\it local}\/ action:
$$\eqalignno{{\cal S}(\varphi,\bar\varphi)&=\int_0^\beta \d{t}\int\d^d{x}\left[
 \bar\varphi(x,t)\left(-{\partial\over\partial t}-{\hbar^2\over2m}\nabla^2 -\mu
\right)\varphi(x,t) \right. \cr &\quad \left.
+{G\over2} \bigl(\bar\varphi(x,t) \varphi(x,t)\bigr)^2\right],&\eqnd\eactBEdel \cr}$$
and the fields satisfy periodic boundary conditions in the euclidean
time direction,
$$\varphi(x,\beta =1/T)=\varphi(x,0) \quad \bar\varphi(x,\beta =1/T)=\bar\varphi(x,0).$$
The strength $G$ of the interaction must be positive,  corresponding to a repulsive interaction, for the boson system to be stable. \par
 It is customary to parametrize the strength $G$ of the pair
potential in terms of the scattering length $a$. In three
dimensions $a= m G/4\pi\hbar^2$.    Furthermore, we assume that
$a\ll \lambda$, where
$$
  \lambda=\hbar\sqrt{   2\pi / m k_BT}
$$
is the thermal wavelength.  (In the following we set $k_B=\hbar=
1$, and write simply $\lambda^2=2\pi/mT$.) \par Note that, unlike
what would happen for a fluid, in a very dilute system even though
the pair-potential is weak, the $N$-body potentials, $N>2$, are
even much smaller and can thus be totally  neglected.
\par
To compute the effects of the interactions on the transition temperature,
we write the equation of state, the relation between particle number density, temperature and chemical potential. The particle number density can be expressed as a sum of the single particle Green's
function over Matsubara frequencies $\omega_\nu=2\pi \nu T$:
$$
n = {T\over \Omega } {\partial \ln {\cal Z}\over \partial \mu}=
\left<\bar\varphi(x,t) \varphi(x,t)\right>=
T\int\lfrac{\d^d k}{(2\pi)^d} \sum_{\nu\in{\Bbb Z}} \tilde G^{(2)}( \omega_\nu,k)\eqnd{\ematsuz}
$$
($\Omega $ is the space volume).
Above the transition, the single particle  Green's function  can
be parametrized as
$$
\left[\tilde G^{(2)}( \omega ,k)\right]^{-1} =- i\omega  - \mu + \lfrac{k^2}{2m}+ \Sigma( \omega ,k).
\eqnd{\egf}
$$
The Bose--Einstein condensation temperature is determined by the point where $ [\tilde G^{(2)}(0, 0)]^{-1}=0$, that is where $\Sigma(0,0) = \mu$.  At this point,
$$
\left[\tilde G^{(2)}( \omega,k )\right]^{-1}  =-i \omega +\lfrac{k^2}{2m} - \left[\Sigma( \omega,k )-\Sigma(0,0)\right].
\eqnd{\egfi}
$$
At $(T_c,\mu_c)$ the Fourier transform of the two-point correlation
function at zero frequency diverges at zero momentum, and so does the correlation length. \par
In the absence of interactions,
$$
 \sum_{\nu\in{\Bbb Z}} \tilde G^{(2)}(k,\omega_\nu) ={\beta \over
\e^{\beta (k^2/2m-\mu)}-1}, $$
$\mu=0$ at the transition, and
$$ n=\zeta(d/2)/\lambda_c^d\, , \eqnd\eBeTcideal
$$
where
$$\lambda_c^2=2\pi/mT_c^0\,,$$
 and $T_c^0$ is the condensation temperature of the ideal gas.

In the presence of weak interactions, the temperature of the Bose--Einstein
condensation becomes the critical temperature of the interacting model, and is
shifted by the interactions. From the theory of critical phenomena we know
that the variation of the critical temperature in systems with dimension $d$
below four depends primarily on contributions from the small momenta or large
distance (which we refer to as the infrared, or IR) region.  This property,
which we later verify explicitly for $d=3$, simplifies the problem, since to
leading order the IR properties are only sensitive to the $\omega_\nu=0$ component.

It follows from the relations \eqns{\egfi,\eBeTcideal} that at leading order in the dilute limit, where only the
$\omega_\nu=0$ Matsubara frequency contributes, the shift in the transition
temperature at fixed density, $\Delta T_c = T_c - T_c^0$, can be related to
the change $\Delta n$ in the density at fixed $T_c$ by \refs {\club}
$$ \lfrac{\Delta T_c}{T_c} =-\lfrac{2}{d}\lfrac{\Delta n}{n},\eqnd{\edelta}
$$
where
$$
\Delta n={4\pi \over\lambda^2}N_d \int_0^\infty {\rm d}k\,
k^{d-1}\left(\lfrac{1}{k^2+{\cal M}(k)}-\lfrac{1}{k^2}\right),
$$
$N_d$ is the usual loop factor \etadpolexp{b} ($N_3=1/2\pi^2$), and
$$
{\cal M}(k)\equiv 2m\left[\Sigma(k,0)-\Sigma(0,0)\right].
$$

Restricting  the fields $ \varphi$ and $\bar \varphi $ to their zero Matsubara frequency components, corresponds to take a classical field limit.
Equivalently,  the time dependence of the fields  $\varphi(x,t),\bar\varphi(x,t)$
in \eactBEdel\ can be neglected.
It is then
convenient to rescale the field $ \varphi$ in order to introduce  the
field theory normalizations used elsewhere in this review, and to parametrize it in terms of two real fields
$\phi_1,\phi_2$:
$$\varphi(t,x)\sim \sqrt{mT}\bigl(\phi_1(x)+i\phi_2(x)\bigr).$$
  The partition function
then reads
$$
 {\cal Z}= \int \left[ \d \phib (x) \right] \exp \left[-{\cal S}(\phib)\right]
 , \eqnd{\eeONpart}
$$
where now ${\cal S}(\phib)={\cal H}/T$ is given by
%\eqnd{\eeLGWphi}:
$$
{\cal S}   \phib  )= \int \left\lbrace{ 1 \over 2} \left[
\partial_{\mu} \phib (x) \right]^2+{1 \over 2}r
\phib^2 (x)+{u \over 4!}  \left[ \phib^2(x) \right]^2 \right\rbrace \d^{d}x\,,
\eqnd\eeactfivOii
$$
with $r=-2m \mu$, and for $d=3$:
$$
u=96 \pi^2\lfrac{a}{\lambda^2} = 12 m^2 T G.\eqnd\euaGrel
$$
where $G$ is the strength of the pair potential in expression \eactBEdel.  In Eq.~\eeactfivOii\ we have kept the dimension
$d$ of the spatial integration arbitrary in order to use dimensional regularization later. The two-point correlation
function $\tilde G^{(2)}(p)$ is related to the two-point vertex
function $\tilde\Gamma^{(2)}(p)$ of the classical statistical
field theory by
$$
\tilde G^{(2)}( \omega =0,p)=\lfrac{2mT}{ \tilde\Gamma^{(2)}(p)}.
$$

The model described by the euclidean action \eeactfivOii\ reduces to the
ordinary $O(2)$ symmetric $(\phib^2)^2$ field theory, which indeed describes the universal properties of the superfluid Helium transition.  As it stands this
field theory suffers from more UV divergences than the
original theory, the higher frequency modes providing a large momentum
cutoff $\sim \sqrt{m T}\sim 1/\lambda$.  This cutoff may be restored when
needed, for instance, by replacing the propagator by the regularized propagator:
$$
{2mT \over  k^2}\rightarrow {1\over {\rm e}^{k^2/2mT}-1}\,.
$$
In fact, as we show later, since the shift of the critical temperature is
dominated by long distance properties it is independent of the precise
cutoff procedure, that is universal.

    A second effect of the non-zero frequency modes is to renormalize the effective coefficients of the euclidean action.  This problem can be explored
by returning to the functional integral representation \eactBEdel\ of the complete quantum
theory and integrating over the non-zero modes perturbatively.  The
corrections generated are of higher order in $a$ and can thus be neglected.

    Because the interactions are weak, one may imagine calculating the change in the transition temperature by perturbation theory.  However, the
perturbative expansion for a critical theory does not exist for any fixed
dimension $d<4$; IR divergences prevent a straightforward calculation.  If one
introduces an infrared cutoff $k_c$ to regulate the momentum integrals, one
finds that perturbation theory breaks down when $k_c \sim a/\lambda^2$, all
terms being then of the same order of magnitude.  To discuss this problem in
more detail, we now generalize the model to $N$ component fields with an $O(N)$ symmetric hamiltonian.

\subsection{The $N$-vector model. Renormalization group}

We consider the $O(N)$ symmetric generalization \eactON\ of the model
corresponding to the euclidean action \eeactfivOii:
$$
{\cal S} ( \phib )= \int \left\lbrace{ 1 \over 2} \left[
\partial_{\mu} \phib (x) \right]^2+{1 \over 2}r
\phib^2 (x)+{1 \over 4!}{u\over N} \left[ \phib^2(x) \right]^2 \right\rbrace \d^{d}x\,,
\eqnd\eeactfivON
$$
where the field $\phib(x)$  now has $N$ components (note the change in the normalization of the interaction strength $u\mapsto u/N$).
The advantage of this generalization is that it provides
us with a tool, the large $N$ expansion, which allows calculating directly at
the critical point.

The goal is to obtain the leading order non-trivial contribution at
criticality (in the massless theory) to
$$
n=2mT \sum_{i=1}^N  \left<\phi_i^2 \right> \equiv {2mT \,
N\,}\rho
$$
with
$$
\rho= \int\lfrac{\d^d k}{(2\pi)^d}\,\lfrac{1}{\tilde\Gamma^{(2)}(k)}\,.\eqnd{\eeONrho}
$$
Here, $\delta_{ij}/{\tilde\Gamma^{(2)}(k)}$ is the connected two-point correlation
function.

We first recover, by a simple renormalization group analysis, the result of
\refs {\club} that the change in $\rho$ due to the interaction is linear in the
coupling constant.  We introduce a large momentum cutoff $\Lambda
\propto\sqrt{mT}\sim 1/\lambda$, and the dimensionless coupling constant
$$
g=\Lambda^{d-4}{u\over N} \propto \left(a\over
\lambda\right)^{d-2}\,. \eqnd{\ecoupling}
$$
At $T_c$ the inverse two-point function in momentum
space satisfies the renormalization group equation \refs {\rbook}
$$
\left(\Lambda{\partial\over\partial \Lambda}+\beta(g){\partial \over\partial
g}-\eta(g)\right)\tilde\Gamma^{(2)}(p,\Lambda,g)=0\,.
$$
This equation combined with dimensional analysis implies that the two-point
function has the general form
$$
\tilde\Gamma^{(2)}(p,\Lambda,g)=p^2 Z(g)F\bigl(p/\Lambda(g)\bigr).
\eqnd{\ef}
$$
On dimensional grounds $\Lambda(g)$ is proportional to $\Lambda$, with
$$\eqalignno{
\beta(g){\partial \ln Z(g)\over \partial g}&= \eta(g), &
\eqnd\ebeta \cr
\beta(g){\partial \ln \Lambda (g)\over \partial g}&= -1 \,. &
\eqnd\elam \cr}
$$
Since
$$\beta(g)=-(4-d) g +(N+8)g^2/48\pi^2+{\cal O}(g^3),$$
$\beta(g)$ is
of order $g$ for small $g$ in $d<4$; similarly
$$\eta(g)=(N+2)g^2/(72(8\pi^2)^2)+{\cal O}(g^3).$$
  Therefore,
$$
 Z(g)=\exp\int_0^g{\eta(g') \over \beta(g')}\d g'=1+{\cal O}(g^2).$$
Thus to leading order $Z(g)$ =1.  The function $\Lambda(g)$ is then obtained by
integrating Eq.~\elam:
$$
\Lambda(g)=g^{1/(4-d)}\Lambda\exp\left[-\int_0^g\d g'\left({1\over
\beta(g')} +{1\over (4-d)g'}\right)\right].
$$
In the generic situation (like in a fluid) $g=O(1)$ , the scale $\Lambda(g)$ is indistinguishable from
the cut-off scale $\Lambda $. Universal behaviour
exists only for momenta $|p|\ll \Lambda $ where
$$\tilde\Gamma^{(2)}(p)\propto p^{2-\eta} \quad {\rm for}\  p\ll \Lambda(g)=O(\Lambda ) .
\eqnd{\eIRscaling}
$$
However, for $g\ll 1$ one finds
$$\Lambda(g) \sim g^{1/(4-d)}\Lambda \ll \Lambda\,.$$
There exists therefore an intermediate scale $\Lambda^{-1}(g)$ between the IR and the microscopic scales.
The scale $\Lambda(g)$ corresponds to a crossover separating a universal long-distance regime governed by the non-trivial zero $g^*$ of the $\beta$-function, from a
universal short distance regime governed by the gaussian fixed point $g=0$,
where
$$
\tilde\Gamma^{(2)}(p)\propto p^2 \qquad \Lambda(g) \ll p\ll \Lambda\,.
$$
In a generic situation $g$ is
of order unity and the universal
large momentum region is absent. \par
Since $g$  is equal to $a/\lambda$ (see Eq.~\ecoupling) which is $\ll 1$, this
condition is satisfied in the present situation. Moreover, unlike what is seen in ordinary critical phenomena, no $(\phib^2)^3$ term is here present, because this would correspond to three-body interactions, which for such dilute systems are even smaller.
\par
In the language of the running or effective coupling constant, the
situation can be described as follows: because the initial
coupling constant at the microscopic scale is very close to the (unstable) gaussian fixed point value, it first moves very slowly away from the fixed point. For a dilatation $\lambda $ such that $g(\lambda )$
all irrelevant couplings are already negligible, justifying the existence of a universal small distance, large momentum regime,
whose physics here is the physics of the Bose--Einstein condensation.
Then, very rapidly, $g(\lambda )$ moves toward the IR fixed point $g^*$, where the universal IR behaviour of the superfluid transition is seen.

We now show that with this condition, $\Delta T_c\propto \Lambda(g)$.
First, from the $g=0$ limit we infer that $F(\infty)=1$ (see Eq.~\ef.)
The function $F(p)$ behaves for large $p$ as
$$
F(p)=1+ O (p^{2d-8}),
$$
up to $\ln p$ factors, as can be verified directly from perturbation theory ($g=0$ is the UV fixed point).
Therefore, the first correction to the density \eeONrho\ is convergent at
large momentum and independent of the cutoff procedure, that is universal,
$$
\delta\rho= \int\lfrac{\d^d p}{(2\pi)^d} {1\over p^2}\left(
{1\over F(p/\Lambda(g))} -{ 1}\right).
$$
Similarly, the IR scaling result (Eq.~\eIRscaling) implies that this
integral is IR convergent.  Setting $p=\Lambda(g)k$, we then find the general
form
$$\delta\rho=[\Lambda(g)]^{d-2}\int {\d^d k \over(2\pi)^d} {1\over
k^2}\left( {1\over F(k)} - 1\right), \eqnd{\edeltarho} $$
the $g$ dependence is entirely contained in $\Lambda(g)$.  For $g$ small
we conclude
$$\lfrac{\delta\rho}{\rho}\propto[\Lambda (g)/\Lambda ]^{d-2}\propto
g^{(d-2)/(4-d)}. $$
In the physical dimension 3 we recover
$$
\lfrac{\delta\rho}{\rho}\propto g \propto an^{1/3}\,,
$$
in agreement with \refs {\club}.  It is important to note that both the
perturbative large momentum region and the non-perturbative IR region
contribute to the integral in Eq.~\edeltarho.  Therefore, we cannot
evaluate it from a perturbative calculation of the function $F(p)$.  However,
as we now show, we can calculate $\delta\rho$ in the form of an $1/N$ expansion.
\par
Finally, the result for generic dimension $d$ shows that the linear behaviour found for $d=3$, that could lead to believe
that the result is in some way perturbative, is just  accidental.
%%
\subsection{The large $N$ expansion at order $1/N$}

We now use the techniques of large $N$ expansion explained in section \scfivN, where the large $N$ limit is taken
at $Nu$ fixed.  To leading order the critical two-point function has simply
its free field form.  However, a non-trivial correction is generated at order
$1/N$; one finds the inverse two-point function \eONpropi\ \refs {\rFAAAH,\rbook},
$$\tilde
\Gamma^{(2)}(p)= p^2 +{2\over N }\int
 \lfrac{\d^{d}q}{(2\pi)^{d}}{1 \over(6/u)+B_\Lambda (q,0)}\left({1 \over
(p+q)^2} -{1 \over q^2}\right) + O(1/ N^2),
\eqnd\eeONpropi
$$
where $B_\Lambda(p,0) $ is the one-loop function defined by Eq.~\ebullecrit:
$$
B_\Lambda (q,0)=\int^\Lambda  \lfrac{\d^{d}k}{(2\pi)^{d}}\lfrac{1}{k^2(k+q)^2}\mathop{=}_{q\to0}b(d)q^{d-4} -{6\over N g^*}\Lambda^{d-4}
+{\cal O}(1/\Lambda^2),
$$
and the large $N$ value of $g^*$ has been used. We recall (Eq.~\econstb)
$$
b(d)=-{\pi
\over\sin(\pi d/2)}  {\Gamma^2 (d/ 2) \over \Gamma (d-1)} N_d\,.
 $$

For $d=3$ one finds $b(3)=1/ 8$.\par
Note that in the large $N$ limit, the chemical potential $\mu$ is
proportional to $1/N$.

    In the large $N$ limit, the $\beta$-function takes the simple form
$\beta(g)=(d-4) g (1-g/g^*)$.  Therefore, the leading
cutoff-dependent correction to $B_\Lambda(q,0)$ combines with
$6/u$ to yield $(6/u)(1-g/g^*)$,
%$(6/)\Lambda^{d-4}(g)$
as expected from renormalization group arguments. This cut-off
dependent correction, however, can be neglected because $g\ll
g^*$. \par
We evaluate
$$
\delta\rho=-{2\over N }\int\lfrac{\d^d p}{(2\pi)^{2d}}{1 \over p^4}
{\d^{d}q \over(6/u)+b(d)q^{d-4}}\left({1 \over
(p+q)^2} -{1 \over q^2}\right)
\eqnd{\ero}
$$
by keeping the dimension $d$ generic and using dimensional regularization.
The integral over $p$ is
$$\eqalign{
 \int\lfrac{\d^d p}{(2\pi)^d}{1 \over p^4}{1\over(p+q)^2}&=
{1\over(4\pi)^{d/2}}{\Gamma(3-d/2)\Gamma(d/2-1)\Gamma(d/2-2)
\over \Gamma(d-3)}q^{d-6}\,. \cr
&={1\over(4\pi)^{d/2}}{\Gamma(d/2-1)\over\Gamma(d-3)}{\pi\over\sin \pi d/2}\,
q^{d-6}\,.\cr}
$$
Note that the singularity at $d=3$, which would apparently entail the
vanishing of the integral, is cancelled in the subsequent $q$ integral, which
reduces to
$$ \eqalign{
\int\lfrac{\d^{d}q}{(2\pi)^{d}}\, {q^{d-6}
\over(6/u)+b(d)q^{d-4}} &=
{N_d \over 4-d} \quad {\pi\over\sin\left(\pi(d-2)/(4-d)\right)} \cr
&\quad \times b^{(2d-6)/(4-d)}\left(6\over u\right)^{(2-d)/(4-d)}.\cr}
$$
In the $d=3$ limit the two integrations in Eq.~\ero\ yield
$(1/32\pi^2)(u/6)$. As expected, $\delta\rho\propto u$:
$$
\delta\rho=-Ku/N \,,\quad K={1\over96\pi^2} \,,
$$
or in terms of the original parameters  $(u/N)=96\pi^2 a/ \lambda^2
$,
$$\delta\rho=-\lfrac{a}{\lambda^2}. \eqnd{\edelrhoa}
$$
Using this result in Eq.~\edelta, we finally obtain the change in
the transition temperature \refs{\rBBZJ}:
$$\eqalignno{
\lfrac{\Delta T_c}{T_c} &=
  \lfrac{8\pi}{3\zeta(3/2)}\lfrac{a}{\lambda} \cr
&=  \lfrac{8\pi}{3\zeta(3/2)^{4/3}} a n^{1/3}
 = 2.33\, a n^{1/3}. & \eqnd{\eresult} \cr}
$$

    Note that although the final result does not depend on $N$ and, therefore,
replacing $N$ by two is easy, the result is only valid for $N$ large.
%The
%result \eresult~is in remarkable agreement with the ($N=2$) value
%$\Delta T_c/T_c^0 \approx (2.2\pm 0.2) an^{1/3}$ in the recent numerical
% simulations of Holzmann and Krauth \refs {\HoKr}.

\medskip {\it Conclusion.}
The properties of the weakly
interacting Bose gas remain dominated by the UV fixed point of the
renormalization
group equations up to very large length scales; this is why we can still
refer to the
Bose--Einstein condensation when discussing the phase transition of the
dilute interacting
Bose gas.  Renormalization group arguments also  confirm
directly that the
shift of the transition temperature at fixed density is proportional to the
dimensionless
combination
$an^{1/3}$ for weak interactions.  This result is non-perturbative, and the
proportionality
coefficient cannot be obtained from perturbation theory.  However, a non-perturbative method, the large $N$ expansion,  allows a
systematic calculation of this coefficient as a power series in $1/N$, where
eventually one has to set $N=2$.The explicit calculation of  the
leading order contribution has been given in this section.  The first correction is formally of order $1/N$
multiplied by a function of the product $aN$ which is fixed in the large $N$
limit.  Because the final result in three dimensions is linear in $a$, the
$1/N$ factor somewhat surprisingly cancels, and the result is independent of
$N$.  As mentioned earlier the $1/N$ correction has now been calculated
\rPA, and yields a rather low $26\%$ correction for $N=2$.
Moreover, the value found agrees within a factor 2 with the most recent numerical estimates \refs{\rarnold{--}\rKPR}.

%%%%%%%%%   \listrefs
