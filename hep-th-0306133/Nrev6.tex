%%%%%%%%%%  April 25 2003 Minor chnges in location of Fig. 6

\nref\rMosZinn{Parts of this section are based on: \lb M. Moshe
and J. Zinn-Justin, ~{\it Nucl. Phys.} B648 (2003) 131}

\nref\rSUSYN{On supersymmetric $O(N)$ quantum field theory at
large $N$ see also in: \lb T. Suzuki, {\it Phys. Rev.} D32 (1985)
1017 and refs. \refs{\rRydnell,\rSuzuki}.} \nref\rRydnell{ G.
Rydnell and R. Gundsmundottir, {\it Nucl. Phys.} B254 (1985) 593.}
\nref\rSuzuki{T. Suzuki and H. Yamamoto, {\it Prog. Theor. Phys.}
 75  (1986) 126. }


\nref\rBHM { The phase structure of the $O(N)$ symmetric,
supersymmetric model in $d=3$ was studied in \lb W.A. Bardeen, K.
Higashijima and M. Moshe, {\it Nucl. Phys.} B250 (1985) 437. }

\nref\Sal{Studies of  the scalar $O(N)\times O(N)$ model in
dimensions 3 and  $4 -\varepsilon$  are found in: \rf P.Salomonson
and B.S. Skagerstam {\it Phys. Lett.} 155B   (1985) 100 and
ref.~\rRabin.}\nref\rRabin{ E.~Rabinovici, B.~Saering and
W.A.~Bardeen, {\it Phys. Rev.} D36 (1987) 562.}

\nref\rEM{The $O(N)\times O(N)$ symmetric, supersymmetric model in
$d=3$ was studied in \lb O. Eyal and M. Moshe, {\it Phys. Lett. }
B178 (1986) 379. }

\nref\rWittSUSY{ The supersymmetric $O(N)$ non linear $\sigma $
model has been described in: \lb E. Witten, {\it Phys. Rev.} D16
(1977) 2991. }

\nref\rFreed { The study of general two-dimensional supersymmetric
non-linear $\sigma $ models has been initiated in: \lb D. Z.
Freedman, P.K. Townsend,  {\it Nucl. Phys.} B177 (1981) 282 }

\nref\rAlvar{ UV properties were first discussed in: \lb L.
Alvarez-Gaume, D. Z. Freedman, {\it Commun. Math. Phys.} 80 (1981)
443.}

 \nref\rMarcus{ The general four-loop $\beta $-function is given
 in:
 \lb
Marcus T. Grisaru, A.E.M. van de Ven, D. Zanon, {\it Nucl. Phys.}
B277 (1986) 409, {\it Phys. Lett.} B173 (1986) 423.}
\nref\rGrac{Critical exponents  of the supersymmetric non-linear
$\sigma$ model are calculated as $1/N$ expansions in:
% critical exponents nu order $1/N^2$, eta %O(1/N^3)
\rf J.A. Gracey, {\it Nucl. Phys.} B348 (1991) 737; {\it ibidem}\/
B352 (1991) 183; {\it Phys. Lett.} B262 (1991) 49. }
 \nref\rFerre{For other
supersymmetric models see for instance \lb P.M. Ferreira, I. Jack,
D.R.T. Jones, {\it Phys. Lett.} B399 (1997) 258, hep-ph/9702304
and ref.~\rFerre.}\nref\rFerre{
 % different SUSY theories
P.M. Ferreira, I. Jack, D.R.T. Jones, C.G. North, {\it Nucl.
Phys.} B504 (1997) 108, hep-ph/9705328.}

\section Supersymmetric models in the large $N$ limit

We have already discussed scalar field theories and
self-interacting fermions in the large $N$ limit. We want now to
investigate how the results are affected by supersymmetry, and
what new properties emerge in this case~\rMosZinn.\par
Unfortunately, not many supersymmetric models  can be constructed
which can be studied by large $N$ techniques. We consider here two
such models which involve an $N$-component scalar superfield, in
three and two euclidean dimensions. First, we solve at large $N$ a
$( \Phi ^2)^2$ supersymmetric field theory. We then examine the
supersymmetric non-linear $\sigma$ model,  very much as we have
done in the non supersymmetric examples.  \sslbl\ssNSUSYsc \par
Both models are the simplest generalization of supersymmetric
quantum mechanics as it naturally appears, for example, in the
study of stochastic evolution equations of Langevin type (see
section \label{\ssCrDY}). \par Again the main issue will be the
phase structures of these models, and the possibility of
spontaneous symmetry breaking \refs{\rSUSYN}.
%

\subsection Supersymmetric scalar field in three dimensions



Apart from the interest in the general phase structure of the
supersymmetric $(  \Phi ^2)^2$ model, it will be of interest here
to study the spontaneous breaking of scale invariance that occurs
in this model at large $N$ . As we will be seen below, in a
certain region of the parameter space, there is only spontaneous
breaking of scale invariance and no explicit breaking. Though one
may expect this as a result of non-renormalization of the coupling
constant, in fact this happens also in the non-supersymmetric case
(see appendix \label{\ssNscaleinv}). When the coupling constant
that binds bosonic and fermionic $O(N)$ quanta is tuned to a value
at which $O(N)$ singlet massless bound states are created. The
resulting massless Goldstone particles appear as a supersymmetric
multiplet of a dilaton and dilatino.

%Apart from the phase structure, another subject that will be of
%interest here is spontaneous breaking of scale invariance. This
%will be discussed in more details in a scalar theory in section
%\label{\ssNscaleinv}, here it will be noticed that supersymmetry
%brings in extra ingredients. When the coupling constant that binds
%bosonic and fermionic $O(N)$ quanta is tuned to a value at which
%$O(N)$ singlet massless bound states are created, spontaneous
%breaking of scale invariance occurs. The resulting massless
%Goldstone particles appear as a supersymmetric multiplet of a
%dilaton and dilatino. The fingerprints of supersymmetry is the
%positive definite ground state energy ${\cal E} \geq 0$.
%Supersymmetry is left unbroken at $ N \to \infty$ and at a certain
%region of the parameter space there is no explicit breaking of
%scale invariance. Though one may expect this as a result of
%non-renormalization of the coupling constant, in fact this
%happened also in the non-supersymmetric case (see section
%\label{\ssNscaleinv}). Explicit breaking of scale invariance
%appears at the next to leading order in $1/N$ as in the scalar
%case in the $d=3$. In the $ N \to \infty $ limit, flat directions
%in parameter space appear in the supersymmetric theory maintaining
%${\cal E}=0$.

\medskip
\noindent {\it Conventions and notation: Supersymmetry and
Majorana spinors in $d=3$}. Since the properties of Majorana
spinors in three euclidean space dimensions may not be universally
known, we briefly recall some of them and explain our notation. In
three dimensions the spin group is $SU(2)$. Then, a spinor
transforms like
$$\psi_U=U\psi\,, \quad U\in SU(2).$$
The role of Dirac $\gamma$ matrices is played by the Pauli $\sigma$ matrices,
$\gamma_\mu\equiv \sigma_\mu$. Moreover, $\sigma_2$ is antisymmetric while
$\sigma_2\sigma_\mu$ is symmetric. This implies
$$\sigma_2\sigma_\mu\sigma_2=-{}^T\!\sigma_\mu \ \Rightarrow\ U^*=\sigma_2 U
\sigma_2\, .$$ A Majorana spinor corresponds to a neutral
fermion and has only two independent components $\psi_1,\psi_2$.
The conjugated spinor is defined by (${}^T$ means transposed)
$$\bar\psi={}^T\!\psi \sigma_2 \ \Leftrightarrow  \ \bar\psi  _\alpha =i  \epsilon  _{\alpha \beta }\psi _\beta\,,\eqnd\eMajo $$
($\epsilon _{\alpha \beta }=-\epsilon _{\beta \alpha }$, $\epsilon _{12}=1$) and thus $\bar\psi$ transforms like
$$[{}^T\!\psi \sigma_2]_U=[{}^T\!\psi \sigma_2] U^\dagger\,  .$$
In the same way, we define a spinor of Grassmann coordinates
$$\bar\theta={}^T\!\theta \sigma_2 \,.$$
Since the only non-vanishing product is $\theta_1\theta_2$, we have
$$\bar\theta_\alpha\theta_\beta=\ud\delta_{\alpha\beta}
\bar\theta\cdot\theta\,.$$
The scalar product of $\bar\theta$ and $\theta$ is
$$\bar\theta\cdot\theta=-2i\theta_1\theta_2\ \Rightarrow
\theta_\alpha\bar\theta_\beta=i\delta_{\alpha\beta}\theta_1\theta_2
\,.$$
 If $\bar \theta',\theta'$ is another pair of coordinates,
because $\sigma_2\sigma_\mu$ is symmetric,  one finds
$$\bar\theta \sigma_\mu \theta'={}^T\!\theta\sigma_2\sigma_\mu\theta'=
-\bar\theta' \sigma_\mu \theta\,, \eqnd\esigtetii $$
and for the same reason
$$\bar \theta\psi=\bar\psi\theta\,.$$
Other useful identities are
$$(\bar\theta\psi)^2=-\ud (\bar\theta\theta)(\bar\psi\psi),\quad
(\bar\theta\sla{p}\psi)^2=\ud p^2(\bar\psi\psi)(\bar\theta\theta). $$
It is convenient to integrate over $\theta_1,\theta_2$ with the measure
$$\d^2\theta\equiv {i\over2}\d\theta_2\d\theta_1\, .$$
Then,
$$\int\d^2\theta\,\bar\theta_\alpha\theta_\beta=\ud
\delta_{\alpha\beta}\,,\quad
\int\d^2\theta\,\bar\theta\cdot\theta=1\,.$$
With this convention the identity kernel $\delta^2(\theta'-\theta)$ in
$\theta$ space is
$$\delta^2(\theta'-\theta)=(\bar\theta'-\bar\theta)\cdot
(\theta'-\theta).
\eqnd\eIdtheta $$
\medskip
{\it Superfields and covariant derivatives.}
A superfield $\Phi(\theta)$ can be expanded in $\theta $:
$$\Phi(\theta)=\varphi+\bar\theta\psi+\ud\bar\theta
\theta F \,.\eqnd\PhiSupField$$
Again, although only two $\theta$ variables are independent, we define the
covariant derivatives ${\rm D}_\alpha$ and $\bar {\rm D}_\alpha$ ($\bar {\rm D}=\sigma_2 {\rm D}$)
$$ {\rm D}_\alpha \equiv
{\partial\over\partial\bar\theta_\alpha}-(\sla{\partial} \theta)_\alpha \,,\quad
\bar {\rm D}_\alpha \equiv{\partial \over\partial \theta_\alpha}
-(\bar\theta\sla{\partial})_\alpha \,. $$
Then the anticommutation relation is
$$\{{\rm D}_\alpha,\bar {\rm D}_\beta\}=-2
[\sla{\partial}]_{\alpha\beta}\,.$$
Also
$$\bar {\rm D}_\alpha {\rm D}_\alpha={\partial\over\partial\theta_\alpha}
{\partial\over\partial\bar\theta_\alpha}-(\bar\theta\sla{\partial})_\alpha
{\partial\over\partial\bar\theta_\alpha}
-{\partial\over\partial\theta_\alpha}(\sla{\partial}\theta)_\alpha
+\bar\theta_\alpha\theta_\alpha\partial^2\,.$$
%Then
%$$\eqalign{{\rm D}_\alpha \Phi&=\psi_\alpha+\theta_\alpha
%F-(\sla{\partial}\theta)_\alpha \varphi+\ud \bar\theta\cdot\theta
%(\sla{\partial}\psi)_\alpha  \cr
%\bar {\rm D}_\alpha \Phi&=-\bar\psi_\alpha-\bar \theta_\alpha F-
%(\bar\theta\sla{\partial})_\alpha \varphi-\ud (\bar\psi\sla{\partial})_\alpha
%\bar\theta\cdot\theta\,, \cr} $$
Since the $\sigma_\mu$ are traceless, using the identity \esigtetii\ one
verifies that $\bar {\rm D} {\rm D}$ can also be written as
$$\bar {\rm D}_\alpha {\rm D}_\alpha={\partial\over\partial\theta_\alpha}
{\partial\over\partial\bar\theta_\alpha}-2(\bar\theta\sla{\partial})_\alpha
{\partial\over\partial\bar\theta_\alpha}
+\bar\theta_\alpha\theta_\alpha\partial^2\,,$$
and, therefore, in component form
$$\bar {\rm D}_\alpha {\rm D}_\alpha \Phi=2F-2\bar\theta\sla{\partial}\psi
 +\bar\theta\theta \partial^2\varphi\,.$$
Moreover,
$$\left(\bar {\rm D}_\alpha {\rm D}_\alpha\right)^2=4\partial ^2.$$
\medskip
{\it Supersymmetry generators and WT identities.} Supersymmetry is generated
by the operators
$$  Q_\alpha={\partial\over\partial\bar\theta_\alpha}+(\sla{\partial} \theta)_\alpha      \,,\quad \bar Q_\alpha ={\partial \over\partial \theta_\alpha}
+(\bar\theta\sla{\partial})_\alpha \,, $$
which anticommute with
${\rm D}_\alpha$ (and thus $\bar {\rm D}_\alpha$).
Then,
$$\{\bar Q_\alpha, Q_\beta\}=2[\sla{\partial}]_{\alpha\beta}\,.$$
Supersymmetry implies WT identities for correlation functions. The $n$-point
function $ \widetilde W^{(n)}(p_k,\theta_k)$ of Fourier components satisfies
$$Q_\alpha \widetilde W^{(n)}\equiv \left[\sum_k {\partial \over\bar \partial \theta^k_\alpha}
-i( \sla{p_k}\theta^k)_\alpha\right] \widetilde W^{(n)}(p,\theta)=0\,.$$
To solve this equation, we set
$$\widetilde W^{(n)}(p,\theta)=F^{(n)}(p,\theta)\exp\left[-{i\over2n}\sum_{jk}\bar\theta_j
(\sla{p_j}-\sla{p_k})\theta_k\right], \eqnd\eSUSYWTs $$ where
$F^{(n)}$ is a symmetric function in the exchange
$\{p_i,\theta_i\} \leftrightarrow \{p_j,\theta_j\}$. It then
satisfies
$$\sum_k {\partial \over\partial \theta^k_\alpha}
F^{(n)}(p,\theta)=0\,,$$
that is, is translation invariant in $\theta$ space.\par
In the case of the two-point function, this leads to the general form
$$\eqalignno{\widetilde W^{(2)}(p,\theta',\theta)
&=A(p^2)\left[1+C(p^2)\delta^2(\theta'-\theta)\right]
\e^{i \bar \theta\slam{p}\theta'} .&\eqnd\eSUSYiipt \cr
&=A(p^2)\left[1+C(p^2)
( \bar \theta'- \bar \theta) (\theta'-\theta)
+i \bar \theta\sla{p}\theta'-\frac{1}{4}p^2 \bar \theta\theta \bar \theta'\theta'
\right] .\cr }$$
Since the vertex functions $\Gamma^{(n)}$ satisfy the same WT identities, they take the same general form.
\medskip
{\it General $O(N)$ symmetric action.}
We now consider the $O(N)$ invariant action
$${\cal S}(\Phi)=\int\d^3 x\,\d^2\theta\left[\ud\bar {\rm D}\Phi\cdot
{\rm D}\Phi+NU(\Phi^2/N)\right], \eqnd\eSUSYact $$
where $\Phi$ is a $N$-component vector.\par
In component notation
$$\int\d^2\theta\,\bar {\rm D}\Phi {\rm D}\Phi=-\int\d^2\theta\,\Phi \bar {\rm D}_\alpha
{\rm D}_\alpha \Phi=-\bar\psi \sla{\partial}\psi
+(\partial_\mu\varphi)^2-F^2. \eqnd\esusykin $$
Then, since
$$\Phi^2=\varphi^2+2\varphi\bar\theta\psi-\ud(\bar\theta\theta)(\bar\psi\psi)
+F\varphi\bar\theta\theta\,,\eqnd\esuperFiii $$
quite generally
%$$\int\d^2\theta\,\Phi^2=-\ud\bar\psi\psi+F\varphi\,.$$
$$\int\d^2\theta\,{\cal U}(\Phi^2)=
{\cal U}'(\varphi^2)\left(-\ud\bar\psi\psi+F\varphi\right)
-{\cal U}''(\varphi^2)(\bar\psi\varphi)(\varphi\psi). $$
In the case of the free theory $U(R)\equiv \mu R$, the action in component
form is
$${\cal S}=\int\d^3 x\,\left[-\ud \bar\psi\sla{\partial}\psi+\ud
(\partial_\mu\varphi)^2 -\ud F^2+\mu\left(-\ud
\bar\psi\psi+F\varphi\right)\right]. $$
After integration over the auxiliary $F$ field, the action becomes
$${\cal S}=\int\d^3 x\,\left[-\ud \bar\psi\sla{\partial}\psi-\ud\mu
\bar\psi\psi +\ud (\partial_\mu\varphi)^2 +\ud \mu^2\varphi^2\right]. $$
For later purpose it is also convenient to notice that in momentum representation
$$[-\bar {\rm D} {\rm D}+2\mu]\delta^2(\theta'-\theta)= 2\left[-2+ \delta^2(\theta'-\theta)\mu  \right]\e^{-i\bar\theta\slam{k}\theta'} .\eqnd\eSUSYker $$
The free propagator can be written as
$$[- \bar {\rm D} {\rm D}+ 2\mu]^{-1}= {1\over4( k^2+\mu^2)}\left(  \bar {\rm D} {\rm D}+ 2\mu\right),$$
or more explicitly
$$[-  \bar {\rm D} {\rm D}+2 \mu]^{-1}\delta^2(\theta'-\theta)
= {1\over  k^2+\mu^2 }\left[1+\ud \mu \delta^2(\theta'-\theta)\right]
\e^{-i\bar\theta\slam{k}\theta'} .\eqnn $$
For a generic super-potential $U( R )$, we find
$$\eqalignno{ {\cal S}&=\int\d^3 x\,\left[-\ud \bar\psi\sla{\partial}\psi
+\ud (\partial_\mu\varphi)^2-\ud  U'(\varphi^2/N)\bar\psi\psi
-U''(\varphi^2/N)(\bar\psi\varphi)(\varphi\psi)/N \right.\cr&\quad\left.+\ud \varphi^2
U'{}^2(\varphi^2/N)\right].&\eqnd\eGenericAction \cr}$$ Note that the theory
violates parity symmetry. Actually, a space reflection is
equivalent to the change $U\mapsto -U$ (see definition \echirpar).
Therefore, theories with $\pm U$ have the same physical properties.\par
Finally, in the calculations that follow we assume,
when necessary, a supersymmetric Pauli--Villars regularization.
%
\subsection Large $N$ limit: superfield formulation

To study the large $N$ limit, we introduce a constraint on $\Phi^2/N=R$,
where $R$ now is a superfield, by integrating over another superfield $L$:
\sslbl \ssNSUSYsad
$${\cal Z}=\int[\d\Phi][\d R][\d L]\e^{-{\cal S}(\Phi,R,L)}, $$
where
$${\cal S}(\Phi,R,L)=\int\d^3x\,
\d^2\theta \left\{\ud\bar {\rm D}\Phi\cdot
{\rm D}\Phi+NU(R)  +  L(\theta)\left[\Phi^2(\theta)-NR(\theta)\right]\right\}.
\eqnd\eactLagm $$
We parameterize the scalar superfields $L$ and $R$ as
$$\eqalignno{L(\theta,x)&=M+\bar\theta\ell+\ud\bar\theta\theta \lambda\,,&\eqnd\eLsupField \cr
 R(\theta,x)&=\rho+\bar\theta\sigma+\ud\bar\theta\theta s\,.&\eqnd\eRsupField\cr}$$
The $\Phi$ integral is now gaussian. As usual we integrate over only
$N-1$ components and keep a test-component $\Phi_1\equiv \phi$. We find
$${\cal Z}=\int[\d\phi][\d R][\d L]\e^{-{\cal
S}_N(\phi,R,L)} \eqnn $$
with the large $N$ action
$$\eqalignno{{\cal S}_N&=\int\d^3 x\,\d^2\theta\left[\ud\bar {\rm D}\phi
{\rm D}\phi+NU(R)+ L \left(\phi^2-NR\right)\right] \cr
&\quad+\ud(N-1){\rm Str}\, \ln\left[-\bar {\rm D} {\rm D}+2L\right].&\eqnd\eactSUSYN
\cr}$$
The two first saddle point equations, obtained by varying $\phi$ and $R$,
are
\eqna\eSUSYsad
$$\eqalignno{2L\phi-\bar {\rm D} {\rm D}\phi&=0\,,&\eSUSYsad{a} \cr
L-U'(R)&=0\,.&\eSUSYsad{b} \cr} $$ The last saddle point
equation, obtained by varying $L$, involves the
super-propagator $\Delta $ of the $\phi$-field. For $N \gg 1$, it reads
$$\eqalignno{R-\phi^2/N& = \left<x,\theta \right| \left[-\bar {\rm D} {\rm D}+2L\right]^{-1}\left|x,\theta \right>=
{1\over (2\pi)^3} \int\d^3 k \,\Delta (  k,\theta ,\theta ).
&\eSUSYsad{c} \cr}$$
The super-propagator $\Delta$  is solution of the equation
$$\left(- \bar {\rm D} {\rm D}+2L(\theta)\right)\Delta(k,\theta,\theta')=
\delta^2(\theta'-\theta) . $$
It is here needed only for $\ell=0$ and $M,\lambda$ constants.
It can be obtained by solving
$$\left(- \bar {\rm D} {\rm D}+2L(\theta)\right)\Phi(\theta)=J(\theta). $$
In Fourier representation and
in terms of its components, the equation reads
$$2M\varphi-2F+2\bar\theta(-i\sla{k}+M)\psi+\bar\theta\theta
[(k^2+\lambda)\varphi+MF] =J(\theta). $$
>From its solution we
infer the form of the propagator
$$\Delta(k,\theta,\theta')={\left[1+\ud
M(\bar\theta\theta+\bar\theta'\theta') -\frac{1}{4}(\lambda+k^2)
\bar\theta\theta\bar\theta'\theta'\right]\over k^2+M^2+\lambda}
-{\bar\theta[i\sla{k}+M]\theta'\over k^2+M^2} \,.\eqnd\esupprop $$
Clearly, one reads in Eq.~\esupprop\ the $\langle\varphi (k)\varphi
(-k) \rangle $ propagator $(k^2+M^2+\lambda)^{-1}$ and the
$\langle\psibar (k)\psi (-k)\rangle$ propagator
$(i\sla{k}+M)/(k^2+M^2)$. The coefficients of
$(\bar\theta\theta+\bar\theta'\theta')$ and of $
(\bar\theta\theta\bar\theta'\theta')$ are the $\langle\varphi
(k)F(-k)\rangle$  and $\langle F(k) F(-k)\rangle $ propagators,
respectively. \par
At coinciding $\theta$ arguments, we obtain
$$\Delta(k,\theta,\theta)={1+ M\bar\theta\theta\over
k^2+M^2+\lambda}-{M \bar\theta \theta\over k^2+M^2}
\,. \eqnd\eSUSYpropcoin $$
Eq.~\eSUSYsad{c}  thus is
$$ R-\phi^2/N = {1\over (2\pi)^3} \int\d^3 k\left[{1+ M\bar\theta\theta \over
k^2+M^2+\lambda}-{M\bar\theta\theta\over k^2+M^2}\right].
 \eqnd\eSUSYsadc  $$
\medskip
{\it Saddle point equations in component form.}
It is now convenient to introduce a notation for the boson mass
$$m_\varphi\equiv m=\sqrt{M^2+\lambda }\,. \eqnn $$
Eq.~\eSUSYsad{a} implies
\eqna\eSUSYsada
$$\eqalignno{F-M\varphi&=0 \,, &\eSUSYsada {a}  \cr
\lambda\varphi+MF&=0\,. &  \eSUSYsada{b}      \cr}$$
Eliminating $F$ between the two equations, we find
$$\varphi m^2=0 \eqnd\eSUSYGold $$
and, thus,
 if the $O(N)$ symmetry is broken the boson mass $m_\varphi$ vanishes.\par
Then, Eq.~\eSUSYsad{b} yields
\eqna\eSUSYsadb
$$\eqalignno{M&=U'(\rho),  & \eSUSYsadb{a} \cr
\lambda=m^2-M^2&=sU''(\rho).&\eSUSYsadb{b} \cr} $$
%$$\int\d^2\theta\,L R=\ud M s+\ud \rho \lambda-\ud \bar\sigma \ell\,.$$
%and we can substitute, generating a $\ud M^2\varphi^2$ contribution..
%Since at the saddle point only commuting variables do not vanish, we do not
%need at this stage the spinor $\ell$.
Using Eq.~\esuperFiii, we write Eq.~\eSUSYsadc\ in component form
(a cut-off is implied) as \eqna\eSUSYii
$$\eqalignno{\rho-\varphi^2/N&={1\over(2\pi)^3}\int{\d^3 p\over p^2+ m^2}\,,
&\eSUSYii{a}\cr
s-2F\varphi/N &={2M\over(2\pi)^3}\int\d^3 p\left({1\over p^2+m^2}
-{1\over p^2+M^2}\right). &\eSUSYii{b}\cr} $$
Introducing the cut-off dependent constant
$$\rho_c={1\over(2\pi)^3}\int^\Lambda {\d^3 p\over p^2}=\Omega _3(0)\,,\eqnd\eRciiidef $$
(see Eqs.~\eqns{\etadepole{--}\eadef}) we rewrite these equations as
\eqna\eSUSYiv
$$\eqalignno{\rho-\varphi^2/N&=\rho_c-{1\over4\pi } m\,, &\eSUSYiv{a}\cr
s-2F\varphi/N &={1\over2\pi}M\left(|M|- m\right).
&\eSUSYiv{b}\cr} $$
Note that the change $U\mapsto -U$ here corresponds to
$$F\mapsto -F\,,\quad s\mapsto -s\,,\quad M\mapsto -M\,.$$
\medskip
{\it Action density.} Finally, we calculate the action  density
$\cal E$ corresponding to the action ${\cal S}_N$
(Eq.~\eactSUSYN), ${\cal E}={\cal S}_N/{\rm volume}$,  for
vanishing fermion fields. We use (see also Eq.~\esusykin)
$$ {1\over2} \int\d^2\theta\,\bar {\rm D}\phi {\rm D}\phi =  -{1\over2} F^2 \,,\quad \int\d^2\theta\,L\phi^2 =M F\varphi  +{1\over2}\lambda \varphi^2 \eqnn $$
and
$${\rm Str}\, \ln\left(-\bar {\rm D} {\rm D}+2L\right)=  \tr\ln(-\partial^2
+M^2+\lambda) - \tr\ln(\sla{\partial}+M).\eqnd\eNSUSYtr $$
Then,
$$\eqalignno{{\cal E}/N&=-\ud F^2/N+\ud s U'(\rho)+MF\varphi/N+\ud \lambda\varphi^2/N-\ud
Ms -\ud \lambda \rho \cr&\quad +\half\tr\ln(-\partial^2 +M^2+\lambda)
- \half\tr\ln(\sla{\partial}+M).&\eqnd\eSUSYV
%\cr
}$$
In $d=3$, the $\tr\ln$ in Eq.~\eSUSYV\ is given by
$$    \half\tr\ln(-\partial^2
+M^2+\lambda) - \half \tr\ln(\sla{\partial}+M)  = \ud
\rho_c\lambda-{1\over12\pi}\left(m^3-|M|^3\right) .
\eqnd\eSupertraceLN  $$ The saddle point equations in component
form are then recovered from  derivatives of $\cal E$ with respect
to the various parameters. Using the saddle point equations
\eSUSYsad{a}, \eSUSYiv{} and \eSUSYsadb{a} to eliminate
$F,s,\rho$, one eliminates also the explicit dependence on the
super-potential $U$ and the expression is simplified into
$${\cal E}/N=\ud M^2\varphi^2/N+  {1\over 24\pi}(m-|M|)^2(m+2|M|)
.\eqnd\eSUSYground
$$
In this form we see that $\cal E$ is positive for all saddle
points, and, as a function of $m$, has an absolute minimum at
$m=|M|$, and thus $\lambda =0$, that is for a supersymmetric
ground state. \par Moreover, Eq.~\eSUSYGold\ implies $M\varphi=0$,
and thus ${\cal E}=0$. Therefore, if a supersymmetric solution
exists, it will have the lowest possible ground state energy  and
any non-supersymmetric solution will have a higher energy. \par
Since $M$, $m$, and $\varphi$ are related by the saddle point
equations, it remains to verify whether such a solution indeed
exists.\par In the supersymmetric situation the saddle point
equations reduce to $s=0$, $M\varphi=0$ and
$$\rho-\rho_c=\varphi^2/N-|M|/4\pi\,,\quad M=U'(\rho).$$
In the $O(N)$ symmetric phase $\varphi=0$ and $|M|=4\pi
(\rho_c-\rho)=|U'(\rho)|$. In the broken phase $U'(\rho)=0$ and
$\varphi^2/N=\rho-\rho_c$. We will show in the next section that
these conditions can be realized by a quadratic function $U(\rho)$
and then in both phases the ground state is supersymmetric and
$\cal E$   vanishes.
%
\subsection Variational calculations

For completeness, we present here the corresponding variational
calculations \rBHM\ and apply the arguments of section
\ssNVarfiv~to the action \eGenericAction. In terms of the two
parameters  \sslbl\ssSUSYNvar \eqna\eR
$$\eqalignno{\rho&=(1/N)\left<\varphi^2(x)\right>_0
={ \varphi^2 \over N} +{1\over(2\pi)^3}\int{\d^3 p\over
p^2+m_\varphi^2} ,&\eR{a}\cr \tilde \rho &=(1/N)\left< \bar\psi
(x)\psi (x)\right>_0 ={2m_\psi \over(2\pi)^3}\int{\d^3 p\over
p^2+m_\psi ^2 }\,, &\eR{b}\cr} $$
the variational energy density
can be written as
$$\eqalignno{{\cal E}_{\rm var}/N&=\ud \tilde \rho
\bigl(m_\psi -U'(\rho)\bigr)+\ud \rho U'{}^2(\rho)-\ud
m_\varphi^2(\rho-\varphi^2/N) \cr &\quad  +\half\tr\ln(-\partial^2
+ m_\varphi^2)- \half\tr\ln(\sla{\partial}+m_\psi
).&\eqnd\eVarEsusy }$$ In the following we choose $m_\psi \ge 0$
and $m_\varphi \ge 0$. \par
Since the variational energy is larger than or equal to the ground state energy, which in a supersymmetric theory is non-negative, it is sufficient to find values
of the three parameters $m_\psi,m_\varphi ,\varphi$ such that ${\cal E}_{\rm var}$ vanishes to prove that the ground state is supersymmetric.
Choosing $m_\psi=m_\varphi=M$ and using Eqs.~\eR{}, we find
$${\cal E}_{\rm var}/N=  U'{}^2(\rho ){\varphi^2\over 2N}+{1\over2(2\pi)^3}\int{\d^3 p\over
p^2+M^2}\left[U'(\rho )-M\right]^2. $$
This expression vanishes whenever $M=U'(\rho )$ together with $M\varphi=0$.
Together with Eq.~\eR{a}, we recover the three supersymmetric saddle point equations we discussed at the end
of section \ssNSUSYsad.
\par
A surprising  feature of the variational
energy density  is the appearance of a divergent contribution
$${\cal E}_{\rm var}/N=\ud \rho_c\bigl(M -U'(\rho)\bigr)^2+\ {\rm finite}\, .$$
Therefore, when the  equation $M =U'(\rho)$ is not
enforced, the variational energy is infinite with the cut-off. \par
Of course, we can also look for the minimum of ${\cal E}_{\rm var}$ by differentiating with respect to the three parameters.
Differentiating ${\cal E}_{\rm var}$ with
respect to $m_\psi $ and using the definition of $\tilde \rho$, we
obtain the fermion mass gap equation
$$m_\psi =U'(\rho).\eqnd\eFerMassGap $$
If this equation is taken into account the variational energy
density simplifies to
$$  {1\over N} {\cal E}_{\rm var}(\ma,\mpsi,\varphi ) = {\mpsi^2\over 2}
{\varphi^2 \over N} + {1\over 24\pi}(\ma-\mpsi)^2(\ma+2\mpsi).
 \eqnd\HartreeVTzeroC $$
 which is positive definite (recall that $m_\psi \ge 0$
and $m_\varphi \ge 0$) and vanishes when $m_\psi = m_\varphi$ and
$\mpsi\varphi = 0$, that is for a supersymmetric ground state with the
$O(N)$ symmetry either broken or unbroken. \par When the positive
fermion and the boson masses $m_\psi $, $m_\varphi$ are identified
with their value in terms of the parameters $M,m$,
$$m_\psi=|M| \quad {\rm and}
\quad m_\varphi=\sqrt{M^2+\lambda},\eqnd\FandBmasses$$ one
recognizes the expression \eSUSYground.  As we have shown this
expression has a unique  minimum $\ma=\mpsi$, which is
supersymmetric. Differentiating then with respect to $ \varphi$,
we find $\varphi m_\psi=0$, one solution $m_\psi=0$ corresponding
to $O(N)$ symmetry breaking, the other $\varphi=0$ to an $O(N)$
symmetric phase.
%\par The three parameters $m_\varphi$, $m_\psi $
%and $\varphi$  are, in fact, related and  it remains to verify in
%this formalism also that the supersymmetric minimum exists.

Differentiating Eq.~\eVarEsusy\ with respect to $m_ \varphi$ and
using the definition of $  \rho$, we obtain the boson mass gap
equation
$$m_\varphi^2=2\rho U'(\rho)U''(\rho)+U'{}^2(\rho)-\tilde \rho U''(\rho) \eqnd\eBosMassGap $$
or, using the value of $m_\psi $,
$$m_\varphi^2-m_\psi ^2=U''(\rho)\left(2m_\psi \rho-\tilde \rho\right).$$
Clearly,   the supersymmetric solution satisfies both gap
equations. In the combination $2m_\psi \rho-\tilde \rho$, we
recognize the parameter $s$ as in Eq.~\eSUSYii{b} ($F$ being taken
from Eq.~\eSUSYsada{a}), and thus the equation becomes
Eq.~\eSUSYsadb{b}:  $\lambda =sU''(\rho)$.\par

Eqs.~\eqns{\eFerMassGap,\eBosMassGap}  have a clear
Schwinger--Dyson  diagrammatic interpretation for a $U(\rho)=\mu
\rho + \half u \rho^2 $ potential (see Eq.~\eGenericAction).
Namely, \eqna\eGaps
$$\eqalignno{m_\psi &=\mu + {u\over N}\left<\varphi^2(x)\right>_0\,,
&\eGaps{a}\cr m_\varphi^2 &= \mu^2 + 4 {\mu u \over
N}\left<\varphi^2(x)\right>_0 + 3 {u^2\over
N^2}\left<\varphi^2(x)\right>_0^2 - {u \over N }\left< \bar\psi
(x)\psi (x)\right>_0 . \cr & &\eGaps{b}  }
$$


%\*


%
\subsection The $\Phi^4$ super-potential in $d=3$: phase structure

We now consider the special example
$$U(R)=\mu R+\ud uR^2\ \Rightarrow\ U'(R)=\mu+ uR\,.$$
The dimensions of the $\theta$ variables and the field $\Phi$ are
$$[\theta]=-\ud\,,\quad[\Phi]=\ud\ \Rightarrow \ [u]=0\,.$$
Power counting thus tells us that the model is renormalizable in
three dimensions. Prior to a more refined analysis, one expects
coupling constant and field renormalizations (with logarithmic
divergences) and a mass renormalization with linear divergences.
Using the solution \eSUSYiipt\ for the two-point function
$\Gamma^{(2)}$, one infers that the coefficient $A(p^2)$ has at
most a logarithmic divergence, which corresponds to the field
renormalization, while the coefficient $C(p^2)$ can have a linear
divergence which corresponds to the mass renormalization. \par
The invariance of physics under the change $U\mapsto -U$ was mentioned above and
seen in Eq.~\eGenericAction\ and the equations that followed. This
invariance will be reflected into the phase structure of the
model. For the quartic potential the equations \eSUSYsadb{} now
are
$$M=\mu+u \rho\,,\quad \lambda=us\,.\eqnd\eSUSYmpot $$
We introduce the critical value  of $\mu$,
$$\mu_c=-u \rho_c  \, . \eqnn $$
Taking into account equations  \eSUSYmpot, one finds that the equations \eSUSYiv{} can now be written as
\eqna\eSUSYivquartic
$$\eqalignno{ M&=\mu-\mu_c + u\varphi^2 /N- {u \over{4\pi}}
\sqrt{M^2+\lambda}\,,&\eSUSYivquartic{a}\cr \lambda &=2u M\varphi^2/N +
{u \over{2\pi}}M\left( |M|-\sqrt{M^2+\lambda} \right)\,.
&\eSUSYivquartic{b}\cr}$$
 Eqs.~\eSUSYivquartic{} relate the fermion
mass $m_\psi=|M|$, the boson mass $m_\varphi = \sqrt{M^2+\lambda}$
and the classical field $\varphi $. The phase structure of the
model is then described by the lowest energy solutions of these
equations in the $\{ \mu-\mu_c , u \}$ plane. Taking into account
the $U \to -U$ symmetry, mentioned above, one can restrict the discussion to $u>0$.
\par
 % and thus it is enough to find the
%solutions in the first and second quadrant in the $\{ \mu-\mu_c ,
%u \}$ plane.
We find, indeed, that supersymmetry is left unbroken
($\lambda = 0$) and the ground state energy ${\cal E}=0$ in each
quadrant in the $\{ \mu-\mu_c , u \}$ plane. This is consistent with Eqs.~\eSUSYivquartic{}
having a common solution with $\lambda =0$ (thus $m_\psi= m_\varphi = |M| $) and $M\varphi=0$.
They then reduce to
$$ M =\mu-\mu_c + u{\varphi^2 \over N} - {u \over{4\pi}}|M|\,,
\quad M\varphi  =0\, . \eqnd\eSUSYiii $$
\medskip
{\it The broken $O(N)$ symmetry phase.} The  $M=0$ solution
implies a spontaneously broken $O(N)$ symmetry, scalar and fermion
$O(N)$ quanta are massless and
$$\varphi^2 =-N(\mu-\mu_c)/u\,, \eqnd\ephi$$
which implies that this solution exists only for $ \mu<\mu_c$. The
solution exists in the fourth (and second) quadrant of the
$\{ \mu-\mu_c , u \}$ plane. Note that this yields the same  exponent $\beta=\ud $
as in the simple $(\varphib^2)^2$ field theory  (Eq.~\expbeta).
\smallskip
{\it The $O(N)$ symmetric phase.} We now choose the solution $\varphi=0$
of Eqs.~\eSUSYiii. Then the equation
$$M=\mu-\mu_c - (u/u_c)|M| \eqnd\eSymmetricPhase$$
yields the common mass $M$ for the fermions and bosons. In
Eq.~\eSymmetricPhase\ we have introduced the special value of the coupling $u$,
$$u_c= 4\pi  \ . \eqnd\egCritical$$
This equation splits into two equations, depending on the sign of $M$.
%$$\cases{\displaystyle M=M_+={\mu-\mu_c \over u/u_c+1} & for $M>0$, \cr
%\displaystyle |M|=-M_-={\mu-\mu_c \over u/u_c-1 } & for $M<0$  .\cr} $$
The first solution
$$M= M_+=(\mu-\mu_c)/(1+u/u_c) >  0\eqnd\eSUSYfivMscal $$
exists only for $\mu>\mu_c$ (first quadrant), as one would normally expect.
Note again that this corresponds to a correlation length exponent $\nu=1$, as in the ordinary $(\varphib^2)^2$ field theory (Eq.~\ecorlenb), though the form of the saddle point equations are different: in the supersymmetric theory this is the free field value.
Moreover, the exponent is independent of $u$, though the term proportional to $u$ is not negligible.  If one takes into account the leading correction coming from regularization (expansion \etadepolii), one finds
$$(1+u/u_c) M = \mu-\mu_c +u a(3)M^2/\Lambda \,.$$
Unlike what happens in the usual $\varphi^4$ field theory, no value of
$u$ can cancel the leading correction to the relation
\eSUSYfivMscal, and therefore no IR fixed point can be identified.
\par
The second solution
$$M=
%m_\varphi=m_\psi=m_-
M_-=(\mu-\mu_c)/(1-u/u_c) < 0 $$
is very peculiar. There are two different situations depending on the position of $u$ with respect to $u_c$: \par
(i) $u>u_c=4\pi$ and then $\mu>\mu_c$:  the solution is degenerate with another $O(N)$ symmetric solution $M_+$.\par
(ii) $u<u_c$ and then $\mu<\mu_c$: the solution is degenerate with a solution of broken $O(N)$ symmetry. \par
\midinsert \epsfxsize=13.5cm \epsfysize=11cm \pspoints=2.pt \vskip
-5.2cm \hskip2cm \epsfbox{phases.eps}
%%%%%%%%%%%%%%%%%%%%%%%%%%%% FIG 7 %%%%%%%%%%%%%
\vskip -5.8cm
%%%%%%%%%%\vskip -5.8cm
%\vskip -5.3 cm
%%%%%%%%%%%%%%%%%%%%%%%%%
\hskip 5.cm $\varphi^2 = 0$ \vskip -.2cm \hskip 1.5cm
$~~~~\mu-\mu_c$ \vskip .2cm \hskip 3.5cm {\bf I} \hskip 3cm {\bf
II} \vskip .4cm \hskip 7.2cm $M_-$ \vskip .3cm \hskip 3.5cm $M_+$
\hskip 3cm $M_+$
%%%%%%%%%%%%%%%%%%%%%%%%%%
%\vskip -.2cm
\vskip 0.2cm
%%%%%%%%%\vskip 0.2cm
%\vskip 1.5 cm
%%%%%%%%%%%%%%%%%%%%%%%%%%%%%%%%%
 \hskip 8.5cm $u/u_c$ \vskip .3cm \hskip 3.5cm $M_-$
\vskip .3cm \hskip 3.5cm {\bf IV} \hskip 3cm {\bf III} \vskip 1cm
\hskip 5cm $\varphi^2 \neq 0$ \vskip 5mm \figure{1mm}{Summary of
the phases of the model in the $ \{ \mu-\mu_c , u \} $ plane.
%at $ T = 0$.
Here   $\ma=\mpsi=|M_\pm|=(\mu-\mu_c)/(u/u_c\pm 1)$. The
lines $u=u_c$ and $\mu-\mu_c=0$ are lines of first and second
order phase transitions.}
\figlbl\phases
\endinsert
 The phase structure is summarized
 in Fig. \label{\phases} in the first and
 second quadrant of the $ \{ \mu-\mu_c, u/u_c \} $
 plane where the following different phases appear:
\smallskip
{\bf Region I : ~~$\mu-\mu_c \geq 0 ~~,~~ {u/ u_c}\leq 1$}:

\noindent Here, there is only one $O(N)$ symmetric, supersymmetric
ground state with $\mpsi=\ma=M_+=(\mu-\mu_c)/(u/u_c+1)$ and
$\varphi^2=0$.
\smallskip

{\bf Region II : ~~$\mu-\mu_c \geq 0 ~~,~~ {u/ u_c}\geq 1$}:

\noindent There are two degenerate
$O(N)$ symmetric ($\varphi=0$) supersymmetric ground states with
masses $\mpsi=\ma=M_+=(\mu-\mu_c)/(u/u_c+1)$ and
$\mpsi=\ma=-M_-=(\mu-\mu_c)/(u/u_c-1)$.

\smallskip

{\bf Region III : ~~$\mu-\mu_c \leq 0 ~~,~~ {u/ u_c}\geq 1$}:

\noindent There is one supersymmetric ground state, it is an
ordered state with broken $O(N)$ symmetry ($\varphi^2 \neq 0$,
$\mpsi=\ma=0$).
\smallskip

{\bf Region IV : ~~$\mu-\mu_c \leq 0 ~~,~~ {u/ u_c}\leq 1$}:

\noindent There are two degenerate ground states: an $O(N)$
symmetric , supersymmetric ground states with masses
$\mpsi=\ma=m_-=(\mu-\mu_c)/(u/u_c-1)$ and $\varphi^2=0$. The
second ground state is a supersymmetric, broken $O(N)$ symmetry
state with $\mpsi=\ma=0$ and $\varphi^2 \neq 0$.


\medskip

{\it The action density.} To exhibit the phase structure in terms
of the variation of the  action density ${\cal E}$, we plot the
expression \eSUSYground, but use only the fermion gap equation in
Eq.~\eSUSYivquartic{a}, in such a way that $\cal E$ remains a
function of $\varphi$ and $\lambda $, or equivalently $\varphi$
and $m=\sqrt{M^2+\lambda }$:
$$  {1\over N} {\cal E}(m,\varphi)
 = \half M^2(m,\varphi ) {\varphi^2 \over N}
 +{1\over 24\pi} \left[m -\left|M(m,\varphi )\right|\right]^2
  \times \left(\ma + 2\left|M(m,\varphi)\right| \right) .
 \eqnd\HartreeVTzeroE $$
Two figures display the restriction of $\cal E$ to $\varphi=0$:
%
$$  {1\over N} {\cal E}(m,\varphi = 0)
=  {1\over 24\pi} \left[m - \left| \mu -\mu_c - (u/u_c)
m \right| \right]^2   \left( m +2\left| \mu -\mu_c
-(u/u_c)m \right| \right).   \eqnd\HartreeVTzeroF  $$
%%%%%%%%%%%%%%%   FIG 7   %%%%%%%%%%%%%%%%%%%
\midinsert \epsfxsize=17.5cm \epsfysize=14cm \pspoints=2.pt \vskip
-7cm \hskip 1cm \epsfbox{figt2a.eps} \figure{1mm}{{\bf Region II}
of Fig.~\phases: The energy density $W(m=\sqrt{M^2+\lambda
},\varphi ) \equiv {1\over N} {\cal E}(m,\varphi) $ as a function
of the boson mass ($m$) and $A$, where $A^2=\varphi^2/u_c$. Two
degenerate, $O(N)$ symmetric phases exist with  massive bosons
(and massive fermions). Here $\mu-\mu_c=1 $ (sets the mass scale)
\ $ u/u_c=1.5 $.} \figlbl\figTIIaaa
\endinsert

\midinsert \epsfxsize=13.5cm \epsfysize=10cm \pspoints=2.pt \vskip
-5cm \hskip 2cm \epsfbox{zerot1.eps} \figure{1mm}{The energy
density $W(m) \equiv {1\over N} {\cal E}(m,\varphi=0) $ from
Eq.~\HartreeVTzeroF\
%$ T = 0$ and
%$\varphi=0$
in region {\bf II}
of Fig.~\phases. Here $\mu-\mu_c=1 $ (sets the mass scale) and
$u/u_c$= varies between 1.6 and 1.2. There are two degenerate
$O(N) $ symmetric SUSY vacua with $\varphi=0 $ at
$m_\psi=m_\varphi=M_+=(\mu-\mu_c)/(u/u_c+1)$ and at
$m_\psi=m_\varphi=-M_-=(\mu-\mu_c)/(u/u_c-1)$. } \figlbl\ZeroTone
\endinsert


%\vskip .5cm
\midinsert \epsfxsize=19cm \epsfysize=15cm \pspoints=2.pt \vskip
-7.5cm \hskip 1cm \epsfbox{figtivaa.eps} \figure{1mm}{{\bf
Region IV :} The  energy density $W(m,\varphi )={1\over N} {\cal
E}(m ,\varphi )$ as given in Eq.~\HartreeVTzeroE\ as a function of
the boson mass ($m$) and $A$, where $A^2=\varphi^2/u_c$. Here
$\mu-\mu_c=-1 ~,~ u/u_c=0.2 $. As seen here  there are two
distinct degenerate phases. One is an ordered phase ($\varphi \neq
0$) with a massless boson and fermion, the other is a symmetric
phase ($\varphi = 0$) with a massive ($m=|M_-|$) boson and
fermion.
 } \figlbl\figTIVaa
%\vskip 1cm
\endinsert

%%%%%%%%%%%%% inserted from 8.3  %%%%%%%%%

 %(this seems to be
%more practical since in the finite energy case we will avoid the
%need to solve a non-linear equation)
%

Several peculiar phase transitions can be easily traced now in
Eq.~\HartreeVTzeroE{}. First, one notes the phase transitions that
occur  when $\mu-\mu_c$ changes sign. When  $ 0 < u <u_c  $
and $\mu > \mu_c$, the system has a non-degenerate $O(N)$ symmetric
ground state with bosons and fermions of mass $M=M_+$. As $\mu -
\mu_c$ changes sign ($0 < u <u_c$ fixed), two degenerate ground
states appear. Either $M=0$ and $\varphi^2=-N(\mu-\mu_c)/u$ or the
system stays in an $O(N)$ symmetric ground state with a mass $|M_-|$
for the bosons and fermions. Similarly, when one goes from $\mu <
\mu_c$ to $\mu > \mu_c$ at $ u>u_c= 4\pi  $ the $O(N)$ symmetry is restored but there are two
degenerated ground states to choose from  $M=M_\pm$.


%%%%%%%%%%%%%%%%%  FIG. 2 in  %%%%%%%%%%%%%%%%%%%%
%${\cal E}(M)$ in Fig.~2 below.
%\bigskip
%\midinsert
%\epsfxsize=2.in
%\epsfysize=2.5in
%\pspoints=1.pt
%\epsfverbosetrue
%\epsffilein=\read0
%\vskip -1.cm
%\epsftsize=10in
%\epsfrsize=10in
%\epsftmp=10in
%\centerline{\epsfbox{figN2.eps} }
%\vskip -3cm
%\epsfbox[160 400.  20. 182]{figN2.eps}
%\figure{1mm}{The ground state energy ${\cal E}(M)$ for fixed $\mu = 0.1$ and
%different $Ng/ 4\pi$ values. Note the mass ratio
%$|M_-/ M_+| \to \infty$ as $Ng/4\pi \to 1$.
%(From right to left: $  Ng/ 4\pi =1.3,\ 1.1,\ 1.05,\ 1.03 $) }
%\figlbl\figNNii
%\endinsert
%\bigskip
%%%%%%%%%%%%%%%%%%%%%%%  FIG. 2 %%%%%%%%%%%%%%%%%%%%%
%\vskip -2.5cm
 In Fig.~\label{\figTIIaaa} and
 Fig.~\label{\ZeroTone}, $W(m=\sqrt{M^2+\lambda },\varphi) \equiv {1\over N}
{\cal E}(m, \varphi)$ from Eq.~\HartreeVTzeroF\ in region {\bf II}
($\mu-\mu_c \geq 0 ~,~ u/u_c \geq 1$) is plotted as a function of
$m$ and $\varphi$.  An unusual transition takes place when one
varies the coupling constant $u$. The transition from the
degenerate vacua at $u/u_c =1.4$ to a non-degenerate  ground state
at $u/u_c= 0.8$ is shown in Fig.~\label{\ZeroTtwo}  (from phase
{\bf II} to phase {\bf I}).


\midinsert \vskip -4.5cm \epsfxsize=15.5cm \epsfysize=10cm
\pspoints=2.pt \hskip 1.7cm \epsfbox{zerot2.eps} \figure{1mm}{The
energy density $W(m)={1\over N} {\cal E}(m,\varphi=0)$ as given in
Eq.~\HartreeVTzeroF.
%$ T = 0$ and
Here $\mu - \mu_c=1 $ and $u/u_c = 1.2$ is changed to $u/u_c =
0.8$ (from region {\bf II} to region {\bf I} of Fig.~\phases).
There are two degenerate $O(N)$ symmetric SUSY vacua   at $u/u_c=
1.2$ with masses  $m_\psi=m_\varphi=|M_\pm|$ where
$|M_\pm|=\mu/(u/u_c \pm 1)$ while at $u/u_c= 0.8$ there is a
non-degenerate vacuum at $m_\psi=m_\varphi=M_+$.} \figlbl\ZeroTtwo

\endinsert

For positive $\mu-\mu_c$, we find two degenerate ground states if
$ u  > u_c $. As $u$ is lowered (at fixed $\mu-\mu_c$), the ground
state with mass $\ma=\mpsi=-M_-$ disappears ($|M_-/ M_+| \to
\infty$) and only the $O(N)$ symmetric phase remains with $M=M_+$
(Figs.~\label{\ZeroTone}  and \label{\ZeroTtwo}). Namely, suppose
we consider at $\{ u>u_c$, $\mu-\mu_c>0\}$ a physical system in a
state denoted by $A$ and defined by $\{\varphi^2=0, M=M_-\}$, such
a system will go into a state $B$ defined by $\{\varphi^2=0,
M=M_+\}$ when $u$ {\bf decreases} and passes the value $u=u_c=
4\pi $. Now, if we consider the reversed process; a physical
system at $\{ u < u_c , \mu-\mu_c> 0 \}$ that is initially in the
ground state $B$ and $u$ {\bf increases }and
 passes  $u=u_c$. There is no reason now for the
system to go through  the reversed transition from $B$ to $A$
since the $O(N)$ symmetric states with $M=M_-$ and $M=M_+$ are
degenerate and the fact that supersymmetry is preserved will avoid
that the energy of state $A$ will go below zero. These peculiar
phase transitions with the  ($|M_-/ M_+| \to \infty$) and with
``infinite hysteresis" in the $A \to B$ transitions are due to the
fact that supersymmetry is left unbroken in the leading order in
$1/N$. If supersymmetry would have been broken by some small
parameter, the lifted degeneracy would be, most probably,
translated into a slow transitions between the otherwise
degenerate ground states.

\smallskip




In section \label{\ssNSUSYT} we will study the transitions between
the different phases of ~Fig.~\phases~ as a function of the
temperature .

\medskip
{\it Special situation.} In general when $\mu=\mu_c$, the mass $M$ vanishes.
However, there is a special case when
$$u=u_c=4\pi  \,. $$
Then the value of $M$ is left undetermined. An accumulation point
of coexisting degenerate ground states exist in the phase
structure shown in Fig.~\phases. The case $\mu=\mu_c$ represents a
scale invariant theory where, however, the $O(N)$ fermionic and
bosonic quanta can have a non-vanishing mass $\ma=\mpsi=|M|$.
Since $u$ has not undergone any perturbative renormalization there is
no explicit breaking of scale invariance at this point. Thus, the
only scale invariance breaking comes from the solution of the gap
equation for $M$, which leaves, however, its numerical value
undetermined. We see a dimensional transmutation from the
dimensionless coupling $u$ that is fixed at a value of $u=u_c$
into an undetermined scale $M$. If $ M \neq 0$ the spontaneous
breaking of scale invariance will require the appearance of a
Goldstone boson at the point $ u= u_c$. The massless Goldstone
boson is associated here with the spontaneous breaking of scale
invariance. Moreover, since the ground state is supersymmetric, we
expect the appearance of a massless $O(N)$ singlet Goldstone boson
(a dilaton) and its massless fermionic partner (a ``dilatino'').
In order to see all this, we now calculate the $\langle LL \rangle$
propagator that will enable us to see these poles in the
appropriate four-point functions.\par
Finally, note that this analysis is only valid in the complete absence of cut-off corrections.
%Otherwise one finds for $u=u_c$ a change in scaling behaviour
%$$M=\sqrt{(\mu_c-\mu)\Lambda /ua(3)}, $$
%provided the argument of the square root is positive, and the
%value $u_c$ appears as a fixed point value.
Otherwise if
$\mu=\mu_c$ and $u$ is tuned as $u=u_c - M_0/\Lambda$, where $M_0$
is an arbitrary mass scale, one finds for the $M_-$ solution:
$$M=- {M_0 \over a(3)u_c^2}\,.$$
\vskip .5cm
\subsection The $\langle LL\rangle$ propagator and massless
bound states of  fermions and bosons


{\it The $\langle LL \rangle$ propagator.}
We now calculate the $\langle LL\rangle$
propagator in the
symmetric phase. Then, the propagators of the fields $\phi$ and $L$ are
decoupled. In the quartic potential example the $R$ field can be
eliminated by gaussian integration. The relevant part of the $L$-action then
reads
$$-{N\over 2u}\int\d^3 x\,\d^2\theta(L-\mu)^2+{1\over2} (N-1) {\rm Str}\,
\ln\left(-\bar {\rm D} {\rm D}+2L\right) . $$
The calculation of the $\langle LL \rangle$ propagator  involves the
super-propagator \esupprop. For the inverse propagator  one
finds
$$\Delta_L^{-1}(p)=-{N\over u}\delta^2(\theta'-\theta)
-2N\int{\d^3 k\over(2\pi)^3}
\Delta(k,\theta,\theta')\Delta(p-k,\theta,\theta')  $$
with here (see Eq.~\eSUSYiipt)
$$\Delta(k,\theta,\theta')={1\over k^2+M^2}\left[1+\ud
M\delta^2(\theta'-\theta) \right]\e^{-i \bar\theta \slam{k}\theta'}\,. $$
Then,
$$\Delta(k,\theta,\theta')\Delta(p-k,\theta,\theta')
={[1+M \delta^2(\theta'-\theta)]\e^{-i \bar\theta \slam{p}\theta'}
\over (k^2+M^2)[(p+k)^2+M^2]}\,. $$
Notice the cancellation of the factor $\e^{-i \bar\theta \slam{k}\theta'}$
which renders the integral more convergent that one could naively expect.
The integral over $k$ then yields the three-dimensional bubble diagram
$$\eqalign{B(p)&={1\over(2\pi)^3}\int{\d^3 k\over  (k^2+M^2)[(p+k)^2+M^2]}\cr
&={1\over4\pi p}{\rm Arctan}(p/2|M|)\,. \cr}
$$
At leading order for $p$ small, we need only $B(0)=1/8\pi |M|$. Then,
$$\Delta_L^{-1}=
-{N\over4\pi |M|}\left[1+(M +4\pi |M|/u )
\delta^2(\theta'-\theta)\right]\e^{-i \bar\theta \slam{p}\theta'}.
$$
Comparing with the expression \eSUSYker, we conclude that for
$M>0$ small the $LL$ propagator corresponds to a super-particle of
mass $2M (1+4\pi/u ) $. For $M<0$ the mass is $2|M(1-4\pi/u)|$.
For $|u-u_c|$ small, it is a bound state and at the special point
$u=u_c$ the mass vanishes.\par
More generally, we find
$$\Delta_L(p)={2\over NB(p)}
{1\over p^2+m^2(p)}\left[1-\ud m(p)\delta^2(\theta'-\theta)\right]
\e^{-i \bar\theta \slam{p}\theta'} \eqnd\eLLprop$$
with
$$m(p)=2M+{1\over  u B(p)}\,.$$
We note that only a mass renormalization is required at leading order, a situation similar to the $\varphi^4$ scalar field theory.
As a consequence, dimensions of fields are not modified.\par

Clearly, the propagation of  the fields $M(x)$  and $\ell(x)$  (of
Eq.~\eLsupField), as indicated in Eq.~\eLLprop, when combined with
the $ L(\theta)\Phi^2$ interaction in Eq.~\eactLagm, namely,
$$  \int\d^2\theta\,L\Phi^2 =M(-\half\bar\psi\psi+F\varphi )
-\varphi\bar\ell \psi+\half\lambda \varphi^2 ,$$
describes the
bound states in the $\varphi\varphi$, $\psi\psi$ and $\psi\varphi$
scattering amplitudes. For example, in the supersymmetric ground
state case and with $\mu-\mu_c=0 $, the $\psi\varphi$ scattering
amplitude $T_{\psi\varphi,\psi\varphi}(p^2)$, in the limit $p^2
\to 0$ satisfies
$$T_{\psi\varphi,\psi\varphi}(p^2) \sim {2u\over
N}\left[1+ {u\over 4\pi}{M\over |M|} +{u\over 2\pi}{i\sla{p} \over
|M|} \right]^{-1}  \to -{4\pi i\over N}{|M|\over \sla{p}} \eqnn $$
for  $M<0$ and $u \to u_c$

One notes here that the fermionic massless bound state pole
appears when a non-zero solution  ($M$) to the gap equation exists
$ (m_\varphi=m_\psi=|M|) $ in the absence of any dimensional
parameters ($\mu-\mu_c=0$).  This happens  when the force between
the massive $\psi$ and $ \varphi$ quanta is determined by $u \to
u_c$. The massless $O(N)$ singlet fermionic bound state excitation
is associated with the spontaneous breaking of scale invariance.
Similarly, the bosonic partner of this massless bound state
excitation can be then seen, at the same value of the parameters,
in $\varphi\varphi$ and $\psi\psi$ scattering amplitudes as
  Eq.~\eLLprop~shows.
\par
At $\mu=\mu_c$ in the generic situation $M=0$, or for
$|p|\to\infty$, we find
$$B(p)={1\over 8p} $$
and, thus,
$$\Delta^{-1}_L(p)=-{N\over u}\delta^2(\theta'-\theta)
-{N\over4 p}\e^{-i \bar\theta \slam{p}\theta'}. $$
As a consequence,
the canonical dimension of the field $L$ is 1, as in
perturbation theory, and the interaction $L\Phi^2$ in Eq.~\eactLagm\
is renormalizable. The $L^2$ term thus is not negligible in the IR limit.
Nevertheless,  as we have noticed, the behaviour of physical quantities
does not depend on $u$. Therefore, we expect that, at least for generic values of $u$, we can describe the same physics with the supersymmetric non-linear $\sigma $-model.
\medskip
{\it $O(N)\times O(N)$ symmetric models.} The dynamics by which
scale invariance can be broken in a theory which has no trace
anomalies in perturbation theory has been studied also in
$O(N)\times O(N)$ symmetric models \refs{\Sal,\rEM}.
%\refs{\Em}
%\refs{\BRS}.
Here one finds that
spontaneous breaking of scale invariance is due to the breakdown
of the internal $O(N)\times O(N)$ symmetry or to non-perturbative mass
generation on a critical surface. The mass of the fermion
 and boson $O(N)$ quanta is arbitrary due to
the appearance of flat directions in the action density. Also the ratios
between the scales associated with breaking of the internal
$O(N)\times O(N)$ symmetry and scale
symmetries are arbitrary on the critical surface.
Massless bound states of fermions and bosons appear
due to the spontaneous breaking of scale invariance.
Here, again, in the large $N$ limit there is no explicit breaking of scale
invariance and the perturbative $\beta$ function vanishes.
The variational ground state energy, which is calculable in the
large $N$ limit, has flat directions which allow spontaneous breaking of
scale invariance by non-perturbative generation of mass scales.
This results in lines of fixed points in the
coupling constant plane which are
associated with dynamical scale symmetry
breaking. The interplay between internal symmetry, scale symmetry
and supersymmetry is reflected in a rich phase structure.
The novel issue in these models are phases in which
the breaking of scale invariance is directly related
to the breaking of the internal symmetry
$O(N)\times O(N) \rightarrow  O(N-1)\times O(N-1)$.

For the $O(N) \times O(N)$ symmetric potential $NU(\Phi _1^2/N ,\Phi_2^2/N ) $ where
%40
$$ U( R _1,R _2) =  \mu  ( R _1+R _2)
 +\ud u   ( R _1 ^2 +  R _2^2 )
 + v R _1R _2\,,   \eqnn $$
one again finds in $d=3$, in the leading order in $1/N$, that  only a mass renormalization is needed. Correspondingly, the critical value of the parameter
$\mu$ now is $\mu_c=  -2 (u + v) \rho_c $.
  Here too, the interesting case is
$\mu = \mu_c$. As in the $O(N)$ symmetric model, one finds that
supersymmetry is left unbroken to leading order in $1/N$. The gap
equations are now
%41
$$\left.\eqalign { m_1\left({{u}\over{4\pi}} + \sgn(\mpone) \right) + m_2\left({v\over{4\pi}}\right) &= u \varphi_1 ^2/N +v \varphi_2^2 /N   \,, \cr
m_1\left({v\over{4\pi}}\right) + m_2\left({{u }\over{4\pi}} +
\sgn(\mptwo)\right ) &= v{\varphi_1 }^2/N+ u{\varphi_2 }^2 /N\,,\cr}\right.\eqnn$$
where non-zero solutions are obtained on lines $v=v(u)$,
which is the condition for spontaneous breaking of scale
invariance. As expected, this is also the condition for the
appearance of massless dilaton and ``dilatino" in the spectrum.
One notes that scale invariance can be broken here as a
consequence of the breakdown of the internal $O(N) \times O(N)$
symmetry. Indeed, the massless dilaton and ``dilatino" appear
either as bound states of massive ``pions" and their supersymmetric
partners or, when the internal symmetry is broken, they are mixed
with the ``sigma" boson and fermion particles.
%\refs{\Em} .
Thus, the non-zero scale that is responsible for the spontaneous
breaking of scale invariance may be set here also by $
\varphi^2~\neq~0$ rather than by $\mpsi=\ma~\neq~0$ only, as was
the case in the $O(N)$ symmetric model. On the lines $v = v(u)$,
one finds that $\cal E$ has flat directions in all dimensional
parameters.

\subsection  Dimensions $2\leq d \leq 3$

In $d$ space--time dimensions, $2\leq d \leq 3$, one can keep in the space of $\gamma $ matrices $\tr{\bf 1}=2$. From the point of view of power counting in perturbation theory the model is super-renormalizable since the dimension of $u$ now is $3-d$.\par
The calculations follow the same pattern discussed above.
Eqs.~\eSUSYii{} are replaced by (with the definition \etadepole)
\eqna\esadNSUSYnls
$$\eqalignno{\rho-\varphi^2/N&=\Omega _d(m), &\esadNSUSYnls{a}\cr
s-2F\varphi/N&=2M\left[\Omega _d(m)-\Omega _d(|M|)\right]. \cr}$$
Taking into account the other saddle point equations, and $\rho_c=\Omega _d(0)$,
one obtains a generalization of Eq.~\eSUSYground:
%\refs{\MT}
$$2 {\cal E}/N =  M^2\varphi^2/N+(M^2-m^2)\left[ \Omega _d(m)-\Omega _d(0)\right] +\int_0^\lambda \d\lambda ' \,
 \Omega _d\left(\sqrt{M^2+\lambda '}\right)  \eqnd \eDminEi $$
and, thus, from Eqs.~\etadepolii\ and \etadpolexp{a}:
$$ {\cal E}/N =\half M^2\varphi^2/N +K(d)\left[ \ud (m^2-M^2)m^{d-2}
-\left(m^d-|M|^{d }\right)/d\right],$$
 where
$$K(d)= -{ \Gamma(1-d/2)\over (4\pi)^{d/2}}.$$
Moreover, from Eq.~\eDminEi,
$${\partial {\cal E}\over\partial m }=- {N\over 2m}  (m^2-M^2)
 \Omega _d'(m),\eqnd \eDminE $$
which has the sign of $m-|M|$ because $\Omega _d'(m)$ is negative. The function, therefore, has a minimum  at $m=|M|$ for all values of $d$, and supersymmetry is maintained in the ground state for all $2\leq d \leq3$.\par
The critical exponents now are $\eta =0$, $\beta=1/2$, and for the mass in the unbroken phase one obtains
$$M=\mu -\mu _c-K(d)u |M|^{d-2}+O(M^2\Lambda ^{d-4}).$$
We see that the l.h.s.~is now negligible, the equation having a solution only for $(\mu -\mu _c)/u>0$. The non-perturbative value  $\nu =1/(d-2)$ of the correlation exponent follows.
One also verifies that the dimension of $L$ remains 1, and thus the $L^2$ term now is negligible in the IR limit, leading immediately to the non-linear $\sigma $-model. However, again one finds no IR fixed point, no value of $u$ cancels
the leading correction to scaling, and therefore the argument leading to the non-linear $\sigma $-model is not as solid as for the usual $(\varphi^2)^2$ field theory.
\medskip
{\it Two dimensions.}
It may be interesting to consider the same model in two
dimensions. The model now is super-renormalizable, but even at $\mu=0$ it is
not chiral invariant since a chiral transformation corresponds to a change
$U\mapsto -U$.\par
The expression \eDminE\  is cut-off independent and has a limit for $d=2$:
 $${\cal E}/N=\ud M^2\varphi^2/N +{1\over8\pi}
\left[m^2-M^2-2M^2\ln(m/M)\right]  ,\eqnd\eSUSYNEnii $$
an expression which again has an absolute minimum at $m=|M|$. Then,
a minimization with respect to $M$ and $\varphi$ yields $M\varphi=0$. At the
minimum $\cal E$ vanishes. \par
Taking into account $\lambda =0$ and Eq.~\esadNSUSYnls{a} in the $d=2$ limit, we obtain the gap equation
$$M=\mu+u\varphi^2/N+{u\over 2\pi}\left[\ln(\Lambda /|M|]+O(1)\right). $$
% $$\rho_c={N\over(2\pi)^2}\int^\Lambda{\d^2 p\over
%p^2+M_0^2}\sim {N\over2\pi}\ln(\Lambda/M_0) \,.\eqnd\eRciidef $$
%The $\lambda$ saddle point equation now reads
%$$\rho-\rho_c=\varphi^2-{N\over4\pi}\left[\ln(M^2+\lambda)-2\ln M_0+1\right]. $$
The solution $M^2=\lambda=0$, thus, is not
acceptable. The $O(N)$ symmetry
is never broken.
%Defining $\mu_c=-u \rho_c$, we obtain the mass $M$ as
%solution to the equation
%$$M=\mu-\mu_c-{u\over4\pi}(2\ln M/M_0+1). $$
When the mass $M$ is small in the cut-off scale, the l.h.s.~is negligible
and
$$m=M\propto  \Lambda  \e^{2\pi\mu /u}. $$
This solution exists only when $-\mu/u$ is positive and large. Its behaviour
suggests immediately a relation with the non-linear $\sigma $-model, which we now briefly examine.

\subsection A supersymmetric non-linear $\sigma $ model at large $N$

We consider the supersymmetric non-linear $\sigma $ model in $d$
dimensions, $2\le d\le 3$. The action \refs{\rWittSUSY{--}\rFerre}
\sslbl\ssSUSYnls
%\eqna\eNLSM
$${\cal S}(\Phi)={1\over 2 \kappa }\int\d^d x\,\d^2\theta\, \bar {\rm D}\Phi\cdot
{\rm D}\Phi   \eqnd\eNLSM $$
involves an $N$-component   scalar superfield
$\Phi$,  which satisfies
$$\Phi\cdot\Phi = N \,.$$
This relation is implemented  by introducing  a superfield
$$L (x,\theta) = M(x) + \bar\theta\ell(x) +\half \bar\theta\theta
\lambda(x) ,$$
where $M(x)$, $\lambda(x)$ and $\ell(x)$ are the Lagrange multiplier fields,
and adding to the action
$${\cal S}_L ={1\over  \kappa }\int\d^d x\,\d^2\theta \, L (x,\theta)
\left[\Phi^2(x,\theta)-N \right] . \eqnd\eNLSML $$
The partition function is given by (${\cal S}(\Phi,L)={\cal S}+{\cal S}_L$):
$${\cal Z} = \int [d\Phi][\d L] \e^{ -  {\cal S  }(\Phi,L) }.$$
In terms of component fields, the total action reads
$$\eqalignno{ {\cal S}={1\over  \kappa }\int \d^dx \{ &\half { \varphi}(-\del^2 +
M^2){ \varphi}
-\half{ {\bar\psi}}(\delslash+M){ \psi} \cr
&+\ud \lambda
({ \varphi}^2- N  )
-\bar\ell({ {\bar\psi}}\cdot \varphi) \}.
&\eqnd\NLSMcomp }$$
As in the case of the $\Phi^4$ theory, we integrate out $N-1$ components
leaving out $\phi=\Phi_1$:
$${\cal Z} = \int [d\phi][\d L] \e^{ - {\cal S}_N(\phi,L) }$$
where
$$   {\cal S}_N(\phi,L)= {1\over\kappa } \int\d^d x\,\d^2\theta  \left[ \half\bar {\rm D}\phi\cdot
{\rm D}\phi+  L \left(\phi^2- N \right) \right]    + {{N-1}\over 2}
{\rm Str}\, \ln( -{\bar {\rm D}}{\rm D} +2 L )  .   \eqnd\eNLSMeff   $$
By varying the effective action with respect to $\phi$, one finds the
saddle point equation
$${\bar {\rm D}}{\rm D}\phi-2L \phi =0\,, \eqnd\eSadNLSMa $$
which implies for constant $\varphi$ and $\psi=0$:
$$ F-M\varphi=0 \qquad {\rm and} \qquad MF+\lambda\varphi = 0 \,.
\eqnd\eSadNLSMaa$$
When the large $N$ action is varied with respect to $L $ and using the
expression in \esupprop~for the $\phi$ propagator, one finds ($N\gg 1$)
$${1\over \kappa}-{\phi^2 \over N}
=  \int{\d^d k\over (2\pi)^d}\left[{1+ M\bar\theta\theta \over
k^2+M^2+\lambda}-{M\bar\theta\theta\over k^2+M^2}\right].
\eqnd\eSadNLSMb$$
We now introduce the boson mass parameter $m=\sqrt{M^2+\lambda }$.
 Eq.~\eSadNLSMb~in  component form (for  $\psi=0$) then reads
\eqna\eSadNLSMbb
$$\eqalignno{1-{\varphi^2\over N}&=
\kappa \,\Omega _d (m ),
&\eSadNLSMbb{a}\cr
M{\varphi^2\over N\kappa } &=M \left[\Omega _d(|M|)- \Omega _d\left( m\right)
\right]. &\eSadNLSMbb{b}\cr} $$
\medskip
{\it Dimension $d=3$.} Introducing the critical (cut-off dependent) value
 $$ \kappa_c = 1/\Omega _3(0)\,,$$
we can write
% $\kappa_\r$, defined by
%$${1\over \kappa_\r}={1\over \kappa}
%-{1\over(2\pi)^3}\int{\d^3 k\over k^2} \eqnd\ekappaR$$
Eqs.~\eSadNLSMbb{} now as
\eqna\eSadNLSMbbb
$$\eqalignno{{1\over \kappa }-{1\over \kappa_c }-{\varphi^2\over N\kappa } &=-
{m\over4\pi } &\eSadNLSMbbb{a}\cr
M{\varphi^2\over N\kappa } &={1\over4\pi}M\left(m-|M|\right).
&\eSadNLSMbbb{b}\cr} $$
The calculation of the ground state energy density
of the non-linear $\sigma $ model follows  similar steps
as in the $(\Phi^2)^2$ model:
$$  {\cal E} /N={ 1\over 2N\kappa } m^2\varphi^2  - \ud(m^2-M^2)\left( {1\over\kappa } -{1\over\kappa_c}\right)
-{1\over12\pi}\left(m^3-|M|^3\right)
   \eqnd\eNLSMenergy $$
and, taking into account the saddle point equation \eSadNLSMbb{a},
 $$ {\cal E} /N={ 1\over 2N\kappa } M^2\varphi^2 +  {1\over 24\pi}(m-|M|)^2(m+2|M|)
,\eqnd\eSUSYnlsground
$$
an expression identical to \eSUSYground, up to the normalization of $\varphi$.
Again, if a supersymmetric solution exists it has the lowest energy.
We thus look for supersymmetric solutions.\par
In the $O(N)$ symmetric phase ($\varphi = 0$)
Eq.~\eSadNLSMbbb{b} is satisfied while
Eq.~\eSadNLSMbbb{a} yields
$$m=|M|=4\pi\left({1\over \kappa _c}-{1\over \kappa }\right).$$
This phase exists for   $\kappa \ge \kappa _c$.\par
In the broken phase $m=M=0$, and Eq.~\eSadNLSMbbb{b} is again satisfied.
Eq.~\eSadNLSMbbb{a} then  yields
$$\varphi^2/N=1-\kappa /\kappa _c \,, \eqnn $$
which is the solution for $  \kappa \le \kappa _c$.\par
Since we have found solutions for all values of $\kappa $, we conclude that
the ground state is always supersymmetric.
At $\kappa _c$, a  transition occurs between a massless phase with broken $O(N)$ symmetry  ($\varphi^2 \neq 0$)  and a symmetric phase with massive fermions and bosons of equal mass.
The supersymmetric non-linear $\sigma $ model and the usual non-linear $\sigma $-model studied in section \ssLTsN, thus, have analogous phase structures, which are less surprising than the structure of $(\Phi^2)^2$ field theory. \par
\medskip
{\it Renormalization group.} From the point of view of power counting, the model is analogous to its non-supersymmetric version (section \ssLTsN), and RG equations take the same form with two RG functions $\beta (\kappa )$ and $\zeta (\kappa )$. Both have been calculated to four loop order.\par
At leading  order for $N\to\infty $, one finds the same critical exponents as in the supersymmetric $(\Phi^2)^2$ field theory or the usual non-linear $\sigma $ model: $\eta=0$ and, in generic dimension $d$: $\nu=1/(d-2)$.
The $\beta $-function has at this order the same form as in the non-supersymmetric model:
$$\beta (\kappa )=(d-2)\kappa (1-\kappa /\kappa _c)+O(1/N).$$
In the supersymmetric model the critical exponents $\nu$ and $\eta$ have been calculated
to order $1/N^2$ and $1/N^3$, respectively. To order $1/N$ one finds
$$\eta= {X_1(d)\over N}+O(1/N^2)\,, \quad \nu={1\over d-2}+O(1/N^2),$$
where $X_1(d)$ is the constant \eXone, $X_1(3)=8/\pi^2$.
\medskip
{\it Dimension $d=2$.}   The phase structure of the supersymmetric
non-linear $\sigma$ model in two dimensions is rather simple, and again analogous to the structure of the usual non-linear $\sigma $-model.
Eq.~\eSadNLSMbbb{a} immediately implies that $M^2+\lambda =0$ is not a solution
and thus the $O(N)$ symmetry remains unbroken $(\varphi^2=0$)
for all values of the coupling constant $\kappa $.
Then, with a suitable normalization of the cut-off $\Lambda $,
$$m=\Lambda \e^{-2\pi/\kappa }. \eqnd\enlsSUSYNii $$
Correspondingly, the action density becomes
$$  {\cal E} /N=   -  {\lambda\over2\kappa}
 -{1\over8\pi}\left[(M^2+\lambda)\ln(M^2+\lambda)
-\lambda\ln  \Lambda ^2
-M^2\ln M^2 -\lambda\right]  \eqnd\eNLSMenergyiicr $$
and, taking into account Eq.~\enlsSUSYNii,
$${\cal E} /N={1\over 8\pi}\left[m^2-M^2+2M^2\ln(M/m)\right].$$
We recognize the expression \eSUSYNEnii, and conclude in the same way
 that a  supersymmetric solution has the lowest energy. Supersymmetry remains unbroken, and the common boson and fermion mass is
$$M=m=\Lambda \e^{-2\pi/\kappa }. \eqnn $$
The model, thus, is UV asymptotically free and the condition $M\ll\Lambda $ implies that non-trivial physics is concentrated
near $\kappa =0$.
%In two dimensions the  ground state energy density
%of the non-linear $\sigma $ model is:
%$$\eqalignno{{\cal E}/N
% &= {1\over {\rm volume} }{\cal S}_{\rm eff}  \cr
%&=\left({M^2+\lambda\over 2}\right){\varphi^2 \over N\kappa }
%-\half\lambda{N\over {\kappa_\r}}
%\cr &\quad -{1\over8\pi}\left[(M^2+\lambda)\ln(M^2+\lambda)
%-\lambda\ln M_0^2
%-M^2\ln M^2 -\lambda\right] , &\eqnd\eNLSMenergyiicr}$$
%or after eliminating $\lambda$ in the symmetric phase
%(there $\lambda=M_0^2\e^{-{4\pi/ k_\r}} -m^2$)
%$$\eqalignno{{\cal E} /N&=
%- {1\over 2\kappa_\r}(M_0^2\e^{-{4\pi/ {\kappa_\r}}}-M^2)
%-{1\over8\pi}
%\left[
%\biggclb M_0^2\e^{-{4\pi/\kappa_\r}}\ln(M_0^2\e^{ -{1/\kappa_\r}})
%\cr  &\quad
%-(M_0^2\e^{-{4\pi/ k_\r}}-M^2 )\ln (M_0^2\e)  -M^2\ln M^2\biggcrb
%\right] .
% &\eqnd\eNLSMenergyiii
%\cr}$$
