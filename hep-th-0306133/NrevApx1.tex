%%%%%%%%%%  MM some minor corrections on June 10
%%%%%%  JZJ April 1, reference and a few typos corrected

\section Spontaneous breaking of scale invariance,
non-trivial fixed points

The subject of scale invariance breaking has a long history in
particle physics. Under very general conditions,
 a $d$-dimensional field theory that is scale
 invariant, is also invariant under the transformations
 of the  conformal group $SO(d+1,1)$ (for $d>2$).
%group 15 dimensional Lie algebra of .
 The scale and conformal current, $S_\mu$ and $K_{\mu \nu}$
are then constructed ~\refs{\rEMTen} from the ``improved"  energy
momentum tensor $\tilde T_{\mu \nu}$. The divergences of these
currents are proportional to the trace of the energy momentum
tensor ${\tilde T_\mu}^\mu$. They are: \sslbl\ssNscaleinv
$$ S^\mu~=~x_\lambda \tilde T^{\mu \lambda}~~~~~~
{\rm and} ~~~~~~K^{\mu \nu} ~=~x^2 \tilde T^{\mu \nu}~-~ 2 x^\mu
x^\lambda {\tilde T_\lambda}^\nu. \eqnd\scaleCurrent $$ There are
two reasonably well understood mechanisms for breaking these
symmetries in a theory which is scale and conformal invariant at
tree level (see, for example, Ref.~\refs{\rAdler}): \par

 {\bf (a)}  Spontaneous breaking of scale invariance of the Nambu--Jona-Lasinio type encountered in the BCS theory of
superconductivity.\par {\bf (b)} Explicit breaking  of scale
invariance, which is expressed at the quantum level by the anomaly
in the trace of the energy momentum tensor, as the result of
radiative corrections. \par In conventional quantum field theories
the two mechanisms occur simultaneously and the spontaneous
breaking of scale symmetry is not normally accompanied by the
appearance of a massless Nambu--Goldstone boson. Thus, for
example, we do not find a massless dilaton in QCD. An interesting
possibility exists, however, in case of theories with an hierarchy
of scales in which a light dilaton will appear as a
pseudo-Goldstone boson associated with the spontaneous breaking of
scale symmetry.
%\refs{\BLL}
 \sslbl\SpnSclBrkg

Finite theories like $N=2$ and $N=4$ supersymmetric Yang--Mills
theories are especially interesting. In these theories there is no
trace anomaly
%\refs{\YM}
and thus mechanism {\bf b} mentioned above does not apply; but the
alternative dynamical generation of mass scales in these cases is
unclear.
%In fact, the accompanied breaking of supersymmetry,
%which one demands at low energy may create difficult problems in
%case of global supersymmetric models.
%(see, for example, Ref.\refs{\EGR} where this was noted).
%The absence of calculable schemes in these theories is partially
%supplemented by simpler models in which the essence of the
%dynamics may be revealed. In particular, the above mentioned
Mechanisms {\bf a} and {\bf b} and related physical issues can be
also analyzed in solvable large $N$ vector theories, $O(N)$
symmetric scalar, and supersymmetric  theories. In particular, we
will mention in this appendix the case of spontaneous breaking of
scale invariance, unaccompanied by explicit breaking (namely, {\bf
a} without {\bf b}). This will be demonstrated in the leading
large $N$ analysis of the euclidean action of an $O(N)$ symmetric
quantum field theory in three dimensions.

We will recall first a few general points on scale invariance. The
scale current is conserved at tree level for a scale invariant
potential
$$V(\phi)=g\phi^n$$
if the trace of the improved energy momentum tensor vanishes,
namely
$$
%S^\nu=x_\mu \tilde T^{\mu \nu} {\rm  ~~~~~and~~~~~~}
\del_\mu S^\mu = \tilde T^\nu_{~\nu} = 0 \eqnd\eScaleCurrent$$
where, in general,
%the improved ($A\neq 0$), conserved energy momentum tensor is
$$
\tilde T^{\mu\nu}={2\over \sqrt{-g}}{\delta\over \delta
g_{\mu\nu}}{\cal S} ( \phi ). \eqnd\TmunuDef$$
Here, $\cal S(\phi)$ is the action on a Riemannian manifold:
$$
{\cal S}(\phi)= \int \sqrt {-g}\d^dx
\ud\{g^{\alpha\beta}(x)\nabla_\alpha\phi(x)\nabla_\beta
\phi(x)-m^2\phi^2(x) - \xi R(x)\phi^2(x)\}, \eqnd\RiemanAction$$
$g
= \det \{g_{\alpha\beta}\}$, $\nabla_\alpha$ is the covariant
derivative (which reduces to $\del_\alpha$ for the scalar $\phi$)
and $R(x)$ is the curvature scalar (trace of the Ricci curvature
tensor). Varying with respect to $h_{\mu\nu}$ near flat space
$g_{\mu\nu}=\eta_{\mu\nu}+h_{\mu\nu}$, one then obtains from
Eq.~\TmunuDef:
$$\tilde T^{\mu\nu}=\del^\mu\phi\del^\nu\phi -
\eta^{\mu\nu}\left(\half (\del\phi)^2 - V(\phi) \right) +
\xi(\eta^{\mu\nu}\del^2 - \del^\mu\del^\nu)\phi^2.
\eqnd\energyMomentum$$
Using the equation of motion, the trace is
given by
$$\eqalignno{ \del_\mu S^\mu &= \tilde T^\nu_{~\nu} \cr
&=(\del\phi)^2\left[1-{d/2}+2 \xi (d-1) \right] + \phi\del^2\phi
\left[-{d/ n}+ 2 \xi (d-1)\right].  &\eqnd\eTrace }$$
So, in $d$ dimensions
$$ \xi ={1\over 4}\left({d-2\over d-1}\right)\quad {\rm
and }\quad n={2d\over d-2}\,. \eqnd\eKandn$$
 ($\xi$ is independent
of the potential, e.g.  $\xi = {1\over 6}$ in $d=4$ \refs{\rEMTen}
).

{}Following the variational method described in section
\ssNVarfiv, one defines the on-shell single particle state of
momentum $\vec p$ and mass $m$ to be used as a variational
parameter as $|\Psi_{\vec p}\rangle = a^\dagger (\vec p) |0\rangle
$ where we normalize by
$$ [a(\vec p_1), a^\dagger (\vec p_2)]=(2\pi)^{d-1}2\omega
\delta^{d-1}(\vec p_1-\vec p_2).$$ The energy is $\omega=(\vec
p^{~2}+m^2)^{1/2}$ and the field operators are
$$\phi(t,\vec x )=\int {d^{d-1}k\over (2\pi)^{d-1}2\omega}
\{\e^{ikx}a^\dagger(\vec k) + \e^{-ikx}a(\vec k) \}. \eqnd\ephiField
$$
 One finds from Eq.\energyMomentum
$$\left<\vec p_2\right|\tilde T^{\mu\nu} \left|\vec p_1\right>
= p_2^\mu p_1^\nu + p_1^\mu p_2^\nu -\eta^{\mu\nu} p_1\cdot p_2  -
2\xi(q^2 \eta^{\mu\nu}-q^\mu q^\nu) + \eta^{\mu\nu}\left<\vec
p_2\right|V(\phi)\left|\vec p_1\right> \eqnd\eEMtensor$$ where
$q^\mu=p_2^\mu-p_1^\mu$.

As an illustration, we will discuss the $d=3$, $ O(N)$ symmetric
model with the Euclidean action
$${\cal S}( \phi)= \int \d^3 x
\left[ \ud{(\partial_{\mu} \phi)}^2~ + V(\phi) \right] \eqnd\PhiSixAction  $$
and the potential
$$V(\phi) = { {\mu_0}^2 \over2} \phi^2  +
 {\lambda_0 \over 4N} {(\phi^2)}^2
 + {\eta_0 \over 6N^2}{(\phi^2)}^3.\eqnd\ePotentialPhiSix $$
Following section \ssNVarfiv, the variational ground state
energy density can be calculated from
$$\eqalignno{ {\cal E}_{\rm var.}/N\mathop{\to}_{N\to \infty } & {1\over 2 (2\pi)^d}
\int \d^d k\, \ln[(k^2+m^2 )/k^2] -\ud m^2 \Omega_3(m) +
%\left<\phib^2(x)/N\right>_0 +
 U(\left<\phib^2(x)/N\right>_0) \cr & = \int^m_0 s ds \Omega_3(s)
 -\ud m^2 \Omega_3(m) +U\bigl(\Omega_3(m)+{\sigma^2\over N}\bigr).&
\eqnd{\eEnergDensVar}\cr}$$
 The only divergence (see Eqs.~\eqns{\etadepole,\esaddleNc}) occurs in
$$\left< \phi^2 / N \right>_0
%\equiv  {1\over V}\int \d^{3} x  \, \left< 0 \right|
 %\phi^2 \left| 0 \right>
 = \Omega_3(m) + {\sigma^2\over N}= \int^\Lambda {\d^3 k\over
{(2\pi)}^{3 } } {1 \over k^2+m^2 } + {{\sigma}^2\over N} =
\Omega_3(0) - { |m|\over 4\pi}  + {{\sigma}^2\over N},
\eqnd\phiExpec $$ and the only renormalization needed in order to
obtain a finite ground state energy density ${\cal
E}(m^2,{\sigma}^2) $ is
%eq24
$$\mu^2 = {\mu_0}^2 + \lambda_0 \Omega_3(0)
 + \eta_0 (\Omega_3(0))^2 , \quad\lambda = \lambda_0 + 2{\eta_0}\Omega_3(0)
\,,\quad \eta = \eta_0 \,.\eqnd{\renorCoupl}
$$
The ground state  energy density is then given, at leading order
for $ N\to\infty $, by
%eq25
$$\eqalignno{ {1\over N}{\cal E}(m^2,{\sigma}^2) & =  {m^3 \over 24\pi}
+{\mu^2 \over 2} \left( {{\sigma}^2\over N} -{|m|\over 4\pi}
\right) +{\lambda\over 4} \left( {{\sigma}^2\over N} -{|m|\over
4\pi} \right)^2  \cr &\quad +{\eta\over 6} \left( {{\sigma}^2\over
N} -{|m|\over 4\pi} \right)^3. & \eqnd{\GrStatEnergy}\cr}
$$
In what follows, we discuss the region of parameter space where the
renormalized dimensional parameters are set to zero $(
\mu^2=0$, $\lambda=0 )$. We show  that the non-trivial
critical end-point $( \mu^2=0$, $\lambda=0$, $\eta=\eta_c=
(4\pi)^2 )$ can govern  the continuum limit of the theory. At this
point the mass $m$ of the quanta created by  $\phi (x)$, as found
from the gap equation, is non zero though all dimensional
parameters are set to zero. The appearance of a propagating mass $m$ is associated with spontaneous breakdown of scale invariance. This
is shown by the appearance of a massless bound state as $\eta \to
\eta_c$ and the vanishing of the trace of the energy momentum
tensor.
\smallskip
{\it Explicit calculation.} %$g^{\mu\nu}m^2$
 The term $\eta^{\mu\nu}\langle\vec
p_2|V(\phi)|\vec p_1\rangle$ in Eq.~\eEMtensor ~contributes  to
$\langle\vec p_2|\tilde T^{\mu\nu} |\vec p_1\rangle$ at the tree
level  a term
$$\eqalignno{ &\mu_0^2
+{\lambda_0\over N}\left<\phi^2\right>_0 +{\eta_0\over
N^2}\left<\phi^2\right>_0^2 \cr &= \mu^2 + \lambda\left(
{{\sigma}^2\over N} -{|m|\over 4\pi}\right)+\eta \left(
{{\sigma}^2\over N} -{|m|\over 4\pi} \right)^2 .
&\eqnd\eGapEtaPhiSix }
$$ Using the gap equation, the contribution to $\langle\vec
p_2|\tilde T^{\mu\nu} |\vec p_1\rangle$ in Eq. \eGapEtaPhiSix~is
given, simply, by $\eta^{\mu\nu}m^2$. In the leading order in
an $1/N$ calculation of $\langle\vec p_2|\tilde T^{\mu\nu} |\vec
p_1\rangle$, one encounters an effective four-point vertex:
$${\bar\lambda \over N}\equiv {\lambda_0\over N}+
2 {\eta\over N^2}\left<\phi^2\right>_0= {\lambda\over N }- \eta
{|m|\over 2\pi N}\,. \eqnd\eEffectivCoupling$$ Summing all bubble
graphs, one finds (using the Euclidean action in
Eq.~\PhiSixAction)
$$\eqalignno{ \left<\vec p_2|\tilde T^{\mu\nu} |\vec p_1\right> &=
p_2^\mu p_1^\nu + p_1^\mu p_2^\nu -\eta^{\mu\nu} p_1\cdot p_2
-\eta^{\mu\nu} m^2 \cr &- \left[\frac{1}{ 4}(q^2
\eta^{\mu\nu}-q^\mu q^\nu) + \bar\lambda
I^{\mu\nu}(q)\right]\left({1\over 1+\bar\lambda B(q) }\right),
&\eqnd\eEMtensori }
$$
where
$$\eqalignno{ B(q)&={1 \over  (2\pi )^{3}} \int { \d
^{3}k \over \left(k^{2}+m^{2} \right) [ \left(q+k
\right)^{2}+m^{2} ]}\cr &={1\over 8\pi}\int_0^1 \d\alpha {1\over
\sqrt{\alpha(1-\alpha)q^2+m^2}}={1\over 4 \pi q}\arctan\left({q\over
2|m|}\right),
%\arcsin[{1\over \sqrt{1+4m^2/q^2}}]
&\eqnd\ediagbuli}
$$
and
$$\eqalignno {I^{\mu\nu}(q)&=\int {\d^3k\over (2\pi)^3}{(q+k)^{\mu}k^\nu +k^\mu(q+k)^\nu
-\eta^{\mu\nu}(q+k)\cdot k -\eta^{\mu\nu}m^2\over
\left(k^{2}+m^{2} \right)[ \left(q+k \right)^{2}+m^{2} ] }  \cr &=
{q^\mu q^\nu - q^2 \eta^{\mu\nu} \over 4\pi } \int_0^1 \d\alpha
{\alpha(\alpha-1)\over \sqrt{m^2+\alpha(\alpha-1)q^2}
}\,.&\eqnd\eImunu }
$$
As mentioned above, we are interested in the theory when all
dimensional, renormalized parameters are set to zero and  the
dimensionless $\eta$ at its critical value:
$$\mu=\lambda=0 \quad {\rm and }\quad \eta=16\pi^2.$$

The first line in Eq.~\eEMtensori\ contributes $\half q^2-2m^2$ to
the trace of the energy momentum tensor. As a result of
spontaneous breaking of scale invariance, we will find that the
term in the second line in Eq.~\eEMtensori\ contributes $-\half
q^2 +2m^2$ and ensures a traceless energy momentum tensor  and a
conserved scale current. Indeed, one notices that the factor
$(1+\bar\lambda B(q) )^{-1}$ is exactly the term that appears in a
four-point function of the $\phi$ field. Thus, the appearance of a
massless bound state in this amplitude is directly associated with
the vanishing of the trace of $T^{\mu\nu}$ as expected.
\par

 In Eq.~\eEMtensori, the
 induced effective coupling ($\bar\lambda$  in Eq.~\eEffectivCoupling)  is
 $$ \bar\lambda = -8\pi |m| $$
 and thus $1+\bar\lambda B(q) = 0 $ at $q^2=0$. A
 massless $\phi-\phi$ bound state is created at this effective binding
 strength. Eq.~\eEMtensori\ can be written
 as
 $$\eqalignno{& \left< \vec p_2|\tilde T^{\mu\nu} |\vec p_1\right> =
p_2^\mu p_1^\nu + p_1^\mu p_2^\nu -\eta^{\mu\nu} p_1\cdot p_2
-\eta^{\mu\nu} m^2 +  {1\over 4}(q^\mu q^\nu - q^2 \eta^{\mu\nu}
)\cr &\quad\times \left[1- 8 m \int_0^1 \d\alpha
{\alpha(1-\alpha)\over \sqrt{m^2+\alpha(1-\alpha){q^2}} }\right]
\left[ 1-m\int_0^1 \d\alpha {1\over \sqrt{m^2+\alpha(1-\alpha) q^2
} }  \right]^{-1}. \cr & &\eqnd\eEMtensorii }
$$
 Indeed, due to the following simple identity
$$\eqalignno{&{1\over 4}\left[1- 8m\int_0^1 \d\alpha {\alpha(1-\alpha)\over
\sqrt{m^2+\alpha(1-\alpha) {q^2} }} \right]\cr &\quad=
\left({1\over4}-{m^2\over q^2}\right) \left( 1-m\int_0^1 \d\alpha
{1\over \sqrt{m^2+\alpha(1-\alpha) q^2}} \right) ,&\eqnd\eIdentity
\cr}
$$
%the massless pole term is exactly cancelled with the rest of the
%quantum contribution
%and thus the simple expression
%$$\eqalignno{ \left< \vec p_2|\tilde T^{\mu\nu} |\vec p_1\right> &=
%p_2^\mu p_1^\nu + p_1^\mu p_2^\nu -g^{\mu\nu} p_1\cdot p_2
%-g^{\mu\nu} m^2 \cr &+  (\{ q^\mu q^\nu - q^2 g^{\mu\nu} \} \{
%{1\over 4 }- {m^2 \over q^2} \}\cr & &\eqnd\eEMtensoriii }
%$$
as clearly seen now in Eq.~\eEMtensorii, the  term in the trace of
$\langle \vec p_2|\tilde T^{\mu\nu} |\vec p_1\rangle$ at tree
level is exactly canceled by the induced quantum correction term
due to the bound state massless Goldstone boson--dilaton pole.
that appears at $\eta=\eta_c$ when $\mu^2=\lambda=0$.

More can be seen on Eq.~\GrStatEnergy:   Clearly, at $\mu^2=\lambda=0$ one
finds ${\cal E }(m^2,{\sigma}^2) \rightarrow - \infty $ as $m
\rightarrow \infty$ unless $0<\eta<\eta_c= (4\pi)^2 $. One also
notices that it is the large negative renormalization of
$\phivec^2$ that makes the energy unbounded from below. In the
regularized theory positivity of $\phivec^2$ is maintained, but
the instability at $ \Lambda \rightarrow \infty $ is now reflected
in the solution of the gap equation. In general, $ m=\Lambda
f(\eta_0)$ where $f(\eta_0)$ is a function of the only
(unrenormalized) dimensionless coupling constant in the theory.
Thus, $m \rightarrow \infty$ as $\Lambda \rightarrow \infty$ {
unless} the theory has an UV fixed point exhibited here by a zero
of $f(\eta_0)$ at some $\eta=\eta_c$. In this case, as seen in
Eq.~\GrStatEnergy, the ground state energy density ${\cal
E}(m^2,{\sigma}^2)$ is flat, namely, is  $m$ independent  after
$\Lambda \to \infty $ has been removed. A cutoff independent,
finite physical mass can appear in the ground state spectrum.
Namely,
$$ m=\Lambda f(\eta_0) \rightarrow
{\rm ~finite~ value ~as~} \Lambda \rightarrow \infty
 \quad {\rm since }\quad \eta_0(\Lambda) \to {\eta_c}^- {\rm ~as~}
\Lambda \to \infty \,. \eqnd{\critEta} $$
One finds
%\refs{\Bmb}
$$ m=\Lambda f(\eta_0) = \Lambda {\pi \over 2}
\left( 1-\sqrt{{\eta_c \over \eta_0} }\right) .
\eqnd{\FinitMass}$$
This gives (from $\partial m/ \partial \Lambda=0 $) a
$\beta(\eta_0)$ function with a UV fixed point at
$\eta_0=\eta_c$. The bare coupling constant is then solved and
one finds
%eq28
$$ \eta_0(\Lambda)=\eta_c +{\tilde \mu \over \Lambda}\quad
{\rm  and~ thus }\quad m={\pi \over 4 \eta_c}\tilde \mu\, ,
\eqnd{\runningEta} $$ where $\tilde \mu $ is a new normalization
scale. The above is just the manifestation of dimensional
transmutation at the non-trivial UV fixed point.
\smallskip
 {\it Summary :} Two facts should be
noted now:  {\bf (a)} In the massive phase described above, scale
invariance has been broken only spontaneously. As seen from
$\eta_0=\eta$, there is no explicit breaking of scale invariance
(at $\mu^2=\lambda=0$), and the perturbative $\beta (\eta)$
function vanishes in the large $N$ limit. Indeed, one finds that
the trace of the energy momentum tensor stays zero. A massless
dilaton --- the Goldstone boson associated with this breaking ---
appears in the ground state spectrum as a reflection of the
Goldstone realization of scale symmetry. ${\cal
E}(m^2,{\sigma}^2)$ is $m$ independent. {\bf (b) } The normal
ordering of $\phi^6$ induces a new $\bar\lambda \phi^4$
interaction, where $\bar \lambda=\lambda-\eta_0 [{m \over 2\pi}]$.
This new interaction guarantees the appearance of the dilaton pole
in the physical amplitudes as $\eta_0 \to \eta_c$ .

Finally, higher orders in $1/N$ introduce an explicit breaking of scale invariance and probably destabilize the finite $N$ scalar theory
in Eq.~\PhiSixAction.


\vfill\eject
