%%%%%% Nrev8 as of March 31 2003

\nref\rGirard{ Early discussions and controversy on the breakdown
of supersymmetry at finite temperature are found in \rf  L.
Girardelo, M.T. Grisaru, P. Salomonson, {\it Nucl. Phys.} B178
(1981) 331 and in \rDas.}
\nref\rDas{ A. Das and M. Kaku, {\it
Phys.Rev} D18 (1978) 4540; A. Das {\it Physics A } 158 (1989) 1.}

\nref\rSenjan{ Restoration of broken internal symmetries were
discussed in \rf A. Riotto and G. Senjanovi\'c, {\it Phys. Rev. Lett.}
79 (1997) 349 [arXiv:hep-ph/9702319]  and references therein. }

\nref\rFotop{The thermodynamics of supersymmetric gauge theories
was studied in  \rf A. Fotopoulos and T.R. Taylor, {\it Phys.
Rev.} D59 (1999) 061701 and references therein. }

\nref\rEspin{The minimal supersymmetric model at finite
temperatures was discussed in \rf J.R. Espinosa, {\it Nucl. Phys.}
B475 (1996) 273 and in ref.~\rLaine.}
\nref\rLaine{ M. Laine, {\it
Nucl. Phys.} B481 (1996) 43, {\it Erratum ibid.} B548 (1999) 637.}



\section  $O(N)$ supersymmetric models  at finite temperature

In this section we would like to find out which of the peculiar
properties of the phase structure seen at $T=0$ in  section
\ssNSUSYsc\ are maintained at finite temperature and how the phase
transition occurs as the temperature, rather than the coupling
constant, is varied~\rMosZinn. \sslbl\ssNSUSYT

Supersymmetry is softly broken at finite temperature as bosons and
fermions behave differently when interacting with a heat bath.
Breakdown of supersymmetry at finite temperature has attracted
much attention since the early interest in supersymmetry and
involved much controversy on its consequences, appearance and
phase structure \refs{\rGirard,\rDas}. Other related issues such as
the restoration of broken internal symmetries at finite
temperature supersymmetric theories were also debated
\refs{\rSenjan}. More recently, there is a continued interest in
the thermodynamics of supersymmetric Yang-Mills theory
\refs{\rFotop} and temperature effects on the minimal
supersymmetric model \refs{\rEspin}.


%
\subsection {The free energy at finite temperature}

Following the notation of section \ssNSUSYsc~to which we refer for
details, we now derive the free energy at finite temperature. The
partition function is given by \sslbl\ssNSUSYfreeE
$$ {\cal Z}=\int[\d\Phi][\d\rho][\d L] \e^{-{\cal S}(\Phi,L,R)}
\eqnd\SUSYaction$$
with
$${\cal S}(\Phi,L,R)=\int\d^3 x\,\d^2\theta\left\{\ud\bar {\rm D}\Phi\cdot
{\rm D}\Phi+NU(R)+L(\theta)\left[\Phi^2(\theta)-NR(\theta)\right]
\right\},$$ where $\Phi $, $L$, $R$ are $N$-component scalar
superfields, parametrized in the form Eq.~\PhiSupField:
$$ \Phi(\theta,x) =
\varphi + \bar \theta \psi + \ud  \bar \theta\theta  F   $$ and
(Eqs.~\eqns{\eLsupField,\eRsupField})
$$L(\theta,x)=M+\bar\theta\ell+\ud\bar\theta\theta \lambda\,,
\quad R(\theta,x)=\rho +
\bar\theta\sigma+\ud\bar\theta\theta s\,.\eqnd\LandRho$$
The Grassmann coordinate  $\theta$ is a two-component
Majorana spinor; $\varphi$, $M $ and $\rho$, are $N$ component
real scalar fields $\psi$, $\ell $ and $\sigma$ are $N$ component,
two-component Majorana spinors $F$, $ \lambda $ and $s$ are $N$
component  auxiliary fields. $\D$ is the covariant derivative,
$\D=\partial /
\partial \bar \theta -\delslash ~\theta $, the integration
measure is $\d^2\theta={i\over 2}d\theta_2 d\theta_1$ and
$\bar\theta_\alpha\theta_\beta =
\half\delta_{\alpha\beta}\bar\theta\theta $.

Integrating out $N-1$ superfield components  of $\Phi$ and keeping
$\Phi_1\equiv \phi$ (the scalar component of the superfield $\phi$
is identified as $\varphi_1 \equiv \varphi$) one finds
$${\cal Z}=\int[\d\phi][\d R][\d L]\e^{-{\cal
S}_N(\phi,R,L)},\eqnd\Nintegration $$ where the  large $N$ action
is
$$\eqalignno{{\cal S}_N=\int\d^3 x\,\d^2\theta &\left[\half\bar \D\phi
\D\phi+NU(R)+ L \left(\phi^2-N R\right)\right] \cr &\quad+
\half(N-1){\rm Str}\, \ln\left[-\bar D
D+2L\right].&\eqnd\LargeNaction \cr}$$ The two saddle point
equations~\eSUSYsad{} are not changed. Only the third equation
\eSUSYsadc\ is affected  by the boundary conditions due to finite
temperature: \eqna\saddle
$$\eqalignno{ 2L\phi-\bar D D\phi&=0\,,&\saddle{a} \cr
  L-U'(R)&=0\,,&\saddle{b} \cr
 R-\phi^2/N&= {1\over N}\tr\Delta(k,\theta,\theta), &\saddle{c}\cr
} $$
$\Delta(k,\theta,\theta)$ is given by \eSUSYpropcoin:
$$
\Delta(k,\theta,\theta)= {1\over k^2+M_T^2+\lambda}+
\bar\theta\theta M_T \left( {1\over k^2+M_T^2+\lambda} - {1\over
k^2+M_T^2} \right). \eqnd\SUSYpropagator$$where $M_T$ is the
expectation value of $M(x)$ at finite temperature $T$. When
written in components, Eq.~\saddle{a} implies \eqna\saddleComponA
$$\eqalignno{F-M_T\varphi&=0 \  , &\saddleComponA {a} \cr
 \lambda\varphi+M_TF&=0 \,,&\saddleComponA{b} \cr} $$
from which  the Goldstone condition $\varphi(\lambda+M_T^2)=0 $
follows.\par Eq.~\saddle{b} implies:
$$M_T=U'(\rho)\ ,\quad  \lambda=sU''(\rho).\eqnd\saddleComponB $$
When calculating the trace in Eq.~\saddle{c}, one has to take into
account that bosons at finite temperature satisfy periodic, and
fermions anti-periodic boundary conditions. Then, combining
Eq.~\eNfivTtp~with the $\theta =0$ part of the finite temperature
Eq.~\saddle{c}, we obtain
$$ \rho-\varphi^2/N = \int {\d^2k\over (2\pi)^2 }{1\over \omega_\varphi (k)}
 \left({1\over2} +{1\over \e^{  \omega_\varphi (k)/T}-1}\right) $$
with
$$\omega_\varphi (k)=\sqrt{M_T^2+\lambda +k^2}. $$
It is convenient here to introduce the boson thermal mass
$$m_T=\sqrt{M_T^2+\lambda}\,.\eqnd\eSUSYbostherm $$
The thermal mass $m_T$ characterizes the decay of correlation
functions in space directions. Note that, on the other hand, $M_T$
does not characterize the decay of fermion correlations, because
fermions have no zero mode, the relevant parameter being
$\sqrt{M_T^2+\pi^2T^2}$. Then,
$$\rho-\varphi^2/N   =\rho_c
  -{ T\over 2\pi}\ln\bigl(2\sinh(m_T/2T)\bigr) \eqnd \esadSUSYNTa $$
where  $\rho_c$ has been defined in Eq.~\eRciiidef.

In the same way, the $\bar\theta \theta $ part of Eq.~\saddle{c}
combined with Eq.~\eNGNTtp~yields
$$\eqalign{s-2F\varphi /N&=2M_T \int {\d^2k\over (2\pi)^2  } \left\{ {1\over 2 \omega _\varphi(k)}
\left({1\over 2}+{1\over \e^{ \omega_\varphi(k)/T}-1}\right)
\right. \cr&\quad \left.
 -{1\over 2 \omega _\psi(k)}\left({1\over2} -{1\over \e^{ \omega_\psi (k)/T}+1}\right)\right\} \cr}$$
with $\omega _\psi=\sqrt{M_T^2+k^2}$. Integrating we obtain
$$ s-2F\varphi /N =  {  T M_T\over  \pi}\left[
 \ln\bigl(2\cosh( M_T/ 2)\bigr)-\ln\bigl(2\sinh(m_T/2T)\bigr)\right].
\eqnd  \esadSUSYNTb $$
%%%%%%%%%%%%%%%%%%%%%%%%%%%%%%%%%%%%%%%%%%%%%%

The action density at finite temperature ${\cal F}$, to leading
order in $1/N$, is given by ($\beta =1/T$)
$${\cal F}={\cal S}_N/{V_2 \beta}\ , \eqnd\freeEnergyT$$
where ${\cal S}_N$ is given in Eq.~\LargeNaction~at constant
fields and $V_2$ is the two dimensional volume.

The supertrace term in Eq.~\LargeNaction~can be calculated at finite
temperature by  using the expressions
Eqs.~\TRboson~and~\eTRfermion:
$$\eqalignno{ {1\over V_2  } {\rm Str}\, &\ln\left [-\bar D D+2L\right] =
%\half
{1\over V_2  }\tr\ln(-\partial^2 +M_T^2+\lambda) -{1\over V_2  }
%\half
\tr\ln(\sla{\partial}+M_T) \cr
  &=2  \int {\d^2k\over {(2\pi)^2} }
  \ln[2\sinh( \beta \omega _\varphi /2 ) ]
  -  2 \int {\d^2k\over {(2\pi)^2} }
  \ln[2\cosh( \beta  \omega _\psi/2)]    \cr
&={1 \over T}\rho_c (m_T^2-M_T^2) -{ 1\over 6\pi T}\left(
m_T^3-|M_T|^3\right) \cr &\quad +2\int {\d^2k\over  (2\pi)^2
}\left\{
  \ln[ 1-\e^{- \beta \omega _\varphi  } ]-  \ln[1+\e^{- \beta  \omega _\psi}]
  \right\}.
  &\eqnd\SuperTraceT \cr} $$
The explicitly subtracted part reduces for $M_T=M$ to the $T= 0$ result.
The other contributions to ${\cal E}$ are the same (up to the change
$M\mapsto M_T$) as in Eq.~\eSUSYV\ and therefore
one finds
$$\eqalignno{ {1\over N}{\cal F}&=
   -  {F^2\over2 N} +
M_T{F\varphi\over N} +   \lambda{\varphi^2\over 2N}  + \half s
(U'(\rho)-M_T)   -{1\over 12\pi}\left(  m_T^3 - |M_T|^3\right)\cr
&\quad + \half\lambda (\rho_c-\rho) +T\int {\d^2k\over  (2\pi)^2
}\left\{
  \ln[ 1-\e^{- \beta \omega _\varphi  } ]-  \ln[1+\e^{- \beta  \omega _\psi}]
  \right\}  \,.
  &\eqnd\freeEnergyTiii \cr }$$
In the limit $T=0$, the free energy in Eq.~\freeEnergyTiii~reduces
to the action density of  Eq.~\eSUSYV,   ${\cal F}(T=0)\equiv{\cal
E}$.\par

Eq.~\freeEnergyTiii\ can be rewritten by using  Eq.~\esadSUSYNTa,
which is also obtained by setting to zero the $ \del/ \del\lambda
$ derivative of Eq.~\freeEnergyTiii, as well as $M_T-U'(\rho)=0$
from Eq.~\saddleComponB~(or equivalently, setting to zero the $
\del/ \del s $ derivative of Eq.~\freeEnergyTiii). One finds
%
$$\eqalignno{&{1\over N} {\cal F} = \half M_T^2 {\varphi^2\over N} + {1\over 24 \pi}(
m_T - |M_T|)^2(m_T +2|M_T|) \cr
&\quad+{T\lambda\over 4\pi
}\ln(1-\e^{-m_T/T})+T \int {\d^2k\over
{(2\pi)^2} } \{ \ln(1-\e^{-\beta \omega_\varphi} )
 - \ln(1 + \e^{-\beta \omega_\psi} ) \}.\hskip12mm
&\eqnd\FreeEnergy }$$
Eq.~\FreeEnergy~is the finite energy version of
Eq.~\HartreeVTzeroC.
 \par
Inserting Eqs.~\esadSUSYNTa~and \esadSUSYNTb~(with $F=M_T\varphi$)
into Eqs.~\saddleComponB~with $U( R )=\mu R  +\half u  R^2$, one
finds ($\mu_c=-u\rho_c$)
 $$\eqalignno{  M_T &=
  \mu-\mu_c   + u {\varphi^2\over N}
 -{u\over 2\pi}T\ln\left(2\sinh\left(\ud  m_T/T \right)\right) , &
 \eqnd\FermionMassi  \cr
{m_T^2-M_T^2 \over uM_T}&={2\varphi^2 \over N}     - {   T  \over  \pi}\left[
\ln\left(2\sinh\left(\ud  m_T/T \right)\right)-\ln\left(2\cosh
\left(\ud  |M_T|/T \right)\right) \right],
\hskip10mm & \eqnd  \esadSUSYNTla\cr} $$
 which are the   gap equations for  $T\neq 0$. \par
 %%%%%%%%%%%%%%%%%%%%%%%%%%%%%%%%%%%
%********* USEFUL?
%After some rearrangement:
%
%$$\eqalignno{&{1\over N} {\cal F}%= \half M^2 {\varphi^2\over N}
%+ {1\over 24 \pi}( \sqrt{M^2+\lambda} - |M|)^2(\sqrt{M^2+\lambda}
%+2|M|) \cr & - {\lambda\over 2} \int {\d^2k\over {(2\pi)^2}
%}{1\over \omega_\varphi} [\e^{\beta \omega_\varphi}-1]^{-1} +T
%\int {\d^2k\over {(2\pi)^2} } \{ \ln(1-\e^{-\beta \omega_\varphi} )
% - \ln(1 + \e^{-\beta \omega_\psi} ) \}
%\cr &
%= \half M_T^2 {\varphi^2\over N} + {1\over 24 \pi}(
%\sqrt{M_T^2+\lambda} - |M_T|)^2(\sqrt{M_T^2+\lambda} +2|M_T|)
%\cr
%&\quad+{T\lambda\over 4\pi }\ln(1-\e^{-\beta\sqrt{M_T^2+\lambda}})+T \int {\d^2k\over {(2\pi)^2} } \{
%\ln(1-\e^{-\beta \omega_\varphi} )
% - \ln(1 + \e^{-\beta \omega_\psi} ) \}\hskip12mm
%&\eqnd\FreeEnergyi }$$
%Eq,~\FreeEnergyi~ agrees with Eq.~\HartreeVTC~which has been derived by
%the Hartree--Fock
%variational approach.
%************ \par
\medskip
{\it Solutions to saddle point equations:}
 One first notes that the r.h.s.~of the gap Eq. \FermionMassi\
 diverges when $m_T\to 0$.
This phenomenon has been already discussed in the example of the
scalar field theory. A finite temperature system in three
dimensions has the property of a statistical system in two
dimensions. Spontaneous breaking of a continuous symmetry is
impossible in two dimensions due to the  IR behaviour of a system
with potential massless Goldstone particles. Therefore, the $O(N)$
symmetry is never broken,  $\varphi=0$, and thus
\eqna\eNTSUSYgapii
 $$\eqalignno{ { M_T \over T} &=
 { \mu-\mu_c\over T}     -{u\over 2\pi} \ln\left(2\sinh\left(\ud   m_T/T
 \right)\right) , &
 \eqnd\eNTSUSYgapii{a} \cr
{M_T^2-m_T^2\over uTM_T}&=    {   1  \over  \pi}\left[
\ln\left(2\sinh\left(\ud  m_T/T \right)\right)-\ln\left(2\cosh \left(\ud
M_T /T \right)\right) \right].
\hskip2mm & \eqnd  \eNTSUSYgapii{b}
\cr} $$
Note that while the expression of $\cal F$ is complicated, its derivative with
respect to $m_T^2 $ remains simple
$${1\over N}{\partial  {\cal F} \over \partial m_T^2 }={(m_T^2-M_T^2)
\over 16\pi m_T  \tanh(m_T/2T)}.
\eqnn $$
Therefore, the minimum still occurs at $m_T=|M_T|$,  but
one verifies that $m_T=|M_T|$ is not a solution to the saddle
point equations. We have only found a lower bound on the free energy density  $\cal F$ as a function of $M_T$:
$$ {\cal F}=NT \int {\d^2k\over (2\pi)^2 }\ln\tanh\left(\ud\beta \sqrt{M_T^2+k^2} \right).
\eqnd\eSUSYTFlow $$
Its derivative with respect to $M_T$ is
$${\partial {\cal F}\over \partial M_T}=-N{T M_T\over 2\pi}\ln\tanh(|M_T|/2T).$$
Therefore the derivative has the sign of $M_T$. The lower bound  has a limit for $M_T\to 0$, which is thus an absolute lower bound:
$${\cal F}=-{7\over 8\pi}\zeta (3)NT^3\,.\eqnd\eSUSYFlow $$
Similarly, the  derivative of $\cal F$ with respect to $ M_T $ at $m_T$ fixed is
$$\eqalignno{{\partial {\cal F}\over\partial  M_T }&=   { M_T T\over 2\pi}
\left[
\ln\bigl(2\cosh(M_T/2T)\bigr)-\ln\bigl(2\sinh(m_T/2T)\bigr)\right] \cr
&={m_T^2-M_T^2 \over 2u}\,. \cr}$$

 We now have to find the solutions to the saddle point
 equations and compare their free energies.\par
Note a first solution in the regime $T\to0$, $\mu<\mu_c$ with $|m_T |\ll T$ and $|M_T\ll T$
Then, from \FermionMassi, we find the boson thermal mass
$$m_T \sim T\e^{-2\pi(\mu_c-\mu)/uT}\,.$$
Eq.~\esadSUSYNTla~yields the other mass parameter
$$M_T \sim {m_T^2 \over \mu_c-\mu}\,.$$
Since both $m_T$ and $M_T$ are very small, the free energy is very
close to the lower bound \eSUSYFlow. The solution found here is a
precursor of the zero temperature phase transition, and
corresponds to $\varphi\ne 0$ in region III and IV of
Fig.~\phases. ~Other solutions  exist in this regime but they
converge, up to exponential corrections, to the finite masses of
the $T=0$ spectrum, and thus their free energy is much larger from
the lower bound \eSUSYTFlow.\par It is expected that even for $T$
larger and $\mu\ge \mu_c$ the continuation of the small mass
solution remains the solution with lowest energy.\par
%\subsection Variational calculations

Variational calculations will produce here similar results to
those obtained from the path integral at $N \to \infty$.
%\par
We know from the general discussion at $T=0$ in section
\ssSUSYNvar\ that one of the gap equations $M_T=U'(\rho)$ is
immediately implied by demanding that the free energy density in
the variational calculations remains finite for large cut-off
$\Lambda $. Therefore, the variational free energy in Eq.~\eVarEsusy\
 (with $\rho$ and $\tilde \rho$  of Eqs.~\eR a ~and
 \eR b~replaced by their thermal expressions) takes the form  identical
to Eq.~\FreeEnergy.


$m_T$ and $M_T$ being related, in what follows,  we take as the
independent variable $m\equiv m_T$ and, using Eq.~\FermionMassi\
for $M_T=M_T(m,\varphi,\mu,u,T)$, ~we discuss
$$W(m,T)={1\over N}{\cal F}
(m_T=m,M_T(m,\varphi,\mu,u,T),T).\eqnd\eWvar$$
Since, as described above, at $T\neq 0$ there is no breaking of
 $O(N)$ symmetry, we will discuss mainly $\varphi=0$ in
Eq.~\eWvar.
%%%%%%%%%%%%%%%%%%%%%%%%%%%%%
%*********** AS INDICATED ABOVE I SUGGEST TO REDUCE
%MOST OF THE DISCUSSION TO $\varphi=0$ SINCE THIS
%THE ONLY POSSIBLE SOLUTION. If you wish you can show one picture
%with $\varphi\ne 0$ just as an illustration but I see no real point in
%carrying a whole discussion with arbitrary $\varphi$.
%\par
%********************************* \par
 % For $u<u_c$ the
%l.h.s.~has a maximum for a finite value of $|M_T/T|$, and
%therefore the equation has either two solutions or no solution
%depending of the value of $(\mu-\mu_c)/T$. But in all cases the
%value of $|M_T|$ will be larger than for the positive solution. We
%conclude that only the positive solution corresponds to a stable
%situation. Clearly this stability is due to the entropy term in
%the ground state free energy. The  entropy negative contribution
%to ${\cal F}$ in Eq.\FreeEnergy~ is larger for the small mass
%quanta (positive $M_T$)than for the larger mass quanta (negative
%$M_T$).

 %appeared in the Hartree--Fock calculations in
%Eq.~\FermionMass.
%\par ************************ ??????????? \par

%%%%%%%%%%%%%   major changes from here  MM Oct. 9  %%%%

%The situation described above is clearly visualized when

%%%%%%%%%%%%%%%%%%%%

 For $T=0$, Eq.~\eWvar\  results in
$ {1\over N}{\cal F}(m,\varphi^2, T=0) ={1\over N}{\cal
E}(m,\varphi^2)$ which has been analyzed in section \ssNSUSYsc. In
particular, see the phase structure in Fig.\phases~ and the
interesting degeneracy found in Regions II and IV and exhibited in
Figs.\ZeroTone-- \ZeroTtwo.

Rather than changing the value of $\mu-\mu_c$ and the coupling $u$
as done in section \ssNSUSYsc, we are interested here to see
the temperature dependence while the parameters $\mu-\mu_c$ and
$u$ are held fixed.

At finite temperature, the effective field theory describes  the
interactions of the fermions and bosons with the heat bath. This
interaction acts like a source of soft breaking of supersymmetry.
The short distance behavior is cured at finite temperature in a
similar way it happens at $T=0$. We will see now that the general
properties of the transitions as a function of the temperature $T$
have a similar character as the transitions seen at $T=0$ when the
coupling constant was varied.


As seen in Fig.~\phases, at $T=0$ there are two regions in the $\{
\mu-\mu_c , u \}$ space where the vacua are degenerate.
%We will
%discuss first these two regions as the temperature is raised from
%zero to a  finite temperature.
These are (region {\bf II :}
$\mu-\mu_c \geq 0$ and $u \geq u_c$ and region {\bf IV : }
$\mu-\mu_c \leq 0$ and $u \leq u_c$). Fig.~\figTIIaaa ~shows the
ground state  energy ($T=0$) in region {\bf II} and Fig.~\figTIVaa ~shows the  ground state energy  in region
{\bf IV}.



We will discuss first region {\bf (II)}  $\mu-\mu_c \geq 0$
and $u \geq u_c$:

\noindent Here, as $T$ increases the ground state with the smaller
mass ($\ma\equiv m =m_+$) has a lower free energy than the heavier
one ($\ma \equiv m=m_- > m_+$) due to a higher entropy
contribution,
%as mentioned above
(Fig.~\label{\figTIIa}).

%\noindent

Peculiar transitions can occur in this system. If the system was
initially at $T=0$ ~in the ground state with a boson mass
$\ma=m_-$, it will eventually go, as $T$ is increased, into the
state with mass $\ma=m_+$, namely, into the lower mass ground
state. This is due to the entropy negative contribution to the
free energy forcing the system to favors the lowest mass state. On
the other hand, a system that started at a high temperature in the
state of low mass $\ma=m_+$ will stay in this state as the system
is cooled and will never roll back into the $m=m_-$ high mass
phase. Favoring the lowest  mass phase as the temperature
increases is a general effect that will occur in any physical
system that is initially (at $T=0$) mass degenerated. Here,
supersymmetry imply that the $m_+$ and $m_-$ vacua are at the same
energy $E=0$ at $T=0$.

\midinsert \epsfxsize=14.5cm \epsfysize=8cm \pspoints=2.pt \vskip
-4cm \hskip 2cm \epsfbox{figtiia.eps} \figure{1mm}{The
 free energy $W(m,\varphi,T)={1\over N} {\cal
F}(m\equiv\ma,\varphi,T)$ as given in Eq.~ \FreeEnergy\ at
$\varphi=0$ as a function of the boson mass ($m$) at different
temperatures. Here $\mu-\mu_c=1 $ (this sets the mass scale),  $
u/u_c=1.5 $ and $T$ varies between $T=0 - 0.5$. At $T = 0$ two
degenerate $O(N)$ symmetric phases exist  with a light $m=m_+$ and
heavier $m=m_- > m_+$ massive boson (and fermion). As the
temperature increases the light mass phase is stronger affected as
its entropy increases faster and becomes the only ground state.}
\figlbl\figTIIa
\endinsert

\noindent Region {\bf (IV)}  $\mu \leq 0$ and $u \leq u_c$: Here
one finds at $T=0$ two distinct degenerate phases. One is an
ordered phase ($\varphi \neq 0$) with a zero boson and fermion
mass, the other is a symmetric phase ($\varphi = 0$) with a
massive ($m=m_-$) boson and fermion. (as seen in
Fig.~\label{\figTIVaa}).

As mentioned above at $T\neq 0$
the ordered phase with the broken
$O(N)$ symmetry,
 ($m=0 , \varphi^2 \neq 0$) disappears
into $\varphi^2 = 0$ symmetric phase and a very small mass ground
state $\ma=m\geq 0$ is created (as seen in Fig.~\label{\figTIVbb}
and in Fig.~\label{\figTIVa}).
 \midinsert
\epsfxsize=13cm
\epsfysize=10cm \pspoints=2.pt \vskip -3.5cm\hskip 2.5cm
\epsfbox{figtivbb.eps} \figure{1mm}{Same as Fig.~\figTIVaa ~in
section~\ssNSUSYsc ~but the temperature has been increased from $T
= 0$ (in Fig.~\figTIVaa) to $T=0.7$ (here).
$W(m,\varphi,T)={1\over N} {\cal F}(m\equiv\ma,\varphi,T)$ as
given in
 Eq.~ \FreeEnergy\ as a function of the
boson mass ($m$)  and $A$, where $A^2=\varphi^2/u_c$.
 Here $\mu-\mu_c=-1 $ and
$ u/u_c=0.2 $. A non-degenerate $O(N)$ symmetric ground state
($\varphi = 0$) appears with a very small boson mass (the non-zero
mass is not seen here due to the limited resolution of the plot).
} \figlbl\figTIVbb
\endinsert

Eventually, also the other $O(N)$ symmetric vacua ($m=m_-,
\varphi^2 = 0$) goes  into the small mass $O(N)$ symmetric ground
state.

\midinsert \epsfxsize=13cm \epsfysize=11cm \pspoints=2.pt \vskip
-5cm \hskip 2.5cm \epsfbox{figtiva.eps} \figure{1mm}{This figure
displays the effect of increasing the temperature from $T = 0$ in
Fig.~\figTIVaa~to $T$ that varies in the range $ 0 \le T\le  0.7$
(Fig.~\figTIVbb ~has $T=0.7$). The  free energy
$W(m,\varphi,T)={1\over N}{\cal F}(m\equiv\ma,\varphi,T)$ as given
in Eq.~ \FreeEnergy\ at $\varphi=0$ is plotted as a function of
the boson mass ($m$)  at different temperatures. Here
$\mu-\mu_c=-1 $, $ u/u_c=0.2 $. At $T = 0$ there are two
degenerate phases: An $O(N)$ symmetric phase, shown here, with a
massive ($m=m_-$) boson and fermion  and an ordered phase
($\varphi \neq 0$) with massless particles (both phases are shown
in Fig.~\figTIVaa).  At finite temperatures the $O(N)$ symmetry is
restored (see Fig.~\figTIVbb) and a small mass ground state
appears, the heavy mass state decays into the small mass ground
state as seen here.} \figlbl\figTIVa
\endinsert

\noindent As in {\bf II}, a system that was initially, at $T=0$
in the $O(N)$ symmetric phase ($\ma=m_- , \varphi^2 = 0$) will
decay into the smaller mass state when the temperature. But upon
cooling the system in the small mass phase will roll into the
ordered phase ($\ma=m=0 , \varphi^2 \neq 0$) at $T=0$. The system
will never roll back into the symmetric phase ($\ma=m_- , \varphi
= 0$).

%\vfill\supereject
\subsection The supersymmetric $O(N)$
non-linear $\sigma $ model at finite temperature

The supersymmetric non-linear $\sigma $ model we consider here has already
been discussed at zero temperature in section \ssSUSYnls\ to which
we refer for details.\sslbl\ssNlsSUSYT
\par
The partition function of the $O(N)$ supersymmetric non-linear $\sigma $ model
in $d $ dimensions is given by
$${\cal Z}=\int[\d\Phi][\d L] \e^{-{\cal S}(\Phi,L)} \eqnd\NLSMpartition$$
with
$${\cal S}(\Phi)={1\over 2 \kappa }\int\d^d x\,\d^2\theta\, \bar D\Phi\cdot
D\Phi +L(\Phi^2-N)  \,, \eqnd\eNLSMT $$
where $\kappa$ is a constant. The scalar superfield
$\Phi(x,\theta)$  is an $N$-component vector:
$$ \Phi(x,\theta) =   \varphi + \bar \theta  \psi +
\ud \bar \theta\theta   F \,,\eqnd{\evecPhi} $$ and the scalar
superfield $L(x,\theta)$ is given by
$$L(x,\theta) = M(x) + \bar\theta \ell(x) +\half \bar\theta\theta
\lambda(x) .\eqnd\L$$ After integrating out $N-1$ superfield
components of $\Phi$, leaving out $\phi=\Phi_1$, one obtains
$${\cal Z} = \int [d\phi][\d L] \e^{ - {\cal S}_N(\phi,L) }$$
where
$$   {\cal S}_N(\phi,L)= {1\over\kappa } \int\d^d x\,\d^2\theta  \left[ \half\bar D\phi\cdot
D\phi+  L \left(\phi^2- N \right) \right]    + {{N-1}\over 2} {\rm
Str}\, \ln( -{\bar D}D +2 L )  .   \eqnd\eNLSTact   $$ Varying the
action $     {\cal S}_N(\phi,L)$
 by varying the superfields
$\phi$ and $L$,   one obtains Eq.~\eSadNLSMa\
and the generalization of Eq.~\eSadNLSMb:
\eqna\eSUSYNTnls
$$\eqalignno{ 2L\phi-\bar D D\phi &=0 \,,&\eSUSYNTnls{a} \cr
 {N\over \kappa}-{\phi^2 \over N}&=\tr\Delta(k,\theta,\theta) .&\eSUSYNTnls{b} \cr}
 $$
The first equation is equivalent to $F=M_T\varphi$ and $\lambda \varphi+FM_T=0$
while the second equation can be compared with Eq.~\saddle{c}, and thus
leads to

(for $N\gg 1$ and $\psi=0$)
\eqna\NLSMComponC
$$\eqalignno{ {1\over \kappa}-{\varphi^2\over N} &=\int {\d^dk\over (2\pi)^d }{1\over \omega_\varphi (k)}
 \left({1\over2} +{1\over \e^{  \omega_\varphi (k)/T}-1}\right), &\NLSMComponC{a}\cr
-{2F\varphi  \over N}&=2M_T \int {\d^dk\over (2\pi)^d  } \left\{ {1\over 2 \omega _\varphi(k)}
\left({1\over 2}+{1\over \e^{ \omega_\varphi(k)/T}-1}\right)
\right. \cr&\quad \left.
 -{1\over 2 \omega _\psi(k)}\left({1\over2} -{1\over \e^{ \omega_\psi (k)/T}+1}\right)\right\}  & \NLSMComponC{b}\cr} $$
with $\omega_\varphi (k)=\sqrt{m_T^2  +k^2}$ and  $\omega_ \psi (k)=\sqrt{M_T^2  +k^2} $.
\medskip
{\it Dimension $d=3$.} As it has been discussed already, the $d=3$
finite temperature theory is analogous from the point of phase
transitions to a two-dimensional theory. Therefore the $O(N)$
symmetry remains unbroken and $\varphi=0$. We thus write the two
gap equations only  in this limit

\eqna\eNLSMsadTi
$$\eqalignno{
 {1\over \kappa _c}  - {1\over \kappa} &=
 {  T\over 2
\pi}\ln\bigl(2\sinh(m_T/2T)\bigr)&\eNLSMsadTi{a}
\cr
0&= {  T M_T\over 2 \pi}\left[\ln\bigl(2\sinh(m_T/2T)\bigr)-
\ln\bigl(2\cosh(M_T/2T)\bigr)\right].
&\eNLSMsadTi{b} } $$
The calculation of the  energy density to leading order for $N\to\infty $
of the non-linear $\sigma $ model at finite temperature follows the similar
steps of the $(\Phi^2)^2$ field theory.
The   free energy at finite temperature is given by
$${\cal F}=T {\cal S}_N / V_2\,, \eqnd\NLSMFEnergyT$$
where ${\cal S}_N$ is given by Eq.~\eNLSTact~at constant fields and
$V_2$ is the two dimensional volume.
The expression has been calculated in section \ssSUSYnls\ but here
the supertrace term in Eq.~\eNLSTact~has to be replaced by the
finite temperature form as
it appears in Eq.~\freeEnergyTiii.
The free energy then is given by
$$\eqalignno{{1\over N} {\cal F}&=      \ud(M_T^2-m_T^2)\left( {1\over\kappa } -{1\over\kappa_c}\right)
-{1\over12\pi}\left(m_T^3-|M_T|^3\right) \cr\quad &
+T\int {\d^2k\over  (2\pi)^2  }\left\{
  \ln[ 1-\e^{- \beta \omega _\varphi  } ]-  \ln[1+\e^{- \beta  \omega _\psi}]\right\}.
&\eqnd\NLSMFreeEnergyTiii \cr}$$
The  free energy in Eq.~\NLSMFreeEnergyTiii~can be compared
with Eq.~\eNLSMenergy\ which gives the ground state  energy ${\cal F}$ at $T=0$.
\par
In section \ssSUSYnls\ we have verified that after using
the zero temperature limit of Eq.~\eNLSMsadTi{a} the energy densities
of the $(\Phi^2)^2$ theory and the non-linear $\sigma $ model become identical, up to a possible rescaling of the field.
It is now clear that the same mechanism works here. Using Eq.~\eNLSMsadTi{a} one indeed finds the expression \FreeEnergy\ (for $\varphi=0$):
$$\eqalignno{&{1\over N} {\cal F} =  {1\over 24 \pi}(
m_T - |M_T|)^2(m_T +2|M_T|) +{T \over 4\pi
}(m_T^2-M_T^2)\ln(1-\e^{-m_T/T})\cr
&\quad+T \int {\d^2k\over
{(2\pi)^2} } \{ \ln(1-\e^{-\beta \omega_\varphi} )
 - \ln(1 + \e^{-\beta \omega_\psi} ) \}.\hskip12mm
&\eqnn\cr} $$
\medskip
{\it Solutions.}
The derivative of $\cal F$ with respect to $|M_T|$ at $m_T$ fixed is
$${\partial {\cal F}\over\partial |M_T|}={|M_T|T\over 2\pi}
\left[
\ln\bigl(2\cosh(M_T/2T)\bigr)-\ln\bigl(2\sinh(m_T/2T)\bigr)\right].$$
It is convenient to introduce the notation
$$X(\kappa ,T)=\exp\left[{2\pi \over T}\left({1\over \kappa_c}-{1\over \kappa }\right)\right].$$
Then the behaviour of $\cal F$ depends on the position of $X$ with respect
to $2$:
$$X=2\ \Leftrightarrow\ T={2\pi\over \ln 2}\left({1\over \kappa_c}-{1\over \kappa }\right), $$
an equation that has a solution only for $\kappa >\kappa _c$.\par
 For $X<2$, the derivative vanishes at $M_T=0$ and is positive
for all $|M_T|>0$. Instead for $X>2$, the derivative vanishes both at $M_T=0$
and
$$|M_T|=2T\ln\left[\ud(X+\sqrt{X^2-4})\right],\eqnd\eSUSYNTnlsMT $$
which are the two solutions of Eq.~\eNLSMsadTi{a}. The minimum of $\cal F$
is located at the second value \eSUSYNTnlsMT. \par
We find therefore an interesting non-analytic behaviour:
$$\cases{M_T=0 &for $X<2$, \cr
M_T=2T\ln\left[\ud(X+\sqrt{X^2-4})\right] & for $X>2$ .\cr}$$
Eq.~\eNLSMsadTi{a} then yields directly the boson thermal mass
$$m_T=2T\ln\left[\ud(X+\sqrt{X^2+4})\right] .$$
For $\kappa <\kappa _c$ and $T\to 0$, we find the asymptotic behaviour
$$m_T\sim  TX(\kappa ,T),$$
which is exponentially small, and $M_T=0$.\par
For $\kappa >\kappa _c$ and $T\to 0$, both $m_T$ and $M_T$ converge
toward the finite $T=0$ limit with
exponentially small corrections.\par
In the opposite high temperature limit $T\gg | 1/\kappa -1/\kappa _c|$,
we find that $m_T$ is asymptotically proportional to $T$:
$$m_T\sim 2T\ln\bigl((1+\sqrt{5})/2\bigr),$$
while  $M_T=0$.\par
It is not clear whether such a result can survive $1/N$ corrections.
 \medskip
{\it Dimension $d=2$.} In generic dimensions $2\le d\le 3$ a phase
transition is not even possible at finite temperature  and $\varphi=0$
(in two dimensions it is even impossible at zero temperature).
The gap equations take the form
$$\eqalignno{{1\over \kappa }&=\Omega _d(m_T)+T^{d-2}f_d(m_T^2/T^2), &\eqnn \cr
0&=M_T\left[\Omega_d(|M_T|)-T^{d-2}g_d(M_T^2/T^2)-
\Omega _d(m)-T^{d-2}f_d(m_T^2/T^2)\right] ,
\hskip6mm &\eqnn \cr}$$
where $f_d$ and $g_d$ are defined in Eqs.~\eThermfz~and \eThermfiis~respectively.
\par
For $d=2$ it is then convenient to introduce the physical mass $m$ solution of
$${1\over \kappa }=\Omega _2(m). $$
The equations can then be rewritten as
$$\Omega _2(m)=\Omega _2(m_T)+  f_2(m_T^2/T^2)  \eqnn $$
and either $M_T=0$ or
 $$ \Omega_2(|M_T|)- g_2(M_T^2/T^2)-\Omega _2(m_T)- f_2(m_T^2/T^2) =0\,.\eqnn $$
The first equation, which determines $m_T$, is identical to the
Eq. \eNTnlsgapii~already  obtained  for the usual non-linear
$\sigma $ model. At low temperature $m_T=m$ up to exponentially
small corrections. At high temperature
$${T\over m_T}\sim{1\over \pi}\ln(m_T/m)\sim {1\over \pi}\ln( T/m),$$
a consequence of the domination of the zero mode and the UV
asymptotic freedom of the non-linear $\sigma $-model.\par
Combining both gap equations we find
$$ \Omega_2(|M_T|)- g_2(M_T^2/T^2)=\Omega _2(m),$$
which is identical to Eq.~\eTGNNgapiii.\par
The analysis of section \ssNTGNNphase\ for $d=2$ has shown that for $T$ large
the equation has no solution and thus $M_T=0$, but it has a solution for $T$ small.
The situation therefore is similar to what has
been encountered in three dimensions. Again an analysis
of $LL$ propagator and $1/N$ corrections is required to
understand whether this result survives beyond the large $N$ limit.
