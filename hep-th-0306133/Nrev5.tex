%%%%%%%%%%  Nrev5 as of Feb. 28 2003
\def\r{\rm r}
\def\thetab{\theta}
\def\bartheta{\skew1\bar\theta}
\def\nub{\nu}
\def\l{m^2}
\def\lii{m^4}
\def\bphi{\bar\varphi}
%%%%%%%%%%%%%%%%%%%%%%%%%%%%%%%%%%%%%%%%%%%%%%%

\nref\rHohen{Dynamical models are reviewed from the RG point of
view in \lb P.C. Hohenberg and B.I. Halperin, {\it Rev. Mod.
Phys.} 49 (1977) 435.}

 \nref\rMart{The dynamic action associated with the
Langevin equation has been introduced in \lb P.C. Martin, E.D.
Siggia and H.A. Rose, {\it Phys. Rev.} A8 (1978) 423. }

\nref\rDomin{For an early discussion of the renormalization of
dynamic theories in a field theory language see for example \lb C.
De Dominicis and L. Peliti, {\it Phys. Rev.} B18 (1978) 353.}

 \nref\rWitt{
The relation between supersymmetry and dissipative Langevin or
Fokker--Planck equations have been shown in \rf E. Witten, {\it
Nucl. Phys.} B188 (1981) 513 and refs.
\refs{\rFeigel{--}\rEgo}.}\nref\rFeigel{ M.V. Feigel'man and A.M.
Tsvelik, {\it Sov. Phys.--JETP} 56 (1982) 823.}\nref\rNaka{ H.
Nakazato, M. Namiki, I. Ohba and K. Okano, {\it Prog. Theor.
Phys.} 70 (1983) 298}\nref\rEgo{ E. Egorian and S. Kalitzin, {\it
Phys. Lett.} 129B (1983) 320.}

 \nref\rZinnJ{
In the discussion of the dissipative dynamics we follow \lb J.
Zinn-Justin, {\it Nucl. Phys.} B275 [FS18] (1986) 135.}

 \nref\rHalper{
Th dynamic exponent $z$ at order $1/N$ is given in \lb B.I.
Halperin, P.C. Hohenberg and S.K. Ma, {\it Phys. Rev. Lett.} 29
(1972) 1548.}
 \nref\rBaus{The non-linear $\sigma$-model has been considered in
 \lb R.
 Bausch,
H.K. Janssen and Y. Yamazaki, {\it Z. Phys.} B37 (1980) 163.}
%%%%%%%%%%%%%%%%%%%%%%%%%%%%%%%%%%%%%%%%%%%%%%%

\section Dissipative dynamics in the large $N$ limit

We now study a dissipative stochastic dynamics described by a
Langevin equation, in the large $N$ limit. From the point of
critical phenomena, the dissipative Langevin equation describes
the simplest time evolution with prescribed equilibrium
distribution, and allows calculating relaxation or correlation
times or time-dependent correlation functions. We recall that the
correlation time diverges at a second order phase transition, a
phenomenon called critical slowing-down, and this leads to
universal long time evolution \rHohen. \par Purely dissipative
stochastic dynamics is a problem interesting in its own right, but
it will also serve as an introduction to the discussion of
supersymmetric models of section \label{\ssNSUSYsc}, since
correlation functions associated with the dissipative Langevin
equation can be calculated  from a functional integral with a
supersymmetric action. The corresponding algebraic structure
generalizes supersymmetric quantum mechanics. \sslbl\ssCrDY
%
\subsection Langevin equation in the large $N$ limit

A general dissipative  Langevin equation for a scalar field $\varphi$
is a stochastic differential equation of the form
$${\partial   \varphi (t,x)\over \partial t}=-{\Omega \over 2}{\delta{\cal A}
\over \delta \varphi (t,x)}+\nu(t,x), \eqnd\eqlandisp $$
where $ \nu(t,x)$ is a gaussian white noise,
$$ \left< \nu (t,x) \right> = 0\,,\qquad
\left< \nu (t,x)\nu (t',x' ) \right>  = \Omega
\,\delta  (t-t'  )\delta (x-x'  ),\eqnd\ebruitb $$
and the constant $\Omega $ characterizes the amplitude of the noise.
This equation generates a time-dependent field distribution, which converges
at large time towards an equilibrium distribution corresponding to the
functional measure $\e^{-{\cal A}(\varphi)}[\d\varphi]$  if it is normalizable.
\medskip
{\it $O(N)$ symmetric models in the large $N$ limit.}
We now consider a model where $\varphi$ is a $N$-component field and the
static action ${\cal A}(\varphi)$ has an $O(N)$ symmetry of the form \eactONgen:
$${\cal A}(\varphi)=\int\d^d x\left[\ud (\partial_\mu\varphi)^2
+NU(\varphi^2/N)\right] . \eqnd\eNAscalgen $$
The corresponding Langevin equation then reads
$$ \dot \varphi_i (t,x)=-\ud \Omega  \left[-\nabla _x^2 +2
U'( \varphi^2/N)\right]\varphi_i(t,x)+\nu _i(t,x)  \eqnd\eNlangev $$
with
$$ \left< \nu_i (t,x) \right> = 0\,,\qquad
\left< \nu_i (t,x)\nu_j (t',x' ) \right>  = \Omega
\,\delta  (t-t'  )\delta (x-x'  )\delta _{ij}. \eqnd\eNbruit  $$
Here, the components $\nu _i$ of the noise are independent variables
and in the large $N$ limit the central limit theorem applies to $O(N)$ scalar
functions of $\nu$. \par
As a boundary condition, we choose
$$\varphi(t=-\infty ,x)=0\,, $$
which ensures equilibrium at any finite time. \par
We then set
$$\rho (t,x)=\varphi^2(t,x)/N\,,\quad m^2=2 U'(\rho ),$$
and assume, as an ansatz, that $m^2$ goes to a constant in the large $N$ limit. The Langevin equation then becomes linear and can be solved. Introducing the
field Fourier components
$$\varphib(t,x)=\int\dd k\,\e^{ikx}\tilde\varphib(t,k),\quad \nub (t,x)=
\int\dd k\,\e^{ikx}\tilde\nub(t,k),$$
one finds
$$\tilde\varphib(t,k)=\int_{-\infty }^t \d\tau \,\e^{-\Omega (k^2+m^2)(t-\tau )/2}\tilde\nu (\tau ,k),$$
with
$$\left<\tilde\nu_i(t,k)\tilde\nu_j(t',k')\right>={1\over(2\pi)^d}\delta (t-t') \delta (k+k')\delta _{ij}\,.$$
The calculation of $\rho (t,x)$ then involves the quantity
with vanishing fluctuations for $N\to\infty $:
$$\sum_i \tilde\nu_i( \tau _1,k_1)\tilde\nu_i(\tau _2,k_2)
\sim {N\over(2\pi)^d} \delta (k_1+k_2)\delta ( \tau _1-\tau _2).$$
Therefore,
$$\rho (t,x)={1\over(2\pi)^d}\int \dd k\int_{-\infty }^t\d\tau\, \e^{-\Omega (k^2+m^2)(t-\tau ) }={1\over(2\pi)^d}\int {\dd k\over k^2+m^2}=\Omega _d(m),$$
a result that is consistent with the ansatz $m$ constant. One then recognizes the static saddle point equation \esaddleN{c}
in the symmetric phase. \par For the broken phase the ansatz or
boundary conditions have to be slightly modified. One verifies
that provided $m=0$, one can impose
$$\varphi(t=-\infty ,x)=\sigma \,,$$
where $\sigma $ is a constant. Then,
$$\tilde\varphib(t,k)=\int_{-\infty }^t \d\tau \,\e^{-\Omega  k^2 (t-\tau )/2}\tilde\nu (\tau ,k)+\sigma \delta (k),$$
and
$$\rho (t,x)=\sigma ^2/N+{1\over(2\pi)^d}\int \dd k\int_{-\infty }^t\d\tau\, \e^{-\Omega  k^2 (t-\tau ) }=\sigma ^2/N+ \Omega _d(0),$$
where one recognizes the static saddle point equation \esaddleN{c} in the broken phase. \par
However, we will not discuss  the problem of the large $N$ expansion in this formalism further, because we now introduce an alternative supersymmetric formalism.
%
\subsection Path integral solution: Supersymmetric formalism

In the case of the purely dissipative Langevin equation \eqlandisp\ with gaussian white noise, it can be shown that dynamic correlation functions can also be expressed in terms of a functional integral that generalizes supersymmetric quantum mechanics \refs{\rMart{--}\rWitt}.\par
%%%%%%%%%%%%%%%%%%%%%%%%%%%%%%%%%%%%%%%
One introduces a superfield  $\Phi$  function of two Grassmann coordinates
$\bar\theta,\theta$:
$$ \Phi  (t,x;\bar \theta ,\theta )=\varphi (t,x) +\theta \bar \psi(t,x)+ \psi (t,x)\bar \theta +\theta \bar \theta
\bphi (t,x)\,,  $$
and supersymmetric covariant derivatives
$$ {\rm \bar D} ={\partial \over \partial \bar\theta},\quad {\rm  D} = {\partial
\over \partial \theta} -\bar\theta{ \partial \over \partial
t},\eqnd{\egensusy}  $$
which satisfy the anticommutation relations
$$ {\rm D}^{2}= {\rm \bar D}^{2}
=0\,, \qquad  {\rm D} {\rm \bar D} + {\rm \bar D} {\rm D}  =-{ \partial
\over \partial  t} \, .  \eqnd\eantrela $$
The generating functional  of $\Phi$-field correlation functions
then is given by
$${\cal Z}(J)=\int[\d\Phi]\e^{-{\cal S}(\Phi)+J\cdot \Phi} $$
with
$$ {\cal S} (\Phi)= \int \d  \bar\theta\, \d  \theta\, \d  t
\left[{ 2 \over \Omega} \int \d ^{d}x\,\bar{\rm  D} \Phi{\rm  D} \Phi+{\cal
A} (\Phi)\right] . \eqnd{\eactLsup} $$
Here $J(x,\bar\theta,\theta)$ is a source for $\Phi$ field:
$$J\cdot \Phi\equiv \int\d t \,\d^d x\,\d\bar\theta\d\theta\,J(t,x,\bar\theta,\theta)
\Phi(t,x,\bar\theta,\theta).$$
Note that with our conventions for the $\theta$ integration measure
$$\delta^2(\theta-\theta')=(\theta-\theta')(\bar\theta-\bar\theta')\,.$$
\par
The generalized action ${\cal S} (\Phi)$ is supersymmetric.
The corresponding supersymmetry generators
 are
$$ {\rm Q}  ={\partial \over \partial\theta},  \qquad \bar {\rm Q}  ={\partial
\over \partial \bar\theta} + \theta{ \partial \over \partial t}\,.
\eqnd{\egensup} $$
Both {\it anticommute}\/ with ${\rm D}$ and ${\rm \bar D}$ and satisfy
$$ {\rm Q}^ 2  = \bar {\rm Q}^ 2
=0\,, \qquad  {\rm Q} \bar {\rm Q} + \bar {\rm Q} {\rm Q}  ={ \partial
\over \partial  t} \, .  \eqnd \erelant  $$
Let us verify, for instance, that
$ \bar {\rm Q} $ is the generator  of a symmetry. We perform
a variation of $\Phi$ of the form
$$ \delta\Phi  (t, \theta ,\bar\theta )= \bar\varepsilon\bar{\rm
Q}\Phi\,, \eqnd{\esuptr} $$
which in component form  reads
$$  \delta \varphi   = \psi \bar \varepsilon\, , \quad \delta
\psi   =0\,, \quad \delta \bar \psi = \left(\bphi-\dot\varphi\right)
\bar \varepsilon\,,  \quad  \delta \bphi  = \dot\psi\bar\varepsilon\, .
  \eqnd \esusytr   $$
The term  ${\cal A}$ is invariant because it does not  depend on
$t$ and $\bar\theta$  explicitly. For the remaining term, the additional property that $
\bar{\rm Q}  $ anticommutes with  ${\rm D}$ and $\bar{\rm D}$ has to be used:
$$\delta\left[\bar{\rm  D} \phi{\rm  D} \phi\right]= \bar{\rm  D}
\left[\bar\varepsilon\bar{\rm Q}\phi\right]{\rm  D} \phi +\bar{\rm  D} \phi{\rm
D} \left[\bar\varepsilon\bar{\rm Q}\phi\right]=\bar\varepsilon\bar{\rm Q}\left[
\bar{\rm  D} \phi{\rm  D} \phi\right].$$
The variation of the action density thus is a total derivative. A similar argument applies to $Q$. This proves that the action is supersymmetric. \par
This supersymmetry is directly related to the property that
the corresponding Fokker--Planck hamiltonian   is equivalent to a positive hamiltonian.
%
\medskip
{\it Static action.} The static action defines the equilibrium distribution.
In what follows we have in mind a static action of the form (this includes
the action \eNAscalgen)
$${\cal A}(\varphi)=\int\d^d x\left[\ud (\partial_\mu \varphi)^2 +\ud
m^2\varphi^2+ V(\varphi)\right]. $$ The propagator $\Delta $ is
the inverse of the kernel
$$K= -{2\over\Omega}[\bar {\rm D},{\rm D}]-\nabla_x ^2+m^2. $$
Introducing  Fourier components, frequency $\omega$
corresponding to time and momentum $k$ corresponding to space,
we can write this operator more explicitly. From
$$ [\bar {\rm D},{\rm D}]\delta ^2( \theta -\theta ')=2+  i\omega(\theta'-\theta)(\bar\theta+\bar\theta')  $$
one infers
$$\eqalign{K(\omega,{\bf k}, \theta',\theta)&=\left\{-{2\over\Omega}[\bar
{\rm D},{\rm D}]+k^2+m^2\right\} \delta^2(\theta-\theta') \cr
& =-{4\over\Omega}
\left[1+\ud i\omega(\theta'-\theta)(\bar\theta+\bar\theta')\right]
%\cr &\quad
+(k^2+m^2) \delta^2(\theta-\theta'). \cr}$$
To obtain the propagator $\Delta $ in superspace, we note
$$\left([\bar {\rm D},{\rm D}]\right)^2 =-\left( 2\bar {\rm D} {\rm D}+i \omega \right)\left( 2 {\rm D}\bar {\rm D}+i \omega \right)  =   -\omega ^2\,.$$
Then,
$$\Delta ={\Omega \over2}{ [\bar {\rm D},{\rm D}]+  \Omega (k^2+m^2)/2\over
 \omega^2+
  \Omega^2  (k^2+m^2  )^2/4} $$
or, more explicitly,
$$\Delta  ( \omega ,{\bf k} , \thetab',\thetab  ) ={\Omega \left[ 1+
\ud i\omega \left(\theta' -\theta \right)
\left(\bar\theta +\bar\theta ' \right)+ \frac{1}{4} \Omega
\left(k^2+m^2\right)\delta^2(\bar\thetab' -\bar\thetab) \right] \over \omega^2+
 { \Omega^2 \over 4} \left(k^2+m^2 \right)^2}.
\eqnd\eprop $$
%
\subsection Ward--Takahashi (WT) identities and renormalization

{\it WT identities.}
The symmetry associated with the $\rm Q$ generator has a simple consequence,
correlation functions are invariant under a translation of the coordinate $
\theta  $. The transformation \esuptr\ has a slightly more complicated
form. Connected correlation functions $W^{ (n )} (t_i,x_i ,
\theta_i ,\bar\theta _i  t)$ and proper vertices
$\Gamma^{  n  )}  (t_i,x_i , \theta_i ,\bar\theta_i
 )$   satisfy  the WT identities  \sslbl\sssStfWT
$$ \bar{\rm Q} W^{ (n )} (t_i,x_i , \theta_i ,\bar\theta_i   )=0\,, \qquad  \bar{\rm Q}\Gamma^{ (n )} (t_i,x_i , \theta_i ,\bar\theta_i   )=0
\eqnd\eward $$
with
$$ \bar{\rm Q}\equiv\sum^{n}_{k=1} \left({\partial \over \partial
\bar\theta_{k}}+\theta_{k}{ \partial \over \partial t_k} \right). $$
After Fourier transformation over time, the operator $\bar{\rm Q}$
takes the form
$$\bar{\rm Q}= \sum^{n}_{k=1} \left({ \partial  \over \partial  \bar\theta
_{k}}-i\omega_{k}\theta_{k} \right).\eqnd\ewardft $$
One verifies  immediately that the propagator \eprop\ satisfies the identity
\eward.
Actually, the general solution can be written as
$$\widetilde W^{(n)}(\omega, {\bf k},\theta,\bar\theta)
=\exp\left[-{i\over4n}\sum_{k,l}(\theta_k-\theta_l)
(\omega_k-\omega_l)
(\bar\theta_k+\bar\theta_l)\right]F^{(n)}(\omega,
{\bf k},\theta,\bar\theta) ,$$ where the function $F^{(n)}(\omega,
{\bf k},\theta,\bar\theta)$ now is invariant under translations of
both $\theta$ and $\bar\theta$.
\medskip
{\it Example: a two-point function}. Let us explore the implications of WT identities for a two-point function. As the relations \eqns{\egensup,\erelant}
show, supersymmetry implies translation invariance on time and $\theta$.
Therefore, any two-point function $W^{(2)}$ can be written as
$$W^{(2)}= A ( t_1 -t_2 ) +  ( \theta_1 - \theta_2
 ) \left[  ( \bar\theta_1 + \bar\theta_2  )B ( t_1 -t_2  ) +
 ( \bar\theta_1 - \bar\theta_2  )C ( t_1 -t_2 )\right] .\eqnn $$
The WT identity \eward\ then implies
$$2B(t) = {\partial A \over \partial t}.\eqnn $$
The WT identity does not determine  the function $C$. An
additional constraint comes from causality. For the two-point
function, it implies that the coefficient of $\theta_1
\bar\theta_2$ vanishes for $t_1 < t_2$ and the coefficient of
$\theta_2 \bar\theta_1$ for $t_2 < t_1$. The last function is thus
determined, up to a possible distribution localized at $t_1 =
t_2$. One  finds
$$2C(t)= - \epsilon(t) {\partial A \over \partial t}, \eqnn $$
where $\epsilon(t)$ is the sign of $t$, and, therefore,
$$W^{(2)}= \left\lbrace 1+ \ud \left(\theta_1 -\theta_2 \right)
\left[ \bar\theta_1 +\bar\theta_2 -  (\bar\theta_1 -\bar\theta_2  )\epsilon
 (t_1-t_2 ) \right]{\partial \over \partial t_1 }
\right\rbrace  A ( t_1-t_2 ).\eqnd\etwopt $$
%
\medskip
{\it Renormalization.}
In the special case of the supersymmetric dynamical action \eactLsup, a
comparison between the two explicit quadratic terms in $\Phi$ of the action
yields the relation between dimensions \rZinnJ
$$ \left[ t \right] - \left[ \bar\theta \right] -
\left[\theta \right] = 0 \ \Rightarrow\left[ \d  t \right] +
\left[ \d  \bar\theta \right] + \left[ \d \theta
\right] =0\,. \eqnd{\edim} $$
(We recall that since integration and differentiation over anticommuting
variables are equivalent operations, the dimension of $ \d \theta
$ is $ -[ \theta] $.)  \par
Therefore, the term proportional to $ {\cal A}  (\Phi  ) $ in the
action has the same canonical  dimension as in the static case:
the power counting is thus the same and the dynamic theory is always
renormalizable in the same space dimension as the static theory. Note
that Eq.~\edim\ also implies
$$ 2[\Phi]  =d+ [t]\,,$$
an equation that relates the dimensions of field and time. \par
One then verifies that supersymmetry is preserved by renormalization
and that the most general supersymmetric renormalized action has the form
$$ {\cal S}_{\r}(\Phi)= \int \d  \bar\theta\, \d \theta\,
\d t \left[{ 2 \over \Omega} Z_\Omega \int \d ^{d}x\, {\rm \bar D}\Phi
{\rm D}\Phi +{\cal A}_{\r}(\Phi) \right] , \eqnd\eactssr $$
where $Z_\Omega$ is the renormalization of the parameter $\Omega $, and thus also of the
scale of time. \par
The renormalized Langevin equation thus remains dissipative; the drift
force derives from an action.
%
\subsection $O(N)$ symmetric models in the large $N$ limit: supersymmetric formalism

We now consider again the  $O(N)$ symmetric Langevin equation \eNlangev\  for an $N$-component field $\varphi$,  corresponding to the static action  \eNAscalgen:
$${\cal A}(\varphi)=\int\d^d x\left[\ud (\partial_\mu\varphi)^2
+NU(\varphi^2/N)\right] . $$
We apply the usual strategy and introduce in the dynamic theory two
superfields $L$ and $R$, which have the form
$$\eqalign{L( \theta)&=l+\theta\bar\ell+\ell\bar\theta+\theta\bar\theta
\lambda\,, \cr
R( \theta)&=\rho+\theta\bar\sigma+\sigma\bar\theta +\theta\bar\theta s\,.
\cr}$$
We implement the condition $R=\Phi^2/N$ by an integral over $L$. The functional
integral takes the form
$${\cal Z}=\int[\d\Phi][\d R][\d L]\e^{-{\cal S}(\Phi,R,L)}  $$
with
$${\cal S}(\Phi,R,L)= \int\d t\d\bar\theta\d\theta\d^d x
\left[{2\over\Omega}{\rm \bar D}\Phi{\rm
D}\Phi+\ud(\partial_\mu\Phi)^2+NU(R) +\ud L(\Phi^2-NR)\right]. $$
We  integrate over $N-1$ components of the $\Phi$ field, keeping one
component $\Phi_1=\phi$ as a test-component. The large $N$ action then reads
$$\eqalignno{{\cal S}_N&=
\int\d t\d\bar\theta\d\theta\d^d x \left[{2\over\Omega}{\rm \bar
D}\phi{\rm D}\phi +\ud(\partial_\mu\phi)^2+NU(R) +\ud
L(\phi^2-NR)\right] \cr &\quad +\ud(N-1)\,{\rm
Str}\,\ln\left\{-2\Omega^{-1} [\bar {\rm D},{\rm D}] -\nabla^2_x
+L\right\}, \cr}$$ where ${\rm Str}$ means trace in the sense of
space, time  and Grassmann coordinates.
\par
At leading order at large $N$, the functional integral can be
calculated by the steepest descent method. The two first saddle
point equations, obtained by varying the superfields $\phi$ and
$R$, are \eqna\eStsad
$$\eqalignno{\left(-{2\over\Omega}[\bar {\rm D},{\rm D}]+L\right)\phi&=0\,,&\eStsad{a} \cr
L-2U'(R)&=0\,.&\eStsad{b} \cr} $$
The last saddle point equation, obtained by varying $L$, involves the
$\phi$ super-propagator $\Delta $:
$$\eqalignno{R-\phi^2/N&={1\over(2\pi)^{d+1}}\int\d\omega \,\d^d k
\, \Delta (\omega ,{\bf k} , \thetab,  \thetab),&\eStsad{c}\cr}$$
where $\omega $ and $\bf k$ refer to the
time and space Fourier components, respectively. \par
The super-propagator  can be calculated, for example, by solving
$$\left\{-2\Omega^{-1} [\bar {\rm D},{\rm D}] +k^2
+L\right\}\phi=J\,.$$ For Eq.~\eStsad{c},  the $\phi$-propagator
in presence of a  $L$ field is needed, but only  for $L$ of the
form
$$L=\l+\theta\bar\theta\lambda\,,$$
with $\l,\lambda$ constants. Then, setting
$$G=k^2+\l+2i\omega/\Omega\,$$
one finds
$$\Delta  ( \omega , {\bf k} , \thetab' ,\thetab  )
={4\Omega^{-1}+G\theta'\bar\theta'+G^*\theta\bar\theta
-\lambda\theta'\bar\theta'\theta\bar\theta \over
GG^*+4\lambda/\Omega} -{\theta'\bar\theta\over G}-
{\theta\bar\theta'\over G^*}\,.$$
At coinciding points $\theta=\theta'$ it reduces to
$$\Delta  (\omega , {\bf k} , \thetab ,\thetab
 )={4\Omega^{-1}+2(k^2+\l)\theta\bar\theta \over GG^*+4\lambda/\Omega}
-{2(k^2+\l)\theta\bar\theta \over GG^*}
\,. $$
%$${\Omega\over  \omega^2+ \displaystyle{
%\Omega^{2} \over 4} \left(k^2+\l \right)^2}.$$
After integration over $\omega$, one obtains %assuming $\lambda$ real!
$${1\over2\pi}\int\d\omega\,\Delta  (\omega , {\bf k} , \thetab ,\thetab
 ) \equiv \bar\Delta({\bf k},\theta)={1+\Omega(k^2+\l)\theta\bar\theta/2\over
\sqrt{(k^2+\l)^2+4\lambda/\Omega}}-\ud\Omega \theta\bar\theta\,. $$
Eq.~\eStsad{c} then becomes
$$R-\phi^2/N={1\over(2\pi)^{d}}\int\d^d k
\,\bar\Delta ({\bf k} , \thetab ).$$ All saddle point equations
reduce to the static equations for $\lambda=0$, and then $F=s=0$,
which implies that supersymmetry is preserved, and the ground
state energy vanishes. Then, no further analysis is necessary. Of
course, we know that supersymmetry is broken when the measure
$\e^{-{\cal A}(\varphi)}[\d\varphi]$ is not normalizable. But this
effect  cannot be seen at leading order in perturbation theory nor
in the large $N$ limit. \par Finally, note  that the
super-propagator formalism  simplifies dynamic $1/N$ calculations.
\smallskip
{\it The action density: saddle point equations in component form.} Alternatively, one can start from the action density   for constant scalar fields and vanishing Grassmann fields
$$\eqalignno{{\cal E}&=-{2\over \Omega }F^2+NsU'(\rho)+{1\over 2}\lambda (\varphi^2-N\rho)+{1\over2}\l(2F\varphi-Ns)\cr&\quad +
{N\Omega \over 4}\int{\d^d k \over (2\pi)^d}\left(\sqrt{(k^2+\l)^2+4\lambda /\Omega }-k^2-\l\right).&\eqnn\cr}$$
By differentiating $\cal E$ with respect to all parameters,
one recovers the saddle point equations in component form:
$$\eqalignno{F&=\Omega \l\varphi/4\,,\quad \l F+\lambda \varphi=0\,, &\eqnd\eLangsada \cr
\l&=2U'(\rho)\,,\quad \lambda =2sU''(\rho),& \eqnd\eLangsadb \cr}$$
and
\eqna\eLangsadx
$$ \eqalignno{\rho-\varphi^2/N&={1\over(2\pi)^d}\int{\d^d k \over \sqrt{(k^2+\l)^2+4\lambda /\Omega }}&\eLangsadx{a}\cr
s-2F\varphi/N&={\Omega \over 2}{1\over(2\pi)^d}\int \d^d k\left[ {(k^2+\l)\over \sqrt{(k^2+\l)^2+4\lambda /\Omega }}-1\right].&\eLangsadx{b}\cr}$$
Note that the same action density is obtained in a large $N$
variational calculation, starting from
$${\cal S}_0(\Phi,L) = \int\d t\d\bar\theta\d\theta\d^d x
\left[{2\over\Omega}{\rm \bar D}\Phi{\rm
D}\Phi+\ud(\partial_\mu\Phi)^2  +\ud L(\Phi-\Phi_0)^2)\right],$$
but then the two equations \eLangsadx{} are constraints, and only
$\Phi_0$ and $L$ are variational parameters. \par
Eliminating $F$ between the two equations \eLangsada, one finds
$$\varphi(\lambda +\Omega\lii/4)=0\,.\eqnn $$
This equation has two solutions: $\varphi=0$ which corresponds to
the $O(N)$ symmetric phase, $\lambda +\Omega\lii/4=0$ which
corresponds to a broken massless phase. \par Eliminating
$F,s,\rho$ from the action density  using the saddle point
equations, one obtains the ground state energy density
%%% **********************
$$\eqalign{{1\over N}{\cal E}&=-{1\over2}\lambda \varphi^2- {\lambda
\over2(2\pi)^d}\int{\d^d k \over \sqrt{(k^2+\l)^2+4\lambda /\Omega
}}\cr\quad&+{ \Omega \over 4}\int{\d^d k \over
(2\pi)^d}\left(\sqrt{(k^2+\l)^2+4\lambda /\Omega }-k^2-\l\right).
\cr}$$ In the symmetric phase the derivative of $\cal E$ with
respect to $\lambda $ is
$${\partial {\cal E}\over \partial \lambda }={N\over \Omega }{\lambda
\over(2\pi)^d}\int{\d^d k \over \left[(k^2+\l)^2+4\lambda
/\Omega\right]^{3/2} }.$$ The minimum is at $\lambda =0$,  that is
at the supersymmetric point, where $\cal E$ vanishes. \par In the
broken symmetry phase $\lambda $ is non-positive. A short
calculation shows that again $\cal E$ is positive. The minimum is
reached for $\l=0$, and therefore $\lambda =0$. The minimum again
is supersymmetric and $\cal E$ then vanishes at the minimum,
independently of the value of $\varphi$.
%
\subsection Quartic potential

We now specialize to the  quartic potential
$$U(R)=\ud rR+{u\over 4!}R^2. $$
Then, the integral over $R$ can be performed, leading to a contribution
$$L=r+\frac{1}{6}uR\ \Rightarrow\ \delta{\cal S}_N=- 3N  (L-r)^2/2u
.$$ Note that for $u>0$ the usual static results are recovered, a
symmetric phase for $r>r_c $ and a broken symmetry phase
otherwise. For $u<0$ one finds the opposite situation, and there
is no sign that the situation is pathological form the static
point of view. The absence of an equilibrium state requires higher
order calculations.
\smallskip
{\it RG equations.}
The RG differential operator \eRGgamln\ acting on dynamic correlation functions takes at $T_c$  ($r=r_c$) the  form
$${\rm D}_{\rm RG}=  \Lambda { \partial \over \partial  \Lambda } +\beta(g){\partial
\over \partial g}+\eta_ \Omega (g)\Omega{ \partial \over \partial \Omega}-{n
\over 2}\eta (g) ,  \eqnn $$
where $g=u\Lambda ^{4-d}/N$ and $\eta_ \Omega(g)$ is a new independent RG function related to the renormalization constant $Z_\Omega $.\par
The solution of the RG equation for the two-point function $\widetilde W^{  (2  )}$ leads to the scaling form
$$\widetilde W^{(2)}  (p,\omega , \thetab  =0 )\sim  p^{-2+\eta -z} G^{  (2  )}  ( \omega /p^{z}   ).
\eqnn $$ The dynamic exponent $z$ which also characterizes, near
the critical point, the divergence of  the correlation time in the
scale of the correlation length, keeps at leading order its
classical value $z=2$.\par At order $1/N$  the $\left<LL\right>$
propagator is needed. In the symmetric phase $\varphi=0$, it is
given by
$$[\Delta_L]^{-1}=-{3\over u}\delta^2(\bar\theta-\bar\theta')+{1\over(2\pi)^{d+1}}
\int\d^d k\,\d \omega'' \,\Delta( \omega '',{\bf k}, \theta,\theta')\Delta(\omega- \omega '',{\bf p-k},\theta,\theta') .$$
In the infrared limit $\omega, {\bf k}\to 0$, it can be evaluated and used to
calculate the $\phi$ two-point function at order $1/N$.
The value of the dynamic exponent $z$ at order $1/N$ follows. It can be written as \rHalper
$$z=2+c\eta\,, \quad c=\left(4\over  4-d\right)\left\{ {dB(\ud d-1, \ud d-1)
\over 8\int_0^{1/2}\d x\,[x(2-x)]^{d/2-2}} -1\right\},$$
where $\eta$ has been given in section \sssfivNRT\ and $B(\alpha ,\beta )$ is
the mathematical $\beta $-function.
\medskip
{\it The dissipative non-linear $\sigma $-model.} Within the framework of the large $N$ expansion, we have shown that the results of the static $(\phib^2)^2$ could be reproduced by the non-linear $\sigma $-model. This result generalizes to the dynamic theory \rBaus.
In terms of the superfield $\Phi$, the functional integral takes a form
$${\cal Z}=\int[\d\Phi]\delta (\Phi^2-N)\exp\left[-{1\over T}\int\d t\d\bar\theta\d\theta\d^d x
\left({2\over\Omega}{\rm \bar D}\Phi{\rm
D}\Phi+\ud(\partial_\mu\Phi)^2\right)\right].$$
Therefore, the strategy is the same as in the static case. Since supersymmetry is not broken, the saddle point equations again reduce to the static equations.

%\listrefs


%\bye
