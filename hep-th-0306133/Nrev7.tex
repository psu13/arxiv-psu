%%%%%%  Nrev7.tex as of Feb. 2003
\def\r{{\rm r}}
\def\phib{\phi}
\def\varphib{\varphi}
\def\psib{\psi}
\def\MS{$\overline{\rm MS}$}
%\nref\rbook{J. Zinn-Justin, 1989, {\it Quantum Field
%Theory and Critical Phenomena}, in particular chap.~28 of third ed., Clarendon
%Press (Oxford 1989, fourth ed.  2002).}
\nref\rKiLin{The study of finite temperature quantum field theory
has been initially motivated by the discussions of cosmological
problems \rf D.A. Kirznits {\it JETP Lett.} 15 (1972) 529 and
refs. \refs{\rKirzn,\rLinde}.}\nref\rKirzn{ D.A. Kirznits and A.D.
Linde, {\it Phys. Lett.} B42 (1972) 471; {\it Ann. Phys.} 101
(1976) 195.}\nref\rLinde{ A.D. Linde {\it Rep. Prog. Phys.} 42
(1979) 389.} \nref\rBWDJ{The possibility of phase transitions in
heavy ion collisions has generated additional interest \rf
 C.W. Bernard, {\it Phys. Rev.} D9 (1974) 3312  and refs. \refs{\rWein{--}\rGross}.}
 \nref\rWein{S. Weinberg, {\it
 Phys. Rev.} D9 (1974) 3357.}\nref\rDolan{  L. Dolan and R. Jackiw, {\it Phys. Rev.}
 D9 (1974) 3320.} \nref\rGross{D.J. Gross, R.D. Pisarski and L.G. Yaffe, {\it
 Rev. Mod. Phys.} 53 (1981) 43.}
\nref\rTreview{Among the various reviews and textbooks see, for
example,  \rf N.P. Landsman and C. van Weert, {\it Phys. Rep.} 145
(1987) 141 and refs. \refs{\rKapus{--}\rBlaizot}.}
\nref\rKapus{J.~I.~Kapusta, {\it Finite Temperature Field Theory},
Cambridge Univ. Press (Cambridge 1989).}\nref\rLeBel{ M. Le
Bellac, {\it Thermal Field Theory}, Cambridge Univ. Press
(Cambridge 1996).}\nref\rMeyer{ H. Meyer-Ortmanns, {\it Rev. Mod.
Phys.} 68 (1996) 473.}\nref\rBlaizot{ J.P. Blaizot, E. Iancu, {\it
Phys. Rep.} 359 (2002) 355, hep-ph/0101103.} \nref\rTRGeq{RG
equations in the context of finite temperature dimensional
reduction are discussed in \rf N.P. Landsman, {\it Nucl. Phys.}
B322  (1989) 498.} \nref\rZJTN{J. Zinn-Justin, SACLAY preprint
-SPH-T-00-055, hep-ph 0005272.} \nref\rTdimred{Early articles on
dimensional reduction include \rf P. Ginsparg, {\it Nucl. Phys.}
B170 (1980) 388 and refs.
\refs{\rAppel{--}\rKajan}.}\nref\rAppel{T. Appelquist and R.D.
Pisarski, {\it Phys. Rev.} D23 (1981) 2305.}\nref\rNadka{ S.
Nadkarni, {\it Phys. Rev.} D27 (1983) 917, {\it ibidem} D38 (1988)
3287.}\nref\rBraaten{ E. Braaten, {\it Phys. Rev. Lett.} 74 (1995)
2164.}\nref\rKajan{ K. Kajantie, K. Rummukainen and M.E.
Shaposhnikov, {\it Nucl. Phys.} B407 (1993) 356.} \nref\rCuSaYe{A.
Chubukov, S. Sachdev and J. Ye, {\it Phys. Rev.} B49 (1994)
11919.} \nref\rNTGN{Numerical simulations concerning the NJL $2+1$
model with $U(1)$ chiral symmetry are reported in \rf
%%The (2+1)-dimensional Gross-Neveu model with a
%U(1) chiral symmetry at non-%%zero temperature
 S.J. Hands, J.B. Kogut, C.G. Strouthos,
   {\it Phys. Lett.} B515 (2001) 407, hep-lat/0107004.
%S. Hands, D.N. Walters, hep-lat/0209140
%% Evidence for BCS Diquark Condensation in the 3+1d Lattice NJL Model
}

\nref\rDaMaRa{ The thermodynamics of the Gross--Neveu and and
Nambu--Jona-Lasinio models at all temperatures and densities at
$d=2$ for $N\to\infty$ is discussed and the existence of
instantons responsible of the symmetry restoration is demonstrated
in \rf R.F. Dashen, S.K. Ma and R. Rajaraman, {\it Phys. Rev.} D11
(1975) 1499.} \nref\rBCMPG{More recently a more complete analysis
has appeared in A. Barducci, R. Casalbuoni, M. Modugno and G.
Pettini, R. Gatto, {\it Phys. Rev.} D51 (1995) 3042.}

 \nref\rSharev{For a review see, for example,\rf
 M.E. Shaposhnikov, {\it Proc. Int. School of Subnuclear Phys.} (Erice
 1996), World Scientific, hep-ph/9610247.}
\nref\rTgauge{Finite temperature calculations in gauge theories are reported in \rf
 P. Arnold and C. Zhai, {\it Phys. Rev.} D51 (1995) 1906 and
 refs.~\refs{\rFodor,\rArnold}.}\nref\rFodor{
Z. Fodor and A. Hebecker, {\it Nucl. Phys.} B432 (1994) 127.}
\nref\rArnold{ P. Arnold and O. Espinosa, {\it Phys. Rev.} D47
(1993) 3546.} \nref\rTSchwinger{The Schwinger model is solved in
\rf
 A.V. Smilga, {\it Phys. Lett.} B278 (1992) 371.}
%%%%%%%%%%%%%%%%%%%%%%%%%%%%
\section Finite temperature field theory in the large $N$ limit

In this section, we first recall a few general properties of Statistical Quantum Field Theory (QFT) at equilibrium, that is QFT at finite temperature \refs{\rKiLin{--}\rTreview}. We argue that when the temperature varies, one generally observes a crossover from a zero temperature $d$ dimensional theory to an effective $d-1$ dimensional field theory at high temperature, a phenomenon called dimensional reduction. This corresponds also to a classical limit with a transition from a statistical quantum field theory to a statistical classical field theory. Note that high temperature here refers to an ultra-relativistic limit where the temperature, in energy unit, is much larger than the physical masses of particles, so that bosons can produce a classical field.  In perturbation theory it is often impossible to describe this crossover and therefore large $N$ techniques may again be useful.
 \sslbl\scFTQFT \par
We illustrate these ideas with several standard examples, $\phi^4$ field theory, the non-linear $\sigma $ model, the Gross--Neveu model and gauge theories.
%
\subsection Finite temperature QFT: general remarks

We first recall some general properties of QFT at thermal equilibrium in $(1,d-1)$ dimensions. We introduce the mode expansion of fields in the euclidean time variable, discuss the conditions under which statistical properties of finite temperature QFT in $(1,d-1)$ dimension can be described by an effective local classical statistical field theory in $d-1$ dimensions, and indicate how to construct it explicitly.
%
\smallskip
{\it Partition function.} Equilibrium properties of QFT
at finite temperature can be derived from the partition function ${\cal Z}=\tr\e^{-H/T}$, where $H$ is the quantum hamiltonian   and $T$ the
temperature (interesting physics related to the addition of
a chemical potential will not be discussed here).
 We will consider,
first,  a  euclidean action ${\cal S}(\phi)$ describing a scalar
boson field $\phi$. The partition function is given by the
functional integral
$${\cal Z}=\int[\d\phi]\exp\left[-{\cal S}(\phi)\right],\eqnd\eFTZp$$
where  ${\cal S}(\phi)$ is the integral of the euclidean
lagrangian density ${\cal L}(\phi)$:
$${\cal S}(\phi)=\int_0^{1/T}\d\tau\int\d^{d-1} x\,{\cal L}(\phi), $$
and the field $\phi$ satisfies {\it periodic boundary conditions}  in
the (euclidean or imaginary) time direction:
$$\phi(\tau=0,x)=\phi(\tau=1/T,x).$$
A QFT may also involve fermions. Fermion fields
$\psi(\tau,x)$, by contrast, satisfy anti-periodic boundary conditions:
$$ \psi(\tau=0,x)=-\psi(\tau=1/T,x).$$
\smallskip
{\it Mode expansion.} As a consequence of the finite temperature boundary conditions, fields have a
Fourier series expansion in the euclidean time direction with quantized
frequencies $\omega_n$ (also called Matsubara frequencies). For boson fields
$$\phib(t,x)=\sum_{n\in {\Bbb Z}}\e^{i\omega_n t}\phib_n(x) ,
\quad \omega_n=2n\pi T\,.
\eqnd\emodexp $$
In the case of fermions, anti-periodic boundary conditions lead to the expansion
$$\psib(t,x)=\sum_{\omega_n=(2n+1)\pi T}\e^{i\omega_n t}\psib_n
(x).\eqnd\efmodexp $$
\smallskip
{\it Remark.} The mode expansion \eqns{\emodexp} is well suited to simple situations where fields belong to a linear space. In the case of
non-linear $\sigma $ models or non-abelian gauge theories, the decomposition in modes leads to problems, because it conflicts with the geometric structure.
\medskip
{\it Classical statistical field theory and thermal mass.}  The quantum partition function \eFTZp\ can also be considered as the partition
function of a classical statistical field theory in $d$ dimensions. In this interpretation finite temperature for the quantum partition function  \eFTZp\ corresponds to a finite size $ \beta =1/T$ in one direction for
the classical partition function. The zero
temperature limit of the quantum theory corresponds to an infinite volume limit of the classical theory.
  \par
An important parameter then is the {\it thermal mass}\/ $m_T$, inverse of the correlation length $\xi_T $, which characterizes the decay of correlations in space directions. A cross-over is expected between a $d $-dimensional
behaviour when the correlation length $\xi_T$ is small compared with $
\beta $,  that is the mass scale $m_T$ is large compared with the
temperature $T$, to the $(d-1)$-dimensional behaviour when $m_T$ is small compared with $T$. Moreover, in this limit macroscopic properties and correlations for  momenta much
smaller than the temperature $T$ or distances much larger than $ \beta
$, can be described by an effective $(d-1)$-dimensional local field theory.
As we shall see, this corresponds to a classical limit, in the sense that classical fields replace quantum fields.
Within this framework, the temperature then plays the role of a large momentum cut-off.
 \par
The ratio $m_T/T$ can be expected to be small in several situations, at high temperature and near a finite temperature phase transition. Moreover, it is also small at low temperature in a third peculiar situation, when a symmetry is broken at zero temperature and no phase transition is possible at finite temperature.
\smallskip
{\it Finite temperature renormalization group.} From the classical
statistical interpretation one learns that general results
obtained in the study of finite size effects also  apply here.
Correlation functions satisfy the renormalization group (RG)
equations of the corresponding $d $-dimensional classical theory
\refs{\rTRGeq,\rZJTN}. Indeed, RG equations are  related to short
distance singularities  and are, therefore, insensitive to finite
sizes. A finite size  affects only  solutions of the RG equations,
because a new dimensionless  RG invariant parameter becomes
available, here for instance the ratio $ m_T/T$.  \par At high
temperature, the ratio $m/T$, where $m$ is the physical mass
scale, goes to zero. This does not automatically imply that
$m_T/T$ is small because  the parameter $m_T$ is identical with
$m$  only in the tree approximation. By solving the RG equations
in terms of the effective coupling at the temperature scale $T$,
one infers that if the effective coupling goes to zero at high
temperature, then the ratio $m_T/T$ really becomes small. Two
examples will be met: the first  corresponds to theories where the
free field theory is an IR fixed point, like $\phi^4_4$ scalar
field theory or QED${}_4$, the second corresponds to UV
asymptotically free field theories. Conversely, when a non-trivial
IR fixed point is present the ratio $m_T/T$ goes to a constant. At
high temperature, one then has  to rearrange the initial
perturbation theory  by adding and subtracting a mass term to
suppress fictitious perturbative large IR contributions.
%%%%%%%%%%%%%%%%%%%%%%%%%%%
\smallskip
{\it Zero mode and large distance behaviour.}
We consider first the example of a free scalar field
theory with the action
$${\cal S}(\phi)=\ud\int_0^{1/T}\d t\int\d^{d-1} x\left[ ( \partial _t \phi)^2+ (\nabla _x\phi )^2+m^2\phi^2\right]. $$
After introducing the mode expansion \emodexp\ into the action and integrating over time, one obtains a $(d-1)$-dimensional euclidean field theory with an infinite number of fields, the modes $\phi_n(x)$. The form of the action,
$${\cal S}(\phi)={1\over 2T}\int\d^{d-1} x\sum_n\left[|\nabla _x\phi_n (x)|^2+
(m^2+4\pi^2n^2 T^2)|\phi_n(x) |^2\right],$$ shows that $\phi_n$
has a mass $\sqrt{m^2+4\pi^2n^2T^2}$. Therefore, at high
temperature all modes become very massive except the zero-mode,
whose mass $m$ governs the decay of $\phi(t,x)$-field correlation
functions in space directions (here $m_T=m$). The large distance,
low momentum, physics can entirely described by an effective
$(d-1)$-dimensional field theory involving only the zero mode.
\par Note, however, that all
fermion modes  become very massive at high  temperature due to the
anti-periodic boundary conditions \efmodexp and generally
decouple.
\par In an interacting theory, if
$m_T$ remains much smaller than the temperature, one expects still
to be able to describe low momentum physics in terms of an
effective $(d-1)$-dimensional euclidean local field theory. The
thermal mass $m_T$ is the mass of the zero-mode in the effective
theory, and all other scalar and fermion modes have masses at
least of order $T$. The effective theory can thus be constructed
by integrating out perturbatively all non-zero modes and
performing a local expansion of the resulting effective action. As
mentioned above, in this framework the temperature $T$ acts as a
large momentum cut-off.
\medskip
{\it Dimensional reduction.}
We now briefly outline the construction of the $(d-1)$-dimensional effective theory in the example of a general scalar QFT, assuming that the mass $m_T$ of the zero-mode remains indeed much smaller than $T$ \refs{\rTdimred,\rSharev}.  We first separate the zero-mode $\varphi(x)$, setting
$$\phi( t,x)=\varphi(x)+\chi(t,x), \eqnd\eFTmodfc $$
where  $\chi$ the
sum of all other modes (Eq.~\emodexp):
$$\chi(t,x)=\sum_{n\ne 0}\e^{i\omega_n t}\phi_n
(x), \quad \omega_n=2n\pi T\,. \eqnn $$
The action ${\cal S}_T(\varphi) $ of the effective theory is then obtained by integrating over $\chi$:
$$\e^{-{\cal S}_T(\varphi)}=\int[\d\chi]\exp[-{\cal S}
(\varphi +\chi)]. \eqnd\eFTefact $$
\smallskip
{\it Leading order.} From the point of view of the $\chi$
integration, the tree approximation corresponds to setting
$\chi=0$ and one simply finds
$${\cal S}_T(\varphi)={1\over T} \int\d^{d-1} x\,{\cal L}(\varphi),\eqnn $$
an action that is obviously local.  \par We note that $T$ plays,
in this leading approximation, the formal role of $ \hbar$, and
the small $T$ expansion corresponds to a loop expansion.
 If the ratio $T/ \Lambda$, which is always assumed to be small,
 is the relevant expansion parameter, which means that the perturbative expansion
 is dominated by large momentum (UV) contributions, then the effective $(d-1)$
dimensional theory can still be studied by perturbative methods. This is
expected when the number $d-1$ of space dimensions is large and field
theories are non-renormalizable. However, another dimensionless
ratio can be found, $m/T$, which at high temperature
is small. This may be the relevant expansion parameter for theories that
are dominated by small momentum (IR) contributions,  a problem that
arises in low dimensions. Then, perturbation theory
is no longer possible or useful.   Actually, the relevant
parameter in the full effective theory  is $m_T/T$.
Therefore, the contributions to the mass of the zero-mode due to quantum
and thermal fluctuations need to be investigated.
\smallskip
{\it Loop corrections.} The integration over non-zero modes
generates non-local interactions.  To study long wave length
phenomena, one can, however, perform  a {\it local expansion}\/ of
the effective action, expansion that breaks down at momenta of
order $T$.  In general, higher order corrections coming from the
integration over non-zero modes generate terms that renormalize
the terms already present at leading order, and additional
interactions are suppressed by powers of $1/T$.  Exceptions are
provided by gauge theories where new low dimensional interactions
are generated as a consequence of the explicit breaking of the
$O(1,d-1)$ symmetry.
\smallskip
{\it Renormalization in the effective theory.} If the initial $(1,d-1)$ dimensional theory has been
renormalized at $T=0$, the complete theory is finite in the formal infinite cut-off
limit because finite size effects do not affect divergences. However, as a consequence of the zero-mode subtraction, cut-off
dependent terms may appear in the reduced $(d-1)$-dimensional action. These terms provide the necessary counter-terms that render the perturbative expansion of the effective field theory finite.
%%%%%%%%%%%%%%%%%%%%%%%%%%%%%%%%%
\def\Pib{ \Pi}
\subsection Scalar quantum field theory at finite temperature for $N$ large

General finite size effects and, therefore, as we have already
discussed, finite temperature physics involve crossover phenomena
between different dimensions when the temperature varies from zero
to infinity. This severely limits the applicability of
perturbation theory (even with RG improvement), in particular
because IR divergences are more severe in lower dimensions. Large
$N$ techniques, however, which are rather insensitive to changes
in the number of dimensions are, therefore, particularly well
suited to study a crossover situation \rZJTN.\par We first
consider again the self-interacting scalar field with $O(N)$
symmetric action, studied at zero temperature in section
\ssNbosgen. The field $\phib$ is an $N$-component vector and the
hamiltonian  \sslbl\ssNTphig
$$ {\cal H}(\Pib, \phib)={1\over2} \int \d ^{d-1} x \, \Pib^2(x)+\Sigma
(\phib) \eqnn $$
with
$$\Sigma (\phib)= \int \d ^{d-1} x \left\lbrace\ud
\left[ \nabla \phib (x) \right]^{2}+ NU\bigl(\phib^2(x) /N\bigr)
\right\rbrace . \eqnd\eactred $$ A cut-off $\Lambda$, as usual, is
implied that renders the QFT UV finite. The  quantum partition
function \eFTZp\ at finite temperature $T$,  $$ {\cal Z}= \int
\left[ \d \phib (x) \right] \exp \left[-{\cal S}(\phib)\right] ,
\eqnn $$ can then be expressed in terms of the action
%$${\cal Z}=\int[\d\phib]\exp[-{\cal S}(\phib)]\,, \eqnn $$
$${\cal S}(\phib)=\int_0^{1/T} \d t\left[\int\d^{d-1} x\ud(\d_t\phib)^2
+\Sigma (\phib)\right], \eqnn $$
where the field $\phib$ satisfies periodic boundary conditions in the euclidean time direction.
%%%%%%%%%%%%%%%%%%%%%%%%%%%%%%%%%%
 \par
Following Eq.~\eactONef, the action density at finite temperature
and large $N$ is
$${1\over N}{\cal F}(\rho,\sigma,m_T)=U(\rho)+\ud m_T^2 (\sigma^2-N\rho)+
{1\over 2\beta V_{d-1} }\tr\ln \left[ - \nabla^2  + m_T^2 \right]
,\eqnd\EBfiniteT
$$
where $V_{d-1}$ is the $d-1$ volume. The $\tr\ln$ contribution is now modified
by the boundary conditions:
$$ \eqalignno{{1\over V_{d-1}}\tr\ln(-\nabla^2 + m_T^2) &=
\int {\d^{d-1}k\over {(2\pi)^{d-1}} }\sum_ {n\in {\Bbb Z}}
 \ln(\omega_n^2+ k^2+m_T^2) \cr &=2  \int
{\d^{d-1}k\over {(2\pi)^{d-1}} }
\ln\left[2\sinh\bigl({\beta\omega(k)/2}\bigr)\right]&\eqnd\TRboson}$$
with $\omega(k)= \sqrt{k^2 +m^2_T}$. The result in Eq.~\TRboson\
is given up to a mass independent infinite constant. We have used
$\omega_n=2n{\pi/\beta}$ and the identity \eqns{\eFTgenidii}. \par
Differentiating  ${\cal F}$ in Eq.~\EBfiniteT \ with respect to
$\sigma $, $\rho $ and $m_T$, one obtains the saddle point
equations. The algebraic transformations which produce the large
$N$ effective action are clearly insensitive to boundary
conditions  and, therefore, only the specific form of the saddle
point equation in Eq.~\esaddleN{c} is modified. Eqs.~\esaddleN{}
become
$$m_T^2\sigma =0\,, \quad m_T^2=2U'(\rho)$$
 and
$$\rho-\sigma ^2/N =G_2(m_T,T ) \eqnd\esaddleNTci $$
%($\sigma ^2=\phib_c^2$)
 with
$$\eqalignno{G_2(m_T, T)&  ={T\over (2\pi)^{d-1}  }\sum_{n\in{\Bbb Z}}\int^\Lambda {\d^{d-1} k\over (2\pi n T )^2+k^2+m_T^2},\cr
 &=\int^\Lambda{\d^{d-1} k\over (2\pi)^{d-1}}{1\over \omega (k)}
 \left({1\over2} +{1\over \e^{\beta  \omega (k)}-1}\right). &\eqnd\eNfivTtp
\cr}$$
The quantum ($T=0 $) and thermal ($ T>0 $ finite) fluctuations are
clearly separated when the two terms in Eq.~\eNfivTtp\ are written
as
%$$  G_2  =[G_2]_{\rm quantum}  +
% [G_2]_{\rm thermal} = \int {\d^ d k\over
%{(2\pi)^d} }\biggclb {1\over k^2 + m^2} + {2\pi \delta (k^2 + m^2)
%\over  \e^{\beta  k_0}-1} \biggcrb . \eqnd\eQuantTerm $$
$$ \eqalignno{ G_2 & =[G_2]_{\rm quantum}  +
 [G_2]_{\rm thermal} \cr
&= {1\over (2\pi)^d}  \int^\Lambda  {\d^ d k\over
  k^2 + m_T^2} +{1\over  (2\pi)^{d-1}} \int {\d^{d-1} k\over \omega (k)} {1\over \e^{\beta  \omega (k)}-1}  . &\eqnd\eQuantTerm  \cr} $$
In particular, as expected, only the zero temperature contribution is cut-off dependent. For $T\to\infty $, the second contribution dominates
and at leading order is identical to the zero-mode contribution.  \par
Note that, alternatively, we could have used the finite size formalism
of section \ssNFSS\ and the corresponding Jacobi function \eJacobi.
\smallskip
{\it Symmetry breaking.} In the broken symmetry phase $\sigma \ne 0$ and thus $m_T=0$. Then,  $G_2$ can
be calculated explicitly:
$$G_2(0,T)=\rho _c+N_{d-1}\Gamma (d-2)\zeta (d-2) T ^{ d-2}, \eqnd\eNTGiilow $$
where $N_d$ is the  loop factor \etadpolexp{b} and $\zeta (s)$ Riemann's
$\zeta$-function. The function $\zeta (s)$ has a pole for $s=1$ and therefore, as expected, a phase transition for $T>0$ is possible only for $d>3$ when the result is IR finite.
The result has a simple
interpretation: at $d=3$, IR divergences come from  the contribution of the
zero-mode  in Eq.~\eNfivTtp\ and are those of a two-dimensional theory, where no
phase transition is possible. This is a direct example of the property of {\it dimensional reduction}\/ $d \mapsto d-1$.  \par
The parameter $\rho$ is the minimum of $U$ and thus takes its $T=0$ value. Finally, the expectation value $\sigma $ is given by
$$\sigma ^2/N=\rho(T=0) -\rho _c-N_{d-1}\Gamma (d-2)\zeta (d-2) T ^{ d-2}.$$
The field expectation value $\sigma $ decreases with the temperature, which
implies that the symmetry is broken at finite temperature only if it is already broken at zero temperature.
The critical temperature $T_c$ can be expressed, for $N\to\infty $,
in terms of the zero temperature expectation value:
$$T_c\propto \left[\sigma (T=0)\right]^{2/(d-2)}. \eqnn $$
A RG analysis actually shows that, in general,  $T_c$ can be related to the crossover scale between Goldstone and critical behaviour (section \label{\ssNTfiv}).
\smallskip
{\it The symmetric phase.} One now finds
$$\rho -\rho _c=\Omega_d(m_T)-\Omega_d(0)+T^{d-2}f_d(m_T^2/T^2 ),\eqnd\eNTphigsym $$
where $\Omega _d$ is defined by Eq.~\etadepole\ and
$$\eqalignno{f_d(z)&=N_{d-1}\int_0^\infty {x^{d-2}\d x \over \sqrt{x^2+z}}{1\over
\exp[\sqrt{x^2+z}]-1} \cr
&=N_{d-1}\int_{\sqrt{z}}^\infty(y^2-z)^{(d-3)/2}{\d y\over \e^y-1}\,
.&\eqnd\eThermfz\cr} $$
In particular,
$$f_d(0)=N_{d-1}\Gamma (d-2)\zeta (d-2),\quad
f'_d(0)=-\ud(d-3)N_{d-1}\Gamma (d-4) \zeta (d-4)
.\eqnd\eTfzzero $$
The other equation is $m_T^2=U'(\rho)$.
In the special limit of a critical potential at $T=0$ (the massless theory), which satisfies $U'(\rho_c)=0$, the equation  can be expanded as
$$m_T^2\sim (\rho -\rho _c)U''(\rho _c)$$
if $U''(\rho _c)$ does not vanish. Then,
$$ m_T^2/U''(\rho_c)\sim \Omega_d(m_T)-\Omega_d(0)+T ^{ d-2}f_d(m_T^2/T^2 ).\eqnd\eNTUgenmTcrit $$
For $d>4$, this equation implies
$$m_T/T\propto (T/\Lambda )^{(d-4)/2}\ll 1 \eqnd \eNTmTfivv$$
and, thus, dimensional reduction is justified. \par
For $d=4$, the conclusion is the same because
 $$\left(m_T \over T\right)^2\sim {2 \pi^2\over3 \ln (\Lambda /T)}\ll 1\,. \eqnd\eNTmTfiviv $$
Instead, for $d<4$, the l.h.s.\ is negligible and, therefore, $m_T$ is proportional to $T $.
These results have a simple interpretation from the RG point of view in the framework of the $(\phib^2)^2$ field theory.
%%%%%%%%%%%%%%%%%%%%%%%%%%%%%%%%%%
\subsection The $(\phib^2)^2$ field theory at finite temperature

We now specialize to the $(\phib^2)^2$ field theory,
that is (section \ssfivNi) \sslbl\ssNTfiv
$$U(\rho )=\ud r \rho +{ u\over 4!}\rho^2\,.$$
We first summarize what can be learned from a simple RG analysis,
and then solve the large $N$ saddle point equations in this case. \par
We define the quantity $r_c(u)$, which has the form of a mass renormalization, as the value of $r$ at which the physical
mass $m$ of the field $\phib$
vanishes at $T=0$. At $T=0$, $r=r_c$, a  transition occurs between a symmetric phase ($r>r_c$) and
a broken symmetry phase ($r<r_c$). We recall that a QFT is meaningful only if
the physical mass $m$ is much smaller than the cut-off $\Lambda$. This implies
either (the famous {\it fine tuning}\/ problem) $|r-r_c| \ll \Lambda^2  $
or, for $N> 1$,  $r<r_c$ which corresponds to a spontaneously broken symmetry with massless Goldstone modes. The latter situation will be examined in
section \label{\ssFTnls} within the more suitable formalism of the non-linear
$\sigma $-model.  \par
It is also convenient to introduce a dimensionless coupling
$$ \lambda = u \Lambda^{d-4}/N  \,, \eqnd\eFTuLg $$
where later $N\lambda $ will be assumed to take generic (i.e.\ not very small) values. \par
%%%%%%%%%%%%%%%%%%%%%%%%%%%%%%%%%%%%%%%%%%
\medskip
{\it RG at finite temperature.}
As we have already stressed, some useful information can be obtained from a RG analysis, which also explains the nature of some of the results
obtained in the large $N$ limit. \par
Correlation functions at finite temperature satisfy the RG equations
of the zero temperature QFT or the $d $-dimensional classical
field theory in infinite volume. The dimension $d=4$ is special, because then
the $\phi^4_4$ theory is just renormalizable.  One important quantity is the ratio $ m_T/T$,
where  $m_T$ governs the decay of correlations in space directions and is also, after dimensional reduction, the mass of the zero-mode in the effective theory.
\smallskip
{\it Higher dimensions}. For $d>4$, the theory is non-renormalizable, which
means that the gaussian fixed point $u=0$ is stable. The coupling constant
$u=N\lambda \Lambda^{4-d}$ is small in the physical domain, and perturbation theory is applicable. At zero temperature, the physical mass in the symmetric phase has the
scaling behaviour of a free or gaussian theory, $m\propto (r-r_c)^{1/2}$.
The leading corrections to the two-point function due to finite temperature
effects, are of order $u$. Therefore, in the symmetric phase, for dimensional
reasons,
$$m_T \propto ( r-r_c+{\rm const.}\ \lambda\Lambda^{4-d} T^{d-2})^{1/2}.$$
If at zero temperature the symmetry is broken ($r<r_c$),   a phase transition thus occurs
at a temperature $T_c$, which scales like
 $$T_c \propto \Lambda \left[(r_c-r)/\Lambda^2 \right]^{1/(d-2)} \gg (r_c-r)^{1/2} \,.$$
The critical temperature is large compared with the $T=0$ crossover mass scale \eNmcrossc.
%%$$T_c\propto m^{2/(d-2)}\Lambda^{(d-4)/(d-2)} \gg m\,.$$
 \par
At high temperature or in the massless theory ($r=r_c$), the thermal
mass behaves like
$$m_T  /T\propto ( T/\Lambda )^{(d-4)/2}\ll 1\,,$$
a behaviour consistent with the form \eNTmTfivv.
The property  $  m_T/T \ll 1 $ implies the validity of dimensional reduction.
\smallskip
{\it Dimension $d=4$.} The $(\phib^2)^2$ theory is just renormalizable and RG equations take the form
$$ \left[ \Lambda{  \partial \over  \partial \Lambda}
+\beta(\lambda){ \partial \over
 \partial \lambda}-{n \over 2}\eta
(\lambda)-\eta_{2}(\lambda)(r-r_c){ \partial \over  \partial r}
\right] \Gamma^{(n)}  (p_{i};r,\lambda,T,\Lambda
 )=0\,. \eqnd\eganTRG $$
The ratio $m_T/T=F( \Lambda /T, \lambda,r/T^2 )$  is dimensionless and
RG invariant, and thus  satisfies
$$ \left[ \Lambda{  \partial \over  \partial \Lambda}
+\beta(\lambda){ \partial \over
 \partial \lambda}-\eta_{2}(\lambda)(r-r_c){ \partial \over  \partial r}
\right] F=0\,.$$
The solution can be written as
$$  m_T/T=F(\Lambda /T, \lambda,  (r-r_c)/T^2)=F\bigl(\ell
\Lambda/T,\lambda(\ell ), ( r(\ell)-r_c)/T^2\bigr),\eqnd\eFTRGmass $$
where $\ell $ is a scale parameter, and $\lambda(\ell ), r(\ell )$ the
corresponding running parameters (or effective parameters at scale $\ell $):
$$\ell {\d \lambda(\ell )\over \d\ell }=\beta\bigl(\lambda(\ell )\bigr), \quad
\ell {\d r(\ell )\over \d\ell }=-\bigl(r(\ell )-r_c\bigr)\eta_2\bigl(\lambda(\ell
)\bigr).$$
The form of the RG $\beta$-function,
$$\beta(\lambda)={(N+8) \over 48\pi^2}\lambda^2+O\left(\lambda^{3}
\right), \eqnn $$
implies that the theory is IR free,  that is $\lambda(\ell )\to 0$ for $\ell
\to 0$. The effective coupling constant at the physical scale is
logarithmically small, implying logarithmic
deviations from naive scaling. To describe physics at the  scale
$T $, one has to
choose $\ell =T/\Lambda  \ll1$  and, thus,
$$\lambda(T/\Lambda  )\sim {48 \pi^2 \over (N+8)\ln(\Lambda
/T)}. \eqnd\eFTgrun $$
Therefore, RG improved perturbation theory can be used to derive the effective action of the reduced theory. In particular, for $r=r_c$ one recovers the behaviour \eNTmTfiviv:
$$\left(m_T\over T\right)^2\sim{N+2\over72} \lambda(T/\Lambda  )\sim {2\pi^2(N+2)\over 3(N+8)}{1\over\ln(\Lambda
/T)}.$$
\smallskip
{\it Dimension  $d=3$.}
The three-dimensional classical theory has an IR fixed point
$\lambda^*$. Then finite size scaling (Eq.~\eFTRGmass) predicts,
in the symmetric phase,
$$ m_T/T= f\bigl((r-r_c)/T^{1/\nu}\bigr),$$
where $\nu$ is the correlation exponent of the three-dimensional
system. Therefore, $m_T$ in general  remains of order $T$ at high
temperature.  \par The zero-mode plays a special role only if
there exist values such that the function $f$ is small (relative
to 1). This would happen near a phase transition, but in an
effective two-dimensional theory a phase transition is impossible
for $N>2$.
%The situation in which for $N\ne 1$ the symmetry is broken at
%zero temperature will be examined in the framework of the non-linear $\sigma $-model starting with %Section \ssFTnls.
%
%%%%%%%%%%%%%%%%%%%%%%%%%%%%%%%%%%%%%%%%%
\medskip
{\it The large $N$ limit.} We now specialize the results of section
\ssNTphig, and verify consistency with the general RG analysis.
\smallskip
{\it Broken symmetry phase.} In the broken symmetry phase $m_T=0$,   $G_2$ is given by Eq.~\eNTGiilow:
$$G_2(0,T)=\rho _c+N_{d-1}\Gamma (d-2)\zeta (d-2) T ^{ d-2}. $$
We have already noted that a phase transition for $T>0$ is possible only for $d>3$ when the result is IR finite. \par
Then, using the last equation $U'(\rho )=0$ in the example of the
$(\phib^2)^2$  theory, one obtains
$$\sigma ^2/N= {6\over Nu}(r_c-r)- N_{d-1}\Gamma (d-2)\zeta (d-2)
T^{d-2} $$
and finds a critical temperature
$$T _c\propto \Lambda \bigl((r_c-r)/\Lambda ^2)^{1/(d-2)}.$$
The value of $T_c$ can be compared with $m_{\rm cr}$, the mass
scale at which a crossover between critical and Goldstone behaviours occurs. \par
For $d>4$,
$$T_c\propto \Lambda (m_{\rm cr}/\Lambda )^{2/(d-2)}\gg m_{\rm cr} $$
 (from Eq.~\eNmcrossc) and, thus, the temperature $T_c$ is large
 compared to the relevant mass scale.
\par For $d=4$, the explicit value is
$$ T _c=(72/Nu)^{1/2}(r_c-r)^{1/2} \propto m_{\rm cr}\sqrt{\ln(\Lambda /m_{\rm cr})},$$
where Eq.~\eNmcrossb\ has been used. In all cases the critical
temperature is large compared to the crossover mass.
\smallskip
{\it The symmetric phase.} The saddle point equations become
$$\eqalign{\rho -\rho _c&=\Omega_d(m_T)-\Omega_d(0)+T^{d-2}f_d(m_T^2/T^2 ), \cr
m_T^2&=(Nu/6)(\rho-\rho_c)+r-r_c\,. \cr}
$$
In the special limit $r=r_c$, which corresponds to the critical point
at zero temperature, Eq.~\eNTUgenmTcrit\ now reads
$$ (6/Nu)m_T^2=\Omega_d(m_T)-\Omega_d(0)+T ^{ d-2}f_d(m_T^2/T^2 ).$$
The behaviour of $m_T/T$ has already been discussed in section \ssNTphig.
The special dimension $d=4$ will be examined in more detail in  section \ssNvarfivT\ in the context of variational calculations, and the general large $N$ limit again in section \label{\ssFTnls}  from the point of view of the non-linear $\sigma $ model.
%In four dimensions instead the critical temperature has the form

\subsection{Variational calculations in the $(\phib^2)^2 $ theory
}

To further compare variational calculations with large $N$ results, we
generalize the zero-temperature calculations of section \ssNVarfiv\ to a finite temperature system. The variational
principle is still based on the inequality \evarineq, but is
applied to a finite temperature system $T=1/\beta  $. For a system
of $N$ non-interacting massive bosons at temperature $T $ in $d-1$
space dimensions, the Helmholtz free energy density
 is given by \sslbl\ssNvarfivT
$$\eqalignno{ {\cal F}_0 =- {1\over V\beta  } \ln  {\cal Z}_0 &
=- {1\over V\beta } \ln \tr \e^{-\beta  H_0} \cr
& =N
  {1\over {(2\pi)^{d-1}} }\int \d^ {d-1}k \left( \half\omega(k) +
T\ln \left(1-\e^{-\beta \omega(k)}\right) \right).&
\eqnd\freeEzero \cr}$$
The inequality \evarTzero\ is replaced by
$$ {\cal F}\le  {\cal F}_{\rm var.}=  {\cal F}_0
+N\left<U\bigl(\phib^2(x)/N\bigr) -\ud m_T^2\bigl(
\phib(x)-\phib_0\bigr)^2/N\right>_0\,,\eqnn $$ where
$\left<\bullet\right>_0$ means average with respect to $\e^{-{\cal
S}_0}$ in a finite temperature, infinite space volume system, and
${\cal S}_0$ again is the trial free action:
$${\cal S}_0=\int \d^ d x \,
 \left({1\over2}{( \partial_{\mu} \phib)}^2 +
 {m_T^2 \over2} \bigl(\phib(x)-\phib_0\bigr)^2 ~\right)  , $$
$m_T^2$, $\phib_0^2 =\sigma ^2$ (in the notation of section \ssNTphig) being the two variational parameters. We now
evaluate the gaussian average in the large $N$ limit:
$$\left<U\bigl(\phib^2(x)/N\bigr)\right>_0\sim U\bigl(\left<\phib^2(x)/N\right>_0\bigr)
,$$
and introduce the parameter $\rho$, which is given by
$$\eqalignno{\rho=\left<\phib^2(x)/N\right>_0&=\phib_0^2/N+{T \over (2\pi)^{d-1}}
\sum_{n\in{\Bbb Z}}\int{\d^{d-1} k\over (2\pi n T )^2+k^2+m_T^2},\cr
 &= \sigma ^2/N+ G_2(m_T,T).&
\eqnd\esaddleNTc \cr}$$
Then, the variational free energy density becomes
$${\cal F}_{\rm var}= {\cal F}_0+ NU(\rho)-\ud m_T^2( N\rho- \sigma ^2) .\eqnd\FvarT $$
The relation
$${ \partial {\cal F}_0\over  \partial m_T^2}=\ud( N \rho-\sigma^2),$$
still holds, in such a way that
$$ { \partial {\cal F}_{\rm var.}\over  \partial m_T^2} =N{ \partial \rho \over
 \partial m_T^2}\left[U'(\rho)-\ud m_T^2\right].  \eqnd\eFvarTfiv $$
The condition of stationarity of ${\cal F}_{\rm var.}$ with respect
to variations of $m_T^2$, together with the equation defining $\rho$, are again
equivalent to the large $N$ saddle point equations.
 \medskip
{\it The $(\phib^2)^2$ field theory in dimension $d=4$.}
We now discuss the example of the $(\phib^2)^2$ field theory in four dimensions:
$$U(\rho)=\ud r \rho+{u\over 4!}\rho^2.$$ From Eqs.~\eqns{\eNTphigsym,\etadepoliv},
one finds
$$\eqalignno{ \rho(m_T^2)  &= \rho_c+ \sigma ^2/N-  {m_T^2 \over 8 \pi^2}
\ln(\Lambda /m_T) + { T^2\over 2 }f_4({m_T^2/T^2}),
 &\eqnd\phifourdimension \cr }$$
where $f_d(z)$ is given in Eq.~\eThermfz.
%%% F changed in f/12, thus a factor 12 in what follows
%$$f(z)={6\over \pi^2}\int_0^\infty \d x
%{x^2 \over \sqrt{x^2+z}}
%\left( {1\over \exp\sqrt{x^2+z}-1 }\right)\eqnd\Fofz$$
The variational free energy of the system
can be obtained from Eq.~\FvarT\ or by integrating Eq.~\eFvarTfiv:
$$\eqalignno{{1\over N}  {\cal F}_{\rm var}( \sigma ,m_T^2)  &= U(\rho)-\ud
 \int^{m_T^2}_0 \d s \ s{ \partial\rho(s)\over
 \partial s}\cr &= {1\over 32\pi^2} m_T^4 \ln( \Lambda/ m_T ) +
{1\over 4}T^4\int^\infty_{m_T^2/ T^2} \d z\, z  f_4'(z) \cr &+{u\over
24}\left\{  \sigma ^2/N +6 {r-r_c \over u} - {1\over 8\pi^2} m_T^2
\ln(\Lambda/ m_T) + {1\over 2}T^2 f_4({m_T^2/T^2}) \right\}^2 \cr
&-{3\over 2} {r^2 \over u}\,.&\eqnd\WfourDT  }$$
The renormalized theory with finite non-zero renormalized coupling $u_\r$
can be reached only with a negative $u$ in a metastable
state mentioned in section \ssNVarfiv. In the limit $\Lambda \to
\infty$,  $u \to 0^-$ and the renormalized free energy can be
calculated in terms of the renormalized
parameters $u_\r,m_T$ and using the gap equation. One finds
$$ \eqalignno{{1\over N}
{\cal F}_{\rm var}( \sigma ,m_T^2) + {3\over 2} {r^2 \over u} =
{m_T^2\over 2}(\sigma^2/N \pm 6{M^2/u_\r})
%\cr &
&- {r^2\over 4}\left\{ {1\over u_\r} + {1\over
16\pi^2}\ln\left({\e^ {3\over 2}M^2\over m_T^2 }\right)\right\}
\cr &-{1\over 4}T^4\int^\infty_{m_T^2/ T^2} \d z\, f_4(z).
&\eqnd\WfourDRe }
$$  %%%%%% TO BE CHECKED
Here  $M$ is a normalization scale and the choice $\pm$ sign
depends on the sign of $M^2/ u_\r$. The basic instability that
originates from the negatively coupled theory is maintained also
in the renormalized expression in Eq.~\WfourDRe.
%\refs{\Bm}\ \refs{\Tar}\ \refs{\Linde}\
One finds that as the
temperature increases, the $O(N)$ symmetric metastable state that
 exists at low temperature becomes even less stable, and above a certain
critical temperature, disappears altogether. The theory  now has only the $O(N)$ broken symmetry phase that
we have found; it is unstable, as mentioned at the bottom of section
\ssNVarfiv, and thus  renders the theory inconsistent.

\subsection The non-linear $\sigma$ model  at finite temperature

We now discuss another, related, example: the non-linear $\sigma$
model because the presence of Goldstone modes introduces some new
features in the analysis. In the perturbative framework, due the non-linear character of the
group representation, one is confronted with difficulties which also
appear in non-abelian gauge theories. Moreover, the non-linear $\sigma $ model
and non-abelian theories share another property: both are UV asymptotically free
in the dimensions in which they are renormalizable.  \sslbl\ssFTnls \par
Finally, we recall that it has been proven in
section \label{\ssLTsN}, within the
framework of the $1/N$ expansion, that the non-linear $\sigma $ model is
equivalent to the $((\phib^2))^2$ field theory (at least for generic
$\phi^4$ coupling), both quantum field theories corresponding to
two different perturbative expansions of the same physical model.
%
\medskip
{\it The non-linear $\sigma $ model.}  The non-linear $\sigma $ model has been studied
at zero temperature in section \ssLTsN. It is an $O(N)$ symmetric
QFT, with an $N$-component scalar field
${\bf S}(t,x)$ which belongs to a sphere,  that is that satisfies the constraint ${\bf
S}^2(t,x)=1$. \par
The partition function of the non-linear $\sigma$ model can be written as
$${\cal Z}= \int \left[\d{\bf S}(t,x)\d\lambda(t,x)\right]
\exp\left[-{\cal S}({\bf S},\lambda)\right] \eqnn  $$
with
$$\eqalignno{{\cal S}({\bf S},\lambda) &= {1 \over 2g}\int_0^{1/T} \d t\,\d^{d-1}x \left[
\bigl( \partial_t {\bf S}(t,x) \bigr)^2\right.\cr&\quad\left.
+\bigl(\nabla{\bf S}(t,x) \bigr)^2  + \lambda(t,x) \bigl({\bf
S}^2(t,x) -1 \bigr) \right],& \eqnd\eactsigla \cr}$$ where the
$\lambda$ integration runs along the imaginary axis and enforces
the constraint ${\bf S}^2(x)=1$. The parameter $g$ is the coupling
constant of the quantum model as well as the temperature of the
corresponding $d $-dimensional classical theory. \par As we have
already explained,  a finite temperature $T$ leads, in the
corresponding  classical theory, to one finite size $ \beta =1/T$
with periodic boundary conditions.
 \smallskip
{\it Finite temperature saddle point equations.}
The non-linear $\sigma $ model has been discussed in the
large $N$ limit at zero temperature in section \ssLTsN\ with a slightly different notation ($T\to g$).
At finite temperature, the saddle point equation \emgNsig{a} remains
unchanged.  The saddle point equation \emgNsig{b} is modified because the frequencies in the time direction are quantized.  It can be expressed in terms of the function
$$ G_2(m_T, T)  ={T\over (2\pi)^{d-1}  }\sum_{n\in{\Bbb Z}}\int^\Lambda {\d^{d-1} k\over (2\pi n T )^2+k^2+m_T^2}  $$
defined earlier (Eq.~\eNfivTtp). In the symmetric phase $ \left<{\bf S}(T)\right>=0$, one finds
$$1= (N-1)\, g \, G_2(m_T,T)
. \eqnd\esaddNTf $$
Here, $m_T$ is the thermal mass and $\xi_T=m_T^{-1}$ has the meaning of a correlation length in  space
directions.  \par
In terms of the functions  \eqns{\etadepole, \eThermfz},
 Eq.~\esaddNTf\ (for $N$ large)  can be rewritten as
$${1 / Ng}=  \Omega_d(m_T ) +T^{d-2} f_d (m_T^2/T^2 ).  \eqnd\eNTnlssad $$
One recovers that a phase transition is possible only if $f_d(0)$ is finite,
which from Eq.~\eTfzzero\ implies $d>3$, a result that can be seen as a consequence of   dimensional reduction.
\medskip
{\it Dimension $d=2$}. We first examine the case $d=2$.
This corresponds to a situation where even at zero temperature  the
$O(N)$ symmetry remains always unbroken. In
the zero temperature QFT, or the infinite volume classical statistical
system, the continuum limit corresponds to $g\ll 1$ and the physical
mass $m$ then is given by  Eq.~\eNsigmii:
$$ 1/ Ng = \Omega _2(m) \ \Rightarrow \
m =\xi_0^{-1}\propto \Lambda \e^{-2\pi/N g}.$$
By subtracting this equation from Eq.~\eNTnlssad\ (the  finite
temperature gap equation), one finds
$$\ln(m_T/m )=\ln(\xi_0/\xi_T)=2\pi f_2 (m_T^2/T^2 ). \eqnd\eNTnlsgapii $$
High temperature corresponds to $ T/m   \gg 1 $ and thus one also
expects $ m_T \gg m  $. The integral \eThermfz\ then is dominated by
the contribution of the zero-mode and, therefore,
$${T\over m_T}  ={1\over \pi} \ln (m_T/m  )\sim  {1\over
\pi} \ln(T/m  ).\eqnd\eFTNnlsi $$
The logarithmic decrease of the ratio $m_T/T$ at high temperature  corresponds to the UV asymptotic freedom of the classical non-linear $\sigma
$ model in two dimensions.
\medskip
{\it Dimensions $d>2$.} In higher dimensions the system has a phase
transition for $T=0$ at a value $g_c$ of the coupling constant. The gap equation can then be rewritten as
 $${1\over Ng}-{1\over Ng_c}=\Omega_d(m_T )-\Omega_d(0)
+T^{d-2}f_d (m_T^2/ T^2 ).\eqnd\eFTnlsgap  $$
For $g>g_c$, the equation can also be expressed in terms of  the physical mass $m$ (Eq.~\emasNsig) as
$$\left[\Omega_d(m )-\Omega_d(m_T )\right]/T^{d-2}=f_d (m_T^2/
T^2 ).\eqnn $$
The behaviour of the system then depends
on the ratio $T/m $.
To obtain more explicit results, one has to distinguish between various possible dimensions.
\medskip
{\it Dimension $d=3$} \refs{\rGNDiiirev,\rCuSaYe}. The absence of phase transition in two dimensions prevents a phase transition at finite temperature. The
gap equation has a scaling form, as predicted by finite size RG arguments. A short calculation yields
$$f_3(s)=-{1\over
2\pi}\ln\left(1-\e^{-\sqrt{s}}\right) $$
and
$$\Omega_3(m_T)-\Omega_3(0)=-{m_T\over 4\pi}\,.$$
For $g>g_c$, after some simple algebra, the gap equation can be written as
$$2\sinh(m_T/2T)=\e^{m/2T} .$$
One verifies that for $m/T$ large (low temperature) $m_T\to m$, and at high
temperature $T\gg m$, $m_T$ becomes proportional to $T$:
$$m_T /T\sim 2\ln\bigl((1+\sqrt{5})/2\bigr).$$
For $g<g_c$,  that is when the symmetry is broken at zero temperature, one has to return to the general form
$$2\sinh(m_T/2T)=\exp\left[-{2\pi  \over
NT}\left({1\over g}-{1\over g_c}\right) \right]. \eqnn $$
One can also introduce the mass scale % and \eqns{\enlsmdef})
$$m_{\rm cr}(g)=  {1\over g}-{1\over g_c}  $$
(see Eq.~\eqns{\eNmcrossa}), which at zero temperature characterizes the crossover between
critical and Goldstone mode behaviours. Then,
 $$2\sinh(m_T/2T)=\e^{-2 \pi m_{\rm cr}/NT} .$$
For $g<g_c$, the zero-mode dominates if  the ratio  $m_T/T $  is small and thus if $ m_{\rm cr}/ T $ is large. This condition is  realized  for all temperatures  if $|g-g_c|$ is not small because then $m_{\rm cr}=O(\Lambda )\gg T$:
this is the situation of chiral perturbation theory, and corresponds to the deep  IR (perturbative) region where only Goldstone particles propagate.
It is also realized  in the critical domain $|g-g_c|$
small, if $T\ll m_{\rm cr}$, that is at low (but non-zero) temperature. Then,
$$ m_T \sim T \e^{-2 \pi m_{\rm cr}/NT} =T  \exp\left[-{2\pi  \over
NT}\left({1\over g}-{1\over g_c}\right) \right]. \eqnd\eFTnlsmLii$$
Note that, when the coupling constant $g$  or
the temperature $T$ go to zero, the mass $m_T$ has the exponential behaviour characteristic of the dimension 2.
This property  of dimensional reduction at low temperature
is somewhat surprising. It is in fact a precursor of the symmetry breaking at zero temperature.
\medskip
{\it Higher dimensions: the critical temperature.} For $d>3$, the quantity $f_d(0)$ is finite
and, therefore, a phase transition at finite temperature is possible, in agreement
with dimensional reduction and the property that a phase transition
is possible in dimensions larger than two (in the case of continuous symmetries). From Eq.~\eFTnlsgap\ one infers
$$T_c^{d-2}={1\over Nf_d(0)}  \left({1\over g}-{1\over g_c}\right).\eqnn $$
Since $f_d(0)$ is positive, this result confirms that a transition is possible only for $g<g_c$,  that is if at zero temperature the symmetry is broken. \par
However, this result is meaningful only if $T\ll \Lambda $ and thus only  for $|g-g_c|$ small.  Then, $T_c$ is proportional to the crossover mass scale $m_{\rm cr}(g)$ (Eq.~\eqns{\eNmcrossa}) between critical and Goldstone behaviours.
\smallskip
{\it Dimension $d=4$.}   Since $f_4(0)=1/12$  one finds
$$T_c^2={12\over N}\left({1\over g}-{1\over g_c}\right)={12 \over N} m_{\rm cr}^2.\eqnn $$

  \par
Another limit of interest is the high temperature or massless limit. For $m_T\ne 0$, one finds an additional cut-off dependence:
$$\Omega_4(m_T)-\Omega_4(0)\sim-{m_T^2\over 8\pi^2}\ln(\Lambda/m_T).$$
At $g=g_c$, one finds that $m_T/T $ decreases logarithmically with the cut-off. At leading order, using $f_4(0)=1/12$,  one obtains
$$(m_T/T )^2={2\pi^2\over 3\ln(\Lambda /T )}, $$
in agreement with Eq.~\eNTmTfiviv.
 \smallskip
{\it Dimension $d=5$.} From $f_5 (0)=\zeta(3)/4\pi^2$, one infers the critical temperature $T_c$:
$$T_c^3 \sim{4\pi^2\over N\zeta(3)}\left({1\over g}-{1\over g_c}\right) ={4\pi^2\over N\zeta(3)}m_{\rm cr}^3\,.$$
 \par
In the massless limit  $g=g_c$,
$$  (m_T/T) ^2\sim {\zeta(3)\over 4\pi^2}{T\over D_5(0)} \eqnd\eFTnlsmiv$$
with $D_5(0)\propto \Lambda $, a result consistent with the behaviour \eNTmTfivv.
%%%%%%%%%%%%%%%%%%%%%%%%%%%%%%
\subsection The Gross--Neveu model at finite temperature

To gain some intuition about the role of fermions at finite temperature, we now
examine a simple model of self-interacting fermions, the Gross--Neveu (GN \refs{\rDaMaRa, \rBCMPG}.  The GN model is described in terms of a $U(\tilde N)$ symmetric action for a
set of $\tilde N$ massless Dirac fermions $\{\psi^i, \bar\psi^i \}$ (for details see section \ssNGNmod): \sslbl\ssGNNFT
$${\cal S} (\bar \psib, \psib )= -\int \d t\,\d^{d-1} x \left[
\bar\psib(t,x)\cdot \sla{ \partial} \psib(t,x) +\frac{1}{2N} G\left(\bar\psib(t,x)\cdot \psib(t,x) \right)^2
\right],\eqnd\eGNact $$
where $N=\tilde N\tr{\bf 1}$   is the total number of fermion components. \par
The GN model has in all dimensions a discrete symmetry  that prevents the addition of a mass term.
In even dimensions it corresponds to a discrete chiral symmetry, and in odd
dimensions to space reflection. Below, to simplify, we
will speak about chiral symmetry, irrespective of dimensions.  \par
The GN model is renormalizable in $d=2$ dimensions, where it is asymptotically
free and the chiral symmetry is always broken at zero temperature.  \par
It has been proven in section \sssGNYN\ that within the
$1/N$ expansion the GN model is equivalent to the GNY (Y for Yukawa)
model, at least for generic couplings: the GNY model has the same symmetry, but contains an elementary
scalar particle  coupled to fermions through a
Yukawa-like interaction, and is renormalizable in four dimensions. This
equivalence provides a simple interpretation to some of the results that
follow. \par
At finite temperature, due to the anti-periodic boundary conditions, fermions
have no zero modes. Therefore, limited insight about
the physics of the model at high  temperature can be gained from
perturbation theory; all fermions
are simply integrated out. Instead, we study here the GN model within the
framework of the $1/N$ expansion.   \par
Effects due to the addition of a chemical potential will
not be considered here. \par  %%% read the corresponding part in largen6
The integration over fermions,
leads to an effective non-local action for a periodic scalar field $\sigma $,
which has already been discussed in the zero temperature limit  in section
\sssGNYN:
$$ {\cal S}_N (\sigma )={N\over2 G}\int_0^{1/T} \d t\, \int
\d^{d-1} x \,
\sigma^2(t,x) -\tilde N\tr \ln\left(\sla{ \partial}+\sigma(\cdot) \right), \eqnd\eGNNactb $$
where $T  $ is the  temperature, and the
$\sigma $ field satisfies periodic boundary conditions in the euclidean time variable.
 \par
In the situations in which the thermal mass of the
$\sigma$ field is small compared with the temperature, one can perform a mode
expansion of the $\sigma $-field, integrate over the non-zero modes and obtain
an effective local $(d-1)$-dimensional action for the zero-mode.  It is important
to realize
that for $T>0$, since the reduced action is local and symmetric in $\sigma \mapsto
-\sigma $, it describes the physics of the Ising transition  with  {\it short range
interactions}, unlike what happens at zero temperature where the fermions are massless.
The question that then arises is the possibility of a breaking of  this
remaining reflexion symmetry.  \par
As we have seen,  a non-trivial perturbation theory is obtained by expanding
for large $N$. If a continuous order phase transition occurs at finite temperature,
IR divergences
generated by the $\sigma $ zero-mode will appear in the $1/N$
perturbation theory at the transition temperature $T_c$ for $d-1<4$.
Below $T_c$, in the same way as at zero temperature, the decay of $\sigma
$ correlation functions in  space directions is characterized by the
saddle point value $M_T$ of the field $\sigma (x)$. Above $T_c$, the correlation
length is also finite in contrast with the zero temperature situation. \par
One then finds two regimes, which have to be dealt with differently.
Near the transition temperature, $1/N$ perturbation theory
is not useful for $d<5$. Instead, one has to perform a mode expansion
of the $\sigma $-field and a local expansion of the dimensionally
reduced action for the $\sigma $ zero-mode. The effective field theory
relevant for long distance properties is of $\sigma ^4$ type (as in
the case of the Ising model) with coefficients depending on coupling
constant and temperature, which has to be studied by the usual RG methods.
Note that this implies the absence of phase transition for $d=2$ at finite temperature.
In the other regime where  the $\sigma $ correlation length is of
order $1/T$, all modes are similar and $1/N$ perturbation theory is directly applicable.
%

\subsection The gap equation

Calling $M_T$ the saddle point value of the field $\sigma (x)$, we
obtain the action density at finite temperature and large
$N$:
$${1\over N}{\cal F}(\rho,M_T) =  {M_T^2 \over 2G} - { T\over V_{d-1}\tr{\bf 1}}\tr \ln\bigl
(\sla{ \partial}+M_T \bigr),\eqnd\eTRfermionM $$
where $V_{d-1}$ is the $d-1$ dimension
volume and
$$ {1\over V_{d-1}}  \tr\ln(\sla{ \partial} + M_T)  =
  {1\over2} \tr{\bf 1}  \int {\d^{d-1}k\over {(2\pi)^{d-1}} }\sum_
{n\in {\Bbb Z}}
 \ln(\omega_n^2+ k^2+M_T^2)   $$
with $\omega_n=(2n+1){\pi T}$. The sum over frequencies  follows from the identity \eqns{\eFTgenidii}, and one obtains
 $$ { 1\over V_{d-1}\tr{\bf 1}} \tr\ln(\sla{ \partial} + M_T) =  \int {\d^{d-1}k\over {(2\pi)^{d-1}} }
\ln\left[2\cosh\bigl( \omega (k)/  2T\bigr)\right]
 \eqnd\eTRfermion $$
with $\omega(k)=\sqrt{k^2+M^2_T}$.
Alternatively, one could use
Schwinger's  representation  of the propagator and another
function of elliptic type
$$ \vartheta_1(s) =\sum_n\e^{-(n+1/2)^2\pi s}.\eqnd\eAellfer $$
The gap equation at finite temperature, obtained by differentiating
${\cal F}$ with respect to $M_T$, again splits
into two equations $M_T =0$ and
$$ 1 /G    = {\cal G}_2(M_T ,T)   \eqnd\eFTGNNsad $$
with
$${\cal G}_2(M_T ,T) =  \int^\Lambda{\d^{d-1} k\over (2\pi)^{d-1}}{1\over \omega(k)
}\left({1\over2} -{1\over \e^{ \omega(k)/T}+1}\right)  .
\eqnd  \eNGNTtp  $$
This is the fermion analogue
of Eqs.~\eqns{\esaddNTf,\eNfivTtp}  in which one recognizes again the sum of quantum and thermal contributions.
Note, however, that the function ${\cal G}_2(M_T ,T)$ has a regular expansion in
$M_T^2$ at $M_T=0$. \par
%$${\beta \over G}={1\over (4\pi)^{(d-1)/2}}\int{\d t \over t^{(d-1)/2}}\e^{-t
%\sigma^2}B(4\pi t/\beta ^2). $$
In terms of the function
$$\eqalignno{g_d(s)&= N_{d-1}\int_0^\infty {x^{d-2}\d x \over
\sqrt{x^2+s}}{1\over \exp[\sqrt{x^2+s}]+1}\cr
&=N_{d-1}\int_{\sqrt{s}}^\infty (y^2-s)^{(d-3)/2}{\d y\over \e^y +1}\,,
 &\eqnd\eThermfiis\cr}
 $$ where $N_d$ is  given in \etadpolexp{b},  the
gap equation can be rewritten as
$$  1 /  G  = \Omega_d(M_T)-T ^{d-2} g_d(  M_T ^2/T^2).  \eqnn  $$
If $d>2$, one can introduce the critical value $G_c$ where $M\equiv M_{T=0}$
vanishes at  zero temperature:
$$  {1 \over G}-{1\over G_c} =\Omega_d(M_T
)-\Omega_d(0)-T ^{d-2} g_d(  M_T ^2/T^2). \eqnd\eTGNNgapii $$
Finally, for $G>G_c$, the equation can be expressed in terms of the fermion physical mass $M\equiv m_\psi $ solution of Eq.~\eNGNmassgap,
$$ 1/G =\Omega_d(M) ,\eqnd\eGNNfms $$
and then reads
$$\Omega_d(M_T)- \Omega_d(M) =T ^{d-2} g_d(  M_T ^2/T^2). \eqnd\eTGNNgapiii $$
%%%%%%%%%%%%%%%%%%%%%%
 \smallskip
{\it The $\sigma $ two-point function.} Since the correlation length $\xi_\sigma $
of the $\sigma $ zero-mode plays a crucial role, we also calculate the
$\sigma $ two-point function  $\Delta_\sigma (p)\equiv \Delta_\sigma
(p_0 =0,p)$ (see Eq.~\eGNsigprop). Note that in what follows we  use the notation $m_\sigma $ for the corresponding temperature-dependent
thermal mass:  $ m_\sigma(T)=1/\xi_\sigma (T)$, which is also the mass of the $\sigma $ field in the dimensionally reduced theory. \par
 For $M_T=0$, one finds
$${1\over N  \Delta_\sigma(p) } ={1 \over  G} -   {\cal G}_2(0,T)   + {T
\over 2 (2\pi)^{d-1} }  p^2 \sum_n \int^\Lambda{\d^{d-1} k \over \left(
k^2+\omega _n^2 \right) \left[(p+k)^2 +\omega _n^2  \right]} . \eqnn $$
For $d> 2$, the propagator can be expressed in terms of the constant $g_d(0)$:
$$ \eqalignno{{1\over N \Delta_\sigma(p) }&={1 \over  G} -{1\over G_c}+
T^{d-2}g_d(0) \cr& \quad   + {T \over 2 (2\pi)^{d-1} }
 p^2 \sum_n \int^\Lambda{\d^{d-1} k \over \left(
k^2+\omega _n^2 \right) \left[(p+k)^2 +\omega _n^2  \right]} .&\eqnd\eNTDsigsym\cr} $$
For $M_T\ne 0$, using the gap equation, one can write the propagator as
$${1\over N   \Delta _\sigma(p)}= { T\left(p^2+4M_T^2 \right)\over 2 (2\pi)^{d-1} }
\sum_n\int^\Lambda{\d^{d-1} k \over \left(
k^2+\omega _n^2+M_T^2 \right) \left[(p+k)^2+\omega _n^2 +M_T^2 \right]}.  \eqnd\eNTfergsig  $$
Therefore, when the symmetry is broken the mass $m_\sigma(T) =2 M_T$, generalizing the zero temperature result. \par
More detailed properties require distinguishing between dimensions.
%%%%%%%%%%%%%%%%%%%%%%%%%%%%%%%%%%%%%%
\medskip %%% Improve
{\it Local expansion.} When the $\sigma $ thermal mass or expectation value is small
compared with $T$, one can perform a mode expansion of the field $\sigma $, retaining only the
zero mode and then a local expansion of the action \eGNNactb,
and study it to all orders in the $1/N$ expansion.  In the reduced theory, $T$ plays the role of a large momentum cut-off.  \par
The first terms of the effective $(d-1)$-dimensional action are
$${\cal S}_{d-1} (\sigma )=N\int\d^{d-1} x \left[\ud Z_\sigma
( \partial_\mu \sigma
)^2+\ud r \sigma ^2 +{1\over 4!}u \sigma^4\right], \eqnd\eFTGNloc $$
where terms of order $\sigma^6$ and $ \partial ^2 \sigma ^4 $ and higher have
been neglected. The three parameters are given by
$$Z_\sigma =\ud   {\cal G}_4(0,T)/T\,, \quad r=\left[1/G-   {\cal G}_2(0,T)\right]/T \,,
\quad u=6   {\cal G}_4(0,T)/T\,, $$
where  ${\cal G}_4(m_T,T)$ can be calculated from
$${\cal G}_4(m_T,T)=-{ \partial \over  \partial m_T^2}{\cal G}_2(m_T,T).$$
For $d<4$, ${\cal G}_4(0,T)$ is finite and thus proportional to $T^{d-4} $.
For $d>2$, after the shift of the
coupling constant, one finds
$$rT  ={1\over G }-{1\over G_c  }-T^{ d-2}g_d(0)= g_d(0)\left(T_c^{d-2}-T^{d-2} \right) . \eqnn $$
 \par
As already explained, the properties of this model are those of the
critical $\phi^4$ theory and, for $d<5$, have to be studied by the usual non-perturbative techniques.
 % %%%%%%%%%%%%%%%%%%%%%%%%%%%%%%%%%%%%%%%
\subsection Phase structure for $N$ large

{\it Phase transition and critical temperature.}
One notes that $\Omega_d(M)-\Omega_d(M_T )$ is always
negative and, thus,  since $\Omega _d$ is a decreasing function, $|M|< |M_T|$. Therefore,  when the chiral symmetry is unbroken at zero temperature,
Eq.~\eTGNNgapiii\ has no solution, $M_T=0$ is the minimum  and the $\sigma\mapsto-\sigma$ symmetry is not
broken at $N=\infty$.   \par
We now assume a situation of symmetry breaking at zero temperature,
which implies $G>G_c$. For $d>2$,
$$g_d(0)=N_{d-1}\left(1-2^{3-d}\right)\Gamma (d-2)\zeta (d-2),$$
is finite  and, thus,  the gap equation \eTGNNgapiii\ has a solution up to a temperature $T_c$  where $M_T$ vanishes:
$$T_c=\left[{ 1\over g_d(0)}\left({1\over G_c}-{1\over
G}\right)\right]^{1/(d-2)}=\left[{\Omega_d(0) -\Omega_d(M)\over g_d(0)}\right]^{1/(d-2)}  .\eqnd\eGNNTTc $$
Moreover, the $p=0$ limit of the $\sigma $ propagator  \eNTDsigsym\ confirms that the symmetric phase is stable only if $T>T_c$.
Therefore, $T_c$ separates two
Ising-like phases, a low temperature  phase with symmetry breaking and a symmetric high
temperature phase. Since ${\cal G}_2(M_T,T)$ is a regular function of $M_T $
at $M_T=0$, one finds that, near $T_c$,
$$M_T^2\propto \Lambda ^{4-d}(T_c^{d-2}-T^{d-2})\propto \Lambda (T_c-T), $$
that is a quasi-gaussian or mean-field behaviour in all dimensions. \par
More specific results require considering various $d$ dimensions separately, first $d>2$, the case $d=2$ requiring a special analysis.
\smallskip
{\it High dimensions.} For $d>4$, the critical temperature scales like
$$T_c \propto M(\Lambda /M)^{(d-4)/(d-2)}\ \Rightarrow M\ll T_c \ll  \Lambda \,,$$
and, thus, $T_c$ is physical and large compared with the particle masses.
 \par
In the symmetric high temperature phase $T>T_c$, the $\sigma$ thermal mass ($\sigma $ inverse correlation length) behaves like
$$m_\sigma^2 \propto T^  2 (T/\Lambda )^{d-4}\left[ 1-(T_c/T)^{d-2}\right], $$
and thus is small with respect to $T$, justifying dimensional reduction.
For $T<T_c$ but $T\gg M$, one finds $|M_T^2-M^2|\ll T^2$ and the property remains true. \sslbl\ssNTGNNphase
\smallskip
{\it Dimension $d=4$.} In the high temperature symmetric phase, the thermal mass $m_\sigma $  is given by
$$m_\sigma^2 \propto {1\over \ln(\Lambda /T)}\left({1 \over  G}
-{1\over G_c}+  T^2g_4(0)\right). $$
The thermal  mass $m_\sigma $ is physical (i.e.~$m_\sigma \ll \Lambda $) only  for $|1/G-1/G_c|\ll \Lambda
^2$, that is in the critical domain of the zero temperature theory.
For $G>G_c$, one can introduce the critical temperature (Eq.~\eGNNTTc):
$$m_\sigma^2 \propto {1\over \ln(\Lambda /T)}(T^2-T_c^2).$$
Eventually, $m_\sigma $ vanishes as $(T-T_c)^{1/2}$ a typical mean-field or gaussian behaviour, and a
phase transition occurs.  Eq.~\eGNNTTc\ yields $T_c$ which,  expressed in terms of the
physical fermion mass $M $,  is given by
$$(T_c/M)^2\sim {3\over \pi^2}\ln(\Lambda /M).$$
The critical temperature is thus large compared with the
physical mass $M$.  \par
In the broken symmetry phase, for $T/M$ finite, the mass parameter $M_T$ remains
close to $M$
and one finds
$$\left(M_T\over M\right)^2=1- 8\pi^2  g_4(M^2/T^2)\left(T\over M\right)^2{1\over \ln (\Lambda /M)}\,.$$
\medskip
{\it Dimension $d=3$} \refs{\rGNDiiirev, \rNTGN}. In the symmetric phase
$$m_\sigma ^2\propto   {T \over  G} -{T\over G_c}+  T^2 g_3(0) . $$
The mass parameter $m_\sigma $ is physical only if $T(1/G-1/G_c)\ll \Lambda ^2$. At
the transition coupling constant
$G_c$, one finds as expected $m_\sigma \propto T$. \par
In the broken symmetry phase since at leading order $\Omega_d(M)-\Omega_d(0)=-M/4\pi$ and
$$g_3(s)={1\over 2\pi}\ln\left(1+\e^{-\sqrt{s}}\right),$$
the gap equation can be written as
$$2\cosh(M_T/2T)=\e^{M/2T},$$
an equation that has a scaling form expected for $d<4$ from the correspondence between GN and GNY models, and the existence
of an IR fixed point in the latter.
The critical temperature  is proportional to the fermion mass:
$$T_c/M={1\over 2\ln2}\,.$$
%the equation
%$$T_c=(2\pi/N')(G-G_c)/GG_c\,.$$
\medskip
{\it Dimension $d=2$} \refs{\rDaMaRa,\rBCMPG}. The situation $d=2$ is doubly special, since at zero temperature chiral
symmetry is always broken and at finite temperature the Ising symmetry is
never broken.  The GN model is renormalizable and UV free. For $N$ large
$$\beta(G)=-{1  \over 2\pi}G^2 . $$
All masses are  proportional to the RG invariant mass scale
$$\Lambda(G) \propto\Lambda \exp\left[-\int^G{\d G'\over \beta (G')}\right] .
$$
In particular, the fermion physical mass   has the form
$$M\propto \Lambda \e^{-2\pi/G}. $$
At finite temperature all masses, in the sense of inverse of the correlation length in the space direction, have a scaling property. For example, the
$\sigma $ mass has the form
$$m_\sigma/T=f(M/T). $$
For $T>M$, one can also express the scaling properties by introducing a temperature
dependent coupling constant $G_T$ defined by
$$\int_G^{G_T}{\d G'\over \beta(G')}=-\ln( \Lambda/T ).$$
At high temperature, $G_T$ decreases like
$$G_T  \sim {2\pi \over \ln( T/M)}.$$
One expects,  therefore, trivial high temperature physics with weakly interacting fermions.  \par
At high temperature  the mass parameter $m_\sigma $ is proportional to $T$
and, therefore, the zero-mode is not different from other modes. Eventually, it decreases
when $T$ approaches $T_c=M/\pi$. At leading order one finds
$$m_\sigma ^2\propto T^2 \ln(\pi T/M)  .\eqnd\eGNNTmii $$
This  result does not imply a phase transition since, for  $m_ \sigma /T\ll1 $, dimensional reduction is justified: the statistical system becomes essentially  one-dimensional  with short-range interactions and thus can have no phase transition. Due to fluctuations the correlation length $1/m_\sigma $ never diverges. \par
For $T<T_c$, the gap equation becomes
%%$${1\over  G}=-{1\over 2\pi }\ln(\Lambda /T ) -g_2(M_T^2/T^2). $$
$$\ln(M/M_T)=2\pi g_2(M_T^2/T^2).$$
%Introducing the physical mass M
%we can rewrite the equation
%$$ {1\over 2\pi}\ln(M/T ) =g_2(M_T^2/T^2).$$
The function $g_2(s)$ is positive, which again implies $M_T<M$, goes to $ \infty$ for $s\to0$
and goes to $0$ for $s\to  \infty $.  At low temperature, $M_T/M$ converges to 1 exponentially in $M/T$.
%$${M_T\over M}-1\sim -(T/M)^{1/2}\e^{-M/T}.$$
At high temperature, $M_T/T$ goes to zero and
$$g_2(s)\sim -{1\over 4\pi}\ln s\,.$$
The equation implies $M\propto T$ and thus has no solution for $T\to\infty$, but instead  solutions at finite temperature, in agreement with Eq.~\eGNNTmii, which shows that $m_\sigma $ vanishes for some value $T_c\propto M$. The existence of non-trivial solutions to the gap equation here implies
only that the $\sigma $ potential has degenerate minima, but as a consequence of fluctuations
the expectation value of $\sigma $ nevertheless vanishes. \par
More precisely, the expansion \eFTGNloc\ can be applied to the $d=2$ example. One obtains a simple model in 1D quantum
mechanics: the quartic anharmonic oscillator. Straightforward considerations
show that the correlation length, inverse of the $\sigma $ mass parameter,
becomes small only when the coefficient of $\sigma^2$ is large and negative.
This happens only at low temperature where the two lowest eigenvalues of the
corresponding quantum hamiltonian are almost degenerate. Then, instantons  relate  the two classical minima of the potential and restore the symmetry.  Calculating the difference between the two lowest eigenvalues
of the  hamiltonian, one obtains a behaviour of the form
$$m_\sigma/T \propto (\ln M /T)^{5/4}\e^{-{\rm const.}(\ln M/T)^{3/2}}. $$
Again, the property that $m_\sigma /T$ is small at low temperature
is a precursor of the zero temperature phase transition.

%
\subsection Abelian gauge theories

The presence of gauge fields introduces some new features in finite temperature field theories, because the $O(d)$ space symmetry is explicitly broken. This affects the
mode decomposition of the gauge field and quantization. Therefore, we  discuss here only QED-like theories, that is abelian gauge fields coupled to fermions with $N$ flavours in $(1,d-1)$ dimensions, because the mode decomposition is gauge invariant  and the quantization simpler  as we now recall, before describing some explicit large $N$ calculations \rTgauge.
\medskip
{\it Mode expansion.} We first describe the effect of a mode
expansion on an abelian gauge field $A_\mu$. We set
$$A_\mu(t,x)=B_\mu(x)+Q_\mu(t,x), $$
where $B_\mu(x)$ is the zero-mode and, thus, $Q_\mu(t,x)$  satisfies
$$\int_0^\beta \d t\,Q_\mu(t,x)=0\,.$$
Gauge transformations then become
$$\delta A_\mu(t,x)= \partial_\mu \left[\varphi_1(t,x)+\varphi_2(t,x) \right] \eqnd\eNTAmugau $$
with
$$\delta Q_\mu(t,x)= \partial_\mu \varphi_1(t,x),\quad
\delta B_\mu(x)= \partial_\mu \varphi_2(t,x).$$
The constraints on $B_\mu$ and $Q_\mu$ then imply
$$\varphi_2(t,x)=F(x)+\Omega t\,,\quad \int_0^\beta \d t\, \varphi_1(t,x)={\rm const. }\
,\quad \varphi_1(0,x)=\varphi_1(\beta ,x), \eqnd\eNTBQgau $$
with $\Omega $ constant. \par
The interpretation of this result is simple, $Q_\mu$ transforms as a gauge field but the
gauge function has no zero-mode, $B_i$ transforms as a $(d-1)$-dimensional
gauge field, $B_0$ behaves like a $(d-1)$-dimensional massless scalar, since a translation of $B_0$ by a constant $\Omega$ leaves the action unchanged.
\medskip
{\it Quantization.} We may quantize by adding to the action a term
quadratic in $ \partial_\mu A_\mu$. Then,
$$  \partial_\mu A_\mu= \partial_t Q_0(t,x) + \partial_i B_i(x)
+ \partial_i Q_i(t,x ),$$
and, therefore,
$$\int_0^\beta \d t( \partial_\mu A_\mu)^2=\beta \bigl( \partial_i B_i(x)\bigr)^2
+\int_0^\beta \d t( \partial_\mu Q_\mu)^2.$$
An integration over the non-zero modes, then leads to a $(d-1)$-dimensional
gauge theory, quantized in the same covariant gauge, coupled to a neutral
massless scalar field.
\smallskip
{\it Massless QED.} If one now considers a massless QED-like theory
$${\cal S} (\bar \psib ,\psib,A _{\mu} ) = \int \d^d  x
\left[{\textstyle{ 1 \over 4e^2}}  F ^{2} _{\mu \nu}(x) -
\bar \psib (x) \cdot\left(\sla{ \partial} + i\Abar \right)
\psib(x) \right] , \eqnd\eactQEDN $$
quantized in the same covariant gauge,
one can integrate out the fermions because, due to
anti-periodic boundary conditions, they have no zero mode.  At one-loop order after fermion integration, the leading local corrections take the same form as in the free case, only coefficients are modified.  \par
Note, however, that one can change the fermion situation by introducing
a  chemical  potential term. \par
Gauge transformations
$$\psib(t,x)=\e^{i \varphi (t,x)}\psib'(t,x), $$
must now preserve the anti-periodicity of the charged fields, and thus be periodic. This implies
$$\varphi( \beta ,x)=\varphi (0,x) \pmod{ 2\pi} .$$
In terms of the decomposition $\varphi=\varphi_1+\varphi_2$, this condition implies that the parameter $\Omega $ in \eNTBQgau\ is quantized:
$$\Omega =2 \pi n T\,,\quad n\in{\Bbb Z}\,.$$
This affects the transformation of the scalar field $B_0$:
$$\delta B_0(x)=2 \pi n T\,$$
and, therefore, the thermodynamic potential is a periodic function of
$B_0$. One consequence of this quantization, then, is that
the field $B_0$, which is massless in the tree approximation, acquires a mass  from quantum corrections, as we verify below by explicit calculations.
\smallskip
{\it Temporal gauge.}
It is also instructive to quantize in the temporal gauge.  To change gauges, we make a gauge transformation with a periodic function
$$ \varphi(t,x)=\int_0^t \d t\, A_0(t',x)-{t\over \beta }\int_0^\beta  \d t\,
A_0(t',x) .$$
Then, $A_0$ is reduced to its zero-mode component, and
the other modes only appear in the gauge fixing function $ \partial_\mu
A_\mu$. Integration over these modes reduces it to $ \partial_i A_i$, where
now only the zero-modes of the other fields $A_i$ contribute.
For what concerns zero-modes, the situation is the same as before,
while the non-zero modes are now quantized in temporal gauge.
This means that one finds $d-1$ families of vector fields with masses $2\pi
n T$, $n\ne 0$, quantized in the unitary, and thus non-renormalizable gauge.
 \par
Let us confirm that the same result is obtained by  quantizing in the temporal gauge $A_0(t,x)=0$ directly:
$$ {\cal S} (\bar \psib ,\psib,A _{\mu}  ) = \int \d^{d-1}x\,\d t
\left[{\textstyle{ 1 \over 4e^2}}\left(2\dot A_i^2+  F ^{2} _{ij}(t,x)
\right)- \bar \psib (t,x) \cdot\left(\sla{ \partial} + i\Abar \right)
\psib(t,x) \right]. \eqnd\eactQEDAze $$
One has to remember that Gauss's law has to be imposed. Therefore,
a projector over gauge invariant states has to be introduced in the functional integral. This can be accomplished by imposing periodic, anti-periodic respectively, boundary conditions in the time direction up to a gauge
transformation:
$$\eqalign{A_i(\beta ,x)&=A_i(0,x)-\beta  \partial_i \varphi(x), \cr
\psib(\beta ,x)&=-\e^{i \beta  \varphi(x)}\psi(0,x),\cr
\bar\psib(\beta ,x)&=-\e^{i \beta  \varphi(x)}\bar\psi(0,x).\cr}
$$
We thus set
$$\eqalign{A_i(t,x)&=A_i'(t,x)-t \partial_i \varphi(x), \cr
\psib(t,x)&=\e^{i t \varphi(x)}\psi'(t,x) ,\cr
\bar\psib(t,x)&=\e^{-i t \varphi(x)}\bar\psi'(t,x) ,\cr} $$
where the fields $A'_i$, $\psi',\bar\psib'$ now are  periodic and anti-periodic, respectively. Two
modifications appear in the action:
$$\eqalign{\int\d x\d t\,( \partial_t A_i)^2\ &\mapsto\ \int\d x\d
t\,( \partial_t A_i)^2 + \beta \int\d x\,( \partial_i \varphi(x))^2 \cr
\int\d x\d t\,\bar \psib (t,x)\gamma_0  \partial_ t \psib(t,x) &
\mapsto \int\d x\d t\,\bar \psib (t,x)\gamma_0(  \partial_ t +i\varphi
(x))\psib(t,x) . \cr}$$
Therefore, $\varphi(x)$ is simply the zero-mode of the $A_0$
component. Not enforcing Gauss's law suppresses this additional mode. \par
The theory has a $(d-1)$-dimensional gauge invariance with the zero-mode
of $A_i(t,x)$ as a gauge field.
\medskip
{\it Fermion integration and large $N$ calculations.} We now consider the action \eactQEDN, and integrate over fermions, to render the $N$-dependence explicit:
$$ {\cal S}_N  (A _{\mu}  ) = \int \d^d x\,
{\textstyle{ 1 \over 4e^2}}  F ^{2} _{\mu \nu}(x) -\tilde N\tr\ln
\left(\sla{ \partial} + i\Abar \right) , \eqnn $$
where $\tilde N$ is the number of charged fermions, and below $N=\tilde N\tr{\bf 1}$. \par
The action density as a function of a constant field $\varphi\equiv A_0$ then is  given by
$${1\over N}{\cal F}(\varphi)=-  {T\over2} \sum_n{1\over (2\pi)^{d-1}}\int\d^{d-1} k\,\ln\left[k^2
+\bigl(\varphi+(2n+1)\pi T\bigr)^2\right]. $$
The sum over $n$  can be performed with the help of the identity \eqns{\eFTgenidii} and one obtains
$${1\over N}{\cal F}(\varphi)=- {T\over2}    \int {\d^{d-1} k\over (2\pi)^{d-1}}\,
\ln\bigl(\cosh(\beta k )+\cos(\beta\varphi) \bigr).\eqnd\eFTdetQED $$
One verifies that the difference ${\cal F}(\varphi)-{\cal F}(0)$   is UV finite
and has a scaling form $T^d f(\varphi/T)$.
This is not surprising since in the zero temperature  limit no
gauge field mass and quartic $\varphi$ potential are generated.  \par
The derivative
$${1\over N}{\cal F}'(\varphi)={1\over2}
{1\over(2\pi)^{d-1}}  \sinh(\beta \varphi)\int {\d^{d-1} k\over \cosh(\beta k)+\cos(\beta\varphi)},$$
is negative for $-\pi<\beta \varphi<0$ and positive for $0<\beta \varphi<\pi$.
The action density has a unique minimum at $\varphi=0$ in the interval
$-\pi<\beta \varphi<\pi$. \par
A special case is $d=2$ for which one finds \rTSchwinger
$$ {\cal F}'(\varphi)=\ud N \varphi/\pi \quad {\rm for}\
|\varphi|<\pi\quad{\rm and\ thus}\ {\cal
E}(\varphi)=\frac{1}{4} N \varphi^2/\pi\,. $$
 \par
Neglecting all $\varphi$ derivatives, one obtains a contribution to
the action ${\cal S}_T$:
$$-\tilde N\tr \ln \left(\sla{ \partial} + i\Bbar \right) \sim -{1\over2} N\int\d^{d-1} x
\int {\d^{d-1} k\over (2\pi)^{d-1}}\,
\ln\bigl(\cosh(\beta k )+\cos(\beta\varphi(x)) \bigr).$$
The coefficient $K_2(d)$ of $\ud\int\d x\,\varphi^2(x)$ follows:
$$\eqalign{ K_2(d)& =NN_{d-1}\Gamma (d-1)\left(1-d^{3-d}\right)\zeta (d-2)T^{d-3}\cr
&=N {8\over
(4\pi)^{d/2 }}\Gamma ( d/2 )\left(2^{d-3}
-1\right)\zeta(d-2)T^{d-3} . \cr}$$
\smallskip
{\it Discussion.} At leading order, the mass term is thus
proportional to $e T^{(d-2)/2}$. If $e$ is generic, that is of order 1
at the microscopic scale $1/\Lambda $, then $e\propto
\Lambda^{(4-d)/2}$ and the scalar mass $m_T$ is proportional to
$(\Lambda/T)^{(4-d)/2} T$. It is thus large with respect to the vector masses for $d<4$ and small for $d>4$.  \par
If one takes into account loop corrections, one finds for $d>4$ a finite coupling
constant renormalization $e \mapsto e_\r$, and the conclusion is not changed.
The zero-mode becomes massive but with a mass small compared with $T$, justifying  mode and local expansions.
 \par
For $d=4$,  QED is IR free,
$$\beta_{e^2}={\tilde N\over 6\pi^2} e^4 +O(e^6),$$
$e_\r$ has to be replaced by the effective coupling constant
$e(T/\Lambda)$, which is logarithmically small:
$$e^2(T/\Lambda)\sim {6\pi^2\over \tilde N\ln(\Lambda/T)}, $$
and the scalar mass thus is still small, although only logarithmically:
$$m^2_\varphi\propto {T^2\over \ln(\Lambda/T)}.$$
The separation between zero and non-zero modes remains justified. High temperature QED shares
some properties with high temperature $\phi^4$ field
theory, and a perturbative expansion for the same reason remains meaningful. \par
Note that if one is interested  in IR physics only, one can in a second
step integrate over the massive scalar field $\varphi$.
 \par
Finally for $d<4$, one finds an IR fixed point and, therefore,
one expects that in massless QED $m_T$ becomes proportional to $T$
and comparable to all other modes, in particular, to all gauge field
non-zero modes that become massive vector fields.
%\smallskip
%{\it Quantization.}
%%%%%%%%%%%%%%%%%%%%%%%%%%%%%%%%%%%%%%%%%%%%%%%%%%%%%%%%
%To quantize the theory we still have to fix the gauge, using for instance
%a covariant gauge. For what concerns the massive modes, they
%are here quantized in a unitary non-renormalizable gauge. Therefore,
%for these also a change of gauge is required for renormalizability purpose.
%This is not difficult in the abelian case because we can make an %independent
%gauge transformation on each vector field. Furthermore, we note that the %vector masses are not renormalized.
%%%%%%%%%%%%%%%%%%%%%%%%%%%%%%%%%%%%%%%%%%%%%%%%%%%%%%%%%
