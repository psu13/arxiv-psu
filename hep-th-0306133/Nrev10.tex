%%%%%% JZJ Apr 1, last line of introduction supressed
%%%%%%% because conclusion section has been removed

\nref\rEMNZJ{The following sections are taken mainly from \rf G.
Eyal, M. Moshe, S. Nishigaki and J. Zinn-Justin, {\it Nucl. Phys.}
B470 (1996) 369, hep-th/9601080.} \nref\rmatrix{For a review on
matrix models and double scaling limit see \rf
 P. Di Francesco, P. Ginsparg and J. Zinn-Justin,
{\it Phys. Rep.} 254 (1995) 1.} \nref\rNscaling{Previous
references on the double scaling limit in vector models include
\rf A. Anderson, R.C. Myers and V. Perival, {\it Phys. Lett.} B254
(1991) 89, {\it Nucl. Phys.} B360 (1991) 463. and
refs.~\refs{\rNishi{--}\rYoneya}.} \nref\rNish{S. Nishigaki and T.
Yoneya, {\it Nucl. Phys.} B348 (1991) 787.} \nref\rDiVe{ P. Di
Vecchia, M. Kato and N. Ohta, {\it Nucl. Phys.} B357 (1991)
495.}\nref\rZinnJ{ J. Zinn-Justin, {\it Phys. Lett.} B257 (1991)
335.}\nref\rDiVecc{ P. Di Vecchia, M. Kato and N. Ohta, {\it Int.
J. Mod. Phys.}A7, (1992)1391.}\nref\rYoneya{T. Yoneya, {\it Prog.
Theo. Phys. Suppl.} 92, 14 (1992).} \nref\rmatrixDi{The $d=1$
matrix problem is discussed in \rf P. Ginsparg and J. Zinn-Justin,
{\it Phys. Lett.} B240 (1990) 333 and
refs.~\refs{\rBrezin{--}\rGrossM}.}\nref\rBrezin{E. Br\'ezin, V. A.
Kazakov, and Al. B. Zamolodchikov, {\it Nucl. Phys.} B338 (1990)
673.}\nref\rPa{ G. Parisi, {\it Phys. Lett.} B238 (1990) 209, 213;
{\it Europhys. Lett.} 11 (1990) 595.}\nref\rGrossM{D. J. Gross and
N. Miljkovic, {\it Phys. Lett.} B238 (1990) 217.}

% large N in 0 and 1 dimensions
\nref\rBBFS{Early large $N$ calculation in zero and one dimensions
are reported in \rf A.J.~Bray, {\it J.~Stat.~Phys.} 11 (1974) 29;
{\it ibidem} {\it J. Phys.} A7 (1974) 2144 and ref.~\rFerrelS.
}\nref\rFerrelS{R.A. Ferrel and D. Scalapino, {\it Phys. Rev.} A9
(1974) 846.}
%
\nref\rNtricrit{References on tricritical behaviour and massless
dilaton include \rf W.A.~Bardeen, M.~Moshe, M.~Bander, {\it
Phys.~Rev.~Lett.} 52 (1984) 1188 and
refs.~\refs{\rDavidKN{--}\rSchnitzer}.}\nref\rDavidKN{ F.~David,
D.A.~Kessler and H.~Neuberger, {\it Phys.~Rev.~Lett.} 53 (1984)
2071, {\it Nucl.~Phys.} B257 [FS14] (1985) 695.}\nref\rKessler{
D.A. Kessler and H. Neuberger, {\it Phys.~Lett.} 157B (1985)
416.}\nref\rDiVecMM{ P. Di Vecchia and M. Moshe, {\it
Phys.~Lett.~}B300 (1993) 49.}\nref\rSchnitzer{ H.J.~Schnitzer,
{\it Mod.~Phys.~Lett.~}A7 (1992) 2449.}

\section{Multicritical points and double scaling limit}

We use now the general formalism presented in section \ssNbosgen\
to discuss various additional issues concerning the general
$N$-vector model with one scalar field, in the large $N$ limit
\rEMNZJ. One obvious application concerns multi-critical points.
Of particular interest are the subtleties involved in the
stability of the phase structure at critical dimensions. The
example of the tricritical $(\phib^2)^3$ theory will illustrate
explicitly, however, the limitations of the large $N$
method.\sslbl\ssdblescal\par Another issue involves the so-called
{\it double scaling limit}. Statistical mechanical properties of
random surfaces as well as randomly branched polymers can be
analyzed within the framework of large $N$ expansion. In the same
manner in which matrix models in their double scaling limit
provide representations  of dynamically triangulated random
surfaces summed on different topologies \rmatrix, $O(N)$ symmetric
vector models represent discretized branched polymers in this
limit \refs{\rNscaling{--}\rYoneya}, where $N\to\infty$ and the
coupling constant $g \to g_{c}$ in a correlated manner. The
surfaces in the case of matrix models, and the randomly branched
polymers in the case of vector models are classified by the
different topologies of their Feynman graphs and thus by powers of
$1/N$. Though matrix theories attract most  attention, a detailed
understanding of these theories exists only for dimensions  $d
\leq 1$ \rmatrixDi. On the other hand, in  many cases, the $O(N)$
vector models can be successfully studied  also  in dimensions $d
> 1$, and thus, provide us with intuition for the search for a
possible description of quantum field theory in terms of extended
objects in four dimensions, which is a long lasting problem in
elementary particle theory.

The phase structure of $O(N)$ vector quantum field theories at $ N
\to \infty $ is generally well understood, there are, however,
certain cases where it is still unclear which of the features
survives at finite $N$, and to what extent. One such problem is
the multicritical behaviour of $O(N)$ models {\it at critical
dimensions}. Here, one finds that in the $N \to \infty$ limit,
there exists a non-trivial UV fixed point, scale invariance is
spontaneously broken, and the one parameter family of ground
states contains a massive vector and a massless bound state, a
Goldstone boson--dilaton. However, since it is unclear whether
this structure is likely to survive for finite $N$, one would like
to know whether it is possible to construct a local field theory
of a massless dilaton via the double scaling limit, where all
orders in $1/N$ contribute. The double scaling limit is viewed as
the limit at which the attraction between the $O(N)$ vector quanta
reaches a value at $g \to g_c$, at which a massless bound state is
formed in the $N \to \infty$ limit, while the mass of the vector
particle stays finite. In this limit, powers of $1/N$ are
compensated by IR singularities and thus all orders in $1/N$
contribute.

The special case of field theory in two dimension is discussed in
section \label{\stwodim}. In higher dimensions a new phenomenon
arises: the possibility of a spontaneous breaking of the $O(N)$
symmetry of the model, associated to the Goldstone phenomenon.
%\par
Before discussing a possible double scaling limit, the critical
and multicritical points of the $O(N)$ vector model are
re-examined in section \label{\smulticr}. In particular, a certain
sign ambiguity that appears in the expansion of the gap equation
is noted, and related to the existence of the IR fixed point in
dimensions $2<d<4$ discussed in section \label{\sssEGRN}.  In
section \label{\ssNtricritical}, the  interesting physical example
of the tricritical point in three dimension is discussed. Some
insight in the problem is obtained there from  variational
calculations.

\par
In section  \label{\sboundst}, we discuss the subtleties and
conditions for the existence of an $O(N)$ singlet massless bound
state along with a small mass  $O(N)$ vector particle excitation.
It is pointed out that the correct massless effective field theory
is obtained after the massive $O(N)$ scalar is integrated out.
Section \label{\sdouble} is  devoted to the double scaling limit
with a particular emphasis on this limit in theories at their
critical dimensions.
%%%%%%%%%%%%%%%
\subsection{The 2D  $O(N)$ symmetric field theory in the double scaling
limit}

We first re-examine and summarize the results for the $O(N)$
symmetric field theory with a potential $NU(\phib^2/N)$, where
$\phib$ is $N$-component field, in the large $N$ limit in two
dimensions. Indeed, there are no phase transitions in two
dimensions and we therefore discuss this case separately. The
action is\sslbl\stwodim
$${\cal S}(\phib)= N\int\d^2 x \left\{\half \left[ \partial_{\mu} \phib (x)
\right]^{2} +U\left(\phib^2/N\right) \right\} ,\eqnd{\eactONgii}$$
where an implicit cut-off $\Lambda$ is always assumed below. Whenever the
explicit dependence in the cut-off will be relevant, we shall assume a
Pauli--Villars's type regularization replacing the propagator by a regularized form
\epropreg.\par
 As  in section \ssNbosgen, one introduces two fields
$\rho(x)$ and $\lambda(x)$ and uses the identity \egeniden.
The large $N$  action obtained by integrating over the field $\phib$ is then
$${\cal S}_N=N\int\d^2 x \left[U(\rho)-\half \lambda
\rho \right]+\half N \tr\ln(-\nabla^2+\lambda ).\eqnd\eactefNgii $$
%
The integral is  evaluated for $N$ large by the steepest descent method.
The saddle point value $\lambda$ is the
$\phib$-field mass squared, and thus we set in general $\lambda=m^2$.  Since in two dimensions
there is no phase transition, $\left<\phib\right>=0$, the three
saddle point equations \esaddleN{} reduce to two:
 \eqna\esaddNgenii
$$\eqalignno{
U'(\rho_s)&=\half m^2\,,&\esaddNgenii{a}\cr
\rho_s&= \Omega _2(m)\,, & \esaddNgenii{b} \cr}$$
where  the function $\Omega_d(m)$ is defined by Eq.~\etadepole.\par
For $m\ll\Lambda$, one finds
$$\Omega_2(m)={1\over2\pi}\ln(\Lambda/m)+{1\over4\pi}\ln (8\pi K)
+O(m^2/\Lambda^2) , $$
where $K$ is a regularization dependent constant.
%\ln (8\pi C) &= \int_0^\infty\d s\left( {1\over
%D(s)}-\theta(1-s)\right).
\par
As was discussed in the case of quantum mechanics in ref.~\rEMNZJ,
a critical point is characterized by the vanishing at zero
momentum of the determinant of second derivatives of the action at
the saddle point. The mass-matrix has then a zero eigenvalue
which, in field theory, corresponds to the appearance of a new
massless excitation other than $\phib$ (this implies also that the
forces between $\phib$ quanta are attractive, a serious problem in
the case of bosons). \par
 In order to obtain the effective action
for this scalar massless mode, we must integrate over one of the
fields. In the field theory case the resulting effective action
can no longer be written in a local form. In order to discuss the
order of the critical point we only need, however, the action for
space independent fields, and thus, for example, we can eliminate
$\lambda$ using the $\lambda$ saddle point equation. The action
density   ${\cal E} (\rho)$ (Eq.~\eNEner) can then be written as
$${1\over N}{\cal E} (\rho)=U(\rho)-{1\over2}\int^{\lambda(\rho)}\d \lambda' \,\lambda'
{\del \over \del \lambda'} \Omega_d(\sqrt{\lambda'}),\eqnd\eWeffrho$$
where at leading order for $\Lambda$ large
$$\lambda(\rho)=8\pi K \Lambda^2\e^{-4\pi\rho}. $$
The expression  \eWeffrho\ is
valid  for any $d$ and will be used  also in section \sdouble.
Here, it reduces to
$${1\over N}{\cal E}(\rho)=U(\rho)+K\Lambda^2\e^{-4\pi\rho}=U(\rho)+\frac{1}{8\pi}m^2
\e^{-4\pi(\rho-\rho_s)}.$$
A multicritical point is defined by the condition
$$ {\cal E}(\rho)- {\cal E}(\rho_s) =O\left((\rho-\rho_s)^n\right).\eqnd\eWcrit $$
This implies the conditions
$$U^{(k)}(\rho_s)=\half (-4\pi)^{k-1}
m^2\quad \ \ {\rm for}\ \ \ 1\le k\le n-1\,.$$ Note that the
coefficients $U^{(k)}(\rho_s)$ are the coupling constants
renormalized at leading order for $N$ large. If $U(\rho)$ is a
polynomial of degree $n-1$ (the minimal polynomial model), the
multicritical condition in Eq.~\eWcrit\ determines the critical
values of renormalized coupling constants as well as $\rho_s$.
\par
When the fields are space-dependent it is, instead,
 simpler to eliminate $\rho$
because the corresponding field equation
$$U'\bigl(\rho(x)\bigr)=\half \lambda(x)  \eqnd\esadroge $$
is local. This equation can be solved by expanding $\rho(x)-\rho_s$ in a power
series in $\lambda(x)-m^2$:
$$\rho(x)-\rho_s= {1\over2 U''(\rho_s)}\bigl(\lambda(x)-m^2\bigr)
+O\left((\lambda-m^2)^2\right) . \eqnd\esolrola $$
The resulting action for
the field $\lambda(x)$ remains non-local but because, as we
shall see,  adding powers of $\lambda$
as well as adding derivatives make
terms less relevant, only the few first terms of a local expansion of the effective action are important. \par
If in the local expansion of the determinant we keep only the two
first terms, we obtain an action containing at leading order a kinetic
term proportional to $(\del_\mu\lambda)^2$ and the interaction
$(\lambda(x)-m^2)^n$:
$${\cal S}_N(\lambda) \sim N\int\d^2 x\left[{1\over96\pi
m^4}(\del_\mu\lambda)^2 + \frac{1}{n!}S_n \bigl(\lambda(x)-m^2)^n\right],$$
where the neglected terms are of order $(\lambda-m^2)^{n+1}$, $\lambda\del^4
\lambda$, and $\lambda^2\del^2\lambda$ and
$$S_n={1\over N}{\cal E}^{(n)}(\rho_s)[2U''(\rho_s)]^{-n}={1\over N}{\cal E}^{(n)}(\rho_s)(-4\pi m^2)^{-n} .$$
Moreover,  we note that, together with the cut-off $\Lambda$, $m$
now also acts as a cut-off in the local expansion. \par
To eliminate the $N$ dependence in the action we have, as in the
example of quantum mechanics \rEMNZJ, to rescale both the field
$\lambda-m^2$ and space:
%
$$\lambda(x)-m^2=\sqrt{48\pi}m^2 N^{-1/2}\varphi(x)\,,\quad x\mapsto
N^{(n-2)/4}x \,.
\eqnd\elamrescal $$
We obtain
$${\cal S}_N(\varphi)
\sim\int\d^2x\left[\half(\partial_\mu\varphi)^2+\frac{1}{n!} g_n
\varphi^n\right] .$$
In the minimal model, where the polynomial $U(\rho)$ has exactly degree $n-1$,
we find $g_n=6(48\pi)^{(n-2)/2}m^2$.

As anticipated, we observe that derivatives and powers of
$\varphi$ are affected by negative powers of $N$, justifying a
local expansion. However, we also note that the cut-offs
($\Lambda$  or the mass $m$) are now also multiplied by
$N^{(n-2)/4}$. Therefore, the large $N$ limit becomes also a large
cut-off limit.
\medskip
{\it Double scaling limit.} The existence of a double scaling
limit relies on the existence of IR singularities due to the
massless or small mass bound state which can compensate the $1/N$
factors appearing in the large $N$ perturbation theory. We refer
the reader to the examples in ref.~\rEMNZJ\ of a simple integral
$(d=0)$ and a quantum mechanical $(d=1)$ example of the double
scaling limit.
\par
 We now add to the action relevant perturbations
$$\delta_k U=v_k(\rho(x)-\rho_s)^k,  \quad 1\le k\le n-2\,,$$
proportional to $\int\d^2x(\lambda-m^2)^k$:
$$\delta_k {\cal S}_N(\lambda)=N S_k \int\d^2x \bigl(\lambda(x)-m^2\bigr)^k ,$$
where the coefficients $S_k$ are functions of the coefficients
$v_k$. After the rescaling of Eq.~\elamrescal, one finds
$$\delta_k {\cal S}_N(\varphi)=\frac{1}{k!} g_k
N^{(n-k)/2}\int\d^2x\,\varphi^k(x)  \quad \ \ \ 1\le k\le n-2\,.
$$
%(the term $k=n-1$ can always be eliminated by a shift of $\varphi(x)$).
However, unlike quantum mechanics \rEMNZJ, it is not sufficient to
scale the coefficients $g_k$ with the power $N^{(k-n)/2}$ in order
to obtain a finite scaling limit. Indeed, perturbation theory is
affected by UV divergences, and we have just noticed that the
cut-off diverges with $N$. In two dimensions the nature of
divergences is very simple: it is entirely due to the
self-contractions of the interaction terms and only one divergent
integral appears
$$\left<\varphi^2(x)\right>={1\over4\pi^2}\int {\d^2 q\over q^2+\mu^2}\,,$$
where $\mu$ is the small mass of the bound state, required as an IR cut-off to
define perturbatively the double scaling limit.
We can then extract the $N$ dependence:
$$\left<\varphi^2(x)\right>={1\over8\pi}(n-2)\ln N+O(1) .$$
Therefore, the coefficients $S_k$ have also to cancel these UV divergences,
and thus have a logarithmic dependence in $N$ superposed to the natural
power obtained from power counting arguments. In general, for any potential,
(Eq.~\eqns{\enormorderInv})
$$V(\varphi)=:V(\varphi):+\left[\sum_{k=1}{1\over2^k
k!}\left<\varphi^2\right>^k \left(\partial\over\partial \varphi\right)^{2k}
\right]:V(\varphi):\,, \eqnd\eNVnormorder $$
where $:V(\varphi):$ is the potential from which self-contractions have been
subtracted (it has been normal-ordered). For example, for $n=3$
$$\varphi^3(x)=:\varphi^3(x):+3\left<\varphi^2\right> \varphi(x) ,$$
and thus the double scaling limit is obtained with
$$N g_1+{1\over16\pi}\ln N g_3  \ \ \ \ \ {\rm held \ \ fixed \ \ as}
\ \ N \to \infty \ .$$ For the example $n=4$ ~(
$\varphi^4(x)=:\varphi^4(x):+6\left<\varphi^2\right>
\varphi(x)^2-3\left<\varphi^2\right>^2$ ) one finds that the
double scaling limit is obtained when
$$ g_1 N^{3/2}\quad {\rm and}\quad N g_2
+{g_4\over8\pi}\ln N \ \ \ {\rm are  \ held \  fixed  \ as}\ \ \
N\to \infty .$$
%
\subsection{The $O(N)$ symmetric model in higher dimensions: phase transitions}

In higher dimensions, a phase transition  associated with  the spontaneous breaking of the $O(N)$ symmetry is possible. In a first part we thus study the $O(N)$
symmetric $NU(\phib^2/N)$ field theory, in the large $N$ limit in
order  to explore the possible phase transitions and identify the
corresponding multicritical points. \par
 The action is\sslbl\smulticr
$${\cal S}(\phib)=  \int\d^d x \left\{\half \left[ \partial_{\mu} \phib (x)
\right]^{2} +NU\left(\phib^2/N\right) \right\} ,\eqnd{\eactONg}$$
where an implicit cut-off $\Lambda$ is again assumed. \par
Following the strategy of section \ssNbosgen, to which we refer for detail,
we again introduce two auxiliary fields $\lambda (x)$, $\rho (x)=\phib^2(x)/N$,
integrate over $N-1$ components of $\phib$,  and obtain the large $N$ action
$${\cal S}_N=N\int\d^d x \left[\half \left(\partial_\mu\sigma\right)^2+
U(\rho)+\half \lambda\left(\sigma^2/N- \rho\right)\right]+\half
(N-1)\tr\ln(-\nabla^2+\lambda ),\eqnd{\eactefNg}$$
with $\sigma (x)\equiv \phib_1(x)$.
%
\medskip
{\it The saddle point equations: the $O(N)$ critical point.}
The large $N$ saddle point equations are given by Eqs.~\esaddleN{}:
 \eqna\esaddNgen
$$\eqalignno{m^2\sigma&=0\,, & \esaddNgen{a} \cr
U'(\rho)&=\half m^2\,,&\esaddNgen{b}\cr
\sigma^2/N&=\rho-\Omega _d(m)\,. & \esaddNgen{c} \cr}$$
In the ordered phase $\sigma\ne0$ and thus $m$ vanishes.
Eq.~\esaddNgen{c} has a solution only for $\rho>\rho_c$,
$$\rho_c={1\over (2\pi)^d }\int^\Lambda {\d^d k \over k^2}\ \ \Rightarrow\
\sigma=\sqrt{\rho-\rho_c}\,.$$
Eq.~\esaddNgen{b} which reduces to $U'(\rho)=0$ then yields the critical temperature. Setting
$$U(\rho)=V(\rho)+\half r\rho ,$$ one finds
$$r_c=-2 V'(\rho_c).$$
In order to find the magnetization critical exponent $\beta$, we
need the relation between the $r$ and $\rho$ near the critical
point.
\par
 In the disordered phase, $\sigma=0$, Eq.~\esaddNgen{c}
relates $\rho$ to the $\phib$-field mass $m$. For $m\ll\Lambda$,
$\rho$ approaches $\rho_c$, and the relation becomes
(Eq.~\etadepolii)
$$\rho-\rho_c =-K(d)
m^{d-2}+a(d)m^2\Lambda^{d-4}
+O\left(m^4\Lambda^{d-6}, m^d \Lambda ^{-2}\right). \eqnd\edmum  $$
The constant $K(d)$ is universal (Eq.~\etadpolexp{a}).
The constant  $a(d)$, which also appears in Eq.~\etadepolii, on the other hand,
depends on the cut-off procedure (Eq.~\eadef). \par
For $2<d<4$, (the situation we  assume below except when stated otherwise)
%the $ O\left(m^d\Lambda^{-2}\right)$  from the
%non-analytic part dominates the corrections to the leading part of this expression.\par
the universal non-analytic part in $m^{d-2}$ dominates.\par
For $d=4$, Eq.~\edmum\ becomes
$$\rho-\rho_c=\frac{1}{8\pi^2} m^2\left(\ln m/\Lambda+\ {\rm const.}\right),$$
and for $d>4$ the analytic contribution dominates and
$$\rho-\rho_c\sim a(d)m^2\Lambda^{d-4}.$$
%$$a(d)= {1\over (2\pi)^d}\int{\d^d k\over k^4}\left(1-{1\over
%D^2(k^2)} \right)  .  \eqnd\eadef $$
\medskip
{\it Critical point.} In a generic situation $V''(\rho_c)=U''(\rho_c)$ does
not vanish, a situation we have examined in section \ssNUfige. We  find in the low temperature phase
$$t=r-r_c\sim -2 U''(\rho_c)(\rho-\rho_c) \ \Rightarrow\ \beta=\half\,.
\eqnd\erminrc$$
This is the case of an ordinary critical point. Stability implies
$U''(\rho_c)>0$ so that $t<0$.\par
At high temperature, in the disordered phase, the $\phib$-field mass $m$ is
given by $2V'(\rho)+ r=m^2$ and thus, using \edmum, at leading order
$$t\sim 2U''(\rho_c)K(d)m^{d-2}.$$
Of course, the
simplest realization of this situation is to take $U(\rho)$ quadratic, and we
recover the $(\phib^2)^2$ field theory.
\medskip
{\it Multicritical points.} A new situation arises if we can
adjust the parameters of the potential in such a way that
$U''(\rho_c)=0$. This can be achieved only if the potential $U$ is
at least cubic. We then expect a tricritical behaviour \rNtricrit.
Higher critical points can be obtained when more derivatives
vanish. We shall examine the general case though, from the point
of view of real phase transitions, higher order critical points
are not especially interesting. Indeed, for continuous symmetries
phase transitions occur only for $d>2$ and quasi-gaussian
behaviour is then obtained for all dimensions $d\ge 3$. The
analysis will, however, be useful in the study of  double scaling
limit. \par
Assuming that the first non-vanishing derivative is
$U^{(n)}(\rho_c)$, we expand further Eq.~\esaddNgen{b}. In the
ordered low temperature phase, we now find
$$t=-{2\over (n-1)!}U^{(n)}(\rho_c)(\rho-\rho_c)^{n-1},\ \Rightarrow\
\sigma\propto (-t)^\beta,\quad \beta={1\over2(n-1)}\,  ,  \eqnd\ecritbeta $$
which is the magnetic exponent  obtained in the mean field approximation for such a multicritical point.
We have in addition the condition $U^{(n)}(\rho_c)>0$.\par
In the high temperature phase, instead,
$$m^2=t+ (-1)^{n-1}{2\over (n-1)!}U^{(n)}(\rho_c)K^{n-1}(d)m^{(n-1)(d-2)}.
\eqnd\emsq$$
For $d>2n/(n-1)$, for $m$ small the equation reduces to $m^2=t$, which yields a simple gaussian behaviour, as expected above the upper-critical dimension. \par
For $d<2n/(n-1)$, we find a peculiar phenomenon, the term in the r.h.s.\ is
always dominant, but depending on the parity of $n$ the equation has solutions
for $t>0$ or $t<0$. For $n$ even, $t$ is positive and we find
$$m\propto t^\nu,\qquad \nu={1\over(n-1)(d-2)},\eqnd\emoft$$
which is a non gaussian behaviour below the critical dimension.
However, for $n$ odd (this includes the tricritical point), $t$ must be
negative,
in such a way that we have now two competing solutions at low temperature.
We have to find out which one is stable. We verify below that only the
ordered phase is stable, so that the correlation length of the $\phib$-field
in the high temperature phase remains always finite. Although these dimensions
do not correspond to physical situations because $d<3$, the result is
peculiar and inconsistent with the $\varepsilon$-expansion.
\par
For $d=2n/(n-1)$, we find a gaussian behaviour without logarithmic
corrections, provided the condition
$$U^{(n)}(\rho_c)< \Omega _c\,,\quad \Omega _c=\ud(n-1)! \left[K(2n/(n-1))\right]^{1-n},\qquad
K(3)=1/(4\pi),\eqnd\etriccond$$
is met. In particular, we will see that the special point
$$ U^{(n)}(\rho_c) =\Omega _c
\eqnd\eendtric$$
 has several peculiarities. \par
We examine, in section \label{\ssNtricritical}, in more detail,  the most interesting example: the tricritical point.
 \medskip
{\it Discussion.} In the tree approximation, the dominant configuration
is given by the minimum of the function $U(\rho)\propto\rho^n$. For $n$ odd,
the function is not bounded from below, but $\rho=0$
is the minimum because by
definition $\rho\ge0$. Here, however, we are in the situation where $U(\rho)
\sim (\rho-\rho_c)^n$ with $\rho_c$ is positive. Thus, this extremum of the
potential is likely to be unstable for $n$ odd. To check the global
stability requires further work. This difficulty shed immediately some doubts about the possibility of studying  such multicritical
points  by the large $N$ method.
\par
Another point to notice concerns renormalization group: the $n=2$ example is
peculiar in the sense that the large $N$ limit exhibits a non-trivial IR fixed
point. For higher values of $n$, no coupling renormalization arises in the
large $N$ limit and only the gaussian fixed point is found. We are in a
situation quite
similar to usual perturbation theory, the $\beta$ function can only be
calculated perturbatively in $1/N$ and below the upper-critical dimension the IR fixed point is outside the $1/N$ perturbative regime.
\medskip
{\it Local stability and the mass matrix.}
The matrix of the general second partial derivatives of the action \eactefNg\
is
$$N\pmatrix{p^2+m^2 & 0 & \sigma \cr
0 & U''(\rho) & -\half \cr
\sigma & -\half & -\frac{1}{2}B_\Lambda(p,m)  \cr}  ,  \eqnd\ematrix$$
where $B_\Lambda(p,m)$ is defined in \ediagbul.\par
We are in position to study the local stability of the critical points.
Since the integration contour for $\lambda=m^2$ should be parallel to the
imaginary axis, a necessary condition for stability is that the determinant
remains negative.
\smallskip
{\it The disordered phase.} Then $\sigma=0$ and thus we have only to study
the $2\times2$ matrix $\bf M$ of the $\rho,m^2$  subspace. Its determinant
must remain negative, which implies
$$\det{\bf M}<0\ \Leftrightarrow\
2U''(\rho)B_\Lambda(p,m)+1>0\,.\eqnd\estable $$
For Pauli--Villars's type regularization, the function
$B_\Lambda(p,m)$ is decreasing
so that this condition is implied by the condition at zero momentum
$$\det{\bf M}<0\ \Leftarrow\ 2U''(\rho)B_\Lambda(0,m)+1>0\,.$$
For $m$ small, we use Eq.~\eBLamze\ and
%find
%$$B_2(0;m^2)=\half(d-2)K(d)m^{d-4}-a(d)\Lambda^{d-4}+O\left(m^{d-2}\right),
%\eqnd\eBmzero$$
at leading order the condition becomes
$$ K(d)(d-2)m^{d-4}U''(\rho)+1>0\,.$$
This condition is always satisfied by a normal critical point since $U''(\rho_c)>0$.
For a multicritical point, and taking into account Eq.~\edmum, one finds
$$(-1)^n {d-2\over (n-2)!}K^{n-1}(d)m^{n(d-2)-d}U^{(n)}(\rho_c)+1>0\,.
\eqnd\multicr $$
We obtain a result consistent with our previous analysis. For $n$ even it is
always satisfied; for $n$ odd, it is always satisfied above the critical
dimension and never below. At the upper-critical dimension we find a condition
on the value of $U^{(n)}(\rho_c)$, which we recognize to be identical to
condition \etriccond\ because then $2/(n-1)=d-2$.
\smallskip
{\it The ordered phase.} Now $m^2=0$ and the determinant $\Delta$ of the
complete matrix is
$$-\Delta>0\ \Leftrightarrow\
2U''(\rho)B_\Lambda(p,0)p^2+p^2+4U''(\rho)\sigma^2>0\,.\eqnd\eorder$$
We recognize a sum of positive quantities, and the condition is always
satisfied. Therefore, in the case where there is a competition with a disordered saddle point, only the ordered one can be stable.
%
%%%%%%%%%%%%%%%%%%%%%%%%%%%%%%%%%%%
\subsection{The tricritical point: variational analysis}

The most interesting physical example is the tricritical point in
three dimensions. Some insight in the problem can be obtained from
the variational calculations of section \ssNVarfiv.
Eq.~\esaddleNcii\ becomes\sslbl\ssNtricritical
$$\rho -\rho _c={\sigma ^2 \over N} -{m \over 4\pi}\,.$$
The variational action density for large $N$ then reads
$${\cal E}_{\rm var.}/N\mathop{\sim}_{N\to \infty }    -{m^3\over 12\pi}+
 U(\rho)-\ud m^2 (\rho-\rho _c) \,,$$
The polynomial of minimal degree near a tricritical point can be parametrized as
$$U(\rho)={1\over3!}U'''(\rho _c) (\rho-\rho _c)^3+{1\over2}t (\rho-\rho _c).$$
In the disordered phase, after elimination of $\rho -\rho _c$,
the variational action density becomes
$$ {\cal E}_{\rm var.}/N\mathop{\sim}_{N\to \infty } {m^3\over24\pi}\left(1-
  U'''(\rho _c)/ \Omega _c\right) -{1\over8\pi}m t \,.$$
For $U'''(\rho _c)/ \Omega _c>1$,  the action is unbounded from below
and, in the absence of other interactions, $m$ is of the order of the cut-off,
independently of the value of $t$, in such that way that the transition has disappeared.  For $U'''(\rho _c)/ \Omega _c<1$, instead, and $t>0$, one finds a minimum such that
$$m^2=\left(1-  U'''(\rho _c)/ \Omega _c\right)^{-1/2} t^{1/2},$$
in agreement with Eq.~\emsq. For $t<0$ the minimum corresponds to $m=0$.
\par
It is also easy to understand from the same arguments what happens for general $d$. The variational action density reads (Eq.~\edmum)
$$ {\cal E}_{\rm var.}/N\mathop{\sim}_{N\to \infty } -{1\over3!}U'''(\rho _c) K^3(d)
m^{3(d-2)}+{d-2\over 2d} K(d)m^{d}-{1\over2}K(d)t m^{d-2}.$$
In the formal situation $d<3$ and for $t=0$, the negative term proportional to $m^{3(d-2)}$ dominates for $m$ small. The minimum is obtained for
$$m^{2(3-d)}=K^2(d)  U'''(\rho _c),$$
that is for a mass of the order of the cut-off. Of course, when $m$ is of the order
of the cut-off, several expressions based on a small $m$ expansion are no longer valid.
For $t>0$, Eq.~\emsq\ now has a different interpretation: it has a solution because $m$ again is of the order
of the cut-off and  $m^2$ is larger than $m^{2(d-2)}$. Therefore, no particle propagates in the high temperature phase. For $t<0$, by contrast, the term in $tm^{d-2}$ dominates for $m$ small
and ${\cal E}_{\rm var.}$ first increases. The solution of Eq.~\emsq\ yields a
small value for $m$, but it corresponds to a maximum. Thus, for all values of
$t\ll \Lambda ^2$, the contribution proportional to $t$ is negligible at the
minimum of the action density and $m$ remains of the order of the cut-off.
This situation is clearly pathological and invalidates the large $N$ expansion. \par
By contrast for $d>3$, the term proportional to $U'''(\rho _c)$ is negligible,
the minimum is given by Eq.~\emsq, and the quasi-gaussian (or mean-field) theory applies.
\par
In the ordered phase, for $d=3$,
$$ {\cal E}_{\rm var.}/N\mathop{\sim}_{N\to \infty }
  {1\over3!}U'''(\rho _c){\sigma ^6\over N^3}+{t\over2}{\sigma ^2\over N}\,.$$
For $t>0$, the minimum corresponds to $\sigma =0$. For $t<0$,
the minimum is given by
$$\sigma ^2/N=[-t /U'''(\rho _c)]^{1/2}.$$
Therefore, in contrast of what is seen in $d<3$, the variational analysis confirms
for $U'''(\rho _c)/ \Omega _c<1$ a non-pathological situation.
\smallskip
{\it Dimension three: the end-point $U'''(\rho _c)=\Omega _c$.}
In the disordered phase the action density reduces to one term. It is interesting to add the first correction for $m$ small
$$ {\cal E}_{\rm var.}/N\mathop{\sim}_{N\to \infty }  -{1\over8\pi}m t
+{a(3)\over 4}{m^4 \over \Lambda }\,.$$
The results now depend on
the sign of $a(3)$. A comment concerning the non-universal
constant $a(d)$, given in \eadef\ , is here in order because,
while its absolute value is irrelevant, its sign plays a role in
the discussion of multicritical points. The relevance of this sign
to the RG properties of the large $N$ limit of the simple
$(\phib^2)^2$ field theories has already mentioned (section
\sssEGRN). For the simplest Pauli--Villars's type regularization,
 $D(k^2) $ is an increasing function and thus $a(d)$ is
finite and positive in dimensions $2 < d < 4$, but this clearly is
not a universal feature.\par
 If $a(3)$ is positive, for $t>0$ the
action density has a minimum at a non-trivial value of $m$:
$$ m\sim \left(t\Lambda \over 8\pi a(3)\right)^{1/3}.$$
For $t$ small, $m$ is small, justifying the analysis, but has a different scaling behaviour, $\nu =1/3$, a property one would expect from an IR unstable fixed point.
For $t<0$, the minimum occurs at $m=0$, as one expects for a phase transition.\par
If $a(3)$ is negative, the action density, in this small $m$ approximation, is not bounded from below and the situation is pathological.
\par
The variational analysis has clarified some peculiarities of the large $N$ limit. However, it does not allow to investigate whether the results survive at $N$ finite.
This is here specially relevant because, for example, the  point $U'''(\rho _c)=\Omega _c$ is at the boundary of a pathological region in parameter space, and its peculiar properties
are expected to be especially sensitive to $1/N$ corrections.
To calculate them, one has to return to the steespest descent method.
%%%%%%%%%%%%%%%%%%%%%%%%%%%%%%%%%%%%%%%%%%%%
\subsection The scalar bound state

In this section, we study the limit of stability in the disordered phase
($\sigma=0$). This is a problem which only arises when $n$ is odd, the first case being provided by the tricritical point.\sslbl\sboundst\par
The mass-matrix has then a zero eigenvalue, which corresponds to the appearance
of a new massless excitation other than $\phib$. Let us denote by $\bf M$ the
$\rho,m^2$ $2\times2$ submatrix. Then,
$$\det{\bf M}=0\ \Leftrightarrow\ 2U''(\rho)B_\Lambda(0,m)+1=0\,.$$
In the two-space the corresponding eigenvector has components
$(\half,U''(\rho))$.
%
\medskip
{\it The small mass region.}
In the small $m$ limit, the equation can be rewritten in terms of the
constant $K(d)$ defined in \etadepolii\ as
$$ (d-2)K(d)m^{d-4}U''(\rho)+1=0\,.\eqnd\ephitwoa $$
Eq.~\ephitwoa\ tells us that $U''(\rho)$ must be small. We are thus
close to a multicritical point. Using the result of the stability analysis,
we obtain
$$(-1)^{n-1}{d-2\over (n-2)!}K^{n-1}(d)m^{n(d-2)-d}U^{(n)}(\rho_c)=1\,.
\eqnd\estabil$$
We immediately notice that this equation has solutions only for $n(d-2)=d$,
that is at the critical dimension. The compatibility then fixes the value of
$U^{(n)}(\rho_c)$. We  find again the point \eendtric, $U^{(n)}(\rho_c)=\Vc$.
If we take into account the leading correction to the small $m$ behaviour
we  find,  instead,
$$U^{(n)}(\rho_c)\Vc^{-1}-1\sim(2n-3){a(d)\over K(d)}\left({m\over\Lambda}
\right)^{4-d}.\eqnd\estabil  $$
This means that when $a(d)>0$, there exists a small region
$U^{(n)}(\rho_c)>\Vc$ where the vector
field is massive with a small mass $m$ and the bound-state massless. The value $\Vc$ is a fixed point value. %% the other case???
\medskip
{\it The scalar field at small mass.} We now extend the analysis to a
situation where the scalar field has a small but non-vanishing mass $M$ and
$m$ is still small. The goal is in particular to explore the neighbourhood of
the special point \eendtric.
Then, the vanishing of the determinant of $\bf M$ implies
$$1+2U''(\rho)B_\Lambda(iM,m)=0\,.\eqnd\emassless$$
Because  $M$ and $m$ are small, this equation still implies
that $\rho$ is close to a point $\rho_c$ where $U''(\rho)$ vanishes.
Since reality imposes $M<2m$, it is easy to verify that this equation has also solutions for only the critical dimension. Then,
$$U^{(n)}(\rho_c)f(m/M)=\Vc\,,\eqnd\eOmegaC$$
where we have set
$$f(z)=\int_0^1\d x\left[1+(x^2-1)/(4z^2)\right]^{d/2-2},\qquad \half<z\,.
\eqnd\efofz$$
In   dimension three, it reduces to
$$f(z)=z\ln\left({2z+1\over 2z-1}\right).$$
$f(z)$ is a decreasing function which diverges for $z=\half$ because $d\le3$.
Thus, we find solutions in the whole region $0<U^{(n)}(\rho_c)<\Omega_c$,
that is, when the multicritical point is locally stable. \par
Evaluating the propagator near the pole, we find the matrix
$${\bf \Delta}={2\over G^2}\left[N\left.{\d B_\Lambda(p,m)\over \d
p^2}\right \vert_{p^2=-M^2}\right]^{-1}{1\over p^2+M^2}\pmatrix{1 &G
\cr G  &  G^2 \cr},\eqnd\edeltapole$$
where we have set
$$ G={2(-K)^{n-2} {\cal E}^{(n)} \over N (n-2)! } m^{4-d}\,. $$ %%% check {\cal E}?
For $m/M$ fixed, the residue goes to zero with $m$ as
$m^{d-2}$ because the derivative of $B$ is of the order of $m^{d-6}$.
Thus the bound state decouples on the multicritical line.
%
%\medskip
%{\it The scalar massless excitation: general situation.}
%Up to now we have
%explored only the case where both the scalar field and the vector field
%propagate. Let us now relax the latter condition, and examine what %happens
%when $m$ is no longer small. The condition $M=0$ then reads
%$$2U''(\rho_s)B_\Lambda(0,m)+1=0   $$
%together with
%$$m^2=2U'(\rho_s),\qquad \rho_s= \Omega _d(m).\eqnd\GapEq $$
%An obvious remark is: there exist solutions only for $U''(\rho_s)<0$, and
%therefore the ordinary critical line can never be approached. In terms
%of the function $ \omega_d (z)$ (Eq.~\etadepole)
%the equations can be rewritten as
%$$\rho_s=\Lambda^{d-2}  \omega (\sqrt{z}),\quad z=2U'(\rho_s)\Lambda^{-%2},\quad
% \Lambda^{d-4} U''(\rho_s) \omega '(\sqrt{z})=\sqrt{z} \,.$$
%The function $ \omega (z)$ in Pauli--Villars's regularization is a decreasing
%function.
%In the same way $-\omega '(z)$ is a positive decreasing function. \par
%The third equation is the condition for the two curves corresponding to %the
%two first ones become tangent. For any value of $z$, we can find %potentials and thus solutions. Let us call $z_s$ such a value and specialize %to cubic potentials. Then,
%$$\rho_s=\Lambda^{d-2} \omega (\sqrt{z_s})\ ,$$
%$$\quad U(\rho)=U'(\rho_s)(\rho-\rho_s)+\half
%U''(\rho_s)(\rho-\rho_s)^2+
%\frac{1}{3!}U^{(3)}(\rho_s)(\rho-\rho_s)^3,\eqnd\Vofrho $$
%which yields a two parameter family of solutions. For $z$ small, we see %that
%for $d<4$ the potential becomes proportional to $(\rho-\rho_c)^3$.

\subsection Stability and double scaling limit

In order to discuss more thoroughly the stability issue and the double scaling
limit, we
now construct the effective action for the scalar bound state. We consider
first only the massless case. We  need the action only in the IR limit, and in
this limit we can integrate out the vector field and the second massive eigenmode.\sslbl\sdouble
\medskip
{\it Integration over the massive modes.}
As we have already explained in section \stwodim, we can integrate over one of
the fields, the second being fixed, and we need  the result only at leading
order. Therefore, we replace in the functional integral
$$\e^{\cal Z}=\int [\d\rho\d\lambda]\exp\left[-\frac{1}{2}N\tr \ln(-\nabla ^2 + \lambda)
+N\int \d^dx\left(- U(\rho) +\half \rho \lambda\right)\right] \eqnd\eZ$$
one of the fields by the solution of the field equation. It is useful to discuss the effective potential of the
massless mode first. This requires calculating the action only for constant fields.
It is then simpler to eliminate $\lambda$. We
assume in this section that $m$ is small (the vector propagates).
For $\lambda\ll\Lambda$, the $\lambda$-equation reads ($d<4$)
$$\rho-\rho_c=-K(d)\lambda^{(d-2)/2}.  \eqnd\erholambda $$
It follows that the resulting action density $ {\cal E}(\rho)$, obtained from
Eq.~\eWeffrho, is
$${1\over N}{\cal E}(\rho)=U(\rho)+{d-2\over 2d (K(d))^{2/(d-2)}}(\rho_c-\rho)^{d/(d-2)}.
\eqnd\effUrho$$
In the sense of the double scaling limit, the criticality conditions are
$$ {\cal E}(\rho)=O\bigl((\rho-\rho_s)^n\bigr).$$
It follows
$$U^{(k)}(\rho_s)=- { K^{1-k}(d)\over2}{\Gamma\bigl(k-d/(d-2)\bigr)\over
\Gamma\bigl(-2/(d-2)\bigr)}m^{d-k(d-2)}\,,\quad 1\le k\le n-1\, .$$
For the potential $U$ of minimal degree, we find
$${1\over N}{\cal E}(\rho)\sim { K^{1-n}(d)\over 2 n!}
{\Gamma\bigl(n-d/(d-2)\bigr)\over
\Gamma\bigl(-2/(d-2)\bigr)}m^{d-n(d-2)}(\rho-\rho_s)^n .$$
\medskip
{\it The double scaling limit.}
We recall here that quite generally one verifies that a non-trivial double
scaling limit may exist only if the resulting field theory of the massless
mode is
super-renormalizable, that is below its upper-critical dimension $d=2n/(n-2)$,
because perturbation theory has to be IR divergent. Equivalently, to
eliminate $N$ from the critical theory, one has to rescale
$$\rho-\rho_s\propto N^{-2\theta}\varphi\,,\quad x\mapsto x N^{(n-2)\theta}
\quad {\rm with}\quad 1/\theta=2n-d(n-2),$$
where $\theta$ has to be positive.
\par
We now specialize to dimension three, since $d<3$ has already been
examined, and the expressions above are valid only for $d<4$.
The normal critical point ($n=3$), which leads to a $\varphi^3$ field theory,
and can be obtained for a quadratic potential $U(\rho)$  (the $(\phib^2)^2$ field theory)
has been discussed in section \ssfivNi. We thus concentrate on the next critical
point $n=4$ where the minimal potential has degree 3.
\medskip %####
{\it The $d=3$ tricritical point.}
The action density  then becomes
$${1\over N}{\cal E}(\rho)= U(\rho)+ \frac{8\pi^2}{3}(\rho_c-\rho)^3. \eqnd\effWthreeD$$
If the potential $U(\rho)$ has degree larger than 3, we obtain
after a local expansion and a rescaling of fields,
$$\rho-\rho_s= -{ 1\over 32\pi^2 \rho_c}
(\lambda-m^2)\propto \varphi/N\,,\quad x\mapsto Nx\,,\eqnd\rescal$$
a simple super-renormalizable $\varphi^4(x)$ field theory.
If we insist,
on the other hand, that the initial theory should be renormalizable, then we remain with
only one candidate, the renormalizable $(\phib^2)^3$ field theory, also
relevant for the tricritical phase transition with $O(N)$ symmetry breaking.
Inspection of  $ {\cal E}(\rho)$ immediately shows a remarkable feature:
because the term added to $U(\rho)$ is itself a polynomial of degree 3,
the critical conditions lead to an action density $ {\cal E}(\varphi)$ that vanishes identically. This result reflects the property that
the two saddle point equations  ($\del {\cal S}/\del \rho = 0$, $
\del {\cal S}/\del \lambda = 0$ in Eqs.~\esaddNgen{}) are proportional
and thus have a continuous one-parameter family of solutions. This results in
a flat effective potential for $\varphi(x)$.
The effective action for $\varphi$ depends only on the derivatives
of $\varphi$, like in the $O(2)$ non-linear $\sigma$ model. \par
We conclude that no non-trivial double scaling limit can be obtained in this way.
In 3 dimensions with a $(\phib^2)^3$ interaction, we can generate at
most a normal critical point $n=3$, but then a simple $(\phib^2)^2$
field theory suffices.
\smallskip
{\it Remark.} The problem of the sign of $a(d)$ discussed in section \smulticr\ has an interesting appearance in $d= 3$ in the small $m^2$ region.
If one keeps the
extra term proportional to $a(d)$ in Eq.~\effUrho, one finds
$${1\over N}{\cal E}(\rho)=U(\rho)+{8\pi^2 \over 3}(\rho_c-\rho)^3
+{a(3)\over\Lambda} 4\pi^2 (\rho_c-\rho)^4.$$
Using now Eq.~\erholambda\ and, as mentioned in section \sboundst,
the fact that in the small $m^2$ region the potential is proportional to
$(\rho-\rho_c)^3$, we can solve for $m^2$.
Since $m^2>0$, the appearance of a phase with small mass depends
on the sign of $a(d)$. Clearly, this shows a
non-commutativity of the limits of $m^2/\Lambda^2 \to 0 $ and $ N \to \infty$.
The small  $m^2$ phase can be reached by a special tuning  and
cannot be reached with an improper sign of $a(d)$.
Calculated in this way, $m^2$ can be made proportional to the deviation  of
the  coefficient of $\rho^3$ in $U(\rho)$ from its critical  value $16\pi^2$.
%
\medskip
{\it A few concluding remarks.}
In this section, we studied of several subtleties in the phase structure of  $O(N)$ vector
models around multicritical points of odd and even orders.
One of the main topics is the understanding of the multicritical behaviour
of these models  at their critical dimensions and the effective field theory
of the $O(N)$-singlet bound state obtained in the $N \to \infty$,  $g \to g_c$
correlated limit. It was pointed out that the integration over massive $O(N)$ singlet modes is essential in order to extract the correct effective field theory of the small mass scalar excitation.
After performing this integration, it has been established here that
the double scaling limit of $(\phib^2)^K$ vector model
in its critical dimension $d=2K/(K-1)$
results in a theory of a free massless $O(N)$ singlet bound state.
This fact is a consequence of the existence of
flat directions at the scale invariant multicritical point in the effective action. In contrast to the case $d < 2K/(K-1)$
where IR singularities compensate powers of $1/N$ in the double scaling
limit,  at  $d=2K/(K-1)$ there is no such compensation and only a
non-interacting effective field theory  of the massless bound state is left.
%\sslbl\sconclud
\par
Another interesting issue in this study is the ambiguity of the
sign of $a(d)$.
The coefficient of $m^2\Lambda^{d-4}$ denoted by $a(d)$ in the expansion
of the gap equation in Eqs.~\esaddNgen{c} and \edmum\ seems to have a
surprisingly important role in the approach to the continuum limit ($\Lambda^2
\gg m^2$). The existence of an IR fixed point  at $g \sim O(N^{-1}),$ as seen
in the $\beta$ function for the unrenormalized coupling constant (section
\sssEGRN), depends on the sign of $a(d)$. Moreover, %as seen in section ???
the existence of a phase with a small mass $m$ for the $O(N)$ vector quanta
and a massless $O(N)$ scalar depends also on the sign of
$a(d)$. It may very well be that the importance of
the sign of $a(d)$ is a mere reflection of the limited  coupling constant
space used to described the model.
%
