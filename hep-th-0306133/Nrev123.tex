%%%%%%%%%%%%%%%  Nrev123 as of April 25 2003
\def\vv{v}
\def\tu{\tau}
\def\ro{\rho}
%%%%%% Unused references are at the end of the file
%%%%%%% Careful, one reference in text.
\nref\rBKStan{As shown by   H.E. Stanley, {\it Phys. Rev.} 176
(1968) 718, the large $N$-limit of the classical $N$-vector model
coincides with the spherical model solved in ref. \rBerlin. }
\nref\rBerlin{ T.H. Berlin and M. Kac, {\it Phys. Rev.} 86 (1952)
821.} \nref\rWWFis{The modern formulation of the RG ideas is due
to: \rf K.G. Wilson, {\it Phys. Rev.} B4 (1971) 3174, {\it
ibidem}\/ 3184; \rf
  and presented in an expanded form in \rWKogut. }
\nref\rWKogut{K.G. Wilson and J. Kogut, {\it Phys. Rep.} 12C
(1974) 75. }
 \nref\rWFisher{The idea of the $\varepsilon$-expansion
is due to \rf K.G. Wilson and M.E. Fisher, {\it Phys. Rev.  Lett.}
28 (1972) 240. } \nref\rAMSDSW{Early work on calculating critical
properties for large $N$ includes \rf R. Abe, {\it Prog. Theor.
Phys.} 48 (1972) 1414; 49 (1973) 113, 1074, 1877 and refs.~\refs{\rMa{--}\rWilson}.} \nref\rMa{S.K. Ma, {\it Phys. Rev.
Lett.} 29 (1972) 1311 ; {\it Phys. Rev.} A7 (1973) 2172 .}
\nref\rSuzuki{M. Suzuki, {\it Phys. Lett.} 42A (1972) 5; {\it
Prog. Theor. Phys.} 49 (1973) 424, 1106, 1440 .} \nref\rFerrel{
R.A. Ferrel and D.J. Scalapino, {\it Phys. Rev. Lett.} 29 (1972)
413.} \nref\rWilson{K.G. Wilson, {\it Phys. Rev.} D7 (1973) 2911.}
\nref \rFAAAH{The spin--spin correlation in zero field is obtained
in \rf M.E. Fisher and A. Aharony, {\it Phys. Rev. Lett.} 31
(1973) 1238 and refs.~\refs{\rAharony,\rAbe }.}\nref\rAharony{ A.
Aharony, {\it Phys. Rev.} B10 (1974) 2834.}\nref\rAbe{ R. Abe and
S. Hikami, {\it Prog. Theor. Phys.} 51 (1974) 1041.}

\nref\rBrWa{The contribution of order $1/N$ to the equation of
state is given in \rf E. Br\'ezin and D.J. Wallace, {\it Phys.
Rev.} B7 (1973) 1967.} \nref\rSKMa{The exponent $\omega$ has been
calculated to order $1/N$ by S.K. Ma, {\it Phys. Rev.} A10 (1974)
1818.} \nref\rMaDGvi{See also the review of S.K. Ma  in {\it Phase
Transitions and Critical Phenomena} vol. 6, C. Domb and M.S. Green
eds. (Academic Press, London 1976).}
%%%%%%%%%%%%%%%%%%%%%%%%%%%
\nref\rBLGZJDBvi{The study of the large $N$ limit by the steepest
descent method is explained in \rf
 E. Br\'ezin, J.C. Le Guillou and J. Zinn-Justin, contribution to {\it Phase
Transitions and Critical Phenomena} vol. 6, C. Domb and M.S. Green
eds. (Academic Press, London 1976). It is applied to more general
scalar field theories in \rHalpern .}\nref\rHalpern{
 M.B. Halpern, {\it Nucl. Phys.} B173 (1980) 504.}
\nref\rBarMos{For a Hartree--Fock variational approach to large
$N$ theories and large $N$  QFT at finite temperature see  \rf
W.A. Bardeen and M. Moshe, {\it Phys. Rev.} D28 (1983) 1372 and
{\it Phys. Rev.} D34 (1986) 1229.} \nref\rBrZJsig{The non-linear
$\sigma $-model is discussed in the spirit of this review in \rf
E. Br\'ezin and  J. Zinn-Justin, {\it Phys. Rev. Lett.} 36 (1976)
691; {\it Phys. Rev.} B14 (1976) 3110.} \nref\rnlslat{Lattice
calculations of the non-linear $\sigma $ model with the large $N$
expansion are reported in \rf M. Campostrini, P. Rossi, {\it
Phys. Lett.} B242 (1990) 81 and in \rBiscari .}\nref\rBiscari{P.
Biscari, M. Campostrini, P. Rossi, {\it Phys. Lett.} B242 (1990)
225.} \nref\rCPN{The $CP(N-1)$ model is discussed in two
dimensions with the large $N$ expansion in \rf M. L\"uscher, {\it
Phys. Lett.} 78B (1978) 465 and in refs.~\refs{\rDadda{--}\rCampos}.}\nref\rDadda{ A. D'Adda, P. Di
Vecchia and M. L\"uscher, {\it Nucl. Phys.} B146 (1978) 63; B152
(1979) 125. }\nref\rMunster{ G. M\"unster, {\it Nucl. Phys.} B218
(1983) 1.} \nref\rDiVecch{P. Di Vecchia, R. Musto, F. Nicodemi,
R. Pettorino and P. Rossi, {\it Nucl. Phys.} B235 (1984)
478.}\nref\rCampos{ M. Campostrini, P. Rossi,
    {\it Phys. Lett.} B272 (1991) 305;
    {\it Phys. Rev.} D45 (1992) 618; Erratum-{\it ibid.} D46 (1992) 2741.}
\nref\rBrCaPe{Some finite size calculations are reported in \rf E.
Br\'ezin, {\it J. Physique (Paris)} 43 (1982) 15 and in refs.~\refs{\rSingh {--}\rCarac}. } \nref\rSingh{ S. Singh and
R.K.~Pathria, {\it Phys. Rev.} B31 (1985) 4483.} \nref\rCompost{
M. Campostrini, P. Rossi, {\it Phys. Lett.} B255 (1991) 89.
}\nref\rCarac{S. Caracciolo and A. Pelissetto, {\it Phys. Rev.}
D58 (1998) 105007, hep-lat/9804001.} \nref\rparamet{The parametric
representation has been introduced in \rf P. Schofield, J.D.
Litster and J.T. Ho, {\it Phys. Rev. Lett.} 23 (1969) 1098  and in
\rJoseph. } \nref\rJoseph{B.D. Josephson, {\it J. Phys. C} 2
(1969) 1113.} \nref\rSymanza{K.~Symanzik, Carg\`ese lectures, DESY
preprint 73/58, Dec 1973; {\it Lett. Nuovo Cim.} 8 (1973) 771;
  {\it    Kyoto 1975, Proceedings, Lecture Notes In Physics}, Berlin 1975, 102-106.
  \rf
A similar analysis in the case of theories renormalizable beyond
perturbation theory (like the non-linear $\sigma $ or Thirring
models) is found in \rParisi .} \nref\rParisi{G. Parisi, {\it
Nucl. Phys.} B100 (1975) 368.}
 \nref\rTriv{On the triviality of $\phi^4$ theory in $d=4$ dimensions
see reference 8 in  K.G. Wilson, {\it Phys. Rev.}   D6 (1972) 419
and the review \rCalla .} \nref\rCalla{ D.J.E. Callaway, {\it
Phys. Rep.} 167 (1988) 241.} \nref\rUVrenorm{B. Lautrup, {\it
Phys. Lett.} B69 (1977) 109 and refs.~\refs{\rtHooft,\rPari}.
{}For more references see also the reprint volume \refs\rLeGuill
.} \nref\rtHooft{ G. 't Hooft, {\it Erice Lectures 1977}.}
\nref\rPari{ G. Parisi, {\it Phys. Lett.} 76B (1978) 65. }
\nref\rLeGuill { {\it Large Order Behaviour of Perturbation
Theory}, J.C. Le Guillou and J. Zinn-Justin eds. (North Holland,
Elsevier Science Pub., Amsterdam 1989).} \nref\rDavid{F. David,
{\it Nucl. Phys.} B209 (1982) 433.} \nref\rHiggsbound{The bound on
the Higgs mass is discussed in \rf R.F. Dashen, H. Neuberger, {\it
Phys. Rev. Lett.} 50 (1983) 1897 and in refs.~\refs{\rHasen,\rLusche}.} \nref\rHasen{ A. Hasenfratz, K. Jansen,
C.B. Lang, T. Neuhaus, H. Yoneyama, {\it Phys. Lett.} B199( 1987)
531. } \nref\rLusche{M. L\"uscher, P. Weisz, {\it Nucl. Phys.}
B290 (1987) 25, B295 (9188) 65, B318 (1989) 705.}
\nref\rSymanzb{K. Symanzik, DESY preprint 77/05, Jan. 1977.}
\nref\rFRG{Functional renormalization group has been discussed in
the large $N$ limit in \rf M. Reuter, N. Tetradis, C. Wetterich,
{\it Nucl. Phys.} B401 (1993) 567 and in refs.~\refs{\rAttan,\rBerges}.}\nref\rAttan{ M. D'Attanasio, T.R.
Morris, {\it Phys. Lett.} B409 (1997) 363, hep-th/9704094.}
\nref\rBerges{J. Berges, N. Tetradis, C. Wetterich, {\it Phys.
Rept.} 363 (2002) 223, hep-ph/0005122.}

\nref \rPeRoVi{Large-$N$ critical behaviour of $O(N)\times O(m)$
spin models is considered in \rf
 A. Pelissetto, P. Rossi and E.
Vicari, {\it Nucl. Phys.}   B607 (2001) 605 and in
\rGracey.}\nref\rGracey{J.A. Gracey, {\it Nucl. Phys.} B644 (2002)
433, hep-th/0209053.}

\nref\rJaStr {see e.g.~R. Jackiw and A. Strominger, {\it Phys.
Lett.} 99B  (1981)133.} \nref\rNsigamiiD{The non-linear $\sigma $
model two-point functions and mass gap in dimension 2 have been
studied by large $N$ techniques in \rf V. F. M\"uller, T Raddatz
and W. R\"uhl, {\it Nucl. Phys.} B251 (1985) 212, {\it Erratum
ibid.}\/ B259 (1985) 745. } \nref\rANPVN{The consistency of the
$1/N$ expansion to all orders has been proven in \rf I. Ya
Aref'eva, E.R. Nissimov and S.J. Pacheva, {\it Commun. Math.
Phys.} 71 (1980) 213, see also ref. \rVasilev. The method presented
here is taken from ref.~\rbook.}\nref\rVasilev{ A.N. Vasil'ev and
M.Yu. Nalimov, {\it Teor. Mat. Fiz.} 55 (1983) 163.}

\nref\rKTOOVPH{At present the longest $1/N$ series for exponents
and amplitudes are found in \rf I. Kondor and T. Temesvari, {\it
J. Physique Lett. (Paris)} 39 (1978) L99 and refs.~\refs{\rOkabe{--}\rVasilevB}. } \nref\rOkabe{Y. Okabe and M. Oku,
{\it Prog. Theor. Phys.} 60 (1978) 1277, 1287; 61 (1979)
443.}\nref\rVasilevA{ A.N. Vasil'ev, Yu.M. Pis'mak and Yu.R.
Honkonen, {\it Teor. Mat. Fiz.} 46 (1981) 157 {\it ibid.} 47
(1981) 465; {\it ibid.} 50 (1982) 195.}\nref\rVasilevB{
A.~N.~Vasilev, M.~Y.~Nalimov and Y.~R.~Khonkonen, {\it Theor.\
Math.\ Phys.} 58 (1984) 111.} \nref \rKTH{See also I. Kondor, T.
Temesvari and L. Herenyi, {\it Phys. Rev.} B22 (1980) 1451.}
\nref\rPeVi{Results concerning the $\beta$-function at order $1/N$
in the massive theory renormalized at zero momentum have been
reported in \rf A. Pelissetto and E. Vicari, {\it Nucl. Phys.}
B519 (1998) 626, cond-mat/9711078.} \nref\rDerMa{In particular, a
calculation of the dimensions of composite operators are reported
and the consequences for the stability of the fixed point of the
non-linear $\sigma$ model discussed in \rf S. E. Derkachov, A. N.
Manashov, {\it Nucl. Phys.} B522 (1998) 301,  hep-th/9710015; {\it
Phys. Rev. Lett.} 79 (1997) 1423, hep-th/9705020.}
\nref\rCamros{See also M. Campostrini, P. Rossi, {\it  Phys.
Lett.} B242 (1990) 81.} \nref\rGraci{The crossover exponent in
$O(N)$ $\phi^4$ theory at $O(1/N^2)$ is given in \rf J.A. Gracey,
{\it Phys. Rev.} E66 (2002) 027102, cond-mat/0206098.}
\nref\rBrHi{S.~Hikami ,  E.~Br\'ezin, {\it J.~Phys.} A 12, 759-770
(1979) [reprinted in {\it Currents Physics - Sources and
Comments,} CPSC 7,  Le Guillou J.C., Zinn-Justin J. eds.
(North-Holland, 1990)].} \nref\rVegAv{H.J. de Vega, {\it Phys.
Lett.} 98B (1981) 280; J. Avan and H.J. de Vega, {\it Phys. Rev.}
D29 (1984) 2891; {\it ibidem} 2904 [all reprinted in {\it Currents
Physics - Sources and Comments,} CPSC 7,  Le Guillou J.C.,
Zinn-Justin J. eds. (North-Holland, 1990)].} \nref\rBZJFSS{E.
Br\'ezin, J. Zinn-Justin, {\it Nucl. Phys.} B257 (1985) 867.}
%%%%%%%%%%%%%%%%%%%%%%%%%%%%%%%%%%%%%%%%%%%%%
%%%%%% Additional references  not cited yet
\nref\rLaRue{Renormalization of operators is discussed in \rf K.
Lang and W. R\"uhl, {\it Nucl. Phys.} B400 (1993) 597; {\it Z.
Phys.} C61 (1994) 459.} \nref\rMalong{ The case of long range
forces has been discussed in \rf S.K. Ma, {\it Phys. Rev.} A7
(1973) 2172.}
%%%%%%%%%%%%%%%%%%%%%%%%%%%%%%%%%%%%%%%%%%%%%%
\section{Scalar field theory for  $N$ large: general formalism and applications}

In this section we present a general formalism that allows
studying $O(N)$ symmetric scalar field theories in the large $N$
limit and, more generally, order by order in a large $ N
$-expansion. \par Of particular interest is the $(\phib^2)^2$
statistical field theory that describes the universal properties
of a number of phase transitions. The  study of phase transitions
and critical phenomena in statistical physics has actually been
one of the early applications of large $N$ techniques. It was
realized that an $N$-component spin-model (the spherical model)
could be solved exactly in the large $N$ limit
\refs{\rBKStan,\rBerlin}, and the solutions revealed scaling laws
and non-trivial (i.e.~non-gaussian or mean-field like) critical
behaviour . Later, following Wilson (and Wilson--Fisher)
\refs{\rWWFis{--}\rWFisher} it was discovered that universal
properties of critical systems  could be derived  from the
$(\phib^2)^2$ field theory within the framework of the formal $
\varepsilon =4-d$ expansion, by a combination of perturbation
theory and renormalization group (RG). The peculiarity of this
scheme, whose reliability in the physical dimension $d=3$, and
thus $\varepsilon=1$, could only be guessed, demanded some
independent confirmation. This was provided in particular by
developing a scheme to solve the $N$-component $(\phib^2)^2$ field
theory in the form of an $1/N$ expansion, whose leading order
yields the results of the spherical spin-model
\refs{\rAMSDSW{--}\rMaDGvi}. \sslbl\scfivN
\par
Here, we first solve more general $O(N)$ symmetric field theories
in the large $N$ limit, reducing the problem to a steepest descent
calculation \refs{\rBLGZJDBvi,\rHalpern}. The $( \phib^2)^2$ field
theory is then discussed more thoroughly from the point of view of
phase transitions and critical phenomena. We also stress the
relation between the large $N$ limit and  variational principles
\refs{\rBarMos}. Moreover, some other issues relevant to particle
physics like triviality, renormalons or Higgs mass are examined in
the large $N$ limit. \par A remarkable implication of the large
$N$ analysis is that two different field theories, the
$(\phib^2)^2$ theory and the non-linear $\sigma $ model, describe
the same critical phenomena, a results that holds to all orders in
the $1/N$ expansion \refs{\rBrZJsig{--}\rBiscari}. This result has
several generalizations, leading for instance to a relation
between the $CP(N-1)$  \refs{\rCPN{--}\rCampos} and the abelian
Higgs models.
\par Finally, large $N$ techniques are well adapted to the
analysis of finite size effects in critical systems \rBrCaPe, a
question we investigate in the more convenient formalism of the
non-linear $\sigma $ model in section \label{\ssNFSS}.
%%%%%%%%%%%%%%%%%%%%%

\subsection Scalar field theory: the large $N$ formalism

We consider an $O(N)$ symmetric euclidean  action (or classical hamiltonian) for an $N$-component scalar field $\phib$: \sslbl\ssNbosgen
$$ {\cal S} ( \phib )= \int \left[\ud \left[
\partial_{\mu} \phib (x) \right]^2+NU\bigl(\phib^2(x)/N\bigr)  \right] \d^{d}x\,,
\eqnd\eactONgen $$
where $U(\rho )$ is a general polynomial, and the explicit $N$ dependence
has been chosen to lead to a large $N$ limit.
The corresponding partition function is  given by a functional integral:
$$ {\cal Z}= \int \left[ \d \phib (x) \right] \exp \left[-{\cal
S}(\phib)\right] . \eqnd\eONpart $$
To render the perturbative expansion finite, a cut-off $ \Lambda $ consistent with the symmetry  is implied. \par
The solution of the model in the large $N$ limit is based on an idea of
mean field  type: it can be expected that, for $N$ large,
$O(N)$ invariant quantities like
$$\phib^2(x)=\sum_{i=1}^N \phi_i^2(x) $$
self-average and therefore have small
fluctuations (as for the central limit theorem this relies on the assumption that the components
$\phi_i$ are somehow  weakly correlated). Thus, for example,
$$\left< \phib^2(x)\phib^2(y)\right>
\mathop{\sim}_{N
\rightarrow \infty}\left< \phib^2(x)\right>  \left<
\phib^2(y)\right> .$$
This observation suggests to take $\phib^2(x)$ as a dynamical variable, rather than $\phib(x)$ itself.
For this purpose, we introduce two additional fields $\lambda$ and $\rho$ and impose the
constraint $\rho(x)=\phib^2(x)/N$ by an integral over $\lambda$. For each point
of space $x$, we use the identity
$$1=N\int\d\rho\,\delta(\phib^2-N\rho)=
{N\over 4i\pi}\int\d\rho\d\lambda\e^{\lambda(\phib^2-N\rho)/2} ,\eqnd\eNbasicid $$
where the $\lambda$ integration contour runs parallel to the imaginary axis.
The insertion of the identity into the integral \eONpart\ yields a new representation of the partition function:
$$ {\cal Z}= \int [ \d \phib][\d\rho][\d\lambda] \exp \left[-{\cal
S}(\phib,\rho,\lambda)\right]   \eqnd\eONpartgen $$
with
$${\cal S}(\phib,\rho,\lambda)= \int \left[\ud \left[
\partial_{\mu} \phib (x) \right]^2+NU\bigl(\rho(x)\bigr) +\ud\lambda(x)
\bigl(\phib^2(x)-N\rho(x)\bigr) \right]
\d^{d}x\,. \eqnd\eactONgenii $$
The  functional integral \eONpartgen\ is then gaussian in $\phib $, the integral
over the field $ \phib $ can be performed and the dependence on $N$ of
the partition function becomes explicit. Actually, it is convenient
to separate the components of $ \phib $ into one component $\sigma
$, and $ N-1 $ components $ \pib $, and integrate over $ \pib $ only
(for $T<T_c$ it may even be convenient to integrate over only $N-2$
components). This does not affect the large $N$ limit but only the $1/N$ corrections. To generate
$\sigma$-correlation functions, we  add also a source $ H(x)$ (a space-dependent magnetic field in the ferromagnetic language) to the action. The partition function then becomes
$$ {\cal Z} (H)= \int[\d\sigma][\d\rho] [\d \lambda] \exp \left[ -{\cal S}_N
(\sigma,\rho,\lambda)+ \int \d^{d}x\, H (x)\sigma (x) \right]
\eqnd{\eZeff} $$
with
$$ \vbox{\eqalignno{ {\cal S}_N (\sigma,\rho,\lambda)= &
\int \left[\ud \left(\partial_{\mu}\sigma \right)^2
+NU(\rho)+ \ud\lambda (x) (\sigma^2 (x)-N\rho(x)) \right] \d^{d}x & \cr
&\quad +\ud (N-1)\tr\ln \left[ - \nabla^2  +\lambda   \right] . &
\eqnd\eactONef \cr}} $$
\smallskip
{\it $\phib^2$-field correlation functions.} In this formalism it is
natural to consider also correlation functions involving the
$\rho$-field which by construction is proportional to the $\phib^2$ composite field. In the framework of phase transitions, near the critical temperature, the $\phib^2$ field plays the role of  the energy operator.
%
\smallskip
{\it Remark.} One can wonder how much one can still generalize this formalism
(with only one vector field). Actually, one can solve also the most general
$O(N)$ symmetric field theory with two derivatives. Indeed, this involves adding
the two terms
$$Z(\phib^2/N)(\partial _\mu\phib)^2,\quad V(\phib^2/N)(\partial _\mu \phib\cdot \phib)^2/N , $$
where $Z$ and $V$ are two arbitrary functions. These terms can be rewritten
$$Z(\rho)(\partial _\mu\phib)^2, \quad NV(\rho)(\partial _\mu \rho )^2, $$
in such a way that the $\phib$ integral remains gaussian and can be performed.
This reduces again the study of the large $N$ limit to the steepest descent method.
%
\subsection Large $ N $ limit: saddle points and phase transitions

We now study the large $ N $ limit, the function $U(\rho)$ being
considered as $N$ independent. If we take $\sigma=O(N^{1/2})$,
$\rho=O(1)$, $\lambda=O(1)$ all terms in ${\cal S}_N$ are of order
$N$ and the functional integral can be calculated for $N$ large by the steepest descent method  \rBLGZJDBvi. \sslbl\ssNUfige
\smallskip
{\it Saddle points.}  We look for a uniform saddle point ($\sigma(x),\rho(x),\lambda(x)$ are space-independent because we look for the ground state, thus excluding instantons or solitons)
$$ \sigma (x)=\sigma\, ,\quad\rho(x)=\rho \quad{\rm and}\quad \lambda (x)=m^2 \eqnd\eNsaddpts $$
because the $\lambda $ saddle point value must be positive. \par
The action density $\cal E$ in zero field $H$ then becomes
$${\cal E}=NU(\rho)+\ud m^2 (\sigma ^2-N\rho)+{ N\over2}\int{\d^d k
\over(2\pi)^d}\,\ln[(k^2+m^2 )/k^2].\eqnd\eNEner $$
Differentiating then ${\cal E}$ with respect to $\sigma$, $\rho$ and
$m^2$, we obtain the saddle point equations
\eqna\esaddleN
$$ \eqalignno{ m^2\sigma & = 0\,, & \esaddleN{a} \cr
\ud m^2&=U'(\rho) , &\esaddleN{b} \cr
\sigma^2/N-\rho+{1 \over  (2\pi)^{d}} \int^\Lambda { \d^{d}k \over k^2+m^2} & =
0\,. &
\esaddleN{c}  \cr} $$
\smallskip
{\it Regularization and large cut-off expansion.}
In the last equation we have now introduced a cut-off $\Lambda$ explicitly. This means,
more precisely, that we have replaced the propagator by some regularized form
$${1\over k^2+m^2}\ \mapsto  \tilde\Delta _\Lambda (k)= {1\over k^2
D(k^2/\Lambda^2)+m^2}\quad {\rm with}\ D(z)=1+O(z ), \eqnd\epropreg $$
where the function $D(z)$ is a function strictly positive for $z> 0$, analytic in
the neighbourhood of the real positive semi-axis, and increasing faster than
$z^{(d-2)/2}$ for $z\to+\infty $. \par
We  set
$${1\over (2\pi)^d }\int^\Lambda {\d^d k \over k^2
+m^2}\equiv {1\over (2\pi)^d }\int \d^d k\, \tilde\Delta _\Lambda (k) \equiv   \Omega_d(m)\equiv \Lambda^{d-2} \omega_d  (m/\Lambda). \eqnd\etadepole $$
Below we need the first terms of the expansion of $\Omega_d(m)$ for
$m^2\to 0$. One finds for $z\to 0$ and $d>2$ an expansion which we
parametrize as  (for details see appendix \label{\apiloop})% right parametrization ? check signs ?
$$\omega_d(z)= \omega _d(0)-K(d)z^{d-2}+a(d)z^2 +O\left(  z^4, z^d\right).
\eqnd\etadepolii $$
The constant $K(d)$ is universal, that is independent of the cut-off procedure:
\eqna\etadpolexp
$$ \eqalignno{ K(d)&=-{1\over
(4\pi)^{d/2}}\Gamma(1-d/2)=- { \pi \over2\sin(\pi d/2)}N_d \, , & \etadpolexp{a} \cr
N_d &={2 \over(4\pi)^{d/2} \Gamma(d/2) }\,,  & \etadpolexp{b}
\cr }
$$
where we have introduced for later purpose the usual loop factor
$N_d$.  \par
The constant $a(d)$, in contrast,
%, which characterizes the leading correction in Eq.~\etadepolii,
depends explicitly on the regularization, that is on the way large momenta are cut,
$$a(d)=N_d \times\cases{ \displaystyle \int_0^\infty k^{d-5}\d k\left(1-{1\over D^2(k^2)} \right)  & for $d<4$,
\cr \displaystyle - \int_0^\infty {k^{d-5} \d k \over D^2(k^2)}  & for $d>4$, \cr}
 \eqnd\eadef $$
but  for
$\varepsilon=4-d\to 0$ satisfies
$$ a(d)\mathop{\sim}_{\varepsilon=4-d\to 0} 1/ (8\pi^2\varepsilon). \eqnd \eaespzero $$
Integrating $\Omega_d(m)$ over $m^2$, we then obtain a finite
expression for the regularized integral arising from the $\phib$
integration and given in \eNEner:
%$$\eqalignno{ \tr\ln\left[ - \nabla^2  + m^2 \right]_\Lambda - \tr\ln( - %\nabla^2)_\Lambda &=\int_0^{m^2}\d s\, \Omega _d(s)
%\cr&\quad  ={1\over (4\pi)^{d/2}}\int
%^{\infty}_{0}\d t\,\varrho  (t\Lambda^2 ){1-\e^{-tm^2} \over t^{1+d/2}} %\,.\hskip8mm&\eqnd\eNtrlnreg \cr} $$
$$  {1\over(2\pi)^d}\int^\Lambda \d^d k\,\ln[(k^2+m^2 )/k^2] =\int_0^{m  }2s\d s\, \Omega _d(s) .
 \eqnd\eNtrlnreg  $$
>From the expansion \etadepolii,  we infer
$$\eqalignno{
\int_0^{m  }2s\d s\, \Omega _d(s) &=-2{K(d)\over d}m^d +\Omega _d(0)m^2+{a(d)\over2}m^4\Lambda^{d-4}\cr&\quad +O(m^{6}\Lambda^{d-6},m^{d+2}\Lambda^{ -2} ).& \eqnd\eNtrlnexp
\cr} $$
Finally, for $d=4$  these expressions have to be modified because a logarithmic contribution appears:
$$ \omega _d(z)- \omega _d(0)\sim {1\over8\pi^2}z^2\ln z \,. \eqnd \etadepoliv $$
\medskip
{\it Phase transitions.}
Eq.~\esaddleN{a} implies either $\sigma=0 $ or $m=0$. We see from the $\tr\ln$ term in expression \eactONef~that
$m$, at this order,  is also the  mass of the  $\pib$ field.
When $\sigma \ne 0$,   the $O(N)$ symmetry is spontaneously broken, $m$ vanishes and the massless $\pib$-field corresponds to the expected $N-1$ Goldstone modes.  If, instead, $\sigma=0$  the $O(N)$  symmetry is
unbroken and the $N$ $\phib$-field components have the same mass $m$. \par
We then note from Eq.~\esaddleN{c} that
the solution $m=0$ can exist only for $d>2$, because at $d=2$ the
integral is IR divergent. This result is consistent with the Mermin--Wagner--Coleman
theorem: in a system with only short range forces a continuous symmetry cannot
be broken for $ d\leq 2$, in the sense that the expectation value $\sigma$ of the order
parameter must necessarily vanish. The potential Goldstone modes are
responsible for this property: being massless, as we expect from
general arguments and verify here, they induce an IR instability for $d\le 2$.
Therefore, we discuss below only the dimensions $d>2$; the dimension
$d=2$ will be examined separately in the more appropriate formalism
of the non-linear $\sigma $ model in section \label{\ssLTsN}.\par
Moreover, we assume now that the polynomial $U(\rho )$ has for $\rho \ge 0$
a unique minimum at a strictly positive value of $\rho $ where
$U''(\rho)$ does not vanish,
otherwise the critical point would turn out to be a multicritical
point,  a situation that  will be studied in section  \label{\ssdblescal}.
 \smallskip
(i) {\it Broken phase.}
When $m=0$, the saddle point Eqs.~\esaddleN{} reduce to
$$U'(\rho)=0 \,,\quad \sigma ^2/N-\rho+{1 \over  (2\pi)^{d}} \int^\Lambda { \d^{d}k \over
k^2} =0\,.$$
The first equation implies that $\rho $ is given by the minimum of $U$ and the second equation then determines the
field expectation value. Clearly a solution can be found only if
$$\rho> \rho_c={1 \over  (2\pi)^{d}} \int^\Lambda { \d^{d}k \over k^2}
=\Omega_d(0),\eqnd\eONrhoc  $$
where the definition \etadepole~has been used, and then
$$\sigma =\sqrt{N(\rho-\rho_c)}. \eqnd\esponmag $$
  \smallskip
(ii) {\it The symmetric phase.} When $\sigma =0$, the saddle point equations \esaddleN{} can be written as
$$ \rho- \rho_c=  \Omega_d(m)-\Omega_d(0) ,\quad m^2=2 U'(\rho)\,. \eqnd\eNsaddsym $$
The first equation \eNsaddsym\ implies $\rho \le \rho _c$. At the value $\rho =\rho _c$ a  transition
takes place between an ordered phase $\rho >\rho _c$ and a symmetric phase
$\rho \le \rho _c$. The condition
$$U'(\rho _c)=0 \eqnd\eNUcrit $$
determines critical potentials. \par

In expression \eactONef\ we see that the
$\sigma$-propagator then becomes  \rFAAAH
$$\Delta_\sigma\mathop{\sim}_{|p|,m\ll \Lambda }{1\over p^2+m^2}\,.\eqnd\eDeltasigN$$
Therefore, $m $ is at this order  the physical mass or  the
inverse of the correlation length $\xi $ of the field $\sigma $ (and thus of all components of the $\phib$-field).
\par
The condition
$m\ll \Lambda $, or equivalently $\xi\gg 1/\Lambda $  defines the critical domain. The first equation \eNsaddsym\ then implies that $\rho-\rho _c $  is small in the critical domain. From the second equation \eNsaddsym\ follows that $U'(\rho )$ is small and thus $\rho $ is close  to the minimum of $U(\rho )$. We can then expand $U(\rho )$ around $\rho _c$:
$$U(\rho)=U'(\rho _c) (\rho -\rho _c)+ \ud U''_c(\rho -\rho _c)^2
+O\bigl( (\rho -\rho _c)^3\bigr), \eqnd\eNUparam $$
and it is convenient to set
$$U'(\rho _c)=\ud \tau \,,\quad |\tau |\ll  \Lambda ^2\,.$$
With this parametrization
$$m^2= 2U''_c(\rho -\rho _c)+ \tau  +O\bigl( (\rho -\rho _c)^2\bigr).$$
With our assumptions  $U''_c $  is strictly positive (the sign ensures that the extremum is a minimum). Then $\tau $ is positive in the symmetric phase, while $\tau <0$ corresponds to the broken phase. \par
At this point we realize that, in the case of a generic critical point, $U(\rho)$ can be approximated by a quadratic polynomial. The problem then reduces to a discussion of the $ (\phib^2 )^2$ field theory
that indeed, in the framework of the $\varepsilon=4-d$ expansion, describes critical phenomena. Therefore, we postpone a more detailed analysis of
the solutions of the saddle point equations and first summarize a few
properties of the $\phi^4$ field theory from the point of view of perturbative RG.
%%%%%%%%%%%%%%%%%%%%%%%%%%%%%%%%%%%%%%%%%%
\subsection $ (\phib^2 )^2$ field theory, renormalization group, universality and large $N$ limit

>From now on, the discussion in this section will be specific to the
$(\phib^2)^2$ field theory. In terms of the initial $N$-component scalar field $\phib$,  we write
the action  as \sslbl\ssfivNi
$$ {\cal S} ( \phib )= \int \left\{{ 1 \over 2} \left[
\partial_{\mu} \phib (x) \right]^2+{1 \over 2}r
\phib^2 (x)+{1\over 4!} {u \over  N} \left[ \phib^2(x) \right]^2 \right\} \d ^{d}x\,.
\eqnd\eactON $$
>From the point of view of classical statistical physics, this model has the interpretation of an effective field theory that encodes large distance properties
of various statistical models near a second order phase transition. In this framework $r$ is a regular function of the temperature $T$ near the critical temperature $T_c$.  To the critical temperature corresponds a value $r_c$ of the parameter $r$ at which the correlation length $\xi$ (the inverse of the
physical mass in field theory language) diverges. For $r$ close to its critical value $r_c$, $\xi(r)\Lambda\gg 1$ and  a
continuum limit can be defined.
\par
We denote by $ \Gamma^{(\ell,n)}$ the vertex or 1PI functions of $\ell$
$\phib^2$ and $n$ $\phib$ fields (the coefficients of the expansion of the
thermodynamic potential) in Fourier representation. We set
$$  r=r_c+ \tu  \eqnd\eTmTc $$
and  $\tu$ thus characterizes the deviation from the critical value $r_c$.
In the symmetric phase ($\tau \ge0$) in zero field, the $\phib$ and $\phib^2$ correlation functions then satisfy, as functions of $\Lambda$, the dimensionless
coupling constant $g=u\Lambda^{4-d}/N$ and the deviation $\tau \ll \Lambda^2$, RG equations:
$$ \left[ \Lambda{ \partial \over \partial
\Lambda} +\beta(g){\partial \over \partial g}-{n \over 2}\eta(g)-\left(\tau{\partial \over\partial  \tau}+\ell\right)\eta_2
(g) \right] \Gamma^{(\ell,n)} =\delta_{n0}\delta_{\ell2} \Lambda^{d-4}B(g),
\eqnd\eRGgamln $$
where $\beta(g)$ is associated with the flow of the coupling constant $g$,
$\eta(g)$ and $\eta_2(g)$ to the anomalous dimensions of the fields $\phib$
and $\phib^2$, and $B(g)$ is associated to the additive renormalization of the $\phib^2$
two-point function. \par
The RG $\beta$-function in dimension $d=4-\varepsilon$,
$$\beta(g,\varepsilon)=-\varepsilon g+{N+8 \over 48\pi^2}g^2+O(g^3), $$
has for $d<4$ a non-trivial   zero
$$ g^*={48\pi^2\varepsilon\over N+8}+O(\varepsilon^2) , $$
which IR attractive since
$$\beta'(g^*)\equiv \omega =\varepsilon+O(\varepsilon^2)>
0\,.\eqnd\edefomega $$
This property is the starting point of the determination of the universal critical properties of the
model within the framework of  the so-called $\varepsilon$-expansion. \par
We explain here, instead, how
the model can be solved  in the large $N$ limit. In this way we will be
able to verify at fixed dimension, in some limit, many results obtained by perturbative methods.
\medskip
{\it Large $N$ limit.}
To the action \eactON~ corresponds the function
$$U(\rho)={1\over2} r\rho+{ u\over4!}\rho^2.\eqnd\eUONii $$
The large $N$ limit is taken at $U(\rho)$ fixed and this
implies with our conventions that $u$, the coefficient of $\phi^4/N$, is fixed. \par
The integral over $\rho$ in \eZeff\ is then gaussian. The integration results in simply replacing $\rho (x)$  by the solution of
$$\frac{1}{6}u\rho (x)+ r=\lambda (x).\eqnd\eONrhola $$
and one obtains the action
$$\eqalignno{{\cal S}_N ( \sigma ,\lambda) &  = {1 \over 2}\int \d ^{d}x \left[
\left(\partial_{\mu}\sigma \right)^2+ \lambda  \sigma^2  - {3N\over u}\lambda^2 +{6 Nr\over u}  \lambda \right] \cr &\quad +{ (N-1
 ) \over 2}  \tr\ln \left[ -\nabla^2 +\lambda (\bullet) \right]
. &\eqnd  \eactONsigla  \cr}$$
Note, however, that the field $\rho$ has a more direct physical interpretation
than the field $\lambda(x) $.
\smallskip
{\it Diagrammatic interpretation.} In  the $(\phib^2)^2$
field theory, the leading perturbative contributions  in the large $N$
limit come from chains of ``bubble'' diagrams of the form displayed in
figure \label{\figfivNi}. These
diagrams asymptotically form a geometric series, which  the
algebraic techniques explained in this section allow to sum.
\midinsert
\epsfxsize=102mm
\epsfysize=12mm
\centerline{\epsfbox{fig29-1.eps}}
\figure{4.mm}{The dominant diagrams in the large $N$ limit.}
\figlbl\figfivNi
\endinsert
%
\medskip
{\it The low temperature phase.} We first assume that $ \sigma $, the expectation value of the
field, does not vanish, and thus the $O(N)$
symmetry is spontaneously broken. The constant $\rho$ is then given by Eq.~\esaddleN{b} which reduces to $U'(\rho )=0$. The solution must satisfy
$\rho>\rho_c$  and Eq.~\esponmag\ then yields
$\sigma=\sqrt{N(\rho-\rho_c)}$ (we recall $\rho_c=\Omega_d(0)$, Eq.~\eONrhoc).
\par
The condition \eNUcrit\ determines the critical potential $U$:
$$U'(\rho_c)=0\ \Rightarrow \ r=r_c=-u\rho _c/6\,.$$
The expectation value of the field vanishes for $r=r_c$, which
thus corresponds to the critical temperature $T_c$. Then  ($r-r_c=\tau$),
$$U'(\rho)=0\ \Rightarrow\ \rho-\rho_c =-(6/u) \tau\,.\eqnn $$
The $O(N)$ symmetry is broken for $\tu<0$,  that is at low temperature, and we
can rewrite Eq.~\esponmag\ as
$$ \sigma^2=- (6/u)\tu\propto  (-\tu)^{2\beta}\quad {\rm with} \
\beta=\ud\, .\eqnd{\expbeta} $$
We find that, for $ N $ large, the exponent $\beta $ remains mean-field like or quasi-gaussian in all dimensions.
\medskip
{\it The high temperature phase.} For $\tu>0$,  that is above $ T_c $, $ \sigma $
vanishes.
%From equation \esaddleN{c}, we verify that $\partial \rho/\partial m$
%is negative and thus $r$ an increasing function of $m$. The minimum
%value of $r$, obtained for $m=0$, is $ r_c$.
Using Eqs.~\eqns{\eONrhoc,\eTmTc} in Eqs.~\esaddleN{b} and \esaddleN{c}, we then find
\eqna\ecorleng
$$\eqalignno{m^2&=(u/6)(\rho-\rho_c)+\tu\,,& \ecorleng{a} \cr
\rho-\rho_c&= \Omega_d(m)-\Omega_d(0) . &\ecorleng{b}\cr} $$
\smallskip
(i) For $ d>4 $,  the expansion \etadepolii\ implies that the leading
contribution to  $\rho-\rho_c$  is proportional to $m^2$,
as the l.h.s.~of Eq.~\ecorleng{a}, and thus,  at leading order,
$$ m^2=\xi^{-2}\sim \tu^{2\nu} \quad  {\rm with} \quad \nu =\ud\,,
\eqnn $$
which is the mean-field or gaussian result for the correlation exponent $\nu$. \par
(ii) For $ 2<d<4 $,   the leading term is now of order $m^{d-2}$:
$$\rho-\rho_c\sim -K(d)m^{d-2}. $$
In Eq.~\ecorleng{a} the leading $m$-dependent contribution  for $m\to 0$
now comes from $\rho-\rho_c$. Keeping only the
leading term in  \etadepolii, we obtain ($\varepsilon=4-d$)
$$ m=\xi^{-1}\sim \tu^{1/(2-\varepsilon)}, \eqnd{\ecorlenb} $$ which
shows that the exponent $ \nu $ is no longer gaussian (or mean-field like):
$$ \nu ={1 \over 2-\varepsilon} ={1 \over d-2}\cdot \eqnd{\enuNlim}
$$
\par
(iii) For $d=4$, the leading $m$-dependent contribution in Eq.~\ecorleng{a} still comes from $\rho-\rho_c$:
$$m^2 \sim {48\pi^2\over u} {\tau \over \ln(\Lambda /m)}.\eqnd\eNfivdivm $$
The correlation length has no longer  a power law behaviour but, instead, the behaviour of the gaussian model  modified by a logarithm. This is typical of a situation
where the gaussian fixed point is stable, in the presence of a marginal operator. \par
(iv) Examining Eq.~\esaddleN{c} for $\sigma=0$ and $d=2$,  we find
that the correlation length becomes large only for $r\to-\infty$.
This peculiar situation will be discussed in the framework of the
non-linear  $\sigma$-model.
\smallskip
{\it Critical limit $\tu=0$}. At $\tu=0$, $m$ vanishes and
from the form \eDeltasigN\ of the $\sigma$-propagator, we find that
the critical exponent $\eta$ remains gaussian for all $d$:
$$\eta=0\ \Rightarrow \ d_\phi=\ud(d-2)\,.\eqnd\eetaNg$$
We verify that for $d\le 4$, the exponents $\beta,\nu,\eta$ satisfy the scaling relation
proven within the framework of the $\varepsilon$-expansion:
$$\beta=\nu d_\phi=\ud\nu(d-2+\eta).$$
%
\medskip
{\it Singular free energy and scaling equation of state.} In a
constant magnetic field $H$ in the $\sigma$ direction, the free
energy density $W(H) $ (defined here as the opposite of the action density $\cal E$ when the saddle point equations are used)   is given by  \rBrWa
$$\eqalign{W(H) &=\ln {\cal Z} /\Omega =- {\cal E}\cr &  =N\left[{3\over 2u}m^4-{3r\over
u}m^2-{1\over2} m^2\sigma^2/N+H\sigma/N- \int_0^{m  }s\d s\,
\Omega _d(s) \right],\cr} $$ where $\Omega$ is the  $d$
dimensional space volume and $\rho$ has been eliminated using
Eq.~\esaddleN{c}. The saddle point values $m^2,\sigma$  are given
by Eq.~\esaddleN{b} and the modified saddle point
Eq.~\esaddleN{a}:
$$m^2 \sigma = H\,. \eqnd{\esaddfld} $$
The magnetization $M$, expectation value of $\phib$, is
$$M= {\partial W\over\partial H}=\sigma\,,\eqnd\eNfivmag $$
because
partial derivatives of $W$ with respect to $m^2 $ and $\sigma$ vanish as a consequence of the saddle point equations.  The thermodynamic potential density ${\cal G}(M)$, Legendre transform of $W(H)  $,
follows:
$$\eqalignno{{\cal G}(M) &=HM-W(H) &\eqnd\eNfivGam \cr
&=N\left[-{3\over 2u}m^4+{3r\over u}m^2+{1\over2} m^2 M^2/N +
\int_0^{m  }s\d s\, \Omega _d(s) \right]. \cr}$$
As a property of
the Legendre transformation, the saddle point equation for $m^2$ is now
obtained by expressing that the derivative of ${\cal G}$ with
respect to $m^2$ vanishes.
\par
The expansion for large $\Lambda$ of  the $\tr\ln$ has been given  in Eq.~\eNtrlnexp.
Introducing $r_c$, one obtains
%$$\eqalign{
%\tr\ln[( \nabla ^2-m^2)/ \nabla ^2]&={1\over(2\pi)^d}\int\d^d
%p\,\ln[(p^2+m^2)/p^2]
%\cr &=-2{K(d)\over d}m^d-{6 r_c\over
%u}m^2+{a(d)\over2}m^4\Lambda^{4-d}+O(m^{6}\Lambda^{d-6}).
%\cr} $$
$${\cal G}(M)/N={3\over 2}\left({1\over u^*}-{1\over u}\right)m^4
+{3(r-r_c)\over u}m^2+{1\over2}m^2 M^2/N-{K(d)\over
d} m^d ,
\eqnn $$
where we have defined
$$u^*={6\over a(d)}\Lambda^\varepsilon.\eqnd\eustari $$
Note that for $d<4$ the term proportional to $m^4$ is
negligible for $m $ small with respect to the singular term $m^d$. Thus, at
leading order in the critical domain,
$${\cal G}(M)/N= {3\over u}\tu m^2+{1\over2}m^2
M^2/N-{K(d)\over d}m^d , \eqnn $$ where $\tu$ has been
defined in \eTmTc.\par
Expressing that the derivative with respect to $m^2$ vanishes,
$$(6/u)\tu + M ^2/N-K(d)m^{d-2}=0\, , $$
we obtain
$$m =\left[{1\over  K(d)}\left((6/u)\tu
+M^2/N\right)\right]^{1/(d-2)}. $$ It follows that the leading

contribution to the thermodynamic
potential, in the critical domain, is given by
$$ {\cal G}(M)/N \sim {(d-2)\over 2d}{1\over\bigl(K(d)
\bigr)^{2/(d-2)}} \left[(6/u)\tu +M^2/N\right]^{d/(d-2)}.\eqnd\ethermscN $$
>From ${\cal G}(M)$ can be derived various other quantities  like the equation
of state, which is obtained by differentiating with respect to $M$. It can be cast into the scaling form
$$H={\partial {\cal G}\over \partial M}= h _0
M^{\delta}f\left(a_0 \tu/M^2 \right), \eqnd \estatNli  $$
where $h_0$ and $a_0$ are normalization constants. The exponent $\delta$ is given by
$$\delta= {d+2 \over d-2}\,, \eqnn $$ in agreement with the general
scaling relation $\delta=d/ d_\phi-1$, and the function
$f(x)$ by
$$f(x)= (1+ x)^{2/(d-2)}. \eqnd{\estatNgb} $$
The asymptotic form of
$f(x)$ for $x$ large implies $\gamma=2/(d-2)$ again in agreement
with the scaling relation $\gamma=\nu(2-\eta)$. Taking into account
the values of the critical exponents $\gamma$ and $\beta$, it is then
easy to verify that the function $f$ satisfies all required
properties like for example Griffith's analyticity. In particular, the equation
of state can be cast into the parametric form
 \rparamet
$$\eqalign{M & =(a_0)^{1/2} R^{1/2}\theta\,,\cr
\tu & =3R\left(1-\theta^2\right),\cr
H& =h_0 R^{\delta/2}\theta\left(3-2\theta^2\right)^{2/(d-2)}.\cr}$$
\medskip
{\it Leading corrections to scaling.} The $m^4$ term yields the
leading corrections to scaling. It is subleading by a power of
$\tu$:
$$m^4/m^d=O(\tu^{(4-d)/(d-2)}).$$
The exponent governing the leading corrections to scaling in the temperature variable is  $\omega \nu $ ($\omega $ is defined in Eq.~\edefomega) and thus \rSKMa
$$\omega\nu=(4-d)/(d-2)\ \Rightarrow\ \omega=4-d\,.\eqnd\eNomega $$
Note that for the special value $u=u^*$ this
correction vanishes.
\medskip
{\it Specific heat exponent. Amplitude ratios.}  Differentiating
twice ${\cal G}(M)$ with respect to $\tu$, we obtain the specific heat
at fixed magnetization
$$C_H\propto   \left[(6/u)\tu
+M^2/N\right]^{(4-d)/(d-2)}.\eqnd\esphCH $$
For $M=0$, we identify the
specific exponent
$$\alpha={4-d\over d-2}, \eqnn $$
which indeed is equal to $2-d\nu$,
as predicted by scaling relations. Among the universal ratios of amplitudes, one can
calculate for example $R^+_\xi$ and $R_c$ (for definitions, see
chapter 29 of ref.~\rbook)
$$(R^+_\xi)^d={4N\over(d-2)^3}{\Gamma(3-d/2)\over(4\pi)^{d/2}},\quad
R_c={4-d\over(d-2)^2}. \eqnn $$
\midinsert
\epsfxsize=50.mm
\epsfysize=16.mm
\centerline{\epsfbox{bulle4.eps}}
\vskip-15.8mm
\centerline{$q$}
\vskip4.6mm
\centerline{$p-q$}
\figure{3.mm}{The ``bubble" diagram $B_\Lambda(p,m)$.}
\figlbl\figbubii
\endinsert
\medskip
{\it The $\lambda$ and $ \phib ^2$ two-point functions.}
In the high temperature phase, differentiating twice the action \eactONef\ with
respect to $\lambda(x),\rho(x)$ and replacing  the field $\lambda(x)$ by its
expectation value $m^2$, we find the $ \lambda $-propagator
 $$ \Delta_\lambda(p)=-{2 \over N} \left[ {6\over
u}+B_\Lambda(p,m)\right]^{-1} \,, \eqnn $$ where $B_\Lambda(p,m)$
is the bubble diagram of figure \figbubii:
$$B_\Lambda(p,m)={1 \over  (2\pi )^{d}} \int^\Lambda { \d
^{d}q
\over \left(q^2+m^2 \right) \left[ \left(p-q \right)^2+m^2
\right]} , \eqnd\ediagbul $$
and the cut-off symbol $\Lambda $ means calculated with a regularized propagator
as in \epropreg.\par
The $\lambda$-propagator is negative because the $\lambda$-field is imaginary. Using the relation \eONrhola, we obtain the $\rho$ two-point function (in the Fourier representation the constant shift only produces a $\delta $-function at $p=0$) and thus as noted in \ssfivNi,  the
$\phib^2$ two-point function
$$\left<\phib^2 \phib^2\right>=N^2\left<\rho\rho\right>=-{12N/u \over
1+(u/6)B_\Lambda(p,m)}.
\eqnn $$
At zero momentum we recover the specific heat. The small $m$ expansion of
$B_\Lambda(0,m)$ can be derived from the expansion \etadepolii.
One finds
$$\eqalignno{B_\Lambda(0,m)&={1\over(2\pi)^d} \int^\Lambda {\d^{d}q \over
\left(q^2+m^2 \right)^2} \cr &=-{\partial \over \partial m^2}
\Omega_d(m) \mathop{=}_{m\ll\Lambda}
 (d/2-1)K(d) m^{-\varepsilon}-a(d)\Lambda^{-\varepsilon}
+\cdots\hskip8mm &\eqnd\eBLamze \cr}$$
The singular part of the specific heat thus
vanishes as $m^{\varepsilon}$, in agreement with Eq.~\esphCH\ for $M=0$.
\par
In the critical theory ($ m=0 $ at this order) for $2\le d< 4$, the
denominator is also dominated at low momentum by the integral
$$B_\Lambda(p,0)= {1 \over \left(2\pi \right)^{d}} \int^\Lambda {
\d^{d}q \over q^2(p-q)^2} \mathop{=}_{2<d<4} b(d)p^{-\varepsilon}
-a(d)\Lambda^{-\varepsilon}
+O\left(\Lambda^{d-6}p^ 2 \right),\eqnd\ebullecrit  $$
where
$$ b (d)=-{\pi \over\sin(\pi d/2)}
{\Gamma^2 (d/ 2) \over \Gamma (d-1)}N_d \,, \eqnd\econstb $$
and thus
$$ \Delta_{\lambda}(p)\mathop{\sim}_{p\to 0} -{2 \over N
b(d)} p^{\varepsilon}. \eqnd{\eprocrit} $$  We again
verify consistency with scaling relations. In particular, we note
that in the large $N$ limit the {\it dimension  of the
field $\lambda$}\/ is
$$[\lambda]=[\rho]=[\phib^2]=d-1/\nu=\ud(d+\varepsilon)=2\,, \eqnn $$
a result important for the $1/N$ perturbation theory.
\smallskip
{\it Remarks.} \par
(i) For $d=4$,  the integral  has a logarithmic behaviour:
$$B_\Lambda(p,0)\mathop{\sim}_{p\ll \Lambda} {1\over 8\pi^2}\ln (\Lambda/p)+\
{\rm const.}\,, \eqnd\eBivasym $$
and still gives {\it the leading contribution}\/ to the inverse propagator
$\Delta_\lambda\propto 1/\ln(\Lambda/p)$.\par
(ii) Note, therefore,
that for $d\le 4$ the contributions generated by the term
proportional to $\lambda^2(x)$ in \eactONef\ are always negligible in the critical domain.
\medskip
{\it The $\left<\sigma \sigma \right>$ two-point function at low temperature.}
In the phase of broken symmetry the action, after translation of
expectation  values, includes a term proportional to $\sigma\lambda$ and thus the
propagators of the fields $\sigma$ and $\lambda$ are elements of a  $2\times2$ matrix $\bf M$:
$${\bf M^{-1}}(p)=\pmatrix{p^2 & \sigma \cr \sigma & -3N/u-\ud N B_\Lambda(p,0)
\cr} \ ,\eqnd\eNsigiipt $$
where $\sigma= \left<\sigma(x)\right>$ and  $B_\Lambda$ is given by Eq.~\ebullecrit.
For $d<4$ at leading order for $|p|\ll \Lambda $, the determinant is given by
$$1/\det{\bf M}(p)\sim-N\left[b(d)p^{d-2}+6\tau /u\right],$$
where the relation \expbeta\ has been used.
For $|r-r_c|\ll \Lambda ^2$, this expression defines a crossover mass scale
$$m_{\rm cr}=(-\tau /u)^{1/(d-2)}\propto \Lambda \bigl((r_c-r) /\Lambda ^2\big)^{1/(d-2)}= \Lambda\bigl((r_c-r) /\Lambda ^2\bigr)^ \nu  , \eqnd\eNmcrossa $$
at which a crossover between Goldstone behaviour ($N-1$ massless free particles) and critical behaviour  ($N$ massless interacting particles) occurs. \par
At $d=4$, the form \eBivasym\ becomes relevant and
$$ m_{\rm cr}^2\propto {r_c-r \over \ln[\Lambda ^2/(r_c-r)]} .\eqnd\eNmcrossb $$
Finally, for $d>4$, $B_\Lambda(p,0)$ has a limit for $p=0$ and therefore
$$ m_{\rm cr}\propto \sqrt{r_c-r }\,.\eqnd\eNmcrossc $$
In all dimensions $m_{\rm cr}$    scales near $r_c$ as the physical mass above $r_c$.
%
\subsection  RG functions and leading corrections to scaling

 {\it The RG functions.} For a more detailed verification of the
consistency between the large $N$ results and RG predictions, we now
calculate RG functions at leading order for $N\to\infty $.
We set (Eq.~\eustari)  \sslbl\sssEGRN
$$ u=Ng\Lambda^{\varepsilon},\quad g^*=u^*\Lambda^{-\varepsilon}/N=
6/(Na)\,,\eqnd\eustar $$
where the constant $ a(d) $ has been defined in
\etadepolii\ and behaves for $\varepsilon=4-d\to 0$ like
$ a(d)\sim 1/ (8\pi^2\varepsilon) $ (Eq.~\eaespzero).
\par
One then verifies
that $m$ solution of
Eqs.~\ecorleng{} satisfies asymptotically for $\Lambda$ large an equation that expresses that it is
RG invariant:
$$ \left( \Lambda{ \partial \over \partial \Lambda} +\beta (g){\partial \over \partial g}-\eta_2(g)\tu{\partial
\over \partial \tu} \right) m(\tu,g,\Lambda)=0\,, \eqnn $$
where in the r.h.s.\ contributions of order $1/\Lambda^2$ have been neglected.
The RG functions $\beta(g)$ and $\eta_2(g)$ are then given by
$$ \eqalignno{ \beta (g) & = -\varepsilon g(1- g/g^*) , &
\eqnd{\ebetaN} \cr
\nu^{-1}(g)= 2+\eta_2(g)& = 2-\varepsilon g/g^*. & \eqnn \cr} $$
When $a(d)$ is positive (but this not true for all regularizations,
see the discussion below), one finds an IR fixed point $g^*$, as
well as exponents  $\omega=\varepsilon$, and $\nu^{-1}=d-2$, in
agreement with Eqs.~\eqns{\eNomega,\enuNlim}. In the framework of the
$\varepsilon$-expansion, $\omega$ is associated with the leading
corrections to scaling. In the large $N$ limit $\omega$ remains
smaller than 2 for $\varepsilon<2$, and this extends the property established near $d=4$ to all dimensions $2\le d\le 4$. \par
Finally, applying the RG
equations to the propagator \eDeltasigN, one finds
$$\eta(g)=0\ , \eqnn $$
a result consistent with the value \eetaNg\ found for $\eta=\eta (g^*)$.
\medskip
{\it Leading corrections to scaling.} From the general RG analysis, we
expect the leading corrections to scaling to vanish for $u=u^*$.
This property has already been verified for the free energy. Let us now
consider the correlation length or mass $m$
given by Eq.~\ecorleng{}. If we keep the leading correction to
the integral for $m$ small (Eq.~\etadepolii), we find
$$ 1- {u\over u^*} +(u/6) K(d) m^{-\varepsilon}
+O\left({m^{2-\varepsilon} \Lambda^{-2}}
\right) ={\tu\over m^2}\,, \eqnd\ecorlengb $$
where Eq.~\eustar~has been used.  We see that the leading
correction again vanishes for $u=u^*$. Actually, all correction terms
suppressed by powers of order $\varepsilon$ for $d\to 4$ vanish
simultaneously as expected from the RG analysis of the $\phi^4$
field theory. Moreover, one verifies that the leading correction is
proportional to $(u-u^*)\tu^{\varepsilon/(2-\varepsilon)}$, which
leads to $\omega\nu=\varepsilon/(2-\varepsilon)$, in agreement with
Eqs.~\eqns{\eNomega,\enuNlim}.
\par
In the same way, if we keep the leading correction to the
$\lambda$-propagator in the critical theory (equation
\ebullecrit), we find
$$ \Delta_{\lambda} (p )=-{2 \over N} \left[{6\over u}-
{6\over u^*}+ b(d)p^{-\varepsilon}
\right]^{-1}, \eqnd{\epropNli}  $$
where terms of order $ \Lambda^{-2} $ and $1/N$ have been neglected.
The leading corrections to scaling again  cancel for $
u=u^{\ast} $ exactly, as expected.
\smallskip
{\it Discussion.} \par
 (i) One can show that a perturbation due to
irrelevant operators is equivalent, at leading order in the critical
region, to a modification of the $(\phib^2)^2$ coupling.  This can
be explicitly  verified here. The amplitude of the leading
correction to scaling has been found to be proportional to
$6/u-a(d)\Lambda^{-\varepsilon}$, where the value of $a(d)$ depends
on the cut-off procedure and thus on contributions of irrelevant
operators. Let us call $u'$ the $(\phib^2)^2$ coupling constant in
another scheme where $a$ is replaced by $a'$. Identifying the
leading correction to scaling, we find the  relation
$${6\Lambda^{\varepsilon} \over u}-a(d)={6\Lambda^{\varepsilon}
\over u'}-a'(d),$$
a homographic relation that is consistent with the special form
\ebetaN\ of the $\beta$-function.\par
 (ii) {\it The sign of
$a(d)$.} It is generally assumed that $a(d )$ is positive for $2<d<4$.  This is indeed what one finds in the simplest regularization schemes, for example
when the function $D(k^2)$ in \epropreg\ is an increasing function of $k^2$.
Moreover, $a(d)$ is always positive near four
dimensions where it diverges like
$$a(d)\mathop{\sim}_{d\to 4} {1\over 8\pi^2\varepsilon}.$$
Then, for $2<d<4$ there exists an IR fixed point, corresponding to a non-trivial zero $u^*$ of the
$\beta$-function. For the value $u=u^*$ the leading corrections to
scaling vanish.\par
However, for $d<4$ fixed  that positivity of $a(d)$ is not assured. For example, in the case of simple lattice
regularizations it has been shown that in $d=3$ the sign is
arbitrary. \par
When $a(d)$ is negative, the RG method for
large $N$ (at least in the perturbative framework) is confronted
with a serious difficulty.  Indeed, the coupling flows in the IR
limit to large values where the large $N$ expansion is no longer
reliable. It is not known whether this signals a real pathology of the model in the RG sense, or is just an artifact of the large $N$ limit. \par
Another way of viewing the problem is to examine directly the
relation between bare and renormalized coupling constant. Calling
$g_{\rm r} m^{4-d}$ the renormalized four-point function at zero
momentum, we find
$$m^{4-d}g_{\rm r}={\Lambda^{4-d}g\over 1+\Lambda^{4-d}g N
B_\Lambda(0,m)/6} . \eqnd\egrenor$$ In the limit $m\ll\Lambda$, the
relation can be written as
$${1\over g_\r}= {1\over g_\r^*}+\left(m\over\Lambda\right)^{4-d}
\left({1\over g}-{N a(d)\over6}\right),\quad {1\over g_\r^*}={(d-2) N K(d)\over
12}.\eqnd\egrenor $$
%%%%% check
We see that when $a(d)<0$, the limiting value $g_\r=g_\r^* $ for $m/ \Lambda=0 $
 cannot be reached by varying $g$ when $m/\Lambda$ is small but finite  (since $g>0$). In
the same way, leading corrections to scaling can no longer be
cancelled.
%
\subsection Small coupling constant and large momentum
expansions for $d<4$

Section \label{\sssfivNRT} is devoted to a systematic discussion of the $1/N$
expansion. However, the $1/N$ correction to
the two-point function will help us to investigate  immediately
the following problem: the perturbative expansion of the massless $\phi^4$ field theory has IR divergences for any dimension $d<4$, although we believe the critical theory to exist.
In the framework of the $1/N$ expansion, instead, the critical theory ($T=T_c, m^2=0$) is defined for any dimension $d<4$. This
implies that the coefficients of the $1/N$ expansion cannot be expanded in
a Taylor series of the coupling constant. \par
To understand the phenomenon, we consider the
$\left<\sigma\sigma\right>$
correlation function at order $1/N$.  At this order only one diagram
contributes (figure \label{\figbubiii}), containing two $\lambda^2\sigma$ vertices.
After mass renormalization,   in the  large cut-off limit, we find \sslbl\ssNplarge \par
$$\Gamma^{(2)}_{\sigma\sigma}(p)= p^2 +{2\over N (2\pi)^{d}}\int
{\d^{d}q \over(6/u)+b(d)q^{-\varepsilon}}\left({1 \over
(p+q)^2} -{1 \over q^2}\right) +O\left({1 \over N^2}\right)
. \eqnd{\eONpropi} $$
We now expand $\Gamma^{(2)}_{\sigma\sigma}$ for $u\to0$.
Note that since the gaussian fixed point is an UV fixed
point, the small coupling expansion is also a large momentum
expansion. \par
An analytic study  then reveals that the integral has an expansion of the form
$$\sum_{k\ge 1} \alpha_k u^k p^{2-k\varepsilon}+\beta_k
u^{(2+2k)/\varepsilon} p^{-2k} .\eqnn $$
The coefficients $\alpha_k,\beta_k$ can be obtained by performing a Mellin
transformation over $u$ on the integral. Indeed, if a function $f(u)$
behaves like $u^t$ for $u$ small, then the Mellin transform
$$M(s)=\int_0^\infty\d u\,u^{-1-s}f(u), $$
has a pole at $s=t$. Applying the transformation to the
integral  and inverting $q$ and $u$ integrations, we have to calculate
the integral
$$\int_0^\infty\d
u\,{u^{-1-s}\over(6/u)+b(d)q^{-\varepsilon}}
={1\over 6}\left(b(d)q^{-\varepsilon}\over
6\right)^{1-s}{\pi
\over \sin\pi s}\,\cdot $$
Then, the value of the remaining $q$ integral follows from the generic
result \eqns{\eintmunu}.\par
The terms with integer powers of $u$ correspond to the formal
perturbative expansion where each integral is calculated for
$\varepsilon$ small enough. $\alpha_k$
has poles at $\varepsilon=(2l+2)/k$ for which  the corresponding power of
$p^2$ is $-l$, that is an integer. One verifies that $\beta_l$ has a pole at
the same value of $\varepsilon$ and that the singular contributions cancel in
the sum \rSymanza. For these dimensions logarithms of $u$ appear in the small $u$ expansion.
\midinsert
\epsfxsize=50.mm
\epsfysize=16.mm
\centerline{\epsfbox{diag2b.eps}}
\vskip-17.mm
\centerline{$\lambda$}
\vskip7.mm
\centerline{$\sigma$}
\figure{3.mm}{The diagram contributing to $\Gamma^{(2)}_{\sigma\sigma}$ at
order  $1/N$.}
\figlbl\figbubiii
\endinsert
%
\subsection Dimension four: triviality, renormalons, Higgs mass

A number of issues concerning the physics of the $(\phib^2)^2$  theory
in four dimensions can be addressed within the framework of the large $N$
expansion. For  simplicity reasons, we consider here only the critical
(i.e.~massless) theory.\sslbl \sstrivia
\medskip
{\it Triviality and UV renormalons.} One
verifies that the renormalized coupling constant $g_\r$,
defined as the  value of the vertex $\left<\sigma\sigma\sigma\sigma\right>$ at
momenta of order $\mu\ll\Lambda$, is given by
$$g_\r={g\over1+\frac{1}{6}N g B_\Lambda(\mu,0)}\,,\eqnd\egfivivren $$
where  $B_\Lambda(p,0)$, which  corresponds to the bubble diagram  (figure \figbubii), is given by Eq.~\eBivasym:
$$B_\Lambda(p,0)\mathop{\sim}_{p\ll \Lambda} {1\over 8\pi^2}\ln (\Lambda/p)+\
{\rm const.}\,.  $$
We see that when the ratio $\mu/\Lambda$ goes to zero, the renormalized
coupling constant vanishes,  independently of the value of $g$ (here $g$ is physical, that is $g>0$). %% (see section \sssLMofiv)
This is the so-called {\it triviality}\/ property. In the traditional presentation of quantum field field, one usually insists in taking the
infinite cut-off $\Lambda$ limit. Here, one finds then only a free field
theory.
Another way of formulating the problem is the following: it seems impossible to
construct in four dimensions a $\phi^4 $ field  theory consistent (in the
sense of satisfying all usual physical requirements) on all scales for non
zero coupling. Of course, in the logic of {\it effective} field theories this
is no longer an issue. The triviality property just implies that the
renormalized or effective charge is logarithmically small as indicated by
Eqs.~\eqns{\egfivivren,\eBivasym}. Note that if $g$ is generic (not too
small) and $\Lambda/\mu$ large, $g_r$ is essentially independent of the
initial coupling constant. The renormalized coupling remains an adjustable, though bounded, quantity only when the bare coupling is small enough and the RG flow is thus very slow.\par
If we proceed formally and, ignoring the problem,
express the leading contribution to the four-point function in terms of the
renormalized constant,
$${g\over 1+{N\over 48\pi^2}g\ln (\Lambda/p)}={g_\r\over 1+{N\over
48\pi^2}g_\r\ln (\mu/p)} \,,$$
we then find that the function has a pole for
$$p=\mu\e^{48\pi^2/(Ng_\r)}.$$
This unphysical pole (called sometimes Landau's ghost) is generated because $g=0$ is an IR fixed point. If we calculate contributions of
higher orders, for example to the two-point function, this pole makes
the loop integrals  diverge. In an expansion in powers of
$g_\r$, each term is instead calculable
but one finds, after renormalization, UV contributions of the type
$$\int^\infty{\d^4 q\over q^6}\left(-{Ng_\r\over48\pi^2}\ln(\mu/q)\right)^k
\mathop{\propto}_{k\to\infty}\left({Ng_\r\over96\pi^2}\right)^k k!\,.$$
The perturbative manifestation of  Landau's ghost is the appearance of
contributions to the perturbation series which are not Borel summable.
This effect is called  UV renormalon effect \rUVrenorm. By contrast the contributions due to the finite momentum region, which can be
evaluated by a semi-classical analysis, are Borel summable, but invisible for
$N$ large.  Note,
finally,  that this UV problem is independent of the mass of the field $\phib$,
which we have taken zero only for simplicity.
\medskip
{\it IR renormalons.}
We now illustrate the problem of IR renormalons with the same example
of the massless  $(\phib^2)^2$ theory (but now zero mass is essential), in
four dimensions, in the large $N$ limit \rDavid. We calculate the contribution
of the small momentum region to the mass renormalization, at
cut-off $\Lambda$ fixed. In the large $N$ limit, the mass renormalization
then  is proportional to (see Eq.~\eONpropi)
$$\int^\Lambda{\d^4 q\over q^2\bigl(1+\frac{1}{6}NgB_\Lambda(q)\bigr)}
\sim\int{\d^4 q\over q^2\bigl(1+{N\over 48\pi^2}g\ln (\Lambda/q) \bigr)}\,.$$
It is easy to expand this expression in powers of the coupling constant $g$.
The term of order $k$ in the limit $k\to\infty$  behaves as $(-1)^k
(N/ 96\pi^2)^k k!$. This contribution has the alternating sign of the
semi-classical contribution. Note that more generally for $N$ finite
on finds $(-\beta_2/2)^k k!$. IR singularities are responsible for
additional, Borel summable, contributions to the large order behaviour.\par
In a theory asymptotically free for large momentum, clearly the roles
of IR and UV singularities are interchanged.
\medskip
{\it The mass of the $\sigma$ field in the phase of broken  symmetry}.
In four dimensions the $\phi^4$ theory is an ingredient of the Standard Model, and the field
$\sigma$ then represents the Higgs field. %%In section \sssmHiggs~we have
%% shown that
With some reasonable assumptions, it is possible to establish for finite $N$
a  semi-quantitative bound on the Higgs mass. Let us examine here this question for $N$ large.\par
In the phase of broken symmetry the action, after translation of
expectation  values, includes a term proportional to $\sigma\lambda$ and thus the
propagators of the fields $\sigma$ and $\lambda$ are elements of the  $2\times2$
matrix $\bf M$  defined in \eNsigiipt,
 where $B_\Lambda(p,0)$ is given by Eq.~\eBivasym. \par
It is convenient to introduce a  RG invariant mass scale $M$, that we  define by
$${48\pi^2\over Ng}+8\pi^2 B_\Lambda(p,0)= \ln(M/p).$$
Then,
$$M\propto \e^{48\pi^2/Ng}\Lambda\,.$$
Poles of the propagator correspond to zeros of  $\det{\bf M} $. Solving
the equation $\det{\bf M} =0$,
one finds for the mass $m_\sigma$ of the field $\sigma$ at this order
$$p^2 \ln (M/p)=-(16\pi^2/N)\sigma^2\ \Rightarrow\ m_\sigma^2\ln (i M/
m_\sigma) =(16\pi^2/N)\sigma^2.$$
The solution of the equation is complex, because the particle
$\sigma$ can  decay into massless Goldstone bosons. At $\sigma$ fixed, the
mass decreases when the cut-off increases or when the coupling  constant goes
to zero. Expressing  that the mass must be smaller than the cut-off, one
obtains an upper-bound on $m_\sigma$ (but which  depends somewhat on the precise regularization) \rHiggsbound. %% that one can compare with the one obtained in the general
%% discussion of section \sssmHiggs.
%
\subsection Other methods. General vector field theories

The large $N$ limit can be obtained by several other algebraic methods.
Without being exhaustive, let us list a few. Schwinger--Dyson equations
for $N$ large lead to a self-consistent equation for the two-point function
\rSymanzb.
Some versions of the Hartree--Fock approximation or variational methods
also yield the large $N$
result as we show in section \label{\ssNVarfiv}. The functional (also called exact) RG takes the form of partial differential equations in the large $N$ limit \rFRG.
\sslbl\ssfivNge \par
>From the point of view of stochastic quantization or critical dynamics the
Langevin equation also becomes linear and self-consistent for $N$ large,
because the fluctuations of $\phib^2(x,t)$ are small. As a byproduct the
large $N$ expansion of the equilibrium equal-time correlation functions
is recovered. This is a topic we study in section \label{\ssCrDY}.
\par
\medskip
{\it General vector field theories.} We have shown how the large $N$ expansion can be generated for a general function $NU(\phib^2/N)$.
This method will be applied
in section \label{\ssdblescal} to the study of multicritical points and
double scaling limit.
\par
We now briefly explain how the algebraic
method of section \ssfivNi~can be generalized to $O(N)$ symmetric actions that depend on several vector fields. Again, the composite fields which are expected to have small
fluctuations, are the $O(N)$ scalars constructed from all $O(N)$ vectors.  One thus introduces pairs of fields and Lagrange multipliers for all independent
$O(N)$ invariant scalar products constructed from the many-component fields \rPeRoVi.
\par
Let us illustrate the idea with the example of two fields $\phib_1$ and $\phib_2$, and  a symmetric interaction  arbitrary function of the three scalar  invariants $\rho_{12}=\phib_{1}\cdot\phib_2/N$, $\rho_{11}=\phib^2_{1}/N$ and $\rho_{22}=\phib^2_2/N$:
$${\cal S}(\phib_1, \phib_2)= \int\d^d x \left\{\ud \left[ \partial_{\mu} \phib_1 (x) \right]^2
+\ud \left[ \partial_{\mu} \phib_2 (x) \right]^2+
NU (\rho_{11},\rho_{12},\rho_{22}) \right\} .\eqnd{\eactONg}$$
We then introduce three pairs of fields $\rho_{ij}(x)$ and $\lambda_{ij}(x)$ and use the identity
$$\eqalignno{&\exp\left[-\int \d^{d}x\,NU (\phib^2_{1}/N, \phib_{1}\cdot\phib_2/N,\phib^2_2/N)\right] \propto \int
\left[\d\rho_{ij}(x)\,\d\lambda_{ij}(x) \right] \cr & \quad \times \exp\left\{-\int
\d^{d}x\left[\sum_{ij}\ud\lambda_{ij}\left(\phib_i\cdot \phib_j-N\rho_{ij}\right) +
NU(\rho_{11},\rho_{12},\rho_{22})\right]\right\}. \hskip7mm &\eqnd\egeniden \cr}$$
The identity \egeniden\ transforms the action into a quadratic form in $\phib_i$,
the integration over $\phib$ can thus be performed.
The large $N$ limit is then again obtained by the steepest descent method.
In the special case in which $U(\ro)$ is a quadratic function, the
integral over all $\rho$'s can also be performed.
If the action is a general $O(N)$ invariant function of $p$ fields $\phib_i$,
it is necessary to introduce $p(p-1)/2$ pairs of $\rho$ and $\lambda $ fields.
%%%%%%%%%%%%%%%%%%%%%%%%%%%%%%%%
\subsection{Variational calculations in large $N$ quantum field theory}

A possible extra insight into the large $N$ limit and its non-perturbative nature
is revealed when  variational methods are employed \rBarMos.
The gaussian variational wave functional reproduces for $N$ large
the saddle point or gap equations of the $1/N$ expansion but also yields a clear
picture of the end-point contribution in the variational
parameter phase space.  It is  applicable for finite values of $N$ though it simplifies for $N$ large. However, as usual with variational methods, it is difficult to improve results systematically, while the large $N$ limit is the leading term in an  $1/N$ expansion.\sslbl\ssNVarfiv
\par
To explain the variational method, we  use again the concrete
example of the $O(N)$ symmetric scalar field theory \eONpart\
%$$ {\cal Z}=\int [\d\phib] \, \e^{-{\cal S}(\phib)}  ,$$
with the euclidean action   \eactONgen:
$$ {\cal S}  ( \phib )= \int \left[\ud \left[
\partial_{\mu} \phib (x) \right]^2+NU\bigl(\phib^2(x)/N\bigr)  \right] \d^{d}x\,.
 $$
In section \label{\scFTQFT} the more general situation
of a  finite euclidean time interval $\beta$  with periodic
boundary conditions will be discussed. The functional integral \eONpart\
then represents the quantum partition function
$${\cal Z}(\beta)=\tr\e^{-\beta H},$$
where $H$ is the quantum hamiltonian corresponding to the euclidean
action, and $\beta =1/T$ the inverse temperature. \par
For any   action  ${\cal S}$ and auxiliary action ${\cal S}_0$,
we can write the identity
$$ {\cal Z} = \int [\d \phib] \e^{-{\cal S}(\phib)}
= {\cal Z}_0 \left< \e^{-({\cal S}-{\cal S}_0)} \right>_0\,,\eqnn $$
where $\left<\bullet\right>_0$ means expectation value with respect
to $\e^{-{\cal S}_0}$.
The variational principle then relies on the general (convexity) inequality
$$\bigbra \e^{-({\cal S}-{\cal S}_0)} \bigket_{\!0}~~\geq
~~\e^{-\bigbra {\cal S}-{\cal S}_0 \bigket_{\!0}}. \eqnd\evarineq $$
The trial action ${\cal S}_0$  then is chosen to maximize the r.h.s., a general strategy that also leads to mean field approximations.\par
In the limit of infinite $d$-dimensional volume, or equivalently
$d-1$-dimensional volume $V_{d-1}$ and zero-temperature ($\beta\to\infty $)
$$\lim_{\beta\to\infty }-{1\over\beta V_{d-1}}\ln{\cal Z}= {\cal E} \,,\quad
\lim_{\beta\to\infty }-{1\over\beta V_{d-1}}\ln{\cal Z}_0= {\cal E}_0 \,,$$
where  ${\cal E}$  and  ${\cal E}_0$ are the ground state energy (or action) densities  corresponding to ${\cal S}$ and ${\cal S}_0$ respectively.
As a consequence  one obtains the  inequality
$$  {\cal E}
  \le {\cal E}_{\rm var.}={\cal E}_0 +{1\over V_{d-1} \beta  } \left<  {\cal S}-{\cal S}_0\right>_0 ,$$
%\refs{\Vol}\ .
One verifies that $V_{d-1} {\cal E}_{\rm var.}$ is also the expectation value
of the hamiltonian corresponding to ${\cal S}$ in the ground state of the
hamiltonian corresponding to ${\cal S}_0$.
In the zero temperature limit   the choice of a trial action ${\cal
S}_0$  thus becomes equivalent to the choice of a trial wave
functional (the ground state) in a Schr\"odinger representation.\par
 %is equivalent
%\refs{\Exep}\
For trial action  ${\cal S}_0$, we take a free field action with a mass
$m$ that is a variational parameter:
$$  {\cal S}_0={1\over2}\int \d^ d x \,
 \left( {(\partial_{\mu} \phib)}^2 +
 m^2   \phib^2 ~\right)  . $$
This corresponds, in the Schr\"odinger representation, to choosing a gaussian wave functional as a variational state:
$$\Psi(\phi)=\exp\left[-{1\over2}\int\d^{d-1}k\,  \tilde \phib(k)  \sqrt{k^2+m^2} \tilde\phib(-k)\right],$$
where $\tilde \phib(k)$ are the Fourier components of $\phib(x)$  or
more generally,
$$\Psi(\phi)=\exp\left[-{1\over2}\int\d^{d-1}x \,\int\d^{d-1}y\, \phib(x) K _m(x-y)\phib(y)\right],$$
where $K _m$ is determined by minimizing ${\cal E}_{\rm var.}$
\rJaStr .\par

The corresponding energy density is %Eq.6
$$ {\cal E}_0 = {N\over 2 (2\pi)^d}\int \d^d k\, \ln[(k^2+m^2 )/k^2] ={N \over
{(2\pi)}^{d-1} }
\int\d^ {d-1}\!k\left[\ud \omega(k)\right],    $$
where $\omega(k)=\sqrt{k^2+m^2}$.\par
The variational energy  then becomes
$$ {\cal E}_{\rm var.}= {N\over 2 (2\pi)^d}\int \d^d k\, \ln[(k^2+m^2 )/k^2]
+N\left<U\bigl(\phib^2(x)/N\bigr)
-\ud m^2 \phib^2(x)/N\right>_0\,.\eqnd\evarTzero $$
The r.h.s.~contains gaussian expectation  values that can be calculated explicitly.
We introduce the parameter
$$\rho=\left<\phib^2(x)/N\right>_0={1\over (2\pi)^d}\int{\d^d k\over k^2+m^2}=\Omega _d(m),
\eqnd\esaddleNc $$
where $\Omega _d(m)$ is defined by Eq.~\etadepole.
The form of ${\cal S}_0$ implies
$${\partial {\cal E}_0\over \partial m^2}=\ud \left<\phib^2\right>=\ud
N\rho\,.$$
The important remark, which explains the role of the large $N$ limit,
is (see also appendix \label{\appnormprod})
%\refs{\Yaffe}\
that  for  $N\to\infty $ the r.h.s.~in Eq.~\evarTzero\ simplifies:
$$\left< U(\phib^2 /N) \right>  = U(\left< \phib^2 /N\right>)
\left[1 + O({1 / N})\right] .\eqnd\eNvarav $$
We then find
$${\cal E}_{\rm var.}/N\mathop{\to}_{N\to \infty } {1\over 2 (2\pi)^d}
\int \d^d k\, \ln[(k^2+m^2 )/k^2] +
 U(\rho)-\ud m^2 \rho \,,$$
an expression identical (for $\sigma =0$) to \eNEner. Note, however, that here Eq.~\esaddleN{c}
is automatically satisfied, since it defines the parameter $\rho $ (Eq.~\esaddleNc).
The quantity $\cal E$ can now be minimized with respect to the free parameter $m^2$, while
without the constraint \esaddleNc\ it has no minimum.
\par
In the broken phase, previous equations have no solution
and one must take as a trial action a free action with a shifted
field:
$${\cal S}_0={1\over2}\int \d^ d x \,
 \left( {(\partial_{\mu} \phib(x))}^2 +
 m^2   (\phib(x)-\phib_0)^2 ~\right)  , $$
where $\phib_0^2=\sigma ^2$ is an additional variational parameter.
Eq.~\esaddleNc~is replaced by
$$\rho=(1/N)\left<\phib^2(x)\right>_0= \Omega _d(m) +\sigma ^2/N\,.\eqnd\esaddleNcii $$
Then,
$${\cal E}_{\rm var.}/N\mathop{=}_{N\to \infty }  {1\over 2 (2\pi)^d}
\int \d^d k\, \ln[(k^2+m^2 )/k^2] +U(\rho)-\ud m^2( \rho-\sigma
^2/N).$$
We  note that again ${\cal E}_{\rm var}$ is identical to
\eNEner~when Eq.~\esaddleN{c} is used. \par
Differentiating the
variational energy with respect to  $m^2$, we first notice
$${\partial \over \partial m^2}{\cal E}_0=\ud\left<(\phib-\phib_0)^2\right>_0=
\ud(N\rho-\sigma ^2)$$
and thus find
$$\eqalign{ {\partial {\cal E}_{\rm var.}\over \partial m^2}&=
\ud N(\rho-\sigma ^2/N)+N {\partial \over \partial
m^2}\left[U(\rho)-\ud m^2( \rho-\sigma ^2/N)\right] \cr
&=N{\partial \rho \over \partial m^2}\left[U'(\rho)-\ud m^2\right].\cr} $$
Since ${\partial \rho / \partial m^2}$ is strictly negative, the
derivative vanishes only when
$$U'(\rho)-\ud m^2=0\,, $$
which, combined with Eq.~\esaddleNcii, yields the set of equations
\esaddleN{b} and \esaddleN{c}.
Differentiating with respect to $\sigma $, we recover Eq.~\esaddleN{a}.
\par
Therefore, in the large $N$ limit the saddle point equations and the
variational equations coincide.
It should, however, be remembered that since we perform a variational
calculation the lowest energy eigenstate is not necessarily the
state determined by the solution of Eqs.~\esaddleN{}.
%\eqns{\gapEq,\Goldston}.
The end-points in the range of variation of $m^2$ and ${\phib_0}^2$
have to be considered as well and the values of ${\cal E}_{\rm var}$ at these
points have to be examined and compared to the extremum values \rBarMos.
%found in Eqs.~\eqns{\gapEq,\Goldston}.\par
\medskip
{\it The $(\phib^2)^2$ field theory at negative coupling for $d=4$.}
We again consider the large $N$ self interacting scalar field in $d=4$
dimensions  with ($\rho\equiv \phib^2/N$)
%e17
 $$U(\rho)={1\over2} r\rho+{u\over4!}\rho^2.        \eqnd \phiPotential $$
The large $N$ limit is defined as $N \to \infty$ holding
$u $, $r$  and the necessary
ultraviolet cutoff, $\Lambda$, fixed.
We have already discussed the triviality of the $(\phib^2)^2$ field theory
in $d=4$ for  positive coupling in section \label{\sstrivia}. Here, we add some comments about the situation for negative coupling. Even though the initial functional integral is defined only for $u>0$, the large $N$ expansion seems to have a well-defined continuation.\par
The only divergence appears in
$$ \rho(m^2)=\left<\phib^2\right>/N=\rho_c- {m^2 \over 8 \pi^2} \ln(\Lambda /m)+\sigma ^2/N \,,\quad
\rho_c= \Omega _4(0)\,,     \eqnn   $$
where a numerical, regularization dependent, constant has been cancelled
by adjusting the definition of the cut-off $\Lambda $.\par
We then obtain the energy density
$$\eqalignno{ { 1 \over N} {\cal E}_{\rm var}( \sigma ,m^2)&=U(\rho)-\ud \int_0^{m^2}
s{\partial \rho(s)\over \partial s}\d s=
U(\rho_c) +{1\over 32\pi^2} m^4( \ln( \Lambda/m)- \frac{1}{4}) \cr
& +{u\over 24}\biggclb \sigma^2/N +6 {r-r_c \over u} - {1\over 8\pi^2}    m^2 \ln(\Lambda/ m)  \biggcrb^2 -{3\over 2} {(r-r_c)^2 \over
u}\,.\hskip8mm &\eqnd\WfourD  }$$
The gap equation   can be expressed in terms of the
renormalized parameters (see Eq.~\egfivivren),
$$ {1 \over u_\r} ={1\over u}  +{1 \over  48\pi^2} \ln(\Lambda/\mu)\,,  \quad
{M^2 \over u_\r}  =  {r-r_c  \over u}  \,,  \eqnd \ecoupmassRe   $$
where $\mu$ is the renormalization scale and $M$ a renormalized mass parameter, as
$${m^2\over u_\r}=-{m^2\ln(\mu/m) \over 48\pi^2}+{M^2\over u_\r}+{\sigma ^2\over 6N}\,.$$
In the  case $u < 0 $ the renormalized coupling constant $u_\r$ does
not necessarily approach zero as the cutoff is removed (if $u
\rightarrow 0^-$ as $\Lambda \to \infty$).
One also finds that the solutions of the gap equation reproduce the past
results in the literature.
%of{\hbox{ Ref. {\ON} .}
However, the ground state energy is lower
at the end point $m^2=0$ value in which the $O(N)$ symmetry is broken
down to $O(N-1)$. One finds in Eq.~\WfourD~that with $u < 0$,
the ground state energy density,
${\cal E}_{\rm var} \rightarrow -\infty$ as
$ \sigma $ is becoming larger. Since ${\cal E}_{\rm var}$, in this case, is
the upper limit to the exact ground state energy density,
one concludes that the theory
is inconsistent for $u< 0$.
%\refs{\ST} .

One finds, however, that when $u<0$ the $m\ne 0$ solution of
%of references {\ON}\ ,
 \esaddleN{} %Eqs.~\eqns{\gapEq,\Goldston},
corresponds to an $O(N)$ symmetric metastable state whose life-time
can be calculated. One finds
%\refs{\Bm}\
that this metastable state is stabilized by a large
tunnelling barrier; its decay rate
is proportional to $\e^{-N}$
and thus, in our large $N$ limit, is an acceptable
ground state for the theory.
However, this metastable state is not a good
candidate for the vacuum of a realistic model because of its
peculiar behaviour at finite temperature
(see  section \label{\ssNvarfivT}).
At first, it may sound strange that an $O(N)$ symmetric state will
become unstable as the temperature is raised. Indeed, one may expect that
the usual behaviour will appear here,
in which, as the temperature increases
the $O(N)$ symmetric phase is stabilized, whereas
the $O(N)$ broken phase,
(which here is the end point $m^2=0$ phase) will be destabilized.
This expectation is based on the fact that at finite temperature,
thermal fluctuations add up to the quantum fluctuations of the field
operator $\phib^2$ .
Thus, one usually expects that a possible
negative ``mass" at low temperature
will turn positive as the temperature
increases, and the possible broken
symmetry will be restored. This, however, is true as long as one
discusses a stable  theory $u>0$. The effect of the temperature
is reversed in a metastable situation (here, $u<0$) since thermal fluctuations help to overcome potential barriers.

The triviality of $ (\phib^2)^2$ for $u > 0 $
and its inconsistency for $u< 0$
have been derived here for the large $N$ limit of
the theory. It seems very likely that these results
persist also at finite $N$.
%as indicated by the results
%of Ref. {\Trivia}.
Interesting suggestions came
from different view points
%\refs {\DN} \refs {\HASEN}
that the
positively coupled theory may have non-trivial
implications on practical physics
issues. In the standard Weak-Electromagnetic theory,
bounds on the Higgs mass can be
%\hbox{ derived \refs{\HASEN}. }
derived (see for example section \sstrivia).
%

\vfill\eject

\section Models on symmetric spaces in the large $N$ limit

A whole class of geometric models involving only scalar fields, for which the non-linear $\sigma $ model is the simplest example, shares  various classical and quantum  properties: The field belongs to a symmetric space, associated to a coset $G/H$, where $G$ is a Lie group and $H$ a compact maximal subgroup. The action is unique, up to a multiplicative factor which is the coupling constant, and takes the general form \sslbl\ssNsymspace
$${\cal S}(\varphi)={1\over 2T}\int\d^d x\, \partial _\mu \varphi^i(x)g_{ij}(\varphi)  \partial _\mu \varphi^j(x), $$
where $g_{ij}$ is the metric tensor on the corresponding manifold $G/H$.
This action leads to an infinite number of classical conservation laws in two dimensions.
The corresponding quantum models are renormalizable in two dimensions, and then UV asymptotically free. In the classical limit the fields are massless, and correspond to the Goldstone bosons of the $G$ symmetry broken down to $H$. Since continuous symmetries cannot be broken in two  dimensions, the
 spectrum of such theories is non-perturbative. In dimension $2+\varepsilon$, one finds a critical coupling constant $T_c$ where a phase transition occurs. \par
To go beyond the $\varepsilon=d-2$ expansion, one can consider using large $N$ techniques.
However, only a subclass can be studied in this way. To this subclass belong the $O(N)$ non-linear
$\sigma $-model and the $CP(N-1)$ model that are discussed below. A number of other models based on
Grassmannian manifolds could also be  investigated, like $O(N)/O(N-p)\times O(p)$ or similarly
$U(N)/ U(N-p)\times U(p)$. More general homogeneous spaces with several coupling constants could also be considered.
%
\subsection The non-linear $\sigma$-model in the large $N$ limit

In reference \rBrZJsig\ it was first shown that universal
quantities could be determined in the form of an $\varepsilon=d-2$ expansion (at least for $N>2$). Again, as for the $\varepsilon=4-d$ expansion, it is somewhat reassuring  that at least in the
limiting case $N\to\infty $,  the results
obtained in this way remained valid even when $ \varepsilon$
is no longer infinitesimal \refs{\rDerMa,\rCamros}. \par
Moreover, the $1/N$ expansion allows exhibiting a remarkable relation between  the non-linear $\sigma$-model  and the  $(\phib^2)^2$ field theory,  a relation expected on physical grounds \rBrZJsig. \par
Finally, large $N$ techniques are well adapted to the analysis of finite size effects in critical systems, a feature we illustrate with a system in a finite volume with periodic boundary conditions.
\sslbl\ssLTsN
\medskip
%%%%%%%%%%%%%%%%
{\it Non-linear $\sigma $-model and $(\phib^2)^2 $ field theory.}
We have noticed that the term proportional to $\int\d^d x\,\lambda^2(x)$ has dimension $4-d$ for $N\to \infty $. It is thus irrelevant in
the critical domain for  all dimensions $d<4$ and can be omitted at leading order (this
also applies to $d=4$ where it is marginal but yields only logarithmic
corrections).
Actually, the constant part in the inverse propagator as written in equation
\epropNli~plays the role of a large momentum cut-off. We thus omit the $\lambda^2$ term in the action \eactONsigla\ after shifting
 the field
$\lambda(x)$ by its expectation value $m^2$  (Eq.~\eNsaddpts),
$\lambda(x) \mapsto m^2+\lambda(x)$:\sslbl\sssfivNRT %\sslbl\ssfivNe
$$\eqalignno{{\cal S}_N ( \sigma ,\lambda) &  = {1 \over 2}\int \d ^{d}x \left[
\left(\partial_{\mu}\sigma \right)^2+ m^2\sigma
^2+ \lambda  \sigma^2  - {3N\over u}\lambda^2  -{6 N\over u} \left(m^2-r\right) \lambda \right] \cr &\quad +{ (N-1
 ) \over 2}  \tr\ln \left[ -\nabla^2 +m^2+\lambda (\bullet) \right]
. &\eqnd  \eacteffb  \cr}$$
If we then work backwards, reintroduce the initial field $\phib$ and
integrate over $ \lambda (x) $, we find
$$ {\cal Z}= \int \left[ \d  \phib (x)\right]\delta
\left[ \phib^2 (x)/N- 6   (m^2-r  )/u
\right] \exp\left[- { 1 \over 2}\int\left(\partial_{\mu} \phib(x)\right)^2
\d ^{d}x\right]. \eqnd{\epartsig} $$
Under this form we recognize the partition function of the $ O(N) $ symmetric
non-linear $ \sigma $-model in an unconventional normalization. We have,
therefore, discovered a remarkable correspondence,  to all orders in an $ 1/N $
expansion, between the non-linear $ \sigma $-model and the $ ( \phib^2 )^2 $ field theory. \par
Actually, reviewing carefully the arguments, one verifies that the identity between the two models has been derived here under the implicit
condition $u_{\phi^4}\propto \Lambda ^{4-d}\gg(m_{\phi})^{4-d}$, which is the generic situation if
the $\phi^4$ interaction is an effective long distance interaction generated
 by some microscopic model. If the condition is not satisfied the situation
is less clear.
\medskip
{\it The large $N$ limit.} We now  write the partition function of the non-linear $\sigma $ model, with slightly more usual notation, as
$${\cal Z}= \int \left[\d\phi(x)\d\lambda(x)\right]
\exp\left[-{\cal S}(\phib,\lambda)\right]  \eqnn  $$
with
$${\cal S}(\phib,\lambda) = {1 \over 2T}\int \d^{d}x \left[ \left(
\partial_{\mu}\phib \right)^2 + \lambda \left(\phib^2 -N \right)\right].
\eqnd\eactsigla $$
By solving  in the large $N$ limit the $\sigma$-model directly, we are able to exhibit more explicitly the correspondence
between the different set of parameters used in the two models.
\par
The field $\phib$ represents a classical spin of size $\sqrt{N}$. The
coupling constant $T$, which is a loop expansion parameter, represents
the temperature of the classical spin model. Eventually, it will be
useful to introduce a dimensionless coupling constant $t$, setting
$$T=\Lambda^{2-d} Nt\,. \eqnd\eNsigtdef $$
\par
We now separate the field $\phib$ into $N-1$ components,
which we call $\pib$ in what follows, and over which we
integrate as we did in section \ssNbosgen,  and a remaining component $\sigma$. We obtain
$${\cal Z}= \int \left[\d\sigma(x)\d\lambda(x)\right]
\exp\left[-{\cal S}_N(\sigma,\lambda)\right]  \eqnn $$
with, for $N\gg 1$,
$${\cal S}_N  (\sigma,\lambda  )=
 {1 \over 2T}\int \left[ \left(\partial_{\mu}\sigma \right)^2+
\left(\sigma^2 (x)-N\right) \lambda (x)
\right] \d^{d}x +{N \over 2}   \tr\ln \left[
-\nabla^2 +\lambda (\cdot) \right] .  \eqnn $$
The large $N$ limit is  taken here at $T $ fixed. The saddle point equations,
analogous to Eqs.~\esaddleN{}, are
\eqna\emgNsig
$$\eqalignno{m^2\sigma &=0\, ,& \emgNsig{a} \cr
\sigma^2/N& = 1 -   \Omega _d(m)T\,,& \emgNsig{b}\cr} $$
where we have set $\left<\lambda(x)\right>=m^2$ and introduced the function \etadepole. At low temperature and $d>2$, $\sigma $
is different from zero and thus $m$, which is the mass of the $\pi$-field,
vanishes. Eq.~\emgNsig{b} yields the spontaneous magnetization:
$$\sigma^ 2 /N = 1 -  \Omega _d(0)T   .\eqnd\emagNsig $$
Setting
$$  T_c = 1/ \Omega _d(0) ,\eqnn $$
we can write Eq.~\emagNsig\ as
$$\sigma^2/N = 1 - T/T_c\, . \eqnd\emgTNsig $$
Thus, $T_c$ is the critical temperature where $\sigma$ vanishes.\par
Above $T_{c}$, $\sigma$ instead vanishes and $m$, which is now the
common mass of the $\pi$- and $\sigma$-field, is for $d>2$
given by
$$ {1 \over  T_c}- {1 \over  T} = \Omega_d(0)-\Omega_d(m)
  .\eqnd\emasNsig $$
The physical mass $m$ is solution of an equation quite similar to \ecorleng{}. In particular, for $d<4$, we recover  the scaling form \ecorlenb~of the correlation length:
$$\Omega_d(0)-\Omega_d(m)
=K(d)m^{d-2} +O(\Lambda^{d-4}m^2) \ \Rightarrow \ \xi=1/ m\propto (T-T_c)^{-1/(d-2)}.$$
\par
In terms of the cut-off $\Lambda$ and the dimensionless coupling $t$
(Eq.~\eNsigtdef), vertex and correlation functions of the non-linear $\sigma$
model satisfy for $d<4$ RG equations:
$$\eqalign{\left(\Lambda {\partial \over \partial\Lambda}+\beta(t){\partial \over
\partial t}- {n\over 2} \zeta(t)\right)\Gamma^{(n)}(p,\Lambda,t)&=0\,, \cr
\left(\Lambda {\partial \over \partial\Lambda}+\beta(t){\partial \over
\partial t}+{n \over2}\zeta(t)\right)W^{(n)}(p,\Lambda,t)&=0\,. \cr} $$
Applying the second equation with $n=1$ to Eq.~\emgTNsig~and
expressing in Eq.~\emasNsig~that $m$ is RG invariant, we determine the RG
functions at leading order for $N$ large:
$$ \beta (t) = (d-2) t(1-t/t_c)\, ,\qquad
 \zeta (t) =(d-2) t/t_c\quad {\rm with} \ t_c=\Lambda ^{d-2}T_c/N\,. \eqnn  $$
If we add to the action \eactsigla~the contribution due to a magnetic field  in
the $\sigma $ direction,
$${\cal S}(\phib,\lambda) \mapsto {\cal S}(\phib,\lambda) -{H\over T}
\int\d^d x\, \sigma (x),$$
we can calculate the free energy density $W(H)=T\ln {\cal Z}(H)/\Omega $ ($\Omega $ is the volume) and  its Legendre
transform (Eqs.~\eqns{\eNfivmag,\eNfivGam}), the thermodynamic potential density  function of the magnetization $M$:
$$ {\cal G}(M)=N{d-2\over2d}{T^{2/(2-d)}\over\bigl(K(d)\bigr)^{2/(d-2)}}
(M^2/N-1+T/T_c)^{d/(d-2)},\eqnn $$
a result that extends the scaling form \ethermscN~to all temperatures below
$T_c$. The calculation of other physical quantities and the
expansion in powers of $1/N$ follow from the considerations of previous sections
and section \sssfivNRT.
\medskip
{\it Two dimensions and the question of Borel summability.} For $d=2$, the
critical  temperature vanishes and the parameter $m$ has the form
$$m \sim \Lambda_0 \e^{-2\pi / T}, \eqnn $$
where (with the definition \epropreg)
$$\ln(\Lambda _0/\Lambda )={ 1\over4\pi}\int_0^\infty {\d s\over s}\left(
{1\over D(s)}-\theta (1-s)\right), $$
in agreement with RG predictions. Note that the field inverse two-point
function in the large $N$-limit is given by
$$\tilde\Gamma^{(2)}_{\sigma\sigma}(p)=p^2 + m^2\,. \eqnd\eNsigmii $$
The mass term vanishes to all orders in the expansion in powers of
the coupling constant $t$, preventing a perturbative calculation of
the mass of the $\sigma$-field. The perturbation series is
trivially not Borel summable. Most likely, this property remains true for the
model at finite $N$. On the other hand, if the $O(N)$ symmetry is broken by
the addition of a term proportional to $ \int\d x\,\sigma(x)$ to the action (a magnetic field), the
physical mass becomes calculable in perturbation theory.
\medskip
{\it Corrections to scaling and the dimension 4.} In Eq.~\emasNsig\
we have neglected corrections to scaling. If we take into account the leading
correction, which becomes increasingly important when $d$ approaches 4 from below,  we get instead
$$ {1\over T_c}-{1\over  T}=K(d)m^{d-2} -a(d)\Lambda ^{d-4}m^2   +O\left(\Lambda ^{d-6}m^4, \Lambda ^{-2}m^d\right),$$
where $a(d)$, as we have already discussed, is a constant that explicitly
depends on the cut-off procedure and can thus be varied by changing
contributions from irrelevant operators.\par
We can compare this result with the solution of Eqs.~\ecorleng{} expanded
at the same order:
$${6\over u}(r-r_c-m^2)=K(d)m^{d-2} -\tilde a(d)\Lambda ^{d-4}m^2   +O\left(\Lambda ^{d-6}m^4\right),$$
where the constant $\tilde a(d)$ has the same formal expression as $a(d)$ but corresponds to a different regularization. Eliminating the leading contribution, we find
$$ {1\over  T_c}-{1\over  T}-{6\over u}(r-r_c)=\Lambda ^{d-4}m^2
\left(\tilde a(d)-{6\over Ng}-a(d)\right).$$
We note that it is thus possible to find a regularization of the non-linear $\sigma $-model that reproduces the effect of the $\phi^4$ coupling constant.
\par
More generally, by comparing with the results of section \sssEGRN, we discover that,
although the non-linear $\sigma$-model superficially depends on one parameter
less than the corresponding $\phib^4$ field theory, actually this parameter is
hidden in the cut-off function. This remark becomes important in the four
dimensional limit where most leading contributions come from the leading
corrections to scaling. For example, for $d=4$ Eq.~\emasNsig\ takes a
different form, the dominant term in the r.h.s.\ is proportional to
$m^2\ln m$. We recognize in the factor $\ln m$ the effective
$\phi^4$ coupling at mass scale $m$. However,
to describe with perturbation theory and RG the physics
of the non-linear $\sigma$ model beyond the $1/N$ expansion,  it is necessary  to return to the $\phi^4$ field theory. This involves addding to the action
the operator $\int\d^d x\,\lambda^2(x)$, which irrelevant for $d<4$,
becomes marginal in four dimensions.
%

%\def\efigiv{4}
%
\subsection $1/N$-expansion and renormalization group: an alternative formulation

{\it Preliminary remarks. Power counting.}
Higher order terms in the steepest descent calculation of the
functional integral \eZeff\ generate a systematic $ 1/N $ expansion.
%Let us first slightly rewrite the action \eactONef\ by shifting the field
%$\lambda(x)$ by its expectation value $m^2$  (Eq.~\eNsaddpts),
%$\lambda(x) \mapsto m^2+\lambda(x)$:\sslbl\sssfivNRT %\sslbl\ssfivNe
%$$\eqalignno{{\cal S}_N ( \sigma ,\lambda) &  = {1 \over 2}\int \d ^{d}x \left[
%\left(\partial_{\mu}\sigma \right)^2+ m^2\sigma
%^2+ \lambda  \sigma^2  - {3N\over u}\lambda^2  -{6 N\over u} \left(m^2-r\right) %\lambda \right] \cr &\quad +{ (N-1
% ) \over 2}  \tr\ln \left[ -\nabla^2 +m^2+\lambda (\bullet) \right]
%. &\eqnd  \eacteffb  \cr}$$

We now analyze the terms in the action \eacteffb~from the
point of view of large $N$ power counting. The
dimension of the field $ \sigma (x) $ is $ (d-2)/2 $. From the critical
behaviour \eprocrit\ of the $\lambda$-propagator, we inferred the
canonical dimension  $[\lambda]$ of the field $ \lambda (x) $:
$$ 2 \left[ \lambda \right] -\varepsilon =d\,, \qquad {\rm i.e.} \quad \left[
\lambda \right] =2\,. $$
As noted above, $\lambda^2$ has dimension $4>d$ and thus is  irrelevant below  four dimensions.
The interaction term $ \int \lambda(x)\sigma^2 (x)\d^d x $ has
dimension zero. It is easy to verify that the non-local interactions
involving the {$ \lambda$-field},  coming from the expansion of the $ \tr\ln $,
have all also  canonical dimension zero:
$$ \left[ \tr \left[ \lambda (x) \left(-\nabla^2 +m^2 \right)^{-1}
\right]^{k} \right] =k \left[ \lambda \right] -2k=0\,. $$
This power counting  has the following implication: in contrast with
usual perturbation theory,  the $ 1/N $ expansion is exactly renormalizable and thus generates only logarithmic
corrections to the leading long distance behaviour for any fixed dimension
$d$, $ 2<d\leq 4$. A similar behaviour is found in  the $ \varepsilon $-expansion (at the IR fixed point) and, thus, one expects here also to be able to
calculate universal quantities like critical exponents for example as power
series in $ 1/N$. However, because the interactions are non-local, it
is not obvious that the general results of renormalization theory
apply here. Therefore, we now construct
an alternative quasi-local field theory, for which the standard RG
analysis is valid, and which reduces to the large $N$ field theory in
some limit \rANPVN.
\medskip
{\it An alternative field theory.}
To be able to use the standard results of renormalization theory, we
reformulate the critical theory to deal with the non-local
interactions. Neglecting corrections to scaling, we start from the
non-linear $\sigma$-model in the form \eactsigla:
$$ \eqalignno{ {\cal Z} & = \int \left[ \d  \lambda (x)
\right] \left[ \d  \phib (x) \right]
\exp\left[-{\cal S} (\phib ,\lambda )\right] , & \eqnn
\cr {\cal S} ( \phib ,\lambda) & = {1 \over 2T}\int \d^{d}x \left[ \left(
\partial_{\mu}\phib \right)^2 + \lambda \left(\phib^2 -N \right)\right].
  & \eqnd{\eactphla}   \cr} $$
The difficulty arises from the $\lambda$-propagator, absent in the
perturbative formulation, and generated by the large $N$ summation.
We thus add to the action \eactphla\ a term quadratic in $\lambda$
that, in the tree approximation of standard perturbation theory, generates a
$\lambda$-propagator of the form \eprocrit.
We thus consider the modified action
$$ {\cal S}_{\vv}  (  \phib ,\lambda ) = {1 \over 2}\int
\d^{d}x \left\{
{1\over T}\left[\left( \partial_{\mu}\phib \right)^2 + \lambda
\left(\phib^2 -N\right)\right]-
{1\over\vv^2}\lambda(-\partial^2)^{-\varepsilon/2}\lambda\right\}
. \eqnd{\eactSg}$$
In the limit where the parameter $\vv$ goes to infinity, the coefficient of
the additional term vanishes and the initial action is recovered. \par
Only the critical theory is discussed  below, and thus the
couplings of all relevant interactions are set to their critical values.
These interactions contain a term linear in $\lambda$ and a polynomial in
$\phib^2$ of degree depending on the dimension. Note that in some discrete
set of dimensions some monomials become just renormalizable. Therefore, we work
in generic dimensions and rely on the property that the quantities we calculate are regular
functions of the dimension. \par
The field theory with the  action \eactSg~can be studied with
standard field theory methods. The peculiar form of the $\lambda$ quadratic
term, which is not strictly local, does not create a problem. Similar terms
are encountered in statistical systems with long range forces. The
main consequence is
that the $\lambda$-field is not renormalized because counter-terms are
always local.\par
It is convenient to rescale $\phib\mapsto \phib\sqrt{T}$, $\lambda \mapsto
\vv\lambda$:
$$ {\cal S}_{\vv} ( \phib ,\lambda ) = {1 \over 2}\int \d^{d}x \left[
\left( \partial_{\mu}\phib \right)^2 + \vv\lambda\phib^2
- \lambda(-\partial^2)^{-\varepsilon/2}\lambda +{\rm relevant\
terms}\right].$$
The renormalized critical action then reads
$$[{\cal S}_\vv]_{\rm ren} =  {1 \over 2}\int \d^{d}x \left[
Z_\phi \left( \partial_{\mu}\phib \right)^2 +\vv_\r Z_{\vv}\lambda\phib^2
- \lambda(-\partial^2)^{-\varepsilon/2}\lambda +{\rm relevant\
terms}\right] . \eqnd\eactSren $$
It follows that the RG equations for vertex functions of $l$
$\lambda$ fields and $n$ $\phib$ fields in the critical theory take the
form
$$\left[\Lambda {\partial \over \partial \Lambda}+
\beta_{\vv^2}(\vv){\partial \over \partial \vv^2}-{n\over 2}\eta(\vv)\right]
\Gamma^{(l,n)}=0\,.\eqnd \eqRGm $$
The solution to the RG equations \eqRGm\ can be written as
$$\Gamma^{(l,n)}(\ell p, \vv,\Lambda)=Z^{-n/2}(\ell)
\ell^{d-2l-n(d-2)/2} \Gamma^{(l,n)}(p, \vv(\ell) \Lambda)
\eqnd\esolRG $$
with the usual definitions
$$\ell{\d \vv^2\over \d \ell}=\beta(\vv(\ell))\,,\quad \ell{\d \ln Z
\over \d \ell}=\eta(\vv(\ell))\, .$$
We can then calculate the RG functions as power series in $1/N$. It
is easy to verify that $\vv^2$ has to be taken of order $1/N$. Therefore,
to generate a $1/N$ expansion, one first has to sum the multiple insertions of
the one-loop $\lambda $ two-point function, contributions that form a
geometric series. The $\lambda$ propagator then becomes
$$\Delta_\lambda(p)=-{2 p^{4-d}\over b(d) D(\vv)}  \eqnn $$
($b(d)$ is given in Eq.~\econstb), where we have defined
$$D(\vv)=2/b(d)+ N\vv^2.$$
We are interested in the neighbourhood of the fixed point $\vv^2=\infty$.
One verifies that the RG function $\eta(\vv)$ approaches the exponent
$\eta$ obtained by direct calculation, and the RG $\beta$-function
behaves like  $\vv^2$. The flow equation
for large coupling constant  becomes
$$\ell{\d \vv^2\over \d \ell}\sim\rho \vv^2  \ \Rightarrow\ \vv^2(\ell)\sim
\ell^{\rho}.\eqnn $$
We then note that to each power of the field $\lambda$ corresponds a power of
$\vv$. It follows that
$$\eqalignno{\Gamma^{(l,n)}(\ell p,\vv,\Lambda)&\propto
v^l(\ell)\ell^{d-2l-n(d-2+\eta)} & \cr
&\propto\ell^{d-(2-\rho/2)l-n(d-2+\eta)} .& \eqnn \cr}
$$
To compare with the result obtained from the perturbative
RG, one has still to take into account that the functions
$\Gamma^{(l,n)}$ defined here are obtained by an additional Legendre
transformation with respect to the source of $\phib^2$. Therefore,
$$ 2-\rho/2=d_{\phib^2}=d-1/\nu \,. \eqnn $$
\midinsert
\epsfxsize=37.6mm
\epsfysize=18.2mm
\centerline{\epsfbox{triangle.eps}}
\figure{3.mm}{Diagram contributing to $\Gamma^{(3)}_{\sigma\sigma\lambda}$ at
order  $1/N$.}
\figlbl\figNtriangl
\endinsert
\midinsert
\epsfxsize=34mm
\epsfysize=27.2mm
\centerline{\epsfbox{triangii.eps}}
\figure{3.mm}{Diagram contributing to $\Gamma^{(3)}_{\sigma\sigma\lambda}$ at
order  $1/N$.}
\figlbl\figNtrianglii
\endinsert

%\def\efigv{5}
\medskip
{\it RG functions at order $1/N$.}
Most calculations at order $1/N$ rely on the evaluation of the generic
integral
$${1\over(2\pi)^d}\int{\d^d q \over (p+q)^{2\mu}
q^{2\nu}}=p^{d-2\mu-2\nu}{\Gamma(\mu+\nu-d/2)\Gamma(d/2-\mu)\Gamma(d/2-\nu)
\over (4\pi)^{d/2}\Gamma(\mu)\Gamma(\nu)\Gamma(d-\mu-\nu)}\,
. \eqnd\eintmunu $$
For later purpose, it is convenient to set
$$X_1={2 N_d\over b(d)}= {4 \Gamma(d-2)\over
\Gamma(d/2)\Gamma(2-d/2)\Gamma^2(d/2-1)}= %2to4 bug
{4 \sin(\pi\varepsilon/2) \Gamma(2-\varepsilon)\over
\pi\Gamma(1-\varepsilon/2) \Gamma(2-\varepsilon/2)} .\eqnd{\eXone}$$
To compare with fixed dimension results, note that $X_1\sim 2(4-d)$ for $d\to 4$
and $X_1\sim(d-2)$ for $d\to 2$.\par
The calculation of the $\left<\phi\phi\right>$ correlation function
at order $1/N$ involves the evaluation of the diagram
of figure  \figbubiii.
 We want to determine the coefficient of
$p^2\ln\Lambda/ p$. Since we work at one-loop order, we can instead
replace the $\lambda$ propagator $q^{-\varepsilon}$ by $q^{2\nu}$ and send the
cut-off to infinity. We then use the result \eintmunu\ with
$\mu=1$. In the limit $2\nu \to-\varepsilon$, the integral has a
pole. The residue of the pole  yields the coefficient of
$p^2\ln\Lambda$ and the finite part contains the $p^2\ln p$ contribution:
$$\tilde\Gamma^{(2)}_{\sigma\sigma}(p)= p^2 +
{\varepsilon\over 4-\varepsilon} {2 N_d\over b(d) D(\vv)} \vv^2
p^2\ln(\Lambda/p) .$$
Expressing that the function satisfies the RG equation, we obtain the
function $\eta(\vv)$.\par
The second RG function can be deduced from the divergent parts
of the $\left<\phi\phi\lambda\right>$ function:
$$\tilde\Gamma^{(3)}_{\sigma\sigma\lambda}=\vv+A_1 \vv^3 D^{-1}(\vv)\ln\Lambda +A_2
\vv^5 D^{-2}(\vv)\ln \Lambda +\ {\rm finite} $$
with
$$\eqalign{A_1 &=-{2\over b(d) } N_d=-X_1\,, \cr
A_2 &=- {4N\over b^2(d)} (d-3)b(d) N_d=-2N(d-3)X_1\,,
\cr} $$
where  $A_1$ and $A_2$ correspond to the diagrams of figures \figNtriangl~and
\figNtrianglii, respectively. \par
Applying the RG equation, one finds at order $1/N$ the relation
$$\beta_{\vv^2}(\vv)=2\vv^2\eta(\vv)-2A_1 \vv^4 D^{-1}(\vv)-2A_2
\vv^6 D^{-2}(\vv) . \eqnn $$
One thus obtains
$$\eqalignno{\eta(\vv) &={ \varepsilon\vv^2 \over 4-\varepsilon}X_1
D^{-1}(\vv), & \eqnn \cr
\beta_{\vv^2}(\vv) &={8 \vv^4 \over 4-\varepsilon}X_1 D^{-1}(\vv)
+4N(1-\varepsilon) \vv^6 X_1 D^{-2}(\vv), &\eqnn \cr}$$
where the first term in $\beta_{\vv^2}$ comes from $A_1$ and $\eta$
and the second from $A_2$.\par
Extracting the large $\vv^2$ behaviour,  one infers
$$\eqalignno{\eta  & =  { \varepsilon \over N (4-\varepsilon)} X_1 +O(1/N^2)
,& \eqnn \cr
\rho&={4(3-\varepsilon)(2-\varepsilon)\over N(4-\varepsilon)}X_1 > 0\,, \cr}$$
and thus
$${1\over\nu}=d-2 + {2(3-\varepsilon)(2-\varepsilon)\over N(4-\varepsilon)}X_1
+O(1/N^2) .\eqnn $$
%
\subsection Some  higher order results

The calculations beyond the order $1/N$ are rather technical. The reason
is easy to understand: because the effective
field theory is  renormalizable in all dimensions $2\leq d \leq 4$, the
dimensional regularization, which is so helpful in perturbative calculations,
can no longer be used. Therefore, either one keeps a true cut-off or one
introduces more sophisticated regularization schemes. For
details the reader is referred to the literature \refs{\rKTOOVPH{--}\rGraci}.
\medskip
{\it Generic dimensions}. The exponents $\gamma$ and  $\eta$
are known up to order $1/N^2$ and $1/N^3$, respectively, in arbitrary
dimensions but the expressions are too complicated to be reproduced here.
The expansion of $\gamma$ up to order
$1/N$ can be directly deduced from the results of the preceding
sections:
$$ \gamma  =  { 1 \over 1- \varepsilon / 2}\left(1-{3 \over
2N}X_1 \right)+O \left({1 \over N ^2} \right) .\eqnn $$
The exponents $ \omega $ and $\theta=\omega\nu$, governing the leading
corrections to scaling, can also be calculated, for example, from the
$\left<\lambda^2\lambda\lambda\right>$ function:
$$ \eqalignno{\omega & = \varepsilon\left(1-{2(3-\varepsilon)^2 \over
(4-\varepsilon)N}X_1\right) +O \left({1 \over N^2}
\right), & \eqnn \cr
\theta=\omega \nu &={\varepsilon \over 2-\varepsilon}
\left(1-{2(3-\varepsilon) \over N}X_1\right) +O \left({1 \over N^2}
\right). & \eqnn \cr}$$
Note that the exponents are regular functions of $\varepsilon$ up to
$\varepsilon =2$ and free of renormalon singularities at $\varepsilon
=0$.\par
The equation of state and the spin--spin correlation function in zero field
are also known at order $ 1/N $, but since the expressions are
complicated we again refer the reader to the literature for details.
\medskip
{\it Three dimensional results.} Let us give the expansion of $\eta$  in
three dimensions at the order presently available:
$$ \eta  = {\eta_{1} \over N}+{\eta_2 \over N^2}+ {\eta_{3} \over N^{3}}
+O \left({1 \over N^{4}} \right) $$
with
$$\eta_1 ={\textstyle{8 \over 3\pi^2}}\, ,\quad \eta_2= -{\textstyle {8
\over 3} \eta_1^2}\, ,\quad
\eta_{3} = \eta_1^3 {\textstyle\left[ -{797 \over 18} - { 61 \over 24}\pi^2
+ {27 \over 8}\psi''(1 / 2 ) + {9 \over 2}\pi^2 \ln 2
\right]}, $$
$\psi(x)$ being the logarithmic derivative of the $\Gamma$ function.\par
The exponent $\gamma$ is  known only up to order $1/N^2$:
$$\gamma  = 2 -{\textstyle {24 \over N \pi^2}} + {\textstyle{64 \over
N^2\pi^{4}}\left({44 \over 9} - \pi^2 \right)+O \left({1 \over N^{3}}
\right)}. $$
Note that the $1/N$ expansion seems to be rapidly divergent and certainly a
direct summation of these terms does not provide precise estimates
of critical exponents in three dimensions for relevant values of $N$.
%%%%%%%%% Refer to large orders of 1/N expansion
\medskip
{\it The nature of the large $N$ expansion.} The large order
behaviour of the $N$ expansion has been determined explicitly in
zero and one dimension (simple integrals and quantum
mechanics) \rBrHi.
%Large-order behaviour of the 1/N expansion in zero and one dimensions,
In quantum field theory some estimates are available in
ref.~\rVegAv.
% Large order in 1/N \phi^4 and nls
All results seem to indicate that the expansion is  divergent in all
dimensions, but Borel summable for dimensions $d<4$.


%
\subsection Finite size effects: the non-linear $\sigma $ model

Because finite size effects involve crossover phenomena between different effective dimensions, large $N$ techniques provide convenient tools to study them \rBrCaPe. It is difficult to discuss systematically finite size
effects because the results depend both on the geometry of the system and on
 boundary conditions. In particular, one must discuss separately boundary
conditions whether they break or not translation invariance. In
the first case new effects appear, which are surface effects, and that we do
not consider here.
We study here only periodic conditions, although they are not the only ones  preserving translation invariance. For systems that have a symmetry, one can
glue the boundaries after having made a group transformation. Thus,
here one could
also discuss anti-periodic conditions or, more generally, fields differing on both sides by a
transformation of the $O(N)$ group.\sslbl\ssNFSS \par
Even with periodic boundary conditions, the number of different possible situations remains large, the finite sizes in different directions may differ,
some sizes may be infinite. Note  that
QFT at finite temperature, which we begin studying in section \scFTQFT, can
be considered as another example of finite size effects, since the
functional integral representing the partition function of a quantum
system at finite temperature is also the partition function of a
classical system with finite size  and periodic
boundary conditions in one dimension.
\par
>From the point of view of the RG, finite size effects, which
only affect the IR domain, do not change UV divergences. RG
equations remain the same, only the solutions are modified due to
the existence of new dimensional parameters. Thus, if finite sizes
are characterized by only one length $L$, solutions will be
functions of an additional  argument like $L/\xi$ where $\xi$ is
the correlation length \rBZJFSS. \par Since we want only to
demonstrate that large $N$ techniques are useful in the context of
finite size effects, we discuss here the non-linear $\sigma
$-model in a simple geometry with periodic boundary conditions
(the geometry of the hypercube  or better hypertorus of linear
size $L$).\par A characteristic property of a system of finite
size is the quantization of momenta, the arguments of field
Fourier components. For periodic boundary conditions, if  $L$ is
the size of the system in a direction $\mu$, we find
$$p_\mu=2\pi n_\mu /L \,,\quad n_\mu\in {\Bbb Z}\,.$$
In particular, in a massless theory in a finite volume the zero mode
${\bf p}=0$   corresponds to an isolated pole of the propagator.
This automatically leads to IR divergences in all dimensions.
Therefore, in Eqs.~\esaddleN{} the solution $\sigma\ne 0$ no longer
exists. This is not surprising:   no phase  transition is expected in a
finite volume.
 \medskip
{\it Finite size scaling in the non-linear $\sigma$ model.}
The gap equation \emgNsig{b} becomes ($N\gg1$)
$$G(m,L,\Lambda )\equiv L^{-d}\sum_{n_\mu\in{\cal Z}^d}{1\over
m^2+(2\pi {\bf n}/L)^2}={1\over  T }\,,
\eqnd\esaddNL $$
where the sums are cut by a cut-off $\Lambda$.\par
To write the equation  in a more  manageable form and be able
to define it for continuous dimension, one uses Schwinger's
representation
$${1\over p^2+m^2}=\int_0^\infty\d s\,\e^{-s(p^2+m^2)}.$$
However, in contrast with  the infinite volume limit,  gaussian
integrals over momenta are here replaced by  infinite sum over
integers which can no longer be calculated exactly. One thus
introduces the function
$$\vartheta_0(s)= \sum^{+ \infty}_{n=- \infty} \e^{-\pi sn^2},
\eqnd\eJacobi $$
related to Jacobi's elliptic function $\theta_3$ by
$$\vartheta_0(s)=\theta_3(0,\e^{-\pi s}).$$
Poisson's transformation allows to prove the useful identity
$$\vartheta_0(s) = s^{-1/2} \vartheta_0\left(1/s\right).\eqnd \ePoisson $$
In terms of $\vartheta_0(s)$, the sums can then be written as
$$G(m,L,\Lambda) =L^{-d}\int^\infty\d s\,
\e^{-sm^2}\vartheta_0^d(4s\pi/L^2), $$
where UV convergence is here ensured by a small $s$ cut-off of order
$1/\Lambda^2$. \par
For $d>2$, we can introduce the  critical temperature $T_c$, which with
the same regularization reads
$${1\over T_c}={1\over(4\pi)^{d/2}}\int^\infty s^{-d/2}\d s\,.$$
It follows from the identity \ePoisson~that the difference between
both integrals is UV convergent for $d<4$. After
a change of variables $4\pi s/L^2 \mapsto s$ (and keeping only the leading
contribution in the critical domain), one finds that
the gap equation can be written as
$$L^{d-2}\left({1\over T}-{1\over T_c}\right)= F(mL)  \eqnd\eFSsigN  $$
with
$$F(z)={1\over 4\pi } \int_0^\infty\d s\,
\left(\e^{-s z^2 /(4\pi)}\vartheta_0^d(s)-s^{-d/2}\right). \eqnn $$
For $|T-T_c|\ll \Lambda^{d-2}$, we thus find a scaling form
which is consistent with  RG predictions:
$$L m(T,L)\equiv L/\xi(T,L)=f\bigl(L/\xi(T,L=\infty)\bigr) , $$
where in addition $f$ is a regular function of $T$ at $T_c$.
More precisely, in the large $N$ limit, using Eq.~\emasNsig~we obtain
$$K(d)\bigl[Lm(T,L=\infty)\bigr]^{d-2}=-F\bigl[L m(T,L)\bigr],$$
and we recall that $d-2=1/\nu+O(1/N)$.
Note that the length $\xi$ has the meaning of a correlation length only for
$\xi<L$. Since $\eta=0$ at this order, $m$ is also directly related to
the magnetic susceptibility $\chi$ in zero field, $\chi=1/m^2$. \par
The function $F(z)$ is decreasing. For $z\to\infty$ the integral is
dominated by the small $s$ region, which corresponds to the infinite
volume limit,
$$F(z)\sim \Gamma(1-d/2) {z^{d-2}\over(4\pi)^{d/2}}=-K(d)z^{d-2} .$$
One then verifies that for $T>T_c$ fixed, $L\to\infty$ and thus for
$mL\to\infty$, one
recovers the infinite volume limit. Alternatively, in the low
temperature phase for $T<T_c$ fixed, $L\to\infty$, $mL$ goes to zero.
Thus, the contribution of the zero mode ${\bf n}=0$ dominates  the
sum in equation
\esaddNL. Using the relation \ePoisson, one then finds
$$\eqalign{F(z)&={1\over z^2}+K_1(d)+O\left(z^2\right), \cr
K_1(d)&={1\over 4\pi } \int_0^\infty\d s\,
\left(\vartheta_0^d(s)-s^{-d/2}-1\right)
, \cr}$$
and thus
$$\chi(L,T)/N={1\over m^2}= \left({1\over T}-{1\over T_c} \right)L^d-  L^2
K_1(d)+O\left(L^{4-d}/(T-T_c)\right)   . \eqnn $$
We see that the susceptibility diverges with the volume, a precursor of
the low temperature phase with broken symmetry.\par
Note, finally, that it is instructive to make a similar analysis
for other boundary conditions that have no zero mode.
\par
For $d=2$, the regime where finite size effects can be seen
corresponds to $T\ln(L\Lambda)=O(1)$, that is to a regime of low
temperature. The zero mode dominates for $T\ln(L\Lambda)\ll 1$, and the
susceptibility then is given by
$$\chi(T,L)\sim  L^2\left[1+O(T\ln(L\Lambda))\right] \,.$$
%
\subsection The $CP(N-1)$ models in the large $N$ limit

We discuss here only one other  family of models based on
symmetric spaces that can be solved in the large $N$ limit, the $CP(N-1)$ models,
because they have been the subject of many studies \rCPN.
In particular, one can show that these models in two dimensions
have instanton solutions.
\par
Again one discovers that, within the framework of the large $N$
expansion,   $CP(N-1)$ models are related to  other models, abelian Higgs models, which are renormalizable in four dimensions.
\def\sigmab{\sigma}
\def\varphib{\varphi}
\smallskip
{\it The models.} The field $\varphi_\alpha $ is an $N$-component complex vector of unit length:
$$ \bar \varphi \cdot \varphi=N\,. \eqnd \ezzbar  $$
In addition two vectors $ \varphi_{\alpha} $ and $ \varphi'_{\alpha} $ are
equivalent if
$$ \varphi'_{\alpha} (x)=\e^{i\Lambda(x)}\varphi_{\alpha}(x)\,.
\eqnd\egaugequi $$
These conditions characterize the symmetric space  $U(N)/[U(N-1) \times U(1)] $,
a  complex Grassmannian manifold,
 which is isomorphic to the complex projective manifold  $CP(N-1)$.\par
%%%%%%%%%%%%%%%%%%%%%%%%%%%%%%%%
One form of the unique symmetric classical
action is
$$ {\cal S}(\varphib, A_\mu) ={1 \over T} \int \d^2x\,
\overline{ {\rm D}_{\mu}\varphib}\cdot {\rm D}_{\mu}\varphib\,, \eqnn $$
in which $T$ is a coupling constant and $ {\rm D}_{\mu}$   the covariant derivative:
$$ {\rm D}_{\mu}=\partial_{\mu}+i A_\mu\,. \eqnn $$
The field $A_\mu$ is a gauge field for the $U(1)$ transformations
$$ \varphib'(x)=\e^{i\Lambda(x)}\varphib (x)\,, \quad A'_\mu(x)=A_\mu(x)-\partial _\mu \Lambda (x).\eqnd{\egaugequi} $$
The $U(N)$ symmetry of the action is obvious  and the gauge symmetry
implements the equivalence  \egaugequi. \par
In the tree approximation the fields are massless, and the $2N-2$ independent
real components correspond to the Goldstone bosons of the broken symmetry
$U(N)\to U(N-1)$, one Goldstone boson being suppressed by the abelian
gauge symmetry (the Higgs mechanism).\par
Since the action contains no kinetic term for   $A_\mu$, the gauge field
is not a dynamical field but only an auxiliary field
that can be integrated out. The action is quadratic in $A$ and the gaussian  integration  results in  replacing in the action $A_\mu$ by the solution of the $A $-field equation
$$NA_\mu=\ud i\left(\overline{  \varphi}\cdot \partial_\mu  \varphi-
\overline{\partial_\mu  \varphi}\cdot \varphi \right) =i\bar
\varphib \cdot \partial_{\mu}\varphib\,,\eqnd\eCPNAphi $$
where Eq.~\ezzbar\ has been used.
After this substitution, the  composite field $\bar \varphib \cdot \partial_{\mu}\varphib/N$ acts as a gauge field.
In the following, however, we find it more convenient to keep $A_\mu$ as an independent  field.\par
%%%%%%%%%%%%%%%%%%%%%%%%%%%%%%%
Note that the $CP(1)$ model is locally isomorphic to the $ O (3)$ non-linear
$ \sigma$-model, with the identification
$$ \phi^i=\bar \varphi_{\alpha} \sigma^i_{\alpha \beta}\varphi_{\beta}\,. \eqnn
$$
\medskip
{\it Large $N$ limit.}
As for the non-linear $\sigma $-model, we   introduce a Lagrange multiplier $\lambda (x) $ to implement the constraint \ezzbar, and obtain the action
$$ {\cal S}(\varphi,A_\mu ,\lambda )={1 \over T} \int \d^{d}x\left[
\overline{ {\rm D}_{\mu}\varphi}\cdot {\rm D}_{\mu}\varphi+ \lambda\left( \bar\varphi\cdot \varphi-N\right)\right] . \eqnn $$
The integral over $\varphi$ now is gaussian and can  be performed. Integrating over $N-1$ components, one finds (with $\varphi\equiv \varphi_1$),
$$ {\cal S}_N(\varphi,A_\mu ,\lambda )={1 \over T} \int \d^{d}x\left[
| {\rm D}_{\mu}\varphi|^2+ \lambda\left(  | \varphi|^2
-N\right)\right] +(N-1)\tr\ln(-{\rm D}_\mu^2+\lambda ). \eqnn $$
Of course, the remaining functional integral over $\varphi,A_\mu
,\lambda$ is well-defined only after a choice of gauge.\par The
gauge field plays no role in the saddle point equations, which are
those of the $O(2N)$ non-linear $\sigma $-model. However, the
model embodies the physics of the abelian Higgs model and of the
Landau--Ginzburg theory of superconductivity. In dimensions $d>2$,
in the broken symmetry phase, the gauge field becomes massive and
the number of Goldstone modes is $2N-2$ instead of $2N-1$. In two
dimensions a phase with massless modes is excluded; the model
exhibits a symmetric phase with confinement because the Coulomb
force is linear.
 \medskip
{\it The abelian Higgs model.} The action of the abelian Higgs model can be written as
$${\cal S}(A_\mu, \varphi)=\int\ \d^{d } x\left[{N\over 4
e^2}F_{\mu\nu}^2+  \overline{{\rm D}_\mu\varphi}\cdot{\rm D}_\mu\varphi + NU(\bar\varphi \cdot \varphi/N)\right], \eqnn $$
where the  potential  is quadratic:
$$U(z)=rz +\frac{1}{6} u z^2. \eqnd\eFTUabHig $$
This model with $N$ charged scalars is renormalizable in $d=4$ dimensions. In dimension $d=4-\varepsilon$, the RG $\beta $-functions are
$$\eqalign{\beta_u&=-\varepsilon u+{1\over 24\pi^2 N}\left[(N+4)u^2-18ue^2+54 e^4\right] ,\cr
\beta_{e^2}&=- \varepsilon e^2+ {1\over 24\pi^2} e^4.\cr} $$
For $d=4$, the origin $e^2=g=0$ is a stable IR fixed point only for $N\ge N_c=90+24\sqrt{15}\approx 183$. Correspondingly, the model has a stable IR fixed point in dimension $4-\varepsilon$  for $N\ge N_c= N_c(d=4)+O(\varepsilon)$. This is of course the situation which prevails in the large $N$ limit, and one finds the IR fixed point
$$u^*= e^2{}^*=24\pi^2\varepsilon+O(\varepsilon^2) .$$
We thus expect both effective couplings $g$ and $e$ to run at low momentum to the IR fixed point and the model to depend only on one parameter.
As in the case of the
non-linear $\sigma $ model, the linear $|\varphi|^4$ theory and the
non-linear $CP(N-1)$ model are equivalent. \par
This result can be verified by the large $N$ techniques.
We introduce the two fields $\rho (x)$ and $\lambda (x)$  as in section \ssNbosgen, to implement the constraint $\rho (x)=\bar\varphi (x)\cdot \varphi(x)/N$. We then obtain an action of the form
$$ {\cal S}(\varphi,A_\mu ,\lambda,\rho  )=  \int \d^{d}x\left[{N\over 4
e^2}F_{\mu\nu}^2+
\overline{ {\rm D}_{\mu}\varphi}\cdot {\rm D}_{\mu}\varphi+ \lambda\left( \bar\varphi\cdot \varphi-N\rho \right)+NU(\rho )\right] . \eqnn $$
We again integrate over $N-1$ components of the complex field and find ($\varphi\equiv \varphi_1$)
$$\eqalignno{ {\cal S}_N(\varphi,A_\mu ,\lambda,\rho  )&= \int \d^{d}x\left[{N\over 4
e^2}F_{\mu\nu}^2+ | {\rm D}_{\mu}\varphi|^2+ \lambda\left(  |
\varphi|^2 -N\rho \right)+NU(\rho )\right] \cr &+(N-1)\tr\ln(-{\rm
D}_\mu^2+\lambda ). &\eqnn\cr} $$ For the scalar field, the
arguments that show that after translation of the expectation
value the four-point interaction is negligible are the same as for
the usual $\phi^4$ theory. In the symmetric phase the expansion of
the $\tr\ln$ yields an additional contribution to the gauge field
propagator of the form
$$\Delta _{\mu\nu}(p)=\ud (p^2\delta _{\mu\nu}-p_\mu p_\nu)B_\Lambda (p,0).$$
We  face the same situation as in section \ssLTsN. For all dimensions $d<4$, $B_\Lambda (p,0)$ behaves for $p$ small like $p^{d-4}$ and, therefore, the contribution to the propagator coming from $F_{\mu\nu}^2$ is negligible. At $d=4$, it is subleading by $1/\ln p$. Therefore, $1/e^2$ is the coefficient of an irrelevant contribution, which can be omitted in the action. The gauge field
then is no longer dynamical and we return to the $CP(N_1)$ model.
%%%%%%%%%%%%%%%%%%%%%%%%%%%%%%%%%%%%%%%%%%%%%%%
