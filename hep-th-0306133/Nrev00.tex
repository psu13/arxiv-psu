%%%%%%%%%   Updated June 10 2003


\input hyperbasics


\input epsf

%%%%%%%%%%%%%%%%%%%%%%%%%%%%%%%%%%%%%%%%%%%%%%%%%%%%%%%%%%%%%%%%%%%%%%%
% ************************* SEVERAL MACROS ****************************
%%%%%%%%%%%%%%%%%%%%%%%%%%%%%%%%%%%%%%%%%%%%%%%%%%%%%%%%%%%%%%%%%%%%%%%
\catcode`\@=11
%%% saclay A4 paper:
\def\unredoffs{\voffset=13mm \hoffset=6.5truemm}
\def\redoffs{\voffset=-12.truemm\hoffset=-3truemm}
\def\speclscape{\special{landscape}}
%----------------------------------------------------------%
\newbox\leftpage \newdimen\fullhsize \newdimen\hstitle \newdimen\hsbody
\newdimen\hdim
%\tolerance=1000
\hfuzz=1pt
%\def\fontflag{cm}
%
\ifx\hyperdef\UNd@FiNeD\def\hyperdef#1#2#3#4{#4}\def\hyperref#1#2#3#4{#4}\fi
\def\newans{y }
%\message{ new or old (y/n)? }\read-1 to\answb
\def\answb{y }
\ifx\answb\newans\message{(This uses normal fonts.)}%
\def\bigans{b }
%\message{ big or little (b/l)? }\read-1 to\answ
\def\answ{b }
\ifx\answ\bigans\message{(Format simple colonne 12pts.}
\magnification=1200 \unredoffs\hsize=147truemm\vsize=219truemm
\hsbody=\hsize \hstitle=\hsize %take default values for unreduced format
%
\else\message{(Format double colonne, 10pts.} \let\l@r=L
\magnification=1000 \vsize=182.5truemm
\redoffs%
\hstitle=122.5truemm\hsbody=122.5truemm\fullhsize=258truemm\hsize=\hsbody
%
\output={
  \almostshipout{\leftline{\vbox{\makeheadline\pagebody\makefootline}}}
\advancepageno%
}
\def\almostshipout#1{\if L\l@r \count1=1 \message{[\the\count0.\the\count1]}
      \global\setbox\leftpage=#1 \global\let\l@r=R
 \else \count1=2
  \shipout\vbox{\speclscape{\hsize\fullhsize}%\makeheadline}
      \hbox to\fullhsize{\box\leftpage\hfil#1}}  \global\let\l@r=L\fi}
\fi
% ****************************************************************************
% fonts, Dirac slash
%%%%%%%%%%%%%%%%%  lfont  %%%%%%%%%%
\def\hskipb#1#2{\hdim=#1#2 \multiply \hdim by 8 \divide \hdim by 10 %
\hskip\hdim}
\def\vskipb#1#2{\hdim=#1#2 \multiply \hdim by 8 \divide \hdim by 10 %
\hskip\hdim}
\def\sla#1{\mkern-1.5mu\raise0.4pt\hbox{$\not$}\mkern1.2mu #1\mkern 0.7mu}
\def\Dbar{\mkern-1.5mu\raise0.4pt\hbox{$\not$}\mkern-.1mu {\rm D}\mkern.1mu}
\def\Abar{\mkern1.mu\raise0.4pt\hbox{$\not$}\mkern-1.3mu A\mkern.1mu}
\def\Bbar{\mkern-0.mu\raise0.4pt\hbox{$\not$}\mkern-.3mu B\mkern 0.6mu}
\newskip\tableskipamount \tableskipamount=8pt plus 3pt minus 3pt
\def\tableskip{\vskip\tableskipamount}
%****************************
\font\elevrm=cmr9 \font\elevit=cmti9 \font\subrm=cmr7
\newdimen\chapskip
\font\twbf=cmssbx10 scaled 1200 \font\ssbx=cmssbx10
\font\twbi=cmmib10 scaled 1200 \font\caprm=cmr9 \font\capit=cmti9
\font\capbf=cmbx9 \font\capsl=cmsl9 \font\capmi=cmmi9
\font\capex=cmex9 \font\capsy=cmsy9 \chapskip=17.5mm
\def\makeheadline{\vbox to 0pt{\vskip-22.5pt
\line{\vbox to8.5pt{}\the\headline}\vss}\nointerlineskip}
%***************************************************
\font\tbfi=cmmib10  %obsolete??
\font\tenbi=cmmib10 \font\ninebi=cmmib9 \font\sevenbi=cmmib7
\font\fivebi=cmmib5 \textfont4=\tenbi \scriptfont4=\sevenbi
\scriptscriptfont4=\fivebi \font\headrm=cmr10 \font\headit=cmti10
\font\twmi=cmmi10 scaled 1200
%****************************
\font\sixrm=cmr6 \font\fiverm=cmr5 \font\sixmi=cmmi6
\font\fivemi=cmmi5 \font\sixsy=cmsy6 \font\fivesy=cmsy5
\font\sixbf=cmbx6 \font\fivebf=cmbx5 \skewchar\capmi='177
\skewchar\sixmi='177 \skewchar\fivemi='177 \skewchar\capsy='60
\skewchar\sixsy='60 \skewchar\fivesy='60

\def\elevenpoint{
\textfont0=\caprm \scriptfont0=\sixrm \scriptscriptfont0=\fiverm
\def\rm{\fam0\caprm}
\textfont1=\capmi \scriptfont1=\sixmi \scriptscriptfont1=\fivemi
\textfont2=\capsy \scriptfont2=\sixsy \scriptscriptfont2=\fivesy
\textfont3=\capex \scriptfont3=\capex \scriptscriptfont3=\capex
\textfont\itfam=\capit \def\it{\fam\itfam\capit} % \it is family 4
\textfont\slfam=\capsl  \def\sl{\fam\slfam\capsl} % \sl is family 5
\textfont\bffam=\capbf \scriptfont\bffam=\sixbf
\scriptscriptfont\bffam=\fivebf
\def\bf{\fam\bffam\capbf} % \bf is family 6
\textfont4=\ninebi \scriptfont4=\sevenbi
\scriptscriptfont4=\fivebi \abovedisplayskip=11pt plus 3pt minus
8pt \belowdisplayskip=\abovedisplayskip
\smallskipamount=2.7pt plus 1pt minus 1pt
\medskipamount=5.4pt plus 2pt minus 2pt
\bigskipamount=10.8pt plus 3.6pt minus 3.6pt
\normalbaselineskip=11pt \setbox\strutbox=\hbox{\vrule height7.8pt
depth3.2pt width0pt} \normalbaselines \rm}

% ****************************************************************************
%       *****     MSSYMB.TeX    *****              20 Sept 91
%   This file contains the definitions for the symbols in the two
%   "extra symbols" fonts created at the American Math. Society.
%       The old fonts msxm et msym have been replaced by msam et msbm.

\catcode`\@=11

\font\tenmsa=msam10 \font\sevenmsa=msam7 \font\fivemsa=msam5
\font\tenmsb=msbm10 \font\sevenmsb=msbm7 \font\fivemsb=msbm5
\newfam\msafam
\newfam\msbfam
\textfont\msafam=\tenmsa  \scriptfont\msafam=\sevenmsa
  \scriptscriptfont\msafam=\fivemsa
\textfont\msbfam=\tenmsb  \scriptfont\msbfam=\sevenmsb
  \scriptscriptfont\msbfam=\fivemsb

\def\hexnumber@#1{\ifcase#1 0\or1\or2\or3\or4\or5\or6\or7\or8\or9\or
    A\or B\or C\or D\or E\or F\fi }

%  The following 13 lines establish the use of the Euler Fraktur font.
%  To use this font, remove % from beginning of these lines.
\font\teneuf=eufm10 \font\seveneuf=eufm7 \font\fiveeuf=eufm5
\newfam\euffam
\textfont\euffam=\teneuf \scriptfont\euffam=\seveneuf
\scriptscriptfont\euffam=\fiveeuf
\def\frak{\ifmmode\let\next\frak@\else
 \def\next{\Err@{Use \string\frak\space only in math mode}}\fi\next}
\def\goth{\ifmmode\let\next\frak@\else
 \def\next{\Err@{Use \string\goth\space only in math mode}}\fi\next}
\def\frak@#1{{\frak@@{#1}}}
\def\frak@@#1{\fam\euffam#1}
%  End definition of Euler Fraktur font.

\edef\msa@{\hexnumber@\msafam} \edef\msb@{\hexnumber@\msbfam}

\mathchardef\boxdot="2\msa@00 \mathchardef\boxplus="2\msa@01
\mathchardef\boxtimes="2\msa@02 \mathchardef\square="0\msa@03
\mathchardef\blacksquare="0\msa@04
\mathchardef\centerdot="2\msa@05 \mathchardef\lozenge="0\msa@06
\mathchardef\blacklozenge="0\msa@07
\mathchardef\circlearrowright="3\msa@08
\mathchardef\circlearrowleft="3\msa@09
\mathchardef\rightleftharpoons="3\msa@0A
\mathchardef\leftrightharpoons="3\msa@0B
\mathchardef\boxminus="2\msa@0C \mathchardef\Vdash="3\msa@0D
\mathchardef\Vvdash="3\msa@0E \mathchardef\vDash="3\msa@0F
\mathchardef\twoheadrightarrow="3\msa@10
\mathchardef\twoheadleftarrow="3\msa@11
\mathchardef\leftleftarrows="3\msa@12
\mathchardef\rightrightarrows="3\msa@13
\mathchardef\upuparrows="3\msa@14
\mathchardef\downdownarrows="3\msa@15
\mathchardef\upharpoonright="3\msa@16
\let\restriction=\upharpoonright
\mathchardef\downharpoonright="3\msa@17
\mathchardef\upharpoonleft="3\msa@18
\mathchardef\downharpoonleft="3\msa@19
\mathchardef\rightarrowtail="3\msa@1A
\mathchardef\leftarrowtail="3\msa@1B
\mathchardef\leftrightarrows="3\msa@1C
\mathchardef\rightleftarrows="3\msa@1D \mathchardef\Lsh="3\msa@1E
\mathchardef\Rsh="3\msa@1F \mathchardef\rightsquigarrow="3\msa@20
\mathchardef\leftrightsquigarrow="3\msa@21
\mathchardef\looparrowleft="3\msa@22
\mathchardef\looparrowright="3\msa@23
\mathchardef\circeq="3\msa@24 \mathchardef\succsim="3\msa@25
\mathchardef\gtrsim="3\msa@26 \mathchardef\gtrapprox="3\msa@27
\mathchardef\multimap="3\msa@28 \mathchardef\therefore="3\msa@29
\mathchardef\because="3\msa@2A \mathchardef\doteqdot="3\msa@2B
\let\Doteq=\doteqdot
\mathchardef\triangleq="3\msa@2C \mathchardef\precsim="3\msa@2D
\mathchardef\lesssim="3\msa@2E \mathchardef\lessapprox="3\msa@2F
\mathchardef\eqslantless="3\msa@30
\mathchardef\eqslantgtr="3\msa@31
\mathchardef\curlyeqprec="3\msa@32
\mathchardef\curlyeqsucc="3\msa@33
\mathchardef\preccurlyeq="3\msa@34 \mathchardef\leqq="3\msa@35
\mathchardef\leqslant="3\msa@36 \mathchardef\lessgtr="3\msa@37
\mathchardef\backprime="0\msa@38
\mathchardef\risingdotseq="3\msa@3A
\mathchardef\fallingdotseq="3\msa@3B
\mathchardef\succcurlyeq="3\msa@3C \mathchardef\geqq="3\msa@3D
\mathchardef\geqslant="3\msa@3E \mathchardef\gtrless="3\msa@3F
\mathchardef\sqsubset="3\msa@40 \mathchardef\sqsupset="3\msa@41
\mathchardef\vartriangleright="3\msa@42
\mathchardef\vartriangleleft="3\msa@43
\mathchardef\trianglerighteq="3\msa@44
\mathchardef\trianglelefteq="3\msa@45
\mathchardef\bigstar="0\msa@46 \mathchardef\between="3\msa@47
\mathchardef\blacktriangledown="0\msa@48
\mathchardef\blacktriangleright="3\msa@49
\mathchardef\blacktriangleleft="3\msa@4A
\mathchardef\vartriangle="0\msa@4D
\mathchardef\blacktriangle="0\msa@4E
\mathchardef\triangledown="0\msa@4F \mathchardef\eqcirc="3\msa@50
\mathchardef\lesseqgtr="3\msa@51 \mathchardef\gtreqless="3\msa@52
\mathchardef\lesseqqgtr="3\msa@53
\mathchardef\gtreqqless="3\msa@54
\mathchardef\Rrightarrow="3\msa@56
\mathchardef\Lleftarrow="3\msa@57 \mathchardef\veebar="2\msa@59
\mathchardef\barwedge="2\msa@5A
\mathchardef\doublebarwedge="2\msa@5B \mathchardef\angle="0\msa@5C
\mathchardef\measuredangle="0\msa@5D
\mathchardef\sphericalangle="0\msa@5E
\mathchardef\varpropto="3\msa@5F \mathchardef\smallsmile="3\msa@60
\mathchardef\smallfrown="3\msa@61 \mathchardef\Subset="3\msa@62
\mathchardef\Supset="3\msa@63 \mathchardef\Cup="2\msa@64
\let\doublecup=\Cup
\mathchardef\Cap="2\msa@65
\let\doublecap=\Cap
\mathchardef\curlywedge="2\msa@66 \mathchardef\curlyvee="2\msa@67
\mathchardef\leftthreetimes="2\msa@68
\mathchardef\rightthreetimes="2\msa@69
\mathchardef\subseteqq="3\msa@6A \mathchardef\supseteqq="3\msa@6B
\mathchardef\bumpeq="3\msa@6C \mathchardef\Bumpeq="3\msa@6D
\mathchardef\lll="3\msa@6E
\let\llless=\lll
\mathchardef\ggg="3\msa@6F
\let\gggtr=\ggg
\mathchardef\circledS="0\msa@73 \mathchardef\pitchfork="3\msa@74
\mathchardef\dotplus="2\msa@75 \mathchardef\backsim="3\msa@76
\mathchardef\backsimeq="3\msa@77 \mathchardef\complement="0\msa@7B
\mathchardef\intercal="2\msa@7C \mathchardef\circledcirc="2\msa@7D
\mathchardef\circledast="2\msa@7E
\mathchardef\circleddash="2\msa@7F
\def\ulcorner{\delimiter"4\msa@70\msa@70 }
\def\urcorner{\delimiter"5\msa@71\msa@71 }
\def\llcorner{\delimiter"4\msa@78\msa@78 }
\def\lrcorner{\delimiter"5\msa@79\msa@79 }
\def\yen{\mathhexbox\msa@55 }
\def\checkmark{\mathhexbox\msa@58 }
\def\circledR{\mathhexbox\msa@72 }
\def\maltese{\mathhexbox\msa@7A }
\mathchardef\lvertneqq="3\msb@00 \mathchardef\gvertneqq="3\msb@01
\mathchardef\nleq="3\msb@02 \mathchardef\ngeq="3\msb@03
\mathchardef\nless="3\msb@04 \mathchardef\ngtr="3\msb@05
\mathchardef\nprec="3\msb@06 \mathchardef\nsucc="3\msb@07
\mathchardef\lneqq="3\msb@08 \mathchardef\gneqq="3\msb@09
\mathchardef\nleqslant="3\msb@0A \mathchardef\ngeqslant="3\msb@0B
\mathchardef\lneq="3\msb@0C \mathchardef\gneq="3\msb@0D
\mathchardef\npreceq="3\msb@0E \mathchardef\nsucceq="3\msb@0F
\mathchardef\precnsim="3\msb@10 \mathchardef\succnsim="3\msb@11
\mathchardef\lnsim="3\msb@12 \mathchardef\gnsim="3\msb@13
\mathchardef\nleqq="3\msb@14 \mathchardef\ngeqq="3\msb@15
\mathchardef\precneqq="3\msb@16 \mathchardef\succneqq="3\msb@17
\mathchardef\precnapprox="3\msb@18
\mathchardef\succnapprox="3\msb@19 \mathchardef\lnapprox="3\msb@1A
\mathchardef\gnapprox="3\msb@1B \mathchardef\nsim="3\msb@1C
%\mathchardef\napprox="3\msb@1D
\mathchardef\ncong="3\msb@1D
\def\napprox{\not\approx}
\mathchardef\varsubsetneq="3\msb@20
\mathchardef\varsupsetneq="3\msb@21
\mathchardef\nsubseteqq="3\msb@22
\mathchardef\nsupseteqq="3\msb@23
\mathchardef\subsetneqq="3\msb@24
\mathchardef\supsetneqq="3\msb@25
\mathchardef\varsubsetneqq="3\msb@26
\mathchardef\varsupsetneqq="3\msb@27
\mathchardef\subsetneq="3\msb@28 \mathchardef\supsetneq="3\msb@29
\mathchardef\nsubseteq="3\msb@2A \mathchardef\nsupseteq="3\msb@2B
\mathchardef\nparallel="3\msb@2C \mathchardef\nmid="3\msb@2D
\mathchardef\nshortmid="3\msb@2E
\mathchardef\nshortparallel="3\msb@2F
\mathchardef\nvdash="3\msb@30 \mathchardef\nVdash="3\msb@31
\mathchardef\nvDash="3\msb@32 \mathchardef\nVDash="3\msb@33
\mathchardef\ntrianglerighteq="3\msb@34
\mathchardef\ntrianglelefteq="3\msb@35
\mathchardef\ntriangleleft="3\msb@36
\mathchardef\ntriangleright="3\msb@37
\mathchardef\nleftarrow="3\msb@38
\mathchardef\nrightarrow="3\msb@39
\mathchardef\nLeftarrow="3\msb@3A
\mathchardef\nRightarrow="3\msb@3B
\mathchardef\nLeftrightarrow="3\msb@3C
\mathchardef\nleftrightarrow="3\msb@3D
\mathchardef\divideontimes="2\msb@3E
\mathchardef\varnothing="0\msb@3F \mathchardef\nexists="0\msb@40
\mathchardef\mho="0\msb@66 \mathchardef\eth="0\msb@67
\mathchardef\eqsim="3\msb@68 \mathchardef\beth="0\msb@69
\mathchardef\gimel="0\msb@6A \mathchardef\daleth="0\msb@6B
\mathchardef\lessdot="3\msb@6C \mathchardef\gtrdot="3\msb@6D
\mathchardef\ltimes="2\msb@6E \mathchardef\rtimes="2\msb@6F
\mathchardef\shortmid="3\msb@70
\mathchardef\shortparallel="3\msb@71
\mathchardef\smallsetminus="2\msb@72
\mathchardef\thicksim="3\msb@73 \mathchardef\thickapprox="3\msb@74
\mathchardef\approxeq="3\msb@75 \mathchardef\succapprox="3\msb@76
\mathchardef\precapprox="3\msb@77
\mathchardef\curvearrowleft="3\msb@78
\mathchardef\curvearrowright="3\msb@79
\mathchardef\digamma="0\msb@7A \mathchardef\varkappa="0\msb@7B
\mathchardef\hslash="0\msb@7D \mathchardef\hbar="0\msb@7E
\mathchardef\backepsilon="3\msb@7F
\def\Bbb{\ifmmode\let\next\Bbb@\else
 \def\next{\errmessage{Use \string\Bbb\space only in math mode}}\fi\next}
\def\Bbb@#1{{\Bbb@@{#1}}}
\def\Bbb@@#1{\fam\msbfam#1}
\font\sacfont=eufm10 scaled 1440 \catcode`\@=12
%%%%%%%%%%%%%%  end lfont  %%%%%%%%%%%%%%%%%%%%%%%%%%%%%
\def\sla#1{\mkern-1.5mu\raise0.4pt\hbox{$\not$}\mkern1.2mu #1\mkern 0.7mu}
\def\Dbar{\mkern-1.5mu\raise0.4pt\hbox{$\not$}\mkern-.1mu {\rm D}\mkern.1mu}
\def\Abar{\mkern1.mu\raise0.4pt\hbox{$\not$}\mkern-1.3mu A\mkern.1mu}
\nopagenumbers
\headline={\ifnum\pageno=1\hfill\else\draftdate\hfil{\headrm\folio}%
\hfil\hphantom{\draftdate}\fi} \else\message{(This uses pseudo
12pts fonts.} \hoffset=8mm \voffset=16mm
\input lfont12 %pour sun
\def\hskipb#1#2{\hskip#1#2}
\def\sla#1{\mkern-1.5mu\raise0.5pt\hbox{$\not$}\mkern1.2mu #1\mkern 0.7mu}
\def\Dbar{\mkern-1.5mu\raise0.5pt\hbox{$\not$}\mkern-.1mu {\rm D}\mkern.1mu}
\def\Abar{\mkern1.mu\raise0.5pt\hbox{$\not$}\mkern-1.3mu A\mkern.1mu}
\fi

%************* end ouptut macros  *********************

% ****************************************************************************
\newcount\yearltd\yearltd=\year\advance\yearltd by -2000
\newif\ifdraftmode
\draftmodefalse
\def\draft{\draftmodetrue{\count255=\time\divide\count255 by 60
\xdef\hourmin{\number\count255}
  \multiply\count255 by-60\advance\count255 by\time
  \xdef\hourmin{\hourmin:\ifnum\count255<10 0\fi\the\count255}}}
\def\draftdate{\ifdraftmode{\headrm\quad (\jobname,\ le
\number\day/\number\month/\number\yearltd\ \ \hourmin)}\else{}\fi}
\newif\iffrancmode
\francmodefalse
% ********* A few math symbols
\def\e{\mathop{\rm e}\nolimits}
\def\sgn{\mathop{\rm sgn}\nolimits}
\def\Im{\mathop{\rm Im}\nolimits}
\def\Re{\mathop{\rm Re}\nolimits}
\def\d{{\rm d}}
\def\ud{{\textstyle{1\over 2}}}
\def\half{\ud}
\def\tr{\mathop{\rm tr}\nolimits}
\def\det{\mathop{\rm det}\nolimits}
\def\del{\partial}
\def\dd{\d^d\hskip-1pt}
\def\ddi{\d^{d-1}\hskip-1pt}
\def\amp{{\rm amp.}}
\def\as{{\rm ,as.}}
\def\div{{\rm div.}}

\def\lqs{\lq\lq}
\chardef\sigmat=27
\def\n{\noindent}
\def\rf{\par\item{}}
\def\nrf{\par\n}
\def\frac#1#2{{\textstyle{#1\over#2}}}

\def\today{\number\day/\number\month/\number\year}
\def\leaderfill{\leaders\hbox to 1em{\hss.\hss}\hfill}
% *************************************************************************
\catcode`\@=11
% ************** double alignment in eqalignno style **********************
\def\deqalignno#1{\displ@y\tabskip\centering \halign to
\displaywidth{\hfil$\displaystyle{##}$\tabskip0pt&$\displaystyle{{}##}$
\hfil\tabskip0pt &\quad
\hfil$\displaystyle{##}$\tabskip0pt&$\displaystyle{{}##}$
\hfil\tabskip\centering& \llap{$##$}\tabskip0pt \crcr #1 \crcr}}
% ************** double eqalign ******************************************
\def\deqalign#1{\null\,\vcenter{\openup\jot\m@th\ialign{
\strut\hfil$\displaystyle{##}$&$\displaystyle{{}##}$\hfil
&&\quad\strut\hfil$\displaystyle{##}$&$\displaystyle{{}##}$
\hfil\crcr#1\crcr}}\,}
%***************************************************************************
% protection macro for undefined macros
\def\xlabel#1{\expandafter\xl@bel#1}\def\xl@bel#1{#1}
\def\label#1{\l@bel #1{\hbox{}}}
\def\l@bel#1{\ifx\UNd@FiNeD#1\message{label \string#1 is undefined.}%
\xdef#1{?.? }\fi{\let\hyperref=\relax\xdef\next{#1}}%
\ifx\next\em@rk\def\next{}%
%\else\ifx\next#1\xlabel#1\fi\fi\next
\else\def\next{#1}\fi\next}
%********* titlepage, headline, section, subsection, sub, appendix *********
%***************************************************************************
%**************** input with check of file existence ***********************
% Warning macro
\def\DefWarn#1{\ifx\UNd@FiNeD#1\else
\immediate\write16{*** WARNING: the label \string#1 is already defined%
***}\fi}%
%NOW WORK syntax \cinput{filename}
\newread\ch@ckfile
\def\cinput#1{\def\filen@me{#1 }% space mandatory after #1 !!
\immediate\openin\ch@ckfile=\filen@me
\ifeof\ch@ckfile\message{<< (\filen@me\ DOES NOT EXIST in this pass)>>}\else%
\closein \ch@ckfile\input\filen@me\fi}
%********* introduce equation number file: for non-causal quotation
\ifx\UNd@FiNeD\prefix\def\prefix{}\fi % correction added
\newread\ch@ckfile
\immediate\openin\ch@ckfile=\jobname.def
\ifeof\ch@ckfile\message{<< (\jobname.def DOES NOT EXIST in this
pass) >>} \else
\def\DefWarn#1{}%
\closein \ch@ckfile%
\input\jobname.def\fi
%**********
% Autre utilitaire
\def\listcontent{%\immediate\closeout\tfile%
\immediate\openin\ch@ckfile=\jobname.tab % space mandatory after tab!!
\ifeof\ch@ckfile\message{no file \jobname.tab, no table of
contents this
pass}%
\else\closein\ch@ckfile\centerline{\bf\iffrancmode Table des
mati\`eres \else Contents\fi}\nobreak\medskip%
{\baselineskip=12pt\parskip=0pt\catcode`\@=11\input\jobname.tab
\catcode`\@=12\bigbreak\bigskip}\fi}
%**************************************************************************
\newcount\nosection
\newcount\nosubsection
\newcount\neqno
\newcount\notenumber
\newcount\nofigure
\newcount\notable
\newcount\noexerc
\newif\ifappmode
\def\equation{\jobname.equ}
\newwrite\equa
% ******************* titlepage **********************************
\def\authorname#1{\centerline{\bf #1}\smallskip}
\def\address#1{\baselineskip14pt #1\bigskip}
\def\saclay{\baselineskip14pt\centerline{CEA-Saclay, Service de Physique
Th\'eorique${}^\dagger$} \centerline{F-91191 Gif-sur-Yvette Cedex,
FRANCE}\bigskip \footnote{}{${^\dagger}$Laboratoire de la
Direction des Sciences de la Mati\`ere du Commissariat \`a
l'Energie Atomique} }
\newdimen\hulp
\def\maketitle#1{
\edef\oneliner##1{\centerline{##1}}
\edef\twoliner##1{\vbox{\parindent=0pt\leftskip=0pt plus
1fill\rightskip=0pt plus 1fill
                     \parfillskip=0pt\relax##1}}
\setbox0=\vbox{#1}\hulp=0.5\hsize
                 \ifdim\wd0<\hulp\oneliner{#1}\else
                 \twoliner{#1}\fi}
\def\preprint#1{\ifdraftmode\gdef\prepname{\jobname/#1}\else%
\gdef\prepname{#1}\fi\hfill{%\sacfont
\expandafter{\prepname}}\vskip20mm}
% **************** beginning
\def\title#1\par{\gdef\titlename{#1}
\maketitle{\ssbx\uppercase\expandafter{\titlename}} \vskip20truemm
\nosection=0 \neqno=0 \notenumber=0 \nofigure=0 \notable=0
\def\prefix{}
\appmodefalse \mark{\the\nosection} \message{#1}
%\immediate\openout\tab=\table
\immediate\openout\equa=\equation }
\def\abstract{\vskip8mm\iffrancmode\centerline{R\'ESUM\'E}\else%
\centerline{ABSTRACT}\fi \smallskip \begingroup\narrower
\elevenpoint\baselineskip10pt}
\def\endabstract{\par\endgroup \bigskip}
%******************************* section ***********************************
\def\section#1\par{\vskip0pt plus.1\vsize\penalty-100\vskip0pt plus-.1
\vsize\bigskip\vskip\parskip
\ifnum\nosection=0\ifappmode\relax\else\writetoc
\fi\fi% ajout
\advance\nosection by 1\global\nosubsection=0\global\neqno=0
\vbox{\noindent\bf{\hyperdef\hypernoname{section}{\prefix\the\nosection}%
{\prefix\the\nosection}\ #1}}
\writetoca{{\string\hyperref{}{section}{\prefix\the\nosection}%
{\prefix\the\nosection}} {#1}} \message{\the\nosection\ #1}
\mark{\the\nosection}\bigskip\noindent }
%%%%%%%%% ******** minor changes
% appendix
\def\appendix#1#2\par{\bigbreak \nosection=0 \appmodetrue \notenumber=0 \neqno=0
\def\prefix{A}
\mark{\the\nosection} \message{APPENDICES} {\centerline{\bf
Appendices}
%%%%%%   *********
\hyperdef\hypernoname{appendix}{\prefix}{
\leftline{\uppercase\expandafter{#1}}
\leftline{\uppercase\expandafter{#2}}}}
\writetoca{\string\hyperref{}{appendix}{\prefix}{Appendices}.\ #1 \ #2}%
}
% **************************** \subsection *************************
\def\subsection#1\par {\vskip0pt plus.05\vsize\penalty-100\vskip0pt
plus-.05\vsize\bigskip\vskip\parskip\advance\nosubsection by 1
\vbox{\noindent\it{\hyperdef\hypernoname{subsection}{\prefix\the\nosection.%
\the\nosubsection}{\prefix\the\nosection.\the\nosubsection\ #1}}}%
\smallskip\noindent
\writetoca{{\string\hyperref{}{subsection}{\prefix\the\nosection.%
\the\nosubsection}{\prefix\the\nosection.\the\nosubsection}} {#1}}
\message{\the\nosection.\the\nosubsection\ #1} }
%
\def\note #1{\advance\notenumber by 1
\footnote{$^{\the\notenumber}$}{\sevenrm #1}}
% ?????
\def\sub#1{\medskip\vskip\parskip
{\indent{\it #1}.}}
%\def\enchapter{\end}
\parindent=1em
\newinsert\margin
\dimen\margin=\maxdimen \count\margin=0 \skip\margin=0pt
%*****************************************************************
% IMPORTANT, new version demands chapter be defined before any section,
% section be defined before any subsection
\def\sslbl#1{\DefWarn#1%
\ifdraftmode{\hfill\escapechar-1{\rlap{\hskip-1mm%
\sevenrm\string#1}}}\fi%
\ifnum\nosection=0\if\prefix{}\xdef#1{}%
\edef\ewrite{\write\equa{{\string#1}}%
\write\equa{}}\ewrite%
\else
\xdef#1{\noexpand\hyperref{}{appendix}{\prefix}{\prefix}}%
\edef\ewrite{\write\equa{{\string#1},\prefix}%
\write\equa{}}\ewrite%
\writedef{#1\leftbracket#1} \fi
\else%
\ifnum\nosubsection=0%
\xdef#1{\noexpand\hyperref{}{section}{\prefix\the\nosection}%
{\prefix\the\nosection}}%
\edef\ewrite{\write\equa{{\string#1},\prefix\the\nosection}%
\write\equa{}}\ewrite%
\writedef{#1\leftbracket#1}
\else%
\xdef#1{\noexpand\hyperref{}{subsection}{\prefix\the\nosection.%
\the\nosubsection}{\prefix\the\nosection.\the\nosubsection}}%
\writedef{#1\leftbracket#1}
\edef\ewrite{\write\equa{{\string#1},\prefix\the\nosection%
.\the\nosubsection}\write\equa{}}\ewrite\fi\fi}%
%**********************************************
\newwrite\tfile \def\writetoca#1{}
%       use this to write file with table of contents
\def\writetoc{\immediate\openout\tfile=\jobname.tab
\def\writetoca##1{{\edef\next{\write\tfile{\noindent ##1 \string\leaderfill%
%{\string\hyperref{}{page}{\noexpand\number\pageno}{\noexpand\number\pageno}}
\noexpand\number\pageno\par}}\next}}}

% ********************* references harvmac style ***********************
%     \ref\label{text}
% generates a number, assigns it to \label, generates an entry.
% To list the refs on a separate page,  \listrefs
%
\def\nolabels{\def\wrlabeL##1{}\def\eqlabeL##1{}\def\reflabeL##1{}}
\def\writelabels{\def\wrlabeL##1{\leavevmode\vadjust{\rlap{\smash%
{\line{{\escapechar=` \hfill\rlap{\sevenrm\hskip.03in\string##1}}}}}}}%
\def\eqlabeL##1{{\escapechar-1\rlap{\sevenrm\hskip.05in\string##1}}}%
\def\reflabeL##1{\noexpand\llap{\noexpand\sevenrm\string\string\string##1}}}
\nolabels

\global\newcount\refno \global\refno=1
\newwrite\rfile
%
\def\ref{[\hyperref{}{reference}{\the\refno}{\the\refno}]\nref}
\def\nref#1{\DefWarn#1%
\xdef#1{[\noexpand\hyperref{}{reference}{\the\refno}{\the\refno}]}%
\writedef{#1\leftbracket#1}%
\ifnum\refno=1\immediate\openout\rfile=\jobname.ref\fi
\chardef\wfile=\rfile\immediate\write\rfile{\noexpand\item{[\noexpand\hyperdef%
\noexpand\hypernoname{reference}{\the\refno}{\the\refno}]\ }%
\reflabeL{#1\hskip.31in}\pctsign}\global\advance\refno
by1\findarg}
%       horrible hack to sidestep tex \write limitation
\def\findarg#1#{\begingroup\obeylines\newlinechar=`\^^M\pass@rg}
{\obeylines\gdef\pass@rg#1{\writ@line\relax #1^^M\hbox{}^^M}%
\gdef\writ@line#1^^M{\expandafter\toks0\expandafter{\striprel@x #1}%
\edef\next{\the\toks0}\ifx\next\em@rk\let\next=\endgroup\else\ifx\next\empty%
\else\immediate\write\wfile{\the\toks0}\fi\let\next=\writ@line\fi\next\relax}}
\def\striprel@x#1{} \def\em@rk{\hbox{}}
%
\def\lref{\begingroup\obeylines\lr@f}
\def\lr@f#1#2{\DefWarn#1\gdef#1{\let#1=\UNd@FiNeD\ref#1{#2}}\endgroup\unskip}
%
\def\semi{;\hfil\break}
\def\addref#1{\immediate\write\rfile{\noexpand\item{}#1}} %now unnecessary
%
\def\listrefs{{}\vfill\supereject\immediate\closeout\rfile\writestoppt
\baselineskip=14pt\centerline{{\bf\iffrancmode R\'eferences\else References%
\fi}}
\bigskip{\parindent=20pt%
\frenchspacing\escapechar=` \input
\jobname.ref\vfill\eject}\nonfrenchspacing}
%
\def\startrefs#1{\immediate\openout\rfile=\jobname.ref\refno=#1}
%
\def\xref{\expandafter\xr@f}\def\xr@f[#1]{#1}
\def\refs#1{\count255=1[\r@fs #1{\hbox{}}]}
\def\r@fs#1{\ifx\UNd@FiNeD#1\message{reflabel \string#1 is undefined.}%
\nref#1{need to supply reference \string#1.}\fi%
\vphantom{\hphantom{#1}}{\let\hyperref=\relax\xdef\next{#1}}%
\ifx\next\em@rk\def\next{}%
\else\ifx\next#1\ifodd\count255\relax\xref#1\count255=0\fi%
\else#1\count255=1\fi\let\next=\r@fs\fi\next}
%************************
\newwrite\lfile
{\escapechar-1\xdef\pctsign{\string\%}\xdef\leftbracket{\string\{}
\xdef\rightbracket{\string\}}\xdef\numbersign{\string\#}}
\def\writedefs{\immediate\openout\lfile=\jobname.def \def\writedef##1{%
{\let\hyperref=\relax\let\hyperdef=\relax\let\hypernoname=\relax
 \immediate\write\lfile{\string\def\string##1\rightbracket}}}}%
\def\writestop{\def\writestoppt{\immediate\write\lfile{\string\pageno%
\the\pageno\string\startrefs\leftbracket\the\refno\rightbracket%
\string\def\string\secsym\leftbracket\secsym\rightbracket%
\string\secno\the\secno\string\meqno\the\meqno}\immediate\closeout\lfile}}
%
\def\writestoppt{}\def\writedef#1{}
\writedefs
% ******
% bibliography: not very satisfactory
\def\biblio\par{\vskip0pt plus.1\vsize\penalty-100\vskip0pt plus-.1
\vsize\bigskip\vskip\parskip \message{Bibliographie}
{\leftline{\bf \hyperdef\hypernoname{biblio}{bib}{Bibliographical
Notes}}} \nobreak\medskip\noindent\frenchspacing
\writetoca{\string\hyperref{}{biblio}{bib}{Bibliographical Notes}}}%
\def\endbiblio{\nonfrenchspacing}
%**************** autre version si plusieurs biblio ************************
\def\biblionote{\iffrancmode Notes Bibliographiques\else Bibliographical Notes
\fi}
\def\beginbib\par{\vskip0pt plus.1\vsize\penalty-100\vskip0pt plus-.1
\vsize\bigskip\vskip\parskip \message{Bibliographie}
{\leftline{\bf \hyperdef\hypernoname{biblio}{\the\nosection}%
{\biblionote}}} \nobreak\medskip\noindent\frenchspacing
\writetoca{\string\hyperref{}{biblio}{\the\nosection}%
{\biblionote}}}%
\def\endbib{\nonfrenchspacing}

% *************** exercises: same comment
\def\Exercises{\iffrancmode Exercices\else Exercises
\fi}
\def\exerc\par{\vskip0pt plus.1\vsize\penalty-100\vskip0pt plus-.1
\vsize\bigskip\vskip\parskip\global\noexerc=0 \message{Exercises}
{\leftline{\bf
\hyperdef\hypernoname{exercise}{\the\nosection}{\Exercises}}}
\nobreak\medskip\noindent\frenchspacing
\writetoca{\string\hyperref{}{exercise}{\the\nosection}{\Exercises}}
}
\def\esubsec{\ifnum\noexerc=0\vskip-12pt\else\vskip0pt plus.05\vsize%
\penalty-100\vskip0pt plus-.05\vsize\bigskip\vskip\parskip\fi%
\global\advance\noexerc by 1
\hyperdef\hypernoname{exercise}{\the\nosection.\the\noexerc}{}%
\vbox{\noindent\it \iffrancmode Exercice\else Exercise\fi\
\the\nosection.\the\noexerc}\smallskip\noindent}
%%%%%%%%%%%%%%%%%%%%%%%%%%%
\def\exelbl#1{\ifdraftmode{\hfill\escapechar-1{\rlap{\hskip-1mm%
\sevenrm\string#1}}}\fi%
{\xdef#1{\noexpand\hyperref{}{exercise}{\the\nosection.\the\noexerc}%
{\the\nosection.\the\noexerc}}}%
\edef\ewrite{\write\equa{{\string#1}\the\nosection.\the\noexerc}%
\write\equa{}}\ewrite%
\writedef{#1\leftbracket#1}}
%*************************************************************************
%Macro de numerotation automatique
%*************************************************************************
% numbering without naming
\def\eqnn{\global\advance\neqno by 1 \ifinner\relax\else%
\eqno\fi(\prefix\the\nosection.\the\neqno)}
%
% numbering and attaching a name: \eqnd{\ename}
\def\eqnd#1{\DefWarn#1%
\global\advance\neqno by 1
{\xdef#1{($\noexpand\hyperref{}{equation}{\prefix\the\nosection.\the\neqno}%
{\prefix\the\nosection.\the\neqno}$)}}%???
\ifinner\relax\else\eqno\fi(\hyperdef\hypernoname{equation}{\prefix\the%
\nosection.\the\neqno}{\prefix\the\nosection.\the\neqno})
\writedef{#1\leftbracket#1}
\ifdraftmode{\escapechar-1{\rlap{\hskip.2mm\sevenrm\string#1}}}\fi
\edef\ewrite{\write\equa{{\string#1},(\prefix\the\nosection.\the\neqno)
{\noexpand\number\pageno}}\write\equa{}}\ewrite}
%
% for eqalignno, allows (1a) (1b)...
\def\checkm@de#1#2{\ifmmode{\def\f@rst##1{##1}\hyperdef\hypernoname{equation}%
{#1}{#2}}\else\hyperref{}{equation}{#1}{#2}\fi}
\def\f@rst#1{\c@t#1a\em@ark}\def\c@t#1#2\em@ark{#1}
\def\eqna#1{\DefWarn#1%
\global\advance\neqno by1\ifdraftmode{\hfill%
\escapechar-1{\rlap{\sevenrm\string#1}}}\fi%
\xdef #1##1{(\noexpand\relax\noexpand%
\checkm@de{\prefix\the\nosection.\the\neqno\noexpand\f@rst{##1}1}%
{\hbox{$\prefix\the\nosection.\the\neqno##1$}})}
\writedef{#1\numbersign1\leftbracket#1{\numbersign1}}%
}
%
% \eqn,\eqnna,eqnnd obsolete pour compatibilite anterieure,
\def\eqn{\eqnn}
\def\eqnna{\eqna}
\def\eqnnd{\eqnd}
%
\def\em@rk{\hbox{}}
\def\xeqn{\expandafter\xe@n}\def\xe@n(#1){#1}
\def\xeqna#1{\expandafter\xe@na#1}\def\xe@na\hbox#1{\xe@nap #1}
\def\xe@nap$(#1)${\hbox{$#1$}}
% \eqns allows to quote several equations, suppressing unnecessary ()
\def\eqns#1{(\e@ns #1{\hbox{}})}
\def\e@ns#1{\ifx\UNd@FiNeD#1\message{eqnlabel \string#1 is undefined.}%
\xdef#1{(?.?)}\fi{\let\hyperref=\relax\xdef\next{#1}}%
\ifx\next\em@rk\def\next{}%
\else\ifx\next#1\xeqn#1\else\def\n@xt{#1}\ifx\n@xt\next#1\else\xeqna#1\fi
\fi\let\next=\e@ns\fi\next}
%**********************************************************************
%*************************** figure macros ****************************
% Pour centrer ajouter 16mm a la taille de la boite
\def\figure#1#2{\global\advance\nofigure by 1 \vglue#1%
\hyperdef\hypernoname{figure}{\the\nofigure}{}%
{\elevenpoint \setbox1=\hbox{#2}
\ifdim\wd1=0pt\centerline{Fig.\ \the\nofigure\hskip0.5mm}%
\else\def\caption{Fig.\ \the\nofigure\quad#2\hskip0mm}
\setbox0=\hbox{\caption} \ifdim\wd0>\hsize\noindent Fig.\
\the\nofigure\quad#2\else
                 \centerline{\caption}\fi\fi}\par}
% le bigskip a la fin a ete enleve!
\def\ffigure{\figure} % obsolete, for compatibility
%***************
%figure alignee a gauche
\def\lfigure#1#2{\global\advance\nofigure by
1\vglue#1%
\hyperdef\hypernoname{figure}{\the\nofigure}{}%
\leftline{\elevenpoint\hskip10truemm  Fig.\ \the\nofigure\quad
#2}}
%***************
\def\figlbl#1{\ifdraftmode{\hfill\escapechar-1{\rlap{\hskip-1mm%
\sevenrm\string#1}}}\fi%
{\xdef#1{\noexpand\hyperref{}{figure}{\the\nofigure}%
{\the\nofigure}}}%
\edef\ewrite{\write\equa{{\string#1}\the\nofigure}%
\write\equa{}}\ewrite%
\writedef{#1\leftbracket#1}}
%****************************
\def\tablbl#1{\global\advance\notable by
1\ifdraftmode{\hfill\escapechar-1{\rlap{\hskip-1mm%
\sevenrm\string#1}}}\fi%
{\xdef#1{\noexpand\hyperref{}{table}{\the\notable}%
{\the\notable}}}%
\hyperdef\hypernoname{table}{\the\notable}{}%
\edef\ewrite{\write\equa{{\string#1}\the\notable}%
\write\equa{}}\ewrite%
\writedef{#1\leftbracket#1}}

%**********************************************
\catcode`@=12

\def\draftend{\vfill\supereject%
\immediate\closeout\equa%\immediate\closeout\tab
\ifdraftmode%\vfill\supereject%
{\bf \titlename},\par ------------ Date \today. -----------\par
%\edef\ewrite{\write\eqdf{}}\ewrite%
%\immediate\closeout\eqdf
\catcode`\&=0 \catcode`\\=10
\input \equation
\catcode`\\=0 \catcode`\&=4\fi
\end}
\def\endchapter{\draftend}
%%%%%%%%%%%%%%%%%%%%%%%%%%%%%%%%%%%%%%%%%%%%%%%%%%%%%%%%%%%%%%%%%
% **************END ****** SEVERAL MACROS ********* END ************
%%%%%%%%%%%%%%%%%%%%%%%%%%%%%%%%%%%%%%%%%%%%%%%%%%%%%%%%%%%%%%%%%%%

%\input zmacxxx

%\input epsf
% figures fig10-3.eps, fig29-1.eps, bulle4.eps, triangle.eps, triangii.eps


%%%%%%%%%%%%%%%%%%%%%%%%%%%%%%%%%%%%%%%%%\draft
%\writedefs
\francmodefalse


\def\*{********** new, please check **********}



\def\slam#1{\mkern-1.5mu\raise-0.2pt\hbox{$\scriptstyle{\not}$}\mkern1.2mu
#1\mkern 0.7mu}
\def\r{{\rm r}}
%\def\slash#1{\mathord{\hskip-.5pt\not\hskip.5pt\mathrel{#1}}}
%$$\e^{i\slam{p}},\e^{i\slash{p}}.\sla{p},\slash{p} $$

\def\D{{\rm D}}
\def\r{{\rm r}}
\def\psib{\psi}
\def\phib{\phi}
\def\pib{\pi}
\def\biglb{\mathopen{\vcenter{\hbox{\tenex\char'2}}}}
\def\bigglb{\mathopen{\vcenter{\hbox{\tenex\char'24}}}}
\def\biggglb{\mathopen{\vcenter{\hbox{\tenex\char'42}}}}
\def\bigrb{\mathclose{\vcenter{\hbox{\tenex\char'3}}}}
\def\biggrb{\mathclose{\vcenter{\hbox{\tenex\char'25}}}}
\def\bigggrb{\mathclose{\vcenter{\hbox{\tenex\char'43}}}}
\def\bigclb{\mathopen{\vcenter{\hbox{\tenex\char'10}}}}
\def\biggclb{\mathopen{\vcenter{\hbox{\tenex\char'32}}}}
\def\bigcrb{\mathclose{\vcenter{\hbox{\tenex\char'11}}}}
\def\biggcrb{\mathclose{\vcenter{\hbox{\tenex\char'33}}}}
\def\bigbra{\mathopen{\vcenter{\hbox{\tenex\char'12}}}}
\def\bigket{\mathclose{\vcenter{\hbox{\tenex\char'13}}}}

%%%%%%%%%%   %%%%%%%%

\def\half{\frac{1}{2}}
\def\lb{\hfil\break}
\def\del{\partial}
\def\pslash{\mathord{\not\mathrel{p}}}
\def\kslash{\mathord{\not\mathrel{k}}}
\def\delslash{\mathord{\not\mathrel{\partial}}}
\def\psibar{\bar\psi}
\def\psvec{\vec\psi}
\def\psvecb{\vec{\bar\psi}}
\def\phivec{\vec \phi}
\def\varphivec{\vec \varphi}
\def\ma{m_\varphi}
\def\mpsi{m_\psi}
\def\del{\partial}
\def\psvec{\vec\psi}
 \def\exphi{\bigbra \Phi ^2 \bigket}
 \def\phiones{{\Phi _1}^2}
 \def\phitwos{{\Phi _2}^2}

\def\mpone{m_{\psi_1}}
\def\mptwo{m_{\psi_2}}
\def\W{{\cal W}(m^2,{\Phi_c}^2)}
\def\psps{N(-\Lambda{\mpsi \over {\pi^2} }
+{{\mpsi |\mpsi|} \over {2\pi} })}
\def\sqr#1#2{{\vcenter{\hrule height.#2pt
     \hbox{\vrule width.#2pt height#1pt \kern#1pt
          \vrule width.#2pt}
       \hrule height.#2pt}}}
\def\sbox{{\mathchoice{\sqr34}{\sqr34}{\sqr23}{\sqr23}}}
\def\bbox{{\mathchoice{\sqr84}{\sqr84}{\sqr53}{\sqr43}}}

% ************** math symbols ********************************************
\def\to{\rightarrow}
\def\half{\frac{1}{2}}
\def\del{\partial}
\def\phib{{\phi}}
\def\Vc{\Omega_c}
\newskip\tableskipamount \tableskipamount=10pt plus 4pt minus 4pt
\def\tableskip{\vskip\tableskipamount}
\def\upar{\uparrow\kern-9.\exec1pt\lower.2pt \hbox{$\uparrow$}}
\def\hrulefill{\leaders\hrule height 0.8pt \hfill}

%%%%%%%%%%%%%%%%%%%%%%%%%%%%%%%%%%%%%%%%%%%%%%%%%%%%%%%%%

%\preprint{SPhT~~ T/02-??? ~~~~~~~Technion~~ 02-PH-???}

\preprint{May 2003}

\title{Quantum Field Theory in the Large $N$ Limit: a review}

\centerline{{\bf Moshe~Moshe}${}^{1 ,a,c}$ and {\bf
Jean~Zinn-Justin}${}^{2,b}$}
\bigskip
{\baselineskip14pt
\centerline{${}^1$\it Department of Physics, Technion - Israel
Institute of Technology,} \centerline{\it Haifa, 32000 ISRAEL}
\smallskip
\centerline{${}^2$\it  Service de Physique Th\'eorique* CEA/Saclay}
\centerline{\it F-91191 Gif-sur-Yvette C\'edex, FRANCE}
\centerline{\it and} \centerline{\it Institut de Math\'ematiques
de Jussieu--Chevaleret,} \centerline{\it Universit\'e de Paris
VII} }
\footnote{}{(a)~email: moshe@physics.technion.ac.il}
\footnote{}{(b)~email: zinn@spht.saclay.cea.fr} \footnote{}{(c)
~Supported in part by the Israel Science Foundation grant number
193-00}
 \footnote{}{${^*}$Laboratoire de la Direction des Sciences
de la Mati\`ere du Commissariat \`a \indent \ l'Energie Atomique
Unit\'e de recherche associ\'ee au CNRS} \vskip 3mm


%%%%%%%%%%%%%%%%%%%%%%%%%%%%%%%%%%%%%%%%%%%%%%%


\abstract We review the solutions of $O(N)$ and $U(N)$ quantum field theories  in
the large $N$ limit and as $1/N$ expansions, in the case of vector
representations. Since invariant composite fields have small
fluctuations for large $N$, the method relies on constructing
effective field theories for composite fields after integration
over the original degrees of freedom. We first solve  a general
scalar $U(\phib^2)$ field theory for $N$ large and discuss various
non-perturbative physical issues such as critical behaviour.  We
show how large $N$ results can also be obtained from variational
calculations.We illustrate these ideas by showing that the large
$N$ expansion allows to relate the $(\phib^2)^2$ theory and the
non-linear $\sigma$-model, models which are renormalizable in
different dimensions. Similarly, a  relation
between $CP(N-1)$ and abelian Higgs models is exhibited.
Large $N$
techniques also allow solving self-interacting fermion models.
A relation between  the Gross--Neveu,   a theory with a four-fermi
self-interaction, and a Yukawa-type theory renormalizable in four
dimensions then follows. We discuss dissipative dynamics,
which is relevant to the approach to equilibrium, and which in
some formulation exhibits quantum mechanics supersymmetry. This
also serves as an introduction to the study of the 3D
supersymmetric quantum  field theory. Large
$N$ methods are useful in problems that involve a crossover between different dimensions. We thus  briefly discuss  finite
size effects, finite temperature scalar and supersymmetric field
theories. We also use large $N$ methods to investigate the weakly interacting Bose gas. The solution of the general scalar $U(\phib^2)$ field theory is then applied to other  issues like tricritical behaviour and double
scaling limit.
\endabstract
\vfill\eject
\listcontent
\vfill\eject
\section Introduction

Studies of the non-perturbative features of quantum field
theories are at the forefront of theoretical physics research.
 Though remarkable progress
has been achieved in recent years, still, some of the
more fundamental questions have only a descriptive answer,
whereas non-perturbative calculable schemes are seldom at hand.
The absence of calculable
dynamics in realistic models is often supplemented
by simpler models in which the essence of the dynamics
is revealed. Such a calculable framework for exploring theoretical
ideas is given by large $N$ quantum field theories. \par
Very early quantum field  theorists have looked for methods to solve
field theory beyond perturbation theory and obtain confirmation of
perturbative results. Moreover, some important physical questions are often intrinsically non-perturbative. Let us mention, for
illustration, the problem of fermion-pair condensation. A number of
similar schemes were proposed, all of mean-field theory nature,
variational methods, self-consistent approximations, all reducing the
interacting theory to a free fermion theory with self-consistently
determined parameters. For example, the quartic fermion self-interaction
$(\bar\psi\psi)^2$ would be replaced by a term  proportional to
$\langle \bar\psi\psi\rangle \bar\psi\psi$, where
$\langle \bar\psi\psi\rangle $ is the free field average. However,
all these methods have several drawbacks:  it is unclear how to
improve the results systematically, the domain of validity of the
approximations are often unknown,  in fact there is no obvious small
parameter.  To return to the fermion example, one realizes that the
approximation would be justified if for some reasons the fluctuations
of the composite field $\bar\psi\psi$ were much smaller than the
fluctuations of the fermion field itself.
\par
Large $N$ techniques solve this problem in the spirit of the
central limit theorem of the theory of probabilities. If the field
has $N$ components, in the large $N$ limit scalar (in the group
sense) composite fields are sums of many terms and therefore may
have small fluctuations (at least if the different terms are sufficiently
uncorrelated). Therefore, if we are able to construct an effective
field theory for the scalars, integrating out the original degrees
of freedom, we can solve the field theory not only in the large
$N$ limit, but also in a systematic $1/N$ expansion. On the
technical level, one notes that in vector representations the
number of independent scalars is finite and independent of $N$,
unlike what happens for matrix representations. This explains why
vector models have been solved much more generally than matrix
models.
\par
 In this review we describe a few applications of large $N$
techniques to quantum field theories (QFT) with $O(N)$ or $U(N)$
symmetries, where the fields are in the {\it vector}\/
representation  \ref\rZJTai{Some sections are
directly inspired from  J. Zinn-Justin, lectures given at {\it 11$^{\rm th}$ Taiwan Spring School}, Taipei 1997, hep-th/9810198.}. A summary of results are presented here  in the
study of the phase structure of quantum field theories. It is
demonstrated that large $N$ results nicely complement results
obtained from more conventional perturbative renormalization group
(RG) \ref\rbook{For a general
background with analogous notation see  J. Zinn-Justin,
{\it Quantum Field Theory and Critical Phenomena}, Clarendon
Press (Oxford 1989, fourth ed.~2002).}. Indeed, the shortcoming of the latter method is that it
mainly applies to gaussian or near gaussian fixed points. This
restricts space dimension to dimensions in which the corresponding
effective QFT is renormalizable, or after dimensional
continuation, to the neighbourhood of such dimensions. In some
cases, large $N$ techniques allow a study in generic dimensions.
In this review we will, in particular, stress two points: first,
it is always necessary to check that the $1/N$ expansion is both
IR finite and renormalizable.  This is essential for the stability of the large
$N$ results and the existence of a $1/N$ expansion. Second, the large $N$
expansion is just a technique, with its own (often unknown)
limitations and it should not be discussed in isolation. Instead,
as we shall do in the following examples, it should be combined, when possible,
with other perturbative techniques and the reliability of the
$1/N$ expansion should be inferred from the general consistency of
all results.\par
Second-order phase transitions in classical
statistical physics provide a first illustration of
the usefulness of the large $N$ expansion. Due to the divergence
of the correlation length at the critical temperature, one finds
that near $T_c$, system share universal properties which can be
described by effective continuum quantum field theories. The
$N$-vector model that we discuss in {\bf Sections \label{\scfivN}} and {\bf \label{\ssNsymspace}} is the
simplest example but it has many applications since it allows to
describe the critical properties of systems like vapour--liquid,
binary mixtures, superfluid Helium or ferromagnetic transitions as
well as the statistical properties of polymers. Before showing
what kind of information can be provided by large $N$ techniques,
we will first shortly recall what can be learned from perturbative
RG methods.
Long distance properties can be described by a
$(\phib^2)^2$ field theory in which analytic calculations can be performed
only in an $\varepsilon=4-d$ expansion. From lattice model considerations, one expects that
the same properties can also be derived from a different
 QFT,  the $O(N)$ non-linear $\sigma$ model, which, however, can be solved only
as an $\varepsilon=d-2$ expansion. It is somewhat surprising
that the same statistical model can be described by two different
theories. Since the results derived in this  way are valid {\it a
priori}\/ only for $\varepsilon$ small, there is no overlap to
test the consistency. The large $N$ expansion  enables one  to
discuss generic dimensions and thus to understand the relation
between both field theories. \par
Similar large $N$ techniques can also be applied to other non-linear models
and we briefly examine the example of the $CP(N-1)$ model. \par
Four-fermi interactions have been
proposed to generate a composite Higgs particle in four
dimensions, as an alternative to a Yukawa-type theory, as one
finds in the Standard Model. Again, in the specific example of
the Gross--Neveu model, in {\bf Section \label{\ssNfermions}}  we will use large $N$
techniques to clarify the relations between these two approaches.
We will finally briefly investigate other
models with chiral properties, like massless QED or the Thirring
model.\par
Preceding the
discussion of supersymmetric models, we study  in {\bf Section \label{\ssCrDY}}
critical dynamics of purely dissipative systems, which are known
to provide a simple field theory extension of supersymmetric quantum mechanics. We
then study in {\bf Section \label{\ssNSUSYsc}} two SUSY models in two and three dimensions: the
supersymmetric $\phi^4$ field theory, which is shown to have a
peculiar phase structure in the large $N$ limit, and a
supersymmetric non-linear $\sigma$ model. \par
Other applications of the large $N$ expansion include finite size effects ({\bf Section \label{\ssNFSS}}) and finite temperature field theory to which we devote {\bf Section \label{\scFTQFT}} for non-supersymmetric theories and  {\bf Section \label{\ssNSUSYT}} for the supersymmetric theories of  {\bf Section \label{\ssNSUSYsc}}. In these situations a dimensional crossover occurs
between the large size or zero temperature situation, where the
infinite volume theory is relevant, to a dimensionally reduced
theory in the small volume or high temperature limit. Both
effective field theories being renormalizable in different
dimensions, perturbative RG cannot describe correctly both
situations. Again, large $N$ techniques help    understanding
the crossover. We often compare in this review the large $N$
results with those obtained by variational methods, since it is
often possible to set up variational calculations which parallel
the large $N$ limit calculations.\par
The effect of weak  interactions on Bose gases at the Bose--Einstein condensation temperature are analyzed in {\bf Section \label{\scBEC}} where large $N$ techniques are
employed for the non-perturbative calculations of physical
quantities.
\par
We then return  in {\bf Section \label{\ssdblescal}} to general scalar boson field theories,
and examine multi-critical points (where the large $N$ technique
will show some obvious limitations), and the double scaling limit,
a toy model for discussing problems encountered in matrix models
of 2D quantum gravity.
\par
In  this review, we also discuss the breaking of scale
invariance. In most quantum field theories spontaneous and
explicit breaking of scale invariance  occur simultaneously and
thus the breaking of scale symmetry is not normally accompanied by
the appearance of a massless Nambu--Goldstone boson. Spontaneous
breaking of scale invariance, unaccompanied by explicit breaking,
associated with a non-zero fixed point and the creation of a
massless bound state is demonstrated in {\bf Section \label{\ssNscaleinv}}.
%\label{\ssbscal}.
In some cases the dynamics by which scale invariance is broken in
a theory which has no trace anomalies in perturbation theory is
directly related to the breaking of internal symmetry. This
appears in the phase structure of $O(N)\times O(N)$
symmetric models  for $N$ large, where the breaking of scale invariance is
directly related to the breaking of the internal symmetry.
%%% $O(N) \times O(N) \mapsto O(N-1) \times O(N-1)$.
The spontaneous breaking of scale invariance in a supersymmetric,
$O(N)$ symmetric vector model in three dimensions was also studied
in {\bf Section \label{\ssNSUSYsc}},
%\label{\ssxxx}
were one finds the creation of a massless fermionic bound state as
the supersymmetric partner of the massless boson in the
supersymmetric ground state.
% as discussed in section \label{\XX}.
 \par



%%%%%%%%%%$$$$$$$
\vfill\eject
%********
%\phi^4 theory , n.l.$\sigma$-model
\input Nrev123
%\input largen1
\vfill\eject
%Fermions, GN model, GNY model
\input Nrev4
%\input largen2
\vfill\eject
%Disipative Dynamics
\input Nrev5
%\input largen5
\vfill\eject
% SUSY Models
\input Nrev6
%\input largen3
\vfill\eject
% Finite Temperature Field Theories
\input Nrev7
%\input largen6
\vfill\eject
% SUSY at finite T
\input Nrev8
%\input largen8.tex
\vfill\eject
% Weakly Interacting Bose gas
\input Nrev9
%\input largen7
\vfill\eject
% V(\phi^2) and double scaling limit
\input Nrev10
%\input largen4

\vfill\eject

\centerline{\bf Acknowledgements }

\noindent MM thanks the Saclay and Fermilab theory groups for
their warm hospitality while different parts of this work was
done.
\nref\rEMTen{ The structure of the energy momentum tensor in
renormalizable quantum field theories and a discussion of explicit
and spontaneous breaking of scale invariance can be found in \lb
C.G. Callan, S. Colman and R. Jackiw {\it Annals of Phys.} 59
(1970) 42;  S. Coleman and R. Jackiw {\it Annals of Phys.}  67
(1971) 552  and reference \refs{\rAdler}.}

\nref\rAdler{A general discussion on breaking of scale invariance
is found in  S.L. Adler, {\it Rev. Mod. Phys.} 54 (1982) 729.}

%%%%%%%%%%%%  \input References.tex
\vfill\eject
%%%%%%%%%%%%%%%%%%%%%  LISTREF at the end of the paper
\listrefs
%%%%%%%%%%%%%%%%%%%%%%%%%%%%%%%%%%

\appendix{}

\input NrevApx1
%\input largen9
% Spontaneous Breaking of Scale Inv.
\section One-loop diagrams

We present some technical details concerning one-loop calculations relevant to the gap equation
at leading order.
%
\subsection The regularized one-loop diagrams

We expand here for $m \ll \Lambda  $ the regularized  one-loop diagram \etadepole\ \sslbl\apiloop
$$\Omega_d(m) =\int{  \d^d k \over (2\pi)^d } \tilde\Delta _\Lambda (k)= {1\over (2\pi)^d }
 \int{  \d^d k \over m^2+ k^2D(k^2/\Lambda ^2)}=\Lambda^{d-2} \omega_d  (m/\Lambda) .\eqnd\eAptadepole $$
The function $D(z)$, $D(0)=1$, is  strictly positive for $z> 0$, analytic in
the neighbourhood of the real positive semi-axis, and increasing faster than
$z^{(d-2)/2}$ for $z\to+\infty $. \par
To extract the coefficients in the expansion we use the Mellin
transform and dimensional  continuation (see section \ssNplarge). We define
$$M(s)=\int_0^\infty \d z\, z^{-1-s} {1\over (2\pi)^d }
 \int{  \d^d k \over z+ k^2D(k^2 )}\,.$$
If the $k$ integral has a contribution of the form $z^\alpha $ in a small $z$ expansion, the Mellin transform has a pole at $s=\alpha $ and the coefficient can be identified. \par
We change variables $z \mapsto zk^2 D(k^2)$ and obtain
$$\eqalign{M(s)&=\int_0^\infty \d z\, {z^{-1-s} \over 1+z}
 \int{   \d^d k\over (2\pi)^d } \bigl( k^2D(k^2 )\bigr)^{-1-s} \cr
&=-{\pi \over \sin(\pi s)}
 \int {   \d^d k\over (2\pi)^d } \bigl( k^2D(k^2 )\bigr)^{-1-s}.\cr}$$
$$ M(s) =\int_0^\infty \d z\, {z^{-1-s} \over 1+z}
 \int{   \d^d k\over (2\pi)^d } \bigl( k^2D(k^2 )\bigr)^{-1-s}
 =-{\pi \over \sin(\pi s)}I_d(s)$$
with
$$I_d(s)= \int {   \d^d k\over (2\pi)^d } \bigl( k^2D(k^2 )\bigr)^{-1-s}=N_d
\int_0^\infty \d k\,k^{d-3-2s}\bigl(  D(k^2 )\bigr)^{-1-s}
. $$
We find two  series, the first  corresponding to the zeros of the sine function
with $s$ non-negative integer (the poles with $s<0$ corresponding to the
singularities in the large $z$ behaviour)
$$ M(s)\mathop{\sim}_{s\to n} (-1)^n  I_d(n){1\over n-s}. $$
Note that the analytic continuation of $I_n(d)$ for low dimensions is obtained by subtracting to $D^{-1-n}$ the first terms of its Taylor series at $k=0$.\par
The second corresponds to the divergence at $k=0$ of the $k$ integral $I(s)$. The residues are obtained by integrating near $k=0$ and expanding $D(k^2)$ in powers of $k^2$
 $$\eqalign{I_d(s)&\sim N_d
\int_0^1\d k\,k^{d-3-2s}\bigl(  D(k^2 )\bigr)^{-1-s} \cr
&\sim {N_d \over d-2-2s}-(1+d/2)D'(0){N_d \over d-2s}+\cdots .\cr} $$
In general the poles are at $s=\ud(d-2)+n $
with  $n$ non-negative integer. Therefore the small $z$ expansion
of $\omega _d( \sqrt{z})$ has the general form
$$\omega _d( \sqrt{z})=\sum_{n=0} \alpha _n(d)z^{(d-2)/2+n}+\beta _n(d)
z^n. $$

\smallskip
{\it Parametrization.}
We parametrize the expansion for  $z\to 0$ and $d>2$
as % right parametrization ? check signs ?
$$\omega_d(z)= \omega _d(0)-K(d)z^{d-2}+a(d)z^2 +O\left(z^{d}, z^4\right).
\eqnd\eAptadepolii $$
The first contribution  is proportional to $\rho _c$ (equation \eONrhoc)
$$ \omega _d(0)=
{1\over(2\pi)^{d }} \int{\dd k\over k^2 D(k^2)}. \eqnn $$
The constant $K(d)$ is independent of the cut-off procedure
because it involves only the leading behaviour of the integrand for $k\to 0$ and thus $D(0)=1$:
\eqna\eApintNcor
$$ \eqalignno{N_d &={2 \over(4\pi)^{d/2} \Gamma(d/2) } & \eApintNcor{a}
\cr  K(d)&=- { \pi \over2\sin(\pi d/2)}N_d=-{1\over
(4\pi)^{d/2}}\Gamma(1-d/2) \, , & \eApintNcor{b} \cr}
$$  where we have introduced   the usual loop factor
$N_d$.  \par
The constant $a(d)$, which characterizes the leading
correction in equation \etadepolii, depends explicitly on the regularization,
i.e.~the way large momenta are cut,
$$a(d)=\cases{\displaystyle - {1\over(2\pi)^{d }}  \int {\dd k \over k^4} {1\over D^2(k^2)}& for $d>4$ \cr
\displaystyle {1\over(2\pi)^{d }}  \int {\dd k \over k^4}\left(1-{1\over D^2(k^2)} \right) & for $d<4$ \cr}  . \eqnd\eApadef   $$
For $d=4$ the integrals diverge at $k=\infty $. For $\varepsilon=4-d\to 0$
$$ a(d)\mathop{\sim}_{\varepsilon=4-d\to 0} 1/ (8\pi^{2}\varepsilon). \eqnd \eApaespzero $$
A logarithmic contribution then appears in the expansion \eAptadepolii:
$$ \omega _d(z)- \omega _d(0)\sim {1\over8\pi^2}z^2\ln z \,. \eqnd \eAptadepoliv $$
%
\subsection Finite temperature

Let us add a few remarks concerning the calculation of Feynman diagrams.
General methods explained in the
framework of finite size scaling can also be used here, involving
Jacobi's elliptic functions. However, more specific techniques are also
available in finite temperature quantum field theory. The idea is the
following: in the mixed $(d-1)$-momentum, time
representation  the propagator is the two-point function $\Delta(t,p)$ of the
harmonic oscillator with frequency $\omega (p)=\sqrt{p^2+m^2}$ and time
interval $\beta=1/T$:
$${1\over p^2+\omega^2+m^2}\mapsto  \Delta(t,p)={1\over 2 \omega(p)}
{\cosh\left[(\beta/2-|t|)\omega (p)\right]\over \sinh\bigl(\beta\omega
(p)/2\bigr)}. $$
Summing over all frequencies is equivalent to set $t=0$. For the simple
one-loop diagram one finds
$${1\over 2\pi }\int{\d\omega \over \omega^2 +p^2+m^2}\mapsto {1\over 2
\omega(p)} {\cosh\bigl(\beta\omega (p)/2)\bigr)\over \sinh\bigl(\beta\omega
(p)/2\bigr)}, $$
This expression can be written in a way that separates quantum and thermal contributions:
$${1\over 2\omega(p)\tanh(\beta\omega(p) /2)}={1\over 2\omega
(p)}+{1\over \omega
(p)(\e^{ \beta\omega (p)}-1)}, $$
where the first term is the zero-temperature result, and the second term,
which involves the relativistic Bose statistical factor,  decreases
exponentially at large momentum.\par
Finally in the example of fermions or gauge theories we can use a more
general identity that can be proven by replacing the sum by a contour integral,
$$\sum_n{x\over (n+\nu)^2+x^2}=\pi {\sinh(2\pi x)\over \cosh(2\pi
x)-\cos(2\pi\nu)},\eqnd\eFTgenid $$
and thus after integration,
$$\ln\bigl(\cosh(2\pi x)-\cos(2\pi \nu)\bigr)-\ln 2=\lim_{N\to \infty}\sum_{n=-N}^N \left[\ln\bigl((n+\nu)^2+x^2\bigr) -2\ln(n+1/2)\right] .\eqnd\eFTgenidii $$
%

\subsection $\Gamma$, $\psi$, $\zeta$ functions: a few useful identities

Two useful identities on the $\Gamma$ function are \sslbl\appGPZid
$$\sqrt{\pi}\,\Gamma(2z)=2^{2z-1}\Gamma(z)\Gamma(z+1/2)\,,\quad
\Gamma(z)\Gamma(1-z)\sin(\pi z)=\pi\,.\eqnd\eGammadup $$
They translate into a relations for the $\psi(z)$ function,
$\psi(z)=\Gamma'(z)/\Gamma(z)$
$$2\psi(2z)=2\ln 2+\psi(z)+\psi(z+1/2),\quad \psi(z)-\psi(1-z)+\pi
/\tan(\pi z)=0\, \eqnn $$
%The two integrals have been used
%$$\eqalign{\int_{0}^\infty\d s\, s^{\alpha/2-1}\left[A(s)-1\right]&=
%2\Gamma(\alpha/2)\zeta(\alpha), \cr
%\int_{0}^\infty\d s\, s^{\alpha/2-1}\left[A(s)-\sqrt{\pi/s}\right]&=
%2\pi^{\alpha-1/2}\Gamma[(1-\alpha)/2]\zeta(1-\alpha), \cr}
%$$
where $\zeta(s)$ is the standard Riemann $\zeta$-function
$$\zeta(s)=\sum_{n \ge 1}{1\over n^s} \,. \eqnd\ezetadef $$
It is also useful  to remember the reflection formula
$$\zeta(s)\Gamma(s/2)=\pi^{s-1/2}\Gamma\bigl((1-s)/2\bigr)\zeta(1-s),
\eqnd\ezetaref  $$
which can be written in different forms using $\Gamma$-function
relations. Moreover
$$\sum_{n=1}{(-1)^n\over n^s}=(2^{1-s}-1)\zeta(s),\eqnd\ezetamin  $$
Finally
$$\eqalignno{\zeta(1+\varepsilon)&={1\over\varepsilon}-\psi(1)+O(\varepsilon),
&\eqnd\ezetaone \cr
\zeta(\varepsilon)&=-\ud(2\pi)^{\varepsilon}+O(\varepsilon^2). &
\eqnd\ezetanul \cr}$$
%
%
\section Gaussian measure and normal product

We assume that we are given some gaussian measure $\exp[-\ud
\sum_{ij} x_i A_{ij} x_j]$ for a finite or infinite number of
random variables $x_i$. In the following applications the index
$i$ will describe a parameter that can be either discrete (e.g.
group theoretic index) or continuous (e.g. space-time). We denote
the expectation values \sslbl\appnormprod
$$\left<x_{i_i}\ldots x_{i_\ell}\right>={\cal N}({\bf A})
\int\d x\, x_{i_i}\ldots x_{i_\ell}\exp[-\ud \sum_{ij} x_i A_{ij}
x_j],
$$
where the normalization ${\cal N}({\bf A})$ is such that
$\left<1\right>=1$. We denote by $\bf \Delta $ the inverse of the
symmetrix (or operator) $\bf A$ and, therefore,
$$\Delta _{ij}=\left<x_ix_j\right>.$$
We now define the normal ordering of a formal power series in the
variables $x_i$  as a linear map, with the following property: If
$V(x)$ is a homogeneous polynomial of degree $n$, the
normal-ordered polynomial $:V(x):$ is a polynomial of degree $n$
such that
$$V(x)-: V(x) :=O(|x|^{n-2}),$$
and
$$\left<: V(x) :x_{i_1}x_{i_2}\ldots x_{i_\ell}\right>=0 \quad
\ \ \ \ \forall \ell<n\,.$$
The normal ordering amounts to subtract from $V(x)$ all terms
corresponding to self-contractions in the sense of Wick's theorem.
\par
We now establish an explicit expression for the normal order of
any formal power series in the variables $x_i$. First, we consider
an exponential $\exp\left[\sum_i h_i x_i\right]$ and we generate
expectation values by a generating function $\exp\left[\sum_i g_i
x_i\right]$. We, therefore, calculate
$$\eqalign{{\cal Z}(g,h)&={\cal N}({\bf A})
\int\d x\, \exp\left[ -{1\over2} \sum_{ij} x_i A_{ij} x_j+\sum_i
(g_i+h_i)x_i\right]\cr &=\exp\left[{1\over2}\sum_{ij}
(g_i+h_i)\Delta _{ij}(g_j+h_j)\right].\cr } \eqnd\eGenFunc $$ The
suppression of the self-contractions is now easy; we divide by the
value for $g=0$. Indeed, then the remaining dependence in $h$, in
the exponential, is linear in $g$. A non vanishing result can only
be obtained by differentiating at least as many times with respect
to $g$ as to $h$ before taking the $g=h=0$ limit. Therefore,
$$ :\e^{{\bf  h} \cdot{\bf x}}:\,=\e^{-  {\bf h}\cdot{\bf \Delta}{\bf  h}/2} \e^{{\bf  h} \cdot{\bf x}}.$$
We now write any function as a Laplace transform
$$V(x)=\int\d h\, \tilde V(h)\e^{{\bf  h} \cdot{\bf x}}$$
and use the linearity of the normal order operation to get
$$:V(x):=\exp\left[-{ 1\over2}\sum_{ij}\Delta _{ij}{\partial \over\partial x_i}
{\partial \over\partial x_j}\right]V(x).\eqnn $$ Note that
analogous expressions can be derived by the same method for
complex or grassmannian gaussian measures.\par We now discuss a
few applications.
\medskip
{\it  Local polynomials in field theory.} In the exemple of a
gaussian functional measure over a field $\varphi$, $\bf \Delta $
is the propagator. If the field theory is invariant under space
translations the propagator is of the form $\Delta (x-y)$. If $V$
is a local function of a field of the form $V[\varphi(x)]$, the
normal order takes the form
$$ :V(\varphi):= \exp\left[-{ 1\over2} \left<\varphi^2(0)\right> \left( {  \partial \over  \partial \varphi   }\right)^2
 \right]V(\varphi ),\eqnd\enormorderfld $$
justifying equation \eNVnormorder.\par
By acting with $\partial/
\partial \varphi $ on both sides of the equation, we infer
$$  {  \partial \over  \partial \varphi   }
:V(\varphi):= :{  \partial \over  \partial \varphi   }  V(\varphi
):\ \ \ \ \ .$$ Namely, Eq.~\enormorderfld\ implies that the normal
ordering commutes with differentiation. Therefore, the relation
\enormorderfld\ can be inverted as
$$ V(\varphi )=:  \exp\left[ { 1\over2}
\left<\varphi^2(0)\right> \left( {  \partial \over  \partial \varphi   }\right)^2
 \right]  V(\varphi ): \ \ \ \ \ . \eqnd\enormorderInv $$
\smallskip
{\it $O(N)$ invariant theories.} With an $O(N)$ invariant gaussian
measure and $N$ component field $\varphi$ the two-point function
takes the form
 $$\Delta _{ij}(x)=\delta _{ij}\Delta (x)
 ={\delta _{ij}\over N}\left<\varphib(x)\cdot
\varphib(0)\right>.$$ The normal ordered polynomial is then given
by
$$ :V(\varphib):= \exp\left[-{ 1\over2N} \left<\varphib^2(0)\right>\sum_i \left( {  \partial \over  \partial \varphi_i   }\right)^2
 \right]V(\varphi ).\eqnn $$
In case of an $O(N)$ invariant local function of the form
$U[\varphib^2(x)/N]$,
$$\sum_i \left( {  \partial \over  \partial \varphi_i   }\right)^2U(\varphib^2 /N)=
2U'(\varphib^2 /N)+{4\over N^2}\varphib^2 U''(\varphib^2 /N), $$
and, thus, setting $\varphib^2 /N=\rho $:
$$:U(\rho ):=\exp\left\{-{ 1\over2N} \left<\varphib^2(0)\right>
\left[2{\partial \over\partial \rho }+{4\rho \over
N}\left({\partial \over\partial \rho
}\right)^2\right]\right\}U(\rho ). $$ The exponential of a first
order derivative is a translation operator. Moreover,
$\left<\varphib^2(0)\right>/N=\left<\rho \right>$. Therefore, the
normal-ordered polynomial can be rewritten
$$:U(\rho ):=\exp\left[ -{2\over N} \left< \rho \right>\rho
  \left({\partial \over\partial \rho }\right)^2\right] U\bigl(\rho-\left<\rho \right>\bigr ). \eqnn $$
In this way the normal ordered expression can be expanded in
powers of $1/N$. At leading order for $N\to\infty $:
$$:U(\rho ):\mathop{\sim}_{N\to \infty }
 U\bigl(\rho-\left<\rho \right>\bigr ). \eqnn $$
This relation,  when written in the form $:U(\rho+\left<\rho
\right> ):\mathop{\sim}
 U\bigl(\rho)$ immediately implies the
result \eNvarav.


\bye
