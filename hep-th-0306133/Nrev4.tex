%%%%%%%%   Nrev4 as of Feb. 28 2003
\def\ggn{u}
\def\tt{\tau}
\def\GG{G}
\nref\rNaJoLa{A four-fermion interaction with $U(1)$ chiral
invariance was proposed by Nambu and Jona-Lasinio  as a basic
mechanism to generate nucleon, scalar and pseudo-scalar
$\sigma,\pi $ masses: \rf Y. Nambu and G. Jona-Lasinio, {\it Phys.
Rev.} 122 (1961) 345.} \nref\rWilGN{The difficulties connected
with this approach (approximate treatment of Dyson--Schwin\-ger
equations without small parameter,  non renormalizable theory with
cut-off) have been partially solved, the $1/N$ expansion
introduced and the existence of IR fixed points pointed out  in
\rf K.G. Wilson, {\it Phys. Rev.} D7 (1973) 2911 and in ref.
\rGN.} \nref\rGN{D.J. Gross and A. Neveu, {\it Phys. Rev.} D10
(1974) 3235.} \nref\rGNZJ{J. Zinn-Justin, {\it Nucl. Phys.}  B367
(1991) 105.} \nref\rGNeps{Calculations of RG functions in
dimensions 2 and $2+\varepsilon$ have been reported  in \rf N.A.
Kivel, A.S. Stepanenko, A.N. Vasil'ev, {\it Nucl.Phys.} B424
(1994) 619, hep-th/9308073 and refs. \refs{\rDerk{--}\rGrace}.}
%%On Calculation of $2+\veps$ RG Functions in the Gross Neveu
%Model from %%Large $N$ Expansions of Critical Exponents
\nref\rDerk{S.E. Derkachov, N.A. Kivel, A.S. Stepanenko, A.N.
Vasiliev, hep-th/9302034.}\nref\rWenzel{ W. Wentzel, {\it Phys.
Lett.} 153B (1985) 297.} \nref\rGrace{ J.A. Gracey, {\it Nucl.
Phys.} B367 (1991) 657.}

\nref\rDHN{The semi-classical spectrum for $d=2$ in the large $N$
limit of the GN model (with discrete chiral invariance) was
obtained from soliton  calculation in \rf R. Dashen, B. Hasslacher
and A. Neveu, {\it Phys. Rev.} D12 (1975) 2443.} \nref\rZZKT{ This
study as well as some additional considerations concerning the
factorization of $S$ matrix elements at order $1/N$ have led to a
conjecture of the exact spectrum at $N$ finite \rf A.B.
Zamolodchikov and Al.B. Zamolodchikov, {\it Phys. Lett.} 72B
(1978) 481 and also in refs. \refs{\rKaro,\rFor}.} \nref\rKaro{ M.
Karowski and H.J. Thun, {\it Nucl. Phys.} B190 (1981)
61.}\nref\rFor{ P. Forgacs, F. Niedermayer and P. Weisz, {\it
Nucl. Phys.} B367 (1991) 123, 144.} \nref\rLow{ The properties of
the NJL model in two dimensions are discussed in \rf J.H.
Lowenstein, {\it Recent Advances in Field Theory and Statistical
Mechanics}, Les Houches 1982, J.B. Zuber and R. Stora eds.,
(Elsevier Science Pub., Amsterdam 1984).} \nref\rKMV{For rigorous
results see \rf C. Kopper, J. Magnen and V. Rivasseau, {\it Comm.
Math. Phys.} 169 (1995) 121.} \nref\rRVW{Approximate functional RG
has also been used in \rf L. Rosa, P. Vitale, C. Wetterich, {\it
Phys. Rev. Lett.} 86 (2001) 958, hep-th/0007093.} \nref\rGNY{The
relation between the GN and GNY models is discussed in \rf A.
Hasenfratz, P. Hasenfratz, K. Jansen, J. Kuti and Y. Shen, {\it
Nucl. Phys.} B365 (1991) 79 and reference \rGNZJ.} \nref\rKLLP{A
comparison in three dimensions between numerical simulations of
the GN model and the expansion to order $\varepsilon^2$ obtained
from the GNY model is reported in \rf L. K\"arkk\"ainen, R.
Lacaze, P.Lacock and B. Petersson, {\it Nucl. Phys.} B415 (1994)
781, Erratum {\it ibidem} B438 (1995) 650 and
\rFocht.}\nref\rFocht{ E. Focht, J. Jerzak and J. Paul, {\it Phys.
Rev.} D53 (1996) 4616.} \nref\rFFJS{ The models are also compared
numerically in dimension two in \rf A.K. De, E. Focht, W. Franski,
J. Jersak and M.A. Stephanow, {\it Phys. Lett.} B308 (1993) 327.}
\nref\rGNtop{The GN and GNY models are related to the physics of
the top quark condensate. For a  review see for instance \rf G.
Cvetic,  {\it Rev. Mod. Phys.} 71 (1999) 513, hep-ph/9702381.}
\nref\rFNRK{Further RG calculations concerning the GN and NJL
models at $d=4$ can be found in \rf P.M. Fishbane and R.E. Norton,
{\it Phys. Rev.} D48 (1993) 4924 and \rRose.}\nref\rRose{ B.
Rosenstein, H.L. Yu and A. Kovner, {\it Phys. Lett.} B314 (1993)
381.} \nref\rGNDiii{The large $N$ expansion of the GN model in
$d=3$ has been discussed in\rf B. Rosenstein, B. Warr and S.H.
Park, {\it Phys. Rev. Lett.} 62 (1989) 1433 and \rGat.} \nref\rGat{
 G. Gat, A. Kovner and B. Rosenstein,
{\it Nucl. Phys.} B385 (1992) 76.} \nref\rGNDiiirev{For an early
review on the large $N$ expansion of the GN model in $d=3$ see B.
Rosenstein, B. Warr and S.H. Park, {\it Phys. Rep.} 205 (1991)
59.} \nref\rGNGracey{A number of $1/N$  calculations concerning
the GN and NJL models have been reported \rf J.A. Gracey, {\it
Phys. Lett.} B342 (1995) 297, hep-th/9410121;
%%   O(1/N^3) Conformal Bootstrap Solution of the  SU(2) x SU(2)
%% Nambu--Jona-Lasinio Model
{\it Phys. Rev.} D50 (1994) 2840; {\it Erratum-ibid.} D59 (1999) 109904,  hep-th/9406162;
%%: The Beta-Function of the Chiral Gross Neveu Model at O(1/N^2)
{\it Int. J. Mod. Phys.} A9 (1994) 727, hep-th/9306107;
%%Computation of Critical Exponent \eta at O(1/N^3) in the Four Fermi Model in
%%Arbitrary Dimensions
{\it Int. J. Mod. Phys.} A9 (1994) 567,  hep-th/9306106;
%%Computation of $\beta(g_c)$ at O(1/N^2) in the O(N) Gross Neveu Model in
%%Arbitrary Dimensions
{\it Phys. Lett.} B308 (1993) 65, hep-th/9305012.}
%%The Nambu-Jona-Lasinio Model at O(1/N^2)

\nref\rSchwinger{A few references on the Schwinger model and its
relation with the confinement problem: \rf J. Schwinger, {\it
Phys. Rev.} 128 (1962) 2425 and refs. \refs{\rLowen
{--}\rCol}.}\nref\rLowen{ J.H. Lowenstein and J.A. Swieca, {\it
Ann. Phys. (NY)} 68 (1971) 172.} \nref\rCasher{A. Casher, J. Kogut
and L. Susskind, {\it Phys. Rev.} D10 (1974) 732.} \nref\rCJS{ S.
Coleman, R. Jackiw and L. Susskind, {\it Ann. Phys. (NY)} 93
(1975) 267.}\nref\rCol{ S. Coleman, {\it Ann. Phys. (NY)} 101
(1976) 239.} \nref\rQEDRG{ For the calculation of the QED RG
$\beta$ function in the $\overline{\rm MS}$  scheme see \rf S.G.
Gorishny, A.L. Kataev and S.A. Larin, {\it Phys. Lett.} B194
(1987) 429; B256 (1991) 81.} \nref\rNSchwinger{A few references on
Schwinger or QED in the large $N$ limit: \rf J.A. Gracey, {\it
Phys. Lett.} B317 (1993) 415, hep-th/9309092; {\it Nucl. Phys.}
B414 (1994) 614, hep-th/9312055  and refs.
\refs{\rEspriu,\rPalan}.} \nref\rEspriu{ D. Espriu, A.
Palanques-Mestre, P. Pascual and R. Tarrach, {\it Z. Phys} C13
(1982) 153.} \nref\rPalan{A. Palanques-Mestre and P. Pascual, {\it
Comm. Math. Phys.} 95 (1984) 277.} \nref\rThirringeps{An early
calculation for dimension $d=2+ \varepsilon$  in the Thirring
model is found in \rf S. Hikami and T. Muta, {\it Prog. Theor.
Phys.} 57 (1977) 785.} \nref\rNThirring{The Thirring model has
been
 investigated for $N$ large in \rf
 S.J. Hands, {\it Phys. Rev.}
D51 (1995) 5816; hep-th/9411016; hep-lat/9806022.}
\nref\rDuHaMe{Recent simulations concerning the 3D Thirring model
are reported in \rf
 Simon Hands, Biagio Lucini, {\it  Phys. Lett.} B461 (1999) 263,
 hep-lat/9906008 and \rDelDeb.}\nref\rDelDeb{
 L. Del Debbio, S.J. Hands, {\it Nucl. Phys.} B552 (1999) 339, hep-lat/9902014.
}

\section Fermions in the large $N$ limit

We now illustrate the large $N$ techniques that we have started
describing in section \ssNbosgen, with the study of models involving
fermions in the vector representation of the $U(N)$ group.
We first explain how a class of self-interacting fermion models can be solved in the large $N$ limit. When these models have a discrete chiral symmetry
they may exhibit two phases, a symmetric phase with massless fermions, and a massive phase where the symmetry is broken.  The simplest realization is
the Gross--Neveu (GN) model, which we study in more detail \refs{\rNaJoLa,\rWilGN}.
The model  is renormalizable in two dimensions, and
describes in perturbation theory only one phase, the symmetric phase.
We also summarize what can  be learned from RG equations and $d=2-\varepsilon$ expansion.
\par
A  different field theory, the Gross--Neveu--Yukawa (GNY)
model has  the same symmetry, but a different field content since fermions interact through their coupling to a scalar field. The  model is renormalizable in four dimensions and, moreover, allows
a perturbative analysis of the chiral phase transition.
We  recall some properties of the  model  using perturbation theory and
 RG equations at and near four dimensions.  We
then show that additional information can be obtained from a large
$N$ expansion, and that a relation between  the GNY and GN models
follows.\par Finally, we also briefly examine QED and the massless
Thirring model in the large $N$ limit.\par In all examples, one of
the physical issues we  explore is the possibility of spontaneous
chiral symmetry breaking and fermion mass generation.
\sslbl\ssNfermions
%
\subsection Large $N$ techniques and fermion self-interactions

We consider a model characterized by a $U(\tilde N)$ symmetric
action for a set of $\tilde N$ Dirac fermions $\{\psi^i,
\bar\psi^i \}$. Since fermions have also a spin, we concentrate on
the simple examples where interactions involve only the scalar
combination both in the $U(\tilde N)$ and the spin group sense:
$$\bar\psib\cdot \psib\equiv \sum_{\alpha ,i}\bar\psi_\alpha^i \psi_\alpha^i.$$
The action can then be written as
 \sslbl\ssNferself
$${\cal S} (\bar \psib, \psib )= -\int \d^d x \left[ \bar\psib(x)\cdot
\sla{\partial} \psib(x) +NU\bigl(\bar\psib(x)\cdot \psib(x)/N \bigr)
\right],\eqnd \eNactferg $$
where $U $ is a general polynomial potential. We have introduced the notation  $N =\tilde N\tr{\bf 1}$, the matrix ${\bf 1}$ being the identity in the space
of Dirac $\gamma $ matrices and $N$ thus the total number of $\psi$ components. Moreover, a chiral invariant regularization (see Eq.~\epropreg) is assumed:
$$\sla{\partial} \mapsto \sla{\partial}\sqrt {D(-\nabla ^2/\Lambda ^2)}\,.$$
\par From the general discussion of section \ssNbosgen, it is now quite
clear how to solve such a model in the large $N$ limit and how to
generate a systematic large $N$ expansion. We introduce two scalar
fields $\sigma$ and $\rho$ and impose the
constraint $\rho(x)=\bar\psib(x)\cdot \psib(x) /N $ by an integral over $\sigma$.
The partition function can then be written as
$$ {\cal Z}= \int [ \d\psib\d\bar\psib][\d\rho][\d\sigma] \exp \left[-{\cal
S}(\bar\psib,\psib,\rho,\sigma)\right]  \eqnd\eONpartgen $$
with
$${\cal S}(\bar\psib,\psib,\rho,\sigma)=- \int \left[ \bar\psib\cdot
\sla{\partial} \psib +NU\bigl(\rho(x)\bigr) +\sigma(x)
\bigl((\bar\psib(x)\cdot \psib (x)-N \rho(x)\bigr) \right]
\d^{d}x\,. \eqnd\eNactfergii $$
In the representation \eONpartgen,  the integration over fermion fields is gaussian and can be performed, the $N$-dependence of
the partition function becoming explicit. Again it is convenient
to integrate only over $ \tilde N-1 $ components. The action suited for large
$N$ calculations then takes the form (now $\psi\equiv \psib_1$)
$$\eqalignno{{\cal S}_N(\bar\psi,\psi,\rho,\sigma)&=- \int \left[ \bar\psi
\sla{\partial} \psi +NU\bigl(\rho(x)\bigr) +\sigma(x)
\bigl((\bar\psib(x) \psib (x)-N \rho(x)\bigr) \right]
\d^{d}x \cr &\quad -(\tilde N-1)\tr \ln\left(\sla{\partial}+\sigma
\right).&
 \eqnd\eNactfergiii \cr} $$  For  large $N$, the action is
 proportional to $N$  and can be calculated
by the steepest descent method. The action density $\cal E$ for constant fields
$\rho,\sigma=M$ reduces to
$${\cal E}(M,\rho )/N=-U(\rho)+ M \rho-{1\over2}\int^\Lambda{\d^d q\over(2\pi)^d}
\ln[(q^2+M^2)/q^2].\eqnn $$
At this order $M$, the $\sigma $ expectation value, is also the fermion mass.
In the continuum limit (or in the critical domain) it must satisfy the physical condition $|M|\ll \Lambda $.
\par
The saddle point equations, expressed in terms of the function \etadepole, are
\eqna\eNsadferg
$$\eqalignno{M&=U'(\rho),&\eNsadferg{a} \cr
 \rho&={M\over(2\pi)^d}\int^\Lambda {\d^d q \over
q^2+M^2} = M \Omega_d(M) \,.  &\eNsadferg{b}\cr} $$
In terms of the function $\Omega_d$, the action density then reads
$${\cal E}(M,\rho )/N=-U(\rho)+ M \rho-\int_0^M s\d s
\, \Omega _d(s).\eqnd\eNEfer $$ From
Eq.~\eNsadferg{b}, we infer that $\rho /M$ is positive and
that  $\rho $ and $M$ vanish simultaneously for $d>1$.  Moreover,
the  condition $|M|\ll \Lambda $ implies that  $\rho $ is small in
the natural cut-off scale. We can, therefore, expand $U$ for
$\rho$  small:
$$U( \rho  )={\cal M}\rho +\ud G\rho ^2+O(\rho ^3).\eqnd\eNferUgen $$
Using Eq.~\eNsadferg{b} to eliminate $\rho $, we infer from Eq.~\eNsadferg{a}
$$M={\cal M}+G M \Omega_d(M)+O(M^2 \Omega_d^2).$$
In particular, the fermion mass $M$ vanishes when $\cal M$ goes to zero except
if the equation $ G\Omega_d(M)=1$
has a solution. The latter condition implies $G>0$,
that is attraction between fermions. For a repulsive interaction ($G<0$), at ${\cal M}=0$ the saddle point is $\rho =M=0$ and  the fermion mass always vanishes.\par
The value ${\cal M}=0$ is natural if the model has a discrete symmetry
$U(\rho )=U(-\rho )$, which prevents the addition of an explicit fermion mass term. Such a symmetry can be realized in the fermion representation, in the form of discrete chiral transformations in even dimensions, and in odd dimensions is a consequence of parity symmetry. In the case of attractive interactions, the issue one can then address concerns the possibility of spontaneous fermion mass generation, consequence of the spontaneous breaking of the symmetry. This is the question we now discuss at  large $N$.
%
\subsection Discrete chiral symmetry and spontaneous mass generation

We now consider models with a discrete symmetry that prevents the addition of a fermion mass (${\cal M}=0$).
In even dimensions it is a discrete chiral symmetry
($\gamma_S \equiv \gamma_{d+1}$) \sslbl\ssNGNgen
$$\psib \mapsto \gamma_S \psib, \quad\bar \psib \mapsto -\bar\psib \gamma_S\,
, \eqnd\echirald $$
while in odd dimensions  it is simply space reflection. Actually, it is possible to find a unique transformation, which corresponds to space reflection in odd dimensions, and makes sense in all dimensions
$${\bf x}=\{x_1  ,\ldots,x_\mu,\ldots, x_{d }\}\mapsto\tilde {\bf
x}=\{x_1 ,\ldots,-x_\mu,\ldots,x_{d }\},
\quad \cases{\psi({\bf x})\mapsto \gamma_\mu \psi(\tilde{\bf x}), \cr \bar\psi({\bf x})\mapsto -\bar\psi(\tilde {\bf x})\gamma_\mu \cr} .\eqnd\echirpar$$
Then, the potential has the expansion
$$U(\rho)=\ud \GG \rho^2 +O(\rho^4)\ \Rightarrow\ \rho \sim
M/\GG\,,\eqnd\eNferUquad $$
where we have assumed an attractive fermion self-interaction  that  excludes multicritical points ($G>0$) . \par
Then, the gap equation  \eNsadferg{b} has two solutions,
a symmetric solution
$M=\rho =0$ and a solution with non-vanishing mass and broken symmetry,
  $|M|  \ll \rho^{1/(d-1)}\ll \Lambda$,  which satisfies
$$1\sim \GG \Omega_d (M)  . \eqnd\eNGNmassgap $$
\smallskip
{\it Two dimensions.} We  first  examine dimension 2,
which is peculiar.   The solution with non-vanishing mass  and broken symmetry leads to
$$1= \GG \Omega_2(M) \ \Rightarrow \
%$$M\propto \Lambda \e^{-2\pi\rho/ M}.$$
%Since $\rho$ is small compared to $\Lambda$, we can expand $U(\rho)$
%and, therefore,
 M\propto \Lambda \e^{-2\pi/ \GG},$$
where $|M|  \ll \Lambda$ implies that the parameter $\GG$  has to be small enough.
The massive solution has always lower energy density, and is therefore realized (see also the variational analysis at the end of the section).
\smallskip
{\it Higher dimensions.} In   dimensions $d>2$,
the massive phase can exist only if the  ratio $\rho/M$ is smaller than
some critical value
$$\rho/M=  \Omega_d(0).$$
Since  $M/\rho \sim \GG$, this implies the existence of  a critical coupling constant
$$G_c=1/ \Omega_d(0). \eqnn $$  For  $\GG<\GG_c $
the symmetry is unbroken and fermions are
massless: $\rho =M=0$.\par
For $\GG>\GG_c$, instead, both the  trivial symmetric solution
$\rho =M=0$ and a massive solution can be found.
In the broken symmetry phase, the gap equation can  be written as
$$ {1\over \GG_c}-{1\over \GG}  = \Omega_d(0)-\Omega_d(M) .$$
A comparison of energies  then shows that the broken phase has a lower energy
than the symmetric phase  (see also the variational
argument at the end of the section).  For
$\GG>\GG_c$, the symmetry is broken and the fermions are
massive.  At the critical value $\GG_c$,
 a phase transition occurs.
\par
Note that while for $d>2$ the massless symmetric phase  $M=\rho=0$ can exist,  for $d=2$ instead the symmetric solution does
not exist anymore.  This phenomenon is associated with the divergence
of momentum integral diverges at $q=0$ in
the saddle point equation  \eNsadferg{b}.
The mechanism is reminiscent of the Goldstone
phenomenon, but with one difference: here it is the symmetric phase that is
massless and therefore does not exist in low dimensions.
\par
Continuum physics in the broken phase is
possible only if $|M|\ll \Lambda$, that is when $\GG$ is close to its critical value, $  \GG  -  \GG_c\ll \Lambda ^{2-d}$.
 \par
Depending on the position of the dimension with respect to 4,
one then finds
$$\eqalign{M\propto \left({1\over \GG_c}-{1\over \GG}\right)^{1/(d-2)}
&{\rm for}\ d<4\,, \cr M\propto{1\over\sqrt{\ln(\Lambda /M)}}
\left({1\over \GG_c}-{1\over \GG}\right)^{1/2} &{\rm for}\ d=4\,,
\cr M\propto   \left({1\over \GG_c}-{1\over \GG}\right)^{1/2}
&{\rm for}\ d>4 \,.\cr}$$  Finally, we note that in the continuum
limit the function $U(\rho)$ can  always be approximated by the
quadratic polynomial \eNferUquad.

\smallskip
{\it The scalar bound state.} In the quadratic approximation relevant to the continuum limit, we can
integrate over $\rho(x)$. The result of the gaussian integration
amounts to replace $\rho(x)$ by the solution of the field equation
$$\rho(x)=\sigma(x)/\GG\,,\eqnd\eNferquadrs $$
and the action becomes
$${\cal S}_N(\bar\psi,\psi,\sigma)=- \int \left[ \bar\psi
\left(\sla{\partial}+\sigma(x)\right) \psi -N\sigma^2(x)/2\GG \right]
\d^{d}x -(\tilde N-1)\tr \ln\left(\sla{\partial}+\sigma
\right).\eqnd\eNactfergiv  $$
It is then instructive to calculate, at leading order, the $\sigma$
two-point function $\Delta_\sigma$ in the massive phase. Differentiating the action twice
with respect to $\sigma(x)$, and then setting $\sigma(x)=M$, one finds
after some algebra
$$\eqalignno{\Delta^{-1}_\sigma(p)&={N \over \GG} - {N  \over (2\pi)^d }
\int^\Lambda{\d^d q \over q^2+M^2}\cr&\quad  + {N  \over 2 (2\pi)^d }
\left(p^2+4M^2 \right)\int^\Lambda{\d^d q \over \left(
q^2+M^2 \right) \left[(p+q)^2 +M^2 \right]} \cr
&= \ud N   \left(p^2+4M^2 \right) B_\Lambda (p,m), &\eqnd\eNfergsig \cr}
$$
where the definition \ediagbul, the  saddle point equation \eNsadferg{b} and the relation
\eNferUquad~have been used.
We see that the inverse propagator vanishes for $ip=2M$ (we use
euclidean conventions), which means that in the broken phase, in the
 quadratic approximation, one finds
a scalar bound state with mass $2M$ (the two-fermion threshold), independently of the dimension $d$ of  space.
\medskip
{\it Variational calculations.} Here, like for the scalar fields in section \ssNVarfiv, it is possible to relate the large $N$ results to results obtained from variational calculations in the large $N$ limit.
One takes as a variational action
$${\cal S}_0  (\bar \psib, \psib )= -\int \d^d x\,\bar\psib\cdot
(\sla{\partial}+M) \psib \,.$$
The arguments follow directly what has been done in the scalar example.
We introduce the parameter
$$\rho = \left<\bar\psib\cdot \psib\right>_0/N={\tilde N \over N} \tr
\int^\Lambda {\d^d k \over (2\pi)^d}  { M-i\sla{k} \over k^2+M^2}= M \Omega _d(M)\,,\eqnd\eNsadferrho $$
where $\langle\bullet\rangle$ means expectation value with respect to $\e^{-{\cal S}_0}$.\par
Then, the variational energy density ${\cal E}_{\rm var}$ is given by
$${\cal E}_{\rm var}/N=-\left<U(\bar\psib\cdot \psib)/N\right>_0+M \left<\bar\psib\cdot \psib\right>_0/N
-\tr\ln(\sla{\partial }+M)/\tr{\bf 1}\, .$$
Again, in the large $N$ limit,
$$\left<U(\bar\psib\cdot \psib)/N\right>_0\sim U(\rho)  $$
and thus
$${\cal E}_{\rm var}/N= -U(\rho)+ M \rho-\int_0^M s\d s \, \Omega _d(s)\,.
\eqnd\eGNeVAR$$ We recognize Eq.~\eNEfer\ but here $\rho $ and $M$
are related by Eq.~\eNsadferrho\ and thus Eq.~\eNsadferg{b} is
automatically satisfied. We have now to look for the minimum of
${\cal E}_{\rm var}$ as a function of $M$. Except for possible
end-point solutions, which are not relevant here, and taking into
account Eq.~\eNEfer, we recover Eq.~\eNsadferg{a}. \par We can now
examine the behaviour of ${\cal E}_{\rm var}$ near the trivial
symmetric solution $M=0$.  From Eq.~\eGNeVAR, one finds
$${1\over N}{\del{\cal E_{\rm var}}\over\del M }=M(1-G\Omega_d(M))
{\del \{ M\Omega_d(M)\}\over \del M}\,.\eqnd\eGNextrem $$
Thus, for $d>2$,
$${\cal E}_{\rm var}/N\mathop{\sim}_{M\to0}  \ud \Omega _d(0)M^2(1-\GG/\GG_c).$$
Therefore, for $\GG<\GG_c$ the massless symmetric phase has lower
energy, while it is  the massive broken phase for $\GG>\GG_c$.\par
\noindent For $d=2$,
$${\cal E}_{\rm var}/N\mathop{\sim}_{M\to0} -\ud G M^2\Omega _2^2(M)$$
and, therefore, the broken phase is always lower.
%
\subsection The Gross--Neveu model

In the general discussion of the fermion self-interaction in the large
$N$ limit, we have  seen that interesting physics can be
studied by considering only a quartic fermion interaction. Then,\sslbl\ssNGNmod
$${\cal S} (\bar \psib, \psib )= -\int \d^d x \left[ \bar\psib\cdot
\sla{\partial} \psib +\frac{1}{2N}\GG\left(\bar\psib\cdot \psib \right)^2 \right].\eqnd\eGNact $$
This characterizes the Gross--Neveu (GN) model which we now study in more detail.
The model illustrates, for $\GG>0$, the mechanism of spontaneous mass
generation and, in even dimensions, chiral symmetry breaking.
It is renormalizable and asymptotically free in two dimensions.
However, as in the case of the non-linear $\sigma$ model, the perturbative
GN model describes only one phase.  \par
Since in the GN model the symmetry breaking mechanism is non-perturbative,
it will eventually be instructive to compare it with a different model
with the same symmetries, but where the mechanism is perturbative: the
Gross--Neveu--Yukawa model.
%
\smallskip
{\it RG equations in two and near two dimensions.} The GN model is renormalizable in two dimensions, and in perturbation
theory describes only the massless symmetric phase.
Perturbative calculations in two dimensions can be made with an IR cut-off
of the form of a mass term ${\cal M}\bar\psi\psi$, which breaks softly the chiral
symmetry. It is possible to use dimensional regularization in
practical calculations. \par
Note that in two dimensions the symmetry group
is really $O( N)$, as one verifies after some relabelling of the
fields. Therefore, the $(\bar \psi \psi)^2$ interaction is
multiplicatively renormalized. \par
In generic non-integer dimensions $d>2$, the situation is more complicated because the algebra of $\gamma$ matrices is infinite-dimensional and an infinite number of four-fermion interactions mix under renormalization. The coupling \eqns{\eNGNefg} thus has the interpretation of
a coupling constant that parametrizes the RG flow that joins the gaussian fixed point to the non-trivial UV fixed point.
 Its flow equation is obtained by first eliminating all other couplings.
This remark is important from the point of view of explicit calculations  in a $d-2$ expansion, but because the problem does not appear at leading order, it does not affect the analysis and we  disregard here this subtlety.\par
It is convenient to introduce here a
dimensionless coupling constant
$$\ggn=G\Lambda^{2-d}/N. \eqnd\eNGNefg $$
As  a function of the cut-off $\Lambda$, the bare vertex functions
satisfy the RG equations  \rGNZJ.
$$\left[ \Lambda{ \partial \over \partial
\Lambda} +\beta (\ggn){\partial \over \partial \ggn}-{n \over
2}\eta_{\psi}(\ggn)-\eta_{\cal M}(\ggn){\cal M}{\partial \over \partial {\cal M}} \right]
 \Gamma^{(n)}\left(p_{i};\ggn,{\cal M},\Lambda \right)=0\, .\eqnd{\eRGGN}$$
A direct calculation of the $\beta$-function in $d=2+\varepsilon$
dimension yields \rGNeps
$$\beta(\ggn)=\varepsilon \ggn-(N-2){\ggn^2 \over
2\pi}+(N-2){\ggn^3 \over4\pi^2}+{(N-2)(N-7)\over 32\pi^3}\ggn^4
+O\left(\ggn^5\right),\eqnd \ebetGNii $$ Here $N=\tilde N\tr{\bf
1}$ is the number of fermion degrees of freedom and thus for $d=2$
$N=2 \tilde N$.\par The special case $N=2$, for which the
$\beta$-function vanishes identically in two dimensions,
corresponds to the Thirring model (since for $N=2$ ~,~ $(\bar\psi
\gamma_{\mu}\psi)^2=-2 (\bar \psi\psi )^2$). The latter model is
also equivalent to the sine-Gordon or the $O(2)$ vector model.

\par Finally, the field and mass RG functions, at the presently
available, order are
$$\eqalignno{\eta_{\psi}(\ggn)&={N-1\over 8 \pi^2}\ggn^2-{(N-1)(N-2) \over
32\pi^3} \ggn^3+{(N-1)(N{}^2-7N+7)\over 128\pi^4}\ggn^4 , \hskip10mm &\eqnd\egamii \cr
\eta_ {\cal M} (\ggn)&={N-1\over 2\pi}\ggn-{N-1\over
8\pi^2}\ggn^2-{(2N-3)(N-2)\over 32\pi^3} \ggn^3+O(\ggn^4). & \cr}
$$
\smallskip
{\it Repulsive interactions.} If $u$ is negative, the form of the $\beta $-function shows that the model is IR free for any dimension $d\ge 2$ (at least for $u$ small enough), chiral symmetry is never broken and, for ${\cal M}=0$,
fermions remain massless. In dimension 2, one finds a gaussian behaviour
modified by logarithms. In higher dimensions the theory is gaussian.
Therefore, in what follows we discuss only the
situation of  an attractive self-interaction, that is $u>0$.
\medskip
{\it Attractive interactions.}
As in the case of the non-linear $\sigma$  model, it is then convenient to express the solutions of the RG
equations \eRGGN~in terms of  a RG invariant mass scale $\Lambda(\ggn)$ or its inverse, a length scale $\xi$ of the type of a correlation
length,
$$\xi^{-1}(\ggn)\equiv \Lambda(\ggn) \propto \Lambda\exp\left[-\int^\ggn{\d
\ggn'\over \beta(\ggn')}\right]\,. \eqnd\eferMass $$
%
We then have to consider separately dimension 2, which is special, and higher dimensions.
\smallskip
{\it Two dimensions} \refs{\rDHN{--}\rKMV}. For $d=2$, the model is UV asymptotically
free. In the chiral limit (${\cal M}=0$) the spectrum, then, is
non-perturbative, and a number of arguments lead to the conclusion
that the chiral symmetry is always broken and a fermion  mass
generated. From the statistical point of view, this corresponds to
a gap in the spectrum of fermion excitations (like in a
superconductor). All masses are proportional to the mass parameter
$\Lambda(\ggn)$ defined in equation \eferMass. For $\ggn$ small
$$\Lambda(\ggn)\propto \Lambda
\ggn^{1/(N-2)}\e^{-2\pi/(N-2)\ggn}\bigl(1+O(\ggn)\bigr)\,. \eqnn $$
We see that the continuum limit, which is reached when the masses are
small compared to the cut-off, corresponds to $\ggn\to 0$.\par
$S$-matrix considerations have then led to the conjecture that, for $N$
finite, the spectrum is
$$m_n=\Lambda(\ggn) {(N-2) \over \pi}\sin\left({n\pi \over
N-2}\right),\quad
n=1,2\ldots <N/2\,,\ N>4\,. $$
The  first values of $N$ are special, the model $N=4$ is
conjectured to be equivalent to two decoupled sine-Gordon models.\par
To each mass value corresponds a representation of the $O( N)$ group.
The nature of the representations leads to the conclusion that $n$ odd
corresponds to fermions and $n$ even to bosons.\par
This result is consistent with the spectrum for $N$ large evaluated by
semi-classical methods. In particular, the ratio of the masses of the
fundamental fermion $\psi$ and the lowest lying boson $\sigma$ is
$${m_{\sigma}\over m_{\psi}}=2\cos\left({\pi \over N-2}\right)=2+O(1/N^2).
\eqnd{\eGNspec}$$
Note that the results about breaking of chiral symmetry, the coupling
 constant dependence of the mass scale, and the ratio of \eGNspec~are
completely consistent with the large $N$ results
found in sections \label{\ssNferself,\ssNGNgen}.
%
\smallskip
{\it Dimension $d=2+\varepsilon$}  \rGNZJ.
As in the case of the $\sigma$-model, asymptotic freedom implies the
existence of a non-trivial UV fixed point $\ggn_c$  in $2+\varepsilon$
dimension:
$$\ggn_c={2\pi \over N-2}\varepsilon\left(1-{\varepsilon \over N-2}\right)
+O\left(\varepsilon^3\right). $$
$\ggn_c$ is also the critical coupling constant for a transition
between a
phase in which the chiral symmetry is spontaneously broken and a massless
small $\ggn$ phase.  \par
Setting $u=u_c$ in the RG functions, one infers the correlation length exponent $\nu$:
$$\nu^{-1}=-\beta'(\ggn_c)=\varepsilon-{\varepsilon^2 \over N-2}
+O\left(\varepsilon^3\right),\eqnd{\expnuii}$$
and the fermion field dimension $[\psi]$:
$$2[\psi]=d-1+\eta_{\psi}(\ggn_c)= 1+\varepsilon+ {N-1 \over
2(N-2)^2}\varepsilon^2 +O\left(\varepsilon^3\right) .\eqnd\expetaii $$
The dimension of the composite field $\sigma=\bar\psib\psib$ is given by
$$[\sigma]=d-1-\eta_M(\ggn_c)=1-{\varepsilon \over N-2}+O(\varepsilon^2). $$
As for the $\sigma$-model, the existence of a non-trivial UV fixed point
implies that large momentum behaviour is not given by perturbation theory
above two dimensions. This explains why the perturbative result that
indicates that  the model cannot be renormalized in higher dimensions, cannot
be trusted.
However, to investigate whether the $\varepsilon$ expansion makes
sense beyond an infinitesimal neighbourhood of dimension two, other methods are required  \rRVW, like the $1/N$ expansion, which is discussed in sections
\ssNferself, \ssNGNgen, \label{\sssGNYN}. \par
%%%%%%%%%%%%%%
\subsection The Gross--Neveu--Yukawa model

The Gross--Neveu--Yukawa (GNY)  and the GN models have the
same chiral and $U(\tilde N)$ symmetries. The action of the
GNY model is (now $\varepsilon=4-d$) \refs{\rGNY{--}\rFFJS}
$${\cal S} (\bar \psib, \psib,\sigma )=\int \d^d x \left[-
\bar\psib\cdot \left(\sla{\partial}+g\Lambda^{\varepsilon/2}\sigma
\right) \psib +\ud\left(\partial_{\mu}\sigma\right)^2+\ud m^2
\sigma^2+{\lambda \over 4!} \Lambda^{\varepsilon}\sigma^4
\right],\eqnd{\eactGNg} $$ where $\sigma$ is an additional scalar
field, $\Lambda$ the momentum cut-off, and $g,\lambda$
dimensionless ``bare",  that is effective coupling constants at
large momentum scale $\Lambda$.\par The action still has a
reflection symmetry, $\sigma$ transforming into $-\sigma$ when the
fermions transform by \echirald. In contrast with the GN model,
however, the chiral transition can  be discussed here by
perturbative methods. An analogous situation has already been
encountered when comparing the $(\phib^2)^2$ field theory with the
non-linear $\sigma$ model. An additional analogy is provided by
the property that the GN model is renormalizable in dimension 2
and the GNY model in four dimensions.
\medskip
{\it The phase transition.} Examining the action \eactGNg, we see that in the
tree approximation when $m^2$ is negative the chiral symmetry is spontaneously
broken. The $\sigma$ expectation value gives a mass to the fermions, a
mechanism reminiscent of the Standard Model of weak-electromagnetic
interactions,
$$m_\psi =g\Lambda ^{\varepsilon/2}\left<\sigma \right>,\eqnn $$
while the $\sigma$ mass  is then
$$m^2_{\sigma}={\lambda \over 3g^2}m^2_{\psi}\,.\eqnd \eratiobf $$
Interactions modify the transition value $m^2_c$ of the parameter
$m^2$ and thus in what follows we set
$$m^2=m^2_c+\tt \,. \eqnn $$
The new parameter $\tt$ plays, in the language of phase
transitions, the role of the deviation from the critical
temperature.\par In order to study the model beyond the tree
approximation, we discuss now shortly RG equations near four
dimensions.
%
\subsection RG equations near  dimension 4

The model \eactGNg\ is trivial above four dimensions,
renormalizable in four dimensions and can thus be studied near
dimension 4 by RG techniques. Five renormalization constants are
required, corresponding to the two
%Calling $\mu$ the renormalization scale, setting $d=4-\varepsilon$, we can
%write the renormalized action:
%$$\eqalignno{ {\cal S}_\r\left(\bar \psib, \psib,\sigma\right)& =\int \d^d x
%\biggl[- Z_{\psi}\bar\psib\cdot\sla{\partial} \psib
%-\Lambda^{\varepsilon/2}gZ_\psi\sigma \bar\psib\cdot \psib & \cr
%&\quad +\ud Z_{\sigma}\left(\partial_{\mu}\sigma\right)^2+\ud (Z_\sigma m^2_c
%+ Z_m t) \sigma^2+\Lambda^{\varepsilon} Z^2_\sigma{\lambda \over 4!}
%\sigma^4 \biggr],\hskip10truemm&\eqnd{\eactGNr} \cr}$$
%where $m^2_c$ is the critical bare mass squared, the critical temperature in
%statistical language, and $t$ characterizes the deviation from the critical
%temperature.
field renormalizations, the $\sigma$ mass, and the two coupling
constants. The RG equations thus involve five RG functions. The
vertex functions $\Gamma^{(\ell,n)}$, for $l$ $\psi$ and $n$
$\sigma$ fields, then satisfy
$$\left(\Lambda{\partial\over \partial \Lambda}+\beta_{g^2}{\partial \over
\partial
g^2}+\beta_{\lambda}{\partial \over \partial\lambda} -\ud \ell\eta_{\psi}-\ud
n\eta_{\sigma}-\eta_m \tt{\partial \over \partial \tt}
\right)\Gamma^{(\ell,n)}=0\,.
\eqnd{\eRGiv}$$
%\midinsert
%\epsfxsize=100.mm
%\epsfysize=45.mm
%\centerline{\epsfbox{fig10-3.eps}}
%\figure{3.mm}{One-loop diagrams: fermions are represented by solid %lines.}
%\figlbl\figNferonel
%\endinsert
%\def\efigvi{6}
%\medskip
%{\it The RG functions.}
The RG functions at one-loop
order are
%involve the calculation of the diagrams of figure \figNferonel. One finds:
$$\eqalignno{\beta_{\lambda}&= -\varepsilon \lambda+{1\over 8\pi^2}
\left({3\over 2}\lambda^2+N\lambda g^2-6N g^4\right), &\eqnd \ebetl  \cr
\beta_{g^2}&= -\varepsilon g^2+{ N+6 \over16\pi^2} g^4, &\eqnd \ebetgde
\cr}$$
where $N=\tilde N\tr{\bf 1}$ is the total number of fermion components. In four
dimensions $\tr {\bf 1}=4$ and thus $N=4\tilde N$.
\smallskip
{\it Dimension 4:} In  dimension 4, the origin
$\lambda=g^2=0$ is IR stable. Indeed  Eq.~\ebetgde\ implies that
$g$ goes to zero, and  Eq.~\ebetl \ then implies that also
$\lambda$ goes to zero. As a consequence, if the bare coupling
constants are generic,  that is if the effective couplings at
cut-off scale are of order 1, the effective couplings at scale
$\mu\ll\Lambda$ are small and become asymptotically independent
from the initial bare couplings. One finds
$$g^2(\mu)\sim {16\pi^2 \over ( N+6)\ln(\Lambda/\mu)}\,,\quad
\lambda(\mu)\sim {16\pi^2 \over
\ln(\Lambda/\mu)}R \eqnd\eGNYgrunas $$ % bug 8 \to 16
where we have defined
$$R={24 N\over ( N+6)\left[( N-6)+\sqrt{ N^2+132N+36}\right]}.
\eqnn $$
This observation allows using renormalized perturbation theory to calculate
physical observables. For example, we can evaluate the ratio between the
masses of the scalar and fermion fields. To minimize quantum corrections we take  for $\mu$ a value of order $\langle \sigma\rangle$. A remarkable consequence
follows: the ratio \eratiobf~of scalar and fermion masses is fixed \refs{\rGNZJ,\rGNtop,\rFNRK}:
$${m^2_{\sigma}\over m^2_{\psi}}={\lambda_* \over
3g_*^2}={8N\over ( N-6)+\sqrt{ N^2+132N+36} }, \eqnd\emassratio $$
while in the classical limit it is arbitrary. Note that in the large $N$ limit
$${m _ \sigma \over m _ \psi }=2\left(1-\frac{15}{N}\right) +O(1/N^2).$$
The ratio has the same limit 2 as in two dimensions (Eq.~\eGNspec), a result that will be
explained by the study of the large $N$ limit in section \label{\sssGNYN}.
\par
Of course, if
the initial bare couplings are ``unnaturally'' small, the ratio
$\Lambda/\mu$ may not be large enough for the asymptotic regime
\eGNYgrunas\ to be reached. The renormalized couplings at scale
$\mu$ may then be even smaller than in Eq.~\eGNYgrunas\ and the
ratio will remain arbitrary.
\smallskip
{\it Dimension $d=4-\varepsilon$.}
One then finds a non-trivial IR fixed point
$$g_*^{2}={16\pi^2\varepsilon \over N +6},\quad \lambda_*=16\pi^2
R\varepsilon \,.\eqnd{\estar}$$
The  matrix of derivatives of the $\beta$-functions has two positive
eigenvalues $\omega _1,\omega _2$,
$$0<\omega_1=\varepsilon < \omega_2=\varepsilon
\sqrt{N^2+132N +36}/(N +6),
\eqnn $$
and thus the fixed point is IR stable.
%The first  eigenvalue is always the smallest.
\par
The field renormalization RG functions at the same order are
$$\eta_{\sigma}={N\over 16\pi^2}  g^2,\qquad
\eta_{\psi}= {1 \over 16\pi^2}  g^2 .\eqnn $$
At the fixed point one finds
$$\eta_{\sigma}={N\varepsilon\over N+6},\quad \eta_{\psi}={\varepsilon
\over (N+6)} , \eqnd\expeps $$
and thus the dimensions $d_\psi$ and $d_\sigma$ of the fields:
$$d_\psi={3\over2}-{N+4 \over2(N+6)}\varepsilon\,,\qquad
d_\sigma=1-{3\over N+6}\varepsilon\,. \eqnn $$
The  RG function $\eta_m$ corresponding
to the mass operator is at one-loop order:
$$\eta_m=-{\lambda \over 16\pi^2}-\eta_{\sigma}\,,$$
and the correlation length exponent $\nu$ is given by % bug lambda^* \to 2R
$${1\over\nu}=2+\eta_m =2- \varepsilon R-  {N\varepsilon\over N+6}=  2-\varepsilon{5N+6+\sqrt{N^2+132N+36}
\over6(N+6)}.\eqnn $$Finally, we can evaluate the ratio of masses
\eratiobf\ at the fixed point:
$${m^2_{\sigma}\over m^2_{\psi}}={\lambda_* \over
3g_*^2}={8N\over (N-6)+\sqrt{N^2+132N+36}} .$$
In $d=4$ and $d=4-\varepsilon$, the existence of an IR fixed point has the
same consequence: if we assume that the $\sigma$ expectation value is much
smaller than the cut-off and that the coupling
constants are generic at the cut-off scale, then {\it the ratio  of fermion
and scalar masses is fixed}.
%
\subsection GNY and GN models in the large $N$ limit

We now show that the GN model plays with respect to the GNY model
\eactGNg~the role the non-linear $\sigma$-model plays with respect
to the $\phi^4$ field theory \rGNZJ. For this purpose we start
from the action \eactGNg~of the GNY model
%The large $N$ behaviour of the expressions of section
%. For example
%we find $\lambda_*\sim 48\pi^2/N$ and $d-2+\eta_{\sigma}=2$. This reminds us
%the $(\phib^2)^2$ field theory and suggests a study of the large $N$ limit.
and integrate over $\tilde N-1$ fermion fields. We also rescale
for convenience $\Lambda^{(4-d)/2}g\sigma$ into $\sigma$, and then
get the large $N$ action \sslbl\sssGNYN
$$\left.\eqalign{ {\cal S}_N (\bar \psi, \psi,\sigma )&=\int
\d^d x \left\{-
\bar\psi \left(\sla{\partial} +\sigma \right) \psi
+\Lambda^{d-4}\left[{1\over2g^2} \bigl(
 \left(\partial_{\mu}\sigma\right)^2+m^2\sigma^2
 \bigr)+{\lambda\sigma^4\over 4!g^4}
 \right]\right\}\cr &\quad -(\tilde N-1)\tr \ln\left(\sla{\partial}+\sigma
\right).\cr}\right.\eqnd\eactefGN  $$
We take the large $N$ limit with $Ng^2$, $N\lambda$ fixed. When
$\sigma$ is of order one, the action is of order $N$ and can be
calculated by the steepest descent method.\par
We denote by ${\cal E}(\sigma)$ the action density for constant field
$\sigma(x)$ and vanishing fermion fields:
$$\eqalignno{{\cal
E}(\sigma)&=\Lambda^{d-4}\left({m^2\over2g^2}\sigma^2+{\lambda\over
4!g^4} \sigma^4 \right) -{\tilde N}\tr
\ln\left(\sla{\partial}+\sigma \right) \cr
&=\Lambda^{d-4}\left({m^2\over2g^2}\sigma^2+{\lambda\over 4!g^4}
\sigma^4 \right) -{ N\over 2}\int^\Lambda{\d^d q\over
(2\pi)^d}\ln[(q^2+\sigma^2)/q^2] .&\eqnn \cr}$$
 The expectation
value of $\sigma$ for $N$ large is a solution to  the {\it gap}\/
equation
$${\cal E}'(\sigma)\Lambda^{4-d}={m^2\over g^2} \sigma+{\lambda \over
6g^4}\sigma^3-N\Lambda^{4-d} \sigma\,  \Omega _d(\sigma )=0\,,\eqnd\evac $$
where, again, we have introduced the function \etadepole.
It is also useful to calculate the second derivative to check
stability of the extrema:
$${\cal E}''(\sigma)\Lambda^{4-d}={m^2\over g^2}+{\lambda \over
2g^4}\sigma^2+N \Lambda^{4-d} \int^\Lambda
{\d^d q\over (2\pi)^d} {\sigma^2-q^2 \over( q^2+\sigma^2)^2}.$$
The solution $\sigma=0$ is stable provided
$${\cal E}''(0)>0\ \Leftrightarrow\ {m^2\over g^2}>N \Lambda^{4-d} \Omega _d(0) .$$
Instead, the non-trivial solution to the gap equation exists only
for
$${m^2\over g^2}<N \Lambda^{4-d}\Omega _d(0) ,$$
but then it is stable. We conclude that the bare mass $m_c$ given by
$${m^2_c\over g^2}=N  \Lambda^{4-d}\Omega _d(0),\eqnd\eTcrit $$
is the critical bare mass (the analogue of the critical temperature for a
classical statistical system) where a phase transition occurs.
The expression  shows that the fermions favour the
chiral transition. In particular when $d$ approaches 2, we observe that
$m^2_c\to +\infty$ which implies that the chiral symmetry is always broken in
two dimensions. Using Eq.~\eTcrit~and setting
$$\tt=\Lambda^{d-4}(m^2-m^2_c)/g^2 , \eqnn $$
we can write the equation for the non-trivial solution as
$$\tt +\Lambda^{d-4}{\lambda \over 6g^4}\sigma^2+N\bigl( \Omega _d(0)-\Omega _d(\sigma )\bigr)=0\,.$$
We now expand $\Omega _d$  for $\sigma$ small (see Eq.~\etadepolii): $$\Omega _d(0)-\Omega _d(\sigma)=K(d)\sigma^{2-\varepsilon}
-a(d)\Lambda^{-\varepsilon}\sigma ^2 + O \left({\sigma^4/
\Lambda^{2+\varepsilon}} \right). \eqnn $$
Keeping only the leading terms for $\tt\to 0$, we obtain for $d<4$ the
scaling behaviour
$$  \sigma\sim \bigl(-\tt/NK(d)\bigr)^{1/(d-2)}.\eqnd{\expbet} $$
Since, at leading order, the fermion mass $m_{\psi}=\sigma$, it
follows  immediately that the exponent $\nu$ is also given by
$$\nu\sim \beta\sim 1/(d-2)\ \Rightarrow\
\eta_{\sigma}=4-d \ \Leftrightarrow \ d_\sigma =\ud(d-2+\eta_{\sigma})=1 \,.\eqnd \eGNNnu $$
At leading order  for $N\to\infty$, $\nu$ has the same value as in the
non-linear $\sigma$-model.\par
At leading order in the scaling limit, the  thermodynamic potential density  then becomes
$${\cal G}(\sigma)= \ud \tt\sigma^2+(N/d)K(d)|\sigma|^d .\eqnd\efpotGN $$
We note that, although in terms of the $\sigma$-field the model
has a simple Ising-like symmetry, the scaling equation of state
for $N$ large is different. This reflects the property that the
fermions become massless at the transition and thus do not
decouple.\par We also read from the large $N$ action that at this
order $\eta_{\psi}=0$. Finally, from the large $N$ action we can
calculate the $\sigma$-propagator at leading order. Quite
generally, using the saddle point equation,  one finds for the
inverse $\sigma$-propagator in the massive phase
$$\eqalignno{\Delta^{-1}_{\sigma}(p)&=\Lambda^{d-4}\left({p^2\over
g^2}+{\lambda \over3g^4 }\sigma^2\right) \cr&\quad + {N
\over 2 (2\pi)^d }\left(p^2+4\sigma^2 \right)\int^\Lambda{\d^d q \over \left(
q^2+\sigma^2 \right) \left[(p+q)^2 +\sigma^2 \right]}.&\eqnd\eGNsigprop \cr}$$
We see that in the scaling limit $p,\sigma\to 0$, the integral yields the
leading contribution. Neglecting corrections to scaling, we find that
the propagator vanishes for $p^2=-4\sigma^2$ which is
just the $\bar\psi\psi$  threshold. Thus, in this limit,  $m_{\sigma}=2
m_{\psi}$ in all dimensions, a result consistent with the $d=2$ and $d=4$ exact
values (Eq.~\eqns{\eGNspec,\emassratio}).\par
At the transition the propagator reduces to
$$\Delta_\sigma\sim {2 \over N b(d)  p^{d-2}} \eqnn
$$
with (Eq.~\econstb)
$$ b(d)=-{\pi \over\sin(\pi d/2)}
{\Gamma^2 (d/ 2) \over \Gamma (d-1)}N_d \,. \eqnn $$ The result is
consistent with the value of $\eta_{\sigma}$ in Eq.~\eGNNnu.
\par Finally, we note that the behaviour of the propagator at the
critical point, $\Delta_\sigma(p) \propto p^{2-d}$, implies that
the field $\sigma$ from the point of view of the large $N$
expansion, for $2\le d\le 4$, has the canonical dimension
$$[\sigma]=1\,. \eqnn $$
\medskip
{\it Corrections to scaling and the IR fixed point.}  The IR fixed point is
determined by demanding the cancellation of the leading corrections to
scaling. In the example of the action density ${\cal E}(\sigma)$, the leading correction to scaling is proportional to
$$\left({\lambda\over4! g^4}-{N a(d)\over 4}\right)\sigma^4 ,$$
($ a(d)\sim{1 / 8\pi^{2}\varepsilon}
$).
We now assume $a(d)>0$, otherwise we are led to problems analogous to those already discussed in section \sssEGRN. Demanding the
cancellation of the coefficient of $\sigma^4$, we obtain a relation between
$\lambda$ and $g^2$,
$$ g_*^4={\lambda_*\over 6N a(d)}
={4\lambda_*\varepsilon \pi^2\over 3N}+O\left(\varepsilon^2\right).$$
In the same way, it is possible to calculate the leading correction to the
$\sigma$-propagator \eGNsigprop. Demanding the cancellation of the leading
correction, we obtain
$${p^2\over g^2_*}+{\lambda_* \over3g_*^4}\sigma^2 -\ud N
\left(p^2+4\sigma^2\right)a(d)=0\,.$$
The coefficient of $\sigma^2$ cancels from the previous relation and
the cancellation of the coefficient of $p^2$ implies
$$g_*^2={2\over Na(d)}
={16\pi^2\varepsilon\over N}
+O\left(\varepsilon^2\right) \ \Rightarrow \
\lambda ^*={192 \pi^2 \varepsilon \over N}\,,$$
in agreement with the $\varepsilon$-expansion for $N$ large.
\medskip
{\it  The relation to the GN model for dimensions $2\le d\le 4$.}
In several examples we have observed that the contributions coming
from the terms $(\partial_{\mu}\sigma)^2$ and $\sigma^4$ in the large
$N$ action could be neglected in the IR critical region for $d\le 4$.
Power counting confirms this property because both terms have a
canonical dimension $4>d$ and are therefore irrelevant.
  We recognize a situation already encountered in
the $(\phib^2)^2$ field theory in the large $N$ limit. In the scaling region
it is possible to omit them and one then finds the action
$${\cal S}_N (\bar \psib, \psib,\sigma )=\int \d^d x \left[-
\bar\psib\cdot \left(\sla{\partial}+\sigma \right) \psib
+\Lambda^{d-4}{ m^2\over2g^2} \sigma^2 \right].\eqnd\eactGN $$
The gaussian integral over the $\sigma$ field can then  be performed explicitly
and yields the action of the GN model
$${\cal S}_N (\bar \psib, \psib )=-\int\d^d x  \left[\bar\psib\cdot
\sla{\partial} \psib +{\Lambda^{4-d} \over
2m^2}g^2\left(\bar\psib\cdot \psib \right)^2 \right].$$ The GN
(with an attractive interaction) and GNY models are thus
equivalent for  large distance physics, that is in the continuum
limit. Again, the arguments above rely on the condition that the
effective coupling constants at cut-off scale are generic. In the
GN model, in the large $N$ limit, the $\sigma$ particle, simply
appears as a $ \bar\psi\psi$ bound state at threshold  \refs{\rGNDiii,\rGNDiiirev}.\par
One may
then wonder whether the corrections to scaling are different.
Indeed superficially it would seem that the GN model depends on a
smaller number of parameters than the GNY model. Again this
problem is only interesting in four dimensions where corrections
to the leading contributions vanish only logarithmically. However,
if we examine the divergences of the term $\tr\ln
\left(\sla{\partial}+\sigma \right)$ in the effective action
\eactefGN\ relevant for the large $N$ limit, we find a local
polynomial in $\sigma$ of the form
$$\int\d^4x\left[A \sigma^2(x)+B\left(\partial_{\mu}\sigma\right)^2 +C
\sigma^4(x)\right].$$ Therefore, the value of the determinant can
be modified by a local polynomial of this form by changing the way
the cut-off is implemented: additional parameters, as in the case
of the non-linear $\sigma$-model, are hidden in the cut-off
procedure. Near two dimensions these operators can be identified
with $(\bar\psi\psi)^2, [\partial_{\mu} (\bar\psi\psi)]^2,
(\bar\psi\psi)^4$. It is clear that by changing the cut-off
procedure, we change the amplitude of higher dimension operators.
These bare operators in the IR limit have a component on all lower
dimensional renormalized operators.  \par Finally, note that we
could have added to the GNY model an explicit breaking term linear
in the $\sigma$ field, which becomes a fermion mass term in the GN
model, and which would have played the role of the magnetic field
of ferromagnets.
%
\subsection The large $N$ expansion

Using the large $N$ dimension of fields and power counting
arguments, one can then prove that the $1/N$ expansion is
renormalizable with arguments quite similar to those presented in
section \sssfivNRT\ \refs{\rZJTai, \rbook,\rGNDiiirev}.
\medskip
{\it Alternative theory.} To prove that the large $N$
expansion is renormalizable, one proceeds as in the case of the scalar theory
in section \sssfivNRT. One starts from a critical action with an additional
term quadratic in $\sigma$ which generates the large $N$ $\sigma$-propagator
already in perturbation theory:
$${\cal S}(\psi,\bar\psi,\sigma)=\int \d^d x\left[-\bar\psi
(\sla{\partial}+\sigma)\psi +{1\over
2\vv^2}\sigma(-\partial^2)^{d/2-1}\sigma\right] .\eqnn $$
The initial theory  is recovered in the limit $\vv\to\infty$. One then
rescales $\sigma$ in $\vv \sigma$. The model is renormalizable without
$\sigma$ field renormalization because divergences generate only local
counter-terms. The renormalized action then reads
$${\cal S}_\r(\psi,\bar\psi,\sigma)=\int \d^d x\left[-Z_\psi\bar\psi
(\sla{\partial}+\vv_\r Z_\vv\sigma)\psi +{1\over
2}\sigma(-\partial^2)^{d/2-1}\sigma\right] .\eqnn $$
RG equations follow:
$$\left[\Lambda {\partial \over \partial \Lambda}+
\beta_{\vv^2}(\vv){\partial \over \partial \vv^2}-{n\over
2}\eta_\psi(\vv)\right] \Gamma^{(l,n)}=0\,.\eqnn $$
Again, the large $N$ expansion is obtained by first summing the bubble
contributions to the $\sigma$-propagator. We define
$$D(\vv)={2\over b(d)}+N \vv^2 .$$
Then, the $\sigma$ propagator for $N$ large reads
$$\left<\sigma\sigma\right>={2\over b(d)D(\vv) p^{d-2}}.\eqnn $$
The solution to the RG equations can be written as
$$\Gamma^{(l,n)}(\ell p, \vv,\Lambda)=Z^{-n/2}(\ell)
\ell^{d-l-n(d-2)/2} \Gamma^{(l,n)}(p, \vv(\ell),\Lambda)
\eqnn $$
with the usual definitions
$$\ell{\d \vv^2\over \d \ell}=\beta(\vv(\ell))\,,\quad \ell{\d \ln Z
\over \d \ell}=\eta_\psi(\vv(\ell))\, .$$
We are interested in the neighbourhood of the fixed point $\vv^2=\infty$.
Then, the RG function $\eta(\vv)$ approaches the exponent
$\eta$. The flow equation
for the coupling constant becomes
$$\ell{\d \vv^2\over \d \ell}=\rho \vv^2 , \ \Rightarrow\ \vv^2(\ell)\sim
\ell^{\rho}. $$
We again note that a correlation function with $l$ $\sigma$ fields becomes
proportional to $\vv^l$. Therefore,
$$\Gamma^{(l,n)}(\ell p, \vv,\Lambda)\propto
\ell^{d-(1-\rho/2)l-n(d-2+\eta_\psi)/2} .
\eqnn  $$
We conclude
$$ d_\sigma=\ud(d-2+\eta_\sigma)=1-\ud\rho \ \Leftrightarrow\
\eta_\sigma=4-d-\rho \,.\eqnn   $$
%
\medskip
{\it RG functions at order $1/N$} \rGNGracey.
A new generic integral is useful here:
$${1\over(2\pi)^d}\int{\d^d q (\sla{p}+\sla{q})\over (p+q)^{2\mu} q^{2\nu}}=
\sla{p}
p^{d-2\mu-2\nu}{\Gamma(\mu+\nu-d/2)\Gamma(d/2-\mu+1)\Gamma(d/2-\nu)
\over (4\pi)^{d/2}\Gamma(\mu)\Gamma(\nu)\Gamma(d-\mu-\nu+1)}
\, . \eqnn $$
We first calculate the
$1/N$ contribution to the fermion two-point function at the critical point
(from a diagram similar to diagram \figbubiii)
$$\Gamma^{(2)}_{\bar\psi\psi}(p)=i\sla{p}+{2i \vv^2\over
b(d)D(\vv)(2\pi)^d}
\int^\Lambda{\d^d q (\sla{p}+\sla{q})\over q^{d-2}(p+q)^2}.$$
We need the coefficient of $\sla{p}\ln\Lambda/p$.
Since we work only at one-loop order, we again replace the $\sigma$
propagator $1/q^{d-2}$ by $1/q^{2\nu}$  and send the cut-off to
infinity. The residue of the pole at $2\nu=d-2$ gives the coefficient
of the term $\sla{p} \ln\Lambda$ and the finite part the $\sla{p}\ln p$
contribution. We find
$$\Gamma^{(2)}_{\bar\psi\psi}(p)=i\sla{p}+{2i \vv^2\over
b(d)D(\vv)}N_d \left(d-2\over d\right)
\sla{p}\ln(\Lambda/p)\,, \eqnn $$
where $N_d$ is the loop factor \etadpolexp{b}.
Expressing that the $\left<\bar \psi \psi\right>$ function satisfies
RG equations, we obtain  immediately the RG function
$$\eta_\psi(\vv)={\vv^2\over D(\vv)}{(d-2) \over d}X_1\,, \eqnn $$
where $X_1$ is given by equation \eXone.
We then calculate the function $\left<\sigma\bar\psi\psi\right>$ at
order $1/N$:
$$\Gamma^{(3)}_{\sigma\bar\psi\psi}(p)=\vv+A_1 D^{-1}(\vv)\vv^3 \ln\Lambda
 \,,$$
where $A_1$ corresponds to the diagram of figure \figNtriangl:
$$A_1 =-{2\over b(d) } N_d =-X_1 \,.$$
The diagram
of figure \figNtrianglii\ vanishes because the $\sigma$ three-point function vanishes for symmetry reasons. \par
The $\beta$-function follows:
$$\beta_{\vv^2}(\vv) ={4(d-1) \vv^4 \over d}X_1 D^{-1}(\vv)
 \eqnn $$
and thus
$$\rho={8(d-1)N_d\over d b(d) N}={4(d-1)\over d N}X_1\,.$$
The exponents $\eta_{\psi}$ and $\eta_\sigma$ at order $1/N$  and, thus,
the corresponding dimensions $d_\psi, d_\sigma$ of the fields, follow:
$$\eqalignno{\eta_{\psi}&= {(d-2)\over d}{X_1\over N}= {(d-2)^2 \over
d}{\Gamma(d-1)\over \Gamma^3(d/2)\Gamma(2-d/2)N}.&\eqnd\eetapsii
\cr 2d_\psi&=d-1-{2(d-2)\over d}{X_1\over N}\,. &\eqnn \cr}$$ For
$d=4-\varepsilon$, we find $\eta_{\psi}\sim \varepsilon/N$, result
consistent with \expeps\ for  $N$ large. Whereas for
$d=2+\varepsilon$, one finds $\eta_{\psi}\sim \varepsilon^2/2N$,
consistent with \expetaii. The dimension of the field $\sigma$ is
$$d_\sigma=\ud(d-2+\eta_\sigma)=1-{2(d-1)\over d N}X_1 +O(1/N{}^2)
.\eqnn $$
A similar evaluation of the $\left<\sigma^2\sigma\sigma\right>$ function
allows to determine the exponent $\nu$ to order $1/N$:
$${1\over\nu}=d-2-{2(d-1)(d-2)\over dN}X_1\,. \eqnn $$
Actually all exponents are known to order $1/N^2$, except $\eta_\psi$ which is
known to order $1/N^3$.
%
%\subsection Massless electrodynamics with $U(N)$ symmetry
\smallskip

We discuss now shortly  two other models with chiral fermions, in
which large $N$ techniques can be applied, massless QED and the
$U(N)$ massless Thirring model.
%{\it Massless QED.}

\subsection Massless electrodynamics with $U(\tilde N)\times U(\tilde N)$ symmetry.

We first consider $\tilde N$ charged massless fermion fields
$\psib,\bar \psib$, interacting through an abelian gauge field
$A_{\mu}$ (massless QED with $\tilde N$ flavours)  \rSchwinger:
$$ {\cal S} (\bar \psib ,\psib,A _{\mu} ) = \int \d^{d}x
\left[{\textstyle{ 1 \over 4e^2}}  F ^{2} _{\mu \nu}(x) - \bar
\psib (x) \cdot\left(\sla{\partial} + i\Abar \right) \psib(x)
\right] . \eqnd\eactQEDN $$ This model possesses, in addition to
the $U(1)$ gauge invariance, a chiral  $U(\tilde N)\times U(\tilde
N)$ symmetry since the fermions are massless. Again, an
interesting question is whether the model exhibits in some
dimensions $2\le d\le 4$ a spontaneous breaking of chiral
symmetry. As before we use here $N={\tilde N} \tr{\bf 1}$, where
$N$ is the total number of fermion  components in $\tilde N$
differently flavoured Dirac fields. \sslbl\ssNQED
%
\smallskip
{\it Dimension $d=4-\varepsilon$.}
In terms of the  coupling constant standard in dimension 4,
$$\alpha\equiv {e^2\over 4\pi}\Lambda ^{-\varepsilon}\,, \eqnn $$
the RG $\beta$-function reads (taking  $\tr {\bf 1}=4$ in the space of
$\gamma$ matrices) \rQEDRG:
$$\eqalignno{\beta(\alpha)&=-\varepsilon \alpha +{2 {\tilde N}\over 3\pi}\alpha^2+
{{\tilde N}\over2\pi^2}\alpha^3-{{\tilde N}(22{\tilde
N}+9)\over144\pi^3}\alpha^4 \cr &-{1\over 64 \pi^4}{\tilde
N}\left[\frac{616}{243}{\tilde N}^2+\left(\frac{416}{9}\zeta(3)
-\frac{380}{27}\right){\tilde N}+23\right]\alpha^5+
O\left(\alpha^6\right). &\eqnd\ebetQED \cr} $$
The model is IR
free  in four dimensions. Therefore no phase  transition is
expected, at least for $e^2$ small enough. A hypothetical  phase
transition would rely on the existence on non-trivial fixed points
outside of the perturbative regime. \par In the perturbative
framework, the model provides an example of the famous triviality
property. For a generic effective coupling constant at cut-off
scale (i.e.~bare coupling), the effective coupling constant at
scale $\mu\ll\Lambda$ is given by
$$\alpha(\mu)\equiv {e^2(\mu)\over 4\pi}\sim {3\pi\over2{\tilde N}\ln(\Lambda/\mu)}.$$
This result can be used to bound the number of charged fields
(the number is not huge).\par
In $4-\varepsilon$ dimension
instead, one finds a non-trivial  IR fixed point corresponding to
a coupling constant
$$e^2_*=24 \pi^2 \varepsilon\Lambda^{\varepsilon}/N\,,$$
($N={\tilde N}\tr{\bf 1}$) and  correlation functions have a
scaling behaviour at large distance. As we have discussed in the
case of the $\phi^4$ field  theory, the effective coupling
constant at large distance becomes close to the IR fixed  point,
except when the  initial coupling constant is very small.
\par
The RG function associated  with the field renormalization is also
known up to order $\alpha^3$:
$$\eta_\psi= \xi {\alpha\over2\pi}-{4{\tilde N}+3\over 16\pi^2}\alpha^2+{40 {\tilde N}^2 +
54{\tilde N}+27 \over 576\pi^3}\alpha^3+O\left(\alpha^4\right),$$
where the gauge is specified by a term $(\partial_\mu
A_\mu)^2/2\xi$,  but it is a non-physical quantity because
it is gauge dependent. The simple dependence of $\eta _\psi$ in $\xi$ reflects the property that
if $\bar\psi(x)\psi(y)$ is not gauge invariant, $\bar\psi(x)\exp[i\oint_y^x A_\mu(s)\d s_\mu]\psi(y) $, instead is.

%
\medskip
{\it The large $N$ limit} \rNSchwinger.
To solve the model for $N\to \infty $, one first integrates over the
fermion fields and one obtains the large $N$ action
$${\cal S}_N  (A _{\mu} )= \int \d^{d}x
\left[{\textstyle{ 1 \over 4e^2}}  F^{2}_{\mu \nu}(x) -{\tilde
N}\tr\ln \left(\sla{\partial} + i\Abar \right)\right] .
\eqnd{\eactQEDN} $$ The  large  $N$ limit ($N={\tilde N}\tr{\bf
1}$) is taken with $e^2N$ fixed. Therefore, at leading order, only
${\cal S}_{N,{\rm F}} (A_\mu)$, the quadratic term in $A_{\mu}$ in the
expansion of the fermion determinant, contributes. A short
calculation yields
$$\left.\eqalign{{\cal S}_{N,{\rm F}} (A_\mu)&=-N \int \d^d k\, A_{\mu}(k)
\left[k^2\delta_{\mu\nu}-k_{\mu}k_{\nu}\right]K(k) A_{\nu}(-k),
\cr {\rm with}\quad K(k)&={d-2\over4 (d-1)}\left[b(d)k^{d-4}
-a(d)\Lambda^{d-4}\right]+O\left(\Lambda^{-2}\right),\cr}\right.
\eqnd\eAAlN$$ where $b(d)$ is the universal constant \econstb, and
$a(d)$ is the constant \eadef\ that depends on the regularization.
\par For $d<4$, the leading term in the IR region comes from the
integral. The behaviour at small momentum of the vector field is
modified, which confirms the existence of a non-trivial IR fixed
point. The fixed point is found by demanding cancellation of the
leading corrections to scaling coming from $F_{\mu\nu}^2$ and the
divergent part of the loop integral,
$$e_*^2={2(d-1)\over (d-2)a(d)}{\Lambda^{4-d}\over N}.$$
However, there is again no indication of chiral symmetry breaking.
Power counting within the $1/N$ expansion confirms that the IR
singularities have been eliminated, because the large $N$ vector
propagator is less singular than in perturbation theory. Of course,
this result is valid only for $N$ large. Since the long range
forces generated by the gauge coupling have not been totally
eliminated, the problem remains open for $d$ not close to four, or
for $e^2$ not very small and $N$ finite. Some numerical
simulations indeed suggest a  chiral phase transition for $d=4$
and $d=3$, ${\tilde N}\le N_c \sim 3$. \par
The exponents
corresponding to the IR fixed point have been calculated up to
order $1/N^2$. At order $1/N$ ($X_1$ is defined  by Eq.~\eXone)
$$\eqalign{\eta_\psi(\xi=0)&=-{(d-1)^2(4-d)\over
d(d-2)} {X_1\over N}+O\left(1/  N ^2\right), \cr \eta_m&
=-4{(d-1)^2\over d(d-2)} {X_1\over N }+O\left(1/   N ^2\right)
,\cr \beta'(\alpha^*)&= 4-d -{(d-3)(d-6)(d-1)^2(4-d)\over d(d-2)}
{X_1\over N }+O\left(1/  N ^2\right)  . \cr}$$  Finally, note that
in the  $d=2$ limit, the integral generates a contribution
$Ne^2/\pi k^2$ times the propagator of the free gauge field
$$K(k)\mathop{\sim}_{d\to 2}{1\over 4 \pi k^2}. $$
As a direct  analysis of the $d=2$  case confirms, this corresponds
to a massive bound state, of mass squared $Ne^2/\pi$. However, for
generic values of the coupling constant, the mass is of the order of
the cut-off $\Lambda$. Only when $e$ is ``unnaturally" small with respect to the
microscopic scale, as one assumes in conventional renormalized
perturbation theory, does this mass correspond in the continuum limit
to a propagating particle.
%
\smallskip
{\it Two dimensions.}
We now assume that the dimensional quantity $e^2$ is small
in the microscopic scale. The model then  is a simple
 extension  of the Schwinger model and can be exactly solved by the same
method. For ${\tilde N}=1$, the model exhibits the simplest example
of a chiral anomaly, illustrates  the property of confinement and
spontaneous chiral symmetry breaking. For ${\tilde N}>1$, the
situation is more subtle. The neutral $\bar\psi\psi$ two-point
function decays algebraically,
$$\left< \bar{\psib}(x)\cdot\psib(x)\bar{\psib}(0)\cdot\psib(0) \right>
\propto x^{2/{\tilde N}-2},$$ indicating the presence of a
massless mode and $\left<\bar\psi\psi\right>=0$. Instead, if we
calculate the two-point function of the composite operator
$${\cal O}_{\tilde N}(x)=\prod_{i=1}^{\tilde N} \bar{\psi}_i(x)\psi_i(x),$$
we find
$$\left<{\cal O}_{\tilde N}(x){\cal O}_{\tilde N}(0)\right>\propto \ {\rm const.}\ .$$
We have thus identified an operator which has a non-zero
expectation value. As a consequence of the fermion antisymmetry,
if we perform a transformation under the group $U({\tilde
N})\times U({\tilde N})$ corresponding to matrices $U_+,U_-$, the
operator is multiplied  by $\det U_+/\det U_-$. The operator thus
is invariant under the group $SU({\tilde N})\times SU({\tilde
N})\times U(1)$. Its non-vanishing expectation value is the sign
of the spontaneous breaking of the remaining $U(1)$ chiral group.

\subsection The $U({\tilde N})$ Thirring model

We now consider the model \refs{\rThirringeps{--}\rDuHaMe}
$${\cal S}(\bar\psib,\psib)= - \int \d^d
x\left[\bar\psi\left(\sla{\partial}+m_0
\right) \psi - \ud g J_{\mu}J_{\mu}\right] ,\eqnd{\eactThir} $$
where
$$J_{\mu}=\bar\psib\gamma_{\mu}\cdot\psib\, . \eqnn $$
The special case ${\tilde N}=1$ corresponds to the simple Thirring
model. In two dimensions, it is then equivalent to a free massless
boson field theory (with mass term for fermions, one obtains the
sine--Gordon model). In order to bosonize the model in $d=2$ and
to study that large $N$ properties, one introduces an abelian
gauge field $A_\mu$ coupled to the current $J_\mu$:
$$\ud g  J_{\mu}J_{\mu} \longmapsto A_{\mu}^2/2g +iA_{\mu}J_{\mu}\, .
\eqnn $$
One then finds massive QED without a $F^2_{\mu\nu}$ gauge kinetic term:
$${\cal S}(A_{\mu},\bar\psi,\psi)= - \int \d^2
x\left[\bar\psi\left(\sla{\partial}+ i\Abar+ m_0 \right)\psi
-A_{\mu}^2 /2g\right] . \eqnd{\eactAmu} $$ If one integrates over
the fermions, the fermion determinant generates a  kinetic term
for the gauge field. For $m_0=0$, the situation is thus similar to
massless QED, except that the gauge field is now massive.
