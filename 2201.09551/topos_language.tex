\documentclass{tac}

 % Prepared using tac.cls and diagxy (if you do not have diagxy, compiling this will require commenting lines 165-168)
% PLEASE READ comments in the text below
% PLEASE NOTE: source files for submission to TAC require a comment like the following
% giving style, packages used, TeX implementation
% TAC style, 2 pp, Xy-pic ver 3.7, MikTeX version 3.1


% NOTE: packages used should be the first part of the preamble

 \usepackage{xy}
% diagxy loads xy, so the line preceding is redundant
\usepackage{newlfont}
\usepackage{amsfonts}
\usepackage{amssymb}
%\usepackage{amsthm}
\usepackage[centertags]{amsmath}
\usepackage{plain}
\xyoption{all}

 \def\hs#1{\hskip#1mm}
\def\vs#1{\vskip#1mm}
\def\a#1#2#3#4#5#6{\ar@<#1ex>@/^#2pc/@{#3}[#4]#5{#6}}%
%#1 moving arrow up or down (right arrow +num or -num; left arrow -num or +num),
%#2 curvature of arrow up or down (+num or -num),
%#3 type of arrow (->, .>, - , etc),
%#4 direction of arrow, right, left, up, or down (r, l, u, d),
%#5 putting arrow name up, middle, or down (right arrow ^, |, _; left arrow _, |, ^)
%#6 arrow name
\def\rar#1#2#3{$\xymatrix{#1:#2\ar[r]&#3}$}
\def\goesto{$\xymatrix{A\ar@{|->}[rr] && B}$}
\def\ph{\phantom{(}}
\newdir{ >}{*{}!/-.2cm/\dir{>}}
\newdir{< }{*{}!/.2cm/\dir{<}}
\newtheorem{thm}{Theorem}[section]
\newtheorem{cor}[thm]{Corollary}
\newtheorem{lem}[thm]{Lemma}
\newtheorem{prop}[thm]{Proposition}
\theoremstyle{definition}
\newtheorem{defn}[thm]{Definition}
\theoremstyle{remark}
\newtheorem{ques}{Question}
\newtheorem{prob}{Problem}
\newtheorem{rem}{Remark}
\newtheorem{alg}{Algorithm}
\newtheorem{con}{Conclusion}
\newtheorem{exm}[thm]{Example}
\newtheorem{pf}{proof}
\newtheorem{nt}{Notice}
\input diagxy

 \def\xypic{\hbox{\rm\Xy-pic}}


 \usepackage[colorlinks=true]{hyperref}
%\hypersetup{allcolors=[rgb]{0.1,0.1,0.4}}

%%%%%%%%%%
\def\mc#1{\mathcal {#1}}
\def\C{\mc C}
\def\A{\mc A}
\def\B{\mc B}
\def\D{\mc D}
\def\F{\mc F}
\def\I{\mc I}
\def\S{\mc S}
\def\X{\mc X}
\def\Y{\mc Y}
\def\T{\mc T}
\def\Z{\mc Z}
\def\E{\mc E}
\def\M{\mc M}


\def\P{\mc P}

\def\W{\mc W}
%%%%%%%%%%%%%

\title{Representing the language of a Topos as Quotient of the Category of Spans}


\author{M. Golshani and A.R Shir Ali Nasab}
 \keywords{Language of topos, Span category, Allegory, Topos, Boolean Topos}
\amsclass{18A32, 18B99, 18C10, 03G30}

\thanks{%
 The first author's research has been supported by a grant from
  IPM (No. 1400030417). The
	second author's research  is partially supported by IPM.}




 \def\xypic{\hbox{\rm\Xy-pic}}



 \newtheorem{thm}{Theorem}

\newtheorem{theorem}{Theorem}
\newtheorem{lem}{Lemma}


\newtheoremrm{rem}{Remark}



 \let\pf\proof
\let\epf\endproof


 \begin{document}

 \maketitle
\begin{abstract}
We use quotients of span categories to introduce the language of a topos. We also study the  logical relations and the quotients of span categories derived from them. As an application we show that the category of Boolean toposes is a reflective subcategory of the category of toposes,
when  the morphisms are logical functors.
 \end{abstract}



\section{Introduction}

The Mitchell-Benabou language \cite{macmor} is a form of the internal language of an elementary topos. In this form, types are the objects of topos and variables are interpreted as identity morphisms $1:A\rightarrow A$ and terms of type $A$ in variables $x_i$ of type $X_i$ are interpreted as morphisms from the product of the $X_i$'s to $A$. Formulas of the language are thus terms of type $\Omega.$

	In \cite{lambscot}, a different approach is presented in which the variables are interpreted as indeterminate morphisms. For every object $A$ in topos $\mathcal T$, a category, $\mathcal T[x]$ is formed by adding an indeterminate morphism $1\rightarrow A$ in a universal manner and $\mathcal T[x]$'s are used to interpret variables and terms of language. In this form, we have to travel between $\mathcal T[x]$'s and there is no integrated setting to have all variables and terms together. In \cite{lambscot} two approaches are introduced to add an indeterminate morphism, namely using:
 \begin{itemize}
 \item Free category generated by a graph, and
	
\item Kleisli category.

\end{itemize}
	In this paper, we follow the method of \cite{lambscot} and add indeterminate morphisms to $\T$ in a universal manner to provide variables and terms. To do this we use the category of spans that has objects as objects of $\mathcal T$ and morphisms as follows:
	$$\xymatrix{&& D \ar[lld]_{s} \ar[rrd]^{f}&&\\
	A&&&&B}$$
	The category of spans and its quotients are used widely in categorical structures, see for example \cite{hoshitho} and \cite{hsty}.
	It has more morphisms than the base category and it makes us able to construct new categories with various features by using its quotients.
	Using spans to add indeterminate morphisms to a category gives us more flexibility and we can add all of the variables in one stage and obtain a category that is cartesian closed and contains all variables and terms. Also all of logical connectives are defined as morphisms of this category. So all of the language is already in the category.
	
	We then introduce some kind of relations generated by endospans which are spans with the same domain and codomain. We also study when a quotient of span category has the same logical property as the base category and in this way, we introduce a class of relations called logical relations. As an application of this kind of relation, we obtain a boolean topos from an elementary topos and using it we show that the category of Boolean toposes is a reflective subcategory of the category of toposes,
when  the morphisms are logical functors.


\section{Preliminaries}
We recall some definitions and preliminaries about spans categories.
We consider {\em categories equipped with a stable system of morphisms}; these are pairs $(\C,\S)$ with a category $\C$ and a class $\mathcal{S}$ of morphisms in $\C$ such that:
\begin{itemize}
	\item
	$\S$ contains all isomorphisms and is closed under composition, and
	\item
	pullbacks of $\S$-morphisms along arbitrary morphisms exist in $\C$ and belong to $\S$.
\end{itemize}
$\S$ is called an stable class.
For objects $A,B$ in $\C$, an span $(s,f)$ with {\em domain} $A$ and {\em codomain} $B$ is given by a pair of morphisms
\begin{center}
	$\xymatrix{A & D\ar[l]_{s}\ar[r]^{f} & B}$
\end{center}
with $s$ in $\S$ and $f$ in $\C$.

For a stable class $\F$, we define a morphism $x:(s,f)\rightarrow(s',f')$ with $x\in\F$, if the following diagram commutes.
\begin{center}
	$\xymatrix{& & D\ar[lld]_{s}\ar[dd]_{x}\ar[rrd]^{f} & & \\
		A & & & & B\\
		& & \tilde{D}\ar[llu]^{s'}\ar[rru]_{f'} & &}$
\end{center}
If there is a morphism $x:(s,f)\rightarrow(s',f')$, write $(s,f)\leq_\F(s',f')$. The equivalence relation generated by $\leq_\F$ is denoted by $\sim_\F$. We form the quotient category of spans $\mathsf{Span}_\F(\C,\S)$ that has objects of $\C$ as objects and equivalence classes $[s,f]_{\sim_\F}$ as morphisms. Composition of its morphisms $[s,f]_{\sim_\F}:A\rightarrow B$ and $[t,g]_{\sim_\F}:B\rightarrow C$ is defined to be $[st',gf']$ as depicted in the following diagram.
\begin{center}
	$\xymatrix{&& P\ar[ld]_{t'}\ar[rd]^{f'} && \\
		& D\ar[ld]_{s}\ar[rd]^{f}\ar@{}[rr]|{pb} & & E\ar[ld]_{t}\ar[rd]^{g} & \\
		A && B && C}$
\end{center}
The composition is well-defined. For simplicity we use the notation $[s,f]_\F$ instead of $[s,f]_{\sim_\F}$.
If $\I$ is the class of isomorphism, $\mathsf{Span}_\I(\C,\S)$ is the ordinary category of spans and we use $[s,f]$ instead of $[s,f]_\I$ for morphisms.
We also mention a useful lemma about $\sim_\F$:
\begin{lemma}\label{describe for stab class rel}
	For an stable class $\F$, $(s,f)\sim_\F(s',f')$ $\iff$ there exist $p,q\in\F$ such that the following diagram commutes.
	$$\xymatrix{&& D\ar[lld]_{s}\ar[rrd]^{f}&&\\
	A&& P\ar[d]_{q}\ar[u]^{p}&&B\\
&&D'\ar[ull]^{s'}\ar[urr]_{f'}&&}$$
	
\end{lemma}

Due to generality of $\sim_\F$ we define a {\em compatible} relation on $\mathsf{Span}(\C,\S)$, that is a relation for $\S$-spans such that
\begin{itemize}
	\item only $\S$-spans with the same domain and codomain may be related;
	\item vertically isomorphic $\S$-spans are related;
	\item horizontal composition from either side preserves the relation.
\end{itemize}
For a compatible equivalence relation $\sim$ we denote the $\sim$-equivalence class of $(s,f)$ by $[s,f]_{\sim}$, or simply by $[s,f]$ when the context makes it clear which relation $\sim$ we are referring to, and we write
$$\mathsf{Span}_{\sim}(\C,\S)$$
for the resulting category.


\section{Adding indeterminate arrows}
Through out this section we let the category $\C$ be a cartesian category.
As in \cite{lambscot}, we will try to add an indeterminate morphism $x:1\rightarrow A$ to the category $\C$ in a universal manner.

For an object $A$ in category $\C$, let $\A$ be the following class:
\begin{center}
	$\A=\lbrace \pi:A^n\times B\rightarrow B: \pi$ is a projection$\rbrace$.
\end{center}
\begin{lemma}
	For every object $A\in\C$, $\A=\lbrace \pi:A^n\times B\rightarrow B: \pi$ is a projection$\rbrace$ is an stable class.
\end{lemma}
\begin{proof}
	Letting $n=0$. So $A^n=1$. Then $\A$ contains isomorphisms.
	Stability under pullback and composition is obvious.
\end{proof}
Let us construct the quotient category of spans
$$\mathsf{Span}_\A(\C,\A).$$
\begin{proposition}
	The map $\mathbf Q$, sending $f$ to $[1,f]_\A$ from $\C$ to $\mathsf{Span}_\A(\C,\A)$ is a functor; furthermore if there is an arrow from terminal object, $1$, to $\A$, then this functor is faithful.
\end{proposition}
\begin{proof}
	Obviously this map is a functor. For morphisms $\xymatrix{B\ar@<1ex>[rr]^{f}\ar@<-1ex>[rr]_{g}&& C}$, let $[1,f]_\A=[1,g]_\A$. By Lemma \ref{describe for stab class rel}, there exist $p,q\in\A$ such that the following diagram commutes.
		$$\xymatrix{&& B\ar[lld]_{1}\ar[rrd]^{f}&&\\
		B&& A^n\times B\ar[d]_{q}\ar[u]^{p}&&C\\
		&&B\ar[ull]^{1}\ar[urr]_{g}&&}$$
	This implies $p=q$. Since there is a morphism $1\rightarrow A$, $p$ is an epimorphism. Then $f=g$, hence functor is faithful.
\end{proof}
\begin{theorem}
	The functor $\mathbf Q:\C\rightarrow \mathsf{Span}_\A(\C,\A)$ preserves finite products.
\end{theorem}
\begin{proof}
	We have to show
	$$\xymatrix{B&&B\times C\ar[ll]_{[1,\pi_B]_\A}\ar[rr]^{[1,\pi_C]_\A}&&C}$$
	is product in $\mathsf{Span}_\A(\C,\A)$, where $\pi_B$ and $\pi_C$ are projections in $\C$.
	
	Let $\xymatrix{B&&D\ar[ll]_{[d_1,f]_\A}\ar[rr]^{[d_2,g]_\A}&&C}$ be given where $[d_1,f]_\A$ and $[d_2,g]_\A$ are depicted below.
	\begin{center}
		$\xymatrix{&& A^n\times D\ar[lld]_{d_1}\ar[rrd]^{f}&&\\
		D&&&&B}$
		\hfil
		$\xymatrix{&& A^m\times D\ar[lld]_{d_2}\ar[rrd]^{g}&&\\
		D&&&&C}$
	\end{center}
	Let $m\leq n$. We have the projection $\pi:A^n\times D\rightarrow A^m\times D$. So $[d_2,g]_\A=[d_2\pi,g\pi]_\A$. Since $d_1$ and $d_2\pi$ are projections from $A^n\times D$ to $D$, we can let $n=m$ and $d_1=d_2$. Set $d=d_1$ and $h=\langle f,g\rangle$.
	clearly $[1,\pi_B]_\A[d,h]_\A=[d,f]_\A$ and $[1,\pi_C]_\A[d,h]_\A=[d,g]_\A$.
	
	To show the uniqueness of $[d,h]_\A$, let $[e,k]_\A$ be a morphism in which $[1,\pi_C][e,k]=[d,g]$ and $[1,\pi_B]_\A[e,k]_\A=[d,f]_\A$.
	
	By Lemma \ref{describe for stab class rel}, morphisms $a,b,a',b'\in\A$ exist such that the following diagrams commutes.
	\begin{center}
		$\xymatrix{&& A^n\times D\ar[lld]_{d}\ar[rrd]^{f}&&\\
			D&& A^{n+r+s}\times D\ar[d]_{b}\ar[u]^{a}&&B\\
			&&A^r\times D\ar[ull]^{e}\ar[urr]_{\pi_Bk}&&}$
		\hfil
		$\xymatrix{&& A^n\times D\ar[lld]_{d}\ar[rrd]^{g}&&\\
			D&& A^{n+r+s'}\times D\ar[d]_{b'}\ar[u]^{a'}&&C\\
			&&A^r\times D\ar[ull]^{e}\ar[urr]_{\pi_Ck}&&}$
	\end{center}
	As before we can let $s=s'$, $a=a'$ and $b=b'$. So the following diagram is commutative.
	$$\xymatrix{&& A^n\times D\ar[lld]_{d}\ar[rrd]^{\langle f,g\rangle}&&\\
		D&& A^{n+r+s}\times D\ar[d]_{b}\ar[u]^{a}&&B\times C\\
		&&A^r\times D\ar[ull]^{e}\ar[urr]_{k}&&}$$
	Hence $[e,k]_\A=[d,h]_\A$.
\end{proof}
Our aim of this section is to add an indeterminate arrow $1\rightarrow A$ to the category $\C$. We construct the category $\mathsf{Span}_\A(\C,\A)$ as a quotient of spans. The morphism $[!_A,1_A]_\A:1\rightarrow A$ is the morphism that we want. We denote $[!_A,1_A]_\A:1\rightarrow A$ by $x$ and the category $\mathsf{Span}_\A(\C,\A)$ by $\C[x]$.
\begin{proposition}
	\begin{itemize}
		\item[(a)]
		$x^n=[!_{A^n},1_{A^n}]$.
		\item[(b)]
		$x^n\times 1_B=[\pi,1_{A^n\times B}]$ where $\pi:A^n\times B\rightarrow B$ is the projection.
	\end{itemize}
\end{proposition}
\begin{proof}
	\begin{itemize}
		\item[(a)]
		For $n=2$, uniqueness of $X^2$ in the following commutative diagram implies $x^2=[!_{A^2},1_{A^n}]_\A$.
		$$\xymatrix{1\ar[d]_{x} && 1\ar[ll]\ar[rr] \ar[d]|{x^2=[!_{A^2},1_{A^n}]_\A} && 1\ar[d]^{x}\\
		A&&A^2\ar[rr]\ar[ll]&&A}$$
		By induction on $n$ we get $x^n=[!_{A^n},1_{A^n}]$.
		\item[(b)]
		Uniqueness of $x^n\times 1_B$ in the following diagram implies $x^n\times 1_B=[\pi,1_{A^n\times B}]$.
		$$\xymatrix{1\ar[dd]_{x^n} && B\ar[ll]\ar[rr] \ar[dd]|{x^n\times 1_B=[\pi,1_{A^n\times B}]_\A} && B\ar[dd]^{1}\\
			&&&&\\
			A^n && A^n\times B\ar[rr]\ar[ll]&&B}$$
	\end{itemize}
\end{proof}
\begin{proposition}
	Suppose the functor $\mathbf F:\C\longrightarrow\C'$ is a finite product preserving functor and $a:1\rightarrow \mathbf F(A)$ is a morphism in $\C'$. There is a unique functor $\mathbf F':\C'\longrightarrow\mathsf{Span}_\A(\C,\A)$ such that $\mathbf F'(x)=a$ and the following triangle commutes.
	$$\xymatrix{\C\ar[rr]\ar[d]_{\mathbf F}&& \mathsf{Span}_\A(\C,\A)\ar[lld]^{\mathbf F'}\\
	\C'&&}$$
\end{proposition}
\begin{proof}
	For a morphism $[p,f]$, depicted as follows:
	$$\xymatrix{B&&A^n\times B\ar[ll]_{p}\ar[rr]^{f}&& C}$$
	we define
$$\mathbf F'[p,f]=\mathbf Ff\mathbf F'(x^n\times 1_B)=\mathbf Ff (a^n\times 1_{\mathbf F(B)}).$$
	To show $\mathbf F'$ is well defined, suppose the following diagram is given:
	$$\xymatrix{&& A^n\times B\ar[lld]_{p}\ar[rrd]^{f}\ar[dd]_{\pi}&&\\
		B&& &&C\\
		&&A^m\times B\ar[ull]^{p'}\ar[urr]_{f'}&&}$$
	We have
$$\mathbf F'[p,f]=\mathbf F(f)(a^n\times 1_{\mathbf F(B)})=\mathbf F(f')\mathbf F(\pi)(a^n\times 1_{\mathbf FB})=\mathbf Ff'(a^m\times 1_{FB})=\mathbf F'[p',f'].$$
	By definition of $\mathbf F'$, we get the commutativity of the triangles and also the uniqueness.
\end{proof}
In the following, we will show the construction of indeterminate morphism is hereditary. For an object $B$ in $\C$, we know $B$ is an object in $\mathsf{Span}_\A(\C,\A)=\C[x]$. We form the following class:
\begin{center}
	$\B=\lbrace [1,\pi]_\A:B^n\times C\rightarrow C: [1,\pi]_\A$ is a projection in $\mathsf{Span}_\A(\C,\A)$$\rbrace$.
\end{center}
We also define the following class:
\begin{center}
	$\A\circ\B=\lbrace A^n\times B^m\times C\rightarrow C: \pi $ is a projection in $\C\rbrace$.
\end{center}
\begin{theorem}
	$\mathsf{Span}_\B(\mathsf{Span}(\C,\A),\B)$ is isomorphic to $\mathsf{Span}_{\A\circ\B}(\C,\A\circ\B)$.
\end{theorem}
\begin{proof}
	We define the map:
	$$[[1,pr]_\A,[p,f]_\A] \longmapsto [pr.p,f]_{\A\circ\B}.$$
	Suppose we have the following commutative diagram.
	$$\xymatrix{&& B^n\times C\ar[lld]_{[1,pr]_\A}\ar[rrd]^{[p,f]_\A}\ar[dd]_{[1,\pi]_\A}&&\\
		C&& &&D\\
		&&B^r\times C\ar[ull]^{[1,pr']_\A}\ar[urr]_{[q,g]_\A}&&}$$
	So we have
\[\begin{array}{ll}
[pr'.q,g]_{\A\circ\B} &=[q,g]_{\A\circ\B}[pr',1]_{\A\circ\B}\\
&=[q,g]_{\A\circ\B}[1,\pi]_{\A\circ\B}[\pi,1]_{\A\circ\B}[pr',1]_{\A\circ\B}\\
&=[p,f]_{\A\circ\B}[pr,1]_{\A\circ\B}\\
&=[pr.p,f]_{\A\circ\B}.
 \end{array}\]
%$[pr'.q,g]_{\A\circ\B}=[q,g]_{\A\circ\B}[pr',1]_{\A\circ\B}=
%[q,g]_{\A\circ\B}[1,\pi]_{\A\circ\B}[\pi,1]_{\A\circ\B}[pr',1]_{\A\circ\B}=[p,f]_{\A\circ\B}[pr,1]_{\A\circ\B}=[pr.p,f]_{\A\circ\B}$.
	Thus the map is well defined. Obviously this map is an isomorphic functor.	
\end{proof}
\begin{corollary}
	The functor $\C\rightarrow \mathsf{Span}_{\A\circ\B}(\C,\A\circ\B)$ which is defined as
	$$f\longmapsto [1,f]_{\A\circ\B}$$
	preserves finite product.
\end{corollary}
\begin{proof}
	This functor in the composition of the following finite product preserving functors:
	$$\C\rightarrow \mathsf{Span}_\A(\C,\A)\rightarrow \mathsf{Span}_\B(\mathsf{Span}_\A(\C,\A),\B)\cong\mathsf{Span}_{\A\circ\B}(\C,\A\circ\B),$$
and hence it preserves finite products.
\end{proof}
Let $\Pi$ be the class of all projections. Obviously $\Pi$ is an stable class. So we can form the following quotient category of spans:
$$\mathsf{Span}_\Pi(\C,\Pi)$$
\begin{theorem}
	The functor $\mathbf Q:\C\longrightarrow \mathsf{Span}_\Pi(\C,\Pi)$,sending $f$ to $[1,f]_\Pi$ preserves products.
\end{theorem}
\begin{proof}
	Let $\xymatrix{C&& C\times D\ar[ll]_{\pi_1}\ar[rr]^{\pi_2}&& D}$ be a product diagram in $\C$. We will show $\xymatrix{C&& C\times D\ar[ll]_{[1,\pi_1]_\Pi}\ar[rr]^{[1,\pi_2]_\Pi}&& D}$ is a product diagram in $\mathsf{Span}_\Pi(\C,\Pi)$.
	Let the diagram $\xymatrix{C&& E\ar[ll]_{[p,f]_\Pi}\ar[rr]^{[q,g]_\Pi}&& D}$ be given, where $[p,f]$ and $[q,g]$ are depicted as follows:
	\begin{center}
		$\xymatrix{&&A\times E\ar[lld]_{p}\ar[rrd]^{f}&&\\
		E&&&&C}$
	\hfil
	$\xymatrix{&&B\times E\ar[lld]_{q}\ar[rrd]^{g}&&\\
		E&&&&D}$
	\end{center}
	By replacing $\Pi$ by $\A\circ\B$, there is a unique morphism $[r,h]:E\rightarrow C\times D$ such that the triangles in the following left diagram are commutative. So the triangles in the following right diagram are commutative as well.
	\begin{center}
		$\xymatrix{C&&C\times D\ar[ll]_{[1,\pi_1]_{\A\circ\B}}\ar[rr]^{[1,\pi_2]_{\A\circ\B}}&& D\\
			&& E\ar[llu]^{[p,f]_{\A\circ\B}}\ar[rru]_{[q,g]_{\A\circ\B}}\ar[u]_{[r,h]_{\A\circ\B}}&&}$
		\hfil
		$\xymatrix{C&&C\times D\ar[ll]_{[1,\pi_1]_{\Pi}}\ar[rr]^{[1,\pi_2]_{\Pi}}&& D\\
			&& E\ar[llu]^{[p,f]_{\Pi}}\ar[rru]_{[q,g]_{\Pi}}\ar[u]_{[r,h]_{\Pi}}&&}$
	\end{center}
	To show the uniqueness of $[r,h]_{\Pi}$, let $[s,k]_\Pi$ be a morphism such that $[1,\pi_1]_\Pi[s,k]_\Pi=[p,f]_\Pi$ and $[1,\pi_2]_\Pi[s,k]_\Pi=[q,g]_\Pi$.
	
	By Lemma \ref{describe for stab class rel}, there exist some projections such that the following diagrams commute.
	\begin{center}
			$\xymatrix{&& X\times E\ar[lld]_{s}\ar[rrd]^{\pi_1 k}&&\\
			E&& Y\times X\times A\times E\ar[d]\ar[u]&&C\\
			&&A\times E\ar[ull]^{p}\ar[urr]_{f}&&}\hfil	
		\xymatrix{&& X\times E\ar[lld]_{s}\ar[rrd]^{\pi_2 k}&&\\
			E&& Z\times X\times B\times E\ar[d]\ar[u]&&D\\
			&&B\times E\ar[ull]^{p}\ar[urr]_{f}&&}$
	\end{center}
	Set $\Pi'=\A\circ\B\circ\X\circ\Y\circ\Z$. In $\mathsf{Span}_{\Pi'}(\C,\Pi')$, we have $[s,k]_{\Pi'}=[r,h]_{\Pi'}$. So $[s,k]_{\Pi}=[r,h]_{\Pi}$. Hence $[r,h]_{\Pi}$ is unique.
\end{proof}
\begin{theorem}
	If $\C$ is a cartesian closed category, then so is $\mathsf{Span}_\Pi(\C,\Pi)$.
\end{theorem}
\begin{proof}
	We want to show evaluation map $ev:B^A\times A\longrightarrow B$ in $\C$ is also evaluation map in $\mathsf{Span}_\Pi(\C,\Pi)$.
	Suppose that the morphism $[p,f]_\Pi$ is given and assume $[p,f]_\Pi$ is depicted as follows:
	$$\xymatrix{&& D\times C\times A\ar[dll]_{p}\ar[drr]^{f}&&\\
		C\times A&&&&B}$$	
	There is the unique morphism $\tilde f$ in $\C$ such that the following diagram commutes.
	$$\xymatrix{B^A\times A\ar[rr]^{ev}&& B\\
	(D\times C)\times A\ar[u]^{\tilde{f} \times 1}\ar[urr]_{f}&&}$$
	We form the following diagram in $\C$ by using product diagrams.
	\begin{center}
		$\xymatrix{B^A & B^A\times A\ar[l]\ar[r] & A\\
		D\times C\ar[u]^{\tilde f} & D\times C\times A\ar[l]\ar[r]\ar[u]|{\tilde f\times 1} & A\ar[u]^{1}}$	
	\hfil
	$\xymatrix{C & C\times A\ar[l]\ar[r] & A\\
		D\times C\ar[u]^{\pi} & D\times C\times A\ar[l]\ar[r]\ar[u]|{\pi\times 1} & A\ar[u]^{1}}$
	\end{center}
	Note that in the above diagram,   the left squares are pullbacks. By using the above diagram, we get the following diagram in $\mathsf{Span}_\Pi(\C,\Pi)$.
	
	$$\xymatrix{B^A && B^A\times A\ar[ll]\ar[rr] && A\\
		C\ar[u]^{[\pi,\tilde f]_\Pi} && C\times A\ar[ll]\ar[rr]\ar[u]|{[p,\tilde f\times 1]_\Pi} && A\ar[u]^{1}}$$
	So $[p,\tilde f\times 1]_\Pi=[\pi,\tilde f]_\Pi\times 1$. Then we get $[1,ev]_\Pi([\pi,\tilde f]_\Pi\times 1)=[p,f]_\Pi$.
	We have to show $[\pi,\tilde f]_\Pi$ is the unique morphism by this property.
	Thus suppose that for another morphism $[\pi',f']_\Pi$, we have $[1,ev]_\Pi([\pi',f']\times 1)=[p,f]$.
	There are $r,s\in\Pi$ such that the following diagram commutes.
	$$\xymatrix{&&D\times C\times A\ar[lldd]_{\pi\times 1}\ar[rd]^{\tilde f\times 1}&&\\
	&&&B^A\times A\ar[rd]^{ev}&\\
	C\times A && L\times E\times D\times C\times A\ar[uu]^{r}\ar[dd]_{s}&& B\\
	&&&B^A\times A\ar[ur]_{ev}&\\
	&& E\times C\times A\ar[lluu]^{\pi'\times 1}\ar[ru]_{f'\times 1}&&}$$
	We can show $r,s$ as
	$$r=pr\times 1:(L\times E\times D\times C)\times A\longrightarrow D\times C\times A$$
	and
	$$s=pr'\times 1:(L\times E\times D\times C)\times A\longrightarrow E\times C\times A$$
	These imply $\tilde f\ pr=f'\ pr'$. So the following commutative diagram implies $[\pi,\tilde f]_\Pi=[\pi',f']_\Pi$ and the uniqueness of $[\pi,\tilde f]_\Pi$ is proved.
	$$\xymatrix{&& D\times C\ar[lld]_{\pi}\ar[rrd]^{\tilde f}&&\\
		C&& L\times E\times D\times C\ar[d]_{pr'}\ar[u]^{pr}&&B^A\\
		&&E\times C\ar[ull]^{\pi'}\ar[urr]_{f'}&&}$$
	Hence $\mathsf{Span}_\Pi(\C,\Pi)$ is cartesian closed.
\end{proof}
For objects $A_1, A_2,..., A_n\in\C$ we have the functor
$$\mathbf Q:\mathsf{Span}_{\A_1\circ\A_2\circ ...\circ\A_{n-1}}(\C,\A_1\circ\A_2\circ ...\circ\A_{n-1})\longrightarrow \mathsf{Span}_{\A_1\circ\A_2\circ ...\circ\A_n}(\C,\A_1\circ\A_2\circ ...\circ\A_{n})$$
So we can form a diagram in category $\mathsf{Cat}$.
\begin{theorem}
	$\mathsf{Span}_\Pi(\C,\Pi)$ is colimit for the above diagram.
\end{theorem}
\begin{proof}
	Obviously
	$$\xymatrix{\mathsf{Span}_{\A_1\circ\A_2\circ ...\circ\A_{n-1}}(\C,\A_1\circ\A_2\circ ...\circ\A_{n-1})\ar[rr]\ar[d] && \mathsf{Span}_\Pi(\C,\Pi)\\
	  \mathsf{Span}_{\A_1\circ\A_2\circ ...\circ\A_n}(\C,\A_1\circ\A_2\circ ...\circ\A_{n})\ar[rru] &&}$$
	is natural. Now suppose we have the natural sink
	$$\xymatrix{\mathsf{Span}_{\A_1\circ\A_2\circ ...\circ\A_{n-1}}(\C,\A_1\circ\A_2\circ ...\circ\A_{n-1})\ar[d]\ar[rrrrrr]^{\ \ \ \mathbf F_{\A_1\circ\A_2\circ ...\circ\A_{n-1}}} &&&&&& \D \\
		 \mathsf{Span}_{\A_1\circ\A_2\circ ...\circ\A_n}(\C,\A_1\circ\A_2\circ ...\circ\A_{n})\ar[rrrrrru]|{\mathbf F_{\A_1\circ\A_2\circ ...\circ\A_n}} &&&&&&}$$
	We define the functor $\mathbf U:\mathsf{Span}_\Pi(\C,\Pi)\longrightarrow \D$ as follows:
	$$[\pi,f]_\Pi\mapsto \mathbf F_\A[\pi,f]_\A$$
	where $[\pi,f]$ is depicted as follows:
	$$\xymatrix{&A\times B\ar[ld]_{\pi}\ar[rd]^{f}&\\
	B&&C}$$
	To show $\mathbf U$ is well defined, suppose we have the following commutative diagram with $p$ projection.
	$$\xymatrix{&& A\times B\ar[lld]_{\pi}\ar[rrd]^{f}&&\\
		B&&&&C\\
		&&A\times D\times B\ar[uu]_{p}\ar[llu]_{\pi'}\ar[rru]_{f'}&&}$$
	So $[\pi,f]_{\A\circ\D}=[\pi',f']_{\A\circ\D}$. By the naturality  of diagram we get
	$$\mathbf U[\pi,f]_\Pi=\mathbf F_\A[\pi,f]_{\A}=\mathbf F_{\A\circ\D}[\pi,f]_{\A\circ\D}=F_{\A\circ\D}[\pi',f']_{\A\circ\D}=\mathbf U[\pi',f']_\Pi$$
	Uniqueness of $\mathbf U$ can be seen easily.
\end{proof}
\section{Language of a topos}
Throughout  this section we assume $\T$ is a topos. We will show the category $\mathsf{Span}_\Pi(\T,\Pi)$ can be seen as the language of topos $\T$.
The language of topos $\T$ has objects of $\T$ as types and morphisms $[!_A,f]_\Pi:1\rightarrow B$ as terms of type $B\in\T$.
We denote terms $[!_A,f]_\Pi:1\rightarrow B$ by $\phi(x):1\rightarrow B$. Here, $x$ is used to denote $[!_A,1_A]:1\rightarrow A$. So $x$ is a term of type $A$. We call $x$ a variable of type $A$. Terms of type $\Omega$ are called formula.



\begin{definition}
	For $\alpha(x)=[!_A,f]_\Pi:1\rightarrow D$, $\beta(y)=[!_B,g]_\Pi:1\rightarrow D$ and $\gamma(z)=[!_C,h]_\Pi:1\rightarrow \mathbf PD$
	\begin{itemize}
		\item
		$\alpha(x)=\beta(y)$ is the formula
		$$\xymatrix{1\ar[rr]^{\langle \alpha,\beta\rangle}&& D\times D\ar[rr]^{[1,\delta_D]_\Pi}&& \Omega}$$
		\item
		$\alpha\varepsilon\gamma$ is the formula
		$\xymatrix{1\ar[rr]^{\langle \alpha,\gamma\rangle}&& D\times \mathbf PD\ar[rr]^{[1,ev]_\Pi}&&\Omega}$
	\end{itemize}
	For formulas $\phi(x)=[!_A,f]:1\rightarrow \Omega$ and $\psi(y)=[!_B,g]:1\rightarrow \Omega$ we define
	\begin{itemize}
		\item $\phi\wedge\psi$ is defined as follows:
		$$\xymatrix{1\ar[rr]^{\langle \phi,\psi\rangle}&& \Omega\times \Omega\ar[rr]^{[1,\wedge]_\Pi}&& \Omega}$$
		\item
		$\phi\vee\psi$ as:
		$$\xymatrix{1\ar[rr]^{\langle \phi,\psi\rangle}&& \Omega\times \Omega\ar[rr]^{[1,\vee]_\Pi }&& \Omega}$$
		\item
		$\phi\implies\psi$ as:
		$$\xymatrix{1\ar[rr]^{\langle \phi,\psi\rangle}&& \Omega\times \Omega\ar[rr]^{[1,\implies]_\Pi }&& \Omega}$$
		\item
		$not\phi$ as:
		$$\xymatrix{1\ar[rr]^{\phi}&&\Omega\ar[rr]^{[1,not]_\Pi}&&\Omega}$$
		\item $\forall\phi(x)=[1,\forall_A\tilde f]$
		\item $\exists\phi(x)=[1,\exists_A\tilde f]$
			
	where $\forall_A$ is right adjoint and $\exists_A$ is left adjoint for $\mathbf{P}(!_A):\Omega\rightarrow \mathbf(P)(A)$ and $\tilde f$ is obtained by the following diagram.
	$$\xymatrix{\mathbf P(A)\times A\ar[rr]^{ev}&&\Omega\\
	1\times A\ar[u]^{\tilde f\times 1}\ar[rr]&&A\ar[u]_{f}}$$
	\item
	For $\phi(x)=[1,f][!_A,1]:1\rightarrow A\rightarrow \Omega$:
	$$\lbrace x\in A: \phi(x)\rbrace$$ is the unique morphism $u$ obtained by the following diagram.
	$$\xymatrix{\mathbf PA\times A\ar[rr]^{[1,ev]_\Pi}&&\Omega\\
	1\times A\ar[u]^{u\times 1}\ar[urr]_{[1,f]_\Pi}&&}$$
	\end{itemize}
\end{definition}
\begin{proposition}
	$\forall\phi(x)$ and $\exists\phi(x)$ are well defined.
\end{proposition}
\begin{proof}
	Suppose we have
	$$\xymatrix{&& A\times C\ar[lld]\ar[rrd]^{f\pi}\ar[dd]_{\pi}&&\\
	1&&&&\Omega\\
	&&A\ar[llu]\ar[rru]_{f}&&}$$
	We have to show $\forall_A\tilde f=\forall_{A\times C}\tilde{f\pi}$. The following diagram implies $\tilde{f\pi}=\mathbf P(\pi)\tilde f$.
	$$\xymatrix{\mathbf P(A\times C)\times (A\times C)\ar[rr] && \Omega\\
	\mathbf PA\times (A\times C)\ar[u]_{\mathbf P\pi\times 1}\ar[rr]^{1\times \pi}&&\mathbf PA\times A\ar[u]_{ev}\\
	1\times (A\times C)\ar[u]_{\tilde f\times 1}\ar[rr]_{1\times \pi}&& 1\times A\ar[u]_{\tilde f\times 1}\ar@/_2.5pc/[uu]_{f}}$$
	By morphisms $!_{A\times C}=!_A\pi:A\times C\rightarrow A \rightarrow 1$ we have the following adjunctions.
	$$\xymatrix{\mathbf P(A\times C)\ar@<1ex>[rr]^{\forall_\pi}&&\mathbf PA\ar@<1ex>[ll]^{\mathbf P\pi}\ar@<1ex>[rr]^{\forall_A}&&\Omega\ar@<1ex>[ll]^{\mathbf P(!_A)}}$$
	So $\forall_{A\times C}=\forall_A\forall_\pi$. In external case we have the following diagrams for $\mathbf P \pi$ and $\forall_\pi$:
	\begin{center}
		$\xymatrix{D\times C\ar[rr]^{d\times 1}\ar[d]\ar@{}[rrd]|{p.b.}&&A\times C\ar[d]^{\pi}\\
		D\ar[rr]_{d}&& A\times 1}$
		\hfil
		$\xymatrix{D\times C\ar[rr]^{d\times 1}\ar[d]_{1\times !_C}\ar@{}[rrd]|{p.b.c.}&&A\times C\ar[d]^{1\times !_C=\pi}\\
		D\times 1\ar[rr]_{d\times 1}&& A\times 1}$
	\end{center}
	By the right one we get $\pi^{-1}d=d\times 1$. So we get $\forall_\pi\mathbf P\pi=1$. This implies
	$$\forall_{A\times C}\tilde{f\pi}=\forall_A\forall_\pi\mathbf P\pi\tilde f=\forall_A\tilde f$$
	
	For the case $\exists\phi(x)$, by \cite[Lemma 2.3.6]{john} we have $\exists_C\mathbf P\pi=1$. So
	$$\exists_{A\times C}\tilde{f\pi}=\exists_A\exists_C\mathbf P\pi \tilde f=\exists_A\tilde{f}$$
	
\end{proof}
\section{Relations generated by a class of endospans}


In the previous sections we  used compatible relations generated by a class of morphisms, especially the class of epimorphisms. Now we consider a more general kind of compatible relations. In this kind we replace morphisms by endospans:
\begin{center}
	$\xymatrix{&& D\ar[lld]_{p}\ar[rrd]^{q}&&\\
 	A&&&&A}$
\end{center}
If in the above endospan $p=q$ and $p$ is isomorphism, it is called endospan of an iso.
\begin{definition}
	\begin{itemize}
		\item
		A class of endospans is called saturated if it contains all of endospans of isos.
		\item
		Suppose $\A$ is a saturated class of endospans. The smallest compatible relation $\sim$ on the category $\mathsf{Span}(\C)$ such that for all $(a,b)$ in $\A$:
		$$(a,b)\sim (1,1)$$
		is called the compatible relation generated by $A$ and it is denoted by $\sim_\A$.
	\end{itemize}
\end{definition}
\begin{proposition}
	For a compatible class of endospans, $\A$, the compatible relation generated by $\A$ is described as follows:
	
	$(h,k)\sim (r,s)$ $\iff$ for decompositions
	\begin{center}
			$(h,k)=(h_n,k_n)...(h_1,k_1) $ and $(r,s)=(r_m,s_m)...(r_1,s_1)$
	\end{center}
	 and
	$(a_1,b_1),...,(a_n,b_n)\in\A$ and $(c_1,d_1),...,(c_m,d_m)\in\A$, we have
	$$(r_m,s_m)(c_m,d_m)...(r_1,s_1)(c_1,d_1)=(h_n,k_n)(a_n,b_n)...(h_1,k_1)(a_1,b_1)$$
\end{proposition}
\begin{proof}
	Obvious
\end{proof}
\begin{example}
	\begin{itemize}
		\item
		Let $\I$ be the class of all endospans of isos. The compatible relation generated by this class is defined as follows:
		$(f,g)\sim(h,k)$ if there is an isomorphism $\phi$ such that $f=h\phi$ and $g=k\phi$. So $\mathsf{Span}_{\sim}(\C)$ is the ordinary category of spans.
		\item
		For a Compatible Class of morphism, $\B$, we can form a compatible class of endospans containing $(b,b)$ for all $b\in \B$. The compatible relation generated by this class of endospans is the same as $\sim_\B$
		\item
		For a morphism $f:A\rightarrow B$ we can form a compatible endospan class by adding the kernel pair of $f:A\rightarrow B$ to $\I$, the class of all endospans of isos.
		\item
		For a morphism $f:A\rightarrow B$ we can form a compatible endospan class by adding the kernel pair of $f:A\rightarrow B$ to the class of endospans containing $(e,e)$ for epimorphisms $e$.
	\end{itemize}
\end{example}
\begin{definition}
	For a morphism $f:A\rightarrow B$, let $K(f)$ be the compatible endospan class containing kernel pair of all morphisms $h$ in which $f=gh$ for some morphism $g$ and $(e,e)$ for all epimorphisms $e$.
\end{definition}
The compatible relations generated by $K(f)$ imply $(p_1,p_2)\sim_{k(f)}(1,1)$, in which $p_1,p_2$ are obtained by the following pullback diagram:
\begin{center}
	$\xymatrix{&&P\ar[lld]_{p_1}\ar[rrd]^{p_2}&&\\
	A\ar[rrd]_{h}&&&&A\ar[lld]^{h}\\
&&B&&}$
\end{center}
where $f=gh$ for some morphism $g$.
\begin{lemma}
Using the above definitions and notations, we have:
	\begin{itemize}
		\item[(a)]
		for an epimorphism $e$, $[1,e]_{K(e)}$ is an isomorphism and its inverse is $[e,1]_{V(e)}$.
		\item[(b)]
		if $f=gh$, then $K(h)\subseteq K(f)$.
	\end{itemize}
\end{lemma}
\begin{proof}
	Obvious.
\end{proof}
\begin{definition}
	For a topos $\T$, a compatible relation $\sim$ on $\mathsf{Span}(\T)$ is called logical if,
	\begin{itemize}
		\item
		$\E\subseteq \sim$
		\item
		for spans $(f,g),(h,k):A\rightarrow C$ and morphism $a:A\rightarrow B$
		\[
		(f,g)\sim(h,k) \implies (\pi_1\forall_{a\times 1}m,\pi_2\forall_{a\times 1}m)\sim(\pi_1\forall_{a\times 1}n,\pi_2\forall_{a\times 1}n)
		\]
		where $m$ and $n$ are the $\M$ parts of $\langle f,g\rangle$ and $\langle h,k\rangle$ respectively.
	\end{itemize}
\end{definition}
The smallest logical relation containing $K(f)$ is denoted by $L(f)$.
\begin{lemma}\label{forall}
	The following diagram is formed by pulling back and $g$ is epi. We have $\forall_{g\times g}\langle q_1,q_2\rangle=\langle p_1,p_2\rangle$.
	\begin{center}
		$\xymatrix{Q\ar@/_1pc/[dd]_{q_1}\ar@/^1pc/[rr]^{q_2}\ar[r]_{v_2}\ar[d]^{v_1}&R\ar[r]^{r}\ar[d]_{r_2}&C\ar[d]^{g}\\
		R\ar[r]^{r_1}\ar[d]_{r}&P\ar[r]^{p_2}\ar[d]_{p_1}&B\ar[d]^{f}\\
		C\ar[r]_{g}&B\ar[r]_{f}&A}$
	\end{center}
\end{lemma}
\begin{proof}
	Let $(g\times g)^{-1}\langle x,y\rangle \le\langle q_1,q_2\rangle$. Then there is $i$ such that
$(g\times g)^{-1}\langle x,y\rangle=\langle q_1,q_2\rangle i$. Set $(g\times g)^{-1}\langle x,y\rangle=\langle x',y'\rangle$ and $\langle x,y\rangle^{-1}(g\times g)=e$. Since $g$ is epi, $(g\times g)$ is epi and since in a topos epimorphisms are stable under pullbacks, we have $e$ is epi as well.
	We have the following equalizer diagrams.
	\begin{center}
		$\xymatrix{P\ar[rr]^{\langle p_1,p_2\rangle}&&B\times B\ar@<1ex>[rr]^{f\pi_1}\ar@<-1ex>[rr]_{f\pi_2}&&A}$
		\hfil
		$\xymatrix{P\ar[rr]^{\langle q_1,q_2\rangle}&&C\times C\ar@<1ex>[rr]^{fg\pi'_1}\ar@<-1ex>[rr]_{fg\pi'_2}&&A}$
	\end{center}
	We have $fg\pi'_1=f\pi_1(g\times g)$ and  $fg\pi'_2=f\pi_2(g\times g)$. So we get
\[\begin{array}{ll}
fxe &=f\pi_1\langle x,y\rangle e\\
&=f\pi_1(g\times g)\langle x',y'\rangle\\
&=fg\pi'_1\langle q_1,q_2\rangle i\\
&=fg\pi'_2\langle q_1,q_2\rangle i\\
&=f\pi_2(g\times g)\langle x',y'\rangle\\
&=f\pi_2\langle x,y\rangle e\\
&=fye.
 \end{array}\]

%\begin{center}
%		$fxe=f\pi_1\langle x,y\rangle e=		
%		f\pi_1(g\times g)\langle x',y'\rangle =fg\pi'_1\langle q_1,q_2\rangle i=fg\pi'_2\langle q_1,q_2\rangle i=f\pi_2(g\times g)\langle %x',y'\rangle=f\pi_2\langle x,y\rangle e=fye$
%	\end{center}
	Since $e$ is epi, $fx=fy$. Then $\langle x,y\rangle\le\langle p_1,p_2\rangle$.
	
	By the following pullback diagrams we get $(g\times g)^{-1}\langle p_1,p_2\rangle=\langle q_1,q_2\rangle$. Then, $\langle x,y\rangle\le\langle p_1,p_2\rangle$ implies $(g\times g)^{-1}\langle x,y\rangle \le\langle q_1,q_2\rangle$.
	\begin{center}
		$\xymatrix{Q\ar[rr]^{v_1}\ar[d]_{\langle q_1,q_2\rangle}&&R\ar[rr]^{r_1}\ar[d]_{\langle r,p_2r_1\rangle}&&P\ar[d]^{\langle p_1,p_2\rangle}\\
		C\times C\ar[rr]_{1\times g} && C\times B\ar[rr]_{g\times 1}&&B\times B}$
	\end{center}
\end{proof}
\begin{corollary}
	Suppose $f=gh$ and $h$ is epi. Then $L(g)\subseteq L(f)$.
\end{corollary}
\begin{proof}
	Let $g=uv$ and consequently $f=uvh$. So the kernel pair of $v$ is related to $(1,1)$ by $L(g)$ and the kernel pair of $vh$ is related to $(1,1)$ by $L(f)$. Since the kernel pairs of $h$ and $vh$ are related by $L(f)$, by using $\forall_{h\times h}$ and Lemma \ref{forall}, kernel pair of $v$ is related to $(1,1)$ by $L(f)$.
\end{proof}
\section{Booleanization of a topos}
Allegories were defined in \cite{freyd} as categories which reflect properties that hold in the category  of relations.
\begin{definition}
	An allegory is a locally ordered $2$-category $\A$ whose hom-posets
	have binary intersections, equipped with an anti-involution $\phi \mapsto \phi^{\circ}$ and satisfying the modular law
	\[
	\psi\phi \cap \chi \leq (\psi \cap \chi\phi^{\circ})\phi,
	\]
	whenever this makes sense.
\end{definition}
\begin{definition}
	Given an allegory $\A,$ let $\mathsf{MAP}(\A)$ denote the subcategory of maps of $\A.$
\end{definition}

A power allegory is a division allegory with some extra properties. First we give the definition of a division allegory and then the definition of power allegory. See \cite{john} for more information.
\begin{definition}(\cite[Definition 3.4.1]{john})
	An allegory $A$ is called a division allegory if, for each $\phi:A\rightarrow B$ and object $C$, the order preserving map $(-)\phi:\A(B,C)\longrightarrow \A(A,C)$ has a right adjoint, which we call right division by $\phi$ and denote $(-)/\phi$.
\end{definition}
Of course, the anti-involution ensures that if we have right division we also have left division $\phi\setminus (-)$ (right adjoint to $\phi(-)$).
We write $(\phi|\psi)$ for
$$(\phi\setminus \psi)\cap (\psi\setminus \phi)^\circ.$$
\begin{definition}\cite{john}
	A division allegory $\A$ is called a power allegory if there is an operation assigning to each object $A$ a morphism $\in_A:PA\rightarrow A$ satisfying $(\in_A|\in_A)=1_{PA}$ and
	$$1_B\le (\phi\setminus\in_A)(\in_A\setminus\phi)$$
	for any $\phi:B\rightarrow A$.
\end{definition}
Given a category $\C$ let $\mathsf{Rel(\C)}$ be such that whose objects are
the same as $\C$ and whose morphisms are relations in $\C.$ The next lemma shows that under some extra conditions on
$\C$, $\mathsf{Rel(\C)}$  is a category.
\begin{lemma} (see \cite[Corollary 3.1.2]{john})
	Suppose $\C$ is a regular category. Then $\mathsf{Rel(\C)}$ is a category.
\end{lemma}
Let us recall that every topos is a regular category and has $(\E,\M)$ factorization structure, where
  $\E$ denotes the class of all epimorphisms and $\M$ denotes the class of all monomorphisms.

As a first step, we would like  to make $\mathsf{Span}_\sim(\T)$ a division allegory.
For a logical relation $\sim$, since $\E\subseteq \sim$, the mapping $Q:\mathsf{Span}_\E(\T)\longrightarrow \mathsf{Span}_\sim(\T)$ defined by $$Q([f,g]_\E)=[f,g]_\sim$$ is a representation of allegories, that means $Q$ preserves $^\circ$ and $\cap$.
\begin{theorem}
	For every Topos $\T$, $\mathsf{Span}_\E(\T)$ is a division allegory.
\end{theorem}
\begin{proof}
	 By \cite[Theorem 3.4.2]{john} and \cite[Theoem 4.2]{hsty}, $\mathsf{Span}_\E(\T)$ is a division allegory defined by
	$$[h,k]_\E/[f,g]_\E=[\pi_1a,\pi_2a]_\E$$
	where $a=\forall_{g\times 1}(f\times 1)^*(m_{\langle h,k\rangle})$, $m_{\langle h,k\rangle}$ is the mono part of $\langle h,k\rangle$
\end{proof}
\begin{theorem}
	For a logical relation $\sim$, $\mathsf{Span}_\sim(\T)$ is a division allegory and
$$Q((-)/[f,g]_\E)=(-)/[f,g]_\sim.$$
\end{theorem}
\begin{proof}
	We define $(-)/[f,g]_\sim:=Q((-)/[f,g]_\E)$. It follows from the definition of logical relation that this definition is well-defined.
	We have
	$$[h,k]_\sim=Q [h,k]_\E\le  Q(([h,k]_\E[f,g]_\E)/[f,g]_\E)=([h,k]_\sim[f,g]_\sim)/[f,g]_\sim$$
	and
	$$([r,s]_\sim/[f,g]_\sim)[f,g]_\sim=Q([r,s]_\E/[f,g]_\E)Q[f,g]_\E=Q(([r,s]_\E/[f,g]_\E)[f,g]_\E)\le Q[r,s]_\E=[r,s]_\sim$$
	So $(-)/[f,g]_\sim$ is right adjoint for $(-)[f,g]_\sim$.
\end{proof}
Since $\T$ is a topos, $\mathsf{Rel}(\T,\E,\M)$ is a power allegory and $\in_A:PA\rightarrow A$ is
$$\xymatrix{&\in_A\ar[ld]\ar[rd]\ar[d]\ &\\
	PA&PA\times A\ar[l]\ar[r]& A}$$
\begin{theorem}
	$\mathsf{Span}_\E(\T)$ is a power allegory.
\end{theorem}
\begin{proof}
	This follows from $\mathsf{Rel}(\T,\E,\M)\cong \mathsf{Span}_\E(\T)$ and that $\in_A:PA\rightarrow A$ in $\mathsf{Span}_\E(\T)$ is defined as in $\mathsf{Rel}(\T,\E,\M)$.
\end{proof}
\begin{theorem}
	For a logical relation $\sim$, $\mathsf{Span}_\sim(\T)$ is a power allegory and $\mathsf{Map(Span}_\sim(\T))$ is a topos.
\end{theorem}
\begin{proof}
	Let $\in_A:PA\rightarrow A$ in $\mathsf{Span}_\sim(\T)$ be $Q(\in_A:PA\rightarrow A)$. Since $Q((-)/[f,g]_\E)=(-)/[f,g]_\sim$ and Q is a representation, we get $\mathsf{Span}_\sim(\T)$ is a power allegory. Thus  \cite[Corollary 3.4.7]{john} implies $\mathsf{Map(Span}_\sim(\T))$ is a topos.
\end{proof}
We use $\eta$ for the functor $\mathsf{Q}:\T\longrightarrow\mathsf{Map(Span}_\sim(\T))$. So we have:
\begin{theorem}\label{logical functor}
	$\eta:\T\longrightarrow \mathsf{Map(Span}_\sim(\T))$ is a logical functor.
\end{theorem}

\iffalse
\section{Forcing}
For a topos $\T$ with terminal object $1$, Suppose $a:A\rightarrow 1$ be a monomorphism.
So $Pa:\Omega\rightarrow PA$ is an epimorphism. We form the logical relation $L(Pa)$. So we get the topos $\mathsf{Map(Span}_{L(Pa)}(\T))$. We denote this topos by $\T(a)$.

By \ref*{logical functor}, the logical functor 	$\eta:\T\longrightarrow \T(a)$ exists.
\begin{definition}
	For $p$ and $q$ in $\mathsf{Sub}(1)$, we say $p$ forces $q$, $p\Vdash q$, if $\T(p)\models q$.
\end{definition}
\begin{proposition}
	For $p,r$ in $\mathsf{Sub}(1)$, suppose $r\le p$. Then the map $\T(p)\longrightarrow \T(r)$ by sending $[f,g]_{L(Pp)}$ to $[f,g]_{L(Pr)}$  is a functor.
\end{proposition}
\begin{proof}
	This comes from the definition of the relations.
\end{proof}
\fi
\section{Morphism classes yielding a logical relation}
\begin{definition}
	We call a class of morphism $\W$  a logical class if
	\begin{itemize}
		\item
		$\W$ is closed under composition and closed under pullbacks and contains isomorphisms,
		\item
		$\E\subseteq \W$,
		\item
		for each $w\in\W$, $\M$-part of $w$ is in $\W$,
		\item
		for monomorphism $m\in\W$ and for monomorphism $f$ and morphism $g$ in $\T$, $\forall_gm$ is in $\W$
		$$\xymatrix{\ar[r]^{m} & \ar[r]^{f} &\ar[d]^{g}\\
		\ar[r]_{\forall_gm} & \ar[r]_{\forall_gf}&}$$
	\end{itemize}
\end{definition}
\begin{theorem}
	$\sim_W$ is a logical relation.
\end{theorem}
\begin{proof}
	This is obvious.
\end{proof}
As an application we will use this technique to make a topos boolean. The morphism $b:1+1\rightarrow \Omega$ is mono in every topos. We will try to make this an isomorphism in a logical manner to construct a boolean topos.
Let the class $\B(\T)$ denote the least logical class containing $b:1+1\rightarrow \Omega$.
\begin{theorem}
	$\mathsf{Map(Span}_{\B(\T)}(\T))$ is a boolean topos.
\end{theorem}
\begin{proof}
	Since $(b,b)\sim_{\B(\T)}(1,1)$, we have $[b,b]_{\B(\T)}=[1,1]_{\B(\T)}$. So $[1,b]_{\B(\T)}$ is a retraction. Because $b$ is mono, we get $[b,1]_{\B(\T)}[1,b]_{\B(\T)}=1$. So $[1,b]$ is an isomorphism in $\mathsf{Span}_{\B(\T)}(\T)$ and then in $\mathsf{Map(Span}_{\B(\T)}(\T))$.
	By Theorem \ref{logical functor}, $\eta:\T\longrightarrow \mathsf{Map(Span}_{\B(\T)}(\T))$ is a logical functor. By  \cite[Corollary 2.2.10]{john}, $\eta$ is cocartesian. So
	$$\eta(b:1+1\rightarrow \Omega)=[1,b]:1+1\rightarrow \Omega.$$
	Thus $\mathsf{Map(Span}_{\B(\T)}(\T))$ is a boolean topos, as claimed.
\end{proof}
We may note that this construction is universal.
\begin{lemma}\label{FB(B"}
	For a logical functor $F:\T\rightarrow\T'$, $F(\B(\T))\subseteq\B(\T').$
\end{lemma}
\begin{proof}
	Since $F$ is a logical functor, it preserves epi, mono and $\forall$. So one can easily verify $F^{-1}(\B(\T'))$ is a logical class. $F(b)=b'$ implies $b\in F^{-1}(\B(\T'))$. So $\B(\T)\subseteq F^{-1}(\B(\T'))$ and we get $F(\B(\T))\subseteq \B(\T')$.
\end{proof}
\begin{theorem}\label{logical functor gives logical functor}
	For a logical functor $F:\T\rightarrow\T'$
	\begin{itemize}
		\item[(a)]
		the mapping $PF:\mathsf{Span}_{\B(\T)}(\T)\longrightarrow\mathsf{Span}_{\B(\T')}(\T')$ by taking $[f,g]_{\B(\T)}$ to $[Ff,Fg]_{\B(\T')}$ is a representation of allegories.
		\item[(b)]
		$\mathsf{Map}(PF):\mathsf{Map(Span}_{\B(\T)}(\T))\longrightarrow\mathsf{Map(Span}_{\B(\T')}(\T'))$ is a logical functor.
	\end{itemize}
\end{theorem}
\begin{proof}
For clause (a), note that  by lemma \ref{FB(B"}, the map is well-defined. The rest of proof comes from the fact that $F$ preserves pullbacks.
	Clause (b) follows from the definition of $\in_A$ in $\mathsf{Span}_{\B(\T)}(\T)$ and $\mathsf{Span}_{\B(\T')}(\T')$.
\end{proof}
Now let $\mathsf{BoolTop}$ be the category with boolean toposes as objects and logical functors as morphisms. This is a subcategory of the category $\mathsf{Top}$ of toposes and logical functors. Using Theorem \ref{logical functor gives logical functor}, we can define the  functor
$$Bool:\mathsf{Top}\longrightarrow \mathsf{BoolTop}$$
in which $Bool(F)$ and $Bool(\T)$ are used instead of $\mathsf{Map}(PF)$ and $\mathsf{Map(Span}_{\B(\T)}(\T))$ respectively.


\begin{theorem}
	$\mathsf{BoolTop}$ is a reflective subcategory of $\mathsf{Top}$.
\end{theorem}
\begin{proof}
	We will show the functor $Bool$ is a left adjoint for the inclusion functor.
	It is enough to show
	$$\eta:\T\rightarrow \mathsf{Map(Span}_{\B(\T)}(\T)=\imath\cdot Bool(\T)$$
	is universal, where $\imath$ denotes the inclusion functor $\imath: \mathsf{BoolTop} \longrightarrow \mathsf{Top}$.

Let $F:\T\rightarrow \T'=\imath(\T')$ be a logical functor and $A$ a boolean topos. Since $F$ is cocartesian and $\T'$ is boolean, $F(b:1+1\rightarrow\Omega)=1+1\rightarrow \Omega$ is an isomorphism in $\T'$. Using Theorem \ref{logical functor gives logical functor}, we have the functor
	$Bool(F):Bool(\T)\rightarrow Bool(\T')$. It is easy to verify $\B(\T')=\E$ and then $Bool(\T')=\T'$. So we have the following commutative triangle.
	$$\xymatrix{\T\ar[rrrr]^{\eta} \ar[d]_{F}&&&& \mathsf{Map(Span}_{\B(\T)}(\T)=\imath\cdot Bool(\T)\ar[lllld]^{Bool(F)}\\
	\T'&&&&}$$
For uniqueness, let $[f,g]$ be a map in $\mathsf{Span}_\B(\T)$. So $[f,1]$ is an iso and the inverse is $[1,f]$. For the functor $G$, set $F=G\eta$. Since $[f,g]=[1,g][f,1]$, we have
\[\begin{array}{ll}
G[f,g]&=G[1,g]G[f,1]\\
&=G[1,g](G[1,f])^{-1}\\
&=G\eta(g)(G\eta(f))^{-1}\\
&=F(g)(F(f))^{-1}\\
&=Bool(F)[f,g].
 \end{array}\]
%$G[f,g]=G[1,g]G[f,1]=G[1,g](G[1,f])^{-1}=G\eta(g)(G\eta(f))^{-1}=F(g)(F(f))^{-1}=Bool(F)[f,g].$
The result follows.
\end{proof}


\begin{thebibliography}{10}
\bibitem{freyd} Freyd, Peter J.; Scedrov, Andre Categories, allegories. North-Holland Mathematical Library, 39. North-Holland Publishing Co., Amsterdam, 1990. xviii+296 pp. ISBN: 0-444-70368-3; 0-444-70367-5.

\bibitem{hoshitho} Hosseini, S. N.; Shir Ali Nasab, A. R.; Tholen, W.; Fraction, restriction, and range categories from stable systems of morphisms. J. Pure Appl. Algebra 224 (2020), no. 9, 106361, 28 pp.

\bibitem{hsty}  Hosseini, S. N.; Shir Ali Nasab, A. R.; Tholen, W.;  Yeganeh, L.;  Quotients of span categories that are allegories and the representation of regular categories, https://arxiv.org/abs/2112.04599



    \bibitem{john} Johnstone, Peter T. Sketches of an elephant: a topos theory compendium. Vol. 1. Oxford Logic Guides, 43. The Clarendon Press, Oxford University Press, New York, 2002. xxii+468+71 pp. ISBN: 0-19-853425-6

\bibitem{lambscot} Lambek, J.; Scott, P. J. Introduction to higher order categorical logic. Cambridge Studies in Advanced Mathematics, 7. Cambridge University Press, Cambridge, 1986. x+293 pp. ISBN: 0-521-24665-2

\bibitem{macmor} Mac Lane, Saunders; Moerdijk, Ieke Sheaves in geometry and logic. A first introduction to topos theory. Corrected reprint of the 1992 edition. Universitext. Springer-Verlag, New York, 1994. xii+629 pp. ISBN: 0-387-97710-4.
\end{thebibliography}

\noindent
School of Mathematics,
\noindent
Institute for Research in Fundamental Sciences (IPM),
\noindent
P.O. Box:
19395-5746,
\noindent
Tehran-Iran.

\noindent
{\emph E-mail address:} golshani.m@gmail.com
\begin{center}
\end{center}
\noindent
Mathematics Department,
Shahid Bahonar University of Kerman,
Kerman, Iran.
\\
{\emph E-mail address:} ashirali@math.uk.ac.ir




\end{document}
