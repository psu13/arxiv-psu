
% !TEX root = replicas_draft.tex

\section{The equation of motion in Lorentzian signature}
\label{app:lorentzian}

The Hilbert transform appearing in the equation of motion \eqref{hilbertp} has a nice interpretation in Lorentzian signature. It is responsible for the dissipation of energy into the thermal bath outside. This makes contact with the Schwarzian equation for black hole evaporation studied in \cite{Engelsoy:2016xyb,Almheiri:2019psf}.

In this appendix we set $n=1$, but allow for CFT operators inserted in the non-gravitational region. The perturbative Schwarzian equation in Euclidean signature is
\be
\p_\tau S + i \kappa \Hilbert \cdot S = i \kappa {\cal F}
\ee
where $S = \delta \{e^{i\theta}, \tau\}$ and
\be
{\cal F} =  T_{yy}(i\tau) - T_{yy}(-i\tau)  \ .
\ee
We separate this into positive and negative frequencies on the Euclidean $\tau$-circle,
\begin{align}
\p_\tau S_+ - i  \kappa S_+ &= i \kappa {\cal F}_+ \\
\p_\tau S_- + i  \kappa S_- &=  i\kappa {\cal F}_- \ .
\end{align}
Here the `$+$' terms include only the non-negative powers of $e^y$, and the `$-$' terms have the negative powers.
Now continuing to Lorentzian signature with $\tau = it$, this becomes
\begin{align}\label{lorentzianshock}
\p_t S_\pm \pm \kappa S_\pm = -\kappa {\cal F}_\pm
\end{align}
This is the Lorentzian equation of motion. As an example, consider a state with two scalar operators ${\cal O}(y_1) {\cal O}(y_2)$ inserted at 
\be
y_1 = L + i \delta , \qquad y_2  = \by_1 =  L - i \delta \ ,
\ee
with $0 < \delta \ll L $. This creates a shockwave that falls into the AdS region at time $t \approx L$. The state is time-symmetric, so there is also a shockwave exiting the AdS region at $t \approx -L$.
The stress tensor is 
\be
T_{yy}(y) = -  \frac{h_O}{2\pi}  \frac{v^2(v_1 -v_2)^2}{(v-v_1)^2(v-v_2)^2} \ ,
\ee
with $v = e^{y}$. The projections onto positive and negative Euclidean frequencies are
\be
{\cal F}_+ = - \frac{h_O}{2\pi} \frac{v^2(v_1 - v_2)^2}{(v-v_1)^2(v-v_2)^2}  \ , \qquad
{\cal F}_-  = \frac{h_O}{2\pi} \frac{v^2(v_1-v_2)^2}{(1-v_1 v)^2(1-v_1 v)^2} \ .
\ee
In Lorentzian signature this becomes
\begin{align}
{\cal F}_+ &=  \frac{h_O\sin^2\delta }{2\pi (\cos\delta  - \cosh(L+t))^2} \\
{\cal F}_- &= -\frac{h_O \sin^2\delta }{ 2\pi (\cos\delta - \cosh(L-t))^2 } 
\end{align}
As $\delta \to 0$, these vanishes away from the singularities, leading to
\begin{align}
\p_t S_+ + \kappa S_+ = - \kappa E_O \delta(t+L) \\
\p_t S_- - \kappa S_- =  \kappa E_O \delta(t-L)  \ ,
\end{align}
where $E_O = h_O/\delta $. 
The delta functions are the shockwaves exiting and entering the AdS region. The signs here, and in particular the extra minus sign from the Hilbert transform, ensure that there is a sensible solution for the Schwarzian, which is time-symmetric and goes to zero as $t\to \pm \infty$. The solution is
\be
S_+ = \Theta(-t-L)
\kappa E_0 e^{\kappa(t+L)} 
\ ,  \qquad
S_- = 
\Theta(t-L) \kappa E_0 e^{\kappa(L-t)} \ .
\ee
For $t>0$, this is essentially the same solution as the evaporating black hole in \cite{Almheiri:2019psf}, which had a shockwave produced by a joining quench rather than an operator insertion. 