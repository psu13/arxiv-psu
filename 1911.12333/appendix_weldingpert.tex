
% !TEX root = replicas_draft.tex

\section{Linearized solution to the welding problem}
\label{app:welding}

Let us start with a discussion of the symmetries of the welding problem   \nref{WeldEqns}. 
First we can imagine doing $SL(2,C)$ transformations of the $z$ plane. These move around the point at infinity, and we would need to   allow a pole  in the functions $F$ or $G$. If we fix that $F(\infty ) = \infty$, then we can then impose
that the functions are holomorphic everywhere, with no poles,  and this group is reduced to just translations, scalings and rotations of the plane $z$. 
None of these transformations change the data for the welding problem which is $\theta(\tau)$. 
In addition, we have two $SL(2,R)$ transformations, one acting on $w$ and one acting on $v$, both preserving the circles 
$|w|=1$ and $|v|=1$. These change the data 
of the welding problem by an $SL(2,R)$ transformation of $e^{ i \theta}$ or $e^{i \tau}$ respectively. They map a solution of a 
welding problem with $\theta(\tau)$ to a solution of a different welding problem given by the transformed function. 
In our combined gravity plus CFT problem, we are integrating over $\theta(\tau)$, so we can look for symmetries that change $\theta(\tau)$.  It turns out that the $SL(2,R)_v$ that acts on the $v$ plane is {\it not} a symmetry. It changes the 
Schwarzian action, for example. On the other hand, the $SL(2,R)_w$ is actually a    gauge symmetry, when we also act with the
$SL(2,R)$ transformation on the possible locations, $w_i$, of the conical singularities. 



Consider a plane with coordinate $w$ inside the unit disk, and $v$ outside, as in fig.~\ref{WandZplanes}. The plane is glued along the unit circle with a gluing function $\theta(\tau)$, where $w = e^{i\theta}$ and $v = e^{i\tau}$. The solution to the welding problem is a pair of functions
\begin{align}
z &= G(w) \qquad (\mbox{inside})\\
z &= F(v) \qquad (\mbox{outside})
\end{align}
where $G$ is holomorphic inside the disk, and $F$ is holomorphic outside the disk.  In this appendix we will solve for $F,G$ perturbatively, assuming the gluing is close to the identity, $\theta(\tau) = \tau + \delta \theta(\tau)$. Here we are considering $\delta \theta(\tau)$ to be a fixed input to the problem of finding $F$ and $G$.

Expand in Fourier modes,
\begin{align}
\theta(\tau) = \tau  + \sum_{m=-\infty}^{\infty} c_m e^{i m \tau} \ , \quad
G(w) = w + \sum_{\ell=0}^{\infty} g_\ell w^{\ell}  \ , \quad
F(v) =  v  + \sum_{\ell=-\infty}^2 f_\ell v^{\ell} \ .
\end{align}
Here $c_m, d_1^{\ell}$, and $d_2^{\ell}$ are considered small. There is an SL(2) ambiguity in the zeroth order solution, which we have gauge-fixed to set these maps to the identity. (Note that this is different from the choice in the main text around eqn \eqref{Fleading}.) The matching condition on the unit circle is
\be
G(e^{i \theta(\tau)}) = F(e^{i\theta}) \ .
\ee
At the linearized level, this sets
\begin{align}
%c_\ell &= - i d_2^{\ell+1} \qquad (\ell \leq -2)\\
%c_\ell &=  i d_1^{\ell+1} \qquad (\ell \geq 2) 
 f_{\ell+1} &= i c_\ell\qquad (\ell \leq -2)\\
g_{\ell+1} &=  -i c_\ell \qquad (\ell \geq 2) \end{align}
and
\begin{align}
ic_{-1} =  f_2 -g_2 
%d_1^0 - d_2^0 
 \ , \quad
ic_0 = f_1 -g_1
% d_1^1 - d_2^1 
\ , \quad 
ic_1 = 
%d_1^2 - d_2^2 
f_2 -g_2 \ .
\end{align}
There an ambiguity by a small  $SL(2,C)$ action on the $z$ plane.   
We can fix it by setting $G(0)=0$, $F(v) = v + $constant, as $v\to \infty$. This amounts to three complex conditions 
that set 
\be
g_0=f_{2} = f_1 =0 
\ee
% If we choose the origin of the gluing map so $\delta \theta(0) = 0$, then we can fix it be setting $z(0) = 0$, $z'(0) = 1$, $z(\infty) = 0$, $z'(\infty) = 1$, which imposes
%\be
%d_1^0 = d_1^1 = d_2^2 = d_2^1 = c_0 = 0 \ .
%\ee
This now implies that we get a unique solution for the remaining coefficients in terms of the $c_m$ 
%
%Then the solution is
\begin{align}
f_l  = i c_{\ell-1} ,   ~~~~{\rm for}~~\ell \leq 0 ~;  \qquad ~~~~~~~~
g_\ell = - i c_{\ell-1} ~,~~~{\rm for}~~~ \ell > 0  \ .
\end{align}
From here we can calculate
\be
v^2 \{ F, v \} =  \sum_{\ell=-\infty}^{-2} \ell(\ell^2-1)i c_\ell v^{\ell } \ .
\ee
Comparing to $\{w, \tau\} = \{ e^{i\theta}, \tau \}$ gives the relation used in the main text,
\be
e^{2i\tau}\{F, v \} = - \delta\{w, \tau\}_- = - ( \delta \theta''' + \delta \theta' )_- \ .
\ee

