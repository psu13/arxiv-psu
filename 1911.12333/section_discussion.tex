% !TEX root = replicas_draft.tex

In this paper, we have exhibited non-perturbative effects that dramatically reduce the late time von Neumann entropy of quantum fields outside a black hole.

The computation of the Renyi entropies corresponds to the expectation value of a swap or cyclic permutation operator in $n$ copies of the theory. Systems with very high entropy have very small, exponentially small,  expectation values for this observable. This means that non-perturbative effects can compete with the naive answers. In particular, the Hawking-like computation of the Renyi entropies of radiation corresponds to a computation on the leading gravitational background. A growing entropy corresponds to an exponentially decreasing expectation value for the cyclic permutation operator. It decreases exponentially as time progresses. 
 For this reason, we need to pay attention to other geometries, with other topologies. These other topologies give exponentially small effects, but they do not continue decreasing with time for long times. 
 Said in this way, the effects are vaguely similar to the ones discussed for corrections of other exponentially small effects 
 \cite{Maldacena:2001kr,Saad:2018bqo,Saad:2019lba,Saad:2019pqd}. 
 Though the Renyi entropies are small,  the von Neumann entropy is large and  the new series of saddles gives rise to a constant von Neumann  
 entropy at late times.  More precisely, we can think of the computation of the Renyi entropies in the two interval case as an insertion of a pair of external cosmic branes in the non-gravitational region. As time progresses these are separated further and further through the wormhole. Eventually the dominant contribution is one where a pair of cosmic branes is created in the gravitational region that ``screen'' the external ones, giving an entropy which is the same as that of two copies of the single interval entropy. 
 
   These other topologies are present as subleading saddles also at short times (perhaps as complex saddles) where we can analyze them using Euclidean methods and then analytically continue. We have only done this analytic continuation for the von Neumann entropies, not the Renyi entropies. It would be interesting to do it more explicitly for the Renyi entropies.
  
  There have been discussions on whether small corrections to the density matrix, of order $e^{ -S_{BH}}$,
 could or could not restore unitarity. These results suggest that they interfere constructively to give rise 
 to the right expression for the entropy. 
  
  This is evidence that including  nonperturbative gravitational effects can indeed lead to results compatible with unitarity.   However, we emphasize that this is not a full  microscopic resolution of the information paradox. We have not given a gravitational description for   the 
   $S$-matrix describing how infalling matter escapes into the radiation. 
   In this sense, these  results are on a footing similar to the Bekenstein-Hawking calculation of the entropy, which uses a Euclidean path integral to compute the right answer but does not give an explicit  Hilbert space picture for what it is counting. 
   In contrast, the Strominger-Vafa computation of the entropy \cite{Strominger:1996sh} gives us an explicit Hilbert space, but not a detailed description of the   microstates in the gravity variables.
    Something similar can be said of the CFT description in AdS/CFT. Hopefully these results will be useful for providing a more explicit map. 
   
   It is amusing to note that wormholes were initially thought to destroy information 
   \cite{Hawking:1987mz,Lavrelashvili:1987jg,Giddings:1987cg}. 
   But more recently the 
   work of \cite{Saad:2019lba,Saad:2019pqd},  as well as the present discussion, and \cite{Penington:2019kki}, suggests that the opposite is true. Wormholes are important for producing results that are compatible with unitarity. 
   For earlier work in this direction see also    \cite{Coleman:1988cy,Giddings:1988cx,Polchinski:1994zs}.
   


We assumed that $c\gg 1$ as a blanket justification for analyzing the equations classically. However, even for small $c \sim 1$, the basic picture for the Page curve can be justified. The basic point is simple. First consider the single interval computation. In that case for $c\sim 1$ we see that the correction to the black hole solution is very small, for all the Renyi entropies. In other words, we find that $A$ is small, and we can probably not distinguish such a small value of $A$ from zero but that does not matter, the geometries and the entropies are basically those of a black hole. Now when we go to two intervals, and we consider the late time situation, then all that really matters is that we can do an OPE-like expansion of the twist operator insertions. The important observation is that the twist operator insertions in the interior of the black hole are very far from each other. This is the fact that the wormhole is getting longer \cite{Hartman:2013qma,Susskind:2014moa}. Then the solution becomes similar to two non-interacting copies of the single interval solution.
The fact that $c$ is small only implies that we will have to wait longer for the island solution to dominate.  We just have to wait a time of order the entropy, $t \propto  {\beta } (S-S_0)/c$ for it to dominate.  

In \cite{Saad:2019lba}, it was argued that pure JT gravity should be interpreted in terms of an average over Hamiltonians. In addition, higher genus corrections were precisely matched. 
This has raised the question of whether the corrections we are discussing in this paper crucially involve an average over Hamiltonians, or whether they would also apply to a system which has a definite Hamiltonian. 
Though JT gravity plus a CFT probably does not define a complete quantum gravity theory, it seems likely that well defined theories could be approximated by   JT gravity plus a CFT. For example, we could imagine an AdS/CFT example that involves an extremal black hole such that it also has a CFT on its geometry. All we need is this low energy description, the theory might have lots of other massive fields which will not drastically participate in the discussion. They might lead to additional saddles, but it seems that they will not correct the saddles we have been discussing. And we have the seen that the saddles we discussed already give an answer consistent with unitarity, at least for the entropy.  In contrast with \cite{Saad:2019lba}, we are not doing the full path integral, we are simply using a saddle point approximation, so the JT gravity plus CFT only needs to be valid around these saddles.
  
    
  As we mentioned in the introduction, the setup in this paper can be 
  viewed as an approximation to some magnetically charged near extremal four dimensional black holes \cite{Maldacena:2018gjk}. 
  But one could analyze more general asymptotically flat black holes and wonder how to define either exactly or approximately the various entropies involved. In particular, to have a sharp definition of the entropy of radiation it seems important to go to null infinity. 
    
  Another interesting question is whether we can give a Lorentzian interpretation to the modification of the density matrix implied by the existence of replica wormholes.  
    
 It has been pointed out that a black hole as seen from outside looks like a system obeying the laws of hydrodynamics. For this reason, it is sometimes thought that gravity is just an approximation that intrinsically loses information. Here we see that if we include the black hole interior, and we do a more complete gravity computation, we can get results compatible with unitarity. The fact that gravity is more than dissipative hydrodynamics is already contained in the Ryu-Takayanagi formula for the fine grained entropy, which shows that the geometry of the interior can discriminate between pure and mixed states for a black hole.  




  



\vspace{1cm}
\textbf{Acknowledgments} We are grateful to  
Tarek Anous, Raphael Bousso, Kanato Goto, Daniel Harlow, Luca Iliesiu, Alexei Kitaev, Alexandru Lupsasca, Raghu Mahajan, Alexei Milekhin, Shiraz Minwalla,  Geoff Penington, Steve Shenker, Julian Sonner, Douglas Stanford, Andrew Strominger, Sandip Trivedi and Zhenbin Yang 
for helpful discussions of this and related work. 
AA and TH thank the organizers of the workshop \textit{Quantum Information In Quantum Gravity} at UC Davis, August 2019, and AA, TH, and JM thank the organizers of the workshop \textit{Quantum Gravity in the Lab} at Google X, November 2019. A.A. is supported by funds from the Ministry of Presidential Affairs, UAE.
The work of ES is supported by the Simons Foundation as part of the Simons Collaboration on the Nonperturbative Bootstrap. The work of TH and AT is supported by DOE grant DE-SC0020397.
J.M. is supported in part by U.S. Department of Energy grant DE-SC0009988 and by the Simons Foundation grant 385600.
