% !TEX root = replicas_draft.tex

\section{Derivation of the gravitational action} 
\la{GravAct}

In this appendix we derive the action that leads to the equation of motion 
\nref{EOMFin}. 

We start with the expansion of the metric near the boundary  \nref{Metdr}
  \nref{DelRhoe}
\be
ds^2 = { 4 dw d\bar w \over (1 - |w|^2)^2}  \left(1 - { 2 \over 3} (1-|w|)^2 U(\theta) + \cdots  \right).
\ee
We now write in terms of the variables $w = e^{-\gamma} e^{ i\theta }$ and expand it in powers of $\gamma$ as 
\be
ds^2 = { d \theta^2 \over \gamma^2 } +  { d \gamma^2 \over \gamma^2 } - { 2 \over 3 } d\theta^2 U(\theta) \,.
\ee
We now equate this to $ds = { d\tau \over \epsilon}$, we set $\theta = \theta(\tau)$  and solve 
for $\gamma$ in a power series 
\be
\gamma = \epsilon  \theta' \left[1  + \epsilon^2 \left( \half {  {\theta''}^2 \over { \theta'}^2 } - { 1 \over 3}
 U(\theta) {\theta'}^2 \right) + \cdots \right].
\ee
We can now compute the tangent vector to the curve $t^\mu$ and the normal vector $n^\mu$ and compute the
extrinsic curvature from 
\be
K = t^\mu t^\nu \nabla_\mu n_\nu = 1 + \epsilon^2 \left[ \{ \theta , \tau \} +  \left( \half + U(\theta)\right) {\theta'}^2 \right].
\ee
Up to the purely topological term, the gravitational action \nref{newgra}
 reduces to the extrinsic curvature term 
 \be
 - I_{\rm grav} = { 1 \over 4 \pi }  { \phi_r \over \epsilon } \int { d \tau \over \epsilon } 2 K =  { 2 \phi_r \over 4\pi \epsilon^2} \int d\tau + {\phi_r \over 2 \pi }  \int d\tau \left[ \{ \theta(\tau) , \tau\} +  \left( \half + U(\theta)\right) {\theta'}^2 \right] + o(\epsilon)\,.
 \ee
 The first term is a purely local divergence that can be viewed as the correction to the vacuum energy. 
 We should also remark that we can always choose a coordinate $x$ where the metric locally looks like the standard Poincare coordinates. In those coordinates the action is simply $\{x,\tau\}$. However, we will have a nontrivial identification for $x$ as we move from $\tau \to \tau + 2\pi $. Here we simplified the boundary condition, it is just $\theta = \theta + 2 \pi$, but we complicated a bit the action. Notice that
 we can think of $U(\theta)$ as a stress tensor, the change of coordinates is basically the same that we use to 
 transform this stress tensor to zero. In other words, $x(\theta)$ is a function which obeys $\{x,\theta \} = \half + U(\theta)$. 
 
The conserved energy of the system is given by 
 \be
  E = {\phi_r \over 2 \pi }    \left[ \{ \theta(\tau) , \tau\} +  
  \left( \half + U(\theta)\right) {\theta'}^2 \right] .
  \ee
  
  
  We now compute $U(\theta)$ for the case when  we put a conical defect at point $A$ in the $w$ plane. We have
  the metric \nref{metrictil} and the change of coordinates \nref{wtildew} which imply that 
  \bea
  ds^2 &=& \left| { d \tilde w \over d w } \right|^2 { 4 |dw|^2 \over (1 - |\tilde w|^2 )^2 }  
  \cr 
   &=& { 4 |dw|^2 \over (1 - |w|^2)^2 } \left[ 1 - { 2 \over 3} (1 - |w|)^2 U(\theta)  + \cdots \right] ~,~~~{\rm as} ~~|w|\to 1 
  \eea
  with 
  \be
   U(\theta) = -\half \left( 1 - { 1 \over n^2} \right)  \frac{(1-A^2)^2   }{ (e^{i\theta}  - A)^2(e^{-i \theta}  - A  )^2}   \,,
   \ee
   which leads to the same action as \nref{Rdefi}
  
   We now would like to derive the equations of motion for this action. In particular, we would like to see that
 as $\theta \to \theta + \delta \theta$ we get the right equations of motion. The change in gravitational action is 
 simple, we just have 
 \be \la{VarSch}
  - \delta I_{\rm grav} = - { \phi_r \over 2 \pi } \int d\tau  { \left[  \{ \theta(\tau) , \tau\} +  ( \half + U(\theta)) {\theta'}^2 \right]' \over \theta' }\, \delta \theta \,.
  \ee


  Now, let us do the variation of the CFT part. 
  Imagine that we choose locally complex coordinates so that 
   \be 
   \log w = s + i \theta  
   \ee
   We also have the outside coordinates $y = \sigma + i \tau$ and we can locally think of the relation between the two in terms of $\log w = i \theta(-iy)$. 
   Now imagine that we do a small change $\theta(\tau) \to \theta + \delta \theta$ with $\delta \theta$ with compact 
   support. This would change the relation between the two sides. However, let us imagine we instead keep the relation fixed, set by $\theta(\tau)$ and we redefine the outside coordinate by an infinitesimal  reparametrization, 
   $ \tilde y = y + \zeta^y $ in such a way that the relation between the new variables is the same as the old one
   \be \la{DefTvf}
   \log w = i \theta (-i \tilde y) = i \theta(-i y) + i \delta \theta(-iy) = i \theta(-iy) + \theta'(-iy) \zeta^y  ~~~~~~\to ~~~~ 
   \zeta^y = i { \delta \theta \over \theta' } 
   \ee
   and we have the complex conjugate expression for $\zeta^{\bar y}$. 
   We can then extend this reparametrization in a non-holomorphic way in the region outside, defining 
   \be
    \tilde \zeta^y = i { \delta \theta(-i y) \over \theta'(-iy ) } h(\sigma) ~,~~~~~~~~ \tilde \zeta^{\bar y} =- i 
   { \delta \theta(i \bar y ) \over \theta'(i \bar y ) } h(\sigma)
   \ee
   where $h(\sigma)$ is one for $\sigma =0$ and quickly goes to zero at $\sigma $ increases. An example is 
   $h(\sigma ) = \theta(\sigma_0 -\sigma )$ for a small $\sigma_0$. 
   This change of coordinates is equivalent to a change in metric 
   \be
   ds^2 = dy d\bar y = d\tilde y d\bar {\tilde  y} - 2 \partial_{\alpha } \zeta^{\beta} d{\tilde y}^\alpha d{\tilde y}^\beta ~,~~~~~
   \delta g_{\alpha \beta } = - 2 \partial_{(\alpha } \zeta_{\beta)} 
   \ee
  This differs from the original metric by some terms that are localized near the point where we are doing the variation.   The relation between $\log w$ and
   the $\tilde y$ variable was the same as it was before we did the variation, due
   to our choice of $\tilde y$ variable in \nref{DefTvf}. Furthermore, far from the
   region where we are doing the variation, both variables coincide. 
  Thus, the only thing we are doing is locally changing the metric of the outside region. 
Using the definition of the stress tensor, $T_{\alpha \beta} = - { 2 \over  \sqrt{g} } { \delta  \over \delta g^{\alpha \beta } } \log Z $, we get  
  \bea  \la{InteRes}
  \delta \log \hat Z_M &=& 
  -
  \half 
  \int d\varphi d \sigma ( T_{yy }   \delta g^{yy} +      T_{\bar y \bar y }  \delta g^{\bar y \bar y } )
 \cr 
  &=&
 - 2
  \int d\varphi d \sigma ( T_{yy }  \partial_{\bar y}  \zeta^{y}  + T_{\bar y \bar y }  \partial_{  y }  \zeta^{\bar y} )\,,
  \eea
  where we used that the background metric is flat and that the trace of the stress tensor is zero. 
     We now use evaluate the derivatives 
  \be
  \partial_{\bar y} \zeta^{y } = { i \over 2 } { \delta \theta(-i y ) \over \theta'(-i y) } h'(\sigma)   ~,~~~~~~~~~~  \partial_{  y} \zeta^{\bar y}= -{ i \over 2 }  { \delta \theta (i\bar y ) \over \theta'(i\bar y) } h'(\sigma) ~,~~~~~~~~~~~~~ h' = - \delta(\sigma -\sigma^0)\,.
\ee
Here we used that the arguments of $\delta \theta$ and $\theta'$ are holomorphic or antiholomorphic, so the derivative receives only a contribution from $h$, which is just a delta function. Inserting this into \nref{InteRes}, integrating over $\sigma$, and taking $\sigma^0\to 0$, we get 
\be
\delta \log \hat Z_M =    i  \int d\tau ( T_{yy } -   T_{\bar y\bar y } ) { \delta \theta \over \theta' }   \,.
\ee
Using \nref{VarSch} we get the appropriate equation \nref{EOMFin} after cancelling the $1/\theta'$ factor
from both sides.   

 