\section{The path space} \label{sec:pathspace}

As mentioned in Sections \ref{sec:inductive_intro} and
\ref{sec:groupoids}, the type theoretic \emph{identity type} is
interpreted homotopy theoretically as the path space.  The path space,
and the corresponding type in Coq, is so important that we will now
carefully describe several basic constructions involving it in the
setting of Coq.
\begin{figure}[ht]
  \centering
  \begin{tikzpicture}[color=mydark, fill=mylight, line width=1pt]
    %% THE MAIN SPACE
    \draw[fill=mylight] plot [smooth cycle,tension=.75] coordinates
    {(-2.5,0) (-2,1.5) (0,1.25) (1.75,1.5) (2.75,0) (1.75,-1) (-1,-1)};
    \draw[fill=white] plot [smooth cycle,tension=.75] coordinates
    {(0,.2) (.5,.45) (1,.2) (.5,-.05) };
    % \draw[color=magenta,line width=.1pt] plot coordinates
    % {(.8,.2) (1.2,.45) (1.7,.2) (1.2,-.05) } -- cycle;
    % \draw[color=magenta,line width=.1pt] plot coordinates
    % {(-2.5,0) (-2,1.5) (0,1.25) (1.75,1.5) (2.75,0) (1.75,-1) (-1,-1)} -- cycle;
    %% The basepoint a:
    \node[circle,fill=mydark,inner sep=0pt,outer sep=4pt,minimum size=.75mm] (aa)
    at (.2,-.4) {};
    \node at (.15,-.6) {$\scriptstyle a$};
    %% The point b:
    \node[circle,fill=mydark,inner sep=0pt,outer sep=4pt,minimum size=.75mm] (bb)
    at (-1,.75) {};
    \node at (-1.2,.9) {$\scriptstyle b$};
    %% Path around hole:
    \draw[color=mydark,->-=.5,line width=.5pt] plot [smooth,tension=.75]
    coordinates { (aa) (1.2,.2) (.4,.8) (bb) };
    \node[font=\tiny] at (1.3,.2) {$k$};
    %% The homotopy:
    \setcounter{myi}{0}
    \foreach \i in {1,...,9}
    {
      \pgfmathsetcounter{myi}{\themyi+9.9}
      \setcounter{myi}{\themyi}
      \draw [mydark,line width=.1pt] plot [smooth,tension=.5] 
      coordinates { (aa) (0-.1*\i,0+.01*\i) (bb) }; 
    }
    \draw [mydark,line width=.5pt,->-=.7] plot [smooth,tension=.5] 
    coordinates { (aa) (-1,.1) (bb) };
    \draw[mydark,line width=.5pt,->-=.4] plot [smooth,tension=.5] coordinates { (aa) (-.1,.01) (bb) };
    \node[circle,inner sep=.2mm,fill=mylight,fill opacity=.4,text opacity=1] at (-.7,.2) {$\scriptstyle p$};
    \node[font=\tiny] at (-.25,.4) {$f$};
    \node[font=\tiny] at (-.8,-.2) {$g$};
    %% Fiber space:
    \draw[dotted,line width=.6pt] (-1,3) to (bb);
    \draw[fill=mylight!50,line width=.8pt] plot [smooth cycle,tension=-.15] coordinates
    {(-2.2,2.5) (-2.2,4.5) (.2,4.5) (.2,2.5) };
    %% Points and paths in the fiber:
    \node[circle,fill=mydark,inner sep=0pt,outer sep=2pt,minimum
    size=.75mm] (ee) at (-.5,3.4) {};
    \node[circle,fill=mydark,inner sep=0pt,outer sep=2pt,minimum
    size=.75mm] (ff) at (-1.2,3.9) {};
    \node[circle,fill=mydark,inner sep=0pt,outer sep=2pt,minimum
    size=.75mm] (gg) at (-1.4,3.1) {};
    \draw[mydark,line width=.4pt,->-=.5,inner sep=0pt, outer sep=0pt] (-1.2,3.9) to (-1.4,3.1);
    \node[font=\tiny] at (-.38,3.5) {$\scriptstyle k$};
    \node[font=\tiny] at (-1.05,4) {$\scriptstyle f$};
    \node[font=\tiny] at (-1.5,3) {$\scriptstyle g$};
    \node[font=\tiny] at (-1.45,3.5) {$\scriptstyle p$};
    \node[font=\tiny] at (.8,4.3) {$\texttt{paths }a\;b$};
    \node[font=\footnotesize] at (-2.55,1.45) {$A$};
  \end{tikzpicture} \label{fig:paths}
  \caption{The path space fibration $\texttt{paths }a$ with the fiber
    over a point $b$.  Here $p$ is a (path) homotopy from $f$ to $g$.}
\end{figure}
In Coq, the path space \verb|paths| is defined as follows.
\begin{center}
\begin{coqcode}
Notation paths := identity.
\end{coqcode}
\end{center}
Here \verb|identity|, like \verb|nat|, is a built-in
inductive type in the Coq system.  We can see how it is defined
inductively using \verb|Print| to find
\begin{center}
  \begin{coqcode}
Inductive identity (A : Type) (a : A) : A -> Type :=
 identity_refl : identity a a.
  \end{coqcode}
\end{center}
That is, for each \verb|a : A|, \verb|identity a| is the
fibration freely generated by a term \verb|identity_refl a| in
the fiber over \verb|a|.  

We add the following line in order to introduce a
slightly shorter notation for the terms \verb|identity_refl|:
\begin{center}
  \begin{coqcode}
Notation idpath := identity_refl.
  \end{coqcode}
\end{center}
That is, for \verb|a : A|, \verb|idpath a : paths a a| is
the \emph{identity path} based at \verb|a|.

Recall that a \emph{path} in a space $A$ is a continuous function
$\varphi:I\to A$ where $I=[0,1]$ is the unit interval.  We say that
$\varphi$ \emph{is a path from a point $a$ of $A$ to a point $b$ of $A$} when
$\varphi(0)=a$ and $\varphi(1)=b$.  Then, the \emph{path space} $A^{I}$ is
the space of paths in $A$ and it comes equipped with two maps
$\partial_{0},\partial_{1}:A^{I}\to A$ given by
$\partial_{i}(\varphi):=\varphi(i)$ for $i=0,1$.  The induced map
$\langle\partial_{0},\partial_{1}\rangle: A^{I}\to A\times A$ is a
fibration which gives a factorization of the diagonal $\Delta:A\to
A\times A$ as 
\begin{center}
  \begin{tikzpicture}[auto]
    \node (UL) at (0,1.25) {$A$};
    \node (UR) at (2.5,1.25) {$A^{I}$};
    \node (B) at (1.25,0) {$A\times A$};
    \draw[->] (UL) to (UR);
    \draw[->,bend right=10pt] (UL) to node[mylabel,swap] {$\Delta$} (B);
    \draw[->,bend left=10pt] (UR) to (B);
  \end{tikzpicture}
\end{center}
where the first map $A\to A^{I}$ is a weak equivalence and the second is the
fibration mentioned above.  Here the first map $A\to A^{I}$ sends a
point $a$ to the constant loop based at $a$.  (That is, this first map
is precisely \verb|idpath|.)  One of the many important contributions of Quillen in
\cite{Quillen:1967uz} was to demonstrate that it is in fact possible
to do homotopy theory without the unit interval provided that one has
the structure of path spaces, weak equivalences, fibrations, and a few
other ingredients.  This is part of the reason that, even though type
theory does not (without adding higher-inductive types or something
similar) provide us with a unit interval, it is still possible to work
with homotopy theoretic structures type theoretically.

\subsection{Groupoid structure of the path space}

We will now describe the groupoid structure which the path space
endows on each type.  These constructions are well-known and their
connection with higher-dimensional groupoids was first noticed by
Hofmann and Streicher \cite{Hofmann:1998ty}.

First, given a path $f$ from $a$ to $b$ in $A$ we would like to be
able to reverse this path to obtain a path from $b$ to $a$.  For
topological spaces this is easy because a path $\varphi:I\to A$ gives
rise to an inverse path $\varphi'$ given by
$\varphi'(t):=\varphi(1-t)$, for $0\leq t\leq 1$.
\begin{center}
  \begin{coqcode}
Definition pathsinv { A : UU } { a b : A } ( f : paths a b ) 
: paths b a.
Proof.
  destruct f. apply idpath. 
Defined.
  \end{coqcode}
\end{center}
Here recall that \verb|destruct| allows us to argue by cases
about terms of inductive types.  Here \verb|f| is of type
\verb|paths a b|, which is inductive, and therefore this tactic
applies.  In this case, there is only one case to consider:
\verb|f| is really the identity path 
\verb|idpath a : paths a a|.  Because the inverse of the identity is the identity we then use
\verb|apply idpath| to complete the proof.  (Note that we could
also have used \verb|exact ( idpath a )| instead of
\verb|apply idpath| here to obtain the same term.)

Next, given a path $f$ as above together with another path $g$ from
$b$ to $c$, we would like to define the composite path from $a$ to $c$
obtained by first traveling along $f$ and then traveling along $g$.
This operation of \emph{path composition} is defined as follows:
\begin{center}
  \begin{coqcode}
Definition pathscomp { A : UU } { a b c : A } ( f : paths a b ) ( g : paths b c ) : paths a c.
Proof.
  destruct f. assumption.
Defined.
  \end{coqcode}
\end{center}
Once again, the proof begins with \verb|destruct f| which
effectively collapses \verb|f| to a constant loop.  In
particular, the result of this is to change the ambient hypotheses so
that \verb|g| is now of type \verb|paths a c| (see Figure
\ref{figure:pathscomp}).  At this stage, the goal matches the type of
\verb|g| and we use \verb|assumption| to let the Coq system
choose \verb|g| as the result of composing \verb|g| with the
identity path.
\begin{figure}[ht]
  \begin{tikzpicture}
    \node[smallcoqbox] (zero)  at (0,0) {%
      \begin{minipage}{4.25cm}
        \footnotesize
        \noindent\verb|A : UU|

        \noindent\verb|a : A|

        \noindent\verb|b : A|
        
        \noindent\verb|c : A|
        
        \noindent\verb|f : paths a b|
        
        \noindent\verb|g : paths b c|

        \noindent\verb|============================|

        \noindent\verb| paths a c|
      \end{minipage}
    };
    \node[anchor=north east, inner sep=2pt] (titlezero) at
    (zero.north east) {\emph{Start of proof}};
    \node[smallcoqbox] (one) at (5.25,0) {%
      \begin{minipage}{4.25cm}
        \footnotesize
        \vphantom{\texttt{b : A}}
        
        \vphantom{\texttt{f : paths a b}}

        \noindent\verb|A : UU|

        \noindent\verb|a : A|
        
        \noindent\verb|c : A|
        
        \noindent\verb|g : paths a c|

        \noindent\verb|============================|

        \noindent\verb| paths a c|
      \end{minipage}
    };
    \node[anchor=north east, inner sep=2pt] (titleone) at
    (one.north east) {\emph{after} \verb|destruct f.|};
  \end{tikzpicture}
  \caption{Coq output during the definition of path composition.}
  \label{figure:pathscomp}
\end{figure}
One immediate consequence of this definition is that the unit law
$f\circ 1_{a}=f$ for $f:a\to b$ holds \emph{on the nose} in the sense
that these two terms (\verb|pathscomp ( idpath a ) f| and 
\verb|f|) are identical in the strong $=$ sense.  On the other
hand, the unit law $1_{b}\circ f=f$ does not hold on the nose.
Instead, it only holds up to the existence of a higher-dimensional
path as described in the following Lemma:
\begin{center}
  \begin{coqcode}
Lemma isrunitalpathscomp { A : UU } { a b : A } ( f : paths a b ) : paths ( pathscomp f ( idpath b ) ) f.
Proof.
  destruct f. apply idpath. 
Defined.
  \end{coqcode}
\end{center}
The proof of this requires little comment (when $f$ becomes itself an
identity path, the composite becomes, by the left-unit law mentioned
above, an identity path).  The one thing to note here is that here
instead of \verb|Definition| we have written \verb|Lemma|.
Although there are some technical differences between these two ways
of defining terms they are for us entirely interchangeable and
therefore we use the appellation ``Lemma'' in keeping with the
traditional mathematical distinction between definitions and lemmas.

That facts that, up to the existence of higher-dimensional paths,
composition of paths is associative and that the inverses given
by \verb|pathsinv| are inverses for composition are recorded as
the terms \verb|isassocpathscomp|, \verb|islinvpathsinv| and
\verb|isrinvpathsinv|.  However, the descriptions of
these terms are omitted in light of the fact that they all
follow the same pattern as the proof of \verb|isrunitalpathscomp|.

\subsection{The functorial action of a continuous map on a path}

Classically, given a continuous map $f:A\to B$ and a path
$\varphi:I\to A$ in $A$, we obtain a corresponding path in $B$ by
composition of continuous functions.  Thinking of spaces as
$\infty$-groupoids, this operation of going from the path $\varphi$ in
$A$ to the path $f\circ\varphi$ in $B$ is the functorial action of $f$
on $1$-cells of the $\infty$-groupoid $A$.  In Coq, this action of
transporting a path in $A$ to a path in $B$ along a continuous map is
given as follows:
\begin{center}
  \begin{coqcode}
Definition maponpaths { A B : UU } ( f : A -> B ) { a a' : A } ( p : paths a a' ) : paths ( f a ) ( f a' ).
Proof.
  destruct p. apply idpath. 
Defined.
  \end{coqcode}
\end{center}
The proof again follows the familiar pattern: when the path $p$ is the
identity path on $a$, the result of applying $f$ should be the
identity path on $f(a)$.  We introduce the following notation for
\verb|maponpaths|:
\begin{center}
  \begin{coqcode}
Notation "f ` p" := ( maponpaths f p ) (at level 30 ).
  \end{coqcode}
\end{center}
This is an example of a general mechanism in Coq for defining
notations, but discussion of this mechanism is outside of the scope of
this article (the crucial point here is that the value 30 tells how
tightly the operation $`$ should bind).
\begin{figure}[ht]
  \centering
  \begin{tikzpicture}
    \draw[fill=mylight] plot [smooth cycle,tension=.75] coordinates
    {(-4,0) (-3.5,1.5) (-2,1) (-.5,1.5) (0,0) (-2,-1) };
    \draw[fill=mylight] plot [smooth cycle,tension=1] coordinates
    {(2.5,0) (4,1) (5.5,0) (4,-1) };
    \node[circle,fill=mydark,inner sep=0pt,outer sep=4pt,minimum
    size=.75mm] (aa) at (-3.5,.5) {};
    \node[circle,fill=mydark,inner sep=0pt,outer sep=4pt,minimum
    size=.75mm] (aaprime) at (-.5,.5) {};
    \draw[color=mydark,->-=.5,line width=.5pt] plot [smooth,tension=.75]
    coordinates { (aa) (-2,-.25) (aaprime) };
    \node at (-3.6,.65) {\footnotesize $a$};
    \node at (-.35,.7) {\footnotesize $a'$};
    \node at (-2.25,-.35) {\footnotesize $p$};
    \node[circle,fill=mydark,inner sep=0pt,outer sep=0pt,minimum
    size=.75mm] (faa) at (3.25,.25) {};
    \node[circle,fill=mydark,inner sep=0pt,outer sep=0pt,minimum
    size=.75mm] (faaprime) at (4.75,-.25) {};
   \draw[color=mydark,->-=.5,line width=.5pt] (faa) to
   node[mylabel,auto,swap] {$f`p$} (faaprime);
   \draw[->,line width=.75,bend left] (.25,.75) to node[auto,mylabel]
   {$f$} (2.25,.75);
   \node at (3,.4) {\footnotesize $f a$};
   \node at (5,-.05) {\footnotesize $f a'$};
   \node at (-4.25,1.35) {$A$};
   \node at (5.25,1.25) {$B$};
  \end{tikzpicture}
  \caption{Representation of \texttt{maponpaths}.}
\end{figure}
We leave it as an exercise for the reader to verify that the
operation \verb|maponpaths| respects identity paths, as well as
composition and inverses of paths.

\section{Transport}\label{sec:transport}

Given a fibration $\pi:E\to B$ together with a path $f$ from $b$ to $b'$
in the base $B$, there is a continuous function $f_{!}:E_{b}\to
E_{b'}$ from the fiber $E_{b}$ of $\pi$ over $b$ to the fiber $E_{b'}$
over $b'$.  This operation $f_{!}$ of \emph{forward transport} along a
path is described in Coq as follows:
\begin{center}
  \begin{coqcode}
Definition transportf { B : UU } ( E : B -> UU ) { b b' : B } 
( f : paths b b' ) : E b -> E b'.
Proof.
  intros e. destruct f. assumption.
Defined.
\end{coqcode}
\end{center}
\begin{figure}[ht]
  \centering
  \begin{tikzpicture}[color=mydark, fill=mylight, line width=1pt,scale=.75]
    %% THE MAIN SPACE
    \draw[fill=mylight] plot [smooth cycle,tension=.75] coordinates
    {(-2.5,0) (-2,1.5) (0,1.25) (1.75,1.5) (2.75,0) (1.75,-1) (-.5,0)};
    \draw[color=mydark,->-=.5,line width=.7pt] plot [smooth,tension=.5]
    coordinates { (-1.75,.5) (-1,.8)(1,0) (1.5,.25) };
    \node[circle,fill=mydark,inner sep=0pt,outer sep=4pt,minimum size=.75mm] (aa)
    at (-1.75,.5) {};
    \node[circle,fill=mydark,inner sep=0pt,outer sep=2pt,minimum size=.75mm] (bb)
    at (1.5,.25) {};
    \draw[dotted,line width=.6pt] (-1.75,3) to (aa);
    \draw[dotted,line width=.6pt] (1.5,3) to (bb);
    \draw[fill=mylight!50,line width=.8pt] plot [smooth cycle,tension=-.15] coordinates
    {(-2.5,2.5) (-2.5,4.5) (-1,4.5) (-1,2.5) };
    \draw[fill=mylight!50,line width=.8pt] plot [smooth cycle,tension=-.15] coordinates
    {(.75,2) (.75,4) (2.25,4) (2.25,2) };
    \node[circle,fill=mydark,inner sep=0pt,outer sep=2pt,minimum
    size=.75mm] (ee)
    at (-2,3.8) {};
    \node[circle,fill=mydark,inner sep=0pt,outer sep=2pt,minimum
    size=.75mm] (eef) at (1.5,3.5) {};
    \draw[->,dotted,line width=.6pt] (ee) to (eef);
    \node[font=\tiny] at (-2.15,3.9) {$\scriptstyle e$};
    \node[font=\tiny] at (1.75,3.7) {$\scriptstyle f_{!}(\!e)$};
    \node at (-1.95,.6) {$\scriptstyle b$};
    \node at (1.7,.4) {$\scriptstyle b'$};
    \node[font=\tiny] at (.25,.5) {$f$};
    \node[font=\tiny] at (-2.95,4.5) {$E(b)$};
    \node[font=\tiny] at (2.75,4) {$E(b')$};
    \node[font=\footnotesize] at (-2.55,1.45) {$B$};
  \end{tikzpicture} \label{fig:transportf}
  \caption{Forward transport.}
\end{figure}
For a path $f$ as above, there is a corresponding operation
$f^{*}:E_{b'}\to E_{b}$ of \emph{backward transport} and it turns out
that $f_{!}$ and $f^{*}$ constitute a homotopy equivalence.

We will turn to briefly discuss homotopy and homotopy equivalence in
the setting of Coq before returning to forward and backward transport.

\subsection{Homotopy and homotopy equivalence}

Recall that for continuous functions $f,g:A\to B$, \emph{a homotopy from $f$
to $g$} is given by a continuous map $h:A\to B^{I}$ such that
\begin{center}
  \begin{tikzpicture}[auto]
    \node (UL) at (0,1.25) {$A$};
    \node (UR) at (2.5,1.25) {$B^{I}$};
    \node (B) at (1.25,0) {$B\times B$};
    \draw[->] (UL) to node[mylabel] {$h$} (UR);
    \draw[->,bend right=10pt] (UL) to node[mylabel,swap] {$\langle f,g\rangle$} (B);
    \draw[->,bend left=10pt] (UR) to (B);
  \end{tikzpicture}
\end{center}
commutes.  

In Coq, the type of homotopies between functions $f,g: A\to B$ is given by
\begin{center}
  \begin{coqcode}
Definition homot { A B : UU } ( f g : A -> B ) := forall x :A, paths ( f x ) ( g x ).
  \end{coqcode}
\end{center}
Here we encounter a new ingredient in Coq: the universal quantifier
\verb|forall|.  From the homotopical point of view, this
operation takes a fibration \verb|E : B -> UU| and gives back the
space \verb|forall x : B, E x| of all continuous sections of the
fibration.  That is, we should think of a point $s$ of this type as
corresponding to a continuous section
\begin{align*}
  \begin{tikzpicture}[auto]
    \node (UL) at (0,1.25) {$B$};
    \node (UR) at (2.5,1.25) {$E$};
    \node (B) at (1.25,0) {$B.$};
    \draw[->] (UL) to node[mylabel] {$s$} (UR);
    \draw[->,bend right=10pt] (UL) to node[mylabel,swap] {$1_{B}$} (B);
    \draw[->,bend left=10pt] (UR) to (B);
  \end{tikzpicture}
\end{align*}
One particular consequence of this is that if we are given a term 
\begin{center}
  \begin{coqcode}
s : ( forall x : B, E x )
  \end{coqcode}
\end{center}
and another term \verb|b : B|, then the term \verb|s| can be
\emph{applied} to the term \verb|b : B| to obtain a term of type 
\verb|E b|.  The result of applying \verb|s| to 
\verb|b| is denoted by
\begin{center}
  \begin{coqcode}
s b : E b.
  \end{coqcode}
\end{center}
We have more below to say about \verb|forall|.

Now, a map $f:A\to B$ is a \emph{homotopy equivalence} when there exists a
map $f':B\to A$ together with homotopies from $f'\circ f$ to $1_{A}$
and from $f\circ f'$ to $1_{B}$.  In this case, we say that $f'$ is a
\emph{homotopy inverse} of $f$.  Two spaces $A$ and $B$ are said to
have the same \emph{homotopy type} when there exists a homotopy
equivalence $f:A\to B$.

In Coq, we define the type of proofs that a map \verb|f : A -> B|
is a homotopy equivalence as follows:
\begin{center}
  \begin{coqcode}
Definition isheq { A B : UU } ( f : A -> B ) := total (fun f' : B -> A => dirprod (homot (funcomp f' f) (idfun _)) (homot (funcomp f f') (idfun _)) ).
  \end{coqcode}
\end{center}
Here it is worth pausing for a moment to consider the meaning of the
type \verb|isheq|.  Intuitively, \verb|isheq f| is the type
consisting of the data which one must provide in order to prove that
\verb|f| is a homotopy equivalence.  That is, a term of type
\verb|isheq f| consists of:
\begin{itemize}
\item a continuous map \verb|f' : B -> A|;
\item a homotopy from \verb|funcomp f' f| to the identity on \verb|B|;
\item a homotopy from \verb|funcomp f f'| to the identity on
  \verb|A|.
\end{itemize}
Indeed, by the definitions of \verb|total| and
\verb|dirprod| the terms of \verb|isheq f| can be regarded
as a tuple of such data.

\subsection{Forward and backward transport}

It turns out that, as mentioned above, the backward transport map
$f^{*}:E_{b'}\to E_{b}$ is a homotopy inverse of forward transport
$f_{!}$.  Denote by \verb|transportb| the backward transport
term.  It is often convenient to break the proofs of larger facts up
into smaller lemmas and we will do just this in order to show that
\verb|transportf E f| is a homotopy equivalence.  In particular,
we begin by proving that $f_{!}\circ f^{*}$ is homotopic to the
identity $1_{E_{b'}}$:
\begin{center}
  \begin{coqcode}
Lemma backandforth { B : UU } { E : B -> UU } { b b' : B } ( f : paths b b' ) ( e : E b' ) : homot ( funcomp ( transportb E f ) ( transportf E f ) ) ( idfun _ ).
Proof.
  intros x. destruct f. apply idpath. 
Defined.
  \end{coqcode}
\end{center}
Next, we prove that $f^{*}\circ f_{!}$ is homotopic to the identity
$1_{E_{b}}$ as \verb|forthandback| (we omit the proof because it
is identical to the proof of \verb|backandforth|):
\begin{center}
  \begin{coqcode}
Lemma forthandback { B : UU } { E : B -> UU } { b b' : B } ( f : paths b b' ) ( e : E b' ) : homot ( funcomp ( transportf E f ) ( transportb E f ) ) ( idfun _ ).
  \end{coqcode}
\end{center}
Using these lemmas we can finally prove that 
\verb|transportf E f| is a homotopy equivalence.
\begin{center}
  \begin{coqcode}
Lemma isheqtransportf { B : UU } ( E : B -> UU ) { b b' : B } ( f : paths b b' ) : isheq ( transportf E f ).
Proof.
  split with ( transportb E f ). split.
  apply backandforth. apply forthandback.
Defined.
  \end{coqcode}
\end{center}
\begin{figure}[ht]
  \begin{tikzpicture}
    \node[smallcoqbox] (zero)  at (0,0) {%
      \begin{minipage}{5.2cm}
        \footnotesize
        \noindent\verb|1 subgoals, subgoal 1|

        ~

        \noindent\verb|B : UU|

        \noindent\verb|E : B -> UU|

        \noindent\verb|b : B|

        \noindent\verb|b' : B|
        
        \noindent\verb|f : paths b b'|

        \noindent\verb|============================|

        \noindent\verb| isheq (transportf E f)|

        \vphantom{\texttt{(transportf E f )) ( idfun ( E b' ) )}}
      \end{minipage}
    };
    \node[anchor=north east, inner sep=2pt] (titlezero) at
    (zero.north east) {\footnotesize\emph{Start of proof}};
    \node[smallcoqbox] (one) at (6,0) {%
      \begin{minipage}{5.2cm}
        \footnotesize
        \noindent\verb|2 subgoals, subgoal 1|
        
        ~
        
        \noindent\verb|B : UU|
        
        \noindent\verb|E : B -> UU|

        \noindent\verb|b : B|
        
        \noindent\verb|b' : B|
        
        \noindent\verb|f : paths b b'|
        
        \noindent\verb|============================|
        
        \noindent\verb| homot (funcomp (transportb E f)|
        
        \noindent\verb|  (transportf E f)) (idfun (E b'))|
      \end{minipage}
    };
    \node[anchor=north east, inner sep=2pt] (titleone) at
    (one.north east) {\footnotesize\emph{after} 
      \verb|split with; split|.};
    \node[smallcoqbox] (two)  at (0,-5.15) {%
      \begin{minipage}{5.2cm}
        \footnotesize
        \noindent\verb|1 subgoals, subgoal 1|

        ~

        \noindent\verb|B : UU|

        \noindent\verb|E : B -> UU|

        \noindent\verb|b : B|

        \noindent\verb|b' : B|
        
        \noindent\verb|f : paths b b'|

        \noindent\verb|============================|

        \noindent\verb| homot (funcomp (transportf E f)|
        
        \noindent\verb|  (transportb E f)) (idfun (E b))|
      \end{minipage}
    };
    \node[anchor=north east, inner sep=2pt] (titletwo) at
    (two.north east) {\footnotesize\emph{after} \verb|apply backandforth|};
  \end{tikzpicture}
  \caption{Coq output during the proof that forward transport is a
    homotopy equivalence.}
  \label{figure:isheqtransportf}
\end{figure}
There are several points to make about this proof.  The initial goal is
to supply a term of type \verb|isheq ( transportf E f )|.  Now,
this type is itself really of the form (you can see this in the proof
by entering \verb|unfold isheq|):
\begin{center}
  \begin{coqcode}
total (fun f' : E b' -> E b => dirprod (homot (funcomp f' (transportf
E f)) (idfun (E b'))) (homot (funcomp (transportf E f) f') (idfun (E b))))
  \end{coqcode}
\end{center}
and in general to construct a term of type \verb|total E|, for 
\verb|E : B -> UU|, it suffices (by virtue of the definition of 
\verb|total|) to give a term \verb|b| of type
\verb|B| together with a term of type \verb|E b|.  This is
captured in Coq by the command \verb|split with| and one should
think of \verb|split with b| as saying to Coq that you will
construct the required term using \verb|b| as the term of type
\verb|B| you are after.  Upon using this command, the goal will
automatically be updated to \verb|E b|.  In this case, 
entering \verb|split with ( transportb E f)| is the way to tell
Coq that we take \verb|transportb E f| to be the homotopy inverse
of \verb|transportf E f|.  So, after entering this
command the new goal becomes
\begin{center}
  \begin{coqcode}
dirprod 
 (homot (funcomp (transportb E f) (transportf E f)) (idfun (E b')))
 (homot (funcomp (transportf E f) (transportb E f)) (idfun (E b)))
  \end{coqcode}
\end{center}
As with \verb|total E|, in order to construct a term of type
\verb|dirprod A B| it suffices to supply terms of both types
\verb|A| and \verb|B|.  When given a goal of
the form \verb|dirprod A B|, we use the \verb|split| tactic
to tell Coq that we will supply separately the
terms of type \verb|A| and \verb|B| individually (as opposed
to providing a term by some other means).  (See Figure \ref{figure:isheqtransportf}
for the result of applying both \verb|split with| and
\verb|split| in the particular proof we are considering.)

The final new ingredient from the proof of \verb|isheqtransportf|
is the appearance of the tactic \verb|apply|.  When you have
proved a result in Coq and you are later
given a goal which is a (more or less direct) consequence of that
the result, then the tactic \verb|apply| will allow
you to apply the result.  In this case, the lemmas
\verb|backandforth| and \verb|forthandback| are exactly the
lemmas required in order to prove the remaining subgoals.

\subsection{Paths in the total space}

Using transport it is possible to give a complete characterization of
paths in the total space of a fibration \verb|E : B -> UU|.
Along these lines, the following lemma gives sufficient conditions for
the existence of a path in the total space:
\begin{center}
  \begin{coqcode}
Lemma pathintotalfiber { B : UU } { E : B -> UU } { x y : total E } ( f : paths ( pr1 x ) ( pr1 y ) ) ( g : paths ( transportf E f ( pr2 x ) ) ( pr2 y ) ) : paths x y.
Proof.
  intros. destruct x as [ x0 x1 ]. destruct y as [ y0 y1 ].  
  simpl in *. destruct f. destruct g. apply idpath.
Defined.
  \end{coqcode}
\end{center}
This lemma shows that, given points \verb|x| and \verb|y| of
the total space, in order to construct a path from \verb|x| to
\verb|y| it suffices to provide the following data:
\begin{itemize}
\item a path \verb|f| from \verb|pr1 x| to \verb|pr1 y|; and
\item a path \verb|g| from the result of transporting
  \verb|pr2 x| along \verb|f| to \verb|pr2 y|. 
\end{itemize}
This is illustrated in Figure \ref{fig:pathintotalfiber} in the
special case where \verb|x| is the pair \verb|pair b e| and
\verb|y| is the pair \verb|pair b' e'|.

Regarding the proof of \verb|pathintotalfiber|, it is worth
mentioning that here the effect of applying 
\verb|destruct x as [ x0 x1 ]| is that it tells Coq that we would like to consider the case where
\verb|x| is really of the form \verb|pair x0 x1|.  The only new
tactic here is \verb|simpl in *| which tells Coq to make any possible
simplifications to the terms appearing in the goal or hypotheses.
For example, in this case, Coq will simplify 
\verb|(pr1 (pair x0 x1))| to \verb|x0|.
\begin{figure}[ht]
  \centering
  \begin{tikzpicture}[color=mydark, fill=mylight, line width=1pt,scale=.75]
    %% THE MAIN SPACE
    \draw[fill=mylight] plot [smooth cycle,tension=.75] coordinates
    {(-2.5,0) (-2,1.5) (0,1.25) (1.75,1.5) (2.75,0) (1.75,-1) (-.5,0)};
    \draw[->-=.5,line width=.7pt] plot [smooth,tension=.5]
    coordinates { (-1.75,.5) (-1,.8)(1,0) (1.5,.25) };
    \node[circle,fill=mydark,inner sep=0pt,outer sep=4pt,minimum size=.75mm] (aa)
    at (-1.75,.5) {};
    \node[circle,fill=mydark,inner sep=0pt,outer sep=2pt,minimum size=.75mm] (bb)
    at (1.5,.25) {};
    \draw[dotted,line width=.6pt] (-1.75,3) to (aa);
    \draw[dotted,line width=.6pt] (1.5,3) to (bb);
    \draw[fill=mylight!50,line width=.8pt] plot [smooth cycle,tension=-.15] coordinates
    {(-2.5,2.5) (-2.5,4.5) (-1,4.5) (-1,2.5) };
    \draw[fill=mylight!50,line width=.8pt] plot [smooth cycle,tension=-.15] coordinates
    {(.75,2) (.75,4) (2.25,4) (2.25,2) };
    \draw[->-=.5,line width=.6pt] plot
      [smooth,tension=1] coordinates { (1.5,3.5) (1.3,3.25) (1.4,2.75) (1.25,2.5) };
    \node[circle,fill=mydark,inner sep=0pt,outer sep=2pt,minimum
    size=.75mm] (ee)
    at (-2,3.8) {};
    \node[circle,fill=mydark,inner sep=0pt,outer sep=2pt,minimum
    size=.75mm] (eef) at (1.5,3.5) {};
    \node[circle,fill=mydark,inner sep=0pt,outer sep=2pt,minimum
      size=.75mm] (eeprime) at (1.25,2.5) {};
    \draw[->,dotted,line width=.6pt] (ee) to (eef);
    \node[font=\tiny] at (-2.15,3.9) {$\scriptstyle e$};
    \node[font=\tiny] at (1.75,3.7) {$\scriptstyle f_{!}(\!e)$};
    \node[font=\tiny] at (1.55,2.8) {$\scriptstyle g$};
    \node[font=\tiny] at (1.1,2.55) {$\scriptstyle e'$};
    \node at (-1.95,.6) {$\scriptstyle b$};
    \node at (1.7,.4) {$\scriptstyle b'$};
    \node[font=\tiny] at (.25,.5) {$f$};
    \node[font=\tiny] at (-2.95,4.5) {$E(b)$};
    \node[font=\tiny] at (2.75,4) {$E(b')$};
    \node[font=\footnotesize] at (-2.55,1.45) {$B$};
  \end{tikzpicture}
  \caption{Paths in the total space.}
  \label{fig:pathintotalfiber}
\end{figure}

On the other hand, if we are given a path \verb|f| from
\verb|x| to \verb|y| in the total space, there is an induced
path in the base given by 
\begin{center}
  \begin{coqcode}
Definition pathintotalfiberpr1 { B : UU } { E : B -> UU } { x y : total E } ( f : paths x y ) : paths ( pr1 x ) ( pr1 y ) := pr1 ` f.
  \end{coqcode}
\end{center}
Furthermore, we may transport \verb|pr2 x| along
\verb|pathintotalfiberpr1 f| and there is a path from the
resulting term to \verb|pr2 y|:
\begin{center}
  \begin{coqcode}
Definition pathintotalfiberpr2 { B : UU } { E : B -> UU } { x y : total E } ( f : paths x y ) : paths (transportf E ( pathintotalfiberpr1 f ) ( pr2 x )) ( pr2 y ).
Proof.
  intros. destruct f. apply idpath.
Defined.
  \end{coqcode}
\end{center}
Finally, we prove that every path in the total space is homotopic to
one obtained using \verb|pathintotalfiber|:
\begin{center}
  \begin{coqcode}
Lemma pathintotalfibercharacterization { B : UU } { E : B -> UU } { x
  y : total E } ( f : paths x y ) : paths f  (pathintotalfiber (pathintotalfiberpr1 f) (pathintotalfiberpr2 f) ).
Proof.
  intros. destruct f. destruct x as [ x0 x1 ]. apply idpath.
Defined.
  \end{coqcode}
\end{center}
