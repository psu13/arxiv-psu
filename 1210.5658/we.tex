\section{Weak equivalences and homotopy equivalences} \label{sec:bh}

In this section we introduce further basic homotopy theoretic notions
including the crucial notion of \emph{weak equivalence}.  Classically,
weak equivalences are those maps which induce isomorphisms on (all
higher) homotopy groups.  Between sufficiently nice spaces weak
equivalences turn out to coincide with homotopy equivalences and the
main goal of this section is to prove this fact in Coq.  The Coq
proofs in this section become increasingly sophisticated and
accordingly we will begin to pass quickly over the more basic features
of the proofs while at the same time giving their mathematical
interpretations.

\subsection{Contractibility}\label{sec:contr}

A space $A$ is said to be \emph{contractible} when the canonical map
$A\to 1$ to the one point space is a homotopy equivalence.  In Coq,
the notion of contractibility is captured by the following definition:
\begin{center}
  \begin{coqcode}
Definition iscontr ( A : UU ) := total ( fun center : A => forall b : A, paths center b ).
  \end{coqcode}
\end{center}
That is, a term of type \verb|iscontr A| consists of the data
required to prove that \verb|A| is contractible:
\begin{itemize}
\item a point \verb|center| of \verb|A|, which we will
  sometimes refer to as the \emph{center of contraction}; and
\item a continuous assignment of, to each \verb|b : A|, a path
  from \verb|center| to \verb|b|.
\end{itemize}
There are various facts about contractible space (e.g., $1$ is
contractible, contractible spaces satisfy the principle of ``proof
irrelevance'', and so forth) which are ready consequences of this
definition.  However, we will leave the investigation of such matters
to the reader and merely include one important fact about contractible
spaces: homotopy retracts of contractible spaces are contractible.  That is,
given continuous functions $r : A \to B$ and $s: B\to A$ together with
a homotopy $p$ as indicated in the following diagram:
\begin{align}\label{eq:hretract}
  \begin{minipage}{.25\linewidth}
    \begin{tikzpicture}[auto]
    \node (UL) at (0,1.5) {$B$};
    \node (UR) at (3,1.5) {$A$};
    \node (LC) at (1.5,0) {$B$};
    \draw[->] (UL) to node[mylabel] {$s$} (UR);
    \draw[->,bend left] (UR) to node[mylabel] {$r$} (LC);
    \draw[->,bend right] (UL) to node[mylabel,swap] {$1_{B}$} (LC);
    \node (ppp) at (1.5,.75) {$\Downarrow$}; 
    \node (pp) at (1.75,.75) {\footnotesize $p$};
  \end{tikzpicture}
  \end{minipage}
\end{align}
if the space $A$ is contractible, then so is the space $B$.  Note that
in the diagram (\ref{eq:hretract}) we employ the convention, familiar
from higher-dimensional category theory, of indicating the homotopy $p$
by a double arrow $\Rightarrow$.  In Coq:
\begin{center}
  \begin{coqcode}
Lemma iscontrretract { A B : UU } { r : A -> B } { s : B -> A } (p : homot ( funcomp s r ) (idfun _)) (is : iscontr A) : iscontr B.
Proof.
  split with ( r ( pr1 is ) ).  intros b. 
  change b with ( idfun B b ). rewrite <- ( p b ). unfold funcomp. 
  apply maponpaths. apply ( pr2 is ).
Defined.
  \end{coqcode}
\end{center}
The output of Coq during the proof can be found in Figure
\ref{fig:iscontrretract}.  First, we need to provide a center of contraction for \verb|B|.
The center of contraction for \verb|A| is the term 
\verb|pr1 is|. By entering 
\verb|split with ( r (pr1 is ) )| we tell Coq to take
the center of contraction of \verb|B| to be the term
\verb|r ( pr1 is )|.  After this, the goal becomes
\begin{center}
  \begin{coqcode}
forall b : B, paths (r (pr1 is)) b
  \end{coqcode}
\end{center}
which is to say that we must prove that there is a path from
\verb|r (pr1 is)| to any other point of \verb|B|.  Next,
after using \verb|intros b|, we enter the command 
\begin{center}
  \begin{coqcode}
change b with ( idfun B b )
  \end{coqcode}
\end{center}
to replace the term \verb|b| in the goal with the \emph{equal}
term
\verb|idfun B b|.  Observe that the term \verb|p b| has type
\begin{center}
  \begin{coqcode}
paths (funcomp s r b) (idfun B b).
  \end{coqcode}
\end{center}
The tactic \verb|rewrite| allows us to replace each occurrence of
a term in the goal by a term which is in the same path component.
In this case, because \verb|p b| is a path from
\verb|funcomp s r b| to \verb|idfun B b|, the result of
rewriting with \verb|rewrite <- (p b)| is to replace each
occurrence of \verb|idfun B b| in the goal by 
\verb|funcomp s r b|.  Here the backward arrow \verb|<-|
indicates that we are rewriting the path \verb|p b| from right to
left.
\begin{figure}[H]
  \begin{tikzpicture}
    \node[smallcoqbox] (zero)  at (0,0) {%
      \begin{minipage}{5.25cm}
        \footnotesize

        \noindent\verb|A : UU|

        \noindent\verb|B : UU|
        
        \noindent\verb|r : A -> B|

        \noindent\verb|s : B -> A|

        \noindent\verb|p : homot (funcomp s r) (idfun B)|
        
        \noindent\verb|is : iscontr A|

        \noindent\verb|b : B|

        \noindent\verb|============================|
        
        \noindent\verb| paths (r (pr1 is)) b|
      \end{minipage}
    };
    \node[anchor=north east, inner sep=2pt] (titlezero) at
    (zero.north east) {\emph{after \texttt{split with} and
        \texttt{intros}}};
    \node[smallcoqbox] (one) at (6.25,0) {%
      \begin{minipage}{5.25cm}
        \footnotesize

        
        \noindent\verb|A : UU|

        \noindent\verb|B : UU|
        
        \noindent\verb|r : A -> B|

        \noindent\verb|s : B -> A|

        \noindent\verb|p : homot (funcomp s r) (idfun B)|
        
        \noindent\verb|is : iscontr A|

        \noindent\verb|b : B|

        \noindent\verb|============================|
        
        \noindent\verb| paths (r (pr1 is)) (idfun B b)|
      \end{minipage}
    };
    \node[anchor=north east, inner sep=2pt] (titleone) at
    (one.north east) {\emph{after} \verb|change b with |\ldots};
    \node[smallcoqbox] (two) at (0,-4.65) {%
      \begin{minipage}{5.25cm}
        \footnotesize
        
        \noindent\verb|A : UU|

        \noindent\verb|B : UU|
        
        \noindent\verb|r : A -> B|

        \noindent\verb|s : B -> A|

        \noindent\verb|p : homot (funcomp s r) (idfun B)|
        
        \noindent\verb|is : iscontr A|

        \noindent\verb|b : B|

        \noindent\verb|============================|
        
        \noindent\verb| paths (r (pr1 is)) (funcomp s r b)|
      \end{minipage}
    };
    \node[anchor=north east, inner sep=2pt] (titletwo) at
    (two.north east) {\emph{after} \verb|rewrite <- (p b)|};
    \node[smallcoqbox] (three) at (6.25,-4.65) {%
      \begin{minipage}{5.25cm}
        \footnotesize
        
        \noindent\verb|A : UU|

        \noindent\verb|B : UU|
        
        \noindent\verb|r : A -> B|

        \noindent\verb|s : B -> A|

        \noindent\verb|p : homot (funcomp s r) (idfun B)|
        
        \noindent\verb|is : iscontr A|

        \noindent\verb|b : B|

        \noindent\verb|============================|
        
        \noindent\verb| paths (r (pr1 is)) (r (s b))|
      \end{minipage}
    };
    \node[anchor=north east, inner sep=2pt] (titlethree) at
    (three.north east) {\emph{after} \verb|unfold funcomp |};
    \node[smallcoqbox] (four) at (0,-9.3) {%
      \begin{minipage}{5.25cm}
        \footnotesize
        
        \noindent\verb|A : UU|

        \noindent\verb|B : UU|
        
        \noindent\verb|r : A -> B|

        \noindent\verb|s : B -> A|

        \noindent\verb|p : homot (funcomp s r) (idfun B)|
        
        \noindent\verb|is : iscontr A|

        \noindent\verb|b : B|

        \noindent\verb|============================|
        
        \noindent\verb| paths (pr1 is) (s b)|
      \end{minipage}
    };
    \node[anchor=north east, inner sep=2pt] (titlefour) at
    (four.north east) {\emph{after} \verb|apply maponpaths|};
  \end{tikzpicture}
  \label{fig:iscontrretract}
  \caption{Coq output during the proof of \texttt{iscontrretract}.}
\end{figure}

Next, we use \verb|unfold funcomp| to replace each occurrence of 
\verb|funcomp| in the goal by its definition.  Then
\verb|apply maponpaths| tells Coq that we will use 
\verb|maponpaths| to construct the goal.  Because Coq recognizes
that the hypotheses of \verb|maponpaths| require us to provide a
path the effect of this is to remove both applications of the function
\verb|r| on both the right and left.  Finally, the required path
can be constructed using \verb|pr2 is|.

It is an immediate consequence of \verb|iscontrretract| that
spaces which are homotopy equivalent to contractible spaces are also
contractible.  This fact is captured by the definitions
\begin{center}
  \begin{coqcode}
Definition iscontrandheq { A B : UU } { f : A -> B } ( p : isheq f ) (is : iscontr A) : iscontr B := iscontrretract (pr1 (pr2 p)) is.
  \end{coqcode}
\end{center}
and
\begin{center}
  \begin{coqcode}
Definition iscontrandheqinv { A B : UU } { f : A -> B } ( p : isheq f ) ( is : iscontr B ) : iscontr A := iscontrretract ( pr2 ( pr2 p ) ) is. 
  \end{coqcode}
\end{center}

\subsection{Homotopy fibers}

Given a map \verb|f : A -> B| and a point \verb|b| of 
\verb|B|, we define the \emph{homotopy fiber of \texttt{f}
  over \texttt{b}} as follows:
\begin{center}
  \begin{coqcode}
Definition hfiber { A B : UU } ( f : A -> B ) ( b : B ) : UU := 
  total ( fun x : A => paths ( f x ) b ).
  \end{coqcode}
\end{center}
The homotopy fiber of $f$ over $b$ is the homotopical analogue of the
ordinary fiber $f^{-1}(b)$.  So, a typical point of the homotopy fiber
is a pair consisting of a point $a$ of $A$ together with a path 
\begin{align*}
  \begin{tikzpicture}
    \node (A) at (0,0) {$f(a)$};
    \node (B) at (2,0) {$b$};
    \draw[->-=.5] (A) to (B);
  \end{tikzpicture}
\end{align*}
in $B$.

There are various reasons for considering homotopy fibers.  Homotopy
fibers play a role for fibrations analogous to that played by ordinary
fibers for fiber bundles.  For us, the interest in homotopy fibers
comes from their presence in the definition of weak equivalences given below.

From the point of view of category theory, the homotopy fiber of $f$
over $b$ is the $\infty$-groupoid version of the comma category
$(f\downarrow b)$.  As in the categorical setting, there are actions
of the operation of taking the homotopy fiber over a fixed point on
maps over the base $B$.  That is, given
continuous maps $f:A\to B$, $g:C\to B$ and $h:A\to C$ together with a
homotopy $p$ from $g\circ h$ to $f$, there is an induced map 
\begin{align*}
  \texttt{hfiber}\;f\;b \to \texttt{hfiber}\;g\;b.
\end{align*}
In Coq, this map is defined as 
\begin{center}
  \begin{coqcode}
Definition hfiberact { A B C : UU } { f : A -> B } { g : C -> B } ( h : A -> C ) ( p : homot ( funcomp h g ) f ) ( b : B ) : hfiber f b -> hfiber g b 
:= fun a => pair (h ( pr1 a )) (pathscomp (p ( pr1 a )) ( pr2 a )).
  \end{coqcode}
\end{center}
That is, \verb|hfiberact h p b| sends an element
\begin{align*}
  \begin{tikzpicture}[auto]
    \node (A) at (0,0) {$f(a)$};
    \node (B) at (2,0) {$b$};
    \draw[->-=.5] (A) to node[mylabel] {$i$} (B);
  \end{tikzpicture}
\end{align*}
of $\texttt{hfiber}\;f\;b$ to the point of $\texttt{hfiber}\;g\;b$
given by the composite path
\begin{align*}
  \begin{tikzpicture}[auto]
    \node (A) at (0,0) {$g\bigl(h(a)\bigr)$};
    \node (B) at (2,0) {$f(a)$};
    \node (C) at (4,0) {$b$.};
    \draw[->-=.5] (A) to node[mylabel] {$p(a)$} (B);
    \draw[->-=.5] (B) to node[mylabel] {$i$} (C);
  \end{tikzpicture}
\end{align*}
As a special case of this construction, when we are given a continuous
maps $f:A\to B$ and $g:B\to C$ together with a point $c$ of $C$, there
is an induced map 
\begin{align*}
\texttt{hfiber}\;(g\circ f)\;c\to\texttt{hfiber}\;g\;c
\end{align*}
obtained by applying 
\verb|hfiberact| with the identity homotopy $1_{g\circ f}$.
In Coq, this is obtained as the following definition:
\begin{center}
  \begin{coqcode}
Definition maponhfiber { A B C : UU } ( f : A -> B ) ( g : B -> C ) ( c : C ) : hfiber ( funcomp f g ) c -> hfiber g c := fun a => pair ( f ( pr1 a ) ) ( pr2 a ).
  \end{coqcode}
\end{center}
In the case where we have a homotopy retract as in (\ref{eq:hretract})
above, we obtain, a further map as follows:
\begin{center}
  \begin{coqcode}
Definition secmaponhfiber { A B : UU } {r : A -> B} { s : B -> A } ( p : homot ( funcomp s r ) ( idfun _ ) ) ( a : A ) : hfiber s a -> hfiber ( funcomp r s ) a := fun b => pair (s ( pr1 b)) (pathscomp (s ` ( p ( pr1 b ) )) ( pr2 b )).
  \end{coqcode}
\end{center}
That is, \verb|secmaponhfiber| sends
\begin{align*}
  \begin{tikzpicture}[auto]
    \node (A) at (0,0) {$s(b)$};
    \node (B) at (2,0) {$a$};
    \draw[->-=.5] (A) to node[mylabel] {$i$} (B);
  \end{tikzpicture}
\end{align*}
in $\texttt{hfiber}\;s\;a$ to the term in $\texttt{hfiber}\;(s\circ
r)\;a$ given by the composite path
\begin{align*}
  \begin{tikzpicture}[auto]
    \node (A) at (0,0) {$s\bigl(r(s(b))\bigr)$};
    \node (B) at (3,0) {$s(b)$};
    \node (C) at (5,0) {$a$.};
    \draw[->-=.5] (A) to node[mylabel] {$s`p(b)$} (B);
    \draw[->-=.5] (B) to node[mylabel] {$i$} (C);
  \end{tikzpicture}
\end{align*}
It turns out that, in this situation, $\texttt{hfiber}\;s\;a$ is a
homotopy retract of $\texttt{hfiber}\;(s\circ r)\;a$:
\begin{align*}
    \begin{tikzpicture}[auto]
    \node (UL) at (0,1.5) {$\texttt{hfiber}\;s\;a$};
    \node (UR) at (4.5,1.5) {$\texttt{hfiber}\;(s\circ r)\;a$};
    \node (LC) at (1.5,0) {$\texttt{hfiber}\;s\;a$};
    \draw[->] (UL) to node[mylabel] {$\texttt{secmaponhfiber}\;p\;a$} (UR);
    \draw[->,bend left] (UR) to node[mylabel] {$\texttt{maponhfiber}\;r\;s\;a$} (LC);
    \draw[->,bend right] (UL) to node[mylabel,swap] {$1_{\texttt{hfiber}\;s\;a}$} (LC);
    \node (ppp) at (1.5,.75) {$\Downarrow$};
  \end{tikzpicture}
\end{align*}
Here, as in (\ref{eq:hretract}), we indicate the existence of a
homotopy in this diagram by a double arrow $\Rightarrow$.  In Coq,
this fact is captured by the following lemma:
\begin{center}
  \begin{coqcode}
Lemma secmaponhfiberissec {A B : UU} { r : A -> B } { s : B -> A } ( p : homot ( funcomp s r ) ( idfun _ ) ) ( a : A ) : homot (funcomp  ( secmaponhfiber p a ) ( maponhfiber r s a )) (idfun _).
Proof.
 intros b. destruct b as [ b i ]. unfold funcomp, idfun in *. simpl. 
 apply (@pathintotalfiber _ _ (pair (r (s b)) _) (pair b i) (p b)).
 rewrite transportfandhfiber. unfold secmaponhfiber. simpl.
 rewrite <- isassocpathscomp. rewrite islinvpathsinv. apply idpath. 
Defined.
  \end{coqcode}
\end{center}
First, after the first application of the \verb|simpl| tactic we
have simplified the goal and hypotheses to the extent that we are now
in the following situation:
\begin{center}
  \begin{coqcode}
  A : UU
  B : UU
  r : A -> B
  s : B -> A
  p : homot (fun x : B => r (s x)) (fun x : B => x)
  a : A
  b : B
  i : paths (s b) a
  ============================
   paths (maponhfiber r s a (secmaponhfiber p a (pair b i))) 
     (pair b i)
  \end{coqcode}
\end{center}
Because we are asked to construct a path in a total space we will make
use of \verb|pathintotalfiber|.  However, in order to help Coq
understand exactly what we are trying to do we must provide some of
the implicit arguments of \verb|pathintotalfiber| explicitly.
This is accomplished here using the symbol \verb|@|.  The
underscores are the arguments which we leave for Coq to guess and the
other arguments are those we are explicitly providing for Coq.  That
is, we are telling Coq that in order to achieve the current goal it
suffices to construct using \verb|pathintotalfiber| a path from
\verb|pair ( r ( s b ) ) _ | to \verb|pair b i| where the
path from \verb|r ( s b )| to \verb|b| we use is 
\verb|p b|.  After this, the goal becomes:
\begin{center}
  \begin{coqcode}
paths
 (transportf (fun x : B => paths (s x) a) (p b)
   (pr2 (pair (r (s b)) (pr2 (secmaponhfiber p a (pair b i))))))
 (pr2 (pair b i))
  \end{coqcode}
\end{center}
Here, as is often the case, it is convenient to know that the result of applying forward
transport can be decomposed in a specific way.  In this case, it turns
out that, for \verb|j| a path from \verb|v| to \verb|w|
in \verb|B| and a path \verb|k : paths ( s v ) a|,
\begin{center}
  \begin{coqcode}
transportf ( fun x : B => paths ( s x ) a ) j k
  \end{coqcode}
\end{center}
is homotopic (in the sense of path homotopy) to the following
composite path:
\begin{align*}
  \begin{tikzpicture}[auto]
    \node (UL) at (0,0) {$s w$};
    \node (UR) at (2,0) {$s v$};
    \node (RR) at (4,0) {$a$.};
    \draw[->-=.5] (UL) to node[mylabel] {$(s ` j )^{-1}$} (UR);
    \draw[->-=.5] (UR) to node[mylabel] {$k$} (RR);
  \end{tikzpicture}
\end{align*}
This fact is captured by the lemma \verb|transportfandhfiber|
which has a trivial proof that we omit.  Returning to the proof of \verb|secmaponhfiberissec|, we rewrite
using \verb|transportfandhfiber| and then simplify using 
\verb|unfold| and \verb|simpl| to arrive at the goal
\begin{center}
  \begin{coqcode}
paths (pathscomp (pathsinv (s ` p b)) (pathscomp (s ` p b) i)) i
  \end{coqcode}
\end{center}
which is an immediate consequence of associativity of path composition
and the fact that \verb|pathsinv| is an inverse with respect to
path composition.

In addition to acting on maps via \verb|hfiberact|, there is also
an action of the homotopy fiber operation on homotopies.  Namely, a
homotopy from a map $f:A\to B$ to a map $g:A\to B$ induces, for $b :
B$, a map $\texttt{hfiber}\;f\;b\to\texttt{hfiber}\;g\;b$ as follows:
\begin{center}
\begin{coqcode}
Definition hfiberandhomot { A B : UU } { f g : A -> B } ( b : B ) ( p : homot f g ) : hfiber f b -> hfiber g b := fun a => pair (pr1 a) (pathscomp (pathsinv (p (pr1 a ))) (pr2 a)). 
\end{coqcode}
\end{center}
Similarly, there is a corresponding map
\begin{align*}
  \texttt{hfiber}\;g\;b\to\texttt{hfiber}\;f\;b
\end{align*}
because the homotopy relation is symmetric.
\begin{center}
\begin{coqcode}
Definition hfiberandhomotinv { A B : UU } { f g : A -> B } ( b : B ) ( p : homot f g ) : hfiber g b -> hfiber f b := fun a => pair ( pr1 a ) (pathscomp ( ( p ( pr1 a ) ) ) ( pr2 a )). 
\end{coqcode}
\end{center}
Finally, these two maps constitute a homotopy equivalence of spaces as
the following lemma confirms:
\begin{center}
\begin{coqcode}
Lemma isheqhfiberandhomot { A B : UU } { f g : A -> B } ( b : B ) ( p : homot f g ) : isheq ( hfiberandhomot b p ).
\end{coqcode}
\end{center}
We leave the proof of \verb|isheqhfiberandhomot| as an exercise
for the reader (the proof can also be found in the accompanying Coq
file). 

\subsection{Weak equivalences}\label{section:weq_def}

Classically, a map $f:A\to B$ is a \emph{weak equivalence} when it
induces isomorphisms on homotopy groups.  However, we will give a
definition, following Voevodsky \cite{Vo2012a}, which makes sense without referring
to homotopy groups.  In Section \ref{section:gradth} below we will
show that these weak equivalences coincide, for the ``nice spaces'' we
are considering, with the homotopy equivalences (and therefore also with
the classical weak equivalences).  We start with the definition:
\begin{center}
  \begin{coqcode}
Definition isweq { A B : UU } ( f : A -> B ) := ( forall b : B, iscontr ( hfiber f b ) ).
  \end{coqcode}
\end{center}
From the ``propositions as types'' point of view, the weak
equivalences correspond to the bijections.  The space of weak
equivalences from $A$ to $B$ is defined as follows:
\begin{center}
  \begin{coqcode}
Definition weq ( A B : UU ) := total ( fun f : A -> B => isweq f ).
  \end{coqcode}
\end{center}
That is, a typical term of type \verb|weq A B| is a pair
consisting of a map \verb|f : A -> B| together with a term of
type \verb|isweq f|.  The most basic example of a weak equivalence
is the identity function \verb|idfun A : A -> A|.  It is
straightforward to construct the required proof that this is a weak
equivalence.  We denote by \verb|isweqidfun A| this term of type 
\verb|isweq ( idfun A )| and we then adopt the following definition:
\begin{center}
  \begin{coqcode}
Definition idweq ( A : UU ) := pair ( idfun A ) ( isweqidfun A ).
  \end{coqcode}
\end{center}
That is, \verb|idweq A| is the representative of the identity
function on \verb|A| as a weak equivalence.  

Given a map \verb|f : A -> B| together with a proof 
\verb|is : isweq f| that it is a weak equivalence, if 
we are given a point \verb|b : B|, then there is a corresponding 
center of contraction in the homotopy fiber of \verb|f| over 
\verb|b| given explicitly by the term \verb|pr1 ( is b )|.
Because it is often convenient to make use the actual term of type 
\verb|A| corresponding to this center of contraction we introduce
the following nomenclature:
\begin{center}
  \begin{coqcode}
Definition weqpreimage { A B : UU } { f : A -> B } ( is : isweq f ) ( b : B ) : A := pr1 ( pr1 ( is b ) ).
  \end{coqcode}
\end{center}
That is, \verb|weqpreimage is b| is the (homotopically) canonical
element in the homotopy fiber of \verb|f| over \verb|b|.
Similarly, we name the path from \verb|f ( weqpreimage is b )| to 
\verb|b| as follows:
\begin{center}
  \begin{coqcode}
Definition weqpreimageeq { A B : UU } { f : A -> B } ( is : isweq f ) ( b : B ) : paths ( f ( weqpreimage is b ) ) b := pr2 ( pr1 ( is b ) ).
  \end{coqcode}
\end{center}
That is, we have:
\begin{align*}
   \begin{tikzpicture}[auto]
    \node (A) at (0,0) {$f(\texttt{weqpreimage is }b)$};
    \node (B) at (5,0) {$b$.};
    \draw[->-=.5] (A) to node[mylabel] {\texttt{weqpreimageeq is }$b$} (B);
  \end{tikzpicture}
\end{align*}
It is also clear from the definition of \verb|weqpreimage| that
if we are given any point \verb|a : A| and path \verb|g|
from \verb|f a| to \verb|b|, there exists a path
\begin{align*}
 \begin{tikzpicture}[auto]
    \node (A) at (0,0) {$\texttt{weqpreimage is }b$};
    \node (B) at (5,0) {$a$.};
    \draw[->-=.5] (A) to node[mylabel] {\texttt{weqpreimageump1 is
      }$b$ $g$} (B);
  \end{tikzpicture}
\end{align*}
We leave the construction of the term \verb|weqpreimageump1| to
the reader.  Finally, the operation of taking the preimage is
injective in the sense that if there is a path
\begin{align*}
  \begin{tikzpicture}[auto]
    \node (A) at (0,0) {$\texttt{weqpreimage is }b$};
    \node (B) at (5,0) {$\texttt{weqpreimage is }b'$,};
    \draw[->-=.5] (A) to node[mylabel] {$g$} (B);
  \end{tikzpicture}
\end{align*}
then there exists an induced path \verb|weqpreimageump2 is g|
from \verb|b| to \verb|b'|.

\subsection{Weak equivalences and homotopy equivalences}\label{section:gradth}

Given a weak equivalence \verb|f : A -> B|, we construct a
homotopy inverse of \verb|f| as follows:
\begin{center}
  \begin{coqcode}
Definition weqinv { A B : UU } ( f : weq A B ) : B -> A := fun x => weqpreimage ( pr2 f ) x.
  \end{coqcode}
\end{center}
Using the observations from Section \ref{section:weq_def} it is
straightforward to prove the following lemmas:
\begin{center}
  \begin{coqcode}
Lemma weqinvislinv { A B : UU } ( f : weq A B ) : homot ( funcomp ( weqinv f ) ( pr1 f ) ) ( idfun _ ).
  \end{coqcode}
\end{center}
and
\begin{center}
  \begin{coqcode}
Lemma weqinvisrinv { A B : UU } ( f : weq A B ) : homot ( funcomp ( pr1 f ) ( weqinv f ) ) ( idfun _ ).
  \end{coqcode}
\end{center}
which together constitute the proof of:
\begin{center}
  \begin{coqcode}
Lemma weqtoheq { A B : UU } { f : A -> B } (is : isweq f) : isheq f.
  \end{coqcode}
\end{center}
Classically it should be noted that the analogue of 
\verb|weqtoheq| only applies in the case of reasonably nice
spaces.  Interestingly, the proof of the converse of \verb|weqtoheq| is not
entirely trivial.  (Categorically, the proof of the converse makes
crucial use of the fact that categorical equivalences can be
transformed into adjoint equivalences \cite{MacLane:1971tv}.)  

The proof of the converse of \verb|weqtoheq| can nicely be broken
up into two lemmas.  The first lemma states that, for a homotopy
retract as in (\ref{eq:hretract}) above, if the homotopy fiber of the
composite $s\circ r$ over a point $a$ of $A$ is contractible, then the
homotopy fiber of $s$ over $a$ is also contractible:
\begin{center}
  \begin{coqcode}
Lemma iscontrhfiberandhretract { A B : UU } { r : A -> B } { s : B -> A } (p : homot ( funcomp s r ) ( idfun _ )) ( a : A ) : iscontr ( hfiber ( funcomp r s ) a ) -> iscontr ( hfiber s a ).
Proof.
 intros is. 
 apply ( @iscontrretract ( hfiber ( funcomp r s ) a ) ( hfiber s a ) 
  ( maponhfiber _ _ _ ) ( secmaponhfiber p a ) ). 
 apply secmaponhfiberissec. assumption. 
Defined.    
  \end{coqcode}
\end{center}
As the Coq code shows, the proof of this fact is an immediate
consequence of \verb|iscontrretract| and
\verb|secmaponhfiberissec|.

The second lemma states that if there is a homotopy from $f:A\to B$ to
$f':A\to B$
and the homotopy fiber of $f'$ over a point $b : B$ is contractible,
then the homotopy fiber of $f$ over $b$ is also contractible:
\begin{center}
  \begin{coqcode}
Lemma iscontrhfiberandhomot { A B : UU } { f f' : A -> B } ( h : homot
f f' ) ( b : B ) : iscontr ( hfiber f' b ) -> iscontr ( hfiber f b ).
Proof.
  intros is. apply ( iscontrandheqinv ( isheqhfiberandhomot b h ) ). 
  assumption.
Defined.
  \end{coqcode}
\end{center}
In this case the proof is an immediate consequence of the fact 
(\verb|iscontrandheqinv|) that being contractible is preserved by
homotopy equivalences together with the fact
(\verb|isheqhfiberandhomot|) that homotopies induce homotopy
equivalences of homotopy fibers.

Using these two lemmas we obtain the converse of \verb|weqtoheq|,
called following Voevodsky the \emph{Grad Theorem}, as follows:
\begin{center}
\begin{coqcode}
Theorem gradth { A B : UU } { f : A -> B } ( is : isheq f ) : isweq f.
Proof.
  intro b. destruct is as [ f' is ]. 
  apply ( @iscontrhfiberandhretract B A f' f ( pr2 is ) ). 
  apply (@iscontrhfiberandhomot B _ (funcomp f' f) (idfun B) (pr1 is)). 
  apply isweqidfun.
Defined.
\end{coqcode}
\end{center}
After applying \verb|intro b| and \verb|destruct| we find
ourselves in the following situation:
\begin{center}
  \begin{coqcode}
  A : UU
  B : UU
  f : A -> B
  f' : B -> A
  is : dirprod (homot (funcomp f' f) (idfun B))
         (homot (funcomp f f') (idfun A))
  b : B
  ============================
   iscontr (hfiber f b)
  \end{coqcode}
\end{center}
Because \verb|f'| is the homotopy inverse of \verb|f| it
then suffices, by \verb|iscontrhfiberandhretract| to show that
the composite $f\circ f'$ has a contractible homotopy fiber over $b$.
This is the effect of the first \verb|apply|, after which the Coq
output is
\begin{center}
  \begin{coqcode}
  A : UU
  B : UU
  f : A -> B
  f' : B -> A
  is : dirprod (homot (funcomp f' f) (idfun B))
         (homot (funcomp f f') (idfun A))
  b : B
  ============================
   iscontr (hfiber (funcomp f' f) b)
  \end{coqcode}
\end{center}
We now observe that there is a homotopy from $f\circ f'$ to the
identity function on $B$ and therefore after applying
\verb|iscontrhfiberandhomot| it remains only to prove that the
identity on $B$ is a weak equivalences, which is precisely the content
of \verb|isweqidfun|. 
