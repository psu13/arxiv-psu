
\section{Some basic inductive types} 
\label{sec:inductive}


First we explain two important notions: induction
and recursion, and how they relate to each other. As a warm up,
we are going to explain them for the case of the natural numbers,
but it is important to keep in mind that for us, the key inductive
and recursive process will take place over more general objects
than the natural numbers. But to understand those, first one
must understand the simpler case of the natural numbers.


\subsection{The inductive type of natural numbers}\label{sec:induction}

Although it is possible to construct arbitrary inductive types in Coq,
we will start by looking at one inductive type which is already
defined in the Coq system.  This is the type \verb|nat| of
natural numbers.  As an inductive type, \verb|nat| is generated
by 
\begin{itemize}
\item the single generator ($0$-ary operation) \verb|0 : nat|; and
\item the single generating function ($1$-ary operation) 
  \verb| S : nat -> nat | (this is the usual \emph{successor function}).
\end{itemize}
In Coq, this inductive type is specified as follows:
\begin{center}
  \begin{coqcode}
Inductive nat := 0 | S : nat -> nat.
  \end{coqcode}
\end{center}
Here the internal Coq command \verb|Inductive| functions
similarly to \verb|Definition| (as we will see below).  For now
the crucial point is to observe that the generating operations of the
inductive type appear on the right of \verb|:=| and are separated
by the symbol $|$.

One of the advantages of working with an inductive type such as the
natural numbers is that functions with inductive domain can be defined
by cases.  To see an example, consider the predecessor function:
\begin{center}
\begin{coqcode}
Definition predecessor ( n : nat ) : nat := 
 match n with
   | 0 => 0
   | S m => m
 end.
\end{coqcode}
\end{center}
This is the way of telling Coq that the predecessor function is the
function $\text{predecessor}:\nat\to\nat$ given by case analysis as
\begin{align*}
  \text{predecessor}(n) & :=
   \begin{cases}
    0 & \text{ if }n=0,\text{ and}\\
    m & \text{ if }n=(m+1).
  \end{cases}
\end{align*}
Because we are using the type \verb|nat| which is predefined in
the Coq system, Coq will recognize that ordinary numerals refer to the
corresponding terms of type \verb|nat|.  So, for example, Coq
knows that \verb|3| is the same as \verb|S ( S ( S 0 ) )|.
Using this, we can play around with the computational abilities of
Coq by entering, say,
\begin{center}
  \begin{coqcode}
Eval compute in predecessor 26.
  \end{coqcode}
\end{center}
Here \verb|Eval compute in| tells Coq that we would like it to
compute the value of the subsequent expression (in this case
\verb|predecessor 26|).  Coq correctly replies
\begin{center}
  \begin{coqcode}
= 25 : nat
  \end{coqcode}
\end{center}
as expected.

The definition of the predecessor function given above using
\verb|match| is a direct definition as described above in
Section \ref{sec:direct_def}.  It is also possible to define the
predecessor function via an indirect definition as follows:
\begin{center}
  \begin{coqcode}
Definition indirect_predecessor ( n : nat ) : nat.
Proof.
  destruct n. exact 0. exact n.
Defined.
  \end{coqcode}
\end{center}
In the proof there are two new tactics.  The first,
\verb|destruct n|, tells the Coq system that we will reason by
cases on the structure of \verb|n| as a term of type
\verb|nat|.  Coq knows that, as a natural number, there are two
cases and in the first case there is no hypothesis necessary (see
Figure \ref{figure:indirect_predecessor}) because \verb|n| is
\verb|0| in this case.  At this stage, we know that we would like
the output of the function to be \verb|0| and we tell Coq this
using the \verb|exact| tactic.  In general, if we enter
\verb|exact| $x$, this tells Coq that the term we are looking for
is \emph{exactly} the term $x$.  Once we have entered
 \verb|exact 0|, Coq moves on to the second possibility: the term is a successor.
Note that in the list of hypotheses at this stage (see Figure
\ref{figure:indirect_predecessor}) the term \verb|n : nat| is
listed and so \emph{superficially} things are just as they were at the
start of the proof.  However, because we earlier employed the
\verb|destruct| tactic, Coq knows that we must now give the
required output of the predecessor function when given the value
\verb|S n|.  As such, we enter \verb|exact n|.  Comparing
\verb|predecessor| and \verb|indirect_predecessor| using the
\verb|Print| command reveals that they are indeed identical terms.
\begin{figure}[ht]
  \begin{tikzpicture}
    \node[smallcoqbox] (zero)  at (0,0) {%
      \begin{minipage}{4.25cm}
        \footnotesize
        \noindent\verb|n : nat|
        
        \noindent\verb|============================|
        
        \noindent\verb| nat|
      \end{minipage}
    };
    \node[anchor=north east, inner sep=2pt] (titlezero) at
    (zero.north east) {\emph{Start of proof}};
    \node[smallcoqbox] (one) at (5.25,0) {%
      \begin{minipage}{4.25cm}
        \footnotesize
        \vphantom{\texttt|n : nat |}
        
        \noindent\verb|============================|
        
        \noindent\verb| nat|
      \end{minipage}
    };
    \node[anchor=north east, inner sep=2pt] (titleone) at
    (one.north east) {\emph{after} \verb|destruct n.|};
    \node[smallcoqbox] (two) at (0,-2.5) {%
      \begin{minipage}{4.25cm}
        \footnotesize
        \noindent\verb|n : nat|
        
        \noindent\verb|============================|
        
        \noindent\verb| nat|
      \end{minipage}
    };
    \node[anchor=north east, inner sep=2pt] (titletwo) at
    (two.north east) {\emph{after} \verb|exact 0.|};
  \end{tikzpicture}
  \label{figure:indirect_predecessor}
  \caption{Coq output during indirect definition of the predecessor function.}
\end{figure}
We will now turn to several of the inductive types more closely
related to the homotopy theoretic side of things.

\subsection{Fibrations and the total space of a fibration} 

Fibrations can be understood as a homotopy theoretic generalization of
the more familiar notion of fiber bundle.  Similarly, they can be
understood as a homotopy theoretic version of Grothendieck fibrations
familiar from category theory and algebraic geometry.  In particular,
for spaces, a fibration is a map
\begin{align*}
\pi \colon E \to B
\end{align*}
which possesses a certain homotopy-lifting property.  In this case we
would refer to $B$ as the \emph{base space} and to $E$ as the
\emph{total space} of the fibration.  Given a point $b$ in $B$, the
\emph{fiber} over $b$, which we sometimes write as $E_{b}$, is just the preimage $\pi^{-1}(b)$ of $b$ under
the map $\pi$.  As mentioned in Section \ref{sec:homotopy_interp}
above, fibrations correspond to
types which depend on parameters (so-called \emph{dependent types}).
In Coq, the way of dealing with dependent types is somewhat different
from the one sketched in Section \ref{sec:homotopy_interp}.  In
particular, given an element \verb|B : UU| a fibration for us is
a term
\begin{center}
  \begin{coqcode}
E : B -> UU.
  \end{coqcode}
\end{center}
The idea that a fibration over a base space $B$ is the same as a map
from $B$ into a suitable universe is a classical idea, especially in
category theory where it is a basic fact that Grothendieck fibrations
over a category $B$ correspond to pseudo-functors from $B$ into the
2-category of small categories.  (These ideas are ubiquitous in
category theory and have been developed in great generality by the
Australian school of category theorists.  A nice exposition of these
ideas can be found in the first section of \cite{Street:1980da}.)

The idea behind this correspondence is that a fibration can be completely recovered from its base space
$B$ together with its fibers by gluing the fibers together in a
coherent way in accordance with the structure of the base space.  The
same can be accomplished in Coq by defining the total space of a
fibration \verb|E : B -> UU| as an inductive type.
\begin{center}
  \begin{coqcode}
Inductive total {B :UU} (E : B -> UU) : UU := 
pair ( x : B ) ( y : E x ).
  \end{coqcode}
\end{center}
Intuitively, \verb|total E| should be thought of as a space consisting of all
pairs \verb|(b,e)|, where \verb|b| is a point of the base
space \verb|B| and \verb|e| in a point of the fiber
\verb|E b| over \verb|b|.

Fiber bundles $E\to B$ are sometimes thought of as a ``twisted'' generalization
of direct products $F\times B\to B$, and the fact that fibrations are
a homotopical generalization of this notion reveals itself type
theoretically by the fact that the total space
construction \verb|total| is a generalization of the construction
of direct products of spaces $A\times B$, which are given by
\begin{center}
  \begin{coqcode}
Definition dirprod { A B : UU } : UU := 
 total ( fun x : A => B ).
  \end{coqcode}
\end{center}
Returning to \verb|total|, we define a projection map
\verb|total E -> B| by 
\begin{center}
  \begin{coqcode}
Definition pr1 { B : UU } { E : B -> UU } : total E -> B := fun z => match z with pair x y => x end.
  \end{coqcode}
\end{center}
This map serves to exhibit \verb|E| as a fibration over
\verb|B|.
