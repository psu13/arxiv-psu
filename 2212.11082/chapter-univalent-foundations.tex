\chapter{The Univalent Foundations of Mathematics}\label{chap:uf}

The univalent foundations program is an approach to mathematics in which mathematics is formalized in dependent type theory, using the homotopy interpretation and the univalence axiom. The homotopy interpretation of type theory fully embraces the idea that between any two elements of a type there is a \emph{type} of identifications, much like between any two points in a topological space there is a \emph{space} of paths between them. This idea was first explored by Awodey and Warren in \cite{AwodeyWarren} and independently by Voevodsky in \cite{Voevodsky06}. With the homotopy interpretation of type theory\index{homotopy interpretation}, outlined in the table below, we think of types as spaces, type families as fibrations, and identifications as paths.
\begin{table}
%\caption{The homotopy interpretation}
\begin{tabular}{ll}
\toprule
\emph{Type theory} &  \emph{Homotopy theory} \\
\midrule
Types  & Spaces \\
Dependent types & Fibrations \\
Elements & Points \\
Dependent pair type & Total space \\
Identity type & Path fibration\\
\bottomrule
\end{tabular}
\end{table}

Voevodsky's univalence axiom characterizes the identity type of the universes in type theory, asserting that for any two types $A$ and $B$ in a universe $\UU$, we have an equivalence
\begin{equation*}
  (A=_{\UU}B)\simeq (A\simeq B).
\end{equation*}
In other words, identifications of types are equivalent to equivalences of types. A consequence of the univalence axiom is that many kinds of isomorphic objects in mathematics, such as isomorphic groups or isomorphic rings, can be identified.

The concept of equivalences generalizes the concept of set-isomorphisms to type theory in a way that is suitable for the homotopy interpretation of type theory. Equivalent types are the same for all practical purposes, just as isomorphic objects are practically the same in everyday mathematics. By the univalence axiom, isomorphic objects get identified.

However, the informal practice of identifying isomorphic objects is technically inconsistent with the set theoretic foundations of mathematics. The extensionality axiom of Zermelo-Fraenkel set theory implies, for instance, that there are many different singleton sets $\{x\}$. All those singleton sets are isomorphic, so the univalence axiom identifies them, which would be inconsistent within Zermelo-Fraenkel set theory. The assumption of the univalence axiom therefore marks our definitive departure from the set-theoretic foundations of mathematics.

Since the univalence axiom characterizes the identity type of the universe, it is important to understand the general task of characterizing the identity type of any given type. It is a crucial observation, which we already made when we discussed the uniqueness of $\refl{}$ in \cref{sec:refl-unique}, that for any $a:A$, the type
\begin{equation*}
  \sm{x:A}a=x  
\end{equation*}
is contractible. Contractible types are types that are singletons up to homotopy, i.e., they are types $A$ that come equipped with a point $a:A$ such that $a=x$ for every $x:A$. We have seen in \cref{prp:contraction-total-space-id} that the total space of all paths starting at $a$ is such a type, so it is an example of a contractible type. The fundamental theorem of identity types asserts that a type family $B$ over $A$ with $b:B(a)$ has a contractible total space
\begin{equation*}
  \sm{x:A}B(x)
\end{equation*}
if and only if $(a=x)\simeq B(x)$ for all $x:A$. The fundamental theorem of identity types can be used to characterize the identity types of virtually any type that we will encounter. Since types are only fully understood if we also have a clear understanding of their identity types, it is one of the core tasks of a homotopy type theorist to characterize identity types, and the fundamental theorem (\cref{thm:id_fundamental}) is the main tool.

Not all types have very complicated identity types. For example, some types have the property that all their identity types are contractible. For example, the types $\emptyt$ and $\unit$ satisfy this condition. Any two terms of such a type can therefore be identified, so in this sense they are \emph{proof irrelevant}. The only thing that matters about such types is whether or not they are inhabited by a term. Analogously, this is also the case for propositions in the propositional calculus or first order logic. Therefore we call such types propositions, and we see that propositions are present in type theory as certain types.

Next, there are the types of which the identity types are propositions. In other words, the identity types of such types have the property of proof irrelevance. We are familiar with this situation from set theory, because equality in set theory is a proposition. Therefore we call such types sets. The types $\N$, $\Z$, and $\Fin{k}$ are all examples of sets.

It is now clear that there is a hierarchy arising: at the bottom of the hierarchy we have the contractible types; then we have the propositions, of which the identity types are contractible; after the propositions we have the sets, of which the identity types are propositions. Defining sets to be of truncation level $0$, we define a type to be of truncation level $k+1$ if its identity types are of truncation level $k$. Types of truncation level $k$ for $k\geq 1$ are also called $k$-types or $k$-groupoids.

This hierarchy of truncation levels is due to Voevodsky, who recognized that, when you are formalizing mathematics in type theory, it is important to specify the truncation level in which you are working. Most mathematics, for example, takes place at truncation level $0$, the level of sets. Groups, rings, posets, and so on are all set-level objects. Categories, on the other hand, are objects of truncation level $1$, the level of the $1$-groupoids. This is because two objects in a category are considered equal if they are isomorphic, and between any two objects in a category there is a set of isomorphisms.

The fundamental theorem of identity types and the basic facts about truncation levels are proved without assuming any axioms. In other words, they are theorems of Martin-L\"of's dependent type theory, as introduced in \cref{chap:type-theory}. In particular, the rules of dependent type theory are sufficient to characterize the identity types of $\Sigma$-types and of the type of natural numbers, and also to prove the disjointness of coproducts. However, there are still two important characterizations of identity types missing: those of $\Pi$-types and those of universes. For those two cases we need axioms:
\begin{enumerate}
\item For any two dependent functions $f,g:\prd{x:A}B(x)$, the canonical map
  \begin{equation*}
    (f=g)\to (f\htpy g)
  \end{equation*}
  that maps $\refl{f}$ to the constant homotopy, is an equivalence.
\item For any two types $A$ and $B$ in a universe $\UU$, the canonical map
  \begin{equation*}
    (A=B)\to (A\simeq B)
  \end{equation*}
  that maps $\refl{A}$ to the identity equivalence, is an equivalence.
\end{enumerate}
The function extensionality axiom (i) characterizes the identity types of $\Pi$-types, and the univalence axiom (ii) characterizes the identity types of universes.

With the addition of the function extensionality axiom and the univalence axiom, we have almost fully specified the univalent foundations of mathematics. The one ingredient missing is that of quotients. In order to obtain quotients, we will postulate two more axioms:
\begin{enumerate}
  \addtocounter{enumi}{2}
\item We will assume that every type $A$ has a propositional truncation.
\item We will assume the type theoretic \emph{replacement axiom}. 
\end{enumerate}

Propositional truncations are the simplest kind of quotients, identifying all the elements in a type $A$. In other words, the propositional truncation of a type $A$ is a proposition $\brck{A}$ that is true if and only if $A$ is inhabited. Using propositional truncations we can construct the homotopy image of a map. A quotient of a type $A$ by an equivalence relation $R$ can then be constructed as the type of all equivalence classes of $R$, i.e., as the image of the map $R:A\to (A\to \prop_\UU)$. With this construction of the quotient, we immediately obtain a surjective map $q:A\to A/R$, and by the univalence axiom it follows that the quotient is \emph{effective}, i.e., that for any $x,y:A$ we have
\begin{equation*}
  (q(x)=q(y))\simeq R(x,y).
\end{equation*}
However, this construction does not guarantee that the quotient $A/R$ is small with respect to the universe $\UU$, because it is constructed as a subtype of the type $A\to\prop_\UU$. This is why we assume the replacement axiom, which will imply that the quotient $A/R$ is \emph{essentially} small. Essentially small types are types that are equivalent to a small type, and the replacement axiom asserts that if $f:A\to B$ is a map from an essentially small type $A$ into a type $B$ of which the identity types are essentially small, then the image of $f$ is also essentially small. The role of the replacement axiom in type theory is similar to the role of the replacement axiom in Zermelo-Fraenkel set theory: to ensure that quotients are small.

We have two goals in this chapter. The first goal is to fully describe the univalent foundations of mathematics and its most important concepts. Our second goal is to show how to we can start doing ordinary mathematics from a univalent point of view. We therefore show how to derive the strong induction principle for the natural numbers using function extensionality; we give a new interpretation of logic in univalent mathematics using our definition of propositions and the propositional truncations; we show how Cantor's diagonal argument works in univalent mathematics; and we introduce finite types, binomial types, set quotients, the univalent type of all groups. We end this chapter with a variant of Russell's paradox, showing that for any univalent universe $\UU$ there can be no type $U:\UU$ that is equivalent to $\UU$. We hope that after seeing these familiar examples, you will be able to do your own mathematics from a univalent point of view.

\section{Equivalences}\label{sec:equivalences}
\index{equivalence|(} 

In this section we will define equivalences of types. However, we have to be a bit careful in how we define the condition for a map to be an equivalence. It turns out to be important that being an equivalence is a \emph{property} of maps, and not a \emph{structure} on maps. In other words, we want to define the type
\begin{equation*}
  \isequiv(f)
\end{equation*}
in such a way that we will be able to prove that the type $\isequiv(f)$ is a \emph{proposition}. Propositions will be defined in \cref{chap:hierarchy}, and in \cref{chap:funext} we will be able to prove that $\isequiv(f)$ is indeed a proposition.

It turns out that if we naively define a function $f$ to be an equivalence if it has an inverse, then we won't be able to show that $\isequiv(f)$ is a property. We will therefore say that $f$ is an equivalence if it has a separate left and right inverse. This may look odd, but when we define equivalences in this way we will be able to show that $\isequiv(f)$ is a property.

\subsection{Homotopies}
\index{homotopy|(}

In type theory we are very limited in constructing identifications of functions. The following example illustrates a case where type theory provides no rules to construct an identification between two maps, even though they are pointwise equal.

\begin{rmk}\label{rmk:negnegbool}
  Consider the negation function $\negbool : \bool\to\bool$\index{neg bool@{$\negbool$}} on the booleans, which was defined in \cref{ex:bool}. Type theory does not provide any means to show that
  \begin{equation*}
    \negbool\circ\negbool=\idfunc.
  \end{equation*}
  The best we can do is to construct an identification\index{neg neg bool@{$\negnegbool$}|textbf}
  \begin{equation*}
    \negnegbool(b) : \negbool(\negbool(b))=b
  \end{equation*}
  for any $b:\bool$. Indeed, $\negnegbool$ is defined using the induction principle of $\bool$, by
  \begin{align*}
    \negnegbool(\btrue) & \defeq \refl{\btrue} \\
    \negnegbool(\bfalse) & \defeq \refl{\bfalse}.
  \end{align*}
  Therefore we see that, while we cannot identify $\negbool\circ\negbool$ with $\idfunc$, we can define a \emph{pointwise identification} between the values of $\negbool\circ\negbool$ and $\idfunc$.
\end{rmk}

The observations in \cref{rmk:negnegbool} are an instance of a general phenomenon in type theory: it is often much easier to construct a \emph{pointwise identification} between the values of two maps, than it is to construct an identification between those two maps. In fact, the prevalent notion of sameness of maps is the notion of pointwise identification. Since they are so important, we will give them a name and call them \emph{homotopies}\index{pointwise identification|see {homotopy}}\index{pointwise identification|textbf}.

\begin{defn}
Let $f,g:\prd{x:A}B(x)$ be two dependent functions. The type of \define{homotopies}\index{homotopy|textbf} from $f$ to $g$ is defined as the type of pointwise identifications, i.e., we define\index{f htpy g@{$f\htpy g$}|see {homotopy}}\index{f htpy g@{$f\htpy g$}|textbf}
\begin{equation*}
f\htpy g \defeq \prd{x:A} f(x)=g(x).
\end{equation*}
\end{defn}

\begin{eg}
  By \cref{rmk:negnegbool} we have a homotopy
  \begin{equation*}
    \negnegbool : \negbool\circ\negbool\htpy\idfunc.
  \end{equation*}
\end{eg}

\begin{rmk}\label{rmk:commuting-diagrams}
  We will use homotopies, for example, to express the commutativity of diagrams. For example, we say that a triangle\index{homotopy!commutative diagram|textbf}
  \begin{equation*}
    \begin{tikzcd}[column sep=tiny]
      A \arrow[rr,"h"] \arrow[dr,swap,"f"] & & B \arrow[dl,"g"] \\
      & X
    \end{tikzcd}
  \end{equation*}
  \define{commutes}\index{commutative diagram|textbf} if it comes equipped with a homotopy $H:f\htpy g\circ h$. Similarly, we say that a square
  \begin{equation*}
    \begin{tikzcd}
      A \arrow[r,"g"] \arrow[d,swap,"f"] & A' \arrow[d,"{f'}"] \\
      B \arrow[r,swap,"h"] & B'
    \end{tikzcd}
  \end{equation*}
  commutes if it comes equipped with a homotopy $h \circ f\htpy f'\circ g$.
\end{rmk}

Note that the type of homotopies $f\htpy g$ is defined for dependent functions, and moreover the type of homotopies is itself a dependent function type. The definition of homotopies is therefore set up in such a way that we may also consider homotopies \emph{between}\index{homotopy!iterated homotopy}\index{iterated homotopies} homotopies, and even further homotopies between those higher homotopies. More concretely, if $H,K:f\htpy g$ are two homotopies, then the type of homotopies $H\htpy K$ between them is just the type
\begin{equation*}
\prd{x:A} H(x)=K(x).
\end{equation*}

\index{groupoid laws!of homotopies|(}
\index{homotopy!groupoid laws|(}
Since homotopies are pointwise identifications, we can use the groupoidal structure of identity types to also define the groupoidal structure of homotopies. In this case, however, we state the groupoid laws as \emph{homotopies} and \emph{homotopies between homotopies} rather than as identifications.

\begin{defn}\label{defn:htpy_groupoid}\index{groupoid laws!of homotopies}
  For any type family $B$ over $A$ we define the operations on homotopies
  \index{homotopy!refl-htpy@{$\reflhtpy$}|textbf}
  \index{refl-htpy@{$\reflhtpy$}|textbf}
  \index{homotopy!inv-htpy@{$\invhtpy$}|textbf}
  \index{inv-htpy@{$\invhtpy$}|textbf}
  \index{homotopy!concathtpy@{$\concathtpy$}|textbf}
  \index{concat-htpy@{$\concathtpy$}|textbf}
  \begin{align*}
    \reflhtpy & : \prd{f:\prd{x:A}B(x)}f\htpy f \\
    \invhtpy & : \prd{f,g:\prd{x:A}B(x)} (f\htpy g)\to(g\htpy f) \\
    \concathtpy & : \prd{f,g,h:\prd{x:A}B(x)} (f\htpy g)\to ((g\htpy h)\to (f\htpy h))
  \end{align*}
  pointwise by
  \begin{align*}
    \reflhtpy(f) & \defeq \lam{x} \refl{f(x)} \\
    \invhtpy(H) & \defeq \lam{x} H(x)^{-1} \\
    \concathtpy(H,K) & \defeq \lam{x}\ct{H(x)}{K(x)}.
  \end{align*}
  We will often write $H^{-1}$ for $\invhtpy(H)$, and $\ct{H}{K}$ for $\concathtpy(H,K)$.
\end{defn}

\begin{prp}
  Homotopies satisfy the groupoid laws:
  \begin{enumerate}
  \item Concatenation of homotopies is associative up to homotopy, i.e., there is a homotopy
    \begin{equation*}
      \assochtpy(H,K,L) : \ct{(\ct{H}{K})}{L}\htpy\ct{H}{(\ct{K}{L})}
    \end{equation*}
    for any homotopies $H:f\htpy g$, $K:g\htpy h$ and $L:h\htpy i$.
  \item Homotopies satisfy the left and right unit laws up to homotopy, i.e., there are homotopies
    \begin{align*}
    \leftunithtpy(H) & : \ct{\reflhtpy_f}{H}\htpy H \\
    \rightunithtpy(H) & : \ct{H}{\reflhtpy_g}\htpy H 
    \end{align*}
    for any homotopy $H$.
  \item Homotopies satisfy the left and right inverse laws up to homotopy, i.e., there are homotopies
    \begin{align*}
      \leftinvhtpy(H) & : \ct{H^{-1}}{H} \htpy \reflhtpy_g \\
      \rightinvhtpy(H) & : \ct{H}{H^{-1}} \htpy \reflhtpy_f
    \end{align*}
    for any homotopy $H$.
  \end{enumerate}
\end{prp}

\begin{proof}
  The homotopy $\assochtpy(H,K,L)$ is defined pointwise by
  \begin{equation*}
    \assochtpy(H,K,L,x) \defeq \assoc(H(x),K(x),L(x)).
  \end{equation*}
  The other homotopies are similarly defined pointwise.
\end{proof}
\index{groupoid laws!of homotopies|)}
\index{homotopy!groupoid laws|)}

Apart from the groupoid operations and their laws, we will occasionally need \emph{whiskering} operations. Whiskering operations are operations that allow us to compose homotopies with functions. There are two situations where we want this:
\begin{equation*}
  \begin{tikzcd}
    A \arrow[r,bend left,""{name=A,below}] \arrow[r,bend right,""{name=B,above}] \arrow[from=A,to=B,draw=none,"\Downarrow" description] & B \arrow[r] & C & A \arrow[r] & B \arrow[r,bend left,""{name=C,below}] \arrow[r,bend right,""{name=D,above}] \arrow[from=C,to=D,draw=none,"\Downarrow" description] & C.
  \end{tikzcd}
\end{equation*}

\begin{defn}
We define the following \define{whiskering}\index{homotopy!whiskering operations|textbf}\index{whiskering operations!of homotopies|textbf} operations on homotopies:
\begin{enumerate}
\item Suppose $H:f\htpy g$ for two functions $f,g:A\to B$, and let $h:B\to C$. We define\index{h . H@{$h\cdot H$}|see {homotopy, whiskering operations}}\index{h . H@{$h\cdot H$}|textbf}
\begin{equation*}
h\cdot H\defeq \lam{x}\ap{h}{H(x)}:h\circ f\htpy h\circ g.
\end{equation*}
\item Suppose $f:A\to B$ and $H:g\htpy h$ for two functions $g,h:B\to C$. We define\index{H . f@{$H\cdot f$}|see {homotopy, whiskering operations}}\index{H . f@{$H\cdot f$}|textbf}
\begin{equation*}
H\cdot f\defeq\lam{x}H(f(x)):g\circ f\htpy h\circ f.
\end{equation*}
\end{enumerate}
\end{defn}
\index{homotopy|)}

\subsection{Bi-invertible maps}

We use homotopies to define sections and retractions of a map $f$, and to define what it means for a map $f$ to be an equivalence.

\begin{defn}
  Let $f:A\to B$ be a function.
  \begin{enumerate}
  \item The type of \define{sections} of $f$\index{section!of a map|textbf}\index{function!section of a map|textbf} is defined to be the type\index{sec(f)@{$\sections(f)$}|textbf}
    \begin{equation*}
      \sections(f) \defeq \sm{g:B\to A} f\circ g\htpy \idfunc[B].
    \end{equation*}
    In other words, a \define{section} of $f$ is a map $g:B\to A$ equipped with a homotopy $f\circ g\htpy \idfunc$. 
  \item The type of \define{retractions} of $f$\index{retraction|textbf}\index{function!has a retraction|textbf} is defined to be the type\index{retr(f)@{$\retractions(f)$}|textbf}
    \begin{equation*}
      \retractions(f) \defeq \sm{h:B\to A} h\circ f\htpy \idfunc[A].
    \end{equation*}
    If a map $f:A \to B$ has a retraction, we also say that $A$ is a \define{retract}\index{retract!of a type|textbf} of $B$.
  \item We say that a function $f:A\to B$ is an \define{equivalence}\index{equivalence|textbf}\index{is an equivalence|textbf}\index{function!is an equivalence|textbf} if it has both a section and a retraction, i.e., if it comes equipped with an element of type\index{is-equiv(f)@{$\isequiv(f)$}|textbf}
    \begin{equation*}
      \isequiv(f)\defeq\sections(f)\times\retractions(f).
    \end{equation*}
    We will write $\eqv{A}{B}$\index{A simeq B@{$\eqv{A}{B}$}|see {equivalence}} for the type $\sm{f:A\to B}\isequiv(f)$ of all equivalences from $A$ to $B$.
    For any equivalence $e:A\simeq B$ we define $e^{-1}$ to be the section of $e$.\index{equivalence!inverse|textbf}\index{inverse!of an equivalence|textbf}
  \end{enumerate}
\end{defn}

\begin{rmk}
An equivalence, as we defined it here, can be thought of as a \emph{bi-invertible map}\index{bi-invertible map|see {equivalence}}, since it comes equipped with a separate left and right inverse. Explicitly, if $f$ is an equivalence, then there are
\begin{align*}
g & : B\to A & h & : B\to A \\
G & : f\circ g \htpy \idfunc[B] & H & : h\circ f \htpy \idfunc[A].
\end{align*}
\end{rmk}

\begin{eg}\label{thm:id_equiv}
  For any type $A$, the identity function $\idfunc:A\to A$ is an equivalence, since it is its own section and its own retraction\index{identity function!is an equivalence}\index{is an equivalence!identity function}
\end{eg}

\begin{eg}\label{ex:neg_equiv}
  Since we have seen in \cref{rmk:negnegbool} that the negation function $\negbool:\bool\to\bool$ on the booleans is its own inverse, it follows that $\negbool$ is an equivalence.\index{neg bool@{$\negbool$}!is an equivalence}\index{is an equivalence!neg bool@{$\negbool$}}
\end{eg}

\begin{eg}\label{eg:is-equiv-succ-Z}
  The successor and predecessor functions on $\Z$ are equivalences by \cref{ex:is-equiv-succ-Z}\index{successor function!on Z@{on $\Z$}!is an equivalence}\index{succ Z@{$\succZ$}!is an equivalence}\index{is an equivalence!succ Z@{$\succZ$}}. Furthermore, the function
  \begin{equation*}
    x\mapsto x+k
  \end{equation*}
  is an equivalence from $\Z$ to $\Z$, for each $k:\Z$. This follows from the group laws on $\Z$, proven in \cref{ex:int_group_laws}. Indeed, the inverse of $x\mapsto x+k$ is the map $x\mapsto x+(-k)$. Finally, it also follows from the group laws on $\Z$ that the map $x\mapsto -x$ is an equivalence.

  The same holds for the finite types: the maps $\succFin_{k}$, $\predFin_{k}$, $\addFin_{k}(x)$ and $\negFin_{k}$ are all equivalences on $\Fin{k}$.
\end{eg}

\begin{rmk}\label{rmk:has-inverse}
  More generally, if $f$ \define{has an inverse}\index{has an inverse|textbf}\index{function!has an inverse|textbf} in the sense that we have a function $g:B\to A$ equipped with homotopies $f\circ g\htpy\idfunc[B]$ and $g\circ f\htpy\idfunc[A]$, then $f$ is an equivalence. We write\index{has-inverse(f)@{$\hasinverse(f)$}}
  \begin{equation*}
    \hasinverse(f)\defeq\sm{g:B\to A} (f\circ g\htpy \idfunc[B])\times (g\circ f\htpy\idfunc[A]).
  \end{equation*}
  However, we did \emph{not} define equivalences to be functions that have inverses. The reason is that we would like that being an equivalence is a \emph{property}, not a non-trivial structure on the map $f$. This fact requires the function extensionality axiom, but we can already say that if a map $f$ is an equivalence, then it has up to homotopy only one section and only one retraction (see \cref{ex:isprop_isequiv}).

  The type $\hasinverse(f)$ on the other hand, turns out to be homotopically complicated. In \cref{ex:is_invertible_id_S1} we will see that the identity function $\idfunc[\sphere{1}]:\sphere{1}\to\sphere{1}$ on the circle is an example of a map for which
  \begin{equation*}
    \hasinverse(\idfunc[\sphere{1}])\simeq \Z.
  \end{equation*}
\end{rmk}

Even though $\isequiv(f)$ and $\hasinverse(f)$ can be wildly different types, there are maps back and forth between the two. We have already observed in \cref{rmk:has-inverse} that there is a map
\begin{equation*}
  \hasinverse(f)\to\isequiv(f).
\end{equation*}
The following proposition gives the converse implication.

\begin{prp}\label{lem:inv_equiv}
  Any map $f:A\to B$ which is an equivalence, can be given the structure of an invertible map\index{equivalence!has an inverse} i.e., there is a map
  \begin{equation*}
    \isequiv(f)\to\hasinverse(f).
  \end{equation*}
\end{prp}

\begin{proof}
First we construct for any equivalence $f$ with right inverse $g$ and left inverse $h$ a homotopy $K:g\htpy h$. For any $y:B$, we have 
\begin{equation*}
\begin{tikzcd}[column sep=huge]
g(y) \arrow[r,equals,"H(g(y))^{-1}"] & hfg(y) \arrow[r,equals,"\ap{h}{G(y)}"] & h(y).
\end{tikzcd}
\end{equation*} 
In other words, the homotopy $K:g\htpy h$ is defined to be $\ct{(H\cdot g)^{-1}}{(h\cdot G)}$.
Using the homotopy $K$ we are able to show that $g$ is also a left inverse of $f$. For $x:A$ we have the identification
\begin{equation*}
\begin{tikzcd}[column sep=large]
gf(x) \arrow[r,equals,"K(f(x))"] & hf(x) \arrow[r,equals,"H(x)"] & x.
\end{tikzcd}\qedhere
\end{equation*}
\end{proof}

\begin{cor}
The inverse of an equivalence is again an equivalence.\index{inverse!of an equivalence!is an equivalence}\index{is an equivalence!inverse of an equivalence}
\end{cor}

\begin{proof}
Let $f:A\to B$ be an equivalence. By \cref{lem:inv_equiv} it follows that the section of $f$ is also a retraction. Therefore it follows that the section is itself an invertible map, with inverse $f$. Hence it is an equivalence.
\end{proof}

\begin{eg}\label{eg:laws-products-coproducts}
  Types, just as sets in classical mathematics, satisfy the usual laws of coproducts and products, such as unit laws, commutativity, and associativity. These laws are formulated as equivalences:\index{unit laws!for coproducts}\index{associativity!of coproducts}\index{zero laws!for cartesian products}\index{unit laws!for cartesian products}\index{commutativity!of coproducts}\index{commutativity!of cartesian products}\index{associativity!of cartesian products}\index{distributivity!of cartesian product over coproduct}\index{coproduct!unit laws}\index{coproduct!associativity}\index{coproduct!commutativity}\index{cartesian product type!zero laws}\index{cartesian product type!unit laws}\index{cartesian product type!commutativity}\index{cartesian product type!associativity}\index{cartesian product type!distributivity over coproducts}
  \begin{align*}
    \emptyt+B & \simeq B & A+\emptyt & \simeq A \\
    A+B & \simeq B+A & (A+B)+C & \simeq A+(B+C) \\
    \emptyt\times B & \simeq \emptyt & A\times\emptyt & \simeq \emptyt\\
    \unit\times B & \simeq B & A\times\unit & \simeq A \\
    A\times B & \simeq B\times A & (A \times B) \times C & \simeq A \times (B \times C) \\
    A\times (B+C) & \simeq (A\times B)+(A\times C) & (A+B)\times C & \simeq (A\times C)+(B\times C).
  \end{align*}
  All of these equivalences are constructed in a similar way: the maps back and forth as well as the required homotopies are constructed using induction, or, more efficiently, using pattern matching. For example, to show that cartesian products distribute from the left over coproducts, we construct maps
  \begin{align*}
    \alpha & : A\times(B+C)\to (A\times B)+(A\times C) \\
    \beta & : (A\times B)+(A\times C)\to A\times(B+C)
  \end{align*}
  as follows:
  \begin{align*}
    \alpha(x,\inl(y)) & \defeq \inl(x,y) & \beta(\inl(x,y)) & \defeq (x,\inl(y)) \\
    \alpha(x,\inr(z)) & \defeq \inr(x,z) & \beta(\inr(x,z)) & \defeq (x,\inr(z)).
  \end{align*}
  The homotopies $G:\alpha\circ\beta\htpy\idfunc$ and $H:\beta\circ\alpha\htpy \idfunc$ are then defined by
  \begin{align*}
    G(\inl(x,y)) & \defeq \refl{} & H(x,\inl(y)) & \defeq \refl{} \\
    G(\inr(x,z)) & \defeq \refl{} & H(x,\inr(z)) & \defeq \refl{}.
  \end{align*}
  We encourage the reader to write out the definitions of at least a few of these equivalences.
\end{eg}

\begin{eg}\label{eg:laws-Sigma-types}
  The laws for cartesian products can be generalized to arbitrary $\Sigma$-types. The absorption laws and unit laws, for instance, are as follows:
  \index{absorption laws!of dependent pair types}\index{dependent pair type!absorption laws}\index{unit laws!for dependent pair types}\index{dependent pair type!unit laws}
  \begin{align*}
    \sm{x:\emptyt}B(x) & \simeq \emptyt & \sm{x:A}\emptyt & \simeq \emptyt \\
    \sm{x:\unit}B(x) & \simeq B(\ttt) & \sm{x:A}\unit & \simeq A.
  \end{align*}
  Note that the right absorption law and the right unit law are exactly the same as the right absorption and unit laws for cartesian products. The left absorption and unit laws are, however, formulated with a type family $B$ over $\emptyt$ and over $\unit$, and therefore they are slightly more general.
  
  Commutativity cannot be generalized to $\Sigma$-types. Associativity, on the other hand, can be expressed in two ways:\index{associativity!of dependent pair types}\index{dependent pair type!associativity}
  \begin{align*}
    \sm{w:\sm{x:A}B(x)}C(w) & \simeq\sm{x:A}\sm{y:B}C(\pair(x,y)) \\
    \sm{w:\sm{x:A}B(x)}C(\proj 1(w),\proj 2(w)) & \simeq \sm{x:A}\sm{y:B(x)}C(x,y). 
  \end{align*}
  In the first of these equivalences associativity is stated using a type family $C$ over $\sm{x:A}B(x)$ while in the second it is stated using a family of types $C(x,y)$ indexed by $x:A$ and $y:B(x)$.
  
  Finally, we note that $\Sigma$ also distributes over coproducts. In other words, there are the following two equivalences:\index{dependent pair type!distributivity over coproducts}\index{distributivity!of S-types over coproducts@{of $\Sigma$-types over coproducts}}
  \begin{align*}
    \sm{x:A}B(x)+C(x) & \simeq \Big(\sm{x:A}B(x)\Big)+\Big(\sm{x:A}C(x)\Big) \\
    \sm{w:A+B}C(w) & \simeq \Big(\sm{x:A}C(\inl(x))\Big)+\Big(\sm{y:B}C(\inr(y))\Big).
  \end{align*}
\end{eg}

\begin{rmk}
    We haven't stated any laws involving function types or dependent function types, because it requires the function extensionality principle to prove them.
\end{rmk}



\subsection{Characterizing the identity types of \texorpdfstring{$\Sigma$-}{dependent pair }types}

\index{identity type!of a Sigma-type@{of a $\Sigma$-type}|(}
\index{dependent pair type!identity type|(}
\index{characterization of identity type!of S-types@{of $\Sigma$-types}|(}
In this section we characterize the identity type of a $\Sigma$-type as a $\Sigma$-type of identity types. Characterizing identity types is a task that a homotopy type theorist routinely performs, so we will follow the general outline of how such a characterization goes:
\begin{enumerate}
\item First we define a binary relation $R:A\to A\to \UU$ on the type $A$ that we are interested in. This binary relation is intended to be equivalent to its identity type.
\item Then we will show that this binary relation is reflexive, by constructing a dependent function of type
  \begin{equation*}
    \prd{x:A}R(x,x)
  \end{equation*}
\item Using the reflexivity we will show that there is a canonical map
  \begin{equation*}
    (x=y)\to R(x,y)
  \end{equation*}
  for every $x,y:A$. This map is just constructed by path induction, using the reflexivity of $R$.
\item Finally, it has to be shown that the map
  \begin{equation*}
    (x=y)\to R(x,y)
  \end{equation*}
  is an equivalence for each $x,y:A$. 
\end{enumerate}
The last step is usually the most difficult, and we will refine our methods for this step in \cref{chap:fundamental}, where we establish the fundamental theorem of identity types.

In this section we consider a type family $B$ over $A$. Given two pairs
\begin{equation*}
  (x,y),(x',y'):\sm{x:A}B(x),
\end{equation*}
if we have a path $\alpha:x=x'$ then we can compare $y:B(x)$ to $y':B(x')$ by first transporting $y$ along $\alpha$, i.e., we consider the identity type
\begin{equation*}
  \tr_B(\alpha,y)=y'.
\end{equation*}
Thus it makes sense to think of $(x,y)$ to be identical to $(x',y')$ if there is an identification $\alpha:x=x'$ and an identification $\beta:\tr_B(\alpha,y)=y'$. In the following definition we turn this idea into a binary relation on the $\Sigma$-type.

\begin{defn}
  We will define a relation\index{Eq Sigma@{$\EqSigma$}|textbf}\index{dependent pair type!Eq Sigma@{$\EqSigma$}|textbf}\index{dependent pair type!observational equality|textbf}\index{observational equality!on S-types@{on $\Sigma$-types}|textbf}
  \begin{equation*}
    \EqSigma : \Big(\sm{x:A}B(x)\Big)\to\Big(\sm{x:A}B(x)\Big)\to\UU
  \end{equation*}
  by defining
  \begin{equation*}
    \EqSigma(s,t)\defeq\sm{\alpha:\proj 1(s)=\proj 1(t)}\tr_B(\alpha,\proj 2(s))=\proj 2 (t).
  \end{equation*}
\end{defn}

\begin{lem}
  The relation $\EqSigma$ is reflexive, i.e., we can construct
  \begin{equation*}
    \reflexiveEqSigma:\prd{s:\sm{x:A}B(x)}\EqSigma(s,s).
  \end{equation*}
\end{lem}

\begin{constr}
  The element $\reflexiveEqSigma$ is constructed by $\Sigma$-induction on $s:\sm{x:A}B(x)$. Thus, it suffices to construct a dependent function of type
  \begin{equation*}
    \prd{x:A}\prd{y:B(x)}\sm{\alpha:x=x}\tr_B(\alpha,y)=y.
  \end{equation*}
  Here we take $\lam{x}\lam{y}(\refl{x},\refl{y})$.
\end{constr}

\begin{defn}
  Consider a type family $B$ over $A$. Then for any $s,t:\sm{x:A}B(x)$ we define a map\index{pair-eq@{$\paireq$}|textbf}
  \begin{equation*}
    \paireq: (s=t)\to \EqSigma(s,t)
  \end{equation*}
  by path induction, taking $\paireq(\refl{s})\defeq\reflexiveEqSigma(s)$.
\end{defn}

\begin{thm}\label{thm:eq_sigma}
  Let $B$ be a type family over $A$. Then the map\index{pair-eq@{$\paireq$}!is an equivalence}\index{is an equivalence!pair-eq@{$\paireq$}}
  \begin{equation*}
    \paireq: (s=t)\to \EqSigma(s,t)
  \end{equation*}
  is an equivalence for every $s,t:\sm{x:A}B(x)$.\index{is an equivalence!pair-eq@{$\paireq$}}
\end{thm}

\begin{proof}
The maps in the converse direction\index{eq-pair@{$\eqpair$}|textbf}
\begin{equation*}
\eqpair : \EqSigma(s,t)\to(\id{s}{t})
\end{equation*}
are defined by repeated $\Sigma$-induction. By $\Sigma$-induction on $s$ and $t$  we see that it suffices to define a map
\begin{equation*}
\eqpair : \Big(\sm{p:x=x'}\id{\tr_B(p,y)}{y'}\Big)\to(\id{(x,y)}{(x',y')}).
\end{equation*}
A map of this type is again defined by $\Sigma$-induction. Thus it suffices to define a dependent function of type
\begin{equation*}
\prd{p:x=x'} (\id{\tr_B(p,y)}{y'}) \to (\id{(x,y)}{(x',y')}).
\end{equation*}
Such a dependent function is defined by double path induction by sending $\pairr{\refl{x},\refl{y}}$ to $\refl{\pairr{x,y}}$. This completes the definition of the function $\eqpair$.

Next, we must show that $\eqpair$ is a section of $\paireq$. In other words, we must construct an identification
\begin{equation*}
\paireq(\eqpair(\alpha,\beta))=\pairr{\alpha,\beta}
\end{equation*}
for each $\pairr{\alpha,\beta}:\sm{\alpha:x=x'}\id{\tr_B(\alpha,y)}{y'}$. We proceed by path induction on $\alpha$, followed by path induction on $\beta$. Then our goal becomes to construct an identification of type
\begin{equation*}
\paireq(\eqpair\pairr{\refl{x},\refl{y}})=\pairr{\refl{x},\refl{y}}
\end{equation*}
By the definition of $\eqpair$ we have $\eqpair\pairr{\refl{x},\refl{y}}\jdeq \refl{\pairr{x,y}}$, and by the definition of $\paireq$ we have $\paireq(\refl{\pairr{x,y}})\jdeq\pairr{\refl{x},\refl{y}}$. Thus we may take $\refl{\pairr{\refl{x},\refl{y}}}$ to complete the construction of the homotopy $\paireq\circ\eqpair\htpy\idfunc$.

To complete the proof, we must show that $\eqpair$ is a retraction of $\paireq$. In other words, we must construct an identification
\begin{equation*}
\eqpair(\paireq(p))=p
\end{equation*}
for each $p:s=t$. We proceed by path induction on $p:s=t$, so it suffices to construct an identification 
\begin{equation*}
\eqpair\pairr{\refl{\proj 1(s)},\refl{\proj 2(s)}}=\refl{s}.
\end{equation*}
Now we proceed by $\Sigma$-induction on $s:\sm{x:A}B(x)$, so it suffices to construct an identification
\begin{equation*}
\eqpair\pairr{\refl{x},\refl{y}}=\refl{(x,y)}.
\end{equation*}
Since $\eqpair\pairr{\refl{x},\refl{y}}$ computes to $\refl{(x,y)}$, we may simply take $\refl{\refl{(x,y)}}$.
\end{proof}
\index{identity type!of a Sigma-type@{of a $\Sigma$-type}|)}
\index{dependent pair type!identity type|)}
\index{characterization of identity type!of S-types@{of $\Sigma$-types}|)}

\begin{exercises}
  \exitem \label{ex:equiv_grpd_ops}Show that the functions\index{inv@{$\invfunc$}!is an equivalence}\index{is an equivalence!inv@{$\invfunc$}}\index{concat@{$\concat$}!is a family of equivalences}\index{is an equivalence!concat(p)@{$\concat(p)$}}\index{concat'@{$\concat'$}!is a family of equivalences}\index{is an equivalence!concat'@{$\concat'(q)$}}\index{tr B@{$\tr_B$}!is a family of equivalences}\index{is an equivalence!tr B(p)@{$\tr_B(p)$}}
  \begin{align*}
    \invfunc & :(\id{x}{y})\to(\id{y}{x}) \\
    \concat(p) & : (\id{y}{z})\to(\id{x}{z}) \\
    \concat'(q) & : (\id{x}{y}) \to (\id{x}{z}) \\
    \tr_B(p) & :B(x)\to B(y)
  \end{align*}
  are equivalences, where $\concat'(q,p)\defeq \ct{p}{q}$\index{concat'@{$\concat'$}|textbf}. Give their inverses explicitly.
  \exitem
  \begin{subexenum}
  \item Use \cref{ex:zero-neq-one-bool} to show that the constant function $\const_b:\bool\to\bool$ is not an equivalence, for any $b:\bool$.\index{booleans!const b is not an equivalence@{$\const_b$ is not an equivalence}}
  \item Show that $\bool\not\simeq \unit$.
  \item Show that $\N\not\simeq \Fin{k}$ for any $k:\N$. 
  \end{subexenum}
  \exitem
  \begin{subexenum}
  \item \label{ex:htpy_equiv}\index{equivalence!closed under homotopies} Consider two functions $f,g:A\to B$ and a homotopy $H:f\htpy g$. Then
    \begin{equation*}
      \isequiv(f)\leftrightarrow\isequiv(g).
    \end{equation*}
  \item Show that for any two homotopic equivalences $e,e':\eqv{A}{B}$, their inverses are also homotopic.
  \end{subexenum}
  \exitem \label{ex:3_for_2}
  Consider a commuting triangle
  \begin{equation*}
    \begin{tikzcd}[column sep=tiny]
      A \arrow[rr,"h"] \arrow[dr,swap,"f"] & & B \arrow[dl,"g"] \\
      & X.
    \end{tikzcd}
  \end{equation*}
  with $H:f\htpy g\circ h$.
  \begin{subexenum}
  \item Suppose that the map $h$ has a section $s:B \to A$. Show that the triangle
    \begin{equation*}
      \begin{tikzcd}[column sep=tiny]
        B \arrow[rr,"s"] \arrow[dr,swap,"g"] & & A \arrow[dl,"f"] \\
        & X.
      \end{tikzcd}
    \end{equation*}
    commutes, and that $f$ has a section if and only if $g$ has a section.
  \item Suppose that the map $g$ has a retraction $r:X\to B$. Show that the triangle
    \begin{equation*}
      \begin{tikzcd}[column sep=tiny]
        A \arrow[rr,"f"] \arrow[dr,swap,"h"] & & X \arrow[dl,"r"] \\
        & B.
      \end{tikzcd}
    \end{equation*}
    commutes, and that $f$ has a retraction if and only if $h$ has a retraction.
  \item (The \define{3-for-2 property} for equivalences.)\index{equivalence!3-for-2 property}\index{3-for-2 property!of equivalences}\index{composition!of equivalences|textbf}\index{equivalence!composition|textbf} Show that if any two of the functions
    \begin{equation*}
      f,\qquad g,\qquad h
    \end{equation*}
    are equivalences, then so is the third. Conclude that any section and any retraction of an equivalence is again an equivalence.
  \end{subexenum}
  \exitem \label{ex:sigma_swap}
  \begin{subexenum}
  \item Let $A$ and $B$ be types, and let $C$ be a family over $x:A,y:B$. Construct an equivalence
    \begin{equation*}
      \eqv{\Big(\sm{x:A}\sm{y:B}C(x,y)\Big)}{\Big(\sm{y:B}\sm{x:A}C(x,y)\Big)}.
    \end{equation*}
  \item Let $A$ be a type, and let $B$ and $C$ be type families over $A$. Construct an equivalence
    \begin{equation*}
      \Big(\sm{u:\sm{x:A}B(x)}C(\proj 1(u))\Big) \simeq \Big(\sm{v:\sm{x:A}C(x)}B(\proj 1(v))\Big).
    \end{equation*}
  \end{subexenum}
  \exitem \label{ex:coproduct_functor}Recall from \cref{rmk:functor-coprod} that coproducts have a \define{functorial action}\index{functorial action!of coproducts}\index{coproduct!functorial action}, i.e., that for every $f:A\to A'$ and every $g:B\to B'$ we have a map
  \begin{equation*}
    f+g:(A+B)\to (A'+B').
  \end{equation*}
  \begin{subexenum}
  \item Show that $\idfunc[A]+\idfunc[B]\htpy \idfunc[A+B]$.
  \item Show that for any two pairs of composable functions
    \begin{equation*}
      \begin{tikzcd}
        A \arrow[r,"f"] & {A'} \arrow[r,"{f'}"] & {A''}
      \end{tikzcd}
      \qquad\text{and}\qquad
      \begin{tikzcd}
        B \arrow[r,"g"] & {B'} \arrow[r,"{g'}"] & {B''}
      \end{tikzcd}
    \end{equation*}
    there is a homotopy $(f'\circ f)+(g'\circ g) \htpy (f'+g')\circ (f+g)$.
  \item Show that if $H:f\htpy f'$ and $K:g\htpy g'$, then there is a homotopy
    \begin{equation*}
      H+K:(f+g)\htpy (f'+g').
    \end{equation*}
  \item \label{ex:coproduct_functor_equivalence}Show that if both $f$ and $g$ are equivalences, then so is $f+g$. (The converse of this statement also holds, see \cref{ex:is-equiv-is-equiv-functor-coprod}.)
  \end{subexenum}
  \exitem
  \begin{subexenum}
  \item Construct for any two maps $f:A \to A'$ and $g:B\to B'$, a map
    \begin{equation*}
      f\times g : A\times B \to A'\times B'
    \end{equation*}
  \item Show that $\idfunc[A]\times\idfunc[B]\htpy\idfunc[A\times B]$.
  \item Show that for any two pairs of composable functions
    \begin{equation*}
      \begin{tikzcd}
        A \arrow[r,"f"] & {A'} \arrow[r,"{f'}"] & {A''}
      \end{tikzcd}
      \qquad\text{and}\qquad
      \begin{tikzcd}
        B \arrow[r,"g"] & {B'} \arrow[r,"{g'}"] & {B''}
      \end{tikzcd}
    \end{equation*}
    there is a homotopy $(f'\circ f)\times(g'\circ g) \htpy (f'\times g')\circ (f\times g)$.
  \item Show that if $H:f\htpy f'$ and $K:g\htpy g'$, then there is a homotopy
    \begin{equation*}
      H\times K:(f\times g)\htpy (f'\times g').
    \end{equation*}
  \item Show that for any two maps $f:A\to A'$ and $g:B\to B'$, the following are equivalent:
    \begin{enumerate}
    \item The map $f\times g$ is an equivalence.
    \item There are functions
      \begin{align*}
        \alpha & : B \to \isequiv(f) \\
        \beta & : A \to \isequiv(g).
      \end{align*}
    \end{enumerate}
  \end{subexenum}
  \exitem\label{ex:laws-Fin} Construct equivalences
  \begin{align*}
    \Fin{k+l} & \simeq \Fin{k}+\Fin{l} \\
    \Fin{kl} & \simeq \Fin{k}\times\Fin{l}.
  \end{align*}
  \exitem A map $f:X\to X$ is said to be \define{finitely cyclic}\index{finitely cyclic type|textbf} if it comes equipped with an element of type
  \begin{equation*}
    \isfinitelycyclic(f)\defeq\prd{x,y:X}\sm{k:\N}f^k(x)=y.
  \end{equation*}
  \begin{subexenum}
  \item Show that any finitely cyclic map is an equivalence.
  \item Show that $\succFin:\Fin{k}\to\Fin{k}$ is finitely cyclic for any $k:\N$.
  \end{subexenum}
\end{exercises}
\index{equivalence|)}

%%% Local Variables:
%%% mode: latex
%%% TeX-master: "hott-intro"
%%% End:

% !TEX root = hott_intro.tex

\section{Contractible types and contractible maps}
\sectionmark{Contractible types and maps}
\label{sec:contractible}

\index{contractible type|(} 
A contractible type is a type which has, up to identification, only one element. In other words, a contractible type is a type that comes equipped with a point, and an identification of this point with any point.

We may think of contractible types as singletons up to homotopy, and indeed we show that the unit type is an example of a contractible type. Moreover, we show that contractible types satisfy an induction principle that is very similar to the induction principle of the unit type.

Another case of an inductive type with a single constructor is the type of identifications $p:a=x$ with a fixed starting point $a:A$. To specify such an identification, we have to give its end point $x:A$ as well as the identification $p:a=x$, and the path induction principle asserts that in order to show something about all such identifications, it suffices to show that thing in the case where the end point is $a$, and the path is $\refl{a}$. This suggests that the total space
\begin{equation*}
  \sm{x:A}a=x
\end{equation*}
of all paths with starting point $a:A$ is contractible. This important fact will be shown in \cref{thm:total_path}, and it is the basis for the fundamental theorem of identity types (\cref{chap:fundamental}).

Next, we introduce the \emph{fiber} of a map $f:A\to B$. The fiber of $f$ at $b:B$ consists of the type of elements $a:A$ equipped with an identification $p:f(a)=b$. In other words, the fiber of $f$ at $b$ is the preimage of $f$ at $b$. In \cref{thm:equiv_contr,thm:contr_equiv} we show that a map is an equivalence if and only if its fibers are contractible. The condition that the fibers of a map are contractible is analogous to the set theoretic notion of bijective map, or $1$-to-$1$-correspondence.

\subsection{Contractible types}

\begin{defn}
  We say that a type $A$ is \define{contractible}\index{contractible type|textbf} if it comes equipped with an element of type\index{is-cont(A)r@{$\iscontr(A)$}|see {contractible type}}
  \begin{equation*}
    \iscontr(A) \defeq \sm{c:A}\prd{x:A}c=x.
  \end{equation*}
  Given a pair $(c,C):\iscontr(A)$, we call $c:A$ the \define{center of contraction}\index{center of contraction|textbf}\index{contractible type!center of contraction|textbf} of $A$, and we call $C:\prd{x:A}c=x$ the \define{contraction}\index{contraction|textbf}\index{contractible type!contraction|textbf} of $A$.
\end{defn}

\begin{rmk}
Suppose $A$ is a contractible type with center of contraction $c$ and contraction $C$. Then the type of $C$ is (judgmentally) equal to the type
\begin{equation*}
\const_c\htpy\idfunc[A].
\end{equation*}
In other words, the contraction $C$ is a \emph{homotopy} from the constant function to the identity function.
\end{rmk}

\begin{eg}
  The unit type is easily seen to be contractible.\index{unit type!is contractible}\index{is contractible!unit type} For the center of contraction we take $\ttt:\unit$. Then we define a contraction $\prd{x:\unit}\ttt=x$ by the induction principle of $\unit$. Applying the induction principle, it suffices to construct an identification of type $\ttt = \ttt$, for which we just take $\refl{\ttt}$.
\end{eg}

\begin{thm}\label{thm:total_path}
For any $a:A$, the type
\begin{equation*}
\sm{x:A}a=x
\end{equation*}
is contractible.\index{identity type!total space is contractible}\index{is contractible!total space of identity type}
\end{thm}

\begin{proof}
  For the center of contraction we take
  \begin{equation*}
    (a,\refl{a}):\sm{x:A}a=x.
  \end{equation*}
  The contraction is constructed in \cref{prp:contraction-total-space-id}.
\end{proof}

\subsection{Singleton induction}
\index{singleton induction|(}
\index{induction principle!singleton induction|(}

Contractible types are singletons up to homotopy. Indeed, every element of a contractible type can be identified with the center of contraction. Therefore we can prove an induction principle for contractible types that is similar to the induction principle of the unit type.

\begin{defn}\label{defn:singleton-induction}
  Suppose $A$ comes equipped with an element $a:A$. Then we say that $A$ satisfies \define{singleton induction}\index{singleton induction|textbf}\index{induction principle!singleton induction|textbf} if for every type family $B$ over $A$, the map\index{ev-pt@{$\evpt$}|textbf}
  \begin{equation*}
    \evpt:\Big(\prd{x:A}B(x)\Big)\to B(a)
  \end{equation*}
  defined by $\evpt(f)\defeq f(a)$ has a section. In other words, if $A$ satisfies singleton induction we have a function and a homotopy\index{ind-sing@{$\singind$}|textbf}\index{comp-sing@{$\singcomp$}|textbf}
  \begin{align*}
    \singind_{a} & : B(a)\to \prd{x:A}B(x) \\
    \singcomp_{a} & : \evpt\circ \singind_{a} \htpy \idfunc
  \end{align*}
  for any type family $B$ over $A$.
\end{defn}

\begin{eg}
  Note that the singleton induction principle is almost the same as the induction principle for the unit type, the difference being that the `computation rule' in the singleton induction for $A$ is stated using an \emph{identification} rather than as a judgmental equality. The unit type\index{unit type!singleton induction} $\unit$ comes equipped with a function
  \begin{equation*}
    \indunit:B(\ttt)\to \prd{x:\unit}B(x)
  \end{equation*}
  for every type family $B$ over $\unit$, satisfying the judgmental equality $\indunit(b,\ttt)\jdeq b$ for every $b:B(\ttt)$ by the computation rule. Therefore, we obtain the homotopy
  \begin{equation*}
    \lam{b}\refl{b}:\evpt\circ\indunit \htpy\idfunc,
  \end{equation*}
  and we conclude that the unit type satisfies singleton induction. 
\end{eg}

\begin{thm}\label{thm:contractible}
Let $A$ be a type. The following are equivalent:\index{is contractible!satisfies singleton induction}\index{singleton induction!is contractible}\index{contractible type!satisfies singleton induction}
\begin{enumerate}
\item The type $A$ is contractible.
\item The type $A$ comes equipped with an element $a:A$, and satisfies singleton induction.
\end{enumerate}
\end{thm}

\begin{proof}
Suppose $A$ is contractible with center of contraction $a$ and contraction $C$. 
First we observe that, without loss of generality, we may assume that $C$ comes equipped with an identification $p:C(a)=\refl{a}$.
To see this, note that we can always define a new contraction $C'$ by
\begin{equation*}
C'(x)\defeq\ct{C(a)^{-1}}{C(x)},
\end{equation*}
which satisfies the requirement by the left inverse law, constructed in \cref{defn:id_invlaw}.

To show that $A$ satisfies singleton induction let $B$ be a type family over $A$, and suppose we have $b:B(a)$. Our goal is to define
\begin{equation*}
  \indsing_a(b):\prd{x:A}B(x).
\end{equation*}
Let $x:A$. Since we have an identification $C(x):a=x$, and an element $b$ in $B(a)$, we may transport $b$ along the path $C(x)$ to obtain
\begin{equation*}
  \indsing_a(b,x)\defeq \tr_B(C(x),b):B(x).
\end{equation*}
Therefore, the function $\indsing_a(b)$ is defined to be the dependent function $\lam{x}\tr_B(C(x),b)$. Now we have to show that $\indsing_a(b,a)=b$. Then we have the identifications
\begin{equation*}
\begin{tikzcd}
\tr_B(C(a),b) \arrow[r,equals,"\ap{\lam{\omega}\tr_B(\omega,b)}{p}"] &[4em] \tr_B(\refl{a},b) \arrow[r,equals,"\refl{b}"] & b.
\end{tikzcd}
\end{equation*}
This shows that the computation rule is satisfied, which completes the proof that $A$ satisfies singleton induction.

For the converse, suppose that $a:A$ and that $A$ satisfies singleton induction. Our goal is to show that $A$ is contractible. For the center of contraction we take the element $a:A$. By singleton induction applied to $B(x)\defeq a=x$ we have the map 
\begin{equation*}
\indsing_{a} : a=a \to \prd{x:A}a=x.
\end{equation*}
Therefore $\indsing_{a}(\refl{a})$ is a contraction.
\end{proof}
\index{singleton induction|)}
\index{induction principle!singleton induction|)}

\subsection{Contractible maps}

\index{contractible map|(}
\begin{defn}
  Let $f:A\to B$ be a function, and let $b:B$. The \define{fiber}\index{fiber|textbf}\index{homotopy fiber|see {fiber}} of $f$ at $b$ is defined to be the type\index{fib f b@{$\fib{f}{b}$}|textbf}
  \begin{equation*}
    \fib{f}{b}\defeq\sm{a:A}f(a)=b.
  \end{equation*}
\end{defn}

In other words, the fiber of $f$ at $b$ is the type of $a:A$ that get mapped by $f$ to $b$.
One may think of the fiber as a type theoretic version of the preimage\index{pre-image|see {fiber}} of a point.

\index{fiber!characterization of identity type|(}
\index{characterization of identity type!of the fiber of a map|(}
\index{identity type!of a fiber|(}
It will be useful to have a characterization of the identity type of a fiber. In order to identify any $(x,p)$ and $(x',p')$ in $\fib{f}{y}$, we may first construct an identification $\alpha:x=x'$. Then we obtain a triangle
\begin{equation*}
  \begin{tikzcd}[column sep=tiny]
    f(x) \arrow[dr,equals,swap,"p"] \arrow[rr,equals,"{\ap{f}{\alpha}}"] & & f(x') \arrow[dl,equals,"{p'}"] \\
    \phantom{f(x')} & y,
  \end{tikzcd}
\end{equation*}
so we may consider the type of identifications $\beta:p=\ct{\ap{f}{\alpha}}{p'}$. We will show that the type of all identifications $(x,p)=(x',p')$ is equivalent to the type of such pairs $(\alpha,\beta)$. 

\begin{defn}
  Let $f:A \to B$ be a map, and let $(x,p),(x',p'):\fib{f}{y}$ for some $y:B$.
  Then we define\index{Eq-fib@{$\Eqfib$}|textbf}\index{fiber!Eq-fib@{$\Eqfib$}|textbf}\index{observational equality!fiber}
  \begin{equation*}
    \Eqfib_f((x,p),(x',p'))\defeq \sm{\alpha:x=x'}p=\ct{\ap{f}{\alpha}}{p'}
  \end{equation*}
  The relation $\Eqfib_f:\fib{f}{y}\to\fib{f}{y}\to\UU$ is a reflexive relation, since we have
  \begin{equation*}
    \lam{(x,p)}(\refl{x},\refl{p}):\prd{(x,p):\fib{f}{y}}\Eqfib_f((x,p),(x,p)).
  \end{equation*}
\end{defn}

\begin{prp}
  Consider a map $f:A\to B$ and let $y:B$. The canonical map
  \begin{equation*}
    ((x,p)=(x',p'))\to\Eqfib_f((x,p),(x',p'))
  \end{equation*}
  induced by the reflexivity of $\Eqfib_f$ is an equivalence for any $(x,p),(x',p'):\fib{f}{y}$.
\end{prp}

\begin{proof}
  The converse map
  \begin{equation*}
    \Eqfib_f((x,p),(x',p'))\to ((x,p)=(x',p'))
  \end{equation*}
  is easily defined by $\Sigma$-induction, and then path induction twice. The homotopies witnessing that this converse map is indeed a right inverse as well as a left inverse are similarly constructed by induction.
\end{proof}
\index{fiber!characterization of identity type|)}
\index{characterization of identity type!of the fiber of a map|)}
\index{identity type!of a fiber|)}

Now we define at the notion of contractible map.

\begin{defn}
We say that a function $f:A\to B$ is \define{contractible}\index{contractible map|textbf} if it comes equipped with an element of type\index{is-contr(f)@{$\iscontr(f)$}|see {contractible map}}\index{is a contractible map|textbf}
\begin{equation*}
\iscontr(f)\defeq\prd{b:B}\iscontr(\fib{f}{b}).
\end{equation*}
\end{defn}

\begin{thm}\label{thm:equiv_contr}
Any contractible map is an equivalence.\index{contractible map!is an equivalence}\index{is an equivalence!contractible map}
\end{thm}

\begin{proof}
Let $f:A\to B$ be a contractible map. Using the center of contraction of each $\fib{f}{y}$, we obtain the dependent function
\begin{align*}
\lam{y}\pairr{g(y),G(y)}:\prd{y:B}\fib{f}{y}.
\end{align*}
Thus, we get map $g:B\to A$, and a homotopy $G:\prd{y:B} f(g(y))=y$. In other words, we get a section of $f$.

It remains to construct a retraction of $f$. Taking $g$ as our retraction, we have to show that $\prd{x:A} g(f(x))=x$. Note that we get an identification $p:f(g(f(x)))=f(x)$ since $g$ is a section of $f$. Therefore, it follows that $(g(f(x)),p):\fib{f}{f(x)}$. Moreover, since $\fib{f}{f(x)}$ is contractible we get an identification $q:\pairr{g(f(x)),p}=\pairr{x,\refl{f(x)}}$. The base path $\ap{\proj 1}{q}$ of this identification is an identification of type $g(f(x))=x$, as desired.
\end{proof}

\subsection{Equivalences are contractible maps}\label{sec:is-contr-map-is-equiv}

In \cref{thm:contr_equiv} we will show the converse to \cref{thm:equiv_contr}, i.e., we will show that any equivalence is a contractible map. We will do this in two steps.

First we introduce a new notion of \emph{coherently invertible map}, for which we can easily show that such maps have contractible fibers. Then we show that any equivalence is a coherently invertible map.

  Recall that an invertible map is a map $f:A\to B$ equipped with $g:B\to A$ and homotopies
  \begin{equation*}
    G : f\circ g \htpy \idfunc\qquad\text{and}\qquad H:g\circ f\htpy \idfunc.
  \end{equation*}
  Then we observe that both $G \cdot f$ and $f \cdot H$ are homotopies of the same type
  \begin{equation*}
    f\circ g\circ f \htpy f.
  \end{equation*}
  A coherently invertible map is an invertible map for which there is a further homotopy $G \cdot f\htpy f\cdot H$.

  \begin{defn}
    Consider a map $f:A\to B$. We say that $f$ is \define{coherently invertible}\index{coherently invertible map|textbf} if it comes equipped with
    \begin{align*}
      g & : B \to A \\
      G & : f \circ g \htpy \idfunc \\
      H & : g \circ f \htpy \idfunc \\
      K & : G \cdot f \htpy f \cdot H.
    \end{align*}
    We will write $\iscohinvertible(f)$\index{is-coh-invertible(f)@{$\iscohinvertible(f)$}|textbf} for the type of quadruples $(g,G,H,K)$.
  \end{defn}

  Although we will encounter the notion of coherently invertible map on some further occasions, the following proposition is our main motivation for considering it.

  \begin{prp}\label{lem:contr-inv}
    Any coherently invertible map has contractible fibers.\index{coherently invertible map!is a contractible map}
  \end{prp}

  \begin{proof}
    Consider a map $f:A\to B$ equipped with
    \begin{align*}
      g & : B \to A \\
      G & : f \circ g \htpy \idfunc \\
      H & : g \circ f \htpy \idfunc \\
      K & : G \cdot f \htpy f \cdot H,
    \end{align*}
    and let $y:B$. Our goal is to show that $\fib{f}{y}$ is contractible. For the center of contraction we take $(g(y),G(y))$. In order to construct a contraction, it suffices to construct a dependent function of type
    \begin{equation*}
      \prd{x:A}\prd{p:f(x)=y}\Eqfib_f((g(y),G(y)),(x,p)).
    \end{equation*}
    By path induction on $p:f(x)=y$ it suffices to construct a dependent function of type
    \begin{equation*}
      \prd{x:A}\Eqfib_f((g(f(x)),G(f(x))),(x,\refl{f(x)})).
    \end{equation*}
    By definition of $\Eqfib_f$, we have to construct for each $x:A$ an identification $\alpha:g(f(x))=x$ equipped with a further identification
    \begin{equation*}
      G(f(x))=\ct{\ap{f}{\alpha}}{\refl{f(x)}}.
    \end{equation*}
    Such a dependent function is constructed as $\lam{x}(H(x),K'(x))$, where the homotopy $H:g\circ f\htpy \idfunc$ is given by assumption, and the homotopy
    \begin{align*}
      K' & : \prd{x:A}G(f(x))=\ct{\ap{f}{H(x)}}{\refl{f(x)}}
    \end{align*}
    is defined as
    \begin{equation*}
      K'\defeq \ct{K}{\rightunithtpy(f\cdot H)^{-1}}.\qedhere
    \end{equation*}
  \end{proof}

  Our next goal is to show that for any map $f:A\to B$ equipped with
  \begin{equation*}
    g:B\to A,\qquad G:f\circ g \htpy \idfunc,\qquad\text{and}\qquad H:g\circ f\htpy \idfunc,
  \end{equation*}
  we can improve the homotopy $G$ to a new homotopy $G':f\circ g\htpy \idfunc$ for which there is a further homotopy
  \begin{equation*}
    f\cdot H\htpy G'\cdot f.
  \end{equation*}
  Note that this situation is analogous to the situation in the proof of \cref{thm:contractible}, where we improved the contraction $C$ so that it satisfied $C(c)=\refl{}$. The extra coherence $f\cdot H\htpy G'\cdot f$ is then used in the proof that the fibers of an equivalence are contractible.

\begin{defn}\label{defn:htpy_nat}\index{homotopy!naturality|textbf}
Let $f,g:A\to B$ be functions, and consider $H:f\htpy g$ and $p:x=y$ in $A$. We define the identification\index{nat-nat@{$\nathtpy$}|textbf}\index{homotopy!nat-htpy@{$\nathtpy$}|textbf}
\begin{equation*}
\nathtpy(H,p) \defeq  \ct{\ap{f}{p}}{H(y)}=\ct{H(x)}{\ap{g}{p}}
\end{equation*}
witnessing that the square
\begin{equation*}
\begin{tikzcd}
f(x) \arrow[r,equals,"H(x)"] \arrow[d,equals,swap,"\ap{f}{p}"] & g(x) \arrow[d,equals,"\ap{g}{p}"] \\
f(y) \arrow[r,equals,swap,"H(y)"] & g(y)
\end{tikzcd}
\end{equation*}
commutes. This square is also called the \define{naturality square}\index{naturality square of homotopies|textbf} of the homotopy $H$ at $p$.
\end{defn}

\begin{constr}
  By path induction on $p$ it suffices to construct an identification
  \begin{equation*}
    \ct{\ap{f}{\refl{x}}}{H(x)}=\ct{H(x)}{\ap{g}{\refl{x}}}
  \end{equation*}
  since $\ap{f}{\refl{x}}\jdeq \refl{f(x)}$ and $\ap{g}{\refl{x}}\jdeq\refl{g(x)}$, and since $\ct{\refl{f(x)}}{H(x)}\jdeq H(x)$, we see that the path $\rightunit(H(x))^{-1}$ is of the asserted type.
\end{constr}

\begin{defn}\label{defn:retraction_swap}
Consider $f:A\to A$ and $H: f\htpy \idfunc[A]$. We construct an identification $H(f(x))=\ap{f}{H(x)}$, for any $x:A$.
\end{defn}

\begin{constr}
By the naturality of homotopies with respect to identifications the square
\begin{equation*}
\begin{tikzcd}[column sep=large]
ff(x) \arrow[d,swap,equals,"\ap{f}{H(x)}"] \arrow[r,equals,"H(f(x))"] & f(x) \arrow[d,equals,"H(x)"] \\
f(x) \arrow[r,swap,equals,"H(x)"] & x
\end{tikzcd}
\end{equation*}
commutes. This gives the desired identification $H(f(x))=\ap{f}{H(x)}$.
\end{constr}

\begin{lem}\label{lem:coherently-invertible}
  Let $f:A\to B$ be a map equipped with an inverse, i.e., consider
  \begin{align*}
    g & : B \to A \\
    G & : f \circ g \htpy \idfunc \\
    H & : g \circ f \htpy \idfunc.
  \end{align*}
  Then there is a homotopy $G':f\circ g\htpy \idfunc$ equipped with a further homotopy
  \begin{equation*}
    K : f\cdot H \htpy G'\cdot f.
  \end{equation*}
  Thus we obtain a map $\hasinverse(f)\to\iscohinvertible(f)$.\index{has-inverse(f)@{$\hasinverse(f)$}!has-inverse(f) to is-coh-invertible(f)@{$\hasinverse(f)\to\iscohinvertible(f)$}}
\end{lem}

\begin{proof}
  For each $y:B$, we construct the identification $G'(y)$ as the concatenation
  \begin{equation*}
    \begin{tikzcd}
      fg(y) \arrow[r,equals,"{G(fg(y))}^{-1}"] &[2.5em] fgfg(y) \arrow[r,equals,"\ap{f}{H(g(y))}"] &[2.5em] fg(y) \arrow[r,equals,"G(y)"] & y.
\end{tikzcd}
  \end{equation*}
  In order to construct a homotopy $f\cdot H \htpy G'\cdot f$, it suffices to show that the square
  \begin{equation*}
    \begin{tikzcd}[column sep=8em]
      fgfgf(x) \arrow[r,equals,"{G(fgf(x))}"] \arrow[d,equals,swap,"\ap{f}{H(gf(x))}"] & fgf(x) \arrow[d,equals,"\ap{f}{H(x)}"] \\
      fgf(x) \arrow[r,equals,swap,"G(f(x))"] & f(x)
    \end{tikzcd}
  \end{equation*}
  commutes for every $x:A$.
  Recall from \cref{defn:retraction_swap} that we have $H(gf(x))=\ap{gf}{H(x)}$. Using this identification, we see that it suffices to show that the square
  \begin{equation*}
    \begin{tikzcd}[column sep=8em]
      fgfgf(x) \arrow[r,equals,"(G\cdot f)(gf(x))"] \arrow[d,equals,swap,"\ap{fgf}{H(x)}"] & fgf(x) \arrow[d,equals,"\ap{f}{H(x)}"] \\
      fgf(x) \arrow[r,equals,swap,"(G\cdot f)(x)"] & f(x)
    \end{tikzcd}
  \end{equation*}
  commutes. Now we observe that this is just a naturality square the homotopy $G\cdot f:fgf\htpy f$, which commutes by \cref{defn:htpy_nat}.
\end{proof}

Now we put the pieces together to conclude that any equivalence has contractible fibers.

\begin{thm}\label{thm:contr_equiv}
Any equivalence is a contractible map.\index{equivalence!is a contractible map}\index{is a contractible map!equivalence}\index{is contractible!fiber of an equivalence}
\end{thm}

\begin{proof}
  We have seen in \cref{lem:contr-inv} that any coherently invertible map is a contractible map. Moreover, any equivalence has the structure of an invertible map by \cref{lem:inv_equiv}, and any invertible map is coherently invertible by \cref{lem:coherently-invertible}.
\end{proof}

The following corollary is very similar to \cref{thm:total_path}, which asserts that the type $\sm{x:A}a=x$ is contractible. However, we haven't yet established that the equivalence $(a=x)\simeq (x=a)$ induces an equivalence on total spaces. However, using the fact that equivalences are contractible maps we can give a direct proof.

\begin{cor}\label{cor:contr_path}
Let $A$ be a type, and let $a:A$. Then the type\index{is contractible!total space of opposite identity type}
\begin{equation*}
\sm{x:A}x=a
\end{equation*}
is contractible.
\end{cor}

\begin{proof}
By \cref{thm:id_equiv}, the identity function is an equivalence. Therefore, the fibers of the identity function are contractible by \cref{thm:contr_equiv}. Note that $\sm{x:A}x=a$ is exactly the fiber of $\idfunc[A]$ at $a:A$.
\end{proof}
\index{contractible map|)}

\begin{exercises}
  \exitem \label{ex:prop_contr}Show that if $A$ is contractible, then for any $x,y:A$ the identity type $x=y$ is also contractible.\index{contractible type!characterization of identity type}\index{is contractible!identity type of contractible type}\index{identity type!of a contractible type}\index{characterization of identity type!of a contractible type}
  \exitem \label{ex:contr_retr}Suppose that $A$ is a retract of $B$. Show that\index{contractible type!closed under retracts}
  \begin{equation*}
    \iscontr(B)\to\iscontr(A).
  \end{equation*}
  \exitem \label{ex:contr_equiv}
  \begin{subexenum}
  \item Show that for any type $A$, the map $\const_\ttt : A\to \unit$ is an equivalence if and only if $A$ is contractible.\index{contractible type!is equivalent to 1@{is equivalent to $\unit$}}
  \item Apply \cref{ex:3_for_2} to show that for any map $f:A\to B$, if any two of the three assertions\index{contractible type!3-for-2 property}\index{3-for-2 property!of contractible types}
    \begin{enumerate}
    \item $A$ is contractible
    \item $B$ is contractible
    \item $f$ is an equivalence
    \end{enumerate}
    hold, then so does the third.
  \end{subexenum}
  \exitem \label{ex:is-not-contractible-Fin}Show that $\Fin{k}$ is not contractible for all $k\neq 1$. 
  \exitem \label{ex:is-contr-prod}Show that for any two types $A$ and $B$, the following are equivalent:
  \index{contractible type!closed under cartesian products}
  \index{is contractible!factor of contractible cartesian product}
  \begin{enumerate}
  \item Both $A$ and $B$ are contractible.
  \item The type $A\times B$ is contractible.
  \end{enumerate}
  \exitem \label{ex:contr_in_sigma} Let $A$ be a contractible type with center of contraction $a:A$. Furthermore, let $B$ be a type family over $A$. Show that the map
  \begin{equation*}
    y\mapsto\pairr{a,y}:B(a)\to\sm{x:A}B(x)
  \end{equation*}
  is an equivalence.\index{left unit law!of Sigma-types@{of $\Sigma$-types}}\index{dependent pair type!left unit law}
  \exitem \label{ex:proj_fiber}Let $B$ be a family of types over $A$, and consider the projection map 
    \begin{equation*}
      \proj 1 : \big(\sm{x:A}B(x)\big)\to A.
    \end{equation*}
  \begin{subexenum}
  \item Show that for any $a:A$, the map
    \begin{equation*}
      \lam{((x,y),p)} \tr_B(p,y) : \fib{\proj 1}{a} \to B(a),
    \end{equation*}
    is an equivalence.\index{type family!fibers of projection map}
  \item Show that the following are equivalent:%
    \index{pr 1@{$\proj 1$}!of contractible family is an equivalence}%
    \index{is an equivalence!pr 1 of contractible family@{$\proj 1$ of contractible family}}
    \begin{enumerate}
    \item The projection map $\proj 1$ is an equivalence.
    \item The type $B(x)$ is contractible for each $x:A$.
    \end{enumerate}
  \item Consider a dependent function $b:\prd{x:A}B(x)$. Show that the following are equivalent:
    \begin{enumerate}
    \item The map
    \begin{equation*}
      \lam{x}(x,b(x)) : A \to \sm{x:A}B(x)
    \end{equation*}
    is an equivalence.
    \item The type $B(x)$ is contractible for each $x:A$.
    \end{enumerate}
  \end{subexenum}
  \exitem \label{ex:fib_replacement}Construct for any map $f:A\to B$ an equivalence $e:\eqv{A}{\sm{y:B}\fib{f}{y}}$ and a homotopy $H:f\htpy \proj 1\circ e$ witnessing that the triangle
  \begin{equation*}
    \begin{tikzcd}[column sep=0em]
      A \arrow[rr,"e"] \arrow[dr,swap,"f"] & & \sm{y:B}\fib{f}{y} \arrow[dl,"\proj 1"] \\
      \phantom{\sm{y:B}\fib{f}{y}} & B
    \end{tikzcd}
  \end{equation*}
  commutes. The projection $\proj 1 : (\sm{y:B}\fib{f}{y})\to B$ is sometimes also called the \define{fibrant replacement}\index{fibrant replacement|textbf} of $f$, because first projection maps are fibrations in the homotopy interpretation of type theory.
\end{exercises}
\index{contractible type|)}

%%% Local Variables:
%%% mode: latex
%%% TeX-master: "hott-intro"
%%% End:

% !TEX root = hott_intro.tex

\section{The fundamental theorem of identity types}\label{chap:fundamental}
\sectionmark{The fundamental theorem}

\index{fundamental theorem of identity types|(}
\index{characterization of identity type!fundamental theorem of identity types|(}
For many types it is useful to have a characterization of their identity types. For example, we have used a characterization of the identity types of the fibers of a map in order to conclude that any equivalence is a contractible map. The fundamental theorem of identity types is our main tool to carry out such characterizations, and with the fundamental theorem it becomes a routine task to characterize an identity type whenever that is of interest. We note that the fundamental theorem also appears as Theorem 5.8.4 in \cite{hottbook}.

In our first application of the fundamental theorem of identity types we show that any equivalence is an embedding. Embeddings are maps that induce equivalences on identity types, i.e., they are the homotopical analogue of injective maps. In our second application we characterize the identity types of coproducts.

Throughout this book we will encounter many more occasions to characterize identity types. For example, we will show in \cref{thm:eq_nat} that the identity type of the natural numbers is equivalent to its observational equality, and we will show in \cref{thm:eq-circle} that the loop space of the circle is equivalent to $\Z$.

In order to prove the fundamental theorem of identity types, we first prove the basic fact that a family of maps is a family of equivalences if and only if it induces an equivalence on total spaces. 

\subsection{Families of equivalences}

\index{family of equivalences|(}
\begin{defn}
Consider a family of maps
\begin{equation*}
f : \prd{x:A}B(x)\to C(x).
\end{equation*}
We define the map\index{tot(f)@{$\tot{f}$}}
\begin{equation*}
\tot{f}:\sm{x:A}B(x)\to\sm{x:A}C(x)
\end{equation*}
by $\lam{(x,y)}(x,f(x,y))$.
\end{defn}

\begin{lem}\label{lem:fib_total}
  For any family of maps $f:\prd{x:A}B(x)\to C(x)$ and any $t:\sm{x:A}C(x)$,
  there is an equivalence\index{fiber!of tot(f)@{of $\tot{f}$}}\index{tot(f)@{$\tot{f}$}!fiber}
  \begin{equation*}
    \eqv{\fib{\tot{f}}{t}}{\fib{f(\proj 1(t))}{\proj 2(t)}}.
  \end{equation*}
\end{lem}

\begin{proof}
  We first define
  \begin{equation*}
    \varphi : \prd{t:\sm{x:A}C(x)} \fib{\tot{f}}{t}\to\fib{f(\proj 1(t))}{\proj 2(t)}
  \end{equation*}
  by pattern matching by
  \begin{equation*}
    \varphi((x,f(x,y)),((x,y),\refl{}))\defeq(y,\refl{}).
  \end{equation*}

  For the proof that $\varphi(t)$ is an equivalence, for each $t:\sm{x:A}C(x)$, we construct a map
  \begin{equation*}
    \psi(t) : \fib{f(\proj 1(t))}{\proj 2(t)}\to\fib{\tot{f}}{t}
  \end{equation*}
  equipped with homotopies $G(t):\varphi(t)\circ\psi(t)\htpy\idfunc$ and $H(t):\psi(t)\circ\varphi(t)\htpy\idfunc$. Each of these definitions is given by pattern matching, as follows:
  \begin{align*}
    \psi((x,f(x,y)),(y,\refl{})) & \defeq ((x,y),\refl{}) \\
    G((x,f(x,y)),(y,\refl{})) & \defeq \refl{} \\
    H((x,f(x,y)),((x,y),\refl{})) & \defeq \refl{}.\qedhere
  \end{align*}
\end{proof}

\begin{thm}\label{thm:fib_equiv}
  Let $f:\prd{x:A}B(x)\to C(x)$ be a family of maps. The following are equivalent:
  \index{is an equivalence!total(f) of family of equivalences@{$\tot{f}$ of family of equivalences}}
  \index{tot(f)@{$\tot{f}$}!of family of equivalences is an equivalence}\index{is family of equivalences!if total(f) is an equivalence@{iff $\tot{f}$ is an equivalence}}
\begin{enumerate}
\item For each $x:A$, the map $f(x)$ is an equivalence. In this case we say that $f$ is a \define{family of equivalences}\index{family of equivalences|textbf}\index{equivalence!family of equivalences|textbf}.
\item The map $\tot{f}:\sm{x:A}B(x)\to\sm{x:A}C(x)$ is an equivalence.
\end{enumerate}
\end{thm}

\begin{proof}
By \cref{thm:equiv_contr,thm:contr_equiv} it suffices to show that $f(x)$ is a contractible map for each $x:A$, if and only if $\tot{f}$ is a contractible map. Thus, we will show that $\fib{f(x)}{c}$ is contractible if and only if $\fib{\tot{f}}{x,c}$ is contractible, for each $x:A$ and $c:C(x)$. However, by \cref{lem:fib_total} these types are equivalent, so the result follows by \cref{ex:contr_equiv}.
\end{proof}

Now consider the situation where we have a map $f:A\to B$, and a family $C$ over $B$. Then we have the map
\begin{equation*}
  \lam{(x,z)}(f(x),z):\sm{x:A}C(f(x))\to\sm{y:B}C(y).
\end{equation*}
We claim that this map is an equivalence when $f$ is an equivalence. The technique to prove this claim is the same as the technique we used in \cref{thm:fib_equiv}: first we note that the fibers are equivalent to the fibers of $f$, and then we use the fact that a map is an equivalence if and only if its fibers are contractible to finish the proof.

The converse of the following lemma does not hold. Why not?

\begin{lem}\label{lem:total-equiv-base-equiv}
  Consider a map $f:A\to B$, and let $C$ be a type family over $B$. If $f$ is an equivalence, then the map
  \begin{equation*}
    \sigma_f(C) \defeq\lam{(x,z)}(f(x),z):\sm{x:A}C(f(x))\to\sm{y:B}C(y)
  \end{equation*}
  is an equivalence.
\end{lem}

\begin{proof}
  We claim that for each $t:\sm{y:B}C(y)$ there is an equivalence
  \begin{equation*}
    \fib{\sigma_f(C)}{t}\simeq \fib{f}{\proj 1(t)}.
  \end{equation*}
  We obtain such an equivalence by constructing the following functions and homotopies:
  \begin{align*}
    \varphi(t) & : \fib{\sigma_f(C)}{t}\to\fib{f}{\proj 1 (t)} & \varphi((f(x),z),((x,z),\refl{})) & \defeq (x,\refl{}) \\
    \psi(t) & : \fib{f}{\proj 1(t)} \to\fib{\sigma_f(C)}{t} & \psi((f(x),z),(x,\refl{})) & \defeq ((x,z),\refl{}) \\
    G(t) & : \varphi(t)\circ\psi(t)\htpy\idfunc & G((f(x),z),(x,\refl{})) & \defeq \refl{} \\
    H(t) & : \psi(t)\circ\varphi(t)\htpy\idfunc & H((f(x),z),((x,z),\refl{})) & \defeq \refl{}.
  \end{align*}
  Now the claim follows, since we see that $\varphi$ is a contractible map if and only if $f$ is a contractible map.
\end{proof}

Now we use \cref{lem:total-equiv-base-equiv} to obtain a generalization of \cref{thm:fib_equiv}.

\begin{defn}\label{defn:toto}
  Consider a map $f:A\to B$ and a family of maps
  \begin{equation*}
    g:\prd{x:A}C(x)\to D(f(x)),
  \end{equation*}
  where $C$ is a type family over $A$, and $D$ is a type family over $B$. In this situation we also say that $g$ is a \define{family of maps over $f$}. Then we define\index{totf(g)@{$\tot[f]{g}$}}
  \begin{equation*}
    \tot[f]{g}:\sm{x:A}C(x)\to\sm{y:B}D(y)
  \end{equation*}
  by $\tot[f]{g}(x,z)\defeq (f(x),g(x,z))$.
\end{defn}

\begin{thm}\label{thm:equiv-toto}
  Suppose that $g$ is a family of maps over $f$ as in \cref{defn:toto}, and suppose that $f$ is an equivalence. Then the following are equivalent:
  \begin{enumerate}
  \item The family of maps $g$ over $f$ is a family of equivalences.
  \item The map $\tot[f]{g}$ is an equivalence.
  \end{enumerate}
\end{thm}

\begin{proof}
  Note that we have a commuting triangle
  \begin{equation*}
    \begin{tikzcd}[column sep=0]
      \sm{x:A}C(x) \arrow[rr,"{\tot[f]{g}}"] \arrow[dr,swap,"\tot{g}"]& & \sm{y:B}D(y) \\
      & \sm{x:A}D(f(x)) \arrow[ur,swap,"{\lam{(x,z)}(f(x),z)}"]
    \end{tikzcd}
  \end{equation*}
  By the assumption that $f$ is an equivalence, it follows that the map
  \begin{equation*}
    \sm{x:A}D(f(x))\to \sm{y:B}D(y)
  \end{equation*}
  is an equivalence. Therefore it follows that $\tot[f]{g}$ is an equivalence if and only if $\tot{g}$ is an equivalence. Now the claim follows, since $\tot{g}$ is an equivalence if and only if $g$ if a family of equivalences.
\end{proof}
\index{family of equivalences|)}

\subsection{The fundamental theorem}

\index{identity system|(}

The fundamental theorem of identity types (\cref{thm:id_fundamental}) is a general theorem that can be used to characterize the identity type of a given type. It describes necessary and sufficient conditions on a type family $B$ over a type $A$ equipped with a point $a:A$ to obtain an equivalence $(a=x)\simeq B(x)$ for each $x:A$.

One of those conditions is that the family $B$ satisfies an induction principle that is similar to the identification elimination principle. Such families are called \emph{identity systems}, which we will introduce now.

\begin{defn}\label{defn:identity-system}
  Let $A$ be a type equipped with a term $a:A$. A \define{(unary) identity system}\index{identity system|textbf}\index{unary identity system|see {identity system}} on $A$ at $a$ consists of a type family $B$ over $A$ equipped with $b:B(a)$, such that for any family of types $P(x,y)$ indexed by $x:A$ and $y:B(x)$,
  the function
  \begin{equation*}
    h\mapsto h(a,b):\Big(\prd{x:A}\prd{y:B(x)}P(x,y)\Big)\to P(a,b)
  \end{equation*}
  has a section.  
\end{defn}

In other words, if $B$ is an identity system on $A$ at $a$ and $P$ is a family of types indexed by $x:A$ and $y:B(x)$, then there is for each $p:P(a,b)$ a dependent function
\begin{equation*}
  f:\prd{x:A}\prd{y:B(x)}P(x,y)
\end{equation*}
such that $f(a,b)=p$. This is of course a variant of identification elimination, where the computation rule is given by an identification rather than as a judgmental equality.

We will state the fundamental theorem of identity types in a way that makes it maximally applicable. The fundamental theorem starts off with assuming a type $A$ equipped with a base point $a:A$, and a type family $B$ over $A$ equipped with a point $b:B(a)$. Furthermore it assumes an arbitrary family of maps
\begin{equation*}
  f:\prd{x:A}(a=x)\to B(x)
\end{equation*}
equipped with an identification $f(a,\refl{a})=b$. The theorem asserts conditions that are equivalent to $f$ being a family of equivalences.

In the setup of the fundamental theorem of identity types we can always construct the family of maps
\begin{equation*}
  f\defeq\pathind_a(b):\prd{x:A}(a=x)\to B(x)
\end{equation*}
for which the judgmental equality $f(a,\refl{a})\jdeq b$ holds. So you may wonder why we choose to formulate the fundamental theorem of identity types using a general family of maps $f$. The reason is that it is somewhat common to apply the fundamental theorem of identity types in order to conclude that $f$ is a family of equivalences, even when $f$ is not by definition the canonical family of maps, and we want to be free to do so.

The most important implication in the fundamental theorem is that (ii) implies (i). Occasionally we will also use the third equivalent statement.

\begin{thm}[The fundamental theorem of identity types]\label{thm:id_fundamental}
Let $A$ be a type with $a:A$, and let $B$ be a type family over $A$ equipped with a point $b:B(a)$. Furthermore, consider a family of maps\index{fundamental theorem of identity types}
\begin{equation*}
  f:\prd{x:A}(a=x)\to B(x)
\end{equation*}
equipped with an identification $f(a,\refl{a})=b$. Then the following are equivalent:
\begin{enumerate}
\item The family of maps $f$ is a family of equivalences.
\item The total space\index{is contractible!total space of an identity system}
\begin{equation*}
\sm{x:A}B(x)
\end{equation*}
is contractible.
\item The family $B$ equipped with $b:B(a)$ is an identity system.
\end{enumerate}
In particular, we see that for any $b:B(a)$, the canonical family of maps
\begin{equation*}
\pathind_a(b):\prd{x:A} (a=x)\to B(x)
\end{equation*}
is a family of equivalences if and only if $\sm{x:A}B(x)$ is contractible.
\end{thm}

\begin{proof}
  First we show that (i) and (ii) are equivalent.
  By \cref{thm:fib_equiv} it follows that the family of maps $f$ is a family of equivalences if and only if it induces an equivalence
  \begin{equation*}
    \eqv{\Big(\sm{x:A}a=x\Big)}{\Big(\sm{x:A}B(x)\Big)}
  \end{equation*}
  on total spaces. We have that $\sm{x:A}a=x$ is contractible, so it follows by \cref{ex:contr_equiv} that $\tot{f}$ is an equivalence if and only if $\sm{x:A}B(x)$ is contractible.

  Now we show that (ii) and (iii) are equivalent. Note that we have the following commuting triangle
  \begin{equation*}
    \begin{tikzcd}[column sep=0]
      \prd{t:\sm{x:A}B(x)}P(t) \arrow[rr,"\evpair"] \arrow[dr,swap,"{\evpt(a,b)}"] & & \prd{x:A}\prd{y:B(x)}P(x,y) \arrow[dl,"{\lam{h}h(a,b)}"] \\
      \phantom{\prd{x:A}\prd{y:B(x)}P(x,y)} & P(a,b)
    \end{tikzcd}
  \end{equation*}
  In this diagram the top map has a section. Therefore it follows by \cref{ex:3_for_2} that the left map has a section if and only if the right map has a section. Recall from \cref{defn:singleton-induction} that the type $\sm{x:A}B(x)$ satisfies singleton induction if and only if the left map in the triangle has a section for each $P$. Therefore we conclude our proof with \cref{thm:contractible}, which shows that the type $\sm{x:A}B(x)$ satisfies singleton induction if and only if it is contractible.
\end{proof}
\index{identity system|)}

\subsection{Equality on the natural numbers}
\index{natural numbers!observational equality|(}
\index{Eq N@{$\EqN$}|(}

As a first application of the fundamental theorem of identity types, we characterize the identity type of the natural numbers. We will use the observational equality $\EqN$ on $\N$. Recall from \cref{defn:obs_nat} that $\EqN$ is defined by
\begin{align*}
  \EqN(\zeroN,\zeroN) & \defeq \unit & \EqN(\zeroN,n+1) & \defeq \emptyt \\
  \EqN(m+1,\zeroN) & \defeq \emptyt & \EqN(m+1,n+1) & \defeq \EqN(m,n).
\end{align*}
This relation is an equivalence relation. In particular, the reflexivity term $\reflEqN(m):\EqN(m,m)$ is defined inductively by
\begin{align*}
  \reflEqN(\zeroN) & \defeq \ttt \\
  \reflEqN(m+1) & \defeq \reflEqN(m).
\end{align*}
Using the reflexivity term, we obtain a canonical map
\begin{equation*}
  (m=n)\to \EqN(m,n)
\end{equation*}
for every $m,n:\N$.

\begin{thm}\label{thm:eq_nat}
  For each $m,n:\N$, the canonical map\index{natural numbers!identity type}\index{natural numbers!characterization of identity type}\index{characterization of identity type!of N@{of $\N$}}\index{identity type!of the natural numbers}
  \begin{equation*}
    (m=n)\to \EqN(m,n)
  \end{equation*}
  is an equivalence.
\end{thm}

\begin{proof}
  By \cref{thm:id_fundamental} it suffices to show that the type
  \begin{equation*}
    \sm{n:\N}\EqN(m,n)
  \end{equation*}
  is contractible, for each $m:\N$. The center of contraction is defined to be $(m,\reflEqN(m))$.

  The contraction
  \begin{equation*}
    \gamma(m):\prd{n:\N}\prd{e:\EqN(m,n)}(m,\reflEqN(m))=(n,e)
  \end{equation*}
  is defined for each $m$ by induction on $m,n:\N$. In the base case we define
  \begin{equation*}
    \gamma(\zeroN,\zeroN,\ttt)\defeq \refl{}.
  \end{equation*}
  If one of $m$ and $n$ is zero and the other is a successor, then the type $\EqN(m,n)$ is empty, so the desired path can be obtained via the induction principle of the empty type.

  The inductive step remains, in which we have to define the identification
  \begin{equation*}
    \gamma(m+1,n+1,e):(m+1,\reflEqN(m+1))=(n+1,e)
  \end{equation*}
  for each $m,n:\N$ equipped with $e:\EqN(m,n)$. We first observe that there is a map
  \begin{equation*}
    \begin{tikzcd}
      \Big(\sm{n:\N}\EqN(m,n)\Big) \arrow[r,"f"] & \Big(\sm{n:\N}\EqN(m+1,n)\Big)
    \end{tikzcd}
  \end{equation*}
  given by $(n,e)\mapsto (n+1,e)$. With this definition of $f$ we have
  \begin{equation*}
    f(m,\reflEqN(m))\jdeq (m+1,\reflEqN(m+1)).
  \end{equation*}
  Therefore we can define
  \begin{equation*}
    \gamma(m+1,n+1,e)\defeq \ap{f}{\gamma(m,n,e)}.\qedhere
  \end{equation*}
\end{proof}
\index{natural numbers!observational equality|)}
\index{Eq N@{$\EqN$}|)}

\subsection{Embeddings}
\index{embedding|(}
In our second application of the fundamental theorem we show that equivalences are embeddings. The notion of embedding is the homotopical analogue of the set theoretic notion of injective map.

\begin{defn}
An \define{embedding}\index{embedding|textbf} is a map $f:A\to B$\index{is an embedding|textbf} that satisfies the property that\index{is an equivalence!action on paths of an embedding}
\begin{equation*}
\apfunc{f}:(\id{x}{y})\to(\id{f(x)}{f(y)})
\end{equation*}
is an equivalence, for every $x,y:A$. We write $\isemb(f)$\index{is-emb(f)@{$\isemb(f)$}} for the type of witnesses that $f$ is an embedding, and we define\index{A hookrightarrow B@{$A\hookrightarrow B$}|see {embedding}}
\begin{equation*}
  A\hookrightarrow B\defeq \sm{f:A\to B}\isemb(f).
\end{equation*}
\end{defn}

Another way of phrasing the following statement is that equivalent types have equivalent identity types.

\begin{thm}
\label{cor:emb_equiv} 
Any equivalence is an embedding.\index{is an embedding!equivalence}\index{equivalence!is an embedding}
\end{thm}

\begin{proof}
Let $e:\eqv{A}{B}$ be an equivalence, and let $x:A$. Our goal is to show that
\begin{equation*}
\apfunc{e} : (\id{x}{y})\to (\id{e(x)}{e(y)})
\end{equation*}
is an equivalence for every $y:A$. By \cref{thm:id_fundamental} it suffices to show that 
\begin{equation*}
\sm{y:A}e(x)=e(y)
\end{equation*}
is contractible. Now observe that there is an equivalence
\begin{samepage}
\begin{align*}
\sm{y:A}e(x)=e(y) & \eqvsym \sm{y:A}e(y)=e(x) \\
& \jdeq \fib{e}{e(x)}
\end{align*}
\end{samepage}
by \cref{thm:fib_equiv}, since for each $y:A$ the map
\begin{equation*}
\invfunc : (e(x)=e(y))\to (e(y)= e(x))
\end{equation*}
is an equivalence by \cref{ex:equiv_grpd_ops}.
The fiber $\fib{e}{e(x)}$ is contractible by \cref{thm:contr_equiv}, so it follows by \cref{ex:contr_equiv} that the type $\sm{y:A}e(x)=e(y)$ is indeed contractible.
\end{proof}
\index{embedding|)}

\subsection{Disjointness of coproducts}

\index{disjointness of coproducts|(}
\index{characterization of identity type!of coproducts|(}
\index{identity type!of a coproduct|(}
\index{coproduct!characterization of identity type|(}
\index{coproduct!disjointness|(}
In our third application of the fundamental theorem of identity types, we characterize the identity types of coproducts. Our goal in this section is to prove the following theorem.

\begin{thm}\label{thm:id-coprod-compute}
Let $A$ and $B$ be types. Then there are equivalences\index{identity type!of a coproduct}\index{characterization of identity type!of coproducts}\index{coproduct!characterization of identity type}
\begin{align*}
(\inl(x)=\inl(x')) & \eqvsym (x = x')\\
(\inl(x)=\inr(y')) & \eqvsym \emptyt \\
(\inr(y)=\inl(x')) & \eqvsym \emptyt \\
(\inr(y)=\inr(y')) & \eqvsym (y=y')
\end{align*}
for any $x,x':A$ and $y,y':B$.
\end{thm}

In order to prove \cref{thm:id-coprod-compute}, we first define
a binary relation $\Eqcoprod_{A,B}$ on the coproduct $A+B$.

\begin{defn}
Let $A$ and $B$ be types. We define\index{Eq-coprod@{$\Eqcoprod_{A,B}$}|textbf}\index{coproduct!Eq-coprod@{$\Eqcoprod_{A,B}$}|textbf}
\begin{equation*}
\Eqcoprod_{A,B} : (A+B)\to (A+B)\to\UU
\end{equation*}
by double induction on the coproduct, postulating
\begin{align*}
\Eqcoprod_{A,B}(\inl(x),\inl(x')) & \defeq (x=x') \\
\Eqcoprod_{A,B}(\inl(x),\inr(y')) & \defeq \emptyt \\
\Eqcoprod_{A,B}(\inr(y),\inl(x')) & \defeq \emptyt \\
\Eqcoprod_{A,B}(\inr(y),\inr(y')) & \defeq (y=y').
\end{align*}
The relation $\Eqcoprod_{A,B}$ is also called the \define{observational equality of coproducts}\index{observational equality!on coproduct types}\index{coproduct!observational equality}.
\end{defn}

\begin{lem}
The observational equality relation $\Eqcoprod_{A,B}$ on $A+B$ is reflexive, and therefore there is a map
\begin{equation*}
\Eqcoprodeq:\prd{s,t:A+B} (s=t)\to \Eqcoprod_{A,B}(s,t).
\end{equation*}
\end{lem}

\begin{constr}
The reflexivity term $\rho$ is constructed by induction on $t:A+B$, using
\begin{align*}
\rho(\inl(x))\defeq \refl{x}  & : \Eqcoprod_{A,B}(\inl(x),\inl(x)) \\
\rho(\inr(y))\defeq \refl{y} & : \Eqcoprod_{A,B}(\inr(y),\inr(y)).\qedhere
\end{align*}
\end{constr}

To show that $\Eqcoprodeq$ is a family of equivalences, we will use the fundamental theorem of identity types, \cref{thm:id_fundamental}. Therefore, we need to prove the following proposition.

\begin{prp}\label{lem:is-contr-total-eq-coprod}
For any $s:A+B$ the total space
\begin{equation*}
\sm{t:A+B}\Eqcoprod_{A,B}(s,t)
\end{equation*}
is contractible.
\end{prp}

\begin{proof}
  For convenience, let us write $E\defeq \Eqcoprod_{A,B}$. By induction on $s$, it suffices to show that the total spaces
  \begin{equation*}
    \sm{t:A+B}E(\inl(x),t) \qquad\text{and}\qquad \sm{t:A+B}E(\inr(y),t)
  \end{equation*}
  are contractible. The two proofs are similar, so we only prove that the type on the left is contractible. By the laws of coproducts and $\Sigma$-types given in \cref{eg:laws-products-coproducts,eg:laws-Sigma-types}, we simply compute
  \begin{samepage}
    \begin{align*}
      & \sm{t:A+B}E(\inl(x),t) \\
      & \eqvsym \Big(\sm{x':A}E(\inl(x),\inl(x'))\Big)+\Big(\sm{y':B}E(\inl(x),\inr(y'))\Big) \\
      & \eqvsym \Big(\sm{x':A}x=x'\Big)+\Big(\sm{y':B}\emptyt\Big) \\
      & \eqvsym \sm{x':A}x=x'.
    \end{align*}%
  \end{samepage}%
  The last type in this computation is contractible by \cref{thm:total_path}, so we conclude that the total space of $E(\inl(x))$ is contractible.
\end{proof}

\begin{proof}[Proof of \cref{thm:id-coprod-compute}]
  The proof is now concluded with an application of \cref{thm:id_fundamental}, using \cref{lem:is-contr-total-eq-coprod}.
\end{proof}
\index{disjointness of coproducts|)}
\index{characterization of identity type!of coproducts|)}
\index{identity type!of a coproduct|)}
\index{coproduct!characterization of identity type|)}
\index{coproduct!disjointness|)}

\subsection{The structure identity principle}\label{sec:structure-identity-principle}
\index{structure identity principle|(}

We often encounter a type consisting of certain objects equipped with further structure. For example, the fiber of a map $f:A\to B$ at $b:B$ is the type of elements $a:A$ equipped with an identification $p:f(a)=b$. Such \emph{structure} types occur all over mathematics, and it is important to have an efficient characterization of their identity types. A general structure type is just a $\Sigma$-type, and we're asking for a characterization of its identity type.

Recall from \cref{thm:eq_sigma} that the identity type of the type $\sm{x:A}B(x)$ at a pair $(a,b)$ can be characterized as
\begin{equation*}
  ((a,b)=(x,y))\simeq \sm{p:a=x}\tr_B(p,b)=y.
\end{equation*}
However, this characterization of the identity type of $\sm{x:A}B(x)$ is not as clear and useful as we like it to be, because it uses the transport function, which is completely generic. Our plan is to use identity systems on $A$ and on $B(a)$ to arrive at a more useful characterization of the identity type of $\sm{x:A}B(x)$.

In order to abstract away this characterization of the identity type of $\sm{x:A}B(x)$, let $C:A\to\UU$ be the family of types given by $C(x)\defeq (a=x)$, and let
\begin{equation*}
  D:\prd{x:A}B(x)\to(C(x)\to\UU)
\end{equation*}
be the family of types given by $D(x,y,p)\defeq \tr_B(p,b)=y$. Then $C$ is an identity system on $A$ at $a$, and the type family $y\mapsto D(a,y,\refl{})$ is an identity system on $B(a)$ at $b$. This suggests the following definition of dependent identity systems.

\begin{defn}
  Consider a type $A$ equipped with an identity system $C$ based at $a:A$, and let $c:C(a)$. Furthermore, consider a type family $B$ over $A$. A \define{dependent identity system}\index{dependent identity system|textbf}\index{identity system!dependent identity system|textbf} over $C$ at $b:B(a)$ consists of a type family
  \begin{equation*}
    D : \prd{x:A} B(x) \to (C(x)\to \UU)
  \end{equation*}
  equipped with an element $d:D(a,b,c)$ such that $y\mapsto D(a,y,c)$ is an identity system at $b$.
\end{defn}

\begin{thm}[Structure identity principle]\label{thm:structure-identity-principle}
  Consider a type family $B$ over $A$, elements $a:A$ and $b:B(a)$, and an identity system $C$ of $A$ with $c:C(a)$. Furthermore, consider a type family
  \begin{equation*}
    D : \prd{x:A} B(x) \to (C(x)\to \UU)
  \end{equation*}
  equipped with an element $d:D(a,b,c)$. Then the following are equivalent:
  \begin{enumerate}
  \item Any family of maps
    \begin{equation*}
      (b=y)\to D(a,y,c)
    \end{equation*}
    indexed by $y:B(a)$ is a family of equivalences.
  \item The total space
    \begin{equation*}
      \sm{y:B(a)}D(a,y,c)
    \end{equation*}
    is contractible.
  \item $D$ is a dependent identity system over $C$ at $b:B(a)$.
  \item Any family of maps
    \begin{equation*}
      ((a,b)=(x,y))\to \sm{z:C(x)}D(x,y,z))
    \end{equation*}
    indexed by $(x,y):\sm{x:A}B(x)$ is a family of equivalences.
  \item The total space
    \begin{equation*}
      \sm{(x,y):\sm{x:A}B(x)}\sm{z:C(x)}D(x,y,z)
    \end{equation*}
    is contractible.
  \item The type family
    \begin{equation*}
      (x,y)\mapsto \sm{z:C(x)}D(x,y,z)
    \end{equation*}
    is an identity system at $(a,b):\sm{x:A}B(x)$.
  \end{enumerate}
\end{thm}

\begin{proof}
  The first three statements as well as the last three statements are equivalent by \cref{thm:id_fundamental}. Therefore it suffices to show that (ii) and (v) are equivalent. Note that there is an equivalence
  \begin{multline*}
    \sm{(x,y):\sm{x:A}B(x)}\sm{z:C(x)}D(x,y,z) \\
    \simeq
    \sm{(x,z):\sm{x:A}C(x)}\sm{y:B(x)}D(x,y,z).
  \end{multline*}
  This equivalence, its inverse, and the homotopies witnessing that the inverse is indeed an inverse are all straightforward to construct using pattern matching. Furthermore, notice that the type $\sm{x:A}C(x)$ is contractible with center of contraction $(a,c)$ since $C$ is assumed to be an identity system at $a:A$. Therefore it follows that
  \begin{equation*}
    \sm{(x,y):\sm{x:A}B(x)}\sm{z:C(x)}D(x,y,z)\simeq\sm{y:B(a)}D(a,y,c).\qedhere
  \end{equation*}
\end{proof}

\begin{eg}
  By the structure identity principle of \cref{thm:structure-identity-principle} in combination with the fundamental theorem of identity types (\cref{thm:id_fundamental}), it becomes completely routine to characterize identity types of structures: We only have to show that the types
  \begin{equation*}
    \sm{x:A}C(x)\qquad\text{and}\qquad\sm{y:B(a)}D(a,y,c)
  \end{equation*}
  are contractible. To illustrate this use of the structure identity principle, we give an alternative characterization of the fiber of a map $f:A \to B$ at $b:B$. We claim that\index{identity type!of a fiber}\index{fiber!characterization of identity type}\index{characterization of identity type!of the fiber of a map}
  \begin{align*}
    ((x,p)=(y,q)) & \simeq \fib{\apfunc{f}}{\ct{p}{q^{-1}}} \\
                  & \jdeq \sm{\alpha:x=y}\ap{f}{\alpha}=\ct{p}{q}^{-1}.
  \end{align*}
  To see this, we apply \cref{thm:structure-identity-principle}. Note that $\sm{y:A}x=y$ is contractible by \cref{thm:total_path} with center of contraction $(x,\refl{f(x)})$. Therefore it suffices to show that the type
  \begin{equation*}
    \sm{q:f(x)=b}\refl{f(x)}=\ct{p}{q}^{-1}
  \end{equation*}
  is contractible. Of course, this type is equivalent to $\sm{q:f(x)=b}p=q$, which is again contractible by \cref{thm:total_path}.
\end{eg}
\index{structure identity principle|)}

\begin{exercises}
  \exitem
  \begin{subexenum}
  \item \label{ex:is-emb-empty}Show that the map $\emptyt\to A$ is an embedding for every type $A$.\index{is an embedding!0 to A@{$\emptyt\to A$}}
  \item \label{ex:is-emb-inl-inr}Show that $\inl:A\to A+B$ and $\inr:B\to A+B$ are embeddings for any two types $A$ and $B$.
    \index{is an embedding!inl (for coproducts)@{$\inl$ (for coproducts)}}
    \index{is an embedding!inr (for coproducts)@{$\inr$ (for coproducts)}}
    \index{inl@{$\inl$}!is an embedding}
    \index{inr@{$\inr$}!is an embedding}
  \item Show that $\inl:A\to A+B$ is an equivalence if and only if $B$ is empty, and that $\inr : B \to A+B$ is an equivalence if and only if $A$ is empty.
  \end{subexenum}
  \exitem Consider an equivalence $e:A\simeq B$. Construct an equivalence
  \begin{equation*}
    p\mapsto \tilde{p}:(e(x)=y)\simeq(x=e^{-1}(y))
  \end{equation*}
  for every $x:A$ and $y:B$, such that the triangle
  \begin{equation*}
    \begin{tikzcd}[column sep=large]
      e(x) \arrow[r,equals,"\ap{e}{\tilde{p}}"] \arrow[dr,equals,swap,"p"] & e(e^{-1}(y)) \arrow[d,equals,"G(y)"] \\
      & y
    \end{tikzcd}
  \end{equation*}
  commutes for every $p:e(x)=y$. In this diagram, the homotopy $G:e\circ e^{-1}\htpy \idfunc$ is the homotopy witnessing that $e^{-1}$ is a section of $e$.
  \exitem Show that\index{embedding!closed under homotopies}
  \begin{equation*}
    (f\htpy g)\to (\isemb(f)\leftrightarrow\isemb(g))
  \end{equation*}
  for any $f,g:A\to B$.
  \exitem \label{ex:emb_triangle}Consider a commuting triangle
  \begin{equation*}
    \begin{tikzcd}[column sep=tiny]
      A \arrow[rr,"h"] \arrow[dr,swap,"f"] & & B \arrow[dl,"g"] \\
      & X
    \end{tikzcd}
  \end{equation*}
  with $H:f\htpy g\circ h$. 
  \begin{subexenum}
  \item Suppose that $g$ is an embedding. Show that $f$ is an embedding if and only if $h$ is an embedding.\index{is an embedding!composite of embeddings}\index{is an embedding!right factor of embedding if left factor is an embedding}
  \item Suppose that $h$ is an equivalence. Show that $f$ is an embedding if and only if $g$ is an embedding.\index{is an embedding!left factor of embedding if right factor is an equivalence}
  \end{subexenum}
  \exitem Consider two embeddings $f:A\hookrightarrow B$ and $g:B\hookrightarrow C$. Show that the following are equivalent:
  \begin{enumerate}
  \item The composite $g\circ f$ is an equivalence.
  \item Both $f$ and $g$ are equivalences.
  \end{enumerate}
  \exitem Consider two maps $f:A\to C$ and $g:B\to C$. Use \cref{ex:is-emb-inl-inr} to show that the following are equivalent:
  \begin{enumerate}
  \item The map $[f,g]:A+B\to C$ is an embedding.
  \item Both $f$ and $g$ are embeddings, and
    \begin{equation*}
      f(a)\neq g(b)
    \end{equation*}
    for all $a:A$ and $b:B$.
  \end{enumerate}
  \exitem \label{ex:is-equiv-is-equiv-functor-coprod}Consider two maps $f:A\to A'$ and $g:B \to B'$.
  \begin{subexenum}
  \item Show that if the map
    \begin{equation*}
      f+g:(A+B)\to (A'+B')
    \end{equation*}
    is an equivalence, then so are both $f$ and $g$ (this is the converse of \cref{ex:coproduct_functor_equivalence}).
  \item \label{ex:is-emb-coprod}Show that $f+g$ is an embedding if and only if both $f$ and $g$ are embeddings.
  \end{subexenum}
  \exitem \label{ex:id_fundamental_retr}
  \begin{subexenum}
  \item Let $f,g:\prd{x:A}B(x)\to C(x)$ be two families of maps. Show that
    \begin{equation*}
      \Big(\prd{x:A}f(x)\htpy g(x)\Big)\to \Big(\tot{f}\htpy \tot{g}\Big). 
    \end{equation*}
  \item Let $f:\prd{x:A}B(x)\to C(x)$ and let $g:\prd{x:A}C(x)\to D(x)$. Show that
    \begin{equation*}
      \tot{\lam{x}g(x)\circ f(x)}\htpy \tot{g}\circ\tot{f}.
    \end{equation*}
  \item For any family $B$ over $A$, show that
    \begin{equation*}
      \tot{\lam{x}\idfunc[B(x)]}\htpy\idfunc.
    \end{equation*}
  \item Let $a:A$, and let $B$ be a type family over $A$. Use \cref{ex:contr_retr} to show that if each $B(x)$ is a retract of $\id{a}{x}$, then $B(x)$ is equivalent to $\id{a}{x}$ for every $x:A$.
    \index{fundamental theorem of identity types!formulation with retractions}
  \item Conclude that for any family of maps
    \index{fundamental theorem of identity types!formulation with sections}
    \begin{equation*}
      f : \prd{x:A} (a=x) \to B(x),
    \end{equation*}
    if each $f(x)$ has a section, then $f$ is a family of equivalences.
  \end{subexenum} 
  \exitem Use \cref{ex:id_fundamental_retr} to show that for any map $f:A\to B$, if
  \begin{equation*}
    \apfunc{f} : (x=y) \to (f(x)=f(y))
  \end{equation*}
  has a section for each $x,y:A$, then $f$ is an embedding.\index{is an embedding!if the action on paths have sections}
  \exitem \label{ex:path-split}(Shulman) We say that a map $f:A\to B$ is \define{path-split}\index{path-split|textbf} if $f$ has a section, and for each $x,y:A$ the map
  \begin{equation*}
    \apfunc{f}(x,y):(x=y)\to (f(x)=f(y))
  \end{equation*}
  also has a section. We write $\pathsplit(f)$\index{path-split(f)@{$\pathsplit(f)$}|textbf} for the type
  \begin{equation*}
    \sections(f)\times\prd{x,y:A}\sections(\apfunc{f}(x,y)).
  \end{equation*}
  Show that for any map $f:A\to B$ the following are equivalent:
  \begin{enumerate}
  \item The map $f$ is an equivalence.
  \item The map $f$ is path-split.\index{is an equivalence!path-split map}
  \end{enumerate}
  \exitem \label{ex:fiber_trans}Consider a triangle
  \begin{equation*}
    \begin{tikzcd}[column sep=small]
      A \arrow[rr,"h"] \arrow[dr,swap,"f"] & & B \arrow[dl,"g"] \\
      & X
    \end{tikzcd}
  \end{equation*}
  with a homotopy $H:f\htpy g\circ h$ witnessing that the triangle commutes. 
  \begin{subexenum}
  \item Construct a family of maps
    \begin{equation*}
      \fibtriangle(h,H):\prd{x:X}\fib{f}{x}\to\fib{g}{x},
    \end{equation*}
    for which the square
    \begin{equation*}
      \begin{tikzcd}[column sep=8em]
        \sm{x:X}\fib{f}{x} \arrow[r,"\tot{\fibtriangle(h,H)}"] \arrow[d] & \sm{x:X}\fib{g}{x} \arrow[d] \\
        A \arrow[r,swap,"h"] & B
      \end{tikzcd}
    \end{equation*}
    commutes, where the vertical maps are as constructed in \cref{ex:fib_replacement}.
  \item Show that $h$ is an equivalence if and only if $\fibtriangle(h,H)$ is a family of equivalences.
  \end{subexenum}
\end{exercises}
\index{fundamental theorem of identity types|)}
\index{characterization of identity type!fundamental theorem of identity types|)}

%%% Local Variables:
%%% mode: latex
%%% TeX-master: "hott-intro"
%%% End:

% !TEX root = hott_intro.tex

\section{Propositions, sets, and the higher truncation levels}
\sectionmark{Truncation levels}\label{chap:hierarchy}

The set theoretic foundations of mathematics arise in two stages. The first stage is to specify the formal system of first order logic; the second stage is to give an axiomatization of set theory in this formal system. Unlike set theory, type theory is its own formal system. The logic of dependent types, as given by the inference rules, is all we need.

However, even though type theory is not built upon a separate system of logic such as first order logic, we can find logic in type theory by recognizing certain types as propositions. Note that the propositions of first order logic have a virtue that could be rather useful sometimes: First order logic does not offer any way to distinguish between any two proofs of the same proposition. Therefore we say that propositions in type theory are those types that have at most one element.

This condition can be expressed with the identity type: any two elements must be equal. Examples of such types include the empty type $\emptyt$ and the unit type $\unit$. We call such types propositions. Propositions are useful, because if we know that a certain type is a proposition, then we know that any of its inhabitants are equal. Many important conditions, such as the condition that a map is an equivalence, will turn out to be propositions. This fact implies that two equivalences $A\simeq B$ are equal if and only if their underlying maps $A\to B$ are equal. However, the claim that being an equivalence is a proposition requires function extensionality, the topic of the next section.

In this section we use the idea of propositions in a different way. After we establish some basic properties of propositions, we will introduce the \emph{sets} as the types of which the identity types are propositions. This is again reminiscent of the situation in set theory, where equality is a predicate in first order logic. We will see in \cref{eg:is-set-nat} that the type of natural numbers is a set.

Next, one might ask about the types of which the identity types are \emph{sets}. Such types are called \emph{$1$-types}. There is an entire hierarchy of special types that arises this way, where a type is said to be a $(k+1)$-type if its identity types are $k$-types. Since the identity types of the $1$-types are sets, we see that sets are in fact $0$-types. Most of mathematics takes place at this level, the level of sets. The types in higher levels, as well as types that do not belong to any finite level in this hierarchy, are studied extensively in synthetic homotopy theory.

However, we can also go a step further down: Since the identity types of sets are propositions, we see that the propositions are $(-1)$-types. Moreover, the identity types of propositions are contractible. Hence we find at the bottom of this hierarchy the contractible types as the $(-2)$-types. There is no point in going down further, since we have seen in \cref{ex:prop_contr} that the identity types of contractible types are again contractible.

\index{truncated type|(}
\index{truncation level|(}

\subsection{Propositions}

\index{proposition|(}
\begin{defn}
A type $A$ is said to be a \define{proposition}\index{proposition|textbf} if its identity types are contractible, i.e., if it comes equipped with a term of type\index{is-prop(A)@{$\isprop(A)$}|textbf}
\begin{equation*}
\isprop(A)\defeq\prd{x,y:A}\iscontr(x=y).
\end{equation*}
Given a universe $\UU$, we define $\prop_\UU$\index{Prop@{$\prop_\UU$}|textbf} to be the type of all small propositions, i.e.,
\begin{equation*}
  \prop_\UU\defeq\sm{X:\UU}\isprop(X).
\end{equation*}
\end{defn}

\begin{eg}\label{eg:prop_contr}
  Any contractible type is a proposition by \cref{ex:prop_contr}\index{contractible type!is a proposition}\index{is a proposition!contractible type}. In particular, the unit type is a proposition. The empty type is also a proposition, since we have\index{empty type!is a proposition}\index{is a proposition!empty type}
\begin{equation*}
\prd{x,y:\emptyt}\iscontr(x=y)
\end{equation*}
by the induction principle of the empty type.
\end{eg}

There are many conditions on a type $A$ that are equivalent to the condition that $A$ is a proposition. In the following proposition we state four such conditions.

\begin{prp}\label{lem:isprop_eq}
  Let $A$ be a type. Then the following are equivalent:
  \begin{enumerate}
  \item The type $A$ is a proposition.
  \item Any two terms of type $A$ can be identified, i.e., there is a dependent function of type\index{is-prop'(A)@{$\isprop'(A)$}}\index{is-prop(A)@{$\isprop(A)$}!is-prop(A) iff is-prop'(A)@{$\isprop(A)\leftrightarrow\isprop'(A)$}}
    \begin{equation*}
      \isprop'(A)\defeq\prd{x,y:A}\id{x}{y}.
    \end{equation*}
  \item The type $A$ is contractible as soon as it is inhabited, i.e., there is a function of type\index{is-prop(A)@{$\isprop(A)$}!is-prop(A) iff A to is-contr(A)@{$\isprop(A)\leftrightarrow(A\to\iscontr(A))$}}
    \begin{equation*}
      A \to \iscontr(A).
    \end{equation*}
  \item The map $\const_\ttt : A\to\unit$ is an embedding.\index{is-prop(A)@{$\isprop(A)$}!is-prop(A) iff is-emb(const star)@{$\isprop(A)\leftrightarrow\isemb(\const_\ttt)$}}
  \end{enumerate}
\end{prp}

\begin{proof}
  If $A$ is a proposition, then we can use the center of contraction of the identity types of $A$ to identify any two terms in $A$. This shows that (i) implies (ii).

  To show that (ii) implies (iii), suppose that $A$ comes equipped with $p:\prd{x,y:A}\id{x}{y}$. Then for any $x:A$ the dependent function $p(x):\prd{y:A}\id{x}{y}$ is a contraction of $A$. Thus we obtain the function
  \begin{equation*}
    \lam{x}(x,p(x)):A\to\iscontr(A).
  \end{equation*}

  To show that (iii) implies (iv), suppose that $A\to\iscontr(A)$. We first make the simple observation that
  \begin{equation*}
    (X\to \isemb(f))\to \isemb(f)
  \end{equation*}
  for any map $f:X\to Y$, so it suffices to show that $A\to\isemb(\const_\ttt)$. However, assuming we have $x:A$, it follows by assumption that $A$ is contractible. Therefore, it follows by \cref{ex:contr_equiv} that the map $\const_\ttt:A\to\unit$ is an equivalence, and any equivalence is an embedding by \cref{cor:emb_equiv}.

  To show that (iv) implies (i), note that if $A\to\unit$ is an embedding, then the identity types of $A$ are equivalent to contractible types and therefore they must be contractible.
\end{proof}

One useful feature of propositions, is that in order to construct an equivalence $e:P\simeq Q$ between propositions, it suffices to construct maps back and forth between them.

\begin{prp}\label{prp:equiv-prop}
  A map $f:P\to Q$ between two propositions $P$ and $Q$ is an equivalence if and only if there is a map $g:Q\to P$. Consequently, we have for any two propositions $P$ and $Q$ that
  \begin{equation*}
    (P\simeq Q) \leftrightarrow (P\leftrightarrow Q).
  \end{equation*}
\end{prp}

\begin{proof}
  Of course, if we have an equivalence $e:P\simeq Q$, then we get maps back and forth between $P$ and $Q$. Therefore it remains to show that
  \begin{equation*}
    (P\leftrightarrow Q) \to (P\simeq Q).
  \end{equation*}
  Suppose we have $f:P\to Q$ and $g:Q\to P$. Then we obtain the homotopies $f\circ g\htpy \idfunc$ and $g\circ f\htpy \idfunc$ by the fact that any two elements in $P$ and $Q$ can be identified. Therefore $f$ is an equivalence with inverse $g$. 
\end{proof}
\index{proposition|)}

\subsection{Subtypes}
\index{subtype|(}

  In set theory, a set $y$ is said to be a subset of a set $x$, if any element of $y$ is an element of $x$, i.e., if the condition
  \begin{equation*}
    \forall_z\, (z\in y)\to (z\in x)
  \end{equation*}
  holds. We have already noted that type theory is different from set theory in that terms in type theory come equipped with a \emph{unique} type. Moreover, in set theory the proposition $x\in y$ is well-formed for any two sets $x$ and $y$, whereas in type theory we can only judge that $a:A$ by applying the rules of inference of type theory in such a manner that we arrive at the conclusion that $a:A$. Because of these differences we must find a different way to talk about subtypes.

  Note that in set theory there is a correspondence between the subsets of a set $x$, and the \emph{predicates} on $x$. A predicate on $x$ is just a proposition $P(z)$ that varies over the elements $z\in x$. Indeed, if $y$ is a subset of $x$, then the corresponding predicate is the proposition $z\in y$. Conversely, if $P$ is a predicate on $x$, then we obtain the subset
  \begin{equation*}
    \{z\in x\mid P(z)\}
  \end{equation*}
  of $x$. This observation suggests that in type theory we should define a subtype of a type $A$ to be a family of propositions over $A$.

\begin{defn}
A type family $B$ over $A$ is said to be a \define{subtype}\index{subtype|textbf} of $A$ if for each $x:A$ the type $B(x)$ is a proposition. When $B$ is a subtype of $A$, we also say that $B(x)$ is a \define{property}\index{property|textbf} of $x:A$.
\end{defn}

One reason why subtypes are important and useful, is that for any
\begin{equation*}
  (x,p),(y,q):\sm{x:A}P(x)
\end{equation*}
in a subtype of $A$, we have $(x,p)=(y,q)$ if and only if $x=y$. In other words, two terms of a subtype of $A$ are equal if and only if they are equal as terms of $A$. This fact is properly expressed using embeddings: we claim that the projection map
\begin{equation*}
  \proj 1 : \Big(\sm{x:A}P(x)\Big)\to A
\end{equation*}
is an embedding, for any subtype $P$ of $A$. This claim can be strengthened slightly. We will prove the following two closely related facts:
\begin{enumerate}
\item A map $f:A\to B$ is an embedding if and only if its fibers are propositions.
\item A family of types $B$ over $A$ is a subtype of $A$ if and only if the projection map
  \begin{equation*}
    \Big(\sm{x:A}B(x)\Big)\to A
  \end{equation*}
  is an embedding.
\end{enumerate}
The first fact is analogous to the fact that a map is an equivalence if and only if its fibers are contractible, which we saw in \cref{thm:contr_equiv,thm:equiv_contr}. To prove the above claims, we will need that propositions are closed under equivalences.

\begin{lem}\label{lem:prop_equiv}
Let $A$ and $B$ be types, and let $e:\eqv{A}{B}$. Then we have\index{proposition!closed under equivalences}
\begin{equation*}
\isprop(A)\leftrightarrow\isprop(B).
\end{equation*}
\end{lem}

\begin{proof}
We will show that $\isprop(B)$ implies $\isprop(A)$. This suffices, because the converse follows from the fact that $e^{-1}:B\to A$ is also an equivalence. 

Since $e$ is assumed to be an equivalence, it follows by \cref{cor:emb_equiv} that
\begin{equation*}
\apfunc{e} : (x=y)\to (e(x)=e(y))
\end{equation*}
is an equivalence for any $x,y:A$. If $B$ is a proposition, then in particular the type $e(x)=e(y)$ is contractible for any $x,y:A$, so the claim follows from \cref{thm:contr_equiv}.
\end{proof}

\begin{thm}\label{thm:embedding}
  Consider a map $f:A\to B$. The following are equivalent:
  \begin{enumerate}
  \item The map $f$ is an embedding.\index{embedding}
  \item The fiber $\fib{f}{b}$ is a proposition for each $b:B$.
  \end{enumerate}
\end{thm}

\begin{proof}
  By the fundamental theorem of identity types, it follows that $f$ is an embedding if and only if
  \begin{equation*}
    \sm{x:A}f(x)=f(y)
  \end{equation*}
  is contractible for each $y:A$. In other words, $f$ is an embedding if and only if $\fib{f}{f(y)}$ is contractible for each $y:A$. Note that we obtain equivalences
  \begin{equation*}
    \fib{f}{f(y)}\simeq \fib{f}{b}
  \end{equation*}
  for any $b:B$ and $p:f(y)=b$, by transporting along $p$. Therefore it follows by \cref{lem:prop_equiv} that $\fib{f}{f(y)}$ is contractible for each $y:A$ if and only if $\fib{f}{b}$ is contractible for each $y:A$, and each $b:B$ such that $p:f(y)=b$. The latter condition holds if and only if we have
  \begin{equation*}
    \fib{f}{b}\to\iscontr(\fib{f}{b})
  \end{equation*}
  for any $b:B$, which is by \cref{lem:isprop_eq} equivalent to the condition that each $\fib{f}{b}$ is a proposition.
\end{proof}

\begin{cor}\label{cor:pr1-embedding}
  Consider a family $B$ of types over $A$. The following are equivalent:
  \begin{enumerate}
  \item The map $\proj 1 : (\sm{x:A}B(x))\to A$ is an embedding.
  \item The type $B(x)$ is a proposition for each $x:A$.
  \end{enumerate}
\end{cor}

\begin{proof}
  This corollary follows at once from \cref{ex:proj_fiber}, where we showed that
  \begin{equation*}
    \fib{\proj 1}{x}\simeq B(x).\qedhere
  \end{equation*}
\end{proof}
\index{subtype|)}

\subsection{Sets}

\index{set|(}
\begin{defn}
  A type $A$ is said to be a \define{set}\index{set|textbf} if its identity types are propositions, i.e., if it comes equipped with a term of type
  \index{is-set(A)@{$\isset(A)$}}\index{is a set}
\begin{equation*}
\isset(A)\defeq \prd{x,y:A}\isprop(\id{x}{y}).
\end{equation*}
\end{defn}

\begin{eg}\label{eg:is-set-nat}
  The type of natural numbers is a set.\index{is a set!natural numbers}\index{natural numbers!is a set}\index{N@{$\N$}!is a set} To see this, recall from \cref{thm:eq_nat} that we have an equivalence
  \begin{equation*}
    (m=n)\simeq \EqN(m,n)
  \end{equation*}
  for every $m,n:\N$. Therefore it suffices to show that each $\EqN(m,n)$ is a proposition. This follows easily by induction on both $m$ and $n$.
\end{eg}

\begin{prp}
  Consider a type $A$. The following are equivalent:
  \begin{enumerate}
  \item The type $A$ is a set.
  \item The type $A$ satisfies \define{axiom K}\index{axiom K|textbf}\index{axiom!axiom K|textbf}, i.e., if and only if it comes equipped with a term of type\index{is-set(A)@{$\isset(A)$}!is-set(A) iff axiom-K(A)@{$\isset(A)\leftrightarrow\axiomK(A)$}}\index{axiom KK(A)@{$\axiomK(A)$}|textbf}
    \begin{equation*}
      \axiomK(A)\defeq\prd{x:A}\prd{p:\id{x}{x}}\id{\refl{x}}{p}.
    \end{equation*}
  \end{enumerate}
\end{prp}

\begin{proof}
If $A$ is a set, then $\id{x}{x}$ is a proposition, so any two of its elements are equal. 
This implies axiom $K$. 

For the converse, if $A$ satisfies axiom $K$, then for any $p,q:\id{x}{y}$ we have $\id{\ct{p}{q^{-1}}}{\refl{x}}$, and hence $\id{p}{q}$. This shows that $\id{x}{y}$ is a proposition, and hence that $A$ is a set.
\end{proof}

\begin{thm}\label{lem:prop_to_id}
Let $A$ be a type, and let $R:A\to A\to\UU$ be a binary relation on $A$ satisfying
\begin{enumerate}
\item Each $R(x,y)$ is a proposition,
\item $R$ is reflexive, as witnessed by $\rho:\prd{x:A}R(x,x)$,
\item There is a map
  \begin{equation*}
    R(x,y)\to (x=y)
  \end{equation*}
  for each $x,y:A$.
\end{enumerate}
Then any family of maps
\begin{equation*}
\prd{x,y:A}(\id{x}{y})\to R(x,y)
\end{equation*}
is a family of equivalences. Consequently, the type $A$ is a set.
\end{thm}

\begin{proof}
Let $f:\prd{x,y:A}R(x,y)\to(\id{x}{y})$. 
Since $R$ is assumed to be reflexive, we also have a family of maps
\begin{equation*}
\pathind_x(\rho(x)):\prd{y:A}(\id{x}{y})\to R(x,y).
\end{equation*}
Since each $R(x,y)$ is assumed to be a proposition, it therefore follows that each $R(x,y)$ is a retract of $\id{x}{y}$. Therefore it follows that $\sm{y:A}R(x,y)$ is a retract of $\sm{y:A}x=y$, which is contractible. We conclude that $\sm{y:A}R(x,y)$ is contractible, and therefore that any family of maps
\begin{equation*}
  \prd{y:A}(x=y)\to R(x,y)
\end{equation*}
is a family of equivalences.

Now it also follows that $A$ is a set, since its identity types are equivalent to propositions, and therefore they are propositions by \cref{lem:prop_equiv}. 
\end{proof}

\begin{thm}[Hedberg]\label{thm:hedberg}
Any type with decidable equality is a set.\index{Hedberg's theorem}\index{set!Hedberg's theorem}
\end{thm}

\begin{proof}
  Let $A$ be a type, and let $d:\prd{x,y:A}(\id{x}{y})+ (x\neq y)$ be the witness that $A$ has decidable equality. Furthermore, let $\UU$ be a universe containing the type $A$. We will prove that $A$ is a set by applying \cref{lem:prop_to_id}.

  For every $x,y:A$, we first define a type family $R'(x,y):((\id{x}{y})+{(x\neq y)})\to\UU$ by
\begin{align*}
R'(x,y,\inl(p)) & \defeq \unit \\
R'(x,y,\inr(p)) & \defeq \emptyt.
\end{align*}
Note that $R'(x,y,q)$ is a proposition for each $x,y:A$ and $q:(\id{x}{y})+(x\neq y)$. 
Now we define $R(x,y)\defeq R'(x,y,d(x,y))$. Then $R$ is a reflexive binary relation on $A$, and furthermore each $R(x,y)$ is a proposition. In order to apply \cref{lem:prop_to_id}, it therefore it remains to show that $R$ implies identity. 

Since $R$ is defined as an instance of $R'$, it suffices to construct a function
\begin{equation*}
  f(q) : R'(q)\to (\id{x}{y}). 
\end{equation*}
for each $q:(\id{x}{y})+(x\neq y)$. Such a function is defined by
\begin{align*}
  f(\inl(p),r) & := p \\
  f(\inr(p),r) & := \exfalso(r).\qedhere
\end{align*}
\end{proof}
\index{set|)}

\subsection{General truncation levels}
\index{truncated type|(}
\index{truncation level|(}

Consider a type $A$ in a universe $\UU$. The conditions
\begin{align*}
  \iscontr(A) & \defeq \sm{a:A}\prd{x:A}a=x \\
  \isprop(A) & \defeq \prd{x,y:A}\iscontr(x=y) \\
  \isset(A) & \defeq \prd{x,y:A}\isprop(x=y)
\end{align*}
define the first few layers of the hierarchy of truncation levels. This hierarchy starts at the level of the contractible types, which we call level $-2$. The next level is the level of propositions, and at level $0$ we have the sets.

The indexing type of the truncation levels, which will be equivalent to the type $\Z_{\geq -2}$ of integers greater than $-2$, is an inductive type $\T$\index{T@{$\T$}|see {truncation level}} equipped with the constructors\index{-2 T@{$\negtwoT$}}\index{succT@{$\succT$}}
\begin{align*}
  \negtwoT & : \T \\
  \succT & : \T\to\T.
\end{align*}
The natural inclusion $i:\N\to \T$ is defined recursively by
\begin{align*}
  i(\zeroN) & \defeq \succT(\succT(\negtwoT)) \\
  i(\succN(n)) & \defeq \succT(i(n)).
\end{align*}
Of course, we will simply write $-2$ for $\negtwoT$ and $k+1$ for $\succT(k)$.

\begin{defn}
We define $\istrunc{} : \T\to\UU\to\UU$ recursively by\index{is-trunc k(A)@{$\istrunc{k}(A)$}}
\begin{align*}
\istrunc{-2}(A) & \defeq \iscontr(A) \\
\istrunc{k+1}(A) & \defeq \prd{x,y:A}\istrunc{k}(\id{x}{y}).\qedhere
\end{align*}
For any type $A$, we say that $A$ is \define{$k$-truncated}\index{k-truncated type@{$k$-truncated type}|see {truncated type}}\index{truncated type|textbf}, or a \define{$k$-type}\index{k-type@{$k$-type}|see {truncated type}}, if there is a term of type $\istrunc{k}(A)$. We also say that a type $A$ is a \define{proper $(k+1)$-type}\index{proper (k+1)-type@{proper $(k+1)$-type}|textbf} if $A$ is a $(k+1)$-type and not a $k$-type.

Given a universe $\UU$, we define the universe $\UU^{\leq k}$ of $k$-truncated types by\index{U leq k@{$\UU^{\leq k}$}|textbf}
\begin{equation*}
  \UU^{\leq k}\defeq\sm{X:\UU}\istrunc{k}(X).
\end{equation*}

Furthermore, we say that a map $f:A\to B$ is $k$-truncated if its fibers are $k$-truncated.\index{k-truncated map@{$k$-truncated map}|see {truncated map}}\index{truncated map|textbf}
\end{defn}

\begin{rmk}
  There is a subtlety in the definition of $\istrunc{}$ regarding universes. Note that the truncation levels are defined with respect to a universe $\UU$. To be completely precise, we should therefore write $\istrunc{k}^{\UU}(A)$ for the type $\istrunc{k}(A)$ defined with respect to the universe $\UU$. If $A$ is also contained in a second universe $\VV$, then it is legitimate to ask whether
  \begin{equation*}
    \istrunc{k}^{\UU}(A)\leftrightarrow\istrunc{k}^{\VV}(A).
  \end{equation*}
  A simple inductive argument shows that this is indeed the case, where the base case follows from the judgmental equalities
  \begin{align*}
    \istrunc{-2}^{\UU}(A) & \jdeq \sm{x:A}\prd{y:A}x=y \\
    \istrunc{-2}^{\VV}(A) & \jdeq \sm{x:A}\prd{y:A}x=y.
  \end{align*}
  We may therefore safely omit explicit reference to the universes when considering truncatedness of a type.
\end{rmk}

We show in the following theorem that the truncation levels are successively contained in one another.

\begin{prp}\label{thm:istrunc_next}
If $A$ is a $k$-type, then $A$ is also a $(k+1)$-type.\index{is-trunc k(A)@{$\istrunc{k}(A)$}!is-trunc k(A) to is-trunc k+1(A)@{$\istrunc{k}(A)\to\istrunc{k+1}(A)$}}
\end{prp}

\begin{proof}
We have seen in \cref{eg:prop_contr} that contractible types are propositions. This proves the base case.
For the inductive step, note that if any $k$-type is also a $(k+1)$-type, then any $(k+1)$-type is a $(k+2)$-type, since its identity types are $k$-types and therefore $(k+1)$-types.
\end{proof}

It is immediate from the proof of \cref{thm:istrunc_next} that the identity types of $k$-types are also $k$-types.

\begin{cor}
  If $A$ is a $k$-type, then its identity types are also $k$-types.\hfill $\square$
\end{cor}

\begin{prp}\label{thm:ktype_eqv}
If $e:\eqv{A}{B}$ is an equivalence, and $B$ is a $k$-type, then so is $A$.\index{truncated type!closed under equivalences}
\end{prp}

\begin{proof}
We have seen in \cref{ex:contr_equiv} that if $B$ is contractible and $e:\eqv{A}{B}$ is an equivalence, then $A$ is also contractible. This proves the base case.

For the inductive step, assume that the $k$-types are stable under equivalences, and consider $e:\eqv{A}{B}$ where $B$ is a $(k+1)$-type. In \cref{cor:emb_equiv} we have seen that
\begin{equation*}
\apfunc{e}:(\id{x}{y})\to(\id{e(x)}{e(y)})
\end{equation*}
is an equivalence for any $x,y$. Note that $\id{e(x)}{e(y)}$ is a $k$-type, so by the induction hypothesis it follows that $\id{x}{y}$ is a $k$-type. This proves that $A$ is a $(k+1)$-type.
\end{proof}

\begin{cor}\label{cor:emb_into_ktype}
If $f:A\to B$ is an embedding, and $B$ is a $(k+1)$-type, then so is $A$.\index{truncated type!closed under embeddings}
\end{cor}

\begin{proof}
By the assumption that $f$ is an embedding, the action on paths
\begin{equation*}
\apfunc{f}:(\id{x}{y})\to (\id{f(x)}{f(y)})
\end{equation*}
is an equivalence for every $x,y:A$. Since $B$ is assumed to be a $(k+1)$-type, it follows that $f(x)=f(y)$ is a $k$-type for every $x,y:A$. Therefore we conclude by \cref{thm:ktype_eqv} that $\id{x}{y}$ is a $k$-type for every $x,y:A$. In other words, $A$ is a $(k+1)$-type.
\end{proof}

We end this section with a theorem that characterizes $(k+1)$-truncated maps. Note that it generalizes \cref{thm:embedding}, which asserts that a map is an embedding if and only if its fibers are propositions.

\begin{thm}\label{thm:trunc_ap}
Let $f:A\to B$ be a map. The following are equivalent:
\begin{enumerate}
\item The map $f$ is $(k+1)$-truncated.\index{truncated map}
\item For each $x,y:A$, the map
\begin{equation*}
\apfunc{f} : (x=y)\to (f(x)=f(y))
\end{equation*}
is $k$-truncated. 
\end{enumerate}
\end{thm}

\begin{proof}
First we show that for any $s,t:\fib{f}{b}$ there is an equivalence
\begin{equation*}
\eqv{(s=t)}{\fib{\apfunc{f}}{\ct{\proj 2(s)}{\proj 2(t)^{-1}}}}
\end{equation*}
We do this by $\Sigma$-induction on $s$ and $t$, and then we calculate
\begin{align*}
(\pairr{x,p}=\pairr{y,q}) & \eqvsym \Eqfib_f((x,p),(y,q)) \\
  & \jdeq \sm{\alpha:x=y} p=\ct{\ap{f}{\alpha}}{q} \\
  & \eqvsym \sm{\alpha:x=y} \ct{\ap{f}{\alpha}}{q}=p \\
& \eqvsym \sm{\alpha:x=y} \ap{f}{\alpha}=\ct{p}{q^{-1}} \\
& \jdeq \fib{\apfunc{f}}{\ct{p}{q^{-1}}}.
\end{align*}
By these equivalences, it follows that if $\apfunc{f}$ is $k$-truncated, then for each $s,t:\fib{f}{b}$ the identity type $s=t$ is equivalent to a $k$-truncated type, and therefore we obtain by \cref{thm:ktype_eqv} that $f$ is $(k+1)$-truncated.

For the converse, note that we have equivalences
\begin{align*}
\fib{\apfunc{f}}{p} & \eqvsym ((x,p)=(y,\refl{f(y)})).
\end{align*}
It follows that if $f$ is $(k+1)$-truncated, then the identity type $(x,p)=(y,\refl{f(y)})$ in $\fib{f}{f(y)}$ is $k$-truncated for any $p:f(x)=f(y)$. We conclude by \cref{thm:ktype_eqv} that the fiber $\fib{\apfunc{f}}{p}$ is $k$-truncated. 
\end{proof}
\index{truncated type|)}
\index{truncation level|)}

\begin{exercises}
  \exitem \label{ex:eq_bool}Show that $\bool$ is a set\index{bool@{$\bool$}!is a set} by applying \cref{lem:prop_to_id} with the observational equality on $\bool$ defined in \cref{ex:obs_bool}.
  \exitem Recall that a \define{partially ordered set (poset)}\index{poset!is a set} is defined to be a type $A$ equipped with a relation
  \begin{equation*}
    \blank\leq\blank : A \to (A \to \prop_\UU)
  \end{equation*}
  that is reflexive, antisymmetric, and transitive. Show that the underlying type of any poset is a set.
  \exitem
  \begin{subexenum}
  \item \label{cor:is-emb-is-injective}
    Show that any injective map $f:A\to B$ into a set $B$ is an embedding, and conclude that $A$ is automatically a set in this case.\index{is an embedding!injective map into a set}\index{injective map!injective maps into sets are embeddings}
  \item Show that $n\mapsto m+n$ is an embedding, for each $m:\N$.\index{add N@{$\addN$}!add N(m) is an embedding@{$\addN(m)$ is an embedding}}\index{is an embedding!add N(m)@{$\addN(m)$}} Moreover, conclude that there is an equivalence\index{N@{$\N$}!leq@{$\leq$}}
    \begin{equation*}
      (m\leq n)\simeq \sm{k:\N}m+k=n.
    \end{equation*}
  \item Show that $n\mapsto mn$ is an embedding\index{mul N@{$\mulN$}!mul N(m) is an embedding if m>0@{$\mulN(m)$ is an embedding if $m>0$}}\index{is an embedding!mul N(m) for m > 0@{$\mulN(m)$ for $m>0$}}, for each nonzero number $m:\N$. Conclude that the divisibility relation\index{d {"|" n}@{$d\mid n$}!is a proposition if d>0@{is a proposition if $d>0$}}\index{is a proposition!d {"|" n} for d>0@{$d\mid n$ for $d>0$}}\index{divisibility on N@{divisibility on $\N$}!is a proposition}
    \begin{equation*}
      d\mid n
    \end{equation*}
    is a proposition for each $d,n:\N$ such that $d>0$. 
  \end{subexenum}
  \exitem \label{ex:set_coprod}
  \begin{subexenum}
  \item Show that for any two contractible types $A$ and $B$, the coproduct $A+B$ is not contractible.
  \item Show that for any two propositions $P$ and $Q$, we have a logical equivalence
    \begin{equation*}
      \iscontr(P+Q)\leftrightarrow P\oplus Q,
    \end{equation*}
    where the \define{exclusive disjunction}\index{exclusive disjunction|textbf} $P\oplus Q$\index{P oplus Q@{$P \oplus Q$}|see {exclusive disjunction}} is defined by
    \begin{equation*}
      P\oplus Q:= (P\times\neg Q)+(Q\times\neg P).
    \end{equation*}
  \item \label{ex:is-prop-coproduct}Show that for any two propositions $P$ and $Q$, the coproduct $P+Q$ is a proposition if and only if $P\to \neg Q$.
  \item Show that for any two $(k+2)$-types $A$ and $B$, the coproduct $A+B$ is again a $(k+2)$-type\index{coproduct!is a truncated type}\index{is a truncated type!coproduct}\index{is a set!coproduct}\index{coproduct!is a set}. Conclude that $\Z$ is a set.\index{Z@{$\Z$}!is a set}
  \end{subexenum}
  \exitem \label{ex:diagonal}Let $A$ be a type, and let the \define{diagonal}\index{diagonal of a type|textbf}\index{d  A@{$\delta_A$}|textbf}\index{d  A@{$\delta_A$}|see {diagonal, of a type}} of $A$ be the map $\delta_A:A\to A\times A$ given by $\lam{x}(x,x)$. 
  \begin{subexenum}
  \item Show that\index{is a proposition!d A is an equivalence@{$\delta_A$ is an equivalence}}
    \begin{equation*}
      {\isequiv(\delta_A)}\leftrightarrow{\isprop(A)}.
    \end{equation*}
  \item Construct an equivalence $\eqv{\fib{\delta_A}{x,y}}{(x=y)}$ for any $x,y:A$.
  \item Show that $A$ is $(k+1)$-truncated if and only if $\delta_A:A\to A\times A$ is $k$-truncated.
  \end{subexenum}
  \exitem \label{ex:istrunc_sigma}
  \begin{subexenum}
  \item Consider a type family $B$ over a $k$-truncated type $A$. Show that the following are equivalent:
    \begin{enumerate}
    \item The type $B(x)$ is $k$-truncated for each $x:A$.
    \item The type $\sm{x:A}B(x)$ is $k$-truncated.\index{is a truncated type!S-type@{$\Sigma$-type}}\index{dependent pair type!is truncated}
    \end{enumerate}
    Hint: for the base case, use \cref{ex:contr_in_sigma,ex:contr_equiv}.
  \item Consider a map $f:A\to B$ into a $k$-type $B$. Show that the following are equivalent:
    \begin{enumerate}
    \item The type $A$ is $k$-truncated.
    \item The map $f$ is $k$-truncated.
    \end{enumerate}
  \end{subexenum}
  \exitem Consider two types $A$ and $B$. Show that the following are equivalent:
    \begin{enumerate}
    \item There are functions
      \begin{align*}
        f & : B \to \istrunc{k+1}(A) \\
        g & : A \to \istrunc{k+1}(B).
      \end{align*}
    \item The type $A\times B$ is $(k+1)$-truncated.
    \end{enumerate}
    Conclude with \cref{ex:is-contr-prod} that, if both $A$ and $B$ come equipped with an element, then both $A$ and $B$ are $k$-truncated if and only if the product $A\times B$ is $k$-truncated.
  \exitem
  \begin{subexenum}
  \item \label{ex:retr_id} Consider a section-retraction pair
    \begin{equation*}
      \begin{tikzcd}
        A \arrow[r,"i"] & B \arrow[r,"r"] & A,
      \end{tikzcd}
    \end{equation*}
    with $H:r\circ i\htpy \idfunc$. Show that $\id{x}{y}$ is a retract of $\id{i(x)}{i(y)}$.\index{retract!identity type}\index{identity type!of retract is retract}
  \item Use \cref{ex:contr_retr} to show that if $A$ is a retract of a $k$-type $B$, then $A$ is also a $k$-type.\index{truncated type!closed under retracts}
  \end{subexenum}
  \exitem Consider an arbitrary type $A$. Recall that concatenation of lists was defined in \cref{ex:lists}. Show that the map\index{list A@{$\lst(A)$}}
  \begin{equation*}
    f:\lst(A)\times\lst(A)\to\lst(A).
  \end{equation*}
  given by $f(x,y)\defeq\concatlist(x,y)$ is $0$-truncated.\index{truncated map!concatenation of lists}\index{concat-list@{$\concatlist$}!is a $0$-truncated map}
  \exitem \label{ex:is-trunc-const}Show that a type $A$ is a $(k+1)$-type if and only if the map $\const_x:\unit\to A$ is $k$-truncated for every $x:A$.
  \exitem \label{ex:is-trunc-comp}Consider a commuting triangle
  \begin{equation*}
    \begin{tikzcd}[column sep=tiny]
      A \arrow[rr,"h"] \arrow[dr,swap,"f"] & & B \arrow[dl,"g"] \\
      & X
    \end{tikzcd}
  \end{equation*}
  with $H: f \htpy g \circ h$, and suppose that $g$ is $k$-truncated. Show that $f$ is $k$-truncated if and only if $h$ is $k$-truncated.
  \exitem Let $f:\prd{x:A}B(x)\to C(x)$ be a family of maps. Show that the following are equivalent:
  \begin{enumerate}
  \item For each $x:A$ the map $f(x)$ is $k$-truncated.
  \item The induced map\index{tot(f)@{$\tot{f}$}!is a truncated map}\index{is a truncated map!tot(f)@{$\tot{f}$}}
    \begin{equation*}
      \tot{f}:\Big(\sm{x:A}B(x)\Big)\to\Big(\sm{x:A}C(x)\Big)
    \end{equation*}
    is $k$-truncated.
  \end{enumerate}
  \exitem \label{ex:is-trunc-fiber-inclusion}Consider a type $A$. Show that the following are equivalent:
  \begin{enumerate}
  \item The type $A$ is $(k+1)$-truncated.
  \item For any type family $B$ over $A$ and any $a:A$, the \define{fiber inclusion}\index{fiber inclusion|textbf}
    \begin{equation*}
      i_a: B(a)\to\sm{x:A}B(x)
    \end{equation*}
    given by $y\mapsto(a,y)$ is a $k$-truncated map.\index{fiber inclusion!is a truncated map}\index{is a truncated map!fiber inclusion}
  \end{enumerate}
  In particular, if $A$ is a set then any fiber inclusion $i_a:B(a)\to\sm{x:A}B(x)$ is an embedding.\index{fiber inclusion!is an embedding}\index{is an embedding!fiber inclusion}
  \exitem \label{ex:isolated-point}Consider a type $A$ equipped with an element $a:A$. We say that $a$ is an \define{isolated element}\index{isolated element|textbf} of $A$ if it comes equipped with an element of type\index{is-isolated(a)@{$\isisolated(a)$}|textbf}
  \begin{equation*}
    \isisolated(a)\defeq\prd{x:A}(a=x)+(a\neq x).
  \end{equation*}
  \begin{subexenum}
  \item Show that $a$ is isolated if and only if the map $\const_a:\unit\to A$ has decidable fibers.
  \item Show that if $a$ is isolated, then $a=x$ is a proposition, for every $x:A$. Conclude that if $a$ is isolated, then the map $\const_a:\unit\to A$ is an embedding.
  \end{subexenum}
\end{exercises}
\index{truncated type|)}
\index{truncation level|)}

%%% Local Variables:
%%% mode: latex
%%% TeX-master: "hott-intro"
%%% End:

\section{Function extensionality}
\label{chap:funext}
\index{function extensionality|(}
\index{axiom!function extensionality|(}

The function extensionality axiom asserts that for any two dependent functions $f,g:\prd{x:A}B(x)$, the type of identifications $f=g$ is equivalent to the type of homotopies $f\htpy g$ from $f$ to $g$. In other words, two (dependent) functions can only be distinguished by their values. The function extensionality axiom therefore provides a characterization of the identity type of (dependent) function types. By the fundamental theorem of identity types it follows immediately that the function extensionality axiom has at least three equivalent forms. There is, however, a fourth useful equivalent form of the function extensionality axiom: the \emph{weak} function extensionality axiom. This axiom asserts that any dependent product of contractible types is again contractible. A simple consequence of the weak function extensionality axiom is that any dependent product of a family of $k$-types is again a $k$-type.

The function extensionality axiom is used to derive many important properties in type theory. One class of such properties are (dependent) universal properties. Universal properties give a characterization of the type of functions into, or out of a type. For example, the universal property of the coproduct $A+B$ characterizes the type of maps $(A+B)\to X$ as the type of pairs of maps $(f,g)$ consisting of $f:A\to X$ and $g:B\to X$, i.e., the universal property of the coproduct $A+B$ is an equivalence
\begin{equation*}
  ((A+B)\to X)\simeq (A\to X)\times (B\to X).
\end{equation*}
Note that there are function types on both sides of this equivalence. Therefore we will need function extensionality in order to construct the homotopies witnessing that the inverse map is both a left and a right inverse. In fact, we leave this particular universal property as \cref{ex:up-coproduct}. The universal properties that we do show in the main text, are the universal properties of $\Sigma$-types and of the identity type. 

We end this section with two further applications of the function extensionality axiom. In the first, \cref{ex:equiv_precomp}, we show that precomposition by an equivalence is again an equivalence. More precisely we show that $f:A\to B$ is an equivalence if and only if for every type family $P$ over $B$, the precomposition map
\begin{equation*}
  \blank\circ f :\Big(\prd{y:B}P(y)\Big)\to \Big(\prd{x:A}P(f(x))\Big)
\end{equation*}
is an equivalence. To prove this fact we will make use of coherently invertible maps, which were introduced in \cref{sec:is-contr-map-is-equiv}. In the second application, \cref{thm:strong-ind-N}, we prove the strong induction principle of the natural numbers. Function extensionality is needed in order to derive the computation rule for the strong induction principle.

Many important consequences of the function extensionality axiom are left as exercises. For example, in \cref{ex:isprop_istrunc} you are asked to show that both $\iscontr(A)$ and $\istrunc{k}(A)$ are propositions, and in \cref{ex:isprop_isequiv} you are asked to show that $
\isequiv(f)$ is a proposition. The universal properties of $\emptyt$, $\unit$, and $A+B$ are left as \cref{ex:up-emptyt,ex:up-unit,ex:up-coproduct}. A few more advanced properties, such as the fact that post-composition
\begin{equation*}
  g\circ\blank : (A\to X)\to (A\to Y)
\end{equation*}
by a $k$-truncated map $g:X\to Y$ is itself a $k$-truncated map, appear in the later exercises. We encourage you to read through all of them, and get at least a basic idea of why they are true.


\subsection{Equivalent forms of function extensionality}

The function extensionality principle characterizes the identity type of an arbitrary dependent function type. It asserts that the type $f=g$ of identifications between two dependent functions is equivalent to the type of homotopies $f\htpy g$. By \cref{thm:id_fundamental}\index{fundamental theorem of identity types} there are three equivalent ways of doing this.

\begin{prp}\label{prp:funext}
  Consider a dependent function $f:\prd{x:A}B(x)$. The following are equivalent:\index{function extensionality|textbf}\index{characterization of identity type!of P-types@{of $\Pi$-types}}\index{dependent function type!characterization of identity type}
  \begin{enumerate}
  \item The \define{function extensionality principle} holds at $f$: for each $g:\prd{x:A}B(x)$, the family of maps
    \begin{equation*}
      \htpyeq:(f=g)\to (f\htpy g)
    \end{equation*}
    defined by $\htpyeq(\refl{f}):=\reflhtpy_{f}$ is a family of equivalences.
  \item The total space
    \begin{equation*}
      \sm{g:\prd{x:A}B(x)}f\htpy g
    \end{equation*}
    is contractible.
  \item
    The principle of \define{homotopy induction}\index{homotopy induction|textbf}\index{induction principle!for homotopies|textbf}:
    for any family of types $P(g,H)$ indexed by $g:\prd{x:A}B(x)$ and $H:f\htpy g$, the evaluation function
    \begin{equation*}
      \Big(\prd{g:\prd{x:A}B(x)}\prd{H:f\htpy g}P(g,H)\Big)\to P(f,\reflhtpy_f),
    \end{equation*}
    given by $s\mapsto s(f,\reflhtpy_f)$, has a section.
  \end{enumerate}
\end{prp}

\begin{proof}
  This theorem follows directly from \cref{thm:id_fundamental}.
\end{proof}

There is, however, yet a fourth condition equivalent to the function extensionality principle: the \emph{weak} function extensionality principle. The weak function extensionality principle asserts that any dependent product of contractible types is again contractible.

The following theorem is stated with respect to an arbitrary universe $\UU$, because we will use it in \cref{thm:funext-univalence} to show that the univalence axiom implies function extensionality.

\begin{thm}\label{thm:funext_wkfunext}
  Consider a universe $\UU$. The following are equivalent:\index{function extensionality}
  \begin{enumerate}
  \item The function extensionality principle holds in $\UU$: For every type family $B$ over $A$ in $\UU$ and any $f,g:\prd{x:A}B(x)$, the map
    \begin{equation*}
      \htpyeq : (f=g)\to (f\htpy g)
    \end{equation*}
    is an equivalence.
  \item The \define{weak function extensionality principle}\index{weak function extensionality|textbf}\index{function extensionality!weak function extensionality|textbf} holds in $\UU$: For every type family $B$ over $A$ in $\UU$ one has\index{contractible type!weak function extensionality|textbf}\index{is contractible!dependent function type}
    \begin{equation*}
      \Big(\prd{x:A}\iscontr(B(x))\Big)\to\iscontr\Big(\prd{x:A}B(x)\Big).
    \end{equation*}
  \end{enumerate}
\end{thm}

\begin{proof}
  First, we show that function extensionality implies weak function extensionality, suppose that each $B(a)$ is contractible with center of contraction $c(a)$ and contraction $C_a:\prd{y:B(a)}c(a)=y$. Then we take $c\defeq \lam{a}c(a)$ to be the center of contraction of $\prd{x:A}B(x)$. To construct the contraction we have to define a term of type
  \begin{equation*}
    \prd{f:\prd{x:A}B(x)}c=f.
  \end{equation*}
  Let $f:\prd{x:A}B(x)$. By function extensionality we have a map ${(c\htpy f)}\to {(c=f)}$, so it suffices to construct a term of type $c\htpy f$. Here we take $\lam{a}C_a(f(a))$. This completes the proof that function extensionality implies weak function extensionality.
  
  It remains to show that weak function extensionality implies function extensionality. By \cref{prp:funext} it suffices to show that the type
  \begin{equation*}
    \sm{g:\prd{x:A}B(x)}f\htpy g
  \end{equation*}
  is contractible for any $f:\prd{x:A}B(x)$. In order to do this, we first note that we have a section-retraction pair
  \begin{align*}
    \Big(\sm{g:\prd{x:A}B(x)}f\htpy g\Big)
    & \stackrel{i}{\longrightarrow} \Big(\prd{x:A}\sm{b:B(x)}f(x)=b\Big) \\
    & \stackrel{r}{\longrightarrow} \Big(\sm{g:\prd{x:A}B(x)}f\htpy g\Big)
  \end{align*}
  Here we have the functions
  \begin{align*}
    i & \defeq \lam{(g,H)}\lam{x}(g(x),H(x)) \\
    r & \defeq \lam{p}\pairr{\lam{x}\proj 1(p(x)),\lam{x}\proj 2(p(x))}.
  \end{align*}
  Their composite is homotopic to the identity function by the computation rule for $\Sigma$-types and the $\eta$-rule for $\Pi$-types:
  \begin{align*}
    r(i(g,H)) & \jdeq r(\lam{x}\pairr{g(x),H(x)}) \\
              & \jdeq \pairr{\lam{x}g(x),\lam{x}H(x)} \\
              & \jdeq \pairr{g,H}.
  \end{align*}
  Now we observe that the type $\prd{x:A}\sm{b:B(x)}f(x)=b$ is a product of contractible types, so it is contractible by our assumption of the weak function extensionality principle. The claim now follows, because retracts of contractible types are contractible by \cref{ex:contr_retr}.
\end{proof}

We will henceforth assume the function extensionality principle as an axiom.

\begin{axiom}[Function Extensionality]\label{axiom:funext}
  \index{function extensionality|textbf}\index{axiom!function extensionality|textbf}\index{identity type!of a Pi-type@{of a $\Pi$-type}}\index{extensionality principle!for functions|textbf}%
  For any type family $B$ over $A$, and any two dependent functions $f,g:\prd{x:A}B(x)$, the map\index{htpy-eq@{$\htpyeq$}|textbf}\index{htpy-eq@{$\htpyeq$}!is an equivalence}\index{is an equivalence!htpy-eq@{$\htpyeq$}}
  \begin{equation*}
    \htpyeq:(f=g)\to (f\htpy g)
  \end{equation*}
  is an equivalence. We will write $\eqhtpy$\index{eq-htpy@{$\eqhtpy$}|textbf} for its inverse.
\end{axiom}

\begin{rmk}
  The function extensionality axiom is added to type theory by adding the rule
  \begin{prooftree}
    \AxiomC{$\Gamma,x:A\vdash B(x)~\type$}
    \AxiomC{$\Gamma\vdash f : \prd{x:A}B(x)$}
    \AxiomC{$\Gamma\vdash g : \prd{x:A}B(x)$}
    \TrinaryInfC{$\Gamma\vdash\funext:\isequiv(\htpyeq_{f,g})$}
  \end{prooftree}
\end{rmk}

In the following theorem we extend the weak function extensionality principle to general truncation levels.

\begin{thm}\label{thm:trunc_pi}\index{k-type@{$k$-type}}
For any type family $B$ over $A$ one has\index{truncated type!dependent function type}
\begin{equation*}
\Big(\prd{x:A}\istrunc{k}(B(x))\Big)\to \istrunc{k}\Big(\prd{x:A}B(x)\Big).
\end{equation*}
\end{thm}

\begin{proof}
The theorem is proven by induction on $k\geq -2$. The base case is just the weak function extensionality principle\index{weak function extensionality}, which was shown to follow from function extensionality in \cref{thm:funext_wkfunext}.

For the inductive step, assume that the $k$-truncated types are closed under $\Pi$-types, and consider a family $B$ of $(k+1)$-truncated types. To show that the type $\prd{x:A}B(x)$ is $(k+1)$-truncated, we have to show that the type $f=g$ is $k$-truncated for every $f,g:\prd{x:A}$. By function extensionality, the type $f=g$ is equivalent to $f\htpy g$ for any two dependent functions $f,g:\prd{x:A}B(x)$. Now observe that $f\htpy g$ is a dependent product of $k$-truncated types, and therefore it is $k$-truncated by the inductive hypothesis. Since the $k$-truncated types are closed under equivalences by \cref{thm:ktype_eqv}, it follows that the type $f=g$ is $k$-truncated.
\end{proof}

\begin{cor}\label{cor:funtype_trunc}\index{truncated type!function type}
Suppose $B$ is a $k$-type. Then $A\to B$ is also a $k$-type, for any type $A$.
\end{cor}

\begin{rmk}
  It follows that $\neg A$ is a proposition for each type $A$. Note that it requires function extensionality even just to prove that $\neg P$ is a proposition for any proposition $P$.
\end{rmk}

\subsection{Identity systems on \texorpdfstring{$\Pi$}{Π}-types}

Recall from \cref{sec:structure-identity-principle} that the \emph{structure identity principle} is a way to obtain an identity system on a $\Sigma$-type. Identity systems were defined in \cref{defn:identity-system}. In this section we will describe how to obtain identity systems on a $\Pi$-type. We will first show that $\Pi$-types distribute over $\Sigma$-types\index{distributivity!of P over S@{of $\Pi$ over $\Sigma$}}. This theorem is sometimes called the \emph{type theoretic principle of choice} because it can be seen as the Curry-Howard interpretation of the axiom of choice\index{type theoretic choice|see {distributivity, of $\Pi$ over $\Sigma$}}.

\begin{thm}\label{thm:choice}
Consider a family of types $C(x,y)$ indexed by $x:A$ and $y:B(x)$. Then the map\index{choice@{$\choice$}|textbf}\index{is an equivalence!choice@{$\choice$}}
\begin{equation*}
  \choice:\Big(\prd{x:A}\sm{y:B(x)}C(x,y)\Big)\to \Big(\sm{f:\prd{x:A}B(x)}\prd{x:A}C(x,f(x))\Big)
\end{equation*}
given by
\begin{equation*}\label{eq:choice}
  \choice(h):=(\lam{x}\proj 1(h(x)),\lam{x}\proj 2(h(x))).
\end{equation*}
is an equivalence.
\end{thm}

\begin{proof}
  We define the map\index{choice -1@{$\choice^{-1}$}|textbf}
  \begin{equation*}
    \choice^{-1}:\Big(\sm{f:\prd{x:A}B(x)}\prd{x:A}C(x,f(x))\Big)\to \prd{x:A}\sm{y:B(x)}C(x,y)
  \end{equation*}
  by $\choice^{-1}(f,g):=\lam{x}(f(x),g(x))$. Then we have to construct homotopies
  \begin{equation*}
    \choice\circ\choice^{-1}\htpy\idfunc,\qquad\text{and}\qquad
    \choice^{-1}\circ\choice\htpy\idfunc.
  \end{equation*}
  For the first homotopy it suffices to construct an identification
  \begin{equation*}
    \choice(\choice^{-1}(f,g))=(f,g)
  \end{equation*}
  for any $f:\prd{x:A}B(x)$ and any $g:\prd{x:A}C(x,f(x))$. We compute the left-hand side as follows:
  \begin{align*}
    \choice(\choice^{-1}(f,g))
    & \jdeq \choice(\lam{x}(f(x),g(x))) \\
    & \jdeq (\lam{x}f(x),\lam{x}g(x)).
  \end{align*}
  By the $\eta$-rule for $\Pi$-types we have the judgmental equalities $f\jdeq \lam{x}f(x)$ and $g\jdeq\lam{x}g(x)$. Therefore we have the identification
  \begin{equation*}
    \refl{(f,g)}:\choice(\choice^{-1}(f,g))=(f,g).
  \end{equation*}
  This completes the construction of the first homotopy.

  For the second homotopy we have to construct an identification
  \begin{equation*}
    \choice^{-1}(\choice(h))=h
  \end{equation*}
  for any $h:\prd{x:A}\sm{y:B(x)}C(x,y)$. We compute the left-hand side as follows:
  \begin{align*}
    \choice^{-1}(\choice(h))
    & \jdeq \choice^{-1}(\lam{x}\proj 1(h(x)),(\lam{x}\proj 2(h(x)))) \\
    & \jdeq \lam{x}(\proj 1(h(x)),\proj 2(h(x)))
  \end{align*}
  However, it is \emph{not} the case that $(\proj 1(h(x)),\proj 2(h(x)))\jdeq h(x)$ for any $h:\prd{x:A}\sm{y:B(x)}C(x,y)$. Nevertheless, we have the identification
  \begin{equation*}
    \eqpair(\refl{},\refl{}):(\proj 1(h(x)),\proj 2(h(x)))= h(x).
  \end{equation*}
  Therefore we obtain the required homotopy by function extensionality:
  \begin{equation*}
    \lam{h}\eqhtpy(\lam{x}\eqpair(\refl{\proj 1(h(x))},\refl{\proj 2(h(x))})):\choice^{-1}\circ\choice\htpy\idfunc.\qedhere
  \end{equation*}
\end{proof}

The fact that $\Pi$-types distribute over $\Sigma$-types has many useful consequences. The most straightforward consequence is the following.

\begin{cor}
  For any two types $A$ and $B$, and any type family $C$ over $B$, we have an equivalence
\begin{equation*}
  \Big(A\to\sm{y:B}C(y)\Big)\simeq\Big(\sm{f:A\to B}\prd{x:A}C(f(x))\Big).
\end{equation*}
\end{cor}

Another direct consequence of the distributivity of $\Pi$-types over $\Sigma$-types is the fact that
\begin{equation*}
  \prd{b:B}\fib{f}{b}\simeq\sm{g:B\to A}f\circ g\htpy \idfunc.
\end{equation*}
In the following corollary we use the distributivity of $\Pi$-types over $\Sigma$-types to show that dependent functions are sections of projection maps.

\begin{cor}\label{ex:pi_sec}
  Consider a type family $B$ over $A$, and consider the projection map
  \begin{equation*}
    \proj 1:\big(\sm{x:A}B(x)\big) \to A.  
  \end{equation*}
  Then we have an equivalence
  \begin{equation*}
    \sections(\proj 1)\simeq\prd{x:A}B(x).
  \end{equation*}
\end{cor}

\begin{proof}
  \cref{thm:choice} gives the first equivalence in the following calculation:
  \begin{align*}
    \sm{h:A\to\sm{x:A}B(x)} \proj 1\circ h\htpy \idfunc
    & \simeq \sm{(f,g):\sm{f:A\to A}\prd{x:A}B(f(x))} f\htpy \idfunc \\
    & \simeq \sm{(f,H):\sm{f:A\to A}f\htpy \idfunc}\prd{x:A}B(f(x)) \\
    & \simeq \prd{x:A}B(x)
  \end{align*}
  In the second equivalence we used \cref{ex:sigma_swap} to swap the family $f\mapsto \prd{x:A}B(f(x))$ with the family $f\mapsto f\htpy\idfunc$, and in the third equivalence we used the fact that
  \begin{equation*}
    \sm{f:A\to A}f\htpy\idfunc
  \end{equation*}
  is contractible, with center of contraction $(\idfunc,\reflhtpy)$. One way to see that it is contractible is by \cref{ex:is-equiv-inv-htpy}. A direct way to see this, is by another application of \cref{thm:choice}. This gives an equivalence
  \begin{equation*}
    \left(\sm{f:A\to A}f\htpy\idfunc\right)\simeq \left(\prd{x:A}\sm{y:A}y=x\right),
  \end{equation*}
  and the right-hand side is a product of contractible types.
\end{proof}

In the final application of distributivity of $\Pi$-types over $\Sigma$-types we obtain a general way of constructing identity systems of $\Pi$-types.

\begin{thm}\label{cor:Eq-Pi}
  Consider a family $B$ of types over $A$, and for each $b:B(a)$ consider an identity system $E(b)$ at $b$. Furthermore, consider a dependent function $f:\prd{x:A}B(x)$. Then the family of types
  \begin{equation*}
    \prd{x:A}E(f(x),g(x))
  \end{equation*}
  indexed by $g:\prd{x:A}B(x)$ is an identity system at $f$.\index{identity system!of a dependent function type}\index{dependent function type!identity system}
\end{thm}

\begin{proof}
  By \cref{thm:id_fundamental} it suffices to show that the type
  \begin{equation*}
    \sm{g:\prd{x:A}B(x)}\prd{x:A}E(f(x),g(x))
  \end{equation*}
  is contractible. By \cref{thm:choice} it follows that this type is equivalent to the type
  \begin{equation*}
    \prd{x:A}\sm{y:B(x)}E(f(x),y).
  \end{equation*}
  This is a product of contractible types because each $E(f(x))$ is an identity system at $f(x):B(x)$. This product is therefore contractible by the weak function extensionality principle.
\end{proof}

\subsection{Universal properties}
The function extensionality principle allows us to prove \emph{universal properties}. Universal properties are characterizations of all maps out of or into a given type, so they are very important. Among other applications, universal properties characterize a type up to equivalence. We prove here the universal properties of dependent pair types and of identity types. In the exercises, you are asked to prove the universal properties of $\unit$, $\emptyt$, and coproducts.

\subsubsection*{The universal property of $\Sigma$-types}
\index{universal property!of S-types@{of $\Sigma$-types}|(}
\index{dependent universal property!of S-types@{of $\Sigma$-types}|(}
\index{dependent pair type!universal property|(}
\index{dependent pair type!dependent universal property|(}

The \define{universal property of $\Sigma$-types} characterizes maps \emph{out of} a dependent pair type $\sm{x:A}B(x)$. It asserts that the map\index{ev-pair@{$\evpair$}|textbf}
\begin{equation*}
\evpair:\Big(\Big(\sm{x:A}B(x)\Big)\to X\Big)\to \Big(\prd{x:A}(B(x)\to X)\Big),
\end{equation*}
given by $f\mapsto\lam{x}\lam{y}f(x,y)$, is an equivalence for any type $X$. In fact, we will prove a slight generalization of this universal property. We will prove the \define{dependent universal property} of $\Sigma$-types, which characterizes \emph{dependent} functions out of $\sm{x:A}B(x)$.

\begin{thm}\label{thm:up-sigma}
  \index{dependent universal property!of S-types@{of $\Sigma$-types}|textbf}
  \index{dependent pair type!dependent universal property|textbf}
Let $B$ be a type family over $A$, and let $C$ be a type family over $\sm{x:A}B(x)$. Then the map\index{ev-pair@{$\evpair$}!is an equivalence}
\begin{equation*}
\evpair:\Big(\prd{z:\sm{x:A}B(x)}C(z)\Big)\to \Big(\prd{x:A}\prd{y:B(x)}C(x,y)\Big),
\end{equation*}
given by $f\mapsto\lam{x}\lam{y}f(x,y)$, is an equivalence.\index{is an equivalence!ev-pair@{$\evpair$}}
\end{thm}

\begin{proof}
The map in the converse direction is obtained by the induction principle of $\Sigma$-types. It is simply the map
\begin{equation*}
\indSigma : \Big(\prd{x:A}\prd{y:B(x)}C(x,y)\Big)\to \Big(\prd{z:\sm{x:A}B(x)}C(z)\Big).
\end{equation*}
By the computation rule for $\Sigma$-types we have the homotopy
\begin{equation*}
\reflhtpy:\evpair\circ\indSigma\htpy\idfunc.
\end{equation*}
This shows that $\indSigma$ is a section of $\evpair$.

To show that $\indSigma\circ\evpair\htpy\idfunc$ we will apply the function extensionality principle. Therefore it suffices to show that $\indSigma(\lam{x}\lam{y}f(x,y))=f$. We apply function extensionality again, so it suffices to show that
\begin{equation*}
\prd{t:\sm{x:A}B(x)}\indSigma\big(\lam{x}\lam{y}f(x,y)\big)(t)=f(t).
\end{equation*}
We obtain this homotopy by another application of $\Sigma$-induction. 
\end{proof}

\begin{cor}\label{cor:times_up_out}
  \index{universal property!of cartesian products|textbf}
  \index{cartesian product type!universal property|textbf}
Let $A$, $B$, and $X$ be types. Then the map\index{ev-pair@{$\evpair$}!is an equivalence}\index{is an equivalence!ev-pair@{$\evpair$}}
\begin{equation*}
\evpair: (A\times B \to X)\to (A\to (B\to X))
\end{equation*}
given by $f\mapsto\lam{a}\lam{b}f(a,b)$ is an equivalence.
\end{cor}
\index{universal property!of S-types@{of $\Sigma$-types}|)}
\index{dependent universal property!of S-types@{of $\Sigma$-types}|)}
\index{dependent pair type!universal property|)}
\index{dependent pair type!dependent universal property|)}

\subsubsection*{The universal property of identity types}
\index{identity type!universal property|(}
\index{identity type!dependent universal property|(}
\index{universal property!of identity types|(}
\index{dependent universal property!of identity types|(}
The universal property of identity types is the fact that families of maps out of the identity type are uniquely determined by their action on the reflexivity identification. More precisely, the map
\begin{equation*}
  \evrefl:\Big(\prd{x:A}(a=x)\to B(x)\Big)\to B(a)
\end{equation*}
given by $\lam{f} f(a,\refl{a})$ is an equivalence, for every type family $B$ over $A$. Since this result is similar to the Yoneda lemma of category theory, the universal property of identity types is sometimes referred to as the \emph{type theoretic Yoneda lemma}. We will prove the \emph{dependent} universal property of identity types, a slight generalization of the universal property.

\begin{thm}\label{thm:yoneda}
  \index{dependent universal property!of identity types|textbf}
  \index{identity type!dependent universal property|textbf}
Consider a type $A$ equipped with $a:A$, and consider a family of types $B(x,p)$ indexed by $x:A$ and $p:a=x$. Then the map\index{ev-refl@{$\evrefl$}|textbf}
\begin{equation*}
\evrefl:\Big(\prd{x:A}\prd{p:a=x}B(x,p)\Big)\to B(a,\refl{a}),
\end{equation*}
given by $\lam{f} f(a,\refl{a})$, is an equivalence.\index{is an equivalence!ev-refl@{$\evrefl$}}\index{ev-refl@{$\evrefl$}!is an equivalence}
\end{thm}

\begin{proof}
  The inverse is the function
  \begin{equation*}
    \pathind_a : B(a,\refl{a})\to \prd{x:A}\prd{p:a=x}B(x,p).
  \end{equation*}
  It is immediate from the computation rule of the path induction principle that $\evrefl\circ\pathind_a\htpy \idfunc$.

To see that $\pathind_a\circ \evrefl\htpy\idfunc$, let $f:\prd{x:A}(a=x)\to B(x,p)$. To show that $\pathind_a(f(a,\refl{a}))=f$ we apply function extensionality twice. Therefore it suffices to show that
\begin{equation*}
\prd{x:A}\prd{p:a=x} \pathind_a(f(a,\refl{a}),x,p)=f(x,p).
\end{equation*}
This follows by path induction on $p$, since $\pathind_a(f(a,\refl{a}),a,\refl{a})\jdeq f(a,\refl{a})$ by the computation rule of path induction.
\end{proof}
\index{identity type!universal property|)}
\index{identity type!dependent universal property|)}
\index{universal property!of identity types|)}
\index{dependent universal property!of identity types|)}

\subsection{Composing with equivalences}

We show in the following theorem that a map $f:A\to B$ is an equivalence if and only if precomposing by $f$ is an equivalence.

\begin{thm}\label{ex:equiv_precomp}
  \index{equivalence!precomposition}
  For any map $f:A\to B$, the following are equivalent:
  \begin{enumerate}
  \item $f$ is an equivalence.
  \item For any type family $P$ over $B$ the map
    \begin{equation*}
      \Big(\prd{y:B}P(y)\Big)\to\Big(\prd{x:A}P(f(x))\Big)
    \end{equation*}
    given by $h\mapsto h\circ f$ is an equivalence.
  \item For any type $X$ the map
    \begin{equation*}
      (B\to X)\to (A\to X)
    \end{equation*}
    given by $g\mapsto g\circ f$ is an equivalence. 
  \end{enumerate}
\end{thm}

\begin{proof}
To show that (i) implies (ii), we first recall from \cref{lem:coherently-invertible} that any equivalence is also coherently invertible. Therefore $f$ comes equipped with
\begin{align*}
g & : B \to A\\
G & : f\circ g \htpy \idfunc[B] \\
H & : g\circ f \htpy \idfunc[A] \\
K & : G\cdot f \htpy f\cdot H.
\end{align*}
Then we define the inverse of $\blank\circ f$ to be the map
\begin{equation*}
\varphi:\Big(\prd{x:A}P(f(x))\Big)\to\Big(\prd{y:B}P(y)\Big)
\end{equation*}
given by $h\mapsto \lam{y}\tr_P(G(y),h(g(y)))$. 

To see that $\varphi$ is a section of $\blank\circ f$, let $h:\prd{x:A}P(f(x))$. By function extensionality it suffices to construct a homotopy $\varphi(h)\circ f\htpy h$. In other words, we have to show that
\begin{equation*}
\tr_P(G(f(x)),h(g(f(x)))=h(x)
\end{equation*}
for any $x:A$. Now we use the additional homotopy $K$ from our assumption that $f$ is coherently invertible. Since we have $K(x):G(f(x))=\ap{f}{H(x)}$ it suffices to show that
\begin{equation*}
\tr_P(\ap{f}{H(x)},h(g(f(x))))=h(x).
\end{equation*}
A simple path-induction argument yields that
\begin{equation*}
\tr_P(\ap{f}{p})\htpy \tr_{P\circ f}(p)
\end{equation*}
for any path $p:x=y$ in $A$, so it suffices to construct an identification
\begin{equation*}
\tr_{P\circ f}(H(x),h(g(f(x))))=h(x).
\end{equation*}
We have such an identification by $\apd{h}{H(x)}$.

To see that $\varphi$ is a retraction of $\blank\circ f$, let $h:\prd{y:B}P(y)$. By function extensionality it suffices to construct a homotopy $\varphi(h\circ f)\htpy h$. In other words, we have to show that
\begin{equation*}
\tr_P(G(y),h(f(g(y))))=h(y)
\end{equation*}
for any $y:B$. We have such an identification by $\apd{h}{G(y)}$. This completes the proof that (i) implies (ii).

Note that (iii) is an immediate consequence of (ii), since we can just choose $P$ to be the constant family $X$.

It remains to show that (iii) implies (i). Suppose that
\begin{equation*}
\blank\circ f:(B\to X)\to (A\to X)
\end{equation*}
is an equivalence for every type $X$. Then its fibers are contractible by \cref{thm:contr_equiv}. In particular, choosing $X\jdeq A$ we see that the fiber
\begin{equation*}
\fib{\blank\circ f}{\idfunc[A]}\jdeq \sm{h:B\to A}h\circ f=\idfunc[A]
\end{equation*}
is contractible. Thus we obtain a function $h:B\to A$ and a homotopy $H:h\circ f\htpy\idfunc[A]$ showing that $h$ is a retraction of $f$. We will show that $h$ is also a section of $f$. To see this, we use that the fiber
\begin{equation*}
\fib{\blank\circ f}{f}\jdeq \sm{i:B\to B} i\circ f=f
\end{equation*}
is contractible (choosing $X\defeq B$). 
Of course we have $(\idfunc[B],\refl{f})$ in this fiber. However we claim that there also is an identification $p:(f\circ h)\circ f=f$, showing that $(f\circ h,p)$ is in this fiber, because
\begin{align*}
(f\circ h)\circ f & \jdeq f\circ (h\circ f) \\
& = f\circ \idfunc[A] \\
& \jdeq f
\end{align*}
From the contractibility of the fiber we obtain an identification $(\idfunc[B],\refl{f})=(f\circ h,p)$. In particular we obtain that $\idfunc[B]=f\circ h$, showing that $h$ is a section of $f$.
\end{proof}

\subsection{The strong induction principle of \texorpdfstring{$\N$}{ℕ}}
\index{strong induction principle!of N@{of $\N$}|(}
\index{natural numbers!strong induction principle|(}

In the final application of the function extensionality principle we prove the strong induction principle for the type of natural numbers. Function extensionality is used to derive the computation rules of the strong induction principle.

\begin{thm}[Strong induction for the natural numbers]\label{thm:strong-ind-N}
  \index{strong induction principle!of N@{of $\N$}|textbf}
  \index{natural numbers!strong induction principle|textbf}
  Consider a type family $P$ over $\N$ equipped with
  \begin{align*}
    p_0 & : P(0) \\
    p_S & : \prd{n:\N}\Big(\prd{m:\N}(m\leq n)\to P(m)\Big)\to P(n+1).
  \end{align*}
  Then there is a dependent function\index{strong-ind@{$\strongindN$}|textbf}
  \begin{equation*}
    \strongindN(p_0,p_S) : \prd{n:\N}P(n)
  \end{equation*}
  that satisfies the following computation rules
  \begin{align*}
    \strongindN(p_0,p_S,0) & = p_0 \\
    \strongindN(p_0,p_S,n+1) & = p_S(n,(\lam{m}\lam{p}\strongindN(p_0,p_S,m))).
  \end{align*}
\end{thm}

In order to construct $\strongindN(p_0,p_S)$, we first define the type family $\tilde{P}$ over $\N$ by
\begin{equation*}
  \tilde{P}(n)\defeq \prd{m:\N} (m\leq n)\to P(m).
\end{equation*}
The idea is then to first use $p_0$ and $p_S$ to construct
\begin{align*}
  \tilde{p}_0 & : \tilde{P}(0) \\
  \tilde{p}_S & :\prd{n:\N}\tilde{P}(n)\to\tilde{P}(n+1).
\end{align*}
The ordinary induction principle of $\N$ then gives a function
\begin{equation*}
  \indN(\tilde{p}_0,\tilde{p}_S):\prd{n:\N}\tilde{P}(n),
\end{equation*}
which can be used to define a function $\prd{n:\N}P(n)$.

Before we start by the proof of \cref{thm:strong-ind-N} we state two lemmas in which we construct $\tilde{p}_0$ and $\tilde{p}_S$ with computation rules of their own. We will assume a type family $P$ over $\N$ equipped with
  \begin{align*}
    p_0 & : P(0) \\
    p_S & : \prd{n:\N}\tilde{P}(n) \to P(n+1),
  \end{align*}
as in the hypotheses of \cref{thm:strong-ind-N}.

\begin{lem}
  There is an element $\tilde{p}_0:\tilde{P}(0)$ that satisfies the judgmental equality
  \begin{equation*}
    \tilde{p}_0(0,p)\jdeq p_0
  \end{equation*}
  for any $p:0\leq 0$.
\end{lem}

\begin{proof}
  The fact that we have such a dependent function $\tilde{p}_0$ follows immediately by induction on $m$ and $p:m\leq 0$.
\end{proof}

\begin{lem}\label{lem:succ-strong-ind-N}
  There is a function
  \begin{equation*}
    \tilde{p}_S : \prd{n:\N}\tilde{P}(n)\to\tilde{P}(n+1)
  \end{equation*}
  equipped with
  \begin{enumerate}
  \item an identification
    \begin{equation*}
      \tilde{p}_S(n,H,m,p) = H(m,q)
    \end{equation*}
    for every $H:\tilde{P}(n)$ and every $p:m\leq n+1$ and $q:m\leq n$, and
  \item an identification
    \begin{equation*}
      \tilde{p}_S(n,H,n+1,p) = p_S(n,H)
    \end{equation*}
    for every $p:n+1\leq n+1$.
  \end{enumerate}
\end{lem}

\begin{proof}
  To define the function $\tilde{p}_S(n,H)$, note that there is a function
  \begin{equation*}
    f : (m\leq n+1)\to (m\leq n)+(m=n+1)\tag{\textasteriskcentered}
  \end{equation*}
  which can be defined by induction on $n$ and $m$. Using the fact that the domain and codomain of this map are both propositions, this function is easily seen to be an equivalence. Therefore we define first a function
  \begin{equation*}
    h(n,H) :\prd{m:\N} ((m\leq n)+(m=n+1))\to P(m)
  \end{equation*}
  by case analysis on $x:(m\leq n)+(m=n+1)$. There are two cases to consider: one where we have $q:m\leq n$, and one where we have $q:m=n+1$. Note that in the second case it suffices to make a definition for $q\jdeq \refl{}$. Therefore we define
  \begin{equation*}
    h(n,H,m,x) =
    \begin{cases}
      H(m,q) & \text{if }x\jdeq\inl(q)\\
      p_S(n,H) & \text{if }x\jdeq\inr(\refl{}).
    \end{cases}
  \end{equation*}
  Now we define $\tilde{p}_S$ by
  \begin{equation*}
    \tilde{p}_S(n,H,m,p)\defeq h(n,H,m,f(p)),
  \end{equation*}
  where $f:(m\leq n+1)\to (m\leq n)+(m=n+1)$ is the map we mentioned in (\textasteriskcentered).
  
  To construct the identifications claimed in (i) and (ii), note that there is an equivalence
  \begin{equation*}
    \big(\tilde{p}_S(n,H,m,p)=y\big)\simeq \big(h(n,H,m,x)=y\big),
  \end{equation*}
  for any $y:P(m)$. This equivalence is obtained from the fact that $f(p)=x$ for any $x:(m\leq n)+(m=n+1)$, i.e., the fact that $(m\leq n)+(m=n+1)$ is a proposition. Now the identifications in (i) and (ii) are obtained as a simple consequence of the computation rule for coproducts.
\end{proof}

We are now ready to finish the proof of \cref{thm:strong-ind-N}.

\begin{proof}[Proof of \cref{thm:strong-ind-N}]
  Using $\tilde{p}_0$ and $\tilde{p}_S$, we obtain by induction on $n$ a function
  \begin{equation*}
    \tilde{s}:\prd{n:\N}\tilde{P}(n)
  \end{equation*}
  satisfying the computation rules
  \begin{align*}
    \tilde{s}(0) & \jdeq \tilde{p}_0 \\
    \tilde{s}(n+1) & \jdeq \tilde{p}_S(n,\tilde{s}(n)).
  \end{align*}
  Now we define
  \begin{equation*}
    \strongindN(p_0,p_S,n) \defeq \tilde{s}(n,n,\reflleqN(n)),
  \end{equation*}
  where $\reflleqN(n):n\leq n$ is the proof of reflexivity of $\leq$.

  It remains to show that $\strongindN$ satisfies the computation rules of the strong induction principle. The identification that computes $\strongindN$ at $0$ is easy to obtain, because we have the judgmental equalities
  \begin{align*}
    \strongindN(p_0,p_S,0) & \jdeq \tilde{s}(0,0,\reflleqN(0)) \\
                                & \jdeq \tilde{p}_{0}(0,\reflleqN(0)) \\
                                & \jdeq p_0.
  \end{align*}
  To construct the identification that computes $\strongindN$ at a successor, we start by a similar computation:
  \begin{align*}
    \strongindN(p_0,p_S,n+1) & \jdeq \tilde{s}(n+1,n+1,\reflleqN(n+1)) \\
                                   & \jdeq \tilde{p}_S(n,\tilde{s}(n),n+1,\reflleqN(n+1)) \\
    & = p_S(n,\tilde{s}(n)).
  \end{align*}
  The last identification is obtained from \cref{lem:succ-strong-ind-N} (ii).
  Therefore we see that, in order to show that
  \begin{equation*}
    p_S(n,\tilde{s}(n))=p_S(n,(\lam{m}\lam{p}\tilde{s}(m,m,\reflleqN(m)))),
  \end{equation*}
  we need to prove that
  \begin{equation*}
    \tilde{s}(n)=\lam{m}\lam{p}\tilde{s}(m,m,\reflleqN(m)).
  \end{equation*}
  Here we apply function extensionality, so it suffices to show that
  \begin{equation*}
    \tilde{s}(n,m,p)=\tilde{s}(m,m,\reflleqN(m))
  \end{equation*}
  for every $m:\N$ and $p:m\leq n$. We proceed by induction on $n:\N$. The base case is trivial. For the inductive step, we note that
  \begin{align*}
    \tilde{s}(n+1,m,p)=\tilde{p}_S(n,\tilde{s}(n),m,p)=\begin{cases}\tilde{s}(n,m,p) & \text{if }m\leq n \\
    p_S(n,\tilde{s}(n)) & \text{if }m=n+1.\end{cases}
  \end{align*}
  Therefore it follows by the inductive hypothesis that
  \begin{equation*}
    \tilde{s}(n+1,m,p)=\tilde{s}(m,m,\reflleqN(m))
  \end{equation*}
  if $m\leq n$ holds. In the remaining case, where $m=n+1$, note that we have
  \begin{align*}
    \tilde{s}(n+1,n+1,\reflleqN(n+1)) & = \tilde{p}_S(n,\tilde{s}(n),n+1,\reflleqN(n+1)) \\
    & = p_S(n,\tilde{s}(n)).
  \end{align*}
  Therefore we see that we also have an identification
  \begin{equation*}
    \tilde{s}(n+1,m,p)=\tilde{s}(m,m,\reflleqN(m))
  \end{equation*}
  when $m=n+1$. This completes the proof of the computation rules for the strong induction principle of $\N$.
\end{proof}
\index{strong induction principle!of N@{of $\N$}|)}
\index{natural numbers!strong induction principle|)}

\begin{exercises}
  \exitem \label{ex:is-equiv-inv-htpy}Show that the functions\index{inv-htpy@{$\invhtpy$}!is an equivalence}\index{concat-htpy@{$\concathtpy$}!is a family of equivalences}\index{concat-htpy'@{$\concathtpy'$}!is a family of equivalences}\index{is an equivalence!inv-htpy@{$\invhtpy$}}\index{is an equivalence!concat-htpy(H)@{$\concathtpy(H)$}}\index{is an equivalence!concat-htpy'(K)@{$\concathtpy'(K)$}}
  \begin{align*}
    \invhtpy & : (f \htpy g) \to (g \htpy f) \\
    \concathtpy(H) & : (g \htpy h) \to (f \htpy h) \\
    \concathtpy'(K) & : (f \htpy g) \to (f \htpy h)
  \end{align*}
  are equivalences for every $f,g,h : \prd{x:A}B(x)$. Here, $\concathtpy'(K)$ is the function defined by $H\mapsto \ct{H}{K}$.
  \exitem Characterize the identity types of the following types:
  \begin{subexenum}
  \item The type $\sm{h:A\to B}h(a)=b$ of \define{pointed maps}\index{pointed map|textbf}, where $a:A$ and $b:B$ are given.
  \item The type $\sm{h:A\to B}f\htpy g\circ h$ of commuting triangles
    \begin{equation*}
      \begin{tikzcd}[column sep=tiny]
        A \arrow[rr,"h"] \arrow[dr,swap,"f"] & & B \arrow[dl,"g"] \\
        & X,
      \end{tikzcd}
    \end{equation*}
    where $f:A\to X$ and $g:B\to X$ are given.
  \item The type $\sm{h:X\to Y}h\circ f\htpy g$ of commuting triangles
    \begin{equation*}
      \begin{tikzcd}[column sep=tiny]
        & A \arrow[dl,swap,"f"] \arrow[dr,"g"] \\
        X \arrow[rr,swap,"h"] & & Y,
      \end{tikzcd}
    \end{equation*}
    where $f:A\to X$ and $g:A\to Y$ are given.
  \item The type $\sm{i:A\to X}\sm{j:B\to Y}j\circ f\htpy g\circ i$ of commuting squares
    \begin{equation*}
      \begin{tikzcd}
        A \arrow[d,swap,"f"] \arrow[r,"i"] & X \arrow[d,"g"] \\
        B \arrow[r,swap,"j"] & Y,
      \end{tikzcd}
    \end{equation*}
    where $f:A\to B$ and $g:X\to Y$ are given.
  \end{subexenum}
  \exitem \label{ex:isprop_istrunc}
  \begin{subexenum}
  \item Show that for any type $A$ the type $\iscontr(A)$ is a proposition\index{is-contr(A)@{$\iscontr(A)$}!is a proposition}\index{is contractible!is a property}\index{is a proposition!is-contr(A)@{$\iscontr(A)$}}.
  \item Show that for any type $A$ and any $k\geq-2$, the type $\istrunc{k}(A)$ is a proposition.\index{istrunc@{$\istrunc{k}$}!is a proposition}\index{is a proposition!istrunc(A)@{$\istrunc{k}(A)$}}
  \end{subexenum}
  \exitem \label{ex:isprop_isequiv}Let $f:A\to B$ be a function.
  \begin{subexenum}
  \item Show that if $f$ is an equivalence, then the type $\sm{g:B\to A}f\circ g\htpy \idfunc$ of sections of $f$ is contractible.
  \item Show that if $f$ is an equivalence, then the type $\sm{h:B\to A}h\circ f\htpy \idfunc$ of retractions of $f$ is contractible.
  \item Show that $\isequiv(f)$ is a proposition.\index{is-equiv(f)@{$\isequiv(f)$}!is a proposition}\index{is a proposition!is-equiv(f)@{$\isequiv(f)$}}
  \item Show that for any two equivalences $e,e':A\simeq B$, the canonical map
    \begin{equation*}
      (e=e')\to (e\htpy e')
    \end{equation*}
    is an equivalence.
  \item Show that the type $A\simeq B$ is a $k$-type if both $A$ and $B$ are $k$-types.\index{is a truncated type!A simeq B@{$A\simeq B$}}
  \end{subexenum}
  \exitem
  \begin{subexenum}
  \item Show that $\pathsplit(f)$\index{path-split!is a proposition}\index{is a proposition!is-path-split(f)@{$\pathsplit(f)$}} and $\iscohinvertible(f)$\index{is-coh-invertible(f)@{$\iscohinvertible(f)$}!is a proposition}\index{is a proposition!is-coh-invertible(f)@{$\iscohinvertible(f)$}} are propositions for any map $f:A\to B$. Conclude that we have equivalences\index{is-equiv(f)@{$\isequiv(f)$}!is-equiv(f) path-split(f)@{$\isequiv(f)\eqvsym\pathsplit(f)$}}\index{is-equiv(f)@{$\isequiv(f)$}!is-equiv(f) is-coh-invertible(f)@{$\isequiv(f)\eqvsym\iscohinvertible(f)$}}
    \begin{equation*}
      \isequiv(f) \eqvsym \pathsplit(f) \eqvsym \iscohinvertible(f).
    \end{equation*}
  \item \label{ex:idfunc_autohtpy}Construct for any type $A$ an equivalence\index{has-inverse(f)@{$\hasinverse(f)$}!has-inverse(id) id htpy id@{$\hasinverse(\idfunc)\simeq (\idfunc\htpy\idfunc)$}}
    \begin{equation*}
      \eqv{\hasinverse(\idfunc[A])}{\Big(\idfunc[A]\htpy\idfunc[A]\Big)}.
    \end{equation*}
    Note: We will use this fact in \cref{ex:is_invertible_id_S1} to show that there
    are types for which $\hasinverse(\idfunc[A])\not\eqvsym\isequiv(\idfunc[A])$.
  \end{subexenum}
  \exitem \label{ex:up-emptyt}Consider a type $A$. Show that the following are equivalent:
  \begin{enumerate}
  \item The type $A$ is empty.
  \item \label{item:dup-empty}The type $\prd{x:A}P(x)$ is contractible for any family $P$ of types over $A$. This property is the \define{dependent universal property of an empty type}\index{dependent universal property!of empty types|textbf}\index{empty type!dependent universal property|textbf}.
  \item \label{item:up-empty}The type $A\to X$ is contractible for any type $X$. This property is the \define{universal property of an empty type}\index{universal property!of empty types|textbf}\index{empty type!universal property|textbf}.
  \end{enumerate}
  \exitem \label{ex:up-unit}Consider a type $A$. Show that the following are equivalent:
  \begin{enumerate}
  \item \label{item:is-contr}The type $A$ is contractible.
  \item \label{item:dup-unit}The type $A$ comes equipped with a point $a:A$, and the map
    \begin{equation*}
      \Big(\prd{x:A}P(x)\Big)\to P(a)
    \end{equation*}
    given by $f\mapsto f(a)$ is an equivalence for any type family $P$ over $A$. This property is the \define{dependent universal property of a contractible type}\index{dependent universal property!of contractible types|textbf}\index{contractible type!dependent universal property|textbf}.
  \item \label{item:up-unit}The type $A$ comes equipped with a point $a:A$, and the map
    \begin{equation*}
      (A\to X)\to X
    \end{equation*}
    given by $f\mapsto f(a)$ is an equivalence for any type $X$. This property is the \define{universal property of a contractible type}\index{universal property!of contractible types|textbf}\index{contractible type!universal property|textbf}.
  \item The type $A$ comes equipped with a point $a:A$, and the map
    \begin{equation*}
      (A\to A)\to A
    \end{equation*}
    given by $f\mapsto f(a)$ is an equivalence.
  \item \label{item:is-equiv-diag-universal}The map
    \begin{equation*}
      X\to (A\to X)
    \end{equation*}
    given by $x\mapsto\lam{y}x$ is an equivalence for any type $X$.
  \item \label{item:is-equiv-diag}The map
    \begin{equation*}
      A\to (A\to A)
    \end{equation*}
    given by $x\mapsto\lam{y}x$ is an equivalence.
  \end{enumerate}
  \exitem \label{ex:up-coproduct}
  Consider two types $A$ and $B$. Show that the map
  \begin{equation*}
    \Big(\prd{z:A+B}P(z)\Big)\to\Big(\prd{x:A}P(\inl(x))\Big)\times\Big(\prd{y:B}P(\inr(b))\Big)
  \end{equation*}
  given by $f\mapsto (f\circ \inl,f\circ \inr)$ is an equivalence for any type family $P$ over $A+B$. This property is the \define{dependent universal property of the coproduct of $A$ and $B$}\index{dependent universal property!of coproducts|textbf}\index{coproduct!dependent universal property|textbf}. Conclude that the map
  \begin{equation*}
    (A+B\to X)\to (A\to X)\times (B\to X)
  \end{equation*}
  given by $f\mapsto (f\circ \inl,f\circ \inr)$ is an equivalence for any type $X$. This latter property is the \define{universal property of the coproduct of $A$ and $B$}\index{universal property!of coproducts|textbf}\index{coproduct!universal property|textbf}.
  \exitem\label{ex:uniqueness-identity-type}Consider a type $A$ equipped with an element $a:A$ and consider a type family $B$ over $A$ equipped with an element $b:B(a)$. Show that the following are equivalent:\index{identity type!uniqueness}
  \begin{enumerate}
  \item The map
    \begin{equation*}
      \ev_b:\Big(\prd{x:A}B(x)\to C(x)\Big)\to C(a)
    \end{equation*}
    given by $\ev_b(h)\defeq h(a,b)$ is an equivalence for any type family $C$ over $A$.
  \item The map
    \begin{equation*}
      h:\prd{x:A}(a=x)\to B(x)
    \end{equation*}
    given by $h(a,\refl{})\defeq b$ is an equivalence.
  \end{enumerate}
  \exitem Prove the \define{universal property of $\N$}\index{universal property!of N@{of $\N$}|textbf}\index{natural numbers!universal property|textbf}: For any type $X$ equipped with $x:X$ and $f:X\to X$, the type
  \begin{equation*}
    \sm{h:\N\to X} (h(\zeroN)= x)\times (h\circ \succN\htpy f\circ h)
  \end{equation*}
  is contractible.
  \exitem Show that $\N$ satisfies \define{ordinal induction}\index{ordinal induction!of N@{of $\N$}}\index{natural numbers!ordinal induction|textbf}, i.e., construct for any type family $P$ over $\N$ a function $\ordindN$ of type
  \begin{equation*}
    \Big(\prd{k:\N} \Big(\prd{m:\N} (m< k) \to P(m)\Big)\to P(k)\Big) \to \prd{n:\N}P(n).
  \end{equation*}
  Moreover, prove that
  \begin{equation*}
    \ordindN(h,n)=h(n,\lam{m}\lam{p}\ordindN(h,m))
  \end{equation*}
  for any $n:\N$ and any $h:\prd{k:\N}\Big(\prd{m:\N}(m<k)\to P(m)\Big)\to P(k)$.
  \exitem \label{ex:equiv-pi} 
  \begin{subexenum}
  \item Consider a family of $k$-truncated maps $f_i:A_i\to B_i$ indexed by $i:I$. Show that the map
    \begin{equation*}
      \lam{h}\lam{i}f_i(h(i)): \Big(\prd{i:I}A_i\Big)\to\Big(\prd{i:I}B_i\Big)
    \end{equation*}
    is also $k$-truncated.
  \item Consider an equivalence $e:I\simeq J$, and a family of equivalences $f_i:A_i\simeq B_{e(i)}$ indexed by $i:I$, where $A$ is a family of types indexed by $I$ and $B$ family of types indexed by $J$. Show that the map
    \begin{equation*}
      \lam{h}\lam{j} f_{e^{-1}(j)}(h(e^{-1}(j))) : \Big(\prd{i:I}A_i\Big)\to\Big(\prd{j:J}B_j\Big)
    \end{equation*}
    is an equivalence.
  \item Consider a family of maps $f_i:A_i\to B_i$ indexed by $i:I$. Show that the following are equivalent:
    \begin{enumerate}
    \item Each $f_i$ is $k$-truncated.
    \item For every map $\alpha:X\to I$, the map
      \begin{equation*}
        \lam{h}\lam{x}f_{\alpha(x)}(h(x)):\Big(\prd{x:X}A_{\alpha(x)}\Big)\to\Big(\prd{x:X}B_{\alpha(x)}\Big)
      \end{equation*}
      is $k$-truncated.
    \end{enumerate}
  \item \label{ex:equiv-postcomp}Show that for any map $f:A\to B$ the following are equivalent:
    \begin{enumerate}
    \item The map $f$ is $k$-truncated.
    \item For every type $X$, the postcomposition function
      \begin{equation*}
        f\circ\blank : (X\to A)\to (X\to B)
      \end{equation*}
      is $k$-truncated.
    \end{enumerate}
    In particular, $f$ is an equivalence if and only if $f\circ\blank$ is an equivalence, and $f$ is an embedding if and only if $f\circ\blank$ is an embedding.
  \end{subexenum}
  \exitem Show that \emph{$\Pi$-types distribute over coproducts}\index{distributivity!of P over coproducts@{of $\Pi$ over coproducts}}\index{dependent function type!distributivity of P over coproducts@{distributivity of $\Pi$ over coproducts}}\index{coproduct!distributivity of P over coproducts@{distributivity of $\Pi$ over coproducts}}, i.e., construct for any type $X$ and any two families $A$ and $B$ over $X$ an equivalence from the type $\prd{x:X}A(x)+B(x)$ to the type
  \begin{equation*}
    \sm{f:X\to\Fin{2}}\Big(\prd{x:X}A(x)^{f(x)=0}\Big)\times\Big(\prd{x:X}B(x)^{f(x)=1}\Big).
  \end{equation*}
  \exitem \label{ex:sec_retr}Consider a commuting triangle 
  \begin{equation*}
    \begin{tikzcd}[column sep=tiny]
      A \arrow[rr,"h"] \arrow[dr,swap,"f"] & & B \arrow[dl,"g"] \\
      & X
    \end{tikzcd}
  \end{equation*}
  with $H:f\htpy g\circ h$.
  \begin{subexenum}
  \item Show that if $h$ has a section, then $\sections(g)$ is a retract of $\sections(f)$.
  \item Show that if $g$ has a retraction, then $\retractions(h)$ is a retract of $\sections(f)$.
  \end{subexenum}
  \exitem \label{ex:triangle_fib}For any two maps $f:A\to X$ and $g:B\to X$, define the type of \define{morphisms from $f$ to $g$ over $X$}\index{morphism from f to g over X@{morphism from $f$ to $g$ over $X$}|textbf} by\index{hom X (f,g)@{$\homslice_X(f,g)$}|textbf}
  \begin{equation*}
    \homslice_X(f,g)\defeq \sm{h:A\to B} f\htpy g\circ h.
  \end{equation*}
  In other words, the type $\homslice_X(f,g)$ is the type of maps $h:A\to B$ equipped with a homotopy witnessing that the triangle
  \begin{equation*}
    \begin{tikzcd}[column sep=tiny]
      A \arrow[dr,swap,"f"] \arrow[rr,dashed,"h"] & & B \arrow[dl,"g"] \\
      & X
    \end{tikzcd}
  \end{equation*}
  commutes.
  \begin{subexenum}
  \item \label{ex:pi-fib}Consider a family $P$ of types over $X$. Show that the map
    \begin{equation*}
      \Big(\prd{x:X}\fib{f}{x}\to P(x)\Big)\to\Big(\prd{a:A}P(f(a))\Big)
    \end{equation*}
    given by $h\mapsto h_{f(a)}(a,\refl{f(a)})$ is an equivalence. 
  \item Construct three equivalences $\alpha$, $\beta$, and $\gamma$ as shown in the following diagram, and show that this triangle commutes:
    \begin{equation*}
      \begin{tikzcd}[column sep=-3em]
        & \homslice_X(f,g) \arrow[dl,swap,"\alpha"] \arrow[dr,"\beta"] & \phantom{\Big(\prd{x:X}\fib{f}{x}\to\fib{g}{x}\Big)} \\
        \Big(\prd{x:X}\fib{f}{x}\to\fib{g}{x}\Big) \arrow[rr,swap,"\gamma"] & & \prd{a:A}\fib{g}{f(a)}.
      \end{tikzcd}
    \end{equation*}
    Given a morphism $(h,H):\homslice_X(f,g)$ over $X$, we also say that $\alpha(h,H)$ is its \define{action on fibers}\index{action on fibers|textbf}\index{morphism from f to g over X@{morphism from $f$ to $g$ over $X$}!action on fibers|textbf}.
  \item \label{ex:fam-equiv}Given $(h,H):\homslice_X(f,g)$, show that the following are equivalent:
    \begin{enumerate}
    \item The map $h:A\to B$ is an equivalence.
    \item The action on fibers
      \begin{equation*}
        \alpha(h,H):\prd{x:X}\fib{f}{x}\to\fib{g}{x}
      \end{equation*}
      is a family of equivalences.
    \item The precomposition function
      \begin{equation*}
        \blank\circ (h,H) : \homslice_X(g,i)\to\homslice_X(f,i)
      \end{equation*}
      given by $(k,K)\circ (h,H) \defeq (k\circ h,\ct{H}{(K\cdot h)})$ is an equivalence for each map $i:C\to X$.
    \end{enumerate}
    Conclude that the type $\sm{h:\eqv{A}{B}} f\htpy g\circ h$ is equivalent to the type of families of equivalences
    \begin{equation*}
      \prd{x:X}\fib{f}{x}\eqvsym\fib{g}{x}.
    \end{equation*} 
  \end{subexenum}
\exitem \label{ex:iso_equiv}Let $A$ and $B$ be sets. Show that type $\eqv{A}{B}$ of equivalences from $A$ to $B$ is equivalent to the type $A\cong B$ of \define{isomorphisms}\index{isomorphism!of sets|textbf}\index{set!isomorphism|textbf} from $A$ to $B$, i.e., the type of quadruples $(f,g,H,K)$ consisting of
  \begin{align*}
    f & : A\to B \\
    g & : B\to A \\
    H & : f\circ g = \idfunc[B] \\
    K & : g\circ f = \idfunc[A].
  \end{align*}
\exitem Suppose that $A:I\to \UU$ is a type family over a set $I$ with decidable equality. Show that
  \begin{equation*}
    \Big(\prd{i:I}\iscontr(A_i)\Big)\leftrightarrow \iscontr\Big(\prd{i:I}A_i\Big).
  \end{equation*}
  \exitem \label{ex:retracts-as-limits}(Shulman) Consider a section-retraction pair
  \begin{equation*}
    \begin{tikzcd}
      A \arrow[r,"i"] & X \arrow[r,"r"] & A
    \end{tikzcd}
  \end{equation*}
  with $H:r\circ i\htpy \idfunc$ and define $f\defeq i\circ r$. Construct an equivalence
  \begin{equation*}
    A\simeq\sm{x:\N\to X}\prd{n:\N}f(x_{n+1})=x_n.
  \end{equation*}
\end{exercises}
\index{function extensionality|)}
\index{axiom!function extensionality|)}

%%% Local Variables:
%%% mode: latex
%%% TeX-master: "hott-intro"
%%% End:

\section{Propositional truncations}\label{sec:propositional-truncation}\label{chap:propositional-truncation}
\index{propositional truncation|(}

It is common in mathematics to express the property that a certain type of objects is inhabited, without imposing extra structure on those objects. For example, when we assert the property that a set is finite, then we only claim that there exists a bijection to a standard finite set $\{0,\ldots,n-1\}$ for some $n$, not that the set is equipped with such a bijection. There is indeed a conceptual difference between a finite set and a set equipped with a bijection to a standard finite set. The latter concept is that of a finite \emph{totally ordered} set. This difference is due to the fact that finiteness is a property, whereas there may be many different bijections to a standard finite set.

A similar observation can be made in the case of the image of a map. Note that being in the image of a given map $f:A\to B$ is a property. When we claim that $b:B$ is in the image of $f$, then we only claim that the type of $a:A$ such that $f(a)=b$ is inhabited. On the other hand, we saw in \cref{ex:fib_replacement} that the type of $b:B$ equipped with an $a:A$ such that $f(a)=b$ is equivalent to the type $A$, i.e., we have an equivalence
\begin{equation*}
  A\simeq \sm{b:B}\sm{a:A}f(a)=b.
\end{equation*}
Something is clearly off here, because the type $A$ is often not a subtype of the type $B$, while we would expect the image of $f$ to be a subtype of $B$. Therefore we see that the type $\sm{a:A}f(a)=b$ does not quite capture the concept of $b$ being in the image of $f$. The difference is again due to the fact that $\fib{f}{b}$ is often not a proposition, whereas we are looking to express the proposition that the preimage of $f$ at $b$ is inhabited.

To correctly capture the concepts of finiteness and the image of a map in type theory, and many further mathematical concepts, we need a way to assert the \emph{proposition} that a type is inhabited. The proposition that a type $A$ is inhabited is called the propositional truncation of $A$.

\subsection{The universal property of propositional truncations}\label{sec:propositional-truncation-up}

The propositional truncation of a type $A$ is a proposition $\brck{A}$ equipped with a map
\begin{equation*}
  \eta:A\to \brck{A}.
\end{equation*}
This map ensures that if we have an element $a:A$, then the proposition $\brck{A}$ that $A$ is inhabited holds. The complete specification of the propositional truncation includes the universal property of the map $\eta$. In this section we will specify in full generality when a map $f:A\to P$ into a proposition $P$ is a propositional truncation.

\begin{defn}
Let $A$ be a type, and let $f:A\to P$ be a map into a proposition $P$. We say that $f$ \define{is a propositional truncation}\index{is a propositional truncation|textbf}\index{propositional truncation!to be a propositional truncation|textbf} of $A$ if for every proposition $Q$, the precomposition map
\begin{equation*}
\blank\circ f:(P\to Q)\to (A\to Q)
\end{equation*}
is an equivalence. This property of $f$ is called the \define{universal property of the propositional truncation of $A$}\index{universal property!of propositional truncations|textbf}\index{propositional truncation!universal property|textbf}.
\end{defn}

\begin{rmk}\label{ex:prop_equiv}
  Using the fact that equivalences are maps that have contractible fibers, we can reformulate the universal property of the propositional truncation. Note that the fiber of the precomposition map $\blank\circ f:(P\to Q) \to (A \to Q)$ at a map $g:A\to Q$ is the type.
  \begin{equation*}
    \sm{h:P\to Q}h\circ f=g
  \end{equation*}
  Therefore we see that if $f$ satisfies the universal property of the propositional truncation, then these fibers are contractible. In other words, for each map $g:A\to Q$ into a proposition $Q$ there is a unique map $h:P\to Q$ for which $h\circ f=g$. We also say that every map $g:A\to Q$ into a proposition $Q$ \emph{extends} uniquely along $f$, as indicated in the diagram
  \begin{equation*}
    \begin{tikzcd}
      A \arrow[d,swap,"f"] \arrow[dr,"g"] \\
      P \arrow[r,dashed] & Q.
    \end{tikzcd}
  \end{equation*}
\end{rmk}

\begin{rmk}\label{rmk:simplified-up-trunc-Prop}
  For any two propositions $P$ and $P'$, a map $f:P\to P'$ is an equivalence if and only if there is a function $g:P'\to P$. To see this, simply note that any such function $g$ is an inverse of $f$, because any two elements in $P$ and any two elements in $P'$ are equal. 
  
  Note that the type $X\to Q$ is a proposition, for any type $X$ and any proposition $Q$. Using the previous observation, it therefore follows that the map $(P\to Q)\to (A\to Q)$ is an equivalence as soon as there is a map in the converse direction. In other words, to prove that a map $f:A\to P$ into a proposition $P$ satisfies the universal property of the propositional truncation of $A$, it suffices to construct a function
  \begin{equation*}
    (A\to Q)\to (P\to Q)
  \end{equation*}
  for every proposition $Q$.
\end{rmk}

In the following proposition we show that the propositional truncation of a type $A$ is uniquely determined up to equivalence, if it exists. In other words, any two propositional truncations of a type $A$ must be equivalent.

\begin{prp}
  Let $A$ be a type, and consider two maps
  \begin{equation*}
    f:A\to P \qquad\text{and}\qquad f':A\to P'
  \end{equation*}
  into two propositions $P$ and $P'$. If any two of the following three assertions hold, so does the third:\index{3-for-2 property!of propositional truncations}\index{propositional truncation!3-for-2 property}
  \begin{enumerate}
  \item\label{item:f-up-trunc-Prop} The map $f$ is a propositional truncation of $A$.
  \item\label{item:f-up-trunc-Prop'} The map $f'$ is a propositional truncation of $A$.
  \item\label{item:equiv-Prop} There is a (unique) equivalence $P\simeq P'$.
  \end{enumerate}
\end{prp}

\begin{proof}
  We first show that \ref{item:f-up-trunc-Prop} and \ref{item:f-up-trunc-Prop'} together imply \ref{item:equiv-Prop}. If $f$ and $f'$ are both propositional truncations of $A$, then we have maps $P\to P'$ and $P'\to P$ by the universal properties of $f$ and $f'$. Since $P$ and $P'$ are both propositions, it follows that $P\simeq P'$. For the uniqueness claim, note that the type $P\simeq P'$ is itself a proposition.

  Finally we show that \ref{item:equiv-Prop} implies that \ref{item:f-up-trunc-Prop} holds if and only if \ref{item:f-up-trunc-Prop'} holds. Suppose we have an equivalence $P\simeq P'$, let $Q$ be an arbitrary proposition, and consider the triangle
  \begin{equation*}
    \begin{tikzcd}[column sep=-1em]
      \phantom{(P'\to Q)} & (A\to Q) \arrow[dl,dashed] \arrow[dr,dashed] \\
      (P\to Q) \arrow[rr,<->] & & (P'\to Q),
    \end{tikzcd}
  \end{equation*}
  where the fact that $(P\to Q)\leftrightarrow (P'\to Q)$ holds follows from the assumption that $P$ is equivalent to $P'$. We see from this triangle that
  \begin{equation*}
    \Big((A\to Q)\to (P\to Q)\Big)\leftrightarrow\Big((A \to Q) \to (P'\to Q)\Big),
  \end{equation*}
  and this implies that \ref{item:f-up-trunc-Prop} holds if and only if \ref{item:f-up-trunc-Prop'} holds.
\end{proof}

\begin{rmk}
  One might be tempted to think that a type is inhabited if and only if it is nonempty. Recall that a type $A$ is nonempty if it satisfies the property $\neg\neg A$. Indeed, the type $\neg\neg A$ is a proposition, and it comes equipped with a map $A\to\neg\neg A$. It is therefore natural to wonder whether the map $A\to\neg\neg A$ satisfies the universal property of the propositional truncation.

  Recall that we have shown in \cref{ex:dn-monad} that any map $A\to\neg\neg Q$ extends to a map $\neg\neg A\to\neg\neg Q$, as indicated in the diagram
\begin{equation*}
  \begin{tikzcd}
    A \arrow[d] \arrow[dr] \\
    \neg\neg A \arrow[r,dashed] & \neg\neg Q.
  \end{tikzcd}
\end{equation*}
It follows that the natural map
\begin{equation*}
  (\neg\neg A\to\neg\neg Q)\to (A\to \neg\neg Q)
\end{equation*}
given by precomposition by $A\to\neg\neg A$ is an equivalence. However, this only gives us a universal property with respect to doubly negated propositions and there is no way to prove the more general universal property of the propositional truncation for the map $A\to\neg\neg A$. In fact, propositional truncations are not guaranteed to exist in Martin L\"of's dependent type theory, the way it is set up in \cref{chap:type-theory}. We will therefore add new rules to the type theory to ensure their existence.
\end{rmk}

\subsection{Propositional truncations as higher inductive types}\label{sec:propositional-truncation-hit}
\index{higher inductive type!propositional truncation|(}
\index{propositional truncation!as higher inductive type|(}

We have given a specification of the propositional truncation of a type $A$, and we have seen that this specification by a universal property determines the propositional truncation up to equivalence if it exists. However, the propositional truncation is not guaranteed to exist, so we will add new rules to the type theory that ensure that any type has a propositional truncation. We do this by presenting the propositional truncation of a type $A$ as a higher inductive type. The propositional truncation $\brck{A}$ of a type $A$ was one of the first examples of a higher inductive type, along with the circle, which we will discuss in \cref{sec:circle,sec:circle-universal-cover}.

The idea of higher inductive types is similar to the idea of ordinary inductive types, with the added feature that constructors of higher inductive types can also be used to generate \emph{identifications}. In other words, higher inductive types may be specified by two kinds of constructors:
\begin{enumerate}
\item The \emph{point constructors} are used to generate elements of the higher inductive types.
\item The \emph{path constructors} are used to generate identifications between elements of the higher inductive type.
\end{enumerate}
The induction principle of the higher inductive type then tells us how to construct sections of families over it. The rules for higher inductive types therefore come in four sets, just as the rules for ordinary inductive types in \cref{sec:inductive}: the formation rule, the constructors, the induction principle, and the computation rules.

\subsubsection{The formation rules and the constructors}
The formation rule of the propositional truncation postulates that for every type $A$ we can form the propositional truncation of $A$. The formation rule is therefore as follows:\index{[[A]]@{$\brck{A}$}|see {propositional truncation}}
  \begin{prooftree}
    \AxiomC{$\Gamma\vdash A~\type$}
    \UnaryInfC{$\Gamma\vdash \brck{A}~\type$.}
  \end{prooftree}
  Furthermore, we will assume that all universes are closed under propositional truncations. In other words, for any universe $\UU$ we will assume the rules

  \medskip
  \begin{minipage}{.4\textwidth}
    \begin{prooftree}
      \AxiomC{}
      \UnaryInfC{$X:\UU\vdash \brckcheck{X}:\UU$}
    \end{prooftree}
  \end{minipage}
  \begin{minipage}{.5\textwidth}
    \begin{prooftree}
      \AxiomC{}
      \UnaryInfC{$X:\UU\vdash \mathcal{T}(\brckcheck{X})\jdeq\brck{\mathcal{T}(X)}~\type$}
    \end{prooftree}
  \end{minipage}

  \medskip
The constructors of a (higher) inductive type tell what structure the type comes equipped with. In the case of a higher inductive type there may be point constructors and path constructors. The point constructors generate elements of the higher inductive type, and the path constructors generate identifications between those elements. In the case of the propositional truncation, there is one point constructor and one path constructor:\index{propositional truncation!eta : A -> [[A]]@{$\eta:A\to\brck{A}$}}\index{propositional truncation!alpha : P (x y : A) x = y@{$\alpha:\prd{x,y:A}x=y$}}
\begin{align*}
  \eta & : A \to \brck{A}\\*
  \alpha & : \prd{x,y:\brck{A}}x=y.
\end{align*}
The point constructor $\eta$ is sometimes called the \define{unit}\index{propositional truncation!unit of propositional truncation}\index{unit of propositional truncation} of the propositional truncation. It gives us that any element of $A$ also generates an element of $\brck{A}$. The path constructor $\alpha$ simply identifies any two elements of $\brck{A}$. Therefore it follows immediately that $\brck{A}$ is a proposition.

\begin{lem}
  For any type $A$, the type $\brck{A}$ is a proposition.\index{propositional truncation!is a proposition}\index{is a proposition!propositional truncation}\hfill $\square$ 
\end{lem}

\subsubsection{The induction principle and computation rules}
\index{induction principle!of propositional truncation|(}
\index{propositional truncation!induction principle|(}
The induction principle for the propositional truncation tells us how to construct dependent functions
\begin{equation*}
  h:\prd{t:\brck{A}}Q(t).
\end{equation*}
The induction principle will imply that such a dependent function $h$ is entirely determined by its behavior on the constructors of $\brck{A}$. The type $\brck{A}$ has two constructors: a point constructor $\eta$ and a path constructor $\alpha$, so we have two cases to consider:
\begin{enumerate}
\item Applying $h$ to points of the form $\eta(a)$ gives us a dependent function
  \begin{equation*}
    h\circ \eta : \prd{a:A}Q(\eta(a)).
  \end{equation*}
  The induction principle of $\brck{A}$ has therefore the requirement that we can construct
  \begin{equation*}
    f:\prd{a:A}Q(\eta(a))
  \end{equation*}
\item To apply $h$ to the paths $\alpha(x,y)$, we need to use the dependent action on paths from \cref{defn:apd}. For each $x,y:\brck{A}$ we obtain an identification
  \begin{equation*}
    \apd{h}{\alpha(x,y)}:\tr_Q(\alpha(x,y),h(x))=h(y)
  \end{equation*}
  in the type $Q(y)$. Note, however, that $h(x)$ and $h(y)$ are not determined by our choice of $f:\prd{a:A}Q(\eta(a))$. The second requirement of the induction principle of $\brck{A}$ is therefore that, no matter what values $h$ takes, they must always be related via the dependent action on paths of $h$. This second requirement is therefore that
  \begin{equation*}
    \tr_P(\alpha(x,y),u)=v
  \end{equation*}
  for any $u:Q(x)$ and $v:Q(y)$. 
\end{enumerate}

\begin{defn}
  The \define{induction principle}\index{induction principle!of propositional truncation|textbf}\index{propositional truncation!induction principle|textbf} of the propositional truncation $\brck{A}$ of $A$ asserts that for any family $Q$ of types over $\brck{A}$, if we have
  \begin{equation*}
    f:\prd{a:A}Q(\eta(a))
  \end{equation*}
  and if we can construct identifications
  \begin{equation*}
    \tr_Q(\alpha(x,y),u)=v
  \end{equation*}
  for any $u:Q(x)$, $v:Q(y)$ and any $x,y:\brck{A}$, then we obtain a dependent function
  \begin{equation*}
    h:\prd{t:\brck{A}}Q(t)
  \end{equation*}
  equipped with a homotopy $h\circ\eta\htpy f$.
\end{defn}

\begin{rmk}
  In fact, a family $Q$ over $\brck{A}$ satisfies the second requirement in the induction principle of the propositional truncation if and only if $Q$ is a family of propositions. To see this, simply note that transporting along $\alpha(x,y)$ is an embedding. Therefore we have
  \begin{equation*}
    (\tr_Q(\alpha(x,y),u)=\tr_Q(\alpha(x,y),v))\simeq (u=v)
  \end{equation*}
  for any $u,v:Q(x)$. By assumption, there is an identification on the left hand side, so any two elements $u$ and $v$ in $Q(x)$ are equal.

  Since the induction principle of the propositional truncation is only applicable to families of propositions over $\brck{A}$, it also follows that there are no interesting computation rules to state: any identification in a proposition just holds.
\end{rmk}
\index{induction principle!of propositional truncation|)}
\index{propositional truncation!induction principle|)}
\index{higher inductive type!propositional truncation|)}
\index{propositional truncation!as higher inductive type|)}

\subsubsection{The universal property}
\index{universal property!of propositional truncations|(}
\index{propositional truncation!universal property|(}
We have now completed the description of the propositional truncation as a higher inductive type, so it is time to show that it meets the specification we gave for the propositional truncations. In other words, we have to show that the map $\eta:A\to\brck{A}$ satisfies the universal property of the propositional truncation.

\begin{thm}
  The map $\eta:A\to\brck{A}$ satisfies the universal property of the propositional truncation.
\end{thm}

\begin{proof}
  In order to prove that $\eta:A\to\brck{A}$ satisfies the universal property of the propositional truncation of $A$, it suffices to construct a map
  \begin{equation*}
    (A\to Q)\to (\brck{A}\to Q)
  \end{equation*}
  for any proposition $Q$. Consider a map $f:A\to Q$. Then we will construct a function $\brck{A}\to Q$ by the induction principle of the propositional truncation. We have to provide a function $A\to Q$, which we have assumed already, and we have to show that
  \begin{equation*}
    \tr_{\lam{x}Q}(\alpha(x,y),u)=v.
  \end{equation*}
  for any $u,v:Q$ and any $x,y:\brck{A}$. However, we have such identifications by the assumption that $Q$ is a proposition, so the proof is complete.
\end{proof}

 One simple application of the universal property of the propositional truncation is that $\brck{\blank}$ acts on functions in a functorial way.

\begin{prp}
  There is a map\index{functorial action!of propositional truncations|textbf}\index{propositional truncation!functorial action|textbf}
  \begin{equation*}
    \brck{\blank}:(A\to B)\to (\brck{A}\to\brck{B})
  \end{equation*}
  for any two types $A$ and $B$, such that
  \begin{align*}
    \brck{\idfunc} & \htpy \idfunc \\
    \brck{g\circ f} & \htpy \brck{g}\circ\brck{f}.
  \end{align*}
\end{prp}

\begin{proof}
  For any $f:A\to B$, the map $\brck{f}:\brck{A}\to\brck{B}$ is defined to be the unique extension
  \begin{equation*}
    \begin{tikzcd}
      A \arrow[d,swap,"\eta"] \arrow[r,"f"] & B \arrow[d,"\eta"] \\
      \brck{A} \arrow[r,dashed,swap,"\brck{f}"] & \brck{B}.
    \end{tikzcd}
  \end{equation*}
  To see that $\brck{\blank}$ preserves identity maps and compositions, simply note that $\idfunc[\brck{A}]$ is an extension of $\idfunc[A]$, and that $\brck{g}\circ\brck{f}$ is an extension of $g\circ f$. Hence the homotopies are obtained by uniqueness.
\end{proof}
\index{universal property!of propositional truncations|)}
\index{propositional truncation!universal property|)}

\subsection{Logic in type theory}\label{sec:logic}
\index{logic|(}

In \cref{sec:modular-arithmetic} we interpreted logic in type theory via the Curry-Howard correspondence, which stipulates that disjunction ($\lor$) is interpreted by coproducts and the existential quantifier ($\exists$) is interpreted by $\Sigma$-types. However, when the existential quantifier is interpreted by $\Sigma$-types, then it is not possible to express certain concepts correctly, such as finiteness of a type or being in the image a map, and therefore we will add a second interpretation of logic in type theory, where logical propositions are interpreted by type theoretic propositions, i.e., the types of truncation level $-1$.

We have seen that the propositions are closed under cartesian products, implication, and dependent products indexed by arbitrary types. However, they are not closed under coproducts, and if $P$ is a family of propositions over a type $A$, then it is not necessarily the case that $\sm{x:A}P(x)$ is a proposition. We will therefore use propositional truncations to interpret disjunctions and existential quantifiers in type theory.

\begin{defn}
  Given two propositions $P$ and $Q$, we define their \define{disjunction}\index{disjunction|textbf}\index{P v Q@{$P\vee Q$}|see {disjunction}}
  \begin{equation*}
    P\vee Q \defeq \brck{P+Q}.
  \end{equation*}
\end{defn}

\begin{prp}
  Consider two propositions $P$ and $Q$. Then the disjunction $P\vee Q$ comes equipped with maps $i:P\to P\vee Q$ and $j:Q\to P\vee Q$. Moreover, the proposition $P\vee Q$ satisfies the universal property of the disjunction\index{universal property!of disjunction|textbf}\index{disjunction!universal property|textbf}: For any proposition $R$, we have
  \begin{equation*}
    (P\vee Q\to R)\leftrightarrow ((P\to R)\times (Q\to R)).
  \end{equation*}
\end{prp}

\begin{proof}
  The maps $i$ and $j$ are defined by
  \begin{align*}
    i & \defeq \eta\circ\inl  \\
    j & \defeq \eta\circ\inr.
  \end{align*}
  Now consider the following composition of maps, for an arbitrary proposition $R$:
  \begin{equation*}
    \begin{tikzcd}
      (P\vee Q\to R) \arrow[r,"\blank\circ\eta"] & (P+Q\to R) \arrow[r,"{h\,\mapsto\,(h\circ \inl,h\circ \inr)}"] &[3.6em] (P\to R)\times (Q\to R).
    \end{tikzcd}
  \end{equation*}
  The first map is an equivalence by the universal property of the propositional truncation, and the second map is an equivalence by the universal property of coproducts (\cref{ex:up-coproduct}).
\end{proof}

\begin{defn}
  Given a family $P$ of propositions over a type $A$, we define the \define{existential quantification}\index{existential quantification|textbf}\index{E (x:A) P(x)@{$\exists_{(x:A)}P(x)$}|see {existential quantification}}\index{E (x:A) P(x)@{$\exists_{(x:A)}P(x)$}|textbf}
  \begin{equation*}
    \exists_{(x:A)}P(x)\defeq \Brck{\sm{x:A}P(x)}.
  \end{equation*}
\end{defn}

\begin{prp}
  Consider a family $P$ of propositions over a type $A$. Then the existential quantification $\exists_{(x:A)}P(x)$ comes equipped with a dependent function
  \begin{equation*}
    \prd{a:A} \big(P(a)\to \exists_{(x:A)}P(x)\big).
  \end{equation*}
  Furthermore, the proposition $\exists_{(x:A)}P(x)$ satisfies the universal property of the existential quantification\index{universal property!of existential quantification|textbf}\index{existential quantification!universal property|textbf}: For any proposition $Q$, we have
  \begin{equation*}
    \Big(\Big(\exists_{(x:A)}P(x)\Big)\to Q\Big)\leftrightarrow\Big(\prd{x:A}P(x)\to Q\Big).
  \end{equation*}
\end{prp}

\begin{proof}
  The dependent function $\varepsilon : \prd{a:A} \big(P(a)\to \exists_{(x:A)}P(x)\big)$ is given by $\varepsilon(a,p):=\eta(a,p)$. Now consider the following composition of maps
  \begin{equation*}
    \begin{tikzcd}[column sep=small]
      \big(\big(\exists_{(x:A)}P(x)\big)\to Q\big) \arrow[r] &
      \big(\big(\sm{x:A}P(x)\big)\to Q\big) \arrow[r] &
      \big(\prd{x:A}P(x)\to Q\big).
    \end{tikzcd}
  \end{equation*}
  The first map in this composite is an equivalence by the universal property of the propositional truncation, and the second map is an equivalence by the universal property of $\Sigma$-types (\cref{thm:up-sigma}).
\end{proof}

In the following table we give an overview of the interpretation of the logical connectives using the propositions in type theory.

\begin{center}
  \begin{tabular}{ll}
    \toprule
    logical connective & interpretation in type theory\index{interpretation of logic in type theory}\index{logic!interpretation of logic in type theory} \\
    \midrule
    $\top$\index{T@{$\top$}|textbf} & $\unit$ \\
    $\bot$\index{T@{$\bot$}|textbf} & $\emptyt$ \\
    $P\Rightarrow Q$\index{implication|textbf} & $P\to Q$ \\
    $P\land Q$\index{conjunction|textbf} & $P\times Q$ \\
    $P\lor Q$\index{disjunction} & $\brck{P+Q}$ \\
    $P\Leftrightarrow Q$\index{bi-implication|textbf} & $P\leftrightarrow Q$ \\
    $\exists_{(x:A)}P(x)$\index{existential quantification} & $\Brck{\sm{x:A}P(x)}$ \\
    $\forall_{(x:A)}P(x)$\index{universal quantification|textbf}\index{A (x:A) P(x)@{$\forall_{(x:A)}P(x)$}|see {universal quantification}} & $\prd{x:A}P(x)$ \\
    \bottomrule
  \end{tabular}
\end{center}
\index{logic|)}

\subsection{Mapping propositional truncations into sets}

The universal property of the propositional truncation only applies when we want to define a map into a proposition. However, in some situations we might want to map the propositional truncation into a type that is not a proposition. Here we will see what we might do in such a case.

One strategy, if we want to define a map $\brck{A}\to X$, is to find a type family $P$ over $X$ such that the type $\sm{x:X}P(x)$ is a proposition. In that case, we may use the universal property of the propositional truncation to obtain a map $\brck{A}\to \sm{x:X}P(x)$ from a map $A\to \sm{x:X}P(x)$, and then we simply compose with the projection map.

\begin{eg}\label{eg:global-choice-decidable-subtype-N}
  Consider a \define{decidable subtype}\index{decidable subtype|textbf} $P$ of the natural numbers, i.e., a subtype $P:\N\to\prop_\UU$ such that each $P(n)$ is decidable. We claim that there is a function
  \begin{equation*}
    \Brck{\sm{x:\N}P(x)}\to\sm{x:\N}P(x).
  \end{equation*}
  Of course, we cannot directly use the universal property of the propositional truncation here. However, there is at most one \emph{minimal} natural number $x$ in $P$. In other words, we claim that the type
  \begin{equation}\label{eq:is-prop-minimal-element}
    \sm{x:\N}P(x)\times\islowerbound_P(x)\tag{\textasteriskcentered}
  \end{equation}
  is a proposition. To see this, note that the type $\islowerbound_P(x)$ is a proposition. By the assumption that each $P(x)$ is a proposition, it now follows that any two natural numbers $x,y:\N$ that are in $P$ and that are both lower bounds of $P$ are equal as elements in the type of \cref{eq:is-prop-minimal-element} if and only if they are equal as natural numbers. Furthermore, since both $x$ and $y$ are lower bounds of $P$, it follows that $x\leq y$ and $y\leq x$, so indeed $x=y$ holds.

  By the observation that the type in \cref{eq:is-prop-minimal-element} is a proposition, we may define a map
  \begin{equation*}
    \Brck{\sm{x:\N}P(x)}\to \sm{x:\N}P(x)\times\islowerbound_P(x)
  \end{equation*}
  by the universal property of the propositional truncation. A map
  \begin{equation*}
    \sm{x:\N}P(x)\to\sm{x:\N}P(x)\times\islowerbound_P(x)
  \end{equation*}
  was constructed in \cref{thm:well-ordering-principle-N} using the decidability of $P$.

  As a corollary of this observation, we observe that there is also a map
  \begin{equation*}
    \Brck{\sm{x:\Fin{k}}P(x)}\to\sm{x:\Fin{k}}P(x)
  \end{equation*}
  for any decidable subtype $P$ over $\Fin{k}$.
\end{eg}

\begin{rmk}\label{rmk:global-choice}
  The function of type
  \begin{equation*}
    \Brck{\sm{x:\N}P(x)}\to\sm{x:\N}P(x)
  \end{equation*}
  we constructed in \cref{eg:global-choice-decidable-subtype-N} for decidable subtypes of $\N$ is a rare case in which it is possible to obtain a function
  \begin{equation*}
    \brck{A}\to A.
  \end{equation*}
  We say that the type $A$ satisfies the \define{principle of global choice} if there is such a function $\brck{A}\to A$. Using the univalence axiom, we will see in \cref{cor:no-global-choice} that not every type satisfies the principle of global choice. 
\end{rmk}

More generally, we may wish to define a map $\brck{A}\to B$ where the type $B$ is a set. In this situation it is helpful to think of the propositional truncation of $A$ as the quotient of the type $A$ by the equivalence relation that relates every two elements of $A$ with each other. Propositional truncations can therefore also be characterized by the universal property of this quotient, which can be used to extend maps $f:A\to B$ to maps $\brck{A}\to B$ when the type $B$ is a set. The idea is that a map $f:A\to B$ into a set $B$ extends to a map $\brck{A}\to B$ if it satisfies $f(x)=f(y)$ for all $x,y:A$.

\begin{defn}\label{defn:weakly-constant}
  A map $f:A\to B$ is said to be \define{weakly constant}\index{weakly constant map|textbf}\index{constant map!weakly constant map|textbf} if it comes equipped with an element of type\index{is-weakly-constant(f)@{$\isweaklyconstant(f)$}|textbf}
  \begin{equation*}
    \isweaklyconstant(f) \defeq \prd{x,y:A}f(x)=f(y).
  \end{equation*}
\end{defn}

\begin{rmk}
  A constant map $A\to B$ is a map of the form $\const_b$. A map $f:A\to B$ is therefore constant if comes equipped with an element $b:B$ and a homotopy $f\htpy \const_b$. This is a stronger notion than the notion of weakly constant maps, which doesn't require there to be an element in $B$.

  One of the differences between constant maps and weakly constant maps manifests itself as follows: A type $A$ is contractible if and only if the identity map on $A$ is constant, while a type $A$ is a proposition if and only if the identity map on $A$ is weakly constant.
\end{rmk}

\begin{lem}
  Consider a commuting triangle
  \begin{equation*}
    \begin{tikzcd}
      A \arrow[d,swap,"\eta"] \arrow[dr,"f"] \\
      \brck{A} \arrow[r,swap,"g"] & B      
    \end{tikzcd}
  \end{equation*}
  where $B$ is an arbitrary type. Then the map $f$ is weakly constant.
\end{lem}

\begin{proof}
  Since $f$ is assumed to be homotopic to $g\circ \eta$, it suffices to show that $g\circ\eta$ is weakly constant. For any $x,y:A$, we have the identification $\alpha(x,y):\eta(x)=\eta(y)$ in $\brck{A}$. Using the action on paths of $g$, we obtain the identification
  \begin{equation*}
    \ap{g}{\alpha(x,y)}:g(\eta(x))=g(\eta(y))
  \end{equation*}
  in $B$.
\end{proof}

We now show, in a theorem due to Kraus \cite{Kraus}, that any weakly constant map $f:A\to B$ into a set $B$ extends uniquely to a map $\brck{A}\to B$. We therefore conclude that, in order to define a map $\brck{A}\to B$ into a set $B$ it suffices to define a map $f:A\to B$ and show that it is weakly constant.

\begin{thm}[Kraus]\label{ex:weakly-constant-map}
  Let $A$ be a type and let $B$ be a set. Then the map\index{universal property!of propositional truncations into sets|textbf}\index{propositional truncation!universal property into sets|textbf}
  \begin{equation*}
    (\brck{A}\to B)\to \sm{f:A\to B}\prd{x,y:A}f(x)=f(y)
  \end{equation*}
  given by $g\mapsto (g\circ\eta,\lam{x}\lam{y}\ap{g}{\alpha(x,y)})$ is an equivalence.
\end{thm}

\begin{proof}
  Consider a map $f:A\to B$ equipped with $H:\prd{x,y:A}f(x)=f(y)$. We first show that $f$ extends in at most one way to a map $\brck{A}\to B$. Let $g,h:\brck{A}\to B$ be two maps equipped with homotopies $f\htpy g\circ\eta$ and $f\htpy h\circ\eta$. In order to construct a homotopy $g\htpy h$, note that each identity type $g(x)=h(x)$ is a proposition by the assumption that $B$ is a set. We can therefore construct a homotopy $g\htpy h$ by the induction principle of propositional truncations. By the induction principle, it suffices to construct a homotopy $g\circ \eta\htpy h\circ\eta$, which we obtain from the homotopies $f\htpy g\circ\eta$ and $f\htpy h\circ\eta$.

  Since we've already proven uniqueness, it remains to construct an extension of the map $f$. We first claim that the type
  \begin{equation*}
    \sm{b:B}\Brck{\sm{x:A}f(x)=b}
  \end{equation*}
  is a proposition. To see this, consider two elements $b$ and $b'$ in this subtype of $B$. It suffices to show that $b=b'$. Since $B$ is assumed to be a set, the identity type $b=b'$ is a proposition. Therefore we may assume an element $x:A$ equipped with $p:f(x)=b$ and an element $x':A$ equipped with $p':f(x')=b'$. Using the assumption that $f$ is weakly constant, we obtain the identification
  \begin{equation*}
    \begin{tikzcd}
      b \arrow[r,equals,"p^{-1}"] & f(x) \arrow[r,equals,"{H(x,x')}"] & f(x') \arrow[r,equals,"{p'}"] & b'.
    \end{tikzcd}
  \end{equation*}
  Now we observe that the map $f:A\to B$ factors uniquely as follows
  \begin{equation*}
    \begin{tikzcd}[column sep=-1.5em]
      A \arrow[rr,dashed,"g"] \arrow[dr,swap,"f"] & & \sm{b:B}\Brck{\sm{x:A}f(x)=b} \arrow[dl,"\proj 1"] \\
      \phantom{\sm{b:B}\Brck{\sm{x:A}f(x)=b}} & B.
    \end{tikzcd}
  \end{equation*}
  Indeed, the map $g$ is given by $x\mapsto(f(x),\eta(x,\refl{}))$. Since the codomain of $g$ is a proposition, we obtain via the universal property of the propositional truncation of $A$ a unique map $h:\brck{A}\to\sm{b:B}\left\|\sm{x:A}f(x)=b\right\|$ equipped with a homotopy $g\htpy h\circ\eta$. Now we obtain the map $\proj 1\circ h:\brck{A}\to B$ equipped with the concatenated homotopy
  \begin{equation*}
    (\proj 1\circ h)\circ\eta \jdeq \proj 1\circ (h\circ\eta) \htpy \proj 1\circ g \htpy f.\qedhere
  \end{equation*}
\end{proof}

\begin{exercises}
  \exitem \label{ex:propositional-truncations-drill}Let $A$ be a type. Show that
  \begin{subexenum}
  \item $\brck{\brck{A}}\leftrightarrow\brck{A}$.
  \item $\brck{\isdecidable(A)}\leftrightarrow\isdecidable\brck{A}$.
  \item $\isdecidable(A)\to (\brck{A}\to A)$.
  \item $\neg\neg\brck{A}\leftrightarrow\neg\neg A$.
  \item $\brck{A}\vee\brck{B}\leftrightarrow \brck{A+B}$.
  \item $\exists_{(x:A)}\brck{B(x)}\leftrightarrow \brck{\sm{x:A}B(x)}$.
  \item $\neg\neg(\brck{A}\to A)$.
  \end{subexenum}
  \exitem Show that the \define{mere equality}\index{mere equality|textbf} relation given by $x,y\mapsto\brck{x=y}$ is an equivalence relation on any type.
  \exitem \label{ex:product-propositional-truncation}Consider two maps $f:A\to P$ and $g:B\to Q$ into propositions $P$ and $Q$. Show that if both $f$ and $g$ are propositional truncations then the map $f\times g : A\times B\to P\times Q$ is also a propositional truncation. Conclude that\index{propositional truncation!distributes over cartesian products}\index{distributivity!of propositional truncations over cartesian products}\index{cartesian product type!distributivity of propositional truncations over products}
  \begin{equation*}
    \brck{A\times B}\simeq \brck{A}\times\brck{B}. 
  \end{equation*}
  \exitem \label{ex:dup-trunc-prop}Consider a map $f:A\to P$ into a proposition $P$. We say that $f$ satisfies the \define{dependent universal property of the propositional truncation}\index{dependent universal property!of propositional truncations|textbf}\index{propositional truncation!dependent universal property|textbf} of $A$, if for any family $Q$ of propositions over $P$, the precomposition function
  \begin{equation*}
    \blank\circ f : \Big(\prd{p:P}Q(p)\Big)\to\Big(\prd{x:A}Q(f(x))\Big)
  \end{equation*}
  is an equivalence. Show that the following are equivalent:
  \begin{enumerate}
  \item\label{item:up-dup-trunc-Prop} The map $f$ is a propositional truncation.
  \item\label{item:dup-dup-trunc-Prop} The map $f$ satisfies the dependent universal property of the propositional truncation.
  \end{enumerate}
  \exitem Consider a map $f:A\to P$ into a proposition $P$.
  \begin{subexenum}
  \item Show that if there is a map $g:P\to A$, then $f$ is a propositional truncation. Conclude that for any type $A$ equipped with a point $a:A$, the constant map
    \begin{equation*}
      \const_\ttt: A\to\unit
    \end{equation*}
    is a propositional truncation of $A$.
  \item Show that if $A$ is a proposition, then $f$ is a propositional truncation if and only if $f$ is an equivalence. Conclude that if $A$ is a proposition, then the identity function $\idfunc:A\to A$ is a propositional truncation.
  \end{subexenum}
  \exitem Consider a type $A$ equipped with an element $d:\isdecidable(A)$.\index{decidable type}
  \begin{subexenum}
  \item Define a function $f:A\to \sm{x:A}\inl(x)=d$ and show that $f$ is a propositional truncation of $A$.
  \item Consider the function $\pi:\isdecidable(A)\to\Fin{2}$ defined by
    \begin{align*}
      \pi(\inl(x)) & := 1 \\
      \pi(\inr(x)) & := 0.
    \end{align*}
    Define a function $g:A\to (\pi(d)=1)$ and show that $g$ is a propositional truncation of $A$.
  \end{subexenum}
  \exitem Consider a family $B$ of $(k+1)$-truncated types over the propositional truncation $\brck{A}$ of a type $A$. Show that the map
  \begin{equation*}
    \Big(\prd{x:\brck{A}}B(x)\Big)\to\Big(\prd{x:A}B(x)\Big)
  \end{equation*}
  given by $f\mapsto f\circ\eta$ is a $k$-truncated map.
  \exitem Consider a universe $\UU$, let $P_1$ and $P_2$ be propositions in $\UU$, and furthermore, let $P$ be a family of propositions in $\UU$ over a type $A$ in $\UU$. Construct the following equivalences:
    \begin{align*}
      \top & \simeq \prd{Q:\prop_\UU}Q\to Q,\\
      \bot & \simeq \prd{Q:\prop_\UU}Q,\\
      \brck{A} & \simeq \prd{Q:\prop_\UU}(A\to Q)\to Q, \\
      P_1\lor P_2 & \simeq \prd{Q:\prop_\UU}(P_1\to Q) \to ((P_2\to Q)\to Q), \\
      P_1\land P_2 & \simeq \prd{Q:\prop_\UU}(P_1\to (P_2\to Q))\to Q, \\
      P_1\Rightarrow P_2 & \simeq \prd{Q:\prop_\UU}P_1\to ((P_2\to Q)\to Q), \\
      \neg A & \simeq \prd{Q:\prop_\UU} A\to Q, \\
      \exists_{(x:A)}P(x) & \simeq \prd{Q:\prop_\UU} \Big(\prd{x:A}P(x)\to Q\Big)\to Q, \\
      \forall_{(x:A)}P(x) & \simeq \prd{Q:\prop_\UU}\prd{x:A}(P(x)\to Q)\to Q\\
      \brck{a = x} & \simeq \prd{Q:A\to\prop_\UU} Q(a)\to Q(x).
    \end{align*}
    These are the \define{impredicative encodings}\index{impredicative encodings}\index{logic!impredicative encodings}\index{T@{$\top$}!impredicative encoding}\index{T@{$\bot$}!impredicative encoding}\index{propositional truncation!impredicative encoding}\index{disjunction!impredicative encoding}\index{conjunction!impredicative encoding}\index{implication!impredicative encoding}\index{negation!impredicative encoding}\index{existential quantification!impredicative encoding}\index{universal quantification!impredicative encoding}\index{mere equality!impredicative encoding} of the logical operators. \emph{Note:} It has the appearance that we could have defined $\brck{A}$ by its impredicative encoding. There is, however, a subtle issue if we take this as a definition: The map
    \begin{equation*}
      A\to\prd{Q:\prop_\UU}(A\to Q)\to Q
    \end{equation*}
    only satisfies the universal property of the propositional truncation with respect to propositions that are equivalent to propositions in $\UU$. 
  \exitem In this exercise we introduce the \define{interval}\index{interval|textbf} as a higher inductive type $\I$\index{I@{$\I$}|see{interval}}\index{I@{$\I$}|textbf}\index{higher inductive type!interval|textbf}, equipped with two point constructors and one path constructor
  \begin{align*}
    \source,\target & : \I \\
    \pathI & : \source=\target.
  \end{align*}
  The induction principle of $\I$ asserts that for any type family $P$ over $\I$, if we have
  \begin{align*}
    u & : P(\source) \\
    v & : P(\target) \\
    p & : \tr_P(\pathI,u)=v,
  \end{align*}
  then there is a section $f:\prd{x:\I}P(x)$ equipped with identifications
  \begin{align*}
    \alpha & : f(\source) = u \\
    \beta & : f(\target) = v 
  \end{align*}
  and an identification $\gamma$ witnessing that the square
  \begin{equation*}
    \begin{tikzcd}[column sep=6em]
      \tr_P(\pathI,f(\source)) \arrow[r,equals,"\ap{\tr_P(\pathI)}{\alpha}"] \arrow[d,equals,swap,"\apd{f}{\pathI}"] & \tr_P(\pathI,u) \arrow[d,equals,"p"] \\
      f(\target) \arrow[r,equals,swap,"\beta"] & v
    \end{tikzcd}
  \end{equation*}
  commutes. Note that the constructors of $\I$ induce a map
    \begin{equation*}
      \varepsilon: \Big(\prd{x:\I}P(x)\Big)\to \Big(\sm{u:P(\source)}\sm{v:P(\target)}\tr_P(\pathI,u)=v\Big).
    \end{equation*}
    given by $f\mapsto (f(\source),f(\target),\apd{f}{\pathI})$.
  \begin{subexenum}
  \item Characterize the identity types of the codomain of the map $\varepsilon$ in the following way: Construct an equivalence from the type $(u,v,q)=(u',v',q')$ to the type
    \begin{equation*}
      \sm{\alpha:u=u'}\sm{\beta:v=v'} \ct{q}{\beta}=\ct{\ap{\tr_P(\pathI)}{\alpha}}{q'},
    \end{equation*}
    for any $(u,v,q)$ and $(u',v',q')$ in the codomain of $\varepsilon$.
  \item Prove the dependent universal property of $\I$, i.e., show that the map $\varepsilon$ is an equivalence. 
  \item Show that $\I$ is contractible.
  \end{subexenum}
\end{exercises}
\index{propositional truncation|)}


%%% Local Variables:
%%% mode: latex
%%% TeX-master: "hott-intro"
%%% End:

\section{Image factorizations}\label{chap:image}

The image of a map $f:A\to X$ can be thought of as the least subtype of $X$ that contains all the values of $f$. More precisely, the image of $f$ is an embedding $i:\im(f)\hookrightarrow X$ that fits in a commuting triangle
\begin{equation*}
  \begin{tikzcd}[column sep=tiny]
    A \arrow[rr,"q"] \arrow[dr,swap,"f"] & & \im(f) \arrow[dl,hook,"i"] \\
    \phantom{\im(f)} & X
  \end{tikzcd}
\end{equation*}
and satisfies the \emph{universal property} of the image of $f$, which states that if a subtype $B\hookrightarrow X$ contains all the values of $f$, then it contains the image of $f$.

\subsection{The image of a map}\label{sec:image-construction}
 
\subsubsection*{The universal property of the image}
\index{universal property!of the image of a map|(}
\index{image of a map!universal property|(}

Recall from \cref{ex:triangle_fib} that we made the following definition:

\begin{defn}
  Let $f:A\to X$ and $g:B\to X$ be maps. A \define{morphism from $f$ to $g$ over $X$}\index{morphism from f to g over X@{morphism from $f$ to $g$ over $X$}|textbf} consists of a map $h:A\to B$ equipped with a homotopy $H:f\htpy g\circ h$ witnessing that the triangle
\begin{equation*}
\begin{tikzcd}[column sep=tiny]
A \arrow[rr,"h"] \arrow[dr,swap,"f"] & & B \arrow[dl,"g"] \\
& X
\end{tikzcd}
\end{equation*}
commutes. Thus, we define the type\index{hom X (f,g)@{$\homslice_X(f,g)$}|see {morphism from $f$ to $g$ over $X$}}
\begin{equation*}
\homslice_X(f,g)\defeq\sm{h:A\to B}f\htpy g\circ h.
\end{equation*}
Composition of morphisms over $X$ is defined by
\begin{equation*}
  (k,K)\circ (h,H) \defeq (k\circ h,\ct{H}{(K\cdot h)}).
\end{equation*}
\end{defn}

\begin{defn}
Consider a commuting triangle
\begin{equation*}
\begin{tikzcd}[column sep=tiny]
A \arrow[rr,"q"] \arrow[dr,swap,"f"] & & I \arrow[dl,"i"] \\
& X
\end{tikzcd}
\end{equation*}
with $H:f\htpy i\circ q$, where $i$ is an embedding\index{embedding}.
We say that $i$ satisfies the \define{universal property of the image of $f$}\index{universal property!of the image of a map|textbf} if the precomposition function
\begin{equation*}
\blank\circ(q,H) : \homslice_X(i,m)\to\homslice_X(f,m)
\end{equation*}
is an equivalence for every embedding $m:B\hookrightarrow X$. 
\end{defn}

\begin{lem}
For any $f:A\to X$ and any embedding\index{embedding} $m:B\to X$, the type $\homslice_X(f,m)$ is a proposition.\index{hom X (f,g)@{$\homslice_X(f,g)$}!is a proposition}
\end{lem}

\begin{proof}
  Recall from \cref{ex:triangle_fib} that the type $\homslice_X(f,m)$ is equivalent to the type
  \begin{equation*}
    \prd{a:A}\fib{m}{f(a)}.
  \end{equation*}
  Furthermore, recall from \cref{thm:embedding} that a map is an embedding if and only if its fibers are propositions.
  Thus we see that the type $\prd{a:A}\fib{m}{f(a)}$ is a product of propositions, hence it is a proposition by \cref{thm:trunc_pi}.
\end{proof}

\begin{prp}\label{prp:simplifly-universal-property-image}
  Consider a commuting triangle
  \begin{equation*}
    \begin{tikzcd}[column sep=tiny]
      A \arrow[rr,"q"] \arrow[dr,swap,"f"] & & I \arrow[dl,"i"] \\
      & X
\end{tikzcd}
  \end{equation*}
  with $H:f\htpy i\circ q$, where $i$ is an embedding. Then the following are equivalent:
  \begin{enumerate}
  \item The embedding $i$ satisfies the universal property of the image inclusion of $f$.
  \item For every embedding $m:B\to X$ there is a map
    \begin{equation*}
      \homslice_X(f,m)\to\homslice_X(i,m).
    \end{equation*}
  \end{enumerate}
\end{prp}

\begin{proof}
Since $\homslice_X(f,m)$ is a proposition for every embedding $m:B\to X$, the claim follows immediately by the observation made in \cref{ex:prop_equiv}.
\end{proof}
\index{universal property!of the image of a map|)}
\index{image of a map!universal property|)}

\subsubsection*{The existence of the image}
\index{image of a map!existence|(}

The image of a map $f:A\to X$ can be defined using the propositional truncation.

\begin{defn}\label{defn:im}
For any map $f:A\to X$ we define the \define{image}\index{image of a map|textbf} of $f$ to be the type\index{im f@{$\im(f)$}|see {image of a map}}\index{im f@{$\im(f)$}|textbf}
\begin{equation*}
\im(f) \defeq \sm{x:X}\brck{\fib{f}{x}}.
\end{equation*}
Furthermore, we define
\begin{enumerate}
\item the \define{image inclusion}\index{image inclusion|textbf}
  \begin{equation*}
    i_f:\im(f)\to X
  \end{equation*}
  to be the projection $\proj 1$,
\item the map
  \begin{equation*}
    q_f:A\to\im(f)
  \end{equation*}
  to be the map given by $q_f(x)\defeq(f(x),\eta(x,\refl{f(x)}))$, and
\item the homotopy $I_f:f\htpy i_f\circ q_f$ witnessing that the triangle
  \begin{equation*}
    \begin{tikzcd}[column sep=tiny]
      A \arrow[rr,"q_f"] \arrow[dr,swap,"f"] & & \im(f) \arrow[dl,"i_f"] \\
      \phantom{\im(f)} & X
    \end{tikzcd}
  \end{equation*}
  commutes, to be given by $I_f(x)\defeq\refl{f(x)}$.
\end{enumerate}
\end{defn}

\begin{prp}
  The image inclusion $i_f:\im(f)\to X$ of any map $f:A\to X$ is an embedding.\index{image inclusion!is an embedding}\index{is an embedding!image inclusion}
\end{prp}

\begin{proof}
  The claim follows directly by \cref{cor:pr1-embedding}, because the type $\brck{\fib{f}{x}}$ is a proposition for each $x:X$.
\end{proof}

\begin{thm}\label{thm:im}
  The image inclusion $i_f:\im(f)\to X$ of any map $f:A\to X$ satisfies the universal property of the image inclusion of $f$.
\end{thm}

\begin{proof}
  Consider an embedding $m:B\hookrightarrow X$. Note that we have a commuting square
  \begin{equation*}
    \begin{tikzcd}[column sep=6em]
      \homslice_X(i_f,m) \arrow[d] \arrow[r] & \homslice_X(f,m) \arrow[d] \\
      \Big(\prd{x:X}\fib{i_f}{x}\to\fib{m}{x}\Big) \arrow[r,swap,"h\mapsto{\lam{x}h_x\circ\varphi_x}"] & \Big(\prd{x:X}\fib{f}{x}\to\fib{m}{x}\Big)
    \end{tikzcd}
  \end{equation*}
  in which all four types are propositions, and the vertical maps are equivalences. Therefore it suffices to construct a map
  \begin{equation*}
    \Big(\prd{x:X}\fib{f}{x}\to\fib{m}{x}\Big)\to\Big(\prd{x:X}\fib{i_f}{x}\to\fib{m}{x}\Big)
  \end{equation*}
  The fiber $\fib{i_f}{x}$ is equivalent to the propositional truncation $\brck{\fib{f}{x}}$ and the type $\fib{m}{x}$ is a proposition by the assumption that $m$ is an embedding. Therefore we obtain the desired map by the universal property of the propositional truncation.
\end{proof}
\index{image of a map!existence|)}

\subsubsection*{The uniqueness of the image}
\index{image of a map!uniqueness|(}

We will now show that the universal property of the image implies that the image is determined uniquely up to equivalence.

\begin{thm}\label{thm:uniqueness-image}
  Let $f$ be a map, and consider two commuting triangles
  \begin{equation*}
    \begin{tikzcd}[column sep=tiny]
      A \arrow[dr,swap,"f"] \arrow[rr,"q"] & & B \arrow[dl,"i"] &[2em] A \arrow[dr,swap,"f"] \arrow[rr,"{q'}"] & & B' \arrow[dl,"{i'}"] \\
      \phantom{B'} & X & \phantom{B'} & \phantom{B'} & X
    \end{tikzcd}
  \end{equation*}
  with $I:f\htpy i\circ q$ and $I':f\htpy i'\circ q'$, in which $i$ and $i'$ are assumed to be embeddings. Then, if any two of the following three properties hold, so does the third:
  \begin{enumerate}
  \item The embedding $i$ satisfies the universal property of the image inclusion of $f$.
  \item The embedding $i'$ satisfies the universal property of the image inclusion of $f$.
  \item The type of equivalences $e:B\simeq B'$ equipped with a homotopy witnessing that the triangle
    \begin{equation*}
      \begin{tikzcd}
        B \arrow[dr,swap,"i"] \arrow[rr,"e"] & & B' \arrow[dl,"{i'}"] \\
        \phantom{B'} & X
      \end{tikzcd}
    \end{equation*}
    commutes is contractible.
  \end{enumerate}
\end{thm}

\begin{proof}
  First, we show that if (i) and (ii) hold, then (iii) holds. Note that the type $\homslice_X(i,i')$ is a proposition, since $i'$ is assumed to be an embedding. Therefore it suffices to show that the unique map $h:B\to B'$ such that the triangle
  \begin{equation*}
    \begin{tikzcd}[column sep=tiny]
      B \arrow[dr,swap,"i"] \arrow[rr,"h"] & & B' \arrow[dl,"{i'}"] \\
      & X
    \end{tikzcd}
  \end{equation*}
  commutes, is an equivalence. To see this, note that by \cref{ex:triangle_fib} it suffices to show that the action on fibers
  \begin{equation*}
    \fib{i}{x}\to\fib{i'}{x}
  \end{equation*}
  is an equivalence for each $x:X$. This follows from the universal property of $i'$, since we similarly obtain a family of maps
  \begin{equation*}
    \fib{i'}{x}\to\fib{i}{x}
  \end{equation*}
  indexed by $x:X$, and the types $\fib{i}{x}$ and $\fib{i'}{x}$ are propositions by the assumptions that $i$ and $i'$ are embeddings.
  
  Now we will show that (iii) implies that (i) holds if and only if (ii) holds. We will assume a morphism $(e,H):\homslice_X(i,i')$ such that the map $e$ is an equivalence. Furthermore, consider an embedding $m:C\to X$. Then the fact that (i) holds if and only if (ii) holds follows from the equivalence
  \begin{equation*}
    \big(\homslice_X(f,m)\to\homslice_X(i,m)\big)\simeq\big(\homslice_X(f,m)\to\homslice_X(i',m)\big).\qedhere
  \end{equation*}
\end{proof}
\index{image of a map!uniqueness|)}

\subsection{Surjective maps}\label{subsec:surjective}

A map $f:A\to B$ is surjective if for every $b:B$ there is an \emph{unspecified} element $a:A$ that maps to $b$. We define this property using the propositional truncation.

\begin{defn}
A map $f:A\to B$ is said to be \define{surjective}\index{surjective map|textbf} if there is an element of type\index{is-surj f@{$\issurj(f)$}|see {surjective map}}
\begin{equation*}
\issurj(f)\defeq \prd{b:B}\brck{\fib{f}{b}}.
\end{equation*}
\end{defn}

\begin{eg}
  Any equivalence is a surjective map, since its fibers are contractible. More generally, any map that has a section is surjective. Those are sometimes called \define{split epimorphisms}. Note that having a section is stronger than surjectivity, since in general we don't have a function $\brck{\fib{f}{b}}\to\fib{f}{b}$.
\end{eg}

In \cref{ex:dup-trunc-prop} we showed the dependent universal property of the propositional truncation: a map $f:A\to B$ into a proposition $B$ satisfies the universal property of the propositional truncation if and only if for every family of propositions $P$ over $B$, the precomposition map
\begin{equation*}
  \blank\circ f : \Big(\prd{b:B}P(b)\Big)\to\Big(\prd{a:A}P(f(a))\Big)
\end{equation*}
is an equivalence. In the following proposition we show that, if we omit the condition that $B$ is a proposition, then $f$ satisfies this dependent universal property if and only if $f$ is surjective.

\begin{prp}\label{prp:surjective}
  Consider a map $f:A\to B$. Then the following are equivalent:
  \begin{enumerate}
  \item \label{prp-item:surjective}The map $f:A\to B$ is surjective.
  \item \label{prp-item:is-equiv-precomp-surjective}The map $f:A\to B$ satisfies the \define{dependent universal property of a surjective map}\index{dependent universal property!of surjective maps|textbf}\index{surjective map!dependent universal property|textbf}: For any family $P$ of propositions over $B$, the precomposition map
    \begin{equation*}
      \blank\circ f : \Big(\prd{y:B}P(y)\Big)\to\Big(\prd{x:A}P(f(x))\Big)
    \end{equation*}
    is an equivalence. In other words, any subtype of $B$ that contains all the elements of the form $f(x)$ contains all the elements of $B$.
  \item \label{prp-item:is-trunc-map-precomp-surjective}For any $k\geq-2$, and for any family $P$ of $(k+1)$-truncated types over $B$, the precomposition map
    \begin{equation*}
      \blank\circ f : \Big(\prd{y:B}P(y)\Big)\to\Big(\prd{x:A}P(f(x))\Big)
    \end{equation*}
    is a $k$-truncated map.
  \end{enumerate}
\end{prp}

\begin{proof}
  To prove that \ref{prp-item:surjective} implies \ref{prp-item:is-equiv-precomp-surjective}, suppose first that $f$ is surjective, and consider the commuting square
  \begin{equation*}
    \begin{tikzcd}[column sep=4.4em]
      \Big(\prd{y:B}P(y)\Big) \arrow[r,"\blank\circ f"] \arrow[d,swap,"h\mapsto\lam{y}\const_{h(y)}"] & \Big(\prd{x:A}P(f(x))\Big)  \\
      \Big(\prd{y:B}\brck{\fib{f}{y}}\to P(y)\Big) \arrow[r,swap,"h\mapsto h(\blank)\circ\eta"] & \Big(\prd{y:B}\fib{f}{y}\to P(y)\Big) \arrow[u,swap,"{h\mapsto\lam{x}h(f(x),(x,\refl{}))}"]
    \end{tikzcd}
  \end{equation*}
  In this square, the bottom map is an equivalence by \cref{ex:equiv-pi} and by the universal property of the propositional truncation of $\fib{f}{y}$. The map on the right is an equivalence by \cref{ex:pi-fib}. Furthermore, the map on the left is an equivalence by \cref{ex:equiv-pi,ex:up-unit}, because the type $\brck{\fib{f}{y}}$ is contractible by the assumption that $f$ is surjective. Therefore it follows that the top map is an equivalence, which completes the proof that \ref{prp-item:surjective} implies \ref{prp-item:is-equiv-precomp-surjective}.

  The proof that \ref{prp-item:is-equiv-precomp-surjective} implies \ref{prp-item:is-trunc-map-precomp-surjective} is by induction on $k$. The base case holds by assumption. For the inductive step, it suffices by \cref{thm:trunc_ap} to show that $\apfunc{\blank\circ f}$ is $k$-truncated for any $g,h:\prd{y:B}P(y)$. Notice that we have a commuting square
  \begin{equation*}
    \begin{tikzcd}[column sep=large]
      (g=h) \arrow[r,"\apfunc{\blank\circ f}"] \arrow[d,swap,"\htpyeq"] & (g\circ f = h\circ f) \arrow[d,"\htpyeq"] \\
      \prd{y:B}g(y)=h(y) \arrow[r,swap,"\blank\circ f"] & \prd{x:A}g(f(x))=h(f(x))
    \end{tikzcd}
  \end{equation*}
  The vertical maps on the left and right are equivalences by function extensionality, and the bottom map is $k$-truncated by the inductive hypothesis. This implies that $\apfunc{\blank\circ f}$ is $k$-truncated.

  To prove that \ref{prp-item:is-trunc-map-precomp-surjective} implies \ref{prp-item:surjective}, note that the assumption in \ref{prp-item:is-trunc-map-precomp-surjective} implies that the precomposition function
  \begin{equation*}
    \blank\circ f : \Big(\prd{y:B}\brck{\fib{f}{y}}\Big)\to\Big(\prd{x:A}\brck{\fib{f}{f(x)}}\Big)
  \end{equation*}
  is an equivalence. Hence it suffices to construct an element of type $\brck{\fib{f}{f(x)}}$ for each $x:A$. This is easy, because we have
  \begin{equation*}
    \eta(x,\refl{f(x)}):\brck{\fib{f}{f(x)}}.\qedhere
  \end{equation*}
\end{proof}

As a corollary we obtain that any surjective map into a proposition satisfies the universal property of the propositional truncation.

\begin{cor}
  For any map $f:A\to P$ into a proposition $P$, the following are equivalent:
  \begin{enumerate}
  \item The map $f$ satisfies the universal property of the propositional truncation of $A$.
  \item The map $f$ is surjective.
  \end{enumerate}
\end{cor}

Using the characterization of surjective maps of \cref{prp:surjective}, we can also give a new characterization of the image of a map. 

\begin{thm}\label{thm:surjective}
Consider a commuting triangle
\begin{equation*}
\begin{tikzcd}[column sep=tiny]
A \arrow[rr,"q"] \arrow[dr,swap,"f"] & & B \arrow[dl,"m"] \\
& X
\end{tikzcd}
\end{equation*}
in which $m$ is an embedding. Then the following are equivalent:
\begin{enumerate}
\item The embedding $m$ satisfies the universal property of the image inclusion of $f$.\index{image of a map!universal property}\index{surjective map!universal property of the image of a map}
\item The map $q$ is surjective.
\end{enumerate}
\end{thm}

\begin{proof}
  First assume that $m$ satisfies the universal property of the image inclusion of $f$, and consider the composite function
  \begin{equation*}
    \begin{tikzcd}
      \Big(\sm{y:B}\brck{\fib{q}{y}}\Big) \arrow[r,"\proj 1"] & B \arrow[r,"m"] & X.
    \end{tikzcd}
  \end{equation*}
  Note that $m\circ\proj 1$ is a composition of embeddings, so it is an embedding. By the universal property of $m$ there is a unique map $h$ for which the triangle
  \begin{equation*}
    \begin{tikzcd}[column sep=0]
      B \arrow[dr,swap,"m"] \arrow[rr,dashed,"h"] & & \sm{y:B}\brck{\fib{q}{y}} \arrow[dl,"m\circ\proj 1"] \\
      \phantom{\sm{y:B}\brck{\fib{q}{y}}} & X
    \end{tikzcd}
  \end{equation*}
  commutes. Now note that $\proj 1\circ h$ is a map such that $m\circ (\proj 1\circ h)\htpy m$. The identity function is another map for which we have $m\circ\idfunc\htpy m$, so it follows by uniqueness that $\proj 1\circ h\htpy \idfunc$. In other words, the map $h$ is a section of the projection map. Therefore we obtain by \cref{ex:pi_sec} a dependent function
  \begin{equation*}
    \prd{b:B}\brck{\fib{q}{b}},
  \end{equation*}
  showing that $q$ is surjective.

  For the converse, suppose that $q$ is surjective. To prove that $m$ satisfies the universal property of the image factorization of $f$, it suffices to construct a map
  \begin{equation*}
    \homslice_X(f,m')\to\homslice_X(m,m'),
  \end{equation*}
  for any embedding $m':B'\to X$. To see that there is such an equivalence, we make the following calculation
  \begin{align*}
    \homslice_X(m,m') &  \simeq \prd{b:B}\fib{m'}{m(b)} \tag{By \cref{ex:triangle_fib}}\\
                         & \simeq \prd{a:A}\fib{m'}{m(q(a))} \tag{By \cref{prp:surjective}}\\
                         & \simeq \prd{a:A}\fib{m'}{f(a)} \tag{By $f\htpy m\circ q$}\\
                         & \simeq \homslice_X(f,m').\tag{By \cref{ex:triangle_fib}}
  \end{align*}
\end{proof}

\begin{cor}
  Every map factors uniquely as a surjective map followed by an embedding.\index{surjective map!factorization}\index{embedding!factorization}
\end{cor}

\begin{proof}
  Consider a map $f:A\to X$, and two factorizations
  \begin{equation*}
    \begin{tikzcd}[column sep=tiny]
      A \arrow[rr,"q"] \arrow[dr,swap,"f"] & & B \arrow[dl,"i"] &[3em] A \arrow[rr,"{q'}"] \arrow[dr,swap,"f"] & & B' \arrow[dl,"{i'}"] \\
      & X & & & X
    \end{tikzcd}
  \end{equation*}
  of $f$ where $m$ and $m'$ are embeddings, and $q$ and $q'$ are surjective. Then both $m$ and $m'$ satisfy the universal property of the image factorization of $f$ by \cref{thm:surjective}. Now it follows by \cref{thm:uniqueness-image} that the type of $(e,H):\homslice_X(i,i')$ in which $e$ is an equivalence, equipped with an identification
  \begin{equation*}
    (e,H)\circ(q,I)=(q',I')
  \end{equation*}
  in $\homslice_X(f,i')$, is contractible.
\end{proof}

\subsection{Cantor's diagonal argument}
\index{Cantor's diagonal argument|(}

Now that we have introduced surjective maps, we are in position to give Cantor's famous diagonal argument, which he used to show that there are infinite sets of different cardinality. The diagonal argument gives a proof that there is no surjective map from $X$ to its power set $\mathcal{P}(X)$. The power set of a type $X$ is of course defined with respect to a universe $\UU$, as the type of families of propositions in $\UU$ indexed by $X$.

\begin{defn}
  Consider a type $X$, and a universe $\UU$. We define the \define{$\UU$-power set}\index{power set|textbf} of $X$ to be\index{P U X@{$\mathcal{P}_\UU(X)$}|see {power set}}
  \begin{equation*}
    \mathcal{P}_{\mathcal{U}}(X)\defeq X\to\prop_\UU.
  \end{equation*}
\end{defn}

\begin{thm}
  For any type $X$ and any universe $\UU$, there is no surjective function
  \begin{equation*}
    f : X \to \mathcal{P}_{\mathcal{U}}(X)
  \end{equation*}
\end{thm}

\begin{proof}
  Consider a function $f:X\to (X\to \prop_\UU)$, and suppose that $f$ is surjective. Following Cantor's diagonalization argument, we define the subset $P:X\to\prop_\UU$ by
  \begin{equation*}
    P(x)\defeq \neg(f(x,x)).
  \end{equation*}
  Our goal is to reach a contradiction and $f$ is assumed to be surjective. Therefore, it suffices to show that
  \begin{equation*}
    \Brck{\sm{x:X}f(x)=P}\to\emptyt.
  \end{equation*}
  The empty type is a proposition, so by the universal property of the propositional truncation it is equivalent to show that
  \begin{equation*}
    \Big(\sm{x:X}f(x)=P\Big)\to\emptyt.
  \end{equation*}
  Consider an element $x:X$ equipped with an identification $f(x)=P$. Our goal is to construct an element of the empty type, i.e, to reach a contradiction. By the identification $f(x)=P$ it follows that 
  \begin{equation*}
    f(x,y)\leftrightarrow P(y)
  \end{equation*}
  for all $y:X$. In particular, it follows that $f(x,x)\leftrightarrow P(x)$. However, since $P(x)$ is defined as $\neg(f(x,x))$, we obtain that $f(x,x)\leftrightarrow\neg(f(x,x))$. By \cref{ex:no-fixed-points-neg} this gives us the desired contradiction.
\end{proof}
\index{Cantor's diagonal argument|)}

\begin{exercises}
  \exitem Consider a commuting triangle
  \begin{equation*}
    \begin{tikzcd}[column sep=tiny]
      A \arrow[dr,swap,"f"] \arrow[rr,"h"] & & B \arrow[dl,"g"] \\
      & X
    \end{tikzcd}
  \end{equation*}
  where $g$ is an embedding.
  \begin{subexenum}
  \item Show that if there is a morphism
    \begin{equation*}
      \begin{tikzcd}[column sep=tiny]
        B \arrow[dr,swap,"g"] \arrow[rr,"k"] & & A \arrow[dl,"f"] \\
        & X
      \end{tikzcd}
    \end{equation*}
    over $X$, then $g$ satisfies the universal property of the image of $f$.
  \item Show that if $f$ is an embedding, then $g$ satisfies the universal property of $f$ if and only if $h$ is an equivalence.
  \end{subexenum}
  \exitem
  \begin{subexenum}
  \item Show that for any proposition $P$, the constant map\index{constant map!is an embedding}
    \begin{equation*}
      \const_\ttt : P \to \unit
    \end{equation*}
    is an embedding. Use this fact to construct an equivalence
    \begin{equation*}
      \Big(\sm{A:\UU}A\hookrightarrow\unit\Big)\simeq\prop_\UU.
    \end{equation*}
  \item Consider a map $f:A\to P$ into a proposition $P$. Show that the following are equivalent:
    \begin{enumerate}
    \item The map $f$ is a propositional truncation of $A$.\index{propositional truncation!universal property of the image of A arrow 1@{universal property of the image of $A\to \unit$}}
    \item The constant map $P\to\unit$ satisfies the universal property of the image of the constant map $A\to\unit$.
    \end{enumerate}
  \end{subexenum}
  \exitem \label{ex:is-equiv-is-emb-is-surjective}Consider a map $f:A\to B$. Show that the following are equivalent:
  \begin{enumerate}
  \item $f$ is an equivalence.\index{is an equivalence!is surjective and an embedding}
  \item $f$ is both surjective and an embedding.
  \end{enumerate}
  \exitem Consider a commuting triangle
  \begin{equation*}
    \begin{tikzcd}[column sep=tiny]
      A \arrow[rr,"h"] \arrow[dr,swap,"f"] & & B \arrow[dl,"g"] \\
      & X
    \end{tikzcd}
  \end{equation*}
  with $H:f\htpy g\circ h$.
  \begin{subexenum}
  \item Show that if $f$ is surjective, then $g$ is surjective.
  \item Show that if both $g$ and $h$ are surjective, then $f$ is surjective.
  \item As a converse to \cref{ex:is-trunc-comp}, show that if $f$ and $h$ are $k$-truncated, then $g$ is also $k$-truncated.
  \end{subexenum}
  \exitem Prove \define{Lawvere's fixed point theorem}\index{Lawvere's fixed point theorem}: For any two types $A$ and $B$, if there is a surjective map $f:A\to B^A$, then for any $h:B\to B$ there exists an $x:B$ such that $h(x)=x$, i.e., show that
  \begin{equation*}
    \Big(\exists_{(f:A\to(A\to B))}\issurj(f)\Big)\to\Big(\forall_{(h:B\to B)}\exists_{(b:B)}h(b)=b\Big).
  \end{equation*}
\end{exercises}
%%% Local Variables:
%%% mode: latex
%%% TeX-master: "hott-intro"
%%% End:

\section{Finite types}\label{chap:finite}

\subsection{Counting in type theory}
\index{counting|(}
When someone counts the elements of a finite set $A$, they go through the elements of $A$ one by one, at each stage keeping track of how many elements have been counted so far. This process results in the number $|A|$ of elements of the set $A$, and moreover it gives a bijection from the standard finite set with $|A|$ elements. In other words, to count the elements of $A$ is to give an equivalence from one of the standard finite sets to the set $A$. We turn this into a definition.

\begin{defn}
  For each type $A$, we define the type\index{count(A)@{$\cnt(A)$}|textbf}
  \begin{equation*}
    \cnt(A)\defeq\sm{k:\N}(\Fin{k}\simeq A).
  \end{equation*}
  The elements of $\cnt(A)$ are called \define{countings}\index{countings of a type|textbf} of $A$. When we have $(k,e):\cnt(A)$, we also say that $A$ \define{has $k$ elements}\index{has k elements@{has $k$ elements}|textbf}.
\end{defn}

Note that the type $\cnt(A)$ is often not a proposition. For instance, different equivalences of type $\Fin{k}\simeq\Fin{k}$ induce different elements of type $\cnt(\Fin{k})$.

\begin{eg}
  It follows immediately from the definition of countings that every standard finite type can be counted in a canonical way: For any $k:\N$ we have $(k,\idfunc) : \cnt(\Fin{k})$. It also follows immediately from the definition of countings that types equipped with countings are closed under equivalences.
\end{eg}

\begin{eg}
  Suppose $A$ comes equipped with a counting $(k,e):\cnt(A)$. Then $k=0$ if and only if $A$ is empty. Indeed, the inverse of $e$ is a map $e^{-1}:A\to\emptyt$. Conversely, if we have $f:\isempty(A)$, then the map $f:A\to\emptyt$ is automatically an equivalence. This shows that $\Fin{k}\simeq\emptyt$, and a short argument by induction on $k$ yields that $k=0$. 
\end{eg}

\begin{eg}
  A type $A$ has one element if and only if it is contractible. Indeed, the type $\Fin{1}$ is contractible, so it follows from the 3-for-2 property of contractible types (\cref{ex:contr_retr}) that there is an equivalence $\Fin{1}\simeq A$ if and only if $A$ is contractible.   
\end{eg}

\begin{eg}\label{rmk:count-decidable-proposition}
  A proposition $P$ comes equipped with a counting if and only if it is decidable. To see this, note that for any type $X$, if we have $(k,e):\cnt(X)$, then it follows that $X$ is decidable. This is shown by induction on $k$. In the case where $k=0$, it follows that $X$ is empty, and hence that $X$ is decidable. In the case where $k$ is a successor, the bijection $e:\Fin{k}\simeq X$ gives us the element $e(\ttt):X$. Again we conclude that $X$ is decidable.

  Conversely, if $P$ is decidable, then we can construct a counting of $P$ by case analysis on $d:P+\neg P$. If $P$ holds, then it is contractible and hence equivalent to $\Fin{1}$. If $\neg P$ holds, then $P$ is equivalent to $\Fin{0}$.  
\end{eg}

\begin{rmk}\label{rmk:has-decidable-equality-count}
  We also note that any type $A$ equipped with a counting $e:\Fin{k}\simeq A$ has decidable equality.\index{has decidable equality!type equipped with a counting} This follows from \cref{prp:has-decidable-equality-Fin}, where we showed that $\Fin{k}$ has decidable equality, for any $k:\N$.
\end{rmk}

\begin{thm}\label{thm:count}
  We make the following claims about countings:
  \begin{enumerate}
  \item\label{item:count-coprod} Consider two types $A$ and $B$. The following are equivalent:
    \begin{enumerate}
    \item Both $A$ and $B$ come equipped with a counting.
    \item The coproduct $A+B$ comes equipped with a counting.
    \end{enumerate}
  \item\label{item:count-Sigma} Consider a type family $B$ indexed by a type $A$. Consider the following three conditions:
    \begin{enumerate}
    \item \label{item:count-Sigma-count-base}The type $A$ comes equipped with a counting.
    \item \label{item:count-Sigma-count-fibers}The type $B(x)$ comes equipped with a counting, for each $x:A$.
    \item \label{item:count-Sigma-count-total}The type $\sm{x:A}B(x)$ comes equipped with a counting.
    \end{enumerate}
    If (a) holds, then (b) holds if and only if (c) holds. Furthermore, if both (b) and (c) hold and if $B$ comes equipped with a section $f:\prd{x:A}B(x)$, then (a) holds.

    Consequently, if $P$ is a subtype of a type $A$ equipped with a counting, then we have
    \begin{equation*}
      \cnt\Big(\sm{x:A}P(x)\Big)\leftrightarrow \prd{x:A}\isdecidable(P(x)).
    \end{equation*}
  \end{enumerate}
\end{thm}

\begin{proof}
  We will first prove the forward direction of \ref{item:count-coprod}. Then we will prove both claims in \ref{item:count-Sigma}, and we will prove the reverse direction of claim \ref{item:count-coprod} last.

  For the forward direction of claim \ref{item:count-coprod}, suppose we have equivalences $e:\Fin{k}\simeq A$ and $f:\Fin{l}\simeq B$. The equivalences $e$ and $f$ induce via \cref{ex:coproduct_functor,ex:laws-Fin} a composite equivalence
  \begin{equation*}
    \begin{tikzcd}
      A+B \arrow[r,"\simeq"] & \Fin{k}+\Fin{l} \arrow[r,"\simeq"] & \Fin{k+l},
    \end{tikzcd}
  \end{equation*}
  from which we obtain an element of type $\cnt(A+B)$.

  Next, we will prove the forward direction in the first claim of \ref{item:count-Sigma}, i.e., we will prove that if $A$ comes equipped with an equivalence $e:\Fin{k}\simeq A$, and if $B$ is a family of types over $A$ equipped with
  \begin{equation*}
    f:\prd{x:A}\cnt(B(x)),
  \end{equation*}
  then the total space $\sm{x:A}B(x)$ also has a counting. The proof is by induction on $k$. Note that in the base case, where $k=0$, the type $\sm{x:A}B(x)$ is empty, so it has a counting. For the inductive step, note $\Sigma$ distributes from the right over coproducts. This gives an equivalence
  \begin{align*}
    \sm{x:A}B(x) & \simeq \sm{x:\Fin{k+1}}B(e(x)) \\
    & \simeq \Big(\sm{x:\Fin{k}}B(e(\inl(x)))\Big)+ B(e(\inr(\ttt))).
  \end{align*}
  The type $\sm{x:\Fin{k}}B(e(\inl(x)))$ has a counting by the inductive hypothesis, and the type $B(e(\inr(\ttt)))$ has a counting by assumption. Therefore, it follows that the total space $\sm{x:A}B(x)$ has a counting.
  
  Now we will prove the converse direction of the first claim in \ref{item:count-Sigma}. Suppose that $A$ comes equipped with $e:\Fin{k}\simeq A$, and that $\sm{x:A}B(x)$ comes equipped with $f:\Fin{l}\simeq\sm{x:A}B(x)$. By \cref{rmk:count-decidable-proposition} it suffices to show that, for $a:A$, the type $B(a)$ is a decidable subtype of $\sm{x:A}B(x)$. Consider the map
  \begin{equation*}
    i: B(a)\to \sm{x:A}B(x)
  \end{equation*}
  given by $b\mapsto (a,b)$. For $(x,y):\sm{x:A}B(x)$, we have the equivalences
  \begin{align*}
    \fibf{i}(x,y) & \simeq \sm{b:B(a)}(a,b)=(x,y) \\
                  & \simeq \sm{b:B(a)}\sm{p:a=x}\tr_B(p,b)=y \\
                  & \simeq \sm{p:a=x}\fib{\tr_B(p)}{y} \\
                  & \simeq a=x.
  \end{align*}
  Here we used that $\tr_B(p)$ is an equivalence, and therefore has contractible fibers. Now note that the type $a=x$ is a decidable proposition by \cref{rmk:has-decidable-equality-count}.

  Next, we will prove the second claim in \ref{item:count-Sigma}. Suppose that $B$ is a family over $A$ that comes equipped with a section $b:\prd{x:A}B(x)$, and suppose that each $B(x)$ has a counting, and that the total space $\sm{x:A}B(x)$ has a counting. Then we have a map
  \begin{equation*}
    g : A\to\sm{x:A}B(x)
  \end{equation*}
  given by $a\mapsto (a,b(a))$. The fibers of $g$ can be computed by the following equivalences:
  \begin{align*}
    \fibf{g}(x,y) & \simeq \sm{a:A}(a,b(a))=(x,y) \\
                  & \simeq \sm{a:A}\sm{p:a=x}\tr_B(p,b(a))=y \\
    & \simeq \tr_B(p,b(x))=y.
  \end{align*}
  Note that the type $\tr_B(p,b(x))=y$ is a decidable proposition by \cref{rmk:has-decidable-equality-count}. Now it follows by the forward direction of the first claim in \ref{item:count-Sigma} that $A$ has a counting.

  It remains to prove the converse direction of \ref{item:count-coprod}. Note that the forward direction of the first claim in \ref{item:count-Sigma} implies that countings on a type $X$ induce countings on any decidable subtype of $X$. Note that both $A$ and $B$ are decidable subtypes of the coproduct $A+B$. Any counting of $A+B$ therefore induces countings of $A$ and of $B$.
\end{proof}

\begin{cor}\label{cor:count-prod}
  Consider two types $A$ and $B$. We make two claims:
  \begin{enumerate}
  \item If both $A$ and $B$ come equipped with a counting, then the product $A\times B$ has a counting.
  \item If the product $A\times B$ comes equipped with a counting, then we have two functions
    \begin{align*}
      B & \to \cnt(A) \\
      A & \to \cnt(B).
    \end{align*}
  \end{enumerate}
\end{cor}

\begin{proof}
  The first claim follows from \ref{item:count-Sigma-count-base} in \cref{thm:count}, and the second claim follows from \ref{item:count-Sigma-count-fibers} in \cref{thm:count}. 
\end{proof}
\index{counting|)}

\subsection{Double counting in type theory}
\index{double counting|(}

In combinatorics, counting arguments often proceed by showing that two finite sets are isomorphic---or, in the language of type theory, by showing that two finite types are equivalent. The idea here is, of course, that when we count the elements of a type twice correctly, then both countings must result in the same number. However, this is something that we must prove before we can use it. In other words, we must show that
\begin{equation*}
  (\Fin{k}\simeq\Fin{l})\to (k=l)
\end{equation*}
for any two natural numbers $k$ and $l$. We will prove this claim as a consequence of the following general fact.

\begin{prp}\label{prp:is-injective-maybe}
  For any two types $X$ and $Y$, there is a map
  \begin{equation*}
    (X+\unit\simeq Y+\unit)\to (X\simeq Y).
  \end{equation*}
\end{prp}

\begin{proof}
  We prove the claim in four steps. We will write $i$ for $\inl:X\to X+\unit$ and also for $\inl:Y\to Y+\unit$, and we will write $\star$ for $\inr(\star):X+\unit$ and also for $\inr(\star):Y+\unit$.
  \begin{enumerate}
  \item We first show that for any equivalence $e:X+\unit\simeq Y+\unit$ and any $x:X$ equipped with an identification $p:e(i(x))=\star$, that there is an element
    \begin{equation*}
      \starvalue(e,x,p):Y
    \end{equation*}
    equipped with an identification
    \begin{equation*}
      \alpha:i(\starvalue(e,x,p))=e(\star).  
    \end{equation*}
    To see this, note that the map $e$ is injective. The elements $i(x)$ and $\star$ are distinct, so it follows that the elements $e(i(x))$ and $e(\star)$ are distinct. In particular, we have $e(\star)\neq\star$. Therefore it follows that there is an element $y:Y$ equipped with an identification $i(y)=e(\star)$. 
  \item Next, we construct for every equivalence $e:X+\unit\simeq Y+\unit$ a map $f:X\to Y$ equipped with identifications
    \begin{align*}
      \beta & : \prd{y:Y} (e(i(x))=i(y))\to (f(x)=y) \\
      \gamma & : \prd{p:e(i(x))=\star} f(x)=\starvalue(e,x,p).
    \end{align*}
    In order to construct the map $f:X\to Y$, we first construct a dependent function
    \begin{equation*}
      f':\prd{x:X}\prd{u:Y+\unit}((e(i(x))=u)\to Y).
    \end{equation*}
    This function is defined by pattern matching on $u$, by
    \begin{align*}
      f'(x,i(y),p) & \defeq y \\
      f'(x,\star,p) & \defeq \starvalue(e,x,p)
    \end{align*}
    Then we define $f(x):=f'(x,e(i(x)),\refl{})$. By the definition of $f'$ it then follows that we have an identification
    \begin{align*}
      f(x) & \jdeq f'(x,e(i(x)),\refl{}) \\
           & = f'(x,i(y),p) \\
           & \jdeq y
    \end{align*}
    for any $y:Y$ and $p:e(i(x))=i(y)$, and that we have an identification
    \begin{align*}
      f(x) & \jdeq f'(x,e(i(x)),\refl{}) \\
           & = f'(x,\ttt,p) \\
           & \jdeq \starvalue(e,x,p)
    \end{align*}
    for any $p:e(i(x))=\ttt$. 
  \item The inverse function $g:Y\to X$ is constructed in the same way as the function $f:X\to Y$, using the equivalence $e^{-1}:Y+\unit\simeq X+\unit$. This function comes equipped with
    \begin{align*}
      \delta & : \prd{x:X}(e^{-1}(i(y))=i(x))\to (g(y)=x) \\
      \varepsilon & : \prd{p:e^{-1}(i(y))=\star}g(y)=\starvalue(e^{-1},y,p). 
    \end{align*}
  \item It remains to show that $f$ and $g$ are inverse to each other. The proof that $g$ is a retraction of $f$ is similar to the proof that $g$ is a section of $f$, so we will only prove the latter. In other words, we will construct an identification
    \begin{equation*}
      f(g(y))=y
    \end{equation*}
    for any $y:Y$. The proof is by case analysis on $(e^{-1}(i(y))=\ttt)+(e^{-1}(i(y))\neq \ttt)$. In the case where $p:e^{-1}(i(y))=\ttt$, we have the identification
    \begin{equation*}
      \varepsilon(p):g(y)=\starvalue(e^{-1},y,p).
    \end{equation*}
    Furthermore, we have the identification
    \begin{equation*}
      \gamma(q) : f(g(y)) = \starvalue(e,g(y),q),
    \end{equation*}
    where $q:e(i(g(y)))=\ttt$ is the composite of the identifications
    \begin{align*}
      e(i(g(y))) & = e(i(\starvalue(e^{-1},y,p))) \\
                 & = e(e^{-1}(\ttt)) \\
      & =\ttt.
    \end{align*}
    Using the identification $\gamma(q)$, we obtain
    \begin{align*}
      i(f(g(y))) & = i(\starvalue(e,g(y),q)) \\
                 & = e(\ttt) \\
                 & = e(e^{-1}(i(y))) \\
                 & = i(y).
    \end{align*}
    Since $i:Y\to Y+\unit$ is injective, it follows that $f(g(y))=y$. 
    \qedhere
  \end{enumerate}
\end{proof}

\begin{thm}\label{thm:is-injective-Fin}
  For any two natural numbers $k$ and $l$, there is a map
  \begin{equation*}
    (\Fin{k}\simeq\Fin{l})\to (k=l).
  \end{equation*}
\end{thm}

\begin{proof}
  The proof is by induction on $k$ and $l$. In the base case, where both $k$ and $l$ are zero, the claim is obvious. If $k$ is zero and $l$ is a successor, then we have $0:\Fin{l}$. Any equivalence $e:\Fin{k}\simeq \Fin{l}$ now gives us the element
  \begin{equation*}
    e^{-1}(0):\emptyt,
  \end{equation*}
  which is of course absurd. Similarly, if $k$ is a successor and $l$ is zero, we obtain $e(0):\emptyt$, which is again absurd. If both $k$ and $l$ are a successor, then we have by \cref{prp:is-injective-maybe} the composite
  \begin{equation*}
    \begin{tikzcd}[column sep=1.5em]
      (\Fin{k+1}\simeq\Fin{l+1}) \arrow[r] & (\Fin{k}\simeq\Fin{l}) \arrow[r] & (k=l) \arrow[r] & (k+1=l+1).
    \end{tikzcd}\qedhere
  \end{equation*}
\end{proof}
\index{double counting|)}

\subsection{Finite types}

The type of all finite types is the subtype of the base universe $\UU_0$ consisting of all types $X$ for which there exists an unspecified equivalence $\Fin{k}\simeq X$ for some $k:\N$.

\begin{defn}\label{defn:finite}
  A type $X$ is said to be \define{finite}\index{finite type|textbf} if it comes equipped with an element of type\index{is-finite@{$\isfinite(X)$}|textbf}
  \begin{equation*}
    \isfinite(X) \defeq \Brck{\sm{k:\N}\Fin{k}\simeq X}
  \end{equation*}
  The type $\F$ of all finite types is defined to be\index{F@{$\F$}|see {finite type}}\index{F@{$\F$}|textbf}
  \begin{equation*}
    \F:=\sm{X:\UU_0}\isfinite(X).
  \end{equation*}
  In other words, the type $\F$ of finite types is the image of the map $\Fin{} : \N \to \UU_0$.
  We also define the type $\BS_k$\index{BS n@{$\BS_n$}|textbf} of \define{$k$-element types} by
  \begin{equation*}
    \BS_k\defeq \sm{X:\UU_0}\brck{\Fin{k}\simeq X}.
  \end{equation*}
\end{defn}

\begin{rmk}
  It follows directly from the definition of finiteness that any type $X$ equipped with a counting is finite. In particular, any $\Fin{k}$ is finite. Furthermore, it follows that if $X$ is equivalent to a finite type $Y$, then $X$ is also finite. Indeed, we can use the functoriality of the propositional truncation to obtain a function
  \begin{equation*}
    \Brck{\sm{k:\N}\Fin{k}\simeq Y}\to\Brck{\sm{k:\N}\Fin{k}\simeq X}
  \end{equation*}
  from a map $\big(\sm{k:\N}\Fin{k}\simeq Y\big)\to\big(\sm{k:\N}\Fin{k}\simeq X\big)$. Given an equivalence $e:X\simeq Y$, such a map is given as the map induced on total spaces from the family of maps $f\mapsto e^{-1}\circ f$.

  Similarly, it follows that any finite type has decidable equality, and that every finite type is a set.
\end{rmk}

In the following proposition we will show that each finite type can be assigned a unique cardinality.

\begin{thm}
  For any type $X$, consider the type $\isfinite'(X)$ defined by\index{is-finite'@{$\isfinite'(X)$}|textbf}
  \begin{equation*}
    \isfinite'(X) \defeq \sm{k:\N}\brck{\Fin{k}\simeq X}.
  \end{equation*}
  Then the type $\isfinite'(X)$\index{is-finite'@{$\isfinite'(X)$}!is a proposition} is a proposition, and there is an equivalence
  \begin{equation*}
    \isfinite(X)\leftrightarrow\isfinite'(X).
  \end{equation*}
  If $X$ is a finite type, then the unique number $k$ such that $\brck{\Fin{k}\simeq X}$ is the \define{cardinality}\index{cardinality!of a finite type|textbf} of $X$. We write $|X|$\index{{"|"}X{"|"}@{$|X|$}|see {cardinality, of a finite type}} for the cardinality of $X$.
\end{thm}

\begin{proof}
  We first prove the claim that the type $\isfinite'(X)$ is a proposition. In other words, we need to show that any two natural numbers $k$ and $k'$ for which there are respective elements of the types $\brck{\Fin{k}\simeq X}$ and $\brck{\Fin{k'}\simeq X}$, can be identified.

  Since the type of natural numbers is a set, the type $k=k'$ is a proposition. Therefore, we may assume that we have equivalences $\Fin{k}\simeq X$ and $\Fin{k'}\simeq X$. Consequently, we have an equivalence $\Fin{k}\simeq\Fin{k'}$. Now it follows from \cref{thm:is-injective-Fin} that $k=k'$.

  The second claim is that the propositions $\isfinite(X)$ and $\isfinite'(X)$ are equivalent, which we will show by constructing functions back and forth.
  Since we have shown that the type $\isfinite'(X)$ is a proposition, we obtain a map $\isfinite(X)\to\isfinite'(X)$ via the universal property of the propositional truncation, from the map
  \begin{equation*}
    \Big(\sm{k:\N}\Fin{k}\simeq X\Big)\to \sm{k:\N}\brck{\Fin{k}\simeq X}
  \end{equation*}
  given by $(k,e)\mapsto (k,\eta(e))$. 
  
  To construct a map $\isfinite'(X)\to\isfinite(X)$, it suffices to construct a map
  \begin{equation*}
    \brck{\Fin{k'}\simeq X}\to \Brck{\sm{k:\N}\Fin{k}\simeq X}
  \end{equation*}
  for each $k':\N$. Again by the universal property of the propositional truncation, we obtain this map from the function
  \begin{equation*}
    (\Fin{k'}\simeq X) \to \Brck{\sm{k:\N}\Fin{k}\simeq X}
  \end{equation*}
  given by $e\mapsto \eta(k',e)$. 
\end{proof}

\begin{cor}
  There is an equivalence
  \begin{equation*}
    \F \simeq \sm{k:\N}\BS_k.
  \end{equation*}
\end{cor}

\begin{proof}
  This equivalence can be obtained by composing the equivalences
  \begin{align*}
    \sm{X:\UU_0}\isfinite(X) & \simeq \sm{X:\UU_0}\sm{k:\N}\brck{\Fin{k}\simeq X} \\
    & \simeq \sm{k:\N}\sm{X:\UU_0}\brck{\Fin{k}\simeq X}. \qedhere
  \end{align*}
\end{proof}

We now aim to extend \cref{thm:count} to obtain some closure properties of finite types. Before we do so, we prove the \textbf{principle of finite choice}.

\begin{prp}\label{prp:finite-choice}
  Consider a type family $B$ over a finite type $A$. Then there is a \define{finite choice}\index{finite choice|textbf}\index{finite type!finite choice|textbf} map
  \begin{equation*}
    \Big(\prd{x:A}\brck{B(x)}\Big)\to\Brck{\prd{x:A}B(x)}
  \end{equation*}
\end{prp}

\begin{proof}
  Note that the type $\big\|\prd{x:A}B(x)\big\|$ is a proposition. Therefore we may assume that the type $A$ comes equipped with a counting $e:\Fin{k}\simeq A$. By this equivalence, it suffices to show that for every type family $B$ over $\Fin{k}$, there is a map
  \begin{equation*}
    \Big(\prd{x:\Fin{k}}\brck{B(x)}\Big)\to\Brck{\prd{x:\Fin{k}}B(x)}.
  \end{equation*}
  We proceed by induction on $k$. In the base case, $\Fin{k}$ is empty and therefore the type $\prd{x:\Fin{k}}B(x)$ is contractible. The asserted function therefore exists.

  For the inductive step, note that by the dependent universal property of coproducts (\cref{ex:up-coproduct}) we have the equivalences
  \begin{align*}
    \Big(\prd{x:\Fin{k+1}}\brck{B(x)}\Big) & \simeq \Big(\prd{x:\Fin{k}}\brck{B(i(x))}\Big)\times \brck{B(\ttt)} \\
    \Brck{\prd{x:\Fin{k}}B(x)} & \simeq \Brck{\Big(\prd{x:\Fin{k}}B(i(x))\Big)\times B(\ttt)}.
  \end{align*}
  Recall from \cref{ex:product-propositional-truncation} that $\brck{X\times Y}\simeq \brck{X}\times\brck{Y}$ for any two types $X$ and $Y$. This fact together with the inductive hypothesis finishes the proof.
\end{proof}

\begin{thm} ~
  \begin{enumerate}
    \item \label{item:coproduct-finite-types}For any two types $X$ and $Y$, the following are equivalent:
    \begin{enumerate}
    \item Both $X$ and $Y$ are finite.
    \item The coproduct $X+Y$ is finite.
    \end{enumerate}
  \item \label{item:product-finite-types}For any two types $X$ and $Y$, we make two claims:
    \begin{enumerate}
    \item If both $X$ and $Y$ are finite, then the cartesian product $X\times Y$ is finite.
    \item If the type $X\times Y$ is finite, then we have two functions
      \begin{align*}
        Y & \to \isfinite(X) \\
        X & \to \isfinite(Y).
      \end{align*}
    \end{enumerate}
  \item \label{item:Sigma-finite-types}Consider a type family $B$ over $A$, and consider the following three conditions:
    \begin{enumerate}
    \item The type $A$ is finite.
    \item The type $B(x)$ is finite for each $x:A$.
    \item The type $\sm{x:A}B(x)$ is finite.
    \end{enumerate}
    If (a) holds, then (b) is equivalent to (c). Moreover, if (b) and (c) hold, then (a) holds if and only if $A$ is a set and the type $\sm{x:A}\neg B(x)$ is finite. Furthermore, if (b) and (c) hold and $B$ has a section, then (a) holds.
  \end{enumerate}
\end{thm}

\begin{proof}
  To prove claim \ref{item:coproduct-finite-types}, first suppose that both $X$ and $Y$ are finite. Since the type $\isfinite(X+Y)$ is a proposition, we may assume that $X$ and $Y$ come equipped with countings. It follows from \cref{thm:count} that $X+Y$ has a counting, so it is finite. Conversely, suppose that the type $X+Y$ is finite. Since the types $\isfinite(X)$ and $\isfinite(Y)$ are both propositions, we may assume that the coproduct $X+Y$ comes equipped with a counting. Again it follows from \cref{thm:count} that the types $X$ and $Y$ have countings, so they are finite.

  The proof of claim \ref{item:product-finite-types} is similar to the proof of claim \ref{item:coproduct-finite-types}, hence we omit it.

  It remains to prove claim \ref{item:Sigma-finite-types}. First, suppose that the type $A$ is finite, and that each $B(x)$ is finite. By \cref{prp:finite-choice} we have a map
  \begin{equation*}
    \Big(\prd{x:A}\isfinite(B(x))\Big)\to \Brck{\prd{x:A}\cnt(B(x))}.
  \end{equation*}
  Since our goal is to construct an element of a proposition, we may therefore assume that each $B(x)$ comes equipped with a counting. We may also assume that $A$ comes equipped with a counting. It follows from \cref{thm:count} that the type $\sm{x:A}B(x)$ has a counting, so it is finite.

  Next, assume that $A$ is finite and that the type $\sm{x:A}B(x)$ is finite, and let $a:A$. The type $\isfinite(B(a))$ is a proposition, so we may assume that the types $A$ and $\sm{x:A}B(x)$ come equipped with countings. It follows from \cref{thm:count} that $B(a)$ has a counting, so it is finite.

  The final claim has two parts. First, assume that each $B(x)$ is finite, that the type $\sm{x:A}B(x)$ is finite, and that the type family $B$ has a section $f:\prd{x:A}B(x)$. It follows that the map
  \begin{equation*}
    A\to\sm{x:A}B(x)
  \end{equation*}
  given by $x\mapsto (x,f(x))$ is a decidable embedding, because the fiber at $(x,y)$ of this map is equivalent to the identity type $f(x)=y$ in $B(x)$, which is a decidable proposition. It follows from the fact that (a) and (b) together imply (c) that $A$ is finite.

  For the remaining part of the final claim, assume that $A$ is a set. Note that the assumption that each $B(x)$ is finite implies that each $B(x)$ is either inhabited or empty. It follows that we have an equivalence
  \begin{equation*}
    A\simeq \Big(\sm{x:A}\brck{B(x)}\Big)+\Big(\sm{x:A}\neg B(x)\Big).
  \end{equation*}
  We assume that the type $\sm{x:A}\neg B(x)$ is finite. In order to show that $A$ is finite, it therefore suffices to show that the type $\sm{x:A}\brck{B(x)}$ is finite. Without loss of generality, we assume that each $B(x)$ is inhabited. To finish the proof, it suffices to show that there is an element of type
  \begin{equation*}
    \Brck{\prd{x:A}B(x)}
  \end{equation*}
  using the assumption that $\prd{x:A}\brck{B(x)}$. To construct such an element, we may assume a counting $e:\Fin{k}\simeq\sm{x:A}B(x)$. We claim that there is a function
  \begin{equation*}
    \brck{B(a)}\to B(a),
  \end{equation*}
  i.e., that the type $B(a)$ satisfies the principle of global choice of \cref{rmk:global-choice} for each $a:A$. Recall from \cref{eg:global-choice-decidable-subtype-N} that the decidable subtypes of $\Fin{k}$ satisfy global choice. Therefore it also follows that the decidable subtypes of $\sm{x:A}B(x)$ satisfy global choice. Thus, it suffices to show that $B(x)$ is a decidable subtype of $\sm{x:A}B(x)$.
  
   The assumption that $A$ is a set implies by \cref{ex:is-trunc-fiber-inclusion} that the fiber inclusion $i_a:B(a)\to\sm{x:A}B(x)$ is an embedding for each $a:A$. Furthermore, we note that we have the following equivalence computing the fibers of $i_a$ at $(x,y)$:
  \begin{equation*}
    \Big(\sm{z:B(a)}(a,z)=(x,y)\Big)\simeq (a=x).
  \end{equation*}
  The type on the left hand side is decidable, so it follows that the type $A$ has decidable equality. We conclude that each $B(a)$ is a decidable subtype of $\sm{x:A}B(x)$.
\end{proof}

\begin{exercises}
  \exitem
  \begin{subexenum}
  \item Construct an equivalence $\Fin{n^m}\simeq(\Fin{m}\to\Fin{n})$. Conclude that if $A$ and $B$ are finite types, then $A\to B$ is finite.
  \item Construct an equivalence $\Fin{n!}\simeq(\Fin{n}\simeq\Fin{n})$. Conclude that if $A$ is finite, then $A\simeq A$ is finite.
  \end{subexenum}
  \exitem Suppose that $A$ is a retract of $B$. Show that $\cnt(B)\to\cnt(A)$.
  Conclude that $\isfinite(B)\to\isfinite(A)$.\index{finite type!closed under retracts}
  \exitem 
  \begin{subexenum}
  \item Consider a family of decidable types $A_i$ indexed by a finite type $I$. Show that the dependent product
    \begin{equation*}
      \prd{i:I}A_i
    \end{equation*}
    is decidable.
  \item Show that $\isemb(f)$ is decidable, for any map $f:I\to J$ between finite types.
  \item Show that $\issurj(f)$ is decidable, for any map $f:I\to J$ between finite types.
  \item Show that $\isequiv(f)$ is decidable, for any map $f:I\to J$ between finite types.
  \end{subexenum}
  \exitem \label{item:quotient-finite-types}Consider a surjective map $f:A\to B$, and suppose that $A$ is finite. Show that the following are equivalent:
    \begin{enumerate}
    \item The type $B$ has decidable equality.
    \item The type $B$ is finite.
    \end{enumerate}
  \exitem Consider a family $B$ of types over $A$.
  \begin{subexenum}
  \item Show that if $A$ is finite and if each $B(x)$ is finite, then the type
    \begin{equation*}
      \prd{x:A}B(x)
    \end{equation*}
    is finite.
  \item Show that if $A$ is finite and if $\prd{x:A}B(x)$ is finite, then we have
    \begin{equation*}
      \Big(\prd{x:A}\brck{B(x)}\Big)\to\Big(\prd{x:A}\isfinite(B(x))\Big).
    \end{equation*}
  \item Show that if $\prd{x:A}B(x)$ is finite and if each $B(x)$ is finite, then $A$ is finite if and only if the following three conditions hold:
    \begin{enumerate}
    \item $A$ has decidable equality.
    \item The type
      \begin{equation*}
        \sm{x:A}|B(x)|\leq 1
      \end{equation*}
      is finite.
    \item The type
      \begin{equation*}
        \prd{x:A}(2\leq|B(x)|)\to B(x)
      \end{equation*}
      is finite.
    \end{enumerate}
  \end{subexenum}
  \exitem Consider two finite types $X$ and $Y$ with $m$ and $n$ elements, respectively, and let $f:X\to Y$ be a map.
  \begin{subexenum}
  \item Show that
    \begin{equation*}
      \isinj(f)\to (m\leq n).
    \end{equation*}
  \item Prove the \define{pigeonhole principle}\index{pigeonhole principle|textbf}\index{finite type!pigeonhole principle|textbf}, i.e., show that
    \begin{equation*}
      (n>m)\to \exists_{(x,x':X)}(x\neq x')\times(f(x)=f(x')).
    \end{equation*}
  \item Show that there is no embedding $\N\hookrightarrow \Fin{k}$, for any $k:\N$.
  \end{subexenum}
  \exitem Consider a finite type $X$.
  \begin{subexenum}
  \item Show that any embedding $f:X\to X$ is an equivalence. Sets $X$ such that every embedding $X\hookrightarrow X$ is an equivalence are also called \define{Dedekind finite}.\index{Dedekind finite type|textbf}\index{finite type!Dedekind finite type|textbf}
  \item Show that any surjective map $f:X\to X$ is an equivalence.
  \end{subexenum}
  \exitem Consider two arbitrary types $A$ and $B$. For any $2$-element type $X$, construct an equivalence
  \begin{equation*}
    (A+B)^X\simeq A^X+X\times (A\times B)+B^X.
  \end{equation*}
  \exitem
  \begin{subexenum}
  \item Consider a set $A$ and an arbitrary type $B$. Show that any embedding $A\hookrightarrow B$ factors uniquely through the embedding $(\unit\hookrightarrow B)\hookrightarrow B$ given by $e\mapsto e(\ttt)$. 
  \item A map $f:A\to B$ is said to be \define{decidable}\index{decidable map} if the type $\fib{f}{b}$ is decidable for all $b:B$. Write $A\demb B$\index{A hookrightarrow d B@{$A \demb B$}|textbf} for the type of decidable embeddings\index{decidable embedding} from $A$ to $B$. Show that for any type $A$ with decidable equality and an arbitrary type $B$, any decidable embedding $A\demb B$ factors uniquely through the embedding $(\unit\demb B)\emb B$.
  \item (Escard\'o) For any two types $A$ and $B$, construct an equivalence
  \begin{equation*}
    ((A+\unit)\simeq(B+\unit))\simeq (\unit\demb (B+\unit))\times(A\simeq B).
  \end{equation*}
  \end{subexenum}
  \exitem
  \begin{subexenum}
  \item For any two types $A$ and $B$, construct an equivalence
    \begin{equation*}
      ((A+\unit)\demb(B+\unit))\simeq (\unit \demb (B+\unit))\times (A\demb B).
    \end{equation*}
  \item Construct an equivalence $\Fin{\fallingfactorial{n}{m}}\simeq(\Fin{m}\hookrightarrow\Fin{n})$, where $\fallingfactorial{n}{m}$ is the \define{$m$-th falling factorial}\index{falling factorial|textbf}\index{(n) m@{$\fallingfactorial{n}{m}$}|see {falling factorial}} of $n$, which is defined recursively by
    \begin{align*}
      \fallingfactorial{0}{0} & \defeq 1 & \fallingfactorial{0}{m+1} & \defeq 0 \\*
      \fallingfactorial{n+1}{0} & \defeq 1 & \fallingfactorial{n+1}{m+1} & \defeq (n+1)\fallingfactorial{n}{m}.
    \end{align*}
    Conclude that if $A$ and $B$ are finite with cardinality $m$ and $n$, then the type $A\hookrightarrow B$ is finite with cardinality $\fallingfactorial{n}{m}$.
  \end{subexenum}
  \exitem
  \begin{subexenum}
  \item Consider an arbitrary type $A$ and a type $B$ with decidable equality. Construct an equivalence
    \begin{equation*}
      ((A+\unit)\twoheadrightarrow(B+\unit))\simeq (B+\unit)\times(A\twoheadrightarrow B)+(A\twoheadrightarrow B+\unit).
    \end{equation*}
  \item Construct an equivalence $\Fin{\numberofsurjectivemaps{m}{n}}\simeq(\Fin{m}\twoheadrightarrow\Fin{n})$, where $\numberofsurjectivemaps{m}{n}$\index{S(m,n)@{$\numberofsurjectivemaps(m,n)$}|textbf} is defined recursively by
    \begin{align*}
      \numberofsurjectivemaps{0}{0} & \defeq 1 \\*
      \numberofsurjectivemaps{0}{n+1} & \defeq 0 \\*
      \numberofsurjectivemaps{m+1}{0} & \defeq 0 \\*
      \numberofsurjectivemaps{m+1}{n+1} & \defeq (n+1)\numberofsurjectivemaps{m}{n}+\numberofsurjectivemaps{m}{n+1}.
    \end{align*}
    Conclude that if $A$ and $B$ are finite with cardinality $m$ and $n$, then the type $A\twoheadrightarrow B$ is finite with cardinality $\numberofsurjectivemaps{m}{n}$. Note: the number $\numberofsurjectivemaps{m}{n}$ is $n!\stirling{m}{n}$, where $\stirling{m}{n}$\index{{{n m}}@{$\stirling{n}{m}$}|see {Stirling number of the second kind}} is the \define{Stirling number of the second kind}\index{Stirling number of the second kind} at $(m,n)$.
  \end{subexenum}
\end{exercises}
%%% Local Variables:
%%% mode: latex
%%% TeX-master: "hott-intro"
%%% End:

\section{The univalence axiom}
\index{univalence axiom|(}
\index{axiom!univalence|(}

The univalence axiom characterizes the identity type of a universe. Roughly speaking, it asserts that equivalent types are equal. The univalence axiom therefore postulates the common mathematical habit of identifying equivalent objects, such as equivalent types, isomorphic groups, isomorphic rings, logically equivalent propositions, subsets with the same elements, and so on. The univalence axiom is due to Voevodsky, who also showed that it is modeled in the simplicial sets. He also showed, in one of his first applications, that the univalence axiom implies function extensionality, which we will also prove here.

One way to think about the univalence axiom is that it \emph{expands} the notion of equality to encapsulate the notion of equivalence. It asserts that for each equivalence $e$ between two types $X$ and $Y$ in a universe $\mathcal{U}$ there is a unique identification $p_e:X=Y$ in the universe $\mathcal{U}$ such that transporting along $p_e$ in the universal type family over $\mathcal{U}$ is homotopic to the original equivalence $e:X\simeq Y$. 

Since there might be many distinct equivalences between two types $X$ and $Y$, there will be equally many identifications those types. The univalence axiom is therefore inconsistent with the commonly assumed axiom that all identity types are propositions, i.e., that all types are sets. Indeed, there are two equivalences $\bool\simeq\bool$, so a univalent universe cannot be a set.

\subsection{Equivalent forms of the univalence axiom}
By the fundamental theorem of identity types, \cref{thm:id_fundamental}, it is immediate that the univalence axiom comes in three equivalent forms.

\begin{thm}\label{thm:univalence}\index{identity type!of a universe}\index{characterization of identity type!of a universe}\index{universe!characterization of identity type}
Consider a universe $\UU$. The following are equivalent:
\begin{enumerate}
\item The universe $\UU$ is \define{univalent}\index{univalent universe|textbf}: For any two types $A,B:\UU$, the map\index{equiv-eq@{$\equiveq$}|textbf}
  \begin{equation*}
    \equiveq:(A=B)\to (A\simeq B)
  \end{equation*}
  given by $\equiveq(\refl{}):=\idfunc$, is an equivalence.
\item The type
\begin{equation*}
\sm{B:\UU}\eqv{A}{B}
\end{equation*}
is contractible for each $A:\UU$.
\item For any type $A:\UU$, the family of types $A\simeq X$ indexed by $X:\UU$ is an identity system on $\UU$. In other words, the universe $\UU$ satisfies the principle of \define{equivalence induction}\index{equivalence induction|textbf}\index{induction principle!for equivalences|textbf}: For every $A:\UU$ and for every type family of types $P(X,e)$ indexed by $X:\UU$ and $e:A\simeq X$, the map
\begin{equation*}
\Big(\prd{X:\UU}\prd{e:\eqv{A}{X}}P(X,e)\Big)\to P(A,\idfunc)
\end{equation*}
given by $f\mapsto f(A,\idfunc)$ has a section.
\end{enumerate}
\end{thm}

\begin{proof}
  The claim is a special case of \cref{thm:id_fundamental}, the fundamental theorem of identity types\index{fundamental theorem of identity types}.
\end{proof}

One way to see that the univalence axiom is plausible, is by observing that all type constructors preserve equivalences. For example, in \cref{thm:equiv-toto} we showed that for any type family $B$ over $A$ and any type family $B'$ over $A'$, if we have an equivalence $e:A\simeq A'$ and family of equivalences $f:\prd{x:A}B(x)\simeq B'(e(x))$, then we obtain an equivalence
\begin{equation*}
  \Big(\sm{x:A}B(x)\Big)\simeq\Big(\sm{x':A'}B'(x')\Big).
\end{equation*}
Under the same assumptions, we showed in \cref{ex:equiv-pi} that we obtain an equivalence
\begin{equation*}
  \Big(\prd{x:A}B(x)\Big)\simeq\Big(\prd{x':A'}B'(x')\Big).
\end{equation*}
Furthermore, for any two elements $x,y:A$ any equivalence $e:A\simeq A'$ induces an equivalence $(x=y)\simeq (e(x)=e(y))$ by \cref{cor:emb_equiv}. In other words, all the standard type formers within a universe $\UU$ are \emph{equivalence invariant}. Since identity types are not assumed to be propositions, we have the possibility to postulate the univalence axiom.

\begin{axiom}[The univalence axiom]\label{axiom:univalence}\index{univalence axiom|textbf}\index{axiom!univalence|textbf}
  We will assume that all the universes generated by \cref{enough-universes}\index{enough universes}\index{universe!enough universes} are univalent. Given a univalent universe $\UU$, we will write $\eqequiv$\index{eq-equiv@{$\eqequiv$}|textbf} for the inverse of $\equiveq$.
\end{axiom}

As a first application of the univalence axiom, let us show that for any type $A$ the type of types in a univalent universe $\UU$ that are equivalent to $A$ is a proposition.

\begin{defn}\label{defn:small-types}
  Consider a univalent universe $\UU$. A type $X$ is said to be \define{$\UU$-small}\index{small type|textbf}\index{U-small type@{$\UU$-small type}|textbf} if it comes equipped with an element of type\index{is-small@{$\issmall_\UU(A)$}|textbf}
  \begin{equation*}
    \issmall_\UU(A)\defeq\sm{X:\UU}A\simeq X.
  \end{equation*}
  Similarly, a map $f:A\to B$ is said to be \define{$\UU$-small}\index{small map|textbf}\index{U-small map@{$\UU$-small map}|textbf} if all of its fibers are $\UU$-small.
\end{defn}

\begin{eg}
  ~
  \begin{enumerate}
  \item Any type in $\UU$ is $\UU$-small.
  \item Any contractible type is $\UU$-small with respect to any universe $\UU$. \index{contractible type!is U-small@{is $\UU$-small}}\index{small type!contractible type}
  \item For any family $P$ of $\UU$-small types over a $\UU$-small type $A$, the dependent product $\prd{x:A}B(x)$ is $\UU$-small.\index{small type!dependent function type}\index{dependent function type!is U-small@{is $\UU$-small}}
  \item The type of $\UU$-small types in $\VV$ is equivalent to the type of $\VV$-small types in $\UU$. This follows from the equivalence
    \begin{equation*}
      \Big(\sm{Y:\VV}\sm{X:\UU}Y\simeq X\Big) \simeq \Big(\sm{X:\UU}\sm{Y:\VV}X\simeq Y\Big).
    \end{equation*}
  \item Any finite type is $\UU$-small for any universe $\UU$.\index{finite type!is U-small@{is $\UU$-small}}\index{small type!finite type} Consequently, we get equivalences
  \begin{equation*}
    \Big(\sm{X:\UU}\isfinite(X)\Big)\simeq\Big(\sm{Y:\VV}\isfinite(Y)\Big)
  \end{equation*}
  for any two univalent universes $\UU$ and $\VV$. This observation is the reason why we usually write $\F$ for the type of finite types (in $\UU$), without referring to its universe.
  \item In \cref{thm:russell} we will show that $\UU$ cannot be $\UU$-small, i.e., that there cannot be a type $U:\UU$ equipped with an equivalence $U\simeq \UU$.
  \end{enumerate}
\end{eg}

\begin{prp}\label{prp:small}
  For any univalent universe $\UU$ and any type $A$, the type $\issmall_\UU(A)$ is a proposition.\index{is-small@{$\issmall_\UU(A)$}!is a proposition}
\end{prp}

\begin{proof}
  By \cref{lem:isprop_eq} it suffices to show that
  \begin{equation*}
    \issmall_\UU(A)\to\iscontr(\issmall_\UU(A)).
  \end{equation*}
  Let $X:\UU$ be a type equipped with $e:A\simeq X$. Then we have an equivalence
  \begin{equation*}
    \Big(\sm{Y:\UU}A\simeq Y\Big)\simeq\Big(\sm{Y:\UU}X\simeq Y\Big).
  \end{equation*}
  The latter type is contractible by \cref{thm:univalence}.
\end{proof}

\begin{cor}
  Consider a univalent universe $\UU$ and a univalent universe $\VV$ containing all types in $\UU$. Then the universe inclusion $i:\UU\to\VV$ is an embedding.
\end{cor}

\begin{proof}
  Since $\VV$ is assumed to be univalent, it follows that
  \begin{equation*}
    \fib{i}{A}\simeq\issmall_\UU(A)
  \end{equation*}
  for any type $A:\VV$. The type $\issmall_\UU(A)$ is a proposition since $\UU$ is univalent. Hence the claim follows by \cref{thm:embedding}.
\end{proof}

\subsection{Propositional extensionality}
\index{propositional extensionality|(}

An important direct consequence of the univalence axiom is the principle of propositional extensionality. This principle asserts that any two logically equivalent propositions $P$ and $Q$ can be identified. Propositional extensionality is an important principle on its own, which is sometimes assumed in formal systems without the univalence axiom.

In order to prove propositional extensionality, we first observe that the univalence axiom also characterizes the identity type of any subuniverse.

\begin{prp}\label{prp:univalence-subuniverse}
  Consider a universe $\UU$, and let $P$ be a family of propositions over $\UU$. Then the family of maps
  \begin{equation*}
    \equiveq:(A=B)\to (\proj 1(A) \simeq \proj 1(B))
  \end{equation*}
  indexed by $A,B:\sm{X:\UU}P(X)$, given by $\equiveq(\refl{}):=\idfunc$ is an equivalence.
\end{prp}

\begin{proof}
  Since $P$ is a subuniverse, it follows from \cref{cor:pr1-embedding} that the projection map is an embedding. Therefore we see that the asserted map is the composite of the equivalences
  \begin{equation*}
    \begin{tikzcd}
      (A=B) \arrow[r,"\apfunc{\proj 1}"] & (\proj 1(A)=\proj 1(B)) \arrow[r,"\equiveq"] &[2em] (\proj 1(A)\simeq \proj 1(B)).
    \end{tikzcd}\qedhere
  \end{equation*}
\end{proof}

\begin{rmk}
  Often, when $P$ is a subuniverse, i.e., a subtype of the a universe $\UU$, we will also write $A$ for the type $\proj 1(A)$ if $A:\sm{X:\UU}P(X)$. Using this shorthand notation, the equivalence in \cref{prp:univalence-subuniverse} is displayed as
  \begin{equation*}
    (A=B)\simeq (A\simeq B).
  \end{equation*}
\end{rmk}

Important examples of subuniverses include the subuniverse $\prop_\UU$ of propositions in $\UU$, the subuniverse $\Set_\UU$ of sets in $\UU$, and the subuniverse $\UU^{\leq k}$ of $k$-truncated types in $\UU$. The subuniverse $\F$ of finite types in $\UU_0$, and the subuniverses $\BS_k$ of $k$-element types are further important subuniverses to which \cref{prp:univalence-subuniverse} applies. Note that by the univalence axiom, any subuniverse is automatically closed under equivalences\index{subuniverse!closed under equivalences}. Indeed, if we have $X\simeq Y$, then we have $P(X)\to P(Y)$ by transporting along the equality $X=Y$ induced by univalence.

\begin{thm}\label{prp:propositional-extensionality}
  Propositions satisfy \define{propositional extensionality}\index{propositional extensionality|textbf}\index{extensionality principle!for propositions|textbf}:
  For any two propositions $P$ and $Q$, the canonical map\index{bi-implication}\index{iff-eq@{$\iffeq$}|textbf}
  \begin{equation*}
    \iffeq:(P=Q)\to (P\iffprop Q)
  \end{equation*}
  defined by $\iffeq(\refl{}):=(\idfunc,\idfunc)$ is an equivalence. It follows that the type $\prop_\UU$ of propositions in $\UU$ is a set.\index{Prop@{$\prop_\UU$}!is a set}\index{univalence axiom!implies propositional extensionality}
\end{thm}

\begin{proof}
  Recall from \cref{ex:isprop_istrunc} that $\isprop(X)$ is a proposition for any type $X$. \cref{prp:univalence-subuniverse} therefore applies, which gives
  \begin{equation*}
    (P=Q)\simeq (P\simeq Q)\simeq (P\leftrightarrow Q).
  \end{equation*}
  The last equivalence follows from \cref{prp:equiv-prop}, using the fact that $(P\simeq Q)$ is a proposition by \cref{ex:isprop_isequiv}.
\end{proof}

\begin{cor}\label{cor:decidable-Prop}
  The type\index{DProp@{$\decidableProp_\UU$}|see {decidable proposition}}\index{DProp@{$\decidableProp_\UU$}|textbf}\index{decidable proposition|textbf}\index{DProp@{$\decidableProp_\UU$}!is equivalent to bool@{is equivalent to $\bool$}}
  \begin{equation*}
    \decidableProp_\UU \defeq \sm{P:\prop_\UU}\isdecidable(P)
  \end{equation*}
  of decidable propositions in any universe $\UU$ is equivalent to $\bool$.
\end{cor}

\begin{proof}
  Note that $\Sigma$ distributes from the left over coproducts, so we have an equivalence
  \begin{equation*}
    \Big(\sm{P:\prop_\UU}P+\neg P\Big)\simeq \Big(\sm{P:\prop_\UU}P\Big)+\Big(\sm{Q:\prop_\UU}\neg Q\Big). 
  \end{equation*}
  Therefore it suffices to show that both $\sm{P:\prop_\UU}P$ and $\sm{Q:\prop_\UU}\neg Q$ are contractible. At the centers of contraction we have $(\unit,\ttt)$ and $(\emptyt,\idfunc)$, respectively. For the contractions, note that both types are subtypes of the types of propositions. Therefore it suffices to show that $\unit=P$ for any proposition $P$ equipped with $p:P$, and that $\emptyt=Q$ for any proposition $Q$ equipped with $q:\neg Q$. Both identifications are obtained immediately from propositional extensionality.
\end{proof}
\index{propositional extensionality|)}

\subsection{Univalence implies function extensionality}
One of the first applications of the univalence axiom was Voevodsky's theorem that the univalence axiom on a universe $\UU$ implies function extensionality for types in $\UU$. The proof uses the fact that weak function extensionality implies function extensionality. We will also make use of the following lemma. 

\begin{lem}\label{lem:postcomp-equiv}
  For any equivalence $e:\eqv{X}{Y}$ in a univalent universe $\UU$, and any type $A$, the post-composition map
  \begin{equation*}
    e\circ\blank : (A \to X) \to (A\to Y)
  \end{equation*}
  is an equivalence.
\end{lem}

Note that this statement was also part of \cref{ex:equiv-postcomp}. That exercise is solved using function extensionality. However, since our present goal is to derive function extensionality from the univalence axiom, we cannot make use of that exercise. Therefore we give a new proof, using the univalence axiom.

\begin{proof}
  Since $\UU$ is assumed to be a univalent universe, it satisfies by \cref{thm:univalence} the principle of equivalence induction. Therefore, it suffices to show that the post-composition map
  \begin{equation*}
    \idfunc\circ\blank : (A\to X)\to (A\to X)
  \end{equation*}
  is an equivalence. This post-composition map is of course just the identity map on $A\to X$, so it is indeed an equivalence.
\end{proof}

\begin{thm}\label{thm:funext-univalence}\index{univalence axiom!implies function extensionality}\index{function extensionality!univalence implies function extensionality}
  For any universe $\UU$, the univalence axiom on $\UU$ implies function extensionality on $\UU$.
\end{thm}

\begin{proof}
  Note that by \cref{thm:funext_wkfunext}\index{weak function extensionality} it suffices to show that univalence implies weak function extensionality. We note that the proof of \cref{thm:funext_wkfunext} also goes through when it is restricted to types in $\UU$.
  
Suppose that $B:A\to \UU$ is a family of contractible types. Our goal is to show that the product $\prd{x:A}B(x)$ is contractible.
Since each $B(x)$ is contractible, the projection map $\proj 1:\big(\sm{x:A}B(x)\big)\to A$ is an equivalence by \cref{ex:proj_fiber}.

Now it follows by \cref{lem:postcomp-equiv} that $\proj1\circ\blank$ is an equivalence. Consequently, it follows from \cref{thm:contr_equiv} that the fibers of
\begin{equation*}
\proj 1\circ\blank : \Big(A\to \sm{x:A}B(x)\Big)\to (A\to A)
\end{equation*}
are contractible. In particular, the fiber at $\idfunc[A]$ is contractible. Therefore it suffices to show that $\prd{x:A}B(x)$ is a retract of $\sm{f:A\to\sm{x:A}B(x)}\proj 1\circ f=\idfunc[A]$. In other words, we will construct a section-retraction pair
\begin{equation*}
\begin{tikzcd}[column sep=1em]
\Big(\prd{x:A}B(x)\Big) \arrow[r,"i"] & \Big(\sm{f:A\to\sm{x:A}B(x)}\proj 1\circ f=\idfunc[A]\Big) \arrow[r,"r"] & \Big(\prd{x:A}B(x)\Big),
\end{tikzcd}
\end{equation*}
with $H:r\circ i\htpy \idfunc$.

We define the function $i$ by
\begin{equation*}
  i(f) \defeq (\lam{x}(x,f(x)),\refl{\idfunc}).
\end{equation*}
To see that this definition is correct, we need to know that
\begin{equation*}
  \lam{x}\proj 1(x,f(x))\jdeq \idfunc.
\end{equation*}
This is indeed the case, by the rule $\lambda$-eq for $\Pi$-types, on \cpageref{page:lambda-eq}.

Next, we define the function $r$. Consider a function $h:A\to \sm{x:A}B(x)$ equipped with an identification $p:\proj 1 \circ h = \idfunc$. Then we have the homotopy $\htpyeq(p):\proj 1 \circ h \htpy \idfunc$. Furthermore, we obtain $\proj 2(h(x)):B(\proj 1(h(x)))$. Using these ingredients, we define $r$ by
\begin{equation*}
  r((h,p),x)\defeq \tr_B(\htpyeq(p,x),\proj 2(h(x))).
\end{equation*}

It remains to construct a homotopy $H:r\circ i\htpy \idfunc$. We simply compute
\begin{align*}
  r(i(f)) & \jdeq r(\lam{x}(x,f(x)),\refl{}) \\
          & \jdeq \tr_B(\htpyeq(\refl{},x),\proj 2(x,f(x))) \\
          & \jdeq \tr_B(\refl{},f(x)) \\
          & \jdeq f(x).
\end{align*}
Thus we see that $r\circ i\jdeq \idfunc$ by an application of the $\eta$-rule for $\Pi$-types. Therefore we simply define $H(f)\defeq\refl{}$.
\end{proof}

\subsection{Maps and families of types}

Using the univalence axiom, we can establish a fundamental relation between maps into a type $A$, and families of types indexed by $A$. A special case of this relation asserts that the type of all pairs $(X,e)$ consisting of a type $X$ and an embedding $e:X\emb A$ is equivalent to the type of all subtypes of $A$, i.e., the type of all families $P$ of propositions indexed by $A$.

\begin{thm}\label{thm:object-classifier}\index{type family}
  For any type $A$ and any univalent universe $\UU$ containing $A$, the map
  \begin{equation*}
    \Big(\sm{X:\UU}X\to A\Big)\to (A\to \UU)
  \end{equation*}
  given by $(X,f)\mapsto\fibf{f}$ is an equivalence.\index{fib f b@{$\fib{f}{b}$}}
\end{thm}

\begin{proof}
  The map in the converse direction is given by
  \begin{equation*}
    B\mapsto \Big(\sm{x:A}B(x),\proj 1\Big). 
  \end{equation*}
  To verify that this map is a section of the asserted map, we have to prove that
  \begin{equation*}
    \fibf{\proj 1}=B
  \end{equation*}
  for any $B:A\to\UU$. By function extensionality and the univalence axiom, this is equivalent to
  \begin{equation*}
    \prd{x:A}\fib{\proj 1}{x}\simeq B(x).
  \end{equation*}
  Such a family of equivalences was constructed in \cref{ex:proj_fiber}.

  It remains to verify that
  \begin{equation*}
    (X,f)=\Big(\sm{x:A}\fib{f}{x},\proj 1\Big). 
  \end{equation*}
  Before we do this, we claim that the identity type
  \begin{equation*}
    (X,f)=(Y,g)
  \end{equation*}
  in the type $\sm{X:\UU}X\to A$ is equivalent to the type of pairs $(e,f)$ consisting of an equivalence $e:X\simeq Y$ equipped with a homotopy $f\htpy g\circ e$. This fact follows from \cref{thm:id_fundamental}, because the type
  \begin{equation*}
    \sm{Y:\UU}\sm{g:Y\to A}\sm{e:X\simeq Y}f\htpy g\circ e
  \end{equation*}
  is contractible by the structure identity principle, \cref{thm:structure-identity-principle}.

  To finish the proof, it therefore suffices to construct an equivalence
  \begin{equation*}
    e:X\simeq \sm{a:A}\fib{f}{a}
  \end{equation*}
  equipped with a homotopy $f\htpy \proj 1\circ e$. Such an equivalence $e$ equipped with a homotopy was constructed in \cref{ex:fib_replacement}.
\end{proof}

The following corollary is so important, that we call it again a theorem.

\begin{thm}\label{thm:classifier-subuniverse}
  Consider a type $A$ and a univalent universe $\UU$ containing $A$. Furthermore, let $P$ be a family of types indexed by $\UU$, and write
  \begin{equation*}
    \UU_P\defeq\sm{X:\UU}P(X).
  \end{equation*}
  Then the map
  \begin{equation*}
    \Big(\sm{X:\UU}\sm{f:X\to A}\prd{a:A}P(\fib{f}{a})\Big)\to (A\to\UU_P)
  \end{equation*}
  given by $(X,f,p)\mapsto \lam{a}(\fib{f}{a},p(a))$ is an equivalence.
\end{thm}

\begin{proof}
  The asserted map is homotopic to the composition of the equivalences
  \begin{align*}
    & \sm{X:\UU}\sm{f:X\to A}\prd{a:A}P(\fib{f}{a}) \\
    & \simeq \sm{(X,f):\sm{X:\UU}X\to A}\prd{a:A}P(\fib{f}{a}) \\
    & \simeq \sm{B:A\to \UU}\prd{a:A}P(B(a)) \\
    & \simeq A\to\sm{X:\UU}P(X).\qedhere
  \end{align*}
\end{proof}

\cref{thm:classifier-subuniverse} applies to any subuniverse. Examples include the subuniverse of $k$-types, for any truncation level $k$, the subuniverse of decidable propositions, the subuniverse of finite types, the subuniverse of inhabited types, and so on. It also applies to type families over $\UU$ that aren't families of propositions. The families $P\defeq\isdecidable$ and $P\defeq\cnt$ are examples.

\begin{cor}\label{cor:subtype}
  Consider a type $A$ and a univalent universe $\UU$ containing $A$. Then the map\index{embedding}\index{subtype}\index{fib f b@{$\fib{f}{b}$}}
  \begin{equation*}
    \Big(\sm{X:\UU}X\emb A\Big)\to (A\to\prop_\UU)
  \end{equation*}
  given by $(X,f)\mapsto \fibf{f}$ is an equivalence.
\end{cor}

In other words, a subtype of a type $A$ is equivalently described as a type $X$ equipped with an embedding $e:X\hookrightarrow A$. This brings us to an important point about equality of subtypes.

\begin{rmk}
  By function extensionality and propositional extensionality, it follows that two subtypes $P,Q:A\to\prop_\UU$ are the same if and only if\index{characterization of identity type!of subtypes of A@{of subtypes of $A$}}\index{subtype!characterization of identity type}
\begin{equation*}
  P(a)\leftrightarrow Q(a)
\end{equation*}
holds for all $a:A$. In other words, two subtypes of $A$ are the same if and only if they contain the same elements of $A$.

On the other hand, by \cref{cor:subtype} we can also consider two types $X$ and $Y$ equipped with embeddings $f:X\emb A$ and $g:Y\emb A$ as subtypes of $A$. Using the structure identity principle, \cref{thm:structure-identity-principle}, we see that the identity type $(X,f)=(Y,g)$ in the type $\sm{X:\UU}X\emb A$ is equivalent to the type
\begin{equation*}
  \sm{e:X\simeq Y}f\htpy g\circ e.
\end{equation*}
In other words, two subtypes $(X,f)$ and $(Y,g)$ of $A$ are equal if and only if there is an equivalence $X\simeq Y$ that is compatible with the embeddings $f:X\emb A$ and $g:Y\emb X$. Indeed, this condition is equivalent to the previous condition that two subtypes are the same if and only if they have the same elements.

We see that the combination of the structure identity principle\index{structure identity principle} and the univalence axiom automatically characterizes equality of subtypes in the most natural way, and we will see similar natural characterizations of identity types throughout the remainder of this book.
\end{rmk}

\subsection{Classical mathematics with the univalence axiom}

In classical mathematics, the axiom of choice asserts that for any collection $X$ of nonempty sets, there is a choice function $f$ such that $f(x)\in x$ for each $x\in X$. The univalence axiom is consistent with the axiom of choice, but we have to be careful in our formulation of the axiom of choice to make it about sets. A naive interpretation that would be applicable to all types, such as the assertion that every family $B$ of inhabited types has a section, is not consistent with univalence. We will use the type $\BS_2$ of $2$-element types for a counterexample.

\begin{prp}\label{prp:Eq-F2}
  The type\index{BS 2@{$\BS_2$}!characterization of identity type}\index{characterization of identity type!of BS 2@{of $\BS_2$}}
  \begin{equation*}
    \sm{X:\BS_2}X
  \end{equation*}
  of pointed $2$-element types is contractible. Consequently, the canonical family of maps
  \begin{equation*}
    (\Fin{2}= X) \to X
  \end{equation*}
  indexed by $X:\BS_2$, is a family of equivalences.
\end{prp}

\begin{proof}
  By the univalence axiom it follows that the type $\sm{X:\BS_2}\Fin{2}\simeq X$ is contractible. In order to show that $\sm{X:\BS_{2}}X$ is contractible, it therefore suffices to show that the map
  \begin{equation*}
    f: (\Fin{2}\simeq X)\to X
  \end{equation*}
  given by $f(e)\defeq e(\star)$, is an equivalence. Since being an equivalence is a proposition by \cref{ex:isprop_isequiv}, we may assume an equivalence $\alpha:\Fin{2}\simeq X$, and we proceed by equivalence induction on $\alpha$. Therefore, it suffices to show that the map
  \begin{equation*}
    f: (\Fin{2}\simeq \Fin{2})\to\Fin{2}
  \end{equation*}
  give by $f(e)\defeq e(\star)$ is an equivalence. Using the notation from \cref{sec:Fin}, we define the inverse map $g$ by
  \begin{align*}
    g(\star) & \defeq \idfunc \\
    g(i(\star)) & \defeq \succFin_2,
  \end{align*}
  and it is a straightforward verification that $f$ and $g$ are inverse to each other.
\end{proof}

\begin{cor}\label{cor:no-section-F2}
  There is no dependent function\index{BS 2@{$\BS_2$}!is not contractible}
  \begin{equation*}
    \prd{X:\BS_2}X.
  \end{equation*}
\end{cor}

\begin{proof}
  By \cref{prp:Eq-F2,ex:equiv-pi}, we have an equivalence
  \begin{equation*}
    \Big(\prd{X:\BS_2}\Fin{2}=X\Big)\simeq \Big(\prd{X:\BS_2}X\Big).
  \end{equation*}
  Note that $\prd{X:\BS_2}\Fin{2}=X$ is the type of contractions of $\BS_2$, using the center of contraction $\Fin{2}$. Therefore it suffices to show that $\BS_2$ is not contractible. Recall from \cref{ex:prop_contr} that the identity types of contractible types are contractible. On the other hand, it follows from \cref{prp:Eq-F2} that the identity type $\Fin{2}=\Fin{2}$ in $\BS_2$ is equivalent to $\Fin{2}$. This type isn't contractible by \cref{ex:is-not-contractible-Fin}. We conclude that $\BS_2$ is not contractible.
\end{proof}

The family $X\mapsto X$ over $\BS_2$ is therefore an example of a family of nonempty types for which there are provably no sections. In the following corollary we conclude more generally that there is no way to construct an element of an arbitrary inhabited type.

\begin{cor}\label{cor:no-global-choice}
  If $\UU$ is a univalent universe, then there is no \define{global choice}\index{global choice|textbf} function
  \begin{equation*}
    \prd{A:\UU}\brck{A}\to A.
  \end{equation*}
\end{cor}

\begin{proof}
  Suppose $f:\prd{A:\UU}\brck{A}\to A$. By restricting $f$ to the type of $2$-element types in $\UU$, we obtain a function
  \begin{equation*}
    \prd{A:\BS_2}\brck{A}\to A.
  \end{equation*}
  Note that every $2$-element type is inhabited, i.e., there is an element of type $\brck{A}$ for every $2$-element type $A$. To see this, consider a type $A:\UU$ such that $\brck{\Fin{2}\simeq A}$. To obtain an element of type $\brck{A}$, we may assume an equivalence $e:\Fin{2}\simeq A$. Then we have $\eta(e(0)):\brck{A}$.

  Since every $2$-element type is inhabited, we obtain a function $\prd{A:\BS_2}A$, which is impossible by \cref{cor:no-section-F2}.
\end{proof}

\cref{cor:no-global-choice} is of philosophical importance. It shows that the \define{principle of global choice} is incompatible with the univalence axiom, i.e., that there is no way to obtain construct a function $\brck{A}\to A$ for all types $A$. In other words, we cannot obtain an element of $A$ merely from the assumption that the type $A$ is inhabited. What is the obstruction? It is the fact that no such choice of an element of $A$ can be invariant under the automorphisms on $A$, i.e., under the self-equivalences on $A$. Indeed, in the example where $A$ is the $2$-element type $\Fin{2}$ there are no fixed point of the equivalence $\succFin_2:\Fin{2}\simeq\Fin{2}$. By the univalence axiom, there is an identification $p:\Fin{2}=\Fin{2}$ in $\UU$, such that $\tr(p,x)=\succFin_2(x)$. If we had a function
\begin{equation*}
  f:\prd{X:\UU}\brck{X}\to X,
\end{equation*}
the dependent action on paths of $f$ would give an identification
\begin{equation*}
  \apd{f}{p}:\succFin_2(f(\Fin{2},p,H))=f(\Fin{2},p,\eta(0)).
\end{equation*}
In other words, it would give us a fixed point for the successor function on $\Fin{2}$.

This is perhaps a good moment to stress that the axiom of choice is really an axiom about \emph{sets}, not about more general types. And indeed, when we restrict the axiom of choice to sets, it turns out to be consistent with the univalence axiom and therefore safe to assume. In this book, however, we will not have many applications for the axiom of choice and therefore we will not assume it, unless we explicitly say otherwise.

\begin{defn}
  The \define{axiom of choice}\index{axiom of choice|textbf}\index{axiom!axiom of choice|textbf} asserts that for any family $B$ of inhabited sets indexed by a set $A$, the type of sections of $B$ is also inhabited, i.e., it asserts that there is an element of type
  \begin{equation*}
    \AC_{\UU}(A,B)\defeq \Big(\prd{x:A}\brck{B(x)}\Big)\to\Brck{\prd{x:A}B(x)},
  \end{equation*}
  for every $A:\Set_\UU$ and $B:A\to\Set_\UU$.
\end{defn}

Similar care has to be taken with the type theoretic formulation of the law of excluded middle. It is again inconsistent to assume that every type is decidable.

\begin{thm}
  There is no \define{global decidability function}\index{global decidability|textbf}
  \begin{equation*}
    \prd{X:\UU}\isdecidable(X). 
  \end{equation*}
\end{thm}

\begin{proof}
  Suppose there is such a dependent function $d$. By restricting $d$ to the subuniverse of $2$-element types, we obtain a dependent function
  \begin{equation*}
    d:\prd{X:\BS_2}\isdecidable(X).
  \end{equation*}
  However, each $2$-element type $X$ is inhabited. By \cref{ex:propositional-truncations-drill} we obtain a function
  \begin{equation*}
    \isdecidable(X)\to X
  \end{equation*}
  for each $2$-element type $X$. Therefore, we obtain from $d$ a dependent function $\prd{X:\BS_2}X$, which does not exist by \cref{cor:no-section-F2}.
\end{proof}

The law of excluded middle is really an axiom of propositional logic, and it is indeed consistent with the univalence axiom that every \emph{proposition} is decidable.

\begin{defn}
  The \define{law of excluded middle}\index{law of excluded middle|textbf}\index{axiom!law of excluded middle|textbf} asserts that every proposition is decidable, i.e.,
  \begin{equation*}
    \LEM_\UU\defeq \prd{P:\prop_\UU}\isdecidable(P).
  \end{equation*}
\end{defn}

We will again not assume the law of excluded middle, unless we say otherwise. Nevertheless, we have seen in \cref{sec:decidability} that some propositions are already decidable without assuming the law of excluded middle, and decidability remains an important concept in type theory and mathematics.

\subsection{The binomial types}
\index{binomial type|(}

To wrap up this section on univalence, we will use the univalence axiom to construct for any two types $A$ and $B$ a type $\dbinomtype{A}{B}$ that has properties similar to the binomial coefficients $\dbinomtype{n}{k}$. Indeed, we will show that if $A$ is an $n$-element type and $B$ is a $k$-element type, then $\dbinomtype{A}{B}$ is an $\binom{n}{k}$-element type. The binomial types are defined using decidable embeddings.

\begin{defn}
  A map $f:A\to B$ is said to be \define{decidable}\index{decidable map|textbf} if it comes equipped with an element of type
  \begin{equation*}
    \isdecidable(f) \defeq \prd{b:B}\isdecidable(\fib{f}{b}).
  \end{equation*}
  We will write $A\demb B$ for the type of \define{decidable embeddings}\index{decidable embedding|textbf}\index{A hookrightarrow d B@{$A \demb B$}|see {decidable embedding}} from $A$ to $B$, i.e., for the type of embeddings that are also decidable maps. 
\end{defn}

\begin{defn}
  Consider a type $A$ and a universe $\UU$. We define the \define{connected component}\index{universe!connected component|textbf}\index{connected component!of a universe|textbf} of $\UU$ at $A$ by\index{U A@{$\UU_A$}|textbf}
  \begin{equation*}
    \UU_A\defeq \sm{X:\UU}\brck{A\simeq X}.
  \end{equation*}
\end{defn}

\begin{eg}
  Note that type $\UU_{\Fin{n}}$ is the type $\BS_n$ of all $n$-element types. Note also that if $A\simeq B$, then $\UU_A\simeq\UU_B$.\index{BS n@{$\BS_n$}}
\end{eg}

\begin{defn}\label{defn:binomial-type}
  Consider two types $A$ and $B$ and a universe $\UU$ containing both $A$ and $B$. We define the \define{binomial type}\index{binomial type|textbf} $\dbinomtype[\UU]{A}{B}$\index{(A B)@{$\dbinomtype{A}{B}$}|see {binomial type}} by
  \begin{equation*}
    \dbinomtype[\UU]{A}{B} \defeq \sm{X:\UU_B}X\demb A.
  \end{equation*}
\end{defn}

\begin{rmk}
  We define the binomial types using decidable embeddings because the usual properties of binomial coefficients generalize most naturally under the extra assumption of decidability. In particular the binomial theorem for types, which is stated as \cref{ex:binomial-theorem} and generalized in \cref{ex:distributive-pi-coprod}, rely on the use of decidable embeddings.
\end{rmk}

\begin{prp}\label{prp:equiv-binom-type}
  Consider two types $A$ and $B$, and a universe $\UU$ containing both $A$ and $B$. Then we have an equivalence
  \begin{equation*}
    \dbinomtype[\UU]{A}{B}\simeq \sum_{(P:A\to\decidableProp_\UU)}\Brck{B\simeq\sm{a:A}P(a)}.
  \end{equation*}
  from the binomial type $\dbinomtype[\UU]{A}{B}$ to the type of decidable subtypes\index{decidable subtype} of $A$ that are merely equivalent to $B$.
\end{prp}

\begin{proof}
  This equivalence follows from \cref{thm:classifier-subuniverse}, by which we have
  \begin{equation*}
    \Big(\sm{X:\UU}X\demb A\Big)\simeq (A\to\decidableProp_\UU).\qedhere
  \end{equation*}
\end{proof}

\begin{rmk}
  Combining \cref{cor:decidable-Prop,prp:equiv-binom-type}, we obtain an equivalence
  \begin{equation*}
    \dbinomtype[\UU]{A}{B}\simeq \sum_{(f:A\to\bool)}\Brck{B\simeq\sm{a:A}f(a)=\btrue}.
  \end{equation*}
  for any universe $\UU$ that contains both $A$ and $B$. This equivalence is important, because the right hand side doesn't depend on the universe $\UU$. Therefore we will simply write $\dbinomtype{A}{B}$ for $\dbinomtype[\UU]{A}{B}$, if the universe $\UU$ contains both $A$ and $B$.
\end{rmk}
  
\begin{lem}\label{prp:binomtype-recursion}
  For any two types $A$ and $B$, we have equivalences\index{binomial type!recursive relations}
  \begin{align*}
    \dbinomtype{\emptyt}{\emptyt} & \simeq \unit & \dbinomtype{A+\unit}{\emptyt} & \simeq \unit \\
    \dbinomtype{\emptyt}{B+\unit} & \simeq \emptyt & \dbinomtype{A+\unit}{B+\unit} & \simeq \dbinomtype{A}{B}+\dbinomtype{A}{B+\unit}.
  \end{align*}
\end{lem}

\begin{proof}
  For the first two equivalences, we prove that $\dbinomtype{X}{\emptyt}$ is contractible for any type $X$. To see this, we first note that the type $\UU_\emptyt$ is contractible. Indeed, at the center of contraction we have the empty type, and any two types that are merely equivalent to the empty type are empty and hence equivalent. Therefore it follows that
  \begin{equation*}
    \dbinomtype{X}{\emptyt}\simeq \emptyt\demb X.
  \end{equation*}
  The type of decidable embeddings $\emptyt\demb X$ is contractible, because the type $\emptyt\to X$ is contractible with the map $\exfalso:\emptyt\to X$ at the center of contraction, which is of course a decidable embedding.

  Next, the fact that the binomial type $\dbinomtype{\emptyt}{B+\unit}$ is empty follows from the fact that the type of maps $X\to\emptyt$ is empty for any type $X$ merely equivalent to $B+\unit$. 

  For the last equivalence we will use \cref{prp:equiv-binom-type}. Using the universal property of $A+\unit$, we see that
  \begin{equation*}
    \dbinomtype{A+\unit}{B+\unit}\simeq \sm{P:A\to\decidableProp_\UU}\sm{Q:\decidableProp_\UU}\brck{(B+\unit)\simeq \sm{a:A}P(a)+ Q}.
  \end{equation*}
  Using the fact that $\decidableProp_\UU\simeq\Fin{2}$, observe that we have an equivalence
  \begin{align*}
    & \sm{Q:\decidableProp_\UU}\brck{(B+\unit)\simeq(\sm{a:A}P(a)+Q)} \\
    & \qquad\simeq \brck{(B+\unit)\simeq(\sm{a:A}P(a)+\unit)}+\brck{(B+\unit)\simeq\sm{a:A}P(a)}.
  \end{align*}
  Furthermore, note that it follows from \cref{prp:is-injective-maybe} that
  \begin{equation*}
    \brck{(B+\unit)\simeq(\sm{a:A}P(a)+\unit)}\simeq\brck{B\simeq\sm{a:A}P(a)}.
  \end{equation*}
  Thus we see that
  \begin{align*}
    \dbinomtype{A+\unit}{B+\unit} & \simeq \Big(\sm{P:A\to\decidableProp_\UU}\brck{B\simeq\sm{a:A}P(a)}\Big) \\
    & \qquad\qquad +\Big(\sm{P:A\to\decidableProp_\UU}\brck{(B+\unit)\simeq\sm{a:A}P(a)}\Big).\qedhere
  \end{align*}
\end{proof}

\begin{thm}
  If $A$ and $B$ are finite types of cardinality $n$ and $k$, respectively, then the type $\dbinomtype{A}{B}$ is finite of cardinality $\binom{n}{k}$.\index{finite type!binomial type}\index{binomial type!is finite}
\end{thm}

\begin{proof}
  The claim that the type $\binomtype{A}{B}$ is finite of cardinality $\binom{n}{k}$ is a proposition, so we may assume $e:\Fin{n}\simeq A$ and $f:\Fin{k}\simeq B$. The claim now follows by induction on $n$ and $k$, using \cref{prp:binomtype-recursion}.
\end{proof}

\begin{rmk}
  It is perhaps remarkable that the type $\sm{X:\UU_B}X\demb A$ is a good generalisation of the binomial coefficients to types. Note that when $A$ and $B$ are finite types of cardinality $n$ and $k$, respectively, then the type $B\demb A$ has a factor $k!$ too many elements. When we seemingly enlarge it by the type $\UU_B$ of all types merely equivalent to $B$, it turns out that we obtain the correct generalisation of the binomial coefficients.

  One reason why it works is that the identity type of $\sm{X:\UU_B}X\demb A$ is characterized, via the univalence axiom, by
  \begin{equation*}
    ((X,f)=(Y,g))\simeq\sm{e:X\simeq Y}f\htpy g\circ e. 
  \end{equation*}
  Therefore it follows that for any two decidable embeddings $f,g:B\demb A$, if $f$ and $g$ are the same up to a permutation on $B$, then we get an identification $(B,f)=(B,g)$ in the type $\sm{X:\UU_B}X\demb A$.

  From a group theoretic perspective we may observe that the automorphism group $B\simeq B$ acts freely on the set of decidable embeddings $\B\demb A$, and the type $\sm{X:\UU_B}X\hookrightarrow A$ can be viewed as the type of orbits of that action. Since this action of $\Aut(B)$ on $B\demb A$ is free, we see that the number of orbits is $\frac{1}{k!}$ times the number of elements in $B\demb A$. 
\end{rmk}
\index{binomial type|)}

\begin{exercises}
  \exitem \label{ex:istrunc_UUtrunc}
  \begin{subexenum}
  \item Use the univalence axiom to show that the type $\sm{A:\UU}\iscontr(A)$ of all contractible types in $\UU$ is contractible.\index{universe!of contractible types}\index{contractible type}
  \item Use the univalence axiom and \cref{ex:isprop_istrunc,ex:isprop_isequiv} to show that the universe of $k$-types\index{universe!of k-types@{of $k$-types}}\index{U leq k@{$\UU^{\leq k}$}}\index{k-type@{$k$-type}!universe of k-types@{universe of $k$-types}}\index{truncated type!universe of k-types@{universe of $k$-types}}\index{universe!of k-types@{of $k$-types}!is a k-type@{is a $k$-type}}\index{U leq k@{$\UU^{\leq k}$}!is a k-type@{is a $k$-type}}
    \begin{equation*}
      \UU^{\leq k}\defeq \sm{X:\UU}\istrunc{k}(X)
    \end{equation*}
    is a $(k+1)$-type, for any $k\geq -2$.
  \item Show that $\prop_\UU$ is not a proposition.\index{universe!of propositions}\index{Prop@{$\prop_\UU$}}
  \item Show that the universe $\Set_\UU$ of sets\index{universe!of sets} in $\UU$ is not a set. \index{Set@{$\Set_\UU$}!is not a set}
  \end{subexenum}
  \exitem Give an example of a type family $B$ over a type $A$ for which the implication
  \begin{equation*}
    \neg\Big(\prd{x:A}B(x)\Big) \to \Big(\sm{x:A}\neg B(x)\Big)
  \end{equation*}
  is false.
  \exitem Show that the law of excluded middle holds if and only if every set has decidable equality.\index{law of excluded middle}\index{axiom!law of excluded middle}
  \exitem Consider a type $A$ and a univalent universe $\UU$ containing $A$. Construct an equivalence
  \begin{equation*}
    A\simeq\sm{B:A\to\UU}\iscontr\left(\sm{a:A}B(a)\right).
  \end{equation*}
  \exitem \label{ex:surjective-precomp}Consider a map $f:A\to B$. Show that the following are equivalent:
  \begin{enumerate}
  \item The map $f$ is surjective.\index{surjective map}
  \item For every set $C$, the precomposition function
    \begin{equation*}
      \blank\circ f:(B\to C)\to (A\to C)
    \end{equation*}
    is an embedding.
  \end{enumerate}
  Hint: To show that (ii) implies (i), use the assumption with the set $C\defeq\prop_\UU$, where $\UU$ is a univalent universe containing both $A$ and $B$.
  \exitem (Escard\'o)\label{ex:idtype-is-emb} Consider a type $A$ in $\UU$. Show that the identity type, seen as a function\index{Id A@{$\idtypevar{A}$}!is an embedding}\index{is an embedding!Id A@{$\idtypevar{A}$}}
  \begin{equation*}
    \idtypevar{} : A \to (A\to\UU),
  \end{equation*}
  is an embedding.
  \exitem
  \begin{subexenum}
  \item For any type $A$ in $\UU$, consider the function\index{S A@{$\Sigma_A$}|see {dependent pair type}}\index{S A@{$\Sigma_A$}!is a k-truncated map@{is a $k$-truncated map}}\index{truncated map!S A@{$\Sigma_A$}}
    \begin{equation*}
      \Sigma_A : (A \to \UU) \to \UU,
    \end{equation*}
    which takes a family $B$ of $\UU$-small types to its $\Sigma$-type. Show that the following are equivalent:
    \begin{enumerate}
    \item The type $A$ is $k$-truncated.
    \item The map $\Sigma_A$ is $k$-truncated.
    \end{enumerate}
    Hint: Construct an equivalence $\fib{\Sigma_A}{X}\simeq (X\to A)$.\index{fiber!of S A@{of $\Sigma_A$}}
  \item Show that the map ${+}:\UU\times\UU\to\UU$, which takes $(A,B)$ to the coproduct $A+B$, is $0$-truncated.\index{A + B@{$A+B$}!is a 0-truncated map@{${\blank}+{\blank}$ is a $0$-truncated map}}\index{truncated map!-+-@{${\blank}+{\blank}$}}
  \end{subexenum}
  \exitem (Escard\'o) Consider a proposition $P$ and a universe $\UU$ containing $P$. Show that the map
  \begin{equation*}
    \Pi_P : (P\to \UU)\to\UU,
  \end{equation*}
  given by $A\mapsto\prd{p:P}A(p)$, is an embedding.\index{P P@{$\Pi_P$}!is an embedding}
  \exitem Consider two types $A$ and $B$ and a universe $\UU$ containing both $A$ and $B$. A \define{binary correspondence}\index{binary correspondence|textbf} $R:A\to(B\to\UU)$ is said to be a \define{function}\index{binary correspondence!function|textbf}\index{function!binary correspondence|textbf} if it satisfies the condition\index{is-function@{$\isfunction(R)$}|textbf}
  \begin{equation*}
    \isfunction(R)\defeq\prd{a:A}\iscontr\Big(\sm{b:B}R(a,b)\Big),
  \end{equation*}
  and $R$ is said to be an \define{opposite function}\index{function!opposite function|textbf}\index{opposite function|textbf} if the \define{opposite correspondence}\index{correspondence!opposite correspondence}\index{opposite correspondence}\index{R op@{$\op{R}$}|see {opposite correspondence}} $\op{R}:B\to(A\to\UU)$ given by $\op{R}(b,a)\defeq R(a,b)$ is functional.
  \begin{subexenum}
  \item Construct an equivalence
    \begin{equation*}
      (A\to B)\simeq \sm{R:A\to(B\to\UU)}\isfunction(R).
    \end{equation*}
  \item Construct an equivalence
    \begin{equation*}
      (A\simeq B)\simeq\sm{R:A\to (B\to\UU)}\isfunction(R)\times\isfunction(\op{R}).
    \end{equation*}
  \end{subexenum}
  \exitem
  \begin{subexenum}
  \item For any $k:\N$, show that the type\index{BS n@{$\BS_n$}}
    \begin{equation*}
      \sm{X:\BS_{k+1}}\Fin{k}\hookrightarrow X
    \end{equation*}
    is contractible.
  \item More generally, construct for any $k,l:\N$ and any $k$-element type $A$ an equivalence
    \begin{equation*}
      \Big(\sm{X:\BS_{k+l}}A\hookrightarrow X\Big)\simeq \BS_l
    \end{equation*}
  \end{subexenum}
  \exitem \label{ex:complement-Fk}
  \begin{subexenum}
  \item For any type $A$, construct an equivalence
  \begin{equation*}
    \UU_A \simeq \sum_{(X:\UU_{A+\unit})}\dbinomtype{X}{\unit}.
  \end{equation*}
  \item For any $k:\N$, construct an equivalence\index{BS n@{$\BS_n$}}
    \begin{equation*}
      \Big(\sm{X:\BS_{k+1}}X\Big)\simeq \BS_k.
    \end{equation*}
    In other words, show that the type of $(k+1)$-element types equipped with a point is equivalent to the type of $k$-element types. Conclude that the type of pointed finite types is equivalent to the type of finite types, i.e., conclude that we have an equivalence\index{F@{$\F$}}
  \begin{equation*}
    \Big(\sm{X:\F}X\Big)\simeq \F.
  \end{equation*}
  \end{subexenum}
  \exitem 
  \begin{subexenum}
  \item Show that for $k\neq 2$, the type\index{BS n@{$\BS_n$}}
    \begin{equation*}
      \prd{X:\BS_k}X\to X
    \end{equation*}
    is contractible. Conclude that the type $\prd{X:\BS_k}X\simeq X$ is also contractible. Hint: Use \cref{ex:complement-Fk}.
  \item Show that the type\index{BS 2@{$\BS_2$}}
    \begin{equation*}
      \prd{X:\BS_2}X\to X
    \end{equation*}
    is equivalent to $\Fin{2}$. Conclude that the type $\prd{X:\BS_2}X\simeq X$ is also equivalent to $\Fin{2}$.
  \end{subexenum}
  \exitem Consider a type $A$.
  \begin{subexenum}
  \item Recall from \cref{ex:isolated-point} that an element $a:A$ is isolated\index{isolated point} if and only if the map $\const_a:\unit\to A$ is a decidable embedding. Construct an equivalence
  \begin{equation*}
    \dbinomtype{A}{\unit}\simeq \sm{a:A}\isisolated(a).
  \end{equation*}
  \item Construct an equivalence
    \begin{equation*}
      \dbinomtype{A}{\unit}\simeq\Big(\sm{X:\UU}(X+\unit)\simeq A\Big).
    \end{equation*}
    Conclude that the map $X\mapsto X+\unit$ on a univalent universe $\UU$ is $0$-truncated.\index{truncated map!-+1@{${\blank}+\unit$}}
  \item More generally, construct an equivalence\index{binomial type}
    \begin{equation*}
      \dbinomtype{A}{B} \simeq \sm{X:\UU_B}\sm{Y:\UU}(X+Y\simeq A).
    \end{equation*}
  \end{subexenum}
  \exitem \label{ex:binomial-theorem}For any $(X,i):\dbinomtype{A}{B}$, we define $A\setminus(X,i)\defeq (A\setminus X,A\setminus i):\dbinomtype{A}{B}$, where
  \begin{align*}
    A\setminus X & \defeq \sm{a:A}\neg(\fib{i}{a}) \\
    A\setminus i & \defeq \proj 1.
  \end{align*}
  Now consider a finite type $X$ and two arbitrary types $A$ and $B$. Construct an equivalence\index{binomial theorem}
  \begin{equation*}
    (A+B)^X\simeq\sm{k:\N}\sm{(Y,i):\dbinomtype{X}{\Fin{k}}}A^Y\times B^{X\setminus Y}.
  \end{equation*}
  \exitem Let $\UU$ be a univalent universe.
  \begin{subexenum}
  \item Consider a section-retraction pair
    \begin{equation*}
      \begin{tikzcd}
        A \arrow[r,"i"] & X \arrow[r,"r"] & A
      \end{tikzcd}
    \end{equation*}
    with $H:r\circ i\htpy \idfunc$. Show that if $X$ is $\UU$-small, then so is $A$. Hint: Use \cref{ex:retracts-as-limits}.\index{small type!retracts of small types are small}\index{retract!retracts of small types are small}
  \item Consider two inhabited types $A$ and $B$. Show that if the product $A\times B$ is $\UU$-small, then so are the types $A$ and $B$.\index{small type!products of small types are small}\index{cartesian product type!products of small types are small}
  \end{subexenum}
  \exitem Consider a finite type $X$ and a univalent universe $\UU$ containing $X$. Show that the type\index{Retr X@{$\Retr_\UU(X)$}!is finite}\index{finite type!Retr X of a finite type X@{$\Retr_\UU(X)$ of a finite type $X$}}
  \begin{equation*}
    \Retr_\UU(X)\defeq \sm{A:\UU}\sm{i:A\to X}\sm{r:X\to A}r\circ i\htpy\idfunc
  \end{equation*}
  of all retracts of $X$ is finite.
  \exitem Consider a $k$-truncated type $X$ and a univalent universe $\UU$ containing $X$. Show that the type
  \begin{equation*}
    \mathsf{Retr}_\UU(X)
  \end{equation*}
  of all retracts of $X$ is $k$-truncated.\index{Retr X@{$\Retr_\UU(X)$}!is a k-truncated type@{is a $k$-truncated type}}
  \exitem \label{ex:surjection-into-k-type}For any type $A$ and any $k\geq-1$, show that the type
  \begin{equation*}
    \sm{X:\typele{k}}A\twoheadrightarrow X
  \end{equation*}
  of $k$-truncated types $X$ equipped with a surjective map $A\twoheadrightarrow X$ is $k$-truncated, even though the type $\typele{k}$ itself is $(k+1)$-truncated.
  \exitem
  \begin{subexenum}
    \item Show that for $k\geq 3$, the type\index{BS n@{$\BS_n$}}
    \begin{equation*}
      \prd{X:\BS_k}(X+X)\emb (X\times X)+\unit
    \end{equation*}
    is empty, even though the inequality $2k\leq k^2+1$ holds for all $k:\N$.
  \item Show that the type\index{BS 2@{$\BS_2$}}
    \begin{equation*}
      \prd{X:\BS_2}(X+X)\emb (X\times X)+\unit
    \end{equation*}
    is equivalent to $\Fin{8}$.
  \end{subexenum}
  \exitem \label{ex:prime}For any natural number $n$ consider the type
  \begin{equation*}
    \tilde{D}_n\defeq\sm{X:\BS_2}\sm{Y:X\to\F}\Big(\Fin{n}\simeq\prd{x:X}Y(x)\Big).
  \end{equation*}
  \begin{subexenum}
  \item Show that $\tilde{D}_{1}\simeq \BS_2$.\index{BS 2@{$\BS_2$}}
  \item Show that $\tilde{D}_{n}$ is contractible if and only if $n$ is prime\index{prime number}.
  \item Show that $\tilde{D}_n$ is a set if and only if $n$ is not a square.
  \end{subexenum}
\end{exercises}
\index{univalence axiom|)}
\index{axiom!univalence|)}

%%% Local Variables:
%%% mode: latex
%%% TeX-master: "hott-intro"
%%% End:

\section{Set quotients}\label{sec:set-quotients}
\index{set quotient|(}

In this section we construct the quotient of a type by an equivalence relation. By an equivalence relation we understand a binary relation $R$ which is reflexive, symmetric, transitive, and moreover, we require that the type $R(x,y)$ relating $x$ and $y$ is a proposition. Therefore, if $\UU$ is a universe that contains $R(x,y)$ for each $x,y:A$, then we can view $R$ as a map
\begin{equation*}
  R:A\to(A\to\prop_\UU).
\end{equation*}
The quotient $A/R$ is constructed as the type of equivalence classes, which is just the image of the map $R:A\to (A\to\prop_\UU)$. This construction of the quotient by an equivalence relation is very much like the construction of a quotient set in classical set theory. Examples of set quotients are abundant in mathematics. We cover two of them in this section: the type of rational numbers and the set truncation of a type.

There is, however, a subtle issue with our construction of the set quotient as the image of the map $R:A\to(A\to\prop_\UU)$. What universe is the quotient $A/R$ in? Note that $\prop_\UU$ is a type in the successor universe $\UU^+$, constructed in \cref{defn:successor-universe}. Therefore the function type $A\to \prop_\UU$ as well as the quotient $A/R$ are also types in $\UU^+$. That seems unfortunate, because in Zermelo-Fraenkel set theory the quotient of a set by an equivalence relation is an ordinary set, and not a more general class.

To address the size issues of set quotients, we will introduce the type theoretic replacement axiom. This axiom is analogous to the replacement axiom in Zermelo-Fraenkel set theory, which asserts that the image of a set under any function is again a set. The type theoretic replacement property asserts that for any map $f:A\to B$ from a type $A$ in $\UU$ to a type $B$ of which the \emph{identity types} are equivalent to types in $\UU$, the image of $f$ is also equivalent to a type in $\UU$. The replacement axiom can either be assumed, or it can be proven from the assumption that universes are closed under certain \emph{higher inductive types}, and it is therefore considered to be a very mild assumption.

\subsection{Equivalence relations and the replacement axiom}

\begin{defn}\label{defn:eq_rel}
Consider a type $A$ and a universe $\UU$. Let $R:A\to (A\to\prop_\UU)$ be a binary relation on $A$ valued in the propositions in $\UU$\index{relation!valued in propositions}. We say that $R$ is an \define{equivalence relation}\index{equivalence relation|textbf}\index{relation!equivalence relation|textbf} if $R$ comes equipped with
\begin{align*}
\rho & : \prd{x:A}R(x,x) \\*
\sigma & : \prd{x,y:A} R(x,y)\to R(y,x) \\*
\tau & : \prd{x,y,z:A} R(x,y)\to (R(y,z)\to R(x,z)),
\end{align*}
witnessing that $R$ is reflexive\index{relation!reflexive}, symmetric\index{relation!symmetric}, and transitive\index{relation!transitive}. We write $\eqrel_\UU(A)$\index{Eq-Rel@{$\eqrel_\UU(A)$}|see {equivalence relation}}\index{Eq-Rel@{$\eqrel_\UU(A)$}|textbf} for the type of all equivalence relations on $A$ valued in the propositions in $\UU$.
\end{defn}

\begin{defn}
  Let $R:A\to (A\to\prop_\UU)$ be an equivalence relation. A subtype $P:A\to \prop_\UU$ is said to be an \define{equivalence class}\index{equivalence class|textbf}\index{equivalence relation!equivalence class|textbf} if it satisfies the condition
  \begin{equation*}
    \isequivalenceclass(P)\defeq\exists_{(x:A)}\forall_{(y:A)}P(y)\leftrightarrow R(x,y).
  \end{equation*}
  We define $A/R$\index{A/R@{$A/R$}|see {set quotient}}\index{set quotient|textbf} to be the type of equivalence classes, i.e., we define
  \begin{equation*}
    A/R\defeq \sm{P:A\to\prop_\UU}\isequivalenceclass(P).
  \end{equation*}
  Furthermore, we define \define{equivalence class of $x:A$}\index{[x] R@{$[x]_R$}|textbf}\index{equivalence relation![x] R@{$[x]_R$}|textbf}\index{set quotient![x] R@{$[x]_R$}|textbf} to be
  \begin{equation*}
    [x]_R\defeq R(x),
  \end{equation*}
  which is indeed an equivalence class. Sometimes we will write $q_R:A\to A/R$ for the map $x\mapsto [x]_R$.\index{q R@{$q_R$}|textbf}
\end{defn}

In other words, $A/R$ is the image of the map $R:A\to (A\to\prop_\UU)$. In the following proposition we characterize the identity type of $A/R$. As a corollary, we obtain equivalences
\begin{equation*}
  ([x]_R=[y]_R)\simeq R(x,y),
\end{equation*}
justifying that the quotient $A/R$ is defined to be the type of equivalence classes. Note that in our characterization of the identity type of $A/R$ we make use of propositional extensionality.

\begin{prp}\label{prp:eq-quotient}
  Let $R:A\to (A\to\prop_\UU)$ be an equivalence relation. Furthermore, consider $x:A$ and an equivalence class $P$. Then the canonical map\index{characterization of identity type!of set quotients}\index{set quotient!characterization of identity type}\index{identity type!of A/R@{of $A/R$}}
  \begin{equation*}
    ([x]_R=P)\to P(x)
  \end{equation*}
  is an equivalence.
\end{prp}

\begin{proof}
  By \cref{thm:id_fundamental} it suffices to show that the total space
  \begin{equation*}
    \sm{P:A/R}P(x)
  \end{equation*}
  is contractible. The center of contraction is of course $[x]_R$, which satisfies $[x]_R(x)$ by reflexivity of $R$. It remains to construct a contraction. Since $\sm{P:A/R}P(x)$ is a subtype of $A/R$, we construct a contraction by showing that
  \begin{equation*}
    [x]_R=P
  \end{equation*}
  whenever $P(x)$ holds. Since $P$ is an equivalence class there exists an element $y:A$ such that $P=[y]_R$. Note that our goal is a proposition, so we may assume that we have such a $y$. From the assumption that $P(x)$ holds, it follows that $R(x,y)$ holds. To complete the proof, it therefore is suffices to show that
  \begin{equation*}
    [x]_R=[y]_R,
  \end{equation*}
  assuming that $R(x,y)$ holds. By function extensionality and propositional extensionality, it is equivalent to show that
  \begin{equation*}
    \prd{z:A}R(x,z)\leftrightarrow R(y,z),
  \end{equation*}
  which follows directly from the assumption that $R$ is an equivalence relation.
\end{proof}

\begin{cor}\label{cor:eq-quotient}
  Consider an equivalence relation $R$ on a type $A$, and let $x,y:A$. Then there is an equivalence
  \begin{equation*}
    ([x]_R=[y]_R)\simeq R(x,y).
  \end{equation*}
\end{cor}

\begin{rmk}
  Notice that type of equivalence classes of an equivalence relation in $\UU$ is a type in the universe $\UU^+$ that contains $\UU$ and every type in $\UU$, or indeed in any universe $\VV$ containing $\UU$ and every type in $\UU$. Indeed, the type
  \begin{equation*}
    \prop_\UU\jdeq\sm{X:\UU}\isprop(X)
  \end{equation*}
  of propositions in $\UU$ is a type in $\VV$. It follows that the type $A\to\prop_\UU$ is a type in $\VV$. The type of equivalence classes of an equivalence relation $R$ on $A$ in $\UU$ is a subtype of $A\to\prop_\UU$ in $\UU$, so we conclude that $A/R$ is a type in $\VV$.
\end{rmk}

In classical mathematics, on the other hand, we consider the class of equivalence classes of an equivalence relation to be a (small) set. We will introduce the replacement axiom in order to ensure that set quotients in type theory are small.

Recall that in set theory, the replacement axiom asserts that for any family of sets $\{X_i\}_{i\in I}$ indexed by a set $I$, there is a set $X[I]$ consisting of precisely those sets $x$ for which there exists an $i\in I$ such that $x\in X_i$. In other words: the image of a set-indexed family of sets is again a set. Without the replacement axiom, $X[I]$ would be a class.

In type theory, we may similarly ask whether the image of a map $X:I\to\UU$ is $\UU$-small, assuming that $I$ is $\UU$-small. The replacement axiom settles a more general variant of this question. The key observation is that the identity types of $\UU$ are $\UU$-small by the univalence axiom. In other words, univalent universes are \emph{locally small} in the following sense.

\begin{defn}\label{defn:locally-small-type}
  Consider a universe $\UU$. A type $A$ is said to be \define{locally $\UU$-small}\index{locally small type|textbf} if the identity type $x=y$ is $\UU$-small for every $x,y:A$.
    We write\index{is-locally-small(A)@{$\islocallysmall_\UU(A)$}}
    \begin{equation*}
      \islocallysmall_\UU(A)\defeq \prd{x,y:A}\issmall_\UU(x=y).
    \end{equation*}
    Similarly, a map $f:A\to B$ is said to be \define{locally $\UU$-small}\index{locally small map|textbf} if all of its fibers are locally $\UU$-small.
\end{defn}

\begin{eg}
  ~
  \begin{enumerate}
  \item Any $\UU$-small type is also locally $\UU$-small.\index{small type!small types are locally small}\index{locally small type!small types are locally small}
  \item Any proposition is locally small with respect to any universe $\UU$.\index{proposition!propositions are locally U-small@{propositions are locally $\UU$-small}}\index{locally small type!propositions are locally small}
  \item Any univalent universe $\UU$ is locally $\UU$-small, because by the univalence axiom we have an equivalence\index{locally small type!U is locally U-small@{$\UU$ is locally $\UU$-small}}\index{universe!U is locally U-small@{$\UU$ is locally $\UU$-small}}
    \begin{equation*}
      (A=B)\simeq (A\simeq B)
    \end{equation*}
    for each $A,B:\UU$, and the type $A\simeq B$ is in $\UU$.
  \item For any family $B$ of locally $\UU$-small types over a $\UU$-small type $A$, the dependent product $\prd{x:A}B(x)$ is locally $\UU$-small.\index{dependent function type!is locally U-small@{is locally $\UU$-small}}\index{locally small type!dependent function type is locally small}
  \end{enumerate}
\end{eg}

We are now ready to assume the replacement axiom.

\begin{axiom}[The replacement axiom]\label{axiom:replacement}
  For any universe $\UU$, we assume that for any map $f:A\to B$ from a $\UU$-small type $A$ into a locally $\UU$-small type $B$, the image of $f$ is $\UU$-small.\index{replacement axiom|textbf}\index{axiom!replacement axiom|textbf}
\end{axiom}

\begin{eg}
  For any type $A:\UU$, the type $\UU_A$ of all types in $\UU$ merely equivalent to $A$ is equivalent to the image of the constant map $\const_A:\unit\to \UU$ is small. Since $\unit$ is small and $\UU$ is locally $\UU$-small, it follows from the replacement axiom that $\UU_A$ is $\UU$-small.\index{U A@{$\UU_A$}!is U-small@{is $\UU$-small}}
\end{eg}

\begin{eg}
  The type $\F$ of all finite types in $\UU$ is equivalent to be the image of the map
  \begin{equation*}
    \Fin{} : \N\to\UU.
  \end{equation*}
  Since $\N$ is $\UU$-small and $\UU$ is locally $\UU$-small, it follows from the replacement axiom that $\F$ is $\UU$-small.\index{F@{$\F$}!is U-small@{is $\UU$-small}}\index{small type!F is U-small@{$\F$ is $\UU$-small}}
\end{eg}

\begin{eg}
  Consider a type $A$ in $\UU$ and an equivalence relation $R$ on $A$ in $\UU$. Then the type $A/R$ is $\UU$-small, since it is equivalent to the image of
  \begin{equation*}
    R:A\to (A\to\prop_\UU),
  \end{equation*}
  which maps the $\UU$-small type $A$ into the locally $\UU$-small type $A\to\prop_\UU$.\index{set quotient!is U-small@{is $\UU$-small}}\index{small type!set quotient is U-small@{set quotient is $\UU$-small}}
\end{eg}

\subsection{The universal property of set quotients}
\index{universal property!of set quotients|(}
\index{set quotient!universal property|(}

The quotient $A/R$ is constructed as the image of $R$, so we obtain a commuting triangle
\begin{equation*}
  \begin{tikzcd}[column sep=-1em]
    A \arrow[rr,"q_R"] \arrow[dr,swap,"R"] & & A/R \arrow[dl,hook,"i_R"] \\
    \phantom{A/R} & \prop_\UU^A,
  \end{tikzcd}
\end{equation*}
and the embedding $i_R:A/R\to\prop_\UU^A$ satisfies the universal property of the image of $R$. This universal property is, however, not the usual universal property of the quotient.

\begin{defn}
  Consider a map $q:A\to B$ into a set $B$ satisfying the property that
  \begin{equation*}
    R(x,y)\to (q(x)=q(y))
  \end{equation*}
  for all $x,y:A$. We say that $q:A\to B$ \define{is a set quotient}\index{is set quotient}\index{set quotient|textbf} of $R$, or that $q$ satisfies the \define{universal property of the set quotient by $R$}\index{universal property!of set quotients|textbf}\index{set quotient!universal property|textbf}, if for every map $f:A\to X$ into a set $X$ such that $f(x)=f(y)$ whenever $R(x,y)$ holds, there is a unique extension
  \begin{equation*}
    \begin{tikzcd}
      A \arrow[d,swap,"q"] \arrow[dr,"f"] \\
      B \arrow[r,dashed] & X.
    \end{tikzcd}
  \end{equation*}
\end{defn}

\begin{rmk}
  Formally, we express the universal property of the quotient by $R$ as follows. Consider a map $q:A\to B$ that satisfies the property that
  \begin{equation*}
    H:\prd{x,y:A}R(x,y)\to (f(x)=f(y)).
  \end{equation*}
  Then there is for any set $X$ a map
  \begin{equation*}
    q^\ast:(B\to X) \to \Big(\sm{f:A\to X}\prd{x,y:A}R(x,y)\to (f(x)=f(y))\Big).
  \end{equation*}
  This map takes a function $h:B\to X$ to the pair
  \begin{equation*}
    q^\ast(h)\defeq(h\circ q,\lam{x}\lam{y}\lam{r}\ap{h}{H_{x,y}(r)}).
  \end{equation*}
  The universal property of the set quotient of $R$ asserts that the map $q^\ast$ is an equivalence for every set $X$. It is important to note that the universal property of set quotients is formulated with respect to sets.
\end{rmk}

\begin{thm}\label{thm:quotient_up}
  Consider a type $A$ and a universe $\UU$ containing $A$. Furthermore, let $R:A\to (A\to \prop_\UU)$ be an equivalence relation\index{equivalence relation}, and consider a map $q:A\to B$ into a set $B$, not necessarily in $\UU$. Then the following are equivalent.
  \begin{enumerate}
  \item \label{item:thm-quotient-up}The map $q$ satisfies the property that
    \begin{equation*}
      q(x)=q(y)
    \end{equation*}
    for every $x,y:A$ for which $R(x,y)$ holds, and moreover $q$ satisfies the universal property of the set quotient of $R$.
  \item \label{item:thm-quotient-effective}The map $q$ is surjective and \define{effective}\index{effective map|textbf}\index{equivalence relation!effective map|textbf}, which means that for each $x,y:A$ we have an equivalence
    \begin{equation*}
      (q(x)=q(y))\simeq R(x,y).
    \end{equation*}
  \item \label{item:thm-quotient-up-image}The map $R:A\to (A\to \prop_\UU)$ extends along $q$ to an embedding
    \begin{equation*}
      \begin{tikzcd}[column sep=tiny]
        A \arrow[rr,"q"] \arrow[dr,swap,"R"] & & B \arrow[dl,dashed,"i"] \\
        & \prop_\UU^A
      \end{tikzcd}
    \end{equation*}
    and the embedding $i$ satisfies the universal property of the image inclusion of $R$.
  \end{enumerate}
\end{thm}

In \cref{thm:quotient_up} we don't assume that $B$ is in the same universe as $A$ and $R$, because we want to apply it to $B\defeq\im(R)$. As we will see below, this extra generality only affects the proof that \ref{item:thm-quotient-effective} implies \ref{item:thm-quotient-up-image}.

\begin{proof}
  We first show that \ref{item:thm-quotient-effective} is equivalent to \ref{item:thm-quotient-up-image}, since this is the easiest part. After that, we will show that \ref{item:thm-quotient-up} is equivalent to \ref{item:thm-quotient-effective}.

  Assume that \ref{item:thm-quotient-up-image} holds. Then $q$ is surjective by \cref{thm:surjective}. Moreover, we have
  \begin{align*}
    R(x,y) & \simeq R(x)=R(y) \\
           & \simeq i(q(x))=i(q(y)) \\
           & \simeq q(x)=q(y)
  \end{align*}
  In this calculation, the first equivalence holds by \cref{cor:eq-quotient}; the second equivalence holds since we have a homotopy $R\htpy i\circ q$; and the third equivalence holds since $i$ is an embedding. This completes the proof that \ref{item:thm-quotient-up-image} implies \ref{item:thm-quotient-effective}.

  Next, we show that \ref{item:thm-quotient-effective} implies \ref{item:thm-quotient-up-image}. First, we want to define a map 
  \begin{equation*}
    i:B\to\prop_\UU^A.
  \end{equation*}
  We would like to define $i(b,a):=(b=q(a))$. This direct definition does not go through, however, because the type $B$ is not assumed to be in $\UU$. Nevertheless, observe that by the assumption that $q$ is surjective and effective, the type $B$ is locally $\UU$-small. To see this, first note that $\issmall_U(X)$ is a proposition for any type $X$ by \cref{prp:small}. Using the assumption that $q$ is surjective, it follows from \cref{prp:surjective} that it suffices to show that $q(a)=q(a')$ is $\UU$-small for each $a,a':A$. This follows by the assumption that $q$ is effective. In particular, the identity type $b=q(a)$ is a $\UU$-small proposition, for every $b:B$ and $a:A$. Let us write $s(b,a)$ for the element of type $\issmall_\UU(b=q(a))$.

  Now consider a universe $\VV$ containing $B$. Then we can define a map
  \begin{equation*}
    j:B\to\Big(A\to\sm{P:\prop_\VV}\issmall_\UU(P)\Big)
  \end{equation*}
  by $j(b,a):=(b=q(a),s(b,a))$, and now we obtain $i$ from $j$ by defining
  \begin{equation*}
    i(b,a):=\proj 1(s(b,a)).
  \end{equation*}
  Note that we have an equivalence $i(b,a)\simeq (b=q(a))$ for every $b:B$ and $a:A$.
  Then the triangle
  \begin{equation*}
    \begin{tikzcd}[column sep=tiny]
      A \arrow[rr,"q"] \arrow[dr,swap,"R"] & & B \arrow[dl,"i"] \\
      & \prop_\UU^A
    \end{tikzcd}
  \end{equation*}
  commutes, since we have an equivalence
  \begin{equation*}
    i(q(a),a') \simeq (q(a)=q(a')) \simeq R(a,a')
  \end{equation*}
  for each $a,a':A$. To show that $i$ is an embedding, recall from \cref{cor:is-emb-is-injective} that it suffices to show that $i$ is injective, i.e., that
  \begin{equation*}
    \prd{b,b':B}(i(b)=i(b'))\to (b=b'),
  \end{equation*}
  since the codomain of $i$ is a set by \cref{prp:propositional-extensionality}. Note that injectivity of a map into a set is a property, and that $q$ is assumed to be surjective. Hence by \cref{prp:surjective} it is sufficient to show that
  \begin{equation*}
    \prd{a,a':A}(i(q(a))=i(q(a')))\to (q(a)=q(a')).
  \end{equation*}
  Since $R\htpy i\circ q$, and $q(a)=q(a')$ is assumed to be equivalent to $R(a,a')$, it suffices to show that
  \begin{equation*}
    \prd{a,a':A}(R(a)=R(a'))\to R(a,a'),
  \end{equation*}
  which follows directly from \cref{cor:eq-quotient}. Thus we have shown that the factorization $R\htpy i\circ q$ factors $R$ as a surjective map followed by an embedding. We conclude by \cref{thm:surjective} that the embedding $i$ satisfies the universal property of the image factorization of $R$, which finishes the proof that \ref{item:thm-quotient-effective} implies \ref{item:thm-quotient-up-image}.
  
  Now we show that \ref{item:thm-quotient-up} implies \ref{item:thm-quotient-effective}. To see that $q$ is surjective if it satisfies the assumptions in \ref{item:thm-quotient-up}, consider the image factorization
  \begin{equation*}
    \begin{tikzcd}[column sep=tiny]
      A \arrow[dr,swap,"q"] \arrow[rr,"q_q"] & & \im(q) \arrow[dl,"i_q"] \\
      \phantom{\im(q)} & B.
    \end{tikzcd}
  \end{equation*}
  We claim that the map $i_q$ has a section. To see this, we first note that we have
  \begin{equation*}
    q_q(x)=q_q(y)
  \end{equation*}
  for any $x,y:A$ satisfying $R(x,y)$, because if $R(x,y)$ holds, then $q(x)=q(y)$ and hence $i_q(q_q(x))=i_q(q_q(y))$ holds and $i_q$ is an embedding. Since $\im(q)$ is a set, we may apply the universal property of $q$ and we obtain a unique extension of $q_q$ along $q$
  \begin{equation*}
    \begin{tikzcd}
      A \arrow[d,swap,"q"] \arrow[dr,"q_q"] \\
      B \arrow[r,dashed,swap,"h"] & \im(q).
    \end{tikzcd}
  \end{equation*}
  Now we observe that the composite $i_q\circ h$ is an extension of $q$ along $q$, so it must be the identity function by uniqueness. Thus we have established that $h$ is a section of $i_q$. Since $i_q$ is an embedding with a section, it follows that $i_q$ is an equivalence. We conclude that $q$ is surjective, because $q$ is the composite $i_q\circ q_q$ of a surjective map followed by an equivalence.

  Now we have to show that the map $q$ is effective, i.e., that $q(x)=q(y)$ is equivalent to $R(x,y)$ for every $x,y:A$. We first apply the universal property of $q$ to obtain for each $x:A$ an extension of $R(x)$ along $q$
  \begin{equation*}
    \begin{tikzcd}
      A \arrow[d,swap,"q"] \arrow[dr,"R(x)"] \\
      B \arrow[r,dashed,swap,"\tilde{R}(x)"] & \prop_\UU.
    \end{tikzcd}
  \end{equation*}
  Since the triangle commutes, we have an equivalence $\tilde{R}(x,q(x'))\simeq R(x,x')$ for each $x':A$. Now we apply \cref{thm:id_fundamental} to see that the canonical family of maps
  \begin{equation*}
    \prd{y:B}(q(x)=y)\to \tilde{R}(x,y)
  \end{equation*}
  is a family of equivalences. Thus, we need to show that the type $\sm{y:B}\tilde{R}(x,y)$ is contractible. For the center of contraction, note that we have $q(x):B$, and the type $\tilde{R}(x,q(x))$ is equivalent to the type $R(x,x)$, which is inhabited by reflexivity of $R$. To construct the contraction, it suffices to show that
  \begin{equation*}
    \prd{y:B}\tilde{R}(x,y)\to (q(x)=y).
  \end{equation*}
  Since this is a property, and since we have already shown that $q$ is a surjective map, we may apply \cref{prp:surjective}, by which it suffices to show that
  \begin{equation*}
    \prd{x':A}\tilde{R}(x,q(x'))\to (q(x)=q(x')).
  \end{equation*}
  Since $\tilde{R}(x,q(x'))\simeq R(x,x')$, this is immediate from our assumption on $q$. Thus we obtain the contraction, and we conclude that we have an equivalence $\tilde{R}(x,y)\simeq (q(x)=y)$ for each $y:B$. It follows that we have an equivalence
  \begin{equation*}
    R(x,y)\simeq (q(x)=q(y))
  \end{equation*}
  for each $x,y:A$, which completes the proof that \ref{item:thm-quotient-up} implies \ref{item:thm-quotient-effective}.
  
  It remains to show that \ref{item:thm-quotient-effective} implies \ref{item:thm-quotient-up}. Assume \ref{item:thm-quotient-effective}, and let $f:A\to X$ be a map into a set $X$, satisfying the property that
  \begin{equation*}
    \prd{a,a':A}R(a,a')\to (f(a)=f(a')).
  \end{equation*}
  Our goal is to show that the type of extensions of $f$ along $q$ is contractible. By \cref{ex:surjective-precomp} it follows that there is at most one such an extension, so it suffices to construct one.

  In order to construct an extension, we will construct for every $b:B$ a term $x:X$ satisfying the property
  \begin{equation*}
    P(x)\defeq \exists_{(a:A)}(f(a)=x)\land (q(a)=b).
  \end{equation*}
  Before we make this construction, we first observe that there is at most one such $x$, i.e., that the type of $x:X$ satisfying $P(x)$ is in fact a proposition. To see this, we need to show that $x=x'$ for any $x,x':X$ satisfying $P(x)$ and $P(x')$. Since $X$ is assumed to be a set, our goal of showing that $x=x'$ is a property. Therefore we may assume that we have $a,a':A$ satisfying
  \begin{align*}
    f(a) & = x & q(a) & = b \\
    f(a') & = x' & q(a') & = b.
  \end{align*}
  It follows from these assumptions that $q(a)=q(a')$, and hence that $R(a,a')$ holds. This in turn implies that $f(a)=f(a')$, and hence that $x=x'$.

  Now let $b:B$. Our goal is to construct an $x:X$ that satisfies the property
  \begin{equation*}
    \exists_{(a:A)}(f(a)=x)\land (q(a)=b).
  \end{equation*}
  Since $q$ is assumed to be surjective, we have $\brck{\fib{q}{b}}$. Moreover, since we have shown that at most one $x:X$ exists with the asserted property, we get to assume that we have $a:A$ satisfying $q(a)=b$. Now we see that $x\defeq f(a)$ satisfies the desired property.

  Thus, we obtain a function $h:B\to X$ satisfying the property that for all $b:B$ there exists an $a:A$ such that
  \begin{equation*}
    f(a)=h(b)\qquad\text{and}\qquad q(a)=b.
  \end{equation*}
  In particular, it follows that $h(q(a))=f(a)$ for all $a:A$, which completes the proof that \ref{item:thm-quotient-effective} implies \ref{item:thm-quotient-up}.  
\end{proof}

\begin{cor}
  Consider an equivalence relation $R$ over a type $A$. Then the quotient map
  \begin{equation*}
    q:A\to A/R
  \end{equation*}
  is surjective and effective, and it satisfies the universal property of the set quotient.
\end{cor}

\cref{thm:quotient_up} can be used to show that the type of equivalence relations is equivalent to the type of sets $X$ equipped with a surjective map $A\twoheadrightarrow X$. This may seem remarkable if you haven't tried \cref{ex:surjection-into-k-type} yet, because at first glance one might think that the type of sets $X$ equipped with a surjective map $A\twoheadrightarrow X$ is a $1$-type, while the type of equivalence relations on $A$ is a set.

\begin{thm}\label{thm:eqrel-surj}
  For any type $A$ and any universe $\UU$ containing $A$, we have an equivalence
  \index{surjective map!surjective maps into sets are set quotients}\index{set quotient!surjective maps into sets are set quotients}
  \begin{equation*}
    \eqrel_\UU(A)\simeq\sm{X:\Set_\UU}A\twoheadrightarrow X.
  \end{equation*}
\end{thm}

\begin{proof}
  Given an equivalence relation $R:A\to(A\to\prop_\UU)$ on $A$ we first use the replacement axiom, by which the set quotient $A/R$ is $\UU$-small, to obtain a set $Q(R):\Set_\UU$, an equivalence $e:Q(R)\simeq A/R$, and a surjective map $f:A\to Q(R)$ such that the triangle
  \begin{equation*}
    \begin{tikzcd}[column sep=1em]
      \phantom{Q(R)} & A \arrow[dl,swap,"f"] \arrow[dr,"q"] & \phantom{Q(R)}\\
      Q(R) \arrow[rr,swap,"e"] & & A/R
    \end{tikzcd}
  \end{equation*}
  commutes. This defines a map
  \begin{equation*}
    \mathcal{Q}_A:\eqrel_\UU(A)\to\sm{X:\Set_\UU}A\twoheadrightarrow X.
  \end{equation*}
  The map $\mathcal{K}_A:\big(\sm{X:\Set_\UU}A\twoheadrightarrow X\big)\to\eqrel_\UU(A)$ is given by
  \begin{equation*}
    \mathcal{K}_A(X,f,x,y)\defeq K_f(x,y) \defeq (f(x)=f(y)).
  \end{equation*}
  Note that $K_f$ is valued in propositions because $X$ is assumed to be a set, and obviously it is an equivalence relation.

  To see that $\mathcal{K}_A(\mathcal{Q}_A(R))=R$ note that by function extensionality and propositional extensionality it follows that two equivalence relations $R$ and $S$ on $A$ are equal if and only if $R(x,y)\leftrightarrow S(x,y)$ for all $x,y:A$. Note that $\mathcal{K}_A(\mathcal{Q}_A(R))(x,y)\leftrightarrow R(x,y)$ holds for all $x,y:A$ if and only if $(q_R(x)=q_R(y))\leftrightarrow R(x,y)$ holds for all $x,y:A$. This follows from \cref{cor:eq-quotient}.

  It remains to show that $\mathcal{Q}_A(\mathcal{K}_A(X,f))=(X,f)$. Note that the type of identifications $(Y,g)=(X,f)$ is by the univalence axiom equivalent to the type
  \begin{equation*}
    \sm{e:Y\simeq X}e\circ g\htpy f.
  \end{equation*}
  Therefore it suffices to construct a commuting triangle
  \begin{equation*}
    \begin{tikzcd}[column sep=1em]
      \phantom{A/K_f} & A \arrow[dl,swap,"q_{K_f}"] \arrow[dr,"f"] & \phantom{A/K_f} \\
      A/K_f \arrow[rr] & & X
    \end{tikzcd}
  \end{equation*}
  We obtain such an equivalence by combining \cref{thm:quotient_up} and \cref{thm:uniqueness-image}.
\end{proof}
\index{universal property!of set quotients|)}
\index{set quotient!universal property|)}


\subsection{Partitions}
\index{partition|(}
\index{set quotient!partition|(}

There are many equivalent ways of stating what an equivalence relation is. We saw in \cref{thm:eqrel-surj} that the type of equivalence relations on $A$ is equivalent to the type of surjective maps out of $A$ into a set. Here we will show that the type of equivalence relations on $A$ is equivalent to the type of partitions of $A$. Another type that is equivalent to the type of equivalence relations of $A$ is the type of set-indexed $\Sigma$-decompositions of $A$, i.e., the type of triples $(X,Y,e)$ consisting of a set $X$, a family $Y$ of inhabited types indexed by $X$, and an equivalence $e:A\simeq \sm{x:X}Y(x)$. The fact that the type of equivalence relations on $A$ is equivalent to the type of set-indexed $\Sigma$-decompositions of $A$ is stated as \cref{ex:sigmadecompositions}

In this section we show that equivalence relations on $A$ are partitions of $A$. Recall that the type of inhabited subtypes of $A$ is defined to be
\begin{equation*}
  \mathcal{P}_{\mathcal{U}}^+(A)\defeq\sm{Q:A\to\prop_\UU}\Brck{\sm{a:A}Q(a)}.
\end{equation*}
The equivalence of equivalence relations and partitions requires some finesse regarding universes. This is why we set up the definition of partitions in the following way.

\begin{defn}
  Let $A$ be a type and let $\UU$ and $\VV$ be two universes. A \define{$(\UU,\VV)$-partition}\index{partition|textbf}\index{set quotient!partition|textbf} of a type $A$ is a subset
  \begin{equation*}
    P:\mathcal{P}_{\UU}^+(A)\to\prop_{\VV}
  \end{equation*}
  of the type of inhabited subsets of $A$ such that for each $x:A$ there is a unique inhabited subset $Q$ of $A$ in $P$ that contains $x$, i.e., if it comes equipped with an element of type
  \begin{equation*}
    \ispartition(P):=\prd{x:A}\iscontr\Big(\sm{Q:\mathcal{P}_{\UU}^+(A)}P(Q)\times Q(x)\Big)
  \end{equation*}
  The type of all $(\UU,\VV)$-partitions of $A$ is defined by
  \begin{equation*}
    \partition_{\UU,\VV}(A)\defeq\sm{P:\mathcal{P}_{\UU}^+(A)\to\prop_{\VV}}\ispartition(P)
  \end{equation*}
\end{defn}

\begin{thm}
  Consider a type $A$, a universe $\UU$, and consider a universe $\VV$ containing both $A$ and every type in $\UU$. Then we have an equivalence
  \begin{equation*}
    \eqrel_{\UU}(A)\simeq\partition_{\UU,\VV}(A).
  \end{equation*}
\end{thm}

\begin{proof}
  Consider an equivalence relation $R$ on $A$. Then we define
  \begin{equation*}
    P:\mathcal{P}_{\UU}^+(A)\to\prop_\VV
  \end{equation*}
  by $P(Q)\defeq\exists_{(x:A)}\forall_{(y:A)}Q(y)\leftrightarrow R(x,y)$. In other words, $P$ is the subtype of equivalence classes of $R$, which are all inhabited. To show that $P$ is a partition of $A$, let $x:A$. The type
  \begin{equation*}
    \sm{Q:\mathcal{P}_{\UU}^+(A)}P(Q)\times Q(x)
  \end{equation*}
  is equivalent to the type
  \begin{equation*}
    \sm{Q:\mathcal{P}_{\UU}^+(A)}\prd{y:A}Q(y)\leftrightarrow R(x,y)
  \end{equation*}
  since the proposition $\exists_{(z:A)}\forall_{(y:A)}Q(y)\leftrightarrow R(z,y)$ is equivalent to the type $\prd{y:A}Q(y)\leftrightarrow R(x,y)$, given an element $q:Q(x)$. By univalence it follows that the latter type is equivalent to the identity type $Q=R(x)$ in $\mathcal{P}_{\UU}^+(A)$, so the total space is contractible. Thus we obtain a map
  \begin{equation*}
    \psi:\eqrel_{\mathcal{U}}(A)\to\partition_{\UU,\VV}(A).
  \end{equation*}
  
  For the converse map, we first define for any $(\UU,\VV)$-partition $P$ of $A$ a binary relation $R_P$ such that $R_P(x)$ is at the center of contraction in the type
  \begin{equation*}
    \sm{Q:\mathcal{P}^+_{\mathcal{U}}(A)}P(Q)\times Q(x).
  \end{equation*}
  In other words, $R_P(x)$ is defined to be the unique block in the partition $P$ such that $R_P(x,x)$ holds. It is immediate from its definition that $R_P(x,y)$ is a proposition in $\UU$. To see that $R_P$ is symmetric, note that if $R_P(x,y)$ holds, then $R_P(x)$ is an element of type
  \begin{equation*}
    \sm{Q:\mathcal{P}^+_{\mathcal{U}}(A)}P(Q)\times Q(y).
  \end{equation*}
  By contractibility, this implies that $R(x)=R(y)$, from which we obtain that $R(y,x)$ holds. To see that $R_P$ is transitive we observe similarly that if $R(x,y)$ and $R(y,z)$ hold, then we have an identification $R(x)=R(y)$ and it follows that $R(x,z)$ holds.  Thus we obtain a map
  \begin{equation*}
    \varphi:\partition_{\UU,\VV}(A)\to\eqrel_{\UU}(A).
  \end{equation*}
  It remains to prove that the maps $\psi$ and $\varphi$ are inverse to each other, first let $R$ be an equivalence relation. In order to show that $\varphi(\psi(R))=R$ it suffices by univalence to show that the equivalence relation obtained from the partition induced by $R$ is given by
  \begin{equation*}
    R'(x,y):=\sm{Q:\mathcal{P}_{\UU}^+(A)}\Big(\exists_{(u:A)}\forall_{(v:A)}Q(v)\leftrightarrow R(u,v)\Big)\times Q(x)\times Q(y).
  \end{equation*}
  is equivalent to $R$. Observe that the proposition $R'(x,y)$ is equivalent to $R(x,x)\times R(x,y)$, which is equivalent to $R(x,y)$. This shows that the composite
  \begin{equation*}
    \begin{tikzcd}
      \eqrel_{\UU}(A) \arrow[r,"\psi"] & \partition_{\UU,\VV}(A) \arrow[r,"\varphi"] & \eqrel_{\UU}(A)
    \end{tikzcd}
  \end{equation*}
  is homotopic to the identity function.

  Finally, we have to show that for any partition $P$ of $A$ and any inhabited subtype $Q$ of $A$ we have $\psi(\varphi(P))(Q)\leftrightarrow P(Q)$. Note that this is a proposition, so we may assume an element $x:A$ such that $Q(x)$ holds. By univalence it follows that $\psi(\varphi(P))(Q)$ holds if and only if $Q=R_P(x)$, where $R_P$ is the equivalence relation constructed in the definition of the map $\varphi$. Now we see that $P(Q)$ holds if and only if $Q$ is in the contractible type
  \begin{equation*}
    \sm{Q':\mathcal{P}^+_{\mathcal{U}}(A)}P(Q')\times Q'(x),
  \end{equation*}
  which is the case if and only if $Q=R_P(x)$. This shows that the composite
  \begin{equation*}
    \begin{tikzcd}
      \partition_{\UU,\VV}(A) \arrow[r,"\varphi"] & \eqrel_{\UU}(A) \arrow[r,"\psi"] & \partition_{\UU,\VV}(A)
    \end{tikzcd}
  \end{equation*}
  is homotopic to the identity function.
\end{proof}
\index{partition|)}
\index{set quotient!partition|)}

\subsection{Unique representatives of equivalence classes}
\index{equivalence class!choice of unique representatives|(}
\index{equivalence relation!choice of unique representatives|(}

A common way to construct set quotients is by showing that the equivalence classes of an equivalence relation have a choice of unique representatives. In this section we show that if there is a choice of unique representatives, then the set quotient can be constructed as the type of those representatives. An important reason to define set quotients as the type of canonical representatives, if that is possible, is that the universe level of the set quotient can be kept as low as possible without needing to appeal to the replacement axiom.

\begin{defn}
  Consider an equivalence relation $R$ on a type $A$, and consider a family of types $C(x)$ indexed by $x:A$. We say that $C$ is a \define{choice of (unique) representatives}\index{choice of unique representatives|textbf}\index{equivalence class!choice of unique representatives|textbf}\index{equivalence relation!choice of unique representatives|textbf} of the equivalence classes of $R$ if $C$ comes equipped with an element of type
  \begin{equation*}
    \ischoiceofrepresentatives(C) \defeq \prd{x:A}\iscontr\Big(\sm{y:A}C(y)\times R(x,y)\Big).
  \end{equation*}
\end{defn}

\begin{thm}\label{thm:choice-of-representatives}
  Consider an equivalence relation $R$ on a type $A$, and let $C$ be a choice of representatives of the equivalence classes of $R$, with $(h(x),c(x),r(x))$ at the center of contraction of $\sm{y:A}C(y)\times R(x,y)$. Then the map
  \begin{equation*}
    q:A\to\sm{x:A}C(x)
  \end{equation*}
  given by $q(x)\defeq(h(x),c(x))$ is a map into a set such that $q(x)=q(y)$ for every $x,y:A$ such that $R(x,y)$ holds, and moreover $q$ satisfies the universal property of the set quotient of $A$ by $R$. 
\end{thm}

\begin{proof}
  First, we will use \cref{lem:prop_to_id} to show that the type $\sm{y:A}C(y)$ is a set, such that
  \begin{equation*}
    ((x,c)=(y,d))\simeq R(x,y)
  \end{equation*}
  for any $(x,c)$ and $(y,d)$ in $\sm{y:A}C(y)$. Note that we have a function
  \begin{equation*}
    R(x,y)\to ((x,c)=(y,d)),
  \end{equation*}
  since for any $r:R(x,y)$ both $(x,c,r)$ and $(y,d,r)$ are elements of the contractible type ${\sm{y:A}C(y)\times R(x,y)}$. Since $R$ is a reflexive relation valued in propositions, the claim follows. In particular, it follows that
  \begin{equation*}
    (q(x)=q(y))\simeq R(x,y)
  \end{equation*}
  for any $x,y:A$, i.e., $q$ is effective.
  
  To prove the universal property of set quotients, note that by characterization \ref{item:thm-quotient-effective} in \cref{thm:quotient_up} it suffices to show that $q$ is surjective and effective. We have already shown above that $q$ is effective, so it remains to show that $q$ is surjective. In fact, we will prove the stronger claim that the projection map
  \begin{equation*}
    \proj 1:\sm{x:A}C(x)\to A  
  \end{equation*}
  is a section of $q$. Let $x:A$ and $c:C(x)$. Then $(x,c,\rho(x))$ is an element of the type
  \begin{equation*}
    \sm{y:A}C(y)\times R(x,y),
  \end{equation*}
  which is contractible with center of contraction $(h(x),c(x),r(x))$. Therefore it follows that $q(x)\jdeq (h(x),c(x))=(x,c)$. In particular, we see that $q(\proj 1(x,c))=(x,c)$, i.e., that $\proj 1$ is a section of $q$.
\end{proof}

\begin{eg}
  In \cref{prp:congruence-eqrel} we constructed the congruence relations $x\equiv y \mod k$ on the natural numbers for every natural number $k$, and in \cref{thm:effective-mod-k,thm:issec-nat-Fin} we showed that the map
  \begin{equation*}
    x\mapsto [x]_{k+1}:\N\to\Fin{k+1}
  \end{equation*}
  is effective and split surjective. By \cref{thm:quotient_up} it follows that the map
  \begin{equation*}
    x\mapsto [x]_{k+1}:\N\to\Fin{k+1}
  \end{equation*}
  satisfies the universal property of the set quotient of the equivalence relation $x,y\mapsto x\equiv y\mod k+1$.

  We also claim that there is a choice of representatives of the congruence relations.\index{congruence relations on N@{congruence relations on $\N$}!choice of unique representatives}\index{choice of unique representatives!for the congruence relations on N@{for the congruence relations on $\N$}} We define our choice of representatives by
  \begin{equation*}
    C(y)\defeq \fib{\natFin}{y},
  \end{equation*}
  where $\natFin:\Fin{k+1}\to\N$ is the inclusion of $\Fin{k+1}$ into $\N$ constructed in \cref{defn:natFin}. To see that $C$ is a choice of representatives, we have to prove that
  \begin{equation*}
    \sm{y:\N}C(y)\times (x\equiv y\mod k+1)
  \end{equation*}
  is contractible for each $x:\N$. At the center of contraction we have the triple $(\natFin([x]_{k+1}),([x]_{k+1},\refl{}),p)$ where $p:x\equiv\natFin([x]_{k+1})\mod k+1$ is the proof obtained via \cref{thm:effective-mod-k,thm:issec-nat-Fin}. In order to construct the contraction, note that both $C(y)$ and $x\equiv y\mod k+1$ are propositions for each $y:\N$. Therefore it suffices to prove that for any $y:\N$ such that $C(y)$ and $x\equiv y\mod k+1$ hold, we have
  \begin{equation*}
    \natFin([x]_{k+1})=y.
  \end{equation*}
  Since $C(y)$ holds, we see that $y=\natFin([y]_{k+1})$. Therefore it suffices to prove that $[x]_{k+1}=[y]_{k+1}$. This follows from \cref{thm:effective-mod-k}, since we assumed $x\equiv y\mod k+1$.
\end{eg}

\begin{eg}
  Consider the type of \define{(integer) fractions}\index{fraction|textbf}\index{integers!integer fractions|textbf}
  \begin{equation*}
    Q\defeq \Z\times\sm{y:\Z}y\neq 0.
  \end{equation*}
  We define an equivalence relation on $Q$ by
  \begin{equation*}
    ((x,y)\sim (x',y'))\defeq (xy'=x'y).
  \end{equation*}
  This equivalence relation has a choice of representatives defined by
  \index{choice of unique representatives!for integer fractions}
  \index{fraction!choice of unique representatives}
  \begin{equation*}
    C(x,y)\defeq (y>0)\land (\gcd(x,y)=1). 
  \end{equation*}
  In other words, we say that $(x,y)$ is a \define{reduced fraction} if $y>0$ and $x$ and $y$ are coprime. 

  To see that $C$ defines a choice of unique representatives, we first need to construct the center of contraction of
  \begin{equation*}
    \sm{q:Q}C(q)\times ((x,y)\sim q).
  \end{equation*}
  Note that if $y<0$ then $(x,y)\sim (-x,-y)$, and we have $-y>0$. It is therefore safe to assume that $y>0$. We claim that
  \begin{equation*}
    (x/\gcd(x,y),y/\gcd(x,y)):Q
  \end{equation*}
  satisfies $C$ and is equivalent to $(x,y)$. It is immediate that $y/\gcd(x,y)>0$ and that $(x,y)\sim(x/\gcd(x,y),y/\gcd(x,y))$. The fact that $x/\gcd(x,y)$ and $y/\gcd(x,y)$ are coprime follows from the fact that
  \begin{equation*}
    \gcd(x/d,y/d)=\gcd(x,y)/d
  \end{equation*}
  for any common divisor $d$ of $x$ and $y$. 
  
  To construct a contraction, let $(x',y'):Q$ such that $C(x',y')$ and $(x,y)\sim (x',y')$. Since $C(q)$ and $(x,y)\sim q$ are propositions for every $q:Q$ it suffices to show that
  \begin{equation*}
    x'=x/\gcd(x,y)\qquad\text{and}\qquad y'=y/\gcd(x,y).
  \end{equation*}
  Since $x'$ and $y'$ are assumed to be coprime, it follows from the equation
  \begin{equation*}
    x'y/\gcd(x,y)=xy'/\gcd(x,y)
  \end{equation*}
  that $x'$ divides $x/\gcd(x,y)$. Similarly $x/\gcd(x,y)$ and $y/\gcd(x,y)$ are coprime, it follows from the same equation that $x/\gcd(x,y)$ divides $x'$, so we conclude that $ux'=x/\gcd(x,y)$ for some $u=\pm 1$. The fact that $vy'=y/\gcd(x,y)$ for some $v=\pm 1$ is proven similarly. However, since both $y$ and $y'$ are positive, and the $\gcd(x,y)$ of any two integers is positive, it follows that $v=1$. Using the assumption that $x'y/\gcd(x,y)=xy'/\gcd(x,y)$, this allows us to deduce that also $u=1$.

  We define the type of \define{rational numbers}\index{Q@{$\Q$}|see {rational numbers}}\index{rational numbers|textbf} by
  \begin{equation*}
    \Q\defeq \sm{(x,y):Q}(y>0)\land \gcd(x,y)=1,
  \end{equation*}
  and we define the quotient map $(x,y)\mapsto x/y:Q\to \Q$ to be the quotient map $q$ in \cref{thm:choice-of-representatives}. By \cref{thm:choice-of-representatives} it also follows that $(x,y)\mapsto x/y$ satisfies the universal property of the set quotient of the equivalence relation $\sim$ on $Q$.
\end{eg}
\index{equivalence class!choice of unique representatives|)}
\index{equivalence relation!choice of unique representatives|)}

\subsection{Set truncations}
\index{set truncation|(}
\index{set quotient!set truncation|(}

An important instance of set quotients in the univalent foundations of mathematics is the notion of set truncation. Analogous to the propositional truncation, the set truncation of a type $A$ is a map $\eta:A\to \trunc{0}{A}$ into a set $\trunc{0}{A}$ such that any map $f:A\to X$ into a set $X$ extends uniquely along $\eta$:
\begin{equation*}
  \begin{tikzcd}
    A \arrow[dr,"f"] \arrow[d,swap,"\eta"] \\
    \trunc{0}{A} \arrow[r,dashed] & X.
  \end{tikzcd}
\end{equation*}
In other words, the set truncation $\eta:A\to\trunc{0}{A}$ is the universal way of mapping $A$ into a set. We first specify what it means for a map $f:A\to B$ into a set $B$ to be a set truncation of $A$.

\begin{defn}
  We say that a map $f:A\to B$ into a set $B$ is a \define{set truncation}\index{set truncation|textbf}\index{set quotient!set truncation|textbf}\index{universal property!of set truncations|textbf}\index{set truncation!universal property|textbf} if the precomposition function
  \begin{equation*}
    \blank\circ f : (B\to X)\to (A\to X)
  \end{equation*}
  is an equivalence for every set $X$. 
\end{defn}

In the following theorem we prove several conditions that are equivalent to being a set truncation.

\begin{thm}\label{thm:set-truncation}
  Consider a map $f:A\to B$ into a set $B$. Then the following are equivalent:
  \begin{enumerate}
  \item\label{item:is-set-truncation} The map $f$ is a set truncation.
  \item\label{item:dup-set-truncation} The map $f$ satisfies the \textbf{dependent universal property}\index{set truncation!dependent universal property|textbf}\index{dependent universal property!of set truncations|textbf} of the set truncation: For every family $X$ of sets over $B$, the precomposition function
    \begin{equation*}
      \blank\circ f : \Big(\prd{b:B}X(b)\Big)\to\Big(\prd{a:A}X(f(a))\Big)
    \end{equation*}
    is an equivalence.
  \item\label{item:is-quotient-set-truncation} The map $f$ is surjective and effective with respect to the equivalence relation $x,y\mapsto\brck{x=y}$, i.e., we have equivalences
    \begin{equation*}
      (f(x)=f(y))\simeq \brck{x=y}
    \end{equation*}
    for every $x,y:A$.
  \end{enumerate}
\end{thm}

\begin{proof}
  The fact that \ref{item:dup-set-truncation} implies \ref{item:is-set-truncation} is immediate. Moreover, the fact that \ref{item:is-set-truncation} is equivalent to \ref{item:is-quotient-set-truncation} follows from the fact that any map $h:A\to X$ into a set $X$ comes equipped with a function
  \begin{equation*}
    \brck{x=y}\to (h(x)=h(y))
  \end{equation*}
  for every $x,y:A$. 
  
  It remains to prove that \ref{item:is-set-truncation} implies \ref{item:dup-set-truncation}. Consider a family $X$ of sets over $B$, and consider the commuting square
  \begin{equation*}
    \begin{tikzcd}[column sep=7em]
      \sm{g:B\to B}\prd{b:B}X(g(b)) \arrow[d,swap,"\simeq"] \arrow[r,"{(g,s)\mapsto (g\circ f,s\circ f)}"] & \sm{h:A\to B}\prd{a:A}X(h(a)) \arrow[d,"\simeq"] \\
      (B\to\sm{b:B}X(b)) \arrow[r,swap,"\blank\circ f"] & (A\to\sm{b:B}X(b))
    \end{tikzcd}
  \end{equation*}
  The side maps are equivalences by the distributivity of $\Pi$ over $\Sigma$, and the bottom map is an equivalence by the assumption that $f$ is a set truncation. Therefore it follows that the top map is an equivalence. Furthermore, note that the map
  \begin{equation*}
    \blank\circ f : (B\to B)\to (A\to B)
  \end{equation*}
  is an equivalence by the assumption that $f$ is a set truncation. Therefore it follows from \cref{thm:equiv-toto} that the map
  \begin{equation*}
    \blank\circ f : \Big(\prd{b:B}X(g(b))\Big)\to \Big(\prd{a:A}X(g(f(a)))\Big)
  \end{equation*}
  is an equivalence for every $g:B\to B$. Now we take $g\defeq \idfunc$ to complete the proof that \ref{item:is-set-truncation} implies \ref{item:dup-set-truncation}.
\end{proof}

\begin{cor}
  On any universe $\UU$, there is an operation $\trunc{0}{\blank}:\UU\to\Set_\UU$\index{[[A]] 0@{$\trunc{0}{A}$}|see {set truncation}} such that every type $A$ in $\UU$ comes equipped with a map
  \begin{equation*}
    \eta:A\to\trunc{0}{A}
  \end{equation*}
  that satisfies the universal property of the set truncation. The set $\trunc{0}{A}$ is called the \define{set truncation}\index{set truncation|textbf} of $A$.
\end{cor}

\begin{proof}
  By \cref{thm:set-truncation} it follows that a map $f:A\to B$ into a set $B$ is a set truncation if and only if it is a quotient map with respect to the equivalence relation $x,y\mapsto\brck{x=y}$. Given a type $A$ in $\UU$, the quotient of $A$ by $x,y\mapsto\brck{x=y}$ is equivalent to a type in $\UU$ by the replacement axiom.
\end{proof}

\begin{cor}
  The set truncation $\eta:A\to\trunc{0}{A}$ is surjective and effective with respect to the equivalence relation $x,y\mapsto\brck{x=y}$, i.e., we have an equivalence
  \begin{equation*}
    (\eta(x)=\eta(y))\simeq \brck{x=y}
  \end{equation*}
  for each $x,y:A$. 
\end{cor}

By this corollary, we may think of the set truncation $\trunc{0}{A}$ of $A$ as the set of connected components of $A$. Indeed, if we have an unspecified identification $\brck{x=y}$ in $A$, then we think of $x$ and $y$ as being in the same connected component. For example, any $k$-element set is a type that is in the same connected component of $\UU$ as the type $\Fin{k}$.

\begin{defn}
  A type $A$ is said to be \define{connected}\index{connected type|textbf} if its set truncation $\trunc{0}{A}$ is contractible. We define\index{is-conn@{$\isconn(A)$}|textbf}
  \begin{equation*}
    \isconn(A)\defeq\iscontr\trunc{0}{A}.
  \end{equation*}
  Furthermore, we say that a map $f:A\to B$ is \define{connected}\index{connected map|textbf} if all its fibers are connected.
\end{defn}

\begin{rmk}
  In particular, every connected type is inhabited, because if $\trunc{0}{A}$ is contractible, then we have equivalences\index{inhabited type!connected types are inhabited}\index{connected type!connected types are inhabited}
  \begin{equation*}
    \brck{A}\simeq (\trunc{0}{A}\to\brck{A}) \simeq (A\to \brck{A}),
  \end{equation*}
  and the latter type contains the unit of the propositional truncation.
\end{rmk}

Using the notion of connectivity, we can add one more property to the list of equivalent characterizations of set truncations given in \cref{thm:set-truncation}.

\begin{thm}\label{thm:unit-set-truncation-connected}
  Consider a map $f:A\to B$ into a set $B$. Then the following are equivalent:
  \begin{enumerate}
  \item \label{item:unit-set-truncation-connected-i}The map $f$ is a set truncation.
  \item \label{item:unit-set-truncation-connected-ii}The map $f$ is connected.
  \end{enumerate}
\end{thm}

\begin{proof}
  First, suppose that $f$ is a set truncation, and consider $b:B$. Our goal is to show that the type
  \begin{equation*}
    \trunc{0}{\fib{f}{b}}
  \end{equation*}
  is contractible. Since $f$ is surjective by \cref{thm:set-truncation}, there exists an element $a:A$ equipped with an identification $f(a)=b$. We are proving a proposition, so it suffices to show that $\trunc{0}{\fib{f}{f(a)}}$ is contractible. At the center of contraction we have
  \begin{equation*}
    \eta(a,\refl{}):\trunc{0}{\fib{f}{f(a)}}.
  \end{equation*}
  In order to construct the contraction, we use the dependent universal property of the set truncation, by which it suffices to construct a function
  \begin{equation*}
    \prd{x:A}\prd{p:f(x)=f(a)}\eta(a,\refl{})=\eta(x,p)
  \end{equation*}
  Recall from \cref{thm:set-truncation} that the map $f$ is effective, so we have an equivalence $e:\brck{x=a}\simeq (f(x)=f(a))$ for every $x:A$. Furthermore, equality in set truncations are propositions, so we may even eliminate the propositional truncation from $\brck{x=a}$. Therefore it suffices to prove
  \begin{equation*}
    \prd{x:A}\prd{p:x=a}\eta(a,\refl{})=\eta(x,e(\eta(p)))
  \end{equation*}
  This is immediate, since $e(\eta(\refl{}))=\refl{}$. This completes the proof of \ref{item:unit-set-truncation-connected-i} implies \ref{item:unit-set-truncation-connected-ii}.

  For the converse, suppose that $f$ is connected, and consider a set $X$. Note that we have a commuting square
  \begin{equation*}
    \begin{tikzcd}[column sep=8em]
      \Big(\prd{b:B}\trunc{0}{\fib{f}{b}}\to X\Big) \arrow[r,"{h\mapsto\lam{b}{t}h(b,\eta(t))}"] & \Big(\prd{b:B}\fib{f}{b}\to X\Big) \arrow[d,swap,"{h\mapsto\lam{a}h(f(a),(a,\refl{}))}"] \\
      (B\to X) \arrow[r,swap,"\blank\circ f"] \arrow[u,"h\mapsto\lam{b}\lam{u}h(b)"] & (A\to X)
    \end{tikzcd}
  \end{equation*}
  In this commuting square, the map on the left is an equivalence since $\trunc{0}{\fib{f}{b}}$ is contractible for each $b:B$. The top map is an equivalence because $X$ is a set, and the right map is an equivalence by \cref{ex:pi-fib}. Therefore it follows that the bottom map is an equivalence, which completes the proof that \ref{item:unit-set-truncation-connected-ii} implies \ref{item:unit-set-truncation-connected-i}.
\end{proof}

\begin{rmk}
  There are truncation operations for every truncation level. That is, we can define for every type $A$ a map $\eta:A\to\trunc{k}{A}$ such that the map
\begin{equation*}
  \blank\circ\eta : (\trunc{k}{A}\to X)\to (A\to X)
\end{equation*}
is an equivalence for every $k$-truncated type $X$. To learn more about general $k$-truncations, we refer to Chapter 7 of \cite{hottbook}.
\end{rmk}
\index{set truncation|)}
\index{set quotient!set truncation|)}


\begin{exercises}
  \exitem Consider a proposition $P$, and define the relation $\sim_P$\index{~ P@{$\sim_P$}|textbf} on $\bool$ by
  \begin{align*}
    (\btrue\sim_P\btrue) & \defeq \unit & (\btrue\sim_P\bfalse) & \defeq P \\
    (\bfalse\sim_P\btrue) & \defeq P & (\bfalse\sim_P\bfalse) & \defeq \unit
  \end{align*}
  \begin{subexenum}
  \item  Show that $\sim_P$ is an equivalence relation.
  \item Consider a universe $\UU$ containing the proposition $P$. Construct an embedding ${\bool/{\sim}_P}\hookrightarrow\prop_\UU$.  
  \item Use the quotient $\bool/\sim_P$ to show that the axiom of choice implies the law of excluded middle.\index{axiom of choice!implies law of excluded middle}
  \end{subexenum}
  \exitem Consider an equivalence relation $R:A\to(A\to\prop_\UU)$ on $A$, where $\UU$ is a universe containing $A$. Show that type type
  \begin{equation*}
    \sm{X:\UU}\sm{f:A\twoheadrightarrow X}\prd{x,y:A}(f(x)=f(y))\simeq R(x,y)
  \end{equation*}
  is contractible.
  \exitem For any type $A$, show that the type of equivalence relations equipped with a choice of representatives of its equivalence classes is equivalent to the type of \define{set-based retracts} of $A$, i.e., the type\index{retract!set-based retract|textbf}\index{set-based retract|textbf}\index{Retr Set U X@{$\Retr_{\Set_\UU}(X)$}|textbf}\index{Retr Set U X@{$\Retr_{\Set_\UU}(X)$}!is equivalent to equivalence relations with canonical representatives}
  \begin{equation*}
    \Retr_{\Set_\UU}(A) \defeq \sm{X:\Set_\UU}\sm{i:X\to A}\sm{q:A\to X} q\circ i\htpy \idfunc.
  \end{equation*}
  \exitem  \label{ex:sigmadecompositions}A \define{$\Sigma$-decomposition}\index{S-decomposition@{$\Sigma$-decomposition}|textbf} of a type $A$ consists of a type $X$ (the \define{indexing type}\index{S-decomposition@{$\Sigma$-decomposition}!indexing type|textbf} of the $\Sigma$-decomposition) equipped with a family $Y$ of inhabited types indexed by $X$ and an equivalence
  \begin{equation*}
    e:A\simeq \sm{x:X}Y(x).
  \end{equation*}
  In other words, the type of all $\Sigma$-decompositions of $A$ is defined by
  \begin{equation*}
    \Sigmadecomposition_\UU(A) \defeq \sm{X:\UU}\sm{Y:X\to\sm{Z:\UU}\brck{Z}}A\simeq\sm{x:X}Y(x).
  \end{equation*}
  \begin{subexenum}
  \item Construct an equivalence
    \begin{equation*}
      \Sigmadecomposition_\UU(A)\simeq \sm{X:\UU}A\twoheadrightarrow X.
    \end{equation*}
  \item A $\Sigma$-decomposition is said to be \define{set-indexed}\index{S-decomposition@{$\Sigma$-decomposition}!set indexed|textbf}\index{set-indexed S-decomposition@{set-indexed $\Sigma$-decomposition}|textbf} if its indexing type is a set. We will write $\Sigmadecomposition_{\Set_\UU}(A)$ for the type of all set-indexed $\Sigma$-decompositions of $A$ in $\UU$. Construct an equivalence
    \begin{equation*}
      \eqrel_\UU(A)\simeq \Sigmadecomposition_{\Set_\UU}(A).
    \end{equation*}
  \end{subexenum}
  \exitem \label{ex:is-surjective-fiber-inclusion}Consider a type $A$ equipped with an element $a:A$. Show that the following are equivalent:
  \begin{enumerate}
  \item The type $A$ is connected.
  \item There is an element of type $\brck{a=x}$ for any $x:A$.
  \item For any family $B$ over $A$, the fiber inclusion\index{fiber inclusion}
    \begin{equation*}
      i_a:B(a)\to\sm{x:A}B(x)
    \end{equation*}
    defined in \cref{ex:is-trunc-fiber-inclusion} is surjective.
  \end{enumerate}
  \exitem \label{ex:poset-reflection}Consider a preorder $(A,\leq)$, and define for any $a:A$ the order preserving map
  \begin{equation*}
    y_a : \preord(\op{(A,\leq)},{(\prop_\UU,{\to})})
  \end{equation*}
  by $y_a(x)\defeq(x\leq a)$. Furthermore, define the \define{poset reflection}\index{poset reflection|textbf}\index{poset!poset reflection}\index{preorder!poset reflection} $\posetreflection{A}$\index{[[A]] Pos@{$\posetreflection{A}$}|see {poset reflection}} to be the image of the map
  \begin{equation*}
    a\mapsto y_a : A\to \preord(\op{(A,\leq)},{(\prop_\UU,{\to})}).
  \end{equation*}
  \begin{subexenum}
  \item Show that the image of the map $a\mapsto y_a$ satisfies the universal property of the set quotient of the equivalence relation\index{set quotient!poset reflection}\index{poset reflection!is set quotient}
    \begin{equation*}
      x,y\mapsto (x\leq y)\land (y\leq x).
    \end{equation*}
  \item Equip the type $\posetreflection{A}$ with the structure of a poset and construct an order preserving map $\eta : A \to \posetreflection{A}$ that satisfies the following universal property: For any poset $P$, any order preserving map $f:A\to P$ extends uniquely along $\eta$ to an order preserving map $g:\posetreflection{A}\to P$, as indicated in the following diagram:\index{universal property!of poset reflections|textbf}\index{poset reflection!universal property|textbf}
  \begin{equation*}
    \begin{tikzcd}
      A \arrow[r,"f"] \arrow[d,swap,"\eta"] & P. \\
      \posetreflection{A} \arrow[ur,dashed]
    \end{tikzcd}
  \end{equation*}
  \end{subexenum}
  \exitem Consider a map $f:A\to B$.
  \begin{subexenum}
  \item Show that the type of maps $\trunc{0}{f}:\trunc{0}{A}\to\trunc{0}{B}$ equipped with a homotopy witnessing that the square
    \begin{equation*}
      \begin{tikzcd}
        A \arrow[r,"f"] \arrow[d,swap,"\eta"] & B \arrow[d,"\eta"] \\
        \trunc{0}{A} \arrow[r,swap,"\trunc{0}{f}"] & \trunc{0}{B}
      \end{tikzcd}
    \end{equation*}
    commutes is contractible.\index{functorial action!set truncation|textbf}\index{set truncation!functorial action|textbf}
  \item Show that if $f$ is injective, then $\trunc{0}{f}:\trunc{0}{A}\to\trunc{0}{B}$ is injective.
  \item Show that the following are equivalent
    \begin{enumerate}
    \item The map $f$ is surjective.
    \item the map $\trunc{0}{f}:\trunc{0}{A}\to\trunc{0}{B}$ is surjective.
    \end{enumerate}
  \item Construct a map $h:\im(f)\to\im\trunc{0}{f}$ such that the squares
    \begin{equation*}
      \begin{tikzcd}
        A \arrow[r,"q_f"] \arrow[d,swap,"\eta"] & \im(f) \arrow[d,swap,"h"] \arrow[r,"i_f"] & B \arrow[d,"\eta"] \\
        \trunc{0}{A} \arrow[r,swap,"q_{\trunc{0}{f}}"] & \im\trunc{0}{f} \arrow[r,swap,"i_{\trunc{0}{f}}"] & \trunc{0}{B}
      \end{tikzcd}
    \end{equation*}
    commute, and show that $h$ is a set truncation of $\im(f)$.
  \end{subexenum}
  \exitem Consider a type $A$, and suppose that $\trunc{0}{A}$ is a finite type with $k$ elements. Show that there exists a map $f:\Fin{k}\to A$ such that $\eta\circ f$ is an equivalence, i.e., prove the proposition
  \begin{equation*}
    \exists_{(f:\Fin{k}\to A)}\isequiv(\eta\circ f).
  \end{equation*}
  \exitem Consider a type $A$ and a universe $\UU$ containing $A$. Let
  \begin{equation*}
    \tau:A\to ((A\to\prop_\UU)\to\prop_\UU)
  \end{equation*}
  be the map defined by $\tau(a)\defeq\lam{f}f(a)$. Show that the map
  \begin{equation*}
    q_\tau : A\to\im(\tau)
  \end{equation*}
  obtained from the image factorization of $A$ is a set truncation of $A$.
  \exitem \label{ex:weakly-path-constant}A map $f:A \to B$ is called \define{weakly path-constant}\index{weakly path constant map|textbf} if it comes equipped with an element of type
  \begin{equation*}
    \isweaklypathconstant(f) : \prd{x,y:A}\prd{p,q:x=y}\ap{f}{p}=\ap{f}{q}.
  \end{equation*}
  In other words, $f$ is weakly path-constant if for each $x,y:A$ the map $\apfunc{f}:(x=y)\to (f(x)=f(y))$ is weakly constant in the sense of \cref{defn:weakly-constant}.
  \begin{subexenum}
  \item Show that every map $\trunc{0}{A}\to B$ is weakly path-constant. Use this to obtain a map
    \begin{equation*}
      \alpha : \Big(\trunc{0}{A}\to B\Big)\to\Big(\sm{f:A\to B}\isweaklypathconstant(f)\Big).
    \end{equation*}
  \item Show that if $B$ is a $1$-type, then the map $\alpha$ is an equivalence. In other words, show that every weakly path-constant map $f:A\to B$ into a $1$-type $B$ has a unique extension
    \begin{equation*}
      \begin{tikzcd}
        A \arrow[r,"f"] \arrow[d,swap,"\eta"] & B. \\
        \trunc{0}{A} \arrow[ur,dashed]
      \end{tikzcd}
    \end{equation*}
  \end{subexenum}
  \exitem Consider two universes $\UU$ and $\VV$. Use the type theoretic replacement axiom to show that the type of locally $\VV$-small types in $\UU$ is equivalent to the type\index{locally small type}
  \begin{equation*}
    \sum_{(Y:\mathsf{Locally\usc{}}\VV\mathsf{\usc{}Small\usc{}Set}_\UU)}\Big(Y\to \sm{Z:\VV}\isconn(Z)\Big)
  \end{equation*}
  of locally $\VV$-small sets $Y$ in $\UU$ equipped with a family of connected types in $\VV$.
  \exitem Show that every finite type is uniquely a product of finitely many finite types of prime cardinality, in the sense that the type
  \begin{equation*}
    \sm{X:\F}\sm{Y:X\to\sm{p:\primeN}\BS_p}\Brck{A\simeq\prd{x:X}Y(x)}
  \end{equation*}
  is connected for every finite type $A$.
  \exitem \label{ex:stirling-type-of-the-second-kind}Consider two types $A$ and $B$. The \define{Stirling type of the second kind}\index{Stirling type of the second kind|textbf}\index{{{A B}}@{$\stirling{A}{B}$}|see {Stirling type of the second kind}} is the type
  \begin{equation*}
    \stirling{A}{B}:=\sm{X:\UU_B}A\twoheadrightarrow X. 
  \end{equation*}
  \begin{subexenum}
  \item Show that if $B$ is a $k$-type, then the type $\stirling{A}{B}$ is also a $k$-type.\index{Stirling type of the second kind!is truncated}\index{truncated type!Stirling type of the second kind}
  \item Suppose that $B$ has decidable equality. Construct an equivalence
  \begin{equation*}
    \stirling{A+\unit}{B+\unit}\simeq (B+\unit)\times\stirling{A}{B+\unit}+\stirling{A}{B}
  \end{equation*}
  \item Suppose that $A$ and $B$ are finite types of cardinality $n$ and $k$. Show that the Stirling type $\stirling{A}{B}$ of the second kind is a finite type of cardinality $\stirling{n}{k}$, where $\stirling{n}{k}$ is the \define{Stirling number of the second kind}.\index{Stirling type of the second kind!is finite}\index{finite type!Stirling type of the second kind}
  \end{subexenum}
  \exitem \label{ex:distributive-pi-coprod}In this exercise we extend the definition of the binomial types to $\trunc{0}{\UU}$ as follows: For a type $X:\UU$ and $k:\trunc{0}{\UU}$, we define\index{binomial type!extended definition|textbf}
  \begin{equation*}
    \binomtype{X}{k}\defeq \sm{Y:\fib{\eta}{k}}Y\demb X.
  \end{equation*}
  Furthermore, for $(Y,i):\binomtype{X}{k}$, define
  \begin{align*}
    X\setminus Y & \defeq \sm{x:X}\neg(\fib{i}{x}). \\
    \complement{i} & \defeq \proj 1.
  \end{align*}
  Now consider a type $X$ and two type families $A$ and $B$ over $X$, and let $\UU$ be a universe containing $X$, $A$, and $B$. Show that the type $\prd{x:X}A(x)+B(x)$ is equivalent to the type\index{distributivity!of P over coproducts@{of $\Pi$ over coproducts}}\index{dependent function type!distributivity of P over coproducts@{distributivity of $\Pi$ over coproducts}}
  \begin{equation*}
    \sm{k:\trunc{0}{\UU}}\sm{(Y,i):\dbinomtype{X}{k}}\Big(\prd{y:Y}A(i(y))\Big)\times\Big(\prd{y:X\setminus Y}B(\complement{i}(y))\Big).
  \end{equation*}
\end{exercises}
\index{set quotient|)}

%%% Local Variables:
%%% mode: latex
%%% TeX-master: "hott-intro"
%%% End:

\section{Groups in univalent mathematics}\label{sec:groups}
\index{group|(}

In this section we demonstrate a very common way to use the univalence axiom\index{univalence axiom}, showing that isomorphic groups can be identified. When you introduce a certain kind of structure in type theory, such as groups or rings, you automatically obtain the type of all such structures. In other words, we define what a group is by defining the type of all groups, we define what a ring is by defining the type of all rings, and so on. The elements of the type of all groups are of course groups, such as the group of integers, integers modulo $k$, automorphism groups, and so on. The next important question is how two elements in the type of groups can be identified. This question is answered with the help of the univalence axiom: isomorphic groups can be identified. This is an instance of the \emph{structure identity principle}\index{structure identity principle}, which we covered in \cref{sec:structure-identity-principle}.

Identifiying isomorphic groups is a common \emph{informal} practice in classical mathematics. For example, by the third isomorphism theorem we have an isomorphism
\begin{equation*}
  (G/N)/(K/N)\cong (G/K)
\end{equation*}
for any sequence $N \trianglelefteq K \trianglelefteq G$ of normal subgroups of $G$, and it is common to simply write $(G/N)/(K/N)=G/K$. Of course, classical mathematicians know that this convention is incompatible with the axioms of Zermelo-Fraenkel set theory, but that does not stop them from applying this useful abuse of notation. In univalent mathematics we make this informal practice precise and formal.

\subsection{The type of all groups}

In order to efficiently characterize the identity type of the type of all groups in a universe $\UU$, we introduce the type of groups in two stages: first we introduce the type of \emph{semigroups}, and then we introduce groups as semigroups that possess a unit element and inverses. Since semigroups can have at most one unit element and since elements of semigroups can have at most one inverse, it follows that the type of groups is a subtype of the type of semigroups, and this will help us with the characterization of the identity type of the type of all groups.

\begin{rmk}
  In order to show that isomorphic (semi)groups can be identified, it has to be part of the definition of a (semi)group that its underlying type is a set. This is an important observation: in many branches of algebra the objects of study are \emph{set-level} structures\index{set-level structure}.

  A notable exception is formed by categories, which are objects at truncation level $1$, i.e., at the level of \emph{groupoids}. We will not cover categories in this book. For more about categories we recommend Chapter 9 of \cite{hottbook}.
\end{rmk}

\begin{defn}
  A \define{semigroup}\index{semigroup|textbf} in a universe $\UU$ is a triple $(G,\mu,\alpha)$ consisting of a set $G$ in $\UU$ equipped with a binary operation $\mu:G\to (G\to G)$ and a homotopy
  \begin{equation*}
    \alpha : \prd{x,y,z:G}\mu(\mu(x,y),z)=\mu(x,\mu(y,z))
  \end{equation*}
  witnessing that $\mu$ is \define{associative}\index{associative|textbf}.
  We write $\semigroup_\UU$\index{Semigroup@{$\semigroup_\UU$}|textbf} for the type of all semigroups in $\UU$, i.e., for the type
  \begin{equation*}
    \sm{G:\Set_\UU}\sm{\mu:G\to(G\to G)}\prd{x,y,z:G}\mu(\mu(x,y),z)=\mu(x,\mu(y,z)).
  \end{equation*}
\end{defn}

\begin{defn}
  A semigroup $G$ is said to be \define{unital}\index{semigroup!unital}\index{unital semigroup} if it comes equipped with a \define{unit}\index{unit!of a unital semigroup} $e:G$ that satisfies the left and right unit laws\index{unit laws!for a unital semigroup}
  \begin{align*}
    \leftunit : \prd{y:G}\mu(e,y)=y \\
    \rightunit : \prd{x:G}\mu(x,e)=x.
  \end{align*}
  We write $\isunital(G)$\index{is-unital@{$\isunital$}} for the type of such triples $(e,\leftunit,\rightunit)$. Unital semigroups are also called \define{monoids}\index{monoid|textbf}, so we define\index{Monoid@{$\monoid_\UU$}}
  \begin{equation*}
    \monoid_\UU\defeq\sm{G:\semigroup_\UU}\isunital(G).
  \end{equation*}
\end{defn}

The unit of a semigroup is of course unique once it exists. In univalent mathematics we express this fact by asserting that the type $\isunital(G)$ is a proposition for each semigroup $G$. In other words, being unital is a \emph{property} of semigroups rather than structure on it. This is typical for univalent mathematics: we express that a structure is a property by proving that this structure is a proposition.

\begin{lem}
  For a semigroup $G$ the type $\isunital(G)$ is a proposition.\index{is-unital@{$\isunital$}!is a proposition}
\end{lem}

\begin{proof}
  Let $G$ be a semigroup. Note that since $G$ is a set, it follows that the types of the left and right unit laws are propositions. Therefore it suffices to show that any two elements $e,e':G$ satisfying the left and right unit laws can be identified. This is easy:
  \begin{equation*}
    e = \mu(e,e') = e'.\qedhere
  \end{equation*}
\end{proof}

\begin{defn}
  Let $G$ be a unital semigroup. We say that $G$ \define{has inverses}\index{unital semigroup!has inverses}\index{semigroup!has inverses} if it comes equipped with an operation $x\mapsto x^{-1}$ of type $G\to G$, satisfying the left and right inverse laws\index{inverse laws!for a group}
  \begin{align*}
    \leftinv & : \prd{x:G}\mu(x^{-1},x)=e \\
    \rightinv & : \prd{x:G}\mu(x,x^{-1}) = e.
  \end{align*}
  We write $\isgroup'(G,e)$\index{is-group'@{$\isgroup'$}|textbf} for the type of such triples $((\blank)^{-1},\leftinv,\rightinv)$, and we write\index{is-group@{$\isgroup$}|textbf}
  \begin{equation*}
    \isgroup(G)\defeq\sm{e:\isunital(G)}\isgroup'(G,e)
  \end{equation*}
  A \define{group}\index{group|textbf} is a unital semigroup with inverses. We write $\group$\index{Group@{$\group_\UU$}|textbf} for the type of all groups in $\UU$.
\end{defn}

\begin{lem}
  For any semigroup $G$ the type $\isgroup(G)$ is a proposition.\index{is-group@{$\isgroup$}!is a proposition}
\end{lem}

\begin{proof}
  We have already seen that the type $\isunital(G)$ is a proposition. Therefore it suffices to show that the type $\isgroup'(G,e)$ is a proposition\index{is-group'@{$\isgroup'$}!is a proposition} for any $e:\isunital(G)$.

  Since a semigroup $G$ is assumed to be a set, we note that the types of the inverse laws are propositions. Therefore it suffices to show that any two inverse operations satisfying the inverse laws are homotopic.

  Let $x\mapsto x^{-1}$ and $x\mapsto x^{-1'}$ be two inverse operations on a unital semigroup $G$, both satisfying the inverse laws. Then we have the following identifications
  \begin{align*}
    x^{-1} & = \mu(e,x^{-1}) \\
    & = \mu(\mu(x^{-1'},x),x^{-1}) \\
    & = \mu(x^{-1'},\mu(x,x^{-1})) \\
    & = \mu(x^{-1'},e) \\
    & = x^{-1'}
  \end{align*}
  for any $x:G$. Thus the two inverses of $x$ are the same, and the claim follows.
\end{proof}

\begin{eg}
  The type $\Z$ of integers\index{Z@{$\Z$}!is a group}\index{group!Z@{$\Z$}} has the structure of a group, with the group operation being addition. The fact that $\Z$ is a set was shown in \cref{ex:set_coprod}, and the group laws were shown in \cref{ex:int_group_laws}. 
\end{eg}

\begin{eg}
  Given a set $X$, we define  the \define{automorphism group}\index{automorphism group|textbf}\index{group!automorphism group of a set|textbf}\index{set!automorphism group|textbf} of $X$ by\index{Aut(X)@{$\Aut(X)$}|see {automorphism group}}
  \begin{equation*}
    \Aut(X)\defeq (X\simeq X).
  \end{equation*}
  The group operation of $\Aut(X)$ is given by composition of equivalences, and the unit of the group is the identity function. An important special case of the automorphism groups is the \define{symmetric group}\index{symmetric group|textbf}\index{S n@{$S_n$}|see {symmetric group}}\index{group!S n@{$S_n$}|textbf}
  \begin{equation*}
    S_n\defeq \Aut(\Fin{n}).
  \end{equation*}
\end{eg}

\subsection{Group homomorphisms}

\begin{defn}
  Let $G$ and $H$ be (semi)groups. A \define{homomorphism}\index{homomorphism!of semigroups|textbf}\index{semigroup!homomorphism|textbf}\index{homomorphism!of groups|textbf}\index{group!homomorphism|textbf}\index{group homomorphism|textbf}\index{semigroup homomorphism|textbf} of (semi)groups from $G$ to $H$ is a pair $(f,\mu_f)$ consisting of a function $f:G\to H$ between their underlying types, and a homotopy
  \begin{equation*}
    \mu_f:\prd{x,y:G} f(\mu_G(x,y))=\mu_H(f(x),f(y))
  \end{equation*}
  witnessing that $f$ preserves the binary operation of $G$. We will write\index{hom(G,H) for semigroups@{$\hom(G,H)$ for semigroups}|textbf}\index{hom(G,H) for groups@{$\hom(G,H)$ for groups}|textbf}
  \begin{equation*}
    \hom(G,H)
  \end{equation*}
  for the type of all (semi)group homomorphisms from $G$ to $H$.
\end{defn}

\begin{rmk}\label{rmk:is-set-hom-semigroup}
  Since it is a property for a function to preserve the multiplication of a semigroup, it follows easily that equality of semigroup homomorphisms is equivalent to the type of homotopies between their underlying functions. In particular, it follows that the type of homomorphisms of semigroups is a set.
\end{rmk}

\begin{rmk}\label{rmk:category-semigroup}
  The \define{identity homomorphism}\index{identity homomorphism!of semigroups|textbf}\index{identity homomorphism!of groups|textbf} on a (semi)group $G$ is defined to be the pair consisting of
  \begin{align*}
    \idfunc & : G \to G \\
    \lam{x}\lam{y}\refl{} & : \prd{x,y:G} \mu_G(x,y) = \mu_G(x,y).
  \end{align*}
  Let $f:G\to H$ and $g:H\to K$ be (semi)group homomorphisms. Then the composite function $g\circ f:G\to K$ is also a (semi)group homomorphism\index{composition!of semigroup homomorphisms|textbf}\index{composition!of group homomorphisms|textbf}, since we have the identifications
  \begin{equation*}
    \begin{tikzcd}
      {g(f(\mu_G(x,y)))} \arrow[r,equals] & {g(\mu_H(f(x),f(y)))} \arrow[r,equals] & {\mu_K(g(f(x)),g(f(y)))}.
    \end{tikzcd}
  \end{equation*}
  Since the identity type of (semi)group homomorphisms is equivalent to the type of homotopies between (semi)group homomorphisms it is easy to see that (semi)group homomorphisms satisfy the laws of a category, i.e., that we have the identifications
  \begin{align*}
    \idfunc\circ f & = f \\
    g\circ \idfunc & = g \\
    (h\circ g) \circ f & = h \circ (g \circ f)
  \end{align*}
  for any composable (semi)group homomorphisms $f$, $g$, and $h$.
\end{rmk}

\begin{defn}
Let $h:\hom(G,H)$ be a homomorphism of (semi)groups. Then $h$ is said to be an \define{isomorphism}\index{group!isomorphism|textbf}\index{isomorphism!of groups|textbf}\index{semigroup!isomorphism|textbf}\index{isomorphism!of semigroups|textbf} if it comes equipped with an element of type $\isiso(h)$\index{is-iso@{$\isiso(h)$}!for semigroup homomorphisms|textbf}\index{is-iso@{$\isiso(h)$}!for group homomorphisms|textbf}, consisting of triples $(h^{-1},p,q)$ consisting of a homomorphism $h^{-1}:\hom(H,G)$ of semigroups and identifications
\begin{equation*}
p:h^{-1}\circ h=\idfunc[G]\qquad\text{and}\qquad q:h\circ h^{-1}=\idfunc[H]
\end{equation*}
witnessing that $h^{-1}$ satisfies the inverse laws\index{inverse laws!for semigroup isomorphisms}\index{inverse laws!for group isomorphisms}We write $G\cong H$ for the type of all isomorphisms of semigroups from $G$ to $H$, i.e.,
\begin{equation*}
G\cong H \defeq \sm{h:\hom(G,H)}\sm{k:\hom(H,G)} (k\circ h = \idfunc[G])\times (h\circ k=\idfunc[H]).
\end{equation*}
\end{defn}

If $f$ is an isomorphism, then its inverse is unique. In other words, being an isomorphism is a property.

\begin{lem}
  For any semigroup homomorphism $h:\hom(G,H)$, the type
  \begin{equation*}
    \isiso(h)
  \end{equation*}
  is a proposition.\index{is-iso@{$\isiso(h)$}!is a proposition} It follows that the type $G\cong H$ is a set for any two semigroups $G$ and $H$.
\end{lem}

\begin{proof}
  Let $k$ and $k'$ be two inverses of $h$. In \cref{rmk:is-set-hom-semigroup} we have observed that the type of semigroup homomorphisms between any two semigroups is a set. Therefore it follows that the types $h\circ k=\idfunc$ and $k\circ h=\idfunc$ are propositions, so it suffices to check that $k=k'$. In \cref{rmk:is-set-hom-semigroup} we also observed that the equality type $k=k'$ is equivalent to the type of homotopies $k\htpy k'$ between their underlying functions. We construct a homotopy $k\htpy k'$ by the usual argument:
  \begin{equation*}
    \begin{tikzcd}
      k(y) \arrow[r,equals] & k(h(k'(y)) \arrow[r,equals] & k'(y).
    \end{tikzcd}\qedhere
  \end{equation*}
\end{proof}

\subsection{Isomorphic groups are equal}

\begin{lem}\label{lem:grp_iso}
  A (semi)group homomorphism $h:\hom(G,H)$ is an isomorphism if and only if its underlying map is an equivalence. Consequently, there is an equivalence
  \begin{equation*}
    (G\cong H)\simeq \sm{e:G\simeq H}\prd{x,y:G}e(\mu_G(x,y))=\mu_H(e(x),e(y))
  \end{equation*}
\end{lem}

\begin{proof}
  If $h:\hom(G,H)$ is an isomorphism, then the inverse semigroup homomorphism also provides an inverse of the underlying map of $h$. Thus we obtain that $h$ is an equivalence. For the converse, suppose that the underlying map of $f:G\to H$ is an equivalence. Then its inverse is also a semigroup homomorphism, since we have
  \begin{align*}
    f^{-1}(\mu_H(x,y)) & = f^{-1}(\mu_H(f(f^{-1}(x)),f(f^{-1}(y)))) \\
               & = f^{-1}(f(\mu_G(f^{-1}(x),f^{-1}(y)))) \\
               & = \mu_G(f^{-1}(x),f^{-1}(y)). \qedhere
  \end{align*}
\end{proof}

\begin{defn}
Let $G$ and $H$ be a semigroups in a univalent universe $\UU$. We define the family of maps\index{iso-eq for semigroups@{$\isoeq$ for semigroups}}
\begin{equation*}
\isoeq : (G=H)\to (G\cong H)
\end{equation*}
indexed by $H:\semigroup_\UU$ by $\isoeq(\refl{})\defeq\idfunc[G]$.
\end{defn}

\begin{thm}\label{thm:iso-eq-semigroup}
Consider a semigroup $G$ in a univalent universe $\UU$. Then the family of maps\index{identity type!of Semigroup@{of $\semigroup_\UU$}}\index{Semigroup@{$\semigroup_\UU$}!identity type}\index{characterization of identity type!of Semigroup@{of $\semigroup_\UU$}}
\begin{equation*}
\isoeq : (G=H)\to (G\cong H)
\end{equation*}
indexed by $H:\semigroup_\UU$ is a family of equivalences.
\end{thm}

\begin{proof}
By the fundamental theorem of identity types \cref{thm:id_fundamental}\index{fundamental theorem of identity types} it suffices to show that the total space
\begin{equation*}
\sm{H:\semigroup_\UU}G\cong H
\end{equation*}
is contractible. Since the type of isomorphisms from $G$ to $H$ is equivalent to the type of equivalences from $G$ to $H$ it suffices to show that the type
\begin{equation*}
  \sm{H:\semigroup_\UU}\sm{e:\eqv{G}{H}}\prd{x,y:G}e(\mu_G(x,y))=\mu_{H}(e(x),e(y)))
\end{equation*}
is contractible. Since $\semigroup_\UU\jdeq\sm{H:\Set_\UU}\hasassociativemul(H)$ we are in position to apply the structure identity principle stated in \cref{thm:structure-identity-principle}. Note that $H\mapsto G\simeq H$ is an identity system on $\Set_\UU$ at the set $G$. By condition (v) of \cref{thm:structure-identity-principle} it therefore suffices to show that the type
\begin{equation*}
  \sm{\mu':\hasassociativemul(G)}\prd{x,y:G}\mu_G(x,y)=\mu'(x,y)
\end{equation*}
is contractible. This follows by function extensionality, since associativity of a binary operation on a set is a proposition.
\end{proof}

\begin{cor}
The type $\semigroup_\UU$ is a $1$-type.\index{Semigroup@{$\semigroup_\UU$}!is a 1-type@{is a $1$-type}}
\end{cor}

\begin{proof}
  The identity types of $\semigroup_\UU$ are sets because they are equivalent to the sets of isomorphisms between semigroups.
\end{proof}

We now turn to the proof that isomorphic groups are equal. Analogously to the map $\isoeq$ of semigroups, we have a map $\isoeq$ of groups. Note, however, that the domain of this map is now the identity type $G=H$ of the \emph{groups} $G$ and $H$, so the maps $\isoeq$ of semigroups and groups are not exactly the same maps.

\begin{defn}
  Let $G$ and $H$ be groups in a univalent universe $\UU$. We define the family of maps\index{iso-eq for groups@{$\isoeq$ for groups}}
  \begin{equation*}
    \isoeq : (G=H)\to (G\cong H)
  \end{equation*}
  indexed by $H:\Group_\UU$ by $\isoeq(\refl{})\defeq\idfunc[G]$.
\end{defn}

\begin{thm}
  For any two groups $G$ and $H$ in a univalent universe $\UU$, the map\index{identity type!of Group@{of $\group_\UU$}}\index{Group@{$\group_\UU$}!characterization of identity type}\index{characterization of identity type!of Group@{of $\Group_\UU$}}
  \begin{equation*}
    \isoeq:(G=H)\to (G\cong H)
  \end{equation*}
  is an equivalence.
\end{thm}

\begin{proof}
  Let $G$ and $H$ be groups in $\UU$, and write $UG$ and $UH$ for their underlying semigroups, respectively. Then we have a commuting triangle
  \begin{equation*}
    \begin{tikzcd}[column sep=0]
      (G=H) \arrow[rr,"\apfunc{\proj 1}"] \arrow[dr,swap,"\isoeq"] & & (UG=UH) \arrow[dl,"\isoeq"] \\
      \phantom{(UG=UH)} & (G\cong H)
    \end{tikzcd}
  \end{equation*}
  Since being a group is a property of semigroups it follows that the projection map $\group_\UU\to\semigroup_\UU$ forgetting the unit and inverses, is an embedding. Thus the top map in this triangle is an equivalence. The map on the right is an equivalence by \cref{thm:iso-eq-semigroup}, so the claim follows by the 3-for-2 property.
\end{proof}

\begin{cor}
  The type of groups is a $1$-type.\index{Group@{$\group_\UU$}!is a 1-type@{is a $1$-type}}
\end{cor}

\subsection{Homotopy groups of types}
\index{homotopy group|(}

Since the identity type gives every type groupoidal structure, we can construct for every type $A$ equipped with a base point $a:A$ a sequence of groups $\pi_n(A,a)$ indexed by $n\geq 1$. In order to construct this sequence of groups, we first define the \emph{loop space} operation, which takes pointed types to pointed types.

\begin{defn}
  The type of \define{pointed types}\index{pointed type|textbf} in a universe $\UU$ is defined as\index{U *@{$\UU_\ast$}|textbf}
  \begin{equation*}
    \UU_\ast\defeq\sm{X:\UU}X.
  \end{equation*}
  Given two pointed types $A$ and $B$ with base points $a$ and $b$ respectively, we define the type of \define{pointed maps}\index{pointed map|textbf}\index{A arrow* B@{$A\to_\ast B$}|see {pointed map}}
  \begin{equation*}
    (A\to_\ast B)\defeq\sm{f:A\to B}f(a)=b.
  \end{equation*}
\end{defn}

\begin{defn}\label{defn:loop-spaces}
  Consider a universe $\UU$. We define the \define{loop space}\index{loop space|textbf}\index{O (A)@{$\loopspace{A}$}|see {loop space}} operation
  \begin{equation*}
    \loopspacesym : \UU_\ast\to\UU_\ast
  \end{equation*}
  by $\loopspace{A,a}\defeq(a=a,\refl{})$. Furthermore, we define for every $A:\UU_\ast$ the \define{iterated loop space}\index{iterated loop space|textbf}\index{O n(A)@{$\loopspace[n]{A}$}|see {iterated loop space}} $\loopspace[n]{A}$ recursively by
  \begin{align*}
    \loopspace[0]{A}\defeq A \\
    \loopspace[n+1]{A}\defeq\loopspace{\loopspace[n]{A}}.
  \end{align*}
\end{defn}

\begin{eg}\label{eg:loop-spaces}
  If $A$ is a pointed $1$-type\index{pointed 1-type@{pointed $1$-type}}, then the loop space $\loopspace{A}$ is a set. Furthermore, it has the structure of a group. Its unit is $\refl{}$, and the group operation is given by concatenation of identifications. This satisfies the group laws, since the group laws are just a special case of the groupoid laws for identity types, constructed in \cref{sec:groupoid}. Thus we see that the loop space of a pointed $1$-type is a group\index{loop space!of a 1-type is a group@{of a $1$-type is a group}}\index{group!loop space of a 1-type@{loop space of a $1$-type}}.
\end{eg}

If $A$ is a pointed type, but not assumed to be $1$-truncated, then we can still get 

\begin{defn}
  Consider a pointed type $A$ with base point $a:A$, and let $n\geq 1$. Then we define the \define{$n$-th homotopy group}\index{homotopy group|textbf}\index{group!homotopy group|textbf} $\pi_n(A)$\index{p  n(A)@{$\pi_n(A)$}|see {homotopy group}} of $A$ at $a$ to be the group with underlying set
  \begin{equation*}
    \pi_n(A)\defeq\trunc{0}{\loopspace[n]{A}}
  \end{equation*}
  The unit of the group is $\eta(\refl{})$ and the group operation is the unique binary operation such that
  \begin{equation*}
    \eta(r)\eta(s)=\eta(\ct{r}{s})
  \end{equation*}
  for every $r,s:\loopspace[n]{A}$. The group $\pi_1(A)$\index{p  1(A)@{$\pi_1(A)$}|see {fundamental group}} of a pointed type is called the \define{fundamental group}\index{fundamental group|textbf}\index{group!fundamental group|textbf} of $A$ at its base point $a:A$.
\end{defn}

\begin{rmk}
  Note that for $n=0$, we can still define the set
  \begin{equation*}
    \pi_0(A)\defeq\trunc{0}{A}.
  \end{equation*}
  However, this set does not necessarily come equipped with the structure of a group.
\end{rmk}

\begin{prp}\label{prp:homotopy-group-loop-space}
  For any pointed type $A$ and any $n\geq 1$ we have an isomorphism
  \begin{equation*}
    \pi_{n+1}(A)\cong \pi_n(\loopspace{A}).
  \end{equation*}
\end{prp}

\begin{proof}
  First, observe that we have a pointed equivalence
  \begin{equation*}
    \loopspace{\loopspace[n]{A}}\equiv_\ast\loopspace[n]{\loopspace{A}}.
  \end{equation*}
  This equivalence is constructed by induction on $n$, and also preserves the concatenation operation. Using this equivalence, we obtain a group isomorphism
  \begin{equation*}
    \pi_{n+1}(A)\jdeq \trunc{0}{\loopspace{\loopspace[n]{A}}}\cong\trunc{0}{\loopspace[n]{\loopspace{A}}} \jdeq \pi_n(\loopspace{A}).\qedhere
  \end{equation*}
\end{proof}

Homotopy groups are important algebraic invariants of a type. For example, they can be used to show that two pointed types $A$ and $B$ are not equivalent by showing that two types $A$ and $B$ have non-isomorphic homotopy groups. The study of homotopy groups of types is an intricate and complicated subject, analogous to algebraic topology. Since the homotopy groups of types are obtained in such a canonical manner from the identity types, which are inductively generated by just the reflexivity identification, the subject of studying homotopy groups of types is also called \emph{synthetic homotopy theory}. In the final section of this book we will show that the fundamental group of the circle, which is introduced as a \emph{higher inductive type}, is $\Z$. In this section we will show that equivalent types have isomorphic homotopy groups, and that the homotopy groups $\pi_n(A)$ are abelian if $n\geq 2$.

\begin{defn}
  Consider a pointed map $f:A\to_\ast B$ between two pointed types $A$ and $B$, where $p:f(a)=b$. Then we define the pointed map\index{functorial action!of O@{of $\loopspacesym$}|textbf}
  \begin{equation*}
    \loopspace{f}:\loopspace{A}\to_\ast\loopspace{B}
  \end{equation*}
  by $\loopspace{f}(r)\defeq \ct{(\ct{p^{-1}}{\ap{f}{r}})}{p}$. The identification witnessing that this is indeed a pointed map is obtained from the fact that $\ap{f}{\refl{}}\jdeq\refl{}$ and $\ct{p^{-1}}{p}=\refl{}$.

  Similarly, we define $\loopspace[n]{f}:\loopspace[n]{A}\to_\ast\loopspace[n]{B}$ recursively by\index{functorial action!of O n@{of $\loopspacesym^n$}|textbf}
  \begin{align*}
    \loopspace[0]{f} & \defeq f \\
    \loopspace[n+1]{f} & \defeq \loopspace{\loopspace[n]{f}}.
  \end{align*}
  The functorial action of $\Omega^n$ together with the functorial action of set truncation yield a functorial action\index{functorial action!of p n@{of $\pi_n$}|textbf}
  \begin{equation*}
    \pi_n(f):\pi_n(A)\to\pi_n(B)
  \end{equation*}
  for every pointed map $f:A\to_\ast B$. 
\end{defn}

\begin{rmk}
  Since action of paths preserves path concatenation, it follows that $\Omega^n(f)$ preserves path concatenation, for each $n\geq 1$. Consequently, the maps
  \begin{equation*}
    \pi_n(f):\pi_n(A)\to\pi_n(B)
  \end{equation*}
  are group homomorphisms.
\end{rmk}

\begin{prp}
  Consider a pointed equivalence $e:A\simeq_\ast B$ between two pointed types $A$ and $B$. Then we obtain group isomorphisms
  \begin{equation*}
    \pi_n(e):\pi_n(A)\cong\pi_n(B)
  \end{equation*}
  for all $n\geq 1$.
\end{prp}

\begin{proof}
  For any pointed equivalence $e:A\simeq_\ast B$ it follows that $\pi_n(e)$ is also an equivalence. Using \cref{lem:grp_iso}, the claim now follows.
\end{proof}

\subsection{The Eckmann-Hilton argument}

\index{Eckmann-Hilton argument|(}
The Eckmann-Hilton argument is used to show that $\pi_n(A)$ is an abelian group for all $n\geq 2$. This is achieved by constructing an identification
\begin{equation*}
  \ct{p}{q}=\ct{q}{p}
\end{equation*}
for all $p,q:\loopspace[2]{A}$. Note that identification elimination is not immediately applicable here, since both $p$ and $q$ are identifications of type $\refl{a}=\refl{a}$ with neither endpoint free. Therefore, we must come up with something else.

\begin{defn}
  Consider a binary operation $f:A\to(B\to C)$. The \define{binary action on paths}\index{binary action on paths|textbf}\index{action on paths!binary action on paths|textbf}\index{action on paths!ap-binary@{$\apbinary_f$}|textbf} of $f$ is the family of functions\index{ap-binary@{$\apbinary_f$}|textbf}
  \begin{equation*}
    \apbinary_f:(x=x')\to ((y=y') \to (f(x,y)=f(x',y'))
  \end{equation*}
  indexed by $x,x':A$ and $y,y':B$ given by $\apbinary_f(\refl{},\refl{})\defeq\refl{}$.
\end{defn}

\begin{lem}\label{lem:laws-ap-binary}
  The binary action on paths of $f:A\to(B\to C)$ satisfies the following laws:
  \begin{align*}
    \apbinary_f(\refl{},q) & = \ap{f(x)}{q} \\
    \apbinary_f(p,\refl{}) & = \ap{f(\blank,y)}{p}
  \end{align*}
  and moreover both triangles in the following diagram commute:
  \begin{equation*}
    \begin{tikzcd}[column sep=10em,row sep=4em]
      f(x,y) \arrow[r,equals,"{\ap{f(\blank,y)}{p}}"] \arrow[d,equals,swap,"{\ap{f(x,\blank)}{q}}"] \arrow[dr,equals,"{\apbinary_f(p,q)}"] & f(x',y) \arrow[d,equals,"{\ap{f(x',\blank)}{q}}"] \\
      f(x,y') \arrow[r,equals,swap,"{\ap{f(\blank,y')}{p}}"] & f(x',y')
    \end{tikzcd}
  \end{equation*}
\end{lem}

\begin{proof}
  The proof is immediate by identification elimination on $p$ and $q$, where applicable.
\end{proof}

\begin{eg}
  One particular binary operation to which we can apply the binary action on paths is concatenation of identifications
  \begin{equation*}
    \ct{\blank}{\blank}:(x=y)\to((y=z)\to (x=z))
  \end{equation*}
  This results in the \define{horizontal concatenation}\index{identity type!horizontal concatenation|textbf}\index{horizontal concatenation|textbf} operation\index{r .h s@{$\ct[h]{r}{s}$}|see {horizontal concatenation}}
  \begin{equation*}
    \ct[h]{\blank}{\blank} : (p=p')\to ((q=q') \to (\ct{p}{q}=\ct{p'}{q'})).
  \end{equation*}
  In other words, for any two identifications $r:p=p'$ and $s:q=q'$ as in the diagram
  \begin{equation*}
    \begin{tikzcd}[column sep=huge]
      x \arrow[r,equals,bend left=30,"p",""{name=A,below}] \arrow[r,equals,bend right=30,""{name=B,above},"{p'}"{below}] \arrow[from=A,to=B,phantom,"r\Downarrow"] & y \arrow[r,equals,bend left=30,"q",""{name=C,below}] \arrow[r,equals,bend right=30,""{name=D,above},"{q'}"{below}] \arrow[from=C,to=D,phantom,"s\Downarrow"] & z.
    \end{tikzcd}
  \end{equation*}
  we obtain $\ct[h]{r}{s}\defeq\apbinary_{\ct{\blank}{\blank}}(r,s):\ct{p}{q}=\ct{p'}{q'}$. The \define{vertical concatenation}\index{identity type!vertical concatenation|textbf}\index{vertical concatenation|textbf} operation, which concatenates $r:p=p'$ and $r':p'=p''$ as in the diagram
  \begin{equation*}
    \begin{tikzcd}[column sep=7em]
      x \arrow[r,equals,bend left=60,"p",""{name=A,below}] \arrow[r,equals,""{name=B},""{name=E,below},"{p'}"{near end}] \arrow[r,equals,bend right=60,"{p''}"{below},""{name=F,above}] \arrow[from=A,to=B,phantom,"r\Downarrow"] \arrow[from=E,to=F,phantom,"{r'\Downarrow}"] 
      & y
    \end{tikzcd}
  \end{equation*}
  is given by ordinary concatenation of identifications.
\end{eg}

\begin{lem}\label{lem:unit-laws-horizontal-concat}
  Horizontal concatenation satisfies the following left and right unit laws:\index{unit laws!for horizontal concatenation}\index{horizontal concatenation!unit laws}
  \begin{align*}
    \ct[h]{\refl{\refl{}}}{s} & = s \\
    \ct[h]{r}{\refl{\refl{}}} & = r.
  \end{align*}
\end{lem}

\begin{proof}
  This follows by identification elimination on $r$ and $s$, or alternatively via \cref{lem:laws-ap-binary}.
\end{proof}

In the following lemma we establish the \define{interchange law} for horizontal and vertical concatenation.

\begin{lem}\label{lem:interchange-law}
Consider a diagram of the form\index{interchange law!of horizontal and vertical concatenation}\index{horizontal concatenation!interchange law}\index{vertical concatenation!interchange law}
\begin{equation*}
\begin{tikzcd}[column sep=7em]
x \arrow[r,equals,bend left=60,"p",""{name=A,below}] \arrow[r,equals,""{name=B},""{name=E,below}] \arrow[r,equals,bend right=60,"{p''}"{below},""{name=F,above}] \arrow[from=A,to=B,phantom,"r\Downarrow"] \arrow[from=E,to=F,phantom,"{r'\Downarrow}"] 
& y \arrow[r,equals,bend left=60,"q",""{name=C,below}] \arrow[r,equals,""{name=G,above},""{name=H,below}] \arrow[r,equals,bend right=60,""{name=D,above},"{q''}"{below}] \arrow[from=C,to=G,phantom,"s\Downarrow"] \arrow[from=H,to=D,phantom,"{s'\Downarrow}"] & z.
\end{tikzcd}
\end{equation*}
Then there is an identification
\begin{equation*}
  \ct[h]{(\ct{r}{r'})}{(\ct{s}{s'})}=\ct{(\ct[h]{r}{s})}{(\ct[h]{r'}{s'})}.
\end{equation*}
\end{lem}

\begin{proof}
  We use path induction on both $r$ and $s$. Then it suffices to show that
  \begin{equation*}
    \ct[h]{(\ct{\refl{}}{r'})}{(\ct{\refl{}}{s'})}=\ct{(\ct[h]{\refl{}}{\refl{}})}{(\ct[h]{r'}{s'})}
  \end{equation*}
  Using the unit laws for ordinary concatenation, we see that both sides reduce to $\ct[h]{r'}{s'}$.
\end{proof}

\begin{thm}
  Consider a pointed type $A$, and let $r,s:\loopspace[2]{A}$. Then there is an identification
  \begin{equation*}
    \ct{r}{s}=\ct{s}{r}
  \end{equation*}
\end{thm}

\begin{proof}
  First we observe that $\ct{r}{s}=\ct[h]{r}{s}$ by the following calculation using the unit laws from \cref{lem:unit-laws-horizontal-concat} and the interchange law from \cref{lem:interchange-law}:
  \begin{align*}
    \ct{r}{s} & = \ct{(\ct[h]{r}{\refl{\refl{}}})}{(\ct[h]{\refl{\refl{}}}{s})} \\
              & = \ct[h]{(\ct{r}{\refl{\refl{}}})}{(\ct{\refl{\refl{}}}{s})} \\
              & = \ct[h]{r}{s}
  \end{align*}
  Similarly, we observe that $\ct[h]{r}{s}=\ct{s}{r}$ by the following calculation:
  \begin{align*}
    \ct[h]{r}{s} & = \ct[h]{(\ct{\refl{\refl{}}}{r})}{(\ct{s}{\refl{\refl{}}})} \\
                 & = \ct{(\ct[h]{\refl{\refl{}}}{s})}{(\ct[h]{r}{\refl{\refl{}}})} \\
                 & = \ct{s}{r}.
  \end{align*}
  These two calculations combined prove the claim.
\end{proof}

\begin{cor}
For $n\geq 2$, the $n$-th homotopy group of any pointed type is abelian.\index{homotopy group!is abelian for n geq 2@{is abelian for $n\geq 2$}}
\end{cor}

\begin{proof}
  By \cref{prp:homotopy-group-loop-space} it follows that $\pi_n(A)$ is isomorphic to the second homotopy group of some pointed type, for every $n\geq 2$. Therefore it suffices to prove the claim for $\pi_2(A)$ for every pointed type $A$.
  
  Our goal is to show that 
  \begin{equation*}
    \prd{r,s:\pi_2(A)} rs=sr.
  \end{equation*}
  Since we are constructing an identification in a set, we can use the dependent universal property of $0$-truncation on both $r$ and $s$, stated in \cref{thm:set-truncation}. Therefore it suffices to show that
  \begin{equation*}
    \prd{r,s:\loopspace[2]{A}} \eta(r)\eta(s)=\eta(s)\eta(r).
  \end{equation*}
  The claim now follows, because
  \begin{equation*}
    \eta(r)\eta(s)=\eta(\ct{r}{s})=\eta(\ct{s}{r})=\eta(s)\eta(r).\qedhere
  \end{equation*}
\end{proof}
\index{homotopy group|)}
\index{Eckmann-Hilton argument|)}

\subsection{Concrete versus abstract groups in univalent mathematics}

In univalent mathematics there is another exciting perspective on group theory. We won't be able to go in full details here, but we can sketch some of key ideas. To learn more about this beautiful univalent perspective on group theory, I recommend the forthcoming \emph{Symmetry} book \cite{symmetrybook}.

We saw in \cref{eg:loop-spaces} that for every pointed connected $1$-type $X$ we obtain a group with underlying type $\loopspace{X}$. All groups can be constructed in this way. In fact, for every group $G$ in $\UU$ the type
\begin{equation*}
  \sm{B:\mathsf{Pointed\usc{}Connected\usc{}}1\mathsf{\usc{}Type}_\UU}G\cong\loopspace{B}
\end{equation*}
of pointed connected $1$-types $B$ equipped with a group isomorphism from $G$ to $\loopspace{B}$ is contractible. We write $BG$ for the unique pointed connected $1$-type whose loop space is isomorphic to $G$. The pointed type $BG$ is also called the \define{delooping}\index{delooping|textbf}\index{group!delooping|textbf} of $G$, or the \define{classifying type}\index{classifying type|textbf}\index{group!classifying type|textbf} of $G$. The fact that the above type is contractible is of course heavily reliant on the univalence axiom.

\begin{eg}
  We have already seen that
  \begin{equation*}
    S_n\cong\loopspace{\BS_n},
  \end{equation*}
  i.e., that the loop space of the type of all finite types of cardinality $n$ is equivalent to the symmetric group $S_n$. The type $\BS_n$ is of course a pointed connected $1$-type, so $BS_n$ is indeed the classifying type of the symmetric group $S_n$.\index{BS n@{$\BS_n$}!is classifying type of symmetric group}
\end{eg}

Since the map
\begin{equation*}
  \loopspacesym:\mathsf{Pointed\usc{}Connected\usc{}}1\mathsf{\usc{}Type}_\UU\to\Grp_\UU
\end{equation*}
is an equivalence, we obtain two perspectives on the type of all groups. The elements of the type $\Grp_\UU$ are groups according to the traditional definition of groups. We call such groups \define{abstract groups}\index{group!abstract group|textbf}\index{abstract group|textbf}. On the other hand, pointed connected $1$-types $B$ are \define{concrete groups}\index{concrete group|textbf}\index{group!concrete group|textbf} in the sense that the contain an object $\ast:B$, and the group $B$ represents is the group of self-identifications (i.e., symmetries) of the base point $\ast:B$. Thus we see that when we present a group as a pointed connected $1$-type, then we \emph{concretely} manifest that group as the group of symmetries of some object.

We can also bring group homomorphisms into the mix: for every group homomorphism $f:G\to H$ the type of pointed maps $b:BG\to_\ast BH$ equipped with a homotopy witnessing 
\begin{equation*}
  \begin{tikzcd}
    G \arrow[d,swap,"\cong"] \arrow[r,"f"] & H \arrow[d,"\cong"] \\
    \loopspace{BG} \arrow[r,swap,"\loopspace{b}"] & \loopspace{BH}
  \end{tikzcd}
\end{equation*}
commutes is contractible. In other words, every group homomorphism $f:G\to H$ has a unique \define{delooping} $Bf:BG\to BH$.

We can do all of group theory in this way. For example, traditionally a $G$-set is defined to be a set $X$ equipped with a group homomorphism $G\to\Aut(X)$. That is, the type of \define{abstract $G$-sets}\index{abstract G-set@{abstract $G$-set}|textbf}\index{group!abstract G-set@{abstract $G$-set}|textbf} is defined to be\index{G-Set@{$G\mathsf{\usc{}}\Set_\UU$}|see {abstract $G$-set}}\index{G-Set@{$G\mathsf{\usc{}}\Set_\UU$}|textbf}
\begin{equation*}
  G\mathsf{\usc{}}\Set_\UU\defeq\sm{X:\Set_\UU}\hom(G,\Aut(X)).
\end{equation*}
However, this definition is equivalent to family $X:BG\to\Set_\UU$ of sets indexed by the classifying type $BG$. Therefore we define \define{concrete $G$-sets}\index{concrete G-set@{concrete $G$-set}|textbf}\index{group!concrete G-set@{concrete $G$-set}|textbf} to be type families $X:BG\to\Set_\UU$. Given a concrete $G$-set $X:BG\to\Set_\UU$, the set being acted upon is the set $X(\ast)$, and the action of $G$ on $X(\ast)$ is given by transport, since the elements of $G$ are equivalent to loops in $BG$.

The type of \define{orbits}\index{orbit}\index{concrete G-set@{concrete $G$-set}!orbit|textbf} of a concrete $G$-set $X:BG\to\Set_\UU$ can then be defined as
\begin{equation*}
  X/G\defeq \sm{u:BG}X(u)
\end{equation*}
and the type of \define{fixed points}\index{fixed point!of a concrete G-set@{of a concrete $G$-set}|textbf}\index{concrete G-set@{concrete $G$-set}!fixed point|textbf} of $X$ can be defined as
\begin{equation*}
  X_G\defeq \prd{u:BG}X(u).
\end{equation*}
To see that these definitions make sense, note that the fiber inclusion $X(\ast)\to X/G$ maps each element in the $G$-set $X$ to its orbit. The fiber inclusion is surjective by \cref{ex:is-surjective-fiber-inclusion}, and it maps two elements $x,y:X(\ast)$ to the same orbit precisely when there is a group element $g$ such that $gx=y$. Similarly, for the type of fixed points notice that each $x:X_G$ determines an element $x_\ast:X(\ast)$, which comes equipped with an identification
\begin{equation*}
  \apd{x}{g}:gx_\ast=x_\ast
\end{equation*}
since the group action of $G$ on $X$ is given by transport.

Also, notice that a subgroup $H$ of $G$ determines an inclusion homomorphism $i:H\to G$, and this inclusion function corresponds uniquely to a pointed map $Bi:BG\to BH$. Since $\loopspace{Bi}$ is an embedding, we note that $Bi$ must be a $0$-truncated map. Therefore, a concrete subgroup of a concrete group $BG$ is defined to be a concrete $G$-set $X$ such that the type of orbits is connected. Such concrete $G$-sets are called \define{transitive}\index{transitive concrete G-set@{transitive concrete $G$-set}|textbf}\index{concrete G-set@{concrete $G$-set}!transitive concrete G-set@{transitive concrete $G$-set}|textbf}.

Dually, we say that a concrete $G$-set $X$ is \define{free}\index{concrete G-set@{concrete $G$-set}!free concrete G-set@{free concrete $G$-set}|textbf}\index{free concrete G-set@{free concrete $G$-set}|textbf} if the type of orbits $X/G$ is a set. To see that this definition makes sense, we use the following generalization of the fundamental theorem of identity types:

\begin{thm}\label{thm:truncated-fundamental}
  Consider a connected type $A$ equipped with an element $a:A$, and consider a family of types $B(x)$ indexed by $x:A$. Then the following are equivalent:\index{fundamental theorem of identity types!generalization to truncated maps}\index{generalized fundamental theorem of identity types!truncated maps}
  \begin{enumerate}
  \item Every family of maps
    \begin{equation*}
      f:\prd{x:A}(a=x)\to B(x)
    \end{equation*}
    is a family of $k$-truncated maps.
  \item The total space
    \begin{equation*}
      \sm{x:A}B(x)
    \end{equation*}
    is $(k+1)$-truncated.
  \end{enumerate}
\end{thm}

\begin{proof}
  Recall from \cref{ex:is-trunc-const} that the total space $\sm{x:A}B(x)$ is $(k+1)$-truncated if and only if the base point inclusion
  \begin{equation*}
    (x,y):\unit\to\sm{x:A}B(x)
  \end{equation*}
  is $k$-truncated for every $(x,y):\sm{x:A}B(x)$. Since the type $A$ is assumed to be connected, this is equivalent to the condition that every base point inclusion of the form
  \begin{equation*}
    (a,y):\unit\to\sm{x:A}B(x)
  \end{equation*}
  is $k$-truncated. Base point inclusions of this form are homotopic to $\tot{f}$, where
  \begin{equation*}
    f:\prd{x:A}(a=x)\to B(x)
  \end{equation*}
  is given by $f(a,\refl{})\defeq y$. The condition that $\tot{f}$ is $k$-truncated is by \cref{lem:fib_total} equivalent to the condition that $f$ is a family of $k$-truncated maps. Furthermore, every family of maps $f:\prd{x:A}(a=x)\to B(x)$ is of the above form by the type theoretic Yoneda lemma (\cref{thm:yoneda}), completing the proof. 
\end{proof}

By the previous theorem it follows that if the type of orbits of a concrete $G$-set $X$ is a set, then the map $g\mapsto gx$ must be an embedding for every $x:X(\ast)$. In other words, the action of $G$ on $X$ is free.

\begin{rmk}
  \cref{thm:truncated-fundamental} can be generalized further. We include this generalization in \cref{ex:connected-fundamental}.
\end{rmk}

\begin{eg}
  Consider two sets $A$ and $B$, and a universe $\UU$ containing both of them. Then the automorphism group $\Aut(B)$ acts on the decidable embeddings $B\demb A$ by precomposition. Its type of orbits is the binomial type\index{binomial type}
  \begin{equation*}
    \dbinomtype{A}{B}\defeq\sm{X:\UU_B}X\demb A,
  \end{equation*}
  which we introduced in \cref{defn:binomial-type}. By \cref{prp:equiv-binom-type} it follows that $\dbinomtype{A}{B}$ is a set, so the action of $\Aut(B)$ on $B\demb A$ is free. Note that we didn't need to assume that $A$ and $B$ are sets: the action of $\Aut(B)$ on $B\demb A$ is always free.

  Similarly, we have an action of the automorphism group $\Aut(B)$ on the surjective maps $A\twoheadrightarrow B$ by postcomposition. Its type of orbits is the stirling type of the second kind\index{Stirling type of the second kind}
  \begin{equation*}
    \stirling{A}{B}\defeq\sm{X:\UU_B}A\twoheadrightarrow X,
  \end{equation*}
  which we introduced in \cref{ex:stirling-type-of-the-second-kind}. Assuming that $B$ is a set, it was shown in \cref{ex:stirling-type-of-the-second-kind} that $\stirling{A}{B}$ is a set. In other words, the action of $\Aut(B)$ on $A\twoheadrightarrow B$ is free.
\end{eg}

\begin{eg}
  In \cref{ex:prime} we introduced the type\index{prime number}
  \begin{equation*}
    \tilde{D}_n\defeq \sm{X:\BS_2}\sm{Y:X\to\F}\Big(\Fin{n}\simeq\prd{x:X}Y(x)\Big).
  \end{equation*}
  Notice that this type is the type of orbits of the $\Z/2$-set $D_n$ given by
  \begin{equation*}
    D_n(X)\defeq \sm{Y:X\to\F}\Big(\Fin{n}\simeq\prd{x:X}Y(x)\Big).
  \end{equation*}
  The fact that this is a family of sets is a nice exercise. Note that there is a surjective morphism of $\Z/2$-sets from $D_n(\Fin{2})$ to the $\Z/2$ set of divisors of $n$, where the action is given by $d\mapsto n/d$. The concrete $\Z/2$-action $D_n$ is transitive precisely when $n$ is either $1$ or a prime, and it is is free precisely when $n$ is not a square. Combining these two observations, we see that $n$ is prime if and only if this action is both transitive and free. In other words, $n$ is prime if and only if the type $\tilde{D}_n$ of orbits is contractible.
\end{eg}

$G$-sets which are both transitive and free are very special. Such $G$-sets are called \define{$G$-torsors}\index{torsor|textbf}\index{group!torsor|textbf}. Note that a $G$-set $X$ is a $G$-torsor if and only if the type of orbits $X/G$ is contractible. By the fundamental theorem of identity types, this implies that the family of maps
\begin{equation*}
  \prd{v:BG}(u=v)\to X(v)
\end{equation*}
is a family of equivalences, where $(u,x)$ is the center of contraction of $X/G$. It follows that a concrete $G$-set $X:BG\to\Set_\UU$ is a $G$-torsor if and only if it is in the image of
\begin{equation*}
  \idtypevar{} : BG\to (BG\to \Set_\UU).
\end{equation*}
However, we know from \cref{ex:idtype-is-emb} that this map is an embedding, so it follows that the type of concrete $G$-torsors is equivalent to $BG$. On the other hand, the type of concrete $G$-torsors is equivalent to the type of abstract $G$-torsors. This suggests that the classifying type $BG$ of any group $G$ can be constructed as the type of abstract $G$-torsors, and this is indeed one way to construct the classifying type of a group $G$.

\begin{exercises}
  \exitem Consider a set $X$ equipped with an associative binary operation $\mu:X\to (X\to X)$, and suppose that
  \begin{enumerate}
  \item The type $X$ is inhabited, i.e., $\brck{X}$ holds.
  \item The maps $\mu(x,\blank)$ and $\mu(\blank,y)$ are equivalences, for each $x,y:X$.
  \end{enumerate}
  Show that $X$ is a group.\index{group}
  \exitem Let $f:\hom(G,H)$ be a group homomorphism\index{homomorphism!of groups}. Show that $f$ preserves units and inverses, i.e., show that\index{group homomorphism!preserves units and inverses}
  \begin{align*}
    f(e_G) & = e_H \\
    f(x^{-1}) & = f(x)^{-1}.
  \end{align*}
  \exitem \label{ex:groupop-embedding}
  Consider a group $G$. Show that the function
  \begin{equation*}
    \mu_G:G\to (G\simeq G)
  \end{equation*}
  is an injective group homomorphism.
  \exitem Let $X$ be a set. Show that the map\index{equiv-eq@{$\equiveq$}!is a group isomorphism}
  \begin{equation*}
    \equiveq : (X=X)\to (\eqv{X}{X})
  \end{equation*}
  is a group isomorphism.
  \exitem Consider a group $G$. Show that the map\index{Z@{$\Z$}!is the free group with one generator|textbf}\index{group!free group with one generator|textbf}
  \begin{equation*}
    \Grp(\Z,G)\to G
  \end{equation*}
  given by $h\mapsto h(\oneZ)$, is an equivalence. In other words, the group $\Z$ satisfies the universal property of the \define{free group on one generator}\index{free group with one generator|textbf}.
  \exitem Give a direct proof and a proof using the univalence axiom of the fact that all semigroup isomorphisms between unital semigroups preserve the unit. Conclude that isomorphic monoids are equal.\index{isomorphism!of semigroups!preserves unit}\index{characterization of identity type!of Monoid@{of $\monoid_\UU$}}\index{identity type!of Monoid@{of $\monoid_\UU$}}
  \exitem \label{ex:dihedral-group}Consider an abelian group $A$, and let $D_A\defeq A+A$\index{D A@{$D_A$}|see {generalized dihedral group}}\index{D A@{$D_A$}|textbf} be the set equipped with $1\defeq\inl(0)$, the binary operation ${\blank}\cdot{\blank}:D_A\to (D_A\to D_A)$ defined by
    \begin{align*}
    \inl(x)\cdot\inl(y) & \defeq \inl(x+y) \\
    \inl(x)\cdot\inr(y) & \defeq \inr(-x+y) \\
    \inr(x)\cdot\inl(y) & \defeq \inr(x+y) \\
    \inr(x)\cdot\inr(y) & \defeq \inl(-x+y),
  \end{align*}
  and the unary operation $(\blank)^{-1}:D_A\to D_A$ defined by
  \begin{align*}
    \inl(x)^{-1} & \defeq \inl(-x) \\
    \inr(x)^{-1} & \defeq \inr(x).
  \end{align*}
  Show that $D_A$ equipped with these operations is a group. The group $D_A$ is called the \define{generalized dihedral group}\index{generalized dihedral group|textbf}\index{group!generalized dihedral group|textbf} on $A$. The (ordinary) \define{dihedral group}\index{dihedral group|textbf}\index{group!dihedral group|textbf} $D_k$\index{D k@{$D_k$}|see {dihedral group}}\index{D k@{$D_k$}|textbf} is defined to be $D_k\defeq D_{\Z/k}$.
  \exitem Recall that a \define{subgroup}\index{subgroup|textbf}\index{group!subgroup|textbf} of a group $G$ in $\UU$ consists of a subtype
  \begin{equation*}
    P:G\to\prop_\UU
  \end{equation*}
  such that $P$ contains the unit and is closed under the group operation and under inverses.
  \begin{subexenum}
  \item Consider a proposition $P$, and let $N_P$ be the subtype of $\Z/2$ given by
    \begin{equation*}
      N_P(x)\defeq (x=0)\vee P.
    \end{equation*}
    Show that $N_P$ is a subgroup of $\Z/2$.
  \item Show that the map $P\mapsto N_P$ is an embedding
    \begin{equation*}
      \prop_\UU\hookrightarrow\subgroup_\UU(\Z/2).
    \end{equation*}
  \end{subexenum}
  \exitem Recall that a \define{normal subgroup}\index{normal subgroup|textbf}\index{group!normal subgroup|textbf} $H$ of a group $G$ is a subgroup of $G$ such that $xyx^{-1}$ is in $H$ for every $y:H$ and $x:G$. Show that the type of normal subgroups of $G$ in $\UU$ is equivalent to the type
  \begin{equation*}
    \sm{H:\Grp_\UU}\sm{f:\hom(G,H)}\issurj(f).
  \end{equation*}
  \exitem \label{ex:commutative-binary-operations}For any type $A$, we define the type of \define{commutative binary operations}\index{commutative binary operation|textbf} on $A$ to be
  \begin{equation*}
    \Big(\sm{X:\BS_2}A^X\Big)\to A.
  \end{equation*}
  If $A$ is a set, show that the map
  \begin{equation*}
    \Big(\Big(\sm{X:\BS_2}A^X\Big)\to A\Big)\to\Big(\sm{f:A\to (A\to A)}\prd{x,y:A}f(x,y)=f(y,x)\Big)
  \end{equation*}
  given by $h\mapsto \lam{x}\lam{y}h(\Fin{2},(x,y))$ is an equivalence. In other words, show that every commutative operation $f:A\to(A\to A)$ extends uniquely along the map $f\mapsto(\Fin{2},f)$ as in the diagram
  \begin{equation*}
    \begin{tikzcd}
      A^{\Fin{2}} \arrow[dr,"\mu"] \arrow[d,swap,"f\mapsto{(\Fin{2},f)}"] \\
      \sm{X:\BS_2}A^X \arrow[r,dashed] & A.
    \end{tikzcd}
  \end{equation*}
  Give an informal explanation of this fact in terms fixed points of the concrete $\Z/2$-action on the set of binary operations $A\to (A\to A)$.
  \exitem Consider a commutative monoid $M$. Define an operation
  \begin{equation*}
    \prd{X:\mathbb{F}} M^X\to M
  \end{equation*}
  that extends the (binary) monoid operation to the finite unordered $n$-tuples of elements in $M$.
  \exitem Show that the type of $3$-element groups is equivalent to the type of $2$-element types.
  \exitem Show that the number of connected components in the type of all groups of order $n$ is as follows, for $n\leq 8$:
  \begin{center}
    \begin{tabular}{rllllllll}
      \emph{order:} & 1 & 2 & 3 & 4 & 5 & 6 & 7 & 8 \\
      \midrule
      \emph{number of groups:} & 1 & 1 & 1 & 2 & 1 & 2 & 1 & 5
    \end{tabular}
  \end{center}
  \exitem \label{ex:connected-fundamental}Consider a subtype\index{fundamental theorem of identity types!full generalization}\index{generalized fundamental theorem of identity types}
  \begin{equation*}
    P:\UU\to\prop_\VV
  \end{equation*}
  of a universe $\UU$. We say that a type $A:\UU$ is a \define{$P$-type}\index{P-type@{$P$-type}|textbf} if $P(A)$ holds, we say that a map $f:A\to B$ is a \define{$P$-map}\index{P-map@{$P$-map}|textbf} if its fibers are $P$-types, and we say that $A$ is \define{$P$-separated}\index{P-separated type@{$P$-separated type}|textbf} if its identity types are $P$-types.
  
  Now consider a connected type $A:\UU$ equipped with an element $a:A$, and consider a family of types $B(x):\UU$ indexed by $x:A$. Show that the following are equivalent:
  \begin{enumerate}
  \item Every family of maps
    \begin{equation*}
      f:\prd{x:A}(a=x)\to B(x)
    \end{equation*}
    is a family of $P$-maps.
  \item The total space
    \begin{equation*}
      \sm{x:A}B(x)
    \end{equation*}
    is $P$-separated.
  \end{enumerate}
  For readers familiar with the notion of $k$-connectedness: Conclude that every $f:\prd{x:A}(a=x)\to B(x)$ is a family of $k$-connected maps if and only if $\sm{x:A}B(x)$ is a $(k+1)$-connected type.
  \exitem Consider a group $G$ in a universe $\UU$ and a pointed connected $1$-type $B$. In analogy with \cref{thm:quotient_up}, show that the following are equivalent:
  \begin{enumerate}
  \item The pointed connected $1$-type $B$ comes equipped with a group homomorphism\index{universal property!of the classifying type of a group|textbf}\index{classifying type!universal property|textbf}
    \begin{equation*}
      \varphi:G \to \loopspace{B}
    \end{equation*}
    and for every pointed connected $1$-type $C$ that comes equipped with a group homomorphism $\psi:G\to \loopspace{C}$ there is a unique pointed map $f:B\to_\ast C$ equipped with a homotopy witnessing that the triangle
    \begin{equation*}
      \begin{tikzcd}[column sep=tiny]
        & G \arrow[dl,swap,"\varphi"] \arrow[dr,"\psi"] \\
        \loopspace{B} \arrow[rr,swap,"\loopspace{f}"] & & \loopspace{C}
      \end{tikzcd}
    \end{equation*}
    commutes.
  \item The pointed connected $1$-type $B$ comes equipped with a group isomorphism
    \begin{equation*}
      \varphi:G\cong \loopspace{B}.
    \end{equation*}
  \item There is an embedding $i:B\hookrightarrow G\mathsf{\usc{}}\Set_\UU$ such that the triangle
    \begin{equation*}
      \begin{tikzcd}[column sep=tiny]
        \unit \arrow[rr] \arrow[dr,swap,"\mathsf{Pr}_G"] & & B \arrow[dl,"i"] \\
        & G\mathsf{\usc{}}\Set_\UU
      \end{tikzcd}
    \end{equation*}
    commutes, where $\mathsf{Pr}_G$ is the \define{principal $G$-set}\index{principal G-set@{principal $G$-set}|textbf}\index{group!principal G-set@{principal $G$-set}|textbf}, i.e., $G$ acting on itself from the left.
  \end{enumerate}
  \exitem Consider a group $G$ and a pointed connected $1$-type $B$ equipped with a group isomorphism
  \begin{equation*}
    \varphi:G\cong \loopspace{B}.
  \end{equation*}
  \begin{subexenum}
  \item Show that the map
    \begin{equation*}
      \ev_\ast:(B\to\Set_\UU)\to \sm{X:\Set_\UU}\hom(G,\Aut(X))
    \end{equation*}
    sending concrete $G$-sets\index{concrete G-set@{concrete $G$-set}}\index{group!concrete G-set@{concrete $G$-set}} to abstract $G$-sets\index{abstract G-set@{abstract $G$-set}}\index{group!abstract G-set@{abstract $G$-set}} defined by
    \begin{equation*}
      \ev_\ast(X)\defeq (X(\ast),g\mapsto \tr_X(\varphi(g)))
    \end{equation*}
    is an equivalence. In the remainder of this exercise we will write $gx$ for $\tr_X(\varphi(g),x)$.
  \item Show that the type $X_G\defeq\prd{u:BG}X(u)$ of concrete fixed points of $X$\index{fixed point!of a concrete G-set@{of a concrete $G$-set}}\index{concrete G-set@{concrete $G$-set}!fixed point} is equivalent to the type
    \begin{equation*}
      \sm{x:X(\ast)}gx=x
    \end{equation*}
    of \define{fixed points} of the abstract $G$-set $\ev_\ast(X)$.\index{fixed point!of an abstract G-set@{of an abstract $G$-set}|textbf}\index{abstract G-set@{abstract $G$-set}!fixed point|textbf}
  \item Show that the type $X/G$ of orbits\index{concrete G-set@{concrete $G$-set}!orbit}\index{orbit} of $X$ is connected if and only if the abstract $G$-set $\ev_\ast(X)$ is transitive in the sense that\index{abstract G-set@{abstract $G$-set}!transitive abstract G-set@{transitive abstract $G$-set}|textbf}\index{transitive abstract G-set@{transitive abstract $G$-set}|textbf}
    \begin{equation*}
      \forall_{(x:X(\ast))}\issurj(g\mapsto gx)
    \end{equation*}
  \item Show that the type $X/G$ of orbits\index{concrete G-set@{concrete $G$-set}!orbit}\index{orbit} of $X$ is a set if and only if the abstract $G$-set $\ev_\ast(X)$ is free\index{abstract G-set@{abstract $G$-set}!free abstract G-set@{free abstract $G$-set}|textbf}\index{free abstract G-set@{free abstract $G$-set}|textbf} in the sense that
    \begin{equation*}
      \forall_{(x:X(\ast))}\isinj(g\mapsto gx).
    \end{equation*}
  \item Show that the type of abstract $G$-torsors\index{torsor}\index{group!torsor} is equivalent to the type of families $X:B\to\Set_\UU$ with contractible total space.\index{torsor}\index{group!torsor}
  \end{subexenum}
  \exitem (Buchholtz) Consider a group $G$ with classifying type $BG$ equipped with a group isomorphism
  \begin{equation*}
    \varphi:G\cong \loopspace{BG}.
  \end{equation*}
  Define the $G$-type $\concretesubgroup_\UU(G) : BG\to\UU$ of \define{concrete subgroups} of $G$ by\index{concrete subgroup|textbf}\index{group!concrete subgroup|textbf}\index{Concrete-Subgroup@{$\concretesubgroup_\UU(G,u)$}|textbf}
  \begin{equation*}
    \concretesubgroup_\UU(G,u)\defeq \sum_{(X:BG\to\Set_\UU)}\sum_{(x:X(u))}\isconn(X/G).
  \end{equation*}
  \begin{subexenum}
  \item Construct an equivalence\index{subgroup}\index{group!subgroup}
    \begin{equation*}
      \concretesubgroup_\UU(G,\ast)\simeq\subgroup_\UU(G).
    \end{equation*}
  \item Show that $G$ acts on $\concretesubgroup_\UU(G,\ast)$ by conjugation\index{conjugation|textbf}\index{group!conjugation|textbf}, i.e., show that for any $g:G$ we have a commuting square
    \begin{equation*}
      \begin{tikzcd}[column sep=1.6em]
        \concretesubgroup_\UU(G,\ast) \arrow[r,"g"] \arrow[d,swap,"\simeq"] & \concretesubgroup_\UU(G,\ast) \arrow[d,"\simeq"] \\
        \subgroup_\UU(G) \arrow[r,swap,"H\mapsto\{ghg^{-1}\mid h\in H\}"] & \subgroup_\UU(G)
      \end{tikzcd}
    \end{equation*}
  \item Conclude that the type of normal subgroups of a group $G$\index{normal subgroup}\index{group!normal subgroup} is equivalent to the type of \define{concrete normal subgroups}\index{concrete normal subgroup|textbf}\index{group!concrete normal subgroup|textbf}
    \begin{equation*}
      \prd{u:BG}\concretesubgroup_\UU(G,u).
    \end{equation*}
  \item Show that the type of normal subgroups of a group $G$ is also equivalent to the type
    \begin{equation*}
      \sm{BH:\concretegroup_\UU}\sm{f:BG\to_\ast BH}\isconn(f)
    \end{equation*}
  \end{subexenum}
\end{exercises}
\index{group|)}

%%% Local Variables:
%%% mode: latex
%%% TeX-master: "hott-intro"
%%% End:

\section{General inductive types}\label{sec:w-types}

\index{inductive type|(}
\index{W-type|(}

Most inductive types we have seen in this book have a finite number of constructors with finite arities. For example, the type $\N$ has two constructors: one constant $\zeroN$ and one unary constructor $\succN$. However, there is no objection to having an nonfinite amount of constructors, possibly with nonfinite arities. W-types are general inductive types that have a \emph{type} of constructors, whose arities are \emph{types}. W-types are therefore specified by a type $A$ of \emph{symbols} for the constructors, and a type family $B$ over $A$ specifying the arities of the constructors that the symbols represent.

An example of a W-type is the type of finitely branching rooted trees. This inductive type has a constructor with arity $X$ for each finite type $X$. In other words, a finitely branching rooted tree is obtained by attaching a finitely many finitely branching rooted trees to a root. The root itself is therefore a finitely branching tree, obtained from the $0$-ary constructor corresponding to the empty type, and if we have any finite family finitely branching rooted trees, we can combine them all into one finitely branching rooted tree by attaching them to a new root.

\subsection{The type of well-founded trees}

\begin{defn}
  Consider a type family $B$ over $A$. The \define{W-type}\index{W-type|textbf} $\W(A,B)$\index{W(A,B)@{$\W(A,B)$}|see {W-type}}\index{W(A,B)@{$W(A,B)$}|textbf} is defined as the inductive type with constructor\index{tree@{$\collect$}|textbf}\index{W-type!tree@{$\collect$}|textbf}\index{inductive type!W-type}
  \begin{equation*}
    \collect : \prd{x:A} (B(x)\to \W(A,B))\to \W(A,B).
  \end{equation*}
  The induction principle\index{W-type!induction principle|textbf}\index{induction principle!of W-types|textbf} of the W-type $\W(A,B)$ asserts that, for any type family $P$ over $\W(A,B)$, any dependent function
  \begin{equation*}
    h : \prd{x:A}\prd{\alpha:B(x)\to \W(A,B)} \Big(\prd{y:B(x)}P(\alpha(x))\Big)\to P(\collect(x,\alpha))
  \end{equation*}
  determines a dependent function\index{ind_W@{$\indW$}|textbf}\index{W-type!ind_W@{$\indW$}|textbf}
  \begin{equation*}
    \indW(h):\prd{x:\W(A,B)}P(x)
  \end{equation*}
  that satisfies the judgmental equality\index{computation rules!for W-types|textbf}\index{W-type!computation rule|textbf}
  \begin{equation*}
    \indW(h,\collect(x,\alpha))\jdeq h(x,\alpha,\lam{y}\indW(h,\alpha(y))).
  \end{equation*}
  The elements of W-types are called \define{(well-founded) trees}\index{well-founded trees|textbf}.
\end{defn}

\begin{rmk}
  Some authors write $\mathsf{sup}$ for the constructor of a W-type. The intuition that $\collect(a,\alpha)$ is a supremum of the family of elements $\alpha(y)$ indexed by $y:B(a)$ is, however, somewhat misleading, because $\collect(a,\alpha)$ does not satisfy the defining properties of a supremum.
\end{rmk}

\begin{rmk}
  When we define a dependent function
  \begin{equation*}
    f:\prd{x:\W(A,B)}P(x)
  \end{equation*}
  via the induction principle of W-types, we will often display that definition by pattern matching\index{pattern matching!for W-types}\index{W-type!pattern matching}. Such definitions are then displayed as
  \begin{equation*}
    f(\collect(x,\alpha))\defeq h(x,\alpha,\lam{y}f(\alpha(y))),
  \end{equation*}
  which contains all the information to carry out the construction via the induction principle of W-types. The advantage of definitions by pattern matching is that they directly display the defining judgmental equality the function being defined.
\end{rmk}

\begin{rmk}\label{rmk:constant-W}
  For any $x:A$, the function
  \begin{equation*}
    \collect(x):(B(x)\to\W(A,B))\to\W(A,B)
  \end{equation*}
  takes a family of elements $\alpha(y):\W(A,B)$ indexed by $y:B(x)$ and collects them into an element $\collect(x,\alpha):\W(A,B)$. Since the element $\collect(x,\alpha)$ has been constructed out of a family $\alpha(y)$ of elements of $\W(A,B)$ indexed by $y:B(x)$, we say that the type $B(x)$ is the \define{arity}\index{arity of constructor W-type|textbf}\index{W-type!arity of constructor|textbf} of $\collect(x,\alpha)$. In other words, there is a function\index{arity@{$\arity$}|textbf}\index{W-type!arity@{$\arity$}|textbf}
  \begin{equation*}
    \arity : \W(A,B)\to\UU
  \end{equation*}
  given by $\arity(\collect(x,\alpha))\defeq B(x)$. The element $x:A$ is the \define{symbol}\index{symbol of a constructor of a W-type|textbf}\index{W-type!symbol of a constructor|textbf} of the operation $\collect(x):(B(x)\to\W(A,B))\to\W(A,B)$. Note that there might be many different symbols $x,y:A$ for which the operations $\collect(x)$ and $\collect(y)$ have equivalent arities, i.e., for which $B(x)\simeq B(y)$.

  Furthermore, the \define{components}\index{component of an element pf a W-type|textbf}\index{W-type!component of an element|textbf} of $\collect(x,\alpha)$ are the elements $\alpha(y):\W(A,B)$ indexed by $y:B(x)$. In other words, we have
  \begin{align*}
    \component & : \prd{w:\W(A,B)} \arity(w) \to \W(A,B),
  \end{align*}
  given by $\component(\collect(x,\alpha))\defeq\alpha$.
  
  In the special case where $B(x)$ is empty, there is exactly one family of elements $\alpha(y):\W(A,B)$ indexed by $y:B(x)$. Therefore, it follows that any $x:A$ such that $B(x)$ is empty induces a constant in the W-type $\W(A,B)$. More precisely, if we are given a map $h:B(x)\to \emptyt$, then we can define the \define{constant}\index{constant element in a W-type|textbf}\index{W-type!constant element|textbf}
  \begin{equation*}
    c_x(h)\defeq \collect(x,\exfalso\circ h).
  \end{equation*}
  The elements of $w:\W(A,B)$ for which the type $B(\arity(w))$ is empty are called the \define{constants} of $\W(A,B)$. In other words, the predicate\index{is-constant@{$\isconstantW$}|textbf}\index{W-type!is-constant@{$\isconstantW$}|textbf}
  \begin{equation*}
    \isconstantW : \W(A,B)\to\prop_\UU
  \end{equation*}
  is defined by $\isconstantW(w)\defeq\isempty(B(\arity(w)))$.
  
  On the other hand, if each type $B(x)$ is inhabited, then there are no such constants and we will see in the following proposition that the W-type $\W(A,B)$ is empty in this case.  
\end{rmk}

\begin{prp}\label{prp:is-empty-W}
  Consider a family $B$ of types over $A$. Then the following are equivalent:
  \begin{enumerate}
  \item For each $x:A$, the type $B(x)$ is nonempty.
  \item The $W$-type $\W(A,B)$ is empty.\index{is empty!W-type}\index{W-type!is empty}
  \end{enumerate}
  In particular, if each $B(x)$ is inhabited, then $\W(A,B)$ is empty.
\end{prp}

\begin{proof}
  To prove that (i) implies (ii), assume that $\neg\neg(B(x))$ holds for each $x:A$. Our goal is to construct a function $f:\W(A,B)\to \emptyt$. By the induction principle of W-types it suffices to construct a function of type
  \begin{equation*}
    \prd{x:A}\prd{\alpha:B(x)\to\W(A,B)}\Big(\prd{y:B(x)}\emptyt\Big)\to\emptyt.
  \end{equation*}
  This type is judgmentally equal to the type
  \begin{equation*}
    \prd{x:A}\prd{\alpha:B(x)\to\W(A,B)}\neg\neg(B(x)),
  \end{equation*}
  so we obtain the desired function from the assumption that $\neg\neg(B(x))$ holds for every $x:A$.

  To prove that (ii) implies (i), suppose that $\W(A,B)$ is empty and let $x:A$. To show that $\neg\neg(B(x))$ holds, assume that $\neg(B(x))$ holds. In other words, assume a function $h:B(x)\to\emptyt$. Then we have the constant element $c_x(h):\W(A,B)$. This is impossible, since $\W(A,B)$ was assumed to be empty.
\end{proof}

\begin{eg}\label{eg:Nat-W}
  Consider the type family $P$ over $\bool$ given by
  \begin{equation*}
    P(\bfalse) \defeq \emptyt \qquad\text{and}\qquad P(\btrue) \defeq \unit.
  \end{equation*}
  We claim that the W-type $N\defeq \W(\bool,P)$\index{natural numbers!as W-type}\index{W-type!natural numbers} is equivalent to $\N$. The idea is that the constructor $\collect$ of $\W(\bool,P)$ splits into one nullary constructor with symbol $\bfalse$ and arity $P(\bfalse)\jdeq\emptyt$, and one unary constructor with symbol $\btrue$ and arity $P(\btrue)\jdeq\unit$.

  More formally, we define the zero element $z:N$ and the successor function $s:N\to N$ by
  \begin{equation*}
    z\defeq \collect(\bfalse,\exfalso) \qquad\text{and}\qquad s(x)\defeq \collect(\btrue,\const_x).
  \end{equation*}
  Thus, we obtain a function $f:\N\to N$ that satisfies $f(\zeroN)\jdeq z$ and $f(\succN(n))\jdeq s(f(n))$. It's inverse $g:N\to \N$ is defined via the induction principle of W-types by
  \begin{align*}
    g(\collect(\bfalse,\alpha)) & \defeq \zeroN \\
    g(\collect(\btrue,\alpha)) & \defeq \succN(g(\alpha(\ttt))).
  \end{align*}
  It is immediate from these definitions that $g(f(n))=n$ for all $n:\N$. It remains to construct an identification $p(x):f(g(x))=x$ for all $x:N$. Such an identification is constructed inductively. First, there is an identification
  \begin{equation*}
    p(\collect(\bfalse,\alpha)) : \collect(\bfalse,\exfalso)=\collect(\bfalse,\alpha)
  \end{equation*}
  by the fact that $\exfalso=\alpha$ for any $\alpha:\emptyt\to N$. Second, there is an identification
  \begin{equation*}
    p(\collect(\btrue,\alpha)) : \collect(\btrue,\const_{\alpha(\ttt)})=\collect(\btrue,\alpha)
  \end{equation*}
  by the fact that $\const_{\alpha(\ttt)}=\alpha$ for any map $\alpha:\unit\to N$. This completes the construction of the equivalence $\N\simeq N$.
\end{eg}

\begin{eg}\label{eg:planar-binary-tree-W}
  Consider the type family $B$ over $\bool$ given by
  \begin{equation*}
    B(\bfalse) \defeq \emptyt \qquad\text{and}\qquad B(\btrue) \defeq \bool.
  \end{equation*}
  Then the W-type $\W(\bool,B)$ is equivalent to the type of \define{oriented binary rooted trees}\index{oriented binary rooted tree|textbf}\index{tree!oriented binary rooted tree|textbf}\index{W-type!oriented binary rooted trees|textbf}, which is the inductive type with constructors\index{oriented binary rooted tree!node@{$\node$}|textbf}\index{oriented binary rooted tree![-,-]@{$[\blank,\blank]$}|textbf}\index{inductive type!oriented binary rooted trees|textbf}
  \begin{align*}
    \node & : \planarBinTree \\
    {[\blank,\blank]} & : \planarBinTree\to (\planarBinTree \to \planarBinTree).
  \end{align*}
  We leave the construction of the equivalence $\planarBinTree\simeq\W(\bool,B)$ as \cref{ex:oriented-bin-tree}. The reason we call the elements of $\planarBinTree$ oriented binary rooted trees is that in a tree of the form $[T_1,T_2]$ we can see by inspection which branch is on the left and which branch is on the right.
\end{eg}

\begin{eg}\label{eg:binary-tree-W}
  Consider the type $A\defeq \unit+\BS_2$, where $\BS_2$ is the type of $2$-element types. We define the family $B$ over $A$ by pattern matching:
  \begin{align*}
    B(\inl(x)) & \defeq \emptyt \\
    B(\inr(X)) & \defeq X.
  \end{align*}
  The type of \define{binary rooted trees}\index{binary rooted tree|textbf}\index{tree!binary rooted tree|textbf}\index{W-type!binary rooted trees|textbf} is the W-type $\W(A,B)$ for this choice of $A$ and $B$. We can also present the type of binary rooted trees as an inductive type with the following constructors:\index{binary rooted tree!node@{$\node$}|textbf}\index{binary rooted tree!bin-tree@{$\collectBinTree$}|textbf}\index{bin-tree@{$\collectBinTree$}|textbf}\index{Bin-Tree@{$\BinTree$}|textbf}\index{inductive type!binary rooted trees|textbf}
  \begin{align*}
    \node & : \BinTree \\
    \collectBinTree & : \prd{X:\BS_2} \BinTree^X\to \BinTree.
  \end{align*}
  There is an important qualitative difference between the type of oriented binary rooted trees and the type of binary rooted trees. Given two distinct oriented binary rooted trees $T_1$ and $T_2$, the two oriented binary rooted trees $[T_1,T_2]$ and $[T_2,T_1]$ will also be distinct. On the other hand, given two binary rooted trees $T_1$ and $T_2$, the binary rooted trees
  \begin{align*}
    & \collectBinTree (\bool,\indbool(T_1,T_2)) \\
    & \collectBinTree(\bool,\indbool(T_2,T_1))
  \end{align*}
  can always be identified. In the terminology of \cref{ex:commutative-binary-operations}, the constructor $\collectBinTree$ of $\BinTree$ is equivalently described as a commutative binary operation on $\BinTree$.
\end{eg}

\begin{eg}\label{eg:finitely-branching-tree-W}
  The W-type $\W(\N,\Fin{})$ is the type of \define{oriented finitely branching rooted trees}\index{tree!oriented finitely branching rooted tree|textbf}\index{oriented finitely branching rooted tree|textbf}\index{W-type!oriented finitely branching rooted trees|textbf}. On the other hand, we define the type of \define{(unoriented) finitely branching rooted trees}\index{tree!finitely branching rooted tree|textbf}\index{finitely branching rooted tree|textbf}\index{W-type!finitely branching rooted trees|textbf} to be the W-type $\W(\F,\mathcal{T})$. The qualitive difference between the types of oriented and unoriented finitely branching rooted trees is similar to the qualitative difference between types of oriented and unoriented binary rooted trees. In the type of oriented finitely branching rooted trees, we record the ordering of the branches while in the type of unoriented finitely branching rooted trees there are identifications between trees that have the same branches up to permutation.
\end{eg}

\subsection{Observational equality of W-types}

\index{observational equality!on W-types|(}
\index{W-type!observational equality|(}

Each element $x:\W(A,B)$ has symbol $\prearity(x):A$ and a family of components $\component(x):B(\prearity(x))\to\W(A,B)$. Therefore, we have a map
\begin{equation*}
  \eta : \W(A,B)\to \sm{x:A}(B(x)\to\W(A,B))
\end{equation*}
given by $\eta(x)\defeq(\prearity(x),\component(x))$.

\begin{prp}\label{prp:algebra-W}
  The map $\eta:\W(A,B)\to\sm{x:A}(B(x)\to\W(A,B))$ is an equivalence.
\end{prp}

\begin{proof}
  We define
  \begin{equation*}
    \varepsilon : \Big(\sm{x:A}(B(x)\to\W(A,B))\Big)\to\W(A,B)
  \end{equation*}
  by $\varepsilon(x,\alpha)\defeq\collect(x,\alpha)$. The fact that $\varepsilon$ is an inverse of $\eta$ follows easily.
\end{proof}

The fact that we have an equivalence
\begin{equation*}
  \W(A,B)\simeq\sm{x:A}(B(x)\to\W(A,B)),
\end{equation*}
suggests a way to characterize the identity type of $\W(A,B)$. Indeed, any equivalence is an embedding, and therefore we also have
\begin{equation*}
  (x=y)\simeq (\eta(x)=\eta(y)).
\end{equation*}
The latter is an identity type in a $\Sigma$-type, which can be characterized as a $\Sigma$-type of identity types. We therefore define the following observational equality relation on $\W(A,B)$.

\begin{defn}
  Suppose $A$ and each $B(x)$ are in $\UU$. We define a binary relation\index{Eq W@{$\EqW$}|textbf}\index{W-type!Eq W@{$\EqW$}|textbf}
  \begin{equation*}
    \EqW : \W(A,B)\to \W(A,B)\to \UU
  \end{equation*}
  recursively by
  \begin{equation*}
    \EqW(\collect(x,\alpha),\collect(y,\beta)) \defeq \sm{p:x=y}\prd{z:B(x)}\,\alpha(z)=\beta(\tr_B(p,z))%\EqW(\alpha(z),\beta(\tr_B(p,z)))
  \end{equation*}
\end{defn}

\begin{thm}\label{thm:EqW}
  The observational equality relation $\EqW$ on $\W(A,B)$ is reflexive, and the canonical map\index{characterization of identity type!of W-types}\index{W-type!characterization of identity type}\index{identity type!of W(A,B)@{of $\W(A,B)$}}
  \begin{equation*}
    (x=y)\to \EqW(x,y)
  \end{equation*}
  is an equivalence for each $x,y:\W(A,B)$. 
\end{thm}

\begin{proof}
  The element $\reflEqW(x):\EqW(x,x)$ is defined recursively as
  \begin{equation*}
    \reflEqW(\collect(x,\alpha))\defeq (\refl{x},\reflhtpy_\alpha).
  \end{equation*}
  This proof of reflexivity induces the canonical map $(x=y)\to\EqW(x,y)$. To show that it is an equivalence for each $x,y:\W(A,B)$, we apply the fundamental theorem of identity types, by which it suffices to show that the type
  \begin{equation*}
    \sm{y:\W(A,B)}\EqW(x,y)
  \end{equation*}
  is contractible for each $x:\W(A,B)$. The center of contraction is the pair $(x,\reflEqW(x))$. For the contraction, we have to construct a function
  \begin{equation*}
    h:\prd{y:\W(A,B)}\prd{p:\EqW(x,y)}\,(x,\reflEqW(x))=(y,p).
  \end{equation*}
  By the induction principle of W-types, it suffices to define
  \begin{equation*}
    h(\collect(y,\beta),(p,H))\defeq (x,(\refl{},\reflhtpy))=(y,(p,H)).
  \end{equation*}
  Here we proceed by identification elimination on $p:x=y$, followed by homotopy induction on the homotopy $H:\alpha\htpy \beta$. Thus, it suffices to construct an identification
  \begin{equation*}
    (x,(\refl{},\reflhtpy))=(x,(\refl{},\reflhtpy)),
  \end{equation*}
  which we have by reflexivity.
\end{proof}

\begin{thm}
  Consider a type family $B$ over a type $A$, and let $k:\T$ be a truncation level. If $A$ is a $(k+1)$-type, then so is $\W(A,B)$.\index{W-type!is truncated}\index{is truncated!W-type}
\end{thm}

\begin{proof}
  Suppose that $A$ is a $(k+1)$-type. In order to show that $\W(A,B)$ is a $(k+1)$-type, we have to show that its identity types are $k$-types. The proof is by induction on $x,y:\W(A,B)$. For $x\jdeq\collect(a,\alpha)$ and $y\jdeq\collect(b,\beta)$, we have the equivalence
  \begin{equation*}
    (\collect(a,\alpha)=\collect(b,\beta))\simeq\sm{p:a=b}\prd{z:B(a)}\,\alpha(z)=\beta(\tr_B(p,z))
  \end{equation*}
  Note that the type $a=b$ is a $k$-type by the assumption that $A$ is a $(k+1)$-type. Furthermore, the type $\alpha(z)=\beta(\tr_B(p,z))$ is a $k$-type by the induction hypothesis. Therefore it follows that the type on the right-hand side of the displayed equivalence is a $k$-type, and this completes the proof.
\end{proof}
\index{observational equality!on W-types|)}
\index{W-type!observational equality|)}


\subsection{Functoriality of W-types}
\index{functorial action!of W-types|(}
\index{W-type!functorial action|(}

\begin{defn}
  Consider a type family $B$ over $A$, and a type family $B'$ over $A'$. Furthermore, consider a map $f:A'\to A$ and a family of equivalences
  \begin{equation*}
    e_x:B'(x)\simeq B(f(x))
  \end{equation*}
  indexed by $x:A'$. Then we define the map $\W(f,e):\W(A',B')\to\W(A,B)$\index{W(f,e)@{$\W(f,e)$}|textbf}\index{W(f,e)@{$\W(f,e)$}|see {W-type, functorial action}}\index{W-type!functorial action|textbf}\index{functorial action!of W-types|textbf} of W-types inductively by
  \begin{equation*}
    \W(f,e)(\collect(x,\alpha))\defeq\collect(f(x),\W(f,g)\circ \alpha\circ e_x^{-1}).
  \end{equation*}
\end{defn}

\begin{lem}\label{lem:fib-W}
  For any morphism $\W(f,e):\W(A',B')\to\W(A,B)$ of W-types and any $\collect(x,\alpha):\W(A,B)$, there is an equivalence\index{fiber!of W(f,e)@{of $\W(f,e)$}}\index{W(f,e)@{$\W(f,e)$}!fiber}
  \begin{equation*}
    \fib{\W(f,e)}{\collect(x,\alpha)} \simeq \fib{f}{x}\times\prd{b:B(x)}\fib{\W(f,e)}{\alpha(b)}.
  \end{equation*}
\end{lem}

\begin{proof}
  First, note that by the characterization in \cref{thm:EqW} of the identity type of $\W(A,B)$, there is an equivalence between the fiber $\fib{\W(f,e)}{\collect(x,\alpha)}$ and the type
  \begin{align*}
    & \sm{x':A'}\sm{\alpha':B'(x')\to\W(A',B')}\sm{p:f(x')=x} \\*
    & \phantom{\sm{x':A'}}\prd{b:B(f(x'))}\W(f,e)(\alpha'(e_{x'}^{-1}(b)))=\alpha(\tr_{B}(p,b)). \\
    \intertext{By rearranging the $\Sigma$-type, we see that this type is equivalent to the type}
    & \sm{(x',p):\fib{f}{x}}\sm{\alpha':B'(x')\to\W(A',B')} \\*
    & \phantom{\sm{x':A'}}\prd{b:B(f(x'))}\W(f,e)(\alpha'(e_{x'}^{-1}(b)))=\alpha(\tr_{B}(p,b)).
  \end{align*}
  Therefore, it suffices to show for each $(x',p):\fib{f}{x}$, that the type
  \begin{equation*}
    \sm{\alpha':B'(x')\to\W(A',B')}\prd{b:B(f(x'))}\W(f,e)(\alpha'(e_{x'}^{-1}(b)))=\alpha(\tr_{B}(p,b))
  \end{equation*}
  is equivalent to the type $\prd{b:B(x)}\fib{\W(f,e)}{\alpha(b)}$. Since we have an identification $p:f(x')=x$ and an equivalence $e_{x'}:B'(x')\simeq B(f(x'))$, it follows that the type above is equivalent to the type
  \begin{equation*}
    \sm{\alpha':B(x)\to\W(A',B')}\prd{b:B(x)}\W(f,e)(\alpha'(b))=\alpha(b).
  \end{equation*}
  By distributivity of $\Pi$ over $\Sigma$, i.e., by \cref{thm:choice}, this type is equivalent to the type
  \begin{equation*}
    \prd{b:B(x)}\sm{w:\W(A',B')}\W(f,e)(w)=\alpha(b),
  \end{equation*}
  completing the proof.
\end{proof}

\begin{thm}
  Consider a morphism $\W(f,e):\W(A,B)\to\W(A',B')$ of W-types. If the map $f:A\to A'$ is $k$-truncated, then so is the map $\W(f,e)$. In particular, if $f$ is an equivalence or an embedding, then so is $\W(f,e)$.\index{W(f,e)@{$\W(f,e)$}!is a truncated map}\index{W(f,e)@{$\W(f,e)$}!is an embedding}\index{W(f,e)@{$\W(f,e)$}!is an equivalence}\index{is an equivalence!W(f,e)@{$\W(f,e)$}}\index{is an embedding!W(f,e)@{$\W(f,e)$}}
\end{thm}

\begin{proof}
  Suppose that the map $f$ is $k$-truncated. We will prove recursively that the fibers of the morphism $\W(f,e)$ on W-types is $k$-truncated. We saw in \cref{lem:fib-W} that there is an equivalence
  \begin{equation*}
    \fib{\W(f,e)}{\collect(x,\alpha)}\simeq \fib{f}{x}\times\prd{b:B(x)}\fib{\W(f,e)}{\alpha(b)}.
  \end{equation*}
  The type $\fib{f}{x}$ is $k$-truncated by assumption, and each of the types
  \begin{equation*}
    \fib{\W(f,e)}{\alpha(b)}
  \end{equation*}
  is $k$-truncated by the inductive hypothesis, so the claim follows.
\end{proof}
\index{functorial action!of W-types|)}
\index{W-type!functorial action|)}


\subsection{The elementhood relation on W-types}
\index{elementhood relation on W-types|(}
\index{W-type!elementhood relation|(}

The elements of a W-type $\W(A,B)$ are constructed out of families of elements of $\W(A,B)$ indexed by a type $B(x)$ for some $x:A$. More precisely, for each $\collect(x,\alpha):\W(A,B)$ we have a family of elements
\begin{equation*}
  \alpha(y):\W(A,B)
\end{equation*}
indexed by $y:B(x)$. Thus, we could say that $\alpha(y)$ is in $\collect(x,\alpha)$, for each $y:B(x)$. More abstractly, we can define an elementhood relation on $\W(A,B)$.

\begin{defn}
  Given a W-type $\W(A,B)$ and a universe $\UU$ containing both $A$ and each type in the family $B$, we define a type-valued relation\index{e@{$\in$}|see {elementhood relation on W-types}|textbf}\index{e@{$\in$}|textbf}\index{W-type!e@{$\in$}|textbf}
  \begin{equation*}
    {\in}:\W(A,B)\to\W(A,B)\to \UU
  \end{equation*}
  by $(x\in \collect(a,\alpha))\defeq \sm{y:B(a)}\alpha(y)=x$. 
\end{defn}

Using the elementhood relation on $\W(A,B)$, we can reformulate the induction principle to, perhaps, a more recognizable form:

\begin{thm}
  For any family $P$ of types over $\W(A,B)$, there is a function\index{induction principle!of W-types}\index{W-type!induction principle}
  \begin{equation*}
    i : \Big(\prd{x:\W(A,B)}\Big(\prd{y:\W(A,B)}(y\in x)\to P(y)\Big)\to P(x)\Big)\to \Big(\prd{x:X}P(x)\Big)
  \end{equation*}
  that comes equipped with an identification
  \begin{equation*}
    i(h,x)=h(x,\lam{y}\lam{e}i(h,y))
  \end{equation*}
  for every $h:\prd{x:\W(A,B)}\Big(\prd{y:\W(A,B)}(y\in x)\to P(y)\Big)\to P(x)$, and every $x:\W(A,B)$.
\end{thm}

\begin{proof}
  For any type family $P$ over $\W(A,B)$, we first define a new type family $\square P$ over $\W(A,B)$ given by
  \begin{equation*}
    \square P(x):=\prd{y:\W(A,B)}(y\in x)\to P(y).
  \end{equation*}
  The family $\square P(x)$ comes equipped with a map
  \begin{equation*}
    \eta : \Big(\prd{x:\W(A,B)}P(x)\Big)\to \Big(\prd{x:\W(A,B)}\square P(x)\Big)
  \end{equation*}
  given by $\eta(f,x,y,e)\defeq f(y)$. Conversely, there is a map
  \begin{equation*}
    \varepsilon(h) : \Big(\prd{y:\W(A,B)}\square P(y)\Big) \to \Big(\prd{x:\W(A,B)}P(x)\Big)
  \end{equation*}
  for every $h:\prd{y:\W(A,B)}\square P (y)\to P(y)$, given by $\varepsilon(h,g,x)\defeq h(x,g(x))$. Note that the induction principle can now be stated as
  \begin{equation*}
    i : \Big(\prd{y:\W(A,B)}\square P (y)\to P(y)\Big)\to\Big(\prd{x:\W(A,B)}P(x)\Big),
  \end{equation*}
  and the computation rule states that
  \begin{equation*}
    i(h,x)=h(x,\eta(i(h),x)).
  \end{equation*}
  Before we prove the induction principle, we prove the intermediate claim that there is a function
  \begin{equation*}
    i' : \Big(\prd{y:\W(A,B)}\square P (y)\to P(y)\Big) \to \Big(\prd{x:\W(A,B)}\square P(x)\Big)
  \end{equation*}
  equipped with an identification
  \begin{equation*}
    j'(h,x,y,e) : i'(h,x,y,e) = h(y,i'(h,y))
  \end{equation*}
  for every $h:\prd{y:\W(A,B)}\square P(y)\to P(y)$ and every $x,y:\W(A,B)$ equipped with $e:y\in x$. Both $i'$ and $j'$ are defined by pattern matching:
  \begin{align*}
    i'(h,\collect(a,f),f(b),(b,\refl{})) & := h(f(b),i'(h,f(b))) \\
    j'(h,\collect(a,f),f(b),(b,\refl{})) & := \refl{}.
  \end{align*}
  Now we define $i(h):=\varepsilon(h,i'(h))$. Note that we have the judgmental equalities
  \begin{align*}
    i(h,x) & \jdeq \varepsilon(h,i'(h),x) \\
           & \jdeq h(x,i'(h,x)), \\
    \intertext{and}
    h(x,\lam{y}\lam{e}i(h,y))
           & \jdeq h(x,\lam{y}\lam{e}\varepsilon(h,i'(h),y)) \\
           & \jdeq h(x,\lam{y}\lam{e}h(y,i'(h,y))).
  \end{align*}
  The computation rule is therefore satisfied by the identification
  \begin{equation*}
    \begin{tikzcd}[column sep=12.5em]
      h(x,i'(h,x)) \arrow[r,equals,"\ap{h(x)}{\eqhtpy(\lam{y}\eqhtpy(j'(h,x,y)))}"] & h(x,\lam{y}\lam{e} h(y,i'(h,y))).
    \end{tikzcd}
    \qedhere
  \end{equation*}
\end{proof}

\subsection{Extensional W-types}\label{sec:extensional-W-types}

It is tempting to think that an element $w:\W(A,B)$ is completely determined by the elements $z:\W(A,B)$ equipped with a proof $z\in w$. However, this may not be the case. For instance, a W-type $\W(A,B)$ might have \emph{two} unary constructors, e.g., when $A\defeq \unit+\bool$ and the family $B$ over $A$ is given by
\begin{align*}
  B(\inl(x)) & \defeq \emptyt \\
  B(\inr(y)) & \defeq \unit.
\end{align*}
If we write $f$ and $g$ for the two unary constructors of $\W(A,B)$, then we see that for any element $w:\W(A,B)$, the elements
\begin{equation*}
  u\defeq\collect(\inr(\bfalse),\const_w)\qquad\text{and}\qquad v\defeq\collect(\inr(\btrue),\const_w)
\end{equation*}
both only contain the element $w$. However, the elements $u$ and $v$ are distinct in $\W(A,B)$.

Something similar happens in the type of oriented binary rooted trees. Given two binary rooted trees $S$ and $T$, there are two ways to combine $S$ and $T$ into a new binary tree: we have $[S,T]$ and $[T,S]$. Both contain precisely the elements $S$ and $T$, but they are distinct. Nevertheless, there are many important W-types in which the elements $w$ are uniquely determined by the elements $z\in w$. Such W-types are called extensional.

\begin{defn}
  We say that a W-type $\W(A,B)$ is \define{extensional}\index{extensional W-type|textbf}\index{W-type!extensionality|textbf}\index{extensionality principle!for W-types|textbf} if the canonical map
  \begin{equation*}
    (x=y)\to\prd{z:\W(A,B)}(z\in x)\simeq (z\in y)
  \end{equation*}
  is an equivalence.
\end{defn}

In the following theorem we give a precise characterization of the inhabited extensional W-types. 

\begin{thm}\label{thm:extensional-W}
  Consider an inhabited W-type $\W(A,B)$. Then the following are equivalent:
  \begin{enumerate}
  \item The W-type $\W(A,B)$ is extensional.
  \item The family $B$ is \define{univalent}\index{type family!univalent type family|textbf}\index{univalent type family|textbf} in the sense that the map
  \begin{equation*}
    \tr_B:(x=y)\to (B(x)\simeq B(y))
  \end{equation*}
  is an equivalence, for every $x,y:A$.
  \end{enumerate}
\end{thm}

\begin{rmk}
  Note that if the W-type $\W(A,B)$ is empty, then it is vacuously extensional. However, we saw in \cref{prp:is-empty-W} that any family $B$ of inhabited types over $A$ gives rise to an empty W-type $\W(A,B)$, so there is no hope of showing that $B$ is a univalent family if $\W(A,B)$ is empty.

  We also note that a type family $B$ over $A$ is univalent if and only if the map $B:A\to \UU$ is an embedding. In other words, the claim in \cref{thm:extensional-W} is that an inhabited W-type $\W(A,B)$ is extensional if and only if $B$ is the canonical type family over a subuniverse $A$ of $\UU$.
\end{rmk}

\begin{proof}
  We will first show that (ii) is equivalent to the following property:
  \begin{enumerate}
  \item[(ii')] The map
    \begin{equation*}
      \tr_B : (\prearity(x)=y)\to (B(\prearity(x))\simeq B(y))
    \end{equation*}
    is an equivalence for every $x:\W(A,B)$ and every $y:A$.
  \end{enumerate}
  Clearly, (ii) implies (ii'). For the converse we use the assumption that $\W(A,B)$ is inhabited. Since the property in (ii) is a proposition, we may assume an element $w:\W(A,B)$. Using $w$, we obtain for every $x:A$ the element
  \begin{equation*}
    \collect(x,\const_w):\W(A,B)
  \end{equation*}
  The symbol of $\collect(x,\const_w)$ is $x$, and therefore the hypothesis that (ii') holds implies that the map $(x=y)\to (B(x)\simeq B(y))$ is an equivalence. This concludes the proof that (ii) is equivalent to (ii'). It remains to show that (i) is equivalent to (ii').

  Let $x:\W(A,B)$. By the fundamental theorem of identity types, the W-type $\W(A,B)$ is extensional if and only if the total space
  \begin{equation*}
    \sm{y:\W(A,B)}\prd{z:\W(A,B)}(z\in x)\simeq (z\in y)
  \end{equation*}
  is contractible, for any $x:\W(A,B)$. When $x$ is of the form $\collect(a,\alpha)$, the type $z\in x$ is just the fiber $\fib{\alpha}{z}$. Using this observation, we see that the above type is equivalent to the type
  \begin{equation*}
    \sm{b:A}\sm{\beta:B(b)\to\W(A,B)}\prd{z:\W(A,B)}\fib{\alpha}{z}\simeq\fib{\beta}{z}.\tag{\textasteriskcentered}
  \end{equation*}
  By \cref{ex:fam-equiv} it follows that this type is equivalent to the type
  \begin{equation*}
    \sm{y:A}\sm{\beta:B(y)\to\W(A,B)}\sm{e:B(x)\simeq B(y)}\alpha\htpy e\circ \beta.
  \end{equation*}
  Note that the type $\sm{\beta:B(y)\to\W(A,B)}\alpha\htpy e\circ\beta$ is contractible for any equivalence $e:B(x)\simeq B(y)$. Therefore, it follows that the above type is contractible if and  only if the type
  \begin{equation*}
    \sm{y:A}B(x)\simeq B(y)
  \end{equation*}
  is contractible, which is the case if and only if the map $(x=y)\to(B(x)\simeq B(y))$ is an equivalence for all $y:A$.
\end{proof}

\begin{eg}
  The type $N$ of \cref{eg:Nat-W}\index{natural numbers!is an extensional W-type}, the type of binary rooted trees \cref{eg:binary-tree-W}\index{binary rooted tree!is an extensional W-type}, and the type of finitely branching rooted trees \cref{eg:finitely-branching-tree-W}\index{finitely branching rooted tree!is an extensional W-type} are all examples extensional W-types. On the other hand, the type of oriented binary rooted trees of \cref{eg:planar-binary-tree-W} and the type of oriented finitely branching rooted trees of \cref{eg:finitely-branching-tree-W} are not extensional.
\end{eg}
\index{elementhood relation on W-types|)}
\index{W-type!elementhood relation|)}


\subsection{Russell's paradox in type theory}\label{subsec:russell}

\index{Russell's paradox|(}
\index{multiset|(}
Russell's paradox tells us that there cannot be a set of all sets. If there were such a set $S$, then we could form the set
\begin{equation*}
  R\defeq \{x\in S\mid x\notin x\},
\end{equation*}
for which we have $R\in R\leftrightarrow R\notin R$, a contradiction. To reproduce Russell's paradox in type theory, we first recall a crucial difference between the type theoritic judgment $a:A$ and the set theoretic proposition $x\in y$. Although the judgment $a:A$ plays a similar role in type theory as the elementhood relation, types and their elements are fundamentally different entities, whereas in Zermelo-Fraenkel set theory there are only sets, and the proposition $x\in y$ can be formed for any two sets $x$ and $y$. In type theory, there is no relation on the universe that is similar to the elementhood relation.

However, we have seen in \cref{sec:extensional-W-types} that it is possible to define an elementhood relation on arbitrary W-types. We will use this elementhood relation on the W-type $\W(\UU,\Ty)$ to derive a paradox analogous to Russell's paradox, and we will see that $\UU$ cannot be equivalent to a type in $\UU$.

The type $\W(\UU,\Ty)$ possesses a lot of further structure. In fact, it can be used to encode constructive set theory in type theory. There is, however, one significant difference with ordinary set theory: the elementhood relation is type-valued. In other words, there may be many ways in which $x\in y$ holds. The type $\W(\UU,\Ty)$ is therefore also called the type of \define{multisets}. It was first studied by Aczel in \cite{AczelCZF}, with refinements in \cite{AczelGambinoCZF}, and in the setting of univalent mathematics it has been studied extensively by Gylterud in \cite{GylterudMultisets}.

\begin{defn}
  Consider a $\UU$ with universal type family $\Ty$. We define the type\index{M_U@{$\multiset{\UU}$}|textbf}\index{M_U@{$\multiset{\UU}$}|see {multiset}}
  \begin{equation*}
    \multiset{\UU} \defeq \W(\UU,\Ty),
  \end{equation*}
  and the elements of $\multiset{\UU}$ are called \define{multisets in $\UU$}\index{multiset|textbf}. We will write\index{{f(x)"|"x:A}@{$\{f(x)\mid x:A\}$}|textbf}\index{multiset!{f(x)"|"x:A}@{$\{f(x)\mid x:A\}$}|textbf}
  \begin{equation*}
    \{f(x)\mid x:A\}
  \end{equation*}
  for the multiset in $\UU$ of the form $\collect(A,f)$. More generally, given an element $t(x_0,\ldots,x_n):\multiset{\UU}$ in context $x_0:A_0,\ldots,x_n:A_n(x_0,\ldots,x_{n-1})$, where each $A_i$ is in $\UU$, we will write\index{{t(x0,...,xn)"|"x0:A0,...,xn:An}@{$\{t(x_0,\ldots,x_n)\mid x_0:A_0,\ldots,x_n:A_n\}$}|textbf}\index{multiset!{t(x0,...,xn)"|"x0:A0,...,xn:An}@{$\{t(x_0,\ldots,x_n)\mid x_0:A_0,\ldots,x_n:A_n\}$}|textbf}
  \begin{equation*}
    \{t(x_0,\ldots,x_n)\mid x_0:A_0,\ldots,x_n:A_n(x_0,\ldots,x_{n-1})\}
  \end{equation*}
  for the multiset in $\UU$ of the form
  \begin{equation*}
    \collect\left(\sm{x_0:A_0}\cdots A_n(x_0,\ldots,x_{n-1}),\lam{(x_0,\ldots,x_n)}t(x_0,\ldots,x_n)\right).
  \end{equation*}

  Given a multiset $X\jdeq \{f(x)\mid x:A\}$ in $\UU$, the \define{cardinality}\index{cardinality!of a multiset|textbf}\index{multiset!cardinality|textbf} of $X$ is the type $A$, and the \define{elements}\index{element of a multiset|textbf}\index{multiset!element|textbf} of $X$ are the multisets $f(x)$ in $\UU$, for each $x:A$.
\end{defn}

In the notation of multisets, the elementhood relation ${\in}:\multiset{\UU}\to\multiset{\UU} \to\UU^+$ is defined by
\begin{equation*}
  (X\in \{g(y)\mid y:B\}) \jdeq \sm{y:B} g(y)=X.
\end{equation*}
In other words, a multiset $X$ is in a multiset of the form $\{g(y)\mid y:B\}$ if and only if $X$ comes equipped with an element $y:B$ and an identification $g(y)=X$. The W-type of multisets is extensional by \cref{thm:extensional-W} and the univalence axiom.

Recall from \cref{defn:small-types}\index{small type}\index{U-small type@{$\UU$-small type}} that for a universe $\UU$, we say that a type $A$ is (essentially) $\UU$-small if $A$ comes equipped with an element of type\index{is-small@{$\issmall_\UU(A)$}}
\begin{equation*}
  \issmall_{\UU}(A)\defeq \sm{X:\UU}A\simeq X. 
\end{equation*}
Our goal in this section is to show, via Russell's paradox, that the universe $\UU$ is not $\UU$-small, i.e., that there cannot be a type $U:\UU$ equipped with an equivalence $\UU\simeq U$. We will use a similar condition of smallness for multisets.

\begin{defn}
  Let $\UU$ and $\VV$ be universes. We say that a multiset $\{f(x)\mid x:A\}$ in $\VV$ \define{is $\UU$-small}\index{small multiset|textbf}\index{U-small multiset@{$\UU$-small multiset}|textbf}\index{multiset!small multiset|textbf} if the type $A$ is $\UU$-small and if each mulitset $f(x)$ in $\VV$ is $\UU$-small. In other words, the type family\index{is-small-M@{$\issmallmultiset{\UU}$}|textbf}\index{multiset!is-small-M@{$\issmallmultiset{\UU}$}|textbf}
  \begin{equation*}
    \issmallmultiset{\UU} : \multiset{\VV}\to \VV\sqcup\UU^+ 
  \end{equation*}
  is defined recursively by
  \begin{equation*}
    \issmallmultiset{\UU}(\{f(x)\mid x:A\}) \defeq \issmall_{\UU}(A)\times \prd{x:A}\issmallmultiset{\UU}(f(x)).
  \end{equation*}
\end{defn}

We will need quite a few properties of smallness before we can reproduce Russell's paradox. We begin with a simple lemma.

\begin{lem}\label{lem:is-small-comprehension-multiset}
  Consider a $\UU$-small multiset $\{f(x)\mid x:A\}$ in $\VV$, and let $B$ be a family of $\UU$-small types over $A$. Then the multiset
  \begin{equation*}
    \{f(x)\mid x:A, y:B(x)\}
  \end{equation*}
  is again $\UU$-small.
\end{lem}

\begin{proof}
  If the multiset $\{f(x)\mid x:A\}$ is $\UU$-small, then the type $A$ is $\UU$-small. By the assumption that $B$ is a family of $\UU$-small types together with the fact that $\UU$-small types are closed under formation of $\Sigma$-types, it follows that the type
  \begin{equation*}
    \sm{x:A}B(x)
  \end{equation*}
  is $\UU$-small. Furthermore, since each $f(x)$ is $\UU$-small, we conclude that the multiset $\{f(x)\mid x:A,y:B(x)\}$ is $\UU$-small. 
\end{proof}

The main purpose of the following lemma is to know that the elementhood relation takes values in the $\UU$-small types, when it is applied to $\UU$-small multisets. We will use the univalence axiom to prove this fact. 

\begin{prp}\label{prp:is-small-elementhood-multiset}
  Consider two univalent universes $\UU$ and $\VV$, and let $X$ and $Y$ be $\UU$-small multisets in $\VV$. We make two claims:
  \begin{enumerate}
  \item The type $X=Y$ is $\UU$-small.
  \item The type $X\in Y$ is $\UU$-small.
  \end{enumerate}
\end{prp}

\begin{proof}
  For the first claim, let $X\jdeq\{f(x)\mid x : A\}$ and let $Y\jdeq\{g(y)\mid y:B\}$. The proof is by induction. Via \cref{thm:EqW} it follows that the type $X=Y$ is equivalent to the type
  \begin{equation*}
    \sm{p:A=B}\prd{x:A}f(x)=g(\equiveq(p)).
  \end{equation*}
  The type $A=B$ is $\UU$-small because it is equivalent to the type $A\simeq B$, which is $\UU$-small. Therefore it suffices to show that the type
  \begin{equation*}
    \prd{x:A}f(x)=g(\equiveq(p))
  \end{equation*}
  is $\UU$-small, for every $p:A=B$. Here we proceed by identification elimination, and the type $\prd{x:A}f(x)=g(x)$ is a product of $\UU$-small types by the induction hypothesis. This concludes the proof of the first claim.

  For the second claim, let $Y\jdeq\{g(y)\mid y:B\}$. Then the type
  \begin{equation*}
    \sm{y:B}g(y)=X
  \end{equation*}
  is a dependent sum of $\UU$-small types, indexed by an $\UU$-small type, which is again $\UU$-small.
\end{proof}

The condition that a multiset $\{f(x)\mid x:A\}$ in $\VV$ is $\UU$-small suggests that there is an `equivalent' multiset in $\UU$. 

\begin{defn}\label{defn:inclusion-small-multisets}
  Given two universes $\UU$ and $\VV$, we define an inclusion function
  \begin{equation*}
    i : \Big(\sm{X:\multiset{\VV}}\issmallmultiset{\UU}(X)\Big)\to\multiset{\UU},
  \end{equation*}
  of the $\UU$-small multisets in $\VV$ into the multisets in $\UU$, inductively by
  \begin{equation*}
    i(\{f(x)\mid x:A\})\defeq \{i(f(e^{-1}(y))) \mid y:B\}.
  \end{equation*}
  for any multiset $\{f(x)\mid x:A\}$ of which the type $A$ is equipped with an equivalence $e:A\simeq B$ for some $B$ in $\UU$, and such that the multiset $f(x)$ in $\VV$ is $\UU$-small for each $x:A$.
\end{defn}

\begin{prp}\label{prp:is-embedding-inclusion-small-multisets}
  The inclusion function $i$ of $\UU$-small multisets in $\VV$ into the multisets in $\UU$ satisfies the following properties
  \begin{enumerate}
  \item For each $\UU$-small multiset $X$ in $\VV$, the multiset $i(X)$ in $\UU$ is $\VV$-small.
  \item The induced map
    \begin{equation*}
      \Big(\sm{X:\multiset{\VV}}\issmallmultiset{\UU}(X)\Big)\to\Big(\sm{Y:\multiset{\UU}}\issmallmultiset{\VV}(Y)\Big)
    \end{equation*}
    is an equivalence.
  \end{enumerate}
  Consequently, the inclusion function $i$ is an embedding.
\end{prp}

\begin{proof}
  To see that $i(\{f(x)\mid x:A\})$ is $\VV$-small for each $\UU$-small multiset $\{f(x)\mid x:A\}$ in $\VV$, note that the assumption that $\{f(x)\mid x:A\}$ is $\UU$-small gives us an equivalence $e:A\simeq B$ and an element $H(x):\issmallmultiset{\UU}(f(x))$ for each $x:A$. The type $B$ is the indexing type of $i(\{f(x)\mid x:A\})$, and $B$ is $\VV$-small because it is equivalent to the type $A$ in $\VV$. Furthermore, each multiset $i(f(e^{-1}(y)))$ is $\VV$-small by the inductive hypothesis. This completes the proof of the first claim.

  We therefore have inclusion functions
  \begin{equation*}
    \begin{tikzcd}
      \Big(\sm{X:\multiset{\VV}}\issmallmultiset{\UU}(X)\Big) \arrow[r,yshift=-.7ex,swap,"i"] &
      \Big(\sm{Y:\multiset{\UU}}\issmallmultiset{\VV}(Y)\Big) \arrow[l,yshift=.7ex,swap,"i"]
    \end{tikzcd}
  \end{equation*}
  To see that the maps $i$ and $i$ are mutual inverses, it suffices to show that $i(i(X))=X$. This follows by induction from the following calculation, where we assume an equivalence $e:A\simeq B$ into a $B$ in $\UU$.
  \begin{align*}
    i(i(\{f(x)\mid x :A\})) & \jdeq i(\{i(f(e^{-1}(y)))\mid y:B\}) \\
                            & \jdeq \{i(i(f(e^{-1}(e(x))))) \mid x:A\} \\
                            & = \{i(i(f(x)))\mid x:A\} \\
                            & = \{f(x)\mid x:A\}.
  \end{align*}
  
  For the last claim, note that we have factored $i$ as an equivalence followed by an embedding
  \begin{equation*}
    \begin{tikzcd}[column sep=2em]
      \Big(\sm{X:\multiset{\VV}}\issmallmultiset{\UU}(X)\Big) \arrow[r] &
      \Big(\sm{Y:\multiset{\UU}}\issmallmultiset{\VV}(Y)\Big) \arrow[r] &
      \multiset{\VV},
    \end{tikzcd}
  \end{equation*}
  and therefore $i$ is an embedding.
\end{proof}

Furthermore, the embedding $i$ induces equivalences on the elementhood relation on multisets.

\begin{prp}\label{prp:elementhood-small-multisets}
  Consider a multiset $X$ in $\UU$ and a multiset $Y$ in $\VV$. Furthermore, suppose that $X$ is $\VV$-small and that $Y$ is $\UU$-small. Then we have
  \begin{equation*}
    (i(X)\in Y)\simeq (X\in i(Y)).
  \end{equation*}
\end{prp}

\begin{proof}
  Let $X\jdeq\{f(x)\mid x:A\}$ and $Y\jdeq\{g(y)\mid y:B\}$. By the assumption that $Y$ is $\UU$-small we have an equivalence $e:B\simeq B'$ to a type $B'$ in $\UU$. Then we have the equivalences
  \begin{align*}
    i(X) \in \{g(y)\mid y:B\} & \jdeq \sm{y:B}g(y)=i(X) \\
                              & \simeq \sm{y:B}i(g(y))=X \\
                              & \simeq \sm{y':B'}i(g(e^{-1}(y')))=X \\
                              & \jdeq X\in i(Y).\qedhere
  \end{align*}
\end{proof}

We are now almost in position to reproduce Russell's paradox. We will need one more ingredient: the universal tree, i.e., the multiset of all multisets in $\UU$. 

\begin{defn}
  Let $\UU$ be a universe. Then we define the \define{universal tree}\index{universal tree|textbf}\index{tree!universal tree|textbf}\index{multiset!universal tree|textbf} $\yggdrasil$ to be the multiset\index{Y_U@{$\yggdrasil$}|see {universal tree}}
  \begin{equation*}
    \yggdrasil:=\{i(X) \mid X:\multiset{\UU}\}
  \end{equation*}
  in $\UU^{+}$, where $i:\multiset{\UU}\to\multiset{\UU^+}$ is the inclusion of the multisets in $\UU$ to the multisets in $\UU^+$ given by the fact that each multiset in $\UU$ is $\UU^+$-small.
\end{defn}

\begin{prp}\label{prp:is-small-universal-tree}
  Consider two universes $\UU$ and $\VV$, and suppose that $\UU$ as well as each $X:\UU$ are $\VV$-small. Then the universal tree $\yggdrasil$ is also $\VV$-small.
\end{prp}

\begin{proof}
  To show that the universal tree $\{i(X)\mid X:\multiset{\UU}\}$ is $\VV$-small, we first have to show that the type $\multiset{\UU}$ is $\VV$-small. This follows from the more general fact that the subuniverse of $\VV$-small types is closed under the formation of W-types. Indeed, if a type $A$ is $\VV$-small, and if $B(x)$ is $\VV$-small for each $x:A$, then we have an equivalence $\alpha:A\simeq A'$ to a type $A'$ in $\VV$, and for each $x':A'$ we have an equivalence $B(\alpha^{-1}(x'))\simeq B'(x')$ in $\VV$. These equivalences induce an equivalence
  \begin{equation*}
    \W(A,B)\simeq \W(A',B')
  \end{equation*}
  into the type $W(A',B')$, which is in $\VV$. This concludes the proof that $\multiset{\UU}$ is $\VV$-small.

  It remains to show that the multiset $i(X)$ in $\UU^+$ is $\VV$-small, for each $X:\multiset{\UU}$. Equivalently, we have to show that each multiset $X$ in $\UU$ is $\VV$-small. This follows by recursion: given a multiset $\{f(x)\mid x:A\}$, the type $A$ is $\VV$-small by assumption, and the multiset $f(x)$ is $\VV$-small by the induction hypothesis.
\end{proof}

We are finally ready to employ \define{Russell's paradox} to prove that a univalent universe cannot be equivalent to any type it contains.\index{Russell's paradox|textbf}

\begin{thm}\label{thm:russell}
  Consider a univalent universe $\UU$. Then $\UU$ cannot be $\UU$-small.\index{universe!U is not U-small@{$\UU$ is not $\UU$-small}}\index{U-small type@{$\UU$-small type}!U is not U-small@{$\UU$ is not $\UU$-small}}\index{small type!U is not U-small@{$\UU$ is not $\UU$-small}}
\end{thm}

\begin{proof}
  Suppose that $\UU$ is $\UU$-small, and consider the multiset
  \begin{equation*}
    R\defeq \{i(X) \mid X:\multiset{\UU}, H : X\notin X\}
  \end{equation*}
  in $\UU^+$, where $i:\multiset{\UU}\to\multiset{\UU^+}$ is the inclusion of the multisets in $\UU$ to the multisets in $\UU^+$ given by the fact that each multiset in $\UU$ is $\UU^+$-small.

  First, we note that $R$ is $\UU$-small. This follows from \cref{lem:is-small-comprehension-multiset}, using the fact that the universal tree $\{i(X)\mid X:\multiset{\UU}\}$ is $\UU$-small by \cref{prp:is-small-universal-tree}, and the fact that $X\in X$ is $\UU$-small by \cref{prp:is-small-elementhood-multiset}.

  Since $R$ is $\UU$-small, there is a multiset $R':\multiset{\UU}$ such that $i(R')=R$. Now it follows that
  \begin{align*}
    R\in R & \simeq \sm{X:\multiset{\UU}}\sm{H:X\notin X} i(X)=R \\
           & \simeq \sm{X:\multiset{\UU}}\sm{H:X\notin X} X=R' \\
           & \simeq R'\notin R' \\
           & \simeq R\notin R.
  \end{align*}
  In the second step we used \cref{prp:is-embedding-inclusion-small-multisets}, where we showed that $i$ is an embedding, and in the last step we used \cref{prp:elementhood-small-multisets}. Now we obtain a contradiction, because it follows from \cref{ex:no-fixed-points-neg} that no type is (logically) equivalent to its own negation.
\end{proof}
\index{Russell's paradox|)}
\index{multiset|)}

\begin{exercises}
  \exitem
  \begin{subexenum}
  \item \label{ex:oriented-bin-tree}Let $B:\bool\to\UU$ be the type family defined in \cref{eg:planar-binary-tree-W}. Construct an equivalence\index{oriented binary rooted tree}\index{tree!oriented binary rooted tree}
    \begin{equation*}
      \planarBinTree\simeq\W(\bool,B).
    \end{equation*}
  \item Prove that $\W(\bool,B)$ is not extensional.
  \end{subexenum}
  \exitem Show that for any univalent universe $\UU$ there is no type $U:\UU$ equipped with a surjection $\UU\twoheadrightarrow U$.
  \exitem For a type family $B$ over $A$, suppose that each $B(x)$ is empty. Show that the type $\W(A,B)$ is equivalent to the type $A$.
  \exitem
  \begin{subexenum}
  \item Show that the elementhood relation ${\in}$ on $\W(A,B)$ is irreflexive, for any type family $B$ over any type $A$.\index{elementhood relation on W-types!is irreflexive}\index{W-type!elementhood relation!is irreflexive}
  \item Use the previous fact along with \cref{prp:elementhood-small-multisets} to give a second proof of the fact that there can be no type $U:\UU$ equipped with an equivalence $\UU\simeq U$.
  \end{subexenum}
  \exitem \label{ex:le-W} For each $x:\W(A,B)$, let ${x <(\blank)}:\W(A,B)\to\UU$ be the type family generated inductively by the following constructors:\index{W-type!strict ordering|textbf}\index{strict ordering!on W-types|textbf}
  \begin{align*}
    i & : \prd{y:\W(A,B)}(x\in y) \to (x < y) \\
    j & : \prd{y,z:\W(A,B)} (y\in z) \to ((x<y) \to (x<z)).
  \end{align*}
  \begin{subexenum}
  \item Show that the type-valued relation $<$ is transitive and irreflexive.\index{W-type!strict ordering!is transitive}\index{strict ordering!on W-types!is transitive}\index{W-type!strict ordering!is irreflexive}\index{strict ordering!on W-types!is irreflexive}
  \item Suppose that the type $\W(A,B)$ is inhabited and suppose that there exists an element $a:A$ for which $B(a)$ is inhabited. Show that the following are equivalent:
    \begin{enumerate}
    \item The type $x<y$ is a proposition for all $x,y:\W(A,B)$.
    \item The type $x\in y$ is a proposition for all $x,y:\W(A,B)$.
    \item The type $A$ is a set and the type $B(a)$ is a proposition for all $a:A$.
    \end{enumerate}
    Thus, in general it is not the case that $<$ is a relation valued in propositions.
  \item Show that $\W(A,B)$ satisfies the following \define{strong induction principle}\index{strong induction principle!of W-types|textbf}\index{W-type!strong induction principle|textbf}: For any type family $P$ over $\W(A,B)$, if there is a function
    \begin{equation*}
      h:\prd{x:\W(A,B)}\Big(\prd{y:\W(A,B)} (y<x)\to P(y)\Big)\to P(x),
    \end{equation*}
    then there is a function $f:\prd{x:\W(A,B)}P(x)$ equipped with an identification
    \begin{equation*}
      f(x)=h(x,\lam{y}\lam{p}f(y))
    \end{equation*}
    for all $x:\W(A,B)$.
  \item Show that there can be no sequence of elements $x:\N\to\W(A,B)$ such that $x_{n+1}< x_n$ for all $n:\N$.\index{W-type!no infinitely descending sequences}
  \end{subexenum}
  \exitem (Awodey, Gambino, Sojakova \cite{AwodeyGambinoSojakova}) For any type family $B$ over $A$, the \define{polynomial endofunctor}\index{polynomial endofunctor|textbf} $P_{A,B}$ acts on types by
  \begin{equation*}
    P_{A,B}(X) \defeq \sm{x:A}X^{B(x)},
  \end{equation*}
  and it takes a map $h:X\to Y$ to the map\index{functorial action!polynomial endofunctor|textbf}
  \begin{equation*}
    P_{A,B}(h) : P_{A,B}(X)\to P_{A,B}(Y)
  \end{equation*}
  defined by $P_{A,B}(h,(x,\alpha)) \defeq (x,h\circ \alpha)$. Furthermore, there is a canonical map
  \begin{equation*}
    (h\htpy h') \to (P_{A,B}(h)\htpy P_{A,B}(h'))
  \end{equation*}
  taking a homotopy $H:h\htpy h'$ to a homotopy $P_{A,B}(H):P_{A,B}(h)\htpy P_{A,B}(h')$. 

  A type $X$ is said to be equipped with the \define{structure of an algebra}\index{polynomial endofunctor!algebra|textbf}\index{algebra of a polynomial endofunctor|textbf} for the polynomial endofunctor $P_{A,B}$ if $X$ comes equipped with a map
  \begin{equation*}
    \mu: P_{A,B}(X)\to X.
  \end{equation*}
  Thus, \define{algebras} for the polynomial endofunctor $P_{A,B}$ are pairs $(X,\mu)$ where $X$ is a type and $\mu:P_{A,B}(X)\to X$. Note that $\W(A,B)$ comes equipped with the structure of an algebra for $P_{A,B}$ by \cref{prp:algebra-W}.
  
  Given two algebras $X$ and $Y$ for the polynomial endofunctor $P_{A,B}$, we say that a map $h:X\to Y$ is equipped with the \define{structure of a homomorphism}\index{polynomial endofunctor!morphism of algebras|textbf}\index{morphism of algebras!polynomial endofunctor|textbf} of algebras if it comes equipped with a homotopy witnessing that the square
  \begin{equation*}
    \begin{tikzcd}[column sep=large]
      P_{A,B}(X) \arrow[d,swap,"\mu_X"] \arrow[r,"P_{A,B}(h)"] & P_{A,B}(Y) \arrow[d,"\mu_Y"] \\
      X \arrow[r,swap,"h"] & Y
    \end{tikzcd}
  \end{equation*}
  commutes. The type $\hom((X,\mu_X),(Y,\mu_Y))$ of homomorphisms of algebras for $P_{A,B}$ is therefore defined as
  \begin{equation*}
    \hom((X,\mu_X),(Y,\mu_Y))\defeq \sm{h:X\to Y}h\circ\mu_X\htpy \mu_Y\circ P_{A,B}(h).
  \end{equation*}
  \begin{subexenum}
  \item For any $(x,\alpha),(y,\beta):P_{A,B}(X)$, construct an equivalence
    \begin{equation*}
      ((x,\alpha)=(y,\beta)) \simeq \sm{p:x=y}\alpha\htpy \beta\circ\tr_B(p).
    \end{equation*}
  \item For any two morphisms $(f,K),(g,L):\hom((X,\mu_X),(Y,\mu_Y))$ of algebras for $P_{A,B}$, construct an equivalence
    \begin{equation*}
      ((f,K)=(g,L))\simeq \sm{H:f\htpy g} \ct{K}{(\mu_Y\cdot P_{A,B}(H))}\htpy \ct{(H\cdot \mu_X)}{L}.
    \end{equation*}
  \item Show that the W-type $\W(A,B)$ equipped with the canonical structure $\varepsilon$ of a $P_{A,B}$-algebra, constructed in \cref{prp:algebra-W}, is a \define{(homotopy) initial $P_{A,B}$-algebra} in the sense that the type
    \begin{equation*}
      \hom((\W(A,B),\varepsilon),(X,\mu))
    \end{equation*}
    is contractible, for each $P_{A,B}$-algebra $(X,\mu)$.\index{W-type!is initial algebra of polynomial endofunctor}
  \end{subexenum}
  \exitem Consider the \define{rank comparison relation}\index{W-type!rank comparison relation|textbf}\index{rank comparison relation!W-type|textbf} ${\preceq} : \W(A,B)\to (\W(A,B)\to\prop_\UU)$ defined recursively by\index{preceq@{$\preceq$}|see {W-type, rank comparison relation}}
  \begin{equation*}
    (\collect(a,\alpha)\preceq\collect(b,\beta)) \defeq \forall_{(x:B(a))}\exists_{(y:B(b))}\,\alpha(x)\preceq \beta(y).
  \end{equation*}
  If $x\preceq y$ holds, we say that $x$ has \define{lower rank} than $y$. Furthermore, we define the \define{strict rank comparison relation}\index{W-type!strict rank comparison relation|textbf}\index{strict rank comparison relation!W-type|textbf} ${\prec}$\index{prec@{$\prec$}|see {W-type, strict rank comparison relation}} on $\W(A,B)$ by
  \begin{equation*}
    (x\prec y)\defeq \exists_{(z\in y)}x\preceq z.
  \end{equation*}
  If $x\prec y$ holds, we say that $x$ has \define{strictly lower rank} than $y$.
  \begin{subexenum}
  \item Show that the rank comparison relation defines a preordering on $\W(A,B)$, i.e., show that $\preceq$ is reflexive and transitve\index{W-type!rank comparison relation!is a preordering}\index{rank comparison relation!W-type!is a preordering}. Furthermore, prove the following properties, in which $<$ is the strict ordering on $\W(A,B)$ defined in \cref{ex:le-W}:
    \begin{enumerate}
    \item $(x \preceq y) \leftrightarrow \forall_{(x'<x)}\exists_{(y'<y)}\,x'\preceq y'$
    \item $(x < y)\to (x\preceq y)$
    \item $(x < y) \to (y \npreceq x)$
    \item $\isconstantW(x)\leftrightarrow \forall_{(y:\W(A,B))}\,x \preceq y$.
    \end{enumerate}
  \item Show that the relation $\prec$ on $\W(A,B)$ is a strict ordering on $\W(A,B)$, i.e., show that it is irreflexive and transitive\index{W-type!strict rank comparison relation!is a strict ordering}\index{strict rank comparison relation!W-type!is a strict ordering}. Furthermore, prove the following properties:
    \begin{enumerate}
    \item $(x < y)\to (x\prec y)$
    \item $(x \prec y)\to (x\preceq y)$
    \item $\forall_{(y\preceq y')}\forall_{(x'\preceq x)}(x\prec y)\to (x'\prec y')$.
    \end{enumerate}
  \end{subexenum}
  Since $\preceq$ defines a preordering on $\W(A,B)$, it follows that the preorder $(\W(A,B),\preceq)$ has a poset reflection, in the sense of \cref{ex:poset-reflection}. We will write
  \begin{equation*}
    \eta : (\W(A,B),\preceq)\to(\rank(A,B),\preceq)
  \end{equation*}
  for the poset reflection of $(\W(A,B),\preceq)$\index{R(A,B)@{$\rank(A,B)$}|see {W-type, rank poset}}\index{W-type!rank poset|textbf}\index{rank poset!W-type|textbf} and its quotient map. We will call the poset $(\mathcal{R}(A,B),\preceq)$ the \define{rank poset} of the W-type $\W(A,B)$.
  \begin{subexenum}[resume]
  \item Show that if each $B(x)$ is finite, then the rank poset $(\rank(A,B),\preceq)$ is either the empty poset, the poset with one element, or it is isomorphic to the poset $(\N,\leq)$. 
  \item Show that the strict ordering $\prec$\index{strict rank comparison relation!W-type!extends to rank poset} extends to a relation $\prec$ on $\rank(A,B)$ with the following properties:
    \begin{enumerate}
    \item We have $(x\prec y)\leftrightarrow (\eta(x)\prec\eta(y))$ for every $x,y:\W(A,B)$.
    \item We have $(x\prec y)\to (x\preceq y)$ for every $x,y:\rank(A,B)$.
    \item The relation $\prec$ is transitive and irreflexive on $\rank(A,B)$.
    \end{enumerate}
    We will call the strictly ordered set $(\mathcal{R}(A,B),\prec)$ the \define{(strict) rank}\index{rank!of W-type}\index{W-type!rank} of the W-type $\W(A,B)$. 
  \item A \define{strictly ordered set}\index{strictly ordered set|textbf} $(X,<)$, i.e., a set $X$ equipped with a transitive, irreflexive relation $<$ valued in the propositions, is said to be \define{well-founded}\index{well-founded relation|textbf} if for any family $P$ of propositions over $X$, the implication
  \begin{equation*}
    \Big(\forall_{(x:X)}\Big(\forall_{(y<x)}P(y)\Big)\to P(x)\Big)\to \forall_{(x:X)}P(x).
  \end{equation*}
  holds. Show that the rank $(\rank(A,B),\prec)$ of $\W(A,B)$ is well-founded.\index{rank!of W-type!is well-founded}\index{W-type!rank!is well-founded}
  \item A strictly ordered set $(X,<)$ is said to be \define{extensional}\index{strict ordering!extensional|textbf}\index{extensional strict ordering|textbf} if the logical equivalence
  \begin{equation*}
    (x=y)\leftrightarrow\forall_{(z:X)}\,(z<x)\leftrightarrow(z<y)
  \end{equation*}
  holds for any $x,y:X$. Show that the rank $(\rank(A,B),\prec)$ of $\W(A,B)$ is extensional.\index{rank!of W-type!is extensional}\index{W-type!rank!is extensional}
  \end{subexenum}
\end{exercises}
\index{inductive type|)}
\index{W-type|)}


%%% Local Variables:
%%% mode: latex
%%% TeX-master: "hott-intro"
%%% End:



%\input{algebra}
%\input{real-numbers}
\cleardoublepage

%%% Local Variables:
%%% mode: latex
%%% TeX-master: "hott-intro"
%%% End:
