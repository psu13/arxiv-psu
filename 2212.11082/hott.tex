% AUTHOR: Egbert Rijke
% LAST MODIFIED: February 2020

\makeatletter


% Display format for sections
%\crefformat{section}{\S#2#1#3}
%\Crefformat{section}{Section~#2#1#3}
%\crefrangeformat{section}{\S\S#3#1#4--#5#2#6}
%\Crefrangeformat{section}{Sections~#3#1#4--#5#2#6}
%\crefmultiformat{section}{\S\S#2#1#3}{ and~#2#1#3}{, #2#1#3}{ and~#2#1#3}
%\Crefmultiformat{section}{Sections~#2#1#3}{ and~#2#1#3}{, #2#1#3}{ and~#2#1#3}
%\crefrangemultiformat{section}{\S\S#3#1#4--#5#2#6}{ and~#3#1#4--#5#2#6}{, #3#1#4--#5#2#6}{ and~#3#1#4--#5#2#6}
%\Crefrangemultiformat{section}{Sections~#3#1#4--#5#2#6}{ and~#3#1#4--#5#2#6}{, #3#1#4--#5#2#6}{ and~#3#1#4--#5#2#6}

% Display format for appendices
%\crefformat{appendix}{Appendix~#2#1#3}
%\Crefformat{appendix}{Appendix~#2#1#3}
%\crefrangeformat{appendix}{Appendices~#3#1#4--#5#2#6}
%\Crefrangeformat{appendix}{Appendices~#3#1#4--#5#2#6}
%\crefmultiformat{appendix}{Appendices~#2#1#3}{ and~#2#1#3}{, #2#1#3}{ and~#2#1#3}
%\Crefmultiformat{appendix}{Appendices~#2#1#3}{ and~#2#1#3}{, #2#1#3}{ and~#2#1#3}
%\crefrangemultiformat{appendix}{Appendices~#3#1#4--#5#2#6}{ and~#3#1#4--#5#2#6}{, #3#1#4--#5#2#6}{ and~#3#1#4--#5#2#6}
%\Crefrangemultiformat{appendix}{Appendices~#3#1#4--#5#2#6}{ and~#3#1#4--#5#2#6}{, #3#1#4--#5#2#6}{ and~#3#1#4--#5#2#6}

%\crefname{part}{Part}{Parts}

%\crefformat{paragraph}{\S#2#1#3}
%\Crefformat{paragraph}{Paragraph~#2#1#3}
%\crefrangeformat{paragraph}{\S\S#3#1#4--#5#2#6}
%\Crefrangeformat{paragraph}{Paragraphs~#3#1#4--#5#2#6}
%\crefmultiformat{paragraph}{\S\S#2#1#3}{ and~#2#1#3}{, #2#1#3}{ and~#2#1#3}
%\Crefmultiformat{paragraph}{Paragraphs~#2#1#3}{ and~#2#1#3}{, #2#1#3}{ and~#2#1#3}
%\crefrangemultiformat{paragraph}{\S\S#3#1#4--#5#2#6}{ and~#3#1#4--#5#2#6}{, #3#1#4--#5#2#6}{ and~#3#1#4--#5#2#6}
%\Crefrangemultiformat{paragraph}{Paragraphs~#3#1#4--#5#2#6}{ and~#3#1#4--#5#2#6}{, #3#1#4--#5#2#6}{ and~#3#1#4--#5#2#6}

% Display format for figures
%\crefname{figure}{Figure}{Figures}
%\crefname{table}{Table}{Tables}

% Use cleveref instead of hyperref's \autoref
\let\autoref\cref


%%%% EQUATION NUMBERING %%%%

% The following hack uses the single theorem counter to number
% equations as well, so that we don't have both Theorem 1.1 and
% equation (1.1).

%\let\c@equation\c@thm

\numberwithin{equation}{subsection}


%%%% ENUMERATE NUMBERING %%%%

% Number the first level of enumerates as (i), (ii), ...
%\renewcommand{\theenumi}{(\roman{enumi})}
%\renewcommand{\labelenumi}{\theenumi}
% \def\makeRRlabeldot#1{\hss\llap{#1}}
\renewcommand\theenumi{{\rm (\roman{enumi})}}
\renewcommand\theenumii{{\rm (\alph{enumii})}}
\renewcommand\theenumiii{{\rm (\arabic{enumiii})}}
\renewcommand\theenumiv{{\rm (\Alph{enumiv})}}

%%%% CITATIONS %%%%

% \let \cite \citep

%%%% INDEX %%%%

%\newcommand{\footstyle}[1]{{\hyperpage{#1}}n} % If you index something that is in a footnote
%\newcommand{\defstyle}[1]{\textbf{\hyperpage{#1}}}  % Style for pageref to a definition

%%% Term being defined
\newcommand{\define}[1]{\textbf{#1}}
\newcommand{\quasidefine}[1]{\emph{#1}}
\newcommand{\indexdef}[1]{\index{#1|defstyle}}   % Index a definition
\newcommand{\indexfoot}[1]{\index{#1|footstyle}} % Index a term in a footnote
\newcommand{\indexsee}[2]{\index{#1|see{#2}}}    % Index "see also"

%% Use \symlabel instead of \label to mark a pageref that you need in the index of symbols
\newcounter{symindex}
\newcommand{\symlabel}[1]{\refstepcounter{symindex}\label{#1}}

%%%%%%%%%%%%%%%%%%%%%%%%%%%%%%%%%%%%%%%%%%%%%%%%%%%%%%%%%%%%%%%%%%%%%%%%%%%%%%%%
%%%% MACROS FOR NOTATION %%%%
% Use these for any notation where there are multiple options.

%%% Definitional equality (used infix) %%%
\newcommand{\jdeq}{\doteq}      % An equality judgment
\newcommand{\defeq}{\coloneqq} %{\vcentcolon\jdeq}  % A judgmental equality currently being defined

%%%%%%%%%%%%%%%%%%%%%%%%%%%%%%%%%%%%%%%%%%%%%%%%%%%%%%%%%%%%%%%%%%%%%%%%%%%%%%%%
%%%% Type constructors and quantifiers

%%% Universal quantification of mere propositions
\newcommand{\fall}[1]{\forall_{(#1)}\,\@ifnextchar\bgroup{\fall}{}}

%%% Existential quantifier %%%
\newcommand{\exis}[1]{\exists_{(#1)}\,\@ifnextchar\bgroup{\exis}{}}

%%% Unique existence %%%
\newcommand{\uexis}[1]{\exists!_{(#1)}\,\@ifnextchar\bgroup{\uexis}{}}

%%% Dependent sums %%%
\def\smsym{\sum}
\newcommand{\@thesum}[1]{\smsym_{(#1)}}
\newcommand{\sm}[1]{\@ifnextchar\bgroup{\@sm{#1}\sm}{\@sm{#1}}}
\newcommand{\@sm}[1]{\mathchoice{{\textstyle\@thesum{#1}}}{\@thesum{#1}}{\@thesum{#1}}{\@thesum{#1}}}

%%% Dependent products %%%
\def\prdsym{\prod}
\newcommand{\@theprd}[1]{\prdsym_{(#1)}}
\newcommand{\prd}[1]{\@ifnextchar\bgroup{\@prd{#1}\prd}{\@prd{#1}}}
\newcommand{\@prd}[1]{\mathchoice{{\textstyle\@theprd{#1}}}{\@theprd{#1}}{\@theprd{#1}}{\@theprd{#1}}}

% \newcommand{\tprd}{\prd}

% Other notations related to dependent sums
\newcommand{\pairr}[1]{{\mathopen{}(#1)\mathclose{}}}
\newcommand{\Pairr}[1]{{\mathopen{}\left(#1\right)\mathclose{}}}
\newcommand{\tup}[2]{(#1,#2)}
\newcommand{\proj}[1]{\ensuremath{\mathsf{pr}_{#1}}\xspace}
\newcommand{\fst}{\ensuremath{\proj1}\xspace}
\newcommand{\snd}{\ensuremath{\proj2}\xspace}

%%%% Dependent products
%\def\prdsym{\mathchoice{{{\textstyle\prod}}}{\prod}{\prod}{\prod}}

%% Call the macro like \prd{x,y:A}{p:x=y} with any number of
%% arguments.  Make sure that whatever comes *after* the call doesn't
%% begin with an open-brace, or it will be parsed as another argument.

%%%% Implicit arguments are possible. In the above example, if x and y are to be
%%%% implicit, then one should write \prd*{x,y:A}{p:x=y}
%%%%
%%%% If one wants to use implicit arguments in the notation for product types,
%%%% a * has to be put before the argument that has to be implicit.
%%%% For example: in $\prd{x:A}*{y:B}{u:P(y)}Q(x,y,u)$, the argument y is
%%%% implicit. Any of the arguments can be made implicit this way.

%%%% First of all, we should make the command \prd search not only for a
%%%% brace, but also for a star. We introduce an auxiliary command that
%%%% determines whether the next character is a star or brace.

% \newcommand{\@ifnextchar@starorbrace}[2]
%   {\@ifnextchar*{#1}{\@ifnextchar\bgroup{#1}{#2}}}

% \newcommand{\@theprd}[1]{\prdsym_{(#1)}}
% \newcommand{\@theiprd}[1]{\prdsym_{\{#1\}}}

%%%% When encountering the \prd command, latex should determine whether it
%%%% should print implicit argument brackets or not.
%%%% First, we have the following switch. Set it to true if implicit arguments
%%%% are desired, or to false if not. Note turning off implicit arguments
%%%% might render some parts of the text harder to comprehend, since in the
%%%% text might appear $f(x)$ where we would have $f(i,x)$ without implicit
%%%% arguments.

% \DeclareOption{implicit_arguments_on}%
%   {\newcommand{\implicitargumentson}{\boolean{true}}}
% \DeclareOption{implicit_arguments_off}%
%   {\newcommand{\implicitargumentson}{\boolean{false}}}

% \ExecuteOptions{implicit_arguments_off}

% \ProcessOptions\relax

% \newcommand{\prd}{\@ifnextchar*{\@iprd}{\@prd}}
% \newcommand{\@prd}[1]
%   {\@ifnextchar@starorbrace
%     {\@tprd{#1}\prd}
%     {\@tprd{#1}}}
% \newcommand{\@tprd}[1]{%
%   \mathchoice{%
%     {{\textstyle\@theprd{#1}}}}{\@theprd{#1}}{\@theprd{#1}}{\@theprd{#1}}}

% \newcommand{\@iprd}[2]{\@ifnextchar@starorbrace%
%   {\@tiprd{#2}\prd}%
%   {\@tiprd{#2}}}
% \newcommand{\@tiprd}[1]{
%   \ifthenelse{\implicitargumentson}
%     {\@@tiprd{#1}}%\@ifnextchar\bgroup{\@tiprd}{}}
%     {\@tprd{#1}}}
% \newcommand{\@@tiprd}[1]{\mathchoice{{\textstyle\@theiprd{#1}}}{\@theiprd{#1}}{\@theiprd{#1}}{\@theiprd{#1}}}

%%% Lambda abstractions.
% Each variable being abstracted over is a separate argument.  If
% there is more than one such argument, they *must* be enclosed in
% braces.  Arguments can be untyped, as in \lam{x}{y}, or typed with a
% colon, as in \lam{x:A}{y:B}. In the latter case, the colons are
% automatically noticed and (with current implementation) the space
% around the colon is reduced.  You can even give more than one variable
% the same type, as in \lam{x,y:A}.
\def\lam#1{{\lambda}\@lamarg#1:\@endlamarg\@ifnextchar\bgroup{.\,\lam}{.\,}}
\def\@lamarg#1:#2\@endlamarg{\if\relax\detokenize{#2}\relax #1\else\@lamvar{\@lameatcolon#2},#1\@endlamvar\fi}
\def\@lamvar#1,#2\@endlamvar{(#2\,{:}\,#1)}
% \def\@lamvar#1,#2{{#2}^{#1}\@ifnextchar,{.\,{\lambda}\@lamvar{#1}}{\let\@endlamvar\relax}}
\def\@lameatcolon#1:{#1}
\let\lamt\lam
% This version silently eats any typing annotation.
\def\lamu#1{{\lambda}\@lamuarg#1:\@endlamuarg\@ifnextchar\bgroup{.\,\lamu}{.\,}}
\def\@lamuarg#1:#2\@endlamuarg{#1}

\mathchardef\usc="2D
%\newcommand{\usc}{\underline{\ }}

% Macros for dependent type theory
\newcommand{\type}{\mathrm{type}}
\newcommand{\Coq}{\textsc{Coq}\xspace}
\newcommand{\Agda}{\textsc{Agda}\xspace}
\newcommand{\NuPRL}{\textsc{NuPRL}\xspace}

% Macros for dependent function types
\newcommand{\idfunc}[1][]{\mathsf{id}_{#1}}
\newcommand{\const}{\mathsf{const}}
\newcommand{\comp}{\mathsf{comp}}
\newcommand{\blank}{\mathord{\hspace{1pt}\text{--}\hspace{1pt}}}
\newcommand{\newdef}{\mathsf{c}}

% Macros for natural numbers
\newcommand{\N}{{\mathbb{N}}}
\newcommand{\zeroN}{0_{\N}}
\newcommand{\oneN}{1_{\N}}
\newcommand{\succN}{\mathsf{succ}_{\N}}
\newcommand{\addN}{\mathsf{add_{\N}}}
\newcommand{\addzeroN}{\mathsf{add\usc{}zero}_\N}
\newcommand{\addsuccN}{\mathsf{add\usc{}succ}_\N}
\newcommand{\addpN}{\mathsf{add'_{\N}}}
\newcommand{\mulN}{\mathsf{mul}_{\N}}
\newcommand{\minN}{\mathsf{min}_{\N}}
\newcommand{\maxN}{\mathsf{max}_{\N}}
\newcommand{\indN}{\ind{\N}}

% Macros for inductive types
\newcommand{\ind}[1]{\mathsf{ind}_{#1}}
\newcommand{\comphtpy}[1]{\mathsf{comp}_{#1}}

\newcommand{\unit}{\ensuremath{\mathbf{1}}\xspace}
\newcommand{\ttt}{\star}
\newcommand{\indunit}{\mathsf{ind}_{\unit}}
\newcommand{\pt}{\mathsf{pt}}

\newcommand{\emptyt}{\emptyset}
\newcommand{\indempty}{\mathsf{ind}_{\emptyt}}
\newcommand{\exfalso}{\mathsf{ex\usc{}falso}}
\newcommand{\isempty}{\mathsf{is\usc{}empty}}

\newcommand{\bool}{\mathsf{bool}}
\newcommand{\btrue}{\mathsf{true}}
\newcommand{\bfalse}{\mathsf{false}}
\newcommand{\bneg}{\mathsf{neg}}
\newcommand{\indbool}{\mathsf{ind\usc{}bool}}
\newcommand{\negbool}{\mathsf{neg\usc{}bool}}

\newcommand{\inlsym}{{\mathsf{inl}}}
\newcommand{\inrsym}{{\mathsf{inr}}}
\newcommand{\inl}{\ensuremath\inlsym\xspace}
\newcommand{\inr}{\ensuremath\inrsym\xspace}
\newcommand{\indcoprod}{\mathsf{ind}_{+}}

\newcommand{\pair}{\mathsf{pair}}
\newcommand{\indSigma}{\mathsf{ind}_{\Sigma}}

\newcommand{\indprod}{\mathsf{ind}_{\times}}

\newcommand{\Z}{{\mathbb{Z}}}
\newcommand{\inpos}{\mathsf{in\usc{}pos}}
\newcommand{\inneg}{\mathsf{in\usc{}neg}}
\newcommand{\indZ}{\mathsf{ind}_{\mathbb{Z}}}
\newcommand{\zeroZ}{0_{\Z}}
\newcommand{\oneZ}{1_{\Z}}
\newcommand{\succZ}{\mathsf{succ}_{\Z}}
\newcommand{\predZ}{\mathsf{pred}_{\Z}}
\newcommand{\addZ}{\mathsf{add}_{\Z}}
\newcommand{\mulZ}{\mathsf{mul}_{\Z}}
\newcommand{\negZ}{\mathsf{neg}_{\Z}}

\newcommand{\lst}{\mathsf{list}}
\newcommand{\nil}{\mathsf{nil}}
\newcommand{\cons}{\mathsf{cons}}
\newcommand{\fold}{\mathsf{fold}}
\newcommand{\foldlist}{\mathsf{fold\usc{}list}}
\newcommand{\maplist}{\mathsf{map\usc{}list}}
\newcommand{\lengthlist}{\mathsf{length\usc{}list}}
\newcommand{\sumlist}{\mathsf{sum\usc{}list}}
\newcommand{\productlist}{\mathsf{product\usc{}list}}
\newcommand{\reverselist}{\mathsf{reverse\usc{}list}}
\newcommand{\concatlist}{\mathsf{concat\usc{}list}}
\newcommand{\flattenlist}{\mathsf{flatten\usc{}list}}

% Macros for identity types
\newcommand{\idsymbin}{=}
\newcommand{\idsym}{{\idsymbin}}
\newcommand{\id}[3][]{
  \@ifnextchar\bgroup
    {#2 \mathbin{\idsym_{#1}} \id[#1]{#3}}
    {#2 \mathbin{\idsym_{#1}} #3}
  }
\newcommand{\idtype}[3][]{\ensuremath{\mathsf{Id}_{#1}(#2,#3)}\xspace}
\newcommand{\idtypevar}[1]{\ensuremath{\mathsf{Id}_{#1}}\xspace}
\newcommand{\refl}[1]{\mathsf{refl}_{#1}}
\newcommand{\pathind}{\mathsf{ind\usc{}eq}}

\newcommand{\concat}{\mathsf{concat}}
\newcommand{\ctsym}{%
  \mathchoice{\mathbin{\raisebox{0.5ex}{$\displaystyle\centerdot$}}}%
             {\mathbin{\raisebox{0.5ex}{$\centerdot$}}}%
             {\mathbin{\raisebox{0.25ex}{$\scriptstyle\,\centerdot\,$}}}%
             {\mathbin{\raisebox{0.1ex}{$\scriptscriptstyle\,\centerdot\,$}}}
  }
\newcommand{\ct}[3][]{
  \@ifnextchar\bgroup
    {#2 \mathbin{\ctsym_{#1}} \ct[#1]{#3}}
    {#2 \mathbin{\ctsym_{#1}} #3}
}

\newcommand{\leftunit}{\mathsf{left\usc{}unit}}
\newcommand{\rightunit}{\mathsf{right\usc{}unit}}
\newcommand{\cohunit}{\mathsf{coh\usc{}unit}}
\newcommand{\invfunc}{\mathsf{inv}}
\newcommand{\leftinv}{\mathsf{left\usc{}inv}}
\newcommand{\rightinv}{\mathsf{right\usc{}inv}}
\newcommand{\cohinv}{\mathsf{coh\usc{}inv}}
\newcommand{\assoc}{\mathsf{assoc}}

\newcommand{\apid}{\mathsf{ap\usc{}id}}
\newcommand{\apcomp}{\mathsf{ap\usc{}comp}}
\newcommand{\aprefl}{\mathsf{ap\usc{}refl}}
\newcommand{\apinv}{\mathsf{ap\usc{}inv}}
\newcommand{\apconcat}{\mathsf{ap\usc{}concat}}

\newcommand{\tr}{\mathsf{tr}}
\newcommand{\mapfunc}[1]{\ensuremath{\mathsf{ap}_{#1}}\xspace}
\newcommand{\map}[2]{\mapfunc{#1}({#2})\xspace}
\let\Ap\map
\newcommand{\mapdepfunc}[1]{\ensuremath{\mathsf{apd}_{#1}}\xspace}
\newcommand{\mapdep}[2]{\ensuremath{\mapdepfunc{#1}\mathopen{}\left(#2\right)\mathclose{}}\xspace}
\let\apfunc\mapfunc
\let\ap\map
\let\apdfunc\mapdepfunc
\let\apd\mapdep

\newcommand{\distributiveinvconcat}{\mathsf{distributive\usc{}inv\usc{}concat}}
\newcommand{\invcon}{\mathsf{inv\usc{}con}}
\newcommand{\coninv}{\mathsf{con\usc{}inv}}
\newcommand{\lift}{\mathsf{lift}}
\newcommand{\leftunitlawaddN}{\mathsf{left\usc{}unit\usc{}law\usc{}add_\N}}
\newcommand{\rightunitlawaddN}{\mathsf{right\usc{}unit\usc{}law\usc{}add_\N}}
\newcommand{\leftsuccessorlawaddN}{\mathsf{left\usc{}successor\usc{}law\usc{}add_\N}}
\newcommand{\rightsuccessorlawaddN}{\mathsf{right\usc{}successor\usc{}law\usc{}add_\N}}
\newcommand{\associativeaddN}{\mathsf{associative\usc{}add_\N}}
\newcommand{\commutativeaddN}{\mathsf{commutative\usc{}add_\N}}

% Macros for universes
\newcommand{\UU}{\mathcal{U}}
\newcommand{\VV}{\mathcal{V}}
\newcommand{\Ty}{\mathcal{T}}
\newcommand{\EqN}{\mathsf{Eq}_{\N}}
\newcommand{\reflEqN}{\mathsf{refl\usc{}Eq}_\N}
\newcommand{\indEqN}{\mathsf{ind\usc{}Eq}_\N}
\newcommand{\EqBool}{\mathsf{Eq\usc{}bool}}
\newcommand{\universalfam}{\Ty}
\newcommand{\distN}{\mathsf{dist}_\N}
\newcommand{\vect}{\mathsf{vector}}
\newcommand{\mat}{\mathsf{matrix}}
\newcommand{\nilvertical}{\mathsf{nil\usc{}vert}}
\newcommand{\nilhorizontal}{\mathsf{nil\usc{}horiz}}
\newcommand{\consvertical}{\mathsf{cons\usc{}vert}}
\newcommand{\conshorizontal}{\mathsf{cons\usc{}horiz}}
\newcommand{\transpose}{\mathsf{transpose}}

% Macros for modular arithmetic
\newcommand{\cgrN}{\mathsf{cong}_\N}%The congruence relation
\renewcommand{\mod}{\ \ \mathsf{mod}\ }
\newcommand{\Fin}[1]{\ensuremath{\mathsf{Fin}}_{#1}}%The finite sets
\newcommand{\indFin}{\mathsf{ind\usc{}Fin}}
\newcommand{\natFin}{\iota}%The inclusion of finite sets in N
\newcommand{\negtwoFin}{\mathsf{neg\usc{}two}}%-2 in Fin
\newcommand{\negoneFin}{\mathsf{neg\usc{}one\usc}}%-1 in Fin
\newcommand{\zeroFin}{\mathsf{zero}}%0 in Fin
\newcommand{\oneFin}{\mathsf{one}}%1 in Fin
\newcommand{\skipzeroFin}{\mathsf{skip\usc{}zero}}%The function that skips 0
\newcommand{\succFin}{\mathsf{succ}}%The successor function on Fin
\newcommand{\skipnegtwoFin}{\mathsf{skip\usc{}neg\usc{}two}}%The function that skips -2
\newcommand{\predFin}{\mathsf{pred}}%The predecessor function
\newcommand{\EqFin}{\mathsf{Eq}}%Observational equality on Fin
\newcommand{\reflEqFin}{\mathsf{refl\usc{}Eq}}%Reflexivity of observational Eq
\newcommand{\swapFin}{\mathsf{swap}}
\newcommand{\addFin}{\mathsf{add}}
\newcommand{\negFin}{\mathsf{neg}}
\newcommand{\mulFin}{\mathsf{mul}}
\newcommand{\with}{~\mathit{with}~}
\newcommand{\reverseFin}{\mathsf{reverse}}
\newcommand{\skipFin}{\delta}
\newcommand{\repeatFin}{\sigma}
\newcommand{\basedN}[1]{\N_{#1}}
\newcommand{\constantbasedN}[1]{\mathsf{constant}_{\basedN{#1}}}
\newcommand{\unaryopbasedN}[1]{\mathsf{unary\usc{}op}_{\basedN{#1}}}
\newcommand{\EqbasedN}[1]{\mathsf{Eq_{\basedN{k}}}}
\newcommand{\convertbasedN}[1]{f_{#1}}
\newcommand{\isfinitelycyclic}{\mathsf{is\usc{}finitely\usc{}cyclic}}
\newcommand{\issplitsurjective}{\mathsf{is\usc{}split\usc{}surjective}}
\newcommand{\classicalFin}{\mathsf{classical\usc{}Fin}}

% Macros for elementary number theory
\DeclareMathOperator{\lcm}{lcm}
\newcommand{\isdecidable}{\mathsf{is\usc{}decidable}}
\newcommand{\isdec}{\isdecidable}
\newcommand{\hasdecidableequality}{\mathsf{has\usc{}decidable\usc{}eq}}
\newcommand{\collatz}{\mathsf{collatz}}
\newcommand{\decidableProp}{\mathsf{DProp}}
\newcommand{\decunit}{\mathsf{dec}_\unit}
\newcommand{\decemptyt}{\mathsf{dec}_\emptyt}
\renewcommand{\gcd}{\mathsf{gcd}}
\newcommand{\isminimal}{\mathsf{is\usc{}minimal}}
\newcommand{\ismaximal}{\mathsf{is\usc{}maximal}}
\newcommand{\islowerbound}{\mathsf{is\usc{}lower\usc{}bound}}
\newcommand{\isupperbound}{\mathsf{is\usc{}upper\usc{}bound}}
\newcommand{\wellorderingprinciple}{\mathsf{w}}
\newcommand{\ismultipleofgcd}{M}
\newcommand{\isbounded}{\mathsf{is\usc{}bounded}}
\newcommand{\maximum}{\mathsf{maximum}}
\newcommand{\minimum}{\mathsf{minimum}}
\newcommand{\isprime}{\mathsf{is\usc{}prime}}
\newcommand{\strongindN}{\mathsf{strong\usc{}ind}_\N}
\newcommand{\reflleqN}{\mathsf{refl\usc{}leq}_\N}
\newcommand{\ordindN}{\mathsf{ord\usc{}ind}_\N}
\newcommand{\gcdeuclid}{\mathsf{gcd}_E}
\newcommand{\isgcd}{\mathsf{is\usc{}gcd}}
\newcommand{\isproperdivisor}{\mathsf{is\usc{}proper\usc{}divisor}}
\newcommand{\isdecidableisprime}{\mathsf{is\usc{}decidable\usc{}is\usc{}prime}}
\newcommand{\booleanization}{\mathsf{booleanization}}
\newcommand{\booleanreflection}{\mathsf{reflect}}
\let\reflect\booleanreflection
\newcommand{\isprimethirtyseven}{\mathsf{is\usc{}prime\usc{}37}}
\newcommand{\primefunction}{\mathsf{prime}}
\newcommand{\primefactors}{\mathsf{prime\usc{}factors}}
\newcommand{\primeN}{\mathsf{prime}}
\newcommand{\Eqlist}{\mathsf{Eq\usc{}list}}
\newcommand{\invmul}{\mathsf{inv\usc{}mul}}

% Macros for equivalences
\newcommand{\negnegbool}{\mathsf{neg\usc{}neg\usc{}bool}}
\newcommand{\htpy}{\sim}
\newcommand{\reflhtpy}{\mathsf{refl\usc{}htpy}}
\newcommand{\invhtpy}{\mathsf{inv\usc{}htpy}}
\newcommand{\concathtpy}{\mathsf{concat\usc{}htpy}}
\newcommand{\leftunithtpy}{\mathsf{left\usc{}unit\usc{}htpy}}
\newcommand{\rightunithtpy}{\mathsf{right\usc{}unit\usc{}htpy}}
\newcommand{\leftinvhtpy}{\mathsf{left\usc{}inv\usc{}htpy}}
\newcommand{\rightinvhtpy}{\mathsf{right\usc{}inv\usc{}htpy}}
\newcommand{\assochtpy}{\mathsf{assoc\usc{}htpy}}
\newcommand{\sections}{\mathsf{sec}}
\renewcommand{\sec}{\sections}
\newcommand{\retractions}{\mathsf{retr}}
\newcommand{\Retr}{\mathsf{Retr}}
\newcommand{\isequiv}{\mathsf{is\usc{}equiv}}
\newcommand{\eqvsym}{\simeq}    % infix symbol
\newcommand{\eqv}[2]{\@ifnextchar\bgroup{#1 \eqvsym \eqv{#2}}{#1 \eqvsym #2}}
\newcommand{\hasinverse}{\mathsf{has\usc{}inverse}}
\newcommand{\EqSigma}{\mathsf{Eq}_{\Sigma}}
\newcommand{\reflexiveEqSigma}{\mathsf{reflexive\usc{}Eq}_{\Sigma}}
\newcommand{\paireq}{\mathsf{pair\usc{}eq}}
\newcommand{\eqpair}{\mathsf{eq\usc{}pair}}
\newcommand{\assocSigma}{\mathsf{assoc\usc{}}\Sigma}
\newcommand{\swapSigma}{\mathsf{swap\usc{}}\Sigma}
\newcommand{\leftunitlawaddZ}{\mathsf{left\usc{}unit\usc{}law\usc{}add\usc{}}\Z}
\newcommand{\rightunitlawaddZ}{\mathsf{right\usc{}unit\usc{}law\usc{}add\usc{}}\Z}
\newcommand{\leftpredecessorlawaddZ}{\mathsf{left\usc{}predecessor\usc{}law\usc{}add\usc{}}\Z}
\newcommand{\rightpredecessorlawaddZ}{\mathsf{right\usc{}predecessor\usc{}law\usc{}add\usc{}}\Z}
\newcommand{\leftsuccessorlawaddZ}{\mathsf{left\usc{}successor\usc{}law\usc{}add\usc{}}\Z}
\newcommand{\rightsuccessorlawaddZ}{\mathsf{right\usc{}successor\usc{}law\usc{}add\usc{}}\Z}
\newcommand{\assocaddZ}{\mathsf{assoc\usc{}add\usc{}}\Z}
\newcommand{\commaddZ}{\mathsf{comm\usc{}add\usc{}}\Z}
\newcommand{\leftinverselawaddZ}{\mathsf{left\usc{}inverse\usc{}law\usc{}add\usc{}}\Z}
\newcommand{\rightinverselawaddZ}{\mathsf{right\usc{}inverse\usc{}law\usc{}add\usc{}}\Z}
\newcommand{\starvalue}{\mathsf{star\usc{}value}}

% Macros for contractible types
\newcommand{\indsing}{\mathsf{ind\usc{}sing}}
\newcommand{\compsing}{\mathsf{comp\usc{}sing}}
\let\singind\indsing
\let\singcomp\compsing
\newcommand{\evrefl}{\mathsf{ev\usc{}refl}}
\newcommand{\iscontr}{\ensuremath{\mathsf{is\usc{}contr}}}
\newcommand{\ev}{\mathsf{ev}}
\newcommand{\evpt}{\mathsf{ev\usc{}pt}}
\newcommand{\evpair}{\mathsf{ev\usc{}pair}}
\newcommand{\evid}{\mathsf{ev\usc{}id}}
\newcommand{\fibf}[1]{\mathsf{fib}_{#1}}
\newcommand{\fib}[2]{\mathsf{fib}_{#1}(#2)}
\newcommand{\Fib}[2]{\mathsf{fib}_{#1}\left(#2\right)}
\newcommand{\Eqfib}{\mathsf{Eq\usc{}fib}}
\newcommand{\iscohinvertible}{\mathsf{is\usc{}coh\usc{}invertible}}
\newcommand{\nathtpy}{\mathsf{nat\usc{}htpy}}

% Macros for fundamental theorem of identity types
\newcommand{\tot}[2][]{\ensuremath{\mathsf{tot}_{#1}(#2)}}
\newcommand{\isemb}{\mathsf{is\usc{}emb}}
\newcommand{\Eqcoprod}{\mathsf{Eq\usc{}copr}}
\newcommand{\Eqcoprodeq}{\mathsf{Eq\usc{}copr\usc{}eq}}
\newcommand{\cohreduction}{\mathsf{coh\usc{}reduction}}
\newcommand{\fibtriangle}{\mathsf{fib\usc{}triangle}}
\newcommand{\isinj}{\mathsf{is\usc{}inj}}

% Macros for truncation levels
\newcommand{\isprop}{\ensuremath{\mathsf{is\usc{}prop}}}
\newcommand{\prop}{\mathsf{Prop}}
\newcommand{\isset}{\ensuremath{\mathsf{is\usc{}set}}}
\newcommand{\axiomK}{\mathsf{axiom\usc{}K}}
\newcommand{\istrunc}[1]{\mathsf{is\usc{}trunc}_{#1}}
\newcommand{\typele}[1]{\UU^{\leq #1}}
\newcommand{\typeleU}[1]{\ensuremath{{#1}\text-\mathsf{Type}_\UU}\xspace}
\newcommand{\typelep}[1]{\ensuremath{{(#1)}\text-\mathsf{Type}}\xspace}
\newcommand{\typelepU}[1]{\ensuremath{{(#1)}\text-\mathsf{Type}_\UU}\xspace}
\let\ntype\typele
\let\ntypeU\typeleU
\let\ntypep\typelep
\let\ntypepU\typelepU
\newcommand{\set}{\ensuremath{\mathsf{Set}}\xspace}
\newcommand{\setU}{\ensuremath{\mathsf{Set}_\UU}\xspace}
\newcommand{\propU}{\ensuremath{\mathsf{Prop}_\UU}\xspace}
\newcommand{\T}{\mathbb{T}}
\newcommand{\negtwoT}{-2_{\T}}
\newcommand{\succT}{\mathsf{succ}_{\T}}
\newcommand{\isisolated}{\mathsf{is\usc{}isolated}}

% Macros for function extensionality
\newcommand{\eqhtpy}{\mathsf{eq\usc{}htpy}}
\newcommand{\htpyeq}{\mathsf{htpy\usc{}eq}}
\newcommand{\funext}{\mathsf{funext}}
\newcommand{\evtruefalse}{\mathsf{ev\usc{}true\usc{}false}}
\newcommand{\evunit}{\mathsf{ev\usc{}unit}}
\newcommand{\precomp}[1]{{#1}^\ast}
\newcommand{\pathsplit}{\mathsf{is\usc{}path\usc{}split}}
\newcommand{\choice}{\mathsf{choice}}
\newcommand{\isinvertible}{\mathsf{is\usc{}invertible}}
\newcommand{\evinlinr}{\mathsf{ev\usc{}inl\usc{}inr}}
\newcommand{\homslice}{\mathsf{hom}}

% Macros for univalence
\newcommand{\eqequiv}{\mathsf{eq\usc{}equiv}}
\newcommand{\equiveq}{\mathsf{equiv\usc{}eq}}
\newcommand{\univalence}{\ensuremath{\mathsf{univalence}}\xspace} % the full axiom
\newcommand{\ptdeqvsym}{\simeq_\ast}    % infix symbol
\newcommand{\ptdeqv}[2]{
  \@ifnextchar\bgroup
    {#1 \ptdeqvsym \ptdeqv{#2}}
    {#1 \ptdeqvsym #2}
}
\newcommand{\isfunction}{\mathsf{is\usc{}function}}
\newcommand{\iffeq}{\mathsf{iff\usc{}eq}}
\newcommand{\opp}[1]{#1^{\mathsf{op}}}
\newcommand{\graph}{\mathsf{graph}}
\newcommand{\iffprop}{\leftrightarrow}
\newcommand{\ac}{\mathsf{ac}}
\newcommand{\AC}{\mathsf{AC}}
\newcommand{\LEM}{\mathsf{LEM}}
\newcommand{\preord}{\mathsf{PreOrd}}
\newcommand{\posetreflection}[1]{\trunc{\mathsf{Pos}}{#1}}
\newcommand{\issmall}{\mathsf{is\usc{}small}}
\newcommand{\islocallysmall}{\mathsf{is\usc{}locally\usc{}small}}
\newcommand{\isdecidableemb}{\mathsf{is\usc{}decidable\usc{}emb}}

% Macros for propositional truncation
\newcommand{\issurj}{\mathsf{is\usc{}surj}}
\newcommand{\F}{\mathbb{F}}
\newcommand{\BS}{BS}
\newcommand{\carthomFam}{\mathsf{cart\usc{}hom\usc{}Fam}}
\newcommand{\indbrck}{\mathsf{ind\usc{}ptrunc}}
\newcommand{\isfinite}{\mathsf{is\usc{}finite}}
\newcommand{\BAut}{\mathsf{BAut}}
\newcommand{\isweaklyconstant}{\mathsf{is\usc{}weakly\usc{}constant}}
\newcommand{\brckcheck}[1]{\brck{#1}\,\check{}\,}
\newcommand{\source}{\mathsf{source}}
\newcommand{\target}{\mathsf{target}}
\newcommand{\pathI}{\mathsf{path}}
\newcommand{\I}{\mathbb{I}}
\newcommand{\numberofelements}{\mathsf{number\usc{}of\usc{}elements}}
\DeclareRobustCommand{\stirling}{\genfrac\{\}{0pt}{}}

% Macros for finite types
\newcommand{\cnt}{\mathsf{count}}
\renewcommand{\complement}[1]{{#1}^{\mathsf{c}}}
\newcommand{\fallingfactorial}[2]{(#1)_{#2}}
\newcommand{\numberofsurjectivemaps}[2]{\mathsf{S}(#1,#2)}

% Macros for groups in univalent mathematics
\DeclareMathOperator{\Aut}{Aut}
\newcommand{\hasassociativemul}{\mathsf{has\usc{}associative\usc{}mul}}
\newcommand{\semigroup}{\mathsf{Semigroup}}
\newcommand{\monoid}{\mathsf{Monoid}}
\newcommand{\group}{\mathsf{Group}}
\newcommand{\isunital}{\mathsf{is\usc{}unital}}
\newcommand{\isgroup}{\mathsf{is\usc{}group}}
\renewcommand{\hom}{\mathsf{hom}}
\newcommand{\isiso}{\mathsf{is\usc{}iso}}
\newcommand{\isoeq}{\mathsf{iso\usc{}eq}}
\newcommand{\iso}{\mathsf{iso}}
\newcommand{\Set}{\mathsf{Set}}
\newcommand{\binomtype}{\binom}
\newcommand{\dbinomtype}[3][]{\binom{#2}{#3}_{#1}}
\newcommand{\demb}{\hookrightarrow_{\mathsf{d}}}
\newcommand{\emb}{\hookrightarrow}

% Macros for set quotients
\newcommand{\eqrel}{\mathsf{Eq\usc{}Rel}}
\newcommand{\isequivalenceclass}{\mathsf{is\usc{}equivalence\usc{}class}}
\newcommand{\ispartition}{\mathsf{is\usc{}partition}}
\newcommand{\partition}{\mathsf{Partition}}
\newcommand{\isweaklypathconstant}{\mathsf{is\usc{}weakly\usc{}path\usc{}constant}}
\newcommand{\Q}{\mathbb{Q}}
\newcommand{\ischoiceofrepresentatives}{\mathsf{is\usc{}choice\usc{}of\usc{}reps}}

% Macros for the circle
\newcommand{\gen}{\mathsf{gen}}
\newcommand{\dgen}{\mathsf{dgen}}
\newcommand{\mulcircle}{\mathsf{mul}_{\sphere{1}}}
\newcommand{\basemulcircle}{\mathsf{base\usc{}mul}_{\sphere{1}}}
\newcommand{\loopmulcircle}{\mathsf{loop\usc{}mul}_{\sphere{1}}}
\newcommand{\htpyidcircle}{H}
\newcommand{\basehtpyidcircle}{\alpha}
\newcommand{\loophtpyidcircle}{\beta}
\newcommand{\invcircle}{\mathsf{inv}_{\sphere{1}}}
\newcommand{\evbase}{\mathsf{ev\usc{}base}}
\newcommand{\segmenthelix}{\mathsf{segment\usc{}helix}}
\newcommand{\universalcovercircle}{\mathcal{E}_{\sphere{1}}}

% Macros for the fundamental cover of the circle

% Macros for pullbacks
\newcommand{\ispullback}{\mathsf{is\usc{}pullback}}
\newcommand{\cone}{\mathsf{cone}}
\newcommand{\conemap}{\mathsf{cone\usc{}map}}
\newcommand{\gap}{\mathsf{gap}}
\newcommand{\fibsquare}{\mathsf{fib\usc{}sq}}

% Macros for pushouts
\newcommand{\cocone}{\mathsf{cocone}}
\newcommand{\coconemap}{\mathsf{cocone\usc{}map}}
\newcommand{\gluesym}{{\mathsf{glue}}}
\newcommand{\jglue}{\ensuremath\gluesym\xspace}
\newcommand{\glue}{\mathsf{glue}}
\newcommand{\susp}{\Sigma}
\newcommand{\suspspec}{\susp^\infty}
\newcommand{\north}{\mathsf{N}}
\newcommand{\south}{\mathsf{S}}
\newcommand{\merid}{\mathsf{merid}}
\newcommand{\sphere}[1]{\mathbf{S}^{#1}}
\newcommand{\Sn}{\mathbf{S}}
\newcommand{\base}{\ensuremath{\mathsf{base}}\xspace}
\newcommand{\lloop}{\ensuremath{\mathsf{loop}}\xspace}
\newcommand{\smashpr}[2]{#1 \wedge #2}
\newcommand{\wedgepr}[2]{#1 \vee #2}
\newcommand{\cuppr}[2]{#1 \mathbin{\cupsym} #2}
\newcommand{\cupsym}{\smallsmile}

% Macros for sequential colimits
\newcommand{\seqin}{\mathsf{in\usc{}seq}}
\newcommand{\inseq}{\mathsf{in\usc{}seq}}

% Macros for connected types
\newcommand{\isconn}{\mathsf{is\usc{}conn}}

% Macros for long exact sequences
\newcommand{\mulsphere}[1]{\mathsf{mul}_{\sphere{#1}}}

% Macros for undirected graphs

\newcommand{\unorderedpair}{\mathsf{unordered\usc{}pair}}
\newcommand{\unorderedtuple}{\mathsf{unordered\usc{}tuple}}
\newcommand{\digraph}{\mathsf{Directed\usc{}Graph}}
\newcommand{\undirectedgraph}{\mathsf{Undirected\usc{}Graph}}

% Macros for univalent combinatorics

\newcommand{\intercalate}{\mathsf{intercalate}}
\newcommand{\latinsquare}{\mathsf{Latin\usc{}Square}}

% Macros for later parts of the book

\newcommand{\cart}{\mathsf{cart}}
\newcommand{\esssmall}{\mathsf{ess\usc{}small}}
\newcommand{\locsmall}{\mathsf{loc\usc{}small}}
\newcommand{\isclassified}{\mathsf{is\usc{}classified}}
\newcommand{\tottriangle}{\mathsf{tot\usc{}triangle}}
\newcommand{\idp}{\mathsf{id}^\ast}
\newcommand{\FibSeq}{\mathsf{Fib\usc{}Seq}}
\newcommand{\equivpt}{\mathsf{equiv\usc{}pt}}
\newcommand{\ptequiv}{\mathsf{pt\usc{}equiv}}
\newcommand{\truncunit}[1]{|#1|}
\DeclareMathOperator{\im}{im} % the image
\newcommand{\leftwhisker}{\mathbin{{\ct}_{\ell}}}
\newcommand{\rightwhisker}{\mathbin{{\ct}_{r}}}
\newcommand{\hct}{\star}
\newcommand{\quotientrestr}{\mathsf{quotient\usc{}restriction}}
\newcommand{\pcttype}[2]{\mathsf{Type}_{#1}^{#2}}
\newcommand{\ft}{\sigma}
\newcommand{\bd}{\tau}
\newcommand{\GSet}[1][G]{#1\mathsf{\usc{}Set}}
\newcommand{\Grp}{\mathsf{Group}}
\newcommand{\sslash}{/\mkern-6mu/}
\newcommand{\EM}{\mathsf{EM}}
\newcommand{\iskuratowskyfinite}{\mathsf{is\usc{}kuratowsky\usc{}finite}}
\newcommand{\unorderedpairs}{\mathsf{unordered\usc{}pairs}}
\newcommand{\ispifinite}[1]{\mathsf{is\usc{}}\pi_{#1}\mathsf{\usc{}finite}}
\newcommand{\Tree}{\mathsf{Tree}}
\newcommand{\partitiontype}{\mathsf{Partition\usc{}Type}}
\newcommand{\ferrersdiagram}{\mathsf{Ferrers\usc{}Diagram}}
\newcommand{\isproper}{\mathsf{is\usc{}proper}}
\newcommand{\cycle}{\mathsf{Cycle}}
\newcommand{\polygon}{\mathsf{Polygon}}
\newcommand{\blockdecomposition}{\mathsf{Decomposition}}
\newcommand{\walk}{\mathsf{walk}}
\newcommand{\trail}{\mathsf{trail}}
\newcommand{\Sigmadecomposition}{\Sigma\mathsf{\usc{}decomp}}
\newcommand{\indecomposabletype}{\mathsf{Indecomposable\usc{}Type}}

% Macros for W-types
\newcommand{\M}{\mathbb{M}}
\newcommand{\supW}{\collect}
\newcommand{\collect}{\tree}
\newcommand{\tree}{\mathsf{tree}}
\newcommand{\wcollect}[2]{[ #1\mid #2]}
\newcommand{\Wcollect}[2]{\left[ #1 \mid #2 \right]}
\newcommand{\ecollect}[2]{\{ #1\mid #2\}}
\newcommand{\Ecollect}[2]{\left\{ #1\mid #2\}}
\newcommand{\W}{\mathsf{W}}
\newcommand{\indW}{\mathsf{ind_W}}
\newcommand{\EqW}{\mathsf{Eq_W}}
\newcommand{\reflEqW}{\mathsf{refl\usc{}Eq_W}}
\newcommand{\hasrank}{\mathsf{has\usc{}rank}}
\newcommand{\issmallmultiset}[1]{\mathsf{is\usc{}small_{\M_{#1}}}}
\newcommand{\multiset}[1]{\mathbb{M}_{#1}}
\newcommand{\BinTree}{\mathsf{Bin\usc{}Tree}}
\newcommand{\leaf}{\mathsf{leaf}}
\newcommand{\node}{\mathsf{node}}
\newcommand{\collectBinTree}{\mathsf{bin\usc{}tree}}
\newcommand{\planarBinTree}{\mathbf{T}_2}
\newcommand{\arity}{\mathsf{arity}}
\newcommand{\component}{\mathsf{component}}
\newcommand{\symbolW}{\mathsf{symbol}}
\newcommand{\prearity}{\symbolW}
\newcommand{\rank}{\mathcal{R}}
\newcommand{\isconstantW}{\mathsf{is\usc{}constant_W}}
\newcommand{\yggdrasil}[1][\UU]{\mathbb{Y}_{#1}}

%%%%%%%%%%%%%%%%%%%%%%%%%%%%%%%%%%%%%%%%%%%%%%%%%%%%%%%%%%%%%%%%%%%%%%%%%%%%%%%%
%%%% Basic types

%%%%%%%%%%%%%%%%%%%%%%%%%%%%%%%%%%%%%%%%%%%%%%%%%%%%%%%%%%%%%%%%%%%%%%%%%%%%%%%%
%%%% Universes, subuniverses and modal operators


%Pointed types
\newcommand{\pointed}[1]{\ensuremath{#1_\bullet}}
%%% (pointed) mapping spaces
\newcommand{\Map}{\mathsf{Map}}


\newcommand{\trunc}[2]{\Vert #2\Vert_{#1}}
\newcommand{\Trunc}[2]{\Big\Vert #2\Big\Vert_{#1}}
\newcommand{\truncf}[1]{\Vert \blank \Vert_{#1}}
\newcommand{\tproj}[3][]{\mathopen{}\left|#3\right|_{#2}^{#1}\mathclose{}}
\newcommand{\tprojf}[2][]{|\blank|_{#2}^{#1}}
\def\pizero{\trunc0}
\newcommand{\brck}[1]{\trunc{}{#1}}
\newcommand{\Brck}[1]{\Trunc{}{#1}}
\newcommand{\bproj}[1]{\tproj{}{#1}}
\newcommand{\bprojf}{\tprojf{}}

%%% modalities %%%
\newcommand{\modal}{\ensuremath{\ocircle}}
\newcommand{\modaltype}{\ensuremath{\type_\modal}}
\newcommand{\modalUU}{\ensuremath{\UU_\modal}}
% \newcommand{\ism}[1]{\ensuremath{\mathsf{is}_{#1}}}
% \newcommand{\ismodal}{\ism{\modal}}
% \newcommand{\existsmodal}{\ensuremath{{\exists}_{\modal}}}
% \newcommand{\existsmodalunique}{\ensuremath{{\exists!}_{\modal}}}
% \newcommand{\modalfunc}{\textsf{\modal-fun}}
% \newcommand{\Ecirc}{\ensuremath{\mathsf{E}_\modal}}
% \newcommand{\Mcirc}{\ensuremath{\mathsf{M}_\modal}}
\newcommand{\mreturn}{\ensuremath{\eta}}
\let\project\mreturn
%\newcommand{\mbind}[1]{\ensuremath{\hat{#1}}}
\newcommand{\ext}{\mathsf{ext}}
%\newcommand{\mmap}[1]{\ensuremath{\bar{#1}}}
%\newcommand{\mjoin}{\ensuremath{\mreturn^{-1}}}
% Subuniverse
\renewcommand{\P}{\ensuremath{\type_{P}}\xspace}
\newcommand{\islocal}[1]{\mathsf{is\usc{}local}_{#1}}
\newcommand{\modalunit}[1][]{{\eta_{#1}}}
\newcommand{\plus}[1]{{{#1}^+}}
\newcommand{\localization}[1]{\mathcal{L}_{#1}}
\newcommand{\ismodal}{\ensuremath{\mathsf{is\usc{}modal}}}
\newcommand{\natunit}{\mathsf{nat\usc{}unit}}
\newcommand{\isetale}{\mathsf{is\usc{}etale}}
\newcommand{\etmap}{\mathsf{etale\usc{}map}}

% Projection and extension for truncations
\let\extendsmb\ext
\newcommand{\extend}[1]{\extendsmb(#1)}

%%% Ordinals and cardinals
\newcommand{\card}{\ensuremath{\mathsf{Card}}\xspace}
\newcommand{\ord}{\ensuremath{\mathsf{Ord}}\xspace}
\newcommand{\ordsl}[2]{{#1}_{/#2}}

%%% Some categories
\newcommand{\uset}{\ensuremath{\mathcal{S}et}\xspace}
\newcommand{\ucat}{\ensuremath{{\mathcal{C}at}}\xspace}
\newcommand{\urel}{\ensuremath{\mathcal{R}el}\xspace}
\newcommand{\uhilb}{\ensuremath{\mathcal{H}ilb}\xspace}
\newcommand{\utype}{\ensuremath{\mathcal{T}\!ype}\xspace}

%%%%%%%%%%%%%%%%%%%%%%%%%%%%%%%%%%%%%%%%%%%%%%%%%%%%%%%%%%%%%%%%%%%%%%%%%%%%%%%%
%%%% Miscellaneous stuff

\newcommand{\fibration}{\twoheadrightarrow}
\newcommand{\Sq}[1]{\mathrm{Sq}^{#1}}


% Function definition with domain and codomain
\newcommand{\function}[4]{\left\{\begin{array}{rcl}#1 &
      \longrightarrow & #2 \\ #3 & \longmapsto & #4 \end{array}\right.}


\newcommand{\Hq}{\mathbb{H}}
\newcommand{\Oc}{\mathbb{O}}
\newcommand{\rprojective}[1]{\mathbb{R}\mathsf{P}^{#1}}
\newcommand{\cprojective}[1]{\mathbb{C}\mathsf{P}^{#1}}
\newcommand{\hprojective}[1]{\mathbb{H}\mathsf{P}^{#1}}
\newcommand{\oprojective}[1]{\mathbb{O}\mathsf{P}^{#1}}
\newcommand{\antipodal}[1]{\mathsf{antipodal}_{#1}}


%%% Nameless objects
\newcommand{\nameless}{\mathord{\hspace{1pt}\underline{\hspace{1ex}}\hspace{1pt}}}

\newcommand{\mentalpause}{\medskip} % Use for "mental" pause, instead of \smallskip or \medskip


%%%%%%%%%%%%%%%%%%%%%%%%%%%%%%%%%%%%%%%%%%%%%%%%%%%%%%%%%%%%%%%%%%%%%%%%%%%%%%%%
%%%% The following is a big unorganized list of new macros that we use in the
%%%% notes. 

\newcommand{\tfW}{\mathsf{W}}
\newcommand{\tfM}{\mathsf{M}}
\newcommand{\mfM}{\modelfont{M}}
\newcommand{\mfN}{\modelfont{N}}
\newcommand{\tfctx}{\mathsf{ctx}}
\newcommand{\mftypfunc}[1]{{\modelfont{typ}^{#1}}}
\newcommand{\mftyp}[2]{{\mftypfunc{#1}(#2)}}
\newcommand{\tftypfunc}[1]{{\mathsf{typ}^{#1}}}
\newcommand{\tftyp}[2]{{\tftypfunc{#1}(#2)}}
\newcommand{\hfibfunc}[1]{\mathsf{fib}_{#1}}
\newcommand{\mappingcone}[1]{\mathcal{C}_{#1}}
\newcommand{\equifib}{\mathsf{equiFib}}
\newcommand{\eqfop}{\mathsf{EqF}}
\newcommand{\eqf}[1]{\eqfop(#1)}
\newcommand{\tfcolim}{\mathsf{colim}}
\newcommand{\colim}{\mathsf{colim}}
\renewcommand{\lim}{\mathsf{lim}}
\newcommand{\tflim}{\mathsf{lim}}
\newcommand{\tfdiag}{\mathsf{diag}}
\newcommand{\tfGraph}{\mathsf{Graph}}
\newcommand{\mfGraph}{\modelfont{Graph}}
\newcommand{\unitGraph}{\unit^\mfGraph}
\newcommand{\UUGraph}{\UU^\mfGraph}
\newcommand{\tfrGraph}{\mathsf{rGraph}}
\newcommand{\mfrGraph}{\modelfont{rGraph}}
\newcommand{\tflimits}{\mathsf{limits}}
\newcommand{\tfcolimits}{\mathsf{colimits}}
\newcommand{\islimiting}{\mathsf{is\usc{}limiting}}
\newcommand{\iscolimiting}{\mathsf{is\usc{}colimiting}}
\newcommand{\islimit}{\mathsf{is\usc{}limit}}
\newcommand{\iscolimit}{\mathsf{is\usc{}colimit}}
\newcommand{\pbcone}{\mathsf{cone_{pb}}}
\newcommand{\tfinj}{\mathsf{inj}}
\newcommand{\tfsurj}{\mathsf{surj}}
\newcommand{\tfepi}{\mathsf{epi}}
\newcommand{\tftop}{\mathsf{top}}
%\newcommand{\sbrck}[1]{\Vert #1\Vert}
%\newcommand{\strunc}[2]{\Vert #2\Vert_{#1}}
\newcommand{\gobjclass}{{\mathsf{U}^\mfGraph}}
\newcommand{\gcharmap}{\mathsf{fib}}
\newcommand{\diagclass}{\mathsf{T}}
\newcommand{\opdiagclass}{\op{\diagclass}}
\newcommand{\equifibclass}{\diagclass^{\eqv{}{}}}
\newcommand{\universe}{\mathsf{U}}
\newcommand{\catid}[1]{{\catfont{id}_{#1}}}
\newcommand{\isleftfib}{\mathsf{is\usc{}left\usc{}fib}}
\newcommand{\isrightfib}{\mathsf{is\usc{}right\usc{}fib}}
\newcommand{\leftLiftings}{\mathsf{leftLiftings}}
\newcommand{\rightLiftings}{\mathsf{rightLiftings}}
\newcommand{\psh}{\mathsf{Psh}}
\newcommand{\rgclass}{\mathsf{\Omega^{RG}}}
\newcommand{\terms}[2][]{| #2 |^{#1}}
\newcommand{\grconstr}[2]
             {\mathchoice % max size is textstyle size.
             {{\textstyle \int_{#1}}#2}% 
             {\int_{#1}#2}%
             {\int_{#1}#2}%
             {\int_{#1}#2}}
\newcommand{\ctxhom}[3][]{\mathsf{hom}_{#1}(#2,#3)}
\newcommand{\graphcharmapfunc}[1]{\gcharmap_{#1}}
\newcommand{\graphcharmap}[2][]{\graphcharmapfunc{#1}(#2)}
\newcommand{\tfexp}[1]{\mathsf{exp}_{#1}}
\newcommand{\tffamfunc}{\mathsf{fam}}
\newcommand{\tffam}[1]{\tffamfunc(#1)}
\newcommand{\tfev}{\mathsf{ev}}
\newcommand{\tfcomp}{\mathsf{comp}}
\newcommand{\smal}{\mathcal{S}}
\renewcommand{\modal}{{\ensuremath{\ocircle}}}
\newcommand{\piw}{\ensuremath{\Pi\mathsf{W}}} %% to be used in conjunction with -pretopos.
%\renewcommand{\sslash}{/\!\!/}
\newcommand{\mprd}[3][]{\Pi^{#1}(#2,#3)}
\newcommand{\msmsym}{\Sigma}
\newcommand{\msm}[2]{\msmsym(#1,#2)}
\newcommand{\mw}[2]{\wtypesym(#1,#2)}
\newcommand{\midt}[1]{\idvartype_#1}
\newcommand{\reflf}[1]{\mathsf{refl}^{#1}}
\newcommand{\tfJ}{\mathsf{J}}
\newcommand{\tftrans}{\mathsf{trans}}
\newcommand{\shapem}{\mathop{\textesh}}

\newcommand{\@gcomp}{\underline}
\newcommand{\gcomp}[1]{{
\renewcommand{\msmsym}{\@gcomp{\Sigma}}
\def\idtypevar##1{\@gcomp{\mathsf{Id}}_{##1}}
\renewcommand{\unit}{\@gcomp{\mathbf{1}}}
\renewcommand{\wtypesym}{\@gcomp{\mathsf{W}}}
\def\reflf##1{\@gcomp{\mathsf{refl}}^{##1}}
#1}}

\newcommand{\tfT}{\mathsf{T}}
\newcommand{\reflsym}{{\mathsf{refl}}}
\newcommand{\strans}[2]{\ensuremath{{#1}_{*}({#2})}}
\newcommand{\eqtype}[1]{\mathsf{Eq}_{#1}}
\newcommand{\eqtoid}[1]{\mathsf{eqtoid}(#1)}
\newcommand{\greek}{\mathrm}
\newcommand{\product}[2]{{#1}\times{#2}}
\newcommand{\pairp}[1]{(#1)}
\newcommand{\jequalizer}[3]{\{#1|#2\jdeq #3\}}
\newcommand{\jequalizerin}[2]{\iota_{#1,#2}}
\newcommand{\tounit}[1]{{!_{#1}}}
\newcommand{\trwk}{\mathsf{trwk}}
\newcommand{\trext}{\mathsf{trext}}
\newcommand{\tfindf}[2][]{\mathsf{ind}_{#2}^{#1}}
\newcommand{\thom}[2]{\mathrm{thom}(#1,#2)}
\newcommand{\thomd}[3]{\mathrm{thom}_{#1}(#2,#3)}

\newcommand{\tfind}[3][]{\tfindf[#1]{#2}(\default@ctxext #3)}
\newcommand{\famtoequifib}{\mathsf{famToEquifib}}
\newcommand{\struct}[1]{\mathop{#1\textnormal{-Struct}}}
\newcommand{\loopspacesym}{\Omega}
\newcommand{\loopspace}[2][]{\loopspacesym^{#1}(#2)}
\newcommand{\join}[3][]{{#2}\ast_{#1}{#3}}
\newcommand{\bigjoinsym}{\mathop{{\scalebox{2.2}{\raisebox{-0.2ex}{$\ast$}}}}}
\newcommand{\@bigjoin}[2]{\bigjoinsym_{(#1)}\,#2}
\newcommand{\bigjoin}[2]{\mathchoice{\textstyle{\@bigjoin{#1}{#2}}}{\@bigjoin{#1}{#2}}{\@bigjoin{#1}{#2}}{\@bigjoin{#1}{#2}}}
\newcommand{\higherequifibsf}{\mathcal}
\newcommand{\higherequifib}[2]{\higherequifibsf{#1}(#2)}
\newcommand{\underlyinggraph}[1]{U(#1)}
\newcommand{\theequifib}[1]{{\def\higherequifibsf{}#1}}
\newcommand{\apbinary}{\mathsf{ap\usc{}binary}}
\newcommand{\subgroup}{\mathsf{Subgroup}}
\newcommand{\concretesubgroup}{\mathsf{Concrete\usc{}Subgroup}}
\newcommand{\concretegroup}{\mathsf{Concrete\usc{}Group}}


%%%%%%%%%%%%%%%%%%%%%%%%%%%%%%%%%%%%%%%%%%%%%%%%%%%%%%%%%%%%%%%%%%%%%%%%%%%%%%%%
%%%% Categories

\newcommand{\inv}[1]{{#1}^{-1}}
\newcommand{\idtoiso}{\ensuremath{\mathsf{idtoiso}}\xspace}
\newcommand{\isotoid}{\ensuremath{\mathsf{isotoid}}\xspace}
\newcommand{\op}[1]{{{#1}^\mathsf{op}}}
\newcommand{\y}{\ensuremath{\mathbf{y}}\xspace}
\newcommand{\dgr}[1]{{#1}^{\dagger}}
\newcommand{\unitaryiso}{\mathrel{\cong^\dagger}}
\newcommand{\cteqv}[2]{\ensuremath{#1 \simeq #2}\xspace}
\newcommand{\cteqvsym}{\simeq}     % Symbol for equivalence of categories

%%% Arrows
\newcommand{\epi}{\ensuremath{\twoheadrightarrow}}
\newcommand{\mono}{\ensuremath{\rightarrowtail}}

%%% Macros for HITs
\newcommand{\cc}{\mathsf{c}}
\newcommand{\pp}{\mathsf{p}}
\newcommand{\cct}{\widetilde{\mathsf{c}}}
\newcommand{\ppt}{\widetilde{\mathsf{p}}}
\newcommand{\Wtil}{\ensuremath{\widetilde{W}}\xspace}

%%%%%%%%%%%%%%%%%%%%%%%%%%%%%%%%%%%%%%%%%%%%%%%%%%%%%%%%%%%%%%%%%%%%%%%%%%%%%%%%
%%%% Set theory

\newcommand{\vset}{\mathsf{set}}  % point constructor for cummulative hierarchy V
\def\cd{\tproj0}
\newcommand{\inj}{\ensuremath{\mathsf{inj}}} % type of injections
\newcommand{\acc}{\ensuremath{\mathsf{acc}}} % accessibility

\newcommand{\atMostOne}{\mathsf{atMostOne}}

\newcommand{\power}[1]{\mathcal{P}(#1)} % power set
\newcommand{\powerp}[1]{\mathcal{P}_+(#1)} % inhabited power set

\newcommand{\ucomp}[1]{\hat{#1}}
\newcommand{\fin}{\mathsf{Fin}}
\newcommand{\finsetf}{\mathsf{Fin}}
\newcommand{\finset}[1]{\finsetf(#1)}

%%%%%%%%%%%%%%%%%%%%%%%%%%%%%%%%%%%%%%%%%%%%%%%%%%%%%%%%%%%%%%%%%%%%%%%%%%%%%%%%
%%%% Group theory

\newcommand{\symmetric}[1]{S_{#1}}

%%%%%%%%%%%%%%%%%%%%%%%%%%%%%%%%%%%%%%%%%%%%%%%%%%%%%%%%%%%%%%%%%%%%%%%%%%%%%%%%

%%%% Introducing logical usage of fonts.
\newcommand{\modelfont}{\mathit} % use 'mf' in command to indicate model font
\newcommand{\catfont}{\mathrm} % use 'cf' in command to indicate cat font

%%%%%%%%%%%%%%%%%%%%%%%%%%%%%%%%%%%%%%%%%%%%%%%%%%%%%%%%%%%%%%%%%%%%%%%%%%%%%%%%
%%%% When investigation pointed structures we use the \pt macro.

\newcommand*{\bN}{{\mathbb{N}}}
\newcommand*{\bF}{{\mathbb{F}}}
\newcommand*{\bZ}{{\mathbb{Z}}}
\newcommand*{\bR}{{\mathbb{R}}}
\newcommand*{\bC}{{\mathbb{C}}}
\newcommand*{\bH}{{\mathbb{H}}}
\newcommand*{\bS}{{\mathbb{S}}}
\newcommand*{\RP}{{\mathbb{R}\mathrm{P}}}
\newcommand*{\CP}{{\mathbb{C}\mathrm{P}}}
\newcommand*{\HP}{{\mathbb{H}\mathrm{P}}}
\newcommand*{\GType}{\mathrm{GType}} % (k-tuply groupal) n-types
\newcommand*{\nGrp}{\mathrm{\texthyphen Group}} % higher groups
\newcommand*{\Group}{\group}
\newcommand*{\AbGroup}{\mathrm{AbGroup}}
\newcommand*{\iDecat}{\mathrm{\infty\text{-}\mathrm{Decat}}}
\newcommand*{\iDisc}{\mathrm{\infty\text{-}\mathrm{Disc}}}
\newcommand*{\leanref}[1]{}
\newcommand*{\ditto}{--- \raisebox{-0.5ex}{''} ---}
\newcommand*\Sym{S}
\DeclareMathOperator\B{B}
%\newcommand*\BS{\mathop{BS}} 
\newcommand*{\dblslash}{\mathbin{/\kern-3pt/}}
\DeclareMathOperator\Th{Th}
\DeclareMathOperator\Decat{Decat}
\DeclareMathOperator\Disc{Disc}
\DeclareMathOperator\aut{Aut}
\DeclarePairedDelimiter\angled{\langle}{\rangle}
\newcommand{\texthyphen}{\text{-}}
\newcommand{\kersym}{\mathcal{K}}
\newcommand{\qsym}{\mathcal{Q}}
\mathchardef\hyph="2D
\newcommand{\kshiftequiv}{\mathsf{kshift\underline{~}equiv}}



%%%%%%%%%%%%%%%%%%%%%%%%%%%%%%%%%%%%%%%%%%%%%%%%%%%%%%%%%%%%%%%%%%%%%%%%%%%%%%%%
\newcommand{\prekersym}{k}

\newcommand{\pto}{\to_{\ast}}
\newcommand{\suspsym}{\Sigma}
\newcommand{\lra}       {\longrightarrow}
\newcommand{\LLL}{\mathcal{L}}
\newcommand{\RRR}{\mathcal{R}}
\newcommand{\sto}{\!\to\!}
\newcommand{\degg}{\mathsf{deg}}
\newcommand{\llra}[1]   {\stackrel{#1}{\lra}}  % labelled long right arrow
\newcommand{\longhookrightarrow}{\lhook\joinrel\lra}
\newcommand{\open}[1]{\mathsf{Op}_{#1}}
\newcommand{\closed}[1]{\mathsf{Cl}_{#1}}
\newcommand{\truncmod}[1]{\mathcal{T}_{#1}}
\newcommand{\shmod}[1]{\mathcal{S}\mathit{h}_{#1}}

\newcommand{\fillers}[1]{\mathsf{fill}_{\mathcal{L},\mathcal{R}}(#1)}
\newcommand{\cL}{\mathcal{L}}
\newcommand{\cR}{\mathcal{R}}
\newcommand{\rsu}[1][\UU]{\mathsf{RSU}_{#1}}
\newcommand{\mdl}[1][\UU]{\mathsf{Mdl}_{#1}}
\newcommand{\lex}[1][\UU]{\mathsf{Lex}_{#1}}
\newcommand{\tpl}[1][\UU]{\mathsf{Top}_{#1}}
\newcommand{\accrsu}[1][\UU]{\mathsf{AccRSU}_{#1}}
\newcommand{\accmdl}[1][\UU]{\mathsf{AccMdl}_{#1}}
\newcommand{\acclex}[1][\UU]{\mathsf{AccLex}_{#1}}

\newcommand{\rcoeq}{\mathsf{rcoeq}}

\newcommand{\tftmfunc}{\mathsf{tm}}
\newcommand{\tftm}[1]{\tftmfunc(#1)}

\DeclarePairedDelimiter\norm{\lVert}{\rVert}


%%%%%%%%%%%%%%%%%%%%%%%%%%%%%%%%%%%%%%%%%%%%%%%%%%%%%%%%%%%%%%%%%%%%%%%%%%%%%%%%
%% Some tikz macros to typeset diagrams uniformly.

\tikzcdset{arrow style=math font}
\tikzset{patharrow/.style={double,double equal sign distance,-,font=\scriptsize}}
\tikzset{description/.style={fill=white,inner sep=2pt}}
\tikzset{fib/.style={->>,font=\scriptsize}}
\tikzset{%
    adjunction/.style={%
        draw=none,
        every to/.append style={%
            edge node={node [sloped, allow upside down, auto=false]{$\dashv$}}}
    }
}

%% Used for extra wide diagrams, e.g. when the label is too large otherwise.
\tikzset{commutative diagrams/column sep/Huge/.initial=18ex}

%%%%%%%%%%%%%%%%%%%%%%%%%%%%%%%%%%%%%%%%%%%%%%%%%%%%%%%%%%%%%%%%%%%%%%%%%%%%%%%%
%%%% New environment for constructions.

%\expandafter\let\expandafter\oldproof\csname\string\proof\endcsname
%\let\oldendproof\endproof
\newenvironment{constr}{\begin{proof}[Construction]}{\end{proof}}

%%%%%%%%%%%%%%%%%%%%%%%%%%%%%%%%%%%%%%%%%%%%%%%%%%%%%%%%%%%%%%%%%%%%%%%%%%%%%%%%
%%%% The following environment for desiderata should not be there. It is better
%%%% to use the issue tracker for desiderata.

\newenvironment{desiderata}{\begingroup\color{blue}\textbf{Desiderata.}}
{\endgroup}

%%%%%%%%%%%%%%%%%%%%%%%%%%%%%%%%%%%%%%%%%%%%%%%%%%%%%%%%%%%%%%%%%%%%%%%%%%%%%%%%
%%%% The following piece of code from tex.stackexchange:
%%%%
%%%% http://tex.stackexchange.com/a/55180/14653
%%%%
%%%% We include it so that inference rules in align environments have enough
%%%% vertical space.

\newlength\minalignvsep

\def\align@preamble{%
   &\hfil
    \setboxz@h{\@lign$\m@th\displaystyle{##}$}%
    \ifnum\row@>\@ne
    \ifdim\ht\z@>\ht\strutbox@
    \dimen@\ht\z@
    \advance\dimen@\minalignvsep
    \ht\strutbox\dimen@
    \fi\fi
    \strut@
    \ifmeasuring@\savefieldlength@\fi
    \set@field
    \tabskip\z@skip
   &\setboxz@h{\@lign$\m@th\displaystyle{{}##}$}%
    \ifnum\row@>\@ne
    \ifdim\ht\z@>\ht\strutbox@
    \dimen@\ht\z@
    \advance\dimen@\minalignvsep
    \ht\strutbox@\dimen@
    \fi\fi
    \strut@
    \ifmeasuring@\savefieldlength@\fi
    \set@field
    \hfil
    \tabskip\alignsep@
}

\minalignvsep.2em

\allowdisplaybreaks

%%%%%%%%%%%%%%%%%%%%%%%%%%%%%%%%%%%%%%%%%%%%%%%%%%%%%%%%%%%%%%%%%%%%%%%%%%%%%%%%

% \setdescription[1]{itemsep=-0.2em}

\makeatother


%%% Local Variables:
%%% mode: latex
%%% TeX-master: "hott-intro"
%%% End:
