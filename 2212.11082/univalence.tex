\section{The univalence axiom}
\index{univalence axiom|(}
\index{axiom!univalence|(}

The univalence axiom characterizes the identity type of a universe. Roughly speaking, it asserts that equivalent types are equal. The univalence axiom therefore postulates the common mathematical habit of identifying equivalent objects, such as equivalent types, isomorphic groups, isomorphic rings, logically equivalent propositions, subsets with the same elements, and so on. The univalence axiom is due to Voevodsky, who also showed that it is modeled in the simplicial sets. He also showed, in one of his first applications, that the univalence axiom implies function extensionality, which we will also prove here.

One way to think about the univalence axiom is that it \emph{expands} the notion of equality to encapsulate the notion of equivalence. It asserts that for each equivalence $e$ between two types $X$ and $Y$ in a universe $\mathcal{U}$ there is a unique identification $p_e:X=Y$ in the universe $\mathcal{U}$ such that transporting along $p_e$ in the universal type family over $\mathcal{U}$ is homotopic to the original equivalence $e:X\simeq Y$. 

Since there might be many distinct equivalences between two types $X$ and $Y$, there will be equally many identifications those types. The univalence axiom is therefore inconsistent with the commonly assumed axiom that all identity types are propositions, i.e., that all types are sets. Indeed, there are two equivalences $\bool\simeq\bool$, so a univalent universe cannot be a set.

\subsection{Equivalent forms of the univalence axiom}
By the fundamental theorem of identity types, \cref{thm:id_fundamental}, it is immediate that the univalence axiom comes in three equivalent forms.

\begin{thm}\label{thm:univalence}\index{identity type!of a universe}\index{characterization of identity type!of a universe}\index{universe!characterization of identity type}
Consider a universe $\UU$. The following are equivalent:
\begin{enumerate}
\item The universe $\UU$ is \define{univalent}\index{univalent universe|textbf}: For any two types $A,B:\UU$, the map\index{equiv-eq@{$\equiveq$}|textbf}
  \begin{equation*}
    \equiveq:(A=B)\to (A\simeq B)
  \end{equation*}
  given by $\equiveq(\refl{}):=\idfunc$, is an equivalence.
\item The type
\begin{equation*}
\sm{B:\UU}\eqv{A}{B}
\end{equation*}
is contractible for each $A:\UU$.
\item For any type $A:\UU$, the family of types $A\simeq X$ indexed by $X:\UU$ is an identity system on $\UU$. In other words, the universe $\UU$ satisfies the principle of \define{equivalence induction}\index{equivalence induction|textbf}\index{induction principle!for equivalences|textbf}: For every $A:\UU$ and for every type family of types $P(X,e)$ indexed by $X:\UU$ and $e:A\simeq X$, the map
\begin{equation*}
\Big(\prd{X:\UU}\prd{e:\eqv{A}{X}}P(X,e)\Big)\to P(A,\idfunc)
\end{equation*}
given by $f\mapsto f(A,\idfunc)$ has a section.
\end{enumerate}
\end{thm}

\begin{proof}
  The claim is a special case of \cref{thm:id_fundamental}, the fundamental theorem of identity types\index{fundamental theorem of identity types}.
\end{proof}

One way to see that the univalence axiom is plausible, is by observing that all type constructors preserve equivalences. For example, in \cref{thm:equiv-toto} we showed that for any type family $B$ over $A$ and any type family $B'$ over $A'$, if we have an equivalence $e:A\simeq A'$ and family of equivalences $f:\prd{x:A}B(x)\simeq B'(e(x))$, then we obtain an equivalence
\begin{equation*}
  \Big(\sm{x:A}B(x)\Big)\simeq\Big(\sm{x':A'}B'(x')\Big).
\end{equation*}
Under the same assumptions, we showed in \cref{ex:equiv-pi} that we obtain an equivalence
\begin{equation*}
  \Big(\prd{x:A}B(x)\Big)\simeq\Big(\prd{x':A'}B'(x')\Big).
\end{equation*}
Furthermore, for any two elements $x,y:A$ any equivalence $e:A\simeq A'$ induces an equivalence $(x=y)\simeq (e(x)=e(y))$ by \cref{cor:emb_equiv}. In other words, all the standard type formers within a universe $\UU$ are \emph{equivalence invariant}. Since identity types are not assumed to be propositions, we have the possibility to postulate the univalence axiom.

\begin{axiom}[The univalence axiom]\label{axiom:univalence}\index{univalence axiom|textbf}\index{axiom!univalence|textbf}
  We will assume that all the universes generated by \cref{enough-universes}\index{enough universes}\index{universe!enough universes} are univalent. Given a univalent universe $\UU$, we will write $\eqequiv$\index{eq-equiv@{$\eqequiv$}|textbf} for the inverse of $\equiveq$.
\end{axiom}

As a first application of the univalence axiom, let us show that for any type $A$ the type of types in a univalent universe $\UU$ that are equivalent to $A$ is a proposition.

\begin{defn}\label{defn:small-types}
  Consider a univalent universe $\UU$. A type $X$ is said to be \define{$\UU$-small}\index{small type|textbf}\index{U-small type@{$\UU$-small type}|textbf} if it comes equipped with an element of type\index{is-small@{$\issmall_\UU(A)$}|textbf}
  \begin{equation*}
    \issmall_\UU(A)\defeq\sm{X:\UU}A\simeq X.
  \end{equation*}
  Similarly, a map $f:A\to B$ is said to be \define{$\UU$-small}\index{small map|textbf}\index{U-small map@{$\UU$-small map}|textbf} if all of its fibers are $\UU$-small.
\end{defn}

\begin{eg}
  ~
  \begin{enumerate}
  \item Any type in $\UU$ is $\UU$-small.
  \item Any contractible type is $\UU$-small with respect to any universe $\UU$. \index{contractible type!is U-small@{is $\UU$-small}}\index{small type!contractible type}
  \item For any family $P$ of $\UU$-small types over a $\UU$-small type $A$, the dependent product $\prd{x:A}B(x)$ is $\UU$-small.\index{small type!dependent function type}\index{dependent function type!is U-small@{is $\UU$-small}}
  \item The type of $\UU$-small types in $\VV$ is equivalent to the type of $\VV$-small types in $\UU$. This follows from the equivalence
    \begin{equation*}
      \Big(\sm{Y:\VV}\sm{X:\UU}Y\simeq X\Big) \simeq \Big(\sm{X:\UU}\sm{Y:\VV}X\simeq Y\Big).
    \end{equation*}
  \item Any finite type is $\UU$-small for any universe $\UU$.\index{finite type!is U-small@{is $\UU$-small}}\index{small type!finite type} Consequently, we get equivalences
  \begin{equation*}
    \Big(\sm{X:\UU}\isfinite(X)\Big)\simeq\Big(\sm{Y:\VV}\isfinite(Y)\Big)
  \end{equation*}
  for any two univalent universes $\UU$ and $\VV$. This observation is the reason why we usually write $\F$ for the type of finite types (in $\UU$), without referring to its universe.
  \item In \cref{thm:russell} we will show that $\UU$ cannot be $\UU$-small, i.e., that there cannot be a type $U:\UU$ equipped with an equivalence $U\simeq \UU$.
  \end{enumerate}
\end{eg}

\begin{prp}\label{prp:small}
  For any univalent universe $\UU$ and any type $A$, the type $\issmall_\UU(A)$ is a proposition.\index{is-small@{$\issmall_\UU(A)$}!is a proposition}
\end{prp}

\begin{proof}
  By \cref{lem:isprop_eq} it suffices to show that
  \begin{equation*}
    \issmall_\UU(A)\to\iscontr(\issmall_\UU(A)).
  \end{equation*}
  Let $X:\UU$ be a type equipped with $e:A\simeq X$. Then we have an equivalence
  \begin{equation*}
    \Big(\sm{Y:\UU}A\simeq Y\Big)\simeq\Big(\sm{Y:\UU}X\simeq Y\Big).
  \end{equation*}
  The latter type is contractible by \cref{thm:univalence}.
\end{proof}

\begin{cor}
  Consider a univalent universe $\UU$ and a univalent universe $\VV$ containing all types in $\UU$. Then the universe inclusion $i:\UU\to\VV$ is an embedding.
\end{cor}

\begin{proof}
  Since $\VV$ is assumed to be univalent, it follows that
  \begin{equation*}
    \fib{i}{A}\simeq\issmall_\UU(A)
  \end{equation*}
  for any type $A:\VV$. The type $\issmall_\UU(A)$ is a proposition since $\UU$ is univalent. Hence the claim follows by \cref{thm:embedding}.
\end{proof}

\subsection{Propositional extensionality}
\index{propositional extensionality|(}

An important direct consequence of the univalence axiom is the principle of propositional extensionality. This principle asserts that any two logically equivalent propositions $P$ and $Q$ can be identified. Propositional extensionality is an important principle on its own, which is sometimes assumed in formal systems without the univalence axiom.

In order to prove propositional extensionality, we first observe that the univalence axiom also characterizes the identity type of any subuniverse.

\begin{prp}\label{prp:univalence-subuniverse}
  Consider a universe $\UU$, and let $P$ be a family of propositions over $\UU$. Then the family of maps
  \begin{equation*}
    \equiveq:(A=B)\to (\proj 1(A) \simeq \proj 1(B))
  \end{equation*}
  indexed by $A,B:\sm{X:\UU}P(X)$, given by $\equiveq(\refl{}):=\idfunc$ is an equivalence.
\end{prp}

\begin{proof}
  Since $P$ is a subuniverse, it follows from \cref{cor:pr1-embedding} that the projection map is an embedding. Therefore we see that the asserted map is the composite of the equivalences
  \begin{equation*}
    \begin{tikzcd}
      (A=B) \arrow[r,"\apfunc{\proj 1}"] & (\proj 1(A)=\proj 1(B)) \arrow[r,"\equiveq"] &[2em] (\proj 1(A)\simeq \proj 1(B)).
    \end{tikzcd}\qedhere
  \end{equation*}
\end{proof}

\begin{rmk}
  Often, when $P$ is a subuniverse, i.e., a subtype of the a universe $\UU$, we will also write $A$ for the type $\proj 1(A)$ if $A:\sm{X:\UU}P(X)$. Using this shorthand notation, the equivalence in \cref{prp:univalence-subuniverse} is displayed as
  \begin{equation*}
    (A=B)\simeq (A\simeq B).
  \end{equation*}
\end{rmk}

Important examples of subuniverses include the subuniverse $\prop_\UU$ of propositions in $\UU$, the subuniverse $\Set_\UU$ of sets in $\UU$, and the subuniverse $\UU^{\leq k}$ of $k$-truncated types in $\UU$. The subuniverse $\F$ of finite types in $\UU_0$, and the subuniverses $\BS_k$ of $k$-element types are further important subuniverses to which \cref{prp:univalence-subuniverse} applies. Note that by the univalence axiom, any subuniverse is automatically closed under equivalences\index{subuniverse!closed under equivalences}. Indeed, if we have $X\simeq Y$, then we have $P(X)\to P(Y)$ by transporting along the equality $X=Y$ induced by univalence.

\begin{thm}\label{prp:propositional-extensionality}
  Propositions satisfy \define{propositional extensionality}\index{propositional extensionality|textbf}\index{extensionality principle!for propositions|textbf}:
  For any two propositions $P$ and $Q$, the canonical map\index{bi-implication}\index{iff-eq@{$\iffeq$}|textbf}
  \begin{equation*}
    \iffeq:(P=Q)\to (P\iffprop Q)
  \end{equation*}
  defined by $\iffeq(\refl{}):=(\idfunc,\idfunc)$ is an equivalence. It follows that the type $\prop_\UU$ of propositions in $\UU$ is a set.\index{Prop@{$\prop_\UU$}!is a set}\index{univalence axiom!implies propositional extensionality}
\end{thm}

\begin{proof}
  Recall from \cref{ex:isprop_istrunc} that $\isprop(X)$ is a proposition for any type $X$. \cref{prp:univalence-subuniverse} therefore applies, which gives
  \begin{equation*}
    (P=Q)\simeq (P\simeq Q)\simeq (P\leftrightarrow Q).
  \end{equation*}
  The last equivalence follows from \cref{prp:equiv-prop}, using the fact that $(P\simeq Q)$ is a proposition by \cref{ex:isprop_isequiv}.
\end{proof}

\begin{cor}\label{cor:decidable-Prop}
  The type\index{DProp@{$\decidableProp_\UU$}|see {decidable proposition}}\index{DProp@{$\decidableProp_\UU$}|textbf}\index{decidable proposition|textbf}\index{DProp@{$\decidableProp_\UU$}!is equivalent to bool@{is equivalent to $\bool$}}
  \begin{equation*}
    \decidableProp_\UU \defeq \sm{P:\prop_\UU}\isdecidable(P)
  \end{equation*}
  of decidable propositions in any universe $\UU$ is equivalent to $\bool$.
\end{cor}

\begin{proof}
  Note that $\Sigma$ distributes from the left over coproducts, so we have an equivalence
  \begin{equation*}
    \Big(\sm{P:\prop_\UU}P+\neg P\Big)\simeq \Big(\sm{P:\prop_\UU}P\Big)+\Big(\sm{Q:\prop_\UU}\neg Q\Big). 
  \end{equation*}
  Therefore it suffices to show that both $\sm{P:\prop_\UU}P$ and $\sm{Q:\prop_\UU}\neg Q$ are contractible. At the centers of contraction we have $(\unit,\ttt)$ and $(\emptyt,\idfunc)$, respectively. For the contractions, note that both types are subtypes of the types of propositions. Therefore it suffices to show that $\unit=P$ for any proposition $P$ equipped with $p:P$, and that $\emptyt=Q$ for any proposition $Q$ equipped with $q:\neg Q$. Both identifications are obtained immediately from propositional extensionality.
\end{proof}
\index{propositional extensionality|)}

\subsection{Univalence implies function extensionality}
One of the first applications of the univalence axiom was Voevodsky's theorem that the univalence axiom on a universe $\UU$ implies function extensionality for types in $\UU$. The proof uses the fact that weak function extensionality implies function extensionality. We will also make use of the following lemma. 

\begin{lem}\label{lem:postcomp-equiv}
  For any equivalence $e:\eqv{X}{Y}$ in a univalent universe $\UU$, and any type $A$, the post-composition map
  \begin{equation*}
    e\circ\blank : (A \to X) \to (A\to Y)
  \end{equation*}
  is an equivalence.
\end{lem}

Note that this statement was also part of \cref{ex:equiv-postcomp}. That exercise is solved using function extensionality. However, since our present goal is to derive function extensionality from the univalence axiom, we cannot make use of that exercise. Therefore we give a new proof, using the univalence axiom.

\begin{proof}
  Since $\UU$ is assumed to be a univalent universe, it satisfies by \cref{thm:univalence} the principle of equivalence induction. Therefore, it suffices to show that the post-composition map
  \begin{equation*}
    \idfunc\circ\blank : (A\to X)\to (A\to X)
  \end{equation*}
  is an equivalence. This post-composition map is of course just the identity map on $A\to X$, so it is indeed an equivalence.
\end{proof}

\begin{thm}\label{thm:funext-univalence}\index{univalence axiom!implies function extensionality}\index{function extensionality!univalence implies function extensionality}
  For any universe $\UU$, the univalence axiom on $\UU$ implies function extensionality on $\UU$.
\end{thm}

\begin{proof}
  Note that by \cref{thm:funext_wkfunext}\index{weak function extensionality} it suffices to show that univalence implies weak function extensionality. We note that the proof of \cref{thm:funext_wkfunext} also goes through when it is restricted to types in $\UU$.
  
Suppose that $B:A\to \UU$ is a family of contractible types. Our goal is to show that the product $\prd{x:A}B(x)$ is contractible.
Since each $B(x)$ is contractible, the projection map $\proj 1:\big(\sm{x:A}B(x)\big)\to A$ is an equivalence by \cref{ex:proj_fiber}.

Now it follows by \cref{lem:postcomp-equiv} that $\proj1\circ\blank$ is an equivalence. Consequently, it follows from \cref{thm:contr_equiv} that the fibers of
\begin{equation*}
\proj 1\circ\blank : \Big(A\to \sm{x:A}B(x)\Big)\to (A\to A)
\end{equation*}
are contractible. In particular, the fiber at $\idfunc[A]$ is contractible. Therefore it suffices to show that $\prd{x:A}B(x)$ is a retract of $\sm{f:A\to\sm{x:A}B(x)}\proj 1\circ f=\idfunc[A]$. In other words, we will construct a section-retraction pair
\begin{equation*}
\begin{tikzcd}[column sep=1em]
\Big(\prd{x:A}B(x)\Big) \arrow[r,"i"] & \Big(\sm{f:A\to\sm{x:A}B(x)}\proj 1\circ f=\idfunc[A]\Big) \arrow[r,"r"] & \Big(\prd{x:A}B(x)\Big),
\end{tikzcd}
\end{equation*}
with $H:r\circ i\htpy \idfunc$.

We define the function $i$ by
\begin{equation*}
  i(f) \defeq (\lam{x}(x,f(x)),\refl{\idfunc}).
\end{equation*}
To see that this definition is correct, we need to know that
\begin{equation*}
  \lam{x}\proj 1(x,f(x))\jdeq \idfunc.
\end{equation*}
This is indeed the case, by the rule $\lambda$-eq for $\Pi$-types, on \cpageref{page:lambda-eq}.

Next, we define the function $r$. Consider a function $h:A\to \sm{x:A}B(x)$ equipped with an identification $p:\proj 1 \circ h = \idfunc$. Then we have the homotopy $\htpyeq(p):\proj 1 \circ h \htpy \idfunc$. Furthermore, we obtain $\proj 2(h(x)):B(\proj 1(h(x)))$. Using these ingredients, we define $r$ by
\begin{equation*}
  r((h,p),x)\defeq \tr_B(\htpyeq(p,x),\proj 2(h(x))).
\end{equation*}

It remains to construct a homotopy $H:r\circ i\htpy \idfunc$. We simply compute
\begin{align*}
  r(i(f)) & \jdeq r(\lam{x}(x,f(x)),\refl{}) \\
          & \jdeq \tr_B(\htpyeq(\refl{},x),\proj 2(x,f(x))) \\
          & \jdeq \tr_B(\refl{},f(x)) \\
          & \jdeq f(x).
\end{align*}
Thus we see that $r\circ i\jdeq \idfunc$ by an application of the $\eta$-rule for $\Pi$-types. Therefore we simply define $H(f)\defeq\refl{}$.
\end{proof}

\subsection{Maps and families of types}

Using the univalence axiom, we can establish a fundamental relation between maps into a type $A$, and families of types indexed by $A$. A special case of this relation asserts that the type of all pairs $(X,e)$ consisting of a type $X$ and an embedding $e:X\emb A$ is equivalent to the type of all subtypes of $A$, i.e., the type of all families $P$ of propositions indexed by $A$.

\begin{thm}\label{thm:object-classifier}\index{type family}
  For any type $A$ and any univalent universe $\UU$ containing $A$, the map
  \begin{equation*}
    \Big(\sm{X:\UU}X\to A\Big)\to (A\to \UU)
  \end{equation*}
  given by $(X,f)\mapsto\fibf{f}$ is an equivalence.\index{fib f b@{$\fib{f}{b}$}}
\end{thm}

\begin{proof}
  The map in the converse direction is given by
  \begin{equation*}
    B\mapsto \Big(\sm{x:A}B(x),\proj 1\Big). 
  \end{equation*}
  To verify that this map is a section of the asserted map, we have to prove that
  \begin{equation*}
    \fibf{\proj 1}=B
  \end{equation*}
  for any $B:A\to\UU$. By function extensionality and the univalence axiom, this is equivalent to
  \begin{equation*}
    \prd{x:A}\fib{\proj 1}{x}\simeq B(x).
  \end{equation*}
  Such a family of equivalences was constructed in \cref{ex:proj_fiber}.

  It remains to verify that
  \begin{equation*}
    (X,f)=\Big(\sm{x:A}\fib{f}{x},\proj 1\Big). 
  \end{equation*}
  Before we do this, we claim that the identity type
  \begin{equation*}
    (X,f)=(Y,g)
  \end{equation*}
  in the type $\sm{X:\UU}X\to A$ is equivalent to the type of pairs $(e,f)$ consisting of an equivalence $e:X\simeq Y$ equipped with a homotopy $f\htpy g\circ e$. This fact follows from \cref{thm:id_fundamental}, because the type
  \begin{equation*}
    \sm{Y:\UU}\sm{g:Y\to A}\sm{e:X\simeq Y}f\htpy g\circ e
  \end{equation*}
  is contractible by the structure identity principle, \cref{thm:structure-identity-principle}.

  To finish the proof, it therefore suffices to construct an equivalence
  \begin{equation*}
    e:X\simeq \sm{a:A}\fib{f}{a}
  \end{equation*}
  equipped with a homotopy $f\htpy \proj 1\circ e$. Such an equivalence $e$ equipped with a homotopy was constructed in \cref{ex:fib_replacement}.
\end{proof}

The following corollary is so important, that we call it again a theorem.

\begin{thm}\label{thm:classifier-subuniverse}
  Consider a type $A$ and a univalent universe $\UU$ containing $A$. Furthermore, let $P$ be a family of types indexed by $\UU$, and write
  \begin{equation*}
    \UU_P\defeq\sm{X:\UU}P(X).
  \end{equation*}
  Then the map
  \begin{equation*}
    \Big(\sm{X:\UU}\sm{f:X\to A}\prd{a:A}P(\fib{f}{a})\Big)\to (A\to\UU_P)
  \end{equation*}
  given by $(X,f,p)\mapsto \lam{a}(\fib{f}{a},p(a))$ is an equivalence.
\end{thm}

\begin{proof}
  The asserted map is homotopic to the composition of the equivalences
  \begin{align*}
    & \sm{X:\UU}\sm{f:X\to A}\prd{a:A}P(\fib{f}{a}) \\
    & \simeq \sm{(X,f):\sm{X:\UU}X\to A}\prd{a:A}P(\fib{f}{a}) \\
    & \simeq \sm{B:A\to \UU}\prd{a:A}P(B(a)) \\
    & \simeq A\to\sm{X:\UU}P(X).\qedhere
  \end{align*}
\end{proof}

\cref{thm:classifier-subuniverse} applies to any subuniverse. Examples include the subuniverse of $k$-types, for any truncation level $k$, the subuniverse of decidable propositions, the subuniverse of finite types, the subuniverse of inhabited types, and so on. It also applies to type families over $\UU$ that aren't families of propositions. The families $P\defeq\isdecidable$ and $P\defeq\cnt$ are examples.

\begin{cor}\label{cor:subtype}
  Consider a type $A$ and a univalent universe $\UU$ containing $A$. Then the map\index{embedding}\index{subtype}\index{fib f b@{$\fib{f}{b}$}}
  \begin{equation*}
    \Big(\sm{X:\UU}X\emb A\Big)\to (A\to\prop_\UU)
  \end{equation*}
  given by $(X,f)\mapsto \fibf{f}$ is an equivalence.
\end{cor}

In other words, a subtype of a type $A$ is equivalently described as a type $X$ equipped with an embedding $e:X\hookrightarrow A$. This brings us to an important point about equality of subtypes.

\begin{rmk}
  By function extensionality and propositional extensionality, it follows that two subtypes $P,Q:A\to\prop_\UU$ are the same if and only if\index{characterization of identity type!of subtypes of A@{of subtypes of $A$}}\index{subtype!characterization of identity type}
\begin{equation*}
  P(a)\leftrightarrow Q(a)
\end{equation*}
holds for all $a:A$. In other words, two subtypes of $A$ are the same if and only if they contain the same elements of $A$.

On the other hand, by \cref{cor:subtype} we can also consider two types $X$ and $Y$ equipped with embeddings $f:X\emb A$ and $g:Y\emb A$ as subtypes of $A$. Using the structure identity principle, \cref{thm:structure-identity-principle}, we see that the identity type $(X,f)=(Y,g)$ in the type $\sm{X:\UU}X\emb A$ is equivalent to the type
\begin{equation*}
  \sm{e:X\simeq Y}f\htpy g\circ e.
\end{equation*}
In other words, two subtypes $(X,f)$ and $(Y,g)$ of $A$ are equal if and only if there is an equivalence $X\simeq Y$ that is compatible with the embeddings $f:X\emb A$ and $g:Y\emb X$. Indeed, this condition is equivalent to the previous condition that two subtypes are the same if and only if they have the same elements.

We see that the combination of the structure identity principle\index{structure identity principle} and the univalence axiom automatically characterizes equality of subtypes in the most natural way, and we will see similar natural characterizations of identity types throughout the remainder of this book.
\end{rmk}

\subsection{Classical mathematics with the univalence axiom}

In classical mathematics, the axiom of choice asserts that for any collection $X$ of nonempty sets, there is a choice function $f$ such that $f(x)\in x$ for each $x\in X$. The univalence axiom is consistent with the axiom of choice, but we have to be careful in our formulation of the axiom of choice to make it about sets. A naive interpretation that would be applicable to all types, such as the assertion that every family $B$ of inhabited types has a section, is not consistent with univalence. We will use the type $\BS_2$ of $2$-element types for a counterexample.

\begin{prp}\label{prp:Eq-F2}
  The type\index{BS 2@{$\BS_2$}!characterization of identity type}\index{characterization of identity type!of BS 2@{of $\BS_2$}}
  \begin{equation*}
    \sm{X:\BS_2}X
  \end{equation*}
  of pointed $2$-element types is contractible. Consequently, the canonical family of maps
  \begin{equation*}
    (\Fin{2}= X) \to X
  \end{equation*}
  indexed by $X:\BS_2$, is a family of equivalences.
\end{prp}

\begin{proof}
  By the univalence axiom it follows that the type $\sm{X:\BS_2}\Fin{2}\simeq X$ is contractible. In order to show that $\sm{X:\BS_{2}}X$ is contractible, it therefore suffices to show that the map
  \begin{equation*}
    f: (\Fin{2}\simeq X)\to X
  \end{equation*}
  given by $f(e)\defeq e(\star)$, is an equivalence. Since being an equivalence is a proposition by \cref{ex:isprop_isequiv}, we may assume an equivalence $\alpha:\Fin{2}\simeq X$, and we proceed by equivalence induction on $\alpha$. Therefore, it suffices to show that the map
  \begin{equation*}
    f: (\Fin{2}\simeq \Fin{2})\to\Fin{2}
  \end{equation*}
  give by $f(e)\defeq e(\star)$ is an equivalence. Using the notation from \cref{sec:Fin}, we define the inverse map $g$ by
  \begin{align*}
    g(\star) & \defeq \idfunc \\
    g(i(\star)) & \defeq \succFin_2,
  \end{align*}
  and it is a straightforward verification that $f$ and $g$ are inverse to each other.
\end{proof}

\begin{cor}\label{cor:no-section-F2}
  There is no dependent function\index{BS 2@{$\BS_2$}!is not contractible}
  \begin{equation*}
    \prd{X:\BS_2}X.
  \end{equation*}
\end{cor}

\begin{proof}
  By \cref{prp:Eq-F2,ex:equiv-pi}, we have an equivalence
  \begin{equation*}
    \Big(\prd{X:\BS_2}\Fin{2}=X\Big)\simeq \Big(\prd{X:\BS_2}X\Big).
  \end{equation*}
  Note that $\prd{X:\BS_2}\Fin{2}=X$ is the type of contractions of $\BS_2$, using the center of contraction $\Fin{2}$. Therefore it suffices to show that $\BS_2$ is not contractible. Recall from \cref{ex:prop_contr} that the identity types of contractible types are contractible. On the other hand, it follows from \cref{prp:Eq-F2} that the identity type $\Fin{2}=\Fin{2}$ in $\BS_2$ is equivalent to $\Fin{2}$. This type isn't contractible by \cref{ex:is-not-contractible-Fin}. We conclude that $\BS_2$ is not contractible.
\end{proof}

The family $X\mapsto X$ over $\BS_2$ is therefore an example of a family of nonempty types for which there are provably no sections. In the following corollary we conclude more generally that there is no way to construct an element of an arbitrary inhabited type.

\begin{cor}\label{cor:no-global-choice}
  If $\UU$ is a univalent universe, then there is no \define{global choice}\index{global choice|textbf} function
  \begin{equation*}
    \prd{A:\UU}\brck{A}\to A.
  \end{equation*}
\end{cor}

\begin{proof}
  Suppose $f:\prd{A:\UU}\brck{A}\to A$. By restricting $f$ to the type of $2$-element types in $\UU$, we obtain a function
  \begin{equation*}
    \prd{A:\BS_2}\brck{A}\to A.
  \end{equation*}
  Note that every $2$-element type is inhabited, i.e., there is an element of type $\brck{A}$ for every $2$-element type $A$. To see this, consider a type $A:\UU$ such that $\brck{\Fin{2}\simeq A}$. To obtain an element of type $\brck{A}$, we may assume an equivalence $e:\Fin{2}\simeq A$. Then we have $\eta(e(0)):\brck{A}$.

  Since every $2$-element type is inhabited, we obtain a function $\prd{A:\BS_2}A$, which is impossible by \cref{cor:no-section-F2}.
\end{proof}

\cref{cor:no-global-choice} is of philosophical importance. It shows that the \define{principle of global choice} is incompatible with the univalence axiom, i.e., that there is no way to obtain construct a function $\brck{A}\to A$ for all types $A$. In other words, we cannot obtain an element of $A$ merely from the assumption that the type $A$ is inhabited. What is the obstruction? It is the fact that no such choice of an element of $A$ can be invariant under the automorphisms on $A$, i.e., under the self-equivalences on $A$. Indeed, in the example where $A$ is the $2$-element type $\Fin{2}$ there are no fixed point of the equivalence $\succFin_2:\Fin{2}\simeq\Fin{2}$. By the univalence axiom, there is an identification $p:\Fin{2}=\Fin{2}$ in $\UU$, such that $\tr(p,x)=\succFin_2(x)$. If we had a function
\begin{equation*}
  f:\prd{X:\UU}\brck{X}\to X,
\end{equation*}
the dependent action on paths of $f$ would give an identification
\begin{equation*}
  \apd{f}{p}:\succFin_2(f(\Fin{2},p,H))=f(\Fin{2},p,\eta(0)).
\end{equation*}
In other words, it would give us a fixed point for the successor function on $\Fin{2}$.

This is perhaps a good moment to stress that the axiom of choice is really an axiom about \emph{sets}, not about more general types. And indeed, when we restrict the axiom of choice to sets, it turns out to be consistent with the univalence axiom and therefore safe to assume. In this book, however, we will not have many applications for the axiom of choice and therefore we will not assume it, unless we explicitly say otherwise.

\begin{defn}
  The \define{axiom of choice}\index{axiom of choice|textbf}\index{axiom!axiom of choice|textbf} asserts that for any family $B$ of inhabited sets indexed by a set $A$, the type of sections of $B$ is also inhabited, i.e., it asserts that there is an element of type
  \begin{equation*}
    \AC_{\UU}(A,B)\defeq \Big(\prd{x:A}\brck{B(x)}\Big)\to\Brck{\prd{x:A}B(x)},
  \end{equation*}
  for every $A:\Set_\UU$ and $B:A\to\Set_\UU$.
\end{defn}

Similar care has to be taken with the type theoretic formulation of the law of excluded middle. It is again inconsistent to assume that every type is decidable.

\begin{thm}
  There is no \define{global decidability function}\index{global decidability|textbf}
  \begin{equation*}
    \prd{X:\UU}\isdecidable(X). 
  \end{equation*}
\end{thm}

\begin{proof}
  Suppose there is such a dependent function $d$. By restricting $d$ to the subuniverse of $2$-element types, we obtain a dependent function
  \begin{equation*}
    d:\prd{X:\BS_2}\isdecidable(X).
  \end{equation*}
  However, each $2$-element type $X$ is inhabited. By \cref{ex:propositional-truncations-drill} we obtain a function
  \begin{equation*}
    \isdecidable(X)\to X
  \end{equation*}
  for each $2$-element type $X$. Therefore, we obtain from $d$ a dependent function $\prd{X:\BS_2}X$, which does not exist by \cref{cor:no-section-F2}.
\end{proof}

The law of excluded middle is really an axiom of propositional logic, and it is indeed consistent with the univalence axiom that every \emph{proposition} is decidable.

\begin{defn}
  The \define{law of excluded middle}\index{law of excluded middle|textbf}\index{axiom!law of excluded middle|textbf} asserts that every proposition is decidable, i.e.,
  \begin{equation*}
    \LEM_\UU\defeq \prd{P:\prop_\UU}\isdecidable(P).
  \end{equation*}
\end{defn}

We will again not assume the law of excluded middle, unless we say otherwise. Nevertheless, we have seen in \cref{sec:decidability} that some propositions are already decidable without assuming the law of excluded middle, and decidability remains an important concept in type theory and mathematics.

\subsection{The binomial types}
\index{binomial type|(}

To wrap up this section on univalence, we will use the univalence axiom to construct for any two types $A$ and $B$ a type $\dbinomtype{A}{B}$ that has properties similar to the binomial coefficients $\dbinomtype{n}{k}$. Indeed, we will show that if $A$ is an $n$-element type and $B$ is a $k$-element type, then $\dbinomtype{A}{B}$ is an $\binom{n}{k}$-element type. The binomial types are defined using decidable embeddings.

\begin{defn}
  A map $f:A\to B$ is said to be \define{decidable}\index{decidable map|textbf} if it comes equipped with an element of type
  \begin{equation*}
    \isdecidable(f) \defeq \prd{b:B}\isdecidable(\fib{f}{b}).
  \end{equation*}
  We will write $A\demb B$ for the type of \define{decidable embeddings}\index{decidable embedding|textbf}\index{A hookrightarrow d B@{$A \demb B$}|see {decidable embedding}} from $A$ to $B$, i.e., for the type of embeddings that are also decidable maps. 
\end{defn}

\begin{defn}
  Consider a type $A$ and a universe $\UU$. We define the \define{connected component}\index{universe!connected component|textbf}\index{connected component!of a universe|textbf} of $\UU$ at $A$ by\index{U A@{$\UU_A$}|textbf}
  \begin{equation*}
    \UU_A\defeq \sm{X:\UU}\brck{A\simeq X}.
  \end{equation*}
\end{defn}

\begin{eg}
  Note that type $\UU_{\Fin{n}}$ is the type $\BS_n$ of all $n$-element types. Note also that if $A\simeq B$, then $\UU_A\simeq\UU_B$.\index{BS n@{$\BS_n$}}
\end{eg}

\begin{defn}\label{defn:binomial-type}
  Consider two types $A$ and $B$ and a universe $\UU$ containing both $A$ and $B$. We define the \define{binomial type}\index{binomial type|textbf} $\dbinomtype[\UU]{A}{B}$\index{(A B)@{$\dbinomtype{A}{B}$}|see {binomial type}} by
  \begin{equation*}
    \dbinomtype[\UU]{A}{B} \defeq \sm{X:\UU_B}X\demb A.
  \end{equation*}
\end{defn}

\begin{rmk}
  We define the binomial types using decidable embeddings because the usual properties of binomial coefficients generalize most naturally under the extra assumption of decidability. In particular the binomial theorem for types, which is stated as \cref{ex:binomial-theorem} and generalized in \cref{ex:distributive-pi-coprod}, rely on the use of decidable embeddings.
\end{rmk}

\begin{prp}\label{prp:equiv-binom-type}
  Consider two types $A$ and $B$, and a universe $\UU$ containing both $A$ and $B$. Then we have an equivalence
  \begin{equation*}
    \dbinomtype[\UU]{A}{B}\simeq \sum_{(P:A\to\decidableProp_\UU)}\Brck{B\simeq\sm{a:A}P(a)}.
  \end{equation*}
  from the binomial type $\dbinomtype[\UU]{A}{B}$ to the type of decidable subtypes\index{decidable subtype} of $A$ that are merely equivalent to $B$.
\end{prp}

\begin{proof}
  This equivalence follows from \cref{thm:classifier-subuniverse}, by which we have
  \begin{equation*}
    \Big(\sm{X:\UU}X\demb A\Big)\simeq (A\to\decidableProp_\UU).\qedhere
  \end{equation*}
\end{proof}

\begin{rmk}
  Combining \cref{cor:decidable-Prop,prp:equiv-binom-type}, we obtain an equivalence
  \begin{equation*}
    \dbinomtype[\UU]{A}{B}\simeq \sum_{(f:A\to\bool)}\Brck{B\simeq\sm{a:A}f(a)=\btrue}.
  \end{equation*}
  for any universe $\UU$ that contains both $A$ and $B$. This equivalence is important, because the right hand side doesn't depend on the universe $\UU$. Therefore we will simply write $\dbinomtype{A}{B}$ for $\dbinomtype[\UU]{A}{B}$, if the universe $\UU$ contains both $A$ and $B$.
\end{rmk}
  
\begin{lem}\label{prp:binomtype-recursion}
  For any two types $A$ and $B$, we have equivalences\index{binomial type!recursive relations}
  \begin{align*}
    \dbinomtype{\emptyt}{\emptyt} & \simeq \unit & \dbinomtype{A+\unit}{\emptyt} & \simeq \unit \\
    \dbinomtype{\emptyt}{B+\unit} & \simeq \emptyt & \dbinomtype{A+\unit}{B+\unit} & \simeq \dbinomtype{A}{B}+\dbinomtype{A}{B+\unit}.
  \end{align*}
\end{lem}

\begin{proof}
  For the first two equivalences, we prove that $\dbinomtype{X}{\emptyt}$ is contractible for any type $X$. To see this, we first note that the type $\UU_\emptyt$ is contractible. Indeed, at the center of contraction we have the empty type, and any two types that are merely equivalent to the empty type are empty and hence equivalent. Therefore it follows that
  \begin{equation*}
    \dbinomtype{X}{\emptyt}\simeq \emptyt\demb X.
  \end{equation*}
  The type of decidable embeddings $\emptyt\demb X$ is contractible, because the type $\emptyt\to X$ is contractible with the map $\exfalso:\emptyt\to X$ at the center of contraction, which is of course a decidable embedding.

  Next, the fact that the binomial type $\dbinomtype{\emptyt}{B+\unit}$ is empty follows from the fact that the type of maps $X\to\emptyt$ is empty for any type $X$ merely equivalent to $B+\unit$. 

  For the last equivalence we will use \cref{prp:equiv-binom-type}. Using the universal property of $A+\unit$, we see that
  \begin{equation*}
    \dbinomtype{A+\unit}{B+\unit}\simeq \sm{P:A\to\decidableProp_\UU}\sm{Q:\decidableProp_\UU}\brck{(B+\unit)\simeq \sm{a:A}P(a)+ Q}.
  \end{equation*}
  Using the fact that $\decidableProp_\UU\simeq\Fin{2}$, observe that we have an equivalence
  \begin{align*}
    & \sm{Q:\decidableProp_\UU}\brck{(B+\unit)\simeq(\sm{a:A}P(a)+Q)} \\
    & \qquad\simeq \brck{(B+\unit)\simeq(\sm{a:A}P(a)+\unit)}+\brck{(B+\unit)\simeq\sm{a:A}P(a)}.
  \end{align*}
  Furthermore, note that it follows from \cref{prp:is-injective-maybe} that
  \begin{equation*}
    \brck{(B+\unit)\simeq(\sm{a:A}P(a)+\unit)}\simeq\brck{B\simeq\sm{a:A}P(a)}.
  \end{equation*}
  Thus we see that
  \begin{align*}
    \dbinomtype{A+\unit}{B+\unit} & \simeq \Big(\sm{P:A\to\decidableProp_\UU}\brck{B\simeq\sm{a:A}P(a)}\Big) \\
    & \qquad\qquad +\Big(\sm{P:A\to\decidableProp_\UU}\brck{(B+\unit)\simeq\sm{a:A}P(a)}\Big).\qedhere
  \end{align*}
\end{proof}

\begin{thm}
  If $A$ and $B$ are finite types of cardinality $n$ and $k$, respectively, then the type $\dbinomtype{A}{B}$ is finite of cardinality $\binom{n}{k}$.\index{finite type!binomial type}\index{binomial type!is finite}
\end{thm}

\begin{proof}
  The claim that the type $\binomtype{A}{B}$ is finite of cardinality $\binom{n}{k}$ is a proposition, so we may assume $e:\Fin{n}\simeq A$ and $f:\Fin{k}\simeq B$. The claim now follows by induction on $n$ and $k$, using \cref{prp:binomtype-recursion}.
\end{proof}

\begin{rmk}
  It is perhaps remarkable that the type $\sm{X:\UU_B}X\demb A$ is a good generalisation of the binomial coefficients to types. Note that when $A$ and $B$ are finite types of cardinality $n$ and $k$, respectively, then the type $B\demb A$ has a factor $k!$ too many elements. When we seemingly enlarge it by the type $\UU_B$ of all types merely equivalent to $B$, it turns out that we obtain the correct generalisation of the binomial coefficients.

  One reason why it works is that the identity type of $\sm{X:\UU_B}X\demb A$ is characterized, via the univalence axiom, by
  \begin{equation*}
    ((X,f)=(Y,g))\simeq\sm{e:X\simeq Y}f\htpy g\circ e. 
  \end{equation*}
  Therefore it follows that for any two decidable embeddings $f,g:B\demb A$, if $f$ and $g$ are the same up to a permutation on $B$, then we get an identification $(B,f)=(B,g)$ in the type $\sm{X:\UU_B}X\demb A$.

  From a group theoretic perspective we may observe that the automorphism group $B\simeq B$ acts freely on the set of decidable embeddings $\B\demb A$, and the type $\sm{X:\UU_B}X\hookrightarrow A$ can be viewed as the type of orbits of that action. Since this action of $\Aut(B)$ on $B\demb A$ is free, we see that the number of orbits is $\frac{1}{k!}$ times the number of elements in $B\demb A$. 
\end{rmk}
\index{binomial type|)}

\begin{exercises}
  \exitem \label{ex:istrunc_UUtrunc}
  \begin{subexenum}
  \item Use the univalence axiom to show that the type $\sm{A:\UU}\iscontr(A)$ of all contractible types in $\UU$ is contractible.\index{universe!of contractible types}\index{contractible type}
  \item Use the univalence axiom and \cref{ex:isprop_istrunc,ex:isprop_isequiv} to show that the universe of $k$-types\index{universe!of k-types@{of $k$-types}}\index{U leq k@{$\UU^{\leq k}$}}\index{k-type@{$k$-type}!universe of k-types@{universe of $k$-types}}\index{truncated type!universe of k-types@{universe of $k$-types}}\index{universe!of k-types@{of $k$-types}!is a k-type@{is a $k$-type}}\index{U leq k@{$\UU^{\leq k}$}!is a k-type@{is a $k$-type}}
    \begin{equation*}
      \UU^{\leq k}\defeq \sm{X:\UU}\istrunc{k}(X)
    \end{equation*}
    is a $(k+1)$-type, for any $k\geq -2$.
  \item Show that $\prop_\UU$ is not a proposition.\index{universe!of propositions}\index{Prop@{$\prop_\UU$}}
  \item Show that the universe $\Set_\UU$ of sets\index{universe!of sets} in $\UU$ is not a set. \index{Set@{$\Set_\UU$}!is not a set}
  \end{subexenum}
  \exitem Give an example of a type family $B$ over a type $A$ for which the implication
  \begin{equation*}
    \neg\Big(\prd{x:A}B(x)\Big) \to \Big(\sm{x:A}\neg B(x)\Big)
  \end{equation*}
  is false.
  \exitem Show that the law of excluded middle holds if and only if every set has decidable equality.\index{law of excluded middle}\index{axiom!law of excluded middle}
  \exitem Consider a type $A$ and a univalent universe $\UU$ containing $A$. Construct an equivalence
  \begin{equation*}
    A\simeq\sm{B:A\to\UU}\iscontr\left(\sm{a:A}B(a)\right).
  \end{equation*}
  \exitem \label{ex:surjective-precomp}Consider a map $f:A\to B$. Show that the following are equivalent:
  \begin{enumerate}
  \item The map $f$ is surjective.\index{surjective map}
  \item For every set $C$, the precomposition function
    \begin{equation*}
      \blank\circ f:(B\to C)\to (A\to C)
    \end{equation*}
    is an embedding.
  \end{enumerate}
  Hint: To show that (ii) implies (i), use the assumption with the set $C\defeq\prop_\UU$, where $\UU$ is a univalent universe containing both $A$ and $B$.
  \exitem (Escard\'o)\label{ex:idtype-is-emb} Consider a type $A$ in $\UU$. Show that the identity type, seen as a function\index{Id A@{$\idtypevar{A}$}!is an embedding}\index{is an embedding!Id A@{$\idtypevar{A}$}}
  \begin{equation*}
    \idtypevar{} : A \to (A\to\UU),
  \end{equation*}
  is an embedding.
  \exitem
  \begin{subexenum}
  \item For any type $A$ in $\UU$, consider the function\index{S A@{$\Sigma_A$}|see {dependent pair type}}\index{S A@{$\Sigma_A$}!is a k-truncated map@{is a $k$-truncated map}}\index{truncated map!S A@{$\Sigma_A$}}
    \begin{equation*}
      \Sigma_A : (A \to \UU) \to \UU,
    \end{equation*}
    which takes a family $B$ of $\UU$-small types to its $\Sigma$-type. Show that the following are equivalent:
    \begin{enumerate}
    \item The type $A$ is $k$-truncated.
    \item The map $\Sigma_A$ is $k$-truncated.
    \end{enumerate}
    Hint: Construct an equivalence $\fib{\Sigma_A}{X}\simeq (X\to A)$.\index{fiber!of S A@{of $\Sigma_A$}}
  \item Show that the map ${+}:\UU\times\UU\to\UU$, which takes $(A,B)$ to the coproduct $A+B$, is $0$-truncated.\index{A + B@{$A+B$}!is a 0-truncated map@{${\blank}+{\blank}$ is a $0$-truncated map}}\index{truncated map!-+-@{${\blank}+{\blank}$}}
  \end{subexenum}
  \exitem (Escard\'o) Consider a proposition $P$ and a universe $\UU$ containing $P$. Show that the map
  \begin{equation*}
    \Pi_P : (P\to \UU)\to\UU,
  \end{equation*}
  given by $A\mapsto\prd{p:P}A(p)$, is an embedding.\index{P P@{$\Pi_P$}!is an embedding}
  \exitem Consider two types $A$ and $B$ and a universe $\UU$ containing both $A$ and $B$. A \define{binary correspondence}\index{binary correspondence|textbf} $R:A\to(B\to\UU)$ is said to be a \define{function}\index{binary correspondence!function|textbf}\index{function!binary correspondence|textbf} if it satisfies the condition\index{is-function@{$\isfunction(R)$}|textbf}
  \begin{equation*}
    \isfunction(R)\defeq\prd{a:A}\iscontr\Big(\sm{b:B}R(a,b)\Big),
  \end{equation*}
  and $R$ is said to be an \define{opposite function}\index{function!opposite function|textbf}\index{opposite function|textbf} if the \define{opposite correspondence}\index{correspondence!opposite correspondence}\index{opposite correspondence}\index{R op@{$\op{R}$}|see {opposite correspondence}} $\op{R}:B\to(A\to\UU)$ given by $\op{R}(b,a)\defeq R(a,b)$ is functional.
  \begin{subexenum}
  \item Construct an equivalence
    \begin{equation*}
      (A\to B)\simeq \sm{R:A\to(B\to\UU)}\isfunction(R).
    \end{equation*}
  \item Construct an equivalence
    \begin{equation*}
      (A\simeq B)\simeq\sm{R:A\to (B\to\UU)}\isfunction(R)\times\isfunction(\op{R}).
    \end{equation*}
  \end{subexenum}
  \exitem
  \begin{subexenum}
  \item For any $k:\N$, show that the type\index{BS n@{$\BS_n$}}
    \begin{equation*}
      \sm{X:\BS_{k+1}}\Fin{k}\hookrightarrow X
    \end{equation*}
    is contractible.
  \item More generally, construct for any $k,l:\N$ and any $k$-element type $A$ an equivalence
    \begin{equation*}
      \Big(\sm{X:\BS_{k+l}}A\hookrightarrow X\Big)\simeq \BS_l
    \end{equation*}
  \end{subexenum}
  \exitem \label{ex:complement-Fk}
  \begin{subexenum}
  \item For any type $A$, construct an equivalence
  \begin{equation*}
    \UU_A \simeq \sum_{(X:\UU_{A+\unit})}\dbinomtype{X}{\unit}.
  \end{equation*}
  \item For any $k:\N$, construct an equivalence\index{BS n@{$\BS_n$}}
    \begin{equation*}
      \Big(\sm{X:\BS_{k+1}}X\Big)\simeq \BS_k.
    \end{equation*}
    In other words, show that the type of $(k+1)$-element types equipped with a point is equivalent to the type of $k$-element types. Conclude that the type of pointed finite types is equivalent to the type of finite types, i.e., conclude that we have an equivalence\index{F@{$\F$}}
  \begin{equation*}
    \Big(\sm{X:\F}X\Big)\simeq \F.
  \end{equation*}
  \end{subexenum}
  \exitem 
  \begin{subexenum}
  \item Show that for $k\neq 2$, the type\index{BS n@{$\BS_n$}}
    \begin{equation*}
      \prd{X:\BS_k}X\to X
    \end{equation*}
    is contractible. Conclude that the type $\prd{X:\BS_k}X\simeq X$ is also contractible. Hint: Use \cref{ex:complement-Fk}.
  \item Show that the type\index{BS 2@{$\BS_2$}}
    \begin{equation*}
      \prd{X:\BS_2}X\to X
    \end{equation*}
    is equivalent to $\Fin{2}$. Conclude that the type $\prd{X:\BS_2}X\simeq X$ is also equivalent to $\Fin{2}$.
  \end{subexenum}
  \exitem Consider a type $A$.
  \begin{subexenum}
  \item Recall from \cref{ex:isolated-point} that an element $a:A$ is isolated\index{isolated point} if and only if the map $\const_a:\unit\to A$ is a decidable embedding. Construct an equivalence
  \begin{equation*}
    \dbinomtype{A}{\unit}\simeq \sm{a:A}\isisolated(a).
  \end{equation*}
  \item Construct an equivalence
    \begin{equation*}
      \dbinomtype{A}{\unit}\simeq\Big(\sm{X:\UU}(X+\unit)\simeq A\Big).
    \end{equation*}
    Conclude that the map $X\mapsto X+\unit$ on a univalent universe $\UU$ is $0$-truncated.\index{truncated map!-+1@{${\blank}+\unit$}}
  \item More generally, construct an equivalence\index{binomial type}
    \begin{equation*}
      \dbinomtype{A}{B} \simeq \sm{X:\UU_B}\sm{Y:\UU}(X+Y\simeq A).
    \end{equation*}
  \end{subexenum}
  \exitem \label{ex:binomial-theorem}For any $(X,i):\dbinomtype{A}{B}$, we define $A\setminus(X,i)\defeq (A\setminus X,A\setminus i):\dbinomtype{A}{B}$, where
  \begin{align*}
    A\setminus X & \defeq \sm{a:A}\neg(\fib{i}{a}) \\
    A\setminus i & \defeq \proj 1.
  \end{align*}
  Now consider a finite type $X$ and two arbitrary types $A$ and $B$. Construct an equivalence\index{binomial theorem}
  \begin{equation*}
    (A+B)^X\simeq\sm{k:\N}\sm{(Y,i):\dbinomtype{X}{\Fin{k}}}A^Y\times B^{X\setminus Y}.
  \end{equation*}
  \exitem Let $\UU$ be a univalent universe.
  \begin{subexenum}
  \item Consider a section-retraction pair
    \begin{equation*}
      \begin{tikzcd}
        A \arrow[r,"i"] & X \arrow[r,"r"] & A
      \end{tikzcd}
    \end{equation*}
    with $H:r\circ i\htpy \idfunc$. Show that if $X$ is $\UU$-small, then so is $A$. Hint: Use \cref{ex:retracts-as-limits}.\index{small type!retracts of small types are small}\index{retract!retracts of small types are small}
  \item Consider two inhabited types $A$ and $B$. Show that if the product $A\times B$ is $\UU$-small, then so are the types $A$ and $B$.\index{small type!products of small types are small}\index{cartesian product type!products of small types are small}
  \end{subexenum}
  \exitem Consider a finite type $X$ and a univalent universe $\UU$ containing $X$. Show that the type\index{Retr X@{$\Retr_\UU(X)$}!is finite}\index{finite type!Retr X of a finite type X@{$\Retr_\UU(X)$ of a finite type $X$}}
  \begin{equation*}
    \Retr_\UU(X)\defeq \sm{A:\UU}\sm{i:A\to X}\sm{r:X\to A}r\circ i\htpy\idfunc
  \end{equation*}
  of all retracts of $X$ is finite.
  \exitem Consider a $k$-truncated type $X$ and a univalent universe $\UU$ containing $X$. Show that the type
  \begin{equation*}
    \mathsf{Retr}_\UU(X)
  \end{equation*}
  of all retracts of $X$ is $k$-truncated.\index{Retr X@{$\Retr_\UU(X)$}!is a k-truncated type@{is a $k$-truncated type}}
  \exitem \label{ex:surjection-into-k-type}For any type $A$ and any $k\geq-1$, show that the type
  \begin{equation*}
    \sm{X:\typele{k}}A\twoheadrightarrow X
  \end{equation*}
  of $k$-truncated types $X$ equipped with a surjective map $A\twoheadrightarrow X$ is $k$-truncated, even though the type $\typele{k}$ itself is $(k+1)$-truncated.
  \exitem
  \begin{subexenum}
    \item Show that for $k\geq 3$, the type\index{BS n@{$\BS_n$}}
    \begin{equation*}
      \prd{X:\BS_k}(X+X)\emb (X\times X)+\unit
    \end{equation*}
    is empty, even though the inequality $2k\leq k^2+1$ holds for all $k:\N$.
  \item Show that the type\index{BS 2@{$\BS_2$}}
    \begin{equation*}
      \prd{X:\BS_2}(X+X)\emb (X\times X)+\unit
    \end{equation*}
    is equivalent to $\Fin{8}$.
  \end{subexenum}
  \exitem \label{ex:prime}For any natural number $n$ consider the type
  \begin{equation*}
    \tilde{D}_n\defeq\sm{X:\BS_2}\sm{Y:X\to\F}\Big(\Fin{n}\simeq\prd{x:X}Y(x)\Big).
  \end{equation*}
  \begin{subexenum}
  \item Show that $\tilde{D}_{1}\simeq \BS_2$.\index{BS 2@{$\BS_2$}}
  \item Show that $\tilde{D}_{n}$ is contractible if and only if $n$ is prime\index{prime number}.
  \item Show that $\tilde{D}_n$ is a set if and only if $n$ is not a square.
  \end{subexenum}
\end{exercises}
\index{univalence axiom|)}
\index{axiom!univalence|)}

%%% Local Variables:
%%% mode: latex
%%% TeX-master: "hott-intro"
%%% End:
