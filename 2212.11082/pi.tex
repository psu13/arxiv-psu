\section{Dependent function types}
\label{sec:pi}

\index{Pi-type@{$\Pi$-type}|see {dependent function type}}
\index{dependent function type|(}
A fundamental concept of dependent type theory is that of a dependent function. A dependent function is a function of which the type of the output may depend on the input. For example, when we concatenate a vector of length $m$ with a vector of length $n$, we obtain a vector of length $m+n$. Dependent functions are a generalization of ordinary functions, because an ordinary function $f:A\to B$ is a function of which the output $f(x)$ has type $B$ regardless of the value of $x$.

\subsection{The rules for dependent function types}
Consider a section $b$ of a family $B$ over $A$ in context $\Gamma$, i.e., consider
\begin{equation*}
  \Gamma,x:A\vdash b(x):B(x).
\end{equation*}
From one point of view, such a section $b$ is an operation or assignment $x\mapsto b(x)$, or a program\index{program}, that takes as input $x:A$ and produces a term $b(x):B(x)$. From a more mathematical point of view we see $b$ as a choice of an element of each $B(x)$. In other words, we may see $b$ as a function that takes $x:A$ to $b(x):B(x)$. Note that the type $B(x)$ of the output may depend on $x:A$. The assignment $x\mapsto b(x)$ is in this sense a \emph{dependent} function. The type of all such dependent functions is called the \define{dependent function type}, and we will write
\begin{equation*}
  \prd{x:A}B(x)
\end{equation*}
for the type of dependent functions. There are four principal rules for $\Pi$-types:
\begin{enumerate}
\item The \emph{formation rule}, which tells us how we may form dependent function types.
\item The \emph{introduction rule}, which tells us how to introduce new terms of dependent function types.
\item The \emph{elimination rule}, which tells us how to use arbitrary terms of dependent function types.
\item The \emph{computation rules}, which tell us how the introduction and elimination rules interact. These computation rules guarantee that every term of a dependent function type is indeed a dependent function taking the values by which it is defined.
\end{enumerate}
In the cases of the formation rule, the introduction rule, and the elimination rule, we also need rules that assert that all the constructions respect judgmental equality. Those rules are called \define{congruence rules}, and they are part of the specification of dependent function types.

\subsubsection{The $\Pi$-formation rule}
\index{dependent function type!formation rule|textbf}
The \define{$\Pi$-formation rule} tells us how $\Pi$-types are constructed. The idea of $\Pi$-types is that $\prd{x:A}B(x)$ is a type of \define{dependent functions}\index{dependent function type|textbf}, for any type family $B$ of types over $A$, so the $\Pi$-formation rule is as follows:\index{rules!for dependent function types!formation|textbf}\index{P (x:A) B(x)@{$\prd{x:A}B(x)$}|see{dependent function type}}\index{P (x:A) B(x)@{$\prd{x:A}B(x)$}|textbf}
\begin{prooftree}
\AxiomC{$\Gamma,x:A\vdash B(x)~\type$}
\RightLabel{$\Pi$.}
\UnaryInfC{$\Gamma\vdash \prd{x:A}B(x)~\type$}
\end{prooftree}
This rule simply states that in order to form the type $\prd{x:A}B(x)$ in context $\Gamma$, we must have a type family $B$ over $A$ in context $\Gamma$.

We also require that the operation of forming dependent function types respects judgmental equality. This is postulated in the \define{congruence rule} for $\Pi$-types:
\index{rules!for dependent function types!congruence|textbf}
\index{dependent function type!congruence rule|textbf}
\begin{prooftree}
\AxiomC{$\Gamma\vdash A\jdeq A'~\type$}
\AxiomC{$\Gamma,x:A\vdash B(x)\jdeq B'(x)~\textrm{type}$}
\RightLabel{$\Pi$-eq.}
\BinaryInfC{$\Gamma\vdash \prd{x:A}B(x)\jdeq\prd{x:A'}B'(x)~\type$}
\end{prooftree}

\subsubsection{The $\Pi$-introduction rule}
The introduction rule for dependent functions tells us how we may construct dependent functions of type $\prd{x:A}B(x)$. The idea is that a dependent function $f:\prd{x:A}B(x)$ is an operation that takes an $x:A$ to $f(x):B(x)$. Hence the introduction rule of dependent functions postulates that, in order to construct a dependent function one has to construct a term $b(x):B(x)$ indexed by $x:A$ in context $\Gamma$, i.e.:
\begin{prooftree}
  \AxiomC{$\Gamma,x:A \vdash b(x) : B(x)$}
  \RightLabel{$\lambda$.}
  \UnaryInfC{$\Gamma\vdash \lam{x}b(x) : \prd{x:A}B(x)$}
\end{prooftree}
This introduction rule%
\index{dependent function type!introduction rule|see {$\lambda$-abstraction}}
for dependent functions is also called the \define{$\lambda$-abstraction rule}%
\index{lambda-abstraction@{$\lambda$-abstraction}|textbf}%
\index{rules!for dependent function types!lambda-abstraction@{$\lambda$-abstraction}|textbf}%
\index{dependent function type!lambda-abstraction@{$\lambda$-abstraction}|textbf}, and we also say that the $\lambda$-abstraction $\lam{x}b(x)$ \define{binds} the variable $x$ in $b$. Just like ordinary mathematicians, we will sometimes write $x\mapsto b(x)$ for a function $\lam{x}b(x)$. The map $n\mapsto n^2$ is an example.

We will also require that $\lambda$-abstraction respects judgmental equality. Therefore we postulate the \define{congruence rule} for $\lambda$-abstraction,
\index{rules!for dependent function types!lambda-congruence@{$\lambda$-congruence}}
\index{lambda-congruence@{$\lambda$-congruence}}
\index{dependent function type!lambda-congruence@{$\lambda$-congruence}}
which asserts that\label{page:lambda-eq}
\begin{prooftree}
  \AxiomC{$\Gamma,x:A \vdash b(x)\jdeq b'(x) : B(x)$}
  \RightLabel{$\lambda$-eq.}
  \UnaryInfC{$\Gamma\vdash \lam{x}b(x)\jdeq \lam{x}b'(x) : \prd{x:A}B(x)$}
\end{prooftree}

\subsubsection{The $\Pi$-elimination rule}

\index{dependent function type!elimination rule|see {evaluation}}
The elimination rule for dependent function types provides us with a way to \emph{use} dependent functions. The way to use a dependent function is to evaluate it at an argument of the domain type. The $\Pi$-elimination rule is therefore also called the \define{evaluation rule}\index{evaluation|textbf}\index{rules!for dependent function types!evaluation|textbf}\index{dependent function type!evaluation|textbf}:
\begin{prooftree}
\AxiomC{$\Gamma\vdash f:\prd{x:A}B(x)$}
\RightLabel{$ev$.}
\UnaryInfC{$\Gamma,x:A\vdash f(x) : B(x)$}
\end{prooftree}
This rule asserts that given a dependent function $f:\prd{x:A}B(x)$ in context $\Gamma$ we obtain a term $f(x)$ of type $B(x)$ indexed by $x:A$ in context $\Gamma$. Again we require that evaluation respects judgmental equality:
\begin{prooftree}
  \AxiomC{$\Gamma\vdash f\jdeq f':\prd{x:A}B(x)$}
  \RightLabel{$ev$-eq.}
  \UnaryInfC{$\Gamma,x:A\vdash f(x)\jdeq f'(x):B(x)$}
\end{prooftree}

\subsubsection{The $\Pi$-computation rules}

\index{dependent function type!computation rules|see {$\beta$- and $\eta$-rules}}
We now postulate rules that specify the behavior of functions. First, we have a rule that asserts that a function of the form $\lam{x}b(x)$ behaves as expected: when we evaluate it at $x:A$, then we obtain the value $b(x):B(x)$. This rule is called the \define{$\beta$-rule}\index{b-rule for P-types@{$\beta$-rule for $\Pi$-types}|textbf}\index{rules!for dependent function types!b-rule@{$\beta$-rule}|textbf}\index{dependent function type!b-rule@{$\beta$-rule}|textbf}
\begin{prooftree}
\AxiomC{$\Gamma,x:A \vdash b(x) : B(x)$}
\RightLabel{$\beta$.}
\UnaryInfC{$\Gamma,x:A \vdash (\lambda y.b(y))(x)\jdeq b(x) : B(x)$}
\end{prooftree}
Second, we postulate a rule that asserts that all elements of a $\Pi$-type are (dependent) functions. This rule is known as the \define{$\eta$-rule}\index{eta-rule@{$\eta$-rule} for Pi-types@{for $\Pi$-types}|textbf}\index{rules!for dependent function types!eta-rule@{$\eta$-rule}|textbf}\index{dependent function type!eta-rule@{$\eta$-rule}|textbf}
\begin{prooftree}
\AxiomC{$\Gamma\vdash f:\prd{x:A}B(x)$}
\RightLabel{$\eta$.}
\UnaryInfC{$\Gamma \vdash \lam{x}f(x) \jdeq f : \prd{x:A}B(x)$}
\end{prooftree}
In other words, the computation rules ($\beta$ and $\eta$) for dependent function types postulate that $\lambda$-abstraction rule and the evaluation rule are mutual inverses. This completes the specification of dependent function types.

\subsection{Ordinary function types}

An important special case of $\Pi$-types arises when both $A$ and $B$ are types in context $\Gamma$. In this case, we can first weaken $B$ by $A$ and then apply the $\Pi$-formation rule to obtain the type $A\to B$ of \emph{ordinary} functions from $A$ to $B$, as in the following derivation:
\begin{prooftree}
\AxiomC{$\Gamma\vdash A~\textrm{type}$}
\AxiomC{$\Gamma\vdash B~\textrm{type}$}
\RightLabel{$W$}
\BinaryInfC{$\Gamma,x:A\vdash B~\textrm{type}$}
\RightLabel{$\Pi$}
\UnaryInfC{$\Gamma\vdash \prd{x:A}B~\textrm{type}$.}
\end{prooftree}
A term $f:\prd{x:A}B$ is a function that takes an argument $x:A$ and returns $f(x):B$. In other words, terms of type $\prd{x:A}B$ are indeed ordinary functions from $A$ to $B$. Therefore, we define the type $A\to B$\index{A arrow B@{$A\to B$}|see {function type}}\index{A arrow B@{$A\to B$}|textbf} of \define{(ordinary) functions}\index{function type|textbf} from $A$ to $B$ by
\begin{equation*}
  A\to B\defeq\prd{x:A}B.
\end{equation*}
If $f:A\to B$ is a function, then the type $A$ is also called the \define{domain} of $f$, and the type $B$ is also called the \define{codomain} of $f$.

Sometimes we will also write $B^A$\index{B^A@{$B^A$}|see {function type}} for the type $A\to B$.  Formally, we make such definitions by adding one more line to the above derivation:
\begin{prooftree}
\AxiomC{$\Gamma\vdash A~\textrm{type}$}
\AxiomC{$\Gamma\vdash B~\textrm{type}$}
\RightLabel{$W$}
\BinaryInfC{$\Gamma,x:A\vdash B~\textrm{type}$}
\RightLabel{$\Pi$}
\UnaryInfC{$\Gamma\vdash \prd{x:A}B~\textrm{type}$}
\UnaryInfC{$\Gamma\vdash A\to B \defeq \prd{x:A}B~\textrm{type}$.}
\end{prooftree}

\begin{rmk}
  More generally, we can make definitions at the end of a derivation if the conclusion is a certain type in context, or if the conclusion is a certain term of a type in context. Suppose, for instance, that we have a derivation
  \begin{prooftree}
    \AxiomC{$\mathcal{D}$}
    \UnaryInfC{$\Gamma\vdash a:A$,}
  \end{prooftree}
  in which the derivation $\mathcal{D}$ makes use of the premises $\mathcal{H}_1$, \ldots,$\mathcal{H}_n$. If we wish to make a definition $\newdef\defeq a$, then we can extend the derivation tree with
  \begin{prooftree}
    \AxiomC{$\mathcal{D}$}
    \UnaryInfC{$\Gamma\vdash a:A$}
    \UnaryInfC{$\Gamma\vdash\newdef\defeq a:A$.}
  \end{prooftree}
  The effect of such a definition is that we have extended our type theory with a new constant $\newdef$, for which the following inference rules are valid
  \begin{center}
    \begin{minipage}{.45\textwidth}
      \begin{prooftree}
        \AxiomC{$\mathcal{H}_1$\quad $\mathcal{H}_2$ \quad \dots \quad $\mathcal{H}_n$}
        \UnaryInfC{$\Gamma\vdash\newdef:A$}
      \end{prooftree}
    \end{minipage}
    \begin{minipage}{.45\textwidth}
      \begin{prooftree}
        \AxiomC{$\mathcal{H}_1$\quad $\mathcal{H}_2$ \quad \dots \quad $\mathcal{H}_n$}
        \UnaryInfC{$\Gamma\vdash\newdef\jdeq a:A$.}
      \end{prooftree}
    \end{minipage}
  \end{center}
  In our example of the definition of the ordinary function type $A\to B$, we therefore have by definition the following valid inference rules
  \begin{center}
    \begin{minipage}{.45\textwidth}
      \begin{prooftree}
        \AxiomC{$\Gamma\vdash A~\textrm{type}$}
        \AxiomC{$\Gamma\vdash B~\textrm{type}$}
        \BinaryInfC{$\Gamma\vdash A\to B~\textrm{type}$}
      \end{prooftree}
    \end{minipage}
    \begin{minipage}{.45\textwidth}
      \begin{prooftree}
        \AxiomC{$\Gamma\vdash A~\textrm{type}$}
        \AxiomC{$\Gamma\vdash B~\textrm{type}$}
        \BinaryInfC{$\Gamma\vdash A\to B\jdeq \prd{x:A}B~\textrm{type}$.}
      \end{prooftree}
    \end{minipage}
  \end{center}
  There are of course many such definitions throughout the development of dependent type theory, the univalent foundations of mathematics, and synthetic homotopy theory. They are all included in the index at the end of this book.
\end{rmk}

\begin{rmk}
  By the term conversion rules of \cref{ex:term_conversion} we can now use the rules for $\lambda$-abstraction, evaluation, and so on, to obtain corresponding rules for the ordinary function type $A\to B$. We give a brief summary of these rules, omitting the congruence rules.\index{rules!for function types}
  \begin{prooftree}
    \AxiomC{$\Gamma\vdash A~\textrm{type}$}
    \AxiomC{$\Gamma\vdash B~\textrm{type}$}
    \RightLabel{$\to$}
    \BinaryInfC{$\Gamma\vdash A\to B~\textrm{type}$}
  \end{prooftree}%
  \begin{center}
    \begin{minipage}{.55\textwidth}
      \begin{prooftree}
        \AxiomC{$\Gamma\vdash B~\textrm{type}$}
        \AxiomC{$\Gamma,x:A\vdash b(x):B$}
        \RightLabel{$\lambda$}
        \BinaryInfC{$\Gamma\vdash \lam{x}b(x):A\to B$}
      \end{prooftree}%
    \end{minipage}
    \begin{minipage}{.35\textwidth}
      \begin{prooftree}
        \AxiomC{$\Gamma\vdash f:A\to B$}
        \RightLabel{$ev$}
        \UnaryInfC{$\Gamma,x:A\vdash f(x):B$}
      \end{prooftree}%
    \end{minipage}
  \end{center}
  \begin{center}
    \begin{minipage}{.55\textwidth}
      \begin{prooftree}
        \AxiomC{$\Gamma\vdash B~\textrm{type}$}
        \AxiomC{$\Gamma,x:A\vdash b(x):B$}
        \RightLabel{$\beta$}
        \BinaryInfC{$\Gamma,x:A\vdash(\lam{y}b(y))(x)\jdeq b(x):B$}
      \end{prooftree}%
    \end{minipage}
    \begin{minipage}{.40\textwidth}
      \begin{prooftree}
        \AxiomC{$\Gamma\vdash f:A\to B$}
        \RightLabel{$\eta$}
        \UnaryInfC{$\Gamma\vdash\lam{x} f(x)\jdeq f:A\to B$}
      \end{prooftree}
    \end{minipage}
  \end{center}
\end{rmk}

Now we can use these rules to construct some familiar functions, such as the identity function $\idfunc:A\to A$ on an arbitrary type $A$, and the composition $g\circ f:A\to C$ of any two functions $f:A\to B$ and $g:B\to C$. 

\begin{defn}
For any type $A$ in context $\Gamma$, we define the \define{identity function}\index{identity function|textbf}\index{function!identity function|textbf} $\idfunc[A]:A\to A$\index{id A@{$\idfunc[A]$}|see {identity function}}\index{id A@{$\idfunc[A]$}|textbf} using the generic term:
\begin{prooftree}
\AxiomC{$\Gamma\vdash A~\textrm{type}$}
\UnaryInfC{$\Gamma,x:A\vdash x:A$}
\UnaryInfC{$\Gamma\vdash \lam{x}x:A\to A$}
\UnaryInfC{$\Gamma\vdash \idfunc[A]\defeq\lam{x}x:A\to A$.}
\end{prooftree}
\end{defn}

The identity function therefore satisfies the following inference rules:
  \begin{center}
    \begin{minipage}{.45\textwidth}
      \begin{prooftree}
        \AxiomC{$\Gamma\vdash A~\textrm{type}$}
        \UnaryInfC{$\Gamma\vdash \idfunc[A]:A\to A$}
      \end{prooftree}
    \end{minipage}
    \begin{minipage}{.45\textwidth}
      \begin{prooftree}
        \AxiomC{$\Gamma\vdash A~\textrm{type}$}
        \UnaryInfC{$\Gamma\vdash \idfunc[A]\jdeq\lam{x}x:A\to A$.}
      \end{prooftree}
    \end{minipage}
  \end{center}

Next, we define the composition of functions. We will introduce the composition operation itself as a function $\comp$ that takes two arguments: the first argument is a function $g:B\to C$, and the second argument is a function $f:A\to B$. The output is a function $\comp(g,f):A\to C$, for which we often write $g\circ f$.

\begin{rmk}
  Since composition is a function that takes multiple arguments, we need to know how to represent such functions. Types of functions with multiple arguments can be formed by iterating the $\Pi$-formation rule or the $\to$-formation rule. For example, a function
  \begin{equation*}
    f:A\to (B\to C)
  \end{equation*}
  takes two arguments: first it takes an argument $x:A$, and the output $f(x)$ has type $B\to C$. This is again a function type, so $f(x)$ is a function that takes an argument $y:B$, and its output $f(x)(y)$ has type $C$. We will usually write $f(x,y)$ for $f(x)(y)$.

  Similarly, when $C(x,y)$ is a family of types indexed by $x:A$ and $y:B(x)$, then we can form the dependent function type $\prd{x:A}\prd{y:B(x)}C(x,y)$. In the special case where $C(x,y)$ is a family of types indexed by two elements $x,y:A$ of the same type, then we often write
  \begin{equation*}
    \prd{x,y:A}C(x,y)
  \end{equation*}
  for the type $\prd{x:A}\prd{y:A}C(x,y)$.

  With the idea of iterating function types, we see that type of the composition operation $\comp$ is
  \begin{equation*}
    (B\to C)\to ((A\to B)\to (A\to C)).
  \end{equation*}
  It is the type of functions, taking a function $g:B\to C$, to the type of functions $(A\to B)\to (A\to C)$. Thus, $\comp(g)$ is again a function, mapping a function $f:A\to B$ to a function of type $A\to C$.
\end{rmk}

\begin{defn}
For any three types $A$, $B$, and $C$ in context $\Gamma$, there is a \define{composition}\index{function!composition|textbf}\index{composition!of functions|textbf} operation
\begin{equation*}
\comp:(B\to C)\to ((A\to B)\to (A\to C)).
\end{equation*}
We will usually write $g\circ f$\index{g o f@{$g\circ f$}|textbf} for $\comp(g,f)$\index{comp(g,f)@{$\comp(g,f)$}|textbf}\index{comp(g,f)@{$\comp(g,f)$}|see {composition, of functions}}.
\end{defn}

\begin{constr}
  The idea of the definition is to define $\comp(g,f)$ to be the function $\lam{x}g(f(x))$. The function $\comp$ is therefore defined as
  \begin{equation*}
    \comp\defeq \lam{g}\lam{f}\lam{x}g(f(x)).
  \end{equation*}
  The derivation we use to construct $\comp$ is as follows:
  \begin{small}
  \begin{prooftree}
    \AxiomC{$\Gamma\vdash A~\type$}
    \AxiomC{$\Gamma\vdash B~\type$}
    \RightLabel{(a)}
    \BinaryInfC{$\Gamma,f:B^A,x:A\vdash f(x):B$}
    \UnaryInfC{$\Gamma,g:C^B,f:B^A,x:A\vdash f(x):B$}
    \AxiomC{$\Gamma\vdash B~\type$}
    \AxiomC{$\Gamma\vdash C~\type$}
    \RightLabel{(b)}
    \BinaryInfC{$\Gamma,g:C^B,y:B\vdash g(y):C$}
    \UnaryInfC{$\Gamma,g:C^B,f:B^A,y:B\vdash g(y):C$}
    \UnaryInfC{$\Gamma,g:C^B,f:B^A,x:A,y:B\vdash g(y):C$}
    \BinaryInfC{$\Gamma,g:C^B,f:B^A,x:A\vdash g(f(x)) : C$}
    \UnaryInfC{$\Gamma,g:C^B,f:B^A\vdash \lam{x}g(f(x)):C^A$}
    \UnaryInfC{$\Gamma,g:B\to C\vdash \lam{f}\lam{x}g(f(x)):B^A\to C^A$}
    \UnaryInfC{$\Gamma\vdash\lam{g}\lam{f}\lam{x}g(f(x)):C^B\to (B^A\to C^A)$}
    \UnaryInfC{$\Gamma\vdash\comp\defeq \lam{g}\lam{f}\lam{x}g(f(x)):C^B\to (B^A\to C^A)$.}
  \end{prooftree}
  \end{small}
  Note, however, that we haven't derived the rules (a) and (b) yet. These rules assert that the \emph{generic functions} of $A\to B$ and $B\to C$ can also be evaluated. The formal derivation of this fact is as follows:
  \begin{prooftree}
    \AxiomC{$\Gamma\vdash A~\type$}
    \AxiomC{$\Gamma\vdash B~\type$}
    \BinaryInfC{$\Gamma\vdash A \to B~\type$}
    \UnaryInfC{$\Gamma,f:A\to B\vdash f:A\to B$}
    \UnaryInfC{$\Gamma,f:A\to B,x:A\vdash f(x):B$.}
  \end{prooftree}
  This completes the construction of $\comp$.
\end{constr}

In the remainder of this section we will see how to use the given rules for function types to derive the laws of a category\index{category laws!for functions} for functions. These are the laws that assert that function composition is associative and that the identity function satisfies the unit laws.

\begin{lem}
Composition of functions is associative\index{associativity!of function composition}, i.e., we can derive
\begin{prooftree}
\AxiomC{$\Gamma\vdash f:A\to B$}
\AxiomC{$\Gamma\vdash g:B\to C$}
\AxiomC{$\Gamma\vdash h:C\to D$}
\TrinaryInfC{$\Gamma \vdash (h\circ g)\circ f\jdeq h\circ(g\circ f):A\to D$.}
\end{prooftree}
\end{lem}

\begin{proof}
  The main idea of the proof is that both $((h\circ g)\circ f)(x)$ and $(h\circ (g\circ f))(x)$ evaluate to $h(g(f(x))$, and therefore $(h\circ g)\circ f$ and $h\circ(g\circ f)$ must be judgmentally equal. This idea is made formal in the following derivation:
  \begin{prooftree}
    \AxiomC{$\Gamma\vdash f:A\to B$}
    \UnaryInfC{$\Gamma,x:A\vdash f(x):B$}
    \AxiomC{$\Gamma\vdash g:B\to C$}
    \UnaryInfC{$\Gamma,y:B\vdash g(y):C$}
    \UnaryInfC{$\Gamma,x:A,y:B\vdash g(y):C$}
    \BinaryInfC{$\Gamma,x:A\vdash g(f(x)):C$}
    \AxiomC{$\Gamma\vdash h:C\to D$}
    \UnaryInfC{$\Gamma,z:C\vdash h(z):D$}
    \UnaryInfC{$\Gamma,x:A,z:C\vdash h(z):D$}
    \BinaryInfC{$\Gamma,x:A\vdash h(g(f(x))):D$}
    \UnaryInfC{$\Gamma,x:A\vdash h(g(f(x)))\jdeq h(g(f(x))):D$}
    \UnaryInfC{$\Gamma,x:A\vdash (h\circ g)(f(x))\jdeq h((g\circ f)(x)):D$}
    \UnaryInfC{$\Gamma,x:A\vdash ((h\circ g)\circ f)(x)\jdeq (h\circ (g \circ f))(x):D$}
    \UnaryInfC{$\Gamma\vdash (h\circ g)\circ f\jdeq h\circ(g\circ f):A\to D$.}
  \end{prooftree}
\end{proof}

\begin{lem}\label{lem:fun_unit}
Composition of functions satisfies the left and right unit laws\index{left unit law|see {unit laws}}\index{right unit law|see {unit laws}}\index{unit laws!for function composition}, i.e., we can derive
\begin{prooftree}
\AxiomC{$\Gamma\vdash f:A\to B$}
\UnaryInfC{$\Gamma\vdash \idfunc[B]\circ f\jdeq f:A\to B$}
\end{prooftree}
and
\begin{prooftree}
\AxiomC{$\Gamma\vdash f:A\to B$}
\UnaryInfC{$\Gamma\vdash f\circ\idfunc[A]\jdeq f:A\to B$.}
\end{prooftree}
\end{lem}

\begin{proof}
  Note that it suffices to derive that $\idfunc(f(x))\jdeq f(x)$ in context $\Gamma,x:A$, because once we derived this equality we can finish the derivation with
  \begin{prooftree}
    \AxiomC{$\vdots$}
    \UnaryInfC{$\Gamma,x:A\vdash\idfunc(f(x))\jdeq f(x):B$}
    \UnaryInfC{$\Gamma\vdash\lam{x}\idfunc(f(x))\jdeq\lam{x}f(x):A\to B$}
    \AxiomC{$\Gamma\vdash f:A\to B$}
    \UnaryInfC{$\Gamma\vdash\lam{x}f(x)\jdeq f:A\to B$}
    \BinaryInfC{$\Gamma\vdash\idfunc\circ f\jdeq f:A\to B$.}  
  \end{prooftree}
  The derivation of the equality $\idfunc(f(x))\jdeq f(x)$ in context $\Gamma,x:A$ is as follows:
  \begin{prooftree}
    \AxiomC{$\Gamma\vdash f:A\to B$}
    \UnaryInfC{$\Gamma,x:A\vdash f(x):B$}
    \AxiomC{$\Gamma\vdash A~\type$}
    \AxiomC{$\Gamma\vdash B~\type$}
    \UnaryInfC{$\Gamma,y:B\vdash\idfunc(y)\jdeq y:B$}
    \BinaryInfC{$\Gamma,x:A,y:B\vdash\idfunc(y)\jdeq y:B$}
    \BinaryInfC{$\Gamma,x:A\vdash\idfunc(f(x))\jdeq f(x):B$.}
  \end{prooftree}
  We leave the right unit law as \cref{ex:fun_right_unit}.
\end{proof}

\begin{exercises}
  \exitem \label{ex:eta_ext}The $\eta$-rule is often seen as a judgmental extensionality principle. Use the $\eta$-rule to show that if $f$ and $g$ take equal values, then they must be equal, i.e., give a derivation for the rule
  \begin{prooftree}
    \def\fCenter{\Gamma}
    \Axiom$\fCenter\vdash f:\prd{x:A}B(x)$
    \noLine
    \UnaryInf$\fCenter\vdash g:\prd{x:A}B(x)$
    \noLine
    \UnaryInf$\fCenter,x:A\vdash f(x)\jdeq g(x):B(x)$
    \UnaryInf$\fCenter\vdash f\jdeq g:\prd{x:A}B(x).$
  \end{prooftree}
  \exitem \label{ex:fun_right_unit}Give a derivation for the right unit law of \cref{lem:fun_unit}.\index{unit laws!for function composition}
  \exitem 
  \begin{subexenum}
  \item Construct the \define{constant map}\index{constant map|textbf}\index{function!constant map|textbf}\index{const x@{$\const_x$}|textbf}\index{function!const@{$\const$}|textbf}
    \begin{prooftree}
      \AxiomC{$\Gamma\vdash A~\textrm{type}$}
      \UnaryInfC{$\Gamma,y:B\vdash \const_y:A\to B$.}
    \end{prooftree}
  \item Show that
    \begin{prooftree}
      \AxiomC{$\Gamma\vdash f:A\to B$}
      \UnaryInfC{$\Gamma,z:C\vdash \const_z\circ f\jdeq\const_z : A\to C$.}
    \end{prooftree}
  \item Show that
    \begin{prooftree}
      \AxiomC{$\Gamma\vdash A~\textrm{type}$}
      \AxiomC{$\Gamma\vdash g:B\to C$}
      \BinaryInfC{$\Gamma,y:B\vdash g\circ\const_y\jdeq \const_{g(y)}:A\to C$.}
    \end{prooftree}
  \end{subexenum}
  \exitem \label{ex:swap}
  \begin{subexenum}
  \item Define the \define{swap function}\index{function!swap|textbf}\index{swap function|textbf}
    \begin{prooftree}
      \AxiomC{$\Gamma\vdash A~\mathrm{type}$}
      \AxiomC{$\Gamma\vdash B~\mathrm{type}$}
      \AxiomC{$\Gamma,x:A,y:B\vdash C(x,y)~\mathrm{type}$}
      \TrinaryInfC{$\Gamma\vdash \sigma:\Big(\prd{x:A}\prd{y:B}C(x,y)\Big)\to\Big(\prd{y:B}\prd{x:A}C(x,y)\Big)$}
    \end{prooftree}
    that swaps the order of the arguments.
  \item Show that
  \end{subexenum}
  \vspace{-.5\baselineskip}
  \begin{small}
    \begin{prooftree}
      \AxiomC{$\Gamma\vdash A~\mathrm{type}$}
      \AxiomC{$\Gamma\vdash B~\mathrm{type}$}
      \AxiomC{$\Gamma,x:A,y:B\vdash C(x,y)~\mathrm{type}$}
      \TrinaryInfC{$\Gamma\vdash \sigma\circ\sigma\jdeq\idfunc:\Big(\prd{x:A}\prd{y:B}C(x,y)\Big)\to \Big(\prd{x:A}\prd{y:B}C(x,y)\Big).$}
    \end{prooftree}
  \end{small}
\end{exercises}
\index{dependent function type|)}

%%% Local Variables:
%%% mode: latex
%%% TeX-master: "hott-intro"
%%% End:
