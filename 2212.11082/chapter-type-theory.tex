\chapter{Martin-L\"of's Dependent Type Theory}
\label{chap:type-theory}%

Dependent type theory is a formal system to organize all mathematical objects, structure, and knowledge. Dependent type theory is about types, or more generally dependent types, and their elements. There are many ways to think about type theory, types, and its elements. Types can be interpreted as sets, i.e., there is an interpretation of type theory into Zermelo-Fraenkel set theory, but there are some important differences between type theory and set theory, and the interpretation of types as sets has significant limitations. One of the differences is that in type theory, every element comes equipped with its type. We will write $a:A$ for the judgment that $a$ is an element of type $A$. This leads us to a second important difference between type theory and set theory. Set theory is axiomatized in the formal system of first order logic, whereas type theory is its own formal system. Types and their elements are constructed by following the rules of this formal system, and the only way to construct an element is to construct it as an element of a previously constructed type. The expression $a:A$ is therefore not considered to be a proposition, i.e., something which one can assert about an arbitrary element and an arbitrary type, but it is considered to be a judgment, i.e., an assessment that is part of the construction of the element $a:A$.

In type theory there is a much stronger focus on equality of elements than there is in set theory. It is said that a type is not fully understood until (i) one understands how to construct an element of the type and (ii) one understands precisely how to show that two elements of the type are equal. Equality in type theory is governed by the identity type. Unlike in classical set theory, where equality is a decidable proposition of first order logic, the \emph{type} $x=y$ of identifications of two elements $x,y:A$ is itself a type, and therefore it could possess intricate further structure. 

Dependent type theory is built up in several stages. At the first stage we give structural rules, which express the general theory of type dependency. There is no ambient deductive system of first order logic in type theory. Type theory is its own deductive system, and the structural rules are at the heart of this system. The basic operations that are governed by the structural rules are substitution and weakening operations. After the general system of dependent type theory has been set up, we introduce the ways in which we can form types. The most fundamental class of types are dependent function types, or $\Pi$-types. They are used for practically everything. Next, we introduce the type of natural numbers, where we use type-dependency to formulate a type-theoretic version of the induction principle. By the type-theoretic nature of this induction principle, it can be used in two ways: it can be used to construct the many familiar operations on $\N$, such as addition and multiplication, and it can also be used to prove properties about those operations.

The next idea is that we can consider induction principles for many other types as well. This leads to the idea of more general inductive types. In \cref{sec:inductive} we introduce the unit type, the empty type, the booleans, coproducts, dependent pair types, and cartesian products. All of these are examples of inductive types, and their induction principles can be used to construct the basic operations on them, as well as to prove properties about those operations.

Then we come to the most characteristic ingredient of Martin L\"of's dependent type theory: the identity type. The identity type $x=_Ay$ is an example of a \emph{dependent} type, because it is indexed by $x,y:A$, and it is inductively generated by the reflexivity element $\refl{x}:x=_Ax$. The catch is, however, that the identity type $x=_Ay$ is just another type, and it could potentially have many different elements.

The last class of types that we introduce are universes. Universes are type families that are closed under the operations of type theory: $\Pi$-types, $\Sigma$-types, identity types, and so on. Universes play a fundamental role in the theory. One important reason for introducing universes is that they can be used to define type families over inductive types via their induction principles. For example, this allows us to define the ordering relations $\leq$ and $<$ on the natural numbers. We will also use the universes to show the Peano axioms asserting that $\succN$ is injective, and that $\zeroN$ is not a successor.

In the final two sections of this chapter, we start developing mathematics in type theory. In \cref{sec:modular-arithmetic} we study the Curry-Howard interpretation, and use it to develop modular arithmetic in type theory. In \cref{sec:decidability} we study the concept of decidability, and use it to obtain basic theorems in elementary number theory, such as the well-ordering theorem, the construction of the greatest common divisor, and the infinitude of primes. Both of these sections can be viewed as tutorials in type theory, designed to give you some practical experience with type theory before diving into the intricacies of the univalent foundations of mathematics.

\section{Dependent type theory}%
\label{sec:dtt}%
\index{dependent type theory|(}%

Dependent type theory is a system of inference rules that can be combined to make \emph{derivations}. In these derivations, the goal is often to construct an element of a certain type. Such an element can be a function if the type of the constructed element is a function type; a proof of a property if the type of the constructed element is a proposition; but it can also be an identification if the type of the constructed element is an identity type, and so on. In some respect, a type is just a collection of mathematical objects and constructing elements of a type is the everyday mathematical task or challenge. The system of inference rules that we call type theory offers a principled way of engaging in mathematical activity.

\subsection{Judgments and contexts in type theory}%
\index{judgment|(}%
\index{context|(}%

A mathematical argument or construction consists of a sequence of deductive steps, each one using finitely many premises in order to get to the next stage in the proof or construction. Such steps can be represented by \define{inference rules}\index{inference rule|see {rule}}, which are written in the form
\begin{prooftree}
  \AxiomC{$\mathcal{H}_1$\quad $\mathcal{H}_2$ \quad \dots \quad $\mathcal{H}_n$}
  \UnaryInfC{$\mathcal{C}$.}
\end{prooftree}
Inference rules contain above the horizontal line\index{horizontal line|see {inference rule}} a finite list $\mathcal{H}_1$, $\mathcal{H}_2$, \dots, $\mathcal{H}_n$ of \emph{judgments} for the \define{premises}\index{inference rule!premises|textbf}\index{premise of an inference rule|textbf}, and below the horizontal line a single judgment $\mathcal{C}$ for the \define{conclusion}\index{inference rule!conclusion|textbf}\index{conclusion of an inference rule|textbf}. The system of dependent type theory is described by a set of such inference rules.

A straightforward example of an inference rule that we will encounter in \cref{sec:pi} when we introduce function types\index{function type}, is the inference rule
\begin{prooftree}
  \AxiomC{$\Gamma\vdash a:A$}
  \AxiomC{$\Gamma\vdash f:A\to B$}
  \BinaryInfC{$\Gamma\vdash f(a):B$.}
\end{prooftree}
This rule asserts that in any context $\Gamma$ we may use an element $a:A$ and a function $f:A\to B$ to obtain an element $f(a):B$. Each of the expressions
\begin{align*}
  \Gamma & \vdash a :A \\*
  \Gamma & \vdash f : A \to B \\*
  \Gamma & \vdash f(a):B
\end{align*}
are examples of judgments.

\begin{defn}\label{defn:judgments}
  There are four kinds of \define{judgments} in Martin-L\"of's dependent type theory:
  \begin{enumerate}
  \item \emph{$A$ is a (well-formed) \define{type} in context $\Gamma$.}
    \index{well-formed type}\index{type}
    We express this judgment as\index{Gamma turnstile A type@{$\Gamma\vdash A~\type$}}\index{judgment!Gamma turnstile A type@{$\Gamma\vdash A~\type$}}
    \begin{equation*}
      \Gamma\vdash A~\type.
    \end{equation*}
  \item \emph{$A$ and $B$ are \define{judgmentally equal types} in context $\Gamma$.}
    \index{judgmental equality!of types} We express this judgment as\index{Gamma turnstile A is B type@{$\Gamma\vdash A\jdeq B~\type$}}\index{judgment!Gamma turnstile A is B type@{$\Gamma\vdash A\jdeq B~\type$}}
    \begin{equation*}
      \Gamma\vdash A \jdeq B~\type.
    \end{equation*}
  \item \emph{$a$ is an \define{element}\index{element|textbf} of type $A$ in context $\Gamma$.} We express this judgment as\index{Gamma turnstile a in A@{$\Gamma\vdash a:A$}}\index{judgment!Gamma turnstile a in A@{$\Gamma\vdash a:A$}}
    \begin{equation*}
      \Gamma \vdash a:A.
    \end{equation*}
  \item \emph{$a$ and $b$ are \define{judgmentally equal elements} of type $A$ in context $\Gamma$.}\index{judgmental equality!of elements} We express this judgment as\index{Gamma turnstile a is b in A@{$\Gamma\vdash a\jdeq b:A$}}\index{judgment!Gamma turnstile a is b in A@{$\Gamma\vdash a\jdeq b:A$}}
    \begin{equation*}
      \Gamma\vdash a\jdeq b:A.
    \end{equation*}
  \end{enumerate}
\end{defn}

We see that any judgment is of the form $\Gamma\vdash\mathcal{J}$, consisting of a \emph{context} $\Gamma$ and a \emph{judgment thesis} $\mathcal{J}$ asserting either that $A$ is a type, that $A$ and $B$ are equal types, that $a$ is an element of type $A$, or that $a$ and $b$ are equal elements of type $A$. The role of a context is to declare what \define{hypothetical elements}\index{hypothetical elements|textbf} are assumed, along with their types. Hypothetical elements are commonly called \define{variables}\index{variable|textbf}.

\begin{defn}\label{defn:context}
  A \define{context}\index{context|textbf} is a finite list of \define{variable declarations}\index{variable declaration|textbf}
\begin{equation}\label{eq:context}
x_1:A_1,~x_2:A_2(x_1),~\ldots,~x_n:A_n(x_1,\ldots,x_{n-1})
\end{equation}
satisfying the condition that for each $1\leq k\leq n$ we can derive the judgment
\begin{equation*}
  x_1:A_1,~\ldots,~x_{k-1}:A_{k-1}(x_1,\ldots,x_{k-2})\vdash A_k(x_1,\ldots,x_{k-1})~\type,
\end{equation*}
using the inference rules of type theory. We may use variable names other than $x_1,\ldots,x_n$, as long as no variable is declared more than once.
\end{defn}

The condition in \cref{defn:context} that each of the hypothetical elements is assigned a type, is checked recursively. In other words, to check that a list of variable declarations as in \cref{eq:context} is a context, one starts on the left and works their way to the right, verifying that each hypothetical elements $x_k$ is assigned a type. 

Note that there is a context of length $0$, the \define{empty context}\index{context!empty context|textbf}\index{empty context|textbf}, which declares no variables. This context satisfies the requirement in \cref{defn:context} vacuously. A list of variable declarations $x_1:A_1$ of length one is a context if and only if $A_1$ is a type in the empty context. We will soon encounter the type $\N$ of natural numbers\index{natural numbers}, which is an example of a type in the empty context.

The next case is that a list of variable declarations $x_1:A_1,~x_2:A_2(x_1)$ of length two is a context if and only if $A_1$ is a type in the empty context, and $A_2(x_1)$ is a type in context $x_1:A_1$. This process repeats itself for longer contexts.
\index{judgment|)}
\index{context|)}

\subsection{Type families}
\index{type family|(}
It is a feature of \emph{dependent} type theory that all judgments are context dependent, and indeed that even the types of the variables in a context may depend on any previously declared variables. For example, if $n$ is a natural number and we know from the context that $n$ is prime, then we don't have enough information yet to decide whether or not $n$ is odd. However, if we also know from the context that $n+2$ is prime, then we can derive that $n$ must be odd. Context dependency is everywhere -- not only in mathematics, but also in language and in everyday life -- and it gives rise to the notion of \emph{type families} and their \emph{sections}.

\begin{defn}
  Consider a type $A$ in context $\Gamma$. A \define{family}\index{family of types|see{type family}}\index{type family|textbf}\index{family of types|textbf} of types over $A$ in context $\Gamma$ is a type $B(x)$ in context $\Gamma,x:A$. In other words, in the situation where
\begin{equation*}
  \Gamma,~x:A\vdash B(x)~\type,
\end{equation*}
we say that $B$ is a family of types over $A$ in context $\Gamma$. Alternatively, we say that $B(x)$ is a type \define{indexed}\index{indexed type|textbf}\index{type!indexed type|textbf} by $x:A$, in context $\Gamma$.
\end{defn}

We think of a type family $B$ over $A$ in context $\Gamma$ as a type $B(x)$ varying along $x:A$. A basic example of a type family occurs when we introduce \emph{identity types}\index{identity type} in \cref{sec:identity}. They are introduced as follows:
\begin{prooftree}
  \AxiomC{$\Gamma\vdash a:A$}
  \UnaryInfC{$\Gamma,~x:A\vdash a=x~\type$.}
\end{prooftree}
This rule asserts that given an element $a:A$ in context $\Gamma$, we may form the type $a=x$ in context $\Gamma,~x:A$. The type $a=x$ in context $\Gamma,~x:A$ is an example of a type family over $A$ in context $\Gamma$.

\begin{defn}
Consider a type family $B$ over $A$ in context $\Gamma$. A \define{section}\index{section of a type family} of the family $B$ over $A$ in context $\Gamma$ is an element of type $B(x)$ in context $\Gamma,x:A$, i.e., in the judgment
\begin{equation*}
  \Gamma,~x:A\vdash b(x):B(x)
\end{equation*}
we say that $b$ is a section of the family $B$ over $A$ in context $\Gamma$. Alternatively, we say that $b(x)$ is an element of type $B(x)$ \define{indexed}\index{indexed element|textbf}\index{element!indexed element|textbf} by $x:A$ in context $\Gamma$.
\end{defn}

Note that in the above situations $A$, $B$, and $b$ also depend on the variables declared in the context $\Gamma$, even though we have not explicitly mentioned them. It is indeed common practice to not mention every variable in the context $\Gamma$ in such situations.
\index{type family|)}

\subsection{Inference rules}\label{sec:rules}

We are now ready to present the system of inference rules that underlies dependent type theory. These rules are known as the \define{structural rules} of type theory, since they establish the basic mathematical framework for type dependency. There are six sets of inference rules:
\begin{enumerate}
\item Rules about the formation of contexts, types, and their elements
\item Rules postulating that judgmental equality is an equivalence relation.
\item Variable conversion rules.
\item Substitution rules.
\item Weakening rules.
\item The generic element.
\end{enumerate}

\subsubsection*{Rules about the formation of contexts, types, and their elements}
In the definition of well-formed contexts, types, and elements we specified that for a type $B(x)$ to be well-formed in context $\Gamma,x:A$, it must be the case that $A$ is a well-formed type in context $\Gamma$. The following rules follow from the presuppositions about contexts, types, and their elements, and may be used freely in derivations:

\begin{center}
  \begin{minipage}{.35\textwidth}
    \begin{prooftree}
      \AxiomC{$\Gamma,x:A\vdash B(x)~\type$}
      \UnaryInfC{$\Gamma\vdash A~\type$}
    \end{prooftree}
    
    \begin{prooftree}
      \AxiomC{$\Gamma\vdash a:A$}
      \UnaryInfC{$\Gamma\vdash A~\type$}
    \end{prooftree}
  \end{minipage}
  \begin{minipage}{.25\textwidth}
    \begin{prooftree}
      \AxiomC{$\Gamma\vdash A\jdeq B~\type$}
      \UnaryInfC{$\Gamma\vdash A~\type$}
    \end{prooftree}
    
    \begin{prooftree}
      \AxiomC{$\Gamma\vdash a\jdeq b:A$}
      \UnaryInfC{$\Gamma\vdash a:A$}
    \end{prooftree}
  \end{minipage}
  \begin{minipage}{.25\textwidth}
    \begin{prooftree}
      \AxiomC{$\Gamma\vdash A\jdeq B~\type$}
      \UnaryInfC{$\Gamma\vdash B~\type$}
    \end{prooftree}
    
    \begin{prooftree}
      \AxiomC{$\Gamma\vdash a\jdeq b:A$}
      \UnaryInfC{$\Gamma\vdash b:A$}
    \end{prooftree}
  \end{minipage}
\end{center}

\subsubsection*{Judgmental equality is an equivalence relation}

\index{rules!for type dependency!judgmental equality is an equivalence relation|(}
The rules postulating that judgmental equality on types and on elements is an equivalence relation simply postulate that these relations are reflexive, symmetric, and transitive\index{judgmental equality!is an equivalence relation}:
\begin{center}
  \begin{small}
    \begin{minipage}{.22\textwidth}
      \begin{center}
        \begin{prooftree}
          \AxiomC{$\Gamma\vdash A~\textrm{type}$}
          \UnaryInfC{$\Gamma\vdash A\jdeq A~\textrm{type}$}
        \end{prooftree}
      \end{center}
    \end{minipage}
    \begin{minipage}{.28\textwidth}
      \begin{center}
        \begin{prooftree}
          \AxiomC{$\Gamma\vdash A\jdeq B~\textrm{type}$}
          \UnaryInfC{$\Gamma\vdash B\jdeq A~\textrm{type}$}
        \end{prooftree}
      \end{center}
    \end{minipage}
    \begin{minipage}{.48\textwidth}
      \begin{prooftree}
        \AxiomC{$\Gamma\vdash A\jdeq B~\textrm{type}$}
        \AxiomC{$\Gamma\vdash B\jdeq C~\textrm{type}$}
        \BinaryInfC{$\Gamma\vdash A\jdeq C~\textrm{type}$}
      \end{prooftree}
    \end{minipage}
    \\
    \bigskip
    \begin{minipage}{.22\textwidth}
      \begin{prooftree}
        \AxiomC{$\Gamma\vdash a:A$}
        \UnaryInfC{$\Gamma\vdash a\jdeq a : A$}
      \end{prooftree}
    \end{minipage}
    \begin{minipage}{.28\textwidth}
      \begin{prooftree}
        \AxiomC{$\Gamma\vdash a\jdeq b:A$}
        \UnaryInfC{$\Gamma\vdash b\jdeq a: A$}
      \end{prooftree}
    \end{minipage}
    \begin{minipage}{.48\textwidth}
      \begin{prooftree}
        \AxiomC{$\Gamma\vdash a\jdeq b : A$}
        \AxiomC{$\Gamma\vdash b\jdeq c: A$}
        \BinaryInfC{$\Gamma\vdash a\jdeq c: A$.}
      \end{prooftree}
    \end{minipage}
  \end{small}
\end{center}
\index{rules!for type dependency!judgmental equality is an equivalence relation|)}

\subsubsection*{Variable conversion rules}
\index{rules!for type dependency!variable conversion|(}
The \define{variable conversion rules}\index{judgmental equality!conversion rules}\index{variable conversion rules|textbf}\index{conversion rule!variable|textbf}\index{rules!for type dependency!variable conversion|textbf} are rules postulating that we can convert the type of a variable to a judgmentally equal type. The first variable conversion rule states that
\begin{prooftree}
\AxiomC{$\Gamma\vdash A\jdeq A'~\textrm{type}$}
\AxiomC{$\Gamma,~x:A,~\Delta\vdash B(x)~\type$}
\BinaryInfC{$\Gamma,~x:A',~\Delta\vdash B(x)~\type$.}
\end{prooftree}
In this conversion rule, the context $\Gamma,~x:A,~\Delta$ is just any extension of the context $\Gamma,~x:A$, i.e., it is a context of the form
\begin{equation*}
  x_1:A_1,~\ldots,~x_{n-1}:A_{n-1},~x:A,~x_{n+1}:A_{n+1},~\ldots,~x_{n+m}:A_{n+m}.
\end{equation*}

Similarly, there are variable conversion rules for judgmental equality of types, for elements, and for judgmental equality of elements. To avoid having to state essentially the same rule four times, we state all four variable conversion rules at once using a \emph{generic judgment thesis} $\mathcal{J}$, which can be any of the four kinds described in \cref{defn:judgments}:
\begin{prooftree}
\AxiomC{$\Gamma\vdash A\jdeq A'~\textrm{type}$}
\AxiomC{$\Gamma,~x:A,~\Delta\vdash \mathcal{J}$}
\BinaryInfC{$\Gamma,~x:A',~\Delta\vdash \mathcal{J}$.}
\end{prooftree}
An analogous \emph{element conversion rule}, stated in \cref{ex:term_conversion}, converting the type of an element to a judgmentally equal type, is derivable using the rules from the rules presented in this section.
\index{rules!for type dependency!variable conversion|)}

\subsubsection*{Substitution}
\index{substitution|(}\index{rules!for type dependency!rules for substitution|(}

Consider an element $f(x):B(x)$ indexed by $x:A$ in context $\Gamma$, and suppose we also have an element $a:A$. Then we can simultaneously substitute $a$ for all occurrences of $x$ in $f(x)$ to obtain a new element $f[a/x]$, which has type $B[a/x]$. A precise definition of substitution requires us to get too deep into the theory of the syntax of type theory, but a mathematician is of course no stranger to substitution. For example, substituting $0$ for $x$ in the polynomial
\begin{equation*}
  1+x+x^2+x^3
\end{equation*}
results in the number $1+0+0^2+0^3$, which can be computed to the value $1$.

Type theoretic substitution is similar. Type theoretic substitution is in fact a bit more general than what we have described above. Suppose we have a type
\begin{equation*}
  \Gamma,~x:A,~y_{1}:B_{1},~\ldots,~y_{n}:B_{n}\vdash C~\textrm{type}
\end{equation*}
and an element $a:A$ in context $\Gamma$. Then we can simultaneously substitute $a$ for all occurrences of $x$ in the types $B_1,\ldots,B_n$ and $C$, to obtain
\begin{equation*}
  \Gamma,~y_{1}:B_{1}[a/x],~\ldots,~y_{n}:B_{n}[a/x]\vdash C[a/x]~\mathrm{type}.
\end{equation*}
Note that the variables $y_{1},~\ldots,y_{n}$ are assigned new types after performing the substitution of $a$ for $x$. Similarly, we can substitute $a$ for $x$ in an element $c:C$ to obtain the element $c[a/x]:C[a/x]$, and we can substitute $a$ for $x$ in a judgmental equality thesis, either of types or elements, by simply substituting on both sides of the equation. The \define{substitution rule} are therefore stated using a generic judgment $\mathcal{J}$:
\begin{prooftree}
\AxiomC{$\Gamma\vdash a:A$}
\AxiomC{$\Gamma,~x:A,~\Delta\vdash \mathcal{J}$}
\RightLabel{$S$.}
\BinaryInfC{$\Gamma,~\Delta[a/x]\vdash \mathcal{J}[a/x]$}
\end{prooftree}
Furthermore, we add two more `congruence rules' for substitution, postulating that substitution by judgmentally equal elements results in judgmentally equal types and elements:
\begin{prooftree}
\AxiomC{$\Gamma\vdash a\jdeq a':A$}
\AxiomC{$\Gamma,~x:A,~\Delta\vdash B~\type$}
\BinaryInfC{$\Gamma,~\Delta[a/x]\vdash B[a/x]\jdeq B[a'/x]~\type$}
\end{prooftree}
\begin{prooftree}
\AxiomC{$\Gamma\vdash a\jdeq a':A$}
\AxiomC{$\Gamma,~x:A,~\Delta\vdash b:B$}
\BinaryInfC{$\Gamma,~\Delta[a/x]\vdash b[a/x]\jdeq b[a'/x]:B[a/x]$.}
\end{prooftree}
To see that these rules make sense, we observe that both $B[a/x]$ and $B[a'/x]$ are types in context $\Delta[a/x]$, provided that $a\jdeq a'$. This is immediate by recursion on the length of $\Delta$.

\begin{defn}
  When $B$ is a family of types over $A$ in context $\Gamma$, and if we have $a:A$, then we also say that $B[a/x]$ is the \define{fiber}\index{type family!fiber of a type family|textbf}\index{fiber of a type family|textbf} of $B$ at $a$. We will usually write $B(a)$ for the fiber of $B$ at $a$.

  When $b$ is a section of the family $B$ over $A$ in context $\Gamma$, we call the element $b[a/x]$ the \define{value} of $b$ at $a$. Again, we will usually write $b(a)$ for the value of $b$ at $a$.
\end{defn}
\index{substitution|)}\index{rules!for type dependency!rules for substitution|)}

\subsubsection*{Weakening}
\index{weakening|(}\index{rules!for type dependency!rules for weakening|(}
If we are given a type $A$ in context $\Gamma$, then any judgment made in a longer context $\Gamma,~\Delta$ can also be made in the context $\Gamma,~x:A,~\Delta$, for a fresh variable $x$. The \define{weakening rule}\index{weakening} asserts that weakening by a type $A$ in context preserves well-formedness and judgmental equality of types and elements.
\begin{prooftree}
\AxiomC{$\Gamma\vdash A~\textrm{type}$}
\AxiomC{$\Gamma,~\Delta\vdash \mathcal{J}$}
\RightLabel{$W$.}
\BinaryInfC{$\Gamma,~x:A,~\Delta \vdash \mathcal{J}$}
\end{prooftree}
This process of expanding the context by a fresh variable of type $A$ is called \define{weakening} (by $A$).

In the simplest situation where weakening applies, we have two types $A$ and $B$ in context $\Gamma$. Then we can weaken $B$ by $A$ as follows
\begin{prooftree}
  \AxiomC{$\Gamma\vdash A~\textrm{type}$}
  \AxiomC{$\Gamma\vdash B~\textrm{type}$}
  \RightLabel{$W$}
  \BinaryInfC{$\Gamma,~x:A\vdash B~\type$}
\end{prooftree}
in order to form the type $B$ in context $\Gamma,~x:A$. The type $B$ in context $\Gamma,~x:A$ is called the \define{constant family}\index{type family!constant family|textbf}\index{constant family|textbf} $B$, or the \define{trivial family}\index{type family!trivial family|textbf}\index{trivial family|textbf} $B$.
\index{weakening|)}\index{rules!for type dependency!rules for weakening|)}

\subsubsection*{The generic elements}
If we are given a type $A$ in context $\Gamma$, then we can weaken $A$ by itself to obtain that $A$ is a type in context $\Gamma,~x:A$. The rule for the \define{generic element}\index{generic element|textbf}\index{rules!for type dependency!generic element|textbf} now asserts that any hypothetical element $x:A$ in the context $\Gamma,~x:A$ is also an element of type $A$ in context $\Gamma,~x:A$.
\begin{prooftree}
\AxiomC{$\Gamma\vdash A~\textrm{type}$}
\RightLabel{$\delta$.}
\UnaryInfC{$\Gamma,~x:A\vdash x:A$}
\end{prooftree}
This rule is also known as the \define{variable rule}\index{variable rule|textbf}\index{rules!for type dependency!variable rule|textbf}. One of the reasons for including the generic element is to make sure that the variables declared in a context---i.e., the hypothetical elements---are indeed \emph{elements}. It also provides the \emph{identity function}\index{identity function} on the type $A$ in context $\Gamma$.

\subsection{Derivations}\label{sec:derivations}

\index{derivation|(}
A \define{derivation}\index{derivation|textbf} in type theory is a finite tree in which each node is a valid rule of inference. At the root of the tree we find the conclusion, and in the leaves of the tree we find the hypotheses. We give two examples of derivations: a derivation showing that any variable can be changed to a fresh one, and a derivation showing that any two variables that do not mutually depend on one another can be swapped in order.

Given a derivation with hypotheses $\mathcal{H}_1,\ldots,\mathcal{H}_n$ and conclusion $\mathcal{C}$, we can form a new inference rule
\begin{prooftree}
  \AxiomC{$\mathcal{H}_1$}
  \AxiomC{$\cdots$}
  \AxiomC{$\mathcal{H}_n$}
  \TrinaryInfC{$\mathcal{C}$.}
\end{prooftree}
Such a rule is called \define{derivable}, because we have a derivation for it. In order to keep proof trees reasonably short and manageable, we use the convention that any derived rules can be used in future derivations.

\subsubsection*{Changing variables}

\index{change of variables}
Variables can always be changed to fresh variables. We show that this is the case by showing that the inference rule\index{rules!for type dependency!change of variables}
\begin{prooftree}
  \AxiomC{$\Gamma,~x:A,~\Delta\vdash \mathcal{J}$}
  \RightLabel{$x'/x$}
  \UnaryInfC{$\Gamma,~x':A,~\Delta[x'/x]\vdash \mathcal{J}[x'/x]$}
\end{prooftree}
is derivable, where $x'$ is a variable that does not occur in the context $\Gamma,~x:A,~\Delta$. 

Indeed, we have the following derivation using substitution, weakening, and the generic element:
\begin{prooftree}
  \AxiomC{$\Gamma\vdash A~\type$}
  \RightLabel{$\delta$}
  \UnaryInfC{$\Gamma,~x':A\vdash x':A$}
  \AxiomC{$\Gamma\vdash A~\type$}
  \AxiomC{$\Gamma,~x:A,~\Delta\vdash \mathcal{J}$}
  \RightLabel{$W$}
  \BinaryInfC{$\Gamma,~x':A,~x:A,~\Delta\vdash \mathcal{J}$}
  \RightLabel{$S$.}
  \BinaryInfC{$\Gamma,~x':A,~\Delta[x'/x]\vdash \mathcal{J}[x'/x]$}
\end{prooftree}
In this derivation it is the application of the weakening rule where we have to check that $x'$ does not occur in the context $\Gamma,~x:A,~\Delta$.

\subsubsection*{Interchanging variables}

The \define{interchange rule}\index{rules!for type dependency!interchange}\index{interchange rule} states that if we have two types $A$ and $B$ in context $\Gamma$, and we make a judgment in context $\Gamma,~x:A,~y:B,~\Delta$, then we can make that same judgment in context $\Gamma,~y:B,~x:A,~\Delta$ where the order of $x:A$ and $y:B$ is swapped. More formally, the interchange rule is the following inference rule
\begin{prooftree}
\AxiomC{$\Gamma\vdash B~\textrm{type}$}
\AxiomC{$\Gamma,~x:A,~y:B,~\Delta\vdash \mathcal{J}$}
\BinaryInfC{$\Gamma,~y:B,~x:A,~\Delta\vdash \mathcal{J}$.}
\end{prooftree}
Just as the rule for changing variables, we claim that the interchange rule is a derivable rule.

The idea of the derivation for the interchange rule is as follows: If we have a judgment
\begin{equation*}
  \Gamma,~x:A,~y:B,~\Delta\vdash\mathcal{J},
\end{equation*}
then we can change the variable $y$ to a fresh variable $y'$ and weaken the judgment to obtain the judgment
\begin{equation*}
  \Gamma,~y:B,~x:A,~y':B,~\Delta[y'/y]\vdash\mathcal{J}[y'/y].
\end{equation*}
Now we can substitute $y$ for $y'$ to obtain the desired judgment $\Gamma,~y:B,~x:A,~\Delta\vdash\mathcal{J}$. The formal derivation is as follows:
\begin{small}
  \begin{prooftree}
    \AxiomC{$\Gamma\vdash B~\textrm{type}$}
    %\RightLabel{$\delta$}
    \UnaryInfC{$\Gamma,~y:B\vdash y:B$}
    %\RightLabel{$W$} 
    \UnaryInfC{$\Gamma,~y:B,~x:A\vdash y:B$}
    \AxiomC{$\Gamma\vdash B~\textrm{type}$}
    \AxiomC{$\Gamma,~x:A,~y:B,~\Delta\vdash \mathcal{J}$}
    %\RightLabel{$y'/y$}
    \UnaryInfC{$\Gamma,~x:A,~y':B,~\Delta[y'/y]\vdash \mathcal{J}[y'/y]$}
    %\RightLabel{$W$}
    \BinaryInfC{$\Gamma,~y:B,~x:A,~y':B,~\Delta[y'/y]\vdash \mathcal{J}[y'/y]$}
    %\RightLabel{$S$.}
    \BinaryInfC{$\Gamma,~y:B,~x:A,~\Delta\vdash \mathcal{J}$}
  \end{prooftree}%
\end{small}
\index{derivation|)}

\begin{exercises}
  \exitem \label{ex:term_conversion}
  \begin{subexenum}
  \item Give a derivation for the following \define{element conversion rule}\index{element conversion rule|textbf}\index{rules!for type dependency!element conversion|textbf}\index{conversion rule!element|textbf}:
    \begin{prooftree}
      \AxiomC{$\Gamma\vdash A\jdeq A'~\textrm{type}$}
      \AxiomC{$\Gamma\vdash a:A$}
      \BinaryInfC{$\Gamma\vdash a:A'$.}
    \end{prooftree}
  \item Give a derivation for the following \define{congruence rule} for element conversion:
    \begin{prooftree}
      \AxiomC{$\Gamma\vdash A\jdeq A'~\textrm{type}$}
      \AxiomC{$\Gamma\vdash a\jdeq b:A$}
      \BinaryInfC{$\Gamma\vdash a\jdeq b:A'$.}
    \end{prooftree}
  \end{subexenum}
\end{exercises}
\index{dependent type theory|)}

%%% Local Variables:
%%% mode: latex
%%% TeX-master: "hott-intro"
%%% End:

\section{Dependent function types}
\label{sec:pi}

\index{Pi-type@{$\Pi$-type}|see {dependent function type}}
\index{dependent function type|(}
A fundamental concept of dependent type theory is that of a dependent function. A dependent function is a function of which the type of the output may depend on the input. For example, when we concatenate a vector of length $m$ with a vector of length $n$, we obtain a vector of length $m+n$. Dependent functions are a generalization of ordinary functions, because an ordinary function $f:A\to B$ is a function of which the output $f(x)$ has type $B$ regardless of the value of $x$.

\subsection{The rules for dependent function types}
Consider a section $b$ of a family $B$ over $A$ in context $\Gamma$, i.e., consider
\begin{equation*}
  \Gamma,x:A\vdash b(x):B(x).
\end{equation*}
From one point of view, such a section $b$ is an operation or assignment $x\mapsto b(x)$, or a program\index{program}, that takes as input $x:A$ and produces a term $b(x):B(x)$. From a more mathematical point of view we see $b$ as a choice of an element of each $B(x)$. In other words, we may see $b$ as a function that takes $x:A$ to $b(x):B(x)$. Note that the type $B(x)$ of the output may depend on $x:A$. The assignment $x\mapsto b(x)$ is in this sense a \emph{dependent} function. The type of all such dependent functions is called the \define{dependent function type}, and we will write
\begin{equation*}
  \prd{x:A}B(x)
\end{equation*}
for the type of dependent functions. There are four principal rules for $\Pi$-types:
\begin{enumerate}
\item The \emph{formation rule}, which tells us how we may form dependent function types.
\item The \emph{introduction rule}, which tells us how to introduce new terms of dependent function types.
\item The \emph{elimination rule}, which tells us how to use arbitrary terms of dependent function types.
\item The \emph{computation rules}, which tell us how the introduction and elimination rules interact. These computation rules guarantee that every term of a dependent function type is indeed a dependent function taking the values by which it is defined.
\end{enumerate}
In the cases of the formation rule, the introduction rule, and the elimination rule, we also need rules that assert that all the constructions respect judgmental equality. Those rules are called \define{congruence rules}, and they are part of the specification of dependent function types.

\subsubsection{The $\Pi$-formation rule}
\index{dependent function type!formation rule|textbf}
The \define{$\Pi$-formation rule} tells us how $\Pi$-types are constructed. The idea of $\Pi$-types is that $\prd{x:A}B(x)$ is a type of \define{dependent functions}\index{dependent function type|textbf}, for any type family $B$ of types over $A$, so the $\Pi$-formation rule is as follows:\index{rules!for dependent function types!formation|textbf}\index{P (x:A) B(x)@{$\prd{x:A}B(x)$}|see{dependent function type}}\index{P (x:A) B(x)@{$\prd{x:A}B(x)$}|textbf}
\begin{prooftree}
\AxiomC{$\Gamma,x:A\vdash B(x)~\type$}
\RightLabel{$\Pi$.}
\UnaryInfC{$\Gamma\vdash \prd{x:A}B(x)~\type$}
\end{prooftree}
This rule simply states that in order to form the type $\prd{x:A}B(x)$ in context $\Gamma$, we must have a type family $B$ over $A$ in context $\Gamma$.

We also require that the operation of forming dependent function types respects judgmental equality. This is postulated in the \define{congruence rule} for $\Pi$-types:
\index{rules!for dependent function types!congruence|textbf}
\index{dependent function type!congruence rule|textbf}
\begin{prooftree}
\AxiomC{$\Gamma\vdash A\jdeq A'~\type$}
\AxiomC{$\Gamma,x:A\vdash B(x)\jdeq B'(x)~\textrm{type}$}
\RightLabel{$\Pi$-eq.}
\BinaryInfC{$\Gamma\vdash \prd{x:A}B(x)\jdeq\prd{x:A'}B'(x)~\type$}
\end{prooftree}

\subsubsection{The $\Pi$-introduction rule}
The introduction rule for dependent functions tells us how we may construct dependent functions of type $\prd{x:A}B(x)$. The idea is that a dependent function $f:\prd{x:A}B(x)$ is an operation that takes an $x:A$ to $f(x):B(x)$. Hence the introduction rule of dependent functions postulates that, in order to construct a dependent function one has to construct a term $b(x):B(x)$ indexed by $x:A$ in context $\Gamma$, i.e.:
\begin{prooftree}
  \AxiomC{$\Gamma,x:A \vdash b(x) : B(x)$}
  \RightLabel{$\lambda$.}
  \UnaryInfC{$\Gamma\vdash \lam{x}b(x) : \prd{x:A}B(x)$}
\end{prooftree}
This introduction rule%
\index{dependent function type!introduction rule|see {$\lambda$-abstraction}}
for dependent functions is also called the \define{$\lambda$-abstraction rule}%
\index{lambda-abstraction@{$\lambda$-abstraction}|textbf}%
\index{rules!for dependent function types!lambda-abstraction@{$\lambda$-abstraction}|textbf}%
\index{dependent function type!lambda-abstraction@{$\lambda$-abstraction}|textbf}, and we also say that the $\lambda$-abstraction $\lam{x}b(x)$ \define{binds} the variable $x$ in $b$. Just like ordinary mathematicians, we will sometimes write $x\mapsto b(x)$ for a function $\lam{x}b(x)$. The map $n\mapsto n^2$ is an example.

We will also require that $\lambda$-abstraction respects judgmental equality. Therefore we postulate the \define{congruence rule} for $\lambda$-abstraction,
\index{rules!for dependent function types!lambda-congruence@{$\lambda$-congruence}}
\index{lambda-congruence@{$\lambda$-congruence}}
\index{dependent function type!lambda-congruence@{$\lambda$-congruence}}
which asserts that\label{page:lambda-eq}
\begin{prooftree}
  \AxiomC{$\Gamma,x:A \vdash b(x)\jdeq b'(x) : B(x)$}
  \RightLabel{$\lambda$-eq.}
  \UnaryInfC{$\Gamma\vdash \lam{x}b(x)\jdeq \lam{x}b'(x) : \prd{x:A}B(x)$}
\end{prooftree}

\subsubsection{The $\Pi$-elimination rule}

\index{dependent function type!elimination rule|see {evaluation}}
The elimination rule for dependent function types provides us with a way to \emph{use} dependent functions. The way to use a dependent function is to evaluate it at an argument of the domain type. The $\Pi$-elimination rule is therefore also called the \define{evaluation rule}\index{evaluation|textbf}\index{rules!for dependent function types!evaluation|textbf}\index{dependent function type!evaluation|textbf}:
\begin{prooftree}
\AxiomC{$\Gamma\vdash f:\prd{x:A}B(x)$}
\RightLabel{$ev$.}
\UnaryInfC{$\Gamma,x:A\vdash f(x) : B(x)$}
\end{prooftree}
This rule asserts that given a dependent function $f:\prd{x:A}B(x)$ in context $\Gamma$ we obtain a term $f(x)$ of type $B(x)$ indexed by $x:A$ in context $\Gamma$. Again we require that evaluation respects judgmental equality:
\begin{prooftree}
  \AxiomC{$\Gamma\vdash f\jdeq f':\prd{x:A}B(x)$}
  \RightLabel{$ev$-eq.}
  \UnaryInfC{$\Gamma,x:A\vdash f(x)\jdeq f'(x):B(x)$}
\end{prooftree}

\subsubsection{The $\Pi$-computation rules}

\index{dependent function type!computation rules|see {$\beta$- and $\eta$-rules}}
We now postulate rules that specify the behavior of functions. First, we have a rule that asserts that a function of the form $\lam{x}b(x)$ behaves as expected: when we evaluate it at $x:A$, then we obtain the value $b(x):B(x)$. This rule is called the \define{$\beta$-rule}\index{b-rule for P-types@{$\beta$-rule for $\Pi$-types}|textbf}\index{rules!for dependent function types!b-rule@{$\beta$-rule}|textbf}\index{dependent function type!b-rule@{$\beta$-rule}|textbf}
\begin{prooftree}
\AxiomC{$\Gamma,x:A \vdash b(x) : B(x)$}
\RightLabel{$\beta$.}
\UnaryInfC{$\Gamma,x:A \vdash (\lambda y.b(y))(x)\jdeq b(x) : B(x)$}
\end{prooftree}
Second, we postulate a rule that asserts that all elements of a $\Pi$-type are (dependent) functions. This rule is known as the \define{$\eta$-rule}\index{eta-rule@{$\eta$-rule} for Pi-types@{for $\Pi$-types}|textbf}\index{rules!for dependent function types!eta-rule@{$\eta$-rule}|textbf}\index{dependent function type!eta-rule@{$\eta$-rule}|textbf}
\begin{prooftree}
\AxiomC{$\Gamma\vdash f:\prd{x:A}B(x)$}
\RightLabel{$\eta$.}
\UnaryInfC{$\Gamma \vdash \lam{x}f(x) \jdeq f : \prd{x:A}B(x)$}
\end{prooftree}
In other words, the computation rules ($\beta$ and $\eta$) for dependent function types postulate that $\lambda$-abstraction rule and the evaluation rule are mutual inverses. This completes the specification of dependent function types.

\subsection{Ordinary function types}

An important special case of $\Pi$-types arises when both $A$ and $B$ are types in context $\Gamma$. In this case, we can first weaken $B$ by $A$ and then apply the $\Pi$-formation rule to obtain the type $A\to B$ of \emph{ordinary} functions from $A$ to $B$, as in the following derivation:
\begin{prooftree}
\AxiomC{$\Gamma\vdash A~\textrm{type}$}
\AxiomC{$\Gamma\vdash B~\textrm{type}$}
\RightLabel{$W$}
\BinaryInfC{$\Gamma,x:A\vdash B~\textrm{type}$}
\RightLabel{$\Pi$}
\UnaryInfC{$\Gamma\vdash \prd{x:A}B~\textrm{type}$.}
\end{prooftree}
A term $f:\prd{x:A}B$ is a function that takes an argument $x:A$ and returns $f(x):B$. In other words, terms of type $\prd{x:A}B$ are indeed ordinary functions from $A$ to $B$. Therefore, we define the type $A\to B$\index{A arrow B@{$A\to B$}|see {function type}}\index{A arrow B@{$A\to B$}|textbf} of \define{(ordinary) functions}\index{function type|textbf} from $A$ to $B$ by
\begin{equation*}
  A\to B\defeq\prd{x:A}B.
\end{equation*}
If $f:A\to B$ is a function, then the type $A$ is also called the \define{domain} of $f$, and the type $B$ is also called the \define{codomain} of $f$.

Sometimes we will also write $B^A$\index{B^A@{$B^A$}|see {function type}} for the type $A\to B$.  Formally, we make such definitions by adding one more line to the above derivation:
\begin{prooftree}
\AxiomC{$\Gamma\vdash A~\textrm{type}$}
\AxiomC{$\Gamma\vdash B~\textrm{type}$}
\RightLabel{$W$}
\BinaryInfC{$\Gamma,x:A\vdash B~\textrm{type}$}
\RightLabel{$\Pi$}
\UnaryInfC{$\Gamma\vdash \prd{x:A}B~\textrm{type}$}
\UnaryInfC{$\Gamma\vdash A\to B \defeq \prd{x:A}B~\textrm{type}$.}
\end{prooftree}

\begin{rmk}
  More generally, we can make definitions at the end of a derivation if the conclusion is a certain type in context, or if the conclusion is a certain term of a type in context. Suppose, for instance, that we have a derivation
  \begin{prooftree}
    \AxiomC{$\mathcal{D}$}
    \UnaryInfC{$\Gamma\vdash a:A$,}
  \end{prooftree}
  in which the derivation $\mathcal{D}$ makes use of the premises $\mathcal{H}_1$, \ldots,$\mathcal{H}_n$. If we wish to make a definition $\newdef\defeq a$, then we can extend the derivation tree with
  \begin{prooftree}
    \AxiomC{$\mathcal{D}$}
    \UnaryInfC{$\Gamma\vdash a:A$}
    \UnaryInfC{$\Gamma\vdash\newdef\defeq a:A$.}
  \end{prooftree}
  The effect of such a definition is that we have extended our type theory with a new constant $\newdef$, for which the following inference rules are valid
  \begin{center}
    \begin{minipage}{.45\textwidth}
      \begin{prooftree}
        \AxiomC{$\mathcal{H}_1$\quad $\mathcal{H}_2$ \quad \dots \quad $\mathcal{H}_n$}
        \UnaryInfC{$\Gamma\vdash\newdef:A$}
      \end{prooftree}
    \end{minipage}
    \begin{minipage}{.45\textwidth}
      \begin{prooftree}
        \AxiomC{$\mathcal{H}_1$\quad $\mathcal{H}_2$ \quad \dots \quad $\mathcal{H}_n$}
        \UnaryInfC{$\Gamma\vdash\newdef\jdeq a:A$.}
      \end{prooftree}
    \end{minipage}
  \end{center}
  In our example of the definition of the ordinary function type $A\to B$, we therefore have by definition the following valid inference rules
  \begin{center}
    \begin{minipage}{.45\textwidth}
      \begin{prooftree}
        \AxiomC{$\Gamma\vdash A~\textrm{type}$}
        \AxiomC{$\Gamma\vdash B~\textrm{type}$}
        \BinaryInfC{$\Gamma\vdash A\to B~\textrm{type}$}
      \end{prooftree}
    \end{minipage}
    \begin{minipage}{.45\textwidth}
      \begin{prooftree}
        \AxiomC{$\Gamma\vdash A~\textrm{type}$}
        \AxiomC{$\Gamma\vdash B~\textrm{type}$}
        \BinaryInfC{$\Gamma\vdash A\to B\jdeq \prd{x:A}B~\textrm{type}$.}
      \end{prooftree}
    \end{minipage}
  \end{center}
  There are of course many such definitions throughout the development of dependent type theory, the univalent foundations of mathematics, and synthetic homotopy theory. They are all included in the index at the end of this book.
\end{rmk}

\begin{rmk}
  By the term conversion rules of \cref{ex:term_conversion} we can now use the rules for $\lambda$-abstraction, evaluation, and so on, to obtain corresponding rules for the ordinary function type $A\to B$. We give a brief summary of these rules, omitting the congruence rules.\index{rules!for function types}
  \begin{prooftree}
    \AxiomC{$\Gamma\vdash A~\textrm{type}$}
    \AxiomC{$\Gamma\vdash B~\textrm{type}$}
    \RightLabel{$\to$}
    \BinaryInfC{$\Gamma\vdash A\to B~\textrm{type}$}
  \end{prooftree}%
  \begin{center}
    \begin{minipage}{.55\textwidth}
      \begin{prooftree}
        \AxiomC{$\Gamma\vdash B~\textrm{type}$}
        \AxiomC{$\Gamma,x:A\vdash b(x):B$}
        \RightLabel{$\lambda$}
        \BinaryInfC{$\Gamma\vdash \lam{x}b(x):A\to B$}
      \end{prooftree}%
    \end{minipage}
    \begin{minipage}{.35\textwidth}
      \begin{prooftree}
        \AxiomC{$\Gamma\vdash f:A\to B$}
        \RightLabel{$ev$}
        \UnaryInfC{$\Gamma,x:A\vdash f(x):B$}
      \end{prooftree}%
    \end{minipage}
  \end{center}
  \begin{center}
    \begin{minipage}{.55\textwidth}
      \begin{prooftree}
        \AxiomC{$\Gamma\vdash B~\textrm{type}$}
        \AxiomC{$\Gamma,x:A\vdash b(x):B$}
        \RightLabel{$\beta$}
        \BinaryInfC{$\Gamma,x:A\vdash(\lam{y}b(y))(x)\jdeq b(x):B$}
      \end{prooftree}%
    \end{minipage}
    \begin{minipage}{.40\textwidth}
      \begin{prooftree}
        \AxiomC{$\Gamma\vdash f:A\to B$}
        \RightLabel{$\eta$}
        \UnaryInfC{$\Gamma\vdash\lam{x} f(x)\jdeq f:A\to B$}
      \end{prooftree}
    \end{minipage}
  \end{center}
\end{rmk}

Now we can use these rules to construct some familiar functions, such as the identity function $\idfunc:A\to A$ on an arbitrary type $A$, and the composition $g\circ f:A\to C$ of any two functions $f:A\to B$ and $g:B\to C$. 

\begin{defn}
For any type $A$ in context $\Gamma$, we define the \define{identity function}\index{identity function|textbf}\index{function!identity function|textbf} $\idfunc[A]:A\to A$\index{id A@{$\idfunc[A]$}|see {identity function}}\index{id A@{$\idfunc[A]$}|textbf} using the generic term:
\begin{prooftree}
\AxiomC{$\Gamma\vdash A~\textrm{type}$}
\UnaryInfC{$\Gamma,x:A\vdash x:A$}
\UnaryInfC{$\Gamma\vdash \lam{x}x:A\to A$}
\UnaryInfC{$\Gamma\vdash \idfunc[A]\defeq\lam{x}x:A\to A$.}
\end{prooftree}
\end{defn}

The identity function therefore satisfies the following inference rules:
  \begin{center}
    \begin{minipage}{.45\textwidth}
      \begin{prooftree}
        \AxiomC{$\Gamma\vdash A~\textrm{type}$}
        \UnaryInfC{$\Gamma\vdash \idfunc[A]:A\to A$}
      \end{prooftree}
    \end{minipage}
    \begin{minipage}{.45\textwidth}
      \begin{prooftree}
        \AxiomC{$\Gamma\vdash A~\textrm{type}$}
        \UnaryInfC{$\Gamma\vdash \idfunc[A]\jdeq\lam{x}x:A\to A$.}
      \end{prooftree}
    \end{minipage}
  \end{center}

Next, we define the composition of functions. We will introduce the composition operation itself as a function $\comp$ that takes two arguments: the first argument is a function $g:B\to C$, and the second argument is a function $f:A\to B$. The output is a function $\comp(g,f):A\to C$, for which we often write $g\circ f$.

\begin{rmk}
  Since composition is a function that takes multiple arguments, we need to know how to represent such functions. Types of functions with multiple arguments can be formed by iterating the $\Pi$-formation rule or the $\to$-formation rule. For example, a function
  \begin{equation*}
    f:A\to (B\to C)
  \end{equation*}
  takes two arguments: first it takes an argument $x:A$, and the output $f(x)$ has type $B\to C$. This is again a function type, so $f(x)$ is a function that takes an argument $y:B$, and its output $f(x)(y)$ has type $C$. We will usually write $f(x,y)$ for $f(x)(y)$.

  Similarly, when $C(x,y)$ is a family of types indexed by $x:A$ and $y:B(x)$, then we can form the dependent function type $\prd{x:A}\prd{y:B(x)}C(x,y)$. In the special case where $C(x,y)$ is a family of types indexed by two elements $x,y:A$ of the same type, then we often write
  \begin{equation*}
    \prd{x,y:A}C(x,y)
  \end{equation*}
  for the type $\prd{x:A}\prd{y:A}C(x,y)$.

  With the idea of iterating function types, we see that type of the composition operation $\comp$ is
  \begin{equation*}
    (B\to C)\to ((A\to B)\to (A\to C)).
  \end{equation*}
  It is the type of functions, taking a function $g:B\to C$, to the type of functions $(A\to B)\to (A\to C)$. Thus, $\comp(g)$ is again a function, mapping a function $f:A\to B$ to a function of type $A\to C$.
\end{rmk}

\begin{defn}
For any three types $A$, $B$, and $C$ in context $\Gamma$, there is a \define{composition}\index{function!composition|textbf}\index{composition!of functions|textbf} operation
\begin{equation*}
\comp:(B\to C)\to ((A\to B)\to (A\to C)).
\end{equation*}
We will usually write $g\circ f$\index{g o f@{$g\circ f$}|textbf} for $\comp(g,f)$\index{comp(g,f)@{$\comp(g,f)$}|textbf}\index{comp(g,f)@{$\comp(g,f)$}|see {composition, of functions}}.
\end{defn}

\begin{constr}
  The idea of the definition is to define $\comp(g,f)$ to be the function $\lam{x}g(f(x))$. The function $\comp$ is therefore defined as
  \begin{equation*}
    \comp\defeq \lam{g}\lam{f}\lam{x}g(f(x)).
  \end{equation*}
  The derivation we use to construct $\comp$ is as follows:
  \begin{small}
  \begin{prooftree}
    \AxiomC{$\Gamma\vdash A~\type$}
    \AxiomC{$\Gamma\vdash B~\type$}
    \RightLabel{(a)}
    \BinaryInfC{$\Gamma,f:B^A,x:A\vdash f(x):B$}
    \UnaryInfC{$\Gamma,g:C^B,f:B^A,x:A\vdash f(x):B$}
    \AxiomC{$\Gamma\vdash B~\type$}
    \AxiomC{$\Gamma\vdash C~\type$}
    \RightLabel{(b)}
    \BinaryInfC{$\Gamma,g:C^B,y:B\vdash g(y):C$}
    \UnaryInfC{$\Gamma,g:C^B,f:B^A,y:B\vdash g(y):C$}
    \UnaryInfC{$\Gamma,g:C^B,f:B^A,x:A,y:B\vdash g(y):C$}
    \BinaryInfC{$\Gamma,g:C^B,f:B^A,x:A\vdash g(f(x)) : C$}
    \UnaryInfC{$\Gamma,g:C^B,f:B^A\vdash \lam{x}g(f(x)):C^A$}
    \UnaryInfC{$\Gamma,g:B\to C\vdash \lam{f}\lam{x}g(f(x)):B^A\to C^A$}
    \UnaryInfC{$\Gamma\vdash\lam{g}\lam{f}\lam{x}g(f(x)):C^B\to (B^A\to C^A)$}
    \UnaryInfC{$\Gamma\vdash\comp\defeq \lam{g}\lam{f}\lam{x}g(f(x)):C^B\to (B^A\to C^A)$.}
  \end{prooftree}
  \end{small}
  Note, however, that we haven't derived the rules (a) and (b) yet. These rules assert that the \emph{generic functions} of $A\to B$ and $B\to C$ can also be evaluated. The formal derivation of this fact is as follows:
  \begin{prooftree}
    \AxiomC{$\Gamma\vdash A~\type$}
    \AxiomC{$\Gamma\vdash B~\type$}
    \BinaryInfC{$\Gamma\vdash A \to B~\type$}
    \UnaryInfC{$\Gamma,f:A\to B\vdash f:A\to B$}
    \UnaryInfC{$\Gamma,f:A\to B,x:A\vdash f(x):B$.}
  \end{prooftree}
  This completes the construction of $\comp$.
\end{constr}

In the remainder of this section we will see how to use the given rules for function types to derive the laws of a category\index{category laws!for functions} for functions. These are the laws that assert that function composition is associative and that the identity function satisfies the unit laws.

\begin{lem}
Composition of functions is associative\index{associativity!of function composition}, i.e., we can derive
\begin{prooftree}
\AxiomC{$\Gamma\vdash f:A\to B$}
\AxiomC{$\Gamma\vdash g:B\to C$}
\AxiomC{$\Gamma\vdash h:C\to D$}
\TrinaryInfC{$\Gamma \vdash (h\circ g)\circ f\jdeq h\circ(g\circ f):A\to D$.}
\end{prooftree}
\end{lem}

\begin{proof}
  The main idea of the proof is that both $((h\circ g)\circ f)(x)$ and $(h\circ (g\circ f))(x)$ evaluate to $h(g(f(x))$, and therefore $(h\circ g)\circ f$ and $h\circ(g\circ f)$ must be judgmentally equal. This idea is made formal in the following derivation:
  \begin{prooftree}
    \AxiomC{$\Gamma\vdash f:A\to B$}
    \UnaryInfC{$\Gamma,x:A\vdash f(x):B$}
    \AxiomC{$\Gamma\vdash g:B\to C$}
    \UnaryInfC{$\Gamma,y:B\vdash g(y):C$}
    \UnaryInfC{$\Gamma,x:A,y:B\vdash g(y):C$}
    \BinaryInfC{$\Gamma,x:A\vdash g(f(x)):C$}
    \AxiomC{$\Gamma\vdash h:C\to D$}
    \UnaryInfC{$\Gamma,z:C\vdash h(z):D$}
    \UnaryInfC{$\Gamma,x:A,z:C\vdash h(z):D$}
    \BinaryInfC{$\Gamma,x:A\vdash h(g(f(x))):D$}
    \UnaryInfC{$\Gamma,x:A\vdash h(g(f(x)))\jdeq h(g(f(x))):D$}
    \UnaryInfC{$\Gamma,x:A\vdash (h\circ g)(f(x))\jdeq h((g\circ f)(x)):D$}
    \UnaryInfC{$\Gamma,x:A\vdash ((h\circ g)\circ f)(x)\jdeq (h\circ (g \circ f))(x):D$}
    \UnaryInfC{$\Gamma\vdash (h\circ g)\circ f\jdeq h\circ(g\circ f):A\to D$.}
  \end{prooftree}
\end{proof}

\begin{lem}\label{lem:fun_unit}
Composition of functions satisfies the left and right unit laws\index{left unit law|see {unit laws}}\index{right unit law|see {unit laws}}\index{unit laws!for function composition}, i.e., we can derive
\begin{prooftree}
\AxiomC{$\Gamma\vdash f:A\to B$}
\UnaryInfC{$\Gamma\vdash \idfunc[B]\circ f\jdeq f:A\to B$}
\end{prooftree}
and
\begin{prooftree}
\AxiomC{$\Gamma\vdash f:A\to B$}
\UnaryInfC{$\Gamma\vdash f\circ\idfunc[A]\jdeq f:A\to B$.}
\end{prooftree}
\end{lem}

\begin{proof}
  Note that it suffices to derive that $\idfunc(f(x))\jdeq f(x)$ in context $\Gamma,x:A$, because once we derived this equality we can finish the derivation with
  \begin{prooftree}
    \AxiomC{$\vdots$}
    \UnaryInfC{$\Gamma,x:A\vdash\idfunc(f(x))\jdeq f(x):B$}
    \UnaryInfC{$\Gamma\vdash\lam{x}\idfunc(f(x))\jdeq\lam{x}f(x):A\to B$}
    \AxiomC{$\Gamma\vdash f:A\to B$}
    \UnaryInfC{$\Gamma\vdash\lam{x}f(x)\jdeq f:A\to B$}
    \BinaryInfC{$\Gamma\vdash\idfunc\circ f\jdeq f:A\to B$.}  
  \end{prooftree}
  The derivation of the equality $\idfunc(f(x))\jdeq f(x)$ in context $\Gamma,x:A$ is as follows:
  \begin{prooftree}
    \AxiomC{$\Gamma\vdash f:A\to B$}
    \UnaryInfC{$\Gamma,x:A\vdash f(x):B$}
    \AxiomC{$\Gamma\vdash A~\type$}
    \AxiomC{$\Gamma\vdash B~\type$}
    \UnaryInfC{$\Gamma,y:B\vdash\idfunc(y)\jdeq y:B$}
    \BinaryInfC{$\Gamma,x:A,y:B\vdash\idfunc(y)\jdeq y:B$}
    \BinaryInfC{$\Gamma,x:A\vdash\idfunc(f(x))\jdeq f(x):B$.}
  \end{prooftree}
  We leave the right unit law as \cref{ex:fun_right_unit}.
\end{proof}

\begin{exercises}
  \exitem \label{ex:eta_ext}The $\eta$-rule is often seen as a judgmental extensionality principle. Use the $\eta$-rule to show that if $f$ and $g$ take equal values, then they must be equal, i.e., give a derivation for the rule
  \begin{prooftree}
    \def\fCenter{\Gamma}
    \Axiom$\fCenter\vdash f:\prd{x:A}B(x)$
    \noLine
    \UnaryInf$\fCenter\vdash g:\prd{x:A}B(x)$
    \noLine
    \UnaryInf$\fCenter,x:A\vdash f(x)\jdeq g(x):B(x)$
    \UnaryInf$\fCenter\vdash f\jdeq g:\prd{x:A}B(x).$
  \end{prooftree}
  \exitem \label{ex:fun_right_unit}Give a derivation for the right unit law of \cref{lem:fun_unit}.\index{unit laws!for function composition}
  \exitem 
  \begin{subexenum}
  \item Construct the \define{constant map}\index{constant map|textbf}\index{function!constant map|textbf}\index{const x@{$\const_x$}|textbf}\index{function!const@{$\const$}|textbf}
    \begin{prooftree}
      \AxiomC{$\Gamma\vdash A~\textrm{type}$}
      \UnaryInfC{$\Gamma,y:B\vdash \const_y:A\to B$.}
    \end{prooftree}
  \item Show that
    \begin{prooftree}
      \AxiomC{$\Gamma\vdash f:A\to B$}
      \UnaryInfC{$\Gamma,z:C\vdash \const_z\circ f\jdeq\const_z : A\to C$.}
    \end{prooftree}
  \item Show that
    \begin{prooftree}
      \AxiomC{$\Gamma\vdash A~\textrm{type}$}
      \AxiomC{$\Gamma\vdash g:B\to C$}
      \BinaryInfC{$\Gamma,y:B\vdash g\circ\const_y\jdeq \const_{g(y)}:A\to C$.}
    \end{prooftree}
  \end{subexenum}
  \exitem \label{ex:swap}
  \begin{subexenum}
  \item Define the \define{swap function}\index{function!swap|textbf}\index{swap function|textbf}
    \begin{prooftree}
      \AxiomC{$\Gamma\vdash A~\mathrm{type}$}
      \AxiomC{$\Gamma\vdash B~\mathrm{type}$}
      \AxiomC{$\Gamma,x:A,y:B\vdash C(x,y)~\mathrm{type}$}
      \TrinaryInfC{$\Gamma\vdash \sigma:\Big(\prd{x:A}\prd{y:B}C(x,y)\Big)\to\Big(\prd{y:B}\prd{x:A}C(x,y)\Big)$}
    \end{prooftree}
    that swaps the order of the arguments.
  \item Show that
  \end{subexenum}
  \vspace{-.5\baselineskip}
  \begin{small}
    \begin{prooftree}
      \AxiomC{$\Gamma\vdash A~\mathrm{type}$}
      \AxiomC{$\Gamma\vdash B~\mathrm{type}$}
      \AxiomC{$\Gamma,x:A,y:B\vdash C(x,y)~\mathrm{type}$}
      \TrinaryInfC{$\Gamma\vdash \sigma\circ\sigma\jdeq\idfunc:\Big(\prd{x:A}\prd{y:B}C(x,y)\Big)\to \Big(\prd{x:A}\prd{y:B}C(x,y)\Big).$}
    \end{prooftree}
  \end{small}
\end{exercises}
\index{dependent function type|)}

%%% Local Variables:
%%% mode: latex
%%% TeX-master: "hott-intro"
%%% End:

\section{The natural numbers}
\label{sec:nat}

\index{inductive type|(}
\index{natural numbers|(}

The set of natural numbers is the most important object in mathematics. We quote Bishop\index{Bishop on the positive integers}, from his Constructivist Manifesto, the first chapter in Foundations of Constructive Analysis \cite{Bishop1967}, where he gives a colorful illustration of its importance to mathematics.

\begin{quote}
  ``The primary concern of mathematics is number, and this means the positive integers. We feel about number the way Kant felt about space. The positive integers and their arithmetic are presupposed by the very nature of our intelligence and, we are tempted to believe, by the very nature of intelligence in general. The development of the theory of the positive integers from the primitive concept of the unit, the concept of adjoining a unit, and the process of mathematical induction carries complete conviction. In the words of Kronecker, the positive integers were created by God. Kronecker would have expressed it even better if he had said that the positive integers were created by God for the benefit of man (and other finite beings). Mathematics belongs to man, not to God. We are not interested in properties of the positive integers that have no descriptive meaning for finite man. When a man proves a positive integer to exist, he should show how to find it. If God has mathematics of his own that needs to be done, let him do it himself.''
\end{quote}

A bit later in the same chapter, he continues:

\begin{quote}
  ``Building on the positive integers, weaving a web of ever more sets and ever more functions, we get the basic structures of mathematics: the rational number system, the real number system, the euclidean spaces, the complex number system, the algebraic number fields, Hilbert space, the classical groups, and so forth. Within the framework of these structures, most mathematics is done. Everything attaches itself to number, and every mathematical statement ultimately expresses the fact that if we perform certain computations within the set of positive integers, we shall get certain results.''
\end{quote}

\subsection{The formal specification of the type of natural numbers}
The type $\N$\index{N@{$\N$}|see {natural numbers}} of \define{natural numbers} is the archetypal example of an inductive type\index{inductive type!natural numbers}. The rules we postulate for the type of natural numbers come in four sets, just as the rules for $\Pi$-types:
\begin{enumerate}
\item The formation rule, which asserts that the type $\N$ can be formed.
\item The introduction rules, which provide the zero element $\zeroN$ and the successor function $\succN$.
\item The elimination rule. This rule is the type theoretic version of the induction principle for $\N$.
\item The computation rules, which assert that any application of the elimination rule behaves as expected on the constructors $\zeroN$ and $\succN$ of $\N$.
\end{enumerate}

\subsubsection{The formation rule of $\N$}

\index{natural numbers!rules for N@{rules for $\N$}!formation}
\index{rules!for N@{for $\N$}!formation}
The type $\N$ is formed by the \define{$\N$-formation} rule
\begin{prooftree}
  \AxiomC{}
  \RightLabel{$\N$-form.}
  \UnaryInfC{$\vdash \N~\type$}
\end{prooftree}
In other words, $\N$ is postulated to be a type in the empty context.

\subsubsection{The introduction rules of $\N$}
Unlike the set of positive integers in Bishop's remarks, Peano's first axiom postulates that $0$ is a natural number. The introduction rules for $\N$ equip it with the \define{zero element} and the \define{successor function}.
\index{natural numbers!rules for N@{rules for $\N$}!introduction rules}
\index{rules!for N@{for $\N$}!introduction rules}
\index{natural numbers!operations on N@{operations on $\N$}!0 N@{$\zeroN$}}
\index{0 N@{$\zeroN$}}
\index{successor function!on N@{on $\N$}}
\index{natural numbers!operations on N@{operations on $\N$}!succ N@{$\succN$}}
\index{succ N@{$\succN$}}

\bigskip
\begin{minipage}{.45\textwidth}
  \begin{prooftree}
    \AxiomC{}
    \UnaryInfC{$\vdash \zeroN:\N$}
  \end{prooftree}
\end{minipage}
\begin{minipage}{.45\textwidth}
  \begin{prooftree}
    \AxiomC{}
    \UnaryInfC{$\vdash \succN:\N\to\N$}
  \end{prooftree}
\end{minipage}

\bigskip
\begin{rmk}
  Every element in type theory always comes equipped with its type. Therefore it is possible in type theory that all elements have a \emph{unique} type. In general, it is therefore good practice to make sure that every element is given a unique name, and in formalized mathematics in computer proof assistants this is even required. For example, the element $\zeroN$ has type $\N$, and it is not also a type of $\Z$. This is why we annotate the terms $\zeroN$ and $\succN$ with their type in the subscript. The type $\Z$ of the integers will be introduced in the next section, which will come equipped with a zero element $\zeroZ$ and a successor function $\succZ$.
\end{rmk}

\subsubsection{The induction principle of $\N$}

\index{natural numbers!rules for N@{rules for $\N$}!elimination|see {induction}}
\index{natural numbers!rules for N@{rules for $\N$}!induction principle|(}
\index{induction principle!of N@{of $\N$}|(}
The classical induction principle of the natural numbers tells us what we have to do in order to show that $\forall_{(n\in\N)}P(n)$ holds, for a predicate $P$ over $\N$. Recall that a predicate $P$ on a set $X$ is just a proposition $P(x)$ about an arbitrary $x\in X$. For example, the assertion that `$n$ is divisible by five' is a predicate on the natural numbers.

In dependent type theory we may think of a type family $P$ over $\N$ as a predicate over $\N$. The type theoretical induction principle of $\N$ is therefore formulated using a type family $P$ over $\N$:\index{ind N@{$\indN$}|textbf}\index{rules!for N@{for $\N$}!induction principle|textbf}\index{natural numbers!indN@{$\indN$}|textbf}
\begin{prooftree}
  \def\fCenter{\Gamma}
  \Axiom$\fCenter, n:\N\vdash P(n)~\type$
  \noLine
  \UnaryInf$\fCenter\ \vdash p_0:P(\zeroN)$
  \noLine
  \UnaryInf$\fCenter\ \vdash p_S:\prd{n:\N}P(n)\to P(\succN(n))$
  \RightLabel{$\N$-ind.}
  \UnaryInf$\fCenter\ \vdash \indN(p_0,p_S):\prd{n:\N} P(n)$
\end{prooftree}
In other words, the type theoretical induction principle of $\N$ tells us what we need to do in order to construct a dependent function $\prd{n:\N}P(n)$. Just as in the classical induction principle, there are two things to be constructed given a type family $P$ over $\N$: in the \define{base case}\index{base case} we need to construct an element $p_0:P(\zeroN)$, and for the \define{inductive step}\index{inductive step} we need to construct a function of type $P(n)\to P(\succN(n))$ for all $n:\N$.

\begin{rmk}
  We might alternatively present the induction principle of $\N$ as the following inference rule
  \begin{prooftree}
    \AxiomC{$\Gamma,n:\N\vdash P(n)~\type$}
    \UnaryInfC{$\Gamma\vdash \indN : P(\zeroN)\to \Big(\Big(\prd{n:\N}P(n)\to P(\succN(n))\Big)\to \prd{n:\N}P(n)\Big)$.}
  \end{prooftree}
  In other words, for any type family $P$ over $\N$ there is a \emph{function} $\indN$ that takes two arguments, one for the base case and one for the inductive step, and returns a section of $P$. We claim that this rule is \emph{interderivable} with the rule $\N$-ind above.
  
  To see that indeed we get such a function from the rule $\N$-ind, we use generic elements. First, we let $\Gamma'$ be the context
  \begin{equation*}
    \Gamma,~p_0:P(\zeroN),~p_S:\prd{n:\N}P(n)\to P(\succN(n)).
  \end{equation*}
  By weakening we obtain that
  \begin{align*}
    & \Gamma',~n:\N\vdash P(n)~\type \\
    & \Gamma'\vdash p_0 : P(\zeroN) \\
    & \Gamma'\vdash p_S : \prd{n:\N}P(n)\to P(\succN(n)).
  \end{align*}
  Therefore, the induction principle of $\N$ provides us with a dependent function
  \begin{equation*}
    \Gamma' \vdash \indN(p_0,p_S) : \prd{n:\N}P(n).
  \end{equation*}
  Now we proceed by $\lambda$-abstraction twice to obtain a function
  \begin{equation*}
    \indN : P(\zeroN)\to \Big(\Big(\prd{n:\N}P(n)\to P(\succN(n))\Big) \to \prd{n:\N}P(n)\Big)
  \end{equation*}
  in the original context $\Gamma$. This shows that we can define the function $\indN$ from the rule $\N$-ind. Conversely, we can derive the rule $\N$-ind from the rule that presents $\indN$ as a function. We conclude that the ``official'' rule $\N$-ind and the rule that presents $\indN$ as a function are indeed interderivable.
\end{rmk}
\index{natural numbers!rules for N@{rules for $\N$}!induction|)}
\index{induction principle!of N@{of $\N$}|)}

\subsubsection{The computation rules of $\N$}

\index{computation rules!for N@{for $\N$}|(}
\index{natural numbers!rules for N@{rules for $\N$}!computation rules|(}
The computation rules for $\N$ postulate that the dependent function
\begin{equation*}
  \indN(p_0,p_S):\prd{n:\N}P(n)
\end{equation*}
behaves as expected when it is applied to $\zeroN$ or a successor. There is one computation rule for each step in the induction principle, covering the base case and the inductive step.

The computation rule for the base case is\index{rules!for N@{for $\N$}!computation rules|(}\index{computation rules!for N@{for $\N$}|textbf}
\begin{prooftree}
    \def\fCenter{\Gamma}
  \Axiom$\fCenter, n:\N\vdash P(n)~\type$
  \noLine
  \UnaryInf$\fCenter\ \vdash p_0:P(\zeroN)$
  \noLine
  \UnaryInf$\fCenter\ \vdash p_S:\prd{n:\N}P(n)\to P(\succN(n))$
  \UnaryInf$\fCenter\ \vdash \indN(p_0,p_S,\zeroN)\jdeq p_0 : P(\zeroN).$
\end{prooftree}
\begin{samepage}
  The computation rule for the inductive step has the same premises as the computation rule for the base case:
  \begin{prooftree}
    \AxiomC{$\cdots$}
    \UnaryInfC{$\Gamma, n:\N \vdash  \indN(p_0,p_S,\succN(n))\jdeq p_S(n,\indN(p_0,p_S,n)) : P(\succN(n))$.}
  \end{prooftree}
\end{samepage}
This completes the formal specification of the type $\N$ of natural numbers.
\index{rules!for N@{for $\N$}!computation rules|)}
\index{computation rules!for N@{for $\N$}|)}
\index{natural numbers!rules for N@{rules for $\N$}!computation rules|)}

\subsection{Addition on the natural numbers}

\index{addition on N@{addition on $\N$}|(}
\index{natural numbers!operations on N@{operations on $\N$}!addition|(}
The type theoretic induction principle of $\N$ can be used to do all the usual constructions of operations on $\N$, and to derive all the familiar properties about natural numbers. Many of those properties, however, require a few more ingredients of Martin-L\"of's dependent type theory. For example, the traditional inductive proof that the triangular numbers can be calculated by
\begin{equation*}
  1+\cdots+n = \frac{n(n+1)}{2}
\end{equation*}
is analogous in type theory, but it requires the identity type to state this equation. We will introduce the identity type in \cref{sec:identity}. Until we have fully specified all the ways of forming types in Martin-L\"of's dependent type theory, we are a bit limited in what we can do with the natural numbers, but at the present stage we can define some of the familiar operations on $\N$. We give in this section the type theoretical construction the \define{addition operation} by induction on $\N$, along with the complete derivation tree.

\begin{defn}\label{defn:addN}
  We define a function\index{add N@{$\addN$}|textbf}\index{natural numbers!operations on N@{operations on $\N$}!add N@{$\addN$}|textbf}\index{addition on N@{addition on $\N$}|textbf}\index{natural numbers!addition|textbf}
  \begin{equation*}
    \addN:\N\to (\N\to\N)
  \end{equation*}
  satisfying the specification
  \begin{align*}
    \addN(m,\zeroN) & \jdeq m \\
    \addN(m,\succN(n)) & \jdeq \succN(\addN(m,n)).
  \end{align*}
  Usually we will write $m+n$ for $\addN(m,n)$.
\end{defn}

\begin{proof}[Construction.]
  We will construct the binary operation $\addN:\N\to(\N\to\N)$ by induction on the second variable. In other words, we will construct an element
  \begin{equation*}
    m:\N \vdash \addN(m):\N\to\N.
  \end{equation*}
  The context $\Gamma$ we work in is therefore $m:\N$. The induction principle of $\N$ is used with the family of types $P(n)\defeq \N$ indexed by $n:\N$ in context $m:\N$. Therefore we need to construct
  \begin{align*}
    m:\N & \vdash \addzeroN(m) : \N\\
    m:\N & \vdash \addsuccN(m) : \N\to(\N\to\N),\\
    \intertext{in order to obtain}
    m:\N & \vdash \addN(m)\defeq\indN(\addzeroN(m),\addsuccN(m)):\N\to\N.
  \end{align*}
  The element $\addzeroN(m):\N$ in context $m:\N$ is of course defined to be $m:\N$, i.e., by the generic element, because adding zero should just be the identity function.
  To see how the function $\addsuccN(m):\N\to(\N\to\N)$ should be defined, we look at the specification of $\addN(m)$ when it is applied to a successor:
  \begin{equation*}
    \addN(m,\succN(n))\jdeq \succN(\addN(m,n)).
  \end{equation*}
  This shows us that we should define
  \begin{equation*}
    \addsuccN(m,n,x)\jdeq \succN(x),
  \end{equation*}
  because with this definition we will have
  \begin{align*}
    \addN(m,\succN(n)) & \jdeq \indN(\addzeroN(m),\addsuccN(m),\succN(n)) \\
                       & \jdeq \addsuccN(m,n,\addN(m,n)) \\
                       & \jdeq \succN(\addN(m,n)).
  \end{align*}
  The formal derivation for the construction of $\addsuccN$ is as follows:
  \begin{prooftree}
    \AxiomC{}
    \UnaryInfC{$\vdash\N~\type$}
    \AxiomC{}
    \UnaryInfC{$\vdash\N~\type$}
    \AxiomC{}
    \UnaryInfC{$\vdash \succN:\N\to\N$}
    \BinaryInfC{$n:\N\vdash \succN:\N\to\N$}
    \BinaryInfC{$m:\N,n:\N \vdash \succN:\N\to\N$}
    \UnaryInfC{$m:\N \vdash \lam{n}\succN:\N\to (\N \to \N)$}
    \UnaryInfC{$m:\N \vdash \addsuccN(m) \defeq \lam{n}\succN:\N\to (\N \to \N)$.}
  \end{prooftree}
  We combine this derivation with the induction principle of $\N$ to complete the construction of addition:
  \begin{prooftree}
    \AxiomC{$\vdots$}
    \UnaryInfC{$m:\N\vdash \addzeroN(m) \defeq m:\N$}
    \AxiomC{$\vdots$}
    \UnaryInfC{$m:\N\vdash \addsuccN(m):\N\to (\N \to \N)$}
    \BinaryInfC{$m:\N\vdash\indN(\addzeroN(m),\addsuccN(m)):\N\to \N$}
    \UnaryInfC{$m:\N\vdash\addN(m)\defeq\indN(\addzeroN(m),\addsuccN(m)):\N\to \N$.}
  \end{prooftree}
  The asserted judgmental equalities then hold by the computation rules for $\N$.
\end{proof}

\begin{rmk}
  By the computation rules for $\N$ it follows that
  \begin{equation*}
    m+\zeroN\jdeq m,\qquad\text{and}\qquad m+\succN(n)\jdeq\succN(m+n).
  \end{equation*}
  A simple consequence of this definition is that $\succN(n)\jdeq n+1$, as one would expect. However, the rules that we provided so far are not sufficient to also conclude that $\zeroN+n\jdeq n$ and $\succN(m) + n\jdeq \succN(m+n)$. In fact, dependent type theory with its inductive types does not provide any means to prove such judgmental equalities.

  Nevertheless, once we have introduced the \emph{identity type} in \cref{sec:identity} we will be able to \emph{identify} $\zeroN+n$ with $n$, and $\succN(m)+n$ with $\succN(m+n)$. See \cref{prp:unit-laws-add-N,prp:successor-laws-add-N}. 
\end{rmk}
\index{addition on N@{addition on $\N$}|)}
\index{natural numbers!operations on N@{operations on $\N$}!addition|)}

\subsection{Pattern matching}

Note that in definition \cref{defn:addN} we stated that $\addN$ is a function of type $\N\to (\N\to\N)$ satisfying the specification
\begin{align*}
    \addN(m,\zeroN) & \jdeq m \\
  \addN(m,\succN(n)) & \jdeq \succN(\addN(m,n)).
\end{align*}
Such a specification is enough to characterize the function $\addN(m)$ entirely, because it postulates the behaviour of $\addN(m)$ at the constructors of $\N$.
It is therefore convenient to present the definition of $\addN$ recursively in the following way:
\begin{align*}
  \addN(m,\zeroN) & \defeq m \\
  \addN(m,\succN(n)) & \defeq \succN(\addN(m,n)).
\end{align*}

More generally, if we want to define a dependent function $f:\prd{n:\N}P(n)$ by induction on $n$, using
\begin{align*}
  p_0 & : P(\zeroN) \\
  p_S & : \prd{n:\N} P(n)\to P(\succN(n)),
\end{align*}
we can present that definition by writing
\begin{align*}
  f(\zeroN) & \defeq p_0 \\
  f(\succN(n)) & \defeq p_S(n,f(n)). 
\end{align*}
When the definition of $f$ is presented in this way, we say that $f$ is defined by \define{pattern matching}\index{pattern matching} on the variable $n$. To see that $f$ is fully specified when it is defined by pattern matching, we have to recover the dependent function
\begin{equation*}
  p_S:\prd{n:\N}P(n)\to P(\succN(n))
\end{equation*}
from the expression $p_S(n,f(n))$ that was used in the definition of $f$. This can of course be done by replacing all occurrences of the term $f(n)$ in the expression $p_S(n,f(n))$ with a fresh variable $x:P(n)$. In other words, when a subexpression of $p_S(n,f(n))$ \emph{matches} $f(n)$, we replace that subexpression by $x$. This is where the name \emph{pattern matching} comes from. Many computer proof assistants have the pattern matching mechanism built in, because it is a concise way of presenting a recursive definition. Another advantage of presenting definitions by pattern matching is that the judgmental equalities by which the object is defined are immediately displayed. Those judgmental equalities are all that is known about the defined object, and often proving things about it amounts to finding a way to apply those judgmental equalities.

Pattern matching can also be used in more complicated situations, such as defining a function by pattern matching on multiple variables, or by iterated pattern matching. For example, an alternative definition of addition on $\N$ could be given by pattern matching on both variables:
\begin{align*}
  \addpN(\zeroN,\zeroN) & \defeq \zeroN \\*
  \addpN(\zeroN,\succN(n)) & \defeq \succN(n) \\*
  \addpN(\succN(m),\zeroN) & \defeq \succN(m) \\*
  \addpN(\succN(m),\succN(n)) & \defeq \succN(\succN(\addpN(m,n)).
\end{align*}

An example of a definition by iterated pattern matching is the \define{Fibonacci function} $F:\N\to\N$. This function is defined by
\begin{align*}
  F(\zeroN) & \defeq \zeroN \\
  F(\oneN) & \defeq \oneN \\
  F(\succN(\succN(n))) & \defeq F(\succN(n))+F(n).
\end{align*}
However, since $F(\succN(\succN(n)))$ is defined using both $F(\succN(n))$ and $F(n)$, it is not immediately clear how to present $F$ by the usual induction principle of $\N$. It is a nice puzzle, which we leave as \cref{ex:fibonacci}, to find a definition of the Fibonacci sequence with the usual induction principle of $\N$. 

\begin{exercises}
  \exitem
  \begin{subexenum}
  \item Define the \define{multiplication} operation
    \index{multiplication!on N@{on $\N$}|textbf}
    \index{natural numbers!operations on N@{operations on $\N$}!mul N@{$\mulN$}|textbf}
    \index{mul N@{$\mulN$}|textbf}
    \begin{equation*}
      \mulN :\N\to(\N\to\N).
    \end{equation*}
  \item Define the \define{exponentiation function} $n,m\mapsto m^n$ of type $\N\to (\N\to \N)$.
  \index{exponentiation function on N@{exponentiation function on $\N$}|textbf}
  \index{natural numbers!operations on N@{operations on $\N$}!exponentiation|textbf}
  \end{subexenum}
  \exitem Define the binary \define{min} and \define{max} functions
  \index{minimum function|textbf}
  \index{maximum function|textbf}
  \index{natural numbers!operations on N@{operations on $\N$}!minN@{$\minN$}|textbf}
  \index{natural numbers!operations on N@{operations on $\N$}!maxN@{$\maxN$}|textbf}
  \begin{equation*}
    \minN,\maxN:\N\to(\N\to\N).
  \end{equation*}
  \exitem
  \begin{subexenum}
  \item Define the \define{triangular numbers}
    \begin{equation*}
      1+\cdots+n.
    \end{equation*}
    \index{triangle number|textbf}
    \index{natural numbers!operations on N@{operations on $\N$}!triangle number|textbf}
  \item Define the \define{factorial} operation $n\mapsto n!$.
  \index{factorial operation|textbf}
  \index{natural numbers!operations on N@{operations on $\N$}!n factorial@{$n"!$}|textbf}
  \end{subexenum}
  \exitem Define the \define{binomial coefficient} $\binom{n}{k}$\index{(n k)@{$\binom{n}{k}$}|see {binomial coefficient}}\index{(n k)@{$\binom{n}{k}$}|textbf}\index{binomial coefficient|textbf}\index{natural numbers@operations on N@{operations on $\N$}!binomial coefficient|textbf} for any $n,k:\N$, making sure that $\binom{n}{k}\jdeq 0$ when $n<k$.
  \index{binomial coefficient|textbf}
  \index{natural numbers!operations on N@{operations on $\N$}!binomial coefficient|textbf}
  \exitem \label{ex:fibonacci}Use the induction principle of $\N$ to define the \define{Fibonacci sequence} as a function $F:\N\to\N$ that satisfies the equations\index{Fibonacci sequence|textbf}\index{natural numbers!operations on N@{operations on $\N$}!Fibonacci sequence|textbf}
  \begin{samepage}
    \begin{align*}
      F(\zeroN) & \jdeq \zeroN \\
      F(\oneN) & \jdeq \oneN \\
      F(\succN(\succN(n))) & \jdeq F(\succN(n))+F(n).
    \end{align*}
  \end{samepage}
  \exitem Define division by two rounded down as a function $\N\to\N$ in two ways: first by pattern matching, and then directly by the induction principle of $\N$.
\end{exercises}
\index{natural numbers|)}

%%% Local Variables:
%%% mode: latex
%%% TeX-master: "hott-intro"
%%% End:

\section{More inductive types}\label{sec:inductive}

In the previous section we introduced the type of natural numbers. Many other types can also be introduced as inductive types. In this section we will see by example how that works. We will introduce the unit type, the empty type, coproducts, dependent pair types, and cartesian products as inductive types, and in the next section the identity type will be introduced as an inductive family of types.

From this section on, we will also start using a more informal style. The inductive types will be specified by a description of their constructors and induction principles in terms of operations on dependent function types, which is more tightly connected with how we will use them, but we will not display the formal rules. It is a good exercise for the reader to formally specify at least some of the inductive types of this section by stating their formal rules.

\subsection{The idea of general inductive types}

Just like the type of natural numbers, other inductive types are also specified by their \emph{constructors}, an \emph{induction principle}, and their \emph{computation rules}: 
\begin{enumerate}
\item The constructors tell what structure the inductive type comes equipped with. There may be any finite number of constructors, even no constructors at all, in the specification of an inductive type. 
\item The induction principle specifies the data that should be provided in order to construct a section of an arbitrary type family over the inductive type. The idea of the induction principle is always the same: in order to define a dependent function $f:\prd{x:A}B(x)$, one has to specify the behaviour of $f$ at the constructors of $A$.
\item The computation rules assert that the inductively defined section agrees on the constructors with the data that was used to define the section. Thus, there is a computation rule for every constructor.
\end{enumerate}
Since any inductively defined function is entirely determined by its behavior on the constructors, we can again present such inductive definitions by pattern matching. Therefore, we will also specify for each inductive type how to give definitions by pattern matching.

\subsection{The unit type}
\index{unit type|(}
\index{inductive type!unit type|(}
A straightforward example of an inductive type is the \emph{unit type}, which has just one constructor. 
Its induction principle is analogous to just the base case of induction on the natural numbers.

\begin{defn}
We define the \define{unit type}\index{1 @{$\unit$}|see {unit type}}\index{1 @{$\unit$}|textbf}\index{unit type|textbf} to be a type $\unit$ equipped with a term\index{unit type!star@{$\ttt$}|textbf}
\begin{equation*}
\ttt:\unit,
\end{equation*}
satisfying the induction principle\index{induction principle!of the unit type|textbf}\index{unit type!induction principle|textbf} that for any family of types $P(x)$ indexed by $x:\unit$, there is a function\index{ind 1@{$\indunit$}|textbf}\index{unit type!indunit@{$\indunit$}|textbf}
\begin{equation*}
\indunit : P(\ttt)\to\prd{x:\unit}P(x)
\end{equation*}
for which the computation rule\index{computation rules!for the unit type|textbf}\index{unit type!computation rules|textbf}
\begin{equation*}
\indunit(p,\ttt) \jdeq p
\end{equation*}
holds. Alternatively, a definition of a dependent function $f:\prd{x:\unit}P(x)$ by induction using $p:P(\ttt)$ can be presented by pattern matching as
\begin{equation*}
  f(\ttt)\defeq p.
\end{equation*}
\end{defn}

A special case of the induction principle arises when $P$ does not actually depend on $\unit$. If we are given a type $A$, then we can first weaken it to obtain the constant family over $\unit$, with value $A$. Then the induction principle of the unit type provides a function
\begin{equation*}
  \indunit : A \to (\unit\to A).
\end{equation*}
In other words, by the induction principle for the unit type we obtain for every $x:A$ a function $\pt_x\defeq\indunit(x):\unit\to A$.\index{pt x@{$\pt_x$}|textbf}
\index{unit type|)}
\index{inductive type!unit type|)}

\subsection{The empty type}
\index{empty type|(}
\index{inductive type!empty type|(}
The empty type is a degenerate example of an inductive type. It does \emph{not} come equipped with any constructors, and therefore there are also no computation rules. The induction principle merely asserts that any type family has a section. In other words: if we assume the empty type has a term, then we can prove anything.

\begin{defn}
We define the \define{empty type}\index{0 @{$\emptyt$}|see {empty type}}\index{0 @{$\emptyt$}|textbf}\index{empty type|textbf} to be a type $\emptyt$ satisfying the induction principle\index{induction principle!of the empty type|textbf}\index{empty type!induction principle|textbf} that for any family of types $P(x)$ indexed by $x:\emptyt$, there is a term\index{ind 0@{$\indempty$}|textbf}\index{empty type!indempty@{$\indempty$}|textbf}
\begin{equation*}
\indempty : \prd{x:\emptyt}P(x).
\end{equation*}
\end{defn}

It is again a special case of the induction principle that we have a function
\begin{equation*}
  \exfalso\defeq\indempty:\emptyt\to A
\end{equation*}
for any type $A$. Indeed, to obtain this function one first weakens $A$ to obtain the constant family over $\emptyt$ with value $A$, and then the induction principle gives the desired function. The function $\exfalso$ can be used to draw any conclusion after deriving a contradiction, because \emph{ex falso quodlibet}.

We can also use the empty type to define the negation operation on types.

\begin{defn}
  For any type $A$ we define \define{negation}\index{negation!of types|textbf}\index{n A@{$\neg A$}|see {negation}|textbf} of $A$ by
  \begin{align*}
    \neg A & \defeq A\to\emptyt.
  \intertext{We also say that a type $A$ \define{is empty} if it comes equipped with an element of type $\neg A$. Therefore, we also define}
    \isempty(A) & \defeq A\to\emptyt.
  \end{align*}
\end{defn}

\begin{rmk}
  Since $\neg A$ is the type of functions from $A$ to $\emptyt$, a proof of $\neg A$ is given by assuming that $A$ holds, and then constructing an element of the empty type. In other words, we prove $\neg A$ by assuming $A$ and deriving a contradiction. This proof technique is called \define{proof of negation}\index{proof of negation}.

  Proofs of negation should not be confused with proofs by contradiction\index{proof by contradiction}. Even though a proof of negation involves deriving a contradiction, in logic a \define{proof by contradiction} of a proposition $P$ is an argument where we conclude that $P$ holds after showing that $\neg P$ implies a contradiction. In other words, a proof by contradiction uses the logical step $\neg\neg P \Rightarrow P$, which is also called \define{double negation elimination}.

  In type theory, however, note that the type $\neg\neg A$ is the type of functions
  \begin{equation*}
    (A\to\emptyt)\to\emptyt.
  \end{equation*}
  This type is quite different from the type $A$ itself, and with the given rules of type theory it is not possible to construct a function $\neg\neg A \to A$ unless more is known about the type $A$. In other words, before one can prove by contradiction that there is an element in $A$, one has to construct a function $\neg\neg A\to A$, and it depends on the specific type $A$ whether this is possible at all. In \cref{ex:dne-is-decidable} we will see a situation where we can indeed construct a function $\neg\neg A\to A$. In practice, however, we will rarely use double negation elimination.
\end{rmk}

In the following proposition we illustrate how to work with the type theoretic definition of negation.

\begin{prp}\label{prp:contravariant-neg}
  For any two types $P$ and $Q$, there is a function
  \begin{equation*}
    (P\to Q)\to (\neg Q \to \neg P).
  \end{equation*}
\end{prp}

\begin{proof}
  The desired function is defined by $\lambda$-abstraction, so we begin by assuming that we have a function $f:P\to Q$. Then we have to construct a function $\neg Q\to\neg P$, which is again constructed by $\lambda$-abstraction. We assume that we have $\tilde{q}:\neg Q$. By our definition of $\neg Q$ we see that $\tilde{q}$ is a function $Q\to\emptyt$. Now we have to construct a term of type $\neg P$, which is the type of functions $P\to\emptyt$. We apply $\lambda$-abstraction once more, so we assume $p:P$. Now we have
  \begin{align*}
    f & : P \to Q \\*
    \tilde{q} & : Q\to \emptyt \\*
    p & : P,
  \end{align*}
  and our goal is to construct a term of the empty type.

  Since we have $f:P\to Q$ and $p:P$, we obtain $f(p):Q$. Moreover, we have $\tilde{q}:Q\to\emptyt$, so we obtain $\tilde{q}(f(p)):\emptyt$. This completes the proof. The function we have constructed is
  \begin{equation*}
    \lam{f}\lam{\tilde{q}}\lam{p}\tilde{q}(f(p)):(P\to Q)\to(\neg Q\to\neg P).\qedhere
  \end{equation*}
\end{proof}

We leave it to the reader to construct the corresponding natural deduction tree, that formally constructs a function
\begin{equation*}
  (P\to Q)\to(\neg Q\to \neg P).
\end{equation*}
\index{empty type|)}
\index{inductive type!empty type|)}

\subsection{Coproducts}\label{sec:coprod}
\index{coproduct|(}
\index{inductive type!coproduct|(}
\begin{defn}
Let $A$ and $B$ be types. We define the \define{coproduct}\index{disjoint sum|see {coproduct}}\index{coproduct|textbf} $A+B$\index{A + B@{$A+B$}|see {coproduct}} to be a type that comes equipped with\index{inl@{$\inl$}|textbf}\index{coproduct!inl@{$\inl$}|textbf}\index{inr@{$\inr$}|textbf}\index{coproduct!inr@{$\inr$}|textbf}
\begin{align*}
\inl & : A \to A+B \\*
\inr & : B \to A+B,
\end{align*}
satisfying the induction principle\index{induction principle!of coproducts|textbf}\index{coproduct!induction principle|textbf} that for any family of types $P(x)$ indexed by $x:A+B$, there is a term\index{ind +@{$\indcoprod$}|textbf}\index{coproduct!ind+@{$\indcoprod$}|textbf}
\begin{equation*}
\indcoprod : \Big(\prd{x:A}P(\inl(x))\Big)\to\Big(\Big(\prd{y:B}P(\inr(y))\Big)\to\prd{z:A+B}P(z)\Big)
\end{equation*}
for which the computation rules\index{computation rules!for coproducts|textbf}\index{coproduct!computation rules|textbf}
\begin{align*}
\indcoprod(f,g,\inl(x)) & \jdeq f(x) \\*
\indcoprod(f,g,\inr(y)) & \jdeq g(y)
\end{align*}
hold. Alternatively, a definition of a dependent function $h:\prd{x:A+B}P(x)$ by induction using $f:\prd{x:A}P(\inl(x))$ and $g:\prd{y:B}P(\inr(y))$ can be presented by pattern matching as
\begin{align*}
  h(\inl(x)) & \defeq f(x) \\*
  h(\inr(y)) & \defeq g(y).
\end{align*}
Sometimes we write $[f,g]$ for the function $\indcoprod(f,g)$. The coproduct of two types is sometimes also called the \define{disjoint sum}.
\end{defn}

By the induction principle of coproducts we obtain a function
\begin{equation*}
  \indcoprod:(A\to X) \to \big((B\to X) \to (A+B\to X)\big)
\end{equation*}
for any type $X$. Note that this special case of the induction principle of coproducts is very similar to the elimination rule of disjunction in first order logic: if $P$, $P'$, and $Q$ are propositions, then we have
\begin{equation*}
  (P\to Q)\to \big((P'\to Q)\to (P\lor P'\to Q)\big).
\end{equation*}
Indeed, we can think of \emph{propositions as types} and of terms as their constructive proofs. Under this interpretation of type theory the coproduct is indeed the disjunction.

\begin{rmk}\label{rmk:functor-coprod}
  A simple application of the induction principle for coproducts gives us a map\index{coproduct!functorial action|textbf}\index{functorial action!of coproducts|textbf}\index{f + g@{$f+g$}|see {functorial action, of coproducts}}\index{f + g@{$f+g$}|textbf}
  \begin{equation*}
    f+g:A+B\to A'+B'
  \end{equation*}
  for every $f:A\to A'$ and $g:B\to B'$. Indeed, the map $f+g$ is defined by
  \begin{align*}
    (f+g)(\inl(x)) & \defeq \inl(f(x)) \\*
    (f+g)(\inr(y)) & \defeq \inr(g(y)).
  \end{align*}
\end{rmk}

\begin{prp}
  Consider two types $A$ and $B$, and suppose that $B$ is empty. Then there is a function
  \begin{equation*}
    (A+B)\to A.
  \end{equation*}
\end{prp}

\begin{rmk}
  In other words, there is a function
  \begin{equation*}
    \isempty(B) \to ((A+B)\to A),
  \end{equation*}
  for any two types $A$ and $B$. Similarly, there is a function
  \begin{equation*}
    \isempty(A)\to ((A+B)\to B),
  \end{equation*}
  for any two types $A$ and $B$.
\end{rmk}

\begin{proof}
  We will construct the function $(A+B)\to A$ with the induction principle of the coproduct $A+B$. Therefore, we must construct two functions:
  \begin{align*}
    f & : A\to A \\*
    g & : B\to A.
  \end{align*}
  The function $f$ is simply defined to be the identity function $\idfunc:A\to A$. Recall that we have assumed that $B$ is empty, so we have a function $\tilde{b}:B\to\emptyt$. Furthermore, we always have the function $\exfalso:\emptyt\to A$. Therefore, we can define $g\defeq \exfalso\circ \tilde{b}$ to complete the proof.
\end{proof}
\index{coproduct|)}
\index{inductive type!coproduct|)}

\subsection{The type of integers}
\index{integers|(}
The set of integers is usually defined as a quotient of the set $\N\times\N$, by the equivalence relation
\begin{equation*}
  ((n,m)\sim (n',m')) := (n+m' = n'+m).
\end{equation*}
We haven't introduced the identity type yet, in order to consider the type of identifications $n+m'=n'+m$, but more importantly there are no quotient types in Martin-L\"of's dependent type theory. We will only discuss quotient types in \cref{sec:set-quotients} after we have assumed the univalence axiom and propositional truncations, because we will use the univalence axiom and propositional truncations to define them and derive their basic properties. Nevertheless, the type of integers is also definable in dependent type theory without set quotients, but we have to settle for a more pedestrian version of the integers that is defined using coproducts.

\begin{defn}
  We define the \define{integers}\index{Z@{$\Z$}|see {integers}} to be the type $\Z\defeq\N+(\unit+\N)$. The type of integers comes equipped with inclusion functions of the positive and negative integers\index{integers!in-pos@{$\inpos$}}\index{integers!in-neg@{$\inneg$}}
  \begin{alignat*}{2}
    \inpos & \defeq \inr\circ\inr\quad & & : \N\to \Z \\*
    \inneg & \defeq \inl\quad & & : \N \to \Z
  \end{alignat*}
  and with the constants\index{integers!-1 Z@{$-1_\Z$}}\index{integers!0 Z@{$0_\Z$}}\index{integers!1 Z@{$1_\Z$}}\index{-1 Z@{$-1_\Z$}}\index{0 Z@{$0_\Z$}}\index{1 Z@{$1_{\Z}$}}
  \begin{align*}
    -1_\Z & \defeq \inneg(0)\\*
    0_\Z & \defeq \inr(\inl(\ttt))\\*
    1_\Z & \defeq \inpos(0).
  \end{align*}
\end{defn}

The definition of the integers as the coproduct $\N+(\unit+\N)$ can be pictured as follows:
\begin{equation*}
  \begin{tikzcd}[column sep=0]
    \phantom{\unit+\N} & \unit \arrow[dr] & & \N \arrow[dl] \\
    \N \arrow[dr] & \phantom{\unit+\N} & \unit+\N \arrow[dl] & \phantom{\unit+\N} \\
    & \Z
  \end{tikzcd}
\end{equation*}

\begin{rmk}\label{lem:Z_ind}
  The type of integers is built entirely out of inductive types. Therefore it is possible to derive an induction principle especially tailored for the type $\Z$, which can be used to define the basic operations on $\Z$, such as the successor map, addition and multiplication. This induction principle asserts that for any type family $P$ over $\Z$,  we can define a dependent function $f:\prd{k:\Z}P(k)$ recursively by
  \begin{align*}
    f(-1_\Z) & \defeq p_{-1} \\*
    f(\inneg(\succN(n))) & \defeq p_{-S}(n,f(\inneg(n))) \\*
    f(0_\Z) & \defeq p_{0} \\*
    f(1_\Z) & \defeq p_{1} \\*
    f(\inpos(\succN(n))) & \defeq p_S(n,f(\inpos(n))),
  \end{align*}
  where the types of $p_{-1}$, $p_{-S}$, $p_0$, $p_1$, and $p_S$ are 
  \begin{align*}
    p_{-1} & :P(-1_\Z) \\*
    p_{-S} & : \prd{n:\N}P(\inneg(n))\to P(\inneg(\succN(n)))\\*
    p_{0} & : P(0_\Z) \\*
    p_{1} & : P(1_\Z) \\*
    p_{S} & : \prd{n:\N}P(\inpos(n))\to P(\inpos(\succN(n))).
  \end{align*}
\end{rmk}

\begin{defn}
We define the \define{successor function}\index{successor function!on Z@{on $\Z$}|textbf} on the integers $\succZ:\Z\to\Z$\index{succ Z@{$\succZ$}|textbf}\index{integers!succ Z@{$\succZ$}|textbf} using the induction principle of \cref{lem:Z_ind}, taking
\begin{align*}
\succZ(-1_\Z) & \defeq \zeroZ \\*
\succZ(\inneg(\succN(n))) & \defeq \inneg(n) \\*
\succZ(0_\Z) & \defeq \oneZ \\*
\succZ(1_\Z) & \defeq \inpos(1_\N) \\*
\succZ(\inpos(\succN(n))) & \defeq \inpos(\succN(\succN(n))).
\end{align*}
\end{defn}
\index{integers|)}

\subsection{Dependent pair types}

\index{dependent pair type|(}
\index{inductive type!dependent pair type|(}

Given a type family $B$ over $A$, we may consider pairs $(a,b)$ of terms, where $a:A$ and $b:B(a)$. Note that the type of $b$ depends on the first term in the pair. Therefore we call such a pair a \define{dependent pair}\index{dependent pair|textbf}. The type of such dependent pairs is the inductive type that is generated by the dependent pairs.

\begin{defn}
  Consider a type family $B$ over $A$.
  The \define{dependent pair type} (or \define{$\Sigma$-type}) \index{dependent pair type|textbf}\index{S-type@{$\Sigma$-type}|see {dependent pair type}}\index{S-type@{$\Sigma$-type}|textbf}is defined to be the inductive type $\sm{x:A}B(x)$ equipped with a \define{pairing function}\index{pairing function|textbf}\index{pair@{$\pair$}|textbf}\index{dependent pair type!pair@{$\pair$}|textbf}
  \begin{equation*}
    \pair :\prd{x:A} \Big(B(x)\to \sm{y:A}B(y)\Big).
  \end{equation*}
  The induction principle\index{induction principle!of Sigma types@{of $\Sigma$-types}|textbf}\index{dependent pair type!induction principle|textbf} for $\sm{x:A}B(x)$ asserts that for any family of types $P(p)$ indexed by $p:\sm{x:A}B(x)$, there is a function\index{dependent pair type!indSigma@{$\indSigma$}|textbf}\index{ind Sigma@{$\indSigma$}|textbf}
  \begin{equation*}
    \indSigma:\Big(\prd{x:A}\prd{y:B(x)}P(\pair(x,y))\Big)\to\Big(\prd{z:\sm{x:A}B(x)}P(z)\Big).
  \end{equation*}
  satisfying the computation rule\index{computation rules!for S-types@{for $\Sigma$-types}|textbf}\index{dependent pair type!computation rule|textbf}
  \begin{equation*}
    \indSigma(g,\pair(x,y))\jdeq g(x,y).
  \end{equation*}
  Alternatively, a definition of a dependent function $f:\prd{z:\sm{x:A}B(x)}P(z)$ by induction using a function $g:\prd{x:A}\prd{y:B(x)}P((x,y))$ can be presented by pattern matching as
  \begin{equation*}
    f(\pair(x,y))\defeq g(x,y).
  \end{equation*}
  We will usually write $(x,y)$ for $\pair(x,y)$\index{(x,y)@{$(x,y)$}|see {dependent pair}}\index{(x,y)@{$(x,y)$}|textbf}.
\end{defn}

The induction principle of $\Sigma$-types can be used to define the projection functions.

\begin{defn}
  Consider a type $A$ and a type family $B$ over $A$.
  \begin{enumerate}
  \item The \define{first projection map}\index{first projection map|textbf}\index{projection map!first projection|textbf}\index{dependent pair type!pr 1@{$\proj 1$}|textbf}\index{pr 1@{$\proj 1$}|textbf}\index{pr 1@{$\proj 1$}|see{first projection map}}
    \begin{equation*}
      \proj 1:\Big(\sm{x:A}B(x)\Big)\to A
    \end{equation*}
    is defined by induction as
    \begin{equation*}
      \proj 1(x,y) \defeq x.
    \end{equation*}
  \item The \define{second projection map}\index{second projection map|textbf}\index{projection map!second projection|textbf}\index{dependent pair type!pr 2@{$\proj 2$}|textbf}\index{pr 2@{$\proj 2$}|textbf}\index{pr 2@{$\proj 2$}|see{second projection map}} is a dependent function
    \begin{equation*}
      \proj 2 : \prd{p:\sm{x:A}B(x)} B(\proj 1(p)),
    \end{equation*}
    defined by induction as
    \begin{equation*}
      \proj 2(x,y) \defeq y.
    \end{equation*}
  \end{enumerate}
\end{defn}
\index{dependent pair type|)}
\index{inductive type!dependent pair type|)}

\begin{rmk}
  If we want to construct a function
  \begin{equation*}
    f:\prd{z:\sm{x:A}B(x)}P(z)
  \end{equation*}
  by $\Sigma$-induction, then we get to assume a pair $(x,y)$ consisting of $x:A$ and $y:B(x)$ and our goal will be to construct an element of type $P(x,y)$. The induction principle of $\Sigma$-types is therefore converse to the \define{currying operation}, a familiar concept from the theory of programming languages, which is given by the function
  \begin{equation*}
    \evpair : \Big(\prd{z:\sm{x:A}B(x)}P(z)\Big)\to \Big(\prd{x:A}\prd{y:B(x)}P(x,y)\Big)
  \end{equation*}
  given by $f\mapsto\lam{x}\lam{y}f(x,y)$. The induction principle $\indSigma$ is therefore also known as the \define{uncurrying operation}. 
\end{rmk}

\index{cartesian product type|(}
\index{inductive type!cartesian product|(}
A common special case of the $\Sigma$-type occurs when the $B$ is a constant family over $A$, i.e., when $B$ is just a type weakened by $A$.
In this case, the type $\sm{x:A}B$ is the type of \emph{ordinary} pairs $(x,y)$ where $x:A$ and $y:B$, where the type of $y$ does not depend on $x$. The type of ordinary pairs $(x,y)$ consisting of $x:A$ and $y:B$ is of course the \emph{product} of $A$ and $B$, so we see that product types arise as a special case of $\Sigma$-types in a similar way to how function types were defined as a special case of $\Pi$-types.

\begin{defn}
  Consider two types $A$ and $B$. Then we define the \define{(cartesian) product}\index{cartesian product type|textbf}\index{product of types|textbf}\index{A x B@{$A\times B$}|see {cartesian product}}\index{A x B@{$A\times B$}|textbf} $A\times B$ of $A$ and $B$ by
  \begin{equation*}
    A\times B \defeq \sm{x:A}B.
  \end{equation*}
\end{defn}

\begin{rmk}
  Since $A\times B$ is defined as a $\Sigma$-type, it follows that cartesian products also satisfy the induction principle of $\Sigma$-types. In this special case, the induction principle\index{induction principle!of cartesian products|textbf}\index{cartesian product type!induction principle|textbf} for $A\times B$ asserts that for any type family $P$ over $A\times B$ there is a function\index{ind times@{$\ind{\times}$}|textbf}\index{cartesian product type!indtimes@{$\ind{\times}$}|textbf}
\begin{equation*}
\ind{\times} : \Big(\prd{x:A}\prd{y:B}P(x,y)\Big)\to\Big(\prd{z:A\times B} P(z)\Big)
\end{equation*}
that satisfies the computation rule\index{computation rules!for cartesian products}\index{cartesian product type!computation rule|textbf}
\begin{align*}
\ind{\times}(g,(x,y)) & \jdeq g(x,y).
\end{align*}
\end{rmk}

The projection maps are defined similarly to the projection maps of $\Sigma$-types. When one thinks of types as propositions\index{propositions as types!conjunction}, then $A\times B$ is interpreted as the conjunction of $A$ and $B$.
\index{cartesian product type|)}
\index{inductive type!cartesian product|)}

\begin{exercises}
  \exitem
  \begin{subexenum}
  \item \label{ex:int_pred}\index{integers|(}\index{predecessor function|textbf}\index{integers!pred Z@{$\predZ$}|textbf}\index{pred Z@{$\predZ$}|textbf}Define the predecessor function $\predZ:\Z\to \Z$.
  \item \label{ex:int_group_ops}Define the group operations\index{add Z@{$\addZ$}|textbf}\index{integers!add Z@{$\addZ$}|textbf}\index{neg Z@{$\negZ$}}\index{integers!neg Z@{$\negZ$}}
    \begin{align*}
      \addZ & : \Z \to (\Z \to \Z) \\*
      \negZ & : \Z \to \Z.
    \end{align*}
  \item \label{ex:mulZ}Define the multiplication operation $\mulZ : \Z \to (\Z \to \Z)$.\index{mul Z@{$\mulZ$}}\index{integers!mul Z@{$\mulZ$}}
  \end{subexenum}
  \exitem \label{ex:bool}The type of \define{booleans}\index{booleans|textbf}\index{bool@{$\bool$}|see {booleans}}\index{bool@{$\bool$}|textbf} is defined to be an inductive type $\bool$ that comes equipped with\index{booleans!true@{$\btrue$}|textbf}\index{booleans!false@{$\bfalse$}|textbf}\index{false@{$\bfalse$}|textbf}\index{true@{$\btrue$}|textbf}
  \begin{equation*}
    \bfalse : \bool\qquad\text{and}\qquad\btrue : \bool.
  \end{equation*}
  The induction principle\index{induction principle!of the booleans|textbf}\index{booleans!induction principle|textbf} of the booleans asserts that for any family of types $P(x)$ indexed by $x:\bool$, there is a term\index{ind_bool@{$\indbool$}|textbf}
  \begin{equation*}
    \indbool : P(\bfalse)\to \Big(P(\btrue)\to \prd{x:\bool}P(x)\Big)
  \end{equation*}
  for which the computation rules\index{computation rules!for the booleans|textbf}\index{booleans!computation rules|textbf}
  \begin{align*}
    \indbool(p_0,p_1,\bfalse) & \jdeq p_0 \\*
    \indbool(p_0,p_1,\btrue) & \jdeq p_1
  \end{align*}
  hold.
  \begin{subexenum}
  \item Construct the \define{boolean negation} function $\negbool:\bool\to\bool$\index{neg-bool@{$\negbool$}|textbf}\index{booleans!neg-bool@{$\negbool$}|textbf}.
  \item Construct the \define{boolean conjunction} operation $\blank\land\blank : \bool\to(\bool\to\bool)$.\index{boolean conjunction|textbf}\index{booleans!conjunction|textbf}
  \item Construct the \define{boolean disjunction} operation $\blank\lor\blank : \bool\to(\bool\to\bool)$.\index{boolean disjunction|textbf}\index{booleans!disjunction|textbf}
  \end{subexenum}
  \exitem Let $P$ and $Q$ be types. We will write $P\leftrightarrow Q$ for the type of \define{bi-implications}\index{bi-implication|textbf} ${(P\to Q)}\times {(Q\to P)}$. Use the fact that $\neg P$\index{negation} is defined as the type $P\to\emptyt$ of functions from $P$ to the empty type to give type theoretic proofs of the constructive tautologies in this exercise.\label{ex:dne-dec}
  \begin{subexenum}
  \item \label{ex:no-fixed-points-neg}Show that
    \begin{enumerate}
    \item $\neg(P\times \neg P)$
    \item $\neg(P\leftrightarrow \neg P)$.
    \end{enumerate}
  \item \label{ex:dn-monad}Construct the following maps in the structure of the \define{double negation monad}:
    \begin{enumerate}
    \item $P\to\neg\neg P$
    \item $(P\to Q)\to(\neg\neg P\to\neg\neg Q)$
    \item $(P\to \neg\neg Q)\to (\neg\neg P \to\neg\neg Q)$.
    \end{enumerate}
  \item Prove that the following double negations of classical laws hold:
    \begin{enumerate}
    \item $\neg\neg(\neg\neg P \to P)$
    \item $\neg\neg(((P\to Q)\to P)\to P)$
    \item $\neg\neg((P\to Q)+(Q\to P))$
    \item $\neg\neg(P+\neg P)$.
    \end{enumerate}
  \item \label{ex:dne-is-decidable}Show that
    \begin{enumerate}
    \item $(P+\neg P)\to(\neg\neg P\to P)$
    \item $\neg\neg(Q\to P)\leftrightarrow ((P+\neg P)\to (Q\to P))$.    
    \end{enumerate}
  \item Prove the following tautologies, showing that $\neg P$, $P\to\neg\neg Q$, and $\neg\neg P\times\neg\neg Q$ are \define{double negation stable}:
    \begin{enumerate}
    \item $\neg\neg\neg P \to \neg P$
    \item $\neg\neg(P \to \neg\neg Q)\to (P\to\neg\neg Q)$
    \item $\neg\neg((\neg\neg P)\times(\neg\neg Q))\to (\neg\neg P)\times(\neg\neg Q)$.
    \end{enumerate}
  \item Show that
    \begin{enumerate}
    \item $\neg\neg(P\times Q)\leftrightarrow (\neg\neg P)\times(\neg\neg Q)$
    \item $\neg\neg(P+Q)\leftrightarrow \neg (\neg P \times \neg Q)$
    \item $\neg\neg(P\to Q)\leftrightarrow (\neg\neg P\to\neg\neg Q)$.
    \end{enumerate}
  \end{subexenum}
\exitem \label{ex:lists}For any type $A$ we can define the type $\lst(A)$\index{list A@{$\lst(A)$}|see {lists in $A$}}\index{list A@{$lst(A)$}|textbf} of \define{lists}\index{lists in A@{lists in $A$}|textbf}\index{inductive type!lists of elements of A@{lists of elements of $A$}|textbf} of elements of $A$ as the inductive type with constructors\index{lists in A@{lists in $A$}!nil@{$\nil$}|textbf}\index{nil@{$\nil$}|textbf}\index{cons(a,l)@{$\cons(a,l)$}|textbf}\index{lists in A@{lists in $A$}!cons@{$\cons$}|textbf}
  \begin{align*}
    \nil & : \lst(A) \\*
    \cons & : A \to (\lst(A) \to \lst(A)).
  \end{align*}
  \begin{subexenum}
  \item Write down the induction principle and the computation rules for $\lst(A)$.\index{induction principle!of list A@{of $\lst(A)$}|textbf}\index{lists in A@{lists in $A$}!induction principle|textbf}
  \item Let $A$ and $B$ be types, suppose that $b:B$, and consider a binary operation $\mu:A\to (B \to B)$. Define a function\index{fold-list@{$\foldlist$}|textbf}\index{lists in A@{lists in $A$}!fold-list@{$\foldlist$}|textbf}
    \begin{equation*}
      \foldlist(\mu) : \lst(A)\to B
    \end{equation*}
    that iterates the operation $\mu$, starting with $\foldlist(\mu,\nil)\defeq b$.
  \item Define the operation
    \begin{equation*}
      \maplist : (A\to B) \to (\lst(A)\to\lst(B))
    \end{equation*}
    for any two types $A$ and $B$.
  \item Define a function $\lengthlist:\lst(A)\to\N$.\index{length-list@{$\lengthlist$}|textbf}\index{lists in A@{lists in $A$}!length-list@{$\lengthlist$}|textbf}
  \item Define the functions\index{sum-list@{$\sumlist$}|textbf}\index{lists in A@{lists in $A$}!sum-list@{$\sumlist$}|textbf}
    \begin{align*}
      \sumlist & : \lst(\N) \to \N \\
      \productlist & : \lst(\N)\to\N,
    \end{align*}
    where $\sumlist$ adds all the elements in a list of natural numbers, and $\productlist$ takes their product.
  \item Define a function\index{concat-list@{$\concatlist$}|textbf}\index{lists in A@{lists in $A$}!concat-list@{$\concatlist$}|textbf}\index{concatenation!of lists|textbf}
    \begin{equation*}
      \concatlist : \lst(A) \to (\lst(A) \to \lst(A))
    \end{equation*}
    that concatenates any two lists of elements in $A$.
  \item Define a function\index{flatten-list@{$\flattenlist$}|textbf}\index{lists in A@{lists in $A$}!flatten-list@{$\flattenlist$}|textbf}
    \begin{equation*}
      \flattenlist : \lst(\lst(A)) \to \lst(A)
    \end{equation*}
    that concatenates all the lists in a lists of lists in $A$.
  \item Define a function $\reverselist : \lst(A) \to \lst(A)$ that reverses the order of the elements in any list.\index{reverse-list@{$\reverselist$}|textbf}\index{lists in A@{lists in $A$}!reverse-list@{$\reverselist$}|textbf}
  \end{subexenum}
\end{exercises}

%%% Local Variables:
%%% mode: latex
%%% TeX-master: "hott-intro"
%%% End:

\section{Identity types}\label{sec:identity}

\index{identity type|(}
\index{inductive type!identity type|(}
From the perspective of types as proof-relevant propositions, how should we think of \emph{equality} in type theory? Given a type $A$, and two elements $x,y:A$, the equality $\id{x}{y}$ should again be a type. Indeed, we want to \emph{use} type theory to prove equalities. \emph{Dependent} type theory provides us with a convenient setting for this: the identity type $\id{x}{y}$ is dependent on $x,y:A$. 

Then, if $\id{x}{y}$ is to be a type, how should we think of the elements of $\id{x}{y}$. An element $p:\id{x}{y}$ witnesses that $x$ and $y$ are equal elements of type $A$. In other words $p:\id{x}{y}$ is an \emph{identification} of $x$ and $y$. In a proof-relevant world, there might be many elements of type $\id{x}{y}$. I.e., there might be many identifications of $x$ and $y$. And, since $\id{x}{y}$ is itself a type, we can form the type $\id{p}{q}$ for any two identifications $p,q:\id{x}{y}$. That is, since $\id{x}{y}$ is a type, we may also use the type theory to prove things \emph{about} identifications (for instance, that two given such identifications can themselves be identified), and we may use the type theory to perform constructions with them. As we will see in this section, we can give every type a groupoidal structure.

Clearly, the equality type should not just be any type dependent on $x,y:A$. Then how do we form the equality type, and what ways are there to use identifications in constructions in type theory? The answer to both these questions is that we will form the identity type as an \emph{inductive} type, generated by just a reflexivity identification providing an identification of $x$ to itself. The induction principle then provides us with a way of performing constructions with identifications, such as concatenating them, inverting them, and so on. Thus, the identity type is equipped with a reflexivity element, and further possesses the structure that are generated by its induction principle and by the type theory. This inductive construction of the identity type is elegant, beautifully simple, but far from trivial!

The situation where two elements can be identified in possibly more than one way is analogous to the situation in \emph{homotopy theory}, where two points of a space can be connected by possibly more than one \emph{path}. Indeed, for any two points $x,y$ in a space, there is a \emph{space of paths} from $x$ to $y$. Moreover, between any two paths from $x$ to $y$ there is a space of \emph{homotopies} between them, and so on. From \cref{chap:uf} on we will take full advantage of this idea in order to develop the univalent foundations of mathematics.

\subsection{The inductive definition of identity types}

\begin{defn}
  Consider a type $A$ and let $a:A$. Then we define the \define{identity type}\index{identity type|textbf} of $A$ at $a$ as an inductive family of types $a =_A x$\index{a = x@{$a = x$}|see {identity type}} indexed by $x:A$, of which the constructor is\index{refl@{$\refl{}$}|textbf}\index{identity type!refl@{$\refl{}$}|textbf}
  \begin{equation*}
    \refl{a}:a=_Aa.
  \end{equation*}
  The induction principle of the identity type\index{identity type!induction principle|textbf}\index{induction principle!of the identity type|textbf} postulates that for any family of types $P(x,p)$ indexed by $x:A$ and $p:a=_A x$, there is a function\index{path-ind@{$\pathind$}|textbf}\index{identity type!path-ind@{$\pathind$}|textbf}
  \begin{equation*}
    \pathind_a:P(a,\refl{a}) \to \prd{x:A}\prd{p:a=_A x} P(x,p)
  \end{equation*}
  that satisfies $\pathind_a(u,a,\refl{a})\jdeq u$, give $u:P(a,\refl{a})$.

  An element of type $a=_A x$ is also called an \define{identification}\index{identification|textbf}\index{identity type!identification|textbf} of $a$ with $x$, and sometimes it is called a \define{path}\index{path|textbf}\index{identity type!path|textbf} from $a$ to $x$.
The induction principle for identity types is sometimes called \define{identification elimination}\index{identification elimination|textbf}\index{induction principle!identification elimination|textbf}\index{identity type!identification elimination|textbf} or \define{path induction}\index{path induction|textbf}\index{identity type!path induction|textbf}\index{induction principle!path induction|textbf}. We also write $\idtypevar{A}$\index{Id A@{$\idtypevar{A}$}|see {identity type}}\index{Id A@{$\idtypevar{A}$}|textbf} for the identity type on $A$, and often we write $a=x$ for the type of identifications of $a$ with $x$, omitting reference to the ambient type $A$.
\end{defn}

\begin{rmk}
  We see that the identity type is not just an inductive type, like the inductive types $\N$, $\emptyt$, and $\unit$ for example, but it is an inductive \emph{family} of types. Even though we have a type $a=_A x$ for any $x:A$, the constructor only provides an element $\refl{a}:a=_A a$, identifying $a$ with itself. The induction principle then asserts that in order to prove something about all identifications of $a$ with some $x:A$, it suffices to prove this assertion about $\refl{a}$ only. We will see in the next sections that this induction principle is strong enough to derive many familiar facts about equality, namely that it is a symmetric and transitive relation, and that all functions preserve equality.
\end{rmk}

\begin{rmk}
  \index{rules!identity type|(}\index{identity type!rules|(}
  Since the identity types require getting used to, we provide the formal rules
  for identity types. The identity type is formed by the formation rule:
  \begin{prooftree}
    \AxiomC{$\Gamma\vdash a:A$}
    \UnaryInfC{$\Gamma,x:A\vdash a=_A x~\type$}
  \end{prooftree}
  The constructor of the identity type is then given by the introduction rule:
  \begin{prooftree}
    \AxiomC{$\Gamma\vdash a:A$}
    \UnaryInfC{$\Gamma\vdash \refl{a}:a=_A a$}
  \end{prooftree}
  The induction principle is now given by the elimination rule:
  \begin{prooftree}
    \AxiomC{$\Gamma\vdash a:A$}
    \AxiomC{$\Gamma,x:A,p:a=_A x\vdash P(x,p)~\type$}
    \BinaryInfC{$\Gamma\vdash \pathind_a:P(a,\refl{a})\to\prd{x:A}\prd{p:a=_A x}P(x,p)$}
  \end{prooftree}
  And finally the computation rule is:
  \begin{prooftree}
    \AxiomC{$\Gamma\vdash a:A$}
    \AxiomC{$\Gamma,x:A,p:a=_A x\vdash P(x,p)~\type$}
    \BinaryInfC{$\Gamma,u:P(a,\refl{a}) \vdash \pathind_a(u,a,\refl{a})\jdeq u : P(a,\refl{a})$}
  \end{prooftree}
  \index{rules!identity type|)}\index{identity type!rules|)}
\end{rmk}

\begin{rmk}
  One might wonder whether it is also possible to form the identity type at a \emph{variable} of type $A$, rather than at an element. This is certainly possible: since we can form the identity type in \emph{any} context, we can form the identity type at a variable $x:A$ as follows:
  \begin{prooftree}
    \AxiomC{$\Gamma,x:A\vdash x:A$}
    \UnaryInfC{$\Gamma,x:A,y:A\vdash x=_A y~\type$}
  \end{prooftree}
  In this way we obtain the `binary' identity type. Its constructor is then also indexed by $x:A$. We have the following introduction rule
  \begin{prooftree}
    \AxiomC{$\Gamma,x:A\vdash x:A$}
    \UnaryInfC{$\Gamma,x:A\vdash \refl{x}:x=_A x$}
  \end{prooftree}
  and similarly we have elimination and computation rules.
\end{rmk}

\subsection{The groupoidal structure of types}\label{sec:groupoid}
\index{groupoid laws!of identifications|(}
We show that identifications can be \emph{concatenated} and \emph{inverted}, which corresponds to the transitivity and symmetry of the identity type.

\begin{defn}\label{defn:id_concat}
Let $A$ be a type. We define the \define{concatenation}\index{concatenation!of identifications|textbf}\index{concat@{$\concat$}|textbf}\index{identity type!concatenation|textbf} operation
\begin{equation*}
\concat : \prd{x,y,z:A} (\id{x}{y})\to ((\id{y}{z})\to (\id{x}{z})).
\end{equation*}
We will write $\ct{p}{q}$ for $\concat(p,q)$.
\end{defn}

\begin{constr}
  We first construct a function
  \begin{equation*}
    f(x):\prd{y:A}(x=y)\to\prd{z:A}(y=z)\to(x=z)
  \end{equation*}
  for any $x:A$. By the induction principle for identity types, it suffices to construct
  \begin{equation*}
    f(x,x,\refl{x}):\prd{z:A} (x=z)\to(x=z).
  \end{equation*}
  Here we have the function $\lam{z}\idfunc[(x=z)]$. The function $f(x)$ we obtain via identity elimination is explicitly thus defined as
  \begin{equation*}
    f(x)\defeq\pathind_x(\lam{z}\idfunc):\prd{y:A} (x=y)\to \prd{z:A} (y=z)\to (x=z).
  \end{equation*}
  To finish the construction of $\concat$, we use \cref{ex:swap} to swap the order of the third and fourth variable of $f$, i.e., we define
  \begin{equation*}
    \concat_{x,y,z}(p,q):=f(x,y,p,z,q).\qedhere
  \end{equation*}
\end{constr}

\begin{defn}\label{defn:id_inv}
Let $A$ be a type. We define the \define{inverse operation}\index{inverse operation!for identifications|textbf}\index{inv@{$\invfunc$}|textbf}\index{identity type!inverse operation|textbf}
\begin{equation*}
\invfunc:\prd{x,y:A} (x=y)\to (y=x).
\end{equation*}
Most of the time we will write $p^{-1}$ for $\invfunc(p)$.
\end{defn}

\begin{constr}
By the induction principle for identity types, it suffices to construct
\begin{equation*}
\invfunc(\refl{x}): x=x,
\end{equation*}
for any $x:A$. Here we take $\invfunc(\refl{x})\defeq \refl{x}$.
\end{constr}

The next question is whether the concatenation and inverting operations on identifications behave as expected. More concretely: is concatenation of identifications associative, does it satisfy the unit laws, and is the inverse of an identification indeed a two-sided inverse?

For example, in the case of associativity we are asking to compare the identifications
\begin{equation*}
  \ct{(\ct{p}{q})}{r}\qquad\text{and}\qquad\ct{p}{(\ct{q}{r})}
\end{equation*}
for any $p:x=y$, $q:y=z$, and $r:z=w$ in a type $A$. The computation rules of the identity type are not strong enough to conclude that $\ct{(\ct{p}{q})}{r}$ and $\ct{p}{(\ct{q}{r})}$ are judgmentally equal. However, both $\ct{(\ct{p}{q})}{r}$ and $\ct{p}{(\ct{q}{r})}$ are elements of the same type: they are identifications of type $x=w$. Since the identity type is a type like any other, we can ask whether there is an \emph{identification}
\begin{equation*}
\ct{(\ct{p}{q})}{r}=\ct{p}{(\ct{q}{r})}.
\end{equation*}
This is a very useful idea: while it is often impossible to show that two elements of the same type are judgmentally equal, it may be the case that those two elements can be \emph{identified}. Indeed, we identify two elements by constructing an element of the identity type, and we can use all the type theory at our disposal in order to construct such an element. In this way we can show, for example, that addition on the natural numbers or on the integers is associative and satisfies the unit laws. And indeed, here we will show that concatenation of identifications is associative and satisfies the unit laws.

\begin{defn}\label{defn:id_assoc}
  Let $A$ be a type and consider three consecutive identifications
  \begin{equation*}
    \begin{tikzcd}
      x \arrow[r,equals,"p"] & y \arrow[r,equals,"q"] & z \arrow[r,equals,"r"] & w
    \end{tikzcd}
  \end{equation*}
  in $A$. We define the \define{associator}\index{associativity!of concatenation of identifications}
  \begin{equation*}
    \assoc(p,q,r) : \ct{(\ct{p}{q})}{r}=\ct{p}{(\ct{q}{r})}.
  \end{equation*}
\end{defn}

\begin{constr}
By the induction principle for identity types it suffices to show that
\begin{equation*}
\prd{z:A}\prd{q:x=z}\prd{w:A}\prd{r:z=w} \ct{(\ct{\refl{x}}{q})}{r}= \ct{\refl{x}}{(\ct{q}{r})}.
\end{equation*}
Let $q:x=z$ and $r:z=w$. Note that by the computation rule of identity types we have a judgmental equality $\ct{\refl{x}}{q}\jdeq q$. Therefore we conclude that
\begin{equation*}
  \ct{(\ct{\refl{x}}{q})}{r}\jdeq \ct{q}{r}.
\end{equation*}
Similarly we have a judgmental equality $\ct{\refl{x}}{(\ct{q}{r})}\jdeq \ct{q}{r}$. Thus we see that the left-hand side and the right-hand side in
\begin{equation*}
  \ct{(\ct{\refl{x}}{q})}{r}=\ct{\refl{x}}{(\ct{q}{r})}
\end{equation*}
are judgmentally equal, so we can simply define $\assoc(\refl{x},q,r)\defeq\refl{\ct{q}{r}}$.
\end{constr}

\begin{defn}\label{defn:id_unit}
Let $A$ be a type. We define the left and right \define{unit law operations}\index{unit laws!for concatenation of identifications}, which assigns to each $p:x=y$ the identifications\index{left unit@{$\leftunit$}|textbf}\index{right unit@{$\rightunit$}|textbf}
\begin{align*}
\leftunit(p) & : \ct{\refl{x}}{p}=p \\
\rightunit(p) & : \ct{p}{\refl{y}}=p,
\end{align*}
respectively.
\end{defn}

\begin{constr}
By identification elimination it suffices to construct
\begin{align*}
\leftunit(\refl{x}) & : \ct{\refl{x}}{\refl{x}} = \refl{x} \\
\rightunit(\refl{x}) & : \ct{\refl{x}}{\refl{x}} = \refl{x}.
\end{align*}
In both cases we take $\refl{\refl{x}}$.
\end{constr}

\begin{defn}\label{defn:id_invlaw}
Let $A$ be a type. We define left and right \define{inverse law operations}\index{inverse law operations!for identifications}\index{left inv@{$\leftinv$}|textbf}\index{right inv@{$\rightinv$}|textbf}
\begin{align*}
\leftinv(p) & : \ct{p^{-1}}{p} = \refl{y} \\
\rightinv(p) & : \ct{p}{p^{-1}} = \refl{x}.
\end{align*}
\end{defn}

\begin{constr}
By identification elimination it suffices to construct
\begin{align*}
\leftinv(\refl{x}) & : \ct{\refl{x}^{-1}}{\refl{x}} = \refl{x} \\
\rightinv(\refl{x}) & : \ct{\refl{x}}{\refl{x}^{-1}} = \refl{x}.
\end{align*}
Using the computation rules we see that
\begin{equation*}
\ct{\refl{x}^{-1}}{\refl{x}}\jdeq \ct{\refl{x}}{\refl{x}}\jdeq\refl{x},
\end{equation*}
so we define $\leftinv(\refl{x})\defeq \refl{\refl{x}}$. Similarly it follows from the computation rules that
\begin{equation*}
\ct{\refl{x}}{\refl{x}^{-1}} \jdeq \refl{x}^{-1}\jdeq \refl{x}
\end{equation*}
so we again define $\rightinv(\refl{x})\defeq\refl{\refl{x}}$. 
\end{constr}

\begin{rmk}
  We have seen that the associator, the unit laws, and the inverse laws, are all proven by constructing an identification of identifications. And indeed, there is nothing that would stop us from considering identifications of those identifications of identifications. We can go up as far as we like in the \emph{tower of identity types}\index{tower of identity types}\index{identity type!tower of identity types}, which is obtained by iteratively taking identity types.

  The iterated identity types give types in homotopy type theory a very intricate structure. One important way of studying this structure is via the homotopy groups of types, a subject that we will gradually be working towards.
\end{rmk}
\index{groupoid laws!of identifications|)}

\subsection{The action on identifications of functions}

\index{action on paths|(}
\index{identity type!action on paths|(}
Using the induction principle of the identity type we can show that every function preserves identifications.
In other words, every function sends identified elements to identified elements.
Note that this is a form of continuity for functions in type theory: if there is an identification that identifies two points $x$ and $y$ of a type $A$, then there also is an identification that identifies the values $f(x)$ and $f(y)$ in the codomain of $f$. 

\begin{defn}\label{defn:ap}
Let $f:A\to B$ be a map. We define the \define{action on paths}\index{function!action on paths|textbf}\index{identity type!action on paths|textbf}\index{action on paths|textbf} of $f$ as an operation\index{ap f@{$\apfunc{f}$}|see {action on paths}}\index{ap f@{$\apfunc{f}$}|textbf}
\begin{equation*}
\apfunc{f} : \prd{x,y:A} (\id{x}{y})\to(\id{f(x)}{f(y)}).
\end{equation*}
Moreover, there are operations\index{ap-id@{$\apid$}|textbf}\index{action on paths!ap-id@{$\apid$}|textbf}\index{ap-comp@{$\apcomp$}|textbf}\index{action on paths!ap-comp@{$\apcomp$}|textbf}
\begin{align*}
\apid_A & : \prd{x,y:A}\prd{p:\id{x}{y}} \id{p}{\ap{\idfunc[A]}{p}} \\
\apcomp(f,g) & : \prd{x,y:A}\prd{p:\id{x}{y}} \id{\ap{g}{\ap{f}{p}}}{\ap{g\circ f}{p}}.
\end{align*}
\end{defn}

\begin{constr}
First we define $\apfunc{f}$ by the induction principle of identity types, taking
\begin{equation*}
\apfunc{f}(\refl{x})\defeq \refl{f(x)}.
\end{equation*}
Next, we construct $\apid_A$ by the induction principle of identity types, taking
\begin{equation*}
\apid_A(\refl{x}) \defeq \refl{\refl{x}}.
\end{equation*}
Finally, we construct $\apcomp(f,g)$ by the induction principle of identity types, taking
\begin{equation*}
\apcomp(f,g,\refl{x}) \defeq \refl{\refl{g(f(x))}}.\qedhere
\end{equation*}
\end{constr}

\begin{defn}\label{defn:ap-preserve}
Let $f:A\to B$ be a map. Then there are identifications\index{ap-refl@{$\aprefl$}|textbf}\index{ap-inv@{$\apinv$}|textbf}\index{ap-concat@{$\apconcat$}|textbf}\index{action on paths!ap-refl@{$\aprefl$}|textbf}\index{action on paths!ap-inv@{$\apinv$}|textbf}\index{action on paths!ap-concat@{$\apconcat$}|textbf}
\begin{align*}
\aprefl(f,x) & : \id{\ap{f}{\refl{x}}}{\refl{f(x)}} \\
\apinv(f,p) & : \id{\ap{f}{p^{-1}}}{\ap{f}{p}^{-1}} \\
\apconcat(f,p,q) & : \id{\ap{f}{\ct{p}{q}}}{\ct{\ap{f}{p}}{\ap{f}{q}}}
\end{align*}
for every $p:\id{x}{y}$ and $q:\id{x}{y}$.
\end{defn}

\begin{constr}
To construct $\aprefl(f,x)$ we simply observe that ${\ap{f}{\refl{x}}}\jdeq {\refl{f(x)}}$, so we take
\begin{equation*}
\aprefl(f,x)\defeq\refl{\refl{f(x)}}.
\end{equation*}
We construct $\apinv(f,p)$ by identification elimination on $p$, taking
\begin{equation*}
\apinv(f,\refl{x}) \defeq \refl{\ap{f}{\refl{x}}}.
\end{equation*}
Finally we construct $\apconcat(f,p,q)$ by identification elimination on $p$, taking
\begin{equation*}
\apconcat(f,\refl{x},q)  \defeq \refl{\ap{f}{q}}.\qedhere
\end{equation*}
\end{constr}
\index{action on paths|)}
\index{identity type!action on paths|)}

\subsection{Transport}

\index{transport|(}
Dependent types also come with an action on identifications: the \emph{transport} functions.
Given an identification $p:\id{x}{y}$ in the base type $A$, we can transport any element $b:B(x)$ to the fiber $B(y)$.

\begin{defn}
Let $A$ be a type, and let $B$ be a type family over $A$.
We will construct a \define{transport}\index{transport|textbf}\index{type family!transport|textbf}\index{identity type!transport|textbf} operation\index{tr B@{$\tr_B$}|textbf}
\begin{equation*}
\tr_B:\prd{x,y:A} (\id{x}{y})\to (B(x)\to B(y)).
\end{equation*}
\end{defn}

\begin{constr}
We construct $\tr_B(p)$ by induction on $p:x=_A y$, taking
\begin{equation*}
\tr_B(\refl{x}) \defeq \idfunc[B(x)].\qedhere
\end{equation*}
\end{constr}

Thus we see that type theory cannot distinguish between identified elements $x$ and $y$, because for any type family $B$ over $A$ one obtains an element of $B(y)$ from the elements of $B(x)$.

As an application of the transport function we construct the \emph{dependent} action on paths\index{dependent action on paths|textbf} of a dependent function $f:\prd{x:A}B(x)$. Note that for such a dependent function $f$, and an identification $p:\id[A]{x}{y}$, it does not make sense to directly compare $f(x)$ and $f(y)$, since the type of $f(x)$ is $B(x)$ whereas the type of $f(y)$ is $B(y)$, which might not be exactly the same type. However, we can first \emph{transport} $f(x)$ along $p$, so that we obtain the element $\tr_B(p,f(x))$ which is of type $B(y)$. Now we can ask whether it is the case that $\tr_B(p,f(x))=f(y)$. The dependent action on paths of $f$ establishes this identification.

\begin{defn}\label{defn:apd}
Given a dependent function $f:\prd{a:A}B(a)$ and an identification $p:\id{x}{y}$ in $A$, we construct an identification\index{apd f@{$\apdfunc{f}$}|textbf}
\begin{equation*}
\apd{f}{p} : \id{\tr_B(p,f(x))}{f(y)}.
\end{equation*}
\end{defn}

\begin{constr}
The identification $\apd{f}{p}$ is constructed by the induction principle for identity types. Thus, it suffices to construct an identification
\begin{equation*}
\apd{f}{\refl{x}}:\id{\tr_B(\refl{x},f(x))}{f(x)}.
\end{equation*}
Since transporting along $\refl{x}$ is the identity function on $B(x)$, we simply take $\apd{f}{\refl{x}}\defeq\refl{f(x)}$. 
\end{constr}
\index{transport|)}

\subsection{The uniqueness of \texorpdfstring{$\refl{}$}{refl}}\label{sec:refl-unique}%

The identity type is an inductive \emph{family} of types. This has some subtle, but important implications. For instance, while the type $a=x$ indexed by $x:A$ is inductively generated by $\refl{a}$, the type $a=a$ is \emph{not} inductively generated by $\refl{a}$. Hence we cannot use the induction principle of identity types to show that $p=\refl{a}$ for any $p:a=a$. The obstacle, which prevents us from applying the induction principle of identity types in this case, is that the endpoint of $p:a=a$ is not free.

Nevertheless, the identity type $a=x$ is generated by a single element $\refl{a}:a=a$, so it is natural to wonder in what sense the reflexivity identification is unique. An identification with an element $a$ is specified by first giving the endpoint $x$ with which we seek to identify $a$, and then giving the identification $p:a=x$. It is therefore only the pair $(a,\refl{a})$ which is unique in the type of all pairs
\begin{equation*}
  (x,p):\sm{x:A}a=x.
\end{equation*}
We prove this fact in the following proposition.

\begin{prp}\label{prp:contraction-total-space-id}
  Consider an element $a:A$. Then there is an identification
  \begin{equation*}
    (a,\refl{a})=y
  \end{equation*}
  in the type $\sm{x:A}a=x$, for any $y:\sm{x:A}a=x$.
\end{prp}

\begin{proof}
  By $\Sigma$-induction it suffices to show that there is an identification
  \begin{equation*}
    (a,\refl{a})=(x,p)
  \end{equation*}
  for any $x:A$ and $p:a=x$. We proceed by the induction principle of identity types.
  Therefore it suffices to show that
  \begin{equation*}
    (a,\refl{a})=(a,\refl{a}).
  \end{equation*}
  We obtain such an identification by reflexivity.
\end{proof}

\cref{prp:contraction-total-space-id} shows that there is, up to identification, only one element in $\Sigma$-type of the identity type. Such types are called contractible, and they are the subject of \cref{sec:contractible}.

\subsection{The laws of addition on \texorpdfstring{$\N$}{ℕ}}\label{subsec:addN}

Now that we have introduced the identity type, we can start proving equations. We will prove here that there are identifications\index{unit laws!for addition on N@{for addition on $\N$}}\index{successor laws!for addition on N@{for addition on $\N$}}\index{associativity!of addition on N@{of addition on $\N$}}\index{commutativity!of addition on N@{of addition on $\N$}}\index{natural numbers!unit laws for addition}\index{natural numbers!successor laws for addition}\index{natural numbers!associativity of addition}\index{natural numbers!commutativity of addition}
\begin{align*}
  0+n & = n & m+0 & = m \\
  \succN(m)+n & = \succN(m+n) & m+\succN(n) & = \succN(m+n) \\
  (m+n)+k & = m+(n+k) & m+n & = n+m.
\end{align*}
The unit laws, associativity, and commutativity of addition are of course familiar. The successor laws will be useful to prove commutativity. In \cref{ex:semi-ring-laws-N} you will be asked to prove the laws of multiplication on $\N$. There will again be \emph{successor laws} as part of this exercise, because they are useful intermediate steps in the more complicated laws.

Recall that addition on the natural numbers is defined in such a way that
\begin{align*}
  m+0 & \jdeq m & m+\succN(n) & \jdeq \succN(m+n).
\end{align*}
These two judgmental equalities are all we currently know about the function $m,n\mapsto m+n$ on $\N$. Consequently, we will have to find ways to apply these two judgmental equalities in our proofs of the laws of addition. Of course, the judgmental equalities coincide with two of the six laws. For the remaining four laws, we will have to proceed by induction on $\N$.

\begin{prp}\label{prp:unit-laws-add-N}
  For any natural number $n$, there are identifications
  \begin{align*}
    \leftunitlawaddN(n) & : 0+n=n \\
    \rightunitlawaddN(n) & : n+0=n.
  \end{align*}
\end{prp}

\begin{proof}
  We can define
  \begin{equation*}
    \rightunitlawaddN(n)\defeq\refl{n},
  \end{equation*}
  because the computation rule for addition gives us that $n+0\jdeq n$.

  It remains to define the left unit law. We proceed by induction on $n$. In the base case we have to show that $0+0=0$, which holds by reflexivity. For the inductive step, assume that we have an identification $p:0+n=n$. Our goal is to show that $0+\succN(n)=\succN(n)$. However, it suffices to construct an identification
  \begin{equation*}
    \succN(0+n)=\succN(n),
  \end{equation*}
  because by the computation rule for addition we have that $0+\succN(n)\jdeq\succN(0+n)$. Now we use the action on paths of $\succN:\N\to\N$ to obtain
  \begin{equation*}
    \ap{\succN}{p}:\succN(0+n)=\succN(n).
  \end{equation*}
  The left unit law is therefore defined by
  \begin{equation*}
    \leftunitlawaddN(n)\defeq\indN(\refl{0},\lam{p}\ap{\succN}{p}).\qedhere
  \end{equation*}
\end{proof}

\begin{prp}\label{prp:successor-laws-add-N}
  For any natural numbers $m$ and $n$, there are identifications
  \begin{align*}
    \leftsuccessorlawaddN(m,n) & : \succN(m)+n=\succN(m+n) \\
    \rightsuccessorlawaddN(m,n) & : m+\succN(n)=\succN(m+n).
  \end{align*}
\end{prp}

\begin{proof}
  We can define
  \begin{equation*}
    \rightsuccessorlawaddN(m,n)\defeq\refl{\succN(m+n)}
  \end{equation*}
  because we have a judgmental equality $m+\succN(n)\jdeq\succN(m+n)$ by the computation rules for $\addN$.

  The left successor law is constructed by induction on $n$. In the base case we have to construct an identification $\succN(m)+0=\succN(m+0)$, which is obtained by reflexivity. For the inductive step, assume that we have an identification $p:\succN(m)+n=\succN(m+n)$. Our goal is to show that
  \begin{equation*}
    \succN(m)+\succN(n)=\succN(m+\succN(n)). 
  \end{equation*}
  Note that we have the judgmental equalities
  \begin{align*}
    \succN(m)+\succN(n) & \jdeq\succN(\succN(m)+n) \\
    \succN(m+\succN(n)) & \jdeq\succN(\succN(m+n))
  \end{align*}
  Therefore it suffices to construct an identification
  \begin{equation*}
    \succN(\succN(m)+n)=\succN(\succN(m+n)).
  \end{equation*}
  Such an identification is given by $\ap{\succN}{p}$.
\end{proof}

\begin{prp}
  Addition on the natural numbers is associative, i.e., for any three natural numbers $m$, $n$, and $k$, there is an identification
  \begin{equation*}
    \associativeaddN(m,n,k):(m+n)+k=m+(n+k).
  \end{equation*}
\end{prp}

\begin{proof}
  We construct $\associativeaddN(m,n,k)$ by induction on $k$. In the base case we have the judgmental equalities
  \begin{equation*}
    (m+n)+0\jdeq m+n\jdeq m+(n+0).
  \end{equation*}
  Therefore we define $\associativeaddN(m,n,0)\defeq\refl{m+n}$.

  For the inductive step, let $p:(m+n)+k=m+(n+k)$. Our goal is to show that
  \begin{equation*}
    (m+n)+\succN(k)=m+(n+\succN(k)).
  \end{equation*}
  Note that we have the judgmental equalities
  \begin{align*}
    (m+n)+\succN(k) & \jdeq \succN((m+n)+k) \\
    m+(n+\succN(k)) & \jdeq m+(\succN(n+k)) \\
                    & \jdeq \succN(m+(n+k))
  \end{align*}
  Therefore it suffices to construct an identification
  \begin{equation*}
    \succN((m+n)+k)=\succN(m+(n+k)),
  \end{equation*}
  which we have by $\ap{\succN}{p}$.
\end{proof}

\begin{prp}
  Addition on the natural numbers is commutative, i.e., for any two natural numbers $m$ and $n$ there is an identification
  \begin{equation*}
    \commutativeaddN(m,n) : m+n=n+m.
  \end{equation*}
\end{prp}

\begin{proof}
  We construct $\commutativeaddN(m,n)$ by induction on $m$. In the base case we have to show that $0+n=n+0$, which holds by the unit laws for $n$, proven in \cref{prp:unit-laws-add-N}.

  For the inductive step, let $p:m+n=n+m$. Our goal is to construct an identification $\succN(m)+n=n+\succN(m)$. Now it is clear why we first proved the successor laws: we compute
  \begin{align*}
    \succN(m)+n & = \succN(m+n) \\
                & = \succN(n+m) \\
                & \jdeq n+\succN(m).
  \end{align*}
  The first identification is obtained by \cref{prp:successor-laws-add-N}, and the second identification is the identification $\ap{\succN}{p}$.
\end{proof}

\begin{exercises}
  \exitem \label{ex:inv_assoc}Show that the operation inverting identifications distributes over the concatenation operation, i.e., construct an identification
  \index{distributivity!of inv over concat@{of $\invfunc$ over $\concat$}}
  \index{identity type!distributive-inv-concat@{$\distributiveinvconcat$}|textbf}
  \begin{align*}
    \distributiveinvconcat(p,q):\id{(\ct{p}{q})^{-1}}{\ct{q^{-1}}{p^{-1}}}.
  \end{align*}
  for any $p:\id{x}{y}$ and $q:\id{y}{z}$.
  \exitem \label{ex:inv_con}For any $p:x=y$, $q:y=z$, and $r:x=z$, construct maps
  \index{identity type!inv-con@{$\invcon$}|textbf}
  \index{inv-con@{$\invcon$}|textbf}
  \index{identity type!con-inv@{$\coninv$}|textbf}
  \index{con-inv@{$\coninv$}|textbf}
  \begin{align*}
    \invcon(p,q,r) & : (\ct{p}{q}=r)\to (q=\ct{p^{-1}}{r}) \\
    \coninv(p,q,r) & : (\ct{p}{q}=r)\to (p=\ct{r}{q^{-1}}).
  \end{align*}
  \exitem Let $B$ be a type family over $A$, and consider an identification $p:\id{a}{x}$ in $A$. Construct for any $b:B(a)$ an identification\index{lift@{$\lift$}|textbf}\index{identity type!lift@{$\lift$}|textbf}
  \begin{equation*}
    \lift_B(p,b) : \id{(a,b)}{(x,\tr_B(p,b))}.
  \end{equation*}
  In other words, an identification $p:x=y$ in the \emph{base type} $A$ \emph{lifts} to an identification in $\sm{x:A}B(x)$ for every element in $B(x)$, analogous to the path lifting property for fibrations in homotopy theory.
  \exitem Consider four consecutive identifications
  \begin{equation*}
    \begin{tikzcd}
      a \arrow[r,equals,"p"] & b \arrow[r,equals,"q"] & c \arrow[r,equals,"r"] & d \arrow[r,equals,"s"] & e
    \end{tikzcd}
  \end{equation*}
  in a type $A$. In this exercise we will show that the \define{Mac Lane pentagon}\index{Mac Lane pentagon|textbf}\index{identity type!Mac Lane pentagon|textbf} for identifications commutes.
  \begin{subexenum}
  \item Construct the five identifications $\alpha_1,\ldots,\alpha_5$ in the pentagon
    \begin{equation*}
      \begin{tikzcd}[column sep=-1.5em]
        &[-2em] \ct{(\ct{(\ct{p}{q})}{r})}{s} \arrow[rr,equals,"\alpha_4"] \arrow[dl,equals,swap,"\alpha_1"] & & \ct{(\ct{p}{q})}{(\ct{r}{s})} \arrow[dr,equals,"\alpha_5"] &[-2em] \\
        \ct{(\ct{p}{(\ct{q}{r})})}{s} \arrow[drr,equals,swap,"\alpha_2"] & & & & \ct{p}{(\ct{q}{(\ct{r}{s})})}, \\
        & & \ct{p}{(\ct{(\ct{q}{r})}{s})} \arrow[urr,equals,swap,"\alpha_3"]
      \end{tikzcd}
    \end{equation*}
    where $\alpha_1$, $\alpha_2$, and $\alpha_3$ run counter-clockwise, and $\alpha_4$ and $\alpha_5$ run clockwise.
  \item Show that
    \begin{equation*}
      \ct{(\ct{\alpha_1}{\alpha_2})}{\alpha_3} = \ct{\alpha_4}{\alpha_5}.
    \end{equation*}
  \end{subexenum}
  \exitem \label{ex:semi-ring-laws-N}In this exercise we show that the operations of addition and multiplication on the natural numbers satisfy the laws of a commutative \define{semi-ring}.%
  \index{semi-ring laws!for N@{for $\N$}}%
  \index{natural numbers!semi-ring laws}%
  \index{associativity!of multiplication on N@{of multiplication on $\N$}}%
  \index{unit laws!for multiplication on N@{for multiplication on $\N$}}%
  \index{commutativity!of multiplication on N@{of multiplication on $\N$}}%
  \index{distributivity!of mulN over addN@{of $\mulN$ over $\addN$}}%
  \index{natural numbers!associativity of multiplicatoin@{associativity of multiplication}}
  \index{natural numbers!unit laws for multiplication}
  \index{natural numbers!zero laws for multiplication}
  \index{natural numbers!commutativity of multiplication}
  \index{natural numbers!distributivity of multiplication over addition}
  \begin{subexenum}
  \item Show that multiplication satisfies the following laws:
    \begin{align*}
      m\cdot 0 & = 0 & m\cdot 1 & = m & m\cdot \succN(n) & = m+m\cdot n \\
      0\cdot m & = 0 & 1\cdot m & = m & \succN(m)\cdot n & = m\cdot n+n.
    \end{align*}
  \item Show that multiplication on $\N$ is commutative:
    \begin{equation*}
      m\cdot n=n\cdot m.
    \end{equation*}
  \item \label{ex:distributive-mul-addN}Show that multiplication on $\N$ distributes over addition from the left and from the right, i.e., show that we have identifications
    \begin{align*}
      m\cdot (n+k) & = m\cdot n + m\cdot k \\
      (m+n)\cdot k & = m\cdot k + n\cdot k.
    \end{align*}
  \item Show that multiplication on $\N$ is associative:
    \begin{align*}
      (m\cdot n)\cdot k & = m\cdot (n\cdot k).
    \end{align*}
  \end{subexenum}
  \exitem \label{ex:is-equiv-succ-Z}Show that
  \begin{equation*}
    \succZ(\predZ(k))=k \qquad\text{and}\qquad \predZ(\succZ(k))=k
  \end{equation*}
  for any $k:\Z$, where $\predZ$ is the predecessor function on the integers, defined in \cref{ex:int_pred}.
  \exitem \label{ex:int_group_laws}\index{integers!group laws} In this exercise we will show that the laws for abelian groups hold for addition on the integers, using the group operations on $\Z$ defined in \cref{ex:int_group_ops}.
  \begin{subexenum}
  \item Show that addition satisfies the left and right unit laws, i.e., show that\index{unit laws!for addition on Z@{for addition on $\Z$}}\index{integers!unit laws for addition}
    \begin{align*}
      0+x & = x \\
      x+0 & = x.
    \end{align*}
  \item Show that the following successor and predecessor laws hold for addition on $\Z$.
    \begin{align*}
      \predZ(x)+y & = \predZ(x+y) & \succZ(x)+y & = \succZ(x+y) \\
      x+\predZ(y) & = \predZ(x+y) & x+\succZ(y) & = \succZ(x+y).
    \end{align*}
  \item Use part (b) to show that addition on the integers is associative and commutative, show that\index{associativity!of addition on Z@{of addition on $\Z$}}\index{commutativity!of addition on Z@{of addition on $\Z$}}\index{integers!associativity of addition}\index{integers!commutativity of addition}
    \begin{align*}
      (x+y)+z & = x + (y+z) \\
      x+y & = y+x.
    \end{align*}
  \item Show that addition satisfies the left and right inverse laws:\index{inverse laws!for addition on Z@{for addition on $\Z$}}\index{integers!inverse laws for addition}
    \begin{align*}
      (-x)+x & =0 \\
      x+(-x) &=0.
    \end{align*}
  \end{subexenum}
  \exitem \label{ex:ring-Z}In this exercise we will show that $\Z$ satisfies the axioms of a \define{ring}\index{ring!integers}\index{integers!is a ring}, using the multiplication operation defined in \cref{ex:mulZ}.
  \begin{subexenum}
  \item Show that multiplication on $\Z$ satisfies the following laws for $0$ and $1$\index{zero laws!for mulZ@{for $\mulZ$}}\index{unit laws!for multiplication on Z@{for multiplication on $\Z$}}\index{mul Z@{$\mulZ$}!unit laws}\index{mul Z@{$\mulZ$}!zero laws}\index{integers!zero laws for multiplication}\index{unit laws for multiplication}:
    \begin{align*}
      0\cdot x & = 0 & 1\cdot x & = x \\
      x\cdot 0 & = 0 & x\cdot 1 & = x.
    \end{align*}
  \item Show that multiplication on $\Z$ satisfies the predecessor and successor laws\index{mul Z@{$\mulZ$}!predecessor laws}\index{mul Z@{$\mulZ$}!successor laws}\index{integers!successor laws for addition}\index{integers!predecessor laws for addition}:
    \begin{align*}
      \predZ(x)\cdot y & = x\cdot y-y & \succZ(x)\cdot y & = x\cdot y + y \\
      x\cdot \predZ(y) & = x\cdot y-x & y\cdot \succZ(y) & = x\cdot y + x.
    \end{align*}
  \item Show that multiplication on $\Z$ distributes over addition, both from the left and from the right\index{mul Z@{$\mulZ$}!distributive over addZ@{distributive over $\addZ$}}\index{distributivity!of mulZ over addZ@{of $\mulZ$ over $\addZ$}}\index{integers!distributivity of multiplication over addition}:
    \begin{align*}
      x\cdot(y+z) & = x\cdot y+ x\cdot z \\
      (x+y)\cdot z & = x\cdot z + y\cdot z.
    \end{align*}
  \item Show that multiplication on $Z$ is associative and commutative\index{associativity!of multiplication on Z@{of multiplication on $\Z$}}\index{mul Z@{$\mulZ$}!associativity}\index{commutativity!of multiplication on Z@{of multiplication on $\Z$}}\index{mul Z@{$\mulZ$}!commutativity}\index{integers!associativity of multiplication}\index{integers!commutativity of multiplication}:
    \begin{align*}
      (x\cdot y)\cdot z & = x\cdot (y\cdot z) \\
      x\cdot y & = y\cdot x.
    \end{align*}
  \end{subexenum}
\end{exercises}

\index{identity type|)}
\index{inductive type!identity type|)}
\index{inductive type|)}

%%% Local Variables:
%%% mode: latex
%%% TeX-master: "hott-intro"
%%% End:

\section{Universes in model categories}
\label{sec:univalence}

Now let \E be a model category and \Fib the full \nfs determined by its fibrations, so that $\Fibka = \Fib\times_\cE \cEka$ denotes the relatively \ka-presentable fibrations.
In an ideal world, the object $\Fibka\in\Ehat$ would be (pseudonaturally equivalent to) a representable presheaf $\E(-,U)$.
But since \E is itself a 1-category rather than an \io-category, this is unreasonable to expect. % (unless all fibrations are monomorphisms, as in the case of the subobject classifier in a 1-topos).

Instead, we will replace \Fibka by a representable presheaf that is ``weakly equivalent'' in some sense.
We do not have a model structure on $\Ehat$ with which to make sense of this, but we can at least use the Yoneda embedding $\E\to\Ehat$ to lift the weak factorization systems of \E.

\begin{defn}\label{defn:afib}
  Let \E be a model category.
  A morphism $\dX\to\dY$ in $\Ehat$ is an \textbf{acyclic fibration} if it has the right lifting property (\cref{defn:2liftorth}) for all morphisms $\E(-,j) : \E(-,A) \to \E(-,B)$, where $j:A\to B$ is a cofibration in \E.
\end{defn}

\begin{rmk}
  If $f:\dX\to\dY$ is a representable morphism, then it is an acyclic fibration in the sense of \cref{defn:afib} if and only if in any pullback
  \begin{equation*}
    \begin{tikzcd}
      \E(-,W) \ar[r] \ar[d] \ar[dr,phantom,near start,"\lrcorner"] & \dX\ar[d,"{f}"] \\
      \E(-,Z) \ar[r] & \dY
    \end{tikzcd}
  \end{equation*}
  the induced map $W\to Z$ is an acyclic fibration in \E.
  This fits a standard pattern for extending pullback-stable properties of morphisms in \E to properties of representable morphisms in \Ehat.
  
  Note that this notion of (acyclic) fibration based on the model structure of \E is unrelated to the 2-categorical notions of (strict, discrete) fibration defined in \cref{sec:2cat}.
\end{rmk}

\begin{defn}
  If \F is a \nfs on \E, a \textbf{universe} for \F is a cofibrant object $U\in\E$ equipped with an acyclic fibration $\E(-,U) \to \F$ in \Ehat.
\end{defn}

That is, a universe is a sort of ``cofibrant replacement'' of $\Fibka$.
(This perspective was introduced informally in~\cite[\sect 3]{shulman:elreedy}.)

\begin{rmk}\label{rmk:universe}
  When $U$ is a universe, the morphism $\E(-,U) \to \F$ corresponds by the Yoneda lemma to an \F-algebra $\pi:\Util\to U$.
  And the fact that the morphism $\E(-,U) \to \F$ is an acyclic fibration means that given the solid arrows below, where $i:A\mono B$ is a cofibration, $f:X\to B$ is an \F-algebra, and both squares of solid arrows are pullbacks and \F-morphisms:
  \begin{equation*}
    \begin{tikzcd}[row sep=small, column sep=small]
      i^*(X) \arrow[rr] \arrow[dd, "g"'] \arrow[rd] &  & \Util \arrow[dd, "\pi"] \\
      & X \arrow[ru, dashed] &  \\
      A \arrow[rd, "i"', tail] \arrow[rr, near start, "h" description] &  & U \\
      & B \arrow[ru, dashed] \arrow[from=uu, near start, "f" description, crossing over] & 
    \end{tikzcd}%\label{eq:u2p}
  \end{equation*}
  there exist the dashed arrows rendering the diagram commutative and the third square also a pullback and an \F-morphism.
  This property of a universe was first noted in the proof of~\cite[Theorem 2.2.1]{klv:ssetmodel} and isolated more abstractly in~\cite[(2$'$)]{shulman:elreedy}, \cite[Corollary 3.11]{cisinski:elegant}, and~\cite{stenzel:thesis} under varying names.

  If the initial object $\emptyset$ is strict (i.e.\ every morphism with codomain $\emptyset$ is an isomorphism) and $\id_\emptyset$ has a unique \F-structure, this implies that every \F-algebra with cofibrant codomain is a pullback of $\pi$ (though not in a unique way).
\end{rmk}

As usual, when \E is cofibrantly generated we can hope to produce such a cofibrant replacement by the small object argument.
However, the colimits in \E used to build cell complexes are no longer colimits in \Ehat; thus we have to restrict to the objects of \Ehat that preserve these particular colimits.

\begin{defn}
  Let \E be a model category.
  We say $\dX\in\Ehat$ is a \textbf{stack for cell complexes} if as a pseudofunctor $\dX:\E\op\to\cGPD$ it preserves (in the weak bicategorical sense) coproducts, pushouts of cofibrations, and transfinite composites of cofibrations.
\end{defn}

\begin{eg}\label{eg:topos-stack}
  If \E is a Grothendieck 1-topos and all cofibrations are monomorphisms, then the trivial \nfs \cE is a stack for cell complexes.
  This because any topos is infinitary extensive~\cite{clw:ext-dist}, adhesive~\cite{ls:adhesive,ls:topadh}, and exhaustive~\cite{nlab:exhaustive,shulman:elreedy}; see~\cite[\sect3]{shulman:elreedy} and~\cite[Lemma 7.5]{sattler:eqvext}.
  More generally,
  \cE being a stack for cell complexes is one of the conditions for the (cofibration, acyclic fibration) weak factorization system of \E to be \emph{suitable} as in~\cite[Definition 3.2]{sattler:eqvext}.
\end{eg}

\begin{lem}\label{thm:nfs-stack}
  If \cE is a stack for cell complexes and $\phi:\F\to\cE$ is a \local \nfs, then \F is also a stack for cell complexes.
\end{lem}
\begin{proof}
  Let \sQ be the class of morphisms $q:\Yhat \to \E(-,Y)$ from \cref{eg:colim-orth} where $Y= \colim_i Y_i$ ranges over coproducts, pushouts of cofibrations, and transfinite composites of cofibrations.
  Since \cE is a stack for cell complexes, $\sQ \perp \cE$; and since \F is \local, by \cref{thm:local} we have $\sQ\perp\phi$.
  Hence $\sQ\perp \F$.
\end{proof}

\begin{lem}\label{thm:icell-acyc}
  If \E is cofibrantly generated with \cI a set of generating cofibrations, \dX and \dY are stacks for cell complexes, and $f:\dX\to\dY$ has the right lifting property in \Ehat against all morphisms $\E(-,j):\E(-,A)\to\E(-,B)$ where $j:A\to B$ is in \cI, then $f$ is an acyclic fibration.
\end{lem}
\begin{proof}
  As usual, any cofibration is a retract of a transfinite composite of pushouts of coproducts of elements of \cI.
  Thus, given that \dX and \dY are stacks for cell complexes, the lifting property carries through all these operations in the usual way.
\end{proof}

We call a pseudofunctor $\dZ\in\Ehat$ \textbf{small-groupoid-valued} if each groupoid $\dZ(A)$ is essentially small.
Note that by definition, for any \nfs \F the map $\phi:\F\to\cE$ has small \emph{fibers}, i.e.\ any given morphism $f:X\to Y$ has a small set of \F-structures; but \F is only small-groupoid-valued if any given object $Y\in\E$ there is a small set of isomorphism classes of \F-algebras with codomain $Y$.
In general we will achieve this by considering $\Fka = \dF\times_\cE \cEka$ as in \cref{sec:relpres}.

\begin{thm}\label{thm:2cat-soa}
  Let \E be a combinatorial model category, and $\dZ\in\Ehat$ a small-groupoid-valued stack for cell complexes.
  Then any morphism $f:\E(-,X) \to\dZ$ in \Ehat factors, up to isomorphism, as $\E(-,X) \xto{\E(-,j)} \E(-,Y) \xto{p} \dZ$, where $j$ is a cofibration in \E and $p$ is an acyclic fibration in \Ehat.
\end{thm}
\begin{proof}
  This is just a bicategorical adaptation of the small object argument.
  Let \cI be a set of generating cofibrations for \E; we will define an \cI-cell complex sequence $X_0 \to X_1 \to \cdots$ in \E, along with maps $f_n : \E(-,X_n) \to \dZ$ and coherent isomorphisms $f_n \circ j_{m,n} \cong f_m$.
  We start with $X_0 = X$ and $f_0 = f$.
  For limit $n$ we let $X_n = \colim_{m<n} X_m$, with $f_n : \E(-,X_n) \to \dZ$ and attendant isomorphisms induced by the fact that \dZ preserves this colimit.

  At a successor stage $n+1$, we let $S_n$ be a set of representatives for isomorphism classes of pseudo-commutative squares
  \[
    \begin{tikzcd}
      \E(-,A) \ar[r] \ar[d,"i"'] \ar[dr,phantom,"\scriptstyle\Downarrow\cong"] & \E(-,X_n) \ar[d] \\
      \E(-,B) \ar[r] & \dZ
    \end{tikzcd}
  \]
  where $i\in \cI$.
  This is a small set, since \dZ is small-groupoid-valued and \E is locally small.
  Now let $X_{n+1}$ be the pushout
  \[
    \begin{tikzcd}
      \coprod_{s\in S_n} A_s \ar[r] \ar[d] \drpushout & X_n \ar[d]\\
      \coprod_{s\in S_n} B_s \ar[r] & X_{n+1}
    \end{tikzcd}
  \]
  Since \dX preserves these coproducts and pushouts, there is an essentially unique induced map $f_{n+1} : X_{n+1}\to Z$ with attendant isomorphisms.

  Finally, since \E is locally presentable, there is a regular cardinal \la such that all domains of morphisms in \cI are \la-presentable.
  Thus, in any square
  \[
    \begin{tikzcd}
      \E(-,A) \ar[r] \ar[d,"i"'] \ar[dr,phantom,"\scriptstyle\Downarrow\cong"] & \E(-,X_\la) \ar[d] \\
      \E(-,B) \ar[r] & \dZ
    \end{tikzcd}
  \]
  the top morphism $A\to X_\la$ factors through $X_n$ for some $n<\la$, and hence there is a lift $B\to X_{n+1} \to X_\la$.
  Therefore, the map $f_{\la} : X_\la \to \dZ$ has right lifting for \cI, and is thus an acyclic fibration by \cref{thm:icell-acyc}.
\end{proof}

\begin{cor}\label{thm:nfs-universe}
  If \E is a Grothendieck 1-topos with a combinatorial model structure in which all cofibrations are monomorphisms, then any small-groupoid-valued \local \nfs \F on \E has a universe.
\end{cor}
\begin{proof}
  By \cref{eg:topos-stack,thm:nfs-stack}, \F is a stack for cell complexes; thus we can apply \cref{thm:2cat-soa} to factor the map $\E(-,\emptyset) \to \F$.
\end{proof}

In some cases such as \cref{eg:pshf-can,eg:rep-cod}, \Fib is \local and hence so is \Fibka.
This includes the universes constructed in~\cite{klv:ssetmodel,shulman:elreedy,cisinski:elegant}.
However, in the general case we need a different approach: we will suppose given a non-full \nfs \F that \emph{is} \local, and an acyclic fibration $\F\to\Fib$. Thus we will be able to apply \cref{thm:nfs-universe} to \F instead.

More generally, for a \nfs \F, let $\uly\F$ denote the image of the map $\phi:\F\to \cE$.
Thus $\uly\F$ is a full \nfs (though not generally \local, even if \F is), and the $\uly\F$-algebras are the morphisms that admit some \F-structure.

\begin{defn}\label{defn:stratified}
  A \nfs \F on a model category \E is \textbf{\stratified} if the map $\F \to \uly\F$ is an acyclic fibration.
  That is, for any pullback
  \[
    \begin{tikzcd}
      X' \ar[d,"f'"'] \ar[r,"g"] \ar[dr,phantom, near start,"\lrcorner"] & X \ar[d,"f"]\\
      Y' \ar[r,"i",tail] & Y
    \end{tikzcd}
  \]
  with $f$ and $f'$ \F-algebras and $i$ a cofibration, there exists a new \F-structure on $f$ making the square an \F-morphism.
\end{defn}

\begin{prop}\label{thm:pre-u2p}
  Let \F be a \local, \stratified, small-groupoid-valued \nfs on a Grothendieck 1-topos that is a combinatorial model category whose cofibrations are monomorphisms.
  Then $\uly\F$ has a universe.
\end{prop}
\begin{proof}
  Apply \cref{thm:nfs-universe} to \F, and observe that the composite $\E(-,U) \to \F \to \uly\F$ of acyclic fibrations is again an acyclic fibration.
\end{proof}

\begin{rmk}
  By \cref{rmk:universe}, if all objects are cofibrant, $\emptyset$ is strict, and $U$ is a universe for a full \nfs \F such that $\id_\emptyset\in\F$, then in fact $\F = \uly{\dRep_\pi}$ (with $\dRep_\pi$ as in \cref{eg:rep-fcos}).
  Conversely, if \E is locally cartesian closed and $\pi:\Util\to U$ is a universe for $\uly{\dRep_\pi}$, then $\dRep_\pi$ is \local (by \cref{eg:rep-local}), \stratified, and small-groupoid-valued.
  Thus the hypotheses of \cref{thm:pre-u2p} are basically optimal.
\end{rmk}

\begin{eg}
  Full \nfss are always \stratified, as is $\F_1 \times_\cE \F_2$ if $\F_1$ and $\F_2$ are.
\end{eg}

\begin{eg}
  If \F is a \stratified \nfs on $\E_2$ and $G:\E_1\to\E_2$ preserves pullbacks (hence also monomorphisms), then $G^{-1}(\F)$ is also \stratified.
\end{eg}

\begin{eg}\label{thm:sec-afib-strat}
  Let \E be a model category and $H$ be a fibred core-endofunctor of \E such that whenever $H_Y(X)\to Y$ has a section, it is an acyclic fibration.
  Then the \nfs $H^*(\cEp)$ is \stratified.
  For given a pullback square
  \begin{equation}
    \begin{tikzcd}
      X' \ar[d,"f'"'] \ar[r,"g"] \ar[dr,phantom, near start,"\lrcorner"] & X \ar[d,"f"]\\
      Y' \ar[r,"i",tail] & Y
    \end{tikzcd}\label{eq:sec-strat-sq}
  \end{equation}
  with $i$ a cofibration, along with sections $s'$ and $s$ of $H_{Y'}(X')\to Y'$ and $H_Y(X)\to Y$, the assumption implies $H_Y(X)\to Y$ is an acyclic fibration.
  Thus we can find a lift in the square
  \[
    \begin{tikzcd}
      Y' \ar[d,"i"',tail] \ar[r,"s"] & H_{Y'}(X') \ar[r,"{H(g,i)}"] & H_Y(X) \ar[d,"\sim",two heads]\\
      Y \ar[rr,equals] & & Y
    \end{tikzcd}
  \]
  giving an $H^*(\cEp)$-structure on $f$ making~\eqref{eq:sec-strat-sq} an $H^*(\cEp)$-morphism.
\end{eg}

Recall that a \textbf{Cisinski model category}~\cite{cisinski:topos,cisinski:local-acyc} is a Grothendieck 1-topos with a combinatorial model structure whose cofibrations are \emph{precisely} the monomorphisms.

\begin{prop}\label{eg:cof-ff-fcos}
  Let \E be a Cisinski model category, \F a \stratified \nfs on \E, and $E$ a cartesian functorial factorization on \E that factors every \F-algebra as an acyclic cofibration followed by a fibration.
  Then the \nfs $\F\times_\cE \dR_E$ (where $\dR_E$ is as in \cref{eg:ff-fcos}) is also \stratified.
\end{prop}
\begin{proof}
  Suppose given the pullback square on the left:
  \[
    \begin{tikzcd}
      X' \ar[d,"f'"'] \ar[r,"g"] \ar[dr,phantom, near start,"\lrcorner"] & X \ar[d,"f"]\\
      Y' \ar[r,"i",tail] & Y\\
      \phantom{E f}
    \end{tikzcd}
    \hspace{2cm}
    \begin{tikzcd}
      X' \ar[r,"g"] \ar[d,tail,"{\lambda_{f'}}"'] \ar[dr,phantom,near end,"\ulcorner"] & X \ar[ddr,tail,"{\lambda_f}"] \ar[d] \\
      E f' \ar[r] \ar[drr,tail,"{\fact g i}"'] & P \ar[dr,"j" description]\\
      && E f.
    \end{tikzcd}
  \]
  where $f$ and $f'$ are \F-algebras with $\dR_E$-structures $r_{f} : E f \to X$ and $r_{f'} : E f' \to X'$.
  Since \F is \stratified, $f$ has a new \F-structure making $(g,i)$ an \F-morphism; so it remains to find a new $\dR_E$-structure $\rtil_{f} : E f \to X$ (so that $f \circ \rtil_f = \rho_f$ and $\rtil_f \circ \lambda_f = \id_X$) such that $(g,i)$ is also an $\dR_E$-morphism, i.e.\ $\rtil_f \circ \fact g i = g \circ r_{f'}$.

  Define $P$ and $j$ by the pushout as on the right above.
  Since $f$ and $f'$ are \F-algebras, $\lambda_{f}$ and $\lambda_{f'}$ are acyclic cofibrations, and in particular monomorphisms.
  By cartesianness, $\fact g i$ is also a monomorphism (being a pullback of $i$) and $X'\cong E f' \times_{E f}X$.
  Thus, $P$ is a union of subobjects of $E f$ in the 1-topos \E, hence $j:P\to E f$ is also a monomorphism.
  Moreover, since $X\to P$ is a pushout of $\lambda_{f'}$, it is also an acyclic cofibration; hence by 2-out-of-3 $j$ is also acyclic.

  Now since $f$ is an \F-algebra, $\rho_f$ is a fibration.
  But $f$ is a retract of $\rho_f$ (by its $\dR_E$-structure $r_f$), hence also a fibration.
  Thus we can find a lift in the square:
  \[
    \begin{tikzcd}
      P \ar[d,"j"'] \ar[r] & X \ar[d,"f"] \\
      E f \ar[r,"{\rho_f}"'] \ar[ur,dotted,"{\rtil_f}" description] & Y
    \end{tikzcd}
  \]
  where the top arrow is induced by $g \circ r_{f'}$ and $\id_X$.
  Such a lift is then an $\dR_E$-structure on $f$ such that $(g,i)$ is an $\dR_E$-morphism, as desired.
\end{proof}

It remains to ensure that our universes are fibrant and univalent.

\begin{defn}\label{defn:nfs-hoinvar}
  Let \E be a locally presentable category with a model structure.
  A \nfs \F is \textbf{homotopy invariant} if every \F-algebra is a fibration, and given any commutative square
  \[
    \begin{tikzcd}
      X' \ar[d,"f'"',two heads] \ar[r,"\sim"] & X \ar[d,"f",two heads]\\
      Y' \ar[r,"\sim"'] & Y
    \end{tikzcd}
  \]
  where $f$ and $f'$ are fibrations and the horizontal maps are weak equivalences, $f$ admits an \F-structure if and only if $f'$ does.
\end{defn}

Of course, if $\uly\F=\Fib$ is the class of all fibrations, then \F is homotopy invariant.
More generally, homotopy invariance is a condition only on $\uly\F$.

We now recall the fundamental ``equivalence extension'' property.
To my knowledge, a version of this property first appeared in~\cite[Theorem 3.4.1]{klv:ssetmodel} in the case of simplicial sets.
It was observed in~\cite[Theorem 3.1]{shulman:elreedy} and~\cite[Remark 2.19]{cisinski:elegant} that the proof generalizes to any simplicial Cisinski model category.
A similar construction for cubical sets appeared under the name ``gluing'' in~\cite{cchm:cubicaltt}, which was then placed in a more abstract setting by~\cite{sattler:eqvext}.

\begin{thm}\label{thm:u4p}
  Let $\E$ be a simplicial Cisinski model category and \F a homotopy invariant \nfs on \E.
  Then there is a \la such that for any $\ka\shgt\la$ and any cofibration $i: A\cof B$, relatively \ka-presentable $\uly\F$-algebras $D_2 \fib B$ and $E_1 \fib A$, and weak equivalence $w: E_1 \toiso E_2$ over $A$, where $E_2 \coloneqq i^* D_2$:
%  Suppose given also a map $g:C\to D_2$ over $B$ and a factorization of $i^*(g) : i^* C \to E_2$ as $w\circ k$ for some $k:i^*(C) \to E_1$, filling out the solid arrows below:
  \begin{equation}
  \begin{tikzcd}[column sep=small]
      % & i^*C \ar[dl,"k"'] \ar[rrr] &&& C \ar[dl,dashed,"h"] \ar[ddr,"g"] \\
      E_1 \arrow[rdd, two heads] \arrow[rrd, "w"] \arrow[rrr, dashed] &  &  & D_1 \arrow[rdd, two heads, dashed] \arrow[rrd, "v", dashed] &  &  \\
      &  & \mathllap{E_2=\;}i^* D_2 \arrow[ld, two heads] \arrow[rrr,crossing over]
      % \ar[from=uul, crossing over]
      &  &  & D_2 \arrow[ld, two heads] \\
      & A \arrow[rrr, tail,"i"'] &  &  & B & 
    \end{tikzcd}\label{eq:u4p}
  \end{equation}
  there exists a relatively \ka-presentable $\uly\F$-algebra $D_1\fib B$ and an equivalence $v: D_1 \toiso D_2$ over $B$ such that $i^*(v)=w$.
  % If \F is \stratified, we can choose the \F-structure of $D_1$ so that the back square is an \F-morphism.
  %, and a map $h:C\to D_1$ with $i^*(h) = k$ and $v \circ h = g$.
\end{thm}
\begin{proof}
  Largely identical to that of~\cite[Theorem 3.1]{shulman:elreedy}.
  The latter statement assumes that \E is a presheaf category, but this is only used to obtain a notion of ``\ka-small morphism'' that is preserved by $i_*$ and by pullback; using \cref{thm:pres-pb,thm:relpres-mono-dp} instead allows \E to be any Grothendieck 1-topos.\footnote{The author claimed in~\cite{shulman:elreedy} that when $\E=\prcs$ it suffices to take $\ka > \card\C$, but Raffael Stenzel has pointed out that this is not enough to ensure that $i_*$ preserves \ka-small morphisms; even in the presheaf case we need some analogue of the relation $\shlt$.}
  (The proof uses that \E is a simplicial model category and has effective unions, to extend deformation retractions along $i$.)
  We conclude $D_1$ is an $\uly\F$-algebra by homotopy invariance, since it is equivalent to $D_2$ over $B$.
  % The application of \stratification is immediate.
\end{proof}

Univalence of our universes will follow from \cref{thm:u4p} as in~\cite{klv:ssetmodel,shulman:elreedy,cisinski:elegant}.
To show that $U$ is a fibrant object,~\cite[Theorem 2.2.1]{klv:ssetmodel} and~\cite[Proposition 2.21]{cisinski:elegant} use minimal fibrations, while~\cite[Lemma 6.3]{shulman:elreedy} uses a Reedy induction; but in fact fibrancy of $U$ is almost immediate from \cref{thm:u4p}.
A similar fact in the restricted situation of cubical-type model structures (where an explicit description of the fibrations is available) appears in~\cite{cchm:cubicaltt,sattler:eqvext}, while the general case was observed in~\cite{stenzel:thesis}.

\begin{thm}\label{thm:uf-fibrant}
  Let $\E$ be a right proper simplicial Cisinski model category, and \F a \local, \stratified, and homotopy invariant \nfs on \E.
  Then there is a regular cardinal \la such that for any regular cardinal $\ka\shgt\la$, there exists a morphism $\pi:\Util \to U$ such that:
  \begin{enumerate}
  \item The \ka-presentable objects in \E are closed under finite limits.\label{item:uf0}
  \item $\pi:\Util \to U$ is a relatively \ka-presentable $\uly\F$-algebra (in particular, a fibration).\label{item:uf1}
  \item Every relatively \ka-presentable $\uly\F$-algebra is a pullback of $\pi$.\label{item:uf2}
  \item The object $U$ is fibrant.\label{item:uf4}
  \item $\pi$ satisfies the univalence axiom.\label{item:uf3}
  \end{enumerate}
\end{thm}
\begin{proof}
  Let $\la_0$ satisfy \cref{thm:u4p}, let $\la_1$ be such that \E has a generating set of acyclic cofibrations with $\la_1$-presentable domains and codomains, let $\la_2$ be such that \E has functorial factorizations that preserve \ka-presentable objects for any $\ka\shgt\la_2$ (such exists since these factorizations are accessible functors), and let $\la_3$ be such that for any $\ka\shgt\la_3$ the \ka-presentable objects are closed under finite limits (which exists by \cref{thm:pres-pb}).
  % \footnote{It is claimed in~\cite[Propositions 7.2]{dug:pres} that a combinatorial model category satisfies this for all sufficiently large \ka, but the proof only shows it for a \shrp class.}
  Let $\la$ be such that $\la\shgt \la_j$ for $j=0,1,2,3$, and assume $\ka\shgt \la$; then $\ka\shgt\la_j$ for all $j$ as well, and in particular~\ref{item:uf0} holds.

  Since \F and $\cEka$ are \local and \stratified, so is $\Fka = {\F\times_\cE \cEka}$.
  Let $\pi:\Util\to U$ be the universe for $\uly\Fka$ obtained from \cref{thm:pre-u2p}; then~\ref{item:uf1} holds trivially.
  And since \cE and \F are stacks for cell complexes, in particular they preserve the initial object; so by \cref{rmk:universe} we have~\ref{item:uf2}.

  Since $\ka\gt\la_1$, to show that $U$ is fibrant~\ref{item:uf4} it suffices to show that it has right lifting for all acyclic cofibrations between \ka-presentable objects.
  Let $i:A\acof B$ be an acyclic cofibration with $A$ and $B$ \ka-presentable, let $h:A\to U$ be a map, and let $E_1 \fib A$ be the pullback of $\pi$ along $h$.
  Since $\pi$ is relatively \ka-presentable, $E_1$ is \ka-presentable.
  Thus since $\ka\shgt\la_2$, we can factor the composite $E_1\fib A \xto{i} B$ as an acyclic cofibration $E_1 \acof D_2$ followed by a fibration $D_2 \fib B$, where $D_2$ is \ka-presentable.
  So since $\ka\shgt\la_3$, $D_2\fib B$ is a relatively \ka-presentable fibration, and by homotopy invariance it is an $\uly\F$-algebra, hence an $\uly\Fka$-algebra.

  Let $E_2\coloneqq i^*(D_2)$; then by right properness the map $E_2 \to D_2$ is a weak equivalence, hence by 2-out-of-3 so is the induced map $E_1 \to E_2$.
  Thus since $\ka\shgt\la_0$, by \cref{thm:u4p} there is an $\uly\Fka$-algebra $D_1 \fib B$ with $i^*(D_1)\cong E_1$.
  Finally, since $U$ is a universe for $\uly\Fka$, by \cref{rmk:universe} there is a map $k:B\to U$ pulling $\pi$ back to $D_1$ such that $k i = h$; so $U$ has right lifting for $i$.

  For univalence~\ref{item:uf3}, we follow~\cite[Theorem 3.4.1]{klv:ssetmodel},~\cite[\sect 2]{shulman:elreedy}, and~\cite[Theorem 3.12]{cisinski:elegant}.
  Let $\Eq(\Util)$ be the universal space of auto-equivalences of $\pi$, as in~\cite[\sect 4]{shulman:elreedy}; it suffices to show that the composite projection
  \(\Eq(\Util) \to U\times U \to U\)
  is an acyclic fibration.
  Now a square
  \[
    \begin{tikzcd}
      A \ar[d,tail,"i"'] \ar[r] & \Eq(\Util) \ar[d]\\
      B \ar[r] & U
    \end{tikzcd}
  \]
  with $i$ a monomorphism yields a diagram of solid arrows~\eqref{eq:u4p} where all fibrations are $\uly\Fka$-algebras.
  Thus, since $\ka\shgt\la_0$ we can fill out the dashed arrows in~\eqref{eq:u4p} with $D_1 \fib B$ also a $\uly\Fka$-algebra; so by \cref{rmk:universe} we can classify it by a map to $U$ extending the given classifying map of $E_1$.
  But this is precisely what we need to specify a lift $B\to \Eq(\Util)$.
\end{proof}

Thus, to build fibrant and univalent universes for relatively \ka-presentable fibrations in a right proper simplicial Cisinski model category, it suffices to find a \local and \stratified \nfs \F such that $\uly\F=\Fib$ is the class of all fibrations.
We have essentially already seen one way to do this: if \E has a set of generating acyclic cofibrations with representable codomains, then $\Fib$ itself has these properties.
This was the approach of~\cite{klv:ssetmodel,shulman:elreedy}; but to deal with the general case we will have to use non-full \nfss.

\begin{rmk}
  Although our primary interest is in constructing univalent universes for \emph{all} (relatively \ka-presentable) fibrations, it is potentially useful that \cref{thm:uf-fibrant} also yields univalent universes for subclasses of fibrations.
  For instance, the \emph{left fibrations} in bisimplicial sets~\cite{vk:yoneda-css,pbb:groth-segal,rasekh:yoneda-ss} are a subclass of the Reedy fibrations, which by~\cite[Remark 2.1.4(a)]{vk:yoneda-css} are characterized by right lifting against a generating set with representable codomains; thus they admit fibrant and univalent universes.
  Such a universe of left fibrations is essentially the ``\oo-category of spaces'' constructed in~\cite{vk:yoneda-css}, although they do not explain how to make it a strict presheaf.
  In fact it is a complete Segal space (this is shown in~\cite[Theorem 2.2.11]{vk:yoneda-css}, and can also be deduced from~\cite[Theorem 4.8]{rasekh:yoneda-ss}), and could be useful for the programme of~\cite{rs:stt} to use that model for ``synthetic \io-category theory''.
  
  We will see another class of examples in \cref{thm:flf-nfs,rmk:modal-univ}.
\end{rmk}


%%% Local Variables:
%%% mode: latex
%%% TeX-master: "univinj"
%%% End:

\section{Modular arithmetic via the Curry-Howard interpretation}\label{sec:modular-arithmetic}

We have now fully described Martin-L\"of's dependent type theory. It is now up to us to start developing some mathematics in it, and Martin-L\"of's dependent type theory is great for elementary mathematics, such as basic number theory, some algebra, and combinatorics. The fundamental idea that is used to develop basic mathematics in type theory is the Curry-Howard interpretation. This is a translation of logic into type theory, which we will use to express concepts of mathematics such as divisibility, the congruence relations, and so on.

We will also introduce the family $\Fin{}$ of the standard finite types, indexed by $\N$, and show how each $\Fin{k+1}$ can be equipped with the group structure of integers modulo $k+1$. Our goal here is to demonstrate how to do those things in type theory, so we will aim for a high degree of accuracy.

\subsection{The Curry-Howard interpretation}\label{sec:Curry-Howard}

The \emph{Curry-Howard interpretation} is an interpretation of logic into type theory. Recall that in type theory there is no separation between the logical framework and the general theory of collections of mathematical objects the way there is in the more traditional setup with Zermelo-Fraenkel set theory, which is postulated by axioms in first order logic. These two aspects of the foundations of mathematics are unified in type theory. The idea of the Curry-Howard interpretation is therefore to express propositions as types, and to think of the elements of those types as their proofs. We illustrate this idea with an example.

\begin{eg}
  A natural number $d$ is said to divide a natural number $n$ if there exists a natural number $k$ such that $d\cdot k=n$. To represent the divisibility predicate in type theory, we need to define a \emph{type}
  \begin{equation*}
    d\mid n,
  \end{equation*}
  of which the elements are witnesses that $d$ divides $n$. In other words, $d\mid n$ should be the type that consists of natural numbers $k$ equipped with an identification $d\cdot k=n$. In general, the type of $x:A$ equipped with $y:B(x)$ is represented as the type $\sm{x:A}B(x)$. The interpretation of the existential quantification ($\exists$) into type theory via the Curry-Howard interpretation is therefore using $\Sigma$-types.
\end{eg}

\begin{defn}
  Consider two natural numbers $d$ and $n$. We say that $d$ \define{divides}\index{divisibility on N@{divisibility on $\N$}|textbf}\index{natural numbers!divisibility|textbf} $n$ if there is a element of type\index{d {"|" n}@{$d\mid n$}|textbf}\index{d {"|" n}@{$d\mid n$}|see{divisibility on $\N$}}
  \begin{equation*}
    d\mid n\defeq \sm{k:\N}d\cdot k=n.
  \end{equation*}
\end{defn}

\begin{rmk}
  This type-theoretical definition of the divisibility relation using $\Sigma$-types has two important consequences:
  \begin{enumerate}
  \item The principal way to show that $d\mid n$ holds is to construct a pair $(k,p)$ consisting of a natural number $k$ and an identification $p:d\cdot k=n$.
  \item The principal way to use a hypothesis $H:d\mid n$ in a proof is to proceed by $\Sigma$-induction on the variable $H$. We then get to assume a natural number $k$ and an identification $p:d\cdot k=n$, in order to proceed with the proof.
  \end{enumerate}
\end{rmk}

\begin{eg}\label{rmk:elementary-facts-div}
  Just as existential quantification ($\exists$) is translated via the Curry-Howard interpretation to $\Sigma$-types, the translation of the universal quantification ($\forall$) in type theory via the Curry-Howard interpretation is to $\Pi$-types. For example, the assertion that every natural number is divisible by $1$ is expressed in type theory as
  \begin{equation*}
    \prd{x:\N} 1\mid x.
  \end{equation*}
  In other words, in order to show that every number $x:\N$ is divisible by $1$ we need to construct a dependent function
  \begin{equation*}
    \lam{x}p(x):\prd{x:\N}1\mid x.
  \end{equation*}
  We do this by constructing an element
  \begin{equation*}
    p(x):\sm{k:\N}1\cdot k=x
  \end{equation*}
  indexed by $x:\N$. Such an element $p(x)$ is constructed as the pair $(x,q(x))$, where the identification $q(x):1\cdot x=x$ is obtained from the left unit law of multiplication on $\N$, which was constructed in \cref{ex:semi-ring-laws-N}.

  Similarly, the type theoretic proof that every natural number $k$ divides $0$, i.e., that $k\mid 0$, is the pair $(0,p)$ consisting of the natural number $0$ and the identification $p:k\cdot 0=0$ obtained from the right annihilation law of multiplication on $\N$. This identification was also constructed in \cref{ex:semi-ring-laws-N}.
\end{eg}

In the following proposition we will see examples of how a hypothesis of type $d\mid x$ can be used.

\begin{prp}\label{prp:div-3-for-2}
  Consider three natural numbers $d$, $x$ and $y$. If $d$ divides any two of the three numbers $x$, $y$, and $x+y$, then it also divides the third.
\end{prp}

\begin{proof}
  We will only show that if $d$ divides $x$ and $y$, then it divides $x+y$. The remaining two claims, that if $d$ divides $y$ and $x+y$ then it divides $x$, and that if $d$ divides $x$ and $x+y$ then it divides $y$, are left as \cref{ex:div-3-for-2}.

  Suppose that $d$ divides both $x$ and $y$. By assumption we have elements
  \begin{equation*}
    H:\sm{k:\N}d\cdot k=x,\qquad\text{and}\qquad K:\sm{k:\N}d\cdot k=y.
  \end{equation*}
  Since the types of the variables $H$ and $K$ are $\Sigma$-types, we proceed by $\Sigma$-induction on $H$ and $K$. Therefore we get to assume a natural number $k:\N$ equipped with an identification $p:d\cdot k=x$, and a natural number $l:\N$ equipped with an identification $q:d\cdot l=y$. Our goal is now to construct an identification
  \begin{equation*}
    d\cdot (k+l)=x+y.
  \end{equation*}
We construct such an identification as a concatenation $\ct{\alpha}{(\ct{\beta}{\gamma})}$, where the types of the identifications $\alpha$, $\beta$, and $\gamma$ are as follows:
  \begin{equation*}
    \begin{tikzcd}
      d\cdot(k+l) \arrow[r,equals,"\alpha"] & d\cdot k+d\cdot l \arrow[r,equals,"\beta"] & x+d\cdot l \arrow[r,equals,"\gamma"] & x+y.
    \end{tikzcd}
  \end{equation*}
  The identification $\alpha$ is obtained from the fact that multiplication on $\N$ distributes over addition, which was shown in \cref{ex:distributive-mul-addN}. The identifications $\beta$ and $\gamma$ are constructed using the action on paths of a function:
  \begin{equation*}
    \beta\defeq\ap{(\lam{t}t+d\cdot l)}{p},\qquad\text{and}\qquad \gamma\defeq \ap{(\lam{t}x+t)}{q}
  \end{equation*}
  To conclude the proof that $d\mid x+y$, note that we have constructed the pair
  \begin{equation*}
    (k+l,\ct{\alpha}{(\ct{\beta}{\gamma})}):\sm{k:\N}d\cdot k=x+y.\qedhere
  \end{equation*}
\end{proof}

The full Curry-Howard interpretation of logic into type theory also involves interpretations of disjunction, conjunction, implication, and equality.

The introduction and elimination rules for disjunction are, for instance,
\begin{equation*}
  \AxiomC{$P$}
  \UnaryInfC{$P\lor Q$}
  \DisplayProof
  \qquad
  \AxiomC{$Q$}
  \UnaryInfC{$P\lor Q$}
  \DisplayProof
  \qquad
  \text{and}
  \qquad
  \AxiomC{$P\Rightarrow R$}
  \AxiomC{$Q\Rightarrow R$}
  \BinaryInfC{$P\lor Q\Rightarrow R$}
  \DisplayProof
\end{equation*}
The two introduction rules assert that $P\lor Q$ holds provided that $P$ holds, and that $P\lor Q$ holds provided that $Q$ holds. These rules are analogous to the introduction rules for coproduct, which assert that there are functions $\inl : A\to A+B$ and $\inr : B \to A+B$. Furthermore, the non-dependent elimination principle for coproducts gives a function
\begin{equation*}
  (A\to C) \to ((B \to C) \to (A+B \to C))
\end{equation*}
for any type $C$, which is again analogous to the elimination rule of disjunction. The Curry-Howard interpretation of disjunction into type theory is therefore as coproducts.

To interpret conjunction into type theory we observe that the introduction rule and elimination rules for conjunction are
\begin{equation*}
  \AxiomC{$P$}
  \AxiomC{$Q$}
  \BinaryInfC{$P\land Q$}
  \DisplayProof
  \qquad
  \text{and}
  \qquad
  \AxiomC{$P\land Q$}
  \UnaryInfC{$P$}
  \DisplayProof
  \qquad
  \AxiomC{$P\land Q$}
  \UnaryInfC{$Q$}
  \DisplayProof
\end{equation*}
Product types possess such structure, where we have the pairing operation $\pair:A\to (B\to A\times B)$ and the projections $\proj 1:A\times B\to A$ and $\proj 2 : A\times B\to B$ give interpretations of the introduction and elimination rules for conjunction. The Curry-Howard interpretation of conjunction into type theory is therefore by products. We summarize the full Curry-Howard interpretation in \cref{table:Curry-Howard}.

\begin{table}[t]
  \begin{tabular}{ll}
    \toprule
    \multicolumn{2}{c}{The Curry-Howard interpretation} \\
    \midrule
    Propositions & Types \\
    Proofs & Elements \\
    Predicates & Type families \\
    $\top$ & $\unit$ \\
    $\bot$ & $\emptyt$ \\
    $P\lor Q$ & $A+B$ \\
    $P\land Q$ & $A\times B$ \\
    $P\Rightarrow Q$ & $A\to B$ \\
    $\neg P$ & $A\to \emptyt$ \\
    $\exists_{x}P(x)$ & $\sm{x:A}B(x)$ \\
    $\forall_{x}P(x)$ & $\prd{x:A}B(x)$ \\
    $x=y$ & $x=y$ \\
    \bottomrule
  \end{tabular}
  \caption{\label{table:Curry-Howard}The Curry-Howard interpretation of logic into type theory.}
\end{table}

\begin{rmk}
  We should note, however, that despite the similarities between logic and type theory that are highlighted in the Curry-Howard interpretation, there are also some differences. One important difference is that types may contain many elements, whereas in logic, propositions are usually considered to be \emph{proof irrelevant}. This means that to establish the truth of a proposition it only matters \emph{whether} it can be proven, not in how many different ways it can be proven. To address this dissimilarity between general types and logic, we will introduce in \cref{chap:uf} a more refined way of interpreting logic into type theory. In \cref{chap:hierarchy} we will define the type $\isprop(A)$, which expresses the property that the type $A$ is a proposition. Furthermore, we will introduce the \emph{propositional truncation} operation in \cref{sec:propositional-truncation}, which we will use to interpret logic into type theory in such a way that all logical assertions are interpreted as types that satisfy the condition of being a proposition.
\end{rmk}

\subsection{The congruence relations on \texorpdfstring{$\N$}{ℕ}}

Relations in the Curry-Howard interpretation of logic into type theory are also type valued. More specifically, a binary relation on a type $A$ is a family of types $R(x,y)$ indexed by $x,y:A$. Such relations are sometimes called \emph{typal}.

\begin{defn}
  Consider a type $A$. A \define{(typal) binary relation} on $A$ is defined to be a family of types $R(x,y)$ indexed by $x,y:A$. Given a binary relation $R$ on $A$, we say that $R$ is \define{reflexive} if it comes equipped with
  \begin{align*}
    \rho & : \prd{x:A}R(x,x), \\
    \intertext{we say that $R$ is \define{symmetric} if it comes equipped with}
    \sigma & : \prd{x,y:A} R(x,y)\to R(y,x), \\
    \intertext{and we say that $R$ is \define{transitive} if it comes equipped with}
    \tau & : \prd{x,y,z:A} R(x,y)\to (R(y,z)\to R(x,z)).
  \end{align*}
  A \define{(typal) equivalence relation} on $A$ is a reflexive, symmetric, and transitive binary typal relation on $A$.
\end{defn}

To define the congruence relation modulo $k$ in type theory using the Curry-Howard interpretation, we will define for any three natural numbers $x$, $y$, and $k$, a \emph{type}
\begin{equation*}
  x\equiv y\mod k
\end{equation*}
consisting of the proofs that $x$ is congruent to $y$ modulo $k$. We will define this type by directly interpreting Gauss' definition of the congruence relations in his \emph{Disquisitiones Arithmeticae} \cite{Gauss}: two numbers $x$ and $y$ are congruent modulo $k$ if $k$ divides the symmetric difference $\distN(x,y)$ between $x$ and $y$. Recall that $\distN(x,y)$ was defined in \cref{ex:distN} recursively by
  \begin{align*}
    \distN(0,0) & \defeq 0 & \distN(0,y+1) & \defeq y+1 \\
    \distN(x+1,0) & \defeq x+1 & \distN(x+1,y+1) & \defeq \distN(x,y).
  \end{align*}

\begin{defn}
  Consider three natural numbers $k,x,y:\N$. We say that $x$ is \define{congruent to $y$ modulo $k$}\index{congruence relations on N@{congruence relations on $\N$}|textbf}\index{natural numbers!congruence relations|textbf} if it comes equipped with an element of type
  \begin{equation*}
    x\equiv y \mod k \defeq k\mid\distN(x,y).
  \end{equation*}
\end{defn}

\begin{eg}
  For example, $k\equiv 0\mod k$. To see this, we have to show that $k\mid\distN(k,0)$. Since $\distN(k,0)=k$ it suffices to show that $k\mid k$. That is, we have to construct a natural number $l$ equipped with an identification $p:kl=k$. Of course, we choose $l\defeq 1$, and the equation $k1=k$ holds by the right unit law for multiplication on $\N$, which was shown in \cref{ex:semi-ring-laws-N}.
\end{eg}

\begin{prp}\label{prp:congruence-eqrel}
  For each $k:\N$, the congruence relation modulo $k$ is an equivalence relation.
\end{prp}

\begin{proof}
  Reflexivity follows from the fact that $\distN(x,x)=0$, and any number divides $0$. Symmetry follows from the fact that $\distN(x,y)=\distN(y,x)$ for any two natural numbers $x$ and $y$.

  The non-trivial part of the claim is therefore transitivity. Here we use the fact that for any three natural numbers $x$, $y$, and $z$, at least one of the equalities
  \begin{align*}
    \distN(x,y)+\distN(y,z) & =\distN(x,z) \\
    \distN(y,z)+\distN(x,z) & =\distN(x,y) \\
    \distN(x,z)+\distN(x,y) & =\distN(y,z)
  \end{align*}
  holds. A formal proof of this fact is given by case analysis on the six possible ways in which $x$, $y$, and $z$ can be ordered:
  \begin{align*}
    x\leq y & \text{ and }y\leq z, & x\leq z & \text{ and }z\leq y, \\
    y\leq z & \text{ and }z\leq x, & y\leq x & \text{ and }x\leq z, \\
    z\leq x & \text{ and }x\leq y, & z\leq y & \text{ and }y\leq x.
  \end{align*}
  Therefore it follows by \cref{ex:distN-triangle-equality} and \cref{prp:div-3-for-2} that ${k\mid\distN(x,z)}$ if ${k\mid\distN(x,y)}$ and ${k\mid\distN(y,z)}$.
\end{proof}

\subsection{The standard finite types}\label{sec:Fin}

The standard finite sets are classically defined as the sets $\{x\in\N\mid x<k\}$. This leads to the question of how to interpret a subset $\{x\in A\mid P(x)\}$ in type theory.

Since type theory is set up in such a way that elements come equipped with their types, subsets aren't formed the same way as in set theory, where the comprehension axiom is used to form the set $\{x\in A\mid P(x)\}$ for any predicate $P$ over $A$. The Curry-Howard interpretation dictates that predicates are interpreted as dependent types. Therefore, a set of elements $x\in A$ such that $P(x)$ holds is interpreted in type theory as the type of terms $x:A$ equipped with an element (a proof) $p:P(x)$. In other words, we interpret a subset $\{x\in A\mid P(x)\}$ as the type $\sm{x:A}P(x)$.

\begin{rmk}
  The alert reader may now have observed that the interpretation of a subset $\{x\in A\mid P(x)\}$ in type theory is the same as the interpretation of the proposition $\exists_{(x\in A)}P(x)$, while indeed the subset $\{x\in A\mid P(x)\}$ has a substantially different role in mathematics than the proposition $\exists_{(x\in A)}P(x)$. This points at a slight problem of the Curry-Howard interpretation of the existential quantifier. While the Curry-Howard interpretation of the existential quantifier is nevertheless useful and important, we will reinterpret the existential quantifier in type theory in \cref{sec:logic}.
\end{rmk}

Since subsets are interpreted as $\Sigma$-types, the `classical' definition of the standard finite types is
\begin{equation*}
  \classicalFin_k:=\sm{x:\N}x<k.
\end{equation*}
This is a perfectly fine definition of the standard finite types. However, the usual definition of the standard finite types in Martin-L\"of's dependent type theory is a more direct, recursive definition, which takes full advantage of the inductive constructions of dependent type theory. 

\begin{defn}\label{defn:fin}
  We define the type family $\Fin{}$ of the \define{standard finite types}\index{Fin k@{$\Fin{k}$}|see {standard finite type}}\index{Fin k@{$\Fin{k}$}|textbf}\index{standard finite type}\index{type family!of standard finite types} over $\N$ recursively by
  \begin{align*}
    \Fin{0} & \defeq \emptyt \\*
    \Fin{k+1} & \defeq \Fin{k}+\unit.
  \end{align*}
  We will write $i$ for the inclusion $\inl:\Fin{k}\to\Fin{k+1}$ and we will write $\ttt$ for the point $\inr(\ttt)$.
\end{defn}

In \cref{ex:classical-Fin} you will be asked to show that the types $\classicalFin_k$ and $\Fin{k}$ are isomorphic.

\begin{rmk}
The type family $\Fin{}$ over $\N$ can be given its own induction principle, which is, at least for the time being, the principal way to make constructions on $\Fin{k}$ for arbitrary $k:\N$ and to prove properties about those constructions. The induction principle of the standard finite types tells us that the family of standard finite types is inductively generated by
\begin{align*}
  i & : \Fin{k}\to\Fin{k+1} \\*
  \ttt & : \Fin{k+1}. 
\end{align*}
In other words, we can define a dependent function $f:\prd{k:\N}\prd{x:\Fin{k}}P_k(x)$ by defining
\begin{align*}
  g_k & : \prd{x:\Fin{k}}P_k(x)\to P_{k+1}(i(x)) \\*
  p_k & : P_{k+1}(\ttt)
\end{align*}
for each $k:\N$. The function $f$ defined in this way then satisfies the judgmental equalities
\begin{align*}
  f_{k+1}(i(x)) & \jdeq g_k(x,f_k(x)) \\*
  f_{k+1}(\ttt) & \jdeq p_k.
\end{align*}
These judgmental equalities completely determine the function $f$, and therefore we may also present such inductive definitions by pattern matching:
  \begin{align*}
    f_{k+1}(i(x)) & \defeq g_k(x,f_k(x)) \\*
    f_{k+1}(\ttt) & \defeq p_k.
  \end{align*}
\end{rmk}

We will often use definitions by pattern matching for two reasons: (i) such definitions are concise, and (ii) they display the judgmental equalities that hold for the defined object. Those judgmental equalities are the only thing we know about that object, and proving a claim about it often amounts to finding a way to apply these judgmental equalities.

To illustrate this way of working with the standard finite types, we define the inclusion functions $\Fin{k}\to\N$, and show that these are injective. In order to show that $\natFin_k$ is injective, we will also show that $\natFin_k$ is bounded.

\begin{defn}\label{defn:natFin}
  We define the inclusion $\natFin_k : \Fin{k}\to\N$ inductively by
  \begin{align*}
    \natFin_{k+1}(i(x)) & \defeq \natFin_{k}(x) \\
    \natFin_{k+1}(\ttt) & \defeq k.
  \end{align*}
\end{defn}

\begin{lem}\label{lem:is-bounded-natFin}
  The function $\natFin:\Fin{k}\to\N$ is bounded, in the sense that $\natFin(x)< k$ for each $x:\Fin{k}$.
\end{lem}

\begin{proof}
  The proof is by induction. In the base case there is nothing to show. In the inductive step, we have the inequalities $\natFin_{k+1}(i(x))\jdeq\natFin_{k}(x)<k<k+1$, where the first inequality holds by the inductive hypothesis, and we also have
  \begin{equation*}
    \natFin_{k+1}(\ttt)\jdeq k<k+1.\qedhere
  \end{equation*}
\end{proof}

\begin{prp}\label{prp:is-injective-natFin}
  The inclusion function $\natFin_k : \Fin{k}\to \N$ is injective, for each $k:\N$.
\end{prp}

\begin{proof}
  We define a function $\alpha_k(x,y):(\natFin_k(x)=\natFin_k(y))\to (x=y)$ recursively by
  \begin{align*}
    \alpha_{k+1}(i(x),i(y),p) & \defeq \ap{i}{\alpha_k(x,y,p)} & \alpha_{k+1}(i(x),\ttt,p) & \defeq \exfalso(f(p)) \\
    \alpha_{k+1}(\ttt,i(y),p) & \defeq \exfalso(g(p)) & \alpha_{k+1}(\ttt,\ttt,p) & \defeq \refl{},
  \end{align*}
  where $f:(\natFin_{k+1}(i(x))=\natFin_{k+1}(\ttt))\to\emptyt$ and $g:(\natFin_{k+1}(\ttt)=\natFin_{k+1}(i(y)))\to\emptyt$ are obtained from the fact that $\natFin_{k+1}(i(z))\jdeq\natFin_k(z)<k$ for any $z:\Fin{k}$, and the fact that $\natFin_{k+1}(\ttt)\jdeq k$.
\end{proof}

\subsection{The natural numbers modulo \texorpdfstring{$k+1$}{k+1}}\label{subsec:finite-types-quotient-maps}

Given an equivalence relation $\sim$ on a set $A$ in classical mathematics, the quotient $A/{\sim}$ comes equipped with a quotient map $q:A\to A/{\sim}$ that satisfies two important properties: (1) The map $q$ satisfies the condition
\begin{equation*}
  q(x)=q(y)\leftrightarrow x\sim y,
\end{equation*}
and (2) the map $q$ is surjective. The first condition is called the \define{effectiveness} of the quotient map.

In classical mathematics, a map $f:A\to B$ is said to be surjective if for every $b\in B$ there exists an element $a\in A$ such that $f(a)=b$. Following the Curry-Howard interpretation, a map $f:A\to B$ is therefore surjective if it comes equipped with a dependent function
\begin{equation*}
  \prd{b:B}\sm{a:A}f(a)=b.
\end{equation*}
However, there is a subtle issue with this interpretation of surjectivity. It is somewhat stronger than the classical notion of surjectivity, because a dependent function $\prd{b:B}\sm{a:A}f(a)=b$ provides for every element $b:B$ an \emph{explicit} element $a:A$ equipped with an explicit identification $p:f(a)=b$, whereas in the classical notion of surjectivity such an element $a\in A$ is merely asserted to exist. To emphasize that the Curry-Howard interpretation of surjectivity is stronger than intended we make the following definition, and we will properly introduce surjective maps in \cref{subsec:surjective}.

\begin{defn}
  Consider a function $f:A\to B$. We say that $f$ is \define{split surjective} if it comes equipped with an element of type
  \begin{equation*}
    \issplitsurjective(f):=\prd{b:B}\sm{a:A}f(a)=b.
  \end{equation*}
\end{defn}

Martin-L\"of's dependent type theory doesn't have a general way of forming quotients of types. However, in the specific case of the congruence relations on $\N$ we can define the type of natural numbers modulo $k+1$ as the standard finite type $\Fin{k+1}$. We will show that $\Fin{k+1}$ comes equipped with a map
\begin{equation*}
  [\blank]_{k+1}:\N\to \Fin{k+1}
\end{equation*}
for each $k:\N$, and we will show in \cref{thm:effective-mod-k,thm:issec-nat-Fin} that this map satisfies conditions (1) and (2) in the split surjective sense.

To prepare for the definition of the quotient map $[\blank]_{k+1}$, we will first define a zero element of $\Fin{k+1}$ and successor function on each $\Fin{k}$. We will also define an auxiliary function $\skipzeroFin_k:\Fin{k}\to\Fin{k+1}$, which is used in the definition of the successor function. The map $[\blank]_{k+1}$ is then defined by iterating the successor function. 

\begin{defn} ~\nopagebreak
  \begin{enumerate}
  \item We define the \define{zero element} $\zeroFin_k:\Fin{k+1}$ recursively by
    \begin{align*}
      \zeroFin_0 & \defeq\ttt \\*
      \zeroFin_{k+1} & \defeq i(\zeroFin_k).
                       \intertext{Since there is a mismatch between the index of $\zeroFin_k$ and the index of its type, we will often simply write $\zeroFin$ or $0$ for the zero element of $\Fin{k+1}$.
    \item We define the function $\skipzeroFin_k:\Fin{k}\to\Fin{k+1}$ recursively by}
      \skipzeroFin_{k+1}(i(x)) & \defeq i(\skipzeroFin_k(x)) \\*
      \skipzeroFin_{k+1}(\ttt) & \defeq \ttt.
    \intertext{\item We define the \define{successor function} $\succFin_k:\Fin{k}\to\Fin{k}$ recursively by}
      \succFin_{k+1}(i(x)) & \defeq \skipzeroFin_k(x) \\*                       
      \succFin_{k+1}(\ttt)    & \defeq \zeroFin_k.
    \end{align*}
  \end{enumerate}
\end{defn}

\begin{defn}
  For any $k:\N$, we define the map $[\blank]_{k+1}:\N\to\Fin{k+1}$ recursively on $x$ by
  \begin{align*}
    [0]_{k+1} & \defeq 0 \\*
    [x+1]_{k+1} & \defeq \succFin_{k+1}[x]_{k+1}.
  \end{align*}
\end{defn}

Our next intermediate goal is to show that $x\equiv \natFin[x]_{k+1}\mod k+1$ for any natural number $x$. This fact is a consequence of the following simple lemma, that will help us compute with the maps $\natFin : \Fin{k}\to\N$.

\begin{lem}\label{lem:nat-Fin}
  We make three claims:
  \begin{enumerate}
  \item For any $k:\N$ there is an identification
    \begin{align*}
      \natFin(\zeroFin_k) & = 0
  \intertext{\item For any $k:\N$ and any $x:\Fin{k}$, we have}
      \natFin(\skipzeroFin_k(x)) & = \natFin(x)+1.
  \intertext{\item For any $k:\N$ and any $x:\Fin{k}$, we have}
      \natFin(\succFin_k(x)) & \equiv \natFin(x)+1 \mod k.
    \end{align*}
  \end{enumerate}
\end{lem}

\begin{proof}
  For the first claim, we define an identification $\alpha_k:\natFin(\zeroFin_k)=0$ recursively by
  \begin{align*}
    \alpha_0 & \defeq \refl{} \\
    \alpha_{k+1} & \defeq \alpha_k.
  \intertext{For the second claim, we define an identification $\beta_k(x):\natFin(\skipzeroFin_k(x))=\natFin(x)+1$ recursively by}
    \beta_{k+1}(i(x)) & \defeq \beta_k(x) \\
    \beta_{k+1}(\ttt) & \defeq \refl{}.
  \end{align*}
  For the third claim, we again define an element $\gamma_k(x):\natFin(\succFin_k(x)) \equiv \natFin(x)+1\mod{k}$ recursively. To obtain
  \begin{equation*}
    \gamma_{k+1}(i(x)) : \natFin(\succFin_{k+1}(i(x))) \equiv\natFin(i(x))+1\mod{k+1},
  \end{equation*}
  we calculate
  \begin{align*}
    \natFin(\succFin_{k+1}(i(x))) & \jdeq \natFin(\skipzeroFin(x)) & & \text{by definition of }\succFin\\
                                  & = \natFin(x)+1 & & \text{by claim (ii).}
  \end{align*}
  Since the congruence relation modulo $k+1$ is reflexive, we obtain $\gamma_{k+1}(i(x))$ from the identification of the above calculation. To obtain
  \begin{equation*}
    \gamma_{k+1}(\ttt) : \natFin(\succFin_{k+1}(\ttt)) \equiv \natFin(\ttt)+1\mod{k+1},
  \end{equation*}
  we calculate
  \begin{align*}
    \natFin(\succFin_{k+1}(\ttt)) & \jdeq \natFin(0) & & \text{by definition of }\succFin \\
                                  & = 0 & & \text{by claim (i)} \\
                                  & \equiv k+1 & & \text{by \cref{rmk:elementary-facts-div}} \\
                                  & \jdeq \natFin(\ttt)+1 & & \text{by definition of }\natFin.\qedhere
  \end{align*}
\end{proof}

\begin{prp}\label{prp:cong-nat-mod-succ}
  For any $x:\N$ we have
  \begin{equation*}
    \natFin[x]_{k+1}\equiv x \mod k+1.
  \end{equation*}
\end{prp}

\begin{proof}
  The proof by induction on $x$. The fact that
  \begin{equation*}
    \natFin[0]_{k+1}\equiv 0 \mod {k+1}
  \end{equation*}
  is immediate from the fact that $\natFin[0]_{k+1}\jdeq\natFin(0)=0$, which was shown in \cref{lem:nat-Fin}. In the inductive step, we have to show that
  \begin{equation*}
    \natFin[x+1]_{k+1}\equiv x+1\mod k+1.
  \end{equation*}
  This follows from the following computation
  \begin{align*}
    \natFin[x+1]_{k+1} & \jdeq \natFin(\succFin_{k+1}[x]_{k+1}) & & \text{by definition of }[\blank]_{k+1} \\
                       & \equiv \natFin[x]_{k+1}+1 & & \text{by \cref{lem:nat-Fin}} \\
                       & \equiv x+1 & & \text{by the inductive hypothesis.}\qedhere
  \end{align*}
\end{proof}

We need one more fact before we can prove \cref{thm:effective-mod-k,thm:issec-nat-Fin}.

\begin{prp}\label{cor:eq-congN}
  For any natural number $x<d$ we have
  \begin{equation*}
  d\mid x\leftrightarrow x=0.  
  \end{equation*}
  Consequently, for any two natural numbers $x$ and $y$ such that $\distN(x,y)<k$, we have
  \begin{equation*}
    x\equiv y\mod k\leftrightarrow x=y.
  \end{equation*}
\end{prp}

\begin{proof}
  Note that the implication $x=0\to d\mid x$ is trivial, so it suffices to prove the forward implication
  \begin{equation*}
    d\mid x \to x=0.
  \end{equation*}
  This implication clearly holds if $x\jdeq 0$. Therefore we only have to show that $d\mid x+1$ implies $x+1=0$, if we assume that $x+1<d$. In other words, we will derive a contradiction from the hypotheses that $x+1<d$ and $d\mid x+1$. To reach a contradiction we use \cref{ex:contradiction-le}, by which it suffices to show that $d\leq x+1$.
  
  We proceed by $\Sigma$-induction on the (unnamed) variable of type $d\mid x+1$, so we get to assume a natural number $k$ equipped with an identification $p:dk=x+1$. In the case where $k\jdeq 0$ we reach an immediate contradiction via \cref{prp:zero-one}, because we obtain that $0=d\cdot 0=x+1$. In the case where $k\jdeq\succN(k')$ it follows that
  \begin{equation*}
    d\leq dk'+ d\jdeq dk = x+1.\qedhere
  \end{equation*}
\end{proof}

\begin{thm}\label{thm:effective-mod-k}
  Consider a natural number $k$. Then we have
  \begin{equation*}
    [x]_{k+1}=[y]_{k+1} \leftrightarrow x\equiv y\mod k+1,
  \end{equation*}
  for any $x,y:\N$.
\end{thm}

\begin{proof}
  First note that, since $\natFin$ is injective by \cref{prp:is-injective-natFin}, we have
  \begin{align*}
    [x]_{k+1}=[y]_{k+1} & \leftrightarrow \natFin[x]_{k+1}=\natFin[y]_{k+1}.
  \end{align*}
  Since the inequalities $\natFin[x]_{k+1}<k+1$ and $\natFin[y]_{k+1}<k+1$ hold by \cref{lem:is-bounded-natFin}, it follows by \cref{cor:eq-congN} that
  \begin{equation*}
    \natFin[x]_{k+1}=\natFin[y]_{k+1}\leftrightarrow \natFin[x]_{k+1}\equiv\natFin[y]_{k+1}\mod k+1.   
  \end{equation*}
  The latter condition is by \cref{prp:cong-nat-mod-succ} equivalent to the condition that $x\equiv y\mod k+1$.
\end{proof}

\begin{thm}\label{thm:issec-nat-Fin}
  For any $x:\Fin{k+1}$ there is an identification
  \begin{equation*}
    [\natFin(x)]_{k+1}=x.
  \end{equation*}
  In other words, the map $[\blank]_{k+1}:\N\to \Fin{k+1}$ is split surjective.
\end{thm}

\begin{proof}
  Since $\natFin:\Fin{k+1}\to\N$ is injective by \cref{prp:is-injective-natFin}, it suffices to show that
  \begin{equation*}
    \natFin[\natFin(x)]_{k+1}=\natFin(x).
  \end{equation*}
  Now observe that $\natFin[\natFin(x)]_{k+1}<k+1$ and $\natFin(x)<k+1$. By \cref{cor:eq-congN} it therefore suffices to show that
  \begin{equation*}
    \natFin[\natFin(x)]_{k+1}\equiv\natFin(x)\mod{k+1}.
  \end{equation*}
  This fact is an instance of \cref{prp:cong-nat-mod-succ}.
\end{proof}

\subsection{The cyclic groups}
We can now define the cyclic groups $\Z/k$ for each $k:\N$. Note that $\Z/k$ must come equipped with the structure of a quotient $\Z/{\equiv}$ of $\Z$ by the congruence relation modulo $k$. In the case where $k\jdeq 0$, we have that $x\equiv y\mod{0}$ if and only if $x=y$. This motivates the following definition:

\begin{defn}\label{defn:Zk}
  We define the type $\Z/k$ for each $k:\N$ by
  \begin{equation*}
    \Z/0\defeq \Z\qquad\text{and}\qquad \Z/{(k+1)}\defeq\Fin{k+1}.
  \end{equation*}
\end{defn}

Recall from \cref{ex:int_group_laws} that $\Z/0$ already comes equipped with the structure of a group, but the group structure on $\Z/{(k+1)}$ remains to be defined.

\begin{defn}
  We define the \define{addition} operation on $\Z/{(k+1)}$ by
  \begin{equation*}
    x+y\defeq[\natFin(x)+\natFin(y)]_{k+1},
  \end{equation*}
  and we define the \define{additive inverse} operation on $\Z/{(k+1)}$ by
  \begin{equation*}
    -x\defeq[\distN(\natFin(x),k+1)]_{k+1}.
  \end{equation*}
\end{defn}

\begin{rmk}
  The following congruences modulo $k+1$ follow immediately from \cref{prp:cong-nat-mod-succ}:
  \begin{align*}
    \natFin(0) & \equiv 0 \\
    \natFin(x+y) & \equiv \natFin(x)+\natFin(y) \\
    \natFin(-x) & \equiv \distN(\natFin(x),k+1).
  \end{align*}
\end{rmk}

Before we show that addition on $\Z/{k}$ satisfies the group laws, we have to show that addition on $\N$ preserves the congruence relation.

\begin{prp}
  Consider $x,y,x',y':\N$. If any two of the following three properties hold, then so does the third:
  \begin{enumerate}
  \item $x\equiv x'\mod k$,
  \item $y\equiv y'\mod k$,
  \item $x+y\equiv x'+y'\mod k$.
  \end{enumerate}
\end{prp}

\begin{proof}
  Recall that the distance function $\distN$ is translation invariant by \cref{ex:translation-invariant-distN}. Therefore it follows that
  \begin{equation}\label{eq:translation-invariant-congN}
    a\equiv b\mod k \leftrightarrow a+c\equiv b+c\mod k.\tag{\textasteriskcentered}
  \end{equation}
  We will use this observation to prove the claim.
  
  First, suppose that $x\equiv x'$ and $y\equiv y'$ modulo $k$. Then it follows by \cref{eq:translation-invariant-congN} that
  \begin{equation*}
    x+y\equiv x'+y\equiv x'+y'.
  \end{equation*}
  This shows that (i) and (ii) together imply (iii).

  Next, suppose that $x\equiv x'$ and $x+y\equiv x'+y'$ modulo $k$. Then it follows that
  \begin{equation*}
    x+y\equiv x'+y'\equiv x+y'.
  \end{equation*}
  Applying \cref{eq:translation-invariant-congN} once more in the reverse direction, we obtain that $y\equiv y'$ modulo $k$. This shows that (i) and (iii) together imply (ii).

  The remaining claim, that (ii) and (iii) together imply (i), follows by commutativity of addition from the fact that (i) and (iii) together imply (ii).
\end{proof}

\begin{thm}
  The addition operation on $\Z/{k}$ satisfies the laws of an abelian group:
  \begin{align*}
    0+x & = x & x+0 & = x \\
    (-x)+x & = 0 & x+(-x) & = 0 \\
    (x+y)+z & = x+(y+z) & x+y & = y+x. 
  \end{align*}
\end{thm}

\begin{proof}
  The fact that the addition operation on $\Z/0$ satisfies the laws of an abelian group was stated as \cref{ex:int_group_laws}. Therefore we will only show that addition on $\Z/{(k+1)}$ satisfies the laws of an abelian group.

  We first note that by commutativity of addition on $\N$, it follows immediately that addition on $\Z/{(k+1)}$ is commutative.

  To prove associativity, note that by \cref{thm:effective-mod-k} it suffices to show that
  \begin{equation*}
    \natFin(x+y)+\natFin(z)\equiv\natFin(x)+\natFin(y+z)\mod k+1.
  \end{equation*}
  Since addition on $\Z/{(k+1)}$ maps preserves the congruence relation, and since we have the congruences
  \begin{align*}
    \natFin(x+y) & \equiv \natFin(x)+\natFin(y) \mod k+1 \\
    \natFin(y+z) & \equiv \natFin(y)+\natFin(z) \mod k+1,
  \end{align*}
  it suffices to show that
  \begin{equation*}
    (\natFin(x)+\natFin(y))+\natFin(z) \equiv \natFin(x)+(\natFin(y)+\natFin(z)) \mod k+1.
  \end{equation*}
  This follows immediately by associativity of addition on $\N$.

  To show that addition on $\Z/{(k+1)}$ satisfies the right unit law, we first observe that it suffices to show that
  \begin{equation*}
    [\natFin(x)+\natFin(0)]_{k+1}=[\natFin(x)]_{k+1}
  \end{equation*}
  because there is an identification $[\natFin(x)]_{k+1}=x$ by \cref{thm:issec-nat-Fin}. By \cref{thm:effective-mod-k} it now suffices tho show that
  \begin{equation*}
    \natFin(x)+\natFin(0)\equiv\natFin(x)\mod k+1. 
  \end{equation*}
  This follows immediately from the fact that $\natFin(0)=0$. The left unit law now follows from the right unit law by commutativity. We leave the inverse laws as an exercise.
\end{proof}

\begin{exercises}
  \exitem \label{ex:div-3-for-2}Complete the proof of \cref{prp:div-3-for-2}.
  \exitem \label{ex:is-poset-div}Show that the divisibility relation satisfies the axioms of a poset, i.e., that it is reflexive, antisymmetric, and transitive.
  \exitem \label{ex:div-factorial}Construct a dependent function
  \begin{equation*}
    \prd{x:\N}(x\neq 0)\to ((x\leq n)\to (x\mid n!))
  \end{equation*}
  for every $n:\N$.
  \exitem Define $1\defeq[1]_{k+1}:\Fin{k+1}$. Show that
  \begin{equation*}
    \succFin_{k+1}(x)=x+1
  \end{equation*}
  for any $x:\Fin{k+1}$.
  \exitem \label{ex:Eq-Fin}The observational equality on $\Fin{k}$ is a binary relation
  \begin{equation*}
    \EqFin_{k}:\Fin{k}\to(\Fin{k}\to\UU_0)
  \end{equation*}
  defined recursively by
  \begin{align*}
    \EqFin_{k+1}(i(x),i(y)) & \defeq \EqFin_k(x,y) & \EqFin_{k+1}(i(x),\ttt) & \defeq \emptyt \\*
    \EqFin_{k+1}(\ttt,i(y)) & \defeq \emptyt & \EqFin_{k+1}(\ttt,\ttt) & \defeq \unit.
  \end{align*}
  \begin{subexenum}
  \item \label{ex:eq-iff-Eq-Fin}Show that
  \begin{equation*}
    (x=y)\leftrightarrow \EqFin_k(x,y)
  \end{equation*}
  for any two elements $x,y:\Fin{k}$.
  \item \label{ex:is-injective-i-Fin}Show that the function $i:\Fin{k}\to\Fin{k+1}$ is injective, for each $k:\N$.
  \item \label{ex:neq-zero-succ-Fin}
  Show that
  \begin{equation*}
    \succFin_{k+1}(i(x))\neq 0
  \end{equation*}
  for any $x:\Fin{k}$.
  \item Show that function $\succFin_k:\Fin{k}\to\Fin{k}$ is injective, for each $k:\N$.
  \end{subexenum}
  \exitem \label{ex:has-inverse-succ-Fin}The predecessor function $\predFin_k:\Fin{k}\to\Fin{k}$ is defined in three steps, just as in the definition of the successor function on $\Fin{k}$.
  \begin{enumerate}
  \item We define the element $\negtwoFin_k:\Fin{k+1}$ by
    \begin{align*}
      \negtwoFin_0 & \defeq\ttt \\*
      \negtwoFin_{k+1} & \defeq i(\ttt).
    \intertext{\item We define the function $\skipnegtwoFin_k:\Fin{k}\to\Fin{k+1}$ recursively by}
      \skipnegtwoFin_{k+1}(i(x)) & \defeq i(i(x)) \\*
      \skipnegtwoFin_{k+1}(\ttt) & \defeq \ttt.
    \intertext{\item Finally, we define the \define{predecessor function} $\predFin_k:\Fin{k}\to\Fin{k}$ recursively by}
      \predFin_{k+1}(i(x)) & \defeq \skipnegtwoFin_k(\predFin_k(x)) \\*                       
      \predFin_{k+1}(\ttt)    & \defeq \negtwoFin_k.
    \end{align*}
  \end{enumerate}
  Show that $\predFin_k$ is an inverse to $\succFin_k$, i.e., construct identifications
  \begin{equation*}
    \succFin_k(\predFin_k(x))=x,\qquad\text{and}\qquad\predFin_k(\succFin_k(x))=x
  \end{equation*}
  for each $x:\Fin{k}$.
  \exitem \label{ex:classical-Fin}Recall that
  \begin{equation*}
    \classicalFin_k:=\sm{x:\N}x<k.
  \end{equation*}
  \begin{subexenum}
  \item Show that
    \begin{equation*}
      (x=y)\leftrightarrow (\proj 1(x)=\proj 1(y))
    \end{equation*}
    for each $x,y:\classicalFin_k$.
  \item By \cref{lem:is-bounded-natFin} it follows that the map $\natFin :\Fin{k}\to\N$ induces a map $\natFin:\Fin{k}\to\classicalFin_k$. Construct a map
    \begin{equation*}
      \alpha_k:\classicalFin_k \to \Fin{k}  
    \end{equation*}
    for each $k:\N$, and show that
  \begin{equation*}
    \alpha_k(\natFin(x)) = x \qquad\text{and}\qquad \natFin(\alpha_k(y)) = y
  \end{equation*}
  for each $x:\Fin{k}$ and each $y:\classicalFin_k$. 
  \end{subexenum}
  \exitem \label{ex:ring-Fin}The multiplication operation $x,y\mapsto xy$ on $\Z/{(k+1)}$ is defined by
  \begin{equation*}
    xy \defeq [\natFin(x)\natFin(y)]_{k+1}.
  \end{equation*}
  \begin{subexenum}
  \item Show that $\natFin(xy)\equiv\natFin(x)\natFin(y)\mod{k+1}$ for each $x,y:\Z/{(k+1)}$.
  \item \label{ex:congruence-mulN}Show that
    \begin{equation*}
      xy\equiv x'y'\mod k
    \end{equation*}
    for any $x,y,x',y':\N$ such that $x\equiv x'$ and $y\equiv y' \mod k$.
  \item Show that multiplication on $\Z/{(k+1)}$ satisfies the laws of a commutative ring:
    \begin{align*}
      (xy)z & = x(yz) & xy & = yx \\
      1x & = x & x1 & = x \\
      x(y+z) & = xy+xz & (x+y)z & = xz+yz.
    \end{align*}
  \end{subexenum}
  \exitem \label{ex:euclidean-division}(Euclidean division) Consider two natural numbers $a$ and $b$.
  \begin{subexenum}
  \item Construct two natural numbers $q$ and $r$ such that $(b\neq 0) \to (r<b)$, along with an identification
    \begin{equation*}
      a=qb+r.
    \end{equation*}
  \item Show that for any four natural numbers $q,q'$ and $r,r'$ such that the implications $(b\neq 0) \to (r<b)$ and $(b\neq 0)\to (r'<b)$ hold, and for which there are identifications
    \begin{equation*}
      a=qb+r\qquad\text{and}\qquad a=q'b+r',
    \end{equation*}
    we have $q=q'$ and $r=r'$.
  \end{subexenum}
  \exitem The type $\N_k$ of \define{$k$-ary natural numbers} is an inductive type with the following constructors:
  \begin{align*}
    \constantbasedN{k} & : \Fin{k}\to\basedN{k} \\
    \unaryopbasedN{k} & : \Fin{k}\to (\basedN{k}\to\basedN{k}).
  \end{align*}
  A $k$-ary natural number can be converted back into an ordinary natural number via the function $\convertbasedN{k}:\basedN{k}\to\N$, which is defined recursively by
  \begin{align*}
    \convertbasedN{k}(\constantbasedN{k}(x)) & \defeq \natFin(x) \\
    \convertbasedN{k}(\unaryopbasedN{k}(x,n)) & \defeq k(\convertbasedN{k}(n)+1)+\natFin(x).
  \end{align*}
  \begin{subexenum}
  \item Show that the type $\basedN{0}$ is empty.
  \item Show that the function $\convertbasedN{k}:\basedN{k}\to\N$ is injective.
  \item Show that the function $\convertbasedN{k+1}:\basedN{k+1}\to\N$ has an inverse, i.e. construct a function
    \begin{equation*}
      g_{k} : \N\to\basedN{k+1}
    \end{equation*}
    equipped with identifications
    \begin{align*}
      \convertbasedN{k+1}(g_k(n)) & = n \\
      g_{k}(\convertbasedN{k+1}(x)) & = x
    \end{align*}
    for each $n:\N$ and each $x:\basedN{k+1}$.
  \end{subexenum}
\end{exercises}
%%% Local Variables:
%%% mode: latex
%%% TeX-master: "hott-intro"
%%% End:

\section{Decidability in elementary number theory}\label{sec:decidability}

Martin-L\"of's dependent type theory is a foundation for constructive mathematics, but in constructive mathematics there is no way to show that $P\lor\neg P$ holds for an arbitrary proposition $P$. Likewise, in type theory there is no way to construct an element of type $A+\neg A$ for an arbitrary type $A$. Consequently, if we want to reason by case analysis over whether $A$ is empty or nonempty, we first have to \emph{show} that $A+\neg A$ holds.

A type $A$ that comes equipped with an element of type $A+\neg A$ is said to be \emph{decidable}. Even though we cannot show that all types are decidable, many types are indeed decidable. Examples include the empty type and any type that comes equipped with a point, such as the type of natural numbers.

Decidability is an important concept with many applications in number theory and finite mathematics, and in this section we will explore the applications of decidability to elementary number theory. For example, the natural numbers satisfy a well-ordering principle with respect to decidable type families over the natural numbers; decidability can be used to construct the greatest common divisor of any two natural numbers; and it can also be used to show that there are infinitely many prime numbers.

\subsection{Decidability and decidable equality}

\begin{defn}
  A type $A$ is said to be \define{decidable}\index{decidable type|textbf} if it comes equipped with an element of type\index{is-decidable@{$\isdecidable(A)$}}
  \begin{equation*}
    \isdecidable(A)\defeq A+\neg A.
  \end{equation*}
  A family $P$ over a type $A$ is said to be \define{decidable}\index{decidable family of types|textbf} if $P(x)$ is decidable for every $x:A$.
\end{defn}

\begin{eg}
  The principal way to show that a type $A$ is decidable is to either construct an element $a:A$, or to construct a function $A\to\emptyt$. For example, the types $\unit$ and $\emptyt$ are decidable. Indeed, we have\index{decidable type!unit type}\index{unit type!is decidable}\index{decidable type!empty type}\index{empty type!is decidable}\index{decidable type!type with an element}
  \begin{align*}
    \inl(\ttt) & :\isdecidable(\unit) \\
    \inr(\idfunc) & : \isdecidable(\emptyt).
  \end{align*}
  Furthermore, any type $A$ equipped with an element $a:A$ is decidable because we have $\inl(a):\isdecidable(A)$ for such $A$.
\end{eg}

\begin{eg}\label{eg:decidability-closure}
  The principal way to use a hypothesis that $A$ is decidable is to proceed by the induction principle of coproducts, i.e., to proceed by case analysis.
  
  For example, if $A$ and $B$ are decidable types, then the types $A+B$, $A\times B$, and $A\to B$ are also decidable. This is straightforward to prove directly by pattern-matching on the variables of type $\isdecidable(A)$ and $\isdecidable(B)$. When we go through these proofs, the familiar truth table emerges:
  \begin{center}
    \begin{tabular}{lllll}
      \toprule
      \multicolumn{5}{c}{$\isdecidable$} \\ \cmidrule{1-5}
      $A$ & $B$ & $A+B$ & $A\times B$ & $A\to B$ \\
      \midrule
      $\inl(a)$ & $\inl(b)$ & $\inl(\inl(a))$ & $\inl(a,b)$ & $\inl(\lam{x}b)$ \\
      $\inl(a)$ & $\inr(g)$ & $\inl(\inl(a))$ & $\inr(g\circ \proj 2)$ & $\inr(\lam{h}g(h(a)))$ \\
      $\inr(f)$ & $\inl(b)$ & $\inl(\inr(b))$ & $\inr(f\circ\proj 1)$ & $\inl(\exfalso\circ f)$ \\
      $\inr(f)$ & $\inr(g)$ & $\inr[f,g]$ & $\inr(f\circ\proj 1)$ & $\inl(\exfalso\circ f)$ \\
      \bottomrule
    \end{tabular}
  \end{center}
  Since $A\to B$ is decidable whenever both $A$ and $B$ are decidable, it also follows that the negation $\neg A$ of any decidable type $A$ is decidable.
\end{eg}

\begin{eg}\label{eg:is-decidable-EqN}
  Since the empty type and the unit type are both decidable types, it also follows that the types $\EqN(m,n)$, $m\leq n$ and $m<n$ are decidable for each $m,n:\N$. The proofs in each of the three cases is by induction on $m$ and $n$.

  For instance, to show that $\EqN(m,n)$ is decidable for each $m,n:\N$, we simply note that the types
  \begin{align*}
    \EqN(\zeroN,\zeroN) & \jdeq \unit \\
    \EqN(\zeroN,\succN(n)) & \jdeq \emptyt \\
    \EqN(\succN(m),\zeroN) & \jdeq \emptyt 
  \end{align*}
  are all decidable, and that the type $\EqN(\succN(m),\succN(n))\jdeq \EqN(m,n)$ is decidable by the inductive hypothesis.
\end{eg}

The fact that $\N$ has decidable observational equality also implies that equality itself is decidable on $\N$. This leads to the general concept of decidable equality, which is important in many results about decidability.

\begin{defn}
  We say that a type $A$ has \define{decidable equality}\index{decidable equality|textbf} if the identity type $x=y$ is decidable for every $x,y:A$. We will write\index{has-decidable-equality(A)@{$\hasdecidableequality(A)$}|textbf}
  \begin{equation*}
    \hasdecidableequality(A)\defeq \prd{x,y:A}\isdecidable(x=y).
  \end{equation*}
\end{defn}

Before we show that $\N$ has decidable equality, let us show that if $A\leftrightarrow B$ and $A$ is decidable, then $B$ must be decidable.

\begin{lem}\label{lem:is-decidable-iff}
  Consider two types $A$ and $B$, and suppose that $A\leftrightarrow B$. Then $A$ is decidable if and only if $B$ is decidable.
\end{lem}

\begin{proof}
  Since we have functions $f:A\to B$ and $g:B\to A$ by assumption, we obtain by \cref{prp:contravariant-neg} the functions
  \begin{align*}
    \tilde{f} & : \neg B\to\neg A \\
    \tilde{g} & : \neg A \to \neg B.
  \end{align*}
  By \cref{rmk:functor-coprod} we have therefore the functions
  \begin{align*}
    f+\tilde{g} & : (A+\neg A) \to (B+\neg B) \\
    g+\tilde{f} & : (B+\neg B) \to (A+\neg A).\qedhere
  \end{align*}
\end{proof}

\begin{prp}\label{prp:has-decidable-equality-N}
  Equality on the natural numbers is decidable.\index{natural numbers!decidable equality}\index{N@{$\N$}!has decidable equality}\index{has decidable equality!natural numbers}\index{decidable equality!of N@{of $\N$}}
\end{prp}

\begin{proof}
  Recall from \cref{prp:Eq-eq-N} that we have
  \begin{equation*}
    (m=n)\leftrightarrow \EqN(m,n).
  \end{equation*}
  The claim therefore follows by \cref{lem:is-decidable-iff}, since we have observed in \cref{eg:is-decidable-EqN} that $\EqN(m,n)$ is decidable for every $m,n:\N$.
\end{proof}

It is certainly not provable with the given rules of type theory that every type has decidable equality. In fact, we will show in \cref{thm:hedberg} that if a type has decidable equality, then it is a \emph{set}. However, it is also not provable that every set has decidable equality unless one assumes the \emph{law of excluded middle}. We will discuss this principle in \cref{sec:logic}. For now, it is important to remember that in order to use decidability, we must first \emph{prove that it holds}, and many familiar types do indeed have decidable equality.

\begin{prp}\label{prp:has-decidable-equality-Fin}
  The standard finite type $\Fin{k}$ has decidable equality for each $k:\N$.\index{Fin k@{$\Fin{k}$}!has decidable equality}\index{has decidable equality!Fin k@{$\Fin{k}$}}\index{decidable equality!of Fin k@{of $\Fin{k}$}}
\end{prp}

\begin{proof}
  Recall from \cref{ex:Eq-Fin} that we constructed an observational equality relation $\EqFin_k$ on $\Fin{k}$ for each $k:\N$, which satisfies
  \begin{equation*}
    (x=y)\leftrightarrow \EqFin_k(x,y).
  \end{equation*}
  The type $\EqFin_k(x,y)$ is decidable, since it is recursively defined using the decidable types $\emptyt$ and $\unit$.
\end{proof}

We can use the fact that the finite types $\Fin{k}$ have decidable equality to show that the divisibility relation on $\N$ is decidable. 

\begin{thm}\label{thm:is-decidable-div-N}
  For any $d,x:\N$, the type $d\mid x$ is decidable.\index{divisibility on N@{divisibility on $\N$}!is decidable}
\end{thm}

\begin{proof}
  Note that $0\mid x$ is decidable because $0\mid x$ if and only if $x=0$, which is decidable by \cref{prp:has-decidable-equality-N}. Therefore it suffices to show that $d+1\mid x$ is decidable.

  By \cref{thm:effective-mod-k} it follows that $d+1\mid x$ holds if and only if we have an identification $[x]_{d+1}=0$ in $\Fin{d+1}$. Therefore the claim follows from the fact that $\Fin{d+1}$ has decidable equality.
\end{proof}

\subsection{Constructions by case analysis}
\index{case analysis|(}
\index{with-abstraction|(}

A common way to construct functions and to prove properties about them is by case analysis. For example, a famous function of Collatz is specified by case analysis on whether $n$ is even or odd:
  \begin{equation*}
  \collatz(n) =
  \begin{cases}
    n/2 & \text{if $n$ is even}\\
    3n+1 & \text{if $n$ is odd.}
  \end{cases}
\end{equation*}
The Collatz function is of course uniquely determined by this specification, but it is important to note that there is a bit of work to be done in order to define the Collatz function according to the rules of dependent type theory. First we note that, since the Collatz function is specified by case analysis on whether $n$ is even or odd, we will have to use a dependent function witnessing the fact that every number is either even or odd. In other words, we will make use of the dependent function
\begin{equation*}
  d:\prd{n:\N}\isdecidable(2\mid n),
\end{equation*}
which we have by \cref{thm:is-decidable-div-N}. The type $\isdecidable(2\mid n)$ is the coproduct $(2\mid n)+(2\nmid n)$, so the idea is to proceed by case analysis on whether $d(n)$ is of the form $\inl(x)$ or $\inr(x)$, i.e., by the induction principle of coproducts. However, $d(n)$ is not a free variable of type $\isdecidable(2\mid n)$. Before we can proceed by induction, we must therefore first \emph{generalize} the element $d(n)$ to a free variable $y:\isdecidable(2\mid n)$. In other words, we will first define a function
\begin{equation*}
  h:\prd{n:\N} (\isdecidable(2\mid n)\to\N)
\end{equation*}
by the induction principle of coproducts, and then we obtain the Collatz function by substituting $d(n)$ for $y$ in $h(n,y)$. Putting these ideas together, we obtain the following type theoretical definition of the Collatz function.

\begin{defn}
  Write $d:\prd{n:\N}\isdecidable(2\mid n)$ for the function deciding $2\mid n$, given in \cref{thm:is-decidable-div-N}.
  \begin{enumerate}
  \item We define a function $h:\prd{n:\N}(\isdecidable(2\mid n)\to \N)$ by
    \begin{align*}
      h(n,\inl(m,p)) & \defeq m \\
      h(n,\inr(f)) & \defeq 3n+1.
    \end{align*}
  \item We define the \define{collatz function}\index{Collatz function|textbf} $\collatz:\N\to \N$ by\index{collatz@{$\collatz$}|textbf}
    \begin{equation*}
      \collatz(n)\defeq h(n,d(n)).
    \end{equation*}
  \end{enumerate}
\end{defn}

\begin{rmk}
  The general ideas behind the formal construction of the Collatz function lead to the type theoretic concept of \emph{with-abstraction}. With-abstraction is a type-theoretically precise generalization of case analysis.
  
  In full generality, if our goal is to define a dependent function $f:\prd{x:A}C(x)$, and we already have a function $g:\prd{x:A}B(x)$, then it suffices to define a dependent function
  \begin{equation*}
    h:\prd{x:A}B(x)\to C(x).
  \end{equation*}
  Indeed, given $g$ and $h$ as above, we can define $f\defeq \lam{x}h(x,g(x))$. In other words, to define $f(x)$ using $g(x):B(x)$, we generalize $g(x)$ to an arbitrary element $y:B(x)$ and proceed to define an element $h(x,y):C(x)$.

  With-abstraction is a concise way to present such a definition. In a definition by with-abstraction, we may write
  \begin{equation*}
    f(x)\with [g(x)/y]\defeq h(x,y),
  \end{equation*}
  to define a function $f:\prd{x:A}C(x)$ that satisfies the judgmental equality $f\jdeq \lam{x}h(x,g(x))$. In other words, $f(x)$ is defined to be $h(x,y)$ with $g(x)$ for $y$.
  
  The definition of the Collatz function can therefore be given by with-abstraction as
  \begin{equation*}
    \collatz(n)\with [d(n)/y]\defeq h(x,y).
  \end{equation*}
  However, recall that the function $h$ was defined by pattern matching on $y$. We can combine with-abstraction and pattern matching to obtain a \emph{direct} definition of the Collatz function that doesn't explicitly mention the function $h$ anymore. This gives us the following concise way to define the Collatz function:
  \begin{align*}
    \collatz(n)\with [d(n)/\inl(m,p)] & \defeq m \\
    \collatz(n)\with [d(n)/\inr(f)] & \defeq 3n+1.
  \end{align*}
  Notice that in addition to the information in the specification of the Collatz function, the definition by with-abstraction also tells us which decision procedure was used to decide whether $n$ is even or not. The combination of with-abstraction and pattern matching, which allows us to skip the explicit definition of the function $h$, is what makes with-abstraction so useful.
  \end{rmk}
  
  Using with-abstraction we can find a slight improvement of the decidability results of $A\to B$ and $A\times B$ in \cref{eg:decidability-closure}, and we will use these improved claims in the construction of the greatest common divisor.

\begin{prp}\label{prp:is-decidable-function-type}
  Consider a decidable type $A$, and let $B$ be a type equipped with a function
  \begin{equation*}
    A\to\isdecidable(B).
  \end{equation*}
  Then the types $A\times B$ and $A\to B$ are also decidable.
\end{prp}

\begin{proof}
  We only prove the claim about the decidability of $A\to B$, since the claim about the decidability of $A\times B$ is proven similarly. Since $A$ is assumed to be decidable, we proceed by case analysis on $A+\neg A$. In the case where we have $f:\neg A$, we have the functions
  \begin{equation*}
    \begin{tikzcd}[column sep=large]
      A \arrow[r,"f"] & \emptyt \arrow[r,"\exfalso"] & B.
    \end{tikzcd}
  \end{equation*}
  Therefore we obtain the element $\inl(\exfalso\circ f):\isdec(A\to B)$. In the case where we have an element $a:A$, we have to construct a function
  \begin{equation*}
    d:(A\to\isdecidable(B))\to\isdecidable(A\to B)
  \end{equation*}
  Given $H:A\to\isdecidable(B)$, we can use with-abstraction to proceed by case analysis on $H(a):B+\neg B$. The function $d$ is therefore defined as
  \begin{align*}
    d(H)\with [H(a)/\inl(b)] & \defeq \inl(\lam{x}b) \\
    d(H)\with [H(a)/\inr(g)] & \defeq \inr(\lam{h}g(h(a))).\qedhere
  \end{align*}
\end{proof}

For a general family of decidable types $P$ over $\N$, we cannot prove that the type
\begin{equation*}
  \prd{x:\N}P(x)
\end{equation*}
is decidable. However, if we know in advance that $P(x)$ holds for any $m\leq x$, then we can decide $\prd{x:\N}P(x)$ by checking the decidability of each $P(x)$ until $m$. 

\begin{prp}\label{prp:is-decidable-pi-type}
  Consider a decidable type family $P$ over $\N$ equipped with a natural number $m$ such that the type
  \begin{equation*}
    \prd{x:\N}(m\leq x)\to P(x)
  \end{equation*}
  is decidable. Then the type $\prd{x:\N}P(x)$ is decidable. 
\end{prp}

\begin{proof}
  Our proof is by induction on $m$, but we will first make sure that the inductive hypothesis will be strong enough by quantifying over all decidable type families over $\N$. Of course, we cannot do this directly. However, by the assumption that there are enough universes (\cref{enough-universes}), there is a universe $\UU$ that contains $P$. We fix this universe, and we will prove by induction on $m$ that for every decidable type family $Q:\N\to\UU$ for which the type
  \begin{equation*}
    \prd{x:\N}(m\leq x)\to Q(x),
  \end{equation*}
  is decidable, the type $\prd{x:\N}Q(x)$ is again decidable.

  In the base case, it follows by assumption that the type $\prd{x:A}Q(x)$ is decidable. For the inductive step, let $Q:\N\to\UU$ be a decidable type family for which the type
  \begin{equation*}
    \prd{x:\N}(m+1\leq x)\to Q(x)
  \end{equation*}
  is decidable. Since $Q$ is assumed to be decidable, we can proceed by case analysis on $Q(0)+\neg Q(0)$. In the case of $\neg Q(0)$, it follows that $\neg \prd{x:\N}Q(x)$. In the case where we have $q:Q(0)$, consider the type family $Q':\N\to\UU$ given by
  \begin{equation*}
    Q'(x)\defeq Q(x+1).
  \end{equation*}
  Then $Q'$ is decidable since $Q$ is decidable, and moreover it follows that the type $\prd{x:\N} ({m\leq x})\to Q'(x)$ is decidable. The inductive hypothesis implies therefore that the type $\prd{x:\N}Q'(x)$ is decidable. In the case where $\neg\prd{x:\N}Q'(x)$, it follows that $\neg\prd{x:\N}Q(x)$, and in the case where we have a function $g:\prd{x:\N}Q'(x)$, we can construct a function $f:\prd{x:\N}Q(x)$ by
  \begin{align*}
    f(0) & \defeq q \\
    f(x+1) & \defeq g(x)\qedhere.
  \end{align*}
\end{proof}

\begin{cor}\label{cor:is-decidable-bounded-pi}
  Consider two decidable families $P$ and $Q$ over $\N$, and suppose that $P$ comes equipped with an upper bound $m$. Then the type
  \begin{equation*}
    \prd{n:\N}P(n)\to Q(n)
  \end{equation*}
  is decidable.
\end{cor}

\begin{proof}
  Since $m$ is assumed to be an upper bound for $P$, it follows $P(n)\to Q(n)$ for any $m\leq n$. With this observation we apply \cref{prp:is-decidable-pi-type}.
\end{proof}
\index{case analysis|)}
\index{with-abstraction|)}

\subsection{The well-ordering principle of \texorpdfstring{$\N$}{ℕ}}
\index{well-ordering principle of N@{well-ordering principle of $\N$}|(}
\index{natural numbers!well-ordering principle|(}

The well-ordering principle of the natural numbers in classical mathematics asserts that any nonempty subset of $\N$ has a least element. To formulate the well-ordering principle in type theory, we will use type families over $\N$ instead of subsets of $\N$. Moreover, the classical well-ordering principle tacitly assumes that subsets are decidable. The type theoretic well-ordering principle of $\N$ is therefore formulated using \emph{decidable} families over $\N$.

\begin{defn}
  Let $P$ be a family over $\N$, not necessarily decidable.
  \begin{enumerate}
  \item We say that a natural number $n$ is a \define{lower bound}\index{lower bound|textbf} for $P$ if it comes equipped with an element of type\index{is-lower-bound P (n)@{$\islowerbound_P(n)$}|textbf}
    \begin{equation*}
      \islowerbound_P(n)\defeq \prd{x:\N}P(x)\to (n\leq x).
    \end{equation*}
  \item We say that a natural number $n$ is an \define{upper bound}\index{upper bound|textbf} for $P$ if it comes equipped with an element of type\index{is-upper-bound P (n)@{$\isupperbound_P(n)$}|textbf}
    \begin{equation*}
      \isupperbound_P(n)\defeq \prd{x:\N}P(x)\to (x\leq n).
    \end{equation*}
  \end{enumerate}
\end{defn}

  A minimal element of $P$ is therefore a natural number $n$ for which $P(n)$ holds, and which is also a lower bound for $P$. The well-ordering principle of $\N$ asserts that such an element exists for any decidable family $P$, as soon as $P(n)$ holds for some $n$.

  \begin{thm}[Well-ordering principle of $\N$]\label{thm:well-ordering-principle-N}\index{well-ordering principle of N@{well-ordering principle of $\N$}|textbf}\index{natural numbers!well-ordering principle|textbf}
    Let $P$ be a decidable family over $\N$, where $d$ witnesses that $P$ is decidable. Then there is a function\index{well-ordering-principle P d@{$\wellorderingprinciple(P,d)$}|textbf}
  \begin{equation*}
    \wellorderingprinciple(P,d):\Big(\sm{n:\N}P(n)\Big)\to\Big(\sm{m:\N}P(m)\times\islowerbound_P(m)\Big).
  \end{equation*}
\end{thm}

\begin{proof}
  By the assumption that there are enough universes (\cref{enough-universes}), there is a universe $\UU$ that contains $P$. Instead of proving the claim for the given type family $P$, we will show by induction on $n:\N$ that there is a function
  \begin{equation*}\label{eq:well-ordering}
    Q(n)\to \Big(\sm{m:\N}Q(m)\times\islowerbound_Q(m)\Big)\tag{\textasteriskcentered}
  \end{equation*}
  for every decidable family $Q:\N\to\UU$. Note that we are now also quantifying over the decidable families $Q:\N\to\UU$. This slightly strengthens the inductive hypothesis, which we will be able to exploit.

  The base case is trivial, since $\zeroN$ is a lower bound of every type family over $\N$. For the inductive step, assume that \cref{eq:well-ordering} holds for every decidable type family $Q:\N\to \UU$. Furthermore, let $Q:\N\to\UU$ be a decidable type family equipped with an element $q:Q(\succN(n))$. Our goal is to construct an element of type
  \begin{equation*}
    \sm{m:\N}Q(m)\times\islowerbound_Q(m).
  \end{equation*}
  Since $Q(\zeroN)$ is assumed to be decidable, it suffices to construct a function
  \begin{equation*}
    (Q(\zeroN)+\neg Q(\zeroN))\to \sm{m:\N}Q(m)\times\islowerbound_Q(m).
  \end{equation*}
  Therefore we can proceed by case analysis on $Q(\zeroN)+\neg Q(\zeroN)$. In the case where we have an element of type $Q(\zeroN)$, it follows immediately that $\zeroN$ must be minimal. In the case where $\neg Q(\zeroN)$, we consider the decidable subset $Q'$ of $\N$ given by
  \begin{equation*}
    Q'(n)\defeq Q(\succN(n)).
  \end{equation*}
  Since we have $q:Q'(n)$, we obtain a minimal element in $Q'$ by the inductive hypothesis. Of course, by the assumption that $Q(\zeroN)$ doesn't hold, the minimal element of $Q'$ is also the minimal element of $Q$.
\end{proof}
\index{well-ordering principle of N@{well-ordering principle of $\N$}|)}
\index{natural numbers!well-ordering principle|)}

\subsection{The greatest common divisor}
\index{natural numbers!greatest common divisor|(}
\index{greatest common divisor|(}

The greatest common divisor of two natural numbers $a$ and $b$ is a natural number $\gcd(a,b)$ that satisfies the property that
\begin{equation*}
  x\mid a\ \text{and}\ x\mid b\qquad\text{if and only if}\qquad x\mid\gcd(a,b)
\end{equation*}
for any $x:\N$. In other words, any number $x:\N$ that divides both $a$ and $b$ also divides the greatest common divisor. Moreover, since $\gcd(a,b)$ divides itself, it follows from the reverse implication that $\gcd(a,b)$ divides both $a$ and $b$.

This property can also be seen as the \emph{specification} of what it means to be a greatest common divisor of $a$ and $b$. In formal developments of mathematics, when you're about to construct an object that satisfies a certain specification, it can be useful to start out with that specification. For example, there is more than one way to define the greatest common divisor. We will define it here in \cref{defn:gcd} using the well-ordering principle, but an alternative definition using Euclid's algorithm is of course just as good, since both definitions satisfy the specification that uniquely characterizes it. Hence we make the following specification of the greatest common divisor.

\begin{defn}\label{defn:is-gcd}
  Consider three natural numbers $a$, $b$, and $d$. We say that $d$ is a \define{greatest common divisor}\index{is a greatest common divisor|textbf}\index{greatest common divisor|textbf} of $a$ and $b$ if it comes equipped with an element of type\index{is-gcd a b d@{$\isgcd_{a,b}(d)$}|textbf}
  \begin{equation*}
    \isgcd_{a,b}(d) \defeq \prd{x:\N} (x\mid a)\times (x\mid b)\leftrightarrow (x\mid d).
  \end{equation*}
\end{defn}

The property of being a greatest common divisor uniquely characterizes the greatest common divisor, in the following sense.

\begin{prp}
  Suppose $d$ and $d'$ are both a greatest common divisor of $a$ and $b$. Then $d=d'$.
\end{prp}

\begin{proof}
  If both $d$ and $d'$ are a greatest common divisor of $a$ and $b$, then both $d$ and $d'$ divide both $a$ and $b$, and hence it follows that $d\mid d'$ and $d'\mid d$. Since the divisibility relation was shown to be a partial order in \cref{ex:is-poset-div}, it follows by antisymmetry that $d=d'$.
\end{proof}

Note that for any two natural numbers $a$ and $b$, the type
\begin{equation*}\label{eq:multiples-of-gcd}
  \sm{n:\N}\prd{x:\N} (x\mid a)\times (x\mid b)\to (x\mid n)\tag{\textasteriskcentered}
\end{equation*}
consists of all the multiples of the common divisors of $a$ and $b$, including $0$. On the other hand, the type
\begin{equation*}\label{eq:common-divisors}
  \sm{n:\N}\prd{x:\N} (x\mid n)\to (x\mid a)\times (x\mid b)\tag{\textasteriskcentered\textasteriskcentered}
\end{equation*}
consists of all the common divisors of $a$ and $b$ except in the case where $a=0$ and $b=0$. In this case, the type in \cref{eq:common-divisors} consists of all natural numbers.

\cref{eq:multiples-of-gcd,eq:common-divisors} provide us with two ways to define the greatest common divisor. We can either define the greatest common divisor of $a$ and $b$ as the greatest natural number in the type in \cref{eq:common-divisors} or we can define it as the least \emph{nonzero} natural number of the type in \cref{eq:multiples-of-gcd}, provided that we make an exception in the case where both $a=0$ and $b=0$. Since we already have established the well-ordering principle of $\N$, we will opt for the second approach. In \cref{ex:maximal-element} you will be asked to show that any \emph{bounded} decidable family over $\N$ has a maximum as soon as it contains some natural number. 

In order to correctly define the greatest common divisor using well-ordering principle of $\N$, we need a slight modification of the type family in \cref{eq:multiples-of-gcd}. We define this family as follows:

\begin{defn}\label{defn:fam-gcd}
  Given $a,b:\N$, we define the type family $\ismultipleofgcd(a,b)$ over $\N$ by
  \begin{align*}
    \ismultipleofgcd(a,b,n) & \defeq (a+b\neq 0) \to (n\neq 0)\times \Big(\prd{x:\N} (x\mid a)\times (x\mid b) \to (x\mid n)\Big).
  \end{align*}
\end{defn}

In other words, if $a+b=0$ then the type $\sm{n:\N}M(a,b,n)$ consist of all the natural numbers. On the other hand, if $a+b\neq 0$ it consists of the nonzero natural numbers $n$ with the property that any common divisor of $a$ and $b$ also divides $n$. These are exactly the nonzero multiples of the greatest common divisor of $a$ and $b$.

Since we intend to apply the well-ordering principle, we must show that the family $\ismultipleofgcd(a,b)$ is decidable. This is a step that one can skip in classical mathematics, because all the subsets of $\N$ are decidable there. However, in our current setting we have no choice but to prove it.

\begin{prp}\label{prp:is-decidable-is-multiple-of-gcd}
  The type family $\ismultipleofgcd(a,b)$ is decidable for each $a,b:\N$.
\end{prp}

\begin{proof}
  The type $a+b\neq 0$ is decidable because it is the negation of the type $a+b=0$, which is decidable by \cref{prp:has-decidable-equality-N}. Therefore it suffices to show that the type
  \begin{equation*}
    (n\neq 0)\times \prd{x:\N} (x\mid a)\times (x\mid b)\to (x\mid n)
  \end{equation*}
  is decidable, and by \cref{prp:is-decidable-function-type} we also get to assume that $a+b\neq 0$. The type $n\neq 0$ is again decidable by \cref{prp:has-decidable-equality-N}, so it suffices to show that the type
  \begin{equation*}
    \prd{x:\N}(x\mid a)\times (x\mid b)\to (x\mid n)
  \end{equation*}
  is decidable. The types $(x\mid a)\times(x\mid b)$ and $(x\mid n)$ are decidable by \cref{thm:is-decidable-div-N}, so by \cref{cor:is-decidable-bounded-pi} it suffices to check that the family of types $(x\mid a)\times (x\mid b)$ indexed by $x:\N$ has an upper bound. If $x$ is a common divisor of $a$ and $b$, then it follows that $x$ divides $a+b$. Furthermore, since we have assumed that $a+b\neq 0$, it follows that $x\leq a+b$. This provides the upper bound.
  \end{proof}

We are almost in position to apply the well-ordering principle of $\N$ to define the greatest common divisor. It just remains to show that there is some $n:\N$ for which $M(a,b,n)$ holds. We prove this in the following lemma.

\begin{lem}\label{lem:exists-multiple-of-gcd}
  There is an element of type $\ismultipleofgcd(a,b,a+b)$. 
\end{lem}

\begin{proof}
  To construct an element of type $\ismultipleofgcd(a,b,a+b)$, assume that $a+b\neq 0$. Then we have tautologically that $a+b\neq 0$, and any common divisor of $a$ and $b$ is also a divisor of $a+b$.
\end{proof}

\begin{defn}\label{defn:gcd}
  We define the \define{greatest common divisor}\index{greatest common divisor|textbf} $\gcd:\N\to (\N\to\N)$\index{gcd@{$\gcd$}|textbf} by the well-ordering principle of $\N$ (\cref{thm:well-ordering-principle-N}) as the least natural number $n$ for which $M(a,b,n)$ holds, using the fact that $M(a,b)$ is a decidable type family (\cref{prp:is-decidable-is-multiple-of-gcd}) and that $M(a,b,a+b)$ always holds (\cref{lem:exists-multiple-of-gcd}).
\end{defn}

\begin{lem}\label{lem:is-zero-gcd}
  For any two natural numbers $a$ and $b$, we have $\gcd(a,b)=0$ if and only if $a+b=0$.
\end{lem}

\begin{proof}
  To prove the forward direction, assume that $\gcd(a,b)=0$. By definition of $\gcd(a,b)$ we have that $\ismultipleofgcd(a,b,\gcd(a,b))$ holds. More explicitly, the implication
  \begin{equation*}
    (a+b\neq 0)\to (\gcd(a,b)\neq 0)\times \prd{x:\N}(x\mid a)\times(x\mid b)\to (x\mid\gcd(a,b))
  \end{equation*}
  holds. However, we have assumed that $\gcd(a,b)=0$, so it follows from the above implication that $\neg(a+b\neq 0)$. In other words, we have $\neg\neg(a+b=0)$. The fact that equality on $\N$ is decidable implies via \cref{ex:dne-is-decidable} that $\neg\neg(a+b=0)\to (a+b=0)$, so we conclude that $a+b=0$.

  For the converse direction, recall that the inequality $\gcd(a,b)\leq a+b$ holds by minimality, since $\ismultipleofgcd(a,b,a+b)$ holds by \cref{lem:exists-multiple-of-gcd}. If $a+b=0$, it therefore follows that $\gcd(a,b)\leq 0$, which implies that $\gcd(a,b)=0$.
\end{proof}

\begin{thm}
  For any two natural numbers $a$ and $b$, the number $\gcd(a,b)$ is a greatest common divisor of $a$ and $b$ in the sense of \cref{defn:is-gcd}.
\end{thm}

\begin{proof}
  We give the proof by case analysis on whether $a+b=0$.
  If we assume that $a+b=0$, then it follows that both $a=0$ and $b=0$, and by \cref{lem:is-zero-gcd} it also follows that $\gcd(a,b)=0$. Since any number divides $0$, the claim follows immediately.

  In the case where $a+b\neq 0$, it follows from \cref{lem:is-zero-gcd} that also $\gcd(a,b)\neq 0$. From the fact that $\ismultipleofgcd(a,b,\gcd(a,b))$ we therefore immediately obtain that
  \begin{equation*}
    \prd{x:\N} (x\mid a)\times (x\mid b)\to (x\mid \gcd(a,b)).
  \end{equation*}
  Therefore it remains to show that if $x$ divides $\gcd(a,b)$, then $x$ divides both $a$ and $b$. By transitivity of the divisibility relation it suffices to show that $\gcd(a,b)$ divides both $a$ and $b$. We will show only that $\gcd(a,b)$ divides $a$, the proof that $\gcd(a,b)$ divides $b$ is similar.

  Since $\gcd(a,b)$ is nonzero, it follows by Euclidean division (\cref{ex:euclidean-division}) that there are numbers $q$ and $r<\gcd(a,b)$ such that
  \begin{equation*}
    a = q\cdot\gcd(a,b)+r.
  \end{equation*}
  From this equation and \cref{prp:div-3-for-2} it follows that any number $x$ which divides both $a$ and $b$ also divides $r$, because we have already noted that any such $x$ divides $\gcd(a,b)$. This observation implies that $r=0$, because we have $r<\gcd(a,b)$ by construction and $\gcd(a,b)$ is minimal. Therefore we conclude that $\gcd(a,b)$ divides $a$.
\end{proof}
\index{natural numbers!greatest common divisor|)}
\index{greatest common divisor|)}

\subsection{The infinitude of primes}
\index{prime number|(}
\index{natural number!prime number|(}

When the natural numbers are ordered by the divisibility relation, the number $1$ is at the bottom. Directly above $1$ are the prime numbers. Above the prime numbers are the multiples of two primes, then the multiples of three primes, and so on. At the top of this ordering we find $0$. For any natural number $n$, the numbers strictly below $n$ are the proper divisors of $n$. A prime number is therefore a number of which has exactly one proper divisor.

\begin{defn}
  ~
  \begin{enumerate}
  \item Consider two natural numbers $d$ and $n$. Then $d$ is said to be a \define{proper divisor}\index{proper divisor|textbf} of $n$ if it comes equipped with an element of type\index{is-proper-divisor n d@{$\isproperdivisor(n,d)$}|textbf}
    \begin{equation*}
      \isproperdivisor(n,d)\defeq (d\neq n)\times (d\mid n).
    \end{equation*}
  \item A natural number $n$ is said to be \define{prime}\index{prime number|textbf}\index{natural numbers!prime number|textbf} if it comes equipped with an element of type\index{is-prime(n)@{$\isprime(n)$}|textbf}
  \begin{equation*}
    \isprime(n)\defeq \prd{x:\N}\isproperdivisor(n,x)\leftrightarrow (x=1).
  \end{equation*}
  \end{enumerate}
\end{defn}

\begin{prp}
  For any $n:\N$, the type $\isprime(n)$ is decidable.\index{is-prime(n)@{$\isprime(n)$}!is decidable}
\end{prp}

\begin{proof}
  We will first show that $\isprime(n)\leftrightarrow\isprime'(n)$, where
  \begin{equation*}
    \isprime'(n)\defeq (n\neq 1)\times \prd{x:\N}\isproperdivisor(n,x)\to (x=1).
  \end{equation*}
  For the forward direction, simply note that $1$ is not a proper divisor of itself, and therefore $1$ is not a prime. For the converse direction, suppose that $n\neq 1$ and that any proper divisor of $n$ is $1$. Then it follows that $1$ is a proper divisor of $n$, which implies that $n$ is prime.

  Now we proceed by showing that the type $\isprime'(n)$ is decidable for every $n:\N$. The proof is by case analysis on whether $n=0$ or $n\neq 0$. In the case where $n=0$, note that any nonzero number is a proper divisor of $0$, and therefore $\isprime'(0)$ doesn't hold. In particular, $\isprime'(0)$ is decidable.
  
  Now suppose that $n\neq 0$. In order to show that the type $\isprime'(n)$ is decidable, note that the type $n\neq 1$ is decidable since it is the negation of the decidable type $n=1$. Therefore it suffices to show that the type
  \begin{equation*}
    \prd{x:\N}\isproperdivisor(n,x)\to (x=1)
  \end{equation*}
  is decidable. Since the types $(x\neq n)\times (x\mid n)$ and $x=1$ are decidable, it follows from \cref{cor:is-decidable-bounded-pi} that it suffices to check that
  \begin{equation*}
    ((x\neq n)\times (x\mid n))\to (x\leq n)
  \end{equation*}
  for any $x:\N$. This follows from the implication $(x\mid n)\to (x\leq n)$, which holds because we have assumed that $n\neq 0$.
\end{proof}

The proof that there are infinitely many primes proceeds by constructing a prime number larger than $n$, for any $n:\N$. The number $n!+1$ is relatively prime with any number $x\leq n$. Therefore there is a least number $n<m$ that is relatively prime with any number $x\leq n$, and it follows that this number $m$ must be prime.

\begin{defn}
  For any two natural numbers $n$ and $m$, we define the type
  \begin{equation*}
    R(n,m)\defeq (n<m)\times \prd{x:\N}(x\leq n)\to ((x\mid m)\to (x=1)). 
  \end{equation*}
\end{defn}

\begin{lem}
  The type $R(n,m)$ is decidable for each $n,m:\N$.
\end{lem}

\begin{proof}
  The type $n<m$ and, and for each $x:\N$ both types $x\leq n$ and $(x\mid m)\to (x=1)$ are decidable, so it follows via \cref{cor:is-decidable-bounded-pi} that the product
  \begin{equation*}
    \prd{x:\N}(x\leq n)\to ((x\mid m)\to (x=1))
  \end{equation*}
  is decidable.
\end{proof}

\begin{lem}\label{lem:succ-factorial-has-one-bounded-divisor}
  There is an element of type $R(n,{n!}+1)$ for each $n:\N$.
\end{lem}

\begin{proof}
  The fact that $n<{n!}+1$ follows from the fact that $n\leq n!$, which is shown by induction. We leave this to the reader, and focus on the second aspect of the claim: that every $x\leq n$ that divides ${n!}+1$ must be equal to $1$.

  To see this, note that any divisor of ${n!}+1$ is automatically nonzero, and recall that any nonzero $x\leq n$ divides $n!$ by \cref{ex:div-factorial}. Therefore it follows that any $x\leq n$ that divides ${n!}+1$ also divides $n!$, and consequently it divides $1$ as well. Now we are done, because if $x$ divides $1$ then $x=1$.
\end{proof}

We finally show that there are infinitely many primes.

\begin{thm}
  For each $n:\N$, there is a prime number $p:\N$ such that $n< p$.\index{infinitude of primes}\index{prime number!infinitude of primes}\index{natural numbers!infinitude of primes}
\end{thm}

\begin{proof}
  It suffices to show that for each \emph{nonzero} $n:\N$, there is a prime number $p:\N$ such that $n\leq p$. Let $n$ be a nonzero natural number.

  Since the type $R(n,m)$ is decidable for each $m:\N$, and since $R(n,{n!}+1)$ holds by \cref{lem:succ-factorial-has-one-bounded-divisor}, it follows by the well-ordering principle of $\N$ (\cref{thm:well-ordering-principle-N}) that there is a minimal $m:\N$ such that $R(n,m)$ holds. In order to prove the theorem, we will show that this number $m$ is prime, i.e., that there is an element of type
  \begin{equation*}
    (m\neq 1)\times \prd{x:\N} \isproperdivisor(m,x)\to (x=1).
  \end{equation*}

  First, we note that $m\neq 1$ because $n<m$ holds by construction, and $n$ is assumed to be nonzero. Therefore it suffices to show that $1$ is the only proper divisor of $m$. Let $x$ be a proper divisor of $m$. Since $R(n,m)$ holds by construction, we will prove that $x=1$ by showing that $x\leq n$ holds.

  Since $m$ is nonzero, it follows from the assumption that $x\mid m$ that $x<m$. By minimality of $m$, it therefore follows that $\neg R(n,x)$ holds. However, any divisor of $x$ is also a divisor of $m$ by transitivity of the divisibility relation. Therefore it follows that any $y\leq n$ that divides $x$ must be $1$. In other words:
  \begin{equation*}
    \prd{y:\N}(y\leq n)\to ((y\mid x)\to (y=1))
  \end{equation*}
  holds. Since $\neg R(n,x)$ holds, we conclude now that $n\nless x$. To finish the proof, it follows that $x\leq n$.
\end{proof}
\index{prime number|)}
\index{natural number!prime number|)}

\subsection{Boolean reflection}\label{sec:boolean-reflection}
\index{boolean reflection|(}

We have shown that the type $\isprime(n)$ is decidable for every $n$. In other words, there is an element $d(n):\isdecidable(\isprime(n))$ for every $n$. In principle, we can therefore check whether any \emph{specific} natural number $n$ is prime by inspecting the element $d(n)$: if it is of the form $\inl(x)$ for some $x:\isprime(n)$, then $n$ is prime; if it is of the form $\inr(f)$ for some $f:\neg\isprime(n)$, then $n$ is not prime. In other words, we evaluate the element $d(n)$ using the computation rules of type theory, and then we see whether $n$ is prime or not.

Computers can perform such evaluations, but it is often unfeasible to carry out such evaluations by hand. Moreover, even for computers the task of evaluating a proof term like $\isdecidableisprime(n)$ may quickly get out of hand. With the formalization of the material in this book, the proof assistant Agda returns a proof term of 430 lines of code when we simply ask it to evaluate the term $\isdecidableisprime(7)$, and it returned a proof term of 69373 lines of code when we asked it to evaluate the term $\isdecidableisprime(37)$. There is a much better way to do this: \emph{boolean reflection}.

\begin{defn}
  For any type $A$ we define the map\index{booleanization@{$\booleanization$}|textbf}
  \begin{equation*}
    \booleanization:\isdecidable(A)\to\bool
  \end{equation*}
  by
  \begin{align*}
    \booleanization(\inl(a)) & \defeq \btrue \\*
    \booleanization(\inr(f)) & \defeq \bfalse.
  \end{align*}
\end{defn}

\begin{thm}[Boolean reflection principle]
  For any type $A$ and any decision $d:\isdecidable(A)$, there is a map\index{boolean reflection|textbf}\index{reflect@{$\booleanreflection$}|textbf}
  \begin{equation*}
    \booleanreflection:(\booleanization(d)=\btrue)\to A
  \end{equation*}
  such that $\booleanreflection(\inl(a))\jdeq a$.
\end{thm}

\begin{proof}
  First, recall that by \cref{ex:obs_bool} there is a map $\gamma:(\bfalse=\btrue)\to \emptyt$. We use this to construct $\booleanreflection$ by pattern matching as follows:
  \begin{align*}
    \booleanreflection(\inl(a),p) & \defeq a \\*
    \booleanreflection(\inr(f),p) & \defeq \exfalso(\gamma(p)).\qedhere
  \end{align*}
\end{proof}

\begin{rmk}
  Since the number 37 is a prime, it follows that the booleanization of the term
  \begin{equation*}
    d(37):\isdecidable(\isprime(37))
  \end{equation*}
  has the value $\booleanization(d(37))\jdeq\btrue$. By boolean reflection it therefore follows that
  \begin{equation}\label{eq:is-prime-37}
    \isprimethirtyseven\defeq \booleanreflection(d(37),\refl{}):\isprime(37).\tag{\textasteriskcentered}
  \end{equation}
  The term in $\isprimethirtyseven$ does not, however, contain any explicit information as to why the number 37 is prime. The reason that it type checks is simply that $d(37)$ is judgmentally equal to some term of the form $\inl(t):\isdecidable(\isprime(37))$ and therefore it follows that $\refl{}$ is an identification of type
  \begin{equation*}
    \booleanization(d(37))=\btrue.
  \end{equation*}
  To see that $\isprimethirtyseven$ is indeed an element of type $\isprime(37)$ therefore requires us to evaluate the term $d(37)$. This is not doable by hand. Computer proof assistants, however, are capable of performing this task. In a proof assistant, we may therefore use boolean reflection to offload the task of evaluating the decision algorithm of a decidable type to the computer. This technique has been essential in the formalization of the Feit-Thompson theorem in Coq \cite{Gonthier}. The book \emph{Mathematical Components} \cite{mathematical-components} contains more information about using boolean reflection effectively in formalized mathematics.

  Do not, however, "solve" your homework problems with boolean reflection. If your teaching assistant cannot evaluate your solution, they will conclude that you haven't demonstrated your clear understanding of the problem.
\end{rmk}
\index{boolean reflection|)}

\begin{exercises}
  \exitem
  \begin{subexenum}
  \item State Goldbach's conjecture\index{Goldbach's conjecture} in type theory.
  \item State the twin prime conjecture\index{twin prime conjecture} in type theory.
  \item State the Collatz conjecture\index{Collatz conjecture} in type theory.
  \end{subexenum}
  \noindent If you have a solution to any of these open problems, you should certainly formalize it before you submit it to the Annals of Mathematics.
  \exitem Show that
  \begin{equation*}
    \isdecidable(\isdecidable(P))\to\isdecidable(P)
  \end{equation*}
  for any type $P$.
  \exitem For any family $P$ of decidable types indexed by $\Fin{k}$, construct a function
  \begin{equation*}
    \neg\Big(\prd{x:\Fin{k}}P(x)\Big)\to\sm{x:\Fin{k}}\neg P(x).
  \end{equation*}
  \exitem
  \begin{subexenum}
  \item Define the \define{prime function} $\primefunction:\N\to\N$\index{prime@{$\primefunction$}|see {prime function}}\index{prime function|textbf}\index{natural numbers!prime function} for which $\primefunction(n)$ is the $n$-th prime.

  \item Define the \define{prime-counting function}\index{prime counting function|textbf} $\pi:\N\to\N$\index{p@{$\pi$}|see {prime counting function}}, which counts for each $n:\N$ the number of primes $p\leq n$.
  \end{subexenum}
  \exitem For any natural number $n$, show that
  \begin{equation*}
    \isprime(n)\leftrightarrow (2\leq n)\times \prd{x:\N} (x\mid n)\to (x=1)+(x=n).
  \end{equation*}
  \exitem Consider two types $A$ and $B$. Show that the following are equivalent:
  \begin{enumerate}
  \item There are functions
    \begin{align*}
      & B \to \hasdecidableequality(A) \\
      & A \to \hasdecidableequality(B).
    \end{align*}
  \item The product $A\times B$ has decidable equality.\index{decidable equality!of cartesian products}
  \end{enumerate}
  Conclude that if both $A$ and $B$ have decidable equality, then so does $A\times B$.
  \exitem Consider two types $A$ and $B$, and consider the observational equality $\Eqcoprod$ on the coproduct $A+B$ defined by
  \begin{align*}
    \Eqcoprod(\inl(x),\inl(x')) & \defeq x= x' & \Eqcoprod(\inl(x),\inr(y')) & \defeq \emptyt \\
    \Eqcoprod(\inr(y),\inl(x')) & \defeq \emptyt & \Eqcoprod(\inr(y),\inr(y')) & \defeq y = y'.
  \end{align*}
  \begin{subexenum}
  \item Show that $(x=y)\leftrightarrow\Eqcoprod(x,y)$ for every $x,y:A+B$.
  \item Show that the following are equivalent:
    \begin{enumerate}
    \item Both $A$ and $B$ have decidable equality.\index{decidable equality!of coproducts}
    \item The coproduct $A+B$ has decidable equality.
    \end{enumerate}
    Conclude that $\Z$ has decidable equality.\index{decidable equality!of Z@{of $\Z$}}\index{Z@{$\Z$}!has decidable equality}\index{integers!decidable equality}\index{has decidable equality!integers}
  \end{subexenum}
  \exitem \label{ex:has-decidable-equality-Sigma}Consider a family $B$ over $A$, and consider the following three conditions:
  \begin{enumerate}
  \item The type $A$ has decidable equality.
  \item The type $B(x)$ has decidable equality for each $x:A$.
  \item The type $\sm{x:A}B(x)$ has decidable equality.\index{decidable equality!of S-types@{of $\Sigma$-types}}
  \end{enumerate}
  Show that if (i) holds, then (ii) and (iii) are equivalent, and show that if $B$ has a section $b:\prd{x:A}B(x)$, then (ii) and (iii) together imply (i).
  \exitem Consider a family $B$ of types over $\Fin{k}$, for some $k:\N$.
  \begin{subexenum}
  \item Show that if each $B(x)$ is decidable, then $\prd{x:\Fin{k}}B(x)$ is again decidable.
  \item Show that if each $B(x)$ has decidable equality, then $\prd{x:\Fin{k}}B(x)$ also has decidable equality.
  \end{subexenum}
  \exitem \label{ex:maximal-element}Consider a decidable type family $P$ over $\N$ equipped with an upper bound $m$.
  \begin{subexenum}
  \item Show that the type $\sm{n:\N}P(n)$ is decidable.
  \item Construct a function
  \begin{equation*}
    \Big(\sm{n:\N}P(n)\Big)\to\Big(\sm{n:\N}P(n)\times\isupperbound_P(n)\Big).
  \end{equation*}
  \item Use the function of part (b) to give a second construction of the greatest common divisor, and verify that it satisfies the specification of \cref{defn:is-gcd}.
  \end{subexenum}
  \exitem \label{ex:bezouts-identity-N}
  \begin{subexenum}
  \item For any three natural numbers $x$, $y$, and $z$, show that the type
    \begin{equation*}
      \sm{k:\N}\sm{l:\N}\distN(kx,ly)=z
    \end{equation*}
    is decidable.
  \item (B\'ezout's identity)\index{Bezout's identity@{B\'ezout's identity}}\index{natural numbers!Bezout's identity@{B\'ezout's identity}} For any two natural numbers $x$ and $y$, construct two natural numbers $k$ and $l$ equipped with an identification
  \begin{equation*}
    \distN(kx,ly)=\gcd(x,y).
  \end{equation*}
  \end{subexenum}
  \exitem
  \begin{subexenum}
  \item Show that every natural number $n\geq 2$ has a prime factor.
  \item Define a function
    \begin{equation*}
      \primefactors : \Big(\sm{n:\N}2\leq n\Big)\to \lst(\N)
    \end{equation*}
    such that $\primefactors(n)$ is an increasing list of primes, and $n$ is the product of the primes in the list $\primefactors(n)$.
  \item Show that any increasing list $l$ of primes of which the product is $n$ is equal to the list $\primefactors(n)$.
  \end{subexenum}
  \exitem Show that there are infinitely many primes $p\equiv 3\mod 4$.
  \exitem Show that for each prime $p$, the ring $\Z/p$ of integers modulo $p$ is a field, i.e., construct a multiplicative inverse
  \begin{equation*}
    (\blank)^{-1} : \prd{x:\Z/p}\to (x\neq 0) \to \Z/p
  \end{equation*}
  equipped with identifications
  \begin{align*}
    x^{-1}x & = 1 & xx^{-1} & = 1.
  \end{align*}
  \exitem Let $F:\N\to\N$ be the Fibonacci sequence. Construct the \define{cofibonacci sequence}\index{cofibonacci sequence|textbf}\index{natural numbers!cofibonacci sequence}, i.e., the function $G:\N\to\N$ such that\index{Fibonacci sequence!has left adjoint}
  \begin{equation*}
    (G_m\mid n) \leftrightarrow (m\mid F_n)
  \end{equation*}
  for all $m,n:\N$. Hint: for $m>0$, $G_m$ is the least $x>0$ such that $m\mid F_x$. 
\end{exercises}

%%% Local Variables:
%%% mode: latex
%%% TeX-master: "hott-intro"
%%% End:

%\input{integers}

\cleardoublepage
%%% Local Variables:
%%% mode: latex
%%% TeX-master: "hott-intro"
%%% End:
