\chapter{Martin-L\"of's Dependent Type Theory}
\label{chap:type-theory}%

Dependent type theory is a formal system to organize all mathematical objects, structure, and knowledge. Dependent type theory is about types, or more generally dependent types, and their elements. There are many ways to think about type theory, types, and its elements. Types can be interpreted as sets, i.e., there is an interpretation of type theory into Zermelo-Fraenkel set theory, but there are some important differences between type theory and set theory, and the interpretation of types as sets has significant limitations. One of the differences is that in type theory, every element comes equipped with its type. We will write $a:A$ for the judgment that $a$ is an element of type $A$. This leads us to a second important difference between type theory and set theory. Set theory is axiomatized in the formal system of first order logic, whereas type theory is its own formal system. Types and their elements are constructed by following the rules of this formal system, and the only way to construct an element is to construct it as an element of a previously constructed type. The expression $a:A$ is therefore not considered to be a proposition, i.e., something which one can assert about an arbitrary element and an arbitrary type, but it is considered to be a judgment, i.e., an assessment that is part of the construction of the element $a:A$.

In type theory there is a much stronger focus on equality of elements than there is in set theory. It is said that a type is not fully understood until (i) one understands how to construct an element of the type and (ii) one understands precisely how to show that two elements of the type are equal. Equality in type theory is governed by the identity type. Unlike in classical set theory, where equality is a decidable proposition of first order logic, the \emph{type} $x=y$ of identifications of two elements $x,y:A$ is itself a type, and therefore it could possess intricate further structure. 

Dependent type theory is built up in several stages. At the first stage we give structural rules, which express the general theory of type dependency. There is no ambient deductive system of first order logic in type theory. Type theory is its own deductive system, and the structural rules are at the heart of this system. The basic operations that are governed by the structural rules are substitution and weakening operations. After the general system of dependent type theory has been set up, we introduce the ways in which we can form types. The most fundamental class of types are dependent function types, or $\Pi$-types. They are used for practically everything. Next, we introduce the type of natural numbers, where we use type-dependency to formulate a type-theoretic version of the induction principle. By the type-theoretic nature of this induction principle, it can be used in two ways: it can be used to construct the many familiar operations on $\N$, such as addition and multiplication, and it can also be used to prove properties about those operations.

The next idea is that we can consider induction principles for many other types as well. This leads to the idea of more general inductive types. In \cref{sec:inductive} we introduce the unit type, the empty type, the booleans, coproducts, dependent pair types, and cartesian products. All of these are examples of inductive types, and their induction principles can be used to construct the basic operations on them, as well as to prove properties about those operations.

Then we come to the most characteristic ingredient of Martin L\"of's dependent type theory: the identity type. The identity type $x=_Ay$ is an example of a \emph{dependent} type, because it is indexed by $x,y:A$, and it is inductively generated by the reflexivity element $\refl{x}:x=_Ax$. The catch is, however, that the identity type $x=_Ay$ is just another type, and it could potentially have many different elements.

The last class of types that we introduce are universes. Universes are type families that are closed under the operations of type theory: $\Pi$-types, $\Sigma$-types, identity types, and so on. Universes play a fundamental role in the theory. One important reason for introducing universes is that they can be used to define type families over inductive types via their induction principles. For example, this allows us to define the ordering relations $\leq$ and $<$ on the natural numbers. We will also use the universes to show the Peano axioms asserting that $\succN$ is injective, and that $\zeroN$ is not a successor.

In the final two sections of this chapter, we start developing mathematics in type theory. In \cref{sec:modular-arithmetic} we study the Curry-Howard interpretation, and use it to develop modular arithmetic in type theory. In \cref{sec:decidability} we study the concept of decidability, and use it to obtain basic theorems in elementary number theory, such as the well-ordering theorem, the construction of the greatest common divisor, and the infinitude of primes. Both of these sections can be viewed as tutorials in type theory, designed to give you some practical experience with type theory before diving into the intricacies of the univalent foundations of mathematics.

\section{Dependent type theory}
\label{ch:dtt}

\index{dependent type theory|(}
Dependent type theory is a system of inference rules that can be combined to make \emph{derivations}. In these derivations, the goal is often to construct a term of a certain type. Such a term can be a function if the type of the constructed term is a function type; a proof of a property if the type of the constructed term is a proposition; an identification if the type of the constructed term is an identity type, and so on. In some respect, a type is just a collection of mathematical objects and constructing terms of a type is the everyday mathematical task or challenge. The system of inference rules that we call type theory offers a principled way of engaging in mathematical activity.

\subsection{Judgments and contexts in type theory}\label{sec:judgments}
\index{judgment|(}
\index{context|(}
A mathematical argument or construction consists of a sequence of deductive steps, each one using finitely many hypotheses in order to get to the next stage in the proof or construction. Such steps can be represented by \define{inference rules}\index{inference rule|see {rule}}, which are written in the form
\begin{prooftree}
  \AxiomC{$\mathcal{H}_1$\quad $\mathcal{H}_2$ \quad \dots \quad $\mathcal{H}_n$}
  \UnaryInfC{$\mathcal{C}$}
\end{prooftree}
Inference rules contain above the horizontal line\index{horizontal line|see {inference rule}} a finite list $\mathcal{H}_1$, $\mathcal{H}_2$, \dots, $\mathcal{H}_n$ of \emph{judgments} for the hypotheses\index{inference rule!hypotheses}, and below the horizontal line a single judgment $\mathcal{C}$ for the conclusion\index{inference rule!conclusion}. The system of dependent type theory is described by a set of such inference rules.

A straightforward example of an inference rule that we will encounter in \cref{ch:pi} when we introduce function types\index{function type}, is the inference rule
\begin{prooftree}
  \AxiomC{$\Gamma\vdash a:A$}
  \AxiomC{$\Gamma\vdash f:A\to B$}
  \BinaryInfC{$\Gamma\vdash f(a):B$}
\end{prooftree}
This rule asserts that in any context $\Gamma$ we may use a term $a:A$ and a function $f:A\to B$ to obtain a term $f(a):B$. Each of the expressions
\begin{align*}
  \Gamma & \vdash a :A \\
  \Gamma & \vdash f : A \to B \\
  \Gamma & \vdash f(a):B
\end{align*}
are examples of judgments. There are four kinds of judgments in type theory:
\begin{enumerate}
\item \emph{$A$ is a (well-formed) \define{type} in context $\Gamma$.}
  \index{well-formed type}\index{type}
  The symbolic expression for this judgment is\index{Gamma turnstile A type@{$\Gamma\vdash A~\type$}}\index{judgment!Gamma turnstile A type@{$\Gamma\vdash A~\type$}}
  \begin{equation*}
    \Gamma\vdash A~\type
  \end{equation*}
\item \emph{$A$ and $B$ are \define{judgmentally equal types} in context $\Gamma$.}
  \index{judgmental equality!of types} The symbolic expression for this judgment is\index{Gamma turnstile A is B type@{$\Gamma\vdash A\jdeq B~\type$}}\index{judgment!Gamma turnstile A is B type@{$\Gamma\vdash A\jdeq B~\type$}}
  \begin{equation*}
    \Gamma\vdash A \jdeq B~\type
  \end{equation*}
\item \emph{$a$ is a (well-formed) \define{term}\index{well-formed term}\index{term} of type $A$ in context $\Gamma$.} The symbolic expression for this judgment is\index{Gamma turnstile a in A@{$\Gamma\vdash a:A$}}\index{judgment!Gamma turnstile a in A@{$\Gamma\vdash a:A$}}
  \begin{equation*}
    \Gamma \vdash a:A
  \end{equation*}
\item \emph{$a$ and $b$ are \define{judgmentally equal terms} of type $A$ in context $\Gamma$.}\index{judgmental equality!of terms} The symbolic expression for this judgment is\index{Gamma turnstile a is b in A@{$\Gamma\vdash a\jdeq b:A$}}\index{judgment!Gamma turnstile a is b in A@{$\Gamma\vdash a\jdeq b:A$}}
  \begin{equation*}
    \Gamma\vdash a\jdeq b:A
  \end{equation*}
\end{enumerate}
Thus we see that any judgment is of the form $\Gamma\vdash\mathcal{J}$, consisting of a context $\Gamma$ and an expression $\mathcal{J}$ asserting that $A$ is a type, that $A$ and $B$ are equal types, that $a$ is a term of type $A$, or that $a$ and $b$ are equal terms of type $A$. The role of a context is to declare what hypothetical terms\index{hypothetical term} are assumed, along with their types. More formally, a \define{context} is an expression of the form
\begin{equation}\label{eq:context}
x_1:A_1,~x_2:A_2(x_1),~\ldots,~x_n:A_n(x_1,\ldots,x_{n-1})
\end{equation}
satisfying the condition that for each $1\leq k\leq n$ we can derive, using the inference rules of type theory, that
\begin{equation}\label{eq:context-condition}
  x_1:A_1,~x_2:A_2(x_1),~\ldots,~x_{k-1}:A_{k-1}(x_1,\ldots,x_{k-2})\vdash A_k(x_1,\ldots,x_{k-1})~\type.
\end{equation}
In other words, to check that an expression of the form \cref{eq:context} is a context, one starts on the left and works their way to the right verifying that each hypothetical term $x_k$ is assigned a well-formed type. Hypothetical terms are commonly called \define{variables}\index{variable}, and we say that a context as in \cref{eq:context} \define{declares the variables}\index{variable declaration} $x_1,\ldots,x_n$. We may use variable names other than $x_1,\ldots,x_n$, as long as no variable is declared more than once.

The condition in \cref{eq:context-condition} that each of the hypothetical terms is assigned a well-formed type, is checked recursively. Note that the context of length $0$ satisfies the requirement in \cref{eq:context-condition} vacuously. This context is called the \define{empty context}\index{context!empty context}\index{empty context}. An expression of the form $x_1:A_1$ is a context if and only if $A_1$ is a well-formed type in the empty context. Such types are called \define{closed types}\index{closed type}\index{type!closed type}. We will soon encounter the type $\N$ of natural numbers\index{natural numbers}, which is an example of a closed type\index{natural numbers!is a closed type}. There is also the notion of \define{closed term}\index{closed term}\index{term!closed term}, which is simply a term in the empty context. The next case is that an expression of the form $x_1:A_1,~x_2:A_2(x_1)$ is a context if and only if $A_1$ is a well-formed type in the empty context, and $A_2(x_1)$ is a well-formed type, given a hypothetical term $x_1:A_1$. This process repeats itself for longer contexts.

It is a feature of \emph{dependent} type theory that all judgments are context-dependent, and indeed that even the types of the variables may depend on any previously declared variables. For example, when we introduce the \emph{identity type}\index{identity type} in \cref{chap:identity}, we make full use of the machinery of type dependency, as is clear from how they are introduced:
\begin{prooftree}
  \AxiomC{$\Gamma\vdash A~\type$}
  \UnaryInfC{$\Gamma,x:A,y:A\vdash x=y~\type$}
\end{prooftree}
This rule asserts that given a type $A$ in context $\Gamma$, we may form a type $x=y$ in context $\Gamma,x:A,y:A$. Note that in order to know that the expression $\Gamma,x:A,y:A$ is indeed a well-formed context, we need to know that $A$ is a well-formed type in context $\Gamma,x:A$. This is an instance of \emph{weakening}\index{weakening}, which we will describe shortly.

In the situation where we have
\begin{equation*}
  \Gamma,x:A\vdash B(x)~\type,
\end{equation*}
we say that $B$ is a \define{family}\index{family}\index{type family} of types over $A$ in context $\Gamma$. Alternatively, we say that $B(x)$ is a type \define{indexed}\index{indexed type}\index{type!indexed} by $x:A$, in context $\Gamma$. Similarly, in the situation where we have
\begin{equation*}
  \Gamma,x:A\vdash b(x):B(x),
\end{equation*}
we say that $b$ is a \define{section}\index{section of a family} of the family $B$ over $A$ in context $\Gamma$. Alternatively, we say that $b(x)$ is a term of type $B(x)$, \define{indexed}\index{indexed term}\index{term!indexed} by $x:A$ in context $\Gamma$. Note that in the above situations $A$, $B$, and $b$ also depend on the variables declared in the context $\Gamma$, even though we have not explicitly mentioned them. It is common practice to not mention every variable in the context $\Gamma$ in such situations.

\index{judgment|)}
\index{context|)}

\subsection{Inference rules}\label{sec:rules}

In this section we present the basic inference rules of dependent type theory. Those rules are valid to be used in any type theoretic derivation. There are only four sets of inference rules:
\begin{enumerate}
\item Rules for judgmental equality 
\item Rules for substitution
\item Rules for weakening
\item The ``variable rule''
\end{enumerate}

\subsubsection*{Judgmental equality}

\index{rules!for type dependency!rules for judgmental equality|(}
In this set of inference rules we ensure that judgmental equality (both on types and on terms) are equivalence relations, and we make sure that in any context $\Gamma$, we can change the type of any variable to a judgmentally equal type.

\begin{samepage}
The rules postulating that judgmental equality on types and on terms is an equivalence relation are as follows\index{judgmental equality!is an equivalence relation}:
\begin{center}
%\begin{small}
\begin{minipage}{.2\textwidth}
\begin{prooftree}
\AxiomC{$\Gamma\vdash A~\textrm{type}$}
\UnaryInfC{$\Gamma\vdash A\jdeq A~\textrm{type}$}
\end{prooftree}
\end{minipage}
\begin{minipage}{.25\textwidth}
\begin{prooftree}
\AxiomC{$\Gamma\vdash A\jdeq A'~\textrm{type}$}
\UnaryInfC{$\Gamma\vdash A'\jdeq A~\textrm{type}$}
\end{prooftree}
\end{minipage}
\begin{minipage}{.5\textwidth}
\begin{prooftree}
\AxiomC{$\Gamma\vdash A\jdeq A'~\textrm{type}$}
\AxiomC{$\Gamma\vdash A'\jdeq A''~\textrm{type}$}
\BinaryInfC{$\Gamma\vdash A\jdeq A''~\textrm{type}$}
\end{prooftree}
\end{minipage}
\\*
\bigskip
\begin{minipage}{.2\textwidth}
\begin{prooftree}
\AxiomC{$\Gamma\vdash a:A$}
\UnaryInfC{$\Gamma\vdash a\jdeq a : A$}
\end{prooftree}
\end{minipage}
\begin{minipage}{.25\textwidth}
\begin{prooftree}
\AxiomC{$\Gamma\vdash a\jdeq a':A$}
\UnaryInfC{$\Gamma\vdash a'\jdeq a: A$}
\end{prooftree}
\end{minipage}
\begin{minipage}{.5\textwidth}
\begin{prooftree}
\AxiomC{$\Gamma\vdash a\jdeq a' : A$}
\AxiomC{$\Gamma\vdash a'\jdeq a'': A$}
\BinaryInfC{$\Gamma\vdash a\jdeq a'': A$}
\end{prooftree}
\end{minipage}
%\end{small}
\end{center}
\end{samepage}

\bigskip
Apart from the rules postulating that judgmental equality is an equivalence relation, there are also \define{variable conversion rules}\index{judgmental equality!conversion rules}\index{variable conversion rules}\index{conversion rule!variable}\index{rules!for type dependency!variable conversion}.
Informally, these are rules stating that if $A$ and $A'$ are judgmentally equal types in context $\Gamma$, then any valid judgment in context $\Gamma,x:A$ is also a valid judgment in context $\Gamma,x:A'$. In other words: we can convert the type of a variable to a judgmentally equal type.

The first variable conversion rule states that
\begin{prooftree}
\AxiomC{$\Gamma\vdash A\jdeq A'~\textrm{type}$}
\AxiomC{$\Gamma,x:A,\Delta\vdash B(x)~\type$}
\BinaryInfC{$\Gamma,x:A',\Delta\vdash B(x)~\type$}
\end{prooftree}
In this conversion rule, the context of the form $\Gamma,x:A,\Delta$ is just any extension of the context $\Gamma,x:A$, i.e., a context of the form
\begin{equation*}
  x_1:A_1,\ldots,x_{n-1}:A_{n-1},x:A,x_{n+1}:A_{n+1},\ldots,x_{n+m}:A_{n+m}.
\end{equation*}

Similarly, there are variable conversion rules for judgmental equality of types, for terms, and for judgmental equality of terms. To avoid having to state essentially the same rule four times, we state all four variable conversion rules at once using a \emph{generic judgment} $\mathcal{J}$, which can be any of the four kinds of judgments.
\begin{prooftree}
\AxiomC{$\Gamma\vdash A\jdeq A'~\textrm{type}$}
\AxiomC{$\Gamma,x:A,\Delta\vdash \mathcal{J}$}
\BinaryInfC{$\Gamma,x:A',\Delta\vdash \mathcal{J}$}
\end{prooftree}
An analogous \emph{term conversion rule}, stated in \cref{ex:term_conversion}, converting the type of a term to a judgmentally equal type, is derivable using the rules for substitution and weakening, and the variable rule.
\index{rules!for type dependency!rules for judgmental equality|)}

\subsubsection*{Substitution}
\index{substitution|(}\index{rules!for type dependency!rules for substitution|(}
If we are given a term $a:A$ in context $\Gamma$, then for any type $B$ in context $\Gamma,x:A,\Delta$ we can simultaneously substitute $a$ for all occurences of the variable $x$ in $\Delta$ and in $B$, to obtain a type $B[a/x]$ in context $\Gamma,\Delta[a/x]$. You are already familiar with simultaneous substitution, e.g., substituting $0$ for $x$ in the polynomial
\begin{equation*}
  1+x+x^2+x^3
\end{equation*}
resuts in the number $1+0+0^2+0^3$, which can be computed to the value $1$. 

Type theoretic substitution is similar. In a bit more detail, suppose we have well-formed type
\begin{equation*}
  x_1:A_1,\ldots,x_{n-1}:A_{n-1},x_n:A_n,x_{n+1}:A_{n+1},\ldots,x_{n+m}:A_{n+m}\vdash B~\textrm{type}
\end{equation*}
and a term $a:A_n$ in context $x_1:A_1,\ldots,x_{n-1}:A_{n-1}$. Then we can form the type
\begin{equation*}
  x_1:A_1,\ldots,x_{n-1}:A_{n-1},x_{n+1}:A_{n+1}[a/x_n],\ldots,x_{n+m}:A_{n+m}[a/x_n]\vdash B[a/x_n]~\textrm{type}
\end{equation*}
by substituting $a$ for all occurences of $x_n$. Note that the variables $x_{n+1},\ldots,x_{n+m}$ are assigned new types after performing the substitution of $a$ for $x_n$.

This operation of substituting $a$ for $x$ is understood to be defined recursively over the length of $\Delta$. When $B$ is a family of types over $A$ and $a:A$, we also say that $B[a/x]$ is the \define{fiber}\index{family!fiber of}\index{fiber!of a family} of $B$ at $a$. We will usually write $B(a)$ for $B[a/x]$. Similarly we obtain for any term $b:B$ in context $\Gamma,x:A,\Delta$ a term $b[a/x]:B[a/x]$. The term $b[a/x]$ is called the \define{value} of $b$ at $a$. When we substitute in a judgmental equality, either of types or terms, we simply subtitute on both sides of the equation.

We can now postulate the \define{substitution rule} as follows:
\begin{prooftree}
\AxiomC{$\Gamma\vdash a:A$}
\AxiomC{$\Gamma,x:A,\Delta\vdash \mathcal{J}$}
\RightLabel{$S$}
\BinaryInfC{$\Gamma,\Delta[a/x]\vdash \mathcal{J}[a/x]$}
\end{prooftree}
In other words, the substitution rule asserts that substitution preserves well-formedness and judgmental equality of types and terms. Furthermore, we postulate that substitution by judgmentally equal terms results in judgmentally equal types
\begin{prooftree}
\AxiomC{$\Gamma\vdash a\jdeq a':A$}
\AxiomC{$\Gamma,x:A,\Delta\vdash B~\type$}
\BinaryInfC{$\Gamma,\Delta[a/x]\vdash B[a/x]\jdeq B[a'/x]~\type$}
\end{prooftree}
and it also results in judgmentally equal terms
\begin{prooftree}
\AxiomC{$\Gamma\vdash a\jdeq a':A$}
\AxiomC{$\Gamma,x:A,\Delta\vdash b:B$}
\BinaryInfC{$\Gamma,\Delta[a/x]\vdash b[a/x]\jdeq b[a'/x]:B[a/x]$}
\end{prooftree}
To see that these rules make sense, we observe that both $B[a/x]$ and $B[a'/x]$ are types in context $\Delta[a/x]$, provided that $a\jdeq a'$. This is immediate by recursion on the length of $\Delta$.
\index{substitution|)}\index{rules!for type dependency!rules for substitution|)}

\subsubsection*{Weakening}
\index{weakening|(}\index{rules!for type dependency!rules for weakening|(}
If we are given a type $A$ in context $\Gamma$, then any judgment made in a longer context $\Gamma,\Delta$ can also be made in the context $\Gamma,x:A,\Delta$, for a fresh variable $x$. The \define{weakening rule}\index{weakening} asserts that weakening by a type $A$ in context preserves well-formedness and judgmental equality of types and terms.
\begin{prooftree}
\AxiomC{$\Gamma\vdash A~\textrm{type}$}
\AxiomC{$\Gamma,\Delta\vdash \mathcal{J}$}
\RightLabel{$W$}
\BinaryInfC{$\Gamma,x:A,\Delta \vdash \mathcal{J}$}
\end{prooftree}
This process of expanding the context by a fresh variable of type $A$ is called \define{weakening} (by $A$).

In the simplest situation where weakening applies, we have two types $A$ and $B$ in context $\Gamma$. Then we can weaken $B$ by $A$ as follows
\begin{prooftree}
  \AxiomC{$\Gamma\vdash A~\textrm{type}$}
  \AxiomC{$\Gamma\vdash B~\textrm{type}$}
  \RightLabel{$W$}
  \BinaryInfC{$\Gamma,x:A\vdash B~\type$}
\end{prooftree}
in order to form the type $B$ in context $\Gamma,x:A$. The type $B$ in context $\Gamma,x:A$ is called the \define{constant family}\index{family!constant family}\index{constant family} $B$, or the \define{trivial family}\index{family!trivial family}\index{trivial family} $B$.
\index{weakening|)}\index{rules!for type dependency!rules for weakening|)}

\subsubsection*{The variable rule}
If we are given a type $A$ in context $\Gamma$, then we can weaken $A$ by itself to obtain that $A$ is a type in context $\Gamma,x:A$. The \define{variable rule}\index{variable rule}\index{rules!for type dependency!variable rule} now asserts that any hypothetical term $x:A$ in context $\Gamma$ is a well-formed term of type $A$ in context $\Gamma,x:A$.
\begin{prooftree}
\AxiomC{$\Gamma\vdash A~\textrm{type}$}
\RightLabel{$\delta$}
\UnaryInfC{$\Gamma,x:A\vdash x:A$}
\end{prooftree}
One of the reasons for including the variable rule is that it provides an \emph{identity function}\index{identity function} on the type $A$ in context $\Gamma$.

\subsection{Derivations}\label{sec:derivations}

\index{derivation|(}
A derivation in type theory is a finite tree in which each node is a valid rule of inference. At the root of the tree we find the conclusion, and in the leaves of the tree we find the hypotheses. We give two examples of derivations: a derivation showing that any variable can be changed to a fresh one, and a derivation showing that any two variables that do not mutually depend on one another can be swapped in order.

Given a derivation with hypotheses $\mathcal{H}_1,\ldots,\mathcal{H}_n$ and conclusion $\mathcal{C}$, we can form a new inference rule
\begin{prooftree}
  \AxiomC{$\mathcal{H}_1$}
  \AxiomC{$\cdots$}
  \AxiomC{$\mathcal{H}_n$}
  \TrinaryInfC{$\mathcal{C}$}
\end{prooftree}
Such a rule is called \define{derivable}, because we have a derivation for it. In order to keep proof trees reasonably short and manageable, we use the convention that any derived rules can be used in future derivations.

\subsubsection*{Changing variables}

\index{change of variables}
Variables can always be changed to fresh variables. We show that this is the case by showing that the inference rule\index{rules!for type dependency!change of variables}
\begin{prooftree}
  \AxiomC{$\Gamma,x:A,\Delta\vdash \mathcal{J}$}
  \RightLabel{$x'/x$}
  \UnaryInfC{$\Gamma,x':A,\Delta[x'/x]\vdash \mathcal{J}[x'/x]$}
\end{prooftree}
is derivable, where $x'$ is a variable that does not occur in the context $\Gamma,x:A,\Delta$. 

Indeed, we have the following derivation using substitution, weakening, and the variable rule:
\begin{prooftree}
  \AxiomC{$\Gamma\vdash A~\type$}
  \RightLabel{$\delta$}
  \UnaryInfC{$\Gamma,x':A\vdash x':A$}
  \AxiomC{$\Gamma\vdash A~\type$}
  \AxiomC{$\Gamma,x:A,\Delta\vdash \mathcal{J}$}
  \RightLabel{$W$}
  \BinaryInfC{$\Gamma,x':A,x:A,\Delta\vdash \mathcal{J}$}
  \RightLabel{$S$}
  \BinaryInfC{$\Gamma,x':A,\Delta[x'/x]\vdash \mathcal{J}[x'/x]$}
\end{prooftree}
In this derivation it is the application of the weakening rule where we have to check that $x'$ does not occur in the context $\Gamma,x:A,\Delta$.

\subsubsection*{Interchanging variables}

The \define{interchange rule}\index{rules!for type dependency!interchange}\index{interchange rule} states that if we have two types $A$ and $B$ in context $\Gamma$, and we make a judgment in context $\Gamma,x:A,y:B,\Delta$, then we can make that same judgment in context $\Gamma,y:B,x:A,\Delta$ where the order of $x:A$ and $y:B$ is swapped. More formally, the interchange rule is the following inference rule
\begin{prooftree}
\AxiomC{$\Gamma\vdash B~\textrm{type}$}
\AxiomC{$\Gamma,x:A,y:B,\Delta\vdash \mathcal{J}$}
\BinaryInfC{$\Gamma,y:B,x:A,\Delta\vdash \mathcal{J}$}
\end{prooftree}
Just as the rule for changing variables, we claim that the interchange rule is a derivable rule.

The idea of the derivation for the interchange rule is as follows: If we have a judgment
\begin{equation*}
  \Gamma,x:A,y:B,\Delta\vdash\mathcal{J},
\end{equation*}
then we can change the variable $y$ to a fresh variable $y'$ and weaken the judgment to obtain the judgment
\begin{equation*}
  \Gamma,y:B,x:A,y':B,\Delta[y'/y]\vdash\mathcal{J}[y'/y].
\end{equation*}
Now we can substitute $y$ for $y'$ to obtain the desired judgment $\Gamma,y:B,x:A,\Delta\vdash\mathcal{J}$. The formal derivation is as follows:
%\begin{small}
\begin{prooftree}
\AxiomC{$\Gamma\vdash B~\textrm{type}$}
\RightLabel{$\delta$}
\UnaryInfC{$\Gamma,y:B\vdash y:B$}
\RightLabel{$W$} 
\UnaryInfC{$\Gamma,y:B,x:A\vdash y:B$}
\AxiomC{$\Gamma,x:A,y:B,\Delta\vdash \mathcal{J}$}
\RightLabel{$y'/y$}
\UnaryInfC{$\Gamma,x:A,y':B,\Delta[y'/y]\vdash \mathcal{J}[y'/y]$}
\RightLabel{$W$}
\UnaryInfC{$\Gamma,y:B,x:A,y':B,\Delta[y'/y]\vdash \mathcal{J}[y'/y]$}
\RightLabel{$S$}
\BinaryInfC{$\Gamma,y:B,x:A,\Delta\vdash \mathcal{J}$}
\end{prooftree}
%\end{small}
\index{derivation|)}

\begin{exercises}
\exercise \label{ex:term_conversion}Give a derivation for the following \define{term conversion rule}\index{term conversion rule}\index{rules!for type dependency!term conversion}\index{term conversion rule}\index{conversion rule!term}:
\begin{prooftree}
\AxiomC{$\Gamma\vdash A\jdeq A'~\textrm{type}$}
\AxiomC{$\Gamma\vdash a:A$}
\BinaryInfC{$\Gamma\vdash a:A'$}
\end{prooftree}
%\exercise Consider a type $A$ in context $\Gamma$. In this exercise we establish a correspondence between types in context $\Gamma,x:A$, and uniform choices of types $B_a$, where $a$ ranges over terms of $A$ in a uniform way. A similar connection is made for terms.
%  \begin{subexenum}
%  \item We define a \define{uniform family}\index{uniform family} over $A$ to consist of a type
%    \begin{equation*}
%      \Delta,\Gamma\vdash B_a~\type
%    \end{equation*}
%    for every context $\Delta$, and every term $\Delta,\Gamma\vdash a:A$, subject to the condition that one can derive
%    \begin{prooftree}
%      \AxiomC{$\Delta\vdash d:D$}
%      \AxiomC{$\Delta,y:D,\Gamma\vdash a:A$}
%      \BinaryInfC{$\Delta,\Gamma\vdash B_a[d/y]\jdeq B_{a[d/y]}~\type$}
%    \end{prooftree}
%    Define a bijection between the set of types in context $\Gamma,x:A$ modulo judgmental equality, and the set of uniform families over $A$ modulo judgmental equality. 
%  \item Consider a type $\Gamma,x:A\vdash B$. We define a \define{uniform term}\index{uniform term} of $B$ over $A$ to consist of a type
%    \begin{equation*}
%      \Delta,\Gamma\vdash b_a:B[a/x]~\type
%    \end{equation*}
%    for every context $\Delta$, and every term $\Delta,\Gamma\vdash a:A$, subject to the condition that one can derive
%    \begin{prooftree}
%      \AxiomC{$\Delta\vdash d:D$}
%      \AxiomC{$\Delta,y:D,\Gamma\vdash a:A$}
%      \BinaryInfC{$\Delta,\Gamma\vdash b_a[d/y]\jdeq b_{a[d/y]}:B[a/x][d/y]$}
%    \end{prooftree}
%    Define a bijection between the set of terms of $B$ in context $\Gamma,x:A$ modulo judgmental equality, and the set of uniform terms of $B$ over $A$ modulo judgmental equality. 
%  \end{subexenum}
\end{exercises}
\index{dependent type theory|)}

\section{Dependent function types}
\label{ch:pi}

\index{Pi-type@{$\Pi$-type}|see {dependent function type}}
\index{dependent function type|(}
A fundamental concept in dependent type theory is that of a dependent function. A dependent function is a function of which the type of the output may depend on the input. They are a generalization of ordinary functions, because an ordinary function $f:A\to B$ is a function of which the output $f(x)$ has type $B$ regardless of the value of $x$.

\subsection{The rules for dependent function types}
Consider a section $b$ of a family $B$ over $A$ in context $\Gamma$, i.e.,
\begin{equation*}
  \Gamma,x:A\vdash b(x):B(x).
\end{equation*}
From one point of view, such a section $b$ is an operation, or a program\index{program}, that takes as input $x:A$ and produces a term $b(x):B(x)$. From a more mathematical point of view we see $b$ as a choice of an element of each $B(x)$. In other words, we may see $b$ as a function that takes $x:A$ to $b(x):B(x)$. Note that the type $B(x)$ of the output is dependent on $x:A$. In this section we postulate rules for the \emph{type} of all such dependent functions: whenever $B$ is a family over $A$ in context $\Gamma$, there is a type
\begin{equation*}
  \prd{x:A}B(x)
\end{equation*}
in context $\Gamma$, consisting of all the dependent functions of which the output at $x:A$ has type $B(x)$. There are four principal rules for $\Pi$-types:
\begin{enumerate}
\item The formation rule, which tells us how we may form dependent function types.
\item The introduction rule, which tells us how to introduce new terms of dependent function types.
\item The elimination rule, which tells us how to use arbitrary terms of dependent function types.
\item The computation rules, which tell us how the introduction and elimination rules interact. These computation rules guarantee that every term of a dependent function type behaves as expected: as a dependent function.
\end{enumerate}
In the cases of the formation rule, the introduction rule, and the elimination rule, we also need rules that assert that all the constructions respect judgmental equality. Those rules are called \define{congruence rules}.

\subsubsection{The $\Pi$-formation rule}
\index{dependent function type!formation rule}
The \define{$\Pi$-formation rule} tells us how $\Pi$-types are constructed. The idea of $\Pi$-types is that for any type family $B$ of types over $A$, there is a type of dependent functions $\prd{x:A}B(x)$, so the $\Pi$-formation rule is as follows:\index{rules!for dependent function types!formation}
\begin{prooftree}
\AxiomC{$\Gamma,x:A\vdash B(x)~\textrm{type}$}
\RightLabel{$\Pi$.}
\UnaryInfC{$\Gamma\vdash \prd{x:A}B(x)~\type$}
\end{prooftree}
This rule simply states that in order for the type $\prd{x:A}B(x)$ to be well-formed in context $\Gamma$, the type $B$ must be a well-formed type in context $\Gamma,x:A$.

We also require that the the operation of forming dependent function types respects judgmental equality. This is postulated in the \define{congruence rule} for $\Pi$-types:
\index{rules!for dependent function types!congruence}
\index{dependent function type!congruence rule}
\begin{prooftree}
\AxiomC{$\Gamma\vdash A\jdeq A'~\type$}
\AxiomC{$\Gamma,x:A\vdash B(x)\jdeq B'(x)~\textrm{type}$}
\RightLabel{$\Pi$-eq.}
\BinaryInfC{$\Gamma\vdash \prd{x:A}B(x)\jdeq\prd{x:A'}B'(x)~\type$}
\end{prooftree}

There is one last rule that we need about the formation of $\Pi$-types, asserting that it does not matter what name we use for the variable $x$ that appears in the expression $\prd{x:A}B(x)$.
More precisely, when $x'$ is a variable that does not occur in the context $\Gamma$, then we postulate that
\index{rules!for dependent function types!change of bound variable}
\index{dependent function type!change of bound variable}
\begin{prooftree}
\AxiomC{$\Gamma,x:A\vdash B(x)~\textrm{type}$}
\RightLabel{$\Pi$-$x'/x$.}
\UnaryInfC{$\Gamma\vdash \prd{x:A}B(x)\jdeq \prd{x':A}B(x')~\type$}
\end{prooftree}
This rule is also known as \define{$\alpha$-conversion} for $\Pi$-types.

\subsubsection{The $\Pi$-introduction rule}
The introduction rule for dependent functions tells us how we may construct dependent functions of type $\prd{x:A}B(x)$. The idea is that a dependent function $f:\prd{x:A}B(x)$ is an operation that takes an $x:A$ to $f(x):B(x)$. Hence the introduction rule of dependent functions postulates that, in order to construct a dependent function one has to construct a term $b(x):B(x)$ in context $x:A$, i.e.:
\begin{prooftree}
  \AxiomC{$\Gamma,x:A \vdash b(x) : B(x)$}
  \RightLabel{$\lambda$.}
  \UnaryInfC{$\Gamma\vdash \lam{x}b(x) : \prd{x:A}B(x)$}
\end{prooftree}
This introduction rule%
\index{dependent function type!introduction rule|see {$\lambda$-abstraction}}
for dependent functions is also called the \define{$\lambda$-abstraction rule}%
\index{lambda-abstraction@{$\lambda$-abstraction}}%
\index{rules!for dependent function types!lambda-abstraction@{$\lambda$-abstraction}}%
\index{dependent function type!lambda-abstraction@{$\lambda$-abstraction}}, and we also say that the $\lambda$-abstraction $\lam{x}b(x)$ \define{binds} the variable $x$ in $b$. Just like ordinary mathematicians, we will sometimes write $x\mapsto b(x)$ for a function $\lam{x}b(x)$. The map $n\mapsto n^2$ is an example.

The $\lambda$-abstraction is also required to respect judgmental equality. Therefore we postulate the \define{congruence rule} for $\lambda$-abstraction,
\index{rules!for dependent function types!lambda-congruence@{$\lambda$-congruence}}
\index{lambda-congruence@{$\lambda$-congruence}}
\index{dependent function type!lambda-congruence@{$\lambda$-congruence}}
which asserts that
\begin{prooftree}
  \AxiomC{$\Gamma,x:A \vdash b(x)\jdeq b'(x) : B(x)$}
  \RightLabel{$\lambda$-eq.}
  \UnaryInfC{$\Gamma\vdash \lam{x}b(x)\jdeq \lam{x}b'(x) : \prd{x:A}B(x)$}
\end{prooftree}

\subsubsection{The $\Pi$-elimination rule}

\index{dependent function type!elimination rule|see {evaluation}}
The elimination rule for dependent function types provides us with a way to \emph{use} dependent functions. The way to use a dependent function is to apply it to an argument of the domain type. The $\Pi$-elimination rule is therefore also called the \define{evaluation rule}\index{evaluation}\index{rules!for dependent function types!evaluation}\index{dependent function type!evaluation}. It asserts that given a dependent function $f:\prd{x:A}B(x)$ in context $\Gamma$ we obtain a term $f(x)$ of type $B(x)$ in context $\Gamma,x:A$. More formally:
\begin{prooftree}
\AxiomC{$\Gamma\vdash f:\prd{x:A}B(x)$}
\RightLabel{$ev$}
\UnaryInfC{$\Gamma,x:A\vdash f(x) : B(x)$}
\end{prooftree}
Again we require that evaluation respects judgmental equality:
\begin{prooftree}
  \AxiomC{$\Gamma\vdash f\jdeq f':\prd{x:A}B(x)$}
  \UnaryInfC{$\Gamma,x:A\vdash f(x)\jdeq f'(x):B(x)$}
\end{prooftree}

\subsubsection{The $\Pi$-computation rules}

\index{dependent function type!computation rules|see {$\beta$- and $\eta$-rules}}
We now postulate rules that specify the behavior of functions. First, we have a rule that asserts that a function of the form $\lam{x}b(x)$ behaves as expected: when we evaluate it at $x:A$, then we obtain the value $b(x):B(x)$. This rule is called the \define{$\beta$-rule}\index{beta-rule@{$\beta$-rule}!for Pi-types@{for $\Pi$-types}}\index{rules!for dependent function types!beta-rule@{$\beta$-rule}}\index{dependent function type!beta-rule@{$\beta$-rule}}
\begin{prooftree}
\AxiomC{$\Gamma,x:A \vdash b(x) : B(x)$}
\RightLabel{$\beta$.}
\UnaryInfC{$\Gamma,x:A \vdash (\lambda y.b(y))(x)\jdeq b(x) : B(x)$}
\end{prooftree}
Second, we postulate a rule that asserts that all elements of a $\Pi$-type are (dependent) functions. This rule is known as the \define{$\eta$-rule}\index{eta-rule@{$\eta$-rule}!for Pi-types@{for $\Pi$-types}}\index{rules!for dependent function types!eta-rule@{$\eta$-rule}}\index{dependent function type!eta-rule@{$\eta$-rule}}
\begin{prooftree}
\AxiomC{$\Gamma\vdash f:\prd{x:A}B(x)$}
\RightLabel{$\eta$.}
\UnaryInfC{$\Gamma \vdash \lam{x}f(x) \jdeq f : \prd{x:A}B(x)$}
\end{prooftree}
In other words, the computation rules ($\beta$ and $\eta$) for dependent function types postulate that $\lambda$-abstraction rule and the evaluation rule are mutual inverses. This completes the specification of dependent function types.

\subsection{Ordinary function types}
Given two types $A$ and $B$ in context $\Gamma$, we can use the rules for $\Pi$-types to form the type $A\to B$ of \emph{ordinary} functions from $A$ to $B$. The type of ordinary functions is obtained by first weakening $B$ by $A$ and subsequently applying the $\Pi$-formation rule, as in the following derivation:
\begin{prooftree}
\AxiomC{$\Gamma\vdash A~\textrm{type}$}
\AxiomC{$\Gamma\vdash B~\textrm{type}$}
\RightLabel{$W$}
\BinaryInfC{$\Gamma,x:A\vdash B~\textrm{type}$}
\RightLabel{$\Pi$}
\UnaryInfC{$\Gamma\vdash \prd{x:A}B~\textrm{type}$}
\end{prooftree}
A term $f:\prd{x:A}B$ is a function that takes an argument $x:A$ and returns $f(x):B$. In other words, terms of type $\prd{x:A}B$ are indeed ordinary functions from $A$ to $B$. Therefore we will write $A\to B$\index{A arrow B@{$A\to B$}|see {function type}} for the \define{type of functions}\index{function type} from $A$ to $B$. Sometimes we will also write $B^A$\index{B^A@{$B^A$}|see {function type}} for the type $A\to B$.

We give a brief summary of the rules specifying ordinary function types, omitting the congruence rules. All of these rules can be derived easily from the corresponding rules for $\Pi$-types.\index{rules!for function types}
\begin{prooftree}
\AxiomC{$\Gamma\vdash A~\textrm{type}$}
\AxiomC{$\Gamma\vdash B~\textrm{type}$}
\RightLabel{$\to$}
\BinaryInfC{$\Gamma\vdash A\to B~\textrm{type}$}
\end{prooftree}%
\begin{center}
\begin{minipage}{.45\textwidth}
\begin{prooftree}
\AxiomC{$\Gamma\vdash B~\textrm{type}$}
\AxiomC{$\Gamma,x:A\vdash b(x):B$}
\RightLabel{$\lambda$}
\BinaryInfC{$\Gamma\vdash \lam{x}b(x):A\to B$}
\end{prooftree}%
\end{minipage}
\begin{minipage}{.45\textwidth}
\begin{prooftree}
\AxiomC{$\Gamma\vdash f:A\to B$}
\RightLabel{$ev$}
\UnaryInfC{$\Gamma,x:A\vdash f(x):B$}
\end{prooftree}%
\end{minipage}
\end{center}
\begin{center}
\begin{minipage}{.45\textwidth}
\begin{prooftree}
\AxiomC{$\Gamma\vdash B~\textrm{type}$}
\AxiomC{$\Gamma,x:A\vdash b(x):B$}
\RightLabel{$\beta$}
\BinaryInfC{$\Gamma,x:A\vdash(\lam{y}b(y))(x)\jdeq b(x):B$}
\end{prooftree}%
\end{minipage}
\begin{minipage}{.45\textwidth}
\begin{prooftree}
\AxiomC{$\Gamma\vdash f:A\to B$}
\RightLabel{$\eta$}
\UnaryInfC{$\Gamma\vdash\lam{x} f(x)\jdeq f:A\to B$}
\end{prooftree}
\end{minipage}
\end{center}

\subsection{The identity function, composition, and their laws}

First, we use the rules of dependent type theory to construct the identity function on an arbitrary type. 

\begin{defn}
For any type $A$ in context $\Gamma$, we define the \define{identity function}\index{identity function}\index{function type!identity function} $\idfunc[A]:A\to A$\index{id A@{$\idfunc[A]$}} using the variable rule:
\begin{prooftree}
\AxiomC{$\Gamma\vdash A~\textrm{type}$}
\UnaryInfC{$\Gamma,x:A\vdash x:A$}
\UnaryInfC{$\Gamma\vdash \idfunc[A]\defeq \lam{x}x:A\to A$}
\end{prooftree}
\end{defn}

Note that we have used the symbol $\defeq$ in the conclusion to define the identity function. A judgment of the form $\Gamma\vdash a\defeq b:A$ should be read as "$b$ is a well-defined term of type $A$ in context $\Gamma$, and we will refer to it as $a$".

By the above construction of the identity function we see that the identity function can be introduced with the following rule
\begin{prooftree}
  \AxiomC{$\Gamma\vdash A~\type$}
  \UnaryInfC{$\Gamma\vdash\idfunc[A]:A\to A$}
\end{prooftree}
Moreover, by the $\beta$-rule we see that the identity function satisfies the following computation rule:
\begin{prooftree}
  \AxiomC{$\Gamma\vdash A~\type$}
  \UnaryInfC{$\Gamma,x:A\vdash \idfunc[A](x)\jdeq x:A$}
\end{prooftree}

Next, we define the composition of functions. We will introduce the composition operation itself as a function $\comp$ that takes two arguments: the first argument is a function $g:B\to C$, and the second argument is a function $f:A\to B$. The output is a function $\comp(g,f):A\to C$, for which we often write $g\circ f$.

Types of functions with multiple arguments can be formed by iterating the $\Pi$-formation rule or the $\to$-formation rule. For example, a function
\begin{equation*}
  f:A\to (B\to C)
\end{equation*}
takes two arguments: first it takes an argument $x:A$, and the output $f(x)$ has type $B\to C$. This is again a function type, so $f(x)$ is a function that takes an argument $y:B$, and its output $f(x)(y)$ has type $C$. We will usually write $f(x,y)$ for $f(x)(y)$. With this idea of iterating function types, we see that type of the composition operation $\comp$ should be
\begin{equation*}
  (B\to C)\to ((A\to B)\to (A\to C)).
\end{equation*}
It is the type of functions, taking a function $g:B\to C$, to the type of functions $(A\to B)\to (A\to C)$. Thus, $\comp(g)$ is again a function, mapping a function $f:A\to B$ to the type of functions $B\to C$. 

\begin{defn}
For any three types $A$, $B$, and $C$ in context $\Gamma$, there is a \define{composition}\index{function type!composition}\index{composition!of functions} operation
\begin{equation*}
\comp:(B\to C)\to ((A\to B)\to (A\to C)),
\end{equation*}
i.e., we can derive
\begin{prooftree}
\AxiomC{$\Gamma\vdash A~\textrm{type}$}
\AxiomC{$\Gamma\vdash B~\textrm{type}$}
\AxiomC{$\Gamma\vdash C~\textrm{type}$}
\TrinaryInfC{$\Gamma\vdash\comp:(B\to C)\to ((A\to B)\to (A\to C))$}
\end{prooftree}
We will usually write $g\circ f$\index{g composed with f@{$g\circ f$}} for $\comp(g,f)$\index{comp(g,f)@{$\comp(g,f)$}}. Moreover, the composition operation satisfies the following computation rule:
\begin{prooftree}
  \AxiomC{$\Gamma\vdash A~\textrm{type}$}
  \AxiomC{$\Gamma\vdash B~\textrm{type}$}
  \AxiomC{$\Gamma\vdash C~\textrm{type}$}
  \TrinaryInfC{$\Gamma,g:B\to C,f:A\to B,x:A \vdash\comp(g,f,x)\jdeq g(f(x)) :C$}
\end{prooftree}
\end{defn}

\begin{constr}
  The idea of the definition is to define $\comp(g,f)$ to be the function $\lam{x}g(f(x))$. The derivation we use to construct $\comp$ is as follows:
  \begin{prooftree}
    \AxiomC{$\Gamma\vdash A~\type$}
    \AxiomC{$\Gamma\vdash B~\type$}
    \BinaryInfC{$\Gamma,f:B^A,x:A\vdash f(x):B$}
    \UnaryInfC{$\Gamma,g:C^B,f:B^A,x:A\vdash f(x):B$}
    \AxiomC{$\Gamma\vdash B~\type$}
    \AxiomC{$\Gamma\vdash C~\type$}
    \BinaryInfC{$\Gamma,g:C^B,y:B\vdash g(y):C$}
    \UnaryInfC{$\Gamma,g:C^B,f:B^A,y:B\vdash g(y):C$}
    \UnaryInfC{$\Gamma,g:C^B,f:B^A,x:A,y:B\vdash g(y):C$}
    \BinaryInfC{$\Gamma,g:C^B,f:B^A,x:A\vdash g(f(x)) : C$}
    \UnaryInfC{$\Gamma,g:C^B,f:B^A\vdash \lam{x}g(f(x)):C^A$}
    \UnaryInfC{$\Gamma,g:B\to C\vdash \lam{f}\lam{x}g(f(x)):B^A\to C^A$}
    \UnaryInfC{$\Gamma\vdash\comp\defeq \lam{g}\lam{f}\lam{x}g(f(x)):C^B\to (B^A\to C^A)$}
  \end{prooftree}
  It is immediate by the $\beta$-rule that the composition operation satisfies the asserted computation rule.
\end{constr}

The rules of function types can be used to derive the laws of a category\index{category laws!for functions} for functions, i.e., we can derive that function composition is associative and that the identity function satisfies the unit laws. In the remainder of this section we will give these derivations.

\begin{lem}
Composition of functions is associative\index{associativity!of function composition}\index{composition!of functions!associativity}, i.e., we can derive
\begin{prooftree}
\AxiomC{$\Gamma\vdash f:A\to B$}
\AxiomC{$\Gamma\vdash g:B\to C$}
\AxiomC{$\Gamma\vdash h:C\to D$}
\TrinaryInfC{$\Gamma \vdash (h\circ g)\circ f\jdeq h\circ(g\circ f):A\to D$}
\end{prooftree}
\end{lem}

\begin{proof}
  The main idea of the proof is that both $((h\circ g)\circ f)(x)$ and $(h\circ (g\circ f))(x)$ evaluate to $h(g(f(x))$, and therefore $(h\circ g)\circ f$ and $h\circ(g\circ f)$ must be judgmentally equal. This idea is made formal in the following derivation:
  \begin{prooftree}
    \AxiomC{$\Gamma\vdash f:A\to B$}
    \UnaryInfC{$\Gamma,x:A\vdash f(x):B$}
    \AxiomC{$\Gamma\vdash g:B\to C$}
    \UnaryInfC{$\Gamma,y:B\vdash g(y):C$}
    \UnaryInfC{$\Gamma,x:A,y:B\vdash g(y):C$}
    \BinaryInfC{$\Gamma,x:A\vdash g(f(x)):C$}
    \AxiomC{$\Gamma\vdash h:C\to D$}
    \UnaryInfC{$\Gamma,z:C\vdash h(z):D$}
    \UnaryInfC{$\Gamma,x:A,z:C\vdash h(z):D$}
    \BinaryInfC{$\Gamma,x:A\vdash h(g(f(x))):D$}
    \UnaryInfC{$\Gamma,x:A\vdash h(g(f(x)))\jdeq h(g(f(x))):D$}
    \UnaryInfC{$\Gamma,x:A\vdash (h\circ g)(f(x))\jdeq h((g\circ f)(x)):D$}
    \UnaryInfC{$\Gamma,x:A\vdash ((h\circ g)\circ f)(x)\jdeq (h\circ (g \circ f))(x):D$}
    \UnaryInfC{$\Gamma\vdash (h\circ g)\circ f\jdeq h\circ(g\circ f):A\to D$.}
  \end{prooftree}
\end{proof}

\begin{lem}\label{lem:fun_unit}
Composition of functions satisfies the left and right unit laws\index{left unit law|see {unit laws}}\index{right unit law|see {unit laws}}\index{unit laws!for function composition}\index{composition!of functions!unit laws}, i.e., we can derive
\begin{prooftree}
\AxiomC{$\Gamma\vdash f:A\to B$}
\UnaryInfC{$\Gamma\vdash \idfunc[B]\circ f\jdeq f:A\to B$}
\end{prooftree}
and
\begin{prooftree}
\AxiomC{$\Gamma\vdash f:A\to B$}
\UnaryInfC{$\Gamma\vdash f\circ\idfunc[A]\jdeq f:A\to B$}
\end{prooftree}
\end{lem}

\begin{proof}
The derivation for the left unit law is
%\begin{prooftree}
%\AxiomC{$\Gamma\vdash f:A\to B$}
%\UnaryInfC{$\Gamma,x:A\vdash f(x):B$}
%\AxiomC{$\Gamma\vdash B~\type$}
%\UnaryInfC{$\Gamma,y:B\vdash \idfunc[B](y)\jdeq y:B$}
%\UnaryInfC{$\Gamma,x:A,y:B\vdash \idfunc[B](y)\jdeq y:B$}
%\BinaryInfC{$\Gamma,x:A\vdash \idfunc[B](f(x))\jdeq f(x):B$}
%\UnaryInfC{$\Gamma,x:A\vdash (\idfunc[B]\circ f)(x)\jdeq f(x):B$}
%\UnaryInfC{$\Gamma\vdash \idfunc[B]\circ f\jdeq f:A\to B$}
%\end{prooftree}
\begin{prooftree}
  \AxiomC{$\Gamma\vdash f:A\to B$}
  \UnaryInfC{$\Gamma,x:A\vdash f(x):B$}
  \AxiomC{$\Gamma\vdash A~\type$}
  \AxiomC{$\Gamma\vdash B~\type$}
  \UnaryInfC{$\Gamma,y:B\vdash\idfunc(y)\jdeq y:B$}
  \BinaryInfC{$\Gamma,x:A,y:B\vdash\idfunc(y)\jdeq y:B$}
  \BinaryInfC{$\Gamma,x:A\vdash\idfunc(f(x))\jdeq f(x):B$}
  \UnaryInfC{$\Gamma\vdash\lam{x}\idfunc(f(x))\jdeq\lam{x}f(x):A\to B$}
  \AxiomC{$\Gamma\vdash f:A\to B$}
  \UnaryInfC{$\Gamma\vdash\lam{x}f(x)\jdeq f:A\to B$}
  \BinaryInfC{$\Gamma\vdash\idfunc\circ f\jdeq f:A\to B$}
\end{prooftree}
The right unit law is left as \cref{ex:fun_right_unit}.
\end{proof}

\begin{exercises}
  \exercise \label{ex:eta_ext}The $\eta$-rule is often seen as an extensionality principle. Use the $\eta$-rule to show that if $f$ and $g$ take equal values, then they must be equal, i.e., give a derivation for the rule
  \begin{prooftree}
    \AxiomC{$\Gamma\vdash f:\prd{x:A}B(x)$}
    \AxiomC{$\Gamma\vdash g:\prd{x:A}B(x)$}
    \AxiomC{$\Gamma,x:A\vdash f(x)\jdeq g(x):B(x)$}
    \TrinaryInfC{$\Gamma\vdash f\jdeq g:\prd{x:A}B(x)$}
  \end{prooftree}
  \exercise \label{ex:fun_right_unit}Give a derivation for the right unit law of \cref{lem:fun_unit}.\index{unit laws!for function composition}
  \exercise Show that the rule
  \begin{prooftree}
    \AxiomC{$\Gamma,x:A \vdash b(x) : B(x)$}
    \RightLabel{$\lambda$-$x'/x$}
    \UnaryInfC{$\Gamma\vdash \lam{x}b(x)\jdeq \lam{x'}b(x') : \prd{x:A}B(x)$}
  \end{prooftree}
  is derivable for any variable $x'$ that does not occur in the context $\Gamma,x:A$.
  \exercise 
  \begin{subexenum}
  \item Construct the \define{constant function}\index{constant function}\index{function!constant function}\index{const x@{$\const_x$}}\index{function!const@{$\const$}}
    \begin{prooftree}
      \AxiomC{$\Gamma\vdash A~\textrm{type}$}
      \UnaryInfC{$\Gamma,y:B\vdash \const_y:A\to B$}
    \end{prooftree}
  \item Show that
    \begin{prooftree}
      \AxiomC{$\Gamma\vdash f:A\to B$}
      \UnaryInfC{$\Gamma,z:C\vdash \const_z\circ f\jdeq\const_z : A\to C$}
    \end{prooftree}
  \item Show that
    \begin{prooftree}
      \AxiomC{$\Gamma\vdash A~\textrm{type}$}
      \AxiomC{$\Gamma\vdash g:B\to C$}
      \BinaryInfC{$\Gamma,y:B\vdash g\circ\const_y\jdeq \const_{g(y)}:A\to C$}
    \end{prooftree}
  \end{subexenum}
%  \exercise In this exercise we generalize the composition operation of non-dependent function types\index{composition!of dependent functions}:
%  \begin{subexenum}
%  \item Define a composition operation for dependent function types
%    \begin{prooftree}
%      \AxiomC{$\Gamma\vdash f:\prd{x:A}B(x)$}
%      \AxiomC{$\Gamma\vdash g:\prd{x:A}\prd{y:B(x)} C(x,y)$}
%      \BinaryInfC{$\Gamma\vdash g\circ' f:\prd{x:A} C(x,f(x))$}
%    \end{prooftree}
%    and show that this operation agrees with ordinary composition when it is specialized to non-dependent function types.
%  \item Show that composition of dependent functions agrees with ordinary composition of functions:
%    \begin{prooftree}
%      \AxiomC{$\Gamma\vdash f:A\to B$}
%      \AxiomC{$\Gamma\vdash g:B\to C$}
%      \BinaryInfC{$\Gamma\vdash (\lam{x}g)\circ' f\jdeq g\circ f:A \to C$}
%    \end{prooftree}
%  \item Show that composition of dependent functions is associative.\index{associativity!of dependent function composition}\index{composition!of dependent functions!associativity}
%  \item Show that composition of dependent functions satisfies the right unit law\index{unit laws!dependent function composition}:
%    \begin{prooftree}
%      \AxiomC{$\Gamma\vdash f:\prd{x:A}B(x)$}
%      \UnaryInfC{$\Gamma\vdash (\lam{x}f)\circ'\idfunc[A]\jdeq f :\prd{x:A}B(x)$}
%    \end{prooftree}
%  \item Show that composition of dependent functions satisfies the left unit law\index{unit laws!dependent function composition}\index{composition!of dependent functions!unit laws}:
%    \begin{prooftree}
%      \AxiomC{$\Gamma\vdash f:\prd{x:A}B(x)$}
%      \UnaryInfC{$\Gamma\vdash (\lam{x}\idfunc[B(x)])\circ' f\jdeq f:\prd{x:A}B(x)$}
%    \end{prooftree}
%  \end{subexenum}
  \exercise \label{ex:swap}
  \begin{subexenum}
  \item Given two types $A$ and $B$ in context $\Gamma$, and a type $C$ in context $\Gamma,x:A,y:B$, define the \define{swap function}\index{function!swap}\index{swap function}
    \begin{equation*}
      \Gamma\vdash \sigma:\Big(\prd{x:A}\prd{y:B}C(x,y)\Big)\to\Big(\prd{y:B}\prd{x:A}C(x,y)\Big)
    \end{equation*}
    that swaps the order of the arguments.
  \item Show that
    \begin{equation*}
      \Gamma\vdash \sigma\circ\sigma\jdeq\idfunc:\Big(\prd{x:A}\prd{y:B}C(x,y)\Big)\to \Big(\prd{x:A}\prd{y:B}C(x,y)\Big).
    \end{equation*}
  \end{subexenum}
\end{exercises}
\index{dependent function type|)}

\section{The natural numbers}

\index{inductive type|(}
\index{natural numbers|(}

The set of natural numbers is the most important object in mathematics. We quote Bishop\index{Bishop on the positive integers}, from his Constructivist Manifesto, the first chapter in Foundations of Constructive Analysis \cite{Bishop1967}, where he gives a colorful illustration of its importance to mathematics.

\begin{quote}
  ``The primary concern of mathematics is number, and this means the positive integers. We feel about number the way Kant felt about space. The positive integers and their arithmetic are presupposed by the very nature of our intelligence and, we are tempted to believe, by the very nature of intelligence in general. The development of the theory of the positive integers from the primitive concept of the unit, the concept of adjoining a unit, and the process of mathematical induction carries complete conviction. In the words of Kronecker, the positive integers were created by God. Kronecker would have expressed it even better if he had said that the positive integers were created by God for the benefit of man (and other finite beings). Mathematics belongs to man, not to God. We are not interested in properties of the positive integers that have no descriptive meaning for finite man. When a man proves a positive integer to exist, he should show how to find it. If God has mathematics of his own that needs to be done, let him do it himself.''
\end{quote}

A bit later in the same chapter, he continues:

\begin{quote}
  ``Building on the positive integers, weaving a web of ever more sets and ever more functions, we get the basic structures of mathematics: the rational number system, the real number system, the euclidean spaces, the complex number system, the algebraic number fields, Hilbert space, the classical groups, and so forth. Within the framework of these structures, most mathematics is done. Everything attaches itself to number, and every mathematical statement ultimately expresses the fact that if we perform certain computations within the set of positive integers, we shall get certain results.''
\end{quote}

\subsection{The formal specification of the type of natural numbers}
The type $\N$\index{N@{$\N$}|see {natural numbers}} of \define{natural numbers} is the archetypal example of an inductive type\index{inductive type!natural numbers}. The rules we postulate for the type of natural numbers come in four sets, just as the rules for $\Pi$-types:
\begin{enumerate}
\item The formation rule, which asserts that the type $\N$ can be formed.
\item The introduction rules, which provide the zero element and the successor function.
\item The elimination rule. This rule is the type theoretic analogue of the induction principle for $\N$.
\item The computation rules, which assert that any application of the elimination rule behaves as expected on the constructors $\zeroN$ and $\succN$ of $\N$.
\end{enumerate}

\subsubsection{The formation rule of $\N$}

\index{natural numbers!rules for N@{rules for $\N$}!formation}
\index{rules!for N@{for $\N$}!formation}
The type $\N$ is formed by the \define{$\N$-formation} rule
\begin{prooftree}
  \AxiomC{}
  \RightLabel{$\N$-form}
  \UnaryInfC{$\vdash \N~\type$.}
\end{prooftree}
In other words, $\N$ is postulated to be a closed type.

\subsubsection{The introduction rules of $\N$}
Unlike the set of positive integers in Bishop's remarks, Peano's first axiom postulates that $0$ is a natural number. The introduction rules for $\N$ equip it with the \define{zero term} and the \define{successor function}.
\index{natural numbers!rules for N@{rules for $\N$}!introduction rules}
\index{rules!for N@{for $\N$}!introduction rules}
\index{natural numbers!operations on N@{operations on $\N$}!0 N@{$\zeroN$}}
\index{0 N@{$\zeroN$}}
\index{successor function!on N@{on $\N$}}
\index{function!succ N@{$\succN$}}
\index{natural numbers!operations on N@{operations on $\N$}!succ N@{$\succN$}}
\index{succ N@{$\succN$}}

\bigskip
\begin{minipage}{.45\textwidth}
  \begin{prooftree}
    \AxiomC{}
    \UnaryInfC{$\vdash \zeroN:\N$}
  \end{prooftree}
\end{minipage}
\begin{minipage}{.45\textwidth}
  \begin{prooftree}
    \AxiomC{}
    \UnaryInfC{$\vdash \succN:\N\to\N$}
  \end{prooftree}
\end{minipage}

\bigskip
\begin{rmk}
  We annotate the terms $\zeroN$ and $\succN$ of type $\N$ with their type in the subscript, as a reminder that $\zeroN$ and $\succN$ are declared to be terms of type $\N$, and not of any other type. In the next chapter we will introduce the type $\Z$ of the integers, on which we can also define a zero term $\zeroZ$, and a successor function $\succZ$. These should be distinguished from the terms $\zeroN$ and $\succN$. In general, we will make sure that every term is given a unique name. In libraries of mathematics formalized in a computer proof assistant it is also the case that every type must be given a unique name.
\end{rmk}

\subsubsection{The elimination rule of $\N$}

\index{natural numbers!rules for N@{rules for $\N$}!elimination|see {induction}}
\index{natural numbers!rules for N@{rules for $\N$}!induction principle|(}
\index{induction principle!of N@{for $\N$}|(}
To prove properties about the natural numbers, we postulate an \emph{induction principle} for $\N$. For a typical example, it is easy to show by induction that
\begin{equation*}
  1+\dots+n=\frac{n(n+1)}{2}.
\end{equation*}
Similarly, we can define operations by recursion on the natural numbers: the Fibonacci sequence\index{Fibonacci sequence}\index{natural numbers!operations on N@{operations on $\N$}!Fibonacci sequence} is defined by $F(0)=0$, $F(1)=1$, and
\begin{equation*}
  F(n+2)=F(n)+F(n+1).
\end{equation*}
Needless to say, we want an induction principle to hold for the natural numbers in type theory and we also want it to be possible to construct operations on the natural numbers by recursion.

In dependent type theory we may think of a type family $P$ over $\N$ as a \emph{predicate} over $\N$. Especially after we introduce a few more type-forming operations, such as $\Sigma$-types and identity types, it will become clear that the language of dependent type theory expressive enough to find definitions of all of the standard concepts and operations of elementary number theory in type theory. Many of those definitions, the ordering relations $\leq$ and $<$ for example, will make use of type dependency. Then, to prove that $P(n)$ `holds' for all $n$ we just have to construct a dependent function
\begin{equation*}
  \prd{n:\N}P(n).
\end{equation*}

The induction principle for the natural numbers in type theory exactly states what one has to do in order to construct such a dependent function, via the following inference rule:\index{ind N@{$\ind{\N}$}}\index{rules!for N@{for $\N$}!induction principle}\index{natural numbers!indN@{$\indN$}}
\begin{prooftree}
  \def\fCenter{\Gamma}
  \Axiom$\fCenter, n:\N\vdash P(n)~\type$
  \noLine
  \UnaryInf$\fCenter\ \vdash p_0:P(\zeroN)$
  \noLine
  \UnaryInf$\fCenter\ \vdash p_S:\prd{n:\N}P(n)\to P(\succN(n))$
  \RightLabel{$\N$-ind}
  \UnaryInf$\fCenter\ \vdash \ind{\N}(p_0,p_S):\prd{n:\N} P(n)$
\end{prooftree}
Just like for the usual induction principle of the natural numbers, there are two things to be constructed given a type family $P$ over $\N$: in the \define{base case}\index{base case} we need to construct a term $p_0:P(\zeroN)$, and for the \define{inductive step}\index{inductive step} we need to construct a function of type $P(n)\to P(\succN(n))$ for all $n:\N$. And this comes at one immediate advantage: induction and recursion in type theory are one and the same thing!

\begin{rmk}
  We might alternatively present the induction principle of $\N$ as the following inference rule
  \begin{prooftree}
    \AxiomC{$\Gamma,n:\N\vdash P(n)~\type$}
    \UnaryInfC{$\Gamma\vdash \indN : P(\zeroN)\to \Big(\Big(\prd{n:\N}P(n)\to P(\succN(n))\Big)\to \prd{n:\N}P(n)\Big)$}
  \end{prooftree}
  In other words, for any type family $P$ over $\N$ there is a \emph{function} $\ind{\N}$ that takes two arguments, one for the base case and one for the inductive step, and returns a section of $P$. Now it is justified to wonder: is this slightly different presentation of induction equivalent to the previous presentation?
  
  To see that indeed we get such a function from the induction principle (rule $\N$-ind above), we note that the induction principle is stated to hold in an \emph{arbitrary} context $\Gamma$. So let us wield the power of type dependency: by weakening and the variable rule we have the following well-formed terms:
  \begin{align*}
    \Gamma,~p_0:P(\zeroN),~p_S:\prd{n:\N}P(n)\to P(\succN(n)) & \vdash p_0 : P(\zeroN) \\
    \Gamma,~p_0:P(\zeroN),~p_S:\prd{n:\N}P(n)\to P(\succN(n)) & \vdash p_S : \prd{n:\N}P(n)\to P(\succN(n)).
  \end{align*}
  Therefore, the induction principle of $\N$ provides us with a term
  \begin{equation*}
    \Gamma,~p_0:P(\zeroN),~p_S:\prd{n:\N}P(n)\to P(\succN(n)) \vdash \indN(p_0,p_S) : \prd{n:\N}P(n).
  \end{equation*}
  By $\lambda$-abstraction we now obtain a function
  \begin{equation*}
    \indN : P(\zeroN)\to \Big(\Big(\prd{n:\N}P(n)\to P(\succN(n))\Big) \to \prd{n:\N}P(n)\Big)
  \end{equation*}
  in context $\Gamma$. Therefore we see that it does not really matter whether we present the induction principle of $\N$ in a more verbose way as an inference rule with the base case and the inductive step as hypotheses, or as a function taking variables for the base case and the inductive step as arguments.
\end{rmk}
\index{natural numbers!rules for N@{rules for $\N$}!induction|)}
\index{induction principle!for N@{for $\N$}|)}

\subsubsection{The computation rules of $\N$}

\index{computation rules!for N@{for $\N$}|(}
\index{natural numbers!rules for N@{rules for $\N$}!computation rules|(}
The \define{computation rules} for $\N$ postulate that the dependent function $\ind{\N}(P,p_0,p_S)$ behaves as expected when it is applied to $\zeroN$ or a successor. There is one computation rule for each step in the induction principle, covering the base case and the inductive step.

The computation rule for the base case is\index{rules!for N@{for $\N$}!computation rules|(}
\begin{prooftree}
    \def\fCenter{\Gamma}
  \Axiom$\fCenter, n:\N\vdash P(n)~\type$
  \noLine
  \UnaryInf$\fCenter\ \vdash p_0:P(\zeroN)$
  \noLine
  \UnaryInf$\fCenter\ \vdash p_S:\prd{n:\N}P(n)\to P(\succN(n))$
  \UnaryInf$\fCenter\ \vdash \ind{\N}(p_0,p_S,\zeroN)\jdeq p_0 : P(\zeroN)$
\end{prooftree}
Similarly, with the same hypotheses as for the computation rule for the base case, the computation rule for the inductive step is
\begin{prooftree}
\AxiomC{$\cdots$}
\UnaryInfC{$\Gamma, n:\N \vdash  \ind{\N}(p_0,p_S,\succN(n))\jdeq p_S(n,\ind{\N}(p_0,p_S,n)) : P(\succN(n))$}
\end{prooftree}\index{rules!for N@{for $\N$}!computation rules|)}

This completes the formal specification of $\N$.
\index{computation rules!for N@{for $\N$}|)}
\index{natural numbers!rules for N@{rules for $\N$}!computation rules|)}

\subsection{Addition on the natural numbers}

\index{addition on N@{addition on $\N$}|(}
\index{natural numbers!operations on N@{operations on $\N$}!addition|(}
\index{function!addition on N@{addition on $\N$}|(}
Using the induction principle of $\N$ we can perform many familiar constructions. 
For instance, we can define the \define{addition operation} by induction on $\N$.

\begin{defn}
  We define a function\index{add N@{$\addN$}}\index{natural numbers!operations on N@{operations on $\N$}!add N@{$\addN$}}
  \begin{equation*}
    \addN:\N\to (\N\to\N)
  \end{equation*}
  satisfying $\addN(\zeroN,n)\jdeq n$ and $\addN(\succN(m),n)\jdeq\succN(\addN(m,n))$. Usually we will write $n+m$ for $\addN(n,m)$.
\end{defn}

We first give an informal construction of the addition operation, explaining the ideas behind the construction. This is important, because there are many binary operations on the natural numbers. The correctness of a formal construction of a term
\begin{equation*}
  \vdash \N\to(\N\to\N)
\end{equation*}
only shows us that we have correctly constructed a binary operation on the natural numbers, but this doesn't tell us that the operation we've defined is deserving of the name addition. There are indeed many binary operations on the natural numbers, such as the $\minN$, $\maxN$, and multiplication operations, so we need to be careful to make sure that the binary operation we are constructing really is the addition operation.

\begin{proof}[Informal construction]
  Our goal is to construct a function of type
  \begin{equation*}
    \vdash \addN:\N\to (\N\to\N).
  \end{equation*}
  By $\lambda$-abstraction it therefore suffices to construct a term
  \begin{equation*}
    m:\N\vdash \addN(m):\N\to\N.
  \end{equation*}
  Such a term is constructed by induction. Since we are defining addition, we want our definition of $\addN$ to be such that
  \begin{align*}
    \addN(m,\zeroN) & \jdeq n \\
    \addN(m,\succN(n)) & \jdeq \succN(\addN(m,n)). 
  \end{align*}
  In other words, our definition of addition is such that $m+0\jdeq m$ and $m+\succN(n)\jdeq \succN(m+n)$.

  The inductive proof requires us to define a term
  \begin{equation*}
    n:\N\vdash \addzeroN(n)\defeq n:\N
  \end{equation*}
  in the base case, and a term
  \begin{equation*}
    n:\N \vdash \addsuccN(n):\N\to(\N\to\N)
  \end{equation*}
  in the inductive step. The result of the inductive proof will then be a function $\addN(n):\N\to\N$ satisfying
  \begin{align*}
    n:\N & \vdash \addN(n,\zeroN) \jdeq \addzeroN(n) : \N \\
    n:\N & \vdash \addN(n,\succN(m)) \jdeq \addsuccN(n,m,\addN(n,m)).
  \end{align*}
  Anticipating these computation rules, we see that the following choices result in an addition operation with the expected behavior:
  \begin{align*}
    n:\N & \vdash \addzeroN(n)\defeq n : \N \\
    n:\N & \vdash \addsuccN(n)\defeq\const_{\succN}:\N\to(\N\to\N).\qedhere
  \end{align*}
\end{proof}

\begin{proof}[Formal derivation]
The derivation for the construction of $\addsuccN$ looks as follows:
\begin{prooftree}
  \AxiomC{}
  \UnaryInfC{$\vdash\N~\type$}
  \AxiomC{}
  \UnaryInfC{$\vdash\N~\type$}
  \AxiomC{}
  \UnaryInfC{$\vdash \succN:\N\to\N$}
  \BinaryInfC{$x:\N\vdash \succN:\N\to\N$}
  \BinaryInfC{$n:\N,x:\N \vdash \succN:\N\to\N$}
  \UnaryInfC{$n:\N \vdash \addsuccN(n) \defeq \lam{x}\succN:\N\to (\N \to \N)$}
\end{prooftree}
We combine this derivation with the induction principle of $\N$ to complete the construction of addition:
\begin{prooftree}
  \AxiomC{$\vdots$}
  \UnaryInfC{$n:\N\vdash \addzeroN(n) \defeq n:\N$}
  \AxiomC{$\vdots$}
  \UnaryInfC{$n:\N\vdash \addsuccN(n):\N\to (\N \to \N)$}
  \BinaryInfC{$n:\N\vdash\addN(n)\jdeq\indN(\addzeroN,\addsuccN):\N\to \N$}
\end{prooftree}
The asserted judgmental equalities then hold by the computation rules for $\N$.
\end{proof}

\begin{rmk}
  When we define a function $f:\prd{n:\N} P(n)$, we will often do so just by indicating its definition on $\zeroN$ and its definition on $\succN(n)$, by writing
  \begin{align*}
    f(\zeroN) & \defeq p_0 \\
    f(\succN(n)) & \defeq p_S(n,f(n)).
  \end{align*}
  For example, the definition of addition on the natural numbers could be given as
  \begin{align*}
    \addN(\zeroN,n) & \defeq n \\
    \addN(\succN(m),n) & \defeq \succN(\addN(m,n)).
  \end{align*}
  This way of defining a function is called \emph{pattern matching}\index{pattern matching}. A more formal inductive argument can be obtained from a definition by pattern matching if it is possible to obtain from the expression $p_S(n,f(n))$ a general dependent function
  \begin{equation*}
    p_S : \prd{n:\N} P(n)\to P(\succN(n)).
  \end{equation*}
  In practice this is usually the case. Computer proof assistants such as Agda have sophisticated algorithms to allow for definitions by pattern matching.
\end{rmk}

\begin{rmk}
  By the computation rules for $\N$ it follows that
  \begin{equation*}
    m+\zeroN\jdeq m,\qquad\text{and}\qquad m+\succN(n)\jdeq\succN(m+n).
  \end{equation*}
  A simple consequence of this definition is that $\succN(n)\jdeq n+1$, as one would expect. However, the rules that we provided so far are not sufficient to also conclude that $\zeroN+n\jdeq n$ and $\succN(m) + n\jdeq \succN(m+n)$. In fact, we will not be able to prove such judgmental equalities. Nevertheless, once we have introduced the \emph{identity type} in \cref{chap:identity} we will be able to \emph{identify} $\zeroN+n$ with $n$, and $\succN(m)+n$ with $\succN(m+n)$. See \cref{ex:semi-ring-laws-N}. 
\end{rmk}
\index{addition on N@{addition on $\N$}|)}
\index{natural numbers!operations on N@{operations on $\N$}!addition|)}
\index{function!addition on N@{addition on $\N$}|)}

\begin{exercises}
\exercise Define the binary \define{min} and \define{max} functions
  \index{minimum function}
  \index{maximum function}
  \index{function!minN@{$\minN$}}
  \index{function!maxN@{$\maxN$}}
  \index{natural numbers!operations on N@{operations on $\N$}!minN@{$\minN$}}
  \index{natural numbers!operations on N@{operations on $\N$}!maxN@{$\maxN$}}
  \begin{equation*}
    \minN,\maxN:\N\to(\N\to\N).
  \end{equation*}
\exercise Define the \define{multiplication} operation
  \index{multiplication!on N@{on $\N$}}
  \index{function!mul N@{$\mulN$}}
  \index{natural numbers!operations on N@{operations on $\N$}!mul N@{$\mulN$}}
  \index{mul N@{$\mulN$}}
  \begin{equation*}
    \mulN :\N\to(\N\to\N).
  \end{equation*}
\exercise Define the \define{exponentiation function} $n,m\mapsto m^n$ of type $\N\to (\N\to \N)$.
  \index{exponentiation function on N@{exponentiation function on $\N$}}
  \index{function!exponentiation on N@{exponentiation on $\N$}}
  \index{natural numbers!operations on N@{operations on $\N$}!exponentiation}
\exercise Define the \define{factorial} operation $n\mapsto n!$.
  \index{factorial operation}
  \index{function!factorial operation}
  \index{natural numbers!operations on N@{operations on $\N$}!n factorial@{$n"!$}}
\exercise Define the \define{binomial coefficient} $\binom{n}{k}$ for any $n,k:\N$, making sure that $\binom{n}{k}\jdeq 0$ when $n<k$.
  \index{binomial coefficient}
  \index{function!binomial coefficient}
  \index{natural numbers!operations on N@{operations on $\N$}!binomial coefficient}
  \exercise Use the induction principle of $\N$ to define the \define{Fibonacci sequence} as a function $F:\N\to\N$ that satisfies the equations
  \begin{align*}
    F(\zeroN) & \jdeq \zeroN \\
    F(\oneN) & \jdeq \oneN \\
    F(\succN(\succN(n))) & \jdeq F(n)+F(\succN(n)).
  \end{align*}
  \index{Fibonacci sequence}
  \index{natural numbers!operations on N@{operations on $\N$}!Fibonacci sequence}
\end{exercises}
\index{natural numbers|)}

\section{More inductive types}

Analogous to the type of natural numbers, many types can be specified as inductive types. In this section we introduce some further examples of inductive types: the unit type, the empty type, the booleans, coproducts, dependent pair types, and cartesian products. We also introduce the type of integers.

\subsection{The idea of general inductive types}

Just like the type of natural numbers, other inductive types are also specified by their \emph{constructors}, an \emph{induction principle}, and their \emph{computation rules}: 
\begin{enumerate}
\item The constructors tell what structure the inductive type comes equipped with. There may any finite number of constructors, even no constructors at all, in the specification of an inductive type. 
\item The induction principle specifies the data that should be provided in order to construct a section of an arbitrary type family over the inductive type. 
\item The computation rules assert that the inductively defined section agrees on the constructors with the data that was used to define the section. Thus, there is a computation rule for every constructor.
\end{enumerate}
The induction principle and computation rules can be generated automatically once the constructors are specified, but it goes beyond the scope of our course to describe general inductive types.
%For a more general treatment of inductive types, we refer to Chapter 5 of \cite{hottbook}.


\subsection{The unit type}
\index{unit type|(}
\index{inductive type!unit type|(}
A straightforward example of an inductive type is the \emph{unit type}, which has just one constructor. 
Its induction principle is analogous to just the base case of induction on the natural numbers.

\begin{defn}
We define the \define{unit type}\index{1 @{$\unit$}|see {unit type}}\index{unit type} to be a closed type $\unit$\index{unit type!is a closed type} equipped with a closed term\index{unit type!star@{$\ttt$}}
\begin{equation*}
\ttt:\unit,
\end{equation*}
satisfying the induction principle\index{induction principle!of unit type}\index{unit type!induction principle} that for any type family of types $P(x)$ indexed by $x:\unit$, there is a term\index{ind 1@{$\indunit$}}\index{unit type!indunit@{$\indunit$}}
\begin{equation*}
\indunit : P(\ttt)\to\prd{x:\unit}P(x)
\end{equation*}
for which the computation rule\index{computation rules!of unit type}\index{unit type!computation rules}
\begin{equation*}
\indunit(p,\ttt) \jdeq p
\end{equation*}
holds. Sometimes we write $\lam{\ttt}p$ for $\indunit(p)$.
\end{defn}

The induction principle can also be used to define ordinary functions out of the unit type. Indeed, given a type $A$ we can first weaken it to obtain the constant family over $\unit$, with value $A$. Then the induction principle of the unit type provides a function
\begin{equation*}
  \indunit : A \to (\unit\to A).
\end{equation*}
In other words, by the induction principle for the unit type we obtain for every $x:A$ a function $\pt_x\defeq\indunit(x):\unit\to A$.\index{ptx@{$\pt_x$}}
\index{unit type|)}
\index{inductive type!unit type|)}

\subsection{The empty type}
\index{empty type|(}
\index{inductive type!empty type|(}
The empty type is a degenerate example of an inductive type. It does \emph{not} come equipped with any constructors, and therefore there are also no computation rules. The induction principle merely asserts that any type family has a section. In other words: if we assume the empty type has a term, then we can prove anything.

\begin{defn}
We define the \define{empty type}\index{0 @{$\emptyt$}|see {empty type}} to be a type $\emptyt$ satisfying the induction principle\index{induction principle!of empty type}\index{empty type!induction principle} that for any family of types $P(x)$ indexed by $x:\empty$, there is a term\index{ind 0@{$\indempty$}}\index{empty type!indempty@{$\indempty$}}
\begin{equation*}
\indempty : \prd{x:\emptyt}P(x).
\end{equation*}
\end{defn}

The induction principle for the empty type can also be used to construct a function
\begin{equation*}
  \emptyt\to A
\end{equation*}
for any type $A$. Indeed, to obtain this function one first weakens $A$ to obtain the constant family over $\emptyt$ with value $A$, and then the induction principle gives the desired function.

Thus we see that from the empty type anything follows. Therefore, we we see that anything follows from $A$, if we have a function from $A$ to the empty type. This motivates the following definition.

\begin{defn}
  For any type $A$ we define \define{negation}\index{negation!of types}\index{neg (A)@{$\neg A$}|see {negation}} of $A$ by
  \begin{equation*}
    \neg A\defeq A\to\emptyt.
  \end{equation*}
\end{defn}

Since $\neg A$ is the type of functions from $A$ to $\emptyt$, a proof of $\neg A$ is given by assuming that $A$ holds, and then deriving a contradiction. This proof technique is called \define{proof of negation}\index{proof of negation}. Proofs of negation are not to be confused with \emph{proofs by contradiction}\index{proof by contradiction}. In type theory there is no way of obtaining a term of type $A$ from a term of type $(A\to \emptyt)\to\emptyt$.
\index{empty type|)}
\index{inductive type!empty type|)}

\subsection{The booleans}
\index{booleans}
\index{inductive type!booleans}

\begin{defn}
We define the \define{booleans}\index{2 @{$\bool$}|see {booleans}} to be a type $\bool$ that comes equipped with\index{booleans!btrue@{$\btrue$}}\index{booleans!bfalse@{$\bfalse$}}\index{0 2@{$\bfalse$}}\index{1 2@{$\btrue$}}
\begin{align*}
\bfalse & : \bool \\
\btrue & : \bool
\end{align*}
satisfying the induction principle\index{induction principle!of booleans}\index{booleans!induction principle} that for any family of types $P(x)$ indexed by $x:\bool$, there is a term\index{ind 2@{$\indbool$}}
\begin{equation*}
\indbool : P(\bfalse)\to \Big(P(\btrue)\to \prd{x:\bool}P(x)\Big)
\end{equation*}
for which the computation rules\index{computation rules!of booleans}\index{booleans!computation rules}
\begin{align*}
\indbool(p_0,p_1,\bfalse) & \jdeq p_0 \\
\indbool(p_0,p_1,\btrue) & \jdeq p_1
\end{align*}
hold.
\end{defn}

Just as in the cases for the unit type and the empty type, the induction principle for the booleans can also be used to construct an ordinary function $\bool\to A$, provided that we can construct two terms of type $A$. Indeed, by the induction principle for the booleans there is a function
\begin{equation*}
  \indbool : A \to (A\to A^\bool)
\end{equation*}
for any type $A$.

\begin{eg}\label{eg:boolean-ops}
  \index{boolean operations|(}\index{boolean logic|(}
  Using the induction principle of $\bool$ we can define all the operations of Boolean algebra\index{boolean algebra}. For example, the \define{boolean negation}\index{booleans!negation} operation $\negbool : \bool \to \bool$\index{negation function!on booleans}\index{neg 2@{$\negbool$}}\index{booleans!neg 2@{$\negbool$}} is defined by
  \begin{align*}
    \negbool(\btrue) & \defeq \bfalse & \negbool(\bfalse) & \defeq \btrue.
  \end{align*}
  The \define{boolean conjunction}\index{booleans!conjunction} operation $\blank\land\blank : \bool \to (\bool\to \bool)$ is defined by
  \begin{align*}
    \btrue\land\btrue & \defeq \btrue & \bfalse\land\btrue & \defeq \bfalse \\
    \btrue\land\bfalse & \defeq \bfalse & \bfalse\land\bfalse & \defeq \bfalse.
  \end{align*}
  The \define{boolean disjunction}\index{booleans!disjunction} operation $\blank\lor\blank : \bool \to (\bool\to \bool)$ is defined by
  \begin{align*}
    \btrue\lor\btrue & \defeq \btrue & \bfalse\lor\btrue & \defeq \btrue \\
    \btrue\lor\bfalse & \defeq \btrue & \bfalse\lor\bfalse & \defeq \bfalse.
  \end{align*}  
  We leave the definitions of some of the other boolean operations as \cref{ex:boolean-operation}. Note that the method of defining the boolean operations by the induction principle of $\bool$ is not that different from defining them by truth tables\index{truth tables}.

  Boolean logic is important, but it won't be very prominent in this course. The reason is simple: in type theory it is more natural to use the `logic' of types that is provided by the inference rules.\index{boolean operations|)}\index{boolean logic|)}
\end{eg}
\index{booleans|)}
\index{inductive type!booleans|)}

\subsection{Coproducts and the type of integers}
\index{coproduct|(}
\index{inductive type!coproduct|(}
\begin{defn}
Let $A$ and $B$ be types. We define the \define{coproduct}\index{disjoint sum|see {coproduct}} $A+B$\index{A + B@{$A+B$}|see {coproduct}} to be a type that comes equipped with\index{inl@{$\inl$}}\index{coproduct!inl@{$\inl$}}\index{inr@{$\inr$}}\index{coproduct!inr@{$\inr$}}
\begin{align*}
\inl & : A \to A+B \\
\inr & : B \to A+B
\end{align*}
satisfying the induction principle\index{induction principle!of coproduct}\index{coproduct!induction principle} that for any family of types $P(x)$ indexed by $x:A+B$, there is a term\index{ind +@{$\ind{+}$}}\index{coproduct!ind+@{$\ind{+}$}}
\begin{equation*}
\ind{+} : \Big(\prd{x:A}P(\inl(x))\Big)\to\Big(\prd{y:B}P(\inr(y))\Big)\to\prd{z:A+B}P(z)
\end{equation*}
for which the computation rules\index{computation rules!of coproduct}\index{coproduct!computation rules}
\begin{align*}
\ind{+}(f,g,\inl(x)) & \jdeq f(x) \\
\inr{+}(f,g,\inr(y)) & \jdeq g(y)
\end{align*}
hold. Sometimes we write $[f,g]$ for $\ind{+}(f,g)$.
\end{defn}

The coproduct of two types is sometimes also called the \define{disjoint sum}. By the induction principle of coproducts it follows that we have a function
\begin{equation*}
  (A\to X) \to \big((B\to X) \to (A+B\to X)\big)
\end{equation*}
for any type $X$. Note that this special case of the induction principle of coproducts is very much like the elimination rule of disjunction in first order logic: if $P$, $P'$, and $Q$ are propositions, then we have
\begin{equation*}
  (P\to Q)\to \big((P'\to Q)\to (P\lor P'\to Q)\big).
\end{equation*}
Indeed, we can think of \emph{propositions as types} and of terms as their constructive proofs. Under this interpretation of type theory the coproduct is indeed the disjunction.

\index{integers|(}
An important example of a type that can be defined using coproducts is the type $\Z$ of integers.\index{coproduct!Z@{$\Z$}}

\begin{defn}
  We define the \define{integers}\index{Z@{$\Z$}|see {integers}} to be the type $\Z\defeq\N+(\unit+\N)$. The type of integers comes equipped with inclusion functions of the positive and negative integers\index{integers!in-pos@{$\inpos$}}\index{integers!in-neg@{$\inneg$}}
  \begin{align*}
    \inpos & \defeq \inr\circ\inr \\
    \inneg & \defeq \inl,
  \end{align*}
  which are both of type $\N\to\Z$, and the constants\index{integers!-1 Z@{$-1_\Z$}}\index{integers!0 Z@{$0_\Z$}}\index{integers!1 Z@{$1_\Z$}}\index{-1 Z@{$-1_\Z$}}\index{0 Z@{$0_\Z$}}\index{1 Z@{$1_{\Z}$}}
  \begin{align*}
    -1_\Z & \defeq \inneg(0)\\
    0_\Z & \defeq \inr(\inl(\ttt))\\
    1_\Z & \defeq \inpos(0).
  \end{align*}
\end{defn}

In the following lemma we derive an induction principle\index{induction principle!of Z@{of $\Z$}}\index{integers!induction principle} for $\Z$, which can be used in many familiar constructions on $\Z$, such as in the definitions of addition and multiplication.

\begin{lem}\label{lem:Z_ind}
  Consider a type family $P$ over $\Z$. If we are given
  \begin{align*}
    p_{-1} & :P(-1_\Z) \\
    p_{-S} & : \prd{n:\N}P(\inneg(n))\to P(\inneg(\succN(n)))\\
    p_{0} & : P(0_\Z) \\
    p_{1} & : P(1_\Z) \\
    p_{S} & : \prd{n:\N}P(\inpos(n))\to P(\inpos(\succN(n))),
  \end{align*}
  then we can construct a dependent function $f:\prd{k:\Z}P(k)$ for which the following judgmental equalities hold:\index{integers!computation rules}\index{computation rules!of Z@{of $\Z$}}
  \begin{align*}
    f(-1_\Z) & \jdeq p_{-1} \\
    f(\inneg(\succN(n))) & \jdeq p_{-S}(n,f(\inneg(n))) \\
    f(0_\Z) & \jdeq p_{0} \\
    f(1_\Z) & \jdeq p_{1} \\
    f(\inpos(\succN(n))) & \jdeq p_S(n,f(\inpos(n))).
  \end{align*}
\end{lem}

\begin{proof}
  Since $\Z$ is the coproduct of $\N$ and $\unit+\N$, it suffices to define
  \begin{align*}
    p_{inl} & : \prd{n:\N}P(\inl(n)) \\
    p_{inr} & : \prd{t:\unit+\N}P(\inr(t)).
  \end{align*}
  Note that $\inneg\jdeq\inl$ and $-1_\Z\jdeq \inneg(\zeroN)$. In order to define $p_{inl}$ we use induction on the natural numbers, so it suffices to define
  \begin{align*}
    p_{-1} & : P(-1) \\
    p_{-S} & : \prd{n:\N} P(\inneg(n))\to P(\inneg(\succN(n))).
  \end{align*}
  Similarly, we proceed by coproduct induction, followed by induction on $\unit$ in the left case and induction on $\N$ on the right case, in order to define $p_{inr}$. 
\end{proof}

As an application we define the successor function on the integers.

\begin{defn}
We define the \define{successor function}\index{successor function!on Z@{on $\Z$}}\index{function!succ Z@{$\succZ$}} on the integers $\succZ:\Z\to\Z$\index{succ Z@{$\succZ$}}\index{integers!succ Z@{$\succZ$}} using the induction principle of \cref{lem:Z_ind}, taking
\begin{align*}
\succZ(-1_\Z) & \defeq 0_\N \\
\succZ(\inneg(\succN(n))) & \defeq \inneg(n) \\
\succZ(0_\Z) & \defeq 1_\N \\
\succZ(1_\Z) & \defeq \inpos(1_\N) \\
\succZ(\inpos(\succN(n))) & \defeq \inpos(\succN(\succN(n))).
\end{align*}
\end{defn}
\index{integers|)}
\index{coproduct|)}
\index{inductive type!coproduct|)}

\subsection{Dependent pair types}

\index{dependent pair type|(}
\index{inductive type!dependent pair type|(}

Given a type family $B$ over $A$, we may consider pairs $(a,b)$ of terms, where $a:A$ and $b:B(a)$. Note that the type of $b$ depends on the first term in the pair, so we call such a pair a \define{dependent pair}\index{dependent pair}.

The \emph{dependent pair type} is an inductive type that is generated by the dependent pairs.


\begin{defn}
  Consider a type family $B$ over $A$.
  The \define{dependent pair type} (or $\Sigma$-type) \index{Sigma-type@{$\Sigma$-type}|see {dependent pair type}}is defined to be the inductive type $\sm{x:A}B(x)$ equipped with a \define{pairing function}\index{pairing function}\index{(-,-)@{$(\blank,\blank)$}}\index{dependent pair type!(-,-)@{$(\blank,\blank)$}}
\begin{equation*}
(\blank,\blank):\prd{x:A} \Big(B(x)\to \sm{y:A}B(y)\Big).
\end{equation*}
The induction principle\index{induction principle!of Sigma types@{of $\Sigma$-types}}\index{dependent pair type!induction principle} for $\sm{x:A}B(x)$ asserts that for any family of types $P(p)$ indexed by $p:\sm{x:A}B(x)$, there is a function\index{dependent pair type!indSigma@{$\ind{\Sigma}$}}\index{ind Sigma@{$\ind{\Sigma}$}}
\begin{equation*}
\ind{\Sigma}:\Big(\prd{x:A}\prd{y:B(x)}P(x,y)\Big)\to\Big(\prd{p:\sm{x:A}B(x)}P(p)\Big).
\end{equation*}
satisfying the computation rule\index{computation rules!of Sigma types@{of $\Sigma$-types}}\index{dependent pair type!computation rule}
\begin{equation*}
\ind{\Sigma}(f,(x,y))\jdeq f(x,y).
\end{equation*}
Sometimes we write $\lam{(x,y)}f(x,y)$ for $\ind{\Sigma}(\lam{x}\lam{y}f(x,y))$. 
\end{defn}

\begin{defn}
Given a type $A$ and a type family $B$ over $A$, the \define{first projection map}\index{first projection map}\index{projection maps!first projection}\index{dependent pair type!pr 1@{$\proj 1$}}\index{pr 1@{$\proj 1$}}\index{function!pr 1@{$\proj 1$}}
\begin{equation*}
\proj 1:\Big(\sm{x:A}B(x)\Big)\to A
\end{equation*}
is defined by induction as
\begin{equation*}
\proj 1\defeq \lam{(x,y)}x.
\end{equation*}
The \define{second projection map}\index{second projection map}\index{projection map!second projection}\index{dependent pair type!pr 2@{$\proj 2$}}\index{pr 2@{$\proj 2$}}\index{function!pr 2@{$\proj 2$}} is a dependent function
\begin{equation*}
\proj 2 : \prd{p:\sm{x:A}B(x)} B(\proj 1(p))
\end{equation*}
defined by induction as
\begin{equation*}
\proj 2\defeq \lam{(x,y)}y.
\end{equation*}
By the computation rule we have
\begin{align*}
\proj 1 (x,y) & \jdeq x \\
\proj 2 (x,y) & \jdeq y.
\end{align*}
\end{defn}
\index{dependent pair type|)}
\index{inductive type!dependent pair type|)}

\subsection{Cartesian products}

\index{cartesian product|(}
\index{inductive type!cartesian product|(}
A special case of the $\Sigma$-type occurs when the $B$ is a constant family over $A$, i.e., when $B$ is just a type.
In this case, the inductive type $\sm{x:A}B(x)$ is generated by \emph{ordinary} pairs $(x,y)$ where $x:A$ and $y:B$. In other words, if $B$ does not depend on $A$, then the type $\sm{x:A}B$ is the \emph{(cartesian) product} $A\times B$.
The cartesian product is a very common special case of the dependent pair type, just as the type $A\to B$ of ordinary functions from $A\to B$ is a common special case of the dependent product. Therefore we provide its specification along with the induction principle for cartesian products.

\begin{defn}
Consider two types $A$ and $B$. The \define{(cartesian) product}\index{product of types}\index{A x B@{$A\times B$}|see {cartesian product}} of $A$ and $B$ is defined as the inductive type $A\times B$ with constructor
\begin{equation*}
(\blank,\blank):A\to (B\to A\times B).
\end{equation*}
The induction principle\index{induction principle!of cartesian products}\index{cartesian product!induction principle} for $A\times B$ asserts that for any type family $P$ over $A\times B$, one has\index{ind times@{$\ind{\times}$}}\index{cartesian product!indtimes@{$\ind{\times}$}}
\begin{equation*}
\ind{\times} : \Big(\prd{x:A}\prd{y:B}P(a,b)\Big)\to\Big(\prd{p:A\times B} P(p)\Big)
\end{equation*}
satisfying the computation rule\index{computation rules!of cartesian product}\index{cartesian product!computation rule} that
\begin{align*}
\ind{\times}(f,(x,y)) & \jdeq f(x,y).
\end{align*}
\end{defn}

The projection maps are defined similarly to the projection maps of $\Sigma$-types. When one thinks of types as propositions\index{propositions as types!conjunction}, then $A\times B$ is interpreted as the conjunction of $A$ and $B$.
\index{cartesian product|)}
\index{inductive type!cartesian product|)}

\begin{exercises}
\exercise
  \index{rules!for unit type}\index{unit type!rules}
  \index{rules!for empty type}\index{empty type!rules}
  \index{rules!for booleans}\index{booleans!rules}
  \index{rules!for coproduct}\index{coproduct!rules}
  \index{rules!for dependent pair type}\index{dependent pair type!rules}
  \index{rules!for cartesian product}\index{cartesian product!rules}
  Write the rules for $\unit$, $\emptyt$, $\bool$, $A+B$, $\sm{x:A}B(x)$, and $A\times B$. As usual, present the rules in four sets:
  \begin{enumerate}
  \item A formation rule.
  \item Introduction rules.
  \item An elimination rule.
  \item Computation rules.
  \end{enumerate}
  \exercise Let $P$ and $Q$ be types. Use the fact that $\neg P$\index{negation} is defined as the type $P\to\emptyt$ of functions from $P$ to the empty type\index{empty type}, to give type theoretic proofs of the following taugologies\index{tautologies} of constructive logic\index{constructive logic}.\label{ex:dne-dec}
  \begin{subexenum}
  \item $P\to\neg\neg P$
  \item $(P\to Q)\to(\neg\neg P\to\neg\neg Q)$
  \item $(P+\neg P)\to(\neg\neg P\to P)$
  \item $\neg\neg(P+\neg P)$
  \item $\neg\neg(\neg\neg P \to P)$
  \item $(P\to \neg\neg Q)\to (\neg\neg P \to\neg\neg Q)$
  \item $\neg\neg\neg P \to \neg P$
  \item $\neg\neg(P \to \neg\neg Q)\to (P\to\neg\neg Q)$
  \item $\neg\neg((\neg\neg P)\times(\neg\neg Q))\to (\neg\neg P)\times(\neg\neg Q)$
  \end{subexenum}
\exercise \label{ex:boolean-operation}Define the following operations of Boolean algebra:\index{boolean algebra}\index{booleans!exclusive disjunction}\index{booleans!implication}\index{booleans!if and only if}\index{booleans!Peirce's arrow}\index{booleans!Sheffer stroke}
  \begin{center}
    \begin{tabular}{ll}
      exclusive disjunction & $p \oplus q$ \\
      implication & $p \Rightarrow q$ \\
      if and only if & $p \Leftrightarrow q$ \\
      Peirce's arrow (neither \dots{} nor) & $p \downarrow q$ \\
      Sheffer stroke (not both) & $p\mid q$.
    \end{tabular}
  \end{center}
  Here $p$ and $q$ range over $\bool$. 
\exercise \label{ex:int_pred}\index{integers|(}\index{predecessor function}\index{function!pred Z@{$\predZ$}}\index{integers!pred Z@{$\predZ$}}\index{pred Z@{$\predZ$}}Define the predecessor function $\predZ:\Z\to \Z$.
\exercise \label{ex:int_group_ops}\index{group operations!on Z@{on $\Z$}}Define the group operations\index{add Z@{$\addZ$}}\index{integers!add Z@{$\addZ$}}\index{neg Z@{$\negZ$}}\index{integers!neg Z@{$\negZ$}}\index{mul Z@{$\mulZ$}}\index{integers!mul Z@{$\mulZ$}}
  \begin{align*}
    \addZ & : \Z \to (\Z \to \Z) \\
    \negZ & : \Z \to \Z,
    \intertext{and define the multiplication}
    \mulZ & : \Z \to (\Z \to \Z).
  \end{align*}
\exercise Construct a function $F:\Z\to\Z$ that extends the Fibonacci sequence\index{Fibonacci sequence}\index{integers!Fibonacci sequence} to the negative integers
  \begin{equation*}
    \ldots,5,-3,2,-1,1,0,1,1,2,3,5,8,13,\ldots
  \end{equation*}
  in the expected way.\index{integers|)}
\exercise \label{ex:one_plus_one} Show that $\unit+\unit$ satisfies the same induction principle\index{induction principle!of booleans} as $\bool$, i.e., define
  \begin{align*}
    t_0 & : \unit + \unit \\
    t_1 & : \unit + \unit,
  \end{align*}
  and show that for any type family $P$ over $\unit+\unit$ there is a function
  \begin{align*}
    \ind{\unit+\unit}:P(t_0)\to \Big(P(t_1)\to \prd{t:\unit+\unit}P(t)\Big)
  \end{align*}
  satisfying
  \begin{align*}
    \ind{\unit+\unit}(p_0,p_1,t_0) & \jdeq p_0 \\
    \ind{\unit+\unit}(p_0,p_1,t_1) & \jdeq p_1.
  \end{align*}
  In other words, \emph{type theory cannot distinguish between the types $\bool$ and $\unit+\unit$.}
\exercise \label{ex:lists}For any type $A$ we can define the type $\lst(A)$\index{list A@{$\lst(A)$}|see {lists in $A$}} of \define{lists}\index{lists in A @{lists in $A$}}\index{inductive type!list A@{$\lst(A)$}} elements of $A$ as the inductive type with constructors\index{lists in A@{lists in $A$}!nil@{$\nil$}}\index{nil@{$\nil$}}\index{cons(a,l)@{$\cons(a,l)$}}\index{lists in A@{lists in $A$}!cons@{$\cons$}}
  \begin{align*}
    \nil & : \lst(A) \\
    \cons & : A \to (\lst(A) \to \lst(A)).
  \end{align*}
  \begin{subexenum}
  \item Write down the induction principle and the computation rules for $\lst(A)$.\index{induction principle!list A@{$\lst(A)$}}\index{lists in A@{lists in $A$}!induction principle}
  \item Let $A$ and $B$ be types, suppose that $b:B$, and consider a binary operation $\mu:A\to (B \to B)$. Define a function\index{fold-list@{$\foldlist$}}\index{lists in A@{lists in $A$}!fold-list@{$\foldlist$}}
    \begin{equation*}
      \foldlist(\mu) : \lst(A)\to B
    \end{equation*}
    that iterates the operation $\mu$, starting with $\foldlist(\mu,\nil)\defeq b$.
  \item Define a function $\lengthlist:\lst(A)\to\N$.\index{length-list@{$\lengthlist$}}\index{lists in A@{lists in $A$}!length-list@{$\lengthlist$}}
  \item Define a function\index{sum-list@{$\sumlist$}}\index{lists in A@{lists in $A$}!sum-list@{$\sumlist$}}
    \begin{equation*}
      \sumlist : \lst(\N) \to \N
    \end{equation*}
    that adds all the elements in a list of natural numbers.
  \item Define a function\index{concat-list@{$\concatlist$}}\index{lists in A@{lists in $A$}!concat-list@{$\concatlist$}}\index{concatenation!of lists}
    \begin{equation*}
      \concatlist : \lst(A) \to (\lst(A) \to \lst(A))
    \end{equation*}
    that concatenates any two lists of elements in $A$.
  \item Define a function\index{flatten-list@{$\flattenlist$}}\index{lists in A@{lists in $A$}!flatten-list@{$\flattenlist$}}
    \begin{equation*}
      \flattenlist : \lst(\lst(A)) \to \lst(A)
    \end{equation*}
    that concatenates all the lists in a lists of lists in $A$.
  \item Define a function $\reverselist : \lst(A) \to \lst(A)$ that reverses the order of the elements in any list.\index{reverse-list@{$\reverselist$}}\index{lists in A@{lists in $A$}!reverse-list@{$\reverselist$}}
  \end{subexenum}
\end{exercises}

\section{Identity types}\label{chap:identity}

\index{identity type|(}
\index{inductive type!identity type|(}
From the perspective of types as proof-relevant propositions, how should we think of \emph{equality} in type theory? Given a type $A$, and two terms $x,y:A$, the equality $\id{x}{y}$ should again be a type. Indeed, we want to \emph{use} type theory to prove equalities. \emph{Dependent} type theory provides us with a convenient setting for this: the equality type $\id{x}{y}$ is dependent on $x,y:A$. 

Then, if $\id{x}{y}$ is to be a type, how should we think of the terms of $\id{x}{y}$. A term $p:\id{x}{y}$ witnesses that $x$ and $y$ are equal terms of type $A$. In other words $p:\id{x}{y}$ is an \emph{identification} of $x$ and $y$. In a proof-relevant world, there might be many terms of type $\id{x}{y}$. I.e., there might be many identifications of $x$ and $y$. And, since $\id{x}{y}$ is itself a type, we can form the type $\id{p}{q}$ for any two identifications $p,q:\id{x}{y}$. That is, since $\id{x}{y}$ is a type, we may also use the type theory to prove things \emph{about} identifications (for instance, that two given such identifications can themselves be identified), and we may use the type theory to perform constructions with them. As we will see shortly, we can give every type a groupoidal structure.

Clearly, the equality type should not just be any type dependent on $x,y:A$. Then how do we form the equality type, and what ways are there to use identifications in constructions in type theory? The answer to both these questions is that we will form the identity type as an \emph{inductive} type, generated by just a reflexivity term providing an identification of $x$ to itself. The induction principle then provides us with a way of performing constructions with identifications, such as concatenating them, inverting them, and so on. Thus, the identity type is equipped with a reflexivity term, and further possesses the structure that are generated by its induction principle and by the type theory. This inductive construction of the identity type is elegant, beautifully simple, but far from trivial!

The situation where two terms can be identified in possibly more than one way is analogous to the situation in \emph{homotopy theory}, where two points of a space can be connected by possibly more than one \emph{path}. Indeed, for any two points $x,y$ in a space, there is a \emph{space of paths} from $x$ to $y$. Moreover, between any two paths from $x$ to $y$ there is a space of \emph{homotopies} between them, and so on. This leads to the homotopy interpretation of type theory, outlined in \cref{tab:homotopy_interpretation}. The connection between homotopy theory and type theory been made precise by the construction of homotopical models of type theory, and it has led to the fruitful research area of \emph{synthetic homotopy theory}, the subfield of \emph{homotopy type theory} that is the topic of this course.

\begin{table}
\begin{center}
\caption{\label{tab:homotopy_interpretation}The homotopy interpretation\index{Homotopy interpretation}}
\begin{tabular}{ll}
\toprule
\emph{Type theory} &  \emph{Homotopy theory} \\
\midrule
Types  & Spaces \\
Dependent types & Fibrations \\
Terms & Points \\
Dependent pair type & Total space \\
Identity type & Path fibration\\
\bottomrule
\end{tabular}
\end{center}
\end{table}

\subsection{The inductive definition of identity types}

\begin{defn}
  Consider a type $A$ and let $a:A$. Then we define the \define{identity type} of $A$ at $a$ as an inductive family of types $a =_A x$\index{a = x@{$a = x$}|see {identity type}} indexed by $x:A$, of which the constructor is\index{refl@{$\refl{}$}}\index{identity type!refl@{$\refl{}$}}
  \begin{equation*}
    \refl{a}:a=_Aa.
  \end{equation*}
  The induction principle of the identity type\index{identity type!induction principle}\index{induction principle!of the identity type} postulates that for any family of types $P(x,p)$ indexed by $x:A$ and $p:a=_A x$, there is a function\index{path-ind@{$\pathind$}}\index{identity type!path-ind@{$\pathind$}}
  \begin{equation*}
    \pathind_a:P(a,\refl{a}) \to \prd{x:A}\prd{p:a=_A x} P(x,p)
  \end{equation*}
  that satisfies $\pathind_a(p,a,\refl{a})\jdeq p$.

  A term of type $a=_A x$ is also called an \define{identification}\index{identification}\index{identity type!identification} of $a$ with $x$, and sometimes it is called a \define{path}\index{path}\index{identity type!path} from $a$ to $x$.
The induction principle for identity types is sometimes called \define{identification elimination}\index{identification elimination}\index{induction principle!identification elimination}\index{identity type!identification elimination} or \define{path induction}\index{path induction}\index{identity type!path induction}\index{induction principle!path induction}. We also write $\idtypevar{A}$\index{Id A@{$\idtypevar{A}$}|see {identity type}} for the identity type on $A$, and often we write $a=x$ for the type of identifications of $a$ with $x$, omitting reference to the ambient type $A$.
\end{defn}

\begin{rmk}
  We see that the identity type is not just an inductive type, like the inductive types $\N$, $\emptyt$, and $\unit$ for example, but it is and inductive \emph{family} of types. Even though we have a type $a=_A x$ for any $x:A$, the constructor only provides a term $\refl{a}:a=_A a$, identifying $a$ with itself. The induction principle then asserts that in order to prove something about all identifications of $a$ with some $x:A$, it suffices to prove this assertion about $\refl{a}$ only. We will see in the next sections that this induction principle is strong enough to derive many familiar facts about equality, namely that it is a symmetric and transitive relation, and that all functions preserve equality.
\end{rmk}

\begin{rmk}
  \index{rules!identity type|(}\index{identity type!rules|(}
  Since the identity types require getting used to, we provide the formal rules
  for identity types. The identity type is formed by the formation rule:
  \begin{prooftree}
    \AxiomC{$\Gamma\vdash a:A$}
    \UnaryInfC{$\Gamma,x:A\vdash a=_A x~\type$}
  \end{prooftree}
  The constructor of the identity type is then given by the introduction rule:
  \begin{prooftree}
    \AxiomC{$\Gamma\vdash a:A$}
    \UnaryInfC{$\Gamma\vdash \refl{a}:a=_A a$}
  \end{prooftree}
  The induction principle is now given by the elimination rule:
  \begin{prooftree}
    \AxiomC{$\Gamma\vdash a:A$}
    \AxiomC{$\Gamma,x:A,p:a=_A x\vdash P(x,p)~\type$}
    \BinaryInfC{$\Gamma\vdash \pathind_a:P(a,\refl{a})\to\prd{x:A}\prd{p:a=_A x}P(x,p)$}
  \end{prooftree}
  And finally the computation rule is:
  \begin{prooftree}
    \AxiomC{$\Gamma\vdash a:A$}
    \AxiomC{$\Gamma,x:A,p:a=_A x\vdash P(x,p)~\type$}
    \BinaryInfC{$\Gamma\vdash \pathind_a(p,a,\refl{a})\jdeq p : P(a,\refl{a})$}
  \end{prooftree}
  \index{rules!identity type|)}\index{identity type!rules|)}
\end{rmk}

\begin{rmk}
  One might wonder whether it is also possible to form the identity type at a \emph{variable} of type $A$, rather than at a term. This is certainly possible: since we can form the identity type in \emph{any} context, we can form the identity type at a variable $x:A$ as follows:
  \begin{prooftree}
    \AxiomC{$\Gamma,x:A\vdash x:A$}
    \UnaryInfC{$\Gamma,x:A,y:A\vdash x=_A y~\type$}
  \end{prooftree}
  In this way we obtain the `binary' identity type. Its constructor is then also indexed by $x:A$. We have the following introduction rule
  \begin{prooftree}
    \AxiomC{$\Gamma,x:A\vdash x:A$}
    \UnaryInfC{$\Gamma,x:A\vdash \refl{x}:x=_A x$}
  \end{prooftree}
  and similarly we have elimination and computation rules.
\end{rmk}

\subsection{The groupoidal structure of types}\label{sec:groupoid}
\index{groupoid laws!of identifications|(}
We show that identifications can be \emph{concatenated} and \emph{inverted}, which corresponds to the transitivity and symmetry of the identity type.

\begin{defn}\label{defn:id_concat}
Let $A$ be a type. We define the \define{concatenation}\index{concatenation!for identifications}\index{concat@{$\concat$}} operation
\begin{equation*}
\concat : \prd{x,y,z:A} (\id{x}{y})\to(\id{y}{z})\to (\id{x}{z}).
\end{equation*}
We will write $\ct{p}{q}$ for $\concat(p,q)$.
\end{defn}

\begin{constr}
We construct the concatenation operation by path induction. It suffices to construct
\begin{equation*}
\concat(\refl{x}):\prd{z:A} (x=z)\to(x=z).
\end{equation*}
Here we take $\concat(\refl{x})_z \jdeq \idfunc[(x=z)]$. 
Explicitly, the term we have constructed is
\begin{equation*}
\lam{x}\pathind_x(\lam{z}\idfunc[(\id{x}{z})]):\prd{x,y:A} (x=y)\to \prd{z:A} (y=z)\to (x=z).
\end{equation*}
To obtain a term of the asserted type we need to swap the order of the arguments $p:x=y$ and $z:A$, using \cref{ex:swap}.
\end{constr}

\begin{defn}\label{defn:id_inv}
Let $A$ be a type. We define the \define{inverse operation}\index{inverse operation!for identifications}\index{inv@{$\invfunc$}}
\begin{equation*}
\invfunc:\prd{x,y:A} (x=y)\to (y=x).
\end{equation*}
Most of the time we will write $p^{-1}$ for $\invfunc(p)$.
\end{defn}

\begin{constr}
We construct the inverse operation by path induction. It suffices to construct
\begin{equation*}
\invfunc(\refl{x}): x=x,
\end{equation*}
for any $x:A$. Here we take $\invfunc(\refl{x})\defeq \refl{x}$.
\end{constr}

The next question is whether the concatenation and inverting operations on paths behave as expected. More concretely, is path concatenation associative, does it satisfy the unit laws, and is the inverse of a path indeed a two-sided inverse?

For example, in the case of associativity we are asking to compare the paths
\begin{equation*}
  \ct{(\ct{p}{q})}{r}\qquad\text{and}\qquad\ct{p}{(\ct{q}{r})}
\end{equation*}
for any $p:x=y$, $q:y=z$, and $r:z=w$ in a type $A$. The computation rules of path induction are not strong enough to conclude that $\ct{(\ct{p}{q})}{r}$ and $\ct{p}{(\ct{q}{r})}$ are judgmentally equal. However, both $\ct{(\ct{p}{q})}{r}$ and $\ct{p}{(\ct{q}{r})}$ are terms of the same type: they are identifications of type $x=w$. Since the identity type is a type like any other, we can ask whether there is an \emph{identification}
\begin{equation*}
\ct{(\ct{p}{q})}{r}=\ct{p}{(\ct{q}{r})}.
\end{equation*}
This is a very useful idea: while it is often impossible to show that two terms of the same type are judgmentally equal, it may be the case that those two terms can be \emph{identified}. Indeed, we identify two terms by constructing a term of the identity type, and we can use all the type theory at our disposal in order to construct such a term. In this way we can show, for example, that addition on the natural numbers or on the integers is associative and satisfies the unit laws. And indeed, here we will show that path concatenation is associative and satisfies the unit laws.

\begin{defn}\label{defn:id_assoc}
  Let $A$ be a type and consider three consecutive paths
  \begin{equation*}
    \begin{tikzcd}
      x \arrow[r,equals,"p"] & y \arrow[r,equals,"q"] & z \arrow[r,equals,"r"] & w
    \end{tikzcd}
  \end{equation*}
  in $A$. We define the \define{associator}\index{associativity!of path concatenation}
  \begin{equation*}
    \assoc(p,q,r) : \ct{(\ct{p}{q})}{r}=\ct{p}{(\ct{q}{r})}.
  \end{equation*}
\end{defn}

\begin{constr}
By path induction it suffices to show that
\begin{equation*}
\prd{z:A}\prd{q:x=z}\prd{w:A}\prd{r:z=w} \ct{(\ct{\refl{x}}{q})}{r}= \ct{\refl{x}}{(\ct{q}{r})}.
\end{equation*}
Let $q:x=z$ and $r:z=w$. Note that by the computation rule of the path induction principle we have a judgmental equality $\ct{\refl{x}}{q}\jdeq q$. Therefore we conclude that
\begin{equation*}
  \ct{(\ct{\refl{x}}{q})}{r}\jdeq \ct{q}{r}.
\end{equation*}
Similarly we have a judgmental equality $\ct{\refl{x}}{(\ct{q}{r})}\jdeq \ct{q}{r}$. Thus we see that the left-hand side and the right-hand side in
\begin{equation*}
  \ct{(\ct{\refl{x}}{q})}{r}=\ct{\refl{x}}{(\ct{q}{r})}
\end{equation*}
are judgmentally equal, so we can simply define $\assoc(\refl{x},q,r)\defeq\refl{\ct{q}{r}}$.
\end{constr}

\begin{defn}\label{defn:id_unit}
Let $A$ be a type. We define the left and right \define{unit law operations}\index{unit law operations!for identifications}, which assigns to each $p:x=y$ the terms\index{left unit@{$\leftunit$}}\index{right unit@{$\rightunit$}}
\begin{align*}
\leftunit(p) & : \ct{\refl{x}}{p}=p \\
\rightunit(p) & : \ct{p}{\refl{y}}=p,
\end{align*}
respectively.
\end{defn}

\begin{constr}
By identification elimination it suffices to construct
\begin{align*}
\leftunit(\refl{x}) & : \ct{\refl{x}}{\refl{x}} = \refl{x} \\
\rightunit(\refl{x}) & : \ct{\refl{x}}{\refl{x}} = \refl{x}.
\end{align*}
In both cases we take $\refl{\refl{x}}$.
\end{constr}

\begin{defn}\label{defn:id_invlaw}
Let $A$ be a type. We define left and right \define{inverse law operations}\index{inverse law operations!for identifications}\index{left inv@{$\leftinv$}}\index{right inv@{$\rightinv$}}
\begin{align*}
\leftinv(p) & : \ct{p^{-1}}{p} = \refl{y} \\
\rightinv(p) & : \ct{p}{p^{-1}} = \refl{x}.
\end{align*}
\end{defn}

\begin{constr}
By identification elimination it suffices to construct
\begin{align*}
\leftinv(\refl{x}) & : \ct{\refl{x}^{-1}}{\refl{x}} = \refl{x} \\
\rightinv(\refl{x}) & : \ct{\refl{x}}{\refl{x}^{-1}} = \refl{x}.
\end{align*}
Using the computation rules we see that
\begin{equation*}
\ct{\refl{x}^{-1}}{\refl{x}}\jdeq \ct{\refl{x}}{\refl{x}}\jdeq\refl{x},
\end{equation*}
so we define $\leftinv(\refl{x})\defeq \refl{\refl{x}}$. Similarly it follows from the computation rules that
\begin{equation*}
\ct{\refl{x}}{\refl{x}^{-1}} \jdeq \refl{x}^{-1}\jdeq \refl{x}
\end{equation*}
so we again define $\rightinv(\refl{x})\defeq\refl{\refl{x}}$. 
\end{constr}

\begin{rmk}
  We have seen that the associator, the unit laws, and the inverse laws, are all proven by constructing an identification of identifications. And indeed, there is nothing that would stop us from considering identifications of those identifications of identifications. We can go up as far as we like in the \emph{tower of identity types}\index{tower of identity types}\index{identity type!tower of identity types}, which is obtained by iteratively taking identity types.

  The iterated identity types give types in homotopy type theory a very intricate structure. One important way of studying this structure is via the homotopy groups of types, a subject that we will gradually be working towards.
\end{rmk}
\index{groupoid laws!of identifications|)}

\subsection{The action on paths of functions}

\index{action on paths|(}
\index{identity type!action on paths|(}
Using the induction principle of the identity type we can show that every function preserves identifications.
In other words, every function sends identified terms to identified terms.
Note that this is a form of continuity for functions in type theory: if there is a path that identifies two points $x$ and $y$ of a type $A$, then there also is a path that identifies the values $f(x)$ and $f(y)$ in the codomain of $f$. 

\begin{defn}\label{defn:ap}
Let $f:A\to B$ be a map. We define the \define{action on paths}\index{function!action on paths} of $f$ as an operation\index{ap f@{$\apfunc{f}$}|see {action on paths}}
\begin{equation*}
\apfunc{f} : \prd{x,y:A} (\id{x}{y})\to(\id{f(x)}{f(y)}).
\end{equation*}
Moreover, there are operations\index{ap-id@{$\apid$}}\index{action on paths!ap-id@{$\apid$}}\index{ap-comp@{$\apcomp$}}\index{action on paths!ap-comp@{$\apcomp$}}
\begin{align*}
\apid_A & : \prd{x,y:A}\prd{p:\id{x}{y}} \id{p}{\ap{\idfunc[A]}{p}} \\
\apcomp(f,g) & : \prd{x,y:A}\prd{p:\id{x}{y}} \id{\ap{g}{\ap{f}{p}}}{\ap{g\circ f}{p}}.
\end{align*}
\end{defn}

\begin{constr}
First we define $\apfunc{f}$ by identity elimination, taking
\begin{equation*}
\apfunc{f}(\refl{x})\defeq \refl{f(x)}.
\end{equation*}
Next, we construct $\apid_A$ by identity elimination, taking
\begin{equation*}
\apid_A(\refl{x}) \defeq \refl{\refl{x}}.
\end{equation*}
Finally, we construct $\apcomp(f,g)$ by identity elimination, taking
\begin{equation*}
\apcomp(f,g,\refl{x}) \defeq \refl{g(f(x))}.\qedhere
\end{equation*}
\end{constr}

\begin{defn}\label{defn:ap-preserve}
Let $f:A\to B$ be a map. Then there are identifications\index{ap-refl@{$\aprefl$}}\index{ap-inv@{$\apinv$}}\index{ap-concat@{$\apconcat$}}\index{action on paths!ap-refl@{$\aprefl$}}\index{action on paths!ap-inv@{$\apinv$}}\index{action on paths!ap-concat@{$\apconcat$}}
\begin{align*}
\aprefl(f,x) & : \id{\ap{f}{\refl{x}}}{\refl{f}(x)} \\
\apinv(f,p) & : \id{\ap{f}{p^{-1}}}{\ap{f}{p}^{-1}} \\
\apconcat(f,p,q) & : \id{\ap{f}{\ct{p}{q}}}{\ct{\ap{f}{p}}{\ap{f}{q}}}
\end{align*}
for every $p:\id{x}{y}$ and $q:\id{x}{y}$.
\end{defn}

\begin{constr}
To construct $\aprefl(f,x)$ we simply observe that ${\ap{f}{\refl{x}}}\jdeq {\refl{f}(x)}$, so we take
\begin{equation*}
\aprefl(f,x)\defeq\refl{\refl{f(x)}}.
\end{equation*}
We construct $\apinv(f,p)$ by identification elimination on $p$, taking
\begin{equation*}
\apinv(f,\refl{x}) \defeq \refl{\ap{f}{\refl{x}}}.
\end{equation*}
Finally we construct $\apconcat(f,p,q)$ by identification elimination on $p$, taking
\begin{equation*}
\apconcat(f,\refl{x},q)  \defeq \refl{\ap{f}{q}}.\qedhere
\end{equation*}
\end{constr}
\index{action on paths|)}
\index{identity type!action on paths|)}

\subsection{Transport}

Dependent types also come with an action on paths: the \emph{transport} functions.
Given an identification $p:\id{x}{y}$ in the base type $A$, we can transport any term $b:B(x)$ to the fiber $B(y)$.
The transport functions have many applications, which we will encounter throughout this course.

\begin{defn}
Let $A$ be a type, and let $B$ be a type family over $A$.
We will construct a \define{transport}\index{transport}\index{family!transport}\index{identity type!transport} operation\index{tr B@{$\tr_B$}}
\begin{equation*}
\tr_B:\prd{x,y:A} (\id{x}{y})\to (B(x)\to B(y)).
\end{equation*}
\end{defn}

\begin{constr}
We construct $\tr_B(p)$ by induction on $p:x=_A y$, taking
\begin{equation*}
\tr_B(\refl{x}) \defeq \idfunc[B(x)].\qedhere
\end{equation*}
\end{constr}

Thus we see that type theory cannot distinguish between identified terms $x$ and $y$, because for any type family $B$ over $A$ one gets a term of $B(y)$ as soon as $B(x)$ has a term.

As an application of the transport function we construct the \emph{dependent} action on paths\index{dependent action on paths}\index{dependent function!dependent action on paths} of a dependent function $f:\prd{x:A}B(x)$. Note that for such a dependent function $f$, and an identification $p:\id[A]{x}{y}$, it does not make sense to directly compare $f(x)$ and $f(y)$, since the type of $f(x)$ is $B(x)$ whereas the type of $f(y)$ is $B(y)$, which might not be exactly the same type. However, we can first \emph{transport} $f(x)$ along $p$, so that we obtain the term $\tr_B(p,f(x))$ which is of type $B(y)$. Now we can ask whether it is the case that $\tr_B(p,f(x))=f(y)$. The dependent action on paths of $f$ establishes this identification.

\begin{defn}\label{defn:apd}
Given a dependent function $f:\prd{a:A}B(a)$ and a path $p:\id{x}{y}$ in $A$, we construct a path\index{apd f@{$\apdfunc{f}$}}
\begin{equation*}
\apd{f}{p} : \id{\tr_B(p,f(x))}{f(y)}.
\end{equation*}
\end{defn}

\begin{constr}
The path $\apd{f}{p}$ is constructed by path induction on $p$. Thus, it suffices to construct a path
\begin{equation*}
\apd{f}{\refl{x}}:\id{\tr_B(\refl{x},f(x))}{f(x)}.
\end{equation*}
Since transporting along $\refl{x}$ is the identity function on $B(x)$, we simply take $\apd{f}{\refl{x}}\defeq\refl{f(x)}$. 
\end{constr}

\begin{exercises}
\exercise
  \begin{subexenum}
  \item State Goldbach's Conjecture\index{Goldbach's Conjecture} in type theory.
  \item State the Twin Prime Conjecture\index{Twin Prime Conjecture} in type theory.
  \end{subexenum}
\exercise \label{ex:inv_assoc}Show that the operation inverting paths distributes over the concatenation operation, i.e., construct an identification
  \index{distributivity!of inv over concat@{of $\invfunc$ over $\concat$}}
  \index{identity type!distributive-inv-concat@{$\distributiveinvconcat$}}
  \begin{align*}
    \distributiveinvconcat(p,q):\id{(\ct{p}{q})^{-1}}{\ct{q^{-1}}{p^{-1}}}.
  \end{align*}
  for any $p:\id{x}{y}$ and $q:\id{y}{z}$.
\exercise \label{ex:inv_con}For any $p:x=y$, $q:y=z$, and $r:x=z$, construct maps
  \index{identity type!inv-con@{$\invcon$}}
  \index{inv-con@{$\invcon$}}
  \index{identity type!con-inv@{$\coninv$}}
  \index{con-inv@{$\coninv$}}
  \begin{align*}
    \invcon(p,q,r) & : (\ct{p}{q}=r)\to (q=\ct{p^{-1}}{r}) \\
    \coninv(p,q,r) & : (\ct{p}{q}=r)\to (p=\ct{r}{q^{-1}}).
  \end{align*}
\exercise Let $B$ be a type family over $A$, and consider a path $p:\id{x}{x'}$ in $A$. Construct for any $y:B(x)$ a path\index{lift@{$\lift$}}\index{identity type!lift@{$\lift$}}
  \begin{equation*}
    \lift_B(p,y) : \id{(x,y)}{(x',\tr_B(p,y))}.
  \end{equation*}
  In other words, a path in the \emph{base type} $A$ \emph{lifts} to a path in the total space $\sm{x:A}B(x)$ for every term over the domain, analogous to the path lifting property for fibrations in homotopy theory.
\exercise \label{ex:semi-ring-laws-N}In this exercise we show that the operations of addition and multiplication on the natural numbers satisfy the laws of a commutative \define{semi-ring}.%
  \index{semi-ring laws!for N@{for $\N$}}%
  \index{natural numbers!semi-ring laws}%
  \index{associativity!of addition on N@{of addition on $\N$}}%
  \index{unit laws!for addition on N@{for addition on $\N$}}%
  \index{commutativity!of addition on N@{of addition on $\N$}}%
  \index{associativity!of multiplication on N@{of multiplication on $\N$}}%
  \index{unit laws!for multiplication on N@{for multiplication on $\N$}}%
  \index{commutativity!of multiplication on N@{of multiplication on $\N$}}%
  \index{distributivity!of mulN over addN@{of $\mulN$ over $\addN$}}%
  \begin{subexenum}
  \item Show that addition satisfies the following laws:
    \begin{align*}
      m+0 & = m & m+\succN(n) & = \succN(m+n) \\
      0+m & = m & \succN(m)+n & = \succN(m+n).
    \end{align*}
  \item Show that addition is associative and commutative, i.e., show that we have identifications
    \begin{align*}
      (m+n)+k & = m+(n+k) \\
      m+n & = n+m.
    \end{align*}
  \item Show that multiplication satisfies the following laws:
    \begin{align*}
      m\cdot 0 & = 0 & m\cdot 1 & = m & m\cdot \succN(n) & = m+m\cdot n \\
      0\cdot m & = 0 & 1\cdot m & = m & \succN(m)\cdot n & = m\cdot n+n.
    \end{align*}
  \item Show that multiplication on $\N$ is commutative:
    \begin{equation*}
      m\cdot n=n\cdot m.
    \end{equation*}
  \item Show that multiplication on $\N$ distributes over addition from the left and from the right, i.e., show that we have identifications
    \begin{align*}
      m\cdot (n+k) & = m\cdot n + m\cdot k \\
      (m+n)\cdot k & = m\cdot k + n\cdot k.
    \end{align*}
  \item Show that multiplication on $\N$ is associative:
    \begin{align*}
      (m\cdot n)\cdot k & = m\cdot (n\cdot k).
    \end{align*}
  \end{subexenum}
\exercise Consider four consecutive identifications
  \begin{equation*}
    \begin{tikzcd}
      a \arrow[r,equals,"p"] & b \arrow[r,equals,"q"] & c \arrow[r,equals,"r"] & d \arrow[r,equals,"s"] & e
    \end{tikzcd}
  \end{equation*}
  in a type $A$. In this exercise we will show that the \define{Mac Lane pentagon}\index{Mac Lane pentagon}\index{identity type!Mac Lane pentagon} for identifications commutes.
  \begin{subexenum}
  \item Construct the five identifications $\alpha_1,\ldots,\alpha_5$ in the pentagon
    \begin{equation*}
      \begin{tikzcd}[column sep=-1.5em]
        &[-2em] \ct{(\ct{(\ct{p}{q})}{r})}{s} \arrow[rr,equals,"\alpha_4"] \arrow[dl,equals,swap,"\alpha_1"] & & \ct{(\ct{p}{q})}{(\ct{r}{s})} \arrow[dr,equals,"\alpha_5"] &[-2em] \\
        \ct{(\ct{p}{(\ct{q}{r})})}{s} \arrow[drr,equals,swap,"\alpha_2"] & & & & \ct{p}{(\ct{q}{(\ct{r}{s})})}, \\
        & & \ct{p}{(\ct{(\ct{q}{r})}{s})} \arrow[urr,equals,swap,"\alpha_3"]
      \end{tikzcd}
    \end{equation*}
    where $\alpha_1$, $\alpha_2$, and $\alpha_3$ run counter-clockwise, and $\alpha_4$ and $\alpha_5$ run clockwise.
  \item Show that
    \begin{equation*}
      \ct{(\ct{\alpha_1}{\alpha_2})}{\alpha_3} = \ct{\alpha_4}{\alpha_5}.
    \end{equation*}
  \end{subexenum}
\end{exercises}

%\item In this exercise we show that the action on paths of a function preserves the groupoid-structure of a type.
%\begin{subexenum}
%\item Construct an identification
%\begin{equation*}
%\mathsf{ap.assoc}(f,p,q,r)
%\end{equation*}
%witnessing that the diagram
%\begin{equation*}
%\begin{tikzcd}[column sep=large]
%\ap{f}{\ct{(\ct{p}{q})}{r}} \arrow[r,equals,"\ap{\apfunc{f}}{\assoc(p,q,r)}"] \arrow[d,swap,equals,"{\mathsf{ap.ct}(f,%\ct{p}{q},r)}"] & \ap{f}{\ct{p}{(\ct{q}{r})}} \arrow[d,equals,"{\mathsf{ap.ct}(f,p,\ct{q}{r})}"] \\ 
%\ct{\ap{f}{\ct{p}{q}}}{\ap{f}{r}} \arrow[dd,equals,near start,"{\mathsf{whisk\usc{}r}(\mathsf{ap.ct}(f,p,q),\ap{f}{r})}"]   & %\ct{\ap{f}{p}}{\ap{f}{\ct{q}{r}}} \arrow[dd,equals,swap,near end,"{\mathsf{whisk\usc{}l}(\ap{f}{p},\mathsf{ap.ct}(f,q,r))}"]  %\\
%\\
%\ct{(\ct{\ap{f}{p}}{\ap{f}{q}})}{\ap{f}{r}} \arrow[r,equals,swap,"{\assoc(\ap{f}{p},\ap{f}{q},\ap{f}{r})}"yshift=-1em] & \ct{\ap{f}{p}}{(\ct{\ap{f}{q}}{\ap{f}{r}})}
%\end{tikzcd}
%\end{equation*}
%commutes.
%\end{subexenum}

\index{identity type|)}
\index{inductive type!identity type|)}
\index{inductive type|)}

\section{Universes in model categories}
\label{sec:univalence}

Now let \E be a model category and \Fib the full \nfs determined by its fibrations, so that $\Fibka = \Fib\times_\cE \cEka$ denotes the relatively \ka-presentable fibrations.
In an ideal world, the object $\Fibka\in\Ehat$ would be (pseudonaturally equivalent to) a representable presheaf $\E(-,U)$.
But since \E is itself a 1-category rather than an \io-category, this is unreasonable to expect. % (unless all fibrations are monomorphisms, as in the case of the subobject classifier in a 1-topos).

Instead, we will replace \Fibka by a representable presheaf that is ``weakly equivalent'' in some sense.
We do not have a model structure on $\Ehat$ with which to make sense of this, but we can at least use the Yoneda embedding $\E\to\Ehat$ to lift the weak factorization systems of \E.

\begin{defn}\label{defn:afib}
  Let \E be a model category.
  A morphism $\dX\to\dY$ in $\Ehat$ is an \textbf{acyclic fibration} if it has the right lifting property (\cref{defn:2liftorth}) for all morphisms $\E(-,j) : \E(-,A) \to \E(-,B)$, where $j:A\to B$ is a cofibration in \E.
\end{defn}

\begin{rmk}
  If $f:\dX\to\dY$ is a representable morphism, then it is an acyclic fibration in the sense of \cref{defn:afib} if and only if in any pullback
  \begin{equation*}
    \begin{tikzcd}
      \E(-,W) \ar[r] \ar[d] \ar[dr,phantom,near start,"\lrcorner"] & \dX\ar[d,"{f}"] \\
      \E(-,Z) \ar[r] & \dY
    \end{tikzcd}
  \end{equation*}
  the induced map $W\to Z$ is an acyclic fibration in \E.
  This fits a standard pattern for extending pullback-stable properties of morphisms in \E to properties of representable morphisms in \Ehat.
  
  Note that this notion of (acyclic) fibration based on the model structure of \E is unrelated to the 2-categorical notions of (strict, discrete) fibration defined in \cref{sec:2cat}.
\end{rmk}

\begin{defn}
  If \F is a \nfs on \E, a \textbf{universe} for \F is a cofibrant object $U\in\E$ equipped with an acyclic fibration $\E(-,U) \to \F$ in \Ehat.
\end{defn}

That is, a universe is a sort of ``cofibrant replacement'' of $\Fibka$.
(This perspective was introduced informally in~\cite[\sect 3]{shulman:elreedy}.)

\begin{rmk}\label{rmk:universe}
  When $U$ is a universe, the morphism $\E(-,U) \to \F$ corresponds by the Yoneda lemma to an \F-algebra $\pi:\Util\to U$.
  And the fact that the morphism $\E(-,U) \to \F$ is an acyclic fibration means that given the solid arrows below, where $i:A\mono B$ is a cofibration, $f:X\to B$ is an \F-algebra, and both squares of solid arrows are pullbacks and \F-morphisms:
  \begin{equation*}
    \begin{tikzcd}[row sep=small, column sep=small]
      i^*(X) \arrow[rr] \arrow[dd, "g"'] \arrow[rd] &  & \Util \arrow[dd, "\pi"] \\
      & X \arrow[ru, dashed] &  \\
      A \arrow[rd, "i"', tail] \arrow[rr, near start, "h" description] &  & U \\
      & B \arrow[ru, dashed] \arrow[from=uu, near start, "f" description, crossing over] & 
    \end{tikzcd}%\label{eq:u2p}
  \end{equation*}
  there exist the dashed arrows rendering the diagram commutative and the third square also a pullback and an \F-morphism.
  This property of a universe was first noted in the proof of~\cite[Theorem 2.2.1]{klv:ssetmodel} and isolated more abstractly in~\cite[(2$'$)]{shulman:elreedy}, \cite[Corollary 3.11]{cisinski:elegant}, and~\cite{stenzel:thesis} under varying names.

  If the initial object $\emptyset$ is strict (i.e.\ every morphism with codomain $\emptyset$ is an isomorphism) and $\id_\emptyset$ has a unique \F-structure, this implies that every \F-algebra with cofibrant codomain is a pullback of $\pi$ (though not in a unique way).
\end{rmk}

As usual, when \E is cofibrantly generated we can hope to produce such a cofibrant replacement by the small object argument.
However, the colimits in \E used to build cell complexes are no longer colimits in \Ehat; thus we have to restrict to the objects of \Ehat that preserve these particular colimits.

\begin{defn}
  Let \E be a model category.
  We say $\dX\in\Ehat$ is a \textbf{stack for cell complexes} if as a pseudofunctor $\dX:\E\op\to\cGPD$ it preserves (in the weak bicategorical sense) coproducts, pushouts of cofibrations, and transfinite composites of cofibrations.
\end{defn}

\begin{eg}\label{eg:topos-stack}
  If \E is a Grothendieck 1-topos and all cofibrations are monomorphisms, then the trivial \nfs \cE is a stack for cell complexes.
  This because any topos is infinitary extensive~\cite{clw:ext-dist}, adhesive~\cite{ls:adhesive,ls:topadh}, and exhaustive~\cite{nlab:exhaustive,shulman:elreedy}; see~\cite[\sect3]{shulman:elreedy} and~\cite[Lemma 7.5]{sattler:eqvext}.
  More generally,
  \cE being a stack for cell complexes is one of the conditions for the (cofibration, acyclic fibration) weak factorization system of \E to be \emph{suitable} as in~\cite[Definition 3.2]{sattler:eqvext}.
\end{eg}

\begin{lem}\label{thm:nfs-stack}
  If \cE is a stack for cell complexes and $\phi:\F\to\cE$ is a \local \nfs, then \F is also a stack for cell complexes.
\end{lem}
\begin{proof}
  Let \sQ be the class of morphisms $q:\Yhat \to \E(-,Y)$ from \cref{eg:colim-orth} where $Y= \colim_i Y_i$ ranges over coproducts, pushouts of cofibrations, and transfinite composites of cofibrations.
  Since \cE is a stack for cell complexes, $\sQ \perp \cE$; and since \F is \local, by \cref{thm:local} we have $\sQ\perp\phi$.
  Hence $\sQ\perp \F$.
\end{proof}

\begin{lem}\label{thm:icell-acyc}
  If \E is cofibrantly generated with \cI a set of generating cofibrations, \dX and \dY are stacks for cell complexes, and $f:\dX\to\dY$ has the right lifting property in \Ehat against all morphisms $\E(-,j):\E(-,A)\to\E(-,B)$ where $j:A\to B$ is in \cI, then $f$ is an acyclic fibration.
\end{lem}
\begin{proof}
  As usual, any cofibration is a retract of a transfinite composite of pushouts of coproducts of elements of \cI.
  Thus, given that \dX and \dY are stacks for cell complexes, the lifting property carries through all these operations in the usual way.
\end{proof}

We call a pseudofunctor $\dZ\in\Ehat$ \textbf{small-groupoid-valued} if each groupoid $\dZ(A)$ is essentially small.
Note that by definition, for any \nfs \F the map $\phi:\F\to\cE$ has small \emph{fibers}, i.e.\ any given morphism $f:X\to Y$ has a small set of \F-structures; but \F is only small-groupoid-valued if any given object $Y\in\E$ there is a small set of isomorphism classes of \F-algebras with codomain $Y$.
In general we will achieve this by considering $\Fka = \dF\times_\cE \cEka$ as in \cref{sec:relpres}.

\begin{thm}\label{thm:2cat-soa}
  Let \E be a combinatorial model category, and $\dZ\in\Ehat$ a small-groupoid-valued stack for cell complexes.
  Then any morphism $f:\E(-,X) \to\dZ$ in \Ehat factors, up to isomorphism, as $\E(-,X) \xto{\E(-,j)} \E(-,Y) \xto{p} \dZ$, where $j$ is a cofibration in \E and $p$ is an acyclic fibration in \Ehat.
\end{thm}
\begin{proof}
  This is just a bicategorical adaptation of the small object argument.
  Let \cI be a set of generating cofibrations for \E; we will define an \cI-cell complex sequence $X_0 \to X_1 \to \cdots$ in \E, along with maps $f_n : \E(-,X_n) \to \dZ$ and coherent isomorphisms $f_n \circ j_{m,n} \cong f_m$.
  We start with $X_0 = X$ and $f_0 = f$.
  For limit $n$ we let $X_n = \colim_{m<n} X_m$, with $f_n : \E(-,X_n) \to \dZ$ and attendant isomorphisms induced by the fact that \dZ preserves this colimit.

  At a successor stage $n+1$, we let $S_n$ be a set of representatives for isomorphism classes of pseudo-commutative squares
  \[
    \begin{tikzcd}
      \E(-,A) \ar[r] \ar[d,"i"'] \ar[dr,phantom,"\scriptstyle\Downarrow\cong"] & \E(-,X_n) \ar[d] \\
      \E(-,B) \ar[r] & \dZ
    \end{tikzcd}
  \]
  where $i\in \cI$.
  This is a small set, since \dZ is small-groupoid-valued and \E is locally small.
  Now let $X_{n+1}$ be the pushout
  \[
    \begin{tikzcd}
      \coprod_{s\in S_n} A_s \ar[r] \ar[d] \drpushout & X_n \ar[d]\\
      \coprod_{s\in S_n} B_s \ar[r] & X_{n+1}
    \end{tikzcd}
  \]
  Since \dX preserves these coproducts and pushouts, there is an essentially unique induced map $f_{n+1} : X_{n+1}\to Z$ with attendant isomorphisms.

  Finally, since \E is locally presentable, there is a regular cardinal \la such that all domains of morphisms in \cI are \la-presentable.
  Thus, in any square
  \[
    \begin{tikzcd}
      \E(-,A) \ar[r] \ar[d,"i"'] \ar[dr,phantom,"\scriptstyle\Downarrow\cong"] & \E(-,X_\la) \ar[d] \\
      \E(-,B) \ar[r] & \dZ
    \end{tikzcd}
  \]
  the top morphism $A\to X_\la$ factors through $X_n$ for some $n<\la$, and hence there is a lift $B\to X_{n+1} \to X_\la$.
  Therefore, the map $f_{\la} : X_\la \to \dZ$ has right lifting for \cI, and is thus an acyclic fibration by \cref{thm:icell-acyc}.
\end{proof}

\begin{cor}\label{thm:nfs-universe}
  If \E is a Grothendieck 1-topos with a combinatorial model structure in which all cofibrations are monomorphisms, then any small-groupoid-valued \local \nfs \F on \E has a universe.
\end{cor}
\begin{proof}
  By \cref{eg:topos-stack,thm:nfs-stack}, \F is a stack for cell complexes; thus we can apply \cref{thm:2cat-soa} to factor the map $\E(-,\emptyset) \to \F$.
\end{proof}

In some cases such as \cref{eg:pshf-can,eg:rep-cod}, \Fib is \local and hence so is \Fibka.
This includes the universes constructed in~\cite{klv:ssetmodel,shulman:elreedy,cisinski:elegant}.
However, in the general case we need a different approach: we will suppose given a non-full \nfs \F that \emph{is} \local, and an acyclic fibration $\F\to\Fib$. Thus we will be able to apply \cref{thm:nfs-universe} to \F instead.

More generally, for a \nfs \F, let $\uly\F$ denote the image of the map $\phi:\F\to \cE$.
Thus $\uly\F$ is a full \nfs (though not generally \local, even if \F is), and the $\uly\F$-algebras are the morphisms that admit some \F-structure.

\begin{defn}\label{defn:stratified}
  A \nfs \F on a model category \E is \textbf{\stratified} if the map $\F \to \uly\F$ is an acyclic fibration.
  That is, for any pullback
  \[
    \begin{tikzcd}
      X' \ar[d,"f'"'] \ar[r,"g"] \ar[dr,phantom, near start,"\lrcorner"] & X \ar[d,"f"]\\
      Y' \ar[r,"i",tail] & Y
    \end{tikzcd}
  \]
  with $f$ and $f'$ \F-algebras and $i$ a cofibration, there exists a new \F-structure on $f$ making the square an \F-morphism.
\end{defn}

\begin{prop}\label{thm:pre-u2p}
  Let \F be a \local, \stratified, small-groupoid-valued \nfs on a Grothendieck 1-topos that is a combinatorial model category whose cofibrations are monomorphisms.
  Then $\uly\F$ has a universe.
\end{prop}
\begin{proof}
  Apply \cref{thm:nfs-universe} to \F, and observe that the composite $\E(-,U) \to \F \to \uly\F$ of acyclic fibrations is again an acyclic fibration.
\end{proof}

\begin{rmk}
  By \cref{rmk:universe}, if all objects are cofibrant, $\emptyset$ is strict, and $U$ is a universe for a full \nfs \F such that $\id_\emptyset\in\F$, then in fact $\F = \uly{\dRep_\pi}$ (with $\dRep_\pi$ as in \cref{eg:rep-fcos}).
  Conversely, if \E is locally cartesian closed and $\pi:\Util\to U$ is a universe for $\uly{\dRep_\pi}$, then $\dRep_\pi$ is \local (by \cref{eg:rep-local}), \stratified, and small-groupoid-valued.
  Thus the hypotheses of \cref{thm:pre-u2p} are basically optimal.
\end{rmk}

\begin{eg}
  Full \nfss are always \stratified, as is $\F_1 \times_\cE \F_2$ if $\F_1$ and $\F_2$ are.
\end{eg}

\begin{eg}
  If \F is a \stratified \nfs on $\E_2$ and $G:\E_1\to\E_2$ preserves pullbacks (hence also monomorphisms), then $G^{-1}(\F)$ is also \stratified.
\end{eg}

\begin{eg}\label{thm:sec-afib-strat}
  Let \E be a model category and $H$ be a fibred core-endofunctor of \E such that whenever $H_Y(X)\to Y$ has a section, it is an acyclic fibration.
  Then the \nfs $H^*(\cEp)$ is \stratified.
  For given a pullback square
  \begin{equation}
    \begin{tikzcd}
      X' \ar[d,"f'"'] \ar[r,"g"] \ar[dr,phantom, near start,"\lrcorner"] & X \ar[d,"f"]\\
      Y' \ar[r,"i",tail] & Y
    \end{tikzcd}\label{eq:sec-strat-sq}
  \end{equation}
  with $i$ a cofibration, along with sections $s'$ and $s$ of $H_{Y'}(X')\to Y'$ and $H_Y(X)\to Y$, the assumption implies $H_Y(X)\to Y$ is an acyclic fibration.
  Thus we can find a lift in the square
  \[
    \begin{tikzcd}
      Y' \ar[d,"i"',tail] \ar[r,"s"] & H_{Y'}(X') \ar[r,"{H(g,i)}"] & H_Y(X) \ar[d,"\sim",two heads]\\
      Y \ar[rr,equals] & & Y
    \end{tikzcd}
  \]
  giving an $H^*(\cEp)$-structure on $f$ making~\eqref{eq:sec-strat-sq} an $H^*(\cEp)$-morphism.
\end{eg}

Recall that a \textbf{Cisinski model category}~\cite{cisinski:topos,cisinski:local-acyc} is a Grothendieck 1-topos with a combinatorial model structure whose cofibrations are \emph{precisely} the monomorphisms.

\begin{prop}\label{eg:cof-ff-fcos}
  Let \E be a Cisinski model category, \F a \stratified \nfs on \E, and $E$ a cartesian functorial factorization on \E that factors every \F-algebra as an acyclic cofibration followed by a fibration.
  Then the \nfs $\F\times_\cE \dR_E$ (where $\dR_E$ is as in \cref{eg:ff-fcos}) is also \stratified.
\end{prop}
\begin{proof}
  Suppose given the pullback square on the left:
  \[
    \begin{tikzcd}
      X' \ar[d,"f'"'] \ar[r,"g"] \ar[dr,phantom, near start,"\lrcorner"] & X \ar[d,"f"]\\
      Y' \ar[r,"i",tail] & Y\\
      \phantom{E f}
    \end{tikzcd}
    \hspace{2cm}
    \begin{tikzcd}
      X' \ar[r,"g"] \ar[d,tail,"{\lambda_{f'}}"'] \ar[dr,phantom,near end,"\ulcorner"] & X \ar[ddr,tail,"{\lambda_f}"] \ar[d] \\
      E f' \ar[r] \ar[drr,tail,"{\fact g i}"'] & P \ar[dr,"j" description]\\
      && E f.
    \end{tikzcd}
  \]
  where $f$ and $f'$ are \F-algebras with $\dR_E$-structures $r_{f} : E f \to X$ and $r_{f'} : E f' \to X'$.
  Since \F is \stratified, $f$ has a new \F-structure making $(g,i)$ an \F-morphism; so it remains to find a new $\dR_E$-structure $\rtil_{f} : E f \to X$ (so that $f \circ \rtil_f = \rho_f$ and $\rtil_f \circ \lambda_f = \id_X$) such that $(g,i)$ is also an $\dR_E$-morphism, i.e.\ $\rtil_f \circ \fact g i = g \circ r_{f'}$.

  Define $P$ and $j$ by the pushout as on the right above.
  Since $f$ and $f'$ are \F-algebras, $\lambda_{f}$ and $\lambda_{f'}$ are acyclic cofibrations, and in particular monomorphisms.
  By cartesianness, $\fact g i$ is also a monomorphism (being a pullback of $i$) and $X'\cong E f' \times_{E f}X$.
  Thus, $P$ is a union of subobjects of $E f$ in the 1-topos \E, hence $j:P\to E f$ is also a monomorphism.
  Moreover, since $X\to P$ is a pushout of $\lambda_{f'}$, it is also an acyclic cofibration; hence by 2-out-of-3 $j$ is also acyclic.

  Now since $f$ is an \F-algebra, $\rho_f$ is a fibration.
  But $f$ is a retract of $\rho_f$ (by its $\dR_E$-structure $r_f$), hence also a fibration.
  Thus we can find a lift in the square:
  \[
    \begin{tikzcd}
      P \ar[d,"j"'] \ar[r] & X \ar[d,"f"] \\
      E f \ar[r,"{\rho_f}"'] \ar[ur,dotted,"{\rtil_f}" description] & Y
    \end{tikzcd}
  \]
  where the top arrow is induced by $g \circ r_{f'}$ and $\id_X$.
  Such a lift is then an $\dR_E$-structure on $f$ such that $(g,i)$ is an $\dR_E$-morphism, as desired.
\end{proof}

It remains to ensure that our universes are fibrant and univalent.

\begin{defn}\label{defn:nfs-hoinvar}
  Let \E be a locally presentable category with a model structure.
  A \nfs \F is \textbf{homotopy invariant} if every \F-algebra is a fibration, and given any commutative square
  \[
    \begin{tikzcd}
      X' \ar[d,"f'"',two heads] \ar[r,"\sim"] & X \ar[d,"f",two heads]\\
      Y' \ar[r,"\sim"'] & Y
    \end{tikzcd}
  \]
  where $f$ and $f'$ are fibrations and the horizontal maps are weak equivalences, $f$ admits an \F-structure if and only if $f'$ does.
\end{defn}

Of course, if $\uly\F=\Fib$ is the class of all fibrations, then \F is homotopy invariant.
More generally, homotopy invariance is a condition only on $\uly\F$.

We now recall the fundamental ``equivalence extension'' property.
To my knowledge, a version of this property first appeared in~\cite[Theorem 3.4.1]{klv:ssetmodel} in the case of simplicial sets.
It was observed in~\cite[Theorem 3.1]{shulman:elreedy} and~\cite[Remark 2.19]{cisinski:elegant} that the proof generalizes to any simplicial Cisinski model category.
A similar construction for cubical sets appeared under the name ``gluing'' in~\cite{cchm:cubicaltt}, which was then placed in a more abstract setting by~\cite{sattler:eqvext}.

\begin{thm}\label{thm:u4p}
  Let $\E$ be a simplicial Cisinski model category and \F a homotopy invariant \nfs on \E.
  Then there is a \la such that for any $\ka\shgt\la$ and any cofibration $i: A\cof B$, relatively \ka-presentable $\uly\F$-algebras $D_2 \fib B$ and $E_1 \fib A$, and weak equivalence $w: E_1 \toiso E_2$ over $A$, where $E_2 \coloneqq i^* D_2$:
%  Suppose given also a map $g:C\to D_2$ over $B$ and a factorization of $i^*(g) : i^* C \to E_2$ as $w\circ k$ for some $k:i^*(C) \to E_1$, filling out the solid arrows below:
  \begin{equation}
  \begin{tikzcd}[column sep=small]
      % & i^*C \ar[dl,"k"'] \ar[rrr] &&& C \ar[dl,dashed,"h"] \ar[ddr,"g"] \\
      E_1 \arrow[rdd, two heads] \arrow[rrd, "w"] \arrow[rrr, dashed] &  &  & D_1 \arrow[rdd, two heads, dashed] \arrow[rrd, "v", dashed] &  &  \\
      &  & \mathllap{E_2=\;}i^* D_2 \arrow[ld, two heads] \arrow[rrr,crossing over]
      % \ar[from=uul, crossing over]
      &  &  & D_2 \arrow[ld, two heads] \\
      & A \arrow[rrr, tail,"i"'] &  &  & B & 
    \end{tikzcd}\label{eq:u4p}
  \end{equation}
  there exists a relatively \ka-presentable $\uly\F$-algebra $D_1\fib B$ and an equivalence $v: D_1 \toiso D_2$ over $B$ such that $i^*(v)=w$.
  % If \F is \stratified, we can choose the \F-structure of $D_1$ so that the back square is an \F-morphism.
  %, and a map $h:C\to D_1$ with $i^*(h) = k$ and $v \circ h = g$.
\end{thm}
\begin{proof}
  Largely identical to that of~\cite[Theorem 3.1]{shulman:elreedy}.
  The latter statement assumes that \E is a presheaf category, but this is only used to obtain a notion of ``\ka-small morphism'' that is preserved by $i_*$ and by pullback; using \cref{thm:pres-pb,thm:relpres-mono-dp} instead allows \E to be any Grothendieck 1-topos.\footnote{The author claimed in~\cite{shulman:elreedy} that when $\E=\prcs$ it suffices to take $\ka > \card\C$, but Raffael Stenzel has pointed out that this is not enough to ensure that $i_*$ preserves \ka-small morphisms; even in the presheaf case we need some analogue of the relation $\shlt$.}
  (The proof uses that \E is a simplicial model category and has effective unions, to extend deformation retractions along $i$.)
  We conclude $D_1$ is an $\uly\F$-algebra by homotopy invariance, since it is equivalent to $D_2$ over $B$.
  % The application of \stratification is immediate.
\end{proof}

Univalence of our universes will follow from \cref{thm:u4p} as in~\cite{klv:ssetmodel,shulman:elreedy,cisinski:elegant}.
To show that $U$ is a fibrant object,~\cite[Theorem 2.2.1]{klv:ssetmodel} and~\cite[Proposition 2.21]{cisinski:elegant} use minimal fibrations, while~\cite[Lemma 6.3]{shulman:elreedy} uses a Reedy induction; but in fact fibrancy of $U$ is almost immediate from \cref{thm:u4p}.
A similar fact in the restricted situation of cubical-type model structures (where an explicit description of the fibrations is available) appears in~\cite{cchm:cubicaltt,sattler:eqvext}, while the general case was observed in~\cite{stenzel:thesis}.

\begin{thm}\label{thm:uf-fibrant}
  Let $\E$ be a right proper simplicial Cisinski model category, and \F a \local, \stratified, and homotopy invariant \nfs on \E.
  Then there is a regular cardinal \la such that for any regular cardinal $\ka\shgt\la$, there exists a morphism $\pi:\Util \to U$ such that:
  \begin{enumerate}
  \item The \ka-presentable objects in \E are closed under finite limits.\label{item:uf0}
  \item $\pi:\Util \to U$ is a relatively \ka-presentable $\uly\F$-algebra (in particular, a fibration).\label{item:uf1}
  \item Every relatively \ka-presentable $\uly\F$-algebra is a pullback of $\pi$.\label{item:uf2}
  \item The object $U$ is fibrant.\label{item:uf4}
  \item $\pi$ satisfies the univalence axiom.\label{item:uf3}
  \end{enumerate}
\end{thm}
\begin{proof}
  Let $\la_0$ satisfy \cref{thm:u4p}, let $\la_1$ be such that \E has a generating set of acyclic cofibrations with $\la_1$-presentable domains and codomains, let $\la_2$ be such that \E has functorial factorizations that preserve \ka-presentable objects for any $\ka\shgt\la_2$ (such exists since these factorizations are accessible functors), and let $\la_3$ be such that for any $\ka\shgt\la_3$ the \ka-presentable objects are closed under finite limits (which exists by \cref{thm:pres-pb}).
  % \footnote{It is claimed in~\cite[Propositions 7.2]{dug:pres} that a combinatorial model category satisfies this for all sufficiently large \ka, but the proof only shows it for a \shrp class.}
  Let $\la$ be such that $\la\shgt \la_j$ for $j=0,1,2,3$, and assume $\ka\shgt \la$; then $\ka\shgt\la_j$ for all $j$ as well, and in particular~\ref{item:uf0} holds.

  Since \F and $\cEka$ are \local and \stratified, so is $\Fka = {\F\times_\cE \cEka}$.
  Let $\pi:\Util\to U$ be the universe for $\uly\Fka$ obtained from \cref{thm:pre-u2p}; then~\ref{item:uf1} holds trivially.
  And since \cE and \F are stacks for cell complexes, in particular they preserve the initial object; so by \cref{rmk:universe} we have~\ref{item:uf2}.

  Since $\ka\gt\la_1$, to show that $U$ is fibrant~\ref{item:uf4} it suffices to show that it has right lifting for all acyclic cofibrations between \ka-presentable objects.
  Let $i:A\acof B$ be an acyclic cofibration with $A$ and $B$ \ka-presentable, let $h:A\to U$ be a map, and let $E_1 \fib A$ be the pullback of $\pi$ along $h$.
  Since $\pi$ is relatively \ka-presentable, $E_1$ is \ka-presentable.
  Thus since $\ka\shgt\la_2$, we can factor the composite $E_1\fib A \xto{i} B$ as an acyclic cofibration $E_1 \acof D_2$ followed by a fibration $D_2 \fib B$, where $D_2$ is \ka-presentable.
  So since $\ka\shgt\la_3$, $D_2\fib B$ is a relatively \ka-presentable fibration, and by homotopy invariance it is an $\uly\F$-algebra, hence an $\uly\Fka$-algebra.

  Let $E_2\coloneqq i^*(D_2)$; then by right properness the map $E_2 \to D_2$ is a weak equivalence, hence by 2-out-of-3 so is the induced map $E_1 \to E_2$.
  Thus since $\ka\shgt\la_0$, by \cref{thm:u4p} there is an $\uly\Fka$-algebra $D_1 \fib B$ with $i^*(D_1)\cong E_1$.
  Finally, since $U$ is a universe for $\uly\Fka$, by \cref{rmk:universe} there is a map $k:B\to U$ pulling $\pi$ back to $D_1$ such that $k i = h$; so $U$ has right lifting for $i$.

  For univalence~\ref{item:uf3}, we follow~\cite[Theorem 3.4.1]{klv:ssetmodel},~\cite[\sect 2]{shulman:elreedy}, and~\cite[Theorem 3.12]{cisinski:elegant}.
  Let $\Eq(\Util)$ be the universal space of auto-equivalences of $\pi$, as in~\cite[\sect 4]{shulman:elreedy}; it suffices to show that the composite projection
  \(\Eq(\Util) \to U\times U \to U\)
  is an acyclic fibration.
  Now a square
  \[
    \begin{tikzcd}
      A \ar[d,tail,"i"'] \ar[r] & \Eq(\Util) \ar[d]\\
      B \ar[r] & U
    \end{tikzcd}
  \]
  with $i$ a monomorphism yields a diagram of solid arrows~\eqref{eq:u4p} where all fibrations are $\uly\Fka$-algebras.
  Thus, since $\ka\shgt\la_0$ we can fill out the dashed arrows in~\eqref{eq:u4p} with $D_1 \fib B$ also a $\uly\Fka$-algebra; so by \cref{rmk:universe} we can classify it by a map to $U$ extending the given classifying map of $E_1$.
  But this is precisely what we need to specify a lift $B\to \Eq(\Util)$.
\end{proof}

Thus, to build fibrant and univalent universes for relatively \ka-presentable fibrations in a right proper simplicial Cisinski model category, it suffices to find a \local and \stratified \nfs \F such that $\uly\F=\Fib$ is the class of all fibrations.
We have essentially already seen one way to do this: if \E has a set of generating acyclic cofibrations with representable codomains, then $\Fib$ itself has these properties.
This was the approach of~\cite{klv:ssetmodel,shulman:elreedy}; but to deal with the general case we will have to use non-full \nfss.

\begin{rmk}
  Although our primary interest is in constructing univalent universes for \emph{all} (relatively \ka-presentable) fibrations, it is potentially useful that \cref{thm:uf-fibrant} also yields univalent universes for subclasses of fibrations.
  For instance, the \emph{left fibrations} in bisimplicial sets~\cite{vk:yoneda-css,pbb:groth-segal,rasekh:yoneda-ss} are a subclass of the Reedy fibrations, which by~\cite[Remark 2.1.4(a)]{vk:yoneda-css} are characterized by right lifting against a generating set with representable codomains; thus they admit fibrant and univalent universes.
  Such a universe of left fibrations is essentially the ``\oo-category of spaces'' constructed in~\cite{vk:yoneda-css}, although they do not explain how to make it a strict presheaf.
  In fact it is a complete Segal space (this is shown in~\cite[Theorem 2.2.11]{vk:yoneda-css}, and can also be deduced from~\cite[Theorem 4.8]{rasekh:yoneda-ss}), and could be useful for the programme of~\cite{rs:stt} to use that model for ``synthetic \io-category theory''.
  
  We will see another class of examples in \cref{thm:flf-nfs,rmk:modal-univ}.
\end{rmk}


%%% Local Variables:
%%% mode: latex
%%% TeX-master: "univinj"
%%% End:

\section{Modular arithmetic via the Curry-Howard interpretation}\label{sec:modular-arithmetic}

We have now fully described Martin-L\"of's dependent type theory. It is now up to us to start developing some mathematics in it, and Martin-L\"of's dependent type theory is great for elementary mathematics, such as basic number theory, some algebra, and combinatorics. The fundamental idea that is used to develop basic mathematics in type theory is the Curry-Howard interpretation. This is a translation of logic into type theory, which we will use to express concepts of mathematics such as divisibility, the congruence relations, and so on.

We will also introduce the family $\Fin{}$ of the standard finite types, indexed by $\N$, and show how each $\Fin{k+1}$ can be equipped with the group structure of integers modulo $k+1$. Our goal here is to demonstrate how to do those things in type theory, so we will aim for a high degree of accuracy.

\subsection{The Curry-Howard interpretation}\label{sec:Curry-Howard}

The \emph{Curry-Howard interpretation} is an interpretation of logic into type theory. Recall that in type theory there is no separation between the logical framework and the general theory of collections of mathematical objects the way there is in the more traditional setup with Zermelo-Fraenkel set theory, which is postulated by axioms in first order logic. These two aspects of the foundations of mathematics are unified in type theory. The idea of the Curry-Howard interpretation is therefore to express propositions as types, and to think of the elements of those types as their proofs. We illustrate this idea with an example.

\begin{eg}
  A natural number $d$ is said to divide a natural number $n$ if there exists a natural number $k$ such that $d\cdot k=n$. To represent the divisibility predicate in type theory, we need to define a \emph{type}
  \begin{equation*}
    d\mid n,
  \end{equation*}
  of which the elements are witnesses that $d$ divides $n$. In other words, $d\mid n$ should be the type that consists of natural numbers $k$ equipped with an identification $d\cdot k=n$. In general, the type of $x:A$ equipped with $y:B(x)$ is represented as the type $\sm{x:A}B(x)$. The interpretation of the existential quantification ($\exists$) into type theory via the Curry-Howard interpretation is therefore using $\Sigma$-types.
\end{eg}

\begin{defn}
  Consider two natural numbers $d$ and $n$. We say that $d$ \define{divides}\index{divisibility on N@{divisibility on $\N$}|textbf}\index{natural numbers!divisibility|textbf} $n$ if there is a element of type\index{d {"|" n}@{$d\mid n$}|textbf}\index{d {"|" n}@{$d\mid n$}|see{divisibility on $\N$}}
  \begin{equation*}
    d\mid n\defeq \sm{k:\N}d\cdot k=n.
  \end{equation*}
\end{defn}

\begin{rmk}
  This type-theoretical definition of the divisibility relation using $\Sigma$-types has two important consequences:
  \begin{enumerate}
  \item The principal way to show that $d\mid n$ holds is to construct a pair $(k,p)$ consisting of a natural number $k$ and an identification $p:d\cdot k=n$.
  \item The principal way to use a hypothesis $H:d\mid n$ in a proof is to proceed by $\Sigma$-induction on the variable $H$. We then get to assume a natural number $k$ and an identification $p:d\cdot k=n$, in order to proceed with the proof.
  \end{enumerate}
\end{rmk}

\begin{eg}\label{rmk:elementary-facts-div}
  Just as existential quantification ($\exists$) is translated via the Curry-Howard interpretation to $\Sigma$-types, the translation of the universal quantification ($\forall$) in type theory via the Curry-Howard interpretation is to $\Pi$-types. For example, the assertion that every natural number is divisible by $1$ is expressed in type theory as
  \begin{equation*}
    \prd{x:\N} 1\mid x.
  \end{equation*}
  In other words, in order to show that every number $x:\N$ is divisible by $1$ we need to construct a dependent function
  \begin{equation*}
    \lam{x}p(x):\prd{x:\N}1\mid x.
  \end{equation*}
  We do this by constructing an element
  \begin{equation*}
    p(x):\sm{k:\N}1\cdot k=x
  \end{equation*}
  indexed by $x:\N$. Such an element $p(x)$ is constructed as the pair $(x,q(x))$, where the identification $q(x):1\cdot x=x$ is obtained from the left unit law of multiplication on $\N$, which was constructed in \cref{ex:semi-ring-laws-N}.

  Similarly, the type theoretic proof that every natural number $k$ divides $0$, i.e., that $k\mid 0$, is the pair $(0,p)$ consisting of the natural number $0$ and the identification $p:k\cdot 0=0$ obtained from the right annihilation law of multiplication on $\N$. This identification was also constructed in \cref{ex:semi-ring-laws-N}.
\end{eg}

In the following proposition we will see examples of how a hypothesis of type $d\mid x$ can be used.

\begin{prp}\label{prp:div-3-for-2}
  Consider three natural numbers $d$, $x$ and $y$. If $d$ divides any two of the three numbers $x$, $y$, and $x+y$, then it also divides the third.
\end{prp}

\begin{proof}
  We will only show that if $d$ divides $x$ and $y$, then it divides $x+y$. The remaining two claims, that if $d$ divides $y$ and $x+y$ then it divides $x$, and that if $d$ divides $x$ and $x+y$ then it divides $y$, are left as \cref{ex:div-3-for-2}.

  Suppose that $d$ divides both $x$ and $y$. By assumption we have elements
  \begin{equation*}
    H:\sm{k:\N}d\cdot k=x,\qquad\text{and}\qquad K:\sm{k:\N}d\cdot k=y.
  \end{equation*}
  Since the types of the variables $H$ and $K$ are $\Sigma$-types, we proceed by $\Sigma$-induction on $H$ and $K$. Therefore we get to assume a natural number $k:\N$ equipped with an identification $p:d\cdot k=x$, and a natural number $l:\N$ equipped with an identification $q:d\cdot l=y$. Our goal is now to construct an identification
  \begin{equation*}
    d\cdot (k+l)=x+y.
  \end{equation*}
We construct such an identification as a concatenation $\ct{\alpha}{(\ct{\beta}{\gamma})}$, where the types of the identifications $\alpha$, $\beta$, and $\gamma$ are as follows:
  \begin{equation*}
    \begin{tikzcd}
      d\cdot(k+l) \arrow[r,equals,"\alpha"] & d\cdot k+d\cdot l \arrow[r,equals,"\beta"] & x+d\cdot l \arrow[r,equals,"\gamma"] & x+y.
    \end{tikzcd}
  \end{equation*}
  The identification $\alpha$ is obtained from the fact that multiplication on $\N$ distributes over addition, which was shown in \cref{ex:distributive-mul-addN}. The identifications $\beta$ and $\gamma$ are constructed using the action on paths of a function:
  \begin{equation*}
    \beta\defeq\ap{(\lam{t}t+d\cdot l)}{p},\qquad\text{and}\qquad \gamma\defeq \ap{(\lam{t}x+t)}{q}
  \end{equation*}
  To conclude the proof that $d\mid x+y$, note that we have constructed the pair
  \begin{equation*}
    (k+l,\ct{\alpha}{(\ct{\beta}{\gamma})}):\sm{k:\N}d\cdot k=x+y.\qedhere
  \end{equation*}
\end{proof}

The full Curry-Howard interpretation of logic into type theory also involves interpretations of disjunction, conjunction, implication, and equality.

The introduction and elimination rules for disjunction are, for instance,
\begin{equation*}
  \AxiomC{$P$}
  \UnaryInfC{$P\lor Q$}
  \DisplayProof
  \qquad
  \AxiomC{$Q$}
  \UnaryInfC{$P\lor Q$}
  \DisplayProof
  \qquad
  \text{and}
  \qquad
  \AxiomC{$P\Rightarrow R$}
  \AxiomC{$Q\Rightarrow R$}
  \BinaryInfC{$P\lor Q\Rightarrow R$}
  \DisplayProof
\end{equation*}
The two introduction rules assert that $P\lor Q$ holds provided that $P$ holds, and that $P\lor Q$ holds provided that $Q$ holds. These rules are analogous to the introduction rules for coproduct, which assert that there are functions $\inl : A\to A+B$ and $\inr : B \to A+B$. Furthermore, the non-dependent elimination principle for coproducts gives a function
\begin{equation*}
  (A\to C) \to ((B \to C) \to (A+B \to C))
\end{equation*}
for any type $C$, which is again analogous to the elimination rule of disjunction. The Curry-Howard interpretation of disjunction into type theory is therefore as coproducts.

To interpret conjunction into type theory we observe that the introduction rule and elimination rules for conjunction are
\begin{equation*}
  \AxiomC{$P$}
  \AxiomC{$Q$}
  \BinaryInfC{$P\land Q$}
  \DisplayProof
  \qquad
  \text{and}
  \qquad
  \AxiomC{$P\land Q$}
  \UnaryInfC{$P$}
  \DisplayProof
  \qquad
  \AxiomC{$P\land Q$}
  \UnaryInfC{$Q$}
  \DisplayProof
\end{equation*}
Product types possess such structure, where we have the pairing operation $\pair:A\to (B\to A\times B)$ and the projections $\proj 1:A\times B\to A$ and $\proj 2 : A\times B\to B$ give interpretations of the introduction and elimination rules for conjunction. The Curry-Howard interpretation of conjunction into type theory is therefore by products. We summarize the full Curry-Howard interpretation in \cref{table:Curry-Howard}.

\begin{table}[t]
  \begin{tabular}{ll}
    \toprule
    \multicolumn{2}{c}{The Curry-Howard interpretation} \\
    \midrule
    Propositions & Types \\
    Proofs & Elements \\
    Predicates & Type families \\
    $\top$ & $\unit$ \\
    $\bot$ & $\emptyt$ \\
    $P\lor Q$ & $A+B$ \\
    $P\land Q$ & $A\times B$ \\
    $P\Rightarrow Q$ & $A\to B$ \\
    $\neg P$ & $A\to \emptyt$ \\
    $\exists_{x}P(x)$ & $\sm{x:A}B(x)$ \\
    $\forall_{x}P(x)$ & $\prd{x:A}B(x)$ \\
    $x=y$ & $x=y$ \\
    \bottomrule
  \end{tabular}
  \caption{\label{table:Curry-Howard}The Curry-Howard interpretation of logic into type theory.}
\end{table}

\begin{rmk}
  We should note, however, that despite the similarities between logic and type theory that are highlighted in the Curry-Howard interpretation, there are also some differences. One important difference is that types may contain many elements, whereas in logic, propositions are usually considered to be \emph{proof irrelevant}. This means that to establish the truth of a proposition it only matters \emph{whether} it can be proven, not in how many different ways it can be proven. To address this dissimilarity between general types and logic, we will introduce in \cref{chap:uf} a more refined way of interpreting logic into type theory. In \cref{chap:hierarchy} we will define the type $\isprop(A)$, which expresses the property that the type $A$ is a proposition. Furthermore, we will introduce the \emph{propositional truncation} operation in \cref{sec:propositional-truncation}, which we will use to interpret logic into type theory in such a way that all logical assertions are interpreted as types that satisfy the condition of being a proposition.
\end{rmk}

\subsection{The congruence relations on \texorpdfstring{$\N$}{ℕ}}

Relations in the Curry-Howard interpretation of logic into type theory are also type valued. More specifically, a binary relation on a type $A$ is a family of types $R(x,y)$ indexed by $x,y:A$. Such relations are sometimes called \emph{typal}.

\begin{defn}
  Consider a type $A$. A \define{(typal) binary relation} on $A$ is defined to be a family of types $R(x,y)$ indexed by $x,y:A$. Given a binary relation $R$ on $A$, we say that $R$ is \define{reflexive} if it comes equipped with
  \begin{align*}
    \rho & : \prd{x:A}R(x,x), \\
    \intertext{we say that $R$ is \define{symmetric} if it comes equipped with}
    \sigma & : \prd{x,y:A} R(x,y)\to R(y,x), \\
    \intertext{and we say that $R$ is \define{transitive} if it comes equipped with}
    \tau & : \prd{x,y,z:A} R(x,y)\to (R(y,z)\to R(x,z)).
  \end{align*}
  A \define{(typal) equivalence relation} on $A$ is a reflexive, symmetric, and transitive binary typal relation on $A$.
\end{defn}

To define the congruence relation modulo $k$ in type theory using the Curry-Howard interpretation, we will define for any three natural numbers $x$, $y$, and $k$, a \emph{type}
\begin{equation*}
  x\equiv y\mod k
\end{equation*}
consisting of the proofs that $x$ is congruent to $y$ modulo $k$. We will define this type by directly interpreting Gauss' definition of the congruence relations in his \emph{Disquisitiones Arithmeticae} \cite{Gauss}: two numbers $x$ and $y$ are congruent modulo $k$ if $k$ divides the symmetric difference $\distN(x,y)$ between $x$ and $y$. Recall that $\distN(x,y)$ was defined in \cref{ex:distN} recursively by
  \begin{align*}
    \distN(0,0) & \defeq 0 & \distN(0,y+1) & \defeq y+1 \\
    \distN(x+1,0) & \defeq x+1 & \distN(x+1,y+1) & \defeq \distN(x,y).
  \end{align*}

\begin{defn}
  Consider three natural numbers $k,x,y:\N$. We say that $x$ is \define{congruent to $y$ modulo $k$}\index{congruence relations on N@{congruence relations on $\N$}|textbf}\index{natural numbers!congruence relations|textbf} if it comes equipped with an element of type
  \begin{equation*}
    x\equiv y \mod k \defeq k\mid\distN(x,y).
  \end{equation*}
\end{defn}

\begin{eg}
  For example, $k\equiv 0\mod k$. To see this, we have to show that $k\mid\distN(k,0)$. Since $\distN(k,0)=k$ it suffices to show that $k\mid k$. That is, we have to construct a natural number $l$ equipped with an identification $p:kl=k$. Of course, we choose $l\defeq 1$, and the equation $k1=k$ holds by the right unit law for multiplication on $\N$, which was shown in \cref{ex:semi-ring-laws-N}.
\end{eg}

\begin{prp}\label{prp:congruence-eqrel}
  For each $k:\N$, the congruence relation modulo $k$ is an equivalence relation.
\end{prp}

\begin{proof}
  Reflexivity follows from the fact that $\distN(x,x)=0$, and any number divides $0$. Symmetry follows from the fact that $\distN(x,y)=\distN(y,x)$ for any two natural numbers $x$ and $y$.

  The non-trivial part of the claim is therefore transitivity. Here we use the fact that for any three natural numbers $x$, $y$, and $z$, at least one of the equalities
  \begin{align*}
    \distN(x,y)+\distN(y,z) & =\distN(x,z) \\
    \distN(y,z)+\distN(x,z) & =\distN(x,y) \\
    \distN(x,z)+\distN(x,y) & =\distN(y,z)
  \end{align*}
  holds. A formal proof of this fact is given by case analysis on the six possible ways in which $x$, $y$, and $z$ can be ordered:
  \begin{align*}
    x\leq y & \text{ and }y\leq z, & x\leq z & \text{ and }z\leq y, \\
    y\leq z & \text{ and }z\leq x, & y\leq x & \text{ and }x\leq z, \\
    z\leq x & \text{ and }x\leq y, & z\leq y & \text{ and }y\leq x.
  \end{align*}
  Therefore it follows by \cref{ex:distN-triangle-equality} and \cref{prp:div-3-for-2} that ${k\mid\distN(x,z)}$ if ${k\mid\distN(x,y)}$ and ${k\mid\distN(y,z)}$.
\end{proof}

\subsection{The standard finite types}\label{sec:Fin}

The standard finite sets are classically defined as the sets $\{x\in\N\mid x<k\}$. This leads to the question of how to interpret a subset $\{x\in A\mid P(x)\}$ in type theory.

Since type theory is set up in such a way that elements come equipped with their types, subsets aren't formed the same way as in set theory, where the comprehension axiom is used to form the set $\{x\in A\mid P(x)\}$ for any predicate $P$ over $A$. The Curry-Howard interpretation dictates that predicates are interpreted as dependent types. Therefore, a set of elements $x\in A$ such that $P(x)$ holds is interpreted in type theory as the type of terms $x:A$ equipped with an element (a proof) $p:P(x)$. In other words, we interpret a subset $\{x\in A\mid P(x)\}$ as the type $\sm{x:A}P(x)$.

\begin{rmk}
  The alert reader may now have observed that the interpretation of a subset $\{x\in A\mid P(x)\}$ in type theory is the same as the interpretation of the proposition $\exists_{(x\in A)}P(x)$, while indeed the subset $\{x\in A\mid P(x)\}$ has a substantially different role in mathematics than the proposition $\exists_{(x\in A)}P(x)$. This points at a slight problem of the Curry-Howard interpretation of the existential quantifier. While the Curry-Howard interpretation of the existential quantifier is nevertheless useful and important, we will reinterpret the existential quantifier in type theory in \cref{sec:logic}.
\end{rmk}

Since subsets are interpreted as $\Sigma$-types, the `classical' definition of the standard finite types is
\begin{equation*}
  \classicalFin_k:=\sm{x:\N}x<k.
\end{equation*}
This is a perfectly fine definition of the standard finite types. However, the usual definition of the standard finite types in Martin-L\"of's dependent type theory is a more direct, recursive definition, which takes full advantage of the inductive constructions of dependent type theory. 

\begin{defn}\label{defn:fin}
  We define the type family $\Fin{}$ of the \define{standard finite types}\index{Fin k@{$\Fin{k}$}|see {standard finite type}}\index{Fin k@{$\Fin{k}$}|textbf}\index{standard finite type}\index{type family!of standard finite types} over $\N$ recursively by
  \begin{align*}
    \Fin{0} & \defeq \emptyt \\*
    \Fin{k+1} & \defeq \Fin{k}+\unit.
  \end{align*}
  We will write $i$ for the inclusion $\inl:\Fin{k}\to\Fin{k+1}$ and we will write $\ttt$ for the point $\inr(\ttt)$.
\end{defn}

In \cref{ex:classical-Fin} you will be asked to show that the types $\classicalFin_k$ and $\Fin{k}$ are isomorphic.

\begin{rmk}
The type family $\Fin{}$ over $\N$ can be given its own induction principle, which is, at least for the time being, the principal way to make constructions on $\Fin{k}$ for arbitrary $k:\N$ and to prove properties about those constructions. The induction principle of the standard finite types tells us that the family of standard finite types is inductively generated by
\begin{align*}
  i & : \Fin{k}\to\Fin{k+1} \\*
  \ttt & : \Fin{k+1}. 
\end{align*}
In other words, we can define a dependent function $f:\prd{k:\N}\prd{x:\Fin{k}}P_k(x)$ by defining
\begin{align*}
  g_k & : \prd{x:\Fin{k}}P_k(x)\to P_{k+1}(i(x)) \\*
  p_k & : P_{k+1}(\ttt)
\end{align*}
for each $k:\N$. The function $f$ defined in this way then satisfies the judgmental equalities
\begin{align*}
  f_{k+1}(i(x)) & \jdeq g_k(x,f_k(x)) \\*
  f_{k+1}(\ttt) & \jdeq p_k.
\end{align*}
These judgmental equalities completely determine the function $f$, and therefore we may also present such inductive definitions by pattern matching:
  \begin{align*}
    f_{k+1}(i(x)) & \defeq g_k(x,f_k(x)) \\*
    f_{k+1}(\ttt) & \defeq p_k.
  \end{align*}
\end{rmk}

We will often use definitions by pattern matching for two reasons: (i) such definitions are concise, and (ii) they display the judgmental equalities that hold for the defined object. Those judgmental equalities are the only thing we know about that object, and proving a claim about it often amounts to finding a way to apply these judgmental equalities.

To illustrate this way of working with the standard finite types, we define the inclusion functions $\Fin{k}\to\N$, and show that these are injective. In order to show that $\natFin_k$ is injective, we will also show that $\natFin_k$ is bounded.

\begin{defn}\label{defn:natFin}
  We define the inclusion $\natFin_k : \Fin{k}\to\N$ inductively by
  \begin{align*}
    \natFin_{k+1}(i(x)) & \defeq \natFin_{k}(x) \\
    \natFin_{k+1}(\ttt) & \defeq k.
  \end{align*}
\end{defn}

\begin{lem}\label{lem:is-bounded-natFin}
  The function $\natFin:\Fin{k}\to\N$ is bounded, in the sense that $\natFin(x)< k$ for each $x:\Fin{k}$.
\end{lem}

\begin{proof}
  The proof is by induction. In the base case there is nothing to show. In the inductive step, we have the inequalities $\natFin_{k+1}(i(x))\jdeq\natFin_{k}(x)<k<k+1$, where the first inequality holds by the inductive hypothesis, and we also have
  \begin{equation*}
    \natFin_{k+1}(\ttt)\jdeq k<k+1.\qedhere
  \end{equation*}
\end{proof}

\begin{prp}\label{prp:is-injective-natFin}
  The inclusion function $\natFin_k : \Fin{k}\to \N$ is injective, for each $k:\N$.
\end{prp}

\begin{proof}
  We define a function $\alpha_k(x,y):(\natFin_k(x)=\natFin_k(y))\to (x=y)$ recursively by
  \begin{align*}
    \alpha_{k+1}(i(x),i(y),p) & \defeq \ap{i}{\alpha_k(x,y,p)} & \alpha_{k+1}(i(x),\ttt,p) & \defeq \exfalso(f(p)) \\
    \alpha_{k+1}(\ttt,i(y),p) & \defeq \exfalso(g(p)) & \alpha_{k+1}(\ttt,\ttt,p) & \defeq \refl{},
  \end{align*}
  where $f:(\natFin_{k+1}(i(x))=\natFin_{k+1}(\ttt))\to\emptyt$ and $g:(\natFin_{k+1}(\ttt)=\natFin_{k+1}(i(y)))\to\emptyt$ are obtained from the fact that $\natFin_{k+1}(i(z))\jdeq\natFin_k(z)<k$ for any $z:\Fin{k}$, and the fact that $\natFin_{k+1}(\ttt)\jdeq k$.
\end{proof}

\subsection{The natural numbers modulo \texorpdfstring{$k+1$}{k+1}}\label{subsec:finite-types-quotient-maps}

Given an equivalence relation $\sim$ on a set $A$ in classical mathematics, the quotient $A/{\sim}$ comes equipped with a quotient map $q:A\to A/{\sim}$ that satisfies two important properties: (1) The map $q$ satisfies the condition
\begin{equation*}
  q(x)=q(y)\leftrightarrow x\sim y,
\end{equation*}
and (2) the map $q$ is surjective. The first condition is called the \define{effectiveness} of the quotient map.

In classical mathematics, a map $f:A\to B$ is said to be surjective if for every $b\in B$ there exists an element $a\in A$ such that $f(a)=b$. Following the Curry-Howard interpretation, a map $f:A\to B$ is therefore surjective if it comes equipped with a dependent function
\begin{equation*}
  \prd{b:B}\sm{a:A}f(a)=b.
\end{equation*}
However, there is a subtle issue with this interpretation of surjectivity. It is somewhat stronger than the classical notion of surjectivity, because a dependent function $\prd{b:B}\sm{a:A}f(a)=b$ provides for every element $b:B$ an \emph{explicit} element $a:A$ equipped with an explicit identification $p:f(a)=b$, whereas in the classical notion of surjectivity such an element $a\in A$ is merely asserted to exist. To emphasize that the Curry-Howard interpretation of surjectivity is stronger than intended we make the following definition, and we will properly introduce surjective maps in \cref{subsec:surjective}.

\begin{defn}
  Consider a function $f:A\to B$. We say that $f$ is \define{split surjective} if it comes equipped with an element of type
  \begin{equation*}
    \issplitsurjective(f):=\prd{b:B}\sm{a:A}f(a)=b.
  \end{equation*}
\end{defn}

Martin-L\"of's dependent type theory doesn't have a general way of forming quotients of types. However, in the specific case of the congruence relations on $\N$ we can define the type of natural numbers modulo $k+1$ as the standard finite type $\Fin{k+1}$. We will show that $\Fin{k+1}$ comes equipped with a map
\begin{equation*}
  [\blank]_{k+1}:\N\to \Fin{k+1}
\end{equation*}
for each $k:\N$, and we will show in \cref{thm:effective-mod-k,thm:issec-nat-Fin} that this map satisfies conditions (1) and (2) in the split surjective sense.

To prepare for the definition of the quotient map $[\blank]_{k+1}$, we will first define a zero element of $\Fin{k+1}$ and successor function on each $\Fin{k}$. We will also define an auxiliary function $\skipzeroFin_k:\Fin{k}\to\Fin{k+1}$, which is used in the definition of the successor function. The map $[\blank]_{k+1}$ is then defined by iterating the successor function. 

\begin{defn} ~\nopagebreak
  \begin{enumerate}
  \item We define the \define{zero element} $\zeroFin_k:\Fin{k+1}$ recursively by
    \begin{align*}
      \zeroFin_0 & \defeq\ttt \\*
      \zeroFin_{k+1} & \defeq i(\zeroFin_k).
                       \intertext{Since there is a mismatch between the index of $\zeroFin_k$ and the index of its type, we will often simply write $\zeroFin$ or $0$ for the zero element of $\Fin{k+1}$.
    \item We define the function $\skipzeroFin_k:\Fin{k}\to\Fin{k+1}$ recursively by}
      \skipzeroFin_{k+1}(i(x)) & \defeq i(\skipzeroFin_k(x)) \\*
      \skipzeroFin_{k+1}(\ttt) & \defeq \ttt.
    \intertext{\item We define the \define{successor function} $\succFin_k:\Fin{k}\to\Fin{k}$ recursively by}
      \succFin_{k+1}(i(x)) & \defeq \skipzeroFin_k(x) \\*                       
      \succFin_{k+1}(\ttt)    & \defeq \zeroFin_k.
    \end{align*}
  \end{enumerate}
\end{defn}

\begin{defn}
  For any $k:\N$, we define the map $[\blank]_{k+1}:\N\to\Fin{k+1}$ recursively on $x$ by
  \begin{align*}
    [0]_{k+1} & \defeq 0 \\*
    [x+1]_{k+1} & \defeq \succFin_{k+1}[x]_{k+1}.
  \end{align*}
\end{defn}

Our next intermediate goal is to show that $x\equiv \natFin[x]_{k+1}\mod k+1$ for any natural number $x$. This fact is a consequence of the following simple lemma, that will help us compute with the maps $\natFin : \Fin{k}\to\N$.

\begin{lem}\label{lem:nat-Fin}
  We make three claims:
  \begin{enumerate}
  \item For any $k:\N$ there is an identification
    \begin{align*}
      \natFin(\zeroFin_k) & = 0
  \intertext{\item For any $k:\N$ and any $x:\Fin{k}$, we have}
      \natFin(\skipzeroFin_k(x)) & = \natFin(x)+1.
  \intertext{\item For any $k:\N$ and any $x:\Fin{k}$, we have}
      \natFin(\succFin_k(x)) & \equiv \natFin(x)+1 \mod k.
    \end{align*}
  \end{enumerate}
\end{lem}

\begin{proof}
  For the first claim, we define an identification $\alpha_k:\natFin(\zeroFin_k)=0$ recursively by
  \begin{align*}
    \alpha_0 & \defeq \refl{} \\
    \alpha_{k+1} & \defeq \alpha_k.
  \intertext{For the second claim, we define an identification $\beta_k(x):\natFin(\skipzeroFin_k(x))=\natFin(x)+1$ recursively by}
    \beta_{k+1}(i(x)) & \defeq \beta_k(x) \\
    \beta_{k+1}(\ttt) & \defeq \refl{}.
  \end{align*}
  For the third claim, we again define an element $\gamma_k(x):\natFin(\succFin_k(x)) \equiv \natFin(x)+1\mod{k}$ recursively. To obtain
  \begin{equation*}
    \gamma_{k+1}(i(x)) : \natFin(\succFin_{k+1}(i(x))) \equiv\natFin(i(x))+1\mod{k+1},
  \end{equation*}
  we calculate
  \begin{align*}
    \natFin(\succFin_{k+1}(i(x))) & \jdeq \natFin(\skipzeroFin(x)) & & \text{by definition of }\succFin\\
                                  & = \natFin(x)+1 & & \text{by claim (ii).}
  \end{align*}
  Since the congruence relation modulo $k+1$ is reflexive, we obtain $\gamma_{k+1}(i(x))$ from the identification of the above calculation. To obtain
  \begin{equation*}
    \gamma_{k+1}(\ttt) : \natFin(\succFin_{k+1}(\ttt)) \equiv \natFin(\ttt)+1\mod{k+1},
  \end{equation*}
  we calculate
  \begin{align*}
    \natFin(\succFin_{k+1}(\ttt)) & \jdeq \natFin(0) & & \text{by definition of }\succFin \\
                                  & = 0 & & \text{by claim (i)} \\
                                  & \equiv k+1 & & \text{by \cref{rmk:elementary-facts-div}} \\
                                  & \jdeq \natFin(\ttt)+1 & & \text{by definition of }\natFin.\qedhere
  \end{align*}
\end{proof}

\begin{prp}\label{prp:cong-nat-mod-succ}
  For any $x:\N$ we have
  \begin{equation*}
    \natFin[x]_{k+1}\equiv x \mod k+1.
  \end{equation*}
\end{prp}

\begin{proof}
  The proof by induction on $x$. The fact that
  \begin{equation*}
    \natFin[0]_{k+1}\equiv 0 \mod {k+1}
  \end{equation*}
  is immediate from the fact that $\natFin[0]_{k+1}\jdeq\natFin(0)=0$, which was shown in \cref{lem:nat-Fin}. In the inductive step, we have to show that
  \begin{equation*}
    \natFin[x+1]_{k+1}\equiv x+1\mod k+1.
  \end{equation*}
  This follows from the following computation
  \begin{align*}
    \natFin[x+1]_{k+1} & \jdeq \natFin(\succFin_{k+1}[x]_{k+1}) & & \text{by definition of }[\blank]_{k+1} \\
                       & \equiv \natFin[x]_{k+1}+1 & & \text{by \cref{lem:nat-Fin}} \\
                       & \equiv x+1 & & \text{by the inductive hypothesis.}\qedhere
  \end{align*}
\end{proof}

We need one more fact before we can prove \cref{thm:effective-mod-k,thm:issec-nat-Fin}.

\begin{prp}\label{cor:eq-congN}
  For any natural number $x<d$ we have
  \begin{equation*}
  d\mid x\leftrightarrow x=0.  
  \end{equation*}
  Consequently, for any two natural numbers $x$ and $y$ such that $\distN(x,y)<k$, we have
  \begin{equation*}
    x\equiv y\mod k\leftrightarrow x=y.
  \end{equation*}
\end{prp}

\begin{proof}
  Note that the implication $x=0\to d\mid x$ is trivial, so it suffices to prove the forward implication
  \begin{equation*}
    d\mid x \to x=0.
  \end{equation*}
  This implication clearly holds if $x\jdeq 0$. Therefore we only have to show that $d\mid x+1$ implies $x+1=0$, if we assume that $x+1<d$. In other words, we will derive a contradiction from the hypotheses that $x+1<d$ and $d\mid x+1$. To reach a contradiction we use \cref{ex:contradiction-le}, by which it suffices to show that $d\leq x+1$.
  
  We proceed by $\Sigma$-induction on the (unnamed) variable of type $d\mid x+1$, so we get to assume a natural number $k$ equipped with an identification $p:dk=x+1$. In the case where $k\jdeq 0$ we reach an immediate contradiction via \cref{prp:zero-one}, because we obtain that $0=d\cdot 0=x+1$. In the case where $k\jdeq\succN(k')$ it follows that
  \begin{equation*}
    d\leq dk'+ d\jdeq dk = x+1.\qedhere
  \end{equation*}
\end{proof}

\begin{thm}\label{thm:effective-mod-k}
  Consider a natural number $k$. Then we have
  \begin{equation*}
    [x]_{k+1}=[y]_{k+1} \leftrightarrow x\equiv y\mod k+1,
  \end{equation*}
  for any $x,y:\N$.
\end{thm}

\begin{proof}
  First note that, since $\natFin$ is injective by \cref{prp:is-injective-natFin}, we have
  \begin{align*}
    [x]_{k+1}=[y]_{k+1} & \leftrightarrow \natFin[x]_{k+1}=\natFin[y]_{k+1}.
  \end{align*}
  Since the inequalities $\natFin[x]_{k+1}<k+1$ and $\natFin[y]_{k+1}<k+1$ hold by \cref{lem:is-bounded-natFin}, it follows by \cref{cor:eq-congN} that
  \begin{equation*}
    \natFin[x]_{k+1}=\natFin[y]_{k+1}\leftrightarrow \natFin[x]_{k+1}\equiv\natFin[y]_{k+1}\mod k+1.   
  \end{equation*}
  The latter condition is by \cref{prp:cong-nat-mod-succ} equivalent to the condition that $x\equiv y\mod k+1$.
\end{proof}

\begin{thm}\label{thm:issec-nat-Fin}
  For any $x:\Fin{k+1}$ there is an identification
  \begin{equation*}
    [\natFin(x)]_{k+1}=x.
  \end{equation*}
  In other words, the map $[\blank]_{k+1}:\N\to \Fin{k+1}$ is split surjective.
\end{thm}

\begin{proof}
  Since $\natFin:\Fin{k+1}\to\N$ is injective by \cref{prp:is-injective-natFin}, it suffices to show that
  \begin{equation*}
    \natFin[\natFin(x)]_{k+1}=\natFin(x).
  \end{equation*}
  Now observe that $\natFin[\natFin(x)]_{k+1}<k+1$ and $\natFin(x)<k+1$. By \cref{cor:eq-congN} it therefore suffices to show that
  \begin{equation*}
    \natFin[\natFin(x)]_{k+1}\equiv\natFin(x)\mod{k+1}.
  \end{equation*}
  This fact is an instance of \cref{prp:cong-nat-mod-succ}.
\end{proof}

\subsection{The cyclic groups}
We can now define the cyclic groups $\Z/k$ for each $k:\N$. Note that $\Z/k$ must come equipped with the structure of a quotient $\Z/{\equiv}$ of $\Z$ by the congruence relation modulo $k$. In the case where $k\jdeq 0$, we have that $x\equiv y\mod{0}$ if and only if $x=y$. This motivates the following definition:

\begin{defn}\label{defn:Zk}
  We define the type $\Z/k$ for each $k:\N$ by
  \begin{equation*}
    \Z/0\defeq \Z\qquad\text{and}\qquad \Z/{(k+1)}\defeq\Fin{k+1}.
  \end{equation*}
\end{defn}

Recall from \cref{ex:int_group_laws} that $\Z/0$ already comes equipped with the structure of a group, but the group structure on $\Z/{(k+1)}$ remains to be defined.

\begin{defn}
  We define the \define{addition} operation on $\Z/{(k+1)}$ by
  \begin{equation*}
    x+y\defeq[\natFin(x)+\natFin(y)]_{k+1},
  \end{equation*}
  and we define the \define{additive inverse} operation on $\Z/{(k+1)}$ by
  \begin{equation*}
    -x\defeq[\distN(\natFin(x),k+1)]_{k+1}.
  \end{equation*}
\end{defn}

\begin{rmk}
  The following congruences modulo $k+1$ follow immediately from \cref{prp:cong-nat-mod-succ}:
  \begin{align*}
    \natFin(0) & \equiv 0 \\
    \natFin(x+y) & \equiv \natFin(x)+\natFin(y) \\
    \natFin(-x) & \equiv \distN(\natFin(x),k+1).
  \end{align*}
\end{rmk}

Before we show that addition on $\Z/{k}$ satisfies the group laws, we have to show that addition on $\N$ preserves the congruence relation.

\begin{prp}
  Consider $x,y,x',y':\N$. If any two of the following three properties hold, then so does the third:
  \begin{enumerate}
  \item $x\equiv x'\mod k$,
  \item $y\equiv y'\mod k$,
  \item $x+y\equiv x'+y'\mod k$.
  \end{enumerate}
\end{prp}

\begin{proof}
  Recall that the distance function $\distN$ is translation invariant by \cref{ex:translation-invariant-distN}. Therefore it follows that
  \begin{equation}\label{eq:translation-invariant-congN}
    a\equiv b\mod k \leftrightarrow a+c\equiv b+c\mod k.\tag{\textasteriskcentered}
  \end{equation}
  We will use this observation to prove the claim.
  
  First, suppose that $x\equiv x'$ and $y\equiv y'$ modulo $k$. Then it follows by \cref{eq:translation-invariant-congN} that
  \begin{equation*}
    x+y\equiv x'+y\equiv x'+y'.
  \end{equation*}
  This shows that (i) and (ii) together imply (iii).

  Next, suppose that $x\equiv x'$ and $x+y\equiv x'+y'$ modulo $k$. Then it follows that
  \begin{equation*}
    x+y\equiv x'+y'\equiv x+y'.
  \end{equation*}
  Applying \cref{eq:translation-invariant-congN} once more in the reverse direction, we obtain that $y\equiv y'$ modulo $k$. This shows that (i) and (iii) together imply (ii).

  The remaining claim, that (ii) and (iii) together imply (i), follows by commutativity of addition from the fact that (i) and (iii) together imply (ii).
\end{proof}

\begin{thm}
  The addition operation on $\Z/{k}$ satisfies the laws of an abelian group:
  \begin{align*}
    0+x & = x & x+0 & = x \\
    (-x)+x & = 0 & x+(-x) & = 0 \\
    (x+y)+z & = x+(y+z) & x+y & = y+x. 
  \end{align*}
\end{thm}

\begin{proof}
  The fact that the addition operation on $\Z/0$ satisfies the laws of an abelian group was stated as \cref{ex:int_group_laws}. Therefore we will only show that addition on $\Z/{(k+1)}$ satisfies the laws of an abelian group.

  We first note that by commutativity of addition on $\N$, it follows immediately that addition on $\Z/{(k+1)}$ is commutative.

  To prove associativity, note that by \cref{thm:effective-mod-k} it suffices to show that
  \begin{equation*}
    \natFin(x+y)+\natFin(z)\equiv\natFin(x)+\natFin(y+z)\mod k+1.
  \end{equation*}
  Since addition on $\Z/{(k+1)}$ maps preserves the congruence relation, and since we have the congruences
  \begin{align*}
    \natFin(x+y) & \equiv \natFin(x)+\natFin(y) \mod k+1 \\
    \natFin(y+z) & \equiv \natFin(y)+\natFin(z) \mod k+1,
  \end{align*}
  it suffices to show that
  \begin{equation*}
    (\natFin(x)+\natFin(y))+\natFin(z) \equiv \natFin(x)+(\natFin(y)+\natFin(z)) \mod k+1.
  \end{equation*}
  This follows immediately by associativity of addition on $\N$.

  To show that addition on $\Z/{(k+1)}$ satisfies the right unit law, we first observe that it suffices to show that
  \begin{equation*}
    [\natFin(x)+\natFin(0)]_{k+1}=[\natFin(x)]_{k+1}
  \end{equation*}
  because there is an identification $[\natFin(x)]_{k+1}=x$ by \cref{thm:issec-nat-Fin}. By \cref{thm:effective-mod-k} it now suffices tho show that
  \begin{equation*}
    \natFin(x)+\natFin(0)\equiv\natFin(x)\mod k+1. 
  \end{equation*}
  This follows immediately from the fact that $\natFin(0)=0$. The left unit law now follows from the right unit law by commutativity. We leave the inverse laws as an exercise.
\end{proof}

\begin{exercises}
  \exitem \label{ex:div-3-for-2}Complete the proof of \cref{prp:div-3-for-2}.
  \exitem \label{ex:is-poset-div}Show that the divisibility relation satisfies the axioms of a poset, i.e., that it is reflexive, antisymmetric, and transitive.
  \exitem \label{ex:div-factorial}Construct a dependent function
  \begin{equation*}
    \prd{x:\N}(x\neq 0)\to ((x\leq n)\to (x\mid n!))
  \end{equation*}
  for every $n:\N$.
  \exitem Define $1\defeq[1]_{k+1}:\Fin{k+1}$. Show that
  \begin{equation*}
    \succFin_{k+1}(x)=x+1
  \end{equation*}
  for any $x:\Fin{k+1}$.
  \exitem \label{ex:Eq-Fin}The observational equality on $\Fin{k}$ is a binary relation
  \begin{equation*}
    \EqFin_{k}:\Fin{k}\to(\Fin{k}\to\UU_0)
  \end{equation*}
  defined recursively by
  \begin{align*}
    \EqFin_{k+1}(i(x),i(y)) & \defeq \EqFin_k(x,y) & \EqFin_{k+1}(i(x),\ttt) & \defeq \emptyt \\*
    \EqFin_{k+1}(\ttt,i(y)) & \defeq \emptyt & \EqFin_{k+1}(\ttt,\ttt) & \defeq \unit.
  \end{align*}
  \begin{subexenum}
  \item \label{ex:eq-iff-Eq-Fin}Show that
  \begin{equation*}
    (x=y)\leftrightarrow \EqFin_k(x,y)
  \end{equation*}
  for any two elements $x,y:\Fin{k}$.
  \item \label{ex:is-injective-i-Fin}Show that the function $i:\Fin{k}\to\Fin{k+1}$ is injective, for each $k:\N$.
  \item \label{ex:neq-zero-succ-Fin}
  Show that
  \begin{equation*}
    \succFin_{k+1}(i(x))\neq 0
  \end{equation*}
  for any $x:\Fin{k}$.
  \item Show that function $\succFin_k:\Fin{k}\to\Fin{k}$ is injective, for each $k:\N$.
  \end{subexenum}
  \exitem \label{ex:has-inverse-succ-Fin}The predecessor function $\predFin_k:\Fin{k}\to\Fin{k}$ is defined in three steps, just as in the definition of the successor function on $\Fin{k}$.
  \begin{enumerate}
  \item We define the element $\negtwoFin_k:\Fin{k+1}$ by
    \begin{align*}
      \negtwoFin_0 & \defeq\ttt \\*
      \negtwoFin_{k+1} & \defeq i(\ttt).
    \intertext{\item We define the function $\skipnegtwoFin_k:\Fin{k}\to\Fin{k+1}$ recursively by}
      \skipnegtwoFin_{k+1}(i(x)) & \defeq i(i(x)) \\*
      \skipnegtwoFin_{k+1}(\ttt) & \defeq \ttt.
    \intertext{\item Finally, we define the \define{predecessor function} $\predFin_k:\Fin{k}\to\Fin{k}$ recursively by}
      \predFin_{k+1}(i(x)) & \defeq \skipnegtwoFin_k(\predFin_k(x)) \\*                       
      \predFin_{k+1}(\ttt)    & \defeq \negtwoFin_k.
    \end{align*}
  \end{enumerate}
  Show that $\predFin_k$ is an inverse to $\succFin_k$, i.e., construct identifications
  \begin{equation*}
    \succFin_k(\predFin_k(x))=x,\qquad\text{and}\qquad\predFin_k(\succFin_k(x))=x
  \end{equation*}
  for each $x:\Fin{k}$.
  \exitem \label{ex:classical-Fin}Recall that
  \begin{equation*}
    \classicalFin_k:=\sm{x:\N}x<k.
  \end{equation*}
  \begin{subexenum}
  \item Show that
    \begin{equation*}
      (x=y)\leftrightarrow (\proj 1(x)=\proj 1(y))
    \end{equation*}
    for each $x,y:\classicalFin_k$.
  \item By \cref{lem:is-bounded-natFin} it follows that the map $\natFin :\Fin{k}\to\N$ induces a map $\natFin:\Fin{k}\to\classicalFin_k$. Construct a map
    \begin{equation*}
      \alpha_k:\classicalFin_k \to \Fin{k}  
    \end{equation*}
    for each $k:\N$, and show that
  \begin{equation*}
    \alpha_k(\natFin(x)) = x \qquad\text{and}\qquad \natFin(\alpha_k(y)) = y
  \end{equation*}
  for each $x:\Fin{k}$ and each $y:\classicalFin_k$. 
  \end{subexenum}
  \exitem \label{ex:ring-Fin}The multiplication operation $x,y\mapsto xy$ on $\Z/{(k+1)}$ is defined by
  \begin{equation*}
    xy \defeq [\natFin(x)\natFin(y)]_{k+1}.
  \end{equation*}
  \begin{subexenum}
  \item Show that $\natFin(xy)\equiv\natFin(x)\natFin(y)\mod{k+1}$ for each $x,y:\Z/{(k+1)}$.
  \item \label{ex:congruence-mulN}Show that
    \begin{equation*}
      xy\equiv x'y'\mod k
    \end{equation*}
    for any $x,y,x',y':\N$ such that $x\equiv x'$ and $y\equiv y' \mod k$.
  \item Show that multiplication on $\Z/{(k+1)}$ satisfies the laws of a commutative ring:
    \begin{align*}
      (xy)z & = x(yz) & xy & = yx \\
      1x & = x & x1 & = x \\
      x(y+z) & = xy+xz & (x+y)z & = xz+yz.
    \end{align*}
  \end{subexenum}
  \exitem \label{ex:euclidean-division}(Euclidean division) Consider two natural numbers $a$ and $b$.
  \begin{subexenum}
  \item Construct two natural numbers $q$ and $r$ such that $(b\neq 0) \to (r<b)$, along with an identification
    \begin{equation*}
      a=qb+r.
    \end{equation*}
  \item Show that for any four natural numbers $q,q'$ and $r,r'$ such that the implications $(b\neq 0) \to (r<b)$ and $(b\neq 0)\to (r'<b)$ hold, and for which there are identifications
    \begin{equation*}
      a=qb+r\qquad\text{and}\qquad a=q'b+r',
    \end{equation*}
    we have $q=q'$ and $r=r'$.
  \end{subexenum}
  \exitem The type $\N_k$ of \define{$k$-ary natural numbers} is an inductive type with the following constructors:
  \begin{align*}
    \constantbasedN{k} & : \Fin{k}\to\basedN{k} \\
    \unaryopbasedN{k} & : \Fin{k}\to (\basedN{k}\to\basedN{k}).
  \end{align*}
  A $k$-ary natural number can be converted back into an ordinary natural number via the function $\convertbasedN{k}:\basedN{k}\to\N$, which is defined recursively by
  \begin{align*}
    \convertbasedN{k}(\constantbasedN{k}(x)) & \defeq \natFin(x) \\
    \convertbasedN{k}(\unaryopbasedN{k}(x,n)) & \defeq k(\convertbasedN{k}(n)+1)+\natFin(x).
  \end{align*}
  \begin{subexenum}
  \item Show that the type $\basedN{0}$ is empty.
  \item Show that the function $\convertbasedN{k}:\basedN{k}\to\N$ is injective.
  \item Show that the function $\convertbasedN{k+1}:\basedN{k+1}\to\N$ has an inverse, i.e. construct a function
    \begin{equation*}
      g_{k} : \N\to\basedN{k+1}
    \end{equation*}
    equipped with identifications
    \begin{align*}
      \convertbasedN{k+1}(g_k(n)) & = n \\
      g_{k}(\convertbasedN{k+1}(x)) & = x
    \end{align*}
    for each $n:\N$ and each $x:\basedN{k+1}$.
  \end{subexenum}
\end{exercises}
%%% Local Variables:
%%% mode: latex
%%% TeX-master: "hott-intro"
%%% End:

\section{Elementary number theory}

One of the things type theory is great for, is for the formalization of mathematics in a computer proof assistant. Those are programs that can compile any type theoretical construction to check that this construction indeed has the type it was claimed it has.

At this point in our development of type theory there are two areas of mathematics that would be natural to try to do in type theory: discrete mathematics and elementary number theory. Indeed, how does one define in type theory the greatest common divisor of two natural numbers, or how does one show that there are infinitely many primes? How does one even formalize that every non-empty subset of the natural numbers has a least element?

To answer these questions we will run into questions of decidability. How do we write a term that decides wheter a number is prime or not? Or indeed, is it even true that every non-empty subset of the natural numbers has a least element? What about the subset of $\N$ that contains $1$, and it contains $0$ if and only if Goldbach's conjecture holds? Finding the least element of this subset is equivalent to settling the conjecture!

Therefore, we will prove the well-foundedness of the natural numbers for decidable subsets of $\N$. In fact, we will show it for decidable families, because sometimes we don't know in advance whether a family of types is in fact a subtype. A consequence of involving decidability in the well-foundedness of the natural numbers is that for many properties one has to prove that they are decidable. Luckily this is the case: many of the familiar properties that one encounters in number theory are indeed decidable.

\subsection{Decidability}

A common way of reasoning in mathematics is via a proof by contradiction: ``in order to show that $P$ holds we show that it cannot be the case that $P$ doesn't hold". There are no inference rules in type theory that allow us to obtain a term of type $P$ from a term of type $\neg\neg P$. However, for some propositions $P$ one can construct a function $\neg\neg P \to P$. The \emph{decidable propositions} from a class of such propositions $P$ for which we can show $\neg\neg P \to P$.

The following definition of decidability is made for general types, even though we will mostly be interested in the decidabilyt of proposition. The reason will become aparent in a moment, when we show that types with decidable equality are sets. This useful theorem would become trivial if we restricted the notion of decidability to propositions.

\begin{defn}
  A type $A$ is said to be decidable if it comes equipped with a term of type
  \begin{equation*}
    \isdecidable(A)\defeq A+\neg A.
  \end{equation*}
  Decidable propositions are called \define{classical}. We will write
  \begin{equation*}
    \classicalprop_\UU \defeq \sm{P:\prop_\UU}\isdecidable(P)
  \end{equation*}
  for the type of all classical propositions (with respect to a universe $\UU$).
\end{defn}

\begin{eg}\label{eg:classical-prop}
  The types $\unit$ and $\emptyt$ are decidable. Indeed, we have
  \begin{align*}
    \decunit & \defeq \inl(\ttt) & & :\isdecidable(\unit) \\
    \decemptyt & \defeq \inr(\idfunc) & & : \isdecidable(\emptyt).\qedhere
  \end{align*}
  Any type $A$ equipped with a point $a:A$ is decidable.
\end{eg}

Since $P$ and $\neg P$ are mutually exclusive cases, it follows that $\isdecidable(P)$ is a proposition. Therefore we see that the type of decidable propositions in a universe $\UU$ form a subtype of the type of all propositions in $\UU$.

\begin{lem}\label{lem:isprop-isdecidable}
  For any proposition $P$, the type $\isdecidable(P)$ is a proposition.
\end{lem}

\begin{proof}
  By \cref{lem:isprop_eq} it suffices to show that
  \begin{equation*}
    \prd{t,t':\isdecidable(P)}t=t.
  \end{equation*}
  We proceed by case analysis on $t$ and $t'$. We have four cases to consider:
  \begin{align*}
    \inl(p) & =\inl(p') & \inr(f) & =\inl(p') \\
    \inl(p) & =\inr(f') & \inr(f) & =\inr(f').
  \end{align*}
  We construct these four identifications as follows:
  \begin{enumerate}
  \item First, we want to show that $\inl(p)=\inl(p')$ for any $p,p':P$. We obtain this identification from the fact that $p=p'$, which we have because $P$ is assumed to be a proposition.
  \item Next, we want to show that $\inl(p)=\inr(f')$ for any $p:P$ and $f':\neg P$. Since we have contradictory assumptions, we obtain $f'(p):\emptyt$. We now obtain the desired identification by applying the function $\emptyt \to (\inl(p)=\inr(f')$.
  \item The construction of an identification $\inr(f)=\inl(p')$ for $f:\neg P$ and $p':P$ is similar. We have $f(p'):\emptyt$, which gives the desired identification via the function $\emptyt\to (\inr(f)=\inl(p'))$.
  \item Finally, we want to show that $\inr(f)=\inr(f')$ for $f,f':\neg P$. The type $\neg P$ is a proposition, so we have an identification $f=f'$ from which we obtain $\inr(f)=\inr(f')$.\qedhere
  \end{enumerate}
\end{proof}

We have seen in \cref{thm:propositional-extensionality} that the univalence axiom implies propositional extensionality. Recall that propositional extensionality is the property that the map
\begin{equation*}
  (P=Q)\to (P\leftrightarrow Q)
\end{equation*}
is an equivalence. We will use this fact here to conclude that $\classicalprop_\UU$ is equivalent to $\bool$.

\begin{prp}
  The type of classical propositions in any universe $\UU$ is equivalent to $\bool$.%
  \index{classical-Prop_U@{$\classicalprop_\UU$}!classical-Prop_U bool@{$\classicalprop_\UU\eqvsym\bool$}}
\end{prp}

\begin{proof}
  Since the empty type and the unit type are decidable propositions, we have a map $\varphi:\bool\to\classicalprop_\UU$ defined by
  \begin{align*}
    \varphi(\btrue) & \defeq (\unit,\decunit) \\
    \varphi(\bfalse) & \defeq (\emptyt,\decemptyt).
  \end{align*}
  Next, we define a map $\psi:\classicalprop_\UU\to\bool$. Let $P$ be a proposition that comes equipped with a term $t:P+\neg P$. To define a boolean, we proceed by case analysis on $t$. The map $\psi$ is thus defined by
  \begin{align*}
    \psi(P,\inl(p)) & \defeq \btrue \\
    \psi(P,\inr(f)) & \defeq \bfalse.
  \end{align*}
  To see that $\psi$ is an inverse of $\varphi$, note that
  \begin{align*}
    \varphi(\psi(P,\inl(p))) & \jdeq (\unit,\decunit) & \psi(\varphi(\btrue)) & \jdeq \btrue \\
    \varphi(\psi(P,\inr(f))) & \jdeq (\emptyt,\decemptyt) & \psi(\varphi(\bfalse)) & \jdeq \bfalse.
  \end{align*}
  It is therefore immediate that $\psi$ is a retract of $\varphi$. However, in order to show that $\psi$ is a section of $\varphi$ we still need to show that
  \begin{align*}
    (\unit,\decunit) & = (P,\inl(p)) \\
    (\emptyt,\decemptyt) & = (P,\inr(f)).
  \end{align*}
  Since $\isdecidable(P)$ is shown to be a proposition in \cref{lem:isprop-isdecidable}, it suffices to show that
  \begin{align*}
    \unit & = P & & \text{if we have }p:P \\
    \emptyt & = P & & \text{if we have }f:\neg P.
  \end{align*}
  In both cases we proceed by propositional extensionality. Therefore we obtain the desired identifications by observing that
  \begin{align*}
    \unit & \leftrightarrow P & & \text{if we have }p:P \\
    \emptyt & \leftrightarrow P & & \text{if we have }f:\neg P.\qedhere
  \end{align*}
\end{proof}

We will now study the concept of decidable equality.

\begin{defn}
  We say that a type $A$ has \define{decidable equality} if the identity type $x=y$ is decidable for every $x,y:A$. Types with decidable equality are also called \define{discrete}.
\end{defn}

\begin{lem}
  For each $m,n:\N$, the types $\EqN(m,n)$, $m\leq n$ and $m<n$ are decidable.
\end{lem}

\begin{proof}
  The proofs in each of the three cases is similar, so we only show that $\EqN(m,n)$ is decidable for each $m,n:\N$. This is done by induction on $m$ and $n$. Note that the types
  \begin{align*}
    \EqN(\zeroN,\zeroN) & \jdeq \unit \\
    \EqN(\zeroN,\succN(n)) & \jdeq \emptyt \\
    \EqN(\succN(m),\zeroN) & \jdeq \emptyt 
  \end{align*}
  are all decidable. Moreover, the type $\EqN(\succN(m),\succN(n))\jdeq \EqN(m,n)$ is decidable by the inductive hypothesis.
\end{proof}

\begin{cor}
  Equality on the natural numbers is decidable.
\end{cor}

\begin{proof}
  Recall from the proof of \cref{thm:eq_nat} that the canonical map
  \begin{equation*}
    (m=n)\simeq \EqN(m,n)
  \end{equation*}
  is an equivalence. Thus we obtain that $(m=n)$ is decidable from the fact that $\EqN(m,n)$ is decidable.
\end{proof}

\begin{comment}
\begin{lem}
  Suppose that $A$ and $B$ are types with decidable equality. Then the coproduct $A+B$ also has decidable equality.
\end{lem}

\begin{proof}
  Our goal is to construct a dependent function
  \begin{equation*}
    d_{A+B} : \prd{z,z':A+B}\isdecidable(z=z').
  \end{equation*}
  This function is constructed by coproduct induction on both $z$ and $z'$, so we have four cases to consider. Recall from \cref{thm:id-coprod-compute} that we have equivalences
  \begin{align*}
    (\inl(x)=\inl(x')) & \simeq (x=x') \\
    (\inl(x)=\inr(y')) & \simeq \emptyt \\
    (\inr(y)=\inl(x')) & \simeq \emptyt \\
    (\inr(y)=\inr(y')) & \simeq (y=y').
  \end{align*}
  Therefore the type $z=z'$ is equivalent to a decidable type in each of the four cases.
\end{proof}

\begin{cor}
  The type $\Z$ has decidable equality.
\end{cor}

\begin{cor}
  For any $n:\N$ the type $\Fin(n)$ has decidable equality. 
\end{cor}
\end{comment}

We have already shown in \cref{thm:eq_nat} that the type of natural numbers is a set. In fact, any type with decidable equality is a set. This fact is known as Hedberg's theorem.

\begin{thm}[Hedberg]
  Any type with decidable equality is a set.
\end{thm}

\begin{proof}
  Let $A$ be a type, and let
  \begin{equation*}
    d:\prd{x,y:A}(x=y)+\neg(x=y).
  \end{equation*}
  Recall from \cref{ex:dne-dec} that $(A+\neg A)\to (\neg\neg A\to A)$ for any type $A$, so we obtain that
  \begin{equation*}
    \prd{x,y:A}\neg\neg(x=y)\to (x=y).
  \end{equation*}
  Now observe that $\neg\neg(x=y)$ is a proposition for each $x,y:A$, and that the relation $x,y\mapsto\neg\neg(x=y)$ is reflexive. Therefore we are in position to apply \cref{lem:prop_to_id} and we conclude that $A$ is a set.
\end{proof}

\subsection{The well-ordering principle for decidable families over \texorpdfstring{$\N$}{ℕ}}

\begin{defn}
  A family $P$ over a type $A$ is said to be decidable if $P(x)$ is decidable for every $x:A$. A \define{decidable subset} of a type $A$ is a map
  \begin{equation*}
    P:A\to\classicalprop.
  \end{equation*}
\end{defn}

\begin{defn}
  Let $P$ be a decidable family over $\N$, and let $n:\N$ be a natural number equipped with $p:P(n)$. We say that $n$ is a \define{minimal $P$-element} if it comes equipped with a term of type
  \begin{equation*}
    \isminimal_P(n,p)\defeq \Big(\prd{m:\N}P(m)\to (n\leq m)\Big)
  \end{equation*}
\end{defn}

Note that the type $\isminimal_P(n,p)$ doesn't depend on $p$. However, it doesn't make much sense that $n$ is a minimal element of $P$ unless we already know that $n$ is in $P$. Indeed, if we would omit the hypothesis that $n$ is in $P$, it would be more accurate to say that $n$ is a \emph{lower bound} of $P$. The following theorem is the well-ordering principle of $\N$. 

\begin{thm}
  Let $P$ be a decidable family over $\N$. Then there is a function
  \begin{equation*}
    \Big(\sm{n:\N}P(n)\Big)\to\Big(\sm{m:\N}{p:P(m)}\isminimal_P(m,p)\Big).
  \end{equation*}
\end{thm}

\begin{proof}
  Consider a universe $\UU$ that contains $P$. We show by induction on $n:\N$ that there is a function
  \begin{equation*}
    Q(n)\to \Big(\sm{m:\N}{p:Q(m)}\isminimal_Q(m,p)\Big) 
  \end{equation*}
  for every decidable family $Q:\N\to\UU$. Note that we performed a swap in the order of quantification, using the universe that contains $P$. This slightly strengthens the inductive hypothesis, which we will be able to exploit.

  The base case is trivial, since $\zeroN$ is the least natural number. For the inductive step, suppose that $Q(\succN(n))$ holds. Note that $Q(\zeroN)$ is assumed to be decidable, so we proceed by case analysis on $Q(\zeroN)+\neg Q(\zeroN)$. Given $q:Q(\zeroN)$, it follows immediately that $\zeroN$ must be minimal. In the case where $\neg Q(\zeroN)$, we consider the decidable subset $Q'$ of $\N$ given by
  \begin{equation*}
    Q'(n)\defeq Q(\succN(n)).
  \end{equation*}
  Since we have $q:Q'(n)$, we obtain a minimal element in $Q'$ by the inductive hypothesis. Of course, by the assumption that $Q(\zeroN)$ doesn't hold, the minimal element of $Q'$ is also the minimal element of $Q$.
\end{proof}

\subsection{The strong induction principle of \texorpdfstring{$\N$}{N}}

\begin{thm}
  For any type family $P$ over $\N$ there an operation
  \begin{equation*}
    \strongindN : P(\zeroN)\to\Big(\prd{n:\N}\Big(\prd{m:\N}(m\leq n)\to P(m)\Big)\to P(n+1)\Big)\to \Big(\prd{n:\N}P(n)\Big).
  \end{equation*}
  Moreover, the operation $\strongindN$ comes equipped with identifications
  \begin{align*}
    \strongindN(p_0,p_S,\zeroN) & = p_0 \\
    \strongindN(p_0,p_S,n+1) & = p_S(n,(\lam{m}\lam{p}\strongindN(p_0,p_S,m))),
  \end{align*}
  for any $p_0:P(\zeroN)$ and $p_S:\prd{n:\N}\Big(\prd{m:\N}(m\leq n)\to P(m)\Big)\to P(n+1)$.
\end{thm}

\begin{proof}
  Consider
  \begin{align*}
    p_0 & : P(\zeroN) \\
    p_S & : \prd{n:\N}\Big(\prd{m:\N}(m\leq n)\to P(m)\Big)\to P(n+1)
  \end{align*}
  
  First, we claim that there is a function
  \begin{equation*}
    \tilde{p}_0 : \prd{m:\N}(m\leq\zeroN)\to P(m)
  \end{equation*}
  that comes equipped with an identification
  \begin{equation*}
    \tilde{p}_0(\zeroN,p)=p_0
  \end{equation*}
  for any $p:\zeroN\leq\zeroN$. The fact that we have such a dependent function $\tilde{p}_0$ follows immediately by induction on $m$ and $p:m\leq \zeroN$.

  Next, we claim that there is a function
  \begin{equation*}
    \tilde{p}_S : \prd{n:\N}\Big(\prd{m:\N}(m\leq n) \to P(m)\Big)\to \Big(\prd{m:\N}(m\leq n+1)\to P(m)\Big)
  \end{equation*}
  equipped with a homotopy
  \begin{equation*}
    \prd{m:\N}\prd{q:m\leq n}{p:m\leq n+1} \tilde{p}_S(n,H,m,p) = H(m,q)
  \end{equation*}
  and an identification
  \begin{equation*}
    \tilde{p}_S(n,H,n+1,p)=p_S(n,H)
  \end{equation*}
  for every $p:n+1\leq n+1$.

  Using $\tilde{p}_0$ and $\tilde{p}_S$, we obtain by induction on $n$ a function
  \begin{equation*}
    \tilde{s}:\prd{n:\N}\prd{m:\N} (m\leq n)\to P(m)
  \end{equation*}
  satisfying the computation rules
  \begin{align*}
    \tilde{s}(\zeroN) & \jdeq \tilde{p}_0 \\
    \tilde{s}(n+1) & \jdeq \tilde{p}_S(n,\tilde{s}(n)).
  \end{align*}
  Now we define
  \begin{equation*}
    \strongindN(p_0,p_S,n) \defeq \tilde{s}(n,n,\reflleqN(n)),
  \end{equation*}
  where $\reflleqN(n):n\leq n$ is the proof of reflexivity of $\leq$.

  It remains to show that $\strongindN$ satisfies the identifications claimed in the statement of the theorem. The identification that computes $\strongindN$ at $\zeroN$ is easy to obtain:
  \begin{align*}
    \strongindN(p_0,p_S,\zeroN) & \jdeq \tilde{s}(\zeroN,\zeroN,\reflleqN(\zeroN)) \\
                                & \jdeq \tilde{p}_{0}(\zeroN,\reflleqN) \\
                                & = p_0.
  \end{align*}
  To construct the identification that computes $\strongindN$ at a successor, we start with a similar computation:
  \begin{align*}
    \strongindN(p_0,p_S,n+1) & \jdeq \tilde{s}(n+1,n+1,\reflleqN(n+1)) \\
                                   & \jdeq \tilde{p}_S(n,\tilde{s}(n),n+1,\reflleqN(n+1)) \\
    & = p_S(n,\tilde{s}(n))
  \end{align*}
  Thus we see that, in order to show that
  \begin{equation*}
    p_S(n,\tilde{s}(n))=p_S(n,(\lam{m}\lam{p}\tilde{s}(m,m,\reflleqN(m)))),
  \end{equation*}
  we need to prove that
  \begin{equation*}
    \tilde{s}(n)=\lam{m}\lam{p}\tilde{s}(m,m,\reflleqN(m)).
  \end{equation*}
  Here we apply function extensionality, so it suffices to show that
  \begin{equation*}
    \tilde{s}(n,m,p)=\tilde{s}(m,m,\reflleqN(m))
  \end{equation*}
  for every $m:\N$ and $p:m\leq n$. We proceed by induction on $n:\N$. The base case is trivial. For the inductive step, we note that
  \begin{align*}
    \tilde{s}(n+1,m,p)=\tilde{p}_S(n,\tilde{s}(n),m,p)=\begin{cases}\tilde{s}(n,m,p) & \text{if }m\leq n \\
    p_S(n,\tilde{s}(n)) & \text{if }m=n+1.\end{cases}
  \end{align*}
  Therefore it follows by the inductive hypothesis that
  \begin{equation*}
    \tilde{s}(n+1,m,p)=\tilde{s}(m,m,\reflleqN(m))
  \end{equation*}
  if $m\leq n$ holds. In the remaining case, where $m=n+1$, note that we have
  \begin{align*}
    \tilde{s}(\succN,\succN,\reflleqN(\succN)) & = \tilde{p}(n,\tilde{s}(n),n+1,\reflleqN(n+1)) \\
    & = p_S(n,\tilde{s}(n)).
  \end{align*}
  Therefore we see that we also have an identification
  \begin{equation*}
    \tilde{s}(n+1,m,p)=\tilde{s}(m,m,\reflleqN(m))
  \end{equation*}
  when $m=n+1$. This completes the proof of the strong induction principle for $\N$.
\end{proof}

\subsection{Defining the greatest common divisor}

\begin{lem}
  For any $d,n:\N$, the type $d\mid n$ is decidable.
\end{lem}

\begin{proof}
  We give the proof by case analysis on $(d=\zeroN)+(d\neq\zeroN)$. If $d=\zeroN$, then $d\mid n$ holds if and only if $\zeroN=n$, which is decidable.

  If $d\neq\zeroN$, then it follows that $n\leq nd$. Therefore we obtain by the well-ordering principle of the natural numbers a minimal $m:\N$ that satisfies the decidable property $n\leq md$. Now we observe that $d\mid n$ holds if and only if $n=md$, which is decidable.
\end{proof}

\begin{defn}
  A type family $P$ over $\N$ is said to be \define{bounded from above} by $m$ for some natural number $m$, if it comes equipped with a term of type
  \begin{equation*}
    \isbounded_m(P) \defeq \prd{n:\N}P(n)\to (n\leq m).
  \end{equation*}
\end{defn}

\begin{defn}
  Let $P$ be a type family over $\N$, and consider $p:P(n)$. We say that $n$ is the maximal $P$-number if it comes equipped with a term of type
  \begin{equation*}
    \ismaximal_P(n,p) \defeq \prd{m:\N} P(m)\to m\leq n.
  \end{equation*}
\end{defn}

In the following lemma we show that if a decidable family $P$ is bounded from above and inhabited, then it possesses a maximal element.

\begin{lem}\label{lem:maximal}
  Consider a decidable type family $P$ over $\N$ which is bounded from above by $m$. Then there is a function
  \begin{equation*}
    \maximum_P:\Big(\sm{n:\N}P(n)\Big)\to\Big(\sm{n:\N}{p:P(n)}\ismaximal_P(n,p)\Big).
  \end{equation*}
\end{lem}

\begin{proof}
  We define the asserted function by induction on $m$. In the base case, if we have $p:P(n)$, then it follows from $n\leq 0$ that $n=0$. It follows by the boundedness of $P$ that $(n,p)$ is maximal.

  In the inductive step we proceed by case analysis on $P(\succN(m))$. This is allowed because $P$ is decidable. If we have $q:P(\succN(m))$, then it follows by the boundedness of $P$ that $(\succN(m),q)$ is maximal. If $\neg P(\succN(m))$, then it follows that $P$ is bounded by $m$, which allows us to proceed by recursion.
\end{proof}

\begin{defn}
  For any two natural numbers $m,n$ we define the \define{greatest common divisor} $\gcd(m,n)$, which satisfies the following two properties:
  \begin{enumerate}
  \item We have both $\gcd(m,n)\mid m$ and $\gcd(m,n)\mid n$.
  \item For any $d:\N$ we have $d\mid \gcd(m,n)$ if and only if both $d\mid m$ and $d\mid n$ hold.
  \end{enumerate}
\end{defn}

\begin{proof}[Construction]
  Consider the type family $P(d)\defeq (d\mid m)\times (d\mid n)$. Then $P$ is bounded from above by $m$. Moreover, $P(1)$ holds since $1\mid n$ for any natural number $n$. Furthermore, the divisibility relation is decidable, so it follows that $P$ is a family of decidable types. Now the greatest common divisor is defined as the maximal $P$-element, which is obtained by \cref{lem:maximal}
\end{proof}

\subsection{The Euclidean algorithm}

It was immediate from our definition of the greatest common divisor of $a$ and $b$ that it indeed divides both $a$ and $b$, and that it is the greatest such number. However, as a program that is supposed to \emph{compute} the greatest common divisor of $a$ and $b$ it performs rather poorly: it checks for every $n$ from $1$ until either $a$ or $b$ whether it is a divisor of both $a$ and $b$, and only then it gives as output the largest common divisor that it has found. In this section we give a new definition of an operation
\begin{equation*}
  \gcdeuclid:\N \to (\N \to \N)
\end{equation*}
following Euclid's algorithm, with the opposite qualities: it will compute rather quicky a value for $\gcdeuclid(a,b)$, but it will be left as something to show that this value is indeed the greatest common divisor of $a$ and $b$.

\begin{defn}
  We define a binary operation
  \begin{equation*}
    \gcdeuclid:\N \to (\N\to\N).
  \end{equation*}
\end{defn}

\begin{proof}
  We will define the operation $\gcdeuclid$ with the \emph{strong} induction principle for $\N$, which was given as \cref{ex:strong-induction}. Thus it suffices to construct a function $\N\to\N$, which will provide the values for $\gcdeuclid(\zeroN)$, and a function
  \begin{equation*}
    h_a:\Big(\prd{x:\N}(x\leq a) \to \N\to\N\Big)\to (\N\to\N),
  \end{equation*}
  for every $a:\N$, which will provide the values for $\gcdeuclid(a+1)$.

  In the base case, we simply define
  \begin{equation*}
    \gcdeuclid(\zeroN)\defeq\idfunc.
  \end{equation*}
  For the inductive step, consider a family of maps $F_x:\N\to\N$ indexed by $x\leq a$. We think of $F_x(b)$ as the value for $\gcdeuclid(x,b)$, so our assumption of having such a family of maps $F_x$ is really the assumption that $\gcdeuclid(x,b)$ is defined for every $x\leq a$. Our goal is to construct a map
  \begin{equation*}
    \gcdeuclid(a+1):\N\to\N
  \end{equation*}
  We proceed by strong induction on $b:\N$. In the base case, we define
  \begin{equation*}
    \gcdeuclid(a+1,\zeroN)\defeq a+1.
  \end{equation*}
  For the inductive step, assume that we have a number $G_y:\N$ for every $y\leq b$. Observe that $(b\leq a)+(a<b)$ holds for any $b:B$, see \cref{ex:order_N}. Thus we can proceed by case analysis to define
  \begin{equation*}
    h_a(F,b+1)\defeq
    \begin{cases}
      F_{(a+1)-(b+1)}(b+1) & \text{if }b\leq a\\
      G_{(b+1)-(a+1)} & \text{if }a<b.
    \end{cases}
  \end{equation*}
  This completes the inductive step, and hence we obtain a binary operation
  $\gcdeuclid$ that satisfies
  \begin{align*}
    \gcdeuclid(\zeroN,b) & \jdeq b \\
    \gcdeuclid(a+1,\zeroN) & \jdeq a+1 \\
    \gcdeuclid(a+1,b+1) & \jdeq \gcdeuclid((a+1)-(b+1),b+1) & & \text{if }b\leq a.\\
    \gcdeuclid(succN(a),b+1) & \jdeq \gcdeuclid(a+1,(b+1)-(a+1)) & & \text{if }a<b.\qedhere            
  \end{align*}
\end{proof}

\begin{prp}
  For each $a,b:\N$, the number $\gcdeuclid(a,b)$ is the greatest common divisor of $a$ and $b$.
\end{prp}


\subsection{The trial division primality test}

\begin{thm}
  For any $n:\N$, the proposition $\isprime(n)$ is decidable.
\end{thm}

It is important to note that, even when we prove that a type such as $\isprime(n)$ is decidable, it is only after we \emph{evaluate} the proof term that we know whether the type under consideration has a term or not. In other words, for any given $n$ we don't know right away whether it is prime or not. Evaluating whether $n$ is prime can be computationally costly, so it may be desirable in any specific situation to give a separate mathematical \emph{argument} that decides whether or not the number is prime.

\subsection{Prime decomposition}

We will show now that any natural number $n>0$ can be written as a product of primes
\begin{equation*}
  n=p_1^{k_1}\cdots p_{m}^{k_m}
\end{equation*}
This prime decomposition is unique if we require that the primes $p_i<p_{i+1}$ for each $0<i<m$. In order to establish these facts in type theory, we first have to define finite products.

\subsection{The infinitude of primes}

\begin{thm}
  There are infinitely many primes.
\end{thm}

\begin{proof}
  We will show that for every $n:\N$ there is a prime number that is larger than $n$. In other words, we will construct a term of type
  \begin{equation*}
    \prd{n:\N}\sm{p:\N}\isprime(p)\times (n\leq p).
  \end{equation*}
  Note that the number $n!+1$ is relatively prime to any number $m\leq n$. Therefore the primes in its prime factorization must all be larger than $n$. Thus, the function that assigns to $n$ the least prime factor of $n!+1$ shows that for any $n:\N$ there is a prime number $p$ that is larger than $n$.
\end{proof}

\begin{cor}
  There is a function
  \begin{equation*}
    \primetype : \N \to \sm{p:\N}\isprime(p)
  \end{equation*}
  that sends $n$ to the $n$-th prime. This function is strictly monotone, so it is an embedding.
\end{cor}

\begin{exercises}
  \exercise Show that for any $f:\Fin(m)\to\Fin(n)$ and any $i:\Fin(n)$, the type $\fib{f}{i}$ is decidable.
  \exercise Consider a decidable type $P(i)$ indexed by $i:\Fin(n)$.
  \begin{subexenum}
  \item Show that the type
    \begin{equation*}
      \prd{i:\Fin(n)}P(i)
    \end{equation*}
    is decidable.
  \item Show that the type
    \begin{equation*}
      \sm{i:\Fin(n)}P(i)
    \end{equation*}
    is decidable.
  \end{subexenum}
  \exercise
  \begin{subexenum}
  \item Show that $\N$ and $\bool$ have decidable equality. Hint: to show that $\N$ has decidable equality, show first that the successor function is injective.
  \item Show that if $A$ and $B$ have decidable equality, then so do $A+B$ and $A\times B$. Conclude that $\Z$ has decidable equality.
  \item Show that if $A$ is a retract of a type $B$ with decidable equality, then $A$ also has decidable equality.
  \end{subexenum}
  \exercise Define the prime-counting function $\pi:\N\to\N$.
  \exercise (The Cantor-Schr\"oder-Bernstein theorem) Let $X$ and $Y$ be two sets with decidable equality, and consider two maps $f:X\to Y$ and $g:Y\to X$, both of which we assume to be injective. Construct an equivalence $X\simeq Y$.
  \exercise For any $k:\Z$, define a function $i\mapsto i+k \mod n$ of type $\Fin(n)\to\Fin(n)$. Show that this function is an equivalence.
  \exercise For any $k:\Z$, define a function $i\mapsto i\cdot k \mod n$ of type $\Fin(n)\to\Fin(n)$. Show that this function is an equivalence if and only if $\gcd(n,k)=1$.
  \exercise Show that
  \begin{equation*}
    \sum_{i=0}^n \binom{n-i}{i}=F_{n+1}
  \end{equation*}
  \exercise Show that if $2^n-1$ is prime, then $n$ is prime.
  \exercise Prove Fermat's little theorem.
  \exercise Extend the definition of the greatest common divisor to all integers.
  \exercise Show that
  \begin{equation*}
    (\Fin(m)\simeq\Fin(n))\leftrightarrow(m=n).
  \end{equation*}
  \exercise Show that $\N$ satisfies \define{ordinal induction}, i.e., construct for any type family $P$ over $\N$ a function of type
  \begin{equation*}
    \ordindN : \Big(\prd{k:\N} \Big(\prd{m:\N} (m< k) \to P(m)\Big)\to P(k)\Big) \to \prd{n:\N}P(n).
  \end{equation*}
  Moreover, prove that
  \begin{equation*}
    \ordindN(h,n)=h(n,\lam{m}\lam{p}\ordindN(h,m))
  \end{equation*}
  for any $n:\N$ and any $h:\prd{k:\N}\Big(\prd{m:\N}(m<k)\to P(m)\Big)\to P(k)$.
  \exercise
  \begin{subexenum}
  \item Show that if $A$ and $B$ have decidable equality, then so do the types $A+B$ and $A\times B$.
  \item Show that $\Z$ and $\Fin(n)$ have decidable equality, for every $n:\N$.
  \end{subexenum}
  \exercise Let $P:\N\to\classicalprop$ be a decidable subset of $\N$.
  \begin{subexenum}
  \item Show that $\sm{m:\N}{p:P(m)}\isminimal_P(m,p)$ is a proposition.
  \item Show that the map
    \begin{equation*}
      \Big(\sm{n:\N}P(n)\Big)\to\Big(\sm{m:\N}{p:P(m)}\isminimal_P(m,p)\Big)
    \end{equation*}
    is a propositional truncation.
  \end{subexenum}
  \exercise Suppose that $A:I\to \UU$ is a type family over a set $I$ with decidable equality. Show that
  \begin{equation*}
    \Big(\prd{i:I}\iscontr(A_i)\Big)\leftrightarrow \iscontr\Big(\prd{i:I}A_i\Big).
  \end{equation*}
\end{exercises}

%\input{integers}

\cleardoublepage
%%% Local Variables:
%%% mode: latex
%%% TeX-master: "hott-intro"
%%% End:
