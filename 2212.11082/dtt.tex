\section{Dependent type theory}%
\label{sec:dtt}%
\index{dependent type theory|(}%

Dependent type theory is a system of inference rules that can be combined to make \emph{derivations}. In these derivations, the goal is often to construct an element of a certain type. Such an element can be a function if the type of the constructed element is a function type; a proof of a property if the type of the constructed element is a proposition; but it can also be an identification if the type of the constructed element is an identity type, and so on. In some respect, a type is just a collection of mathematical objects and constructing elements of a type is the everyday mathematical task or challenge. The system of inference rules that we call type theory offers a principled way of engaging in mathematical activity.

\subsection{Judgments and contexts in type theory}%
\index{judgment|(}%
\index{context|(}%

A mathematical argument or construction consists of a sequence of deductive steps, each one using finitely many premises in order to get to the next stage in the proof or construction. Such steps can be represented by \define{inference rules}\index{inference rule|see {rule}}, which are written in the form
\begin{prooftree}
  \AxiomC{$\mathcal{H}_1$\quad $\mathcal{H}_2$ \quad \dots \quad $\mathcal{H}_n$}
  \UnaryInfC{$\mathcal{C}$.}
\end{prooftree}
Inference rules contain above the horizontal line\index{horizontal line|see {inference rule}} a finite list $\mathcal{H}_1$, $\mathcal{H}_2$, \dots, $\mathcal{H}_n$ of \emph{judgments} for the \define{premises}\index{inference rule!premises|textbf}\index{premise of an inference rule|textbf}, and below the horizontal line a single judgment $\mathcal{C}$ for the \define{conclusion}\index{inference rule!conclusion|textbf}\index{conclusion of an inference rule|textbf}. The system of dependent type theory is described by a set of such inference rules.

A straightforward example of an inference rule that we will encounter in \cref{sec:pi} when we introduce function types\index{function type}, is the inference rule
\begin{prooftree}
  \AxiomC{$\Gamma\vdash a:A$}
  \AxiomC{$\Gamma\vdash f:A\to B$}
  \BinaryInfC{$\Gamma\vdash f(a):B$.}
\end{prooftree}
This rule asserts that in any context $\Gamma$ we may use an element $a:A$ and a function $f:A\to B$ to obtain an element $f(a):B$. Each of the expressions
\begin{align*}
  \Gamma & \vdash a :A \\*
  \Gamma & \vdash f : A \to B \\*
  \Gamma & \vdash f(a):B
\end{align*}
are examples of judgments.

\begin{defn}\label{defn:judgments}
  There are four kinds of \define{judgments} in Martin-L\"of's dependent type theory:
  \begin{enumerate}
  \item \emph{$A$ is a (well-formed) \define{type} in context $\Gamma$.}
    \index{well-formed type}\index{type}
    We express this judgment as\index{Gamma turnstile A type@{$\Gamma\vdash A~\type$}}\index{judgment!Gamma turnstile A type@{$\Gamma\vdash A~\type$}}
    \begin{equation*}
      \Gamma\vdash A~\type.
    \end{equation*}
  \item \emph{$A$ and $B$ are \define{judgmentally equal types} in context $\Gamma$.}
    \index{judgmental equality!of types} We express this judgment as\index{Gamma turnstile A is B type@{$\Gamma\vdash A\jdeq B~\type$}}\index{judgment!Gamma turnstile A is B type@{$\Gamma\vdash A\jdeq B~\type$}}
    \begin{equation*}
      \Gamma\vdash A \jdeq B~\type.
    \end{equation*}
  \item \emph{$a$ is an \define{element}\index{element|textbf} of type $A$ in context $\Gamma$.} We express this judgment as\index{Gamma turnstile a in A@{$\Gamma\vdash a:A$}}\index{judgment!Gamma turnstile a in A@{$\Gamma\vdash a:A$}}
    \begin{equation*}
      \Gamma \vdash a:A.
    \end{equation*}
  \item \emph{$a$ and $b$ are \define{judgmentally equal elements} of type $A$ in context $\Gamma$.}\index{judgmental equality!of elements} We express this judgment as\index{Gamma turnstile a is b in A@{$\Gamma\vdash a\jdeq b:A$}}\index{judgment!Gamma turnstile a is b in A@{$\Gamma\vdash a\jdeq b:A$}}
    \begin{equation*}
      \Gamma\vdash a\jdeq b:A.
    \end{equation*}
  \end{enumerate}
\end{defn}

We see that any judgment is of the form $\Gamma\vdash\mathcal{J}$, consisting of a \emph{context} $\Gamma$ and a \emph{judgment thesis} $\mathcal{J}$ asserting either that $A$ is a type, that $A$ and $B$ are equal types, that $a$ is an element of type $A$, or that $a$ and $b$ are equal elements of type $A$. The role of a context is to declare what \define{hypothetical elements}\index{hypothetical elements|textbf} are assumed, along with their types. Hypothetical elements are commonly called \define{variables}\index{variable|textbf}.

\begin{defn}\label{defn:context}
  A \define{context}\index{context|textbf} is a finite list of \define{variable declarations}\index{variable declaration|textbf}
\begin{equation}\label{eq:context}
x_1:A_1,~x_2:A_2(x_1),~\ldots,~x_n:A_n(x_1,\ldots,x_{n-1})
\end{equation}
satisfying the condition that for each $1\leq k\leq n$ we can derive the judgment
\begin{equation*}
  x_1:A_1,~\ldots,~x_{k-1}:A_{k-1}(x_1,\ldots,x_{k-2})\vdash A_k(x_1,\ldots,x_{k-1})~\type,
\end{equation*}
using the inference rules of type theory. We may use variable names other than $x_1,\ldots,x_n$, as long as no variable is declared more than once.
\end{defn}

The condition in \cref{defn:context} that each of the hypothetical elements is assigned a type, is checked recursively. In other words, to check that a list of variable declarations as in \cref{eq:context} is a context, one starts on the left and works their way to the right, verifying that each hypothetical elements $x_k$ is assigned a type. 

Note that there is a context of length $0$, the \define{empty context}\index{context!empty context|textbf}\index{empty context|textbf}, which declares no variables. This context satisfies the requirement in \cref{defn:context} vacuously. A list of variable declarations $x_1:A_1$ of length one is a context if and only if $A_1$ is a type in the empty context. We will soon encounter the type $\N$ of natural numbers\index{natural numbers}, which is an example of a type in the empty context.

The next case is that a list of variable declarations $x_1:A_1,~x_2:A_2(x_1)$ of length two is a context if and only if $A_1$ is a type in the empty context, and $A_2(x_1)$ is a type in context $x_1:A_1$. This process repeats itself for longer contexts.
\index{judgment|)}
\index{context|)}

\subsection{Type families}
\index{type family|(}
It is a feature of \emph{dependent} type theory that all judgments are context dependent, and indeed that even the types of the variables in a context may depend on any previously declared variables. For example, if $n$ is a natural number and we know from the context that $n$ is prime, then we don't have enough information yet to decide whether or not $n$ is odd. However, if we also know from the context that $n+2$ is prime, then we can derive that $n$ must be odd. Context dependency is everywhere -- not only in mathematics, but also in language and in everyday life -- and it gives rise to the notion of \emph{type families} and their \emph{sections}.

\begin{defn}
  Consider a type $A$ in context $\Gamma$. A \define{family}\index{family of types|see{type family}}\index{type family|textbf}\index{family of types|textbf} of types over $A$ in context $\Gamma$ is a type $B(x)$ in context $\Gamma,x:A$. In other words, in the situation where
\begin{equation*}
  \Gamma,~x:A\vdash B(x)~\type,
\end{equation*}
we say that $B$ is a family of types over $A$ in context $\Gamma$. Alternatively, we say that $B(x)$ is a type \define{indexed}\index{indexed type|textbf}\index{type!indexed type|textbf} by $x:A$, in context $\Gamma$.
\end{defn}

We think of a type family $B$ over $A$ in context $\Gamma$ as a type $B(x)$ varying along $x:A$. A basic example of a type family occurs when we introduce \emph{identity types}\index{identity type} in \cref{sec:identity}. They are introduced as follows:
\begin{prooftree}
  \AxiomC{$\Gamma\vdash a:A$}
  \UnaryInfC{$\Gamma,~x:A\vdash a=x~\type$.}
\end{prooftree}
This rule asserts that given an element $a:A$ in context $\Gamma$, we may form the type $a=x$ in context $\Gamma,~x:A$. The type $a=x$ in context $\Gamma,~x:A$ is an example of a type family over $A$ in context $\Gamma$.

\begin{defn}
Consider a type family $B$ over $A$ in context $\Gamma$. A \define{section}\index{section of a type family} of the family $B$ over $A$ in context $\Gamma$ is an element of type $B(x)$ in context $\Gamma,x:A$, i.e., in the judgment
\begin{equation*}
  \Gamma,~x:A\vdash b(x):B(x)
\end{equation*}
we say that $b$ is a section of the family $B$ over $A$ in context $\Gamma$. Alternatively, we say that $b(x)$ is an element of type $B(x)$ \define{indexed}\index{indexed element|textbf}\index{element!indexed element|textbf} by $x:A$ in context $\Gamma$.
\end{defn}

Note that in the above situations $A$, $B$, and $b$ also depend on the variables declared in the context $\Gamma$, even though we have not explicitly mentioned them. It is indeed common practice to not mention every variable in the context $\Gamma$ in such situations.
\index{type family|)}

\subsection{Inference rules}\label{sec:rules}

We are now ready to present the system of inference rules that underlies dependent type theory. These rules are known as the \define{structural rules} of type theory, since they establish the basic mathematical framework for type dependency. There are six sets of inference rules:
\begin{enumerate}
\item Rules about the formation of contexts, types, and their elements
\item Rules postulating that judgmental equality is an equivalence relation.
\item Variable conversion rules.
\item Substitution rules.
\item Weakening rules.
\item The generic element.
\end{enumerate}

\subsubsection*{Rules about the formation of contexts, types, and their elements}
In the definition of well-formed contexts, types, and elements we specified that for a type $B(x)$ to be well-formed in context $\Gamma,x:A$, it must be the case that $A$ is a well-formed type in context $\Gamma$. The following rules follow from the presuppositions about contexts, types, and their elements, and may be used freely in derivations:

\begin{center}
  \begin{minipage}{.35\textwidth}
    \begin{prooftree}
      \AxiomC{$\Gamma,x:A\vdash B(x)~\type$}
      \UnaryInfC{$\Gamma\vdash A~\type$}
    \end{prooftree}
    
    \begin{prooftree}
      \AxiomC{$\Gamma\vdash a:A$}
      \UnaryInfC{$\Gamma\vdash A~\type$}
    \end{prooftree}
  \end{minipage}
  \begin{minipage}{.25\textwidth}
    \begin{prooftree}
      \AxiomC{$\Gamma\vdash A\jdeq B~\type$}
      \UnaryInfC{$\Gamma\vdash A~\type$}
    \end{prooftree}
    
    \begin{prooftree}
      \AxiomC{$\Gamma\vdash a\jdeq b:A$}
      \UnaryInfC{$\Gamma\vdash a:A$}
    \end{prooftree}
  \end{minipage}
  \begin{minipage}{.25\textwidth}
    \begin{prooftree}
      \AxiomC{$\Gamma\vdash A\jdeq B~\type$}
      \UnaryInfC{$\Gamma\vdash B~\type$}
    \end{prooftree}
    
    \begin{prooftree}
      \AxiomC{$\Gamma\vdash a\jdeq b:A$}
      \UnaryInfC{$\Gamma\vdash b:A$}
    \end{prooftree}
  \end{minipage}
\end{center}

\subsubsection*{Judgmental equality is an equivalence relation}

\index{rules!for type dependency!judgmental equality is an equivalence relation|(}
The rules postulating that judgmental equality on types and on elements is an equivalence relation simply postulate that these relations are reflexive, symmetric, and transitive\index{judgmental equality!is an equivalence relation}:
\begin{center}
  \begin{small}
    \begin{minipage}{.22\textwidth}
      \begin{center}
        \begin{prooftree}
          \AxiomC{$\Gamma\vdash A~\textrm{type}$}
          \UnaryInfC{$\Gamma\vdash A\jdeq A~\textrm{type}$}
        \end{prooftree}
      \end{center}
    \end{minipage}
    \begin{minipage}{.28\textwidth}
      \begin{center}
        \begin{prooftree}
          \AxiomC{$\Gamma\vdash A\jdeq B~\textrm{type}$}
          \UnaryInfC{$\Gamma\vdash B\jdeq A~\textrm{type}$}
        \end{prooftree}
      \end{center}
    \end{minipage}
    \begin{minipage}{.48\textwidth}
      \begin{prooftree}
        \AxiomC{$\Gamma\vdash A\jdeq B~\textrm{type}$}
        \AxiomC{$\Gamma\vdash B\jdeq C~\textrm{type}$}
        \BinaryInfC{$\Gamma\vdash A\jdeq C~\textrm{type}$}
      \end{prooftree}
    \end{minipage}
    \\
    \bigskip
    \begin{minipage}{.22\textwidth}
      \begin{prooftree}
        \AxiomC{$\Gamma\vdash a:A$}
        \UnaryInfC{$\Gamma\vdash a\jdeq a : A$}
      \end{prooftree}
    \end{minipage}
    \begin{minipage}{.28\textwidth}
      \begin{prooftree}
        \AxiomC{$\Gamma\vdash a\jdeq b:A$}
        \UnaryInfC{$\Gamma\vdash b\jdeq a: A$}
      \end{prooftree}
    \end{minipage}
    \begin{minipage}{.48\textwidth}
      \begin{prooftree}
        \AxiomC{$\Gamma\vdash a\jdeq b : A$}
        \AxiomC{$\Gamma\vdash b\jdeq c: A$}
        \BinaryInfC{$\Gamma\vdash a\jdeq c: A$.}
      \end{prooftree}
    \end{minipage}
  \end{small}
\end{center}
\index{rules!for type dependency!judgmental equality is an equivalence relation|)}

\subsubsection*{Variable conversion rules}
\index{rules!for type dependency!variable conversion|(}
The \define{variable conversion rules}\index{judgmental equality!conversion rules}\index{variable conversion rules|textbf}\index{conversion rule!variable|textbf}\index{rules!for type dependency!variable conversion|textbf} are rules postulating that we can convert the type of a variable to a judgmentally equal type. The first variable conversion rule states that
\begin{prooftree}
\AxiomC{$\Gamma\vdash A\jdeq A'~\textrm{type}$}
\AxiomC{$\Gamma,~x:A,~\Delta\vdash B(x)~\type$}
\BinaryInfC{$\Gamma,~x:A',~\Delta\vdash B(x)~\type$.}
\end{prooftree}
In this conversion rule, the context $\Gamma,~x:A,~\Delta$ is just any extension of the context $\Gamma,~x:A$, i.e., it is a context of the form
\begin{equation*}
  x_1:A_1,~\ldots,~x_{n-1}:A_{n-1},~x:A,~x_{n+1}:A_{n+1},~\ldots,~x_{n+m}:A_{n+m}.
\end{equation*}

Similarly, there are variable conversion rules for judgmental equality of types, for elements, and for judgmental equality of elements. To avoid having to state essentially the same rule four times, we state all four variable conversion rules at once using a \emph{generic judgment thesis} $\mathcal{J}$, which can be any of the four kinds described in \cref{defn:judgments}:
\begin{prooftree}
\AxiomC{$\Gamma\vdash A\jdeq A'~\textrm{type}$}
\AxiomC{$\Gamma,~x:A,~\Delta\vdash \mathcal{J}$}
\BinaryInfC{$\Gamma,~x:A',~\Delta\vdash \mathcal{J}$.}
\end{prooftree}
An analogous \emph{element conversion rule}, stated in \cref{ex:term_conversion}, converting the type of an element to a judgmentally equal type, is derivable using the rules from the rules presented in this section.
\index{rules!for type dependency!variable conversion|)}

\subsubsection*{Substitution}
\index{substitution|(}\index{rules!for type dependency!rules for substitution|(}

Consider an element $f(x):B(x)$ indexed by $x:A$ in context $\Gamma$, and suppose we also have an element $a:A$. Then we can simultaneously substitute $a$ for all occurrences of $x$ in $f(x)$ to obtain a new element $f[a/x]$, which has type $B[a/x]$. A precise definition of substitution requires us to get too deep into the theory of the syntax of type theory, but a mathematician is of course no stranger to substitution. For example, substituting $0$ for $x$ in the polynomial
\begin{equation*}
  1+x+x^2+x^3
\end{equation*}
results in the number $1+0+0^2+0^3$, which can be computed to the value $1$.

Type theoretic substitution is similar. Type theoretic substitution is in fact a bit more general than what we have described above. Suppose we have a type
\begin{equation*}
  \Gamma,~x:A,~y_{1}:B_{1},~\ldots,~y_{n}:B_{n}\vdash C~\textrm{type}
\end{equation*}
and an element $a:A$ in context $\Gamma$. Then we can simultaneously substitute $a$ for all occurrences of $x$ in the types $B_1,\ldots,B_n$ and $C$, to obtain
\begin{equation*}
  \Gamma,~y_{1}:B_{1}[a/x],~\ldots,~y_{n}:B_{n}[a/x]\vdash C[a/x]~\mathrm{type}.
\end{equation*}
Note that the variables $y_{1},~\ldots,y_{n}$ are assigned new types after performing the substitution of $a$ for $x$. Similarly, we can substitute $a$ for $x$ in an element $c:C$ to obtain the element $c[a/x]:C[a/x]$, and we can substitute $a$ for $x$ in a judgmental equality thesis, either of types or elements, by simply substituting on both sides of the equation. The \define{substitution rule} are therefore stated using a generic judgment $\mathcal{J}$:
\begin{prooftree}
\AxiomC{$\Gamma\vdash a:A$}
\AxiomC{$\Gamma,~x:A,~\Delta\vdash \mathcal{J}$}
\RightLabel{$S$.}
\BinaryInfC{$\Gamma,~\Delta[a/x]\vdash \mathcal{J}[a/x]$}
\end{prooftree}
Furthermore, we add two more `congruence rules' for substitution, postulating that substitution by judgmentally equal elements results in judgmentally equal types and elements:
\begin{prooftree}
\AxiomC{$\Gamma\vdash a\jdeq a':A$}
\AxiomC{$\Gamma,~x:A,~\Delta\vdash B~\type$}
\BinaryInfC{$\Gamma,~\Delta[a/x]\vdash B[a/x]\jdeq B[a'/x]~\type$}
\end{prooftree}
\begin{prooftree}
\AxiomC{$\Gamma\vdash a\jdeq a':A$}
\AxiomC{$\Gamma,~x:A,~\Delta\vdash b:B$}
\BinaryInfC{$\Gamma,~\Delta[a/x]\vdash b[a/x]\jdeq b[a'/x]:B[a/x]$.}
\end{prooftree}
To see that these rules make sense, we observe that both $B[a/x]$ and $B[a'/x]$ are types in context $\Delta[a/x]$, provided that $a\jdeq a'$. This is immediate by recursion on the length of $\Delta$.

\begin{defn}
  When $B$ is a family of types over $A$ in context $\Gamma$, and if we have $a:A$, then we also say that $B[a/x]$ is the \define{fiber}\index{type family!fiber of a type family|textbf}\index{fiber of a type family|textbf} of $B$ at $a$. We will usually write $B(a)$ for the fiber of $B$ at $a$.

  When $b$ is a section of the family $B$ over $A$ in context $\Gamma$, we call the element $b[a/x]$ the \define{value} of $b$ at $a$. Again, we will usually write $b(a)$ for the value of $b$ at $a$.
\end{defn}
\index{substitution|)}\index{rules!for type dependency!rules for substitution|)}

\subsubsection*{Weakening}
\index{weakening|(}\index{rules!for type dependency!rules for weakening|(}
If we are given a type $A$ in context $\Gamma$, then any judgment made in a longer context $\Gamma,~\Delta$ can also be made in the context $\Gamma,~x:A,~\Delta$, for a fresh variable $x$. The \define{weakening rule}\index{weakening} asserts that weakening by a type $A$ in context preserves well-formedness and judgmental equality of types and elements.
\begin{prooftree}
\AxiomC{$\Gamma\vdash A~\textrm{type}$}
\AxiomC{$\Gamma,~\Delta\vdash \mathcal{J}$}
\RightLabel{$W$.}
\BinaryInfC{$\Gamma,~x:A,~\Delta \vdash \mathcal{J}$}
\end{prooftree}
This process of expanding the context by a fresh variable of type $A$ is called \define{weakening} (by $A$).

In the simplest situation where weakening applies, we have two types $A$ and $B$ in context $\Gamma$. Then we can weaken $B$ by $A$ as follows
\begin{prooftree}
  \AxiomC{$\Gamma\vdash A~\textrm{type}$}
  \AxiomC{$\Gamma\vdash B~\textrm{type}$}
  \RightLabel{$W$}
  \BinaryInfC{$\Gamma,~x:A\vdash B~\type$}
\end{prooftree}
in order to form the type $B$ in context $\Gamma,~x:A$. The type $B$ in context $\Gamma,~x:A$ is called the \define{constant family}\index{type family!constant family|textbf}\index{constant family|textbf} $B$, or the \define{trivial family}\index{type family!trivial family|textbf}\index{trivial family|textbf} $B$.
\index{weakening|)}\index{rules!for type dependency!rules for weakening|)}

\subsubsection*{The generic elements}
If we are given a type $A$ in context $\Gamma$, then we can weaken $A$ by itself to obtain that $A$ is a type in context $\Gamma,~x:A$. The rule for the \define{generic element}\index{generic element|textbf}\index{rules!for type dependency!generic element|textbf} now asserts that any hypothetical element $x:A$ in the context $\Gamma,~x:A$ is also an element of type $A$ in context $\Gamma,~x:A$.
\begin{prooftree}
\AxiomC{$\Gamma\vdash A~\textrm{type}$}
\RightLabel{$\delta$.}
\UnaryInfC{$\Gamma,~x:A\vdash x:A$}
\end{prooftree}
This rule is also known as the \define{variable rule}\index{variable rule|textbf}\index{rules!for type dependency!variable rule|textbf}. One of the reasons for including the generic element is to make sure that the variables declared in a context---i.e., the hypothetical elements---are indeed \emph{elements}. It also provides the \emph{identity function}\index{identity function} on the type $A$ in context $\Gamma$.

\subsection{Derivations}\label{sec:derivations}

\index{derivation|(}
A \define{derivation}\index{derivation|textbf} in type theory is a finite tree in which each node is a valid rule of inference. At the root of the tree we find the conclusion, and in the leaves of the tree we find the hypotheses. We give two examples of derivations: a derivation showing that any variable can be changed to a fresh one, and a derivation showing that any two variables that do not mutually depend on one another can be swapped in order.

Given a derivation with hypotheses $\mathcal{H}_1,\ldots,\mathcal{H}_n$ and conclusion $\mathcal{C}$, we can form a new inference rule
\begin{prooftree}
  \AxiomC{$\mathcal{H}_1$}
  \AxiomC{$\cdots$}
  \AxiomC{$\mathcal{H}_n$}
  \TrinaryInfC{$\mathcal{C}$.}
\end{prooftree}
Such a rule is called \define{derivable}, because we have a derivation for it. In order to keep proof trees reasonably short and manageable, we use the convention that any derived rules can be used in future derivations.

\subsubsection*{Changing variables}

\index{change of variables}
Variables can always be changed to fresh variables. We show that this is the case by showing that the inference rule\index{rules!for type dependency!change of variables}
\begin{prooftree}
  \AxiomC{$\Gamma,~x:A,~\Delta\vdash \mathcal{J}$}
  \RightLabel{$x'/x$}
  \UnaryInfC{$\Gamma,~x':A,~\Delta[x'/x]\vdash \mathcal{J}[x'/x]$}
\end{prooftree}
is derivable, where $x'$ is a variable that does not occur in the context $\Gamma,~x:A,~\Delta$. 

Indeed, we have the following derivation using substitution, weakening, and the generic element:
\begin{prooftree}
  \AxiomC{$\Gamma\vdash A~\type$}
  \RightLabel{$\delta$}
  \UnaryInfC{$\Gamma,~x':A\vdash x':A$}
  \AxiomC{$\Gamma\vdash A~\type$}
  \AxiomC{$\Gamma,~x:A,~\Delta\vdash \mathcal{J}$}
  \RightLabel{$W$}
  \BinaryInfC{$\Gamma,~x':A,~x:A,~\Delta\vdash \mathcal{J}$}
  \RightLabel{$S$.}
  \BinaryInfC{$\Gamma,~x':A,~\Delta[x'/x]\vdash \mathcal{J}[x'/x]$}
\end{prooftree}
In this derivation it is the application of the weakening rule where we have to check that $x'$ does not occur in the context $\Gamma,~x:A,~\Delta$.

\subsubsection*{Interchanging variables}

The \define{interchange rule}\index{rules!for type dependency!interchange}\index{interchange rule} states that if we have two types $A$ and $B$ in context $\Gamma$, and we make a judgment in context $\Gamma,~x:A,~y:B,~\Delta$, then we can make that same judgment in context $\Gamma,~y:B,~x:A,~\Delta$ where the order of $x:A$ and $y:B$ is swapped. More formally, the interchange rule is the following inference rule
\begin{prooftree}
\AxiomC{$\Gamma\vdash B~\textrm{type}$}
\AxiomC{$\Gamma,~x:A,~y:B,~\Delta\vdash \mathcal{J}$}
\BinaryInfC{$\Gamma,~y:B,~x:A,~\Delta\vdash \mathcal{J}$.}
\end{prooftree}
Just as the rule for changing variables, we claim that the interchange rule is a derivable rule.

The idea of the derivation for the interchange rule is as follows: If we have a judgment
\begin{equation*}
  \Gamma,~x:A,~y:B,~\Delta\vdash\mathcal{J},
\end{equation*}
then we can change the variable $y$ to a fresh variable $y'$ and weaken the judgment to obtain the judgment
\begin{equation*}
  \Gamma,~y:B,~x:A,~y':B,~\Delta[y'/y]\vdash\mathcal{J}[y'/y].
\end{equation*}
Now we can substitute $y$ for $y'$ to obtain the desired judgment $\Gamma,~y:B,~x:A,~\Delta\vdash\mathcal{J}$. The formal derivation is as follows:
\begin{small}
  \begin{prooftree}
    \AxiomC{$\Gamma\vdash B~\textrm{type}$}
    %\RightLabel{$\delta$}
    \UnaryInfC{$\Gamma,~y:B\vdash y:B$}
    %\RightLabel{$W$} 
    \UnaryInfC{$\Gamma,~y:B,~x:A\vdash y:B$}
    \AxiomC{$\Gamma\vdash B~\textrm{type}$}
    \AxiomC{$\Gamma,~x:A,~y:B,~\Delta\vdash \mathcal{J}$}
    %\RightLabel{$y'/y$}
    \UnaryInfC{$\Gamma,~x:A,~y':B,~\Delta[y'/y]\vdash \mathcal{J}[y'/y]$}
    %\RightLabel{$W$}
    \BinaryInfC{$\Gamma,~y:B,~x:A,~y':B,~\Delta[y'/y]\vdash \mathcal{J}[y'/y]$}
    %\RightLabel{$S$.}
    \BinaryInfC{$\Gamma,~y:B,~x:A,~\Delta\vdash \mathcal{J}$}
  \end{prooftree}%
\end{small}
\index{derivation|)}

\begin{exercises}
  \exitem \label{ex:term_conversion}
  \begin{subexenum}
  \item Give a derivation for the following \define{element conversion rule}\index{element conversion rule|textbf}\index{rules!for type dependency!element conversion|textbf}\index{conversion rule!element|textbf}:
    \begin{prooftree}
      \AxiomC{$\Gamma\vdash A\jdeq A'~\textrm{type}$}
      \AxiomC{$\Gamma\vdash a:A$}
      \BinaryInfC{$\Gamma\vdash a:A'$.}
    \end{prooftree}
  \item Give a derivation for the following \define{congruence rule} for element conversion:
    \begin{prooftree}
      \AxiomC{$\Gamma\vdash A\jdeq A'~\textrm{type}$}
      \AxiomC{$\Gamma\vdash a\jdeq b:A$}
      \BinaryInfC{$\Gamma\vdash a\jdeq b:A'$.}
    \end{prooftree}
  \end{subexenum}
\end{exercises}
\index{dependent type theory|)}

%%% Local Variables:
%%% mode: latex
%%% TeX-master: "hott-intro"
%%% End:
