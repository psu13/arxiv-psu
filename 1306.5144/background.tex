%!TEX root = all.tex
% ******************************************************************
% ** Title:            The 2-category theory of quasi-categories
% **                  Background
% ** Precis:        
% ** Author:           Emily Riehl and Dominic Verity
% ** Commenced:        2/3/2012
% ******************************************************************



  \section{Background on quasi-categories}\label{sec:background}

	We start by reviewing some basic concepts and notations. 

  \begin{obs}[size]\label{obs:size-conventions}
    In this paper matters of size will not be of great importance. However, for definiteness we shall adopt the usual conceit of assuming that we have fixed an inaccessible cardinal which then determines a corresponding Grothendieck universe, members of which will be called {\em sets\/}; we refer to everything else as {\em classes}.  A category is {\em small\/} if it has sets of objects and arrows; a category is {\em locally small\/} if each of its hom-sets is small. We shall write $\Set$ to denote the large and locally small category of all sets and functions between them.

    When discussing the existence of limits and colimits we shall implicitly assume that these are indexed by small categories. Correspondingly, completeness and cocompletess properties will implicitly reference the existence of small limits and small colimits.
     \end{obs}

    \subsection{Some standard simplicial notation}\label{subsect:simplicial.notation}

    \begin{ntn}[simplicial operators]\label{ntn:simp.op}
        As usual, we let $\Del+$ denote the algebraists' (skeletal) category of all finite ordinals and order preserving maps between them and let $\Del$ denote the topologists' full subcategory of non-zero ordinals. Following tradition, we write $[n]$ for the ordinal $n+1$ as an object of $\Del+$ and refer to arrows of $\Del+$ as {\em simplicial operators}.  We will generally use lower case Greek letters $\alpha,\beta,\gamma\colon[m]\to[n]$ to denote simplicial operators. We will also use the following standard notation and nomenclature throughout:
        \begin{itemize}
            \item The injective maps in $\Del+$ are called {\em face operators}. For each $j\in[n]$,  $\face_n^j\colon[n-1]\to[n]$ denotes the {\em elementary face operator\/} distinguished by the fact that its image does not contain the integer $j$.
            \item The surjective maps in $\Del+$ are called {\em degeneracy operators}. For each $j\in[n]$, we write $\degen_n^j\colon[n+1]\to[n]$ to denote the {\em elementary degeneracy operator} determined by the property that two integers in its domain map to the integer $j$ in its codomain.
      %  \item We write $\tdegen_n\colon[n]\to[0]$ and $\aug_n\colon[-1]\to[n]$ to denote the unique such simplicial operators.
        \end{itemize}
        Unless doing so would introduce an ambiguity, we tend to reduce notational clutter by dropping the subscripts of these elementary operators.
    \end{ntn}

    \begin{ntn}[(augmented) simplicial sets]\label{ntn:simplicial-sets}
        Let $\sSet$ denote the functor category $\Set^{\Del\op}$,  the category of all {\em simplicial sets\/} and {\em simplicial maps\/} between them. 

        If $X$ is a simplicial set then $X_n$ will denote its value at the object $[n]\in\Del$, called its set of $n$-simplices, and if $f\colon X\to Y$ is a simplicial map then $f_n\colon X_n\to Y_n$ denotes its component at $[n]\in\Del$.

        It is common to think of simplicial sets as being right $\Del$-sets and use the (right) action notation $x\cdot\alpha$ to denote the element of $X_n$ obtained by applying the image under $X$ of a simplicial operator $\alpha \colon [n] \to [m]$ to an element $x\in X_m$. Exploiting this notation,  the functoriality of a simplicial set $X$ may be expressed in terms of the familiar action axioms $(x\cdot\alpha)\cdot\beta = x\cdot(\alpha\circ\beta)$ and $x\cdot\id = x$ and the naturality of a simplicial map $f\colon X\to Y$ corresponds to the action preservation identity $f(x\cdot\alpha)=f(x)\cdot\alpha$. 

A subset $Y\subseteq X$ of a simplicial set $X$ is said to be a {\em simplicial subset\/} of $X$ if it is closed  under  right action by all simplicial operators. If $S$ is a subset of $X$ then there is a smallest simplicial subset of $X$ containing $S$, the simplicial subset of $X$ {\em generated by\/} $S$.

We adopt the same notational conventions for {\em augmented simplicial sets}, objects of the functor category $\Set^{\Del+\op}$, which we denote by $\asSet$.
    \end{ntn}

    \begin{rec}[augmentation]\label{rec:augmentation}
       There is a canonical forgetful functor $\asSet\to\sSet$ constructed by pre-composition with the inclusion functor $\Del\inc\Del+$. Rather than give this functor a name, we prefer instead to allow context to determine whether an augmented simplicial set should be regarded as being a simplicial set by forgetting its augmentation. 

        Left and right Kan extension along $\Del\inc\Del+$ provides left and right adjoints to this forgetful functor, both of which are fully faithful. The left adjoint gives a simplicial set $X$ the {\em initial augmentation\/} $X\to\cpts X$ by its set of path components. The right adjoint gives $X$ the {\em terminal augmentation\/} $X\to {*}$ by the singleton set. We say that an augmented simplicial set is {\em initially\/} (resp.~{\em terminally}) {\em augmented\/} if the counit (resp.~unit) of the appropriate adjunction is  an isomorphism.

        Each $(-1)$-simplex $x$ in an augmented simplicial set $X$ is associated with a terminally augmented sub-simplicial set consisting of those simplices whose $(-1)$-face is $x$. These components are mutually disjoint and their disjoint union is the whole of $X$, providing a canonical decomposition of $X$ as a disjoint union of terminally augmented simplicial sets.
    \end{rec}

    \begin{ntn}[some important (augmented) simplicial sets]

        We fix notation for some important (augmented) simplicial sets. 
        \begin{itemize}
            \item The {\em standard $n$-simplex\/} $\Del^n$ is defined to be the contravariant representable on the ordinal $[n]\in\Del+$. In other words, $\Del^n_m$ is the set of simplicial operators $\alpha\colon[m]\to[n]$ which are acted upon by pre-composition.
            \item The {\em boundary\/} of the standard $n$-simplex $\boundary\Del^n$ is defined to be the simplicial subset of $\Del^n$ consisting of those simplicial operators which are not degeneracy operators. This is the simplicial subset of $\Del^n$ generated by the set of its $(n-1)$-dimensional faces.
            \item The {\em $(n,k)$-horn\/} $\Horn^{n,k}$ (for $n\in\mathbb{N}$ and $0\leq k\leq n$) is the simplicial subset of $\Del^n$ generated by the set $\{\face^i_n\mid 0\leq i\leq n \text{ and } i\neq k\}$ of $(n-1)$-dimensional faces.  Alternatively, we can describe $\Horn^{n,k}$ as the simplicial subset of those simplicial operators $\alpha\colon[m]\to[n]$ for which $\im(\alpha)\cup\{k\}\neq[n]$.
            \item We say that $\Horn^{n,k}$ is an {\em inner horn\/} if $0<k<n$; if $k=0$ or $k=n$, it is an {\em outer horn}.
        \end{itemize}

We have overloaded our notation above to refer interchangeably to objects of $\sSet$ or $\asSet$. There is no ambiguity since in each case the underlying simplicial set of one of these objects in $\asSet$ is  the corresponding object in $\sSet$. As an augmented simplicial set each of the objects above is terminally augmented.

        When $\alpha\colon [n]\to [m]$ is a simplicial operator we use the same symbol to denote the corresponding simplicial map $\alpha\colon\Del^n\to\Del^m$ which acts by post-composing with $\alpha$. In particular, $\face^j_n\colon\Del^{n-1}\to\Del^n$, $\degen^j_n\colon\Del^{n+1}\to\Del^n$,  $\tdegen_n\colon\Del^n\to\Del^0$, and $\aug_n:\Del^{-1}\to\Del^n$ denote the simplicial maps corresponding to the simplicial operators introduced in~\ref{ntn:simp.op} above.
    \end{ntn}

    \begin{ntn}[faces of \protect{$\Del^n$}]\label{ntn:faces-by-vertices}
It is useful to identify a non-degenerate simplex in the standard $n$-simplex $\Delta^n$ simply by naming its vertices. We use the notation $\fbv{v_0,v_1,v_2,...,v_m}$ to denote the simplicial operator $[m]\to [n]$ which maps $i\in[m]$ to $v_i\in[n]$. Let $\Del^{\fbv{v_0,v_1,...,v_m}}$ denote the smallest simplicial subset of $\Del^n$ which contains the face $\fbv{v_0,v_1,...,v_m}$.

      %More generally, in the case where $X$ is a simplicial set whose simplices are {\em determined by their vertices}, in the sense that if $x, x'\in X$ are simplices with the same $0$-faces then $x=x'$, we shall adopt the convention of annotating an $n$-simplex of $X$ by listing its $0$-faces in order $\fbv{x_0,x_1,...,x_n}$. The nerves of pre-ordered sets have simplices which are determined by their vertices and the class of all such simplicial sets is closed under product and subset. Indeed, any simplicial set in this class may be obtained, up to isomorphism, as a simplicial subset of a nerve of a pre-ordered set.
    \end{ntn}
    
    \begin{ntn}[internal hom]\label{ntn:simplicial-hom-space}
    Like any presheaf category, the category of simplicial sets is cartesian closed. We write $Y^X$ for the exponential, equivalently the {\em internal hom\/} or simply {\em hom-space}, from $X$ to $Y$. By the defining adjunction and the Yoneda lemma, an $n$-simplex in $Y^X$ is a simplicial map $X \times \Del^n \to Y$. Its faces and degeneracies are computed by pre-composing with the appropriate maps between the representables.
    \end{ntn}
    
    \subsection{Quasi-categories}

    \begin{defn}[quasi-categories]
      A {\em quasi-category} is a simplicial set $A$ which possesses the {\em right lifting property\/} with respect to all {\em inner horn inclusions\/} $\Horn^{n,k}\inc\Del^n$ ($n \geq 2$, $0<k<n$). A simplicial map between quasi-categories will be called a {\em functor}. We write $\qCat$ for the full subcategory of $\sSet$ consisting of the quasi-categories and functors.
    \end{defn}
    

    \begin{rec}[the homotopy category]\label{rec:hty-category}
      Let $\Cat$ denote the category of all small categories and functors between them. There is an adjunction
      \begin{equation*}
        \adjdisplay \ho-|\nrv:\Cat->\sSet. 
      \end{equation*}
      given by the nerve construction and its left adjoint. Since the nerve construction is fully faithful, we typically regard $\Cat$ as being a full subcategory of $\sSet$ and elide explicit mention of the functor $\nrv$. The nerve of any category is a quasi-category, so we may equally well regard $\Cat$ as being a reflective full subcategory of $\qCat$.

 When $A$ is a quasi-category, $\ho{A}$ is sensibly called its {\em homotopy category}; it has:
 \begin{itemize}
 \item \textbf{objects} the 0-simplices of $A$,
\item \textbf{arrows} equivalence classes of 1-simplices of $A$ which share the same boundaries, and
\item \textbf{composition} determined by the property that $k = g f$ in $\ho{A}$ if and only if there exists a 2-simplex $a$ in $A$ with $a\cdot\face^0=g$, $a\cdot\face^2=f$ and $a\cdot\face^1=k$.
\end{itemize}
See, e.g., \cite[\S 1.2.3]{Lurie:2009fk}. To emphasise the analogy with categories, we draw a 1-simplex $f$ of $A$ as an arrow with domain $f \cdot \face^1$ and codomain $f \cdot \face^0$. With these conventions, a 2-simplex $a$ of $A$ witnessing the identity $k = g f$ in $\ho{A}$ takes the form:
  \begin{equation*}
  \xymatrix{ & \cdot \ar[dr]^g \ar@{}[d]|(.6){a} \\ \cdot \ar[ur]^f \ar[rr]_k & & \cdot}    
\end{equation*} 
      
  Identity arrows in $\ho{A}$ are represented by degenerate 1-simplices. Hence, the composition axiom defines what it means for a parallel pair of 1-simplices  $f,f' \colon x \to y$  to represent the same morphism in $\ho{A}$: this is the case if and only if there exist 2-simplices of each of  (equivalently, any one of) the following forms
  \begin{equation}\label{eq:homotopy-of-1-simplices} \xymatrix{ & y \ar[dr]^{y\cdot \degen^0} & & & y \ar[dr]^{y\cdot\degen^0}& & & x \ar[dr]^f & & & x \ar[dr]^{f'} \\ x \ar[ur]^f \ar[rr]_{f'} & & y & x \ar[ur]^{f'} \ar[rr]_f & & y & x \ar[ur]^{x \cdot\degen^0} \ar[rr]_{f'} & & y & x \ar[ur]^{x\cdot\degen^0} \ar[rr]_f & & y}\end{equation} 
    In this case, we say that $f$ and $f'$ are {\em homotopic relative to their boundary}.
      
Both of the functors $\ho$ and $\nrv$ are {\em cartesian}, preserving all finite products; see \cite[B.0.15]{Joyal:2008tq} or \cite[18.1.1]{Riehl:2014kx}.    
    \end{rec}

\begin{ntn}
 Let $\catone$, $\cattwo$, or $\catthree$  denote the one-point \fbox{$\bullet$}, \emph{generic arrow} \fbox{$\bullet\to\bullet$}, and \emph{generic composed pair} \fbox{$\bullet\to\bullet\to\bullet$} \emph{categories} respectively. Under our identification of categories with their nerves, these categories are identified with the standard simplices $\Del^0$, $\Del^1$, and $\Del^2$ respectively.
\end{ntn}
    

The terms {\em model category\/} and {\em model structure\/}  refer to closed model structures in the sense of Quillen~\cite{Quillen:1967:Model}.

    \begin{rec}[the model category of quasi-categories]\label{rec:qmc-quasicat}
    The quasi-categories are precisely the fibrant-cofibrant objects in a combinatorial model structure on simplicial sets due to Joyal, a proof of which can be found in \cite[\S 6.5]{Verity:2007:wcs1}.
       For our purposes here, it will be enough to recall that Joyal's model structure is completely determined by the fact that it has: 
      \begin{itemize}
      \item {\em weak equivalences\/}, which are  those simplicial maps $w\colon X\to Y$ for which each functor $\ho(A^w)\colon\ho(A^Y)\to\ho(A^X)$ is an equivalence of categories for all quasi-categories $A$,
      \item {\em cofibrations\/}, which are simply the injective simplicial maps. In particular all objects are cofibrant in this model structure, and
      \item {\em fibrations between fibrant objects\/}, which are those functors of quasi-categories which possess the right lifting property with respect to:
      \begin{itemize}
      \item all inner horn inclusions $\Horn^{n,k}\inc\Del^n$ ($n\geq 2$, $0<k<n$), and
      \item (either one of) the monomorphisms $\Del^0\inc \iso$, where $\iso$ denotes the {\em generic isomorphism category\/} $\bullet\cong\bullet$.
      \end{itemize}
      To emphasise the analogy with 1-category theory, we call the fibrations between fibrant objects {\em isofibrations}.
      \end{itemize}
    \end{rec}

The Joyal model structure for quasi-categories is \emph{cartesian}, the meaning of which requires the following construction.


\begin{rec}[Leibniz constructions]\label{rec:leibniz}
If we are given a bifunctor
$ \otimes \colon \lcat{K} \times \lcat{L} \to \lcat{M}$ 
  whose codomain  possesses all pushouts, then the {\em Leibniz construction\/} provides us with a bifunctor $ \leib\otimes \colon \lcat{K}\mapcat \times \lcat{L}\mapcat \to \lcat{M}\mapcat$
  between arrow categories, which carries a pair of objects $f \in \lcat{K}\mapcat$ and $g \in \lcat{L}\mapcat$ to an object $f\leib\otimes g \in \lcat{M}\mapcat$ defined to be the map induced by the universal property of the pushout in the following diagram:
  \begin{equation}
  \xymatrix@=2em{
    {K\otimes L} \ar[d]_{K\otimes g} \ar[r]^{f\otimes L} &
    {K'\otimes L} \ar[d] \ar@/^4ex/[ddr]^{K'\otimes g} & \\
    {K\otimes L'} \ar[r] \ar@/_4ex/[rrd]_{f\otimes L'} &
    {(K'\otimes L) \cup_{K\otimes L} (K\otimes L') } \poexcursion
    \ar@{-->}[dr]_{f\leib\otimes g} & \\
    & & {K'\otimes L'}
  }
  \end{equation}
  The action of this functor on the arrows of $\lcat{K}\mapcat$ and $\lcat{L}\mapcat$ is the canonical one induced by the functoriality of $\otimes$ and the universal property of the pushout in the diagram above. In the case where the bifunctor $\otimes$ defines a monoidal product, the Leibniz bifunctor $\leib\otimes$ is frequently called the {\em pushout product}.
In the context of a bifunctor $\hom \colon \lcat{K}\op \times \lcat{L} \to \lcat{M}$, the dual construction, defined using pullbacks in $\lcat{M}$, is preferred.  We refer the reader to \cite[\S\ref*{reedy:sec:Leibniz-Reedy}]{RiehlVerity:2013kx} for a full account of this construction and its properties. %In particular, in the few instances where Reedy category theory is invoked in proofs appearing below, we make use of the notational conventions established therein.  
\end{rec}

\begin{rec}[cartesian model categories]\label{rec:cart-modcat}
The cartesianness of the Joyal model structure may be formulated in the following equivalent forms:
  \begin{enumerate}
    \item If $i\colon X\tcof Y$ and $j\colon U\tcof V$ are both cofibrations (monomorphisms) then so is their {\em Leibniz product\/} $i\leib\times j\colon (Y\times U)\cup_{X\times U} (X\times V)\tcof (Y\times V)$. Furthermore, if $i$  or $j$ is a trivial cofibration then so is $i\leib\times j$.
    \item If $i\colon X\tcof Y$ is a cofibration (monomorphism) and $p\colon A\tfib B$ is a fibration then their {\em Leibniz hom\/} $\leib\hom(i,p)\colon A^Y\tfib B^Y\times_{B^X} A^X$ is also a fibration. Furthermore, if $i$ is a trivial cofibration or $p$ is a trivial fibration then $\leib\hom(i,p)$ is also a trivial fibration.
  \end{enumerate}
  In particular, if $A$ is a quasi-category then we may apply the second of these formulations to the unique isofibration $!\colon A\to 1$ and monomorphisms $\emptyset\inc X$ and $i\colon X\inc Y$ to show that $A^X$ is again a quasi-category and that the pre-composition functor $A^i\colon A^Y\tfib A^X$ is an isofibration. 
\end{rec}

\begin{obs}[closure properties of isofibrations]\label{obs:isofibration-closure}
As a consequence of \ref{rec:qmc-quasicat} and \ref{rec:cart-modcat}, the isofibrations enjoy the following closure properties:
\begin{itemize}
\item  The isofibrations are closed under products, pullbacks, retracts, and transfinite limits of towers (as fibrations between fibrant objects).
\item The isofibrations are also closed under the Leibniz hom $\leib\hom(i,-)$ for any monomorphism $i$ and, in particular, under exponentiation $(-)^X$ for any simplicial set $X$ (as fibrations between fibrant objects in a cartesian model category).
\end{itemize}
\end{obs}

\subsection{Isomorphisms and marked simplicial sets}

  \begin{defn}[isomorphisms in quasi-categories]\label{defn:equivalences} 
    When $A$ is a quasi-category, we say that a 1-simplex $a\in A_1$ is an {\em isomorphism\/} if and only if the corresponding arrow of its homotopy category $\ho{A}$ is an isomorphism in the usual sense.
\end{defn}

Others use the term ``equivalences'' for the isomorphisms in a quasi-category, but we believe our terminology is less ambiguous: no stricter notion of isomorphism exists. 

When working with isomorphisms in quasi-categories, it will sometimes be convenient to work in the category of {\em marked simplicial sets\/} as defined by Lurie~\cite{Lurie:2009fk}.

    \begin{defn}[marked simplicial sets]
      A {\em marked simplicial set} $X$ is a simplicial set equipped with a specified subset of \emph{marked} $1$-simplices $mX\subseteq X_1$ containing all the degenerate 1-simplices. A map of marked simplicial sets is a map of underlying simplicial sets that carries marked 1-simplices to marked 1-simplices. While the category $\msSet$ of marked simplicial sets is not quite as well behaved as $\sSet$ it is nevertheless a \emph{quasitopos}, which implies that it is complete, cocomplete, and (locally) cartesian closed (see \cite[Observation 11]{Verity:2007:wcs1} and \cite{Street:2003:WomCats}).

      The functor $\msSet\to\sSet$ which forgets markings has both a left and a right adjoint. This left adjoint, dubbed {\em flat\/} by Lurie, makes a simplicial set $X$ into a marked simplicial set $X^\flat$ by giving it the minimal marking in which only the degenerate $1$-simplices are marked. Conversely, this right adjoint, which Lurie calls {\em sharp}, makes $X$ into a marked simplicial set $X^\sharp$ by giving it the maximal marking in which all $1$-simplices are marked. If $X$ is already a marked simplicial set then we will use the notation $X^\flat$ and $X^\sharp$ for the marked simplicial sets obtained by applying the flat or sharp construction (respectively) to the underlying simplicial set of $X$.

      In general, we will identify simplicial sets with their minimally marked variants, allowing us to extend the notation introduced above to the marked context. Any variation to this rule will be commented upon as we go along.
    \end{defn}
    
        \begin{rmk}[stratified simplicial sets]
Earlier authors, including Roberts~\cite{Roberts:1978:Complicial}, Street~\cite{Street:1987:Oriental}, and Verity~\cite{Verity:2008sr,Verity:2007:wcs1}, have studied a more general notion of {\em stratification}. A {\em stratified simplicial set\/} is again a simplicial set $X$ equipped with a specified subset of simplices which, in that context, are said to be {\em thin}. A stratification may contain simplices of arbitrary dimension and it must again contain all degenerate simplices. Stratifications are used to build structures called {\em complicial sets\/}, which model homotopy coherent higher categories in much the way that quasi-categories model homotopy coherent categories.
    \end{rmk}

    \begin{rec}[products and exponentiation]\label{rec:marked-prod-exp}
The product in $\msSet$ of marked simplicial sets $X$ and $Y$ is formed by taking the product of underlying simplicial sets and marking those $1$-simplices $(x,y)\in X\times Y$ which have $x$ marked in $X$ and $y$ marked in $Y$.

      An exponential (internal hom) $Y^X$ in marked simplicial sets has $n$-simplices which correspond to maps $k\colon X\times\Del^n\to Y$ of marked simplicial sets and has marked $1$-simplices those $k$ which extend along the canonical inclusion $X\times\Del^1\inc X\times(\Del^1)^\sharp$ to give a (uniquely determined) map $k'$ \[\xymatrix{ X \times \Del^1 \ar[r]^-k \ar@{_(->}[d] & Y \\ X \times (\Del^1)^\sharp \ar[ur]_-{k'}}\] That is, a marked 1-simplex in $Y^X$ is a map $k'\colon X \times (\Del^1)^\sharp \to Y$ of marked simplicial sets; see \cite[\S 3.1.3]{Lurie:2009fk}.
     The only $1$-simplices which are not marked in $X\times\Del^1$ but are marked in $X\times(\Del^1)^\sharp$ are pairs of the form $(x,\id_{[1]})$ in which $x$ is marked in $X$. It follows that a marked simplicial map $k\colon X\times\Del^1\to Y$ extends along $X\times\Del^1\inc X\times(\Del^1)^\sharp$, and thus represents a marked $1$-simplex in $Y^X$, if and only if for all marked $1$-simplices $x$ in $X$ the $1$-simplex $k(x,\id_{[1]})$ is marked in $Y$.
    \end{rec}

    \begin{rec}[isomorphisms and markings]\label{rmk:equiv-markings}
A quasi-category $A$ becomes a marked simplicial set $A^\natural$ with the {\em natural marking}, under which a 1-simplex is marked if and only if it is an isomorphism. When we regard an object as being a quasi-category in the marked setting we will always assume that it carries the natural marking without comment. A functor $f\colon A\to B$ between quasi-categories automatically preserves natural markings simply because the corresponding functor $\ho(f)\colon\ho{A}\to\ho{B}$ preserves isomorphisms.
\end{rec}

\begin{ntn}
      In this context it is useful to adopt the special marking convention for horns ($n \geq 1$, $0\leq k \leq n$) under which we
      \begin{itemize}
      \item write $\Del^{n:k}$ for the marked simplicial set obtained from the standard minimally marked simplex $\Del^n$ by also marking the edge $\fbv{0,1}$ in the case $k=0$ and marking the edge $\fbv{n-1,n}$ in the case $k=n$,
      \item inherit the marking of the horn $\Horn^{n,k}$ from that of $\Del^{n:k}$, and
      \item use $\Horn^{n,k}\inc\Del^{n:k}$ to denote the marked inclusion of this horn into its corresponding specially marked simplex.
      \end{itemize}
\end{ntn}

      Using these conventions we may recast Joyal's ``special horn filler'' result \cite[1.3]{Joyal:2002:QuasiCategories} simply as follows.

\begin{prop}[Joyal]\label{prop:joyal-special-horn} A naturally marked quasi-category has the right lifting property with respect to all marked horn inclusions $\Horn^{n,k}\inc\Del^{n:k}$, for $n\geq 1$ and $0\leq k \leq n$.
\end{prop}

 An important corollary is that a Kan complex is precisely a quasi-category in which every 1-simplex is an isomorphism \cite[1.4]{Joyal:2002:QuasiCategories}.

    \begin{rec}[the model structure of naturally marked quasi-categories]\label{rec:qmc-quasi-marked}
There is a model structure on the category of marked simplicial sets whose fibrant-cofibrant objects are precisely the naturally marked quasi-categories (see Lurie~\cite[\S 3.1]{Lurie:2009fk} or Verity~\cite[\S 6.5]{Verity:2007:wcs1}). This model category is  combinatorial and cartesian and is completely characterised by the fact that it has:
      \begin{itemize}
        \item {\em weak equivalences\/} which are those maps $w\colon X\to Y$ of marked simplicial sets for which $\ho(A^w)\colon\ho(A^Y)\to\ho(A^X)$ is an equivalence of categories for all (naturally marked) quasi-categories $A$,
        \item {\em cofibrations\/} which are simply the injective maps of marked simplicial sets, and
        \item {\em fibrations between fibrant objects\/} which are the isofibrations of naturally marked quasi-categories.
      \end{itemize}
      Here, the exponential $A^X$ is the internal hom in the category of marked simplicial sets $\msSet$. The functor $\ho\colon\msSet\to\Cat$ is the left adjoint to the nerve functor $\nrv\colon\Cat\to\msSet$ which carries a category $\scat{C}$ to the marked simplicial set whose underlying simplicial set is the usual nerve and in which a $1$-simplex is marked if and only if it is an isomorphism in $\scat{C} \cong \ho{\scat{C}}$. The left adjoint $h$ sends a marked simplicial set $X$ to the localisation of its homotopy category $hX$ at the set of marked edges. Note that in the case of a naturally marked quasi-category $A^\natural$, $h(A^\natural) = hA$, the usual homotopy category of the quasi-category.

      By \cite[7.14]{Joyal:2007kk}, a cofibration is a weak equivalence if and only if it has the left lifting property with respect to the fibrations between fibrant objects.
        In particular, in this model structure all of the special marked horn inclusions $\Horn^{n,k}\inc\Del^{n:k}$ ($n \geq 1$, $0\leq k\leq n$) and the inclusion $(\Del^1)^\sharp\inc\iso$ of the marked 1-simplex into the naturally marked isomorphism category are trivial cofibrations (see \cite[B.10, B.15]{DuggerSpivak:2011ms}). This proves that an isomorphism $\Del^1 \to A$ in a quasi-category may always be extended to a functor $\iso \to A$ \cite[1.6]{Joyal:2002:QuasiCategories}.
\end{rec}

  \begin{obs}[natural markings, internal homs, and products]\label{obs:nat-mark-homs}
The product of two naturally marked quasi-categories is again a naturally marked quasi-category.
By cartesianness of the marked model structure, if $A$ is a naturally marked quasi-category and $X$ is any marked simplicial set then the exponential $A^X$ is again a naturally marked quasi-category.    In summary, the fully faithful natural marking functor $\natural\colon\qCat\to\msSet$ is a cartesian closed functor, in the sense that it preserves products and internal homs.
    \end{obs}

  The content of observation \ref{obs:nat-mark-homs} is  more profound than one might initially suspect. It might be summarised by the slogan ``a natural transformation of functors is an isomorphism if and only if it is a {\em pointwise isomorphism}''. The precise meaning of this slogan is encoded in the following result.

            \begin{lem}[pointwise isomorphisms are isomorphisms]\label{lem:pointwise-equiv}
Let $X$ be a marked simplicial set and let $A$ be a naturally marked quasi-category.  A $1$-simplex $k\colon X\times\Del^1\to A$ is marked in $A^X$ if and only if for all $0$-simplices $x$ in $X$ the $1$-simplex $k(x\cdot\degen^0,\id_{[1]})$ is marked in $A$.
       \end{lem}
Here is the intuition for this result. The component of a map $k \colon X \times \Del^1 \to A$ at a 1-simplex $f \colon a \to b$ in $X$      is a diagram $\Del^1 \times \Del^1 \to A$ \begin{equation}\label{eq:pointwise-equivalence-square}\xymatrix@=35pt{ \cdot \ar[d]_{k(f,\face^1)} \ar[r]^{k(a,\id_{[1]})} \ar[dr]|{k(f,\id_{[1]})} & \cdot \ar[d]^{k(f,\face^0)} \\ \cdot \ar[r]_{k(b,\id_{[1]})} & \cdot}\end{equation} If $f$ is marked and $k$ is a marked map, then the verticals are marked in $A$. If $A$ is a naturally marked quasi-category, then if the horizontals, the ``components'' of $k$, are marked, then so is the diagonal edge, simply because isomorphisms compose. If this is the case for all marked 1-simplices $f$, then $k$ is marked in $A^X$ by the definition of the internal hom in $\msSet$.
               
      \begin{proof}
As recalled in~\ref{rec:marked-prod-exp}, $k$ is a marked $1$-simplex in $A^X$ if and only if $k(f,\id_{[1]})$ is marked in $A$ for all marked edges $f$ of $X$. In particular, the edges $k(x\cdot\degen^0,\id_{[1]})$ are necessarily marked in $A$ if $k$ is marked in $A^X$. We show that this condition is sufficient to detect the marked edges $k \in (A^X)_1$.

 The $2$-simplex $(f\cdot\degen^0,\degen^1)$ of $X\times\Del^1$ can be drawn as follows:
        \[\left(    \vcenter{
        \xymatrix{ & \cdot \ar[dr]^{f} \ar@{}[d]|(.6){f \cdot \degen^0} \\ \cdot \ar@{=}[ur]^{(f \cdot \face^1)\cdot \degen^0} \ar[rr]_{f} & & \cdot} }\quad,\quad   \vcenter{
        \xymatrix{ & \cdot \ar@{=}[dr]^{\degen^0} \ar@{}[d]|(.6){\degen^1} \\ \cdot \ar[ur]^{{\id_{[1]}}} \ar[rr]_{\id_{[1]}} & & \cdot} } \right) \]
Applying $k$, the $2$-simplex $k(f\cdot\degen^0,\degen^1)$ of $A$ witnesses the fact that $k(f,\id_{[1]})$ is a composite of $k(f,\degen^0)$ and $k((f\cdot\face^1)\cdot\degen^0,\id_{[1]})$. 
        
        Now when $f$ is marked in $X$, the edge $(f,\degen^0)$ is marked in $X\times \Del^1$, so it follows that $k(f,\degen^0)$ is marked in $A$. By assumption the $1$-simplex $k((f\cdot\face^1)\cdot\degen^0,\id_{[1]})$ is also marked in $A$. The isomorphisms, that is to say naturally marked $1$-simplices, compose in $A$ simply because isomorphisms compose in the category $\ho{A}$, so it follows that $k(f,\id_{[1]})$ is marked in $A$. 
    \end{proof}

Recall that the marked edges in a naturally marked quasi-category are precisely the isomorphisms. Reinterpetting Lemma \ref{lem:pointwise-equiv} in the unmarked context, we have proven:

\begin{cor}\label{cor:pointwise-equiv}
For any quasi-category $A$ and simplicial set $X$, an edge $k \in (A^X)_1$ is an isomorphism if and only if each of its components $k(x) \in A_1$, defined by evaluating at each vertex $x \in X_0$, are isomorphisms.
\end{cor}



  \begin{obs}\label{obs:marked-arrow-subcat}
    If $A$ is a naturally marked quasi-category then pre-composition by the inclusion $\Del^1\inc(\Del^1)^\sharp$ gives rise to an inclusion $A^{(\Del^1)^\sharp}\inc A^{\Del^1}$ of naturally marked quasi-categories. Taking transposes, we see that Lemma \ref{lem:pointwise-equiv} may be recast as saying that a marked simplicial map $k\colon X\to A^{\Del^1}$ has a (necessarily unique) lift as the dotted arrow in
    \begin{equation*}
      \xymatrix@=2em{
        & {A^{(\Del^1)^\sharp}}\ar@{u(->}[d] \\
        {X}\ar[r]_{k}\ar@{-->}[ur] &
        {A^{\Del^1}}
      }
    \end{equation*}
    if and only if $k$ maps each 0-simplex $x\in X$ to an object $k(x)\in A^{\Del^1}$ which corresponds to a marked arrow of $A$. In other words, the map $A^{(\Del^1)^\sharp}\inc A^{\Del^1}$ is a fully faithful inclusion which identifies $A^{(\Del^1)^\sharp}$ with the full sub-quasi-category of $A^{\Del^1}$ whose objects are the isomorphisms in $A$.
  \end{obs}

\subsection{Join and slice}\label{subsec:join}

Particularly to facilitate comparisons between our development of the theory of quasi-categories, using the enriched category theories of 2-categories and simplicial categories, and the more traditional accounts following Joyal and Lurie, we review Joyal's slice and join constructions, introduced in \cite{Joyal:2002:QuasiCategories}. Unlike in the classical treatments, these technical combinatorial details will be of secondary importance for us, and for that reason, we encourage the reader to skip this section upon first reading, referring back only as necessary. A more leisurely account of the combinatorial work reviewed here can be found in an earlier version of this paper \cite[\S A]{RiehlVerity:2015tt-v3}.

  \begin{defn}[joins and d{\'e}calage]\label{defn:join-dec} The algebraists' skeletal category $\Del+$ of all finite ordinals and order preserving maps supports a canonical strict (non-symmetric) monoidal structure $(\Del+,\oplus,[-1])$ in which $\oplus$ denotes the {\em ordinal sum\/} given 
  \begin{itemize}
    \item for objects $[n],[m]\in\Del+$ by $[n]\oplus[m] \defeq [n+m+1]$,
    \item for arrows $\alpha\colon[n]\to[n'], \beta\colon[m]\to[m']$ by $\alpha\oplus\beta\colon[n+m+1]\to[n'+m'+1]$ defined by
  \begin{equation*}
    \alpha\oplus\beta(i) =
    \begin{cases}
    \alpha(i)& \text{if $i\leq n$,} \\
    \beta(i-n-1) + n' + 1& \text{otherwise.}
    \end{cases}
  \end{equation*}
  \end{itemize}
By Day convolution, this bifunctor extends to a (non-symmetric) monoidal closed structure $(\asSet, \join, \Del^{-1}, \dec_l, \dec_r)$ on the category of augmented simplicial sets. Here the monoidal operation $\join$ is known as the {\em simplicial join\/} and its closures $\dec_l$ and $\dec_r$ are known as the {\em left and right d{\'e}calage constructions}, respectively.   To fix handedness, we declare that for each augmented simplicial set $X$ the functor $\dec_l(X,{-})$ (resp.\ $\dec_r(X,{-})$) is right adjoint to $X\join{-}$ (resp.\ ${-}\join X$).

The join $X\join Y$ of augmented simplicial sets $X$ and $Y$ may be described explicitly as follows:
  \begin{itemize}
    \item it has simplices pairs $(x,y) \in (X\join Y)_{r+s+1}$ with $x\in X_r$, $y\in Y_s$,
    \item if $(x,y)$ is a simplex of $X\join Y$ with $x \in X_r$ and $y \in Y_s$ and $\alpha\colon[n]\to[r+s+1]$ is a simplicial operator in $\Del+$, then $\alpha$ may be uniquely decomposed as $\alpha=\alpha_1\join\alpha_2$ with $\alpha_1\colon[n_1]\to[r]$ and $\alpha_2\colon[n_2]\to[s]$, and we define $(x,y)\cdot\alpha\defeq (x\cdot\alpha_1,y\cdot\alpha_2)$. 
  \end{itemize}
If $f\colon X\to X'$ and $g\colon Y\to Y'$ are simplicial maps then the simplicial map $f\join g\colon X\join Y\to X'\join Y'$  carries the simplex $(x,y)\in X\join Y$ to the simplex $(f(x),g(y))\in X'\join Y'$. 
  \end{defn}

  \begin{defn}[d{\'e}calage and slices]\label{defn:slices}
The d{\'e}calage functors can be used to define Joyal's \emph{slice} construction for a map $f\colon X\to A$ of simplicial. 
Fixing a simplicial set $X$ and identifying the category $\sSet$  of simplicial sets with the full subcategory of terminally augmented simplicial sets in $\asSet$, we define a functor
    \begin{equation*}
      {-}\mathbin{\bar\join} X\colon \xy<0em,0em>*+{\sSet}\ar <6em,0em>*+{X\slice\sSet}\endxy\mkern30mu
      (\text{resp.\ } 
      X\mathbin{\bar\join}{-}\colon \xy<0em,0em>*+{\sSet}\ar <6em,0em>*+{X\slice\sSet}\endxy)
    \end{equation*}
    which carries a simplicial set $Y\in\sSet$ to the object ${*}\join X\colon X\cong\Del^{-1}\join X\to Y\join X$ (resp.\ $X\join{*}\colon X\cong X\join\Del^{-1} \to X\join Y$) induced by the map ${*}\colon\Del^{-1}\to Y$ corresponding to the unique $(-1)$-simplex of $Y$. This functor preserves all colimits, and thus admits a right adjoint that we now describe explicitly.
    
Observe that the $(-1)$-dimensional simplices of $\dec_r(X, A)$ (resp.\ $\dec_l(X,A)$) are in bijective correspondence with simplicial maps $f\colon X\to A$. So if we are given such a simplicial map we may, by recollection \ref{rec:augmentation}, extract the component of $\dec_r(X,A)$ (resp.\ $\dec_l(X,A)$) consisting of those simplices whose $(-1)$-face is $f$, which we denote by $\slicer{A}{f}$ (resp.\ $\slicel{f}{A}$) and call the {\em slice of $A$ over (resp.\ under) $f$}. Now it is a matter of an easy calculation to demonstrate directly that $\slicer{A}{f}$ (resp. $\slicel{f}{A}$) has the universal property required of the right adjoint to ${-}\bar\join X$ (resp.\ $X\bar\join {-}$) at the object $f\colon X\to A$ of $X\slice\sSet$.

    In other words, these d{\'e}calages admit the following canonical decompositions as disjoint unions of (terminally augmented) slices: 
    \begin{equation*}
      \dec_r(X,A)=\bigsqcup_{f\colon X\to A} (\slicer{A}{f})\mkern30mu
      \dec_l(X,A)=\bigsqcup_{f\colon X\to A} (\slicel{f}{A})
    \end{equation*}

    We think of the slice $\slicel{f}{A}$ as being the simplicial set of {\em cones under the diagram $f$\/} and we think of the dual slice $\slicer{A}{f}$ as being the simplicial set of {\em cones over the diagram $f$}.
  \end{defn}
  
  \begin{obs}[slices of quasi-categories]\label{obs:slice-and-qcats}
    A direct computation from the explicit description of the join construction given above demonstrates that the Leibniz join (see recollection~\ref{rec:leibniz}) of a horn and a boundary $(\Horn^{n,k}\inc\Del^n)\leib\join(\boundary\Del^m\inc\Del^m)$ is again isomorphic to a single horn $\Horn^{n+m+1,k}\inc\Del^{n+m+1}$. Dually the Leibniz join $(\boundary\Del^n\inc\Del^n)\leib\join(\Horn^{m,k}\inc\Del^m)$ is isomorphic to the single horn $\Horn^{n+m+1,n+k+1}\inc\Del^{n+m+1}$. 

    Combining these computations with the properties of the Leibniz construction developed in \cite[\S\ref*{reedy:sec:Leibniz-Reedy}]{RiehlVerity:2013kx}, we may show that an augmented simplicial set $A$ has the right lifting property with respect to all (inner) horn inclusions then so do the left and right d{\'e}calages $\dec_l(X,A)$ and $\dec_r(X,A)$ for any augmented simplicial set $X$. In particular, this tells us that if $f\colon X\to A$ is any map of simplicial sets and $A$ is a quasi-category then the slices $\slicel{f}{A}$ and $\slicer{A}{f}$ are also quasi-categories. 

    Working in the marked context, we may extend this result to Leibniz joins with specially marked outer horns. That then allows us to prove that if $p\colon A\to B$ is an isofibration of quasi-categories and $f\colon X\to A$ is any simplicial map then the induced simplicial maps $p\colon \slicer{A}{f}\to \slicer{B}{pf}$ and $p\colon \slicel{f}{A}\to \slicel{pf}{B}$ are also isofibrations of quasi-categories.
  \end{obs}

A variant of the join and slice constructions, also due to Joyal, is more closely related to the enriched categorical comma constructions that we will use here.

 \begin{defn}[fat join]\label{def:fat-join}
    We define the {\em fat join} of two simplicial sets $X$ and $Y$ to be the simplicial set $X\fatjoin Y$ constructed by means of the following pushout:
    \begin{equation}\label{eq:fat-join-def}
      \xymatrix@R=2em@C=4em{
        {({X}\times{Y})\sqcup({X}\times{Y})}\ar[r]^-{\pi_X\sqcup\pi_Y}
        \ar[d]_{\langle{X}\times\face^1\times{Y},
          {X}\times\face^0\times{Y}\rangle} &
        {{X}\sqcup{Y}}\ar[d] \\
        {{X}\times\Del^1\times{Y}} \ar[r] &
        {{X}\fatjoin{Y}}\poexcursion
      }
    \end{equation}
    We may extend this construction to simplicial maps in the obvious way to give us a bifunctor $\fatjoin\colon\sSet\times\sSet\to\sSet$, and it is clear that this preserves connected colimits in each variable. It does not preserve all colimits  because the coproduct bifunctor $\sqcup$ (as used in the top right hand corner of the defining pushout above) fails to preserve coproducts  in each variable (while it does preserve connected colimits). In particular, a fat join of a simplicial set $X$ with the empty simplicial set, rather than being empty, is isomorphic  to $X$ itself.

%    Now if we again identify the category of simplicial sets $\sSet$ with the full subcategory of terminally augmented simplicial sets then we may extend our fat join to a bifunctor on augmented simplicial sets. To do this we start by observing that every augmented simplicial set $X$ may be written canonically as a coproduct $\bigsqcup_{i\in I} X_i$ in $\asSet$ of terminally augmented simplicial sets. So if $X = \bigsqcup_{i\in I} X_i$ and $Y=\bigsqcup_{j\in J} Y_j$ are two augmented simplicial sets with terminally augmented components $X_i$ and $Y_j$, then we define $X\join Y$ to be the coproduct $\bigsqcup_{i\in I, j\in J} X_i\fatjoin Y_j$ in $\asSet$. We extend this to maps of augmented simplicial sets in the obvious way, using the fact that we may decompose such maps into families of maps between terminally augmented components. It is now a routine matter to verify that, when regarded as being a bifunctor on $\asSet$, the fat join does indeed preserve {\em all\/} colimits in each variable. Consequently, since $\asSet$ is a presheaf category, it follows that the fat join bifunctor on $\asSet$ has both left and right closures $\fatdec_l(X,A)$ and $\fatdec_r(X,A)$, called  {\em left and right fat d{\'e}calage\/} respectively, which notation we fix by declaring that if $X$ is an augmented simplicial set then $X\fatjoin{-}\dashv\fatdec_l(X,{-})$ and ${-}\fatjoin X\dashv\fatdec_r(X,{-})$.

The fat join of two non-empty simplicial sets $X$ and $Y$ may be described more concretely as the simplicial set obtained by taking the quotient of ${X}\times\Del^1\times{Y}$ under the simplicial congruence relating the pairs of $r$-simplices
    \begin{equation}\label{eq:fat-join-cong}
    (x,0,y)\sim (x,0,y')\quad \text{and}\quad (x,1,y) \sim (x',1,y) 
   \end{equation}
   where $0$ and $1$ denote the constant operators $[r]\to[1]$. We use square bracketed triples $[x,\beta,y]_\sim$ to denote equivalence classes under $\sim$. 
  \end{defn}

  \begin{defn}[fat slice]\label{defn:fat-slices}
    Replaying Joyal's slice construction of Definition~\ref{defn:slices}, if $X$ is a simplicial set, we may use the fat join to construct a functor
    \begin{equation*}
      {-}\mathbin{\bar\fatjoin} X\colon \xy<0em,0em>*+{\sSet}\ar <6em,0em>*+{X\slice\sSet}\endxy\mkern30mu
      (\text{resp.\ } 
      X\mathbin{\bar\fatjoin}{-}\colon \xy<0em,0em>*+{\sSet}\ar <6em,0em>*+{X\slice\sSet}\endxy)
    \end{equation*}
    which carries a simplicial set $Y\in\sSet$ to the object ${*}\fatjoin X\colon X\cong\Del^{-1}\fatjoin X\to Y\fatjoin X$ (resp.\ $X\fatjoin{*}\colon X\cong X\fatjoin\Del^{-1} \to X\fatjoin Y$). These functors admit right adjoints whose value at an object $f\colon X\to A$ of $X\slice\sSet$ is denoted $\fatslicer{A}{f}$ (resp.\ $\fatslicel{f}{A}$) and is called the {\em fat slice of $A$ over} (resp.\ \emph{under}) $f$.
    \end{defn}

  \begin{obs}[comparing join constructions]
    When $\beta\colon [n]\to [1]$ is a simplicial operator let $\hat{n}_\beta$ denote the largest integer in the set $\{-1\}\cup\{i\in[n]\mid \beta(i)=0\}$  and let $\check{n}_\beta=n-1-\hat{n}_\beta$. Define an associated pair $\hat\beta\colon[\hat{n}_\beta]\to[n]$ and $\check\beta\colon[\check{n}_\beta]\to[n]$ of simplicial face operators in $\Del+$ by $\hat\beta(i) = i$ for all $i\in[\hat{n}_\beta]$ and $\check\beta(j)=j+\hat{n}_\beta + 1$ for all $j\in[\check{n}_\beta]$. 
    
    Now if $X$ and $Y$ are (terminally augmented) simplicial sets we may define a map $\bar{s}^{X,Y}$ which carries an $n$-simplex $(x,\beta,y)$ of $X\times \Del^1\times Y$ to the $n$-simplex $(x\cdot\hat\beta,y\cdot\check\beta)$ of $X\join Y$. A straightforward calculation demonstrates that this map commutes with the simplicial actions on these sets and is thus a simplicial map. Furthermore, the family of simplicial maps $\bar{s}^{X,Y}\colon X\times \Del^1\times Y \to X\join Y$ is natural in $X$ and $Y$.

    Of course, since $X$ and $Y$ are terminally augmented, we also have canonical maps $l^{X,Y}\colon X\cong X\join\Del^{-1}\to X\join Y$ and $r^{X,Y}\colon Y\cong\Del^{-1}\join Y\to X\join Y$ and we may assemble all these maps together into a commutative square 
    \begin{equation}\label{eq:join-comp-def}
      \xymatrix@R=2em@C=4em{
        {({X}\times{Y})\sqcup({X}\times{Y})}\ar[r]^-{\pi_X\sqcup\pi_Y}
        \ar[d]_{\langle{X}\times\face^1\times{Y},
          {X}\times\face^0\times{Y}\rangle} &
        {{X}\sqcup{Y}}\ar[d]^{\langle l^{X,Y}, r^{X,Y}\rangle} \\
        {{X}\times\Del^1\times{Y}} \ar[r]_{\bar{s}^{X,Y}} &
        {{X}\join{Y}}
      }
    \end{equation}
    whose maps are all natural in $X$ and $Y$. Using the defining universal property of fat join, as given in~\eqref{eq:fat-join-def}, these squares induce maps $s^{X,Y}\colon X\fatjoin Y\to X\join Y$ which are again natural in $X$ and $Y$. Should we so wish, we may now take suitable coproducts of these maps to canonically extend this family of simplicial maps to a natural transformation between the extended fat join and join bifunctors on augmented simplicial sets.

    More explicitly, if $n,m\geq 0$, then $\bar{s}^{n,m}\colon\Del^n\times\Del^1\times\Del^m\to\Del^{n+m+1}$ is the unique simplicial map determined by the (order preserving) action on vertices given by:
    \begin{equation}\label{eq:tnm-def}
      \bar{s}^{n,m}(i,j,k) = 
      \begin{cases}
        i & \text{if $j=0$, and} \\
        k+n+1 & \text{if $j=1$.}
      \end{cases}
    \end{equation}
    This takes simplices related under the congruence defined in~\eqref{eq:fat-join-def} of Definition~\ref{def:fat-join} to the same simplex and thus induces a unique map $s^{n,m}\colon\Del^n\fatjoin\Del^m\to\Del^n\join\Del^m$ on the quotient simplicial set.
  \end{obs}

  \begin{prop}\label{prop:join-fatjoin-equiv}
    For all simplicial sets $X$ and $Y$, the  map $s^{X,Y}\colon X\fatjoin Y\to X\join Y$ is a weak equivalence in the Joyal model structure.
  \end{prop}
\begin{proof} For proof see \cite[4.2.1.2]{Lurie:2009fk} or \cite[A.4.11]{RiehlVerity:2015tt-v3}.

\end{proof}


  \begin{lem}\label{lem:slices-quillen}
 For any simplicial set $X$, the slice and fat slice adjunctions
    \begin{align*}
  		\adjdisplay \textstyle X\bar\join{-}-|{} :X\slice\sSet->\sSet. & 
      \mkern20mu & 
  		\adjdisplay \textstyle {-}\bar\join X-|{} :X\slice\sSet->\sSet. \\
  		\adjdisplay \textstyle X\bar\fatjoin{-}-|{}:X\slice\sSet->\sSet. & 
      \mkern20mu & 
  		\adjdisplay \textstyle {-}\bar\fatjoin X-|{}:X\slice\sSet->\sSet.
    \end{align*}
    of Definitions~\ref{defn:slices} and~\ref{defn:fat-slices} are Quillen adjunctions with respect to the Joyal model structure on $\sSet$ and the corresponding sliced model structure on $X\slice\sSet$.
  \end{lem}
  
  \begin{proof}
    By~\cite[7.15]{Joyal:2007kk} it is enough to check that in each of these adjunctions the left adjoint preserves cofibrations and the right adjoint preserves fibrations between fibrant objects. From the explicit descriptions of the join and fat join, it is not difficult to see that the left adjoints preserve monomorphisms of simplicial sets. Observations~\ref{obs:slice-and-qcats} tells us that if $p \colon A \to B$ is an isofibration of quasi-categories and $f \colon X \to A$ is any simplicial map, then the induced simplicial maps $p\colon \slicer{A}{f}\to \slicer{B}{pf}$ and $p\colon \slicel{f}{A}\to \slicel{pf}{B}$ are also isofibrations of quasi-categories. The corresponding result for fat slices is a special case of Lemma \ref{lem:comma-obj-maps} below.%SAY MORE
  \end{proof}

  Finally, we arrive at the advertised comparison result relating the slice and fat slice constructions.
  
  \begin{prop}[slices and fat slices of a quasi-category are equivalent]\label{prop:slice-fatslice-equiv}
    Suppose that $X$ is any simplicial set, that $\sSet$ carries the Joyal model structure, and that $X\slice\sSet$ carries the associated sliced model structure. Then the comparison maps $s^{X,Y}\colon X\fatjoin Y\to X\join Y$ furnish us with natural transformations $s^{X,{-}}\colon X\bar\fatjoin{-}\to X\bar\join{-}$ and $s^{{-},X}\colon {-}\bar\fatjoin X\to {-}\bar\join X$ which are pointwise weak equivalences. Furthermore, these induce natural transformations  on corresponding right adjoints, whose components $e_l^f\colon \slicel{f}{A}\to \fatslicel{f}{A} $ and $e_r^f\colon \slicer{A}{f}\to \fatslicer{A}{f}$ at an object $f\colon X\to A$ of $X\slice\sSet$ are equivalences of quasi-categories whenever $A$ is a quasi-category.
  \end{prop}
  
  \begin{proof}
    The assertions involving left adjoints were proven in Proposition~\ref{prop:join-fatjoin-equiv}. The Quillen adjunctions established in Lemma~\ref{lem:slices-quillen} allow us to apply the standard result in model category theory~\cite[1.4.4]{Hovey:1999fk} that a natural transformation between left Quillen functors has components which are weak equivalences at each cofibrant object (which fact we have already established) if and only if the  induced natural transformation between the corresponding right Quillen functors has components which are weak equivalences at each fibrant object. Now simply observe that an object $f\colon X\to A$ is fibrant in $X\slice\sSet$ if and only if $A$ is a quasi-category. 
  \end{proof}
  
  \begin{rmk}\label{rmk:map-slices}
    Suppose that $f\colon B\to A$ and $g\colon C\to A$ are two simplicial maps. We generalise our slice and fat slice notation by using $\slicer{g}{f}$, $\fatslicer{g}{f}$, $\slicel{f}{g}$ and $\fatslicel{f}{g}$ to denote the objects constructed in the following pullback diagrams
    \begin{equation}
      \xymatrix@=2em{
        {\slicer{g}{f}} \pbexcursion\ar[r]\ar[d] & {\slicer{A}{f}}\ar[d]^\pi \\
        {C}\ar[r]_g & {A}
      }
      \mkern50mu
      \xymatrix@=2em{
        {\fatslicer{g}{f}} \pbexcursion\ar[r]\ar[d] & {\fatslicer{A}{f}}\ar[d]^\pi \\
        {C}\ar[r]_g & {A}
      }
      \mkern50mu
      \xymatrix@=2em{
        {\slicel{f}{g}} \pbexcursion\ar[r]\ar[d] & {\slicel{f}{A}}\ar[d]^\pi \\
        {C}\ar[r]_g & {A}
      }
      \mkern50mu
      \xymatrix@=2em{
        {\fatslicel{f}{g}} \pbexcursion\ar[r]\ar[d] & {\fatslicel{f}{A}}\ar[d]^\pi \\
        {C}\ar[r]_g & {A}
      }
    \end{equation}
    in which the maps labelled $\pi$ denote the various canonical projection maps. We call these the slices and fat slices of $g$ over and under $f$ respectively. 
     We have isomorphisms $\slicer{g\op}{f\op} \cong (\slicel{f}{g})\op$ and $\fatslicer{g\op}{f\op} \cong (\fatslicel{f}{g})\op$. % Similarly $g\op\downarrow f\op \cong (f\downarrow g)\op$.    
    
    When $A$ is a quasi-cat\-e\-go\-ry the projection maps $\pi$ are all isofibrations that commute with the comparison equivalences $e_l^f\colon \slicel{f}{A}\to \fatslicel{f}{A}$ and $e_r^f\colon \slicer{A}{f}\to \fatslicer{A}{f}$ of Proposition \ref{prop:join-fatjoin-equiv}. These maps are equivalences  of fibrant objects in the sliced Joyal model structure on $\sSet\slice A$. Pullback along any map in a model category is always a right Quillen functor of sliced model structures, so Ken Brown's lemma tells us that the pullbacks are again equivalences.
  \end{rmk}



