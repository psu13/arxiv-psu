%!TEX root=all.tex
% ***************************************************************
% ** Title:            Dom's Standard Macros
% ** Author:           Dominic Verity.
% ** Commenced:        9/7/2009
% ***************************************************************

% A useful conditional construct.
\newcommand{\ifundef}[1]{\expandafter\ifx\csname#1\endcsname\relax}

% Font fiddles

% import these fonts by hand here to avoid clashes with usual blackboard bold usage.
%\pdfmapfile{+bbold.map}
\newcommand{\bbefamily}{\fontencoding{U}\fontfamily{bbold}\selectfont}
\newcommand{\textbbe}[1]{{\bbefamily #1}}
\DeclareMathAlphabet{\mathbbe}{U}{bbold}{m}{n}

\makeatletter

\def\re@DeclareMathSymbol#1#2#3#4{%
    \let#1=\undefined
    \DeclareMathSymbol{#1}{#2}{#3}{#4}}

% Top and Bottom stolen from txsymb
\ifundef{Top}
  \DeclareSymbolFont{tcSyC}{U}{txsyc}{m}{n}
  \SetSymbolFont{tcSyC}{bold}{U}{txsyc}{bx}{n}
  \DeclareFontSubstitution{U}{txsyc}{m}{n}

  \re@DeclareMathSymbol{\Top}{\mathord}{tcSyC}{120}
  \re@DeclareMathSymbol{\Bot}{\mathord}{tcSyC}{121}
\fi

% Symbols for pushout and pullback diagram shapes stolen
% from MnSymbol
\ifundef{righthalfcup}
  \DeclareFontFamily{U}{MnSymbolC}{}
  \DeclareSymbolFont{mnSyC}{U}{MnSymbolC}{m}{n}
  \SetSymbolFont{mnSyC}{bold}{U}{MnSymbolC}{b}{n}
  \DeclareFontShape{U}{MnSymbolC}{m}{n}{
      <-6>  MnSymbolC5
     <6-7>  MnSymbolC6
     <7-8>  MnSymbolC7
     <8-9>  MnSymbolC8
     <9-10> MnSymbolC9
    <10-12> MnSymbolC10
    <12->   MnSymbolC12}{}
  \DeclareFontShape{U}{MnSymbolC}{b}{n}{
      <-6>  MnSymbolC-Bold5
     <6-7>  MnSymbolC-Bold6
     <7-8>  MnSymbolC-Bold7
     <8-9>  MnSymbolC-Bold8
     <9-10> MnSymbolC-Bold9
    <10-12> MnSymbolC-Bold10
    <12->   MnSymbolC-Bold12}{}
  
  \re@DeclareMathSymbol{\righthalfcup}{\mathord}{mnSyC}{184}
  \re@DeclareMathSymbol{\lefthalfcap}{\mathord}{mnSyC}{185}
\fi

\DeclareFontFamily{U}{MnSymbolA}{}
\DeclareSymbolFont{mnSyA}{U}{MnSymbolA}{m}{n}
\SetSymbolFont{mnSyA}{bold}{U}{MnSymbolA}{b}{n}
\DeclareFontShape{U}{MnSymbolA}{m}{n}{
    <-6>  MnSymbolA5
   <6-7>  MnSymbolA6
   <7-8>  MnSymbolA7
   <8-9>  MnSymbolA8
   <9-10> MnSymbolA9
  <10-12> MnSymbolA10
  <12->   MnSymbolA12}{}
\DeclareFontShape{U}{MnSymbolA}{b}{n}{
    <-6>  MnSymbolA-Bold5
   <6-7>  MnSymbolA-Bold6
   <7-8>  MnSymbolA-Bold7
   <8-9>  MnSymbolA-Bold8
   <9-10> MnSymbolA-Bold9
  <10-12> MnSymbolA-Bold10
  <12->   MnSymbolA-Bold12}{}

\re@DeclareMathSymbol{\twoheadedswarrow}{\mathord}{mnSyA}{30}

\makeatother

\def\sdagger{{\!\text{\mdseries\textdagger}}}

% ***************************************************************
% ** Description:      Miscellaneous bits and pieces.            
% ***************************************************************

% *** Now some general definitions ***

% *** Fiddling with boxes, depths etc. ***

\newcommand{\mlaux}[3]{\setbox0=\hbox{$\mathsurround=0pt #2{#3}$}%
  \dimen0=\dp0\advance\dimen0 by \ht0\lower#1\dimen0\box0}
\newcommand{\mlower}[2]{\mathpalette{\mlaux{#1}}{#2}}

\newcommand{\makellapm}[2]{\hbox to 0pt{\hss$\mathsurround=0pt #1{#2}$}}
\newcommand{\llapm}{\relax\mathpalette\makellapm}
\newcommand{\makerlapm}[2]{\hbox to 0pt{$\mathsurround=0pt #1{#2}$\hss}}
\newcommand{\rlapm}{\relax\mathpalette\makerlapm}
\newcommand{\makelapm}[2]{\hbox to 0pt{\hss$\mathsurround=0pt #1{#2}$\hss}}
\newcommand{\lapm}{\relax\mathpalette\makelapm}

\newcommand{\makeushort}[3]{%
	\setbox0=\hbox{$\mathsurround=0pt #2{#3}$}%
	\hbox to 1\wd0{\hss\underbar{\hbox to #1\wd0{\hss\box0\hss}}\hss}}
\newcommand{\ushort}[1]{\relax\mathpalette{\makeushort{#1}}}

% Macro to typeset part of a math formula at a bigger size
\def\makebigger#1#2#3{\scalebox{#1}{$\mathsurround=0pt #2{#3}$}}
\def\bigger#1#2{{\relax\mathpalette{\makebigger{#1}}{#2}}}

\def\scaleuphalf{1.0954}
\def\scaleupone{1.2}
\def\scaleuptwo{1.44}

% Duals / Superscripted postfix ops

\newcommand{\dual}{^\circ}
\newcommand{\oth}{^{\mathord{\text{th}}}}
\newcommand{\ost}{^{\mathord{\text{st}}}}
\newcommand{\ond}{^{\mathord{\text{nd}}}}
\newcommand{\op}{^{\mathord{\text{\rm op}}}}
\newcommand{\co}{^{\mathord{\text{\rm co}}}}
\newcommand{\coop}{^{\mathord{\text{\rm coop}}}}
\newcommand{\vop}{^{\mathord{\text{\rm vop}}}}
\newcommand{\hop}{^{\mathord{\text{\rm hop}}}}
\newcommand{\hvop}{^{\mathord{\text{\rm hvop}}}}
\newcommand{\refld}{^{\mathord{\text{\rm r}}}}
\newcommand{\rev}{^{\mathord{\text{\rm rev}}}}
\newcommand{\rhv}{^r}
\newcommand{\lhv}{^l}
\newcommand{\eqv}{^e}
\newcommand{\tr}{^{\text{\rm t}}}
\newcommand{\mapcat}{^\cattwo}
\newcommand{\fp}{_{\mathord{\text{\em fp}}}}

% ordinal stuff

\newcommand{\st}{^{\text{st}}}
\newcommand{\nd}{^{\text{nd}}}
\newcommand{\rd}{^{\text{rd}}}
\renewcommand{\th}{^{\text{th}}}

% General mathematical connectives etc.

\newcommand{\orelse}{\mathrel{\text{or}}}
\newcommand{\also}{\mathrel{\text{and}}}
\newcommand{\defeq}{\mathrel{:=}}
\newcommand{\eqdef}{\mathrel{=:}}

% ***************************************************************
% ** Description:      Some useful mathematical operators.            
% ***************************************************************

\makeatletter

\def\newmop{\@ifstar{\@newmop m}{\@newmop o}}
\def\@newmop#1{\@ifnextchar[{\@@newmop #1}{\@@@newmop #1}}
\def\@@newmop#1[#2]{\@declmathop #1#2}
\def\@@@newmop#1#2{\expandafter\@declmathop\expandafter #1\csname #2\endcsname{#2}}

\makeatother

%new xypic tails
%\newdir{ >}{{}*!/-10pt/@{>}}
\newdir{ |}{{}*!/-5pt/@{|}}

% General operations on maps etc.
\newmop{im}
\newmop{coim}
\newmop{dom}
\newmop{cod}
\newmop{id}
\newmop{Map}

\newmop{obj}
\newmop{arr}
\newmop{sq}
\newmop{norm}

\newmop{el}
\def\card{\#}

\newmop{ev}

%Misc

\newmop{Ext}
\newmop{icon}
\newmop{pbk}

% (Weak) factorisation systems.
\newmop{cell}
\newmop{cof}
\newmop{fib}

% Bisimplicial sets
\newmop{diag}
\def\Wbar{\overline{W}}

% Colimits and limits
\newmop*{colim}
\newmop*{holim}

% Kan extension
\newmop[\lan]{lan}
\newmop[\ran]{ran}

% ***************************************************************
% ** Description:      General categorical notations
% ***************************************************************

\newcommand{\comma}{\mathbin{\downarrow}}

\newcommand{\unit}{\eta}
\newcommand{\counit}{\varepsilon}

\makeatletter
\newcommand{\rotatemath}[2]{\rotatebox[origin=c]{180}{$\m@th #1{#2}$}}
\makeatother

\newcommand{\Coproj}{\mathpalette\rotatemath\Pi}
\newcommand{\cprod}{\mathbin{\mathpalette\rotatemath\Pi}}
\newcommand{\ip}{\mathord{\mathpalette\rotatemath\pi}}

\newcommand{\yonmap}[1]{\ulcorner{#1}\urcorner}
\newcommand{\kancon}{\tilde}

\newcommand{\yoneda}{\mathscr{Y}\!}

% projections for products, pullbacks and comma objects
\newmop[\pr]{pr}
\newmop[\dm]{d} 

% inclusions for coproducts and pushouts
\newmop[\cpr]{in}

% Pushout and Pullback ``corners'' for xypic diagrams.

\newcommand{\pocorner}{\hbox to 8pt{{\vrule height8pt depth0pt width0.5pt}%
    \vbox to 8pt{{\hrule height0.5pt width7.5pt depth0pt}\vfill}}}
\newcommand{\poexcursion}{\save[]-<15pt,-15pt>*{\pocorner}\restore}
\newcommand{\pbcorner}{\vbox to 0pt{\kern 4pt\hbox to 0pt{\kern 4pt%
      \vbox{{\hrule height0.5pt width7.5pt depth0pt}}%
      {\vrule height8pt depth0pt width0.5pt}\hss}\vss}}
\newcommand{\pbexcursion}{\save[]+<5pt,-5pt>*{\pbcorner}\restore}
\newcommand{\pbdiamond}{\save[]+<0pt,-5pt>*{\rotatebox{-45}{$\pbcorner$}}\restore}


% ***************************************************************
% ** Description:      Tensors, actions etc.            
% ***************************************************************

\newcommand{\hact}{\cdot_h}
\newcommand{\vact}{\cdot_v}
\newcommand{\act}[1]{\cdot_{#1}}

\newcommand{\pwr}{\pitchfork}
\newcommand{\tns}{\ast}
\newcommand{\wcolim}{\circledast}
\newcommand{\leibwcolim}{\leib\wcolim}
\newcommand{\wlim}[2]{\{#1,#2\}}
\newcommand{\leibwlim}[2]{\{#1,#2\}^{\wedge}}

\newmop{cls}

\newcommand{\leib}[1]{\mathbin{\widehat{#1}}}
\newcommand{\wleib}{\widehat}

\newcommand{\etimes}{\mathbin{\ushort{0.5}{\mathord\times}}}

\newcommand{\pbtimes}[1]{\mathbin{\mathop{\times}_{#1}}}
\newcommand{\pbotimes}[1]{\mathbin{\mathop{\otimes}_{#1}}}

% ***************************************************************
% ** Description:      Standard categories. 
% ***************************************************************

% Macros for typesetting the names of different kinds of category

\newcommand{\category}[1]{\underline{\smash[b]{\text{\rm{#1}}}}}
\newcommand{\bicat}[1]{\underline{\smash[b]{\mathcal{#1}}}}
\newcommand{\twocat}[1]{\bicat{\text{\rm\em #1}}}
\newcommand{\tcat}{\lcat}
\newcommand{\bifun}{\underline}
\newcommand{\trans}{\underline}
\newcommand{\icat}{\mathbb}
\newcommand{\lcat}{\mathcal}
\newcommand{\scat}{\mathbf}
\newcommand{\stcat}{\scat}
\newcommand{\sspan}[2]{{}_{#1}\scat{sSet}_{#2}}
\newcommand{\qspan}[2]{{}_{#1}\scat{qCat}_{#2}}

\newcommand{\catthree}{{\bigger{1.12}{\mathbbe{3}}}}
\newcommand{\cattwo}{{\bigger{1.12}{\mathbbe{2}}}}
\newcommand{\catone}{{\bigger{1.16}{\mathbbe{1}}}}
\newcommand{\iso}{{\bigger\scaleuphalf{\mathbb{I}}}}

% Magic with Delta

\makeatletter

\def\Del@Sym{{\bigger\scaleuphalf{\mathbbe{\Delta}}}}

\def\del@fn{\futurelet\del@next}
\def\del@dn{\def\del@next}

\def\parsedel@{%
  \ifx +\del@next \del@dn+{\Del@Sym_{\mathord{+}}}%
  \else \del@dn {\del@fn\parsedel@@}%
  \fi\del@next}

\def\parsedel@@{%
  \ifx\space@\del@next \expandafter\del@dn\space{\del@fn\parsedel@@}%
  \else\ifx [\del@next \del@dn[{\del@fn\parsedel@@@}%
  \else\ifx _\del@next \del@dn{\Delta}%
  \else\ifx ^\del@next \del@dn{\Delta}%
  \else \del@dn{\Del@Sym}%
  \fi\fi\fi\fi\del@next}

\def\parsedel@@@{%
  \ifx\space@\del@next \expandafter\del@dn\space{\del@fn\parsedel@@@}%
  \else\ifx t\del@next \del@dn t{\Del@Sym_\infty\del@fn\parsedel@@@@}%
  \else\ifx b\del@next \del@dn b{\Del@Sym_{-\infty}\del@fn\parsedel@@@@}%
  \else \del@dn{\errmessage{unexpected modifier}}%
  \fi\fi\fi\del@next}

\def\parsedel@@@@{%
  \ifx\space@\del@next \expandafter\del@dn\space{\del@fn\parsedel@@@@}%
  \else\ifx ]\del@next \del@dn]{}%
  \else \del@dn{\errmessage{expecting close of option block}}%
  \fi\fi\del@next}

\def\Del{\del@fn\parsedel@}

\def\DelTop{\Del[t]}
\def\DelBot{\Del[b]}

\makeatother

\newcommand{\Horn}{\Lambda}

% Standard categories

\newcommand{\aDelta}{\Del+}

\newcommand{\Set}{\category{Set}}
\newcommand{\Cat}{\category{Cat}}

\newcommand{\fact}{\category{fact}}

\newcommand{\eCat}[1]{#1\text{-}\Cat} % Enriched categories

\newcommand{\Gph}{\category{Gph}}
\newcommand{\sCat}{\sSet\text{-}\Cat}
\newcommand{\sSet}{\category{sSet}}
\newcommand{\ssSet}{\category{ssSet}}%bisimplicial sets
\newcommand{\qCat}{\category{qCat}}
\newcommand{\Adj}{\category{Adj}}
\newcommand{\Mnd}{\category{Mnd}}
\newcommand{\Cmd}{\category{Cmd}}
\newcommand{\asSet}{\sSet_{\mathord{+}}}
\newcommand{\msSet}{\category{msSet}} % marked simplicial sets = strat
\newcommand{\amsSet}{\msSet_{\mathord{+}}}

\newcommand{\twoCat}{\eCat{2}}

% legacy names for a few things
\newcommand{\Simp}{\sSet}
\newcommand{\aSimp}{\asSet}
\newcommand{\sSimp}{\category{sSimp}}
\newcommand{\Strat}{\sSimp}
\newcommand{\asSimp}{\sSimp_{\mathord{+}}}

\newcommand{\Span}[1]{\category{Span}(#1)}
\newcommand{\Mod}[2]{{}_{#1}\category{Mod}_{#2}}
\newcommand{\qMod}[2]{{}_{#1}\category{Mod}_{#2}'}

\newcommand{\BiSimp}{\category{BiSimp}}

\newcommand{\genericarr}{\{\bullet\to\bullet\}}

% Notation for Adj related structures
\newcommand{\Atom}{\text{Atom}}
\newcommand{\Fillable}{\text{Fill}}

% Shapes of diagrams for pullbacks and pushouts
\newcommand{\pbshape}{{\mathord{\bigger\scaleupone\righthalfcup}}}
\newcommand{\poshape}{{\mathord{\bigger\scaleupone\lefthalfcap}}}

% ***************************************************************
% ** Description:      Simplicial Set / Model Category notation.            
% ***************************************************************

% Elementary operators in the theory of (stratified) simplicial sets.

\newcommand{\face}{\delta}
\newcommand{\vertex}{\nu}
\newcommand{\degen}{\sigma}
\newcommand{\aug}{\iota}
\newcommand{\tdegen}{\varsigma}
\newcommand{\thop}{\varrho}
\newcommand{\genop}[1]{\{{#1}\}}

\newcommand{\faceact}{\mathrm{d}}
\newcommand{\degenact}{\mathrm{s}}
\newcommand{\vertexact}{\mathrm{v}}
\newcommand{\tdegenact}{\mathrm{e}}
\newcommand{\augact}{\mathrm{i}}

\newcommand{\ssub}{\subseteq_s}

% face by vertices (fbv)
\newcommand{\sembl}{\mathopen{\mathord[\mkern-3mu\mathord[}}
\newcommand{\sembr}{\mathclose{\mathord]\mkern-3mu\mathord]}}
\newcommand{\fbv}[1]{\{{#1}\}}

% Partition operators.

\newcommand{\partinj}{\Bot}
\newcommand{\partproj}{\Top}

% other simplicial notation

\newcommand{\join}{\star}
\newcommand{\fatjoin}{\mathbin\diamond}
\newmop{dec}
\newmop{fatdec}
\newmop{slc}
\newmop{fatslc}
\newcommand{\fatslice}{\mathbin{\mkern-1mu{/}\mkern-5mu{/}\mkern-1mu}}
\newcommand{\fatslicel}[2]{\vphantom{#2}^{{#1}\fatslice{}}\mkern-2mu{#2}}
\newcommand{\fatslicer}[2]{{#1\mkern-1mu}_{{}\fatslice{#2}}}
\newcommand{\slice}{/}
\newcommand{\slicel}[2]{\vphantom{#2}^{{#1}\slice{}}\mkern-2mu{#2}}
\newcommand{\slicer}[2]{{#1\mkern-1mu}_{{}\slice{#2}}}
\newcommand{\slicelr}[3]{\vphantom{#2}^{{#1}\slice{}}\mkern-2mu{#2}_{{}\slice{#3}}}
\newcommand{\underlie}{\overline}
\newmop{ir} % the interval representation functor
\newmop{incl}

% nerves etc
\newcommand{\nrv}{N}
\newcommand{\ho}{h}
\newcommand{\kan}{k}

\newcommand{\nrvhc}{\nrv}
\newcommand{\gC}{\mathfrak{C}}%left adjoint to the homotopy coherent nerve

% Notation associated with Reedy categories

\newcommand{\Reedy}{\category{Reedy}}

\newcommand{\boundary}{\partial}
\newcommand{\coboundary}{\dot\partial}
\newcommand{\direct}{\overrightarrow}
\newcommand{\inverse}{\overleftarrow}
\newcommand{\reedycat}[1]{(\scat{#1},\direct{\scat{#1}},\inverse{\scat{#1}})}
\newcommand{\reedyclass}[3]{#1^{#2}_#3}
\newcommand{\twar}[1]{\scat{#1}\wr\scat{#1}}

\newcommand{\latch}{L}
\newcommand{\match}{M}

\newcommand{\bdrymap}{b}
\newcommand{\hornmap}{h}

\newcommand{\cofibrep}{L}
\newcommand{\fibrep}{R}
\newcommand{\coresol}{\Psi}
\newcommand{\resol}{\Phi}

\newcommand{\cfr}{_{\text{\rm c}}}
\newcommand{\fbr}{_{\text{\rm f}}}
\newcommand{\cfbr}{_{\text{\rm fc}}}
\newcommand{\srs}{_{\text{\rm sr}}}
\newcommand{\crs}{_{\text{\rm cr}}}

\def\reedyfilt#1_#2{#1_{\leq #2}}
\newmop{sk}
\newmop{cosk}
\newmop{res}

%decoration
\newcommand{\cart}{{\mathrm{cart}}}

\newcommand{\homsp}[1]{\hom_{\lcat{#1}}}

% General model category notation

\newcommand{\mclass}{\mathcal}
\newcommand{\lp}{\mathrel\square}
\newcommand{\lpclass}{{}^\square}

\newcommand{\Mono}{\mclass{M}\mkern-2mu\text{\it ono}}
\newcommand{\Epi}{\mclass{E}\mkern-2mu\text{\it pi}}

\newmop{map}
\newmop{Ho}

\newcommand{\cpts}{\pi_0}

\newmop[\pth]{path}
\newmop{cyl}

% Standard notations for operations on (iterated) functor categories

\newmop[\const]{c}
  
% Some standard model categories

\newcommand{\Kan}{\category{Kan}}
\newcommand{\Quasi}{\category{Quasi}}
\newcommand{\Diag}{\category{Diag}}

\newcommand{\dcsQuasi}{\category{dcsQuasi}}

% Augmentation
\def\iaug{^\bot}
\def\taug{^\top}

% collage and its right adjoint 
% (NB: this is probably in the wrong place)

\newmop{coll}
\newmop{wgt}

% spaces of homotopy coherent adjunctions

\newcommand{\counits}{\mathrm{counit}}
\newcommand{\cohadjs}{\mathrm{cohadj}}
\newcommand{\leftadjs}{\mathrm{leftadj}}
%\newmop{counits}
%\newmop{cohadjs}
%\newmop{leftadjs}

% ***************************************************************
% ** Description:      In-line arrows.            
% ***************************************************************

\newdir{ >}{{}*!/-7pt/@{>}}
\newdir{u(}{{}*!/-4pt/@^{(}}
\newdir{d(}{{}*!/-4pt/@_{(}}
\newdir{|>}{%
  !/4.5pt/@{|}*:(1,-.2)@^{>}*:(1,+.2)@_{>}*+@{}}

% New style, simpler, inline arrows. To match these in xypic diagrams
% use cm arrow heads there.

\makeatletter
\def\makeslashed#1#2#3#4#5{#1{\mathpalette{\sla@{#2}{#3}{#4}}{#5}}}

\def\@mathlower#1#2#3{\setbox0=\hbox{$\m@th#2#3$}\lower#1\ht0\box0}
\def\mathlower#1#2{\mathpalette{\@mathlower{#1}}{#2}}
\makeatother


\newcommand{\epi}{\twoheadrightarrow}
\newcommand{\inc}{\hookrightarrow}
\newcommand{\mono}{\rightarrowtail}
\newcommand{\tcof}{\hookrightarrow}
\newcommand{\tfib}{\twoheadrightarrow}
\newcommand{\fibt}{\twoheadleftarrow}

\newcommand{\xtfib}[1]{\xtwoheadrightarrow{#1}}
\newcommand{\xfibt}[1]{\xtwoheadleftarrow{#1}}

\newcommand{\prof}{\makeslashed\mathbin\shortmid{-0.1}{0.16}\to}
\newcommand{\mat}{\makeslashed\mathbin\circ{-0.1}{0.1}\to}

\newcommand{\we}{\xrightarrow{\mkern10mu{\smash{\mathlower{0.6}{\sim}}}\mkern10mu}}
\newcommand{\weleft}{\xleftarrow{\mkern10mu{\smash{\mathlower{0.6}{\sim}}}\mkern10mu}}
\newcommand{\trvfib}{\xtwoheadrightarrow{\smash{\mathlower{1.2}{\sim}}}}
\newcommand{\trvcof}{\xhookrightarrow{\mkern8mu{\smash{\mathlower{1}{\sim}}}\mkern12mu}}

% 2-cells

\newcommand{\To}{\Rightarrow}

% ***************************************************************
% ** Description:      Macros to support mixed variance tensor 
% **                   style sub/super-script notation            
% ***************************************************************

% Note - the \tn macro is not "re-entrant".

\makeatletter

\def\tens@fn{\futurelet\tens@next}
\def\tens@dn{\def\tens@nextcont}
\newtoks\tens@toks
\def\addtotens@toks#1{\tens@toks=\expandafter{\the\tens@toks#1}}

\def\parsetens@@{%
    \ifx\space@\tens@next \expandafter\tens@dn\space{\tens@fn\parsetens@@}%
    \else\ifx ^\tens@next \tens@dn ^##1{\parsetens@procsep^\addtotens@toks{##1}%
      \tens@fn\parsetens@@}%
    \else\ifx _\tens@next \tens@dn _##1{\parsetens@procsep_\addtotens@toks{##1}%
      \tens@fn\parsetens@@}%
    \else\tens@dn{\ifx *\tens@last \else\addtotens@toks\egroup\fi\the\tens@toks}%
    \fi\fi\fi\tens@nextcont}

\def\parsetens@procsep#1{%
  \ifx *\tens@last \addtotens@toks{#1}\addtotens@toks\bgroup%
  \else\ifx \tens@last\tens@next \addtotens@toks,%
  \else \addtotens@toks\egroup\addtotens@toks\bgroup%
    \addtotens@toks\egroup\addtotens@toks{#1}\addtotens@toks\bgroup% 
  \fi\fi\let\tens@last\tens@next}

\newcommand{\tn}[1]{\let\tens@last=*\tens@toks={#1}\tens@fn\parsetens@@}

\makeatother

\newcommand{\tlambda}{\tn\lambda}

% ***************************************************************
% ** Description:      Some standard xypic diagrams            
% ***************************************************************

\def\adjdisplay#1-|#2:#3->#4.{{%
    \xymatrix@R=0em@!C=2.5em{%
      *+[l]{#3} \ar@/_0.55pc/[rr]_-{#2} & {\bot} &
      *+[r]{#4}\ar@/_0.55pc/[ll]_-{#1}}}}

\def\adjdisplaytwo#1-|#2:#3->#4.{{% 
\xymatrix@=1.2em{
      {#3}\ar@/_1.5ex/[rr]_-{#2}^-{}="one"
      & & {#4}
      \ar@/_1.5ex/[ll]_-{#1}^-{}="two" 
      \ar@{}"one";"two"|{\bot}
    }}}

\def\tripleadjdisplay#1-|#2-|#3:#4->#5.{{%
\xymatrix@=2.4em{ 
{#4}\ar[r]|{#2} &
{#5} \ar@/_3ex/[l]_{#1}^{\bot} \ar@/^3ex/[l]_{\bot}^{#3}}
}}

\def\adjinline#1-|#2:#3->#4.{{#1}\dashv{#2}:#3\to #4}

\newcommand{\pent}[1]{
  \xybox{
    \POS (0,-15)*+{\a}="0", 
         (-14,-5)*+{\b}="1", 
         (-9,12)*+{\c}="2", 
         (9,12)*+{\d}="3", 
         (14,-5)*+{\e}="4"
    \POS"0" \ar "1"^{\labelstyle \ab}|{}="01"
    \POS"1" \ar "2"^{\labelstyle \bc}|{}="12"
    \POS"2" \ar "3"^{\labelstyle \cd}|{}="23"
    \POS"3" \ar "4"^{\labelstyle \de}|{}="34"
    \POS"0" \ar "4"_{\labelstyle \ae}|{}="04"
    \ifcase #1
    \POS"0" \ar "2"|{\labelstyle \ac}="02"
    \POS"0" \ar "3"|{\labelstyle \ad}="03"
    \POS"02";"1"**{}, ?(0.3) \ar@{=>} ?(0.7)^{\labelstyle \abc}
    \POS"03";"2"**{}, ?(0.25) \ar@{=>} ?(0.5)_{\labelstyle \acd}
    \POS"04";"3"**{}, ?(0.2) \ar@{=>} ?(0.4)_{\labelstyle \ade}
    \or
    \POS"1" \ar "3"|{\labelstyle \bd}="13"
    \POS"1" \ar "4"|{\labelstyle \be}="14"
    \POS"13";"2"**{}, ?(0.3) \ar@{=>} ?(0.7)_{\labelstyle \bcd}
    \POS"14";"3"**{}, ?(0.25) \ar@{=>} ?(0.5)_{\labelstyle \bde}
    \POS"04";"1"**{}, ?(0.25) \ar@{=>} ?(0.5)_{\labelstyle \abe}
    \or
    \POS"2" \ar "4"|{\labelstyle \ce}="24"
    \POS"0" \ar "2"|{\labelstyle \ac}="02"
    \POS"02";"1"**{}, ?(0.3) \ar@{=>} ?(0.7)^{\labelstyle \abc}
    \POS"04";"2"**{}, ?(0.2) \ar@{=>} ?(0.35)_{\labelstyle \ace}
    \POS"24";"3"**{}, ?(0.2) \ar@{=>} ?(0.6)^{\labelstyle \cde}
    \or
    \POS"1" \ar "3"|{\labelstyle \bd}="13"
    \POS"0" \ar "3"|{\labelstyle \ad}="03"
    \POS"04";"3"**{}, ?(0.2) \ar@{=>} ?(0.4)_{\labelstyle \ade}
    \POS"13";"2"**{}, ?(0.3) \ar@{=>} ?(0.7)_{\labelstyle \bcd}
    \POS"03";"1"**{}, ?(0.25) \ar@{=>} ?(0.5)^{\labelstyle \abd}
    \or
    \POS"2" \ar "4"|{\labelstyle \ce}="24"
    \POS"1" \ar "4"|{\labelstyle \be}="14"
    \POS"24";"3"**{}, ?(0.2) \ar@{=>} ?(0.6)^{\labelstyle \cde}
    \POS"04";"1"**{}, ?(0.25) \ar@{=>} ?(0.5)_{\labelstyle \abe}
    \POS"14";"2"**{}, ?(0.25) \ar@{=>} ?(0.5)^{\labelstyle \bce}
    \else\fi
  }
}

\newcommand{\pentofpent}[1]{
  \def\baselen{#1}
  \begin{xy}
    0;<\baselen,0mm>:
    *{\xybox{
        \POS(0,-4)*[o]{\pent 0}="zero"
        \POS(16,40)*[o]{\pent 3}="three"
        \POS(72,40)*[o]{\pent 1}="one"
        \POS(88,-4)*[o]{\pent 4}="four"
        \POS(44,-36)*[o]{\pent 2}="two"
        \ar@<1ex>"zero";"three"^-{\objectstyle\abcd}
        \ar@<1ex>"three";"one"^-{\objectstyle\abde}
        \ar@<1ex>"one";"four"^-{\objectstyle\bcde}
        \ar@<-1ex>"zero";"two"_-{\objectstyle\acde}
        \ar@<-1ex>"two";"four"_-{\objectstyle\abce}
        \ar@{=>}(44,-5);(44,+15)^{\objectstyle\abcde}
     }}
  \end{xy}
}

% Local Variables:
% mode: LaTeX
% TeX-master: "main.tex"
% TeX-PDF-mode: t
% TeX-parse-self: t
% TeX-auto-save: t
% End: 
