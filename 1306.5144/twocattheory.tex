%!TEX root = all.tex
% ******************************************************************
% ** Title:            The 2-category theory of quasi-categories
% **                   2-Categorical Arguments
% ** Precis:        
% ** Author:           Emily Riehl and Dominic Verity
% ** Commenced:        2/3/2012
% ******************************************************************

  \section{The 2-category of quasi-categories}\label{sec:twocat}

  The full subcategory $\qCat$ of quasi-categories and functors is closed in $\sSet$ under products and internal homs. 
   It follows that $\qCat$ is cartesian closed and that it becomes a full simplicial sub-category of $\sSet$ under its usual self enrichment. We denote this self-enriched category of quasi-categories, whose simplicial hom-spaces are given by exponentiation, by $\qCat_\infty$.

  In this section, we study a corresponding (strict) 2-category of quasi-categories $\qCat_2$ first introduced by Joyal \cite{Joyal:2008tq}. This should be thought of as being a kind of quotient of $\qCat_\infty$ whose 2-cells (1-arrows in the hom-spaces) are replaced by homotopy classes of such and in which higher dimensional information in the hom-spaces is discarded.  At first blush, it might seem that such a process would destroy far too much information to be of any great use. However, much of this paper is devoted to showing, perhaps quite surprisingly, that we may develop a great deal of the elementary category theory of quasi-categories within the 2-category $\qCat_2$ alone. Our first step in this direction will be to recognise that much of this category theory may be encoded in the weak 2-universal properties of certain constructions in this 2-category.

  In this section, we introduce the 2-category $\qCat_2$ of quasi-categories and establish a few of its basic properties. In particular, we define a particular notion of weak 2-limit appropriate to this context and show that $\qCat_2$ admits certain weak 2-limit constructions. In later sections, we use the structures introduced here to transport classical categorical proofs into the quasi-categorical context. 

	\subsection{Relating 2-categories and simplicially enriched categories}
	
\begin{ntn}[simplicial categories and 2-categories]
  The category of simplicial sets $\sSet$ is complete, cocomplete, and cartesian closed, so in particular it supports a well developed enriched category theory. We refer to $\sSet$-enriched categories simply as {\em simplicial categories\/} and the enriched functors between them as {\em simplicial functors}.

In a simplicial category $\tcat{C}$, we call the $n$-simplices of one of its simplicial hom-spaces $\tcat{C}(A,B)$ its {\em $n$-arrows\/} from $A$ to $B$. The composition operation of $\tcat{C}$ restricts to make the graph of the objects and $n$-arrows of $\tcat{C}$ into a category which we shall call $\tcat{C}_n$, for which $\tcat{C}_n(A,B) = \tcat{C}(A,B)_n$. Furthermore, if $\alpha\colon[n]\to[m]$ is a simplicial operator then its action on arrows gives rise to an identity-on-objects functor $\tcat{C}_m\to\tcat{C}_n$.

The category of all (small) categories $\Cat$ is also complete, cocomplete, and cartesian closed, so it too supports an enriched category theory. We refer to $\Cat$-enriched categories as {\em 2-categories\/} and the enriched functors between then as {\em 2-functors}. In a 2-category $\tcat{C}$, we follow convention and refer to its objects as {\em 0-cells}, the objects in its hom-categories as {\em 1-cells}, and the arrows in its hom-categories as {\em 2-cells}. 

  We refer the reader to Kelly's canonical tome~\cite{Kelly:2005:ECT} for the standard exposition of the yoga of enriched category theory. We also strongly recommend Kelly and Street~\cite{kelly.street:2} and Kelly~\cite{Kelly:1989fk} as elementary introductions to 2-categories and their attendant 2-limit notions. In particular, we encourage the reader to familiarise him- or herself with the rubric of pasting composition discussed in~\cite{kelly.street:2}.
\end{ntn}


  Recollection~\ref{rec:hty-category} reminds us that $\Cat$ may be regarded as a reflective subcategory of $\sSet$, or indeed $\qCat$, via the adjunction $\ho \dashv \nrv$: the natural map $X \to hX$ is an isomorphism if and only if $X$ is (the nerve of) a category.  The fact that $h\colon \sSet\to \Cat$ preserves binary products implies, and in fact is equivalent to, the observation that if $C$ is a category and $X$ is a simplicial set then their internal hom $C^X$ in $\sSet$ is again a category. The proof in \cite[B.0.16]{Joyal:2008tq} is as follows: there is a canonical map of simplicial sets $C^{hX} \to C^X$. Fixing $X$ and varying $C$ these maps define the components of a natural transformation between two right adjoints $\Cat\to\sSet$. This map is invertible because the transposed natural transformation $h(X\times Y) \to hX \times hY$ is an isomorphism.

Recollection~\ref{rec:cart-modcat} tells us the corresponding result for quasi-categories, this being that internal homs whose target objects are quasi-categories are themselves quasi-categories. In particular, it follows that each of the categories $\Cat$, $\qCat$, and $\sSet$ is cartesian closed and that the various inclusions of one into another preserve finite products and internal homs. In particular, we may regard the self-enriched categories $\Cat$ and $\qCat$ as being full simplicial subcategories of $\sSet$ under its self enrichment.  We write $\Cat_2$ for this 2-category of categories, regarded as a full subcategory of $\qCat_\infty$. 


\begin{obs}\label{obs:simp-to-2-cats}
Using the fact that $\ho$ and $\nrv$ both preserve finite products, we may construct an induced adjunction
	\begin{equation*}
		\adjdisplay \ho_*-|\nrv_*:\twoCat->\sCat. 
	\end{equation*}
	between the categories of 2-categories and simplicial categories respectively. The functors in this adjunction are obtained by applying $\nrv$ and $\ho$  to the hom-objects of an enriched category on one side of this adjunction to obtain a corresponding enriched category on the other side. Here again $\nrv_*$ is fully faithful, so it is natural to regard $\twoCat$ as being a reflective full subcategory both of $\sCat$ and of its full subcategory $\qCat$--$\Cat$ of categories enriched in quasi-categories. Indeed, for our purposes here it suffices to consider the restricted adjunction \[ \adjdisplay \ho_* -| \nrv_* : \twoCat -> \eCat\qCat.\] 

Given a quasi-categorically enriched category $\tcat{C}$, the 2-category $\ho_*\tcat{C}$ is a quotient of sorts. The underlying unenriched categories of $\tcat{C}$ and $\ho_*\tcat{C}$ coincide, but 2-cells in $\ho_*\tcat{C}$ are homotopy classes of 1-arrows in $\tcat{C}$. These homotopy classes are defined using relations witnessed by the 2-arrows. All higher dimensional cells are discarded. On regarding $\ho_*\tcat{C}$ as a simplicially enriched category we see that the unit of the adjunction $\ho_*\dashv\nrv_*$ provides us with a canonical simplicial quotient functor $\tcat{C}\to\ho_*\tcat{C}$.
\end{obs}

  Our identification of categories with their nerves also leads us to regard 2-categories as certain special kinds of simplicial categories. Under this identification, a 1-cell (resp.\ 2-cell) in a 2-category can equally well be regarded as being a 0-arrow (resp.\ 1-arrow) in the corresponding simplicial category.  

\subsection{The 2-category of quasi-categories}

\begin{defn}[the 2-category of quasi-categories]\label{def:qCat-2}
  In particular, applying the functor $\ho_*$ to the quasi-categorically enriched category $\qCat_\infty$, we obtain an associated 2-category $\qCat_2 := h_*\qCat_\infty$ whose hom-categories are given by
  \begin{equation}
    \label{eq:qCat2homdefn} \hom'(A,B) \defeq \ho(B^A).
  \end{equation} 
Using the  description of $h$ given in \ref{rec:hty-category}, we find that the objects of $\qCat_2$ are quasi-categories; the 1-cells are maps of quasi-categories, which we have agreed to call \emph{functors}; and the 2-cells, which we shall call \emph{natural transformations}, are certain homotopy classes of 1-simplices in the internal hom $B^A$. 

  More explicitly, a 2-cell $f\Rightarrow g$ between parallel functors $f,g \colon A \rightrightarrows B$  is an equivalence class represented by a simplicial map $\alpha\colon A\times \Del^1\to B$ making the following diagram \[ \vcenter{ \xymatrix{ A \times \Del^0 \cong A \ar[dr]^f \ar[d]_{A\times\face^1} \\ A \times \Del^1 \ar[r]^-\alpha & B \\ A \times \Del^0 \cong A \ar[ur]_g \ar[u]^{A\times\face^0}}}\] commute. The displayed map $\alpha$ is a 1-simplex in $B^A$ from the vertex $f$ to the vertex $g$. Two such 1-simplices represent the same 2-cell if and only if they are connected by a homotopy (in the sense of \eqref{eq:homotopy-of-1-simplices}) which fixes their common domain $f$ and  codomain $g$. 

We adopt common 2-categorical notation, writing $\alpha\colon f\Rightarrow g$ to denote a 2-cell of $\qCat_2$ which is represented by a simplicial map $\alpha\colon A\times \Del^1\to B$. So if $\alpha,\beta\colon f\Rightarrow g$ are two such represented 2-cells then when we write $\alpha=\beta$ we will not mean that any two particular representing maps $\alpha,\beta\colon A\times\Del^1\to B$ are literally equal but instead that they are appropriately homotopic.

  The 2-category $\qCat_2$ and the simplicial category $\qCat_\infty$ both have the same underlying ordinary category $\qCat$. Furthermore, we know that if $A$ and $B$ are both categories regarded as quasi-categories (via the nerve functor) then $B^A\in \qCat$ is also a category and so $B^A\cong \ho(B^A)$. This in turn implies that the full sub-2-category of $\qCat_2$ spanned by the categories is itself equivalent to $\Cat_2$; we shall identify these from here on. 
  
  The fact that the homotopy category functor $\ho$ preserves finite products allows us to canonically enrich it to a simplicial functor $\ho\colon\qCat_\infty\to\Cat_2$. Specifically we take its action on the hom-space $B^A$ to be the map obtained as the adjoint transpose of the composite $\ho(B^A)\times\ho(A)\cong\ho(B^A\times A)\stackrel{\ho(\ev)}\longrightarrow\ho(B)$.
\end{defn}

\begin{obs}[pointwise isomorphisms are isomorphisms (reprise)]\label{obs:pointwise-iso-reprise}
  We say that a 2-cell $\alpha\colon f\Rightarrow g\colon A\to B$ of $\qCat_2$ is a {\em pointwise isomorphism\/} if and only if for all functors $a\colon \Del^0\to A$ (objects of $A$) the whiskered composite 2-cell $\alpha a \colon fa\Rightarrow ga\colon\Del^0\to B$ is an isomorphism in $\hom'(\Del^0, B) = \ho{B}$. Using this notion, Corollary~\ref{cor:pointwise-equiv} may be recast to posit that $\alpha$ is a pointwise isomorphism in $\qCat_2$ if and only if it is a genuine isomorphism in $\hom'(A,B)=\ho(B^A)$.
\end{obs}

Since $\qCat_\infty$ is the self enrichment of $\qCat$ under its cartesian product,  it is cartesian closed as a quasi-categorically enriched category. We now show that the 2-category $\qCat_2$ inherits the corresponding property:

\begin{prop}\label{prop:qcat2closed} $\qCat_2$ is cartesian closed as a 2-category.
\end{prop}

\begin{proof}
We show that the terminal object, binary products, and internal hom of the quasi-categorically enriched category $\qCat_\infty$ possess the corresponding 2-categorical universal properties. Specifically, we need to demonstrate the existence of canonical isomorphisms
\begin{align*}
  \hom'(A, \Del^0) &\cong \catone \\
  \hom'(A, B\times C) &\cong \hom'(A, B)\times\hom'(A,C) \\
  \hom'(A, C^B) &\cong \hom'(A\times B, C)
\end{align*}
of categories which are natural in all variables.

To establish each of these we simply apply the homotopy category functor $\ho$ to translate the corresponding $\qCat$-enriched universal properties to $\Cat$-enriched ones, as expressed in terms of the hom-categories defined in~\eqref{eq:qCat2homdefn}. 

Because $\Delta^0$ is a terminal object in the simplicially enriched sense, i.e., because $(\Delta^0)^A \cong \Delta^0$, it is also terminal in the 2-categorical sense: applying $\ho$, the canonical isomorphism
  \[
    \hom'(A,\Del^0) = h( (\Delta^0)^A) \cong h( \Delta^0) \cong \catone
  \] 
 asserts that the hom-category from $A$ to $\Delta^0$ is the terminal category. 

  In a similar fashion, since $\qCat_\infty$ is cartesian closed we know that $B \times C$ is a simplicially enriched product, as expressed by the canonical isomorphisms $(B \times C)^A \cong B^A \times C^A$. Applying $\ho$ we get:
\begin{align*}
    \hom'(A, B\times C) = h((B \times C)^A) & \cong h(B^A \times C^A) \\
    &\cong h(B^A) \times h(C^A) = \hom'(A,B)\times\hom'(A,C).
\end{align*} 

  Finally, the cartesian closure of $\qCat_\infty$  gives rise to isomorphisms $(C^B)^A\cong C^{A \times B}$, to which we may apply the homotopy category functor $\ho$ to obtain the isomorphism 
  \[ 
    \hom'(A\times B, C) = h(C^{A \times B}) \cong h((C^B)^A) = \hom'(A,C^B)
  \] 
  which says that $C^B$ defines an internal hom for the 2-category $\qCat_2$.
\end{proof}

As for any cartesian closed 2-category, the exponential defines a 2-functor $\qCat_2\op \times \qCat_2 \to \qCat_2$.

\begin{defn}[the 2-category of all simplicial sets]\label{defn:2-cat.of.all.simpsets}
  The category $\sSet$ of all simplicial sets is cartesian closed, so we can apply the functor $\ho_*\colon\sCat\to\twoCat$ to its self-enrichment. This provides us with a 2-category $\sSet_2 \defeq \ho_*\sSet$ of all simplicial sets, which has $\qCat_2$ as a full sub-2-category. On occasion, we make slightly implicit use of this larger 2-category. However, we generally choose not to distinguish it notationally from $\qCat_2$, leaving whatever disambiguation is required to the context. 
\end{defn}

\begin{rmk}\label{rmk:exp2functor} 
Exponentiation in the cartesian closed simplicial category $\sSet$ restricts to a simplicial cotensor functor $\sSet\op\times\qCat_\infty\to\qCat_\infty$.

Proposition~\ref{prop:qcat2closed} extends immediately to show that the 2-category of all simplicial sets is again cartesian closed as a 2-category.  Applying $\ho_*$, we obtain a 2-functor $\sSet_2\op\times\qCat_2\to\qCat_2$.  In particular, it follows that exponentiation by any simplicial set $X$ defines a 2-functor $(-)^X \colon \qCat_2 \to \qCat_2$.
  \end{rmk}

\begin{defn}[equivalences in 2-categories]
A 1-cell $u\colon A\to B$ in a 2-category $\tcat{C}$ is an {\em equivalence\/} if and only if there exists a 1-cell $v\colon B\to A$, called its {\em equivalence inverse}, and a pair of 2-isomorphisms $uv\cong\id_B$ and $vu\cong\id_A$. 
\end{defn}

The equivalences of a 2-category $\tcat{C}$ are preserved by all 2-functors since they are defined by 2-equational conditions. Consequently, if $u\colon A\to B$ is an equivalence in $\tcat{C}$ then, applying the representable 2-functor $\tcat{C}(X,-)$,  the functor $\tcat{C}(X,u)\colon\tcat{C}(X,A)\to \tcat{C}(X,B)$ is an equivalence of hom-categories. A basic 2-categorical fact, whose proof is left to the reader, is that these \emph{representably-defined equivalences} are necessarily equivalences in $\tcat{C}$.

\begin{lem}\label{lem:2-cat.equivs}
A 1-cell $u\colon A\to B$ in a 2-category $\tcat{C}$ is an equivalence if and only if $\tcat{C}(X,u)\colon\tcat{C}(X,A)\to \tcat{C}(X,B)$ is an equivalence of hom-categories for all objects $X \in \tcat{C}$. 
\end{lem}

%\begin{rmk}\label{rmk:weak.equivs.2-cat}
  Our central thesis is that the category theory of quasi-categories developed by Joyal, Lurie, and others is captured by $\qCat_2$. For this, it is essential that the standard notion of equivalence of quasi-categories---weak equivalence in the Joyal model structure---is encoded in the 2-category.

  To that end, observe that the description of the weak equivalences given in \ref{rec:qmc-quasicat} may be recast in our 2-categorical framework: by definition, a simplicial map $u\colon X\to Y$ is a weak equivalence in Joyal's model structure if and only if for all quasi-categories $A$ the functor $\hom'(u,A)\colon\hom'(Y,A)\to\hom'(X,A)$ is an equivalence of hom-categories.
%\end{rmk}

  Combining this description with Proposition \ref{prop:qcat2closed} and Lemma \ref{lem:2-cat.equivs} we obtain the following straightforward results:

\begin{prop}\label{prop:equivsareequivs} A functor between quasi-categories is a weak equivalence in the Joyal model structure if and only if it is an equivalence in the 2-category $\qCat_2$.
\end{prop}
\begin{proof}
The weak equivalences between quasi-categories are the representably defined equivalences in the dual 2-category $\qCat_2\op$. Equivalence in a 2-category is a self dual notion, so these coincide with the equivalences in $\qCat_2$.
\end{proof}

\begin{prop}\label{prop:equivsareequivs2}
  A simplicial map $u\colon X\to Y$ is a weak equivalence in the Joyal model structure if and only if for all quasi-categories $A$ the pre-composition functor $A^u\colon A^Y\to A^X$ is an equivalence in the 2-category $\qCat_2$.
\end{prop}

\begin{proof}
  By Lemma~\ref{lem:2-cat.equivs}, $A^u\colon A^Y\to A^X$ is an equivalence in $\qCat_2$ if and only if for all quasi-categories $B$ the functor $\hom'(B,A^u) \colon \hom'(B,A^Y) \to \hom'(B,A^X)$ is an equivalence of hom-categories. Taking duals, $\hom'(B,A^u)$ is isomorphic to $\hom'(u,A^B) \colon \hom'(Y,A^B) \to \hom'(X,A^B)$. Hence, it suffices to show that  $u\colon X\to Y$ is a weak equivalence in Joyal's model structure if and only if $\hom'(u, A^B)$ is an equivalence of hom-categories for all quasi-categories $A$ and $B$, which is the case because $B^A$ is again a quasi-category.
\end{proof}

\subsection{Weak 2-limits}\label{subsec:weak-2-limits}

Finite products aside, the 2-category $\qCat_2$ has few 2-limits. However, we shall soon discover that it has a number of important \emph{weak 2-limits} whose universal properties will be repeatedly exploited in the remainder of this paper.

\begin{defn}[smothering functors]\label{defn:smothering}
  A functor between categories is {\em smothering\/} if and only if it is surjective on objects, full, and conservative (reflects isomorphisms). Equivalently, a functor is smothering if and only if it possesses the right lifting property with respect to the set of functors
  \[ \left\{ \vcenter{\xymatrix@C=5pt{ \emptyset \ar@{u(->}[d] \\ \bullet }},  \vcenter{\xymatrix@C=5pt{ \bullet &  \ar@{u(->}[d] & \bullet  \\ \bullet \ar[rr] & {~} &  \bullet }} , \vcenter{\xymatrix@C=5pt{ \bullet \ar[rr] & \ar@{u(->}[d] & \bullet \\ \bullet \ar[rr] & {~} & \bullet \ar@<.8ex>[ll] }}  \right\}  = \left\{ \vcenter{\xymatrix@C=5pt{ \emptyset \ar@{u(->}[d] \\ \catone }},  \vcenter{\xymatrix@C=5pt{   \catone\sqcup\catone \ar@{u(->}[d]   \\   \cattwo  }} , \vcenter{\xymatrix@C=5pt{ \cattwo \ar@{u(->}[d]   \\  \iso }}  \right\}  \]
  Consequently, the class of smothering functors contains all surjective equivalences and is closed under composition, retract, and pullback along arbitrary functors. By composing lifting problems, we see that all smothering functors are isofibrations, in the sense that they have the right lifting property with respect to either inclusion $\catone\inc \iso$. It is easily checked that if $f$ is a functor which is surjective on objects and arrows, as is true for a smothering functor, and a composite $gf$ is smothering, then so is the functor $g$.
\end{defn}

The following very simple lemma will be of significant utility later on.

\begin{lem}[fibres of smothering functors]\label{lem:smothering}
  Each fibre of a smothering functor is a non-empty connected groupoid. 
\end{lem}

\begin{proof}
  Suppose that $f\colon A\to B$ is a smothering functor. The fact that it is surjective on objects implies immediately that its fibres are non-empty. Furthermore, if $a$ and $a'$ are both objects of $A$ in the fibre of $f$ over some object $b$ in $B$,  then the fullness of $f$ implies that we may find an arrow $\tau\colon a\to a'$ in $A$ with $f(\tau)=\id_b$, thus demonstrating that the fibres are  connected. Finally, if $\tau\colon a\to a'$ is an arrow of $A$ which lies in the fibre of $f$ over $b$, in other words if $f(\tau) = \id_b$, then by conservativity of $f$ we know that $\tau$ is an isomorphism. Hence, these fibres are groupoids. 
\end{proof}

  We have chosen the term smothering here to evoke the image that these are surjective covering functors in quite a strong sense.  Of course, we have placed our tongues firmly in our cheeks while introducing this nomenclature. Smothering functors can fruitfully be thought of as being a certain variety of weak surjective equivalences.

We weaken the standard theory of {\em weighted 2-limits\/} (see e.g., \cite{Kelly:1989fk}) as follows. 

\begin{defn}[weak 2-limits in a 2-category]\label{defn:weak2limit}
  Suppose that $\stcat{A}$ is a small 2-category, that $D \colon \stcat{A} \to \tcat{C}$ is a diagram in a 2-category $\tcat{C}$, and that $W\colon\stcat{A}\to\Cat_2$ is a 2-functor, which we shall refer to as a {\em weight}. If $P$ is an object in $\tcat{C}$ then a {\em cone with summit $P$ over $D$ weighted by\/} $W$ is a 2-natural transformation $c\colon W\To \tcat{C}(P,D(-))$. 

  For each object $K$ of $\tcat{C}$, composition with such a cone induces a functor
  \begin{equation}\label{eq:cone-induced}
    c_K\colon\tcat{C}(K,P)\longrightarrow \lim(W, \tcat{C}(K,D(-)))\cong\int_{a\in\stcat{A}} \tcat{C}(K,D(a))^{W(a)}
  \end{equation}
  where the expression on the right denotes the usual category of 2-natural transformations from W to the 2-functor $\tcat{C}(K,D(-))$, the 2-limit of $\tcat{C}(K,D(-))$  weighted by $W$. The family of maps \eqref{eq:cone-induced} is 2-natural in $K$.

  We say that the cone $c$ displays $P$ as a {\em weak $2$-limit of $D$ weighted by\/} $W$ if and only if the map in \eqref{eq:cone-induced} is a smothering functor for all objects $K\in\tcat{C}$.
\end{defn}

  While we feel obliged to give the last definition in its full, slightly unsightly, generality. However, the reader need not become an expert in the technology of weighted 2-limits in order to read the rest of the paper. We shall only work with certain simple varieties of weak 2-limits in $\qCat_2$, whose weak 2-universal properties we shall describe explicitly.

The fact that the fibres of a smothering functor are connected groupoids is the key ingredient in the proof of the following lemma.

\begin{lem}\label{lem:unique-weak-2-limits} Weak 2-limits are unique up to equivalence: the summits of any two weak 2-limit over a common diagram with a fixed weight are equivalent via an equivalence that commutes with the legs of the limit cones.
\end{lem}

\begin{proof}
Given a pair of cones $c\colon W\To \tcat{C}(P,D(-))$ and $c'\colon W\To \tcat{C}(P',D(-))$ that display  $P$ and $P'$ as weak 2-limits of $D$ weighted by $W$, then for each $K \in \tcat{C}$ we have a pair of smothering functors:
  \begin{equation*}
    \tcat{C}(K,P)\stackrel{c_K}\longrightarrow
    \lim(W, \tcat{C}(K,D(-)))
    \stackrel{c'_{K}}\longleftarrow\tcat{C}(K,P')
  \end{equation*}
  Taking $K=P$, consider the identity 1-cell $\id_P$, an object in the hom-category $\tcat{C}(P,P)$. Since $c'_P$ is surjective on objects, there is a 1-cell $u\colon P\to P'$, an object in $\tcat{C}(P,P')$, such that $c'_{P}(u) = c_{P}(\id_P)$. Exchanging the role of $P$ and $P'$, we also find a 1-cell $u'\colon P'\to P$ such that $c_{P'}(u')=c'_{P'}(\id_{P'})$. These definitions ensure that $u$ and $u'$ commute with the legs of the limit cones.

  Now we can apply the 2-naturality properties of the functors $c_K$ and $c'_K$ to show that
  \begin{align*}
    c_{P}(u'u) &= \lim(W, \tcat{C}(u,D(-)))(c_{P'}(u')) && \text{naturality of family $c_K$}\\
    &= \lim(W, \tcat{C}(u,D(-)))(c'_{P'}(\id_{P'})) && \text{definition of $u'$}\\
    &= c'_{P}(u) && \text{naturality of family $c'_K$}\\
    & = c_{P}(\id_P) && \text{definition of $u$.}
  \end{align*} 
  In other words, $u'u$ and $\id_P$ are both in the same fibre of $c_P$, and so they are isomorphic in that fibre since $c_P$ is a smothering functor. Dually, $uu'$ and $\id_{P'}$ are both in the same fibre of $c'_{P'}$ from which it follows that they too are isomorphic in that fibre. It follows that $u\colon P\to P'$ and $u'\colon P'\to P$ are equivalence inverses. 
\end{proof}

The only diagrams we will consider are indexed by small 1-categories $\stcat{A}$.   Because $\qCat_2$ and $\qCat_\infty$ have the same underlying category, a diagram $D \colon \stcat{A}\to\qCat$ is equally a 2-functor $D\colon\stcat{A}\to\qCat_2$ and a simplicial functor $D\colon\stcat{A}\to\qCat_\infty$. A weight $W\colon\stcat{A}\to\Cat$ for a 2-limit can be regarded as a weight for a simplicial limit by composing with the subcategory inclusion $\Cat\inc\sSet$. Our general strategy will be to show that  the simplicial weighted limit $\lim(W,D)$ exists in $\qCat_\infty$ and that it has the weak 2-universal property expected of the weak 2-limit of $D$ in $\qCat_2$. The following lemma allows us to considerably simplify the class of functors \eqref{eq:cone-induced} that we will need to consider.

\begin{lem}\label{lem:weak-simplification} Fix a small 1-category $\stcat{A}$ and a weight $W \colon \stcat{A} \to \Cat$. Suppose $\mclass{D}$ is a class of diagrams $D\colon\stcat{A} \to \qCat$ that is closed under exponentiation by quasi-categories, in the sense that if $D$ is in the class $\mclass{D}$ then so is $D(-)^X$ for any quasi-category $X$. Then $\qCat_2$ admits weak $W$-weighted 2-limits of this class of diagrams if and only if, for all $D \in \mclass{D}$, the canonical functor
\[ \ho(\lim(W,D)) \to \lim(W,h(D(-)))\] is smothering.
\end{lem}
\begin{proof}
By Definition \ref{defn:weak2limit}, to show that the simplicial weighted limit $\lim(W,D)$ defines a weak 2-limit of a diagram $D \colon \stcat{A} \to \qCat$ in the class $\mclass{D}$, we must show that  for each quasi-category $X$ the canonical comparison map
  \begin{equation}\label{eq:qCat.wl.comp.1}
    \hom'(X,\lim(W,D)) \longrightarrow \lim(W,\hom'(X,D(-)))
  \end{equation}
  is a smothering functor. Recall that $\hom'(X,-) = \ho((-)^X)$. The right adjoint simplicial functor $(-)^X\colon\qCat_\infty\to\qCat_\infty$ preserves all simplicial weighted limits; in other words, the canonical comparison map $\lim(W,D)^X\to\lim(W,D(-)^X)$ is an isomorphism. Thus, the comparison functor~\eqref{eq:qCat.wl.comp.1} is isomorphic to the functor:
  \begin{equation*}
    \ho(\lim(W,D(-)^X)) \longrightarrow \lim(W,\ho(D(-)^X)).
  \end{equation*}
By hypothesis, the diagram $D(-)^X$ is in $\mclass{D}$. Thus, to prove that $\qCat_2$ admits weak 2-limits of the diagrams in $\mclass{D}$, it suffices to show that for all diagrams $D\in\mclass{D}$ the comparison map
  \begin{equation*}
    \ho(\lim(W,D)) \longrightarrow \lim(W,\ho{(D(-))})
  \end{equation*}
  is smothering. 
\end{proof}


\begin{obs}[cones whose summits are not quasi-categories]
The classes of diagrams $\mclass{D}$ we will consider are in fact closed under exponentiation by all simplicial sets. The proof of Lemma \ref{lem:weak-simplification} can then be used to extend the 2-universal properties of the weak 2-limits of $\qCat_2$ constructed here to cones whose summits are arbitrary simplicial sets.  Abstractly speaking, this tells us that the inclusion 2-functor $\qCat_2\inc\sSet_2$ preserves the weak 2-limits of diagrams in $\mclass{D}$. 
In order to avoid repeated remarks of this kind throughout the remainder of this paper, our notation will tacitly signal when this is so by use of the letter ``$X$'' for the object of $\qCat_2$ or $\qCat_\infty$ that could equally be replaced by any simplicial set. By contrast, the letters ``$A$'', ``$B$'', and ``$C$'' refer only to quasi-categories.
\end{obs}

As our first example of a weak 2-limit in $\qCat_2$ we examine cotensors with the generic arrow $\cattwo$. 
Recall we write $A^\cattwo$ for the quasi-category $A^{\Delta^1}$ using our convention that categories are identified with their nerves. We invite the reader to verify that the natural functor $\ho(A^\cattwo) \to (\ho{A})^\cattwo$ is not an isomorphism: it is neither injective on objects nor faithful. However, it is a smothering functor. In other words:

\begin{prop}\label{prop:weak-cotensors} 
The exponential $A^\cattwo$ is a weak cotensor of $A$ by $\cattwo$ in 
$\qCat_2$.
\end{prop}

\begin{proof}
   By Lemma~\ref{lem:weak-simplification}, it suffices to prove that for any quasi-category $A$, the canonical functor \[ \ho(A^\cattwo) \longrightarrow (\ho{A})^\cattwo\] is a smothering functor. Certainly this map is surjective on objects, simply because every arrow in $\ho{A}$ is represented by a 1-simplex in the quasi-category $A$. 

To prove fullness, suppose given a commutative square in $\ho{A}$ and choose arbitrary 1-simplices representing each morphism  and their common composite \begin{equation}\label{eq:arrowinhA}\xymatrix{ \cdot \ar[d]_f \ar[r]^a  \ar[dr]|k & \cdot \ar[d]^g \\ \cdot \ar[r]_b & \cdot}\end{equation} Because $A$ is a quasi-category, any relation between morphisms in $\ho{A}$ is witnessed by a 2-simplex with any choice of representative 1-simplices as its boundary. Hence, we may choose 2-simplices witnessing the fact that $k$ is a composite of $a$ with $g$ and of $f$ with $b$ as displayed. 
 \begin{equation}\label{eq:liftedarrowinA2} \xymatrix{ \cdot \ar[d]_f \ar[r]^{a} \ar[dr]|k^*+{\labelstyle\sim}_*+{\labelstyle \sim}& \cdot \ar[d]^g \\ \cdot \ar[r]_{b} & \cdot}\end{equation}
These two 2-simplices define a map $\Delta^1  \to A^{\Delta^1} = A^\cattwo$, which represents an arrow in the category $h(A^{\cattwo})$ whose image is the specified commutative square.

To prove conservativity, suppose given a map in $h(A^{\cattwo})$ represented by a diagram  \eqref{eq:liftedarrowinA2} whose image  \eqref{eq:arrowinhA} is an isomorphism in $(\ho{A})^\cattwo$, meaning that $a$ and $b$ are isomorphisms in $\ho{A}$, in which case $a$ and $b$ are isomorphisms in the quasi-category $A$. Lemma~\ref{lem:pointwise-equiv} tells us immediately that this diagram is an isomorphism in $A^\cattwo$; compare with \eqref{eq:pointwise-equivalence-square}.
\end{proof}

\begin{rmk}
  A generalisation of this argument shows that if $\scat{C}$ is a free category and $A$ is a quasi-category then the exponential $A^\scat{C}$ is the weak cotensor of $A$ by $\scat{C}$ in $\qCat_2$. Conservativity of the canonical comparison $\ho(A^\scat{C})\to(\ho A)^\scat{C}$ follows from Lemma~\ref{lem:pointwise-equiv}. Its surjectivity on objects makes use of the fact that the inclusion of the \emph{spine} of an $n$-simplex, the simplicial subset spanned by the edges $\fbv{i,i+1}$ in $\Del^n$, is a trivial cofibration for all $n \geq 1$. Fullness is similar.

One should note, however, that this result does not hold for exponentiation by arbitrary categories $\scat{C}$. For example,  $A^{\cattwo\times\cattwo}$ is not the weak cotensor of $A$ by the product category $\cattwo\times\cattwo$ in $\qCat_2$.
\end{rmk}

\begin{prop}
  The exponential $A^\iso$ is a weak cotensor of $A$ by the generic isomorphism $\iso$ in 
  $\qCat_2$. 
\end{prop}

\begin{proof}
 By Lemma \ref{lem:weak-simplification}, it suffices to show that
  \begin{equation*}
    \ho(A^\iso)\longrightarrow \ho(A)^\iso
  \end{equation*}
  is a smothering functor. This is easiest to do by arguing in the marked context. 

By Observation~\ref{obs:nat-mark-homs}, $A^\iso$ may equally well be regarded as an internal hom of naturally marked quasi-categories in $\msSet$. Recollection~\ref{rec:qmc-quasi-marked} tells us that the inclusion $\cattwo^\sharp\inc\iso$ is a trivial cofibration in the marked model structure. Because the marked model structure is cartesian closed, the restriction functor  $A^\iso\to A^{\cattwo^\sharp}$  is a trivial fibration.  Immediately from their defining lifting properties, trivial fibrations of quasi-categories are carried by $\ho$ to functors which are surjective on objects and fully faithful, the so-called {\em surjective equivalences}, so it follows that $\ho(A^\iso)\to \ho(A^{\cattwo^\sharp})$ is a surjective equivalence. Furthermore, in the case where $A$ is an actual category, the functor $A^\iso\to A^{\cattwo^\sharp}$ is an isomorphism. So we obtain a commutative square
  \begin{equation*}
    \xymatrix{
      {\ho(A^\iso)} \ar[r]\ar[d] &
      {\ho(A)^\iso} \ar[d]^{\cong}\\
      {\ho(A^{\cattwo^\sharp})}\ar[r] &
      {\ho(A)^{\cattwo^\sharp}}
    }
  \end{equation*}
of functors between categories in which the left hand vertical is a surjective equivalence. By the composition and cancellation results described in \ref{defn:smothering}, the upper horizontal map in this square is a smothering functor if and only if the lower horizontal map is smothering.

The smothering functors are stable under pullback, so to complete our proof, we will show that  for any naturally marked quasi-category $A$ the square
\begin{equation*}
    \xymatrix{
      {\ho(A^{\cattwo^\sharp})}\pbexcursion \ar[r]\ar[d] &
      {\ho(A)^{\cattwo^\sharp}} \ar[d]\\
      {\ho(A^{\cattwo})}\ar[r] &
      {\ho(A)^{\cattwo}}
    }
  \end{equation*}
  is a pullback;  we know from Proposition~\ref{prop:weak-cotensors} that the lower horizontal map is a smothering functor.  This follows from the definition of the natural marking: a 1-simplex in $A$ is marked if and only if it is an isomorphism, which is the case if and only if it represents an isomorphism in $\ho{A}$. \end{proof}

\begin{prop}\label{prop:weak-homotopy-pullbacks} The 2-category $\qCat_2$ admits weak 2-pullbacks along isofibrations: if the square
\begin{equation*}
  \xymatrix{
    {B\times_A C}\pbexcursion\ar[r]^-{\pi_2} \ar@{->>}[d]_-{\pi_1} 
    &  C\ar@{->>}[d]^g \\
    {B} \ar[r]_f & A
  }
\end{equation*}
is a pullback in simplicial sets for which $B$, $A$, and $C$ are quasi-categories and $g$ is an isofibration, then $B\times_A C$ is a quasi-category and it is a weak 2-pullback of $g$ along $f$ in the 2-category $\qCat_2$.
\end{prop}

\begin{proof}
  The statement only applies to pullbacks of those diagrams of shape $B\xrightarrow{f} A \xleftarrow{g} C$ for which the map $g$ is an isofibration. However, Observation~\ref{obs:isofibration-closure} tells us that any exponentiated isofibration $g^X\colon C^X\to A^X$ is again an isofibration, and so we are in a position to apply Lemma~\ref{lem:weak-simplification}.

   It remains to show that the canonical comparison functor
  \begin{equation*}
    \ho(B\times_A C)\longrightarrow \ho{B}\times_{\ho{A}} \ho{C}
  \end{equation*}
  is smothering. This functor is actually bijective on objects, since in both categories an object consists simply of a pair $(b,c)$ of 0-simplices $b\in B$ and $c\in C$ with $f(b)=g(c)$. 
  
  For fullness, suppose we are given two such pairs $(b,c)$ and $(b',c')$. An arrow between these objects in $\ho{B} \times_{\ho{A}} \ho{C}$ consists of a pair of equivalence classes represented by 1-simplices $\beta \colon b \to b'$ and $\gamma \colon c \to c'$ which both map to the same equivalence class in $\ho{A}$ under $f$ and $g$ respectively. This latter condition simply posits that $f(\beta)$ and $g(\gamma)$ are homotopic relative to their endpoints in $A$; such a homotopy is represented by a 2-simplex with $2^{\text{nd}}$ face $g(\gamma)$, $1^{\text{st}}$ face $f(\beta)$, and $0^{\text{th}}$ face degenerate. This information provides us with a lifting problem between $\Horn^{2,1} \to \Delta^2$ and $g$, which we may solve because $g$ is an isofibration. The resulting filler supplies us with a 1-simplex $\gamma' \colon c \to c'$ for which $g(\gamma')=f(\beta)$ and a homotopy of $\gamma'$ and $\gamma$ (relative to their endpoints) which shows these represent the same arrow in $\ho{C}$. In other words, $(\beta,\gamma')$ is a 1-simplex in $B\times_A C$ that represents an arrow of $\ho(B\times_A C)$ from $(b,c)$ to $(b',c')$ and this arrow maps to the originally chosen arrow in $\ho{B}\times_{\ho{A}}\ho{C}$.

The proof of conservativity is simplified by arguing in the marked model structure.  Giving our quasi-categories $A$, $B$, and $C$ the natural marking, the isofibration $g$ becomes a fibration in the marked model structure. It follows that the pullback is a fibrant object and hence naturally marked. Consequently, a 1-simplex $(\beta,\gamma)$ of $B\times_A C$ represents an isomorphism in $\ho(B\times_A C)$ if and only if it is marked, and this is the case if and only if $\beta$ is marked in $B$ and $\gamma$ is marked in $C$. Now, this latter condition holds if and only if $\beta$ is invertible in $\ho{B}$ and $\gamma$ is invertible in $\ho{C}$ and these conditions together are equivalent to the pair $(\beta,\gamma)$ being invertible as an arrow in the category $\ho{B}\times_{\ho{A}} \ho{C}$.
\end{proof} 

  \begin{defn}[comma objects]\label{def:comma-obj}
 Given a pair of functors $B\xrightarrow{f} A \xleftarrow{g} C$ between quasi-categories, we define the {\em comma object\/} $f\comma g$ to be the simplicial set constructed by forming the following pullback:
    \begin{equation*}
      \xymatrix@=2.5em{
        {f\comma g}\pbexcursion \ar[r]\ar[d]_{p} &
        {A^\cattwo} \ar[d] \\
        {C\times B} \ar[r]_-{g\times f} & {A\times A}
      }
    \end{equation*}
\end{defn}

The right-hand vertical  is defined by restricting along the boundary inclusion $\Del^0\sqcup\Del^0\cong\boundary\Del^1\inc\Del^1$ and then composing with the symmetry isomorphism $A \times A \cong A \times A$. In a subsequent paper, we will think of the comma object $f \comma g$ as a module, with $C$ acting on the left and with $B$ acting on the right, which is the reason for our convention.

\begin{lem}\label{lem:comma-obj} The simplicial set $f \comma g$ is a quasi-category and the projection functors $p_0\defeq\pi_B\circ p \colon f\comma g\tfib B$ and $p_1\defeq\pi_C\circ p\colon f\comma g\tfib C$ are isofibrations.
\end{lem}
\begin{proof}
    The right hand vertical in the pullback square above is isomorphic to the simplicial map $A^{\Del^1}\to A^{\boundary\Del^1}$ and is thus, by \ref{rec:cart-modcat}, an isofibration whenever $A$ is a quasi-category. Consequently, since the product $C\times B$ is again a quasi-category, $p\colon f\comma g\to C\times B$ is  an isofibration and $f\comma g$ is a quasi-category. The projection functors $\pi_C\colon C\times B\tfib C$ and $\pi_B\colon C\times B\tfib B$ are both isofibrations because $B$ and $C$ are fibrant, so it follows that the domain and codomain projection maps $p_0 \colon f\comma g\tfib B$ and $p_1\colon f\comma g\tfib C$ are also isofibrations.
    \end{proof}
        
\begin{lem}[maps induced between comma objects]\label{lem:comma-obj-maps}
A commutative diagram
     \begin{equation*}
       \xymatrix{
         {B}\ar[r]^{f}\ar@{->>}[d]_-{r} & 
         {A}\ar@{->>}[d]_-{q} & 
         {C}\ar[l]_{g}\ar@{->>}[d]^-{s} \\
         {\bar{B}}\ar[r]_{\bar{f}} & 
         {\bar{A}} & 
         {\bar{C}}\ar[l]^{\bar{g}} 
       }
     \end{equation*} 
     in $\qCat$ in which the vertical maps are (trivial) fibrations in the Joyal model structure, induces a (trivial) fibration $r \comma_q s \colon f \comma g \tfib \bar{f}\comma\bar{g}$ between comma quasi-categories.
\end{lem}
\begin{proof}
Consider the commutative diagram
     \begin{equation*}
       \xymatrix@C=4.5em@R=2.5em{
         {C\times B}\ar[r]^-{g\times f}\ar@{->>}[d]_-{s\times r} & 
         {A\times A}\ar@{->}[d]_-{q\times q} & 
         {A^\cattwo}\ar[l]_-{(p_1,p_0)}\ar@{->}[d]^-{q^\cattwo} 
         \save "1,3"-<2.5em,1.5em>*+{P}\ar[l]\ar[d]\ar@{<<.}[]_(0.6)l \restore \\
         {\bar{C}\times\bar{B}}\ar[r]_-{\bar{g}\times\bar{f}} & 
         {\bar{A}\times\bar{A}} & 
         {\bar{A}^\cattwo}\ar[l]^-{(p_1,p_0)} 
       }
     \end{equation*}
     in which $P$ denotes the pullback of the maps $q\times q$ and $(p_1,p_0)$ and $l$ is the unique map induced into it by the right hand square. The pullbacks of the two horizontal lines are the comma objects $f\comma g$ and $\bar{f}\comma\bar{g}$ respectively. So this diagram induces a unique map $r\comma_q s\colon f\comma g\to\bar{f}\comma\bar{g}$ of comma objects which makes the manifest cube commute.
     
     The (trivial) fibrations of any model category are closed under product, so the map $s\times r$ is a (trivial) fibration in the Joyal model structure.  The induced map $l$ is isomorphic to the Leibniz hom $\leib\hom(\boundary\Del^1\inc\Del^1, q\colon A\tfib \bar{A})$; a recalled in \ref{rec:cart-modcat}, cartesianness of the Joyal model structure implies that $l$ is a (trivial) fibration.
The induced map $r\comma_q s\colon f\comma g\tfib\bar{f}\comma\bar{g}$ is again a (trivial) fibration because it factors as a composite of pullbacks of the (trivial) fibrations $s\times r$ and $l$.
   \end{proof}

\begin{prop}\label{prop:weakcomma} For any functors $B \xrightarrow{f} A \xleftarrow{g} C$  of quasi-categories, the comma quasi-category $f\comma g$ is a weak comma object in $\qCat_2$.
\end{prop}
\begin{proof}
    Again, Lemma~\ref{lem:weak-simplification} applies, so it suffices to show that the canonical comparison
  \begin{equation}\label{eq:commacat-comp}
    \ho(f\comma g) \longrightarrow \ho(f)\comma\ho(g)
  \end{equation}
  is a smothering functor. Here the target category is just the usual comma category constructed in $\Cat$. By definition,  $f\comma g \cong (C\times B)\times_{(A\times A)} A^\cattwo$ and consequently we find that we may express the functor in~\eqref{eq:commacat-comp} as a composite:
  \begin{equation*}
    \ho((C\times B)\times_{(A\times A)} A^\cattwo)
    \longrightarrow
    \ho(C\times B)\times_{\ho(A\times A)} \ho(A^\cattwo)
    \longrightarrow
    \ho(C\times B)\times_{\ho(A\times A)} \ho(A)^\cattwo
  \end{equation*}
  The first of these maps is the canonical comparison functor studied in Proposition~\ref{prop:weak-homotopy-pullbacks}, so we know that it is smothering. The second of these maps is a pullback of the canonical comparison functor discussed in Proposition~\ref{prop:weak-cotensors}; since smothering functors are stable under pullback, it too is a smothering functor. We obtain the required result from the fact that smothering functors compose.
\end{proof}

\begin{obs}[unpacking the universal property of weak comma objects]\label{obs:unpacking-weak-comma-objects}
  The smothering functors
  \begin{equation}\label{eq:weak-comma-prop}
    \hom'(X,f\comma g)\longrightarrow\hom'(X,f)\comma\hom'(X,g)
  \end{equation}
  which express the weak 2-universal property of the quasi-category $f\comma g$ are induced by composition with a cone:
  \begin{equation}\label{eq:standard-comma-pic}
    \xymatrix@=10pt{
      & f \downarrow g \ar[dl]_{p_1} \ar[dr]^{p_0} \ar@{}[dd]|(.4){\psi}|{\Leftarrow}  \\ 
      C \ar[dr]_g & & B \ar[dl]^f \\ 
      & A}
  \end{equation}
  The data displayed in \eqref{eq:standard-comma-pic} is the image of the identity 1-cell under \eqref{eq:weak-comma-prop} in the case $X = f\comma g$. The weak universal property of this comma cone has three aspects, corresponding to the surjectivity on objects, fullness, and conservativity of the smothering functor \eqref{eq:weak-comma-prop}, which we refer to as 
 {\em 1-cell induction\/}, {\em 2-cell induction}, and {\em 2-cell conservativity}.


Surjectivity on objects of the functor~\eqref{eq:weak-comma-prop} simply says that for any comma cone
  \begin{equation}\label{eq:comma-cone}
    \xymatrix@=10pt{
       & X \ar[dl]_{c} \ar[dr]^{b} \ar@{}[dd]|(.4){\alpha}|{\Leftarrow}  \\ 
       C \ar[dr]_g & & B \ar[dl]^f \\ 
       & A}
  \end{equation}
  over our diagram there exists a map $a\colon X \to f \comma g$ which factors $b\colon X\to B$ and $c\colon X\to C$ through $p_0\colon f\comma g\to B$ and $p_1\colon f\comma g\to C$ respectively and which whiskers with the 2-cell $\psi\colon fp_0\Rightarrow gp_1$ to give the 2-cell $\alpha\colon fb\Rightarrow gc$; diagrammatically speaking, \emph{1-cell induction} produces a functor $a \colon X \to f \comma g$ from a 2-cell $\alpha \colon fb \To gc$ so that:
  \begin{equation}\label{eq:comma-ind-1cell-prop}
    \vcenter{\xymatrix@=10pt{
       & X \ar[dl]_{c} \ar[dr]^{b} \ar@{}[dd]|(.4){\alpha}|{\Leftarrow}  \\ 
       C \ar[dr]_g & & B \ar[dl]^f \\ 
       & A
    }}
    \mkern20mu = \mkern20mu
    \vcenter{\xymatrix@=10pt{
      & f \downarrow g \ar[dl]_{p_1} \ar[dr]^{p_0} \ar@{}[dd]|(.4){\psi}|{\Leftarrow}  \\ 
      C \ar[dr]_g & & B \ar[dl]^f \\ 
      & A
      \save "1,2"+<0pt,40pt>*+{X}\ar "1,2" _-a\restore
      }}
  \end{equation}

Fullness of \eqref{eq:weak-comma-prop} tells us that if we are given a pair of functors $a,a'\colon X\to f\comma g$ and a pair of 2-cells
  \begin{equation}\label{eq:comma-ind-2cell-data}
    \vcenter{\xymatrix@=10pt{
      & {X}\ar[dl]_{a'}\ar[dr]^{a}
      \ar@{}[dd]|(.4){\tau_0}|{\Leftarrow} & \\
      {f\comma g}\ar[dr]_{p_0} & & 
      {f\comma g}\ar[dl]^{p_0} \\
      & B &
    }}
    \mkern30mu\text{and}\mkern30mu
    \vcenter{\xymatrix@=10pt{
      & {X}\ar[dl]_{a'}\ar[dr]^{a}
      \ar@{}[dd]|(.4){\tau_1}|{\Leftarrow} & \\
      {f\comma g}\ar[dr]_{p_1} & & 
      {f\comma g}\ar[dl]^{p_1} \\
      & C &
    }}
  \end{equation}
  with the property that 
    \begin{equation}\label{eq:comma-ind-2cell-compat}
      \xymatrix@=10pt{ & X \ar[dl]_{a'} \ar[dr]^a \ar@{}[dd]|(.4){\tau_1}|{\Leftarrow} & &  & \ar@{}[dd]|{\displaystyle =} & &  & X \ar[dl]_{a'} \ar[dr]^a  \ar@{}[dd]|(.4){\tau_0}|{\Leftarrow} \\ f \downarrow g  \ar[dr]_{p_1} & & f \downarrow g \ar[dl]|{p_1} \ar[dr]^{p_0}   \ar@{}[dd]|(.4){\psi}|{\Leftarrow} & & & &   f \downarrow g \ar[dl]_{p_1} \ar[dr]|{p_0}  \ar@{}[dd]|(.4){\psi}|{\Leftarrow}  & & f \downarrow g \ar[dl]^{p_0} \\ & C \ar[dr]_g & & B \ar[dl]^f & & C \ar[dr]_g & & B \ar[dl]^f & &  \\ & & A & &  &  & A}
    \end{equation}
  then there exists a 2-cell $\tau \colon a \Rightarrow a'$, defined by \emph{2-cell induction}, satisfying the equalities 
  \[     \vcenter{\xymatrix@=10pt{
      & {X}\ar[dl]_{a'}\ar[dr]^{a}
      \ar@{}[dd]|(.4){\tau_0}|{\Leftarrow} & \\
      {f\comma g}\ar[dr]_{p_0} & & 
      {f\comma g}\ar[dl]^{p_0} \\
      & B &
    }}    \mkern20mu = \mkern20mu
    \vcenter{\xymatrix@=30pt{ X \ar@/^2ex/[d]^a \ar@/_2ex/[d]_{a'} \ar@{}[d]|(.4){\tau}|{\Leftarrow}  \\ f \downarrow g \ar[d]^{p_0} \\ B}}
      \mkern30mu\text{and}\mkern30mu
    \vcenter{\xymatrix@=10pt{
      & {X}\ar[dl]_{a'}\ar[dr]^{a}
      \ar@{}[dd]|(.4){\tau_1}|{\Leftarrow} & \\
      {f\comma g}\ar[dr]_{p_1} & & 
      {f\comma g}\ar[dl]^{p_1} \\
      & C &
    }}     \mkern20mu = \mkern20mu
        \vcenter{\xymatrix@=30pt{ X \ar@/^2ex/[d]^a \ar@/_2ex/[d]_{a'} \ar@{}[d]|(.4){\tau}|{\Leftarrow}  \\ f \downarrow g \ar[d]^{p_1} \\ C}}.    \]

    Finally, conservativity of \eqref{eq:weak-comma-prop} tells us that if we are given a 2-cell $\tau\colon a\Rightarrow a' \colon X \to f \comma g$ then if the whiskered composites $p_0\tau$ and $p_1\tau$, as shown in the previous diagram, are isomorphisms in $\hom'(X,B)$ and $\hom'(X,C)$ respectively, then $\tau$ is also an isomorphism in $\hom'(X,f\comma g)$; this is \emph{2-cell conservativity}.
 \end{obs}

\begin{lem}[1-cell induction is unique up to isomorphism]\label{lem:1cell-ind-uniqueness}
Any two 1-cells $a,a' \colon X \to f \comma g$ over a weak comma object \eqref{eq:standard-comma-pic} that are induced by the same comma cone $\alpha \colon fb \To gc$ are isomorphic over $C \times B$. 
\end{lem}
\begin{proof}
This follows from Lemma~\ref{lem:smothering}, which demonstrates that fibres of smothering functors are connected groupoids, or can be proven directly. From the defining property of induced 1-cells displayed in~\eqref{eq:comma-ind-1cell-prop} it follows that $p_0 a = p_0 a'$, $p_1 a = p_1 a'$, and $\psi a = \psi a'$. We can regard the first two of these equalities as being identity 2-cells of the form displayed in~\eqref{eq:comma-ind-2cell-data}. Then the third of these equalities may be re-interpreted as positing the compatibility property displayed in~\eqref{eq:comma-ind-2cell-compat} for those identity 2-cells. So we may apply the 2-cell induction property of $f\comma g$ to obtain a 2-cell $\tau\colon a\Rightarrow a'$ whose whiskered composites with $p_0$ and $p_1$ are the identity 2-cells corresponding to the equalities $p_0 a = p_0 a'$ and $p_1 a = p_1 a'$ respectively. This then allows us to apply the 2-cell conservativity property of our weak comma object to show that $\tau\colon a\Rightarrow a'$ is an isomorphism.
 \end{proof}

\subsection{Slices of the category of quasi-categories}\label{subsec:slice-2cats-of-qcats}

\begin{defn}[enriching the slices of $\qCat$]\label{defn:enriched-slice}
  For a quasi-category $A$, we will write $\qCat/A$ for the full subcategory of the usual slice category whose objects are isofibrations $E\tfib A$. Where not otherwise stated, we shall restrict our attention to these subcategories of isofibrations: these are the subcategories of fibrant objects in slices of Joyal's model structure and so are better behaved when viewed from the perspective of formal quasi-category theory than the slice categories of all maps with fixed codomain.
   
   The category $\qCat/A$ has two enrichments of interest to us here. Let $\qCat_2\slice A$ and $\qCat_\infty\slice A$ denote the 2-category and simplicial category (respectively) whose objects are the isofibrations with codomain $A$ and whose hom-category and simplicial hom-space (respectively) between $p \colon E \tfib A$ and $q \colon F \tfib A$ are defined by the pullbacks 
   \begin{equation}\label{eq:slice-hom-objects}  
    \xymatrix{
      \hom'_A(p,q) \pbexcursion \ar[d] \ar[r] & \hom'(E,F) \ar[d]^-{\hom'(E,q)} &  &   \hom_A(p,q) \pbexcursion \ar[r] \ar[d] & F^E \ar@{->>}[d]^-{q^E} \\ 
       \catone \ar[r]_-p & \hom'(E,A) & &     \Delta^0 \ar[r]_-p & A^E}
   \end{equation}
  The objects of $\hom'_A(p,q)$ and the vertices of $\hom_A(p,q)$ are exactly the morphisms from $p$ to $q$ in $\qCat/A$. The morphisms in $\hom'_A(p,q)$, 2-cells in the 2-category $\qCat_2\slice A$, are natural transformations between functors $E \to F$ in $\qCat_2$ whose whiskered composite with $q$ is the identity 2-cell on $p$. Since $q\colon F\tfib A$ is an isofibration we know that $q^E\colon F^E\tfib A^E$ is also an isofibration as is its pullback $\hom_A(p,q)\tfib\Del^0$; hence, $\hom_A(p,q)$ is a quasi-category. In other words, $\qCat_\infty\slice A$ is enriched in quasi-categories.
\end{defn}

\begin{obs}[pushforward]\label{obs:fibred-pushforward} If $f \colon B \tfib A$ is an isofibration of quasi-categories then post-composition defines a simplicial functor $f_* \colon \qCat_\infty\slice B \to \qCat_\infty\slice A$ and a 2-functor $f_* \colon \qCat_2\slice B \to \qCat_2\slice A$. 
\end{obs}

One reason for our particular interest in the simplicial categories $\qCat_\infty\slice A$ has to do with the following observation.  Simplicially enriched limits are defined up to isomorphism and thus assemble into a simplicial functor. The universal property defining weak 2-limits, however, lacks a uniqueness statement of sufficient strength to make them assemble into a (strict) 2-functor. In particular:

\begin{obs}[pullback]\label{obs:fibred-pullback}
Consider any functor $f \colon B \to A$ between quasi-categories. Pullback along $f$ defines a functor $f^* \colon \qCat/A \to \qCat/B$, but it cannot be extended to a 2-functor between slice 2-categories $\qCat_2\slice A$ and $\qCat_2\slice B$ in any canonical way. On the other hand, pullback is a genuine simplicial limit in $\qCat_\infty$ and so it does define a simplicial functor $f^* \colon \qCat_\infty\slice A \to \qCat_\infty\slice B$, which in turn gives rise to a 2-functor  $f^* \colon \ho_*(\qCat_\infty\slice A) \to \ho_*(\qCat_\infty\slice B)$ on application of $\ho_*\colon\eCat{\sSet}\to\twoCat$.  The remarks apply equally to the larger slice categories of all maps with fixed codomain.
\end{obs}




\begin{obs}[comparing the 2-categories $\qCat_2\slice A$ and $\ho_*(\qCat_\infty\slice A)$]
The 2-categories $\qCat_2\slice A$ to $\ho_*(\qCat_\infty\slice A)$ have the same 0-cells and 1-cells; however it is not the case that their 2-cells coincide. If we are given a parallel pair of 1-cells \[ \xymatrix@=1.5em{ E \ar@{->>}[dr]_p \ar@/^1ex/[rr]^f \ar@/_1ex/[rr]_g & & F \ar@{->>}[dl]^q \\ & A}\] a 2-cell from $f$ to $g$ in
  \begin{description}
    \item[$\qCat_2\slice A$] is a homotopy class of 1-simplices $f \to g$ in $F^E$ that whisker with $q$ to the homotopy class of the degenerate 1-simplex on $p$.
    \item[$\ho_*(\qCat_\infty\slice A)$] is a homotopy class represented by a 1-simplex $f \to g$ in the fibre of $q^E\colon F^E\tfib A^E$ over the vertex $p\in A^E$ under homotopies which are also constrained to that fibre.
  \end{description}
  Note here that the notion of homotopy involved in the description of 2-cells in $\ho_*(\qCat_\infty\slice A)$ is more refined (identifies fewer simplices) than that given for 2-cells in $\qCat_2\slice A$. Each homotopy class representing a 2-cell in $\qCat_2\slice A$ may actually split into a number of distinct homotopy classes representing 2-cells in $\ho_*(\qCat_\infty\slice A)$.
\end{obs}



  Consequently, it is not the case that these two enrichments of $\qCat\slice A$ to a 2-category are identical. However, they are related by a 2-functor whose properties we now enumerate.

\begin{defn}[smothering 2-functor]\label{defn:smothering-2-functor}
  A 2-functor $F\colon\tcat{C}\to\tcat{D}$ is said to be a {\em smothering 2-functor\/} if it is surjective on 0-cells and \emph{locally smothering}, i.e., if for all 0-cells $K$ and $K'$ in $\tcat{C}$ the action $F\colon \tcat{C}(K,K')\to\tcat{D}(FK,FK')$ of $F$ on the hom-category from $K$ to $K'$ is a smothering functor.

Note that smothering 2-functors are also conservative at the level of 1-cells in the sense appropriate to 2-category theory; that is to say if $k\colon K\to K'$ is a 1-cell in $\tcat{C}$ for which $Fk$ is an equivalence in $\tcat{D}$ then $k$ is an equivalence in $\tcat{C}$.
\end{defn}

\begin{prop}\label{prop:slice-smothering-2-functor} There exists a canonical 2-functor $\ho_*(\qCat_\infty\slice A)\to\qCat_2\slice A$  which acts identically on 0-cells and 1-cells and is a smothering 2-functor.
\end{prop}
\begin{proof}
  To construct the required 2-functor, apply the homotopy category functor $\ho$ to the defining pullback square for $\hom_A(p,q)$ in~\eqref{eq:slice-hom-objects} to obtain a square which then induces a functor $h(\hom_A(p,q))\to \hom'_A(p,q)$ by the pullback property of the defining square for $\hom'_A(p,q)$. It is a routine matter now to check that we may assemble these actions on hom-categories together to give a 2-functor which acts as the identity on the common underlying category $\qCat\slice A$ of these 2-categories. 

  To show that this 2-functor is smothering, we already know that it acts bijectively on 0-cells, so all that remains is to show that each $h(\hom_A(p,q))\to \hom'_A(p,q)$ is a smothering functor. This fact follows by direct application of Proposition~\ref{prop:weak-homotopy-pullbacks} to the defining pullbacks~\eqref{eq:slice-hom-objects}.
 \end{proof}

Our next aim is to develop a useful principle by which to recognise those 1-cells of $\ho_*(\qCat_\infty\slice A)$ which are equivalences in there. To achieve this, we must first explore the 2-categorical properties of the isofibrations between quasi-categories.

 \begin{defn}[representably defined isofibrations in 2-categories]\label{defn:representable-isofibrations}
  A 1-cell $p\colon B\to A$ in a 2-category $\tcat{C}$ is said to be a {\em representably defined isofibration\/} (or just an \emph{isofibration}) if and only if for each object $X\in\tcat{C}$ the functor $\tcat{C}(X,p)\colon\tcat{C}(X,B)\to\tcat{C}(X,A)$ is an isofibration of categories (has the right lifting property with respect to the inclusion $\catone\inc\iso$).  In more explicit terms, this means that for any diagram \[ \xymatrix{ \ar@{}[dr]|(.7){\alpha\cong} & B \ar[d]^p  & \ar@{}[d]|{\displaystyle\rightsquigarrow} &  \ar@{}[dr]|{\beta\cong} & B \ar[d]^p \\ X \ar[ur]^b \ar[r]_a & A &&  X \ar@/^1.5ex/[ur]^b \ar@/_1.5ex/[ur]_*!<-3pt,+3pt>{\labelstyle x} \ar[r]_a & A}\] consisting of 1-cells $a$ and $b$ and a 2-isomorphism $\alpha \colon pb \cong a$, there exists a 1-cell $x$ and 2-isomorphism $\beta \colon b \cong x$ so that $p \beta = \alpha$ and $px = a$.
\end{defn}

\begin{lem}\label{lem:representable-isofibration} If $p \colon B \tfib A$ is an isofibration between quasi-categories, then $p$ is a representably defined isofibration in $\qCat_2$.
\end{lem}
\begin{proof} 
For any simplicial set $X$,   $p^X\colon B^X\tfib A^X$ is also an isofibration  and in particular has the right lifting property with respect to $\catone\inc\iso$. Using the standard homotopy coherence result, recalled in \ref{rec:qmc-quasi-marked}, that an isomorphism in the homotopy category of a quasi-category can be extended to a functor with domain $\iso$, it follows that $\hom'(X,p)\colon\hom'(X,B)\to\hom'(X,A)$ also has the right lifting property with respect to $\catone\inc\iso$. Thus $\hom'(X,p)$ is 
  an isofibration of categories, which shows that the isofibrations of quasi-categories are representably defined  in the 2-category $\qCat_2$.
  \end{proof}

    The following lemma, stated here in the special case of $\qCat_2$, applies equally to any slice 2-category whose objects are isofibrations.

  \begin{lem}\label{lem:proj-is-1-conservative}
     The canonical projection 2-functor $\qCat_2\slice A\to \qCat_2$ is conservative on 1-cells in the appropriate 2-categorical sense: if
\begin{equation}\label{eq:equiv-to-lift}
  \xymatrix@=1.5em{
    {E}\ar[rr]^w\ar@{->>}[dr]_p && {F}\ar@{->>}[dl]^q \\
    & {A} &
  }
\end{equation}
is a 1-cell in $\qCat_2\slice A$ for which $w\colon E\to F$ admits an equivalence inverse $w' \colon F \to E$ in $\qCat_2$, then $w$  is  an equivalence in the slice 2-category $\qCat_2\slice A$.
  \end{lem}
  \begin{proof}
By a standard 2-categorical argument, we may choose  2-isomorphisms $\alpha\colon w'w\cong\id_E$ and $\beta\colon\id_F\cong ww'$ which display $w'$ as a left adjoint equivalence inverse to $w$ in $\qCat_2$. As $p$ is an isofibration in $\qCat_2$, the isomorphism $q\beta\colon q\cong qww' = pw'$ can be lifted along $p$ to give a 1-cell $\bar{w}\colon F\to E$ with $p\bar{w} = q$ and a 2-isomorphism $\gamma\colon \bar{w}\cong w'$ with $p\gamma=q\beta$. The first of these equations tells us that $\bar{w}$ is a 1-cell in $\qCat_2\slice A$.  Using the second of these equations and the triangle identities relating $\alpha$ and $\beta$, we see that the isomorphisms $\alpha\cdot\gamma w\colon \bar{w} w\cong\id_E$ and $w\gamma^{-1}\cdot\beta\colon\id_F\cong w\bar{w}$ are 2-cells in $\qCat_2\slice A$: 
\begin{align*}
p(\alpha \cdot \gamma w) &= p\alpha \cdot p\gamma w = qw\alpha \cdot q\beta w = q\id_w \\ q(w\gamma^{-1}\cdot \beta) &= qw\gamma^{-1} \cdot q\beta = p\gamma^{-1} \cdot q\beta = \id_p. 
\end{align*}
These isomorphisms display $\bar{w}$ as an equivalence inverse to $w$ in $\qCat_2\slice A$.
  \end{proof}

  \begin{cor}\label{cor:recog-fibred-equivs}
    The 1-cell depicted in~\eqref{eq:equiv-to-lift} is an equivalence in $\ho_*(\qCat_\infty\slice A)$ if and only if $w\colon E\to F$ is an equivalence in $\qCat_2$. 
  \end{cor}

  \begin{proof}
  By  Proposition~\ref{prop:slice-smothering-2-functor} and Lemma \ref{lem:proj-is-1-conservative}, the canonical 2-functors $\ho_*(\qCat_\infty\slice A)\to\qCat_2\slice A$ and $\qCat_2\slice A\to \qCat_2$ are both conservative on 1-cells, so their composite is also conservative on 1-cells. The result follows immediately.
  \end{proof}
  

  \begin{defn}[fibred equivalence]\label{defn:fibred-equivalence}
  A functor $w \colon E \to F$ between quasi-categories equipped with specified isofibrations $p\colon E \tfib A$ and $q \colon F \tfib A$ is an {\em equivalence fibred over $A$\/}, or just a {\em fibred equivalence}, if it is an equivalence in $\ho_*(\qCat_\infty\slice A)$. By Corollary \ref{cor:recog-fibred-equivs}, any equivalence in $\qCat_2$ which commutes with the maps down to $A$ is a fibred equivalence. Unpacking the definition, a fibred equivalence admits an equivalence inverse $w' \colon F \to E$ over $A$ together with isomorphisms $\alpha \colon w' w \cong \id_E \in E^E$ and $\beta \colon \id_F \cong w w' \in F^F$ represented by 1-simplices that compose with $p$ and $q$ to degenerate 1-simplices.
    \end{defn}

Corollary \ref{cor:recog-fibred-equivs} allows us to lift equivalences in $\qCat_2/A$ to fibred equivalences, which can be pulled back along a functor $f \colon B \to A$ as described in Observation \ref{obs:fibred-pullback}. The lifting arguments developed here relied upon the assumption that the simplicial categories in which we work have hom-spaces which are quasi-categories, which is why our default is to assume that the objects of our slice categories $\qCat_2\slice A$ and $\qCat_\infty\slice A$ are isofibrations.
  
\subsection{A strongly universal characterisation of weak comma objects}

We may use properties of the 2-categorical slice $\qCat_2\slice (C \times B)$ to characterise the weak comma objects of $\qCat_2$ in terms of a \emph{strict} 1-categorical universal property. We present this technical result here and then use it to good effect in section~\ref{sec:limits}, where we demonstrate how to characterise limits and colimits that exist in a quasi-category in purely 2-categorical terms.

For this subsection we shall assume, contrary to our notational convention elsewhere, that $\qCat_2\slice(C\times B)$ denotes the unrestricted slice 2-category whose objects are all functors with codomain $C\times B$.

\begin{obs}[uniqueness of 1-cell induction revisited]\label{obs:1cell-ind-uniqueness-reloaded}
Any 1-cell $a\colon X\to f\comma g$ induced by the comma cone~\eqref{eq:comma-cone} may be regarded as a 1-cell
  \begin{equation*}
    \xymatrix@=1em{
      {X}\ar[dr]_(0.3){(c,b)}\ar[rr]^{a}
      && *+!L(0.5){f\comma g}\ar[dl]^(0.3){(p_1,p_0)} \\
      & {C\times B}&
    }
  \end{equation*}
  in $\qCat_2\slice(C\times B)$. If we are given a second 1-cell $a'\colon X\to f\comma g$ which is also induced by the same comma cone then the argument of Lemma~\ref{lem:1cell-ind-uniqueness} delivers us a 2-cell
  \begin{equation}\label{eq:induced-1-cell-comparison}
    \xymatrix@=1.5em{
      {X}\ar[dr]_(0.3){(c,b)} 
      \ar@/^1.5ex/[rr]^{a}_{}="one" \ar@/_1.5ex/[rr]_{a'}^{}="two" \ar@{=>}"one";"two"^{\tau}
      && *+!L(0.5){f\comma g}\ar[dl]^(0.3){(p_1,p_0)} \\
      & {C\times B}&
    }
  \end{equation}
in $\qCat_2\slice(C\times B)$, which is moreover an isomorphism; this is what we meant by the assertion that any pair of functors defined by 1-cell induction over the same comma cone are isomorphic over $C \times B$.
 Conversely, by 2-cell conservativity of the comma quasi-category $f \downarrow g$, any 2-cell of $\qCat_2\slice(C\times B)$ of the form depicted in~\eqref{eq:induced-1-cell-comparison} is an isomorphism. Thus, the hom-category $\hom'_{C\times B}((c,b),(p_1,p_0))$ is a groupoid, whose connected components comprise those 1-cells induced by a common cone \eqref{eq:comma-cone}.
\end{obs}

\begin{obs}\label{obs:squares-set}
  For each object $(c,b)\colon X\to C\times B$ of $\qCat/(C\times B)$ we have a set $\sq_{g,f}(c,b)$ of 2-cells as depicted in~\eqref{eq:comma-cone}. This construction may be extended immediately to a contravariant functor $\sq_{g,f}\colon(\qCat/(C\times B))\op\to\Set$, which carries a morphism
  \begin{equation*}
    \xymatrix@=1em{
      {X}\ar[dr]_(0.3){(c,b)}\ar[rr]^{u}
      && *+!L(0.5){Y}\ar[dl]^(0.3){(\bar{c},\bar{b})} \\
      & {C\times B}&
    }
  \end{equation*}
  of $\qCat/(C\times B)$ to the function $\sq_{g,f}(u)$ which maps a 2-cell $\beta$ of $\sq_{g,f}(\bar{c},\bar{b})$ to the whiskered 2-cell $\beta u$ in $\sq_{g,f}(c,b)$. 


  \end{obs}

\begin{obs}\label{obs:groupoid-components}
There is a product-preserving  functor $\pi^g_0\colon\Cat\to\Set$ that sends a category to  the set of connected components of its sub-groupoid of isomorphisms. We may apply $\pi^g_0$ to the hom-categories of a 2-category $\tcat{C}$ to construct a category $(\pi^g_0)_*\tcat{C}$.  Any isomorphism $K\cong L$ in the category $(\pi^g_0)_*\tcat{C}$ can be lifted to a corresponding equivalence in $\tcat{C}$ by picking representatives $w\colon K\to L$ and $w'\colon L\to K$ in $\tcat{C}$ for the isomorphism and its inverse. The 2-isomorphisms $\alpha\colon w'w\cong\id_K$ and $\beta\colon ww'\cong\id_L$ which witness these as equivalence inverses in $\tcat{C}$ arise by choosing 2-cells which witness the mutual inverse identities $w'w=\id_K$ and $ww'=\id_L$ in $(\pi^g_0)_*\tcat{C}$.
\end{obs}


\begin{lem}\label{lem:sq-as-a-functor}
The functor $\sq_{g,f}$ factorises through the quotient functor $\qCat/(C\times B)\to(\pi^g_0)_*(\qCat_2\slice(C\times B))$ to define a functor
\begin{equation}\label{eq:the-real-sq-functor}
    \sq_{g,f}\colon(\pi^g_0)_*(\qCat_2\slice(C\times B))\op\longrightarrow\Set.
    \end{equation}
\end{lem}
\begin{proof}
If we are given a 2-cell
  \begin{equation*}
    \xymatrix@=1.5em{
      {X}\ar[dr]_(0.3){(c,b)} 
      \ar@/^1.5ex/[rr]^{u}_{}="one" \ar@/_1.5ex/[rr]_{u'}^{}="two" \ar@{=>}"one";"two"^{\tau}
      && *+!L(0.5){Y}\ar[dl]^(0.3){(\bar{c},\bar{b})} \\
      & {C\times B}&
    }
  \end{equation*}
  in $\qCat_2\slice(C\times B)$ and a 2-cell $\beta \in \sq_{g,f}(\bar{c},\bar{b})$ then the middle four interchange rule for the horizontal composite of the 2-cells $\beta$ and $\tau$ provides us with a commutative square
   \begin{equation*}
    \xymatrix@=1.5em{
      {f\bar{b}u} \ar@{=>}[r]^{\beta u}\ar@{=>}[d]_{f\bar{b}\tau} & 
      {g\bar{c}u}\ar@{=>}[d]^{g\bar{c}\tau} \\
      {f\bar{b}u'} \ar@{=>}[r]_{\beta u'} & {g\bar{c}u'}
    }
  \end{equation*} 
  whose vertical arrows are the identities on $fb$ and $gc$ respectively. Hence, $\beta u = \beta u'$, and we conclude that if $u$ and $u'$ are 1-cells in the same connected component of the category $\hom'_{C \times B}((c,b),(\bar{c},\bar{b}))$ then the functions $\sq_{g,f}(u)$ and $\sq_{g,f}(u')$ are identical.
\end{proof}

  This functor allows us to expose another aspect of the weak 2-universal property of weak comma objects: namely that the comma cone formed from the cospan $B \xrightarrow{f} A \xleftarrow{g} C$ represents the functor \eqref{eq:the-real-sq-functor}.

\begin{lem}\label{lem:cpts-and-comma-2-cells}
  The weakly universal comma cone 
  \begin{equation}\label{eq:first-comma-cone}
    \xymatrix@=10pt{
      & f \downarrow g \ar[dl]_{p_1} \ar[dr]^{p_0} \ar@{}[dd]|(.4){\psi}|{\Leftarrow}  \\ 
      C \ar[dr]_g & & B \ar[dl]^f \\ 
      & A}
  \end{equation}
  provides us with an element $\psi\in\sq_{g,f}(p_1,p_0)$ which is universal, in the usual sense, for the functor $\sq_{g,f}\colon(\pi^g_0)_*(\qCat_2\slice(C\times B))\op\to\Set$. Furthermore, any comma cone
   \begin{equation}\label{eq:other-comma-cone}
    \xymatrix@=10pt{
      & Q \ar[dl]_{q_1} \ar[dr]^{q_0} \ar@{}[dd]|(.4){\phi}|{\Leftarrow}  \\ 
      C \ar[dr]_g & & B \ar[dl]^f \\ 
      & A}
  \end{equation} 
  for which the 2-cell $\phi\in\sq_{g,f}(q_1,q_0)$ is a universal element of the functor $\sq_{g,f}$ displays $Q$ as a weak comma object in $\qCat_2$.
\end{lem}

\begin{proof}
  For each object $(c,b)\colon X\to C\times B$ of $\qCat_2\slice(C\times B)$ the element $\psi\in\sq_{g,f}(p_1,p_0)$ induces a function
  \begin{equation*}
\pi^g_0(\hom'_{C\times B}((c,b),(p_1,p_0)))\longrightarrow \sq_{g,f}(c,b)
  \end{equation*}
  which carries a functor $a\colon X\to f\comma g$ representing an element of the set on the left to the whiskered composite $\psi a$ on the right. The element $\psi\in\sq_{g,f}(p_1,p_0)$ is universal for $\sq_{g,f}$ if and only if each of those functions is a bijection. Surjectivity follows directly from the 1-cell induction property of $f\comma g$, and injectivity follows from the reformulation of Lemma \ref{lem:smothering} discussed in Observation~\ref{obs:1cell-ind-uniqueness-reloaded}. 

If $\phi\in\sq_{g,f}(q_1,q_0)$ is another element which is universal for $\sq_{g,f}$, then by Yoneda's lemma the objects $(p_1,p_0)\colon f\comma g\to C\times B$ and $(q_1,q_0)\colon Q\to C\times B$ are isomorphic in the category $(\pi^g_0)_*(\qCat_2\slice(C\times B))$ via an isomorphism whose action under $\sq_{g,f}$ carries $\psi\in\sq_{g,f}(p_1,p_0)$ to $\phi\in\sq_{g,f}(q_1,q_0)$. Proceeding as in Observation \ref{obs:groupoid-components}, we may pick representatives of this isomorphism and its inverse to provide a pair of 1-cells
\begin{equation*}
  \xymatrix@=1.5em{
    {Q} \ar@/_1ex/[rr]_w 
    \ar[dr]_(0.4){(q_1,q_0)} &&
    *+!L(0.5){f\comma g} \ar@/_1ex/[ll]_{w'}
    \ar[dl]^(0.4){(p_1,p_0)}  \\
    & {C\times B}
  }
\end{equation*}
which are related by a pair of 2-isomorphisms $\alpha\colon w'w\cong \id_{Q}$ and $\beta\colon ww'\cong\id_{f\comma g}$ in the slice 2-category $\qCat_2\slice(C\times B)$. The fact that this isomorphism carries $\phi$ to $\psi$ under the action of $\sq_{g,f}$ provides the 2-cellular equations $\psi w = \phi$ and $\phi w' = \psi$. 

  To prove the 1-cell induction property for the comma cone~\eqref{eq:other-comma-cone} suppose that we are given a comma cone~\eqref{eq:comma-cone}. The 1-cell induction property of $f\comma g$ provides us with a 1-cell $a\colon X\to f\comma g$ with the defining property that $p_0 a = b$, $p_1 a = c$, and $\psi a = \alpha$. The functor $w'\colon f\comma g\to Q$ satisfies the equations $q_0 w' = p_0$, $q_1 w' = p_1$, and $\phi w' = \psi$, so we have $q_0 w' a = p_0 a = b$, $q_1 w' a = p_1 a = c$, and $\phi w' a = \psi a = \alpha$. This demonstrates that $w' a\colon X\to Q$ is a 1-cell induced by the comma cone~\eqref{eq:comma-cone} with respect to the comma cone~\eqref{eq:other-comma-cone}.

  To prove the 2-cell induction property for the comma cone~\eqref{eq:other-comma-cone} suppose that we are given a pair of 1-cells $a,a'\colon X\to Q$ and a pair of 2-cells $\tau_0 \colon q_0 a \Rightarrow q_0 a'$ and $\tau_1 \colon q_1 a \Rightarrow q_1 a'$ satisfying the condition given in~\eqref{eq:comma-ind-2cell-compat} with respect to the comma cone~\eqref{eq:other-comma-cone}. The 1-cells $w a, w a'\colon X\to f\comma g$ and the 2-cells $\tau_0 \colon p_0 w a = q_0 a \Rightarrow q_0 a' = p_0 w a'$ and $\tau_1 \colon p_1 w a = q_1 a \Rightarrow q_1 a' = p_1 w a'$ also satisfy the condition given in~\eqref{eq:comma-ind-2cell-compat} with respect to the comma cone~\eqref{eq:first-comma-cone}. Hence, the 2-cell induction property of $f\comma g$ ensures that we have a 2-cell $\mu\colon w a \Rightarrow w a'$ with the defining properties that $p_0 \mu = \tau_0$ and $p_1\mu = \tau_1$. Combining this with the invertible 2-cell $\alpha\colon w'w\cong\id_Q$, we may construct a 2-cell 
  \begin{equation*}
    \xymatrix@C=4em{
      {\tau \defeq a} \ar@{=>}[r]_-{\cong}^-{\alpha^{-1} a} &
      {w'wa} \ar@{=>}[r]^{w'\mu} &
      {w'wa'} \ar@{=>}[r]_-{\cong}^-{\alpha a'} & {a'}
    }
  \end{equation*}
Because $\alpha$ is a 2-cell in the endo-hom-category in $\qCat_2\slice(C\times B)$ on the object $(q_1,q_0)\colon Q\to C\times B$, $q_0\alpha = \id_{q_0}$ and $q_1\alpha = \id_{q_1}$. It follows that $q_0 \tau = q_0 w' \mu = p_0 \mu = \tau_0$ and $q_1 \tau = q_1 w' \mu = p_1 \mu = \tau_1$, 
  which demonstrates that $\tau\colon a\Rightarrow a'$ satisfies the defining properties required of a 2-cell induced by the pair of 2-cells $\tau_0$ and $\tau_1$.

  The proof of 2-cell conservativity is of a similar ilk and is left to the reader.
\end{proof}


