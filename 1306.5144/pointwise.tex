%!TEX root = all.tex
% ******************************************************************
% ** Title:            The 2-category theory of quasi-categories
% **                   adjunctions
% ** Precis:        
% ** Author:           Emily Riehl and Dominic Verity
% ** Commenced:        2/3/2012
% ******************************************************************


\section{Pointwise universal properties}\label{sec:pointwise}

We have seen that limits and adjunctions can be encoded as absolute lifting diagrams in $\qCat_2$. In this section, we prove a theorem that allows such diagrams to be identified in practice: we show that absolute left or right lifting diagrams can be defined ``pointwise''  by specifying initial or terminal objects, respectively, in the appropriate comma or slice quasi-categories; the definition of Joyal's slice quasi-categories is recalled in \ref{defn:slices}.

 We conclude by proving a corollary of this result: that simplicial Quillen adjunctions between simplicial model categories are adjunctions of quasi-categories.  Adjunctions in homotopical contexts are commonly presented as Quillen adjunctions,  which can be replaced  by adjunctions of this type in good set-theoretical cases \cite{RezkSchwedeShipley:2001ss}. This result implies that such adjunctions can be imported into the quasi-categorical context.

\subsection{Pointwise absolute lifting}

Immediately from Definition \ref{defn:families.of.diagrams}, absolute lifting diagrams are preserved by pre-composition by all functors and, in particular, under evaluation at a vertex in the domain quasi-category.

\begin{defn}[pointwise universal property of absoluting lifting diagrams]\label{defn:pointwise-abs-lifting}
If the left-hand diagram
  \begin{equation}\label{eq:pointwise-absRlifting}
    \vcenter{\xymatrix{ \ar@{}[dr]|(.7){\Downarrow\lambda} & B \ar[d]^f \\ C \ar[r]_g \ar[ur]^\ell & A}} \qquad \rightsquigarrow \qquad   \vcenter{\xymatrix{ \ar@{}[dr]|(.7){\Downarrow\lambda c} & B \ar[d]^f \\ \Del^0 \ar[r]_{gc} \ar[ur]^{\ell c} & A}}
  \end{equation}
  is an absolute lifting diagram and $c$ is an object of $C$ then pre-composition by the functor $c\colon\Del^0\to C$ gives a 2-cell $\lambda c\colon f\ell c\Rightarrow gc$ which displays $\ell c\colon\Del^0\to B$ as an absolute right lifting of $gc\colon\Del^0\to A$ through $f\colon B\to A$. The family of absolute lifting diagrams as displayed on the right encode  the \emph{pointwise universal property} of the absolute lifting diagram displayed on the left.
\end{defn}

A special case of Proposition \ref{prop:right.liftings.as.fibred.terminal.objects} provides an alternate characterisation of a pointwise absolute lifting property:

\begin{lem}\label{lem:pointwise-terminal}
Given functors $g \colon C \to A$ and $f \colon B \to A$ the data of a pointwise absolute right lifting diagram at a vertex $c \in C$ is equally the data of a terminal object in the comma or slice quasi-categories $f \comma gc \simeq \slicer{f}{gc}$.
\end{lem}

Lemma~\ref{lem:slice-equiv-comma} supplies an equivalence $f \comma gc \simeq \slicer{f}{gc}$ along which we may transport terminal objects. Lemma \ref{lem:pointwise-terminal} demonstrates that if $g$ admits an absolute right lifting through $f$, then $f\comma gc \simeq \slicer{f}{gc}$ has a terminal object, for each vertex $c$ in the domain of $g$. In fact, these terminal objects suffice to demonstrate the existence of an absolute right lifting:

\begin{thm}\label{thm:pointwise} The functor $g\colon C\to A$ admits an absolute right lifting through the functor $f\colon B\to A$ if and only if for all objects $c$ of $C$ the quasi-category $f\comma gc\simeq\slicer{f}{gc}$ has a terminal object.
\end{thm}









\begin{proof}[Proof of Theorem \ref{thm:pointwise}]
Suppose each $\slicer{f}{gc}$ has a terminal object $\lambda_c \colon fb \to gc$, i.e., suppose we can fill any sphere $\partial\Delta^n \to \slicer{f}{gc}$ with $n \geq 1$ whose final vertex is $\lambda_c$. Unpacking the definition, we have assumed that we can solve any lifting problem 
\begin{equation}\label{eq:term.obj.assumption} \vcenter{\xymatrix{\boundary \Delta^n \ar[d] \ar[r] & \slicer{f}{gc} \\ \Delta^n \ar@{-->}[ur]}}\qquad\leftrightsquigarrow \qquad \vcenter{\xymatrix@C=25pt@R=30pt@!0{ \boundary\Delta^n \ar[rrrr] \ar[dd] \ar[dr] & && & B \ar[dr]^f \\ & \Lambda^{n+1,n+1} \ar[dd] \ar[rrrr] & & & & A \\ \Delta^n \ar[dr]_{\face^{n+1}} \ar@{-->}[uurrrr] \\ & \Delta^{n+1} \ar@{-->}[uurrrr]}}\end{equation} 
in $\qCat^\cattwo$ for which the $\fbv{n,n+1}$ edge of the $\Lambda^{n+1,n+1}$-horn in $A$ is $\lambda_c$.

It follows that we can solve any extension problem 
\begin{equation}\label{eq:term.obj.extension}\xymatrix@C=70pt@R=30pt@!0{  \boundary\Delta^n \times \Delta^{\fbv{0}} \ar[rr] \ar[dd] \ar[dr] &  & B \ar[dr]^f \\ &  \boundary\Delta^n \times \Delta^1\cup \Delta^n \times \Delta^{\fbv{1}}  \ar[dd] \ar[rr] &  & A \\ \Delta^n \times \Delta^{\fbv{0}} \ar[dr] \ar@{-->}[uurr]|\hole \\ & \Delta^n \times \Delta^1 \ar@{-->}[uurr]}\end{equation} for which the image of the edge between the vertices $(n,0)$ and $(n,1)$ is $\lambda_c$: The filler is constructed by inductively choosing images for the shuffles of $\Delta^n \times \Delta^1$  starting from the filled end of the specified cylinder. The images for all but the last shuffle are defined by filling the obvious inner horns in $A$. The final shuffle is attached by filling a $\Lambda^{n+1,n+1}$-horn in $A$ precisely of the form \eqref{eq:term.obj.assumption}.

We are interested in extension problems \eqref{eq:term.obj.extension} where the $n$-simplex in $A$ given as one end of the cylinder is in the image of some specified $n$-simplex of $C$ under $g$; these are precisely the data specified by a lifting problem \begin{equation}\label{eq:term.obj.lifting} \xymatrix{\Delta^0 \ar[r]_-{\fbv{n}} \ar@/^2ex/[rr]^{\lambda_c} & \boundary\Delta^n \ar[d] \ar[r] & f \comma g \ar[d]^{q_1} \\ &  \Delta^n \ar@{-->}[ur] \ar[r] & C}\end{equation} in which case the extension of \eqref{eq:term.obj.extension} provides a solution. We have just shown that any lifting problem \eqref{eq:term.obj.lifting} in which the final vertex of the sphere maps to a terminal object $\lambda_c \in \slicer{f}{gc}$ has a solution. By Lemma \ref{lem:RARI-lifting}, this tells us that $q_1 \colon f \comma g \to C$ admits a right adjoint right inverse $t \colon C \to f \comma g$, which by Proposition \ref{prop:right.liftings.as.fibred.terminal.objects} encodes the data of an absolute right lifting diagram, as displayed on the bottom right.
\[ 
    \vcenter{\xymatrix@=0.8em{
      & \save []+<0pt,1em>*+{C}\ar[d]^-{t}\ar@{=}@/_1.5ex/[ddl]
      \ar@/^1.5ex/[ddr]^{\ell}\restore & \\
      & {f\comma g}\ar[dr]_{q_0}\ar[dl]^{q_1} & \\
      {C}\ar[dr]_{g} & {\scriptstyle\Leftarrow\psi} &
      {B}\ar[dl]^{f} \\
      & {A} &
    }}
    \mkern20mu = \mkern20mu
    \vcenter{\xymatrix@=0.7em{
      & {C}\ar@{=}[dl]\ar[dr]^{\ell} & \\
      {C}\ar[dr]_{g} & {\scriptstyle\Leftarrow\lambda} & {B}\ar[dl]^{f} \\
      & {A} &
    }}
\]
\end{proof}



Theorem \ref{thm:pointwise} provides a useful criterion for the existence of absolute lifting diagrams. The following corollary supplies the corresponding detection result, identifying when a candidate lifting diagram has the desired universal property. The lifting property implies that each of its fibres admit terminal objects, a definition that will be introduced in the next section.

\begin{cor}\label{cor:pointwise}
  A triangle 
  \[    \xymatrix{ \ar@{}[dr]|(.7){\Downarrow\lambda} & B \ar[d]^f \\ C \ar[r]_g \ar[ur]^\ell & A}\] displays $\ell$ as an absolute right lifting of $g$ through $f$ if and only if it has that property pointwise.
\end{cor}

\begin{proof}
Necessity of the pointwise absolute lifting property of Definition \ref{defn:pointwise-abs-lifting} is immediate. Conversely, the assumed pointwise lifting tells us, in particular, that for each object $c$ in $C$ the slice quasi-category $f\comma gc\simeq\slicer{f}{gc}$ has a terminal object. Consequently, we may apply Theorem \ref{thm:pointwise} to construct a functor $\ell'\colon C\to A$ and 2-cell $\lambda'\colon f\ell'\Rightarrow g$ which displays $\ell'$ as an absolute right lifting of $g$ through $f$. 
  
The universal property of $(\ell',\lambda')$ applied to the triangle $(\ell,\lambda)$ provides us with a unique 2-cell $\tau\colon\ell\Rightarrow\ell'$ with the defining property that $\lambda'\cdot f\tau =\lambda$. Now both of the 2-cells $\lambda$ and $\lambda'$ possess the pointwise lifting property, the first by assumption and the second by construction. In other words, for all objects $c$ in $C$ the 2-cell $\lambda c\colon f\ell c\Rightarrow g c$ (respectively $\lambda' c\colon f\ell' c\Rightarrow g c$) displays $\ell c$ (respectively $\ell' c$) as an absolute right lifting of $g c$ through $f$ for all objects $c$ of $C$. Furthermore, the defining property of $\tau$ whiskers to tell us that $\lambda' c\cdot f(\tau c)=\lambda c$, so since $\lambda c$ and $\lambda' c$ both possess the absolute right lifting property it follows that $\tau c$ is an isomorphism. Applying Observation~\ref{obs:pointwise-iso-reprise}, we find that $\tau\colon\ell\Rightarrow\ell'$ is an isomorphism and thus that the given triangle is isomorphic to the absolute right lifting that we constructed and is thus itself an absolute right lifting.
\end{proof}

Proposition \ref{prop:families.of.diagrams}, which states that a quasi-category admits limits of a family of diagrams of a fixed shape if and only if it admits limits of each individual diagram in the family, is a special case of Theorem \ref{thm:pointwise}.

\begin{proof}[Proof of Proposition~\ref{prop:families.of.diagrams}]
If $A$ admits limits of each diagram in a family $k \colon K \to A^X$, then Proposition~\ref{prop:limits.are.limits} implies that for each vertex $\overline{d} \in K$, $\slicer{c}{k\overline{d}}$ has a terminal object. By Theorem~\ref{thm:pointwise}, it follows that $k$ admits an absolute right lifting along $c \colon A \to A^X$, i.e., $A$ admits limits of the family of diagrams $k \colon K \to A^X$.
\end{proof}

%Corollary \ref{cor:pointwise} also allows us to prove that Lurie's definition of adjunction given in \cite[5.2.2.8]{Lurie:2009fk} is equivalent to the 2-categorical definition.

%\begin{prop}\label{prop:joyal-adj-equals-lurie-adj} A pair of functors $f \colon B \to A$ and $u \colon A \to B$ together with a 2-cell $\epsilon \colon fu \To \id _A \in A^A$ define an adjunction $f \dashv u$ with counit $\epsilon$ if and only if the functor $B \comma u \to f \comma A$ over $A \times B$ induced by $\epsilon$, as described in proposition \ref{prop:adjointequiv}, pulls back over any pair of vertices $(a,b) \in A \times B$  to define an equivalence $b \comma ua \simeq fb \comma a$ of hom-spaces.
%\end{prop}
%\begin{proof}
%Observation \ref{obs:pointwise-adjoint-correspondence} demonstrated that the counit of an adjunction in the 2-category of quasi-categories has this property. 
%\end{proof}

\subsection{Simplicial Quillen adjunctions are adjunctions of quasi-categories}

Now we use Theorem \ref{thm:pointwise} to prove the assertions made in Example \ref{ex:simp.quillen.adj}: namely that any simplicial Quillen adjunction between simplicial model categories descends to an adjunction of quasi-categories. Another proof of this result is given in \cite[5.2.4.6]{Lurie:2009fk}. 

Recall that the quasi-category associated to a simplicial model category $\lcat{A}$ is defined by restricting to the full simplicial subcategory $\lcat{A}_{cf}$ of fibrant-cofibrant objects and then applying the homotopy coherent nerve $\nrvhc\colon \sSet\text{-}\Cat \to \sSet$. 

\begin{thm}\label{thm:simplicial-Quillen-adjunction} A simplicial Quillen adjunction \[\adjdisplay f -| u : \lcat{A} -> \lcat{B}.\] between simplicial model categories gives rise to an adjunction between the quasi-categories $\nrvhc\lcat{A}_{cf}$ and $\nrvhc\lcat{B}_{cf}$.
\end{thm}
\begin{proof}
We introduce a pair of simplicial categories  $\coll(f,\lcat{A})$ and $\coll(\lcat{B},u)$, with $\lcat{B}$ and $\lcat{A}$ as full subcategories that are jointly surjective on objects. Declare the hom-spaces from $a \in \lcat{A}$ to $b \in \lcat{B}$ to be empty and define \[ \coll(f,\lcat{A})(b,a) \defeq \lcat{A}(fb,a) \qquad \coll(\lcat{B},u)(b,a) \defeq \lcat{B}(b,ua).\] The simplicial adjunction $f\dashv u$ is encoded in the proposition that the simplicial categories $\coll(f,\lcat{A})$ and $\coll(\lcat{B},u)$ are isomorphic under $\lcat{B} \coprod \lcat{A}$.

Now write $\coll(f,\lcat{A})_{cf} \cong \coll(\lcat{B},u)_{cf}$ for the full simplicial sub-categories spanned by the fibrant-cofibrant objects of $\lcat{A}$ and $\lcat{B}$.  Via these restrictions, we obtain a diagram \[ \lcat{B}_{cf} \hookrightarrow \coll(f,\lcat{A})_{cf} \cong \coll(\lcat{B},u)_{cf} \hookleftarrow \lcat{A}_{cf}\] of locally Kan simplicial categories. Applying the homotopy coherent nerve, we have a pair of isomorphic cospans in $\qCat_2$: \[\xymatrix{ \ar@{}[dr]|(.7){\Uparrow\psi} & \nrvhc\lcat{A}_{cf} \ar[d] & &  \ar@{}[dr]|(.7){\Downarrow\beta} & \nrvhc\lcat{B}_{cf} \ar[d]^i  \\ \nrvhc\lcat{B}_{cf} \ar@{-->}[ur]^{\overline{f}} \ar[r] & \nrvhc\coll(f,\lcat{A})_{cf} & & \nrvhc\lcat{A}_{cf} \ar@{-->}[ur]^{\overline{u}} \ar[r]_-j &\nrvhc\coll(\lcat{B},u)_{cf} }\] Our objective is to define an absolute left lifting $(\overline{f}, \psi)$ and an absolute right lifting $(\overline{u},\beta)$. Proposition \ref{prop:absliftingtranslation2} and its dual then provides a fibred equivalence \[ \overline{f} \comma \nrvhc\lcat{A}_{cf} \simeq \nrvhc\lcat{B}_{cf} \comma \overline{u}\] over $\nrvhc\lcat{A}_{cf} \times \nrvhc\lcat{B}_{cf}$, which by Proposition \ref{prop:adjointequivconverse} implies that $\adjinline \overline{f} -| \overline{u} : \nrvhc\lcat{A}_{cf} -> \nrvhc\lcat{B}_{cf}.$ is an adjunction of quasi-categories.

The arguments building the absolute right lifting diagram $(\overline{u},\beta)$ and the absolute left lifting diagram $(\overline{f},\psi)$ are entirely dual. Interpreting the statement of Theorem \ref{thm:pointwise} in this context, we are asked to produce, for each fibrant-cofibrant object $a \in \lcat{A}$, a terminal object in $\slicer{i}{ja}$, defined to be the pullback of the slice  quasi-category $\slicer{(\nrvhc\coll(\lcat{B},u)_{cf})}{a}$ along the natural inclusion $i \colon N\lcat{B}_{cf} \to N\coll(\lcat{B},u)_{cf}$. To that end, choose a  cofibrant replacement $q \colon t \to ua$ in the model category $\lcat{B}$ such that the map $q$ is a trivial fibration. It follows that whenever $b \in \lcat{B}$ is cofibrant, the natural map $q_*\colon \lcat{B}(b,t) \to \lcat{B}(b,ua)$ is a trivial fibration between Kan complexes. We claim that $q$ is terminal in $\slicer{i}{ja}$.

Let $\gC$ denote the left adjoint to the homotopy coherent nerve. Unpacking the definition, an $n$-simplex in $\slicer{i}{ja}$ is \[\xymatrix{  \Delta^n \ar[d]_{\face^{n+1}} \ar[r] & \nrvhc\lcat{B}_{cf} \ar[d]  & & \gC\Delta^n\ar[d]_{\face^{n+1}} \ar[r] & \lcat{B}_{cf} \ar[d]^i\\ \Delta^n\join\Delta^0\cong  \Delta^{n+1} \ar[r] & \nrvhc\coll(\lcat{B},u)_{cf} & \leftrightsquigarrow & \gC\Delta^{n+1} \ar[r] & \coll(\lcat{B},u)_{cf} \\ \Delta^{\fbv{n+1}} \ar[u] \ar[ur]_a  &  & & \catone \ar[u]^{\mathrm{last}} \ar[ur]_a } \]  The vertex $q \in \nrvhc\coll(\lcat{B},u)_{cf}$ is terminal if and only if we can extend any diagram of simplicial functors 
\begin{equation}\label{eq:simp.adj.extension}\xymatrix@C=30pt@R=35pt@!0{ \gC\boundary\Delta^n \ar[rrrr] \ar[dd] \ar[dr] & && & \lcat{B}_{cf} \ar[dr] \\ & \gC\Lambda^{n+1,n+1} \ar[dd] \ar[rrrr] & & & & \coll(\lcat{B},u)_{cf} \\ \gC\Delta^n \ar[dr]_{\face^{n+1}} \ar@{-->}[uurrrr] \\ & \gC\Delta^{n+1} \ar@{-->}[uurrrr]}\end{equation} in which the unique vertex in the hom-space between the objects $n$ and $n+1$ in the simplicial category $\gC\Lambda^{n+1,n+1}$ is mapped to $q \in \lcat{B}(t,ua)$. 

The simplicial categories $\gC\Lambda^{n+1,n+1}$ and $\gC\Delta^{n+1}$ have objects $0,\ldots, n+1$ and all but two of the same hom-spaces, the only exceptions being the hom-spaces from $0$ to $n$ and to $n+1$. We have $\gC\Delta^{n+1}(0,n) \cong (\Delta^1)^{n-1}$ and $\gC\Delta^{n+1}(0,n+1)\cong(\Delta^1)^n$, while $\gC\Lambda^{n+1,n+1}(0,n) \cong \boundary(\Delta^1)^{n-1}$ and $\gC\Lambda^{n+1,n+1}(0,n+1)$ is the open box $B \hookrightarrow (\Delta^1)^n$ with the interior of the $n$-cube and one face removed \cite[1.1.5.10]{Lurie:2009fk} and \cite[16.5.10]{Riehl:2014kx}. In this way, writing $b \in \lcat{B}$ for the image of the object 0, the extension problem \eqref{eq:simp.adj.extension} in the category of simplicial categories reduces to an extension problem
\[\xymatrix@C=30pt@R=35pt@!0{ \boundary(\Delta^1)^{n-1} \ar[rrrr] \ar@{ >->}[dd] \ar[dr] & && & \lcat{B}(b,t) \ar@{->>}[dr]^{q_*}_(.4){\rotatebox{135}{$\labelstyle\sim$}} \\ & B \ar@{ >->}[dd]^{\rotatebox{90}{$\labelstyle\sim$}} \ar[rrrr] & & & & \lcat{B}(b,ua) \\ (\Delta^1)^{n-1} \ar[dr] \ar@{-->}[uurrrr] \\ & (\Delta^1)^n \ar@{-->}[uurrrr]}\] in the category of simplicial sets. For the reader's convenience, we have used the standard decorations to mark cofibrations, fibrations, and weak equivalences in Quillen's model structure on simplicial sets.

The extension \eqref{eq:simp.adj.extension} may be achieved by first  extending along the map $B \hookrightarrow (\Delta^1)^n$ in the Kan complex $\lcat{B}(b,ua)$. This chooses an image under the map $q_*$ for the $(n-1)$-cube missing from the box $B$. An $(n-1)$-cube in $\lcat{B}(b,t)$ with this image can be found by lifting the cofibration $\boundary(\Delta^1)^{n-1} \hookrightarrow (\Delta^1)^{n-1}$ against the trivial fibration $q_*$.
\end{proof}



