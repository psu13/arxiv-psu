\pdfoutput=1
% ******************************************************************
% ** Title:            The 2-category theory of quasi-categories
% **                   Main file
% ** Precis:        
% ** Author:           Emily Riehl and Dominic Verity
% ** Commenced:        2/3/2012
% ******************************************************************

\documentclass[12pt,reqno]{amsart}

\usepackage{a4wide}
\usepackage{amsmath}
\usepackage{amssymb}
\usepackage{amsxtra}
\usepackage{amsthm}
%\usepackage{MnSymbol}
\usepackage{mathtools}
\usepackage{extpfeil}
\usepackage{enumitem}
\usepackage[mathscr]{eucal}
\usepackage{graphicx}
%\usepackage{asymptote}
%\usepackage{ifnextok}
\usepackage[T1]{fontenc}

\usepackage[all]{xy}
\SelectTips{cm}{}
\SilentMatrices
\ifx\pdfoutput\undefined
  \xyoption{dvips}
\fi

%\usepackage[silent]{fixme}
\usepackage{hyperref}
\usepackage{slashed}

% draft settings

%\usepackage[inline]{showlabels}
%\fxsetup{
%    status=draft,
%    layout={marginclue,noinline}}
%\fxsetface{inline}{\itshape}

% Standard macro definitions

% Macros shared among templates

\usepackage[utf8]{inputenc}

\usepackage{graphicx}
\setkeys{Gin}{width=\linewidth,totalheight=\textheight,keepaspectratio}

\definecolor{darkblue}{HTML}{00416A}

\usepackage{longtable}
\usepackage{booktabs}
\usepackage{amssymb}
\usepackage{amsmath}
\usepackage{amsthm}
%\usepackage{commath}
\usepackage{boxedminipage}
\usepackage{microtype}

\usepackage{makeidx}
\usepackage{hyperref}

% attempts to prevent margin figures from being cut off
\usepackage{marginfix}
\usepackage[morefloats=100]{morefloats}

\newtheorem{Definition}{Definition}
\newtheorem{Theorem}{Theorem}
\newtheorem{Lemma}{Lemma}
\newtheorem{Exercise}{Exercise}
\newtheorem{Fact}{Fact}
\newtheorem{Proposition}{Proposition}
\newtheorem{Assumption}{Assumption}
\newenvironment{Algorithm}{\begin{center}\begin{boxedminipage}{0.92\textwidth}}{\end{boxedminipage}\end{center}}
\newenvironment{Proof}{\begin{proof}}{\end{proof}}
\newtheorem*{summary*}{Summary}
\newenvironment{Summary}{\begin{center}\begin{minipage}{0.92\textwidth}\begin{summary*}}{\end{summary*}\end{minipage}\end{center}\medskip}
\newenvironment{EmphBox}{\begin{center}\begin{minipage}{0.8\textwidth}}{\end{minipage}\end{center}\medskip}

\providecommand{\tightlist}{%
  \setlength{\itemsep}{0pt}\setlength{\parskip}{0pt}}

\renewcommand{\bot}{\perp}
\renewcommand{\hat}{\widehat}



% Global Stuff

% Adjust list environments.

\setlist{}
\setenumerate{leftmargin=*,labelindent=\parindent}
\setitemize{leftmargin=*,labelindent=0.5\parindent}
\setdescription{leftmargin=1em}

% Theorem declarations.

\swapnumbers

\theoremstyle{plain}
\newtheorem{thm}{Theorem}[subsection]
\newtheorem{lem}[thm]{Lemma}
\newtheorem{cor}[thm]{Corollary}
\newtheorem{conj}[thm]{Conjecture}
\newtheorem{prop}[thm]{Proposition}

\theoremstyle{definition}
\newtheorem{defn}[thm]{Definition}
\newtheorem{ex}[thm]{Example}
\newtheorem{ntn}[thm]{Notation}
\newtheorem{hyp}[thm]{Hypothesis}

\theoremstyle{remark}
\newtheorem{obs}[thm]{Observation}
\newtheorem{rec}[thm]{Recall}
\newtheorem{rmk}[thm]{Remark}
\newtheorem{exs}[thm]{Exercise}

% Number equations in subsections
\makeatletter
\let\c@equation\c@thm
\makeatother
\numberwithin{equation}{subsection}

% General page formatting tweaks.

\raggedbottom

% Import asymptote code for drawing string diagrams
% for arrows in Adj.

%\begin{asydef}
%  import "adjunction" as adjunction;
%\end{asydef}

% Macros for cross reference between these papers

\newcommand{\refI}[1]{I.\ref*{found:#1}}
\newcommand{\refII}[1]{II.\ref*{cohadj:#1}}

% Titles, Authors etc.

\title{The 2-category theory of quasi-categories}
\date{$6^{\text{th}}$ May 2015}

\author[Riehl]{Emily Riehl}
\address{
  Department of Mathematics \\
  Harvard University \\
  Cambridge, MA 02138\\
  USA
}
\email{eriehl@math.harvard.edu}

\author[Verity]{Dominic Verity}
\address{
  Centre of Australian Category Theory \\
  Macquarie University \\
  NSW 2109 \\
  Australia
}
\email{dominic.verity@mq.edu.au}

\newlabel{reedy:eq:reedy's.po}{{1.1}{2}{Introduction\relax }{equation.1.1}{}}
\newlabel{reedy:eq:reedy's.delta}{{1.2}{3}{Introduction\relax }{equation.1.2}{}}
\newlabel{reedy:ntn:index-stuff}{{1.4}{4}{general index notation\relax }{thm.1.4}{}}
\newlabel{reedy:ex:tensor-cotensor}{{1.6}{5}{tensors and cotensors\relax }{thm.1.6}{}}
\newlabel{reedy:defn:ends-and-coends}{{1.7}{6}{ends and coends\relax }{thm.1.7}{}}
\newlabel{reedy:defn:cat-of-elts}{{1.8}{6}{categories of elements\relax }{thm.1.8}{}}
\newlabel{reedy:obs:coend-in-sets}{{1.9}{7}{coends in sets\relax }{thm.1.9}{}}
\newlabel{reedy:eq:pb-coends-in-sets}{{1.10}{7}{coends in sets\relax }{equation.1.10}{}}
\newlabel{reedy:eq:wlimformula}{{1.12}{7}{weighted limits and colimits\relax }{equation.1.12}{}}
\newlabel{reedy:ex:weighted-conical}{{1.13}{7}{terminal weights\relax }{thm.1.13}{}}
\newlabel{reedy:ex:weighted-yoneda}{{1.14}{8}{representable weights\relax }{thm.1.14}{}}
\newlabel{reedy:eq:yoneda}{{1.15}{8}{representable weights\relax }{equation.1.15}{}}
\newlabel{reedy:ex:matching.preview}{{1.17}{8}{\relax }{thm.1.17}{}}
\newlabel{reedy:obs:weighted.as.ordinary}{{1.19}{8}{weighted limits as ordinary limits\relax }{thm.1.19}{}}
\newlabel{reedy:obs:wcolim-in-sets}{{1.21}{8}{weighted colimits in sets\relax }{thm.1.21}{}}
\newlabel{reedy:sec:reedy}{{2}{9}{Reedy categories\relax }{section.2}{}}
\newlabel{reedy:defn:reedy}{{2.1}{9}{Reedy categories\relax }{thm.2.1}{}}
\newlabel{reedy:ex:poset.reedy}{{2.3}{9}{finite posets\relax }{thm.2.3}{}}
\newlabel{reedy:ex:pushout.reedy.alt}{{2.4}{9}{the pushout diagram\relax }{thm.2.4}{}}
\newlabel{reedy:ex:parallel.pair.reedy}{{2.5}{10}{the parallel pair\relax }{thm.2.5}{}}
\newlabel{reedy:defn:factorisations}{{2.8}{10}{categories of factorisations\relax }{thm.2.8}{}}
\newlabel{reedy:lem:reedy-fact-connect}{{2.9}{10}{\relax }{thm.2.9}{}}
\newlabel{reedy:sec:latching}{{3}{11}{Latching and matching objects\relax }{section.3}{}}
\newlabel{reedy:rec:skel-coskel}{{3.1}{11}{skeleta and coskeleta\relax }{thm.3.1}{}}
\newlabel{reedy:eq:4}{{3.2}{12}{skeleta and coskeleta\relax }{equation.3.2}{}}
\newlabel{reedy:obs:skel-colim}{{3.3}{12}{\relax }{thm.3.3}{}}
\newlabel{reedy:eq:5}{{3.4}{12}{\relax }{equation.3.4}{}}
\newlabel{reedy:lem:skeleta-colimit}{{3.5}{13}{\relax }{thm.3.5}{}}
\newlabel{reedy:eq:6}{{3.6}{13}{\relax }{equation.3.6}{}}
\newlabel{reedy:obs:skel-reps}{{3.7}{13}{skeleta of representables and factorisations\relax }{thm.3.7}{}}
\newlabel{reedy:eq:sk-rep-coend}{{3.8}{13}{skeleta of representables and factorisations\relax }{equation.3.8}{}}
\newlabel{reedy:lem:inductive.defn}{{3.10}{15}{inductive definition of diagrams\relax }{thm.3.10}{}}
\newlabel{reedy:obs:inductive.defn}{{3.11}{15}{inductive definition of natural transformations\relax }{thm.3.11}{}}
\newlabel{reedy:ex:sequence-induction}{{3.12}{15}{\relax }{thm.3.12}{}}
\newlabel{reedy:defn:latching}{{3.13}{16}{latching and matching objects\relax }{thm.3.13}{}}
\newlabel{reedy:ex:simp-latching-1}{{3.14}{16}{\relax }{thm.3.14}{}}
\newlabel{reedy:obs:weights-latching}{{3.15}{16}{weights for latching and matching objects\relax }{thm.3.15}{}}
\newlabel{reedy:eq:7}{{3.16}{16}{weights for latching and matching objects\relax }{equation.3.16}{}}
\newlabel{reedy:lem:sk-rep-reedy}{{3.17}{17}{skeleta of the representables of a Reedy category\relax }{thm.3.17}{}}
\newlabel{reedy:obs:cons-skeleta}{{3.18}{17}{characterising the boundary of a representable\relax }{thm.3.18}{}}
\newlabel{reedy:ex:sequence.latching}{{3.19}{17}{\relax }{thm.3.19}{}}
\newlabel{reedy:ex:pushout.latching}{{3.20}{17}{\relax }{thm.3.20}{}}
\newlabel{reedy:ex:parallel.pair.latching}{{3.21}{18}{\relax }{thm.3.21}{}}
\newlabel{reedy:ex:simp-latching-2}{{3.22}{18}{\relax }{thm.3.22}{}}
\newlabel{reedy:obs:latching.ordinary.colimit}{{3.23}{18}{latching objects as ordinary colimits\relax }{thm.3.23}{}}
\newlabel{reedy:sec:Leibniz-Reedy}{{4}{19}{Leibniz constructions and the Reedy model structure\relax }{section.4}{}}
\newlabel{reedy:obs:box-product}{{4.2}{19}{Leibniz's formula\relax }{thm.4.2}{}}
\newlabel{reedy:defn:leibniz}{{4.4}{20}{the Leibniz construction\relax }{thm.4.4}{}}
\newlabel{reedy:eq:13}{{4.5}{20}{the Leibniz construction\relax }{equation.4.5}{}}
\newlabel{reedy:ex:boundary-prod}{{4.6}{20}{\relax }{thm.4.6}{}}
\newlabel{reedy:obs:leibniz-isos}{{4.7}{20}{Leibniz and structural isomorphisms\relax }{thm.4.7}{}}
\newlabel{reedy:lem:leibniz-cocts}{{4.8}{21}{Leibniz and colimit preservation\relax }{thm.4.8}{}}
\newlabel{reedy:eq:2varadj}{{4.9}{21}{\S \ Leibniz constructions\relax }{equation.4.9}{}}
\newlabel{reedy:lem:leibniz-close}{{4.10}{21}{Leibniz and closures\relax }{thm.4.10}{}}
\newlabel{reedy:obs:leibniz-lifting-properties}{{4.11}{22}{Leibniz and lifting properties\relax }{thm.4.11}{}}
\newlabel{reedy:defn:relative-maps}{{4.14}{22}{relative latching and matching maps\relax }{thm.4.14}{}}
\newlabel{reedy:obs:relative.lifting}{{4.16}{23}{relative latching and matching maps and lifting problems\relax }{thm.4.16}{}}
\newlabel{reedy:eq:i-lift-p}{{4.17}{23}{relative latching and matching maps and lifting problems\relax }{equation.4.17}{}}
\newlabel{reedy:thm:model-structure}{{4.18}{24}{the Reedy model structure\relax }{thm.4.18}{}}
\newlabel{reedy:sec:Leibniz-rel-cell-cx}{{5}{24}{Leibniz constructions and cell complexes\relax }{section.5}{}}
\newlabel{reedy:obs:leibniz-comp}{{5.1}{24}{Leibniz and composites\relax }{thm.5.1}{}}
\newlabel{reedy:defn:transfinite-composites}{{5.3}{25}{transfinite composites\relax }{thm.5.3}{}}
\newlabel{reedy:def:rel-cell}{{5.4}{25}{cell complexes\relax }{thm.5.4}{}}
\newlabel{reedy:eq:10}{{5.5}{26}{cell complexes\relax }{equation.5.5}{}}
\newlabel{reedy:ntn:phi-map}{{5.6}{26}{\relax }{thm.5.6}{}}
\newlabel{reedy:lem:leibniz-tcofp}{{5.7}{26}{Leibniz bifunctors and cell complexes\relax }{thm.5.7}{}}
\newlabel{reedy:eq:11}{{5.8}{27}{Leibniz constructions and cell complexes\relax }{equation.5.8}{}}
\newlabel{reedy:obs:leibniz-tcofp}{{5.9}{27}{\relax }{thm.5.9}{}}
\newlabel{reedy:ex:latch-comp}{{5.10}{28}{relative latching maps of composites II\relax }{thm.5.10}{}}
\newlabel{reedy:cor:leibniz-tcofp}{{5.11}{28}{\relax }{thm.5.11}{}}
\newlabel{reedy:prop:leibniz-tcofp}{{5.12}{28}{\relax }{thm.5.12}{}}
\newlabel{reedy:sec:cellular}{{6}{29}{Cellular presentations and Reedy categories\relax }{section.6}{}}
\newlabel{reedy:obs:two-sided}{{6.1}{29}{skeleta of two-sided representables\relax }{thm.6.1}{}}
\newlabel{reedy:obs:building-skeleta-rep}{{6.2}{29}{building up for skeleta of representables\relax }{thm.6.2}{}}
\newlabel{reedy:prop:building-up}{{6.3}{30}{general building up\relax }{thm.6.3}{}}
\newlabel{reedy:eq:skel-seq}{{6.4}{31}{general building up\relax }{equation.6.4}{}}
\newlabel{reedy:eq:c-cell}{{6.5}{31}{general building up\relax }{equation.6.5}{}}
\newlabel{reedy:eq:skel-seq-cells}{{6.6}{31}{Cellular presentations and Reedy categories\relax }{equation.6.6}{}}
\newlabel{reedy:cor:building-up}{{6.7}{32}{\relax }{thm.6.7}{}}
\newlabel{reedy:cor:B-cell-complex}{{6.8}{32}{\relax }{thm.6.8}{}}
\newlabel{reedy:ex:EZ.simp.set}{{6.9}{33}{\relax }{thm.6.9}{}}
\newlabel{reedy:sec:proof}{{7}{33}{Proof of the Reedy model structure\relax }{section.7}{}}
\newlabel{reedy:lem:triv-cof-char}{{7.1}{33}{\relax }{thm.7.1}{}}
\newlabel{reedy:lem:lifting}{{7.3}{34}{lifting\relax }{thm.7.3}{}}
\newlabel{reedy:lem:factorisation}{{7.4}{34}{factorisation\relax }{thm.7.4}{}}
\newlabel{reedy:eq:factorisation-defn}{{7.5}{35}{Proof of the Reedy model structure\relax }{equation.7.5}{}}
\newlabel{reedy:prop:reedy.cof.gen}{{7.7}{35}{\relax }{thm.7.7}{}}
\newlabel{reedy:sec:hocolim}{{8}{36}{Homotopy limits and colimits\relax }{section.8}{}}
\newlabel{reedy:ex:hocoeq}{{8.2}{37}{homotopy coequalisers\relax }{thm.8.2}{}}
\newlabel{reedy:ex:hoeq}{{8.3}{37}{homotopy equalisers\relax }{thm.8.3}{}}
\newlabel{reedy:prop:reedy.model.dual}{{8.4}{38}{\relax }{thm.8.4}{}}
\newlabel{reedy:ex:mapping}{{8.5}{38}{mapping telescopes\relax }{thm.8.5}{}}
\newlabel{reedy:eq:sequence}{{8.6}{38}{mapping telescopes\relax }{equation.8.6}{}}
\newlabel{reedy:eq:sequence.replacement}{{8.7}{38}{mapping telescopes\relax }{equation.8.7}{}}
\newlabel{reedy:ex:hopushout}{{8.8}{38}{homotopy pushouts\relax }{thm.8.8}{}}
\newlabel{reedy:eq:pushout.compare}{{8.9}{39}{homotopy pushouts\relax }{equation.8.9}{}}
\newlabel{reedy:ex:stupid-simplicial}{{8.10}{39}{\relax }{thm.8.10}{}}
\newlabel{reedy:sec:connected-weights}{{9}{39}{Connected weights\relax }{section.9}{}}
\newlabel{reedy:prop:2/3-SM7}{{9.1}{39}{\relax }{thm.9.1}{}}
\newlabel{reedy:obs:fibrant.constants}{{9.2}{40}{\relax }{thm.9.2}{}}
\newlabel{reedy:cor:connected.weights}{{9.4}{40}{\relax }{thm.9.4}{}}
\newlabel{reedy:sec:simplicial}{{10}{41}{Simplicial model categories and geometric realization\relax }{section.10}{}}
\newlabel{reedy:thm:simp.model.cat}{{10.3}{42}{\relax }{thm.10.3}{}}
\newlabel{reedy:eq:simpmodelreedy}{{10.4}{42}{Simplicial model categories and geometric realization\relax }{equation.10.4}{}}
\newlabel{reedy:obs:geo-filt}{{10.7}{43}{skeletal filtration of geometric realization\relax }{thm.10.7}{}}
\newlabel{reedy:eq:reedy's.po2}{{10.8}{43}{skeletal filtration of geometric realization\relax }{equation.10.8}{}}
\newlabel{reedy:tocindent-1}{0pt}
\newlabel{reedy:tocindent0}{15.01021pt}
\newlabel{reedy:tocindent1}{26.75734pt}
\newlabel{reedy:tocindent2}{0pt}
\newlabel{reedy:tocindent3}{0pt}

\newlabel{cohadj:sec:intro-adj-data}{{1.1}{2}{Adjunction data}{subsection.1.1}{}}
\newlabel{cohadj:eq:sampletriangleidentities}{{1.1.1}{3}{Adjunction data}{equation.1.1.1}{}}
\newlabel{cohadj:eq:horn-1}{{1.1.2}{3}{Adjunction data}{equation.1.1.2}{}}
\newlabel{cohadj:eq:middle.four}{{1.1.3}{4}{Adjunction data}{equation.1.1.3}{}}
\newlabel{cohadj:eq:horn-2}{{1.1.4}{4}{Adjunction data}{equation.1.1.4}{}}
\newlabel{cohadj:sec:computads}{{2}{7}{Simplicial computads}{section.2}{}}
\newlabel{cohadj:subsec:computads}{{2.1}{7}{Simplicial categories and simplicial computads}{subsection.2.1}{}}
\newlabel{cohadj:ntn:whiskering}{{2.1.2}{8}{whiskering in a simplicial category}{thm.2.1.2}{}}
\newlabel{cohadj:ntn:generic-cattwo}{{2.1.3}{8}{the generic $n$-arrow}{thm.2.1.3}{}}
\newlabel{cohadj:defn:computads}{{2.1.4}{8}{(relative) simplicial computads}{thm.2.1.4}{}}
\newlabel{cohadj:obs:simp-computad-char}{{2.1.5}{8}{an explicit characterisation of simplicial computads}{thm.2.1.5}{}}
\newlabel{cohadj:eq:computad-arrrow-decomp}{{2.1.6}{9}{an explicit characterisation of simplicial computads}{equation.2.1.6}{}}
\newlabel{cohadj:ex:gothic-C}{{2.1.10}{9}{}{thm.2.1.10}{}}
\newlabel{cohadj:ex:homotopy-coherent-simplex}{{2.1.11}{10}{}{thm.2.1.11}{}}
\newlabel{cohadj:subsec:subcomputads}{{2.2}{10}{Simplicial subcomputads}{subsection.2.2}{}}
\newlabel{cohadj:ex:mono-subcomputad}{{2.2.3}{10}{}{thm.2.2.3}{}}
\newlabel{cohadj:defn:generated-subcomputad}{{2.2.4}{11}{}{thm.2.2.4}{}}
\newlabel{cohadj:ex:simp-cat.(co)skeleta}{{2.2.5}{11}{(co)skeleta of simplicial categories}{thm.2.2.5}{}}
\newlabel{cohadj:prop:simp-computad-maps}{{2.2.6}{11}{}{thm.2.2.6}{}}
\newlabel{cohadj:sec:generic-adj}{{3}{12}{The generic adjunction}{section.3}{}}
\newlabel{cohadj:ssec:graphical}{{3.1}{12}{A graphical calculus for the simplicial category \protect \texorpdfstring {$\protect \Adj $}{Adj}}{subsection.3.1}{}}
\newlabel{cohadj:eq:samplesquiggle}{{3.1.1}{13}{A graphical calculus for the simplicial category \protect \texorpdfstring {$\protect \Adj $}{Adj}}{equation.3.1.1}{}}
\newlabel{cohadj:def:strict-und-squiggles}{{3.1.2}{14}{strictly undulating squiggles}{thm.3.1.2}{}}
\newlabel{cohadj:item:strict-und-squiggle-1}{{{{(i)}}}{14}{strictly undulating squiggles}{Item.1}{}}
\newlabel{cohadj:item:strict-und-squiggle-2}{{{{(ii)}}}{14}{strictly undulating squiggles}{Item.2}{}}
\newlabel{cohadj:item:und-squiggle-2}{{{{(ii)$'$}}}{14}{strictly undulating squiggles}{Item.3}{}}
\newlabel{cohadj:def:comp-squiggles}{{3.1.3}{14}{composing squiggles}{thm.3.1.3}{}}
\newlabel{cohadj:eq:samplesquiggle2}{{3.1.6}{15}{simplicial action on strictly undulating squiggles}{equation.3.1.6}{}}
\newlabel{cohadj:obs:interval.rep}{{3.1.7}{17}{interval representation}{thm.3.1.7}{}}
\newlabel{cohadj:obs:squiggle-vertices}{{3.1.9}{18}{the vertices of an arrow in $\Adj $}{thm.3.1.9}{}}
\newlabel{cohadj:prop:adjcomputad}{{3.1.10}{19}{}{thm.3.1.10}{}}
\newlabel{cohadj:ex:sample-adjunction-data}{{3.1.11}{19}{adjunction data in $\Adj $}{thm.3.1.11}{}}
\newlabel{cohadj:eq:ualpha-faces}{{3.1.12}{20}{adjunction data in $\Adj $}{equation.3.1.12}{}}
\newlabel{cohadj:ssec:2-cat-adj}{{3.2}{21}{The simplicial category \texorpdfstring {$\protect \Adj $}{Adj} as a 2-category}{subsection.3.2}{}}
\newlabel{cohadj:prop:adj.2-cat}{{3.2.2}{21}{}{thm.3.2.2}{}}
\newlabel{cohadj:eq:samplesquiggles4}{{3.2.3}{21}{The simplicial category \texorpdfstring {$\protect \Adj $}{Adj} as a 2-category}{equation.3.2.3}{}}
\newlabel{cohadj:eq:samplesquiggle3}{{3.2.4}{22}{The simplicial category \texorpdfstring {$\protect \Adj $}{Adj} as a 2-category}{equation.3.2.4}{}}
\newlabel{cohadj:rmk:hammock}{{3.2.5}{22}{}{thm.3.2.5}{}}
\newlabel{cohadj:eq:karol}{{3.2.6}{23}{}{equation.3.2.6}{}}
\newlabel{cohadj:ssec:adj-is-adj}{{3.3}{23}{The 2-categorical universal property of \protect \texorpdfstring {$\protect \Adj $}{Adj}}{subsection.3.3}{}}
\newlabel{cohadj:obs:2-cat.as.simp-cat}{{3.3.1}{24}{2-categories as simplicial categories}{thm.3.3.1}{}}
\newlabel{cohadj:eq:genadjdata}{{3.3.3}{24}{the adjunction in \protect \texorpdfstring {$\protect \Adj $}{Adj}}{equation.3.3.3}{}}
\newlabel{cohadj:prop:2-cat-univ-Adj}{{3.3.4}{24}{a 2-categorical universal property of \protect \texorpdfstring {$\protect \Adj $}{Adj}}{thm.3.3.4}{}}
\newlabel{cohadj:cor:schanuel-street-iso}{{3.3.5}{25}{}{thm.3.3.5}{}}
\newlabel{cohadj:obs:delta.adj.duality}{{3.3.6}{26}{adjunctions in \protect \texorpdfstring {$\protect \Del +$}{Delta+}}{thm.3.3.6}{}}
\newlabel{cohadj:eq:elem.op.adj}{{3.3.7}{26}{adjunctions in \protect \texorpdfstring {$\protect \Del +$}{Delta+}}{equation.3.3.7}{}}
\newlabel{cohadj:rmk:schanuel-street}{{3.3.8}{26}{the Schanuel and Street 2-category $\Adj $}{thm.3.3.8}{}}
\newlabel{cohadj:sec:adjunction-data}{{4}{28}{Adjunction data}{section.4}{}}
\newlabel{cohadj:ssec:fillable}{{4.1}{28}{Fillable arrows}{subsection.4.1}{}}
\newlabel{cohadj:obs:fillable-dist-face}{{4.1.2}{29}{fillable arrows and distinguished faces}{thm.4.1.2}{}}
\newlabel{cohadj:lem:fillablecodim1}{{4.1.3}{29}{}{thm.4.1.3}{}}
\newlabel{cohadj:lem:fillablecodim1'}{{4.1.4}{30}{}{thm.4.1.4}{}}
\newlabel{cohadj:ssec:parental}{{4.2}{30}{Parental subcomputads}{subsection.4.2}{}}
\newlabel{cohadj:defn:parental-subcomputad}{{4.2.2}{30}{parental subcomputads of $\Adj $}{thm.4.2.2}{}}
\newlabel{cohadj:ex:parental-subcomputad}{{4.2.3}{31}{}{thm.4.2.3}{}}
\newlabel{cohadj:ex:non-parental-subcomputad}{{4.2.4}{31}{a non-example}{thm.4.2.4}{}}
\newlabel{cohadj:ex:another-parental-subcomputad}{{4.2.5}{31}{}{thm.4.2.5}{}}
\newlabel{cohadj:ntn:generic-catthree}{{4.2.6}{31}{}{thm.4.2.6}{}}
\newlabel{cohadj:eq:3[i]-square}{{4.2.7}{32}{}{equation.4.2.7}{}}
\newlabel{cohadj:def:funct-for-fillable}{{4.2.9}{32}{}{thm.4.2.9}{}}
\newlabel{cohadj:lem:ext-par-subcomp}{{4.2.10}{32}{extending parental subcomputads}{thm.4.2.10}{}}
\newlabel{cohadj:eq:parental-adj-pushout-1}{{4.2.11}{33}{extending parental subcomputads}{equation.4.2.11}{}}
\newlabel{cohadj:eq:parental-adj-pushout-2}{{4.2.12}{33}{extending parental subcomputads}{equation.4.2.12}{}}
\newlabel{cohadj:cor:ext-par-subcomp}{{4.2.13}{34}{}{thm.4.2.13}{}}
\newlabel{cohadj:eq:big-pushout}{{4.2.14}{34}{}{equation.4.2.14}{}}
\newlabel{cohadj:prop:ext-par-subcomp}{{4.2.15}{34}{}{thm.4.2.15}{}}
\newlabel{cohadj:itm:one}{{{{(i)}}}{34}{}{Item.4}{}}
\newlabel{cohadj:itm:two}{{{{(ii)}}}{34}{}{Item.5}{}}
\newlabel{cohadj:ssec:hocoh}{{4.3}{36}{Homotopy Coherent Adjunctions}{subsection.4.3}{}}
\newlabel{cohadj:obs:adjunctions-in-qcat-cats}{{4.3.2}{36}{adjunctions in a quasi-categorically enriched category}{thm.4.3.2}{}}
\newlabel{cohadj:obs:int-univ}{{4.3.4}{37}{the internal universal property of the counit}{thm.4.3.4}{}}
\newlabel{cohadj:lem:rel-lift-terminal}{{4.3.5}{38}{a relative universal property of terminal objects}{thm.4.3.5}{}}
\newlabel{cohadj:prop:rel-int-univ}{{4.3.6}{39}{the relative internal universal property of the counit}{thm.4.3.6}{}}
\newlabel{cohadj:obs:ext-quasicat-homs}{{4.3.7}{39}{}{thm.4.3.7}{}}
\newlabel{cohadj:thm:hty-coherence-exist}{{4.3.8}{40}{}{thm.4.3.8}{}}
\newlabel{cohadj:thm:hty-coherence-exist-I}{{4.3.9}{41}{homotopy coherence of adjunctions I}{thm.4.3.9}{}}
\newlabel{cohadj:thm:hty-coherence-exist-II}{{4.3.11}{42}{homotopy coherence of adjunctions II}{thm.4.3.11}{}}
\newlabel{cohadj:def:adjunction-data}{{4.3.13}{42}{}{thm.4.3.13}{}}
\newlabel{cohadj:ssec:uniqueness}{{4.4}{42}{Homotopical uniqueness of homotopy coherent adjunctions}{subsection.4.4}{}}
\newlabel{cohadj:obs:icon-defn}{{4.4.1}{43}{simplicial enrichment of simplicial categories}{thm.4.4.1}{}}
\newlabel{cohadj:lem:isofib-icon}{{4.4.2}{43}{}{thm.4.4.2}{}}
\newlabel{cohadj:lem:conservative-icon}{{4.4.4}{45}{}{thm.4.4.4}{}}
\newlabel{cohadj:lem:cohadj.space.Kan}{{4.4.6}{46}{}{thm.4.4.6}{}}
\newlabel{cohadj:prop:hty-uniqueness}{{4.4.7}{46}{}{thm.4.4.7}{}}
\newlabel{cohadj:eq:trans-lift-prob-1}{{4.4.8}{47}{Homotopical uniqueness of homotopy coherent adjunctions}{equation.4.4.8}{}}
\newlabel{cohadj:obs:epsilon-icon-expl}{{4.4.9}{47}{}{thm.4.4.9}{}}
\newlabel{cohadj:thm:hty-uniqueness-I}{{4.4.11}{48}{}{thm.4.4.11}{}}
\newlabel{cohadj:prop:counits-proj-trivial}{{4.4.12}{48}{}{thm.4.4.12}{}}
\newlabel{cohadj:lem:fib-contract-fibres}{{4.4.13}{48}{}{thm.4.4.13}{}}
\newlabel{cohadj:prop:leftadjs-proj-trivial}{{4.4.17}{49}{}{thm.4.4.17}{}}
\newlabel{cohadj:thm:hty-uniqueness-II}{{4.4.18}{50}{}{thm.4.4.18}{}}
\newlabel{cohadj:sec:weighted}{{5}{50}{Weighted limits in \protect \texorpdfstring {$\protect \qCat _\infty $}{qCat}}{section.5}{}}
\newlabel{cohadj:subsec:weighted}{{5.1}{50}{Weighted limits and colimits}{subsection.5.1}{}}
\newlabel{cohadj:defn:cotensors}{{5.1.1}{50}{cotensors}{thm.5.1.1}{}}
\newlabel{cohadj:eq:wlimformula}{{5.1.3}{51}{weighted limits}{equation.5.1.3}{}}
\newlabel{cohadj:eq:wlim-def-prop}{{5.1.4}{51}{weighted limits}{equation.5.1.4}{}}
\newlabel{cohadj:ex:rep-weights}{{5.1.5}{51}{representable weights}{thm.5.1.5}{}}
\newlabel{cohadj:eq:yoneda}{{5.1.6}{51}{representable weights}{equation.5.1.6}{}}
\newlabel{cohadj:obs:weighted-cocontinuity}{{5.1.7}{51}{}{thm.5.1.7}{}}
\newlabel{cohadj:ex:diagrams-weighted-limit}{{5.1.8}{52}{diagrams}{thm.5.1.8}{}}
\newlabel{cohadj:ex:commaweightedlimit}{{5.1.9}{52}{comma quasi-categories}{thm.5.1.9}{}}
\newlabel{cohadj:ex:homotopyweighted}{{5.1.10}{52}{homotopy limits as weighted limits}{thm.5.1.10}{}}
\newlabel{cohadj:lem:lanweights}{{5.1.11}{52}{weighted limits and Kan extensions}{thm.5.1.11}{}}
\newlabel{cohadj:subsec:weighted-qcat}{{5.2}{53}{Weighted limits in the quasi-categorical context}{subsection.5.2}{}}
\newlabel{cohadj:defn:proj-cof}{{5.2.1}{53}{projective cofibrations}{thm.5.2.1}{}}
\newlabel{cohadj:prop:projwlims2}{{5.2.2}{53}{}{thm.5.2.2}{}}
\newlabel{cohadj:eq:layer-in-tower}{{5.2.3}{53}{Weighted limits in the quasi-categorical context}{equation.5.2.3}{}}
\newlabel{cohadj:prop:projwlims}{{5.2.4}{54}{}{thm.5.2.4}{}}
\newlabel{cohadj:prop:proj-wlim-homotopical}{{5.2.6}{54}{}{thm.5.2.6}{}}
\newlabel{cohadj:rmk:projwlims}{{5.2.7}{54}{}{thm.5.2.7}{}}
\newlabel{cohadj:rmk:categories-special-case}{{5.2.9}{55}{2-categorical weighted limits and quasi-categorical weighted limits}{thm.5.2.9}{}}
\newlabel{cohadj:subsec:collage}{{5.3}{55}{The collage construction}{subsection.5.3}{}}
\newlabel{cohadj:obs:coll-right-adj}{{5.3.2}{55}{a right adjoint to the collage construction}{thm.5.3.2}{}}
\newlabel{cohadj:prop:projcofchar2}{{5.3.3}{56}{}{thm.5.3.3}{}}
\newlabel{cohadj:eq:pushout-transform}{{5.3.4}{56}{The collage construction}{equation.5.3.4}{}}
\newlabel{cohadj:prop:projcofchar}{{5.3.5}{58}{}{thm.5.3.5}{}}
\newlabel{cohadj:sec:formal}{{6}{58}{The formal theory of homotopy coherent monads}{section.6}{}}
\newlabel{cohadj:ssec:formalmonads}{{6.1}{59}{Weighted limits for the formal theory of monads}{subsection.6.1}{}}
\newlabel{cohadj:obs:mnd-weights}{{6.1.3}{59}{weights on $\Mnd $}{thm.6.1.3}{}}
\newlabel{cohadj:defn:W+}{{6.1.4}{60}{monad resolutions}{thm.6.1.4}{}}
\newlabel{cohadj:eq:resolution2}{{6.1.5}{60}{monad resolutions}{equation.6.1.5}{}}
\newlabel{cohadj:defn:W-}{{6.1.6}{61}{}{thm.6.1.6}{}}
\newlabel{cohadj:defn:EMobject}{{6.1.7}{61}{quasi-category of algebras}{thm.6.1.7}{}}
\newlabel{cohadj:lem:projcof1}{{6.1.8}{61}{}{thm.6.1.8}{}}
\newlabel{cohadj:rmk:expl-htycoh-alg}{{6.1.10}{61}{}{thm.6.1.10}{}}
\newlabel{cohadj:eq:expl-htycoh-alg-cond}{{6.1.11}{61}{}{equation.6.1.11}{}}
\newlabel{cohadj:eq:algebraresolution}{{6.1.12}{62}{}{equation.6.1.12}{}}
\newlabel{cohadj:ex:monadicadj}{{6.1.14}{63}{monadic adjunction}{thm.6.1.14}{}}
\newlabel{cohadj:eq:monadicadjunctionweights}{{6.1.15}{63}{monadic adjunction}{equation.6.1.15}{}}
\newlabel{cohadj:ssec:monadicequivs}{{6.2}{63}{Conservativity of the monadic forgetful functor}{subsection.6.2}{}}
\newlabel{cohadj:prop:conservative-char}{{6.2.2}{63}{}{thm.6.2.2}{}}
\newlabel{cohadj:cor:uTconservative}{{6.2.3}{64}{}{thm.6.2.3}{}}
\newlabel{cohadj:obs:understanding-monadic-forgetful-functor}{{6.2.4}{64}{}{thm.6.2.4}{}}
\newlabel{cohadj:rmk:u-map}{{6.2.5}{65}{}{thm.6.2.5}{}}
\newlabel{cohadj:ssec:algcolims}{{6.3}{65}{Colimit representation of algebras}{subsection.6.3}{}}
\newlabel{cohadj:eq:splitcoeq}{{6.3.1}{65}{Colimit representation of algebras}{equation.6.3.1}{}}
\newlabel{cohadj:eq:splitgeorealizations}{{6.3.2}{66}{}{equation.6.3.2}{}}
\newlabel{cohadj:rec:splitgeorealizations}{{6.3.3}{66}{constructing the triangle in theorem \refI {thm:splitgeorealizations}}{thm.6.3.3}{}}
\newlabel{cohadj:eq:canonical-2-cell}{{6.3.4}{66}{constructing the triangle in theorem \refI {thm:splitgeorealizations}}{equation.6.3.4}{}}
\newlabel{cohadj:defn:u-split-aug}{{6.3.5}{66}{$u$-split augmented simplicial objects}{thm.6.3.5}{}}
\newlabel{cohadj:eq:u-split-pullback}{{6.3.6}{67}{$u$-split augmented simplicial objects}{equation.6.3.6}{}}
\newlabel{cohadj:prop:uTcreates}{{6.3.7}{67}{}{thm.6.3.7}{}}
\newlabel{cohadj:thm:uTcreates}{{6.3.8}{67}{}{thm.6.3.8}{}}
\newlabel{cohadj:obs:uTcreates}{{6.3.9}{67}{}{thm.6.3.9}{}}
\newlabel{cohadj:eq:colim.diag.1}{{6.3.10}{67}{}{equation.6.3.10}{}}
\newlabel{cohadj:eq:colimit-diag-ut-split}{{6.3.11}{69}{Colimit representation of algebras}{equation.6.3.11}{}}
\newlabel{cohadj:eq:colimit-diag-ut-split-under}{{6.3.12}{69}{Colimit representation of algebras}{equation.6.3.12}{}}
\newlabel{cohadj:eq:Wscdefn}{{6.3.14}{70}{a direct description of $S(u^t)$}{equation.6.3.14}{}}
\newlabel{cohadj:obs:canonical.alg.pres}{{6.3.15}{70}{}{thm.6.3.15}{}}
\newlabel{cohadj:eq:usplitpushout}{{6.3.16}{70}{}{equation.6.3.16}{}}
\newlabel{cohadj:thm:colim-rep-algebras}{{6.3.17}{71}{canonical colimit representation of algebras}{thm.6.3.17}{}}
\newlabel{cohadj:eq:canonical-colimits}{{6.3.18}{71}{canonical colimit representation of algebras}{equation.6.3.18}{}}
\newlabel{cohadj:eq:downstairs-colimit}{{6.3.19}{71}{canonical colimit representation of algebras}{equation.6.3.19}{}}
\newlabel{cohadj:sec:monadicity}{{7}{72}{Monadicity}{section.7}{}}
\newlabel{cohadj:ssec:comparison}{{7.1}{72}{Comparison with the monadic adjunction}{subsection.7.1}{}}
\newlabel{cohadj:lem:actionquotient}{{7.1.3}{73}{}{thm.7.1.3}{}}
\newlabel{cohadj:eq:actionquotient}{{7.1.4}{73}{}{equation.7.1.4}{}}
\newlabel{cohadj:obs:lanW-}{{7.1.5}{73}{}{thm.7.1.5}{}}
\newlabel{cohadj:eq:monadicadjunctionweight}{{7.1.6}{74}{}{equation.7.1.6}{}}
\newlabel{cohadj:eq:monadiccomparison}{{7.1.10}{74}{comparison with the monadic adjunction}{equation.7.1.10}{}}
\newlabel{cohadj:ssec:monadicity}{{7.2}{74}{The monadicity theorem}{subsection.7.2}{}}
\newlabel{cohadj:eq:usplit}{{7.2.2}{75}{$u$-split simplicial objects}{equation.7.2.2}{}}
\newlabel{cohadj:eq:usplithyp}{{7.2.3}{75}{$u$-split simplicial objects}{equation.7.2.3}{}}
\newlabel{cohadj:thm:monadiccomparisonadj}{{7.2.4}{75}{monadicity I}{thm.7.2.4}{}}
\newlabel{cohadj:eq:Ldefn}{{7.2.5}{75}{The monadicity theorem}{equation.7.2.5}{}}
\newlabel{cohadj:eq:Labs-lifting}{{7.2.6}{76}{The monadicity theorem}{equation.7.2.6}{}}
\newlabel{cohadj:thm:monadicity}{{7.2.7}{77}{monadicity II}{thm.7.2.7}{}}
\newlabel{cohadj:tocindent-1}{0pt}
\newlabel{cohadj:tocindent0}{15.01021pt}
\newlabel{cohadj:tocindent1}{20.88377pt}
\newlabel{cohadj:tocindent2}{34.49158pt}
\newlabel{cohadj:tocindent3}{0pt}

\setcounter{tocdepth}{2}

\subjclass[2010]{%
  Primary  18G55, 55U35, 55U40; %
  Secondary 18A05, 18D20, 18G30, 55U10%18D35, 18F99
}

\begin{document}
  
  \ifpdf
  \DeclareGraphicsExtensions{.pdf, .jpg, .tif}
  \else
  \DeclareGraphicsExtensions{.eps, .jpg}
  \fi
  

\begin{abstract}
In this paper we re-develop the foundations of the category theory of quasi-categories (also called $\infty$-categories) using 2-category theory. We show that Joyal's strict 2-category of quasi-categories admits certain weak 2-limits, among them weak comma objects. We use these comma quasi-categories to encode universal properties relevant to limits, colimits, and adjunctions and prove the expected theorems relating these notions. These universal properties have an alternate form as absolute lifting diagrams in the 2-category, which we show are determined pointwise by the existence of certain initial or terminal vertices, allowing for the easy production of examples.

All the quasi-categorical notions introduced here are equivalent to the established ones but our proofs are independent and more ``formal''. In particular, these results generalise immediately to model categories enriched over quasi-categories.
\end{abstract}

\maketitle
\tableofcontents


\chapter*{Introduction}


This book was born of research in category theory, brought to life by the
ongoing vigorous debate on how to quantify biological diversity, given
strength by information theory, and fed by the ancient field of functional
equations.  It applies the power of the axiomatic method to a biological
problem of pressing concern, but it also presents new advances in `pure'
mathematics that stand in their own right, independently of any
application.

The starting point is the connection between diversity and entropy.  We
will discover:
% 
\begin{itemize}
\item
how Shannon entropy, originally defined for communications engineering, can
also be understood through biological diversity (Chapter~\ref{ch:shannon});
% (Section~\ref{sec:ent-div});

\item
how deformations of Shannon entropy express
a spectrum of viewpoints on the meaning of biodiversity
(Chapter~\ref{ch:def});

\item
how these deformations \emph{provably} provide the only reasonable
abundance-based measures of diversity (Chapter~\ref{ch:value});
% (Section~\ref{sec:total-hill});

\item
how to derive such results from characterization theorems for the
power means, of which we prove several, some new (Chapters~\ref{ch:mns}
and~\ref{ch:prob}). 
% Section~\ref{sec:mult-means}). 
\end{itemize}
% 
Complementing the classical techniques of these proofs is a large-scale
categorical programme, which has produced both new mathematics and new
measures of diversity now used in scientific applications.  For example, we
will find:
% 
\begin{itemize}
\item
that many invariants of size from across the breadth of mathematics
(including cardinality, volume, surface area, fractional dimension, and
both topological and algebraic notions of Euler characteristic) arise
from one single invariant, defined in the wide generality of enriched
categories (Chapter~\ref{ch:sim});
% (Sections~\ref{sec:mag} and~\ref{sec:mag-geom});

\item
a way of measuring diversity that reflects not only the varying abundances
of species (as is traditional), but also the varying similarilities between
them, or, more generally, any notion of the values of the species
(Chapters~\ref{ch:sim} and~\ref{ch:value});

\item
that these diversity measures belong to the extended family of measures
of size (Chapter~\ref{ch:sim});
% (Sections~\ref{sec:mag} and~\ref{sec:mag-geom});

\item
a `best of all possible worlds': an abundance distribution on any given set
of species that maximizes diversity from an infinite number of viewpoints
simultaneously (Chapter~\ref{ch:sim});
% (Section~\ref{sec:max});

\item
an extension of Shannon entropy from its classical context of finite sets
to distributions on a metric space or a graph (Chapter~\ref{ch:sim}),
obtained by translating the similarity-sensitive diversity measures into
the language of entropy.
\end{itemize}
% 
Shannon entropy is a fundamental concept of information theory, but
information theory contains many riches besides.  We will mine them,
discovering:
% 
\begin{itemize}
\item
how the concept of relative entropy not only touches subjects from
Bayesian inference to coding theory to Riemannian geometry, but also
provides a way of quantifying local diversity within a larger context
(Chapter~\ref{ch:rel});

\item
quantitative methods for identifying particularly unusual or atypical parts of
an ecological community (Chapter~\ref{ch:mm}, drawing on work of Reeve
et al.~\cite{HPD}).
\end{itemize}
% 
The main narrative thread is modest in its mathematical prerequisites.  But
we also take advantage of some more specialized bodies of knowledge (large
deviation theory, the theory of operads, and the theory of finite fields),
establishing:
% 
\begin{itemize}
\item
how probability theory can be used to solve functional equations
(Chapter~\ref{ch:prob}, following work of Aubrun and Nechita~\cite{AuNe});  

\item
a streamlined characterization of information loss, as a natural consequence
of categorical and operadic thinking (Chapters~\ref{ch:loss} and~\ref{ch:cat});

\item
that the concept of entropy is (provably) inescapable even in the
pure-mathematical heartlands of category theory, algebra and topology,
quite separately from its importance in scientific
applications (Chapter~\ref{ch:cat});

\item
the right definition of entropy for probability distributions whose
`probabilities' are elements of the ring $\Zp$ of integers modulo a
prime~$p$ (Chapter~\ref{ch:p}, drawing on work of
Kontsevich~\cite{KontOHL}). 
\end{itemize}
% 
The question of how to quantify diversity is far more mathematically
profound than is generally appreciated.  This book makes the case that the
theory of diversity measurement is fertile soil for new
mathematics, just as much as the neighbouring but far more thoroughly
worked field of information theory.

\introbreak

What \emph{is} the problem of quantifying diversity?%
% 
\index{diversity measure}%
\index{diversity}
% 
Briefly, it is to take a biological community and extract from it a
numerical measure of its `diversity' (whatever that should mean).
% 
This task is certainly beset with practical problems: for instance, field
ecologists recording woodland animals will probably observe the noisy, the
brightly-coloured and the gregarious more frequently than the quiet, the
camouflaged and the shy.  There are also statistical difficulties: if a
survey of one community finds $10$ different species in a sample of $50$
individuals, and a survey of another finds $18$ different species in a
sample of $100$, which is more diverse?

However, we will not be concerned with either the practical or the
statistical difficulties.  Instead, we will focus on a fundamental
conceptual problem: in an ideal world where we have complete, perfect data,
how can we quantify diversity in a meaningful and logical way?

In both the news media and the scientific literature, the most common
meaning given to the word `diversity' (or `biodiversity') is simply the
number of species present.  Certainly this is an important quantity.
However, it is not always very informative.  For instance, the number of
species of great ape\index{apes} on the planet is~$8$
(Example~\ref{eg:hill-apes}), but $99.99$\% of all great apes belong to
just one species: us.  In terms of global ecology, it is arguably more accurate
to say that there is effectively only one species of great ape.

An example illustrates the spectrum of possible interpretations of the
concept of diversity.  Consider two bird\index{birds} communities:
\[
\lbl{p:intro-birds}
\lengths
\begin{picture}(120,55)(0,2)
% 
% In what follows, each bird is 8\unitlength high, and \unitlength=0.9mm,
% so it's 7.2mm high. Now each bird is 214x184 pixels, so that's 184/7.2 =
% 25.555... pixels/mm, i.e. 25.555... x 25.4 = 649.111... pixels/inch. That
% exceeds the 600 dpi requirement.
% 
\cell{10}{8}{bl}{\includegraphics[height=8\unitlength]{birdF_std.png}}
\cell{10}{16.2}{bl}{\includegraphics[height=8\unitlength]{birdF_std.png}}
\cell{10}{24.4}{bl}{\includegraphics[height=8\unitlength]{birdF_std.png}}
\cell{10}{32.6}{bl}{\includegraphics[height=8\unitlength]{birdF_std.png}}
\cell{10}{40.8}{bl}{\includegraphics[height=8\unitlength]{birdF_std.png}}
\cell{10}{49}{bl}{\includegraphics[height=8\unitlength]{birdF_std.png}}
\cell{20.5}{8}{bl}{\includegraphics[height=8\unitlength]{birdA_std.png}}
\cell{31}{8}{bl}{\includegraphics[height=8\unitlength]{birdG_std.png}}
\cell{41.5}{8}{bl}{\includegraphics[height=8\unitlength]{birdB_std.png}}
\cell{30}{2}{b}{A}
\cell{70}{8}{bl}{\includegraphics[height=8\unitlength]{birdF_std.png}}
\cell{70}{16.2}{bl}{\includegraphics[height=8\unitlength]{birdF_std.png}}
\cell{70}{24.4}{bl}{\includegraphics[height=8\unitlength]{birdF_std.png}}
\cell{80.5}{8}{bl}{\includegraphics[height=8\unitlength]{birdA_std.png}}
\cell{80.5}{16.2}{bl}{\includegraphics[height=8\unitlength]{birdA_std.png}}
\cell{80.5}{24.4}{bl}{\includegraphics[height=8\unitlength]{birdA_std.png}}
\cell{91}{8}{bl}{\includegraphics[height=8\unitlength]{birdG_std.png}}
\cell{91}{16.2}{bl}{\includegraphics[height=8\unitlength]{birdG_std.png}}
\cell{91}{24.4}{bl}{\includegraphics[height=8\unitlength]{birdG_std.png}}
\cell{85}{2}{b}{B}
\end{picture}
\]
In community~A, there are four species, but the majority of individuals
belong to a single dominant species.  Community~B contains the first three
species in equal abundance, but the fourth is absent.  Which community,
A~or~B, is more diverse?

One viewpoint%
%
\index{viewpoint!diversity@on diversity} 
% 
is that the presence of \emph{species} is what matters.  Rare
species count for as much as common ones: every species is precious.  From
this viewpoint, community~A is more diverse, simply because more species
are present.  The abundances of species are irrelevant; presence or absence
is all that matters.

But there is an opposing viewpoint that prioritizes the balance of
\emph{communities}.  Common species are important; they are the ones that
exert the most influence on the community.  Community~A has a single very
common species, which has largely outcompeted the others, whereas
community~B has three common species, evenly balanced.  From this
viewpoint, community~B is more diverse.

These two viewpoints are the two ends of a continuum.  More precisely,
there is a continuous one-parameter family $(D_q)_{q \in [0, \infty]}$ of
diversity measures encoding this spectrum of viewpoints.  Low values of $q$
attach high importance to rare species; for example, $D_0$ measures
community~A as more diverse than community~B.  When $q$ is high, $D_q$ is
most strongly influenced by the balance of more common species; thus,
$D_\infty$ judges~B to be more diverse.  No single viewpoint is right or
wrong.  Different scientists adopt different viewpoints (that is, different
values of $q$) for different purposes, as the literature amply attests
(Examples~\ref{egs:hill}).

Long ago, it was realized that the concept of diversity is closely related
to the concept of entropy.  Entropy appears in dozens of guises across
dozens of branches of science, of which thermodynamics is probably the most
famous.  (The introduction to Chapter~\ref{ch:shannon} gives a long but
highly incomplete list.)  The most simple incarnation is Shannon entropy,
which is a real number associated with any probability distribution on a
finite set.  It is, in fact, the logarithm of the diversity measure $D_1$.
Most often, Shannon entropy is explained and understood through the theory
of coding; indeed, we provide such an explanation here.  But the diversity
interpretation provides a new perspective.

For example, the diversity measures $D_q$, known in ecology as the
Hill%
%
\index{Hill number}
%
numbers, are the exponentials of what information theorists know as the
R\'enyi%
%
\index{Renyi entropy@R\'enyi entropy} 
% 
entropies.  From the very beginning of information theory, an
important role has been played by characterization theorems: results
stating that any measure (of information, say) satisfying a list of
desirable properties must be of a particular form (a scalar
multiple of Shannon entropy, say).  But what counts as a desirable
property depends on one's perspective.  We will prove that the Hill numbers
$D_q$ are, in a precise sense, the only measures of diversity with
certain natural properties (Theorem~\ref{thm:total-hill}).  This theorem
translates into a new characterization of the R\'enyi entropies, but it is
not one that necessarily would have been thought of from a purely
information-theoretic perspective.  

However, something is missing.  In the real world, diversity is understood
as involving not only the number and abundances of the species, but also
how \emph{different} they are.  (For example, this affects
conservation\index{conservation} policy; see the OECD quotation on
p.~\pageref{p:oecd-quote}.)  We describe the remedy in
Chapter~\ref{ch:sim}, defining a family of diversity measures that take
account of the varying similarity%
%
\index{similarity!species@of species} 
% 
between species, while still incorporating the
spectrum of viewpoints discussed above.  This definition unifies into one
family a large number of the diversity measures proposed and used in the
ecological and genetics literature.

This family of diversity measures first appeared in a paper in
\emph{Ecology}~\cite{MDISS}, but it can also be understood and motivated
from a purely mathematical perspective.  The classical R\'enyi entropies
are a family of real numbers assigned to any probability distribution on a
finite \emph{set}.  By factoring in the differences or distances
between points (species), we extend this to a family of real numbers
assigned to any probability distribution on a finite
\emph{metric\index{metric!space} space}.
In the extreme case where $d(x, y) = \infty$ for all distinct points $x$
and $y$, we recover the R\'enyi entropies.  In this way, the
similarity-sensitive diversity measures extend the definition of R\'enyi
entropy from sets to metric spaces.

Different values of the viewpoint parameter $q \in [0, \infty]$ produce
different judgements on which of two distributions is the more diverse.
But it turns out that for any metric space (or in biological terms, any
set of species), there is a single distribution that maximizes%
%
\lbl{p:max-intro}
% 
diversity from all viewpoints simultaneously.  For a generic finite metric
space, this maximizing distribution is unique.  Thus, almost every finite
metric space carries a canonical probability distribution (not usually
uniform).  The maximum%
%
\index{maximum diversity} 
% 
diversity itself is also independent of $q$, and is therefore a numerical
invariant of metric spaces.  This invariant has geometric significance in
its own right (Section~\ref{sec:mag-geom}).

We go further.  One might wish to evaluate an ecological community in a way
that takes into account some notion of the values\index{value} of the
species (such as 
phylogenetic distinctiveness).  Again, there is a sensible family of
measures that does this job, extending not only the similarity-sensitive
diversity measures just described, but also further measures already
existing in the ecological literature.  The word `sensible' can be made
precise: as soon as we subject an abstract measure of the value of a
community to some basic logical requirements, it is forced to belong to a
certain one-parameter family $(\sigma_q)$ (Theorem~\ref{thm:val-char}),
which are essentially the R\'enyi \emph{relative} entropies.

Information theory also helps us to analyse the diversity of
metacommunities,\index{metacommunity} that is, ecological communities made
up of a number of smaller communities such as geographical regions.  The
established notions of relative entropy, conditional entropy and mutual
information provide meaningful measures of the structure of a metacommunity
(Chapter~\ref{ch:mm}).  But we will do more than simply translate
information theory into ecological language.  For example, the new
characterization of the R\'enyi entropies mentioned above is a byproduct of
the characterization theorem for measures of ecological value.  In this
way, the theory of diversity gives back to information theory.

\introbreak

The scientific importance of biological diversity goes far beyond the
obvious setting of conservation\index{conservation} of animals and plants.
Certainly such conservation efforts are important, and the need for
meaningful measures of diversity is well-appreciated in that context.  For
example, Vane-Wright%
%
\index{Vane-Wright, Richard} 
% 
et al.~\cite{VWHW} wrote thirty years ago of the `agony of choice'
in conservation of flora and fauna, and emphasized how crucial it is to use
the right diversity measures.

But most life is microscopic.  Nee~\cite{NeeMTM}%
%
\index{Nee, Sean} 
% 
argued in 2004 that
% 
\begin{quote}
{}[w]e are still at the very beginning of a golden age of biodiversity
discovery, driven largely by the advances in molecular biology and a new
open-mindedness about where life might be found,%
%
\index{microbial systems}
\end{quote}
% 
and that
% 
\begin{quote}
all of the marvels in biodiversity's new bestiary are invisible.
\end{quote}
% 
Even excluding exotic new discoveries of microscopic life, two recent lines
of research illustrate important uses of diversity measures at the
microbial level.

First, the extensive use of antimicrobial drugs on animals
unfortunate enough to be born into the modern meat industry is commonly
held to be a cause of antimicrobial resistance in pathogens affecting
humans.  However, a 2012 study of Mather%
%
\index{Mather, Alison} 
% 
et al.~\cite{MMMR} suggests that the causality may be more complex.  By
analysing the diversity of antimicrobial%
%
\index{antimicrobial resistance}%
\index{microbial systems}
% 
resistance in \emph{Salmonella} taken from animal populations on the one
hand, and from human populations on the other, the authors concluded that
the animal population is `unlikely to be the major source of resistance'
for humans, and that `current policy emphasis on restricting antimicrobial
use in domestic animals may be overly simplistic'.  The diversity measures
used in this analysis were the Hill numbers $D_q$ mentioned above and
central to this book.

Second, the increasing problem of obesity in humans has prompted research
into causes and treatments, and there is evidence of a negative
correlation between obesity and diversity of the gut%
%
\index{gut microbiome} 
% 
microbiome (Turnbaugh et al.~\cite{THYC,TQFM}).  Almost all traditional
measures of diversity rely on a division of organisms into species or other
taxonomic groups, but in this case, only a fraction of the microbial
species concerned have been isolated and classified taxonomically.
Researchers in this field therefore use DNA sequence data, applying
sophisticated but somewhat arbitrary clustering algorithms to create
artificial species-like groups (`operational taxonomic units').  On the
other hand, the similarity-sensitive diversity measures mentioned above and
introduced in Chapter~\ref{ch:sim} can be applied directly to the sequence
data, bypassing the clustering step and producing a measure of genetic
diversity.  A test case was carried out in Leinster and
Cobbold~\cite{MDISS} (Example~4), with results that supported the
conclusions of Turnbaugh et al.

Despite the wide variety of uses of diversity measures in biology, none of
the mathematics presented in this text is intrinsically biological.%
% 
\index{diversity measure!applications of}
% 
Indeed, the mathematics of diversity was being developed as early as 1912
by the economist\index{economics} Corrado
Gini~\cite{Gini}%
%
\index{Gini, Corrado} 
% 
(best known for the Gini
coefficient of disparity of wealth), and by the statistician Udny
Yule%
%
\index{Yule, G. Udny} 
% 
in the 1940s for the analysis of lexical%
%
\index{lexical diversity}%
\index{diversity!lexical}
% 
diversity in literature~\cite{Yule}.  Some of the diversity measures most
common in ecology have recently been used to analyse the ethnic and
sociological diversity of judges (Barton and Moran~\cite{BaMo}), and the
similarity-sensitive diversity measures that are the subject of
Chapter~\ref{ch:sim} have been used not only in multiple ecological
contexts (as listed after Example~\ref{eg:devries}), but also in
non-biological applications such as computer%
%
\index{computer network security} 
% 
network security (Wang et al.~\cite{WZJS}).% 
% 
\index{diversity measure!applications of}

In mathematical terms, simple diversity measures such as the Hill numbers
are invariants of a probability distribution on a finite set.  The
similarity-sensitive diversity measures are defined for any probability
distribution on a finite set with an assigned degree of similarity between
each pair of points.  (This includes any finite metric space or graph.)
The value measures are defined for any finite set equipped with a
probability distribution and an assignment of a nonnegative value to each
element.  The metacommunity measures are defined for any probability
distribution on the cartesian product of a pair of finite sets.  Much of
this text is written using ecological terminology, but the mathematics is
entirely general.\lbl{p:general}

\introbreak

This work grew out of a general category-theoretic%
%
\index{category theory} 
%
study of size.\index{size} In many parts of mathematics, there is a
canonical notion of the size of the objects of study: sets have
cardinality, vector spaces have dimension, subsets of Euclidean space have
volume, topological spaces have Euler%
%
\index{Euler characteristic}
%
characteristic, and so on.  Typically, such measures of size satisfy
analogues of the elementary inclusion-exclusion and multiplicativity
formulas for counting finite sets:
% 
\begin{align*}
\mg{X \cup Y}   &= \mg{X} + \mg{Y} - \mg{X \cap Y},     \\
\mg{X \times Y} &= \mg{X} \cdot \mg{Y}.
\end{align*}
% 
(The interpretation of Euler characteristic as the topological analogue of
cardinality is not as well known as it should be; this is an insight of
Schanuel%
%
\index{Schanuel, Stephen} 
% 
on which we elaborate in Section~\ref{sec:mag}.)  From a
categorical perspective, it is natural to seek a single invariant unifying
all of these measures of size.

Some unification is achieved by defining a notion of size for categories
themselves, called \emph{magnitude}\index{magnitude} or Euler
characteristic.  (Finiteness hypotheses are required, but will not be
mentioned in this overview.)  This definition already brings together
several established invariants of size~\cite{ECC}: cardinality of
sets, and the various notions of Euler characteristic for partially ordered
sets, topological spaces, and even orbifolds (whose Euler characteristics
are in general not integers).  The theory of magnitude of categories is
closely related to the theory of M\"obius--Rota inversion for partially
ordered sets~\cite{RotaFCT,NMI}.

But the decisive, unifying step is the generalization of the definition of
magnitude from categories to the wider class of \emph{enriched}%
%
\index{enriched category} 
% 
categories~\cite{MMS}, which includes not only categories
themselves, but also metric spaces, graphs, and the additive categories
that are a staple of homological algebra.

The definition of the magnitude of an enriched category unifies still more
established invariants of size.  For example, in the representation theory
of associative algebras, one frequently considers the indecomposable
projective modules, which form an additive category.  The magnitude of that
additive category turns out to be the Euler form of a certain canonical
module, defined as an alternating sum of dimensions of $\Ext$ groups
(equation~\eqref{eq:ip}).  Magnitude for enriched categories can also be
realized as the Euler characteristic of a certain Hochschild-like homology%
%
\index{magnitude!homology}
% 
theory of enriched categories, in the same sense that the Jones polynomial
for knots is the Euler characteristic of Khovanov homology~\cite{Khov}.
This was established in recent work led by Shulman~\cite{MHECMS}, building
on the case of magnitude homology for graphs previously developed by
Hepworth and Willerton~\cite{HeWi}. 

Since any metric\index{metric!space} space can be regarded as an enriched
category, the general definition of the magnitude of an enriched category
gives, in particular, a definition of the magnitude $\mg{X} \in \R$ of a
metric space $X$.  Unlike the other special cases just mentioned, this
invariant is essentially new.

Recent, increasingly sophisticated, work in analysis has connected
magnitude with classical invariants of geometric measure.  For example, for
a compact subset $X \sub \R^n$ satisfying certain regularity conditions, if
one is given the magnitude of all of the rescalings $tX$ of $X$ (for $t >
0$), then one can recover:
% 
\begin{itemize}
\item 
the Minkowski\index{dimension} dimension of $X$ (one of the principal
notions of fractional dimension), a result proved by Meckes using results
in potential theory (Theorem~\ref{thm:mink});

\item
the volume\index{volume} of $X$, a result proved by Barcel\'o and Carbery
using PDE methods (Theorem~\ref{thm:bc});

\item
the surface%
%
\index{surface area} 
%
area of $X$, a result proved by Gimperlein and Goffeng using
global analysis (or more specifically, tools for computing heat trace
asymptotics; Theorem~\ref{thm:gg}).   
\end{itemize}
% 
Gimperlein and Goffeng also proved an asymptotic
inclusion-exclusion%
%
\index{inclusion-exclusion principle}
%
principle:
\[
\mg{t(X \cup Y)} + \mg{t(X \cap Y)} - \mg{tX} - \mg{tY}
\to 0
\]
as $t \to \infty$, for sufficiently regular $X, Y \sub \R^n$
(Section~\ref{sec:mag-geom}).  This is another manifestation of the
cardinality-like nature of magnitude.

We have seen that every finite metric space $X$ has an unambiguous maximum
diversity $\Dmax{X} \in \R$, defined in terms of the similarity-sensitive
diversity measures (p.~\pageref{p:max-intro}).  We have also seen that $X$
has a magnitude $\mg{X} \in \R$.  These two real numbers are not in general
equal (ultimately because probabilities or species abundances are forbidden
to be negative),%
%
\index{negative!probability}
%
but they are closely related.  Indeed, $\Dmax{X}$ is always equal to the
magnitude of some \emph{subspace} of $X$, and in important families of
cases is equal to the magnitude of $X$ itself.  So, magnitude is closely
related to maximum diversity.  Indeed, this relationship was exploited by
Meckes%
%
\index{Meckes, Mark} 
% 
to prove the result on Minkowski dimension.

There is a historical surprise.  Although this author arrived at the
definition of the magnitude of a metric space by the route of enriched
category theory, it had already arisen in earlier work on the
quantification of biodiversity.  In 1994, the environmental scientists
Andrew Solow%
%
\index{Solow, Andrew} 
%
and Stephen\label{p:sp-mag} Polasky%
%
\index{Polasky, Stephen} 
%
carried out a probabilistic analysis of the benefits of high biodiversity
(\cite{SoPo}, Section~4), and isolated a particular quantity that they
called the `effective%
%
\index{effective number!species@of species}
% 
number of species'.  They did not
investigate it mathematically, merely remarking mildly that it `has
some appealing properties'.  It is exactly our magnitude.

\introbreak

Ecologists began to propose quantitative definitions of biological
diversity in the mid-twentieth century~\cite{SimpMD,WhitVSM}, setting in
motion more than sixty years of heated debate, with dozens of further proposed
diversity measures, hundreds of scholarly papers, at least one book devoted
to the subject~\cite{Magu}, and consequently, for some, despair (expressed
as early as 1971 in a famously-titled paper of Hurlbert~\cite{Hurl}).%
%
\index{Hurlbert, Stuart}  
% 
Meanwhile, parallel debates were taking place in genetics and other
disciplines.

The connections between diversity measurement on the one hand, and
information theory and category theory on the other, are fruitful for both
mathematics and biology.  But any measure of biological diversity must be
justifiable in purely biological terms, rather than by borrowing authority
from information theory, category theory, or any other field.  The
ecologist E.~C.~Pielou%
% 
\index{Pielou, Evelyn Chrystalla} 
% 
warned against attaching ecological significance to diversity measures for
anything other than ecological reasons:
% 
\begin{quote}
It should not be (but it is) necessary to emphasize that the object of
calculating indices of diversity is to solve, not to create, problems.  The
indices are merely numbers, useful in some circumstances but not in all.
[\ldots] Indices should be calculated for the light (not the shadow) they
cast on genuine ecological problems.
\end{quote}
% 
(\cite{PielME}, p.~293).

In a series of incisive papers beginning in 2006, the
conservationist and botanist Lou Jost%
%
\index{Jost, Lou}
%
insisted that whatever diversity measures one uses, they must exhibit
\emph{logical behaviour}%
%
\index{diversity measure!logical behaviour of} 
%
\cite{JostED,JostPDI,JostGST,JostMBD}.  For
example, Shannon entropy is commonly used as a diversity measure by
practising ecologists, and it does behave logically if one is only using it
to ask whether one community is more or less diverse than another.  But as Jost
observed, any attempt to reason about percentage changes in diversity using
Shannon entropy runs into logical absurdities: Examples~\ref{eg:plague}
and~\ref{eg:oil} describe the plague that exterminates $90\%$ of species
but only causes a $17\%$ drop in `diversity', and the oil drilling that
simultaneously destroys \emph{and} preserves $83\%$ of the `diversity' of
an ecosystem.  It is, in fact, the \emph{exponential} of Shannon entropy
that should be used for this purpose.

In this sense, origin stories are irrelevant.  Inventing new diversity
measures is easy, and it is nearly as easy to tell a story of how a new
measure fits with some intuitive idea of diversity, or to justify it in
terms of its importance in some related discipline.  But if a measure does
not pass basic logical tests (as in Section~\ref{sec:prop-hill}), it is
useless or worse.

Jost noted that all of the Hill numbers $D_q$ do behave logically.  Again,
we go further: Theorem~\ref{thm:total-hill} states that the Hill numbers
are in fact the \emph{only} measures of diversity satisfying certain
logically fundamental properties.  (At least, this is so for the simple
model of a community in terms of species abundances only.)  This is the
ideal of the axiomatic approach: to prove results stating that if one
wishes to have a measure with such-and-such properties, then it can only be
one of \emph{these} measures.

Mathematically, such results belong to the field of functional equations.  We
review a small corner of this vast and classical theory, beginning with the
fact that the only measurable functions $f \from \R \to \R$ satisfying the
Cauchy functional equation $f(x + y) = f(x) + f(y)$ are the linear mappings
$x \mapsto cx$.  Building on classical results, we obtain new axiomatic
characterizations of a variety of measures of diversity, entropy and value.
We also explain a new method, pioneered by Aubrun%
%
\index{Aubrun, Guillaume}
%
and Nechita%
%
\index{Nechita, Ion} 
%
in
2011~\cite{AuNe}, for solving functional equations by harnessing the power
of probability theory.  This produces new characterizations of the $\ell^p$
norms and the power means.

Characterization theorems for the power%
%
\index{power mean} 
%
means are, in fact, the engine of
this book (Chapter~\ref{ch:mns}).  By definition, the power mean of order
$t$ of real numbers $x_1, \ldots, x_n$, weighted by a probability
distribution $(p_1, \ldots, p_n)$, is
\[
M_t(\vc{p}, \vc{x}) 
=
\Biggl( \sum_{i = 1}^n p_i x_i^t \Biggr)^{1/t}.
\]
The power means $(M_t)_{t \in \R}$ form a one-parameter family of
operations, and the central place that they occupy in this text is explained
by their relationship with several other important one-parameter families:
the Hill numbers, the R\'enyi entropies, the $q$-logarithms, the
$q$-logarithmic entropies (also known as Tsallis entropies), the value
measures of Chapter~\ref{ch:value}, and the $\ell^p$-norms.  We will prove
characterization theorems for all of these families, in each case finding a
short list of properties that determines them uniquely.

\introbreak

Much of this text can be described as `mathematical%
% 
\index{mathematical anthropology} 
% 
anthropology'.  The mathematical anthropologist begins by observing that
some group of scientists attaches great importance to a particular object
or concept: homotopy theorists talk a lot about simplicial sets, harmonic
analysts constantly use the Fourier transform, ecologists often count the
number of species present in a community, and so on.  The next step is to
ask: why do they attach such importance to that particular thing, not
something slightly different?  Is it the \emph{only} object that enjoys the
useful properties that it enjoys?  If not, why do they use the object they
use, and not some other object with those properties?  And if it \emph{is}
the only object with those properties, can we prove it?  For
example, 2008 work of Alesker, Artstein-Avidan and Milman~\cite{AAAM}
proved that the Fourier transform is, in fact, the only transform that enjoys
its familiar properties.

This is the animating spirit of the field of functional equations.
But there is another field that has been
enormously successful in mathematical anthropology: category%
\index{category theory} 
% 
theory.  There, objects of mathematical interest are typically
characterized by universal%
%
\index{universal property} 
% 
properties.  For instance, the tensor product $M \otimes N$ of modules $M$
and $N$ is the universal module equipped with a bilinear map $M \times N
\to M \otimes N$; the Hilbert space completion $\hat{X}$ of an inner
product space $X$ is the universal Hilbert space equipped with an isometry
$X \to \hat{X}$; the real interval $[0, 1]$ is the universal bipointed
topological space equipped with a map $[0, 1] \to [0, 1] \vee [0, 1]$
(Theorem~2.2 of Leinster~\cite{GTSS} and Theorem~2.5 of
Leinster~\cite{GSSO}, building on results of Freyd~\cite{FreARA}).  Any
universal property involves uniqueness at two levels: the literal
uniqueness of a connecting \emph{map}, and the fact that the universal
property characterizes the \emph{object} possessing it uniquely up to
isomorphism.  Thus, category theory is a potent tool for proving
characterization theorems.

We demonstrate this with a categorically-motivated characterization theorem
for entropy (Baez, Fritz and Leinster~\cite{CETIL}).  Briefly put, the
probability distributions on finite sets form an operad\index{operad}, we
construct a certain universal category acted on by that operad, and this
leads naturally to the concept of Shannon entropy.  The categorical
approach amounts to a shift of emphasis from the entropy of a probability
space (an object) to the amount of information lost by a deterministic
process (a map).

The moral of this result is that entropy is not just something for applied
scientists.  It emerges inevitably from a general categorical machine,
given as its inputs nothing more obscure than the real line and the
standard topological simplices.  In other words, even in algebra and
topology, entropy is inescapable.

To demonstrate the strength of the axiomatic approach, we finish by
applying it to an entity of purely mathematical interest: entropy modulo a
prime number.  The topic was first introduced as a curiosity by
Kontsevich,%
%
\index{Kontsevich, Maxim} 
% 
as a byproduct of work on polylogarithms~\cite{KontOHL}.  Just as any real
probability distribution $\ppi = (\pi_1, \ldots, \pi_n)$ has a Shannon
entropy $H_\R(\ppi) \in \R$, one can define, for any
prime%
%
\index{entropy!modulo a prime} 
%
$p$ and `probabilities' $\pi_1, \ldots, \pi_n \in \Zp$, a kind of entropy
$H_p(\ppi) \in \Zp$.  The functional forms are quite different:
\[
\begin{array}{rcll}
H_\R(\pi_1, \ldots, \pi_n)      &
= &
\displaystyle
-\sum_{1 \leq i \leq n} \pi_i \log \pi_i
&\in \R, \\[1.5ex]
H_p(\pi_1, \ldots, \pi_n)       &
=&
\displaystyle
-\sum_{\substack{0 \leq r_1, \ldots, r_n < p\\ r_1 + \cdots + r_n = p}}
\frac{\pi_1^{r_1} \cdots \pi_n^{r_n}}{r_1! \cdots r_n!}
&\in \Zp.
\end{array}
\]
One would probably not guess that the second formula is the correct mod~$p$
analogue of the first.  However, the definition is fully justified by a
characterization theorem strictly analogous to the one that characterizes
real Shannon entropy.  And from the categorical perspective, there is a
strictly analogous characterization of information loss mod~$p$.  In short,
the apparatus developed for the real field can be successfully applied to
the field of integers modulo a prime.

\newpage
\introbreak

Finally, this book aims to challenge outdated conceptions of what
applied%
%
\index{applied mathematics} 
% 
mathematics can look like.  Too often, `applied mathematics' is
subconsciously understood to mean `methods of analysis applied to problems
of physics'.  (Or, worse, `applied' is taken to be a euphemism for
`unrigorous'.)  Those applications are certainly enormously important.
However, this excessively narrow interpretation ignores the glittering
array of applications of other parts of mathematics to other kinds of
problem.  It is mere historical accident that a researcher using PDEs in
the study of fluids is usually called an applied mathematician, but one
applying category theory to the design of programming languages is not.

Mathematicians are coming to appreciate that applications of their subject
to biology are enormously fruitful and, with the revolution in the
availability of genetic data, will only grow.  Mackey and Maini asked and
answered the question `What has mathematics done for
biology?'~\cite{MaMa},%
% 
\index{Mackey, Michael}%
\index{Maini, Philip}
% 
quoting the evolutionary biologist and slime mould specialist
John%
%
\index{Bonner, John} 
% 
Bonner on the `rocking back and forth between the reality of experimental
facts and the dream world of hypotheses'.  They reviewed some major
contributions, including striking success stories in ecology, epidemiology,
developmental biology, physiology, and neuro-oncology.  But still, most of
the work cited there (and most of mathematical biology as a whole) uses
parts of mathematics traditionally thought of as `applied', such as
differential equations, dynamical systems, and stochastic analysis.

The reality is that many parts of mathematics conventionally called `pure'
are now being successfully applied in diverse contexts, both biological and
otherwise.  Knot theory has solved longstanding problems in genetic
recombination (Buck and Flapan~\cite{BuFlPKC,BuFlTCK}).  Group theory has
illuminated virus structure (Twarock, Valiunas and Zappa~\cite{TVZ}).
Topological data analysis, founded on the theory of persistent%
%
\index{persistent homology} 
%
homology and
calling on the power of algebraic topology, succeeded in identifying a
hitherto unknown subtype of breast cancer with a 100\% survival rate
(Nicolau, Levine and Carlsson~\cite{NLC}; see Lesnick~\cite{Lesn} for an
expository account).  Order theory, topos theory and classical logic have
all been employed in the quest for improved ways of specifying, modelling
and designing concurrent systems (Nygaard and Winskel~\cite{NyWi}; Joyal,
Nielsen and Winskel~\cite{JNW}; Hennessy and Milner~\cite{HeMi}).  And,
famously, number theory is used to both provide and undermine security of
communications on the internet (Hales~\cite{HaleNBD}).  All of these are
real applications of mathematics.  None is `applied mathematics' as
traditionally construed.

But applications are not the only product of applied mathematics.  It also
\emph{nourishes} the core of mathematics, providing new questions, answers,
and perspectives.  Mathematics applied to physics has done this from
Archimedes to Newton to Witten.  Reed~\cite{Reed} lists dozens of ways in
which mathematics applied to biology is doing it now.  The developments
surveyed in this book provide further evidence that a body of mathematics
can simultaneously be entirely rigorous, be applied effectively to another
branch of science, use parts of mathematics that do not fit the narrow
stereotype of `applied mathematics', and produce new results that are
significant and satisfying from a purely mathematical aesthetic.






\section*{Background}		\label{p:background}


\concept{Category Theory}

Here is a summary of the categorical background and terminology needed in order
to read the entire paper.  The reader who isn't familiar with everything below
shouldn't be put off: each individual Definition only uses some of it.

I assume familiarity with \demph{categories}, \demph{functors},
\demph{natural transformations}, \demph{adjunctions}, \demph{limits}, and
\demph{monads} and their \demph{algebras}.  Limits include \demph{products},
\demph{pullbacks} (with the pullback of a diagram $X \go Z \og Y$ sometimes
written $X \times_Z Y$), and \demph{terminal objects} (written $1$,
especially for the terminal set $\{ * \}$); we also use
\demph{initial objects}.  A monad $(T,\eta,\mu)$ is often abbreviated to $T$.

I make no mention of the difference between sets and classes (`small
and large collections').  All the Definitions are really of \emph{small}
weak $n$-category.

Let \cat{C} be a category.  $X\in \cat{C}$ means that $X$ is an object of
$\cat{C}$, and $\cat{C}(X,Y)$ is the set of morphisms (or \demph{maps}, or
\demph{arrows}) from $X$ to $Y$ in \cat{C}.  If $f\in \cat{C}(X,Y)$ then $X$
is the \demph{domain} or \demph{source} of $f$, and $Y$ the \demph{codomain}
or \demph{target}.

\Set\ is the category (sets $+$ functions), and \Cat\ is (categories $+$
functors).  A set is just a \demph{discrete category} (one in which the only
maps are the identities).

$\cat{C}^\op$ is the \demph{opposite} or \demph{dual} of a category
\cat{C}.  $\ftrcat{\cat{C}}{\cat{D}}$ is the category of functors from
$\cat{C}$ to $\cat{D}$ and natural transformations between them.  Any object
$X$ of $\cat{C}$ induces a functor $\cat{C}(X, \dashbk): \cat{C} \go \Set$,
and a natural transformation from $\cat{C}(X, \dashbk)$ to
$F: \cat{C} \go \Set$ is the same thing as an element of $FX$ (the
\demph{Yoneda Lemma}); dually for $\cat{C}(\dashbk,X): \cat{C}^\op \go \Set$.

A functor $F: \cat{C} \go \cat{D}$ is an \demph{equivalence} if these
equivalent conditions hold: (i) $F$ is full, faithful and essentially
surjective on objects; (ii) there exist a functor $G: \cat{D} \go \cat{C}$ (a
\demph{pseudo-inverse} to $F$) and natural isomorphisms $\eta: 1 \go GF$,
$\epsln: FG \go 1$ ; (iii) as~(ii), but with $(F,G,\eta,\epsln)$ also being
an adjunction.

Any set $\cat{D}_0$ of objects of a category \cat{C} determines a \demph{full
subcategory} \cat{D} of \cat{C}, with object-set $\cat{D}_0$ and
$\cat{D}(X,Y) = \cat{C}(X,Y)$.  Every category \cat{C} has a
\demph{skeleton}: a subcategory whose inclusion into \cat{C} is an
equivalence and in which no two distinct objects are isomorphic.  If $F, G:
\cat{C} \go \Set$, $GX \sub FX$ for each $X \in \cat{C}$, and $F$ and $G$
agree on morphisms of \cat{C}, then $G$ is a
\demph{subfunctor} of $F$.

A \demph{total order} on a set $I$ is a reflexive transitive
relation $\leq$ such that if $i \neq j$ then exactly one of $i\leq j$ and
$j\leq i$ holds.  
$(I,\leq)$ can be seen as a category with object-set $I$
in which each hom-set has at most one element.
An \demph{order-preserving map} $(I,\leq) \go (I',\leq')$
is a function $f$ such that $i \leq j \implies f(i) \leq' f(j)$.

Let \Del\ be the category with objects $[k]=\{0,\ldots,k\}$ for $k\geq 0$,
and order-preserving functions as maps.  A \demph{simplicial set} is a
functor $\Delop \go \Set$.  Every category \cat{C} has a \demph{nerve} (the
simplicial set $N\cat{C}: [k] \goesto \Cat([k],\cat{C})$), giving a full and
faithful functor $N: \Cat \go \ftrcat{\Delop}{\Set}$.  So \Cat\ is equivalent
to the full subcategory of \ftrcat{\Delop}{\Set} with objects $\{ X \such X
\iso N\cat{C} \textrm{ for some } \cat{C} \}$; there are various
characterizations of such $X$, but we come to that in the main text.

Leftovers: a \demph{monoid} is a set (or more generally, an object of a
monoidal category) with an associative binary operation and a two-sided unit.
\Cat\ is monadic over the category of directed graphs.  The \demph{natural
numbers} start at $0$.


\clearpage



\concept{Strict $n$-Categories}


If \cat{V} is a category with finite products then there is a category
$\cat{V}\hyph\Cat$ of \cat{V}-enriched categories and \cat{V}-enriched
functors, and this itself has finite products.  (A \demph{\cat{V}-enriched
category} is just like an ordinary category, except that the `hom-sets' are
now objects of \cat{V}.)  Let $0\hyph\Cat = \Set$ and, for $n\geq 0$,
$(n+1)\hyph\Cat = (n\hyph\Cat)\hyph\Cat$; a \demph{strict $n$-category} is an
object of $n\hyph\Cat$.  Note that $1\hyph\Cat = \Cat$.

Any finite-product-preserving functor $U: \cat{V} \go \cat{W}$
induces a finite-product-preserving functor $U_*: \cat{V}\hyph\Cat \go
\cat{W}\hyph\Cat$, so we can define functors $U_n: (n+1)\hyph\Cat \go
n\hyph\Cat$ by taking $U_0$ to be the objects functor and $U_{n+1} =
(U_n)_*$.  The category $\omega\hyph\Cat$ of \demph{strict
$\omega$-categories} is the limit of the diagram
\[
\cdots 
\goby{U_{n+1}} (n+1)\hyph\Cat  \goby{U_n} 
\cdots
\goby{U_1} 1\hyph\Cat = \Cat
\goby{U_0} 0\hyph\Cat = \Set.
\]

Alternatively: a \demph{globular set} (or \demph{$\omega$-graph}) $A$
consists of sets and functions
\[
\cdots 
\parpair{s}{t} A_m  \parpair{s}{t} A_{m-1} \parpair{s}{t} 
\cdots 
\parpair{s}{t} A_0
\]
such that for $m\geq 2$ and $\alpha\in A_m$, $ss(\alpha) = st(\alpha)$ and
$ts(\alpha) = tt(\alpha)$.  An element of $A_m$ is called an
\demph{$m$-cell}, and we draw a $0$-cell $a$ as $\gzeros{a}$, a $1$-cell $f$
as $\gfsts{a}\gones{f}\glsts{b}$ (where $a=s(f), b=t(f)$), a 2-cell $\alpha$
as $\gfsts{a}\gtwos{f}{g}{\alpha}\glsts{b}$, etc.  For $m > p \geq 0$, write
$ A_m \times_{A_p} A_m = \{ (\alpha',\alpha) \in A_m \times A_m \such
s^{m-p}(\alpha') = t^{m-p}(\alpha) \}$.

A \demph{strict $\omega$-category} is a globular set $A$ together with a
function $\ofdim{p}: A_m \times_{A_p} A_m \go A_m$ (\demph{composition}) for
each $m > p \geq 0$ and a function $i: A_m \go A_{m+1}$ (\demph{identities},
usually written $i(\alpha) = 1_\alpha$) for each $m\geq 0$, such that
%
\begin{enumerate}
\item 	\label{part:strict-n:source-comp}
if $m > p \geq 0$ and
$(\alpha',\alpha) \in A_m \times_{A_p} A_m$ then
\[
\begin{array}{llll}
s(\alpha' \ofdim{p} \alpha) = 
s(\alpha) 			&	
\textrm{and}			&
t(\alpha' \ofdim{p} \alpha) = 
t(\alpha') 			&
\textrm{for }
m=p+1	\\
s(\alpha' \ofdim{p} \alpha) = 
s(\alpha') \ofdim{p} s(\alpha)	&
\textrm{and}			&
t(\alpha' \ofdim{p} \alpha) = 
t(\alpha') \ofdim{p} t(\alpha)	&
\textrm{for }
m\geq p+2	
\end{array}
\]
\item  	\label{part:strict-n:source-id}
if $m\geq 0$ and $\alpha\in A_m$ then $s(i(\alpha)) = \alpha =
t(i(\alpha))$ 
\item \label{part:strict-n:ass-and-id} 
if $m > p \geq 0$ and $\alpha \in A_m$ then $i^{m-p}(t^{m-p}(\alpha))
\ofdim{p} \alpha = \alpha = \alpha \ofdim{p}$\linebreak
$i^{m-p}(s^{m-p}(\alpha))$; if also $\alpha', \alpha''$ are such that
$(\alpha'', \alpha'), (\alpha', \alpha) \in A_m \times_{A_p} A_m$, then
$(\alpha'' \ofdim{p} \alpha') \ofdim{p} \alpha = \alpha'' \ofdim{p} (\alpha'
\ofdim{p} \alpha)$
\item  	\label{part:strict-n:int}
if $p>q\geq 0$ and $(f',f) \in A_p \times_{A_q} A_p$ then
$i(f') \ofdim{q} i(f) = i(f' \ofdim{q} f)$; 
if also $m>p$ and $\alpha,\alpha',\beta,\beta'$ are such that
$(\beta',\beta), (\alpha', \alpha) \in A_m \times_{A_p} A_m$, 
$(\beta',\alpha'), (\beta, \alpha) \in A_m \times_{A_q} A_m$,
then 
$(\beta' \ofdim{p} \beta) \ofdim{q} (\alpha' \ofdim{p} \alpha) 
= 
(\beta' \ofdim{q} \alpha') \ofdim{p} (\beta \ofdim{q} \alpha)$.
\end{enumerate}

The composition $\ofdim{p}$ is `composition of $m$-cells by gluing along
$p$-cells'.  Pictures for $(m,p) = (2,1), (1,0), (2,0)$ are in the
Bicategories section below. 

\demph{Strict $n$-categories} are defined similarly, but with the globular
set only going up to $A_n$.  \demph{Strict $n$- and $\omega$-functors} are
maps of globular sets preserving composition and identities; the categories
$n\hyph\Cat$ and $\omega\hyph\Cat$ thus defined are equivalent to the ones
defined above.  The comments below on the two alternative definitions of
bicategory give an impression of how this equivalence works. 

\clearpage




\concept{Bicategories}

Bicategories are the traditional and best-known formulation of `weak
2-category'.  

A \demph{bicategory} $B$ consists of
%
\begin{itemize}
\item a set $B_0$, whose elements $a$ are called \demph{0-cells} or
\demph{objects} of $B$ and drawn
$\gzeros{a}$
\item for each $a,b \in B_0$, a category $B(a,b)$, whose objects $f$ are
called \demph{1-cells} and drawn $\gfsts{a} \gones{f} \glsts{b}$, whose
arrows $\alpha: f \go g$ are called \demph{2-cells} and drawn $\gfsts{a}
\gtwos{f}{g}{\alpha} \glsts{b}$, and whose composition $\gfsts{a}
\gthrees{f}{g}{h}{\alpha}{\beta} \glsts{b} \goesto \gfsts{a}
\gtwos{f}{h}{\!\!\!\!\!\! \beta \sof \alpha} \glsts{b}$ is called
\demph{vertical composition} of 2-cells
\item for each $a \in B_0$, an object $1_a \in B(a,a)$ (the \demph{identity}
on $a$); and for each $a,b,c \in B_0$, a functor $B(b,c) \times B(a,b) \go
B(a,c)$, which on objects is called \emph{1-cell composition},
$\gfsts{a}\gones{f}\gblws{b}\gones{g}\glsts{c} \goesto 
\gfsts{a}\gones{g\sof f}\glsts{c}$, and on arrows is called \demph{horizontal
composition} of 2-cells, $\gfsts{a} \gtwos{f}{g}{\alpha} \gfbws{a'}
\gtwos{f'}{g'}{\alpha'} \glsts{a''} \goesto \gfsts{a} \gtwos{f' \sof f}{g' \sof
g}{\!\!\!\!\!\! \alpha' * \alpha} \glsts{a''}$ 
\item \demph{coherence 2-cells}: for each $f \in B(a,b), g \in B(b,c), h \in
B(c,d)$, an \demph{associativity isomorphism} $\xi_{h,g,f}: (h\of g)\of f
\go h\of (g\of f)$; and for each $f \in B(a,b)$, \demph{unit isomorphisms}
$\lambda_f: 1_b \of f \go f$ and $\rho_f: f \of 1_a \go f$
\end{itemize}
%
satisfying the following \demph{coherence axioms}:
%
\begin{itemize}
\item $\xi_{h,g,f}$ is natural in $h$, $g$ and $f$, and $\lambda_f$ and
$\rho_f$ are natural in $f$
\item if $f \in B(a,b), g \in B(b,c), h \in B(c,d), k
\in B(d,e)$, then 
$
\xi_{k,h,g\sof f} \,\of\, \xi_{k\sof h, g, f} = 
(1_k * \xi_{h,g,f}) \,\of\, \xi_{k,h\sof g,f} \,\of\, (\xi_{k,h,g} * 1_f)
$
(the \demph{pentagon axiom});
and if $f \in B(a,b), g \in B(b,c)$, then 
$
\rho_g * 1_f =
(1_g * \lambda_f) \,\of\, \xi_{g,1_b,f}
$
(the \demph{triangle axiom}).
\end{itemize}

An alternative definition is that a bicategory consists of sets and functions
$B_2 \parpair{s}{t} B_1 \parpair{s}{t} B_0$ satisfying $ss=st$ and $ts=tt$,
together with functions determining composition, identities and coherence
cells (in the style of the second definition of strict $\omega$-category
above).  The idea is that $B_m$ is the set of $m$-cells and that $s$ and $t$
give the source and target of a cell.  Strict 2-categories can be identified
with bicategories in which the coherence 2-cells are all identities.

A 1-cell $\gfsts{a}\gones{f}\glsts{b}$ in a bicategory $B$ is called an
\demph{equivalence} if there exists a 1-cell $\gfsts{b}\gones{g}\glsts{a}$
such that $g\of f \iso 1_a$ and $f\of g \iso 1_b$.  

A \demph{monoidal category} can be defined as a bicategory with only one
0-cell: for if the 0-cell is called $\star$ then the bicategory just consists
of a category $B(\star,\star)$ equipped with an object $I$, a functor
$\otimes: B(\star,\star)^2 \go B(\star,\star)$, and associativity and unit
isomorphisms satisfying coherence axioms.

We can consider \demph{strict functors} of bicategories, in which composition
etc is preserved strictly; more interesting are \demph{weak functors} $F$, in
which there are isomorphisms $Fg \of Ff \go F(g \of f)$, $1_{Fa} \go
F(1_a)$ satisfying coherence axioms.

%!TEX root = all.tex
% ******************************************************************
% ** Title:            The 2-category theory of quasi-categories
% **                   2-Categorical Arguments
% ** Precis:        
% ** Author:           Emily Riehl and Dominic Verity
% ** Commenced:        2/3/2012
% ******************************************************************

  \section{The 2-category of quasi-categories}\label{sec:twocat}

  The full subcategory $\qCat$ of quasi-categories and functors is closed in $\sSet$ under products and internal homs. 
   It follows that $\qCat$ is cartesian closed and that it becomes a full simplicial sub-category of $\sSet$ under its usual self enrichment. We denote this self-enriched category of quasi-categories, whose simplicial hom-spaces are given by exponentiation, by $\qCat_\infty$.

  In this section, we study a corresponding (strict) 2-category of quasi-categories $\qCat_2$ first introduced by Joyal \cite{Joyal:2008tq}. This should be thought of as being a kind of quotient of $\qCat_\infty$ whose 2-cells (1-arrows in the hom-spaces) are replaced by homotopy classes of such and in which higher dimensional information in the hom-spaces is discarded.  At first blush, it might seem that such a process would destroy far too much information to be of any great use. However, much of this paper is devoted to showing, perhaps quite surprisingly, that we may develop a great deal of the elementary category theory of quasi-categories within the 2-category $\qCat_2$ alone. Our first step in this direction will be to recognise that much of this category theory may be encoded in the weak 2-universal properties of certain constructions in this 2-category.

  In this section, we introduce the 2-category $\qCat_2$ of quasi-categories and establish a few of its basic properties. In particular, we define a particular notion of weak 2-limit appropriate to this context and show that $\qCat_2$ admits certain weak 2-limit constructions. In later sections, we use the structures introduced here to transport classical categorical proofs into the quasi-categorical context. 

	\subsection{Relating 2-categories and simplicially enriched categories}
	
\begin{ntn}[simplicial categories and 2-categories]
  The category of simplicial sets $\sSet$ is complete, cocomplete, and cartesian closed, so in particular it supports a well developed enriched category theory. We refer to $\sSet$-enriched categories simply as {\em simplicial categories\/} and the enriched functors between them as {\em simplicial functors}.

In a simplicial category $\tcat{C}$, we call the $n$-simplices of one of its simplicial hom-spaces $\tcat{C}(A,B)$ its {\em $n$-arrows\/} from $A$ to $B$. The composition operation of $\tcat{C}$ restricts to make the graph of the objects and $n$-arrows of $\tcat{C}$ into a category which we shall call $\tcat{C}_n$, for which $\tcat{C}_n(A,B) = \tcat{C}(A,B)_n$. Furthermore, if $\alpha\colon[n]\to[m]$ is a simplicial operator then its action on arrows gives rise to an identity-on-objects functor $\tcat{C}_m\to\tcat{C}_n$.

The category of all (small) categories $\Cat$ is also complete, cocomplete, and cartesian closed, so it too supports an enriched category theory. We refer to $\Cat$-enriched categories as {\em 2-categories\/} and the enriched functors between then as {\em 2-functors}. In a 2-category $\tcat{C}$, we follow convention and refer to its objects as {\em 0-cells}, the objects in its hom-categories as {\em 1-cells}, and the arrows in its hom-categories as {\em 2-cells}. 

  We refer the reader to Kelly's canonical tome~\cite{Kelly:2005:ECT} for the standard exposition of the yoga of enriched category theory. We also strongly recommend Kelly and Street~\cite{kelly.street:2} and Kelly~\cite{Kelly:1989fk} as elementary introductions to 2-categories and their attendant 2-limit notions. In particular, we encourage the reader to familiarise him- or herself with the rubric of pasting composition discussed in~\cite{kelly.street:2}.
\end{ntn}


  Recollection~\ref{rec:hty-category} reminds us that $\Cat$ may be regarded as a reflective subcategory of $\sSet$, or indeed $\qCat$, via the adjunction $\ho \dashv \nrv$: the natural map $X \to hX$ is an isomorphism if and only if $X$ is (the nerve of) a category.  The fact that $h\colon \sSet\to \Cat$ preserves binary products implies, and in fact is equivalent to, the observation that if $C$ is a category and $X$ is a simplicial set then their internal hom $C^X$ in $\sSet$ is again a category. The proof in \cite[B.0.16]{Joyal:2008tq} is as follows: there is a canonical map of simplicial sets $C^{hX} \to C^X$. Fixing $X$ and varying $C$ these maps define the components of a natural transformation between two right adjoints $\Cat\to\sSet$. This map is invertible because the transposed natural transformation $h(X\times Y) \to hX \times hY$ is an isomorphism.

Recollection~\ref{rec:cart-modcat} tells us the corresponding result for quasi-categories, this being that internal homs whose target objects are quasi-categories are themselves quasi-categories. In particular, it follows that each of the categories $\Cat$, $\qCat$, and $\sSet$ is cartesian closed and that the various inclusions of one into another preserve finite products and internal homs. In particular, we may regard the self-enriched categories $\Cat$ and $\qCat$ as being full simplicial subcategories of $\sSet$ under its self enrichment.  We write $\Cat_2$ for this 2-category of categories, regarded as a full subcategory of $\qCat_\infty$. 


\begin{obs}\label{obs:simp-to-2-cats}
Using the fact that $\ho$ and $\nrv$ both preserve finite products, we may construct an induced adjunction
	\begin{equation*}
		\adjdisplay \ho_*-|\nrv_*:\twoCat->\sCat. 
	\end{equation*}
	between the categories of 2-categories and simplicial categories respectively. The functors in this adjunction are obtained by applying $\nrv$ and $\ho$  to the hom-objects of an enriched category on one side of this adjunction to obtain a corresponding enriched category on the other side. Here again $\nrv_*$ is fully faithful, so it is natural to regard $\twoCat$ as being a reflective full subcategory both of $\sCat$ and of its full subcategory $\qCat$--$\Cat$ of categories enriched in quasi-categories. Indeed, for our purposes here it suffices to consider the restricted adjunction \[ \adjdisplay \ho_* -| \nrv_* : \twoCat -> \eCat\qCat.\] 

Given a quasi-categorically enriched category $\tcat{C}$, the 2-category $\ho_*\tcat{C}$ is a quotient of sorts. The underlying unenriched categories of $\tcat{C}$ and $\ho_*\tcat{C}$ coincide, but 2-cells in $\ho_*\tcat{C}$ are homotopy classes of 1-arrows in $\tcat{C}$. These homotopy classes are defined using relations witnessed by the 2-arrows. All higher dimensional cells are discarded. On regarding $\ho_*\tcat{C}$ as a simplicially enriched category we see that the unit of the adjunction $\ho_*\dashv\nrv_*$ provides us with a canonical simplicial quotient functor $\tcat{C}\to\ho_*\tcat{C}$.
\end{obs}

  Our identification of categories with their nerves also leads us to regard 2-categories as certain special kinds of simplicial categories. Under this identification, a 1-cell (resp.\ 2-cell) in a 2-category can equally well be regarded as being a 0-arrow (resp.\ 1-arrow) in the corresponding simplicial category.  

\subsection{The 2-category of quasi-categories}

\begin{defn}[the 2-category of quasi-categories]\label{def:qCat-2}
  In particular, applying the functor $\ho_*$ to the quasi-categorically enriched category $\qCat_\infty$, we obtain an associated 2-category $\qCat_2 := h_*\qCat_\infty$ whose hom-categories are given by
  \begin{equation}
    \label{eq:qCat2homdefn} \hom'(A,B) \defeq \ho(B^A).
  \end{equation} 
Using the  description of $h$ given in \ref{rec:hty-category}, we find that the objects of $\qCat_2$ are quasi-categories; the 1-cells are maps of quasi-categories, which we have agreed to call \emph{functors}; and the 2-cells, which we shall call \emph{natural transformations}, are certain homotopy classes of 1-simplices in the internal hom $B^A$. 

  More explicitly, a 2-cell $f\Rightarrow g$ between parallel functors $f,g \colon A \rightrightarrows B$  is an equivalence class represented by a simplicial map $\alpha\colon A\times \Del^1\to B$ making the following diagram \[ \vcenter{ \xymatrix{ A \times \Del^0 \cong A \ar[dr]^f \ar[d]_{A\times\face^1} \\ A \times \Del^1 \ar[r]^-\alpha & B \\ A \times \Del^0 \cong A \ar[ur]_g \ar[u]^{A\times\face^0}}}\] commute. The displayed map $\alpha$ is a 1-simplex in $B^A$ from the vertex $f$ to the vertex $g$. Two such 1-simplices represent the same 2-cell if and only if they are connected by a homotopy (in the sense of \eqref{eq:homotopy-of-1-simplices}) which fixes their common domain $f$ and  codomain $g$. 

We adopt common 2-categorical notation, writing $\alpha\colon f\Rightarrow g$ to denote a 2-cell of $\qCat_2$ which is represented by a simplicial map $\alpha\colon A\times \Del^1\to B$. So if $\alpha,\beta\colon f\Rightarrow g$ are two such represented 2-cells then when we write $\alpha=\beta$ we will not mean that any two particular representing maps $\alpha,\beta\colon A\times\Del^1\to B$ are literally equal but instead that they are appropriately homotopic.

  The 2-category $\qCat_2$ and the simplicial category $\qCat_\infty$ both have the same underlying ordinary category $\qCat$. Furthermore, we know that if $A$ and $B$ are both categories regarded as quasi-categories (via the nerve functor) then $B^A\in \qCat$ is also a category and so $B^A\cong \ho(B^A)$. This in turn implies that the full sub-2-category of $\qCat_2$ spanned by the categories is itself equivalent to $\Cat_2$; we shall identify these from here on. 
  
  The fact that the homotopy category functor $\ho$ preserves finite products allows us to canonically enrich it to a simplicial functor $\ho\colon\qCat_\infty\to\Cat_2$. Specifically we take its action on the hom-space $B^A$ to be the map obtained as the adjoint transpose of the composite $\ho(B^A)\times\ho(A)\cong\ho(B^A\times A)\stackrel{\ho(\ev)}\longrightarrow\ho(B)$.
\end{defn}

\begin{obs}[pointwise isomorphisms are isomorphisms (reprise)]\label{obs:pointwise-iso-reprise}
  We say that a 2-cell $\alpha\colon f\Rightarrow g\colon A\to B$ of $\qCat_2$ is a {\em pointwise isomorphism\/} if and only if for all functors $a\colon \Del^0\to A$ (objects of $A$) the whiskered composite 2-cell $\alpha a \colon fa\Rightarrow ga\colon\Del^0\to B$ is an isomorphism in $\hom'(\Del^0, B) = \ho{B}$. Using this notion, Corollary~\ref{cor:pointwise-equiv} may be recast to posit that $\alpha$ is a pointwise isomorphism in $\qCat_2$ if and only if it is a genuine isomorphism in $\hom'(A,B)=\ho(B^A)$.
\end{obs}

Since $\qCat_\infty$ is the self enrichment of $\qCat$ under its cartesian product,  it is cartesian closed as a quasi-categorically enriched category. We now show that the 2-category $\qCat_2$ inherits the corresponding property:

\begin{prop}\label{prop:qcat2closed} $\qCat_2$ is cartesian closed as a 2-category.
\end{prop}

\begin{proof}
We show that the terminal object, binary products, and internal hom of the quasi-categorically enriched category $\qCat_\infty$ possess the corresponding 2-categorical universal properties. Specifically, we need to demonstrate the existence of canonical isomorphisms
\begin{align*}
  \hom'(A, \Del^0) &\cong \catone \\
  \hom'(A, B\times C) &\cong \hom'(A, B)\times\hom'(A,C) \\
  \hom'(A, C^B) &\cong \hom'(A\times B, C)
\end{align*}
of categories which are natural in all variables.

To establish each of these we simply apply the homotopy category functor $\ho$ to translate the corresponding $\qCat$-enriched universal properties to $\Cat$-enriched ones, as expressed in terms of the hom-categories defined in~\eqref{eq:qCat2homdefn}. 

Because $\Delta^0$ is a terminal object in the simplicially enriched sense, i.e., because $(\Delta^0)^A \cong \Delta^0$, it is also terminal in the 2-categorical sense: applying $\ho$, the canonical isomorphism
  \[
    \hom'(A,\Del^0) = h( (\Delta^0)^A) \cong h( \Delta^0) \cong \catone
  \] 
 asserts that the hom-category from $A$ to $\Delta^0$ is the terminal category. 

  In a similar fashion, since $\qCat_\infty$ is cartesian closed we know that $B \times C$ is a simplicially enriched product, as expressed by the canonical isomorphisms $(B \times C)^A \cong B^A \times C^A$. Applying $\ho$ we get:
\begin{align*}
    \hom'(A, B\times C) = h((B \times C)^A) & \cong h(B^A \times C^A) \\
    &\cong h(B^A) \times h(C^A) = \hom'(A,B)\times\hom'(A,C).
\end{align*} 

  Finally, the cartesian closure of $\qCat_\infty$  gives rise to isomorphisms $(C^B)^A\cong C^{A \times B}$, to which we may apply the homotopy category functor $\ho$ to obtain the isomorphism 
  \[ 
    \hom'(A\times B, C) = h(C^{A \times B}) \cong h((C^B)^A) = \hom'(A,C^B)
  \] 
  which says that $C^B$ defines an internal hom for the 2-category $\qCat_2$.
\end{proof}

As for any cartesian closed 2-category, the exponential defines a 2-functor $\qCat_2\op \times \qCat_2 \to \qCat_2$.

\begin{defn}[the 2-category of all simplicial sets]\label{defn:2-cat.of.all.simpsets}
  The category $\sSet$ of all simplicial sets is cartesian closed, so we can apply the functor $\ho_*\colon\sCat\to\twoCat$ to its self-enrichment. This provides us with a 2-category $\sSet_2 \defeq \ho_*\sSet$ of all simplicial sets, which has $\qCat_2$ as a full sub-2-category. On occasion, we make slightly implicit use of this larger 2-category. However, we generally choose not to distinguish it notationally from $\qCat_2$, leaving whatever disambiguation is required to the context. 
\end{defn}

\begin{rmk}\label{rmk:exp2functor} 
Exponentiation in the cartesian closed simplicial category $\sSet$ restricts to a simplicial cotensor functor $\sSet\op\times\qCat_\infty\to\qCat_\infty$.

Proposition~\ref{prop:qcat2closed} extends immediately to show that the 2-category of all simplicial sets is again cartesian closed as a 2-category.  Applying $\ho_*$, we obtain a 2-functor $\sSet_2\op\times\qCat_2\to\qCat_2$.  In particular, it follows that exponentiation by any simplicial set $X$ defines a 2-functor $(-)^X \colon \qCat_2 \to \qCat_2$.
  \end{rmk}

\begin{defn}[equivalences in 2-categories]
A 1-cell $u\colon A\to B$ in a 2-category $\tcat{C}$ is an {\em equivalence\/} if and only if there exists a 1-cell $v\colon B\to A$, called its {\em equivalence inverse}, and a pair of 2-isomorphisms $uv\cong\id_B$ and $vu\cong\id_A$. 
\end{defn}

The equivalences of a 2-category $\tcat{C}$ are preserved by all 2-functors since they are defined by 2-equational conditions. Consequently, if $u\colon A\to B$ is an equivalence in $\tcat{C}$ then, applying the representable 2-functor $\tcat{C}(X,-)$,  the functor $\tcat{C}(X,u)\colon\tcat{C}(X,A)\to \tcat{C}(X,B)$ is an equivalence of hom-categories. A basic 2-categorical fact, whose proof is left to the reader, is that these \emph{representably-defined equivalences} are necessarily equivalences in $\tcat{C}$.

\begin{lem}\label{lem:2-cat.equivs}
A 1-cell $u\colon A\to B$ in a 2-category $\tcat{C}$ is an equivalence if and only if $\tcat{C}(X,u)\colon\tcat{C}(X,A)\to \tcat{C}(X,B)$ is an equivalence of hom-categories for all objects $X \in \tcat{C}$. 
\end{lem}

%\begin{rmk}\label{rmk:weak.equivs.2-cat}
  Our central thesis is that the category theory of quasi-categories developed by Joyal, Lurie, and others is captured by $\qCat_2$. For this, it is essential that the standard notion of equivalence of quasi-categories---weak equivalence in the Joyal model structure---is encoded in the 2-category.

  To that end, observe that the description of the weak equivalences given in \ref{rec:qmc-quasicat} may be recast in our 2-categorical framework: by definition, a simplicial map $u\colon X\to Y$ is a weak equivalence in Joyal's model structure if and only if for all quasi-categories $A$ the functor $\hom'(u,A)\colon\hom'(Y,A)\to\hom'(X,A)$ is an equivalence of hom-categories.
%\end{rmk}

  Combining this description with Proposition \ref{prop:qcat2closed} and Lemma \ref{lem:2-cat.equivs} we obtain the following straightforward results:

\begin{prop}\label{prop:equivsareequivs} A functor between quasi-categories is a weak equivalence in the Joyal model structure if and only if it is an equivalence in the 2-category $\qCat_2$.
\end{prop}
\begin{proof}
The weak equivalences between quasi-categories are the representably defined equivalences in the dual 2-category $\qCat_2\op$. Equivalence in a 2-category is a self dual notion, so these coincide with the equivalences in $\qCat_2$.
\end{proof}

\begin{prop}\label{prop:equivsareequivs2}
  A simplicial map $u\colon X\to Y$ is a weak equivalence in the Joyal model structure if and only if for all quasi-categories $A$ the pre-composition functor $A^u\colon A^Y\to A^X$ is an equivalence in the 2-category $\qCat_2$.
\end{prop}

\begin{proof}
  By Lemma~\ref{lem:2-cat.equivs}, $A^u\colon A^Y\to A^X$ is an equivalence in $\qCat_2$ if and only if for all quasi-categories $B$ the functor $\hom'(B,A^u) \colon \hom'(B,A^Y) \to \hom'(B,A^X)$ is an equivalence of hom-categories. Taking duals, $\hom'(B,A^u)$ is isomorphic to $\hom'(u,A^B) \colon \hom'(Y,A^B) \to \hom'(X,A^B)$. Hence, it suffices to show that  $u\colon X\to Y$ is a weak equivalence in Joyal's model structure if and only if $\hom'(u, A^B)$ is an equivalence of hom-categories for all quasi-categories $A$ and $B$, which is the case because $B^A$ is again a quasi-category.
\end{proof}

\subsection{Weak 2-limits}\label{subsec:weak-2-limits}

Finite products aside, the 2-category $\qCat_2$ has few 2-limits. However, we shall soon discover that it has a number of important \emph{weak 2-limits} whose universal properties will be repeatedly exploited in the remainder of this paper.

\begin{defn}[smothering functors]\label{defn:smothering}
  A functor between categories is {\em smothering\/} if and only if it is surjective on objects, full, and conservative (reflects isomorphisms). Equivalently, a functor is smothering if and only if it possesses the right lifting property with respect to the set of functors
  \[ \left\{ \vcenter{\xymatrix@C=5pt{ \emptyset \ar@{u(->}[d] \\ \bullet }},  \vcenter{\xymatrix@C=5pt{ \bullet &  \ar@{u(->}[d] & \bullet  \\ \bullet \ar[rr] & {~} &  \bullet }} , \vcenter{\xymatrix@C=5pt{ \bullet \ar[rr] & \ar@{u(->}[d] & \bullet \\ \bullet \ar[rr] & {~} & \bullet \ar@<.8ex>[ll] }}  \right\}  = \left\{ \vcenter{\xymatrix@C=5pt{ \emptyset \ar@{u(->}[d] \\ \catone }},  \vcenter{\xymatrix@C=5pt{   \catone\sqcup\catone \ar@{u(->}[d]   \\   \cattwo  }} , \vcenter{\xymatrix@C=5pt{ \cattwo \ar@{u(->}[d]   \\  \iso }}  \right\}  \]
  Consequently, the class of smothering functors contains all surjective equivalences and is closed under composition, retract, and pullback along arbitrary functors. By composing lifting problems, we see that all smothering functors are isofibrations, in the sense that they have the right lifting property with respect to either inclusion $\catone\inc \iso$. It is easily checked that if $f$ is a functor which is surjective on objects and arrows, as is true for a smothering functor, and a composite $gf$ is smothering, then so is the functor $g$.
\end{defn}

The following very simple lemma will be of significant utility later on.

\begin{lem}[fibres of smothering functors]\label{lem:smothering}
  Each fibre of a smothering functor is a non-empty connected groupoid. 
\end{lem}

\begin{proof}
  Suppose that $f\colon A\to B$ is a smothering functor. The fact that it is surjective on objects implies immediately that its fibres are non-empty. Furthermore, if $a$ and $a'$ are both objects of $A$ in the fibre of $f$ over some object $b$ in $B$,  then the fullness of $f$ implies that we may find an arrow $\tau\colon a\to a'$ in $A$ with $f(\tau)=\id_b$, thus demonstrating that the fibres are  connected. Finally, if $\tau\colon a\to a'$ is an arrow of $A$ which lies in the fibre of $f$ over $b$, in other words if $f(\tau) = \id_b$, then by conservativity of $f$ we know that $\tau$ is an isomorphism. Hence, these fibres are groupoids. 
\end{proof}

  We have chosen the term smothering here to evoke the image that these are surjective covering functors in quite a strong sense.  Of course, we have placed our tongues firmly in our cheeks while introducing this nomenclature. Smothering functors can fruitfully be thought of as being a certain variety of weak surjective equivalences.

We weaken the standard theory of {\em weighted 2-limits\/} (see e.g., \cite{Kelly:1989fk}) as follows. 

\begin{defn}[weak 2-limits in a 2-category]\label{defn:weak2limit}
  Suppose that $\stcat{A}$ is a small 2-category, that $D \colon \stcat{A} \to \tcat{C}$ is a diagram in a 2-category $\tcat{C}$, and that $W\colon\stcat{A}\to\Cat_2$ is a 2-functor, which we shall refer to as a {\em weight}. If $P$ is an object in $\tcat{C}$ then a {\em cone with summit $P$ over $D$ weighted by\/} $W$ is a 2-natural transformation $c\colon W\To \tcat{C}(P,D(-))$. 

  For each object $K$ of $\tcat{C}$, composition with such a cone induces a functor
  \begin{equation}\label{eq:cone-induced}
    c_K\colon\tcat{C}(K,P)\longrightarrow \lim(W, \tcat{C}(K,D(-)))\cong\int_{a\in\stcat{A}} \tcat{C}(K,D(a))^{W(a)}
  \end{equation}
  where the expression on the right denotes the usual category of 2-natural transformations from W to the 2-functor $\tcat{C}(K,D(-))$, the 2-limit of $\tcat{C}(K,D(-))$  weighted by $W$. The family of maps \eqref{eq:cone-induced} is 2-natural in $K$.

  We say that the cone $c$ displays $P$ as a {\em weak $2$-limit of $D$ weighted by\/} $W$ if and only if the map in \eqref{eq:cone-induced} is a smothering functor for all objects $K\in\tcat{C}$.
\end{defn}

  While we feel obliged to give the last definition in its full, slightly unsightly, generality. However, the reader need not become an expert in the technology of weighted 2-limits in order to read the rest of the paper. We shall only work with certain simple varieties of weak 2-limits in $\qCat_2$, whose weak 2-universal properties we shall describe explicitly.

The fact that the fibres of a smothering functor are connected groupoids is the key ingredient in the proof of the following lemma.

\begin{lem}\label{lem:unique-weak-2-limits} Weak 2-limits are unique up to equivalence: the summits of any two weak 2-limit over a common diagram with a fixed weight are equivalent via an equivalence that commutes with the legs of the limit cones.
\end{lem}

\begin{proof}
Given a pair of cones $c\colon W\To \tcat{C}(P,D(-))$ and $c'\colon W\To \tcat{C}(P',D(-))$ that display  $P$ and $P'$ as weak 2-limits of $D$ weighted by $W$, then for each $K \in \tcat{C}$ we have a pair of smothering functors:
  \begin{equation*}
    \tcat{C}(K,P)\stackrel{c_K}\longrightarrow
    \lim(W, \tcat{C}(K,D(-)))
    \stackrel{c'_{K}}\longleftarrow\tcat{C}(K,P')
  \end{equation*}
  Taking $K=P$, consider the identity 1-cell $\id_P$, an object in the hom-category $\tcat{C}(P,P)$. Since $c'_P$ is surjective on objects, there is a 1-cell $u\colon P\to P'$, an object in $\tcat{C}(P,P')$, such that $c'_{P}(u) = c_{P}(\id_P)$. Exchanging the role of $P$ and $P'$, we also find a 1-cell $u'\colon P'\to P$ such that $c_{P'}(u')=c'_{P'}(\id_{P'})$. These definitions ensure that $u$ and $u'$ commute with the legs of the limit cones.

  Now we can apply the 2-naturality properties of the functors $c_K$ and $c'_K$ to show that
  \begin{align*}
    c_{P}(u'u) &= \lim(W, \tcat{C}(u,D(-)))(c_{P'}(u')) && \text{naturality of family $c_K$}\\
    &= \lim(W, \tcat{C}(u,D(-)))(c'_{P'}(\id_{P'})) && \text{definition of $u'$}\\
    &= c'_{P}(u) && \text{naturality of family $c'_K$}\\
    & = c_{P}(\id_P) && \text{definition of $u$.}
  \end{align*} 
  In other words, $u'u$ and $\id_P$ are both in the same fibre of $c_P$, and so they are isomorphic in that fibre since $c_P$ is a smothering functor. Dually, $uu'$ and $\id_{P'}$ are both in the same fibre of $c'_{P'}$ from which it follows that they too are isomorphic in that fibre. It follows that $u\colon P\to P'$ and $u'\colon P'\to P$ are equivalence inverses. 
\end{proof}

The only diagrams we will consider are indexed by small 1-categories $\stcat{A}$.   Because $\qCat_2$ and $\qCat_\infty$ have the same underlying category, a diagram $D \colon \stcat{A}\to\qCat$ is equally a 2-functor $D\colon\stcat{A}\to\qCat_2$ and a simplicial functor $D\colon\stcat{A}\to\qCat_\infty$. A weight $W\colon\stcat{A}\to\Cat$ for a 2-limit can be regarded as a weight for a simplicial limit by composing with the subcategory inclusion $\Cat\inc\sSet$. Our general strategy will be to show that  the simplicial weighted limit $\lim(W,D)$ exists in $\qCat_\infty$ and that it has the weak 2-universal property expected of the weak 2-limit of $D$ in $\qCat_2$. The following lemma allows us to considerably simplify the class of functors \eqref{eq:cone-induced} that we will need to consider.

\begin{lem}\label{lem:weak-simplification} Fix a small 1-category $\stcat{A}$ and a weight $W \colon \stcat{A} \to \Cat$. Suppose $\mclass{D}$ is a class of diagrams $D\colon\stcat{A} \to \qCat$ that is closed under exponentiation by quasi-categories, in the sense that if $D$ is in the class $\mclass{D}$ then so is $D(-)^X$ for any quasi-category $X$. Then $\qCat_2$ admits weak $W$-weighted 2-limits of this class of diagrams if and only if, for all $D \in \mclass{D}$, the canonical functor
\[ \ho(\lim(W,D)) \to \lim(W,h(D(-)))\] is smothering.
\end{lem}
\begin{proof}
By Definition \ref{defn:weak2limit}, to show that the simplicial weighted limit $\lim(W,D)$ defines a weak 2-limit of a diagram $D \colon \stcat{A} \to \qCat$ in the class $\mclass{D}$, we must show that  for each quasi-category $X$ the canonical comparison map
  \begin{equation}\label{eq:qCat.wl.comp.1}
    \hom'(X,\lim(W,D)) \longrightarrow \lim(W,\hom'(X,D(-)))
  \end{equation}
  is a smothering functor. Recall that $\hom'(X,-) = \ho((-)^X)$. The right adjoint simplicial functor $(-)^X\colon\qCat_\infty\to\qCat_\infty$ preserves all simplicial weighted limits; in other words, the canonical comparison map $\lim(W,D)^X\to\lim(W,D(-)^X)$ is an isomorphism. Thus, the comparison functor~\eqref{eq:qCat.wl.comp.1} is isomorphic to the functor:
  \begin{equation*}
    \ho(\lim(W,D(-)^X)) \longrightarrow \lim(W,\ho(D(-)^X)).
  \end{equation*}
By hypothesis, the diagram $D(-)^X$ is in $\mclass{D}$. Thus, to prove that $\qCat_2$ admits weak 2-limits of the diagrams in $\mclass{D}$, it suffices to show that for all diagrams $D\in\mclass{D}$ the comparison map
  \begin{equation*}
    \ho(\lim(W,D)) \longrightarrow \lim(W,\ho{(D(-))})
  \end{equation*}
  is smothering. 
\end{proof}


\begin{obs}[cones whose summits are not quasi-categories]
The classes of diagrams $\mclass{D}$ we will consider are in fact closed under exponentiation by all simplicial sets. The proof of Lemma \ref{lem:weak-simplification} can then be used to extend the 2-universal properties of the weak 2-limits of $\qCat_2$ constructed here to cones whose summits are arbitrary simplicial sets.  Abstractly speaking, this tells us that the inclusion 2-functor $\qCat_2\inc\sSet_2$ preserves the weak 2-limits of diagrams in $\mclass{D}$. 
In order to avoid repeated remarks of this kind throughout the remainder of this paper, our notation will tacitly signal when this is so by use of the letter ``$X$'' for the object of $\qCat_2$ or $\qCat_\infty$ that could equally be replaced by any simplicial set. By contrast, the letters ``$A$'', ``$B$'', and ``$C$'' refer only to quasi-categories.
\end{obs}

As our first example of a weak 2-limit in $\qCat_2$ we examine cotensors with the generic arrow $\cattwo$. 
Recall we write $A^\cattwo$ for the quasi-category $A^{\Delta^1}$ using our convention that categories are identified with their nerves. We invite the reader to verify that the natural functor $\ho(A^\cattwo) \to (\ho{A})^\cattwo$ is not an isomorphism: it is neither injective on objects nor faithful. However, it is a smothering functor. In other words:

\begin{prop}\label{prop:weak-cotensors} 
The exponential $A^\cattwo$ is a weak cotensor of $A$ by $\cattwo$ in 
$\qCat_2$.
\end{prop}

\begin{proof}
   By Lemma~\ref{lem:weak-simplification}, it suffices to prove that for any quasi-category $A$, the canonical functor \[ \ho(A^\cattwo) \longrightarrow (\ho{A})^\cattwo\] is a smothering functor. Certainly this map is surjective on objects, simply because every arrow in $\ho{A}$ is represented by a 1-simplex in the quasi-category $A$. 

To prove fullness, suppose given a commutative square in $\ho{A}$ and choose arbitrary 1-simplices representing each morphism  and their common composite \begin{equation}\label{eq:arrowinhA}\xymatrix{ \cdot \ar[d]_f \ar[r]^a  \ar[dr]|k & \cdot \ar[d]^g \\ \cdot \ar[r]_b & \cdot}\end{equation} Because $A$ is a quasi-category, any relation between morphisms in $\ho{A}$ is witnessed by a 2-simplex with any choice of representative 1-simplices as its boundary. Hence, we may choose 2-simplices witnessing the fact that $k$ is a composite of $a$ with $g$ and of $f$ with $b$ as displayed. 
 \begin{equation}\label{eq:liftedarrowinA2} \xymatrix{ \cdot \ar[d]_f \ar[r]^{a} \ar[dr]|k^*+{\labelstyle\sim}_*+{\labelstyle \sim}& \cdot \ar[d]^g \\ \cdot \ar[r]_{b} & \cdot}\end{equation}
These two 2-simplices define a map $\Delta^1  \to A^{\Delta^1} = A^\cattwo$, which represents an arrow in the category $h(A^{\cattwo})$ whose image is the specified commutative square.

To prove conservativity, suppose given a map in $h(A^{\cattwo})$ represented by a diagram  \eqref{eq:liftedarrowinA2} whose image  \eqref{eq:arrowinhA} is an isomorphism in $(\ho{A})^\cattwo$, meaning that $a$ and $b$ are isomorphisms in $\ho{A}$, in which case $a$ and $b$ are isomorphisms in the quasi-category $A$. Lemma~\ref{lem:pointwise-equiv} tells us immediately that this diagram is an isomorphism in $A^\cattwo$; compare with \eqref{eq:pointwise-equivalence-square}.
\end{proof}

\begin{rmk}
  A generalisation of this argument shows that if $\scat{C}$ is a free category and $A$ is a quasi-category then the exponential $A^\scat{C}$ is the weak cotensor of $A$ by $\scat{C}$ in $\qCat_2$. Conservativity of the canonical comparison $\ho(A^\scat{C})\to(\ho A)^\scat{C}$ follows from Lemma~\ref{lem:pointwise-equiv}. Its surjectivity on objects makes use of the fact that the inclusion of the \emph{spine} of an $n$-simplex, the simplicial subset spanned by the edges $\fbv{i,i+1}$ in $\Del^n$, is a trivial cofibration for all $n \geq 1$. Fullness is similar.

One should note, however, that this result does not hold for exponentiation by arbitrary categories $\scat{C}$. For example,  $A^{\cattwo\times\cattwo}$ is not the weak cotensor of $A$ by the product category $\cattwo\times\cattwo$ in $\qCat_2$.
\end{rmk}

\begin{prop}
  The exponential $A^\iso$ is a weak cotensor of $A$ by the generic isomorphism $\iso$ in 
  $\qCat_2$. 
\end{prop}

\begin{proof}
 By Lemma \ref{lem:weak-simplification}, it suffices to show that
  \begin{equation*}
    \ho(A^\iso)\longrightarrow \ho(A)^\iso
  \end{equation*}
  is a smothering functor. This is easiest to do by arguing in the marked context. 

By Observation~\ref{obs:nat-mark-homs}, $A^\iso$ may equally well be regarded as an internal hom of naturally marked quasi-categories in $\msSet$. Recollection~\ref{rec:qmc-quasi-marked} tells us that the inclusion $\cattwo^\sharp\inc\iso$ is a trivial cofibration in the marked model structure. Because the marked model structure is cartesian closed, the restriction functor  $A^\iso\to A^{\cattwo^\sharp}$  is a trivial fibration.  Immediately from their defining lifting properties, trivial fibrations of quasi-categories are carried by $\ho$ to functors which are surjective on objects and fully faithful, the so-called {\em surjective equivalences}, so it follows that $\ho(A^\iso)\to \ho(A^{\cattwo^\sharp})$ is a surjective equivalence. Furthermore, in the case where $A$ is an actual category, the functor $A^\iso\to A^{\cattwo^\sharp}$ is an isomorphism. So we obtain a commutative square
  \begin{equation*}
    \xymatrix{
      {\ho(A^\iso)} \ar[r]\ar[d] &
      {\ho(A)^\iso} \ar[d]^{\cong}\\
      {\ho(A^{\cattwo^\sharp})}\ar[r] &
      {\ho(A)^{\cattwo^\sharp}}
    }
  \end{equation*}
of functors between categories in which the left hand vertical is a surjective equivalence. By the composition and cancellation results described in \ref{defn:smothering}, the upper horizontal map in this square is a smothering functor if and only if the lower horizontal map is smothering.

The smothering functors are stable under pullback, so to complete our proof, we will show that  for any naturally marked quasi-category $A$ the square
\begin{equation*}
    \xymatrix{
      {\ho(A^{\cattwo^\sharp})}\pbexcursion \ar[r]\ar[d] &
      {\ho(A)^{\cattwo^\sharp}} \ar[d]\\
      {\ho(A^{\cattwo})}\ar[r] &
      {\ho(A)^{\cattwo}}
    }
  \end{equation*}
  is a pullback;  we know from Proposition~\ref{prop:weak-cotensors} that the lower horizontal map is a smothering functor.  This follows from the definition of the natural marking: a 1-simplex in $A$ is marked if and only if it is an isomorphism, which is the case if and only if it represents an isomorphism in $\ho{A}$. \end{proof}

\begin{prop}\label{prop:weak-homotopy-pullbacks} The 2-category $\qCat_2$ admits weak 2-pullbacks along isofibrations: if the square
\begin{equation*}
  \xymatrix{
    {B\times_A C}\pbexcursion\ar[r]^-{\pi_2} \ar@{->>}[d]_-{\pi_1} 
    &  C\ar@{->>}[d]^g \\
    {B} \ar[r]_f & A
  }
\end{equation*}
is a pullback in simplicial sets for which $B$, $A$, and $C$ are quasi-categories and $g$ is an isofibration, then $B\times_A C$ is a quasi-category and it is a weak 2-pullback of $g$ along $f$ in the 2-category $\qCat_2$.
\end{prop}

\begin{proof}
  The statement only applies to pullbacks of those diagrams of shape $B\xrightarrow{f} A \xleftarrow{g} C$ for which the map $g$ is an isofibration. However, Observation~\ref{obs:isofibration-closure} tells us that any exponentiated isofibration $g^X\colon C^X\to A^X$ is again an isofibration, and so we are in a position to apply Lemma~\ref{lem:weak-simplification}.

   It remains to show that the canonical comparison functor
  \begin{equation*}
    \ho(B\times_A C)\longrightarrow \ho{B}\times_{\ho{A}} \ho{C}
  \end{equation*}
  is smothering. This functor is actually bijective on objects, since in both categories an object consists simply of a pair $(b,c)$ of 0-simplices $b\in B$ and $c\in C$ with $f(b)=g(c)$. 
  
  For fullness, suppose we are given two such pairs $(b,c)$ and $(b',c')$. An arrow between these objects in $\ho{B} \times_{\ho{A}} \ho{C}$ consists of a pair of equivalence classes represented by 1-simplices $\beta \colon b \to b'$ and $\gamma \colon c \to c'$ which both map to the same equivalence class in $\ho{A}$ under $f$ and $g$ respectively. This latter condition simply posits that $f(\beta)$ and $g(\gamma)$ are homotopic relative to their endpoints in $A$; such a homotopy is represented by a 2-simplex with $2^{\text{nd}}$ face $g(\gamma)$, $1^{\text{st}}$ face $f(\beta)$, and $0^{\text{th}}$ face degenerate. This information provides us with a lifting problem between $\Horn^{2,1} \to \Delta^2$ and $g$, which we may solve because $g$ is an isofibration. The resulting filler supplies us with a 1-simplex $\gamma' \colon c \to c'$ for which $g(\gamma')=f(\beta)$ and a homotopy of $\gamma'$ and $\gamma$ (relative to their endpoints) which shows these represent the same arrow in $\ho{C}$. In other words, $(\beta,\gamma')$ is a 1-simplex in $B\times_A C$ that represents an arrow of $\ho(B\times_A C)$ from $(b,c)$ to $(b',c')$ and this arrow maps to the originally chosen arrow in $\ho{B}\times_{\ho{A}}\ho{C}$.

The proof of conservativity is simplified by arguing in the marked model structure.  Giving our quasi-categories $A$, $B$, and $C$ the natural marking, the isofibration $g$ becomes a fibration in the marked model structure. It follows that the pullback is a fibrant object and hence naturally marked. Consequently, a 1-simplex $(\beta,\gamma)$ of $B\times_A C$ represents an isomorphism in $\ho(B\times_A C)$ if and only if it is marked, and this is the case if and only if $\beta$ is marked in $B$ and $\gamma$ is marked in $C$. Now, this latter condition holds if and only if $\beta$ is invertible in $\ho{B}$ and $\gamma$ is invertible in $\ho{C}$ and these conditions together are equivalent to the pair $(\beta,\gamma)$ being invertible as an arrow in the category $\ho{B}\times_{\ho{A}} \ho{C}$.
\end{proof} 

  \begin{defn}[comma objects]\label{def:comma-obj}
 Given a pair of functors $B\xrightarrow{f} A \xleftarrow{g} C$ between quasi-categories, we define the {\em comma object\/} $f\comma g$ to be the simplicial set constructed by forming the following pullback:
    \begin{equation*}
      \xymatrix@=2.5em{
        {f\comma g}\pbexcursion \ar[r]\ar[d]_{p} &
        {A^\cattwo} \ar[d] \\
        {C\times B} \ar[r]_-{g\times f} & {A\times A}
      }
    \end{equation*}
\end{defn}

The right-hand vertical  is defined by restricting along the boundary inclusion $\Del^0\sqcup\Del^0\cong\boundary\Del^1\inc\Del^1$ and then composing with the symmetry isomorphism $A \times A \cong A \times A$. In a subsequent paper, we will think of the comma object $f \comma g$ as a module, with $C$ acting on the left and with $B$ acting on the right, which is the reason for our convention.

\begin{lem}\label{lem:comma-obj} The simplicial set $f \comma g$ is a quasi-category and the projection functors $p_0\defeq\pi_B\circ p \colon f\comma g\tfib B$ and $p_1\defeq\pi_C\circ p\colon f\comma g\tfib C$ are isofibrations.
\end{lem}
\begin{proof}
    The right hand vertical in the pullback square above is isomorphic to the simplicial map $A^{\Del^1}\to A^{\boundary\Del^1}$ and is thus, by \ref{rec:cart-modcat}, an isofibration whenever $A$ is a quasi-category. Consequently, since the product $C\times B$ is again a quasi-category, $p\colon f\comma g\to C\times B$ is  an isofibration and $f\comma g$ is a quasi-category. The projection functors $\pi_C\colon C\times B\tfib C$ and $\pi_B\colon C\times B\tfib B$ are both isofibrations because $B$ and $C$ are fibrant, so it follows that the domain and codomain projection maps $p_0 \colon f\comma g\tfib B$ and $p_1\colon f\comma g\tfib C$ are also isofibrations.
    \end{proof}
        
\begin{lem}[maps induced between comma objects]\label{lem:comma-obj-maps}
A commutative diagram
     \begin{equation*}
       \xymatrix{
         {B}\ar[r]^{f}\ar@{->>}[d]_-{r} & 
         {A}\ar@{->>}[d]_-{q} & 
         {C}\ar[l]_{g}\ar@{->>}[d]^-{s} \\
         {\bar{B}}\ar[r]_{\bar{f}} & 
         {\bar{A}} & 
         {\bar{C}}\ar[l]^{\bar{g}} 
       }
     \end{equation*} 
     in $\qCat$ in which the vertical maps are (trivial) fibrations in the Joyal model structure, induces a (trivial) fibration $r \comma_q s \colon f \comma g \tfib \bar{f}\comma\bar{g}$ between comma quasi-categories.
\end{lem}
\begin{proof}
Consider the commutative diagram
     \begin{equation*}
       \xymatrix@C=4.5em@R=2.5em{
         {C\times B}\ar[r]^-{g\times f}\ar@{->>}[d]_-{s\times r} & 
         {A\times A}\ar@{->}[d]_-{q\times q} & 
         {A^\cattwo}\ar[l]_-{(p_1,p_0)}\ar@{->}[d]^-{q^\cattwo} 
         \save "1,3"-<2.5em,1.5em>*+{P}\ar[l]\ar[d]\ar@{<<.}[]_(0.6)l \restore \\
         {\bar{C}\times\bar{B}}\ar[r]_-{\bar{g}\times\bar{f}} & 
         {\bar{A}\times\bar{A}} & 
         {\bar{A}^\cattwo}\ar[l]^-{(p_1,p_0)} 
       }
     \end{equation*}
     in which $P$ denotes the pullback of the maps $q\times q$ and $(p_1,p_0)$ and $l$ is the unique map induced into it by the right hand square. The pullbacks of the two horizontal lines are the comma objects $f\comma g$ and $\bar{f}\comma\bar{g}$ respectively. So this diagram induces a unique map $r\comma_q s\colon f\comma g\to\bar{f}\comma\bar{g}$ of comma objects which makes the manifest cube commute.
     
     The (trivial) fibrations of any model category are closed under product, so the map $s\times r$ is a (trivial) fibration in the Joyal model structure.  The induced map $l$ is isomorphic to the Leibniz hom $\leib\hom(\boundary\Del^1\inc\Del^1, q\colon A\tfib \bar{A})$; a recalled in \ref{rec:cart-modcat}, cartesianness of the Joyal model structure implies that $l$ is a (trivial) fibration.
The induced map $r\comma_q s\colon f\comma g\tfib\bar{f}\comma\bar{g}$ is again a (trivial) fibration because it factors as a composite of pullbacks of the (trivial) fibrations $s\times r$ and $l$.
   \end{proof}

\begin{prop}\label{prop:weakcomma} For any functors $B \xrightarrow{f} A \xleftarrow{g} C$  of quasi-categories, the comma quasi-category $f\comma g$ is a weak comma object in $\qCat_2$.
\end{prop}
\begin{proof}
    Again, Lemma~\ref{lem:weak-simplification} applies, so it suffices to show that the canonical comparison
  \begin{equation}\label{eq:commacat-comp}
    \ho(f\comma g) \longrightarrow \ho(f)\comma\ho(g)
  \end{equation}
  is a smothering functor. Here the target category is just the usual comma category constructed in $\Cat$. By definition,  $f\comma g \cong (C\times B)\times_{(A\times A)} A^\cattwo$ and consequently we find that we may express the functor in~\eqref{eq:commacat-comp} as a composite:
  \begin{equation*}
    \ho((C\times B)\times_{(A\times A)} A^\cattwo)
    \longrightarrow
    \ho(C\times B)\times_{\ho(A\times A)} \ho(A^\cattwo)
    \longrightarrow
    \ho(C\times B)\times_{\ho(A\times A)} \ho(A)^\cattwo
  \end{equation*}
  The first of these maps is the canonical comparison functor studied in Proposition~\ref{prop:weak-homotopy-pullbacks}, so we know that it is smothering. The second of these maps is a pullback of the canonical comparison functor discussed in Proposition~\ref{prop:weak-cotensors}; since smothering functors are stable under pullback, it too is a smothering functor. We obtain the required result from the fact that smothering functors compose.
\end{proof}

\begin{obs}[unpacking the universal property of weak comma objects]\label{obs:unpacking-weak-comma-objects}
  The smothering functors
  \begin{equation}\label{eq:weak-comma-prop}
    \hom'(X,f\comma g)\longrightarrow\hom'(X,f)\comma\hom'(X,g)
  \end{equation}
  which express the weak 2-universal property of the quasi-category $f\comma g$ are induced by composition with a cone:
  \begin{equation}\label{eq:standard-comma-pic}
    \xymatrix@=10pt{
      & f \downarrow g \ar[dl]_{p_1} \ar[dr]^{p_0} \ar@{}[dd]|(.4){\psi}|{\Leftarrow}  \\ 
      C \ar[dr]_g & & B \ar[dl]^f \\ 
      & A}
  \end{equation}
  The data displayed in \eqref{eq:standard-comma-pic} is the image of the identity 1-cell under \eqref{eq:weak-comma-prop} in the case $X = f\comma g$. The weak universal property of this comma cone has three aspects, corresponding to the surjectivity on objects, fullness, and conservativity of the smothering functor \eqref{eq:weak-comma-prop}, which we refer to as 
 {\em 1-cell induction\/}, {\em 2-cell induction}, and {\em 2-cell conservativity}.


Surjectivity on objects of the functor~\eqref{eq:weak-comma-prop} simply says that for any comma cone
  \begin{equation}\label{eq:comma-cone}
    \xymatrix@=10pt{
       & X \ar[dl]_{c} \ar[dr]^{b} \ar@{}[dd]|(.4){\alpha}|{\Leftarrow}  \\ 
       C \ar[dr]_g & & B \ar[dl]^f \\ 
       & A}
  \end{equation}
  over our diagram there exists a map $a\colon X \to f \comma g$ which factors $b\colon X\to B$ and $c\colon X\to C$ through $p_0\colon f\comma g\to B$ and $p_1\colon f\comma g\to C$ respectively and which whiskers with the 2-cell $\psi\colon fp_0\Rightarrow gp_1$ to give the 2-cell $\alpha\colon fb\Rightarrow gc$; diagrammatically speaking, \emph{1-cell induction} produces a functor $a \colon X \to f \comma g$ from a 2-cell $\alpha \colon fb \To gc$ so that:
  \begin{equation}\label{eq:comma-ind-1cell-prop}
    \vcenter{\xymatrix@=10pt{
       & X \ar[dl]_{c} \ar[dr]^{b} \ar@{}[dd]|(.4){\alpha}|{\Leftarrow}  \\ 
       C \ar[dr]_g & & B \ar[dl]^f \\ 
       & A
    }}
    \mkern20mu = \mkern20mu
    \vcenter{\xymatrix@=10pt{
      & f \downarrow g \ar[dl]_{p_1} \ar[dr]^{p_0} \ar@{}[dd]|(.4){\psi}|{\Leftarrow}  \\ 
      C \ar[dr]_g & & B \ar[dl]^f \\ 
      & A
      \save "1,2"+<0pt,40pt>*+{X}\ar "1,2" _-a\restore
      }}
  \end{equation}

Fullness of \eqref{eq:weak-comma-prop} tells us that if we are given a pair of functors $a,a'\colon X\to f\comma g$ and a pair of 2-cells
  \begin{equation}\label{eq:comma-ind-2cell-data}
    \vcenter{\xymatrix@=10pt{
      & {X}\ar[dl]_{a'}\ar[dr]^{a}
      \ar@{}[dd]|(.4){\tau_0}|{\Leftarrow} & \\
      {f\comma g}\ar[dr]_{p_0} & & 
      {f\comma g}\ar[dl]^{p_0} \\
      & B &
    }}
    \mkern30mu\text{and}\mkern30mu
    \vcenter{\xymatrix@=10pt{
      & {X}\ar[dl]_{a'}\ar[dr]^{a}
      \ar@{}[dd]|(.4){\tau_1}|{\Leftarrow} & \\
      {f\comma g}\ar[dr]_{p_1} & & 
      {f\comma g}\ar[dl]^{p_1} \\
      & C &
    }}
  \end{equation}
  with the property that 
    \begin{equation}\label{eq:comma-ind-2cell-compat}
      \xymatrix@=10pt{ & X \ar[dl]_{a'} \ar[dr]^a \ar@{}[dd]|(.4){\tau_1}|{\Leftarrow} & &  & \ar@{}[dd]|{\displaystyle =} & &  & X \ar[dl]_{a'} \ar[dr]^a  \ar@{}[dd]|(.4){\tau_0}|{\Leftarrow} \\ f \downarrow g  \ar[dr]_{p_1} & & f \downarrow g \ar[dl]|{p_1} \ar[dr]^{p_0}   \ar@{}[dd]|(.4){\psi}|{\Leftarrow} & & & &   f \downarrow g \ar[dl]_{p_1} \ar[dr]|{p_0}  \ar@{}[dd]|(.4){\psi}|{\Leftarrow}  & & f \downarrow g \ar[dl]^{p_0} \\ & C \ar[dr]_g & & B \ar[dl]^f & & C \ar[dr]_g & & B \ar[dl]^f & &  \\ & & A & &  &  & A}
    \end{equation}
  then there exists a 2-cell $\tau \colon a \Rightarrow a'$, defined by \emph{2-cell induction}, satisfying the equalities 
  \[     \vcenter{\xymatrix@=10pt{
      & {X}\ar[dl]_{a'}\ar[dr]^{a}
      \ar@{}[dd]|(.4){\tau_0}|{\Leftarrow} & \\
      {f\comma g}\ar[dr]_{p_0} & & 
      {f\comma g}\ar[dl]^{p_0} \\
      & B &
    }}    \mkern20mu = \mkern20mu
    \vcenter{\xymatrix@=30pt{ X \ar@/^2ex/[d]^a \ar@/_2ex/[d]_{a'} \ar@{}[d]|(.4){\tau}|{\Leftarrow}  \\ f \downarrow g \ar[d]^{p_0} \\ B}}
      \mkern30mu\text{and}\mkern30mu
    \vcenter{\xymatrix@=10pt{
      & {X}\ar[dl]_{a'}\ar[dr]^{a}
      \ar@{}[dd]|(.4){\tau_1}|{\Leftarrow} & \\
      {f\comma g}\ar[dr]_{p_1} & & 
      {f\comma g}\ar[dl]^{p_1} \\
      & C &
    }}     \mkern20mu = \mkern20mu
        \vcenter{\xymatrix@=30pt{ X \ar@/^2ex/[d]^a \ar@/_2ex/[d]_{a'} \ar@{}[d]|(.4){\tau}|{\Leftarrow}  \\ f \downarrow g \ar[d]^{p_1} \\ C}}.    \]

    Finally, conservativity of \eqref{eq:weak-comma-prop} tells us that if we are given a 2-cell $\tau\colon a\Rightarrow a' \colon X \to f \comma g$ then if the whiskered composites $p_0\tau$ and $p_1\tau$, as shown in the previous diagram, are isomorphisms in $\hom'(X,B)$ and $\hom'(X,C)$ respectively, then $\tau$ is also an isomorphism in $\hom'(X,f\comma g)$; this is \emph{2-cell conservativity}.
 \end{obs}

\begin{lem}[1-cell induction is unique up to isomorphism]\label{lem:1cell-ind-uniqueness}
Any two 1-cells $a,a' \colon X \to f \comma g$ over a weak comma object \eqref{eq:standard-comma-pic} that are induced by the same comma cone $\alpha \colon fb \To gc$ are isomorphic over $C \times B$. 
\end{lem}
\begin{proof}
This follows from Lemma~\ref{lem:smothering}, which demonstrates that fibres of smothering functors are connected groupoids, or can be proven directly. From the defining property of induced 1-cells displayed in~\eqref{eq:comma-ind-1cell-prop} it follows that $p_0 a = p_0 a'$, $p_1 a = p_1 a'$, and $\psi a = \psi a'$. We can regard the first two of these equalities as being identity 2-cells of the form displayed in~\eqref{eq:comma-ind-2cell-data}. Then the third of these equalities may be re-interpreted as positing the compatibility property displayed in~\eqref{eq:comma-ind-2cell-compat} for those identity 2-cells. So we may apply the 2-cell induction property of $f\comma g$ to obtain a 2-cell $\tau\colon a\Rightarrow a'$ whose whiskered composites with $p_0$ and $p_1$ are the identity 2-cells corresponding to the equalities $p_0 a = p_0 a'$ and $p_1 a = p_1 a'$ respectively. This then allows us to apply the 2-cell conservativity property of our weak comma object to show that $\tau\colon a\Rightarrow a'$ is an isomorphism.
 \end{proof}

\subsection{Slices of the category of quasi-categories}\label{subsec:slice-2cats-of-qcats}

\begin{defn}[enriching the slices of $\qCat$]\label{defn:enriched-slice}
  For a quasi-category $A$, we will write $\qCat/A$ for the full subcategory of the usual slice category whose objects are isofibrations $E\tfib A$. Where not otherwise stated, we shall restrict our attention to these subcategories of isofibrations: these are the subcategories of fibrant objects in slices of Joyal's model structure and so are better behaved when viewed from the perspective of formal quasi-category theory than the slice categories of all maps with fixed codomain.
   
   The category $\qCat/A$ has two enrichments of interest to us here. Let $\qCat_2\slice A$ and $\qCat_\infty\slice A$ denote the 2-category and simplicial category (respectively) whose objects are the isofibrations with codomain $A$ and whose hom-category and simplicial hom-space (respectively) between $p \colon E \tfib A$ and $q \colon F \tfib A$ are defined by the pullbacks 
   \begin{equation}\label{eq:slice-hom-objects}  
    \xymatrix{
      \hom'_A(p,q) \pbexcursion \ar[d] \ar[r] & \hom'(E,F) \ar[d]^-{\hom'(E,q)} &  &   \hom_A(p,q) \pbexcursion \ar[r] \ar[d] & F^E \ar@{->>}[d]^-{q^E} \\ 
       \catone \ar[r]_-p & \hom'(E,A) & &     \Delta^0 \ar[r]_-p & A^E}
   \end{equation}
  The objects of $\hom'_A(p,q)$ and the vertices of $\hom_A(p,q)$ are exactly the morphisms from $p$ to $q$ in $\qCat/A$. The morphisms in $\hom'_A(p,q)$, 2-cells in the 2-category $\qCat_2\slice A$, are natural transformations between functors $E \to F$ in $\qCat_2$ whose whiskered composite with $q$ is the identity 2-cell on $p$. Since $q\colon F\tfib A$ is an isofibration we know that $q^E\colon F^E\tfib A^E$ is also an isofibration as is its pullback $\hom_A(p,q)\tfib\Del^0$; hence, $\hom_A(p,q)$ is a quasi-category. In other words, $\qCat_\infty\slice A$ is enriched in quasi-categories.
\end{defn}

\begin{obs}[pushforward]\label{obs:fibred-pushforward} If $f \colon B \tfib A$ is an isofibration of quasi-categories then post-composition defines a simplicial functor $f_* \colon \qCat_\infty\slice B \to \qCat_\infty\slice A$ and a 2-functor $f_* \colon \qCat_2\slice B \to \qCat_2\slice A$. 
\end{obs}

One reason for our particular interest in the simplicial categories $\qCat_\infty\slice A$ has to do with the following observation.  Simplicially enriched limits are defined up to isomorphism and thus assemble into a simplicial functor. The universal property defining weak 2-limits, however, lacks a uniqueness statement of sufficient strength to make them assemble into a (strict) 2-functor. In particular:

\begin{obs}[pullback]\label{obs:fibred-pullback}
Consider any functor $f \colon B \to A$ between quasi-categories. Pullback along $f$ defines a functor $f^* \colon \qCat/A \to \qCat/B$, but it cannot be extended to a 2-functor between slice 2-categories $\qCat_2\slice A$ and $\qCat_2\slice B$ in any canonical way. On the other hand, pullback is a genuine simplicial limit in $\qCat_\infty$ and so it does define a simplicial functor $f^* \colon \qCat_\infty\slice A \to \qCat_\infty\slice B$, which in turn gives rise to a 2-functor  $f^* \colon \ho_*(\qCat_\infty\slice A) \to \ho_*(\qCat_\infty\slice B)$ on application of $\ho_*\colon\eCat{\sSet}\to\twoCat$.  The remarks apply equally to the larger slice categories of all maps with fixed codomain.
\end{obs}




\begin{obs}[comparing the 2-categories $\qCat_2\slice A$ and $\ho_*(\qCat_\infty\slice A)$]
The 2-categories $\qCat_2\slice A$ to $\ho_*(\qCat_\infty\slice A)$ have the same 0-cells and 1-cells; however it is not the case that their 2-cells coincide. If we are given a parallel pair of 1-cells \[ \xymatrix@=1.5em{ E \ar@{->>}[dr]_p \ar@/^1ex/[rr]^f \ar@/_1ex/[rr]_g & & F \ar@{->>}[dl]^q \\ & A}\] a 2-cell from $f$ to $g$ in
  \begin{description}
    \item[$\qCat_2\slice A$] is a homotopy class of 1-simplices $f \to g$ in $F^E$ that whisker with $q$ to the homotopy class of the degenerate 1-simplex on $p$.
    \item[$\ho_*(\qCat_\infty\slice A)$] is a homotopy class represented by a 1-simplex $f \to g$ in the fibre of $q^E\colon F^E\tfib A^E$ over the vertex $p\in A^E$ under homotopies which are also constrained to that fibre.
  \end{description}
  Note here that the notion of homotopy involved in the description of 2-cells in $\ho_*(\qCat_\infty\slice A)$ is more refined (identifies fewer simplices) than that given for 2-cells in $\qCat_2\slice A$. Each homotopy class representing a 2-cell in $\qCat_2\slice A$ may actually split into a number of distinct homotopy classes representing 2-cells in $\ho_*(\qCat_\infty\slice A)$.
\end{obs}



  Consequently, it is not the case that these two enrichments of $\qCat\slice A$ to a 2-category are identical. However, they are related by a 2-functor whose properties we now enumerate.

\begin{defn}[smothering 2-functor]\label{defn:smothering-2-functor}
  A 2-functor $F\colon\tcat{C}\to\tcat{D}$ is said to be a {\em smothering 2-functor\/} if it is surjective on 0-cells and \emph{locally smothering}, i.e., if for all 0-cells $K$ and $K'$ in $\tcat{C}$ the action $F\colon \tcat{C}(K,K')\to\tcat{D}(FK,FK')$ of $F$ on the hom-category from $K$ to $K'$ is a smothering functor.

Note that smothering 2-functors are also conservative at the level of 1-cells in the sense appropriate to 2-category theory; that is to say if $k\colon K\to K'$ is a 1-cell in $\tcat{C}$ for which $Fk$ is an equivalence in $\tcat{D}$ then $k$ is an equivalence in $\tcat{C}$.
\end{defn}

\begin{prop}\label{prop:slice-smothering-2-functor} There exists a canonical 2-functor $\ho_*(\qCat_\infty\slice A)\to\qCat_2\slice A$  which acts identically on 0-cells and 1-cells and is a smothering 2-functor.
\end{prop}
\begin{proof}
  To construct the required 2-functor, apply the homotopy category functor $\ho$ to the defining pullback square for $\hom_A(p,q)$ in~\eqref{eq:slice-hom-objects} to obtain a square which then induces a functor $h(\hom_A(p,q))\to \hom'_A(p,q)$ by the pullback property of the defining square for $\hom'_A(p,q)$. It is a routine matter now to check that we may assemble these actions on hom-categories together to give a 2-functor which acts as the identity on the common underlying category $\qCat\slice A$ of these 2-categories. 

  To show that this 2-functor is smothering, we already know that it acts bijectively on 0-cells, so all that remains is to show that each $h(\hom_A(p,q))\to \hom'_A(p,q)$ is a smothering functor. This fact follows by direct application of Proposition~\ref{prop:weak-homotopy-pullbacks} to the defining pullbacks~\eqref{eq:slice-hom-objects}.
 \end{proof}

Our next aim is to develop a useful principle by which to recognise those 1-cells of $\ho_*(\qCat_\infty\slice A)$ which are equivalences in there. To achieve this, we must first explore the 2-categorical properties of the isofibrations between quasi-categories.

 \begin{defn}[representably defined isofibrations in 2-categories]\label{defn:representable-isofibrations}
  A 1-cell $p\colon B\to A$ in a 2-category $\tcat{C}$ is said to be a {\em representably defined isofibration\/} (or just an \emph{isofibration}) if and only if for each object $X\in\tcat{C}$ the functor $\tcat{C}(X,p)\colon\tcat{C}(X,B)\to\tcat{C}(X,A)$ is an isofibration of categories (has the right lifting property with respect to the inclusion $\catone\inc\iso$).  In more explicit terms, this means that for any diagram \[ \xymatrix{ \ar@{}[dr]|(.7){\alpha\cong} & B \ar[d]^p  & \ar@{}[d]|{\displaystyle\rightsquigarrow} &  \ar@{}[dr]|{\beta\cong} & B \ar[d]^p \\ X \ar[ur]^b \ar[r]_a & A &&  X \ar@/^1.5ex/[ur]^b \ar@/_1.5ex/[ur]_*!<-3pt,+3pt>{\labelstyle x} \ar[r]_a & A}\] consisting of 1-cells $a$ and $b$ and a 2-isomorphism $\alpha \colon pb \cong a$, there exists a 1-cell $x$ and 2-isomorphism $\beta \colon b \cong x$ so that $p \beta = \alpha$ and $px = a$.
\end{defn}

\begin{lem}\label{lem:representable-isofibration} If $p \colon B \tfib A$ is an isofibration between quasi-categories, then $p$ is a representably defined isofibration in $\qCat_2$.
\end{lem}
\begin{proof} 
For any simplicial set $X$,   $p^X\colon B^X\tfib A^X$ is also an isofibration  and in particular has the right lifting property with respect to $\catone\inc\iso$. Using the standard homotopy coherence result, recalled in \ref{rec:qmc-quasi-marked}, that an isomorphism in the homotopy category of a quasi-category can be extended to a functor with domain $\iso$, it follows that $\hom'(X,p)\colon\hom'(X,B)\to\hom'(X,A)$ also has the right lifting property with respect to $\catone\inc\iso$. Thus $\hom'(X,p)$ is 
  an isofibration of categories, which shows that the isofibrations of quasi-categories are representably defined  in the 2-category $\qCat_2$.
  \end{proof}

    The following lemma, stated here in the special case of $\qCat_2$, applies equally to any slice 2-category whose objects are isofibrations.

  \begin{lem}\label{lem:proj-is-1-conservative}
     The canonical projection 2-functor $\qCat_2\slice A\to \qCat_2$ is conservative on 1-cells in the appropriate 2-categorical sense: if
\begin{equation}\label{eq:equiv-to-lift}
  \xymatrix@=1.5em{
    {E}\ar[rr]^w\ar@{->>}[dr]_p && {F}\ar@{->>}[dl]^q \\
    & {A} &
  }
\end{equation}
is a 1-cell in $\qCat_2\slice A$ for which $w\colon E\to F$ admits an equivalence inverse $w' \colon F \to E$ in $\qCat_2$, then $w$  is  an equivalence in the slice 2-category $\qCat_2\slice A$.
  \end{lem}
  \begin{proof}
By a standard 2-categorical argument, we may choose  2-isomorphisms $\alpha\colon w'w\cong\id_E$ and $\beta\colon\id_F\cong ww'$ which display $w'$ as a left adjoint equivalence inverse to $w$ in $\qCat_2$. As $p$ is an isofibration in $\qCat_2$, the isomorphism $q\beta\colon q\cong qww' = pw'$ can be lifted along $p$ to give a 1-cell $\bar{w}\colon F\to E$ with $p\bar{w} = q$ and a 2-isomorphism $\gamma\colon \bar{w}\cong w'$ with $p\gamma=q\beta$. The first of these equations tells us that $\bar{w}$ is a 1-cell in $\qCat_2\slice A$.  Using the second of these equations and the triangle identities relating $\alpha$ and $\beta$, we see that the isomorphisms $\alpha\cdot\gamma w\colon \bar{w} w\cong\id_E$ and $w\gamma^{-1}\cdot\beta\colon\id_F\cong w\bar{w}$ are 2-cells in $\qCat_2\slice A$: \[ p(\alpha \cdot \gamma w) = p\alpha \cdot p\gamma w = qw\alpha \cdot q\beta w = q\id_w \qquad q(w\gamma^{-1}\cdot \beta) = qw\gamma^{-1} \cdot q\beta = p\gamma^{-1} \cdot q\beta = \id_p. \]
These isomorphisms display $\bar{w}$ as an equivalence inverse to $w$ in $\qCat_2\slice A$.
  \end{proof}

  \begin{cor}\label{cor:recog-fibred-equivs}
    The 1-cell depicted in~\eqref{eq:equiv-to-lift} is an equivalence in $\ho_*(\qCat_\infty\slice A)$ if and only if $w\colon E\to F$ is an equivalence in $\qCat_2$. 
  \end{cor}

  \begin{proof}
  By  Proposition~\ref{prop:slice-smothering-2-functor} and Lemma \ref{lem:proj-is-1-conservative}, the canonical 2-functors $\ho_*(\qCat_\infty\slice A)\to\qCat_2\slice A$ and $\qCat_2\slice A\to \qCat_2$ are both conservative on 1-cells, so their composite is also conservative on 1-cells. The result follows immediately.
  \end{proof}
  

  \begin{defn}[fibred equivalence]\label{defn:fibred-equivalence}
  A functor $w \colon E \to F$ between quasi-categories equipped with specified isofibrations $p\colon E \tfib A$ and $q \colon F \tfib A$ is an {\em equivalence fibred over $A$\/}, or just a {\em fibred equivalence}, if it is an equivalence in $\ho_*(\qCat_\infty\slice A)$. By Corollary \ref{cor:recog-fibred-equivs}, any equivalence in $\qCat_2$ which commutes with the maps down to $A$ is a fibred equivalence. Unpacking the definition, a fibred equivalence admits an equivalence inverse $w' \colon F \to E$ over $A$ together with isomorphisms $\alpha \colon w' w \cong \id_E \in E^E$ and $\beta \colon \id_F \cong w w' \in F^F$ represented by 1-simplices that compose with $p$ and $q$ to degenerate 1-simplices.
    \end{defn}

Corollary \ref{cor:recog-fibred-equivs} allows us to lift equivalences in $\qCat_2/A$ to fibred equivalences, which can be pulled back along a functor $f \colon B \to A$ as described in Observation \ref{obs:fibred-pullback}. The lifting arguments developed here relied upon the assumption that the simplicial categories in which we work have hom-spaces which are quasi-categories, which is why our default is to assume that the objects of our slice categories $\qCat_2\slice A$ and $\qCat_\infty\slice A$ are isofibrations.
  
\subsection{A strongly universal characterisation of weak comma objects}

We may use properties of the 2-categorical slice $\qCat_2\slice (C \times B)$ to characterise the weak comma objects of $\qCat_2$ in terms of a \emph{strict} 1-categorical universal property. We present this technical result here and then use it to good effect in section~\ref{sec:limits}, where we demonstrate how to characterise limits and colimits that exist in a quasi-category in purely 2-categorical terms.

For this subsection we shall assume, contrary to our notational convention elsewhere, that $\qCat_2\slice(C\times B)$ denotes the unrestricted slice 2-category whose objects are all functors with codomain $C\times B$.

\begin{obs}[uniqueness of 1-cell induction revisited]\label{obs:1cell-ind-uniqueness-reloaded}
Any 1-cell $a\colon X\to f\comma g$ induced by the comma cone~\eqref{eq:comma-cone} may be regarded as a 1-cell
  \begin{equation*}
    \xymatrix@=1em{
      {X}\ar[dr]_(0.3){(c,b)}\ar[rr]^{a}
      && *+!L(0.5){f\comma g}\ar[dl]^(0.3){(p_1,p_0)} \\
      & {C\times B}&
    }
  \end{equation*}
  in $\qCat_2\slice(C\times B)$. If we are given a second 1-cell $a'\colon X\to f\comma g$ which is also induced by the same comma cone then the argument of Lemma~\ref{lem:1cell-ind-uniqueness} delivers us a 2-cell
  \begin{equation}\label{eq:induced-1-cell-comparison}
    \xymatrix@=1.5em{
      {X}\ar[dr]_(0.3){(c,b)} 
      \ar@/^1.5ex/[rr]^{a}_{}="one" \ar@/_1.5ex/[rr]_{a'}^{}="two" \ar@{=>}"one";"two"^{\tau}
      && *+!L(0.5){f\comma g}\ar[dl]^(0.3){(p_1,p_0)} \\
      & {C\times B}&
    }
  \end{equation}
in $\qCat_2\slice(C\times B)$, which is moreover an isomorphism; this is what we meant by the assertion that any pair of functors defined by 1-cell induction over the same comma cone are isomorphic over $C \times B$.
 Conversely, by 2-cell conservativity of the comma quasi-category $f \downarrow g$, any 2-cell of $\qCat_2\slice(C\times B)$ of the form depicted in~\eqref{eq:induced-1-cell-comparison} is an isomorphism. Thus, the hom-category $\hom'_{C\times B}((c,b),(p_1,p_0))$ is a groupoid, whose connected components comprise those 1-cells induced by a common cone \eqref{eq:comma-cone}.
\end{obs}

\begin{obs}\label{obs:squares-set}
  For each object $(c,b)\colon X\to C\times B$ of $\qCat/(C\times B)$ we have a set $\sq_{g,f}(c,b)$ of 2-cells as depicted in~\eqref{eq:comma-cone}. This construction may be extended immediately to a contravariant functor $\sq_{g,f}\colon(\qCat/(C\times B))\op\to\Set$, which carries a morphism
  \begin{equation*}
    \xymatrix@=1em{
      {X}\ar[dr]_(0.3){(c,b)}\ar[rr]^{u}
      && *+!L(0.5){Y}\ar[dl]^(0.3){(\bar{c},\bar{b})} \\
      & {C\times B}&
    }
  \end{equation*}
  of $\qCat/(C\times B)$ to the function $\sq_{g,f}(u)$ which maps a 2-cell $\beta$ of $\sq_{g,f}(\bar{c},\bar{b})$ to the whiskered 2-cell $\beta u$ in $\sq_{g,f}(c,b)$. 


  \end{obs}

\begin{obs}\label{obs:groupoid-components}
There is a product-preserving  functor $\pi^g_0\colon\Cat\to\Set$ that sends a category to  the set of connected components of its sub-groupoid of isomorphisms. We may apply $\pi^g_0$ to the hom-categories of a 2-category $\tcat{C}$ to construct a category $(\pi^g_0)_*\tcat{C}$.  Any isomorphism $K\cong L$ in the category $(\pi^g_0)_*\tcat{C}$ can be lifted to a corresponding equivalence in $\tcat{C}$ by picking representatives $w\colon K\to L$ and $w'\colon L\to K$ in $\tcat{C}$ for the isomorphism and its inverse. The 2-isomorphisms $\alpha\colon w'w\cong\id_K$ and $\beta\colon ww'\cong\id_L$ which witness these as equivalence inverses in $\tcat{C}$ arise by choosing 2-cells which witness the mutual inverse identities $w'w=\id_K$ and $ww'=\id_L$ in $(\pi^g_0)_*\tcat{C}$.
\end{obs}


\begin{lem}\label{lem:sq-as-a-functor}
The functor $\sq_{g,f}$ factorises through the quotient functor $\qCat/(C\times B)\to(\pi^g_0)_*(\qCat_2\slice(C\times B))$ to define a functor
\begin{equation}\label{eq:the-real-sq-functor}
    \sq_{g,f}\colon(\pi^g_0)_*(\qCat_2\slice(C\times B))\op\longrightarrow\Set.
    \end{equation}
\end{lem}
\begin{proof}
If we are given a 2-cell
  \begin{equation*}
    \xymatrix@=1.5em{
      {X}\ar[dr]_(0.3){(c,b)} 
      \ar@/^1.5ex/[rr]^{u}_{}="one" \ar@/_1.5ex/[rr]_{u'}^{}="two" \ar@{=>}"one";"two"^{\tau}
      && *+!L(0.5){Y}\ar[dl]^(0.3){(\bar{c},\bar{b})} \\
      & {C\times B}&
    }
  \end{equation*}
  in $\qCat_2\slice(C\times B)$ and a 2-cell $\beta \in \sq_{g,f}(\bar{c},\bar{b})$ then the middle four interchange rule for the horizontal composite of the 2-cells $\beta$ and $\tau$ provides us with a commutative square
   \begin{equation*}
    \xymatrix@=1.5em{
      {f\bar{b}u} \ar@{=>}[r]^{\beta u}\ar@{=>}[d]_{f\bar{b}\tau} & 
      {g\bar{c}u}\ar@{=>}[d]^{g\bar{c}\tau} \\
      {f\bar{b}u'} \ar@{=>}[r]_{\beta u'} & {g\bar{c}u'}
    }
  \end{equation*} 
  whose vertical arrows are the identities on $fb$ and $gc$ respectively. Hence, $\beta u = \beta u'$, and we conclude that if $u$ and $u'$ are 1-cells in the same connected component of the category $\hom'_{C \times B}((c,b),(\bar{c},\bar{b}))$ then the functions $\sq_{g,f}(u)$ and $\sq_{g,f}(u')$ are identical.
\end{proof}

  This functor allows us to expose another aspect of the weak 2-universal property of weak comma objects: namely that the comma cone formed from the cospan $B \xrightarrow{f} A \xleftarrow{g} C$ represents the functor \eqref{eq:the-real-sq-functor}.

\begin{lem}\label{lem:cpts-and-comma-2-cells}
  The weakly universal comma cone 
  \begin{equation}\label{eq:first-comma-cone}
    \xymatrix@=10pt{
      & f \downarrow g \ar[dl]_{p_1} \ar[dr]^{p_0} \ar@{}[dd]|(.4){\psi}|{\Leftarrow}  \\ 
      C \ar[dr]_g & & B \ar[dl]^f \\ 
      & A}
  \end{equation}
  provides us with an element $\psi\in\sq_{g,f}(p_1,p_0)$ which is universal, in the usual sense, for the functor $\sq_{g,f}\colon(\pi^g_0)_*(\qCat_2\slice(C\times B))\op\to\Set$. Furthermore, any comma cone
   \begin{equation}\label{eq:other-comma-cone}
    \xymatrix@=10pt{
      & Q \ar[dl]_{q_1} \ar[dr]^{q_0} \ar@{}[dd]|(.4){\phi}|{\Leftarrow}  \\ 
      C \ar[dr]_g & & B \ar[dl]^f \\ 
      & A}
  \end{equation} 
  for which the 2-cell $\phi\in\sq_{g,f}(q_1,q_0)$ is a universal element of the functor $\sq_{g,f}$ displays $Q$ as a weak comma object in $\qCat_2$.
\end{lem}

\begin{proof}
  For each object $(c,b)\colon X\to C\times B$ of $\qCat_2\slice(C\times B)$ the element $\psi\in\sq_{g,f}(p_1,p_0)$ induces a function
  \begin{equation*}
\pi^g_0(\hom'_{C\times B}((c,b),(p_1,p_0)))\longrightarrow \sq_{g,f}(c,b)
  \end{equation*}
  which carries a functor $a\colon X\to f\comma g$ representing an element of the set on the left to the whiskered composite $\psi a$ on the right. The element $\psi\in\sq_{g,f}(p_1,p_0)$ is universal for $\sq_{g,f}$ if and only if each of those functions is a bijection. Surjectivity follows directly from the 1-cell induction property of $f\comma g$, and injectivity follows from the reformulation of Lemma \ref{lem:smothering} discussed in Observation~\ref{obs:1cell-ind-uniqueness-reloaded}. 

If $\phi\in\sq_{g,f}(q_1,q_0)$ is another element which is universal for $\sq_{g,f}$, then by Yoneda's lemma the objects $(p_1,p_0)\colon f\comma g\to C\times B$ and $(q_1,q_0)\colon Q\to C\times B$ are isomorphic in the category $(\pi^g_0)_*(\qCat_2\slice(C\times B))$ via an isomorphism whose action under $\sq_{g,f}$ carries $\psi\in\sq_{g,f}(p_1,p_0)$ to $\phi\in\sq_{g,f}(q_1,q_0)$. Proceeding as in Observation \ref{obs:groupoid-components}, we may pick representatives of this isomorphism and its inverse to provide a pair of 1-cells
\begin{equation*}
  \xymatrix@=1.5em{
    {Q} \ar@/_1ex/[rr]_w 
    \ar[dr]_(0.4){(q_1,q_0)} &&
    *+!L(0.5){f\comma g} \ar@/_1ex/[ll]_{w'}
    \ar[dl]^(0.4){(p_1,p_0)}  \\
    & {C\times B}
  }
\end{equation*}
which are related by a pair of 2-isomorphisms $\alpha\colon w'w\cong \id_{Q}$ and $\beta\colon ww'\cong\id_{f\comma g}$ in the slice 2-category $\qCat_2\slice(C\times B)$. The fact that this isomorphism carries $\phi$ to $\psi$ under the action of $\sq_{g,f}$ provides the 2-cellular equations $\psi w = \phi$ and $\phi w' = \psi$. 

  To prove the 1-cell induction property for the comma cone~\eqref{eq:other-comma-cone} suppose that we are given a comma cone~\eqref{eq:comma-cone}. The 1-cell induction property of $f\comma g$ provides us with a 1-cell $a\colon X\to f\comma g$ with the defining property that $p_0 a = b$, $p_1 a = c$, and $\psi a = \alpha$. The functor $w'\colon f\comma g\to Q$ satisfies the equations $q_0 w' = p_0$, $q_1 w' = p_1$, and $\phi w' = \psi$, so we have $q_0 w' a = p_0 a = b$, $q_1 w' a = p_1 a = c$, and $\phi w' a = \psi a = \alpha$. This demonstrates that $w' a\colon X\to Q$ is a 1-cell induced by the comma cone~\eqref{eq:comma-cone} with respect to the comma cone~\eqref{eq:other-comma-cone}.

  To prove the 2-cell induction property for the comma cone~\eqref{eq:other-comma-cone} suppose that we are given a pair of 1-cells $a,a'\colon X\to Q$ and a pair of 2-cells $\tau_0 \colon q_0 a \Rightarrow q_0 a'$ and $\tau_1 \colon q_1 a \Rightarrow q_1 a'$ satisfying the condition given in~\eqref{eq:comma-ind-2cell-compat} with respect to the comma cone~\eqref{eq:other-comma-cone}. The 1-cells $w a, w a'\colon X\to f\comma g$ and the 2-cells $\tau_0 \colon p_0 w a = q_0 a \Rightarrow q_0 a' = p_0 w a'$ and $\tau_1 \colon p_1 w a = q_1 a \Rightarrow q_1 a' = p_1 w a'$ also satisfy the condition given in~\eqref{eq:comma-ind-2cell-compat} with respect to the comma cone~\eqref{eq:first-comma-cone}. Hence, the 2-cell induction property of $f\comma g$ ensures that we have a 2-cell $\mu\colon w a \Rightarrow w a'$ with the defining properties that $p_0 \mu = \tau_0$ and $p_1\mu = \tau_1$. Combining this with the invertible 2-cell $\alpha\colon w'w\cong\id_Q$, we may construct a 2-cell 
  \begin{equation*}
    \xymatrix@C=4em{
      {\tau \defeq a} \ar@{=>}[r]_-{\cong}^-{\alpha^{-1} a} &
      {w'wa} \ar@{=>}[r]^{w'\mu} &
      {w'wa'} \ar@{=>}[r]_-{\cong}^-{\alpha a'} & {a'}
    }
  \end{equation*}
Because $\alpha$ is a 2-cell in the endo-hom-category in $\qCat_2\slice(C\times B)$ on the object $(q_1,q_0)\colon Q\to C\times B$, $q_0\alpha = \id_{q_0}$ and $q_1\alpha = \id_{q_1}$. It follows that $q_0 \tau = q_0 w' \mu = p_0 \mu = \tau_0$ and $q_1 \tau = q_1 w' \mu = p_1 \mu = \tau_1$, 
  which demonstrates that $\tau\colon a\Rightarrow a'$ satisfies the defining properties required of a 2-cell induced by the pair of 2-cells $\tau_0$ and $\tau_1$.

  The proof of 2-cell conservativity is of a similar ilk and is left to the reader.
\end{proof}



%!TEX root = all.tex
% ******************************************************************
% ** Title:            The 2-category theory of quasi-categories
% **                   adjunctions
% ** Precis:        
% ** Author:           Emily Riehl and Dominic Verity
% ** Commenced:        2/3/2012
% ******************************************************************

\section{Adjunctions of quasi-categories}\label{sec:qcatadj}

We begin our 2-categorical development of quasi-category theory by introducing the appropriate notion of adjunction, following Joyal. As observed in \cite{kelly.street:2} and elsewhere, adjunctions can be defined internally to any 2-category and the proofs of many of their familiar properties can be internalised similarly.

\setcounter{thm}{0}
\begin{defn}[adjunction]\label{defn:adjunction}
An {\em adjunction\/}  \[ \adjdisplay f-| u : A ->B .\] in a 2-category consists of objects $A,B$; 1-cells $f \colon B \to A$, $u \colon A \to B$; and {\em unit\/} and {\em counit\/} 2-cells $\eta \colon \id_B \Rightarrow uf$, $\epsilon \colon fu \Rightarrow \id_A$ satisfying the triangle identities.
\[\xymatrix@=1.5em{ & B \ar[dr]^f \ar@{}[d]|(.6){\Downarrow\epsilon} \ar@{=}[rr] &  \ar@{}[d]|(.4){\Downarrow\eta} & B \ar@{}[d]^*+{=} & B &&   B \ar@{=}[rr] \ar[dr]_f & \ar@{}[d]|(.4){\Downarrow \eta} & B \ar[dr]^f \ar@{}[d]|(.6){\Downarrow\epsilon} & {\mkern40mu}\ar@{}[d]^*+{=} &  B \ar@/^2ex/[d]^f \ar@/_2ex/[d]_f \ar@{}[d]|(.4){\id_f}|(.6){=} & \\A \ar[ur]^u \ar@{=}[rr] & &  A \ar[ur]_u & {\mkern40mu} & A \ar@/^2ex/[u]^u \ar@/_2ex/[u]_u \ar@{}[u]|(.4){=}|(.6){\id_u}  && &A \ar[ur]_u \ar@{=}[rr] & & A & A }\]
\end{defn}

In particular, an \emph{adjunction between quasi-categories} is an adjunction in the 2-category $\qCat_2$.  As always we identify the unit and counit 2-cells with the simplicial maps \[ \xymatrix@=1.5em{ B \ar[d]_{i_0} \ar@{=}[dr] & & &  A \ar[r]^u \ar[d]_{i_0} & B \ar[d]^f  \\ B \times \Del^1 \ar[r]_-{\eta} & B & \mathrm{and} &  A \times \Del^1 \ar[r]^-{\epsilon} & A \\ B \ar[u]^{i_1} \ar[r]_f & A \ar[u]_u & &  A \ar[u]^{i_1} \ar@{=}[ur]}\] (1-simplices in $B^B$ and $A^A$ respectively) representing the unit and counit respectively. Because $B^A$ and $A^B$ are quasi-categories we know, from the description of the homotopy category of a quasi-category given in Recollection~\ref{rec:hty-category}, that for any choice of representatives of the unit and counit there exist maps \[ \alpha \colon A \times \Del^2 \to B \qquad \mathrm{and} \qquad \beta \colon B \times \Del^2 \to A\] (2-simplices in $B^A$ and $A^B$ respectively) which witness the triangle identities in the  sense that their boundaries have the form 
 \[ \xymatrix@=1em{ & ufu \ar@{}[d]|(.6){\alpha} \ar[dr]^{u\epsilon} & & & fuf \ar@{}[d]|(.6){\beta} \ar[dr]^{\epsilon f} \\ u \ar[ur]^{\eta u} \ar[rr]_{\id_u} & & u & f \ar[ur]^{f\eta} \ar[rr]_{\id_f} & & f}\]

\begin{ex}
On account of the fully-faithful inclusion $\Cat_2 \inc \qCat_2$, any adjunction of categories gives rise to an adjunction of quasi-categories with canonical representatives for the unit and counit. Conversely, the 2-functor $\ho \colon \qCat_2 \to \Cat_2$ carries any adjunction of quasi-categories to an adjunction between their respective homotopy categories.
\end{ex}

\begin{ex} The homotopy coherent nerve, introduced in \cite{Cordier:1982:HtyCoh} and studied in \cite{Cordier:1986:HtyCoh}, defines a 2-functor from the 2-category of topologically enriched categories, continuous functors, and enriched natural transformations to $\qCat_2$. This 2-functor factors through the 2-category of locally Kan simplicial categories, simplicial functors, and simplicial natural transformations; the locally Kan simplicial categories are the cofibrant objects in Berger's model structure \cite{Bergner:2007fk}.  Hence, any enriched adjunction between topological or fibrant simplicial categories gives rise to an adjunction of quasi-categories by passing to homotopy coherent nerves. As in the unenriched case, there exist canonical representatives for the unit and counit defined by applying the homotopy coherent nerve to the corresponding enriched natural transformations.
\end{ex}

\begin{ex}\label{ex:simp.quillen.adj}
Any simplicially enriched Quillen adjunction between simplicial model categories descends to an adjunction between the associated quasi-categories, constructed by restricting to the fibrant-cofibrant objects and then applying the homotopy coherent nerve. This restriction is necessary to define the quasi-category associated to a simplicial model category; the homotopy coherent nerve of a simplicial category might not be a quasi-category if the simplicial category is not locally Kan. The subcategory of fibrant-cofibrant objects of a simplicial model category is locally Kan, and furthermore the hom-space bifunctor preserves weak equivalences in both variables; it is common to say that only between fibrant-cofibrant objects are the simplicial hom-spaces guaranteed to have the ``correct'' homotopy type. 

In contrast with the topological case, some care is required to define the functors constituting the adjunction; the point-set level functors will not do because neither adjoint need land directly in the fibrant-cofibrant objects. We prove that  a simplicial Quillen adjunction descends to an adjunction of quasi-categories in Theorem~\ref{thm:simplicial-Quillen-adjunction}.
\end{ex}

Adjunctions can also be constructed internally to $\qCat_2$ using its weak 2-limits, as we shall see in the next section. Later, we will also meet adjunctions constructions using limits or colimits defined internally to a quasi-category.

\subsection{Right adjoint right inverse adjunctions}\label{subsec:RARI} 

We begin by studying an important class of adjunctions whose counit 2-cells are isomorphisms.

\begin{defn}\label{defn:RARI}
A 1-cell $f \colon B \to A$ in a 2-category admits a \emph{right adjoint right inverse} (abbreviated \emph{RARI}) if it admits a right adjoint $u \colon A \to B$ so that the counit of the adjunction $f \dashv u$ is an isomorphism.
\end{defn}

In the situation of Definition \ref{defn:RARI}, $f$ defines a \emph{left adjoint left inverse} (abbreviated \emph{LALI}) to $u$. When the counit of $f \dashv u$ is an isomorphism, the whiskered composites $f\eta$ and $\eta u$ of the unit must also be isomorphisms. Indeed, to construct an adjunction of this form it suffices to give 2-cells with these properties, as demonstrated by the following 2-categorical lemma. 

\begin{lem}\label{lem:adjunction.from.isos} Suppose we are given a pair of 1-cells $u \colon A \to B$ and $f\colon B\to A$ and a 2-isomorphism $fu \cong \id_A$ in a 2-category. If there exists a 2-cell $\eta' \colon \id_B \Rightarrow uf$ with the property that $f\eta'$ and $\eta' u$ are 2-isomorphisms, then $f$ is left adjoint to $u$. Furthermore, in the special case where $u$ is a section of $f$, then $f$ is left adjoint to $u$ with the counit of the adjunction an identity.
\end{lem}
\begin{proof} 
Let $\epsilon \colon fu \To \id_A$ be the isomorphism, taken to be the identity in the case where $u$ is a section of $f$. We will define an adjunction $f \dashv u$ with counit $\epsilon$ by modifying $\eta' \colon \id_B \To uf$.  The ``triangle identity composite'' $\theta\defeq u\epsilon \cdot \eta' u \colon u \To u$ defines an automorphism of $u$.  Define 
\[ \eta \defeq \xymatrix{ \id_B \ar@{=>}[r]^-{\eta'} & uf \ar@{=>}[r]^-{\theta^{-1}f} & uf.}\] Immediately, $u\epsilon \cdot \eta u = \id_u$, as is verified by the calculation: 
\begin{equation}\label{eq:triangle-calculation-1} \xymatrix@R=1.2em@C=2.5em{ u \ar@{=>}[dr]_\theta \ar@{=>}[r]^-{\eta' u} & ufu \ar@{=>}[r]^{\theta^{-1} fu} \ar@{=>}[d]^{u \epsilon} & ufu \ar@{=>}[d]^{u \epsilon} \\ & u \ar@{=>}[r]_{\theta^{-1}} & u}\end{equation} 

The other triangle identity composite $\phi\defeq \epsilon f \cdot f \eta $ is an isomorphism, as a composite of isomorphisms, and also an idempotent:
\begin{equation}\label{eq:triangle-calculation-2} \xymatrix@R=1.2em@C=2.5em{ f \ar@{=>}[d]_{f \eta} \ar@{=>}[r]^-{f \eta} & fuf \ar@{=}[dr] \ar@{=>}[d]_{f\eta uf} \\ fuf \ar@{=>}[d]_{\epsilon f} \ar@{=>}[r]_-{fuf\eta} & fufuf \ar@{=>}[d]^{\epsilon f} \ar@{=>}[r]_-{fu\epsilon f} & fuf \ar@{=>}[d]^{\epsilon f} \\ f \ar@{=>}[r]_-{f\eta} & fuf \ar@{=>}[r]_-{\epsilon f} & f}\end{equation} But any idempotent isomorphism is an identity: the isomorphism $\phi$ can be cancelled from both sides of the idempotent equation $\phi \cdot \phi = \phi$. Hence, $\epsilon f \cdot f \eta = \id_f$, proving the second triangle identity.
\end{proof}

\begin{rmk}[idempotent isomorphisms]\label{rmk:idempotent-isomorphisms} Because $\qCat_2$ has many weak but few strict 2-limits, it is frequently easier to show that a 2-cell is an isomorphism than to show that it is an identity. When we desire an identity and not merely an isomorphism,  we will make frequent use of the trick that any idempotent isomorphism is an identity.
\end{rmk}

We now show that for any functor $\ell \colon C \to B$, the codomain projection functor $\pi_1 \colon B \comma \ell \to C$  admits a right adjoint right inverse, the ``identity functor'' $i \colon C \to B \comma \ell$ defined below. Here the right adjoint $i$ defines a section to the left adjoint $p_i$. Taking the counit of $i \dashv \pi_1$ to be an identity, as permitted by Lemma \ref{lem:adjunction.from.isos}, the adjunction lifts to the slice 2-category $\qCat_2/C$.


\begin{lem}\label{lem:technicalsliceadjunction}
  Suppose that $\ell\colon C\to B$ is a functor of quasi-categories and let $i\colon C\to B\comma\ell$ be any functor induced by the identity comma cone:
  \begin{equation}\label{eq:technicalsliceadjunction}
    \vcenter{\xymatrix@=1em{
      & {C}\ar@{=}[dl]\ar[dr]^{\ell} & \\
      {C}\ar[rr]_{\ell} && {B}
      \ar@{} "1,2";"2,2" |(0.6){\textstyle =}
    }}
    \mkern20mu = \mkern20mu
    \vcenter{\xymatrix@=1em{
      & {C}\ar[d]^-i \ar@/^/[ddr]^\ell \ar@/_/@{=}[ddl] & \\
      & {B\comma\ell}\ar[dl]|{p_1}\ar[dr]|{p_0} & \\
      {C}\ar[rr]_{\ell} && {B}
      \ar@{} "2,2";"3,2" |(0.6){\Leftarrow\phi}
    }}
  \end{equation}
  Then $i\colon C\to B\comma\ell$ is right adjoint to the codomain projection functor $p_1\colon B\comma\ell\to C$ in the slice 2-category $\qCat_2\slice C$ 
  \[    \xymatrix@=1.2em{
      {C}\ar@{=}[dr]\ar@/_1.5ex/[rr]_-{i}^-{}="one"
      & & *+!L(0.5){B\comma\ell}\ar@{->>}[dl]^-{p_1}
      \ar@/_1.5ex/[ll]_-{p_1}^-{}="two" \\
      & {C} &
      \ar@{}"one";"two"|{\bot}
    }
    \]
 and the counit may be chosen to be an identity 2-cell.
\end{lem}


\begin{proof}
By construction, $i$ is a section to the isofibration $p_1$ and, accordingly, we may take the counit of the postulated adjunction to be the identity $p_1 i = \id_C$. Now a 2-cell $\nu \colon \id_{B\comma \ell} \Rightarrow i p_1$ provides us with a 2-cell in $\qCat_2\slice C$ which satisfies the triangle identities with respect to that counit if and only if $p_1\nu$ and $\nu i$ are identity 2-cells. 

  We construct a suitable 2-cell $\nu\colon \id_{B\comma \ell} \Rightarrow i p_1$ by applying the 2-cell induction property of $B\comma\ell$ to the pair of 2-cells $\phi\colon p_0 \Rightarrow \ell p_1 = p_0 i p_1$ and $\id_{p_1}\colon p_1 = p_1 i p_1$; here, the compatibility condition of~\eqref{eq:comma-ind-2cell-compat} reduces to the trivial pasting identity \[ \vcenter{\xymatrix@=0.7em{ & B\comma \ell \ar@/^2ex/[ddr]^{p_0}   \ar@/^2ex/[ddl]|*+<3pt>{\scriptstyle p_1} \ar@/_2ex/[ddl]_{p_1}  \ar@{}[ddl]|{=} \\ &  \ar@{}[dr]|(.3){\Leftarrow\phi} \\ C \ar[rr]_\ell & & B}} \mkern20mu = \mkern20mu\vcenter{ \xymatrix@=0.7em{ & B\comma \ell \ar@/^2ex/[ddr]^{p_0}  \ar@/_2ex/[ddr]|*+<3pt>{\scriptstyle\ell p_1} \ar@{}[ddr]|{\Leftarrow\phi}  \ar@/_2ex/[ddl]_{p_1}  \\ & \ar@{}[dl]|(0.3){=} & \\ C \ar[rr]_\ell & & B}}  \] By construction, $\nu\colon \id_{B\comma \ell} \Rightarrow i p_1$ is a 2-cell satisfying $p_0\nu = \phi$ and $p_1\nu = \id_{p_1}$.

To show that $\nu i$ is an isomorphism,  observe that $p_0\nu i = \phi i = \id_\ell$ and $p_1\nu i = \id_{p_1} i = \id_{p_1 i} = \id_{\id_C}$, so  using the 2-cell conservativity property of $B\comma\ell$ we conclude that $\nu i$ is an isomorphism.  By Lemma \ref{lem:adjunction.from.isos} this suffices; indeed, applying middle-four interchange to $\nu i \cdot \nu i$ and the equation $p_1\nu=\id_{p_1}$,  $\nu i$ can be seen to be an idempotent isomorphism and thus an identity.
\end{proof}

In general, if a (representable) isofibration $f \colon B \tfib A$ admits a right adjoint right inverse $u$, then the counit of the RARI adjunction may be chosen to be an identity. Lemma \ref{lem:representable-isofibration}, which shows that an isofibration between quasi-categories defines a representable isofibration in $\qCat_2$, will allow us to make frequent use of this ``strictification'' result.

\begin{lem}\label{lem:isofibration-RARI} If $f \colon B \tfib A$ is a representable isofibration in a 2-category $\tcat{C}$ admitting a right adjoint right inverse $u' \colon A \to B$, then there exists a 1-cell $u \colon A \to B$ that is right adjoint right inverse to $f$ with identity counit.
\end{lem}
\begin{proof}
We construct the functor $u\colon A\to B$ and an isomorphism $\beta\colon u' \cong u$ by applying the universal property of the isofibration $f\colon B\tfib A$ to the counit $\epsilon'\colon fu'\cong\id_A$.
\[\xymatrix{ \ar@{}[dr]|(.7){\epsilon'\cong} & B \ar@{->>}[d]^f  & \ar@{}[d]|{\displaystyle\rightsquigarrow} &  \ar@{}[dr]|{\beta\cong} & B \ar@{->>}[d]^f \\ A \ar[ur]^{u'} \ar@{=}[r] & A &&  A \ar@/^1.5ex/[ur]^{u'} \ar@/_1.5ex/[ur]_{u} \ar@{=}[r] & A}\] By construction $fu = \id_A$. The composite
 $\eta \defeq \xymatrix@1{\id_B \ar@{=>}[r]^{\eta'} & u'f \ar@{=>}[r]^{\beta f} & uf}$ of the original unit $\eta'$ with the lifted isomorphism $\beta$ defines a 2-cell that whiskers with $f$ and $u$ to isomorphisms, permitting the application of Lemma \ref{lem:adjunction.from.isos} to conclude.
\end{proof}



\subsection{Terminal objects as adjoint functors}\label{subsec:terminal}


A quasi-category $A$ has a terminal object if and only if the projection functor $! \colon A \to \Del^0$ admits a right adjoint right inverse:

\begin{defn}[terminal objects]\label{defn:terminal}
An object $t$ in a quasi-category $A$ is {\em terminal\/} if there is an adjunction \[ \adjdisplay !-| t :\Del^0 -> A . \]  
 \end{defn}

Dually, of course, an object in $A$ is initial just when it defines a left adjoint left inverse to $! \colon A \to \Delta^0$.


 \begin{ex}[slices have terminal objects]\label{ex:slice-terminal}
For any object $a$  of a quasi-category $A$, there is an adjunction \[\adjdisplay !-| i : \Del^0 -> A\comma a .\] whose right adjoint, defining the terminal object of $A \comma a$, is any vertex of $A \comma a$ that is isomorphic to the degenerate 1-simplex $a\cdot\degen^0\colon a\to a$. This functor whiskers with the comma cone to an identity 2-cell:
\[     \vcenter{\xymatrix@=1em{
      & {\Del^0}\ar@{=}[dl]\ar[dr]^{a} & \\
      {\Del^0}\ar[rr]_{a} && {A}
      \ar@{} "1,2";"2,2" |(0.6){\textstyle =}
    }}
    \mkern20mu = \mkern20mu
    \vcenter{\xymatrix@=1em{
      & {\Del^0}\ar[d]^-i \ar@/^/[ddr]^a \ar@/_/@{=}[ddl] & \\
      & {A\comma a}\ar[dl]|{p_1}\ar[dr]|{p_0} & \\
      {\Del^0}\ar[rr]_{a} && {A}
      \ar@{} "2,2";"3,2" |(0.6){\Leftarrow\phi}
    }}\]
Thus, the adjunction $!\dashv i$ is a special case of Lemma \ref{lem:technicalsliceadjunction}.
\end{ex} 


Lemma \ref{lem:adjunction.from.isos} allows us to describe the minimal information required to display a terminal object.

\begin{lem}[minimal information required to display a terminal object]\label{lem:min-term-pres}
  To demonstrate that an object $t$ is terminal in $A$ it is enough to provide a unit 2-cell $\eta\colon\id_A\Rightarrow t!$ for which the whiskered composite $\eta t$ is an isomorphism.
 \end{lem}

When $A$ is a category this presentation is neither more nor less than the well known observation that an object $t$ is terminal in $A$ if and only if there exists a cocone on the identity diagram with vertex $t$ whose component at $t$ is an isomorphism. The proof of this lemma applies in any 2-category which possesses a 2-terminal object.

\begin{proof}
The categories $\hom'(\Del^0,\Del^0)$ and $\hom'(A,\Del^0)$ are both isomorphic to the terminal category $\catone$, so the counit is necessarily taken to be the identity and one of the triangle identities arises trivially. By Lemma \ref{lem:adjunction.from.isos} it remains only to provide a unit $\eta \colon \id_A \To t!$ for which the whiskered composition $\eta t$ is an isomorphism.  Specialising the proof of Lemma \ref{lem:adjunction.from.isos}, it follows formally that $\eta t \colon t \To t$ is an idempotent isomorphism and hence an identity, as required.
\end{proof}

The following straightforward 2-categorical lemma provides us with a useful ``external'' characterisation of terminal objects in quasi-categories.

\begin{lem}\label{lem:adj-ext-univ}
  Suppose we are given a pair of 1-cells $u\colon A\to B$ and $f\colon B\to A$ and a 2-cell $\epsilon\colon fu\Rightarrow\id_A$ in a 2-category $\tcat{C}$. Then $f$ is left adjoint to $u$ with counit $\epsilon$ in $\tcat{C}$ if and only if for all 0-cells $X\in\tcat{C}$ the functor $\tcat{C}(X,f)\colon\tcat{C}(X,B)\to\tcat{C}(X,A)$ is left adjoint to $\tcat{C}(X,u)\colon\tcat{C}(X,A)\to\tcat{C}(X,B)$, in the usual sense, with counit $\tcat{C}(X,\epsilon)$. 
\end{lem}

\begin{proof}
  The only if direction is immediate on observing that $\tcat{C}(X,-)$ is a 2-functor and thus preserves adjunctions. For the converse, we observe that the family of units of the adjunctions $\tcat{C}(X,f)\dashv\tcat{C}(X,u)$ is 2-natural in $X$ and so the 2-categorical Yoneda lemma provides us with a 2-cell $\eta\colon\id_B\Rightarrow uf$ with the property that $\tcat{C}(X,\eta)$ and $\tcat{C}(X,\epsilon)$ are unit and counit of the adjunction  $\tcat{C}(X,f)\dashv\tcat{C}(X,u)$. A further application of the 2-categorical Yoneda lemma demonstrates that the triangle identities for $\eta$ and $\epsilon$ follow immediately from those for $\tcat{C}(X,\eta)$ and $\tcat{C}(X,\epsilon)$.
\end{proof}

\begin{prop}\label{prop:terminal-ext-univ}
A vertex $t$ in a quasi-category $A$  is terminal if and only if for all $X$ the constant functor $\xymatrix@1{{X}\ar[r]^{!} & {\Del^0}\ar[r]^t & {A}}$ is terminal, in the usual sense, in the hom-category $\hom'(X,A)$.
\end{prop}

\begin{proof}
Apply Lemma~\ref{lem:adj-ext-univ} to the functors $t\colon\Del^0\to A$ and $!\colon A\to \Del^0$ and the identity natural transformation $!t = \id_{\Del^0}$.
\end{proof}

We conclude by comparing our definition of terminal object with its antecedent.


\begin{ex}\label{ex:terminaldefn} Joyal defines a vertex $t$ in a quasi-category $A$ to be terminal if and only if any sphere $\partial\Delta^n \to A$ whose final vertex is $t$ has a filler \cite[4.1]{Joyal:2002:QuasiCategories}. In Proposition~\ref{prop:terminalconverse}, we will show that Joyal's definition is equivalent to ours. For the moment, however, we shall at least take some satisfaction in convincing ourselves directly that his notion implies ours.

Supposing that $t \in A$ is terminal in Joyal's sense, then to define an adjunction $\adjinline !-| t :\Del^0 -> A .$ we wish to define a unit $\eta\colon\id_A\Rightarrow t!$ for which $\eta t$ is an identity. This unit is represented by a map \[ \xymatrix@=1.5em{ A \ar[d]_-{i_0} \ar@{=}[dr] \\ A \times \Delta^1 \ar[r]_-\eta & A \\  A \ar[u]^{i_1} \ar[r]_{!} & \Delta^0 \ar[u]_t} \] which we define as follows. For each $a \in A_0$, use the universal property of $t$ to choose a 1-simplex $\eta a \colon \Delta^1 \to A$ from $a$ to $t$. We take care to pick $\eta t$ to be the degenerate 1-simplex at $t$, thus ensuring that the 2-cell $\eta t$ will be the identity at $t$ as required by Lemma \ref{lem:min-term-pres}.

To define $\eta \colon A \to A^{\Delta^1}$ it suffices to inductively specify maps $\Delta^n \xrightarrow{\sigma} A \xrightarrow{\eta} A^{\Delta^1}$ for each non-degenerate $\sigma \in A_n$ compatibly with taking faces of $\sigma$. The map $\eta (\sigma \times \id_{\Delta^1}) \colon \Delta^n \times \Delta^1 \to A$ should be thought of as the component of $\eta$ at $\sigma$. The chosen 1-simplices $\eta a$ define the components at the vertices $a \in A_0$.

For each non-degerate $\alpha \colon a \to a' \in A_1$, define a cylinder $\Delta^1 \times \Delta^1 \to A$ as follows. The 1-skeleton consists of the displayed 1-simplices.
\[ \xymatrix{ a \ar[d]_\alpha \ar[r]^{\eta a} \ar[dr]|{\eta a} & t \ar@{=}[d]^{t\cdot\sigma^0} \\ a' \ar[r]_{\eta a'} & t} \] 
One shuffle is defined by degenerating $\eta a$. The other is chosen by applying the universal property of $t$ to the sphere formed by $\alpha$, $\eta a$, and $\eta a'$.

Continuing inductively, suppose we have chosen, for each $\sigma \in A_n$, a cylinder $\Delta^n \times \Delta^1 \to A$ from $\sigma$ to the degenerate $n$-simplex at $t$ in such a way that these choices are compatible with the face and degeneracy maps from the $n$-truncation $\sk_n\Del$ of $\Del$. Given a non-degenerate simplex $\tau \in A_{n+1}$, this simplex together with the $(n+1)$-simplices chosen for each of its $n$-dimensional faces $\tau\delta^i$ form an $(n+2)$-sphere with final vertex $t$, and we may choose a filler $\hat{\tau} \in A_{n+2}$. Define the requisite cylinder, the component of $\eta$ at $\tau$, to be the composite \[ \Delta^{n+1} \times \Delta^1 \xrightarrow{q} \Delta^{n+2} \xrightarrow{\hat{\tau}} A\] of $\hat{\tau}$ with the map induced by the functor $q \colon [n+1] \times [1] \to [n+2]$ defined by $q(i,0) = i$ and $q(i,1) = n+2$. By construction, $\hat{\tau} \face^i = \hat{\tau \face^i}$ for each $0 \leq i \leq n+1$, that is, the $i^{\th}$ face of the sphere whose filler defines $\hat{\tau}$ is the $(n+1)$-simplex chosen to fill the corresponding sphere for $\tau\delta^i$; thus the cylinder for $\tau$ is chosen compatibly with its faces. 
\end{ex}

This example will be generalised in Proposition \ref{prop:limitsasadjunctions} to limits of arbitrary shape.





\subsection{Basic theory}

A key advantage to our 2-categorical definition of adjunctions is that formal category theory supplies easy proofs of a number of desired results.

\begin{prop}\label{prop:adj-comp} A pair of adjunctions $\adjinline f -| u : A -> B.$ and $\adjinline f' -| u' : B -> C.$  in a 2-category compose to give an adjunction $\adjinline ff' -| u'u : A -> C.$. In particular, we may compose adjunctions of quasi-categories.
\end{prop}

\begin{proof}
The unit and counit of the composite adjunction are \[ \xymatrix@=10pt{ C \ar[dr]_{f'} \ar@{=}[rrrr] & & \ar@{}[d]|(.4){\Downarrow\eta'}& & C & & & C \ar[dr]^{f'} \ar@{}[d]|(.6){\Downarrow\epsilon'} \\ & B \ar[dr]_f \ar@{=}[rr] & \ar@{}[d]|(.4){\Downarrow\eta} & B \ar[ur]_{u'}  & & & B \ar[ur]^{u'} \ar@{=}[rr] & \ar@{}[d]|(.6){\Downarrow\epsilon} & B \ar[dr]^f \\  & & A \ar[ur]_u & &  & A \ar[ur]^u \ar@{=}[rrrr] & & & & A} \qedhere\] 
\end{proof}

Recall Proposition \ref{prop:equivsareequivs}, which demonstrates that equivalences in $\qCat_2$ are exactly the weak equivalences between quasi-categories in the Joyal model structure. The following classical 2-categorical result allows us to promote any equivalence to an adjoint equivalence (cf.~\cite[IV.4.1]{Maclane:1971:CWM}):

\begin{prop}\label{prop:equivtoadjoint} Any equivalence $w\colon A\to B$ in a 2-category may be promoted to an adjoint equivalence in which $w$ may be taken to be either the left or right adjoint. In particular, we may promote equivalences of quasi-categories to adjoint equivalences.
\end{prop}

\begin{proof}
  This is an immediate corollary of Lemma~\ref{lem:adjunction.from.isos}. 
\end{proof}

 \begin{prop}\label{prop:expadj} Suppose $\adjinline f -| u : A -> B.$ is an adjunction of quasi-categories. For any simplicial set $X$ and any quasi-category $C$, \[ \adjdisplay f^X-| u^X : A^X-> B^X .\qquad \text{and} \qquad \adjdisplay C^u -| C^f : C^A -> C^B .\] are adjunctions of quasi-categories. 
 \end{prop}
 \begin{proof}
By \ref{prop:qcat2closed} and \ref{rmk:exp2functor}, exponentiation defines 2-functors $(-)^X \colon \qCat_2 \to \qCat_2$ and $C^{(-)} \colon \qCat_2\op \to \qCat_2$, which preserve adjunctions.
 \end{proof}

As an easy corollary of the last few results, terminal objects are preserved by right adjoints, initial objects are preserved by left adjoints, and they are both preserved by equivalences.

\begin{prop}\label{prop:terminaldefn} If $u \colon A \to B$ is a right adjoint or an equivalence of quasi-categories and $t$ is a terminal object of $A$, then $ut$ is a terminal object in $B$.
\end{prop}
\begin{proof} By Proposition~\ref{prop:equivtoadjoint}, if $u$ is an equivalence then it may be promoted to a right adjoint, which reduces preservation by equivalences to preservation by right adjoints. Now Proposition~\ref{prop:adj-comp} tells us that we may compose the adjunction in which $u$ features with that which displays $t$ as a terminal object in $A$ to obtain an adjunction which displays $ut$ as a terminal object in $B$.
\end{proof}

\subsection{The universal property of adjunctions}

An essential point in the proof of the main existence theorem of \cite{RiehlVerity:2012hc} is that adjunctions between quasi-categories, while defined equationally, satisfy a universal property. In the terminology introduced there, any adjunction between quasi-categories extends to a \emph{homotopy coherent adjunction}. By contrast, a monad in $\qCat_2$ need not underlie a homotopy coherent monad. In this subsection, we provide several forms of the universal property held by an adjunction. 

Given an adjunction, we form the comma quasi-categories 
\begin{equation}\label{eq:commaobjdefn} \xymatrix@=1.5em{ f \comma A \ar[d]_{(p_1,p_0)} \ar[r] \pbexcursion & A^\cattwo \ar[d]  & & B \comma u \ar[d]_{(q_1,q_0)} \ar[r] \pbexcursion & B^\cattwo \ar[d] \\ A \times B \ar[r]_{\id_A \times f} & A \times A & & A \times B \ar[r]_{u \times \id_B} & B \times B}\end{equation} as in Definition~\ref{def:comma-obj}.  These quasi-categories are equipped with 2-cells
\[
\xymatrix@=15pt{ & f \comma A \ar[dl]_{p_1} \ar[dr]^{p_0} \ar@{}[d]|(.6){\Leftarrow\alpha} & && & B \comma u \ar[dl]_{q_1} \ar[dr]^{q_0} \ar@{}[d]|(.6){\Leftarrow\beta} \\ A & & B \ar[ll]^f && B \ar[rr]_u & & A}
\]
satisfying the weak 2-universal properties detailed in Observation~\ref{obs:unpacking-weak-comma-objects}. Mimicking the standard argument, we derive a fibred equivalence $f \comma A \simeq B \comma u$ from the unit and counit of our adjunction.

\begin{prop}\label{prop:adjointequiv} If $\adjinline f -| u : A -> B.$ is an adjunction of quasi-categories, then there is a fibred equivalence between the objects $(p_1,p_0)\colon f \comma A\tfib A\times B$ and $(q_1,q_0)\colon B \comma u\tfib A\times B$. 
\end{prop}
\begin{proof}
The composite 2-cells displayed on the left of the equalities below give rise to functors $w\colon B \comma u \to f \comma A$ and $w'\colon f \comma A \to B \comma u$ by 1-cell induction: 
\begin{equation*}\xymatrix@C=10pt{ & B \comma u \ar[dl]_{q_1} \ar[dr]^{q_0} \ar@{}[d]|(.6){\Leftarrow\beta} &  &  & B \comma u \ar[d]^{w}  & && & f \comma A \ar[dl]_{p_1} \ar[dr]^{p_0} \ar@{}[d]|(.6){\Leftarrow\alpha} & & & f \comma A \ar[d]^{w'}  \\ A \ar@{=}[dr] \ar[rr]^u & \ar@{}[d]|{\Leftarrow\epsilon} & B  \ar[dl]^f & = & f \comma A \ar[dl]_{p_1} \ar[dr]^{p_0} \ar@{}[d]|(.6){\Leftarrow\alpha} & & & A  \ar[dr]_u & \ar@{}[d]|{\Leftarrow\eta} & B \ar@{=}[dl] \ar[ll]_f & = & B \comma u \ar[dl]_{q_1} \ar[dr]^{q_0} \ar@{}[d]|(.6){\Leftarrow\beta}  \\ & A & & A  & & B \ar[ll]^f & & & B & & A \ar[rr]_u & & B}\end{equation*} 
By these defining pasting identities, the induced functors provide us with 1-cells 
\begin{equation*}
    \xymatrix{ f\comma A \ar@{->>}[dr]_{(p_1,p_0)} \ar@/^1ex/[rr]^{w'} & & B \comma u \ar@{->>}[dl]^{(q_1,q_0)} \ar@/^1ex/[ll]^w \\ & A \times B} 
\end{equation*}
in the slice 2-category $\qCat_2\slice(A\times B)$ commuting with the canonical isofibrations to $A \times B$. These identities give rise to the following sequence of pasting identities 
\begin{equation*}
\xymatrix@C=10pt{ & f \comma A \ar[d]^{w'} & & & f \comma A \ar[d]^{w'} & & & f \comma A \ar[dl]_{p_1} \ar[dr]^{p_0}\ar@{}[d]|(.6){\Leftarrow\alpha} & \ar@{}[dr]|*+{=} & & f \comma A  \ar[dl]_{p_1} \ar[dr]^{p_0} \ar@{}[d]|(.6){\Leftarrow\alpha} \\ & B \comma u \ar[d]^{w} & \ar@{}[d]|*+{=} & & B \comma u \ar[dl]_{q_1} \ar[dr]^{q_0} \ar@{}[d]|(.6){\Leftarrow\beta}  & {=}  & A  \ar[dr]_u \ar@{=}[dd] &\ar@{}[d]|{\Leftarrow\eta} & B \ar[ll]_f \ar@{=}[dl]  & A & & B \ar[ll]^f \\ & f \comma A \ar[dl]_{p_1} \ar[dr]^{p_0} \ar@{}[d]|(.6){\Leftarrow\alpha} & & A \ar@{=}[dr] \ar[rr]^u & \ar@{}[d]|(0.4){\Leftarrow\epsilon}  & B  \ar[dl]^f  &\ar@{}[r]|(0.4){\Leftarrow\epsilon} &  B \ar[dl]^f  &  \\ A & & B \ar[ll]^f & & A & & A& &  }
\end{equation*} 
in which the last step is an application of one of the triangle identities of the adjunction $f\dashv u$. This tells us that the endo-1-cells $ww'$ and $\id_{f\comma A}$ on the object $(p_1,p_0)\colon f\comma A\tfib A\times B$ in $\qCat_2\slice(A\times B)$ both map to the same 2-cell $\alpha$ under the whiskering operation. Applying Lemma \ref{lem:1cell-ind-uniqueness} (or Observation~\ref{obs:1cell-ind-uniqueness-reloaded}), 
we find that  $ww'$ and $\id_{f\comma A}$ are connected by a 2-isomorphism in $\qCat_2\slice(A\times B)$. A dual argument provides us with a 2-isomorphism between the 1-cells $w' w$ and $\id_{B\comma u}$ in the groupoid of endo-cells on $(q_1,q_0)\colon B\comma u\tfib A\times B$. This data provides us with an equivalence in the slice 2-category $\qCat_2\slice(A\times B)$, which we may lift along the smothering 2-functor of Proposition~\ref{prop:slice-smothering-2-functor} to give a fibred equivalence over $A\times B$.
\end{proof}

Just as in ordinary category theory, the Proposition~\ref{prop:adjointequiv} has a converse:

\begin{prop}\label{prop:adjointequivconverse} Suppose we are given functors $u \colon A \to B$ and $f\colon B\to A$ between quasi-categories. If there is a fibred equivalence between $(p_1,p_0)\colon f \comma A\tfib A\times B$ and $(q_1,q_0)\colon B \comma u\tfib A\times B$, then $f$ is left adjoint to $u$.
\end{prop}

    Schematically the proof of this result proceeds by observing that the image of the identity morphism at $f$ under the equivalence $f \comma A \simeq B \comma u$ defines a candidate unit for the desired adjunction. This can then be shown to have the appropriate universal property; the proof, however is slightly subtle. We delay it to the next section, where it will appear as a special case of a more general result needed there.

\begin{obs}[the hom-spaces of a quasi-category]\label{obs:pointwise-adjoint-correspondence} One model for the hom-space between a pair of objects $a$ and $a'$ in a quasi-category $A$ is the comma quasi-category $a \comma a'$, denoted by $\mathrm{Hom}_A(a,a')$ in~ \cite{Lurie:2009fk}. Proposition~\ref{prop:weakcomma} tells us that the canonical comparison $\ho(a\comma a')\to \ho(a)\comma \ho(a')$ from the homotopy category of this hom-space is a smothering functor. Its codomain $\ho(a)\comma\ho(a')$ is a comma category of arrows between a fixed pair of objects in the category $\ho A$, so it is simply the discrete category whose objects are the arrows from $a$ to $a'$ in $\ho A$. It follows from conservativity of the smothering functor that all arrows in $\ho(a\comma a')$ and thus also $a\comma a'$ are isomorphisms; hence,  $a\comma a'$ is a Kan complex by Joyal's result \cite[1.4]{Joyal:2002:QuasiCategories}.

By Observation~\ref{obs:fibred-pullback}, the fibred equivalence of Proposition~\ref{prop:adjointequiv} may be pulled back along the functor $(a,b)\colon\Del^0\to A\times B$ associated with any pair of vertices $a\in A$ and $b\in B$ to give an equivalence $fb \comma a \simeq b \comma ua$ of hom-spaces. This should be thought of as a quasi-categorical analog of the usual adjoint correspondence defined for arrows between a fixed pair of objects $b \in B$ and $a \in A$. 
\end{obs}

\begin{rmk}\label{rmk:vs-lurie-adjunction}
Observation \ref{obs:pointwise-adjoint-correspondence} demonstrates that the 2-categorical definition of an adjunction implies the definition of adjunction given by Lurie in \cite[5.2.2.8]{Lurie:2009fk}. As his definition has a more complicated form, we prefer not to recall it here. It is in fact precisely equivalent to Joyal's 2-categorical definition \ref{defn:adjunction}. Our preferred proof that Lurie's definition implies Joyal's makes use of the fact that the domain and codomain projections from comma quasi-categories are, respectively, cartesian and cocartesian fibrations. A proof will appear in \cite{RiehlVerity:2015fy}, which gives new 2-categorical definitions of these notions, which, when interpreted in $\qCat_2$, recapture precisely the (co)cartesian fibrations of \cite{Lurie:2009fk}.
\end{rmk}

We may apply Proposition~\ref{prop:adjointequiv}  to give a converse to Example~\ref{ex:terminaldefn}, proving that our notion of terminal objects is equivalent to Joyal's. The proof requires one combinatorial lemma, which relates certain comma quasi-categories with Joyal's slices, which are recalled in \ref{defn:slices} and \ref{rmk:map-slices}.

\begin{lem}\label{lem:slice-equiv-comma} For any vertex $a$ in a quasi-category $A$, there is an equivalence
\[\xymatrix@=1em{ \slicer{A}{a} \ar@{->>}[dr] \ar[rr]^-\sim & & A \comma a \ar@{->>}[dl] \\ & A} \] over $A$, which pulls back along any $f \colon B \to A$ to define an equivalence $\slicer{f}{a} \simeq f \comma a$ over $B$.
\end{lem}
\begin{proof}
The result follows from an isomorphism $A \comma a \cong \fatslicer{A}{a}$ between the comma and the fat slice construction reviewed in Definition \ref{defn:fat-slices}. The map $\slicer{A}{a} \to A \comma a$ and the equivalence over $A$ are then special cases of Proposition \ref{prop:slice-fatslice-equiv}. To establish the isomorphism, it suffices to show that $A \comma a$ has the universal property that defines $\fatslicer{A}{a}$.  By adjunction, a map $X \to \fatslicer{A}{a}$ corresponds to a commutative square, as displayed on the left:
\[ \vcenter{\xymatrix{ X \coprod X \ar[d] \ar[r]^-{\pi_X \coprod !} & X \coprod \Del^0 \ar[d]^{ (f, a)} \\ X \times \Del^1 \ar[r]_-k & A}} \qquad \leftrightsquigarrow \qquad \vcenter{\xymatrix{ X \ar[r]^-k \ar[d]_{(!,f)} & A^{\Del^1} \ar[d] \\  \Del^0 \times A \ar[r]_-{ a \times \id_A} & A \times A}}\]
which transposes to the commutative square displayed on the right. The data of the right-hand square is precisely that of a map $X \to A \comma a$ by the universal property of the pullback \ref{def:comma-obj} defining the comma quasi-category.

The isomorphism $A \comma a \cong \fatslicer{A}{a}$ pulls back to define an isomorphism $f \comma a \cong \fatslicer{f}{a}$. The map $\slicer{f}{a} \to f \comma a$ is then an equivalence over $B$ by Remark \ref{rmk:map-slices}.
\end{proof} 


\begin{prop}\label{prop:terminalconverse} A vertex $t \in A$ is terminal in the sense of  Joyal's \cite[4.1]{Joyal:2002:QuasiCategories} if and only if \[\adjdisplay !-| t :\Del^0 -> A . \] is an adjunction of quasi-categories. 
\end{prop}
\begin{proof}
The ``if'' direction is Example~\ref{ex:terminaldefn}. For the converse implication, an adjunction $! \dashv t$ gives rise to an equivalence between $!\comma \Delta^0 \cong A$ and $A \comma t$ over $A$ by Proposition~\ref{prop:adjointequiv}. Hence, by the 2-of-3 property of equivalences, the isofibration $A \comma t \tfib A$ is a trivial fibration. Lemma \ref{lem:slice-equiv-comma} supplies an equivalence \[\xymatrix@=1em{ \slicer{A}{t} \ar@{->>}[dr]_\sim \ar[rr]^-\sim & & A \comma t \ar@{->>}[dl] \\ & A} \] between our comma quasi-category and Joyal's slice quasi-category; see \ref{defn:slices} for a definition. Applying the 2-of-3 property again, it follows that the isofibration $\slicer{A}{t} \tfib A$ is a trivial fibration; the right lifting property against the boundary inclusions $\boundary\Delta^n \to \Delta^n$ says precisely that $t \in A$ is terminal in Joyal's sense.
\end{proof}

One reason for our particular interest in terminal objects is to show that the units and counits of adjunctions have universal properties which may be expressed ``pointwise'' in terms of certain outer horn filler conditions.

\begin{prop}[the pointwise universal property of an adjunction]\label{prop:pointwise-univ-adj}
    Suppose that we are given an adjunction 
    \begin{equation*}
        \adjdisplay f -| u : A -> B.
    \end{equation*}
    of quasi-categories with unit $\eta\colon\id_B\Rightarrow uf$ and counit $\epsilon\colon fu\Rightarrow \id_A$. Then for each $a \in A$ the (fat) slice quasi-category $f\comma a\simeq\slicer{f}{a}$ has terminal object $\epsilon a\colon fua\to a$, namely the component of the counit $\epsilon$ at $a$.
\end{prop}

\begin{proof}
    From Proposition~\ref{prop:adjointequiv}, the adjunction $f\dashv u$ gives rise to the equivalence $f \comma A \simeq B \comma u$ fibred over $A \times B$. By Observation~\ref{obs:fibred-pullback}, for each  $a\in A$, the fibred equivalence pulls back along the functor $(a,\id_B)\colon B\to A\times B$ to give a fibred equivalence
    \begin{equation}
      \xymatrix{ f\comma a \ar@{->>}[dr]_{p_0} \ar@/^1ex/[rr]^{w'} & & B \comma ua \ar@{->>}[dl]^{q_0} \ar@/^1ex/[ll]^w \\ & B}
    \end{equation}
  over $B$. 

By Example~\ref{ex:slice-terminal}, we know that $B\comma ua$ has the identity map $ua\cdot\degen^0\colon ua\to ua$ as its terminal object, and by Proposition~\ref{prop:terminaldefn} we know that terminal objects transport along equivalences, so it follows that $f\comma a$ also has terminal object $w'(ua\cdot\degen^0)$. It is now easily checked, from the definition of $w'$ given in Proposition~\ref{prop:adjointequiv}, that $w'(ua\cdot\degen^0)$ is isomorphic to $\epsilon a\colon fua\to a$. The desired result follows on transporting this terminal object along the equivalence between $f\comma a$ and $\slicer{f}{a}$ provided by the geometry result of Lemma \ref{lem:slice-equiv-comma}.
\end{proof}

Of course, the unit of an adjunction of quasi-categories satisfies a dual universal property.

\begin{obs}[unpacking this pointwise universal property of an adjunction] \label{obs:universal-property-of-epsilon}
    Unpacking the definitions in Remark~\ref{rmk:map-slices} and Definition~\ref{defn:slices} we see that a map $X\to \slicer{f}{a}$ corresponds to a pair of maps $b\colon X\to B$ and $\alpha\colon X\join\Del^0\to A$ which make the diagram 
    \[
        \xymatrix@=1.5em{ 
            X \ar[d] \ar[r]^f & B \ar[d]^b \\ X \join \Del^0 \ar[r]^-{\alpha} & A \\ \Del^0 \ar[u] \ar[ur]_{a}} \] commute.

By Proposition \ref{prop:terminalconverse}, we know that $\epsilon a\colon fua\to a$ is terminal in $\slicer{f}{a}$ is terminal if and only if every sphere $\boundary\Del^{n-1}\to \slicer{f}{a}$ whose last vertex is $\epsilon a$ may be filled to a simplex. Applying our description of maps into $\slicer{f}{a}$ and observing that $\Del^{n-1}\join\Del^0\cong\Del^n$ and $\boundary\Del^{n-1}\join\Del^0\cong\Horn^{n,n}$, we see that $\epsilon a$ being terminal means that if we are given
    \begin{itemize} 
        \item a horn $\Horn^{n,n} \to A$, with $n \geq 2$ together with
        \item a sphere $\partial\Delta^{n-1} \to B$ whose composite with $f$ is the boundary of the missing face of the horn, with the property that
        \item  the final edge of the horn  is $\epsilon a$ 
    \end{itemize} 
    then there is 
    \begin{itemize}
        \item a simplex $\Delta^n \to A$ filling the given horn and
        \item a simplex $\Delta^{n-1} \to B$ filling the given sphere, with the property that
        \item the $n\th$ face of the filling $n$-simplex in $B$ is the simplex obtained by applying $f$ to the filling $(n-1)$-simplex in $A$.
    \end{itemize} 

    For $n=2$, this situation is summarised by the following schematic:
 \[ \vcenter{ \xymatrix@=1.2em{ & fua \ar[dr]^{\epsilon} & \\ fb \ar[rr]_\alpha & & a}} b \in B_0\quad \rightsquigarrow \vcenter{\xymatrix@=1.2em{ & fua \ar[dr]^{\epsilon} \ar@{}[d]|(.6){\sigma} &\\ fb \ar[ur]^{f\beta} \ar[rr]_\alpha & & a}}  \mkern10mu \sigma \in A_2,\ \beta \colon b \to ua \in B_1\]
  \end{obs}

\begin{obs}[the relative universal property of an adjunction]
For any quasi-category $X$ the 2-functor $\hom'(X,{-})\colon \qCat_2\to\Cat$ carries an adjunction $f\dashv u\colon A\to B$ of quasi-categories to an adjunction $\hom'(X,f)\dashv \hom'(X,u)\colon\hom'(X,A)\to\hom'(X,B)$ of categories. Extending Lemma~\ref{lem:adj-ext-univ}, a standard and easily established fact of 2-category theory is that $f\colon B\to A$ has a right adjoint in $\qCat_2$ if and only if for each quasi-category $X$ the functor $\hom'(X,f)\colon\hom'(X,B)\to\hom'(X,A)$  has a right adjoint. We might call this observation the {\em external\/} universal property of an adjunction.

    There is a closely related {\em internal\/} or {\em relative\/} universal property of adjunctions in $\qCat_2$, which arises instead from Remark~\ref{rmk:exp2functor} that the cotensor $(-)^X\colon\qCat_2\to\qCat_2$ is also a 2-functor. Applying this cotensor 2-functor to the adjunction $f\dashv u$ we obtain its relative universal property simply as the pointwise universal property of the adjunction $f^X\dashv u^X\colon A^X\to B^X$ as derived in Proposition~\ref{prop:pointwise-univ-adj} and expressed explicitly in Observation~\ref{obs:universal-property-of-epsilon}. The relative universal property of adjunctions will become a key tool in the proof that any adjoint functor between quasi-categories extends to a homotopy coherent adjunction; see  \cite{RiehlVerity:2012hc}.  
\end{obs}

Another application of Proposition \ref{prop:adjointequiv} allows us to show that an isofibration between quasi-categories admits a right adjoint right inverse if and only if the following lifting property holds.

\begin{lem}[right adjoint right inverse as a lifting property]\label{lem:RARI-lifting}
  An isofibration $f \colon B \tfib A$ of quasi-categories admits a right adjoint right inverse if and only if for all $a \in A_0$ there exists $ua \in B_0$ with $fua = a$ and so that any lifting problem with $n \geq 1$
\begin{equation}\label{eq:RARI-lifting} \xymatrix{ \Delta^0 \ar[r]_{\fbv{n}} \ar@/^2ex/[rr]^{ua} & \boundary\Delta^n \ar@{u(->}[d] \ar[r] & B \ar@{->>}[d]^f \\ & \Delta^n \ar@{-->}[ur] \ar[r] & A} \end{equation}
has a solution.
\end{lem}
\begin{proof}
If $u$ is the right adjoint right inverse, then $fu = \id_A$ and there is a trivial fibration $B \comma u \trvfib f \comma fu \cong f \comma A$ over $A \times B$ defined by applying $f$ (Lemma \ref{lem:comma-obj-maps} proves that this map is an isofibration and Proposition \ref{prop:adjointequiv} shows that it is an equivalence). This trivial fibration pulls back over any vertex $a \in A_0$ to define a trivial fibration $B \comma ua \trvfib f \comma a$. The domain and codomain are equivalent to Joyal's slices by Lemma \ref{lem:slice-equiv-comma}, so the isofibration $\slicer{B}{ua} \tfib \slicer{f}{a}$ is also a trivial fibration:
\[ \xymatrix{ \boundary\Delta^{n-1} \ar[r] \ar[d] & \slicer{B}{ua} \ar[d] \\ \Delta^n \ar[r] \ar@{-->}[ur] & \slicer{f}{a} \cong B \times_A \slicer{A}{a}}\] In adjoint form, this is the lifting property of \eqref{eq:RARI-lifting}.

Conversely, the lifting property \eqref{eq:RARI-lifting} can be used to inductively define a section $u \colon A \to B$ of $f$ extending the choices $ua \in B_0$ for $a \in A_0$. The inclusion $\sk_0A\hookrightarrow A$ can be expressed as a countable composite of pushouts of coproducts of maps $\boundary\Del^n\hookrightarrow\Del^n$ with $n \geq 1$, and each intermediate lifting problem required to define a lift
\[ \xymatrix{ \Delta^0 \ar[r]_-{a} \ar@/^2ex/[rr]^{ua} & \sk_0 A \ar@{u(->}[d] \ar[r] & B \ar@{->>}[d]^f \\ & A \ar@{-->}[ur]^u \ar@{=}[r] & A}\]
will have the form of \eqref{eq:RARI-lifting}. To show that $u$ is a right adjoint right inverse to $f$, it suffices, by Lemma \ref{lem:adjunction.from.isos} to define a 2-cell $\eta \colon \id_B \To uf$ that whiskers with $u$ and with $f$ to isomorphisms. We construct a representative for $\eta$ by solving the lifting problem
\[\xymatrix{ B \coprod B \ar[d] \ar[rr]^{\id_B \coprod uf} & & A \ar[d]^{f} \\ B \times \Delta^1 \ar@{-->}[urr]^\eta \ar[r]_-{\pi_B} & B \ar[r]_f & A}\]
By construction $f\eta=\id_f$. 

To show that $\eta u$ is an isomorphism it suffices, by Corollary \ref{cor:pointwise-equiv}, to check that its components $\eta u(a) \colon ua \to ufua=ua$ are isomorphisms in $A$. Inverse isomorphisms can be found by elementary applications of the lifting property \eqref{eq:RARI-lifting}, whose details we leave to the reader.
\end{proof}


\subsection{Fibred adjunctions}\label{subsec:fibred.adjunction}

Fibred equivalences over $A$, i.e., equivalences in $\ho_*(\qCat_\infty\slice  A)$, are preferable to equivalences in the slice 2-category $\qCat_2\slice  A$ because the former can be pulled back along arbitrary maps $f \colon B \to A$; see Observation~\ref{obs:fibred-pullback}. Precisely the same kind of reasoning applies to adjunctions in $\qCat_2\slice  A$. 

\begin{defn}[fibred adjunctions]\label{defn:fibred.adj}
  We refer to adjunctions in $\ho_*(\qCat_\infty\slice A)$ as \emph{adjunctions fibred over $A$} or simply \emph{fibred adjunctions}.
\end{defn}

    Our aim in this section is to show that any adjunction in $\qCat_2\slice A$ can  be lifted to an adjunction fibred over $A$, i.e., to an adjunction in $\ho_*(\qCat_\infty\slice A)$. In particular, such a result will allow us to prove that any adjunction in $\qCat_2\slice  A$ may be pulled back along any functor $f\colon B\to A$. We shall use this result to define a loops--suspension adjunction on any quasi-category with appropriate finite limits and colimits (cf.\ Proposition~\ref{prop:loops-suspension}).

    Recall from Proposition \ref{prop:slice-smothering-2-functor} that the canonical 2-functor $\ho_*(\qCat_\infty\slice A) \to \qCat_2\slice A$ is a smothering 2-functor. Consequently, the following 2-categorical lemma is key:

\begin{lem}\label{lem:missed-lemma} Suppose $F \colon \tcat{C} \to \tcat{D}$ is a smothering 2-functor. Then any adjunction in $\tcat{D}$ can be lifted to an adjunction in $\tcat{C}$. Furthermore, if we have previously specified a lift of the objects, 1-cells, and either the unit or counit of the adjunction in $\tcat{D}$, then there is a lift of the remaining 2-cell that combines with the previously specified data to define an adjunction in $\tcat{C}$.
\end{lem}
\begin{proof}
We use surjectivity on objects and local surjectivity on arrows to define $u \colon A \to B$ and $f\colon B\to A$ in $\tcat{C}$ lifting the objects and 1-cells of the downstairs adjunction. Then we use local fullness to define lifts $\epsilon \colon fu \Rightarrow \id_A$ and $\eta' \colon \id_B \Rightarrow uf$ of the downstairs counit and unit. If desired, we can regard $A$, $B$, $f$, $u$ and $\epsilon$ as ``previously specified''. We will show that $f \dashv u$ by modifying the 2-cell $\eta'$. The details are similar to the proof of Lemma \ref{lem:adjunction.from.isos}.

We define a 2-cell $\theta\colon u\Rightarrow u$ as the ``triangle identity composite'' $\theta\defeq u\epsilon \cdot \eta' u$ and observe that $F\theta = \id_{Fu}$. Applying the local conservativity of the action of $F$ on 2-cells, we conclude that $\theta$ is an isomorphism. Define the 2-cell $\eta \colon \id_B \Rightarrow uf$ to be the composite $\eta \defeq \theta^{-1} f \cdot \eta'$. Because $F\theta$ is an identity,  $F\eta$ and $F\eta'$ lift the same downstairs 2-cell. We claim that this data forms an adjunction in $\tcat{C}$.

The diagram \eqref{eq:triangle-calculation-1} demonstrates that $u\epsilon \cdot \eta u = \id_u$. The diagram \eqref{eq:triangle-calculation-2} demonstrates that the other triangle identity composite $\phi\defeq \epsilon f \cdot f \eta $ is an idempotent. Finally observe that the component parts we've composed to make $\phi$ all map by $F$ to the corresponding components of the original adjunction in $\lcat{L}$. It follows that $F\phi$ is equal to the corresponding triangle identity composite in $\lcat{L}$ and so is an identity. Consequently, applying the local conservativity of $F$ on 2-cells we find that $\phi$ is an isomorphism. Because all idempotent isomorphisms are identities,  it follows that $\epsilon f \cdot f \eta = \id_f$ as required.
\end{proof} 

\begin{cor}\label{cor:missed-lemma}
  Every adjunction in $\qCat_2\slice A$ lifts to an adjunction fibred over $A$.
\end{cor}

\begin{proof}
  Combine Proposition~\ref{prop:slice-smothering-2-functor} and Lemma~\ref{lem:missed-lemma}.
\end{proof}

\begin{ex}\label{ex:fibred-technical-slice-adjunction}
Corollary \ref{cor:missed-lemma} allows us to lift the adjunction $\adjinline p_1 -| i : C -> B\comma \ell.$ of Lemma \ref{lem:technicalsliceadjunction} to a fibred adjunction over $C$ whose counit is an identity.
\end{ex}

\begin{ex}[fibred isofibration RARIs]\label{ex:isofib-section.fibred.adjunction} 
Lemma \ref{lem:isofibration-RARI} demonstrates that any right adjoint right inverse to an isofibration $f \colon B \tfib A$ can be modified to produce a RARI $f \dashv u$ with an identity counit. This latter adjunction provides us with an adjunction in $\qCat_2\slice A$ which we may lift into $\ho_*(\qCat_\infty\slice A)$ to give an adjunction
\begin{equation}\label{eq:fibred.terminal}
  \xymatrix@=1.5em{
    {A}\ar@/_1.2ex/[rr]_u\ar@{=}[dr] & {\bot} & 
    {B}\ar@/_1.2ex/[ll]_f\ar@{->>}[dl]^{f} \\
    & A &
  }
\end{equation}
which is fibred over $A$. In essence, this latter fibred adjunction expresses the fact that each of the fibres of the isofibration $f\colon B\tfib A$ has a terminal object.
\end{ex} 


\begin{obs}\label{obs:isofib-section.fibred.adjunction}
   Applying the 2-functor $\hom'_A(p,{-})$ represented by an isofibration $p\colon E\tfib A$ to the fibred adjunction in~\eqref{eq:fibred.terminal} we obtain an adjunction
  \begin{equation*}
    \adjdisplay f\circ{-} -| u\circ{-} : 
    \hom'_A(p,\id_A) -> \hom'_A(p, f).
  \end{equation*}
  of hom-categories. Now the identity functor $\id_A$ is the 2-terminal object of the 2-category $\qCat_2\slice A$, so it follows that $\hom'_A(p,\id_A)\cong\catone$. Hence, the displayed adjunction amounts simply to the assertion that $up$ is a terminal object of the category $\hom'_A(p, f)$. Consequently, applying Lemma~\ref{lem:adj-ext-univ}, we discover that there exists a fibred adjunction of the form displayed in~\eqref{eq:fibred.terminal} if and only if for all isofibrations $p\colon E\tfib A$ the composite map $up\colon E\to B$ is a terminal object of the hom-category $\hom'_A(p,f)$.
\end{obs}

A final example of a fibred adjunction describes the ``composition'' functor $A^{\Horn^{2,1}} \to A^\cattwo$ that fills a (2,1)-horn and then restricts to the missing face as the right and left adjoint, respectively, to the pair of functors that extend a 1-simplex into a composable pair by using the identities at its domain and codomain.

\begin{ex}\label{ex:comp.ident.adj}
  There exists a pair of adjunctions
  \begin{equation*}
    \xymatrix@C=10em@R=1ex{
      {{\Del^1}}\ar[r]|*+{\scriptstyle \face^1} & {{\Del^2}}
      \ar@/^2.5ex/[l]^{\degen^0}_{}="l" \ar@/_2.5ex/[l]_{\degen_1}^{}="u"
      \ar@{} "u";"l" |(0.2){\bot} |(0.8){\bot}
    }
  \end{equation*}
  of ordered sets, whose units and counits arise as the equalities $\degen^0\face^1 = \degen^1\face^1=\id_{\Del^1}$ and the inequalities $\face^1\degen^0 < \id_{[2]} < \face^1\degen^1$. Now if $A$ is a quasi-category, we may apply Proposition~\ref{prop:expadj} to construct the associated pair of adjunctions
  \begin{equation*}
    \xymatrix@C=10em{
      {A^{\Del^2}}
      \ar[r]|*+{\scriptstyle A^{\face^1}} &
      {A^{\Del^1}}
      \ar@/^2.5ex/[l]^{A^{\degen^1}}_{}="l" \ar@/_2.5ex/[l]_{A^{\degen_0}}^{}="u"
      \ar@{} "u";"l" |(0.2){\bot} |(0.8){\bot} 
    }
  \end{equation*}
  Here the upper adjunction has identity unit and the lower adjunction has identity counit. So it follows from Example~\ref{ex:isofib-section.fibred.adjunction} that this is a pair of adjunctions fibred over $A^{\Del^1}$ with respect to the projections $A^{\face^1}\colon A^{\Del^2}\tfib A^{\Del^1}$ and $\id_{A^{\Del^1}}\colon A^{\Del^1} \tfib A^{\Del^1}$. 

Because the horn inclusion $\Horn^{2,1}\inc\Del^2$ is a trivial cofibration in Joyal's model structure, the associated restriction isofibration $p\colon A^{\Del^2}\tfib A^{\Horn^{2,1}}$ is an equivalence of quasi-categories fibred over $A^{\Horn^{2,1}}$. By Proposition~\ref{prop:equivtoadjoint} (applied to $\qCat_2/A^{\Horn^{2,1}}$) and Corollary~\ref{cor:missed-lemma}, the fibred equivalence formed by $p$ and a chosen inverse $p'$ can be promoted to a pair of adjoint equivalences $p \dashv p' \dashv p$ fibred over $A^{\Horn^{2,1}}$. On account of the pushout diagram defining the (2,1)-horn,  $A^{\Horn^{2,1}}$ is isomorphic to the pullback:
\begin{equation*}
  \xymatrix@=2em{ \Horn^{2,1} \pbexcursion & \Del^1 \ar[l]_-{\face^2} & & 
    {A^{\Horn^{2,1}}}\pbexcursion
    \ar[r]^-{\pi_0}\ar[d]_-{\pi_1} & {A^\cattwo}\ar@{->>}[d]^-{p_1} \\ \Del^1 \ar[u]^{\face^0} & \Del^0 \ar[u]_{\face^0}\ar[l]^-{\face^1} & &
    {A^\cattwo}\ar@{->>}[r]_-{p_0} & A
  }
\end{equation*}

Now we may take the pushforward of the fibred adjunctions of the last two paragraphs along the isofibrations $(A^{\fbv{1}},A^{\fbv{0}})\colon A^{\Del^1}\tfib A\times A$ and $(A^{\fbv{2}},A^{\fbv{0}})\colon A^{\Horn^{2,1}}\tfib A\times A$ respectively to obtain adjunctions fibred over $A\times A$. Composing these we obtain a pair of adjunctions 
  \begin{equation}\label{eq:comp.ident.adj}
    \xymatrix@C=10em{
      *+[l]{A^{\Horn^{2,1}}\cong A^\cattwo\times_AA^\cattwo}
      \ar[r]|*+{\scriptstyle m} &
      {A^\cattwo}
      \ar@/^2.5ex/[l]^{i_1}_{}="l" \ar@/_2.5ex/[l]_{i_0}^{}="u"
      \ar@{} "u";"l" |(0.2){\bot} |(0.8){\bot} 
    }
  \end{equation}
  which are fibred over $A\times A$ with respect to the projections $(p_1,p_0)\colon A^\cattwo\tfib A\times A$ and $(p_1\pi_1,p_0\pi_0)\colon A^{\Horn^{2,1}}\tfib A\times A$. Here the upper adjunction has isomorphic unit and the lower adjunction has isomorphic counit. The functors $i_0$ and $i_1$ degenerate the domain and codomain respectively of a given 1-simplex to form a (2,1)-horn. The map $m$ is a ``composition'' functor.
\end{ex}


%!TEX root = all.tex
% ******************************************************************
% ** Title:           The 2-category theory of quasi-categories
% **                   limits and colimits
% ** Precis:        
% ** Author:           Emily Riehl and Dominic Verity
% ** Commenced:        2/3/2012
% ******************************************************************


\section{Limits and colimits}\label{sec:limits}

In this section, we demonstrate that limits and colimits of individual diagrams in a quasi-category can be encoded as {\em absolute right and left liftings\/} in the 2-category $\qCat_2$. The proof that this definition is equivalent to the standard one makes use of the fact that absolute lifting diagrams in $\qCat_2$ can be detected by an equivalence of suitably defined comma quasi-categories. This observation, combined with Example~\ref{ex:adjasabslifting}, also supplies the proof of Proposition~\ref{prop:adjointequivconverse}, completing the unfinished business from the previous section. 

We begin with a general definition:

\setcounter{thm}{0}
\begin{defn}\label{defn:abs-right-lift} In a 2-category, an \emph{absolute right lifting diagram}   consists of the data \begin{equation}\label{eq:absRlifting}\xymatrix{ \ar@{}[dr]|(.7){\Downarrow\lambda} & B \ar[d]^f \\ C \ar[r]_g \ar[ur]^\ell & A}\end{equation} with the universal property that if we are given any 2-cell $\chi$ of the form depicted to the left of the following equality
\begin{equation}\label{eq:abs-lifting-property}
    \vcenter{\xymatrix{ X \ar[d]_c \ar[r]^b \ar@{}[dr]|{\Downarrow\chi} & B \ar[d]^f \\ C \ar[r]_g & A}} \mkern20mu = \mkern20mu \vcenter{\xymatrix{ X \ar[d]_c \ar[r]^b \ar@{}[dr]|(.3){\exists !\Downarrow}|(.7){\Downarrow\lambda} & B \ar[d]^f \\ C \ar[ur]|(.4)*+<2pt>{\scriptstyle\ell} \ar[r]_g & A}}
 \end{equation} 
 then it admits a unique factorisation of the form displayed to the right of that equality. When this condition holds for the diagram in~\eqref{eq:absRlifting} we say that it {\em displays $\ell$ as an absolute right lifting of $g$ through $f$}.
\end{defn}

\begin{ex}\label{ex:adjasabslifting} The counit of an adjunction $\adjinline f-|u:A->B.$ defines an absolute right lifting diagram 
  \begin{equation}\label{eq:adjasabslifting}
    \xymatrix{ \ar@{}[dr]|(.7){\Downarrow\epsilon} & B \ar[d]^f \\ A \ar[ur]^u \ar[r]_{\id_A} & A}  
  \end{equation}
  and, conversely, if this diagram displays $u$ as an absolute right lifting of the identity on its domain through $f$ then $f$ is left adjoint to $u$ with counit 2-cell $\epsilon$. 
\end{ex}

\begin{proof} 
This is a standard 2-categorical result. The 2-functor represented by $X$ carries an adjunction $f \dashv u$ to an adjunction whose counit has the universal property described in~\eqref{eq:abs-lifting-property} for the 2-cell \eqref{eq:adjasabslifting}.

Conversely, given an  absolute right lifting diagram~\eqref{eq:adjasabslifting}, we take this 2-cell to be the counit and define the unit by applying the universal property of this absolute right lifting to the identity 2-cell:
\begin{equation}\label{eq:unitdefn} 
  \vcenter{\xymatrix{ B \ar[r]^{\id_B} \ar[d]_f \ar@{}[dr]|{\Downarrow \id_f} & B \ar[d]^f \\ A \ar[r]_{\id_{A}} & A}} = \vcenter{\xymatrix{ B \ar[r]^{\id_B} \ar[d]_f \ar@{}[dr]|(.3){\Downarrow \eta}|(.7){\Downarrow\epsilon} & B \ar[d]^f \\ A \ar[r]_{\id_{A}} \ar[ur]|*+{\scriptstyle u} & A}} 
  \end{equation}
  This defining equation establishes one of the triangle identities. The other is obtained by pasting $\epsilon$ on the left of both of the 2-cells of \eqref{eq:unitdefn} and applying the uniqueness statement in the universal property of the absolute right lifting:
 \[  \vcenter{\xymatrix{ \ar@{}[dr]|(.7){\Downarrow\epsilon} & B \ar[r]^{\id_B} \ar[d]_f \ar@{}[dr]|{\Downarrow \id_f} & B \ar[d]^f \\ A \ar[r]_{\id_{A}} \ar[ur]^{u} & A \ar[r]_{\id_{A}} & A}} = \vcenter{\xymatrix{  \ar@{}[dr]|(.7){\Downarrow\epsilon}& B \ar[r]^{\id_B} \ar[d]_f \ar@{}[dr]|(.3){\Downarrow \eta}|(.7){\Downarrow\epsilon} & B \ar[d]^f \\ A \ar[r]_{\id_{A}} \ar[ur]^{u} &  A \ar[r]_{\id_{A}} \ar[ur]|*+{\scriptstyle u} & A}}\rightsquigarrow \vcenter{\xymatrix{ & B  \ar@{}[dl]|{\Downarrow\id_{u}}  \\ A \ar@/^2.25ex/[ur]^{u} \ar@/_2.25ex/[ur]_{u} &  }} = \vcenter{\xymatrix{  \ar@{}[dr]|(.7){\Downarrow\epsilon}& B \ar[r]^{\id_B} \ar[d]_f \ar@{}[dr]|(.3){\Downarrow \eta}& B\\ A \ar[r]_{\id_{A}} \ar[ur]^{u} &  A  \ar[ur]|*+{\scriptstyle u} & }}\qedhere \]
\end{proof}

\subsection{Absolute liftings and comma objects}

We now specialise to the 2-category $\qCat_2$. Our aim is to use its weak comma objects to re-express the universal property of absolute lifting diagrams and describe various procedures through which they may be detected.

Given any diagram in $\qCat_2$ of the form displayed in~\eqref{eq:absRlifting} in $\qCat_2$ we may form comma objects $B \comma \ell$ and $f \comma g$ with canonical comma cones:
\begin{equation}\label{eq:comma-cones}
    \vcenter{ \xymatrix{ B \comma \ell \ar[d]_{p_1} \ar[r]^-{p_0} \ar@{}[dr]|(.3){\Downarrow\phi} & B  & & f \comma g \ar[d]_{q_1} \ar[r]^-{q_0}  \ar@{}[dr]|{\Downarrow\psi} & B \ar[d]^f \\ C \ar[ur]_\ell & & &C \ar[r]_g & A }}
\end{equation}
Pasting the canonical cone associated with $B \comma \ell$ onto the triangle~\eqref{eq:absRlifting} we obtain a comma cone which induces a functor $w\colon B \comma \ell \to f \comma g$ by the 1-cell induction property of $f\comma g$. Recall this means that $w$ makes the following pasting equality hold
\begin{equation}\label{eq:w-def-prop}
  \vcenter{\xymatrix@=1.2em{
    & {B\comma\ell}\ar[dl]_{p_1}\ar[dr]^{p_0} & \\
    {C} \ar[dr]_{g}\ar[rr]|*+{\scriptstyle\ell} && {B}\ar[dl]^{f} \\
    & A &  
    \ar@{} "1,2";"3,2" |(0.3){\Leftarrow\phi} |(0.7){\Leftarrow\lambda}
  }}
  \mkern 20mu = \mkern20mu
  \vcenter{\xymatrix@=1.2em{
    & {B\comma\ell}\ar[d]^{w}\ar@/_1.5ex/[ddl]_{p_1}\ar@/^1.5ex/[ddr]^{p_0} & \\
    & {f\comma g}\ar[dl]^{q_1}\ar[dr]_{q_0} & \\
    {C}\ar[dr]_{g} & & {B}\ar[dl]^{f} \\
    & {A} & 
    \ar@{} "2,2";"4,2" |{\Leftarrow\psi}
  }}
\end{equation}
and in particular may be regarded as being a 1-cell in the slice 2-category $\qCat_2\slice(C\times B)$ from $(p_1,p_0)\colon f\comma g \tfib C\times B$ to $(q_1,q_0)\colon B\comma\ell\tfib C\times B$.



\begin{prop}\label{prop:absliftingtranslation} The data of \eqref{eq:absRlifting} defines an absolute right lifting in $\qCat_2$ if and only if the induced map $w\colon B \comma \ell \to f \comma g$ of~\eqref{eq:w-def-prop} is an equivalence.
\end{prop} 
\begin{proof}
    For each pair of functors $b\colon X\to B$ and $c\colon X\to C$ as in~\eqref{eq:abs-lifting-property} observe that $\sq_{g,f}(c,b)$ (cf.\ Observation~\ref{obs:squares-set}) is simply the set of those 2-cells of the form depicted in the square on the left of the equality in~\eqref{eq:abs-lifting-property} and that $\sq_{\ell,B}(c,b)$ is the set of those 2-cells which inhabit the upper left triangle of the diagram to the right of that same equality. Define
  \begin{equation*}
    \xymatrix@C=8em{
      {\sq_{\ell,B}(c,b)}\ar[r]^{k^{\lambda}_{(c,b)}} &
      {\sq_{g,f}(c,b)}
    }
  \end{equation*}
  to be the function which takes each triangle in its domain and pastes it onto our candidate lifting diagram~\eqref{eq:absRlifting} to obtain a corresponding square as depicted in~\eqref{eq:abs-lifting-property}. This family of functions is natural in $(c,b)\colon X\to C\times B$ in the sense that they are the components of a natural transformation $k^\lambda$ between the functors
\[ \xymatrix{ (\pi^g_0)_*(\qCat_2\slice(C\times B))\op \ar@<1.5ex>[r]^-{\sq_{\ell,B}} \ar@<-1.5ex>[r]_-{\sq_{g,f}} \ar@{}[r]|-{\Downarrow k^\gamma} & \Set}\] 
of Lemma~\ref{lem:sq-as-a-functor}. By construction,  the triangle in~\eqref{eq:absRlifting} is an absolute right lifting if and only if $k^\lambda\colon \sq_{\ell,B}\Rightarrow \sq_{g,f}$ is a natural isomorphism.

Now   consider a commutative square of natural transformations
  \begin{equation*}
    \xymatrix@C=5em{
      {\pi^g_0(\hom'_{C\times B}(-,(p_1,p_0)))} 
      \ar[r]^{u\circ -}\ar[d]_{\cong} &
      {\pi^g_0(\hom'_{C\times B}(-,(q_1,q_0)))}
      \ar[d]^{\cong} \\
      {\sq_{\ell,B}}\ar[r]_{k} &
      {\sq_{g,f}}
    }
  \end{equation*}
  between presheaves on $(\pi^g_0)_*(\qCat_2\slice(C\times B))$, in which the vertical isomorphisms are those induced by the weakly universal comma cones of~\eqref{eq:comma-cones} as discussed in Lemma~\ref{lem:cpts-and-comma-2-cells}. Applying Yoneda's lemma and the definition of $(\pi^g_0)_*(\qCat_2\slice(C\times B))$, this square provides us with a canonical bijection between the set of natural transformations $k\colon\sq_{\ell,B}\Rightarrow\sq_{g,f}$ and the set of isomorphism classes of 1-cells
  \begin{equation}\label{eq:induced-u-from-nattrans-k}
    \xymatrix@=1em{
      {B\comma\ell}\ar@{->>}[dr]_(0.3){(p_1,p_0)}\ar[rr]^{u}
      && *+!L(0.5){f\comma g}\ar@{->>}[dl]^(0.3){(q_1,q_0)} \\
      & {C\times B}&
    }
  \end{equation}
  in $\qCat_2\slice(C\times B)$.  By the Yoneda lemma, $k\colon\sq_{\ell,B}\Rightarrow\sq_{g,f}$ is a natural isomorphism if and only if the corresponding $u\colon B\comma\ell\to f\comma g$ is an isomorphism in $(\pi^g_0)_*(\qCat_2\slice(C\times B))$. By Observation~\ref{obs:groupoid-components}, this holds if and only if $u$ is an equivalence in $\qCat_2\slice(C\times B)$. By Lemma~\ref{lem:proj-is-1-conservative},  this is the case if and only if $u$ is an equivalence in $\qCat_2$. 

In particular, the natural transformation $k^\lambda\colon\sq_{\ell,B}\Rightarrow\sq_{g,f}$ constructed from the 2-cell~\eqref{eq:absRlifting} corresponds  to the isomorphism class of those induced 1-cells $w\colon B\comma\ell\to f\comma g$ over $C\times B$ which satisfy the pasting identity displayed in~\eqref{eq:w-def-prop}. We have just shown that  the triangle in~\eqref{eq:absRlifting} is an absolute lifting diagram if and only if $k^\lambda\colon\sq_{\ell,B}\Rightarrow\sq_{g,f}$ is a natural isomorphism, which is the case  if and only if $w\colon B \comma \ell \to f \comma g$ is an equivalence. 
\end{proof}








\begin{rmk}
There is nothing in the proof of the Proposition \ref{prop:absliftingtranslation}, or in those of the results upon which it relies, which depends upon the vertex $X$ in~\eqref{eq:abs-lifting-property} being a quasi-category. The essential point here is that the space of maps out of any simplicial set $X$ taking values in a quasi-category is still a quasi-category. Consequently, we find that absolute lifting diagrams in $\qCat_2$ possess the factorisation property displayed in~\eqref{eq:abs-lifting-property} for 2-cells whose 0-cellular domains $X$ are general simplicial sets.  
\end{rmk}

For certain applications, it will be important to have a strengthened version of Proposition~\ref{prop:absliftingtranslation} which says that from {\em any\/} equivalence $B\comma \ell \simeq f \comma g$ fibred over $C\times B$ we may construct a 2-cell which displays $\ell$ as an absolute right lifting of $g$ through $f$. This result, Proposition~\ref{prop:absliftingtranslation2} below, proceeds directly from the following technical lemma:

\begin{lem}\label{lem:represented-nat-trans}
  For all natural transformations $k\colon\sq_{\ell,B}\Rightarrow\sq_{g,f}$ there exists a unique 2-cell $\lambda$ of the form depicted in~\eqref{eq:absRlifting} such that $k$ is equal to the natural transformation $k^\lambda$ defined by pasting a 2-cell in a triangle over $\ell$ with $\lambda$ to form a 2-cell in a square over $f$ and $g$.
\end{lem}

\begin{proof}
  A 2-cell in the triangle~\eqref{eq:absRlifting} is simply an element of $\sq_{g,f}(C,\ell)$, so we may construct our candidate 2-cell $\lambda$ from the natural transformation $k\colon\sq_{\ell,B}\Rightarrow\sq_{g,f}$ by applying it to the identity 2-cell  in $\sq_{\ell,B}(C,\ell)$; that is, we take $\lambda\defeq k_{(C,\ell)}(\id_\ell)$. 

  Lemma~\ref{lem:cpts-and-comma-2-cells} reveals that $\sq_{\ell,B}$ is a representable functor whose universal element is the 2-cell $\phi\in\sq_{\ell,B}(p_1,p_0)$ of the weakly universal cone~\eqref{eq:comma-cones} displaying $B\comma\ell$. So Yoneda's lemma tells us that in order to show that our original natural transformation $k$ is equal to $k^\lambda$ it is enough to check that they both map $\phi$ to the same element of $\sq_{g,f}(p_1,p_0)$.

To do this, first observe that the functor $i \colon C \to B\comma\ell$ defined in Lemma~\ref{lem:technicalsliceadjunction} can be regarded as a morphism in $(\pi^g_0)_*(\qCat_2\slice(C\times B))$. Its defining property, that $\phi i = \id_\ell$, may then be re-expressed as the equality $\sq_{\ell,B}(i)(\phi) = \id_\ell$ relating $\id_\ell\in\sq_{\ell,B}(C,\ell)$ and $\phi\in\sq_{\ell,B}(p_1,p_0)$. By naturality of $k$, this then allows us to obtain a similar relationship between the 2-cell $\lambda$  and the image $\mu\defeq k_{(p_1,p_0)}(\phi)$ of $\phi$ under $k$, as given by the following computation: $\sq_{g,f}(i)(k_{(p_1,p_0)}(\phi)) = k_{(C,\ell)}(\sq_{\ell,B}(i)(\phi)) = k_{(C,\ell)}(\id_\ell) = \lambda$. By the definition of the map $\sq_{g,f}(i)$, this relationship may be expressed as a pasting equality:
  \begin{equation}\label{eq:rel-mu-lambda}
    \vcenter{\xymatrix@=0.8em{
      & {C} \ar@{=}[dl]\ar[dr]^\ell & \\
      {C}\ar[dr]_g & {\scriptstyle\Leftarrow\lambda} &
      {B}\ar[dl]^f \\
      & {A} &
    }}
    \mkern20mu = \mkern20mu
    \vcenter{\xymatrix@=0.8em{
      & \save []+<0pt,1em>*+{C}\ar[d]^i\ar@{=}@/_1.5ex/[ddl]\ar@/^1.5ex/[ddr]^\ell \restore & \\
      & {B\comma\ell}\ar[dl]^{p_1}\ar[dr]_{p_0} & \\
      {C}\ar[dr]_g & {\scriptstyle\Leftarrow\mu} & 
      {B}\ar[dl]^f \\
      & {A} &
    }}
  \end{equation}

By definition, $k^\lambda$ acts on $\phi$ by pasting it to the 2-cell $\lambda$ as depicted in the diagram on the left hand side of the following computation:
  \begin{equation*}
    \vcenter{\xymatrix@=1em{
      & & {B\comma\ell}\ar[dl]_{p_1}
      \ar[dd]^{p_0}_{}="one"\\
      & {C} \ar@{=}[dl]\ar[dr]_(0.4)\ell 
      \ar@{} "one" |(0.6){\Leftarrow\phi} & \\
      {C}\ar[dr]_g & {\scriptstyle\Leftarrow\lambda} &
      {B}\ar[dl]^f \\
      & {A} &
    }}
    \mkern5mu = \mkern5mu
    \vcenter{\xymatrix@=0.8em{
      & & {B\comma\ell}\ar[dl]_{p_1}\ar[dddd]^{p_0} \\
      & {C}\ar[dd]^i\ar@/_1.5ex/@{=}[dddl]
      \ar@/^1.5ex/[dddr]^(0.3)\ell="one" 
      & \\ & & \\
      & {B\comma\ell}\ar[dl]^{p_1}\ar[dr]_{p_0} & \\
      {C}\ar[dr]_g & {\scriptstyle\Leftarrow\mu} & 
      {B}\ar[dl]^f \\
      & {A} &
      \ar@{}"1,3";"one"|(0.7){\Leftarrow\phi} 
    }}
    \mkern5mu = \mkern5mu
    \vcenter{\xymatrix@=0.8em{
      & & {B\comma\ell}\ar[dl]_{p_1}\ar[dddd]^{p_0} 
      \ar@{=}@/^1.5ex/[dddl]_{}="one"\\
      & {C}\ar[dd]^i\ar@/_1.5ex/@{=}[dddl]
      & \\ & & \\
      & {B\comma\ell}\ar[dl]^{p_1}\ar[dr]_{p_0} & \\
      {C}\ar[dr]_g & {\scriptstyle\Leftarrow\mu} & 
      {B}\ar[dl]^f \\
      & {A} &
      \ar@{}"2,2";"one"|(0.6){\Leftarrow\nu} 
    }}
    \mkern5mu = \mkern5mu
    \vcenter{\xymatrix@=0.8em{
      & {B\comma\ell}\ar[dl]_{p_1}\ar[dr]^{p_0} & \\
      {C}\ar[dr]_g & {\scriptstyle\Leftarrow\mu} & 
      {B}\ar[dl]^f \\
      & {A} &
    }}
  \end{equation*}
  To elaborate, the first step in this calculation is simply an application of the equality given in~\eqref{eq:rel-mu-lambda}. Its second step follows from the first of the defining properties of the unit $\nu\colon\id_{B\comma\ell}\Rightarrow i p_1$ of the adjunction $p_1\dashv i$ of Lemma~\ref{lem:technicalsliceadjunction}, those being that $p_0\nu=\phi$ and $p_1\nu=\id_{p_1}$. The third of these equalities follows on observing that the pasting depicted on its left is simply the horizontal composite of the 2-cells $\mu$ and $\nu$, which may be expressed as the vertical composite $qp_1\nu\cdot \mu$ in which the second factor is an identity by the second defining property of $\nu$. 
  
  In other words, this calculation demonstrates that $k^\lambda_{(p_1,p_0)}(\phi)=\mu$ which is in turn equal to $k_{(p_1,p_0)}(\phi)$, by definition. Consequently, Yoneda's lemma tells us that $k = k^\lambda$ as required. Finally, the fact that $\lambda$ is the unique 2-cell with the property that $k=k^\lambda$ follows immediately from the patent fact that $\lambda = k^\lambda_{(C,\ell)}(\id_\ell)$.
\end{proof}

As an immediate corollary, we have the following important result:

\begin{prop}\label{prop:absliftingtranslation2} Suppose we are given functors $f\colon B\to A$, $g\colon C\to A$, and $\ell\colon C\to B$ of quasi-categories. Then the construction depicted in~\eqref{eq:w-def-prop} provides us with a bijection between 2-cells of the form 
\begin{equation}\label{eq:absRlifting.2}
  \xymatrix{ \ar@{}[dr]|(.7){\Downarrow\lambda} & B \ar[d]^f \\ C \ar[r]_g \ar[ur]^\ell & A}
\end{equation}
and isomorphism classes of 1-cells
\begin{equation}\label{eq:induced-from-lambda}
  \xymatrix@=1em{
    {B\comma\ell}\ar@{->>}[dr]_(0.3){(p_1,p_0)}\ar[rr]^{w}
    && *+!L(0.5){f\comma g}\ar@{->>}[dl]^(0.3){(q_1,q_0)} \\
    & {C\times B}&
  }
\end{equation}
in $\qCat_2\slice(C\times B)$. Furthermore, this 2-cell $\lambda$ displays $\ell$ as an absolute right lifting of $g$ through $f$ if and only if any representative $w$ of the corresponding isomorphism class of functors is an equivalence.
\end{prop}

\begin{proof}
Lemma \ref{lem:represented-nat-trans} provides a canonical bijection between 2-cells \eqref{eq:absRlifting.2} and natural transformations $\sq_{\ell,B} \To \sq_{g,f}$. 
The proof of Proposition \ref{prop:absliftingtranslation} establishes a canonical bijection between natural transformations $\sq_{\ell,B} \To \sq_{g,f}$ and isomorphism classes of 1-cells \eqref{eq:induced-from-lambda}. Proposition \ref{prop:absliftingtranslation} then concludes that $\lambda$ displays $\ell$ as an absolute right lifting of $g$ through $f$ if and only if any representative $w$ of the corresponding isomorphism class of functors is an equivalence.
\end{proof}

As a special case,  if $f\comma A$ and $B \comma u$ are equivalent over $A \times B$, then $f$ is left adjoint to $u$.

\begin{proof}[Proof of Proposition~\ref{prop:adjointequivconverse}]
If $f\comma A$ and $B \comma u$ are equivalent over $A \times B$, then Proposition \ref{prop:absliftingtranslation2} provides us with a corresponding 2-cell $\epsilon\colon fu\Rightarrow\id_A$, which displays $u$ as an absolute right lifting of $\id_A$ through $f$. By Example~\ref{ex:adjasabslifting}, this provides us with enough information to conclude that $f$ is left adjoint to $u$ with counit $\epsilon$. 
\end{proof}

  A second characterisation of absolute right liftings in $\qCat_2$ relates them to the possession of terminal objects by the fibres of $q_1\colon f\comma g\tfib C$. To explain this relationship, start by applying Observation~\ref{obs:1cell-ind-uniqueness-reloaded} to show that arbitrary pairs $(\ell,\lambda)$ as depicted in~\eqref{eq:absRlifting.2} correspond to isomorphism classes of functors
  \begin{equation*}
    \xymatrix@=0.8em{
      {C}\ar[dr]_-{(C,\ell)}\ar[rr]^{t}
      && *+!L(0.5){f\comma g}\ar@{->>}[dl]^-{(q_1,q_0)} \\
      & {C\times B} &
    }
  \end{equation*}
  over $C\times B$ defined by 1-cell induction
  \begin{equation}\label{eq:t-for-lim-def-prop}
    \vcenter{\xymatrix@=0.8em{
      & \save []+<0pt,1em>*+{C}\ar[d]^-{t}\ar@{=}@/_1.5ex/[ddl]
      \ar@/^1.5ex/[ddr]^{\ell}\restore & \\
      & {f\comma g}\ar[dr]_{q_0}\ar[dl]^{q_1} & \\
      {C}\ar[dr]_{g} & {\scriptstyle\Leftarrow\psi} &
      {B}\ar[dl]^{f} \\
      & {A} &
    }}
    \mkern20mu = \mkern20mu
    \vcenter{\xymatrix@=0.7em{
      & {C}\ar@{=}[dl]\ar[dr]^{\ell} & \\
      {C}\ar[dr]_{g} & {\scriptstyle\Leftarrow\lambda} & {B}\ar[dl]^{f} \\
      & {A} &
    }}
  \end{equation}
The following proposition relates the universal properties of pairs $(\ell,\lambda)$ and corresponding functors $t$.

\begin{prop}\label{prop:right.liftings.as.fibred.terminal.objects} The 2-cell $\lambda$ shown in~\eqref{eq:absRlifting.2} displays $\ell$ as an absolute right lifting of the functor $g$ through $f$ if and only if the induced functor $t\colon C\to f\comma g$  of \eqref{eq:t-for-lim-def-prop} features in a fibred adjunction:
  \begin{equation}\label{eq:fibred.terminal.2}
    \xymatrix@=1.2em{
      {C}\ar@{=}[dr]\ar@/_1.5ex/[rr]_-{t}^-{}="one"
      & & *+!L(0.5){f\comma g}\ar@{->>}[dl]^-{q_1}
      \ar@/_1.5ex/[ll]_-{q_1}^-{}="two" \\
      & {C} &
      \ar@{}"one";"two"|{\bot}
    }
  \end{equation}
that is, if and only if $t$ defines a right adjoint right inverse to the isofibration $q_1$.
\end{prop}
\begin{proof}
First assume that the triangle in~\eqref{eq:absRlifting.2} is an absolute right lifting diagram and apply Proposition~\ref{prop:absliftingtranslation} to show that the associated functor $w\colon B\comma\ell\to f\comma g$ is a fibred equivalence with equivalence inverse $w'$. Applying Proposition~\ref{prop:equivtoadjoint} in $\qCat_2/(C\times B)$ and Corollary~\ref{cor:missed-lemma}, this may be promoted to a fibred adjoint equivalence $w'\dashv w$ over $C\times B$. Its pushforward along the projection $C\times B\tfib C$ is an adjoint equivalence fibred over $C$. 

Example~\ref{ex:fibred-technical-slice-adjunction} provides us with an adjunction $p_1\dashv i\colon C\to B\comma\ell$  also fibred over $C$. Composing these, we obtain an adjunction $p_1w'\dashv wi\colon C\to f\comma g$  again fibred over $C$. From the defining properties of $w$ and $i$, as described in~\eqref{eq:w-def-prop} and~\eqref{eq:technicalsliceadjunction}, it is clear that $wi$ is a 1-cell induced over $\psi$ by the comma cone $\lambda$, and so we may infer, by Observation~\ref{obs:1cell-ind-uniqueness-reloaded}, that it is isomorphic to $t$ over $C$. Furthermore, $w'$ is fibred over $C\times B$ so $p_1w' = q_1$, and the fibred adjunction $p_1w'\dashv wi$ reduces to a fibred adjunction $q_1\dashv t$ as required.

   For the converse, assume that we have a fibred adjunction of the form given in~\eqref{eq:fibred.terminal.2}. We must show that for any 2-cell $\mu$ 
  \begin{equation}\label{eq:limit-lifting-prop}
    \vcenter{\xymatrix@=1.5em{
      {Y}\ar[r]^{b}\ar[d]_-{c} &
      {B}\ar[d]^-{f} \\
      {C}\ar@{}[ur]|{\Downarrow\mu}\ar[r]_-{g} &
      {A}
    }}
    \mkern20mu = \mkern20mu
    \vcenter{\xymatrix@=1.5em{
      {Y}\ar[r]^{b}\ar[d]_-{c} &
      {B}\ar[d]^-{f} \\
      {C}\ar[ur]|*+<2pt>{\scriptstyle\ell}="one"\ar[r]_-{g} &
      {A}
      \ar@{} "1,1";"one"|(0.6){\Downarrow \exists!\tau}
      \ar@{} "one";"2,2"|(0.4){\Downarrow\lambda}
    }}
  \end{equation}
there exits a unique 2-cell $\tau$ which makes this pasting equation hold. 

  To do this, start by applying the 1-cell induction property of $f\comma g$ to the comma cone $\mu$ to give a functor $m\colon Y\to f\comma g$ so that
  \begin{equation}\label{eq:m-def-prop}
    \vcenter{\xymatrix@=0.7em{
      & {Y}\ar[dd]^{m}\ar@/_1.5ex/[dddl]_{c}
      \ar@/^1.5ex/[dddr]^{b} & \\
      &&\\
      & {f\comma g}\ar[dr]_{q_0}\ar[dl]^{q_1} & \\
      {C}\ar[dr]_{g} & {\scriptstyle\Leftarrow\psi} &
      {B}\ar[dl]^{f} \\
      & {A} &
    }}
    \mkern20mu = \mkern20mu
    \vcenter{\xymatrix@=0.7em{
      & {Y}\ar[dl]_{c}\ar[dr]^{b} & \\
      {C}\ar[dr]_{g} & {\scriptstyle\Leftarrow\mu} & {B}\ar[dl]^{f} \\
      & {A} &
    }}
  \end{equation}
A 2-cell $\tau\colon b\Rightarrow \ell c$ satisfying~\eqref{eq:limit-lifting-prop} gives rise to a 
  2-cell $\nu$ from $m\colon Y\to f\comma g$ to the composite functor $tc\colon Y\to f\comma g$ over $C$ by 2-cell induction: notice that the fact that we require $\nu$ to be a 2-cell over $C$ means that the equation $q_1\nu = \id_{p_1}$ must hold, which tells us that the second 2-cell of its inducing pair must  be $\id_{p_1}$.  The compatibility condition expressed in~\eqref{eq:comma-ind-2cell-compat} for the pair $(\tau,\id_{p_1})$ reduces to the pasting equality~\eqref{eq:limit-lifting-prop} by direct application of the defining properties for $m$ and $t$ given in~\eqref{eq:m-def-prop} and~\eqref{eq:t-for-lim-def-prop}. Conversely, if $\nu\colon m\Rightarrow tc$ is any 2-cell over $C$ then the whiskered 2-cell $\tau\defeq q_0\nu\colon b \Rightarrow \ell c$ satisfies~\eqref{eq:limit-lifting-prop}.

Extending Definition~\ref{defn:enriched-slice}, the map $c$ defines a 2-functor $\hom'_C(c,-)\colon\qCat_2\slice C \to \Cat_2$. As in Observation~\ref{obs:isofib-section.fibred.adjunction}, this 2-functor carries the postulated fibred adjunction $q_1\dashv t$ to a terminal object  $tc\colon Y\to f\comma g$ in the hom-category $\hom'_C(c,q_1)$. It follows that there exists a unique 2-cell $\nu\colon m\Rightarrow tc$ over $C$; hence, the 2-cell $q_0\nu\colon b \Rightarrow \ell c$ provides us with a solution to~\eqref{eq:limit-lifting-prop}. Furthermore if $\tau\colon b\Rightarrow \ell c$ is any other 2-cell which solves that pasting equality then the 2-cell it induces must necessarily be the unique such $\nu\colon m\Rightarrow tc$, and consequently we have the equality $\tau = q_0\nu$. This demonstrates that the solution to~\eqref{eq:limit-lifting-prop} is unique.
\end{proof}

\begin{obs}\label{obs:right.liftings.as.fibred.terminal.objects}
  The upshot of Proposition \ref{prop:right.liftings.as.fibred.terminal.objects} is that if the projection $q_1\colon f\comma g\tfib C$ has a fibred right adjoint~\eqref{eq:fibred.terminal.2}, then we may compose it with the weakly universal cone associated with $f\comma g$ to obtain an absolute right lifting of $g$ through $f$.
\end{obs}

This characterisation of absolute right liftings leads to the following generalisation of a classical result:

\begin{prop}\label{prop:translated.lifting}
  There exists an absolute right lifting
  \begin{equation}\label{eq:orig.lifting}
    \xymatrix{ \ar@{}[dr]|(.7){\Downarrow\lambda} & B \ar[d]^f \\ C \ar[r]_g \ar[ur]^\ell & A}
  \end{equation}
  if and only if there exists an absolute right lifting
  \begin{equation}\label{eq:translated.lifting}
    \xymatrix{ \ar@{}[dr]|(.7){\Downarrow\hat\lambda} & {f\comma A} \ar@{->>}[d]^{p_1} \\ C \ar[r]_g \ar[ur]^{\hat\ell} & A}
  \end{equation}
  Furthermore, the 2-cell $\hat\lambda$ is necessarily an isomorphism and $\hat\ell$ may be chosen so as to make it an identity.
\end{prop}

\begin{proof} Write $(r_1,r_0)\colon p_1\comma g \tfib C \times f\comma A$ for the projection defined by the comma quasi-category construction~\ref{def:comma-obj}. Directly from this definition, there exists a canonical isomorphism $p_1\comma g \cong A\comma g\times_A f\comma A$ commuting with the projections to $C \times f \comma A$. Applying Proposition~\ref{prop:right.liftings.as.fibred.terminal.objects}, our aim is to use a fibred right adjoint to $q_1$ to construct a fibred right adjoint to $r_1$ and vice versa. 
  \begin{equation}\label{eq:ran.as.fibred.adj}
 \xymatrix@=1.2em{
      {C}\ar@{=}[dr]\ar@/_1.5ex/[rr]_-{t}^-{}="one"
      & & *+!L(0.5){f\comma g}\ar@{->>}[dl]^-{q_1}
      \ar@/_1.5ex/[ll]_-{q_1}^-{}="two" \\
      & {C} &
      \ar@{}"one";"two"|{\bot}
    }    \qquad  \quad  
     \xymatrix@=1.2em{
      {C}\ar@{=}[dr]\ar@/_1.5ex/[rr]^-{}="one"
      & & *+!L(0.5){p_1\comma g}\ar@{->>}[dl]^-{r_1}
      \ar@/_1.5ex/[ll]_-{r_1}^-{}="two" \\
      & {C} &
      \ar@{}"one";"two"|{\bot}
    }
  \end{equation}

To that end, pull back  the ``composition--identity'' fibred adjunctions~\eqref{eq:comp.ident.adj} along the functor $g\times p_1\colon C\times f\comma A\to A\times A$ to obtain a pair of adjunctions 
  \begin{equation}\label{eq:adj.for.trans}
    \xymatrix@C=14em{
      {p_1\comma g\cong A\comma g\times_A f\comma A}
      \ar[]!R(0.72);[r]|*+{\scriptstyle m} &
      {f\comma g}
      \ar@/^2.5ex/[l]!R(0.72)^{i_1}_{}="l" \ar@/_2.5ex/[l]!R(0.72)_{i_0}^{}="u"
      \ar@{} "u";"l" |(0.2){\bot} |(0.8){\bot} 
    }
  \end{equation}
  fibred over $C\times f\comma A$. Pushing forward along the projection $C\times f\comma A\tfib C$, we may regard the adjunctions \eqref{eq:adj.for.trans} as fibred over $C$ with respect to the isofibrations $r_1\colon p_1\comma g\tfib C$ and $q_1\colon f\comma g\tfib C$. 

  With this adjunction in our armoury our result is essentially immediate. If we are given the left-hand fibred adjunction~\eqref{eq:ran.as.fibred.adj} witnessing the existence of the absolute right lifting of $g$ through $f$ then we may compose it with the lower fibred adjunction of~\eqref{eq:adj.for.trans} to obtain the right-hand fibred adjunction~\eqref{eq:ran.as.fibred.adj}, providing us with an absolute right lifting of $g$ through $p_1$. Conversely, we may go back in the other direction by composing the right-hand fibred adjunction with the upper fibred adjunction of~\eqref{eq:adj.for.trans} to obtain an adjunction of the type on the left of~\eqref{eq:ran.as.fibred.adj}.

  All that remains is to check the final clause of the proposition. To that end, Observation~\ref{obs:right.liftings.as.fibred.terminal.objects} tells us that we may construct an absolute right lifting of $g$ through $p_1$ by composing the right adjoint functor 
\begin{equation*}
  \xymatrix{
    {C}\ar[r]^-{t} & {f\comma g}\ar[r]^-{i_1} & {p_1\comma g}
  }
\end{equation*}
where $t$ is the fibred right adjoint of~\eqref{eq:ran.as.fibred.adj}, with the comma cone that displays $p_1\comma g$ as a weak comma object. By construction, the 2-cell of that cone is the restriction
\begin{equation*}
  \xymatrix{
    {p_1\comma g} \ar[r] & {A\comma g}\ar[r] & {A^\cattwo}\ar@{}[]!R(0.5);[rr]!L(0.5)|{\Downarrow}\ar@/^1.5ex/[]!R(0.5);[rr]!L(0.5)^{p_0}\ar@/_1.5ex/[]!R(0.5);[rr]!L(0.5)_{p_1} && {A}
  }
\end{equation*}
of the 2-cell which displays $A^\cattwo$ as a weak cotensor. Hence, the 2-cell $\hat\lambda$ constructed by Proposition~\ref{prop:right.liftings.as.fibred.terminal.objects}  is equal to 
\begin{equation*}
  \xymatrix{
    {C}\ar[r]^-{t} & {f\comma g}\ar[r] & {A^\cattwo}\ar[r]^-{i_1} & {A^\cattwo\times_A A^\cattwo}\ar[r]^-{\pi_1} & {A^\cattwo}\ar@{}[]!R(0.5);[rr]!L(0.5)|{\Downarrow}\ar@/^1.5ex/[]!R(0.5);[rr]!L(0.5)^{p_0}\ar@/_1.5ex/[]!R(0.5);[rr]!L(0.5)_{p_1} && {A}
    }
\end{equation*}
  and, consulting the definition of $i_1$ given in Example~\ref{ex:comp.ident.adj}, it is straightforward to verify that the composite of the last three cells above is equal to the identity 2-cell on $p_1\colon A^\cattwo\tfib A$. Consequently, the 2-cell in our absolute right lifting is also an identity as required.
\end{proof}

\subsection{Limits and colimits as absolute lifting diagrams}

A diagram in a quasi-category $A$ is just a map $d \colon X \to A$ of simplicial sets. In particular, when $X$ is the nerve of a small category and $A$ is the homotopy coherent nerve of a locally Kan simplicial category, a diagram is precisely a \emph{homotopy coherent diagram} in the sense of Cordier, Porter, Vogt, and others \cite{Cordier:1986:HtyCoh}.

\begin{ntn}
  From here on  we use $c\colon A\to A^X$ to denote the {\em constant diagram map}: the adjoint transpose of the projection map $\pi_A\colon A\times X\to A$. Furthermore, we shall notationally identify functors $f\colon X\to A$ and natural transformations $\alpha\colon f\Rightarrow g\colon X\to A$ with their adjoint transposes $f\colon\Del^0\to A^X$ and $\alpha\colon f\Rightarrow g\colon \Del^1\to A^X$ respectively.
\end{ntn}

\begin{defn}\label{defn:limit} We say that an absolute right lifting diagram 
    \begin{equation}\label{eq:genericlimit}
      \xymatrix{ \ar@{}[dr]|(.7){\Downarrow\lambda} & A \ar[d]^c \\ \Delta^0 \ar[ur]^\ell \ar[r]_d& A^X}
    \end{equation}
    {\em displays the vertex $\ell\in A$ as the limit of the diagram $d\colon X\to A$}. The 2-cell $\lambda$, which we may equally regard as going from the constant diagram  $X \xrightarrow{!} \Del^0 \xrightarrow{\ell} A$ to $d$, is called the \emph{limiting cone}. Dually, we say that an absolute left lifting diagram 
  \begin{equation}\label{eq:genericcolimit} 
    \xymatrix{ \ar@{}[dr]|(.7){\Uparrow\lambda} & A \ar[d]^c \\ \Delta^0 \ar[ur]^\ell \ar[r]_d & A^X}
  \end{equation} 
  {\em displays the vertex $\ell\in A$ as the colimit of the diagram $d\colon X\to A$}.  Here again the 2-cell $\lambda$, from $d$  to the constant diagram $X \xrightarrow{!} \Del^0 \xrightarrow{\ell} A$, is called the \emph{colimiting cone}.
\end{defn}

\begin{rmk}
For the most part in what follows, we shall present our results in terms of limits and absolute right liftings only. Of course, these arguments all admit the obvious duals which apply to colimits and absolute left liftings. Indeed the results of this section and the last are almost exclusively matters of formal 2-category theory. Their duals follow by re-interpreting these arguments in the dual 2-category $\qCat_2\co$ obtained by reversing the direction of all 2-cells.
\end{rmk}

A special case of Proposition~\ref{prop:right.liftings.as.fibred.terminal.objects} gives an alternative definition of limits and colimits in a quasi-category.

\begin{prop}\label{prop:limits.as.terminal.objects} A limit of $d \colon X \to A$ is a terminal object in the quasi-category $c\comma d$, and conversely a terminal object defines a limit.
\end{prop}
\begin{proof}
A limiting cone defines a vertex in the comma quasi-category $c\comma d$ by 1-cell induction; Lemma~\ref{lem:1cell-ind-uniqueness} and Proposition~\ref{prop:right.liftings.as.fibred.terminal.objects} tell us this vertex is unique up to isomorphism and terminal. Conversely, Proposition~\ref{prop:right.liftings.as.fibred.terminal.objects} implies that the data of a terminal object in $c\comma d$ defines a limit object $\ell\in A$ and a limiting cone $\lambda$ in the sense of Definition~\ref{defn:limit}.
\end{proof}

An important corollary of Proposition~\ref{prop:limits.as.terminal.objects} is that our definition of limit agrees with the existing ones in the literature. As discussed in section \ref{subsec:join} and seen already in the proof of Proposition \ref{prop:terminalconverse}, this proof makes use of an equivalence between Joyal's slice construction and our comma construction. In this case we show that the quasi-category of cones $c \comma d$ over a diagram $d \colon X \to A$ is equivalent to Joyal's quasi-category of cones $\slicer{A}{d}$, recalled in \ref{defn:slices}. 

\begin{lem}\label{lem:cone-equiv-fatcone} For any diagram $d \colon X \to A$ in a quasi-category $A$, there is an equivalence
\[ \xymatrix{ \slicer{A}{d} \ar[rr]^\simeq \ar@{->>}[dr]_{\pi} & & c \comma d \ar@{->>}[dl]^{q_0} \\ & A}\] of quasi-categories over $A$.
\end{lem}
\begin{proof}
As in the proof of Lemma \ref{lem:slice-equiv-comma}, we will demonstrate an isomorphism $c \comma d \cong \fatslicer{A}{d}$ over $A$ between the quasi-category of cones and the fat slice construction on $d \colon X \to A$ defined in \ref{defn:fat-slices}. Via this isomorphism, the equivalence $\slicer{A}{d}\simeq c \comma d$ is a special case of the equivalence of Proposition~\ref{prop:slice-fatslice-equiv}.

To establish the isomorphism, it suffices to show that $c \comma d$ has the universal property that defines $\fatslicer{A}{d}$.  By adjunction, a map $Y \to \fatslicer{A}{d}$ corresponds to a commutative square, as displayed on the left:
\[ \vcenter{   
      \xymatrix@R=2em@C=4em{
        {(Y\times X)\sqcup(Y\times X)}\ar[r]^-{\pi_Y\sqcup\pi_X}
        \ar[d] &
        {Y\sqcup X}\ar[d]^{\langle f, d\rangle} \\
        {Y\times\Del^1\times X} \ar[r]_-{k} &
        {A}
      }
} \qquad \leftrightsquigarrow \qquad \vcenter{      \xymatrix@R=2em@C=4em{
        {Y}\ar[r]^{k}\ar[d]_{(!,f)} & {(A^X)^{\Del^1}} 
        \ar[d] \\
        { \Del^0\times A}\ar[r]_-{d\times c} & {A^X\times A^X}
      }}\]
which transposes to the commutative square displayed on the right. The data of the right-hand square is precisely that of a map $Y \to c \comma d$ by the universal property of the pullback \ref{def:comma-obj} defining the comma quasi-category.
\end{proof}

Joyal defines a limit of a diagram $d \colon X \to A$ to be a terminal vertex $t$ in the slice quasi-category $\slicer{A}{d}$, thought of as the ``quasi-category of cones'' over $d$.  If $\pi\colon \slicer{A}{d}\tfib A$ denotes the canonical projection then such a limiting cone displays $\ell\defeq \pi t$ as a limit of $d$. 

\begin{prop}\label{prop:limits.are.limits}
The notion of limit and limit cone introduced in Definition \ref{defn:limit} is equivalent to the notion of limit and limit cone introduced by Joyal in \cite[4.5]{Joyal:2002:QuasiCategories}.
\end{prop}
\begin{proof}
By Proposition~\ref{prop:limits.as.terminal.objects} tells us that our definition can be recast in a corresponding form: as a terminal vertex $t$ in  the comma quasi-category $c\comma d$. Our ``quasi-category of cones'' is  equipped with a projection $q_0 \colon c \comma d \to A$, and by Proposition~\ref{prop:right.liftings.as.fibred.terminal.objects} such a limiting cone displays $\ell\defeq q_0t$ as a limit of $d$.

Lemma \ref{lem:cone-equiv-fatcone} supplies an equivalence over $A$ between the quasi-category of cones and Joyal's slice quasi-category $\slicer{A}{d}$. Applying Proposition~\ref{prop:terminaldefn}, our preservation result for terminal objects, we see that this equivalence maps a limit cone in Joyal's sense to a limit cone in our sense and vice versa. Furthermore, since this is an equivalence over $A$, it follows that these corresponding cones display the same vertex $\ell$ as the limit of $d$.
\end{proof}

\begin{defn}\label{defn:families.of.diagrams}
  A {\em family $k$ of diagrams of shape $X$\/} in a quasi-category $A$ is simply a functor $k\colon K\to A^X$. In many cases, $K$ will  be the full sub-quasi-category of $A^X$ determined by some set of diagrams and $k$ will be the inclusion $K\inc A^X$.

  We say that $A$ {\em admits limits of the family of diagrams $k\colon K\to A^X$\/} if there exists an absolute right lifting diagram:
\begin{equation}\label{eq:limits.of.a.family}
      \xymatrix{ \ar@{}[dr]|(.7){\Downarrow\lambda} & A \ar[d]^c \\ K \ar[ur]^\lim \ar[r]_k & A^X}
\end{equation}
  Furthermore, we shall simply say that $A$ admits {\em all limits of shape\/} $X$ if it admits limits of the family of all diagrams $A^X$. 

  A diagram $d\colon X\to A$ is said to be a member of the family $k$ if it is a vertex in the image of $k$, that is to say if there is a vertex $\bar{d}\in K$ such that $d=k\bar{d}$. It is trivially verified, directly from the universal property of absolute right liftings, that if $A$ admits limits of the family of diagrams $k$ and $d$ is a member of the family $k$ then the restricted triangle
  \begin{equation*}
      \xymatrix{ \ar@{}[dr]|(.7){\Downarrow\lambda\bar{d}} & A \ar[d]^c \\ \Del^0 \ar[ur]^{\lim \bar{d}} \ar[r]_d & A^X}
  \end{equation*}
  is again an absolute right lifting, thus providing us with a limit of individual diagram $d$. Our use of the adjective ``absolute'' here coincides with its usual meaning: absolute lifting diagrams are preserved by pre-composition by all functors.
\end{defn}

This result has the following converse, whose proof we delay to section~\ref{sec:pointwise}:

\begin{prop}\label{prop:families.of.diagrams}
  If $A$ admits the limit of each individual diagram $d\colon X\to A$ in the family $k\colon K\to A^X$ then it admits limits of the family of diagrams $k$.
\end{prop}

As a special case of Example \ref{ex:adjasabslifting}:

\begin{prop}\label{prop:limitsasadjunctions} A quasi-category $A$ has all limits of shape $X$ if and only if there exists an adjunction \[ \adjdisplay c -| \lim : A^X -> A.\]
\end{prop}

A key advantage of our 2-categorical definition of (co)limits in any quasi-category is that it permits us to use standard 2-categorical arguments to give easy proofs of the expected categorical theorems.

\begin{prop}\label{prop:RAPL} Right adjoints preserve limits.
\end{prop}

Our proof will closely follow the classical one. Given a diagram $d\colon X \to A$ and a right adjoint $u \colon A \to B$ to some functor $f$, a cone with summit $b$ over $ud$ transposes to a cone with summit $fb$ over $d$, which factorises uniquely through the limit cone. This factorisation transposes back across the adjunction to show that the image of the limit cone under $u$ defines a limit over $ud$.

\begin{proof}
Suppose that $A$ admits limits of a family of diagrams $k\colon K\to A^X$ as witnessed by an absolute right lifting diagram~\eqref{eq:limits.of.a.family}. Given an adjunction $f \dashv u$, and hence by Proposition \ref{prop:expadj} an adjunction $f^X \dashv u^X$, we must show that \[\xymatrix{ \ar@{}[dr]|(.7){\Downarrow\lambda} & A \ar[d]^-{c} \ar[r]^u & B \ar[d]^-{c} \\ K \ar[ur]^\lim \ar[r]_k& A^X \ar[r]_{u^X} & B^X}\] is an absolute right lifting diagram. Given a square
\[\xymatrix{ Y \ar[d]_{a} \ar[rr]^b \ar@{}[drr]|{\Downarrow\chi} & & B \ar[d]^-{c} \\ K \ar[r]_-{k} & A^X \ar[r]_{u^X} & B^X} \] we first transpose across the adjunction, by composing with $f$ and the counit. 
\[\vcenter{\xymatrix{ Y \ar[d]_-{a} \ar[rr]^b \ar@{}[drr]|{\Downarrow\chi} & & B \ar[d]^-{c} \ar[r]^f & A \ar[d]^-{c}  \\ K \ar[r]_-{k} & A^X \ar@{=}@/_3.5ex/[rr]^{\Downarrow\epsilon^X} \ar[r]^{u^X} & B^X \ar[r]^{f^X} & A^X}} = \vcenter{\xymatrix{ Y \ar@{}[drr]|(.3){\exists !\Downarrow\zeta}|(.7){\Downarrow\lambda} \ar[d]_-{a} \ar[r]^b & B \ar[r]^f & A \ar[d]^-{c} \\ K \ar[urr]_(0.4){\lim} \ar[rr]_{k} & & A^X}} \] Applying the universal property of the absolute right lifting diagram~\eqref{eq:limits.of.a.family} produces a factorisation $\zeta$, which may then be transposed back across the adjunction by composing with $u$ and the unit.
\[  \vcenter{\xymatrix{ Y \ar@{}[drr]|(.3){\exists !\Downarrow\zeta}|(.7){\Downarrow\lambda} \ar[d]_-{a} \ar[r]^b & B \ar@{=}@/^3.5ex/[rr]_{\Downarrow\eta} \ar[r]|f & A \ar[d]^-{c} \ar[r]_u & B \ar[d]^-{c} \\ K \ar[urr]_(0.4){\lim} \ar[rr]_-{k} & & A^X \ar[r]_{u^X} & B^X}}= \vcenter{\xymatrix{ Y \ar[d]_-{a} \ar[rr]^b \ar@{}[drr]|{\Downarrow\chi} & & B \ar[d]^-{c} \ar@{=}@/^3.5ex/[rr]_{\Downarrow\eta} \ar[r]_f & A \ar[d]^-{c} \ar[r]_u & B \ar[d]^-{c}  \\ K \ar[r]_-{k} & A^X \ar@{=}@/_3.5ex/[rr]^{\Downarrow\epsilon^X} \ar[r]^{u^X} & B^X \ar[r]^{f^X} & A^X \ar[r]_{u^X} & B^X}}  \] \[ = \vcenter{\xymatrix{ Y \ar[d]_-{a} \ar[rr]^b \ar@{}[drr]|{\Downarrow\chi} & & B \ar[d]^-{c} \ar@{=}@/^3.5ex/[rr]  &  & B \ar[d]^-{c}  \\ K \ar[r]_-{k} & A^X \ar@{=}@/_3.5ex/[rr]^{\Downarrow\epsilon^X} \ar[r]^{u^X} & B^X \ar[r]|{f^X}  \ar@{=}@/^3.5ex/[rr]_{\Downarrow\eta^X}& A^X \ar[r]_{u^X} & B^X}}  = \vcenter{\xymatrix{ Y \ar[d]_-{a} \ar[rr]^b \ar@{}[drr]|{\Downarrow\chi} & & B \ar[d]^-{c} \\ K \ar[r]_-{k} & A^X \ar[r]_{u^X} & B^X}}\] Here the second equality is immediate from the definition of $\eta^X$ and the third is by the triangle identity defining the adjunction $f^X \dashv u^X$. The pasted composite of $\zeta$ and $\eta$ is the desired factorisation of $\chi$ through $\lambda$. 

The proof that this factorisation is unique, which again parallels the classical argument, is left to the reader: the essential point is that the transposes are unique.
\end{proof}

\begin{cor}\label{cor:equivprescolim} Equivalences preserve limits and colimits.
\end{cor}
\begin{proof} This follows immediately from Propositions \ref{prop:RAPL} and \ref{prop:equivtoadjoint}.
\end{proof}

\begin{obs}\label{obs:transpose-abs-lifting}
  Under the 2-adjunction $-\times Y\dashv (-)^Y$ triangles of the form 
  \begin{equation}\label{eq:untransposed}
    \xymatrix{
      & {B}\ar[d]^-{f} \\
      {K\times Y} \ar[ur]^{\ell} \ar[r]_-{k} 
      & {A} \ar@{}[ul]|(0.35){\Downarrow \lambda}
    }
  \end{equation}
  correspond to transposed diagrams:
  \begin{equation}\label{eq:transposed}
    \xymatrix{
      & {B^Y}\ar[d]^-{f^Y} \\
      {K} \ar[ur]^{\hat\ell} \ar[r]_-{\hat{k}} 
      & {A^Y} \ar@{}[ul]|(0.35){\Downarrow \hat\lambda}
    }
  \end{equation}
  Furthermore, if the first of these triangles is an absolute right lifting then so is the second one. To prove this, we must show that we can uniquely factorise the 2-cell in a square \[ \xymatrix{ Z \ar[d]_-{u} \ar[r]^-{v} \ar@{}[dr]|{\Downarrow\alpha} & B^Y \ar[d]^-{f^Y} \\ K \ar[r]_-{\hat{k}} & A^Y} \] through the 2-cell $\hat\lambda$ in~\eqref{eq:transposed}. Transposing that square under the 2-adjunction, we obtain the square on the left of the following diagram: \[ \vcenter{\xymatrix{ Z \times Y \ar[d]_-{\tilde{u}} \ar[r]^-{\tilde{v}} \ar@{}[dr]|{\Downarrow \tilde{\alpha}} & B \ar[d]^-{f} \\ K\times Y \ar[r]_-{k} & A}} = \vcenter{\xymatrix{ Z \times Y \ar[d]_-{\tilde{u}} \ar[r]^-{\tilde{v}} \ar@{}[dr]|(.3){\exists !\Downarrow}|(.7){\Downarrow\lambda} & B \ar[d]^-{f} \\ K\times Y \ar[r]_-{k} \ar[ur]_(0.4){\ell} & A}}\] The unique factorisation on the right arises from the universal property of the absolute lifting diagram~\eqref{eq:untransposed}, and its transpose provides the desired unique factorisation of $\alpha$.
\end{obs}

\begin{prop}[pointwise limits in functor quasi-categories]\label{prop:pointwise-limits-in-functor-quasi-categories}
  If a quasi-category $A$ admits limits of the family of diagrams $k\colon K\to A^X$ of shape $X$ then the functor quasi-category $A^Y$ admits limits of the corresponding family of diagrams $k^Y\colon K^Y\to (A^X)^Y\cong(A^Y)^X$ of shape $X$.
\end{prop}

\begin{proof}
  On precomposing the absolute right lifting that displays the limits of the family $k\colon K\to A^X$ \eqref{eq:limits.of.a.family} by the evaluation map $\ev\colon K^Y\times Y\to K$, we obtain an absolute right lifting diagram whose adjoint transpose under the 2-adjunction ${-}\times Y\dashv (-)^Y$ is the triangle
  \begin{equation*}
      \xymatrix{ \ar@{}[dr]|(.7){\Downarrow\lambda^Y} & A^Y \ar[d]^-{c^Y} \\ K^Y \ar[ur]^{\lim^Y} \ar[r]_-{k^Y} & (A^X)^Y}
\end{equation*}
  By the last observation, this is again an absolute right lifting diagram which, on composition with the canonical isomorphism $(A^X)^Y\cong(A^Y)^X$, displays $\lim^Y$ as the family of limits required in the statement.
\end{proof}

Proposition \ref{prop:limitsasadjunctions} tells us that if $A$ has all limits of shape $X$, then there is a functor $\lim \colon A^X \to A$ that is right adjoint to the constant functor $c \colon A \to A^X$. In ordinary category theory we often deploy another adjunction related to the existence of limits of shape $X$, this being the restriction--right Kan extension adjunction between diagrams of shape $X$ and diagrams whose shape is that of a cone over $X$.

The shape of a cone over a diagram of shape $X$ is given by the simplicial set $\Del^0\join X$, defined using Joyal's join construction of Definition~\ref{defn:join-dec}.

\begin{prop}\label{prop:ran.adj.limits} A quasi-category $A$ admits limits of the family of diagrams $k\colon K\to A^X$ of shape $X$ if and only if there exists an absolute right lifting diagram
\begin{equation*}
  \xymatrix{
    & *+[r]{A^{\Del^0\join X}}\ar@{->>}[d]^-{\res} \\
    {K} \ar[ur]^{\ran} \ar[r]_-{k} 
    & *+[r]{A^X} \ar@{}[ul]|(0.3){\Downarrow\lambda}
  }
\end{equation*}
in which  $\res$  is the restriction isofibration given by pre-composition with the inclusion $X\inc \Del^0\join X$. Furthermore, when these equivalent conditions hold $\lambda$ is necessarily an isomorphism and, indeed, we may choose $\ran$ so that $\lambda$ is an identity.
\end{prop}

\begin{proof}
By Proposition~\ref{prop:join-fatjoin-equiv}, the canonical comparison $\Del^0\fatjoin X\to\Del^0\join X$ is a weak equivalence in Joyal's model structure. So if $A$ is a quasi-category, it follows, by Proposition~\ref{prop:equivsareequivs2}, that the associated pre-composition functor $A^{\Del^0\join X}\to A^{\Del^0\fatjoin X}$ is an equivalence of quasi-categories. Now the contravariant exponential functor $A^{({-})}\colon \sSet\op\to \qCat$ carries colimits to limits so it is immediate, from Definition~\ref{eq:fat-join-def}, that we have a pullback \[ \xymatrix{ A^{\Delta^0\fatjoin X} \pbexcursion \ar[r] \ar[d] & A^{X \times \Delta^1} \ar[d] \\ A \times A^X\cong A^{\Delta^0 \sqcup X}  \ar[r] & A^{X \sqcup X} \cong A^X \times A^X}\] from which we see that $A^{\Delta^0\fatjoin X}$ is isomorphic to the comma quasi-category $c\comma A^X$. It is now easily checked that a triangle of the form given in the statement is an absolute right lifting if and only if the following rearranged triangle 
\begin{equation*}
  \vcenter{\xymatrix{
      & {c\comma A^X} \ar@{->>}[d]^-{p_1} \\
      {K} \ar[ur]^{\ran} \ar[r]_-{k} 
      & {A^X} \ar@{}[ul]|(0.3){\Downarrow\lambda}
  }} \mkern20mu \defeq \mkern20mu
  \vcenter{\xymatrix{
    & *+[r]{A^{\Del^0\join X}}\ar@{->>}[d]^-{\res}\ar[r]^-{\sim} &  {c\comma A^X} \ar@{->>}[dl]^{p_1}\\
    {K} \ar[ur]^{\ran} \ar[r]_-{k} 
    & *+[r]{A^X} \ar@{}[ul]|(0.3){\Downarrow\lambda} &
  }}
\end{equation*}
has that property; now the current result is merely a special case of Proposition~\ref{prop:translated.lifting}.
\end{proof}

\begin{cor}\label{cor:ran.adj.limits} A quasi-category $A$ admits all limits of shape $X$ if and only if the restriction functor associated with the inclusion $X\inc \Del^0\join X$ has a fibred right adjoint \[ 
  \xymatrix@=1.5em{
    {A^X}\ar@/_1.2ex/[rr]_-\ran\ar@{=}[dr] \ar@{}[rr]|*{\bot} && 
    *+!L(0.5){A^{\Del^0\join X}}\ar@/_1.2ex/[ll]_-\res \ar@{->>}[dl]^{\res} \\
    & A^X &
  }
\] 
\end{cor}
\begin{proof}
Since the restriction functor $A^{\Del^0\join X} \tfib A^X$ is an isofibration, we may follow Example~\ref{ex:isofib-section.fibred.adjunction} and pick its right adjoint so that the counit of the adjunction $\res \dashv\ran$ is an identity. By Corollary~\ref{cor:missed-lemma}, this adjunction lifts to an adjunction fibred over $A^X$.
\end{proof}

As an application of some significant classical interest, we may use Proposition~\ref{prop:ran.adj.limits} to construct a loops--suspension adjunction in any pointed quasi-category admitting certain pullbacks and pushouts.

\begin{defn}[pointed quasi-categories]
    A \emph{zero object} in a quasi-category is an object in there that is both initial and terminal. We say that a quasi-category $A$ is {\em pointed\/} if it has a zero object and write $*\in A$ for that object. We call the counit $\rho\colon *!\Rightarrow\id_A$ of the adjunction $\adjinline * -| ! : A -> \Del^0.$ the {\em family of points\/} of the objects of $A$ and call the unit $\xi\colon \id_A\Rightarrow *!$ of the adjunction $\adjinline ! -| * : A -> \Del^0.$ the {\em family of co-points\/} of the objects of $A$.
\end{defn}

\begin{ntn}[pushout and pullback diagrams]\label{ntn:pb.po.joins}
  We shall adopt the following notation for certain important diagram shapes which arise naturally as simplicial subsets of the square $\Del^1\times\Del^1$:
  \begin{itemize}
    \item $\pbshape$ will denote the simplicial subset $(\Del^1\times\Del^{\fbv{1}})\cup(\Del^{\fbv{1}}\times\Del^1)$, and
    \item $\poshape$  will denote the simplicial subset $(\Del^1\times\Del^{\fbv{0}})\cup(\Del^{\fbv{0}}\times\Del^1)$.
  \end{itemize}
  Of course, $\pbshape$ and $\poshape$ are the shapes of pullback and pushout diagrams,   isomorphic to the horns $\Horn^{2,2}$ and $\Horn^{2,0}$ respectively.    The joins $\Del^0\join\pbshape$ and $\poshape\join\Del^0$ are each isomorphic to the square $\Del^1\times\Del^1$. These isomorphisms identify the canonical inclusions of those joins with the corresponding subset inclusions $\pbshape\inc\Del^1\times\Del^1$ and $\poshape\inc\Del^1\times\Del^1$ respectively.
\end{ntn}

\begin{defn}[pushouts and pullbacks in quasi-categories]
  A {\em pullback\/} in a quasi-category is a limit of a diagram of shape $\pbshape$. Dually a {\em pushout\/} in a quasi-category is a colimit of a diagram of shape $\poshape$. 
\end{defn}

\begin{obs}\label{obs:loops.diag.fam}
The family of points of a pointed quasi-category $A$ may be represented by a simplicial map $\rho\colon A\to A^\cattwo$. Now the pullback diagram shape $\pbshape$ may be represented as a glueing of two copies of $\cattwo$ identified at their initial vertex, so it follows that $A^\pbshape$ may be constructed as a pullback of two copies of $A^\cattwo$ along the projection $p_1\colon A^\cattwo\tfib A$. Consequently, two copies of $\rho$ give rise to a functor $\bar\rho\colon A\to A^\pbshape$. This functor maps each object $a$ of $A$ to a pushout diagram with outer vertices $*$, inner vertex $a$, and maps two copies of the component of $\rho$ at $a$. Dually we may define a corresponding functor $\bar\xi\colon A\to A^\poshape$ using two copies of the family of co-points.
\end{obs}

\begin{defn}[loop spaces and suspensions]\label{defn:loop.susp}
  We say that a pointed quasi-category $A$ admits the construction of {\em loop spaces\/} if it admits limits of the family of diagrams $\bar\rho\colon A\to A^\pbshape$.   Dually, we say that $A$ admits the construction of {\em suspensions\/} if  it admits colimits of the family of diagrams $\bar\xi\colon A\to A^\poshape$. These constructions, when they exist, are displayed by absolute right and left liftings
\begin{equation*}
  \xymatrix{ 
    \ar@{}[dr]|(.7){\Downarrow} & A \ar[d]^c \\ 
    A \ar[ur]^\Omega \ar[r]_{\bar\rho} & A^\pbshape
  }
  \mkern80mu
  \xymatrix{ 
    \ar@{}[dr]|(.7){\Uparrow} & A \ar[d]^c \\ 
    A \ar[ur]^\Sigma \ar[r]_{\bar\xi} & A^\poshape
  }
\end{equation*}
in which $\Omega$ is called the {\em loop space functor\/} and $\Sigma$ is called the {\em suspension functor}. Of course, if $A$ admits all pullbacks (resp.\ pushouts) then, as a special case, it admits the construction of loop spaces (resp. suspensions).
\end{defn}

\begin{ex}
  In the quasi-category of spaces, which we construct by applying the homotopy coherent nerve to the simplicially enriched category of Kan complexes, pushouts and pullbacks are constructed by taking classical homotopy pushouts and pullbacks. The quasi-category of pointed spaces is simply the slice under $\Del^0$ and its pushouts and pullbacks may be computed as in the quasi-category of spaces. It follows, therefore, that the loop space and suspension constructions in this quasi-category coincide with the usual notions in classical homotopy theory.
\end{ex}

The following proposition promotes our classical intuition about the relationship between loop and suspension constructions to a genuine adjunction of quasi-categories. To keep our proof as simple and transparent as possible, we choose to assume that the quasi-category here admits all pushouts and pullbacks, leaving it to the reader to generalise this result to one in which we only assume the existence of loop spaces and suspensions.

\begin{prop}\label{prop:loops-suspension} Suppose that $A$ is a pointed quasi-category which admits all pushouts and pullbacks. Then $A$ has a loops--suspension adjunction \[ \adjdisplay \Sigma -| \Omega: A -> A.\]
\end{prop}

\begin{proof}
  By Corollary~\ref{cor:ran.adj.limits} and the ruminations of~\ref{ntn:pb.po.joins}, the hypothesis that $A$ has pullbacks and pushouts implies that there are adjunctions
\begin{equation}\label{eq:pullback.pushout.adj}    
  \xymatrix@R=0em@!C=8em{
    {A^\pbshape}
    \ar@/_0.55pc/[r]!L(0.5)_-{\ran} 
    \ar@{}[r]!L(0.5)|-{\displaystyle\bot} 
    \ar@{<-}@/^0.55pc/[r]!L(0.5)^-{\res} & 
    {A^{\Delta^1\times\Delta^1}} & 
    {A^\poshape}
    \ar@/_0.55pc/[l]!R(0.5)_-{\lan}
    \ar@{<-}@/^0.55pc/[l]!R(0.5)^-{\res}  
    \ar@{}[l]!R(0.5)|-{\displaystyle\bot}
  }
\end{equation}
which are fibred over $A^\pbshape$ and $A^\poshape$, respectively. Now the inclusion of $\Del^0\sqcup\Del^0$ into $\Del^1\times\Del^1$ which picks out the vertices $(1,0)$ and $(0,1)$ factorises through each of the subsets $\pbshape$ and $\poshape$ and therefore induces restriction isofibrations $A^\pbshape\tfib A\times A$ and $A^\poshape\tfib A\times A$. So we may push forward our fibred adjunctions along these isofibrations to obtain a composable pair of adjunctions fibred over $A\times A$. Composing these and pulling back  along $(*,*)\colon \Del^0\to A\times A$, we obtain an adjunction
\begin{equation}\label{eq:loop.susp.var}
  \adjdisplay \overline\Sigma -| \overline\Omega : A^\pbshape_* -> A^\poshape_*.
\end{equation}
where $A^\pbshape_*\subseteq A^\pbshape$ and $A^\poshape_*\subseteq A^\poshape$ are the sub-quasi-categories of pullback and pushout diagrams whose outer vertices are pinned at the zero object $*$. 

The family of points $\rho\colon A\to A^\cattwo$ discussed in Observation~\ref{obs:loops.diag.fam} factorises through the sub-quasi-category $*\comma A\subseteq A^\cattwo$; hence,  the family of diagrams $\bar\rho\colon A\to A^\pbshape$ for the loop space construction also factorises through $A^\pbshape_*\subseteq A^\pbshape$. Furthermore, it is clear that the pullback expressing $A^\pbshape$ in terms of two copies of $A^\cattwo$ restricts to the pullback expressing $A^\pbshape_*$ in terms of two copies of $*\comma A$ in the following diagram:
\begin{equation*}
  \xymatrix@=1.5em{ 
    A\ar[dr]|*+{\scriptstyle\bar\rho}\ar@/^1.5ex/[drr]^\rho
    \ar@/_1.5ex/[ddr]_\rho &&\\
    & A^\pbshape_* \pbexcursion \ar@{->>}[d]
    \ar@{->>}[r] & {*\comma A} \ar@{->>}[d]^-{p_1} \\ 
    & {* \comma A} \ar@{->>}[r]_-{p_1} & A
  }
\end{equation*}

We claim that each functor in this diagram is an equivalence. To show this start by observing that the initiality of $*$ in $A$ implies that the isofibration $p_1$ is an equivalence, as is its right inverse $\rho$ by the 2-of-3 property. Trivial fibrations are stable under pullback, so the two projections from $A^\pbshape_*$ are equivalences, as is $\bar\rho$ by the 2-of-3 property. Observe also that the functor which restricts each pullback diagram to its inner vertex is an isofibration left inverse to $\bar\rho$ and so, by the 2-of-3 property, it too is an equivalence. The dual argument shows that the family of diagrams $\bar\xi\colon A\to A^\poshape$ for the suspension construction also factorises through $A^\poshape_*\subseteq A^\poshape$ to give an equivalence $\bar\xi\colon A\to A^\poshape_*$ with left inverse the isofibration that restricts each pullback diagram to its inner vertex.

  Now we may promote the equivalences $\bar\rho$ and $\bar\xi$  to adjoint equivalences and compose them with the adjunction~\eqref{eq:loop.susp.var}. The right adjoint in this composite adjunction is equal to the composite $\xymatrix@1{{A}\ar[r]^-{\bar\rho} & {A^\pbshape}\ar[r]^-{\ran} & {A^{\Del^1\times\Del^1}}\ar[r]^-{\res} & {A}}$ in which the last map is the restriction functor associated with the inclusion of $\Del^0$ as the vertex $(0,0)$ of $\Del^1\times\Del^1$. The composite of these last two functors is the pullback functor $\lim\colon A^\pbshape\to A$, so pre-composing it with $\bar\rho\colon A\to A^\pbshape$ produces a functor which picks out limits of the diagrams in the family $\bar\rho$. This must therefore be isomorphic to the loop space functor $\Omega$ by Definition~\ref{defn:loop.susp}. A dual argument demonstrates that the left adjoint in the composite adjunction is isomorphic to the suspension functor $\Sigma$, thus completing the verification that the adjunction we have constructed is the one asked for in the statement.
\end{proof}


\subsection{Geometric realisations of simplicial objects}

A classical result from simplicial homotopy theory states that if a simplicial object admits an augmentation together with a splitting, also called a contracting homotopy or simply ``extra degeneracies'', then the augmentation is homotopy equivalent to its geometric realisation. More precisely, the augmented simplicial object, a diagram of shape $\Del+\op$, defines a colimit cone over the restriction of this diagram to $\Del\op$. 

In this section, we import these ideas into the quasi-categorical context, proving that if a simplicial object in a quasi-category admits an augmentation and a splitting then the augmentation is its quasi-categorical colimit.
Again, the result is not new (cf.~\cite[6.1.3.16]{Lurie:2009fk}), but our proof closely mirrors the classical one (see, e.g.,~\cite{Meyer:84ba}). Specifically, we show that the structure of the contracting homotopies define an absolute left extension diagram in $\Cat$. Furthermore, this universal property is witnessed equationally and so is preserved by any 2-functor.  Dual remarks apply to cosimplicial objects admitting a coaugmentation and a splitting.

The first step is to describe the shape of a split simplicial object. There are two choices, distinguished by whether we choose a ``forwards'' or ``backwards'' contracting homotopy. The corresponding categories are opposites. Let $\Del[t]$ and $\Del[b]$ denote the subcategories of $\Del$ consisting of those maps that preserve the top or bottom element respectively in each ordinal. There is an inclusion  $[0]\oplus -\colon \Del+ \inc \Del[b]$ which freely adjoins a bottom element. Note the degree shift: this functor sends the initial object $[-1] \in \Del+$ to the zero object $[0]\in\Del[b]$. 

A simplicial object is \emph{augmented} if it admits an extension to $\Del+\op$ and \emph{split} if it admits a further extension to $\Del[t] \cong \Del[b]\op$. Evaluating at $[0] \in \Del[t]$ yields the augmentation. Restriction along the inclusion $\Del\op \inc \Del+\op\inc \Del[t]$ yields the original diagram. We will prove:

\begin{thm}\label{thm:splitgeorealizations} For any quasi-category $B$, the canonical diagram \[ \xymatrix{ \ar@{}[dr]|(.7){\Uparrow} & B \ar[d]^c \\ B^{\Del[t]} \ar[ur]^{\ev_0} \ar[r]_{\res} & B^{\Del\op}}\] is an absolute left lifting diagram. Hence, given any simplicial object admitting an augmentation and a splitting, the augmented simplicial object defines a colimit cone over the original simplicial object. Furthermore, such colimits are preserved by any functor.
\end{thm}

Our proof uses a 2-categorical lemma.

\begin{lem}\label{lem:doms2catlemma} Suppose given an adjunction in a slice 2-category $C\slice\tcat{C}$
\[\vcenter{\xymatrix@R=30pt{ & C \ar[dl]|b_{\rotatebox{45}{$\labelstyle\perp$}} \ar[dr]^a & \\ \ar@/^3ex/@{-->}[ur]^c B \ar@/^1ex/[rr]^f  \ar@{}[rr]|\perp & & A \ar@/^1ex/[ll]^u }}\] If $b$ admits a left adjoint $c$ in $\tcat{C}$ with unit $\iota$, then the 2-cell $f\iota \colon f \Rightarrow fbc=ac$ exhibits $c$ as an absolute left lifting of $f$ through $a$.
\end{lem}
\begin{proof}
Let $\nu$ be the counit of $c \dashv b$, and write $\eta$ and $\epsilon$ for the unit and counit of the adjunction $f\dashv u$; because this adjunction is under $C$ we have $\epsilon a = \id_a$ and $\eta b =\id_b$. Any 2-cell $\chi$ of the form displayed below factorises through $f\iota$ as follows
\[\xymatrix{ X \ar[d]_x \ar[r]^y \ar@{}[dr]|{\Uparrow\chi} & C \ar[d]^a \ar@{}[dr]|{\displaystyle =}  & X \ar[d]_x \ar[r]^y \ar@{}[dr]|{\Uparrow\chi} & C \ar[d]^a   \ar@{}[dr]|{\displaystyle =}   &X \ar@{}[dr]|{\Uparrow\chi} \ar[d]_x \ar[r]^y & C \ar[d]_a \ar[r]^a \ar[dr]|b & A   \ar@{}[dr]|{\displaystyle =} & X \ar[d]_x \ar[r]^y \ar@{}[dr]|{\Uparrow\chi} & C \ar[d]_a \ar[dr]^b \ar@{=}[rr] & \ar@{}[d]|{\Uparrow\nu} &  C \ar[d]^a  \\  B \ar[r]_f & A & B \ar[r]^f \ar@{=}[dr] & A \ar[d]|u \ar@{}[dl]|(.3){\Uparrow\eta} \ar@{=}[dr]  &  B \ar[r]^f \ar@{=}@/_3.5ex/[rr]^{\Uparrow\eta} & A \ar[r]^u & B \ar[u]_f &  B \ar[r]^f \ar@{=}@/_3.5ex/[rr]^{\Uparrow\eta} & A \ar[r]^u  & B \ar[ur]^c \ar[r]_f & A \ar@{}[ul]|(.3){\Uparrow f\iota} \\ & & & B \ar[r]_f \ar@{}[ur]|(.3){\Uparrow\epsilon} & A }\]
using a triangle identity for each adjunction and the fact that $\epsilon a = \id_a$. Such factorisations are unique because the 2-cell $\zeta$ can be recovered from the pasted composite with $f\iota$: \[\xymatrix{  X \ar[d]_x \ar[r]^y \ar@{}[dr]|(.3){\Uparrow\zeta}|(.7){\Uparrow f\iota} & C \ar[d]_a \ar[dr]^b \ar@{=}[rr] & \ar@{}[d]|{\Uparrow\nu} &  C  \\ B \ar[ur]|c \ar[r]^f \ar@{=}@/_3.5ex/[rr]^{\Uparrow\eta} & A \ar[r]^u & B \ar[ur]^c & {\displaystyle =} } \xymatrix{ X \ar[d]_x \ar[r]^y \ar@{}[dr]|(.3){\Uparrow\zeta}|(.7){\Uparrow\iota} & C \ar[d]|b \ar[dr]|a \ar[drr]^b \ar@{=}[rr]& & C \ar@{}[dl]|(.4){\Uparrow\nu} \\B \ar[ur]|c \ar@{=}[r] & B  \ar[r]_f \ar@{=}@/_3.5ex/[rr]^{\Uparrow\eta} & A \ar[r]_u & B \ar[u]_c} = \xymatrix{ X \ar[d]_x \ar[r]^y \ar@{}[dr]|(.3){\Uparrow\zeta}|(.7){\Uparrow\iota} & C \ar[d]|b \ar@{}[dr]|(.3){\Uparrow\nu} \ar@{=}[r] & C    \\  B \ar[ur]|c \ar@{=}[r] & B \ar[ur]_c & {\displaystyle =}  }
\xymatrix{ X \ar[d]_x \ar[r]^y \ar@{}[dr]|(.3){\Uparrow\zeta} & C \\ B \ar[ur]_c & }\qedhere
\] 
\end{proof}

\begin{proof}[Proof of Theorem \ref{thm:splitgeorealizations}] The inclusion $\Del\op \hookrightarrow \Del[t]$ admits a left adjoint. One way to define it is to present  $\Del\op$ via the ``interval representation'': after employing a degree shift $[n] \mapsto [n+1]$, $\Del\op$ is the subcategory of $\Del+$ consisting of ordinals with distinct top and bottom elements and maps that preserve these. Most generally, we might think of the interval representation as the diagonal composite functor in the pullback diagram \[\xymatrix{ \Del+\op \pbexcursion \ar[d] \ar[r] & \Del[t] \ar@{_(->}[d] \\ \Del[b] \ar@{^(->}[r] & \Del+}\] The arrows $\Del[b] \leftarrow \Del+\op \to \Del[t]$ extend the category indexing augmented simplicial objects by introducing extra maps that define ``extra degeneracies'' either on the left or on the right. The restricted functor $\Del\op \to \Del[t]$ is the inclusion described above. It has a left adjoint: a map $\alpha \colon [k] \to [n+1]$ in $\Del[t]$ is given by a map $[n] \to [k]$ in $\Del$ that sends $i\in [n]$, thought of as a ``gap'' between adjacent elements in $[n+1]$, to the minimal $j \in [k]$ so that $\alpha(j) = i+1$. 

 For any quasi-category $B$, the 2-functor $B^{(-)} \colon \Cat_2\op \to \qCat_2$ carries the
adjoint functors 
\[\xymatrix@R=30pt{ & \catone \ar[dl]^{\labelstyle[0]} & \\ \Del[t] \ar@/^1ex/[rr] \ar@{}[rr]|\perp  \ar@/^2.5ex/[ur]^{!} \ar@{}[ur]^*-{\rotatebox{45}{$\labelstyle\perp$}} & & \Del\op \ar@/^/[ll] \ar[ul]_{!}}\]
to an adjunction in the slice 2-category $B\slice\qCat_2$
\[\xymatrix@R=30pt{ & B \ar[dl]^{\labelstyle c} \ar[dr]^c & \\ B^{\Del[t]} \ar@/^1ex/[rr]^{\res} \ar@{}[rr]|\perp  \ar@/^2.5ex/[ur]^{\ev_0} \ar@{}[ur]^*-{\rotatebox{45}{$\labelstyle\perp$}} & & B^{\Del\op} \ar@/^/[ll] }\] The 2-cell defined by whiskering $\res$ with the unit of $\ev_0\dashv c$ is the 2-cell $\res \Rightarrow c \cdot \ev_0$ obtained by applying the 2-functor $B^{-}$ to the unique 2-cell
\[\xymatrix{ \Del\op \ar@{^(->}[rr] \ar[dr]_{!} & \ar@{}[d]|(.35){\Downarrow} & \Del[t] \\ & \catone \ar[ur]_{[0]} & } \] that exists because $[0] \in \Del[t]$ is terminal. The result now follows from Lemma \ref{lem:doms2catlemma}.

It remains only to prove the last statement. Given any functor $f \colon B \to A$, the diagrams \[ \vcenter{\xymatrix{ \ar@{}[dr]|(.7){\Uparrow} & B \ar[d]^c \ar[r]^f & A \ar[d]^c \\ B^{\Del[t]} \ar[ur]^{\ev_0} \ar[r]_{\res} & B^{\Del\op} \ar[r]_{f^{\Del\op}} & A^{\Del\op}}} = \vcenter{ \xymatrix{ &  \ar@{}[dr]|(.7){\Uparrow} & B \ar[d]^c \\ B^{\Del[t]} \ar[r]_{f^{\Del[t]}} & A^{\Del[t]} \ar[ur]^{\ev_0} \ar[r]_{\res} & A^{\Del\op}}}\] coincide by bifunctoriality of the internal hom 2-functor in $\qCat_2$. In particular, the left-hand side inherits the universal property of the right-hand side.
\end{proof}

\begin{ex} Theorem \ref{thm:splitgeorealizations} can be used to prove that any object in the quasi-category of algebras associated to a coherent monad is a homotopy colimit of a canonical simplicial object of free algebras. See \cite{RiehlVerity:2012hc} and \cite{RiehlVerity:2013cp}.
\end{ex}



%!TEX root = all.tex
% ******************************************************************
% ** Title:            The 2-category theory of quasi-categories
% **                   adjunctions
% ** Precis:        
% ** Author:           Emily Riehl and Dominic Verity
% ** Commenced:        2/3/2012
% ******************************************************************


\section{Pointwise universal properties}\label{sec:pointwise}

We have seen that limits and adjunctions can be encoded as absolute lifting diagrams in $\qCat_2$. In this section, we prove a theorem that allows such diagrams to be identified in practice: we show that absolute left or right lifting diagrams can be defined ``pointwise''  by specifying initial or terminal objects, respectively, in the appropriate comma or slice quasi-categories; the definition of Joyal's slice quasi-categories is recalled in \ref{defn:slices}.

 We conclude by proving a corollary of this result: that simplicial Quillen adjunctions between simplicial model categories are adjunctions of quasi-categories.  Adjunctions in homotopical contexts are commonly presented as Quillen adjunctions,  which can be replaced  by adjunctions of this type in good set-theoretical cases \cite{RezkSchwedeShipley:2001ss}. This result implies that such adjunctions can be imported into the quasi-categorical context.

\subsection{Pointwise absolute lifting}

Immediately from Definition \ref{defn:families.of.diagrams}, absolute lifting diagrams are preserved by pre-composition by all functors and, in particular, under evaluation at a vertex in the domain quasi-category.

\begin{defn}[pointwise universal property of absoluting lifting diagrams]\label{defn:pointwise-abs-lifting}
If the left-hand diagram
  \begin{equation}\label{eq:pointwise-absRlifting}
    \vcenter{\xymatrix{ \ar@{}[dr]|(.7){\Downarrow\lambda} & B \ar[d]^f \\ C \ar[r]_g \ar[ur]^\ell & A}} \qquad \rightsquigarrow \qquad   \vcenter{\xymatrix{ \ar@{}[dr]|(.7){\Downarrow\lambda c} & B \ar[d]^f \\ \Del^0 \ar[r]_{gc} \ar[ur]^{\ell c} & A}}
  \end{equation}
  is an absolute lifting diagram and $c$ is an object of $C$ then pre-composition by the functor $c\colon\Del^0\to C$ gives a 2-cell $\lambda c\colon f\ell c\Rightarrow gc$ which displays $\ell c\colon\Del^0\to B$ as an absolute right lifting of $gc\colon\Del^0\to A$ through $f\colon B\to A$. The family of absolute lifting diagrams as displayed on the right encode  the \emph{pointwise universal property} of the absolute lifting diagram displayed on the left.
\end{defn}

A special case of Proposition \ref{prop:right.liftings.as.fibred.terminal.objects} provides an alternate characterisation of a pointwise absolute lifting property:

\begin{lem}\label{lem:pointwise-terminal}
Given functors $g \colon C \to A$ and $f \colon B \to A$ the data of a pointwise absolute right lifting diagram at a vertex $c \in C$ is equally the data of a terminal object in the comma or slice quasi-categories $f \comma gc \simeq \slicer{f}{gc}$.
\end{lem}

Lemma~\ref{lem:slice-equiv-comma} supplies an equivalence $f \comma gc \simeq \slicer{f}{gc}$ along which we may transport terminal objects. Lemma \ref{lem:pointwise-terminal} demonstrates that if $g$ admits an absolute right lifting through $f$, then $f\comma gc \simeq \slicer{f}{gc}$ has a terminal object, for each vertex $c$ in the domain of $g$. In fact, these terminal objects suffice to demonstrate the existence of an absolute right lifting:

\begin{thm}\label{thm:pointwise} The functor $g\colon C\to A$ admits an absolute right lifting through the functor $f\colon B\to A$ if and only if for all objects $c$ of $C$ the quasi-category $f\comma gc\simeq\slicer{f}{gc}$ has a terminal object.
\end{thm}









\begin{proof}[Proof of Theorem \ref{thm:pointwise}]
Suppose each $\slicer{f}{gc}$ has a terminal object $\lambda_c \colon fb \to gc$, i.e., suppose we can fill any sphere $\partial\Delta^n \to \slicer{f}{gc}$ with $n \geq 1$ whose final vertex is $\lambda_c$. Unpacking the definition, we have assumed that we can solve any lifting problem 
\begin{equation}\label{eq:term.obj.assumption} \vcenter{\xymatrix{\boundary \Delta^n \ar[d] \ar[r] & \slicer{f}{gc} \\ \Delta^n \ar@{-->}[ur]}}\qquad\leftrightsquigarrow \qquad \vcenter{\xymatrix@C=25pt@R=30pt@!0{ \boundary\Delta^n \ar[rrrr] \ar[dd] \ar[dr] & && & B \ar[dr]^f \\ & \Lambda^{n+1,n+1} \ar[dd] \ar[rrrr] & & & & A \\ \Delta^n \ar[dr]_{\face^{n+1}} \ar@{-->}[uurrrr] \\ & \Delta^{n+1} \ar@{-->}[uurrrr]}}\end{equation} 
in $\qCat^\cattwo$ for which the $\fbv{n,n+1}$ edge of the $\Lambda^{n+1,n+1}$-horn in $A$ is $\lambda_c$.

It follows that we can solve any extension problem 
\begin{equation}\label{eq:term.obj.extension}\xymatrix@C=70pt@R=30pt@!0{  \boundary\Delta^n \times \Delta^{\fbv{0}} \ar[rr] \ar[dd] \ar[dr] &  & B \ar[dr]^f \\ &  \boundary\Delta^n \times \Delta^1\cup \Delta^n \times \Delta^{\fbv{1}}  \ar[dd] \ar[rr] &  & A \\ \Delta^n \times \Delta^{\fbv{0}} \ar[dr] \ar@{-->}[uurr]|\hole \\ & \Delta^n \times \Delta^1 \ar@{-->}[uurr]}\end{equation} for which the image of the edge between the vertices $(n,0)$ and $(n,1)$ is $\lambda_c$: The filler is constructed by inductively choosing images for the shuffles of $\Delta^n \times \Delta^1$  starting from the filled end of the specified cylinder. The images for all but the last shuffle are defined by filling the obvious inner horns in $A$. The final shuffle is attached by filling a $\Lambda^{n+1,n+1}$-horn in $A$ precisely of the form \eqref{eq:term.obj.assumption}.

We are interested in extension problems \eqref{eq:term.obj.extension} where the $n$-simplex in $A$ given as one end of the cylinder is in the image of some specified $n$-simplex of $C$ under $g$; these are precisely the data specified by a lifting problem \begin{equation}\label{eq:term.obj.lifting} \xymatrix{\Delta^0 \ar[r]_-{\fbv{n}} \ar@/^2ex/[rr]^{\lambda_c} & \boundary\Delta^n \ar[d] \ar[r] & f \comma g \ar[d]^{q_1} \\ &  \Delta^n \ar@{-->}[ur] \ar[r] & C}\end{equation} in which case the extension of \eqref{eq:term.obj.extension} provides a solution. We have just shown that any lifting problem \eqref{eq:term.obj.lifting} in which the final vertex of the sphere maps to a terminal object $\lambda_c \in \slicer{f}{gc}$ has a solution. By Lemma \ref{lem:RARI-lifting}, this tells us that $q_1 \colon f \comma g \to C$ admits a right adjoint right inverse $t \colon C \to f \comma g$, which by Proposition \ref{prop:right.liftings.as.fibred.terminal.objects} encodes the data of an absolute right lifting diagram, as displayed on the bottom right.
\[ 
    \vcenter{\xymatrix@=0.8em{
      & \save []+<0pt,1em>*+{C}\ar[d]^-{t}\ar@{=}@/_1.5ex/[ddl]
      \ar@/^1.5ex/[ddr]^{\ell}\restore & \\
      & {f\comma g}\ar[dr]_{q_0}\ar[dl]^{q_1} & \\
      {C}\ar[dr]_{g} & {\scriptstyle\Leftarrow\psi} &
      {B}\ar[dl]^{f} \\
      & {A} &
    }}
    \mkern20mu = \mkern20mu
    \vcenter{\xymatrix@=0.7em{
      & {C}\ar@{=}[dl]\ar[dr]^{\ell} & \\
      {C}\ar[dr]_{g} & {\scriptstyle\Leftarrow\lambda} & {B}\ar[dl]^{f} \\
      & {A} &
    }}
\]
\end{proof}



Theorem \ref{thm:pointwise} provides a useful criterion for the existence of absolute lifting diagrams. The following corollary supplies the corresponding detection result, identifying when a candidate lifting diagram has the desired universal property. The lifting property implies that each of its fibres admit terminal objects, a definition that will be introduced in the next section.

\begin{cor}\label{cor:pointwise}
  A triangle 
  \[    \xymatrix{ \ar@{}[dr]|(.7){\Downarrow\lambda} & B \ar[d]^f \\ C \ar[r]_g \ar[ur]^\ell & A}\] displays $\ell$ as an absolute right lifting of $g$ through $f$ if and only if it has that property pointwise.
\end{cor}

\begin{proof}
Necessity of the pointwise absolute lifting property of Definition \ref{defn:pointwise-abs-lifting} is immediate. Conversely, the assumed pointwise lifting tells us, in particular, that for each object $c$ in $C$ the slice quasi-category $f\comma gc\simeq\slicer{f}{gc}$ has a terminal object. Consequently, we may apply Theorem \ref{thm:pointwise} to construct a functor $\ell'\colon C\to A$ and 2-cell $\lambda'\colon f\ell'\Rightarrow g$ which displays $\ell'$ as an absolute right lifting of $g$ through $f$. 
  
The universal property of $(\ell',\lambda')$ applied to the triangle $(\ell,\lambda)$ provides us with a unique 2-cell $\tau\colon\ell\Rightarrow\ell'$ with the defining property that $\lambda'\cdot f\tau =\lambda$. Now both of the 2-cells $\lambda$ and $\lambda'$ possess the pointwise lifting property, the first by assumption and the second by construction. In other words, for all objects $c$ in $C$ the 2-cell $\lambda c\colon f\ell c\Rightarrow g c$ (respectively $\lambda' c\colon f\ell' c\Rightarrow g c$) displays $\ell c$ (respectively $\ell' c$) as an absolute right lifting of $g c$ through $f$ for all objects $c$ of $C$. Furthermore, the defining property of $\tau$ whiskers to tell us that $\lambda' c\cdot f(\tau c)=\lambda c$, so since $\lambda c$ and $\lambda' c$ both possess the absolute right lifting property it follows that $\tau c$ is an isomorphism. Applying Observation~\ref{obs:pointwise-iso-reprise}, we find that $\tau\colon\ell\Rightarrow\ell'$ is an isomorphism and thus that the given triangle is isomorphic to the absolute right lifting that we constructed and is thus itself an absolute right lifting.
\end{proof}

Proposition \ref{prop:families.of.diagrams}, which states that a quasi-category admits limits of a family of diagrams of a fixed shape if and only if it admits limits of each individual diagram in the family, is a special case of Theorem \ref{thm:pointwise}.

\begin{proof}[Proof of Proposition~\ref{prop:families.of.diagrams}]
If $A$ admits limits of each diagram in a family $k \colon K \to A^X$, then Proposition~\ref{prop:limits.are.limits} implies that for each vertex $\overline{d} \in K$, $\slicer{c}{k\overline{d}}$ has a terminal object. By Theorem~\ref{thm:pointwise}, it follows that $k$ admits an absolute right lifting along $c \colon A \to A^X$, i.e., $A$ admits limits of the family of diagrams $k \colon K \to A^X$.
\end{proof}

%Corollary \ref{cor:pointwise} also allows us to prove that Lurie's definition of adjunction given in \cite[5.2.2.8]{Lurie:2009fk} is equivalent to the 2-categorical definition.

%\begin{prop}\label{prop:joyal-adj-equals-lurie-adj} A pair of functors $f \colon B \to A$ and $u \colon A \to B$ together with a 2-cell $\epsilon \colon fu \To \id _A \in A^A$ define an adjunction $f \dashv u$ with counit $\epsilon$ if and only if the functor $B \comma u \to f \comma A$ over $A \times B$ induced by $\epsilon$, as described in proposition \ref{prop:adjointequiv}, pulls back over any pair of vertices $(a,b) \in A \times B$  to define an equivalence $b \comma ua \simeq fb \comma a$ of hom-spaces.
%\end{prop}
%\begin{proof}
%Observation \ref{obs:pointwise-adjoint-correspondence} demonstrated that the counit of an adjunction in the 2-category of quasi-categories has this property. 
%\end{proof}

\subsection{Simplicial Quillen adjunctions are adjunctions of quasi-categories}

Now we use Theorem \ref{thm:pointwise} to prove the assertions made in Example \ref{ex:simp.quillen.adj}: namely that any simplicial Quillen adjunction between simplicial model categories descends to an adjunction of quasi-categories. Another proof of this result is given in \cite[5.2.4.6]{Lurie:2009fk}. 

Recall that the quasi-category associated to a simplicial model category $\lcat{A}$ is defined by restricting to the full simplicial subcategory $\lcat{A}_{cf}$ of fibrant-cofibrant objects and then applying the homotopy coherent nerve $\nrvhc\colon \sSet\text{-}\Cat \to \sSet$. 

\begin{thm}\label{thm:simplicial-Quillen-adjunction} A simplicial Quillen adjunction \[\adjdisplay f -| u : \lcat{A} -> \lcat{B}.\] between simplicial model categories gives rise to an adjunction between the quasi-categories $\nrvhc\lcat{A}_{cf}$ and $\nrvhc\lcat{B}_{cf}$.
\end{thm}
\begin{proof}
We introduce a pair of simplicial categories  $\coll(f,\lcat{A})$ and $\coll(\lcat{B},u)$, with $\lcat{B}$ and $\lcat{A}$ as full subcategories that are jointly surjective on objects. Declare the hom-spaces from $a \in \lcat{A}$ to $b \in \lcat{B}$ to be empty and define \[ \coll(f,\lcat{A})(b,a) \defeq \lcat{A}(fb,a) \qquad \coll(\lcat{B},u)(b,a) \defeq \lcat{B}(b,ua).\] The simplicial adjunction $f\dashv u$ is encoded in the proposition that the simplicial categories $\coll(f,\lcat{A})$ and $\coll(\lcat{B},u)$ are isomorphic under $\lcat{B} \coprod \lcat{A}$.

Now write $\coll(f,\lcat{A})_{cf} \cong \coll(\lcat{B},u)_{cf}$ for the full simplicial sub-categories spanned by the fibrant-cofibrant objects of $\lcat{A}$ and $\lcat{B}$.  Via these restrictions, we obtain a diagram \[ \lcat{B}_{cf} \hookrightarrow \coll(f,\lcat{A})_{cf} \cong \coll(\lcat{B},u)_{cf} \hookleftarrow \lcat{A}_{cf}\] of locally Kan simplicial categories. Applying the homotopy coherent nerve, we have a pair of isomorphic cospans in $\qCat_2$: \[\xymatrix{ \ar@{}[dr]|(.7){\Uparrow\psi} & \nrvhc\lcat{A}_{cf} \ar[d] & &  \ar@{}[dr]|(.7){\Downarrow\beta} & \nrvhc\lcat{B}_{cf} \ar[d]^i  \\ \nrvhc\lcat{B}_{cf} \ar@{-->}[ur]^{\overline{f}} \ar[r] & \nrvhc\coll(f,\lcat{A})_{cf} & & \nrvhc\lcat{A}_{cf} \ar@{-->}[ur]^{\overline{u}} \ar[r]_-j &\nrvhc\coll(\lcat{B},u)_{cf} }\] Our objective is to define an absolute left lifting $(\overline{f}, \psi)$ and an absolute right lifting $(\overline{u},\beta)$. Proposition \ref{prop:absliftingtranslation2} and its dual then provides a fibred equivalence \[ \overline{f} \comma \nrvhc\lcat{A}_{cf} \simeq \nrvhc\lcat{B}_{cf} \comma \overline{u}\] over $\nrvhc\lcat{A}_{cf} \times \nrvhc\lcat{B}_{cf}$, which by Proposition \ref{prop:adjointequivconverse} implies that $\adjinline \overline{f} -| \overline{u} : \nrvhc\lcat{A}_{cf} -> \nrvhc\lcat{B}_{cf}.$ is an adjunction of quasi-categories.

The arguments building the absolute right lifting diagram $(\overline{u},\beta)$ and the absolute left lifting diagram $(\overline{f},\psi)$ are entirely dual. Interpreting the statement of Theorem \ref{thm:pointwise} in this context, we are asked to produce, for each fibrant-cofibrant object $a \in \lcat{A}$, a terminal object in $\slicer{i}{ja}$, defined to be the pullback of the slice  quasi-category $\slicer{(\nrvhc\coll(\lcat{B},u)_{cf})}{a}$ along the natural inclusion $i \colon N\lcat{B}_{cf} \to N\coll(\lcat{B},u)_{cf}$. To that end, choose a  cofibrant replacement $q \colon t \to ua$ in the model category $\lcat{B}$ such that the map $q$ is a trivial fibration. It follows that whenever $b \in \lcat{B}$ is cofibrant, the natural map $q_*\colon \lcat{B}(b,t) \to \lcat{B}(b,ua)$ is a trivial fibration between Kan complexes. We claim that $q$ is terminal in $\slicer{i}{ja}$.

Let $\gC$ denote the left adjoint to the homotopy coherent nerve. Unpacking the definition, an $n$-simplex in $\slicer{i}{ja}$ is \[\xymatrix{  \Delta^n \ar[d]_{\face^{n+1}} \ar[r] & \nrvhc\lcat{B}_{cf} \ar[d]  & & \gC\Delta^n\ar[d]_{\face^{n+1}} \ar[r] & \lcat{B}_{cf} \ar[d]^i\\ \Delta^n\join\Delta^0\cong  \Delta^{n+1} \ar[r] & \nrvhc\coll(\lcat{B},u)_{cf} & \leftrightsquigarrow & \gC\Delta^{n+1} \ar[r] & \coll(\lcat{B},u)_{cf} \\ \Delta^{\fbv{n+1}} \ar[u] \ar[ur]_a  &  & & \catone \ar[u]^{\mathrm{last}} \ar[ur]_a } \]  The vertex $q \in \nrvhc\coll(\lcat{B},u)_{cf}$ is terminal if and only if we can extend any diagram of simplicial functors 
\begin{equation}\label{eq:simp.adj.extension}\xymatrix@C=30pt@R=35pt@!0{ \gC\boundary\Delta^n \ar[rrrr] \ar[dd] \ar[dr] & && & \lcat{B}_{cf} \ar[dr] \\ & \gC\Lambda^{n+1,n+1} \ar[dd] \ar[rrrr] & & & & \coll(\lcat{B},u)_{cf} \\ \gC\Delta^n \ar[dr]_{\face^{n+1}} \ar@{-->}[uurrrr] \\ & \gC\Delta^{n+1} \ar@{-->}[uurrrr]}\end{equation} in which the unique vertex in the hom-space between the objects $n$ and $n+1$ in the simplicial category $\gC\Lambda^{n+1,n+1}$ is mapped to $q \in \lcat{B}(t,ua)$. 

The simplicial categories $\gC\Lambda^{n+1,n+1}$ and $\gC\Delta^{n+1}$ have objects $0,\ldots, n+1$ and all but two of the same hom-spaces, the only exceptions being the hom-spaces from $0$ to $n$ and to $n+1$. We have $\gC\Delta^{n+1}(0,n) \cong (\Delta^1)^{n-1}$ and $\gC\Delta^{n+1}(0,n+1)\cong(\Delta^1)^n$, while $\gC\Lambda^{n+1,n+1}(0,n) \cong \boundary(\Delta^1)^{n-1}$ and $\gC\Lambda^{n+1,n+1}(0,n+1)$ is the open box $B \hookrightarrow (\Delta^1)^n$ with the interior of the $n$-cube and one face removed \cite[1.1.5.10]{Lurie:2009fk} and \cite[16.5.10]{Riehl:2014kx}. In this way, writing $b \in \lcat{B}$ for the image of the object 0, the extension problem \eqref{eq:simp.adj.extension} in the category of simplicial categories reduces to an extension problem
\[\xymatrix@C=30pt@R=35pt@!0{ \boundary(\Delta^1)^{n-1} \ar[rrrr] \ar@{ >->}[dd] \ar[dr] & && & \lcat{B}(b,t) \ar@{->>}[dr]^{q_*}_(.4){\rotatebox{135}{$\labelstyle\sim$}} \\ & B \ar@{ >->}[dd]^{\rotatebox{90}{$\labelstyle\sim$}} \ar[rrrr] & & & & \lcat{B}(b,ua) \\ (\Delta^1)^{n-1} \ar[dr] \ar@{-->}[uurrrr] \\ & (\Delta^1)^n \ar@{-->}[uurrrr]}\] in the category of simplicial sets. For the reader's convenience, we have used the standard decorations to mark cofibrations, fibrations, and weak equivalences in Quillen's model structure on simplicial sets.

The extension \eqref{eq:simp.adj.extension} may be achieved by first  extending along the map $B \hookrightarrow (\Delta^1)^n$ in the Kan complex $\lcat{B}(b,ua)$. This chooses an image under the map $q_*$ for the $(n-1)$-cube missing from the box $B$. An $(n-1)$-cube in $\lcat{B}(b,t)$ with this image can be found by lifting the cofibration $\boundary(\Delta^1)^{n-1} \hookrightarrow (\Delta^1)^{n-1}$ against the trivial fibration $q_*$.
\end{proof}




%\appendix
%%!TEX root = all.tex
% ******************************************************************
% ** Title:            The 2-category theory of quasi-categories
% **                   geometric background
% ** Precis:        
% ** Author:           Emily Riehl and Dominic Verity
% ** Commenced:        2/3/2012
% ******************************************************************


\newcommand{\captionwidth}{14cm}

\newcommand{\agt}{\,\rlap{\lower 3.5 pt \hbox{$\mathchar \sim$}} \raise 1pt
 \hbox {$>$}\,}
\newcommand{\alt}{\,\rlap{\lower 3.5 pt \hbox{$\mathchar \sim$}} \raise 1pt
 \hbox {$<$}\,}

\newcommand{\gsim}{\;\rlap{\lower 3.5 pt \hbox{$\mathchar \sim$}} \raise 1pt
 \hbox {$>$}\;}
\newcommand{\lsim}{\;\rlap{\lower 3.5 pt \hbox{$\mathchar \sim$}} \raise 1pt
 \hbox {$<$}\;}
\newcommand{\dd}{{\rm d}}                  % differential
\newcommand{\bld}[1]{\boldmath{$#1$}}      % bold math symbols
\newcommand{\sprod}[2]{#1\hspace{-.1em}\cdot\hspace{-.1em}#2}   % dot-product
\newcommand{\llangle}{\left\langle}        
\newcommand{\rrangle}{\right\rangle}

\renewcommand{\Re}{{\rm Re}}
\renewcommand{\Im}{{\rm Im}}
\newcommand{\msbar}{$\overline{\rm MS}$}

\newcommand{\yp}{y}
\newcommand{\logtwos}{L_{tW}}
\newcommand{\logtwms}{l_{tW}}
\newcommand{\logctheta}{l_\Theta}
\newcommand{\logmuW}{l_{\mu W}}

\newcommand{\logqmms}{l_{qm}}
\newcommand{\logqmos}{L_m}
\newcommand{\logmsms}{l_s}
\newcommand{\logmsos}{L_{ms}}
\newcommand{\logmusos}{L_{s\mu}}
\newcommand{\logqmums}{l_{q\mu}}
\newcommand{\logmum}{l_{\mu m}}






\section{Geometry}\label{app:geometry}

Our approach to developing the category theory of quasi-categories makes use of the enriched category theories of 2-categories and simplicial categories. Traditional accounts of quasi-category theory have instead employed ``d{\'e}calage'' constructions to define and develop the theory of limits and colimits, adjunctions, and so forth. In this appendix, we demonstrate that these approaches are entirely equivalent by showing that d{\'e}calage constructions may be obtained, in an up to  equivalence sense, using constructions involving the comma quasi-categories introduced in definition~\ref{def:comma-obj}.

The literature already contains a proof that these two constructions are equivalent; see for instance Lurie~\cite[4.2.1.5]{Lurie:2009fk}. However, given the importance of this result to our work, we  beg the indulgence of the reader and devote this appendix to providing a very concrete, explicit, and self-contained presentation of this result.  

  We begin in section \ref{subsec:join} by reviewing Joyal's join and slice constructions, the d{\'e}calage-style constructions mentioned above. The left and right slices associated to a map $f \colon X \to A$ whose codomain is a quasi-category can be interpreted as the quasi-category of cones under and over $f$ respectively. In section \ref{subsec:cones}, we described a variant ``fat cone'' construction as a comma quasi-category. The quasi-categories of fat cones appeared in the definition of limits and colimits expressed by proposition~\ref{prop:limits.as.terminal.objects}. In section \ref{subsec:fatjoin}, we introduce a fattened version of the join and slice constructions and prove that the fat slice construction is isomorphic to the fat cone construction. It remains only to show that ordinary slices and fat slice are equivalent. We prove this in section \ref{subsec:relating} by proving that a natural map from the fat join to the join is an equivalence for any pair of simplicial sets.
 
  \subsection{Joins and slices}\label{subsec:join}

  \begin{rec}[joins and d{\'e}calage]\label{rec:join-dec} The algebraists' skeletal category $\Del+$ of all finite ordinals and order preserving maps supports a canonical strict (non-symmetric) monoidal structure $(\Del+,\oplus,[-1])$ in which $\oplus$ denotes the {\em ordinal sum\/} given 
  \begin{itemize}
    \item for objects $[n],[m]\in\Del+$ by $[n]\oplus[m] \defeq [n+m+1]$,
    \item for arrows $\alpha\colon[n]\to[n'], \beta\colon[m]\to[m']$ by $\alpha\oplus\beta\colon[n+m+1]\to[n'+m'+1]$ defined by
  \begin{equation*}
    \alpha\oplus\beta(i) =
    \begin{cases}
    \alpha(i)& \text{if $i\leq n$,} \\
    \beta(i-n-1) + n' + 1& \text{otherwise.}
    \end{cases}
  \end{equation*}
  \end{itemize}
By Day convolution, this bifunctor extends to a (non-symmetric) monoidal closed structure $(\asSet, \join, \Del^{-1}, \dec_l, \dec_r)$ on the category of augmented simplicial sets. Here the monoidal operation $\join$ is known as the {\em simplicial join\/} and its closures $\dec_l$ and $\dec_r$ are known as the {\em left and right d{\'e}calage constructions}, respectively.   To fix handedness, we declare that for each augmented simplicial set $X$ the functor $\dec_l(X,{-})$ (resp.\ $\dec_r(X,{-})$) is right adjoint to $X\join{-}$ (resp.\ ${-}\join X$).

  If $X$ and $Y$ are augmented simplicial sets then their join $X\join Y$ may be described explicitly as follows:
  \begin{itemize}
    \item it has simplices pairs $(x,y) \in (X\join Y)_{r+s+1}$ with $x\in X_r$, $y\in Y_s$,
    \item if $(x,y)$ is a simplex of $X\join Y$ with $x \in X_r$ and $y \in Y_s$ and $\alpha\colon[n]\to[r+s+1]$ is a simplicial operator in $\Del+$, then $\alpha$ may be uniquely decomposed as $\alpha=\alpha_1\join\alpha_2$ with $\alpha_1\colon[n_1]\to[r]$ and $\alpha_2\colon[n_2]\to[s]$, and we define $(x,y)\cdot\alpha\defeq (x\cdot\alpha_1,y\cdot\alpha_2)$. 
  \end{itemize}
  Furthermore, if $f\colon X\to X'$ and $g\colon Y\to Y'$ are simplicial maps then the simplicial map $f\join g\colon X\join Y\to X'\join Y'$ simply carries the simplex $(x,y)\in X\join Y$ to the simplex $(f(x),g(y))\in X'\join Y'$. 
  \end{rec}

  \begin{defn}[d{\'e}calage and slices]\label{defn:slices}
    In~\cite{Joyal:2002:QuasiCategories}, Joyal introduces a {\em slice\/} construction for maps $f\colon X\to A$ of simplicial sets. To describe this construction, we start by identifying the category of simplicial sets $\sSet$ with the full subcategory of terminally augmented simplicial sets in $\asSet$, fixing a simplicial set $X$ and defining a functor 
    \begin{equation*}
      {-}\mathbin{\bar\join} X\colon \xy<0em,0em>*+{\sSet}\ar <6em,0em>*+{X\slice\sSet}\endxy\mkern30mu
      (\text{resp.\ } 
      X\mathbin{\bar\join}{-}\colon \xy<0em,0em>*+{\sSet}\ar <6em,0em>*+{X\slice\sSet}\endxy)
    \end{equation*}
    which carries a simplicial set $Y\in\sSet$ to the object ${*}\join X\colon X\cong\Del^{-1}\join X\to Y\join X$ (resp.\ $X\join{*}\colon X\cong X\join\Del^{-1} \to X\join Y$) induced by the map ${*}\colon\Del^{-1}\to Y$ corresponding to the unique $(-1)$-simplex of $Y$. Now we may show that this functor preserves all colimits, from which fact we may infer that it possesses a right adjoint $\slc^X_r({-})$ (resp,\ $\slc^X_l({-})$). 
    
    However, we prefer to derive these right adjoints from the d{\'e}calage construction of recollection~\ref{rec:join-dec}. Specifically, observe that the $(-1)$-dimensional simplices of $\dec_r(X, A)$ (resp.\ $\dec_l(X,A)$) are in bijective correspondence with simplicial maps $f\colon X\to A$. So if we are given such a simplicial map we may, by recollection \ref{rec:augmentation}, extract the component of $\dec_r(X,A)$ (resp.\ $\dec_l(X,A)$) consisting of those simplices whose $(-1)$-face is $f$, which we denote by $\slicer{A}{f}$ (resp.\ $\slicel{f}{A}$) and call the {\em slice of $A$ over (resp.\ under) $f$}. Now it is a matter of an easy calculation to demonstrate directly that $\slicer{A}{f}$ (resp. $\slicel{f}{A}$) has the universal property required of the right adjoint to ${-}\bar\join X$ (resp.\ $X\bar\join {-}$) at the object $f\colon X\to A$ of $X\slice\sSet$.

    In other words, these d{\'e}calages admit the following canonical decompositions as disjoint unions of (terminally augmented) slices: 
    \begin{equation*}
      \dec_r(X,A)=\bigsqcup_{f\colon X\to A} (\slicer{A}{f})\mkern30mu
      \dec_l(X,A)=\bigsqcup_{f\colon X\to A} (\slicel{f}{A})
    \end{equation*}

    We think of the slice $\slicel{f}{A}$ as being the simplicial set of {\em cones under the diagram $f$\/} and we think of the dual slice $\slicer{A}{f}$ as being the simplicial set of {\em cones over the diagram $f$}.
  \end{defn}
  
  \begin{obs}[slices of quasi-categories]\label{obs:slice-and-qcats}
    A direct computation from the explicit description of the join construction given above demonstrates that the Leibniz join (see recollection~\ref{rec:leibniz}) of a horn and a boundary $(\Horn^{n,k}\inc\Del^n)\leib\join(\boundary\Del^m\inc\Del^m)$ is again isomorphic to a single horn $\Horn^{n+m+1,k}\inc\Del^{n+m+1}$. Dually the Leibniz join $(\boundary\Del^n\inc\Del^n)\leib\join(\Horn^{m,k}\inc\Del^m)$ is isomorphic to the single horn $\Horn^{n+m+1,n+k+1}\inc\Del^{n+m+1}$. 

    Combining these computations with the properties of the Leibniz construction developed in \cite[section~\ref*{reedy:sec:Leibniz-Reedy}]{RiehlVerity:2013kx}, we may show that an augmented simplicial set $A$ has the right lifting property with respect to all (inner) horn inclusions then so do the left and right d{\'e}calages $\dec_l(X,A)$ and $\dec_r(X,A)$ for any augmented simplicial set $X$. In particular, this tells us that if $f\colon X\to A$ is any map of simplicial sets and $A$ is a quasi-category then the slices $\slicel{f}{A}$ and $\slicer{A}{f}$ are also quasi-categories. 

    Working in the marked context, we may extend this result to Leibniz joins with specially marked outer horns. That then allows us to prove that if $p\colon A\to B$ is an isofibration of quasi-categories and $f\colon X\to A$ is any simplicial map then the induced simplicial maps $\slc^X_r(p)\colon \slicer{A}{f}\to \slicer{B}{pf}$ and $\slc^X_l(p)\colon \slicel{f}{A}\to \slicel{pf}{B}$ are also isofibrations of quasi-categories.
  \end{obs}

  \subsection{Fat cones}\label{subsec:cones}
  
  \begin{obs}[internal homs and fat cones]\label{obs:fat-cones}
    It is common in category theory to define a cone over (resp.\ cone under) a diagram $f\colon X\to A$ with vertex $a\in A$ to be a natural transformation $\pi\colon a\Rightarrow f$ (resp. $\iota\colon f\Rightarrow a$). Here we use the notation $a$ to denote both an object of $A$ and the corresponding constant functor $X\to A$ at that object. 

    We might reasonably hope to generalise this notion of cone to the quasi-categorical context, by saying that cone over (resp.\ under) a diagram $f\colon X\to A$, whose target is a quasi-category $A$, with vertex $a \in A_0$, is a simplicial map $\pi\colon X\times\Del^1\to A$ (resp.~$\iota\colon X\times\Del^1\to A$) for which $a=\pi\circ(X\times\face^1)$ and $f=\pi\circ(X\times\face^0)$  (resp.~$f=\iota\circ(X\times\face^1)$ and $a=\iota\circ(X\times\face^0)$). Here again, we use the notation $a$ to denote the constant simplicial map which carries every $n$-simplex $x$ of $X$ to the degenerate $n$-simplex on the vertex $a$. This cone notion gives rise to a different construction which is designed to look and behave like a quasi-category of cones over (or under) a diagram. 

    We may make this intuition precise using the comma objects of definition~\ref{def:comma-obj}. To that end suppose that $f\colon X\to A$ is a simplicial map which we can regard equally as being a $0$-simplex of the internal hom $A^X$ or as a simplicial map $\Del^0\to A^X$. Define the {\em constant diagram map\/} $\const\colon A\to A^X$ to be the adjoint transpose of the projection $\pi_A\colon A\times X\to A$. We may form the comma object $f\comma c$ in the following pullback 
    \begin{equation}\label{eq:fat-cones}
      \xymatrix@=2.5em{
        {f\comma c}\pbexcursion \ar[r]\ar@{->>}[d]_{p} &
        {(A^X)^\cattwo} \ar@{->>}[d]^{(p_1,p_0)} \\
        {A\times \Del^0} \ar[r]_-{c\times f} & {A^X\times A^X}
      }
    \end{equation}
    and we choose to call this the simplicial set of {\em fat cones under\/} the diagram $f\colon X\to A$. Dually we call the comma object $c\comma f$ the simplicial set of fat cones {\em over\/} the diagram $f$. Unwinding this definition, we find that a 0-simplex of $f\comma c$, that is to say a fat cone under $f$, is no more nor less than a cone in the  sense just discussed.
  \end{obs}
  
  \begin{obs}[fat cones and quasi-categories]\label{obs:fat-cone-quasicat}
    Because $A$ is supposed to be a quasi-category then we know by the comments at the end of recollection~\ref{rec:qmc-quasicat} that $A^X$ is also a quasi-category, as is $\Del^0$. So the discussion in definition~\ref{def:comma-obj} tells us that the simplicial set of fat cones $f\comma c$ (resp.\ $c\comma f$) is again a quasi-category. Furthermore, if $p\colon A\tfib B$ is an isofibration between quasi-categories then the map $p^X\colon A^X\tfib B^X$ is also an isofibration by observation~\ref{obs:isofibration-closure}. So we may apply the result of observation~\ref{obs:comma-obj-maps} to the diagram
    \begin{equation*}
      \xymatrix{
        {\Del^0}\ar[r]^{f}\ar@{=}[d] & 
        {A^X}\ar@{->>}[d]_-{p^X} & 
        {A}\ar[l]_-{c}\ar@{->>}[d]^-{p} \\
        {\Del^0}\ar[r]_{pf} & 
        {B^X} & 
        {B}\ar[l]^-{c} 
      }
    \end{equation*}
    to show that the induced maps $f\comma c\tfib pf\comma c$ and $c\comma f\tfib c\comma pf$ between the quasi-categories of fat cones are also isofibrations.
  \end{obs}
  
  \subsection{Fat joins and fat slices}\label{subsec:fatjoin}

  \begin{rmk}[relating slices and fat cone constructions]
    One might na{\"\i}vely expect that the two cone notions we have met thus far, these being Joyal's slices and our fat cone construction respectively, actually coincide. In general this is certainly a forlorn hope, as fat cones contain many more simplices than the corresponding Joyal cones. However in the case where $A$ is actually a category it is a classical fact that these notions are isomorphic, and we might at least hope that they are also related {\em up to equivalence\/} in the quasi-categorical context.

This kind of result is of great significance to the study limits and colimits in quasi-categories. In order to take advantage of the weak universal properties of the comma quasi-category construction, we have defined a limit (resp.\ colimit) of a diagram $f\colon X\to A$ to be be a terminal (resp.\ initial) object in the quasi-category $c\comma f$ (resp.\ $f\comma c$) of fat cones. On the other hand, in Joyal~\cite{Joyal:2002:QuasiCategories} and Lurie~\cite{Lurie:2009fk} one finds a limit (resp.\ colimit) of $f$ defined to be a terminal (resp.\ initial) object in the slice $\slicer{A}{f}$ (resp. $\slicel{f}{A}$). To reconcile these definitions it will be enough to demonstrate that the slice and fat cone constructions are related by appropriate equivalences of quasi-categories.
  \end{rmk}

  \begin{defn}[fat join and fat d{\'e}calage]\label{def:fat-join}
    We define the {\em fat join} of two simplicial sets $X$ and $Y$ to be the simplicial set $X\fatjoin Y$ constructed by means of the following pushout:
    \begin{equation}\label{eq:fat-join-def}
      \xymatrix@R=2em@C=4em{
        {({X}\times{Y})\sqcup({X}\times{Y})}\ar[r]^-{\pi_X\sqcup\pi_Y}
        \ar[d]_{\langle{X}\times\face^1\times{Y},
          {X}\times\face^0\times{Y}\rangle} &
        {{X}\sqcup{Y}}\ar[d] \\
        {{X}\times\Del^1\times{Y}} \ar[r] &
        {{X}\fatjoin{Y}}\poexcursion
      }
    \end{equation}
    We may extend this construction to simplicial maps in the obvious way to give us a bifunctor $\fatjoin\colon\sSet\times\sSet\to\sSet$, and it is clear that this preserves {\em connected\/} colimits in each variable. We might caution the reader here in regard to the preservation of {\em all\/} colimits in each variable; this result does not hold simply because the coproduct bifunctor $\sqcup$ (as used in the top right hand corner of the defining pushout above) fails to preserve coproducts  in each variable (while it does preserve connected colimits). In particular, a fat join of a simplicial set $X$ with the empty simplicial set is not itself empty: it is canonically isomorphic to $X$ itself.

    Now if we again identify the category of simplicial sets $\sSet$ with the full subcategory of terminally augmented simplicial sets then we may extend our fat join to a bifunctor on augmented simplicial sets. To do this we start by observing that every augmented simplicial set $X$ may be written canonically as a coproduct $\bigsqcup_{i\in I} X_i$ in $\asSet$ of terminally augmented simplicial sets. So if $X = \bigsqcup_{i\in I} X_i$ and $Y=\bigsqcup_{j\in J} Y_j$ are two augmented simplicial sets with terminally augmented components $X_i$ and $Y_j$, then we define $X\join Y$ to be the coproduct $\bigsqcup_{i\in I, j\in J} X_i\fatjoin Y_j$ in $\asSet$. We extend this to maps of augmented simplicial sets in the obvious way, using the fact that we may decompose such maps into families of maps between terminally augmented components. It is now a routine matter to verify that, when regarded as being a bifunctor on $\asSet$, the fat join does indeed preserve {\em all\/} colimits in each variable. Consequently, since $\asSet$ is a presheaf category, it follows that the fat join bifunctor on $\asSet$ has both left and right closures $\fatdec_l(X,A)$ and $\fatdec_r(X,A)$, called  {\em left and right fat d{\'e}calage\/} respectively, which notation we fix by declaring that if $X$ is an augmented simplicial set then $X\fatjoin{-}\dashv\fatdec_l(X,{-})$ and ${-}\fatjoin X\dashv\fatdec_r(X,{-})$.

    Returning to the defining pushout~\eqref{eq:fat-join-def} it will be of use to observe that the fat join of two {\em non-empty\/} simplicial sets $X$ and $Y$ may be described more concretely as the simplicial set obtained by taking the quotient of ${X}\times\Del^1\times{Y}$ under the simplicial congruence relating the pairs of $r$-simplices
    \begin{equation}\label{eq:fat-join-cong}
    (x,0,y)\sim (x,0,y')\quad \text{and}\quad (x,1,y) \sim (x',1,y) 
    \end{equation}
    where $0$ and $1$ denote the constant operators $[r]\to[1]$. We use square bracketed triples $[x,\beta,y]_\sim$ to denote equivalence classes under $\sim$. 
  \end{defn}

  \begin{defn}[fat slice]\label{defn:fat-slices}
    Replaying Joyal's slice construction of definition~\ref{defn:slices}, if $X$ is a simplicial set, we may use the fat join to construct a functor
    \begin{equation*}
      {-}\mathbin{\bar\fatjoin} X\colon \xy<0em,0em>*+{\sSet}\ar <6em,0em>*+{X\slice\sSet}\endxy\mkern30mu
      (\text{resp.\ } 
      X\mathbin{\bar\fatjoin}{-}\colon \xy<0em,0em>*+{\sSet}\ar <6em,0em>*+{X\slice\sSet}\endxy)
    \end{equation*}
    which carries a simplicial set $Y\in\sSet$ to the object ${*}\fatjoin X\colon X\cong\Del^{-1}\fatjoin X\to Y\fatjoin X$ (resp.\ $X\fatjoin{*}\colon X\cong X\fatjoin\Del^{-1} \to X\fatjoin Y$) and use fat d{\'e}calage to show that it has a right adjoint $\fatslc^X_r({-})$ (resp.\ $\fatslc^X_l({-})$). The value of this right adjoint at an object $f\colon X\to A$ of $X\slice\sSet$ is denoted $\fatslicer{A}{f}$ (resp.\ $\fatslicel{f}{A}$) and is called the {\em fat slice of $A$ over (resp.\ under) $f$}. 
    \end{defn}
    
    \begin{obs}[fat slice vs fat cone]\label{obs:fat-slices}
    If $Y$ is any simplicial set and $f\colon X\to A$ is a simplicial map, then simplicial maps $Y\to \fatslicer{A}{f}$ are in bijective correspondence with simplicial maps $k\colon Y\fatjoin X\to A$ for which the following square commutes:
    \begin{equation*}
      \xymatrix@=2em{
        {\Del^{-1}\fatjoin X}\ar[d]_{{*}\fatjoin X}\ar@{}[r]|-{\textstyle\cong} &
        {X}\ar[d]^f \\
        {Y\fatjoin X}\ar[r]_-{k} & {A}
      }
    \end{equation*}
    Using the universal property of the defining pushout~\eqref{eq:fat-join-def} we see that such maps are themselves in bijective correspondence with pairs of maps $\overline{k}\colon Y\times\Del^1\times X\to A$ and $g\colon Y\to A$ which make the square
    \begin{equation*}
      \xymatrix@R=2em@C=4em{
        {(Y\times X)\sqcup(Y\times X)}\ar[r]^-{\pi_Y\sqcup\pi_X}
        \ar[d]_{\langle Y\times\face^1\times X,
          Y\times\face^0\times X\rangle} &
        {Y\sqcup X}\ar[d]^{\langle g, f\rangle} \\
        {Y\times\Del^1\times X} \ar[r]_-{\overline{k}} &
        {A}
      }
    \end{equation*}
    commute. Taking the transpose $\hat{k}\colon Y\to (A^X)^{\Del^1}$ of $\overline{k}$ under the adjunction ${-}\times\Del^1\times X\dashv ((-)^{X})^{\Del^1}$ we see that the above square commutes if and only if the dual square
    \begin{equation*}
      \xymatrix@R=2em@C=4em{
        {Y}\ar[r]^{\hat{k}}\ar[d]_{(!,g)} & {(A^X)^{\Del^1}} 
        \ar[d]^{((A^X)^{\face^0}, (A^X)^{\face^1})} \\
        { \Del^0\times A}\ar[r]_-{f\times c} & {A^X\times A^X}
      }
    \end{equation*}
    commutes. Finally, consulting the defining pullback~\eqref{eq:fat-cones} for the fat cone construction, we see that pairs $(g,\hat{k})$ which make the square above commute are in bijective correspondence with simplicial maps $Y\to c\comma f$. 
  \end{obs}
  
With this argument we have proven:
  
  \begin{prop}\label{prop:fatsliceisfatcone}
  The fat slice construction $\fatslicer{A}{f}$ (resp.\ $\fatslicel{f}{A}$) is isomorphic to the fat cone construction $c\comma f$ (resp.\ $f\comma c$).\qed
  \end{prop}

  \subsection{Relating joins and fat joins}\label{subsec:relating}

  \begin{obs}[comparing join constructions]
    When $\beta\colon [n]\to [1]$ is a simplicial operator let $\hat{n}_\beta$ denote the largest integer in the set $\{-1\}\cup\{i\in[n]\mid \beta(i)=0\}$  and let $\check{n}_\beta=n-1-\hat{n}_\beta$. Define an associated pair $\hat\beta\colon[\hat{n}_\beta]\to[n]$ and $\check\beta\colon[\check{n}_\beta]\to[n]$ of simplicial face operators in $\Del+$ by $\hat\beta(i) = i$ for all $i\in[\hat{n}_\beta]$ and $\check\beta(j)=j+\hat{n}_\beta + 1$ for all $j\in[\check{n}_\beta]$. 
    
    Now if $X$ and $Y$ are (terminally augmented) simplicial sets we may define a map $\bar{s}^{X,Y}$ which carries an $n$-simplex $(x,\beta,y)$ of $X\times \Del^1\times Y$ to the $n$-simplex $(x\cdot\hat\beta,y\cdot\check\beta)$ of $X\join Y$. A straightforward calculation, using the explicit description of the simplicial action on $X\join Y$ given at the end of recollection~\ref{rec:join-dec}, demonstrates that this map commutes with the simplicial actions on these sets and is thus a simplicial map. Furthermore, the family of simplicial maps $\bar{s}^{X,Y}\colon X\times \Del^1\times Y \to X\join Y$ is natural in $X$ and $Y$.

    Of course, since $X$ and $Y$ are terminally augmented, we also have canonical maps $l^{X,Y}\colon X\cong X\join\Del^{-1}\to X\join Y$ and $r^{X,Y}\colon Y\cong\Del^{-1}\join Y\to X\join Y$ and we may assemble all these maps together into a commutative square 
    \begin{equation}\label{eq:join-comp-def}
      \xymatrix@R=2em@C=4em{
        {({X}\times{Y})\sqcup({X}\times{Y})}\ar[r]^-{\pi_X\sqcup\pi_Y}
        \ar[d]_{\langle{X}\times\face^1\times{Y},
          {X}\times\face^0\times{Y}\rangle} &
        {{X}\sqcup{Y}}\ar[d]^{\langle l^{X,Y}, r^{X,Y}\rangle} \\
        {{X}\times\Del^1\times{Y}} \ar[r]_{\bar{s}^{X,Y}} &
        {{X}\join{Y}}
      }
    \end{equation}
    whose maps are all natural in $X$ and $Y$. Using the defining universal property of fat join, as given in~\eqref{eq:fat-join-def}, these squares induce maps $s^{X,Y}\colon X\fatjoin Y\to X\join Y$ which are again natural in $X$ and $Y$. Should we so wish, we may now take suitable coproducts of these maps to canonically extend this family of simplicial maps to a natural transformation between the extended fat join and join bifunctors on augmented simplicial sets.

    More explicitly, if $n,m\geq 0$, then $\bar{s}^{n,m}\colon\Del^n\times\Del^1\times\Del^m\to\Del^{n+m+1}$ is the unique simplicial map determined by the (order preserving) action on vertices given by:
    \begin{equation}\label{eq:tnm-def}
      \bar{s}^{n,m}(i,j,k) = 
      \begin{cases}
        i & \text{if $j=0$, and} \\
        k+n+1 & \text{if $j=1$.}
      \end{cases}
    \end{equation}
    This takes simplices related under the congruence defined in~\eqref{eq:fat-join-def} of definition~\ref{def:fat-join} to the same simplex and thus induces a unique map $s^{n,m}\colon\Del^n\fatjoin\Del^m\to\Del^n\join\Del^m$ on the quotient simplicial set.
  \end{obs}

Our immediate aim, achieved in proposition~\ref{prop:join-fatjoin-equiv}, is to show that the maps $s^{X,Y}$ are weak equivalences in the Joyal model structure for any pair of simplicial sets $X$ and $Y$. To that end,  we will use explicit cylinder objects for the model structure of naturally marked quasi-categories  to prove that certain maps are weak equivalences in the Joyal model structure. We first consider the na\"{i}ve choice and then use lemma~\ref{lem:pointwise-equiv} to construct the cylinder object which we will make use of below.%Dom, is this okay?

  \begin{obs}\label{obs:Kan-cylinder}  When $n=1$, the two specially marked $1$-horn inclusions are simply the inclusion maps $\face^1,\face^0\colon\Del^0\inc(\Del^1)^\sharp$; by recollection~\ref{rec:qmc-quasi-marked} these are trivial cofibrations in the model structure of naturally marked quasi-categories. So if $X$ is any marked simplicial set then the canonical ``end point'' inclusions $i_0,i_1\colon X\inc X\times(\Del^1)^\sharp$ are both trivial cofibrations, simply because they are isomorphic to the maps obtained by taking the product of the cofibrant object $X$ with the trivial cofibrations $\face^1,\face^0\colon\Del^0\inc(\Del^1)^\sharp$ in a cartesian model structure. It follows then that the maps $i_0,i_1\colon X\inc X\times(\Del^1)^\sharp$ and the projection map $\pi\colon X\times(\Del^1)^\sharp\to X$ display $X\times(\Del^1)^\sharp$ as a canonical cylinder object for $X$ in the model structure of naturally marked quasi-categories.
\end{obs}

    \begin{defn}\label{defn:Joyal-cylinder}
      For a marked simplicial set $X$, define $\cyl(X)$ to be the marked simplicial set derived from $X\times\Del^1$ by also marking any $1$-simplex of the form $(x\cdot\degen^0,\id_{[1]})$ for some $0$-simplex $x\in X$. Observe that  $\cyl(X)$ is a marked superset of $X\times\Del^1$ and a marked subset of $X\times(\Del^1)^\sharp$. Consequently the projection map $\pi\colon X\times(\Del^1)^\sharp\to X$ restricts to a map $\pi\colon \cyl(X)\to X$ and the ``end point'' inclusion maps $i_0,i_1\colon X\inc X\times\Del^1$ extend to maps $i_0,i_1\colon X\inc\cyl(X)$.
    \end{defn}

    \begin{lem}\label{lem:cyl-obj}
      The end point inclusion maps $i_0,i_1\colon X\inc\cyl(X)$ and the projection map $\pi\colon\cyl(X)\to X$ present $\cyl(X)$ as a cylinder object for $X$ in the model structure of naturally marked quasi-categories.
    \end{lem}

    \begin{proof}
      This is mostly immediate from the definitions above, but we do need to demonstrate that the map $\pi\colon\cyl(X)\to X$ is a weak equivalence in the model structure of naturally marked quasi-categories. Equivalently, by the 2-of-3 property for weak equivalences, we may show that either of the inclusions $i_0, i_1\colon X\inc\cyl(X)$ is a trivial cofibration in that model structure. In other words, we must show that $i_0$ possesses the left lifting property with respect to all isofibrations $p\colon A\to B$ of naturally marked quasi-categories. 
      
      The proof of this lifting result is illustrated in the following diagram:
      \begin{equation*}
      \xymatrix@C=10em@R=1.5em{
        {X}\ar[r]^u\ar@{u(->}[d]_-{i_0} & {A}\ar[d]^p \\
        {\cyl(X)}\ar[r]^-v 
        \save +<5em,-2em>*+{X\times(\Del^1)^\sharp}="one"\restore
        & {B}
        \ar@{u(->} "2,1";"one" \ar@{..>} "one";"2,2"_-{v'}
        \ar@{-->} "one";"1,2"^(0.6){l}
      }
      \end{equation*}
      Here we are asked to construct a lift in the square of maps whose horizontals are labelled $u$ and $v$ and whose verticals are $i_0$ and $p$. So we start by observing that the dotted extension $v'$ of $v$ along the inclusion $\cyl(X)\inc X\times(\Del^1)^\sharp$ exists by lemma~\ref{lem:pointwise-equiv} and the assumption that $B$ is a naturally marked quasi-category. As just observed, the inclusion $i_0\colon X\inc\cyl(X)\inc X\times(\Del^1)^\sharp$ is a trivial cofibration in the model structure of naturally marked quasi-categories. So the dashed lift $l$ exists, because $p$ is an isofibration by assumption, and we may restrict that map along the inclusion $\cyl(X)\inc X\times(\Del^1)^\sharp$ to give the required lift for our original square. 
    \end{proof}

The following combinatorial proposition is absolutely crucial in the proof that the join and fat join constructions are equivalent.

  \begin{prop}\label{prop:join-fatjoin-equiv-simplices}
    For all $n,m\geq 0$, the comparison simplicial map $s^{n,m}\colon\Del^n\fatjoin\Del^m\to\Del^n\join\Del^m$ is a deformation retraction in the Joyal model structure and in particular is a weak equivalence. Furthermore, if $n$ or $m$ is equal to $-1$ then $s^{n,m}$ is an isomorphism.
  \end{prop}

  \begin{proof}
    We start by thinking of the map $\bar{s}^{n,m}\colon \Del^n\times\Del^1\times\Del^m\to\Del^{n+m+1}$ in terms of its action on $0$-simplices. In other words, we regard it as being the order preserving map from $[n]\times[1]\times[m]$ to $[n+m+1]$ described in~\eqref{eq:tnm-def} above. Now we may define an order preserving map $\bar{t}^{n,m}\colon [n+m+1]\to [n]\times[1]\times[m]$ pointing in the opposite direction by 
    \begin{equation*}
      \bar{t}^{n,m}(i) =
      \begin{cases}
        (i,0,0) & \text{if $i\leq n$ and} \\
        (n,1,i-n-1) & \text{if $i>n$}
      \end{cases}
    \end{equation*}
    and comment immediately that $\bar{s}^{n,m}\circ \bar{t}^{n,m} = \id_{[n+m+1]}$. To compare the obverse composite $\bar{t}^{n,m}\circ \bar{s}^{n,m}$ with the identity on $[n]\times[1]\times[m]$, first observe that this particular composite is given by the following explicit formula:
    \begin{equation*}
      (\bar{t}^{n,m}\circ \bar{s}^{n,m})(i,j,k) = 
      \begin{cases}
        (i,0,0) & \text{if $j=0$ and} \\
        (n, 1, k) & \text{if $j=1$.}
      \end{cases}
    \end{equation*}
    Now we may define a related order preserving endo-map $\bar{u}^{n,m}$ on $[n]\times[1]\times[m]$ by
    \begin{equation*}
      \bar{u}^{n,m}(i,j,k) =
      \begin{cases}
        (i,0,0) & \text{if $j=0$ and} \\
        (i,1,k) & \text{if $j=1$}
      \end{cases}
    \end{equation*}
    and observe that in the pointwise ordering on such maps we have $\bar{u}^{n,m} \leq \bar{t}^{n,m}\circ \bar{s}^{n,m}$ and $\bar{u}^{n,m} \leq \id_{[n]\times[1]\times[m]}$. Taking nerves, the map $\bar{t}^{n,m}$ becomes a simplicial map from $\Del^{n+m+1}$ to $\Del^n\times\Del^1\times\Del^m$, the map $\bar{u}^{n,m}$ becomes a simplicial endo-map on $\Del^n\times\Del^1\times\Del^m$ and the inequalities of the last sentence become $1$-simplices      $\bar{h}^{n,m},\bar{k}^{n,m}$ in $(\Del^n\times\Del^1\times\Del^m)^{\Del^n\times\Del^1\times\Del^m}$
    \[\xymatrix{ \Del^n\times\Del^1\times\Del^m \ar[d]_{\face^1} \ar[dr]^{\bar{u}^{n,m}} &  \\ (\Del^n \times\Del^1\times\Del^m)\times\Del^1 \ar[r]^-{\bar{h}^{n,m}} & \Del^n\times\Del^1\times\Del^m  \\ \Del^n\times\Del^1\times\Del^m\ar[u]^{\face^0} \ar[ur]_*+{\labelstyle \bar{t}^{n,m}\circ\bar{s}^{n,m}} & } \xymatrix{ \Del^n\times\Del^1\times\Del^m\ar[d]_{\face^1} \ar[dr]^{\bar{u}^{n,m}} \\ (\Del^n\times\Del^1\times\Del^m)\times\Del^1\ar[r]^-{\bar{k}^{n,m}} & \Del^n\times\Del^1\times\Del^m\\   \Del^n\times\Del^1\times\Del^m \ar[u]^{\face^0} \ar[ur]_*+{\labelstyle \id_{\Del^n\times\Del^1\times\Del^m}}}  \]
which connect the $0$-simplices $\bar{u}^{n,m}$ to $\bar{t}^{n,m}\circ \bar{s}^{n,m}$ and $\id_{\Del^n\times\Del^1\times\Del^m}$ respectively. The composites $\bar{h}^{n,m}\circ(\bar{t}^{n,m}\times\Del^1), \bar{k}^{n,m}\circ(\bar{t}^{n,m}\times\Del^1)\colon \Del^{n+m+1}\times\Del^1\to\Del^n\times\Del^1\times\Del^m$ are both equal to the degenerate $1$-simplex derived from the $0$-simplex $\bar{t}^{n,m}$ in $(\Del^n\times\Del^1\times\Del^m)^{\Del^{n+m+1}}$.

    Passing to quotients under the congruence $\sim$ defined in~\eqref{eq:fat-join-cong}, it is easily verified that these maps induce simplicial maps $t^{n,m}\colon\Del^n\join\Del^m\to\Del^n\fatjoin\Del^m$, $u^{n,m}\colon\Del^n\fatjoin\Del^m\to\Del^n\fatjoin\Del^m$, and $h^{n,m}, k^{n,m}\colon(\Del^n\fatjoin\Del^m)\times\Del^1\to\Del^n\fatjoin\Del^m$ which also satisfy the algebraic identities discussed in the last paragraph. Now if we are to use this data to show that $s^{n,m}$ is a deformation retraction in Joyal's model structure, then we must show that the maps $h^{n,m}$ and $k^{n,m}$ give rise to homotopies between the map $u^{n,m}$ and the maps $t^{n,m}\circ s^{n,m}$ and $\id_{\Del^n\fatjoin\Del^m}$ respectively. To that end, lemma~\ref{lem:cyl-obj} tells us that it would be enough to show that the maps $h^{n,m}$ and $k^{n,m}$ extend along the inclusion $(\Del^n\fatjoin\Del^m)\times\Del^1\inc\cyl(\Del^n\fatjoin\Del^m)$. On consulting definition~\ref{defn:Joyal-cylinder}, we find that we must verify that for each $0$-simplex $[{i},{j},{k}]_\sim$ of $\Del^n\fatjoin\Del^m$ the $1$-simplex $([{i},{j},{k}]_\sim\cdot\degen^0,\id_{[1]})$ of $(\Del^n\fatjoin\Del^n)\times\Del^1$ is mapped by $h^{n,m}$ and $k^{n,m}$ to marked, and thus degenerate, simplices in $\Del^n\fatjoin\Del^m$. This, however, is a matter of routine verification, which we leave to the reader.
  \end{proof}
  
  
  
  We will extend this result to all simplicial sets presently, but first we shall need the following technical result. A review of the Reedy category theory necessary to understand its statement and proof, written in part for this purpose, can be found in \cite{RiehlVerity:2013kx}.
  
  \begin{lem}\label{lem:cofib-joins}
    If $i\colon X\inc Y$ and $j\colon U\inc V$ are both monomorphisms of terminally augmented simplicial sets, then so is their Leibniz join $(i\colon X\inc Y)\leib\join(j\colon U\inc V)$ and their Leibniz fat join $(i\colon X\inc Y)\leib\fatjoin(j\colon U\inc V)$. In particular, it follows that the latching maps of the associated functors $F_{\join},F_{\fatjoin}\colon \Del+\times\Del+\to\asSet$ given by $F_{\join}^{n,m} \defeq \Del^n\join\Del^m$ and $F_{\fatjoin}^{n,m} \defeq \Del^n\fatjoin\Del^m$ are  monomorphisms.
  \end{lem}
  
  \begin{proof}
    The explicit descriptions of the join and fat join given in recollection~\ref{rec:join-dec} and observation~\ref{def:fat-join} provide us with natural isomorphisms
    \begin{align*}
      (X\join U)_n &\cong X_n\sqcup \left(\coprod_{i=0,...,n-1} X_{n-i-1}\times U_i\right) \sqcup U_n \\
      (X\fatjoin U)_n &\cong X_n\sqcup (X_n\times D_n \times U_n) \sqcup U_n
    \end{align*}
    when $n\geq 0$ where $D_n$ denotes the set of those $n$-simplices of $\Del^1$ which are neither of the constant operators $0,1\colon[n]\to[1]$. Using these expressions, it is easy to verify that each map in the commutative squares
    \begin{equation*}
      \xymatrix@R=2em@C=3em{
        {(X\join U)_n}\ar@{u(->}[r]^{(X\join j)_n}\ar@{u(->}[d]_{(i\join U)_n} &
        {(X\join V)_n}\ar@{u(->}[d]^{(i\join V)_n} \\
        {(Y\join U)_n}\ar@{u(->}[r]_{(Y\join j)_n} & {(Y\join V)_n}
      }
      \mkern30mu
      \xymatrix@R=2em@C=3em{
        {(X\fatjoin U)_n}\ar@{u(->}[r]^{(X\fatjoin j)_n}\ar@{u(->}[d]_{(i\fatjoin U)_n} &
        {(X\fatjoin V)_n}\ar@{u(->}[d]^{(i\fatjoin V)_n} \\
        {(Y\fatjoin U)_n}\ar@{u(->}[r]_{(Y\fatjoin j)_n} & {(Y\fatjoin V)_n}
      }
    \end{equation*}
    is a monomorphism and that both squares are pullbacks in $\Set$. By the pasting property of such squares in $\Set$, the pushouts of the upper horizontal and left-hand vertical maps may be constructed as the joint images of their lower horizontal maps and their right hand vertical maps within the sets in their lower right hand corners. However, since the pushouts of $\sSet$ are constructed pointwise in $\Set$, it follows that the Leibniz join $i \leib\join j$ and Leibniz fat join $i \leib\fatjoin j$, which are induced out of these particular pushouts by these squares, may be written as inclusions of simplicial subsets into the simplicial sets $Y\join V$ and $Y\fatjoin V$, and hence are monomorphisms.
    
        Now on consulting observation~\ref*{reedy:obs:weights-latching} and example~\ref*{reedy:ex:boundary-prod} in \cite{RiehlVerity:2013kx}, we see that the latching maps of the functors $F_{\join}$ and $F_{\fatjoin}$ at $([n],[m])\in\Del+\times\Del+$ may be expressed in terms of the weighted colimit formulae:
    \begin{equation}\label{eq:latching-exprs-1}
      \begin{aligned}
        L^{n,m} F_{\join} &\cong ((\boundary\Del^n\inc\Del^n)\leib\etimes(\boundary\Del^m\inc\Del^m)) \wcolim_{\Del+\times\Del+} F_{\join} \mkern10mu\text{and}\\
        L^{n,m} F_{\fatjoin} &\cong ((\boundary\Del^n\inc\Del^n)\leib\etimes(\boundary\Del^m\inc\Del^m)) \wcolim_{\Del+\times\Del+} F_{\fatjoin} 
      \end{aligned}
    \end{equation}
    Here we are using $\etimes$ to denote the exterior product; cf.~\cite[\ref*{reedy:obs:box-product}]{RiehlVerity:2013kx}.
   Now, a routine application of Yoneda's lemma, in the form given in \cite[example~\ref{reedy:ex:weighted-yoneda}]{RiehlVerity:2013kx}, and a calculation using the fact that the join and fat join operations are cocontinuous in each variable (as bifunctors on the category of augmented simplicial sets) reveals that we have canonical isomorphisms: 
    \begin{align}
      X\join U &\cong \int^{[n],[m]\in\Del+} (X_n\times U_m)\tns(\Del^n\join\Del^m) \cong (X\etimes U)\wcolim_{\Del+\times\Del+} F_{\join} \mkern10mu\text{and} \notag  \\
      X\fatjoin U &\cong \int^{[n],[m]\in\Del+} (X_n\times U_m)\tns(\Del^n\fatjoin\Del^m) \cong (X\etimes U)\wcolim_{\Del+\times\Del+} F_{\fatjoin}  \label{eq:fatjoinformulae}
    \end{align}
    These pass to isomorphisms of the corresponding Leibniz operations and we may then apply them to show that the expressions \eqref{eq:latching-exprs-1} reduce to:
    \begin{equation*}
      L^{n,m} F_{\join} \cong (\boundary\Del^n\inc\Del^n)\leib\join(\boundary\Del^m\inc\Del^m) \mkern10mu\text{and}\mkern10mu
      L^{n,m} F_{\fatjoin} \cong (\boundary\Del^n\inc\Del^n)\leib\fatjoin(\boundary\Del^m\inc\Del^m)
    \end{equation*}
    Now apply the result established in the first part of the lemma to conclude that these latching maps are monomorphisms as stated.
  \end{proof}

  \begin{prop}\label{prop:join-fatjoin-equiv}
    For all simplicial sets $X$ and $Y$, the  map $s^{X,Y}\colon X\fatjoin Y\to X\join Y$ is a weak equivalence in the Joyal model structure.
  \end{prop}

  \begin{proof}
    From \eqref{eq:fatjoinformulae} we know that $X\join Y$ and $X\fatjoin Y$ are naturally isomorphic to the colimits of the functors $F_{\join}, F_{\fatjoin}\colon\Del+\times\Del+\to\sSet$ weighted by $X\etimes Y$. Furthermore, the natural transformation $s^{X,Y}\colon X\fatjoin Y\to X\join Y$ restricts to give a natural transformation $s\colon F_{\fatjoin}\to F_{\join}$ and the naturality of $s^{X,Y}$ ensures that the canonical isomorphisms fit into a commutative square: 
    \begin{equation*}
      \xymatrix@=2em{
        {X\fatjoin Y} \ar@{}[r]|-{\textstyle\cong}\ar[d]_{s^{X,Y}} &
        {(X\etimes Y)\wcolim_{\Del+\times\Del+} F_{\fatjoin}}
        \ar[d]^{(X\etimes Y)\wcolim_{\Del+\times\Del+} s} \\
        {X\join Y} \ar@{}[r]|-{\textstyle\cong} &
        {(X\etimes Y)\wcolim_{\Del+\times\Del+} F_{\join}}
      }
    \end{equation*}
    When $\sSet$ carries the Joyal model structure, lemma~\ref{lem:cofib-joins} asserts that $F_{\join}$ and $F_{\fatjoin}$ are cofibrant in the corresponding Reedy model structure on $\sSet^{\Del+\times\Del+}$, and proposition~\ref{prop:join-fatjoin-equiv-simplices} tells us that $s\colon F_{\fatjoin}\to F_{\join}$ is a pointwise weak equivalence. 
    
    Now the Eilenberg-Zilber lemma (cf.~\cite[II.3.1, pp.~26-27]{GabrielZisman:1967:CFHT}) implies that the latching maps of any (augmented) double simplicial set are monomorphisms. In particular, $X \etimes Y$ is Reedy cofibrant, and we may apply \cite[proposition~\ref*{reedy:prop:2/3-SM7}]{RiehlVerity:2013kx} to show that the functor $(X\etimes Y)\wcolim_{\Del+\times\Del+}{-}$ is a left Quillen functor. Consequently, Ken Brown's lemma (see \cite[1.1.12]{Hovey:1999fk} for example) now applies to show that $(X\etimes Y)\wcolim_{\Del+\times\Del+}{-}$ carries the pointwise weak equivalence $s\colon F_{\fatjoin}\to F_{\join}$ of Reedy cofibrant objects to a weak equivalence in $\sSet$. However, the commutative square above tells us that this latter map is isomorphic to $s^{X,Y}\colon X\fatjoin Y\to X\join Y$ which is thus also a weak equivalence as postulated.
  \end{proof}
  
  \begin{lem}\label{lem:slices-quillen}
 For any simplicial set $X$, the slice and fat slice adjunctions
    \begin{align*}
  		\adjdisplay \textstyle X\bar\join{-}-|\textstyle \slc^X_l:X\slice\sSet->\sSet. & 
      \mkern20mu & 
  		\adjdisplay \textstyle {-}\bar\join X-|\textstyle \slc^X_r:X\slice\sSet->\sSet. \\
  		\adjdisplay \textstyle X\bar\fatjoin{-}-|\textstyle \fatslc^X_l:X\slice\sSet->\sSet. & 
      \mkern20mu & 
  		\adjdisplay \textstyle {-}\bar\fatjoin X-|\textstyle \fatslc^X_r:X\slice\sSet->\sSet.
    \end{align*}
    of definitions~\ref{defn:slices} and~\ref{defn:fat-slices} are Quillen adjunctions with respect to the Joyal model structure on $\sSet$ and the corresponding sliced model structure on $X\slice\sSet$.
  \end{lem}
  
  \begin{proof}
    By~\cite[7.15]{Joyal:2007kk} it is enough to check that in each of these adjunctions the left adjoint preserves cofibrations and the right adjoint preserves fibrations between fibrant objects. Preservation of cofibrations by these left adjoints follows immediately from lemma~\ref{lem:cofib-joins}, since in the Joyal model structure they are simply the monomorphisms of simplicial sets. Preservation of fibrations of fibrant objects by these right adjoints follows immediately from observations~\ref{obs:slice-and-qcats} and~\ref{obs:fat-cone-quasicat}: if $p \colon A \to B$ is an isofibration of quasi-categories and $f \colon X \to A$ is any simplicial map, then the induced simplicial maps $\slc^X_r(p)\colon \slicer{A}{f}\to \slicer{B}{pf}$ and $\slc^X_l(p)\colon \slicel{f}{A}\to \slicel{pf}{B}$ are also isofibrations of quasi-categories and similarly for fat slices.
  \end{proof}
  
  Finally, we arrive at the advertised comparison result relating the slice and fat slice constructions.
  
  \begin{prop}[slices and fat slices of a quasi-category are equivalent]\label{prop:slice-fatslice-equiv}
    Suppose that $X$ is any simplicial set, that $\sSet$ carries the Joyal model structure, and that $X\slice\sSet$ carries the associated sliced model structure. Then the comparison maps $s^{X,Y}\colon X\fatjoin Y\to X\join Y$ furnish us with natural transformations $s^{X,{-}}\colon X\bar\fatjoin{-}\to X\bar\join{-}$ and $s^{{-},X}\colon {-}\bar\fatjoin X\to {-}\bar\join X$ which are pointwise weak equivalences. The components $e_l^f\colon \slicel{f}{A}\to \fatslicel{f}{A} \cong f\comma c$ and $e_r^f\colon \slicer{A}{f}\to \fatslicer{A}{f}\cong c\comma f$ at an object $f\colon X\to A$ of $X\slice\sSet$ of the induced natural transformations on right adjoints  are equivalences of quasi-categories whenever $A$ is a quasi-category.
  \end{prop}
  
  \begin{proof}
    The assertions involving left adjoints are the content of Proposition~\ref{prop:join-fatjoin-equiv}. The result described in the last sentence of the statement then follows from from the fact that, by lemma~\ref{lem:slices-quillen}, all of these adjunctions are Quillen adjunctions. Specifically, that fact allows us to apply the standard result in model category theory~\cite[1.4.4]{Hovey:1999fk} that a natural transformation between left Quillen functors has components which are weak equivalences at each cofibrant object (which fact we have already established) if and only if the  induced natural transformation between the corresponding right Quillen functors has components which are weak equivalences at each fibrant object. Now simply observe that an object $f\colon X\to A$ is fibrant in $X\slice\sSet$ if and only if $A$ is a quasi-category.
  \end{proof}
  
  \begin{rmk}\label{rmk:map-slices}
    Suppose that $f\colon B\to A$ and $g\colon C\to A$ are two simplicial maps. We generalise our slice and fat slice notation by using $\slicer{g}{f}$, $\fatslicer{g}{f}$, $\slicel{f}{g}$ and $\fatslicel{f}{g}$ to denote the objects constructed in the following pullback diagrams
    \begin{equation}
      \xymatrix@=2em{
        {\slicer{g}{f}} \pbexcursion\ar[r]\ar[d] & {\slicer{A}{f}}\ar[d]^\pi \\
        {C}\ar[r]_g & {A}
      }
      \mkern50mu
      \xymatrix@=2em{
        {\fatslicer{g}{f}} \pbexcursion\ar[r]\ar[d] & {\fatslicer{A}{f}}\ar[d]^\pi \\
        {C}\ar[r]_g & {A}
      }
      \mkern50mu
      \xymatrix@=2em{
        {\slicel{f}{g}} \pbexcursion\ar[r]\ar[d] & {\slicel{f}{A}}\ar[d]^\pi \\
        {C}\ar[r]_g & {A}
      }
      \mkern50mu
      \xymatrix@=2em{
        {\fatslicel{f}{g}} \pbexcursion\ar[r]\ar[d] & {\fatslicel{f}{A}}\ar[d]^\pi \\
        {C}\ar[r]_g & {A}
      }
    \end{equation}
    in which the maps labelled $\pi$ denote the various canonical projection maps. We call these the slices and fat slices of $g$ over and under $f$ respectively. 
     We have isomorphisms $\slicer{g\op}{f\op} \cong (\slicel{f}{g})\op$ and $\fatslicer{g\op}{f\op} \cong (\fatslicel{f}{g})\op$. Similarly $g\op\downarrow f\op \cong (f\downarrow g)\op$.    The canonical isomorphisms of proposition \ref{prop:fatsliceisfatcone} may  be pulled back to provide us with canonical isomorphisms $\fatslicer{g}{f}\cong cg\comma f$ and $\fatslicel{f}{g}\cong f\comma cg$. 
    
    When $A$ is a quasi-cat\-e\-go\-ry the projection maps $\pi$ are all isofibrations that commute with the comparison equivalences $e_l^f\colon \slicel{f}{A}\to \fatslicel{f}{A}$ and $e_r^f\colon \slicer{A}{f}\to \fatslicer{A}{f}$ of proposition \ref{prop:join-fatjoin-equiv}. In other words, these comparisons are fibred equivalences (cf.~definition~\ref{defn:fibred-equivalence}), which pull back to define equivalences $e_l^f\colon \slicel{f}{g}\to \fatslicel{f}{g}$ and $e_r^f\colon \slicer{g}{f}\to \fatslicer{g}{f}$ between slices of the map $g$ under and over $f$. (Alternatively, these maps are equivalences  of fibrant objects in the sliced Joyal model structure on $\sSet\slice A$. Pullback along any map in a model category is always a right Quillen functor of sliced model structures, so Ken Brown's lemma tells us that the pullbacks are again equivalences.)
  \end{rmk}


\input{../common/footer}



\bibliographystyle{abbrv}
\bibliography{index}

% \listoffixmes

\end{document}

