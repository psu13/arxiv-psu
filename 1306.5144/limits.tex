%!TEX root = all.tex
% ******************************************************************
% ** Title:           The 2-category theory of quasi-categories
% **                   limits and colimits
% ** Precis:        
% ** Author:           Emily Riehl and Dominic Verity
% ** Commenced:        2/3/2012
% ******************************************************************


\section{Limits and colimits}\label{sec:limits}

In this section, we demonstrate that limits and colimits of individual diagrams in a quasi-category can be encoded as {\em absolute right and left liftings\/} in the 2-category $\qCat_2$. The proof that this definition is equivalent to the standard one makes use of the fact that absolute lifting diagrams in $\qCat_2$ can be detected by an equivalence of suitably defined comma quasi-categories. This observation, combined with Example~\ref{ex:adjasabslifting}, also supplies the proof of Proposition~\ref{prop:adjointequivconverse}, completing the unfinished business from the previous section. 

We begin with a general definition:

\setcounter{thm}{0}
\begin{defn}\label{defn:abs-right-lift} In a 2-category, an \emph{absolute right lifting diagram}   consists of the data \begin{equation}\label{eq:absRlifting}\xymatrix{ \ar@{}[dr]|(.7){\Downarrow\lambda} & B \ar[d]^f \\ C \ar[r]_g \ar[ur]^\ell & A}\end{equation} with the universal property that if we are given any 2-cell $\chi$ of the form depicted to the left of the following equality
\begin{equation}\label{eq:abs-lifting-property}
    \vcenter{\xymatrix{ X \ar[d]_c \ar[r]^b \ar@{}[dr]|{\Downarrow\chi} & B \ar[d]^f \\ C \ar[r]_g & A}} \mkern20mu = \mkern20mu \vcenter{\xymatrix{ X \ar[d]_c \ar[r]^b \ar@{}[dr]|(.3){\exists !\Downarrow}|(.7){\Downarrow\lambda} & B \ar[d]^f \\ C \ar[ur]|(.4)*+<2pt>{\scriptstyle\ell} \ar[r]_g & A}}
 \end{equation} 
 then it admits a unique factorisation of the form displayed to the right of that equality. When this condition holds for the diagram in~\eqref{eq:absRlifting} we say that it {\em displays $\ell$ as an absolute right lifting of $g$ through $f$}.
\end{defn}

\begin{ex}\label{ex:adjasabslifting} The counit of an adjunction $\adjinline f-|u:A->B.$ defines an absolute right lifting diagram 
  \begin{equation}\label{eq:adjasabslifting}
    \xymatrix{ \ar@{}[dr]|(.7){\Downarrow\epsilon} & B \ar[d]^f \\ A \ar[ur]^u \ar[r]_{\id_A} & A}  
  \end{equation}
  and, conversely, if this diagram displays $u$ as an absolute right lifting of the identity on its domain through $f$ then $f$ is left adjoint to $u$ with counit 2-cell $\epsilon$. 
\end{ex}

\begin{proof} 
This is a standard 2-categorical result. The 2-functor represented by $X$ carries an adjunction $f \dashv u$ to an adjunction whose counit has the universal property described in~\eqref{eq:abs-lifting-property} for the 2-cell \eqref{eq:adjasabslifting}.

Conversely, given an  absolute right lifting diagram~\eqref{eq:adjasabslifting}, we take this 2-cell to be the counit and define the unit by applying the universal property of this absolute right lifting to the identity 2-cell:
\begin{equation}\label{eq:unitdefn} 
  \vcenter{\xymatrix{ B \ar[r]^{\id_B} \ar[d]_f \ar@{}[dr]|{\Downarrow \id_f} & B \ar[d]^f \\ A \ar[r]_{\id_{A}} & A}} = \vcenter{\xymatrix{ B \ar[r]^{\id_B} \ar[d]_f \ar@{}[dr]|(.3){\Downarrow \eta}|(.7){\Downarrow\epsilon} & B \ar[d]^f \\ A \ar[r]_{\id_{A}} \ar[ur]|*+{\scriptstyle u} & A}} 
  \end{equation}
  This defining equation establishes one of the triangle identities. The other is obtained by pasting $\epsilon$ on the left of both of the 2-cells of \eqref{eq:unitdefn} and applying the uniqueness statement in the universal property of the absolute right lifting:
 \[  \vcenter{\xymatrix{ \ar@{}[dr]|(.7){\Downarrow\epsilon} & B \ar[r]^{\id_B} \ar[d]_f \ar@{}[dr]|{\Downarrow \id_f} & B \ar[d]^f \\ A \ar[r]_{\id_{A}} \ar[ur]^{u} & A \ar[r]_{\id_{A}} & A}} = \vcenter{\xymatrix{  \ar@{}[dr]|(.7){\Downarrow\epsilon}& B \ar[r]^{\id_B} \ar[d]_f \ar@{}[dr]|(.3){\Downarrow \eta}|(.7){\Downarrow\epsilon} & B \ar[d]^f \\ A \ar[r]_{\id_{A}} \ar[ur]^{u} &  A \ar[r]_{\id_{A}} \ar[ur]|*+{\scriptstyle u} & A}}\rightsquigarrow \vcenter{\xymatrix{ & B  \ar@{}[dl]|{\Downarrow\id_{u}}  \\ A \ar@/^2.25ex/[ur]^{u} \ar@/_2.25ex/[ur]_{u} &  }} = \vcenter{\xymatrix{  \ar@{}[dr]|(.7){\Downarrow\epsilon}& B \ar[r]^{\id_B} \ar[d]_f \ar@{}[dr]|(.3){\Downarrow \eta}& B\\ A \ar[r]_{\id_{A}} \ar[ur]^{u} &  A  \ar[ur]|*+{\scriptstyle u} & }}\qedhere \]
\end{proof}

\subsection{Absolute liftings and comma objects}

We now specialise to the 2-category $\qCat_2$. Our aim is to use its weak comma objects to re-express the universal property of absolute lifting diagrams and describe various procedures through which they may be detected.

Given any diagram in $\qCat_2$ of the form displayed in~\eqref{eq:absRlifting} in $\qCat_2$ we may form comma objects $B \comma \ell$ and $f \comma g$ with canonical comma cones:
\begin{equation}\label{eq:comma-cones}
    \vcenter{ \xymatrix{ B \comma \ell \ar[d]_{p_1} \ar[r]^-{p_0} \ar@{}[dr]|(.3){\Downarrow\phi} & B  & & f \comma g \ar[d]_{q_1} \ar[r]^-{q_0}  \ar@{}[dr]|{\Downarrow\psi} & B \ar[d]^f \\ C \ar[ur]_\ell & & &C \ar[r]_g & A }}
\end{equation}
Pasting the canonical cone associated with $B \comma \ell$ onto the triangle~\eqref{eq:absRlifting} we obtain a comma cone which induces a functor $w\colon B \comma \ell \to f \comma g$ by the 1-cell induction property of $f\comma g$. Recall this means that $w$ makes the following pasting equality hold
\begin{equation}\label{eq:w-def-prop}
  \vcenter{\xymatrix@=1.2em{
    & {B\comma\ell}\ar[dl]_{p_1}\ar[dr]^{p_0} & \\
    {C} \ar[dr]_{g}\ar[rr]|*+{\scriptstyle\ell} && {B}\ar[dl]^{f} \\
    & A &  
    \ar@{} "1,2";"3,2" |(0.3){\Leftarrow\phi} |(0.7){\Leftarrow\lambda}
  }}
  \mkern 20mu = \mkern20mu
  \vcenter{\xymatrix@=1.2em{
    & {B\comma\ell}\ar[d]^{w}\ar@/_1.5ex/[ddl]_{p_1}\ar@/^1.5ex/[ddr]^{p_0} & \\
    & {f\comma g}\ar[dl]^{q_1}\ar[dr]_{q_0} & \\
    {C}\ar[dr]_{g} & & {B}\ar[dl]^{f} \\
    & {A} & 
    \ar@{} "2,2";"4,2" |{\Leftarrow\psi}
  }}
\end{equation}
and in particular may be regarded as being a 1-cell in the slice 2-category $\qCat_2\slice(C\times B)$ from $(p_1,p_0)\colon f\comma g \tfib C\times B$ to $(q_1,q_0)\colon B\comma\ell\tfib C\times B$.



\begin{prop}\label{prop:absliftingtranslation} The data of \eqref{eq:absRlifting} defines an absolute right lifting in $\qCat_2$ if and only if the induced map $w\colon B \comma \ell \to f \comma g$ of~\eqref{eq:w-def-prop} is an equivalence.
\end{prop} 
\begin{proof}
    For each pair of functors $b\colon X\to B$ and $c\colon X\to C$ as in~\eqref{eq:abs-lifting-property} observe that $\sq_{g,f}(c,b)$ (cf.\ Observation~\ref{obs:squares-set}) is simply the set of those 2-cells of the form depicted in the square on the left of the equality in~\eqref{eq:abs-lifting-property} and that $\sq_{\ell,B}(c,b)$ is the set of those 2-cells which inhabit the upper left triangle of the diagram to the right of that same equality. Define
  \begin{equation*}
    \xymatrix@C=8em{
      {\sq_{\ell,B}(c,b)}\ar[r]^{k^{\lambda}_{(c,b)}} &
      {\sq_{g,f}(c,b)}
    }
  \end{equation*}
  to be the function which takes each triangle in its domain and pastes it onto our candidate lifting diagram~\eqref{eq:absRlifting} to obtain a corresponding square as depicted in~\eqref{eq:abs-lifting-property}. This family of functions is natural in $(c,b)\colon X\to C\times B$ in the sense that they are the components of a natural transformation $k^\lambda$ between the functors
\[ \xymatrix{ (\pi^g_0)_*(\qCat_2\slice(C\times B))\op \ar@<1.5ex>[r]^-{\sq_{\ell,B}} \ar@<-1.5ex>[r]_-{\sq_{g,f}} \ar@{}[r]|-{\Downarrow k^\gamma} & \Set}\] 
of Lemma~\ref{lem:sq-as-a-functor}. By construction,  the triangle in~\eqref{eq:absRlifting} is an absolute right lifting if and only if $k^\lambda\colon \sq_{\ell,B}\Rightarrow \sq_{g,f}$ is a natural isomorphism.

Now   consider a commutative square of natural transformations
  \begin{equation*}
    \xymatrix@C=5em{
      {\pi^g_0(\hom'_{C\times B}(-,(p_1,p_0)))} 
      \ar[r]^{u\circ -}\ar[d]_{\cong} &
      {\pi^g_0(\hom'_{C\times B}(-,(q_1,q_0)))}
      \ar[d]^{\cong} \\
      {\sq_{\ell,B}}\ar[r]_{k} &
      {\sq_{g,f}}
    }
  \end{equation*}
  between presheaves on $(\pi^g_0)_*(\qCat_2\slice(C\times B))$, in which the vertical isomorphisms are those induced by the weakly universal comma cones of~\eqref{eq:comma-cones} as discussed in Lemma~\ref{lem:cpts-and-comma-2-cells}. Applying Yoneda's lemma and the definition of $(\pi^g_0)_*(\qCat_2\slice(C\times B))$, this square provides us with a canonical bijection between the set of natural transformations $k\colon\sq_{\ell,B}\Rightarrow\sq_{g,f}$ and the set of isomorphism classes of 1-cells
  \begin{equation}\label{eq:induced-u-from-nattrans-k}
    \xymatrix@=1em{
      {B\comma\ell}\ar@{->>}[dr]_(0.3){(p_1,p_0)}\ar[rr]^{u}
      && *+!L(0.5){f\comma g}\ar@{->>}[dl]^(0.3){(q_1,q_0)} \\
      & {C\times B}&
    }
  \end{equation}
  in $\qCat_2\slice(C\times B)$.  By the Yoneda lemma, $k\colon\sq_{\ell,B}\Rightarrow\sq_{g,f}$ is a natural isomorphism if and only if the corresponding $u\colon B\comma\ell\to f\comma g$ is an isomorphism in $(\pi^g_0)_*(\qCat_2\slice(C\times B))$. By Observation~\ref{obs:groupoid-components}, this holds if and only if $u$ is an equivalence in $\qCat_2\slice(C\times B)$. By Lemma~\ref{lem:proj-is-1-conservative},  this is the case if and only if $u$ is an equivalence in $\qCat_2$. 

In particular, the natural transformation $k^\lambda\colon\sq_{\ell,B}\Rightarrow\sq_{g,f}$ constructed from the 2-cell~\eqref{eq:absRlifting} corresponds  to the isomorphism class of those induced 1-cells $w\colon B\comma\ell\to f\comma g$ over $C\times B$ which satisfy the pasting identity displayed in~\eqref{eq:w-def-prop}. We have just shown that  the triangle in~\eqref{eq:absRlifting} is an absolute lifting diagram if and only if $k^\lambda\colon\sq_{\ell,B}\Rightarrow\sq_{g,f}$ is a natural isomorphism, which is the case  if and only if $w\colon B \comma \ell \to f \comma g$ is an equivalence. 
\end{proof}








\begin{rmk}
There is nothing in the proof of the Proposition \ref{prop:absliftingtranslation}, or in those of the results upon which it relies, which depends upon the vertex $X$ in~\eqref{eq:abs-lifting-property} being a quasi-category. The essential point here is that the space of maps out of any simplicial set $X$ taking values in a quasi-category is still a quasi-category. Consequently, we find that absolute lifting diagrams in $\qCat_2$ possess the factorisation property displayed in~\eqref{eq:abs-lifting-property} for 2-cells whose 0-cellular domains $X$ are general simplicial sets.  
\end{rmk}

For certain applications, it will be important to have a strengthened version of Proposition~\ref{prop:absliftingtranslation} which says that from {\em any\/} equivalence $B\comma \ell \simeq f \comma g$ fibred over $C\times B$ we may construct a 2-cell which displays $\ell$ as an absolute right lifting of $g$ through $f$. This result, Proposition~\ref{prop:absliftingtranslation2} below, proceeds directly from the following technical lemma:

\begin{lem}\label{lem:represented-nat-trans}
  For all natural transformations $k\colon\sq_{\ell,B}\Rightarrow\sq_{g,f}$ there exists a unique 2-cell $\lambda$ of the form depicted in~\eqref{eq:absRlifting} such that $k$ is equal to the natural transformation $k^\lambda$ defined by pasting a 2-cell in a triangle over $\ell$ with $\lambda$ to form a 2-cell in a square over $f$ and $g$.
\end{lem}

\begin{proof}
  A 2-cell in the triangle~\eqref{eq:absRlifting} is simply an element of $\sq_{g,f}(C,\ell)$, so we may construct our candidate 2-cell $\lambda$ from the natural transformation $k\colon\sq_{\ell,B}\Rightarrow\sq_{g,f}$ by applying it to the identity 2-cell  in $\sq_{\ell,B}(C,\ell)$; that is, we take $\lambda\defeq k_{(C,\ell)}(\id_\ell)$. 

  Lemma~\ref{lem:cpts-and-comma-2-cells} reveals that $\sq_{\ell,B}$ is a representable functor whose universal element is the 2-cell $\phi\in\sq_{\ell,B}(p_1,p_0)$ of the weakly universal cone~\eqref{eq:comma-cones} displaying $B\comma\ell$. So Yoneda's lemma tells us that in order to show that our original natural transformation $k$ is equal to $k^\lambda$ it is enough to check that they both map $\phi$ to the same element of $\sq_{g,f}(p_1,p_0)$.

To do this, first observe that the functor $i \colon C \to B\comma\ell$ defined in Lemma~\ref{lem:technicalsliceadjunction} can be regarded as a morphism in $(\pi^g_0)_*(\qCat_2\slice(C\times B))$. Its defining property, that $\phi i = \id_\ell$, may then be re-expressed as the equality $\sq_{\ell,B}(i)(\phi) = \id_\ell$ relating $\id_\ell\in\sq_{\ell,B}(C,\ell)$ and $\phi\in\sq_{\ell,B}(p_1,p_0)$. By naturality of $k$, this then allows us to obtain a similar relationship between the 2-cell $\lambda$  and the image $\mu\defeq k_{(p_1,p_0)}(\phi)$ of $\phi$ under $k$, as given by the following computation: $\sq_{g,f}(i)(k_{(p_1,p_0)}(\phi)) = k_{(C,\ell)}(\sq_{\ell,B}(i)(\phi)) = k_{(C,\ell)}(\id_\ell) = \lambda$. By the definition of the map $\sq_{g,f}(i)$, this relationship may be expressed as a pasting equality:
  \begin{equation}\label{eq:rel-mu-lambda}
    \vcenter{\xymatrix@=0.8em{
      & {C} \ar@{=}[dl]\ar[dr]^\ell & \\
      {C}\ar[dr]_g & {\scriptstyle\Leftarrow\lambda} &
      {B}\ar[dl]^f \\
      & {A} &
    }}
    \mkern20mu = \mkern20mu
    \vcenter{\xymatrix@=0.8em{
      & \save []+<0pt,1em>*+{C}\ar[d]^i\ar@{=}@/_1.5ex/[ddl]\ar@/^1.5ex/[ddr]^\ell \restore & \\
      & {B\comma\ell}\ar[dl]^{p_1}\ar[dr]_{p_0} & \\
      {C}\ar[dr]_g & {\scriptstyle\Leftarrow\mu} & 
      {B}\ar[dl]^f \\
      & {A} &
    }}
  \end{equation}

By definition, $k^\lambda$ acts on $\phi$ by pasting it to the 2-cell $\lambda$ as depicted in the diagram on the left hand side of the following computation:
  \begin{equation*}
    \vcenter{\xymatrix@=1em{
      & & {B\comma\ell}\ar[dl]_{p_1}
      \ar[dd]^{p_0}_{}="one"\\
      & {C} \ar@{=}[dl]\ar[dr]_(0.4)\ell 
      \ar@{} "one" |(0.6){\Leftarrow\phi} & \\
      {C}\ar[dr]_g & {\scriptstyle\Leftarrow\lambda} &
      {B}\ar[dl]^f \\
      & {A} &
    }}
    \mkern5mu = \mkern5mu
    \vcenter{\xymatrix@=0.8em{
      & & {B\comma\ell}\ar[dl]_{p_1}\ar[dddd]^{p_0} \\
      & {C}\ar[dd]^i\ar@/_1.5ex/@{=}[dddl]
      \ar@/^1.5ex/[dddr]^(0.3)\ell="one" 
      & \\ & & \\
      & {B\comma\ell}\ar[dl]^{p_1}\ar[dr]_{p_0} & \\
      {C}\ar[dr]_g & {\scriptstyle\Leftarrow\mu} & 
      {B}\ar[dl]^f \\
      & {A} &
      \ar@{}"1,3";"one"|(0.7){\Leftarrow\phi} 
    }}
    \mkern5mu = \mkern5mu
    \vcenter{\xymatrix@=0.8em{
      & & {B\comma\ell}\ar[dl]_{p_1}\ar[dddd]^{p_0} 
      \ar@{=}@/^1.5ex/[dddl]_{}="one"\\
      & {C}\ar[dd]^i\ar@/_1.5ex/@{=}[dddl]
      & \\ & & \\
      & {B\comma\ell}\ar[dl]^{p_1}\ar[dr]_{p_0} & \\
      {C}\ar[dr]_g & {\scriptstyle\Leftarrow\mu} & 
      {B}\ar[dl]^f \\
      & {A} &
      \ar@{}"2,2";"one"|(0.6){\Leftarrow\nu} 
    }}
    \mkern5mu = \mkern5mu
    \vcenter{\xymatrix@=0.8em{
      & {B\comma\ell}\ar[dl]_{p_1}\ar[dr]^{p_0} & \\
      {C}\ar[dr]_g & {\scriptstyle\Leftarrow\mu} & 
      {B}\ar[dl]^f \\
      & {A} &
    }}
  \end{equation*}
  To elaborate, the first step in this calculation is simply an application of the equality given in~\eqref{eq:rel-mu-lambda}. Its second step follows from the first of the defining properties of the unit $\nu\colon\id_{B\comma\ell}\Rightarrow i p_1$ of the adjunction $p_1\dashv i$ of Lemma~\ref{lem:technicalsliceadjunction}, those being that $p_0\nu=\phi$ and $p_1\nu=\id_{p_1}$. The third of these equalities follows on observing that the pasting depicted on its left is simply the horizontal composite of the 2-cells $\mu$ and $\nu$, which may be expressed as the vertical composite $qp_1\nu\cdot \mu$ in which the second factor is an identity by the second defining property of $\nu$. 
  
  In other words, this calculation demonstrates that $k^\lambda_{(p_1,p_0)}(\phi)=\mu$ which is in turn equal to $k_{(p_1,p_0)}(\phi)$, by definition. Consequently, Yoneda's lemma tells us that $k = k^\lambda$ as required. Finally, the fact that $\lambda$ is the unique 2-cell with the property that $k=k^\lambda$ follows immediately from the patent fact that $\lambda = k^\lambda_{(C,\ell)}(\id_\ell)$.
\end{proof}

As an immediate corollary, we have the following important result:

\begin{prop}\label{prop:absliftingtranslation2} Suppose we are given functors $f\colon B\to A$, $g\colon C\to A$, and $\ell\colon C\to B$ of quasi-categories. Then the construction depicted in~\eqref{eq:w-def-prop} provides us with a bijection between 2-cells of the form 
\begin{equation}\label{eq:absRlifting.2}
  \xymatrix{ \ar@{}[dr]|(.7){\Downarrow\lambda} & B \ar[d]^f \\ C \ar[r]_g \ar[ur]^\ell & A}
\end{equation}
and isomorphism classes of 1-cells
\begin{equation}\label{eq:induced-from-lambda}
  \xymatrix@=1em{
    {B\comma\ell}\ar@{->>}[dr]_(0.3){(p_1,p_0)}\ar[rr]^{w}
    && *+!L(0.5){f\comma g}\ar@{->>}[dl]^(0.3){(q_1,q_0)} \\
    & {C\times B}&
  }
\end{equation}
in $\qCat_2\slice(C\times B)$. Furthermore, this 2-cell $\lambda$ displays $\ell$ as an absolute right lifting of $g$ through $f$ if and only if any representative $w$ of the corresponding isomorphism class of functors is an equivalence.
\end{prop}

\begin{proof}
Lemma \ref{lem:represented-nat-trans} provides a canonical bijection between 2-cells \eqref{eq:absRlifting.2} and natural transformations $\sq_{\ell,B} \To \sq_{g,f}$. 
The proof of Proposition \ref{prop:absliftingtranslation} establishes a canonical bijection between natural transformations $\sq_{\ell,B} \To \sq_{g,f}$ and isomorphism classes of 1-cells \eqref{eq:induced-from-lambda}. Proposition \ref{prop:absliftingtranslation} then concludes that $\lambda$ displays $\ell$ as an absolute right lifting of $g$ through $f$ if and only if any representative $w$ of the corresponding isomorphism class of functors is an equivalence.
\end{proof}

As a special case,  if $f\comma A$ and $B \comma u$ are equivalent over $A \times B$, then $f$ is left adjoint to $u$.

\begin{proof}[Proof of Proposition~\ref{prop:adjointequivconverse}]
If $f\comma A$ and $B \comma u$ are equivalent over $A \times B$, then Proposition \ref{prop:absliftingtranslation2} provides us with a corresponding 2-cell $\epsilon\colon fu\Rightarrow\id_A$, which displays $u$ as an absolute right lifting of $\id_A$ through $f$. By Example~\ref{ex:adjasabslifting}, this provides us with enough information to conclude that $f$ is left adjoint to $u$ with counit $\epsilon$. 
\end{proof}

  A second characterisation of absolute right liftings in $\qCat_2$ relates them to the possession of terminal objects by the fibres of $q_1\colon f\comma g\tfib C$. To explain this relationship, start by applying Observation~\ref{obs:1cell-ind-uniqueness-reloaded} to show that arbitrary pairs $(\ell,\lambda)$ as depicted in~\eqref{eq:absRlifting.2} correspond to isomorphism classes of functors
  \begin{equation*}
    \xymatrix@=0.8em{
      {C}\ar[dr]_-{(C,\ell)}\ar[rr]^{t}
      && *+!L(0.5){f\comma g}\ar@{->>}[dl]^-{(q_1,q_0)} \\
      & {C\times B} &
    }
  \end{equation*}
  over $C\times B$ defined by 1-cell induction
  \begin{equation}\label{eq:t-for-lim-def-prop}
    \vcenter{\xymatrix@=0.8em{
      & \save []+<0pt,1em>*+{C}\ar[d]^-{t}\ar@{=}@/_1.5ex/[ddl]
      \ar@/^1.5ex/[ddr]^{\ell}\restore & \\
      & {f\comma g}\ar[dr]_{q_0}\ar[dl]^{q_1} & \\
      {C}\ar[dr]_{g} & {\scriptstyle\Leftarrow\psi} &
      {B}\ar[dl]^{f} \\
      & {A} &
    }}
    \mkern20mu = \mkern20mu
    \vcenter{\xymatrix@=0.7em{
      & {C}\ar@{=}[dl]\ar[dr]^{\ell} & \\
      {C}\ar[dr]_{g} & {\scriptstyle\Leftarrow\lambda} & {B}\ar[dl]^{f} \\
      & {A} &
    }}
  \end{equation}
The following proposition relates the universal properties of pairs $(\ell,\lambda)$ and corresponding functors $t$.

\begin{prop}\label{prop:right.liftings.as.fibred.terminal.objects} The 2-cell $\lambda$ shown in~\eqref{eq:absRlifting.2} displays $\ell$ as an absolute right lifting of the functor $g$ through $f$ if and only if the induced functor $t\colon C\to f\comma g$  of \eqref{eq:t-for-lim-def-prop} features in a fibred adjunction:
  \begin{equation}\label{eq:fibred.terminal.2}
    \xymatrix@=1.2em{
      {C}\ar@{=}[dr]\ar@/_1.5ex/[rr]_-{t}^-{}="one"
      & & *+!L(0.5){f\comma g}\ar@{->>}[dl]^-{q_1}
      \ar@/_1.5ex/[ll]_-{q_1}^-{}="two" \\
      & {C} &
      \ar@{}"one";"two"|{\bot}
    }
  \end{equation}
that is, if and only if $t$ defines a right adjoint right inverse to the isofibration $q_1$.
\end{prop}
\begin{proof}
First assume that the triangle in~\eqref{eq:absRlifting.2} is an absolute right lifting diagram and apply Proposition~\ref{prop:absliftingtranslation} to show that the associated functor $w\colon B\comma\ell\to f\comma g$ is a fibred equivalence with equivalence inverse $w'$. Applying Proposition~\ref{prop:equivtoadjoint} in $\qCat_2/(C\times B)$ and Corollary~\ref{cor:missed-lemma}, this may be promoted to a fibred adjoint equivalence $w'\dashv w$ over $C\times B$. Its pushforward along the projection $C\times B\tfib C$ is an adjoint equivalence fibred over $C$. 

Example~\ref{ex:fibred-technical-slice-adjunction} provides us with an adjunction $p_1\dashv i\colon C\to B\comma\ell$  also fibred over $C$. Composing these, we obtain an adjunction $p_1w'\dashv wi\colon C\to f\comma g$  again fibred over $C$. From the defining properties of $w$ and $i$, as described in~\eqref{eq:w-def-prop} and~\eqref{eq:technicalsliceadjunction}, it is clear that $wi$ is a 1-cell induced over $\psi$ by the comma cone $\lambda$, and so we may infer, by Observation~\ref{obs:1cell-ind-uniqueness-reloaded}, that it is isomorphic to $t$ over $C$. Furthermore, $w'$ is fibred over $C\times B$ so $p_1w' = q_1$, and the fibred adjunction $p_1w'\dashv wi$ reduces to a fibred adjunction $q_1\dashv t$ as required.

   For the converse, assume that we have a fibred adjunction of the form given in~\eqref{eq:fibred.terminal.2}. We must show that for any 2-cell $\mu$ 
  \begin{equation}\label{eq:limit-lifting-prop}
    \vcenter{\xymatrix@=1.5em{
      {Y}\ar[r]^{b}\ar[d]_-{c} &
      {B}\ar[d]^-{f} \\
      {C}\ar@{}[ur]|{\Downarrow\mu}\ar[r]_-{g} &
      {A}
    }}
    \mkern20mu = \mkern20mu
    \vcenter{\xymatrix@=1.5em{
      {Y}\ar[r]^{b}\ar[d]_-{c} &
      {B}\ar[d]^-{f} \\
      {C}\ar[ur]|*+<2pt>{\scriptstyle\ell}="one"\ar[r]_-{g} &
      {A}
      \ar@{} "1,1";"one"|(0.6){\Downarrow \exists!\tau}
      \ar@{} "one";"2,2"|(0.4){\Downarrow\lambda}
    }}
  \end{equation}
there exits a unique 2-cell $\tau$ which makes this pasting equation hold. 

  To do this, start by applying the 1-cell induction property of $f\comma g$ to the comma cone $\mu$ to give a functor $m\colon Y\to f\comma g$ so that
  \begin{equation}\label{eq:m-def-prop}
    \vcenter{\xymatrix@=0.7em{
      & {Y}\ar[dd]^{m}\ar@/_1.5ex/[dddl]_{c}
      \ar@/^1.5ex/[dddr]^{b} & \\
      &&\\
      & {f\comma g}\ar[dr]_{q_0}\ar[dl]^{q_1} & \\
      {C}\ar[dr]_{g} & {\scriptstyle\Leftarrow\psi} &
      {B}\ar[dl]^{f} \\
      & {A} &
    }}
    \mkern20mu = \mkern20mu
    \vcenter{\xymatrix@=0.7em{
      & {Y}\ar[dl]_{c}\ar[dr]^{b} & \\
      {C}\ar[dr]_{g} & {\scriptstyle\Leftarrow\mu} & {B}\ar[dl]^{f} \\
      & {A} &
    }}
  \end{equation}
A 2-cell $\tau\colon b\Rightarrow \ell c$ satisfying~\eqref{eq:limit-lifting-prop} gives rise to a 
  2-cell $\nu$ from $m\colon Y\to f\comma g$ to the composite functor $tc\colon Y\to f\comma g$ over $C$ by 2-cell induction: notice that the fact that we require $\nu$ to be a 2-cell over $C$ means that the equation $q_1\nu = \id_{p_1}$ must hold, which tells us that the second 2-cell of its inducing pair must  be $\id_{p_1}$.  The compatibility condition expressed in~\eqref{eq:comma-ind-2cell-compat} for the pair $(\tau,\id_{p_1})$ reduces to the pasting equality~\eqref{eq:limit-lifting-prop} by direct application of the defining properties for $m$ and $t$ given in~\eqref{eq:m-def-prop} and~\eqref{eq:t-for-lim-def-prop}. Conversely, if $\nu\colon m\Rightarrow tc$ is any 2-cell over $C$ then the whiskered 2-cell $\tau\defeq q_0\nu\colon b \Rightarrow \ell c$ satisfies~\eqref{eq:limit-lifting-prop}.

Extending Definition~\ref{defn:enriched-slice}, the map $c$ defines a 2-functor $\hom'_C(c,-)\colon\qCat_2\slice C \to \Cat_2$. As in Observation~\ref{obs:isofib-section.fibred.adjunction}, this 2-functor carries the postulated fibred adjunction $q_1\dashv t$ to a terminal object  $tc\colon Y\to f\comma g$ in the hom-category $\hom'_C(c,q_1)$. It follows that there exists a unique 2-cell $\nu\colon m\Rightarrow tc$ over $C$; hence, the 2-cell $q_0\nu\colon b \Rightarrow \ell c$ provides us with a solution to~\eqref{eq:limit-lifting-prop}. Furthermore if $\tau\colon b\Rightarrow \ell c$ is any other 2-cell which solves that pasting equality then the 2-cell it induces must necessarily be the unique such $\nu\colon m\Rightarrow tc$, and consequently we have the equality $\tau = q_0\nu$. This demonstrates that the solution to~\eqref{eq:limit-lifting-prop} is unique.
\end{proof}

\begin{obs}\label{obs:right.liftings.as.fibred.terminal.objects}
  The upshot of Proposition \ref{prop:right.liftings.as.fibred.terminal.objects} is that if the projection $q_1\colon f\comma g\tfib C$ has a fibred right adjoint~\eqref{eq:fibred.terminal.2}, then we may compose it with the weakly universal cone associated with $f\comma g$ to obtain an absolute right lifting of $g$ through $f$.
\end{obs}

This characterisation of absolute right liftings leads to the following generalisation of a classical result:

\begin{prop}\label{prop:translated.lifting}
  There exists an absolute right lifting
  \begin{equation}\label{eq:orig.lifting}
    \xymatrix{ \ar@{}[dr]|(.7){\Downarrow\lambda} & B \ar[d]^f \\ C \ar[r]_g \ar[ur]^\ell & A}
  \end{equation}
  if and only if there exists an absolute right lifting
  \begin{equation}\label{eq:translated.lifting}
    \xymatrix{ \ar@{}[dr]|(.7){\Downarrow\hat\lambda} & {f\comma A} \ar@{->>}[d]^{p_1} \\ C \ar[r]_g \ar[ur]^{\hat\ell} & A}
  \end{equation}
  Furthermore, the 2-cell $\hat\lambda$ is necessarily an isomorphism and $\hat\ell$ may be chosen so as to make it an identity.
\end{prop}

\begin{proof} Write $(r_1,r_0)\colon p_1\comma g \tfib C \times f\comma A$ for the projection defined by the comma quasi-category construction~\ref{def:comma-obj}. Directly from this definition, there exists a canonical isomorphism $p_1\comma g \cong A\comma g\times_A f\comma A$ commuting with the projections to $C \times f \comma A$. Applying Proposition~\ref{prop:right.liftings.as.fibred.terminal.objects}, our aim is to use a fibred right adjoint to $q_1$ to construct a fibred right adjoint to $r_1$ and vice versa. 
  \begin{equation}\label{eq:ran.as.fibred.adj}
 \xymatrix@=1.2em{
      {C}\ar@{=}[dr]\ar@/_1.5ex/[rr]_-{t}^-{}="one"
      & & *+!L(0.5){f\comma g}\ar@{->>}[dl]^-{q_1}
      \ar@/_1.5ex/[ll]_-{q_1}^-{}="two" \\
      & {C} &
      \ar@{}"one";"two"|{\bot}
    }    \qquad  \quad  
     \xymatrix@=1.2em{
      {C}\ar@{=}[dr]\ar@/_1.5ex/[rr]^-{}="one"
      & & *+!L(0.5){p_1\comma g}\ar@{->>}[dl]^-{r_1}
      \ar@/_1.5ex/[ll]_-{r_1}^-{}="two" \\
      & {C} &
      \ar@{}"one";"two"|{\bot}
    }
  \end{equation}

To that end, pull back  the ``composition--identity'' fibred adjunctions~\eqref{eq:comp.ident.adj} along the functor $g\times p_1\colon C\times f\comma A\to A\times A$ to obtain a pair of adjunctions 
  \begin{equation}\label{eq:adj.for.trans}
    \xymatrix@C=14em{
      {p_1\comma g\cong A\comma g\times_A f\comma A}
      \ar[]!R(0.72);[r]|*+{\scriptstyle m} &
      {f\comma g}
      \ar@/^2.5ex/[l]!R(0.72)^{i_1}_{}="l" \ar@/_2.5ex/[l]!R(0.72)_{i_0}^{}="u"
      \ar@{} "u";"l" |(0.2){\bot} |(0.8){\bot} 
    }
  \end{equation}
  fibred over $C\times f\comma A$. Pushing forward along the projection $C\times f\comma A\tfib C$, we may regard the adjunctions \eqref{eq:adj.for.trans} as fibred over $C$ with respect to the isofibrations $r_1\colon p_1\comma g\tfib C$ and $q_1\colon f\comma g\tfib C$. 

  With this adjunction in our armoury our result is essentially immediate. If we are given the left-hand fibred adjunction~\eqref{eq:ran.as.fibred.adj} witnessing the existence of the absolute right lifting of $g$ through $f$ then we may compose it with the lower fibred adjunction of~\eqref{eq:adj.for.trans} to obtain the right-hand fibred adjunction~\eqref{eq:ran.as.fibred.adj}, providing us with an absolute right lifting of $g$ through $p_1$. Conversely, we may go back in the other direction by composing the right-hand fibred adjunction with the upper fibred adjunction of~\eqref{eq:adj.for.trans} to obtain an adjunction of the type on the left of~\eqref{eq:ran.as.fibred.adj}.

  All that remains is to check the final clause of the proposition. To that end, Observation~\ref{obs:right.liftings.as.fibred.terminal.objects} tells us that we may construct an absolute right lifting of $g$ through $p_1$ by composing the right adjoint functor 
\begin{equation*}
  \xymatrix{
    {C}\ar[r]^-{t} & {f\comma g}\ar[r]^-{i_1} & {p_1\comma g}
  }
\end{equation*}
where $t$ is the fibred right adjoint of~\eqref{eq:ran.as.fibred.adj}, with the comma cone that displays $p_1\comma g$ as a weak comma object. By construction, the 2-cell of that cone is the restriction
\begin{equation*}
  \xymatrix{
    {p_1\comma g} \ar[r] & {A\comma g}\ar[r] & {A^\cattwo}\ar@{}[]!R(0.5);[rr]!L(0.5)|{\Downarrow}\ar@/^1.5ex/[]!R(0.5);[rr]!L(0.5)^{p_0}\ar@/_1.5ex/[]!R(0.5);[rr]!L(0.5)_{p_1} && {A}
  }
\end{equation*}
of the 2-cell which displays $A^\cattwo$ as a weak cotensor. Hence, the 2-cell $\hat\lambda$ constructed by Proposition~\ref{prop:right.liftings.as.fibred.terminal.objects}  is equal to 
\begin{equation*}
  \xymatrix{
    {C}\ar[r]^-{t} & {f\comma g}\ar[r] & {A^\cattwo}\ar[r]^-{i_1} & {A^\cattwo\times_A A^\cattwo}\ar[r]^-{\pi_1} & {A^\cattwo}\ar@{}[]!R(0.5);[rr]!L(0.5)|{\Downarrow}\ar@/^1.5ex/[]!R(0.5);[rr]!L(0.5)^{p_0}\ar@/_1.5ex/[]!R(0.5);[rr]!L(0.5)_{p_1} && {A}
    }
\end{equation*}
  and, consulting the definition of $i_1$ given in Example~\ref{ex:comp.ident.adj}, it is straightforward to verify that the composite of the last three cells above is equal to the identity 2-cell on $p_1\colon A^\cattwo\tfib A$. Consequently, the 2-cell in our absolute right lifting is also an identity as required.
\end{proof}

\subsection{Limits and colimits as absolute lifting diagrams}

A diagram in a quasi-category $A$ is just a map $d \colon X \to A$ of simplicial sets. In particular, when $X$ is the nerve of a small category and $A$ is the homotopy coherent nerve of a locally Kan simplicial category, a diagram is precisely a \emph{homotopy coherent diagram} in the sense of Cordier, Porter, Vogt, and others \cite{Cordier:1986:HtyCoh}.

\begin{ntn}
  From here on  we use $c\colon A\to A^X$ to denote the {\em constant diagram map}: the adjoint transpose of the projection map $\pi_A\colon A\times X\to A$. Furthermore, we shall notationally identify functors $f\colon X\to A$ and natural transformations $\alpha\colon f\Rightarrow g\colon X\to A$ with their adjoint transposes $f\colon\Del^0\to A^X$ and $\alpha\colon f\Rightarrow g\colon \Del^1\to A^X$ respectively.
\end{ntn}

\begin{defn}\label{defn:limit} We say that an absolute right lifting diagram 
    \begin{equation}\label{eq:genericlimit}
      \xymatrix{ \ar@{}[dr]|(.7){\Downarrow\lambda} & A \ar[d]^c \\ \Delta^0 \ar[ur]^\ell \ar[r]_d& A^X}
    \end{equation}
    {\em displays the vertex $\ell\in A$ as the limit of the diagram $d\colon X\to A$}. The 2-cell $\lambda$, which we may equally regard as going from the constant diagram  $X \xrightarrow{!} \Del^0 \xrightarrow{\ell} A$ to $d$, is called the \emph{limiting cone}. Dually, we say that an absolute left lifting diagram 
  \begin{equation}\label{eq:genericcolimit} 
    \xymatrix{ \ar@{}[dr]|(.7){\Uparrow\lambda} & A \ar[d]^c \\ \Delta^0 \ar[ur]^\ell \ar[r]_d & A^X}
  \end{equation} 
  {\em displays the vertex $\ell\in A$ as the colimit of the diagram $d\colon X\to A$}.  Here again the 2-cell $\lambda$, from $d$  to the constant diagram $X \xrightarrow{!} \Del^0 \xrightarrow{\ell} A$, is called the \emph{colimiting cone}.
\end{defn}

\begin{rmk}
For the most part in what follows, we shall present our results in terms of limits and absolute right liftings only. Of course, these arguments all admit the obvious duals which apply to colimits and absolute left liftings. Indeed the results of this section and the last are almost exclusively matters of formal 2-category theory. Their duals follow by re-interpreting these arguments in the dual 2-category $\qCat_2\co$ obtained by reversing the direction of all 2-cells.
\end{rmk}

A special case of Proposition~\ref{prop:right.liftings.as.fibred.terminal.objects} gives an alternative definition of limits and colimits in a quasi-category.

\begin{prop}\label{prop:limits.as.terminal.objects} A limit of $d \colon X \to A$ is a terminal object in the quasi-category $c\comma d$, and conversely a terminal object defines a limit.
\end{prop}
\begin{proof}
A limiting cone defines a vertex in the comma quasi-category $c\comma d$ by 1-cell induction; Lemma~\ref{lem:1cell-ind-uniqueness} and Proposition~\ref{prop:right.liftings.as.fibred.terminal.objects} tell us this vertex is unique up to isomorphism and terminal. Conversely, Proposition~\ref{prop:right.liftings.as.fibred.terminal.objects} implies that the data of a terminal object in $c\comma d$ defines a limit object $\ell\in A$ and a limiting cone $\lambda$ in the sense of Definition~\ref{defn:limit}.
\end{proof}

An important corollary of Proposition~\ref{prop:limits.as.terminal.objects} is that our definition of limit agrees with the existing ones in the literature. As discussed in section \ref{subsec:join} and seen already in the proof of Proposition \ref{prop:terminalconverse}, this proof makes use of an equivalence between Joyal's slice construction and our comma construction. In this case we show that the quasi-category of cones $c \comma d$ over a diagram $d \colon X \to A$ is equivalent to Joyal's quasi-category of cones $\slicer{A}{d}$, recalled in \ref{defn:slices}. 

\begin{lem}\label{lem:cone-equiv-fatcone} For any diagram $d \colon X \to A$ in a quasi-category $A$, there is an equivalence
\[ \xymatrix{ \slicer{A}{d} \ar[rr]^\simeq \ar@{->>}[dr]_{\pi} & & c \comma d \ar@{->>}[dl]^{q_0} \\ & A}\] of quasi-categories over $A$.
\end{lem}
\begin{proof}
As in the proof of Lemma \ref{lem:slice-equiv-comma}, we will demonstrate an isomorphism $c \comma d \cong \fatslicer{A}{d}$ over $A$ between the quasi-category of cones and the fat slice construction on $d \colon X \to A$ defined in \ref{defn:fat-slices}. Via this isomorphism, the equivalence $\slicer{A}{d}\simeq c \comma d$ is a special case of the equivalence of Proposition~\ref{prop:slice-fatslice-equiv}.

To establish the isomorphism, it suffices to show that $c \comma d$ has the universal property that defines $\fatslicer{A}{d}$.  By adjunction, a map $Y \to \fatslicer{A}{d}$ corresponds to a commutative square, as displayed on the left:
\[ \vcenter{   
      \xymatrix@R=2em@C=4em{
        {(Y\times X)\sqcup(Y\times X)}\ar[r]^-{\pi_Y\sqcup\pi_X}
        \ar[d] &
        {Y\sqcup X}\ar[d]^{\langle f, d\rangle} \\
        {Y\times\Del^1\times X} \ar[r]_-{k} &
        {A}
      }
} \qquad \leftrightsquigarrow \qquad \vcenter{      \xymatrix@R=2em@C=4em{
        {Y}\ar[r]^{k}\ar[d]_{(!,f)} & {(A^X)^{\Del^1}} 
        \ar[d] \\
        { \Del^0\times A}\ar[r]_-{d\times c} & {A^X\times A^X}
      }}\]
which transposes to the commutative square displayed on the right. The data of the right-hand square is precisely that of a map $Y \to c \comma d$ by the universal property of the pullback \ref{def:comma-obj} defining the comma quasi-category.
\end{proof}

Joyal defines a limit of a diagram $d \colon X \to A$ to be a terminal vertex $t$ in the slice quasi-category $\slicer{A}{d}$, thought of as the ``quasi-category of cones'' over $d$.  If $\pi\colon \slicer{A}{d}\tfib A$ denotes the canonical projection then such a limiting cone displays $\ell\defeq \pi t$ as a limit of $d$. 

\begin{prop}\label{prop:limits.are.limits}
The notion of limit and limit cone introduced in Definition \ref{defn:limit} is equivalent to the notion of limit and limit cone introduced by Joyal in \cite[4.5]{Joyal:2002:QuasiCategories}.
\end{prop}
\begin{proof}
By Proposition~\ref{prop:limits.as.terminal.objects} tells us that our definition can be recast in a corresponding form: as a terminal vertex $t$ in  the comma quasi-category $c\comma d$. Our ``quasi-category of cones'' is  equipped with a projection $q_0 \colon c \comma d \to A$, and by Proposition~\ref{prop:right.liftings.as.fibred.terminal.objects} such a limiting cone displays $\ell\defeq q_0t$ as a limit of $d$.

Lemma \ref{lem:cone-equiv-fatcone} supplies an equivalence over $A$ between the quasi-category of cones and Joyal's slice quasi-category $\slicer{A}{d}$. Applying Proposition~\ref{prop:terminaldefn}, our preservation result for terminal objects, we see that this equivalence maps a limit cone in Joyal's sense to a limit cone in our sense and vice versa. Furthermore, since this is an equivalence over $A$, it follows that these corresponding cones display the same vertex $\ell$ as the limit of $d$.
\end{proof}

\begin{defn}\label{defn:families.of.diagrams}
  A {\em family $k$ of diagrams of shape $X$\/} in a quasi-category $A$ is simply a functor $k\colon K\to A^X$. In many cases, $K$ will  be the full sub-quasi-category of $A^X$ determined by some set of diagrams and $k$ will be the inclusion $K\inc A^X$.

  We say that $A$ {\em admits limits of the family of diagrams $k\colon K\to A^X$\/} if there exists an absolute right lifting diagram:
\begin{equation}\label{eq:limits.of.a.family}
      \xymatrix{ \ar@{}[dr]|(.7){\Downarrow\lambda} & A \ar[d]^c \\ K \ar[ur]^\lim \ar[r]_k & A^X}
\end{equation}
  Furthermore, we shall simply say that $A$ admits {\em all limits of shape\/} $X$ if it admits limits of the family of all diagrams $A^X$. 

  A diagram $d\colon X\to A$ is said to be a member of the family $k$ if it is a vertex in the image of $k$, that is to say if there is a vertex $\bar{d}\in K$ such that $d=k\bar{d}$. It is trivially verified, directly from the universal property of absolute right liftings, that if $A$ admits limits of the family of diagrams $k$ and $d$ is a member of the family $k$ then the restricted triangle
  \begin{equation*}
      \xymatrix{ \ar@{}[dr]|(.7){\Downarrow\lambda\bar{d}} & A \ar[d]^c \\ \Del^0 \ar[ur]^{\lim \bar{d}} \ar[r]_d & A^X}
  \end{equation*}
  is again an absolute right lifting, thus providing us with a limit of individual diagram $d$. Our use of the adjective ``absolute'' here coincides with its usual meaning: absolute lifting diagrams are preserved by pre-composition by all functors.
\end{defn}

This result has the following converse, whose proof we delay to section~\ref{sec:pointwise}:

\begin{prop}\label{prop:families.of.diagrams}
  If $A$ admits the limit of each individual diagram $d\colon X\to A$ in the family $k\colon K\to A^X$ then it admits limits of the family of diagrams $k$.
\end{prop}

As a special case of Example \ref{ex:adjasabslifting}:

\begin{prop}\label{prop:limitsasadjunctions} A quasi-category $A$ has all limits of shape $X$ if and only if there exists an adjunction \[ \adjdisplay c -| \lim : A^X -> A.\]
\end{prop}

A key advantage of our 2-categorical definition of (co)limits in any quasi-category is that it permits us to use standard 2-categorical arguments to give easy proofs of the expected categorical theorems.

\begin{prop}\label{prop:RAPL} Right adjoints preserve limits.
\end{prop}

Our proof will closely follow the classical one. Given a diagram $d\colon X \to A$ and a right adjoint $u \colon A \to B$ to some functor $f$, a cone with summit $b$ over $ud$ transposes to a cone with summit $fb$ over $d$, which factorises uniquely through the limit cone. This factorisation transposes back across the adjunction to show that the image of the limit cone under $u$ defines a limit over $ud$.

\begin{proof}
Suppose that $A$ admits limits of a family of diagrams $k\colon K\to A^X$ as witnessed by an absolute right lifting diagram~\eqref{eq:limits.of.a.family}. Given an adjunction $f \dashv u$, and hence by Proposition \ref{prop:expadj} an adjunction $f^X \dashv u^X$, we must show that \[\xymatrix{ \ar@{}[dr]|(.7){\Downarrow\lambda} & A \ar[d]^-{c} \ar[r]^u & B \ar[d]^-{c} \\ K \ar[ur]^\lim \ar[r]_k& A^X \ar[r]_{u^X} & B^X}\] is an absolute right lifting diagram. Given a square
\[\xymatrix{ Y \ar[d]_{a} \ar[rr]^b \ar@{}[drr]|{\Downarrow\chi} & & B \ar[d]^-{c} \\ K \ar[r]_-{k} & A^X \ar[r]_{u^X} & B^X} \] we first transpose across the adjunction, by composing with $f$ and the counit. 
\[\vcenter{\xymatrix{ Y \ar[d]_-{a} \ar[rr]^b \ar@{}[drr]|{\Downarrow\chi} & & B \ar[d]^-{c} \ar[r]^f & A \ar[d]^-{c}  \\ K \ar[r]_-{k} & A^X \ar@{=}@/_3.5ex/[rr]^{\Downarrow\epsilon^X} \ar[r]^{u^X} & B^X \ar[r]^{f^X} & A^X}} = \vcenter{\xymatrix{ Y \ar@{}[drr]|(.3){\exists !\Downarrow\zeta}|(.7){\Downarrow\lambda} \ar[d]_-{a} \ar[r]^b & B \ar[r]^f & A \ar[d]^-{c} \\ K \ar[urr]_(0.4){\lim} \ar[rr]_{k} & & A^X}} \] Applying the universal property of the absolute right lifting diagram~\eqref{eq:limits.of.a.family} produces a factorisation $\zeta$, which may then be transposed back across the adjunction by composing with $u$ and the unit.
\[  \vcenter{\xymatrix{ Y \ar@{}[drr]|(.3){\exists !\Downarrow\zeta}|(.7){\Downarrow\lambda} \ar[d]_-{a} \ar[r]^b & B \ar@{=}@/^3.5ex/[rr]_{\Downarrow\eta} \ar[r]|f & A \ar[d]^-{c} \ar[r]_u & B \ar[d]^-{c} \\ K \ar[urr]_(0.4){\lim} \ar[rr]_-{k} & & A^X \ar[r]_{u^X} & B^X}}= \vcenter{\xymatrix{ Y \ar[d]_-{a} \ar[rr]^b \ar@{}[drr]|{\Downarrow\chi} & & B \ar[d]^-{c} \ar@{=}@/^3.5ex/[rr]_{\Downarrow\eta} \ar[r]_f & A \ar[d]^-{c} \ar[r]_u & B \ar[d]^-{c}  \\ K \ar[r]_-{k} & A^X \ar@{=}@/_3.5ex/[rr]^{\Downarrow\epsilon^X} \ar[r]^{u^X} & B^X \ar[r]^{f^X} & A^X \ar[r]_{u^X} & B^X}}  \] \[ = \vcenter{\xymatrix{ Y \ar[d]_-{a} \ar[rr]^b \ar@{}[drr]|{\Downarrow\chi} & & B \ar[d]^-{c} \ar@{=}@/^3.5ex/[rr]  &  & B \ar[d]^-{c}  \\ K \ar[r]_-{k} & A^X \ar@{=}@/_3.5ex/[rr]^{\Downarrow\epsilon^X} \ar[r]^{u^X} & B^X \ar[r]|{f^X}  \ar@{=}@/^3.5ex/[rr]_{\Downarrow\eta^X}& A^X \ar[r]_{u^X} & B^X}}  = \vcenter{\xymatrix{ Y \ar[d]_-{a} \ar[rr]^b \ar@{}[drr]|{\Downarrow\chi} & & B \ar[d]^-{c} \\ K \ar[r]_-{k} & A^X \ar[r]_{u^X} & B^X}}\] Here the second equality is immediate from the definition of $\eta^X$ and the third is by the triangle identity defining the adjunction $f^X \dashv u^X$. The pasted composite of $\zeta$ and $\eta$ is the desired factorisation of $\chi$ through $\lambda$. 

The proof that this factorisation is unique, which again parallels the classical argument, is left to the reader: the essential point is that the transposes are unique.
\end{proof}

\begin{cor}\label{cor:equivprescolim} Equivalences preserve limits and colimits.
\end{cor}
\begin{proof} This follows immediately from Propositions \ref{prop:RAPL} and \ref{prop:equivtoadjoint}.
\end{proof}

\begin{obs}\label{obs:transpose-abs-lifting}
  Under the 2-adjunction $-\times Y\dashv (-)^Y$ triangles of the form 
  \begin{equation}\label{eq:untransposed}
    \xymatrix{
      & {B}\ar[d]^-{f} \\
      {K\times Y} \ar[ur]^{\ell} \ar[r]_-{k} 
      & {A} \ar@{}[ul]|(0.35){\Downarrow \lambda}
    }
  \end{equation}
  correspond to transposed diagrams:
  \begin{equation}\label{eq:transposed}
    \xymatrix{
      & {B^Y}\ar[d]^-{f^Y} \\
      {K} \ar[ur]^{\hat\ell} \ar[r]_-{\hat{k}} 
      & {A^Y} \ar@{}[ul]|(0.35){\Downarrow \hat\lambda}
    }
  \end{equation}
  Furthermore, if the first of these triangles is an absolute right lifting then so is the second one. To prove this, we must show that we can uniquely factorise the 2-cell in a square \[ \xymatrix{ Z \ar[d]_-{u} \ar[r]^-{v} \ar@{}[dr]|{\Downarrow\alpha} & B^Y \ar[d]^-{f^Y} \\ K \ar[r]_-{\hat{k}} & A^Y} \] through the 2-cell $\hat\lambda$ in~\eqref{eq:transposed}. Transposing that square under the 2-adjunction, we obtain the square on the left of the following diagram: \[ \vcenter{\xymatrix{ Z \times Y \ar[d]_-{\tilde{u}} \ar[r]^-{\tilde{v}} \ar@{}[dr]|{\Downarrow \tilde{\alpha}} & B \ar[d]^-{f} \\ K\times Y \ar[r]_-{k} & A}} = \vcenter{\xymatrix{ Z \times Y \ar[d]_-{\tilde{u}} \ar[r]^-{\tilde{v}} \ar@{}[dr]|(.3){\exists !\Downarrow}|(.7){\Downarrow\lambda} & B \ar[d]^-{f} \\ K\times Y \ar[r]_-{k} \ar[ur]_(0.4){\ell} & A}}\] The unique factorisation on the right arises from the universal property of the absolute lifting diagram~\eqref{eq:untransposed}, and its transpose provides the desired unique factorisation of $\alpha$.
\end{obs}

\begin{prop}[pointwise limits in functor quasi-categories]\label{prop:pointwise-limits-in-functor-quasi-categories}
  If a quasi-category $A$ admits limits of the family of diagrams $k\colon K\to A^X$ of shape $X$ then the functor quasi-category $A^Y$ admits limits of the corresponding family of diagrams $k^Y\colon K^Y\to (A^X)^Y\cong(A^Y)^X$ of shape $X$.
\end{prop}

\begin{proof}
  On precomposing the absolute right lifting that displays the limits of the family $k\colon K\to A^X$ \eqref{eq:limits.of.a.family} by the evaluation map $\ev\colon K^Y\times Y\to K$, we obtain an absolute right lifting diagram whose adjoint transpose under the 2-adjunction ${-}\times Y\dashv (-)^Y$ is the triangle
  \begin{equation*}
      \xymatrix{ \ar@{}[dr]|(.7){\Downarrow\lambda^Y} & A^Y \ar[d]^-{c^Y} \\ K^Y \ar[ur]^{\lim^Y} \ar[r]_-{k^Y} & (A^X)^Y}
\end{equation*}
  By the last observation, this is again an absolute right lifting diagram which, on composition with the canonical isomorphism $(A^X)^Y\cong(A^Y)^X$, displays $\lim^Y$ as the family of limits required in the statement.
\end{proof}

Proposition \ref{prop:limitsasadjunctions} tells us that if $A$ has all limits of shape $X$, then there is a functor $\lim \colon A^X \to A$ that is right adjoint to the constant functor $c \colon A \to A^X$. In ordinary category theory we often deploy another adjunction related to the existence of limits of shape $X$, this being the restriction--right Kan extension adjunction between diagrams of shape $X$ and diagrams whose shape is that of a cone over $X$.

The shape of a cone over a diagram of shape $X$ is given by the simplicial set $\Del^0\join X$, defined using Joyal's join construction of Definition~\ref{defn:join-dec}.

\begin{prop}\label{prop:ran.adj.limits} A quasi-category $A$ admits limits of the family of diagrams $k\colon K\to A^X$ of shape $X$ if and only if there exists an absolute right lifting diagram
\begin{equation*}
  \xymatrix{
    & *+[r]{A^{\Del^0\join X}}\ar@{->>}[d]^-{\res} \\
    {K} \ar[ur]^{\ran} \ar[r]_-{k} 
    & *+[r]{A^X} \ar@{}[ul]|(0.3){\Downarrow\lambda}
  }
\end{equation*}
in which  $\res$  is the restriction isofibration given by pre-composition with the inclusion $X\inc \Del^0\join X$. Furthermore, when these equivalent conditions hold $\lambda$ is necessarily an isomorphism and, indeed, we may choose $\ran$ so that $\lambda$ is an identity.
\end{prop}

\begin{proof}
By Proposition~\ref{prop:join-fatjoin-equiv}, the canonical comparison $\Del^0\fatjoin X\to\Del^0\join X$ is a weak equivalence in Joyal's model structure. So if $A$ is a quasi-category, it follows, by Proposition~\ref{prop:equivsareequivs2}, that the associated pre-composition functor $A^{\Del^0\join X}\to A^{\Del^0\fatjoin X}$ is an equivalence of quasi-categories. Now the contravariant exponential functor $A^{({-})}\colon \sSet\op\to \qCat$ carries colimits to limits so it is immediate, from Definition~\ref{eq:fat-join-def}, that we have a pullback \[ \xymatrix{ A^{\Delta^0\fatjoin X} \pbexcursion \ar[r] \ar[d] & A^{X \times \Delta^1} \ar[d] \\ A \times A^X\cong A^{\Delta^0 \sqcup X}  \ar[r] & A^{X \sqcup X} \cong A^X \times A^X}\] from which we see that $A^{\Delta^0\fatjoin X}$ is isomorphic to the comma quasi-category $c\comma A^X$. It is now easily checked that a triangle of the form given in the statement is an absolute right lifting if and only if the following rearranged triangle 
\begin{equation*}
  \vcenter{\xymatrix{
      & {c\comma A^X} \ar@{->>}[d]^-{p_1} \\
      {K} \ar[ur]^{\ran} \ar[r]_-{k} 
      & {A^X} \ar@{}[ul]|(0.3){\Downarrow\lambda}
  }} \mkern20mu \defeq \mkern20mu
  \vcenter{\xymatrix{
    & *+[r]{A^{\Del^0\join X}}\ar@{->>}[d]^-{\res}\ar[r]^-{\sim} &  {c\comma A^X} \ar@{->>}[dl]^{p_1}\\
    {K} \ar[ur]^{\ran} \ar[r]_-{k} 
    & *+[r]{A^X} \ar@{}[ul]|(0.3){\Downarrow\lambda} &
  }}
\end{equation*}
has that property; now the current result is merely a special case of Proposition~\ref{prop:translated.lifting}.
\end{proof}

\begin{cor}\label{cor:ran.adj.limits} A quasi-category $A$ admits all limits of shape $X$ if and only if the restriction functor associated with the inclusion $X\inc \Del^0\join X$ has a fibred right adjoint \[ 
  \xymatrix@=1.5em{
    {A^X}\ar@/_1.2ex/[rr]_-\ran\ar@{=}[dr] \ar@{}[rr]|*{\bot} && 
    *+!L(0.5){A^{\Del^0\join X}}\ar@/_1.2ex/[ll]_-\res \ar@{->>}[dl]^{\res} \\
    & A^X &
  }
\] 
\end{cor}
\begin{proof}
Since the restriction functor $A^{\Del^0\join X} \tfib A^X$ is an isofibration, we may follow Example~\ref{ex:isofib-section.fibred.adjunction} and pick its right adjoint so that the counit of the adjunction $\res \dashv\ran$ is an identity. By Corollary~\ref{cor:missed-lemma}, this adjunction lifts to an adjunction fibred over $A^X$.
\end{proof}

As an application of some significant classical interest, we may use Proposition~\ref{prop:ran.adj.limits} to construct a loops--suspension adjunction in any pointed quasi-category admitting certain pullbacks and pushouts.

\begin{defn}[pointed quasi-categories]
    A \emph{zero object} in a quasi-category is an object in there that is both initial and terminal. We say that a quasi-category $A$ is {\em pointed\/} if it has a zero object and write $*\in A$ for that object. We call the counit $\rho\colon *!\Rightarrow\id_A$ of the adjunction $\adjinline * -| ! : A -> \Del^0.$ the {\em family of points\/} of the objects of $A$ and call the unit $\xi\colon \id_A\Rightarrow *!$ of the adjunction $\adjinline ! -| * : A -> \Del^0.$ the {\em family of co-points\/} of the objects of $A$.
\end{defn}

\begin{ntn}[pushout and pullback diagrams]\label{ntn:pb.po.joins}
  We shall adopt the following notation for certain important diagram shapes which arise naturally as simplicial subsets of the square $\Del^1\times\Del^1$:
  \begin{itemize}
    \item $\pbshape$ will denote the simplicial subset $(\Del^1\times\Del^{\fbv{1}})\cup(\Del^{\fbv{1}}\times\Del^1)$, and
    \item $\poshape$  will denote the simplicial subset $(\Del^1\times\Del^{\fbv{0}})\cup(\Del^{\fbv{0}}\times\Del^1)$.
  \end{itemize}
  Of course, $\pbshape$ and $\poshape$ are the shapes of pullback and pushout diagrams,   isomorphic to the horns $\Horn^{2,2}$ and $\Horn^{2,0}$ respectively.    The joins $\Del^0\join\pbshape$ and $\poshape\join\Del^0$ are each isomorphic to the square $\Del^1\times\Del^1$. These isomorphisms identify the canonical inclusions of those joins with the corresponding subset inclusions $\pbshape\inc\Del^1\times\Del^1$ and $\poshape\inc\Del^1\times\Del^1$ respectively.
\end{ntn}

\begin{defn}[pushouts and pullbacks in quasi-categories]
  A {\em pullback\/} in a quasi-category is a limit of a diagram of shape $\pbshape$. Dually a {\em pushout\/} in a quasi-category is a colimit of a diagram of shape $\poshape$. 
\end{defn}

\begin{obs}\label{obs:loops.diag.fam}
The family of points of a pointed quasi-category $A$ may be represented by a simplicial map $\rho\colon A\to A^\cattwo$. Now the pullback diagram shape $\pbshape$ may be represented as a glueing of two copies of $\cattwo$ identified at their initial vertex, so it follows that $A^\pbshape$ may be constructed as a pullback of two copies of $A^\cattwo$ along the projection $p_1\colon A^\cattwo\tfib A$. Consequently, two copies of $\rho$ give rise to a functor $\bar\rho\colon A\to A^\pbshape$. This functor maps each object $a$ of $A$ to a pushout diagram with outer vertices $*$, inner vertex $a$, and maps two copies of the component of $\rho$ at $a$. Dually we may define a corresponding functor $\bar\xi\colon A\to A^\poshape$ using two copies of the family of co-points.
\end{obs}

\begin{defn}[loop spaces and suspensions]\label{defn:loop.susp}
  We say that a pointed quasi-category $A$ admits the construction of {\em loop spaces\/} if it admits limits of the family of diagrams $\bar\rho\colon A\to A^\pbshape$.   Dually, we say that $A$ admits the construction of {\em suspensions\/} if  it admits colimits of the family of diagrams $\bar\xi\colon A\to A^\poshape$. These constructions, when they exist, are displayed by absolute right and left liftings
\begin{equation*}
  \xymatrix{ 
    \ar@{}[dr]|(.7){\Downarrow} & A \ar[d]^c \\ 
    A \ar[ur]^\Omega \ar[r]_{\bar\rho} & A^\pbshape
  }
  \mkern80mu
  \xymatrix{ 
    \ar@{}[dr]|(.7){\Uparrow} & A \ar[d]^c \\ 
    A \ar[ur]^\Sigma \ar[r]_{\bar\xi} & A^\poshape
  }
\end{equation*}
in which $\Omega$ is called the {\em loop space functor\/} and $\Sigma$ is called the {\em suspension functor}. Of course, if $A$ admits all pullbacks (resp.\ pushouts) then, as a special case, it admits the construction of loop spaces (resp. suspensions).
\end{defn}

\begin{ex}
  In the quasi-category of spaces, which we construct by applying the homotopy coherent nerve to the simplicially enriched category of Kan complexes, pushouts and pullbacks are constructed by taking classical homotopy pushouts and pullbacks. The quasi-category of pointed spaces is simply the slice under $\Del^0$ and its pushouts and pullbacks may be computed as in the quasi-category of spaces. It follows, therefore, that the loop space and suspension constructions in this quasi-category coincide with the usual notions in classical homotopy theory.
\end{ex}

The following proposition promotes our classical intuition about the relationship between loop and suspension constructions to a genuine adjunction of quasi-categories. To keep our proof as simple and transparent as possible, we choose to assume that the quasi-category here admits all pushouts and pullbacks, leaving it to the reader to generalise this result to one in which we only assume the existence of loop spaces and suspensions.

\begin{prop}\label{prop:loops-suspension} Suppose that $A$ is a pointed quasi-category which admits all pushouts and pullbacks. Then $A$ has a loops--suspension adjunction \[ \adjdisplay \Sigma -| \Omega: A -> A.\]
\end{prop}

\begin{proof}
  By Corollary~\ref{cor:ran.adj.limits} and the ruminations of~\ref{ntn:pb.po.joins}, the hypothesis that $A$ has pullbacks and pushouts implies that there are adjunctions
\begin{equation}\label{eq:pullback.pushout.adj}    
  \xymatrix@R=0em@!C=8em{
    {A^\pbshape}
    \ar@/_0.55pc/[r]!L(0.5)_-{\ran} 
    \ar@{}[r]!L(0.5)|-{\displaystyle\bot} 
    \ar@{<-}@/^0.55pc/[r]!L(0.5)^-{\res} & 
    {A^{\Delta^1\times\Delta^1}} & 
    {A^\poshape}
    \ar@/_0.55pc/[l]!R(0.5)_-{\lan}
    \ar@{<-}@/^0.55pc/[l]!R(0.5)^-{\res}  
    \ar@{}[l]!R(0.5)|-{\displaystyle\bot}
  }
\end{equation}
which are fibred over $A^\pbshape$ and $A^\poshape$, respectively. Now the inclusion of $\Del^0\sqcup\Del^0$ into $\Del^1\times\Del^1$ which picks out the vertices $(1,0)$ and $(0,1)$ factorises through each of the subsets $\pbshape$ and $\poshape$ and therefore induces restriction isofibrations $A^\pbshape\tfib A\times A$ and $A^\poshape\tfib A\times A$. So we may push forward our fibred adjunctions along these isofibrations to obtain a composable pair of adjunctions fibred over $A\times A$. Composing these and pulling back  along $(*,*)\colon \Del^0\to A\times A$, we obtain an adjunction
\begin{equation}\label{eq:loop.susp.var}
  \adjdisplay \overline\Sigma -| \overline\Omega : A^\pbshape_* -> A^\poshape_*.
\end{equation}
where $A^\pbshape_*\subseteq A^\pbshape$ and $A^\poshape_*\subseteq A^\poshape$ are the sub-quasi-categories of pullback and pushout diagrams whose outer vertices are pinned at the zero object $*$. 

The family of points $\rho\colon A\to A^\cattwo$ discussed in Observation~\ref{obs:loops.diag.fam} factorises through the sub-quasi-category $*\comma A\subseteq A^\cattwo$; hence,  the family of diagrams $\bar\rho\colon A\to A^\pbshape$ for the loop space construction also factorises through $A^\pbshape_*\subseteq A^\pbshape$. Furthermore, it is clear that the pullback expressing $A^\pbshape$ in terms of two copies of $A^\cattwo$ restricts to the pullback expressing $A^\pbshape_*$ in terms of two copies of $*\comma A$ in the following diagram:
\begin{equation*}
  \xymatrix@=1.5em{ 
    A\ar[dr]|*+{\scriptstyle\bar\rho}\ar@/^1.5ex/[drr]^\rho
    \ar@/_1.5ex/[ddr]_\rho &&\\
    & A^\pbshape_* \pbexcursion \ar@{->>}[d]
    \ar@{->>}[r] & {*\comma A} \ar@{->>}[d]^-{p_1} \\ 
    & {* \comma A} \ar@{->>}[r]_-{p_1} & A
  }
\end{equation*}

We claim that each functor in this diagram is an equivalence. To show this start by observing that the initiality of $*$ in $A$ implies that the isofibration $p_1$ is an equivalence, as is its right inverse $\rho$ by the 2-of-3 property. Trivial fibrations are stable under pullback, so the two projections from $A^\pbshape_*$ are equivalences, as is $\bar\rho$ by the 2-of-3 property. Observe also that the functor which restricts each pullback diagram to its inner vertex is an isofibration left inverse to $\bar\rho$ and so, by the 2-of-3 property, it too is an equivalence. The dual argument shows that the family of diagrams $\bar\xi\colon A\to A^\poshape$ for the suspension construction also factorises through $A^\poshape_*\subseteq A^\poshape$ to give an equivalence $\bar\xi\colon A\to A^\poshape_*$ with left inverse the isofibration that restricts each pullback diagram to its inner vertex.

  Now we may promote the equivalences $\bar\rho$ and $\bar\xi$  to adjoint equivalences and compose them with the adjunction~\eqref{eq:loop.susp.var}. The right adjoint in this composite adjunction is equal to the composite $\xymatrix@1{{A}\ar[r]^-{\bar\rho} & {A^\pbshape}\ar[r]^-{\ran} & {A^{\Del^1\times\Del^1}}\ar[r]^-{\res} & {A}}$ in which the last map is the restriction functor associated with the inclusion of $\Del^0$ as the vertex $(0,0)$ of $\Del^1\times\Del^1$. The composite of these last two functors is the pullback functor $\lim\colon A^\pbshape\to A$, so pre-composing it with $\bar\rho\colon A\to A^\pbshape$ produces a functor which picks out limits of the diagrams in the family $\bar\rho$. This must therefore be isomorphic to the loop space functor $\Omega$ by Definition~\ref{defn:loop.susp}. A dual argument demonstrates that the left adjoint in the composite adjunction is isomorphic to the suspension functor $\Sigma$, thus completing the verification that the adjunction we have constructed is the one asked for in the statement.
\end{proof}


\subsection{Geometric realisations of simplicial objects}

A classical result from simplicial homotopy theory states that if a simplicial object admits an augmentation together with a splitting, also called a contracting homotopy or simply ``extra degeneracies'', then the augmentation is homotopy equivalent to its geometric realisation. More precisely, the augmented simplicial object, a diagram of shape $\Del+\op$, defines a colimit cone over the restriction of this diagram to $\Del\op$. 

In this section, we import these ideas into the quasi-categorical context, proving that if a simplicial object in a quasi-category admits an augmentation and a splitting then the augmentation is its quasi-categorical colimit.
Again, the result is not new (cf.~\cite[6.1.3.16]{Lurie:2009fk}), but our proof closely mirrors the classical one (see, e.g.,~\cite{Meyer:84ba}). Specifically, we show that the structure of the contracting homotopies define an absolute left extension diagram in $\Cat$. Furthermore, this universal property is witnessed equationally and so is preserved by any 2-functor.  Dual remarks apply to cosimplicial objects admitting a coaugmentation and a splitting.

The first step is to describe the shape of a split simplicial object. There are two choices, distinguished by whether we choose a ``forwards'' or ``backwards'' contracting homotopy. The corresponding categories are opposites. Let $\Del[t]$ and $\Del[b]$ denote the subcategories of $\Del$ consisting of those maps that preserve the top or bottom element respectively in each ordinal. There is an inclusion  $[0]\oplus -\colon \Del+ \inc \Del[b]$ which freely adjoins a bottom element. Note the degree shift: this functor sends the initial object $[-1] \in \Del+$ to the zero object $[0]\in\Del[b]$. 

A simplicial object is \emph{augmented} if it admits an extension to $\Del+\op$ and \emph{split} if it admits a further extension to $\Del[t] \cong \Del[b]\op$. Evaluating at $[0] \in \Del[t]$ yields the augmentation. Restriction along the inclusion $\Del\op \inc \Del+\op\inc \Del[t]$ yields the original diagram. We will prove:

\begin{thm}\label{thm:splitgeorealizations} For any quasi-category $B$, the canonical diagram \[ \xymatrix{ \ar@{}[dr]|(.7){\Uparrow} & B \ar[d]^c \\ B^{\Del[t]} \ar[ur]^{\ev_0} \ar[r]_{\res} & B^{\Del\op}}\] is an absolute left lifting diagram. Hence, given any simplicial object admitting an augmentation and a splitting, the augmented simplicial object defines a colimit cone over the original simplicial object. Furthermore, such colimits are preserved by any functor.
\end{thm}

Our proof uses a 2-categorical lemma.

\begin{lem}\label{lem:doms2catlemma} Suppose given an adjunction in a slice 2-category $C\slice\tcat{C}$
\[\vcenter{\xymatrix@R=30pt{ & C \ar[dl]|b_{\rotatebox{45}{$\labelstyle\perp$}} \ar[dr]^a & \\ \ar@/^3ex/@{-->}[ur]^c B \ar@/^1ex/[rr]^f  \ar@{}[rr]|\perp & & A \ar@/^1ex/[ll]^u }}\] If $b$ admits a left adjoint $c$ in $\tcat{C}$ with unit $\iota$, then the 2-cell $f\iota \colon f \Rightarrow fbc=ac$ exhibits $c$ as an absolute left lifting of $f$ through $a$.
\end{lem}
\begin{proof}
Let $\nu$ be the counit of $c \dashv b$, and write $\eta$ and $\epsilon$ for the unit and counit of the adjunction $f\dashv u$; because this adjunction is under $C$ we have $\epsilon a = \id_a$ and $\eta b =\id_b$. Any 2-cell $\chi$ of the form displayed below factorises through $f\iota$ as follows
\[\xymatrix{ X \ar[d]_x \ar[r]^y \ar@{}[dr]|{\Uparrow\chi} & C \ar[d]^a \ar@{}[dr]|{\displaystyle =}  & X \ar[d]_x \ar[r]^y \ar@{}[dr]|{\Uparrow\chi} & C \ar[d]^a   \ar@{}[dr]|{\displaystyle =}   &X \ar@{}[dr]|{\Uparrow\chi} \ar[d]_x \ar[r]^y & C \ar[d]_a \ar[r]^a \ar[dr]|b & A   \ar@{}[dr]|{\displaystyle =} & X \ar[d]_x \ar[r]^y \ar@{}[dr]|{\Uparrow\chi} & C \ar[d]_a \ar[dr]^b \ar@{=}[rr] & \ar@{}[d]|{\Uparrow\nu} &  C \ar[d]^a  \\  B \ar[r]_f & A & B \ar[r]^f \ar@{=}[dr] & A \ar[d]|u \ar@{}[dl]|(.3){\Uparrow\eta} \ar@{=}[dr]  &  B \ar[r]^f \ar@{=}@/_3.5ex/[rr]^{\Uparrow\eta} & A \ar[r]^u & B \ar[u]_f &  B \ar[r]^f \ar@{=}@/_3.5ex/[rr]^{\Uparrow\eta} & A \ar[r]^u  & B \ar[ur]^c \ar[r]_f & A \ar@{}[ul]|(.3){\Uparrow f\iota} \\ & & & B \ar[r]_f \ar@{}[ur]|(.3){\Uparrow\epsilon} & A }\]
using a triangle identity for each adjunction and the fact that $\epsilon a = \id_a$. Such factorisations are unique because the 2-cell $\zeta$ can be recovered from the pasted composite with $f\iota$: \[\xymatrix{  X \ar[d]_x \ar[r]^y \ar@{}[dr]|(.3){\Uparrow\zeta}|(.7){\Uparrow f\iota} & C \ar[d]_a \ar[dr]^b \ar@{=}[rr] & \ar@{}[d]|{\Uparrow\nu} &  C  \\ B \ar[ur]|c \ar[r]^f \ar@{=}@/_3.5ex/[rr]^{\Uparrow\eta} & A \ar[r]^u & B \ar[ur]^c & {\displaystyle =} } \xymatrix{ X \ar[d]_x \ar[r]^y \ar@{}[dr]|(.3){\Uparrow\zeta}|(.7){\Uparrow\iota} & C \ar[d]|b \ar[dr]|a \ar[drr]^b \ar@{=}[rr]& & C \ar@{}[dl]|(.4){\Uparrow\nu} \\B \ar[ur]|c \ar@{=}[r] & B  \ar[r]_f \ar@{=}@/_3.5ex/[rr]^{\Uparrow\eta} & A \ar[r]_u & B \ar[u]_c} = \xymatrix{ X \ar[d]_x \ar[r]^y \ar@{}[dr]|(.3){\Uparrow\zeta}|(.7){\Uparrow\iota} & C \ar[d]|b \ar@{}[dr]|(.3){\Uparrow\nu} \ar@{=}[r] & C    \\  B \ar[ur]|c \ar@{=}[r] & B \ar[ur]_c & {\displaystyle =}  }
\xymatrix{ X \ar[d]_x \ar[r]^y \ar@{}[dr]|(.3){\Uparrow\zeta} & C \\ B \ar[ur]_c & }\qedhere
\] 
\end{proof}

\begin{proof}[Proof of Theorem \ref{thm:splitgeorealizations}] The inclusion $\Del\op \hookrightarrow \Del[t]$ admits a left adjoint. One way to define it is to present  $\Del\op$ via the ``interval representation'': after employing a degree shift $[n] \mapsto [n+1]$, $\Del\op$ is the subcategory of $\Del+$ consisting of ordinals with distinct top and bottom elements and maps that preserve these. Most generally, we might think of the interval representation as the diagonal composite functor in the pullback diagram \[\xymatrix{ \Del+\op \pbexcursion \ar[d] \ar[r] & \Del[t] \ar@{_(->}[d] \\ \Del[b] \ar@{^(->}[r] & \Del+}\] The arrows $\Del[b] \leftarrow \Del+\op \to \Del[t]$ extend the category indexing augmented simplicial objects by introducing extra maps that define ``extra degeneracies'' either on the left or on the right. The restricted functor $\Del\op \to \Del[t]$ is the inclusion described above. It has a left adjoint: a map $\alpha \colon [k] \to [n+1]$ in $\Del[t]$ is given by a map $[n] \to [k]$ in $\Del$ that sends $i\in [n]$, thought of as a ``gap'' between adjacent elements in $[n+1]$, to the minimal $j \in [k]$ so that $\alpha(j) = i+1$. 

 For any quasi-category $B$, the 2-functor $B^{(-)} \colon \Cat_2\op \to \qCat_2$ carries the
adjoint functors 
\[\xymatrix@R=30pt{ & \catone \ar[dl]^{\labelstyle[0]} & \\ \Del[t] \ar@/^1ex/[rr] \ar@{}[rr]|\perp  \ar@/^2.5ex/[ur]^{!} \ar@{}[ur]^*-{\rotatebox{45}{$\labelstyle\perp$}} & & \Del\op \ar@/^/[ll] \ar[ul]_{!}}\]
to an adjunction in the slice 2-category $B\slice\qCat_2$
\[\xymatrix@R=30pt{ & B \ar[dl]^{\labelstyle c} \ar[dr]^c & \\ B^{\Del[t]} \ar@/^1ex/[rr]^{\res} \ar@{}[rr]|\perp  \ar@/^2.5ex/[ur]^{\ev_0} \ar@{}[ur]^*-{\rotatebox{45}{$\labelstyle\perp$}} & & B^{\Del\op} \ar@/^/[ll] }\] The 2-cell defined by whiskering $\res$ with the unit of $\ev_0\dashv c$ is the 2-cell $\res \Rightarrow c \cdot \ev_0$ obtained by applying the 2-functor $B^{-}$ to the unique 2-cell
\[\xymatrix{ \Del\op \ar@{^(->}[rr] \ar[dr]_{!} & \ar@{}[d]|(.35){\Downarrow} & \Del[t] \\ & \catone \ar[ur]_{[0]} & } \] that exists because $[0] \in \Del[t]$ is terminal. The result now follows from Lemma \ref{lem:doms2catlemma}.

It remains only to prove the last statement. Given any functor $f \colon B \to A$, the diagrams \[ \vcenter{\xymatrix{ \ar@{}[dr]|(.7){\Uparrow} & B \ar[d]^c \ar[r]^f & A \ar[d]^c \\ B^{\Del[t]} \ar[ur]^{\ev_0} \ar[r]_{\res} & B^{\Del\op} \ar[r]_{f^{\Del\op}} & A^{\Del\op}}} = \vcenter{ \xymatrix{ &  \ar@{}[dr]|(.7){\Uparrow} & B \ar[d]^c \\ B^{\Del[t]} \ar[r]_{f^{\Del[t]}} & A^{\Del[t]} \ar[ur]^{\ev_0} \ar[r]_{\res} & A^{\Del\op}}}\] coincide by bifunctoriality of the internal hom 2-functor in $\qCat_2$. In particular, the left-hand side inherits the universal property of the right-hand side.
\end{proof}

\begin{ex} Theorem \ref{thm:splitgeorealizations} can be used to prove that any object in the quasi-category of algebras associated to a coherent monad is a homotopy colimit of a canonical simplicial object of free algebras. See \cite{RiehlVerity:2012hc} and \cite{RiehlVerity:2013cp}.
\end{ex}


