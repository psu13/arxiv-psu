%!TEX root = all.tex
% ******************************************************************
% ** Title:            The 2-category theory of quasi-categories
% **                   geometric background
% ** Precis:        
% ** Author:           Emily Riehl and Dominic Verity
% ** Commenced:        2/3/2012
% ******************************************************************


\newcommand{\captionwidth}{14cm}

\newcommand{\agt}{\,\rlap{\lower 3.5 pt \hbox{$\mathchar \sim$}} \raise 1pt
 \hbox {$>$}\,}
\newcommand{\alt}{\,\rlap{\lower 3.5 pt \hbox{$\mathchar \sim$}} \raise 1pt
 \hbox {$<$}\,}

\newcommand{\gsim}{\;\rlap{\lower 3.5 pt \hbox{$\mathchar \sim$}} \raise 1pt
 \hbox {$>$}\;}
\newcommand{\lsim}{\;\rlap{\lower 3.5 pt \hbox{$\mathchar \sim$}} \raise 1pt
 \hbox {$<$}\;}
\newcommand{\dd}{{\rm d}}                  % differential
\newcommand{\bld}[1]{\boldmath{$#1$}}      % bold math symbols
\newcommand{\sprod}[2]{#1\hspace{-.1em}\cdot\hspace{-.1em}#2}   % dot-product
\newcommand{\llangle}{\left\langle}        
\newcommand{\rrangle}{\right\rangle}

\renewcommand{\Re}{{\rm Re}}
\renewcommand{\Im}{{\rm Im}}
\newcommand{\msbar}{$\overline{\rm MS}$}

\newcommand{\yp}{y}
\newcommand{\logtwos}{L_{tW}}
\newcommand{\logtwms}{l_{tW}}
\newcommand{\logctheta}{l_\Theta}
\newcommand{\logmuW}{l_{\mu W}}

\newcommand{\logqmms}{l_{qm}}
\newcommand{\logqmos}{L_m}
\newcommand{\logmsms}{l_s}
\newcommand{\logmsos}{L_{ms}}
\newcommand{\logmusos}{L_{s\mu}}
\newcommand{\logqmums}{l_{q\mu}}
\newcommand{\logmum}{l_{\mu m}}






\section{Geometry}\label{app:geometry}

Our approach to developing the category theory of quasi-categories makes use of the enriched category theories of 2-categories and simplicial categories. Traditional accounts of quasi-category theory have instead employed ``d{\'e}calage'' constructions to define and develop the theory of limits and colimits, adjunctions, and so forth. In this appendix, we demonstrate that these approaches are entirely equivalent by showing that d{\'e}calage constructions may be obtained, in an up to  equivalence sense, using constructions involving the comma quasi-categories introduced in definition~\ref{def:comma-obj}.

The literature already contains a proof that these two constructions are equivalent; see for instance Lurie~\cite[4.2.1.5]{Lurie:2009fk}. However, given the importance of this result to our work, we  beg the indulgence of the reader and devote this appendix to providing a very concrete, explicit, and self-contained presentation of this result.  

  We begin in section \ref{subsec:join} by reviewing Joyal's join and slice constructions, the d{\'e}calage-style constructions mentioned above. The left and right slices associated to a map $f \colon X \to A$ whose codomain is a quasi-category can be interpreted as the quasi-category of cones under and over $f$ respectively. In section \ref{subsec:cones}, we described a variant ``fat cone'' construction as a comma quasi-category. The quasi-categories of fat cones appeared in the definition of limits and colimits expressed by proposition~\ref{prop:limits.as.terminal.objects}. In section \ref{subsec:fatjoin}, we introduce a fattened version of the join and slice constructions and prove that the fat slice construction is isomorphic to the fat cone construction. It remains only to show that ordinary slices and fat slice are equivalent. We prove this in section \ref{subsec:relating} by proving that a natural map from the fat join to the join is an equivalence for any pair of simplicial sets.
 
  \subsection{Joins and slices}\label{subsec:join}

  \begin{rec}[joins and d{\'e}calage]\label{rec:join-dec} The algebraists' skeletal category $\Del+$ of all finite ordinals and order preserving maps supports a canonical strict (non-symmetric) monoidal structure $(\Del+,\oplus,[-1])$ in which $\oplus$ denotes the {\em ordinal sum\/} given 
  \begin{itemize}
    \item for objects $[n],[m]\in\Del+$ by $[n]\oplus[m] \defeq [n+m+1]$,
    \item for arrows $\alpha\colon[n]\to[n'], \beta\colon[m]\to[m']$ by $\alpha\oplus\beta\colon[n+m+1]\to[n'+m'+1]$ defined by
  \begin{equation*}
    \alpha\oplus\beta(i) =
    \begin{cases}
    \alpha(i)& \text{if $i\leq n$,} \\
    \beta(i-n-1) + n' + 1& \text{otherwise.}
    \end{cases}
  \end{equation*}
  \end{itemize}
By Day convolution, this bifunctor extends to a (non-symmetric) monoidal closed structure $(\asSet, \join, \Del^{-1}, \dec_l, \dec_r)$ on the category of augmented simplicial sets. Here the monoidal operation $\join$ is known as the {\em simplicial join\/} and its closures $\dec_l$ and $\dec_r$ are known as the {\em left and right d{\'e}calage constructions}, respectively.   To fix handedness, we declare that for each augmented simplicial set $X$ the functor $\dec_l(X,{-})$ (resp.\ $\dec_r(X,{-})$) is right adjoint to $X\join{-}$ (resp.\ ${-}\join X$).

  If $X$ and $Y$ are augmented simplicial sets then their join $X\join Y$ may be described explicitly as follows:
  \begin{itemize}
    \item it has simplices pairs $(x,y) \in (X\join Y)_{r+s+1}$ with $x\in X_r$, $y\in Y_s$,
    \item if $(x,y)$ is a simplex of $X\join Y$ with $x \in X_r$ and $y \in Y_s$ and $\alpha\colon[n]\to[r+s+1]$ is a simplicial operator in $\Del+$, then $\alpha$ may be uniquely decomposed as $\alpha=\alpha_1\join\alpha_2$ with $\alpha_1\colon[n_1]\to[r]$ and $\alpha_2\colon[n_2]\to[s]$, and we define $(x,y)\cdot\alpha\defeq (x\cdot\alpha_1,y\cdot\alpha_2)$. 
  \end{itemize}
  Furthermore, if $f\colon X\to X'$ and $g\colon Y\to Y'$ are simplicial maps then the simplicial map $f\join g\colon X\join Y\to X'\join Y'$ simply carries the simplex $(x,y)\in X\join Y$ to the simplex $(f(x),g(y))\in X'\join Y'$. 
  \end{rec}

  \begin{defn}[d{\'e}calage and slices]\label{defn:slices}
    In~\cite{Joyal:2002:QuasiCategories}, Joyal introduces a {\em slice\/} construction for maps $f\colon X\to A$ of simplicial sets. To describe this construction, we start by identifying the category of simplicial sets $\sSet$ with the full subcategory of terminally augmented simplicial sets in $\asSet$, fixing a simplicial set $X$ and defining a functor 
    \begin{equation*}
      {-}\mathbin{\bar\join} X\colon \xy<0em,0em>*+{\sSet}\ar <6em,0em>*+{X\slice\sSet}\endxy\mkern30mu
      (\text{resp.\ } 
      X\mathbin{\bar\join}{-}\colon \xy<0em,0em>*+{\sSet}\ar <6em,0em>*+{X\slice\sSet}\endxy)
    \end{equation*}
    which carries a simplicial set $Y\in\sSet$ to the object ${*}\join X\colon X\cong\Del^{-1}\join X\to Y\join X$ (resp.\ $X\join{*}\colon X\cong X\join\Del^{-1} \to X\join Y$) induced by the map ${*}\colon\Del^{-1}\to Y$ corresponding to the unique $(-1)$-simplex of $Y$. Now we may show that this functor preserves all colimits, from which fact we may infer that it possesses a right adjoint $\slc^X_r({-})$ (resp,\ $\slc^X_l({-})$). 
    
    However, we prefer to derive these right adjoints from the d{\'e}calage construction of recollection~\ref{rec:join-dec}. Specifically, observe that the $(-1)$-dimensional simplices of $\dec_r(X, A)$ (resp.\ $\dec_l(X,A)$) are in bijective correspondence with simplicial maps $f\colon X\to A$. So if we are given such a simplicial map we may, by recollection \ref{rec:augmentation}, extract the component of $\dec_r(X,A)$ (resp.\ $\dec_l(X,A)$) consisting of those simplices whose $(-1)$-face is $f$, which we denote by $\slicer{A}{f}$ (resp.\ $\slicel{f}{A}$) and call the {\em slice of $A$ over (resp.\ under) $f$}. Now it is a matter of an easy calculation to demonstrate directly that $\slicer{A}{f}$ (resp. $\slicel{f}{A}$) has the universal property required of the right adjoint to ${-}\bar\join X$ (resp.\ $X\bar\join {-}$) at the object $f\colon X\to A$ of $X\slice\sSet$.

    In other words, these d{\'e}calages admit the following canonical decompositions as disjoint unions of (terminally augmented) slices: 
    \begin{equation*}
      \dec_r(X,A)=\bigsqcup_{f\colon X\to A} (\slicer{A}{f})\mkern30mu
      \dec_l(X,A)=\bigsqcup_{f\colon X\to A} (\slicel{f}{A})
    \end{equation*}

    We think of the slice $\slicel{f}{A}$ as being the simplicial set of {\em cones under the diagram $f$\/} and we think of the dual slice $\slicer{A}{f}$ as being the simplicial set of {\em cones over the diagram $f$}.
  \end{defn}
  
  \begin{obs}[slices of quasi-categories]\label{obs:slice-and-qcats}
    A direct computation from the explicit description of the join construction given above demonstrates that the Leibniz join (see recollection~\ref{rec:leibniz}) of a horn and a boundary $(\Horn^{n,k}\inc\Del^n)\leib\join(\boundary\Del^m\inc\Del^m)$ is again isomorphic to a single horn $\Horn^{n+m+1,k}\inc\Del^{n+m+1}$. Dually the Leibniz join $(\boundary\Del^n\inc\Del^n)\leib\join(\Horn^{m,k}\inc\Del^m)$ is isomorphic to the single horn $\Horn^{n+m+1,n+k+1}\inc\Del^{n+m+1}$. 

    Combining these computations with the properties of the Leibniz construction developed in \cite[section~\ref*{reedy:sec:Leibniz-Reedy}]{RiehlVerity:2013kx}, we may show that an augmented simplicial set $A$ has the right lifting property with respect to all (inner) horn inclusions then so do the left and right d{\'e}calages $\dec_l(X,A)$ and $\dec_r(X,A)$ for any augmented simplicial set $X$. In particular, this tells us that if $f\colon X\to A$ is any map of simplicial sets and $A$ is a quasi-category then the slices $\slicel{f}{A}$ and $\slicer{A}{f}$ are also quasi-categories. 

    Working in the marked context, we may extend this result to Leibniz joins with specially marked outer horns. That then allows us to prove that if $p\colon A\to B$ is an isofibration of quasi-categories and $f\colon X\to A$ is any simplicial map then the induced simplicial maps $\slc^X_r(p)\colon \slicer{A}{f}\to \slicer{B}{pf}$ and $\slc^X_l(p)\colon \slicel{f}{A}\to \slicel{pf}{B}$ are also isofibrations of quasi-categories.
  \end{obs}

  \subsection{Fat cones}\label{subsec:cones}
  
  \begin{obs}[internal homs and fat cones]\label{obs:fat-cones}
    It is common in category theory to define a cone over (resp.\ cone under) a diagram $f\colon X\to A$ with vertex $a\in A$ to be a natural transformation $\pi\colon a\Rightarrow f$ (resp. $\iota\colon f\Rightarrow a$). Here we use the notation $a$ to denote both an object of $A$ and the corresponding constant functor $X\to A$ at that object. 

    We might reasonably hope to generalise this notion of cone to the quasi-categorical context, by saying that cone over (resp.\ under) a diagram $f\colon X\to A$, whose target is a quasi-category $A$, with vertex $a \in A_0$, is a simplicial map $\pi\colon X\times\Del^1\to A$ (resp.~$\iota\colon X\times\Del^1\to A$) for which $a=\pi\circ(X\times\face^1)$ and $f=\pi\circ(X\times\face^0)$  (resp.~$f=\iota\circ(X\times\face^1)$ and $a=\iota\circ(X\times\face^0)$). Here again, we use the notation $a$ to denote the constant simplicial map which carries every $n$-simplex $x$ of $X$ to the degenerate $n$-simplex on the vertex $a$. This cone notion gives rise to a different construction which is designed to look and behave like a quasi-category of cones over (or under) a diagram. 

    We may make this intuition precise using the comma objects of definition~\ref{def:comma-obj}. To that end suppose that $f\colon X\to A$ is a simplicial map which we can regard equally as being a $0$-simplex of the internal hom $A^X$ or as a simplicial map $\Del^0\to A^X$. Define the {\em constant diagram map\/} $\const\colon A\to A^X$ to be the adjoint transpose of the projection $\pi_A\colon A\times X\to A$. We may form the comma object $f\comma c$ in the following pullback 
    \begin{equation}\label{eq:fat-cones}
      \xymatrix@=2.5em{
        {f\comma c}\pbexcursion \ar[r]\ar@{->>}[d]_{p} &
        {(A^X)^\cattwo} \ar@{->>}[d]^{(p_1,p_0)} \\
        {A\times \Del^0} \ar[r]_-{c\times f} & {A^X\times A^X}
      }
    \end{equation}
    and we choose to call this the simplicial set of {\em fat cones under\/} the diagram $f\colon X\to A$. Dually we call the comma object $c\comma f$ the simplicial set of fat cones {\em over\/} the diagram $f$. Unwinding this definition, we find that a 0-simplex of $f\comma c$, that is to say a fat cone under $f$, is no more nor less than a cone in the  sense just discussed.
  \end{obs}
  
  \begin{obs}[fat cones and quasi-categories]\label{obs:fat-cone-quasicat}
    Because $A$ is supposed to be a quasi-category then we know by the comments at the end of recollection~\ref{rec:qmc-quasicat} that $A^X$ is also a quasi-category, as is $\Del^0$. So the discussion in definition~\ref{def:comma-obj} tells us that the simplicial set of fat cones $f\comma c$ (resp.\ $c\comma f$) is again a quasi-category. Furthermore, if $p\colon A\tfib B$ is an isofibration between quasi-categories then the map $p^X\colon A^X\tfib B^X$ is also an isofibration by observation~\ref{obs:isofibration-closure}. So we may apply the result of observation~\ref{obs:comma-obj-maps} to the diagram
    \begin{equation*}
      \xymatrix{
        {\Del^0}\ar[r]^{f}\ar@{=}[d] & 
        {A^X}\ar@{->>}[d]_-{p^X} & 
        {A}\ar[l]_-{c}\ar@{->>}[d]^-{p} \\
        {\Del^0}\ar[r]_{pf} & 
        {B^X} & 
        {B}\ar[l]^-{c} 
      }
    \end{equation*}
    to show that the induced maps $f\comma c\tfib pf\comma c$ and $c\comma f\tfib c\comma pf$ between the quasi-categories of fat cones are also isofibrations.
  \end{obs}
  
  \subsection{Fat joins and fat slices}\label{subsec:fatjoin}

  \begin{rmk}[relating slices and fat cone constructions]
    One might na{\"\i}vely expect that the two cone notions we have met thus far, these being Joyal's slices and our fat cone construction respectively, actually coincide. In general this is certainly a forlorn hope, as fat cones contain many more simplices than the corresponding Joyal cones. However in the case where $A$ is actually a category it is a classical fact that these notions are isomorphic, and we might at least hope that they are also related {\em up to equivalence\/} in the quasi-categorical context.

This kind of result is of great significance to the study limits and colimits in quasi-categories. In order to take advantage of the weak universal properties of the comma quasi-category construction, we have defined a limit (resp.\ colimit) of a diagram $f\colon X\to A$ to be be a terminal (resp.\ initial) object in the quasi-category $c\comma f$ (resp.\ $f\comma c$) of fat cones. On the other hand, in Joyal~\cite{Joyal:2002:QuasiCategories} and Lurie~\cite{Lurie:2009fk} one finds a limit (resp.\ colimit) of $f$ defined to be a terminal (resp.\ initial) object in the slice $\slicer{A}{f}$ (resp. $\slicel{f}{A}$). To reconcile these definitions it will be enough to demonstrate that the slice and fat cone constructions are related by appropriate equivalences of quasi-categories.
  \end{rmk}

  \begin{defn}[fat join and fat d{\'e}calage]\label{def:fat-join}
    We define the {\em fat join} of two simplicial sets $X$ and $Y$ to be the simplicial set $X\fatjoin Y$ constructed by means of the following pushout:
    \begin{equation}\label{eq:fat-join-def}
      \xymatrix@R=2em@C=4em{
        {({X}\times{Y})\sqcup({X}\times{Y})}\ar[r]^-{\pi_X\sqcup\pi_Y}
        \ar[d]_{\langle{X}\times\face^1\times{Y},
          {X}\times\face^0\times{Y}\rangle} &
        {{X}\sqcup{Y}}\ar[d] \\
        {{X}\times\Del^1\times{Y}} \ar[r] &
        {{X}\fatjoin{Y}}\poexcursion
      }
    \end{equation}
    We may extend this construction to simplicial maps in the obvious way to give us a bifunctor $\fatjoin\colon\sSet\times\sSet\to\sSet$, and it is clear that this preserves {\em connected\/} colimits in each variable. We might caution the reader here in regard to the preservation of {\em all\/} colimits in each variable; this result does not hold simply because the coproduct bifunctor $\sqcup$ (as used in the top right hand corner of the defining pushout above) fails to preserve coproducts  in each variable (while it does preserve connected colimits). In particular, a fat join of a simplicial set $X$ with the empty simplicial set is not itself empty: it is canonically isomorphic to $X$ itself.

    Now if we again identify the category of simplicial sets $\sSet$ with the full subcategory of terminally augmented simplicial sets then we may extend our fat join to a bifunctor on augmented simplicial sets. To do this we start by observing that every augmented simplicial set $X$ may be written canonically as a coproduct $\bigsqcup_{i\in I} X_i$ in $\asSet$ of terminally augmented simplicial sets. So if $X = \bigsqcup_{i\in I} X_i$ and $Y=\bigsqcup_{j\in J} Y_j$ are two augmented simplicial sets with terminally augmented components $X_i$ and $Y_j$, then we define $X\join Y$ to be the coproduct $\bigsqcup_{i\in I, j\in J} X_i\fatjoin Y_j$ in $\asSet$. We extend this to maps of augmented simplicial sets in the obvious way, using the fact that we may decompose such maps into families of maps between terminally augmented components. It is now a routine matter to verify that, when regarded as being a bifunctor on $\asSet$, the fat join does indeed preserve {\em all\/} colimits in each variable. Consequently, since $\asSet$ is a presheaf category, it follows that the fat join bifunctor on $\asSet$ has both left and right closures $\fatdec_l(X,A)$ and $\fatdec_r(X,A)$, called  {\em left and right fat d{\'e}calage\/} respectively, which notation we fix by declaring that if $X$ is an augmented simplicial set then $X\fatjoin{-}\dashv\fatdec_l(X,{-})$ and ${-}\fatjoin X\dashv\fatdec_r(X,{-})$.

    Returning to the defining pushout~\eqref{eq:fat-join-def} it will be of use to observe that the fat join of two {\em non-empty\/} simplicial sets $X$ and $Y$ may be described more concretely as the simplicial set obtained by taking the quotient of ${X}\times\Del^1\times{Y}$ under the simplicial congruence relating the pairs of $r$-simplices
    \begin{equation}\label{eq:fat-join-cong}
    (x,0,y)\sim (x,0,y')\quad \text{and}\quad (x,1,y) \sim (x',1,y) 
    \end{equation}
    where $0$ and $1$ denote the constant operators $[r]\to[1]$. We use square bracketed triples $[x,\beta,y]_\sim$ to denote equivalence classes under $\sim$. 
  \end{defn}

  \begin{defn}[fat slice]\label{defn:fat-slices}
    Replaying Joyal's slice construction of definition~\ref{defn:slices}, if $X$ is a simplicial set, we may use the fat join to construct a functor
    \begin{equation*}
      {-}\mathbin{\bar\fatjoin} X\colon \xy<0em,0em>*+{\sSet}\ar <6em,0em>*+{X\slice\sSet}\endxy\mkern30mu
      (\text{resp.\ } 
      X\mathbin{\bar\fatjoin}{-}\colon \xy<0em,0em>*+{\sSet}\ar <6em,0em>*+{X\slice\sSet}\endxy)
    \end{equation*}
    which carries a simplicial set $Y\in\sSet$ to the object ${*}\fatjoin X\colon X\cong\Del^{-1}\fatjoin X\to Y\fatjoin X$ (resp.\ $X\fatjoin{*}\colon X\cong X\fatjoin\Del^{-1} \to X\fatjoin Y$) and use fat d{\'e}calage to show that it has a right adjoint $\fatslc^X_r({-})$ (resp.\ $\fatslc^X_l({-})$). The value of this right adjoint at an object $f\colon X\to A$ of $X\slice\sSet$ is denoted $\fatslicer{A}{f}$ (resp.\ $\fatslicel{f}{A}$) and is called the {\em fat slice of $A$ over (resp.\ under) $f$}. 
    \end{defn}
    
    \begin{obs}[fat slice vs fat cone]\label{obs:fat-slices}
    If $Y$ is any simplicial set and $f\colon X\to A$ is a simplicial map, then simplicial maps $Y\to \fatslicer{A}{f}$ are in bijective correspondence with simplicial maps $k\colon Y\fatjoin X\to A$ for which the following square commutes:
    \begin{equation*}
      \xymatrix@=2em{
        {\Del^{-1}\fatjoin X}\ar[d]_{{*}\fatjoin X}\ar@{}[r]|-{\textstyle\cong} &
        {X}\ar[d]^f \\
        {Y\fatjoin X}\ar[r]_-{k} & {A}
      }
    \end{equation*}
    Using the universal property of the defining pushout~\eqref{eq:fat-join-def} we see that such maps are themselves in bijective correspondence with pairs of maps $\overline{k}\colon Y\times\Del^1\times X\to A$ and $g\colon Y\to A$ which make the square
    \begin{equation*}
      \xymatrix@R=2em@C=4em{
        {(Y\times X)\sqcup(Y\times X)}\ar[r]^-{\pi_Y\sqcup\pi_X}
        \ar[d]_{\langle Y\times\face^1\times X,
          Y\times\face^0\times X\rangle} &
        {Y\sqcup X}\ar[d]^{\langle g, f\rangle} \\
        {Y\times\Del^1\times X} \ar[r]_-{\overline{k}} &
        {A}
      }
    \end{equation*}
    commute. Taking the transpose $\hat{k}\colon Y\to (A^X)^{\Del^1}$ of $\overline{k}$ under the adjunction ${-}\times\Del^1\times X\dashv ((-)^{X})^{\Del^1}$ we see that the above square commutes if and only if the dual square
    \begin{equation*}
      \xymatrix@R=2em@C=4em{
        {Y}\ar[r]^{\hat{k}}\ar[d]_{(!,g)} & {(A^X)^{\Del^1}} 
        \ar[d]^{((A^X)^{\face^0}, (A^X)^{\face^1})} \\
        { \Del^0\times A}\ar[r]_-{f\times c} & {A^X\times A^X}
      }
    \end{equation*}
    commutes. Finally, consulting the defining pullback~\eqref{eq:fat-cones} for the fat cone construction, we see that pairs $(g,\hat{k})$ which make the square above commute are in bijective correspondence with simplicial maps $Y\to c\comma f$. 
  \end{obs}
  
With this argument we have proven:
  
  \begin{prop}\label{prop:fatsliceisfatcone}
  The fat slice construction $\fatslicer{A}{f}$ (resp.\ $\fatslicel{f}{A}$) is isomorphic to the fat cone construction $c\comma f$ (resp.\ $f\comma c$).\qed
  \end{prop}

  \subsection{Relating joins and fat joins}\label{subsec:relating}

  \begin{obs}[comparing join constructions]
    When $\beta\colon [n]\to [1]$ is a simplicial operator let $\hat{n}_\beta$ denote the largest integer in the set $\{-1\}\cup\{i\in[n]\mid \beta(i)=0\}$  and let $\check{n}_\beta=n-1-\hat{n}_\beta$. Define an associated pair $\hat\beta\colon[\hat{n}_\beta]\to[n]$ and $\check\beta\colon[\check{n}_\beta]\to[n]$ of simplicial face operators in $\Del+$ by $\hat\beta(i) = i$ for all $i\in[\hat{n}_\beta]$ and $\check\beta(j)=j+\hat{n}_\beta + 1$ for all $j\in[\check{n}_\beta]$. 
    
    Now if $X$ and $Y$ are (terminally augmented) simplicial sets we may define a map $\bar{s}^{X,Y}$ which carries an $n$-simplex $(x,\beta,y)$ of $X\times \Del^1\times Y$ to the $n$-simplex $(x\cdot\hat\beta,y\cdot\check\beta)$ of $X\join Y$. A straightforward calculation, using the explicit description of the simplicial action on $X\join Y$ given at the end of recollection~\ref{rec:join-dec}, demonstrates that this map commutes with the simplicial actions on these sets and is thus a simplicial map. Furthermore, the family of simplicial maps $\bar{s}^{X,Y}\colon X\times \Del^1\times Y \to X\join Y$ is natural in $X$ and $Y$.

    Of course, since $X$ and $Y$ are terminally augmented, we also have canonical maps $l^{X,Y}\colon X\cong X\join\Del^{-1}\to X\join Y$ and $r^{X,Y}\colon Y\cong\Del^{-1}\join Y\to X\join Y$ and we may assemble all these maps together into a commutative square 
    \begin{equation}\label{eq:join-comp-def}
      \xymatrix@R=2em@C=4em{
        {({X}\times{Y})\sqcup({X}\times{Y})}\ar[r]^-{\pi_X\sqcup\pi_Y}
        \ar[d]_{\langle{X}\times\face^1\times{Y},
          {X}\times\face^0\times{Y}\rangle} &
        {{X}\sqcup{Y}}\ar[d]^{\langle l^{X,Y}, r^{X,Y}\rangle} \\
        {{X}\times\Del^1\times{Y}} \ar[r]_{\bar{s}^{X,Y}} &
        {{X}\join{Y}}
      }
    \end{equation}
    whose maps are all natural in $X$ and $Y$. Using the defining universal property of fat join, as given in~\eqref{eq:fat-join-def}, these squares induce maps $s^{X,Y}\colon X\fatjoin Y\to X\join Y$ which are again natural in $X$ and $Y$. Should we so wish, we may now take suitable coproducts of these maps to canonically extend this family of simplicial maps to a natural transformation between the extended fat join and join bifunctors on augmented simplicial sets.

    More explicitly, if $n,m\geq 0$, then $\bar{s}^{n,m}\colon\Del^n\times\Del^1\times\Del^m\to\Del^{n+m+1}$ is the unique simplicial map determined by the (order preserving) action on vertices given by:
    \begin{equation}\label{eq:tnm-def}
      \bar{s}^{n,m}(i,j,k) = 
      \begin{cases}
        i & \text{if $j=0$, and} \\
        k+n+1 & \text{if $j=1$.}
      \end{cases}
    \end{equation}
    This takes simplices related under the congruence defined in~\eqref{eq:fat-join-def} of definition~\ref{def:fat-join} to the same simplex and thus induces a unique map $s^{n,m}\colon\Del^n\fatjoin\Del^m\to\Del^n\join\Del^m$ on the quotient simplicial set.
  \end{obs}

Our immediate aim, achieved in proposition~\ref{prop:join-fatjoin-equiv}, is to show that the maps $s^{X,Y}$ are weak equivalences in the Joyal model structure for any pair of simplicial sets $X$ and $Y$. To that end,  we will use explicit cylinder objects for the model structure of naturally marked quasi-categories  to prove that certain maps are weak equivalences in the Joyal model structure. We first consider the na\"{i}ve choice and then use lemma~\ref{lem:pointwise-equiv} to construct the cylinder object which we will make use of below.%Dom, is this okay?

  \begin{obs}\label{obs:Kan-cylinder}  When $n=1$, the two specially marked $1$-horn inclusions are simply the inclusion maps $\face^1,\face^0\colon\Del^0\inc(\Del^1)^\sharp$; by recollection~\ref{rec:qmc-quasi-marked} these are trivial cofibrations in the model structure of naturally marked quasi-categories. So if $X$ is any marked simplicial set then the canonical ``end point'' inclusions $i_0,i_1\colon X\inc X\times(\Del^1)^\sharp$ are both trivial cofibrations, simply because they are isomorphic to the maps obtained by taking the product of the cofibrant object $X$ with the trivial cofibrations $\face^1,\face^0\colon\Del^0\inc(\Del^1)^\sharp$ in a cartesian model structure. It follows then that the maps $i_0,i_1\colon X\inc X\times(\Del^1)^\sharp$ and the projection map $\pi\colon X\times(\Del^1)^\sharp\to X$ display $X\times(\Del^1)^\sharp$ as a canonical cylinder object for $X$ in the model structure of naturally marked quasi-categories.
\end{obs}

    \begin{defn}\label{defn:Joyal-cylinder}
      For a marked simplicial set $X$, define $\cyl(X)$ to be the marked simplicial set derived from $X\times\Del^1$ by also marking any $1$-simplex of the form $(x\cdot\degen^0,\id_{[1]})$ for some $0$-simplex $x\in X$. Observe that  $\cyl(X)$ is a marked superset of $X\times\Del^1$ and a marked subset of $X\times(\Del^1)^\sharp$. Consequently the projection map $\pi\colon X\times(\Del^1)^\sharp\to X$ restricts to a map $\pi\colon \cyl(X)\to X$ and the ``end point'' inclusion maps $i_0,i_1\colon X\inc X\times\Del^1$ extend to maps $i_0,i_1\colon X\inc\cyl(X)$.
    \end{defn}

    \begin{lem}\label{lem:cyl-obj}
      The end point inclusion maps $i_0,i_1\colon X\inc\cyl(X)$ and the projection map $\pi\colon\cyl(X)\to X$ present $\cyl(X)$ as a cylinder object for $X$ in the model structure of naturally marked quasi-categories.
    \end{lem}

    \begin{proof}
      This is mostly immediate from the definitions above, but we do need to demonstrate that the map $\pi\colon\cyl(X)\to X$ is a weak equivalence in the model structure of naturally marked quasi-categories. Equivalently, by the 2-of-3 property for weak equivalences, we may show that either of the inclusions $i_0, i_1\colon X\inc\cyl(X)$ is a trivial cofibration in that model structure. In other words, we must show that $i_0$ possesses the left lifting property with respect to all isofibrations $p\colon A\to B$ of naturally marked quasi-categories. 
      
      The proof of this lifting result is illustrated in the following diagram:
      \begin{equation*}
      \xymatrix@C=10em@R=1.5em{
        {X}\ar[r]^u\ar@{u(->}[d]_-{i_0} & {A}\ar[d]^p \\
        {\cyl(X)}\ar[r]^-v 
        \save +<5em,-2em>*+{X\times(\Del^1)^\sharp}="one"\restore
        & {B}
        \ar@{u(->} "2,1";"one" \ar@{..>} "one";"2,2"_-{v'}
        \ar@{-->} "one";"1,2"^(0.6){l}
      }
      \end{equation*}
      Here we are asked to construct a lift in the square of maps whose horizontals are labelled $u$ and $v$ and whose verticals are $i_0$ and $p$. So we start by observing that the dotted extension $v'$ of $v$ along the inclusion $\cyl(X)\inc X\times(\Del^1)^\sharp$ exists by lemma~\ref{lem:pointwise-equiv} and the assumption that $B$ is a naturally marked quasi-category. As just observed, the inclusion $i_0\colon X\inc\cyl(X)\inc X\times(\Del^1)^\sharp$ is a trivial cofibration in the model structure of naturally marked quasi-categories. So the dashed lift $l$ exists, because $p$ is an isofibration by assumption, and we may restrict that map along the inclusion $\cyl(X)\inc X\times(\Del^1)^\sharp$ to give the required lift for our original square. 
    \end{proof}

The following combinatorial proposition is absolutely crucial in the proof that the join and fat join constructions are equivalent.

  \begin{prop}\label{prop:join-fatjoin-equiv-simplices}
    For all $n,m\geq 0$, the comparison simplicial map $s^{n,m}\colon\Del^n\fatjoin\Del^m\to\Del^n\join\Del^m$ is a deformation retraction in the Joyal model structure and in particular is a weak equivalence. Furthermore, if $n$ or $m$ is equal to $-1$ then $s^{n,m}$ is an isomorphism.
  \end{prop}

  \begin{proof}
    We start by thinking of the map $\bar{s}^{n,m}\colon \Del^n\times\Del^1\times\Del^m\to\Del^{n+m+1}$ in terms of its action on $0$-simplices. In other words, we regard it as being the order preserving map from $[n]\times[1]\times[m]$ to $[n+m+1]$ described in~\eqref{eq:tnm-def} above. Now we may define an order preserving map $\bar{t}^{n,m}\colon [n+m+1]\to [n]\times[1]\times[m]$ pointing in the opposite direction by 
    \begin{equation*}
      \bar{t}^{n,m}(i) =
      \begin{cases}
        (i,0,0) & \text{if $i\leq n$ and} \\
        (n,1,i-n-1) & \text{if $i>n$}
      \end{cases}
    \end{equation*}
    and comment immediately that $\bar{s}^{n,m}\circ \bar{t}^{n,m} = \id_{[n+m+1]}$. To compare the obverse composite $\bar{t}^{n,m}\circ \bar{s}^{n,m}$ with the identity on $[n]\times[1]\times[m]$, first observe that this particular composite is given by the following explicit formula:
    \begin{equation*}
      (\bar{t}^{n,m}\circ \bar{s}^{n,m})(i,j,k) = 
      \begin{cases}
        (i,0,0) & \text{if $j=0$ and} \\
        (n, 1, k) & \text{if $j=1$.}
      \end{cases}
    \end{equation*}
    Now we may define a related order preserving endo-map $\bar{u}^{n,m}$ on $[n]\times[1]\times[m]$ by
    \begin{equation*}
      \bar{u}^{n,m}(i,j,k) =
      \begin{cases}
        (i,0,0) & \text{if $j=0$ and} \\
        (i,1,k) & \text{if $j=1$}
      \end{cases}
    \end{equation*}
    and observe that in the pointwise ordering on such maps we have $\bar{u}^{n,m} \leq \bar{t}^{n,m}\circ \bar{s}^{n,m}$ and $\bar{u}^{n,m} \leq \id_{[n]\times[1]\times[m]}$. Taking nerves, the map $\bar{t}^{n,m}$ becomes a simplicial map from $\Del^{n+m+1}$ to $\Del^n\times\Del^1\times\Del^m$, the map $\bar{u}^{n,m}$ becomes a simplicial endo-map on $\Del^n\times\Del^1\times\Del^m$ and the inequalities of the last sentence become $1$-simplices      $\bar{h}^{n,m},\bar{k}^{n,m}$ in $(\Del^n\times\Del^1\times\Del^m)^{\Del^n\times\Del^1\times\Del^m}$
    \[\xymatrix{ \Del^n\times\Del^1\times\Del^m \ar[d]_{\face^1} \ar[dr]^{\bar{u}^{n,m}} &  \\ (\Del^n \times\Del^1\times\Del^m)\times\Del^1 \ar[r]^-{\bar{h}^{n,m}} & \Del^n\times\Del^1\times\Del^m  \\ \Del^n\times\Del^1\times\Del^m\ar[u]^{\face^0} \ar[ur]_*+{\labelstyle \bar{t}^{n,m}\circ\bar{s}^{n,m}} & } \xymatrix{ \Del^n\times\Del^1\times\Del^m\ar[d]_{\face^1} \ar[dr]^{\bar{u}^{n,m}} \\ (\Del^n\times\Del^1\times\Del^m)\times\Del^1\ar[r]^-{\bar{k}^{n,m}} & \Del^n\times\Del^1\times\Del^m\\   \Del^n\times\Del^1\times\Del^m \ar[u]^{\face^0} \ar[ur]_*+{\labelstyle \id_{\Del^n\times\Del^1\times\Del^m}}}  \]
which connect the $0$-simplices $\bar{u}^{n,m}$ to $\bar{t}^{n,m}\circ \bar{s}^{n,m}$ and $\id_{\Del^n\times\Del^1\times\Del^m}$ respectively. The composites $\bar{h}^{n,m}\circ(\bar{t}^{n,m}\times\Del^1), \bar{k}^{n,m}\circ(\bar{t}^{n,m}\times\Del^1)\colon \Del^{n+m+1}\times\Del^1\to\Del^n\times\Del^1\times\Del^m$ are both equal to the degenerate $1$-simplex derived from the $0$-simplex $\bar{t}^{n,m}$ in $(\Del^n\times\Del^1\times\Del^m)^{\Del^{n+m+1}}$.

    Passing to quotients under the congruence $\sim$ defined in~\eqref{eq:fat-join-cong}, it is easily verified that these maps induce simplicial maps $t^{n,m}\colon\Del^n\join\Del^m\to\Del^n\fatjoin\Del^m$, $u^{n,m}\colon\Del^n\fatjoin\Del^m\to\Del^n\fatjoin\Del^m$, and $h^{n,m}, k^{n,m}\colon(\Del^n\fatjoin\Del^m)\times\Del^1\to\Del^n\fatjoin\Del^m$ which also satisfy the algebraic identities discussed in the last paragraph. Now if we are to use this data to show that $s^{n,m}$ is a deformation retraction in Joyal's model structure, then we must show that the maps $h^{n,m}$ and $k^{n,m}$ give rise to homotopies between the map $u^{n,m}$ and the maps $t^{n,m}\circ s^{n,m}$ and $\id_{\Del^n\fatjoin\Del^m}$ respectively. To that end, lemma~\ref{lem:cyl-obj} tells us that it would be enough to show that the maps $h^{n,m}$ and $k^{n,m}$ extend along the inclusion $(\Del^n\fatjoin\Del^m)\times\Del^1\inc\cyl(\Del^n\fatjoin\Del^m)$. On consulting definition~\ref{defn:Joyal-cylinder}, we find that we must verify that for each $0$-simplex $[{i},{j},{k}]_\sim$ of $\Del^n\fatjoin\Del^m$ the $1$-simplex $([{i},{j},{k}]_\sim\cdot\degen^0,\id_{[1]})$ of $(\Del^n\fatjoin\Del^n)\times\Del^1$ is mapped by $h^{n,m}$ and $k^{n,m}$ to marked, and thus degenerate, simplices in $\Del^n\fatjoin\Del^m$. This, however, is a matter of routine verification, which we leave to the reader.
  \end{proof}
  
  
  
  We will extend this result to all simplicial sets presently, but first we shall need the following technical result. A review of the Reedy category theory necessary to understand its statement and proof, written in part for this purpose, can be found in \cite{RiehlVerity:2013kx}.
  
  \begin{lem}\label{lem:cofib-joins}
    If $i\colon X\inc Y$ and $j\colon U\inc V$ are both monomorphisms of terminally augmented simplicial sets, then so is their Leibniz join $(i\colon X\inc Y)\leib\join(j\colon U\inc V)$ and their Leibniz fat join $(i\colon X\inc Y)\leib\fatjoin(j\colon U\inc V)$. In particular, it follows that the latching maps of the associated functors $F_{\join},F_{\fatjoin}\colon \Del+\times\Del+\to\asSet$ given by $F_{\join}^{n,m} \defeq \Del^n\join\Del^m$ and $F_{\fatjoin}^{n,m} \defeq \Del^n\fatjoin\Del^m$ are  monomorphisms.
  \end{lem}
  
  \begin{proof}
    The explicit descriptions of the join and fat join given in recollection~\ref{rec:join-dec} and observation~\ref{def:fat-join} provide us with natural isomorphisms
    \begin{align*}
      (X\join U)_n &\cong X_n\sqcup \left(\coprod_{i=0,...,n-1} X_{n-i-1}\times U_i\right) \sqcup U_n \\
      (X\fatjoin U)_n &\cong X_n\sqcup (X_n\times D_n \times U_n) \sqcup U_n
    \end{align*}
    when $n\geq 0$ where $D_n$ denotes the set of those $n$-simplices of $\Del^1$ which are neither of the constant operators $0,1\colon[n]\to[1]$. Using these expressions, it is easy to verify that each map in the commutative squares
    \begin{equation*}
      \xymatrix@R=2em@C=3em{
        {(X\join U)_n}\ar@{u(->}[r]^{(X\join j)_n}\ar@{u(->}[d]_{(i\join U)_n} &
        {(X\join V)_n}\ar@{u(->}[d]^{(i\join V)_n} \\
        {(Y\join U)_n}\ar@{u(->}[r]_{(Y\join j)_n} & {(Y\join V)_n}
      }
      \mkern30mu
      \xymatrix@R=2em@C=3em{
        {(X\fatjoin U)_n}\ar@{u(->}[r]^{(X\fatjoin j)_n}\ar@{u(->}[d]_{(i\fatjoin U)_n} &
        {(X\fatjoin V)_n}\ar@{u(->}[d]^{(i\fatjoin V)_n} \\
        {(Y\fatjoin U)_n}\ar@{u(->}[r]_{(Y\fatjoin j)_n} & {(Y\fatjoin V)_n}
      }
    \end{equation*}
    is a monomorphism and that both squares are pullbacks in $\Set$. By the pasting property of such squares in $\Set$, the pushouts of the upper horizontal and left-hand vertical maps may be constructed as the joint images of their lower horizontal maps and their right hand vertical maps within the sets in their lower right hand corners. However, since the pushouts of $\sSet$ are constructed pointwise in $\Set$, it follows that the Leibniz join $i \leib\join j$ and Leibniz fat join $i \leib\fatjoin j$, which are induced out of these particular pushouts by these squares, may be written as inclusions of simplicial subsets into the simplicial sets $Y\join V$ and $Y\fatjoin V$, and hence are monomorphisms.
    
        Now on consulting observation~\ref*{reedy:obs:weights-latching} and example~\ref*{reedy:ex:boundary-prod} in \cite{RiehlVerity:2013kx}, we see that the latching maps of the functors $F_{\join}$ and $F_{\fatjoin}$ at $([n],[m])\in\Del+\times\Del+$ may be expressed in terms of the weighted colimit formulae:
    \begin{equation}\label{eq:latching-exprs-1}
      \begin{aligned}
        L^{n,m} F_{\join} &\cong ((\boundary\Del^n\inc\Del^n)\leib\etimes(\boundary\Del^m\inc\Del^m)) \wcolim_{\Del+\times\Del+} F_{\join} \mkern10mu\text{and}\\
        L^{n,m} F_{\fatjoin} &\cong ((\boundary\Del^n\inc\Del^n)\leib\etimes(\boundary\Del^m\inc\Del^m)) \wcolim_{\Del+\times\Del+} F_{\fatjoin} 
      \end{aligned}
    \end{equation}
    Here we are using $\etimes$ to denote the exterior product; cf.~\cite[\ref*{reedy:obs:box-product}]{RiehlVerity:2013kx}.
   Now, a routine application of Yoneda's lemma, in the form given in \cite[example~\ref{reedy:ex:weighted-yoneda}]{RiehlVerity:2013kx}, and a calculation using the fact that the join and fat join operations are cocontinuous in each variable (as bifunctors on the category of augmented simplicial sets) reveals that we have canonical isomorphisms: 
    \begin{align}
      X\join U &\cong \int^{[n],[m]\in\Del+} (X_n\times U_m)\tns(\Del^n\join\Del^m) \cong (X\etimes U)\wcolim_{\Del+\times\Del+} F_{\join} \mkern10mu\text{and} \notag  \\
      X\fatjoin U &\cong \int^{[n],[m]\in\Del+} (X_n\times U_m)\tns(\Del^n\fatjoin\Del^m) \cong (X\etimes U)\wcolim_{\Del+\times\Del+} F_{\fatjoin}  \label{eq:fatjoinformulae}
    \end{align}
    These pass to isomorphisms of the corresponding Leibniz operations and we may then apply them to show that the expressions \eqref{eq:latching-exprs-1} reduce to:
    \begin{equation*}
      L^{n,m} F_{\join} \cong (\boundary\Del^n\inc\Del^n)\leib\join(\boundary\Del^m\inc\Del^m) \mkern10mu\text{and}\mkern10mu
      L^{n,m} F_{\fatjoin} \cong (\boundary\Del^n\inc\Del^n)\leib\fatjoin(\boundary\Del^m\inc\Del^m)
    \end{equation*}
    Now apply the result established in the first part of the lemma to conclude that these latching maps are monomorphisms as stated.
  \end{proof}

  \begin{prop}\label{prop:join-fatjoin-equiv}
    For all simplicial sets $X$ and $Y$, the  map $s^{X,Y}\colon X\fatjoin Y\to X\join Y$ is a weak equivalence in the Joyal model structure.
  \end{prop}

  \begin{proof}
    From \eqref{eq:fatjoinformulae} we know that $X\join Y$ and $X\fatjoin Y$ are naturally isomorphic to the colimits of the functors $F_{\join}, F_{\fatjoin}\colon\Del+\times\Del+\to\sSet$ weighted by $X\etimes Y$. Furthermore, the natural transformation $s^{X,Y}\colon X\fatjoin Y\to X\join Y$ restricts to give a natural transformation $s\colon F_{\fatjoin}\to F_{\join}$ and the naturality of $s^{X,Y}$ ensures that the canonical isomorphisms fit into a commutative square: 
    \begin{equation*}
      \xymatrix@=2em{
        {X\fatjoin Y} \ar@{}[r]|-{\textstyle\cong}\ar[d]_{s^{X,Y}} &
        {(X\etimes Y)\wcolim_{\Del+\times\Del+} F_{\fatjoin}}
        \ar[d]^{(X\etimes Y)\wcolim_{\Del+\times\Del+} s} \\
        {X\join Y} \ar@{}[r]|-{\textstyle\cong} &
        {(X\etimes Y)\wcolim_{\Del+\times\Del+} F_{\join}}
      }
    \end{equation*}
    When $\sSet$ carries the Joyal model structure, lemma~\ref{lem:cofib-joins} asserts that $F_{\join}$ and $F_{\fatjoin}$ are cofibrant in the corresponding Reedy model structure on $\sSet^{\Del+\times\Del+}$, and proposition~\ref{prop:join-fatjoin-equiv-simplices} tells us that $s\colon F_{\fatjoin}\to F_{\join}$ is a pointwise weak equivalence. 
    
    Now the Eilenberg-Zilber lemma (cf.~\cite[II.3.1, pp.~26-27]{GabrielZisman:1967:CFHT}) implies that the latching maps of any (augmented) double simplicial set are monomorphisms. In particular, $X \etimes Y$ is Reedy cofibrant, and we may apply \cite[proposition~\ref*{reedy:prop:2/3-SM7}]{RiehlVerity:2013kx} to show that the functor $(X\etimes Y)\wcolim_{\Del+\times\Del+}{-}$ is a left Quillen functor. Consequently, Ken Brown's lemma (see \cite[1.1.12]{Hovey:1999fk} for example) now applies to show that $(X\etimes Y)\wcolim_{\Del+\times\Del+}{-}$ carries the pointwise weak equivalence $s\colon F_{\fatjoin}\to F_{\join}$ of Reedy cofibrant objects to a weak equivalence in $\sSet$. However, the commutative square above tells us that this latter map is isomorphic to $s^{X,Y}\colon X\fatjoin Y\to X\join Y$ which is thus also a weak equivalence as postulated.
  \end{proof}
  
  \begin{lem}\label{lem:slices-quillen}
 For any simplicial set $X$, the slice and fat slice adjunctions
    \begin{align*}
  		\adjdisplay \textstyle X\bar\join{-}-|\textstyle \slc^X_l:X\slice\sSet->\sSet. & 
      \mkern20mu & 
  		\adjdisplay \textstyle {-}\bar\join X-|\textstyle \slc^X_r:X\slice\sSet->\sSet. \\
  		\adjdisplay \textstyle X\bar\fatjoin{-}-|\textstyle \fatslc^X_l:X\slice\sSet->\sSet. & 
      \mkern20mu & 
  		\adjdisplay \textstyle {-}\bar\fatjoin X-|\textstyle \fatslc^X_r:X\slice\sSet->\sSet.
    \end{align*}
    of definitions~\ref{defn:slices} and~\ref{defn:fat-slices} are Quillen adjunctions with respect to the Joyal model structure on $\sSet$ and the corresponding sliced model structure on $X\slice\sSet$.
  \end{lem}
  
  \begin{proof}
    By~\cite[7.15]{Joyal:2007kk} it is enough to check that in each of these adjunctions the left adjoint preserves cofibrations and the right adjoint preserves fibrations between fibrant objects. Preservation of cofibrations by these left adjoints follows immediately from lemma~\ref{lem:cofib-joins}, since in the Joyal model structure they are simply the monomorphisms of simplicial sets. Preservation of fibrations of fibrant objects by these right adjoints follows immediately from observations~\ref{obs:slice-and-qcats} and~\ref{obs:fat-cone-quasicat}: if $p \colon A \to B$ is an isofibration of quasi-categories and $f \colon X \to A$ is any simplicial map, then the induced simplicial maps $\slc^X_r(p)\colon \slicer{A}{f}\to \slicer{B}{pf}$ and $\slc^X_l(p)\colon \slicel{f}{A}\to \slicel{pf}{B}$ are also isofibrations of quasi-categories and similarly for fat slices.
  \end{proof}
  
  Finally, we arrive at the advertised comparison result relating the slice and fat slice constructions.
  
  \begin{prop}[slices and fat slices of a quasi-category are equivalent]\label{prop:slice-fatslice-equiv}
    Suppose that $X$ is any simplicial set, that $\sSet$ carries the Joyal model structure, and that $X\slice\sSet$ carries the associated sliced model structure. Then the comparison maps $s^{X,Y}\colon X\fatjoin Y\to X\join Y$ furnish us with natural transformations $s^{X,{-}}\colon X\bar\fatjoin{-}\to X\bar\join{-}$ and $s^{{-},X}\colon {-}\bar\fatjoin X\to {-}\bar\join X$ which are pointwise weak equivalences. The components $e_l^f\colon \slicel{f}{A}\to \fatslicel{f}{A} \cong f\comma c$ and $e_r^f\colon \slicer{A}{f}\to \fatslicer{A}{f}\cong c\comma f$ at an object $f\colon X\to A$ of $X\slice\sSet$ of the induced natural transformations on right adjoints  are equivalences of quasi-categories whenever $A$ is a quasi-category.
  \end{prop}
  
  \begin{proof}
    The assertions involving left adjoints are the content of Proposition~\ref{prop:join-fatjoin-equiv}. The result described in the last sentence of the statement then follows from from the fact that, by lemma~\ref{lem:slices-quillen}, all of these adjunctions are Quillen adjunctions. Specifically, that fact allows us to apply the standard result in model category theory~\cite[1.4.4]{Hovey:1999fk} that a natural transformation between left Quillen functors has components which are weak equivalences at each cofibrant object (which fact we have already established) if and only if the  induced natural transformation between the corresponding right Quillen functors has components which are weak equivalences at each fibrant object. Now simply observe that an object $f\colon X\to A$ is fibrant in $X\slice\sSet$ if and only if $A$ is a quasi-category.
  \end{proof}
  
  \begin{rmk}\label{rmk:map-slices}
    Suppose that $f\colon B\to A$ and $g\colon C\to A$ are two simplicial maps. We generalise our slice and fat slice notation by using $\slicer{g}{f}$, $\fatslicer{g}{f}$, $\slicel{f}{g}$ and $\fatslicel{f}{g}$ to denote the objects constructed in the following pullback diagrams
    \begin{equation}
      \xymatrix@=2em{
        {\slicer{g}{f}} \pbexcursion\ar[r]\ar[d] & {\slicer{A}{f}}\ar[d]^\pi \\
        {C}\ar[r]_g & {A}
      }
      \mkern50mu
      \xymatrix@=2em{
        {\fatslicer{g}{f}} \pbexcursion\ar[r]\ar[d] & {\fatslicer{A}{f}}\ar[d]^\pi \\
        {C}\ar[r]_g & {A}
      }
      \mkern50mu
      \xymatrix@=2em{
        {\slicel{f}{g}} \pbexcursion\ar[r]\ar[d] & {\slicel{f}{A}}\ar[d]^\pi \\
        {C}\ar[r]_g & {A}
      }
      \mkern50mu
      \xymatrix@=2em{
        {\fatslicel{f}{g}} \pbexcursion\ar[r]\ar[d] & {\fatslicel{f}{A}}\ar[d]^\pi \\
        {C}\ar[r]_g & {A}
      }
    \end{equation}
    in which the maps labelled $\pi$ denote the various canonical projection maps. We call these the slices and fat slices of $g$ over and under $f$ respectively. 
     We have isomorphisms $\slicer{g\op}{f\op} \cong (\slicel{f}{g})\op$ and $\fatslicer{g\op}{f\op} \cong (\fatslicel{f}{g})\op$. Similarly $g\op\downarrow f\op \cong (f\downarrow g)\op$.    The canonical isomorphisms of proposition \ref{prop:fatsliceisfatcone} may  be pulled back to provide us with canonical isomorphisms $\fatslicer{g}{f}\cong cg\comma f$ and $\fatslicel{f}{g}\cong f\comma cg$. 
    
    When $A$ is a quasi-cat\-e\-go\-ry the projection maps $\pi$ are all isofibrations that commute with the comparison equivalences $e_l^f\colon \slicel{f}{A}\to \fatslicel{f}{A}$ and $e_r^f\colon \slicer{A}{f}\to \fatslicer{A}{f}$ of proposition \ref{prop:join-fatjoin-equiv}. In other words, these comparisons are fibred equivalences (cf.~definition~\ref{defn:fibred-equivalence}), which pull back to define equivalences $e_l^f\colon \slicel{f}{g}\to \fatslicel{f}{g}$ and $e_r^f\colon \slicer{g}{f}\to \fatslicer{g}{f}$ between slices of the map $g$ under and over $f$. (Alternatively, these maps are equivalences  of fibrant objects in the sliced Joyal model structure on $\sSet\slice A$. Pullback along any map in a model category is always a right Quillen functor of sliced model structures, so Ken Brown's lemma tells us that the pullbacks are again equivalences.)
  \end{rmk}


\input{../common/footer}
