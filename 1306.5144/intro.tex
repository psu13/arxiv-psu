%!TEX root = all.tex
% ******************************************************************
% ** Title:            The 2-category theory of quasi-categories
% **                  introduction
% ** Precis:        
% ** Author:           Emily Riehl and Dominic Verity
% ** Commenced:        2/3/2012
% ******************************************************************


\section{Introduction}

Quasi-categories, also called $\infty$-categories, were introduced by J.~Michael Boardman and Rainer Vogt under the name ``weak Kan complexes'' in their book \cite{Boardman:1973xo}. Their aim was to describe the weak composition structure enjoyed by homotopy coherent natural transformations between homotopy coherent diagrams. Other examples of quasi-categories include ordinary categories (via the nerve functor) and topological spaces (via the total singular complex functor), which are Kan complexes: quasi-categories in which every 1-morphism is invertible. Topological and simplicial (model) categories also have associated quasi-categories (via the homotopy coherent nerve). Quasi-categories provide a convenient model for $(\infty,1)$-categories: categories weakly enriched in $\infty$-groupoids or topological spaces. Following the program of Boardman and Vogt, many homotopy coherent structures naturally organise themselves into a quasi-category.

For this reason, it is desirable to extend the definitions and theorems of ordinary category theory into the $(\infty,1)$-categorical and specifically into the quasi-categorical context. As categories form a full subcategory of quasi-categories, a principle guiding the quasi-categorical definitions is that these should restrict to the classically understood categorical concepts on this full subcategory. In this way, we think of quasi-category theory as an extension of category theory---and indeed use the same notion for a category and the quasi-category formed by its nerve.

There has been significant work (particularly if measured by page count) towards the development of the category theory of quasi-categories, the most well-known being the articles and unpublished manuscripts of Andr\'{e} Joyal \cite{Joyal:2002:QuasiCategories,Joyal:2007kk,Joyal:2008tq} and the books of Jacob Lurie \cite{Lurie:2009fk,Lurie:2012uq}. Other early work includes the PhD thesis of Joshua Nichols-Barrer \cite{NicholsBarrer:2007oq}.  More recent foundational developments are contained in work of David Gepner and Rune Haugseng \cite{GepnerHaugseng:2013ec}, partially joint with Thomas Nikolaus \cite{GepnerHaugsengNikolaus:2015lc}. Applications of quasi-category theory, for instance to derived algebraic geometry, are already too numerous to mention individually.

Our project is to provide a second generation, {\em formal\/} category theory of quasi-categories, developed from the ground up. Each definition given here is equivalent to the established one, 
but we find our development to be more intuitive and the proofs to be simpler. Our hope is that this self-contained account will be more approachable to the outsider hoping to better understand the foundations of the quasi-category theory he or she may wish to use.

In this paper, we use 2-category theory to develop the category theory of quasi-categories.  The starting point is a (strict) 2-category of quasi-categories $\qCat_2$ defined as a quotient of the simplicially enriched category of quasi-categories $\qCat_\infty$.  The underlying category of both enriched categories is the usual category of quasi-categories and simplicial maps, here called simply ``functors''. We translate simplicial universal properties into 2-categorical ones: for instance, the simplicially enriched universal properties of finite products and the hom-spaces between quasi-categories imply that the 2-category $\qCat_2$ is cartesian closed. Importantly, equivalences in the 2-category $\qCat_2$ are precisely the (weak) equivalences of quasi-categories introduced by Joyal, which means that this 2-category appropriately captures the homotopy theory of quasi-categories.

Aside from finite products, $\qCat_2$ admits few strict 2-limits. However, it admits several important weak 2-limits of a sufficiently strict variety with which to develop formal category theory. Weak 2-limits in $\qCat_2$ are not unique up to isomorphism; rather their universal properties characterise these objects up to equivalence, exactly as one would expect in the $(\infty,1)$-categorical context. We show that $\qCat_2$ admits weak cotensors by categories freely generated by a graph (including, in particular, the walking arrow) and weak comma objects, which we use  to encode the universal properties associated to limits, colimits, adjunctions, and so forth.

A complementary paper \cite{RiehlVerity:2012hc} will showcase a corresponding ``internal'' approach to this theory. The basic observation is that the simplicial category of quasi-categories $\qCat_\infty$ is closed under the formation of weighted limits whose weights are projectively cofibrant simplicial functors. Examples include Bousfield-Kan style homotopy limits and a variety of weighted limits relating to homotopy coherent adjunctions.

In \cite{RiehlVerity:2012hc}, we show that any adjunction of quasi-categories can be extended to a {\em homotopy coherent adjunction}, by which we mean a simplicial functor whose domain is a particular cofibrant simplicial category that we describe in great detail. Unlike previous renditions of coherent adjunction data, our formulation is symmetric: in particular, a homotopy coherent adjunction restricts to a homotopy coherent monad and to a homotopy coherent comonad on the two quasi-categories under consideration. As a consequence of its cofibrancy, various weights extracted from the free homotopy coherent adjunction are projectively cofibrant simplicial functors. We use these to define the quasi-category of algebras associated to a homotopy coherent monad and provide a formal proof of the monadicity theorem of Jon Beck. More details can be found there.

\subsection{A generalisation}

In hopes that our proofs would be  more readily absorbed in familiar language, we have neglected to state our results in their most general setting, referencing only the simplicially enriched full subcategory of quasi-categories $\qCat_\infty$. Nonetheless, a key motivation for our project is that our proofs apply to more general settings which are also of interest.  

Consider a Quillen model category that is enriched as a model category relative to the Joyal model structure on simplicial sets and in which every fibrant object is also cofibrant. Then its full simplicial subcategory of fibrant objects is what we call an $\infty$-\emph{cosmos}; a simple list of axioms, weaker than the model category axioms, will be described in a future paper. Weak equivalences and fibrations between fibrant objects will play the role of the equivalences and isofibrations here. Examples of Quillen model categories which satisfy these conditions include Joyal's model category of quasi-categories and any model category of complete Segal spaces in a suitably well behaved model category. The canonical example \cite{Joyal:2007kk,Rezk:2001sf} is certainly included under this heading but we have in mind more general ``Rezk spaces'' as well. Given a well-behaved model category $\lcat{M}$, the localisation of the Reedy model structure on the category $\lcat{M}^{\Del\op}$ whose fibrant objects are complete Segal objects is enriched as a model category over the Joyal model structure on simplicial sets. All of the definitions that are stated and theorems that are proven here apply representably to any $\infty$-cosmos, being a simplicial category whose hom-spaces are quasi-categories and whose quotient 2-category admits the same weak 2-limits utilised here.


\subsection{Outline}

Our approach to the foundations of quasi-category theory is independent of the existing developments with one exception: we accept as previously proven the Joyal model structure for quasi-categories on simplicial sets and the model structure for naturally marked quasi-categories on marked simplicial sets. So that a reader can begin his or her acquaintance with the subject by reading this paper, we begin with a comprehensive background review in section \ref{sec:background}, where we also establish our notational conventions. 

In section \ref{sec:twocat}, we introduce the 2-category of quasi-categories $\qCat_2$ and investigate its basic properties. Of primary importance is the particular notion of weak 2-limit introduced here. Following \cite{Kelly:1989fk}, a strict 2-limit can be defined representably: the hom-categories mapping into the 2-limit are required to be naturally isomorphic to the corresponding 2-limit of hom-categories formed in $\Cat$. In our context, there is a canonical functor from the former category to the latter but it is not an isomorphism. Rather it is what we term a {\em smothering functor}: surjective on objects, full, and conservative. We develop the basic theory of these weak 2-limits and prove that $\qCat_2$ admits certain weak cotensors, weak 2-pullbacks, and weak comma objects.

In section \ref{sec:qcatadj}, we begin to develop the formal category theory of quasi-categories by introducing adjunctions between quasi-categories, which are defined simply to be adjunctions in the 2-category $\qCat_2$; this definition was first considered by Joyal. It follows immediately that adjunctions are preserved by pre- and post-composition, since these define 2-functors on $\qCat_2$. Any equivalence of quasi-categories extends to an adjoint equivalence, and that any adjunction between Kan complexes is automatically an adjoint equivalence. We describe an alternate form of the universal property of an adjunction which will be a key ingredient in the proof of the main existence theorem of \cite{RiehlVerity:2012hc}. Finally, we show that many of our adjunctions are in fact {\em fibred}, meaning that they are also adjunctions in the 2-category obtained as a quotient of the simplicial category of isofibrations over a fixed quasi-category. Any map between the base quasi-categories defines a pullback 2-functor, which then preserves fibred equivalences, fibred adjunctions, and so forth.

In section \ref{sec:limits}, we define limits and colimits in a quasi-category in terms of absolute right and left lifting diagrams in $\qCat_2$. A key technical theorem provides an equivalent definition as a fibred equivalence of comma quasi-categories. We prove the expected results relating limits and colimits to adjunctions: that right adjoints preserve limits, that limits of a fixed shape can be encoded as adjoints to constant diagram functors provided these exist, that limits and limit cones assemble into right Kan extensions along the join functor, and so on. As an application of these general results, we give a quick proof that any quasi-category admitting pullbacks, pushouts, and a zero object has a ``loops--suspension'' adjunction. This forms the basis for the notion of a {\em stable\/} quasi-category.

We conclude section \ref{sec:limits} with an example particularly well suited to our 2-categorical approach that will reappear in the proof of the monadicity theorem in \cite{RiehlVerity:2012hc}: generalising a classical result from simplicial homotopy theory, we show that if a simplicial object in a quasi-category admits an augmentation and ``extra degeneracies'', then the augmentation is its quasi-categorical colimit and also encodes the canonical colimit cone. Our proof is entirely 2-categorical. There exists an absolute left extension diagram in $\Cat$ involving $\Del$ and related categories and furthermore this 2-universal property is witnessed equationally by various adjunctions. Such universal properties are preserved by any 2-functor---for instance, homming into a quasi-category---and the result follows immediately.

Having established the importance of absolute lifting diagrams, which characterise limits, colimits, and adjunctions in the quasi-categorical context, it is important to develop tools which can be used to show that such diagrams exist in $\qCat_2$. This is the aim of section \ref{sec:pointwise}. In this section,  we show that a cospan $B \xrightarrow{f} A \xleftarrow{g} C$ admits an absolute right lifting of $g$ along $f$ if and only if for each object $c \in C$, the slice (or comma) quasi-category from $f$ down to $gc$ has a terminal object. In practice, this ``pointwise'' universal property is much easier to check than the global one encoded by the absolute lifting diagram. 

To illustrate, we use this theorem to show that any simplicial Quillen adjunction between simplicial model categories defines an adjunction of quasi-categories. The proof of this result is more subtle than one might suppose. The quasi-category associated to a simplicial model category is defined by applying the homotopy coherent nerve to the subcategory of fibrant-cofibrant objects---in general, the mapping spaces between arbitrary objects need not have the ``correct'' homotopy type. On account of this restriction, the point-set level left and right adjoints do not directly descend to functors between these quasi-categories so the quasi-categorical adjunction must be defined in some other way.

We conclude this paper with a technical appendix proving that the comma quasi-categories used here are equivalent to the slice quasi-categories introduced by Joyal \cite{Joyal:2002:QuasiCategories}. It follows that the categorical definitions introduced in this paper coincide with the definitions found in the existing literature.

\subsection{Acknowledgments}

During the preparation of this work the authors were supported by DARPA through the AFOSR grant number HR0011-10-1-0054-DOD35CAP and by the Australian Research Council through Discovery grant number DP1094883. The first-named author was also supported by an NSF postdoctoral research fellowship DMS-1103790. A careful reading by an anonymous referee has lead to numerous improvements in the exposition throughout this article. We would also like to extend personal thanks to Mike Hopkins without whose support and encouragement this work would not exist.


