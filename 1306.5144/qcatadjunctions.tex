%!TEX root = all.tex
% ******************************************************************
% ** Title:            The 2-category theory of quasi-categories
% **                   adjunctions
% ** Precis:        
% ** Author:           Emily Riehl and Dominic Verity
% ** Commenced:        2/3/2012
% ******************************************************************

\section{Adjunctions of quasi-categories}\label{sec:qcatadj}

We begin our 2-categorical development of quasi-category theory by introducing the appropriate notion of adjunction, following Joyal. As observed in \cite{kelly.street:2} and elsewhere, adjunctions can be defined internally to any 2-category and the proofs of many of their familiar properties can be internalised similarly.

\setcounter{thm}{0}
\begin{defn}[adjunction]\label{defn:adjunction}
An {\em adjunction\/}  \[ \adjdisplay f-| u : A ->B .\] in a 2-category consists of objects $A,B$; 1-cells $f \colon B \to A$, $u \colon A \to B$; and {\em unit\/} and {\em counit\/} 2-cells $\eta \colon \id_B \Rightarrow uf$, $\epsilon \colon fu \Rightarrow \id_A$ satisfying the triangle identities.
\[\xymatrix@=1.5em{ & B \ar[dr]^f \ar@{}[d]|(.6){\Downarrow\epsilon} \ar@{=}[rr] &  \ar@{}[d]|(.4){\Downarrow\eta} & B \ar@{}[d]^*+{=} & B &&   B \ar@{=}[rr] \ar[dr]_f & \ar@{}[d]|(.4){\Downarrow \eta} & B \ar[dr]^f \ar@{}[d]|(.6){\Downarrow\epsilon} & {\mkern40mu}\ar@{}[d]^*+{=} &  B \ar@/^2ex/[d]^f \ar@/_2ex/[d]_f \ar@{}[d]|(.4){\id_f}|(.6){=} & \\A \ar[ur]^u \ar@{=}[rr] & &  A \ar[ur]_u & {\mkern40mu} & A \ar@/^2ex/[u]^u \ar@/_2ex/[u]_u \ar@{}[u]|(.4){=}|(.6){\id_u}  && &A \ar[ur]_u \ar@{=}[rr] & & A & A }\]
\end{defn}

In particular, an \emph{adjunction between quasi-categories} is an adjunction in the 2-category $\qCat_2$.  As always we identify the unit and counit 2-cells with the simplicial maps \[ \xymatrix@=1.5em{ B \ar[d]_{i_0} \ar@{=}[dr] & & &  A \ar[r]^u \ar[d]_{i_0} & B \ar[d]^f  \\ B \times \Del^1 \ar[r]_-{\eta} & B & \mathrm{and} &  A \times \Del^1 \ar[r]^-{\epsilon} & A \\ B \ar[u]^{i_1} \ar[r]_f & A \ar[u]_u & &  A \ar[u]^{i_1} \ar@{=}[ur]}\] (1-simplices in $B^B$ and $A^A$ respectively) representing the unit and counit respectively. Because $B^A$ and $A^B$ are quasi-categories we know, from the description of the homotopy category of a quasi-category given in Recollection~\ref{rec:hty-category}, that for any choice of representatives of the unit and counit there exist maps \[ \alpha \colon A \times \Del^2 \to B \qquad \mathrm{and} \qquad \beta \colon B \times \Del^2 \to A\] (2-simplices in $B^A$ and $A^B$ respectively) which witness the triangle identities in the  sense that their boundaries have the form 
 \[ \xymatrix@=1em{ & ufu \ar@{}[d]|(.6){\alpha} \ar[dr]^{u\epsilon} & & & fuf \ar@{}[d]|(.6){\beta} \ar[dr]^{\epsilon f} \\ u \ar[ur]^{\eta u} \ar[rr]_{\id_u} & & u & f \ar[ur]^{f\eta} \ar[rr]_{\id_f} & & f}\]

\begin{ex}
On account of the fully-faithful inclusion $\Cat_2 \inc \qCat_2$, any adjunction of categories gives rise to an adjunction of quasi-categories with canonical representatives for the unit and counit. Conversely, the 2-functor $\ho \colon \qCat_2 \to \Cat_2$ carries any adjunction of quasi-categories to an adjunction between their respective homotopy categories.
\end{ex}

\begin{ex} The homotopy coherent nerve, introduced in \cite{Cordier:1982:HtyCoh} and studied in \cite{Cordier:1986:HtyCoh}, defines a 2-functor from the 2-category of topologically enriched categories, continuous functors, and enriched natural transformations to $\qCat_2$. This 2-functor factors through the 2-category of locally Kan simplicial categories, simplicial functors, and simplicial natural transformations; the locally Kan simplicial categories are the cofibrant objects in Berger's model structure \cite{Bergner:2007fk}.  Hence, any enriched adjunction between topological or fibrant simplicial categories gives rise to an adjunction of quasi-categories by passing to homotopy coherent nerves. As in the unenriched case, there exist canonical representatives for the unit and counit defined by applying the homotopy coherent nerve to the corresponding enriched natural transformations.
\end{ex}

\begin{ex}\label{ex:simp.quillen.adj}
Any simplicially enriched Quillen adjunction between simplicial model categories descends to an adjunction between the associated quasi-categories, constructed by restricting to the fibrant-cofibrant objects and then applying the homotopy coherent nerve. This restriction is necessary to define the quasi-category associated to a simplicial model category; the homotopy coherent nerve of a simplicial category might not be a quasi-category if the simplicial category is not locally Kan. The subcategory of fibrant-cofibrant objects of a simplicial model category is locally Kan, and furthermore the hom-space bifunctor preserves weak equivalences in both variables; it is common to say that only between fibrant-cofibrant objects are the simplicial hom-spaces guaranteed to have the ``correct'' homotopy type. 

In contrast with the topological case, some care is required to define the functors constituting the adjunction; the point-set level functors will not do because neither adjoint need land directly in the fibrant-cofibrant objects. We prove that  a simplicial Quillen adjunction descends to an adjunction of quasi-categories in Theorem~\ref{thm:simplicial-Quillen-adjunction}.
\end{ex}

Adjunctions can also be constructed internally to $\qCat_2$ using its weak 2-limits, as we shall see in the next section. Later, we will also meet adjunctions constructions using limits or colimits defined internally to a quasi-category.

\subsection{Right adjoint right inverse adjunctions}\label{subsec:RARI} 

We begin by studying an important class of adjunctions whose counit 2-cells are isomorphisms.

\begin{defn}\label{defn:RARI}
A 1-cell $f \colon B \to A$ in a 2-category admits a \emph{right adjoint right inverse} (abbreviated \emph{RARI}) if it admits a right adjoint $u \colon A \to B$ so that the counit of the adjunction $f \dashv u$ is an isomorphism.
\end{defn}

In the situation of Definition \ref{defn:RARI}, $f$ defines a \emph{left adjoint left inverse} (abbreviated \emph{LALI}) to $u$. When the counit of $f \dashv u$ is an isomorphism, the whiskered composites $f\eta$ and $\eta u$ of the unit must also be isomorphisms. Indeed, to construct an adjunction of this form it suffices to give 2-cells with these properties, as demonstrated by the following 2-categorical lemma. 

\begin{lem}\label{lem:adjunction.from.isos} Suppose we are given a pair of 1-cells $u \colon A \to B$ and $f\colon B\to A$ and a 2-isomorphism $fu \cong \id_A$ in a 2-category. If there exists a 2-cell $\eta' \colon \id_B \Rightarrow uf$ with the property that $f\eta'$ and $\eta' u$ are 2-isomorphisms, then $f$ is left adjoint to $u$. Furthermore, in the special case where $u$ is a section of $f$, then $f$ is left adjoint to $u$ with the counit of the adjunction an identity.
\end{lem}
\begin{proof} 
Let $\epsilon \colon fu \To \id_A$ be the isomorphism, taken to be the identity in the case where $u$ is a section of $f$. We will define an adjunction $f \dashv u$ with counit $\epsilon$ by modifying $\eta' \colon \id_B \To uf$.  The ``triangle identity composite'' $\theta\defeq u\epsilon \cdot \eta' u \colon u \To u$ defines an automorphism of $u$.  Define 
\[ \eta \defeq \xymatrix{ \id_B \ar@{=>}[r]^-{\eta'} & uf \ar@{=>}[r]^-{\theta^{-1}f} & uf.}\] Immediately, $u\epsilon \cdot \eta u = \id_u$, as is verified by the calculation: 
\begin{equation}\label{eq:triangle-calculation-1} \xymatrix@R=1.2em@C=2.5em{ u \ar@{=>}[dr]_\theta \ar@{=>}[r]^-{\eta' u} & ufu \ar@{=>}[r]^{\theta^{-1} fu} \ar@{=>}[d]^{u \epsilon} & ufu \ar@{=>}[d]^{u \epsilon} \\ & u \ar@{=>}[r]_{\theta^{-1}} & u}\end{equation} 

The other triangle identity composite $\phi\defeq \epsilon f \cdot f \eta $ is an isomorphism, as a composite of isomorphisms, and also an idempotent:
\begin{equation}\label{eq:triangle-calculation-2} \xymatrix@R=1.2em@C=2.5em{ f \ar@{=>}[d]_{f \eta} \ar@{=>}[r]^-{f \eta} & fuf \ar@{=}[dr] \ar@{=>}[d]_{f\eta uf} \\ fuf \ar@{=>}[d]_{\epsilon f} \ar@{=>}[r]_-{fuf\eta} & fufuf \ar@{=>}[d]^{\epsilon f} \ar@{=>}[r]_-{fu\epsilon f} & fuf \ar@{=>}[d]^{\epsilon f} \\ f \ar@{=>}[r]_-{f\eta} & fuf \ar@{=>}[r]_-{\epsilon f} & f}\end{equation} But any idempotent isomorphism is an identity: the isomorphism $\phi$ can be cancelled from both sides of the idempotent equation $\phi \cdot \phi = \phi$. Hence, $\epsilon f \cdot f \eta = \id_f$, proving the second triangle identity.
\end{proof}

\begin{rmk}[idempotent isomorphisms]\label{rmk:idempotent-isomorphisms} Because $\qCat_2$ has many weak but few strict 2-limits, it is frequently easier to show that a 2-cell is an isomorphism than to show that it is an identity. When we desire an identity and not merely an isomorphism,  we will make frequent use of the trick that any idempotent isomorphism is an identity.
\end{rmk}

We now show that for any functor $\ell \colon C \to B$, the codomain projection functor $\pi_1 \colon B \comma \ell \to C$  admits a right adjoint right inverse, the ``identity functor'' $i \colon C \to B \comma \ell$ defined below. Here the right adjoint $i$ defines a section to the left adjoint $p_i$. Taking the counit of $i \dashv \pi_1$ to be an identity, as permitted by Lemma \ref{lem:adjunction.from.isos}, the adjunction lifts to the slice 2-category $\qCat_2/C$.


\begin{lem}\label{lem:technicalsliceadjunction}
  Suppose that $\ell\colon C\to B$ is a functor of quasi-categories and let $i\colon C\to B\comma\ell$ be any functor induced by the identity comma cone:
  \begin{equation}\label{eq:technicalsliceadjunction}
    \vcenter{\xymatrix@=1em{
      & {C}\ar@{=}[dl]\ar[dr]^{\ell} & \\
      {C}\ar[rr]_{\ell} && {B}
      \ar@{} "1,2";"2,2" |(0.6){\textstyle =}
    }}
    \mkern20mu = \mkern20mu
    \vcenter{\xymatrix@=1em{
      & {C}\ar[d]^-i \ar@/^/[ddr]^\ell \ar@/_/@{=}[ddl] & \\
      & {B\comma\ell}\ar[dl]|{p_1}\ar[dr]|{p_0} & \\
      {C}\ar[rr]_{\ell} && {B}
      \ar@{} "2,2";"3,2" |(0.6){\Leftarrow\phi}
    }}
  \end{equation}
  Then $i\colon C\to B\comma\ell$ is right adjoint to the codomain projection functor $p_1\colon B\comma\ell\to C$ in the slice 2-category $\qCat_2\slice C$ 
  \[    \xymatrix@=1.2em{
      {C}\ar@{=}[dr]\ar@/_1.5ex/[rr]_-{i}^-{}="one"
      & & *+!L(0.5){B\comma\ell}\ar@{->>}[dl]^-{p_1}
      \ar@/_1.5ex/[ll]_-{p_1}^-{}="two" \\
      & {C} &
      \ar@{}"one";"two"|{\bot}
    }
    \]
 and the counit may be chosen to be an identity 2-cell.
\end{lem}


\begin{proof}
By construction, $i$ is a section to the isofibration $p_1$ and, accordingly, we may take the counit of the postulated adjunction to be the identity $p_1 i = \id_C$. Now a 2-cell $\nu \colon \id_{B\comma \ell} \Rightarrow i p_1$ provides us with a 2-cell in $\qCat_2\slice C$ which satisfies the triangle identities with respect to that counit if and only if $p_1\nu$ and $\nu i$ are identity 2-cells. 

  We construct a suitable 2-cell $\nu\colon \id_{B\comma \ell} \Rightarrow i p_1$ by applying the 2-cell induction property of $B\comma\ell$ to the pair of 2-cells $\phi\colon p_0 \Rightarrow \ell p_1 = p_0 i p_1$ and $\id_{p_1}\colon p_1 = p_1 i p_1$; here, the compatibility condition of~\eqref{eq:comma-ind-2cell-compat} reduces to the trivial pasting identity \[ \vcenter{\xymatrix@=0.7em{ & B\comma \ell \ar@/^2ex/[ddr]^{p_0}   \ar@/^2ex/[ddl]|*+<3pt>{\scriptstyle p_1} \ar@/_2ex/[ddl]_{p_1}  \ar@{}[ddl]|{=} \\ &  \ar@{}[dr]|(.3){\Leftarrow\phi} \\ C \ar[rr]_\ell & & B}} \mkern20mu = \mkern20mu\vcenter{ \xymatrix@=0.7em{ & B\comma \ell \ar@/^2ex/[ddr]^{p_0}  \ar@/_2ex/[ddr]|*+<3pt>{\scriptstyle\ell p_1} \ar@{}[ddr]|{\Leftarrow\phi}  \ar@/_2ex/[ddl]_{p_1}  \\ & \ar@{}[dl]|(0.3){=} & \\ C \ar[rr]_\ell & & B}}  \] By construction, $\nu\colon \id_{B\comma \ell} \Rightarrow i p_1$ is a 2-cell satisfying $p_0\nu = \phi$ and $p_1\nu = \id_{p_1}$.

To show that $\nu i$ is an isomorphism,  observe that $p_0\nu i = \phi i = \id_\ell$ and $p_1\nu i = \id_{p_1} i = \id_{p_1 i} = \id_{\id_C}$, so  using the 2-cell conservativity property of $B\comma\ell$ we conclude that $\nu i$ is an isomorphism.  By Lemma \ref{lem:adjunction.from.isos} this suffices; indeed, applying middle-four interchange to $\nu i \cdot \nu i$ and the equation $p_1\nu=\id_{p_1}$,  $\nu i$ can be seen to be an idempotent isomorphism and thus an identity.
\end{proof}

In general, if a (representable) isofibration $f \colon B \tfib A$ admits a right adjoint right inverse $u$, then the counit of the RARI adjunction may be chosen to be an identity. Lemma \ref{lem:representable-isofibration}, which shows that an isofibration between quasi-categories defines a representable isofibration in $\qCat_2$, will allow us to make frequent use of this ``strictification'' result.

\begin{lem}\label{lem:isofibration-RARI} If $f \colon B \tfib A$ is a representable isofibration in a 2-category $\tcat{C}$ admitting a right adjoint right inverse $u' \colon A \to B$, then there exists a 1-cell $u \colon A \to B$ that is right adjoint right inverse to $f$ with identity counit.
\end{lem}
\begin{proof}
We construct the functor $u\colon A\to B$ and an isomorphism $\beta\colon u' \cong u$ by applying the universal property of the isofibration $f\colon B\tfib A$ to the counit $\epsilon'\colon fu'\cong\id_A$.
\[\xymatrix{ \ar@{}[dr]|(.7){\epsilon'\cong} & B \ar@{->>}[d]^f  & \ar@{}[d]|{\displaystyle\rightsquigarrow} &  \ar@{}[dr]|{\beta\cong} & B \ar@{->>}[d]^f \\ A \ar[ur]^{u'} \ar@{=}[r] & A &&  A \ar@/^1.5ex/[ur]^{u'} \ar@/_1.5ex/[ur]_{u} \ar@{=}[r] & A}\] By construction $fu = \id_A$. The composite
 $\eta \defeq \xymatrix@1{\id_B \ar@{=>}[r]^{\eta'} & u'f \ar@{=>}[r]^{\beta f} & uf}$ of the original unit $\eta'$ with the lifted isomorphism $\beta$ defines a 2-cell that whiskers with $f$ and $u$ to isomorphisms, permitting the application of Lemma \ref{lem:adjunction.from.isos} to conclude.
\end{proof}



\subsection{Terminal objects as adjoint functors}\label{subsec:terminal}


A quasi-category $A$ has a terminal object if and only if the projection functor $! \colon A \to \Del^0$ admits a right adjoint right inverse:

\begin{defn}[terminal objects]\label{defn:terminal}
An object $t$ in a quasi-category $A$ is {\em terminal\/} if there is an adjunction \[ \adjdisplay !-| t :\Del^0 -> A . \]  
 \end{defn}

Dually, of course, an object in $A$ is initial just when it defines a left adjoint left inverse to $! \colon A \to \Delta^0$.


 \begin{ex}[slices have terminal objects]\label{ex:slice-terminal}
For any object $a$  of a quasi-category $A$, there is an adjunction \[\adjdisplay !-| i : \Del^0 -> A\comma a .\] whose right adjoint, defining the terminal object of $A \comma a$, is any vertex of $A \comma a$ that is isomorphic to the degenerate 1-simplex $a\cdot\degen^0\colon a\to a$. This functor whiskers with the comma cone to an identity 2-cell:
\[     \vcenter{\xymatrix@=1em{
      & {\Del^0}\ar@{=}[dl]\ar[dr]^{a} & \\
      {\Del^0}\ar[rr]_{a} && {A}
      \ar@{} "1,2";"2,2" |(0.6){\textstyle =}
    }}
    \mkern20mu = \mkern20mu
    \vcenter{\xymatrix@=1em{
      & {\Del^0}\ar[d]^-i \ar@/^/[ddr]^a \ar@/_/@{=}[ddl] & \\
      & {A\comma a}\ar[dl]|{p_1}\ar[dr]|{p_0} & \\
      {\Del^0}\ar[rr]_{a} && {A}
      \ar@{} "2,2";"3,2" |(0.6){\Leftarrow\phi}
    }}\]
Thus, the adjunction $!\dashv i$ is a special case of Lemma \ref{lem:technicalsliceadjunction}.
\end{ex} 


Lemma \ref{lem:adjunction.from.isos} allows us to describe the minimal information required to display a terminal object.

\begin{lem}[minimal information required to display a terminal object]\label{lem:min-term-pres}
  To demonstrate that an object $t$ is terminal in $A$ it is enough to provide a unit 2-cell $\eta\colon\id_A\Rightarrow t!$ for which the whiskered composite $\eta t$ is an isomorphism.
 \end{lem}

When $A$ is a category this presentation is neither more nor less than the well known observation that an object $t$ is terminal in $A$ if and only if there exists a cocone on the identity diagram with vertex $t$ whose component at $t$ is an isomorphism. The proof of this lemma applies in any 2-category which possesses a 2-terminal object.

\begin{proof}
The categories $\hom'(\Del^0,\Del^0)$ and $\hom'(A,\Del^0)$ are both isomorphic to the terminal category $\catone$, so the counit is necessarily taken to be the identity and one of the triangle identities arises trivially. By Lemma \ref{lem:adjunction.from.isos} it remains only to provide a unit $\eta \colon \id_A \To t!$ for which the whiskered composition $\eta t$ is an isomorphism.  Specialising the proof of Lemma \ref{lem:adjunction.from.isos}, it follows formally that $\eta t \colon t \To t$ is an idempotent isomorphism and hence an identity, as required.
\end{proof}

The following straightforward 2-categorical lemma provides us with a useful ``external'' characterisation of terminal objects in quasi-categories.

\begin{lem}\label{lem:adj-ext-univ}
  Suppose we are given a pair of 1-cells $u\colon A\to B$ and $f\colon B\to A$ and a 2-cell $\epsilon\colon fu\Rightarrow\id_A$ in a 2-category $\tcat{C}$. Then $f$ is left adjoint to $u$ with counit $\epsilon$ in $\tcat{C}$ if and only if for all 0-cells $X\in\tcat{C}$ the functor $\tcat{C}(X,f)\colon\tcat{C}(X,B)\to\tcat{C}(X,A)$ is left adjoint to $\tcat{C}(X,u)\colon\tcat{C}(X,A)\to\tcat{C}(X,B)$, in the usual sense, with counit $\tcat{C}(X,\epsilon)$. 
\end{lem}

\begin{proof}
  The only if direction is immediate on observing that $\tcat{C}(X,-)$ is a 2-functor and thus preserves adjunctions. For the converse, we observe that the family of units of the adjunctions $\tcat{C}(X,f)\dashv\tcat{C}(X,u)$ is 2-natural in $X$ and so the 2-categorical Yoneda lemma provides us with a 2-cell $\eta\colon\id_B\Rightarrow uf$ with the property that $\tcat{C}(X,\eta)$ and $\tcat{C}(X,\epsilon)$ are unit and counit of the adjunction  $\tcat{C}(X,f)\dashv\tcat{C}(X,u)$. A further application of the 2-categorical Yoneda lemma demonstrates that the triangle identities for $\eta$ and $\epsilon$ follow immediately from those for $\tcat{C}(X,\eta)$ and $\tcat{C}(X,\epsilon)$.
\end{proof}

\begin{prop}\label{prop:terminal-ext-univ}
A vertex $t$ in a quasi-category $A$  is terminal if and only if for all $X$ the constant functor $\xymatrix@1{{X}\ar[r]^{!} & {\Del^0}\ar[r]^t & {A}}$ is terminal, in the usual sense, in the hom-category $\hom'(X,A)$.
\end{prop}

\begin{proof}
Apply Lemma~\ref{lem:adj-ext-univ} to the functors $t\colon\Del^0\to A$ and $!\colon A\to \Del^0$ and the identity natural transformation $!t = \id_{\Del^0}$.
\end{proof}

We conclude by comparing our definition of terminal object with its antecedent.


\begin{ex}\label{ex:terminaldefn} Joyal defines a vertex $t$ in a quasi-category $A$ to be terminal if and only if any sphere $\partial\Delta^n \to A$ whose final vertex is $t$ has a filler \cite[4.1]{Joyal:2002:QuasiCategories}. In Proposition~\ref{prop:terminalconverse}, we will show that Joyal's definition is equivalent to ours. For the moment, however, we shall at least take some satisfaction in convincing ourselves directly that his notion implies ours.

Supposing that $t \in A$ is terminal in Joyal's sense, then to define an adjunction $\adjinline !-| t :\Del^0 -> A .$ we wish to define a unit $\eta\colon\id_A\Rightarrow t!$ for which $\eta t$ is an identity. This unit is represented by a map \[ \xymatrix@=1.5em{ A \ar[d]_-{i_0} \ar@{=}[dr] \\ A \times \Delta^1 \ar[r]_-\eta & A \\  A \ar[u]^{i_1} \ar[r]_{!} & \Delta^0 \ar[u]_t} \] which we define as follows. For each $a \in A_0$, use the universal property of $t$ to choose a 1-simplex $\eta a \colon \Delta^1 \to A$ from $a$ to $t$. We take care to pick $\eta t$ to be the degenerate 1-simplex at $t$, thus ensuring that the 2-cell $\eta t$ will be the identity at $t$ as required by Lemma \ref{lem:min-term-pres}.

To define $\eta \colon A \to A^{\Delta^1}$ it suffices to inductively specify maps $\Delta^n \xrightarrow{\sigma} A \xrightarrow{\eta} A^{\Delta^1}$ for each non-degenerate $\sigma \in A_n$ compatibly with taking faces of $\sigma$. The map $\eta (\sigma \times \id_{\Delta^1}) \colon \Delta^n \times \Delta^1 \to A$ should be thought of as the component of $\eta$ at $\sigma$. The chosen 1-simplices $\eta a$ define the components at the vertices $a \in A_0$.

For each non-degerate $\alpha \colon a \to a' \in A_1$, define a cylinder $\Delta^1 \times \Delta^1 \to A$ as follows. The 1-skeleton consists of the displayed 1-simplices.
\[ \xymatrix{ a \ar[d]_\alpha \ar[r]^{\eta a} \ar[dr]|{\eta a} & t \ar@{=}[d]^{t\cdot\sigma^0} \\ a' \ar[r]_{\eta a'} & t} \] 
One shuffle is defined by degenerating $\eta a$. The other is chosen by applying the universal property of $t$ to the sphere formed by $\alpha$, $\eta a$, and $\eta a'$.

Continuing inductively, suppose we have chosen, for each $\sigma \in A_n$, a cylinder $\Delta^n \times \Delta^1 \to A$ from $\sigma$ to the degenerate $n$-simplex at $t$ in such a way that these choices are compatible with the face and degeneracy maps from the $n$-truncation $\sk_n\Del$ of $\Del$. Given a non-degenerate simplex $\tau \in A_{n+1}$, this simplex together with the $(n+1)$-simplices chosen for each of its $n$-dimensional faces $\tau\delta^i$ form an $(n+2)$-sphere with final vertex $t$, and we may choose a filler $\hat{\tau} \in A_{n+2}$. Define the requisite cylinder, the component of $\eta$ at $\tau$, to be the composite \[ \Delta^{n+1} \times \Delta^1 \xrightarrow{q} \Delta^{n+2} \xrightarrow{\hat{\tau}} A\] of $\hat{\tau}$ with the map induced by the functor $q \colon [n+1] \times [1] \to [n+2]$ defined by $q(i,0) = i$ and $q(i,1) = n+2$. By construction, $\hat{\tau} \face^i = \hat{\tau \face^i}$ for each $0 \leq i \leq n+1$, that is, the $i^{\th}$ face of the sphere whose filler defines $\hat{\tau}$ is the $(n+1)$-simplex chosen to fill the corresponding sphere for $\tau\delta^i$; thus the cylinder for $\tau$ is chosen compatibly with its faces. 
\end{ex}

This example will be generalised in Proposition \ref{prop:limitsasadjunctions} to limits of arbitrary shape.





\subsection{Basic theory}

A key advantage to our 2-categorical definition of adjunctions is that formal category theory supplies easy proofs of a number of desired results.

\begin{prop}\label{prop:adj-comp} A pair of adjunctions $\adjinline f -| u : A -> B.$ and $\adjinline f' -| u' : B -> C.$  in a 2-category compose to give an adjunction $\adjinline ff' -| u'u : A -> C.$. In particular, we may compose adjunctions of quasi-categories.
\end{prop}

\begin{proof}
The unit and counit of the composite adjunction are \[ \xymatrix@=10pt{ C \ar[dr]_{f'} \ar@{=}[rrrr] & & \ar@{}[d]|(.4){\Downarrow\eta'}& & C & & & C \ar[dr]^{f'} \ar@{}[d]|(.6){\Downarrow\epsilon'} \\ & B \ar[dr]_f \ar@{=}[rr] & \ar@{}[d]|(.4){\Downarrow\eta} & B \ar[ur]_{u'}  & & & B \ar[ur]^{u'} \ar@{=}[rr] & \ar@{}[d]|(.6){\Downarrow\epsilon} & B \ar[dr]^f \\  & & A \ar[ur]_u & &  & A \ar[ur]^u \ar@{=}[rrrr] & & & & A} \qedhere\] 
\end{proof}

Recall Proposition \ref{prop:equivsareequivs}, which demonstrates that equivalences in $\qCat_2$ are exactly the weak equivalences between quasi-categories in the Joyal model structure. The following classical 2-categorical result allows us to promote any equivalence to an adjoint equivalence (cf.~\cite[IV.4.1]{Maclane:1971:CWM}):

\begin{prop}\label{prop:equivtoadjoint} Any equivalence $w\colon A\to B$ in a 2-category may be promoted to an adjoint equivalence in which $w$ may be taken to be either the left or right adjoint. In particular, we may promote equivalences of quasi-categories to adjoint equivalences.
\end{prop}

\begin{proof}
  This is an immediate corollary of Lemma~\ref{lem:adjunction.from.isos}. 
\end{proof}

 \begin{prop}\label{prop:expadj} Suppose $\adjinline f -| u : A -> B.$ is an adjunction of quasi-categories. For any simplicial set $X$ and any quasi-category $C$, \[ \adjdisplay f^X-| u^X : A^X-> B^X .\qquad \text{and} \qquad \adjdisplay C^u -| C^f : C^A -> C^B .\] are adjunctions of quasi-categories. 
 \end{prop}
 \begin{proof}
By \ref{prop:qcat2closed} and \ref{rmk:exp2functor}, exponentiation defines 2-functors $(-)^X \colon \qCat_2 \to \qCat_2$ and $C^{(-)} \colon \qCat_2\op \to \qCat_2$, which preserve adjunctions.
 \end{proof}

As an easy corollary of the last few results, terminal objects are preserved by right adjoints, initial objects are preserved by left adjoints, and they are both preserved by equivalences.

\begin{prop}\label{prop:terminaldefn} If $u \colon A \to B$ is a right adjoint or an equivalence of quasi-categories and $t$ is a terminal object of $A$, then $ut$ is a terminal object in $B$.
\end{prop}
\begin{proof} By Proposition~\ref{prop:equivtoadjoint}, if $u$ is an equivalence then it may be promoted to a right adjoint, which reduces preservation by equivalences to preservation by right adjoints. Now Proposition~\ref{prop:adj-comp} tells us that we may compose the adjunction in which $u$ features with that which displays $t$ as a terminal object in $A$ to obtain an adjunction which displays $ut$ as a terminal object in $B$.
\end{proof}

\subsection{The universal property of adjunctions}

An essential point in the proof of the main existence theorem of \cite{RiehlVerity:2012hc} is that adjunctions between quasi-categories, while defined equationally, satisfy a universal property. In the terminology introduced there, any adjunction between quasi-categories extends to a \emph{homotopy coherent adjunction}. By contrast, a monad in $\qCat_2$ need not underlie a homotopy coherent monad. In this subsection, we provide several forms of the universal property held by an adjunction. 

Given an adjunction, we form the comma quasi-categories 
\begin{equation}\label{eq:commaobjdefn} \xymatrix@=1.5em{ f \comma A \ar[d]_{(p_1,p_0)} \ar[r] \pbexcursion & A^\cattwo \ar[d]  & & B \comma u \ar[d]_{(q_1,q_0)} \ar[r] \pbexcursion & B^\cattwo \ar[d] \\ A \times B \ar[r]_{\id_A \times f} & A \times A & & A \times B \ar[r]_{u \times \id_B} & B \times B}\end{equation} as in Definition~\ref{def:comma-obj}.  These quasi-categories are equipped with 2-cells
\[
\xymatrix@=15pt{ & f \comma A \ar[dl]_{p_1} \ar[dr]^{p_0} \ar@{}[d]|(.6){\Leftarrow\alpha} & && & B \comma u \ar[dl]_{q_1} \ar[dr]^{q_0} \ar@{}[d]|(.6){\Leftarrow\beta} \\ A & & B \ar[ll]^f && B \ar[rr]_u & & A}
\]
satisfying the weak 2-universal properties detailed in Observation~\ref{obs:unpacking-weak-comma-objects}. Mimicking the standard argument, we derive a fibred equivalence $f \comma A \simeq B \comma u$ from the unit and counit of our adjunction.

\begin{prop}\label{prop:adjointequiv} If $\adjinline f -| u : A -> B.$ is an adjunction of quasi-categories, then there is a fibred equivalence between the objects $(p_1,p_0)\colon f \comma A\tfib A\times B$ and $(q_1,q_0)\colon B \comma u\tfib A\times B$. 
\end{prop}
\begin{proof}
The composite 2-cells displayed on the left of the equalities below give rise to functors $w\colon B \comma u \to f \comma A$ and $w'\colon f \comma A \to B \comma u$ by 1-cell induction: 
\begin{equation*}\xymatrix@C=10pt{ & B \comma u \ar[dl]_{q_1} \ar[dr]^{q_0} \ar@{}[d]|(.6){\Leftarrow\beta} &  &  & B \comma u \ar[d]^{w}  & && & f \comma A \ar[dl]_{p_1} \ar[dr]^{p_0} \ar@{}[d]|(.6){\Leftarrow\alpha} & & & f \comma A \ar[d]^{w'}  \\ A \ar@{=}[dr] \ar[rr]^u & \ar@{}[d]|{\Leftarrow\epsilon} & B  \ar[dl]^f & = & f \comma A \ar[dl]_{p_1} \ar[dr]^{p_0} \ar@{}[d]|(.6){\Leftarrow\alpha} & & & A  \ar[dr]_u & \ar@{}[d]|{\Leftarrow\eta} & B \ar@{=}[dl] \ar[ll]_f & = & B \comma u \ar[dl]_{q_1} \ar[dr]^{q_0} \ar@{}[d]|(.6){\Leftarrow\beta}  \\ & A & & A  & & B \ar[ll]^f & & & B & & A \ar[rr]_u & & B}\end{equation*} 
By these defining pasting identities, the induced functors provide us with 1-cells 
\begin{equation*}
    \xymatrix{ f\comma A \ar@{->>}[dr]_{(p_1,p_0)} \ar@/^1ex/[rr]^{w'} & & B \comma u \ar@{->>}[dl]^{(q_1,q_0)} \ar@/^1ex/[ll]^w \\ & A \times B} 
\end{equation*}
in the slice 2-category $\qCat_2\slice(A\times B)$ commuting with the canonical isofibrations to $A \times B$. These identities give rise to the following sequence of pasting identities 
\begin{equation*}
\xymatrix@C=10pt{ & f \comma A \ar[d]^{w'} & & & f \comma A \ar[d]^{w'} & & & f \comma A \ar[dl]_{p_1} \ar[dr]^{p_0}\ar@{}[d]|(.6){\Leftarrow\alpha} & \ar@{}[dr]|*+{=} & & f \comma A  \ar[dl]_{p_1} \ar[dr]^{p_0} \ar@{}[d]|(.6){\Leftarrow\alpha} \\ & B \comma u \ar[d]^{w} & \ar@{}[d]|*+{=} & & B \comma u \ar[dl]_{q_1} \ar[dr]^{q_0} \ar@{}[d]|(.6){\Leftarrow\beta}  & {=}  & A  \ar[dr]_u \ar@{=}[dd] &\ar@{}[d]|{\Leftarrow\eta} & B \ar[ll]_f \ar@{=}[dl]  & A & & B \ar[ll]^f \\ & f \comma A \ar[dl]_{p_1} \ar[dr]^{p_0} \ar@{}[d]|(.6){\Leftarrow\alpha} & & A \ar@{=}[dr] \ar[rr]^u & \ar@{}[d]|(0.4){\Leftarrow\epsilon}  & B  \ar[dl]^f  &\ar@{}[r]|(0.4){\Leftarrow\epsilon} &  B \ar[dl]^f  &  \\ A & & B \ar[ll]^f & & A & & A& &  }
\end{equation*} 
in which the last step is an application of one of the triangle identities of the adjunction $f\dashv u$. This tells us that the endo-1-cells $ww'$ and $\id_{f\comma A}$ on the object $(p_1,p_0)\colon f\comma A\tfib A\times B$ in $\qCat_2\slice(A\times B)$ both map to the same 2-cell $\alpha$ under the whiskering operation. Applying Lemma \ref{lem:1cell-ind-uniqueness} (or Observation~\ref{obs:1cell-ind-uniqueness-reloaded}), 
we find that  $ww'$ and $\id_{f\comma A}$ are connected by a 2-isomorphism in $\qCat_2\slice(A\times B)$. A dual argument provides us with a 2-isomorphism between the 1-cells $w' w$ and $\id_{B\comma u}$ in the groupoid of endo-cells on $(q_1,q_0)\colon B\comma u\tfib A\times B$. This data provides us with an equivalence in the slice 2-category $\qCat_2\slice(A\times B)$, which we may lift along the smothering 2-functor of Proposition~\ref{prop:slice-smothering-2-functor} to give a fibred equivalence over $A\times B$.
\end{proof}

Just as in ordinary category theory, the Proposition~\ref{prop:adjointequiv} has a converse:

\begin{prop}\label{prop:adjointequivconverse} Suppose we are given functors $u \colon A \to B$ and $f\colon B\to A$ between quasi-categories. If there is a fibred equivalence between $(p_1,p_0)\colon f \comma A\tfib A\times B$ and $(q_1,q_0)\colon B \comma u\tfib A\times B$, then $f$ is left adjoint to $u$.
\end{prop}

    Schematically the proof of this result proceeds by observing that the image of the identity morphism at $f$ under the equivalence $f \comma A \simeq B \comma u$ defines a candidate unit for the desired adjunction. This can then be shown to have the appropriate universal property; the proof, however is slightly subtle. We delay it to the next section, where it will appear as a special case of a more general result needed there.

\begin{obs}[the hom-spaces of a quasi-category]\label{obs:pointwise-adjoint-correspondence} One model for the hom-space between a pair of objects $a$ and $a'$ in a quasi-category $A$ is the comma quasi-category $a \comma a'$, denoted by $\mathrm{Hom}_A(a,a')$ in~ \cite{Lurie:2009fk}. Proposition~\ref{prop:weakcomma} tells us that the canonical comparison $\ho(a\comma a')\to \ho(a)\comma \ho(a')$ from the homotopy category of this hom-space is a smothering functor. Its codomain $\ho(a)\comma\ho(a')$ is a comma category of arrows between a fixed pair of objects in the category $\ho A$, so it is simply the discrete category whose objects are the arrows from $a$ to $a'$ in $\ho A$. It follows from conservativity of the smothering functor that all arrows in $\ho(a\comma a')$ and thus also $a\comma a'$ are isomorphisms; hence,  $a\comma a'$ is a Kan complex by Joyal's result \cite[1.4]{Joyal:2002:QuasiCategories}.

By Observation~\ref{obs:fibred-pullback}, the fibred equivalence of Proposition~\ref{prop:adjointequiv} may be pulled back along the functor $(a,b)\colon\Del^0\to A\times B$ associated with any pair of vertices $a\in A$ and $b\in B$ to give an equivalence $fb \comma a \simeq b \comma ua$ of hom-spaces. This should be thought of as a quasi-categorical analog of the usual adjoint correspondence defined for arrows between a fixed pair of objects $b \in B$ and $a \in A$. 
\end{obs}

\begin{rmk}\label{rmk:vs-lurie-adjunction}
Observation \ref{obs:pointwise-adjoint-correspondence} demonstrates that the 2-categorical definition of an adjunction implies the definition of adjunction given by Lurie in \cite[5.2.2.8]{Lurie:2009fk}. As his definition has a more complicated form, we prefer not to recall it here. It is in fact precisely equivalent to Joyal's 2-categorical definition \ref{defn:adjunction}. Our preferred proof that Lurie's definition implies Joyal's makes use of the fact that the domain and codomain projections from comma quasi-categories are, respectively, cartesian and cocartesian fibrations. A proof will appear in \cite{RiehlVerity:2015fy}, which gives new 2-categorical definitions of these notions, which, when interpreted in $\qCat_2$, recapture precisely the (co)cartesian fibrations of \cite{Lurie:2009fk}.
\end{rmk}

We may apply Proposition~\ref{prop:adjointequiv}  to give a converse to Example~\ref{ex:terminaldefn}, proving that our notion of terminal objects is equivalent to Joyal's. The proof requires one combinatorial lemma, which relates certain comma quasi-categories with Joyal's slices, which are recalled in \ref{defn:slices} and \ref{rmk:map-slices}.

\begin{lem}\label{lem:slice-equiv-comma} For any vertex $a$ in a quasi-category $A$, there is an equivalence
\[\xymatrix@=1em{ \slicer{A}{a} \ar@{->>}[dr] \ar[rr]^-\sim & & A \comma a \ar@{->>}[dl] \\ & A} \] over $A$, which pulls back along any $f \colon B \to A$ to define an equivalence $\slicer{f}{a} \simeq f \comma a$ over $B$.
\end{lem}
\begin{proof}
The result follows from an isomorphism $A \comma a \cong \fatslicer{A}{a}$ between the comma and the fat slice construction reviewed in Definition \ref{defn:fat-slices}. The map $\slicer{A}{a} \to A \comma a$ and the equivalence over $A$ are then special cases of Proposition \ref{prop:slice-fatslice-equiv}. To establish the isomorphism, it suffices to show that $A \comma a$ has the universal property that defines $\fatslicer{A}{a}$.  By adjunction, a map $X \to \fatslicer{A}{a}$ corresponds to a commutative square, as displayed on the left:
\[ \vcenter{\xymatrix{ X \coprod X \ar[d] \ar[r]^-{\pi_X \coprod !} & X \coprod \Del^0 \ar[d]^{ (f, a)} \\ X \times \Del^1 \ar[r]_-k & A}} \qquad \leftrightsquigarrow \qquad \vcenter{\xymatrix{ X \ar[r]^-k \ar[d]_{(!,f)} & A^{\Del^1} \ar[d] \\  \Del^0 \times A \ar[r]_-{ a \times \id_A} & A \times A}}\]
which transposes to the commutative square displayed on the right. The data of the right-hand square is precisely that of a map $X \to A \comma a$ by the universal property of the pullback \ref{def:comma-obj} defining the comma quasi-category.

The isomorphism $A \comma a \cong \fatslicer{A}{a}$ pulls back to define an isomorphism $f \comma a \cong \fatslicer{f}{a}$. The map $\slicer{f}{a} \to f \comma a$ is then an equivalence over $B$ by Remark \ref{rmk:map-slices}.
\end{proof} 


\begin{prop}\label{prop:terminalconverse} A vertex $t \in A$ is terminal in the sense of  Joyal's \cite[4.1]{Joyal:2002:QuasiCategories} if and only if \[\adjdisplay !-| t :\Del^0 -> A . \] is an adjunction of quasi-categories. 
\end{prop}
\begin{proof}
The ``if'' direction is Example~\ref{ex:terminaldefn}. For the converse implication, an adjunction $! \dashv t$ gives rise to an equivalence between $!\comma \Delta^0 \cong A$ and $A \comma t$ over $A$ by Proposition~\ref{prop:adjointequiv}. Hence, by the 2-of-3 property of equivalences, the isofibration $A \comma t \tfib A$ is a trivial fibration. Lemma \ref{lem:slice-equiv-comma} supplies an equivalence \[\xymatrix@=1em{ \slicer{A}{t} \ar@{->>}[dr]_\sim \ar[rr]^-\sim & & A \comma t \ar@{->>}[dl] \\ & A} \] between our comma quasi-category and Joyal's slice quasi-category; see \ref{defn:slices} for a definition. Applying the 2-of-3 property again, it follows that the isofibration $\slicer{A}{t} \tfib A$ is a trivial fibration; the right lifting property against the boundary inclusions $\boundary\Delta^n \to \Delta^n$ says precisely that $t \in A$ is terminal in Joyal's sense.
\end{proof}

One reason for our particular interest in terminal objects is to show that the units and counits of adjunctions have universal properties which may be expressed ``pointwise'' in terms of certain outer horn filler conditions.

\begin{prop}[the pointwise universal property of an adjunction]\label{prop:pointwise-univ-adj}
    Suppose that we are given an adjunction 
    \begin{equation*}
        \adjdisplay f -| u : A -> B.
    \end{equation*}
    of quasi-categories with unit $\eta\colon\id_B\Rightarrow uf$ and counit $\epsilon\colon fu\Rightarrow \id_A$. Then for each $a \in A$ the (fat) slice quasi-category $f\comma a\simeq\slicer{f}{a}$ has terminal object $\epsilon a\colon fua\to a$, namely the component of the counit $\epsilon$ at $a$.
\end{prop}

\begin{proof}
    From Proposition~\ref{prop:adjointequiv}, the adjunction $f\dashv u$ gives rise to the equivalence $f \comma A \simeq B \comma u$ fibred over $A \times B$. By Observation~\ref{obs:fibred-pullback}, for each  $a\in A$, the fibred equivalence pulls back along the functor $(a,\id_B)\colon B\to A\times B$ to give a fibred equivalence
    \begin{equation}
      \xymatrix{ f\comma a \ar@{->>}[dr]_{p_0} \ar@/^1ex/[rr]^{w'} & & B \comma ua \ar@{->>}[dl]^{q_0} \ar@/^1ex/[ll]^w \\ & B}
    \end{equation}
  over $B$. 

By Example~\ref{ex:slice-terminal}, we know that $B\comma ua$ has the identity map $ua\cdot\degen^0\colon ua\to ua$ as its terminal object, and by Proposition~\ref{prop:terminaldefn} we know that terminal objects transport along equivalences, so it follows that $f\comma a$ also has terminal object $w'(ua\cdot\degen^0)$. It is now easily checked, from the definition of $w'$ given in Proposition~\ref{prop:adjointequiv}, that $w'(ua\cdot\degen^0)$ is isomorphic to $\epsilon a\colon fua\to a$. The desired result follows on transporting this terminal object along the equivalence between $f\comma a$ and $\slicer{f}{a}$ provided by the geometry result of Lemma \ref{lem:slice-equiv-comma}.
\end{proof}

Of course, the unit of an adjunction of quasi-categories satisfies a dual universal property.

\begin{obs}[unpacking this pointwise universal property of an adjunction] \label{obs:universal-property-of-epsilon}
    Unpacking the definitions in Remark~\ref{rmk:map-slices} and Definition~\ref{defn:slices} we see that a map $X\to \slicer{f}{a}$ corresponds to a pair of maps $b\colon X\to B$ and $\alpha\colon X\join\Del^0\to A$ which make the diagram 
    \[
        \xymatrix@=1.5em{ 
            X \ar[d] \ar[r]^f & B \ar[d]^b \\ X \join \Del^0 \ar[r]^-{\alpha} & A \\ \Del^0 \ar[u] \ar[ur]_{a}} \] commute.

By Proposition \ref{prop:terminalconverse}, we know that $\epsilon a\colon fua\to a$ is terminal in $\slicer{f}{a}$ is terminal if and only if every sphere $\boundary\Del^{n-1}\to \slicer{f}{a}$ whose last vertex is $\epsilon a$ may be filled to a simplex. Applying our description of maps into $\slicer{f}{a}$ and observing that $\Del^{n-1}\join\Del^0\cong\Del^n$ and $\boundary\Del^{n-1}\join\Del^0\cong\Horn^{n,n}$, we see that $\epsilon a$ being terminal means that if we are given
    \begin{itemize} 
        \item a horn $\Horn^{n,n} \to A$, with $n \geq 2$ together with
        \item a sphere $\partial\Delta^{n-1} \to B$ whose composite with $f$ is the boundary of the missing face of the horn, with the property that
        \item  the final edge of the horn  is $\epsilon a$ 
    \end{itemize} 
    then there is 
    \begin{itemize}
        \item a simplex $\Delta^n \to A$ filling the given horn and
        \item a simplex $\Delta^{n-1} \to B$ filling the given sphere, with the property that
        \item the $n\th$ face of the filling $n$-simplex in $B$ is the simplex obtained by applying $f$ to the filling $(n-1)$-simplex in $A$.
    \end{itemize} 

    For $n=2$, this situation is summarised by the following schematic:
 \[ \vcenter{ \xymatrix@=1.2em{ & fua \ar[dr]^{\epsilon} & \\ fb \ar[rr]_\alpha & & a}} b \in B_0\quad \rightsquigarrow \vcenter{\xymatrix@=1.2em{ & fua \ar[dr]^{\epsilon} \ar@{}[d]|(.6){\sigma} &\\ fb \ar[ur]^{f\beta} \ar[rr]_\alpha & & a}}  \mkern10mu \sigma \in A_2,\ \beta \colon b \to ua \in B_1\]
  \end{obs}

\begin{obs}[the relative universal property of an adjunction]
For any quasi-category $X$ the 2-functor $\hom'(X,{-})\colon \qCat_2\to\Cat$ carries an adjunction $f\dashv u\colon A\to B$ of quasi-categories to an adjunction $\hom'(X,f)\dashv \hom'(X,u)\colon\hom'(X,A)\to\hom'(X,B)$ of categories. Extending Lemma~\ref{lem:adj-ext-univ}, a standard and easily established fact of 2-category theory is that $f\colon B\to A$ has a right adjoint in $\qCat_2$ if and only if for each quasi-category $X$ the functor $\hom'(X,f)\colon\hom'(X,B)\to\hom'(X,A)$  has a right adjoint. We might call this observation the {\em external\/} universal property of an adjunction.

    There is a closely related {\em internal\/} or {\em relative\/} universal property of adjunctions in $\qCat_2$, which arises instead from Remark~\ref{rmk:exp2functor} that the cotensor $(-)^X\colon\qCat_2\to\qCat_2$ is also a 2-functor. Applying this cotensor 2-functor to the adjunction $f\dashv u$ we obtain its relative universal property simply as the pointwise universal property of the adjunction $f^X\dashv u^X\colon A^X\to B^X$ as derived in Proposition~\ref{prop:pointwise-univ-adj} and expressed explicitly in Observation~\ref{obs:universal-property-of-epsilon}. The relative universal property of adjunctions will become a key tool in the proof that any adjoint functor between quasi-categories extends to a homotopy coherent adjunction; see  \cite{RiehlVerity:2012hc}.  
\end{obs}

Another application of Proposition \ref{prop:adjointequiv} allows us to show that an isofibration between quasi-categories admits a right adjoint right inverse if and only if the following lifting property holds.

\begin{lem}[right adjoint right inverse as a lifting property]\label{lem:RARI-lifting}
  An isofibration $f \colon B \tfib A$ of quasi-categories admits a right adjoint right inverse if and only if for all $a \in A_0$ there exists $ua \in B_0$ with $fua = a$ and so that any lifting problem with $n \geq 1$
\begin{equation}\label{eq:RARI-lifting} \xymatrix{ \Delta^0 \ar[r]_{\fbv{n}} \ar@/^2ex/[rr]^{ua} & \boundary\Delta^n \ar@{u(->}[d] \ar[r] & B \ar@{->>}[d]^f \\ & \Delta^n \ar@{-->}[ur] \ar[r] & A} \end{equation}
has a solution.
\end{lem}
\begin{proof}
If $u$ is the right adjoint right inverse, then $fu = \id_A$ and there is a trivial fibration $B \comma u \trvfib f \comma fu \cong f \comma A$ over $A \times B$ defined by applying $f$ (Lemma \ref{lem:comma-obj-maps} proves that this map is an isofibration and Proposition \ref{prop:adjointequiv} shows that it is an equivalence). This trivial fibration pulls back over any vertex $a \in A_0$ to define a trivial fibration $B \comma ua \trvfib f \comma a$. The domain and codomain are equivalent to Joyal's slices by Lemma \ref{lem:slice-equiv-comma}, so the isofibration $\slicer{B}{ua} \tfib \slicer{f}{a}$ is also a trivial fibration:
\[ \xymatrix{ \boundary\Delta^{n-1} \ar[r] \ar[d] & \slicer{B}{ua} \ar[d] \\ \Delta^n \ar[r] \ar@{-->}[ur] & \slicer{f}{a} \cong B \times_A \slicer{A}{a}}\] In adjoint form, this is the lifting property of \eqref{eq:RARI-lifting}.

Conversely, the lifting property \eqref{eq:RARI-lifting} can be used to inductively define a section $u \colon A \to B$ of $f$ extending the choices $ua \in B_0$ for $a \in A_0$. The inclusion $\sk_0A\hookrightarrow A$ can be expressed as a countable composite of pushouts of coproducts of maps $\boundary\Del^n\hookrightarrow\Del^n$ with $n \geq 1$, and each intermediate lifting problem required to define a lift
\[ \xymatrix{ \Delta^0 \ar[r]_-{a} \ar@/^2ex/[rr]^{ua} & \sk_0 A \ar@{u(->}[d] \ar[r] & B \ar@{->>}[d]^f \\ & A \ar@{-->}[ur]^u \ar@{=}[r] & A}\]
will have the form of \eqref{eq:RARI-lifting}. To show that $u$ is a right adjoint right inverse to $f$, it suffices, by Lemma \ref{lem:adjunction.from.isos} to define a 2-cell $\eta \colon \id_B \To uf$ that whiskers with $u$ and with $f$ to isomorphisms. We construct a representative for $\eta$ by solving the lifting problem
\[\xymatrix{ B \coprod B \ar[d] \ar[rr]^{\id_B \coprod uf} & & A \ar[d]^{f} \\ B \times \Delta^1 \ar@{-->}[urr]^\eta \ar[r]_-{\pi_B} & B \ar[r]_f & A}\]
By construction $f\eta=\id_f$. 

To show that $\eta u$ is an isomorphism it suffices, by Corollary \ref{cor:pointwise-equiv}, to check that its components $\eta u(a) \colon ua \to ufua=ua$ are isomorphisms in $A$. Inverse isomorphisms can be found by elementary applications of the lifting property \eqref{eq:RARI-lifting}, whose details we leave to the reader.
\end{proof}


\subsection{Fibred adjunctions}\label{subsec:fibred.adjunction}

Fibred equivalences over $A$, i.e., equivalences in $\ho_*(\qCat_\infty\slice  A)$, are preferable to equivalences in the slice 2-category $\qCat_2\slice  A$ because the former can be pulled back along arbitrary maps $f \colon B \to A$; see Observation~\ref{obs:fibred-pullback}. Precisely the same kind of reasoning applies to adjunctions in $\qCat_2\slice  A$. 

\begin{defn}[fibred adjunctions]\label{defn:fibred.adj}
  We refer to adjunctions in $\ho_*(\qCat_\infty\slice A)$ as \emph{adjunctions fibred over $A$} or simply \emph{fibred adjunctions}.
\end{defn}

    Our aim in this section is to show that any adjunction in $\qCat_2\slice A$ can  be lifted to an adjunction fibred over $A$, i.e., to an adjunction in $\ho_*(\qCat_\infty\slice A)$. In particular, such a result will allow us to prove that any adjunction in $\qCat_2\slice  A$ may be pulled back along any functor $f\colon B\to A$. We shall use this result to define a loops--suspension adjunction on any quasi-category with appropriate finite limits and colimits (cf.\ Proposition~\ref{prop:loops-suspension}).

    Recall from Proposition \ref{prop:slice-smothering-2-functor} that the canonical 2-functor $\ho_*(\qCat_\infty\slice A) \to \qCat_2\slice A$ is a smothering 2-functor. Consequently, the following 2-categorical lemma is key:

\begin{lem}\label{lem:missed-lemma} Suppose $F \colon \tcat{C} \to \tcat{D}$ is a smothering 2-functor. Then any adjunction in $\tcat{D}$ can be lifted to an adjunction in $\tcat{C}$. Furthermore, if we have previously specified a lift of the objects, 1-cells, and either the unit or counit of the adjunction in $\tcat{D}$, then there is a lift of the remaining 2-cell that combines with the previously specified data to define an adjunction in $\tcat{C}$.
\end{lem}
\begin{proof}
We use surjectivity on objects and local surjectivity on arrows to define $u \colon A \to B$ and $f\colon B\to A$ in $\tcat{C}$ lifting the objects and 1-cells of the downstairs adjunction. Then we use local fullness to define lifts $\epsilon \colon fu \Rightarrow \id_A$ and $\eta' \colon \id_B \Rightarrow uf$ of the downstairs counit and unit. If desired, we can regard $A$, $B$, $f$, $u$ and $\epsilon$ as ``previously specified''. We will show that $f \dashv u$ by modifying the 2-cell $\eta'$. The details are similar to the proof of Lemma \ref{lem:adjunction.from.isos}.

We define a 2-cell $\theta\colon u\Rightarrow u$ as the ``triangle identity composite'' $\theta\defeq u\epsilon \cdot \eta' u$ and observe that $F\theta = \id_{Fu}$. Applying the local conservativity of the action of $F$ on 2-cells, we conclude that $\theta$ is an isomorphism. Define the 2-cell $\eta \colon \id_B \Rightarrow uf$ to be the composite $\eta \defeq \theta^{-1} f \cdot \eta'$. Because $F\theta$ is an identity,  $F\eta$ and $F\eta'$ lift the same downstairs 2-cell. We claim that this data forms an adjunction in $\tcat{C}$.

The diagram \eqref{eq:triangle-calculation-1} demonstrates that $u\epsilon \cdot \eta u = \id_u$. The diagram \eqref{eq:triangle-calculation-2} demonstrates that the other triangle identity composite $\phi\defeq \epsilon f \cdot f \eta $ is an idempotent. Finally observe that the component parts we've composed to make $\phi$ all map by $F$ to the corresponding components of the original adjunction in $\lcat{L}$. It follows that $F\phi$ is equal to the corresponding triangle identity composite in $\lcat{L}$ and so is an identity. Consequently, applying the local conservativity of $F$ on 2-cells we find that $\phi$ is an isomorphism. Because all idempotent isomorphisms are identities,  it follows that $\epsilon f \cdot f \eta = \id_f$ as required.
\end{proof} 

\begin{cor}\label{cor:missed-lemma}
  Every adjunction in $\qCat_2\slice A$ lifts to an adjunction fibred over $A$.
\end{cor}

\begin{proof}
  Combine Proposition~\ref{prop:slice-smothering-2-functor} and Lemma~\ref{lem:missed-lemma}.
\end{proof}

\begin{ex}\label{ex:fibred-technical-slice-adjunction}
Corollary \ref{cor:missed-lemma} allows us to lift the adjunction $\adjinline p_1 -| i : C -> B\comma \ell.$ of Lemma \ref{lem:technicalsliceadjunction} to a fibred adjunction over $C$ whose counit is an identity.
\end{ex}

\begin{ex}[fibred isofibration RARIs]\label{ex:isofib-section.fibred.adjunction} 
Lemma \ref{lem:isofibration-RARI} demonstrates that any right adjoint right inverse to an isofibration $f \colon B \tfib A$ can be modified to produce a RARI $f \dashv u$ with an identity counit. This latter adjunction provides us with an adjunction in $\qCat_2\slice A$ which we may lift into $\ho_*(\qCat_\infty\slice A)$ to give an adjunction
\begin{equation}\label{eq:fibred.terminal}
  \xymatrix@=1.5em{
    {A}\ar@/_1.2ex/[rr]_u\ar@{=}[dr] & {\bot} & 
    {B}\ar@/_1.2ex/[ll]_f\ar@{->>}[dl]^{f} \\
    & A &
  }
\end{equation}
which is fibred over $A$. In essence, this latter fibred adjunction expresses the fact that each of the fibres of the isofibration $f\colon B\tfib A$ has a terminal object.
\end{ex} 


\begin{obs}\label{obs:isofib-section.fibred.adjunction}
   Applying the 2-functor $\hom'_A(p,{-})$ represented by an isofibration $p\colon E\tfib A$ to the fibred adjunction in~\eqref{eq:fibred.terminal} we obtain an adjunction
  \begin{equation*}
    \adjdisplay f\circ{-} -| u\circ{-} : 
    \hom'_A(p,\id_A) -> \hom'_A(p, f).
  \end{equation*}
  of hom-categories. Now the identity functor $\id_A$ is the 2-terminal object of the 2-category $\qCat_2\slice A$, so it follows that $\hom'_A(p,\id_A)\cong\catone$. Hence, the displayed adjunction amounts simply to the assertion that $up$ is a terminal object of the category $\hom'_A(p, f)$. Consequently, applying Lemma~\ref{lem:adj-ext-univ}, we discover that there exists a fibred adjunction of the form displayed in~\eqref{eq:fibred.terminal} if and only if for all isofibrations $p\colon E\tfib A$ the composite map $up\colon E\to B$ is a terminal object of the hom-category $\hom'_A(p,f)$.
\end{obs}

A final example of a fibred adjunction describes the ``composition'' functor $A^{\Horn^{2,1}} \to A^\cattwo$ that fills a (2,1)-horn and then restricts to the missing face as the right and left adjoint, respectively, to the pair of functors that extend a 1-simplex into a composable pair by using the identities at its domain and codomain.

\begin{ex}\label{ex:comp.ident.adj}
  There exists a pair of adjunctions
  \begin{equation*}
    \xymatrix@C=10em@R=1ex{
      {{\Del^1}}\ar[r]|*+{\scriptstyle \face^1} & {{\Del^2}}
      \ar@/^2.5ex/[l]^{\degen^0}_{}="l" \ar@/_2.5ex/[l]_{\degen_1}^{}="u"
      \ar@{} "u";"l" |(0.2){\bot} |(0.8){\bot}
    }
  \end{equation*}
  of ordered sets, whose units and counits arise as the equalities $\degen^0\face^1 = \degen^1\face^1=\id_{\Del^1}$ and the inequalities $\face^1\degen^0 < \id_{[2]} < \face^1\degen^1$. Now if $A$ is a quasi-category, we may apply Proposition~\ref{prop:expadj} to construct the associated pair of adjunctions
  \begin{equation*}
    \xymatrix@C=10em{
      {A^{\Del^2}}
      \ar[r]|*+{\scriptstyle A^{\face^1}} &
      {A^{\Del^1}}
      \ar@/^2.5ex/[l]^{A^{\degen^1}}_{}="l" \ar@/_2.5ex/[l]_{A^{\degen_0}}^{}="u"
      \ar@{} "u";"l" |(0.2){\bot} |(0.8){\bot} 
    }
  \end{equation*}
  Here the upper adjunction has identity unit and the lower adjunction has identity counit. So it follows from Example~\ref{ex:isofib-section.fibred.adjunction} that this is a pair of adjunctions fibred over $A^{\Del^1}$ with respect to the projections $A^{\face^1}\colon A^{\Del^2}\tfib A^{\Del^1}$ and $\id_{A^{\Del^1}}\colon A^{\Del^1} \tfib A^{\Del^1}$. 

Because the horn inclusion $\Horn^{2,1}\inc\Del^2$ is a trivial cofibration in Joyal's model structure, the associated restriction isofibration $p\colon A^{\Del^2}\tfib A^{\Horn^{2,1}}$ is an equivalence of quasi-categories fibred over $A^{\Horn^{2,1}}$. By Proposition~\ref{prop:equivtoadjoint} (applied to $\qCat_2/A^{\Horn^{2,1}}$) and Corollary~\ref{cor:missed-lemma}, the fibred equivalence formed by $p$ and a chosen inverse $p'$ can be promoted to a pair of adjoint equivalences $p \dashv p' \dashv p$ fibred over $A^{\Horn^{2,1}}$. On account of the pushout diagram defining the (2,1)-horn,  $A^{\Horn^{2,1}}$ is isomorphic to the pullback:
\begin{equation*}
  \xymatrix@=2em{ \Horn^{2,1} \pbexcursion & \Del^1 \ar[l]_-{\face^2} & & 
    {A^{\Horn^{2,1}}}\pbexcursion
    \ar[r]^-{\pi_0}\ar[d]_-{\pi_1} & {A^\cattwo}\ar@{->>}[d]^-{p_1} \\ \Del^1 \ar[u]^{\face^0} & \Del^0 \ar[u]_{\face^0}\ar[l]^-{\face^1} & &
    {A^\cattwo}\ar@{->>}[r]_-{p_0} & A
  }
\end{equation*}

Now we may take the pushforward of the fibred adjunctions of the last two paragraphs along the isofibrations $(A^{\fbv{1}},A^{\fbv{0}})\colon A^{\Del^1}\tfib A\times A$ and $(A^{\fbv{2}},A^{\fbv{0}})\colon A^{\Horn^{2,1}}\tfib A\times A$ respectively to obtain adjunctions fibred over $A\times A$. Composing these we obtain a pair of adjunctions 
  \begin{equation}\label{eq:comp.ident.adj}
    \xymatrix@C=10em{
      *+[l]{A^{\Horn^{2,1}}\cong A^\cattwo\times_AA^\cattwo}
      \ar[r]|*+{\scriptstyle m} &
      {A^\cattwo}
      \ar@/^2.5ex/[l]^{i_1}_{}="l" \ar@/_2.5ex/[l]_{i_0}^{}="u"
      \ar@{} "u";"l" |(0.2){\bot} |(0.8){\bot} 
    }
  \end{equation}
  which are fibred over $A\times A$ with respect to the projections $(p_1,p_0)\colon A^\cattwo\tfib A\times A$ and $(p_1\pi_1,p_0\pi_0)\colon A^{\Horn^{2,1}}\tfib A\times A$. Here the upper adjunction has isomorphic unit and the lower adjunction has isomorphic counit. The functors $i_0$ and $i_1$ degenerate the domain and codomain respectively of a given 1-simplex to form a (2,1)-horn. The map $m$ is a ``composition'' functor.
\end{ex}

