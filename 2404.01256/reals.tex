\section{The real numbers}
\label{sec:real-numbers-object}

We review the construction of the Dedekind real numbers, and formulate it in a way that makes it easy to calculate the object of Dedekind reals in a parameterized realizability topos.
%
We also show that the Cauchy reals are sequence-avoiding in any parameterized realizability topos.

\subsection{The Dedekind real numbers}
\label{sec:dedek-real-numb}

In this section we work in higher-order intuitionistic logic, which can be interpreted in any topos.
%
Common mathematical constructions are available, as well as the standard number sets: the natural numbers $\objN$, the integers~$\objZ$ and the rationals~$\objQ$.

\begin{definition}
  \label{def:dedekind-reals}%
  A \defemph{Dedekind cut} is a pair $(L, U) \in \pow{\objQ} \times \pow{\objQ}$ of subsets of rationals, satisfying the following conditions, where $q$ and~$r$ range over~$\objQ$:
  %
  \begin{enumerate}
  \item $L$ and $U$ are inhabited: $\some{q} q \in L$ and $\some{r} r \in U$,
  \item $L$ is lower-rounded and $U$ upper-rounded:
    % 
    \begin{equation*}
      q \in L \liff \some{r} q < r \land r \in L 
      \qquad\text{and}\qquad
      r \in U \liff \some{q} q \in U \land q < r,
    \end{equation*}
  \item $L$ is below $U$: $q \in L \land r \in U \lthen q < r$,
  \item $L$ and $U$ are located: $q < r \lthen q \in L \lor r \in U$.
  \end{enumerate}
  %
  We write $\Cut{L,U}$ for the conjunction of the above conditions.
  %
  The set of \defemph{Dedekind reals} is
  %
  \begin{equation*}
    \RRd \defeq \set{ (L, U) \in \pow{\objQ} \times \pow{\objQ} \such \Cut{L, U} }.
  \end{equation*}
  % 
\end{definition}

The symbols $\RRd$ and $\RR$ both denote the set of Dedekind reals. We normally use $\RRd$ when referring to the object of Dedekind reals in a topos, or when we want to contrast the Dedekind reals with other kinds of reals.

Lower-roundedness may be split into two separate conditions:
%
\begin{itemize}
\item $L$ is lower: $q < r \land r \in L \lthen q \in L$,
\item $L$ is rounded: $q \in L \lthen \some{r} q < r \in r \in L$.
\end{itemize}
%
Upper-roundedness may be decomposed analogously.

Many textbooks construct the reals by using one-sided cuts. Again, this works classically but requires special care when done constructively, and in any case the symmetry of double-sided cuts streamlines the development, even classically. 

\begin{propositionC}
  \label{prop:RRd-stable-equality}%
  For all $x, y \in \RRd$, if $\lnot\lnot (x = y)$ then $x = y$.
\end{propositionC}

\begin{proof}
  Suppose $x = (L_x, U_x)$, $y = (L_y, U_y)$ and $\lnot\lnot(x = y)$.
  %
  We only prove $L_x \subseteq L_y$, as the other three inclusions are proved symmetrically.
  Suppose $q \in L_x$. Because $L_x$ is rounded, there is $r \in \QQ$ such that $q < r \in L_x$.
  From $\lnot\lnot (L_x = L_y)$ follows that $\lnot\lnot (r \in L_y)$.
  %
  Because~$y$ is located, $q \in L_y$ or $r \in U_y$. In the first case we are done, while the second case cannot happen, for if $r \in U_y$ then $\lnot (r \in L_y)$, contradicting $\lnot\lnot (r \in L_y)$.
\end{proof}

\begin{propositionC}
  \label{prop:stradle-closelyd}
  For any cut $(L, U)$ and $k \in \NN$ there exists $q \in \QQ$ such that $q - 2^{-k} \in L$ and $q + 2^{-k} \in U$.
\end{propositionC}

\begin{proof}
  There are $s \in L$ and $t \in U$. Let us show by induction on $j \in \NN$ that there are $u \in L$ and $v \in U$ such that $v - u \leq (2/3)^j (t - s)$. At $j = 0$ we take $u = s$ and $v = t$.
  %
  The induction step from $j$ to $j+1$ proceeds as follows. By the induction hypothesis there are $u' \in L$ and $v' \in U$ such that $v' - u' \leq (2/3)^j (t - s)$. By locatedness $(2 u' + v')/3 \in L$ or $(u' + 2 v')/3 \in U$. In the first case we take $u = (2 u' + v')/3$ and $v = v'$, and in the second $u = u'$ and $v = (u' + 2 v')/3$.

  Now let $k \in \NN$ be given. There is $j \in \NN$ such that $(2/3)^j (t - s) < 2^{-k}$. We proved that there exist $u \in L$ and $v \in U$ such that $v - u \leq (2/3)^j (t - s) < 2^{-k}$. We may take $q = (u + v)/2$ because
  $q - 2^{-k} < v - 2^{-k} < u$ hence $q - 2^{-k} \in L$, and similarly for $q + 2^{-k} \in U$.
\end{proof}

To facilitate calculations in parameterized realizability, we find an object that is isomorphic to~$\RRd$ but whose interpretation in assemblies is straightforward.
%
For any set $A$ and subset $S \subseteq A$, let $\compl{S} \defeq \set{x \in A \such \neg (x \in S)}$ be the complement of~$S$, and
%
\begin{equation*}
  \powcl{A} \defeq \set{S \in \pow{A} \such \all{x \in A} \neg\neg(x \in S) \lthen x \in S}
\end{equation*}
%
the set of $\neg\neg$-stable subsets of~$A$. Just like $\pow{A}$ is isomorphic to the set $\Omega^A$ of characteristic maps on~$A$, so $\powcl{A}$ is isomorphic to $\ClProp^A$, where
%
\begin{equation*}
  \ClProp \defeq \set{p \in \Omega \such \neg\neg p \lthen p}
\end{equation*}
%
is the set of $\neg\neg$-stable truth values. It is a complete Boolean algebra that can be used instead of~$\Omega$ in the definition of Dedekind cuts, like this.

\begin{definition}
  A \defemph{classical Dedekind cut} is a pair $(L, U) \in \powcl{\objQ} \times \powcl{\objQ}$
  of $\neg\neg$-stable subsets of rationals, satisfying the following conditions, where $r$ and $q$ range over~$\objQ$:
  %
  \begin{enumerate}
  \item $L$ and $U$ are not empty: $\lnot \all{q} q \not\in L$ and $\lnot \all{r} r \not\in U$,
  \item $L$ is lower and $U$ upper:
    % 
    \begin{equation*}
      q < r \land r \in L \lthen q \in L
      \quad\text{and}\quad
      q \in U \land q < r \lthen r \in U,
    \end{equation*}
  \item $L$ has no maximum and $U$ no minimum:
    % 
    \begin{equation*}
      (\all{r} r \in L \lthen r \leq q) \lthen q \not\in L
      \quad\text{and}\quad
      (\all{q} q \in U \lthen r \leq q) \lthen r \not\in U,
    \end{equation*}
  \item $L$ is below $U$: $q \in L \land r \in U \lthen q < r$,
  \item $L$ and $U$ are tight: $q \not\in L \land r \not\in U \lthen r \leq q$.
  \end{enumerate}
  %
  We write $\ClCut{L, U}$ for the conjunction of the above conditions.
  %
  The set of \defemph{classical Dedekind reals} is
  %
  \begin{equation*}
    \RRcl \defeq \set{ (L, U) \in \powcl{\objQ} \times \powcl{\objQ} \such \ClCut{L, U}}.
  \end{equation*}
\end{definition}

Dedekind cuts turn out to be those classical Dedekind cuts that have arbitrarily good rational approximations.

\begin{theoremC}
  \label{thm:reals-sub-classical}
  The set of Dedekind reals $\RRd$ is isomorphic to
  %
  \begin{equation}
    \label{eq:reals-sub-classical}%
    R_d \defeq \set{ (L, U) \in \RRcl \such 
       \all{k \in \NN} \some{q \in \QQ} q - 2^{-k} \in L \land q + 2^{-k} \in U
     }.
   \end{equation}
\end{theoremC}

\begin{proof}
  Let $f : \RRd \to R_d$ take a cut to its double complement, $f(L, U) = (\dcompl{L}, \dcompl{U})$.
  %
  To see that it is well-defined, we must verify that $(\dcompl{L}, \dcompl{U})$ is a classical cut and that
  %
  \begin{equation}
    \label{eq:mvv-reals}
    \all{k \in \NN} \some{q \in \QQ} q - 2^{-k} \in \dcompl{L} \land q + 2^{-k} \in \dcompl{U}.
  \end{equation}
  %
  We dispense with~\eqref{eq:mvv-reals} first: for any $k \in \NN$, by \cref{prop:stradle-closelyd} there is $q \in \QQ$ such that $q - 2^{-k} \in L \subseteq \dcompl{L}$ and $q + 2^{-k} \in U \subseteq \dcompl{U}$.

  Let us verify that $(\dcompl{L}, \dcompl{U})$ is a classical cut, where we spell out only the conditions for $L$, as the reasoning for $U$ is symmetric:
  %
  \begin{enumerate}
  \item $\dcompl{L}$ is non-empty because $L$ is inhabited and $L \subseteq \dcompl{L}$.
  \item $\dcompl{L}$ is lower: if $q < r$ and $r \in \dcompl{L}$ then $q \in \compl{L}$ would imply $r \in \compl{L}$, hence $q \in \dcompl{L}$.
  \item $\dcompl{L}$ has no maximum: if  $\all{r \in \dcompl{L}} r \leq q$ then $\all{r \in L} r \leq q$, hence $q \in \compl{L} = \compl{(\dcompl{L})}$.
  \item $\dcompl{L}$ is below $\dcompl{U}$: if $q \in \dcompl{L}$ and $r \in \dcompl{U}$ then $\lnot\lnot(q < r)$, therefore $q < r$.
  \item $\dcompl{L}$ and $\dcompl{U}$ are tight: suppose $q \not\in \dcompl{L}$ and $r \not\in \dcompl{U}$. If $q < r$ then either $q \in L$, which contradicts $q \not\in \dcompl{L}$, or $r \in U$, which contradicts $q \not\in \dcompl{U}$. Therefore $r \leq q$.
  \end{enumerate}
  %

  It remains to be shown that $f$ is a bijection. For injectivity, suppose $x = (L_x, U_x)$ and $y = (L_y, U_y)$ are cuts such that $\dcompl{L_x} = \dcompl{L_x}$ and $\dcompl{U_y} = \dcompl{U_y}$. Then $\lnot\lnot (x = y)$ and by 
  \cref{prop:RRd-stable-equality}, $x = y$.

  To establish surjectivity of~$f$, take any $(L, U) \in R_d$ and define
  %
  \begin{align*}
    \hat{L} &\defeq \set{q \in \QQ \such \some{r \in \QQ} q < r \land r \in L},
    \\
    \hat{U} &\defeq \set{r \in \QQ \such \some{q \in \QQ} q \in U \land q < r}.
  \end{align*}
  %
  Let us show that $(\hat{L}, \hat{U})$ is a cut, where again we verify only the conditions for~$\hat{L}$ when symmetry permits us to do so:
  %
  \begin{enumerate}
  \item $\hat{L}$ is inhabited, because by~\eqref{eq:mvv-reals} there is $q \in \QQ$ such that $q - 2^{0} \in L$, therefore $q - 2 \in \hat{L}$.
  \item $\hat{L}$ is obviously lower, and it is rounded too: if $q \in \hat{L}$ then $q < r$ and $r \in L$ for some $r$,
    hence $(q + r)/2 \in \hat{L}$ because $(q + r)/2 < r$.
  \item $\hat{L}$ is below $\hat{U}$: if $q \in \hat{L}$ and $r \in \hat{U}$ then there are $s$ and $t$ such that $q < s$, $s \in L$, $t \in U$, and $t < r$, therefore $q < s < t < r$.
  \item To see that $(\hat{L},  \hat{U})$ is located, consider any $q < r$. There is $k \in \NN$ such that $2^{-k+1} < r - q$, and there is $s \in \QQ$ such that $s - 2^{-k} \in L$ and $s + 2^{-k} \in U$. Either $q < s - 2^{-k}$, in which case $q \in \hat{L}$, or $s + 2^{-k} < r$, in which case $r \in \hat{U}$.
  \end{enumerate}
  %
  Finally, we claim that $\dcompl{\hat{L}} = L$ and $\dcompl{\hat{U}} = U$.
  %
  Notice that $\hat{L} \subseteq L$ because $L$ is lower, hence $\dcompl{\hat{L}} \subseteq \dcompl{L} = L$.
  %
  To show $L \subseteq \dcompl{\hat{L}}$, suppose $q \in L$. If we had $q \in \compl{\hat{L}}$, then it would follow that $\all{r \in L} r \leq q$, and because $L$ has no maximum also $q \not\in L$, a contradiction, hence we conclude $q \in \dcompl{\hat{L}}$.
\end{proof}

\subsection{The Dedekind reals in parameterized realizability}
\label{sec:dedek-reals-param}

We seek an explicit description of the object of Dedekind reals in a parameterized realizability topos. Rather than interpreting \cref{def:dedekind-reals} directly, we compute the classical Dedekind cuts and use~\cref{thm:reals-sub-classical}.

\begin{proposition}
  \label{prop:PRT-classical-reals}%
  In a parameterized realizability topos, the object of classical Dedekind reals is isomorphic to $\nabla\RR$.
\end{proposition}

\begin{proof}
  In \cref{sec:assembly-nabl-negn} we saw that $\nabla\two$ is isomorphic to~$\ClProp$,
  hence $\powcl{\objQ}$ is isomorphic to ${\nabla\two}^\objQ$, which in turn is isomorphic to $\nabla\left(\pow{\QQ}\right)$.
  Because the definition of $\ClCut{L,U}$ uses only logical connectives and relations that
  are preserved by~$\nabla$, the object $\RRcl$ is isomorphic to
  %
  \begin{equation*}
    \nabla \set{(L, U) \in \pow{\QQ} \times \pow{\QQ} \such \ClCut{L, U}}.
  \end{equation*}
  %
  We are done because under $\nabla$ above appears the classical definition of~$\RR$.
\end{proof}

\begin{corollary}
  \label{cor:dedekind-characterization}%
  In $\PRT{\AA, \PP}$ the object of Dedekind reals is (isomorphic to) the assembly $\RRd$ whose underlying set is
  %
  $
    |\RRd| = \set{x \in \RR \such \Ex{\RRd}(x) \neq \emptyset}
  $
  %
  and whose existence predicate is
  %
  \begin{equation*}
    \Ex{\RRd}(x) \defeq
    \set{\R{r} \in \AA \such
      \all{p \in \PP}
      \all{k \in \NN}
      \some{q \in \QQ}
      |x - q| < 2^{-k} \land
      \R{r} \, \numeral{k} \rz[p] \Ex{\objQ}(q)
    }.
  \end{equation*}
  %
\end{corollary}

\begin{proof}
  By \cref{prop:PRT-classical-reals,thm:reals-sub-classical} the object of Dedekind reals is (isomorphic to) the
  sub-assembly of $\nabla{\RR}$ whose existence predicate is the realizability interpretation of~\eqref{eq:reals-sub-classical}, which is what~$\Ex{\RRd}$ is.
\end{proof}

The takeaway is that a realizer of a Dedekind real~$x$ computes arbitrarily precise rational approximations of~$x$ which \emph{may} depend on the parameter~$p$.

Recall that the strict order $<$ on $\RRd$ is defined by
%
\begin{equation*}
  x < y \defiff
  \some{q \in \QQ} q \in U_x \land q \in U_y.
\end{equation*}
%
Thus $x < y$ is realized by a rational that is wedged between~$x$ and~$y$.
However, the following lemma shows that we need not bother because the order is $\neg\neg$-stable.

\begin{lemma}
  \label{lem:lt-stable}%
  In $\PRT{\AA, \PP}$,
  %
  \begin{enumerate}
  \item $\all{x, y \in \RRd} x \neq y \lthen x < y \lor y < x$, and
  \item $\all{x, y \in \RRd} \neg \neg (x < y) \lthen x < y$.
  \end{enumerate}
\end{lemma}

\begin{proof}
  Deriving the second statement from the first one is an exercise in intuitionistic logic,
  so we only describe how to realize the first statement.
  %
  Suppose $x, y \in \carrier{\RRd}$, $\R{r} \in \Ex{\RRd}(x)$, $\R{s} \in \Ex{\RRd}(y)$ and~$p \in \PP$.
  Search for the least~$k \in \NN$ such that $|\rat{\R{r} \app[p] k} - \rat{\R{s} \app[p] k}| > 2^{-k+1}$.
  It will certainly be found because $x \neq y$.
  %
  With~$k$ in hand, compare $\rat{\R{r} \app[p] k}$ and $\rat{\R{s} \app[p] k}$.
  If $\rat{\R{r} \app[p] k} < \rat{\R{s} \app[p] k}$ then
  %
  \begin{equation*}
    x < \rat{\R{r} \app[p] k} + 2^{-k}
      < \rat{\R{s} \app[p] k} - 2^{-k}
      < y,
  \end{equation*}
  %
  therefore $x < y$ may be realized by $\rat{\R{r} \app[p] k} + 2^{-k}$.
  %
  If $\rat{\R{s} \app[p] k} < \rat{t \app[p] k}$ then $y < x$ symmetrically.
\end{proof}


\subsection{The Cauchy reals}
\label{sec:cauchy-reals}
%
It is instructive to compare the Dedekind and Cauchy reals in parameterized realizability.
%
We identify the Cauchy reals as those Dedekind reals that are limits of Cauchy sequences of rationals. Since we work without the axiom of countable choice, rapid convergence should be imposed.

\begin{definition}
  \label{def:cauchy-real}
  A \defemph{Cauchy real} is a Dedekind real $x \in \RRd$ which is the limit of a rapidly converging rational sequence, i.e., the set of Cauchy reals is
  %
  \begin{equation*}
    \RRc \defeq \set{x \in \RRd \such \some{q \in \objQ^\objN} \all{n \in \objN} |x - q_n| < 2^{-n}}.
  \end{equation*}
\end{definition}

It would make no difference if we used the classical reals:

\begin{propositionC}
  \label{prop:cauch-real-sub-classical}%
  %
  The set of Cauchy reals $\RRc$ is isomorphic to
  %
  \begin{equation}
    \label{eq:cauchy-characterization}%
    R_c \defeq \set{x \in \RRcl \such \some{q \in \objQ^\objN} \all{n \in \objN} |x - q_n| < 2^{-n}}.
  \end{equation}
\end{propositionC}

% NOTE: here and elsewhere we are silently assuming that the classical reals form a ring and a lattice, which they do.
% (They form a field, and are partially ordered, but do not satisfy ∀ x y z . x < y ⇒ x < z ∨ z < y.)

\begin{proof}
  Any classical real $x \in \RRcl$ that is the limit of a rapidly converging rational sequence is also a Dedekind real.
\end{proof}

The previous proposition tells us how to compute the object of Cauchy reals.

\begin{corollary}
  \label{cor:cauchy-characterization}%
  In $\PRT{\AA, \PP}$ the object of Cauchy reals is (isomorphic to) the assembly $\RRc$ whose
  underlying set is
  $
    |\RRc| \defeq \set{x \in \RR \such \Ex{\RRc}(x) \neq \emptyset}
  $
  and the existence predicate is
  %
  \begin{multline*}
    \Ex{\RRc}(x) \defeq
    \{\R{r} \in \AA \such \\
      \some{q \in \QQ^\NN}
      \all{p \in \PP}
      \all{k \in \NN}
      |x - q_k| < 2^{-k} \land
      \R{r} \, \numeral{k} \rz[p] \Ex{\objQ}(q_k)
   \}.
  \end{multline*}
\end{corollary}

\begin{proof}
  As in \cref{cor:dedekind-characterization}, the existence predicate $\Ex{\RRc}$ is the realizability interpretation of \eqref{eq:cauchy-characterization}.
\end{proof}

If we write the existence predicate for the Dedekind reals in the equivalent form\footnote{The equivalence with the definition given in \cref{cor:dedekind-characterization} relies on countable choice \emph{outside} the topos.}
%
\begin{multline*}
  \Ex{\RRd}(x) \defeq
  \{\R{r} \in \AA \such \\
    \all{p \in \PP}
    \some{q \in \QQ^\NN}
    \all{k \in \NN}
    |x - q_k| < 2^{-k} \land
    \R{r} \, \numeral{k} \rz[p] \Ex{\objQ}(q_k)
  \},
\end{multline*}
%
the difference between the Dedekind and Cauchy reals is seen to be just one of switching the quantifiers:
a rapid sequence representing a Dedekind real may depend on the parameter, whereas one representing a Cauchy real may not.
\begin{theorem}
  \label{thm:cauchy-uncountable}%
  In $\PRT{\AA,\PP}$ the Cauchy reals are sequence-avoiding.
\end{theorem}

\begin{proof}
  For $\R{r} \in \AA$ to realize
  %
  \begin{equation*}
    \all{f \in \RRc^\objN}
    \some{x \in \RRc}
    \all{n \in \objN}
    f n \neq x
  \end{equation*}
  %
  it has to satisfy the following condition:
  for all $f \in \carrier{\RRc^\objN}$, $\R{b} \in \Ex{\RRc^\objN}(f)$ and $p \in \PP$
  there is $x \in \carrier{\RRc}$ such that $\all{n \in \NN} f(n) \neq x$ and
  %
  \begin{equation*}
    \some{t \in \QQ^\NN}
    \all{k \in \NN}
    |x - t_k| < 2^{-k}
    \land
    \all{p' \in \PP} (\R{r} \app[p] \R{b}) \app[p'] k \in \Ex{\objQ}(t_k).
  \end{equation*}
  %
  So suppose $f \in \carrier{\RRc^\objN}$, $\R{b} \in \Ex{\RRc^\objN}(f)$ and $p \in \PP$.
  By unraveling the meaning of $\R{b} \in \Ex{\RRc^\objN}(f)$ we find out that
  there is a map $s : \NN \times \NN \to \QQ$, which depends only on~$\R{b}$ and~$p$,
  such that
  %
  \begin{equation*}
    \all{n, k \in \NN}
    |f(n) - s(n, k)| < 2^{-k}
    \land
    \all{p' \in \PP}
    (\R{b} \app[p] n) \app[p'] k \in \Ex{\objQ}(s(n, k)).
  \end{equation*}
  %
  We define sequences $t, u : \NN \to \QQ$ which depend only on~$s$, such that $0 < u_n - t_n < 2^{-n-1}$ and $f(n) < u_n \lor t_n < f(n)$, for all $n \in \NN$. Set $t_0 \defeq s(0,0) + 1$ and $u_0 \defeq s_0 + \frac{1}{4}$.
  Assuming $t_n$ and $u_n$ have been constructed, let $d \defeq \frac{u_n - t_n}{4}$, $m \defeq \min \set{m \in \NN \such 2^{-m} < d}$, and define
  %
  \begin{equation*}
    (t_{n+1}, u_{n+1}) \defeq
    \begin{cases}
      (t_n, t_n + d) &\text{if $t_n + 2 d < s(n, m)$,} \\
      (u_n - d, u_n) &\text{otherwise.}
    \end{cases}
  \end{equation*}
  %
  % Verification that the conditions are met:
  % * f(0) < s(0,0) + 1 = t_0
  % * u_0 - t_0 = 1/2 < 2^0
  % * u_{n+1} - t_{n+1} = d = (u_n - t_n) / 4 < 2^{-n-1} / 4 < 2^{-n-1} / 2 = 2^{-n-2}
  % * if t_n + 2 d < s(n, m) then u_{n+1} = t_n + d < t_n + 2 d - d < s(n, m) - d < s(n,m) - 2^{-m} < f(n)
  % * if s(n,m) ≤ t_n + 2 d = u_n - 2 d then f(n) < s(n,m) + 2^{-m} < s(n,m) + d ≤ u_n - 2 d + d = u_n - d = t_{n+1}
  %
  The real $x \defeq \lim_n t_n = \lim_n u_n$ satisfies $\all{n \in \NN} f(n) \neq x$ and
  depends only on $\R{b}$ and $p$, because it is defined in terms of~$s$.
  %
  It remains to exhibit~$\R{r} \in \AA$, independent of all parameters, such that $(\R{r} \app[p] \R{b}) \app[p'] k \in \Ex{\objQ}(t_k)$ for all $k \in \NN$ and $p' \in \PP$. It takes the form $\R{r} \defeq \abstr{b} \abstr{k} e$ where~$e$ computes a realizer for~$t_k$, as described in the above procedure, and relying on the fact that $(\R{b} \app[p] n) \app[p'] k$ computes a realizer for~$s(n, k)$.
\end{proof}

Let us investigate where the proof of \cref{thm:cauchy-uncountable} gets stuck if we replace the Cauchy reals with the Dedekind reals.
For $\R{r} \in \AA$ to realize
% 
\begin{equation*}
  \all{f \in \RRd^\objN}
  \some{x \in \RRd}
  \all{n \in \objN}
  f n \neq x
\end{equation*}
%
it has to satisfy the following condition:
for all $f \in \carrier{\RRd^\objN}$, $\R{b} \in \Ex{\RRd^\objN}(f)$ and $p \in \PP$
there is $x \in \carrier{\RRd}$ such that $\all{n \in \NN} f(n) \neq x$ and
% 
\begin{equation*}
  \all{k \in \NN}
  \all{p' \in \PP}
  \some{t \in \QQ}
  |x - t| < 2^{-k}
  \land
  (\R{r} \app[p] \R{b}) \app[p'] k \in \Ex{\objQ}(t).
\end{equation*}
%
So suppose $f \in \carrier{\RRd^\objN}$, $\R{b} \in \Ex{\RRd^\objN}(f)$ and $p \in \PP$.
By unraveling the meaning of $\R{b} \in \Ex{\RRd^\objN}(f)$ we find out that
for every $p' \in \PP$ there is a map $s : \NN \times \NN \to \QQ$, which depends on~$\R{b}$ and~$p$ \emph{as well as~$p'$}, such that
%
\begin{equation*}
  \all{n, k \in \NN}
  |f(n) - s(n, k)| < 2^{-k}
  \land
  (\R{b} \app[p] n) \app[p'] k \in \Ex{\objQ}(s(n, k)).
\end{equation*}
%
At this point we are stuck: if we continue by constructing a sequence $t : \NN \to \QQ$ as in the proof, its limit $\lim_n t_n$ will depend on~$p'$, but we need a real~$x$ that does not.
For a specific ppca $(\AA, \PP)$ one might find some way of constructing~$x$ that avoids dependence on~$p'$,
although there can be such no general construction as that would contradict~\cref{thm:countable-reals}.

Finally, let us show that in $\PRT{\AA, \PP}$ the carrier sets of Cauchy and Dedekind reals coincide.
%
Recall that $x \in \RRd$ is \defemph{strongly irrational}\footnote{In intuitionistic mathematics this is a stronger notion than being \defemph{irrational}, which means $x \neq q$ for all~$q \in \objQ$.} when $x < q \lor q < x$ for all $q \in \objQ$.

\begin{lemmaC}
  \label{lem:rat-irrat-cauchy}
  Every rational and every strongly irrational Dedekind real is a Cauchy real.
\end{lemmaC}

\begin{proof}
  Clearly, every rational number is a Cauchy real.
  %
  Suppose $x \in \RRd$ is strongly irrational. Then we may find a rational sequence converging rapidly to~$x$ by simple bisection, as follows. There are rational $t_0$ and $u_0$ such that $t_0 < x < u_0$ and $u_0 - t_0 < 1$.
  Given rational $t_n < x < u_n$ such that $u_n - t_n < 2^{-n}$, let $m \defeq \frac{t_n + u_n}{2}$ and define
  %
  \begin{equation*}
    (t_{n+1}, u_{n+1}) =
    \begin{cases}
      (t_n, m) & \text{if $x < m$,} \\
      (m, u_n) & \text{if $m < x$.}
    \end{cases}
  \end{equation*}
  %
  Then both $t$ and $u$ are rapid rational sequences converging to~$x$.
\end{proof}

\begin{proposition}
  \label{prop:carrier-Rd-eq-Rc}%
  In a parameterized realizability topos, $\carrier{\RRd} = \carrier{\RRc}$.
\end{proposition}

\begin{proof}
  Combine \cref{lem:lt-stable} and \cref{lem:rat-irrat-cauchy} to conclude that
  in $\PRT{\AA, \PP}$ every rational and every irrational Dedekind real is a Cauchy real.
  Of course, there is no Dedekind real which is neither rational nor irrational.
\end{proof}

To summarize, the difference between $\RRc$ and $\RRd$ is not one of extent but one of \emph{structure}:
the identity map $x \mapsto x$ is realized as a morphism $\RRc \to \RRd$, but not necessarily as a morphism $\RRd \to \RRc$.


%%% Local Variables:
%%% mode: latex
%%% TeX-master: "countable-reals"
%%% End:
 