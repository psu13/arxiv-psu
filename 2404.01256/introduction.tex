\section{Introduction}
\label{sec:introduction}

Georg Cantor's theorem \cite{cantor74:_ueber_eigen_inbeg_zahlen} about uncountability of the real numbers is one of the most widely known results of modern mathematics.
%
It is taught to all students of mathematics, and features regularly in popular accounts of mathematics.
%
Cantor proved his theorem by the diagonalization method, whose originality is attested by persistent attacks that are diligently parried by editors~\cite{hodges98:_editor_recal_some_hopel_paper}.
%
It is unwise to participate in such quixotic attempts at bringing down the uncountability of the reals.

Nevertheless, we shall defy Cantor by constructing a mathematical universe, a topos\footnote{A topos is a finitely complete cartesian-closed category with a subobject classifier. Every topos is a model of higher-order intuitionistic logic~\cite{jim86:_introd_higher_order_categ_logic,johnstone02:_sketc_eleph}.} to be precise, in which the natural numbers map surjectively onto the reals.
%
We must of course pay a price: the topos is intuitionistic, as it invalidates the law of excluded middle and the axiom of choice.
%
Let us therefore first investigate how these two principles contribute to uncountability of the reals.

Say that a set $A$ is \defemph{sequence-avoiding} if for every sequence $(a_n)_n$ in~$A$ there exists $x \in A$ such that $x \neq a_n$ for all $n \in \NN$. Clearly, when this is the case there can be no surjection $\NN \to A$.
%
Cantor's proof~\cite{cantor74:_ueber_eigen_inbeg_zahlen} showing that the reals are sequence-avoiding uses the method of nested intervals, which goes as follows.

\begin{theorem}
  \label{thm:R-uncountable}
  %
  The set of reals is sequence-avoiding.
\end{theorem}

\begin{proof}
  Let any sequence $(a_n)_n$ of reals be given.
  %
  It suffices to construct a sequence of closed intervals $[x_n, y_n]_n$ such that, for all $n \geq 0$:
  %
  \begin{enumerate}[a)]
  \item $x_n \leq x_{n+1} < y_{n+1} \leq y_n$,
  \item\label{it:width-converge} $y_n - x_n < 2^{-n}$,
  \item $a_n < x_{n+1}$ or $y_{n+1} < a_n$.
  \end{enumerate}
  %
  Indeed, the limit $\ell \defeq \lim_n x_n = \lim_n y_n$ will exist and satisfy $a_n \neq \ell$ for all $n \in \NN$, because either $a_n < x_{n+1} \leq \ell$ or $\ell \leq y_{n+1} < a_n$.

  Start with any $[x_0, y_0]$ satisfying the first two conditions, for example $[0, \sfrac{1}{2}]$, and define recursively
  %
  \begin{equation}
    \label{eq:R-uncountable}%
    [x_{n+1}, y_{n+1}] \defeq
    \begin{cases}
      [x_n, (4 x_n + y_n)/5] &\text{if $(3 x_n + 2 y_n)/5< a_n$,}\\
      [(x_n + 4 y_n)/5, y_n] &\text{if $a_n < (2 x_n + 3 y_n)/5$}.
    \end{cases}
  \end{equation}
  %
  % Verification:
  % * if 3 x_n + 2 y_n < a_n:
  %   (a) x_n = x_{n+1} < (4 x_n + y_n)/5 = y_{n+1} ≤ y_n
  %   (b) y_n - x_n = (y_n - x_n)/5 < (y_n - x_n)/2 < 2^{-n-1}
  %   (c) y_n = (4 x_n + y_n)/5 < (2 x_n + 3 y_n)/5 < a_n
  %
  % * if a_n < (2 x_n + 3 y_n)/5:
  %   (a) x_n < (x_n + 4 y_n)/5 = x_{n+1} ≤ y_{n+1} = y_n.
  %   (b) y_n - x_n = (y_n - x_n)/5 < (y_n - x_n)/2 < 2^{-n-1}
  %   (c) a_n < (2 x_n + 3 y_n)/5 < (x_n + 4 y_n) = x_n
  In both cases the relevant inequalities are satisfied.
\end{proof}

How precisely does \eqref{eq:R-uncountable} specify a sequence, when the two options overlap when $(3 x_n + 2 y_n)/5 < a_n < (2 x_n + 3 y_n)/5$?
%
We may \emph{choose} one for each~$n$ by appealing to the axiom of dependent choice,\footnote{Dependent choice is a version of the axiom of choice in which the next choice may depend on the previous ones. The proof of \cref{thm:R-uncountable} can be improved to rely on the weaker countable choice only: choose ahead of time, for all rationals $q < r$ and $n \geq 0$, either $a_n < q$ or $r < a_n$. Then make sure $(x_n)_n$ and $(y_n)_n$ are rational sequences by picking rational $x_0$ and $y_0$, and follow the choices so made when choosing an option in~\eqref{eq:R-uncountable}.} or \emph{remove} the overlap by using the law of excluded middle, which allows us to take the first option if available and the second one otherwise. When both principles are proscribed, the proof crumbles.

Mathematics without the law of excluded middle and the axiom of countable choice is \emph{intuitionistic} or \emph{constructive}.\footnote{Some schools of constructive mathematics accept the axiom of countable choice, notably Erret Bishop's~\cite{bishop67:_found_const_analy}, but we eschew it because it validates \cref{thm:R-uncountable}.} Among the available varieties~\cite{bridges87:_variet_const_mathem},
intuitionistic higher-order logic~\cite{jim86:_introd_higher_order_categ_logic} suits us because it is valid in any topos.
%
In many respects intuitionistic and classical mathematics are alike, but there are also important differences, which we review before proceeding.

\subsection{Countable and uncountable sets in intuitionistic mathematics}
\label{sec:countabe-set-intuit}

Classically equivalent notions may bifurcate in intuitionistic mathematics, and countability is one of them.
We say that a set $A$ is \defemph{countable} if there is a surjection\footnote{The disjoint sum $\one + A$ with the singleton set $\one = \set{\star}$ make it possible to enumerate the~$\emptyset$ with the sequence of~$\star$'s.} $\NN \to \one + A$, which for an inhabited\footnote{A set is $A$ inhabited if there exists $x \in A$, thus not empty. The converse is not available intuitionistically.} set $A$ is equivalent to existence of a surjection $\NN \to A$. A set is \defemph{uncountable} if it is not countable. A sequence-avoiding set is uncountable.

Definitions of countability in terms of injection into~$\NN$ misbehave intuitionistically,
because a subset of a countable set need not be countable. The phenomenon appears in its extreme form in the
realizability topos over infinite-time Turing machines~\cite{bauer15:_baire}. In it the set\footnote{We allow ourselves to refer to the objects of a topos as ``sets'', especially when arguing internally to a topos. After all,  intuitionistic higher-order logic is a form of structural set theory, see~\cite[Sect.~2.2]{taylor99:_pract_found_mathem} and~\cite{lawvere03:_sets_mathem}.} of binary sequences $\Cantor$, where $\two \defeq \set{0,1}$, and the reals $\RR$ are both sequence-avoiding, respectively by \cref{cor:cantor-diagonal} and \cref{thm:R-uncountable} combined with the fact that a realizability topos validates the axiom of dependent choice.
At the same time, in this topos both $\Cantor$ and~$\RR$ embed into~$\NN$!

One should not succumb to a Skolem-style paradox~\cite{skolem23:_einig_bemer_begru_mengen} by conflating external and internal notions of countability.
%
Every set appearing in a countable model of Zermelo-Fraenkel set theory is externally countable because the whole model is, but the model validates excluded middle and hence internal uncountability of the reals.
%
In the effective topos~\cite{hyland82} the reals are internally uncountable because the topos validates Dependent choice, hence \cref{thm:R-uncountable} works again. At the same time, the reals are internally a quotient of a subset of natural numbers, and since its elements are the Turing-computable reals, there are only countably many of them, externally.
%
The moral is that we should \emph{always} consider countability internally to a topos. The interested reader may consult~\cite{blechschmidt18} for an overview of uncountability of the reals in various other toposes.

Georg Cantor gave another proof by diagonalization \cite{cantor79:_ueber_punkt}, which is intuitionistically valid.
%
(We shall mark intuitionistically valid theorems with an asterisk to keep a record of those that are applicable internally to a topos.)

\begin{theoremC}[Cantor]
  \label{cor:cantor-diagonal}
  The set of binary sequences $\Cantor$ is sequence-avoiding.
\end{theoremC}

\begin{proof}
  Let $f : \two \to \two$ be a map without fixed points, namely $f(x) \defeq 1 - x$.
  Given any $e : \NN \to \Cantor$, the map $k \mapsto f(e(k)(k))$ differs from $e(n)$ because $e(n)(n) \neq f(e(n)(n))$.
\end{proof}

Beware, uncountability of $\Cantor$ and~$\RR$ are unrelated intuitionistically.
%
We cannot exhibit a surjection in either direction,\footnote{It is consistent with intuitionistic logic~\cite[Sect.~4.6]{troelstra88:_const_mathem} that every map $\RR \to \Cantor$ is continuous, hence constant. In reverse direction, an epimorphism $\Cantor \to [0,1]$ in the topos of sheaves on~$\RR$ would have a continuous right inverse on a small enough interval, but such an inverse would have to be constant.}
and in particular we cannot show that every real has a binary, or decimal, expansion. Speaking intuitionistically, the common proofs of Cantor's theorem that apply diagonalization to decimal expansions of reals prove only that the set of decimal sequences is sequence-avoiding, not the reals themselves.

Likewise, observing that the powerset\footnote{Another surprise: $\Cantor$ and $\pow{\NN}$ need not be isomorphic. The former comprises the \emph{decidable} subsets of~$\NN$, which are those that either contain any given number or do not. In the absence of the law of excluded middle we cannot show that every subset of~$\NN$ is decidable.} $\pow{\NN}$ is sequence-avoiding says nothing about countability of~$\RR$.
%
And since a countable set may contain an uncountable subset, embedding $\Cantor$ into~$\RR$ is not helpful either.

\subsection{Real numbers in intuitionistic mathematics}
\label{sec:real-numb-intu}

The reals may be constructed in several ways~\cite[Sect.~D4.7]{johnstone02:_sketc_eleph}: the \emph{Cauchy reals} are a quotient of the set of Cauchy sequences of rationals, the \emph{Dedekind reals} are formed as Dedekind cuts of rationals, and the \emph{MacNeille reals} are the conditional order-completion of rationals.
%
Classically these all yield isomorphic ordered fields, but generally differ in a constructive setting.\footnote{The Cauchy and Dedekind reals coincide when the axiom of countable choice holds, see the paragraph preceding~\cite[Thm.~D4.7.12]{johnstone02:_sketc_eleph}.} Which ones should we use?

The real numbers, whatever they are, ought to form a metrically complete space. Thus we must disqualify the Cauchy reals, because without the axiom of countable choice we cannot show that a Cauchy sequence of Cauchy reals has a limit which is a Cauchy real.

MacNeille reals are pleasingly complete: every inhabited bounded subset has a supremum.
This can be used to prove that they are sequence-avoiding without the law of excluded middle, the axiom of choice, and without diagonalization.

\begin{theoremC}
  \label{thm:fixed-point-R-uncountable}
  %
  If every inhabited bounded set of reals has a supremum then the reals are sequence-avoiding.
\end{theoremC}

\begin{proof}
  See \cite{blechschmidt19:_knast_tarsk} for details. Briefly, given a sequence of reals $(a_n)_n$,
  the map $f : [0,2] \to [0,2]$, defined by
  %
  \begin{equation*}
    f(x) \defeq \sup
    \set{ 2^{-n_1} + \cdots + 2^{-n_k} \such
      \max \set{a_{n_1}, a_{n_2}, \ldots, a_{n_k}} < x
    },
  \end{equation*}
  %
  where $n_1 < n_2 < \cdots < n_k$,
  is well-defined by order-completeness and is monotone.
  %
  If $a_n \leq f(a_n)$ then
  %
  \begin{equation*}
    a_n + 2^{-n} \leq
    f(a_n) + 2^{-n} \leq
    f(a_n + 2^{-n}),
  \end{equation*}
  %
  therefore if a term of the sequence is a post-fixed point then it has a post-fixed point above it.
  Consequently, the largest post-fixed point of~$f$, which exists by Knaster-Tarski theorem\footnote{The Knaster-Tarski theorem is intuitionistically valid. It states that a monotone map on a complete lattice has the largest fixed-point, which is also the largest post-fixed point, i.e., the largest $x$ satisfying $x \leq f(x)$.}~\cite{Knaster28,Tarski55}, cannot be a term of the sequence.
\end{proof}

Unfortunately, the MacNeille reals may fail to be even a local ring~\cite[Thm.~D4.7.11]{johnstone02:_sketc_eleph}. We therefore adhere to the generally accepted view that the Dedekind reals are the most suitable intuitionistic notion of real numbers. We shall review their construction in detail in \cref{sec:dedek-real-numb}.

\subsection{Overview}
\label{sec:overview}

Having found no intuitionistic proofs of uncountability of the Dedekind reals, it is time to lay out a battle plan.

Our weapon of choice against the diagonalization method will be non-diagonalizable sequences constructed by Joseph Miller~\cite{miller04:_cont_deg}. Speaking imprecisely, a sequence in $[0,1]$ is non-diagonalizable when any real in $[0,1]$ computed by an oracle Turing machine, uniformly in oracles representing the sequence, already appears in the sequence.
A precise definition and Miller's construction are reviewed in \cref{sec:non-diag-sequ}.

In order to construct a topos in which a non-diagonalizable sequence appears, we devise a new kind of realizability in which realizers depend on parameters and logical formulas are realized uniformly in a given set of parameters. \Cref{sec:parameterized-part-comb} introduces parameterized partial combinatory algebras, which serve in \cref{sec:unif-real} to define parameterized realizability toposes using the tripos-to-topos construction. We felt it prudent to provide a hefty amount of details in these sections, even though the connoisseurs will recognize a variation on a familiar theme.

In \Cref{sec:real-numbers-object} we review the construction of Dedekind reals and formulate it in a way that makes it easy to compute the object of Dedekind reals in a parameterized realizability topos. We also calculate the object of Cauchy reals, and show that it is uncountable.

Victory is achieved in \cref{sec:topos-with-countable}. We focus on a particular parameterized realizability topos whose realizers are oracle-computable partial maps, with oracles ranging over representations of a given non-diagonalizable sequence~$\mil : \NN \to [0,1]$. We prove that $\mil$ appears in the topos as an epimorphism, from which countability of the reals follows.
We also show that the Hilbert cube is countable in the topos.

In \cref{sec:analysis-topos-tt} we take a closer look at the reals in the new topos.
%
First we derive the finite- and infinite-dimensional Brouwer's fixed-point theorem as trivial consequences of Lawvere's fixed-point theorem. The 1-dimensional case yields the intermediate value theorem, as well as the constructive taboo\footnote{In constructive mathematics we sometimes refer to a constructively undecided statement as a ``taboo'', especially when the statement is a consequence of excluded middle.} $\all{x \in \RR} x \leq 0 \lor x \geq 0$.
%
We show that all maps $\RR \to \RR$ are continuous, a statement known in constructive mathematics as the KLST theorem.
%
Finally, we observe that there is a countable cover of $[0,1]$ by open intervals whose lengths add up to any desired $0 < \epsilon < 1$, whence such a cover has no finite subcover. In contrast, we show that any countable cover by open intervals with rational endpoints must have a finite subcover. It is a strange topos indeed.

%%% Local Variables:
%%% mode: latex
%%% TeX-master: "countable-reals"
%%% End:
