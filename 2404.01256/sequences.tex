\section{Non-diagonalizable sequences}
\label{sec:non-diag-sequ}

As a preparation for the realizability model constructed in \cref{sec:topos-with-countable} we review the construction of non-diagonalizable sequences developed by Joseph Miller~\cite{miller04:_cont_deg}.

\subsection{Oracle-computable maps and coding of objects}
\label{sec:oracle-comp-maps}

We let lower Greek letters $\alpha$, $\beta \in \Cantor$ denote infinite binary sequences, and refer to them as \defemph{oracles}.

Given an oracle $\alpha \in \Cantor$, a partial map $f : \NN \parto \NN$ is \defemph{$\alpha$-computable} if it is computed by a Turing machine with access to the oracle~$\alpha$~\cite[Sect.~9.2]{rogers67:_theor_recur_funct_effec_comput}. Each such machine can be coded as a number, yielding a numbering $\pr[\alpha]{0}, \pr[\alpha]{1}, \pr[\alpha]{2}, \ldots$ of all partial $\alpha$-computable maps.
%
The codes describing machines are independent of the oracles. For example, there is a single index $i \in \NN$ such that $\pr[\alpha]{i} = \alpha$ for all $\alpha \in \Cantor$, and for any partial computable $f : \NN \parto \NN$ there is $j \in \NN$ such that $\pr[\alpha]{j} = f$ for all $\alpha \in \Cantor$.

Next we set up coding of mathematical objects.
%
Let $\pair{\Box, \Box} : \NN \times \NN \to \NN$ be a computable bijection, also known as a \defemph{pairing},
and let $\pi_1, \pi_2 : \NN \to \NN$ be the associated computable projections $\pi_1 \pair{m, n} = m$ and $\pi_2 \pair{m, n} = n$.
%
Let $\rat{} : \NN \to \QQ$ be a computable bijection enumerating the rationals.
%
Say that $f : \NN \to \NN$ \defemph{represents} $x \in \RR$ when
%
$\all{n \in \NN} |x - \rat{f(n)}| < 2^{-n}$.
%
That is, $f$ enumerates (codes of) rationals that converge to~$x$ with convergence modulus~$2^{-n}$. We call such a sequence \defemph{rapidly} converging.

An oracle~$\alpha$ can be construed as the binary digit expansion of a real $\rrep(\alpha) \in [0,1]$, namely
%
\begin{equation*}
  \rrep(\alpha) \defeq \sum\nolimits_{i=0}^\infty \alpha(i) \cdot 2^{-i-1}.
\end{equation*}
%
The resulting map $\rrep : \Cantor \to [0,1]$ is a continuous surjection.\footnote{It is possible to arrange for~$\rrep$ to be an open quotient map, but for our purposes a continuous map that is classically surjective will do.}
%
We may convert oracles qua binary digits expansions to representing maps, as there is $\R{w} \in \NN$ such that, for all $\alpha \in \Cantor$, the map $\pr[\alpha]{\R{w}}$ is total and it represents~$\rrep(\alpha)$. Concretely, let $\pr[\alpha]{\R{w}}(n)$ compute~$j$ such that $\rat{j} = \sum\nolimits_{i=0}^{n} \alpha(i) \cdot 2^{-i-1}$.

Finally, we provide coding of sequences $\NN \to [0,1]$ with oracles. For this purpose define $\srep : \Cantor \to [0,1]^\NN$ by
%
\begin{equation*}
  \srep(\alpha)(n) \defeq \rrep(m \mapsto \alpha(\pair{n,m}),
\end{equation*}
%
which too is a continuous surjection.
Once again we may convert oracles to representing maps, because there is~$\R{v} \in \NN$ such that $\pr[\alpha]{\pr[\alpha]{\R{v}}(n)}$ represents $\srep(\alpha)(n)$, for all $\alpha \in \Cantor$ and $n \in \NN$.


\subsection{Miller sequences}
\label{sec:miller-sequences}

Let us recall why in a duel between an oracle and diagonalization the latter wins.
%
Given $\alpha \in \Cantor$, say that~$x \in \RR$ is \defemph{$\alpha$-computable} if there is $n \in \NN$ such that $\pr[\alpha]{n}$ represents~$x$.
%
One might hope to construct an oracle~$\alpha \in \Cantor$ representing a sequence $a = \srep(\alpha) : \NN \to [0,1]$ that enumerates all $\alpha$-computable reals in~$[0,1]$, so that any $\alpha$-computable attempt to generate a real avoiding~$a$ would fail.
%
But this is not possible, because the diagonalization procedure described in the proof of~\cref{thm:R-uncountable} is itself $\alpha$-computable.
%
In particular, the choice in~\eqref{eq:R-uncountable} can be carried out $\alpha$-computably, one just has to compute a sufficiently precise rational approximation of~$a_n$.
%
The approximation, and therefore the choice and the resulting limit~$\ell$, may depend on~$\alpha$, but this in itself is not a problem.

An ingenuous insight of Joseph Miller's~\cite{miller04:_cont_deg} was that diagonalization \emph{can} be overcome, if we require oracle computations of reals to depend only on the sequence~$a$, and not the oracle representing it. In the following definition and elsewhere we write $\invim{f}$ for the inverse image map of~$f$.

\begin{definition}
  \label{def:sequence-computable}
  Given a sequence $a : \NN \to [0,1]$, say that $x \in \RR$ is \defemph{$a$-computable} if there is $n \in \NN$, called an \defemph{$a$-index}, such that $\pr[\alpha]{n}$ represents~$x$, for all $\alpha \in \invim{\srep}(a)$.
  %
  If $n$ is an $a$-index, we define $\rcomp{a}{n}$ to be the real computed by $\pr[\alpha]{n}$, for any oracle $\alpha \in \invim{\srep}(a)$. Otherwise, $\rcomp{a}{n}$ is undefined.
\end{definition}

The diagonalization procedure from the proof of~\cref{thm:R-uncountable} is not $a$-computable in the sense of the above definition. Perhaps there is another one that is? No.

\begin{theorem}[Miller]
  \label{thm:miller-sequence}%
  There exists a sequence $\mil : \NN \to [0,1]$ such that, for all $n \in \NN$, if
  $n$ is an $\mil$-index then $\mil(n) = \rcomp{\mil}{n}$.
\end{theorem}

Miller briefly mentions ``diagonally not computably diagonalizable'' as a possible name for a sequence satisfying the stated condition. We shall call it a \defemph{Miller sequence}.
%
In the remainder of this section we recount the original construction~\cite[Thm.~6.3]{miller04:_cont_deg}.

\subsubsection{The interval domain}
\label{sec:interval-domain}

Let
%
\begin{equation*}
  \II \defeq \set{[u,v] \subseteq [0,1] \such 0 \leq u \leq v \leq 1}
\end{equation*}
%
be the collection of all closed sub-intervals of~$[0,1]$.
If we think of an interval $[u,v]$ as an approximate real, then it makes sense to order~$\II$ by reverse inclusion~$\supseteq$ so that the zero-width intervals~$[u,u]$ are the maximal elements.

Not every $\pr[\alpha]{n}$ represents a real, but it can be seen to represent an element of~$\II$, as follows.
%
For $\alpha \in \Cantor$ and $n, j \in \NN$ let the \emph{$j$-th truncation} $\prx[\alpha]{n}{j} : \NN \to \set{\star} \cup \NN$ be
%
\begin{equation*}
  \prx[\alpha]{n}{j}(k) \defeq
  \begin{cases}
    \pr[\alpha]{n}(k) &
      \begin{aligned}[t]
        &\text{if the $n$-th machine with oracle~$\alpha$ applied} \\
        &\text{to~$k$ terminates in at most $j$ steps,}
      \end{aligned}
    \\
    \star &
    \text{otherwise.}
  \end{cases}
\end{equation*}
%
Define $H^\alpha_n : \NN \to \II$ by
%
\begin{equation*}
  H^\alpha_n(\pair{j, k}) \defeq
  \begin{cases}
    [\rat{m} - 2^{-k}, \rat{m} + 2^{-k}] &
      \text{if $\prx[\alpha]{n}{j}(k) = m$,} \\
    [0,1] & \text{if $\prx[\alpha]{n}{j}(k) = \star$}
  \end{cases}
\end{equation*}
% 
and $I^\alpha_n : \NN \to \II$ by $I^\alpha_n(0) \defeq [0,1]$ and
%
\begin{equation*}
  I^\alpha_n(k+1) \defeq
  \begin{cases}
    I^\alpha_n(k) \cap H^\alpha_n(k) &
      \text{if $I^\alpha_n(k) \cap H^\alpha_n(k) \neq \emptyset$}
    \\
    I^\alpha_n(k) & \text{otherwise.}
  \end{cases}
\end{equation*}
%
For any given~$n$ and~$k$, the endpoints of $I^\alpha_n(k)$ depend only on a finite prefix of~$\alpha$. Thus they are continuous in parameter~$\alpha$ with respect to the product topology on $[0,1]^\NN$ and the discrete topology on~$\QQ$.

We get a nested sequence of closed intervals
%
\begin{equation*}
  [0,1] = I^\alpha_n(0) \supseteq I^\alpha_n(1) \supseteq I^\alpha_n(2) \supseteq \cdots
\end{equation*}
%
whose intersection is a closed interval $\mathbf{I}^\alpha_n \defeq \bigcap_{k \in \NN} I^\alpha_n(k)$.
The endpoints of $\mathbf{I}^\alpha_n$ are $\alpha$-computable as \emph{lower} and \emph{upper} reals. Indeed, we can $\alpha$-computably enumerate a non-decreasing sequence of rationals whose supremum is the left-end point, and a non-increasing sequence of rationals whose infimum is the right-end point of~$\mathbf{I}^\alpha_n$.
%
Moreover, $\mathbf{I}^\alpha_n = [x,x]$ when $\pr[\alpha]{n}$ represents $x \in [0,1]$.

The story now repeats at the level of sequences. Given $a : \NN \to [0,1]$ and $n \in \NN$, for each~$\alpha$ such that $\alpha \in \invim{\srep}(a)$ the corresponding map $\pr[\alpha]{n}$ computes an interval~$\mathbf{I}^\alpha_n$ that depends on~$\alpha$, but we seek one that depends on~$a$ only.
%
The convex hull of the~$\mathbf{I}^\alpha_n$'s is the smallest interval that does the job:
%
\begin{equation*}
  \mathbf{J}^a_n \defeq
  \textstyle
  \hull \left(
    \bigcup_{\alpha \in \invim{\srep}(a)} \mathbf{I}^\alpha_n
  \right).
\end{equation*}
%
This is a closed interval because the union appearing in it is closed, even compact, for it is the projection of the set
%
\begin{equation*}
  C \defeq \set{
    (\alpha, x) \in \Cantor \times [0,1] \such
    \alpha \in \invim{\srep}(a) \land x \in \mathbf{I}^\alpha_n
  },
\end{equation*}
%
which we claim to be compact.
It suffices to check that $C$ is closed. Its membership relation is
%
\begin{align*}
  (\alpha, x) \in C
  &\liff
  \alpha \in \invim{\srep}(a) \land x \in \mathbf{I}^\alpha_n \\
  &\liff
  \alpha \in \invim{\srep}(a) \land \all{k \in \NN} x \in I^\alpha_n(k).
\end{align*}
%
This is a closed relation because~$\srep$ is continuous, and the endpoints of $I^\alpha_n(k)$ vary continuously in $\alpha$, $n$ and $k$.
%
If $n$ happens to be an $a$-index for~$x \in [0,1]$, then $\mathbf{J}^a_n = [x,x]$ because $\mathbf{I}^\alpha_n = [x,x]$ for all $\alpha \in \invim{\srep}(a)$.
%

The endpoints of $\mathbf{J}^a_n$ are more complicated than those of $\mathbf{I}^\alpha_n$. There is an $\alpha$-computable double sequence of rationals $q_{i,j}$ such that the left endpoint is $\inf_i \sup_j q_{i,j}$, and dually for the right endpoint.


\subsubsection{Construction of a Miller sequence}
\label{sec:constr-mill-sequ}

We prove \cref{thm:miller-sequence} by using the following generalization of Kakutani's fixed-point theorem, which itself is a generalization of Brouwer's fixed-point theorem. Depending on one's point of view, it is ironic or fascinating that such very classical theorems\footnote{Brouwer's fixed-point theorem has no constructive proofs, because a result of Orevkov's~\cite{orevkov63} implies that in the effective topos there is a continuous map $[0,1]^2 \to [0,1]^2$ which moves every point by a positive distance.} are used to construct an intuitionistic topos.

\begin{theorem}
  \label{thm:generalized-Brouwer}%
  If $F \subseteq [0,1]^\NN \times [0,1]^\NN$ is a closed set such that for each $a \in [0,1]^\NN$, the set $F[a] \defeq \set{b \in [0,1]^\NN \such (a, b) \in F}$ is non-empty and convex, then there is $\mil \in [0,1]^\NN$ such that $(\mil,\mil) \in F$.
\end{theorem}

\begin{proof}
  The statement is not easy to credit properly; see the paragraph after \cite[Thm~6.1]{miller04:_cont_deg} for a discussion which proposes \cite{Eilenberg1946} as the earliest work implying the statement given here.
\end{proof}

\Cref{thm:generalized-Brouwer} is a fixed-point theorem because a closed set $F \subseteq [0,1]^\NN \times [0,1]^\NN$ can be construed as the graph of an upper semicontinuous multivalued map taking each $a \in [0,1]^\NN$ to the non-empty set $F[a]$.
%
The sequence~$\mil$ is a fixed point in the sense that $\mil \in F[\mil]$.

In our case, we take $F$ to be essentially $\mathbf{J}$:
%
\begin{equation*}
  F \defeq \set{(a, b) \in [0,1]^\NN \times [0,1]^\NN \such \all{n \in \NN} b(n) \in \mathbf{J}^a_n},
\end{equation*}
%
or expressed as a multivalued map,
%
\begin{equation*}
  \textstyle
  F[a] \defeq \prod_{n \in \NN} \mathbf{J}^a_n.
\end{equation*}
%
% Verification the above are the same thing:
% b ∈ F[a] iff
% ∀ n . b(n) ∈ J^a_n iff
% (a, b) ∈ F
%
Let us verify the conditions of the theorem.
%
Obviously, $F[a]$ is non-empty and convex for all $a \in [0,1]^\NN$.
%
To see that~$F$ is closed, we unravel its definition in logical form:
%
\begin{align*}
  (a, b) \in F
  &\liff \all{n \in \NN} b(n) \in \mathbf{J}^a_n \\
  &\liff\textstyle
    \all{n \in \NN} b(n) \in
    \hull \left(
      \bigcup \set{ \mathbf{I}^\alpha_n \such \srep(\alpha) = a }
    \right) \\
  &\liff\textstyle
    \begin{aligned}[t]
      &\all{n \in \NN}
      \some{u, v \in [0,1]}
      u \leq b(n) \leq v \land {} \\
      &\quad
      (\some{\alpha \in \invim{\srep}(a)} u \in \mathbf{I}^\alpha_n)
      \land
      (\some{\beta \in \invim{\srep}(a)} v \in \smash{\mathbf{I}^\beta_n})
    \end{aligned}
  \\
  &\liff\textstyle
    \begin{aligned}[t]
      &\all{n \in \NN}
      \some{u, v \in [0,1]}
      u \leq b(n) \leq v \land {} \\
      &\quad
      (\some{\alpha \in \invim{\srep}(a)} \all{m \in \NN} u \in I^\alpha_n(m))
      \land {} \\
      &\quad
      (\some{\beta \in \invim{\srep}(a)} \all{m \in \NN} v \in I^\beta_n(m))
    \end{aligned}
\end{align*}
%
This is a closed condition: $\forall$ and $\land$ correspond to intersection, $\exists \alpha \in \invim{\srep}(a)$ to projection along the compact set $\invim{\srep}(a) \defeq \set{ \alpha \in \Cantor \such \srep(\alpha) = a}$, the relation $\leq$ is closed, projecting the $n$-th component $b(n)$ is continuous, and the endpoints of $I^\alpha_n(m)$ depend continuously on its parameters, in particular~$\alpha$.

It remains to verify that a fixed point $\mil \in F[\mil]$ is a Miller sequence. If $n \in \NN$ is a $\mil$-index of $x \in [0,1]$ then $\mil \in F[\mil]$ implies $\mil(n) \in \mathbf{J}^a_n = [x, x]$, hence $\mil(n) = x$, as required.

%%% Local Variables:
%%% mode: latex
%%% TeX-master: "countable-reals"
%%% End:
