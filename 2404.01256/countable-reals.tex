\documentclass[12pt]{amsart}

% For e-reader, small margins
%\usepackage[margin=5mm]{geometry}

\usepackage[T1]{fontenc}
\usepackage[utf8]{inputenc}

\usepackage{tikz}

% Fancy math fonts
\usepackage{amsmath}
\usepackage{amsfonts}
\usepackage{amssymb}
\usepackage{upgreek}
\usepackage{xfrac} % for neat fraction \sfrac

\usepackage{amsaddr} % place author info on the first page

% I am not sure about all these fonts (Andrej)
\usepackage[helvratio=.93,p,theoremfont]{newpxtext} % Serif palatino font
\usepackage[vvarbb,smallerops,bigdelims]{newpxmath} % Math palatino font
\usepackage[scaled=.85]{beramono} % Monospace font
\usepackage[scr=rsfso,cal=boondoxo]{mathalfa} % Mathcal from STIX, unslanted a bit

% A simpler choice of fonts:
% \usepackage{times}

\usepackage[english]{babel}

\usepackage{amsthm}
\usepackage[hidelinks]{hyperref}
\usepackage[capitalise,nameinlink]{cleveref}
\usepackage{xypic}
\usepackage{xcolor}
\usepackage[shortlabels]{enumitem}

%\usepackage{amsmath,amsthm,amscd,amssymb}
\usepackage[papersize={6.8in, 10.0in}, left=.5in, right=.5in, top=.6in, bottom=.9in]{geometry}
\linespread{1.1}
\sloppy
\raggedbottom
\pagestyle{plain}

% these include amsmath and that can cause trouble in older docs.
\input{../helpers/cmrsum}
\makeatletter

\DeclareFontFamily{OMX}{MnSymbolE}{}
\DeclareSymbolFont{largesymbolsX}{OMX}{MnSymbolE}{m}{n}
\DeclareFontShape{OMX}{MnSymbolE}{m}{n}{
    <-6>  MnSymbolE5
   <6-7>  MnSymbolE6
   <7-8>  MnSymbolE7
   <8-9>  MnSymbolE8
   <9-10> MnSymbolE9
  <10-12> MnSymbolE10
  <12->   MnSymbolE12}{}

\DeclareMathSymbol{\downbrace}    {\mathord}{largesymbolsX}{'251}
\DeclareMathSymbol{\downbraceg}   {\mathord}{largesymbolsX}{'252}
\DeclareMathSymbol{\downbracegg}  {\mathord}{largesymbolsX}{'253}
\DeclareMathSymbol{\downbraceggg} {\mathord}{largesymbolsX}{'254}
\DeclareMathSymbol{\downbracegggg}{\mathord}{largesymbolsX}{'255}
\DeclareMathSymbol{\upbrace}      {\mathord}{largesymbolsX}{'256}
\DeclareMathSymbol{\upbraceg}     {\mathord}{largesymbolsX}{'257}
\DeclareMathSymbol{\upbracegg}    {\mathord}{largesymbolsX}{'260}
\DeclareMathSymbol{\upbraceggg}   {\mathord}{largesymbolsX}{'261}
\DeclareMathSymbol{\upbracegggg}  {\mathord}{largesymbolsX}{'262}
\DeclareMathSymbol{\braceld}      {\mathord}{largesymbolsX}{'263}
\DeclareMathSymbol{\bracelu}      {\mathord}{largesymbolsX}{'264}
\DeclareMathSymbol{\bracerd}      {\mathord}{largesymbolsX}{'265}
\DeclareMathSymbol{\braceru}      {\mathord}{largesymbolsX}{'266}
\DeclareMathSymbol{\bracemd}      {\mathord}{largesymbolsX}{'267}
\DeclareMathSymbol{\bracemu}      {\mathord}{largesymbolsX}{'270}
\DeclareMathSymbol{\bracemid}     {\mathord}{largesymbolsX}{'271}

\def\horiz@expandable#1#2#3#4#5#6#7#8{%
  \@mathmeasure\z@#7{#8}%
  \@tempdima=\wd\z@
  \@mathmeasure\z@#7{#1}%
  \ifdim\noexpand\wd\z@>\@tempdima
    $\m@th#7#1$%
  \else
    \@mathmeasure\z@#7{#2}%
    \ifdim\noexpand\wd\z@>\@tempdima
      $\m@th#7#2$%
    \else
      \@mathmeasure\z@#7{#3}%
      \ifdim\noexpand\wd\z@>\@tempdima
        $\m@th#7#3$%
      \else
        \@mathmeasure\z@#7{#4}%
        \ifdim\noexpand\wd\z@>\@tempdima
          $\m@th#7#4$%
        \else
          \@mathmeasure\z@#7{#5}%
          \ifdim\noexpand\wd\z@>\@tempdima
            $\m@th#7#5$%
          \else
           #6#7%
          \fi
        \fi
      \fi
    \fi
  \fi}

\def\overbrace@expandable#1#2#3{\vbox{\m@th\ialign{##\crcr
  #1#2{#3}\crcr\noalign{\kern2\p@\nointerlineskip}%
  $\m@th\hfil#2#3\hfil$\crcr}}}
\def\underbrace@expandable#1#2#3{\vtop{\m@th\ialign{##\crcr
  $\m@th\hfil#2#3\hfil$\crcr
  \noalign{\kern2\p@\nointerlineskip}%
  #1#2{#3}\crcr}}}

\def\overbrace@#1#2#3{\vbox{\m@th\ialign{##\crcr
  #1#2\crcr\noalign{\kern2\p@\nointerlineskip}%
  $\m@th\hfil#2#3\hfil$\crcr}}}
\def\underbrace@#1#2#3{\vtop{\m@th\ialign{##\crcr
  $\m@th\hfil#2#3\hfil$\crcr
  \noalign{\kern2\p@\nointerlineskip}%
  #1#2\crcr}}}

\def\bracefill@#1#2#3#4#5{$\m@th#5#1\leaders\hbox{$#4$}\hfill#2\leaders\hbox{$#4$}\hfill#3$}

\def\downbracefill@{\bracefill@\braceld\bracemd\bracerd\bracemid}
\def\upbracefill@{\bracefill@\bracelu\bracemu\braceru\bracemid}

\DeclareRobustCommand{\downbracefill}{\downbracefill@\textstyle}
\DeclareRobustCommand{\upbracefill}{\upbracefill@\textstyle}

\def\upbrace@expandable{%
  \horiz@expandable
    \upbrace
    \upbraceg
    \upbracegg
    \upbraceggg
    \upbracegggg
    \upbracefill@}
\def\downbrace@expandable{%
  \horiz@expandable
    \downbrace
    \downbraceg
    \downbracegg
    \downbraceggg
    \downbracegggg
    \downbracefill@}

\DeclareRobustCommand{\overbrace}[1]{\mathop{\mathpalette{\overbrace@expandable\downbrace@expandable}{#1}}\limits}
\DeclareRobustCommand{\underbrace}[1]{\mathop{\mathpalette{\underbrace@expandable\upbrace@expandable}{#1}}\limits}

\makeatother


\usepackage{microtype}

% hyperref last because otherwise some things go wrong.
\usepackage{hyperref}
\hypersetup{colorlinks=true
,breaklinks=true
,urlcolor=blue
,anchorcolor=blue
,citecolor=blue
,filecolor=blue
,linkcolor=blue
,menucolor=blue
,linktocpage=true}
\hypersetup{
bookmarksopen=true,
bookmarksnumbered=true,
bookmarksopenlevel=10
}

% make sure there is enough TOC for reasonable pdf bookmarks.
\setcounter{tocdepth}{3}

%\usepackage[dotinlabels]{titletoc}
%\titlelabel{{\thetitle}.\quad}
%\titleformat{\section}[block]
  {\fillast\medskip}
  {{\thesection. }}
  {1ex minus .1ex}
  {\scshape}
 
\titleformat*{\subsection}{\itshape}
\titleformat*{\subsubsection}{\itshape}

\setcounter{tocdepth}{2}

\titlecontents{section}
              [2.3em] 
              {\bigskip}
              {{\contentslabel{2.3em}}\large\scshape}
              {\hspace*{-2.3em}}
              {\titlerule*[1pc]{}\contentspage}
              
\titlecontents{subsection}
              [4.7em] 
              {}
              {{\contentslabel{2.3em}}}
              {\hspace*{-2.3em}}
              {\titlerule*[.5pc]{}\contentspage}

% hopefully not used.           
\titlecontents{subsubsection}
              [7.9em]
              {}
              {{\contentslabel{3.3em}}}
              {\hspace*{-3.3em}}
              {\titlerule*[.5pc]{}\contentspage}
%\makeatletter
\renewcommand\tableofcontents{%
    \section*{\contentsname
        \@mkboth{%
           \MakeLowercase\contentsname}{\MakeLowercase\contentsname}}%
    \@starttoc{toc}%
    }
\def\@oddhead{{\scshape\rightmark}\hfil{\small\scshape\thepage}}%
\def\sectionmark#1{%
      \markright{\MakeLowercase{%
        \ifnum \c@secnumdepth >\m@ne
          \thesection\quad
        \fi
        #1}}}
        
\makeatother



%\makeatletter

 \def\small{%
  \@setfontsize\small\@xipt{13pt}%
  \abovedisplayskip 8\p@ \@plus3\p@ \@minus6\p@
  \belowdisplayskip \abovedisplayskip
  \abovedisplayshortskip \z@ \@plus3\p@
  \belowdisplayshortskip 6.5\p@ \@plus3.5\p@ \@minus3\p@
  \def\@listi{%
    \leftmargin\leftmargini
    \topsep 9\p@ \@plus3\p@ \@minus5\p@
    \parsep 4.5\p@ \@plus2\p@ \@minus\p@
    \itemsep \parsep
  }%
}%
 \def\footnotesize{%
  \@setfontsize\footnotesize\@xpt{12pt}%
  \abovedisplayskip 10\p@ \@plus2\p@ \@minus5\p@
  \belowdisplayskip \abovedisplayskip
  \abovedisplayshortskip \z@ \@plus3\p@
  \belowdisplayshortskip 6\p@ \@plus3\p@ \@minus3\p@
  \def\@listi{%
    \leftmargin\leftmargini
    \topsep 6\p@ \@plus2\p@ \@minus2\p@
    \parsep 3\p@ \@plus2\p@ \@minus\p@
    \itemsep \parsep
  }%
}%
\def\open@column@one#1{%
 \ltxgrid@info@sw{\class@info{\string\open@column@one\string#1}}{}%
 \unvbox\pagesofar
  \gdef\thepagegrid{one}%
 \global\pagegrid@col#1%
 \global\pagegrid@cur\@ne
 \global\count\footins\@m
 \set@column@hsize\pagegrid@col
 \set@colht
}%

\def\frontmatter@abstractheading{%
\bigskip
 \begingroup
  \centering\large
  \abstractname
  \par\bigskip
 \endgroup
}%

\makeatother

%\DeclareSymbolFont{CMlargesymbols}{OMX}{cmex}{m}{n}
%\DeclareMathSymbol{\sum}{\mathop}{CMlargesymbols}{"50}
%\pdfbookmark[1]{Introduction}{Introduction}

%%% Theorem-like environments
{\theoremstyle{plain}
\newtheorem{theorem}{Theorem}[section]
\newtheorem{theoremC}[theorem]{Theorem${}^\star$}
\newtheorem{proposition}[theorem]{Proposition}
\newtheorem{propositionC}[theorem]{Proposition${}^\star$}
\newtheorem{corollary}[theorem]{Corollary}
\newtheorem{corollaryC}[theorem]{Corollary${}^\star$}
\newtheorem{lemma}[theorem]{Lemma}
\newtheorem{lemmaC}[theorem]{Lemma${}^\star$}
\newtheorem{fact}[theorem]{Fact}
}

{\theoremstyle{definition}
\newtheorem{definition}[theorem]{Definition}
\newtheorem{example}[theorem]{Example}
}

\crefname{theoremC}{Theorem}{Theorems}
\crefname{propositionC}{Proposition}{Propositions}
\crefname{corollaryC}{Corollary}{Corollaries}
\crefname{lemmaC}{Lemma}{Lemmas}


% Equations numbered with sections
\numberwithin{equation}{section}

% Macros shared among templates

\usepackage[utf8]{inputenc}

\usepackage{graphicx}
\setkeys{Gin}{width=\linewidth,totalheight=\textheight,keepaspectratio}

\definecolor{darkblue}{HTML}{00416A}

\usepackage{longtable}
\usepackage{booktabs}
\usepackage{amssymb}
\usepackage{amsmath}
\usepackage{amsthm}
%\usepackage{commath}
\usepackage{boxedminipage}
\usepackage{microtype}

\usepackage{makeidx}
\usepackage{hyperref}

% attempts to prevent margin figures from being cut off
\usepackage{marginfix}
\usepackage[morefloats=100]{morefloats}

\newtheorem{Definition}{Definition}
\newtheorem{Theorem}{Theorem}
\newtheorem{Lemma}{Lemma}
\newtheorem{Exercise}{Exercise}
\newtheorem{Fact}{Fact}
\newtheorem{Proposition}{Proposition}
\newtheorem{Assumption}{Assumption}
\newenvironment{Algorithm}{\begin{center}\begin{boxedminipage}{0.92\textwidth}}{\end{boxedminipage}\end{center}}
\newenvironment{Proof}{\begin{proof}}{\end{proof}}
\newtheorem*{summary*}{Summary}
\newenvironment{Summary}{\begin{center}\begin{minipage}{0.92\textwidth}\begin{summary*}}{\end{summary*}\end{minipage}\end{center}\medskip}
\newenvironment{EmphBox}{\begin{center}\begin{minipage}{0.8\textwidth}}{\end{minipage}\end{center}\medskip}

\providecommand{\tightlist}{%
  \setlength{\itemsep}{0pt}\setlength{\parskip}{0pt}}

\renewcommand{\bot}{\perp}
\renewcommand{\hat}{\widehat}



%%%%%%%%%%%%%%%%%%%%%%%%%%%%%%%%%%%%%%%%%%%%%%%%%%

\title{The countable reals}

\author{Andrej Bauer}
% It is likely there is a better way to specify double affiliation
\address{Faculty of Mathematics and Physics, University of Ljubljana, Slovenia}
\address{Institute of Mathematics, Physics and Mechanics, Slovenia}
\email{Andrej.Bauer@andrej.com}
% amsaddr package seems unable to handle \urladdr, so it just gets placed at the end
% \urladdr{https://www.andrej.com/}

\thanks{This material is based upon work supported by the Air Force Office of Scientific Research under award number FA9550-21-1-0024.}

\author{James E.~Hanson}
\email{jhanson9@umd.edu}
% amsaddr package seems unable to handle \urladdr, so it just gets placed at the end
%\urladdr{https://james-hanson.github.io}
\address{University of Maryland, College Park, USA}

\begin{document}

\begin{abstract}
  We construct a topos in which the Dedekind reals are countable.

  To accomplish this, we first define a new kind of toposes that we call \emph{parameterized realizability toposes}. They are built from partial combinatory algebras whose application operation depends on a parameter, and in which realizers operate uniformly with respect to a given parameter set.

  Our topos is the parameterized realizability topos whose realizers are oracle-computable partial maps, with oracles serving as parameters and ranging over the representations of a non-diagonalizable sequence, discovered by Joseph Miller. It is a sequence of reals in $[0,1]$ that is non-diagonalizable in the sense that any real in $[0,1]$ that is oracle-computable, uniformly in oracles representing the sequence, must already appear in the sequence.
  %
  The Dedekind reals are countable in the topos because the non-diagonalizable sequence appears in it as an epimorphism.

  The topos is intuitionistic, as it invalidates both the law of excluded middle and the axiom of countable choice. The Cauchy reals are uncountable. The Hilbert cube is countable, from which Brouwer's fixed-point theorem follows as an easy corollary of Lawvere's fixed-point theorem. From the 1-dimensional Brouwer's fixed-point theorem we obtain the intermediate value theorem and the lesser limited principle of omniscience. The Kreisel-Lacombe-Shoenfield-Tseitin theorem stating that all real-valued maps are continuous is valid, because the usual proof is uniform with respect to oracles.
  %
  Lastly, the closed interval $[0,1]$, being countable, can trivially be covered by a sequence of open intervals whose lengths add up to any prescribed $0 < \epsilon < 1$, and such a cover has no finite subcover.
  However, we show that any sequence of open intervals with rational endpoints covering $[0,1]$ must has a finite subcover.
\end{abstract}

\maketitle



\section{Introduction \label{sec:introduction}}

When probed at very short wavelengths, QCD is essentially a theory of
free \index{Partons}`partons' --- quarks and gluons --- which only
scatter off one another through relatively small quantum corrections,
that can be systematically calculated. 
But at longer wavelengths, of order the size of the proton $\sim
1\mathrm{fm} = 10^{-15}\mathrm{m}$,  
we see strongly bound towers of hadron resonances emerge, with string-like
potentials building up if we try to separate their partonic
constituents. Due to our
inability to perform analytic calculations in 
strongly coupled field theories, QCD is therefore 
still only partially solved. Nonetheless,  all its features, across all
distance scales, are believed to be encoded in a single one-line
formula of alluring simplicity; the
\index{QCD!Lagrangian}%
Lagrangian\footnote{Throughout these notes we let it be implicit that
  ``Lagrangian'' really refers to Lagrangian density, ${\cal L}$, the
  four-dimensional space-time integral of which is the action.} of QCD.

The consequence for collider physics is that some parts of QCD can be
calculated in terms of the fundamental parameters of the Lagrangian,
whereas others must be expressed through models or functions whose effective 
parameters are not a priori calculable but which can be constrained
by fits to data. 
However, even in the absence of a
perturbative expansion, there are still several strong theorems which
hold, and which can be used to give relations between seemingly
different processes. (This is, e.g., the reason it makes sense to 
measure the partonic substructure of the proton in $ep$ collisions and
then re-use the same parametrisations for $pp$
collisions.) Thus, in the chapters 
dealing with phenomenological models we shall emphasise that the loss
of a factorised perturbative expansion is not equivalent to a total
loss of predictivity.   

An alternative approach would be to give up on calculating QCD 
and use leptons instead. Formally, this amounts to summing inclusively over
strong-interaction phenomena, when such are present. While such a
strategy might succeed in replacing what we do know about QCD by
``unity'', however, even the most adamant chromophobe would acknowledge
the following basic facts of collider physics for the next decade(s): 
1) At the LHC, the initial states are
hadrons, and hence, at
the very least, well-understood and precise parton distribution
functions (PDFs) will be required; 2) high precision will mandate
 calculations to higher orders in perturbation theory, 
which in turn will involve more QCD; 3) the requirement of lepton
\emph{isolation} makes the very definition of a lepton
 depend implicitly on QCD and 4) 
 the rate of jets that are misreconstructed as leptons in
 the experiment depends explicitly on it. 
And, 5) though many new-physics signals \emph{do} give observable
signals in the lepton sector, this is far from guaranteed, nor is it
exclusive when it occurs. 
 It would therefore be  unwise not to attempt to solve QCD to the best
 of our ability, the better to prepare ourselves for both the largest
 possible discovery reach and the highest attainable subsequent
 precision. 

Furthermore, QCD is the richest gauge theory we have so far
 encountered. Its emergent phenomena, unitarity properties, colour structure, 
 non-perturbative dynamics, quantum vs.\ classical limits, 
interplay between scale-invariant and
 scale-dependent properties, and its wide
 range of phenomenological applications, are still very much topics of
 active investigation, about which we continue to learn.  

In addition, or perhaps as a consequence, the field of QCD is
currently experiencing something of a revolution. On the perturbative
side, new methods to compute scattering amplitudes with very high
particle multiplicities are being developed, together with advanced
techniques for combining such amplitudes with all-orders resummation
frameworks. On the non-perturbative side, the wealth of data on
soft-physics processes from the LHC is
forcing us to reconsider the reliability of the standard fragmentation
models, and heavy-ion collisions are providing new insights into
the collective behavior of hadronic matter. The
study of cosmic rays impinging on the Earth's
atmosphere challenges our ability to extrapolate fragmentation models
from collider energy scales to the region of ultra-high energy cosmic
rays. And finally, dark-matter annihilation processes in space  may produce 
hadrons, whose spectra are sensitive to the modeling 
of fragmentation.

In the following, we shall focus on QCD for mainstream 
collider physics. This includes the basics of SU(3), colour factors, the running
of $\alpha_s$, factorisation, 
hard processes, infrared safety, parton showers and matching, event generators, hadronisation, and the so-called underlying event. 
While not covering everything, hopefully these topics can also serve
at least as stepping stones to more specialised
issues that have been left out, such as twistor-inspired techniques, 
heavy flavours, polarisation, or forward physics, or to topics more tangential to
other fields, such as axions, lattice QCD, or heavy-ion physics.  

\subsection{A First Hint of Colour}
Looking for new physics, as we do now at the LHC, it is instructive to 
consider the story of the discovery of colour. The first hint was
arguably the $\Delta^{++}$ \index{Baryons}baryon, discovered in 
1951~\cite{Brueckner:1952zz}. The title and part of the abstract from this
historical paper are reproduced in \figRef{fig:Delta}.
\begin{figure}[t]
\begin{center}
\begin{tabular}{c}
\colorbox{gray}{\includegraphics*[scale=0.75]{DeltaTitle.pdf}}\\[5mm]
\hspace*{2mm}\begin{minipage}{0.88\textwidth}
\small\sl  ``[...] It is concluded that the apparently anomalous features of the
scattering can be interpreted to be an indication of a resonant
meson-nucleon interaction corresponding to a nucleon isobar with spin
$\frac32$, isotopic spin $\frac32$, and with an excitation energy of
$277\,$MeV.''\\[1mm]
\end{minipage}
\end{tabular}
\caption{The title and part of the abstract of the 1951 paper
  \cite{Brueckner:1952zz} (published in 1952) in which the existence 
  of the $\Delta^{++}$ baryon was deduced, based on data from Sachs and
  Steinberger at Columbia~\cite{Chedester:1951sc}  and from Anderson,
  Fermi, Nagle, et al.~at Chicago~\cite{Fermi:1952zz}. Further studies 
  at Chicago were quickly performed
  in~\cite{Anderson:1952nw,Anderson:1952zza}. See also the memoir by
  Nagle~\cite{nagle1984delta}. 
\label{fig:Delta}}  
\end{center}
\end{figure}
In the context of the \index{Quarks}quark model --- which first
had to be developed, successively joining together the notions of 
spin, isospin, strangeness, and 
the \index{Eightfold way}eightfold way\footnote{In physics, the ``eightfold way''
refers to the classification of the lowest-lying pseudoscalar
\index{Mesons}mesons and 
\index{SU(3)!Of Flavour}%
spin-1/2 \index{Baryons}baryons within \index{Octet}octets in SU(3)-flavour space ($u,d,s$). The
$\Delta^{++}$ is part of a spin-3/2 baryon \index{Decuplet}decuplet, a ``tenfold way'' in this
terminology.} 
--- the \index{Flavour}flavour and spin content of the $\Delta^{++}$
baryon is: 
\begin{equation}
\left\vert \Delta^{++} \right> = \left\vert
\,u_\uparrow\ u_\uparrow\ u_\uparrow \right>~,
\end{equation} 
clearly a highly symmetric configuration. However, since 
the $\Delta^{++}$ is a fermion, it must have an overall
antisymmetric wave function. In 1965, fourteen years after its
discovery, this was finally understood by the introduction of colour
\index{SU(3)}%
\index{SU(3)!Of Colour}%
as a new quantum number associated with the group SU(3)
\cite{Greenberg:1964pe,Han:1965pf}. The $\Delta^{++}$ wave function can now be made
antisymmetric by arranging its three quarks antisymmetrically 
in this new degree of freedom, 
\begin{equation}
\left\vert \Delta^{++} \right> = \epsilon^{ijk} \left\vert
\,u_{i\uparrow}\ u_{j\uparrow}\ u_{k\uparrow}\right>~,
\end{equation} 
hence solving the mystery.

More direct experimental tests of the number of colours were provided first by
measurements of the decay width of $\pi^0\to \gamma\gamma$ decays, which 
is proportional to $N_C^2$, 
and later by the famous ``R'' ratio in
$e^+e^-$ collisions ($R=\sigma(e^+e^-\to q\bar{q})/\sigma(e^+e^-\to
\mu^+\mu^-)$), which is proportional to $N_C$, see
e.g.~\cite{Dissertori:2003pj}. 
Below, in \SecRef{sec:L} we shall see how to
calculate such colour factors. 

\subsection{The Lagrangian of QCD \label{sec:L}}
\index{QCD!Lagrangian}%
Quantum Chromodynamics is based on the gauge group
\index{SU(3)}$\mrm{SU(3)}$, the 
Special Unitary group in 3 (complex) dimensions, whose elements 
are the set of unitary $3\times 3$ matrices with determinant one. 
\index{Fundamental representation}%
\index{SU(3)!Fundamental representation}%
Since there are 9 linearly independent unitary complex
matrices\footnote{A complex $N\times N$ matrix has $2N^2$ degrees of
  freedom, on which unitarity provides $N^2$ constraints.}, one of
which has determinant $-1$, there are a total of 8
independent directions in this matrix space, corresponding to eight
different generators as compared
with the single one of QED. In the context of QCD, we normally
represent this group using the 
so-called \emph{fundamental}, or \emph{defining}, representation, in
which the generators of $\mrm{SU(3)}$ appear as a set of eight traceless and
hermitean matrices, to which we return below.  
We shall refer to indices enumerating
the rows and columns of these matrices  (from 1 to 3) as
\emph{fundamental} indices, and we use the letters $i$,
$j$, $k$, \ldots, to denote them.
\index{Adjoint representation}%
\index{SU(3)!Adjoint representation}%
We refer to indices enumerating the generators (from 1 to 8),
as \emph{adjoint} 
indices\footnote{The dimension of the \emph{adjoint}, or
  \emph{vector}, representation is equal to the number of generators,
  $N^2-1=8$ for $\mrm{SU(3)}$, while the  
\index{Fundamental representation}%
\index{SU(3)!Fundamental representation}%
dimension of the fundamental representation is
  the degree of the group, $N=3$ for $\mrm{SU(3)}$.}, and we use the first
letters of the alphabet ($a$, $b$, $c$, \ldots) to denote them. 
These matrices can operate both on each other (representing
combinations of successive gauge transformations) and on a set of
$3$-vectors, the latter of 
which represent \index{Quarks}quarks in colour 
space; the quarks are \emph{triplets} under $\mrm{SU(3)}$. The matrices can be
thought of as representing gluons in colour 
space (or, more precisely, the gauge transformations carried out by
gluons), hence there are
eight different gluons; the gluons are \emph{octets} under $\mrm{SU(3)}$. 

\index{QCD!Lagrangian}%
The Lagrangian density of QCD is 
\begin{equation}
{\cal L} = \bar{\psi}_q^i(i\gamma^\mu)(D_\mu)_{ij}\psi_q^j - m_q
\bar{\psi}_q^i\psi_{qi} - \frac14 F^a_{\mu\nu}F^{a\mu\nu}~,\label{eq:L}
\end{equation}
where $\psi_q^i$ denotes a quark field with
(fundamental) colour index $i$, 
$\psi_q = ({\textcolor{red}{\psi_{qR}}},{\color{green}\psi_{qG}}, 
{\color{blue}\psi_{qB}})^T$, 
$\gamma^\mu$ is a Dirac matrix that expresses the
vector nature of the strong interaction, with $\mu$ being a Lorentz
vector index, $m_q$ allows for the
possibility of non-zero \index{Quarks}quark masses (induced by the
standard Higgs 
mechanism or similar), $F^a_{\mu\nu}$ is the gluon field strength 
tensor for a gluon\footnote{The definition of the gluon field strength
  tensor will be given below in \eqRef{eq:F}.} with (adjoint) 
colour index $a$ (i.e., $a\in[1,\ldots,8]$), 
and $D_\mu$ is the covariant derivative in QCD,
\begin{equation}
(D_{\mu})_{ij} = \delta_{ij}\partial_\mu - i g_s t_{ij}^a A_\mu^a~,\label{eq:D}
\end{equation}
\index{QCD!Coupling}
with $g_s$ the \index{alphaS@$\alpha_s$}strong coupling (related to
$\alpha_s$ by $g_s^2 = 4\pi 
\alpha_s$; we return to the strong coupling in more detail below), 
$A^a_\mu$  the gluon field with 
colour index $a$, and $t_{ij}^a$ proportional to the hermitean and
traceless \index{Gell-Mann matrices|see{SU(3)}}Gell-Mann matrices of $\mrm{SU(3)}$, 
\index{SU(3)!Generators}%
\begin{equation}
\mbox{\includegraphics*[scale=1.0]{gell-mann}}~.
\end{equation}
These generators are just the $\mrm{SU(3)}$ analogs of the
Pauli matrices in 
$\mrm{SU(2)}$. 
By convention, the constant of proportionality is normally
taken to 
be 
\begin{equation}
t^a_{ij} = \frac12 \lambda^a_{ij}~. \label{eq:t}
\end{equation}
\index{QCD!Coupling}
This choice in turn determines the normalisation of the coupling
$g_s$, via \eqRef{eq:D}, and
fixes the values of the $\mrm{SU(3)}$ \index{Casimirs}Casimirs and structure constants, to which we return below. 

An example of the colour flow for a
quark-gluon interaction in colour 
space is given in \figRef{fig:qg}.
\begin{figure}[t]
\begin{center}
\begin{minipage}[h]{4.6cm}
\begin{center}
$A^1_\mu$\\
\includegraphics*[scale=0.75]{qgv.pdf}\\[-3mm]
$\psi_{q\textcolor{green}{G}}$\hfill$\psi_{q\textcolor{red}{R}}$
\end{center}
\end{minipage}~~~
\parbox{0.4\textwidth}{
$
\begin{array}{ccccc}
\propto & - \frac{i}{2} g_s & \bar{\psi}_{q\color{red}R}  & \lambda^{1} & \psi_{q\color{green}G} 
\\[2mm]
= & -\frac{i}{2}g_s & \left(\begin{array}{ccc} \textcolor{red}{1} & \color{green} 0 &
  \color{blue} 0 
\end{array}\right) & 
\left(\begin{array}{ccc}
0 & 1 & 0  \\
1 & 0 & 0 \\
0 & 0 & 0
\end{array}\right) & 
 \left(\begin{array}{c}
\textcolor{red}{0} \\
\color{green}1 \\
\color{blue}0
\end{array}\right) \end{array}
$}
\caption{Illustration of a 
\index{Quarks}\index{Gluons}$qqg$ vertex in QCD, before
  summing/averaging over colours: a gluon in a state represented by $\lambda^1$
  interacts with quarks in the states $\psi_{qR}$ and
  $\psi_{qG}$. \label{fig:qg}}
\end{center}
\end{figure}
Normally, of course, we sum over all the colour indices, so this
example merely gives a pictorial representation of what one particular
(non-zero) term in the colour sum looks like.


\subsection{Colour Factors}
\index{QCD!Colour factors}
\index{Colour factors}%
\index{Colour-space indices|see{Colour connections}}%
\index{Matrix elements}%
Typically, we do not measure colour in the final state ---
instead we average over all possible incoming colours and sum over all
possible outgoing ones, wherefore QCD scattering amplitudes (squared) in
practice always contain sums over quark fields contracted with
\index{SU(3)!Generators}Gell-Mann matrices. These contractions in turn
produce traces  
which yield the \index{Colour factors}\emph{colour factors} that are associated to each QCD
process, and which basically count the number of ``paths through
colour space'' that the process at hand can take\footnote{The
  convention choice represented by \eqRef{eq:t} introduces a
  ``spurious'' factor of 2 for each power of the coupling $\alpha_s$. 
Although one could in principle absorb that factor into a redefinition
of the coupling, effectively redefining the normalisation of ``unit
colour charge'', the standard definition of $\alpha_s$ is now so
entrenched that alternative choices would be counter-productive, at
least in the context of a pedagogical review.}.

A very simple example of a colour factor is given by the decay process $Z\to
q\bar{q}$. This vertex contains a simple $\delta_{ij}$ in colour
space; the outgoing quark and antiquark must have identical 
(anti-)col\-ours. Squaring the corresponding matrix element and summing over
final-state colours yields a colour factor of
\begin{equation}
e^+e^-\to Z \to q\bar{q}~~~:~~~\sum_{\mrm{colours}}|M|^2 \propto
\delta_{ij}\delta_{ji} = \mrm{Tr}\{\delta\} = N_C = 3~,
\end{equation}
since $i$ and $j$ are quark (i.e., 3-dimensional
fundamental) indices. This factor corresponds directly to the 3 different
``paths through colour space'' that the process at hand can take; the
produced quarks can be red, green, or blue. 

A next-to-simplest example is given by $q\bar{q}\to
\gamma^*/Z\to\ell^+\ell^-$ (usually referred to as the
\index{Drell-Yan}Drell-Yan 
process~\cite{Drell:1970wh}),  
which is just a crossing of the previous one. By crossing
symmetry, the squared matrix element, including the colour factor, is
exactly the same as before, but since the quarks are here incoming, we
must \emph{average} rather than sum over their colours, leading to
\begin{equation}
q\bar{q}\to Z\to e^+e^-~~~:~~~\frac{1}{9}\sum_{\mrm{colours}}|M|^2 \propto \frac19\delta_{ij}\delta_{ji} = \frac19 \mrm{Tr}\{\delta\} = \frac13~,
\end{equation}
where the colour factor now expresses a \emph{suppression} which can
be interpreted as due to the fact that only quarks of matching colours
are able to collide and produce a $Z$ boson. The chance that a quark
and an antiquark picked at random from the colliding hadrons have 
matching colours is $1/N_C$. 
\begin{figure}[t]
\end{figure}

Similarly, $\ell q \to
\ell q$ via $t$-channel photon exchange (usually called Deep
Inelastic Scattering --- \index{DIS}\index{Deep inelastic scattering|see{DIS}}DIS --- with ``deep'' referring to a 
large virtuality of the exchanged photon), constitutes yet another
crossing of the same basic process, 
see \figRef{fig:Zcrossings}. \index{Colour factors}The colour factor in this case 
comes out as unity. 
\begin{figure}[t]
\centering\vspace*{-8mm}
\begin{tabular}{ccc}
\rotatebox{360}{\includegraphics*[scale=0.93]{ee2qq}} \ \ 
& \ \ \includegraphics*[scale=0.93,angle=180,origin=c]{ee2qq}
\ \ & \ \ \includegraphics*[scale=0.9,angle=297,origin=c]{ee2qq}\\
Hadronic $Z$ decay & \index{Drell-Yan}Drell-Yan & \index{DIS}DIS \\[1mm]
$e^-e^+ \to \gamma^*/Z^0 \to q\bar{q}$ &
$q\bar{q} \to \gamma^*/Z^0 \to \ell^+\ell^-$ &
$\ell \bar{q} \stackrel{\gamma^*/Z^*}{\to} \ell \bar{q}$
\\[2mm] 
$\propto N_C$ & $\propto 1/N_C$ & $\propto 1$
\end{tabular}
\caption{Illustration of the three crossings of the interaction of a
  lepton current (black) with a \index{Quarks}quark current (red) 
  via an intermediate photon or
  $Z$ boson, with corresponding colour factors. \label{fig:Zcrossings}}
\end{figure}

To illustrate what happens when we insert (and sum over)
quark-gluon
vertices, such as the one depicted in \figRef{fig:qg}, we take
the process $Z\to3\,$jets. \index{Colour factors}The colour factor for
this process can be 
computed as follows, with the accompanying illustration showing a
corresponding diagram (squared) with explicit colour-space indices on
each vertex:\\
\index{Colour connections}
\begin{equation}
\mbox{
\begin{tabular}{cc}
\parbox{5.2cm}{
$Z \to qg\bar{q}$~~~:~~~\\
\[
\begin{array}{rcl}
\displaystyle\sum_{\mrm{colours}}|M|^2 & \propto & \displaystyle
\delta_{ij}t_{jk}^a t_{k\ell
    }^a\delta_{\ell i} \\
& = & \displaystyle
\mrm{Tr}\{t^at^a\}\\[4mm] & = & \displaystyle
  \frac12\mrm{Tr}\{\delta\} = 4~,
\end{array}
\]}
&
\parbox{8.5cm}{\includegraphics*[scale=0.6]{colFacZ3.pdf}
}
\end{tabular}}
\end{equation}
where the last $\mrm{Tr}\{\delta\} = 8$, since the trace runs over
the 8-dimensional adjoint indices. If we
want to ``count the paths through colour space'', we should leave out
the factor $\frac12$ which comes from the normalisation convention for
the $t$ matrices, \eqRef{eq:t}, hence this process can take 8
different paths through colour space, one for each gluon basis state.

The tedious task of taking traces over $t$
matrices can be greatly alleviated by use of the relations given in
\TabRef{tab:lambda}.  
\index{Traces in SU(3)|see{SU(3)}}%
\index{SU(3)!Trace relations}%
\index{QCD!Trace relations|see{SU(3)}}%
\begin{table}
\begin{center}
\scalebox{1.04}{\begin{tabular}{ccc}
\toprule
\index{SU(3)!Trace relations}Trace Relation & Indices & Occurs in Diagram Squared
\\
\midrule
$\mrm{Tr}\{t^at^b\} = T_R\, \delta^{ab}$ & $a,b\in[1,\ldots,8]$
& \parbox[c]{4cm}{\includegraphics*[scale=0.5]{traces1}}\\
$\sum_a t^a_{ij}t^a_{jk} = C_F\, \delta_{ik}$ &%
\parbox[c]{3cm}{\begin{center}
$a\in[1,\ldots,8]$\\
$i,j,k\in[1,\ldots,3]$\end{center}}
& \parbox[c]{4cm}{\includegraphics*[scale=0.5]{traces2}}\\
$\sum_{c,d} f^{acd} f^{bcd} = C_A\, \delta^{ab}$ & $a,b,c,d\in[1,\ldots,8]$
& \parbox[c]{4cm}{\includegraphics*[scale=0.5]{traces3}}\\
$ t^a_{ij}t^a_{k\ell} = T_R \left(\delta_{jk}\delta_{i\ell}
- \frac{1}{N_C}\delta_{ij}\delta_{k\ell}\right)$ & $i,j,k,\ell\in[1,\ldots,3]$
& \parbox[c]{4cm}{\includegraphics*[scale=0.5]{traces4}}\hspace*{-0.2cm}(Fierz)\\
\bottomrule
\end{tabular}}
\caption{Trace relations for $t$ matrices (convention-independent). 
 More relations
  can be found in \cite[Section 1.2]{Ellis:1991qj} and in 
  \cite[Appendix A.3]{Peskin:1995ev}.
\label{tab:lambda}}
\end{center}
\end{table}
In the standard normalisation convention for the \index{SU(3)}$\mrm{SU(3)}$ generators,
\eqRef{eq:t}, the \index{Casimirs}Casimirs of $\mrm{SU(3)}$ appearing in
\TabRef{tab:lambda} are\footnote{See, e.g., \cite[Appendix
    A.3]{Peskin:1995ev} for how to obtain the Casimirs in other
  normalisation conventions. As an example, choosing $t^a_{ij} = \lambda_{ij}^a/\sqrt{2}$ would yield $T_R=1$, $C_F=T_R(N_C^2-1)/N_C=8/3$, $C_A=3$.} 
\index{Casimirs}\index{TR@$T_R$}\index{CA@$C_A$}\index{CF@$C_F$}
\begin{equation}
T_R = \frac12 \hspace*{2cm} C_F = \frac43 \hspace*{2cm} C_A = N_C = 3~.
\end{equation}
In addition, the gluon self-coupling on the third line in
\TabRef{tab:lambda} involves factors of $f^{abc}$. These
\index{QCD!Structure constants|see{SU(3)}}%
are called the \index{SU(3)!Structure constants}\emph{structure constants} of QCD and they enter via 
the non-Abelian term in the \index{Gluons}gluon field strength tensor appearing in
\eqRef{eq:L}, 
\begin{equation}
F^a_{\mu\nu} = \underbrace{\partial_\mu A_\nu^a - \partial_\nu
  A^a_\mu}_{\mathrm{Abelian}} +
\underbrace{ g_s f^{abc} A_\mu^b A_\nu^c}_{\mathrm{non-Abelian}}~. \label{eq:F}
\end{equation}

\noindent\begin{minipage}[t]{0.46\textwidth}
The structure constants of $\mrm{SU(3)}$ are listed in the table to the
right. They define the \emph{adjoint}, or \emph{vector}, representation of $\mrm{SU(3)}$
and are related to the fundamental-representation generators via the
commutator relations
\begin{equation}
t^at^b - t^bt^a = [t^a,t^b] = i f^{abc} t_c~,
\end{equation} 
or equivalently,
\begin{equation}
if^{abc}~=~2\mrm{Tr}\{t^c[t^a,t^b]\}~.
\end{equation}
Thus, it is a matter of choice whether one prefers to express colour
space on a basis of fundamental-representation $t$ matrices, or via
the structure constants $f$, and one can go back and forth between the
two.
\end{minipage}%
\hfill%
\colorbox{darkgray}{%
\colorbox{lightgray}{%
\begin{minipage}[t]{0.46\textwidth}
\vspace*{3mm}\begin{center}
\textbf{Structure Constants of SU(3)}
\begin{equation}
f_{123} = 1
\end{equation}
\begin{equation}
f_{147} = f_{246} = f_{257} = f_{345} = \frac12
\end{equation}
\begin{equation}
f_{156} = f_{367} = -\frac12
\end{equation}
\begin{equation}
f_{458} = f_{678} = \frac{\sqrt{3}}{2}
\end{equation}
Antisymmetric in all indices\\[3mm]
All other $f_{abc}=0$\vspace*{3mm}\\
\end{center}
\end{minipage}%
}}\vskip1mm

\begin{figure}[t]
\begin{center}
\begin{minipage}[h]{4.6cm}
\begin{center}
$A_\nu^4(k_2)$\\
\includegraphics*[scale=0.75]{ggv.pdf}\\[-3mm]
$A^6_\rho(k_1)$\hfill$A_\mu^2(k_3)$
\end{center}
\end{minipage}~~~
\parbox{0.35\textwidth}{
$
\begin{array}{cccc}
\propto & - g_s \ f^{246} \!\! & \!\! [ (k_3 - k_2)^\rho g^{\mu\nu}  \\ 
& & +(k_2 - k_1)^\mu g^{\nu\rho} \\ 
& &+(k_1 - k_3)^\nu g^{\rho\mu}]
\end{array}
$}\vspace*{1mm}
\caption{Illustration of a \index{Gluons}$ggg$ vertex in QCD, before
  summing/averaging over colours: interaction between gluons in the 
  states $\lambda^2$, $\lambda^4$, and $\lambda^6$ is represented by
  the structure constant $f^{246}$. 
\label{fig:gg}}
\end{center}
\end{figure}
 Expanding the $F_{\mu\nu}F^{\mu\nu}$ term of the
Lagrangian using \eqRef{eq:F}, we see that there is a 3-gluon and a
4-gluon vertex that involve $f^{abc}$, the latter of which has two
powers of $f$ and two powers of the coupling. 

Finally, the last line of \TabRef{tab:lambda} is not really a trace
relation but instead a useful so-called Fierz transformation, which
expresses products of $t$ matrices in terms of Kronecker $\delta$ functions. 
It is often used, for instance, in shower Monte Carlo
applications, to assist in mapping between colour flows in $N_C = 3$,
in which cross sections and splitting probabilities are calculated, 
and those in $N_C\to\infty$ (``leading colour''), used to represent colour flow in
the MC ``event record''.

A \index{Gluons}gluon self-interaction vertex is
illustrated in \figRef{fig:gg}, to be compared with the quark-gluon
one in \figRef{fig:qg}. We remind the reader that gauge boson
self-interactions are a hallmark of non-Abelian theories and that their
presence leads to some of the main differences between QED and
QCD. One should also keep in mind 
that the \index{Colour factors}colour factor for the vertex in \figRef{fig:gg}, \index{CA@$C_A$}$C_A$, 
is roughly twice as large as that for a quark, \index{CF@$C_F$}$C_F$.

\subsection{The Strong Coupling \label{sec:coupling}}
\index{QCD!Coupling}
\index{Jets}
\index{alphaS@$\alpha_s$}To first approximation, QCD is 
\index{QCD!Scale invariance}\emph{scale invariant}. That is, if one
``zooms in'' on a QCD jet, one will find a repeated self-similar 
pattern of jets within jets within jets, reminiscent of
fractals. 
In the context of QCD, this property was originally 
called \index{Lightcone scaling|see{QCD Scale invariance}}light-cone scaling, or 
\index{Bjorken scaling|see{QCD Scale invariance}}Bj{\o}rken scaling. 
This type of scaling is closely related to the class of
angle-preserving symmetries, called \index{Conformal
invariance}\emph{conformal} symmetries. In physics 
today, the terms ``conformal'' and ``scale invariant'' are used 
interchangeably\footnote{Strictly speaking, conformal symmetry is more
restrictive than just scale invariance, but examples of
scale-invariant field theories that are not conformal are rare.}.
Conformal invariance is a mathematical property of several
QCD-``like'' theories which are now being studied (such as $N=4$
supersymmetric relatives of QCD). It is also 
related to the physics of so-called ``unparticles'', though that is a
relation that goes beyond the scope of these lectures.

Regardless of the labelling, 
if the  \index{alphaS@$\alpha_s$}strong coupling did not run (we shall
return to the running 
of the coupling below), Bj{\o}rken scaling would be absolutely true. QCD
would be a theory with a fixed coupling, the same at all scales. 
This simplified picture already captures some of the most important
properties of QCD, as we shall discuss presently.  

\index{QCD!Scale invariance}%
In the limit of exact Bj{\o}rken scaling --- QCD at fixed coupling
--- properties of high-energy interactions are determined 
only by \emph{dimensionless} kinematic quantities, such as scattering
angles (pseudorapidities) and ratios of energy
scales\footnote{Originally, the observed approximate agreement with
this was used as a powerful argument
for pointlike substructure in hadrons; since measurements at different
energies are sensitive to different resolution scales, independence of the absolute
energy scale is indicative of the absence of other fundamental
scales in the problem and hence of pointlike constituents.}.
For applications of QCD to high-energy collider physics, an important
consequence of Bj{\o}rken scaling is thus that the rate of 
\index{Parton showers}%
\index{Bremsstrahlung|see{Parton showers}}
bremsstrahlung
jets, with a given transverse momentum, scales in direct proportion to
the hardness 
of the fundamental partonic scattering process they are produced in
association with. This agrees well with our intuition about accelerated
charges; the harder you ``kick'' them, the harder the radiation they
produce.  

For instance, in the limit of exact scaling, a
measurement of the rate of 10-GeV jets produced in association with an
ordinary $Z$ 
boson could be used as a direct prediction of the rate of 100-GeV jets
that would be 
produced in association with a 900-GeV $Z'$ boson, and so 
forth. Our intuition about how many bremsstrahlung jets a given type of
process is likely to have should therefore be governed first and
foremost by the \emph{ratios} of scales that appear in that particular
process, as has been  highlighted in a number of studies focusing on
the mass and $p_\perp$ scales appearing, e.g., in
Beyond-the-Standard-Model (BSM) 
physics processes
\cite{Plehn:2005cq,Alwall:2008qv,Papaefstathiou:2009hp,Krohn:2011zp}. 
\index{QCD!Scale invariance}Bj{\o}rken scaling 
\index{Scale invariance|see{QCD}}
is also fundamental to the understanding of jet substructure in QCD, see, e.g.,
\cite{Vermilion:2011nm,Altheimer:2012mn}.  

\index{alphaS@$\alpha_s$!Running coupling}%
On top of the underlying scaling behavior, the running coupling will
introduce a dependence on the absolute scale, implying more radiation
at low scales than at high ones. The running is logarithmic with
\index{alphaS@$\alpha_s$!beta function}%
energy, and is governed by the so-called \emph{beta function}, 
\index{alphaS@$\alpha_s$}
\begin{equation}
Q^2 \frac{\partial \alpha_s}{\partial Q^2} = \frac{\partial
  \alpha_s}{\partial \ln Q^2} =
\beta(\alpha_s)~, \label{eq:running}
\end{equation}
where the function driving the energy dependence, the \index{Beta function}{beta
  function}, is defined as
\begin{equation}
\beta(\alpha_s) = -\alpha_s^2(b_0 +
b_1\alpha_s + b_2\alpha_s^2 + \ldots)~,\label{eq:beta}
\end{equation}
with LO (1-loop) and NLO (2-loop) coefficients
\begin{eqnarray}
b_0 & = & \frac{11C_A - 4 T_R n_f}{12\pi}~,\\[3mm]
b_1 & = & \frac{17C_A^2 - 10 T_R C_A n_f - 6 T_R C_F n_f}{24\pi^2} ~=~
\frac{153-19 n_f}{24\pi^2}~.\label{eq:b}
\end{eqnarray}
In the $b_0$ coefficient, the first term is due to
\index{Gluons!Contribution to beta function}gluon loops while the
second is due to \index{Quarks!Contribution to beta function}quark
ones. Similarly, the first 
term of the $b_1$ coefficient arises from double gluon loops,
while the second and third represent mixed quark-gluon ones. 
At higher loop orders, the $b_i$ coefficients depend explicitly on the
renormalisation scheme that is used. A brief discussion can be found in the
PDG review on QCD~\cite{pdg2012}, with more elaborate ones
contained in \cite{Dissertori:2003pj,Ellis:1991qj}. 
Note that, if there are additional coloured particles beyond the
Standard-Model ones, loops involving those particles enter
 at energy scales above the masses of the
new particles, thus modifying the  \index{alphaS@$\alpha_s$}running of the coupling at high scales. 
This is discussed, e.g., for supersymmetric models in
\cite{Martin:1997ns}. For the running of other SM couplings, see
e.g.,~\cite{Langacker:2010zza}. 

\index{alphaS@$\alpha_s$!Running coupling}%
Numerically, the value of the  \index{alphaS@$\alpha_s$}strong coupling is usually specified by
giving its value at the specific 
reference scale $Q^2=M^2_Z$, from which we can obtain its
value at any other scale by solving \eqRef{eq:running}, 
\begin{equation}
\alpha_s(Q^2) = \alpha_s(M_Z^2) \frac{1}{1+b_0
  \alpha_s(M_Z^2)\ln\frac{Q^2}{M_Z^2} + {\cal O}(\alpha_s^2)}~,
\label{eq:alphaq2}
\end{equation}
with relations including the ${\cal O}(\alpha_s^2)$ terms 
available, e.g., in \cite{Ellis:1991qj}. 
Relations between scales 
not involving $M_Z^2$ can obviously be obtained by just replacing $M_Z^2$
by some other scale $Q'^2$ everywhere in \eqRef{eq:alphaq2}. A
comparison of running at one- and two-loop order, in both cases starting from
$\alpha_s(M_Z)=0.12$, is given in \figRef{fig:asRun}.
\begin{figure}[t]
\centering
\includegraphics*[scale=0.45]{vc-alphaS.pdf}
\caption{Illustration of the running of
 $\alpha_s$ at 1- (open 
  circles) and 2-loop
  order (filled circles), 
starting from the same value of $\alpha_s(M_Z)=0.12$. 
\label{fig:asRun}}
\end{figure}
As is evident from the figure, the 2-loop running is somewhat faster
than the 1-loop one.

\index{alphaS@$\alpha_s$!Running coupling}%
As an application, let us prove that the 
logarithmic running of the coupling implies that an intrinsically 
multi-scale problem can be converted to a single-scale one, up to
corrections suppressed by two powers of $\alpha_s$, 
by taking the geometric mean of the scales involved. This follows from
expanding an arbitrary product of individual  \index{alphaS@$\alpha_s$}$\alpha_s$ factors around an
arbitrary scale $\mu$, using \eqRef{eq:alphaq2}, 
\begin{eqnarray}
\alpha_s(\mu_1)\alpha_s(\mu_2)\cdots\alpha_s(\mu_n) & = &
\prod_{i=1}^{n} \alpha_s(\mu) \left(1 +
b_0\,\alpha_s\ln\left(\frac{\mu^2}{\mu_i^2}\right) + {\cal O}(\alpha_s^2)\right)
\nonumber\\[2mm]
& = & \alpha_s^n(\mu) \left(1 + b_0\, \alpha_s \ln \left(
 \frac{\mu^{2n}}{\mu_1^2\mu_2^2\cdots\mu_n^2}\right) +  {\cal
   O}(\alpha_s^2) \right)~,
\end{eqnarray}
whereby the specific single-scale choice $\mu^n =
\mu_1\mu_2\cdots\mu_n$ (the geometric mean) can
be seen to push the difference between the two sides of the equation one order higher
than would be the case for any other combination of scales\footnote{In
  a fixed-order calculation, the individual scales $\mu_i$,
would correspond, e.g., to the $n$ hardest scales appearing in an infrared
safe sequential clustering algorithm applied to the given momentum
configuration.}. 

The appearance of the number of \index{Flavour}flavours, $n_f$, in $b_0$ implies that the
slope of the running depends on the number of contributing
\index{Flavour}flavours. Since full QCD is best approximated by $n_f=3$
below the charm threshold, by $n_f=4$ and $5$ from there to the $b$
and $t$ thresholds, respectively, and then by $n_f=6$ at scales
higher than $m_t$, it is therefore important to be aware that 
the running changes slope across quark \index{Flavour}flavour
thresholds. Likewise, it would change across the threshold for any coloured
new-physics particles that might exist, with a magnitude depending on
the particles' colour and spin quantum numbers.

\index{alphaS@$\alpha_s$!Running coupling}%
\index{alphaS@$\alpha_s$}
The negative overall sign of \eqRef{eq:beta}, combined with the fact
that $b_0 > 0$ (for $n_f \le 16$), leads to the famous
result\footnote{
Perhaps the highest pinnacle of fame for \eqRef{eq:beta} was reached
when the sign of it featured in an episode of the TV series ``Big Bang
Theory''.} 
that the QCD coupling effectively \emph{decreases} with
 energy, called \index{Asymptotic freedom}asymptotic 
freedom, for the discovery of which the \index{Nobel prize}Nobel prize in physics was
awarded to D.~Gross, H.~Politzer, and F.~Wilczek in 2004. An extract
of the prize announcement runs as follows:
\begin{center}
\begin{minipage}{0.84\textwidth}
\sl  What this year's Laureates discovered was something that, at
first sight, seemed completely contradictory. The interpretation of
their mathematical result was that the closer the quarks are to each
other, the \emph{weaker} is the ``colour charge''. When the quarks are
really close to each other, the force is so weak that they behave
almost as free particles\footnote{More correctly, it is the coupling
  rather than the  
  force which becomes weak as the distance decreases. 
  The $1/r^2$ Coulomb singularity of the force is only dampened, not removed, 
  by the diminishing coupling.}. 
This phenomenon is called ``asymptotic
freedom''. The converse is true when the quarks move apart: the force
becomes stronger when the distance increases\footnote{More correctly,
 it is the potential which grows, linearly, while the force becomes
 constant.}. 
\end{minipage}
\end{center}

\index{Running coupling|see{alphaS@$\alpha_s$}}%
\index{alphaS@$\alpha_s$!Running coupling}%
Among the consequences of \index{Asymptotic freedom}asymptotic freedom is that perturbation
theory becomes better behaved at higher absolute energies, due to the
effectively decreasing coupling. Perturbative calculations for our
900-GeV $Z'$ boson from before should therefore be slightly faster
converging than equivalent calculations for the 90-GeV one. 
Furthermore, since the running of  \index{alphaS@$\alpha_s$}$\alpha_s$ explicitly
breaks Bj{\o}rken scaling, we also expect to see small changes in jet
shapes and in jet production ratios as we vary the energy. For
instance, since high-$p_\perp$ jets
start out with a smaller effective coupling, their intrinsic shape
(irrespective of boost effects) is
somewhat narrower than for low-$p_\perp$ jets, an issue which can be
important for jet calibration. Our current understanding of the
running of the QCD coupling is summarised by the plot in
\figRef{fig:alphas}, taken from a recent comprehensive review by S.\ Bethke
\cite{pdg2012,Bethke:2012jm}. A complementary up-to-date overview of
$\alpha_s$ determinations can be found in~\cite{d'Enterria:2015toz}. 

\index{alphaS@$\alpha_s$!Running coupling}%
As a final remark on \index{Asymptotic freedom}asymptotic freedom, note
that the decreasing 
value of the  \index{alphaS@$\alpha_s$}strong coupling with energy must eventually cause it to
become comparable to the electromagnetic and weak ones, at some energy
scale. Beyond that point, which may lie at energies of order
$10^{15}-10^{17}\,$GeV (though it may be lower if as yet undiscovered
particles generate large corrections to the running), 
we do not know  what the further evolution of the combined theory will 
actually look like, or whether it will continue to exhibit
\index{Asymptotic freedom}asymptotic
freedom. 

\index{alphaS@$\alpha_s$}%
\index{alphaS@$\alpha_s$!Running coupling}%
\index{alphaS@$\alpha_s$!LambdaQCD@$\Lambda_{\mathrm{QCD}}$}%
Now consider what happens when we run the coupling in the other
direction, towards smaller energies. 
\begin{figure}[t]
\begin{center}\hspace*{-0.25cm}
\parbox[c]{3.1cm}{\includegraphics*[scale=0.65]{arr-ir.pdf}}
\parbox[c]{8cm}{\includegraphics*[scale=0.5]{asq-2011.pdf}}\hspace*{-1mm}
\parbox[c]{3.1cm}{\includegraphics*[scale=0.65]{arr-uv.pdf}}
\caption{Illustration of the running of $\alpha_s$ in a theoretical
  calculation (band) and in physical processes at
  different characteristic scales, from
  \cite{pdg2012,Bethke:2012jm}. The little kinks at $Q=m_{c}$ and
  $Q=m_b$ are
  caused by discontinuities in the running across the flavour
  thresholds.\label{fig:alphas}}  
\end{center}           
\end{figure}
Taken at face value, the numerical value of the coupling diverges
rapidly at scales below 1 GeV, as illustrated by the curves
disappearing off the left-hand edge of the plot in
\figRef{fig:alphas}. To make this divergence
explicit, one can rewrite
\eqRef{eq:alphaq2} in the following form, 
 \index{alphaS@$\alpha_s$}
\begin{equation}
\alpha_s(Q^2) = \frac{1}{b_0 \ln \frac{Q^2}{\Lambda^2}}~,\label{eq:alphasLam}
\end{equation}
where 
\begin{equation}
\Lambda \sim 200\, \mbox{MeV}
\end{equation}
\index{alphaS@$\alpha_s$!LambdaQCD@$\Lambda_{\mathrm{QCD}}$}%
\index{alphaS@$\alpha_s$!Landau Pole|see{$\Lambda_{\mathrm{QCD}}$}}%
\index{LambdaQCD@$\Lambda_{\mathrm{QCD}}$|see{alphaS@$\alpha_s$}}%
specifies the energy scale at which the perturbative coupling would nominally become
infinite, called the Landau pole. (Note, however, that this only
parametrises the purely \emph{perturbative} result, which is not
reliable at \index{Strong coupling}strong coupling, so \eqRef{eq:alphasLam} should 
not be taken to imply that the physical behavior of full QCD should
exhibit a divergence for $Q\to \Lambda$.) 

\index{alphaS@$\alpha_s$}%
\index{alphaS@$\alpha_s$!Running coupling}%
\index{alphaS@$\alpha_s$!LambdaQCD@$\Lambda_{\mathrm{QCD}}$}%
Finally, one should be aware that there is a multitude of different
ways of defining both $\Lambda$ and $\alpha_s(M_Z)$. At the very
least, the numerical value one obtains depends both on the
renormalisation scheme used (with the dimensional-regularisation-based
``modified minimal subtraction'' scheme, $\overline{\mbox{MS}}$, being the
most common one) and on the perturbative order of the calculations 
used to extract them. As a rule of thumb, fits to experimental data typically yield 
smaller values for $\alpha_s(M_Z)$ the higher the order of the
calculation used to extract it (see, e.g.,
\cite{Bethke:2009jm,Dissertori:2009ik,Bethke:2012jm,pdg2012}), with  $
\alpha_s(M_Z)\vert_{\mrm{LO}} \gsim \alpha_s(M_Z)\vert_{\mrm{NLO}}
\gsim \alpha_s(M_Z)\vert_{\mrm{NNLO}}$. 
Further, since the number of \index{Flavour}flavours changes the slope
of the running, the location of the Landau pole for fixed
$\alpha_s(M_Z)$ depends explicitly on the number of \index{Flavour}flavours used in
the running. Thus each value of $n_f$ is associated with its own
value of $\Lambda$, with the following matching relations across
thresholds guaranteeing continuity of the coupling at one loop,
\index{LambdaQCD@$\Lambda_{\mathrm{QCD}}$|see{$\alpha_s$}}
\index{alphaS@$\alpha_s$!LambdaQCD@$\Lambda_{\mathrm{QCD}}$}%
\begin{eqnarray}
n_f = 5 \leftrightarrow 6 ~~~:~~~~~~\Lambda_6 = \Lambda_5
  \left(\frac{\Lambda_5}{m_t}\right)^{\frac{2}{21}} & & 
\Lambda_5 = \Lambda_6
  \left(\frac{m_t}{\Lambda_6}\right)^{\frac{2}{23}} ~, \\[2mm]
n_f = 4 \leftrightarrow 5 ~~~:~~~~~~\Lambda_5 = \Lambda_4
  \left(\frac{\Lambda_4}{m_b}\right)^{\frac{2}{23}} & & 
\Lambda_4 = \Lambda_5
  \left(\frac{m_b}{\Lambda_5}\right)^{\frac{2}{25}} ~, \\[2mm]
n_f = 3 \leftrightarrow 4 ~~~:~~~~~~\Lambda_4 = \Lambda_3 
  \left(\frac{\Lambda_3}{m_c}\right)^{\frac{2}{25}} & &
\Lambda_3 = \Lambda_4 
  \left(\frac{m_c}{\Lambda_4}\right)^{\frac{2}{27}} ~.
\end{eqnarray}

\index{alphaS@$\alpha_s$}%
\index{alphaS@$\alpha_s$!Running coupling}%
It is sometimes stated that QCD only has a single free
parameter, the  \index{alphaS@$\alpha_s$}strong coupling. 
However, even in the perturbative
region, the beta function depends explicitly on the number of
quark \index{Flavour}flavours, as we have seen, and thereby also on the quark
masses. Furthermore, in the non-perturbative region around or below
$\Lambda_{\mrm{QCD}}$, the value of the 
perturbative coupling, as obtained, e.g., from \eqRef{eq:alphasLam},
gives little or no insight into the behavior of the full theory. 
Instead, universal functions (such as parton densities, form factors,
fragmentation functions, etc), effective theories (such as the
Operator Product Expansion, Chiral Perturbation Theory, or Heavy Quark
Effective Theory), or phenomenological models (such as Regge Theory or
the String and Cluster Hadronisation Models) must be used, which in
turn depend on additional non-perturbative parameters whose relation to, e.g.,
$\alpha_s(M_Z)$, is not a priori known. 

\index{Lattice QCD}
For some of these questions,
such as hadron masses, lattice QCD can furnish important
additional insight, but for multi-scale and/or time-evolution
problems, the applicability of lattice methods is still severely
restricted; the lattice formulation of QCD requires 
  a Wick rotation to
  Euclidean space. The time-coordinate can then be treated on an
  equal footing with the other dimensions, but intrinsically
  Minkowskian problems, such as the time evolution of a system, are
   inaccessible. The limited size of current lattices
  also severely constrain the scale hierarchies that it is possible to
  ``fit'' between the lattice spacing and the lattice size. 

\index{Landau pole|see{$\alpha_s$}}%
\index{QCD!Landau Pole|see{$\alpha_s$}}%
\index{Renormalisation|see{$\alpha_s$}}%
\index{QCD!Renormalisation|see{$\alpha_s$}}%

\subsection{Colour States}
\index{Coherence}%
A final example of the application of the underlying $\mrm{SU(3)}$ group
theory to QCD is given by considering which colour states we can
obtain by combinations of quarks and gluons. The simplest example of
this is the combination of a quark and antiquark. We can form a total
of nine different colour-anticolour combinations, which fall into two
irreducible representations of $\mrm{SU(3)}$:
\begin{equation}
3 \otimes \overline{3} = 8 \oplus 1~.\label{eq:33bar}
\end{equation}
The singlet corresponds to the symmetric wave function 
$\frac{1}{\sqrt{3}}\left(\left|R\bar{R}\right>+\left|G\bar{G}\right>+\left|B\bar{B}\right>\right)$, 
which is invariant under $\mrm{SU(3)}$ transformations (the definition of a
singlet). The other eight linearly independent 
combinations (which can be represented by one for each Gell-Mann
matrix, with the singlet corresponding to the identity matrix) transform
into each other under $\mrm{SU(3)}$. Thus, although we sometimes talk about
colour-singlet states as 
being made up, e.g., of ``red-antired'', that is not quite precise
language. The actual state $\left|R\bar{R}\right>$ is \emph{not} a
pure colour singlet.  Although it does
have a non-zero \emph{projection} onto the singlet wave function
above, it also has non-zero projections onto the two members of
the octet that correspond to the diagonal Gell-Mann
matrices. Intuitively, one can also easily realise this by noting that
an $\mrm{SU(3)}$ rotation of $\left|R\bar{R}\right>$ would in general turn it into a
different state, say $\left|B\bar{B}\right>$, whereas a true colour singlet
would be invariant. 
Finally, we can also realise from \eqRef{eq:33bar} that a random
(colour-uncorrelated) quark-antiquark pair has a $1/N^2=1/9$ 
chance to be in an overall colour-singlet state; otherwise it is in
an octet. 

Similarly, there are also nine possible quark-quark (or
antiquark-antiquark) combinations, six of which are symmetric
under interchange of the two quarks and three of which are antisymmetric:
\index{Sextet}%
\begin{equation}
6 ~=~ \left(\begin{array}{c}
\left|RR\right>\\
\left|GG\right>\\
\left|BB\right>\\
\frac{1}{\sqrt{2}}\left(\left|RG\right> + \left|GR\right>\right)\\
\frac{1}{\sqrt{2}}\left(\left|GB\right> + \left|BG\right>\right)\\
\frac{1}{\sqrt{2}}\left(\left|BR\right> + \left|RB\right>\right)
\end{array}\right)
~~~~~~~~~
\bar{3} = \left(\begin{array}{c}
\frac{1}{\sqrt{2}}\left(\left|RG\right> - \left|GR\right>\right)\\
\frac{1}{\sqrt{2}}\left(\left|GB\right> - \left|BG\right>\right)\\
\frac{1}{\sqrt{2}}\left(\left|BR\right> - \left|RB\right>\right)
\end{array}\right)~.
\end{equation}
The members of the sextet transform into (linear combinations of) 
each other under $\mrm{SU(3)}$ transformations, and similarly for the
members of the antitriplet, hence neither of these can be reduced
further. The breakdown into
irreducible $\mrm{SU(3)}$ multiplets is therefore
\begin{equation}
3 \otimes 3 = 6 \oplus \overline{3}~.
\end{equation}
Thus, an uncorrelated pair of quarks has a $1/3$ chance to add to an overall
anti-triplet state (corresponding to coherent
superpositions like ``red + green = antiblue''\footnote{In the context of
  hadronisation models, 
  this coherent superposition of two quarks in an overall antitriplet
  state is sometimes called a
  \index{Diquarks}``diquark'' (at low $m_{qq}$)
  \index{String junctions}or a ``string junction'' (at high $m_{qq}$), see
  \secRef{sec:stringModel}; it corresponds to the antisymmatric ``red
  + green = antiblue'' combination needed to create a baryon
  wavefunction. }); otherwise it is in an overall 
sextet state. 

Note that the emphasis on
the quark-(anti)quark pair being \emph{uncorrelated} is important;
production processes that correlate the produced partons, like $Z\to q\bar{q}$ or $g\to q\bar{q}$, will
project out specific components (here the singlet and octet,
respectively). 
Note also that, if the quark
and (anti)quark are on opposite sides of the universe (i.e., living in
two different hadrons), the QCD \emph{dynamics} will not care what
overall colour state they 
are in, so for the formation of multi-partonic states in QCD, obviously the
spatial part of the wave functions (causality at the very least) 
will also play a role. Here, we are considering \emph{only} the colour part
of the wave functions. 
Some additional examples are 
\begin{eqnarray}
8\otimes 8 & = & 27 \oplus 10 \oplus \overline{10} \oplus 8 \oplus 8
\oplus 1 ~,\\ 
3 \otimes 8 & = & 15 \oplus 6 \oplus 3~,\\
3 \otimes 6 & = & 10 \oplus 8~,\\
3\otimes3\otimes3 & = & (6 \oplus \overline{3}) \otimes 3 = 10 \oplus 8
\oplus 8 \oplus 1 ~.
\end{eqnarray}
Physically, the 27 in the first line corresponds to a completely
incoherent addition of the colour charges of two gluons;
\index{Decuplet}the decuplets are slightly more coherent (with a lower
total colour charge), the octets
yet more, and the singlet corresponds to the combination of two gluons
that have precisely equal and opposite colour charges, so that their
total charge is zero. 
Further extensions and generalisations of these combination rules can
\index{Young tableaux}be obtained, e.g., using the method of Young
tableaux~\cite{young1901,youngSagan}.  


\section{Non-diagonalizable sequences}
\label{sec:non-diag-sequ}

As a preparation for the realizability model constructed in \cref{sec:topos-with-countable} we review the construction of non-diagonalizable sequences developed by Joseph Miller~\cite{miller04:_cont_deg}.

\subsection{Oracle-computable maps and coding of objects}
\label{sec:oracle-comp-maps}

We let lower Greek letters $\alpha$, $\beta \in \Cantor$ denote infinite binary sequences, and refer to them as \defemph{oracles}.

Given an oracle $\alpha \in \Cantor$, a partial map $f : \NN \parto \NN$ is \defemph{$\alpha$-computable} if it is computed by a Turing machine with access to the oracle~$\alpha$~\cite[Sect.~9.2]{rogers67:_theor_recur_funct_effec_comput}. Each such machine can be coded as a number, yielding a numbering $\pr[\alpha]{0}, \pr[\alpha]{1}, \pr[\alpha]{2}, \ldots$ of all partial $\alpha$-computable maps.
%
The codes describing machines are independent of the oracles. For example, there is a single index $i \in \NN$ such that $\pr[\alpha]{i} = \alpha$ for all $\alpha \in \Cantor$, and for any partial computable $f : \NN \parto \NN$ there is $j \in \NN$ such that $\pr[\alpha]{j} = f$ for all $\alpha \in \Cantor$.

Next we set up coding of mathematical objects.
%
Let $\pair{\Box, \Box} : \NN \times \NN \to \NN$ be a computable bijection, also known as a \defemph{pairing},
and let $\pi_1, \pi_2 : \NN \to \NN$ be the associated computable projections $\pi_1 \pair{m, n} = m$ and $\pi_2 \pair{m, n} = n$.
%
Let $\rat{} : \NN \to \QQ$ be a computable bijection enumerating the rationals.
%
Say that $f : \NN \to \NN$ \defemph{represents} $x \in \RR$ when
%
$\all{n \in \NN} |x - \rat{f(n)}| < 2^{-n}$.
%
That is, $f$ enumerates (codes of) rationals that converge to~$x$ with convergence modulus~$2^{-n}$. We call such a sequence \defemph{rapidly} converging.

An oracle~$\alpha$ can be construed as the binary digit expansion of a real $\rrep(\alpha) \in [0,1]$, namely
%
\begin{equation*}
  \rrep(\alpha) \defeq \sum\nolimits_{i=0}^\infty \alpha(i) \cdot 2^{-i-1}.
\end{equation*}
%
The resulting map $\rrep : \Cantor \to [0,1]$ is a continuous surjection.\footnote{It is possible to arrange for~$\rrep$ to be an open quotient map, but for our purposes a continuous map that is classically surjective will do.}
%
We may convert oracles qua binary digits expansions to representing maps, as there is $\R{w} \in \NN$ such that, for all $\alpha \in \Cantor$, the map $\pr[\alpha]{\R{w}}$ is total and it represents~$\rrep(\alpha)$. Concretely, let $\pr[\alpha]{\R{w}}(n)$ compute~$j$ such that $\rat{j} = \sum\nolimits_{i=0}^{n} \alpha(i) \cdot 2^{-i-1}$.

Finally, we provide coding of sequences $\NN \to [0,1]$ with oracles. For this purpose define $\srep : \Cantor \to [0,1]^\NN$ by
%
\begin{equation*}
  \srep(\alpha)(n) \defeq \rrep(m \mapsto \alpha(\pair{n,m}),
\end{equation*}
%
which too is a continuous surjection.
Once again we may convert oracles to representing maps, because there is~$\R{v} \in \NN$ such that $\pr[\alpha]{\pr[\alpha]{\R{v}}(n)}$ represents $\srep(\alpha)(n)$, for all $\alpha \in \Cantor$ and $n \in \NN$.


\subsection{Miller sequences}
\label{sec:miller-sequences}

Let us recall why in a duel between an oracle and diagonalization the latter wins.
%
Given $\alpha \in \Cantor$, say that~$x \in \RR$ is \defemph{$\alpha$-computable} if there is $n \in \NN$ such that $\pr[\alpha]{n}$ represents~$x$.
%
One might hope to construct an oracle~$\alpha \in \Cantor$ representing a sequence $a = \srep(\alpha) : \NN \to [0,1]$ that enumerates all $\alpha$-computable reals in~$[0,1]$, so that any $\alpha$-computable attempt to generate a real avoiding~$a$ would fail.
%
But this is not possible, because the diagonalization procedure described in the proof of~\cref{thm:R-uncountable} is itself $\alpha$-computable.
%
In particular, the choice in~\eqref{eq:R-uncountable} can be carried out $\alpha$-computably, one just has to compute a sufficiently precise rational approximation of~$a_n$.
%
The approximation, and therefore the choice and the resulting limit~$\ell$, may depend on~$\alpha$, but this in itself is not a problem.

An ingenuous insight of Joseph Miller's~\cite{miller04:_cont_deg} was that diagonalization \emph{can} be overcome, if we require oracle computations of reals to depend only on the sequence~$a$, and not the oracle representing it. In the following definition and elsewhere we write $\invim{f}$ for the inverse image map of~$f$.

\begin{definition}
  \label{def:sequence-computable}
  Given a sequence $a : \NN \to [0,1]$, say that $x \in \RR$ is \defemph{$a$-computable} if there is $n \in \NN$, called an \defemph{$a$-index}, such that $\pr[\alpha]{n}$ represents~$x$, for all $\alpha \in \invim{\srep}(a)$.
  %
  If $n$ is an $a$-index, we define $\rcomp{a}{n}$ to be the real computed by $\pr[\alpha]{n}$, for any oracle $\alpha \in \invim{\srep}(a)$. Otherwise, $\rcomp{a}{n}$ is undefined.
\end{definition}

The diagonalization procedure from the proof of~\cref{thm:R-uncountable} is not $a$-computable in the sense of the above definition. Perhaps there is another one that is? No.

\begin{theorem}[Miller]
  \label{thm:miller-sequence}%
  There exists a sequence $\mil : \NN \to [0,1]$ such that, for all $n \in \NN$, if
  $n$ is an $\mil$-index then $\mil(n) = \rcomp{\mil}{n}$.
\end{theorem}

Miller briefly mentions ``diagonally not computably diagonalizable'' as a possible name for a sequence satisfying the stated condition. We shall call it a \defemph{Miller sequence}.
%
In the remainder of this section we recount the original construction~\cite[Thm.~6.3]{miller04:_cont_deg}.

\subsubsection{The interval domain}
\label{sec:interval-domain}

Let
%
\begin{equation*}
  \II \defeq \set{[u,v] \subseteq [0,1] \such 0 \leq u \leq v \leq 1}
\end{equation*}
%
be the collection of all closed sub-intervals of~$[0,1]$.
If we think of an interval $[u,v]$ as an approximate real, then it makes sense to order~$\II$ by reverse inclusion~$\supseteq$ so that the zero-width intervals~$[u,u]$ are the maximal elements.

Not every $\pr[\alpha]{n}$ represents a real, but it can be seen to represent an element of~$\II$, as follows.
%
For $\alpha \in \Cantor$ and $n, j \in \NN$ let the \emph{$j$-th truncation} $\prx[\alpha]{n}{j} : \NN \to \set{\star} \cup \NN$ be
%
\begin{equation*}
  \prx[\alpha]{n}{j}(k) \defeq
  \begin{cases}
    \pr[\alpha]{n}(k) &
      \begin{aligned}[t]
        &\text{if the $n$-th machine with oracle~$\alpha$ applied} \\
        &\text{to~$k$ terminates in at most $j$ steps,}
      \end{aligned}
    \\
    \star &
    \text{otherwise.}
  \end{cases}
\end{equation*}
%
Define $H^\alpha_n : \NN \to \II$ by
%
\begin{equation*}
  H^\alpha_n(\pair{j, k}) \defeq
  \begin{cases}
    [\rat{m} - 2^{-k}, \rat{m} + 2^{-k}] &
      \text{if $\prx[\alpha]{n}{j}(k) = m$,} \\
    [0,1] & \text{if $\prx[\alpha]{n}{j}(k) = \star$}
  \end{cases}
\end{equation*}
% 
and $I^\alpha_n : \NN \to \II$ by $I^\alpha_n(0) \defeq [0,1]$ and
%
\begin{equation*}
  I^\alpha_n(k+1) \defeq
  \begin{cases}
    I^\alpha_n(k) \cap H^\alpha_n(k) &
      \text{if $I^\alpha_n(k) \cap H^\alpha_n(k) \neq \emptyset$}
    \\
    I^\alpha_n(k) & \text{otherwise.}
  \end{cases}
\end{equation*}
%
For any given~$n$ and~$k$, the endpoints of $I^\alpha_n(k)$ depend only on a finite prefix of~$\alpha$. Thus they are continuous in parameter~$\alpha$ with respect to the product topology on $[0,1]^\NN$ and the discrete topology on~$\QQ$.

We get a nested sequence of closed intervals
%
\begin{equation*}
  [0,1] = I^\alpha_n(0) \supseteq I^\alpha_n(1) \supseteq I^\alpha_n(2) \supseteq \cdots
\end{equation*}
%
whose intersection is a closed interval $\mathbf{I}^\alpha_n \defeq \bigcap_{k \in \NN} I^\alpha_n(k)$.
The endpoints of $\mathbf{I}^\alpha_n$ are $\alpha$-computable as \emph{lower} and \emph{upper} reals. Indeed, we can $\alpha$-computably enumerate a non-decreasing sequence of rationals whose supremum is the left-end point, and a non-increasing sequence of rationals whose infimum is the right-end point of~$\mathbf{I}^\alpha_n$.
%
Moreover, $\mathbf{I}^\alpha_n = [x,x]$ when $\pr[\alpha]{n}$ represents $x \in [0,1]$.

The story now repeats at the level of sequences. Given $a : \NN \to [0,1]$ and $n \in \NN$, for each~$\alpha$ such that $\alpha \in \invim{\srep}(a)$ the corresponding map $\pr[\alpha]{n}$ computes an interval~$\mathbf{I}^\alpha_n$ that depends on~$\alpha$, but we seek one that depends on~$a$ only.
%
The convex hull of the~$\mathbf{I}^\alpha_n$'s is the smallest interval that does the job:
%
\begin{equation*}
  \mathbf{J}^a_n \defeq
  \textstyle
  \hull \left(
    \bigcup_{\alpha \in \invim{\srep}(a)} \mathbf{I}^\alpha_n
  \right).
\end{equation*}
%
This is a closed interval because the union appearing in it is closed, even compact, for it is the projection of the set
%
\begin{equation*}
  C \defeq \set{
    (\alpha, x) \in \Cantor \times [0,1] \such
    \alpha \in \invim{\srep}(a) \land x \in \mathbf{I}^\alpha_n
  },
\end{equation*}
%
which we claim to be compact.
It suffices to check that $C$ is closed. Its membership relation is
%
\begin{align*}
  (\alpha, x) \in C
  &\liff
  \alpha \in \invim{\srep}(a) \land x \in \mathbf{I}^\alpha_n \\
  &\liff
  \alpha \in \invim{\srep}(a) \land \all{k \in \NN} x \in I^\alpha_n(k).
\end{align*}
%
This is a closed relation because~$\srep$ is continuous, and the endpoints of $I^\alpha_n(k)$ vary continuously in $\alpha$, $n$ and $k$.
%
If $n$ happens to be an $a$-index for~$x \in [0,1]$, then $\mathbf{J}^a_n = [x,x]$ because $\mathbf{I}^\alpha_n = [x,x]$ for all $\alpha \in \invim{\srep}(a)$.
%

The endpoints of $\mathbf{J}^a_n$ are more complicated than those of $\mathbf{I}^\alpha_n$. There is an $\alpha$-computable double sequence of rationals $q_{i,j}$ such that the left endpoint is $\inf_i \sup_j q_{i,j}$, and dually for the right endpoint.


\subsubsection{Construction of a Miller sequence}
\label{sec:constr-mill-sequ}

We prove \cref{thm:miller-sequence} by using the following generalization of Kakutani's fixed-point theorem, which itself is a generalization of Brouwer's fixed-point theorem. Depending on one's point of view, it is ironic or fascinating that such very classical theorems\footnote{Brouwer's fixed-point theorem has no constructive proofs, because a result of Orevkov's~\cite{orevkov63} implies that in the effective topos there is a continuous map $[0,1]^2 \to [0,1]^2$ which moves every point by a positive distance.} are used to construct an intuitionistic topos.

\begin{theorem}
  \label{thm:generalized-Brouwer}%
  If $F \subseteq [0,1]^\NN \times [0,1]^\NN$ is a closed set such that for each $a \in [0,1]^\NN$, the set $F[a] \defeq \set{b \in [0,1]^\NN \such (a, b) \in F}$ is non-empty and convex, then there is $\mil \in [0,1]^\NN$ such that $(\mil,\mil) \in F$.
\end{theorem}

\begin{proof}
  The statement is not easy to credit properly; see the paragraph after \cite[Thm~6.1]{miller04:_cont_deg} for a discussion which proposes \cite{Eilenberg1946} as the earliest work implying the statement given here.
\end{proof}

\Cref{thm:generalized-Brouwer} is a fixed-point theorem because a closed set $F \subseteq [0,1]^\NN \times [0,1]^\NN$ can be construed as the graph of an upper semicontinuous multivalued map taking each $a \in [0,1]^\NN$ to the non-empty set $F[a]$.
%
The sequence~$\mil$ is a fixed point in the sense that $\mil \in F[\mil]$.

In our case, we take $F$ to be essentially $\mathbf{J}$:
%
\begin{equation*}
  F \defeq \set{(a, b) \in [0,1]^\NN \times [0,1]^\NN \such \all{n \in \NN} b(n) \in \mathbf{J}^a_n},
\end{equation*}
%
or expressed as a multivalued map,
%
\begin{equation*}
  \textstyle
  F[a] \defeq \prod_{n \in \NN} \mathbf{J}^a_n.
\end{equation*}
%
% Verification the above are the same thing:
% b ∈ F[a] iff
% ∀ n . b(n) ∈ J^a_n iff
% (a, b) ∈ F
%
Let us verify the conditions of the theorem.
%
Obviously, $F[a]$ is non-empty and convex for all $a \in [0,1]^\NN$.
%
To see that~$F$ is closed, we unravel its definition in logical form:
%
\begin{align*}
  (a, b) \in F
  &\liff \all{n \in \NN} b(n) \in \mathbf{J}^a_n \\
  &\liff\textstyle
    \all{n \in \NN} b(n) \in
    \hull \left(
      \bigcup \set{ \mathbf{I}^\alpha_n \such \srep(\alpha) = a }
    \right) \\
  &\liff\textstyle
    \begin{aligned}[t]
      &\all{n \in \NN}
      \some{u, v \in [0,1]}
      u \leq b(n) \leq v \land {} \\
      &\quad
      (\some{\alpha \in \invim{\srep}(a)} u \in \mathbf{I}^\alpha_n)
      \land
      (\some{\beta \in \invim{\srep}(a)} v \in \smash{\mathbf{I}^\beta_n})
    \end{aligned}
  \\
  &\liff\textstyle
    \begin{aligned}[t]
      &\all{n \in \NN}
      \some{u, v \in [0,1]}
      u \leq b(n) \leq v \land {} \\
      &\quad
      (\some{\alpha \in \invim{\srep}(a)} \all{m \in \NN} u \in I^\alpha_n(m))
      \land {} \\
      &\quad
      (\some{\beta \in \invim{\srep}(a)} \all{m \in \NN} v \in I^\beta_n(m))
    \end{aligned}
\end{align*}
%
This is a closed condition: $\forall$ and $\land$ correspond to intersection, $\exists \alpha \in \invim{\srep}(a)$ to projection along the compact set $\invim{\srep}(a) \defeq \set{ \alpha \in \Cantor \such \srep(\alpha) = a}$, the relation $\leq$ is closed, projecting the $n$-th component $b(n)$ is continuous, and the endpoints of $I^\alpha_n(m)$ depend continuously on its parameters, in particular~$\alpha$.

It remains to verify that a fixed point $\mil \in F[\mil]$ is a Miller sequence. If $n \in \NN$ is a $\mil$-index of $x \in [0,1]$ then $\mil \in F[\mil]$ implies $\mil(n) \in \mathbf{J}^a_n = [x, x]$, hence $\mil(n) = x$, as required.

%%% Local Variables:
%%% mode: latex
%%% TeX-master: "countable-reals"
%%% End:

\section{Parameterized partial combinatory algebras}
\label{sec:parameterized-part-comb}

We seek a topos in which a Miller sequence is an epimorphism from the natural numbers to the Dedekind reals. Some sort of realizability model seems appropriate, although it cannot be an ordinary realizability topos, as those validate the axiom of countable choice.
%
\Cref{def:sequence-computable} specifies that $n \in \NN$ realizes $x \in [0,1]$ when it does so \emph{parameterically} in oracles representing $a : \NN \to [0,1]$. Therefore, in the present section we develop a general notion of parameterized computational models.

Let us take a moment to introduce notation that is commonly used in realizability theory. We already wrote $f : A \parto B$ to indicate a \defemph{partial map}, which is a map $f : A' \to B$ defined on a subset $A' \subseteq B$. For $x \in A$ we write $\defined{f(x)}$ when $f(x)$ is defined, i.e., when $x \in A'$. More generally, we write $\defined{e}$ when the expression~$e$ is well-defined, and hence so are all of its subexpressions. If $e_1$ and $e_2$ are two possibly undefined expressions, then $e_1 \kleq e_2$ means that if one is defined then so is the other and they are equal. In contrast, $e_1 = e_2$ asserts that both sides are defined and equal.

In realizability theory logical statements are witnessed by \emph{realizers}, which may be numbers, $\lambda$-terms, sequences or other data. A realizer is meant to represent computational evidence of a statement.
For instance, a realizer for $\some{x} \phi(x)$ encodes a specific $a$ for which $\phi(a)$ holds as well as a realizer for $\phi(a)$, and a realizer for $\phi \to \psi$ encodes a procedure for converting realizers for $\phi$ into realizers for $\psi$. In~\cref{sec:heyt-prealg-struct} we shall make these ideas precise.

In Stephen Kleene's original realizability interpretation of Heyting arithmetic~\cite{KleeneSC:intint} the realizers were numbers, whereas in a typical modern framework they are elements of a structure first defined by Solomon Feferman~\cite{feferman75}:

\begin{definition}
  \label{def:pca}%
  A \defemph{partial combinatory algebra (pca)} is given by a \defemph{carrier set} $\AA$, and a partial \defemph{application} operation ${\app} : \AA \times \AA \parto \AA$, such that there exist \defemph{basic combinators} $\combK, \combS \in \AA$ satisfying, for all $\R{a}, \R{b}, \R{c} \in \AA$,
  %
  \begin{align*}
    &\defined{(\combK \app \R{a})}, &
    (\combK \app \R{a}) \app \R{b} &= \R{a}, \\
    &\defined{((\combS \app \R{a}) \app \R{b})}, &
    ((\combS \app \R{a}) \app \R{b}) \app \R{c} &\kleq (\R{a} \app \R{c}) \app (\R{b} \app \R{c}).
  \end{align*}
\end{definition}

\noindent
To make notation more economical, we write application $\R{a} \app \R{b}$ as juxtaposition $\R{a} \, \R{b}$ and associate it to the left, $\R{a} \, \R{b} \, \R{c} = (\R{a} \, \R{b}) \, \R{c}$.

A non-trivial pca has much richer structure as it may seem at first sight.
%
For instance, we may encode in it the natural numbers and partial computable functions, as we shall for parameterized pcas in \cref{sec:progr-with-ppcas}.

\begin{example}
  \label{ex:pca-K-1}%
  The so-called Kleene's first algebra is the pca with the carrier set $\KK_1 \defeq \NN$ and application $m \cdot n \defeq \pr{m}(n)$, where $\pr{m}$ is the $m$-th partial computable map.
  %
  For any oracle $\alpha \in \Cantor$ the relativized version $\KK[\alpha]_1$ has the same carrier set and application $m \cdot n \defeq \pr[\alpha]{m}(n)$.
\end{example}

We refer to~\cite{oosten08:_realiz} for further examples of pcas and press on to the definition of parameterized pcas.

\begin{definition}
  \label{def:ppca}%
  A \defemph{parameterized partial combinatory algebra (ppca)} is given by
  %
  \begin{itemize}
  \item a \defemph{carrier set} $\AA$, whose elements are called \defemph{realizers},
  \item a non-empty \defemph{parameter set} $\PP$, whose elements are called \defemph{parameters},
  \item a partial \defemph{application} operation ${\app} : \PP \times \AA \times \AA \parto \AA$,
  \end{itemize}
  %
  such that there exist \defemph{basic combinators} $\combK, \combS \in \AA$, satisfying, for all $p, q \in \PP$ and $\R{a}, \R{b}, \R{c} \in \AA$,
  %
  \begin{align*}
    &\combK \app[p] \R{a} = \combK \app[q] \R{a},
    &
    \combK \app[p] \R{a} \app[p] \R{b} &= \R{a},
    \\
    &\combS \app[p] \R{a} = \combS \app[q] \R{a},
    &
    \combS \app[p] \R{a} \app[p] \R{b} \app[p] \R{c} &\kleq (\R{a} \app[p] \R{c}) \app[p] (\R{b} \app[p] \R{c}),
    \\
    &\combS \app[p] \R{a} \app[p] \R{b} = \combS \app[q] \R{a} \app[q] \R{b}.
  \end{align*}
  %
\end{definition}

The equations in the left column imply $\defined{(\combK \app[p] \R{a})}$ and $\defined{(\combS \app[p] \R{a} \app[p] \R{b})}$ for all $p \in \PP$ and $\R{a}, \R{b} \in \AA$.
%
For better readability we continued to write application as juxtaposition and let it associate to the left,
but we still need a better way of displaying the parameters, which we do by writing $p \at e$ to specify
that all applications in expression~$e$ should use parameter~$p$. We assign~$\at$ a lower precedence than to application so that $p \at e_1 \app e_2 = p \at (e_1 \app e_2)$. For example, the above equations can be written as
%
\begin{align*}
  &p \at \combK \, \R{a} = q \at \combK \, \R{a},
  &p \at \combK \, \R{a} \, \R{b} &= \R{a},
  \\
  &p \at \combS \, \R{a} = q \at \combS \, \R{a},
  &p \at \combS \, \R{a} \, \R{b} \, \R{c} &\kleq p \at (\R{a} \, \R{c}) \, (\R{b} \, \R{c}),
  \\
  &p \at \combS \, \R{a} \, \R{b} = q \at \combS \, \R{a} \, \R{b}.
\end{align*}
%
We sometimes write parentheses around $p \at e$ to improve readability, especially in equations.
A formal account of notation $p \at e$ is given in \cref{sec:comb-compl-ppcas}.


When no confusion may arise, we take the liberty of referring to a ppca $(\AA, \PP, {\cdot})$ just by the pair $(\AA, \PP)$.

\begin{example}
  An ordinary pca may be construed as a ppca whose parameter set is a singleton.
\end{example}

\begin{example}
  \label{ex:oracle-ppca}
  %
  The following is our main motivating example.
  %
  Recall from \cref{sec:oracle-comp-maps} that $\pr[\alpha]{m}$ stands for the partial $\alpha$-computable map coded by~$m$.
  %
  Let the carrier of the ppca be $\KK \defeq \NN$,
  the parameter set any non-empty set of oracles $\PP \subseteq \Cantor$,
  and application $m \app[\alpha] n \defeq \pr[\alpha]{m}(n)$.

  The combinator~$\combK$ is the code of a machine which accepts input~$n$ and outputs the code of a machine that always outputs~$n$. Such a machine does not consult the oracle, and neither does the machine that always outputs~$n$, hence $\combK \app[\alpha] n = \combK \app[\beta] n$ for all $n \in \NN$.

  To obtain~$\combS$, we apply the relativized smn theorem~\cite[Sect.~III.1.5]{soare87:_recur_enumer_sets_degrees} to first get a computable map $f : \NN \times \NN \to \NN$ such that
  %
  \begin{equation*}
    \pr[\alpha]{f(k, m)}(n) \simeq \pr[\alpha]{\pr[\alpha]{k}(n)}(\pr[\alpha]{m}(n)).
  \end{equation*}
  %
  We apply the theorem again to get a computable $g : \NN \to \NN$ such that
  %
  $\pr[\alpha]{g(k)}(m) = f(k, m)$, and let $\combS \in \NN$ be such that $\pr[\alpha]{\combS} = g$.
  %
  For all $\alpha \in \PP$ and $k, m \in \NN$ we have
  %
  \begin{equation*}
    \combS \app[\alpha] k =
    \pr[\alpha]{\combS}(k) =
    g(k)
  \end{equation*}
  %
  and
  %
  \begin{equation*}
    \combS \app[\alpha] k \app[\alpha] m =
    \pr[\alpha]{\pr[\alpha]{\combS}(k)}(m) =
    \pr[\alpha]{g(k)}(m) =
    f(k, m).
  \end{equation*}
  %
  Being computable, $g$ and $f$ do not depend on the oracle, therefore
  $\combS \app[\alpha] k = \combS \app[\beta] k$
  and
  $\combS \app[\alpha] k \app[\alpha] m = \combS \app[\beta] k \app[\beta] m$ for all $\beta \in \PP$.
  Finally, the defining equation for~$f$ guarantees that
  $\combS \app[\alpha] k \app[\alpha] m \app[\alpha] n \simeq
   (k \app[\alpha] n) \app[\alpha] (m \app[\alpha] n)$.
\end{example}


\begin{example}
  \label{ex:general-oracle-ppca}%
  %
  The previous example generalizes to Jaap van Oosten's construction~\cite[Thm~1.7.5]{oosten08:_realiz} which from any pca $(\AA, {\cdot})$ and a partial map $\xi : \AA \parto \AA$ constructs a new pca $(\AA^\xi, {\app[\xi]})$ with~$\xi$ acting as an oracle.
  The carrier set is unchanged $\AA^\xi \defeq \AA$, while application~$\app[\xi]$ represents a dialogue between a computation in~$\AA$ and the oracle~$\xi$.
  %
  When the construction is applied to $\KK_1$ and $\alpha \in \Cantor$, we obtain a pca that is (isomorphic to)
  the relativized pca $\KK[\alpha]_1$ from \cref{ex:pca-K-1}.

  As it turns out, the construction is uniform in~$\xi$, in the sense that $\combK_\xi, \combS_\xi \in \AA^\xi$ do not depend on~$\xi$, and neither do $\combK_\xi \app[\xi] \R{a}$, $\combS_\xi \app[\xi] \R{a}$, and $\combS_\xi \app[\xi] \R{a} \app[\xi] \R{b}$.
  %
  The conditions for having a ppca are thus met: the carrier set is $\AA$ and the parameter set is any non-empty
  subset $\PP \subseteq \AA \parto \AA$.
\end{example}

% Andrew Swan says every ppca is of the above form.


\subsection{Combinatory completeness of ppcas}
\label{sec:comb-compl-ppcas}

Partial combinatory algebras have the so-called property of \emph{combinatory completeness}.
We formulate an analogous notion for parameterized pcas.

The set of~\defemph{expressions in variables $x_1, \ldots, x_n$} over a ppca $(\AA, \PP)$ is defined inductively:
any variable $x_i$ is an expression, any constant $\R{a} \in \AA$ is an expression, and a formal application $e_1 \cdot e_2$ is an expression if~$e_1$ and~$e_2$ are.
%
We continue to write application as juxtaposition and associate it to the left.
%
An expression~$e$ in no variables is \defemph{closed}. For any $p \in \PP$ and a closed expression~$e$,
define $p \at e$ recursively by
%
\begin{align*}
  p \at \R{a} &\defeq \R{a} &&\text{if $\R{a} \in \AA$,}
  \\
  (p \at e_1 \app e_2) &\defeq (p \at e_1) \app[p] (p \at e_2)
  &&\text{if $e_1$ and $e_2$ are closed expressions.}
\end{align*}
%
Note that $p \at e$ may be undefined.

For a variable $x$ and an expression $e$, let the \defemph{abstraction} $\abstr{x} e$ be the expression defined inductively as
%
\begin{align*}
  \abstr{x} y &\defeq \combK \, y & &\text{if $y$ is a variable distinct from $x$} \\
  \abstr{x} x &\defeq \combS \, \combK \, \combK \\
  \abstr{x} \R{a} &\defeq \combK \, a & &\text{for a constant $\R{a} \in \AA$} \\
  \abstr{x} e_1 e_2 &\defeq \combS \, (\abstr{x} e_1) \, (\abstr{x} e_2).
\end{align*}
%
We write $e[\R{a}_1/x_1, \ldots, \R{a}_n/x_n]$ for $e$ with $\R{a}_i$'s substituted for~$x_i$'s. We abbreviate the substitution $[\R{a}_1/x_1, \ldots, \R{a}_n/x_n]$ as $[\vec{\R{a}}/\vec{x}]$, which allows us to write $e[\vec{\R{a}}/\vec{x}]$. To be precise, substitution is defined as follows:
%
\begin{align*}
  x_i [\vec{\R{a}}/\vec{x}] &\defeq \R{a}_i \\
  y [\vec{\R{a}}/\vec{x}] &\defeq y &&\text{if $y \not\in \set{x_1, \ldots, x_n}$} \\
  \R{b} [\vec{\R{a}}/\vec{x}] &\defeq \R{b} &&\text{if $\R{b} \in \AA$} \\
  (e_1 \, e_2) [\vec{\R{a}}/\vec{x}] &\defeq (e_1[\vec{\R{a}}/\vec{x}]) \, (e_2[\vec{\R{a}}/\vec{x}]).
\end{align*}

\begin{lemma}
  \label{lem:abstr-subst-commute}%
  If $y \not\in \set{x_1, \ldots, x_n}$ then $(\abstr{y} e)[\vec{\R{a}}/\vec{x}] = \abstr{y} (e[\vec{\R{a}}/\vec{x}])$.
\end{lemma}

\begin{proof}
  A straightforward induction on the structure of~$e$.
\end{proof}


\begin{lemma}
  \label{lem:abstr-p-defined}
  For any expression~$e$ in variables $x_1, \ldots, x_n, y$, the value $p \at (\abstr{y} e)[\vec{\R{a}}/\vec{x}]$ is defined for all $p \in \PP$ and $\R{a}_1, \ldots, \R{a}_n \in \AA$.
\end{lemma}

\begin{proof}
  We proceed by induction on the structure of~$e$:
  % 
  \begin{itemize}
  \item if $e = x_i$ then $p \at (\abstr{y} e)[\vec{\R{a}}/\vec{x}] = \combK \app[p] \R{a}_i$, which is defined,
  \item if $e = y$ then $p \at (\abstr{y} e)[\vec{\R{a}}/\vec{x}] = \combS \app[p] \combK \app[p] \combK$, which is defined,
  \item if $e = \R{a}$ for $\R{a} \in \AA$ then $p \at (\abstr{y} e)[\vec{\R{a}}/\vec{x}] = \combK \app[p] \R{a}$, which is defined,
  \item if $e = e_1 \app e_2$ then
    % 
    $
    p \at (\abstr{y} e)[\vec{\R{a}}/\vec{x}] =
    \combS \app[p] ((\abstr{y} e_1)[\vec{\R{a}}/\vec{x}]) \app[p] ((\abstr{y} e_2)[\vec{\R{a}}/\vec{x}])
    $,
    % 
    which is defined because by induction hypotheses both arguments of~$\combS$ are defined.
    \qedhere
  \end{itemize}
\end{proof}

\begin{lemma}
  \label{lem:abstr-compute}%
  %
  For any expression $e$ in variables $x_1, \ldots, x_n, y$, parameter $p \in \PP$, and $\R{a}_1, \ldots, \R{a}_n, \R{b} \in \AA$, 
  % 
  \begin{equation*}
    p \at ((\abstr{y} e) \, \R{b})[\vec{\R{a}}/\vec{x}] \kleq p \at e[\vec{\R{a}}/\vec{x}, \R{b}/y].
  \end{equation*}
\end{lemma}

\begin{proof}
  We proceed by induction on the structure of~$e$. If $e = x_i$ then
%
\begin{equation*}
  p \at ((\abstr{y} x_i) \, \R{b})[\vec{\R{a}}/\vec{x}] \kleq
  p \at \combK \, \R{a}_i \, \, \R{b} =
  p \at \R{a}_i =
  p \at x_i[\vec{\R{a}}/\vec{x}, \R{b}/y].
\end{equation*}
%
If $e = y$ then
%
\begin{equation*}
  p \at ((\abstr{y} y) \, \R{b})[\vec{\R{a}}/\vec{x}] \kleq
  p \at \combS \, \combK \, \combK \, \R{b} =
  p \at \R{b} =
  p \at y[\vec{\R{a}}/\vec{x}, \R{b}/y].
\end{equation*}
%
If $e = \R{a} \in \AA$ then
%
\begin{equation*}
  p \at ((\abstr{y} \R{a}) \, \R{b})[\vec{\R{a}}/\vec{x}] \kleq
  p \at \combK \, \R{a} \, \, \R{b} =
  p \at \R{a} =
  p \at \R{a}[\vec{\R{a}}/\vec{x}, \R{b}/y].
\end{equation*}
%
Finally, if $e = e_1 \, e_2$ then
%
\begin{align*}
  p \at ((\abstr{y} e_1 \, e_2) \, \R{b})[\vec{\R{a}}/\vec{x}]
  &\kleq
  p \at \combS \, ((\abstr{y} e_1)[\vec{\R{a}}/\vec{x}]) \, ((\abstr{y} e_2)[\vec{\R{a}}/\vec{x}]) \, \R{b} \\
  &\kleq
  p \at ((\abstr{y} e_1)[\vec{\R{a}}/\vec{x}] \, \R{b}) \, ((\abstr{y} e_2)[\vec{\R{a}}/\vec{x}] \, \R{b}) \\
  &\kleq
  p \at (e_1[\vec{\R{a}}/\vec{x},\R{b}/y]) \, (e_2[\vec{\R{a}}/\vec{x},\R{b}/y]) \\
  &\kleq
  p \at (e_1 \, e_2)[\vec{\R{a}}/\vec{x},\R{b}/y].
\end{align*}
%
The passage from the first to the second row is secured by \cref{lem:abstr-p-defined}, and from the third to the fourth by the induction hypotheses.
\end{proof}

Let us give a name to those expressions that are independent of the parameter.

\begin{definition}
  \label{def:uniform}%
  A closed expression~$e$ is \defemph{uniform} when $p \at e = q \at e$ for all $p, q \in \PP$.
  When this is the case, there is a unique $\ucode{e} \in \AA$ such that $p \at e = \ucode{e}$ for all $p \in \PP$.
\end{definition}

\Cref{def:ppca} postulates that $\combK \, \R{a}$ and $\combS \, \R{a} \, \R{b}$ are uniform for all $\R{a}, \R{b} \in \AA$. In subsequent calculations we shall frequently use the fact that $p \at \ucode{e} = p \at e$ when~$e$ is uniform.

\begin{lemma}
  \label{lem:abstr-uniform}%
  A closed abstraction $\abstr{x} e$ is uniform.
\end{lemma}

\begin{proof}
  We proceed by induction on the structure of~$e$.
  If $e$ is the variable $x$ then $\abstr{x} e = \combS \, \combK \, \combK$, which is uniform.
  If $e$ is a constant $\R{a} \in \AA$ then $\abstr{x} e = \combK \, \R{a}$, which is uniform.
  If $e = e_1 \, e_2$ then $\abstr{x} e = \combS \, (\abstr{x} e_1) \, (\abstr{x} e_2)$, which is uniform by induction hypotheses.
\end{proof}


\begin{theorem}[Combinatory completeness]
  For any expression~$e$ over a ppca $(\AA, \PP)$ in variables~$x_1, \ldots, x_n, x_{n+1}$, there is $e^{*} \in \AA$ such that, for all $p \in \PP$ and $\R{a}_1, \ldots, \R{a}_{n+1} \in \AA$, the expression $e^{*} \, \R{a}_1 \cdots \R{a}_n$ is uniform and
  \begin{equation*}
    (p \at e^{*} \, \R{a}_1 \cdots \R{a}_{n+1})
    \kleq
    (p \at e[\R{a}_1/x_1, \ldots, \R{a}_{n+1}/x_{n+1}]).
  \end{equation*}
\end{theorem}

\begin{proof}
  The usual proof for ordinary partial combinatory algebras can be mimicked.
  %
  Let $e_{n+1} \defeq \abstr{x_{n+1}} e$ and $e_k = \abstr{x_k} e_{k+1}$ for $k = 1, \ldots, n$.
  By \cref{lem:abstr-uniform}, $e_1$ is uniform, so $e^{*} \defeq \ucode{e_1}$ is well defined,
  and we claim it is the element we are looking for.
  %
  Given $p \in \PP$ and $\R{a}_1, \ldots, \R{a}_{n+1} \in \AA$, \cref{lem:abstr-compute,lem:abstr-subst-commute} imply
  %
  \begin{align*}
    p \at e_1 \, \R{a}_1 \cdots \R{a}_n
    &\kleq p \at (e_2[\R{a}_1/x_1]) \, \R{a}_2 \cdots \R{a}_n \\
    &\kleq p \at (e_3[\R{a}_1/x_1, \R{a}_2/x_2]) \, \R{a}_3 \cdots \R{a}_n \\
    &\qquad \vdots \\
    &\kleq p \at (\abstr{x_{n+1}} e)[\R{a}_1/x_1, \ldots, \R{a}_n/x_n] \\
    &\kleq p \at \abstr{x_{n+1}} (e [\R{a}_1/x_1, \ldots, \R{a}_n/x_n]).
  \end{align*}
  %
  The last row is defined by \cref{lem:abstr-p-defined} and uniform by \cref{lem:abstr-uniform},
  therefore so is the first one.
  Finally, \cref{lem:abstr-compute} implies
  %
  \begin{equation*}
    p \at e_1 \, \R{a}_1 \cdots \R{a}_{n+1}
    \kleq 
    p \at e[\R{a}_1/x_1, \ldots, \R{a}_{n+1}/x_{n+1}]. \qedhere
  \end{equation*}
\end{proof}

\subsection{Programming with ppcas}
\label{sec:progr-with-ppcas}

Combinatory completeness can be used to write complex programs in any ppca, just like in ordinary partial combinatory algebras. For example, $\comb{id} \defeq \ucode{\abstr{x} x} = \ucode{\comb{s} \, \comb{k} \, \comb{k}}$ realizes the identity map.
%
More interesting are pairing, projections, booleans and the conditional:
%
\begin{align*}
  \combPair &\defeq \ucode{\abstr{x y z}{z\, x\, y}},
  &
  \combIf &\defeq \ucode{\abstr{x} x},
  \\
  \combFst &\defeq \ucode{\abstr{z}{z \, (\abstr{x\,y} x)}},
  &
  \combTrue &\defeq \ucode{\abstr{x\,y} x},
  \\
  \combSnd &\defeq \ucode{\abstr{z}{z \, (\abstr{x\,y} y)}},
  &
  \combFalse &\defeq \ucode{\abstr{x\,y} y}.
\end{align*}
%
These are all uniform by \cref{lem:abstr-uniform}. They satisfy the expected equations parameter-wise, for all $p \in \PP$ and $\R{a}, \R{b} \in \AA$:
%
\begin{align*}
  (p \at \combFst \, (\combPair \, \R{a} \, \R{b})) &= \R{a}, &
  (p \at \combIf \, \combTrue \, \R{a} \, \R{b}) &= \R{a}, \\
  (p \at \combSnd \, (\combPair \, \R{a} \, \R{b})) &= \R{b}, &
  (p \at \combIf \, \combFalse \, \R{a} \, \R{b}) &= \R{b}.
\end{align*}

Natural numbers are encoded as \defemph{Curry numerals}:
%
\begin{align*}
  \numeral{0} &\defeq \ucode{\combS\, \combK\, \combK},
  &
  \numeral{n+1} &\defeq \ucode{\combPair \, \combFalse \, \numeral{n}}
\end{align*}
%
Successor, predecessor and zero-testing are defined as
%
\begin{align}
  \comb{succ} &\defeq \ucode{\abstr{x}{\combPair \, \combFalse \,x}}, \label{eq:comb-succ} \\ 
  \comb{iszero} &\defeq \ucode{\combFst}, \notag\\
  \comb{pred} &\defeq
  \ucode{\abstr{x}{\combIf\, (\comb{iszero}\, x)\, \numeral{0}\, (\combSnd\, x)}}. \notag
\end{align}
%
These are again uniform and satisfy the expected equations.

To get recursive definitions going, we define the fixed-point combinators~$\comb{Y}$ and~$\comb{Z}$:
%
%
\begin{align*}
  \R{W} &\defeq \ucode{\abstr{x \, y} y \, (x \, x \, y)},
  &
  \comb{Y} &\defeq \ucode{\R{W} \, \R{W}},
  \\
  \R{X} &\defeq \ucode{\abstr{x\, y\, z} y \, (x \, x \, y) \, z},
  &
  \comb{Z} &\defeq \ucode{\R{X} \, \R{X}}.
\end{align*}
%
Then for all $p \in \PP$ and $\R{f}, \R{a} \in \AA$, $\comb{Z} \, \R{f}$ is uniform,
%
\begin{equation*}
  p \at \comb{Y} \, \R{f} \kleq p \at \R{f} \, (\comb{Y} \, \R{f})
  \qquad\text{and}\qquad
  p \at \comb{Z} \, \R{f} \, \R{a} \kleq p \at \R{f} \, (\comb{Z} \, \R{f}) \, \R{a}.
\end{equation*}
%
% Verification that Y and Z work:
%
% p | Y f ≃
% p | W W f ≃
% p | f (W W f) ≃
% p | f (Y f)
%
% p | Z f a ≃
% p | X X f a ≃
% p | f (X X f) a ≃
% p | f (Z f a) a.
%
For instance, by repeatedly using \cref{lem:abstr-compute} we compute
%
\begin{equation*}
  p \at \comb{Z} \, \R{f} \, \R{a} \kleq
  p \at \R{X} \, \R{X} \, \R{f} \, \R{a} \kleq
  p \at \R{f} \, (\R{X} \, \R{X} \, \R{f}) \, \R{a} \kleq
  p \at \R{f} \, (\comb{Z} \, \R{f}) \, \R{a}.
\end{equation*}
%
With $\comb{Y}$ in hand primitive recursion on natural numbers is realized as
%
\begin{equation*}
  \comb{primrec} \defeq
  \ucode{\abstr{x \, \R{f} \, m} ((\comb{Z} \, \R{R}) \, x \, \R{f} \, m \, \comb{id})}
\end{equation*}
%
where
%
\begin{equation*}
  \R{R} \defeq \ucode{
      \abstr{r \, x \, \R{f} \, m}
      \comb{if} \, (\comb{iszero} \, m) \,
          (\combK \, x) \,
          (\abstr{y} \R{f} \, (\comb{pred} \, m) \, (r \, x \, \R{f} \, (\comb{pred} \, m) \, \comb{id}))
  }.
\end{equation*}
%
It satisfies, for all $p \in \PP$, $\R{a}, \R{f} \in \AA$ and $n \in \NN$,
%
\begin{align*}
  (p \at \comb{primrec} \, \R{a} \, \R{f} \, \numeral{0}) &= \R{a},
  &
  (p \at \comb{primrec} \, \R{a} \, \R{f} \, \numeral{n + 1}) &\kleq
  \R{f} \, \numeral{n} \, (\comb{primrec} \, \R{a} \, \R{f} \, \numeral{n}).
\end{align*}

% Verification that primrec works:
%
% primec a f 0
% = (Z R) a f 0 id
% = R (Z R) a f 0 id
% = if (iszero 0) (k a) (...) id
% = (if true (k a) (...)) id
% = k a id
% = a
%
% primrec a f (n+1)
% = (Z R) a f (n+1) id
% = R (Z R) a f (n+1) id
% = (if (iszero (n+1)) (k a) (<y> f (pred (n+1)) ((R Z) a f (pred (n+1)) id))) id
% = (if false (k a) (<y> f (pred (n+1)) ((R Z) a f (pred (n+1)) id))) id
% = (<y> f (pred (n+1)) ((R Z) a f (pred (n+1)) id)) id
% = f (pred (n+1) ((R Z) a f (pred (n+1)) id)
% = f n ((R Z) a f n id)
% = f n (primrec a f n)
%
% Note: the trailing id is there (I think) to make sure both branches of the conditional are defined

\begin{example}
  \label{ex:numers-vs-numerals}
  It will be useful to know that in the ppca $(\KK, \PP)$ from \cref{ex:oracle-ppca} numbers can be converted to numerals and vice versa. For this purpose we construct realizers $\combNum, \combCur \in \KK$ such that for all $\alpha \in \PP$ and $n \in \NN$
  % 
  \begin{equation*}
    \alpha \at \combNum \, \numeral{n} = \alpha \at n
    \quad\text{and}\quad
    \alpha \at \combCur \, n = \alpha \at \numeral{n}.
  \end{equation*}
  % 
  To convert numerals to numbers, observe that there is $s \in \NN$, independent of~$\alpha$, such that $\pr[\alpha]{s}(n) = n + 1$, and define
  % 
  $\combNum \defeq \ucode{\comb{primrec} \, 0 \, s}$.
  %
  To implement the reverse translation, we apply
  the relativized Kleene's recursion theorem~\cite[Sect.~III.1.6]{soare87:_recur_enumer_sets_degrees}
  to find $r \in \NN$, independent of~$\alpha$, such that
  % 
  \begin{equation*}
    \pr[\alpha]{r}(n) =
    \begin{cases}
      \ucode{\numeral{0}} & \text{if $n = 0$,}\\
      \pr[\alpha]{\ucode{\comb{succ}}}(\pr[\alpha]{r}(n-1)) & \text{if $n > 0$.}
    \end{cases}
  \end{equation*}
  % 
  We may take $\combCur \defeq r$ because
  % 
  $\alpha \at r \, 0 = \pr[\alpha]{r}(0) = \ucode{\numeral{0}} = (\alpha \at \numeral{0})$
  and, assuming $\alpha \at r \, n = \alpha \at \numeral{n}$ for the sake of the induction step,
  % 
  \begin{multline*}
    \alpha \at r \, (n+1) =
    \pr[\alpha]{r}(n+1) =
    \pr[\alpha]{\ucode{\comb{succ}}}(\pr[\alpha]{r}(n)) = \\
    \alpha \at \comb{succ} \, (r \, n) =
    \alpha \at \comb{succ} \, \numeral{n} =
    \alpha \at \numeral{n + 1}.
  \end{multline*}
\end{example}

%%% Local Variables:
%%% mode: latex
%%% TeX-master: "countable-reals"
%%% End:
 
\section{Parameterized realizability}
\label{sec:unif-real}

We next devise a notion of realizability based on ppcas that captures the uniformity of oracle computations from \cref{sec:non-diag-sequ}.
%
We use the tripos-to-topos construction~\cite{hyland80:_tripos}, a general technique for defining toposes. We refer the readers to~\cite[Sect.~S2.1]{oosten08:_realiz} for background material.
%
For the remainder of this section we fix a ppca $(\AA, \PP)$.

\subsection{The parameterized realizability tripos}
\label{sec:tripos-built-from}

In the first step of the construction we shall define a contravariant functor
%
\begin{equation*}
  \PredSymbol : \op{\Set} \to \Heyt,
\end{equation*}
%
from sets to Heyting prealgebras, satisfying further conditions to be given later.
%
% AB: We pick sets because that's what we need to get a realizability topos. Maybe there's no need to create
% confusion here. Experts know, and the rest will just wonder why we're mentioning this.
Recall that a Heyting prealgebra $(H, {\leq})$ is a set $H$ with a reflexive transitive relation~$\leq$ with elements $\bot$, $\top$ and binary operations $\land$, $\lor$, $\limply$, satisfying the laws of intuitionistic propositional calculus.

For any set~$X$, define
%
\begin{equation*}
  \Pred[\AA,\PP]{X} \defeq (\pow{\AA}^X, {\leq_X}),
\end{equation*}
%
where the preorder~$\leq_X$ will be defined momentarily.
%
When no confusion can arise, we abbreviate $\Pred[\AA,\PP]{X}$ as $\Pred{X}$.

An element $\phi \in \Pred{X}$ is called a \defemph{(tripos) predicate} on~$X$.
%
We say that $\R{a} \in \phi(x)$ \emph{realizes} $\phi(x)$.
%
For a closed expression~$e$ over~$\AA$, we define $e \rz[p] \phi(x)$ by
%
\begin{equation*}
  e \rz[p] \phi(x)
  \defiff
  \defined{(p \at e)} \land (p \at e) \in \phi(x)
\end{equation*}
%
and read it as ``$e$ realizes $\phi(x)$ at $p$''.
%
The preoder on $\Pred{X}$ is defined as follows, for $\phi, \psi \in \Pred{X}$:
%
\begin{equation*}
  \phi \leq_X \psi
  \defiff
    \some{\R{a} \in \AA}
    \all{x \in X}
    \all{\R{b} \in \phi(x)}
    \all{p \in \PP}
    \R{a} \, \R{b} \rz[p] \psi(x).
\end{equation*}
%
We say that $a$ satisfying the above condition \emph{realizes} $\phi \leq_X \psi$.
Reflexivity of~$\leq_X$ is realized by $\ucode{\abstr{x} x}$. For transitivity, one checks that if $\R{a}$ realizes $\phi \leq_X \psi$ and $\R{b}$ realizes $\psi \leq_X \chi$ then $\ucode{\abstr{x} \R{b} \, (\R{a} \, x)}$ realizes $\phi \leq_X \chi$.

In order for $\PredSymbol$ to be a bona fide functor, we let it take a map $r : Y \to X$ to the \defemph{reindexing} map $\invim{r} : \Pred{X} \to \Pred{Y}$, which acts by precomposition $\invim{r} \phi = \phi \circ r$. This is obviously contravariant and functorial, and we shall check that $\invim{r}$ is a homomorphism below.

In order to show that $\PredSymbol$ is a tripos, we must verify the following conditions:
%
\begin{enumerate}
\item For every set $X$ the poset $\Pred{X}$ is a Heyting prealgebra (\cref{sec:heyt-prealg-struct}).
\item Reindexing is a homomorphism of Heyting prealgebras (\cref{sec:monot-reind}).
\item Universal and existential quantifiers exist for $\Pred{X}$ (\cref{sec:quantifiers}).
\item There is a generic element (\cref{sec:generic-element}).
\end{enumerate}
%
The arguments are quite similar to those for the tripos arising from an ordinary pca~\cite[Prop.~1.2.1]{oosten08:_realiz}, one only has to pay attention to the presence of parameters.

\subsection{The Heyting prealgebra structure}
\label{sec:heyt-prealg-struct}

The Heyting structure on $\Pred{X}$ is as follows:
%
{\allowdisplaybreaks
\begin{align*}
  \top(x) &\defeq \AA,\\
  \bot(x) &\defeq \emptyset,\\
  (\phi \land \psi)(x) &\defeq \set{\R{a} \in \AA \such
      \all{p \in \PP}
      \combFst \, \R{a} \rz[p] \phi(x) \land
      \combSnd \, \R{a} \rz[p] \psi(x)
  },
  \\
  (\phi \lor \psi)(x) &\defeq \{\R{a} \in \AA \such
    \all{p \in \PP}
    \begin{aligned}[t]
    &(p \at \combFst \, \R{a} = \ucode{\combTrue} \land \combSnd \, \R{a} \rz[p] \phi(x))
    \lor {} \\
    &(p \at \combFst \, \R{a} = \ucode{\combFalse} \land \combSnd \, \R{a} \rz[p] \psi(x)) \},
    \end{aligned}
  \\
  (\phi \limply \psi)(x) &\defeq \set{\R{a} \in \AA \such
    \all{p \in \PP} \all{\R{b} \in \phi(x)} \R{a} \, \R{b} \rz[p] \psi(x)}.
\end{align*}
}
%
The above is like the analogous Heyting structure for ordinary pcas, except that realizers must be uniform in~$p \in \PP$. Next, we verify that the given operations satisfy the laws of a Heyting prealgebra.

\subsubsection{Falsehood and truth}
\label{sec:falsehood-truth}

Both $\bot \leq_X \phi$ and $\phi \leq_X \top$ are realized by $\combK \, \combK$.

\subsubsection{Conjunction}
\label{sec:conjunction}

We need to verify that, for all $\phi, \psi, \chi \in \Pred{X}$,
%
\begin{equation*}
  (\chi \leq_X \phi) \land (\chi \leq_X \psi) \iff \chi \leq_X \phi \land \psi.
\end{equation*}
%
If $\R{a}$ and $\R{b}$ realize $\chi \leq_X \phi$ and $\chi \leq_X \psi$, respectively, then $\chi \leq_X \phi \land \psi$ is realized by $\R{c} \defeq \ucode{\abstr{u} \combPair \, (\R{a} \, u) \, (\R{b} \, u)}$. Indeed, for any $x \in X$, $p \in \PP$ and $\R{d} \in \chi(x)$, we have
%
\begin{equation*}
  p \at \combFst \, (\R{c} \, \R{d})
  \kleq
  p \at \combFst \, (\combPair \, (\R{a} \, \R{d}) \, (\R{b} \, \R{d}))
  \kleq
  p \at \R{a} \, \R{d}.
\end{equation*}
%
Because $\R{a} \, \R{d} \rz[p] \phi(x)$, it follows that $\combFst \, (\R{c} \, \R{d}) \rz[p] \phi(x)$.
The argument for the second component is analogous.

Conversely, if $\R{a}$ realizes $\chi \leq_X \phi \land \psi$ then $\R{b} \defeq \ucode{\abstr{u} \combFst \, (\R{a} \, u)}$ and $\R{c} \defeq \ucode{\abstr{v} \combSnd \, (\R{a} \, v)}$ realize $\chi \leq_X \phi$ and $\chi \leq_X \psi$, respectively. Indeed, for any $x \in X$, $p \in \PP$ and $\R{d} \in \chi(x)$, we have $\R{a} \, \R{d} \rz[p] (\phi \land \psi)(x)$ and $p \at \R{b} \, (\R{a} \, \R{d}) = p \at \combFst \, (\R{a} \, \R{d})$, hence $\R{b} \, (\R{a} \, \R{d}) \rz[p] \phi(x)$.
The argument for $\R{c}$ and $\chi \leq_X \psi$ is analogous.

\subsubsection{Disjunction}
\label{sec:disjunction}

Disjunction is characterized by
%
\begin{equation*}
  (\phi \leq_X \chi) \land (\psi \leq_X \chi) \iff \phi \lor \psi \leq_X \chi.
\end{equation*}
%
If $\R{a}$ and $\R{b}$ respectively realize $\phi \leq_X \chi$ and $\psi \leq_X \chi$, then $\phi \lor \psi \leq_X \chi$ is realized by
%
\begin{equation*}
  \R{c} \defeq
  \ucode{\abstr{u} \combIf \, (\combFst \, u) \, (\R{a} \, (\combSnd \, u)) \, (\R{b} \, (\combSnd \, u))}.
\end{equation*}
%
Consider any $x \in X$, $p \in \PP$ and $\R{d} \in (\phi \lor \psi)(x)$.
If $p \at \combFst \, \R{d} = \ucode{\combTrue}$ then
%
\begin{equation*}
  p \at \R{c} \, \R{d}
  \kleq
  p \at \combIf \, (\combFst \, \R{d}) \, (\R{a} \, (\combSnd \, \R{d})) \, (\R{b} \, (\combSnd \, \R{d}))
  \kleq
  p \at \R{a} \, (\combSnd \, \R{d}),
\end{equation*}
%
and since $\R{a} \, (\combSnd \, \R{d}) \rz[p] \chi(x)$ also $\R{c} \, \R{d} \rz[p] \chi(x)$.
%
If $p \at \combFst \, \R{d} = \ucode{\combFalse}$ then
%
\begin{equation*}
  p \at \R{c} \, \R{d}
  \kleq
  p \at \combIf \, (\combFst \, \R{d}) \, (\R{a} \, (\combSnd \, \R{d})) \, (\R{b} \, (\combSnd \, \R{d}))
  \kleq
  p \at \R{b} \, (\combSnd \, \R{d}),
\end{equation*}
%
and since $\R{b} \, (\combSnd \, \R{d}) \rz[p] \chi(x)$ also $\R{c} \, \R{d} \rz[p] \chi(x)$.

Conversely, if $\R{c}$ realizes $\phi \lor \psi \leq_X \chi$ then $\phi \leq_X \chi$ and $\psi \leq_X \chi$ are respectively realized by $\R{a} \defeq \ucode{\abstr{u} \R{c} \, (\combPair \, \combTrue \, u)}$ and $\R{b} \defeq \ucode{\abstr{v} \R{c} \, (\combPair \, \combFalse \, v)}$.
%
Indeed, for any $x \in X$, $p \in \PP$ and $\R{d} \in \phi(x)$ we have $\combPair \, \combTrue \, \R{d} \rz[p] (\phi \lor \psi)(x)$, hence $\R{c} \, (\combPair \, \combTrue \, \R{d}) \rz[p] \chi(x)$. Now $\R{a} \, \R{d} \rz[p] \chi(x)$ holds because
$p \at \R{a} \, \R{d} \kleq p \at \R{c} \, (\combPair \, \combTrue \, \R{d})$.
The argument for $\R{b}$ and $\psi \leq_X \chi$ is analogous.

\subsubsection{Implication}
\label{sec:implication}

Implication is characterzied by the adjunction
%
\begin{equation*}
   \phi \leq_X \psi \limply \chi\iff \phi \land \psi \leq_X \chi.
\end{equation*}
%
If $\R{a}$ realizes $\phi \leq_X \psi \limply \chi$ then $\R{b} \defeq \ucode{\abstr{x} \R{a} \, (\combFst \, x) \, (\combSnd \, x)}$ realizes $\phi \land \psi \leq_X \chi$. Indeed, for any $x \in X$, $p \in \PP$ and $\R{c} \in (\phi \land \psi)(x)$ we have
%
$
  p \at \R{b} \, \R{c}
  \kleq
  p \at \R{a} \, (\combFst \, \R{c}) \, (\combSnd \, \R{c})
$
%
and $\R{a} \, (\combFst \, \R{c}) \, (\combSnd \, \R{c}) \rz[p] \chi(x)$, hence $\R{b} \, \R{c} \rz[p] \chi(x)$.

Conversely, if $\R{b}$ realizes $\phi \land \psi \leq_X \chi$, then $\R{a} \defeq \ucode{\abstr{u} \abstr{v} \R{b} \, (\combPair \, u \, v)}$ realizes $\phi \leq_X \psi \limply \chi$.
To see this, we must verify for any $x \in X$, $p \in \PP$ and $\R{c} \in \phi(x)$ that $\R{a} \, \R{c} \rz[p] (\psi \limply \chi)(x)$.
%
Consider any $q \in \PP$ and $\R{d} \in \psi(x)$.
%
By \cref{lem:abstr-uniform}
%
\begin{equation*}
  p \at \R{a} \, \R{c} =
  p \at \abstr{v} \R{b} \, (\combPair \, \R{c} \, v) =
  q \at \abstr{v} \R{b} \, (\combPair \, \R{c} \, v) =
  q \at \R{a} \, \R{c},
\end{equation*}
%
hence
%
$
  q \at (\R{a} \app[p] \R{c}) \, \R{d} \kleq
  q \at \R{a} \, \R{c} \, \R{d} \kleq
  q \at \R{b} \, (\combPair \, \R{c} \, \R{d})
$,
%
and because $\R{b} \, (\combPair \, \R{c} \, \R{d}) \rz[q] \chi(x)$, it follows that $(\R{a} \app[p] \R{c}) \, \R{d} \rz[q] \chi(x)$.

\subsubsection{Negation}
\label{sec:negation}

In intuitionistic logic negation $\neg \phi$ is defined as $\phi \limply \bot$. A short calculation reveals that
%
\begin{align*}
  (\neg \phi)(x) &= \set{\R{a} \in \AA \such \phi(x) = \emptyset} \\
  (\neg\neg \phi)(x) &= \set{\R{a} \in \AA \such \phi(x) \neq \emptyset}.
\end{align*}
%

\subsection{Reindexing preserves the Heyting structure}
\label{sec:monot-reind}

We should not forget to check that $\invim{r} : \Pred{X} \to \Pred{Y}$ induced by $r : Y \to X$ is a homomorphism of Heyting prealgebras. This is easy, one just checks directly that $\invim{r}$ commutes with the logical connectives by unfolding the definitions. For example,
%
\begin{equation*}
  \R{a} \rz \invim{r}(\phi \limply \psi)(y)
  \iff
  \all{\R{b} \in \phi(r(y))}
  \all{p \in \PP}
  \R{a} \, \R{b} \rz[p] \phi(r(y))
\end{equation*}
%
and
%
\begin{equation*}
  \R{a} \rz (\invim{r}\phi \limply \invim{r}\psi)(y)
  \iff
  \all{\R{b} \in \phi(r(y))}
  \all{p \in \PP}
  \R{a} \, \R{b} \rz[p] \phi(r(y)),
\end{equation*}
%
which are the same condition.

\subsection{The quantifiers}
\label{sec:quantifiers}

Let $r : Y \to X$ be a map. The universal and existential quantifiers along~$r$ are monotone maps
%
\begin{equation*}
  \exists_r, \forall_r : \Pred{Y} \to \Pred{X},
\end{equation*}
%
such that, for all $\phi \in \Pred{Y}$ and $\psi \in \Pred{X}$,
%
\begin{equation*}
  \exists_r \phi \leq_X \psi \iff \phi \leq_Y \invim{r} \psi
  \qquad\text{and}\qquad
  \psi \leq_X \forall_r \phi \iff \invim{r} \psi \leq_Y \psi.
\end{equation*}
%
(The usual quantifiers correspond to $r : X \times Y \to X$ being the first projection.)
%
We may take the following definition of the existential quantifier:
%
\begin{align*}
  (\exists_r \phi)(x) \defeq
   \set{\R{a} \in \AA \such \some{y \in Y} r(y) = x \land \R{a} \in \phi(y)}.
\end{align*}
%
If $\R{a}$ realizes $\exists_r \phi \leq_X \psi$ then it also realizes $\phi \leq_Y \invim{r} \psi$:
%
for any $y \in Y$, $p \in \PP$ and $\R{b} \in \phi(y)$ we have $\R{b} \in (\exists_r \phi)(r(y))$, therefore $\R{a} \, \R{b} \rz[p] \psi(r(y))$.
%
Conversely, if $\R{a}$ realizes $\phi \leq_Y \invim{r} \psi$ then it also realizes $\exists_r \phi \leq_X \psi$: for any $x \in X$, $p \in \PP$ and $\R{b} \in (\exists_r \phi)(x)$, we have $r(y) = x$ for some $y \in Y$ such that $\R{b} \in \phi(y)$, hence $\R{a} \, \R{b} \rz[p] \psi(r(y))$ and $\R{a} \, \R{b} \rz[p] \psi(x)$.

Next, the definition of the universal quantifier is
%
\begin{multline*}
  (\forall_r \phi)(x) \defeq
   \{\R{a} \in \AA \such
     \all{y \in Y} r(y) = x \lthen
     \all{\R{b} \in \AA} \all{q \in \PP}
     \R{a} \, \R{b} \rz[q] \phi(y)
   \}.
\end{multline*}
%
If~$\R{a}$ realizes $\psi \leq_X \forall_r \phi$ then $\R{b} \defeq \ucode{\abstr{x} \R{a} \, x \, \combK}$ realizes $\invim{r}\psi \leq_Y \phi$:
%
for any $y \in Y$, $p \in \PP$, and $\R{c} \in \psi(r(y))$, we have $\R{a} \, \R{c} \rz[p] (\forall_r \phi)(r(y))$, therefore $\R{a} \, \R{c} \, \combK \rz[p] \phi(y)$ and $p \at \R{b} \, \R{c} = p \at \R{a} \, \R{c} \, \combK$, giving the required $\R{b} \, \R{c} \rz[p] \phi(y)$.
%
Conversely, if $\R{b}$ realizes $\invim{r}\psi \leq_Y \phi$ then $\R{a} \defeq \ucode{\abstr{x} \abstr{d} b \, x}$ realizes $\psi \leq_X \forall_r \phi$: consider any $x \in X$, $p \in \PP$, $\R{c} \in \psi(x)$. To show $\R{a} \, \R{c} \rz[p] (\forall_r \phi)(x)$, first note that $\defined{(p \at \R{a} \, \R{c})}$. Suppose $y \in Y$ is such that $r(y) = x$, and consider any $\R{d} \in \AA$ and $q \in \PP$.
%
By \cref{lem:abstr-uniform}
%
\begin{equation*}
  p \at \R{a} \, \R{c} =
  p \at \abstr{\R{d}} \R{b} \, \R{c} =
  q \at \abstr{\R{d}} \R{b} \, \R{c} =
  q \at \R{a} \, \R{c}
\end{equation*}
%
therefore
%
$
  q \at (\R{a} \app[p] \R{c}) \, \R{d} \kleq
  q \at \R{a} \, \R{c} \, \R{d} \kleq
  q \at \R{b} \, \R{c}
$.
%
From $\R{c} \in \psi(r(y))$ it follows that $\R{b} \, \R{c} \rz[q] \phi(x)$, therefore $(\R{a} \app[p] \R{c}) \, \R{d} \rz[q] \phi(x)$.

The reader may have expected the following, simpler definition of the universal quantifier
%
\begin{equation}
  \label{eq:alternative-forall}%
  (\forall'_r \phi)(x) \defeq
   \{\R{a} \in \AA \such
     \all{y \in Y} r(y) = x \lthen
     \R{a} \in \phi(y)
   \},
\end{equation}
%
which works, but only when~$r$ is surjective.
%
It is easy to check that $\forall'_r \phi \leq_X \forall_r \phi$ is realized by~$\combK$.
% Verification that k is such a realizer:
% Consider x ∈ X, a ∈ (∀'_r ϕ)(x) and p ∈ P.
% Then (p | k a) is defined.
% We verify that k a ⊩_p (∀'_r ϕ)(x):
%  Suppose y ∈ Y, r y = x, b ∈ 𝔸, q ∈ ℙ.
%  Then (q | k a b) = a, so it is defined.
%  And (q | k a b) ∈ ϕ(y) because (q | k a b) = a and a ∈ ϕ(y).
The converse $\forall_r \phi \leq \forall'_r \phi$ is realized by $\R{c} \defeq \abstr{x} x \, \combK$. Too see this, consider any $x \in X$, $\R{a} \in (\forall_r \phi)(x)$ and $p \in \PP$.
First, $p \at \R{c} \, \R{a} \simeq p \at \R{a} \, \combK$  is defined because~$r$ is surjective.
Second, if $y \in Y$ and $r(y) = x$ then $\R{a} \, \combK \rz[p] \phi(y)$ and $p \at \R{c} \, \R{a} = p \at \R{a} \, \combK$, therefore $\R{a} \, \combK \in \phi(y)$.

It remains to verify the Beck-Chevalley condition, which states that, given a pullback in~$\Set$
%
\begin{equation*}
  \xymatrix{
    {Y} \pullbackcorner
    \ar[r]^{r} \ar[d]_{u} 
    &
    {X} \ar[d]^{v}
    \\
    {Z} \ar[r]_{q}
    &
    {W}
  }
\end{equation*}
%
$\forall_r \circ \invim{u}$ and $\invim{v} \circ \forall_q$ are equivalent.
%
For $\phi \in \Pred{Z}$, $x \in X$
the condition $\R{a} \in ((\forall_r \circ \invim{u}) \phi)(x)$ unfolds to
%
\begin{equation}
  \label{eq:bc-1}%
  \all{y \in Y} r(y) = x \lthen \R{a} \in \phi(u(y)),
\end{equation}
%
while $\R{a} \in ((\invim{v} \circ \forall_q) \phi)(x)$ unfolds to
%
\begin{equation}
  \label{eq:bc-2}%
  \all{z \in Z} q(z) = v(x) \lthen \R{a} \in \phi(z).
\end{equation}
%
Let us show that these are equivalent conditions. Suppose $\R{a}$ satisfies \eqref{eq:bc-1} and $z \in Z$ is such that $q(z) = v(x)$. Because the square is a pullback there is a unique $y \in Y$ such that $r(y) = x$ and $u(y) = z$. By \eqref{eq:bc-1} we get $\R{a} \in \phi(u(y))$ which is the same as $\R{a} \in \phi(z)$.
%
Conversely, if $\R{a}$ satisfies \eqref{eq:bc-2} and there is a $y \in Y$ such that $r(y) = x$, then we instantiate \eqref{eq:bc-2} with $z = u(y)$ to obtain the desired $\R{a} \in \phi(u(y))$.

\subsection{The generic element}
\label{sec:generic-element}

Because $\Set$ is cartesian closed, the remaining requirement for a tripos is the existence of a generic element, see the remark following~\cite[Definition~2.12]{oosten08:_realiz}. Specifically, we seek a set $S$ and $\sigma \in \Pred{S}$ such that, for all $X$ and $\phi \in \Pred{X}$, there exists $r_\phi : X \to S$ for which $\phi$ and $\invim{r_\phi} \sigma$ are isomorphic.

Once again, we just reuse the generic element for a tripos based on a pca, namely $S \defeq \pow{\AA}$ and $\sigma \defeq \id[\pow{\AA}]$. This obviously works because $\invim{\phi} \id[\pow{\AA}] = \phi$ for any $\phi \in \Pred{X}$.

\subsection{Tripos logic}
\label{sec:tripos-logic}

A formula $\phi$ built from logical connectives, quantifiers, and tripos predicates
whose free variables $x_1, \ldots, x_n$ range over the sets $X_1, \ldots, X_n$,
determines a tripos predicate
%
\begin{equation*}
  [x_1 \of X_1, \ldots, x_n \of X_n \such \phi] : X_1 \times \cdots \times X_n \to \pow{\AA},
\end{equation*}
%
which we sometimes abbreviate as $[x_1, \ldots, x_n \such \phi]$ or just $[\phi]$.
The case $n = 0$ yields and element of $\pow{\AA}$.

More precisely, the logical connectives appearing in~$\phi$ are interpreted as the corresponding Heyting operations from \cref{sec:heyt-prealg-struct}.
A universally quantified formula $\all{y \of Y} \psi$, where $\psi$ is a formula in variables $x_1, \ldots, x_n$ and $y$, is interpreted as quantification along the projection
%
\begin{equation*}
  r : X_1 \times \cdots \times X_n \times Y \to X_1 \times \cdots \times X_n,
\end{equation*}
%
as in \cref{sec:quantifiers}, and similarly for $\some{y \of Y} \psi$.

\begin{example}
  \label{example:tripos-forall-exists}
  Given a tripos predicate $\psi \in \Pred{X \times Y}$ with an inhabited set~$X$, the formula
  %
  \begin{equation*}
    \all{x \of X} \some{y \of Y} \psi(x,y)
  \end{equation*}
  %
  has no free variables, and so determines an element of $\pow{\AA}$,
  which we compute using \eqref{eq:alternative-forall} to be
  %
  \begin{align*}
    \R{a} \in [\all{x \of X} \some{y \of Y} \psi(x,y)]
    &\iff
      \all{u \in X}
      \R{a} \in [\some{y} \psi(u, y)]
    \\
    &\iff
      \all{u \in X}
      \some{v \in Y}
      \R{a} \in \psi(u, v)
  \end{align*}
  %
  Note that $\R{a}$ may not depend on~$u$ and~$v$.
  This is a rather strong uniformity condition, stemming from the fact that realizers receive no information about the elements of underlying sets. When we pass from the tripos to the topos, the situation will be rectified by equipping sets with suitable realizability relations, see \cref{example:topos-forall-exists}.
\end{example}

We say that a formula $\phi$ in variables $x_1 \of X_1, \ldots, x_n \of X_n$ is \defemph{valid}, written as
%
\begin{equation*}
  x_1 \of X_1, \ldots, x_n \of X_n \models \phi,
\end{equation*}
%
when its interpretation is (equivalent to) the top predicate in $\Pred{X_1 \times \cdots \times X_n}$. This happens precisely when there is $\R{a} \in \AA$ such that $\R{a} \rz[p] [\phi](u_1, \ldots, u_n)$ for all $u_1 \in X_1, \ldots, u_n \in X_n$ and $p \in \PP$.


\subsection{The parameterized realizability topos on a ppca}
\label{sec:unif-real-topos}

Having defined a tripos, we employ the tripos-to-topos construction~\cite[\S2.2]{oosten08:_realiz} to construct a topos from it.

\begin{definition}
  The \defemph{parameterized realizability topos} $\PRT{\AA, \PP}$ on the ppca $(\AA, \PP, {\cdot})$ is the topos arising from the tripos-to-topos construction applied to the tripos~$\PredSymbol[\AA, \PP]$.
\end{definition}

We recall how the construction works.
%
An object $X = (|X|, {\eq[X]})$ of the topos is a set~$|X|$ with a tripos predicate ${\eq[X]} \in \Pred{|X| \times |X|}$, called the \defemph{equality predicate}, which is a partial equivalence relation in the sense of tripos logic:
%
\begin{align*}
  x \of |X|, y \of |X| &\models x \eq[X] y \limply y \eq[X] x,
  \\
  x \of |X|, y \of |X|, z \of |X| &\models x \eq[X] y \limply y \eq[X] z \limply x \eq[X] z.
\end{align*}
%
Specifically, this means that there are $\R{a}, \R{b} \in \AA$ such that:
%
\begin{enumerate}
\item for all $x, y \in |X|$, $\R{c} \in (x \eq[X] y)$, and $p \in \PP$, we have $\R{a} \, \R{c} \rz[p] y \eq[X] x$,
\item for all $x, y, z \in |X|$, $\R{c} \in (x \eq[X] y)$, $\R{d} \in (y \eq[X] z)$, and $p \in \PP$, we have $\R{b} \, \R{c} \, \R{d} \rz[p] x \eq[X] z$.
\end{enumerate}
%
Henceforth we shall refrain from such explicit unfolding of formulas into statements about realizers, and instead rely on the fact that a formula is valid in the tripos logic if it has an intuitionistic proof.

The equality predicate $\eq[X]$ endows $|X|$ with a notion of equality that is witnessed by realizers.
However, because we did not require reflexivity of~$\eq[X]$, there may be elements which are not equal to themselves.
To manage the anomaly, we define the \defemph{existence predicate} $\Ex{X} \in \Pred{|X|}$ by
%
\begin{equation*}
  \Ex{X}(x) \defeq (x \eq[X] x).
\end{equation*}
%
A realizer $\R{a} \in \Ex{X}(x)$ can be thought of as witnessing the fact that $x \in X$. When $\Ex{X}(x) = \emptyset$, the element $x$ ``does not exist'' from the point of view of the topos.
%
We shall strategically use $\Ex{X}(x)$ to disregard such non-existent elements.\footnote{%
It turns out that~$X$ is isomorphic to $(X', {\eq[X]})$ where $X' \defeq \set{x \in |X| \such (x \eq[X] x) \neq \emptyset}$, but insisting that $(x \eq[X] x) \neq \emptyset$ does not lead to any improvements.%
}

A morphism $F : X \to Y$ is represented by a predicate $F \in \Pred{|X| \times |Y|}$ which is a functional, i.e., one satisfying the following conditions, with $x, x' \of |X|$ and $y, y' \of |Y|$:
%
\begin{align*}
  x, y &\models F(x,y) \limply \Ex{X}(x) \land \Ex{Y}(y)
     & &\text{(strict)} \\
  x, x', y, y' &\models F(x,y) \land x \eq[X] x' \land y \eq[Y] y' \limply F(x', y')
     & &\text{(relational)} \\
  x, y, y' &\models F(x, y) \land F(x, y') \limply y \eq[Y] y'
     & &\text{(single-valued)} \\
  x &\models \Ex{X}(x) \limply \some{y \of Y} F(x, y)
     & &\text{(total)}
\end{align*}
%
Single-valuedness and totality are familiar conditions, while the other two ensure that~$F$
behaves with respect to existence and equality predicates. Note how the antecedent $\Ex{X}(x)$ in the totality condition allows~$F$ to ignore non-existing elements of~$X$.
%
Two such relations represent the same morphism if they are equivalent as tripos predicates.

To actually compute~$F$, we use a realizer~$s$ for its strictness and a realizer~$t$ for its totality to define the realizer $f \defeq \ucode{\abstr{x} \combSnd (s \, (t \, x))}$, which works as follows: for any $x \in X$ and $\R{a} \in \Ex{X}(x)$ there is $y \in Y$ such that $r \, \R{a} \rz[p] \Ex{Y}(y)$ for all~$p \in \PP$,
and because $F$ is single-valued, $y$ is unique up to $\eq[Y]$.

The identity morphism on~$X$ is represented by $\eq[X]$,
and the composition of $F : X \to Y$ and $G : Y \to Z$ by the tripos predicate
%
\begin{equation*}
  (G \circ F)(x, z) \defeq \some{y \of Y} F(x, y) \land G(y, z).
\end{equation*}
%
The relevant conditions may be checked by reasoning in intuitionistic logic.

The terminal object in the topos is $\one \defeq (\set{\star}, {\eq[\one]})$, where $(\star \eq[\one] \star) \defeq \AA$.
%
The subobject classifier is the object $\Omega \defeq (\pow{\AA}, \eq[\Omega])$ whose equality predicate is logical equivalence,
%
$
  (\phi \eq[\Omega] \psi) \defeq
  (\phi \to \psi) \land (\psi \to \phi)
$.
%
Truth $T : \one \to \Omega$ is represented by the tripos predicate $T(\star, \phi) \defeq \phi$.


\subsection{Topos logic}
\label{sec:internal-logic-topos}

The topos logic differs from the tripos logic because it accounts for the equality and existence predicates. We refer to~\cite[\S2.3]{oosten08:_realiz} for details, and give here an overview that will suffice for our purposes.

In the topos logic, the predicates on an object $X$ are its subobjects, which turn out to be in
bijective correspondence with equivalence classes of \defemph{strict predicates}~\cite[Thm.~2.2.1]{oosten08:_realiz}, i.e., those $\phi \in \Pred{|X|}$ that satisfy
%
\begin{align*}
  x \of X &\models \phi(x) \limply \Ex{X}(x) & &\text{(strict)} \\
  x \of X, y \of X &\models \phi(x) \land x \eq[X] y \limply \phi(y) & &\text{(relational)}
\end{align*}
%
The tripos falsehood is strict, and the tripos conjunction, disjunction, and the existential quantifier preserve strictness, hence these are computed in the topos in the same way as in the tripos. Truth, implication, and the universal quantifier require modification. We distinguish notationally between the tripos and topos logic by writing
``$\limply$'', ``$\forall y \of Y$'', and ``$\exists y \of Y$'' in the former, and
``$\lthen$'',  ``$\forall y \in Y$'', and ``$\exists y \in Y$'' in the latter.

First, the topos truth $\top$ qua predicate on~$X$ is the tripos predicate $\Ex{X}$. Indeed, this is a strict predicate, and for any strict predicate $\phi \in \Pred{X}$ the implication $\phi(x) \to \Ex{X}(x)$ is valid by strictness of~$\phi$. Because the top predicate has changed, we must also adjust validity: a strict predicate $\phi \in \Pred{X}$ is topos-valid when the tripos validates
%
\begin{equation*}
  x \of X \models \Ex{X}(x) \to \phi(x).
\end{equation*}
%
Explicitly, there exists $\R{a} \in \AA$ such that for all $x \in |X|$, $\R{b} \in \Ex{X}(x)$ and $p \in \PP$ we have $\R{a} \, \R{b} \rz[p] \phi(x)$.

Second, the topos implication $\phi \lthen \psi$ of strict predicates  $\phi$ and $\psi$ on~$X$ is represented by the strict predicate
%
\begin{equation*}
  [x \of X \mid \Ex{X}(x) \land (\phi(x) \limply \psi(x))].
\end{equation*}
%
Explicitly, $\R{a} \in (\phi \lthen \psi)(x)$ when for all $p \in \PP$
%
\begin{equation*}
  (\combFst \, \R{a} \rz[p] \Ex{X}(x))
  \land
  (\combSnd \, \R{a} \rz[p] \phi(x) \to \psi(x)).
\end{equation*}

Third, if $\phi$ is a strict predicate on $X \times Y$, the topos universal $\all{y \in Y} \phi(x, y)$ is represented by the strict predicate
%
\begin{equation*}
  [x \of X \mid \Ex{X}(x) \land \all{y \of |Y|} (\Ex{Y}(y) \limply \phi(x,y))].
\end{equation*}
%
Assuming $|Y|$ is inhabited, $\R{a} \in (\all{y \in Y} \phi(x, y))$ when for all $p \in \PP$
%
\begin{equation*}
  (\combFst \, \R{a} \rz[p] \Ex{X}(x))
  \land
  \all{y \in |Y|} \all{\R{b} \in \Ex{Y}(y)} \combSnd \, \R{a} \, \R{b} \rz[p] \phi(x, y).
\end{equation*}
%
The first conjunct just makes sure that non-existent~$x$ do not get in the way. The second one is more interesting, as it adjusts the unreasonable uniformity of tripos~$\forall$ by providing $\combSnd \, \R{a}$ with a realizer of~$y \in |Y|$.

One might expect the topos existential $\some{y \in Y} \phi(x, y)$ to be
%
\begin{equation*}
  [x \of X \such \some{y \of Y} \Ex{Y}(y) \land \phi(x, y)],
\end{equation*}
%
but we can reuse $\some{y \of Y} \phi(x, y)$, for if $\R{a} \in (\some{y \of Y} \phi(x, y))$
and~$s$ realizes strictness of~$\phi$ then $s \, \R{a} \in \Ex{Y}(y)$ for some $y \in |Y|$.

In contrast to the tripos logic, the topos logic is equipped with equality.
%
Unsurprisingly, equality on~$X$ is represented by $\eq[X]$, one just needs to check that this is indeed a strict predicate.
%
More generally, equality of morphisms $F, G : X \to Y$ is represented by the predicate
%
\begin{equation*}
  [x \of X \such \some{y \of Y} F(x, y) \land G(x, y)].
\end{equation*}

\begin{example}
  \label{example:topos-forall-exists}%
  Suppose $\carrier{X}$ is inhabited, and $\phi \in \Pred{\carrier{X} \times \carrier{Y}}$.
  A short calculations shows that $\all{x \in X} \some{y \in Y} \phi(x, y)$ is realized
  when there is $\R{a} \in \AA$ such that
  %
  \begin{equation*}
    \all{x \in |X|}
    \all{\R{b} \in \Ex{X}(x)}
    \all{p \in \PP}
    \some{y \in |Y|}
    \R{a} \, \R{b} \rz[p] \phi(x, y).
  \end{equation*}
  %
  Note that the unreasonable uniformity of \cref{example:tripos-forall-exists} has been rectified,
  as~$\R{b}$ is passed to~$\R{a}$.
\end{example}


\subsection{Parameterized assemblies}
\label{sec:unif-assemblies}

Direct manipulation of topos objects, and especially morphisms, can be cumbersome. Fortunately, the
subcategory of assemblies~\cite[Sect.~2.4]{oosten08:_realiz} is significantly easier to work with and already contains most objects of interest.

The idea is to take existence predicates as primary.
%
Define a \defemph{(parameterized) assembly} $X = (|X|, \Ex{X})$ to be a set $|X|$ with a tripos predicate $\Ex{X} \in \Pred{|X|}$, called the \defemph{existence predicate}, such that $\Ex{X}(x) \neq \emptyset$ for all $x \in |X|$.
%
Also define a \defemph{(parameterized) assembly map} $f : X \to Y$ to be a map $f : |X| \to |Y|$ which is realized by some $\R{a} \in \AA$, meaning: for all $x \in |X|$, $\R{b} \in \Ex{X}(x)$ and $p \in \PP$, we have $\R{a} \, \R{b} \rz[p] \Ex{Y}(f(x))$.
%
Assembly maps are closed under composition and include the identity maps, so we have a category $\PAsm{\AA, \PP}$. 

Given an assembly~$X$, let $\eq[X]$ be the tripos predicate on $|X| \times |X|$, defined by
%
\begin{equation*}
  (x \eq[X] x') \defeq \set{\R{a} \in \Ex{X}(x) \such x = x'}.
\end{equation*}
%
Thus $x \eq[X] x'$ is empty when $x \neq x'$ and equals $\Ex{X}(x)$ when $x = x'$.
%
It is evident that $x \eq[X] x'$ is an equality predicate on~$|X|$, hence the assembly~$X$ may be construed as the topos object $(|X|, \eq[X])$.
%
Not every topos object arises this way, for instance the subobject classifier~$\Omega$.

To get a functorial embedding of assemblies into the topos, we map an assembly map $f : X \to Y$ to the topos morphism $F : X \to Y$ where
%
\begin{equation*}
  F(x, y) \defeq \set{\R{b} \in \AA \such
    f(x) = y
    \land
    \all{p \in \PP}
    \combFst \, \R{b} \rz[p] \Ex{X}(x)
    \land
    \combSnd \, \R{b} \rz[p] \Ex{Y}(y)}.
\end{equation*}
%
This is a functional relation, for if~$\R{a}$ realizes~$f$ then $\ucode{\abstr{x} \combPair \, x \, (\R{a} \, x)}$ realizes totality of~$F$.
%
The passage from assemblies to topos objects constitutes a full and faithful embedding $\PAsm{\AA, \PP} \to \PRT{\AA, \PP}$. Only fullness deserves attention. Suppose $X$ and $Y$ are assemblies and $F : X \to Y$ a morphism between the induced topos objects. Because $Y$ is an assembly and $F$ is single-valued, each $x \in |X|$ has at most one $y \in |Y|$ such that $F(x, y) \neq \emptyset$. Therefore, we may define a map $f : |X| \to |Y|$ by
%
\begin{equation*}
  f(x) = y \defiff F(x,y) \neq \emptyset.
\end{equation*}
%
If $\R{a} \in \AA$ realizes totality of $F$ then $\ucode{\abstr{x} \combSnd \, (\R{a} \, x)}$ realizes~$f$ as an assembly map.

\subsection{Some distinguished assemblies}
\label{sec:distinguished-assemblies}

We review certain objects of the topos that will play a role in the construction of the object of the Dedekind reals.

\subsubsection{Natural numbers, integers, and rational numbers}
\label{sec:natur-numb-integ}

The natural numbers object is the assembly $\objN \defeq (\NN, \Ex{\objN})$ where $\Ex{\objN}(n) \defeq \set{\numeral{n}}$, so that each number is realized by the corresponding Curry numeral.
%
The induction principle is realized by the primitive recursor $\comb{primrec}$ from \cref{sec:progr-with-ppcas}.

The objects of integers and rational numbers are the assemblies
%
\begin{equation*}
  \objZ \defeq (\ZZ, \Ex{\objZ})
  \quad\text{\and}\quad
  \objQ \defeq (\QQ, \Ex{\objQ}),
\end{equation*}
%
whose existence predicates are induced by computable enumerations.
For the integers we can use
%
\begin{equation*}
  \Ex{\objZ}(k) \defeq
  \begin{cases}
    \set{\numeral{2 k}}     & \text{if $k \geq 0$,} \\
    \set{\numeral{1 - 2 k}} & \text{if $k < 0$.}
  \end{cases}
\end{equation*}
%
For the rationals we may reuse the bijection $\rat{} : \NN \to \QQ$ from \cref{sec:oracle-comp-maps},
%
\begin{equation*}
  \Ex{\objQ}(\rat{n}) \defeq \set{\numeral{n}}.
\end{equation*}
%
Any other reasonable codings would result in isomorphic objects.
%
Arithmetical operations are realized and the order relation is decidable, i.e., the statement
$\all{x, y} x < y \lor y \leq x$ is realized, both for $x, y \in \ZZ$ and for $x, y \in \QQ$.

\subsubsection{Products and exponentials}
\label{sec:prod-expon}

The category of parameterized assemblies is cartesian closed. The product of $X$ and $Y$ is the assembly
%
\begin{equation*}
  X \times Y \defeq (|X| \times |Y|, \Ex{X \times Y})
\end{equation*}
%
where
%
\begin{equation*}
  \Ex{X \times Y}(x, y) \defeq
  \set{\R{a} \in \AA \such
  \all{p \in \PP} \combFst \, \R{a} \rz[p] \Ex{X}(x) \land \combSnd \, \R{a} \rz[p] \Ex{Y}(y)}.
\end{equation*}
%
To construct the exponential~$Y^X$, we define its existence predicate, for any $f : |X| \to |Y|$, by
%
\begin{equation*}
  \Ex{Y^X}(f) \defeq
  \set{\R{a} \in \AA \such
    \all{x \in X} \all{\R{b} \in \Ex{X}(x)} \all{p \in \PP}
      \R{a} \, \R{b} \rz[p] \Ex{Y}(f(x))
  },
\end{equation*}
%
and set $|Y^X| \defeq \set{f : |X| \to |Y| \such \Ex{Y^X}(f) \neq \emptyset}$.

\begin{proposition}
  \label{prop:markov-principle}%
  Markov's principle
  %
  \begin{equation*}
    \all{f \in \two^\objN} \neg \neg (\some{n \in \objN} f n = 1) \lthen \some{n \in \objN} f n = 1
  \end{equation*}
  %
  is valid.
\end{proposition}

\begin{proof}
  The principle is realized by a program that searches for the least $n$ such that $f n \neq 0$:
  %
  \begin{equation*}
    \abstr{f} \abstr{r}
    \comb{Z} \, (\abstr{s} \abstr{n}
         \combIf \,
         (\comb{iszero} \,
         (f \, n) \,
         (s \, (\comb{succ} \, n)) \,
         n))
     \, \numeral{0}.
  \end{equation*}
  %
  The assumption $\neg\neg{\some{n \in \objN} f(n) = 1}$ ensures that the search will succeed.\footnote{As is typical of realizability models, we are relying on mete-level Markov's principle to realize Markov's principle: since it is impossible that the search will run forever, it will find what it is looking for.}
\end{proof}

\begin{example}
  Let us contrast $\forall\exists$ statements and exponentials. Consider a non-empty assembly~$X$, an assembly~$Y$, and
  a strict predicate $\phi \in \Pred{|X| \times |Y|}$, with $s \in \AA$ witnessing its strictness.
  %
  Validity of $\all{x \in X} \some{y \in Y} \phi(x,y)$ is equivalent to there being $\R{a} \in \AA$ such that
  %
  \begin{equation*}
    \all{x \in |X|} \all{\R{b} \in \Ex{X}(x)} \all{p \in \PP} \some{y \in |Y|} \R{a} \, \R{b} \rz[p] \phi(x,y).
  \end{equation*}
  %
  The realizer $\R{c} \defeq \ucode{\abstr{\R{b}} \combSnd \, (s \, (\R{a} \, \R{b}))}$ satisfies, for any $x \in |X|$, $\R{b} \in \Ex{X}(x)$ and $p \in \PP$, that there is $y \in |Y|$ such that $\R{c} \, \R{b} \rz[p] \Ex{Y}(y)$. However, $\R{c}$ need not realize a choice map $f : X \to Y$ because~$y$ may depend on~$\R{b}$ and~$p$.
  %
  Thus in general the axiom of choice
  %
  \begin{equation*}
    (\all{x \in X} \some{y \in Y} \phi(x, y))
    \lthen
    \some{f \in Y^X} \all{x \in X} \phi(x, f(x)))
  \end{equation*}
  %
  is not realized, even in case that $\Ex{X}(x)$ is a singleton for all~$x \in \carrier{X}$, because there is still dependence on the parameter. In particular, countable choice may fail, as it does in the topos~$\TT{\mil}$ from \cref{sec:topos-with-countable}.
\end{example}

\subsubsection{Sub-assemblies}
\label{sec:sub-assemblies}

Suppose $\phi \in \Pred{|X|}$ is a strict predicate on an assembly~$X$. (Notice that $\phi$ is automatically relational because $(x \eq[X] y) \neq \emptyset$ implies $x = y$.) We define the sub-assembly $\set{x \of X \such \phi(x)}$ to have the underlying set
%
\begin{equation*}
  |\set{x \of X \such \phi(x)}| \defeq \set{x \in |X| \such \phi(x) \neq \emptyset}
\end{equation*}
%
and the existence predicate $\Ex{\set{x \of X \such \phi(x)}}(x) \defeq \phi(x)$.
%
Then the canonical map $\set{x \of X \such \phi(x)} \to X$ is realized by any realizer for strictness of~$\phi$.
It is the monomorphism characterized by the predicate~$\phi$.


\subsubsection{Constant assemblies}
\label{sec:constant-assemblies}

Define the \defemph{constant (parameterized) assembly} on a set $S$ to be $\nabla S \defeq (S, \Ex{\nabla S})$ with $\Ex{\nabla S}(x) \defeq \AA$.
%
The existence predicate is maximally uninformative, because all elements of~$S$ share all realizers.
%
Consequently, given any assembly $X$, every map $f : |X| \to S$ is realized, say by~$\combK$.
In particular, every map $f : S \to T$ between sets is an assembly map $\nabla f \defeq f : \nabla S \to \nabla T$,
which makes $\nabla$ a functor from sets to assemblies, see~\cite[Sect.~2.4]{oosten08:_realiz} for details.

\subsubsection{The assembly $\nabla\two$ and $\neg\neg$-stable predicates}
\label{sec:assembly-nabl-negn}

Of particular interest is the assembly $\nabla \two$ because it classifies $\neg\neg$-stable predicates on any assembly~$X$ (and more generally on any topos object).
%
On the one hand, ${\nabla\two}^X$ is isomorphic to $\nabla(\two^{|X|})$ because every map $|X| \to \two$ is realized.
On the other, $\two^{|X|}$ qua Heyting algebra is equivalent to the Heyting prealgebra of $\neg\neg$-stable strict predicates on~$X$. Too see this, observe that a strict predicate~$\phi$ on~$X$ is $\neg\neg$-stable when
%
\begin{equation*}
  x \of X \models \Ex{X}(x) \to ((\phi(x) \lthen \bot) \lthen \bot) \lthen \phi(x),
\end{equation*}
%
%% Computation of ¬¬-stability of a strict relational ϕ ∈ Pred(X)
%
% Note: if ϕ → Ex then Ex → ϕx → P is equivalent to ϕx → P
%
% Double negation in the topos:
% ¬¬ϕx iff
% ((ϕx ⇒ ⊥ ) ⇒ ⊥) iff
% Ex → (Ex → ϕx → ⊥) → ⊥ iff
% Ex → (ϕx → ⊥) → ⊥
%
% ¬¬-stability of ϕ:
% ¬¬ϕx ⇒ ϕx iff
% Ex → ¬¬ϕx → ϕx iff
% Ex → ((Ex → (ϕx → ⊥)) → ⊥) → ϕx iff
% Ex → ((ϕx → ⊥) → ⊥) → ϕx
%
which amounts to there being $\R{a} \in \AA$ such that, for all $x \in |X|$, $\R{b} \in \Ex{X}(x)$ and $p \in \PP$,
if $\phi(x) \neq \emptyset$ then $\R{a} \, \R{b} \rz[p] \phi(x)$.
%
Therefore, $\phi$ is equivalent to the strict predicate
%
$x \mapsto \set{a \in \AA \such \phi(x) \neq \emptyset}$,
%
which in turn corresponds to a unique map $|X| \to \two$, obtained when~$\emptyset$ and~$\AA$ are mapped to $0$ and~$1$, respectively.






%%% Local Variables:
%%% mode: latex
%%% TeX-master: "countable-reals"
%%% End:
 
\section{The real numbers}
\label{sec:real-numbers-object}

We review the construction of the Dedekind real numbers, and formulate it in a way that makes it easy to calculate the object of Dedekind reals in a parameterized realizability topos.
%
We also show that the Cauchy reals are sequence-avoiding in any parameterized realizability topos.

\subsection{The Dedekind real numbers}
\label{sec:dedek-real-numb}

In this section we work in higher-order intuitionistic logic, which can be interpreted in any topos.
%
Common mathematical constructions are available, as well as the standard number sets: the natural numbers $\objN$, the integers~$\objZ$ and the rationals~$\objQ$.

\begin{definition}
  \label{def:dedekind-reals}%
  A \defemph{Dedekind cut} is a pair $(L, U) \in \pow{\objQ} \times \pow{\objQ}$ of subsets of rationals, satisfying the following conditions, where $q$ and~$r$ range over~$\objQ$:
  %
  \begin{enumerate}
  \item $L$ and $U$ are inhabited: $\some{q} q \in L$ and $\some{r} r \in U$,
  \item $L$ is lower-rounded and $U$ upper-rounded:
    % 
    \begin{equation*}
      q \in L \liff \some{r} q < r \land r \in L 
      \qquad\text{and}\qquad
      r \in U \liff \some{q} q \in U \land q < r,
    \end{equation*}
  \item $L$ is below $U$: $q \in L \land r \in U \lthen q < r$,
  \item $L$ and $U$ are located: $q < r \lthen q \in L \lor r \in U$.
  \end{enumerate}
  %
  We write $\Cut{L,U}$ for the conjunction of the above conditions.
  %
  The set of \defemph{Dedekind reals} is
  %
  \begin{equation*}
    \RRd \defeq \set{ (L, U) \in \pow{\objQ} \times \pow{\objQ} \such \Cut{L, U} }.
  \end{equation*}
  % 
\end{definition}

The symbols $\RRd$ and $\RR$ both denote the set of Dedekind reals. We normally use $\RRd$ when referring to the object of Dedekind reals in a topos, or when we want to contrast the Dedekind reals with other kinds of reals.

Lower-roundedness may be split into two separate conditions:
%
\begin{itemize}
\item $L$ is lower: $q < r \land r \in L \lthen q \in L$,
\item $L$ is rounded: $q \in L \lthen \some{r} q < r \in r \in L$.
\end{itemize}
%
Upper-roundedness may be decomposed analogously.

Many textbooks construct the reals by using one-sided cuts. Again, this works classically but requires special care when done constructively, and in any case the symmetry of double-sided cuts streamlines the development, even classically. 

\begin{propositionC}
  \label{prop:RRd-stable-equality}%
  For all $x, y \in \RRd$, if $\lnot\lnot (x = y)$ then $x = y$.
\end{propositionC}

\begin{proof}
  Suppose $x = (L_x, U_x)$, $y = (L_y, U_y)$ and $\lnot\lnot(x = y)$.
  %
  We only prove $L_x \subseteq L_y$, as the other three inclusions are proved symmetrically.
  Suppose $q \in L_x$. Because $L_x$ is rounded, there is $r \in \QQ$ such that $q < r \in L_x$.
  From $\lnot\lnot (L_x = L_y)$ follows that $\lnot\lnot (r \in L_y)$.
  %
  Because~$y$ is located, $q \in L_y$ or $r \in U_y$. In the first case we are done, while the second case cannot happen, for if $r \in U_y$ then $\lnot (r \in L_y)$, contradicting $\lnot\lnot (r \in L_y)$.
\end{proof}

\begin{propositionC}
  \label{prop:stradle-closelyd}
  For any cut $(L, U)$ and $k \in \NN$ there exists $q \in \QQ$ such that $q - 2^{-k} \in L$ and $q + 2^{-k} \in U$.
\end{propositionC}

\begin{proof}
  There are $s \in L$ and $t \in U$. Let us show by induction on $j \in \NN$ that there are $u \in L$ and $v \in U$ such that $v - u \leq (2/3)^j (t - s)$. At $j = 0$ we take $u = s$ and $v = t$.
  %
  The induction step from $j$ to $j+1$ proceeds as follows. By the induction hypothesis there are $u' \in L$ and $v' \in U$ such that $v' - u' \leq (2/3)^j (t - s)$. By locatedness $(2 u' + v')/3 \in L$ or $(u' + 2 v')/3 \in U$. In the first case we take $u = (2 u' + v')/3$ and $v = v'$, and in the second $u = u'$ and $v = (u' + 2 v')/3$.

  Now let $k \in \NN$ be given. There is $j \in \NN$ such that $(2/3)^j (t - s) < 2^{-k}$. We proved that there exist $u \in L$ and $v \in U$ such that $v - u \leq (2/3)^j (t - s) < 2^{-k}$. We may take $q = (u + v)/2$ because
  $q - 2^{-k} < v - 2^{-k} < u$ hence $q - 2^{-k} \in L$, and similarly for $q + 2^{-k} \in U$.
\end{proof}

To facilitate calculations in parameterized realizability, we find an object that is isomorphic to~$\RRd$ but whose interpretation in assemblies is straightforward.
%
For any set $A$ and subset $S \subseteq A$, let $\compl{S} \defeq \set{x \in A \such \neg (x \in S)}$ be the complement of~$S$, and
%
\begin{equation*}
  \powcl{A} \defeq \set{S \in \pow{A} \such \all{x \in A} \neg\neg(x \in S) \lthen x \in S}
\end{equation*}
%
the set of $\neg\neg$-stable subsets of~$A$. Just like $\pow{A}$ is isomorphic to the set $\Omega^A$ of characteristic maps on~$A$, so $\powcl{A}$ is isomorphic to $\ClProp^A$, where
%
\begin{equation*}
  \ClProp \defeq \set{p \in \Omega \such \neg\neg p \lthen p}
\end{equation*}
%
is the set of $\neg\neg$-stable truth values. It is a complete Boolean algebra that can be used instead of~$\Omega$ in the definition of Dedekind cuts, like this.

\begin{definition}
  A \defemph{classical Dedekind cut} is a pair $(L, U) \in \powcl{\objQ} \times \powcl{\objQ}$
  of $\neg\neg$-stable subsets of rationals, satisfying the following conditions, where $r$ and $q$ range over~$\objQ$:
  %
  \begin{enumerate}
  \item $L$ and $U$ are not empty: $\lnot \all{q} q \not\in L$ and $\lnot \all{r} r \not\in U$,
  \item $L$ is lower and $U$ upper:
    % 
    \begin{equation*}
      q < r \land r \in L \lthen q \in L
      \quad\text{and}\quad
      q \in U \land q < r \lthen r \in U,
    \end{equation*}
  \item $L$ has no maximum and $U$ no minimum:
    % 
    \begin{equation*}
      (\all{r} r \in L \lthen r \leq q) \lthen q \not\in L
      \quad\text{and}\quad
      (\all{q} q \in U \lthen r \leq q) \lthen r \not\in U,
    \end{equation*}
  \item $L$ is below $U$: $q \in L \land r \in U \lthen q < r$,
  \item $L$ and $U$ are tight: $q \not\in L \land r \not\in U \lthen r \leq q$.
  \end{enumerate}
  %
  We write $\ClCut{L, U}$ for the conjunction of the above conditions.
  %
  The set of \defemph{classical Dedekind reals} is
  %
  \begin{equation*}
    \RRcl \defeq \set{ (L, U) \in \powcl{\objQ} \times \powcl{\objQ} \such \ClCut{L, U}}.
  \end{equation*}
\end{definition}

Dedekind cuts turn out to be those classical Dedekind cuts that have arbitrarily good rational approximations.

\begin{theoremC}
  \label{thm:reals-sub-classical}
  The set of Dedekind reals $\RRd$ is isomorphic to
  %
  \begin{equation}
    \label{eq:reals-sub-classical}%
    R_d \defeq \set{ (L, U) \in \RRcl \such 
       \all{k \in \NN} \some{q \in \QQ} q - 2^{-k} \in L \land q + 2^{-k} \in U
     }.
   \end{equation}
\end{theoremC}

\begin{proof}
  Let $f : \RRd \to R_d$ take a cut to its double complement, $f(L, U) = (\dcompl{L}, \dcompl{U})$.
  %
  To see that it is well-defined, we must verify that $(\dcompl{L}, \dcompl{U})$ is a classical cut and that
  %
  \begin{equation}
    \label{eq:mvv-reals}
    \all{k \in \NN} \some{q \in \QQ} q - 2^{-k} \in \dcompl{L} \land q + 2^{-k} \in \dcompl{U}.
  \end{equation}
  %
  We dispense with~\eqref{eq:mvv-reals} first: for any $k \in \NN$, by \cref{prop:stradle-closelyd} there is $q \in \QQ$ such that $q - 2^{-k} \in L \subseteq \dcompl{L}$ and $q + 2^{-k} \in U \subseteq \dcompl{U}$.

  Let us verify that $(\dcompl{L}, \dcompl{U})$ is a classical cut, where we spell out only the conditions for $L$, as the reasoning for $U$ is symmetric:
  %
  \begin{enumerate}
  \item $\dcompl{L}$ is non-empty because $L$ is inhabited and $L \subseteq \dcompl{L}$.
  \item $\dcompl{L}$ is lower: if $q < r$ and $r \in \dcompl{L}$ then $q \in \compl{L}$ would imply $r \in \compl{L}$, hence $q \in \dcompl{L}$.
  \item $\dcompl{L}$ has no maximum: if  $\all{r \in \dcompl{L}} r \leq q$ then $\all{r \in L} r \leq q$, hence $q \in \compl{L} = \compl{(\dcompl{L})}$.
  \item $\dcompl{L}$ is below $\dcompl{U}$: if $q \in \dcompl{L}$ and $r \in \dcompl{U}$ then $\lnot\lnot(q < r)$, therefore $q < r$.
  \item $\dcompl{L}$ and $\dcompl{U}$ are tight: suppose $q \not\in \dcompl{L}$ and $r \not\in \dcompl{U}$. If $q < r$ then either $q \in L$, which contradicts $q \not\in \dcompl{L}$, or $r \in U$, which contradicts $q \not\in \dcompl{U}$. Therefore $r \leq q$.
  \end{enumerate}
  %

  It remains to be shown that $f$ is a bijection. For injectivity, suppose $x = (L_x, U_x)$ and $y = (L_y, U_y)$ are cuts such that $\dcompl{L_x} = \dcompl{L_x}$ and $\dcompl{U_y} = \dcompl{U_y}$. Then $\lnot\lnot (x = y)$ and by 
  \cref{prop:RRd-stable-equality}, $x = y$.

  To establish surjectivity of~$f$, take any $(L, U) \in R_d$ and define
  %
  \begin{align*}
    \hat{L} &\defeq \set{q \in \QQ \such \some{r \in \QQ} q < r \land r \in L},
    \\
    \hat{U} &\defeq \set{r \in \QQ \such \some{q \in \QQ} q \in U \land q < r}.
  \end{align*}
  %
  Let us show that $(\hat{L}, \hat{U})$ is a cut, where again we verify only the conditions for~$\hat{L}$ when symmetry permits us to do so:
  %
  \begin{enumerate}
  \item $\hat{L}$ is inhabited, because by~\eqref{eq:mvv-reals} there is $q \in \QQ$ such that $q - 2^{0} \in L$, therefore $q - 2 \in \hat{L}$.
  \item $\hat{L}$ is obviously lower, and it is rounded too: if $q \in \hat{L}$ then $q < r$ and $r \in L$ for some $r$,
    hence $(q + r)/2 \in \hat{L}$ because $(q + r)/2 < r$.
  \item $\hat{L}$ is below $\hat{U}$: if $q \in \hat{L}$ and $r \in \hat{U}$ then there are $s$ and $t$ such that $q < s$, $s \in L$, $t \in U$, and $t < r$, therefore $q < s < t < r$.
  \item To see that $(\hat{L},  \hat{U})$ is located, consider any $q < r$. There is $k \in \NN$ such that $2^{-k+1} < r - q$, and there is $s \in \QQ$ such that $s - 2^{-k} \in L$ and $s + 2^{-k} \in U$. Either $q < s - 2^{-k}$, in which case $q \in \hat{L}$, or $s + 2^{-k} < r$, in which case $r \in \hat{U}$.
  \end{enumerate}
  %
  Finally, we claim that $\dcompl{\hat{L}} = L$ and $\dcompl{\hat{U}} = U$.
  %
  Notice that $\hat{L} \subseteq L$ because $L$ is lower, hence $\dcompl{\hat{L}} \subseteq \dcompl{L} = L$.
  %
  To show $L \subseteq \dcompl{\hat{L}}$, suppose $q \in L$. If we had $q \in \compl{\hat{L}}$, then it would follow that $\all{r \in L} r \leq q$, and because $L$ has no maximum also $q \not\in L$, a contradiction, hence we conclude $q \in \dcompl{\hat{L}}$.
\end{proof}

\subsection{The Dedekind reals in parameterized realizability}
\label{sec:dedek-reals-param}

We seek an explicit description of the object of Dedekind reals in a parameterized realizability topos. Rather than interpreting \cref{def:dedekind-reals} directly, we compute the classical Dedekind cuts and use~\cref{thm:reals-sub-classical}.

\begin{proposition}
  \label{prop:PRT-classical-reals}%
  In a parameterized realizability topos, the object of classical Dedekind reals is isomorphic to $\nabla\RR$.
\end{proposition}

\begin{proof}
  In \cref{sec:assembly-nabl-negn} we saw that $\nabla\two$ is isomorphic to~$\ClProp$,
  hence $\powcl{\objQ}$ is isomorphic to ${\nabla\two}^\objQ$, which in turn is isomorphic to $\nabla\left(\pow{\QQ}\right)$.
  Because the definition of $\ClCut{L,U}$ uses only logical connectives and relations that
  are preserved by~$\nabla$, the object $\RRcl$ is isomorphic to
  %
  \begin{equation*}
    \nabla \set{(L, U) \in \pow{\QQ} \times \pow{\QQ} \such \ClCut{L, U}}.
  \end{equation*}
  %
  We are done because under $\nabla$ above appears the classical definition of~$\RR$.
\end{proof}

\begin{corollary}
  \label{cor:dedekind-characterization}%
  In $\PRT{\AA, \PP}$ the object of Dedekind reals is (isomorphic to) the assembly $\RRd$ whose underlying set is
  %
  $
    |\RRd| = \set{x \in \RR \such \Ex{\RRd}(x) \neq \emptyset}
  $
  %
  and whose existence predicate is
  %
  \begin{equation*}
    \Ex{\RRd}(x) \defeq
    \set{\R{r} \in \AA \such
      \all{p \in \PP}
      \all{k \in \NN}
      \some{q \in \QQ}
      |x - q| < 2^{-k} \land
      \R{r} \, \numeral{k} \rz[p] \Ex{\objQ}(q)
    }.
  \end{equation*}
  %
\end{corollary}

\begin{proof}
  By \cref{prop:PRT-classical-reals,thm:reals-sub-classical} the object of Dedekind reals is (isomorphic to) the
  sub-assembly of $\nabla{\RR}$ whose existence predicate is the realizability interpretation of~\eqref{eq:reals-sub-classical}, which is what~$\Ex{\RRd}$ is.
\end{proof}

The takeaway is that a realizer of a Dedekind real~$x$ computes arbitrarily precise rational approximations of~$x$ which \emph{may} depend on the parameter~$p$.

Recall that the strict order $<$ on $\RRd$ is defined by
%
\begin{equation*}
  x < y \defiff
  \some{q \in \QQ} q \in U_x \land q \in U_y.
\end{equation*}
%
Thus $x < y$ is realized by a rational that is wedged between~$x$ and~$y$.
However, the following lemma shows that we need not bother because the order is $\neg\neg$-stable.

\begin{lemma}
  \label{lem:lt-stable}%
  In $\PRT{\AA, \PP}$,
  %
  \begin{enumerate}
  \item $\all{x, y \in \RRd} x \neq y \lthen x < y \lor y < x$, and
  \item $\all{x, y \in \RRd} \neg \neg (x < y) \lthen x < y$.
  \end{enumerate}
\end{lemma}

\begin{proof}
  Deriving the second statement from the first one is an exercise in intuitionistic logic,
  so we only describe how to realize the first statement.
  %
  Suppose $x, y \in \carrier{\RRd}$, $\R{r} \in \Ex{\RRd}(x)$, $\R{s} \in \Ex{\RRd}(y)$ and~$p \in \PP$.
  Search for the least~$k \in \NN$ such that $|\rat{\R{r} \app[p] k} - \rat{\R{s} \app[p] k}| > 2^{-k+1}$.
  It will certainly be found because $x \neq y$.
  %
  With~$k$ in hand, compare $\rat{\R{r} \app[p] k}$ and $\rat{\R{s} \app[p] k}$.
  If $\rat{\R{r} \app[p] k} < \rat{\R{s} \app[p] k}$ then
  %
  \begin{equation*}
    x < \rat{\R{r} \app[p] k} + 2^{-k}
      < \rat{\R{s} \app[p] k} - 2^{-k}
      < y,
  \end{equation*}
  %
  therefore $x < y$ may be realized by $\rat{\R{r} \app[p] k} + 2^{-k}$.
  %
  If $\rat{\R{s} \app[p] k} < \rat{t \app[p] k}$ then $y < x$ symmetrically.
\end{proof}


\subsection{The Cauchy reals}
\label{sec:cauchy-reals}
%
It is instructive to compare the Dedekind and Cauchy reals in parameterized realizability.
%
We identify the Cauchy reals as those Dedekind reals that are limits of Cauchy sequences of rationals. Since we work without the axiom of countable choice, rapid convergence should be imposed.

\begin{definition}
  \label{def:cauchy-real}
  A \defemph{Cauchy real} is a Dedekind real $x \in \RRd$ which is the limit of a rapidly converging rational sequence, i.e., the set of Cauchy reals is
  %
  \begin{equation*}
    \RRc \defeq \set{x \in \RRd \such \some{q \in \objQ^\objN} \all{n \in \objN} |x - q_n| < 2^{-n}}.
  \end{equation*}
\end{definition}

It would make no difference if we used the classical reals:

\begin{propositionC}
  \label{prop:cauch-real-sub-classical}%
  %
  The set of Cauchy reals $\RRc$ is isomorphic to
  %
  \begin{equation}
    \label{eq:cauchy-characterization}%
    R_c \defeq \set{x \in \RRcl \such \some{q \in \objQ^\objN} \all{n \in \objN} |x - q_n| < 2^{-n}}.
  \end{equation}
\end{propositionC}

% NOTE: here and elsewhere we are silently assuming that the classical reals form a ring and a lattice, which they do.
% (They form a field, and are partially ordered, but do not satisfy ∀ x y z . x < y ⇒ x < z ∨ z < y.)

\begin{proof}
  Any classical real $x \in \RRcl$ that is the limit of a rapidly converging rational sequence is also a Dedekind real.
\end{proof}

The previous proposition tells us how to compute the object of Cauchy reals.

\begin{corollary}
  \label{cor:cauchy-characterization}%
  In $\PRT{\AA, \PP}$ the object of Cauchy reals is (isomorphic to) the assembly $\RRc$ whose
  underlying set is
  $
    |\RRc| \defeq \set{x \in \RR \such \Ex{\RRc}(x) \neq \emptyset}
  $
  and the existence predicate is
  %
  \begin{multline*}
    \Ex{\RRc}(x) \defeq
    \{\R{r} \in \AA \such \\
      \some{q \in \QQ^\NN}
      \all{p \in \PP}
      \all{k \in \NN}
      |x - q_k| < 2^{-k} \land
      \R{r} \, \numeral{k} \rz[p] \Ex{\objQ}(q_k)
   \}.
  \end{multline*}
\end{corollary}

\begin{proof}
  As in \cref{cor:dedekind-characterization}, the existence predicate $\Ex{\RRc}$ is the realizability interpretation of \eqref{eq:cauchy-characterization}.
\end{proof}

If we write the existence predicate for the Dedekind reals in the equivalent form\footnote{The equivalence with the definition given in \cref{cor:dedekind-characterization} relies on countable choice \emph{outside} the topos.}
%
\begin{multline*}
  \Ex{\RRd}(x) \defeq
  \{\R{r} \in \AA \such \\
    \all{p \in \PP}
    \some{q \in \QQ^\NN}
    \all{k \in \NN}
    |x - q_k| < 2^{-k} \land
    \R{r} \, \numeral{k} \rz[p] \Ex{\objQ}(q_k)
  \},
\end{multline*}
%
the difference between the Dedekind and Cauchy reals is seen to be just one of switching the quantifiers:
a rapid sequence representing a Dedekind real may depend on the parameter, whereas one representing a Cauchy real may not.
\begin{theorem}
  \label{thm:cauchy-uncountable}%
  In $\PRT{\AA,\PP}$ the Cauchy reals are sequence-avoiding.
\end{theorem}

\begin{proof}
  For $\R{r} \in \AA$ to realize
  %
  \begin{equation*}
    \all{f \in \RRc^\objN}
    \some{x \in \RRc}
    \all{n \in \objN}
    f n \neq x
  \end{equation*}
  %
  it has to satisfy the following condition:
  for all $f \in \carrier{\RRc^\objN}$, $\R{b} \in \Ex{\RRc^\objN}(f)$ and $p \in \PP$
  there is $x \in \carrier{\RRc}$ such that $\all{n \in \NN} f(n) \neq x$ and
  %
  \begin{equation*}
    \some{t \in \QQ^\NN}
    \all{k \in \NN}
    |x - t_k| < 2^{-k}
    \land
    \all{p' \in \PP} (\R{r} \app[p] \R{b}) \app[p'] k \in \Ex{\objQ}(t_k).
  \end{equation*}
  %
  So suppose $f \in \carrier{\RRc^\objN}$, $\R{b} \in \Ex{\RRc^\objN}(f)$ and $p \in \PP$.
  By unraveling the meaning of $\R{b} \in \Ex{\RRc^\objN}(f)$ we find out that
  there is a map $s : \NN \times \NN \to \QQ$, which depends only on~$\R{b}$ and~$p$,
  such that
  %
  \begin{equation*}
    \all{n, k \in \NN}
    |f(n) - s(n, k)| < 2^{-k}
    \land
    \all{p' \in \PP}
    (\R{b} \app[p] n) \app[p'] k \in \Ex{\objQ}(s(n, k)).
  \end{equation*}
  %
  We define sequences $t, u : \NN \to \QQ$ which depend only on~$s$, such that $0 < u_n - t_n < 2^{-n-1}$ and $f(n) < u_n \lor t_n < f(n)$, for all $n \in \NN$. Set $t_0 \defeq s(0,0) + 1$ and $u_0 \defeq s_0 + \frac{1}{4}$.
  Assuming $t_n$ and $u_n$ have been constructed, let $d \defeq \frac{u_n - t_n}{4}$, $m \defeq \min \set{m \in \NN \such 2^{-m} < d}$, and define
  %
  \begin{equation*}
    (t_{n+1}, u_{n+1}) \defeq
    \begin{cases}
      (t_n, t_n + d) &\text{if $t_n + 2 d < s(n, m)$,} \\
      (u_n - d, u_n) &\text{otherwise.}
    \end{cases}
  \end{equation*}
  %
  % Verification that the conditions are met:
  % * f(0) < s(0,0) + 1 = t_0
  % * u_0 - t_0 = 1/2 < 2^0
  % * u_{n+1} - t_{n+1} = d = (u_n - t_n) / 4 < 2^{-n-1} / 4 < 2^{-n-1} / 2 = 2^{-n-2}
  % * if t_n + 2 d < s(n, m) then u_{n+1} = t_n + d < t_n + 2 d - d < s(n, m) - d < s(n,m) - 2^{-m} < f(n)
  % * if s(n,m) ≤ t_n + 2 d = u_n - 2 d then f(n) < s(n,m) + 2^{-m} < s(n,m) + d ≤ u_n - 2 d + d = u_n - d = t_{n+1}
  %
  The real $x \defeq \lim_n t_n = \lim_n u_n$ satisfies $\all{n \in \NN} f(n) \neq x$ and
  depends only on $\R{b}$ and $p$, because it is defined in terms of~$s$.
  %
  It remains to exhibit~$\R{r} \in \AA$, independent of all parameters, such that $(\R{r} \app[p] \R{b}) \app[p'] k \in \Ex{\objQ}(t_k)$ for all $k \in \NN$ and $p' \in \PP$. It takes the form $\R{r} \defeq \abstr{b} \abstr{k} e$ where~$e$ computes a realizer for~$t_k$, as described in the above procedure, and relying on the fact that $(\R{b} \app[p] n) \app[p'] k$ computes a realizer for~$s(n, k)$.
\end{proof}

Let us investigate where the proof of \cref{thm:cauchy-uncountable} gets stuck if we replace the Cauchy reals with the Dedekind reals.
For $\R{r} \in \AA$ to realize
% 
\begin{equation*}
  \all{f \in \RRd^\objN}
  \some{x \in \RRd}
  \all{n \in \objN}
  f n \neq x
\end{equation*}
%
it has to satisfy the following condition:
for all $f \in \carrier{\RRd^\objN}$, $\R{b} \in \Ex{\RRd^\objN}(f)$ and $p \in \PP$
there is $x \in \carrier{\RRd}$ such that $\all{n \in \NN} f(n) \neq x$ and
% 
\begin{equation*}
  \all{k \in \NN}
  \all{p' \in \PP}
  \some{t \in \QQ}
  |x - t| < 2^{-k}
  \land
  (\R{r} \app[p] \R{b}) \app[p'] k \in \Ex{\objQ}(t).
\end{equation*}
%
So suppose $f \in \carrier{\RRd^\objN}$, $\R{b} \in \Ex{\RRd^\objN}(f)$ and $p \in \PP$.
By unraveling the meaning of $\R{b} \in \Ex{\RRd^\objN}(f)$ we find out that
for every $p' \in \PP$ there is a map $s : \NN \times \NN \to \QQ$, which depends on~$\R{b}$ and~$p$ \emph{as well as~$p'$}, such that
%
\begin{equation*}
  \all{n, k \in \NN}
  |f(n) - s(n, k)| < 2^{-k}
  \land
  (\R{b} \app[p] n) \app[p'] k \in \Ex{\objQ}(s(n, k)).
\end{equation*}
%
At this point we are stuck: if we continue by constructing a sequence $t : \NN \to \QQ$ as in the proof, its limit $\lim_n t_n$ will depend on~$p'$, but we need a real~$x$ that does not.
For a specific ppca $(\AA, \PP)$ one might find some way of constructing~$x$ that avoids dependence on~$p'$,
although there can be such no general construction as that would contradict~\cref{thm:countable-reals}.

Finally, let us show that in $\PRT{\AA, \PP}$ the carrier sets of Cauchy and Dedekind reals coincide.
%
Recall that $x \in \RRd$ is \defemph{strongly irrational}\footnote{In intuitionistic mathematics this is a stronger notion than being \defemph{irrational}, which means $x \neq q$ for all~$q \in \objQ$.} when $x < q \lor q < x$ for all $q \in \objQ$.

\begin{lemmaC}
  \label{lem:rat-irrat-cauchy}
  Every rational and every strongly irrational Dedekind real is a Cauchy real.
\end{lemmaC}

\begin{proof}
  Clearly, every rational number is a Cauchy real.
  %
  Suppose $x \in \RRd$ is strongly irrational. Then we may find a rational sequence converging rapidly to~$x$ by simple bisection, as follows. There are rational $t_0$ and $u_0$ such that $t_0 < x < u_0$ and $u_0 - t_0 < 1$.
  Given rational $t_n < x < u_n$ such that $u_n - t_n < 2^{-n}$, let $m \defeq \frac{t_n + u_n}{2}$ and define
  %
  \begin{equation*}
    (t_{n+1}, u_{n+1}) =
    \begin{cases}
      (t_n, m) & \text{if $x < m$,} \\
      (m, u_n) & \text{if $m < x$.}
    \end{cases}
  \end{equation*}
  %
  Then both $t$ and $u$ are rapid rational sequences converging to~$x$.
\end{proof}

\begin{proposition}
  \label{prop:carrier-Rd-eq-Rc}%
  In a parameterized realizability topos, $\carrier{\RRd} = \carrier{\RRc}$.
\end{proposition}

\begin{proof}
  Combine \cref{lem:lt-stable} and \cref{lem:rat-irrat-cauchy} to conclude that
  in $\PRT{\AA, \PP}$ every rational and every irrational Dedekind real is a Cauchy real.
  Of course, there is no Dedekind real which is neither rational nor irrational.
\end{proof}

To summarize, the difference between $\RRc$ and $\RRd$ is not one of extent but one of \emph{structure}:
the identity map $x \mapsto x$ is realized as a morphism $\RRc \to \RRd$, but not necessarily as a morphism $\RRd \to \RRc$.


%%% Local Variables:
%%% mode: latex
%%% TeX-master: "countable-reals"
%%% End:
 
\section{A topos with countable reals}
\label{sec:topos-with-countable}

Given a Miller sequence~$\mil$, as in~\cref{sec:non-diag-sequ}, let~$\MM{\mil} \defeq \invim{\srep}(\mil)$ be the set of all oracles representing~$\mil$ where $\srep$ is the representing map from \cref{sec:oracle-comp-maps}. Define $\TT{\mil} \defeq \PRT{\KK,\MM{\mil}}$ to be the parameterized realizability topos constructed from the ppca $\KK$ with oracles $\MM{\mil}$, see \cref{ex:oracle-ppca}.

\subsection{Countability of the reals}
\label{sec:countability-reals}
%
Let us immediately address countability of the Dedekind reals in~$\TT{\mil}$. 
We first reduce the problem to countability of the closed interval.

\begin{lemmaC}
  \label{lem:R-contable-iff-I-countable}%
  The real numbers are countable if, and only if, the closed unit interval is countable.
\end{lemmaC}

\begin{proof}
  If $\RR$ is countable then $[0,1]$ is countable because there is a retraction $\RR \to [0,1]$, for instance
  $x \mapsto \max(0, \min(1, x))$.
  %
  Conversely, given $e : \NN \to [0,1]$ is a surjection, we claim that $e' : \NN \to \RR$, defined by
  $e'(\pair{m, n}) \defeq m \cdot (2 \cdot e(n) - 1)$, is a surjection also. For any $x \in \RR$ there is $m \in \NN$ such that
  $-m < x < m$, and there is $n \in \NN$ such that $e(n) = \frac{x + m}{2 m}$, hence $e'(\pair{m, n}) = x$.
  %
  % computation of e'(<m, n>):
  % e'(<m, n>) = m (2 e(n) - 1)
  %           = m (2 ((x + m)/(2m)) - 1)
  %           = m ((x + m)/m - 1)
  %           = m (x/m)
  %           = x.
\end{proof}

Next, we obtain custom descriptions of the assemblies of natural numbers and the closed unit interval.

\begin{lemma}
  \label{lem:nno-assembly}%
  In the topos $\TT{\mil}$ the natural numbers object is isomorphic to the assembly $\objN$ with carrier
  $\carrier{\objN} \defeq \NN$ and the existence predicate
  %
  $\Ex{\objN}(n) \defeq \set{n}$.
\end{lemma}

\begin{proof}
  In \cref{sec:natur-numb-integ} we saw that the natural numbers object is the assembly with carrier
  $\NN$ and existence predicate $n \mapsto \set{\numeral{n}}$. The assembly from the statement is isomorphic
  to it because we may convert between $\numeral{n}$ and $n$ using the combinators~$\combNum$ and~$\combCur$ from \cref{ex:numers-vs-numerals}.
\end{proof}

Henceforth we use~$\objN$ from \cref{lem:nno-assembly} as the standard natural numbers object. The practical consequence is that we may eschew Curry numerals and instead use numbers directly.

\begin{lemma}
  \label{lem:interval-assembly}%
  In the topos $\TT{\mil}$ the closed unit interval is isomorphic to the assembly $\objI$ with carrier
  $\carrier{\objI} \defeq [0,1] \cap \carrier{\RRd}$ and the existence predicate
  %
  $\Ex{\objI}(x) \defeq \set{m \in \KK \such \mil(m) = x}$.
\end{lemma}

\begin{proof}
  The sub-assembly $\set{x \in \RRd \such 0 \leq x \land x \leq 1}$ has $[0,1] \cap \carrier{\RRd}$ as its carrier set. Its existence predicate is the tripos predicate
  %
  $[x \of \RRd \such 0 \leq x \land x \leq 1]$,
  %
  which is $\neg\neg$-stable, and therefore equivalent to $\Ex{\RRd}(x)$ restricted to~$\carrier{\objI}$.
  %
  Thus, it suffices to show that the tripos logic validates
  %
  \begin{equation}
    \label{eq:objI-ER-to-EI}
    %
    x \of \carrier{\objI} \such \Ex{\RRd}(x) \to \Ex{\objI}(x).
  \end{equation}
  %
  and
  %
  \begin{equation}
    \label{eq:objI-EI-to-ER}
    x \of \carrier{\objI} \such \Ex{\objI}(x) \to \Ex{\RRd}(x).
  \end{equation}
  %
  By \cref{cor:dedekind-characterization}, $\R{r} \in \Ex{\RRd}(x)$ is equivalent to
  %
  \begin{equation}
    \label{eq:objI-rz-x}
    \all{\alpha \in \MM{\mil}}
    \all{k \in \NN}
    |x - \rat{\pr[\alpha]{\R{r}}(k)}| < 2^{-k},
  \end{equation}
  %
  where we used $\objN$ from \cref{lem:nno-assembly}.
  %
  Condition \eqref{eq:objI-rz-x} states that~$\R{r}$ is a $\mil$-index for~$x$ in the sense of \cref{def:sequence-computable}, therefore $\mil(\R{r}) = x$ and we may realize \eqref{eq:objI-ER-to-EI} with $\ucode{\abstr{r} r}$.

  It remains to realize \eqref{eq:objI-EI-to-ER}.
  %
  In \cref{sec:oracle-comp-maps} we obtained $\R{v} \in \NN$ such that, if~$\alpha$ codes $\mil$ then
  $\pr[\alpha]{\R{v}}(m)$ represents $\mil(m)$, for all $m \in \NN$.
  Clearly, $\R{v}$ realizes \eqref{eq:objI-EI-to-ER}.
\end{proof}

\begin{theorem}
  \label{thm:countable-reals}
  In the topos $\TT{\mil}$ there is an epimorphism from natural numbers to Dedekind reals.
\end{theorem}

\begin{proof}
  By \cref{lem:R-contable-iff-I-countable,lem:interval-assembly} it suffices to show that the Miller sequence $\mil : \objN \to \objI$, which is realized by $\ucode{\abstr{m} m}$, is an epimorphism. This is so because
    %
  \begin{equation*}
    \all{x \in \objI} \some{m \in \objN} \mil(m) = x
  \end{equation*}
  %
  is trivially realized by $\ucode{\abstr{m}{m}}$ as well.
  %
  % Verification: the topos statement computes to the following tripos predicate:
  % [ ∀ x ∈ I . ∃ m ∈ N . μ(m) = x ] =
  % [ ∀ x : |I| . E_I(x) → ∃ m ∈ ℕ . E_I(μ(m)) ∩ E_I(x) ] =
  % [ ∀ x : |I| . E_I(x) → ∃ m ∈ ℕ . E_I(μ(m)) ∩ E_I(x) ]
  %
  % Given any x ∈ |I| and r ∈ E_I(x), there is m ∈ ℕ such that r = i(m) and μ(m) = x.
  % So we get E_I(μ(m)) ∩ E_I(x) = E_I(x) and (α | (⟨m⟩ m) r) = (α | r), as required.
\end{proof}

\subsection{What else is countable?}
\label{sec:what-else-countable}
%
Given that \cref{thm:fixed-point-R-uncountable,thm:countable-reals,thm:cauchy-uncountable} sandwich the countable Dedekind reals between uncountable Cauchy reals and uncountable MacNeille reals, it is natural to wonder about which classically uncountable spaces are countable in~$\TT{\mil}$.

Products, sums and images of countable sets are countable, which gives basic examples of countable spaces, such as Euclidean spaces $\RRd^n$, hypercubes~$\objI^n$, the unit circle $T \defeq \set{(x,y) \in \RRd \times \RRd \such x^2 + y^2 = 1}$, $n$-spheres, etc.

William F.~Lawvere's~\cite{lawvere69} fixed-point theorem is a source of uncountable sets.

\begin{theoremC}[Lawvere]
  \label{thm:lawvere}
  If $e : A \to B^A$ is surjective then $f : B \to B$ has a fixed point.
\end{theoremC}

\begin{proof}
  Because $e$ is surjective,
  there is $a \in A$ such that $e(a) = (x \mapsto f(e(x)(x)))$, whence $e(a)(a) = f(e(a)(a))$.
\end{proof}

As soon as there is a fixed-point free map $X \to X$, there is no surjection $\objN \to X^\objN$, by the contra-positive of Lawvere's theorem. We already noted in \cref{cor:cantor-diagonal} this to be the case for Cantor space.
%
Two further examples are the countable powers $\RRd^\objN$ and $T^\objN$ of the Dedekind reals and the unit circle, which are uncountable because (non-trivial) translations of~$\RRd$ and rotations of~$T$ have no fixed points.

How about the Hilbert cube $\objI^\objN$?
One might attempt to enumerate it by composing $\mil : \objN \to \objI$ with a space-filling curve $\objI \to \objI^\objN$. However, even constructing just a square-filling curve $\objI \to \objI \times \objI$ in~$\TT{\mil}$ seems impossible,\footnote{One cannot construct intuitionistically a square-filling curve $[0,1] \to [0,1] \times [0,1]$ because there is no such curve in the topos of sheaves on the closed unit square, although countable choice suffices. We do not know whether there is a square-filling curve in~$\TT{\mil}$.}
 so we take another route.
We first need a lemma showing that the elements of $\objI^\objN$ take a special form.

\begin{lemma}
  \label{lem:flattening-realizers}
  There is a total computable function $\ell : \NN \times \NN \to \NN$ such that $\mil(\ell(m,n)) = f(n)$ for all $f \in \carrier{\objI^\objN}$, $m \in \Ex{\objI^\objN}(f)$, and $n \in \NN$.
\end{lemma}

\begin{proof}
  In \cref{sec:oracle-comp-maps} we obtained $\R{v} \in \NN$ such that $\pr[\alpha]{\R{v}}(n)$ represents $\mil(n)$ for all~$n \in \NN$. The map $\ell : \NN \times \NN \to \NN$,
  %
  \begin{equation*}
    \ell(m, n) \defeq \ucode{\abstr{x} \R{v} \, (m \, n) \, x},
  \end{equation*}
  %
  is well-defined by \cref{lem:abstr-uniform} and is computable.

  Now consider any $f \in \carrier{\objI^\objN}$ and $m \in \Ex{\objI^\objN}(f)$.
  %
  Given any $n \in \NN$, we establish $\mil(\ell(m, n)) = f(n)$ by verifying that $\rcomp{\mil}{\ell(m,n)} = f(n)$.
  %
  For any $\alpha \in \MM{\mil}$ and $k \in \NN$,
  %
  \begin{align*}
    \pr[\alpha]{\ell(m,n)}(k)
    &\kleq \alpha \at (\abstr{x} \R{v} \, (m \, n) \, x) \, k \\
    &\kleq \alpha \at \R{v} \, (m \, n) \, k \\
    &\kleq
         \pr[\alpha]{
           \pr[\alpha]{\R{v}}(
             \pr[\alpha]{m}(n)
           )
         }(k),
  \end{align*}
  %
  therefore $\pr[\alpha]{\ell(m,n)} = \pr[\alpha]{\pr[\alpha]{\R{v}}(\pr[\alpha]{m}(n))}$, which is a representation
  of $f(n)$.
\end{proof}

A curious consequence of \cref{lem:flattening-realizers} is that every $f \in \carrier{\objI^\objN}$ is equal to $\mil\circ g$ for some total $g : \NN \to \NN$ that is computable without an oracle.

\begin{theorem}
  \label{thm:hilbert-countable}%
  In the topos $\TT{\mil}$ the Hilbert cube~$\objI^\objN$ is countable.
\end{theorem}

\begin{proof}
  Let $\ell : \NN \times \NN \to \NN$ be as in Lemma~\ref{lem:flattening-realizers} and
  $\R{l} \in \NN$ a realizer for~$\ell$, which exists because~$\ell$ is computable.
  % 
  The map $e : \carrier{\objN} \to \carrier{\objI^\objN}$, defined by $e(m)(n) \defeq \mil(\ell(m,n))$, is realized by
  $\ucode{\abstr{m n} \R{l} \, (\combPair \, m \, n)}$.
  % 
  To show that $e$ is an epimorphism it suffices to prove that
 % 
  \begin{equation*}
    \all{f \in \objI^\objN}
    \some{m \in \objN}
    \all{n \in \objN}
    \mil(\ell(m,n)) = f(n)
  \end{equation*}
  %
  is realized by~$\ucode{\abstr{x} x}$.
  %
  By unfolding the realizability interpretation we find that this amounts to
  %
  \begin{equation*}
    \all{\alpha \in \MM{\mil}}
    \all{\R{b} \in \Ex{\objI^\objN}(f)}
    \all{n \in \NN}
    \R{b} \, n \rz[\alpha] \mil(\ell(\R{b},n)) = f(n).
  \end{equation*}
  %
  This is indeed true by \cref{lem:flattening-realizers} and the fact that~$\R{b}$ realizes~$f$.
\end{proof}

\section{\texorpdfstring{Mathematics in the topos~$\TT{\mil}$}{Mathematics in the topos Tμ}}
\label{sec:analysis-topos-tt}

We devote the last section to exploring a little further the peculiar new mathematical world~$\TT{\mil}$.

\subsection{Brouwer's fixed-point theorem}
\label{sec:brouwers-fixed-point}
%
The reader may have noticed already that having a surjection $\objN \to \objI^\objN$ is precisely the antecedent of Lawvere's theorem, which allows us to easily prove Brouwer's fixed-point theorem.

\begin{theorem}[Brouwer's fixed-point theorem]
  \label{thm:internal-brouwer}%
  In the topos $\TT{\mil}$ every map $\objI^\objN \to \objI^\objN$ has a fixed point,
  and so does every map $\objI^n \to \objI^n$, for every $n \in \NN$.
\end{theorem}

\begin{proof}
  Combining \cref{thm:hilbert-countable} and Lawvere's \cref{thm:lawvere} yields a fixed point of any map $f : \objI \to \objI$. When $n = 0$ the statement is trivial. For the remaining cases, note that the evident bijections $\objN \to \objN \times \objN$ and $\objN \to \set{1, \ldots, n} \times \objN$ induce bijections $\objI^\objN \to (\objI^\objN)^\objN$ and $\objI^\objN \to (\objI^n)^\objN$.
  Composing these with the surjection $e : \objN \to \objI^\objN$ yields surjections $\objN \to (\objI^\objN)^\objN$ and $\objN \to (\objI^n)^\objN$, respectively, so Lawvere's theorem applies again.
\end{proof}

We stated Brouwer's fixed-point theorem for \emph{all} maps, not just the continuous ones, but this is a mirage because all maps are continuous in~$\TT{\mil}$, as we show in \cref{sec:continuity-maps}.

We take a moment to remark that \cref{thm:internal-brouwer} is a fairly unusual property for an intuitionistic topos to have because it implies a constructive taboo, namely the so-called \emph{Limited Lesser Principle of Omniscience (LLPO)}.

\begin{corollary}[LLPO]
  \label{cor:llpo}%
  In the topos $\TT{\mil}$ every Dedekind real is non-negative or non-positive.
\end{corollary}

\begin{proof}
  Given any $x \in \RRd$, the map $y \mapsto \max(0, \min(1, y + x))$ has a fixed point~$y \in \objI$.
  Either $\sfrac{1}{3} < y$ or $y < \sfrac{2}{3}$.
  %
  If $\sfrac{1}{3} < y$ then $x \geq 0$, because $x < 0$ would imply $y = \max(0, \min(1, y + x)) = \max(0, y + x) < y$.
  %
  If $y < \sfrac{2}{3}$ then $x \leq 0$, because $x > 0$ would imply $y = \max(0, \min(1, y + x)) = \min(1, y + x) > y$.
\end{proof}

Sometimes LLPO is phrased as follows: if $a : \objN \to \set{0,1}$ attains value~$1$ at most once, then either $\all{n} a_{2 n} = 0$ or $\all{n} a_{2 n + 1} = 0$. This form follows from \cref{cor:llpo}: if $\sum_{n} a_n \cdot (- \sfrac{1}{2})^n$ is non-negative then $\all{n} a_{2 n} = 0$ and if it is non-positive then $\all{n} a_{2 n + 1} = 0$.

In~\cref{sec:comp-clos-interv} we shall use the following variant of Brouwer's fixed point theorem for partial maps with $\neg\neg$-stable domains of definition.

\begin{theorem}
  \label{thm:partial-brouwer}%
  For every $n \in \NN$, the topos $\TT{\mil}$ validates
  %
  \begin{equation*}
    \all{\phi \in \objI^n \to \ClProp}
    \all{f \in {\set{x \in \objI^n \such \phi(x)}} \to \objI^n}
    \some{y \in \objI^n}
    \phi(y) \lthen f(y) = y.
  \end{equation*}
\end{theorem}

\begin{proof}
  We demonstrate the proof for $n = 2$, which is the instance used in \cref{prop:drinking-buddies}.
  %
  We seek $\R{r} \in \NN$ such that, for all $\alpha \in \MM{\mil}$,
  $\phi : \carrier{\objI^2} \to \set{\bot, \top}$, and
  $f : \set{x \in \carrier{\objI^2} \such \phi(x)} \to \carrier{\objI^2}$
  with $\R{f} \in \NN$ satisfying
  %
  \begin{equation*}
    \all{x \in \carrier{\objI^2}}
    \all{\R{x} \in \Ex{\objI^2}(x)}
    \all{\beta \in \MM{\mil}}
    \phi(x) \lthen (\beta \at \R{f} \, \R{x}) \in \Ex{\objI^2}(f(x)),
  \end{equation*}
  %
  there is $y \in \carrier{\objI^2}$ such that $(\alpha \at \R{r} \, \R{f}) \in \Ex{\objI^2}(y)$ and if $\phi(y)$ then $f(y) = y$. Recalling the fixed-point combinator~$\comb{Z}$ from \cref{sec:progr-with-ppcas}, we define
  %
  \begin{align*}
    \R{r} &\defeq \ucode{\comb{Z} \,
                (\abstr{s \, g} g \,
                    (\combPair \,
                      (\abstr{k} \combFst \, (s \, g) \, k) \,
                      (\abstr{k} \combSnd \, (s \, g) \, k)%
                    )
                )},\\
    \R{a} &\defeq \ucode{\abstr{k} \combFst \, (\R{r} \, \R{f}) \, k},\\
    \R{b} &\defeq \ucode{\abstr{k} \combSnd \, (\R{r} \, \R{f}) \, k},
  \end{align*}
  %
  and $y \defeq (\mil(\R{a}), \mil(\R{b}))$.
  %
  Suppose $\phi(y)$ holds. Then
  %
  $\alpha \at \R{f} \, (\combPair \, \R{a} \, \R{b})$ is defined and
  $(\alpha \at \R{f} \, (\combPair \, \R{a} \, \R{b}) \in \Ex{\objI^2}(f(y))$.
  % r := Z (<r' g> g (pair (<k> fst (r' g) k) (<k> snd (r' g) k)))
  %
  % α | r f =
  % α | Z (<r' g> g (pair (<k> fst (r' g) k) (<k> snd (r' g) k))) f =
  % α | (<r' g> g (pair (<k> fst (r' g) k) (<k> snd (r' g) k))) r f =
  % α | f (pair (<k> fst (r f) k) (<k> snd (r f) k))
  % α | f (pair a b)
  Because $\alpha \at \R{f} \, (\combPair \, \R{a} \, \R{b}) = \alpha \at \R{r} \, \R{f}$,
  it suffices to show that $\R{r} \, \R{f} \in \Ex{\objI^2}(y)$.
  This is the case because $\R{r} \, \R{f}$ realizes an ordered pair, hence its components
  $\combFst \, (\R{r} \, \R{f})$ and $\combSnd \, (\R{r} \, \R{f})$ are defined, and respectively
  compute the same sequences as $\R{a}$ and $\R{b}$.
\end{proof}

\subsection{The intermediate value theorem}
\label{sec:interm-value-theor}
%
The 1-dimensional Brouwer's fixed-point theorem and the Intermediate value theorem are derivable from each other.
%

\begin{lemmaC}
  \label{lem:max-neq-eq}%
  If $\max(a, b) \neq a$ then $\max(a, b) = b$, and similarly for $\min$.
\end{lemmaC}

\begin{proof}
  If $\max(a, b) \neq a$ then $\neg (b \leq a)$.
  By \cref{prop:RRd-stable-equality} it suffices to show that $\neg (\max(a, b) \neq b)$.
  If $\max(a, b) \neq b$ then $\neg (a \leq b)$, which together with $\neg (b \leq a)$ yields a contradiction.
\end{proof}

\begin{theorem}(Intermediate value theorem)
  In the topos $\TT{\mil}$, if $f : \objI \to \RRd$ satisfies $f(0) < 0 < f(1)$ then $f(x) = 0$ for some $x \in \objI$.
\end{theorem}

\begin{proof}
  Given such an~$f$, define $g : \objI \to \objI$ by
  %
  $g(x) \defeq \max(0, \min(1, x - f(x)))$.
  %
  By \cref{thm:internal-brouwer} there is $x \in \objI$ such that
  %
  \begin{equation*}
    \max(0, \min(1, x - f(x))) = x.
  \end{equation*}
  %
  Use \cref{lem:max-neq-eq} lemma to derive $\min(1, x - f(x)) = x$ and once more to derive $x - f(x) = x$,
  yielding $f(x) = 0$.
\end{proof}

\subsection{All maps are continuous}
\label{sec:continuity-maps}
%
Say that $f : \RR \to \RR$ \defemph{jumps at $x$} if there is $\epsilon > 0$ such that $|f(y) - f(x)| > \epsilon$ for all $y > x$. Countability of reals is at odds with existence of such explicitly discontinuous maps.

\begin{propositionC}
  If there is a map with a jump then $\RR$ is uncountable.
\end{propositionC}

\begin{proof}
  Without loss of generality we consider $f : \RR \to \RR$ such that~$f(0) = 0$ and $f(x) > 1$ for all $x > 0$.
  %
  Given a sequence $a : \NN \to \RR$, we construct a real avoiding it by using~$f$ to make decisions in the
  construction of nested intervals $[u_n, v_n]$, as follows. Set $[u_0, v_0] \defeq [0, 1]$ or any other desired initial interval. Assuming $[u_n, v_n]$ has been constructed, let $t \defeq f(\max(0, a_n - (u_n + v_n)/2))$, and
  %
  \begin{equation*}
    [u_{n+1}, v_{n+1}] \defeq
    \begin{cases}
      [u_n, (3 u_n + v_n)/4] &\text{if $t > \sfrac{1}{3}$,} \\
      [(u_n + 3 v_n)/4, v_n] &\text{if $t < \sfrac{2}{3}$.}
    \end{cases}
  \end{equation*}
  %
  The interval $[u_{n+1}, v_{n+1}]$ is well-defined because exactly one of the cases holds.
  Certainly at least one holds, and if they both did then $\sfrac{1}{3} < t < \sfrac{2}{3}$, whence neither $\max(0, a_n - (u_n + v_n)/2) = 0$ nor $\max(0, a_n - (u_n + v_n)/2) > 0$, an impossibility.
  %
  Furthermore, $[u_{n+1}, v_{n+1}]$ avoids~$a_n$ because $t > \sfrac{1}{3}$ implies $a_n \geq (u_n + v_n)/2$ and $t < \sfrac{2}{3}$ implies $a_n \leq (u_n + v_n)/2$.
  %
  The real $x \defeq \lim_n u_n = \lim_n v_n$ thus avoids the sequence~$a$.
\end{proof}

Thus in~$\TT{\mil}$ there are no maps with jumps, and we can do better than that.

\begin{theorem}[KLST]
  In the topos $\TT{\mil}$ all maps $\RRd \to \RRd$ are continuous.
\end{theorem}

\begin{proof}
  The theorem bears the initials of its authors Kreisel, Lacombe, Shoenfield~\cite{KreiselLacombeShoenfield59} and Tseitin~\cite{Tseitin67}. We repurpose a proof that uses the Recursion theorem, checking along the way that it relativizes and is uniform with respect to oracles.

  It suffices to realize continuity at~$0$, specifically the statement
  %
  \begin{equation*}
    \all{f \in \RRd^{\RRd}}
    \all{k \in \objN}
    \some{m \in \objN}
    \all{x \in \RRd}
    |x| < 2^{-m} \lthen |f(x) - f(0)| < 2^{-k + 3},
  \end{equation*}
  %
  which amounts to having a realizer $\R{klst} \in \NN$ such that,
  %
  \begin{multline*}
    \all{f \in \carrier{\RRd^{\RRd}}}
    \all{\R{f} \in \Ex{\RRd^{\RRd}}(f)}
    \all{k \in \NN}
    \all{\alpha \in \MM{\mil}}
    \some{m \in \NN} \\
    (\alpha \at \R{klst} \, \R{f} \, k) = m
    \land
    \all{x \in \carrier{\RRd}}
    |x| < 2^{-m} \lthen |f(x) - f(0)| < 2^{-k + 3}.
  \end{multline*}
  %
  Rather than attempting to write down $\R{klst}$ explicitly, we shall describe an $\alpha$-computable procedure, uniform in~$\alpha \in \MM{\mil}$, which takes as input $\R{f} \in \Ex{\RRd^\RRd}(f)$ and $k \in \NN$, as above, and outputs a suitable~$m \in \NN$ that may depend on all the parameters, including~$\alpha$.

  We start with some auxiliary definitions.
  %
  Let $\R{zero} \in \NN$ be such that $\rat{\R{zero}} = 0$.
  Let $\theta : \NN \times \NN \to \NN$ be a computable map satisfying
  %
  \begin{equation*}
    \pr[\alpha]{\theta(n, j)}(i) =
    \begin{cases}
      \R{zero} &\text{if $i < n$,} \\
      j       &\text{if $i \geq n$.}
    \end{cases}
  \end{equation*}
  %
  If $|\rat{j}| < 2^{-n}$ then $\theta(n, j) \in \Ex{\RRd}(\rat{j})$.

  Next, define $g : \NN \parto \NN$ by $g(i) \defeq \alpha \at \R{f} \, i \, k$ and
  $h : \NN \times \NN \to \NN \cup \set{\star}$ by
  %
  \begin{equation*}
    h(i, t) \defeq \prx[\alpha]{\pr[\alpha]{\R{f}}(i)}{t}(k).
  \end{equation*}
  %
  If the computation $\alpha \at \R{f} \, i \, k$ terminates within~$t$ steps then $h(i, t) = g(i)$, otherwise $h(i, t) = \star$. For any $x \in \carrier{\RRd}$ and $\R{x} \in \Ex{\RRd}(x)$ we have $|f(x) - \rat{g(\R{x})}| < 2^{-k}$, and if $h(\R{x}, t) \neq \star$ then $|f(x) - \rat{h(\R{x}, t)}| < 2^{-k}$ as well.

  By the relativized Kleene's recursion theorem there is a computable map $r : \NN \times \NN \to \NN$, independent of~$\alpha$, such that for all $t \in \NN$:
  %
  \begin{itemize}
  \item if $h(r(\R{f}, k),t) = \star$ then $\pr[\alpha]{r(\R{f}, k)}(t) = \R{zero}$,
  \item if $m \leq t$ is the least number such that $h(r(\R{f}, k), m) \neq \star$ then
    %
    \begin{equation*}
      \pr[\alpha]{r(\R{f}, k)}(t) \simeq
      \min\nolimits_j (|\rat{j}| < 2^{-m} \land |\rat{h(r(\R{f}, k), m)} - \rat{g(\theta(m,j))}| \geq 2^{-k+1}).
    \end{equation*}
  \end{itemize}

  We let our $\alpha$-computable procedure output the least~$m \in \NN$ satisfying $h(r(\R{f}, k), m) \neq \star$.
  %
  Of course, we need to argue that such an~$m$ exists and that
  %
  \begin{equation}
    \label{eq:klst-1}
    \all{x \in \carrier{\RRd}}
    |x| < 2^{-m} \lthen |f(x) - f(0)| < 2^{-k + 3}.
  \end{equation}
  %
  If $h(r(\R{f}, k), t) = \star$ for all $t \in \NN$ then $r(\R{f}, k) \in \Ex{\RRd}(0)$,
  therefore $g(r(\R{f}, k))$ is defined and so $h(r(\R{f}, k), t) = g(r(\R{f}, k)) \neq \star$ for a sufficiently large~$t$, a contradiction.
  It is thus impossible for~$m$ not to exist, so it exists.\footnote{A necessary meta-level application of Markov's principle~\cite{beeson84:_churc}.}

  We claim that $|\rat{h(r(\R{f},k),m)} - f(q)| < 2^{-k+1}$ for all $q \in \QQ$ satisfying $|q| < 2^{-m}$.
  Consider any such~$q$, and suppose the contrary inequality $|\rat{h(r(\R{f},k),m)} - f(q)| \geq 2^{-k+1}$ would hold.
  Then there is a least~$j$ such that $|\rat{j}| < 2^{-m}$ and $|\rat{h(r(\R{f},k),m)} - f(\rat{j})| \geq 2^{-k+1}$,
  in which case $\pr[\alpha]{r(\R{f},k)} = \pr[\alpha]{\theta(m,j)}$ and $r(\R{f},k), \theta(m,j) \in \Ex{\RRd}(\rat{j})$, therefore
  %
  \begin{equation*}
    |\rat{h(r(\R{f}, k), m)} - \rat{g(\theta(m,j))}| \leq
    |\rat{h(r(\R{f}, k), m)} - f(\rat{j})| + |f(\rat{j}) - \rat{g(\theta(m,j))}| < 2^{-k + 1}.
  \end{equation*}
  %
  At the same time $|\rat{h(r(\R{f}, k), m)} - \rat{g(\theta(m,j))}| \geq 2^{-k + 1}$ by the definition of~$r(\R{f}, k)$, a contradiction.

  To establish~\eqref{eq:klst-1}, consider any $x \in \carrier{\RRd}$ with $\R{x} \in \Ex{\RRd}(x)$ such that $|x| < 2^{-m}$,
  and suppose $|f(x) - f(0)| \geq 2^{-k + 3}$ were the case. Then we could solve the Halting problem relative to an oracle~$\alpha \in \MM{\mil}$, as follows. There is $\ell \in \NN$ such that $|x| + 2^{-\ell} < 2^{-m}$. Define $\xi_\alpha : \NN \times \NN \times \NN \to \NN$ by
  %
  \begin{equation*}
    \xi_\alpha(i, j, n) \defeq
    \begin{cases}
      \alpha \at \R{x} \, n  &\text{if $n \leq \ell$ or $\prx[\alpha]{i}{n}(j) = \star$,} \\
      \alpha \at \R{x} \, n' &\text{if $n' \leq n$ least such that $\ell < n'$ and $\prx[\alpha]{i}{n'}(j) \neq \star$.}
    \end{cases}
  \end{equation*}
  %
  The sequence $n \mapsto \xi_\alpha(i, j, n)$ represents a real~$y$ such that $|y| < 2^{-m}$.
  %
  If $\pr[\alpha]{i}(j)$ is undefined then $x = y$ and so $|f(y) - f(0)| \geq 2^{-k+3}$.
  %
  If $\pr[\alpha]{i}(j)$ is defined then $y \in \QQ$ and so by the above claim
  %
  \begin{equation*}
    |f(y) - f(0)| \le |f(y) - \rat{h(r(\R{f}, k), m)}| + |\rat{h(r(\R{f}, k), m)} - f(0)| < 2^{-k+2}.
  \end{equation*}
  %
  To decide whether $\pr[\alpha]{i}(j)$ is defined we compute a sufficiently good approximation of $|f(y) - f(0)|$ to be able to tell whether $|f(y) - f(0)| < 2^{-k+3}$ or $|f(y) - f(0)| > 2^{-k+2}$.
\end{proof}


\subsection{Compactness of the closed interval}
\label{sec:comp-clos-interv}
%
In constructive mathematics various classically equivalent notions of compactness diverge~\cite{bridges02:_compac_contin_const_revis}.
%
We focus on the Heine-Borel compactness of the closed unit interval, which states that every open cover has a finite subcover, as it is the most interesting one in the topos~$\TT{\mil}$.

Say that a sequence of open intervals $(a_0, b_0), (a_1, b_1), \ldots$ forms a \defemph{singular cover} of $[0,1]$ if it covers the interval, but the sum of lengths $\sum_{i \in \NN} b_i - a_1$ is less than~$1$. 
%
Of course, such a thing does not exist classically. In the topos~$\TT{\mil}$ it is readily manufactured from the enumeration $\mil : \objN \to \objI$, just take any $0 < \epsilon < 1$ and set
%
\begin{equation*}
  (a_i, b_i) \defeq (\mil_i - \epsilon \cdot 2^{-i-1}, \mil_i + \epsilon \cdot 2^{-i-1}).
\end{equation*}
%
The $i$-th interval covers $\mil_i$ and $\sum_{i \in \NN} b_i - a_1 = \epsilon$.
Consequently, the Heine-Borel property fails strongly.

\begin{theoremC}
  \label{thm:singular-cover}%
  %
  If $(a_0, b_0), (a_1, b_1), \ldots$ is a singular cover of~$[0,1]$ then for every $n \in \NN$ the set $[0,1] \setminus \bigcup_{i < n} (a_i, b_i)$ is inhabited.
\end{theoremC}

\begin{proof}
  We prove the following stronger statement:
  %
  if $[a_0, b_0], \ldots, [a_n, b_n]$ are closed intervals and $(u, v)$ is an open interval such that $\sum_{i=1}^n b_i - a_i < v - u$, then there is $x \in (u, v)$ which is not in any $[a_i, b_i]$.

  Suppose first that all the endpoints are rational numbers, so that comparisons between them are decidable.
  %
  For each $k = 0, \ldots, n$ we compute a list of pairwise disjoint intervals with rational endpoints
  %
  \begin{equation*}
    (u_{k,1}, v_{k,1}), \ldots, (u_{k, m_k}, v_{k, m_k})
  \end{equation*}
  %
  such that $m_k > 0$, $\sum_{i=k+1}^n b_i - a_i < \sum_{j=1}^{m_k} v_{k,j} - u_{k,j}$, and
  %
  \begin{equation*}
    \textstyle
    (u, v) \setminus \bigcup_{i=1}^k [a_i, b_i] = \bigcup_{j=1}^{m_k} (u_{k,j}, v_{k,j}).
  \end{equation*}
  %
  Start with $m_0 \defeq 1$ and $(u_{0,1}, v_{0,1}) = (u, v)$.
  To progress to $(k+1)$-th stage, replace $(u_{k,j}, v_{k,j})$ with the difference $(u_{k,j}, v_{k,j}) \setminus [a_{k+1}, b_{k+1}]$, which is a union of zero, one, or two disjoint open intervals.
  %
  Also note that the total length decreases by at most $b_{k+1} - a_{k+1}$.
  %
  In the end we may take the midpoint of $(u_{n,1}, v_{n,1})$ to be the desired~$x$.

  When the endpoints are real numbers we may slightly enlarge each $[a_i, b_i]$ to an interval with rational endpoints\footnote{Doing so requires only finitely many choices, which can be carried out by induction, without appealing to the axiom of choice.} and slightly shrink $(u,v)$ to an interval with rational endpoints, while preserving
  $\sum_{i=1}^n b_i - a_i < v - u$.
\end{proof}

The situation is reminiscent of the effective topos~\cite[Sect.~6.4.2]{troelstra88:_const_mathem}, except that there one has to work harder to construct a singular cover because the closed unit interval is not countable. Also note that the sum of lengths of the intervals constructed in \cref{thm:singular-cover} is exactly~$\epsilon$, whereas in the effective topos the sum fails to converge, but its partial sums are bounded by~$\epsilon$.

This however is not all that can be said about the Heine-Borel compactness of the closed unit interval in~$\TT{\mil}$.
%
Observe that the singular cover constructed above consists of intervals whose endpoints are real numbers.
%
Can we also construct one whose endpoints are rational? Surprisingly, no.
%
To see why this is the case we need a bit of preparation.

To lay the groundwork for the proof of \cref{lem:fix-point-free-map}, we explain how to constructively extend certain maps to larger domains.
%
Given $f : [a,b] \to \RR$ and $g : [b, c] \to \RR$ such that $f(b) = g(b)$, there is a map $h : [a,c] \to \RR$ that extends them, namely
%
\begin{equation*}
  h(x) \defeq f(\min(x, b)) + g(\max(x, b)) - f(b).
\end{equation*}
%
The construction can be iterated to give an extension of any finite number of matching maps defined on abutting closed intervals.

Second, consider a solid rectangle~$ABCD$ and points $A'$, $D'$, $E$, as shown in \cref{fig:rectangle}.
%
\begin{figure}[ht]
  \centering
  \begin{tikzpicture}[scale=0.85]
    \fill[color=white!90!black] rectangle (2,3) ;
    \draw (2,0) -- (2,3) ;
    \draw[thick]
       (2,3) node[anchor=west] {$D$} --
       (0,3) node[anchor=east] {$C$} --
       (0,0) node[anchor=east] {$B$} --
       (2,0) node[anchor=west] {$A$} ;
    \draw (0, -2) node[anchor=east] {$A'$} -- (0,0) ; \fill (0, -2) circle (0.05) ;
    \draw (0, 5) node[anchor=east] {$D'$} -- (0,3) ; \fill (0, 5) circle (0.05) ;
    \fill (3.5, 1.5) node[anchor=north west] {$E$} circle (0.05) ;
    \fill (1.25, 1.95) circle (0.05) node[anchor=north west] {$P$} ;
    \fill (0, 2.2) circle (0.05) node[anchor=east] {$r(P)$} ;
    \draw (3.5, 1.5) -- (0, 2.2) ;
    \draw[thin,dashed] (3.5, 1.5) -- (0, 5) ;
    \draw[thin,dashed] (3.5, 1.5) -- (0, -2) ;
  \end{tikzpicture}
  \caption{Extending maps from three sides to the rectangle}
  \label{fig:rectangle}
\end{figure}
%
Define $r : ABCD \to A'D'$ by mapping any point~$P$ to the intersection of $A'D'$ and the straight line through~$E$ and~$P$.
%
Now given maps $f : AB \to \RR$, $g : BC \to \RR$ and $h : CD \to \RR$ satisfying $f(B) = g(B)$ and $g(C) = h(C)$,
we may construct an extension $j : ABCD \to \RR$: transfer the maps~$f$ and~$h$ along congruences $AB \cong A'B$ and $CD \cong CD'$ to maps $f' : A'B \to \RR$ and $h' : CD' \to \RR$, let $i : A'D' \to \RR$ be an extension of $f$, $g'$ and~$h'$ obtained by an application of the previously described technique, and define $j \defeq i \circ r$.

The following lemma is an adaptation of a construction going back to~\cite{orevkov63}, see also \cite[Thm.~IV.10.1]{beeson85:_found_const_mathem}.

\begin{lemmaC}
  \label{lem:fix-point-free-map}%
  Suppose $(a_0, b_0), (a_1, b_1), (a_2, b_2), \ldots$ are open intervals with rational endpoints.
  %
  There is a continuous map $h : ([0,1] \cap \bigcup_{i \in \NN} (a_i, b_i))^2 \to [0,1]^2$ such that, $h$ has a fixed point if, and only if, there is $n \in \NN$ such that  $[0,1] \subseteq (a_0, b_0) \cup \cdots \cup (a_n, b_n)$.
\end{lemmaC}

\begin{proof}
  %
  All intervals considered in the proof have rational endpoints, so we just call them ``intervals''.
  %
  Throughout, we shall depend on decidability of the linear order on~$\QQ$, for example to test inclusion of one interval in another, or to tell whether a finite sequence of intervals with rational endpoints covers~$[0,1]$.

  We first consider the situation when the intervals $(a_i, b_i)$ are well-behaved in the following sense:
  %
  \begin{itemize}
  \item no interval shares an endpoint with $(0,1)$: $a_i, b_i \not\in \set{0,1}$ for all~$i$,
  \item there are no abutting intervals: $b_i \neq a_j$ for all $i, j$, and
  \item no interval is contained in another: $(a_i, b_i) \not\subseteq (a_j, b_j)$ for all $i \neq j$.
  \end{itemize}
  %
  Under these circumstances $(a_i, b_i)$ and $(a_j, b_j)$ are either disjoint with a positive distance between them, or they partially overlap on an open interval. Also, an interval $(a_i, b_i)$ overlaps with at most two other intervals.

  Define $V_k \defeq [0,1] \cap \bigcup_{i = 0}^k [a_i, b_i]$, and let $\partial[0,1]^2$ be the boundary of the unit square.\footnote{More precisely, $\partial[0,1]^2$ is the topological border of $[0,1]^2$ qua subset of the plane -- which need not coincide with the union of the four sides of the square.}
  %
  We construct maps
  %
  \begin{equation*}
    f_k : \partial[0,1]^2 \cup V_k^2 \to [0,1]^2
  \end{equation*}
  %
  so that each $f_k$ extends $f_{k-1}$. In addition, we ensure that if $[0,1] \neq V_k$ then~$f_k$ does not have a fixed point and its image is contained in $\partial[0,1]^2$.

  Let $f_{-1} : \partial[0,1]^2 \to \partial[0,1]^2$ be the rotation of~$\partial[0,1]^2$ by a right angle. For each $k \in \NN$, construct $f_k$ from~$f_{k-1}$ as follows:
  %
  \begin{enumerate}
  \item
    If $V_{k-1} = V_k$ then $f_k \defeq f_{k-1}$.
  \item
    %
    If $V_{k-1} \neq V_k \neq [0,1] $ then $V_k$ is~$V_{k-1}$ properly extended by $[a_k, b_k]$. The left-hand side of \cref{fig:fp-free} depicts a typical situation, where the light gray region is~$V_{k-1}$ and the dark gray the area newly contributed by~$[a_k, b_k]$.
    %
    (The assumption that the intervals are well-behaved makes sure that the dark gray strips have positive width and heights, and that at most one gap is filled at a time.)
    %
\begin{figure}[tp]
  \centering
  \begin{tikzpicture}[baseline=(current bounding box.center),scale=4]
    % Unit square
    \draw[very thick, color=white!70!black] (0,0) rectangle +(1,1) ;
    % Intervals: [0.1, 0.2], [0.3, 0.4], [0.5, 0.7], [0.8, 1.0]
    % column #1
    \fill[fill=white!70!black] (0.1, 0.1) rectangle +(0.1,0.1) ;
    \fill[fill=white!70!black] (0.1, 0.3) rectangle +(0.1,0.1) ;
    \fill[fill=white!70!black] (0.1, 0.5) rectangle +(0.1,0.2) ;
    \fill[fill=white!70!black] (0.1, 0.8) rectangle +(0.1,0.2) ;
    % column #2
    \fill[fill=white!70!black] (0.3, 0.1) rectangle +(0.1,0.1) ;
    \fill[fill=white!70!black] (0.3, 0.3) rectangle +(0.1,0.1) ;
    \fill[fill=white!70!black] (0.3, 0.5) rectangle +(0.1,0.2) ;
    \fill[fill=white!70!black] (0.3, 0.8) rectangle +(0.1,0.2) ;
    % column #3
    \fill[fill=white!70!black] (0.5, 0.1) rectangle +(0.2,0.1) ;
    \fill[fill=white!70!black] (0.5, 0.3) rectangle +(0.2,0.1) ;
    \fill[fill=white!70!black] (0.5, 0.5) rectangle +(0.2,0.2) ;
    \fill[fill=white!70!black] (0.5, 0.8) rectangle +(0.2,0.2) ;
    % column #4
    \fill[fill=white!70!black] (0.8, 0.1) rectangle +(0.2,0.1) ;
    \fill[fill=white!70!black] (0.8, 0.3) rectangle +(0.2,0.1) ;
    \fill[fill=white!70!black] (0.8, 0.5) rectangle +(0.2,0.2) ;
    \fill[fill=white!70!black] (0.8, 0.8) rectangle +(0.2,0.2) ;
    % new interval [0.4,0.5]
    \fill[fill=white!40!black] (0.4, 0.1) rectangle +(0.1,0.1) ;
    \fill[fill=white!40!black] (0.4, 0.3) rectangle +(0.1,0.4) ;
    \fill[fill=white!40!black] (0.4, 0.5) rectangle +(0.1,0.2) ;
    \fill[fill=white!40!black] (0.4, 0.8) rectangle +(0.1,0.2) ;
    \fill[fill=white!40!black] (0.1, 0.4) rectangle +(0.1,0.1) ;
    \fill[fill=white!40!black] (0.3, 0.4) rectangle +(0.4,0.1) ;
    \fill[fill=white!40!black] (0.5, 0.4) rectangle +(0.2,0.1) ;
    \fill[fill=white!40!black] (0.8, 0.4) rectangle +(0.2,0.1) ;
  \end{tikzpicture}
  \hfil
  \begin{tikzpicture}[baseline=(current bounding box.center),scale=8]
    % Outer square
    \fill [fill=white!70!black] (0.3,0.3) rectangle +(0.1,0.1) ;
    \fill [fill=white!70!black] (0.3,0.5) rectangle +(0.1,0.2) ;
    \fill [fill=white!70!black] (0.5,0.5) rectangle +(0.2,0.2) ;
    \fill [fill=white!70!black] (0.5,0.3) rectangle +(0.2,0.1) ;
    \fill[fill=white!40!black] (0.4, 0.3) rectangle +(0.1,0.4) ;
    \fill[fill=white!40!black] (0.4, 0.5) rectangle +(0.1,0.2) ;
    \fill[fill=white!40!black] (0.3, 0.4) rectangle +(0.4,0.1) ;
    \fill[fill=white!40!black] (0.5, 0.4) rectangle +(0.2,0.1) ;
    \draw [thick, white!40!black, text=black]
        (0.7,0.5) -- (0.7,0.7) -- (0.5,0.7) -- (0.45,0.7) node[anchor=south] {$s_3$} -- (0.4,0.7) -- (0.3,0.7) --
        (0.3, 0.5) -- (0.3, 0.45) node[anchor=east]  {$s_2$} --
        (0.3,0.3) -- (0.45,0.3) node[anchor=north] {$s_1$} -- (0.7,0.3) -- (0.7, 0.4) ;
    \draw [very thick, white!40!black, text=black, dashed] (0.7,0.4) -- (0.7,0.45) node[anchor=west]  {$s_4$} -- (0.7,0.5) ;
  \end{tikzpicture}
  \caption{A step in the construction of~$f$ from \cref{lem:fix-point-free-map}}
  \label{fig:fp-free}
\end{figure}
    %
    We obtain~$f_k$ by extending~$f_{k-1}$ to the dark gray area separately on each rectangular component. For example, consider the central component, shown separately on the right-hand side of the figure. Because $V_k \neq [0,1]^2$ at least one of the line segments $s_1, \ldots, s_4$ is not contained in~$V_{k-1}$, say~$s_4$. Using the techniques described above, first extend $f_{k-1}$ to the other line segments, in our case $s_1, s_2, s_3$, all the while making sure that its image is contained in $\partial[0,1]^2$, and then to the entire component.
    %
    The reader may verify easily that the same approach works in other cases.
  \item
    If $V_{k-1} \neq V_k = [0,1]$ then~$[a_k, b_k]$ fills in the last gap in~$[0,1]$.
    %
    We visualize the situation by re-interpreting the right-hand side of the figure as showing~$[0,1]^2$, where
    light gray is~$V_k$ and dark gray the newly contributed area, except that this time all four segments $s_1, \ldots, s_4$ are already in the domain of~$f_{k-1}$. Pick a point in the interior of the dark gray area and declare it to be a fixed point of~$f_k$, then extend $f_k$ to the rest of the square in a piece-wise linear fashion, using the entire square as the codomain of~$f_k$.
    %
    The reader may verify that the same approach works when the dark gray area is adjacent to~$\partial[0,1]^2$, in which case it is shaped like the letter~L.
  \end{enumerate}
  %
  Notice that $V_k \neq [0,1]$ implies that~$f_k$ has no fixed points. Indeed, if $t = f_k(t)$ then $t \in \partial[0,1]^2$, which would make~$t$ a fixed point of~$f_{-1}$.

  Let $h$ be the union of $f_k$'s, restricted to $([0,1]^2 \cap \bigcup_{i \in \NN} (a_i, b_i))^2$.
  %
  We must verify that~$h$ has the required property.
  %
  If $[0,1] \subseteq (a_0, b_0) \cup \cdots \cup (a_n, b_n)$ for some $n \in \NN$, then $V_n = [0,1]$, so~$h$ has a fixed point by construction of~$f_n$.
  %
  Conversely, if~$t$ is a fixed point of~$h$ then $t \in ([0,1] \cap (a_n, b_n))^2$ for some~$n \in \NN$, hence~$f_n$ has a fixed point, which is only possible if $V_n = [0,1]$, but then $[0,1] \subseteq (a_0, b_0) \cup \cdots \cup (a_n, b_n)$ because the intervals are well-behaved.


  It remains to remove the requirement that the intervals be well-behaved.
  %
  Given any sequence of intervals $(a_0, b_0), (a_1, b_1), \ldots$, we define a new well-behaved sequence with the same union, which has a finite subcover of~$[0,1]$ if, and only if, $(a_0, b_0), (a_1, b_1), \ldots$ does.
  %
  We may then apply the above construction to the new sequence.

  Let $p_i$ be the $i$-th prime, and $P_i \defeq p_1 \cdots p_i$ the product of the first~$i$ primes.
  %
  For $i \in \NN$ and $m \in \ZZ$ let
  %
  \begin{equation*}
    \textstyle
    c_{i,m} \defeq \frac{1 + 2 m \cdot p_i}{P_i}
    \qquad\text{and}\qquad
    d_{i,m} \defeq \frac{1 + (2 m + 3) \cdot p_i}{P_{i+1}}.
  \end{equation*}
  %
  % Verification of claims:
  %
  % * the equation (1 + k · p_i)/P_i = 0 has no solutions: obvious
  % * the equation (1 + k · p_i)/P_i = 1 has no solutions: it is equivalent to 1 = p_i · (P_{i-1} - k)
  % * the equation (1 + m · p_i)/P_i = (1 + n · p_j)/P_j has no solutions:
  %   wlog assume i < j and observe that the equation is equivalent to
  %      (1 + m · p_i) P_j = (1 + n · p_j) · P_i
  %      (1 + m · p_i) p_{i+1} ⋯ p_j = 1 + n · p_j
  %      (1 + m · p_i) p_{i+1} ⋯ p_j - n · p_j = 1
  %   LHS is divisible by p_j and RHS is not.
  % From the above it follows that the intervals (c_{i,m}, d_{i,m}) have no common endpoints and the endpoints
  % are never 0 or 1.
  %
  % Verification that (c_{i,m}, d_{i,m}) are well-behaved: 
  %   c_i,m                 d_i,m         c_i,m+2                 d_i,m+2
  %   (-----------------------)            (-----------------------)
  %                     (-----------------------)
  %                    c_i,m+1                 d_i,m+1
  %
  % * c_i,m+1 < d_i,m: reduces to 2 (m + 1) < 2m + 3
  % * d_i,m < c_i,m+2: reduces to 2 m + 3 < 2 (m + 2)
  % * c_i,m+2 < d_i,m+1: reduces to 2 (m + 2) < 2 (m + 1) + 3
  %
  No two intervals $(c_{i,m}, d_{i,m})$ and $(c_{j,n}, d_{j,n})$ share an endpoint, and their endpoints are all different from $0$ and $1$. Also, for a fixed~$i$ the intervals $(c_{i,m}, d_{i,m})$ form a well-behaved cover of~$\RR$.

  We enumerate some of the intervals $(c_{i,m}, d_{i,m})$ in phases, each phase contributing finitely many intervals. In the $i$-th phase we include those $(c_{i,m}, d_{i,m})$ that are contained in $(a_0, b_0) \cup \cdots \cup (a_i, b_i)$ but are not contained in any of the intervals enumerated so far.
  %
  This way we obtain a well-behaved sequence, as the construction of $c_{i,m}$ and $d_{i,m}$ guarantees that intervals are not abutting and that their endpoints avoid $0$ and $1$; and an interval cannot contain another from the same stage, nor from an earlier one as it is too narrow.

  Obviously, the newly enumerated intervals cover at most $\bigcup_{k \in \NN} (a_k, b_k)$. They cover all of it, because any $x \in (a_k, b_k)$ is covered at the latest by the stage at which the widths of $(c_{i,m}, d_{i,m})$'s are smaller than the distance of~$x$ to the endpoints $a_k$ and $b_k$.
  %
  Finally, if $(a_0, b_0) \cup \cdots \cup (a_k, b_k)$ cover $[0,1]$, then they do so with a bit of overlap. There is a stage~$i$ such that the widths of $(c_{i,m}, d_{i,m})$'s are smaller than the overlap, so $[0,1]$ will be covered at least by the $i$-th stage.
\end{proof}

When the previous lemma is combined with Brouwer's fixed point theorem, a variant of Heine-Borel compactness of~$[0,1]$ emerges.

\begin{corollary}
  In the topos~$\TT{\mil}$, a countable cover of the closed unit interval by open intervals with rational endpoints has a finite subcover.
\end{corollary}

\begin{proof}
  Let $h : \objI^2 \to \objI^2$ be the map from \cref{lem:fix-point-free-map} for the given cover of~$\objI$.
  %
  By \cref{thm:internal-brouwer} it has a fixed point, therefore \cref{lem:fix-point-free-map} ensures that~$\objI$ is covered already by a finite subcover.
\end{proof}

We can improve on the corollary to give a variant of the Drinker paradox\footnote{The ``paradox'' states that in every non-empty pub there is a person, such that if the person is drinking then everyone is drinking. It is a non-constructive principle~\cite{warren2018drinker}.} for sufficiently tame predicates on the closed unit interval.

\begin{proposition}[Drinking Buddies Principle]
  \label{prop:drinking-buddies}%
  In the topos~$\TT{\mil}$, suppose $U$ is a countable union of open intervals with rational endpoints.
  There are $x, y \in \objI$ such that $x \in U \land y \in U$ if, and only if, $\all{z \in \objI} z \in U$.
\end{proposition}

\begin{proof}
  Let $(a_0, b_0), (a_1, b_1), \ldots$ be a sequence of intervals with rational endpoints and $U \defeq \bigcup_{i \in \NN} (a_i, b_i)$ and $h : (\objI \cap U)^2 \to \objI^2$ the map from \cref{lem:fix-point-free-map} for the given intervals. We claim that $U$ is a $\neg\neg$-stable subset of~$\RR$. One way to see this is to recall from \cref{lem:lt-stable} that~$<$ is $\neg\neg$-stable, and observe that there is a map $t : \RR \to \RR$ such that $U = \set{x \in \RR \such t(x) > 0}$, for instance a weighted sum of “tent maps” errected on the intervals,
  %
  \begin{equation*}
    \textstyle
    t(x) = \sum_{n \in \NN} 2^{-n} \cdot \max (0, 1 - |\max (a_n, \min (b_n, x)) - (a_n + b_n)/2|).
  \end{equation*}
  %
  Therefore, \cref{thm:partial-brouwer} applies to~$h$ to give $(x, y) \in \objI^2$ such that if $x, y \in U$ then $h(x,y) = (x,y)$, whence $\objI \subseteq U$ by \cref{lem:fix-point-free-map}.
\end{proof}

We are not certain what the principle is good for, apart from obliging one to test its veracity with a buddy in a pub.

%%% Local Variables:
%%% mode: latex
%%% TeX-master: "countable-reals"
%%% End:


\subsubsection*{Acknowledgment}

We thank Ingo Blechschmidt, Joseph Miller, and Andrew Swan for valuable suggestions and engaging discussions.


% \input{extras.tex}

\bibliographystyle{plain}
\bibliography{references}

\appendix
%\input{old-sequences.tex}


\end{document}