\documentclass[11pt]{amsart}

% For e-reader, small margins
%\usepackage[margin=5mm]{geometry}

\usepackage[T1]{fontenc}
\usepackage[utf8]{inputenc}

\usepackage{tikz}

% Fancy math fonts
\usepackage{amsmath}
\usepackage{amsfonts}
\usepackage{amssymb}
\usepackage{upgreek}
\usepackage{xfrac} % for neat fraction \sfrac

\usepackage{amsaddr} % place author info on the first page

% I am not sure about all these fonts (Andrej)
\usepackage[scaled=.97,helvratio=.93,p,theoremfont]{newpxtext} % Serif palatino font
\usepackage[vvarbb,smallerops,bigdelims]{newpxmath} % Math palatino font
\usepackage[scaled=.85]{beramono} % Monospace font
\usepackage[scr=rsfso,cal=boondoxo]{mathalfa} % Mathcal from STIX, unslanted a bit

% A simpler choice of fonts:
% \usepackage{times}

\usepackage[english]{babel}

\usepackage{amsthm}
\usepackage[hidelinks]{hyperref}
\usepackage[capitalise,nameinlink]{cleveref}
\usepackage{xypic}
\usepackage{xcolor}
\usepackage[shortlabels]{enumitem}

%%% Theorem-like environments
{\theoremstyle{plain}
\newtheorem{theorem}{Theorem}[section]
\newtheorem{theoremC}[theorem]{Theorem${}^\star$}
\newtheorem{proposition}[theorem]{Proposition}
\newtheorem{propositionC}[theorem]{Proposition${}^\star$}
\newtheorem{corollary}[theorem]{Corollary}
\newtheorem{corollaryC}[theorem]{Corollary${}^\star$}
\newtheorem{lemma}[theorem]{Lemma}
\newtheorem{lemmaC}[theorem]{Lemma${}^\star$}
\newtheorem{fact}[theorem]{Fact}
}

{\theoremstyle{definition}
\newtheorem{definition}[theorem]{Definition}
\newtheorem{example}[theorem]{Example}
}

\crefname{theoremC}{Theorem}{Theorems}
\crefname{propositionC}{Proposition}{Propositions}
\crefname{corollaryC}{Corollary}{Corollaries}
\crefname{lemmaC}{Lemma}{Lemmas}


% Equations numbered with sections
\numberwithin{equation}{section}

%%%% MACROS FOR NOTATION %%%%
% Use these for any notation where there are multiple options.

%%% Notes and exercise sections
\makeatletter
\newcommand{\sectionNotes}{\phantomsection\section*{Notes}\addcontentsline{toc}{section}{Notes}\markright{\textsc{\@chapapp{} \thechapter{} Notes}}}
\newcommand{\sectionExercises}[1]{\ifdef{\OPTexerciseperpage}{\newpage}{}\phantomsection\section*{Exercises}\addcontentsline{toc}{section}{Exercises}\markright{\textsc{\@chapapp{} \thechapter{} Exercises}}}
\makeatother

%%% Definitional equality (used infix) %%%
\newcommand{\jdeq}{\equiv}      % An equality judgment
\let\judgeq\jdeq
%\newcommand{\defeq}{\coloneqq}  % An equality currently being defined
\newcommand{\defeq}{\vcentcolon\equiv}  % A judgmental equality currently being defined

%%% Term being defined
\newcommand{\define}[1]{\textbf{#1}}

%%% Vec (for example)

\newcommand{\Vect}{\ensuremath{\mathsf{Vec}}}
\newcommand{\Fin}{\ensuremath{\mathsf{Fin}}}
\newcommand{\fmax}{\ensuremath{\mathsf{fmax}}}
\newcommand{\seq}[1]{\langle #1\rangle}

%%% Dependent products %%%
\def\prdsym{\textstyle\prod}
%% Call the macro like \prd{x,y:A}{p:x=y} with any number of
%% arguments.  Make sure that whatever comes *after* the call doesn't
%% begin with an open-brace, or it will be parsed as another argument.
\makeatletter
% Currently the macro is configured to produce
%     {\textstyle\prod}(x:A) \; {\textstyle\prod}(y:B),{\ }
% in display-math mode, and
%     \prod_{(x:A)} \prod_{y:B}
% in text-math mode.
% \def\prd#1{\@ifnextchar\bgroup{\prd@parens{#1}}{%
%     \@ifnextchar\sm{\prd@parens{#1}\@eatsm}{%
%         \prd@noparens{#1}}}}
\def\prd#1{\@ifnextchar\bgroup{\prd@parens{#1}}{%
    \@ifnextchar\sm{\prd@parens{#1}\@eatsm}{%
    \@ifnextchar\prd{\prd@parens{#1}\@eatprd}{%
    \@ifnextchar\;{\prd@parens{#1}\@eatsemicolonspace}{%
    \@ifnextchar\\{\prd@parens{#1}\@eatlinebreak}{%
    \@ifnextchar\narrowbreak{\prd@parens{#1}\@eatnarrowbreak}{%
      \prd@noparens{#1}}}}}}}}
\def\prd@parens#1{\@ifnextchar\bgroup%
  {\mathchoice{\@dprd{#1}}{\@tprd{#1}}{\@tprd{#1}}{\@tprd{#1}}\prd@parens}%
  {\@ifnextchar\sm%
    {\mathchoice{\@dprd{#1}}{\@tprd{#1}}{\@tprd{#1}}{\@tprd{#1}}\@eatsm}%
    {\mathchoice{\@dprd{#1}}{\@tprd{#1}}{\@tprd{#1}}{\@tprd{#1}}}}}
\def\@eatsm\sm{\sm@parens}
\def\prd@noparens#1{\mathchoice{\@dprd@noparens{#1}}{\@tprd{#1}}{\@tprd{#1}}{\@tprd{#1}}}
% Helper macros for three styles
\def\lprd#1{\@ifnextchar\bgroup{\@lprd{#1}\lprd}{\@@lprd{#1}}}
\def\@lprd#1{\mathchoice{{\textstyle\prod}}{\prod}{\prod}{\prod}({\textstyle #1})\;}
\def\@@lprd#1{\mathchoice{{\textstyle\prod}}{\prod}{\prod}{\prod}({\textstyle #1}),\ }
\def\tprd#1{\@tprd{#1}\@ifnextchar\bgroup{\tprd}{}}
\def\@tprd#1{\mathchoice{{\textstyle\prod_{(#1)}}}{\prod_{(#1)}}{\prod_{(#1)}}{\prod_{(#1)}}}
\def\dprd#1{\@dprd{#1}\@ifnextchar\bgroup{\dprd}{}}
\def\@dprd#1{\prod_{(#1)}\,}
\def\@dprd@noparens#1{\prod_{#1}\,}

% Look through spaces and linebreaks
\def\@eatnarrowbreak\narrowbreak{%
  \@ifnextchar\prd{\narrowbreak\@eatprd}{%
    \@ifnextchar\sm{\narrowbreak\@eatsm}{%
      \narrowbreak}}}
\def\@eatlinebreak\\{%
  \@ifnextchar\prd{\\\@eatprd}{%
    \@ifnextchar\sm{\\\@eatsm}{%
      \\}}}
\def\@eatsemicolonspace\;{%
  \@ifnextchar\prd{\;\@eatprd}{%
    \@ifnextchar\sm{\;\@eatsm}{%
      \;}}}

%%% Lambda abstractions.
% Each variable being abstracted over is a separate argument.  If
% there is more than one such argument, they *must* be enclosed in
% braces.  Arguments can be untyped, as in \lam{x}{y}, or typed with a
% colon, as in \lam{x:A}{y:B}. In the latter case, the colons are
% automatically noticed and (with current implementation) the space
% around the colon is reduced.  You can even give more than one variable
% the same type, as in \lam{x,y:A}.
\def\lam#1{{\lambda}\@lamarg#1:\@endlamarg\@ifnextchar\bgroup{.\,\lam}{.\,}}
\def\@lamarg#1:#2\@endlamarg{\if\relax\detokenize{#2}\relax #1\else\@lamvar{\@lameatcolon#2},#1\@endlamvar\fi}
\def\@lamvar#1,#2\@endlamvar{(#2\,{:}\,#1)}
% \def\@lamvar#1,#2{{#2}^{#1}\@ifnextchar,{.\,{\lambda}\@lamvar{#1}}{\let\@endlamvar\relax}}
\def\@lameatcolon#1:{#1}
\let\lamt\lam
% This version silently eats any typing annotation.
\def\lamu#1{{\lambda}\@lamuarg#1:\@endlamuarg\@ifnextchar\bgroup{.\,\lamu}{.\,}}
\def\@lamuarg#1:#2\@endlamuarg{#1}

%%% Dependent products written with \forall, in the same style
\def\fall#1{\forall (#1)\@ifnextchar\bgroup{.\,\fall}{.\,}}

%%% Existential quantifier %%%
\def\exis#1{\exists (#1)\@ifnextchar\bgroup{.\,\exis}{.\,}}

%%% Dependent sums %%%
\def\smsym{\textstyle\sum}
% Use in the same way as \prd
\def\sm#1{\@ifnextchar\bgroup{\sm@parens{#1}}{%
    \@ifnextchar\prd{\sm@parens{#1}\@eatprd}{%
    \@ifnextchar\sm{\sm@parens{#1}\@eatsm}{%
    \@ifnextchar\;{\sm@parens{#1}\@eatsemicolonspace}{%
    \@ifnextchar\\{\sm@parens{#1}\@eatlinebreak}{%
    \@ifnextchar\narrowbreak{\sm@parens{#1}\@eatnarrowbreak}{%
        \sm@noparens{#1}}}}}}}}
\def\sm@parens#1{\@ifnextchar\bgroup%
  {\mathchoice{\@dsm{#1}}{\@tsm{#1}}{\@tsm{#1}}{\@tsm{#1}}\sm@parens}%
  {\@ifnextchar\prd%
    {\mathchoice{\@dsm{#1}}{\@tsm{#1}}{\@tsm{#1}}{\@tsm{#1}}\@eatprd}%
    {\mathchoice{\@dsm{#1}}{\@tsm{#1}}{\@tsm{#1}}{\@tsm{#1}}}}}
\def\@eatprd\prd{\prd@parens}
\def\sm@noparens#1{\mathchoice{\@dsm@noparens{#1}}{\@tsm{#1}}{\@tsm{#1}}{\@tsm{#1}}}
\def\lsm#1{\@ifnextchar\bgroup{\@lsm{#1}\lsm}{\@@lsm{#1}}}
\def\@lsm#1{\mathchoice{{\textstyle\sum}}{\sum}{\sum}{\sum}({\textstyle #1})\;}
\def\@@lsm#1{\mathchoice{{\textstyle\sum}}{\sum}{\sum}{\sum}({\textstyle #1}),\ }
\def\tsm#1{\@tsm{#1}\@ifnextchar\bgroup{\tsm}{}}
\def\@tsm#1{\mathchoice{{\textstyle\sum_{(#1)}}}{\sum_{(#1)}}{\sum_{(#1)}}{\sum_{(#1)}}}
\def\dsm#1{\@dsm{#1}\@ifnextchar\bgroup{\dsm}{}}
\def\@dsm#1{\sum_{(#1)}\,}
\def\@dsm@noparens#1{\sum_{#1}\,}

%%% W-types
\def\wtypesym{{\mathsf{W}}}
\def\wtype#1{\@ifnextchar\bgroup%
  {\mathchoice{\@twtype{#1}}{\@twtype{#1}}{\@twtype{#1}}{\@twtype{#1}}\wtype}%
  {\mathchoice{\@twtype{#1}}{\@twtype{#1}}{\@twtype{#1}}{\@twtype{#1}}}}
\def\lwtype#1{\@ifnextchar\bgroup{\@lwtype{#1}\lwtype}{\@@lwtype{#1}}}
\def\@lwtype#1{\mathchoice{{\textstyle\mathsf{W}}}{\mathsf{W}}{\mathsf{W}}{\mathsf{W}}({\textstyle #1})\;}
\def\@@lwtype#1{\mathchoice{{\textstyle\mathsf{W}}}{\mathsf{W}}{\mathsf{W}}{\mathsf{W}}({\textstyle #1}),\ }
\def\twtype#1{\@twtype{#1}\@ifnextchar\bgroup{\twtype}{}}
\def\@twtype#1{\mathchoice{{\textstyle\mathsf{W}_{(#1)}}}{\mathsf{W}_{(#1)}}{\mathsf{W}_{(#1)}}{\mathsf{W}_{(#1)}}}
\def\dwtype#1{\@dwtype{#1}\@ifnextchar\bgroup{\dwtype}{}}
\def\@dwtype#1{\mathsf{W}_{(#1)}\,}

\newcommand{\suppsym}{{\mathsf{sup}}}
\newcommand{\supp}{\ensuremath\suppsym\xspace}

\def\wtypeh#1{\@ifnextchar\bgroup%
  {\mathchoice{\@lwtypeh{#1}}{\@twtypeh{#1}}{\@twtypeh{#1}}{\@twtypeh{#1}}\wtypeh}%
  {\mathchoice{\@@lwtypeh{#1}}{\@twtypeh{#1}}{\@twtypeh{#1}}{\@twtypeh{#1}}}}
\def\lwtypeh#1{\@ifnextchar\bgroup{\@lwtypeh{#1}\lwtypeh}{\@@lwtypeh{#1}}}
\def\@lwtypeh#1{\mathchoice{{\textstyle\mathsf{W}^h}}{\mathsf{W}^h}{\mathsf{W}^h}{\mathsf{W}^h}({\textstyle #1})\;}
\def\@@lwtypeh#1{\mathchoice{{\textstyle\mathsf{W}^h}}{\mathsf{W}^h}{\mathsf{W}^h}{\mathsf{W}^h}({\textstyle #1}),\ }
\def\twtypeh#1{\@twtypeh{#1}\@ifnextchar\bgroup{\twtypeh}{}}
\def\@twtypeh#1{\mathchoice{{\textstyle\mathsf{W}^h_{(#1)}}}{\mathsf{W}^h_{(#1)}}{\mathsf{W}^h_{(#1)}}{\mathsf{W}^h_{(#1)}}}
\def\dwtypeh#1{\@dwtypeh{#1}\@ifnextchar\bgroup{\dwtypeh}{}}
\def\@dwtypeh#1{\mathsf{W}^h_{(#1)}\,}


\makeatother

% Other notations related to dependent sums
\let\setof\Set    % from package 'braket', write \setof{ x:A | P(x) }.
\newcommand{\pair}{\ensuremath{\mathsf{pair}}\xspace}
\newcommand{\tup}[2]{(#1,#2)}
\newcommand{\proj}[1]{\ensuremath{\mathsf{pr}_{#1}}\xspace}
\newcommand{\fst}{\ensuremath{\proj1}\xspace}
\newcommand{\snd}{\ensuremath{\proj2}\xspace}
\newcommand{\ac}{\ensuremath{\mathsf{ac}}\xspace} % not needed in symbol index

%%% recursor and induction
\newcommand{\rec}[1]{\mathsf{rec}_{#1}}
\newcommand{\ind}[1]{\mathsf{ind}_{#1}}
\newcommand{\indid}[1]{\ind{=_{#1}}} % (Martin-Lof) path induction principle for identity types
\newcommand{\indidb}[1]{\ind{=_{#1}}'} % (Paulin-Mohring) based path induction principle for identity types

%%% Uniqueness principles
\newcommand{\uniq}[1]{\mathsf{uniq}_{#1}}

% Paths in pairs
\newcommand{\pairpath}{\ensuremath{\mathsf{pair}^{\mathord{=}}}\xspace}
% \newcommand{\projpath}[1]{\proj{#1}^{\mathord{=}}}
\newcommand{\projpath}[1]{\ensuremath{\apfunc{\proj{#1}}}\xspace}
\newcommand{\pairct}{\ensuremath{\mathsf{pair}^{\mathord{\ct}}}\xspace}

%%% For quotients %%%
%\newcommand{\pairr}[1]{{\langle #1\rangle}}
\newcommand{\pairr}[1]{{\mathopen{}(#1)\mathclose{}}}
\newcommand{\Pairr}[1]{{\mathopen{}\left(#1\right)\mathclose{}}}

% \newcommand{\type}{\ensuremath{\mathsf{Type}}} % this command is overridden below, so it's commented out
\newcommand{\im}{\ensuremath{\mathsf{im}}} % the image

%%% 2D path operations
\newcommand{\leftwhisker}{\mathbin{{\ct}_{\mathsf{l}}}}  % was \ell
\newcommand{\rightwhisker}{\mathbin{{\ct}_{\mathsf{r}}}} % was r
\newcommand{\hct}{\star}

%%% modalities %%%
\newcommand{\modal}{\ensuremath{\ocircle}}
\let\reflect\modal
\newcommand{\modaltype}{\ensuremath{\type_\modal}}
% \newcommand{\ism}[1]{\ensuremath{\mathsf{is}_{#1}}}
% \newcommand{\ismodal}{\ism{\modal}}
% \newcommand{\existsmodal}{\ensuremath{{\exists}_{\modal}}}
% \newcommand{\existsmodalunique}{\ensuremath{{\exists!}_{\modal}}}
% \newcommand{\modalfunc}{\textsf{\modal-fun}}
% \newcommand{\Ecirc}{\ensuremath{\mathsf{E}_\modal}}
% \newcommand{\Mcirc}{\ensuremath{\mathsf{M}_\modal}}
\newcommand{\mreturn}{\ensuremath{\eta}}
\let\project\mreturn
%\newcommand{\mbind}[1]{\ensuremath{\hat{#1}}}
\newcommand{\ext}{\mathsf{ext}}
%\newcommand{\mmap}[1]{\ensuremath{\bar{#1}}}
%\newcommand{\mjoin}{\ensuremath{\mreturn^{-1}}}
% Subuniverse
\renewcommand{\P}{\ensuremath{\type_{P}}\xspace}

%%% Localizations
% \newcommand{\islocal}[1]{\ensuremath{\mathsf{islocal}_{#1}}\xspace}
% \newcommand{\loc}[1]{\ensuremath{\mathcal{L}_{#1}}\xspace}

%%% Identity types %%%
\newcommand{\idsym}{{=}}
\newcommand{\id}[3][]{\ensuremath{#2 =_{#1} #3}\xspace}
\newcommand{\idtype}[3][]{\ensuremath{\mathsf{Id}_{#1}(#2,#3)}\xspace}
\newcommand{\idtypevar}[1]{\ensuremath{\mathsf{Id}_{#1}}\xspace}
% A propositional equality currently being defined
\newcommand{\defid}{\coloneqq}

%%% Dependent paths
\newcommand{\dpath}[4]{#3 =^{#1}_{#2} #4}

%%% singleton
% \newcommand{\sgl}{\ensuremath{\mathsf{sgl}}\xspace}
% \newcommand{\sctr}{\ensuremath{\mathsf{sctr}}\xspace}

%%% Reflexivity terms %%%
% \newcommand{\reflsym}{{\mathsf{refl}}}
\newcommand{\refl}[1]{\ensuremath{\mathsf{refl}_{#1}}\xspace}

%%% Path concatenation (used infix, in diagrammatic order) %%%
\newcommand{\ct}{%
  \mathchoice{\mathbin{\raisebox{0.5ex}{$\displaystyle\centerdot$}}}%
             {\mathbin{\raisebox{0.5ex}{$\centerdot$}}}%
             {\mathbin{\raisebox{0.25ex}{$\scriptstyle\,\centerdot\,$}}}%
             {\mathbin{\raisebox{0.1ex}{$\scriptscriptstyle\,\centerdot\,$}}}
}

%%% Path reversal %%%
\newcommand{\opp}[1]{\mathord{{#1}^{-1}}}
\let\rev\opp

%%% Coherence paths %%%
\newcommand{\ctassoc}{\mathsf{assoc}} % associativity law

%%% Transport (covariant) %%%
\newcommand{\trans}[2]{\ensuremath{{#1}_{*}\mathopen{}\left({#2}\right)\mathclose{}}\xspace}
\let\Trans\trans
%\newcommand{\Trans}[2]{\ensuremath{{#1}_{*}\left({#2}\right)}\xspace}
\newcommand{\transf}[1]{\ensuremath{{#1}_{*}}\xspace} % Without argument
%\newcommand{\transport}[2]{\ensuremath{\mathsf{transport}_{*} \: {#2}\xspace}}
\newcommand{\transfib}[3]{\ensuremath{\mathsf{transport}^{#1}(#2,#3)\xspace}}
\newcommand{\Transfib}[3]{\ensuremath{\mathsf{transport}^{#1}\Big(#2,\, #3\Big)\xspace}}
\newcommand{\transfibf}[1]{\ensuremath{\mathsf{transport}^{#1}\xspace}}

%%% 2D transport
\newcommand{\transtwo}[2]{\ensuremath{\mathsf{transport}^2\mathopen{}\left({#1},{#2}\right)\mathclose{}}\xspace}

%%% Constant transport
\newcommand{\transconst}[3]{\ensuremath{\mathsf{transportconst}}^{#1}_{#2}(#3)\xspace}
\newcommand{\transconstf}{\ensuremath{\mathsf{transportconst}}\xspace}

%%% Map on paths %%%
\newcommand{\mapfunc}[1]{\ensuremath{\mathsf{ap}_{#1}}\xspace} % Without argument
\newcommand{\map}[2]{\ensuremath{{#1}\mathopen{}\left({#2}\right)\mathclose{}}\xspace}
\let\Ap\map
%\newcommand{\Ap}[2]{\ensuremath{{#1}\left({#2}\right)}\xspace}
\newcommand{\mapdepfunc}[1]{\ensuremath{\mathsf{apd}_{#1}}\xspace} % Without argument
% \newcommand{\mapdep}[2]{\ensuremath{{#1}\llparenthesis{#2}\rrparenthesis}\xspace}
\newcommand{\mapdep}[2]{\ensuremath{\mapdepfunc{#1}\mathopen{}\left(#2\right)\mathclose{}}\xspace}
\let\apfunc\mapfunc
\let\ap\map
\let\apdfunc\mapdepfunc
\let\apd\mapdep

%%% 2D map on paths
\newcommand{\aptwofunc}[1]{\ensuremath{\mathsf{ap}^2_{#1}}\xspace}
\newcommand{\aptwo}[2]{\ensuremath{\aptwofunc{#1}\mathopen{}\left({#2}\right)\mathclose{}}\xspace}
\newcommand{\apdtwofunc}[1]{\ensuremath{\mathsf{apd}^2_{#1}}\xspace}
\newcommand{\apdtwo}[2]{\ensuremath{\apdtwofunc{#1}\mathopen{}\left(#2\right)\mathclose{}}\xspace}

%%% Identity functions %%%
\newcommand{\idfunc}[1][]{\ensuremath{\mathsf{id}_{#1}}\xspace}

%%% Homotopies (written infix) %%%
\newcommand{\htpy}{\sim}

%%% Other meanings of \sim
\newcommand{\bisim}{\sim}       % bisimulation
\newcommand{\eqr}{\sim}         % an equivalence relation

%%% Equivalence types %%%
\newcommand{\eqv}[2]{\ensuremath{#1 \simeq #2}\xspace}
\newcommand{\eqvspaced}[2]{\ensuremath{#1 \;\simeq\; #2}\xspace}
\newcommand{\eqvsym}{\simeq}    % infix symbol
\newcommand{\texteqv}[2]{\ensuremath{\mathsf{Equiv}(#1,#2)}\xspace}
\newcommand{\isequiv}{\ensuremath{\mathsf{isequiv}}}
\newcommand{\qinv}{\ensuremath{\mathsf{qinv}}}
\newcommand{\ishae}{\ensuremath{\mathsf{ishae}}}
\newcommand{\linv}{\ensuremath{\mathsf{linv}}}
\newcommand{\rinv}{\ensuremath{\mathsf{rinv}}}
\newcommand{\biinv}{\ensuremath{\mathsf{biinv}}}
\newcommand{\lcoh}[3]{\mathsf{lcoh}_{#1}(#2,#3)}
\newcommand{\rcoh}[3]{\mathsf{rcoh}_{#1}(#2,#3)}
\newcommand{\hfib}[2]{{\mathsf{fib}}_{#1}(#2)}

%%% Map on total spaces %%%
\newcommand{\total}[1]{\ensuremath{\mathsf{total}(#1)}}

%%% Universe types %%%
%\newcommand{\type}{\ensuremath{\mathsf{Type}}\xspace}
\newcommand{\UU}{\ensuremath{\mathcal{U}}\xspace}
\let\bbU\UU
\let\type\UU
% Universes of truncated types
\newcommand{\typele}[1]{\ensuremath{{#1}\text-\mathsf{Type}}\xspace}
\newcommand{\typeleU}[1]{\ensuremath{{#1}\text-\mathsf{Type}_\UU}\xspace}
\newcommand{\typelep}[1]{\ensuremath{{(#1)}\text-\mathsf{Type}}\xspace}
\newcommand{\typelepU}[1]{\ensuremath{{(#1)}\text-\mathsf{Type}_\UU}\xspace}
\let\ntype\typele
\let\ntypeU\typeleU
\let\ntypep\typelep
\let\ntypepU\typelepU
\renewcommand{\set}{\ensuremath{\mathsf{Set}}\xspace}
\newcommand{\setU}{\ensuremath{\mathsf{Set}_\UU}\xspace}
\newcommand{\prop}{\ensuremath{\mathsf{Prop}}\xspace}
\newcommand{\propU}{\ensuremath{\mathsf{Prop}_\UU}\xspace}
%Pointed types
\newcommand{\pointed}[1]{\ensuremath{#1_\bullet}}

%%% Ordinals and cardinals
\newcommand{\card}{\ensuremath{\mathsf{Card}}\xspace}
\newcommand{\ord}{\ensuremath{\mathsf{Ord}}\xspace}
\newcommand{\ordsl}[2]{{#1}_{/#2}}

%%% Univalence
\newcommand{\ua}{\ensuremath{\mathsf{ua}}\xspace} % the inverse of idtoeqv
\newcommand{\idtoeqv}{\ensuremath{\mathsf{idtoeqv}}\xspace}
\newcommand{\univalence}{\ensuremath{\mathsf{univalence}}\xspace} % the full axiom

%%% Truncation levels
\newcommand{\iscontr}{\ensuremath{\mathsf{isContr}}}
\newcommand{\contr}{\ensuremath{\mathsf{contr}}} % The path to the center of contraction
\newcommand{\isset}{\ensuremath{\mathsf{isSet}}}
\newcommand{\isprop}{\ensuremath{\mathsf{isProp}}}
% h-propositions
% \newcommand{\anhprop}{a mere proposition\xspace}
% \newcommand{\hprops}{mere propositions\xspace}

%%% Homotopy fibers %%%
%\newcommand{\hfiber}[2]{\ensuremath{\mathsf{hFiber}(#1,#2)}\xspace}
\let\hfiber\hfib

%%% Bracket/squash/truncation types %%%
% \newcommand{\brck}[1]{\textsf{mere}(#1)}
% \newcommand{\Brck}[1]{\textsf{mere}\Big(#1\Big)}
% \newcommand{\trunc}[2]{\tau_{#1}(#2)}
% \newcommand{\Trunc}[2]{\tau_{#1}\Big(#2\Big)}
% \newcommand{\truncf}[1]{\tau_{#1}}
%\newcommand{\trunc}[2]{\Vert #2\Vert_{#1}}
\newcommand{\trunc}[2]{\mathopen{}\left\Vert #2\right\Vert_{#1}\mathclose{}}
\newcommand{\ttrunc}[2]{\bigl\Vert #2\bigr\Vert_{#1}}
\newcommand{\Trunc}[2]{\Bigl\Vert #2\Bigr\Vert_{#1}}
\newcommand{\truncf}[1]{\Vert \blank \Vert_{#1}}
\newcommand{\tproj}[3][]{\mathopen{}\left|#3\right|_{#2}^{#1}\mathclose{}}
\newcommand{\tprojf}[2][]{|\blank|_{#2}^{#1}}
\def\pizero{\trunc0}
%\newcommand{\brck}[1]{\trunc{-1}{#1}}
%\newcommand{\Brck}[1]{\Trunc{-1}{#1}}
%\newcommand{\bproj}[1]{\tproj{-1}{#1}}
%\newcommand{\bprojf}{\tprojf{-1}}

\newcommand{\brck}[1]{\trunc{}{#1}}
\newcommand{\bbrck}[1]{\ttrunc{}{#1}}
\newcommand{\Brck}[1]{\Trunc{}{#1}}
\newcommand{\bproj}[1]{\tproj{}{#1}}
\newcommand{\bprojf}{\tprojf{}}

% Big parentheses
\newcommand{\Parens}[1]{\Bigl(#1\Bigr)}

% Projection and extension for truncations
\let\extendsmb\ext
\newcommand{\extend}[1]{\extendsmb(#1)}

%
%%% The empty type
\newcommand{\emptyt}{\ensuremath{\mathbf{0}}\xspace}

%%% The unit type
\newcommand{\unit}{\ensuremath{\mathbf{1}}\xspace}
\newcommand{\ttt}{\ensuremath{\star}\xspace}

%%% The two-element type
\newcommand{\bool}{\ensuremath{\mathbf{2}}\xspace}
\newcommand{\btrue}{{1_{\bool}}}
\newcommand{\bfalse}{{0_{\bool}}}

%%% Injections into binary sums and pushouts
\newcommand{\inlsym}{{\mathsf{inl}}}
\newcommand{\inrsym}{{\mathsf{inr}}}
\newcommand{\inl}{\ensuremath\inlsym\xspace}
\newcommand{\inr}{\ensuremath\inrsym\xspace}

%%% The segment of the interval
\newcommand{\seg}{\ensuremath{\mathsf{seg}}\xspace}

%%% Free groups
\newcommand{\freegroup}[1]{F(#1)}
\newcommand{\freegroupx}[1]{F'(#1)} % the "other" free group

%%% Glue of a pushout
\newcommand{\glue}{\mathsf{glue}}

%%% Colimits
\newcommand{\colim}{\mathsf{colim}}
\newcommand{\inc}{\mathsf{inc}}
\newcommand{\cmp}{\mathsf{cmp}}

%%% Circles and spheres
\newcommand{\Sn}{\mathbb{S}}
\newcommand{\base}{\ensuremath{\mathsf{base}}\xspace}
\newcommand{\lloop}{\ensuremath{\mathsf{loop}}\xspace}
\newcommand{\surf}{\ensuremath{\mathsf{surf}}\xspace}

%%% Suspension
\newcommand{\susp}{\Sigma}
\newcommand{\north}{\mathsf{N}}
\newcommand{\south}{\mathsf{S}}
\newcommand{\merid}{\mathsf{merid}}

%%% Blanks (shorthand for lambda abstractions)
\newcommand{\blank}{\mathord{\hspace{1pt}\text{--}\hspace{1pt}}}

%%% Nameless objects
\newcommand{\nameless}{\mathord{\hspace{1pt}\underline{\hspace{1ex}}\hspace{1pt}}}

%%% Some decorations
%\newcommand{\bbU}{\ensuremath{\mathbb{U}}\xspace}
% \newcommand{\bbB}{\ensuremath{\mathbb{B}}\xspace}
\newcommand{\bbP}{\ensuremath{\mathbb{P}}\xspace}

%%% Some categories
\newcommand{\uset}{\ensuremath{\mathcal{S}et}\xspace}
\newcommand{\ucat}{\ensuremath{{\mathcal{C}at}}\xspace}
\newcommand{\urel}{\ensuremath{\mathcal{R}el}\xspace}
\newcommand{\uhilb}{\ensuremath{\mathcal{H}ilb}\xspace}
\newcommand{\utype}{\ensuremath{\mathcal{T}\!ype}\xspace}

% Pullback corner
\newbox\pbbox
\setbox\pbbox=\hbox{\xy \POS(65,0)\ar@{-} (0,0) \ar@{-} (65,65)\endxy}
\def\pb{\save[]+<3.5mm,-3.5mm>*{\copy\pbbox} \restore}

% Macros for the categories chapter
\newcommand{\inv}[1]{{#1}^{-1}}
\newcommand{\idtoiso}{\ensuremath{\mathsf{idtoiso}}\xspace}
\newcommand{\isotoid}{\ensuremath{\mathsf{isotoid}}\xspace}
\newcommand{\op}{^{\mathrm{op}}}
\newcommand{\y}{\ensuremath{\mathbf{y}}\xspace}
\newcommand{\dgr}[1]{{#1}^{\dagger}}
\newcommand{\unitaryiso}{\mathrel{\cong^\dagger}}
\newcommand{\cteqv}[2]{\ensuremath{#1 \simeq #2}\xspace}
\newcommand{\cteqvsym}{\simeq}     % Symbol for equivalence of categories

%%% Natural numbers
\newcommand{\N}{\ensuremath{\mathbb{N}}\xspace}
%\newcommand{\N}{\textbf{N}}
\let\nat\N
\newcommand{\natp}{\ensuremath{\nat'}\xspace} % alternative nat in induction chapter

\newcommand{\zerop}{\ensuremath{0'}\xspace}   % alternative zero in induction chapter
\newcommand{\suc}{\mathsf{succ}}
\newcommand{\sucp}{\ensuremath{\suc'}\xspace} % alternative suc in induction chapter
\newcommand{\add}{\mathsf{add}}
\newcommand{\ack}{\mathsf{ack}}
\newcommand{\ite}{\mathsf{iter}}
\newcommand{\assoc}{\mathsf{assoc}}
\newcommand{\dbl}{\ensuremath{\mathsf{double}}}
\newcommand{\dblp}{\ensuremath{\dbl'}\xspace} % alternative double in induction chapter


%%% Lists
\newcommand{\lst}[1]{\mathsf{List}(#1)}
\newcommand{\nil}{\mathsf{nil}}
\newcommand{\cons}{\mathsf{cons}}
\newcommand{\lost}[1]{\mathsf{Lost}(#1)}

%%% Vectors of given length, used in induction chapter
\newcommand{\vect}[2]{\ensuremath{\mathsf{Vec}_{#1}(#2)}\xspace}

%%% Integers
\newcommand{\Z}{\ensuremath{\mathbb{Z}}\xspace}
\newcommand{\Zsuc}{\mathsf{succ}}
\newcommand{\Zpred}{\mathsf{pred}}

%%% Rationals
\newcommand{\Q}{\ensuremath{\mathbb{Q}}\xspace}

%%% Function extensionality
\newcommand{\funext}{\mathsf{funext}}
\newcommand{\happly}{\mathsf{happly}}

%%% A naturality lemma
\newcommand{\com}[3]{\mathsf{swap}_{#1,#2}(#3)}

%%% Code/encode/decode
\newcommand{\code}{\ensuremath{\mathsf{code}}\xspace}
\newcommand{\encode}{\ensuremath{\mathsf{encode}}\xspace}
\newcommand{\decode}{\ensuremath{\mathsf{decode}}\xspace}

% Function definition with domain and codomain
\newcommand{\function}[4]{\left\{\begin{array}{rcl}#1 &
      \longrightarrow & #2 \\ #3 & \longmapsto & #4 \end{array}\right.}

%%% Cones and cocones
\newcommand{\cone}[2]{\mathsf{cone}_{#1}(#2)}
\newcommand{\cocone}[2]{\mathsf{cocone}_{#1}(#2)}
% Apply a function to a cocone
\newcommand{\composecocone}[2]{#1\circ#2}
\newcommand{\composecone}[2]{#2\circ#1}
%%% Diagrams
\newcommand{\Ddiag}{\mathscr{D}}

%%% (pointed) mapping spaces
\newcommand{\Map}{\mathsf{Map}}

%%% The interval
\newcommand{\interval}{\ensuremath{I}\xspace}
\newcommand{\izero}{\ensuremath{0_{\interval}}\xspace}
\newcommand{\ione}{\ensuremath{1_{\interval}}\xspace}

%%% Arrows
\newcommand{\epi}{\ensuremath{\twoheadrightarrow}}
\newcommand{\mono}{\ensuremath{\rightarrowtail}}

%%% Sets
\newcommand{\bin}{\ensuremath{\mathrel{\widetilde{\in}}}}

%%% Semigroup structure
\newcommand{\semigroupstrsym}{\ensuremath{\mathsf{SemigroupStr}}}
\newcommand{\semigroupstr}[1]{\ensuremath{\mathsf{SemigroupStr}}(#1)}
\newcommand{\semigroup}[0]{\ensuremath{\mathsf{Semigroup}}}

%%% Macros for the formal type theory
\newcommand{\emptyctx}{\ensuremath{\cdot}}
\newcommand{\production}{\vcentcolon\vcentcolon=}
\newcommand{\conv}{\downarrow}
\newcommand{\ctx}{\ensuremath{\mathsf{ctx}}}
\newcommand{\wfctx}[1]{#1\ \ctx}
\newcommand{\oftp}[3]{#1 \vdash #2 : #3}
\newcommand{\jdeqtp}[4]{#1 \vdash #2 \jdeq #3 : #4}
\newcommand{\judg}[2]{#1 \vdash #2}
\newcommand{\tmtp}[2]{#1 \mathord{:} #2}

% rule names
\newcommand{\rform}{\textsc{form}}
\newcommand{\rintro}{\textsc{intro}}
\newcommand{\relim}{\textsc{elim}}
\newcommand{\rcomp}{\textsc{comp}}
\newcommand{\runiq}{\textsc{uniq}}
\newcommand{\Weak}{\mathsf{Wkg}}
\newcommand{\Vble}{\mathsf{Vble}}
\newcommand{\Exch}{\mathsf{Exch}}
\newcommand{\Subst}{\mathsf{Subst}}

%%% Macros for HITs
\newcommand{\cc}{\mathsf{c}}
\newcommand{\pp}{\mathsf{p}}
\newcommand{\cct}{\widetilde{\mathsf{c}}}
\newcommand{\ppt}{\widetilde{\mathsf{p}}}
\newcommand{\Wtil}{\ensuremath{\widetilde{W}}\xspace}

%%% Macros for n-types
\newcommand{\istype}[1]{\mathsf{is}\mbox{-}{#1}\mbox{-}\mathsf{type}}
\newcommand{\nplusone}{\ensuremath{(n+1)}}
\newcommand{\nminusone}{\ensuremath{(n-1)}}
\newcommand{\fact}{\mathsf{fact}}

%%% Macros for homotopy
\newcommand{\kbar}{\overline{k}} % Used in van Kampen's theorem

%%% Macros for induction
\newcommand{\natw}{\ensuremath{\mathbf{N^w}}\xspace}
\newcommand{\zerow}{\ensuremath{0^\mathbf{w}}\xspace}
\newcommand{\sucw}{\ensuremath{\mathsf{succ}^{\mathbf{w}}}\xspace}
\newcommand{\nalg}{\nat\mathsf{Alg}}
\newcommand{\nhom}{\nat\mathsf{Hom}}
\newcommand{\ishinitw}{\mathsf{isHinit}_{\mathsf{W}}}
\newcommand{\ishinitn}{\mathsf{isHinit}_\nat}
\newcommand{\w}{\mathsf{W}}
\newcommand{\walg}{\w\mathsf{Alg}}
\newcommand{\whom}{\w\mathsf{Hom}}

%%% Macros for real numbers
\newcommand{\RC}{\ensuremath{\mathbb{R}_\mathsf{c}}\xspace} % Cauchy
\newcommand{\RD}{\ensuremath{\mathbb{R}_\mathsf{d}}\xspace} % Dedekind
\newcommand{\R}{\ensuremath{\mathbb{R}}\xspace}           % Either
\newcommand{\barRD}{\ensuremath{\bar{\mathbb{R}}_\mathsf{d}}\xspace} % Dedekind completion of Dedekind

\newcommand{\close}[1]{\sim_{#1}} % Relation of closeness
\newcommand{\closesym}{\mathord\sim}
\newcommand{\rclim}{\mathsf{lim}} % HIT constructor for Cauchy reals
\newcommand{\rcrat}{\mathsf{rat}} % Embedding of rationals into Cauchy reals
\newcommand{\rceq}{\mathsf{eq}_{\RC}} % HIT path constructor
\newcommand{\CAP}{\mathcal{C}}    % The type of Cauchy approximations
\newcommand{\Qp}{\Q_{+}}
\newcommand{\apart}{\mathrel{\#}}  % apartness
\newcommand{\dcut}{\mathsf{isCut}}  % Dedekind cut
\newcommand{\cover}{\triangleleft} % inductive cover
\newcommand{\intfam}[3]{(#2, \lam{#1} #3)} % family of rational intervals

% Macros for the Cauchy reals construction
\newcommand{\bsim}{\frown}
\newcommand{\bbsim}{\smile}

\newcommand{\hapx}{\diamondsuit\approx}
\newcommand{\hapname}{\diamondsuit}
\newcommand{\hapxb}{\heartsuit\approx}
\newcommand{\hapbname}{\heartsuit}
\newcommand{\tap}[1]{\bullet\approx_{#1}\triangle}
\newcommand{\tapname}{\triangle}
\newcommand{\tapb}[1]{\bullet\approx_{#1}\square}
\newcommand{\tapbname}{\square}

%%% Macros for surreals
\newcommand{\NO}{\ensuremath{\mathsf{No}}\xspace}
\newcommand{\surr}[2]{\{\,#1\,\big|\,#2\,\}}
\newcommand{\LL}{\mathcal{L}}
\newcommand{\RR}{\mathcal{R}}
\newcommand{\noeq}{\mathsf{eq}_{\NO}} % HIT path constructor

\newcommand{\ble}{\trianglelefteqslant}
\newcommand{\blt}{\vartriangleleft}
\newcommand{\bble}{\sqsubseteq}
\newcommand{\bblt}{\sqsubset}

\newcommand{\hle}{\diamondsuit\preceq}
\newcommand{\hlt}{\diamondsuit\prec}
\newcommand{\hlname}{\diamondsuit}
\newcommand{\hleb}{\heartsuit\preceq}
\newcommand{\hltb}{\heartsuit\prec}
\newcommand{\hlbname}{\heartsuit}
% \newcommand{\tle}{(\bullet\preceq\triangle)}
% \newcommand{\tlt}{(\bullet\prec\triangle)}
\newcommand{\tle}{\triangle\preceq}
\newcommand{\tlt}{\triangle\prec}
\newcommand{\tlname}{\triangle}
% \newcommand{\tleb}{(\bullet\preceq\square)}
% \newcommand{\tltb}{(\bullet\prec\square)}
\newcommand{\tleb}{\square\preceq}
\newcommand{\tltb}{\square\prec}
\newcommand{\tlbname}{\square}

%%% Macros for set theory
\newcommand{\vset}{\mathsf{set}}  % point constructor for cummulative hierarchy V
\def\cd{\tproj0}
\newcommand{\inj}{\ensuremath{\mathsf{inj}}} % type of injections
\newcommand{\acc}{\ensuremath{\mathsf{acc}}} % accessibility

\newcommand{\atMostOne}{\mathsf{atMostOne}}

\newcommand{\power}[1]{\mathcal{P}(#1)} % power set
\newcommand{\powerp}[1]{\mathcal{P}_+(#1)} % inhabited power set

%%%% THEOREM ENVIRONMENTS %%%%

% The cleveref package provides \cref{...} which is like \ref{...}
% except that it automatically inserts the type of the thing you're
% referring to, e.g. it produces "Theorem 3.8" instead of just "3.8"
% (and hyperref makes the whole thing a hyperlink).  This saves a slight amount
% of typing, but more importantly it means that if you decide later on
% that 3.8 should be a Lemma or a Definition instead of a Theorem, you
% don't have to change the name in all the places you referred to it.

% The following hack improves on this by using the same counter for
% all theorem-type environments, so that after Theorem 1.1 comes
% Corollary 1.2 rather than Corollary 1.1.  This makes it much easier
% for the reader to find a particular theorem when flipping through
% the document.
\makeatletter
\def\defthm#1#2#3{%
  %% Ensure all theorem types are numbered with the same counter
  \newaliascnt{#1}{thm}
  \newtheorem{#1}[#1]{#2}
  \aliascntresetthe{#1}
  %% This command tells cleveref's \cref what to call things
  \crefname{#1}{#2}{#3}% following brace must be on separate line to support poorman cleveref sed file
}

% Now define a bunch of theorem-type environments.
\newtheorem{thm}{Theorem}[section]
\crefname{thm}{Theorem}{Theorems}
%\defthm{prop}{Proposition}   % Probably we shouldn't use "Proposition" in this way
\defthm{cor}{Corollary}{Corollaries}
\defthm{lem}{Lemma}{Lemmas}
\defthm{axiom}{Axiom}{Axioms}
% Since definitions and theorems in type theory are synonymous, should
% we actually use the same theoremstyle for them?
\theoremstyle{definition}
\defthm{defn}{Definition}{Definitions}
\theoremstyle{remark}
\defthm{rmk}{Remark}{Remarks}
\defthm{eg}{Example}{Examples}
\defthm{egs}{Examples}{Examples}
\defthm{notes}{Notes}{Notes}
% Number exercises within chapters, with their own counter.
\newtheorem{ex}{Exercise}[chapter]
\crefname{ex}{Exercise}{Exercises}

% Display format for sections
\crefformat{section}{\S#2#1#3}
\Crefformat{section}{Section~#2#1#3}
\crefrangeformat{section}{\S\S#3#1#4--#5#2#6}
\Crefrangeformat{section}{Sections~#3#1#4--#5#2#6}
\crefmultiformat{section}{\S\S#2#1#3}{ and~#2#1#3}{, #2#1#3}{ and~#2#1#3}
\Crefmultiformat{section}{Sections~#2#1#3}{ and~#2#1#3}{, #2#1#3}{ and~#2#1#3}
\crefrangemultiformat{section}{\S\S#3#1#4--#5#2#6}{ and~#3#1#4--#5#2#6}{, #3#1#4--#5#2#6}{ and~#3#1#4--#5#2#6}
\Crefrangemultiformat{section}{Sections~#3#1#4--#5#2#6}{ and~#3#1#4--#5#2#6}{, #3#1#4--#5#2#6}{ and~#3#1#4--#5#2#6}

% Display format for appendices
\crefformat{appendix}{Appendix~#2#1#3}
\Crefformat{appendix}{Appendix~#2#1#3}
\crefrangeformat{appendix}{Appendices~#3#1#4--#5#2#6}
\Crefrangeformat{appendix}{Appendices~#3#1#4--#5#2#6}
\crefmultiformat{appendix}{Appendices~#2#1#3}{ and~#2#1#3}{, #2#1#3}{ and~#2#1#3}
\Crefmultiformat{appendix}{Appendices~#2#1#3}{ and~#2#1#3}{, #2#1#3}{ and~#2#1#3}
\crefrangemultiformat{appendix}{Appendices~#3#1#4--#5#2#6}{ and~#3#1#4--#5#2#6}{, #3#1#4--#5#2#6}{ and~#3#1#4--#5#2#6}
\Crefrangemultiformat{appendix}{Appendices~#3#1#4--#5#2#6}{ and~#3#1#4--#5#2#6}{, #3#1#4--#5#2#6}{ and~#3#1#4--#5#2#6}

\crefname{part}{Part}{Parts}

% Number subsubsections
\setcounter{secnumdepth}{5}

% Display format for figures
\crefname{figure}{Figure}{Figures}

%%%% EQUATION NUMBERING %%%%

% The following hack uses the single theorem counter to number
% equations as well, so that we don't have both Theorem 1.1 and
% equation (1.1).
\let\c@equation\c@thm
\numberwithin{equation}{section}


%%%% ENUMERATE NUMBERING %%%%

% Number the first level of enumerates as (i), (ii), ...
\renewcommand{\theenumi}{(\roman{enumi})}
\renewcommand{\labelenumi}{\theenumi}


%%%% MARGINS %%%%

% This is a matter of personal preference, but I think the left
% margins on enumerates and itemizes are too wide.
\setitemize[1]{leftmargin=2em}
\setenumerate[1]{leftmargin=*}

% Likewise that they are too spaced out.
\setitemize[1]{itemsep=-0.2em}
\setenumerate[1]{itemsep=-0.2em}

%%% Notes %%%
\def\noteson{%
\gdef\note##1{\mbox{}\marginpar{\color{blue}\textasteriskcentered\ ##1}}}
\gdef\notesoff{\gdef\note##1{\null}}
\noteson

\newcommand{\Coq}{\textsc{Coq}\xspace}
\newcommand{\Agda}{\textsc{Agda}\xspace}
\newcommand{\NuPRL}{\textsc{NuPRL}\xspace}

%%%% CITATIONS %%%%

% \let \cite \citep

%%%% INDEX %%%%

\newcommand{\footstyle}[1]{{\hyperpage{#1}}n} % If you index something that is in a footnote
\newcommand{\defstyle}[1]{\textbf{\hyperpage{#1}}}  % Style for pageref to a definition

\newcommand{\indexdef}[1]{\index{#1|defstyle}}   % Index a definition
\newcommand{\indexfoot}[1]{\index{#1|footstyle}} % Index a term in a footnote
\newcommand{\indexsee}[2]{\index{#1|see{#2}}}    % Index "see also"


%%%% Standard phrasing or spelling of common phrases %%%%

\newcommand{\ZF}{Zermelo--Fraenkel}
\newcommand{\CZF}{Constructive \ZF{} Set Theory}

\newcommand{\LEM}[1]{\ensuremath{\mathsf{LEM}_{#1}}\xspace}
\newcommand{\choice}[1]{\ensuremath{\mathsf{AC}_{#1}}\xspace}

%%%% MISC %%%%

\newcommand{\mentalpause}{\medskip} % Use for "mental" pause, instead of \smallskip or \medskip

%% Use \symlabel instead of \label to mark a pageref that you need in the index of symbols
\newcounter{symindex}
\newcommand{\symlabel}[1]{\refstepcounter{symindex}\label{#1}}

% Local Variables:
% mode: latex
% TeX-master: "hott-online"
% End:


%%%%%%%%%%%%%%%%%%%%%%%%%%%%%%%%%%%%%%%%%%%%%%%%%%

\title{The countable reals}

\author{Andrej Bauer}
% It is likely there is a better way to specify double affiliation
\address{Faculty of Mathematics and Physics, University of Ljubljana, Slovenia}
\address{Institute of Mathematics, Physics and Mechanics, Slovenia}
\email{Andrej.Bauer@andrej.com}
% amsaddr package seems unable to handle \urladdr, so it just gets placed at the end
% \urladdr{https://www.andrej.com/}

\thanks{This material is based upon work supported by the Air Force Office of Scientific Research under award number FA9550-21-1-0024.}

\author{James E.~Hanson}
\email{jhanson9@umd.edu}
% amsaddr package seems unable to handle \urladdr, so it just gets placed at the end
%\urladdr{https://james-hanson.github.io}
\address{University of Maryland, College Park, USA}

\begin{document}

\begin{abstract}
  We construct a topos in which the Dedekind reals are countable.

  To accomplish this, we first define a new kind of toposes that we call \emph{parameterized realizability toposes}. They are built from partial combinatory algebras whose application operation depends on a parameter, and in which realizers operate uniformly with respect to a given parameter set.

  Our topos is the parameterized realizability topos whose realizers are oracle-computable partial maps, with oracles serving as parameters and ranging over the representations of a non-diagonalizable sequence, discovered by Joseph Miller. It is a sequence of reals in $[0,1]$ that is non-diagonalizable in the sense that any real in $[0,1]$ that is oracle-computable, uniformly in oracles representing the sequence, must already appear in the sequence.
  %
  The Dedekind reals are countable in the topos because the non-diagonalizable sequence appears in it as an epimorphism.

  The topos is intuitionistic, as it invalidates both the law of excluded middle and the axiom of countable choice. The Cauchy reals are uncountable. The Hilbert cube is countable, from which Brouwer's fixed-point theorem follows as an easy corollary of Lawvere's fixed-point theorem. From the 1-dimensional Brouwer's fixed-point theorem we obtain the intermediate value theorem and the lesser limited principle of omniscience. The Kreisel-Lacombe-Shoenfield-Tseitin theorem stating that all real-valued maps are continuous is valid, because the usual proof is uniform with respect to oracles.
  %
  Lastly, the closed interval $[0,1]$, being countable, can trivially be covered by a sequence of open intervals whose lengths add up to any prescribed $0 < \epsilon < 1$, and such a cover has no finite subcover.
  However, we show that any sequence of open intervals with rational endpoints covering $[0,1]$ must has a finite subcover.
\end{abstract}

\maketitle


\chapter*{Introduction}
\markboth{\textsc{Introduction}}{}
\addcontentsline{toc}{chapter}{Introduction}
\setcounter{page}{1}
\pagenumbering{arabic}


\emph{Homotopy type theory} is a new branch of mathematics that combines aspects of several different fields in a surprising way. It is based on a recently discovered connection between \emph{homotopy theory} and \emph{type theory}.
Homotopy theory is an outgrowth of algebraic topology and homological algebra, with relationships to higher category theory; while type theory is a branch of mathematical logic and theoretical computer science.
Although the connections between the two are currently the focus of intense investigation, it is increasingly clear that they are just the beginning of a subject that will take more time and more hard work to fully understand.
It touches on topics as seemingly distant as the homotopy groups of spheres, the algorithms for type checking, and the definition of weak $\infty$-groupoids.

Homotopy type theory also brings new ideas into the very foundation of mathematics.
\index{foundations, univalent}%
On the one hand, there is Voevodsky's subtle and beautiful \emph{univalence axiom}. 
\index{univalence axiom}%
The univalence axiom implies, in particular, that isomorphic structures can be identified, a principle that mathematicians have been happily using on workdays, despite its incompatibility with the ``official'' doctrines of conventional foundations.
On the other hand, we have \emph{higher inductive types}, which provide direct, logical descriptions of some of the basic spaces and constructions of homotopy theory: spheres, cylinders, truncations, localizations, etc.
Both ideas are impossible to capture directly in classical set-theoretic foundations, but when combined in homotopy type theory, they permit an entirely new kind of ``logic of homotopy types''.
\index{foundations}%

This suggests a new conception of foundations of mathematics, with intrinsic homotopical content, an ``invariant'' conception of the objects of mathematics --- and convenient machine implementations, which can serve as a practical aid to the working mathematician.
This is the \emph{Univalent Foundations} program.
The present book is intended as a first systematic exposition of the basics of univalent foundations, and a collection of examples of this new style of reasoning --- but without requiring the reader to know or learn any formal logic, or to use any computer proof assistant.

% This enlarges the page by one line in letter format. Use sparringly.
\OPTwidow

We emphasize that homotopy type theory is a young field, and univalent foundations is very much a work in progress. 
This book should be regarded as a ``snapshot'' of just one portion of the field, taken at the time it was written, rather than a polished exposition of a completed edifice. 
As we will discuss briefly later, there are many aspects of homotopy type theory that are not yet fully understood --- and some that are not even touched upon here. 
The ultimate theory will almost certainly not look exactly like the one described in this book, but it will surely be \emph{at least} as capable and powerful; we therefore believe that univalent foundations will eventually become a viable alternative to set theory as the ``implicit foundation'' for the unformalized mathematics done by most mathematicians.

\subsection*{Type theory}

Type theory was originally invented by Bertrand Russell \cite{Russell:1908},\index{Russell, Bertrand} as a device for blocking the paradoxes in the logical foundations of mathematics  that were under investigation at the time.
It was developed further by many people over the next few decades, particularly Church~\cite{Church:1940tu,Church:1941tc} who combined it with his \textit{$\lambda$-calculus}.
Although it is not generally regarded as the foundation for classical mathematics, set theory being more customary, type theory still has numerous applications, especially in computer science and the theory of programming languages~\cite{Pierce-TAPL}.
\index{programming}%
\index{type theory}%
\index{lambda-calculus@$\lambda$-calculus}%
Per Martin-L\"{o}f \cite{Martin-Lof-1972,Martin-Lof-1973,Martin-Lof-1979,martin-lof:bibliopolis}, among others,
developed a ``predicative'' modification of Church's type system, which is now usually called dependent, constructive, intuitionistic, or simply \emph{Martin\--L\"of type theory}. This is the basis of the system that we consider here; it was originally intended as a rigorous framework for the formalization of constructive mathematics.  In what follows, we will often use ``type theory'' to refer specifically to this system and similar ones, although type theory as a subject is much broader (see \cite{somma,kamar} for the history of type theory).

In type theory, unlike set theory, objects are classified using a primitive notion of \emph{type}, similar to the data-types used in programming languages.  These elaborately structured types can be used to express detailed specifications of the objects classified, giving rise to principles of reasoning about these objects.  To take a very simple example, the objects of a product type $A\times B$ are known to be of the form $\pairr{a,b}$, and so one automatically knows how to construct them and how to decompose them. Similarly, an object of function type $A\to B$ can be acquired from an object of type $B$ parametrized by objects of type $A$, and can be evaluated at an argument of type $A$.  This rigidly predictable behavior of all objects (as opposed to set theory's more liberal formation principles, allowing inhomogeneous sets) is one aspect of type theory that has led to its extensive use in verifying the correctness of computer programs.  The clear reasoning principles associated with the construction of types also form the basis of modern \emph{computer proof assistants},%
\index{proof!assistant}%
\indexsee{computer proof assistant}{proof assistant}
\index{mathematics!formalized}%
which are used for formalizing mathematics and verifying the correctness of formalized proofs.  We return to this aspect of type theory below.  

One problem in understanding type theory from a mathematical point of view, however, has always been that the basic concept of \emph{type} is unlike that of \emph{set} in ways that have been hard to make precise.  We believe that the new idea of regarding types, not as strange sets (perhaps constructed without using classical logic), but as spaces, viewed from the perspective of homotopy theory, is a significant step forward.  In particular, it solves the problem of understanding how the notion of equality of elements of a type differs from that of elements of a set.

In homotopy theory one is concerned with spaces
\index{topological!space}%
and continuous mappings between them, 
\index{function!continuous!in classical homotopy theory}%
up to homotopy.  A \emph{homotopy}
\index{homotopy!topological}%
between a pair of continuous maps $f : X \to Y$
and  $g : X\to Y$ is 
a continuous map $H : X \times [0, 1] \to Y$ satisfying
$H(x, 0) = f (x)$  and $H(x, 1) = g(x)$. The homotopy $H$ may be thought of as a ``continuous deformation'' of $f$ into $g$. The spaces $X$ and $Y$ are said to be \emph{homotopy equivalent},
\index{homotopy!equivalence!topological}%
$\eqv X Y$, if there are continuous maps going back and forth, the composites of which are homotopical to the respective identity mappings, i.e., if they are isomorphic ``up to homotopy''.  Homotopy equivalent spaces have the same algebraic invariants (e.g., homology, or the fundamental group), and are said to have the same \emph{homotopy type}.

\subsection*{Homotopy type theory}

Homotopy type theory (HoTT) interprets type theory from a homotopical perspective.
In homotopy type theory, we regard the types as ``spaces'' (as studied in homotopy theory) or higher groupoids, and the logical constructions (such as the product $A\times B$) as homotopy-invariant constructions on these spaces.
In this way, we are able to manipulate spaces directly without first having to develop point-set topology (or any combinatorial replacement for it, such as the theory of simplicial sets).
To briefly explain this perspective, consider first the basic concept of type theory, namely that
the \emph{term} $a$ is of \emph{type} $A$, which is written:
\[ a:A. \]
This expression is traditionally thought of as akin to:
\begin{center}
``$a$ is an element of the set $A$''.
\end{center}
However, in homotopy type theory we think of it instead as:
\begin{center}
``$a$ is a point of the space $A$''.
\end{center}
\index{continuity of functions in type theory@``continuity'' of functions in type theory}%
Similarly, every function $f : A\to B$ in type theory is regarded as a continuous map from the space $A$ to the space $B$.

We should stress that these ``spaces'' are treated purely homotopically, not topologically.
For instance, there is no notion of ``open subset'' of a type or of ``convergence'' of a sequence of elements of a type.
We only have ``homotopical'' notions, such as paths between points and homotopies between paths, which also make sense in other models of homotopy theory (such as simplicial sets).
Thus, it would be more accurate to say that we treat types as \emph{$\infty$-groupoids}\index{.infinity-groupoid@$\infty$-groupoid}; this is a name for the ``invariant objects'' of homotopy theory which can be presented by topological spaces,
\index{topological!space}%
simplicial sets, or any other model for homotopy theory.
However, it is convenient to sometimes use topological words such as ``space'' and ``path'', as long as we remember that other topological concepts are not applicable.

(It is tempting to also use the phrase \emph{homotopy type}
\index{homotopy!type}%
for these objects, suggesting the dual interpretation of ``a type (as in type theory) viewed homotopically'' and ``a space considered from the point of view of homotopy theory''.
The latter is a bit different from the classical meaning of ``homotopy type'' as an \emph{equivalence class} of spaces modulo homotopy equivalence, although it does preserve the meaning of phrases such as ``these two spaces have the same homotopy type''.)

The idea of interpreting types as structured objects, rather than sets, has a long pedigree, and is known to clarify various mysterious aspects of type theory.
For instance, interpreting types as sheaves helps explain the intuitionistic nature of type-theoretic logic, while interpreting them as partial equivalence relations or ``domains'' helps explain its computational aspects.
It also implies that we can use type-theoretic reasoning to study the structured objects, leading to the rich field of categorical logic.
The homotopical interpretation fits this same pattern: it clarifies the nature of \emph{identity} (or equality) in type theory, and allows us to use type-theoretic reasoning in the study of homotopy theory.

The key new idea of the homotopy interpretation is that the logical notion of identity $a = b$ of two objects $a, b: A$ of the same type $A$ can be understood as the existence of a path $p : a \leadsto b$ from point $a$ to point $b$ in the space $A$.
This also means that two functions $f, g: A\to B$ can be identified if they are homotopic, since a homotopy is just a (continuous) family of paths $p_x: f(x) \leadsto g(x)$ in $B$, one for each $x:A$.
In type theory, for every type $A$ there is a (formerly somewhat mysterious) type $\idtypevar{A}$ of identifications of two objects of $A$; in homotopy type theory, this is just the \emph{path space} $A^I$ of all continuous maps $I\to A$ from the unit interval.
\index{unit!interval}%
\index{interval!topological unit}%
\index{path!topological}%
\index{topological!path}%
In this way, a term $p : \idtype[A]{a}{b}$ represents a path $p : a \leadsto b$ in $A$. 

The idea of homotopy type theory arose around 2006 in independent work by Awodey and Warren~\cite{AW} and Voevodsky~\cite{VV}, but it was inspired by 
Hofmann and Streicher's earlier groupoid interpretation~\cite{hs:gpd-typethy}.
Indeed, higher-dimensional category theory (particularly the theory of weak $\infty$-groupoids) is now known to be intimately connected to homotopy theory, as proposed by Grothendieck and now being studied intensely by mathematicians of both sorts.
The original semantic models of Awodey--Warren and Voevodsky use well-known notions and techniques from homotopy theory which are now also in use in higher category theory, such as  Quillen model categories and Kan\index{Kan complex} simplicial sets\index{simplicial!sets}.
\index{Quillen model category}%
\index{model category}%

In particular, Voevodsky constructed an interpretation of type theory in Kan simplicial sets, and recognized that this interpretation satisfied a further crucial property which he dubbed \emph{univalence}.
This had not previously been considered in type theory (although Church's principle of extensionality for propositions turns out to be a very special case of it, and Hofmann and Streicher had considered another special case under the name ``universe extensionality'').
Adding univalence to type theory in the form of a new axiom has far-reaching consequences, many of which are natural, simplifying and compelling.
The univalence axiom also further strengthens the homotopical view of type theory, since it holds in the simplicial model and other related models, while failing under the view of types as sets.

\subsection*{Univalent foundations}

Very briefly, the basic idea of the univalence axiom can be explained as follows.
In type theory, one can have a universe $\UU$, the terms of which are themselves types, $A : \UU$, etc.
Those types that are terms of $\UU$ are commonly called \emph{small} types.
\index{type!small}%
\index{small!type}%
Like any type, $\UU$ has an identity type $\idtypevar{\UU}$, which expresses the identity relation $A = B$ between small types.
Thinking of types as spaces, $\UU$ is a space, the points of which are spaces; to understand its identity type, we must ask, what is a path $p : A \leadsto B$ between spaces in $\UU$?
The univalence axiom says that such paths correspond to homotopy equivalences $\eqv A B$, (roughly) as explained above.
A bit more precisely, given any (small) types $A$ and $B$, in addition to the primitive type $\idtype[\UU]AB$ of identifications of $A$ with $B$, there is the defined type $\texteqv AB$ of equivalences from $A$ to $B$.
Since the identity map on any object is an equivalence, there is a canonical map,
\[\idtype[\UU]AB\to\texteqv AB.\]
The univalence axiom states that this map is itself an equivalence.
At the risk of oversimplifying, we can state this succinctly as follows:

\begin{description}\index{univalence axiom}%
\item[Univalence Axiom:]  $\eqvspaced{(A = B)}{(\eqv A B)}$.
\end{description}
%
In other words, identity is equivalent to equivalence. \index{identity}% 
In particular, one may say that ``equivalent types are identical''.
However, this phrase is somewhat misleading, since it may sound like a sort of ``skeletality'' condition which \emph{collapses} the notion of equivalence to coincide with identity, whereas in fact univalence is about \emph{expanding} the notion of identity so as to coincide with the (unchanged) notion of equivalence.

From the homotopical point of view, univalence implies that spaces of the same homotopy type are connected by a path in the universe $\UU$, in accord with the intuition of a classifying space for (small) spaces.
From the logical point of view, however, it is a radically new idea: it says that isomorphic things can be identified!  Mathematicians are of course used to identifying isomorphic structures in practice, but they generally do so by ``abuse of notation''\index{abuse!of notation}, or some other informal device, knowing that the objects involved are not ``really'' identical.  But in this new foundational scheme, such structures can be formally identified, in the logical sense that every property or construction involving one also applies to the other. Indeed, the identification is now made explicit, and properties and constructions can be systematically transported along it.  Moreover, the different ways in which such identifications may be made themselves form a structure that one can (and should!)\ take into account.

Thus in sum, for points $A$ and $B$ of the universe $\UU$ (i.e., small types), the univalence axiom identifies the following three notions:
\begin{itemize}
\item (logical) an identification $p:A=B$ of $A$ and $B$
\item (topological) a path $p:A \leadsto B$ from $A$ to $B$ in $\UU$
\item (homotopical) an equivalence $p:\eqv A B$ between $A$ and $B$.
\end{itemize}

\subsection*{Higher inductive types}\index{type!higher inductive}%

One of the classical advantages of type theory is its simple and effective techniques for working with inductively defined structures.
The simplest nontrivial inductively defined structure is the natural numbers, which is inductively generated by zero and the successor function.
From this statement one can algorithmically\index{algorithm} extract the principle of mathematical induction, which characterizes the natural numbers.
More general inductive definitions encompass lists and well-founded trees of all sorts, each of which is characterized by a corresponding ``induction principle''.
This includes most data structures used in certain programming languages; hence the usefulness of type theory in formal reasoning about the latter.
If conceived in a very general sense, inductive definitions also include examples such as a disjoint union $A+B$, which may be regarded as ``inductively'' generated by the two injections $A\to A+B$ and $B\to A+B$.
The ``induction principle'' in this case is ``proof by case analysis'', which characterizes the disjoint union.

In homotopy theory, it is natural to consider also ``inductively defined spaces'' which are generated not merely by a collection of \emph{points}, but also by collections of \emph{paths} and higher paths.
Classically, such spaces are called \emph{CW complexes}.
\index{CW complex}%
For instance, the circle $S^1$ is generated by a single point and a single path from that point to itself.
Similarly, the 2-sphere $S^2$ is generated by a single point $b$ and a single two-dimensional path from the constant path at $b$ to itself, while the torus $T^2$ is generated by a single point, two paths $p$ and $q$ from that point to itself, and a two-dimensional path from $p\ct q$ to $q\ct p$.

By using the identification of paths with identities in homotopy type theory, these sort of ``inductively defined spaces'' can be characterized in type theory by ``induction principles'', entirely analogously to classical examples such as the natural numbers and the disjoint union.
The resulting \emph{higher inductive types}
\index{type!higher inductive}%
give a direct ``logical'' way to reason about familiar spaces such as spheres, which (in combination with univalence) can be used to perform familiar arguments from homotopy theory, such as calculating homotopy groups of spheres, in a purely formal way.
The resulting proofs are a marriage of classical homotopy-theoretic ideas with classical type-theoretic ones, yielding new insight into both disciplines.

Moreover, this is only the tip of the iceberg: many abstract constructions from homotopy theory, such as homotopy colimits, suspensions, Postnikov towers, localization, completion, and spectrification, can also be expressed as higher inductive types.
Many of these are classically constructed using Quillen's ``small object argument'', which can be regarded as a finite way of algorithmically describing an infinite CW complex presentation\index{presentation!of a space as a CW complex} of a space, just as ``zero and successor'' is a finite algorithmic\index{algorithm} description of the infinite set of natural numbers.
Spaces produced by the small object argument are infamously complicated and difficult to understand; the type-theoretic approach is potentially much simpler, bypassing the need for any explicit construction by giving direct access to the appropriate ``induction principle''.
Thus, the combination of univalence and higher inductive types suggests the possibility of a revolution, of sorts, in the practice of homotopy theory.


\subsection*{Sets in univalent foundations}

\index{set|(}%

We have claimed that univalent foundations can eventually serve as a foundation for ``all'' of mathematics, but so far we have discussed 
only homotopy theory.  Of course, there are many specific examples of the use of type theory without the new homotopy type theory features to formalize mathematics,
\index{mathematics!formalized}%
\index{theorem!Feit--Thompson}%
\index{theorem!odd-order}%
\index{Feit--Thompson theorem}%
\index{odd-order theorem}%
such as the recent formalization of the Feit--Thompson odd-order theorem in \Coq~\cite{gonthier}.

But the traditional view is that mathematics is founded on set theory, in the sense that all mathematical objects and constructions can be coded into a theory such as Zermelo--Fraenkel set theory (ZF).
\index{set theory!Zermelo--Fraenkel}%
\indexsee{Zermelo-Fraenkel set theory}{set theory}%
\indexsee{ZF}{set theory}%
\indexsee{ZFC}{set theory}%
However, it is well-established by now that for most mathematics outside of set theory proper, the intricate hierarchical membership structure of sets in ZF is really unnecessary: a more ``structural'' theory, such as Lawvere's\index{Lawvere} Elementary Theory of the Category of Sets~\cite{lawvere:etcs-long}, suffices.
\index{Elementary Theory of the Category of Sets}%

In univalent foundations, the basic objects are ``homotopy types'' rather than sets, but we can \emph{define} a class of types which behave like sets.
Homotopically, these can be thought of as spaces in which every connected component is contractible, i.e.\ those which are homotopy equivalent to a discrete space.
\index{discrete!space}%
It is a theorem  that the category of such ``sets'' satisfies Lawvere's\index{Lawvere} axioms (or related ones, depending on the details of the theory).
Thus, any sort of mathematics that can be represented in an ETCS-like theory (which, experience suggests, is essentially all of mathematics) can equally well be represented in univalent foundations.  

This supports the claim that univalent foundations is at least as good as existing foundations of mathematics.
A mathematician working in univalent foundations can build structures out of sets in a familiar way, with more general homotopy types waiting in the foundational background until there is need of them.
For this reason, most of the applications in this book have been chosen to be areas where univalent foundations has something \emph{new} to contribute that distinguishes it from existing foundational systems.

Unsurprisingly, homotopy theory and category theory are two of these, but perhaps less obvious is that univalent foundations has something new and interesting to offer even in subjects such as set theory and real analysis.
For instance, the univalence axiom allows us to identify isomorphic structures, while higher inductive types allow direct descriptions of objects by their universal properties.
Thus we can generally avoid resorting to arbitrarily chosen representatives or transfinite iterative constructions.
In fact, even the objects of study in ZF set theory can be characterized, inside the sets of univalent foundations, by such an inductive universal property.

\index{set|)}%


\subsection*{Informal type theory}

\index{mathematics!formalized|(defstyle}%
\index{informal type theory|(defstyle}%
\index{type theory!informal|(defstyle}%
\index{type theory!formal|(}%
One difficulty often encountered by the classical mathematician when faced with learning about type theory is that it is usually presented as a fully or partially formalized deductive system.
This style, which is very useful for proof-theoretic investigations, is not particularly convenient for use in applied, informal reasoning.
Nor is it even familiar to most working mathematicians, even those who might be interested in foundations of mathematics.
One objective of the present work is to develop an informal style of doing mathematics in univalent foundations that is at once rigorous and precise, but is also closer to the language and style of presentation of everyday mathematics.

In present-day mathematics, one usually constructs and reasons about mathematical objects in a way that could in principle, one presumes, be formalized in a system of elementary set theory, such as ZFC --- at least given enough ingenuity and patience.
For the most part, one does not even need to be aware of this possibility, since it largely coincides with the condition that a proof be ``fully rigorous'' (in the sense that all mathematicians have come to understand intuitively through education and experience).
But one does need to learn to be careful about a few aspects of ``informal set theory'': the use of collections too large or inchoate to be sets; the axiom of choice and its equivalents; even (for undergraduates) the method of proof by contradiction; and so on.
Adopting a new foundational system such as homotopy type theory as the \emph{implicit formal basis} of informal reasoning will require adjusting some of one's instincts and practices.
The present text is intended to serve as an example of this ``new kind of mathematics'', which is still informal, but could now in principle be formalized in homotopy type theory, rather than ZFC, again given enough ingenuity and patience.

It is worth emphasizing that, in this new system, such formalization can have real practical benefits.
The formal system of type theory is suited to computer systems and has been implemented in existing proof assistants.
\index{proof!assistant}%
A proof assistant is a computer program which guides the user in construction of a fully formal proof, only allowing valid steps of reasoning.
It also provides some degree of automation, can search libraries for existing theorems, and can even extract numerical algorithms\index{algorithm} \index{extraction of algorithms} from the resulting (constructive) proofs.

We believe that this aspect of the univalent foundations program distinguishes it from other approaches to foundations, potentially providing a new practical utility for the working mathematician.
Indeed, proof assistants based on older type theories have already been used to formalize substantial mathematical proofs, such as the four-color theorem\index{theorem!four-color} \index{four-color theorem} and the Feit--Thompson theorem.
Computer implementations of univalent foundations are presently works in progress (like the theory itself).
\index{proof!assistant}%
However, even its currently available implementations (which are mostly small modifications to existing proof assistants such as \Coq and 
\Agda) have already demonstrated their worth, not only in the formalization of known proofs, but in the discovery of new ones.
Indeed, many of the proofs described in this book were actually \emph{first} done in a fully formalized form in a proof assistant, and are only now being ``unformalized'' for the first time --- a reversal of the usual relation between formal and informal mathematics.

One can imagine a not-too-distant future when it will be possible for mathematicians to verify the correctness of their own papers by working within the system of univalent foundations, formalized in a proof assistant, and that doing so will become as natural as typesetting their own papers in \TeX.
%(Whether this proves to be the publishers' dream or their nightmare remains to be seen.) 
In principle, this could be equally true for any other foundational system, but we believe it to be more practically attainable using univalent foundations, as witnessed by the present work and its formal counterpart.

\index{type theory!formal|)}%
\index{informal type theory|)}%
\index{type theory!informal|)}%
\index{mathematics!formalized|)}%

\subsection*{Constructivity} 

\index{mathematics!constructive|(}%

One of the most striking differences between classical\index{mathematics!classical} foundations and type theory is the idea of \emph{proof relevance}, according to which mathematical statements, and even their proofs, become first-class mathematical objects.
In type theory, we represent mathematical statements by types, which can be regarded simultaneously as both mathematical constructions and mathematical assertions, a conception also known as \emph{propositions as types}.
\index{proposition!as types}%
Accordingly, we can regard a term $a : A$ as both an element of the type $A$ (or in homotopy type theory, a point of the space $A$), and at the same time, a proof of the proposition $A$.
To take an example, suppose we have sets $A$ and $B$ (discrete spaces),
\index{discrete!space}%
and consider the statement ``$A$ is isomorphic to $B$''.
In type theory, this can be rendered as:
\begin{narrowmultline*}
  \mathsf{Iso}(A,B) \defeq \narrowbreak
  \sm{f : A\to B}{g : B\to A}\Big(\big(\tprd{x:A} g(f(x)) = x\big) \times \big(\tprd{y:B}\, f(g(y)) = y\big)\Big).
\end{narrowmultline*}
%
Reading the type constructors $\Sigma, \Pi, \times$  here  as ``there exists'', ``for all'', and ``and'' respectively yields the usual formulation of ``$A$ and $B$ are isomorphic''; on the other hand, reading them as sums and products yields the \emph{type of all isomorphisms} between $A$ and $B$!  To prove that $A$ and $B$ are isomorphic, one  constructs a proof $p : \mathsf{Iso}(A,B)$, which is therefore the same  as constructing an isomorphism between $A$ and $B$, i.e., exhibiting a pair of functions $f, g$ together with \emph{proofs} that their composites are the respective identity maps.  The latter proofs, in turn, are nothing but homotopies of the appropriate sorts.  In this way, \emph{proving a proposition is the same as constructing an element of some particular type.}
In particular, to prove a statement of the form ``$A$ and $B$'' is just to prove $A$ and to prove $B$, i.e., to give an element of the type $A\times B$.
And to prove that $A$ implies $B$ is just to find an element of $A\to B$, i.e.\ a function from $A$ to $B$ (determining a mapping of proofs of $A$ to proofs of $B$).

The logic of propositions-as-types is flexible and supports many variations, such as using only a subclass of types to represent propositions.
In homotopy type theory, there are natural such subclasses arising from the fact that the system of all types, just like spaces in classical homotopy theory, is ``stratified'' according to the dimensions in which their higher homotopy structure exists or collapses.
In particular, Voevodsky has found a purely type-theoretic definition of \emph{homotopy $n$-types}, corresponding to spaces with no nontrivial homotopy information above dimension $n$.
(The $0$-types are the ``sets'' mentioned previously as satisfying Lawvere's axioms\index{Lawvere}.)
Moreover, with higher inductive types, we can universally ``truncate'' a type into an $n$-type; in classical homotopy theory this would be its $n^{\mathrm{th}}$ Postnikov\index{Postnikov tower} section.\index{n-type@$n$-type}
Particularly important for logic is the case of homotopy $(-1)$-types, which we call \emph{mere propositions}.
Classically, every $(-1)$-type is empty or contractible; we interpret these possibilities as the truth values ``false'' and ``true'' respectively.

Using all types as propositions yields a very ``constructive'' conception of logic; for more on this, see~\cite{kolmogorov,TroelstraI,TroelstraII}.
For instance, every proof that something exists carries with it enough information to actually find such an object; and every proof that ``$A$ or $B$'' holds is either a proof that $A$ holds or a proof that $B$ holds.
Thus, from every proof we can automatically extract an algorithm;\index{algorithm} \index{extraction of algorithms} this can be very useful in applications to computer programming.

On the other hand, however, this logic does diverge from the traditional understanding of existence proofs in mathematics.
In particular, it does not faithfully represent certain important classical principles of reasoning, such as the axiom of choice and the law of excluded middle.
For these we need to use the ``$(-1)$-truncated'' logic, in which only the homotopy $(-1)$-types represent propositions.

\index{axiom!of choice}%
More specifically, consider on one hand the \emph{axiom of choice}: ``if for every $x: A$ there exists a $y:B$ such that $R(x,y)$, there is a function $f : A\to B$ such that for all $x:A$ we have $R(x, f(x))$.''
The pure propositions-as-types notion of ``there exists'' is strong enough to make this statement simply provable --- yet it does not have all the consequences of the usual axiom of choice.
However, in $(-1)$-truncated logic, this statement is not automatically true, but is a strong assumption with the same sorts of consequences as its counterpart in classical\index{mathematics!classical} set theory.

\index{excluded middle}%
\index{univalence axiom}%
On the other hand, consider the \emph{law of excluded middle}: ``for all $A$, either $A$ or not $A$.''
Interpreting this in the pure propositions-as-types logic yields a statement that is inconsistent with the univalence axiom.
For since proving ``$A$'' means exhibiting an element of it, this assumption would give a uniform way of selecting an element from every nonempty type --- a sort of Hilbertian choice operator.
Univalence implies that the element of $A$ selected by such a choice operator must be invariant under all self-equivalences of $A$, since these are identified with self-identities and every operation must respect identity; but clearly some types have automorphisms with no fixed points, e.g.\ we can swap the elements of a two-element type.
\index{automorphism!fixed-point-free}%
However, the ``$(-1)$-truncated law of excluded middle'', though also not automatically true, may consistently be assumed with most of the same consequences as in classical mathematics.

In other words, while the pure propositions-as-types logic is ``constructive'' in the strong algorithmic sense mentioned above, the default $(-1)$-truncated logic is ``constructive'' in a different sense (namely, that of the logic formalized by Heyting under the name ``intuitionistic''); and to the latter we may freely add the axioms of choice and excluded middle to obtain a logic that may be called ``classical''.
Thus, homotopy type theory is compatible with both constructive and classical conceptions of logic, and many more besides.
\index{logic!constructive vs classical}%
Indeed, the homotopical perspective reveals that classical and constructive logic can coexist, as endpoints of a spectrum of different systems, with an infinite number of possibilities in between (the homotopy $n$-types for $-1 < n < \infty$).
We may speak of ``\LEM{n}'' and ``\choice{n}'', with $\choice{\infty}$ being provable and \LEM{\infty} inconsistent with univalence, while $\choice{-1}$ and $\LEM{-1}$ are the versions familiar to classical mathematicians (hence in most cases it is appropriate to assume the subscript $(-1)$ when none is given).  Indeed, one can even have useful systems in which only \emph{certain} types satisfy such further ``classical'' principles, while types in general remain ``constructive''.\index{excluded middle}\index{axiom!of choice}%%

It is worth emphasizing that univalent foundations does not \emph{require} the use of constructive or intuitionistic logic.\index{logic!intuitionistic}\index{logic!constructive} %
Most of classical mathematics which depends on the law of excluded middle and the axiom of choice can be performed in univalent foundations, simply by assuming that these two principles hold (in their proper, $(-1)$-truncated, form).
However, type theory does encourage avoiding these principles when they are unnecessary, for several reasons.

First of all, every mathematician knows that a theorem is more powerful when proven using fewer assumptions, since it applies to more examples.
The situation with \choice{} and \LEM{} is no different:
type theory admits many interesting ``nonstandard'' models, such as in sheaf toposes,\index{topos} where classicality principles such as \choice{} and \LEM{} tend to fail.
Homotopy type theory admits similar models in higher toposes, such as are studied in~\cite{ToenVezzosi02,Rezk05,lurie:higher-topoi}.
Thus, if we avoid using these principles, the theorems we prove will be valid internally to all such models.

Secondly, one of the additional virtues of type theory is its computable character.
In addition to being a foundation for mathematics, type theory is a formal theory of computation, and can be treated as a powerful programming language.
\index{programming}%
From this perspective, the rules of the system cannot be chosen arbitrarily the way set-theoretic axioms can: there must be a harmony between them which allows all proofs to be ``executed'' as programs.
We do not yet fully understand the new principles introduced by homotopy type theory, such as univalence and higher inductive types, from
this point of view, but the basic outlines are emerging; see, for example,~\cite{lh:canonicity}.
It has been known for a long time, however, that principles such as \choice{} and \LEM{} are fundamentally antithetical to computability, since they assert baldly that certain things exist without giving any way to compute them.
Thus, avoiding them is necessary to maintain the character of type theory as a theory of computation.

Fortunately, constructive reasoning is not as hard as it may seem.
In some cases, simply by rephrasing some definitions, a theorem can be made constructive and its proof more elegant.
Moreover, in univalent foundations this seems to happen more often.
For instance:
\begin{enumerate}
\item In set-theoretic foundations, at various points in homotopy theory and category theory one needs the axiom of choice to perform transfinite constructions.
  But with higher inductive types, we can encode these constructions directly and constructively.
  In particular, none of the ``synthetic'' homotopy theory in \cref{cha:homotopy} requires \LEM{} or \choice{}.
\item In set-theoretic foundations, the statement ``every fully faithful and essentially surjective functor is an equivalence of categories'' is equiv\-a\-lent to the axiom of choice.
  But with the univalence axiom, it is just \emph{true}; see \cref{cha:category-theory}.
\item In set theory, various circumlocutions are required to obtain notions of ``cardinal number'' and ``ordinal number'' which canonically represent isomorphism classes of sets and well-ordered sets, respectively --- possibly involving the axiom of choice or the axiom of foundation.
  But with univalence and higher inductive types, we can obtain such representatives directly by truncating the universe; see \cref{cha:set-math}.
\item In set-theoretic foundations, the definition of the real numbers as equivalence classes of Cauchy sequences requires either the law of excluded middle or the axiom of (countable) choice to be well-behaved.
  But with higher inductive types, we can give a version of this definition which is well-behaved and avoids any choice principles; see \cref{cha:real-numbers}.
\end{enumerate}
Of course, these simplifications could as well be taken as evidence that the new methods will not, ultimately, prove to be really constructive.  However, we emphasize again that the reader does not have to care, or worry, about constructivity in order to read this book.  The point is that in all of the above examples, the version of the theory we give has independent advantages, whether or not \LEM{} and \choice{} are assumed to be available.  Constructivity, if attained, will be an added bonus.\index{constructivity}%

Given this discussion of adding new principles such as univalence, higher inductive types, \choice{}, and \LEM{}, one may wonder whether the resulting system remains consistent.
(One of the original virtues of type theory, relative to set theory, was that it can be seen to be consistent by proof-theoretic means).
As with any foundational system, consistency\index{consistency} is a relative question: ``consistent with respect to what?''
The short answer is that all of the constructions and axioms considered in this book have a model in the category of Kan\index{Kan complex} complexes, due to Voevodsky~\cite{klv:ssetmodel} (see~\cite{ls:hits} for higher inductive types).
Thus, they are known to be consistent relative to ZFC (with as many inaccessible cardinals
\index{inaccessible cardinal}\index{consistency}%
as we need nested univalent universes).
Giving a more traditionally type-theoretic account of this consistency is work in progress (see,
e.g.,~\cite{lh:canonicity,coquand2012constructive}).

We summarize the different points of view of the type-theoretic operations in \cref{tab:pov}.

\begin{table}[htb]
  \centering
  \OPTsmalltable
 \begin{tabular}{lllll}
    \toprule
       Types && Logic & Sets & Homotopy\\ \addlinespace[2pt]
    \midrule
       $A$ && proposition & set & space\\ \addlinespace[2pt]
       $a:A$ && proof & element & point \\ \addlinespace[2pt]
       $B(x)$ && predicate & family of sets & fibration \\ \addlinespace[2pt]
       $b(x) : B(x)$ && conditional proof & family of elements & section\\ \addlinespace[2pt]
       $\emptyt, \unit$ && $\bot, \top$ & $\emptyset, \{ \emptyset \}$ & $\emptyset, *$\\ \addlinespace[2pt]
       $A + B$ && $A\vee B$ & disjoint union & coproduct\\ \addlinespace[2pt]
       $A\times B$ && $A\wedge B$ & set of pairs & product space\\ \addlinespace[2pt]
       $A\to B$ && $A\Rightarrow B$ & set of functions & function space\\ \addlinespace[2pt]
       $\sm{x:A}B(x)$ &&  $\exists_{x:A}B(x)$ & disjoint sum & total space\\ \addlinespace[2pt]
       $\prd{x:A}B(x)$ &&  $\forall_{x:A}B(x)$ & product & space of sections\\ \addlinespace[2pt]
       $\mathsf{Id}_{A}$ && equality $=$ & $\setof{\pairr{x,x} | x\in A}$ & path space $A^I$ \\ \addlinespace[2pt]
    \bottomrule
  \end{tabular}
  \caption{Comparing points of view on type-theoretic operations}\label{tab:pov}
\end{table}

\index{mathematics!constructive|)}%

\subsection*{Open problems} 

\index{open!problem|(}%

For those interested in contributing to this new branch of mathematics, it may be encouraging to know that there are many interesting open questions.

\index{univalence axiom!constructivity of}%
Perhaps the most pressing of them is the ``constructivity'' of the Univalence Axiom, posed by Voevodsky in \cite{Universe-poly}.
The basic system of type theory follows the structure of Gentzen's natural deduction. Logical connectives are defined by their introduction rules, and have elimination rules justified by computation rules. Following this pattern, and using Tait's computability method, originally designed to analyse G\"odel's Dialectica interpretation, one can show the property of \emph{normalization} for type theory. This in turn implies important properties such as decidability of type-checking (a crucial property since type-checking corresponds to proof-checking, and one can argue that we should be able to ``recognize a proof when we see one''), and the so-called ``canonicity\index{canonicity} property'' that any closed term of the type of natural numbers reduces to a numeral. This last property, and the uniform structure of introduction/elimination rules, are lost when one extends type theory with an axiom, such as the axiom of function extensionality, or the univalence axiom. Voevodsky has formulated a precise mathematical conjecture connected to this question of canonicity for type theory extended with the axiom of Univalence: given a closed term of the type of natural numbers, is it always possible to find a numeral and a proof that this term is equal to this numeral, where this proof of equality may itself use the univalence axiom? More generally, an important issue is whether it is possible to provide a constructive justification of the univalence axiom.
What about if one adds other homotopically motivated constructions, like higher inductive types?
These questions remain open at the present time, although methods are currently being developed to try to find answers.

Another basic issue is the difficulty of working with types, such as the natural numbers, that are essentially sets (i.e., discrete spaces),
\index{discrete!space}%
containing only trivial paths.
At present, homotopy type theory can really only characterize spaces up to homotopy equivalence, which means that these ``discrete spaces'' may only be \emph{homotopy equivalent} to discrete spaces.
Type-theoretically, this means there are many paths that are equal to reflexivity, but not \emph{judgmentally} equal to it (see \cref{sec:types-vs-sets} for the meaning of ``judgmentally'').
While this homotopy-invariance has advantages, these ``meaningless'' identity terms do introduce needless complications into arguments and constructions, so it would be convenient to have a systematic way of eliminating or collapsing them.
% In some cases, the proliferation of such superfluous identity terms makes it very difficult or impossible to formulate what should be a straightforward concept, such as the definition of a (semi-)simplicial type.

A more specialized, but no less important, problem is the relation between homotopy type theory and the research on \emph{higher toposes}%
\index{.infinity1-topos@$(\infty,1)$-topos}
currently happening at the intersection of higher category theory and homotopy theory.
There is a growing conviction among those familiar with both subjects that they are intimately connected.
For instance, the notion of a univalent universe should coincide with that of an object classifier, while higher inductive types should be an ``elementary'' reflection of local presentability.
More generally, homotopy type theory should be the ``internal language'' of $(\infty,1)$-toposes, just as intuitionistic higher-order logic is the internal language of ordinary 1-toposes.
Despite this general consensus, however, details remain to be worked out --- in particular, questions of coherence and strictness remain to be addressed  --- and doing so will undoubtedly lead to further insights into both concepts.

\index{mathematics!formalized}%
But by far the largest field of work to be done is in the ongoing formalization of everyday mathematics in this new system.
Recent successes in formalizing some facts from basic homotopy theory and category theory have been encouraging; some of these are described in \cref{cha:homotopy,cha:category-theory}.
Obviously, however, much work remains to be done.

\index{open!problem|)}%

The homotopy type theory community maintains a web site and group blog at \url{http://homotopytypetheory.org}, as well as a discussion email list.
Newcomers are always welcome!


\subsection*{How to read this book}

This book is divided into two parts.
\cref{part:foundations}, ``Foundations'', develops the fundamental concepts of homotopy type theory.
This is the mathematical foundation on which the development of specific subjects is built, and which is required for the understanding of the univalent foundations approach. To a programmer, this is ``library code''.
Since univalent foundations is a new and different kind of mathematics, its basic notions take some getting used to; thus \cref{part:foundations} is fairly extensive.

\cref{part:mathematics}, ``Mathematics'', consists of four chapters that build on the basic notions of \cref{part:foundations} to exhibit some of the new things we can do with univalent foundations in four different areas of mathematics: homotopy theory (\cref{cha:homotopy}), category theory (\cref{cha:category-theory}), set theory (\cref{cha:set-math}), and real analysis (\cref{cha:real-numbers}).
The chapters in \cref{part:mathematics} are more or less independent of each other, although occasionally one will use a lemma proven in another.

A reader who wants to seriously understand univalent foundations, and be able to work in it, will eventually have to read and understand most of \cref{part:foundations}.
However, a reader who just wants to get a taste of univalent foundations and what it can do may understandably balk at having to work through over 200 pages before getting to the ``meat'' in \cref{part:mathematics}.
Fortunately, not all of \cref{part:foundations} is necessary in order to read the chapters in \cref{part:mathematics}.
Each chapter in \cref{part:mathematics} begins with a brief overview of its subject, what univalent foundations has to contribute to it, and the necessary background from \cref{part:foundations}, so the courageous reader can turn immediately to the appropriate chapter for their favorite subject.
For those who want to understand one or more chapters in \cref{part:mathematics} more deeply than this, but are not ready to read all of \cref{part:foundations}, we provide here a brief summary of \cref{part:foundations}, with remarks about which parts are necessary for which chapters in \cref{part:mathematics}.

\cref{cha:typetheory} is about the basic notions of type theory, prior to any homotopical interpretation.
A reader who is familiar with Martin-L\"of type theory can quickly skim it to pick up the particulars of the theory we are using.
However, readers without experience in type theory will need to read \cref{cha:typetheory}, as there are many subtle differences between type theory and other foundations such as set theory.

\cref{cha:basics} introduces the homotopical viewpoint on type theory, along with the basic notions supporting this view, and describes the homotopical behavior of each component of the type theory from \cref{cha:typetheory}.
It also introduces the \emph{univalence axiom} (\cref{sec:compute-universe}) --- the first of the two basic innovations of homotopy type theory.
Thus, it is quite basic and we encourage everyone to read it, especially \crefrange{sec:equality}{sec:basics-equivalences}.

\cref{cha:logic} describes how we represent logic in homotopy type theory, and its connection to classical logic as well as to constructive and intuitionistic logic.
Here we define the law of excluded middle, the axiom of choice, and the axiom of propositional resizing (although, for the most part, we do not need to assume any of these in the rest of the book), as well as the \emph{propositional truncation} which is essential for representing traditional logic.
This chapter is essential background for \cref{cha:set-math,cha:real-numbers}, less important for \cref{cha:category-theory}, and not so necessary for \cref{cha:homotopy}.

\cref{cha:equivalences,cha:induction} study two special topics in detail: equivalences (and related notions) and generalized inductive definitions.
While these are important subjects in their own rights and provide a deeper understanding of homotopy type theory, for the most part they are not necessary for \cref{part:mathematics}.
Only a few lemmas from \cref{cha:equivalences} are used here and there, while the general discussions in \cref{sec:bool-nat,sec:strictly-positive,sec:generalizations} are helpful for providing the intuition required for \cref{cha:hits}.
The generalized sorts of inductive definition discussed in \cref{sec:generalizations} are also used in a few places in \cref{cha:set-math,cha:real-numbers}.

\cref{cha:hits} introduces the second basic innovation of homotopy type theory --- \emph{higher inductive types} --- with many examples.
Higher inductive types are the primary object of study in \cref{cha:homotopy}, and some particular ones play important roles in \cref{cha:set-math,cha:real-numbers}.
They are not so necessary for \cref{cha:category-theory}, although one example is used in \cref{sec:rezk}.

Finally, \cref{cha:hlevels} discusses homotopy $n$-types and related notions such as $n$-connected types.
These notions are important for \cref{cha:homotopy}, but not so important in the rest of \cref{part:mathematics}, although the case $n=-1$ of some of the lemmas are used in \cref{sec:piw-pretopos}.

This completes \cref{part:foundations}.
As mentioned above, \cref{part:mathematics} consists of four largely unrelated chapters, each describing what univalent foundations has to offer to a particular subject.

Of the chapters in \cref{part:mathematics}, \cref{cha:homotopy} (Homotopy theory) is perhaps the most radical.
Univalent foundations has a very different ``synthetic'' approach to homotopy theory in which homotopy types are the basic objects (namely, the types) rather than being constructed using topological spaces or some other set-theoretic model.
This enables new styles of proof for classical theorems in algebraic topology, of which we present a sampling, from $\pi_1(\Sn^1)=\Z$ to the Freudenthal suspension theorem.

In \cref{cha:category-theory} (Category theory), we develop some basic (1-)category theory, adhering to the principle of the univalence axiom that \emph{equality is isomorphism}.
This has the pleasant effect of ensuring that all definitions and constructions are automatically invariant under equivalence of categories: indeed, equivalent categories are equal just as equivalent types are equal.
(It also has connections to higher category theory and higher topos theory.)

\cref{cha:set-math} (Set theory) studies sets in univalent foundations.
The category of sets has its usual properties, hence provides a foundation for any mathematics that doesn't need homotopical or higher-categorical structures.
We also observe that univalence makes cardinal and ordinal numbers a bit more pleasant, and that higher inductive types yield a cumulative hierarchy satisfying the usual axioms of Zermelo--Fraenkel set theory.

In \cref{cha:real-numbers} (Real numbers), we summarize the construction of Dedekind real numbers, and then observe that higher inductive types allow a definition of Cauchy real numbers that avoids some associated problems in constructive mathematics.
Then we sketch a similar approach to Conway's surreal numbers.

Each chapter in this book ends with a Notes section, which collects historical comments, references to the literature, and attributions of results, to the extent possible.
We have also included Exercises at the end of each chapter, to assist the reader in gaining familiarity with doing mathematics in univalent foundations.

Finally, recall that this book was written as a massively collaborative effort by a large number of people.
We have done our best to achieve consistency in terminology and notation, and to put the mathematics in a linear sequence that flows logically, but it is very likely that some imperfections remain.
We ask the reader's forgiveness for any such infelicities, and welcome suggestions for improvement of the next edition.


% Local Variables:
% TeX-master: "hott-online"
% End:

\section{Sequential colimits}

\emph{Note: This chapter currently contains only the statements of the definitions and theorems, but no proofs. I hope to make a complete version available soon.}

\subsection{The universal property of sequential colimits}

Type sequences are diagrams of the following form.
\begin{equation*}
\begin{tikzcd}
A_0 \arrow[r,"f_0"] & A_1 \arrow[r,"f_1"] & A_2 \arrow[r,"f_2"] & \cdots.
\end{tikzcd}
\end{equation*}
Their formal specification is as follows.

\begin{defn}
An \define{(increasing) type sequence} $\mathcal{A}$ consists of
\begin{align*}
A & : \N\to\UU \\
f & : \prd{n:\N} A_n\to A_{n+1}. 
\end{align*}
\end{defn}

In this section we will introduce the sequential colimit of a type sequence.
The sequential colimit includes each of the types $A_n$, but we also identify each $x:A_n$ with its value $f_n(x):A_{n+1}$. 
Imagine that the type sequence $A_0\to A_1\to A_2\to\cdots$ defines a big telescope, with $A_0$ sliding into $A_1$, which slides into $A_2$, and so forth.

As usual, the sequential colimit is characterized by its universal property.

\begin{defn}
\begin{enumerate}
\item A \define{(sequential) cocone} on a type sequence $\mathcal{A}$ with vertex $B$ consists of
\begin{align*}
h & : \prd{n:\N} A_n\to B \\
H & : \prd{n:\N} h_n\htpy h_{n+1}\circ f_n.
\end{align*}
We write $\mathsf{cocone}(B)$ for the type of cocones with vertex $B$.
\item Given a cocone $(h,H)$ with vertex $B$ on a type sequence $\mathcal{A}$ we define the map
\begin{equation*}
\mathsf{cocone\usc{}map}(h,H) : (B\to C)\to \mathsf{cocone}(C)
\end{equation*}
given by $f\mapsto (\lam{n}f\circ h_n,\lam{n}{x}\mathsf{ap}_f(H_n(x)))$.
\item We say that a cocone $(h,H)$ with vertex $B$ is \define{colimiting} if $\mathsf{cocone\usc{}map}(h,H)$ is an equivalence for any type $C$.
\end{enumerate}
\end{defn}

\begin{thm}\label{thm:sequential_up}
Consider a cocone $(h,H)$ with vertex $B$ for a type sequence $\mathcal{A}$. The following are equivalent:
\begin{enumerate}
\item The cocone $(h,H)$ is colimiting.
\item The cocone $(h,H)$ is inductive in the sense that for every type family $P:B\to \UU$, the map
\begin{align*}
\Big(\prd{b:B}P(b)\Big)\to {}& \sm{h:\prd{n:\N}{x:A_n}P(h_n(x))}\\ 
& \qquad \prd{n:\N}{x:A_n} \mathsf{tr}_P(H_n(x),h_n(x))={h_{n+1}(f_n(x))}
\end{align*}
given by
\begin{equation*}
s\mapsto (\lam{n}s\circ h_n,\lam{n}{x} \mathsf{apd}_{s}(H_n(x)))
\end{equation*}
has a section.
\item The map in (ii) is an equivalence.
\end{enumerate}
\end{thm}

\subsection{The construction of sequential colimits}

We construct sequential colimits using pushouts.

\begin{defn}
Let $\mathcal{A}\jdeq (A,f)$ be a type sequence. We define the type $A_\infty$ as a pushout
\begin{equation*}
\begin{tikzcd}[column sep=large]
\tilde{A}+\tilde{A} \arrow[r,"{[\idfunc,\sigma_{\mathcal{A}}]}"] \arrow[d,swap,"{[\idfunc,\idfunc]}"] & \tilde{A} \arrow[d,"\inr"] \\
\tilde{A} \arrow[r,swap,"\inl"] & A_\infty.
\end{tikzcd}
\end{equation*}
\end{defn}

\begin{defn}
The type $A_\infty$ comes equipped with a cocone structure consisting of
\begin{align*}
\mathsf{seq\usc{}in} & : \prd{n:\N} A_n\to A_\infty \\
\mathsf{seq\usc{}glue} & : \prd{n:\N}{x:A_n} \mathsf{in}_n(x)=\mathsf{in}_{n+1}(f_n(x)).
\end{align*}
\end{defn}

\begin{constr}
We define
\begin{align*}
\mathsf{seq\usc{}in}(n,x)\defeq \inr(n,x) \\
\mathsf{seq\usc{}glue}(n,x)\defeq \ct{\glue(\inl(n,x))^{-1}}{\glue(\inr(n,x))}.
\end{align*}
\end{constr}

\begin{thm}
Consider a type sequence $\mathcal{A}$, and write $\tilde{A}\defeq\sm{n:\N}A_n$. Moreover, consider the map
\begin{equation*}
\sigma_{\mathcal{A}}:\tilde{A}\to\tilde{A}
\end{equation*}
defined by $\sigma_{\mathcal{A}}(n,a)\defeq (n+1,f_n(a))$. Furthermore, consider a cocone $(h,H)$ with vertex $B$.
The following are equivalent:
\begin{enumerate}
\item The cocone $(h,H)$ with vertex $B$ is colimiting.
\item The defining square
\begin{equation*}
\begin{tikzcd}[column sep=large]
\tilde{A}+\tilde{A} \arrow[r,"{[\idfunc,\sigma_{\mathcal{A}}]}"] \arrow[d,swap,"{[\idfunc,\idfunc]}"] & \tilde{A} \arrow[d,"{\lam{(n,x)}h_n(x)}"] \\
\tilde{A} \arrow[r,swap,"{\lam{(n,x)}h_n(x)}"] & A_\infty,
\end{tikzcd}
\end{equation*}
of $A_\infty$ is a pushout square.
\end{enumerate}
\end{thm}

\subsection{Descent for sequential colimits}

\begin{defn}
The type of \define{descent data} on a type sequence $\mathcal{A}\jdeq (A,f)$ is defined to be
\begin{equation*}
\mathsf{Desc}(\mathcal{A}) \defeq \sm{B:\prd{n:\N}A_n\to\UU}\prd{n:\N}{x:A_n}\eqv{B_n(x)}{B_{n+1}(f_n(x))}.
\end{equation*}
\end{defn}

\begin{defn}
We define a map
\begin{equation*}
\mathsf{desc\usc{}fam} : (A_\infty\to\UU)\to\mathsf{Desc}(\mathcal{A})
\end{equation*}
by $B\mapsto (\lam{n}{x}B(\mathsf{seq\usc{}in}(n,x)),\lam{n}{x}\mathsf{tr}_B(\mathsf{seq\usc{}glue}(n,x)))$.
\end{defn}

\begin{thm}
The map 
\begin{equation*}
\mathsf{desc\usc{}fam} : (A_\infty\to\UU)\to\mathsf{Desc}(\mathcal{A})
\end{equation*}
is an equivalence.
\end{thm}

\begin{defn}
A \define{cartesian transformation} of type sequences from $\mathcal{A}$ to $\mathcal{B}$ is a pair $(h,H)$ consisting of
\begin{align*}
h & : \prd{n:\N} A_n\to B_n \\
H & : \prd{n:\N} g_n\circ h_n \htpy h_{n+1}\circ f_n,
\end{align*}
such that each of the squares in the diagram
\begin{equation*}
\begin{tikzcd}
A_0 \arrow[d,swap,"h_0"] \arrow[r,"f_0"] & A_1 \arrow[d,swap,"h_1"] \arrow[r,"f_1"] & A_2 \arrow[d,swap,"h_2"] \arrow[r,"f_2"] & \cdots \\
B_0 \arrow[r,swap,"g_0"] & B_1 \arrow[r,swap,"g_1"] & B_2 \arrow[r,swap,"g_2"] & \cdots
\end{tikzcd}
\end{equation*}
is a pullback square. We define
\begin{align*}
\mathsf{cart}(\mathcal{A},\mathcal{B}) & \defeq\sm{h:\prd{n:\N}A_n\to B_n} \\
& \qquad\qquad \sm{H:\prd{n:\N}g_n\circ h_n\htpy h_{n+1}\circ f_n}\prd{n:\N}\mathsf{is\usc{}pullback}(h_n,f_n,H_n),
\end{align*}
and we write
\begin{equation*}
\mathsf{Cart}(\mathcal{B}) \defeq \sm{\mathcal{A}:\mathsf{Seq}}\mathsf{cart}(\mathcal{A},\mathcal{B}).
\end{equation*}
\end{defn}

\begin{defn}
We define a map
\begin{equation*}
\mathsf{cart\usc{}map}(\mathcal{B}) : \Big(\sm{X':\UU}X'\to X\Big)\to\mathsf{Cart}(\mathcal{B}).
\end{equation*}
which associates to any morphism $h:X'\to X$ a cartesian transformation of type sequences into $\mathcal{B}$.
\end{defn}

\begin{thm}
The operation $\mathsf{cart\usc{}map}(\mathcal{B})$ is an equivalence.
\end{thm}

\subsection{The flattening lemma for sequential colimits}

The flattening lemma for sequential colimits essentially states that sequential colimits commute with $\Sigma$. 

\begin{lem}
Consider
\begin{align*}
B & : \prd{n:\N}A_n\to\UU \\
g & : \prd{n:\N}{x:A_n}\eqv{B_n(x)}{B_{n+1}(f_n(x))}.
\end{align*}
and suppose $P:A_\infty\to\UU$ is the unique family equipped with
\begin{align*}
e & : \prd{n:\N}\eqv{B_n(x)}{P(\mathsf{seq\usc{}in}(n,x))}
\end{align*}
and homotopies $H_n(x)$ witnessing that the square
\begin{equation*}
\begin{tikzcd}[column sep=7em]
B_n(x) \arrow[r,"g_n(x)"] \arrow[d,swap,"e_n(x)"] & B_{n+1}(f_n(x)) \arrow[d,"e_{n+1}(f_n(x))"] \\
P(\mathsf{seq\usc{}in}(n,x)) \arrow[r,swap,"{\mathsf{tr}_P(\mathsf{seq\usc{}glue}(n,x))}"] & P(\mathsf{seq\usc{}in}(n+1,f_n(x)))
\end{tikzcd}
\end{equation*}
commutes. Then $\sm{t:A_\infty}P(t)$ satisfies the universal property of the sequential colimit of the type sequence
\begin{equation*}
\begin{tikzcd}
\sm{x:A_0}B_0(x) \arrow[r,"{\tot[f_0]{g_0}}"] & \sm{x:A_1}B_1(x) \arrow[r,"{\tot[f_1]{g_1}}"] & \sm{x:A_2}B_2(x) \arrow[r,"{\tot[f_2]{g_2}}"] & \cdots.
\end{tikzcd}
\end{equation*}
\end{lem}

In the following theorem we rephrase the flattening lemma in using cartesian transformations of type sequences.

\begin{thm}
Consider a commuting diagram of the form
\begin{equation*}
\begin{tikzcd}[column sep=small,row sep=small]
A_0 \arrow[rr] \arrow[dd] & & A_1 \arrow[rr] \arrow[dr] \arrow[dd] &[-.9em] &[-.9em] A_2 \arrow[dl] \arrow[dd] & & \cdots \\
& & & X \arrow[from=ulll,crossing over] \arrow[from=urrr,crossing over] \arrow[from=ur,to=urrr] \\
B_0 \arrow[rr] \arrow[drrr] & & B_1 \arrow[rr] \arrow[dr] & & B_2 \arrow[rr] \arrow[dl] & & \cdots \arrow[dlll] \\
& & & Y \arrow[from=uu,crossing over] 
\end{tikzcd}
\end{equation*}
If each of the vertical squares is a pullback square, and $Y$ is the sequential colimit of the type sequence $B_n$, then $X$ is the sequential colimit of the type sequence $A_n$. 
\end{thm}

\begin{cor}
Consider a commuting diagram of the form
\begin{equation*}
\begin{tikzcd}[column sep=small,row sep=small]
A_0 \arrow[rr] \arrow[dd] & & A_1 \arrow[rr] \arrow[dr] \arrow[dd] &[-.9em] &[-.9em] A_2 \arrow[dl] \arrow[dd] & & \cdots \\
& & & X \arrow[from=ulll,crossing over] \arrow[from=urrr,crossing over] \arrow[from=ur,to=urrr] \\
B_0 \arrow[rr] \arrow[drrr] & & B_1 \arrow[rr] \arrow[dr] & & B_2 \arrow[rr] \arrow[dl] & & \cdots \arrow[dlll] \\
& & & Y \arrow[from=uu,crossing over] 
\end{tikzcd}
\end{equation*}
If each of the vertical squares is a pullback square, then the square
\begin{equation*}
\begin{tikzcd}
A_\infty \arrow[r] \arrow[d] & X \arrow[d] \\
B_\infty \arrow[r] & Y
\end{tikzcd}
\end{equation*} 
is a pullback square.
\end{cor}

\subsection{Constructing the propositional truncation}\label{sec:propositional-truncation-constr}
The propositional truncation can be used to construct the image of a map, so we construct that first. We construct the propositional truncation of $A$ via a construction called the \define{join construction}, as the colimit of the sequence of join-powers of $A$
\begin{equation*}
  \begin{tikzcd}
    A \arrow[r] & \join{A}{A} \arrow[r] & \join{A}{(\join{A}{A})} \arrow[r] & \cdots
  \end{tikzcd}
\end{equation*}
The join-powers of $A$ are defined recursively on $n$, by taking\footnote{In this definition, the case $A^{\ast1}\defeq A$ is slightly redundant because we have an equivalence
\begin{equation*}
  \join{A}{\emptyt}\simeq A.
\end{equation*}
Nevertheless, it is nice to have that $A^{\ast 1}\jdeq A$.}
\begin{align*}
  A^{\ast0} & \defeq \emptyt \\
  A^{\ast 1} & \defeq A \\
  A^{\ast(n+2)} & \defeq \join{A}{A^{\ast (n+1)}}.
\end{align*}
We will write $A^{\ast\infty}$ for the colimit of the sequence
\begin{equation*}
  \begin{tikzcd}
    A \arrow[r,"\inr"] & \join{A}{A} \arrow[r,"\inr"] & \join{A}{(\join{A}{A})} \arrow[r,"\inr"] & \cdots.
  \end{tikzcd}
\end{equation*}
The sequential colimit $A^{\ast\infty}$ comes equipped with maps $\inseq_n:A^{\ast (n+1)}\to A^{\ast\infty}$, and we will write
\begin{equation*}
  \eta\defeq\inseq_0:A\to A^{\ast\infty}.
\end{equation*}
Our goal is to show $A^{\ast\infty}$ is a proposition, and that $\eta:A\to A^{\ast\infty}$ satisfies the universal property of the propositional truncation of $A$. Before showing that $A^{\ast\infty}$ is indeed a proposition, let us show in two steps that for any proposition $P$, the map
\begin{equation*}
  (A^{\ast\infty}\to P)\to (A\to P)
\end{equation*}
is indeed an equivalence. 

\begin{lem}\label{lem:extend_join_prop}
Suppose $f:A\to P$, where $A$ is any type and $P$ is a proposition.
Then the precomposition function
\begin{equation*}
\blank\circ\inr:(\join{A}{B}\to P)\to (B\to P)
\end{equation*}
is an equivalence, for any type $B$.
\end{lem}

\begin{proof}
  Since the precomposition function
  \begin{equation*}
    \blank\circ\inr:(\join{A}{B}\to P)\to (B\to P)
  \end{equation*}
  is a map between propositions, it suffices to construct a map
  \begin{equation*}
    (B\to P)\to (\join{A}{B}\to P).
  \end{equation*}
  Let $g:B\to P$. Then the square
  \begin{equation*}
    \begin{tikzcd}
      A\times B \arrow[r,"\proj 2"] \arrow[d,swap,"\proj 1"] & B \arrow[d,"g"] \\
      A \arrow[r,swap,"f"] & P
    \end{tikzcd}
  \end{equation*}
  commutes since $P$ is a proposition. Therefore we obtain a map $\join{A}{B}\to P$ by the universal property of the join.
\end{proof}

\begin{prp}\label{prp:universal-property-brck}
Let $A$ be a type, and let $P$ be a proposition. Then the function
\begin{equation*}
\blank\circ \eta : (A^{\ast\infty}\to P)\to (A\to P)
\end{equation*}
is an equivalence. 
\end{prp}

\begin{proof}
  Since the map
  \begin{equation*}
    \blank\circ \eta : (A^{\ast\infty}\to P)\to (A\to P)
  \end{equation*}
  is a map between propositions, it suffices to construct a map in the converse direction.

  Let $f:A\to P$. First, we show by recursion that there are maps
  \begin{equation*}
    f_n:A^{\ast(n+1)}\to P.
  \end{equation*}
  The map $f_0$ is of course just defined to be $f$. Given $f_n:A^{\ast(n+1)}$ we obtain $f_{n+1}:\join{A}{A^{\ast(n+1)}}\to P$ by \cref{lem:extend_join_prop}. Because $P$ is assumed to be a proposition it is immediate that the maps $f_n$ form a cocone with vertex $P$ on the sequence
  \begin{equation*}
    \begin{tikzcd}
      A \arrow[r,"\inr"] & \join{A}{A} \arrow[r,"\inr"] & \join{A}{(\join{A}{A})} \arrow[r,"\inr"] & \cdots.
    \end{tikzcd}
  \end{equation*}
  From this cocone we obtain the desired map $(A^{\ast\infty}\to P)$.
\end{proof}

\begin{prp}\label{prp:isprop-infjp}
The type $A^{\ast\infty}$ is a proposition for any type $A$.
\end{prp}

\begin{proof}
  By \cref{lem:isprop_eq} it suffices to show that
  \begin{equation*}
    A^{\ast\infty}\to \iscontr(A^{\ast\infty}).
  \end{equation*}
  Since the type $\iscontr(A^{\ast\infty})$ is already known to be a proposition by \cref{ex:isprop_istrunc}, it follows from \cref{prp:universal-property-brck} that it suffices to show that
\begin{equation*}
A\to \iscontr(A^{\ast\infty}).
\end{equation*}

Let $x:A$. To see that $A^{\ast\infty}$ is contractible it suffices by \cref{ex:seqcolim_contr} to show that $\inr:A^{\ast n}\to A^{\ast(n+1)}$ is homotopic to the constant function $\const_{\inl(x)}$. However, we get a homotopy $\const_{\inl(x)}\htpy \inr$ immediately from the path constructor $\glue$.  
\end{proof}

All the definitions are now in place to define the propositional truncation of a type.

\begin{defn}
  For any type $A$ we define the type
  \begin{equation*}
    \trunc{-1}{A}\defeq A^{\ast\infty},
  \end{equation*}
  and we define $\eta:A\to\trunc{-1}{A}$ to be the constructor $\seqin_0$ of the sequential colimit $A^{\ast\infty}$. Often we simply write $\brck{A}$ for $\trunc{-1}{A}$.
\end{defn}

The type $\trunc{-1}{A}$ is a proposition by \cref{prp:isprop-infjp}, and
\begin{equation*}
  \eta:A\to\trunc{-1}{A}
\end{equation*}
satisfies the universal property of propositional truncation by \cref{prp:universal-property-brck}.

\begin{prp}
  The propositional truncation operation is functorial in the sense that for any map $f:A\to B$ there is a unique map $\brck{f}:\brck{A}\to\brck{B}$ such that the square
  \begin{equation*}
    \begin{tikzcd}
      A \arrow[r,"f"] \arrow[d,swap,"\eta"] & B \arrow[d,"\eta"] \\
      \brck{A} \arrow[r,swap,"\brck{f}"] & \brck{B}
    \end{tikzcd}
  \end{equation*}
  commutes. Moreover, there are homotopies
  \begin{align*}
    \brck{\idfunc[A]} & \htpy \idfunc[\brck{A}] \\
    \brck{g\circ f} & \htpy \brck{g}\circ\brck{f}.
  \end{align*}
\end{prp}

\begin{proof}
  The functorial action of propositional truncation is immediate by the universal property of propositional truncation. To see that the functorial action preserves the identity, note that the type of maps $\brck{A}\to\brck{A}$ for which the square
  \begin{equation*}
    \begin{tikzcd}
      A \arrow[r,"\idfunc"] \arrow[d,swap,"\eta"] & A \arrow[d,"\eta"] \\
      \brck{A} \arrow[r,densely dotted] & \brck{A}
    \end{tikzcd}
  \end{equation*}
  commutes is contractible. Since this square commutes for both $\brck{\idfunc}$ and for $\idfunc$, it must be that they are homotopic. The proof that the functorial action of propositional truncation preserves composition is similar.
\end{proof}

\subsection{Proving type theoretical replacement}

Our goal is now to show that the image of a map $f:A\to B$ from an essentially small type $A$ into a locally small type $B$ is again essentially small. This property is called the type theoretic replacement property. In order to prove this property, we have to find another construction of the image of a map. In order to make this construction, we define a join operation on maps.

\begin{defn}
  Consider two maps $f:A\to X$ and $g:B\to X$ with a common codomain $X$.
  \begin{enumerate}
  \item We define the type $\join[X]{A}{B}$ as the pushout
    \begin{equation*}
      \begin{tikzcd}
        A\times_X B \arrow[r,"\pi_2"] \arrow[d,swap,"\pi_1"] & B \arrow[d,"\inr"] \\
        A \arrow[r,swap,"\inl"] & \join[X]{A}{B}.
      \end{tikzcd}
    \end{equation*}
  \item We define the \define{join} $\join{f}{g}:\join[X]{A}{B}\to X$ to be the unique map for which the diagram
        \begin{equation*}
      \begin{tikzcd}
        A\times_X B \arrow[r,"\pi_2"] \arrow[d,swap,"\pi_1"] & B \arrow[d,"\inr"] \arrow[ddr,bend left=15,"g"] \\
        A \arrow[r,swap,"\inl"] \arrow[drr,bend right=15,swap,"f"]  & \join[X]{A}{B} \arrow[dr,densely dotted,swap,"\join{f}{g}"] \\
        & & X
      \end{tikzcd}
    \end{equation*}
  \end{enumerate}
\end{defn}

The reason to call the map $\join{f}{g}$ the join of $f$ and $g$ is that the fiber of $\join{f}{g}$ at any $x:X$ is equivalent to the join of the fibers of $f$ and $g$ at $x$.

\begin{lem}
  Consider two maps $f:A\to X$ and $g:B\to X$. Then there is an equivalence
  \begin{equation*}
    \fib{\join{f}{g}}{x}\simeq\join{\fib{f}{x}}{\fib{g}{x}}
  \end{equation*}
  for any $x:X$.
\end{lem}

\begin{proof}
  Consider the commuting cube
  \begin{equation*}
    \begin{tikzcd}
      & \fib{f}{x}\times\fib{g}{x} \arrow[dl] \arrow[dr] \arrow[d] \\
      \fib{f}{x} \arrow[d] & A\times_X B \arrow[dl] \arrow[dr] & \fib{g}{x} \arrow[d] \arrow[dl,crossing over] \\
      A \arrow[dr] & \unit \arrow[from=ul,crossing over] \arrow[d] & B \arrow[dl] \\
      & X
    \end{tikzcd}
  \end{equation*}
  In this cube, the bottom square is a canonical pullback square. The two squares in the front are pullbacks by \cref{lem:fib_pb}, and the top square is a pullback square by \cref{lem:prod_pb}. Therefore it follows by \cref{rmk:strongly-cartesian} that all the faces of this cube are pullback squares, and hence by \cref{thm:effectiveness-pullback} we obtain that the square
  \begin{equation*}
    \begin{tikzcd}
      \join{\fib{f}{x}}{\fib{g}{x}} \arrow[d,densely dotted] \arrow[r] & \unit \arrow[d] \\
      \join[X]{A}{B} \arrow[r,swap,"\join{f}{g}"] & X
    \end{tikzcd}
  \end{equation*}
  is a pullback square. Now the claim follows by the uniqueness of pullbacks, which was shown in \cref{cor:uniquely-unique-pullback}.
\end{proof}

\begin{lem}
Consider a map $f:A\to X$, an embedding $m:U\to X$, and $h:\mathrm{hom}_X(f,m)$. Then the map
\begin{equation*}
\mathrm{hom}_X(\join{f}{g},m)\to \mathrm{hom}_X(g,m)
\end{equation*}
is an equivalence for any $g:B\to X$.
\end{lem}

\begin{proof}
Note that both types are propositions, so any equivalence can be used to prove the claim. Thus, we simply calculate
\begin{align*}
\mathrm{hom}_X(\join{f}{g},m) & \eqvsym \prd{x:X}\fib{\join{f}{g}}{x}\to \fib{m}{x} \\
& \eqvsym \prd{x:X}\join{\fib{f}{x}}{\fib{g}{x}}\to\fib{m}{x} \\
& \eqvsym \prd{x:X}\fib{g}{x}\to\fib{m}{x} \\
& \eqvsym \mathrm{hom}_X(g,m).
\end{align*}
The first equivalence holds by \cref{ex:triangle_fib}; the second equivalence holds by \cref{ex:fib_join}, also using \cref{ex:equiv_precomp,lem:postcomp_equiv} where we established that that pre- and postcomposing by an equivalence is an equivalence; the third equivalence holds by \cref{lem:extend_join_prop,lem:postcomp_equiv}; the last equivalence again holds by \cref{ex:triangle_fib}.
\end{proof}

For the construction of the image of $f:A\to X$ we observe that if we are given an embedding $m:U\to X$ and a map $(i,I):\mathrm{hom}_X(f,m)$, then $(i,I)$ extends uniquely along $\inr:A\to \join[X]{A}{A}$ to a map $\mathrm{hom}_X(\join{f}{f},m)$. This extension again extends uniquely along $\inr:\join[X]{A}{A}\to \join[X]{A}{(\join[X]{A}{A})}$ to a map $\mathrm{hom}_X(\join{f}{(\join{f}{f})},m)$ and so on, resulting in a diagram of the form
\begin{equation*}
\begin{tikzcd}
A \arrow[dr] \arrow[r,"\inr"] & \join[X]{A}{A} \arrow[d,densely dotted] \arrow[r,"\inr"] & \join[X]{A}{(\join[X]{A}{A})} \arrow[dl,densely dotted] \arrow[r,"\inr"] & \cdots \arrow[dll,densely dotted,bend left=10] \\
& U
\end{tikzcd}
\end{equation*}

\begin{defn}
Suppose $f:A\to X$ is a map. Then we define the \define{fiberwise join powers} 
\begin{equation*}
f^{\ast n}:A_X^{\ast n} X.
\end{equation*}
\end{defn}

\begin{constr}
Note that the operation $(B,g)\mapsto (\join[X]{A}{B},\join{f}{g})$ defines an endomorphism on the type
\begin{equation*}
\sm{B:\UU}B\to X.
\end{equation*}
We also have $(\emptyt,\ind{\emptyt})$ and $(A,f)$ of this type. For $n\geq 1$ we define
\begin{align*}
A_X^{\ast (n+1)} & \defeq \join[X]{A}{A_X^{\ast n}} \\
f^{\ast (n+1)} & \defeq \join{f}{f^{\ast n}}.\qedhere
\end{align*}
\end{constr}

\begin{defn}
We define $A_X^{\ast\infty}$ to be the sequential colimit of the type sequence
\begin{equation*}
\begin{tikzcd}
A_X^{\ast 0} \arrow[r] & A_X^{\ast 1} \arrow[r,"\inr"] & A_X^{\ast 2} \arrow[r,"\inr"] & \cdots.
\end{tikzcd}
\end{equation*}
Since we have a cocone
\begin{equation*}
\begin{tikzcd}
A_X^{\ast 0} \arrow[r] \arrow[dr,swap,"f^{\ast 0}" near start] & A_X^{\ast 1} \arrow[r,"\inr"] \arrow[d,swap,"f^{\ast 1}" near start] & A_X^{\ast 2} \arrow[r,"\inr"] \arrow[dl,swap,"f^{\ast 2}" xshift=1ex] & \cdots \arrow[dll,bend left=10] \\
& X
\end{tikzcd}
\end{equation*}
we also obtain a map $f^{\ast\infty}:A_X^{\ast\infty}\to X$ by the universal property of $A_X^{\ast\infty}$. 
\end{defn}

\begin{lem}\label{lem:finfjp_up}
Let $f:A\to X$ be a map, and let $m:U\to X$ be an embedding. Then the function
\begin{equation*}
\blank\circ \seqin_0: \mathrm{hom}_X(f^{\ast\infty},m)\to \mathrm{hom}_X(f,m)
\end{equation*}
is an equivalence. 
\end{lem}

\begin{thm}\label{lem:isprop_infjp}
For any map $f:A\to X$, the map $f^{\ast\infty}:A_X^{\ast\infty}\to X$ is an embedding that satisfies the universal property of the image inclusion of $f$.
\end{thm}

\begin{lem}
Consider a commuting square
\begin{equation*}
\begin{tikzcd}
A \arrow[r] \arrow[d] & B \arrow[d] \\
C \arrow[r] & D.
\end{tikzcd}
\end{equation*}
\begin{enumerate}
\item If the square is cartesian, $B$ and $C$ are essentially small, and $D$ is locally small, then $A$ is essentially small.
\item If the square is cocartesian, and $A$, $B$, and $C$ are essentially small, then $D$ is essentially small. 
\end{enumerate}
\end{lem}

\begin{cor}
Suppose $f:A\to X$ and $g:B\to X$ are maps from essentially small types $A$ and $B$, respectively, to a locally small type $X$. Then $A\times_X B$ is again essentially small. 
\end{cor}

\begin{lem}
Consider a type sequence
\begin{equation*}
\begin{tikzcd}
A_0 \arrow[r,"f_0"] & A_1 \arrow[r,"f_1"] & A_2 \arrow[r,"f_2"] & \cdots
\end{tikzcd}
\end{equation*}
where each $A_n$ is essentially small. Then its sequential colimit is again essentially small. 
\end{lem}

\begin{thm}\label{thm:replacement}
  For any map $f:A\to B$ from an essentially small type $A$ into a locally small type $B$, the image of $f$ is again essentially small.
\end{thm}

\begin{cor}
  Consider a $\UU$-small type $A$, and an equivalence relation $R$ over $A$ valued in the $\UU$-small propositions. Then the set quotient $A/R$ is essentially small.
\end{cor}

\begin{exercises}
\exercise \label{ex:seqcolim_shift}
Show that the sequential colimit of a type sequence
\begin{equation*}
\begin{tikzcd}
A_0 \arrow[r,"f_0"] & A_1 \arrow[r,"f_1"] & A_2 \arrow[r,"f_2"] & \cdots
\end{tikzcd}
\end{equation*}
is equivalent to the sequential colimit of its shifted type sequence
\begin{equation*}
\begin{tikzcd}
A_1 \arrow[r,"f_1"] & A_2 \arrow[r,"f_2"] & A_3 \arrow[r,"f_3"] & \cdots.
\end{tikzcd}
\end{equation*}
  \exercise Let
  \begin{tikzcd}
    P_0 \arrow[r] & P_1 \arrow[r] & P_2 \arrow[r] & \cdots
  \end{tikzcd}
  be a sequence of propositions. Show that
  \begin{equation*}
    \eqv{\colim_n(P_n)}{\exists_{(n:\N)} P_n}.
  \end{equation*}
\exercise \label{ex:seqcolim_contr}Consider a type sequence
\begin{equation*}
\begin{tikzcd}
A_0 \arrow[r,"f_0"] & A_1 \arrow[r,"f_1"] & A_2 \arrow[r,"f_2"] & \cdots
\end{tikzcd}
\end{equation*}
and suppose that $f_n\htpy \mathsf{const}_{a_{n+1}}$ for some $a_n:\prd{n:\N}A_n$. Show that the sequential colimit is contractible.
\exercise Define the $\infty$-sphere $\sphere{\infty}$ as the sequential colimit of
\begin{equation*}
\begin{tikzcd}
\sphere{0} \arrow[r,"f_0"] & \sphere{1} \arrow[r,"f_1"] & \sphere{2} \arrow[r,"f_2"] & \cdots
\end{tikzcd}
\end{equation*}
where $f_0:\sphere{0}\to\sphere{1}$ is defined by $f_0(\bfalse)\jdeq \inl(\ttt)$ and $f_0(\btrue)\jdeq \inr(\ttt)$, and $f_{n+1}:\sphere{n+1}\to\sphere{n+2}$ is defined as $\susp(f_n)$. Use \cref{ex:seqcolim_contr} to show that $\sphere{\infty}$ is contractible.
\exercise Consider a type sequence
\begin{equation*}
\begin{tikzcd}
A_0 \arrow[r,"f_0"] & A_1 \arrow[r,"f_1"] & A_2 \arrow[r,"f_2"] & \cdots
\end{tikzcd}
\end{equation*}
in which $f_n:A_n\to A_{n+1}$ is weakly constant in the sense that
\begin{equation*}
\prd{x,y:A_n} f_n(x)=f_n(y)
\end{equation*}
Show that $A_\infty$ is a mere proposition.
\exercise Show that $\N$ is the sequential colimit of
\begin{equation*}
  \begin{tikzcd}
    \Fin(0) \arrow[r,"\inl"] & \Fin(1) \arrow[r,"\inl"] & \Fin(2) \arrow[r,"\inl"] & \cdots.
  \end{tikzcd}
\end{equation*}
\end{exercises}

\section{Parameterized partial combinatory algebras}
\label{sec:parameterized-part-comb}

We seek a topos in which a Miller sequence is an epimorphism from the natural numbers to the Dedekind reals. Some sort of realizability model seems appropriate, although it cannot be an ordinary realizability topos, as those validate the axiom of countable choice.
%
\Cref{def:sequence-computable} specifies that $n \in \NN$ realizes $x \in [0,1]$ when it does so \emph{parameterically} in oracles representing $a : \NN \to [0,1]$. Therefore, in the present section we develop a general notion of parameterized computational models.

Let us take a moment to introduce notation that is commonly used in realizability theory. We already wrote $f : A \parto B$ to indicate a \defemph{partial map}, which is a map $f : A' \to B$ defined on a subset $A' \subseteq B$. For $x \in A$ we write $\defined{f(x)}$ when $f(x)$ is defined, i.e., when $x \in A'$. More generally, we write $\defined{e}$ when the expression~$e$ is well-defined, and hence so are all of its subexpressions. If $e_1$ and $e_2$ are two possibly undefined expressions, then $e_1 \kleq e_2$ means that if one is defined then so is the other and they are equal. In contrast, $e_1 = e_2$ asserts that both sides are defined and equal.

In realizability theory logical statements are witnessed by \emph{realizers}, which may be numbers, $\lambda$-terms, sequences or other data. A realizer is meant to represent computational evidence of a statement.
For instance, a realizer for $\some{x} \phi(x)$ encodes a specific $a$ for which $\phi(a)$ holds as well as a realizer for $\phi(a)$, and a realizer for $\phi \to \psi$ encodes a procedure for converting realizers for $\phi$ into realizers for $\psi$. In~\cref{sec:heyt-prealg-struct} we shall make these ideas precise.

In Stephen Kleene's original realizability interpretation of Heyting arithmetic~\cite{KleeneSC:intint} the realizers were numbers, whereas in a typical modern framework they are elements of a structure first defined by Solomon Feferman~\cite{feferman75}:

\begin{definition}
  \label{def:pca}%
  A \defemph{partial combinatory algebra (pca)} is given by a \defemph{carrier set} $\AA$, and a partial \defemph{application} operation ${\app} : \AA \times \AA \parto \AA$, such that there exist \defemph{basic combinators} $\combK, \combS \in \AA$ satisfying, for all $\R{a}, \R{b}, \R{c} \in \AA$,
  %
  \begin{align*}
    &\defined{(\combK \app \R{a})}, &
    (\combK \app \R{a}) \app \R{b} &= \R{a}, \\
    &\defined{((\combS \app \R{a}) \app \R{b})}, &
    ((\combS \app \R{a}) \app \R{b}) \app \R{c} &\kleq (\R{a} \app \R{c}) \app (\R{b} \app \R{c}).
  \end{align*}
\end{definition}

\noindent
To make notation more economical, we write application $\R{a} \app \R{b}$ as juxtaposition $\R{a} \, \R{b}$ and associate it to the left, $\R{a} \, \R{b} \, \R{c} = (\R{a} \, \R{b}) \, \R{c}$.

A non-trivial pca has much richer structure as it may seem at first sight.
%
For instance, we may encode in it the natural numbers and partial computable functions, as we shall for parameterized pcas in \cref{sec:progr-with-ppcas}.

\begin{example}
  \label{ex:pca-K-1}%
  The so-called Kleene's first algebra is the pca with the carrier set $\KK_1 \defeq \NN$ and application $m \cdot n \defeq \pr{m}(n)$, where $\pr{m}$ is the $m$-th partial computable map.
  %
  For any oracle $\alpha \in \Cantor$ the relativized version $\KK[\alpha]_1$ has the same carrier set and application $m \cdot n \defeq \pr[\alpha]{m}(n)$.
\end{example}

We refer to~\cite{oosten08:_realiz} for further examples of pcas and press on to the definition of parameterized pcas.

\begin{definition}
  \label{def:ppca}%
  A \defemph{parameterized partial combinatory algebra (ppca)} is given by
  %
  \begin{itemize}
  \item a \defemph{carrier set} $\AA$, whose elements are called \defemph{realizers},
  \item a non-empty \defemph{parameter set} $\PP$, whose elements are called \defemph{parameters},
  \item a partial \defemph{application} operation ${\app} : \PP \times \AA \times \AA \parto \AA$,
  \end{itemize}
  %
  such that there exist \defemph{basic combinators} $\combK, \combS \in \AA$, satisfying, for all $p, q \in \PP$ and $\R{a}, \R{b}, \R{c} \in \AA$,
  %
  \begin{align*}
    &\combK \app[p] \R{a} = \combK \app[q] \R{a},
    &
    \combK \app[p] \R{a} \app[p] \R{b} &= \R{a},
    \\
    &\combS \app[p] \R{a} = \combS \app[q] \R{a},
    &
    \combS \app[p] \R{a} \app[p] \R{b} \app[p] \R{c} &\kleq (\R{a} \app[p] \R{c}) \app[p] (\R{b} \app[p] \R{c}),
    \\
    &\combS \app[p] \R{a} \app[p] \R{b} = \combS \app[q] \R{a} \app[q] \R{b}.
  \end{align*}
  %
\end{definition}

The equations in the left column imply $\defined{(\combK \app[p] \R{a})}$ and $\defined{(\combS \app[p] \R{a} \app[p] \R{b})}$ for all $p \in \PP$ and $\R{a}, \R{b} \in \AA$.
%
For better readability we continued to write application as juxtaposition and let it associate to the left,
but we still need a better way of displaying the parameters, which we do by writing $p \at e$ to specify
that all applications in expression~$e$ should use parameter~$p$. We assign~$\at$ a lower precedence than to application so that $p \at e_1 \app e_2 = p \at (e_1 \app e_2)$. For example, the above equations can be written as
%
\begin{align*}
  &p \at \combK \, \R{a} = q \at \combK \, \R{a},
  &p \at \combK \, \R{a} \, \R{b} &= \R{a},
  \\
  &p \at \combS \, \R{a} = q \at \combS \, \R{a},
  &p \at \combS \, \R{a} \, \R{b} \, \R{c} &\kleq p \at (\R{a} \, \R{c}) \, (\R{b} \, \R{c}),
  \\
  &p \at \combS \, \R{a} \, \R{b} = q \at \combS \, \R{a} \, \R{b}.
\end{align*}
%
We sometimes write parentheses around $p \at e$ to improve readability, especially in equations.
A formal account of notation $p \at e$ is given in \cref{sec:comb-compl-ppcas}.


When no confusion may arise, we take the liberty of referring to a ppca $(\AA, \PP, {\cdot})$ just by the pair $(\AA, \PP)$.

\begin{example}
  An ordinary pca may be construed as a ppca whose parameter set is a singleton.
\end{example}

\begin{example}
  \label{ex:oracle-ppca}
  %
  The following is our main motivating example.
  %
  Recall from \cref{sec:oracle-comp-maps} that $\pr[\alpha]{m}$ stands for the partial $\alpha$-computable map coded by~$m$.
  %
  Let the carrier of the ppca be $\KK \defeq \NN$,
  the parameter set any non-empty set of oracles $\PP \subseteq \Cantor$,
  and application $m \app[\alpha] n \defeq \pr[\alpha]{m}(n)$.

  The combinator~$\combK$ is the code of a machine which accepts input~$n$ and outputs the code of a machine that always outputs~$n$. Such a machine does not consult the oracle, and neither does the machine that always outputs~$n$, hence $\combK \app[\alpha] n = \combK \app[\beta] n$ for all $n \in \NN$.

  To obtain~$\combS$, we apply the relativized smn theorem~\cite[Sect.~III.1.5]{soare87:_recur_enumer_sets_degrees} to first get a computable map $f : \NN \times \NN \to \NN$ such that
  %
  \begin{equation*}
    \pr[\alpha]{f(k, m)}(n) \simeq \pr[\alpha]{\pr[\alpha]{k}(n)}(\pr[\alpha]{m}(n)).
  \end{equation*}
  %
  We apply the theorem again to get a computable $g : \NN \to \NN$ such that
  %
  $\pr[\alpha]{g(k)}(m) = f(k, m)$, and let $\combS \in \NN$ be such that $\pr[\alpha]{\combS} = g$.
  %
  For all $\alpha \in \PP$ and $k, m \in \NN$ we have
  %
  \begin{equation*}
    \combS \app[\alpha] k =
    \pr[\alpha]{\combS}(k) =
    g(k)
  \end{equation*}
  %
  and
  %
  \begin{equation*}
    \combS \app[\alpha] k \app[\alpha] m =
    \pr[\alpha]{\pr[\alpha]{\combS}(k)}(m) =
    \pr[\alpha]{g(k)}(m) =
    f(k, m).
  \end{equation*}
  %
  Being computable, $g$ and $f$ do not depend on the oracle, therefore
  $\combS \app[\alpha] k = \combS \app[\beta] k$
  and
  $\combS \app[\alpha] k \app[\alpha] m = \combS \app[\beta] k \app[\beta] m$ for all $\beta \in \PP$.
  Finally, the defining equation for~$f$ guarantees that
  $\combS \app[\alpha] k \app[\alpha] m \app[\alpha] n \simeq
   (k \app[\alpha] n) \app[\alpha] (m \app[\alpha] n)$.
\end{example}


\begin{example}
  \label{ex:general-oracle-ppca}%
  %
  The previous example generalizes to Jaap van Oosten's construction~\cite[Thm~1.7.5]{oosten08:_realiz} which from any pca $(\AA, {\cdot})$ and a partial map $\xi : \AA \parto \AA$ constructs a new pca $(\AA^\xi, {\app[\xi]})$ with~$\xi$ acting as an oracle.
  The carrier set is unchanged $\AA^\xi \defeq \AA$, while application~$\app[\xi]$ represents a dialogue between a computation in~$\AA$ and the oracle~$\xi$.
  %
  When the construction is applied to $\KK_1$ and $\alpha \in \Cantor$, we obtain a pca that is (isomorphic to)
  the relativized pca $\KK[\alpha]_1$ from \cref{ex:pca-K-1}.

  As it turns out, the construction is uniform in~$\xi$, in the sense that $\combK_\xi, \combS_\xi \in \AA^\xi$ do not depend on~$\xi$, and neither do $\combK_\xi \app[\xi] \R{a}$, $\combS_\xi \app[\xi] \R{a}$, and $\combS_\xi \app[\xi] \R{a} \app[\xi] \R{b}$.
  %
  The conditions for having a ppca are thus met: the carrier set is $\AA$ and the parameter set is any non-empty
  subset $\PP \subseteq \AA \parto \AA$.
\end{example}

% Andrew Swan says every ppca is of the above form.


\subsection{Combinatory completeness of ppcas}
\label{sec:comb-compl-ppcas}

Partial combinatory algebras have the so-called property of \emph{combinatory completeness}.
We formulate an analogous notion for parameterized pcas.

The set of~\defemph{expressions in variables $x_1, \ldots, x_n$} over a ppca $(\AA, \PP)$ is defined inductively:
any variable $x_i$ is an expression, any constant $\R{a} \in \AA$ is an expression, and a formal application $e_1 \cdot e_2$ is an expression if~$e_1$ and~$e_2$ are.
%
We continue to write application as juxtaposition and associate it to the left.
%
An expression~$e$ in no variables is \defemph{closed}. For any $p \in \PP$ and a closed expression~$e$,
define $p \at e$ recursively by
%
\begin{align*}
  p \at \R{a} &\defeq \R{a} &&\text{if $\R{a} \in \AA$,}
  \\
  (p \at e_1 \app e_2) &\defeq (p \at e_1) \app[p] (p \at e_2)
  &&\text{if $e_1$ and $e_2$ are closed expressions.}
\end{align*}
%
Note that $p \at e$ may be undefined.

For a variable $x$ and an expression $e$, let the \defemph{abstraction} $\abstr{x} e$ be the expression defined inductively as
%
\begin{align*}
  \abstr{x} y &\defeq \combK \, y & &\text{if $y$ is a variable distinct from $x$} \\
  \abstr{x} x &\defeq \combS \, \combK \, \combK \\
  \abstr{x} \R{a} &\defeq \combK \, a & &\text{for a constant $\R{a} \in \AA$} \\
  \abstr{x} e_1 e_2 &\defeq \combS \, (\abstr{x} e_1) \, (\abstr{x} e_2).
\end{align*}
%
We write $e[\R{a}_1/x_1, \ldots, \R{a}_n/x_n]$ for $e$ with $\R{a}_i$'s substituted for~$x_i$'s. We abbreviate the substitution $[\R{a}_1/x_1, \ldots, \R{a}_n/x_n]$ as $[\vec{\R{a}}/\vec{x}]$, which allows us to write $e[\vec{\R{a}}/\vec{x}]$. To be precise, substitution is defined as follows:
%
\begin{align*}
  x_i [\vec{\R{a}}/\vec{x}] &\defeq \R{a}_i \\
  y [\vec{\R{a}}/\vec{x}] &\defeq y &&\text{if $y \not\in \set{x_1, \ldots, x_n}$} \\
  \R{b} [\vec{\R{a}}/\vec{x}] &\defeq \R{b} &&\text{if $\R{b} \in \AA$} \\
  (e_1 \, e_2) [\vec{\R{a}}/\vec{x}] &\defeq (e_1[\vec{\R{a}}/\vec{x}]) \, (e_2[\vec{\R{a}}/\vec{x}]).
\end{align*}

\begin{lemma}
  \label{lem:abstr-subst-commute}%
  If $y \not\in \set{x_1, \ldots, x_n}$ then $(\abstr{y} e)[\vec{\R{a}}/\vec{x}] = \abstr{y} (e[\vec{\R{a}}/\vec{x}])$.
\end{lemma}

\begin{proof}
  A straightforward induction on the structure of~$e$.
\end{proof}


\begin{lemma}
  \label{lem:abstr-p-defined}
  For any expression~$e$ in variables $x_1, \ldots, x_n, y$, the value $p \at (\abstr{y} e)[\vec{\R{a}}/\vec{x}]$ is defined for all $p \in \PP$ and $\R{a}_1, \ldots, \R{a}_n \in \AA$.
\end{lemma}

\begin{proof}
  We proceed by induction on the structure of~$e$:
  % 
  \begin{itemize}
  \item if $e = x_i$ then $p \at (\abstr{y} e)[\vec{\R{a}}/\vec{x}] = \combK \app[p] \R{a}_i$, which is defined,
  \item if $e = y$ then $p \at (\abstr{y} e)[\vec{\R{a}}/\vec{x}] = \combS \app[p] \combK \app[p] \combK$, which is defined,
  \item if $e = \R{a}$ for $\R{a} \in \AA$ then $p \at (\abstr{y} e)[\vec{\R{a}}/\vec{x}] = \combK \app[p] \R{a}$, which is defined,
  \item if $e = e_1 \app e_2$ then
    % 
    $
    p \at (\abstr{y} e)[\vec{\R{a}}/\vec{x}] =
    \combS \app[p] ((\abstr{y} e_1)[\vec{\R{a}}/\vec{x}]) \app[p] ((\abstr{y} e_2)[\vec{\R{a}}/\vec{x}])
    $,
    % 
    which is defined because by induction hypotheses both arguments of~$\combS$ are defined.
    \qedhere
  \end{itemize}
\end{proof}

\begin{lemma}
  \label{lem:abstr-compute}%
  %
  For any expression $e$ in variables $x_1, \ldots, x_n, y$, parameter $p \in \PP$, and $\R{a}_1, \ldots, \R{a}_n, \R{b} \in \AA$, 
  % 
  \begin{equation*}
    p \at ((\abstr{y} e) \, \R{b})[\vec{\R{a}}/\vec{x}] \kleq p \at e[\vec{\R{a}}/\vec{x}, \R{b}/y].
  \end{equation*}
\end{lemma}

\begin{proof}
  We proceed by induction on the structure of~$e$. If $e = x_i$ then
%
\begin{equation*}
  p \at ((\abstr{y} x_i) \, \R{b})[\vec{\R{a}}/\vec{x}] \kleq
  p \at \combK \, \R{a}_i \, \, \R{b} =
  p \at \R{a}_i =
  p \at x_i[\vec{\R{a}}/\vec{x}, \R{b}/y].
\end{equation*}
%
If $e = y$ then
%
\begin{equation*}
  p \at ((\abstr{y} y) \, \R{b})[\vec{\R{a}}/\vec{x}] \kleq
  p \at \combS \, \combK \, \combK \, \R{b} =
  p \at \R{b} =
  p \at y[\vec{\R{a}}/\vec{x}, \R{b}/y].
\end{equation*}
%
If $e = \R{a} \in \AA$ then
%
\begin{equation*}
  p \at ((\abstr{y} \R{a}) \, \R{b})[\vec{\R{a}}/\vec{x}] \kleq
  p \at \combK \, \R{a} \, \, \R{b} =
  p \at \R{a} =
  p \at \R{a}[\vec{\R{a}}/\vec{x}, \R{b}/y].
\end{equation*}
%
Finally, if $e = e_1 \, e_2$ then
%
\begin{align*}
  p \at ((\abstr{y} e_1 \, e_2) \, \R{b})[\vec{\R{a}}/\vec{x}]
  &\kleq
  p \at \combS \, ((\abstr{y} e_1)[\vec{\R{a}}/\vec{x}]) \, ((\abstr{y} e_2)[\vec{\R{a}}/\vec{x}]) \, \R{b} \\
  &\kleq
  p \at ((\abstr{y} e_1)[\vec{\R{a}}/\vec{x}] \, \R{b}) \, ((\abstr{y} e_2)[\vec{\R{a}}/\vec{x}] \, \R{b}) \\
  &\kleq
  p \at (e_1[\vec{\R{a}}/\vec{x},\R{b}/y]) \, (e_2[\vec{\R{a}}/\vec{x},\R{b}/y]) \\
  &\kleq
  p \at (e_1 \, e_2)[\vec{\R{a}}/\vec{x},\R{b}/y].
\end{align*}
%
The passage from the first to the second row is secured by \cref{lem:abstr-p-defined}, and from the third to the fourth by the induction hypotheses.
\end{proof}

Let us give a name to those expressions that are independent of the parameter.

\begin{definition}
  \label{def:uniform}%
  A closed expression~$e$ is \defemph{uniform} when $p \at e = q \at e$ for all $p, q \in \PP$.
  When this is the case, there is a unique $\ucode{e} \in \AA$ such that $p \at e = \ucode{e}$ for all $p \in \PP$.
\end{definition}

\Cref{def:ppca} postulates that $\combK \, \R{a}$ and $\combS \, \R{a} \, \R{b}$ are uniform for all $\R{a}, \R{b} \in \AA$. In subsequent calculations we shall frequently use the fact that $p \at \ucode{e} = p \at e$ when~$e$ is uniform.

\begin{lemma}
  \label{lem:abstr-uniform}%
  A closed abstraction $\abstr{x} e$ is uniform.
\end{lemma}

\begin{proof}
  We proceed by induction on the structure of~$e$.
  If $e$ is the variable $x$ then $\abstr{x} e = \combS \, \combK \, \combK$, which is uniform.
  If $e$ is a constant $\R{a} \in \AA$ then $\abstr{x} e = \combK \, \R{a}$, which is uniform.
  If $e = e_1 \, e_2$ then $\abstr{x} e = \combS \, (\abstr{x} e_1) \, (\abstr{x} e_2)$, which is uniform by induction hypotheses.
\end{proof}


\begin{theorem}[Combinatory completeness]
  For any expression~$e$ over a ppca $(\AA, \PP)$ in variables~$x_1, \ldots, x_n, x_{n+1}$, there is $e^{*} \in \AA$ such that, for all $p \in \PP$ and $\R{a}_1, \ldots, \R{a}_{n+1} \in \AA$, the expression $e^{*} \, \R{a}_1 \cdots \R{a}_n$ is uniform and
  \begin{equation*}
    (p \at e^{*} \, \R{a}_1 \cdots \R{a}_{n+1})
    \kleq
    (p \at e[\R{a}_1/x_1, \ldots, \R{a}_{n+1}/x_{n+1}]).
  \end{equation*}
\end{theorem}

\begin{proof}
  The usual proof for ordinary partial combinatory algebras can be mimicked.
  %
  Let $e_{n+1} \defeq \abstr{x_{n+1}} e$ and $e_k = \abstr{x_k} e_{k+1}$ for $k = 1, \ldots, n$.
  By \cref{lem:abstr-uniform}, $e_1$ is uniform, so $e^{*} \defeq \ucode{e_1}$ is well defined,
  and we claim it is the element we are looking for.
  %
  Given $p \in \PP$ and $\R{a}_1, \ldots, \R{a}_{n+1} \in \AA$, \cref{lem:abstr-compute,lem:abstr-subst-commute} imply
  %
  \begin{align*}
    p \at e_1 \, \R{a}_1 \cdots \R{a}_n
    &\kleq p \at (e_2[\R{a}_1/x_1]) \, \R{a}_2 \cdots \R{a}_n \\
    &\kleq p \at (e_3[\R{a}_1/x_1, \R{a}_2/x_2]) \, \R{a}_3 \cdots \R{a}_n \\
    &\qquad \vdots \\
    &\kleq p \at (\abstr{x_{n+1}} e)[\R{a}_1/x_1, \ldots, \R{a}_n/x_n] \\
    &\kleq p \at \abstr{x_{n+1}} (e [\R{a}_1/x_1, \ldots, \R{a}_n/x_n]).
  \end{align*}
  %
  The last row is defined by \cref{lem:abstr-p-defined} and uniform by \cref{lem:abstr-uniform},
  therefore so is the first one.
  Finally, \cref{lem:abstr-compute} implies
  %
  \begin{equation*}
    p \at e_1 \, \R{a}_1 \cdots \R{a}_{n+1}
    \kleq 
    p \at e[\R{a}_1/x_1, \ldots, \R{a}_{n+1}/x_{n+1}]. \qedhere
  \end{equation*}
\end{proof}

\subsection{Programming with ppcas}
\label{sec:progr-with-ppcas}

Combinatory completeness can be used to write complex programs in any ppca, just like in ordinary partial combinatory algebras. For example, $\comb{id} \defeq \ucode{\abstr{x} x} = \ucode{\comb{s} \, \comb{k} \, \comb{k}}$ realizes the identity map.
%
More interesting are pairing, projections, booleans and the conditional:
%
\begin{align*}
  \combPair &\defeq \ucode{\abstr{x y z}{z\, x\, y}},
  &
  \combIf &\defeq \ucode{\abstr{x} x},
  \\
  \combFst &\defeq \ucode{\abstr{z}{z \, (\abstr{x\,y} x)}},
  &
  \combTrue &\defeq \ucode{\abstr{x\,y} x},
  \\
  \combSnd &\defeq \ucode{\abstr{z}{z \, (\abstr{x\,y} y)}},
  &
  \combFalse &\defeq \ucode{\abstr{x\,y} y}.
\end{align*}
%
These are all uniform by \cref{lem:abstr-uniform}. They satisfy the expected equations parameter-wise, for all $p \in \PP$ and $\R{a}, \R{b} \in \AA$:
%
\begin{align*}
  (p \at \combFst \, (\combPair \, \R{a} \, \R{b})) &= \R{a}, &
  (p \at \combIf \, \combTrue \, \R{a} \, \R{b}) &= \R{a}, \\
  (p \at \combSnd \, (\combPair \, \R{a} \, \R{b})) &= \R{b}, &
  (p \at \combIf \, \combFalse \, \R{a} \, \R{b}) &= \R{b}.
\end{align*}

Natural numbers are encoded as \defemph{Curry numerals}:
%
\begin{align*}
  \numeral{0} &\defeq \ucode{\combS\, \combK\, \combK},
  &
  \numeral{n+1} &\defeq \ucode{\combPair \, \combFalse \, \numeral{n}}
\end{align*}
%
Successor, predecessor and zero-testing are defined as
%
\begin{align}
  \comb{succ} &\defeq \ucode{\abstr{x}{\combPair \, \combFalse \,x}}, \label{eq:comb-succ} \\ 
  \comb{iszero} &\defeq \ucode{\combFst}, \notag\\
  \comb{pred} &\defeq
  \ucode{\abstr{x}{\combIf\, (\comb{iszero}\, x)\, \numeral{0}\, (\combSnd\, x)}}. \notag
\end{align}
%
These are again uniform and satisfy the expected equations.

To get recursive definitions going, we define the fixed-point combinators~$\comb{Y}$ and~$\comb{Z}$:
%
%
\begin{align*}
  \R{W} &\defeq \ucode{\abstr{x \, y} y \, (x \, x \, y)},
  &
  \comb{Y} &\defeq \ucode{\R{W} \, \R{W}},
  \\
  \R{X} &\defeq \ucode{\abstr{x\, y\, z} y \, (x \, x \, y) \, z},
  &
  \comb{Z} &\defeq \ucode{\R{X} \, \R{X}}.
\end{align*}
%
Then for all $p \in \PP$ and $\R{f}, \R{a} \in \AA$, $\comb{Z} \, \R{f}$ is uniform,
%
\begin{equation*}
  p \at \comb{Y} \, \R{f} \kleq p \at \R{f} \, (\comb{Y} \, \R{f})
  \qquad\text{and}\qquad
  p \at \comb{Z} \, \R{f} \, \R{a} \kleq p \at \R{f} \, (\comb{Z} \, \R{f}) \, \R{a}.
\end{equation*}
%
% Verification that Y and Z work:
%
% p | Y f ≃
% p | W W f ≃
% p | f (W W f) ≃
% p | f (Y f)
%
% p | Z f a ≃
% p | X X f a ≃
% p | f (X X f) a ≃
% p | f (Z f a) a.
%
For instance, by repeatedly using \cref{lem:abstr-compute} we compute
%
\begin{equation*}
  p \at \comb{Z} \, \R{f} \, \R{a} \kleq
  p \at \R{X} \, \R{X} \, \R{f} \, \R{a} \kleq
  p \at \R{f} \, (\R{X} \, \R{X} \, \R{f}) \, \R{a} \kleq
  p \at \R{f} \, (\comb{Z} \, \R{f}) \, \R{a}.
\end{equation*}
%
With $\comb{Y}$ in hand primitive recursion on natural numbers is realized as
%
\begin{equation*}
  \comb{primrec} \defeq
  \ucode{\abstr{x \, \R{f} \, m} ((\comb{Z} \, \R{R}) \, x \, \R{f} \, m \, \comb{id})}
\end{equation*}
%
where
%
\begin{equation*}
  \R{R} \defeq \ucode{
      \abstr{r \, x \, \R{f} \, m}
      \comb{if} \, (\comb{iszero} \, m) \,
          (\combK \, x) \,
          (\abstr{y} \R{f} \, (\comb{pred} \, m) \, (r \, x \, \R{f} \, (\comb{pred} \, m) \, \comb{id}))
  }.
\end{equation*}
%
It satisfies, for all $p \in \PP$, $\R{a}, \R{f} \in \AA$ and $n \in \NN$,
%
\begin{align*}
  (p \at \comb{primrec} \, \R{a} \, \R{f} \, \numeral{0}) &= \R{a},
  &
  (p \at \comb{primrec} \, \R{a} \, \R{f} \, \numeral{n + 1}) &\kleq
  \R{f} \, \numeral{n} \, (\comb{primrec} \, \R{a} \, \R{f} \, \numeral{n}).
\end{align*}

% Verification that primrec works:
%
% primec a f 0
% = (Z R) a f 0 id
% = R (Z R) a f 0 id
% = if (iszero 0) (k a) (...) id
% = (if true (k a) (...)) id
% = k a id
% = a
%
% primrec a f (n+1)
% = (Z R) a f (n+1) id
% = R (Z R) a f (n+1) id
% = (if (iszero (n+1)) (k a) (<y> f (pred (n+1)) ((R Z) a f (pred (n+1)) id))) id
% = (if false (k a) (<y> f (pred (n+1)) ((R Z) a f (pred (n+1)) id))) id
% = (<y> f (pred (n+1)) ((R Z) a f (pred (n+1)) id)) id
% = f (pred (n+1) ((R Z) a f (pred (n+1)) id)
% = f n ((R Z) a f n id)
% = f n (primrec a f n)
%
% Note: the trailing id is there (I think) to make sure both branches of the conditional are defined

\begin{example}
  \label{ex:numers-vs-numerals}
  It will be useful to know that in the ppca $(\KK, \PP)$ from \cref{ex:oracle-ppca} numbers can be converted to numerals and vice versa. For this purpose we construct realizers $\combNum, \combCur \in \KK$ such that for all $\alpha \in \PP$ and $n \in \NN$
  % 
  \begin{equation*}
    \alpha \at \combNum \, \numeral{n} = \alpha \at n
    \quad\text{and}\quad
    \alpha \at \combCur \, n = \alpha \at \numeral{n}.
  \end{equation*}
  % 
  To convert numerals to numbers, observe that there is $s \in \NN$, independent of~$\alpha$, such that $\pr[\alpha]{s}(n) = n + 1$, and define
  % 
  $\combNum \defeq \ucode{\comb{primrec} \, 0 \, s}$.
  %
  To implement the reverse translation, we apply
  the relativized Kleene's recursion theorem~\cite[Sect.~III.1.6]{soare87:_recur_enumer_sets_degrees}
  to find $r \in \NN$, independent of~$\alpha$, such that
  % 
  \begin{equation*}
    \pr[\alpha]{r}(n) =
    \begin{cases}
      \ucode{\numeral{0}} & \text{if $n = 0$,}\\
      \pr[\alpha]{\ucode{\comb{succ}}}(\pr[\alpha]{r}(n-1)) & \text{if $n > 0$.}
    \end{cases}
  \end{equation*}
  % 
  We may take $\combCur \defeq r$ because
  % 
  $\alpha \at r \, 0 = \pr[\alpha]{r}(0) = \ucode{\numeral{0}} = (\alpha \at \numeral{0})$
  and, assuming $\alpha \at r \, n = \alpha \at \numeral{n}$ for the sake of the induction step,
  % 
  \begin{multline*}
    \alpha \at r \, (n+1) =
    \pr[\alpha]{r}(n+1) =
    \pr[\alpha]{\ucode{\comb{succ}}}(\pr[\alpha]{r}(n)) = \\
    \alpha \at \comb{succ} \, (r \, n) =
    \alpha \at \comb{succ} \, \numeral{n} =
    \alpha \at \numeral{n + 1}.
  \end{multline*}
\end{example}

%%% Local Variables:
%%% mode: latex
%%% TeX-master: "countable-reals"
%%% End:
 
\section{Parameterized realizability}
\label{sec:unif-real}

We next devise a notion of realizability based on ppcas that captures the uniformity of oracle computations from \cref{sec:non-diag-sequ}.
%
We use the tripos-to-topos construction~\cite{hyland80:_tripos}, a general technique for defining toposes. We refer the readers to~\cite[Sect.~S2.1]{oosten08:_realiz} for background material.
%
For the remainder of this section we fix a ppca $(\AA, \PP)$.

\subsection{The parameterized realizability tripos}
\label{sec:tripos-built-from}

In the first step of the construction we shall define a contravariant functor
%
\begin{equation*}
  \PredSymbol : \op{\Set} \to \Heyt,
\end{equation*}
%
from sets to Heyting prealgebras, satisfying further conditions to be given later.
%
% AB: We pick sets because that's what we need to get a realizability topos. Maybe there's no need to create
% confusion here. Experts know, and the rest will just wonder why we're mentioning this.
Recall that a Heyting prealgebra $(H, {\leq})$ is a set $H$ with a reflexive transitive relation~$\leq$ with elements $\bot$, $\top$ and binary operations $\land$, $\lor$, $\limply$, satisfying the laws of intuitionistic propositional calculus.

For any set~$X$, define
%
\begin{equation*}
  \Pred[\AA,\PP]{X} \defeq (\pow{\AA}^X, {\leq_X}),
\end{equation*}
%
where the preorder~$\leq_X$ will be defined momentarily.
%
When no confusion can arise, we abbreviate $\Pred[\AA,\PP]{X}$ as $\Pred{X}$.

An element $\phi \in \Pred{X}$ is called a \defemph{(tripos) predicate} on~$X$.
%
We say that $\R{a} \in \phi(x)$ \emph{realizes} $\phi(x)$.
%
For a closed expression~$e$ over~$\AA$, we define $e \rz[p] \phi(x)$ by
%
\begin{equation*}
  e \rz[p] \phi(x)
  \defiff
  \defined{(p \at e)} \land (p \at e) \in \phi(x)
\end{equation*}
%
and read it as ``$e$ realizes $\phi(x)$ at $p$''.
%
The preoder on $\Pred{X}$ is defined as follows, for $\phi, \psi \in \Pred{X}$:
%
\begin{equation*}
  \phi \leq_X \psi
  \defiff
    \some{\R{a} \in \AA}
    \all{x \in X}
    \all{\R{b} \in \phi(x)}
    \all{p \in \PP}
    \R{a} \, \R{b} \rz[p] \psi(x).
\end{equation*}
%
We say that $a$ satisfying the above condition \emph{realizes} $\phi \leq_X \psi$.
Reflexivity of~$\leq_X$ is realized by $\ucode{\abstr{x} x}$. For transitivity, one checks that if $\R{a}$ realizes $\phi \leq_X \psi$ and $\R{b}$ realizes $\psi \leq_X \chi$ then $\ucode{\abstr{x} \R{b} \, (\R{a} \, x)}$ realizes $\phi \leq_X \chi$.

In order for $\PredSymbol$ to be a bona fide functor, we let it take a map $r : Y \to X$ to the \defemph{reindexing} map $\invim{r} : \Pred{X} \to \Pred{Y}$, which acts by precomposition $\invim{r} \phi = \phi \circ r$. This is obviously contravariant and functorial, and we shall check that $\invim{r}$ is a homomorphism below.

In order to show that $\PredSymbol$ is a tripos, we must verify the following conditions:
%
\begin{enumerate}
\item For every set $X$ the poset $\Pred{X}$ is a Heyting prealgebra (\cref{sec:heyt-prealg-struct}).
\item Reindexing is a homomorphism of Heyting prealgebras (\cref{sec:monot-reind}).
\item Universal and existential quantifiers exist for $\Pred{X}$ (\cref{sec:quantifiers}).
\item There is a generic element (\cref{sec:generic-element}).
\end{enumerate}
%
The arguments are quite similar to those for the tripos arising from an ordinary pca~\cite[Prop.~1.2.1]{oosten08:_realiz}, one only has to pay attention to the presence of parameters.

\subsection{The Heyting prealgebra structure}
\label{sec:heyt-prealg-struct}

The Heyting structure on $\Pred{X}$ is as follows:
%
{\allowdisplaybreaks
\begin{align*}
  \top(x) &\defeq \AA,\\
  \bot(x) &\defeq \emptyset,\\
  (\phi \land \psi)(x) &\defeq \set{\R{a} \in \AA \such
      \all{p \in \PP}
      \combFst \, \R{a} \rz[p] \phi(x) \land
      \combSnd \, \R{a} \rz[p] \psi(x)
  },
  \\
  (\phi \lor \psi)(x) &\defeq \{\R{a} \in \AA \such
    \all{p \in \PP}
    \begin{aligned}[t]
    &(p \at \combFst \, \R{a} = \ucode{\combTrue} \land \combSnd \, \R{a} \rz[p] \phi(x))
    \lor {} \\
    &(p \at \combFst \, \R{a} = \ucode{\combFalse} \land \combSnd \, \R{a} \rz[p] \psi(x)) \},
    \end{aligned}
  \\
  (\phi \limply \psi)(x) &\defeq \set{\R{a} \in \AA \such
    \all{p \in \PP} \all{\R{b} \in \phi(x)} \R{a} \, \R{b} \rz[p] \psi(x)}.
\end{align*}
}
%
The above is like the analogous Heyting structure for ordinary pcas, except that realizers must be uniform in~$p \in \PP$. Next, we verify that the given operations satisfy the laws of a Heyting prealgebra.

\subsubsection{Falsehood and truth}
\label{sec:falsehood-truth}

Both $\bot \leq_X \phi$ and $\phi \leq_X \top$ are realized by $\combK \, \combK$.

\subsubsection{Conjunction}
\label{sec:conjunction}

We need to verify that, for all $\phi, \psi, \chi \in \Pred{X}$,
%
\begin{equation*}
  (\chi \leq_X \phi) \land (\chi \leq_X \psi) \iff \chi \leq_X \phi \land \psi.
\end{equation*}
%
If $\R{a}$ and $\R{b}$ realize $\chi \leq_X \phi$ and $\chi \leq_X \psi$, respectively, then $\chi \leq_X \phi \land \psi$ is realized by $\R{c} \defeq \ucode{\abstr{u} \combPair \, (\R{a} \, u) \, (\R{b} \, u)}$. Indeed, for any $x \in X$, $p \in \PP$ and $\R{d} \in \chi(x)$, we have
%
\begin{equation*}
  p \at \combFst \, (\R{c} \, \R{d})
  \kleq
  p \at \combFst \, (\combPair \, (\R{a} \, \R{d}) \, (\R{b} \, \R{d}))
  \kleq
  p \at \R{a} \, \R{d}.
\end{equation*}
%
Because $\R{a} \, \R{d} \rz[p] \phi(x)$, it follows that $\combFst \, (\R{c} \, \R{d}) \rz[p] \phi(x)$.
The argument for the second component is analogous.

Conversely, if $\R{a}$ realizes $\chi \leq_X \phi \land \psi$ then $\R{b} \defeq \ucode{\abstr{u} \combFst \, (\R{a} \, u)}$ and $\R{c} \defeq \ucode{\abstr{v} \combSnd \, (\R{a} \, v)}$ realize $\chi \leq_X \phi$ and $\chi \leq_X \psi$, respectively. Indeed, for any $x \in X$, $p \in \PP$ and $\R{d} \in \chi(x)$, we have $\R{a} \, \R{d} \rz[p] (\phi \land \psi)(x)$ and $p \at \R{b} \, (\R{a} \, \R{d}) = p \at \combFst \, (\R{a} \, \R{d})$, hence $\R{b} \, (\R{a} \, \R{d}) \rz[p] \phi(x)$.
The argument for $\R{c}$ and $\chi \leq_X \psi$ is analogous.

\subsubsection{Disjunction}
\label{sec:disjunction}

Disjunction is characterized by
%
\begin{equation*}
  (\phi \leq_X \chi) \land (\psi \leq_X \chi) \iff \phi \lor \psi \leq_X \chi.
\end{equation*}
%
If $\R{a}$ and $\R{b}$ respectively realize $\phi \leq_X \chi$ and $\psi \leq_X \chi$, then $\phi \lor \psi \leq_X \chi$ is realized by
%
\begin{equation*}
  \R{c} \defeq
  \ucode{\abstr{u} \combIf \, (\combFst \, u) \, (\R{a} \, (\combSnd \, u)) \, (\R{b} \, (\combSnd \, u))}.
\end{equation*}
%
Consider any $x \in X$, $p \in \PP$ and $\R{d} \in (\phi \lor \psi)(x)$.
If $p \at \combFst \, \R{d} = \ucode{\combTrue}$ then
%
\begin{equation*}
  p \at \R{c} \, \R{d}
  \kleq
  p \at \combIf \, (\combFst \, \R{d}) \, (\R{a} \, (\combSnd \, \R{d})) \, (\R{b} \, (\combSnd \, \R{d}))
  \kleq
  p \at \R{a} \, (\combSnd \, \R{d}),
\end{equation*}
%
and since $\R{a} \, (\combSnd \, \R{d}) \rz[p] \chi(x)$ also $\R{c} \, \R{d} \rz[p] \chi(x)$.
%
If $p \at \combFst \, \R{d} = \ucode{\combFalse}$ then
%
\begin{equation*}
  p \at \R{c} \, \R{d}
  \kleq
  p \at \combIf \, (\combFst \, \R{d}) \, (\R{a} \, (\combSnd \, \R{d})) \, (\R{b} \, (\combSnd \, \R{d}))
  \kleq
  p \at \R{b} \, (\combSnd \, \R{d}),
\end{equation*}
%
and since $\R{b} \, (\combSnd \, \R{d}) \rz[p] \chi(x)$ also $\R{c} \, \R{d} \rz[p] \chi(x)$.

Conversely, if $\R{c}$ realizes $\phi \lor \psi \leq_X \chi$ then $\phi \leq_X \chi$ and $\psi \leq_X \chi$ are respectively realized by $\R{a} \defeq \ucode{\abstr{u} \R{c} \, (\combPair \, \combTrue \, u)}$ and $\R{b} \defeq \ucode{\abstr{v} \R{c} \, (\combPair \, \combFalse \, v)}$.
%
Indeed, for any $x \in X$, $p \in \PP$ and $\R{d} \in \phi(x)$ we have $\combPair \, \combTrue \, \R{d} \rz[p] (\phi \lor \psi)(x)$, hence $\R{c} \, (\combPair \, \combTrue \, \R{d}) \rz[p] \chi(x)$. Now $\R{a} \, \R{d} \rz[p] \chi(x)$ holds because
$p \at \R{a} \, \R{d} \kleq p \at \R{c} \, (\combPair \, \combTrue \, \R{d})$.
The argument for $\R{b}$ and $\psi \leq_X \chi$ is analogous.

\subsubsection{Implication}
\label{sec:implication}

Implication is characterzied by the adjunction
%
\begin{equation*}
   \phi \leq_X \psi \limply \chi\iff \phi \land \psi \leq_X \chi.
\end{equation*}
%
If $\R{a}$ realizes $\phi \leq_X \psi \limply \chi$ then $\R{b} \defeq \ucode{\abstr{x} \R{a} \, (\combFst \, x) \, (\combSnd \, x)}$ realizes $\phi \land \psi \leq_X \chi$. Indeed, for any $x \in X$, $p \in \PP$ and $\R{c} \in (\phi \land \psi)(x)$ we have
%
$
  p \at \R{b} \, \R{c}
  \kleq
  p \at \R{a} \, (\combFst \, \R{c}) \, (\combSnd \, \R{c})
$
%
and $\R{a} \, (\combFst \, \R{c}) \, (\combSnd \, \R{c}) \rz[p] \chi(x)$, hence $\R{b} \, \R{c} \rz[p] \chi(x)$.

Conversely, if $\R{b}$ realizes $\phi \land \psi \leq_X \chi$, then $\R{a} \defeq \ucode{\abstr{u} \abstr{v} \R{b} \, (\combPair \, u \, v)}$ realizes $\phi \leq_X \psi \limply \chi$.
To see this, we must verify for any $x \in X$, $p \in \PP$ and $\R{c} \in \phi(x)$ that $\R{a} \, \R{c} \rz[p] (\psi \limply \chi)(x)$.
%
Consider any $q \in \PP$ and $\R{d} \in \psi(x)$.
%
By \cref{lem:abstr-uniform}
%
\begin{equation*}
  p \at \R{a} \, \R{c} =
  p \at \abstr{v} \R{b} \, (\combPair \, \R{c} \, v) =
  q \at \abstr{v} \R{b} \, (\combPair \, \R{c} \, v) =
  q \at \R{a} \, \R{c},
\end{equation*}
%
hence
%
$
  q \at (\R{a} \app[p] \R{c}) \, \R{d} \kleq
  q \at \R{a} \, \R{c} \, \R{d} \kleq
  q \at \R{b} \, (\combPair \, \R{c} \, \R{d})
$,
%
and because $\R{b} \, (\combPair \, \R{c} \, \R{d}) \rz[q] \chi(x)$, it follows that $(\R{a} \app[p] \R{c}) \, \R{d} \rz[q] \chi(x)$.

\subsubsection{Negation}
\label{sec:negation}

In intuitionistic logic negation $\neg \phi$ is defined as $\phi \limply \bot$. A short calculation reveals that
%
\begin{align*}
  (\neg \phi)(x) &= \set{\R{a} \in \AA \such \phi(x) = \emptyset} \\
  (\neg\neg \phi)(x) &= \set{\R{a} \in \AA \such \phi(x) \neq \emptyset}.
\end{align*}
%

\subsection{Reindexing preserves the Heyting structure}
\label{sec:monot-reind}

We should not forget to check that $\invim{r} : \Pred{X} \to \Pred{Y}$ induced by $r : Y \to X$ is a homomorphism of Heyting prealgebras. This is easy, one just checks directly that $\invim{r}$ commutes with the logical connectives by unfolding the definitions. For example,
%
\begin{equation*}
  \R{a} \rz \invim{r}(\phi \limply \psi)(y)
  \iff
  \all{\R{b} \in \phi(r(y))}
  \all{p \in \PP}
  \R{a} \, \R{b} \rz[p] \phi(r(y))
\end{equation*}
%
and
%
\begin{equation*}
  \R{a} \rz (\invim{r}\phi \limply \invim{r}\psi)(y)
  \iff
  \all{\R{b} \in \phi(r(y))}
  \all{p \in \PP}
  \R{a} \, \R{b} \rz[p] \phi(r(y)),
\end{equation*}
%
which are the same condition.

\subsection{The quantifiers}
\label{sec:quantifiers}

Let $r : Y \to X$ be a map. The universal and existential quantifiers along~$r$ are monotone maps
%
\begin{equation*}
  \exists_r, \forall_r : \Pred{Y} \to \Pred{X},
\end{equation*}
%
such that, for all $\phi \in \Pred{Y}$ and $\psi \in \Pred{X}$,
%
\begin{equation*}
  \exists_r \phi \leq_X \psi \iff \phi \leq_Y \invim{r} \psi
  \qquad\text{and}\qquad
  \psi \leq_X \forall_r \phi \iff \invim{r} \psi \leq_Y \psi.
\end{equation*}
%
(The usual quantifiers correspond to $r : X \times Y \to X$ being the first projection.)
%
We may take the following definition of the existential quantifier:
%
\begin{align*}
  (\exists_r \phi)(x) \defeq
   \set{\R{a} \in \AA \such \some{y \in Y} r(y) = x \land \R{a} \in \phi(y)}.
\end{align*}
%
If $\R{a}$ realizes $\exists_r \phi \leq_X \psi$ then it also realizes $\phi \leq_Y \invim{r} \psi$:
%
for any $y \in Y$, $p \in \PP$ and $\R{b} \in \phi(y)$ we have $\R{b} \in (\exists_r \phi)(r(y))$, therefore $\R{a} \, \R{b} \rz[p] \psi(r(y))$.
%
Conversely, if $\R{a}$ realizes $\phi \leq_Y \invim{r} \psi$ then it also realizes $\exists_r \phi \leq_X \psi$: for any $x \in X$, $p \in \PP$ and $\R{b} \in (\exists_r \phi)(x)$, we have $r(y) = x$ for some $y \in Y$ such that $\R{b} \in \phi(y)$, hence $\R{a} \, \R{b} \rz[p] \psi(r(y))$ and $\R{a} \, \R{b} \rz[p] \psi(x)$.

Next, the definition of the universal quantifier is
%
\begin{multline*}
  (\forall_r \phi)(x) \defeq
   \{\R{a} \in \AA \such
     \all{y \in Y} r(y) = x \lthen
     \all{\R{b} \in \AA} \all{q \in \PP}
     \R{a} \, \R{b} \rz[q] \phi(y)
   \}.
\end{multline*}
%
If~$\R{a}$ realizes $\psi \leq_X \forall_r \phi$ then $\R{b} \defeq \ucode{\abstr{x} \R{a} \, x \, \combK}$ realizes $\invim{r}\psi \leq_Y \phi$:
%
for any $y \in Y$, $p \in \PP$, and $\R{c} \in \psi(r(y))$, we have $\R{a} \, \R{c} \rz[p] (\forall_r \phi)(r(y))$, therefore $\R{a} \, \R{c} \, \combK \rz[p] \phi(y)$ and $p \at \R{b} \, \R{c} = p \at \R{a} \, \R{c} \, \combK$, giving the required $\R{b} \, \R{c} \rz[p] \phi(y)$.
%
Conversely, if $\R{b}$ realizes $\invim{r}\psi \leq_Y \phi$ then $\R{a} \defeq \ucode{\abstr{x} \abstr{d} b \, x}$ realizes $\psi \leq_X \forall_r \phi$: consider any $x \in X$, $p \in \PP$, $\R{c} \in \psi(x)$. To show $\R{a} \, \R{c} \rz[p] (\forall_r \phi)(x)$, first note that $\defined{(p \at \R{a} \, \R{c})}$. Suppose $y \in Y$ is such that $r(y) = x$, and consider any $\R{d} \in \AA$ and $q \in \PP$.
%
By \cref{lem:abstr-uniform}
%
\begin{equation*}
  p \at \R{a} \, \R{c} =
  p \at \abstr{\R{d}} \R{b} \, \R{c} =
  q \at \abstr{\R{d}} \R{b} \, \R{c} =
  q \at \R{a} \, \R{c}
\end{equation*}
%
therefore
%
$
  q \at (\R{a} \app[p] \R{c}) \, \R{d} \kleq
  q \at \R{a} \, \R{c} \, \R{d} \kleq
  q \at \R{b} \, \R{c}
$.
%
From $\R{c} \in \psi(r(y))$ it follows that $\R{b} \, \R{c} \rz[q] \phi(x)$, therefore $(\R{a} \app[p] \R{c}) \, \R{d} \rz[q] \phi(x)$.

The reader may have expected the following, simpler definition of the universal quantifier
%
\begin{equation}
  \label{eq:alternative-forall}%
  (\forall'_r \phi)(x) \defeq
   \{\R{a} \in \AA \such
     \all{y \in Y} r(y) = x \lthen
     \R{a} \in \phi(y)
   \},
\end{equation}
%
which works, but only when~$r$ is surjective.
%
It is easy to check that $\forall'_r \phi \leq_X \forall_r \phi$ is realized by~$\combK$.
% Verification that k is such a realizer:
% Consider x ∈ X, a ∈ (∀'_r ϕ)(x) and p ∈ P.
% Then (p | k a) is defined.
% We verify that k a ⊩_p (∀'_r ϕ)(x):
%  Suppose y ∈ Y, r y = x, b ∈ 𝔸, q ∈ ℙ.
%  Then (q | k a b) = a, so it is defined.
%  And (q | k a b) ∈ ϕ(y) because (q | k a b) = a and a ∈ ϕ(y).
The converse $\forall_r \phi \leq \forall'_r \phi$ is realized by $\R{c} \defeq \abstr{x} x \, \combK$. Too see this, consider any $x \in X$, $\R{a} \in (\forall_r \phi)(x)$ and $p \in \PP$.
First, $p \at \R{c} \, \R{a} \simeq p \at \R{a} \, \combK$  is defined because~$r$ is surjective.
Second, if $y \in Y$ and $r(y) = x$ then $\R{a} \, \combK \rz[p] \phi(y)$ and $p \at \R{c} \, \R{a} = p \at \R{a} \, \combK$, therefore $\R{a} \, \combK \in \phi(y)$.

It remains to verify the Beck-Chevalley condition, which states that, given a pullback in~$\Set$
%
\begin{equation*}
  \xymatrix{
    {Y} \pullbackcorner
    \ar[r]^{r} \ar[d]_{u} 
    &
    {X} \ar[d]^{v}
    \\
    {Z} \ar[r]_{q}
    &
    {W}
  }
\end{equation*}
%
$\forall_r \circ \invim{u}$ and $\invim{v} \circ \forall_q$ are equivalent.
%
For $\phi \in \Pred{Z}$, $x \in X$
the condition $\R{a} \in ((\forall_r \circ \invim{u}) \phi)(x)$ unfolds to
%
\begin{equation}
  \label{eq:bc-1}%
  \all{y \in Y} r(y) = x \lthen \R{a} \in \phi(u(y)),
\end{equation}
%
while $\R{a} \in ((\invim{v} \circ \forall_q) \phi)(x)$ unfolds to
%
\begin{equation}
  \label{eq:bc-2}%
  \all{z \in Z} q(z) = v(x) \lthen \R{a} \in \phi(z).
\end{equation}
%
Let us show that these are equivalent conditions. Suppose $\R{a}$ satisfies \eqref{eq:bc-1} and $z \in Z$ is such that $q(z) = v(x)$. Because the square is a pullback there is a unique $y \in Y$ such that $r(y) = x$ and $u(y) = z$. By \eqref{eq:bc-1} we get $\R{a} \in \phi(u(y))$ which is the same as $\R{a} \in \phi(z)$.
%
Conversely, if $\R{a}$ satisfies \eqref{eq:bc-2} and there is a $y \in Y$ such that $r(y) = x$, then we instantiate \eqref{eq:bc-2} with $z = u(y)$ to obtain the desired $\R{a} \in \phi(u(y))$.

\subsection{The generic element}
\label{sec:generic-element}

Because $\Set$ is cartesian closed, the remaining requirement for a tripos is the existence of a generic element, see the remark following~\cite[Definition~2.12]{oosten08:_realiz}. Specifically, we seek a set $S$ and $\sigma \in \Pred{S}$ such that, for all $X$ and $\phi \in \Pred{X}$, there exists $r_\phi : X \to S$ for which $\phi$ and $\invim{r_\phi} \sigma$ are isomorphic.

Once again, we just reuse the generic element for a tripos based on a pca, namely $S \defeq \pow{\AA}$ and $\sigma \defeq \id[\pow{\AA}]$. This obviously works because $\invim{\phi} \id[\pow{\AA}] = \phi$ for any $\phi \in \Pred{X}$.

\subsection{Tripos logic}
\label{sec:tripos-logic}

A formula $\phi$ built from logical connectives, quantifiers, and tripos predicates
whose free variables $x_1, \ldots, x_n$ range over the sets $X_1, \ldots, X_n$,
determines a tripos predicate
%
\begin{equation*}
  [x_1 \of X_1, \ldots, x_n \of X_n \such \phi] : X_1 \times \cdots \times X_n \to \pow{\AA},
\end{equation*}
%
which we sometimes abbreviate as $[x_1, \ldots, x_n \such \phi]$ or just $[\phi]$.
The case $n = 0$ yields and element of $\pow{\AA}$.

More precisely, the logical connectives appearing in~$\phi$ are interpreted as the corresponding Heyting operations from \cref{sec:heyt-prealg-struct}.
A universally quantified formula $\all{y \of Y} \psi$, where $\psi$ is a formula in variables $x_1, \ldots, x_n$ and $y$, is interpreted as quantification along the projection
%
\begin{equation*}
  r : X_1 \times \cdots \times X_n \times Y \to X_1 \times \cdots \times X_n,
\end{equation*}
%
as in \cref{sec:quantifiers}, and similarly for $\some{y \of Y} \psi$.

\begin{example}
  \label{example:tripos-forall-exists}
  Given a tripos predicate $\psi \in \Pred{X \times Y}$ with an inhabited set~$X$, the formula
  %
  \begin{equation*}
    \all{x \of X} \some{y \of Y} \psi(x,y)
  \end{equation*}
  %
  has no free variables, and so determines an element of $\pow{\AA}$,
  which we compute using \eqref{eq:alternative-forall} to be
  %
  \begin{align*}
    \R{a} \in [\all{x \of X} \some{y \of Y} \psi(x,y)]
    &\iff
      \all{u \in X}
      \R{a} \in [\some{y} \psi(u, y)]
    \\
    &\iff
      \all{u \in X}
      \some{v \in Y}
      \R{a} \in \psi(u, v)
  \end{align*}
  %
  Note that $\R{a}$ may not depend on~$u$ and~$v$.
  This is a rather strong uniformity condition, stemming from the fact that realizers receive no information about the elements of underlying sets. When we pass from the tripos to the topos, the situation will be rectified by equipping sets with suitable realizability relations, see \cref{example:topos-forall-exists}.
\end{example}

We say that a formula $\phi$ in variables $x_1 \of X_1, \ldots, x_n \of X_n$ is \defemph{valid}, written as
%
\begin{equation*}
  x_1 \of X_1, \ldots, x_n \of X_n \models \phi,
\end{equation*}
%
when its interpretation is (equivalent to) the top predicate in $\Pred{X_1 \times \cdots \times X_n}$. This happens precisely when there is $\R{a} \in \AA$ such that $\R{a} \rz[p] [\phi](u_1, \ldots, u_n)$ for all $u_1 \in X_1, \ldots, u_n \in X_n$ and $p \in \PP$.


\subsection{The parameterized realizability topos on a ppca}
\label{sec:unif-real-topos}

Having defined a tripos, we employ the tripos-to-topos construction~\cite[\S2.2]{oosten08:_realiz} to construct a topos from it.

\begin{definition}
  The \defemph{parameterized realizability topos} $\PRT{\AA, \PP}$ on the ppca $(\AA, \PP, {\cdot})$ is the topos arising from the tripos-to-topos construction applied to the tripos~$\PredSymbol[\AA, \PP]$.
\end{definition}

We recall how the construction works.
%
An object $X = (|X|, {\eq[X]})$ of the topos is a set~$|X|$ with a tripos predicate ${\eq[X]} \in \Pred{|X| \times |X|}$, called the \defemph{equality predicate}, which is a partial equivalence relation in the sense of tripos logic:
%
\begin{align*}
  x \of |X|, y \of |X| &\models x \eq[X] y \limply y \eq[X] x,
  \\
  x \of |X|, y \of |X|, z \of |X| &\models x \eq[X] y \limply y \eq[X] z \limply x \eq[X] z.
\end{align*}
%
Specifically, this means that there are $\R{a}, \R{b} \in \AA$ such that:
%
\begin{enumerate}
\item for all $x, y \in |X|$, $\R{c} \in (x \eq[X] y)$, and $p \in \PP$, we have $\R{a} \, \R{c} \rz[p] y \eq[X] x$,
\item for all $x, y, z \in |X|$, $\R{c} \in (x \eq[X] y)$, $\R{d} \in (y \eq[X] z)$, and $p \in \PP$, we have $\R{b} \, \R{c} \, \R{d} \rz[p] x \eq[X] z$.
\end{enumerate}
%
Henceforth we shall refrain from such explicit unfolding of formulas into statements about realizers, and instead rely on the fact that a formula is valid in the tripos logic if it has an intuitionistic proof.

The equality predicate $\eq[X]$ endows $|X|$ with a notion of equality that is witnessed by realizers.
However, because we did not require reflexivity of~$\eq[X]$, there may be elements which are not equal to themselves.
To manage the anomaly, we define the \defemph{existence predicate} $\Ex{X} \in \Pred{|X|}$ by
%
\begin{equation*}
  \Ex{X}(x) \defeq (x \eq[X] x).
\end{equation*}
%
A realizer $\R{a} \in \Ex{X}(x)$ can be thought of as witnessing the fact that $x \in X$. When $\Ex{X}(x) = \emptyset$, the element $x$ ``does not exist'' from the point of view of the topos.
%
We shall strategically use $\Ex{X}(x)$ to disregard such non-existent elements.\footnote{%
It turns out that~$X$ is isomorphic to $(X', {\eq[X]})$ where $X' \defeq \set{x \in |X| \such (x \eq[X] x) \neq \emptyset}$, but insisting that $(x \eq[X] x) \neq \emptyset$ does not lead to any improvements.%
}

A morphism $F : X \to Y$ is represented by a predicate $F \in \Pred{|X| \times |Y|}$ which is a functional, i.e., one satisfying the following conditions, with $x, x' \of |X|$ and $y, y' \of |Y|$:
%
\begin{align*}
  x, y &\models F(x,y) \limply \Ex{X}(x) \land \Ex{Y}(y)
     & &\text{(strict)} \\
  x, x', y, y' &\models F(x,y) \land x \eq[X] x' \land y \eq[Y] y' \limply F(x', y')
     & &\text{(relational)} \\
  x, y, y' &\models F(x, y) \land F(x, y') \limply y \eq[Y] y'
     & &\text{(single-valued)} \\
  x &\models \Ex{X}(x) \limply \some{y \of Y} F(x, y)
     & &\text{(total)}
\end{align*}
%
Single-valuedness and totality are familiar conditions, while the other two ensure that~$F$
behaves with respect to existence and equality predicates. Note how the antecedent $\Ex{X}(x)$ in the totality condition allows~$F$ to ignore non-existing elements of~$X$.
%
Two such relations represent the same morphism if they are equivalent as tripos predicates.

To actually compute~$F$, we use a realizer~$s$ for its strictness and a realizer~$t$ for its totality to define the realizer $f \defeq \ucode{\abstr{x} \combSnd (s \, (t \, x))}$, which works as follows: for any $x \in X$ and $\R{a} \in \Ex{X}(x)$ there is $y \in Y$ such that $r \, \R{a} \rz[p] \Ex{Y}(y)$ for all~$p \in \PP$,
and because $F$ is single-valued, $y$ is unique up to $\eq[Y]$.

The identity morphism on~$X$ is represented by $\eq[X]$,
and the composition of $F : X \to Y$ and $G : Y \to Z$ by the tripos predicate
%
\begin{equation*}
  (G \circ F)(x, z) \defeq \some{y \of Y} F(x, y) \land G(y, z).
\end{equation*}
%
The relevant conditions may be checked by reasoning in intuitionistic logic.

The terminal object in the topos is $\one \defeq (\set{\star}, {\eq[\one]})$, where $(\star \eq[\one] \star) \defeq \AA$.
%
The subobject classifier is the object $\Omega \defeq (\pow{\AA}, \eq[\Omega])$ whose equality predicate is logical equivalence,
%
$
  (\phi \eq[\Omega] \psi) \defeq
  (\phi \to \psi) \land (\psi \to \phi)
$.
%
Truth $T : \one \to \Omega$ is represented by the tripos predicate $T(\star, \phi) \defeq \phi$.


\subsection{Topos logic}
\label{sec:internal-logic-topos}

The topos logic differs from the tripos logic because it accounts for the equality and existence predicates. We refer to~\cite[\S2.3]{oosten08:_realiz} for details, and give here an overview that will suffice for our purposes.

In the topos logic, the predicates on an object $X$ are its subobjects, which turn out to be in
bijective correspondence with equivalence classes of \defemph{strict predicates}~\cite[Thm.~2.2.1]{oosten08:_realiz}, i.e., those $\phi \in \Pred{|X|}$ that satisfy
%
\begin{align*}
  x \of X &\models \phi(x) \limply \Ex{X}(x) & &\text{(strict)} \\
  x \of X, y \of X &\models \phi(x) \land x \eq[X] y \limply \phi(y) & &\text{(relational)}
\end{align*}
%
The tripos falsehood is strict, and the tripos conjunction, disjunction, and the existential quantifier preserve strictness, hence these are computed in the topos in the same way as in the tripos. Truth, implication, and the universal quantifier require modification. We distinguish notationally between the tripos and topos logic by writing
``$\limply$'', ``$\forall y \of Y$'', and ``$\exists y \of Y$'' in the former, and
``$\lthen$'',  ``$\forall y \in Y$'', and ``$\exists y \in Y$'' in the latter.

First, the topos truth $\top$ qua predicate on~$X$ is the tripos predicate $\Ex{X}$. Indeed, this is a strict predicate, and for any strict predicate $\phi \in \Pred{X}$ the implication $\phi(x) \to \Ex{X}(x)$ is valid by strictness of~$\phi$. Because the top predicate has changed, we must also adjust validity: a strict predicate $\phi \in \Pred{X}$ is topos-valid when the tripos validates
%
\begin{equation*}
  x \of X \models \Ex{X}(x) \to \phi(x).
\end{equation*}
%
Explicitly, there exists $\R{a} \in \AA$ such that for all $x \in |X|$, $\R{b} \in \Ex{X}(x)$ and $p \in \PP$ we have $\R{a} \, \R{b} \rz[p] \phi(x)$.

Second, the topos implication $\phi \lthen \psi$ of strict predicates  $\phi$ and $\psi$ on~$X$ is represented by the strict predicate
%
\begin{equation*}
  [x \of X \mid \Ex{X}(x) \land (\phi(x) \limply \psi(x))].
\end{equation*}
%
Explicitly, $\R{a} \in (\phi \lthen \psi)(x)$ when for all $p \in \PP$
%
\begin{equation*}
  (\combFst \, \R{a} \rz[p] \Ex{X}(x))
  \land
  (\combSnd \, \R{a} \rz[p] \phi(x) \to \psi(x)).
\end{equation*}

Third, if $\phi$ is a strict predicate on $X \times Y$, the topos universal $\all{y \in Y} \phi(x, y)$ is represented by the strict predicate
%
\begin{equation*}
  [x \of X \mid \Ex{X}(x) \land \all{y \of |Y|} (\Ex{Y}(y) \limply \phi(x,y))].
\end{equation*}
%
Assuming $|Y|$ is inhabited, $\R{a} \in (\all{y \in Y} \phi(x, y))$ when for all $p \in \PP$
%
\begin{equation*}
  (\combFst \, \R{a} \rz[p] \Ex{X}(x))
  \land
  \all{y \in |Y|} \all{\R{b} \in \Ex{Y}(y)} \combSnd \, \R{a} \, \R{b} \rz[p] \phi(x, y).
\end{equation*}
%
The first conjunct just makes sure that non-existent~$x$ do not get in the way. The second one is more interesting, as it adjusts the unreasonable uniformity of tripos~$\forall$ by providing $\combSnd \, \R{a}$ with a realizer of~$y \in |Y|$.

One might expect the topos existential $\some{y \in Y} \phi(x, y)$ to be
%
\begin{equation*}
  [x \of X \such \some{y \of Y} \Ex{Y}(y) \land \phi(x, y)],
\end{equation*}
%
but we can reuse $\some{y \of Y} \phi(x, y)$, for if $\R{a} \in (\some{y \of Y} \phi(x, y))$
and~$s$ realizes strictness of~$\phi$ then $s \, \R{a} \in \Ex{Y}(y)$ for some $y \in |Y|$.

In contrast to the tripos logic, the topos logic is equipped with equality.
%
Unsurprisingly, equality on~$X$ is represented by $\eq[X]$, one just needs to check that this is indeed a strict predicate.
%
More generally, equality of morphisms $F, G : X \to Y$ is represented by the predicate
%
\begin{equation*}
  [x \of X \such \some{y \of Y} F(x, y) \land G(x, y)].
\end{equation*}

\begin{example}
  \label{example:topos-forall-exists}%
  Suppose $\carrier{X}$ is inhabited, and $\phi \in \Pred{\carrier{X} \times \carrier{Y}}$.
  A short calculations shows that $\all{x \in X} \some{y \in Y} \phi(x, y)$ is realized
  when there is $\R{a} \in \AA$ such that
  %
  \begin{equation*}
    \all{x \in |X|}
    \all{\R{b} \in \Ex{X}(x)}
    \all{p \in \PP}
    \some{y \in |Y|}
    \R{a} \, \R{b} \rz[p] \phi(x, y).
  \end{equation*}
  %
  Note that the unreasonable uniformity of \cref{example:tripos-forall-exists} has been rectified,
  as~$\R{b}$ is passed to~$\R{a}$.
\end{example}


\subsection{Parameterized assemblies}
\label{sec:unif-assemblies}

Direct manipulation of topos objects, and especially morphisms, can be cumbersome. Fortunately, the
subcategory of assemblies~\cite[Sect.~2.4]{oosten08:_realiz} is significantly easier to work with and already contains most objects of interest.

The idea is to take existence predicates as primary.
%
Define a \defemph{(parameterized) assembly} $X = (|X|, \Ex{X})$ to be a set $|X|$ with a tripos predicate $\Ex{X} \in \Pred{|X|}$, called the \defemph{existence predicate}, such that $\Ex{X}(x) \neq \emptyset$ for all $x \in |X|$.
%
Also define a \defemph{(parameterized) assembly map} $f : X \to Y$ to be a map $f : |X| \to |Y|$ which is realized by some $\R{a} \in \AA$, meaning: for all $x \in |X|$, $\R{b} \in \Ex{X}(x)$ and $p \in \PP$, we have $\R{a} \, \R{b} \rz[p] \Ex{Y}(f(x))$.
%
Assembly maps are closed under composition and include the identity maps, so we have a category $\PAsm{\AA, \PP}$. 

Given an assembly~$X$, let $\eq[X]$ be the tripos predicate on $|X| \times |X|$, defined by
%
\begin{equation*}
  (x \eq[X] x') \defeq \set{\R{a} \in \Ex{X}(x) \such x = x'}.
\end{equation*}
%
Thus $x \eq[X] x'$ is empty when $x \neq x'$ and equals $\Ex{X}(x)$ when $x = x'$.
%
It is evident that $x \eq[X] x'$ is an equality predicate on~$|X|$, hence the assembly~$X$ may be construed as the topos object $(|X|, \eq[X])$.
%
Not every topos object arises this way, for instance the subobject classifier~$\Omega$.

To get a functorial embedding of assemblies into the topos, we map an assembly map $f : X \to Y$ to the topos morphism $F : X \to Y$ where
%
\begin{equation*}
  F(x, y) \defeq \set{\R{b} \in \AA \such
    f(x) = y
    \land
    \all{p \in \PP}
    \combFst \, \R{b} \rz[p] \Ex{X}(x)
    \land
    \combSnd \, \R{b} \rz[p] \Ex{Y}(y)}.
\end{equation*}
%
This is a functional relation, for if~$\R{a}$ realizes~$f$ then $\ucode{\abstr{x} \combPair \, x \, (\R{a} \, x)}$ realizes totality of~$F$.
%
The passage from assemblies to topos objects constitutes a full and faithful embedding $\PAsm{\AA, \PP} \to \PRT{\AA, \PP}$. Only fullness deserves attention. Suppose $X$ and $Y$ are assemblies and $F : X \to Y$ a morphism between the induced topos objects. Because $Y$ is an assembly and $F$ is single-valued, each $x \in |X|$ has at most one $y \in |Y|$ such that $F(x, y) \neq \emptyset$. Therefore, we may define a map $f : |X| \to |Y|$ by
%
\begin{equation*}
  f(x) = y \defiff F(x,y) \neq \emptyset.
\end{equation*}
%
If $\R{a} \in \AA$ realizes totality of $F$ then $\ucode{\abstr{x} \combSnd \, (\R{a} \, x)}$ realizes~$f$ as an assembly map.

\subsection{Some distinguished assemblies}
\label{sec:distinguished-assemblies}

We review certain objects of the topos that will play a role in the construction of the object of the Dedekind reals.

\subsubsection{Natural numbers, integers, and rational numbers}
\label{sec:natur-numb-integ}

The natural numbers object is the assembly $\objN \defeq (\NN, \Ex{\objN})$ where $\Ex{\objN}(n) \defeq \set{\numeral{n}}$, so that each number is realized by the corresponding Curry numeral.
%
The induction principle is realized by the primitive recursor $\comb{primrec}$ from \cref{sec:progr-with-ppcas}.

The objects of integers and rational numbers are the assemblies
%
\begin{equation*}
  \objZ \defeq (\ZZ, \Ex{\objZ})
  \quad\text{\and}\quad
  \objQ \defeq (\QQ, \Ex{\objQ}),
\end{equation*}
%
whose existence predicates are induced by computable enumerations.
For the integers we can use
%
\begin{equation*}
  \Ex{\objZ}(k) \defeq
  \begin{cases}
    \set{\numeral{2 k}}     & \text{if $k \geq 0$,} \\
    \set{\numeral{1 - 2 k}} & \text{if $k < 0$.}
  \end{cases}
\end{equation*}
%
For the rationals we may reuse the bijection $\rat{} : \NN \to \QQ$ from \cref{sec:oracle-comp-maps},
%
\begin{equation*}
  \Ex{\objQ}(\rat{n}) \defeq \set{\numeral{n}}.
\end{equation*}
%
Any other reasonable codings would result in isomorphic objects.
%
Arithmetical operations are realized and the order relation is decidable, i.e., the statement
$\all{x, y} x < y \lor y \leq x$ is realized, both for $x, y \in \ZZ$ and for $x, y \in \QQ$.

\subsubsection{Products and exponentials}
\label{sec:prod-expon}

The category of parameterized assemblies is cartesian closed. The product of $X$ and $Y$ is the assembly
%
\begin{equation*}
  X \times Y \defeq (|X| \times |Y|, \Ex{X \times Y})
\end{equation*}
%
where
%
\begin{equation*}
  \Ex{X \times Y}(x, y) \defeq
  \set{\R{a} \in \AA \such
  \all{p \in \PP} \combFst \, \R{a} \rz[p] \Ex{X}(x) \land \combSnd \, \R{a} \rz[p] \Ex{Y}(y)}.
\end{equation*}
%
To construct the exponential~$Y^X$, we define its existence predicate, for any $f : |X| \to |Y|$, by
%
\begin{equation*}
  \Ex{Y^X}(f) \defeq
  \set{\R{a} \in \AA \such
    \all{x \in X} \all{\R{b} \in \Ex{X}(x)} \all{p \in \PP}
      \R{a} \, \R{b} \rz[p] \Ex{Y}(f(x))
  },
\end{equation*}
%
and set $|Y^X| \defeq \set{f : |X| \to |Y| \such \Ex{Y^X}(f) \neq \emptyset}$.

\begin{proposition}
  \label{prop:markov-principle}%
  Markov's principle
  %
  \begin{equation*}
    \all{f \in \two^\objN} \neg \neg (\some{n \in \objN} f n = 1) \lthen \some{n \in \objN} f n = 1
  \end{equation*}
  %
  is valid.
\end{proposition}

\begin{proof}
  The principle is realized by a program that searches for the least $n$ such that $f n \neq 0$:
  %
  \begin{equation*}
    \abstr{f} \abstr{r}
    \comb{Z} \, (\abstr{s} \abstr{n}
         \combIf \,
         (\comb{iszero} \,
         (f \, n) \,
         (s \, (\comb{succ} \, n)) \,
         n))
     \, \numeral{0}.
  \end{equation*}
  %
  The assumption $\neg\neg{\some{n \in \objN} f(n) = 1}$ ensures that the search will succeed.\footnote{As is typical of realizability models, we are relying on mete-level Markov's principle to realize Markov's principle: since it is impossible that the search will run forever, it will find what it is looking for.}
\end{proof}

\begin{example}
  Let us contrast $\forall\exists$ statements and exponentials. Consider a non-empty assembly~$X$, an assembly~$Y$, and
  a strict predicate $\phi \in \Pred{|X| \times |Y|}$, with $s \in \AA$ witnessing its strictness.
  %
  Validity of $\all{x \in X} \some{y \in Y} \phi(x,y)$ is equivalent to there being $\R{a} \in \AA$ such that
  %
  \begin{equation*}
    \all{x \in |X|} \all{\R{b} \in \Ex{X}(x)} \all{p \in \PP} \some{y \in |Y|} \R{a} \, \R{b} \rz[p] \phi(x,y).
  \end{equation*}
  %
  The realizer $\R{c} \defeq \ucode{\abstr{\R{b}} \combSnd \, (s \, (\R{a} \, \R{b}))}$ satisfies, for any $x \in |X|$, $\R{b} \in \Ex{X}(x)$ and $p \in \PP$, that there is $y \in |Y|$ such that $\R{c} \, \R{b} \rz[p] \Ex{Y}(y)$. However, $\R{c}$ need not realize a choice map $f : X \to Y$ because~$y$ may depend on~$\R{b}$ and~$p$.
  %
  Thus in general the axiom of choice
  %
  \begin{equation*}
    (\all{x \in X} \some{y \in Y} \phi(x, y))
    \lthen
    \some{f \in Y^X} \all{x \in X} \phi(x, f(x)))
  \end{equation*}
  %
  is not realized, even in case that $\Ex{X}(x)$ is a singleton for all~$x \in \carrier{X}$, because there is still dependence on the parameter. In particular, countable choice may fail, as it does in the topos~$\TT{\mil}$ from \cref{sec:topos-with-countable}.
\end{example}

\subsubsection{Sub-assemblies}
\label{sec:sub-assemblies}

Suppose $\phi \in \Pred{|X|}$ is a strict predicate on an assembly~$X$. (Notice that $\phi$ is automatically relational because $(x \eq[X] y) \neq \emptyset$ implies $x = y$.) We define the sub-assembly $\set{x \of X \such \phi(x)}$ to have the underlying set
%
\begin{equation*}
  |\set{x \of X \such \phi(x)}| \defeq \set{x \in |X| \such \phi(x) \neq \emptyset}
\end{equation*}
%
and the existence predicate $\Ex{\set{x \of X \such \phi(x)}}(x) \defeq \phi(x)$.
%
Then the canonical map $\set{x \of X \such \phi(x)} \to X$ is realized by any realizer for strictness of~$\phi$.
It is the monomorphism characterized by the predicate~$\phi$.


\subsubsection{Constant assemblies}
\label{sec:constant-assemblies}

Define the \defemph{constant (parameterized) assembly} on a set $S$ to be $\nabla S \defeq (S, \Ex{\nabla S})$ with $\Ex{\nabla S}(x) \defeq \AA$.
%
The existence predicate is maximally uninformative, because all elements of~$S$ share all realizers.
%
Consequently, given any assembly $X$, every map $f : |X| \to S$ is realized, say by~$\combK$.
In particular, every map $f : S \to T$ between sets is an assembly map $\nabla f \defeq f : \nabla S \to \nabla T$,
which makes $\nabla$ a functor from sets to assemblies, see~\cite[Sect.~2.4]{oosten08:_realiz} for details.

\subsubsection{The assembly $\nabla\two$ and $\neg\neg$-stable predicates}
\label{sec:assembly-nabl-negn}

Of particular interest is the assembly $\nabla \two$ because it classifies $\neg\neg$-stable predicates on any assembly~$X$ (and more generally on any topos object).
%
On the one hand, ${\nabla\two}^X$ is isomorphic to $\nabla(\two^{|X|})$ because every map $|X| \to \two$ is realized.
On the other, $\two^{|X|}$ qua Heyting algebra is equivalent to the Heyting prealgebra of $\neg\neg$-stable strict predicates on~$X$. Too see this, observe that a strict predicate~$\phi$ on~$X$ is $\neg\neg$-stable when
%
\begin{equation*}
  x \of X \models \Ex{X}(x) \to ((\phi(x) \lthen \bot) \lthen \bot) \lthen \phi(x),
\end{equation*}
%
%% Computation of ¬¬-stability of a strict relational ϕ ∈ Pred(X)
%
% Note: if ϕ → Ex then Ex → ϕx → P is equivalent to ϕx → P
%
% Double negation in the topos:
% ¬¬ϕx iff
% ((ϕx ⇒ ⊥ ) ⇒ ⊥) iff
% Ex → (Ex → ϕx → ⊥) → ⊥ iff
% Ex → (ϕx → ⊥) → ⊥
%
% ¬¬-stability of ϕ:
% ¬¬ϕx ⇒ ϕx iff
% Ex → ¬¬ϕx → ϕx iff
% Ex → ((Ex → (ϕx → ⊥)) → ⊥) → ϕx iff
% Ex → ((ϕx → ⊥) → ⊥) → ϕx
%
which amounts to there being $\R{a} \in \AA$ such that, for all $x \in |X|$, $\R{b} \in \Ex{X}(x)$ and $p \in \PP$,
if $\phi(x) \neq \emptyset$ then $\R{a} \, \R{b} \rz[p] \phi(x)$.
%
Therefore, $\phi$ is equivalent to the strict predicate
%
$x \mapsto \set{a \in \AA \such \phi(x) \neq \emptyset}$,
%
which in turn corresponds to a unique map $|X| \to \two$, obtained when~$\emptyset$ and~$\AA$ are mapped to $0$ and~$1$, respectively.






%%% Local Variables:
%%% mode: latex
%%% TeX-master: "countable-reals"
%%% End:
 
\chapter{Real numbers}
\label{cha:real-numbers}

\index{real numbers|(}%
Any foundation of mathematics worthy of its name must eventually address the construction of real numbers as understood by mathematical analysis, namely as a complete archimedean ordered field.
\index{ordered field}%
There are two notions of completeness. The one by Cauchy requires that the reals be closed under limits of Cauchy sequences\index{Cauchy!sequence}, while the stronger one by Dedekind requires closure under Dedekind cuts.\index{cut!Dedekind}
These lead to two ways of constructing reals, which we study in \cref{sec:dedekind-reals} and \cref{sec:cauchy-reals}, respectively. In \cref{RD-final-field,RC-initial-Cauchy-complete} we characterize the two constructions in terms of universal properties: the Dedekind reals are the final archimedean ordered field, and the Cauchy reals the initial Cauchy complete archimedean ordered field.

In traditional constructive mathematics,
\index{mathematics!constructive}%
real numbers always seem to require certain compromises. For example, the Dedekind reals work better with power sets or some other form of impredicativity, while Cauchy reals work well in the presence of countable choice.
\index{axiom!of choice!countable}%
However, we give a new construction of the Cauchy reals as a higher inductive-inductive type that seems to be a third possibility, which requires neither power sets nor countable choice.

In~\cref{sec:comp-cauchy-dedek} we compare the two constructions of reals. The Cauchy reals are included in the Dedekind reals. They coincide if excluded middle or countable choice holds, but in general the inclusion might be proper.

In~\cref{sec:compactness-interval} we consider three notions of compactness of the closed interval~$[0,1]$. We first show that $[0,1]$ is metrically compact\indexdef{metrically compact}\indexdef{compactness!metric} in the sense that it is complete and totally bounded, and that uniformly continuous maps on metrically compact spaces behave as expected. In contrast, the Bolzano--Weierstra\ss{} property that every sequence has a convergent subsequence implies the limited principle of omniscience, which is an instance of excluded middle. Finally, we discuss Heine--Borel compactness. A naive formulation of the finite subcover property does not work, but a proof relevant notion of inductive covers does.
This section is basically standard constructive analysis.

The development of real numbers and analysis in homotopy type theory can be easily made compatible with classical mathematics. By assuming excluded middle and the axiom of choice we get standard classical analysis:\index{mathematics!classical}\index{classical!analysis} the Dedekind and Cauchy reals coincide, foundational questions about the impredicative nature of the Dedekind reals disappear, and the interval is as compact as it could be.

We close the chapter by constructing Conway's surreals as a higher inductive-inductive type in \cref{sec:surreals};
the construction is more natural in univalent type theory than in  classical set theory.

In addition to the basic theory of \cref{cha:basics,cha:logic}, as noted above we use ``higher inductive-inductive types'' for the Cauchy reals and the surreals: these combine the ideas of \cref{cha:hits} with the notion of inductive-inductive type mentioned in \cref{sec:generalizations}.
We will also frequently use the traditional logical notation described in \cref{subsec:prop-trunc}, and the fact (proven in \cref{sec:piw-pretopos}) that our ``sets'' behave the way we would expect.

Note that the total space of the universal cover of the circle, which
in \cref{subsec:pi1s1-homotopy-theory} played a role similar to ``the real numbers'' in
classical algebraic topology, is \emph{not} the type of reals we are looking for. That
type is contractible, and thus equivalent to the singleton type, so it cannot be equipped
with a non-trivial algebraic structure.



\section{The field of rational numbers}
\label{sec:field-rati-numb}

\indexdef{rational numbers}%
\indexsee{number!rational}{rational numbers}%
We first construct the rational numbers \Q, as the reals can then be seen as a completion
of~\Q. An expert will point out that \Q could be replaced by any approximate field,
\indexdef{field!approximate}%
i.e., a subring of \Q in which arbitrarily precise approximate inverses
\index{inverse!approximate}%
exist. An example is the
ring of dyadic rationals,
\index{rational numbers!dyadic}%
which are those of the form $n/2^k$.
If we were implementing constructive mathematics on a computer,
an approximate field would be more suitable, but we leave such finesse for those
who care about the digits of~$\pi$.

We constructed the integers \Z in \cref{sec:set-quotients} as a quotient of $\N\times
\N$, and observed that this quotient is generated by an idempotent. In
\cref{sec:free-algebras} we saw that \Z is the free group on \unit; we could similarly
show that it is the free commutative ring\index{ring} on \emptyt. The field of rationals \Q is
constructed along the same lines as well, namely as the quotient
%
\[ \Q \defeq (\Z \times \N)/{\approx} \]
%
where
\[ (u,a) \approx (v,b) \defeq (u (b + 1) = v (a + 1)). \]
%
In other words, a pair $(u, a)$ represents the rational number $u / (1 + a)$. There can be
no division by zero because we cunningly added one to the denominator~$a$. Here too we
have a canonical choice of representatives, namely fractions in lowest terms. Thus we may
apply \cref{lem:quotient-when-canonical-representatives} to obtain a set \Q, which
again has a decidable equality.
\index{decidable!equality}%

We do not bother to write down the arithmetical operations on \Q as we trust that our readers
know how to compute with fractions even in the case when one is added to the denominator.
Let us just record the conclusion that there is an entirely unproblematic construction of
the ordered field of rational numbers \Q, with a decidable equality and decidable order.
It can also be characterized as the initial ordered field.
\index{initial!ordered field}%

\symlabel{positive-rationals}
\indexdef{rational numbers!positive}%
\indexdef{positive!rational numbers}%
Let $\Qp = \setof{ q : \Q | q > 0 }$ be the type of positive rational numbers.

\section{Dedekind reals}
\label{sec:dedekind-reals}

\index{real numbers!Dedekind|(}%
Let us first recall the basic idea of Dedekind's construction. We use two-sided Dedekind
cuts, as opposed to an often used one-sided version, because the symmetry makes
constructions more elegant, and it works constructively as well as classically.
\index{mathematics!constructive}%
A \emph{Dedekind cut}\index{cut!Dedekind} consists of a pair $(L, U)$ of subsets $L, U \subseteq \Q$, called the
\emph{lower} and \emph{upper cut} respectively, which are:
%
\begin{enumerate}
\item \emph{inhabited:} there are $q \in L$ and $r \in U$,
\item \emph{rounded:} $q \in L \Leftrightarrow \exis {r \in \Q} q < r \land r \in L$
  and $r \in U \Leftrightarrow \exis {q \in \Q} q \in U \land q < r$,
  \index{rounded!Dedekind cut}
\item \emph{disjoint:} $\lnot (q \in L \land q \in U)$, and
\item \emph{located:} $q < r \Rightarrow q \in L \lor r \in U$.
  \index{locatedness}%
\end{enumerate}
%
Reading the roundedness condition from left to right tells us that cuts are \emph{open},
\index{open!cut}%
and from right to left that they are \emph{lower}, respectively \emph{upper}, sets. The
locatedness condition states that there is no large gap between $L$ and $U$. Because cuts
are always open, they never include the ``point in between'', even when it is rational. A
typical Dedekind cut looks like this:
%
\begin{center}
  \begin{tikzpicture}[x=\textwidth]
    \draw[<-),line width=0.75pt] (0,0) -- (0.297,0) node[anchor=south east]{$L\ $};
    \draw[(->,line width=0.75pt] (0.300, 0) node[anchor=south west]{$\ U$} -- (0.9, 0) ;
  \end{tikzpicture}
\end{center}
%
We might naively translate the informal definition into type theory by saying that a cut
is a pair of maps $L, U : \Q \to \prop$. But we saw in \cref{subsec:prop-subsets} that
$\prop$ is an ambiguous\index{typical ambiguity} notation for $\prop_{\UU_i}$ where~$\UU_i$ is a universe. Once we
use a particular $\UU_i$ to define cuts, the type of reals will reside in the next
universe $\UU_{i+1}$, a property of reals two levels higher in $\UU_{i+2}$, a property of
subsets of reals in $\UU_{i+3}$, etc. In principle we should be able to keep track of the
universe levels\index{universe level}, especially with the help of a proof assistant, but doing so here would
just burden us with bureaucracy that we prefer to avoid. We shall therefore make a
simplifying assumption that a single type of propositions $\Omega$ is sufficient for all
our purposes.

In fact, the construction of the Dedekind reals is quite resilient to logical
manipulations. There are several ways in which we can make sense of using a single type
$\Omega$:
%
\begin{enumerate}

\item We could identify $\Omega$ with the ambiguous $\prop$ and track all the universes
  that appear in definitions and constructions.

\item We could assume the propositional resizing axiom,
  \index{propositional!resizing}%
  as in \cref{subsec:prop-subsets}, which essentially collapses the $\prop_{\UU_i}$'s to the
  lowest level\index{universe level}, which we call $\Omega$.

\item A classical mathematician who is not interested in the intricacies of type-theoretic
  universes or computation may simply assume the law of excluded middle~\eqref{eq:lem} for
  mere propositions so that $\Omega \jdeq \bool$.
  \index{excluded middle}
  This not only eradicates questions about
  levels\index{universe level} of $\prop$, but also turns everything we do into the standard classical\index{mathematics!classical}
  construction of real numbers.

\item On the other end of the spectrum one might ask for a minimal requirement that makes
  the constructions work. The condition that a mere predicate be a Dedekind cut is
  expressible using only conjunctions, disjunctions, and existential quantifiers\index{quantifier!existential} over~\Q, which
  is a countable set. Thus we could take $\Omega$ to be the initial \emph{$\sigma$-frame},
  \index{initial!sigma-frame@$\sigma$-frame}%
  \index{sigma-frame@$\sigma$-frame!initial|defstyle}%
  i.e., a lattice\index{lattice} with countable joins\index{join!in a lattice} in which binary meets distribute over countable
  joins. (The initial $\sigma$-frame cannot be the two-point lattice $\bool$ because
  $\bool$ is not closed under countable joins, unless we assume excluded middle.) This
  would lead to a construction of~$\Omega$ as a higher inductive-inductive type, but one
  experiment of this kind in \cref{sec:cauchy-reals} is enough.
\end{enumerate}

In all of the above cases $\Omega$ is a set.
%
Without further ado, we translate the informal definition into type theory.
Throughout this chapter, we use the
logical notation from \cref{defn:logical-notation}.

\begin{defn} \label{defn:dedekind-reals}
  A \define{Dedekind cut}
  \indexsee{Dedekind!cut}{cut, Dedekind}%
  \indexdef{cut!Dedekind}%
  is a pair $(L, U)$ of mere predicates $L : \Q \to \Omega$ and $U
  : \Q \to \Omega$ which is:
  %
  \begin{enumerate}
  \item \emph{inhabited:} $\exis{q : \Q} L(q)$ and $\exis{r : \Q} U(r)$,
  \item \emph{rounded:} for all $q, r : \Q$,
    \index{rounded!Dedekind cut}
    %
    \begin{align*}
      L(q) &\Leftrightarrow \exis{r : \Q} (q < r) \land L(r)
      \qquad\text{and}\\
      U(r) &\Leftrightarrow \exis{q : \Q} (q < r) \land U(q),
    \end{align*}
  \item \emph{disjoint:} $\lnot (L(q) \land U(q))$ for all $q : \Q$,
  \item \emph{located:} $(q < r) \Rightarrow L(q) \lor U(r)$ for all $q, r : \Q$.
  \index{locatedness}%
  \end{enumerate}
  %
  We let $\dcut(L, U)$ denote the conjunction of these conditions. The type of
  \define{Dedekind reals} is
  \indexsee{Dedekind!real numbers}{real numbers, De\-de\-kind}%
  \indexdef{real numbers!Dedekind}%
  %
  \begin{equation*}
    \RD \defeq \setof{ (L, U) : (\Q \to \Omega) \times (\Q \to \Omega) | \dcut(L,U)}.
  \end{equation*}
\end{defn}

It is apparent that $\dcut(L, U)$ is a mere proposition, and since $\Q \to \Omega$ is a
set the Dedekind reals form a set too. See
\cref{ex:RD-extended-reals,ex:RD-lower-cuts,ex:RD-interval-arithmetic} for variants of
Dedekind cuts which lead to extended reals, lower and upper reals, and the interval
domain.

There is an embedding $\Q \to \RD$ which associates with each rational $q : \Q$ the cut
$(L_q, U_q)$ where
%
\begin{equation*}
  L_q(r) \defeq (r < q)
  \qquad\text{and}\qquad
  U_q(r) \defeq (q < r).
\end{equation*}
%
We shall simply write $q$ for the cut $(L_q, U_q)$ associated with a rational number.

\subsection{The algebraic structure of Dedekind reals}
\label{sec:algebr-struct-dedek}

The construction of the algebraic and order-theoretic structure of Dedekind reals proceeds
as usual in intuitionistic logic. Rather than dwelling on details we point out the
differences between the classical\index{mathematics!classical} and intuitionistic setup. Writing $L_x$ and $U_x$ for
the lower and upper cut of a real number $x : \RD$, we define addition as%
%
\indexdef{addition!of Dedekind reals}%
\begin{align*}
  L_{x + y}(q) &\defeq \exis{r, s : \Q} L_x(r) \land L_y(s) \land q = r + s, \\
  U_{x + y}(q) &\defeq \exis{r, s : \Q} U_x(r) \land U_y(s) \land q = r + s,
\end{align*}
%
and the additive inverse by
%
\begin{align*}
  L_{-x}(q) &\defeq \exis{r : \Q} U_x(r) \land q = - r, \\
  U_{-x}(q) &\defeq \exis{r : \Q} L_x(r) \land q = - r.
\end{align*}
%
With these operations $(\RD, 0, {+}, {-})$ is an abelian\index{group!abelian} group. Multiplication is a bit
more cumbersome:
%
\indexdef{multiplication!of Dedekind reals}%
\begin{align*}
  L_{x \cdot y}(q) &\defeq
  \begin{aligned}[t]
    \exis{a, b, c, d : \Q} & L_x(a) \land U_x(b) \land L_y(c) \land U_y(d) \land {}\\
                           & \qquad q < \min (a \cdot c, a \cdot d, b \cdot c, b \cdot d),
  \end{aligned} \\
  U_{x \cdot y}(q) &\defeq
  \begin{aligned}[t]
    \exis{a, b, c, d : \Q} & L_x(a) \land U_x(b) \land L_y(c) \land U_y(d) \land {}\\
                           & \qquad \max (a \cdot c, a \cdot d, b \cdot c, b \cdot d) < q.
  \end{aligned}
\end{align*}
%
\index{interval!arithmetic}%
These formulas are related to multiplication of intervals in interval arithmetic, where
intervals $[a,b]$ and $[c,d]$ with rational endpoints multiply to the interval
%
\begin{equation*}
  [a,b] \cdot [c,d] =
  [\min(a c, a d, b c, b d), \max(a c, a d, b c, b d)].
\end{equation*}
%
For instance, the formula for the lower cut can be read as saying that $q < x \cdot y$
when there are intervals $[a,b]$ and $[c,d]$ containing $x$ and $y$, respectively, such
that $q$ is to the left of $[a,b] \cdot [c,d]$. It is generally useful to think of an
interval $[a,b]$ such that $L_x(a)$ and $U_x(b)$ as an approximation of~$x$, see
\cref{ex:RD-interval-arithmetic}.

We now have a commutative ring\index{ring} with unit
\index{unit!of a ring}%
$(\RD, 0, 1, {+}, {-}, {\cdot})$. To treat
multiplicative inverses, we must first introduce order. Define $\leq$ and $<$ as
%
\begin{align*}
  (x \leq y) &\ \defeq \ \fall{q : \Q} L_x(q) \Rightarrow L_y(q), \\
  (x < y)    &\ \defeq \ \exis{q : \Q} U_x(q) \land L_y(q).
\end{align*}

\begin{lem} \label{dedekind-in-cut-as-le}
  For all $x : \RD$ and $q : \Q$, $L_x(q) \Leftrightarrow (q < x)$ and $U_x(q)
  \Leftrightarrow (x < q)$.
\end{lem}

\begin{proof}
  If $L_x(q)$ then by roundedness there merely is $r > q$ such that $L_x(r)$, and since
  $U_q(r)$ it follows that $q < x$. Conversely, if $q < x$ then there is $r : \Q$ such
  that $U_q(r)$ and $L_x(r)$, hence $L_x(q)$ because $L_x$ is a lower set. The other half
  of the proof is symmetric.
\end{proof}

\index{partial order}%
\index{transitivity!of . for reals@of $<$ for reals}
\index{transitivity!of . for reals@of $\leq$ for reals}
\index{relation!irreflexive}
\index{irreflexivity!of . for reals@of $<$ for reals}
The relation $\leq$ is a partial order, and $<$ is transitive and irreflexive. Linearity
\index{order!linear}%
\index{linear order}%
%
\begin{equation*}
  (x < y) \lor (y \leq x)
\end{equation*}
%
is valid if we assume excluded middle, but without it we get weak linearity
%
\index{order!weakly linear}
\index{weakly linear order}
\begin{equation} \label{eq:RD-linear-order}
  (x < y) \Rightarrow (x < z) \lor (z < y).
\end{equation}
%
At first sight it might not be clear what~\eqref{eq:RD-linear-order} has to do with
linear order. But if we take $x \jdeq u - \epsilon$ and $y \jdeq u + \epsilon$ for
$\epsilon > 0$, then we get
%
\begin{equation*}
  (u - \epsilon < z) \lor (z < u + \epsilon).
\end{equation*}
%
This is linearity ``up to a small numerical error'', i.e., since it is unreasonable to
expect that we can actually compute with infinite precision, we should not be surprised
that we can decide~$<$ only up to whatever finite precision we have computed.

To see that~\eqref{eq:RD-linear-order} holds, suppose $x < y$. Then there merely exists $q : \Q$ such that $U_x(q)$ and
$L_y(q)$. By roundedness there merely exist $r, s : \Q$ such that $r < q < s$, $U_x(r)$
and $L_y(s)$. Then, by locatedness $L_z(r)$ or $U_z(s)$. In the first case we get $x < z$
and in the second $z < y$.

Classically, multiplicative inverses exist for all numbers which are different from zero.
However, without excluded middle, a stronger condition is required. Say that $x, y : \RD$
are \define{apart}
\indexdef{apartness}%
from each other, written $x \apart y$, when $(x < y) \lor (y < x)$:
%
\symlabel{apart}
\begin{equation*}
  (x \apart y) \defeq (x < y) \lor (y < x).
\end{equation*}
%
If $x \apart y$, then $\lnot (x = y)$.
The converse is true if we assume excluded middle, but is not provable constructively.
\index{mathematics!constructive}%
Indeed, if $\lnot (x = y)$ implies $x\apart y$, then a little bit of excluded middle follows; see \cref{ex:reals-apart-neq-MP}.

\begin{thm} \label{RD-inverse-apart-0}
  A real is invertible if, and only if, it is apart from $0$.
\end{thm}

\begin{rmk}
  We observe that a real is invertible if, and only if, it is merely
  invertible.  Indeed, the same is true in any ring,\index{ring} since a ring is a set, and
  multiplicative inverses are unique if they exist.  See the discussion
  following \cref{cor:UC}.
\end{rmk}

\begin{proof}
  Suppose $x \cdot y = 1$. Then there merely exist $a, b, c, d : \Q$ such that
  $a < x < b$, $c < y < d$ and $0 < \min (a c, a d, b c, b d)$. From $0 < a c$ and $0 < b c$ it follows
  that $a$, $b$, and $c$ are either all positive or all negative.
  Hence either $0 < a < x$ or $x < b < 0$, so that $x \apart 0$.

  Conversely, if $x \apart 0$ then
  %
  \begin{align*}
    L_{x^{-1}}(q) &\defeq
    \exis{r : \Q} U_x(r) \land ((0 < r \land q r < 1) \lor (r < 0 \land 1 < q r))
    \\
    U_{x^{-1}}(q) &\defeq
    \exis{r : \Q} L_x(r) \land ((0 < r \land q r > 1) \lor (r < 0 \land 1 > q r))
  \end{align*}
  %
  defines the desired inverse. Indeed, $L_{x^{-1}}$ and $U_{x^{-1}}$ are inhabited because
  $x \apart 0$.
\end{proof}

\index{ordered field!archimedean}%
\index{dense}%
\indexsee{order-dense}{dense}%
The archimedean principle can be stated in several ways. We find it most illuminating in the
form which says that $\Q$ is dense in $\RD$.

\begin{thm}[Archimedean principle for $\RD$] \label{RD-archimedean}
  %
  For all $x, y : \RD$ if $x < y$ then there merely exists $q : \Q$ such that
  $x < q < y$.
\end{thm}

\begin{proof}
  By definition of $<$.
\end{proof}

Before tackling completeness of Dedekind reals, let us state precisely what algebraic
structure they possess. In the following definition we are not aiming at a minimal
axiomatization, but rather at a useful amount of structure and properties.

\begin{defn} \label{ordered-field} An \define{ordered field}
  \indexdef{ordered field}%
  \indexsee{field!ordered}{ordered field}%
  is a set $F$ together with
  constants $0$, $1$, operations $+$, $-$, $\cdot$, $\min$, $\max$, and mere relations
  $\leq$, $<$, $\apart$ such that:
  %
  \begin{enumerate}
  \item $(F, 0, 1, {+}, {-}, {\cdot})$ is a commutative ring with unit;
    \index{unit!of a ring}%
    \index{ring}%
  \item $x : F$ is invertible if, and only if, $x \apart 0$;
  \item $(F, {\leq}, {\min}, {\max})$ is a lattice;
  \item the strict order $<$ is transitive, irreflexive,
    \index{relation!irreflexive}
    \index{irreflexivity!of . in a field@of $<$ in a field}%
    and weakly linear ($x < y \Rightarrow x < z \lor z < y$);\index{transitivity!of . in a field@of $<$ in a field}
    \index{order!weakly linear}
    \index{weakly linear order}
    \index{strict!order}%
    \index{order!strict}%
  \item apartness $\apart$ is irreflexive, symmetric and cotransitive ($x \apart y \Rightarrow x \apart z \lor y \apart z$);
    \index{relation!irreflexive}
    \index{irreflexivity!of apartness}%
    \indexdef{relation!cotransitive}%
    \index{cotransitivity of apartness}%
  \item for all $x, y, z : F$:
    %
    \begin{align*}
      x \leq y &\Leftrightarrow \lnot (y < x), &
      x < y \leq z &\Rightarrow x < z, \\
      x \apart y &\Leftrightarrow (x < y) \lor (y < x), &
      x \leq y < z &\Rightarrow x < z, \\
      x \leq y &\Leftrightarrow x + z \leq y + z, &
      x \leq y \land 0 \leq z &\Rightarrow x z \leq y z, \\
      x < y &\Leftrightarrow x + z < y + z, &
      0 < z \Rightarrow (x < y &\Leftrightarrow x z < y z), \\
      0 < x + y &\Rightarrow 0 < x \lor 0 < y, &
      0 &< 1.
    \end{align*}
  \end{enumerate}
  %
  Every such field has a canonical embedding $\Q \to F$. An ordered field is
  \define{archimedean}
  \indexdef{ordered field!archimedean}%
  \indexsee{archimedean property}{ordered field, archi\-mede\-an}%
  when for all $x, y : F$, if $x < y$ then there merely exists $q :
  \Q$ such that $x < q < y$.
\end{defn}

\begin{thm} \label{RD-archimedean-ordered-field}
  The Dedekind reals form an ordered archimedean field.
\end{thm}

\begin{proof}
  We omit the proof in the hope that what we have demonstrated so far makes the theorem
  plausible.
\end{proof}

\subsection{Dedekind reals are Cauchy complete}
\label{sec:RD-cauchy-complete}

Recall that $x : \N \to \Q$ is a \emph{Cauchy sequence}\indexdef{Cauchy!sequence} when it satisfies
%
\begin{equation} \label{eq:cauchy-sequence}
  \prd{\epsilon : \Qp} \sm{n : \N} \prd{m, k \geq n} |x_m - x_k| < \epsilon.
\end{equation}
%
Note that we did \emph{not} truncate the inner existential because we actually want to
compute rates of convergence---an approximation without an error estimate carries little
useful information. By \cref{thm:ttac}, \eqref{eq:cauchy-sequence} yields a function $M
: \Qp \to \N$, called the \emph{modulus of convergence}\indexdef{modulus!of convergence}, such that $m, k \geq M(\epsilon)$
implies $|x_m - x_k| < \epsilon$. From this we get $|x_{M(\delta/2)} - x_{M(\epsilon/2)}|<
\delta + \epsilon$ for all $\delta, \epsilon : \Qp$. In fact, the map $(\epsilon \mapsto
x_{M(\epsilon/2)}) : \Qp \to \Q$ carries the same information about the limit as the
original Cauchy condition~\eqref{eq:cauchy-sequence}. We shall work with these
approximation functions rather than with Cauchy sequences.

\begin{defn} \label{defn:cauchy-approximation}
  A \define{Cauchy approximation}
  \indexdef{Cauchy!approximation}%
  is a map $x : \Qp \to \RD$ which satisfies
  %
  \begin{equation}
    \label{eq:cauchy-approx}
    \fall{\delta, \epsilon :\Qp} |x_\delta - x_\epsilon| < \delta + \epsilon.
  \end{equation}
  %
  The \define{limit}
  \index{limit!of a Cauchy approximation}%
  of a Cauchy approximation $x : \Qp \to \RD$ is a number $\ell : \RD$ such
  that
  %
  \begin{equation*}
    \fall{\epsilon, \theta : \Qp} |x_\epsilon - \ell| < \epsilon + \theta.
  \end{equation*}
\end{defn}

\begin{thm} \label{RD-cauchy-complete}
  Every Cauchy approximation in $\RD$ has a limit.
\end{thm}

\begin{proof}
  Note that we are showing existence, not mere existence, of the limit.
  Given a Cauchy approximation $x : \Qp \to \RD$, define
  %
  \begin{align*}
    L_y(q) &\defeq \exis{\epsilon, \theta : \Qp} L_{x_\epsilon}(q + \epsilon + \theta),\\
    U_y(q) &\defeq \exis{\epsilon, \theta : \Qp} U_{x_\epsilon}(q - \epsilon - \theta).
  \end{align*}
  %
  It is clear that $L_y$ and $U_y$ are inhabited, rounded, and disjoint. To establish
  locatedness, consider any $q, r : \Q$ such that $q < r$. There is $\epsilon : \Qp$ such
  that $5 \epsilon < r - q$. Since $q + 2 \epsilon < r - 2 \epsilon$ merely
  $L_{x_\epsilon}(q + 2 \epsilon)$ or $U_{x_\epsilon}(r - 2 \epsilon)$. In the first case
  we have $L_y(q)$ and in the second $U_y(r)$.

  To show that $y$ is the limit of $x$, consider any $\epsilon, \theta : \Qp$. Because
  $\Q$ is dense in $\RD$ there merely exist $q, r : \Q$ such that
  %
  \begin{narrowmultline*}
    x_\epsilon - \epsilon - \theta/2 < q < x_\epsilon - \epsilon - \theta/4
    < x_\epsilon < \\
    x_\epsilon + \epsilon + \theta/4 < r < x_\epsilon + \epsilon + \theta/2,
  \end{narrowmultline*}
  %
  and thus $q < y < r$. Now either $y < x_\epsilon + \theta/2$ or $x_\epsilon - \theta/2 < y$.
  In the first case we have
  %
  \begin{equation*}
    x_\epsilon - \epsilon - \theta/2 < q < y < x_\epsilon + \theta/2,
  \end{equation*}
  %
  and in the second
  %
  \begin{equation*}
    x_\epsilon - \theta/2 < y < r < x_\epsilon + \epsilon + \theta/2.
  \end{equation*}
  %
  In either case it follows that $|y - x_\epsilon| < \epsilon + \theta$.
\end{proof}

For sake of completeness we record the classic formulation as well.

\begin{cor}
  Suppose $x : \N \to \RD$ satisfies the Cauchy condition~\eqref{eq:cauchy-sequence}. Then
  there exists $y : \RD$ such that
  %
  \begin{equation*}
    \prd{\epsilon : \Qp} \sm{n : \N} \prd{m \geq n} |x_m - y| < \epsilon.
  \end{equation*}
\end{cor}

\begin{proof}
  By \cref{thm:ttac} there is $M : \Qp \to \N$ such that $\bar{x}(\epsilon) \defeq
  x_{M(\epsilon/2)}$ is a Cauchy approximation. Let $y$ be its limit, which exists by
  \cref{RD-cauchy-complete}. Given any $\epsilon : \Qp$, let $n \defeq M(\epsilon/4)$
  and observe that, for any $m \geq n$,
  %
  \begin{narrowmultline*}
    |x_m - y| \leq |x_m - x_n| + |x_n - y| =
    |x_m - x_n| + |\bar{x}(\epsilon/2) - y| < \narrowbreak
    \epsilon/4 + \epsilon/2 + \epsilon/4 = \epsilon.\qedhere
  \end{narrowmultline*}
\end{proof}

\subsection{Dedekind reals are Dedekind complete}
\label{sec:RD-dedekind-complete}

We obtained $\RD$ as the type of Dedekind cuts on $\Q$. But we could have instead started
with any archimedean ordered field $F$ and constructed Dedekind cuts\index{cut!Dedekind} on $F$. These would
again form an archimedean ordered field $\bar{F}$, the \define{Dedekind completion of $F$},%
\index{completion!Dedekind}%
\indexsee{Dedekind!completion}{completion, Dedekind}
with $F$ contained as a subfield. What happens if we apply this construction to
$\RD$, do we get even more real numbers? The answer is negative. In fact, we shall prove a
stronger result: $\RD$ is final.

Say that an ordered field~$F$ is \define{admissible for $\Omega$}
\indexsee{admissible!ordered field}{ordered field, admissible}%
\indexdef{ordered field!admissible}%
when the strict order
$<$ on~$F$ is a map ${<} : F \to F \to \Omega$.

\begin{thm} \label{RD-final-field}
  Every archimedean ordered field which is admissible for $\Omega$ is a subfield of~$\RD$.
\end{thm}

\begin{proof}
  Let $F$ be an archimedean ordered field. For every $x : F$ define $L_x, U_x : \Q \to
  \Omega$ by
  %
  \begin{equation*}
    L_x(q) \defeq (q < x)
    \qquad\text{and}\qquad
    U_x(q) \defeq (x < q).
  \end{equation*}
  %
  (We have just used the assumption that $F$ is admissible for $\Omega$.)
  Then $(L_x, U_x)$ is a Dedekind cut.\index{cut!Dedekind} Indeed, the cuts are inhabited and rounded because
  $F$ is archimedean and $<$ is transitive, disjoint because $<$ is irreflexive, and
  located because $<$ is a weak linear order. Let $e : F \to \RD$ be the map $e(x) \defeq (L_x,
  U_x)$.

  We claim that $e$ is a field embedding which preserves and reflects the order. First of
  all, notice that $e(q) = q$ for a rational number $q$. Next we have the equivalences,
  for all $x, y : F$,
  %
  \begin{narrowmultline*}
    x < y \Leftrightarrow
    (\exis{q : \Q} x < q < y) \Leftrightarrow \narrowbreak
    (\exis{q : \Q} U_x(q) \land L_y(q)) \Leftrightarrow
    e(x) < e(y),
  \end{narrowmultline*}
  %
  so $e$ indeed preserves and reflects the order. That $e(x + y) = e(x) + e(y)$ holds
  because, for all $q : \Q$,
  %
  \begin{equation*}
    q < x + y \Leftrightarrow
    \exis{r, s : \Q} r < x \land s < y \land q = r + s.
  \end{equation*}
  %
  The implication from right to left is obvious. For the other direction, if $q < x +
  y$ then there merely exists $r : \Q$ such that $q - y < r < x$, and by taking $s \defeq
  q - r$ we get the desired $r$ and $s$. We leave preservation of multiplication by $e$ as
  an exercise.
\end{proof}

To establish that the Dedekind cuts on $\RD$ do not give us anything new, we need just one
more lemma.

\begin{lem} \label{lem:cuts-preserve-admissibility}
  If $F$ is admissible for $\Omega$ then so is its Dedekind completion.
  \index{completion!Dedekind}%
\end{lem}

\begin{proof}
  Let $\bar{F}$ be the Dedekind completion of $F$. The strict order on $\bar{F}$ is
  defined by
  %
  \begin{equation*}
    ((L,U) < (L',U')) \defeq \exis{q : \Q} U(q) \land L'(q).
  \end{equation*}
  %
  Since $U(q)$ and $L'(q)$ are elements of $\Omega$, the lemma holds as long as $\Omega$
  is closed under conjunctions and countable existentials, which we assumed from the outset.
\end{proof}


\begin{cor} \label{RD-dedekind-complete}
  %
  \indexdef{complete!ordered field, Dedekind}%
  \indexdef{Dedekind!completeness}%
  The Dedekind reals are Dedekind complete: for every real-valued Dedekind cut $(L, U)$
  there is a unique $x : \RD$ such that $L(y) = (y < x)$ and $U(y) = (x < y)$ for all $y :
  \RD$.
\end{cor}

\begin{proof}
  By \cref{lem:cuts-preserve-admissibility} the Dedekind completion $\barRD$ of $\RD$
  is admissible for $\Omega$, so by \cref{RD-final-field} we have an embedding $\barRD
  \to \RD$, as well as an embedding $\RD \to \barRD$. But these embeddings must be
  isomorphisms, because their compositions are order-preserving field homomorphisms\index{homomorphism!field} which
  fix the dense subfield~$\Q$, which means that they are the identity. The corollary now
  follows immediately from the fact that $\barRD \to \RD$ is an isomorphism.
\end{proof}

\index{real numbers!Dedekind|)}%

\section{Cauchy reals}
\label{sec:cauchy-reals}

\index{real numbers!Cauchy|(}%
\index{completion!Cauchy|(}%
\indexsee{Cauchy!completion}{completion, Cauchy}%
The Cauchy reals are, by intent, the completion of \Q under limits of Cauchy sequences.\index{Cauchy!sequence}
In the classical construction of the Cauchy reals, we consider the set $\mathcal{C}$ of all Cauchy sequences in \Q and then form a suitable quotient $\mathcal{C}/{\approx}$.
Then, to show that $\mathcal{C}/{\approx}$ is Cauchy complete, we consider a Cauchy sequence $x : \N \to \mathcal{C}/{\approx}$, lift it to a sequence of sequences $\bar{x} : \N \to \mathcal{C}$, and construct the limit of $x$ using $\bar{x}$. However, the lifting of~$x$ to $\bar{x}$ uses
the axiom of countable choice (the instance of~\eqref{eq:ac} where $X=\N$) or the law of excluded middle, which we may wish to avoid.
\indexdef{axiom!of choice!countable}%
Every construction of reals whose last step is a quotient suffers from this deficiency.
There are three common ways out of the conundrum in constructive mathematics:
\index{mathematics!constructive}%
%
\index{bargaining}%
\begin{enumerate}
\item Pretend that the reals are a setoid $(\mathcal{C}, {\approx})$, i.e., the type of
  Cauchy sequences $\mathcal{C}$ with a coincidence\index{coincidence, of Cauchy approximations} relation attached to it by
  administrative decree. A sequence of reals then simply \emph{is} a sequence of Cauchy
  sequences representing them.
\item Give in to temptation and accept the axiom of countable choice. After all, the axiom
  is valid in most models of constructive mathematics based on a computational viewpoint,
  such as realizability models.
\item Declare the Cauchy reals unworthy and construct the Dedekind reals instead.
  Such a verdict is perfectly valid in certain contexts, such as in sheaf-theoretic models of constructive mathematics.
  However, as we saw in \cref{sec:dedekind-reals}, the constructive Dedekind reals have their own problems.
\end{enumerate}

Using higher inductive types, however, there is a fourth solution, which we believe to be preferable to any of the above, and interesting even to a classical mathematician.
The idea is that the Cauchy real numbers should be the \emph{free complete metric space}\index{free!complete metric space} generated by~\Q.
In general, the construction of a free gadget of any sort requires applying the gadget operations repeatedly many times to the generators.
For instance, the elements of the free group on a set $X$ are not just binary products and inverses of elements of $X$, but words obtained by iterating the product and inverse constructions.
Thus, we might naturally expect the same to be true for Cauchy completion, with the relevant ``operation'' being ``take the limit of a Cauchy sequence''.
(In this case, the iteration would have to take place transfinitely, since even after infinitely many steps there will be new Cauchy sequences to take the limit of.)

The argument referred to above shows that if excluded middle or countable choice hold, then Cauchy completion is very special: when building the completion of a space, it suffices to stop applying the operation after \emph{one step}.
This may be regarded as analogous to the fact that free monoids and free groups can be given explicit descriptions in terms of (reduced) words.
However, we saw in \cref{sec:free-algebras} that higher inductive types allow us to construct free gadgets \emph{directly}, whether or not there is also an explicit description available.
In this section we show that the same is true for the Cauchy reals (a similar technique would construct the Cauchy completion of any metric space; see \cref{ex:metric-completion}).
Specifically, higher inductive types allow us to \emph{simultaneously} add limits of Cauchy sequences and quotient by the coincidence relation, so that we can avoid the problem of lifting a sequence of reals to a sequence of representatives.
\index{completion!Cauchy|)}%


\subsection{Construction of Cauchy reals}
\label{sec:constr-cauchy-reals}

The construction of the Cauchy reals $\RC$ as a higher inductive type is a bit more subtle than that of the free algebraic structures considered in \cref{sec:free-algebras}.
We intend to include a ``take the limit'' constructor whose input is a Cauchy sequence of reals, but the notion of ``Cauchy sequence of reals'' depends on having some way to measure the ``distance'' between real numbers.
In general, of course, the distance between two real numbers will be another real number, leading to a potentially problematic circularity.

However, what we actually need for the notion of Cauchy sequence of reals is not the general notion of ``distance'', but a way to say that ``the distance\index{distance} between two real numbers is less than $\epsilon$'' for any $\epsilon:\Qp$.
This can be represented by a family of binary relations, which we will denote $\mathord{\close\epsilon} : \RC\to\RC\to \prop$.
The intended meaning of $x \close\epsilon y$ is $|x - y| < \epsilon$, but since we do not have notions of subtraction, absolute value, or inequality available yet (we are only just defining $\RC$, after all), we will have to define these relations $\close\epsilon$ at the same time as we define $\RC$ itself.
And since $\close\epsilon$ is a type family indexed by two copies of $\RC$, we cannot do this with an ordinary mutual (higher) inductive definition; instead we have to use a \emph{higher inductive-inductive definition}.
\index{inductive-inductive type!higher}

Recall from \cref{sec:generalizations} that the ordinary notion of inductive-inductive definition allows us to define a type and a type family indexed by it by simultaneous induction.
Of course, the ``higher'' version of this allows both the type and the family to have path constructors as well as point constructors.
We will not attempt to formulate any general theory of higher inductive-inductive definitions, but hopefully the description we will give of $\RC$ and $\close\epsilon$ will make the idea transparent.

\begin{rmk}
  We might also consider a \emph{higher inductive-recursive definition}, in which $\close\epsilon$ is defined using the \emph{recursion} principle of $\RC$, simultaneously with the \emph{inductive} definition of $\RC$.
  We choose the inductive-inductive route instead for two reasons.
  Firstly, higher inductive-re\-cur\-sive definitions seem to be more difficult to justify in homotopical semantics.
  Secondly, and more importantly, the inductive-inductive definition yields a more powerful induction principle, which we will need in order to develop even the basic theory of Cauchy reals.
\end{rmk}

Finally, as we did for the discussion of Cauchy completeness of the Dedekind reals in \cref{sec:RD-cauchy-complete}, we will work with \emph{Cauchy approximations} (\cref{defn:cauchy-approximation}) instead of Cauchy sequences.
Of course, our Cauchy approximations will now consist of Cauchy reals, rather than Dedekind reals or rational numbers.

\begin{defn}\label{defn:cauchy-reals}
  Let $\RC$ and the relation $\closesym:\Qp \times \RC \times \RC \to \type$ be the following higher inductive-inductive type family.
  The type $\RC$ of \define{Cauchy reals}
  \indexdef{real numbers!Cauchy}%
  \indexsee{Cauchy!real numbers}{real numbers, Cau\-chy}%
  is generated by the following constructors:
  \begin{itemize}
  \item \emph{rational points:}
    for any $q : \Q$ there is a real $\rcrat(q)$.
    \index{rational numbers!as Cauchy real numbers}%
  \item \emph{limit points}:
    for any $x : \Qp \to \RC$ such that
    %
    \begin{equation}
      \label{eq:RC-cauchy}
      \fall{\delta, \epsilon : \Qp} x_\delta \close{\delta + \epsilon} x_\epsilon
    \end{equation}
    %
    there is a point $\rclim(x) : \RC$. We call $x$ a \define{Cauchy approximation}.
    \indexdef{Cauchy!approximation}%
    \index{limit!of a Cauchy approximation}%
    %
  \item \emph{paths:}
    for $u, v : \RC$ such that
    %
    \begin{equation}
      \label{eq:RC-path}
      \fall{\epsilon : \Qp} u \close\epsilon v
    \end{equation}
    %
    then there is a path $\rceq(u, v) : \id[\RC]{u}{v}$.
  \end{itemize}
  Simultaneously, the type family $\closesym:\RC\to\RC\to\Qp \to\type$ is generated by the following constructors.
  Here $q$ and $r$ denote rational numbers; $\delta$, $\epsilon$, and $\eta$ denote positive rationals; $u$ and $v$ denote Cauchy reals; and $x$ and $y$ denote Cauchy approximations:
  \begin{itemize}
  \item for any $q,r,\epsilon$, if $-\epsilon < q - r < \epsilon$, then $\rcrat(q) \close\epsilon \rcrat(r)$,
  \item for any $q,y,\epsilon,\delta$, if $\rcrat(q) \close{\epsilon - \delta} y_\delta$, then $\rcrat(q) \close{\epsilon} \rclim(y)$,
  \item for any $x,r,\epsilon,\delta$, if $x_\delta \close{\epsilon - \delta} \rcrat(r)$, then $\rclim(x) \close\epsilon \rcrat(r)$,
  \item for any $x,y,\epsilon,\delta,\eta$, if $x_\delta \close{\epsilon - \delta - \eta} y_\eta$, then $\rclim(x) \close\epsilon \rclim(y)$,
  \item for any $u,v,\epsilon$, if $\xi,\zeta : u \close{\epsilon} v$, then $\xi=\zeta$ (propositional truncation).
  \end{itemize}
\end{defn}

\mentalpause

The first constructor of $\RC$ says that any rational number can be regarded as a real number.
The second says that from any Cauchy approximation to a real number, we can obtain a new real number called its ``limit''.
And the third expresses the idea that if two Cauchy approximations coincide, then their limits are equal.

The first four constructors of $\closesym$ specify when two rational numbers are close, when a rational is close to a limit, and when two limits are close.
In the case of two rational numbers, this is just the usual notion of $\epsilon$-closeness for rational numbers, whereas the other cases can be derived by noting that each approximant $x_\delta$ is supposed to be within $\delta$ of the limit $\rclim(x)$.

We remind ourselves of proof-relevance: a real number obtained from $\rclim$ is represented not
just by a Cauchy approximation $x$, but also a proof $p$ of~\eqref{eq:RC-cauchy}, so we
should technically have written $\rclim(x,p)$ instead of just $\rclim(x)$.
A similar observation also applies to $\rceq$ and~\eqref{eq:RC-path}, but we shall write just
$\rceq : u = v$ instead of $\rceq(u, v, p) : u = v$. These abuses of notation are
mitigated by the fact that we are omitting mere propositions and information that is
readily guessed.
Likewise, the last constructor of $\mathord{\close\epsilon}$ justifies our leaving the other four nameless.

We are immediately able to populate $\RC$ with many real numbers. For suppose $x : \N \to
\Q$ is a traditional Cauchy sequence\index{Cauchy!sequence} of rational numbers, and let $M : \Qp \to \N$ be its
modulus of convergence. Then $\rcrat \circ x \circ M : \Qp \to \RC$ is a Cauchy
approximation, using the first constructor of $\closesym$ to produce the necessary witness.
Thus, $\rclim(\rcrat \circ x \circ m)$ is a real number. Various famous
real numbers such as $\sqrt{2}$, $\pi$, $e$, \dots{} are all limits of such Cauchy sequences of
rationals.

\subsection{Induction and recursion on Cauchy reals}
\label{sec:induct-recurs-cauchy}

In order to do anything useful with $\RC$, of course, we need to give its induction principle.
As is the case whenever we inductively define two or more objects at once, the basic induction principle for $\RC$ and $\closesym$ requires a simultaneous induction over both at once.
Thus, we should expect it to say that assuming two type families over $\RC$ and $\closesym$, respectively, together with data corresponding to each constructor, there exist sections of both of these families.
However, since $\closesym$ is indexed on two copies of $\RC$, the precise dependencies of these families is a bit subtle.
The induction principle will apply to any pair of type families:
\begin{align*}
A&:\RC\to\type\\
B&:\prd{x,y:\RC} A(x) \to A(y) \to \prd{\epsilon:\Qp} (x\close\epsilon y) \to \type.
\end{align*}
The type of $A$ is obvious, but the type of $B$ requires a little thought.
Since $B$ must depend on $\closesym$, but $\closesym$ in turn depends on two copies of $\RC$ and one copy of $\Qp$, it is fairly obvious that $B$ must also depend on the variables $x,y:\RC$ and $\epsilon:\Qp$ as well as an element of $(x\close\epsilon y)$.
What is slightly less obvious is that $B$ must also depend on $A(x)$ and $A(y)$.

This may be more evident if we consider the non-dependent case (the recursion principle), where $A$ is a simple type (rather than a type family).
In this case we would expect $B$ not to depend on $x,y:\RC$ or $x\close\epsilon y$.
But the recursion principle (along with its associated uniqueness principle) is supposed to say that $\RC$ with $\close\epsilon$ is an ``initial object'' in some category, so in this case the dependency structure of $A$ and $B$ should mirror that of $\RC$ and $\close\epsilon$: that is, we should have $B:A\to A\to \Qp \to \type$.
Combining this observation with the fact that, in the dependent case, $B$ must also depend on $x,y:\RC$ and $x\close\epsilon y$, leads inevitably to the type given above for $B$.

\symlabel{RC-recursion}
It is helpful to think of $B$ as an $\epsilon$-indexed family of relations between the types $A(x)$ and $A(y)$.
With this in mind, we may write $B(x,y,a,b,\epsilon,\xi)$ as $(x,a) \bsim_\epsilon^\xi (y,b)$.
Since $\xi:x\close\epsilon y$ is unique when it exists, we generally omit it from the notation and write $(x,a) \bsim_\epsilon (y,b)$; this is harmless as long as we keep in mind that this relation is only defined when $x\close\epsilon y$.
We may also sometimes simplify further and write $a\bsim_\epsilon b$, with $x$ and $y$ inferred from the types of $a$ and $b$, but sometimes it will be necessary to include them for clarity.

\index{induction principle!for Cauchy reals}%
Now, given a type family $A:\RC\to\type$ and a family of relations $\bsim$ as above, the hypotheses of the induction principle consist of the following data, one for each constructor of $\RC$ or $\closesym$:
\begin{itemize}
\item For any $q : \Q$, an element $f_q:A(\rcrat(q))$.
\item For any Cauchy approximation $x$, and any $a:\prd{\epsilon:\Qp} A(x_\epsilon)$ such that
  \begin{equation}
    \fall{\delta, \epsilon : \Qp}
    (x_\delta,a_\delta) \bsim_{\delta+\epsilon} (x_\epsilon,a_\epsilon),
    \label{eq:depCauchyappx}
  \end{equation}
  an element $f_{x,a}:A(\rclim(x))$.
  We call such $a$ a \define{dependent Cauchy approximation}
  \indexdef{Cauchy!approximation!dependent}%
  \indexsee{approximation, Cauchy}{Cauchy approximation}%
  \indexdef{dependent!Cauchy approximation}%
  over $x$.
\item For $u, v : \RC$ such that $h:\fall{\epsilon : \Qp} u \close\epsilon v$, and all $a:A(u)$ and $b:A(v)$ such that
  $\fall{\epsilon:\Qp} (u,a) \bsim_\epsilon (v,b)$,
  a dependent path $\dpath{A}{\rceq(u,v)}{a}{b}$.
\item For $q,r:\Q$ and $\epsilon:\Qp$, if $-\epsilon < q - r < \epsilon$, we have
  \narrowequation{(\rcrat(q),f_q) \bsim_\epsilon (\rcrat(r),f_r).}
\item For $q:\Q$ and $\delta,\epsilon:\Qp$ and $y$ a Cauchy approximation, and $b$ a dependent Cauchy approximation over $y$, if $\rcrat(q) \close{\epsilon - \delta} y_\delta$, then
  \[(\rcrat(q),f_q) \bsim_{\epsilon-\delta} (y_\delta,b_\delta)
  \;\Rightarrow\;
  (\rcrat(q),f_q) \bsim_\epsilon (\rclim(y),f_{y,b}).\]
\item Similarly, for $r:\Q$ and $\delta,\epsilon:\Qp$ and $x$ a Cauchy approximation, and $a$ a dependent Cauchy approximation over $x$, if $x_\delta \close{\epsilon - \delta} \rcrat(r)$, then
  \[(x_\delta,a_\delta) \bsim_{\epsilon-\delta} (\rcrat(r),f_r)
  \;\Rightarrow\;
  (\rclim(x),f_{x,a}) \bsim_\epsilon (\rcrat(q),f_r).
  \]
\item For $\epsilon,\delta,\eta:\Qp$ and $x,y$ Cauchy approximations, and $a$ and $b$ dependent Cauchy approximations over $x$ and $y$ respectively, if we have $x_\delta \close{\epsilon - \delta - \eta} y_\eta$, then
  \[ (x_\delta,a_\delta) \bsim_{\epsilon - \delta - \eta} (y_\eta,b_\eta)
  \;\Rightarrow\;
  (\rclim(x),f_{x,a}) \bsim_\epsilon (\rclim(y),f_{y,b}).\]
\item For $\epsilon:\Qp$ and $x,y:\RC$ and $\xi,\zeta:x\close{\epsilon} y$, and $a:A(x)$ and $b:A(y)$, any two elements of $(x,a) \bsim_\epsilon^\xi (y,b)$ and $(x,a) \bsim_\epsilon^\zeta (y,b)$ are dependently equal over $\xi=\zeta$.
  Note that as usual, this is equivalent to asking that $\bsim$ takes values in mere propositions.
\end{itemize}
Under these hypotheses, we deduce functions
\begin{align*}
  f&:\prd{x:\RC} A(x)\\
  g&:\prd{x,y:\RC}{\epsilon:\Qp}{\xi:x\close{\epsilon} y}
  (x,f(x)) \bsim_\epsilon^\xi (y,f(y))
\end{align*}
which compute as expected:
\begin{align}
  f(\rcrat(q)) &\defeq f_q, \label{eq:rcsimind1}\\
  f(\rclim(x)) &\defeq f_{x,(f,g)[x]}. \label{eq:rcsimind2}
\end{align}
Here $(f,g)[x]$ denotes the result of applying $f$ and $g$ to a Cauchy approximation $x$ to obtain a dependent Cauchy approximation over $x$.
That is, we define $(f,g)[x]_\epsilon \defeq f(x_\epsilon) : A(x_\epsilon)$, and then for any $\epsilon,\delta:\Qp$ we have $g(x_\epsilon,x_\delta,\epsilon+\delta,\xi)$ to witness the fact that $(f,g)[x]$ is a dependent Cauchy approximation, where $\xi: x_\epsilon \close{\epsilon+\delta} x_\delta$ arises from the assumption that $x$ is a Cauchy approximation.

We will never use this notation again, so don't worry about remembering it.
Generally we use the pattern-matching convention, where $f$ is defined by equations such as~\eqref{eq:rcsimind1} and~\eqref{eq:rcsimind2} in which the right-hand side of~\eqref{eq:rcsimind2} may involve the symbols $f(x_\epsilon)$ and an assumption that they form a dependent Cauchy approximation.

However, this induction principle is admittedly still quite a mouthful.
To help make sense of it, we observe that it contains as special cases two separate induction principles for~$\RC$ and for~$\closesym$.
Firstly, suppose given only a type family $A:\RC\to\type$, and define $\bsim$ to be constant at \unit.
Then much of the required data becomes trivial, and we are left with:
\begin{itemize}
\item for any $q : \Q$, an element $f_q:A(\rcrat(q))$,
\item for any Cauchy approximation $x$, and any $a:\prd{\epsilon:\Qp} A(x_\epsilon)$, an element $f_{x,a}:A(\rclim(x))$,
\item for $u, v : \RC$ and $h:\fall{\epsilon : \Qp} u \close\epsilon v$, and $a:A(u)$ and $b:A(v)$, we have $\dpath{A}{\rceq(u,v)}{a}{b}$.
\end{itemize}
Given these data, the induction principle yields a function $f:\prd{x:\RC} A(x)$ such that
\begin{align*}
  f(\rcrat(q)) &\defeq f_q,\\
  f(\rclim(x)) &\defeq f_{x,f(x)}.
\end{align*}
We call this principle \define{$\RC$-induction}; it says essentially that if we take $\close\epsilon$ as given, then $\RC$ is inductively generated by its constructors.

Note that, if $A$ is a mere property, then the third hypothesis in $\RC$-induction is automatic (we will see in a moment that these are in fact equivalent statements).
Thus, we may prove mere properties of real numbers by simply proving them for rationals and for limits of Cauchy approximations.
Here is an example.

\begin{lem} \label{lem:close-reflexive}
  For any $u:\RC$ and $\epsilon:\Qp$, we have $u\close\epsilon u$.
\end{lem}
\begin{proof}
  Define $A(u) \defeq \fall{\epsilon:\Qp} (u\close\epsilon u)$.
  Since this is a mere proposition (by the last constructor of $\closesym$), by $\RC$-induction, it suffices to prove it when $u$ is $\rcrat(q)$ and when $u$ is $\rclim(x)$.
  In the first case, we obviously have $|q-q|<\epsilon$ for any $\epsilon$, hence $\rcrat(q) \close\epsilon \rcrat(q)$ by the first constructor of $\closesym$.
  %
  And in the second case, we may assume inductively that $x_\delta \close\epsilon x_\delta$ for all $\delta,\epsilon:\Qp$.
  Then in particular, we have $x_{\epsilon/3} \close{\epsilon/3} x_{\epsilon/3}$, whence $\rclim(x) \close{\epsilon} \rclim(x)$ by the fourth constructor of $\closesym$.
\end{proof}

From \cref{lem:close-reflexive}, we infer that a direct application of $\RC$-induction only has a chance to succeed if the family $A:\RC\to\type$ is a mere property.
To see this, fix $u:\RC$.
Taking $v$ to be $u$, the third hypothesis of $\RC$-induction tells us that, for any $a : A(u)$, we have $\dpath{A}{\rceq(u,u)}{a}{a}$.
Given a point $b : A(u)$ in addition, we also get $\dpath{A}{\rceq(u,u)}{a}{b}$.
From the definition of the dependent path type, we conclude that the combination of these two paths implies $a = b$, i.e.\ all points in $A(u)$ are equal.

\begin{thm}\label{thm:Cauchy-reals-are-a-set}
  $\RC$ is a set.
\end{thm}
\begin{proof}
  We have just shown that the mere relation
  \narrowequation{P(u,v) \defeq \fall{\epsilon:\Qp} (u\close\epsilon v)}
  is reflexive.
  Since it implies identity, by the path constructor of $\RC$, the result follows from \cref{thm:h-set-refrel-in-paths-sets}.
\end{proof}

We can also show that although $\RC$ may not be a quotient of the set of Cauchy sequences of \emph{rationals}, it is nevertheless a quotient of the set of Cauchy sequences of \emph{reals}.
(Of course, this is not a valid \emph{definition} of $\RC$, but it is a useful property.)
We define the type of Cauchy approximations to be
%
\symlabel{cauchy-approximations}%
\index{Cauchy!approximation!type of}%
\begin{equation*}
  \CAP \defeq
  \setof{ x : \Qp \to \RC |
    \fall{\epsilon, \delta : \Qp} x_\delta \close{\delta + \epsilon} x_\epsilon
  }.
\end{equation*}
The second constructor of $\RC$ gives a function $\rclim:\CAP\to\RC$.

\begin{lem} \label{RC-lim-onto}
  Every real merely is a limit point: $\fall{u : \RC} \exis{x : \CAP} u = \rclim(x)$.
  In other words, $\rclim:\CAP\to\RC$ is surjective.
\end{lem}
\begin{proof}
  By $\RC$-induction, we may divide into cases on $u$.
  Of course, if $u$ is a limit $\rclim(x)$, the statement is trivial.
  So suppose $u$ is a rational point $\rcrat(q)$; we claim $u$ is equal to $\rclim(\lam{\epsilon} \rcrat(q))$.
  By the path constructor of $\RC$, it suffices to show $\rcrat(q) \close\epsilon \rclim(\lam{\epsilon} \rcrat(q))$ for all $\epsilon:\Qp$.
  And by the second constructor of $\closesym$, for this it suffices to find $\delta:\Qp$ such that $\rcrat(q)\close{\epsilon-\delta} \rcrat(q)$.
  But by the first constructor of $\closesym$, we may take any $\delta:\Qp$ with $\delta<\epsilon$.
\end{proof}

%

\begin{lem} \label{RC-lim-factor}
  If $A$ is a set and $f : \CAP \to A$ respects coincidence\index{coincidence!of Cauchy approximations} of Cauchy approximations, in the sense that
  %
  \begin{equation*}
    \fall{x, y : \CAP} \rclim(x) = \rclim(y) \Rightarrow f(x) = f(y),
  \end{equation*}
  %
  then $f$ factors uniquely through $\rclim : \CAP \to \RC$.
\end{lem}
\begin{proof}
  Since $\rclim$ is surjective, by \cref{lem:images_are_coequalizers}, $\RC$ is the quotient of $\CAP$ by the kernel pair\index{kernel!pair} of $\rclim$.
  But this is exactly the statement of the lemma.
\end{proof}

For the second special case of the induction principle, suppose instead that we take $A$ to be constant at $\unit$.
In this case, $\bsim$ is simply an $\epsilon$-indexed family of relations on $\epsilon$-close pairs of real numbers, so we may write $u\bsim_\epsilon v$ instead of $(u,\ttt)\bsim_\epsilon (v,\ttt)$.
Then the required data reduces to the following, where $q, r$ denote rational numbers, $\epsilon, \delta, \eta$ positive rational numbers, and $x, y$ Cauchy approximations:
\begin{itemize}
\item if $-\epsilon < q - r < \epsilon$, then
  $\rcrat(q) \bsim_\epsilon \rcrat(r)$,
\item if $\rcrat(q) \close{\epsilon - \delta} y_\delta$ and
  $\rcrat(q)\bsim_{\epsilon-\delta} y_\delta$,
  then $\rcrat(q) \bsim_\epsilon \rclim(y)$,
\item if $x_\delta \close{\epsilon - \delta} \rcrat(r)$ and
  $x_\delta \bsim_{\epsilon-\delta} \rcrat(r)$,
  then $\rclim(y) \bsim_\epsilon \rcrat(q)$,
\item if $x_\delta \close{\epsilon - \delta - \eta} y_\eta$ and
  $x_\delta\bsim_{\epsilon - \delta - \eta} y_\eta$,
  then $\rclim(x) \bsim_\epsilon \rclim(y)$.
\end{itemize}
The resulting conclusion is $\fall{u,v:\RC}{\epsilon:\Qp} (u\close\epsilon v) \to (u \bsim_\epsilon v)$.
We call this principle \define{$\closesym$-induction}; it says essentially that if we take $\RC$ as given, then $\close\epsilon$ is inductively generated (as a family of types) by \emph{its} constructors.
For example, we can use this to show that $\closesym$ is symmetric.

\begin{lem}\label{thm:RCsim-symmetric}
  For any $u,v:\RC$ and $\epsilon:\Qp$, we have $(u\close\epsilon v) = (v\close\epsilon u)$.
\end{lem}
\begin{proof}
  Since both are mere propositions, by symmetry it suffices to show one implication.
  Thus, let $(u\bsim_\epsilon v) \defeq (v\close\epsilon u)$.
  By $\closesym$-induction, we may reduce to the case that $u\close\epsilon v$ is derived from one of the four interesting constructors of $\closesym$.
  In the first case when $u$ and $v$ are both rational, the result is trivial (we can apply the first constructor again).
  In the other three cases, the inductive hypothesis (together with commutativity of addition in $\Q$) yields exactly the input to another of the constructors of $\closesym$ (the second and third constructors switch, while the fourth stays put).
\end{proof}

The general induction principle, which we may call \define{$(\RC,\closesym)$-induction}, is therefore a sort of joint $\RC$-induction and $\closesym$-induction.
Consider, for instance, its non-dependent version, which we call \define{$(\RC,\closesym)$-recursion}, which is the one that we will have the most use for.
\index{recursion principle!for Cauchy reals}%
Ordinary $\RC$-recursion tells us that to define a function $f : \RC \to A$ it suffices to:
\begin{enumerate}
\item for every $q : \Q$ construct $f(\rcrat(q)) : A$,
\item for every Cauchy approximation $x : \Qp \to \RC$, construct $f(x) : A$,
  assuming that $f(x_\epsilon)$ has already been defined for all $\epsilon : \Qp$,
\item prove $f(u) = f(v)$ for all $u, v : \RC$ satisfying $\fall{\epsilon:\Qp} u\close\epsilon v$.\label{item:rcrec3}
\end{enumerate}
However, it is generally quite difficult to show~\ref{item:rcrec3} without knowing something about how $f$ acts on $\epsilon$-close Cauchy reals.
The enhanced principle of $(\RC,\closesym)$-recursion remedies this deficiency, allowing us to specify an \emph{arbitrary} ``way in which $f$ acts on $\epsilon$-close Cauchy reals'', which we can then prove to be the case by a simultaneous induction with the definition of $f$.
This is the family of relations $\bsim$.
Since $A$ is independent of $\RC$, we may assume for simplicity that $\bsim$ depends only on $A$ and $\Qp$, and thus there is no ambiguity in writing $a\bsim_\epsilon b$ instead of $(u,a) \bsim_\epsilon (v,b)$.
In this case, defining a function $f:\RC\to A$ by $(\RC,\closesym)$-recursion requires the following cases (which we now write using the pattern-matching convention).
\begin{itemize}
\item For every $q : \Q$, construct $f(\rcrat(q)) : A$.
\item For every Cauchy approximation $x : \Qp \to \RC$, construct $f(\rclim(x)) : A$, assuming inductively that $f(x_\epsilon)$ has already been defined for all $\epsilon : \Qp$ and form a ``Cauchy approximation with respect to $\bsim$'', i.e.\ that $\fall{\epsilon,\delta:\Qp} (f(x_\epsilon) \bsim_{\epsilon+\delta} f(x_\delta))$.
\item Prove that the relations $\bsim$ are \emph{separated}, i.e.\ that, for any $a,b:A$,
  \indexdef{relation!separated family of}%
  \indexdef{separated family of relations}%
\narrowequation{(\fall{\epsilon:\Qp} a\bsim_\epsilon b) \Rightarrow (a=b).}
\item Prove that if $-\epsilon< q-r <\epsilon$ for $q,r:\Q$, then $f(\rcrat(q))\bsim_\epsilon f(\rcrat(r))$.
\item For any $q:\Q$ and any Cauchy approximation $y$, prove that
\narrowequation{f(\rcrat(q)) \bsim_\epsilon f(\rclim(y)),} assuming inductively that $\rcrat(q)\close{\epsilon-\delta} y_\delta$ and $f(\rcrat(q)) \bsim_{\epsilon-\delta} f(y_\delta)$ for some $\delta:\Qp$, and that $\eta \mapsto f(x_\eta)$ is a Cauchy approximation with respect to $\bsim$.
\item For any Cauchy approximation $x$ and any $r:\Q$, prove that
\narrowequation{f(\rclim(x)) \bsim_\epsilon f(\rcrat(r)),}
assuming inductively that $x_\delta \close{\epsilon-\delta} \rcrat(r)$ and $f(x_\delta) \bsim_{\epsilon-\delta} f(\rcrat(r))$ for some $\delta:\Qp$, and that $\eta\mapsto f(x_\eta)$ is a Cauchy approximation with respect to $\bsim$.
\item For any Cauchy approximations $x,y$, prove that
\narrowequation{f(\rclim(x)) \bsim_\epsilon f(\rclim(y)),}
assuming inductively that $x_\delta \close{\epsilon-\delta-\eta} y_\eta$ and $f(x_\delta) \bsim_{\epsilon-\delta-\eta} f(y_\eta)$ for some $\delta,\eta:\Qp$, and that $\theta\mapsto f(x_\theta)$ and $\theta\mapsto f(y_\theta)$ are Cauchy approximations with respect to $\bsim$.
\end{itemize}
Note that in the last four proofs, we are free to use the specific definitions of $f(\rcrat(q))$ and $f(\rclim(x))$ given in the first two data.
However, the proof of separatedness must apply to \emph{any} two elements of $A$, without any relation to $f$: it is a sort of ``admissibility'' condition on the family of relations $\bsim$.
Thus, we often verify it first, immediately after defining $\bsim$, before going on to define $f(\rcrat(q))$ and $f(\rclim(x))$.

Under the above hypotheses, $(\RC,\closesym)$-recursion yields a function $f:\RC\to A$ such that $f(\rcrat(q))$ and $f(\rclim(x))$ are judgmentally equal to the definitions given for them in the first two clauses.
Moreover, we may also conclude
\begin{equation}
  \fall{u,v:\RC}{\epsilon:\Qp} (u\close\epsilon v) \to (f(u) \bsim_\epsilon f(v)).\label{eq:RC-sim-recursion-extra}
\end{equation}

As a paradigmatic example, $(\RC,\closesym)$-recursion allows us to extend functions defined on $\Q$ to all of $\RC$, as long as they are sufficiently continuous.
\index{function!continuous}%

\begin{defn}\label{defn:lipschitz}
  A function $f:\Q\to\RC$ is \define{Lipschitz}
  \indexdef{function!Lipschitz}%
  \indexdef{Lipschitz!function}%
  \indexdef{Lipschitz!constant}%
  \indexdef{constant!Lipschitz}%
  if there exists $L:\Qp$ (the \define{Lipschitz constant}) such that
  \[ |q - r|<\epsilon \Rightarrow (f(q) \close{L\epsilon} f(r)) \]
  for all $\epsilon:\Qp$ and $q,r:\Q$.
  %
  Similarly, $g:\RC\to\RC$ is \define{Lipschitz} if there exists $L:\Qp$ such that
  \[ (u\close\epsilon v) \Rightarrow (g(u) \close{L\epsilon} g(v)) \]
  for all $\epsilon:\Qp$ and $u,v:\RC$..
\end{defn}

In particular, note that by the first constructor of $\closesym$, if $f:\Q\to\Q$ is Lipschitz in the obvious sense, then so is the composite $\Q\xrightarrow{f} \Q \to \RC$.

\begin{lem}\label{RC-extend-Q-Lipschitz}
  Suppose $f : \Q \to \RC$ is Lipschitz with constant $L : \Qp$.
  Then there exists a Lipschitz map $\bar{f} : \RC \to \RC$, also with constant $L$, such that $\bar{f}(\rcrat(q)) \jdeq f(q)$ for all $q:\Q$.
\end{lem}

\begin{proof}
  % Uniqueness follows directly from \cref{RC-continuous-eq}.
  We define $\bar{f}$ by $(\RC,\closesym)$-recursion, with codomain $A\defeq \RC$.
  We define the relation $\mathord{\bsim}: \RC \to \RC \to \Qp \to \prop$ to be
  \begin{align*}
    (u \bsim_\epsilon v) &\defeq (u \close{L\epsilon} v).
  \end{align*}
  For $q : \Q$, we define
  %
  \begin{equation*}
    \bar{f}(\rcrat(q)) \defeq \rcrat(f(q)).
  \end{equation*}
  %
  For a Cauchy approximation $x : \Qp \to \RC$, we define
  %
  \begin{equation*}
    \bar{f}(\rclim(x)) \defeq \rclim(\lamu{\epsilon : \Qp} \bar{f}(x_{\epsilon/L})).
  \end{equation*}
  %
  For this to make sense, we must verify that $y \defeq \lamu{\epsilon : \Qp} \bar{f}(x_{\epsilon/L})$ is a Cauchy approximation.
  However, the inductive hypothesis for this step is that for any $\delta,\epsilon:\Qp$ we have $\bar{f}(x_\delta) \bsim_{\delta+\epsilon} \bar{f}(x_\epsilon)$, i.e.\ $\bar{f}(x_\delta) \close{L\delta+L\epsilon} \bar{f}(x_\epsilon)$.
  Thus we have
  \[y_\delta \jdeq f(x_{\delta/L}) \close{\delta + \epsilon} f(x_{\epsilon/L})   \jdeq y_\epsilon. \]

  For proving separatedness, we simply observe that $\fall{\epsilon:\Qp} a\bsim_\epsilon b$ means $\fall{\epsilon:\Qp} a\close{L\epsilon} b$, which implies $\fall{\epsilon:\Qp}a\close\epsilon b$ and thus $a=b$.

  To complete the $(\RC,\closesym)$-recursion, it remains to verify the four conditions on $\bsim$.
  This basically amounts to proving that $\bar f$ is Lipschitz for all the four constructors of $\closesym$.
  \begin{enumerate}
  \item When $u$ is $\rcrat(q)$ and $v$ is $\rcrat(r)$ with $-\epsilon < |q-r| <\epsilon$, the assumption that $f$ is Lipschitz yields $f(q) \close{L\epsilon} f(r)$, hence $\bar{f}(\rcrat(q)) \bsim_\epsilon \bar{f}(\rcrat(r))$ by definition.
  \item When $u$ is $\rclim(x)$ and $v$ is $\rcrat(q)$ with $x_\eta \close{\epsilon - \eta} \rcrat(q)$, then the
      inductive hypothesis is $\bar{f}(x_\eta) \close{L \epsilon - L \eta} \rcrat(f(q))$, which proves
      \narrowequation{\bar{f}(\rclim(x)) \close{L \epsilon} \bar{f}(\rcrat(q))}
      by the third constructor of $\closesym$.
  \item The symmetric case when $u$ is rational and $v$ is a limit is essentially identical.
  \item When $u$ is $\rclim(x)$ and $v$ is $\rclim(y)$, with $\delta, \eta : \Qp$ such that $x_\delta \close{\epsilon - \delta - \eta} y_\eta$,
      the inductive hypothesis is $\bar{f}(x_\delta) \close{L \epsilon - L \delta - L \eta} \bar{f}(y_\eta)$, which proves $\bar{f}(\rclim(x)) \close{L
        \epsilon} \bar{f}(\rclim(y))$ by the fourth constructor of $\closesym$.
  \end{enumerate}
  This completes the $(\RC,\closesym)$-recursion, and hence the construction of $\bar f$.
  The desired equality $\bar f(\rcrat(q))\jdeq f(q)$ is exactly the first computation rule for $(\RC,\closesym)$-recursion, and the additional condition~\eqref{eq:RC-sim-recursion-extra} says exactly that $\bar f$ is Lipschitz with constant $L$.
\end{proof}

At this point we have gone about as far as we can without a better characterization of $\closesym$.
We have specified, in the constructors of $\closesym$, the conditions under which we want Cauchy reals of the two different forms to be $\epsilon$-close.
However, how do we know that in the resulting inductive-inductive type family, these are the \emph{only} witnesses to this fact?
We have seen that inductive type families (such as identity types, see \cref{sec:identity-systems}) and higher inductive types have a tendency to contain ``more than was put into them'', so this is not an idle question.

In order to characterize $\closesym$ more precisely, we will define a family of relations $\approx_\epsilon$ on $\RC$ \emph{recursively}, so that they will compute on constructors, and prove that this family is equivalent to $\close\epsilon$.

\begin{thm}\label{defn:RC-approx}
  There is a family of mere relations $\mathord\approx:\RC\to\RC\to\Qp\to\prop$ such that
  \begin{align}
    (\rcrat(q) \approx_\epsilon \rcrat(r))  &\defeq
    (-\epsilon < q - r < \epsilon)\label{eq:RCappx1}\\
    (\rcrat(q) \approx_\epsilon \rclim(y)) &\defeq
    \exis{\delta : \Qp} \rcrat(q) \approx_{\epsilon - \delta} y_\delta\label{eq:RCappx2}\\
    (\rclim(x) \approx_\epsilon \rcrat(r)) &\defeq
    \exis{\delta : \Qp} x_\delta \approx_{\epsilon - \delta} \rcrat(r)\label{eq:RCappx3}\\
    (\rclim(x) \approx_\epsilon \rclim(y)) &\defeq
    \exis{\delta, \eta : \Qp} x_\delta \approx_{\epsilon - \delta - \eta} y_\eta.\label{eq:RCappx4}
  \end{align}
  Moreover, we have
  \begin{gather}
    (u \approx_\epsilon v) \Leftrightarrow \exis{\theta:\Qp} (u \approx_{\epsilon-\theta} v) \label{RC-sim-rounded}\\
    (u \approx_\epsilon v) \to (v\close\delta w) \to (u\approx_{\epsilon+\delta} w)\label{eq:RC-sim-rtri}\\
    (u \close\epsilon v) \to (v\approx_\delta w) \to (u\approx_{\epsilon+\delta} w)\label{eq:RC-sim-ltri}.
  \end{gather}
\end{thm}

The additional conditions~\eqref{RC-sim-rounded}--\eqref{eq:RC-sim-ltri} turn out to be required in order to make the inductive definition go through.
Condition~\eqref{RC-sim-rounded} is called being \define{rounded}.
\indexsee{relation!rounded}{rounded relation}%
\indexdef{rounded!relation}%
Reading it from right to left gives \define{monotonicity} of $\approx$,
\index{monotonicity}%
\index{relation!monotonic}%
%
\begin{equation*}
  (\delta < \epsilon) \land (u \approx_\delta v) \Rightarrow (u \approx_\epsilon v)
\end{equation*}
%
while reading it left to right to \define{openness} of $\approx$,
\index{open!relation}%
\index{relation!open}%
%
\begin{equation*}
  (u \approx_\epsilon v) \Rightarrow \exis{\delta : \Qp} (\delta < \epsilon) \land (u \approx_\delta v).
\end{equation*}
%
Conditions~\eqref{eq:RC-sim-rtri} and~\eqref{eq:RC-sim-ltri} are forms of the triangle inequality, which say that $\approx$ is a ``module'' over $\closesym$ on both sides.

\begin{proof}
  We will define $\mathord\approx:\RC\to\RC\to\Qp\to\prop$ by double $(\RC,\closesym)$-recursion.
  First we will apply $(\RC,\closesym)$-recursion with codomain the subset of $\RC\to\Qp\to\prop$ consisting of those families of predicates which are rounded and satisfy the one appropriate form of the triangle inequality.
  Thinking of these predicates as half of a binary relation, we will write them as $(u,\epsilon) \mapsto (\hapx_\epsilon u)$, with the symbol $\hapname$ referring to the whole relation.
  Now we can write $A$ precisely as
  \begin{multline*}
    A \defeq\; \Bigg\{ \hapname :\RC\to\Qp\to\prop \;\bigg|\; \\
    \Big(\fall{u:\RC}{\epsilon:\Qp}
    \big((\hapx_\epsilon u) \Leftrightarrow \exis{\theta:\Qp} (\hapx_{\epsilon-\theta} u)\big)\Big)  \\
    \land \Big(\fall{u,v:\RC}{\eta,\epsilon:\Qp} (u\close\epsilon v) \to\\
    \big((\hapx_\eta u) \to (\hapx_{\eta+\epsilon} v) \big) \land \big((\hapx_\eta v) \to (\hapx_{\eta+\epsilon} u) \big)\Big)\Bigg\}
  \end{multline*}
  As usual with subsets, we will use the same notation for an inhabitant of $A$ and its first component $\hapname$.
  As the family of relations required for $(\RC,\closesym)$-recursion, we consider the following, which will ensure the other form of the triangle inequality:
  \begin{narrowmultline*}
    (\hapname \bsim_\epsilon \hapbname ) \defeq \narrowbreak
    \fall{u:\RC}{\eta:\Qp} ((\hapx_\eta u) \to (\hapxb_{\epsilon+\eta} u))
    \land \narrowbreak
    ((\hapxb_\eta u) \to (\hapx_{\epsilon+\eta} u)).
  \end{narrowmultline*}
  We observe that these relations are separated.
  For assuming
  \narrowequation{\fall{\epsilon:\Qp} (\hapname \bsim_\epsilon \hapbname),}
  to show $\hapname = \hapbname$ it suffices to show $(\hapx_\epsilon u) \Leftrightarrow (\hapxb_\epsilon u)$ for all $u:\RC$.
  But $\hapx_\epsilon u$ implies $\hapx_{\epsilon-\theta} u$ for some $\theta$, by roundedness, which together with $\hapname \bsim_\epsilon \hapbname$ implies $\hapxb_\epsilon u$; and the converse is identical.

  Now the first two data the recursion principle requires are the following.
  \begin{itemize}
  \item For any $q:\Q$, we must give an element of $A$, which we denote $(\rcrat(q)\approx_{(\blank)} \blank)$.
  \item For any Cauchy approximation $x$, if we assume defined a function $\Qp \to A$, which we will denote by $\epsilon \mapsto (x_\epsilon \approx_{(\blank)} \blank)$, with the property that
    % \[ \fall{u,v:\RC}{\delta,\epsilon,\eta:\Qp} (x_\delta \approx_\eta u) \to (u\close{\delta+\epsilon} v) \to (x_\epsilon \approx_{\eta+\delta+\epsilon} v) \]
    \begin{equation}
      \fall{u:\RC}{\delta,\epsilon,\eta:\Qp} (x_\delta \approx_\eta u) \to (x_\epsilon \approx_{\eta+\delta+\epsilon} u),\label{eq:appxrec2}
    \end{equation}
    we must give an element of $A$, which we write as $(\rclim(x)\approx_{(\blank)} \blank)$.
  \end{itemize}
  In both cases, we give the required definition by using a nested $(\RC,\closesym)$-recursion, with codomain the subset of $\Qp\to\prop$ consisting of rounded families of mere propositions.
  Thinking of these propositions as zero halves of a binary relation, we will write them as $\epsilon \mapsto (\tap{\epsilon})$, with the symbol $\tapname$ referring to the whole family.
  Now we can write the codomain of these inner recursions precisely as
  \begin{narrowmultline*}
    C \defeq
    \bigg\{ \tapname :\Qp\to\prop \;\;\Big|\;\; \narrowbreak
    \fall{\epsilon:\Qp} \Big((\tap\epsilon) \Leftrightarrow \exis{\theta:\Qp} (\tap{\epsilon-\theta})\Big)\bigg\}
  \end{narrowmultline*}
  We take the required family of relations to be the remnant of the triangle inequality:
  \begin{narrowmultline*}
    (\tapname \bbsim_\epsilon \tapbname) \defeq
    \fall{\eta:\Qp} ((\tap\eta) \to (\tapb{\epsilon+\eta})) \land
    \narrowbreak
    ((\tapb\eta) \to (\tap{\epsilon+\eta})).
  \end{narrowmultline*}
  These relations are separated by the same argument as for $\bsim$, using roundedness of all elements of $C$.

  Note that if such an inner recursion succeeds, it will yield a family of predicates $\hapname : \RC\to\Qp\to \prop$ which are rounded
(since their image in $\Qp\to\prop$ lies in $C$) and satisfy
  \[ \fall{u,v:\RC}{\epsilon:\Qp} (u\close\epsilon v) \to \big((\hapx_{(\blank)} u) \bbsim_\epsilon (\hapx_{(\blank)} u)\big). \]
  Expanding out the definition of $\bbsim$, this yields precisely the third condition for $\hapname$ to belong to $A$; thus it is exactly what we need.

  It is at this point that we can give the definitions~\eqref{eq:RCappx1}--\eqref{eq:RCappx4}, as the first two clauses of each of the two inner recursions, corresponding to rational points and limits.
  In each case, we must verify that the relation is rounded and hence lies in $C$.
  In the rational-rational case~\eqref{eq:RCappx1} this is clear, while in the other cases it follows from an inductive hypothesis.
  (In~\eqref{eq:RCappx2} the relevant inductive hypothesis is that $(\rcrat(q) \approx_{(\blank)} y_\delta) : C$, while in~\eqref{eq:RCappx3} and~\eqref{eq:RCappx4} it is that $(x_\delta \approx_{(\blank)} \blank) : A$.)

  The remaining data of the sub-recursions consist of showing that \eqref{eq:RCappx1}--\eqref{eq:RCappx4} satisfy the triangle inequality on the right with respect to the constructors of $\closesym$.
  There are eight cases --- four in each sub-recursion --- corresponding to the eight possible ways that $u$, $v$, and $w$ in~\eqref{eq:RC-sim-rtri} can be chosen to be rational points or limits.
  First we consider the cases when $u$ is $\rcrat(q)$.
  \begin{enumerate}
  \item Assuming $\rcrat(q)\approx_\phi \rcrat(r)$ and $-\epsilon<|r-s|<\epsilon$, we must show $\rcrat(q)\approx_{\phi+\epsilon} \rcrat(s)$.
    But by definition of $\approx$, this reduces to the triangle inequality for rational numbers.
  \item We assume $\phi,\epsilon,\delta:\Qp$ such that $\rcrat(q)\approx_\phi \rcrat(r)$ and $\rcrat(r) \close{\epsilon-\delta} y_\delta$, and inductively that
    \begin{equation}
      \fall{\psi:\Qp}(\rcrat(q) \approx_{\psi} \rcrat(r)) \to (\rcrat(q) \approx_{\psi+\epsilon-\delta} y_\delta).\label{eq:RCappx-rtri-rrl1}
    \end{equation}
    We assume also that $\psi,\delta\mapsto (\rcrat(q) \approx_{\psi} y_\delta)$ is a Cauchy approximation with respect to $\bbsim$, i.e.\
    \begin{equation}
      \fall{\psi,\xi,\zeta:\Qp} (\rcrat(q) \approx_{\psi} y_\xi) \to (\rcrat(q) \approx_{\psi+\xi+\zeta} y_\zeta),\label{eq:RCappx-rtri-rrl2}
    \end{equation}
    although we do not need this assumption in this case.
    Indeed, \eqref{eq:RCappx-rtri-rrl1} with $\psi\defeq \phi$ yields immediately $\rcrat(q) \approx_{\phi+\epsilon-\delta} y_\delta$, and hence $\rcrat(q) \approx_{\phi+\epsilon} \rclim(y)$ by definition of $\approx$.
  \item We assume $\phi,\epsilon,\delta:\Qp$ such that $\rcrat(q)\approx_\phi \rclim(y)$ and $y_\delta \close{\epsilon-\delta} \rcrat(r)$, and inductively that
    \begin{gather}
      \fall{\psi:\Qp}(\rcrat(q) \approx_{\psi} y_\delta) \to (\rcrat(q) \approx_{\psi+\epsilon-\delta} \rcrat(r)).\label{eq:RCappx-rtri-rlr1}\\
      \fall{\psi,\xi,\zeta:\Qp} (\rcrat(q) \approx_{\psi} y_\xi) \to (\rcrat(q) \approx_{\psi+\xi+\zeta} y_\zeta).\label{eq:RCappx-rtri-rlr2}
    \end{gather}
    By definition, $\rcrat(q)\approx_\phi \rclim(y)$ means that we have $\theta:\Qp$ with $\rcrat(q) \approx_{\phi-\theta} y_\theta$.
    By assumption~\eqref{eq:RCappx-rtri-rlr2}, therefore, we have also $\rcrat(q) \approx_{\phi+\delta} y_\delta$, and then by~\eqref{eq:RCappx-rtri-rlr1} it follows that $\rcrat(q) \approx_{\phi+\epsilon} \rcrat(r)$, as desired.
  \item We assume $\phi,\epsilon,\delta,\eta:\Qp$ such that $\rcrat(q)\approx_\phi \rclim(y)$ and $y_\delta \close{\epsilon-\delta-\eta} z_\eta$, and inductively that
    \begin{gather}
      \fall{\psi:\Qp}(\rcrat(q) \approx_{\psi} y_\delta) \to (\rcrat(q) \approx_{\psi+\epsilon-\delta-\eta} z_\eta), \label{eq:RCappx-rtri-rll1}\\
      \fall{\psi,\xi,\zeta:\Qp} (\rcrat(q) \approx_{\psi} y_\xi) \to (\rcrat(q) \approx_{\psi+\xi+\zeta} y_\zeta), \label{eq:RCappx-rtri-rll2}\\
      \fall{\psi,\xi,\zeta:\Qp} (\rcrat(q) \approx_{\psi} z_\xi) \to (\rcrat(q) \approx_{\psi+\xi+\zeta} z_\zeta). \label{eq:RCappx-rtri-rll3}
    \end{gather}
    Again, $\rcrat(q)\approx_\phi \rclim(y)$ means we have $\xi:\Qp$ with $\rcrat(q) \approx_{\phi-\xi} y_\xi$, while~\eqref{eq:RCappx-rtri-rll2} then implies $\rcrat(q) \approx_{\phi+\delta} y_\delta$ and~\eqref{eq:RCappx-rtri-rll1} implies $\rcrat(q) \approx_{\phi+\epsilon-\eta} z_\eta$.
    But by definition of $\approx$, this implies $\rcrat(q) \approx_{\phi+\epsilon} \rclim(z)$ as desired.
  \end{enumerate}
  Now we move on to the cases when $u$ is $\rclim(x)$, with $x$ a Cauchy approximation.
  In this case, the ambient inductive hypothesis of the definition of $(\rclim(x) \approx_{(\blank)} {\blank}) : A$ is that we have ${(x_\delta \approx_{(\blank)} {\blank})}: A$, so that in addition to being rounded they satisfy the triangle inequality on the right.
  \begin{enumerate}\setcounter{enumi}{4}
  \item Assuming $\rclim(x)\approx_\phi \rcrat(r)$ and $-\epsilon<|r-s|<\epsilon$, we must show $\rclim(x)\approx_{\phi+\epsilon} \rcrat(s)$.
    By definition of $\approx$, the former means $x_\delta \approx_{\phi-\delta} \rcrat(r)$, so that above triangle inequality implies $x_\delta \approx_{\epsilon+\phi-\delta} \rcrat(s)$, hence $\rclim(x)\approx_{\phi+\epsilon} \rcrat(s)$ as desired.
  \item We assume $\phi,\epsilon,\delta:\Qp$ such that $\rclim(x)\approx_\phi \rcrat(r)$ and $\rcrat(r) \close{\epsilon-\delta} y_\delta$, and two unneeded inductive hypotheses.
    %
    By definition, we have $\eta:\Qp$ such that $x_\eta \approx_{\phi-\eta} \rcrat(r)$, so the inductive triangle inequality gives $x_\eta \approx_{\phi+\epsilon-\eta-\delta} y_\delta$.
    The definition of $\approx$ then immediately yields $\rclim(x) \approx_{\phi+\epsilon} \rclim(y)$.
  \item We assume $\phi,\epsilon,\delta:\Qp$ such that $\rclim(x)\approx_\phi \rclim(y)$ and $y_\delta \close{\epsilon-\delta} \rcrat(r)$, and two unneeded inductive hypotheses.
    By definition we have $\xi,\theta:\Qp$ such that $x_\xi \approx_{\phi-\xi-\theta} y_\theta$.
    Since $y$ is a Cauchy approximation, we have $y_\theta \close{\theta+\delta} y_\delta$, so the inductive triangle inequality gives $x_\xi \approx_{\phi+\delta-\xi} y_\delta$ and then $x_\xi \close{\phi+\epsilon-\xi} \rcrat(r)$.
    The definition of $\approx$ then gives $\rclim(x) \approx_{\phi+\epsilon}\rcrat(r)$, as desired.
  \item Finally, we assume $\phi,\epsilon,\delta,\eta:\Qp$ such that $\rclim(x)\approx_\phi \rclim(y)$ and $y_\delta \close{\epsilon-\delta-\eta} z_\eta$.
    Then as before we have $\xi,\theta:\Qp$ with $x_\xi \approx_{\phi-\xi-\theta} y_\theta$, and two applications of the triangle inequality suffices as before.
  \end{enumerate}

  This completes the two inner recursions, and thus the definitions of the families of relations $(\rcrat(q)\approx_{(\blank)}\blank)$ and $(\rclim(x)\approx_{(\blank)}\blank)$.
  Since all are elements of $A$, they are rounded and satisfy the triangle inequality on the right with respect to $\closesym$.
% , and satisfy~\eqref{eq:appxrec2}.
  What remains is to verify the conditions relating to $\bsim$, which is to say that these relations satisfy the triangle inequality on the \emph{left} with respect to the constructors of $\closesym$.
  The four cases correspond to the four choices of rational or limit points for $u$ and $v$ in~\eqref{eq:RC-sim-ltri}, and since they are all mere propositions, we may apply $\RC$-induction and assume that $w$ is also either rational or a limit.
  This yields another eight cases, whose proofs are essentially identical to those just given; so we will not subject the reader to them.
\end{proof}

We can now prove:

\begin{thm}\label{thm:RC-sim-characterization}
  For any $u,v:\RC$ and $\epsilon:\Qp$ we have $(u\close\epsilon v) = (u\approx_\epsilon v)$.
\end{thm}
\begin{proof}
  Since both are mere propositions, it suffices to prove bidirectional implication.
  For the left-to-right direction, we use $\closesym$-induction applied to $C(u,v,\epsilon)\defeq (u\approx_\epsilon v)$.
  Thus, it suffices to consider the four constructors of $\closesym$.
  In each case, $u$ and $v$ are specialized to either rational points or limits, so that the definition of $\approx$ evaluates, and the inductive hypothesis always applies.

  For the right-to-left direction, we use $\RC$-induction to assume that $u$ and $v$ are rational points or limits, allowing $\approx$ to evaluate.
  But now the definitions of $\approx$, and the inductive hypotheses, supply exactly the data required for the relevant constructors of $\closesym$.
\end{proof}

\index{encode-decode method}%
Stretching a point, one might call $\approx$ a fibration of ``codes'' for $\closesym$, with the two directions of the above proof being \encode and \decode respectively.
By the definition of $\approx$, from \cref{thm:RC-sim-characterization} we get equivalences
\begin{align*}
  (\rcrat(q) \close\epsilon \rcrat(r))  &=
  (-\epsilon < q - r < \epsilon)\\
  (\rcrat(q) \close\epsilon \rclim(y)) &=
  \exis{\delta : \Qp} \rcrat(q) \close{\epsilon - \delta} y_\delta\\
  (\rclim(x) \close\epsilon \rcrat(r)) &=
  \exis{\delta : \Qp} x_\delta \close{\epsilon - \delta} \rcrat(r)\\
  (\rclim(x) \close\epsilon \rclim(y)) &=
  \exis{\delta, \eta : \Qp} x_\delta \close{\epsilon - \delta - \eta} y_\eta.
\end{align*}
Our proof also provides the following additional information.

\begin{cor}
  \index{triangle!inequality for R@inequality for $\RC$}%
  \indexsee{inequality!triangle}{triangle inequality}%
  $\closesym$ is rounded\index{rounded!relation} and satisfies the triangle inequality:
    \begin{gather}
      \eqvspaced{
        (u \close\epsilon v)
      }{
        \exis{\theta : \Qp} u \close{\epsilon - \theta} v
      }\\
      (u\close\epsilon v) \to (v\close\delta w) \to (u\close{\epsilon+\delta} w). \label{item:RC-sim-triangle}
    \end{gather}
\end{cor}
% \begin{proof}
%   The construction of $\approx$ showed simultaneously that it is rounded, and satisfies ``triangle inequalities'' such as
%   \[ (u\approx_\epsilon v) \to (v\close\delta w) \to (u\approx_{\epsilon+\delta} w). \]
%   Thus, both properties follow from \cref{thm:RC-sim-characterization}.
% \end{proof}

With the triangle inequality in hand, we can show that ``limits'' of Cauchy approximations actually behave like limits.

\begin{lem}\label{thm:RC-sim-lim}
  For any $u:\RC$, Cauchy approximation $y$, and $\epsilon,\delta:\Qp$, if $u\close\epsilon y_\delta$ then $u\close{\epsilon+\delta} \rclim(y)$.
\end{lem}
\begin{proof}
  We use $\RC$-induction on $u$.
  If $u$ is $\rcrat(q)$, then this is exactly the second constructor of $\closesym$.
  Now suppose $u$ is $\rclim(x)$, and that each $x_\eta$ has the property that for any $y,\epsilon,\delta$, if $x_\eta\close\epsilon y_\delta$ then $x_\eta \close{\epsilon+\delta} \rclim(y)$.
  In particular, taking $y\defeq x$ and $\delta\defeq\eta$ in this assumption, we conclude that $x_\eta \close{\eta+\theta} \rclim(x)$ for any $\eta,\theta:\Qp$.

  Now let $y,\epsilon,\delta$ be arbitrary and assume $\rclim(x) \close\epsilon y_\delta$.
  By roundedness, there is a $\theta$ such that $\rclim(x) \close{\epsilon-\theta} y_\delta$.
  Then by the above observation, for any $\eta$ we have $x_\eta \close{\eta+\theta/2} \rclim(x)$, and hence $x_\eta \close{\epsilon+\eta-\theta/2} y_\delta$ by the triangle inequality.
  Hence, the fourth constructor of $\closesym$ yields $\rclim(x) \close{\epsilon+2\eta+\delta-\theta/2} \rclim(y)$.
  Thus, if we choose $\eta \defeq \theta/4$, the result follows.
\end{proof}

\begin{lem}\label{thm:RC-sim-lim-term}
  For any Cauchy approximation $y$ and any $\delta,\eta:\Qp$ we have $y_\delta \close{\delta+\eta} \rclim(y)$.
\end{lem}
\begin{proof}
  Take $u\defeq y_\delta$ and $\epsilon\defeq \eta$ in the previous lemma.
\end{proof}

\begin{rmk}
  We might have expected to have $y_\delta \close{\delta} \rclim(y)$, but this fails in examples.
  For instance, consider $x$ defined by $x_\epsilon \defeq \epsilon$.
  Its limit is clearly $0$, but we do not have $|\epsilon - 0 |<\epsilon$, only $\le$.
\end{rmk}

As an application, \cref{thm:RC-sim-lim-term} enables us to show that the extensions of Lipschitz functions from \cref{RC-extend-Q-Lipschitz} are unique.

\begin{lem}\label{RC-continuous-eq}
  \index{function!continuous}%
  Let $f,g:\RC\to\RC$ be continuous, in the sense that
  \[ \fall{u:\RC}{\epsilon:\Qp}\exis{\delta:\Qp}\fall{v:\RC} (u\close\delta v) \to (f(u) \close\epsilon f(v)) \]
  and analogously for $g$.
  If $f(\rcrat(q))=g(\rcrat(q))$ for all $q:\Q$, then $f=g$.
\end{lem}
\begin{proof}
  We prove $f(u)=g(u)$ for all $u$ by $\RC$-induction.
  The rational case is just the hypothesis.
  Thus, suppose $f(x_\delta)=g(x_\delta)$ for all $\delta$.
  We will show that $f(\rclim(x))\close\epsilon g(\rclim(x))$ for all $\epsilon$, so that the path constructor of $\RC$ applies.

  Since $f$ and $g$ are continuous, there exist $\theta,\eta$ such that for all $v$, we have
  \begin{align*}
    (\rclim(x)\close\theta v) &\to (f(\rclim(x)) \close{\epsilon/2} f(v))\\
    (\rclim(x)\close\eta v) &\to (g(\rclim(x)) \close{\epsilon/2} g(v)).
  \end{align*}
  Choosing $\delta < \min(\theta,\eta)$, by \cref{thm:RC-sim-lim-term} we have both $\rclim(x)\close\theta y_\delta$ and $\rclim(x)\close\eta y_\delta$.
  Hence
  \[ f(\rclim(x)) \close{\epsilon/2} f(y_\delta) = g(y_\delta) \close{\epsilon/2} g(\rclim(x))\]
  and thus $f(\rclim(x))\close\epsilon g(\rclim(x))$ by the triangle inequality.
\end{proof}

\subsection{The algebraic structure of Cauchy reals}
\label{sec:algebr-struct-cauchy}

We first define the additive structure $(\RC, 0, {+}, {-})$. Clearly, the additive unit element
$0$ is just $\rcrat(0)$, while the additive inverse ${-} : \RC \to \RC$ is obtained as the
extension of the additive inverse ${-} : \Q \to \Q$, using \cref{RC-extend-Q-Lipschitz}
with Lipschitz constant~$1$. We have to work a bit harder for addition.

\begin{lem} \label{RC-binary-nonexpanding-extension}
  Suppose $f : \Q \times \Q \to \Q$ satisfies, for all $q, r, s : \Q$,
  %
  \begin{equation*}
    |f(q, s) - f(r, s)| \leq |q - r|
    \qquad\text{and}\qquad
    |f(q, r) - f(q, s)| \leq |r - s|.
  \end{equation*}
  %
  Then there is a function $\bar{f} : \RC \times \RC \to \RC$ such that
  $\bar{f}(\rcrat(q), \rcrat(r)) = f(q,r)$ for all $q, r : \Q$. Furthermore,
  for all $u, v, w : \RC$ and $q : \Qp$,
  %
  \begin{equation*}
    u \close\epsilon v \Rightarrow \bar{f}(u,w) \close\epsilon \bar{f}(v,w)
    \quad\text{and}\quad
    v \close\epsilon w \Rightarrow \bar{f}(u,v) \close\epsilon \bar{f}(u,w).
  \end{equation*}
\end{lem}

\begin{proof}
  We use $(\RC, {\closesym})$-recursion to construct the curried form of $\bar{f}$ as a map
  $\RC \to A$ where $A$ is the space of non-expanding\index{function!non-expanding}\index{non-expanding function} real-valued
  functions:
  %
  \begin{equation*}
    A \defeq
    \setof{ h : \RC \to \RC |
      \fall{\epsilon : \Qp} \fall{u, v : \RC}
      u \close\epsilon v \Rightarrow h(u) \close\epsilon h(v)
    }.
  \end{equation*}
  %
  We shall also need a suitable $\bsim_\epsilon$ on $A$, which we define as
  %
  \begin{equation*}
    (h \bsim_\epsilon k) \defeq \fall{u : \RC} h(u) \close\epsilon k(u).
  \end{equation*}
  %
  Clearly, if $\fall{\epsilon : \Qp} h \bsim_\epsilon k$ then $h(u) = k(u)$ for all $u :
  \RC$, so $\bsim$ is separated.

  For the base case we define $\bar{f}(\rcrat(q)) : A$, where $q : \Q$, as the
  extension of the Lipschitz map $\lam{r} f(q,r)$ from $\Q \to \Q$ to $\RC \to \RC$, as
  constructed in \cref{RC-extend-Q-Lipschitz} with Lipschitz constant~$1$. Next, for a
  Cauchy approximation $x$, we define $\bar{f}(\rclim(x)) : \RC \to \RC$ as
  %
  \begin{equation*}
    \bar{f}(\rclim(x))(v) \defeq \rclim (\lam{\epsilon} \bar{f}(x_\epsilon)(v)).
  \end{equation*}
  %
  For this to be a valid definition, $\lam{\epsilon} \bar{f}(x_\epsilon)(v)$ should be a
  Cauchy approximation, so consider any $\delta, \epsilon : \Q$. Then by assumption
  $\bar{f}(x_\delta) \bsim_{\delta + \epsilon} \bar{f}(x_\epsilon)$, hence
  $\bar{f}(x_\delta)(v) \close{\delta + \epsilon} \bar{f}(x_\epsilon)(v)$. Furthermore,
  $\bar{f}(\rclim(x))$ is non-expanding because $\bar{f}(x_\epsilon)$ is such by induction
  hypothesis. Indeed, if $u \close\epsilon v$ then, for all $\epsilon : \Q$,
  %
  \begin{equation*}
    \bar{f}(x_{\epsilon/3})(u) \close{\epsilon/3} \bar{f}(x_{\epsilon/3})(v),
  \end{equation*}
  %
  therefore $\bar{f}(\rclim(x))(u) \close\epsilon \bar{f}(\rclim(x))(v)$ by the fourth constructor of $\closesym$.

  We still have to check four more conditions, let us illustrate just one. Suppose
  $\epsilon : \Qp$ and for some $\delta : \Qp$ we have $\rcrat(q) \close{\epsilon - \delta}
  y_\delta$ and $\bar{f}(\rcrat(q)) \bsim_{\epsilon - \delta} \bar{f}(y_\delta)$. To show
  $\bar{f}(\rcrat(q)) \bsim_\epsilon \bar{f}(\rclim(y))$, consider any $v : \RC$ and observe that
  %
  \begin{equation*}
    \bar{f}(\rcrat(q))(v) \close{\epsilon - \delta} \bar{f}(y_\delta)(v).
  \end{equation*}
  %
  Therefore, by the second constructor of $\closesym$, we have
  \narrowequation{\bar{f}(\rcrat(q))(v) \close\epsilon \bar{f}(\rclim(y))(v)}
  as required.
\end{proof}

We may apply \cref{RC-binary-nonexpanding-extension} to any bivariate rational function
which is non-expanding separately in each variable. Addition is such a function, therefore
we get ${+} : \RC \times \RC \to \RC$.
\indexdef{addition!of Cauchy reals}%
Furthermore, the extension is unique as long as we
require it to be non-expanding in each variable, and just as in the univariate case,
identities on rationals extend to identities on reals. Since composition of non-expanding
maps is again non-expanding, we may conclude that addition satisfies the usual properties,
such as commutativity and associativity.
\index{associativity!of addition!of Cauchy reals}%
Therefore, $(\RC, 0, {+}, {-})$ is a commutative
group.

We may also apply \cref{RC-binary-nonexpanding-extension} to the functions $\min : \Q \times
\Q \to \Q$ and $\max : \Q \times \Q \to \Q$, which turns $\RC$ into a lattice. The partial
order $\leq$ on $\RC$ is defined in terms of $\max$ as
%
\symlabel{leq-RC}
\index{order!non-strict}%
\index{non-strict order}%
\begin{equation*}
  (u \leq v) \defeq (\max(u, v) = v).
\end{equation*}
%
The relation $\leq$ is a partial order because it is such on $\Q$, and the axioms of a
partial order are expressible as equations in terms of $\min$ and $\max$, so they transfer
to $\RC$.

\index{absolute value}%
Another function which extends to $\RC$ by the same method is the absolute value $|{\blank}|$.
Again, it has the expected properties because they transfer from $\Q$ to $\RC$.

\symlabel{lt-RC}
From $\leq$ we get the strict order $<$ by
\index{strict!order}%
\index{order!strict}%
%
\begin{equation*}
  (u < v) \defeq \exis{q, r : \Q} (u \leq \rcrat(q)) \land (q < r) \land (\rcrat(r) \leq v).
\end{equation*}
%
That is, $u < v$ holds when there merely exists a pair of rational numbers $q < r$ such that $x \leq
\rcrat(q)$ and $\rcrat(r) \leq v$. It is not hard to check that $<$ is irreflexive and
transitive, and has other properties that are expected for an ordered field.
The archimedean principle follows directly from the definition of~$<$.

\index{ordered field!archimedean}%
\begin{thm}[Archimedean principle for $\RC$] \label{RC-archimedean}
  %
  For every $u, v : \RC$ such that $u < v$ there merely exists $q : \Q$ such that $u < q < v$.
\end{thm}

\begin{proof}
  From $u < v$ we merely get $r, s : \Q$ such that $u \leq r < s \leq v$, and we may take $q
  \defeq (r + s) / 2$.
\end{proof}

We now have enough structure on $\RC$ to express $u \close\epsilon v$ with standard concepts.

\begin{lem}\label{thm:RC-le-grow}
  If $q:\Q$ and $u:\RC$ satisfy $u\le \rcrat(q)$, then for any $v:\RC$ and $\epsilon:\Qp$, if $u\close\epsilon v$ then $v\le \rcrat(q+\epsilon)$.
\end{lem}
\begin{proof}
  Note that the function $\max(\rcrat(q),\blank):\RC\to\RC$ is Lipschitz with constant $1$.
  First consider the case when $u=\rcrat(r)$ is rational.
  For this we use induction on $v$.
  If $v$ is rational, then the statement is obvious.
  If $v$ is $\rclim(y)$, we assume inductively that for any $\epsilon,\delta$, if $\rcrat(r)\close\epsilon y_\delta$ then $y_\delta \le \rcrat(q+\epsilon)$, i.e.\ $\max(\rcrat(q+\epsilon),y_\delta)=\rcrat(q+\epsilon)$.

  Now assuming $\epsilon$ and $\rcrat(r)\close\epsilon \rclim(y)$, we have $\theta$ such that $\rcrat(r)\close{\epsilon-\theta} \rclim(y)$, hence $\rcrat(r)\close\epsilon y_\delta$ whenever $\delta<\theta$.
  Thus, the inductive hypothesis gives $\max(\rcrat(q+\epsilon),y_\delta)=\rcrat(q+\epsilon)$ for such $\delta$.
  But by definition,
  \[\max(\rcrat(q+\epsilon),\rclim(y)) \jdeq \rclim(\lam{\delta} \max(\rcrat(q+\epsilon),y_\delta)).\]
  Since the limit of an eventually constant Cauchy approximation is that constant, we have
  \[\max(\rcrat(q+\epsilon),\rclim(y)) = \rcrat(q+\epsilon),\] hence $\rclim(y)\le \rcrat(q+\epsilon)$.

  Now consider a general $u:\RC$.
  Since $u\le \rcrat(q)$ means $\max(\rcrat(q),u)=\rcrat(q)$, the assumption $u\close\epsilon v$ and the Lipschitz property of $\max(\rcrat(q),-)$ imply $\max(\rcrat(q),v) \close\epsilon \rcrat(q)$.
  Thus, since $\rcrat(q)\le \rcrat(q)$, the first case implies $\max(\rcrat(q),v) \le \rcrat(q+\epsilon)$, and hence $v\le \rcrat(q+\epsilon)$ by transitivity of $\le$.
\end{proof}

\begin{lem}\label{thm:RC-lt-open}
  Suppose $q:\Q$ and $u:\RC$ satisfy $u<\rcrat(q)$.  Then:
  \begin{enumerate}
  \item For any $v:\RC$ and $\epsilon:\Qp$, if $u\close\epsilon v$ then $v< \rcrat(q+\epsilon)$.\label{item:RCltopen1}
  \item There exists $\epsilon:\Qp$ such that for any $v:\RC$, if $u\close\epsilon v$ we have $v<\rcrat(q)$.\label{item:RCltopen2}
  \end{enumerate}
\end{lem}
\begin{proof}
  By definition, $u<\rcrat(q)$ means there is $r:\Q$ with $r<q$ and $u\le \rcrat(r)$.
  Then by \cref{thm:RC-le-grow}, for any $\epsilon$, if $u\close\epsilon v$ then $v\le \rcrat(r+\epsilon)$.
  Conclusion~\ref{item:RCltopen1} follows immediately since $r+\epsilon<q+\epsilon$, while for~\ref{item:RCltopen2} we can take any $\epsilon <q-r$.
\end{proof}

We are now able to show that the auxiliary relation $\closesym$ is what we think it is.

\begin{thm} \label{RC-sim-eqv-le}
  \index{distance}%
  $\eqv{(u \close\epsilon v)}{(|u - v| < \rcrat(\epsilon))}$
  for all $u, v : \RC$ and $\epsilon : \Qp$.
\end{thm}
\begin{proof}
  The Lipschitz properties of subtraction and absolute value imply that if $u\close\epsilon v$, then $|u-v| \close\epsilon |u-u| = 0$.
  Thus, for the left-to-right direction, it will suffice to show that if $u\close\epsilon 0$, then $|u|<\rcrat(\epsilon)$.
  We proceed by $\RC$-induction on $u$.

  If $u$ is rational, the statement follows immediately since absolute value and order extend the standard ones on $\Qp$.
  If $u$ is $\rclim(x)$, then by roundedness we have $\theta:\Qp$ with $\rclim(x)\close{\epsilon-\theta} 0$.
  By the triangle inequality, therefore, we have $x_{\theta/3} \close{\epsilon-2\theta/3} 0$, so the inductive hypothesis yields $|x_{\theta/3}|<\rcrat(\epsilon-2\theta/3)$.
  But $x_{\theta/3} \close{2\theta/3} \rclim(x)$, hence $|x_{\theta/3}| \close{2\theta/3} |\rclim(x)|$ by the Lipschitz property, so \cref{thm:RC-lt-open}\ref{item:RCltopen1} implies $|\rclim(x)|<\rcrat(\epsilon)$.

  In the other direction, we use $\RC$-induction on $u$ and $v$.
  If both are rational, this is the first constructor of $\closesym$.

  If $u$ is $\rcrat(q)$ and $v$ is $\rclim(y)$, we assume inductively that for any $\epsilon,\delta$, if $|\rcrat(q)-y_\delta|<\rcrat(\epsilon)$ then $\rcrat(q) \close{\epsilon} y_\delta$.
  Fix an $\epsilon$ such that $|\rcrat(q) - \rclim(y)|<\rcrat(\epsilon)$.
  Since $\Q$ is order-dense in $\RC$, there exists $\theta<\epsilon$ with $|\rcrat(q) - \rclim(y)|<\rcrat(\theta)$.
  Now for any $\delta,\eta$ we have $\rclim(y)\close{2\delta} y_\delta$, hence by the Lipschitz property
  \[ |\rcrat(q) - \rclim(y)| \close{\delta+\eta} |\rcrat(q) - y_\delta|. \]
  Thus, by \cref{thm:RC-lt-open}\ref{item:RCltopen1}, we have $|\rcrat(q) - y_\delta| < \rcrat(\theta+2\delta)$.
  So by the inductive hypothesis, $\rcrat(q) \close{\theta+2\delta} y_\delta$, and thus $\rcrat(q)\close{\theta+4\delta} \rclim(y)$ by the triangle inequality.
  Thus, it suffices to choose $\delta \defeq (\epsilon-\theta)/4$.

  The remaining two cases are entirely analogous.
\end{proof}

\indexdef{multiplication!of Cauchy reals}%
Next, we would like to equip $\RC$ with multiplicative structure. For each $q : \Q$ the
map $r \mapsto q \cdot r$ is Lipschitz with constant\footnote{We defined Lipschitz
  constants as \emph{positive} rational numbers.} $|q| + 1$, and so we can extend it to
multiplication by $q$ on the real numbers. Therefore $\RC$ is a vector space\index{vector!space} over $\Q$.
In general, we can define multiplication of real numbers as
%
\begin{equation}
  u \cdot v \defeq
  {\textstyle \frac{1}{2}} \cdot ((u + v)^2 - u^2 - v^2),\label{mult-from-square}
\end{equation}
%
so we just need squaring\index{squaring function} $u \mapsto u^2$ as a map $\RC \to \RC$. Squaring is not a
Lipschitz map, but it is Lipschitz on every bounded domain, which allows us to patch it
together. Define the open and closed intervals
%
\indexdef{interval!open and closed}%
\indexdef{open!interval}%
\indexdef{closed!interval}%
\begin{equation*}
  [u,v] \defeq \setof{ x : \RC | u \leq x \leq v }
  \qquad\text{and}\qquad
  (u,v) \defeq \setof{ x : \RC | u < x < v }.
\end{equation*}
%
Although technically an element of $[u,v]$ or $(u,v)$ is a Cauchy real number together with a proof, since the latter inhabits a mere proposition it is uninteresting.
Thus, as is common with subset types, we generally write simply $x:[u,v]$ whenever $x:\RC$ is such that $u\leq x \leq v$, and similarly.

\begin{thm} \label{RC-squaring}
  %
  There exists a unique function ${(\blank)}^2 : \RC \to \RC$ which extends squaring $q \mapsto
  q^2$ of rational numbers and satisfies
  %
  \begin{equation*}
    \fall{n : \N}
    \fall{u, v : [-n, n]}
    |u^2 - v^2| \leq 2 \cdot n \cdot |u - v|.
  \end{equation*}
\end{thm}

\begin{proof}
  We first observe that for every $u : \RC$ there merely exists $n : \N$ such that $-n
  \leq u \leq n$, see \cref{ex:traditional-archimedean}, so the map
  %
  \begin{equation*}
    e : \Parens{\sm{n : \N} [-n, n]} \to \RC
    \qquad\text{defined by}\qquad
    e(n, x) \defeq x
  \end{equation*}
  %
  is surjective. Next, for each $n : \N$, the squaring map
  %
  \begin{equation*}
    s_n : \setof{ q : \Q | -n \leq q \leq n } \to \Q
    \qquad\text{defined by}\qquad
    s_n(q) \defeq q^2
  \end{equation*}
  %
  is Lipschitz with constant $2 n$, so we can use \cref{RC-extend-Q-Lipschitz} to
  extend it to a map $\bar{s}_n : [-n, n] \to \RC$ with Lipschitz constant $2 n$, see
  \cref{RC-Lipschitz-on-interval} for details. The maps $\bar{s}_n$ are compatible: if
  $m < n$ for some $m, n : \N$ then $s_n$ restricted to $[-m, m]$ must agree with $s_m$
  because both are Lipschitz, and therefore continuous in the sense
  of~\cref{RC-continuous-eq}. Therefore, by \cref{lem:images_are_coequalizers} the map
  %
  \begin{equation*}
    \Parens{\sm{n : \N} [-n, n]} \to \RC,
    \qquad\text{given by}\qquad
    (n, x) \mapsto s_n(x)
  \end{equation*}
  %
  factors uniquely through $\RC$ to give us the desired function.
\end{proof}

At this point we have the ring structure of the reals and the archimedean order. To
establish $\RC$ as an archimedean ordered field, we still need inverses.

\begin{thm}
  \index{apartness}%
  A Cauchy real is invertible if, and only if, it is apart from zero.
\end{thm}

\begin{proof}
  First, suppose $u : \RC$ has an inverse $v : \RC$ By the archimedean principle there is $q :
  \Q$ such that $|v| < q$. Then $1 = |u v| < |u| \cdot v < |u| \cdot q$ and hence $|u| >
  1/q$, which is to say that $u \apart 0$.

  For the converse we construct the inverse map
  %
  \begin{equation*}
    ({\blank})^{-1} : \setof{ u : \RC | u \apart 0 } \to \RC
  \end{equation*}
  %
  by patching together functions, similarly to the construction of squaring in
  \cref{RC-squaring}. We only outline the main steps. For every $q : \Q$ let
  %
  \begin{equation*}
    [q, \infty) \defeq \setof{u : \RC | q \leq u}
    \qquad\text{and}\qquad
    (-\infty, q] \defeq \setof{u : \RC | u \leq -q}.
  \end{equation*}
  %
  Then, as $q$ ranges over $\Qp$, the types $(-\infty, q]$ and $[q, \infty)$ jointly cover
  $\setof{u : \RC | u \apart 0}$. On each such $[q, \infty)$ and $(-\infty, q]$ the
  inverse function is obtained by an application of \cref{RC-extend-Q-Lipschitz}
  with Lipschitz constant $1/q^2$. Finally, \cref{lem:images_are_coequalizers}
  guarantees that the inverse function factors uniquely through $\setof{ u : \RC | u
    \apart 0 }$.
\end{proof}

We summarize the algebraic structure of $\RC$ with a theorem.

\begin{thm} \label{RC-archimedean-ordered-field}
  The Cauchy reals form an archimedean ordered field.
\end{thm}

\subsection{Cauchy reals are Cauchy complete}
\label{sec:cauchy-reals-cauchy-complete}

We constructed $\RC$ by closing $\Q$ under limits of Cauchy approximations, so it better
be the case that $\RC$ is Cauchy complete. Thanks to \cref{RC-sim-eqv-le} there is no
difference between a Cauchy approximation $x : \Qp \to \RC$ as defined in the construction
of $\RC$, and a Cauchy approximation in the sense of \cref{defn:cauchy-approximation}
(adapted to $\RC$).

Thus, given a Cauchy approximation $x : \Qp \to \RC$ it is quite natural to expect that
$\rclim(x)$ is its limit, where the notion of limit is defined as in
\cref{defn:cauchy-approximation}. But this is so by \cref{RC-sim-eqv-le} and
\cref{thm:RC-sim-lim-term}. We have proved:

\begin{thm}
  Every Cauchy approximation in $\RC$ has a limit.
\end{thm}

An archimedean ordered field in which every Cauchy approximation has a limit is called
\define{Cauchy complete}.
\indexdef{Cauchy!completeness}%
\indexdef{complete!ordered field, Cauchy}%
\index{ordered field}%
The Cauchy reals are the least such field.

\begin{thm} \label{RC-initial-Cauchy-complete}
  The Cauchy reals embed into every Cauchy complete ar\-chi\-me\-de\-an ordered field.
\end{thm}

\begin{proof}
  \index{limit!of a Cauchy approximation}%
  Suppose $F$ is a Cauchy complete archimedean ordered field. Because limits are unique,
  there is an operator $\lim$ which takes Cauchy approximations in $F$ to their limits. We
  define the embedding $e : \RC \to F$ by $(\RC, {\closesym})$-recursion as
  %
  \begin{equation*}
    e(\rcrat(q)) \defeq q
    \qquad\text{and}\qquad
    e(\rclim(x)) \defeq \lim (e \circ x).
  \end{equation*}
  %
  A suitable $\bsim$ on $F$ is
  %
  \begin{equation*}
    (a \bsim_\epsilon b) \defeq |a - b| < \epsilon.
  \end{equation*}
  %
  This is a separated relation because $F$ is archimedean. The rest of the clauses for
  $(\RC, {\closesym})$-recursion are easily checked. One would also have to check that $e$ is
  an embedding of ordered fields which fixes the rationals.
\end{proof}

\index{real numbers!Cauchy|)}%

\section{Comparison of Cauchy and Dedekind reals}
\label{sec:comp-cauchy-dedek}

\index{real numbers!Dedekind|(}%
\index{real numbers!Cauchy|(}%
\index{depression|(}

Let us also say something about the relationship between the Cauchy and Dedekind reals. By
\cref{RC-archimedean-ordered-field}, $\RC$ is an archimedean ordered field. It is also
admissible\index{ordered field!admissible} for $\Omega$, as can be easily checked. (In case $\Omega$ is the initial
$\sigma$-frame
\index{initial!sigma-frame@$\sigma$-frame}%
\index{sigma-frame@$\sigma$-frame!initial}%
it takes a simple induction, while in other cases it is immediate.)
Therefore, by \cref{RD-final-field} there is an embedding of ordered fields
%
\begin{equation*}
  \RC \to \RD
\end{equation*}
%
which fixes the rational numbers.
(We could also obtain this from \cref{RC-initial-Cauchy-complete,RD-cauchy-complete}.)
In general we do not expect $\RC$ and $\RD$ to coincide
without further assumptions.

\begin{lem} \label{lem:untruncated-linearity-reals-coincide}
  %
  If for every $x : \RD$ there merely exists
  %
  \begin{equation}
    \label{eq:untruncated-linearity}
    c : \prd{q, r : \Q} (q < r) \to (q < x) + (x < r)
  \end{equation}
  %
  then the Cauchy and Dedekind reals coincide.
\end{lem}

\begin{proof}
  Note that the type in~\eqref{eq:untruncated-linearity} is an untruncated variant
  of~\eqref{eq:RD-linear-order}, which states that~$<$ is a weak linear order.
  We already know that $\RC$ embeds into $\RD$, so it suffices to show that every Dedekind
  real merely is the limit of a Cauchy sequence\index{Cauchy!sequence} of rational numbers.

  Consider any $x : \RD$. By assumption there merely exists $c$ as in the statement of the
  lemma, and by inhabitation of cuts\index{cut!Dedekind} there merely exist $a, b : \Q$ such that $a < x < b$.
  We construct a sequence\index{sequence} $f : \N \to \setof{ \pairr{q, r} \in \Q \times \Q | q < r }$ by
  recursion:
  %
  \begin{enumerate}
  \item Set $f(0) \defeq \pairr{a, b}$.
  \item Suppose $f(n)$ is already defined as $\pairr{q_n, r_n}$ such that $q_n < r_n$.
    Define $s \defeq (2 q_n + r_n)/3$ and $t \defeq (q_n + 2 r_n)/3$. Then $c(s,t)$
    decides between $s < x$ and $x < t$. If it decides $s < x$ then we set $f(n+1) \defeq
    \pairr{s, r_n}$, otherwise $f(n+1) \defeq \pairr{q_n, t}$.
  \end{enumerate}
  %
  Let us write $\pairr{q_n, r_n}$ for the $n$-th term of the sequence~$f$. Then it is easy
  to see that $q_n < x < r_n$ and $|q_n - r_n| \leq (2/3)^n \cdot |q_0 - r_0|$ for all $n
  : \N$. Therefore $q_0, q_1, \ldots$ and $r_0, r_1, \ldots$ are both Cauchy sequences
  converging to the Dedekind cut~$x$. We have shown that for every $x : \RD$ there merely
  exists a Cauchy sequence converging to $x$.
\end{proof}

The lemma implies that either countable choice or excluded middle suffice for coincidence
of $\RC$ and $\RD$.

\begin{cor} \label{when-reals-coincide}
  \index{axiom!of choice!countable}%
  \index{excluded middle}%
  If excluded middle or countable choice holds then $\RC$ and $\RD$ are equivalent.
\end{cor}

\begin{proof}
  If excluded middle holds then $(x < y) \to (x < z) + (z < y)$ can be proved: either $x <
  z$ or $\lnot (x < z)$. In the former case we are done, while in the latter we get $z <
  y$ because $z \leq x < y$. Therefore, we get~\eqref{eq:untruncated-linearity} so that we
  can apply \cref{lem:untruncated-linearity-reals-coincide}.

  Suppose countable choice holds. The set $S = \setof{ \pairr{q, r} \in \Q \times \Q | q <
    r }$ is equivalent to $\N$, so we may apply countable choice to the statement that $x$
  is located,
  %
  \begin{equation*}
    \fall{\pairr{q, r} : S} (q < x) \lor (x < r).
  \end{equation*}
  %
  Note that $(q < x) \lor (x < r)$ is expressible as an existential statement $\exis{b :
    \bool} (b = \bfalse \to q < x) \land (b = \btrue \to x < r)$. The (curried form) of
  the choice function is then precisely~\eqref{eq:untruncated-linearity} so that
  \cref{lem:untruncated-linearity-reals-coincide} is applicable again.
\end{proof}

\index{real numbers!Dedekind|)}%
\index{real numbers!Cauchy|)}%
\index{real numbers!agree}%

\index{depression|)}

\section{Compactness of the interval}
\label{sec:compactness-interval}

\index{mathematics!classical|(}%
\index{mathematics!constructive|(}%

We already pointed out that our constructions of reals are entirely compatible with
classical logic. Thus, by assuming the law of excluded middle~\eqref{eq:lem} and the axiom
of choice~\eqref{eq:ac} we could develop classical analysis,\index{classical!analysis}\index{analysis!classical} which would essentially
amount to copying any standard book on analysis.

\index{analysis!constructive}%
\index{constructive!analysis}%
Nevertheless, anyone interested in computation, for example a numerical analyst, ought to
be curious about developing analysis in a computationally meaningful setting. That
analysis in a constructive setting is even possible was demonstrated by~\cite{Bishop1967}.
As a sample of the differences and similarities between classical and constructive
analysis we shall briefly discuss just one topic---compactness of the closed interval
$[0,1]$ and a couple of theorems surrounding the concept.

Compactness is no exception to the common phenomenon in constructive mathematics that
classically equivalent notions bifurcate. The three most frequently used notions of
compactness are:
%
\indexdef{compactness}%
\begin{enumerate}
\item \define{metrically compact:} ``Cauchy complete and totally bounded'',
  \indexdef{metrically compact}%
  \indexdef{compactness!metric}%
\item \define{Bolzano--Weierstra\ss{} compact:} ``every sequence has a convergent subsequence'',
  \index{compactness!Bolzano--Weierstrass@Bolzano--Weierstra\ss{}}%
  \indexsee{Bolzano--Weierstrass@Bolzano--Weierstra\ss{}}{compactness}%
  \index{sequence}%
\item \define{Heine--Borel compact:} ``every open cover has a finite subcover''.
  \index{compactness!Heine--Borel}%
  \indexsee{Heine--Borel}{compactness}%
\end{enumerate}
%
These are all equivalent in classical mathematics.
Let us see how they fare in homotopy type theory. We can use either the Dedekind or the
Cauchy reals, so we shall denote the reals just as~$\R$. We first recall several basic
definitions.

\indexsee{space!metric}{metric space}
\index{metric space|(}%

\begin{defn} \label{defn:metric-space}
  A \define{metric space}
  \indexdef{metric space}%
  $(M, d)$ is a set $M$ with a map $d : M \times M \to \R$
  satisfying, for all $x, y, z : M$,
  %
  \begin{align*}
    d(x,y) &\geq 0, &
    d(x,y) &= d(y,x), \\
    d(x,y) &= 0 \Leftrightarrow x = y, &
    d(x,z) &\leq d(x,y) + d(y,z).
  \end{align*}
  %
\end{defn}

\begin{defn} \label{defn:complete-metric-space}
  A \define{Cauchy approximation}
  \index{Cauchy!approximation}%
  in $M$ is a sequence $x : \Qp \to M$ satisfying
  %
  \begin{equation*}
    \fall{\delta, \epsilon} d(x_\delta, x_\epsilon) < \delta + \epsilon.
  \end{equation*}
  %
  \index{limit!of a Cauchy approximation}%
  The \define{limit} of a Cauchy approximation $x : \Qp \to M$ is a point $\ell : M$
  satisfying
  %
  \begin{equation*}
    \fall{\epsilon, \theta : \Qp} d(x_\epsilon, \ell) < \epsilon + \theta.
  \end{equation*}
  %
  \indexdef{metric space!complete}%
  \indexdef{complete!metric space}%
  A \define{complete metric space} is one in which every Cauchy approximation has a limit.
\end{defn}

\begin{defn} \label{defn:total-bounded-metric-space}
  For a positive rational $\epsilon$, an \define{$\epsilon$-net}
  \indexdef{epsilon-net@$\epsilon$-net}%
  in a metric space $(M,
  d)$ is an element of
  %
  \begin{equation*}
    \sm{n : \N}{x_1, \ldots, x_n : M}
    \fall{y : M} \exis{k \leq n} d(x_k, y) < \epsilon.
  \end{equation*}
  %
  In words, this is a finite sequence of points $x_1, \ldots, x_n$ such that every point
  in $M$ merely is within $\epsilon$ of some~$x_k$.

  A metric space $(M, d)$ is \define{totally bounded}
  \indexdef{totally bounded metric space}%
  \indexdef{metric space!totally bounded}%
  when it has $\epsilon$-nets of all
  sizes:
  %
  \begin{equation*}
    \prd{\epsilon : \Qp}
    \sm{n : \N}{x_1, \ldots, x_n : M}
    \fall{y : M} \exis{k \leq n} d(x_k, y) < \epsilon.
  \end{equation*}
\end{defn}

\begin{rmk}
  In the definition of total boundedness we used sloppy notation $\sm{n : \N}{x_1, \ldots, x_n : M}$. Formally, we should have written $\sm{x : \lst{M}}$ instead,
  where $\lst{M}$ is the inductive type of finite lists\index{type!of lists} from \cref{sec:bool-nat}.
  However, that would make the rest of the statement a bit more cumbersome to express.
\end{rmk}

Note that in the definition of total boundedness we require pure existence of an
$\epsilon$-net, not mere existence. This way we obtain a function which assigns to each
$\epsilon : \Qp$ a specific $\epsilon$-net. Such a function might be called a ``modulus of
total boundedness''. In general, when porting classical metric notions to homotopy type
theory, we should use propositional truncation sparingly, typically so that we avoid
asking for a non-constant map from $\R$ to $\Q$ or $\N$. For instance, here is the
``correct'' definition of uniform continuity.

\begin{defn} \label{defn:uniformly-continuous}
  A map $f : M \to \R$ on a metric space is \define{uniformly continuous}
  \indexdef{function!uniformly continuous}%
  \indexdef{uniformly continuous function}%
  when
  %
  \begin{equation*}
    \prd{\epsilon : \Qp}
    \sm{\delta : \Qp}
    \fall{x, y : M}
    d(x,y) < \delta \Rightarrow |f(x) - f(y)| < \epsilon.
  \end{equation*}
  %
  In particular, a uniformly continuous map has a modulus of uniform continuity\indexdef{modulus!of uniform continuity},
  which is a function that assigns to each $\epsilon$ a corresponding $\delta$.
\end{defn}

Let us show that $[0,1]$ is compact in the first sense.

\begin{thm} \label{analysis-interval-ctb}
  \index{compactness!metric}%
  \index{interval!open and closed}%
  The closed interval $[0,1]$ is complete and totally bounded.
\end{thm}

\begin{proof}
  Given $\epsilon : \Qp$, there is $n : \N$ such that $2/k < \epsilon$, so we may take the
  $\epsilon$-net $x_i = i/k$ for $i = 0, \ldots, k-1$. This is an $\epsilon$-net because,
  for every $y : [0,1]$ there merely exists $i$ such that $0 \leq i < k$ and $(i -
  1)/k < y < (i+1)/k$, and so $|y - x_i| < 2/k < \epsilon$.

  For completeness of $[0,1]$, consider a Cauchy approximation $x : \Qp \to
  [0,1]$ and let $\ell$ be its limit in $\R$. Since $\max$ and $\min$ are Lipschitz maps,
  the retraction $r : \R \to [0,1]$ defined by $r(x) \defeq \max(0, \min(1, x))$ commutes
  with limits of Cauchy approximations, therefore
  %
  \begin{equation*}
    r(\ell) =
    r (\lim x) =
    \lim (r \circ x) =
    r (\lim x) =
    \ell,
  \end{equation*}
  %
  which means that $0 \leq \ell \leq 1$, as required.
\end{proof}

We thus have at least one good notion of compactness in homotopy type theory.
Unfortunately, it is limited to metric spaces because total boundedness is a metric
notion. We shall consider the other two notions shortly, but first we prove that a
uniformly continuous map on a totally bounded space has a \define{supremum},
\indexsee{least upper bound}{supremum}%
i.e.\ an upper bound which is less than or equal to all other upper bounds.

\begin{thm} \label{ctb-uniformly-continuous-sup}
  %
  \indexdef{supremum!of uniformly continuous function}%
  A uniformly continuous map $f : M \to \R$ on a totally bounded metric space
  $(M, d)$ has a supremum $m : \R$. For every $\epsilon : \Qp$ there exists $u : M$ such
  that $|m - f(u)| < \epsilon$.
\end{thm}

\begin{proof}
  Let $h : \Qp \to \Qp$ be the modulus of uniform continuity of~$f$.
  We define an approximation $x : \Qp \to \R$ as follows: for any $\epsilon : \Q$ total
  boundedness of $M$ gives a $h(\epsilon)$-net $y_0, \ldots, y_n$. Define
  %
  \begin{equation*}
    x_\epsilon \defeq \max (f(y_0), \ldots, f(y_n)).
  \end{equation*}
  %
  We claim that $x$ is a Cauchy approximation. Consider any $\epsilon, \eta : \Q$, so that
  %
  \begin{equation*}
    x_\epsilon \jdeq \max (f(y_0), \ldots, f(y_n))
    \quad\text{and}\quad
    x_\eta \jdeq \max (f(z_0), \ldots, f(z_m))
  \end{equation*}
  %
  for some $h(\epsilon)$-net $y_0, \ldots, y_n$ and $h(\eta)$-net $z_0, \ldots, z_m$.
  Every $z_i$ is merely $h(\epsilon)$-close to some $y_j$, therefore $|f(z_i) - f(y_j)| <
  \epsilon$, from which we may conclude that
  %
  \begin{equation*}
    f(z_i) < \epsilon + f(y_j) \leq \epsilon + x_\epsilon,
  \end{equation*}
  %
  therefore $x_\eta < \epsilon + x_\epsilon$. Symmetrically we obtain $x_\eta < \eta +
  x_\eta$, therefore $|x_\eta - x_\epsilon| < \eta + \epsilon$.

  We claim that $m \defeq \lim x$ is the supremum of~$f$. To prove that $f(x) \leq m$ for
  all $x : M$ it suffices to show $\lnot (m < f(x))$. So suppose to the contrary that $m <
  f(x)$. There is $\epsilon : \Qp$ such that $m + \epsilon < f(x)$. But now merely for
  some $y_i$ participating in the definition of $x_\epsilon$ we get $|f(x) - f(y_i) <
  \epsilon$, therefore $m < f(x) - \epsilon < f(y_i) \leq m$, a contradiction.

  We finish the proof by showing that $m$ satisfies the second part of the theorem, because
  it is then automatically a least upper bound. Given any $\epsilon : \Qp$, on one hand
  $|m - f(x_{\epsilon/2})| < 3 \epsilon/4$, and on the other $|f(x_{\epsilon/2}) - f(y_i)| <
  \epsilon/4$ merely for some $y_i$ participating in the definition of $x_{\epsilon/2}$,
  therefore by taking $u \defeq y_i$ we obtain $|m - f(u)| < \epsilon$ by triangle
  inequality.
\end{proof}

Now, if in \cref{ctb-uniformly-continuous-sup} we also knew that $M$ were complete, we
could hope to weaken the assumption of uniform continuity to continuity, and strengthen
the conclusion to existence of a point at which the supremum is attained. The usual proofs
of these improvements rely on the facts that in a complete totally bounded space
%
\begin{enumerate}
\item continuity implies uniform continuity, and
\item every sequence has a convergent subsequence.
\end{enumerate}
%
The first statement follows easily from Heine--Borel compactness, and the second is just
Bolzano--Weierstra\ss{} compactness.
\index{compactness!Bolzano--Weierstrass@Bolzano--Weierstra\ss{}}%
Unfortunately, these are both somewhat problematic. Let
us first show that Bolzano--Weierstra\ss{} compactness implies an instance of excluded middle
known as the \define{limited principle of omniscience}:
\indexsee{axiom!limited principle of omniscience}{limited principle of omniscience}%
\indexdef{limited principle of omniscience}%
for every $\alpha : \N \to \bool$,
%
\begin{equation} \label{eq:lpo}
  \Parens{\sm{n : \N} \alpha(n) = \btrue} +
  \Parens{\prd{n : \N} \alpha(n) = \bfalse}.
\end{equation}
%
Computationally speaking, we would not expect this principle to hold, because it asks us to decide
whether infinitely many values of a function are~$\bfalse$.

\begin{thm} \label{analysis-bw-lpo}
  %
  Bolzano--Weierstra\ss{} compactness of $[0,1]$ implies the limited principle of omniscience.
  \index{compactness!Bolzano--Weierstrass@Bolzano--Weierstra\ss{}}%
\end{thm}

\begin{proof}
  Given any $\alpha : \N \to \bool$, define the sequence\index{sequence} $x : \N \to [0,1]$ by
  %
  \begin{equation*}
    x_n \defeq
    \begin{cases}
      0 & \text{if $\alpha(k) = \bfalse$ for all $k < n$,}\\
      1 & \text{if $\alpha(k) = \btrue$ for some $k < n$}.
    \end{cases}
  \end{equation*}
  %
  If the Bolzano--Weierstra\ss{} property holds, there exists a strictly increasing $f : \N \to
  \N$ such that $x \circ f$ is a Cauchy sequence\index{Cauchy!sequence}. For a sufficiently large $n :
  \N$ the $n$-th term $x_{f(n)}$ is within $1/6$ of its limit. Either $x_{f(n)} < 2/3$ or
  $x_{f(n)} > 1/3$. If $x_{f(n)} < 2/3$ then~$x_n$ converges to $0$ and so $\prd{n : \N}
  \alpha(n) = \bfalse$. If $x_{f(n)} > 1/3$ then $x_{f(n)} = 1$, therefore $\sm{n : \N}
  \alpha(n) = \btrue$.
\end{proof}

While we might not mourn Bolzano--Weierstra\ss{} compactness too much, it seems harder to live
without Heine--Borel compactness, as attested by the fact that both classical mathematics
and Brouwer's Intuitionism accepted it. As we do not want to wade too deeply into general
topology, we shall work with basic open sets. In the case of $\R$ these are the open
intervals with rational endpoints. A family of such intervals, indexed by a type~$I$,
would be a map
%
\begin{equation*}
  \mathcal{F} : I \to \setof{(q, r) : \Q \times \Q | q < r},
\end{equation*}
%
with the idea that a pair of rationals $(q, r)$ with $q < r$ determines the type $\setof{ x : \R | q < x < r}$. It is slightly more convenient to allow degenerate intervals as well, so we take a
\define{family of basic intervals}
\indexdef{family!of basic intervals}%
\indexdef{interval!family of basic}%
to be a map
%
\begin{equation*}
  \mathcal{F} : I \to \Q \times \Q.
\end{equation*}
%
To be quite precise, a family is a dependent pair $(I, \mathcal{F})$, not just
$\mathcal{F}$. A \define{finite family of basic intervals} is one indexed by $\setof{ m :
  \N | m < n}$ for some $n : \N$. We usually present it by a finite list $[(q_0, r_0), \ldots,
(q_{n-1}, r_{n-1})]$. Finally, a \define{finite subfamily}\indexdef{subfamily, finite, of intervals} of $(I, \mathcal{F})$ is given
by a list of indices $[i_1, \ldots, i_n]$ which then determine the finite family
$[\mathcal{F}(i_1), \ldots, \mathcal{F}(i_n)]$.

As long as we are aware of the distinction between a pair $(q, r)$ and the corresponding
interval $\setof{ x : \R | q < x < r}$, we may safely use the same notation $(q, r)$ for
both. Intersections\indexdef{intersection!of intervals} and inclusions\indexdef{inclusion!of intervals}\indexdef{containment!of intervals} of intervals are expressible in terms of their
endpoints:
%
\symlabel{interval-intersection}
\symlabel{interval-subset}
\begin{align*}
  (q, r) \cap (s, t) &\ \defeq\  (\max(q, s), \min(r, t)),\\
  (q, r) \subseteq (s, t) &\ \defeq\ (q < r \Rightarrow s \leq q < r \leq t).
\end{align*}
%
We say that $\intfam{i}{I}{(q_i, r_i)}$ \define{(pointwise) covers $[a,b]$}
\indexdef{interval!pointwise cover}%
\indexdef{cover!pointwise}%
\indexdef{pointwise!cover}%
when
%
\begin{equation} \label{eq:cover-pointwise-truncated}
  \fall{x : [a,b]} \exis{i : I} q_i < x < r_i.
\end{equation}
%
The \define{Heine--Borel compactness for $[0,1]$}
\indexdef{compactness!Heine--Borel}%
states that every covering family of $[0,1]$
merely has a finite subfamily which still covers $[0,1]$.

\index{depression}
\begin{thm} \label{classical-Heine-Borel}
  \index{excluded middle}%
  If excluded middle holds then $[0,1]$ is Heine--Borel compact.
\end{thm}

\begin{proof}
  Assume for the purpose of reaching a contradiction that a family $\intfam{i}{I}{(a_i,
    b_i)}$ covers $[0,1]$ but no finite subfamily does. We construct a sequence of closed
  intervals $[q_n, r_n]$ which are nested, their sizes shrink to~$0$, and none of them is covered
  by a finite subfamily of $\intfam{i}{I}{(a_i, b_i)}$.

  We set $[q_0, r_0] \defeq [0,1]$. Assuming $[q_n, r_n]$ has been constructed, let $s
  \defeq (2 q_n + r_n)/3$ and $t \defeq (q_n + 2 r_n)/3$. Both $[q_n, t]$ and $[s, r_n]$
  are covered by $\intfam{i}{I}{(a_i, b_i)}$, but they cannot both have a finite subcover,
  or else so would $[q_n, r_n]$. Either $[q_n, t]$ has a finite subcover or it does not.
  If it does we set $[q_{n+1}, r_{n+1}] \defeq [s, r_n]$, otherwise we set $[q_{n+1},
  r_{n+1}] \defeq [q_n, t]$.

  The sequences $q_0, q_1, \ldots$ and $r_0, r_1, \ldots$ are both Cauchy and they
  converge to a point $x : [0,1]$ which is contained in every $[q_n, r_n]$.
  There merely exists $i : I$ such that $a_i < x < b_i$. Because the sizes of the
  intervals $[q_n, r_n]$ shrink to zero, there is $n : \N$ such that $a_i < q_n \leq x
  \leq r_n < b_i$, but this means that $[q_n, r_n]$ is covered by a single interval $(a_i,
  b_i)$, while at the same time it has no finite subcover. A contradiction.
\end{proof}

Without excluded middle, or a pinch of Brouwerian Intuitionism, we seem to be stuck.
Nevertheless, Heine--Borel compactness of $[0,1]$ \emph{can} be recovered in a constructive
setting, in a fashion that is still compatible with classical mathematics! For this to be
done, we need to revisit the notion of cover. The trouble with
\eqref{eq:cover-pointwise-truncated} is that the truncated existential allows a space to
be covered in any haphazard way, and so computationally speaking, we stand no chance of
merely extracting a finite subcover. By removing the truncation we get
%
\begin{equation} \label{eq:cover-pointwise}
  \prd{x : [0,1]} \sm{i : I} q_i < x < r_i,
\end{equation}
%
which might help, were it not too demanding of covers. With this definition we
could not even show that $(0,3)$ and $(2,5)$ cover $[1,4]$ because that would amount
to exhibiting a non-constant map $[1,4] \to \bool$, see
\cref{ex:reals-non-constant-into-Z}.  Here we can take a lesson from ``pointfree topology''
\index{pointfree topology}%
\index{topology!pointfree}%
(i.e.\ locale theory):
\index{locale}%
the notion of cover ought to be expressed in terms of open sets, without
reference to points. Such a ``holistic'' view of space will then allow us to analyze the
notion of cover, and we shall be able to recover Heine--Borel compactness.  Locale
theory uses power sets,
\index{power set}%
which we could obtain by assuming propositional resizing;
\index{propositional!resizing}%
but instead we can steal ideas from the predicative cousin of locale theory,
\index{mathematics!predicative}%
which is called ``formal topology''.
\index{formal!topology}%

\index{acceptance|(}

Suppose that we have a family $\pairr{I, \mathcal{F}}$ and an interval $(a, b)$. How might
we express the fact that $(a,b)$ is covered by the family, without referring to points?
Here is one: if $(a, b)$ equals some $\mathcal{F}(i)$ then it is covered by the family.
And another one: if $(a,b)$ is covered by some other family $(J, \mathcal{G})$, and in
turn each $\mathcal{G}(j)$ is covered by $\pairr{I, \mathcal{F}}$, then $(a,b)$ is covered
$\pairr{I, \mathcal{F}}$. Notice that we are listing \emph{rules} which can be used to
\emph{deduce} that $\pairr{I, \mathcal{F}}$ covers $(a,b)$. We should find sufficiently
good rules and turn them into an inductive definition.

\begin{defn} \label{defn:inductive-cover}
  %
  The \define{inductive cover $\cover$}
  \indexdef{inductive!cover}%
  \indexdef{cover!inductive}%
  is a mere relation
  %
  \begin{equation*}
    {\cover} : (\Q \times \Q) \to \Parens{\sm{I : \type} (I \to \Q \times \Q)} \to \prop
  \end{equation*}
  %
  defined inductively by the following rules, where $q, r, s, t$ are rational numbers and
  $\pairr{I, \mathcal{F}}$, $\pairr{J, \mathcal{G}}$ are families of basic intervals:
  %
  \begin{enumerate}

  \item \emph{reflexivity:}
    \index{reflexivity!of inductive cover}%
    $\mathcal{F}(i) \cover \pairr{I, \mathcal{F}}$ for all $i : I$,

  \item \emph{transitivity:}
    \index{transitivity!of inductive cover}%
    if $(q, r) \cover \pairr{J, \mathcal{G}}$ and $\fall{j : J} \mathcal{G}(j) \cover \pairr{I,\mathcal{F}}$
    then $(q, r) \cover \pairr{I, \mathcal{F}}$,

  \item \emph{monotonicity:}
    \index{monotonicity!of inductive cover}%
    if $(q, r) \subseteq (s, t)$ and $(s,t) \cover \pairr{I, \mathcal{F}}$ then $(q, r) \cover
    \pairr{I, \mathcal{F}}$,

  \item \emph{localization:}
    \index{localization of inductive cover}%
    if $(q, r) \cover (I, \mathcal{F})$ then $(q, r) \cap (s, t) \cover
    \intfam{i}{I}{(\mathcal{F}(i) \cap (s, t))}$.

  \item \label{defn:inductive-cover-interval-1}
    if $q < s < t < r$ then $(q, r) \cover [(q, t), (r, s)]$,

  \item \label{defn:inductive-cover-interval-2}
    $(q, r) \cover \intfam{u}{\setof{ (s,t) : \Q \times \Q | q < s < t < r}}{u}$.
  \end{enumerate}
\end{defn}

The definition should be read as a higher-inductive type in which the listed rules are
point constructors, and the type is $(-1)$-truncated. The first four clauses are of a
general nature and should be intuitively clear. The last two clauses are specific to the
real line: one says that an interval may be covered by two intervals if they overlap,
while the other one says that an interval may be covered from within. Incidentally, if $r
\leq q$ then $(q, r)$ is covered by the empty family by the last clause.

Inductive covers enjoy the Heine--Borel property, the proof of which requires a lemma.

\begin{lem} \label{reals-formal-topology-locally-compact}
  Suppose $q < s < t < r$ and $(q, r) \cover \pairr{I, \mathcal{F}}$. Then there merely
  exists a finite subfamily of $\pairr{I, \mathcal{F}}$ which inductively covers $(s, t)$.
\end{lem}

\begin{proof}
  We prove the statement by induction on $(q, r) \cover \pairr{I, \mathcal{F}}$. There are
  six cases:
  %
  \begin{enumerate}

  \item Reflexivity: if $(q, r) = \mathcal{F}(i)$ then by monotonicity $(s, t)$ is covered
    by the finite subfamily $[\mathcal{F}(i)]$.

  \item Transitivity:
    suppose $(q, r) \cover \pairr{J, \mathcal{G}}$ and $\fall{j : J} \mathcal{G}(j) \cover
    \pairr{I, \mathcal{F}}$. By the inductive hypothesis there merely exists
    $[\mathcal{G}(j_1), \ldots, \mathcal{G}(j_n)]$ which covers $(s, t)$.
    Again by the inductive hypothesis, each of $\mathcal{G}(j_k)$ is covered by a finite
    subfamily of $\pairr{I, \mathcal{F}}$, and we can collect these into a finite
    subfamily which covers $(s, t)$.

  \item Monotonicity:
    if $(q, r) \subseteq (u, v)$ and $(u, v) \cover \pairr{I, \mathcal{F}}$ then we may
    apply the inductive hypothesis to $(u, v) \cover \pairr{I, \mathcal{F}}$ because $u <
    s < t < v$.

  \item Localization:
    suppose $(q', r') \cover \pairr{I, \mathcal{F}}$ and $(q, r) = (q', r') \cap (a, b)$.
    Because $q' < s < t < r'$, by the inductive hypothesis there is a finite subcover
    $[\mathcal{F}(i_1), \ldots, \mathcal{F}(i_n)]$ of $(s, t)$. We also know that $a < s <
    t < b$, therefore $(s, t) = (s, t) \cap (a, b)$ is covered by
    $[\mathcal{F}(i_1) \cap (a,b), \ldots, \mathcal{F}(i_n) \cap (a,b)]$, which is a
    finite subfamily of $\intfam{i}{I}{(\mathcal{F}(i) \cap (a, b))}$.

  \item If $(q, r) \cover [(q, v), (u, r)]$ for some $q < u < v < r$ then by monotonicity
    $(s, t) \cover [(q, v), (u, r)]$.

  \item Finally, $(s, t) \cover \intfam{z}{\setof{ (u,v):\Q \times \Q | q < u < v < r}}{z}$ by
    reflexivity. \qedhere
  \end{enumerate}
\end{proof}

Say that \define{$\pairr{I, \mathcal{F}}$ inductively covers
  $[a, b]$} when there merely exists $\epsilon : \Qp$ such that $(a - \epsilon, b +
\epsilon) \cover \pairr{I, \mathcal{F}}$.

\begin{cor} \label{interval-Heine-Borel}
  \index{compactness!Heine-Borel}%
  \index{interval!open and closed}%
  A closed interval is Heine--Borel compact for inductive covers.
\end{cor}

\begin{proof}
  Suppose $[a, b]$ is inductively covered by $\pairr{I, \mathcal{F}}$, so there merely is
  $\epsilon : \Qp$ such that $(a - \epsilon, b + \epsilon) \cover \pairr{I, \mathcal{F}}$.
  By \cref{reals-formal-topology-locally-compact} there is a finite subcover of
  $(a - \epsilon/2, b + \epsilon/2)$, which is therefore a finite subcover of $[a, b]$.
\end{proof}

Experience from formal topology\index{topology!formal} shows that the rules for inductive covers are sufficient
for a constructive development of pointfree topology. But we can also provide our own
evidence that they are a reasonable notion.

\begin{thm} \label{inductive-cover-classical}
  \mbox{}
  %
  \begin{enumerate}
  \item An inductive cover is also a pointwise cover.
  \item Assuming excluded middle, a pointwise cover is also an inductive cover.
  \end{enumerate}
\end{thm}

\begin{proof}
  \mbox{}
  %
  \begin{enumerate}

  \item
    Consider a family of basic intervals $\pairr{I, \mathcal{F}}$, where we write $(q_i,
    r_i) \defeq \mathcal{F}(i)$, an interval $(a,b)$ inductively covered by $\pairr{I,
      \mathcal{F}}$, and $x$ such that $a < x < b$.
    %
    We prove by induction on $(a,b) \cover \pairr{I, \mathcal{F}}$ that there merely
    exists $i : I$ such that $q_i < x < r_i$. Most cases are pretty obvious, so we show
    just two. If $(a,b) \cover \pairr{I, \mathcal{F}}$ by reflexivity, then there merely
    is some $i : I$ such that $(a,b) = (q_i, r_i)$ and so $q_i < x < r_i$. If $(a,b)
    \cover \pairr{I, \mathcal{F}}$ by transitivity via $\intfam{j}{J}{(s_j, t_j)}$ then by
    the inductive hypothesis there merely is $j : J$ such that $s_j < x < t_j$, and then since
    $(s_j, t_j) \cover \pairr{I, \mathcal{F}}$ again by the inductive hypothesis there merely
    exists $i : I$ such that $q_i < x < r_i$. Other cases are just as exciting.

  \item Suppose $\intfam{i}{I}{(q_i, r_i)}$ pointwise covers $(a, b)$. By
    \cref{defn:inductive-cover-interval-2} of \cref{defn:inductive-cover} it
    suffices to show that $\intfam{i}{I}{(q_i, r_i)}$ inductively covers $(c, d)$ whenever
    $a < c < d < b$, so consider such $c$ and $d$. By \cref{classical-Heine-Borel}
    there is a finite subfamily $[i_1, \ldots, i_n]$ which already pointwise covers $[c,
    d]$, and hence $(c,d)$. Let $\epsilon : \Qp$ be a Lebesgue number
    \index{Lebesgue number}
    for $(q_{i_1}, r_{i_1}), \ldots, (q_{i_n}, r_{i_n})$ as in
    \cref{ex:finite-cover-lebesgue-number}. There is a positive $k : \N$ such that $2 (d - c)/k
    < \min(1, \epsilon)$. For $0 \leq i \leq k$ let
    %
    \begin{equation*}
      c_k \defeq ((k - i) c + i d) / k.
    \end{equation*}
    %
    The intervals $(c_0, c_2)$, $(c_1, c_3)$, \dots, $(c_{k-2}, c_k)$ inductively cover
    $(c,d)$ by repeated use of transitivity and~\cref{defn:inductive-cover-interval-1}
    in \cref{defn:inductive-cover}. Because their widths are below $\epsilon$ each of
    them is contained in some $(q_i, r_i)$, and we may use transitivity and monotonicity to
    conclude that $\intfam{i}{I}{(q_i, r_i)}$ inductively cover $(c, d)$. \qedhere
  \end{enumerate}
\end{proof}

The upshot of the previous theorem is that, as far as classical mathematics is concerned,
there is no difference between a pointwise and an inductive cover. In particular, since it
is consistent to assume excluded middle in homotopy type theory, we cannot exhibit an
inductive cover which fails to be a pointwise cover. Or to put it in a different way, the
difference between pointwise and inductive covers is not what they cover but in the
\emph{proofs} that they cover.

We could write another book by going on like this, but let us stop here and hope that we
have provided ample justification for the claim that analysis can be developed in homotopy
type theory. The curious reader should consult \cref{ex:mean-value-theorem} for
constructive versions of the intermediate value theorem.

\index{acceptance|)}

\index{mathematics!classical|)}%
\index{mathematics!constructive|)}%

\section{The surreal numbers}
\label{sec:surreals}

\index{surreal numbers|(}%

In this section we consider another example of a higher inductive-in\-duc\-tive type, which draws together many of our threads: Conway's field \NO of \emph{surreal numbers}~\cite{conway:onag}.
The surreal numbers are the natural common generalization of the (Dedekind) real numbers (\cref{sec:dedekind-reals}) and the ordinal numbers (\cref{sec:ordinals}).
Conway, working in classical\index{mathematics!classical} mathematics with excluded middle and Choice, defines a surreal number to be a pair of \emph{sets} of surreal numbers, written $\surr L R$, such that every element of $L$ is strictly less than every element of $R$.
This obviously looks like an inductive definition, but there are three issues with regarding it as such.

Firstly, the definition requires the relation of (strict) inequality between surreals, so that relation must be defined simultaneously with the type \NO of surreals.
(Conway avoids this issue by first defining \emph{games}\index{game!Conway}, which are like surreals but omit the compatibility condition on $L$ and $R$.)
As with the relation $\closesym$ for the Cauchy reals, this simultaneous definition could \emph{a priori} be either inductive-inductive or inductive-recursive.
We will choose to make it inductive-inductive, for the same reasons we made that choice for $\closesym$.

Moreover, we will define strict inequality $<$ and non-strict inequality $\le$ for surreals separately (and mutually inductively).
Conway defines $<$ in terms of $\le$, in a way which is sensible classically but not constructively.
\index{mathematics!constructive}%
Furthermore, a negative definition of $<$ would make it unacceptable as a hypothesis of the constructor of a higher inductive type (see \cref{sec:strictly-positive}).

Secondly, Conway says that $L$ and $R$ in $\surr L R$ should be ``sets of surreal numbers'', but the naive meaning of this as a predicate $\NO\to\prop$ is not positive, hence cannot be used as input to an inductive constructor.
However, this would not be a good type-theoretic translation of what Conway means anyway, because in set theory the surreal numbers form a proper class, whereas the sets $L$ and $R$ are true (small) sets, not arbitrary subclasses of \NO.
In type theory, this means that \NO will be defined relative to a universe \UU, but will itself belong to the next higher universe $\UU'$, like the sets \ord and \card of ordinals and cardinals, the cumulative hierarchy $V$, or even the Dedekind reals in the absence of propositional resizing.
\index{propositional!resizing}%
We will then require the ``sets'' $L$ and $R$ of surreals to be \UU-small, and so it is natural to represent them by \emph{families} of surreals indexed by some \UU-small type.
(This is all exactly the same as what we did with the cumulative hierarchy in \cref{sec:cumulative-hierarchy}.)
That is, the constructor of surreals will have type
\[ \prd{\LL,\RR:\UU} (\LL\to\NO) \to (\RR\to \NO) \to (\text{some condition}) \to \NO \]
which is indeed strictly positive.\index{strict!positivity}

Finally, after giving the mutual definitions of \NO and its ordering, Conway declares two surreal numbers $x$ and $y$ to be \emph{equal} if $x\le y$ and $y\le x$.
This is naturally read as passing to a quotient of the set of ``pre-surreals'' by an equivalence relation.
%(In set-theoretic foundations, one has to us an additional trick to deal with large equivalence classes.)
However, in the absence of the axiom of choice, such a quotient presents the same problem as the quotient in the usual construction of Cauchy reals: it will no longer be the case that a pair of families \emph{of surreals} yield a new surreal $\surr L R$, since we cannot necessarily ``lift'' $L$ and $R$ to families of pre-surreals.
Of course, we can solve this problem in the same way we did for Cauchy reals, by using a \emph{higher} inductive-inductive definition.

\begin{defn}\label{defn:surreals}
  The type \NO of \define{surreal numbers},
  \indexdef{surreal numbers}%
  \indexsee{number!surreal}{surreal numbers}%
  along with the relations $\mathord<:\NO\to\NO\to\type$ and $\mathord\le:\NO\to\NO\to\type$, are defined higher inductive-inductively as follows.
  The type \NO has the following constructors.
  \begin{itemize}
  \item For any $\LL,\RR:\UU$ and functions $\LL\to \NO$ and $\RR\to \NO$, whose values we write as $x^L$ and $x^R$ for $L:\LL$ and $R:\RR$ respectively, if $\fall{L:\LL}{R:\RR} x^L<x^R$, then there is a surreal number $x$.
  \item For any $x,y:\NO$ such that $x\le y$ and $y\le x$, we have $\noeq(x,y):x=y$.
  \end{itemize}
  We will refer to the inputs of the first constructor as a \define{cut}.
  \indexdef{cut!of surreal numbers}%
  If $x$ is the surreal number constructed from a cut, then the notation $x^L$ will implicitly assume $L:\LL$, and similarly $x^R$ will assume $R:\RR$.
  In this way we can usually avoid naming the indexing types $\LL$ and $\RR$, which is convenient when there are many different cuts under discussion.
  Following Conway, we call $x^L$ a \emph{left option}\indexdef{option of a surreal number} of $x$ and $x^R$ a \emph{right option}.

  The path constructor implies that different cuts can define the same surreal number.
  Thus, it does not make sense to speak of the left or right options of an arbitrary surreal number $x$, unless we also know that $x$ is defined by a particular cut.
  Thus in what follows we will say, for instance, ``given a cut defining a surreal number $x$'' in contrast to ``given a surreal number $x$''.

  The relation $\le$ has the following constructors.
  \index{non-strict order}%
  \index{order!non-strict}%
  \begin{itemize}
  \item Given cuts defining two surreal numbers $x$ and $y$, if $x^L<y$ for all $L$, and $x<y^R$ for all $R$, then $x\le y$.
  \item Propositional truncation:
    for any $x,y:\NO$, if $p,q:x\le y$, then $p=q$.
  \end{itemize}
  And the relation $<$ has the following constructors.
  \index{strict!order}%
  \index{order!strict}%
  \begin{itemize}
    % Don't technically need x in the first one and y in the second one to be defined by cuts?
  \item Given cuts defining two surreal numbers $x$ and $y$, if there is an $L$ such that $x\le y^L$, then $x<y$.
  \item Given cuts defining two surreal numbers $x$ and $y$, if there is an $R$ such that $x^R\le y$, then $x<y$.
  \item Propositional truncation: for any $x,y:\NO$, if $p,q:x<y$, then $p=q$.
  \end{itemize}
\end{defn}

\noindent
We compare this with Conway's definitions:
\begin{itemize}\footnotesize
\item[-] If $L,R$ are any two sets of numbers, and no member of $L$ is $\ge$ any member of $R$, then there is a number $\surr L R$.
  All numbers are constructed in this way.
\item[-] $x\ge y$ iff (no $x^R\le y$ and $x\le$ no $y^L$).
\item[-] $x=y$ iff ($x \ge y$ and $y\ge x$).
\item[-] $x>y$ iff ($x\ge y$ and $y\not\ge x$).
\end{itemize}
The inclusion of $x\ge y$ in the definition of $x>y$ is unnecessary if all objects are [surreal] numbers rather than ``games''\index{game!Conway}.
Thus, Conway's $<$ is just the negation of his $\ge$, so that his condition for $\surr L R$ to be a surreal is the same as ours.
Negating Conway's $\le$ and canceling double negations, we arrive at our definition of $<$, and we can then reformulate his $\le$ in terms of $<$ without negations.

We can immediately populate $\NO$ with many surreal numbers.
Like Conway, we write
\symlabel{surreal-cut}
\[\surr{x,y,z,\dots}{u,v,w,\dots}\]
for the surreal number defined by a cut where $\LL\to\NO$ and $\RR\to\NO$ are families described by $x,y,z,\dots$ and $u,v,w,\dots$.
Of course, if $\LL$ or $\RR$ are $\emptyt$, we leave the corresponding part of the notation empty.
There is an unfortunate clash with the standard notation $\setof{x:A | P(x)}$ for subsets, but we will not use the latter in this section.
\begin{itemize}
\item We define $\iota_{\nat}:\nat\to\NO$ recursively by
  \begin{align*}
    \iota_{\nat}(0) &\defeq \surr{}{},\\
    \iota_\nat(\suc(n)) &\defeq \surr{\iota_\nat(n)}{}.
  \end{align*}
  That is, $\iota_\nat(0)$ is defined by the cut consisting of $\emptyt\to\NO$ and $\emptyt\to\NO$.
  Similarly, $\iota_\nat(\suc(n))$ is defined by $\unit\to\NO$ (picking out $\iota_\nat(n)$) and $\emptyt\to\NO$.
\item Similarly, we define $\iota_{\Z}:\Z\to\NO$ using the sign-case recursion principle (\cref{thm:sign-induction}):
  \begin{align*}
    \iota_{\Z}(0) &\defeq \surr{}{},\\
    \iota_\Z(n+1) &\defeq \surr{\iota_\Z(n)}{} & &\text{$n\ge 0$,}\\
    \iota_\Z(n-1) &\defeq \surr{}{\iota_\Z(n)} & &\text{$n\le 0$.}
  \end{align*}
\item By a \define{dyadic rational}
  \indexdef{rational numbers!dyadic}%
  \indexsee{dyadic rational}{rational numbers, dyadic}%
  we mean a pair $(a,n)$ where $a:\Z$ and $n:\nat$, and such that if $n>0$ then $a$ is odd.
  We will write it as $a/2^n$, and identify it with the corresponding rational number.
  If $\Q_D$ denotes the set of dyadic rationals, we define $\iota_{\Q_D}:\Q_D\to\NO$ by induction on $n$:
  \begin{align*}
    \iota_{\Q_D}(a/2^0) &\defeq \iota_\Z(a),\\
    \iota_{\Q_D}(a/2^n) &\defeq \surr{\iota_{\Q_D}(a/2^n - 1/2^n)}{\iota_{\Q_D}(a/2^n + 1/2^n)},
    \quad \text{for $n>0$.}
  \end{align*}
  Here we use the fact that if $n>0$ and $a$ is odd, then $a/2^n \pm 1/2^n$ is a dyadic rational with a smaller denominator than $a/2^n$.
\item We define $\iota_{\RD}:\RD\to\NO$,\label{reals-into-surreals} where $\RD$ is (any version of) the Dedekind reals from \cref{sec:dedekind-reals}, by
  \begin{align*}
    \iota_{\RD}(x) &\defeq
    \surr{q\in\Q_D \text{ such that } q<x}{q\in\Q_D \text{ such that } x<q}.
  \end{align*}
  Unlike in the previous cases, it is not obvious that this extends $\iota_{\Q_D}$ when we regard dyadic rationals as Dedekind reals.
  This follows from the simplicity theorem (\cref{thm:NO-simplicity}).
\item Recall the type \ord of \emph{ordinals}\index{ordinal} from \cref{sec:ordinals}, which is well-ordered by the relation $<$, where $A<B$ means that $A = \ordsl B b$ for some $b:B$.
  We define $\iota_{\ord}:\ord\to\NO$\label{ord-into-surreals} by well-founded recursion (\cref{thm:wfrec}) on $\ord$:
  \begin{equation*}
    \iota_{\ord}(A) \defeq
    \surr{\iota_\ord(\ordsl A a) \text{ for all } a:A}{}.
  \end{equation*}
  It will also follow from the simplicity theorem that $\iota_\ord$ restricted to finite ordinals agrees with $\iota_\nat$.
  (We caution the reader, however, that unlike the above examples, $\iota_\ord$ is not constructively injective unless we restrict it to a smaller class of ordinals; see \cref{ex:ord-into-surreals,ex:hiit-plump}.)
\item A few more interesting examples taken from Conway:
  \begin{align*}
    \omega &\defeq \surr{0,1,2,3,\dots}{} \qquad\text{(also an ordinal)}\\
    -\omega &\defeq \surr{}{\dots,-3,-2,-1,0}\\
    1/\omega &\defeq \textstyle\surr{0}{1,\frac12,\frac14,\frac18,\dots}\\
    \omega-1 &\defeq \surr{0,1,2,3,\dots}{\omega}\\
    \omega/2 &\defeq \surr{0,1,2,3,\dots}{\dots,\omega-2,\omega-1,\omega}.
  \end{align*}
\end{itemize}

In identifying surreal numbers presented by different cuts, the following simple observation is useful.

\begin{thm}[Conway's simplicity theorem]\label{thm:NO-simplicity}
  \index{simplicity theorem}%
  \index{theorem!Conway's simplicity}%
  Suppose $x$ and $z$ are surreal numbers defined by cuts, and that the following hold.
  \begin{itemize}
  \item $x^L < z < x^R$ for all $L$ and $R$.
  \item For every left option $z^L$ of $z$, there exists a left option $x^{L'}$ with $z^L\le x^{L'}$.
  \item For every right option $z^R$ of $z$, there exists a right option $x^{R'}$ with $x^{R'}\le z^R$.
  \end{itemize}
  Then $x=z$.
\end{thm}
\begin{proof}
  Applying the path constructor of $\NO$, we must show $x\le z$ and $z\le x$.
  The first entails showing $x^L<z$ for all $L$, which we assumed, and $x<z^R$ for all $R$.
  But by assumption, for any $z^R$ there is an $x^{R'}$ with $x^{R'}\le z^R$ hence $x<z^R$ as desired.
  Thus $x\le z$; the proof of $z\le x$ is symmetric.
\end{proof}

\index{induction principle!for surreal numbers}
In order to say much more about surreal numbers, however, we need their induction principle.
The mutual induction principle for $(\NO,\le,<)$ applies to three families of types:
\begin{align*}
  A &: \NO\to\type\\
  B &: \prd{x,y:\NO}{a:A(x)}{b:A(y)} (x\le y) \to \type\\
  C &: \prd{x,y:\NO}{a:A(x)}{b:A(y)} (x<y) \to \type.
\end{align*}
As with the induction principle for Cauchy reals, it is helpful to think of $B$ and $C$ as families of relations between the types $A(x)$ and $A(y)$.
\symlabel{NO-recursion}
Thus we write $B(x,y,a,b,\xi)$ as $(x,a) \ble^\xi (y,b)$ and $C(x,y,a,b,\xi)$ as $(x,a) \blt^\xi (y,b)$.
Similarly, we usually omit the $\xi$ since it inhabits a mere proposition and so is uninteresting, and we may often omit $x$ and $y$ as well, writing simply $a\ble b$ or $a\blt b$.
With these notations, the hypotheses of the induction principle are the following.
\begin{itemize}
\item For any cut defining a surreal number $x$, together with
  \begin{enumerate}
  \item for each $L$, an element $a^L:A(x^L)$, and
  \item for each $R$, an element $a^R:A(x^R)$, such that
  \item for all $L$ and $R$ we have $(x^L,a^L) \blt (x^R,a^R)$
  \end{enumerate}
  there is a specified element $f_a:A(x)$.
  We call such data a \define{dependent cut}
  \indexdef{cut!of surreal numbers!dependent}%
  \indexdef{dependent!cut}%
  over the cut defining~$x$.
\item For any $x,y:\NO$ with $a:A(x)$ and $b:A(y)$, if $x\le y$ and $y\le x$ and also $(x,a) \ble (y,b)$
  and $(y,b) \ble (x,a)$,
  then $\dpath{A}{\noeq}{a}{b}$.
\item Given cuts defining two surreal numbers $x$ and $y$, and dependent cuts $a$ over $x$ and $b$ over $y$, such that for all $L$ we have $x^L<y$ and $(x^L,a^L)\blt (y,f_b)$,
  and for all $R$ we have $x<y^R$ and $(x,f_a) \blt (y^R,b^R)$,
  then $(x,f_a) \ble (y,f_b)$.
\item $\ble$ takes values in mere propositions.
\item Given cuts defining two surreal numbers $x$ and $y$, dependent cuts $a$ over $x$ and $b$ over $y$, and an $L_0$ such that $x\le y^{L_0}$ and $(x,f_a) \ble (y^{L_0},b^{L_0})$,
  we have $(x,f_a) \blt (y,f_b)$.
\item Given cuts defining two surreal numbers $x$ and $y$, dependent cuts $a$ over $x$ and $b$ over $y$, and an ${R_0}$ such that $x^{R_0}\le y$ together with $(x^{R_0},a^{R_0}),\ble (y,f_b)$,
  we have $(x,f_a) \blt (y,f_b)$.
\item $\blt$ takes values in mere propositions.
\end{itemize}
Under these hypotheses we deduce a function $f:\prd{x:\NO} A(x)$ such that
\begin{align}
  f(x) &\;\jdeq\; f_{f[x]} \label{eq:noind1}\\
  (x\le y) &\;\Rightarrow\; (x,f(x)) \ble (y,f(y)) \notag\\
  (x< y) &\;\Rightarrow\; (x,f(x)) \blt (y,f(y)). \notag
\end{align}
In the computation rule~\eqref{eq:noind1} for the point constructor, $x$ is a surreal number defined by a cut, and $f[x]$ denotes the dependent cut over $x$ defined by applying $f$ (and using the fact that $f$ takes $<$ to $\blt$).
As usual, we will generally use pattern-matching notation, where the definition of $f$ on a cut $\surr{x^L}{x^R}$ may use the symbols $f(x^L)$ and $f(x^R)$ and the assumption that they form a dependent cut.

As with the Cauchy reals, we have special cases resulting from trivializing some of $A$, $\ble$, and~$\blt$.
Taking $\ble$ and $\blt$ to be constant at \unit, we have \define{\NO-induction}, which for simplicity we state only for mere properties:
\begin{itemize}
\item Given $P:\NO\to\prop$, if $P(x)$ holds whenever $x$ is a surreal number defined by a cut such that $P(x^L)$ and $P(x^R)$ hold for all
$L$ and $R$, then $P(x)$ holds for all $x:\NO$.
\end{itemize}
This should be compared with Conway's remark:
\begin{quote}\footnotesize
  In general when we wish to establish a proposition $P(x)$ for all numbers $x$, we will prove it inductively by deducing $P(x)$ from the truth of all the propositions $P(x^L)$ and $P(x^R)$.
  We regard the phrase ``all numbers are constructed in this way'' as justifying the legitimacy of this procedure.
\end{quote}
With $\NO$-induction, we can prove

\begin{thm}[Conway's Theorem 0]\label{thm:NO-refl-opt}\
  \index{theorem!Conway's 0}%
  \begin{enumerate}
  \item For any $x:\NO$, we have $x\le x$.\label{item:NO-le-refl}
  \item For any $x:\NO$ defined by a cut, we have $x^L <x$ and $x<x^R$ for all $L$ and $R$.\label{item:NO-lt-opt}
  \end{enumerate}
\end{thm}
\begin{proof}
  Note first that if $x\le x$, then whenever $x$ occurs as a left option of some cut $y$, we have $x<y$ by the first constructor of $<$, and similarly whenever $x$ occurs as a right option of a cut $y$, we have $y<x$ by the second constructor of $<$.
  In particular,~\ref{item:NO-le-refl}$\Rightarrow$\ref{item:NO-lt-opt}.

  We prove~\ref{item:NO-le-refl} by $\NO$-induction on $x$.
  Thus, assume $x$ is defined by a cut such that $x^L\le x^L$ and $x^R \le x^R$ for all $L$ and $R$.
  But by our observation above, these assumptions imply $x^L<x$ and $x<x^R$ for all $L$ and $R$, yielding $x\le x$ by the constructor of $\le$.
\end{proof}

\begin{cor}\label{thm:NO-set}
  \NO is a 0-type.
%  (As with $V$, it might be confusing to say that it is a ``set''.)
\end{cor}
\begin{proof}
  The mere relation $R(x,y)\defeq (x\le y) \land (y\le x)$ implies identity by the path constructor of $\NO$, and contains the diagonal by \cref{thm:NO-refl-opt}\ref{item:NO-le-refl}.
  Thus, \cref{thm:h-set-refrel-in-paths-sets} applies.
\end{proof}

By contrast, Conway's Theorem 1 (transitivity of $\le$) is somewhat harder to establish with our definition; see \cref{thm:NO-unstrict-transitive}.

% Of course, we also have:

% \begin{lem}
%   Every surreal number is merely defined by a cut.
% \end{lem}
% \begin{proof}
%   Obvious by $\NO$-induction.
% \end{proof}

We will also need the joint recursion principle, \define{$(\NO,\le,<)$-recursion}.
It is convenient to state this as follows.
Suppose $A$ is a type equipped with relations $\mathord\ble:A\to A\to\prop$ and $\mathord\blt:A\to A\to\prop$.
Then we can define $f:\NO\to A$ by doing the following.
\begin{enumerate}
\item For any $x$ defined by a cut, assuming $f(x^L)$ and $f(x^R)$ to be defined such that $f(x^L)\blt f(x^R)$ for all $L$ and $R$, we must define $f(x)$.  (We call this the \emph{primary clause} of the recursion.)\label{item:NO-rec-primary}
\item Prove that $\ble$ is \emph{antisymmetric}\index{relation!antisymmetric}: if $a\ble b$ and $b\ble a$, then $a=b$.
\item For $x,y$ defined by cuts such that $x^L<y$ for all $L$ and $x<y^R$ for all $R$, and assuming inductively that $f(x^L)\blt f(y)$ for all $L$, $f(x)\blt f(y^R)$ for all $R$, and also that $f(x^L)\blt f(x^R)$ and $f(y^L)\blt f(y^R)$ for all $L$ and $R$, we must prove $f(x)\ble f(y)$.
\item For $x,y$ defined by cuts and an $L_0$ such that $x\le y^{L_0}$, and assuming inductively that $f(x)\ble f(y^{L_0})$, and also that $f(x^L)\blt f(x^R)$ and $f(y^L)\blt f(y^R)$ for all $L$ and $R$, we must prove $f(x)\blt f(y)$.
\item For $x,y$ defined by cuts and an $R_0$ such that $x^{R_0}\le y$, and assuming inductively that $f(x^{R_0})\ble f(y)$, and also that $f(x^L)\blt f(x^R)$ and $f(y^L)\blt f(y^R)$ for all $L$ and $R$, we must prove $f(x)\blt f(y)$.\label{item:NO-rec-last}
\end{enumerate}
The last three clauses can be more concisely described by saying we must prove that $f$ (as defined in the first clause) takes $\le$ to $\ble$ and $<$ to $\blt$.
We will refer to these properties by saying that \emph{$f$ preserves inequalities}.
Moreover, in proving that $f$ preserves inequalities, we may assume the particular instance of $\le$ or $<$ to be obtained from one of its constructors, and we may also use inductive hypotheses that $f$ preserves all inequalities appearing in the input to that constructor.

If we succeed at~\ref{item:NO-rec-primary}--\ref{item:NO-rec-last} above, then we obtain $f:\NO\to A$, which computes on cuts as specified by~\ref{item:NO-rec-primary}, and which preserves all inequalities:
%
\begin{narrowmultline*}
  \fall{x,y:\NO}\Big((x\le y) \to (f(x)\ble f(y))\Big) \land
  \narrowbreak
  \Big((x< y) \to (f(x)\blt f(y))\Big).
\end{narrowmultline*}
%
Like $(\RC,\closesym)$-recursion for the Cauchy reals, this recursion principle is essential for defining functions on $\NO$, since we cannot first define a function on ``pre-surreals'' and only later prove that it respects the notion of equality.

\begin{eg}
  Let us define the \emph{negation} function $\NO\to\NO$.
  We apply the joint recursion principle with $A\defeq\NO$, with $(x\ble y)\defeq (y\le x)$, and $(x\blt y)\defeq (y< x)$.
  Clearly this $\ble$ is antisymmetric.

  For the main clause in the definition, we assume $x$ defined by a cut, with $-x^L$ and $-x^R$ defined such that $-x^L \blt -x^R$ for all $L$ and $R$.
  By definition, this means $-x^R< -x^L$ for all $L$ and $R$, so we can define $-x$ by the cut $\surr{-x^R}{-x^L}$.
  This notation, which follows Conway, refers to the cut whose left options are indexed by the type $\RR$ indexing the right options of $x$, and whose right options are indexed by the type $\LL$ indexing the left options of $x$, with the corresponding families $\RR\to\NO$ and $\LL\to\NO$ defined by composing those for $x$ with negation.

  We now have to verify that $f$ preserves inequalities.
  \begin{itemize}
  \item For $x\le y$, we may assume $x^L<y$ for all $L$ and $x < y^R$ for all $R$, and show $-y\le -x$.
    But inductively, we may assume $-y <-x^L$ and $-y^R<-x$, which gives the desired result, by definition of $-y$, $-x$, and the constructor of $\le$.
  \item For $x<y$, in the first case when it arises from some $x\le y^{L_0}$, we may inductively assume $-y^{L_0} \le -x$, in which case $-y<-x$ follows by the constructor of $<$.
  \item Similarly, if $x<y$ arises from $x^{R_0}\le y$, the inductive hypothesis is $-y \le -x^R$, yielding $-y<-x$ again.
  \end{itemize}
\end{eg}

To do much more than this, however, we will need to characterize the relations $\le$ and $<$ more explicitly, as we did for the Cauchy reals in \cref{thm:RC-sim-characterization}.
Also as there, we will have to simultaneously prove a couple of essential properties of these relations, in order for the induction to go through.

\begin{thm}\label{defn:No-codes}
  There are relations $\mathord\preceq:\NO\to\NO\to\prop$ and $\mathord\prec:\NO\to\NO\to\prop$ such that if $x$ and $y$ are surreals defined by cuts, then
  \begin{align*}
    (x\preceq y) &\defeq
    \big(\fall{L} x^L\prec y\big) \land \big(\fall{R} x\prec y^R\big)\\
    (x\prec y) &\defeq
    \big(\exis{L} x\preceq y^L\big) \lor \big(\exis{R} x^R \preceq y\big).
  \end{align*}
  Moreover, we have
  \begin{equation}\label{eq:NO-codes-unstrict}
    (x\prec y) \to (x\preceq y)
  \end{equation}
  and all the reasonable transitivity properties making $\prec$ and $\preceq$ into a ``bimodule''\index{bimodule} over $\le$ and $<$:
  \begin{equation}\label{eq:NO-codes-transitivity}
    \begin{array}{c@{\hspace{1cm}}c}
      (x \le y) \to (y\preceq z) \to (x\preceq z) &
      (x \preceq y) \to (y\le z) \to (x\preceq z) \\
      (x \le y) \to (y\prec z) \to (x\prec z) &
      (x \preceq y) \to (y< z) \to (x\prec z) \\
      (x < y) \to (y\preceq z) \to (x\prec z) &
      (x \prec y) \to (y\le z) \to (x\prec z).
  \end{array}
  \end{equation}
\end{thm}

\begin{proof}
  We define $\preceq$ and $\prec$ by double $(\NO,\le,<)$-induction on $x,y$.
  The first induction is a simple recursion, whose codomain is the subset $A$ of $(\NO\to\prop)\times (\NO\to\prop)$ consisting of pairs of predicates of which one implies the other and which satisfy ``transitivity on the right'', i.e.~\eqref{eq:NO-codes-unstrict} and the right column of~\eqref{eq:NO-codes-transitivity} with $(x\preceq \blank)$ and $(x\prec \blank)$ replaced by the two given predicates.
  As in the proof of \cref{defn:RC-approx}, we regard these predicates as half of binary relations, writing them as $y\mapsto (\hle y)$ and $y\mapsto (\hlt y)$, with $\hlname$ denoting the pair of relations.
  % The precise definition of $A$ is
  % \begin{align*}
  %   A\defeq \bigg\{ \hlname : (\NO\to\prop)\times (\NO\to\prop) \;\bigg|\;\\
  %   \begin{split}
  %     \fall{y,z:\NO}
  %     &\Big( (\hle y) \to (y\le z) \to (\hle z) \Big)\\
  %     \land\; &\Big( (\hle y) \to (y< z) \to (\hlt z) \Big)\\
  %     \land\; &\Big( (\hlt y) \to (y\le z) \to (\hlt z) \Big)\\
  %     \land\; &\Big( (\hlt y) \to (y< z) \to (\hlt z) \Big) \bigg\}
  %   \end{split}
  % \end{align*}
  We equip $A$ with the following two relations:
  \begin{align*}
    (\hlname \ble \hlbname) &\defeq
    \fall{y:\NO} \Big( (\hleb y) \to (\hle y) \Big) \land
    \Big( (\hltb y) \to (\hlt y) \Big),\\
    (\hlname \blt \hlbname) &\defeq
    \fall{y:\NO} \Big( (\hleb y) \to (\hlt y) \Big).
    %\land \Big( (\hltb y) \to (\hlt y) \Big)
  \end{align*}
  Note that $\ble$ is antisymmetric, since if $\hlname \ble \hlbname$ and $\hlbname \ble \hlname$, then $(\hleb y) \Leftrightarrow (\hle y)$ and $(\hltb y) \Leftrightarrow (\hlt y)$ for all $y$, hence $\hlname=\hlbname$ by univalence for mere propositions and function extensionality.
  Moreover, to say that a function $\NO\to A$ preserves inequalities is exactly to say that, when regarded as a pair of binary relations on $\NO$, it satisfies ``transitivity on the left'' (the left column of~\eqref{eq:NO-codes-transitivity}).

  Now for the primary clause of the recursion, we assume given $x$ defined by a cut, and relations $(x^L \prec \blank)$, $(x^R \prec \blank)$, $(x^L \preceq \blank)$, and $(x^R \preceq \blank)$ for all $L$ and $R$, of which the strict ones imply the non-strict ones, which satisfy transitivity on the right, and such that
  \begin{equation}\label{eq:NO-prec-outer-IH}
    \fall{L,R}{y:\NO}\Big( (x^R\preceq y) \to (x^L \prec y) \Big).
    % \land\Big( (x^R \prec y) \to (x^L \prec y) \Big)
  \end{equation}
  We now have to define $(x\prec y)$ and $(x\preceq y)$ for all $y$.
  Here in contrast to \cref{defn:RC-approx}, rather than a nested recursion, we use a nested induction, in order to be able to inductively use transitivity on the left with respect to the inequalities $x^L<x$ and $x<x^R$.
  Define $A':\NO\to\type$ by taking $A'(y)$ to be the subset $A'$ of $\prop\times\prop$ consisting of two mere propositions, denoted $\tle y$ and $\tlt y$ (with $\tlname:A'(y)$), such that
  \begin{gather}
    (\tlt y) \to (\tle y)\\
    \fall{L} (\tle y)\to (x^L\prec y) \label{eq:NO-prec-IHL}\\
    \fall{R} (x^R \preceq y) \to (\tlt y) \label{eq:NO-prec-IHR}.
  \end{gather}
  Using notation analogous to $\ble$ and $\blt$, we equip $A'$ with the two relations defined for $\tlname:A'(y)$ and $\tlbname:A'(z)$ by
  \begin{align*}
    (\tlname \bble \tlbname) &\defeq
    \Big((\tle y) \to (\tleb z)\Big) \land \Big((\tlt y) \to (\tltb z)\Big)\\
    (\tlname \bblt \tlbname) &\defeq
    \Big((\tle y) \to (\tltb z)\Big). % \land \Big(\tlt \to \tltb\Big).
  \end{align*}
  % (These are the type families $B$ and $C$ in the general induction principle.)
  Again, $\bble$ is evidently antisymmetric in the appropriate sense.
  Moreover, a function $\prd{y:\NO} A'(y)$ which preserves inequalities is precisely a pair of predicates of which one implies the other, which satisfy transitivity on the right, and transitivity on the left with respect to the inequalities $x^L<x$ and $x<x^R$.
  Thus, this inner induction will provide what we need to complete the primary clause of the outer recursion.

  For the primary clause of the inner induction, we assume also given $y$ defined by a cut, and properties $(x\prec y^L)$, $(x\prec y^R)$, $(x\preceq y^L)$, and $(x\preceq y^R)$ for all $L$ and $R$, with the strict ones implying the non-strict ones, transitivity on the left with respect to $x^L<x$ and $x<x^R$, and on the right with respect to $y^L<y^R$.
  % \begin{equation}
  %   \fall{L,R}\Big((x \preceq y^L) \to (x \prec y^R)\Big) % \land \Big((x \prec y^L) \to (x\prec y^R)\Big).
  %   \label{eq:NO-prec-inner-IH}
  % \end{equation}
  We can now give the definitions specified in the theorem statement:
  \begin{align}
    (x\preceq y) &\defeq
    (\fall{L} x^L\prec y) \land (\fall{R} x\prec y^R), \label{eq:NO-preceq-def}\\
    (x\prec y) &\defeq
    (\exis{L} x\preceq y^L) \lor (\exis{R} x^R \preceq y).\label{eq:NO-prec-def}
  \end{align}
  For this to define an element of $A'(y)$, we must show first that $(x\prec y) \to (x\preceq y)$.
  The assumption $x\prec y$ has two cases.
  On one hand, if there is $L_0$ with $x\preceq y^{L_0}$, then by transitivity on the right with respect to $y^{L_0}<y^R$, we have $x\prec y^R$ for all $R$.
  Moreover, by transitivity on the left with respect to $x^L<x$, we have $x^L \prec y^{L_0}$ for any $L$, hence $x^L\prec y$ by transitivity on the right.
  Thus, $x\preceq y$.

  On the other hand, if there is $R_0$ with $x^{R_0}\preceq y$, then by~\eqref{eq:NO-prec-outer-IH}, we have $x^L \prec y$ for all $L$.
  And by transitivity on the left and right with respect to $x<x^{R_0}$ and $y<y^R$, we have $x\prec y^R$ for any $R$.
  Thus, $x\preceq y$.

  We also need to show that these definitions are transitive on the left with respect to $x^L<x$ and $x<x^R$.
  But if $x\preceq y$, then $x^L\prec y$ for all $L$ by definition; while if $x^R\preceq y$, then $x\prec y$ also by definition.

  Thus,~\eqref{eq:NO-preceq-def} and~\eqref{eq:NO-prec-def} do define an element of $A'(y)$.
  We now have to verify that this definition preserves inequalities, as a dependent function into $A'$, i.e.\ that these relations are transitive on the right.
  Remember that in each case, we may assume inductively that they are transitive on the right with respect to all inequalities arising in the inequality constructor.
  \begin{itemize}
  \item Suppose $x\preceq y$ and $y\le z$, the latter arising from $y^L<z$ and $y<z^R$ for all $L$ and $R$.
    Then the inductive hypothesis (of the inner recursion) applied to $y<z^R$ yields $x\prec z^R$ for any $R$.
    Moreover, by definition $x\preceq y$ implies that $x^L \prec y$ for any $L$, so by the inductive hypothesis of the outer recursion we have $x^L \prec z$.
    Thus, $x\preceq z$.
  \item Suppose $x\preceq y$ and $y<z$.
    First, suppose $y<z$ arises from $y\le z^{L_0}$.
    Then the inner inductive hypothesis applied to $y\le z^{L_0}$ yields $x \preceq z^{L_0}$, hence $x\prec z$.

    Second, suppose $y<z$ arises from $y^{R_0}\le z$.
    Then by definition, $x\preceq y$ implies $x\prec y^{R_0}$, and then the inner inductive hypothesis for $y^{R_0}\le z$ yields $x\prec z$.
  \item Suppose $x\prec y$ and $y\le z$, the latter arising from $y^L<z$ and $y<z^R$ for all $L$ and $R$.
    By definition, $x\prec y$ implies there merely exists $R_0$ with $x^{R_0}\preceq y$ or $L_0$ with $x\preceq y^{L_0}$.
    If $x^{R_0}\preceq y$, then the outer inductive hypothesis yields $x^{R_0}\preceq z$, hence $x\prec z$.
    If $x\preceq y^{L_0}$, then the inner inductive hypothesis for $y^{L_0}<z$ (which holds by the constructor of $y\le z$) yields $x\prec z$.
  % \item Suppose $x\prec y$ and $y<z$.
  %   First, suppose $y<z$ arises from $y\le z^{L_0}$.
  %   Then the inner inductive hypothesis for $y\le z^{L_0}$ yields $x\prec z^{L_0}$, hence $x\preceq z^{L_0}$; thus $x\prec z$.

  %   Second, suppose $y<z$ arises from $y^{R_0}\le z$.
  %   Then by definition, $x\prec y$ implies there merely exists $R_1$ with $x^{R_1}\preceq y$ or $L_1$ with $x\preceq y^{L_1}$.
  %   If $x^{R_1}\preceq y$, then the outer inductive hypothesis implies $x^{R_1}\prec z$, hence $x^{R_1}\preceq z$, and thus $x\prec z$.
  %   And if $x\preceq y^{L_1}$, then the inner inductive hypothesis applied to $y^{L_1}<y^{R_0}$ (which comes from $y$ being defined as a cut) and $y^{R_0}\le z$ yields $x\prec z$.
  \end{itemize}
  This completes the inner induction.
  Thus, for any $x$ defined by a cut, we have $(x\prec \blank)$ and $(x\preceq \blank)$ defined by~\eqref{eq:NO-preceq-def} and~\eqref{eq:NO-prec-def}, and transitive on the right.

  To complete the outer recursion, we need to verify these definitions are transitive on the left.
  After a $\NO$-induction on $z$, we end up with three cases that are essentially identical to those just described above for transitivity on the right.
  Hence, we omit them.
\end{proof}

\begin{thm}\label{thm:NO-encode-decode}
  For any $x,y:\NO$ we have $(x<y)=(x\prec y)$ and $(x\le y)=(x\preceq y)$.
\end{thm}
\begin{proof}
  From left to right, we use $(\NO,\le,<)$-induction where $A(x)\defeq\unit$, with $\preceq$ and $\prec$ supplying the relations $\ble$ and $\blt$.
  In all the constructor cases, $x$ and $y$ are defined by cuts, so the definitions of $\preceq$ and $\prec$ evaluate, and the inductive hypotheses apply.

  From right to left, we use $\NO$-induction to assume that $x$ and $y$ are defined by cuts.
  But now the definitions of $\preceq$ and $\prec$, and the inductive hypotheses, supply exactly the data required for the relevant constructors of $\le$ and $<$.
  % From right to left, we first prove by $\NO$-induction on $x$ that for any $y:\NO$ we have $(x\prec y) \to (x<y)$ and $(x\preceq y) \to (x\le y)$.
  % Thus, we assume this to be true for all $x^L$ and $x^R$ in a cut, and show it for the resulting $x:\NO$.
  % Next, we prove by $\NO$-induction on $y$ that $(x\prec y) \to (x<y)$ and $(x\preceq y) \to (x\le y)$, hence we assume it to be true for all $y^L$ and $y^R$ in a cut, and show it for the resulting $y:\NO$.
  % Now since $x$ and $y$ are both defined by cuts, $x\preceq y$ means that $x^L\prec y$ and $x\prec y^R$ for all $L$ and $R$.
  % By the inductive hypotheses, this gives $x^L<y$ and $x<y^R$, hence $x\le y$ by the constructor of $\le$.
  % Similarly, $x\prec y$ yields merely an $R_0$ with $x^{R_0}\preceq y$ or an $L_0$ with $x\preceq y^{L_0}$.
  % Hence merely $x^{R_0}\le y$ or $x\le y^{L_0}$ by the inductive hypothesis, so $x<y$ by a constructor.
\end{proof}

\begin{cor}\label{thm:NO-unstrict-transitive}
  The relations $\le$ and $<$ on $\NO$ satisfy
  \[ \fall{x,y:\NO} (x<y) \to (x\le y) \]
  and are transitive:
  \index{transitivity!of . for surreals@of $<$ for surreals}
  \index{transitivity!of . for surreals@of $\leq$ for surreals}
  \begin{gather*}
    (x\le y) \to (y\le z) \to (x\le z)\\
    (x\le y) \to (y< z) \to (x< z)\\
    (x< y) \to (y\le z) \to (x< z).
  \end{gather*}
\end{cor}

As with the Cauchy reals, the joint $(\NO,\le,<)$-recursion principle remains essential when defining all operations on $\NO$.

\begin{eg}\label{eg:surreal-addition}
\index{addition!of surreal numbers}%
We define $\mathord+:\NO\to\NO\to\NO$ by recursion on the first argument, followed by induction on the second argument.
For the outer recursion, we take the codomain to be the subset of $\NO\to\NO$ consisting of functions $g$ such that $(x<y) \to (g(x)<g(y))$ and $(x\le y) \to (g(x)\le g(y))$ for all $x,y$.
For such $g,h$ we define $(g\ble h)\defeq \fall{x:\NO} g(x)\le h(x)$ and $(g\blt h)\defeq \fall{x:\NO} g(x)< h(x)$.
Clearly $\ble$ is antisymmetric.

For the primary clause of the recursion, we suppose $x$ defined by a cut, that the functions $(x^L+\blank)$ and $(x^R+\blank)$ are defined, preserve inequalities, and satisfy $x^L+y<x^R+y$, and we define $(x+\blank)$.
As in \cref{defn:No-codes}, rather than an inner recursion, we use an inner induction into the family $A:\NO\to\type$, where $A(y)$ is the subset of those $z:\NO$ such that each $x^L + y < z$ and each $x^R + y > z$.
We equip $A$ with the relations $\le$ and $<$ induced from $\NO$, so that antisymmetry is obvious.
For the primary clause of the inner recursion, we suppose also $y$ defined by a cut, with each $x+y^L$ and $x+y^R$ defined and satisfying $x^L+y^L < x+y^L$, $x^L+y^R < x+y^R$, $x+y^L < x^R + y^L$, and $x+y^R < x^R+y^R$ (these come from the additional conditions imposed on elements of $A(y)$), and also $x+y^L < x+y^R$ (since the elements $x+y^L$ and $x+y^R$ of $A(y)$ form a dependent cut).
Now we give Conway's definition:
\[ x+y \defeq \surr{x^L+y, x+y^L}{x^R+y,x+y^R}. \]
In other words, the left options of $x+y$ are all numbers of the form $x^L+y$ for some left option $x^L$, or $x+y^L$ for some left option $y^L$.
We must show that each of these left options is less than each of these right options:
\begin{itemize}
\item $x^L+y < x^R+y$ by the outer inductive hypothesis.
\item $x^L+y < x^L + y^R < x + y^R$, the first since $(x^L+\blank)$ preserves inequalities, and the second since $x+y^R : A(y^R)$.
\item $x+y^L < x^R+ y^L < x^R + y$, the first since $x+y^L : A(y^L)$ and the second since $(x^R+\blank)$ preserves inequalities.
\item $x+y^L < x+y^R$ by the inner inductive hypothesis (specifically, the fact that we have a dependent cut).
\end{itemize}
We also have to show that $x+y$ thusly defined lies in $A(y)$, i.e.\ that $x^L + y < x+y$ and $x+y < x^R + y$; but this is true by \cref{thm:NO-refl-opt}\ref{item:NO-lt-opt}.

Next we have to verify that the definition of $(x+\blank)$ preserves inequality:
\begin{itemize}
\item If $y\le z$ arises from knowing that $y^L<z$ and $y<z^R$ for all $L$ and $R$, then the inner inductive hypothesis gives $x+y^L<x+z$ and $x+y < x+z^R$, while the outer inductive hypotheses give $x^L+y \le x^L+z$ and $x^R+ y \le x^R+z$.
  Moreover, since $x^R+y$ is by definition a right option of $x+y$, we have $x+y < x^R+y$.
  Similarly, we find that $x^L+z$ is a left option of $x+z$, so that $x^L+z < x+z$.
  Thus, using transitivity, we have $x^L+y < x+z$ and $x+y < x^R+z$; so we may conclude $x+y \le x+z$ by the constructor of $\le$.
\item If $y<z$ arises from an $L_0$ with $y\le z^{L_0}$, then inductively $x+y \le x+z^{L_0}$, hence $x+y<x+z$ since $x+z^{L_0}$ is a right option of $x+z$.
\item Similarly, if $y<z$ arises from $y^{R_0}\le z$, then $x+y<x+z$ since $x+y^{R_0}\le x+z$.
\end{itemize}
This completes the inner induction.
For the outer recursion, we have to verify that $+$ preserves inequality on the left as well.
After an $\NO$-induction, this proceeds in exactly the same way.
\end{eg}

\index{acceptance|(}%
\index{mathematics!formalized}%
In the Appendix to Part Zero of~\cite{conway:onag}, Conway discusses how the surreal numbers may be formalized in ZFC set theory: by iterating along the ordinals and passing to sets of representatives of lowest rank for each equivalence class, or by representing numbers with ``sign-expansions''.
He then remarks that
\begin{quote}\footnotesize
  The curiously complicated nature of these constructions tells us more about the nature of formalizations within ZF than about our system of numbers\dots
\end{quote}
and goes on to advocate for a general theory of ``permissible kinds of construction'' which should include
\begin{enumerate}\footnotesize
\item Objects may be created from earlier objects in any reasonably constructive fashion.\label{item:conway1}
\item Equality among the created objects can be any desired equivalence relation.\label{item:conway2}
\end{enumerate}
\noindent
Condition~\ref{item:conway1} can be naturally read as justifying general principles of \emph{inductive definition}, such as those presented in \cref{sec:strictly-positive,sec:generalizations}.
In particular, the condition of strict positivity for constructors can be regarded as a formalization of what it means to be ``reasonably constructive''.
Condition~\ref{item:conway2} then suggests we should extend this to \emph{higher} inductive definitions of all sorts, in which we can impose path constructors making objects equal in any reasonable way.
For instance, in the next paragraph Conway says:
\begin{quote}\footnotesize
  \dots we could also, for instance, freely create a new object $(x,y)$ and call it the ordered pair of $x$ and $y$.
  We could also create an ordered pair $[x,y]$ different from $(x,y)$ but co-existing with it\dots
  If instead we wanted to make $(x,y)$ into an unordered pair, we could define equality by means of the equivalence relation $(x,y)=(z,t)$ if and only if $x=z,y=t$ \emph{or} $x=t,y=z$.
\end{quote}
The freedom to introduce new objects with new names, generated by certain forms of constructors, is precisely what we have in the theory of inductive definitions.
Just as with our two copies of the natural numbers $\nat$ and $\nat'$ in \cref{sec:appetizer-univalence}, if we wrote down an identical definition to the cartesian product type $A\times B$, we would obtain a distinct product type $A\times' B$ whose canonical elements we could freely write as $[x,y]$.
And we could make one of these a type of unordered pairs by adding a suitable path constructor. % (and perhaps 0-truncating).

To be sure, Conway's point was not to complain about ZF in particular, but to argue against all foundational theories at once:
\begin{quote}\footnotesize
  \dots this proposal is not of any particular theory as an alternative to ZF\dots{}
  What is proposed is instead that we give ourselves the freedom to create arbitrary mathematical theories of these kinds, but prove a metatheorem which ensures once and for all that any such theory could be formalized in terms of any of the standard foundational theories.
\end{quote}
One might respond that, in fact, univalent foundations is not one of the ``standard foundational theories'' which Conway had in mind, but rather the \emph{metatheory} in which we may express our ability to create new theories, and about which we may prove Conway's metatheorem.
For instance, the surreal numbers are one of the ``mathematical theories'' Conway has in mind, and we have seen that they can be constructed and justified inside univalent foundations.
Similarly, Conway remarked earlier that
\begin{quote}\footnotesize
  \dots set theory would be such a theory, sets being constructed from earlier ones by processes corresponding to the usual axioms, and the equality relation being that of having the same members.
\end{quote}
This description closely matches the higher-inductive construction of the cumulative hierarchy of set theory in \cref{sec:cumulative-hierarchy}.
Conway's metatheorem would then correspond to the fact we have referred to several times that we can construct a model of univalent foundations inside ZFC (which is outside the scope of this book).

However, univalent foundations is so rich and powerful in its own right that it would be foolish to relegate it to only a metatheory in which to construct set-like theories.
We have seen that even at the level of sets (0-types), the higher inductive types in univalent foundations yield direct constructions of objects by their universal properties (\cref{sec:free-algebras}), such as a constructive theory of Cauchy completion (\cref{sec:cauchy-reals}).
But most importantly, the potential to model homotopy theory and category theory directly in the foundational system (\cref{cha:homotopy,cha:category-theory}) gives univalent foundations an advantage which no set-theoretic foundation can match.
\index{acceptance|)}%

\index{surreal numbers|)}%

\sectionNotes

Defining algebraic operations on Dedekind reals, especially multiplication, is both somewhat tricky and tedious.
There are several ways to get arithmetic going: each has its own advantages, but they all seem to require some technical work.
For instance, Richman~\cite{Richman:reals} defines multiplication on the Dedekind reals first on the positive cuts and then extends it algebraically to all Dedekind cuts, while Conway~\cite{conway:onag} has observed that the definition of multiplication for surreal numbers works well for Dedekind reals.

Our treatment of the Dedekind reals borrows many ideas from~\cite{BauerTaylor09} where the Dedekind reals are constructed in the context of Abstract Stone Duality.
\index{Abstract Stone Duality}%
This is a (restricted) form of simply typed $\lambda$-calculus with a distinguished object $\Sigma$ which classifies open sets, and by duality also the closed ones. In~\cite{BauerTaylor09} you can also find detailed proofs of the basic properties of arithmetical operations.

The fact that $\RC$ is the least Cauchy complete archimedean ordered field, as was proved in \cref{RC-initial-Cauchy-complete}, indicates that our Cauchy reals probably coincide with the Escard{\'o}-Simpson reals~\cite{EscardoSimpson:01}.
\index{real numbers!Escardo-Simpson@Escard\'o-Simpson}%
It would be interesting to check\index{open!problem} whether this is really the case. The notion of Escard{\'o}-Simpson reals, or more precisely the corresponding closed interval, is interesting because it can be stated in any category with finite products.

In constructive set theory augmented by the ``regular extension axiom'', one may also try to define Cauchy completion by closing under limits of Cauchy sequences with a transfinite iteration.
It would also be interesting to check whether this construction agrees with ours.

It is constructive folklore that coincidence of Cauchy and Dedekind reals requires dependent choice but it is less well known that countable choice suffices. Recall that \define{dependent choice}
\indexdef{axiom!of choice!dependent}%
\index{axiom!of choice!countable}%
\index{total!relation}%
states that for a total relation $R$ on $A$, by which we mean $\fall{x : A} \exis{y : A} R(x,y)$, and for any $a : A$ there merely exists $f : \N \to A$ such that $f(0) = a$ and $R(f(n), f(n+1))$ for all $n : \N$. Our \cref{when-reals-coincide} uses the typical trick for converting an application of dependent choice to one using countable choice. Namely, we use countable choice once to make in advance all the choices that could come up, and then use the choice function to avoid the dependent choices.

The intricate relationship between various notions of compactness in a constructive
setting is discussed in \cite{bridges2002compactness}. Palmgren~\cite{Palmgren:FT} has a
good comparison between pointwise analysis and
pointfree topology.

The surreal numbers were defined by~\cite{conway:onag}, using a sort of inductive definition but without justifying it explicitly in terms of any foundational system.
For this reason, some later authors have tended to use sign-expansions or other more explicit presentations which can be coded more obviously into set theory.
The idea of representing them in type theory was first considered by Hancock, while
Setzer and Forsberg~\cite{forsbergfinite} noted that the surreals and their inequality relations $<$ and $\le$ naturally form an inductive-inductive definition.
The \emph{higher} inductive-inductive version presented here, which builds in the correct notion of equality for surreals, is new.


\sectionExercises

\begin{ex}\label{ex:alt-dedekind-reals}
 Give an alternative definition of the Dedekind reals by first defining the square and then use \cref{mult-from-square}.
 Check that one obtains a commutative ring.
\end{ex}

\begin{ex} \label{ex:RD-extended-reals}
  %
  Suppose we remove the boundedness condition in \cref{defn:dedekind-reals}.
  Then we obtain the \define{extended reals}
  \indexdef{real numbers!extended}%
  \indexdef{extended real numbers}%
  which contain $-\infty \defeq
  (\emptyt, \Q)$ and $\infty \defeq (\Q, \emptyt)$. Which definitions of arithmetical
  operations on cuts still make sense for extended reals? What algebraic structure do we
  get?
\end{ex}

\begin{ex} \label{ex:RD-lower-cuts}
  %
  By considering one-sided cuts we obtain \define{lower} and \define{upper} Dedekind reals,
  \indexdef{real numbers!Dedekind!upper}%
  \indexdef{real numbers!Dedekind!lower}%
  \indexdef{lower Dedekind reals}%
  \indexdef{upper Dedekind reals}%
  \index{cut!Dedekind}%
  respectively. For example, a lower real is given by a predicate $L : \Q \to \Omega$
  which is
  %
  \begin{enumerate}
  \item \emph{inhabited:} $\exis{q : \Q} L(q)$ and
  \item \emph{rounded:} $L(q) = \exis{r : \Q} q < r \land L(r)$.
    \index{rounded!Dedekind cut}
  \end{enumerate}
  %
  (We could also require $\exis{r : \Q} \lnot L(r)$ to exclude the cut $\infty \defeq
  \Q$.) Which arithmetical operations can you define on the lower reals? In particular,
  what happens with the additive inverse?
\end{ex}

\begin{ex} \label{ex:RD-interval-arithmetic}
  %
  \index{interval!arithmetic}%
  Suppose we remove the locatedness condition in \cref{defn:dedekind-reals}.
  Then we obtain the \define{interval domain}
  \indexdef{interval!domain}%
  $\mathbb{I}$ because cuts are allowed
  to have ``gaps'', which are just intervals. Define the partial order $\sqsubseteq$ on
  $\mathbb{I}$ by
  %
  \begin{narrowmultline*}
    ((L, U) \sqsubseteq (L', U'))
    \defeq \narrowbreak
    (\fall{q : \Q} L(q) \Rightarrow L'(q)) \land
    (\fall{q : \Q} U(q) \Rightarrow U'(q)).
  \end{narrowmultline*}
  %
  What are the maximal elements of $\mathbb{I}$ with respect to $\mathbb{I}$? Define the
  ``endpoint'' operations which assign to an element of the interval domain its lower and
  upper endpoints. Are the endpoints reals, lower reals, or upper reals (see
  \cref{ex:RD-lower-cuts})? Which definitions of arithmetical operations on cuts still
  make sense for the interval domain?
\end{ex}

\begin{ex} \label{ex:RD-lt-vs-le}
  Show that, for all $x, y : \RD$,
  %
  \begin{equation*}
    \lnot (x < y) \Rightarrow y \leq x
  \end{equation*}
  %
  and
  %
  \begin{equation*}
    \eqv{(x \leq y)}{\Parens{\prd{\epsilon : \Qp} x < y + \epsilon}}.
  \end{equation*}
  %
  Does $\lnot (x \leq y)$ imply $y < x$?
\end{ex}

\begin{ex} \label{ex:reals-non-constant-into-Z}
  \mbox{}
  %
  \begin{enumerate}
  \item
    Assuming excluded middle, construct a non-constant map $\RD \to \Z$.
  \item
    Suppose $f : \RD \to \Z$ is a map such that $f(0) = 0$ and $f(x) \neq 0$ for all $x >
    0$. Derive from this the limited principle of omniscience~\eqref{eq:lpo}.
\index{limited principle of omniscience}%
  \end{enumerate}
\end{ex}

\begin{ex} \label{ex:traditional-archimedean}
  \index{ordered field!archimedean}%
  Show that in an ordered field $F$, density of $\Q$ and the traditional archimedean axiom
  are equivalent:
  %
  \begin{equation*}
    (\fall{x, y : F} x < y \Rightarrow \exis{q : \Q} x < q < y)
    \Leftrightarrow
    (\fall{x : F} \exis{k : \Z} x < k).
  \end{equation*}
\end{ex}

\begin{ex} \label{RC-Lipschitz-on-interval} Suppose $a, b : \Q$ and $f : \setof{ q : \Q |
    a \leq q \leq b } \to \RC$ is Lipschitz with constant~$L$. Show that there exists a unique
  extension $\bar{f} : [a,b] \to \RC$ of $f$ which is Lipschitz with
  constant~$L$. Hint: rather than redoing \cref{RC-extend-Q-Lipschitz} for closed
  intervals, observe that there is a retraction $r : \RC \to [-n,n]$ and apply
  \cref{RC-extend-Q-Lipschitz} to $f \circ r$.
\end{ex}

\begin{ex} \label{ex:metric-completion}
  \index{completion!of a metric space}%
  Generalize the construction of $\RC$ to construct the Cauchy completion of any metric space. First, think about which notion of real numbers is most natural as the codomain for the distance\index{distance} function of a metric space. Does it matter? Next, work out the details of two constructions:
  %
  \begin{enumerate}
  \item Follow the construction of Cauchy reals to define the completion of a metric space as an inductive-inductive type closed under limits of Cauchy sequences.\index{Cauchy!sequence}
  \item Use the following construction due to Lawvere~\cite{lawvere:metric-spaces}\index{Lawvere} and Richman~\cite{Richman00thefundamental}, where the completion of a metric space $(M, d)$ is given as the type of \define{locations}.
    \indexdef{location}%
    A location is a function $f : M \to \R$ such that
    %
    \begin{enumerate}
    \item $f(x) \geq |f(y) - d(x,y)|$ for all $x, y : M$, and
    \item $\inf_{x \in M} f(x) = 0$, by which we mean $\fall{\epsilon : \Qp} \exis{x : M} |f(x)| < \epsilon$ and $\fall{x : M} f(x) \geq 0$.
    \end{enumerate}
    %
    The idea is that $f$ looks like it is measuring the distance from a point.
  \end{enumerate}
  %
  \index{universal!property!of metric completion}%
  Finally, prove the following universal property of metric completions: a locally uniformly continuous map from a metric space to a Cauchy complete metric space extends uniquely to a locally uniformly continuous map on the completion. (We say that a map is \define{locally uniformly continuous}
  \indexdef{function!locally uniformly continuous}%
  \indexdef{locally uniformly continuous map}%
  if it is uniformly continuous on open balls.)
\end{ex}

\index{metric space|)}%

\begin{ex} \label{ex:reals-apart-neq-MP}
  \define{Markov's principle}
  \indexdef{axiom!Markov's principle}%
  \indexdef{Markov's principle}%
  says that for all $f : \nat \to \bool$,
  %
  \begin{equation*}
    (\lnot \lnot \exis{n : \nat} f(n) = \btrue)
    \Rightarrow
    \exis{n : \nat} f(n) = \btrue.
  \end{equation*}
  %
  This is a particular instance of the law of double negation~\eqref{eq:ldn}. Show that
  $\fall{x, y: \RD} x \neq y \Rightarrow x \apart y$ implies Markov's principle. Does the
  converse hold as well?
\end{ex}

\begin{ex} \label{ex:reals-apart-zero-divisors}
  \index{apartness}%
  Verify that the following ``no zero divisors'' property holds for the real numbers:
  $x y \apart 0 \Leftrightarrow x \apart 0 \land y \apart 0$.
\end{ex}

\begin{ex} \label{ex:finite-cover-lebesgue-number}
  %
  Suppose $(q_1, r_1), \ldots, (q_n, r_n)$ pointwise cover $(a, b)$. Then there is
  $\epsilon : \Qp$ such that whenever $a < x < y < b$ and $|x - y| < \epsilon$
  then there merely exists $i$ such that $q_i < x < r_i$ and $q_i < y < r_i$. Such an
  $\epsilon$ is called a \define{Lebesgue number}
  \indexdef{Lebesgue number}%
  for the given cover.
\end{ex}

\begin{ex} \label{ex:mean-value-theorem}
  %
  Prove the following approximate version of the intermediate value theorem:
  %
  \begin{quote}
    \emph{
      If $f : [0,1] \to \R$ is uniformly continuous and $f(0) < 0 < f(1)$ then
      for every $\epsilon : \Qp$ there merely exists $x : [0,1]$ such that $|f(x)| <
      \epsilon$.
    }
  \end{quote}
  %
  Hint: do not try to use the bisection method because it leads to the axiom of choice.
  Instead, approximate $f$ with a piecewise linear map. How do you construct a piecewise
  linear map?
\end{ex}

\begin{ex}\label{ex:knuth-surreal-check}
  Check whether everything in~\cite{knuth74:_surreal_number} can be done using the higher
  inductive-inductive surreals of \cref{sec:surreals}.
\end{ex}

\begin{ex}\label{ex:reals-into-surreals}
  Recall the function $\iota_{\RD}:\RD\to\NO$ defined on page~\pageref{reals-into-surreals}.
  \begin{enumerate}
  \item Show that $\iota_{\RD}$ is injective.
  \item There are obvious extensions of $\iota_{\RD}$ to the extended reals (\cref{ex:RD-extended-reals}) and the interval domain (\cref{ex:RD-interval-arithmetic}).
    Are they injective?
  \end{enumerate}
\end{ex}

\begin{ex}\label{ex:ord-into-surreals}
  Show that the function $\iota_{\ord}:\ord\to\NO$ defined on page~\pageref{ord-into-surreals} is injective if and only if \LEM{} holds.
\end{ex}

\begin{ex}\label{ex:hiit-plump}
  Define a type $\mathsf{POrd}$ equipped with binary relations $\le$ and $<$ by mimicking the definition of \NO but using only left options.
  \begin{enumerate}
  \item Construct a map $j:\mathsf{POrd} \to \NO$ and show that it is an embedding.
  \item Show that $\mathsf{POrd}$ is an ordinal (in the next higher universe, like \ord) under the relation $<$.
  \item Assuming propositional resizing, show that $\mathsf{POrd}$ is equivalent to the subset
    \[\setof{A:\ord | \mathsf{isPlump}(A)}\]
    of \ord from \cref{ex:plump-ordinals}.
    Conclude that $\iota_{\ord}:\ord\to\NO$ is injective when restricted to plump ordinals.
  \end{enumerate}
  In the absence of propositional resizing, we may still refer to elements of $\mathsf{POrd}$ (or their images in \NO) as \define{plump ordinals}.\index{ordinal!plump}\index{plump!ordinal}
\end{ex}

\begin{ex}\label{ex:pseudo-ordinals}
  Define a surreal number to be a \define{pseudo-ordinal}\index{pseudo-ordinal}\index{ordinal!pseudo-} if it is equal to a cut $\surr{x^L}{}$ with no right options (but its left options may themselves have right options).
  Show that the statement ``every pseudo-ordinal is a plump ordinal'' is equivalent to \LEM{}.
\end{ex}

\begin{ex}\label{ex:double-No-recursion}
  Note that \cref{defn:No-codes} and \cref{eg:surreal-addition} both use a similar pattern to define a function $\NO \to \NO \to B$: an outer \NO-recursion whose codomain is the set of order-preserving functions $\NO\to B$, followed by an inner \NO-induction into a family $A:\NO\to\type$ where $A(y)$ is a subset of $B$ ensuring that the inequalities $x^L<x$ and $x<x^R$ are also preserved.
  Formulate and prove a general principle of ``double \NO-recursion'' that generalizes these proofs.
\end{ex}

\index{real numbers|)}%

%%% Local Variables:
%%% mode: latex
%%% TeX-master: "hott-online"
%%% End:

\section{A topos with countable reals}
\label{sec:topos-with-countable}

Given a Miller sequence~$\mil$, as in~\cref{sec:non-diag-sequ}, let~$\MM{\mil} \defeq \invim{\srep}(\mil)$ be the set of all oracles representing~$\mil$ where $\srep$ is the representing map from \cref{sec:oracle-comp-maps}. Define $\TT{\mil} \defeq \PRT{\KK,\MM{\mil}}$ to be the parameterized realizability topos constructed from the ppca $\KK$ with oracles $\MM{\mil}$, see \cref{ex:oracle-ppca}.

\subsection{Countability of the reals}
\label{sec:countability-reals}
%
Let us immediately address countability of the Dedekind reals in~$\TT{\mil}$. 
We first reduce the problem to countability of the closed interval.

\begin{lemmaC}
  \label{lem:R-contable-iff-I-countable}%
  The real numbers are countable if, and only if, the closed unit interval is countable.
\end{lemmaC}

\begin{proof}
  If $\RR$ is countable then $[0,1]$ is countable because there is a retraction $\RR \to [0,1]$, for instance
  $x \mapsto \max(0, \min(1, x))$.
  %
  Conversely, given $e : \NN \to [0,1]$ is a surjection, we claim that $e' : \NN \to \RR$, defined by
  $e'(\pair{m, n}) \defeq m \cdot (2 \cdot e(n) - 1)$, is a surjection also. For any $x \in \RR$ there is $m \in \NN$ such that
  $-m < x < m$, and there is $n \in \NN$ such that $e(n) = \frac{x + m}{2 m}$, hence $e'(\pair{m, n}) = x$.
  %
  % computation of e'(<m, n>):
  % e'(<m, n>) = m (2 e(n) - 1)
  %           = m (2 ((x + m)/(2m)) - 1)
  %           = m ((x + m)/m - 1)
  %           = m (x/m)
  %           = x.
\end{proof}

Next, we obtain custom descriptions of the assemblies of natural numbers and the closed unit interval.

\begin{lemma}
  \label{lem:nno-assembly}%
  In the topos $\TT{\mil}$ the natural numbers object is isomorphic to the assembly $\objN$ with carrier
  $\carrier{\objN} \defeq \NN$ and the existence predicate
  %
  $\Ex{\objN}(n) \defeq \set{n}$.
\end{lemma}

\begin{proof}
  In \cref{sec:natur-numb-integ} we saw that the natural numbers object is the assembly with carrier
  $\NN$ and existence predicate $n \mapsto \set{\numeral{n}}$. The assembly from the statement is isomorphic
  to it because we may convert between $\numeral{n}$ and $n$ using the combinators~$\combNum$ and~$\combCur$ from \cref{ex:numers-vs-numerals}.
\end{proof}

Henceforth we use~$\objN$ from \cref{lem:nno-assembly} as the standard natural numbers object. The practical consequence is that we may eschew Curry numerals and instead use numbers directly.

\begin{lemma}
  \label{lem:interval-assembly}%
  In the topos $\TT{\mil}$ the closed unit interval is isomorphic to the assembly $\objI$ with carrier
  $\carrier{\objI} \defeq [0,1] \cap \carrier{\RRd}$ and the existence predicate
  %
  $\Ex{\objI}(x) \defeq \set{m \in \KK \such \mil(m) = x}$.
\end{lemma}

\begin{proof}
  The sub-assembly $\set{x \in \RRd \such 0 \leq x \land x \leq 1}$ has $[0,1] \cap \carrier{\RRd}$ as its carrier set. Its existence predicate is the tripos predicate
  %
  $[x \of \RRd \such 0 \leq x \land x \leq 1]$,
  %
  which is $\neg\neg$-stable, and therefore equivalent to $\Ex{\RRd}(x)$ restricted to~$\carrier{\objI}$.
  %
  Thus, it suffices to show that the tripos logic validates
  %
  \begin{equation}
    \label{eq:objI-ER-to-EI}
    %
    x \of \carrier{\objI} \such \Ex{\RRd}(x) \to \Ex{\objI}(x).
  \end{equation}
  %
  and
  %
  \begin{equation}
    \label{eq:objI-EI-to-ER}
    x \of \carrier{\objI} \such \Ex{\objI}(x) \to \Ex{\RRd}(x).
  \end{equation}
  %
  By \cref{cor:dedekind-characterization}, $\R{r} \in \Ex{\RRd}(x)$ is equivalent to
  %
  \begin{equation}
    \label{eq:objI-rz-x}
    \all{\alpha \in \MM{\mil}}
    \all{k \in \NN}
    |x - \rat{\pr[\alpha]{\R{r}}(k)}| < 2^{-k},
  \end{equation}
  %
  where we used $\objN$ from \cref{lem:nno-assembly}.
  %
  Condition \eqref{eq:objI-rz-x} states that~$\R{r}$ is a $\mil$-index for~$x$ in the sense of \cref{def:sequence-computable}, therefore $\mil(\R{r}) = x$ and we may realize \eqref{eq:objI-ER-to-EI} with $\ucode{\abstr{r} r}$.

  It remains to realize \eqref{eq:objI-EI-to-ER}.
  %
  In \cref{sec:oracle-comp-maps} we obtained $\R{v} \in \NN$ such that, if~$\alpha$ codes $\mil$ then
  $\pr[\alpha]{\R{v}}(m)$ represents $\mil(m)$, for all $m \in \NN$.
  Clearly, $\R{v}$ realizes \eqref{eq:objI-EI-to-ER}.
\end{proof}

\begin{theorem}
  \label{thm:countable-reals}
  In the topos $\TT{\mil}$ there is an epimorphism from natural numbers to Dedekind reals.
\end{theorem}

\begin{proof}
  By \cref{lem:R-contable-iff-I-countable,lem:interval-assembly} it suffices to show that the Miller sequence $\mil : \objN \to \objI$, which is realized by $\ucode{\abstr{m} m}$, is an epimorphism. This is so because
    %
  \begin{equation*}
    \all{x \in \objI} \some{m \in \objN} \mil(m) = x
  \end{equation*}
  %
  is trivially realized by $\ucode{\abstr{m}{m}}$ as well.
  %
  % Verification: the topos statement computes to the following tripos predicate:
  % [ ∀ x ∈ I . ∃ m ∈ N . μ(m) = x ] =
  % [ ∀ x : |I| . E_I(x) → ∃ m ∈ ℕ . E_I(μ(m)) ∩ E_I(x) ] =
  % [ ∀ x : |I| . E_I(x) → ∃ m ∈ ℕ . E_I(μ(m)) ∩ E_I(x) ]
  %
  % Given any x ∈ |I| and r ∈ E_I(x), there is m ∈ ℕ such that r = i(m) and μ(m) = x.
  % So we get E_I(μ(m)) ∩ E_I(x) = E_I(x) and (α | (⟨m⟩ m) r) = (α | r), as required.
\end{proof}

\subsection{What else is countable?}
\label{sec:what-else-countable}
%
Given that \cref{thm:fixed-point-R-uncountable,thm:countable-reals,thm:cauchy-uncountable} sandwich the countable Dedekind reals between uncountable Cauchy reals and uncountable MacNeille reals, it is natural to wonder about which classically uncountable spaces are countable in~$\TT{\mil}$.

Products, sums and images of countable sets are countable, which gives basic examples of countable spaces, such as Euclidean spaces $\RRd^n$, hypercubes~$\objI^n$, the unit circle $T \defeq \set{(x,y) \in \RRd \times \RRd \such x^2 + y^2 = 1}$, $n$-spheres, etc.

William F.~Lawvere's~\cite{lawvere69} fixed-point theorem is a source of uncountable sets.

\begin{theoremC}[Lawvere]
  \label{thm:lawvere}
  If $e : A \to B^A$ is surjective then $f : B \to B$ has a fixed point.
\end{theoremC}

\begin{proof}
  Because $e$ is surjective,
  there is $a \in A$ such that $e(a) = (x \mapsto f(e(x)(x)))$, whence $e(a)(a) = f(e(a)(a))$.
\end{proof}

As soon as there is a fixed-point free map $X \to X$, there is no surjection $\objN \to X^\objN$, by the contra-positive of Lawvere's theorem. We already noted in \cref{cor:cantor-diagonal} this to be the case for Cantor space.
%
Two further examples are the countable powers $\RRd^\objN$ and $T^\objN$ of the Dedekind reals and the unit circle, which are uncountable because (non-trivial) translations of~$\RRd$ and rotations of~$T$ have no fixed points.

How about the Hilbert cube $\objI^\objN$?
One might attempt to enumerate it by composing $\mil : \objN \to \objI$ with a space-filling curve $\objI \to \objI^\objN$. However, even constructing just a square-filling curve $\objI \to \objI \times \objI$ in~$\TT{\mil}$ seems impossible,\footnote{One cannot construct intuitionistically a square-filling curve $[0,1] \to [0,1] \times [0,1]$ because there is no such curve in the topos of sheaves on the closed unit square, although countable choice suffices. We do not know whether there is a square-filling curve in~$\TT{\mil}$.}
 so we take another route.
We first need a lemma showing that the elements of $\objI^\objN$ take a special form.

\begin{lemma}
  \label{lem:flattening-realizers}
  There is a total computable function $\ell : \NN \times \NN \to \NN$ such that $\mil(\ell(m,n)) = f(n)$ for all $f \in \carrier{\objI^\objN}$, $m \in \Ex{\objI^\objN}(f)$, and $n \in \NN$.
\end{lemma}

\begin{proof}
  In \cref{sec:oracle-comp-maps} we obtained $\R{v} \in \NN$ such that $\pr[\alpha]{\R{v}}(n)$ represents $\mil(n)$ for all~$n \in \NN$. The map $\ell : \NN \times \NN \to \NN$,
  %
  \begin{equation*}
    \ell(m, n) \defeq \ucode{\abstr{x} \R{v} \, (m \, n) \, x},
  \end{equation*}
  %
  is well-defined by \cref{lem:abstr-uniform} and is computable.

  Now consider any $f \in \carrier{\objI^\objN}$ and $m \in \Ex{\objI^\objN}(f)$.
  %
  Given any $n \in \NN$, we establish $\mil(\ell(m, n)) = f(n)$ by verifying that $\rcomp{\mil}{\ell(m,n)} = f(n)$.
  %
  For any $\alpha \in \MM{\mil}$ and $k \in \NN$,
  %
  \begin{align*}
    \pr[\alpha]{\ell(m,n)}(k)
    &\kleq \alpha \at (\abstr{x} \R{v} \, (m \, n) \, x) \, k \\
    &\kleq \alpha \at \R{v} \, (m \, n) \, k \\
    &\kleq
         \pr[\alpha]{
           \pr[\alpha]{\R{v}}(
             \pr[\alpha]{m}(n)
           )
         }(k),
  \end{align*}
  %
  therefore $\pr[\alpha]{\ell(m,n)} = \pr[\alpha]{\pr[\alpha]{\R{v}}(\pr[\alpha]{m}(n))}$, which is a representation
  of $f(n)$.
\end{proof}

A curious consequence of \cref{lem:flattening-realizers} is that every $f \in \carrier{\objI^\objN}$ is equal to $\mil\circ g$ for some total $g : \NN \to \NN$ that is computable without an oracle.

\begin{theorem}
  \label{thm:hilbert-countable}%
  In the topos $\TT{\mil}$ the Hilbert cube~$\objI^\objN$ is countable.
\end{theorem}

\begin{proof}
  Let $\ell : \NN \times \NN \to \NN$ be as in Lemma~\ref{lem:flattening-realizers} and
  $\R{l} \in \NN$ a realizer for~$\ell$, which exists because~$\ell$ is computable.
  % 
  The map $e : \carrier{\objN} \to \carrier{\objI^\objN}$, defined by $e(m)(n) \defeq \mil(\ell(m,n))$, is realized by
  $\ucode{\abstr{m n} \R{l} \, (\combPair \, m \, n)}$.
  % 
  To show that $e$ is an epimorphism it suffices to prove that
 % 
  \begin{equation*}
    \all{f \in \objI^\objN}
    \some{m \in \objN}
    \all{n \in \objN}
    \mil(\ell(m,n)) = f(n)
  \end{equation*}
  %
  is realized by~$\ucode{\abstr{x} x}$.
  %
  By unfolding the realizability interpretation we find that this amounts to
  %
  \begin{equation*}
    \all{\alpha \in \MM{\mil}}
    \all{\R{b} \in \Ex{\objI^\objN}(f)}
    \all{n \in \NN}
    \R{b} \, n \rz[\alpha] \mil(\ell(\R{b},n)) = f(n).
  \end{equation*}
  %
  This is indeed true by \cref{lem:flattening-realizers} and the fact that~$\R{b}$ realizes~$f$.
\end{proof}

\section{\texorpdfstring{Mathematics in the topos~$\TT{\mil}$}{Mathematics in the topos Tμ}}
\label{sec:analysis-topos-tt}

We devote the last section to exploring a little further the peculiar new mathematical world~$\TT{\mil}$.

\subsection{Brouwer's fixed-point theorem}
\label{sec:brouwers-fixed-point}
%
The reader may have noticed already that having a surjection $\objN \to \objI^\objN$ is precisely the antecedent of Lawvere's theorem, which allows us to easily prove Brouwer's fixed-point theorem.

\begin{theorem}[Brouwer's fixed-point theorem]
  \label{thm:internal-brouwer}%
  In the topos $\TT{\mil}$ every map $\objI^\objN \to \objI^\objN$ has a fixed point,
  and so does every map $\objI^n \to \objI^n$, for every $n \in \NN$.
\end{theorem}

\begin{proof}
  Combining \cref{thm:hilbert-countable} and Lawvere's \cref{thm:lawvere} yields a fixed point of any map $f : \objI \to \objI$. When $n = 0$ the statement is trivial. For the remaining cases, note that the evident bijections $\objN \to \objN \times \objN$ and $\objN \to \set{1, \ldots, n} \times \objN$ induce bijections $\objI^\objN \to (\objI^\objN)^\objN$ and $\objI^\objN \to (\objI^n)^\objN$.
  Composing these with the surjection $e : \objN \to \objI^\objN$ yields surjections $\objN \to (\objI^\objN)^\objN$ and $\objN \to (\objI^n)^\objN$, respectively, so Lawvere's theorem applies again.
\end{proof}

We stated Brouwer's fixed-point theorem for \emph{all} maps, not just the continuous ones, but this is a mirage because all maps are continuous in~$\TT{\mil}$, as we show in \cref{sec:continuity-maps}.

We take a moment to remark that \cref{thm:internal-brouwer} is a fairly unusual property for an intuitionistic topos to have because it implies a constructive taboo, namely the so-called \emph{Limited Lesser Principle of Omniscience (LLPO)}.

\begin{corollary}[LLPO]
  \label{cor:llpo}%
  In the topos $\TT{\mil}$ every Dedekind real is non-negative or non-positive.
\end{corollary}

\begin{proof}
  Given any $x \in \RRd$, the map $y \mapsto \max(0, \min(1, y + x))$ has a fixed point~$y \in \objI$.
  Either $\sfrac{1}{3} < y$ or $y < \sfrac{2}{3}$.
  %
  If $\sfrac{1}{3} < y$ then $x \geq 0$, because $x < 0$ would imply $y = \max(0, \min(1, y + x)) = \max(0, y + x) < y$.
  %
  If $y < \sfrac{2}{3}$ then $x \leq 0$, because $x > 0$ would imply $y = \max(0, \min(1, y + x)) = \min(1, y + x) > y$.
\end{proof}

Sometimes LLPO is phrased as follows: if $a : \objN \to \set{0,1}$ attains value~$1$ at most once, then either $\all{n} a_{2 n} = 0$ or $\all{n} a_{2 n + 1} = 0$. This form follows from \cref{cor:llpo}: if $\sum_{n} a_n \cdot (- \sfrac{1}{2})^n$ is non-negative then $\all{n} a_{2 n} = 0$ and if it is non-positive then $\all{n} a_{2 n + 1} = 0$.

In~\cref{sec:comp-clos-interv} we shall use the following variant of Brouwer's fixed point theorem for partial maps with $\neg\neg$-stable domains of definition.

\begin{theorem}
  \label{thm:partial-brouwer}%
  For every $n \in \NN$, the topos $\TT{\mil}$ validates
  %
  \begin{equation*}
    \all{\phi \in \objI^n \to \ClProp}
    \all{f \in {\set{x \in \objI^n \such \phi(x)}} \to \objI^n}
    \some{y \in \objI^n}
    \phi(y) \lthen f(y) = y.
  \end{equation*}
\end{theorem}

\begin{proof}
  We demonstrate the proof for $n = 2$, which is the instance used in \cref{prop:drinking-buddies}.
  %
  We seek $\R{r} \in \NN$ such that, for all $\alpha \in \MM{\mil}$,
  $\phi : \carrier{\objI^2} \to \set{\bot, \top}$, and
  $f : \set{x \in \carrier{\objI^2} \such \phi(x)} \to \carrier{\objI^2}$
  with $\R{f} \in \NN$ satisfying
  %
  \begin{equation*}
    \all{x \in \carrier{\objI^2}}
    \all{\R{x} \in \Ex{\objI^2}(x)}
    \all{\beta \in \MM{\mil}}
    \phi(x) \lthen (\beta \at \R{f} \, \R{x}) \in \Ex{\objI^2}(f(x)),
  \end{equation*}
  %
  there is $y \in \carrier{\objI^2}$ such that $(\alpha \at \R{r} \, \R{f}) \in \Ex{\objI^2}(y)$ and if $\phi(y)$ then $f(y) = y$. Recalling the fixed-point combinator~$\comb{Z}$ from \cref{sec:progr-with-ppcas}, we define
  %
  \begin{align*}
    \R{r} &\defeq \ucode{\comb{Z} \,
                (\abstr{s \, g} g \,
                    (\combPair \,
                      (\abstr{k} \combFst \, (s \, g) \, k) \,
                      (\abstr{k} \combSnd \, (s \, g) \, k)%
                    )
                )},\\
    \R{a} &\defeq \ucode{\abstr{k} \combFst \, (\R{r} \, \R{f}) \, k},\\
    \R{b} &\defeq \ucode{\abstr{k} \combSnd \, (\R{r} \, \R{f}) \, k},
  \end{align*}
  %
  and $y \defeq (\mil(\R{a}), \mil(\R{b}))$.
  %
  Suppose $\phi(y)$ holds. Then
  %
  $\alpha \at \R{f} \, (\combPair \, \R{a} \, \R{b})$ is defined and
  $(\alpha \at \R{f} \, (\combPair \, \R{a} \, \R{b}) \in \Ex{\objI^2}(f(y))$.
  % r := Z (<r' g> g (pair (<k> fst (r' g) k) (<k> snd (r' g) k)))
  %
  % α | r f =
  % α | Z (<r' g> g (pair (<k> fst (r' g) k) (<k> snd (r' g) k))) f =
  % α | (<r' g> g (pair (<k> fst (r' g) k) (<k> snd (r' g) k))) r f =
  % α | f (pair (<k> fst (r f) k) (<k> snd (r f) k))
  % α | f (pair a b)
  Because $\alpha \at \R{f} \, (\combPair \, \R{a} \, \R{b}) = \alpha \at \R{r} \, \R{f}$,
  it suffices to show that $\R{r} \, \R{f} \in \Ex{\objI^2}(y)$.
  This is the case because $\R{r} \, \R{f}$ realizes an ordered pair, hence its components
  $\combFst \, (\R{r} \, \R{f})$ and $\combSnd \, (\R{r} \, \R{f})$ are defined, and respectively
  compute the same sequences as $\R{a}$ and $\R{b}$.
\end{proof}

\subsection{The intermediate value theorem}
\label{sec:interm-value-theor}
%
The 1-dimensional Brouwer's fixed-point theorem and the Intermediate value theorem are derivable from each other.
%

\begin{lemmaC}
  \label{lem:max-neq-eq}%
  If $\max(a, b) \neq a$ then $\max(a, b) = b$, and similarly for $\min$.
\end{lemmaC}

\begin{proof}
  If $\max(a, b) \neq a$ then $\neg (b \leq a)$.
  By \cref{prop:RRd-stable-equality} it suffices to show that $\neg (\max(a, b) \neq b)$.
  If $\max(a, b) \neq b$ then $\neg (a \leq b)$, which together with $\neg (b \leq a)$ yields a contradiction.
\end{proof}

\begin{theorem}(Intermediate value theorem)
  In the topos $\TT{\mil}$, if $f : \objI \to \RRd$ satisfies $f(0) < 0 < f(1)$ then $f(x) = 0$ for some $x \in \objI$.
\end{theorem}

\begin{proof}
  Given such an~$f$, define $g : \objI \to \objI$ by
  %
  $g(x) \defeq \max(0, \min(1, x - f(x)))$.
  %
  By \cref{thm:internal-brouwer} there is $x \in \objI$ such that
  %
  \begin{equation*}
    \max(0, \min(1, x - f(x))) = x.
  \end{equation*}
  %
  Use \cref{lem:max-neq-eq} lemma to derive $\min(1, x - f(x)) = x$ and once more to derive $x - f(x) = x$,
  yielding $f(x) = 0$.
\end{proof}

\subsection{All maps are continuous}
\label{sec:continuity-maps}
%
Say that $f : \RR \to \RR$ \defemph{jumps at $x$} if there is $\epsilon > 0$ such that $|f(y) - f(x)| > \epsilon$ for all $y > x$. Countability of reals is at odds with existence of such explicitly discontinuous maps.

\begin{propositionC}
  If there is a map with a jump then $\RR$ is uncountable.
\end{propositionC}

\begin{proof}
  Without loss of generality we consider $f : \RR \to \RR$ such that~$f(0) = 0$ and $f(x) > 1$ for all $x > 0$.
  %
  Given a sequence $a : \NN \to \RR$, we construct a real avoiding it by using~$f$ to make decisions in the
  construction of nested intervals $[u_n, v_n]$, as follows. Set $[u_0, v_0] \defeq [0, 1]$ or any other desired initial interval. Assuming $[u_n, v_n]$ has been constructed, let $t \defeq f(\max(0, a_n - (u_n + v_n)/2))$, and
  %
  \begin{equation*}
    [u_{n+1}, v_{n+1}] \defeq
    \begin{cases}
      [u_n, (3 u_n + v_n)/4] &\text{if $t > \sfrac{1}{3}$,} \\
      [(u_n + 3 v_n)/4, v_n] &\text{if $t < \sfrac{2}{3}$.}
    \end{cases}
  \end{equation*}
  %
  The interval $[u_{n+1}, v_{n+1}]$ is well-defined because exactly one of the cases holds.
  Certainly at least one holds, and if they both did then $\sfrac{1}{3} < t < \sfrac{2}{3}$, whence neither $\max(0, a_n - (u_n + v_n)/2) = 0$ nor $\max(0, a_n - (u_n + v_n)/2) > 0$, an impossibility.
  %
  Furthermore, $[u_{n+1}, v_{n+1}]$ avoids~$a_n$ because $t > \sfrac{1}{3}$ implies $a_n \geq (u_n + v_n)/2$ and $t < \sfrac{2}{3}$ implies $a_n \leq (u_n + v_n)/2$.
  %
  The real $x \defeq \lim_n u_n = \lim_n v_n$ thus avoids the sequence~$a$.
\end{proof}

Thus in~$\TT{\mil}$ there are no maps with jumps, and we can do better than that.

\begin{theorem}[KLST]
  In the topos $\TT{\mil}$ all maps $\RRd \to \RRd$ are continuous.
\end{theorem}

\begin{proof}
  The theorem bears the initials of its authors Kreisel, Lacombe, Shoenfield~\cite{KreiselLacombeShoenfield59} and Tseitin~\cite{Tseitin67}. We repurpose a proof that uses the Recursion theorem, checking along the way that it relativizes and is uniform with respect to oracles.

  It suffices to realize continuity at~$0$, specifically the statement
  %
  \begin{equation*}
    \all{f \in \RRd^{\RRd}}
    \all{k \in \objN}
    \some{m \in \objN}
    \all{x \in \RRd}
    |x| < 2^{-m} \lthen |f(x) - f(0)| < 2^{-k + 3},
  \end{equation*}
  %
  which amounts to having a realizer $\R{klst} \in \NN$ such that,
  %
  \begin{multline*}
    \all{f \in \carrier{\RRd^{\RRd}}}
    \all{\R{f} \in \Ex{\RRd^{\RRd}}(f)}
    \all{k \in \NN}
    \all{\alpha \in \MM{\mil}}
    \some{m \in \NN} \\
    (\alpha \at \R{klst} \, \R{f} \, k) = m
    \land
    \all{x \in \carrier{\RRd}}
    |x| < 2^{-m} \lthen |f(x) - f(0)| < 2^{-k + 3}.
  \end{multline*}
  %
  Rather than attempting to write down $\R{klst}$ explicitly, we shall describe an $\alpha$-computable procedure, uniform in~$\alpha \in \MM{\mil}$, which takes as input $\R{f} \in \Ex{\RRd^\RRd}(f)$ and $k \in \NN$, as above, and outputs a suitable~$m \in \NN$ that may depend on all the parameters, including~$\alpha$.

  We start with some auxiliary definitions.
  %
  Let $\R{zero} \in \NN$ be such that $\rat{\R{zero}} = 0$.
  Let $\theta : \NN \times \NN \to \NN$ be a computable map satisfying
  %
  \begin{equation*}
    \pr[\alpha]{\theta(n, j)}(i) =
    \begin{cases}
      \R{zero} &\text{if $i < n$,} \\
      j       &\text{if $i \geq n$.}
    \end{cases}
  \end{equation*}
  %
  If $|\rat{j}| < 2^{-n}$ then $\theta(n, j) \in \Ex{\RRd}(\rat{j})$.

  Next, define $g : \NN \parto \NN$ by $g(i) \defeq \alpha \at \R{f} \, i \, k$ and
  $h : \NN \times \NN \to \NN \cup \set{\star}$ by
  %
  \begin{equation*}
    h(i, t) \defeq \prx[\alpha]{\pr[\alpha]{\R{f}}(i)}{t}(k).
  \end{equation*}
  %
  If the computation $\alpha \at \R{f} \, i \, k$ terminates within~$t$ steps then $h(i, t) = g(i)$, otherwise $h(i, t) = \star$. For any $x \in \carrier{\RRd}$ and $\R{x} \in \Ex{\RRd}(x)$ we have $|f(x) - \rat{g(\R{x})}| < 2^{-k}$, and if $h(\R{x}, t) \neq \star$ then $|f(x) - \rat{h(\R{x}, t)}| < 2^{-k}$ as well.

  By the relativized Kleene's recursion theorem there is a computable map $r : \NN \times \NN \to \NN$, independent of~$\alpha$, such that for all $t \in \NN$:
  %
  \begin{itemize}
  \item if $h(r(\R{f}, k),t) = \star$ then $\pr[\alpha]{r(\R{f}, k)}(t) = \R{zero}$,
  \item if $m \leq t$ is the least number such that $h(r(\R{f}, k), m) \neq \star$ then
    %
    \begin{equation*}
      \pr[\alpha]{r(\R{f}, k)}(t) \simeq
      \min\nolimits_j (|\rat{j}| < 2^{-m} \land |\rat{h(r(\R{f}, k), m)} - \rat{g(\theta(m,j))}| \geq 2^{-k+1}).
    \end{equation*}
  \end{itemize}

  We let our $\alpha$-computable procedure output the least~$m \in \NN$ satisfying $h(r(\R{f}, k), m) \neq \star$.
  %
  Of course, we need to argue that such an~$m$ exists and that
  %
  \begin{equation}
    \label{eq:klst-1}
    \all{x \in \carrier{\RRd}}
    |x| < 2^{-m} \lthen |f(x) - f(0)| < 2^{-k + 3}.
  \end{equation}
  %
  If $h(r(\R{f}, k), t) = \star$ for all $t \in \NN$ then $r(\R{f}, k) \in \Ex{\RRd}(0)$,
  therefore $g(r(\R{f}, k))$ is defined and so $h(r(\R{f}, k), t) = g(r(\R{f}, k)) \neq \star$ for a sufficiently large~$t$, a contradiction.
  It is thus impossible for~$m$ not to exist, so it exists.\footnote{A necessary meta-level application of Markov's principle~\cite{beeson84:_churc}.}

  We claim that $|\rat{h(r(\R{f},k),m)} - f(q)| < 2^{-k+1}$ for all $q \in \QQ$ satisfying $|q| < 2^{-m}$.
  Consider any such~$q$, and suppose the contrary inequality $|\rat{h(r(\R{f},k),m)} - f(q)| \geq 2^{-k+1}$ would hold.
  Then there is a least~$j$ such that $|\rat{j}| < 2^{-m}$ and $|\rat{h(r(\R{f},k),m)} - f(\rat{j})| \geq 2^{-k+1}$,
  in which case $\pr[\alpha]{r(\R{f},k)} = \pr[\alpha]{\theta(m,j)}$ and $r(\R{f},k), \theta(m,j) \in \Ex{\RRd}(\rat{j})$, therefore
  %
  \begin{equation*}
    |\rat{h(r(\R{f}, k), m)} - \rat{g(\theta(m,j))}| \leq
    |\rat{h(r(\R{f}, k), m)} - f(\rat{j})| + |f(\rat{j}) - \rat{g(\theta(m,j))}| < 2^{-k + 1}.
  \end{equation*}
  %
  At the same time $|\rat{h(r(\R{f}, k), m)} - \rat{g(\theta(m,j))}| \geq 2^{-k + 1}$ by the definition of~$r(\R{f}, k)$, a contradiction.

  To establish~\eqref{eq:klst-1}, consider any $x \in \carrier{\RRd}$ with $\R{x} \in \Ex{\RRd}(x)$ such that $|x| < 2^{-m}$,
  and suppose $|f(x) - f(0)| \geq 2^{-k + 3}$ were the case. Then we could solve the Halting problem relative to an oracle~$\alpha \in \MM{\mil}$, as follows. There is $\ell \in \NN$ such that $|x| + 2^{-\ell} < 2^{-m}$. Define $\xi_\alpha : \NN \times \NN \times \NN \to \NN$ by
  %
  \begin{equation*}
    \xi_\alpha(i, j, n) \defeq
    \begin{cases}
      \alpha \at \R{x} \, n  &\text{if $n \leq \ell$ or $\prx[\alpha]{i}{n}(j) = \star$,} \\
      \alpha \at \R{x} \, n' &\text{if $n' \leq n$ least such that $\ell < n'$ and $\prx[\alpha]{i}{n'}(j) \neq \star$.}
    \end{cases}
  \end{equation*}
  %
  The sequence $n \mapsto \xi_\alpha(i, j, n)$ represents a real~$y$ such that $|y| < 2^{-m}$.
  %
  If $\pr[\alpha]{i}(j)$ is undefined then $x = y$ and so $|f(y) - f(0)| \geq 2^{-k+3}$.
  %
  If $\pr[\alpha]{i}(j)$ is defined then $y \in \QQ$ and so by the above claim
  %
  \begin{equation*}
    |f(y) - f(0)| \le |f(y) - \rat{h(r(\R{f}, k), m)}| + |\rat{h(r(\R{f}, k), m)} - f(0)| < 2^{-k+2}.
  \end{equation*}
  %
  To decide whether $\pr[\alpha]{i}(j)$ is defined we compute a sufficiently good approximation of $|f(y) - f(0)|$ to be able to tell whether $|f(y) - f(0)| < 2^{-k+3}$ or $|f(y) - f(0)| > 2^{-k+2}$.
\end{proof}


\subsection{Compactness of the closed interval}
\label{sec:comp-clos-interv}
%
In constructive mathematics various classically equivalent notions of compactness diverge~\cite{bridges02:_compac_contin_const_revis}.
%
We focus on the Heine-Borel compactness of the closed unit interval, which states that every open cover has a finite subcover, as it is the most interesting one in the topos~$\TT{\mil}$.

Say that a sequence of open intervals $(a_0, b_0), (a_1, b_1), \ldots$ forms a \defemph{singular cover} of $[0,1]$ if it covers the interval, but the sum of lengths $\sum_{i \in \NN} b_i - a_1$ is less than~$1$. 
%
Of course, such a thing does not exist classically. In the topos~$\TT{\mil}$ it is readily manufactured from the enumeration $\mil : \objN \to \objI$, just take any $0 < \epsilon < 1$ and set
%
\begin{equation*}
  (a_i, b_i) \defeq (\mil_i - \epsilon \cdot 2^{-i-1}, \mil_i + \epsilon \cdot 2^{-i-1}).
\end{equation*}
%
The $i$-th interval covers $\mil_i$ and $\sum_{i \in \NN} b_i - a_1 = \epsilon$.
Consequently, the Heine-Borel property fails strongly.

\begin{theoremC}
  \label{thm:singular-cover}%
  %
  If $(a_0, b_0), (a_1, b_1), \ldots$ is a singular cover of~$[0,1]$ then for every $n \in \NN$ the set $[0,1] \setminus \bigcup_{i < n} (a_i, b_i)$ is inhabited.
\end{theoremC}

\begin{proof}
  We prove the following stronger statement:
  %
  if $[a_0, b_0], \ldots, [a_n, b_n]$ are closed intervals and $(u, v)$ is an open interval such that $\sum_{i=1}^n b_i - a_i < v - u$, then there is $x \in (u, v)$ which is not in any $[a_i, b_i]$.

  Suppose first that all the endpoints are rational numbers, so that comparisons between them are decidable.
  %
  For each $k = 0, \ldots, n$ we compute a list of pairwise disjoint intervals with rational endpoints
  %
  \begin{equation*}
    (u_{k,1}, v_{k,1}), \ldots, (u_{k, m_k}, v_{k, m_k})
  \end{equation*}
  %
  such that $m_k > 0$, $\sum_{i=k+1}^n b_i - a_i < \sum_{j=1}^{m_k} v_{k,j} - u_{k,j}$, and
  %
  \begin{equation*}
    \textstyle
    (u, v) \setminus \bigcup_{i=1}^k [a_i, b_i] = \bigcup_{j=1}^{m_k} (u_{k,j}, v_{k,j}).
  \end{equation*}
  %
  Start with $m_0 \defeq 1$ and $(u_{0,1}, v_{0,1}) = (u, v)$.
  To progress to $(k+1)$-th stage, replace $(u_{k,j}, v_{k,j})$ with the difference $(u_{k,j}, v_{k,j}) \setminus [a_{k+1}, b_{k+1}]$, which is a union of zero, one, or two disjoint open intervals.
  %
  Also note that the total length decreases by at most $b_{k+1} - a_{k+1}$.
  %
  In the end we may take the midpoint of $(u_{n,1}, v_{n,1})$ to be the desired~$x$.

  When the endpoints are real numbers we may slightly enlarge each $[a_i, b_i]$ to an interval with rational endpoints\footnote{Doing so requires only finitely many choices, which can be carried out by induction, without appealing to the axiom of choice.} and slightly shrink $(u,v)$ to an interval with rational endpoints, while preserving
  $\sum_{i=1}^n b_i - a_i < v - u$.
\end{proof}

The situation is reminiscent of the effective topos~\cite[Sect.~6.4.2]{troelstra88:_const_mathem}, except that there one has to work harder to construct a singular cover because the closed unit interval is not countable. Also note that the sum of lengths of the intervals constructed in \cref{thm:singular-cover} is exactly~$\epsilon$, whereas in the effective topos the sum fails to converge, but its partial sums are bounded by~$\epsilon$.

This however is not all that can be said about the Heine-Borel compactness of the closed unit interval in~$\TT{\mil}$.
%
Observe that the singular cover constructed above consists of intervals whose endpoints are real numbers.
%
Can we also construct one whose endpoints are rational? Surprisingly, no.
%
To see why this is the case we need a bit of preparation.

To lay the groundwork for the proof of \cref{lem:fix-point-free-map}, we explain how to constructively extend certain maps to larger domains.
%
Given $f : [a,b] \to \RR$ and $g : [b, c] \to \RR$ such that $f(b) = g(b)$, there is a map $h : [a,c] \to \RR$ that extends them, namely
%
\begin{equation*}
  h(x) \defeq f(\min(x, b)) + g(\max(x, b)) - f(b).
\end{equation*}
%
The construction can be iterated to give an extension of any finite number of matching maps defined on abutting closed intervals.

Second, consider a solid rectangle~$ABCD$ and points $A'$, $D'$, $E$, as shown in \cref{fig:rectangle}.
%
\begin{figure}[ht]
  \centering
  \begin{tikzpicture}[scale=0.85]
    \fill[color=white!90!black] rectangle (2,3) ;
    \draw (2,0) -- (2,3) ;
    \draw[thick]
       (2,3) node[anchor=west] {$D$} --
       (0,3) node[anchor=east] {$C$} --
       (0,0) node[anchor=east] {$B$} --
       (2,0) node[anchor=west] {$A$} ;
    \draw (0, -2) node[anchor=east] {$A'$} -- (0,0) ; \fill (0, -2) circle (0.05) ;
    \draw (0, 5) node[anchor=east] {$D'$} -- (0,3) ; \fill (0, 5) circle (0.05) ;
    \fill (3.5, 1.5) node[anchor=north west] {$E$} circle (0.05) ;
    \fill (1.25, 1.95) circle (0.05) node[anchor=north west] {$P$} ;
    \fill (0, 2.2) circle (0.05) node[anchor=east] {$r(P)$} ;
    \draw (3.5, 1.5) -- (0, 2.2) ;
    \draw[thin,dashed] (3.5, 1.5) -- (0, 5) ;
    \draw[thin,dashed] (3.5, 1.5) -- (0, -2) ;
  \end{tikzpicture}
  \caption{Extending maps from three sides to the rectangle}
  \label{fig:rectangle}
\end{figure}
%
Define $r : ABCD \to A'D'$ by mapping any point~$P$ to the intersection of $A'D'$ and the straight line through~$E$ and~$P$.
%
Now given maps $f : AB \to \RR$, $g : BC \to \RR$ and $h : CD \to \RR$ satisfying $f(B) = g(B)$ and $g(C) = h(C)$,
we may construct an extension $j : ABCD \to \RR$: transfer the maps~$f$ and~$h$ along congruences $AB \cong A'B$ and $CD \cong CD'$ to maps $f' : A'B \to \RR$ and $h' : CD' \to \RR$, let $i : A'D' \to \RR$ be an extension of $f$, $g'$ and~$h'$ obtained by an application of the previously described technique, and define $j \defeq i \circ r$.

The following lemma is an adaptation of a construction going back to~\cite{orevkov63}, see also \cite[Thm.~IV.10.1]{beeson85:_found_const_mathem}.

\begin{lemmaC}
  \label{lem:fix-point-free-map}%
  Suppose $(a_0, b_0), (a_1, b_1), (a_2, b_2), \ldots$ are open intervals with rational endpoints.
  %
  There is a continuous map $h : ([0,1] \cap \bigcup_{i \in \NN} (a_i, b_i))^2 \to [0,1]^2$ such that, $h$ has a fixed point if, and only if, there is $n \in \NN$ such that  $[0,1] \subseteq (a_0, b_0) \cup \cdots \cup (a_n, b_n)$.
\end{lemmaC}

\begin{proof}
  %
  All intervals considered in the proof have rational endpoints, so we just call them ``intervals''.
  %
  Throughout, we shall depend on decidability of the linear order on~$\QQ$, for example to test inclusion of one interval in another, or to tell whether a finite sequence of intervals with rational endpoints covers~$[0,1]$.

  We first consider the situation when the intervals $(a_i, b_i)$ are well-behaved in the following sense:
  %
  \begin{itemize}
  \item no interval shares an endpoint with $(0,1)$: $a_i, b_i \not\in \set{0,1}$ for all~$i$,
  \item there are no abutting intervals: $b_i \neq a_j$ for all $i, j$, and
  \item no interval is contained in another: $(a_i, b_i) \not\subseteq (a_j, b_j)$ for all $i \neq j$.
  \end{itemize}
  %
  Under these circumstances $(a_i, b_i)$ and $(a_j, b_j)$ are either disjoint with a positive distance between them, or they partially overlap on an open interval. Also, an interval $(a_i, b_i)$ overlaps with at most two other intervals.

  Define $V_k \defeq [0,1] \cap \bigcup_{i = 0}^k [a_i, b_i]$, and let $\partial[0,1]^2$ be the boundary of the unit square.\footnote{More precisely, $\partial[0,1]^2$ is the topological border of $[0,1]^2$ qua subset of the plane -- which need not coincide with the union of the four sides of the square.}
  %
  We construct maps
  %
  \begin{equation*}
    f_k : \partial[0,1]^2 \cup V_k^2 \to [0,1]^2
  \end{equation*}
  %
  so that each $f_k$ extends $f_{k-1}$. In addition, we ensure that if $[0,1] \neq V_k$ then~$f_k$ does not have a fixed point and its image is contained in $\partial[0,1]^2$.

  Let $f_{-1} : \partial[0,1]^2 \to \partial[0,1]^2$ be the rotation of~$\partial[0,1]^2$ by a right angle. For each $k \in \NN$, construct $f_k$ from~$f_{k-1}$ as follows:
  %
  \begin{enumerate}
  \item
    If $V_{k-1} = V_k$ then $f_k \defeq f_{k-1}$.
  \item
    %
    If $V_{k-1} \neq V_k \neq [0,1] $ then $V_k$ is~$V_{k-1}$ properly extended by $[a_k, b_k]$. The left-hand side of \cref{fig:fp-free} depicts a typical situation, where the light gray region is~$V_{k-1}$ and the dark gray the area newly contributed by~$[a_k, b_k]$.
    %
    (The assumption that the intervals are well-behaved makes sure that the dark gray strips have positive width and heights, and that at most one gap is filled at a time.)
    %
\begin{figure}[tp]
  \centering
  \begin{tikzpicture}[baseline=(current bounding box.center),scale=4]
    % Unit square
    \draw[very thick, color=white!70!black] (0,0) rectangle +(1,1) ;
    % Intervals: [0.1, 0.2], [0.3, 0.4], [0.5, 0.7], [0.8, 1.0]
    % column #1
    \fill[fill=white!70!black] (0.1, 0.1) rectangle +(0.1,0.1) ;
    \fill[fill=white!70!black] (0.1, 0.3) rectangle +(0.1,0.1) ;
    \fill[fill=white!70!black] (0.1, 0.5) rectangle +(0.1,0.2) ;
    \fill[fill=white!70!black] (0.1, 0.8) rectangle +(0.1,0.2) ;
    % column #2
    \fill[fill=white!70!black] (0.3, 0.1) rectangle +(0.1,0.1) ;
    \fill[fill=white!70!black] (0.3, 0.3) rectangle +(0.1,0.1) ;
    \fill[fill=white!70!black] (0.3, 0.5) rectangle +(0.1,0.2) ;
    \fill[fill=white!70!black] (0.3, 0.8) rectangle +(0.1,0.2) ;
    % column #3
    \fill[fill=white!70!black] (0.5, 0.1) rectangle +(0.2,0.1) ;
    \fill[fill=white!70!black] (0.5, 0.3) rectangle +(0.2,0.1) ;
    \fill[fill=white!70!black] (0.5, 0.5) rectangle +(0.2,0.2) ;
    \fill[fill=white!70!black] (0.5, 0.8) rectangle +(0.2,0.2) ;
    % column #4
    \fill[fill=white!70!black] (0.8, 0.1) rectangle +(0.2,0.1) ;
    \fill[fill=white!70!black] (0.8, 0.3) rectangle +(0.2,0.1) ;
    \fill[fill=white!70!black] (0.8, 0.5) rectangle +(0.2,0.2) ;
    \fill[fill=white!70!black] (0.8, 0.8) rectangle +(0.2,0.2) ;
    % new interval [0.4,0.5]
    \fill[fill=white!40!black] (0.4, 0.1) rectangle +(0.1,0.1) ;
    \fill[fill=white!40!black] (0.4, 0.3) rectangle +(0.1,0.4) ;
    \fill[fill=white!40!black] (0.4, 0.5) rectangle +(0.1,0.2) ;
    \fill[fill=white!40!black] (0.4, 0.8) rectangle +(0.1,0.2) ;
    \fill[fill=white!40!black] (0.1, 0.4) rectangle +(0.1,0.1) ;
    \fill[fill=white!40!black] (0.3, 0.4) rectangle +(0.4,0.1) ;
    \fill[fill=white!40!black] (0.5, 0.4) rectangle +(0.2,0.1) ;
    \fill[fill=white!40!black] (0.8, 0.4) rectangle +(0.2,0.1) ;
  \end{tikzpicture}
  \hfil
  \begin{tikzpicture}[baseline=(current bounding box.center),scale=8]
    % Outer square
    \fill [fill=white!70!black] (0.3,0.3) rectangle +(0.1,0.1) ;
    \fill [fill=white!70!black] (0.3,0.5) rectangle +(0.1,0.2) ;
    \fill [fill=white!70!black] (0.5,0.5) rectangle +(0.2,0.2) ;
    \fill [fill=white!70!black] (0.5,0.3) rectangle +(0.2,0.1) ;
    \fill[fill=white!40!black] (0.4, 0.3) rectangle +(0.1,0.4) ;
    \fill[fill=white!40!black] (0.4, 0.5) rectangle +(0.1,0.2) ;
    \fill[fill=white!40!black] (0.3, 0.4) rectangle +(0.4,0.1) ;
    \fill[fill=white!40!black] (0.5, 0.4) rectangle +(0.2,0.1) ;
    \draw [thick, white!40!black, text=black]
        (0.7,0.5) -- (0.7,0.7) -- (0.5,0.7) -- (0.45,0.7) node[anchor=south] {$s_3$} -- (0.4,0.7) -- (0.3,0.7) --
        (0.3, 0.5) -- (0.3, 0.45) node[anchor=east]  {$s_2$} --
        (0.3,0.3) -- (0.45,0.3) node[anchor=north] {$s_1$} -- (0.7,0.3) -- (0.7, 0.4) ;
    \draw [very thick, white!40!black, text=black, dashed] (0.7,0.4) -- (0.7,0.45) node[anchor=west]  {$s_4$} -- (0.7,0.5) ;
  \end{tikzpicture}
  \caption{A step in the construction of~$f$ from \cref{lem:fix-point-free-map}}
  \label{fig:fp-free}
\end{figure}
    %
    We obtain~$f_k$ by extending~$f_{k-1}$ to the dark gray area separately on each rectangular component. For example, consider the central component, shown separately on the right-hand side of the figure. Because $V_k \neq [0,1]^2$ at least one of the line segments $s_1, \ldots, s_4$ is not contained in~$V_{k-1}$, say~$s_4$. Using the techniques described above, first extend $f_{k-1}$ to the other line segments, in our case $s_1, s_2, s_3$, all the while making sure that its image is contained in $\partial[0,1]^2$, and then to the entire component.
    %
    The reader may verify easily that the same approach works in other cases.
  \item
    If $V_{k-1} \neq V_k = [0,1]$ then~$[a_k, b_k]$ fills in the last gap in~$[0,1]$.
    %
    We visualize the situation by re-interpreting the right-hand side of the figure as showing~$[0,1]^2$, where
    light gray is~$V_k$ and dark gray the newly contributed area, except that this time all four segments $s_1, \ldots, s_4$ are already in the domain of~$f_{k-1}$. Pick a point in the interior of the dark gray area and declare it to be a fixed point of~$f_k$, then extend $f_k$ to the rest of the square in a piece-wise linear fashion, using the entire square as the codomain of~$f_k$.
    %
    The reader may verify that the same approach works when the dark gray area is adjacent to~$\partial[0,1]^2$, in which case it is shaped like the letter~L.
  \end{enumerate}
  %
  Notice that $V_k \neq [0,1]$ implies that~$f_k$ has no fixed points. Indeed, if $t = f_k(t)$ then $t \in \partial[0,1]^2$, which would make~$t$ a fixed point of~$f_{-1}$.

  Let $h$ be the union of $f_k$'s, restricted to $([0,1]^2 \cap \bigcup_{i \in \NN} (a_i, b_i))^2$.
  %
  We must verify that~$h$ has the required property.
  %
  If $[0,1] \subseteq (a_0, b_0) \cup \cdots \cup (a_n, b_n)$ for some $n \in \NN$, then $V_n = [0,1]$, so~$h$ has a fixed point by construction of~$f_n$.
  %
  Conversely, if~$t$ is a fixed point of~$h$ then $t \in ([0,1] \cap (a_n, b_n))^2$ for some~$n \in \NN$, hence~$f_n$ has a fixed point, which is only possible if $V_n = [0,1]$, but then $[0,1] \subseteq (a_0, b_0) \cup \cdots \cup (a_n, b_n)$ because the intervals are well-behaved.


  It remains to remove the requirement that the intervals be well-behaved.
  %
  Given any sequence of intervals $(a_0, b_0), (a_1, b_1), \ldots$, we define a new well-behaved sequence with the same union, which has a finite subcover of~$[0,1]$ if, and only if, $(a_0, b_0), (a_1, b_1), \ldots$ does.
  %
  We may then apply the above construction to the new sequence.

  Let $p_i$ be the $i$-th prime, and $P_i \defeq p_1 \cdots p_i$ the product of the first~$i$ primes.
  %
  For $i \in \NN$ and $m \in \ZZ$ let
  %
  \begin{equation*}
    \textstyle
    c_{i,m} \defeq \frac{1 + 2 m \cdot p_i}{P_i}
    \qquad\text{and}\qquad
    d_{i,m} \defeq \frac{1 + (2 m + 3) \cdot p_i}{P_{i+1}}.
  \end{equation*}
  %
  % Verification of claims:
  %
  % * the equation (1 + k · p_i)/P_i = 0 has no solutions: obvious
  % * the equation (1 + k · p_i)/P_i = 1 has no solutions: it is equivalent to 1 = p_i · (P_{i-1} - k)
  % * the equation (1 + m · p_i)/P_i = (1 + n · p_j)/P_j has no solutions:
  %   wlog assume i < j and observe that the equation is equivalent to
  %      (1 + m · p_i) P_j = (1 + n · p_j) · P_i
  %      (1 + m · p_i) p_{i+1} ⋯ p_j = 1 + n · p_j
  %      (1 + m · p_i) p_{i+1} ⋯ p_j - n · p_j = 1
  %   LHS is divisible by p_j and RHS is not.
  % From the above it follows that the intervals (c_{i,m}, d_{i,m}) have no common endpoints and the endpoints
  % are never 0 or 1.
  %
  % Verification that (c_{i,m}, d_{i,m}) are well-behaved: 
  %   c_i,m                 d_i,m         c_i,m+2                 d_i,m+2
  %   (-----------------------)            (-----------------------)
  %                     (-----------------------)
  %                    c_i,m+1                 d_i,m+1
  %
  % * c_i,m+1 < d_i,m: reduces to 2 (m + 1) < 2m + 3
  % * d_i,m < c_i,m+2: reduces to 2 m + 3 < 2 (m + 2)
  % * c_i,m+2 < d_i,m+1: reduces to 2 (m + 2) < 2 (m + 1) + 3
  %
  No two intervals $(c_{i,m}, d_{i,m})$ and $(c_{j,n}, d_{j,n})$ share an endpoint, and their endpoints are all different from $0$ and $1$. Also, for a fixed~$i$ the intervals $(c_{i,m}, d_{i,m})$ form a well-behaved cover of~$\RR$.

  We enumerate some of the intervals $(c_{i,m}, d_{i,m})$ in phases, each phase contributing finitely many intervals. In the $i$-th phase we include those $(c_{i,m}, d_{i,m})$ that are contained in $(a_0, b_0) \cup \cdots \cup (a_i, b_i)$ but are not contained in any of the intervals enumerated so far.
  %
  This way we obtain a well-behaved sequence, as the construction of $c_{i,m}$ and $d_{i,m}$ guarantees that intervals are not abutting and that their endpoints avoid $0$ and $1$; and an interval cannot contain another from the same stage, nor from an earlier one as it is too narrow.

  Obviously, the newly enumerated intervals cover at most $\bigcup_{k \in \NN} (a_k, b_k)$. They cover all of it, because any $x \in (a_k, b_k)$ is covered at the latest by the stage at which the widths of $(c_{i,m}, d_{i,m})$'s are smaller than the distance of~$x$ to the endpoints $a_k$ and $b_k$.
  %
  Finally, if $(a_0, b_0) \cup \cdots \cup (a_k, b_k)$ cover $[0,1]$, then they do so with a bit of overlap. There is a stage~$i$ such that the widths of $(c_{i,m}, d_{i,m})$'s are smaller than the overlap, so $[0,1]$ will be covered at least by the $i$-th stage.
\end{proof}

When the previous lemma is combined with Brouwer's fixed point theorem, a variant of Heine-Borel compactness of~$[0,1]$ emerges.

\begin{corollary}
  In the topos~$\TT{\mil}$, a countable cover of the closed unit interval by open intervals with rational endpoints has a finite subcover.
\end{corollary}

\begin{proof}
  Let $h : \objI^2 \to \objI^2$ be the map from \cref{lem:fix-point-free-map} for the given cover of~$\objI$.
  %
  By \cref{thm:internal-brouwer} it has a fixed point, therefore \cref{lem:fix-point-free-map} ensures that~$\objI$ is covered already by a finite subcover.
\end{proof}

We can improve on the corollary to give a variant of the Drinker paradox\footnote{The ``paradox'' states that in every non-empty pub there is a person, such that if the person is drinking then everyone is drinking. It is a non-constructive principle~\cite{warren2018drinker}.} for sufficiently tame predicates on the closed unit interval.

\begin{proposition}[Drinking Buddies Principle]
  \label{prop:drinking-buddies}%
  In the topos~$\TT{\mil}$, suppose $U$ is a countable union of open intervals with rational endpoints.
  There are $x, y \in \objI$ such that $x \in U \land y \in U$ if, and only if, $\all{z \in \objI} z \in U$.
\end{proposition}

\begin{proof}
  Let $(a_0, b_0), (a_1, b_1), \ldots$ be a sequence of intervals with rational endpoints and $U \defeq \bigcup_{i \in \NN} (a_i, b_i)$ and $h : (\objI \cap U)^2 \to \objI^2$ the map from \cref{lem:fix-point-free-map} for the given intervals. We claim that $U$ is a $\neg\neg$-stable subset of~$\RR$. One way to see this is to recall from \cref{lem:lt-stable} that~$<$ is $\neg\neg$-stable, and observe that there is a map $t : \RR \to \RR$ such that $U = \set{x \in \RR \such t(x) > 0}$, for instance a weighted sum of “tent maps” errected on the intervals,
  %
  \begin{equation*}
    \textstyle
    t(x) = \sum_{n \in \NN} 2^{-n} \cdot \max (0, 1 - |\max (a_n, \min (b_n, x)) - (a_n + b_n)/2|).
  \end{equation*}
  %
  Therefore, \cref{thm:partial-brouwer} applies to~$h$ to give $(x, y) \in \objI^2$ such that if $x, y \in U$ then $h(x,y) = (x,y)$, whence $\objI \subseteq U$ by \cref{lem:fix-point-free-map}.
\end{proof}

We are not certain what the principle is good for, apart from obliging one to test its veracity with a buddy in a pub.

%%% Local Variables:
%%% mode: latex
%%% TeX-master: "countable-reals"
%%% End:


\subsubsection*{Acknowledgment}

We thank Ingo Blechschmidt, Joseph Miller, and Andrew Swan for valuable suggestions and engaging discussions.


% \input{extras.tex}

\bibliographystyle{plain}
\bibliography{references}

\appendix
%\input{old-sequences.tex}


\end{document}