\section{Discussion and related work}
\label{sec:discussion-and-related-work}

% summary with elements of future work

We set out to give a detailed and general mathematical definition of dependent type theories, accomplished by an analysis of their traditional accounts.
%
Having completed the task, let us take stock of what has been accomplished.

We calibrated abstraction at the level that keeps a connection with concrete syntax, but also clearly identifies the category-theoretic structure underlying the abstract syntax.
%
As such, our work may serve as a theoretical grounding and a guideline for practical implementations of type theories on one hand, and on the other as a ladder to be climbed and discarded by those who wish to ascend to higher, more abstract viewpoints of dependent type theories.

There is not much to remark on our treatment of raw syntax, as the topic has been studied before and is well understood, except to remark that scope systems have served us well as a general approach to scoping and binding of variables.

More interesting are the subsequent stages of our definition.
%
Giving the definition in generality forces us to isolate and precisely define various notions that are traditionally treated only informally, and often only passed on in folklore rather than in writing. 
%
For instance, we articulate precisely the distinction between the syntactic specification of an inference rule, and the scheme of closure conditions that it begets --- a distinction which, once seen, was implicitly present all along, but which is not generally appreciated or consciously articulated.

We initially considered raw theories as just a stepping stone towards the definition of well-presented type theories, but have come to feel that they are of significant intrinsic interest.
%
Their simplicity makes them easy to work with, and even though they permit deviations from the orthodox teachings, they boast with a surprisingly rich collection of meta-theorems.

The well-behavedness properties of raw rules and raw type theories, such as presuppositivity, tightness, and congruity, took some effort to define and explain, but quickly paid off.
%
On a technical level, they allowed us to fine-tune the requirements that enable the various meta-theorems.
%
More importantly, once we incorporated them into our type-theoretic vocabulary they streamlined communication and invigorated the mind where there used to be just an uneasy adherence to heuristic techniques.

We hope that our selection of meta-theorems is illustrative enough to inspire further generally applicable meta-theorems, and comprehensive enough to relieve future designers of type theories from having to redo the work.
%
We have intentionally restrained any category-theoretic analysis of the landscape we explore, but it will be visible in the background to many readers, demanding future exploration of the categorical structure of the syntactic notions, both for its own sake and to connect with a general categorical semantics.
%
This, of course, we hope to return to in future work.

How widely does our definition of type theories cast its net? It takes little effort to enumerate many examples of interest, such as variants of Martin-Löf type theory~\citep{martin-lof:introduction}, in both its intensional and extensional incarnations, homotopy type theory~\citep{hott-book}, simple type theory~\citep{church32:_set_postul_found_logic}, some presentations of the higher-order logic of toposes~\citep{lambek-scott-book}, etc. But it is equally easy to list counter-examples: System F~\citep{system-F,Reynolds74} and related type systems that directly quantify over all types, pure type systems~\citep{pure-type-systems}, cubical type theories~\citep{cohen15:_cubic_type_theor}, cohesive type theories~\citep{cohesive-tt}, etc. Can such a diverse collection of formalisms be unified under the umbrella of an even more general definition of type theories? Doubtlessly, our work can be pushed and stretched in various directions, and we hope it will.  But we also state again that we do not intend our definitions to be definitive or prescriptive, nor consider our methods to be superior to others. After all, type theory is an open-ended idea.

\medskip

% related work

Several years ago Vladimir Voevodsky's relentless inquiries into the precise mathematical underpinnings of type theories motivated us to undertake the study of general type theories. We were hardly alone to do so. Voevodsky himself initially worked to develop the framework of B-systems and C-systems~\citep{B-systems,C-systems}, but these will remain tragically unfinished. There is by now a spectrum of various approaches to the meta-theory of type theories, which cannot be justly reviewed in the remaining space. We only mention a selection of contemporaneous developments and their relation to our work.

%%%%% LF: defend against it one last time

\paragraph*{Logical framework approaches}

When discussing and presenting this work, we have often been asked why we bothered, when logical frameworks (LF)~\citep{harper-honsell-plotkin:framework,pfenning:logical-frameworks} already give a satisfactory definition of type theories.

The main answer is that most work with logical frameworks does not give a general definition in the same sense that we are looking for.
%
It sets up a framework within which many type theories may be defined, but that is not quite the same thing.
%
Indeed, at the point when we were first embarking on the present project, no definition in a generality close to our aims had been proposed in the LF literature (to our knowledge), nor could we see how to do so using LF methods.
%
Since then, Uemura has succeeded in giving a very clean general definition using the LF; we discuss that work in detail below.

A significant secondary motivation for the present approach, though, was to directly recover the standard “naïve” presentation of particular type theories, in specific examples.
%
This problem is, by design, something that LF-based approaches bypass entirely.
%
One may argue --- as some have --- that this desire is misguided: that since LF-based approaches are so much cleaner, naïve syntax should be discarded as obsolete, and LF-embedded presentations of theories preferred as primitive.
%
We however find that view somewhat unsatisfactory, for several reasons.

Firstly, even if we \emph{should} be always reading type theories as LF presentations, we \emph{have not} been.  At least within the literature on constructive type theory in the tradition of Martin-Löf, most work still uses the naïve reading, including the work of Martin Hofmann~\citep{hofmann:syntax-and-semantics} and others~\citep{martin-lof:introduction,streicher91:_seman,hott-book}.
%
Or rather, most of the literature stays silent about the issue, but where the intended reading is made clear, it tends to be the naïve one.
%
Secondly, most work using LF approaches explicitly comments on the setup, and often gives or cites adequacy theorems.
%
This seems to suggest that writers agree that the correctness of LF presentations of type theories rests, in part, on their connection to the naïve presentations.
%
Finally, it seems very difficult to adopt a position of \emph{completely} discarding the naïve readings, and reading all presentations of type theories always as shorthands for their LF embedding.
%
This is because the framework is itself a type theory, whose presentation must sooner or later be given the naïve reading, rather than as embedded in a further framework --- it cannot be “turtles all the way”.

We therefore believe it is important to have both the naïve and LF-based definitions of syntax defined and developed in as wide a generality as possible (as well as other approaches, such as those of categorical logic).
%
The naïve approach is most natural and conceptually basic.
%
The LF approach is cleaner and simpler to set up, and easier to analyse and apply for some purposes.
%
Both should be available, and connected by an adequacy theorem, not just in special cases but in generality.

%%%%%%

\paragraph*{Uemura’s general type theories}

Another general definition of type theories has recently been given by Taichi Uemura.
%
Indeed, \citep{uemura19:_gener_framew_seman_type_theor} provides two definitions, one semantic and one syntactic, and shows their correspondence with a general initiality theorem.

In terms of generality, Uemura’s definition essentially subsumes ours and indeed generalises much further, encompassing type theories with different judgement forms, while still retaining enough structure to allow proofs of important type-theoretic meta\-theorems.
%
It therefore quite satisfactorily solves one of the major goals we set out to solve with the present project.

On inspection, however, Uemura’s approach is sufficiently different that it is complementary with our approach, rather than subsuming it.
%
His syntactic definition is given via a particularly ingenious use of a logical framework; as with other LF-based approaches, this keeps the setup very clean, but means that in specific examples, it does not so closely recover the standard naïve reading of the theory in question.

It does not, therefore, address our secondary goal of taking seriously the naïve reading of syntax, and directly recovering it in examples.
%
We therefore hope that it should be possible in the future to connect our syntactic definition with Uemura’s by means of a generalised adequacy theorem, and show that for theories with Martin-Löf’s original four judgment forms, the two approaches are equivalent.

%%%%%%

\paragraph*{Other general definitions of dependent type theories}
\label{sec:other-related-work}

Independently of our work, Guillaume Brunerie has proposed a general syntactic definition of dependent type theories~\citep{brunerie20}, and is formalising it in Agda~\citep{brunerie:_agda}.
%
His approach is very similar to ours, which we see as a welcome convergence of ideas.

Valery Isaev has also proposed a definition of dependent type theories, in \citep{isaev17:_algeb}.
%
His approach is semantic, avoiding syntax with binding, and defining dependent type theories as certain essentially algebraic theories extending the theory of categories with families.
%
The generality of his definition seems to be roughly similar to ours, but a precise comparison seems slightly subtle to state, and is beyond the scope of the present paper.

%%%%%%

\paragraph*{Meta-theory of type theory in type theory}

When one takes type theory seriously as a foundation of mathematics, it is natural and even imperative, to develop the meta-theory of type theories in type theory itself. Whereas in the LF approach the expressivity of the ambient formalism is curbed to ensure an adequacy theorem, here we gladly trade adequacy for working in a full-fledged dependent type theory.

Such a project has been undertaken by Thorsten Altenkirch and Ambrus Kaposi~\citep{qiit-tt}. Broadly speaking, a \emph{specific} object type theory is constructed in one fell swoop as a quotient-inductive-inductive type (QIIT) that incorporates all judgement forms, the structural and the specific rules. The inductive character of the definition automatically provides the correct notion of derivation, while the ambient type theory guarantees that only derivable judgements can be constructed --- the ``raw'' stage is completely side-stepped. The quotienting capabilities favourably relate the ambient propositional equality with the object-level judgmental equality. From a semantic point of view, the construction is the type-theoretic analogue of an initial-model construction. The ingenuity of the definition allows one to prove many meta-theorems quite effortlessly, especially with the aid of a proof assistant.

Our bottom-up approach can add little to the setup in terms of abstraction, but can possibly provide useful clues on how to pass from the case-by-case presentations of object type theories to a single type whose inhabitants are (presentations of) general type theories. For instance, the type of well-presented type theories would have to improve on our staged definitions by joining them into a single mutually recursive inductive definition that would incorporate the above QIIT construction of a single object-level theory, suitably adapted, as the realisation of a well-presented theory.

