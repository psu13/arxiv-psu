
\newif\ifdraft
\draftfalse

%% Input and output encoding:
\usepackage[T1]{fontenc}
\usepackage[utf8]{inputenc}  % to allow unicode in source

%% AMS and other general math packages:
\usepackage{amsthm}
\usepackage{amsfonts}
\usepackage{amsmath}
\usepackage{amssymb}

\usepackage{xypic} % Commutative diagrams
\usepackage{booktabs} % for reasonable looking tables
\usepackage{mathtools}

%% To include References in TOC
\usepackage[nottoc,numbib]{tocbibind}
\setcounter{tocdepth}{1}

%% Citation settings conforming to MSCS
\usepackage[round,authoryear]{natbib}

%% geometry

\ifdraft
%% trim overall pdf page margins, without changing text body geometry (for better on-screen viewing in most editors):
  \usepackage{calc}
  \newdimen{\tightmargin} \setlength{\tightmargin}{0.7cm}
  \usepackage[
    paperwidth=\textwidth+2\tightmargin,
    paperheight=\textheight+\footskip+2\tightmargin,
    margin=\tightmargin,
    bottom=\tightmargin+\footskip
  ]{geometry}
\fi
  
%% Font selection

%% Andrej’s preferred font setup
\usepackage{times}
\usepackage{upgreek} % for sans-serif Greek

%% Peter’s preferred font setup
%\usepackage{microtype}
%\usepackage{libertine}
%\usepackage[libertine]{newtxmath}
%\usepackage{upgreek} % for sans-serif Greek
%\useosf % old-style numerals in text (for dates, page numbers, etc).
% Must be placed after \usepackage{newtxmath}, so that numerals in math are lining.
%\usepackage[scaled=0.96]{zi4}  % type-writer font for URLs in bibliography

%% General style packages:
\usepackage{xcolor} % for colors of hyperlinks
\definecolor{darkgreen}{rgb}{0,0.2,0}
\definecolor{darkred}{rgb}{0.25,0,0}
\definecolor{darkblue}{rgb}{0,0,0.3}
\PassOptionsToPackage{obeyspaces}{url}
%\usepackage[breaklinks,colorlinks,citecolor=darkgreen,linkcolor=darkblue,urlcolor=darkblue]{hyperref}
% Set the URL fonts to something less jarring
%\urlstyle{rm}


\usepackage{enumitem} % for customising enumerated lists
% Reduce spacing around lists
\setlist{itemsep=0.25ex}
% \setlist{nosep} % reduce everywhere
% \setlist{noitemsep} % reduce between items

\usepackage[capitalise,nameinlink,sort]{cleveref} % NOTE: be careful if re-organising preamble, “cleveref” must be loaded last or nearly so (see documentation for details)

%% Miscellaneous packages

\usepackage{ifmtarg}
\usepackage{mathpartir} % for inference rules

%% Definition of theorem environment

{
\theoremstyle{plain}
% The [equation] argument makes sure equations and thorems are counted together
\newtheorem{theorem}{Theorem}[section]
%\newtheorem*{theorem*}{Theorem}
\newtheorem{proposition}[theorem]{Proposition}
\newtheorem{lemma}[theorem]{Lemma}
\newtheorem{corollary}[theorem]{Corollary}
}
{
\theoremstyle{definition}
\newtheorem{definition}[theorem]{Definition}
\newtheorem{example}[theorem]{Example}
\newtheorem{examples}[theorem]{Examples}
\newtheorem{nonexample}[theorem]{Non-example}
}

% % Remarks are not used because Andrej deletes them faster than Peter can put them in.
% {
% \theoremstyle{remark}
% \newtheorem{remark}[theorem]{Remark}
% }

% Include section number in numbered equations
\numberwithin{equation}{section}

% versions with qed included, based on amsmath faq suggestion
\makeatletter
\newenvironment{propositionwithqed}[1][]
  {\@ifmtarg{#1}{\begin{proposition}}{\begin{proposition}[#1]}\pushQED{\qed}}
  {\popQED\end{proposition}}
\makeatother

%%%% Organisation

%% For the footnote blurb on the title page
\newcommand{\extrafootertext}[1]{%
    \bgroup
    \renewcommand\thefootnote{\fnsymbol{footnote}}%
    \renewcommand\thempfootnote{\fnsymbol{mpfootnote}}%
    \footnotetext[0]{#1}%
    \egroup
}

%% save value of enumerate counter, for resuming later
\newcounter{saveenumi}
\newcommand{\saveitem}{\setcounter{saveenumi}{\value{enumi}}}
\newcommand{\restoreitem}{\setcounter{enumi}{\value{saveenumi}}}



%%% Local Variables:
%%% mode: latex
%%% End: