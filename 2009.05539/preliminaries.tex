\section{Preliminaries}
\label{sec:preliminaries}

We begin by setting up definitions and terminology of a general mathematical nature that we will use throughout.

\subsection{Families}

For several reasons, we work with \emph{families} in places where classical treatments would use either \emph{subsets} of, or \emph{lists} from, a given set.
%
While the term is standard (e.g. ``the product of a family of rings''), we make rather more central use of it than is usual, so we establish some notations and terminology.

\begin{definition}
  \label{def:family}%
  Given a set $X$, a \defemph{family $K$ of elements of $X$} (or briefly, a \defemph{family on~$X$}) consists of an index set $\famindex K$ and a map $\famev{K} : \famindex{K} \to X$.
  %
  We let $\Fam X$ denote the collection of all families on~$X$, and use the \defemph{family comprehension} notation $\famtuple{e_i \in X}{i \in I}$ for the family indexed by~$I$ that maps~$i$ to~$e_i$. A family may be explicitly described by displaying the association of indices to values. For example, we may write $\family{0 \mto e_0, 1 \mto e_1, 2 \mto e_2}$ for the family $\famtuple{e_i}{i \in \set{0,1,2}}$.
\end{definition}

\begin{example}
  Any \emph{subset} $A \subseteq X$ can be viewed as a family $\famtuple{i \in X}{i \in A}$, with $\famindex A$ as $A$ itself and $\famev{A}$ the inclusion $A \injto X$.
  %
  Motivated by this, we will often speak of a family~$K$ as if it were a subset, writing $x \in K$ rather than $x \in \famindex K$, and treating such $x$ itself as an element of $X$ rather than explicitly writing $\famev{K}(x)$.
\end{example}

\begin{example}
  Any \emph{list} $\ell = [x_0,\ldots,x_n]$ of elements of $X$ can be viewed as a family, with $\famindex \ell = \{ 0, \ldots, n \}$ and $\famev{\ell}(i) = x_i$, or equivalently $\ell = \family{0 \mto x_0, \ldots, n \mto x_n}$.
  %
  We will often use list notation to present concrete examples of families.
\end{example}

Working constructively, it is quite important to keep the distinction between families and subsets where classical treatments would confound them.
%
For instance, a propositional theory is usually classically defined as a \emph{set} of propositions; we would instead use a \emph{family} of propositions.
%
In a derivation over the theory, uses of axioms therefore end up “tagged” with elements of the index set of the theory, typically explaining how a certain proposition arises as an axiom (since the same proposition might occur as an instance of axiom schemes in multiple ways).
%
These record constructive content which may be needed for, say, interpreting axioms according to a proof by cases over the axiom schemes of the theory.

Our use of families where most traditional treatments use lists --- e.g.\ for specifying the argument types of a constructor --- is less mathematically significant.
%
It is partly to avoid baking in assumptions of finiteness or ordering where they are not required; but it is mostly motivated just by the formalisation, where families provide a more appropriate abstraction.


\begin{definition}
  \label{def:family-map}%
  A \defemph{map of families} $f : K \to L$ between families~$K$ and~$L$ on~$X$ is a map
  $f : \famindex K \to \famindex L$ such that $\famev{L} \circ f = \famev{K}$.
\end{definition}

We shall notate such a map as $\fammap{f(x)}{x \in \famindex K}$. Indeed, the notation $\fammap{f(x)}{x \in A}$ works for \emph{any} maps $f : A \to B$, as it is just an alternative way of writing $\lambda$-abstractions.

Families and their maps form a category $\Fam X$, which is precisely the slice category $\Set/X$.
%
A map $r : X \to Y$ yields a functorial action $\act{r} : \Fam X \to \Fam Y$ which takes $K \in \Fam X$ to the family $\act{r} K$ with $\famindex (\act{r} K) = \famindex K$ and $\famev{\act{r} K} = r \circ \famev{K}$. It is perhaps clearer to write down the action in terms of family comprehension: $\act{r} \famtuple{e_i}{i \in I} = \famtuple{r(e_i)}{i \in I}$.

\begin{definition}
  \label{def:family-map-over}%
  Given a function $r : X \to Y$ and families $K$, $L$ on $X$, $Y$ respectively, a \defemph{map $f : K \to L$ over $r$} is a map $f : \act{r} K \to L$; equivalently, a map $f : \famindex K \to \famindex L$ forming a commutative square over $r$.
\end{definition}


\subsection{Closure systems}

The general machinery of derivations as closure systems occurs throughout logic, and is independent of the specific syntax or judgements of the logical systems involved.

\begin{definition}
  \label{def:closure-rule}\label{def:closure-system}%
  A \defemph{closure rule $(P, c)$} on a set~$X$ consists of a family $P$ of elements in~$X$, its \defemph{premises}, and a \defemph{conclusion} $c \in X$.
  %
  A \defemph{closure system~$\csS$} on a set~$X$ is a family of closure rules on~$X$, where we respectively write $\premises R$ and $\conclusion R$ for the premises and the conclusion corresponding to a rule $R \in \csS$.
  %
  We write $\ClosureRule X$ and $\ClosureSystem X$ for the collections of closure rules and closure systems on~$X$, respectively.
\end{definition}

As is tradition, we display a closure rule with premises $[p_1,\ldots,p_n]$ and conclusion~$c$ as
%
\[
  \inferrule{p_1 \quad \cdots \quad p_n}{c}
\]
%

The constructions of closure rules and closure systems are evidently functorial in the ambient set.
%
A map $f : X \to Y$ sends a rule $R \in \ClosureRule X$ to the rule $\act{f} R \in \ClosureRule Y$ with $\premises (\act{f} R) \defeq \famtuple{f(p)}{p \in \premises R}$ and $\conclusion (\act{f} R) \defeq f(\conclusion R)$.
%
Similarly, a closure system $\csS$ on $X$ is taken to the closure system $\act{f} \csS$ on $Y$, defined by $\act{f} \csS \defeq \famtuple{\act{f} r}{r \in \csS}$.

\begin{definition}
  A \defemph{simple map} $\csS \to \csT$ between closure systems $\csS$ and $\csT$ on~$X$ is just a map between them as families.
  %
  More generally, a \defemph{simple map $\bar{f} : \csS \to \csT$ over $f : X \to Y$} from $\csS \in \ClosureSystem X$ to $\csT \in \ClosureSystem Y$
  is just a simple map $\bar{f} : \act{f}\csS \to \csT$, or equivalently a family map~$\bar{f}$ over $\act{f} : \ClosureRule X \to \ClosureRule Y$.
\end{definition}

A closure system yields a notion of derivation:

\begin{definition}
  \label{def:closure-system-derivation}%
  Given a closure system $\csS$ on $X$, a family $H$ of elements in $X$, and an element $c \in X$, the
  \defemph{derivations~$\derivation{\csS}{H}{c}$ of~$c$ from hypotheses~$H$} are inductively generated
  by:
  %
  \begin{enumerate}
  \item for every $h \in H$, there is a corresponding derivation $\hyp h \in \derivation{\csS}{H}{h}$,
  \item for every rule $R \in \csS$ and a map $D \in \prod_{p \in \premises R} \derivation{\csS}{H}{p}$ there is a derivation $\der{R}{D} \in \derivation{\csS}{H}{\conclusion R}$.
  \end{enumerate}
\end{definition}

In the second clause above $D$ is a dependent map, i.e., for each $p \in \premises R$ we have $D_p \in \derivation{\csS}{H}{p}$.
%
We do not shy away from using products of families and dependent maps when the situation demands them.

The elements of $\derivation{\csS}{H}{c}$ may be seen as well-founded trees with edges and nodes suitably labelled from $X$, $\csS$, and $H$.
%
We take such inductively generated families of sets as primitive;
%
their existence may be secured one way or another, depending on the ambient mathematical foundations.
%
The essential feature of derivations, which we rely on, is the structural induction principle they provide.

It is easy to check that derivations are functorial in simple maps of closure systems, in a suitable sense:
%
\begin{propositionwithqed}
  A simple map $\bar{f} : \csS_X \to \csS_Y$ of closure systems over $f : X \to Y$ acts on derivations as $\act{\bar{f}} : \derivation{\csS_X}{H}{c} \to \derivation{\csS_Y}{\act{f}H}{f(c)}$ for each $H$ and $c$.
  %
  The action is moreover functorial, in that $\act{\idmap} = \idmap$ and $\act{(\bar{f} \circ \bar{g})} = \act{\bar{f}} \circ \act{\bar{g}}$.
\end{propositionwithqed}

Often, one wants a more general notion of map, sending each rule of the source system not necessarily to a single rule of the target system, but instead to a \emph{derived} rule:

\begin{definition}
  \label{def:derivation-grafting}%
  A \defemph{derivation of a rule} $R$ over a closure system $\csS$ is a derivation of $\conclusion R$ from $\premises R$ over $\csS$.
  %
  Given such a derivation, we call $R$ a \defemph{derived rule} of $\csS$, or say $R$ is \defemph{derivable over $\csS$}.
  %
  A \defemph{map of closure systems} $\bar{f} : \csS \to \csT$ over $f : X \to Y$ is a function giving, for each rule $R$ of $C$, a derivation of $\act{f}R$ in $\csT$.
\end{definition}

To show that maps of closure systems preserve derivability, we need a \emph{grafting} operation on derivations.

\begin{lemma}
  \label{lem:hypotheses-grafting}
  %
  Given an ambient closure system $\csS$, suppose $D$ is a derivation of $c$ from hypotheses~$H$ over $\csS$, and for each $h \in H$, $D_h$ is a derivation of $h$ from $H'$.
  %
  Then there is a derivation of $c$ from $H'$ over $\csS$.
\end{lemma}

\begin{proof}
  The derivation of~$c$ from~$H'$ is constructed inductively from a derivation of~$c$ from~$H$:
  %
  \begin{enumerate}
  \item if $c$ is derived as one of the hypotheses $h \in H$, then $D_h$ derives $c$ from $H'$,
  \item if $\der{R}{D'}$ derives $c$ from $H$, then for each $p \in \premises R$ we inductively obtain a derivation $D''_p$ of $p$ from $H'$ from the corresponding derivation~$D'_p$ of $p$ from $H$, and assemble these into the derivation $\der{R}{D''}$ of~$c$ from~$H'$. \qedhere
  \end{enumerate}
\end{proof}

\begin{definition}
  A closure system map $\bar{f} : \csS \to \csT$ over $f : X \to Y$ \defemph{acts on derivations}: if $D$ is a derivation of $c$ from $H$ over $\csS$, there is a derivation $\act{\bar{f}}D$ of $f(c)$ from $\act{f}H$ over $\csT$.
\end{definition}

\begin{proof}
  $\act{\bar{f}}D$ is defined by recursion on $D$.
  %
  Wherever $D$ uses a rule $R$ of $\csS$, with derivations $D_h$ of the premises, $\act{\bar{f}}D$ uses the given derivation $\bar{f}(R)$ of $\act{f}R$, with the derivations $\act{\bar{f}}D_h$ grafted in at the hypotheses.
\end{proof}

Categorically, grafting can be recognised as the multiplication operation of a monad structure on derivations, and our maps of closure systems can be seen as Kleisli maps for this monad (relative to simple maps).
%
One can thus show that they form a category, that the action on derivations is functorial, and so on.
%
We do not make this precise here, as it is not required for the present paper.

\subsection{Well-founded orders}

There will be several occasions when we shall have to prevent dependency cycles (between premises of a rule, or between rules of a type theory). For this purpose we review a notion of well-foundedness which accomplishes the task.

\begin{definition}
  \label{def:well-founded-order}%
  A \defemph{strict partial order} on a set $A$ is an irreflexive and transitive relation~$<$ on~$A$.
  %
  A subset $S \subseteq A$ is \defemph{$<$-progressive} when, for all $x \in A$,
  %
  \begin{equation*}
    (\all{y \in A} y < x \lthen y \in S) \lthen x \in S.
  \end{equation*}
  %
  A \defemph{well-founded order} is a strict partial order~$<$ in which a subset is the entire set as soon as it is $<$-progressive.
  %
  For each $x \in A$, the \defemph{initial segment} $\initialSegment{i} \defeq \set{y \in A \such y < x}$ is the set of elements preceding~$x$ with respect to the order.
\end{definition}

\noindent
In terms of an induction principle a strict partial order is well-founded when, for every predicate~$\varphi$ on~$A$,
%
\begin{equation*}
  \all{x \in A}{
    (\all{y \in A} y < x \lthen \varphi(y)) \lthen \varphi(x)
  } \lthen
  \all{x \in A}{\varphi(x)}.
\end{equation*}
%
Classically there are many equivalent definitions of well-founded orders.
%
Constructively, the situation is more complicated, cf.\ \cite[\textsection 2.5]{taylor:practical-foundations}; this definition is one of the most standard, and the most suited to our purpose. 


%%% Local Variables:
%%% mode: latex
%%% TeX-master: "general-type-theories"
%%% End:
