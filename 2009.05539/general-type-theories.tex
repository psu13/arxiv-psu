\documentclass[12pt]{article}

\usepackage[colorlinks=true
,breaklinks=true
,urlcolor=blue
,anchorcolor=blue
,citecolor=blue
,filecolor=blue
,linkcolor=blue
,menucolor=blue
,linktocpage=true]{hyperref}
\hypersetup{
bookmarksopen=true,
bookmarksnumbered=true,
bookmarksopenlevel=10
}



%%%%%%%%%%%%%%%%%%%%%%%%%%%%%%%%
%   preamble.tex                 %
%%%%%%%%%%%%%%%%%%%%%%%%%%%%%%%%%
%
\hyphenation{Mars-den} \hyphenation{co-isotropic}
% style
%
%\pagestyle{headings} 
%\newtheorem{Definition}{Definition}[section]
%\newtheorem{Lemma}[Definition]{Lemma}
%\newtheorem{Theorem}[Definition]{Theorem}
%\newtheorem{Proposition}[Definition]{Proposition}
%\newtheorem{Example}[Definition]{Example}
%\newtheorem{Corollary}[Definition]{Corollary}
%\newtheorem{Construction}[Definition]{Construction}
%\topmargin = 2 cm
%\oddsidemargin = 3 cm
%\evensidemargin = 3 cm
\setcounter{secnumdepth}{5}
%
% general commands
%
%\newcommand{\ac}{\addtocounter} \newcommand{\setc}Ftcounter}
\newcommand{\se}{\section} \newcommand{\su}{\subsection}
\newcommand{\susu}{\subsubsection} \newcommand{\pa}{\paragraph}
\newcommand{\supa}{\subparagraph}
\newcommand{\eo}{\setcounter{equation}{0}}
%\renewcommand{\theequation}{\thesection.\arabic{equation}}
\renewcommand{\ll}{\label} \newcommand{\beq}{\begin{equation}}
\newcommand{\eeq}{\end{equation}} 
\newcommand{\bea}{\begin{eqnarray}}
\newcommand{\eea}{\end{eqnarray}} \newcommand{\nn}{\nonumber}
\newcommand{\bib}{\bibitem} \newcommand{\ci}{\cite}
\newcommand{\bll}{\\ \mbox{}\hfill \\} \newcommand{\fn}{\footnote}
%\renewcommand{\sp}{\samepage}
%
% text abbreviations
%
\newcommand{\lac}{\circlearrowright}
\newcommand{\rac}{\circlearrowleft}
\newcommand{\tps}{transition probability space}
\newcommand{\tp}{transition probability}
\newcommand{\tpies}{transition probabilities} \newcommand{\qm}{quantum
mechanics} \newcommand{\clm}{classical mechanics}
\newcommand{\ca}{$C^*$-algebra} \newcommand{\jba}{JB-algebra}
\newcommand{\jlba}{JLB-algebra} \newcommand{\rep}{representation}
\newcommand{\irrep}{irreducible representation}
\newcommand{\Hs}{Hilbert space} \newcommand{\Bs}{Banach space}
\newcommand{\BA}{Banach algebra} \newcommand{\mom}{momentum map}
\newcommand{\sta}{$\mbox{}^*$-algebra}
\newcommand{\sthom}{$\mbox{}^*$-homomorphism}
\newcommand{\staut}{$\mbox{}^*$-automorphism}
\newcommand{\stiso}{$\mbox{}^*$-isomorphism}
\newcommand{\MW}{Marsden--Weinstein} \newcommand{\HCM}{Hilbert
$C^*$-module} \newcommand{\vN}{von Neumann} \newcommand{\vna}{von
Neumann algebra}
\newcommand{\op}{^{\mbox{\tiny op}}}
%
% single mathsymbols
%
\newcommand{\pl}{\hbar}
\newcommand{\id}{\mbox{\rm id}}
\newcommand{\sqre}{{\mathchoice\sqr{7}4\sqr{7}4\sqr{6}3\sqr{6}3}}
\newcommand{\pdb}{\overline{\partial}}
\newcommand{\partialb}{\overline{\partial}}
\newcommand{\zb}{\overline{z}} \newcommand{\ovl}{\overline}
\newcommand{\wt}{\widetilde} \newcommand{\til}{\tilde}
\newcommand{\raw}{\rightarrow} \newcommand{\rat}{\mapsto}
\newcommand{\law}{\leftarrow} \newcommand{\Raw}{\Rightarrow}
\newcommand{\hraw}{\hookrightarrow} \newcommand{\Law}{\Leftarrow}
\newcommand{\lraw}{\leftrightarrow}
\newcommand{\LRaw}{\Leftrightarrow}
\newcommand{\rlh}{\rightleftharpoons} \newcommand{\n}{\|}
\newcommand{\ot}{\otimes} 
\newcommand{\la}{\langle} \newcommand{\ra}{\rangle}
\newcommand{\na}{\bigtriangledown} \newcommand{\wed}{\wedge}
\renewcommand{\Re}{{\rm Re}\,} \renewcommand{\Im}{{\rm Im}\,}
\newcommand{\rst}{\upharpoonright} \newcommand{\cov}{\nabla}
\newcommand{\x}{\times} \newcommand{\hb}{\hbar}
\newcommand{\lti}{\ltimes} \newcommand{\ran}{{\rm ran}}
% 
% composite math symbols
%
\newcommand{\Tr}{\mbox{\rm Tr}\,} 
\DeclareMathOperator{\Ad}{Ad}
\newcommand{\Co}{{\rm Co}} \newcommand{\ad}{{\rm ad}}
\newcommand{\co}{{\rm co}} \newcommand{\Ci}{{\rm Ci}}
\newcommand{\Is}{{\rm Is}} \newcommand{\sn}{\parallel_{\infty}}
\newcommand{\KH}{\mathfrak{B}_0({\mathcal H})}
 \newcommand{\BH}{\mathcal{B}({\mathcal H})} \newcommand{\diri}{\int^{\oplus}}
\newcommand{\cin}{C^{\infty}} \newcommand{\cci}{C^{\infty}_c}
\newcommand{\half}{\mbox{\footnotesize $\frac{1}{2}$}}
\newcommand{\third}{\mbox{\footnotesize $\frac{1}{3}$}}
\newcommand{\quar}{\mbox{\footnotesize $\frac{1}{4}$}}
\newcommand{\sixth}{\mbox{\footnotesize $\frac{1}{6}$}}
\newcommand{\eighth}{\mbox{\footnotesize $\frac{1}{8}$}}
\newcommand{\twelfth}{\mbox{\footnotesize $\frac{1}{12}$}}
\newcommand{\eb}{\partial_e K} \newcommand{\Hh}{{\mathcal H}_{\hbar}}
\newcommand{\PHh}{{\mathbb P}{\mathcal H}_{\hbar}} \newcommand{\PH}{{\mathbb
P}{\mathcal H}} \newcommand{\SH}{{\mathbb S}{\mathcal H}}
\newcommand{\AOOP}{\mathfrak{A}^{00}_{\R}({\mathcal P})}
\newcommand{\AOP}{\mathfrak{A}^0_{\R}({\mathcal P})}
\newcommand{\TAP}{\til{\mathfrak{A}}({\mathcal P})} 
\newcommand{\lp}{{\mathcal L}({\mathcal P})}
\newcommand{\Ah}{\mathfrak{A}^{\hbar}} \newcommand{\Ar}{\mathfrak{A}_{\mathbb
R}} \newcommand{\Br}{\mathfrak{B}_{\mathbb R}} 
 \newcommand{\Hlg}{{\mathcal H}_{\chi}}
\newcommand{\Hug}{{\mathcal H}^{\chi}} \newcommand{\plg}{\pi_{\chi}}
\newcommand{\pug}{\pi^{\chi}} \newcommand{\Ulg}{U_{\chi}}
\newcommand{\Uug}{U^{\chi}} \newcommand{\PA}{{\mathcal P}(\mathfrak{A})}
\newcommand{\SA}{{\mathcal S}(\mathfrak{A})} \newcommand{\LP}{{\mathcal L}({\mathcal
P})} \newcommand{\q}{{\mathcal Q}_{\hbar}}
\newcommand{\lho}{\lim_{\hbar\rightarrow 0}} \newcommand{\tsr}{T^*\mathbb
R^n} \newcommand{\tr}{T\mathbb R^n} \newcommand{\qw}{{\mathcal Q}_{\hbar}^W}
\newcommand{\qp}{\CQ_{\hbar}^{\mbox{\tiny pos}}}
\newcommand{\qh}{q_{\hbar}} \newcommand{\sgh}{\sigma_{\hbar}}
\newcommand{\rhh}{\rho_{\hbar}} \newcommand{\psh}{\psi_{\hbar}}
\newcommand{\phh}{varphi_{\hbar}} \newcommand{\qb}{{\mathcal
Q}_{\hbar}^{B}} \newcommand{\lt}{L^2(\mathbb R^n)}
\newcommand{\klt}{\mathfrak{B}_0(L^2(\mathbb R^n))} 
 \newcommand{\hn}{\mathfrak{h}_n}
\newcommand{\Kh}{{\mathcal K}_{\hbar}}
\newcommand{\upq}{U_{\frac{1}{\hbar}}} \newcommand{\qd}{\dot{q}}
\DeclareMathOperator{\supp}{supp}
\newcommand{\ddt}{\frac{d}{dt}}
\newcommand{\tio}{_{|t=0}} \newcommand{\Ug}{{\mathcal U}(\mathfrak{g}_{\mathbb
C})} \newcommand{\Ugg}{{\mathcal U}_{\Gm}(\mathfrak{g}_{\mathbb C})}
\newcommand{\Ugh}{{\mathcal U}_{\Gm/\hbar}(\mathfrak{g}_{\mathbb C})}
\newcommand{\Agg}{\mathfrak{A}^{\hbar}_{\Gm}(\mathfrak{g})}
\newcommand{\inv}{^{-1}} \newcommand{\sa}{_{\R}}
\newcommand{\Exp}{{\rm Exp}} 
%\newcommand{\SE}{\sqrt{{\rm Exp}}}
\newcommand{\tih}{\times_H} 
\newcommand{\gp}{\mathfrak{g}^{\ta}_\mathsf{P}} \newcommand{\AutP}{{\rm
Aut}(\mathsf{P})} \newcommand{\AutG}{{\rm Aut}(G)}
\newcommand{\DiffQ}{{\rm Diff}(Q)} 
\newcommand{\SHG}{\mathsf{H}^{\ch}} \newcommand{\HG}{{\mathcal H}^{\ch}}
\newcommand{\Hg}{{\mathcal H}_{\ch}} \newcommand{\POA}{T^*\mathsf{P}^{\mathcal
O}_\mathbf{A}} \newcommand{\PO}{T^*\mathsf{P}^{\mathcal O}}
\newcommand{\daw}{\stackrel{\leftarrow}{\Rightarrow}}
\newcommand{\QQ}{Q\times Q}
\newcommand{\aaw}{\stackrel{\rightarrow}{\rightarrow}
\stackrel{\mbox{\tiny $TQ$}}{\mbox{\tiny $Q$}}}
\newcommand{\CPW}{C^{\infty}_{\mbox{\tiny PW}}}
\newcommand{\bb}{\rangle_{\mathfrak{B}}}
 \newcommand{\ba}{\rangle_{\mathfrak{A}}}
\newcommand{\Meq}{\stackrel{M}{\sim}}
\newcommand{\pir}{\pi_{\mbox{\tiny R}}}
\newcommand{\pil}{\pi_{\mbox{\tiny L}}}
 \newcommand{\DA}{\Delta(\mathfrak{A})}
\newcommand{\er}{\eqref}
%\newcommand{\Co}{C^*(A,{\mathbb I})}
%
% Greek  
%
\newcommand{\al}{\alpha} \newcommand{\bt}{\beta}
\newcommand{\gm}{\gamma} \newcommand{\Gm}{\Gamma}
\newcommand{\dl}{\delta} \newcommand{\Dl}{\Delta}
\newcommand{\ep}{\epsilon} \newcommand{\varep}{\varepsilon}
\newcommand{\zt}{\zeta} \newcommand{\et}{\eta}
 \newcommand{\vth}{\vartheta}
\newcommand{\io}{\iota} \newcommand{\kp}{\kappa}
\newcommand{\lm}{\lambda} \newcommand{\Lm}{\Lambda}
\newcommand{\rh}{\rho} \newcommand{\sg}{\sigma}
\newcommand{\Sg}{\Sigma} \newcommand{\ta}{\tau} \newcommand{\ph}{\phi}
\newcommand{\Ph}{\Phi} \newcommand{\phv}{\varphi}
\newcommand{\ch}{\chi} \newcommand{\ps}{\psi} \newcommand{\Ps}{\Psi}
\newcommand{\om}{\omega} \newcommand{\Om}{\Omega}
%\newcommand{\Up}{\Upsilon}
%
% German
%
\newcommand{\A}{\mathfrak{A}} \newcommand{\B}{\mathfrak{B}}
\newcommand{\GC}{\mathfrak{C}} \newcommand{\GE}{\mathfrak{E}}
\newcommand{\GF}{\mathfrak{F}} \newcommand{\GG}{\mathfrak{G}}
\newcommand{\GI}{\mathfrak{I}} \newcommand{\K}{\mathbb{K}}
\newcommand{\GM}{\mathfrak{M}} \newcommand{\GN}{\mathfrak{N}}
\newcommand{\GU}{\mathfrak{U}} \newcommand{\GW}{\mathfrak{W}}
\newcommand{\GS}{\mathfrak{S}} \newcommand{\g}{\mathfrak{g}}
\newcommand{\GQ}{\mathfrak{Q}} 
\newcommand{\h}{\mathfrak{h}} 
\newcommand{\m}{\mathfrak{m}} \newcommand{\Gl}{\mathfrak{l}}
\newcommand{\GP}{\mathfrak{P}} \newcommand{\s}{\mathfrak{s}}
\renewcommand{\t}{\mathfrak{t}}
%
% Calligraphic
%
\newcommand{\CA}{{\mathcal A}} \newcommand{\CB}{{\mathcal B}}
\newcommand{\CC}{{\mathcal C}} \newcommand{\CF}{{\mathcal F}}
%\newcommand{\CD}{{\mathcal D}} 
\newcommand{\CE}{{\mathcal E}}
\newcommand{\CG}{{\mathcal G}} \renewcommand{\H}{{\mathcal H}}
\newcommand{\CJ}{{\mathcal J}} \newcommand{\CI}{{\mathcal I}}
\newcommand{\CK}{{\mathcal K}}   \newcommand{\CL}{{\mathcal L}}   
 \newcommand{\CM}{{\mathcal M}}
\newcommand{\CN}{{\mathcal N}} \newcommand{\CS}{{\mathcal S}}
\newcommand{\CO}{{\mathcal O}} \newcommand{\CP}{{\mathcal P}}
\newcommand{\CQ}{{\mathcal Q}} \newcommand{\CR}{{\mathcal R}}
\newcommand{\CT}{{\mathcal T}} \newcommand{\CV}{{\mathcal V}}
\newcommand{\CW}{{\mathcal W}} \renewcommand{\P}{{\mathcal P}}
\renewcommand{\L}{\label}
\newcommand{\CZ}{{\mathcal Z}}
%
% blackboard
%
\newcommand{\C}{{\mathbb C}} \newcommand{\D}{{\mathbb D}}
\newcommand{\BP}{{\mathbb P}} \newcommand{\I}{{\mathbb I}}
\newcommand{\N}{{\mathbb N}} \newcommand{\R}{{\mathbb R}}
\newcommand{\T}{{\mathbb T}} \newcommand{\Z}{{\mathbb Z}}
%
% sans serif
%
\newcommand{\SSB}{\mathsf{B}} \newcommand{\SG}{\mathsf{G}}
\newcommand{\SSH}{\mathsf{H}} \newcommand{\SU}{\mathsf{U}}
\newcommand{\SSP}{\mathsf{P}} \newcommand{\SSp}{\mathsf{p}}
\newcommand{\SSV}{\mathsf{V}} \newcommand{\SM}{\mathsf{M}}
\newcommand{\SSS}{\mathsf{S}} %
% bold
%
\newcommand{\bofe}{\mathbf{e}} \newcommand{\bofu}{\mathbf{u}}
\newcommand{\bofp}{\mathbf{p}} \newcommand{\bfp}{\mathbf{p}}
\newcommand{\bfe}{\mathbf{e}} \newcommand{\bfu}{\mathbf{u}}
\newcommand{\bg}{\mathbf{g}} \newcommand{\bR}{\mathbf{R}}
\newcommand{\bA}{\mathbf{A}} \newcommand{\bF}{\mathbf{F}} \makeatletter
\newskip\tempskip \def\endproof{{\parfillskip24\p@ plus\@ne
fil\@@par}\tempskip\prevdepth
\ifdim\lastskip=\z@\tempskip\z@\else\vskip-\lastskip
\ifdim\tempskip>4\p@ \tempskip.5\tempskip \else \tempskip\z@\fi\fi
\nobreak\vskip-\baselineskip\vskip-\tempskip\noindent\hbox
to\hsize{\hfill
$\blacksquare$}\par\vskip\tempskip\vskip\abovedisplayskip\@doendpe}
\makeatother \makeatletter
\newskip\tempskip \def\endiproof{{\parfillskip24\p@ plus\@ne
fil\@@par}\tempskip\prevdepth
\ifdim\lastskip=\z@\tempskip\z@\else\vskip-\lastskip
\ifdim\tempskip>4\p@ \tempskip.5\tempskip \else \tempskip\z@\fi\fi
\nobreak\vskip-\baselineskip\vskip-\tempskip\noindent\hbox
to\hsize{\hfill
$\Box$}\par\vskip\tempskip\vskip\abovedisplayskip\@doendpe}
\makeatother \newcommand{\enp}{\endproof}
\newcommand{\eip}{\endiproof}
%%%%%%%%%%%%%%%%%%%%%%%%%%%%%%%%%%%%%%%%%%%%%%%%%%%%%%%%%%%%%%%%%%%%%%%%%%%
\newcommand{\LPo}{\mbox{\rm \textsf{LPoisson}}}
\newcommand{\otB}{\hat{\otimes}_{\mathfrak{B}}}
\newcommand{\otA}{\hat{\otimes}_{\mathfrak{A}}}
\newcommand{\otq}{\hat{\otimes}}
\newcommand{\otc}{\circledcirc}
\newcommand{\otg}{\circledast}
\newcommand{\had}{|\Lm|^{1/2}}
\newcommand{\Me}{Morita equivalent}
\newcommand{\siM}{\stackrel{M}{\sim}}
\newcommand{\siMs}{\stackrel{M}{\sim s}}
\newcommand{\Rep}{\mbox{\rm Rep}}
\newcommand{\End}{\mbox{\rm End}}
\newcommand{\Hom}{\mbox{\rm Hom}}
\newcommand{\Der}{\mbox{\rm Der}}
\newcommand{\sym}{\mbox{\rm sym}}
\newcommand{\Ri}{\mbox{\textsf{Rings}}}
\newcommand{\Ca}{\mbox{\textsf{C}$\mbox{}^*$}}
\newcommand{\Wa}{\mbox{\textsf{W}$\mbox{}^*$}}
\newcommand{\Gr}{\mbox{\textsf{G}}}
\newcommand{\Grb}{\mbox{\textsf{G'}}}
\newcommand{\LG}{\mbox{\textsf{LG}}}
%\newcommand{\MG}{\mbox{\textsf{MG}}}
\newcommand{\LGc}{\mbox{\textsf{LGc}}}
\newcommand{\SyG}{\mbox{\textsf{SG}}}
\newcommand{\Po}{\mbox{\textsf{Poisson}}}
\newcommand{\Alg}{\mbox{\textsf{Alg}}}
\newcommand{\LGt}{\mbox{\textsf{L}$\tilde{\mathsf{G}}$}}
 


% Macros shared among templates

\usepackage[utf8]{inputenc}

\usepackage{graphicx}
\setkeys{Gin}{width=\linewidth,totalheight=\textheight,keepaspectratio}

\definecolor{darkblue}{HTML}{00416A}

\usepackage{longtable}
\usepackage{booktabs}
\usepackage{amssymb}
\usepackage{amsmath}
\usepackage{amsthm}
%\usepackage{commath}
\usepackage{boxedminipage}
\usepackage{microtype}

\usepackage{makeidx}
\usepackage{hyperref}

% attempts to prevent margin figures from being cut off
\usepackage{marginfix}
\usepackage[morefloats=100]{morefloats}

\newtheorem{Definition}{Definition}
\newtheorem{Theorem}{Theorem}
\newtheorem{Lemma}{Lemma}
\newtheorem{Exercise}{Exercise}
\newtheorem{Fact}{Fact}
\newtheorem{Proposition}{Proposition}
\newtheorem{Assumption}{Assumption}
\newenvironment{Algorithm}{\begin{center}\begin{boxedminipage}{0.92\textwidth}}{\end{boxedminipage}\end{center}}
\newenvironment{Proof}{\begin{proof}}{\end{proof}}
\newtheorem*{summary*}{Summary}
\newenvironment{Summary}{\begin{center}\begin{minipage}{0.92\textwidth}\begin{summary*}}{\end{summary*}\end{minipage}\end{center}\medskip}
\newenvironment{EmphBox}{\begin{center}\begin{minipage}{0.8\textwidth}}{\end{minipage}\end{center}\medskip}

\providecommand{\tightlist}{%
  \setlength{\itemsep}{0pt}\setlength{\parskip}{0pt}}

\renewcommand{\bot}{\perp}
\renewcommand{\hat}{\widehat}



\usepackage[noBBpl,sc]{mathpazo}
\linespread{1.05}
\usepackage[papersize={6.9in, 10.0in}, left=.5in, right=.5in, top=1in, bottom=.9in]{geometry}
\sloppy
\raggedbottom
\usepackage[small]{titlesec}

% these include amsmath and that can cause trouble in older docs.
\input{../helpers/cmrsum}
\makeatletter

\DeclareFontFamily{OMX}{MnSymbolE}{}
\DeclareSymbolFont{largesymbolsX}{OMX}{MnSymbolE}{m}{n}
\DeclareFontShape{OMX}{MnSymbolE}{m}{n}{
    <-6>  MnSymbolE5
   <6-7>  MnSymbolE6
   <7-8>  MnSymbolE7
   <8-9>  MnSymbolE8
   <9-10> MnSymbolE9
  <10-12> MnSymbolE10
  <12->   MnSymbolE12}{}

\DeclareMathSymbol{\downbrace}    {\mathord}{largesymbolsX}{'251}
\DeclareMathSymbol{\downbraceg}   {\mathord}{largesymbolsX}{'252}
\DeclareMathSymbol{\downbracegg}  {\mathord}{largesymbolsX}{'253}
\DeclareMathSymbol{\downbraceggg} {\mathord}{largesymbolsX}{'254}
\DeclareMathSymbol{\downbracegggg}{\mathord}{largesymbolsX}{'255}
\DeclareMathSymbol{\upbrace}      {\mathord}{largesymbolsX}{'256}
\DeclareMathSymbol{\upbraceg}     {\mathord}{largesymbolsX}{'257}
\DeclareMathSymbol{\upbracegg}    {\mathord}{largesymbolsX}{'260}
\DeclareMathSymbol{\upbraceggg}   {\mathord}{largesymbolsX}{'261}
\DeclareMathSymbol{\upbracegggg}  {\mathord}{largesymbolsX}{'262}
\DeclareMathSymbol{\braceld}      {\mathord}{largesymbolsX}{'263}
\DeclareMathSymbol{\bracelu}      {\mathord}{largesymbolsX}{'264}
\DeclareMathSymbol{\bracerd}      {\mathord}{largesymbolsX}{'265}
\DeclareMathSymbol{\braceru}      {\mathord}{largesymbolsX}{'266}
\DeclareMathSymbol{\bracemd}      {\mathord}{largesymbolsX}{'267}
\DeclareMathSymbol{\bracemu}      {\mathord}{largesymbolsX}{'270}
\DeclareMathSymbol{\bracemid}     {\mathord}{largesymbolsX}{'271}

\def\horiz@expandable#1#2#3#4#5#6#7#8{%
  \@mathmeasure\z@#7{#8}%
  \@tempdima=\wd\z@
  \@mathmeasure\z@#7{#1}%
  \ifdim\noexpand\wd\z@>\@tempdima
    $\m@th#7#1$%
  \else
    \@mathmeasure\z@#7{#2}%
    \ifdim\noexpand\wd\z@>\@tempdima
      $\m@th#7#2$%
    \else
      \@mathmeasure\z@#7{#3}%
      \ifdim\noexpand\wd\z@>\@tempdima
        $\m@th#7#3$%
      \else
        \@mathmeasure\z@#7{#4}%
        \ifdim\noexpand\wd\z@>\@tempdima
          $\m@th#7#4$%
        \else
          \@mathmeasure\z@#7{#5}%
          \ifdim\noexpand\wd\z@>\@tempdima
            $\m@th#7#5$%
          \else
           #6#7%
          \fi
        \fi
      \fi
    \fi
  \fi}

\def\overbrace@expandable#1#2#3{\vbox{\m@th\ialign{##\crcr
  #1#2{#3}\crcr\noalign{\kern2\p@\nointerlineskip}%
  $\m@th\hfil#2#3\hfil$\crcr}}}
\def\underbrace@expandable#1#2#3{\vtop{\m@th\ialign{##\crcr
  $\m@th\hfil#2#3\hfil$\crcr
  \noalign{\kern2\p@\nointerlineskip}%
  #1#2{#3}\crcr}}}

\def\overbrace@#1#2#3{\vbox{\m@th\ialign{##\crcr
  #1#2\crcr\noalign{\kern2\p@\nointerlineskip}%
  $\m@th\hfil#2#3\hfil$\crcr}}}
\def\underbrace@#1#2#3{\vtop{\m@th\ialign{##\crcr
  $\m@th\hfil#2#3\hfil$\crcr
  \noalign{\kern2\p@\nointerlineskip}%
  #1#2\crcr}}}

\def\bracefill@#1#2#3#4#5{$\m@th#5#1\leaders\hbox{$#4$}\hfill#2\leaders\hbox{$#4$}\hfill#3$}

\def\downbracefill@{\bracefill@\braceld\bracemd\bracerd\bracemid}
\def\upbracefill@{\bracefill@\bracelu\bracemu\braceru\bracemid}

\DeclareRobustCommand{\downbracefill}{\downbracefill@\textstyle}
\DeclareRobustCommand{\upbracefill}{\upbracefill@\textstyle}

\def\upbrace@expandable{%
  \horiz@expandable
    \upbrace
    \upbraceg
    \upbracegg
    \upbraceggg
    \upbracegggg
    \upbracefill@}
\def\downbrace@expandable{%
  \horiz@expandable
    \downbrace
    \downbraceg
    \downbracegg
    \downbraceggg
    \downbracegggg
    \downbracefill@}

\DeclareRobustCommand{\overbrace}[1]{\mathop{\mathpalette{\overbrace@expandable\downbrace@expandable}{#1}}\limits}
\DeclareRobustCommand{\underbrace}[1]{\mathop{\mathpalette{\underbrace@expandable\upbrace@expandable}{#1}}\limits}

\makeatother

\begin{document}

\title{A general definition of dependent type theories}

\author{Andrej Bauer
  \and Philipp G.~Haselwarter
  \and Peter LeFanu Lumsdaine}

\date{September 11, 2020}

\maketitle

\extrafootertext{The authors benefited greatly from visits funded by the COST Action \href{https://eutypes.cs.ru.nl}{\emph{EUTypes}} CA15123.}
\extrafootertext{This material is based upon work supported by the U.S.~Air Force Office of Scientific Research under award number FA9550-17-1-0326, grant number 12595060.}


\begin{abstract}
  We define a general class of dependent type theories, encompassing Martin-Löf’s intuitionistic type theories and variants and extensions.
  %
  The primary aim is pragmatic: to unify and organise their study, allowing results and constructions to be given in reasonable generality, rather than just for specific theories.
  %
  Compared to other approaches, our definition stays closer to the direct or naïve reading of syntax, yielding the traditional presentations of specific theories as closely as possible.
  
  Specifically, we give three main definitions: \emph{raw type theories}, a minimal setup for discussing dependently typed derivability; \emph{acceptable type theories}, including extra conditions ensuring well-behavedness; and \emph{well-presented type theories}, generalising how in traditional presentations, the well-behavedness of a type theory is established step by step as the type theory is built up.
  %
  Following these, we show that various fundamental fitness-for-purpose metatheorems hold in this generality.

  Much of the present work has been formalised in the proof assistant Coq.
\end{abstract}

\newpage
\tableofcontents
\newpage

\newpage


\section{Introduction \label{sec:introduction}}

When probed at very short wavelengths, QCD is essentially a theory of
free \index{Partons}`partons' --- quarks and gluons --- which only
scatter off one another through relatively small quantum corrections,
that can be systematically calculated. 
But at longer wavelengths, of order the size of the proton $\sim
1\mathrm{fm} = 10^{-15}\mathrm{m}$,  
we see strongly bound towers of hadron resonances emerge, with string-like
potentials building up if we try to separate their partonic
constituents. Due to our
inability to perform analytic calculations in 
strongly coupled field theories, QCD is therefore 
still only partially solved. Nonetheless,  all its features, across all
distance scales, are believed to be encoded in a single one-line
formula of alluring simplicity; the
\index{QCD!Lagrangian}%
Lagrangian\footnote{Throughout these notes we let it be implicit that
  ``Lagrangian'' really refers to Lagrangian density, ${\cal L}$, the
  four-dimensional space-time integral of which is the action.} of QCD.

The consequence for collider physics is that some parts of QCD can be
calculated in terms of the fundamental parameters of the Lagrangian,
whereas others must be expressed through models or functions whose effective 
parameters are not a priori calculable but which can be constrained
by fits to data. 
However, even in the absence of a
perturbative expansion, there are still several strong theorems which
hold, and which can be used to give relations between seemingly
different processes. (This is, e.g., the reason it makes sense to 
measure the partonic substructure of the proton in $ep$ collisions and
then re-use the same parametrisations for $pp$
collisions.) Thus, in the chapters 
dealing with phenomenological models we shall emphasise that the loss
of a factorised perturbative expansion is not equivalent to a total
loss of predictivity.   

An alternative approach would be to give up on calculating QCD 
and use leptons instead. Formally, this amounts to summing inclusively over
strong-interaction phenomena, when such are present. While such a
strategy might succeed in replacing what we do know about QCD by
``unity'', however, even the most adamant chromophobe would acknowledge
the following basic facts of collider physics for the next decade(s): 
1) At the LHC, the initial states are
hadrons, and hence, at
the very least, well-understood and precise parton distribution
functions (PDFs) will be required; 2) high precision will mandate
 calculations to higher orders in perturbation theory, 
which in turn will involve more QCD; 3) the requirement of lepton
\emph{isolation} makes the very definition of a lepton
 depend implicitly on QCD and 4) 
 the rate of jets that are misreconstructed as leptons in
 the experiment depends explicitly on it. 
And, 5) though many new-physics signals \emph{do} give observable
signals in the lepton sector, this is far from guaranteed, nor is it
exclusive when it occurs. 
 It would therefore be  unwise not to attempt to solve QCD to the best
 of our ability, the better to prepare ourselves for both the largest
 possible discovery reach and the highest attainable subsequent
 precision. 

Furthermore, QCD is the richest gauge theory we have so far
 encountered. Its emergent phenomena, unitarity properties, colour structure, 
 non-perturbative dynamics, quantum vs.\ classical limits, 
interplay between scale-invariant and
 scale-dependent properties, and its wide
 range of phenomenological applications, are still very much topics of
 active investigation, about which we continue to learn.  

In addition, or perhaps as a consequence, the field of QCD is
currently experiencing something of a revolution. On the perturbative
side, new methods to compute scattering amplitudes with very high
particle multiplicities are being developed, together with advanced
techniques for combining such amplitudes with all-orders resummation
frameworks. On the non-perturbative side, the wealth of data on
soft-physics processes from the LHC is
forcing us to reconsider the reliability of the standard fragmentation
models, and heavy-ion collisions are providing new insights into
the collective behavior of hadronic matter. The
study of cosmic rays impinging on the Earth's
atmosphere challenges our ability to extrapolate fragmentation models
from collider energy scales to the region of ultra-high energy cosmic
rays. And finally, dark-matter annihilation processes in space  may produce 
hadrons, whose spectra are sensitive to the modeling 
of fragmentation.

In the following, we shall focus on QCD for mainstream 
collider physics. This includes the basics of SU(3), colour factors, the running
of $\alpha_s$, factorisation, 
hard processes, infrared safety, parton showers and matching, event generators, hadronisation, and the so-called underlying event. 
While not covering everything, hopefully these topics can also serve
at least as stepping stones to more specialised
issues that have been left out, such as twistor-inspired techniques, 
heavy flavours, polarisation, or forward physics, or to topics more tangential to
other fields, such as axions, lattice QCD, or heavy-ion physics.  

\subsection{A First Hint of Colour}
Looking for new physics, as we do now at the LHC, it is instructive to 
consider the story of the discovery of colour. The first hint was
arguably the $\Delta^{++}$ \index{Baryons}baryon, discovered in 
1951~\cite{Brueckner:1952zz}. The title and part of the abstract from this
historical paper are reproduced in \figRef{fig:Delta}.
\begin{figure}[t]
\begin{center}
\begin{tabular}{c}
\colorbox{gray}{\includegraphics*[scale=0.75]{DeltaTitle.pdf}}\\[5mm]
\hspace*{2mm}\begin{minipage}{0.88\textwidth}
\small\sl  ``[...] It is concluded that the apparently anomalous features of the
scattering can be interpreted to be an indication of a resonant
meson-nucleon interaction corresponding to a nucleon isobar with spin
$\frac32$, isotopic spin $\frac32$, and with an excitation energy of
$277\,$MeV.''\\[1mm]
\end{minipage}
\end{tabular}
\caption{The title and part of the abstract of the 1951 paper
  \cite{Brueckner:1952zz} (published in 1952) in which the existence 
  of the $\Delta^{++}$ baryon was deduced, based on data from Sachs and
  Steinberger at Columbia~\cite{Chedester:1951sc}  and from Anderson,
  Fermi, Nagle, et al.~at Chicago~\cite{Fermi:1952zz}. Further studies 
  at Chicago were quickly performed
  in~\cite{Anderson:1952nw,Anderson:1952zza}. See also the memoir by
  Nagle~\cite{nagle1984delta}. 
\label{fig:Delta}}  
\end{center}
\end{figure}
In the context of the \index{Quarks}quark model --- which first
had to be developed, successively joining together the notions of 
spin, isospin, strangeness, and 
the \index{Eightfold way}eightfold way\footnote{In physics, the ``eightfold way''
refers to the classification of the lowest-lying pseudoscalar
\index{Mesons}mesons and 
\index{SU(3)!Of Flavour}%
spin-1/2 \index{Baryons}baryons within \index{Octet}octets in SU(3)-flavour space ($u,d,s$). The
$\Delta^{++}$ is part of a spin-3/2 baryon \index{Decuplet}decuplet, a ``tenfold way'' in this
terminology.} 
--- the \index{Flavour}flavour and spin content of the $\Delta^{++}$
baryon is: 
\begin{equation}
\left\vert \Delta^{++} \right> = \left\vert
\,u_\uparrow\ u_\uparrow\ u_\uparrow \right>~,
\end{equation} 
clearly a highly symmetric configuration. However, since 
the $\Delta^{++}$ is a fermion, it must have an overall
antisymmetric wave function. In 1965, fourteen years after its
discovery, this was finally understood by the introduction of colour
\index{SU(3)}%
\index{SU(3)!Of Colour}%
as a new quantum number associated with the group SU(3)
\cite{Greenberg:1964pe,Han:1965pf}. The $\Delta^{++}$ wave function can now be made
antisymmetric by arranging its three quarks antisymmetrically 
in this new degree of freedom, 
\begin{equation}
\left\vert \Delta^{++} \right> = \epsilon^{ijk} \left\vert
\,u_{i\uparrow}\ u_{j\uparrow}\ u_{k\uparrow}\right>~,
\end{equation} 
hence solving the mystery.

More direct experimental tests of the number of colours were provided first by
measurements of the decay width of $\pi^0\to \gamma\gamma$ decays, which 
is proportional to $N_C^2$, 
and later by the famous ``R'' ratio in
$e^+e^-$ collisions ($R=\sigma(e^+e^-\to q\bar{q})/\sigma(e^+e^-\to
\mu^+\mu^-)$), which is proportional to $N_C$, see
e.g.~\cite{Dissertori:2003pj}. 
Below, in \SecRef{sec:L} we shall see how to
calculate such colour factors. 

\subsection{The Lagrangian of QCD \label{sec:L}}
\index{QCD!Lagrangian}%
Quantum Chromodynamics is based on the gauge group
\index{SU(3)}$\mrm{SU(3)}$, the 
Special Unitary group in 3 (complex) dimensions, whose elements 
are the set of unitary $3\times 3$ matrices with determinant one. 
\index{Fundamental representation}%
\index{SU(3)!Fundamental representation}%
Since there are 9 linearly independent unitary complex
matrices\footnote{A complex $N\times N$ matrix has $2N^2$ degrees of
  freedom, on which unitarity provides $N^2$ constraints.}, one of
which has determinant $-1$, there are a total of 8
independent directions in this matrix space, corresponding to eight
different generators as compared
with the single one of QED. In the context of QCD, we normally
represent this group using the 
so-called \emph{fundamental}, or \emph{defining}, representation, in
which the generators of $\mrm{SU(3)}$ appear as a set of eight traceless and
hermitean matrices, to which we return below.  
We shall refer to indices enumerating
the rows and columns of these matrices  (from 1 to 3) as
\emph{fundamental} indices, and we use the letters $i$,
$j$, $k$, \ldots, to denote them.
\index{Adjoint representation}%
\index{SU(3)!Adjoint representation}%
We refer to indices enumerating the generators (from 1 to 8),
as \emph{adjoint} 
indices\footnote{The dimension of the \emph{adjoint}, or
  \emph{vector}, representation is equal to the number of generators,
  $N^2-1=8$ for $\mrm{SU(3)}$, while the  
\index{Fundamental representation}%
\index{SU(3)!Fundamental representation}%
dimension of the fundamental representation is
  the degree of the group, $N=3$ for $\mrm{SU(3)}$.}, and we use the first
letters of the alphabet ($a$, $b$, $c$, \ldots) to denote them. 
These matrices can operate both on each other (representing
combinations of successive gauge transformations) and on a set of
$3$-vectors, the latter of 
which represent \index{Quarks}quarks in colour 
space; the quarks are \emph{triplets} under $\mrm{SU(3)}$. The matrices can be
thought of as representing gluons in colour 
space (or, more precisely, the gauge transformations carried out by
gluons), hence there are
eight different gluons; the gluons are \emph{octets} under $\mrm{SU(3)}$. 

\index{QCD!Lagrangian}%
The Lagrangian density of QCD is 
\begin{equation}
{\cal L} = \bar{\psi}_q^i(i\gamma^\mu)(D_\mu)_{ij}\psi_q^j - m_q
\bar{\psi}_q^i\psi_{qi} - \frac14 F^a_{\mu\nu}F^{a\mu\nu}~,\label{eq:L}
\end{equation}
where $\psi_q^i$ denotes a quark field with
(fundamental) colour index $i$, 
$\psi_q = ({\textcolor{red}{\psi_{qR}}},{\color{green}\psi_{qG}}, 
{\color{blue}\psi_{qB}})^T$, 
$\gamma^\mu$ is a Dirac matrix that expresses the
vector nature of the strong interaction, with $\mu$ being a Lorentz
vector index, $m_q$ allows for the
possibility of non-zero \index{Quarks}quark masses (induced by the
standard Higgs 
mechanism or similar), $F^a_{\mu\nu}$ is the gluon field strength 
tensor for a gluon\footnote{The definition of the gluon field strength
  tensor will be given below in \eqRef{eq:F}.} with (adjoint) 
colour index $a$ (i.e., $a\in[1,\ldots,8]$), 
and $D_\mu$ is the covariant derivative in QCD,
\begin{equation}
(D_{\mu})_{ij} = \delta_{ij}\partial_\mu - i g_s t_{ij}^a A_\mu^a~,\label{eq:D}
\end{equation}
\index{QCD!Coupling}
with $g_s$ the \index{alphaS@$\alpha_s$}strong coupling (related to
$\alpha_s$ by $g_s^2 = 4\pi 
\alpha_s$; we return to the strong coupling in more detail below), 
$A^a_\mu$  the gluon field with 
colour index $a$, and $t_{ij}^a$ proportional to the hermitean and
traceless \index{Gell-Mann matrices|see{SU(3)}}Gell-Mann matrices of $\mrm{SU(3)}$, 
\index{SU(3)!Generators}%
\begin{equation}
\mbox{\includegraphics*[scale=1.0]{gell-mann}}~.
\end{equation}
These generators are just the $\mrm{SU(3)}$ analogs of the
Pauli matrices in 
$\mrm{SU(2)}$. 
By convention, the constant of proportionality is normally
taken to 
be 
\begin{equation}
t^a_{ij} = \frac12 \lambda^a_{ij}~. \label{eq:t}
\end{equation}
\index{QCD!Coupling}
This choice in turn determines the normalisation of the coupling
$g_s$, via \eqRef{eq:D}, and
fixes the values of the $\mrm{SU(3)}$ \index{Casimirs}Casimirs and structure constants, to which we return below. 

An example of the colour flow for a
quark-gluon interaction in colour 
space is given in \figRef{fig:qg}.
\begin{figure}[t]
\begin{center}
\begin{minipage}[h]{4.6cm}
\begin{center}
$A^1_\mu$\\
\includegraphics*[scale=0.75]{qgv.pdf}\\[-3mm]
$\psi_{q\textcolor{green}{G}}$\hfill$\psi_{q\textcolor{red}{R}}$
\end{center}
\end{minipage}~~~
\parbox{0.4\textwidth}{
$
\begin{array}{ccccc}
\propto & - \frac{i}{2} g_s & \bar{\psi}_{q\color{red}R}  & \lambda^{1} & \psi_{q\color{green}G} 
\\[2mm]
= & -\frac{i}{2}g_s & \left(\begin{array}{ccc} \textcolor{red}{1} & \color{green} 0 &
  \color{blue} 0 
\end{array}\right) & 
\left(\begin{array}{ccc}
0 & 1 & 0  \\
1 & 0 & 0 \\
0 & 0 & 0
\end{array}\right) & 
 \left(\begin{array}{c}
\textcolor{red}{0} \\
\color{green}1 \\
\color{blue}0
\end{array}\right) \end{array}
$}
\caption{Illustration of a 
\index{Quarks}\index{Gluons}$qqg$ vertex in QCD, before
  summing/averaging over colours: a gluon in a state represented by $\lambda^1$
  interacts with quarks in the states $\psi_{qR}$ and
  $\psi_{qG}$. \label{fig:qg}}
\end{center}
\end{figure}
Normally, of course, we sum over all the colour indices, so this
example merely gives a pictorial representation of what one particular
(non-zero) term in the colour sum looks like.


\subsection{Colour Factors}
\index{QCD!Colour factors}
\index{Colour factors}%
\index{Colour-space indices|see{Colour connections}}%
\index{Matrix elements}%
Typically, we do not measure colour in the final state ---
instead we average over all possible incoming colours and sum over all
possible outgoing ones, wherefore QCD scattering amplitudes (squared) in
practice always contain sums over quark fields contracted with
\index{SU(3)!Generators}Gell-Mann matrices. These contractions in turn
produce traces  
which yield the \index{Colour factors}\emph{colour factors} that are associated to each QCD
process, and which basically count the number of ``paths through
colour space'' that the process at hand can take\footnote{The
  convention choice represented by \eqRef{eq:t} introduces a
  ``spurious'' factor of 2 for each power of the coupling $\alpha_s$. 
Although one could in principle absorb that factor into a redefinition
of the coupling, effectively redefining the normalisation of ``unit
colour charge'', the standard definition of $\alpha_s$ is now so
entrenched that alternative choices would be counter-productive, at
least in the context of a pedagogical review.}.

A very simple example of a colour factor is given by the decay process $Z\to
q\bar{q}$. This vertex contains a simple $\delta_{ij}$ in colour
space; the outgoing quark and antiquark must have identical 
(anti-)col\-ours. Squaring the corresponding matrix element and summing over
final-state colours yields a colour factor of
\begin{equation}
e^+e^-\to Z \to q\bar{q}~~~:~~~\sum_{\mrm{colours}}|M|^2 \propto
\delta_{ij}\delta_{ji} = \mrm{Tr}\{\delta\} = N_C = 3~,
\end{equation}
since $i$ and $j$ are quark (i.e., 3-dimensional
fundamental) indices. This factor corresponds directly to the 3 different
``paths through colour space'' that the process at hand can take; the
produced quarks can be red, green, or blue. 

A next-to-simplest example is given by $q\bar{q}\to
\gamma^*/Z\to\ell^+\ell^-$ (usually referred to as the
\index{Drell-Yan}Drell-Yan 
process~\cite{Drell:1970wh}),  
which is just a crossing of the previous one. By crossing
symmetry, the squared matrix element, including the colour factor, is
exactly the same as before, but since the quarks are here incoming, we
must \emph{average} rather than sum over their colours, leading to
\begin{equation}
q\bar{q}\to Z\to e^+e^-~~~:~~~\frac{1}{9}\sum_{\mrm{colours}}|M|^2 \propto \frac19\delta_{ij}\delta_{ji} = \frac19 \mrm{Tr}\{\delta\} = \frac13~,
\end{equation}
where the colour factor now expresses a \emph{suppression} which can
be interpreted as due to the fact that only quarks of matching colours
are able to collide and produce a $Z$ boson. The chance that a quark
and an antiquark picked at random from the colliding hadrons have 
matching colours is $1/N_C$. 
\begin{figure}[t]
\end{figure}

Similarly, $\ell q \to
\ell q$ via $t$-channel photon exchange (usually called Deep
Inelastic Scattering --- \index{DIS}\index{Deep inelastic scattering|see{DIS}}DIS --- with ``deep'' referring to a 
large virtuality of the exchanged photon), constitutes yet another
crossing of the same basic process, 
see \figRef{fig:Zcrossings}. \index{Colour factors}The colour factor in this case 
comes out as unity. 
\begin{figure}[t]
\centering\vspace*{-8mm}
\begin{tabular}{ccc}
\rotatebox{360}{\includegraphics*[scale=0.93]{ee2qq}} \ \ 
& \ \ \includegraphics*[scale=0.93,angle=180,origin=c]{ee2qq}
\ \ & \ \ \includegraphics*[scale=0.9,angle=297,origin=c]{ee2qq}\\
Hadronic $Z$ decay & \index{Drell-Yan}Drell-Yan & \index{DIS}DIS \\[1mm]
$e^-e^+ \to \gamma^*/Z^0 \to q\bar{q}$ &
$q\bar{q} \to \gamma^*/Z^0 \to \ell^+\ell^-$ &
$\ell \bar{q} \stackrel{\gamma^*/Z^*}{\to} \ell \bar{q}$
\\[2mm] 
$\propto N_C$ & $\propto 1/N_C$ & $\propto 1$
\end{tabular}
\caption{Illustration of the three crossings of the interaction of a
  lepton current (black) with a \index{Quarks}quark current (red) 
  via an intermediate photon or
  $Z$ boson, with corresponding colour factors. \label{fig:Zcrossings}}
\end{figure}

To illustrate what happens when we insert (and sum over)
quark-gluon
vertices, such as the one depicted in \figRef{fig:qg}, we take
the process $Z\to3\,$jets. \index{Colour factors}The colour factor for
this process can be 
computed as follows, with the accompanying illustration showing a
corresponding diagram (squared) with explicit colour-space indices on
each vertex:\\
\index{Colour connections}
\begin{equation}
\mbox{
\begin{tabular}{cc}
\parbox{5.2cm}{
$Z \to qg\bar{q}$~~~:~~~\\
\[
\begin{array}{rcl}
\displaystyle\sum_{\mrm{colours}}|M|^2 & \propto & \displaystyle
\delta_{ij}t_{jk}^a t_{k\ell
    }^a\delta_{\ell i} \\
& = & \displaystyle
\mrm{Tr}\{t^at^a\}\\[4mm] & = & \displaystyle
  \frac12\mrm{Tr}\{\delta\} = 4~,
\end{array}
\]}
&
\parbox{8.5cm}{\includegraphics*[scale=0.6]{colFacZ3.pdf}
}
\end{tabular}}
\end{equation}
where the last $\mrm{Tr}\{\delta\} = 8$, since the trace runs over
the 8-dimensional adjoint indices. If we
want to ``count the paths through colour space'', we should leave out
the factor $\frac12$ which comes from the normalisation convention for
the $t$ matrices, \eqRef{eq:t}, hence this process can take 8
different paths through colour space, one for each gluon basis state.

The tedious task of taking traces over $t$
matrices can be greatly alleviated by use of the relations given in
\TabRef{tab:lambda}.  
\index{Traces in SU(3)|see{SU(3)}}%
\index{SU(3)!Trace relations}%
\index{QCD!Trace relations|see{SU(3)}}%
\begin{table}
\begin{center}
\scalebox{1.04}{\begin{tabular}{ccc}
\toprule
\index{SU(3)!Trace relations}Trace Relation & Indices & Occurs in Diagram Squared
\\
\midrule
$\mrm{Tr}\{t^at^b\} = T_R\, \delta^{ab}$ & $a,b\in[1,\ldots,8]$
& \parbox[c]{4cm}{\includegraphics*[scale=0.5]{traces1}}\\
$\sum_a t^a_{ij}t^a_{jk} = C_F\, \delta_{ik}$ &%
\parbox[c]{3cm}{\begin{center}
$a\in[1,\ldots,8]$\\
$i,j,k\in[1,\ldots,3]$\end{center}}
& \parbox[c]{4cm}{\includegraphics*[scale=0.5]{traces2}}\\
$\sum_{c,d} f^{acd} f^{bcd} = C_A\, \delta^{ab}$ & $a,b,c,d\in[1,\ldots,8]$
& \parbox[c]{4cm}{\includegraphics*[scale=0.5]{traces3}}\\
$ t^a_{ij}t^a_{k\ell} = T_R \left(\delta_{jk}\delta_{i\ell}
- \frac{1}{N_C}\delta_{ij}\delta_{k\ell}\right)$ & $i,j,k,\ell\in[1,\ldots,3]$
& \parbox[c]{4cm}{\includegraphics*[scale=0.5]{traces4}}\hspace*{-0.2cm}(Fierz)\\
\bottomrule
\end{tabular}}
\caption{Trace relations for $t$ matrices (convention-independent). 
 More relations
  can be found in \cite[Section 1.2]{Ellis:1991qj} and in 
  \cite[Appendix A.3]{Peskin:1995ev}.
\label{tab:lambda}}
\end{center}
\end{table}
In the standard normalisation convention for the \index{SU(3)}$\mrm{SU(3)}$ generators,
\eqRef{eq:t}, the \index{Casimirs}Casimirs of $\mrm{SU(3)}$ appearing in
\TabRef{tab:lambda} are\footnote{See, e.g., \cite[Appendix
    A.3]{Peskin:1995ev} for how to obtain the Casimirs in other
  normalisation conventions. As an example, choosing $t^a_{ij} = \lambda_{ij}^a/\sqrt{2}$ would yield $T_R=1$, $C_F=T_R(N_C^2-1)/N_C=8/3$, $C_A=3$.} 
\index{Casimirs}\index{TR@$T_R$}\index{CA@$C_A$}\index{CF@$C_F$}
\begin{equation}
T_R = \frac12 \hspace*{2cm} C_F = \frac43 \hspace*{2cm} C_A = N_C = 3~.
\end{equation}
In addition, the gluon self-coupling on the third line in
\TabRef{tab:lambda} involves factors of $f^{abc}$. These
\index{QCD!Structure constants|see{SU(3)}}%
are called the \index{SU(3)!Structure constants}\emph{structure constants} of QCD and they enter via 
the non-Abelian term in the \index{Gluons}gluon field strength tensor appearing in
\eqRef{eq:L}, 
\begin{equation}
F^a_{\mu\nu} = \underbrace{\partial_\mu A_\nu^a - \partial_\nu
  A^a_\mu}_{\mathrm{Abelian}} +
\underbrace{ g_s f^{abc} A_\mu^b A_\nu^c}_{\mathrm{non-Abelian}}~. \label{eq:F}
\end{equation}

\noindent\begin{minipage}[t]{0.46\textwidth}
The structure constants of $\mrm{SU(3)}$ are listed in the table to the
right. They define the \emph{adjoint}, or \emph{vector}, representation of $\mrm{SU(3)}$
and are related to the fundamental-representation generators via the
commutator relations
\begin{equation}
t^at^b - t^bt^a = [t^a,t^b] = i f^{abc} t_c~,
\end{equation} 
or equivalently,
\begin{equation}
if^{abc}~=~2\mrm{Tr}\{t^c[t^a,t^b]\}~.
\end{equation}
Thus, it is a matter of choice whether one prefers to express colour
space on a basis of fundamental-representation $t$ matrices, or via
the structure constants $f$, and one can go back and forth between the
two.
\end{minipage}%
\hfill%
\colorbox{darkgray}{%
\colorbox{lightgray}{%
\begin{minipage}[t]{0.46\textwidth}
\vspace*{3mm}\begin{center}
\textbf{Structure Constants of SU(3)}
\begin{equation}
f_{123} = 1
\end{equation}
\begin{equation}
f_{147} = f_{246} = f_{257} = f_{345} = \frac12
\end{equation}
\begin{equation}
f_{156} = f_{367} = -\frac12
\end{equation}
\begin{equation}
f_{458} = f_{678} = \frac{\sqrt{3}}{2}
\end{equation}
Antisymmetric in all indices\\[3mm]
All other $f_{abc}=0$\vspace*{3mm}\\
\end{center}
\end{minipage}%
}}\vskip1mm

\begin{figure}[t]
\begin{center}
\begin{minipage}[h]{4.6cm}
\begin{center}
$A_\nu^4(k_2)$\\
\includegraphics*[scale=0.75]{ggv.pdf}\\[-3mm]
$A^6_\rho(k_1)$\hfill$A_\mu^2(k_3)$
\end{center}
\end{minipage}~~~
\parbox{0.35\textwidth}{
$
\begin{array}{cccc}
\propto & - g_s \ f^{246} \!\! & \!\! [ (k_3 - k_2)^\rho g^{\mu\nu}  \\ 
& & +(k_2 - k_1)^\mu g^{\nu\rho} \\ 
& &+(k_1 - k_3)^\nu g^{\rho\mu}]
\end{array}
$}\vspace*{1mm}
\caption{Illustration of a \index{Gluons}$ggg$ vertex in QCD, before
  summing/averaging over colours: interaction between gluons in the 
  states $\lambda^2$, $\lambda^4$, and $\lambda^6$ is represented by
  the structure constant $f^{246}$. 
\label{fig:gg}}
\end{center}
\end{figure}
 Expanding the $F_{\mu\nu}F^{\mu\nu}$ term of the
Lagrangian using \eqRef{eq:F}, we see that there is a 3-gluon and a
4-gluon vertex that involve $f^{abc}$, the latter of which has two
powers of $f$ and two powers of the coupling. 

Finally, the last line of \TabRef{tab:lambda} is not really a trace
relation but instead a useful so-called Fierz transformation, which
expresses products of $t$ matrices in terms of Kronecker $\delta$ functions. 
It is often used, for instance, in shower Monte Carlo
applications, to assist in mapping between colour flows in $N_C = 3$,
in which cross sections and splitting probabilities are calculated, 
and those in $N_C\to\infty$ (``leading colour''), used to represent colour flow in
the MC ``event record''.

A \index{Gluons}gluon self-interaction vertex is
illustrated in \figRef{fig:gg}, to be compared with the quark-gluon
one in \figRef{fig:qg}. We remind the reader that gauge boson
self-interactions are a hallmark of non-Abelian theories and that their
presence leads to some of the main differences between QED and
QCD. One should also keep in mind 
that the \index{Colour factors}colour factor for the vertex in \figRef{fig:gg}, \index{CA@$C_A$}$C_A$, 
is roughly twice as large as that for a quark, \index{CF@$C_F$}$C_F$.

\subsection{The Strong Coupling \label{sec:coupling}}
\index{QCD!Coupling}
\index{Jets}
\index{alphaS@$\alpha_s$}To first approximation, QCD is 
\index{QCD!Scale invariance}\emph{scale invariant}. That is, if one
``zooms in'' on a QCD jet, one will find a repeated self-similar 
pattern of jets within jets within jets, reminiscent of
fractals. 
In the context of QCD, this property was originally 
called \index{Lightcone scaling|see{QCD Scale invariance}}light-cone scaling, or 
\index{Bjorken scaling|see{QCD Scale invariance}}Bj{\o}rken scaling. 
This type of scaling is closely related to the class of
angle-preserving symmetries, called \index{Conformal
invariance}\emph{conformal} symmetries. In physics 
today, the terms ``conformal'' and ``scale invariant'' are used 
interchangeably\footnote{Strictly speaking, conformal symmetry is more
restrictive than just scale invariance, but examples of
scale-invariant field theories that are not conformal are rare.}.
Conformal invariance is a mathematical property of several
QCD-``like'' theories which are now being studied (such as $N=4$
supersymmetric relatives of QCD). It is also 
related to the physics of so-called ``unparticles'', though that is a
relation that goes beyond the scope of these lectures.

Regardless of the labelling, 
if the  \index{alphaS@$\alpha_s$}strong coupling did not run (we shall
return to the running 
of the coupling below), Bj{\o}rken scaling would be absolutely true. QCD
would be a theory with a fixed coupling, the same at all scales. 
This simplified picture already captures some of the most important
properties of QCD, as we shall discuss presently.  

\index{QCD!Scale invariance}%
In the limit of exact Bj{\o}rken scaling --- QCD at fixed coupling
--- properties of high-energy interactions are determined 
only by \emph{dimensionless} kinematic quantities, such as scattering
angles (pseudorapidities) and ratios of energy
scales\footnote{Originally, the observed approximate agreement with
this was used as a powerful argument
for pointlike substructure in hadrons; since measurements at different
energies are sensitive to different resolution scales, independence of the absolute
energy scale is indicative of the absence of other fundamental
scales in the problem and hence of pointlike constituents.}.
For applications of QCD to high-energy collider physics, an important
consequence of Bj{\o}rken scaling is thus that the rate of 
\index{Parton showers}%
\index{Bremsstrahlung|see{Parton showers}}
bremsstrahlung
jets, with a given transverse momentum, scales in direct proportion to
the hardness 
of the fundamental partonic scattering process they are produced in
association with. This agrees well with our intuition about accelerated
charges; the harder you ``kick'' them, the harder the radiation they
produce.  

For instance, in the limit of exact scaling, a
measurement of the rate of 10-GeV jets produced in association with an
ordinary $Z$ 
boson could be used as a direct prediction of the rate of 100-GeV jets
that would be 
produced in association with a 900-GeV $Z'$ boson, and so 
forth. Our intuition about how many bremsstrahlung jets a given type of
process is likely to have should therefore be governed first and
foremost by the \emph{ratios} of scales that appear in that particular
process, as has been  highlighted in a number of studies focusing on
the mass and $p_\perp$ scales appearing, e.g., in
Beyond-the-Standard-Model (BSM) 
physics processes
\cite{Plehn:2005cq,Alwall:2008qv,Papaefstathiou:2009hp,Krohn:2011zp}. 
\index{QCD!Scale invariance}Bj{\o}rken scaling 
\index{Scale invariance|see{QCD}}
is also fundamental to the understanding of jet substructure in QCD, see, e.g.,
\cite{Vermilion:2011nm,Altheimer:2012mn}.  

\index{alphaS@$\alpha_s$!Running coupling}%
On top of the underlying scaling behavior, the running coupling will
introduce a dependence on the absolute scale, implying more radiation
at low scales than at high ones. The running is logarithmic with
\index{alphaS@$\alpha_s$!beta function}%
energy, and is governed by the so-called \emph{beta function}, 
\index{alphaS@$\alpha_s$}
\begin{equation}
Q^2 \frac{\partial \alpha_s}{\partial Q^2} = \frac{\partial
  \alpha_s}{\partial \ln Q^2} =
\beta(\alpha_s)~, \label{eq:running}
\end{equation}
where the function driving the energy dependence, the \index{Beta function}{beta
  function}, is defined as
\begin{equation}
\beta(\alpha_s) = -\alpha_s^2(b_0 +
b_1\alpha_s + b_2\alpha_s^2 + \ldots)~,\label{eq:beta}
\end{equation}
with LO (1-loop) and NLO (2-loop) coefficients
\begin{eqnarray}
b_0 & = & \frac{11C_A - 4 T_R n_f}{12\pi}~,\\[3mm]
b_1 & = & \frac{17C_A^2 - 10 T_R C_A n_f - 6 T_R C_F n_f}{24\pi^2} ~=~
\frac{153-19 n_f}{24\pi^2}~.\label{eq:b}
\end{eqnarray}
In the $b_0$ coefficient, the first term is due to
\index{Gluons!Contribution to beta function}gluon loops while the
second is due to \index{Quarks!Contribution to beta function}quark
ones. Similarly, the first 
term of the $b_1$ coefficient arises from double gluon loops,
while the second and third represent mixed quark-gluon ones. 
At higher loop orders, the $b_i$ coefficients depend explicitly on the
renormalisation scheme that is used. A brief discussion can be found in the
PDG review on QCD~\cite{pdg2012}, with more elaborate ones
contained in \cite{Dissertori:2003pj,Ellis:1991qj}. 
Note that, if there are additional coloured particles beyond the
Standard-Model ones, loops involving those particles enter
 at energy scales above the masses of the
new particles, thus modifying the  \index{alphaS@$\alpha_s$}running of the coupling at high scales. 
This is discussed, e.g., for supersymmetric models in
\cite{Martin:1997ns}. For the running of other SM couplings, see
e.g.,~\cite{Langacker:2010zza}. 

\index{alphaS@$\alpha_s$!Running coupling}%
Numerically, the value of the  \index{alphaS@$\alpha_s$}strong coupling is usually specified by
giving its value at the specific 
reference scale $Q^2=M^2_Z$, from which we can obtain its
value at any other scale by solving \eqRef{eq:running}, 
\begin{equation}
\alpha_s(Q^2) = \alpha_s(M_Z^2) \frac{1}{1+b_0
  \alpha_s(M_Z^2)\ln\frac{Q^2}{M_Z^2} + {\cal O}(\alpha_s^2)}~,
\label{eq:alphaq2}
\end{equation}
with relations including the ${\cal O}(\alpha_s^2)$ terms 
available, e.g., in \cite{Ellis:1991qj}. 
Relations between scales 
not involving $M_Z^2$ can obviously be obtained by just replacing $M_Z^2$
by some other scale $Q'^2$ everywhere in \eqRef{eq:alphaq2}. A
comparison of running at one- and two-loop order, in both cases starting from
$\alpha_s(M_Z)=0.12$, is given in \figRef{fig:asRun}.
\begin{figure}[t]
\centering
\includegraphics*[scale=0.45]{vc-alphaS.pdf}
\caption{Illustration of the running of
 $\alpha_s$ at 1- (open 
  circles) and 2-loop
  order (filled circles), 
starting from the same value of $\alpha_s(M_Z)=0.12$. 
\label{fig:asRun}}
\end{figure}
As is evident from the figure, the 2-loop running is somewhat faster
than the 1-loop one.

\index{alphaS@$\alpha_s$!Running coupling}%
As an application, let us prove that the 
logarithmic running of the coupling implies that an intrinsically 
multi-scale problem can be converted to a single-scale one, up to
corrections suppressed by two powers of $\alpha_s$, 
by taking the geometric mean of the scales involved. This follows from
expanding an arbitrary product of individual  \index{alphaS@$\alpha_s$}$\alpha_s$ factors around an
arbitrary scale $\mu$, using \eqRef{eq:alphaq2}, 
\begin{eqnarray}
\alpha_s(\mu_1)\alpha_s(\mu_2)\cdots\alpha_s(\mu_n) & = &
\prod_{i=1}^{n} \alpha_s(\mu) \left(1 +
b_0\,\alpha_s\ln\left(\frac{\mu^2}{\mu_i^2}\right) + {\cal O}(\alpha_s^2)\right)
\nonumber\\[2mm]
& = & \alpha_s^n(\mu) \left(1 + b_0\, \alpha_s \ln \left(
 \frac{\mu^{2n}}{\mu_1^2\mu_2^2\cdots\mu_n^2}\right) +  {\cal
   O}(\alpha_s^2) \right)~,
\end{eqnarray}
whereby the specific single-scale choice $\mu^n =
\mu_1\mu_2\cdots\mu_n$ (the geometric mean) can
be seen to push the difference between the two sides of the equation one order higher
than would be the case for any other combination of scales\footnote{In
  a fixed-order calculation, the individual scales $\mu_i$,
would correspond, e.g., to the $n$ hardest scales appearing in an infrared
safe sequential clustering algorithm applied to the given momentum
configuration.}. 

The appearance of the number of \index{Flavour}flavours, $n_f$, in $b_0$ implies that the
slope of the running depends on the number of contributing
\index{Flavour}flavours. Since full QCD is best approximated by $n_f=3$
below the charm threshold, by $n_f=4$ and $5$ from there to the $b$
and $t$ thresholds, respectively, and then by $n_f=6$ at scales
higher than $m_t$, it is therefore important to be aware that 
the running changes slope across quark \index{Flavour}flavour
thresholds. Likewise, it would change across the threshold for any coloured
new-physics particles that might exist, with a magnitude depending on
the particles' colour and spin quantum numbers.

\index{alphaS@$\alpha_s$!Running coupling}%
\index{alphaS@$\alpha_s$}
The negative overall sign of \eqRef{eq:beta}, combined with the fact
that $b_0 > 0$ (for $n_f \le 16$), leads to the famous
result\footnote{
Perhaps the highest pinnacle of fame for \eqRef{eq:beta} was reached
when the sign of it featured in an episode of the TV series ``Big Bang
Theory''.} 
that the QCD coupling effectively \emph{decreases} with
 energy, called \index{Asymptotic freedom}asymptotic 
freedom, for the discovery of which the \index{Nobel prize}Nobel prize in physics was
awarded to D.~Gross, H.~Politzer, and F.~Wilczek in 2004. An extract
of the prize announcement runs as follows:
\begin{center}
\begin{minipage}{0.84\textwidth}
\sl  What this year's Laureates discovered was something that, at
first sight, seemed completely contradictory. The interpretation of
their mathematical result was that the closer the quarks are to each
other, the \emph{weaker} is the ``colour charge''. When the quarks are
really close to each other, the force is so weak that they behave
almost as free particles\footnote{More correctly, it is the coupling
  rather than the  
  force which becomes weak as the distance decreases. 
  The $1/r^2$ Coulomb singularity of the force is only dampened, not removed, 
  by the diminishing coupling.}. 
This phenomenon is called ``asymptotic
freedom''. The converse is true when the quarks move apart: the force
becomes stronger when the distance increases\footnote{More correctly,
 it is the potential which grows, linearly, while the force becomes
 constant.}. 
\end{minipage}
\end{center}

\index{Running coupling|see{alphaS@$\alpha_s$}}%
\index{alphaS@$\alpha_s$!Running coupling}%
Among the consequences of \index{Asymptotic freedom}asymptotic freedom is that perturbation
theory becomes better behaved at higher absolute energies, due to the
effectively decreasing coupling. Perturbative calculations for our
900-GeV $Z'$ boson from before should therefore be slightly faster
converging than equivalent calculations for the 90-GeV one. 
Furthermore, since the running of  \index{alphaS@$\alpha_s$}$\alpha_s$ explicitly
breaks Bj{\o}rken scaling, we also expect to see small changes in jet
shapes and in jet production ratios as we vary the energy. For
instance, since high-$p_\perp$ jets
start out with a smaller effective coupling, their intrinsic shape
(irrespective of boost effects) is
somewhat narrower than for low-$p_\perp$ jets, an issue which can be
important for jet calibration. Our current understanding of the
running of the QCD coupling is summarised by the plot in
\figRef{fig:alphas}, taken from a recent comprehensive review by S.\ Bethke
\cite{pdg2012,Bethke:2012jm}. A complementary up-to-date overview of
$\alpha_s$ determinations can be found in~\cite{d'Enterria:2015toz}. 

\index{alphaS@$\alpha_s$!Running coupling}%
As a final remark on \index{Asymptotic freedom}asymptotic freedom, note
that the decreasing 
value of the  \index{alphaS@$\alpha_s$}strong coupling with energy must eventually cause it to
become comparable to the electromagnetic and weak ones, at some energy
scale. Beyond that point, which may lie at energies of order
$10^{15}-10^{17}\,$GeV (though it may be lower if as yet undiscovered
particles generate large corrections to the running), 
we do not know  what the further evolution of the combined theory will 
actually look like, or whether it will continue to exhibit
\index{Asymptotic freedom}asymptotic
freedom. 

\index{alphaS@$\alpha_s$}%
\index{alphaS@$\alpha_s$!Running coupling}%
\index{alphaS@$\alpha_s$!LambdaQCD@$\Lambda_{\mathrm{QCD}}$}%
Now consider what happens when we run the coupling in the other
direction, towards smaller energies. 
\begin{figure}[t]
\begin{center}\hspace*{-0.25cm}
\parbox[c]{3.1cm}{\includegraphics*[scale=0.65]{arr-ir.pdf}}
\parbox[c]{8cm}{\includegraphics*[scale=0.5]{asq-2011.pdf}}\hspace*{-1mm}
\parbox[c]{3.1cm}{\includegraphics*[scale=0.65]{arr-uv.pdf}}
\caption{Illustration of the running of $\alpha_s$ in a theoretical
  calculation (band) and in physical processes at
  different characteristic scales, from
  \cite{pdg2012,Bethke:2012jm}. The little kinks at $Q=m_{c}$ and
  $Q=m_b$ are
  caused by discontinuities in the running across the flavour
  thresholds.\label{fig:alphas}}  
\end{center}           
\end{figure}
Taken at face value, the numerical value of the coupling diverges
rapidly at scales below 1 GeV, as illustrated by the curves
disappearing off the left-hand edge of the plot in
\figRef{fig:alphas}. To make this divergence
explicit, one can rewrite
\eqRef{eq:alphaq2} in the following form, 
 \index{alphaS@$\alpha_s$}
\begin{equation}
\alpha_s(Q^2) = \frac{1}{b_0 \ln \frac{Q^2}{\Lambda^2}}~,\label{eq:alphasLam}
\end{equation}
where 
\begin{equation}
\Lambda \sim 200\, \mbox{MeV}
\end{equation}
\index{alphaS@$\alpha_s$!LambdaQCD@$\Lambda_{\mathrm{QCD}}$}%
\index{alphaS@$\alpha_s$!Landau Pole|see{$\Lambda_{\mathrm{QCD}}$}}%
\index{LambdaQCD@$\Lambda_{\mathrm{QCD}}$|see{alphaS@$\alpha_s$}}%
specifies the energy scale at which the perturbative coupling would nominally become
infinite, called the Landau pole. (Note, however, that this only
parametrises the purely \emph{perturbative} result, which is not
reliable at \index{Strong coupling}strong coupling, so \eqRef{eq:alphasLam} should 
not be taken to imply that the physical behavior of full QCD should
exhibit a divergence for $Q\to \Lambda$.) 

\index{alphaS@$\alpha_s$}%
\index{alphaS@$\alpha_s$!Running coupling}%
\index{alphaS@$\alpha_s$!LambdaQCD@$\Lambda_{\mathrm{QCD}}$}%
Finally, one should be aware that there is a multitude of different
ways of defining both $\Lambda$ and $\alpha_s(M_Z)$. At the very
least, the numerical value one obtains depends both on the
renormalisation scheme used (with the dimensional-regularisation-based
``modified minimal subtraction'' scheme, $\overline{\mbox{MS}}$, being the
most common one) and on the perturbative order of the calculations 
used to extract them. As a rule of thumb, fits to experimental data typically yield 
smaller values for $\alpha_s(M_Z)$ the higher the order of the
calculation used to extract it (see, e.g.,
\cite{Bethke:2009jm,Dissertori:2009ik,Bethke:2012jm,pdg2012}), with  $
\alpha_s(M_Z)\vert_{\mrm{LO}} \gsim \alpha_s(M_Z)\vert_{\mrm{NLO}}
\gsim \alpha_s(M_Z)\vert_{\mrm{NNLO}}$. 
Further, since the number of \index{Flavour}flavours changes the slope
of the running, the location of the Landau pole for fixed
$\alpha_s(M_Z)$ depends explicitly on the number of \index{Flavour}flavours used in
the running. Thus each value of $n_f$ is associated with its own
value of $\Lambda$, with the following matching relations across
thresholds guaranteeing continuity of the coupling at one loop,
\index{LambdaQCD@$\Lambda_{\mathrm{QCD}}$|see{$\alpha_s$}}
\index{alphaS@$\alpha_s$!LambdaQCD@$\Lambda_{\mathrm{QCD}}$}%
\begin{eqnarray}
n_f = 5 \leftrightarrow 6 ~~~:~~~~~~\Lambda_6 = \Lambda_5
  \left(\frac{\Lambda_5}{m_t}\right)^{\frac{2}{21}} & & 
\Lambda_5 = \Lambda_6
  \left(\frac{m_t}{\Lambda_6}\right)^{\frac{2}{23}} ~, \\[2mm]
n_f = 4 \leftrightarrow 5 ~~~:~~~~~~\Lambda_5 = \Lambda_4
  \left(\frac{\Lambda_4}{m_b}\right)^{\frac{2}{23}} & & 
\Lambda_4 = \Lambda_5
  \left(\frac{m_b}{\Lambda_5}\right)^{\frac{2}{25}} ~, \\[2mm]
n_f = 3 \leftrightarrow 4 ~~~:~~~~~~\Lambda_4 = \Lambda_3 
  \left(\frac{\Lambda_3}{m_c}\right)^{\frac{2}{25}} & &
\Lambda_3 = \Lambda_4 
  \left(\frac{m_c}{\Lambda_4}\right)^{\frac{2}{27}} ~.
\end{eqnarray}

\index{alphaS@$\alpha_s$}%
\index{alphaS@$\alpha_s$!Running coupling}%
It is sometimes stated that QCD only has a single free
parameter, the  \index{alphaS@$\alpha_s$}strong coupling. 
However, even in the perturbative
region, the beta function depends explicitly on the number of
quark \index{Flavour}flavours, as we have seen, and thereby also on the quark
masses. Furthermore, in the non-perturbative region around or below
$\Lambda_{\mrm{QCD}}$, the value of the 
perturbative coupling, as obtained, e.g., from \eqRef{eq:alphasLam},
gives little or no insight into the behavior of the full theory. 
Instead, universal functions (such as parton densities, form factors,
fragmentation functions, etc), effective theories (such as the
Operator Product Expansion, Chiral Perturbation Theory, or Heavy Quark
Effective Theory), or phenomenological models (such as Regge Theory or
the String and Cluster Hadronisation Models) must be used, which in
turn depend on additional non-perturbative parameters whose relation to, e.g.,
$\alpha_s(M_Z)$, is not a priori known. 

\index{Lattice QCD}
For some of these questions,
such as hadron masses, lattice QCD can furnish important
additional insight, but for multi-scale and/or time-evolution
problems, the applicability of lattice methods is still severely
restricted; the lattice formulation of QCD requires 
  a Wick rotation to
  Euclidean space. The time-coordinate can then be treated on an
  equal footing with the other dimensions, but intrinsically
  Minkowskian problems, such as the time evolution of a system, are
   inaccessible. The limited size of current lattices
  also severely constrain the scale hierarchies that it is possible to
  ``fit'' between the lattice spacing and the lattice size. 

\index{Landau pole|see{$\alpha_s$}}%
\index{QCD!Landau Pole|see{$\alpha_s$}}%
\index{Renormalisation|see{$\alpha_s$}}%
\index{QCD!Renormalisation|see{$\alpha_s$}}%

\subsection{Colour States}
\index{Coherence}%
A final example of the application of the underlying $\mrm{SU(3)}$ group
theory to QCD is given by considering which colour states we can
obtain by combinations of quarks and gluons. The simplest example of
this is the combination of a quark and antiquark. We can form a total
of nine different colour-anticolour combinations, which fall into two
irreducible representations of $\mrm{SU(3)}$:
\begin{equation}
3 \otimes \overline{3} = 8 \oplus 1~.\label{eq:33bar}
\end{equation}
The singlet corresponds to the symmetric wave function 
$\frac{1}{\sqrt{3}}\left(\left|R\bar{R}\right>+\left|G\bar{G}\right>+\left|B\bar{B}\right>\right)$, 
which is invariant under $\mrm{SU(3)}$ transformations (the definition of a
singlet). The other eight linearly independent 
combinations (which can be represented by one for each Gell-Mann
matrix, with the singlet corresponding to the identity matrix) transform
into each other under $\mrm{SU(3)}$. Thus, although we sometimes talk about
colour-singlet states as 
being made up, e.g., of ``red-antired'', that is not quite precise
language. The actual state $\left|R\bar{R}\right>$ is \emph{not} a
pure colour singlet.  Although it does
have a non-zero \emph{projection} onto the singlet wave function
above, it also has non-zero projections onto the two members of
the octet that correspond to the diagonal Gell-Mann
matrices. Intuitively, one can also easily realise this by noting that
an $\mrm{SU(3)}$ rotation of $\left|R\bar{R}\right>$ would in general turn it into a
different state, say $\left|B\bar{B}\right>$, whereas a true colour singlet
would be invariant. 
Finally, we can also realise from \eqRef{eq:33bar} that a random
(colour-uncorrelated) quark-antiquark pair has a $1/N^2=1/9$ 
chance to be in an overall colour-singlet state; otherwise it is in
an octet. 

Similarly, there are also nine possible quark-quark (or
antiquark-antiquark) combinations, six of which are symmetric
under interchange of the two quarks and three of which are antisymmetric:
\index{Sextet}%
\begin{equation}
6 ~=~ \left(\begin{array}{c}
\left|RR\right>\\
\left|GG\right>\\
\left|BB\right>\\
\frac{1}{\sqrt{2}}\left(\left|RG\right> + \left|GR\right>\right)\\
\frac{1}{\sqrt{2}}\left(\left|GB\right> + \left|BG\right>\right)\\
\frac{1}{\sqrt{2}}\left(\left|BR\right> + \left|RB\right>\right)
\end{array}\right)
~~~~~~~~~
\bar{3} = \left(\begin{array}{c}
\frac{1}{\sqrt{2}}\left(\left|RG\right> - \left|GR\right>\right)\\
\frac{1}{\sqrt{2}}\left(\left|GB\right> - \left|BG\right>\right)\\
\frac{1}{\sqrt{2}}\left(\left|BR\right> - \left|RB\right>\right)
\end{array}\right)~.
\end{equation}
The members of the sextet transform into (linear combinations of) 
each other under $\mrm{SU(3)}$ transformations, and similarly for the
members of the antitriplet, hence neither of these can be reduced
further. The breakdown into
irreducible $\mrm{SU(3)}$ multiplets is therefore
\begin{equation}
3 \otimes 3 = 6 \oplus \overline{3}~.
\end{equation}
Thus, an uncorrelated pair of quarks has a $1/3$ chance to add to an overall
anti-triplet state (corresponding to coherent
superpositions like ``red + green = antiblue''\footnote{In the context of
  hadronisation models, 
  this coherent superposition of two quarks in an overall antitriplet
  state is sometimes called a
  \index{Diquarks}``diquark'' (at low $m_{qq}$)
  \index{String junctions}or a ``string junction'' (at high $m_{qq}$), see
  \secRef{sec:stringModel}; it corresponds to the antisymmatric ``red
  + green = antiblue'' combination needed to create a baryon
  wavefunction. }); otherwise it is in an overall 
sextet state. 

Note that the emphasis on
the quark-(anti)quark pair being \emph{uncorrelated} is important;
production processes that correlate the produced partons, like $Z\to q\bar{q}$ or $g\to q\bar{q}$, will
project out specific components (here the singlet and octet,
respectively). 
Note also that, if the quark
and (anti)quark are on opposite sides of the universe (i.e., living in
two different hadrons), the QCD \emph{dynamics} will not care what
overall colour state they 
are in, so for the formation of multi-partonic states in QCD, obviously the
spatial part of the wave functions (causality at the very least) 
will also play a role. Here, we are considering \emph{only} the colour part
of the wave functions. 
Some additional examples are 
\begin{eqnarray}
8\otimes 8 & = & 27 \oplus 10 \oplus \overline{10} \oplus 8 \oplus 8
\oplus 1 ~,\\ 
3 \otimes 8 & = & 15 \oplus 6 \oplus 3~,\\
3 \otimes 6 & = & 10 \oplus 8~,\\
3\otimes3\otimes3 & = & (6 \oplus \overline{3}) \otimes 3 = 10 \oplus 8
\oplus 8 \oplus 1 ~.
\end{eqnarray}
Physically, the 27 in the first line corresponds to a completely
incoherent addition of the colour charges of two gluons;
\index{Decuplet}the decuplets are slightly more coherent (with a lower
total colour charge), the octets
yet more, and the singlet corresponds to the combination of two gluons
that have precisely equal and opposite colour charges, so that their
total charge is zero. 
Further extensions and generalisations of these combination rules can
\index{Young tableaux}be obtained, e.g., using the method of Young
tableaux~\cite{young1901,youngSagan}.  


\section{Preliminaries}
\label{sec:preliminaries}

We begin by setting up definitions and terminology of a general mathematical nature that we will use throughout.

\subsection{Families}

For several reasons, we work with \emph{families} in places where classical treatments would use either \emph{subsets} of, or \emph{lists} from, a given set.
%
While the term is standard (e.g. ``the product of a family of rings''), we make rather more central use of it than is usual, so we establish some notations and terminology.

\begin{definition}
  \label{def:family}%
  Given a set $X$, a \defemph{family $K$ of elements of $X$} (or briefly, a \defemph{family on~$X$}) consists of an index set $\famindex K$ and a map $\famev{K} : \famindex{K} \to X$.
  %
  We let $\Fam X$ denote the collection of all families on~$X$, and use the \defemph{family comprehension} notation $\famtuple{e_i \in X}{i \in I}$ for the family indexed by~$I$ that maps~$i$ to~$e_i$. A family may be explicitly described by displaying the association of indices to values. For example, we may write $\family{0 \mto e_0, 1 \mto e_1, 2 \mto e_2}$ for the family $\famtuple{e_i}{i \in \set{0,1,2}}$.
\end{definition}

\begin{example}
  Any \emph{subset} $A \subseteq X$ can be viewed as a family $\famtuple{i \in X}{i \in A}$, with $\famindex A$ as $A$ itself and $\famev{A}$ the inclusion $A \injto X$.
  %
  Motivated by this, we will often speak of a family~$K$ as if it were a subset, writing $x \in K$ rather than $x \in \famindex K$, and treating such $x$ itself as an element of $X$ rather than explicitly writing $\famev{K}(x)$.
\end{example}

\begin{example}
  Any \emph{list} $\ell = [x_0,\ldots,x_n]$ of elements of $X$ can be viewed as a family, with $\famindex \ell = \{ 0, \ldots, n \}$ and $\famev{\ell}(i) = x_i$, or equivalently $\ell = \family{0 \mto x_0, \ldots, n \mto x_n}$.
  %
  We will often use list notation to present concrete examples of families.
\end{example}

Working constructively, it is quite important to keep the distinction between families and subsets where classical treatments would confound them.
%
For instance, a propositional theory is usually classically defined as a \emph{set} of propositions; we would instead use a \emph{family} of propositions.
%
In a derivation over the theory, uses of axioms therefore end up “tagged” with elements of the index set of the theory, typically explaining how a certain proposition arises as an axiom (since the same proposition might occur as an instance of axiom schemes in multiple ways).
%
These record constructive content which may be needed for, say, interpreting axioms according to a proof by cases over the axiom schemes of the theory.

Our use of families where most traditional treatments use lists --- e.g.\ for specifying the argument types of a constructor --- is less mathematically significant.
%
It is partly to avoid baking in assumptions of finiteness or ordering where they are not required; but it is mostly motivated just by the formalisation, where families provide a more appropriate abstraction.


\begin{definition}
  \label{def:family-map}%
  A \defemph{map of families} $f : K \to L$ between families~$K$ and~$L$ on~$X$ is a map
  $f : \famindex K \to \famindex L$ such that $\famev{L} \circ f = \famev{K}$.
\end{definition}

We shall notate such a map as $\fammap{f(x)}{x \in \famindex K}$. Indeed, the notation $\fammap{f(x)}{x \in A}$ works for \emph{any} maps $f : A \to B$, as it is just an alternative way of writing $\lambda$-abstractions.

Families and their maps form a category $\Fam X$, which is precisely the slice category $\Set/X$.
%
A map $r : X \to Y$ yields a functorial action $\act{r} : \Fam X \to \Fam Y$ which takes $K \in \Fam X$ to the family $\act{r} K$ with $\famindex (\act{r} K) = \famindex K$ and $\famev{\act{r} K} = r \circ \famev{K}$. It is perhaps clearer to write down the action in terms of family comprehension: $\act{r} \famtuple{e_i}{i \in I} = \famtuple{r(e_i)}{i \in I}$.

\begin{definition}
  \label{def:family-map-over}%
  Given a function $r : X \to Y$ and families $K$, $L$ on $X$, $Y$ respectively, a \defemph{map $f : K \to L$ over $r$} is a map $f : \act{r} K \to L$; equivalently, a map $f : \famindex K \to \famindex L$ forming a commutative square over $r$.
\end{definition}


\subsection{Closure systems}

The general machinery of derivations as closure systems occurs throughout logic, and is independent of the specific syntax or judgements of the logical systems involved.

\begin{definition}
  \label{def:closure-rule}\label{def:closure-system}%
  A \defemph{closure rule $(P, c)$} on a set~$X$ consists of a family $P$ of elements in~$X$, its \defemph{premises}, and a \defemph{conclusion} $c \in X$.
  %
  A \defemph{closure system~$\csS$} on a set~$X$ is a family of closure rules on~$X$, where we respectively write $\premises R$ and $\conclusion R$ for the premises and the conclusion corresponding to a rule $R \in \csS$.
  %
  We write $\ClosureRule X$ and $\ClosureSystem X$ for the collections of closure rules and closure systems on~$X$, respectively.
\end{definition}

As is tradition, we display a closure rule with premises $[p_1,\ldots,p_n]$ and conclusion~$c$ as
%
\[
  \inferrule{p_1 \quad \cdots \quad p_n}{c}
\]
%

The constructions of closure rules and closure systems are evidently functorial in the ambient set.
%
A map $f : X \to Y$ sends a rule $R \in \ClosureRule X$ to the rule $\act{f} R \in \ClosureRule Y$ with $\premises (\act{f} R) \defeq \famtuple{f(p)}{p \in \premises R}$ and $\conclusion (\act{f} R) \defeq f(\conclusion R)$.
%
Similarly, a closure system $\csS$ on $X$ is taken to the closure system $\act{f} \csS$ on $Y$, defined by $\act{f} \csS \defeq \famtuple{\act{f} r}{r \in \csS}$.

\begin{definition}
  A \defemph{simple map} $\csS \to \csT$ between closure systems $\csS$ and $\csT$ on~$X$ is just a map between them as families.
  %
  More generally, a \defemph{simple map $\bar{f} : \csS \to \csT$ over $f : X \to Y$} from $\csS \in \ClosureSystem X$ to $\csT \in \ClosureSystem Y$
  is just a simple map $\bar{f} : \act{f}\csS \to \csT$, or equivalently a family map~$\bar{f}$ over $\act{f} : \ClosureRule X \to \ClosureRule Y$.
\end{definition}

A closure system yields a notion of derivation:

\begin{definition}
  \label{def:closure-system-derivation}%
  Given a closure system $\csS$ on $X$, a family $H$ of elements in $X$, and an element $c \in X$, the
  \defemph{derivations~$\derivation{\csS}{H}{c}$ of~$c$ from hypotheses~$H$} are inductively generated
  by:
  %
  \begin{enumerate}
  \item for every $h \in H$, there is a corresponding derivation $\hyp h \in \derivation{\csS}{H}{h}$,
  \item for every rule $R \in \csS$ and a map $D \in \prod_{p \in \premises R} \derivation{\csS}{H}{p}$ there is a derivation $\der{R}{D} \in \derivation{\csS}{H}{\conclusion R}$.
  \end{enumerate}
\end{definition}

In the second clause above $D$ is a dependent map, i.e., for each $p \in \premises R$ we have $D_p \in \derivation{\csS}{H}{p}$.
%
We do not shy away from using products of families and dependent maps when the situation demands them.

The elements of $\derivation{\csS}{H}{c}$ may be seen as well-founded trees with edges and nodes suitably labelled from $X$, $\csS$, and $H$.
%
We take such inductively generated families of sets as primitive;
%
their existence may be secured one way or another, depending on the ambient mathematical foundations.
%
The essential feature of derivations, which we rely on, is the structural induction principle they provide.

It is easy to check that derivations are functorial in simple maps of closure systems, in a suitable sense:
%
\begin{propositionwithqed}
  A simple map $\bar{f} : \csS_X \to \csS_Y$ of closure systems over $f : X \to Y$ acts on derivations as $\act{\bar{f}} : \derivation{\csS_X}{H}{c} \to \derivation{\csS_Y}{\act{f}H}{f(c)}$ for each $H$ and $c$.
  %
  The action is moreover functorial, in that $\act{\idmap} = \idmap$ and $\act{(\bar{f} \circ \bar{g})} = \act{\bar{f}} \circ \act{\bar{g}}$.
\end{propositionwithqed}

Often, one wants a more general notion of map, sending each rule of the source system not necessarily to a single rule of the target system, but instead to a \emph{derived} rule:

\begin{definition}
  \label{def:derivation-grafting}%
  A \defemph{derivation of a rule} $R$ over a closure system $\csS$ is a derivation of $\conclusion R$ from $\premises R$ over $\csS$.
  %
  Given such a derivation, we call $R$ a \defemph{derived rule} of $\csS$, or say $R$ is \defemph{derivable over $\csS$}.
  %
  A \defemph{map of closure systems} $\bar{f} : \csS \to \csT$ over $f : X \to Y$ is a function giving, for each rule $R$ of $C$, a derivation of $\act{f}R$ in $\csT$.
\end{definition}

To show that maps of closure systems preserve derivability, we need a \emph{grafting} operation on derivations.

\begin{lemma}
  \label{lem:hypotheses-grafting}
  %
  Given an ambient closure system $\csS$, suppose $D$ is a derivation of $c$ from hypotheses~$H$ over $\csS$, and for each $h \in H$, $D_h$ is a derivation of $h$ from $H'$.
  %
  Then there is a derivation of $c$ from $H'$ over $\csS$.
\end{lemma}

\begin{proof}
  The derivation of~$c$ from~$H'$ is constructed inductively from a derivation of~$c$ from~$H$:
  %
  \begin{enumerate}
  \item if $c$ is derived as one of the hypotheses $h \in H$, then $D_h$ derives $c$ from $H'$,
  \item if $\der{R}{D'}$ derives $c$ from $H$, then for each $p \in \premises R$ we inductively obtain a derivation $D''_p$ of $p$ from $H'$ from the corresponding derivation~$D'_p$ of $p$ from $H$, and assemble these into the derivation $\der{R}{D''}$ of~$c$ from~$H'$. \qedhere
  \end{enumerate}
\end{proof}

\begin{definition}
  A closure system map $\bar{f} : \csS \to \csT$ over $f : X \to Y$ \defemph{acts on derivations}: if $D$ is a derivation of $c$ from $H$ over $\csS$, there is a derivation $\act{\bar{f}}D$ of $f(c)$ from $\act{f}H$ over $\csT$.
\end{definition}

\begin{proof}
  $\act{\bar{f}}D$ is defined by recursion on $D$.
  %
  Wherever $D$ uses a rule $R$ of $\csS$, with derivations $D_h$ of the premises, $\act{\bar{f}}D$ uses the given derivation $\bar{f}(R)$ of $\act{f}R$, with the derivations $\act{\bar{f}}D_h$ grafted in at the hypotheses.
\end{proof}

Categorically, grafting can be recognised as the multiplication operation of a monad structure on derivations, and our maps of closure systems can be seen as Kleisli maps for this monad (relative to simple maps).
%
One can thus show that they form a category, that the action on derivations is functorial, and so on.
%
We do not make this precise here, as it is not required for the present paper.

\subsection{Well-founded orders}

There will be several occasions when we shall have to prevent dependency cycles (between premises of a rule, or between rules of a type theory). For this purpose we review a notion of well-foundedness which accomplishes the task.

\begin{definition}
  \label{def:well-founded-order}%
  A \defemph{strict partial order} on a set $A$ is an irreflexive and transitive relation~$<$ on~$A$.
  %
  A subset $S \subseteq A$ is \defemph{$<$-progressive} when, for all $x \in A$,
  %
  \begin{equation*}
    (\all{y \in A} y < x \lthen y \in S) \lthen x \in S.
  \end{equation*}
  %
  A \defemph{well-founded order} is a strict partial order~$<$ in which a subset is the entire set as soon as it is $<$-progressive.
  %
  For each $x \in A$, the \defemph{initial segment} $\initialSegment{i} \defeq \set{y \in A \such y < x}$ is the set of elements preceding~$x$ with respect to the order.
\end{definition}

\noindent
In terms of an induction principle a strict partial order is well-founded when, for every predicate~$\varphi$ on~$A$,
%
\begin{equation*}
  \all{x \in A}{
    (\all{y \in A} y < x \lthen \varphi(y)) \lthen \varphi(x)
  } \lthen
  \all{x \in A}{\varphi(x)}.
\end{equation*}
%
Classically there are many equivalent definitions of well-founded orders.
%
Constructively, the situation is more complicated, cf.\ \cite[\textsection 2.5]{taylor:practical-foundations}; this definition is one of the most standard, and the most suited to our purpose. 


%%% Local Variables:
%%% mode: latex
%%% TeX-master: "general-type-theories"
%%% End:

\section{Raw syntax}
\label{sec:raw-syntax}

In this section, we set out our treatment of raw syntax with binding, sometimes called ``pre-syntax'' to indicate that no typing information is present at this stage.
%
There is nothing essentially novel --- briefly, we use a standard modern treatment, closely inspired by that of \citep{fiore-plotkin-turi}, but focus on concrete constructions rather than categorical characterisations.
%
So we take raw expressions as inductively generated trees, and scope systems, developed below, for keeping track of variable scopes and binding.
%
We spell out the details in order to have a self-contained presentation tailored to our requirements, and to set up terminology and notation we will use later.

\subsection{Scope systems}

We first address the question of how to treat variables and binding.
%
Should we use terms with named variables up to $\alpha$-equivalence, or de Bruijn indices, or reuse the binding structure of a framework language?
%
The last option is appealing, as it dispenses with many cumbersome details,
but we shall avoid it precisely because we want to confront the cumbersome details of type theory.
%
% No, the paragraph should stay, it is not apologetic. It sets up the correct expectations while placating the LF lovers. And we don't need forward references to everything, the reader will be able to find the discussion on LF, if we write it.

Rather than choosing a particular answer, we formulate and use a general structure for binding, abstracting away the implementation-specific details of several approaches, but retaining the common structure required for defining syntax.

\begin{definition}
  \label{def:scope-system}
  A \defemph{scope system} consists of:
  %
  \begin{enumerate}
  \item a collection of \defemph{scopes}~$\cat{S}$;
  \item for each scope $\gamma$, a set of \defemph{positions}~$\position{\gamma}$;
  \item an \defemph{empty scope~$\emptyscope$} with no positions, $\position{\emptyscope} = \emptyset$;
  \item a \defemph{singleton scope} $\singletonscope$ with a unique position, $\position{\singletonscope} = 1$;
  \item operations giving for all scopes $\gamma$ and $\delta$ a \defemph{sum scope} $\sumscope{\gamma}{\delta}$, and functions
    %
    \begin{equation*}
      \xymatrix{
        {\position{\gamma}} \ar[r]^{\inlscope} &
        {\position{\sumscope{\gamma}{\delta}}} &
        {\position{\delta}} \arrow[l]_{\inrscope}
      }
    \end{equation*}
    %
    exhibiting $\position{\sumscope{\gamma}{\delta}}$ as a coproduct of
    $\position{\gamma}$ and $\position{\delta}$.
  \end{enumerate}
\end{definition}

\noindent
%
A scope may be seen as “a context, without the type expressions”: in raw syntax, one cares about what variables are in scope, without yet caring about their types.

The singleton scope $\singletonscope$ is not needed for most of the development of general type theories --- in the present paper, it is used only for sequential contexts (\cref{sec:sequential-contexts}) and notions building on these.
%
However, it is present in all examples of interest, so we include it in the general definition.

We will also use scopes to describe \emph{binders}:
%
if some primitive symbol $S$ binds $\gamma$ variables in its $i$-th argument, then for an instance of $S$ in scope~$\delta$, the $i$-th argument will be an expression in scope $\sumscope{\delta}{\gamma}$.
%
Most traditional constructors bind finitely many variables; to facilitate this, we let $\numscope{n}$ denote the sum of $n \in \N$ copies of $\singletonscope$, which also provides alternative notations $\numscope{0}$ and $\numscope{1}$ for the empty and singleton scopes, respectively.

\begin{example} \label{ex:de-bruijn-scope-systems}
  \defemph{De Bruijn indices} and \defemph{de Bruijn levels} can be seen as scope systems, with $\N$ as the set of scopes, $\position{n} \defeq \{0, 1, \ldots, n-1\}$, $\emptyscope \defeq 0$, and $\sumscope{m}{n} \defeq m + n$.

  The difference lies just in the choice of coproduct inclusions $\inlscope{} : \position{m} \to \position{m + n} \leftarrow \position{n} : \inrscope{}$.
  %
  Setting $\inlscope(i) \defeq i + n$, $\inrscope(j) = j$ gives de Bruijn \emph{indices}, as going under a binder increments the variables outside it.
  %
  Setting $\inlscope(i) \defeq i$, $\inrscope(j) = j + n$ gives de Bruijn \emph{levels}, as higher positions go to the innermost-bound variables.
  %
  Over these scope systems, our syntax precisely recovers standard de Bruijn-style syntax, as in \citep{deBruijn:Lambda:1972} and subsequent work.
\end{example}

Both the de Bruijn scope systems are \emph{strict} in the sense that we have equalities $\sumscope{\gamma}{\emptyscope} = \gamma = \sumscope{\emptyscope}{\gamma}$ and $\sumscope{(\sumscope{\gamma}{\delta})}{\eta} = \sumscope{\gamma}{(\sumscope{\delta}{\eta})}$, whereas general scope systems provide only canonical isomorphisms.
%
The equalities help reduce bureaucracy in several proofs, so we shall occasionally indulge in assuming that we work with a strict scope system.
%
The doubtful reader may consult the formalisation, which makes no such assumptions.
%
% We definitely use the strictness in lemmas about substitution.

\begin{example}
  The \defemph{finite sets} system takes scopes to be finite sets, along with any choice of coproducts.
  %
  It may be prudent to restrict to a small collection, say the hereditarily finite sets.
  %
  Syntax over these scope systems gives a concrete implementation of categorical approaches such as \citep{fiore-plotkin-turi}.
\end{example}

\begin{example}
  Scope systems are not intrinsically linked to dependent type theories, but provide a useful discipline for syntax of other systems.
  %
  For instance, in geometric logic, the infinitary disjunction is usually given with a side condition that the free variables of the disjunction must remain finite \cite[D1.1.3(xi)]{johnstone:elephant-ii}.
  %
  By using finite scopes, we can make the finiteness condition explicit from the start, and dispense with the side condition.
  %
  This is similar in spirit to~\citep{fiore-plotkin-turi} and more closely mirrors the categorical semantics.
  %
  By contrast, the classical Hilbert-type logics $\mathcal{L}_{\alpha,\beta}$ of \citep{karp:1964-book} allow genuinely infinite contexts and binders, which can be obtained by taking scopes to be ordinals $\delta \in \alpha$, with $\position{\delta} \defeq \delta$.
\end{example}

\begin{example}
  The traditional syntax using named variables, for both free and bound variables, is \emph{not} an example of a scope system.
  %
  In that approach, a fresh variable is not introduced by summing with a scope, but rather by a multivalued map allowing extension by \emph{any} unused name.
\end{example}

Some implementations of syntax treat free and bound variables separately, for instance \emph{locally nameless syntax}~\citep{mckinna93:_pure_type_system_formal} uses concrete names for free variables but de Bruijn indices for bound variables.
%
Scope systems as defined here do not subsume such syntax, but could be generalised to do so.

Categorically, a scope system can be viewed precisely as a category~$\Scope$ with initial and terminal objects, binary coproducts, and a full and faithful functor into $(\Set, 0, +, 1)$ preserving this structure.
%
The categorical structure is induced from $\Set$: morphisms $\gamma \to \delta$ are taken as functions $\position{\gamma} \to \position{\delta}$ --- we call these \defemph{renamings}.
%
Two of these, $r : \gamma \to \delta$ and $r' : \gamma' \to \delta'$, give a sum map $\sumscope{r}{r'} : \sumscope{\gamma}{\gamma'} \to \sumscope{\delta}{\delta'}$ arising from the universal property of coproducts.

For the remainder of the paper, we fix a scope system. To make concrete examples readable, we shall write them using the de Bruijn scopes from \cref{ex:de-bruijn-scope-systems}. They can be easily adapted to any other scope system.

\subsection{Arities and signatures}
\label{sec:arities-signatures}


While the arity of a simple algebraic operation is just the number of its arguments, the situation is complicated in type theory by the presence of \emph{binders}. Each argument of a type-theoretic constructor may be a term or a type, and some of its variables may be bound by the constructor. We thus need a suitable notion of arity.

\begin{definition}
  \label{def:syntactic-class}\label{def:arity}%
  By \defemph{syntactic classes}, we mean the two formal symbols $\Ty$ and $\Tm$, representing \defemph{types} and \defemph{terms} respectively.
  %
  An \defemph{arity}~$\alpha$ is a family of pairs $(c, \gamma)$ where $c$~is a syntactic class and $\gamma$ is a scope.
  %
  We call the indices of~$\alpha$ \defemph{arguments} and write $\args \alpha$ for the index set of~$\alpha$.
  %
  Thus, each argument $i \in \args \alpha$ has an associated syntactic class $\argclass{\alpha}{i}$ and a scope $\argbinder{\alpha}{i}$, which we call the \defemph{binder} associated with the argument~$i$, and we can write~$\alpha$ as $\alpha = \famtuple{(\argclass \alpha i, \argbinder \alpha i)}{i \in \args \alpha}$.
\end{definition}

\begin{example}
  In Martin-Löf type theory with the de Bruijn scope system, the constructor $\symPi$ has arity $[(\Ty,0),(\Ty,1)]$.
  %
  That is, the arity has two type arguments, and binds one variable in the second argument.
  %
  A simpler example is the successor symbol in arithmetic whose arity is $[(\Tm,0)]$, because it takes one term argument and binds nothing.
  %
  Still simpler, the arity of a constant symbol is the empty family.
\end{example}

Note that arities express only the basic syntactic information and do not specify the types of term arguments and bound variables, which will be encoded later by typing rules.

\begin{definition}
  \label{def:signature}%
  A \defemph{signature} $\Sigma$ is a family of pairs of a syntactic class and an arity.
  %
  We call the elements of its index set \defemph{symbols}. Thus each symbol $S \in \Sigma$ has an associated syntactic class $\class{S}$ and an arity $\arity{S}$.
  %
  A \defemph{type symbol} is one whose associated syntactic class is~$\Ty$, and a \defemph{term symbol} is one whose associated syntactic class is~$\Tm$.
  %
  The \defemph{arguments} $\args S$ of~$S$ are the arguments of its arity $\arity{S}$. Each argument $i \in \args S$ has an associated syntactic class $\argclass{S}{i}$ and binder $\argbinder{S}{i}$, as prescribed by the arity~$\arity{S}$.
\end{definition}

\begin{example} \label{ex:pi-types-arities}
  The following signature describes the usual constructors for dependent products:
 \begin{align*}
   \symPi & \mapsto (\Ty, [(\Ty,0),(\Ty,1)]), \\
   \symlambda & \mapsto (\Tm, [(\Ty,0), (\Ty,1), (\Tm,1)]), \\
   \symapp & \mapsto (\Tm, [(\Ty,0), (\Ty,1), (\Tm,0),(\Tm,0)]).
 \end{align*}
 %
 Let us spell out the last line. The symbol $\symapp$ builds a term, because its syntactic class is~$\Tm$, from four arguments. The first and the second arguments are types, with one variable bound in the second argument, while the third and the fourth arguments are terms. We thus expect an application term to be written as $\symapp(A, B, s, t)$, with one variable getting bound in~$B$.
\end{example}

\begin{definition}
  \label{def:signature-map}%
  A \defemph{signature map} $F : \Sigma \to \Sigma'$ is a map of families between them, that is, a function from symbols of $\Sigma$ to symbols of $\Sigma'$, preserving the arities and syntactic classes.
  %
  Signatures and maps between them form a category~$\Sig$.
\end{definition}

\subsection{Raw syntax}

Once a signature $\Sigma$ is given, we know how to build type and term expressions over it. We call this part of the setup ``raw'' syntax to emphasise its purely syntactic nature.

\begin{definition}
  \label{def:raw-syntax}%
  The \defemph{raw syntax} over $\Sigma$ consists of the collections of \defemph{raw type expressions} $\ExprTy{\Sigma}{\gamma}$ and \defemph{raw term expressions} $\ExprTm{\Sigma}{\gamma}$, which are generated inductively for any scope~$\gamma$ as follows:
  %
  \begin{enumerate}
  \item for every position $i \in \position{\gamma}$, there is a \defemph{variable} expression $\synvar{i} \in \ExprTm{\Sigma}{\gamma}$;
  \item for every symbol $S \in \Sigma$ of syntactic class $\class{S}$, and a map
    %
    \begin{equation*}
      \textstyle
      e \in \prod_{i \in \args S} \Expr{\argclass{S}{i}}{\Sigma}{\sumscope{\gamma}{\argbinder{S}{i}}},
    \end{equation*}
    %
    there is an expression $S(e) \in \Expr{\class{S}}{\Sigma}{\gamma}$, the \defemph{application} of~$S$ to arguments~$e$.
  \end{enumerate}
\end{definition}

Let us walk through the definition.
%
The first clause states that the positions of~$\gamma$ play the role of available variable names, still without any typing information.
%
The second clause explains how to build an expression with a symbol~$S$: for each argument $i \in \args S$, an expression $e_i$ of a suitable syntactic class must be provided, where $e_i$ may refer to variable names given by~$\gamma$ as well as the variables that are bound by~$S$ in the $i$-th argument.
%
The expressions $e_i$ are conveniently collected into a function~$e$.
%
When writing down concrete examples we write the arguments as tuples.

\begin{example}%
  \label{ex:symbol-pi-arity}
  The symbol $\symPi$ has arity $[(\Ty,0),(\Ty,1)]$.
  %
  So if $A$ is a type expression with free variables amongst $\sumscope{\gamma}{\emptyscope}$ (which is isomorphic to $\gamma$), and $B$ is a type expression with free variables in $\sumscope{\gamma}{1}$ (which has an extra free variable available), then $\synPi[A,B]$ is a type expression with free variables in $\gamma$.
\end{example}

\begin{definition}
  \label{renaming-action}%
  The \defemph{action of a renaming} $r : \gamma \to \delta$ on expressions is the map $\rename{r} : \Expr{c}{\Sigma}{\gamma} \to \Expr{c}{\Sigma}{\delta}$, defined by structural recursion:
  %
  \begin{align*}
    \rename{r} (\synvar{i}) &\defeq \synvar{r(i)}, \\
    \rename{r} (S(e)) &\defeq
      S(\tuple{\rename{(\sumscope{r}{\idmap[\argbinder{S}{i}]})}(e_i)}{i \in \args{S}}).
  \end{align*}
  %
\end{definition}

\noindent
Note how the definition uses the functorial action of~$\sumscope{}{}$ to extend the renaming~$r$ when it descends under the binders of a symbol.

\begin{definition}
  \label{def:signature-map-action}%
  The \defemph{action of a signature map} $F : \Sigma \to \Sigma'$ on expressions is the map $\act{F} : \Expr{c}{\Sigma}{\gamma} \to \Expr{c}{\Sigma'}{\gamma}$, defined by structural recursion:
  %
  \begin{align*}
    \act{F} (\synvar{i}) & \defeq \synvar{i}, \\
    \act{F} (S(e)) &\defeq F(S)(\act{F} \circ e).
  \end{align*}
\end{definition}

\begin{propositionwithqed}
  \label{prop:commutation-renaming-signature-map}%
  The actions by renamings and signature maps commute with each other, and respect identities and composition.
  %
  That is, they make expressions into a functor 
  $\Expr{}{}{} : \Sig \times \Scope \to \Set^{\set{\Ty, \Tm}}$. \qedhere
\end{propositionwithqed}


\subsection{Substitution}
\label{sec:raw-substitutions}

We next spell out substitution as an operation on raw expressions, and note its basic properties.

\begin{definition}
  \label{def:raw-substitution}%
  A \defemph{raw substitution} $f : \gamma \to \delta$ over a signature $\Sigma$ is a map $f : \position{\delta} \to \ExprTm{\Sigma}{\gamma}$.
%
The \defemph{extension $\sumscope{f}{\eta} : \sumscope{\gamma}{\eta} \to \sumscope{\delta}{\eta}$ by a scope~$\eta$} is the substitution
%
\begin{align*}
  (\sumscope{f}{\eta})(\inlscope(i)) &\defeq \act{{\inlscope}}(f(i)) & &\text{if $i \in \position{\delta}$}, \\
  (\sumscope{f}{\eta})(\inrscope(j)) &\defeq \synvar{\inrscope{(j)}} & &\text{if $j \in \position{\eta}$}.
\end{align*}
%
The (contravariant) \defemph{action of $f$ on an expression $e \in \Expr{c}{\Sigma}{\delta}$} gives the expression $\tca{f} e \in \Expr{c}{\Sigma}{\gamma}$, as follows:
%
\label{def:raw-substitution-action}
\begin{align*}
  \tca{f} (\synvar{i}) &\defeq f(i), \\
  \tca{f} (S (e)) &\defeq S(\fammap{\tca{(\sumscope{f}{\argbinder{S}{i}})} e_i}{i \in \arg S}).
\end{align*}
\end{definition}

\begin{example}
  \label{ex:de-bruijn-substitution}%
  The above definition, specialized to the de Bruijn scope systems of \cref{ex:de-bruijn-scope-systems}, precisely recovers the usual definition of substitution with de Bruijn indices or levels.
  %
  Their explicit shift operators are abstracted away, in our setup, as renaming under coproduct inclusions.
\end{example}

\begin{definition}
  \label{def:variable-renaming}
  Any renaming $r : \gamma \to \delta$ induces a substitution $\bar{r} : \delta \to \gamma$, with $\bar{r}(i) \defeq \synvar{r(i)}$.
  %
  In particular, each scope $\delta$ has an \defemph{identity substitution} $\idmap[\delta] : i \mapsto \synvar{i}$.
  %
  Substitutions $f : \gamma \to \delta$ and $g : \delta \to \theta$ may be \defemph{composed} to give a substitution $g \circ f : \gamma \to \theta$, defined by $(g \circ f)(k) \defeq \tca{f}(g(k))$.
\end{definition}

We often write $r$ instead of $\bar{r}$, a slight notational abuse grounded in the next proposition.

\begin{proposition}
  For all suitable renamings $r$, substitutions $f$, $g$, and expressions $e$:
  \begin{enumerate}
  \item Substitution generalises renaming: $\tca{\bar{r}}e = \act{r}e$.
  \item Identity substitution is trivial: $\tca{\idmap[\delta]}e = e$.
  \item Substitution commutes with renaming:
    %
    \begin{equation*}
      \act{r}(\tca{f}e) = \tca{(i \mapsto \act{r}f(i))} e
      \qquad\text{and}\qquad
      \tca{f}(\act{r}e) = \tca{(i \mapsto f(r(i)))} e.
    \end{equation*}

  \item Substitution respects composition: $\tca{f} (\tca{g} e) = \tca{(g \circ f)}$.
  \item Composition of substitutions is unital and associative:
    %
    \begin{equation*}
      f = \idmap \circ f = f \circ \idmap
      \qquad\text{and}\qquad
      f \circ (g \circ h) = (f \circ g) \circ h.
    \end{equation*}
  \end{enumerate}
\end{proposition}

\begin{proof}
  All direct by structural induction on expressions, as in the standard proofs for de Bruijn syntax. 
\end{proof}

The interaction of substitutions with signature maps is similarly straightforward: signature maps act functorially on raw substitutions, and the constructions of this section are natural with respect to the action. More precisely:

\begin{propositionwithqed}
  Given a signature map $F : \Sigma \to \Sigma'$, and a raw substitution $f : \delta \to \gamma$ over $\Sigma$, there is a raw substitution $\act{F} f : \delta \to \gamma$ over $\Sigma'$ given by $(\act{F} f)(i) = \act{F}(f(i))$, and the action respects composition and identities in $F$.
  %
  Moreover, given such $F$, for all suitable $f$ and $e$, we have $\act{F}(\tca{f} e) = \tca{(\act{F}f)} (\act{F}e)$; similarly, for all suitable $f$, $g$, we have $\act{F}(f \circ g) = \act{F}f \circ \act{F}g$. 
\end{propositionwithqed}

\subsection{Metavariable extensions and instantiations}
\label{sec:metas-and-instantiations}

As mentioned in the introduction, when we write down the \emph{rules} of type theories, we will need to extend the ambient signature~$\Sigma$ by new symbols to represent the metavariables of the rule.

For instance, consider the constructor $\symPi$, with arity $[(\Ty,0),(\Ty,1)]$.
%
In the rule for~$\symPi$ formation (\cref{ex:symbol-pi-arity}), we shall use two new symbols $\symA$, $\symB$, corresponding to the arguments of~$\symPi$ in the premises of the rule.
%
The classes and arities of these new symbols are given by their classes and binders as arguments of~$\symPi$: they are both type symbols; the first one takes no arguments, and the second one takes one term argument.

\begin{definition}
  \label{def:simple-arity}%
  The \defemph{simple arity~$\simplearity \gamma$} of a scope~$\gamma$ is the arity indexed by the positions~$\position{\gamma}$, and whose arguments all have syntactic class $\Tm$ with no binding, i.e., $\simplearity \gamma \defeq \famtuple{(\Tm, \emptyscope)}{i \in \position{\gamma}}$.
\end{definition}

\begin{definition}
  \label{def:metavariable-extensions}%
  The \defemph{metavariable extension $\mvextend{\Sigma}{\alpha}$ of~$\Sigma$ by arity $\alpha$} is the signature indexed by $\famindex{\Sigma} + \args{\alpha}$, defined by
  %
  \begin{equation*}
    (\mvextend{\Sigma}{\alpha})_{\inl(S)} \defeq \Sigma_S
    \qquad\text{and}\qquad
    (\mvextend{\Sigma}{\alpha})_{\inr(i)} \defeq (\argclass{\alpha}{i}, \simplearity (\argbinder{\alpha}{i})).
  \end{equation*}
  %
  We usually treat the injection of symbols $\Sigma \to \mvextend{\Sigma}{\alpha}$ as an inclusion, writing $S$ instead of $\inl(S)$;
  %
  and for each $i \in \args{\alpha}$, we write $\synmeta{i}$ for the corresponding new symbol $\inr(i)$ of $\mvextend{\Sigma}{\alpha}$, the \defemph{metavariable symbol} for $i$.
\end{definition}

\begin{example}
  The symbol $\symlambda$ has arity
  %
  $
    [(\Ty,0), (\Ty,1), (\Tm,1)]
  $.
  %
  It thus gives rise to the metavariable extension $\Sigma + [(\Ty,0), (\Ty,1), (\Tm,1)]$, which adjoins to~$\Sigma$ three metavariable symbols $\synmeta{0}$, $\synmeta{1}$, $\synmeta{2}$, which for readability we may rename to $\symA$, $\symB$, $\symb{t}$, with classes and arities $(\Ty, [])$, $(\Ty, [(\Tm, 0)])$, and $(\Tm, [(\Tm,0)])$ respectively.
  %
  That is, $\symA$ and $\symB$ are type symbols and $\symb{t}$ a term symbol, with the latter two each taking a term argument.
  %
  These will then appear in the rule for $\symlambda$, to denote the arguments of a generic instance of $\symlambda$.
  %
  We will often use more readable names for metavariable symbols, as here, without further comment.
\end{example}

Let $\gamma$ be a scope and $\synmeta{i}(e)$ a raw expression over signature $\mvextend{\Sigma}{\alpha}$ and~$\gamma$.
Then~$e$ is a map which assigns to each $j \in \position{\argbinder{\alpha}{i}}$ an expression in $\ExprTm{\mvextend{\Sigma}{\alpha}}{\gamma}$ because $\sumscope{\gamma}{\emptyscope} = \gamma$.
Thus we may construe $e$ as a raw substitution $e : \argbinder{\alpha}{i} \to \gamma$.
With this in mind, the following definition explains how metavariables are replaced with expressions.

\begin{definition}
  \label{def:instantiation}%
  Given a signature~$\Sigma$, an arity $\alpha$, and a scope~$\gamma$, an \defemph{instantiation~$I \in \Inst \Sigma \gamma \alpha$} of $\alpha$ in scope $\gamma$ is a family of expressions $I_i \in \Expr{\argclass{i}{\alpha}}{\Sigma}{\sumscope{\gamma}{\argbinder{i}{\alpha}}}$, for each $i \in \args{\alpha}$.
  %
  Such an instantiation acts on expressions $e \in \Expr{c}{\mvextend{\Sigma}{\alpha}}{\delta}$ to give expressions $\act{I} e \in \Expr{c}{\Sigma}{\sumscope{\gamma}{\delta}}$, by
replacing each occurrence of $\synmeta{i}$ in $e$ with a copy of~$I_i$, with the arguments of $\synmeta{i}$ recursively substituted for the corresponding variables in $I_i$:
%
\begin{align*}
  \act{I} (\synmeta{i}(e)) &\defeq \tca{(\sumscope{\gamma}{e})} I_i, \\
  \act{I} (\synvar{j})     &\defeq \synvar{\inr(j)}, \\
  \act{I} (S(e))           &\defeq S \, (\fammap{\act{I} e_j}{j \in \args{S}}).
\end{align*}
  We call $\act{I}e$ the \defemph{instantiation of $e$ with $I$}.
\end{definition}

\begin{example} \label{ex:app-instantiation}
  Anticipating \cref{ex:raw-rule-app}, the rule for function application will be written as follows, where for readability $x$ stands for $\synvar{0}$:
  \begin{equation*}
  \inferrule
    { \istype{ }{\symA} \\
      \istype{x \of \symA}{\symB(x)} \\
      \isterm{}{\symb{s}}{\symPi(\symA, \symB(x)}) \\
      \isterm{}{\symb{t}}{\symA}
    }{
      \isterm{}{\symapp(\symA, \symB(x), \symb{s}, \symb{t})}{\symB(\symb{t})}
    }
  \end{equation*}
  %
  All expressions are in the metavariable extension $\mvextend{\Sigma}{\arity{\symapp}}$, where $\Sigma$ is some ambient signature including $\symPi$ and $\symapp$, and $\arity{\symapp} = [(\Ty,0), (\Ty,1), (\Tm,0),(\Tm,0)]$ as in \cref{ex:pi-types-arities}.
  %
  The symbols $\symA$, $\symB$, $\symb{s}$, $\symb{t}$ are the metavariable symbols of this extension.
  %
  An instantiation of $\arity{\symapp}$ in scope $\gamma$ consists of expressions $A \in \Expr{\Ty}{\Sigma}{\gamma}$, $B \in \Expr{\Ty}{\Sigma}{\sumscope{\gamma}{\singletonscope}}$, and $s, t \in \Expr{\Tm}{\Sigma}{\gamma}$.
  %
  Instantiating the conclusion with these expressions, over some context $\Gamma$, gives the judgement $\isterm{\Gamma}{\symapp(A, B, s, t)}{B[t/x]}$.
\end{example}

Building on the above, instantiations also act on other objects built out of expressions, including substitutions and other instantiations;
%
at the same time, being themselves syntactic objects, instantiations are acted upon by substitutions and signature maps;
%
and all of these are suitably natural and functorial, as follows.

\begin{definition}
  \label{def:instantiation-actions}%
  %
  Given an instantiation $I \in \Inst{\Sigma}{\gamma}{\alpha}$:
  \begin{enumerate}

  \item The \defemph{instantiation $I$ acts on a substitution} $f : \delta' \to \delta$ over $\mvextend{\Sigma}{\alpha}$ to give a substitution $\act{I} f : \sumscope{\gamma}{\delta'} \to \sumscope{\gamma}{\delta}$, defined by
    \begin{align*}
      (\act{I}f)(\inlscope i) & \defeq \inlscope i & (i \in \gamma), \\
      (\act{I}f)(\inrscope j) & \defeq \act{I}(f j) & (j \in \delta).
    \end{align*}

  \item The \defemph{instantiation $I$ acts on an instantiation} $J \in \Inst{\mvextend{\Sigma}{\alpha}}{\delta}{\beta}$ to give an instantiation $\act{I}J \in \Inst{\Sigma}{\sumscope{\gamma}{\delta}}{\beta}$, defined by $(\act{I}J)_j \defeq \act{I}(J_j)$.  

  \item A \defemph{substitution} $f : \delta \to \gamma$ over $\Sigma$ \defemph{acts on the instantiation~$I$} to give an instantiation $\tca{f} I \in \Inst{\Sigma}{\delta}{\alpha}$, defined by $(\tca{f}I)_i \defeq \tca{(\sumscope{f}{\argbinder{\alpha}{i}})}I_i$.

  \item The \defemph{instantiation $I$ is translated along a signature map $F : \Sigma \to \Sigma'$} to give an instantiation $\act{F} I \in \Inst{\Sigma'}{\gamma}{\alpha}$, defined by $(\act{F}I)_i \defeq \act{F}(I_i)$.
  \end{enumerate}
\end{definition}

\begin{propositionwithqed}
  \label{prop:instantiation-boilerplate}%
  %
  For all suitable instantiations $I$, $J$, signatures maps $F$, $G$, substitutions $f$, $g$, arities $\alpha$, expressions $e$, and scopes $\gamma$, $\delta$, $\theta$:
  %
  \begin{enumerate}

  \item
    %
    Translation along signature maps is functorial:
    %
    % F : Σ → Σ'
    % G : Σ' → Σ''
    % I ∈ Inst Σ γ α
    % F_* I ∈ Inst Σ' γ α
    % G_* (F_* I) ∈ Inst Σ'' γ α
    \begin{equation*}
       \act{(G \circ F)} I = \act{G} (\act{F} I)
      \qquad\text{and}\qquad
      \act{{\idmap[\Sigma]}} I = I.
    \end{equation*}

  \item
    %
    The actions of substitutions and instantiations on expressions and on each other are natural with respect to translation along signature maps:
    %
    % e ∈ Exprᶜ (Σ + α) δ
    % I ∈ Inst Σ γ α
    % I_* e ∈ Exprᶜ Σ (γ ⊕ δ)
    % F : Σ → Σ'
    % F_* (I_* e) ∈ Exprᶜ Σ' (γ ⊕ δ) -- LHS 1
    % F + α : Σ + α → Σ' + α'
    % (F + α)_* e ∈ Exprᶜ (Σ' + α) δ
    % F_* I ∈ Inst Σ' γ α
    % (F_* I)_* ((F + α)_* e) ∈ Exprᶜ (Σ' + α) (γ + δ) -- RHS 1
    \begin{align*}
      \act{F}(\act{I} e) & = \act{(\act{F} I)} (\act{(\mvextend{F}{\alpha})} e), \\
      \act{F}(\act{I} f) & = \act{(\act{F} I)} (\act{(\mvextend{F}{\alpha})} f), \\
      \act{F}(\act{I} J) & = \act{(\act{F} I)} (\act{(\mvextend{F}{\alpha})} J), \\
      \act{F}(\tca{f} I) & = \tca{(\act{F} f)} (\act{F} I).
    \end{align*}

  \item
    %
    The action of substitutions on instantiations is functorial:
    %
    \begin{equation*}
      \tca{(f \circ g)} I = \tca{g} (\tca{f} I)
      \qquad\text{and}\qquad
      \tca{{\idmap[\gamma]}} I = I.
    \end{equation*}

  \item
    %
    The action of instantiations is natural with respect to substitutions:
    \begin{align*}
      % instantiate_substitute_instantiation:
      % I ∈ Inst Σ γ α
      % f : ξ → γ over Σ
      % f^* I ∈ Inst Σ ξ α
      % e ∈ Exprᶜ (Σ + α) δ
      % (f^* I)_* e ∈ Exprᶜ Σ (ξ ⊕ δ) -- RHS
      % I_* e ∈ Exprᶜ Σ (γ ⊕ δ)
      % f ⊕ δ : ξ ⊕ δ → γ ⊕ δ
      % (f ⊕ δ)^* (I_* e) ∈ Exprᶜ Σ (ξ ⊕ δ) -- LHS
      \act{(\tca{f} I)} e & = \tca{(\sumscope{f}{\delta})} (\act{I} e), \\
      % instantiate_substitute:
      % e ∈ Epxrᶜ (Σ + α) γ
      % g : δ → γ over (Σ + α)
      % g^* e ∈ Exprᶜ (Σ + α) δ
      % I ∈ Inst Σ ξ α
      % I_* (g^* e) ∈ Epxrᶜ Σ (ξ ⊕ δ) -- LHS
      % I_* e ∈ Exprᶜ Σ (ξ ⊕ γ)
      % I_* g : ξ ⊕ δ → ξ ⊕ γ over Σ
      % (I_* g)^* (I_* e) ∈ Epxrᶜ Σ (ξ ⊕ δ) -- RHS
      \act{I} (\tca{g} e) &= \tca{(\act{I} g)} (\act{I} e).
    \end{align*}

  \item
    %
    The action of instantiations is “associative” in the sense that
    % J ∈ Inst (Σ + α) δ β
    % J_* e ∈ Exprᶜ (Σ + α) (δ ⊕ ε) -- elided inclusion of e into (Σ + α) + β
    % I ∈ Inst Σ α γ
    % J ∈ Inst (Σ + α) β δ
    % e ∈ Exprᶜ (Σ + β) ε
    % (ι₀ + β) : (Σ + β) → ((Σ + α) + β)
    % (ι₀ + β)_* e ∈ Exprᶜ ((Σ + α) + β) ε
    % J_* ((ι₀ + β)_* e) ∈ Exprᶜ (Σ + α) (δ ⊕ ε)
    % I_* (J_* e) ∈ Exprᶜ Σ (γ ⊕ (δ + ε)) -- LHS
    % I_* J ∈ Inst Σ (γ ⊕ δ) β
    % (I_* J)_* e ∈ Exprᶜ Σ ((γ ⊕ δ) ⊕ ε) -- RHS
    \begin{equation*}
      % instantiate_instantiate_expression:
      \act{(\act{I} J)} e = \act{I}(\act{J} (\act{(\mvextend{\inl}{\beta})} e))
    \end{equation*}
    %
    holds modulo the canonical associativity isomorphism between the scopes $\sumscope{(\sumscope{\gamma}{\delta})}{\theta}$ and $\sumscope{\gamma}{(\sumscope{\delta}{\theta})}$ of the left- and right-hand sides. \qedhere
  \end{enumerate}
\end{propositionwithqed}

The above properties, while routine to prove, are a lot of boilerplate.
%
They can be incorporated, if desired, into the statement that syntax forms a \emph{scope-graded monad} on signatures in the sense of~\citep{orchard-wadler-eades}, and that instantiations are certain Kleisli maps for this monad.
%
As ever, however, we emphasise in this paper the elementary viewpoint, rather than categorical abstraction.

%%% Local Variables:
%%% mode: latex
%%% End:

\section{Raw type theories}
\label{sec:typing}
% not “sec:raw-type-theories” since that would be the more specific subsection below

Having described raw syntax, we proceed with the formulation of raw type theories. These hold enough information to prevent syntactic irregularities, and can be used to specify derivations and derivability, but are qualified as ``raw'' because they allow arbitrariness and abnormalities that are generally considered undesirable.

\subsection{Raw contexts}

\begin{definition}
  \label{def:raw-context}%
  A \defemph{raw context~$\Gamma$} is a scope~$\gamma$ together with a family on $\ExprTy{\Sigma}{\gamma}$ indexed by $\position \gamma$, i.e., a map $\position{\gamma} \to \ExprTy{\Sigma}{\gamma}$.
  %
  The positions of $\gamma$ are also called the \defemph{variables} of~$\Gamma$.
  % 
  We often write just~$\Gamma$ for~$\gamma$, e.g., $i \in \position{\Gamma}$ instead of $i \in \position{\gamma}$, and $\Expr{c}{\Sigma}{\Gamma}$ instead of $\Expr{c}{\Sigma}{\gamma}$.
  % 
  We use a subscript for the application of a context $\Gamma$ to a variable $i$, such that $\Gamma_i$ is the type expression at index $i$.
\end{definition}

This definition is somewhat non-traditional for dependent type theories, in a couple of ways.
%
Contexts are more usually defined as lists, so their variables are ordered, and the type of each variable is assumed to depend only on earlier variables, i.e.~$A_i \in \ExprTy{\Sigma}{i}$.
%
In our definition, the variables form an arbitrary scope, with no order assumed; and each type may \emph{a priori} depend on any variables of the context.

One may describe the two approaches as \emph{sequential} and \emph{flat} contexts, respectively.
%
The flat notion contains all the information needed when contexts are \emph{used} in derivations; we view the sequentiality as extra information that may be provided later by a derivation of well-formedness of a context, cf.\ \cref{sec:sequential-contexts}, but that is not needed at the raw level.

\begin{example}%
  \label{ex:de-bruijn-context}
  With the de Bruijn index scope system, a raw context $\Gamma$ of scope $n$ may be written as
  $\famtuple{A_i \in \ExprTy \Sigma n}{i \in \position n}$. A more familiar way to display $\Gamma$ is as list $[(n-1) \mto A_{n-1}, \ldots, 0 \mto A_0]$, where each $A_i \in \ExprTy{\Sigma}{n}$, but as raw contexts follow the flat approach, we should not think of this as imposing an order on the variables, and hence the list $[0 \mto A_0, \ldots, (n-1) \mto A_{n-1}]$ denotes the same context $\Gamma$.
  %
  Note, by the way, that at this stage contexts on de Bruijn indices are indistinguishable from contexts on de Bruijn levels.
  %
  The difference becomes apparent once we consider context extension, and the scope coproduct inclusions come into play.
\end{example}

\begin{definition}%
  \label{def:context-extension}%
  Let $\Gamma$ be a raw context, $\delta$ be a scope, and $\Delta : \position{\delta} \to \ExprTy{\Sigma}{\sumscope {\position{\Gamma}} {\delta}}$ a family of expressions indexed by $\position \delta$.
  %
  The \defemph{context extension}~$\ctxextend \Gamma \Delta$ is the raw context of scope~$\sumscope {\position \Gamma} \delta$, defined as
  %
  \begin{equation*}
    (\ctxextend \Gamma \Delta)_{\inlscope j} \defeq (\act \inlscope \circ\, \Gamma)_j
    \qquad\text{and}\qquad
    (\ctxextend \Gamma \Delta)_{\inrscope k} \defeq \Delta_k.
  \end{equation*}
  %
  In other words, the extended raw context~$\ctxextend \Gamma \Delta$ is the map $[\act \inlscope \circ\, \Gamma, \Delta]$ induced by the universal property of the coproduct~$\position{\sumscope {\position{\Gamma}} \delta}$.
\end{definition}

\begin{example}
To continue \cref{ex:de-bruijn-context}, we can consider how context extension works for de Bruijn indices. Let $\Gamma = [2 \mto A_2, 1 \mto A_1, 0 \mto A_0]$, where each $A_i \in \ExprTy{\Sigma}{3}$, and $\Delta = [1 \mto B_1, 0 \mto B_0]$ with $B_j \in \ExprTy{\Sigma}{3 + 2}$. The coproduct inclusions are $\inlscope i = i + 2$ and $\inrscope j = j$. The context $\ctxextend \Gamma \Delta$ has scope $3 + 2 = 5$, and is given by $[\act {(i \mapsto i + 2)} \circ\, \Gamma, \Delta]$, which computes to $[4 \mto \act \inlscope A_2, 3 \mto \act \inlscope A_1, 2 \mto \act \inlscope A_0, 1 \mto B_1, 0 \mto B_0]$. The effect is that the variables from $\Gamma$ are renamed according to the scope of $\Delta$, and the renaming acts on the associated type expressions accordingly, i.e., by shifting all variables by 2.
\end{example}

Note that with raw contexts, we weaken types when extending a context.
%
In approaches using sequential contexts with scoped syntax, weakening is instead performed when types are taken \emph{out} of a context: that is, the variable rule (precisely stated) concludes $\isterm{\Gamma}{\synvar{i}}{\rename{\iota}\Gamma_i}$, where $\iota : i \to n$ is the inclusion of an initial segment into the full context.

In concrete examples, we will write contexts in a more traditional style, as lists of variables names with their associated types:
\[  x_1 \of A_1, \ldots, x_n \of A_n. \]

Like other syntactic objects, raw contexts are acted upon by signature maps and instantiations. The functoriality and commutation properties for these actions follow directly from the corresponding properties for expressions.

\begin{definition}
  Given a signature map $F : \Sigma \to \Sigma'$ and a raw context $\Gamma$ over $\Sigma$, the \defemph{translation of $\Gamma$ by $F$} is the raw context $\act{F} \Gamma$ over $\Sigma'$, with $\position{\act{F} \Gamma} \defeq \position{\Gamma}$ and $(\act{F}\Gamma)_i \defeq \act{F}\Gamma_i$. 
\end{definition}

\begin{propositionwithqed}
  The action of signature maps on raw contexts is functorial:
  %
  \begin{equation*}
    \act{F}(\act{G} \Gamma) = \act{(F \circ G)}\Gamma
    \qquad\text{and}\qquad
    \act{{\idmap[\Sigma]}} \Gamma = \Gamma,
  \end{equation*}
  %
  for all suitable $F$, $G$, $\Gamma$.
\end{propositionwithqed}

The action of instantiations is a little more subtle.
%
Acting pointwise on the expressions of the context  is not enough, since the instantiated expressions lie in a larger scope.
%
We need to supply extra types for the extra scope, i.e., the scope of the instantiation must itself underlie a raw context.

\begin{definition}
  An \defemph{instantiation $I$ in context $\Gamma$} over $\Sigma$ for arity $\alpha$ is an instantiation $I \in \Inst{\Sigma}{\position \Gamma}{\alpha}$.
  %
  Then for any raw context $\Delta$ over $\mvextend{\Sigma}{\alpha}$, the \defemph{context instantiation $\act{(I,\Gamma)}\Delta$} is the context over $\Sigma$ with scope $\sumscope{\position{\Gamma}}{\position{\Delta}}$ and with type expressions
  \begin{align*}
    \act{(I,\Gamma)}\Delta_{\inlscope(i)} &\defeq \tca{\inlscope}\Gamma_i & (i \in \gamma), \\
    \act{(I,\Gamma)}\Delta_{\inrscope(j)}&\defeq \act{I}\Delta_j & (j \in \delta).
  \end{align*}

  Briefly, $\act{(I,\Gamma)}\Delta$ is the context extension of $\Gamma$ by the instantiations of the types of $\Delta$.
\end{definition}

\begin{propositionwithqed}
  Given instantiations 
  %
  \begin{align*}
    &\text{$I \in \Inst{\Sigma}{\position \Gamma}{\alpha}$ in context $\Gamma$,}\\
    &\text{$K \in \Inst{\mvextend{\Sigma}{\alpha}}{\position \Delta}{\beta}$ in context $\Delta$, and}\\
    &\text{$\Theta \in \Inst{\mvextend{(\mvextend{\Sigma}{\alpha})}{\beta}}{\position \Theta}{\gamma}$ in context $\Theta$},
  \end{align*}
  %
  the equation
  %
  \begin{equation*}
    \act{(I,\Gamma)} (\act{(K,\Delta)} \Theta) =
    \act{(\act{I}K, \act{(I,\Gamma)} \Delta)} \Theta
  \end{equation*}
  %
  holds modulo the canonical isomorphism between the scopes
  $\sumscope{\position \Gamma}{(\sumscope{\position \Delta}{\position \Theta})}$ and $\sumscope{(\sumscope{\position \Gamma}{\position \Delta})}{\position \Theta}$ of their right- and left-hand sides.
  % OLD FORMULATION:
  % Given instantiations $I$ of $\alpha$ over $\Sigma$ in context $\Gamma$ and $K$ of $\beta$ over $\mvextend{\Sigma}{\alpha}$ in context $\Delta$, and a further context $\Theta$ over $\mvextend{(\mvextend{\Sigma}{\alpha})}{\beta}$, we have $\act{(I,\Gamma)}(\act{(K,\Delta)}{\Theta}) = \act{(\act{I}K,\act{(I,\Gamma)}\Delta)}\Theta$, modulo the associativity renaming between their scopes $\sumscope{\gamma}{(\sumscope{\delta}{\theta})}$, $\sumscope{(\sumscope{\gamma}{\delta})}{\theta}$.
\end{propositionwithqed}

\subsection{Judgements}

Our type theories have four primitive judgement forms, following Martin-Löf \citep{martin-lof:bibliopolis}:
%
\begin{center}
\begin{tabular}{ll}
  $A \type$           &\qquad ``$A$ is a type'' \\
  $t : A$             &\qquad ``term $t$ has type $A$'' \\
  $A \judgeq B$       &\qquad ``$A$ and $B$ are equal as types''  \\
  $s \judgeq t : A$   &\qquad ``$s$ and $t$ are equal as terms of type $A$''
\end{tabular}
\end{center}
%
These are represented with symbols $\Ty$, $\Tm$, $\TyEq$, and $\TmEq$, respectively.
%
For our needs we need to describe the judgement forms quite precisely. In fact, the following elaboration may seem a bit \emph{too} precise, but we found it quite useful in the formalisation to make explicit all the concepts involved and distinctions between them.

Each judgement form has a family of \defemph{boundary slots} and possibly a \defemph{head slot}, where each slot has an associated syntactic class, as follows:
%
\begin{center}
\begin{tabular}{lll}
  Form & Boundary & Head \\
  \midrule
  $\Ty$ & $[]$ & $\Ty$ \\
  $\Tm$ & $[\Ty]$ & $\Tm$ \\
  $\TyEq$ & $[\Ty, \Ty]$ & \\
  $\TmEq$ & $[\Tm, \Tm, \Ty]$ & \\
\end{tabular}
\end{center}
%
The table encodes the familiar constituent parts of the judgement forms:
%
\begin{enumerate}
\item ``$A \type$'' has no boundary slots; the head slot, indicated by $A$, is a type.
\item ``$t : A$'' has one boundary type slot indicated by $A$, called the \defemph{underlying type}; the head, indicated by $t$, is a term.
\item ``$A \equiv B$'' has two type slots indicated by $A$ and $B$, called the \defemph{left-hand side} and \defemph{right-hand side}; there is no head.
\item ``$s \equiv t : A$'' has two term slots indicated by $s$ and $t$, and a type slot indicated by~$A$, called the \defemph{left-hand side}, the \defemph{right-hand side} and the \defemph{underlying type}, respectively; there is no head.
\end{enumerate}
%
The \defemph{slots} of a judgement form are the slots of its boundary, and the head, if present.

\begin{definition}
  \label{def:judgement}
  %
  Given a raw context $\Gamma$ and a judgement form~$\phi$, a \defemph{hypothetical judgement} of that form over~$\Gamma$ is a map $J$ taking each slot of $\phi$ of syntactic class $c$ to an element of $\Expr{c}{\Sigma}{\Gamma}$.
  %
  We write $\Judg{\Sigma}$ for the set of all hypothetical judgements over~$\Sigma$.
  %
  The types of $\Gamma$ are the \defemph{hypotheses} and $J$ is the \defemph{thesis} of the judgement.
  %
  When there is no ambiguity, we will (following traditional usage) speak of a \defemph{judgment} to mean either a whole hypothetical judgement $\Gamma \typesjudgement J$, or just a thesis $J$.
 
  A hypothetical judgement is an \defemph{object judgement} if it is a term or a type judgement, and an \defemph{equality judgement} if it is a type or a term equality judgement.
\end{definition}

\begin{example}
  The boundary and the head slots of the judgement form $\Tm$ are $[\Ty]$ and $\Tm$, respectively. Thus a hypothetical judgement of this form over a raw context~$\Gamma$ is a map taking the slot in the boundary to a type expression $A \in \ExprTy{\Sigma}{\Gamma}$ and the head slot to a term expression $t \in \ExprTm{\Sigma}{\Gamma}$.
  %
  This corresponds precisely to the information conveyed by a traditional hypothetical term judgement ``$\isterm{\Gamma}{t}{A}$''.
\end{example}

In view of the preceding example we shall write a hypothetical judgement over $\Gamma$ given by a map $J$ in the traditional type-theoretic way
%
\begin{equation*}
  \Gamma \typesjudgement J
\end{equation*}
%
where the elements of $J$ are displayed in the corresponding slots.

Just like raw contexts, judgements are acted on by signature maps and instantiations.

\begin{definition} \label{def:judgements-functorial}
  Given a signature map $F : \Sigma \to \Sigma'$ and a judgement $\Gamma \typesjudgement J$ over $\Sigma$, the translation $\act{F}(\Gamma \typesjudgement J)$ is the judgement $\act{F} \Gamma \typesjudgement \act{F} J$ over $\Sigma'$ of the same form, where the thesis $\act{F} J$ is $J$ with $\act{F}$ applied pointwise to each expression.
\end{definition}

\begin{propositionwithqed} \label{prop:judgements-functorial}
  This action is functorial: $\act{F}(\act{G} (\Gamma \typesjudgement J)) = \act{(FG)}(\Gamma \typesjudgement J)$, and $\act{(\idmap[\Sigma])}(\Gamma \typesjudgement J) = (\Gamma \typesjudgement J)$, for all suitable $F$, $G$, $\Gamma$, $J$.
\end{propositionwithqed}

\begin{definition} \label{def:instantiate-judgement}
  Given a signature $\Sigma$, a hypothetical judgement $\Delta \typesjudgement J$ over a metavariable extension $\mvextend \Sigma \alpha$, a raw context $\Gamma$ over $\Sigma$, and an instantiation~$I$ of $\alpha$ in context $\Gamma$, the \defemph{judgement instantiation} $\act{(I,\Gamma)}{(\Delta \typesjudgement J)}$ is the judgement
  $\ctxextend{\Gamma}{\act{I} \Delta} \typesjudgement {\act I J}$ over $\Sigma$, where the thesis $\act I J$ is just $J$ with $\act{I}$ applied pointwise.
\end{definition}

\begin{propositionwithqed} \label{prop:instantiate-instantiate-judgement}
   Given instantiations
   %
   \begin{align*}
     & \text{$I \in \Inst{\Sigma}{\position \Gamma}{\alpha}$ in context $\Gamma$ and}\\
     & \text{$K \in \Inst{\mvextend{\Sigma}{\alpha}}{\position \Delta}{\beta}$ in context $\Delta$},
   \end{align*}
   and a judgement $\Theta \typesjudgement J$ over $\mvextend{(\mvextend{\Sigma}{\alpha})}{\beta}$, the equation
   %
   \begin{equation*}
     \act{(I,\Gamma)}(\act{(K,\Delta)}{(\Theta \types J)}) = \act{(\act{I}K,\act{(I,\Gamma)}\Delta)}(\Theta \types J)
   \end{equation*}
   %
   holds modulo the canonical associativity renaming between their scopes.
\end{propositionwithqed}

\subsection{Boundaries}
\label{sec:boundaries}

In many places, one wants to consider data amounting to a hypothetical judgement without a head expression (if it is of object form, and so should have a head).
%
For instance, a \emph{goal} or \emph{obligation} in a proof assistant is specified by such data;
%
or when adjoining a new well-formed rule to a type theory, before picking a fresh symbol for it (if it is an object rule), the conclusion is specified by such data. 

These entities crop up frequently, and seem almost as fundamental as judgements, so deserve a name.

\begin{definition}
  Given a raw context $\Gamma$ and a judgement form~$\phi$, a \defemph{hypothetical boundary} of form $\phi$ over~$\Gamma$ is a map $B$ taking each \emph{boundary} slot of~$\phi$ of syntactic class $c$ to an element of  $\Expr{c}{\Sigma}{\Gamma}$.
\end{definition}

We display boundaries as judgements with a hole~$\bdryhead$ where the head should stand, or with $\qjudgeq$ in place of $\judgeq$:
%
\begin{equation*}
  \bdryhead \type
  \qquad\qquad
  \bdryhead : A
  \qquad\qquad
  A \qjudgeq B
  \qquad\qquad
  s \qjudgeq t : A
\end{equation*}
%
Since equality judgements have no heads, there is no difference in data between an equality judgement and an equality boundary, but there is one of sense: the judgement $A \judgeq B$ asserts an equality holds, whereas the boundary $A \qjudgeq B$ is a goal to be established or postulated.
%
Analogously,  $\bdryhead \type$ and $\bdryhead : A$ can be read as goals, the former that a type be constructed, and the latter that~$A$ be inhabited.

The terminology about judgements, as well as many constructions, carry over to boundaries.
%
In particular, the action of signature maps and instantiations on boundaries is defined just as in \cref{def:judgements-functorial,def:instantiate-judgement}, and enjoys analogous properties to \cref{prop:judgements-functorial,prop:instantiate-instantiate-judgement}.

Finally, and crucially, boundaries can be completed to judgements.
%
The data required depends on the form: completing an object boundary requires a head expression; completing an equality boundary, just a change of view.
%
\begin{definition} \label{def:completion-boundary}
  Let $B$ be a boundary in scope $\gamma$ over $\Sigma$.
  \begin{enumerate}
  \item If $B$ is of object form, then given an expression $e$ of the class of $B$ in scope $\gamma$, write $\plug{B}{e}$ for the \defemph{completion} of $B$ with head $e$, a judgement over $\gamma$.
  \item If $B$ is of equality form, then the completion of $B$ is just $B$ itself, viewed as a judgement.
  \end{enumerate}
\end{definition}

\begin{propositionwithqed}
  Completion of boundaries is natural with respect to signature maps: $\act{F}(\plug{B}{e}) = \plug{(\act{F}B)}{\act{F}e}$.
\end{propositionwithqed}

\subsection{Raw rules} \label{sec:raw-rules}

We now come to raw rules, syntactic entities that capture the notion of ``templates'' that are traditionally used to display the inference rules of a type theory.
%
The raw rules include all the information needed in order to be \emph{used}, for defining derivations and derivability of judgements --- but they do not yet include the extra properties we typically check when considering rules, and which guarantee good properties of the resulting derivability predicates.
%
We return to these later, in \cref{sec:acceptable-rules}.

\begin{definition}
  \label{def:raw-rule}%
  A \defemph{raw rule} $R$ over a signature $\Sigma$ consists of an arity $\arity{R}$,
  %
  together with a family of judgements over the extended signature $\mvextend{\Sigma}{\arity{R}}$, the \defemph{premises} of~$R$,
  %
  and one more judgement over $\mvextend{\Sigma}{\arity{R}}$, the \defemph{conclusion} of $R$.
  %
  An \defemph{object rule} is one whose conclusion is an object judgement, otherwise it is an \defemph{equality rule}.
\end{definition}

\begin{example}%
\label{ex:raw-rule-app}
  Following on from \cref{ex:pi-types-arities}, the raw rule for function application has arity
  %
  \begin{equation*}
    \arity{\symapp} = [(\Ty,0), (\Ty,1), (\Tm,0),(\Tm,0)].
  \end{equation*}
  %
  Writing $\symA$, $\symB$, $\symb{s}$, $\symb{t}$ for the metavariable symbols of the extended signature $\mvextend{\Sigma}{\arity{\symapp}}$, the premises of the rule are the four-element family:
  %
  \begin{equation*}
    [\;
    \istype{}{\symA}, \quad
    \istype{[0 \mto \symA]}{\symB(\synvar{0})}, \quad
    \isterm{}{\symb{s}}{\symPi(\symA, \symB(\synvar{0})}), \quad
    \isterm{}{\symb{t}}{\symA}
    \;] 
  \end{equation*}
  %
  and its conclusion is the hypothetical judgement
  %
  \begin{equation*}
    \isterm{}{\symapp(\symA, \symB(\synvar{0}), \symb{s}, \symb{t})}{\symB(\symb{t})}.
  \end{equation*}
  %
  Of course, the traditional type-theoretic way of displaying such a rule is
  %
  \begin{equation*}
  \inferrule
    { \istype{ }{\symA} \\
      \istype{x \of \symA}{\symB(x)} \\
      \isterm{}{\symb{s}}{\symPi(\symA, \symB(x)}) \\
      \isterm{}{\symb{t}}{\symA}
    }{
      \isterm{}{\symapp(\symA, \symB(x), \symb{s}, \symb{t})}{\symB(\symb{t})}
    }
  \end{equation*}
  %
  It may seem surprising that we have $\symB(x)$ and $\symB(\symb{t})$ rather than, say, $B$ and $B[\symb{t}/x]$, since this style is usually apologised for as an abuse of notation.
  %
  Here, it is precise and formal; $\symB$ is a metavariable \emph{symbol} in $\mvextend{\Sigma}{\arity{\symapp}}$, so it is applied to arguments when used in the syntax.
  %
  Also note that the occurrences of~$x$ in the third premise and the conclusion are implicitly bound by~$\symPi$ and~$\symapp$, as can be discerned from their arities.
  %
  When we instantiate the rule below with actual expressions $A$, $B$, $s$ and~$t$, then $\symB(x)$ and $\symB(\symb{t})$ will be translated into $B$ and $B[t/x]$ respectively.
\end{example}

\begin{definition}%
  \label{def:raw-rule-fmap}
  The functorial \defemph{action} of a signature map $F : \Sigma \to \Sigma'$ is the map $\act{F}$ which takes a rule $R$ over $\Sigma$ to the rule~$\act{F} R$ over~$\Sigma'$ whose arity is the arity~$\arity{R}$ of~$R$, and its premises and conclusion are those of $R$, all translated along the action of the induced map $\mvextend{F}{\arity{R}} : \mvextend{\Sigma}{\arity{R}} \to \mvextend{\Sigma'}{\arity{R}}$.
\end{definition}

A raw rule should not itself be thought of as a closure rule (though formally it is one), but rather as a template specifying a whole family of closure rules.

\begin{definition}
  \label{def:induced-closure-rule}\label{def:associated-closure-system}
  Given a rule $R$, a raw context $\Gamma$, and an instantiation $I$ of its arity~$\arity{R}$ over $\Gamma$, all over a signature~$\Sigma$, the \defemph{instantiation of $R$ under $I$, $\Gamma$}, is the closure rule $\act{(I,\Gamma)} R$ on $\Judg{\Sigma}$ whose premises and conclusion are the instantiations of the corresponding judgements of~$R$ under~$I$, $\Gamma$.
  %
  The \defemph{closure system $\clos R$ associated to $R$} is the family
  $\famtuple{\act{(I,\Gamma)} R}{\Gamma \in \Context{\Sigma},\, I \in \Inst{\Sigma}{\Gamma}{\arity R}}$
  of all such instantiations.
\end{definition}

\begin{example}
  Continuing \cref{ex:raw-rule-app}, the raw rule $R_\symapp$ for application gives the closure system $\clos(R_\symapp)$, containing for each raw context~$\Gamma$ (with scope~$\gamma$) and expressions $A, B \in \ExprTy{\Sigma}{\gamma}$, $s, t \in \ExprTm{\Sigma}{\gamma}$ a closure rule
   %
   \begin{align*}
    \inferrule
     { \istype{\Gamma}{A} \\
       \istype{\Gamma, x \of A}{B} \\
       \isterm{\Gamma}{s}{\symPi(A,B)} \\
       \isterm{\Gamma}{t}{A}
     }{
       \isterm{\Gamma}{\symapp(A, B, s, t)}{B[t/x]}.
     }
  \end{align*}
  
  This is visually similar to $R_\symapp$ itself, but not to be confused with it.
  %
  The instantiation is a single closure rule, written over the ambient signature $\Sigma$; the original raw rule, written over $\mvextend{\Sigma}{\arity{\symapp}}$, is a template specifying the whole family of such closure rules.
  %
  In the raw rule, $\symA$, $\symB$, $\symb{s}$, $\symb{t}$ are metavariable symbols from the extended signature $\mvextend{\Sigma}{\arity{\symapp}}$; in the instantiatiaion, the symbols $A$, $B$, $s$, $t$ (note the difference in fonts) are the actual syntactic expressions the raw rule was instantiated with.
\end{example}

This construction of $\clos$ formalises the usual informal explanation that a single written rule is a shorthand for a scheme of closure conditions, with the quantification of the scheme inferred from the written rule.

\begin{proposition} \label{prop:closure-system-of-raw-rule-under-signature-map}
  %
  The construction $\clos$ is \emph{laxly natural} in signature maps,
  % 
  in that given $F : \Sigma \to \Sigma'$ and a rule $R$ over $\Sigma$, there is an induced simple map of closure systems $\clos R \to \clos \act{F}R$, over $\act{F} : \Judg \Sigma \to \Judg \Sigma'$.
\end{proposition}

\begin{proof}
  For each instantiation $I \in \Inst{\Sigma}{\position \Gamma}{\arity{R}}$, we have $\act{F}I \in \Inst{\Sigma'}{\position \Gamma}{\arity{R}}$ and $\act{(\act{F} I, \act{F} \Gamma)} R = \act{F} (\act{(I, \Gamma)} R)$.
\end{proof}

This is lax in the sense that the resulting map $\clos R \to \clos \act{F} R$ will not usually be an isomorphism: in general, not every instantiation of $\arity{R}$ over~$\Sigma'$ is of the form~$\act{F} I$.
%
This illustrates the need for considering raw rules formally, rather than just viewing a type theory as a collection of closure rules: when translating a type theory between signatures, we want not just the translations of the original closure rules, but all instantiations of the translated raw rules.

One might hope for $\clos$ to be similarly laxly natural under instantiations.
%
However, this is not so straightforwardly true; we will return to this in \cref{prop:instantiation-of-closure-system-of-raw-rule}, once the structural rules are introduced, and show a weaker form of naturality.

\subsection{Structural rules}

The rules used in derivations over a type theory will fall into two groups:
%
\begin{enumerate}
\item the \defemph{structural rules}, governing generalities common to all type theories;
\item the \defemph{specific rules} of the particular type theory.
\end{enumerate}

The structural rules over a signature~$\Sigma$ are a family of closure rules on $\Judg{\Sigma}$, which we now lay out.
%
They are divided into four families:
%
\begin{itemize}
\item the variable rules,
\item rules stating that equality is an equivalence relation,
\item rules for conversion of terms and term equations between equal types, and
\item rules for substitutions,
\end{itemize}
%
We have chosen the rules so that the development of the general setup requires no hard meta-theorems, as far as possible. In particular, we include the substitution rules into the formalism so that we can postpone proving elimination of substitution until \cref{sec:elimination-substitution}. You might have expected to see congruence rules among the structural rules, but those we take care of separately in \cref{sec:congruence-rules} because they depend on the specific rules.

The first three families of structural rules are straightforward.

\begin{definition}
  \label{def:variable-rule}%
  For each raw context $\Gamma$ over a signature $\Sigma$, and for each $i \in \position{\Gamma}$, the corresponding \emph{variable rule} is the closure rule
  % 
  \begin{equation*}
    \infer{
      \istype{\Gamma}{\Gamma_i}
    }{
      \isterm{\Gamma}{\synvar{i}}{\Gamma_i}
    }
  \end{equation*}
  %
  Taken together, the variable rules form a family indexed by such pairs $(\Gamma, i)$.
\end{definition}

While this had to be given directly as a family of closure rules, the next two groups of structural rules can be expressed as raw rules.

\begin{definition}
  \label{def:equivalence-relation-rule}%
  %
  The \defemph{raw equivalence relation rules} are the following raw rules:
  %
\begin{mathpar}
%
\infer
  {
    \istype{}\symA
  }{
    \eqtype{}{\symA}{\symA}
  }
%
\and
%
\infer{
  \istype{}{\symA}
  \\
  \istype{}{\symB}
  \\
  \eqtype{}{\symA}{\symB}
}{
  \eqtype{}{\symB}{\symA}
}
%
\and
%
\infer{
  \istype{}{\symA}
  \\
  \istype{}{\symB}
  \\
  \istype{}{\symC}
  \\
  \eqtype{}{\symA}{\symB}
  \\
  \eqtype{}{\symB}{\symC}
}{
  \eqtype{}{\symA}{\symC}
}
%
\and
%
\infer
  {
   \istype{} \symA
   \\
   \isterm{}{\symb{s}} \symA
  }{
    \eqterm{}{\symb{s}}{\symb{s}} \symA
  }
%
\and
%
\infer{
  \istype{}{\symA}
  \\
  \isterm{}{\symb{s}} \symA
  \\
  \isterm{}{\symb{t}} \symA
  \\
  \eqterm{}{\symb{s}}{\symb{t}}\symA
}{
  \eqterm{}{\symb{t}}{\symb{s}}\symA
}
%
\and
%
\infer{
  \istype{}{\symA}
  \\
  \isterm{}{\symb{s}} \symA
  \\
  \isterm{}{\symb{t}} \symA
  \\
  \isterm{}{\symb{u}} \symA
  \\
  \eqterm{}{\symb{s}}{\symb{t}} \symA
  \\
  \eqterm{}{\symb{t}}{\symb{u}} \symA
}{
  \eqterm{}{\symb{s}}{\symb{u}} \symA
}
\end{mathpar}
%
The \defemph{equivalence relation rules over~$\Sigma$} is the sum of the closure systems associated to the above equivalence relation rules, over a given signature~$\Sigma$.
\end{definition}

We trust the reader to be able read off the arities of the metavariable symbols appearing in raw rules.
%
For instance, from the use of~$\symA$ and~$\symb{s}$ in the above term reflexivity rule
we can tell that the rule has arity $[(\Ty,0),(\Tm,0)]$.

The conversion rules are written as raw rules, as well.

\begin{definition}
  \label{def:conversion-rule}%
  The \defemph{raw conversion rules} are the following raw rules:
  %
  \begin{mathpar}
  \infer{
    \istype{}{\symA}
    \\
    \istype{}{\symB}
    \\
    \isterm{}{\symb{s}}{\symA}
    \\
    \eqtype{} \symA \symB
  }{
    \isterm{}{\symb{s}}{\symB}
  }
  %
  \and
  %
  \infer{
    \istype{}{\symA}
    \\
    \istype{}{\symB}
    \\
    \isterm{}{\symb{s}}{\symA}
    \\
    \isterm{}{\symb{t}}{\symA}
    \\
    \eqterm{}{\symb{s}}{\symb{t}}{\symA}
    \\
    \eqtype{}{\symA} \symB
  }{
    \eqterm{}{\symb{s}}{\symb{t}}{\symB}
  }
\end{mathpar}
%
Again, the \defemph{conversion rules over $\Sigma$} is the sum of the closure systems associated to the above conversion rules, over a signature~$\Sigma$.
\end{definition}

The remaining groups are the substitution and equality-substitution rules.

The substitution rule should formalize the notion that ``well-typed'' substitutions preserve derivability of judgements.
%
Treatments taking simultaneous substitution as primitive usually say something like: a raw substitution $f : \Delta \to \Gamma$ is well-typed if $\isterm{\Delta}{f(i)}{\tca{f}\Gamma_i}$ for each $i \in \position{\Gamma}$.
%
However, taking all these judgements as premises in the substitution rule is rather profligate: most substitutions in practice act non-trivially only on a small part of the context.
%
For instance, a single-variable substitution may be represented as a raw substitution $\Gamma \to \ctxextend{\Gamma}{A}$ acting trivially on $\Gamma$, so no checking should be required there.
%
Indeed, in treatments taking single-variable substitution as primitive, only require checking of the substituted expression.
%
To abstract this situation, we define the substitution rule as follows. Recall that a subset $X \subseteq Y$ is \emph{complemented} when $X \cup (Y \setminus X) = Y$, a condition that is vacuously true in classical logic.

\begin{definition}
  \label{def:substitution-rule}%
  % NB: the names of the sets K and L is chosen so because I, J, T are taken (instantiation, judgement, theory)!
  A \defemph{raw substitution} $f : \Delta \to \Gamma$ \defemph{acts trivially at $i \in \position{\Gamma}$} when there is some (necessarily unique) $j \in \position{\Delta}$ such that $f(i) = \synvar{j}$ and $\Delta_j = \tca{f} \Gamma_i$.
  %
  Given a complemented subset $K \subseteq \position{\Gamma}$ on which~$f$ acts trivially, the corresponding \defemph{substitution rule} is the closure rule
  %
  \begin{equation}
    \label{eq:substitution-rule}
    \infer{
      \Gamma \typesjudgement J
      \\\\
     \text{for each $i \in \position{\Gamma} \setminus K$:} \quad
     \isterm{\Delta}{f(i)}{\tca{f}\Gamma_i}
    }{
      \Delta \typesjudgement \tca{f}J
    }
  \end{equation}
  %
  The substitution rules form a family of closure rules, indexed by $f : \Delta \to \Gamma$, $K$, and $\Gamma \typesjudgement J$.
\end{definition}

\noindent
%
In the above definition, $K$ is thought of as a set of positions at which~$f$ is
\emph{guaranteed} to act trivially, but it may also do so outside~$K$, as there is no harm in checking positions at which~$f$ acts trivially.

The substitution rules are formulated carefully for another, more technical reason.
%
In inductions over derivations (e.g.\ for \cref{lem:admissibility-substitution}), when a substitution descends under a binder, it gets extended to act trivially on the  variables introduced by the binder.
%
Within the inductive cases, we may not yet have enough information to conclude that the types of the bound variables are well-formed, but we can rely on the trivial action of the substitution.
%
Keeping substitution rules flexible and economical in this way therefore keeps these inductive proofs much cleaner.

Along similar lines, we have rules stating that substitution of equal terms gives equal results.
%
\begin{definition}
  \label{def:equality-substitution-rule}%
  %
  Raw substitutions $f, g : \Delta \to \Gamma$ \defemph{act jointly trivially} at $i \in \position{\Gamma}$ when there is some (necessarily unique) $j \in \position{\Delta}$ such that $f(i) = g(i) = \synvar{j}$ and $\Delta_j = \tca{f} \Gamma_i = \tca{g} \Gamma_i$.
  %
  Given a complemented subset $K \subseteq \position{\Gamma}$ on which $f$ and $g$ act jointly trivially,
  the corresponding \defemph{equality-substitution rules} are the closure rules
  %
  \begin{mathpar}
    \infer{
      \istype{\Gamma}{A}
      \\\\
      \text{for each $i \in \position{\Gamma} \setminus K$:}\\\\
      \isterm \Delta {f(i)} {\tca f \Gamma_i}\\
      \isterm \Delta {g(i)} {\tca g \Gamma_i} \\
      \eqterm{\Delta}{f(i)}{g(i)}{\tca{f}\Gamma_i}
    }{
      \eqtype{\Delta}{\tca{f}A}{\tca{g}A}
    }
    %
    \and
    %
    \infer{
      \isterm{\Gamma}{t}{A}
      \\\\
      \text{for each $i \in \position{\Gamma} \setminus K$:}\\\\
      \isterm \Delta {f(i)} {\tca f \Gamma_i}\\
      \isterm \Delta {g(i)} {\tca g \Gamma_i} \\
      \eqterm{\Delta}{f(i)}{g(i)}{\tca{f}\Gamma_i}
    }{
      \eqterm{\Delta}{\tca{f} t}{\tca{g} t}{\tca{f} A}
    }
  \end{mathpar}
  %
  The equality-substitution rules form a family of closure rules, indexed by $f : \Delta \to \Gamma$, $K$, and either $\istype{\Gamma}{A}$ or $\isterm{\Gamma}{t}{A}$.
\end{definition}

\begin{definition}
  \label{def:structural-rules}
  The \defemph{structural rules over $\Sigma$}, denoted $\StructuralRules \Sigma$, is the sum of the families of closure rules set out above: the variable, equivalence relation, conversion, substitution, and equality-substitution rules.
\end{definition}

\begin{proposition}
  \label{prop:structural-rules-under-signature-map}%
  Given a signature map $F : \Sigma \to \Sigma'$, there is a simple map of closure systems $\StructuralRules \Sigma \to \StructuralRules \Sigma'$ over $\act{F} : \Judg \Sigma \to \Judg \Sigma'$.
\end{proposition}

\begin{proof}
  This is straightforward, amounting to checking that for each instance of a structural rule over $\Sigma$, $F$ acts on the data to give an instance of the same structural rule over $\Sigma'$, and the resulting closure condition is the translation along $\act{F}$ of the original closure condition over $\Sigma$.
\end{proof}

Before giving a similar statement about instantiations of structural rules, we must first tie up the loose end from above about instantiation of closure systems of raw rules.

\begin{proposition} \label{prop:instantiation-of-closure-system-of-raw-rule}
  Let $R$ be a raw rule over $\Sigma$, $\mvextend{R}{\beta}$ its translation to an extension $\mvextend{\Sigma}{\beta}$, and $I$ an instantiation of~$\beta$ in some context~$\Gamma$.
  %
  Then there is a closure system map $\clos (\mvextend{R}{\beta}) \to \clos R + \StructuralRules \Sigma$, over $\act{(I,\Gamma)} : \Judg{(\mvextend{\Sigma}{\beta})} \to \Judg \Sigma$.
\end{proposition}

\begin{proof}
  We need to show that for each instantiation $K \in \Inst{\mvextend{\Sigma}{\beta}}{\position \Delta}{\arity{R}}$ in some context~$\Delta$, the closure condition $\act{I} (\act{(K,\Gamma)} R)$ is derivable from $\clos R + \StructuralRules \Sigma$.

  Given such $K$ and $\Gamma$, we can instantiate both under $I$ to get an instantiation of~$R$ over $\Sigma$.
  %
  We might hope that $\act{(\act{I}K,\act{I}\Gamma)}R = \act{I}\left(\act{(K,\Gamma)}(\mvextend{R}{\beta}\right)$;
  %
  by \cref{prop:instantiate-instantiate-judgement}, we see that this does not strictly, but only up to an associativity renaming in each judgement.

  The substitution structural rule comes to our rescue here.
  %
  For each judgement $J$ involved in $R$, with context $\Theta$, the associativity renamings give substitutions between $\act{I}(\act{K}\Theta)$ and $\act{(\act{I}K)}\Theta$ acting trivially at every position, so the substitution rule lets us derive $\act{I}(\act{K}J)$ from $\act{(\act{I}K)}J$ (with no further premises), and vice versa.

  The desired derivation of $\act{I}\left(\act{(K,\Gamma)}(\mvextend{R}{\beta}\right)$ from $\clos R + \StructuralRules \Sigma$ therefore consists of $\act{(\act{I}K,\act{I}\Gamma)}R$, together with an instance of the substitution rule after the conclusion and before each premise, implementing the associativity renamings.
\end{proof}

\begin{proposition}  \label{prop:instantiation-of-structural-rules}%
  Let $I \in \Inst{\Sigma}{\position \Gamma}{\alpha}$ be an instantiation in context $\Gamma$.
  %
  Then there is a closure system map $\StructuralRules (\mvextend{\Sigma}{\alpha}) \to \StructuralRules \Sigma$, over $\act{(I,\Gamma)} : \Judg {(\mvextend{\Sigma}{\alpha})} \to \Judg{\Sigma}$.
\end{proposition}

\begin{proof}
  For the structural rules given as raw rules, the required derivations are given by \cref{prop:instantiation-of-closure-system-of-raw-rule}.
  
  For the other structural rules, we start as in \cref{prop:structural-rules-under-signature-map}.
  %
  Given an instance of a structural rule over $\mvextend{\Sigma}{\alpha}$, we instantiate its data under $I$ to get an instance of the same structural rule over $\Sigma$.
  %
  Wrapping this instance in associativity renamings, derived by the substitution rule as in \cref{prop:instantiation-of-closure-system-of-raw-rule}, gives the required derivation of the instantiation of the original instance.
\end{proof}

\subsection{Congruence rules}
\label{sec:congruence-rules}

Congruence rules, which state that judgemental equality commutes with type and term symbols, are peculiar enough to demand special attention.

They are present in almost all type theories, but rarely explicitly written out, and are often classified as structural rules.
%
We reserve that term for the rules of the preceding section, which are independent of the specific theory under consideration.
%
Congruence rules, by contrast, depend on the specific rules of a theory; for instance, the congruence rule for~$\symPi$ is determined by the formation rule for~$\symPi$.

In this section we define how any object rule determines an associated congruence rule.
%
We first set up an auxiliary definition, associating equality judgements to object judgements.

\begin{definition}
  \label{def:judgement-associated-congruence}
  For signature maps $\ell, r : \Sigma \to \Sigma'$ and an object judgement $J$ over~$\Sigma$, we define the equality judgment $\tca{(\ell,r)}J$ over~$\Sigma'$ by
  %
  \begin{align*}
     \tca{(\ell,r)}(\istype{\Gamma}{A})
    \ &\defeq \ 
    (\eqtype{\act{\ell} \Gamma}{\act{\ell} A}{\act{r} A}),
    \\
    \tca{(\ell,r)}(\isterm{\Gamma}{t}{A})
    \  &\defeq \ 
    (\eqterm{\act{\ell} \Gamma}{\act{\ell} t}{\act r t}{\act{\ell} A}).
  \end{align*}
\end{definition}

\begin{definition}
  \label{def:congruence-rule}%
  Suppose $R$ is a raw object rule over a signature $\Sigma$, with premises $\famtuple{P_i}{i \in I}$ and conclusion $C$.
  %
  Let $\phi_i$ be the judgement form of $P_i$, and take $I_{\ob} \defeq \set{i \in I \such \phi_i \in \set{\Ty, \Tm}}$, the set of object premises of~$R$.
  %
  The \defemph{associated congruence rule} $\congrule{R}$ is a raw rule with arity
  $\arity{\congrule{R}} \defeq \arity{R} + \arity{R}$, defined as follows, where
  $\ell, r : \mvextend{\Sigma}{\arity{R}} \to \mvextend{\Sigma}{\arity{\congrule{R}}}$ are signature maps
  %
  \begin{align*}
    \ell(\inl(S)) &\defeq \inl(S), &
    r(\inl(S)) &\defeq \inl(S), \\
    \ell(\inr(M)) &\defeq \inr(\inl(M)), &
    r(\inr(M)) &\defeq \inr(\inr(M)):
  \end{align*}
  %
  \begin{enumerate}
  %
  \item The premises of $\congrule{R}$ are indexed by the set $I + I + I_{\ob}$, and are given by:
    \begin{enumerate}
    \item the $\inl(i)$-th premise is $\act{\ell} P_i$,
    \item the $\inr(j)$-th premise is $\act{r} P_j$,
    \item the $\iota_2(k)$-th premise is the equality $\tca{(\ell,r)} P_k$, cf.\ \cref{def:judgement-associated-congruence}.
    \end{enumerate}

  \item The conclusion of $\congrule{R}$ is $\tca{(\ell,r)}C$.
  \end{enumerate}
\end{definition}

\begin{example}
  \label{ex:pi-congruence-rule}%
  \Cref{def:congruence-rule} works as expected. For example, the congruence rule
  associated with the usual product formation rule
  %
  \begin{equation*}
    \inferrule{
      \istype{}{\symA} \\
      \istype{x \of \symA}{\symB(x)}
    }{
      \istype{}{\symPi(\symA, \symB(x))}
    }
  \end{equation*}
  %
  comes out to be
  %
  \begin{equation*}
    \inferrule{
      \istype{}{\symA'} \\
      \istype{x \of \symA'}{\symB'(x)} \\\\
      \istype{}{\symA''} \\
      \istype{x \of \symA''}{\symB''(x)} \\
      \\\\
      \eqtype{}{\symA'}{\symA''}
      \\
      \eqtype{x \of \symA'}{\symB'(x)}{\symB''(x)}
    }{
      \eqtype{}{\symPi(\symA', \symB''(x))}{\symPi(\symA'', \symB''(x))}
    }
  \end{equation*}
\end{example}


\subsection{Raw type theories}

After a considerable amount of preparation, we are finally in position to formulate what a rudimentary general type theory is.

\begin{definition}
  \label{def:raw-type-theory}%
  A \defemph{raw type theory}~$T$ over a signature $\Sigma$ is a family of raw rules over~$\Sigma$.
\end{definition}

\begin{definition} \label{def:closure-system-of-type-theory}
  The \defemph{associated closure system} of a raw type theory~$T$ over~$\Sigma$ is the closure system $\clos T \defeq \StructuralRules \Sigma + \coprod_{R \in T} \clos R$ on $\Judg{\Sigma}$; that is, it consists of the structural rules for~$\Sigma$, and the closure rules generated by the instantiations of the rules of~$T$.
  %
  A \defemph{derivation in $T$} is a derivation over the closure system $\clos T$, in the sense of \cref{def:closure-system-derivation}.
\end{definition}

Note that we have not included the congruence rules into the closure system associated with a raw type theory. Instead, the presence of congruence rules will be required separately as a well-behavedness condition in \cref{sec:acceptable-type-theories}.
%
Derivability and admissibility of rules may now be defined as follows.

\begin{definition} \label{def:derivable-raw-rule}
  Let $T$ be a raw type theory over~$\Sigma$, and $R$ a raw rule over~$\Sigma$.
  %
  \begin{enumerate}
  \item $R$ is \defemph{derivable} from~$T$ if its conclusion is derivable from its premises, over $\mvextend{T}{\arity{R}}$.
  \item $R$ is \defemph{admissible} for~$T$ if for every instance $\act{(I,\Gamma)}R$,
    its conclusion is derivable if its premises are derivable, all over~$T$.
  \end{enumerate}
\end{definition}

We record the basic category-theoretic structure of raw type theories.

\begin{definition}
  Given a signature map $F : \Sigma \to \Sigma'$, and raw type theories $T$, $T'$ over $\Sigma$ and $\Sigma'$ respectively, a \defemph{simple map $\bar{F} : T \to T'$ over $F$} is a family map $T \to T'$ over $\act{F} : \RawRule{\Sigma} \to \RawRule{\Sigma'}$.
  %
  Such $\bar{F}$ is thus a map giving for each rule~$R$ of~$T$ a rule~$\bar{F}(R)$ of~$T'$, whose premises and conclusion are the translations along~$F$ of those of~$R$.
  %
  There are evident identity simple maps, and composites over composites of signature maps, forming a category over the category of signatures.

  Furthermore, a signature map $F : \Sigma \to \Sigma'$ \defemph{acts on a raw type theory} $T$ over $\Sigma$, to give a raw type theory $\act{F}(T)$
  over $\Sigma'$, which consists of the translations $\act{F} R$ of the rules~$R$ of~$T$.
  %
  As with family maps, maps $T \to T'$ over $F$ correspond precisely to maps $\act{F} T \to T'$ over $\idmap[\Sigma']$.
  %
  In the case of the inclusion to a metavariable extension $\inl : \Sigma \to \mvextend{\Sigma}{\alpha}$, we write $\mvextend{T}{\alpha}$ for the translation $\act{{\inl}} T$ of $T$ to $\mvextend{\Sigma}{\alpha}$. 
\end{definition}

\begin{proposition}%
  \label{prop:cl-functorial-simple-maps} \label{prop:derivations-functorial-simple-maps}
  The construction $\clos{}$ is functorial in simple maps:
  %
  a simple map of raw type theories $\bar{F} : T \to T'$ over $F : \Sigma \to \Sigma'$ induces a map $\act{\bar{F}} : \clos{T} \to \clos{T'}$ over $\act{F} : \Judg{\Sigma} \to \Judg{\Sigma'}$,
  %
  and hence provides a translation of any derivation~$D \in \derivation{T}{H}{(\Gamma \typesjudgement J)}$ to a derivation $\act{\bar{F}} D \in \derivation{T'}{\act{F} H}{(\act{F} \Gamma \typesjudgement \act{F} J)}$.
\end{proposition}

\begin{proof}
  Direct from the functoriality and naturality properties of structural rules (\cref{prop:structural-rules-under-signature-map}) and of the closure systems associated to raw rules (\cref{prop:closure-system-of-raw-rule-under-signature-map}).
\end{proof}

\begin{corollary} \label{cor:derivations-functorial-signature-maps}
  A signature map $F : \Sigma \to \Sigma'$ acts on
  $D \in \derivation{T}{H}{(\Gamma \typesjudgement J)}$ to give a derivation
  $\act{F} D \in \derivation{\act{F} T}{\act{F} H}{\act{F}(\Gamma \typesjudgement J)}$, functorially so.
\end{corollary}

\begin{proof}
 By \cref{prop:derivations-functorial-simple-maps}, using the canonical simple map $T \to \act{F}T$ over~$F$.
\end{proof}

We use the previously corollary quite frequently to translate a derivation over a raw type theory to its extension. We mostly leave such applications implicit, as they are easily detected.

Instantiations also preserve derivability, but this is a significantly more involved construction --- more so than one might expect --- bringing together many earlier constructions and lemmas, and relying in particular on almost all the properties of \cref{prop:instantiation-boilerplate}.

\begin{proposition} \label{cor:instantiation-acts-on-flattening}
  Given a raw type theory $T$ over $\Sigma$, an instantiation $I \in \Inst{\Sigma}{\Gamma}{\alpha}$ induces a closure system map $\act{(I,\Gamma)} : \clos{\mvextend{T}{\alpha}} \to \clos{T}$ over $\act{(I,\Gamma)} : \Judg{\mvextend{\Sigma}{\alpha}} \to \Judg{\Sigma}$, where $\mvextend{T}{\alpha}$ is the translation of~$T$ by the inclusion $\Sigma \to \mvextend{\Sigma}{\alpha}$.
\end{proposition}

\begin{proof}
  Again, direct from similar properties of structural rules (\cref{prop:instantiation-of-structural-rules}) and closure systems of raw rules (\cref{prop:instantiation-of-closure-system-of-raw-rule}).
\end{proof}

\begin{corollary} \label{cor:instantiation-of-derivations}
  Let $T$ be a raw type theory over~$\Sigma$.
  %
  An instantiation $I \in \Inst{\Sigma}{\Gamma}{\alpha}$ acts on a derivation $D \in \derivation{\mvextend{T}{\alpha}}{H}{(\Delta \typesjudgement J)}$ to give the \defemph{instantiation} $\act{(I, \Gamma)} D \in \derivation{T}{\act{(I, \Gamma)} H}{\act{I}(\Delta \typesjudgement J)}$.
  %
\end{corollary}

Note that the hypotheses $H$ and the judgement $\Delta \typesjudgement J$ in the statement reside in the translation $\mvextend{T}{\alpha}$ by the inclusion $\Sigma \to \mvextend{\Sigma}{\alpha}$.


\subsection{Summary}

\emph{Raw type theories} give a conceptually minimal way to make precise what is meant by traditional specifications of type theories, and a similarly minimal amount of data from which to define \emph{derivability} on judgements.

As the name suggests, raw type theories are not a finished product.
%
Type theories in nature almost always satisfy further well-formedness properties, and are rejected by audiences if they do not.
%
In the next two sections, we will discuss these well-formedness properties.

In some ways, raw type theories may therefore be viewed as an unnatural or undesirable notion.
%
However, most of the well-behavedness properties --- or rather, the conditions on rules implying well-behavedness --- themselves involve checking derivability of certain judgements.
%
So raw type theories, as the minimal data for defining derivability, give a natural intermediate stage on the way to our main definition of “reasonable” type theories.

%%% Local Variables:
%%% mode: latex
%%% End:

\section{Well-behavedness properties}
\label{sec:well-behavedness}

In this section we identify easily-checked syntactic properties of the rules specifying a type theory, and prove basic fitness-for-purpose meta-theorems, which together articulate the rules-of-thumb that researchers habitually use to verify that some collection of inference rules defines a “reasonable” type theory.

\subsection{Acceptable rules}
\label{sec:acceptable-rules}

Not all raw rules are deemed reasonable from a type-theoretic point of view.
%
But what standard of “reasonable” are we aiming to delineate?
%
Essentially, the same as for the axioms of a theory in first-order logic: the axioms must be well-formed enough to be given some meaning, although that meaning may be “false”, “wrong”, or otherwise unexpected.

Consider for instance the following possible modifications of the rule for $\symapp$, all written as raw rules:
\begin{gather}
\label{eq:example-app-1}
\inferrule{
  \istype {} {\symA} \\ 
  \istype {x \of \symA} {\symB(x)} \\
  \isterm {} {\symf} {\synPi[\symA,\symB(x)]} \\
  \isterm {} {\syma} {\symA}
}{
  \isterm {} {\symapp{(\symA,\symB(x),\symf,\syma)}} {\symB(\syma)}
}
\\[1ex]
\label{eq:example-app-2}
\inferrule{
  \istype {} {\symA} \\ 
  \istype {x \of \symA} {\symB(x)} \\
  \isterm {} {\symf} {\symA} \\
  \isterm {} {\syma} {\symA}
}{
  \isterm {} {\symapp{(\symA,\symB(x),\symf,\syma)}} {\symB(\syma)}
}
\\[1ex]
\label{eq:example-app-3}
\inferrule{
  \istype {} {\symA} \\ 
  \istype {x \of \symA} {\symB(x)} \\
  \isterm {} {\symf} {\synPi[\symA,\symB(x)]} \\
  \isterm {} {\syma} {\synPi[\symA,\symB(x)]}
}{
  \isterm {} {\symapp{(\symA,\symB(x),\symf,\syma)}} {\symB(\syma)}
}
\end{gather}

The first is the usual rule for $\symapp$, and should certainly be considered acceptable.

The second asks the argument $\symf$ to be of type $\symA$.
%
This is “wrong” under the usual reading of $\symPi$ and $\symapp$, but not entirely meaningless: one can introduce $\symapp$ with this typing rule, and obtain a well-behaved (if bizarre) type theory.
%
So this should be accepted as a type-theoretic rule.

The third asks the argument $\syma$ to be of type $\synPi[\symA,\symB(x)]$.
%
This is “not even wrong”: the conclusion purports to introduce a term of type $\symB(\syma)$, but that is not a well-formed type, since $\symB$ expects an argument of type $\symA$, so $\syma$ is not suitable (at least in the absence of other rules implying that $\eqtype{}{\symA}{\synPi[\symA,\symB(x)]}$).
%
This will therefore \emph{not} be an acceptable rule.

Another unacceptable rule would be:
%
\begin{gather}
  \label{eq:example-app-4}
  \inferrule{
    \istype {} {\symA} \\ 
    \istype {x \of \symA} {\symB(x)} \\
    \isterm {} {\symf} {\synPi[\symA,\symB(x)]} \\
    \isterm {} {\syma} {\symA}  \\
    \isterm {} {\syma} {\synPi[\symA,\symB(x)]} \\
  }{
    \isterm {} {\symapp{(\symA,\symB(x),\symf,\syma)}} {\symB(\syma)}
  }
\end{gather}
%
This is again clearly nonsense: it introduces $\syma$ twice, with two different types.

There are rules which are not uncommon in practice, but which we will not accept directly, such as:
\begin{gather}
\label{eq:example-app-5}
\inferrule{
   \isterm {} {\symf} {\synPi[\symA,\symB(x)]} \\
   \isterm {} {\syma} {\symA}
}{
   \isterm {} {\symapp{(\symA,\symB(x),\symf,\syma)}} {\symB(\syma)}
}
\end{gather}
%
While the rule is completely reasonable, making sense of it is rather subtle: checking, for instance, that $\symB(\syma)$ in the conclusion is well-formed requires applying some kind of inversion principle, to the type $\synPi[\symA,\symB(x)]$ from the premises.
%
Whether such an inversion principle is available depends on the particularities of the type theory under consideration.
%
In general, we want acceptable rules to be more straightforwardly and robustly well-behaved, so we expect that every metavariable used by the rule is explicitly introduced by some (unique) premise.

Finally, some rules have variant forms given by moving simple premises into the context of the conclusion.
%
For example, the rule for application is sometimes given as
%
\begin{equation}
\label{eq:example-app-6}
\inferrule{
  \istype {} {\symA} \\ 
  \istype {x \of \symA} {\symB(x)}
}{
 \isterm
   {x \of \symA, y \of \synPi[\symA, \symB(x)]}
   {\symapp{(\symA,\symB(x),y,x)}}
   {\symB(x)}
}
\end{equation}

This variant has been called the \emph{hypothetical} form, in contrast to the \emph{universal} form~\eqref{eq:example-app-1}.
%
With substitution included as a structural rule, the two forms are equivalent: each is derivable from the other.
%
In the absence of a substitution rule, they are not equivalent; the hypothetical form is too weak.
%
We have also heard it argued that the universal form should be seen as conceptually prior.
%
So both forms are arguably reasonable; but the universal form \eqref{eq:example-app-1}, with empty conclusion context, has the clearer claim, and no generality is lost by restricting to such forms.

Summarising the above discussion, there are several simple syntactic criteria commonly used as rules-of-thumb to determine “reasonability” of rules.
%
We now formally define these criteria, and collect them into a definition of \emph{acceptability} of rules.

\begin{definition}
  \label{def:tight-rule}%
  Suppose $R$ is a raw rule with arity $\arity{R}$ over a signature $\Sigma$. We say that $R$ is \defemph{tight} when
  there exists a bijection $\beta$ between the arguments of $\arity{R}$ and the object premises of $R$,
  such that for each argument $i$ of $\arity{R}$,
  %
  \begin{enumerate}
    \item\label{item:tight-rule-ctx} the context of the premise $\beta(i)$ has the scope $\argbinder{\arity{R}}{i}$;
    \item\label{item:tight-rule-jf} the judgement form of the premise $\beta(i)$ is $\argclass{\arity{R}}{i}$;
    \item\label{item:tight-rule-hd} the head expression of the premise $\beta(i)$ is $\synmeta{i}(\fammap{\synvar{j}}{j \in \argbinder{\arity{R}}{i}})$.
  \end{enumerate}
\end{definition}

Note that the bijection~$\beta$ is unique, if it exists.
%
The definition of tightness is admittedly a bit technical, but it captures a well-formedness condition of rules which is familiar but infrequently discussed explicitly. Namely, a rule is tight if its object premises provide the ``typing'' of its metavariable symbols.

Tightness alone does not suffice to make a rule reasonable, e.g., the rule~\eqref{eq:example-app-3} is tight but still broken because the type expression $\symB(\syma)$ is senseless. We need another condition which ensures that the type and term expressions appearing in the rule make sense.

\begin{definition}
  To each judgement $\Gamma \types J$, we associate the family of \defemph{presuppositions} $\Presup {(\Gamma \types J)}$, defined as the judgements formed over $\Gamma$ by placing the boundary slots of $J$ in the head position as follows:
  \begin{align*}
  \Presup {(\istype \Gamma A)} &\defeq [ \; ], \\
  \Presup {(\isterm \Gamma s A)} &\defeq [ \istype \Gamma A ], \\
  \Presup {(\eqtype \Gamma A B)} &\defeq [ \istype \Gamma A, \istype \Gamma B ], \\
  \Presup {(\eqterm \Gamma s t A)} &\defeq [ \istype \Gamma A, \isterm \Gamma s A, \isterm \Gamma t A ].
  \end{align*}
\end{definition}

We shall need to know later on that presuppositions are natural with respect to the action of signature maps, instantiations, and raw substitutions.

\begin{proposition}
  \label{prop:presuppositions-action-signature-map}
  %
  Let $\Gamma \types J$ be a judgement over~$\Sigma$, and $F : \Sigma \to \Sigma'$ a signature map. Then $\Presup {(\act{F} \Gamma \types \act{F} J)} = \act{F}(\Presup {(\Gamma \types J)})$.
\end{proposition}

\begin{proof}
  This is clear, for instance the presupposition of $\isterm{\act{F} \Gamma}{\act{F} s}{\act{F} A}$ is $\istype{\act{F} \Gamma}{\act{F} A}$, which is precisely what we get when $F$ acts on $\istype{\Gamma}{A}$, the presupposition of $\isterm{\Gamma}{s}{A}$.
\end{proof}

The reasoning that established the analogous statements about the actions of instantiations and raw substitutions is similarly easy.

\begin{propositionwithqed}
  \label{prop:presuppositions-action-instantiation}
  %
  Let $\Gamma \types J$ be a judgement over a metavariable extension $\mvextend{\Sigma}{\alpha}$ and $I \in \Inst{\Sigma}{\gamma}{\alpha}$ an instantiation.
  Then $\Presup {(\act{I} \Gamma \types \act{I} J)} = \act{I}(\Presup {(\Gamma \types J)})$.
\end{propositionwithqed}

\begin{propositionwithqed}
  \label{prop:presuppositions-action-substitution}
  %
  Let $\Gamma \types J$ be a judgement and $f : \Delta \to \Gamma$ a raw substitution. Every presupposition of $\Delta \types \tca{f} J$
  has the form $\Delta \types \tca{f} J'$, where $\Gamma \vdash J'$ is a presupposition of $\Gamma \types J'$.
\end{propositionwithqed}

There is a weaker and a stronger condition that we can impose on a rule with regards to the presuppositions of its conclusion.

\begin{definition}%
  \label{def:weakly-presuppositive-rule}%
  Let $T$ be a raw type theory over a signature $\Sigma$ and $R$ a raw rule over~$\Sigma$:
  %
  \begin{enumerate}
  \item a raw rule $R$ is \defemph{weakly presuppositive over~$T$} when every presupposition of the conclusion of~$R$ is derivable in $T$ (translated from $\Sigma$ to $\mvextend{\Sigma}{\arity{R}}$) from the premises of~$R$ and the presuppositions of the premises of~$R$,

  \item a raw rule~$R$ is \defemph{presuppositive over~$T$} when all presuppositions of the conclusion and of the premises of~$R$ are derivable in $T$ (translated from $\Sigma$ to $\mvextend{\Sigma}{\arity{R}}$) from the premises of~$R$.
  \end{enumerate}
\end{definition}

As far as derivability is concerned, weakly presuppositive rules are good enough, for a rule cannot be applied unless its premises have already been derived, in which case their presuppositions will be derivable as well --- which is the gist of the proof of \cref{thm:presuppositions}.
%
However, if we were to give a meaning to a raw rule on its own, we would be hard-pressed to explain what the premises are about, unless their presuppositions were derivable as well, hence we take the stronger variant as the standard one.


\begin{definition}
  \label{def:acceptable-rule}%
  A raw rule $R$ is \defemph{acceptable} for a raw type theory $T$ if it is tight,
  presuppositive over~$T$, and has empty conclusion context.
\end{definition}

\begin{example}
  \parbox{0pt}{}
  %
  \begin{enumerate}

  \item The above rules~\eqref{eq:example-app-1},~\eqref{eq:example-app-2},~\eqref{eq:example-app-3}, and~\eqref{eq:example-app-6} are tight.

  \item The above rules~\eqref{eq:example-app-1},~\eqref{eq:example-app-2},~\eqref{eq:example-app-4}, and~\eqref{eq:example-app-6} are presuppositive.

  \item The rule
    %
    \begin{equation*}
      \infer{ }{\istype{}{\symA}}
    \end{equation*}
    %
    which allows us to infer that every type expression is a type, is not tight.

  \item If $S$ is a type symbol with the empty arity, the rule
    %
    \begin{equation*}
      \infer { } {\istype {} {S ()}}
    \end{equation*}
    %
    is presuppositive and tight.

  \item Symmetry of type equality comes in two versions:
  %
  \begin{equation*}
    \infer{
      \eqtype{}{\symA}{\symB}
    }{
      \eqtype{}{\symB}{\symA}
    }
    %
    \qquad\qquad
    %
    \infer{
      \istype{}{\symA}
      \\
      \istype{}{\symB}
      \\
      \eqtype{}{\symA}{\symB}
    }{
      \eqtype{}{\symB}{\symA}
    }
  \end{equation*}
  %
  The left-hand one is not tight and is presuppositive, and the right-hand one is tight and presuppositive.
  \end{enumerate}
\end{example}

\begin{proposition}
  The congruence rule associated to an acceptable object rule is acceptable.
\end{proposition}
%
\begin{proof}
  Let~$R$ be a tight and presuppositive raw object rule over a raw type theory~$T$ with premises $\famtuple{\Gamma_i \types J_i}{i \in I}$.
  %
  There exists a bijection~$\beta_R$ between object premises of~$R$ and the arguments of~$\arity R$.

  \cref{def:congruence-rule} lays out the associated congruence rule~$\congrule{R}$. Its arity is $\arity{\congrule{R}} = \arity R + \arity R$ and its premises are indexed by $I + I + I_{\ob}$, where $I_{\ob}$ is the set of object premises of~$R$.
  The bijection $\beta_{\congrule{R}}$ witnessing tightness of~$\congrule{R}$ is given by
  %
  \begin{equation*}
    \beta_{\congrule{R}} (\inl(i)) \defeq \inl(\beta_R(i))
    \qquad\text{and}\qquad
    \beta_{\congrule{R}} (\inr(j)) \defeq \inr(\beta_R(j)).
  \end{equation*}
  %
  Let us verify that the properties for tightness of~$\congrule{R}$ required in \cref{def:tight-rule} follow directly from the tightness of~$R$.
  %
  For any $\inl(i) \in  \args {\arity{\congrule{R}}}$:
  %
  \begin{enumerate}

  \item[\eqref{item:tight-rule-ctx}] The context of the premise $\beta_{\congrule{R}}(\inl(i)) = \inl(\beta_R(i))$ is $\act \ell \Gamma_i$. The signature map~$\ell : \mvextend{\Sigma}{\arity R} \to \mvextend{\Sigma}{\arity{\congrule{R}}}$ does not change the underlying scope of~$\Gamma_i$, and thus~$\act \ell \Gamma_i$ has the same underlying scope as~$\Gamma_i$, which equals $\argbinder {\arity R} i$ because~$R$ is tight. Furthermore, $\argbinder {\arity R} i = \argbinder {\arity{\congrule{R}}} (\inl(i))$, as required.

  \item[\eqref{item:tight-rule-jf}] The premise~$\beta_{\congrule{R}}(\inl(i))$ has judgement form~$\argclass{\arity{\congrule{R}}}{\inl(i)}$ by the analogous reasoning.

  \item[\eqref{item:tight-rule-hd}] The head of the premise~$\beta_{\congrule{R}}(\iota_i(i))$ is~$\act \ell e$ where $e = \synmeta{i}(\fammap{\synvar{j}}{j \in \argbinder{\arity{R}}{i}})$ by tightness of~$R$. We need to show that $\act \ell e = \synmeta{\inl(i)}(\fammap{\synvar{j}}{j \in \argbinder{\arity{\congrule{R}}}{\inl(i)}})$, but this equation holds by the definitions of~$\ell$ and of~$\arity{\congrule{R}}$.
  \end{enumerate}
  %
  The case of $\inr(j) \in \args {\arity{\congrule{R}}}$ is symmetric.

  We also need to show that all presuppositions of the premises and the conclusion of~$\congrule{R}$ are derivable in~$T_{\congrule{R}}$ from the premises of~$\congrule{R}$, where~$T_{\congrule{R}} = \act {\inl{}} \circ T$ is the translation of~$T$ along $\inl : \Sigma \to \mvextend \Sigma {\arity{\congrule{R}}}$.

  Consider the premise $\Gamma_{\inl(i)} \types J_{\inl(i)}$ at index $\inl(i)$ for some $i \in I$. By \cref{prop:presuppositions-action-signature-map}, a presupposition of this premise is a presupposition $P = (\Gamma_i \types J')$ of the corresponding premise in~$R$, translated along the signature map~$\ell$.
  %
  By presuppositivity of~$R$, the judgement~$P$ is derivable from~$T_R = \act {\inl{}} \circ T$, the translation of~$T$ along $\inl{} : \Sigma \to \mvextend \Sigma {\arity R}$. By \cref{cor:derivations-functorial-signature-maps}, we can translate such a derivation of~$P$ along~$\act \ell$, yielding a derivation in~$T' = \act \ell \circ T_R$, where~$T'$ is~$T_R$ translated along~$\ell$. But $T' = \act \ell \circ T_R = \act \ell \circ \act {\inl{}} \circ T = \act {\inl{}} T = T_{\congrule{R}}$, so we obtain a derivation in the correct theory.

  The case of a premise indexed by $\inr(j)$ with $j \in I$ is similar, but the last step requires translation along the signature map $r = id_{\mvextend \Sigma {\arity R}} + \inr$ instead, mapping the metavariable symbols of $\arity R$ to the right-hand side metavariables of~$\congrule{R}$.

  A premise~$P$ indexed by $\iota_2(k)$ is an equality associated to the $k$-th object premise of~$R$. The presuppositions of~$P$ are derived by the corresponding object premises~$\inl(k)$ and~$\inr(k)$, and in the case of a term equation, the presupposition of the left-hand side.

  A presupposition of the conclusion is derivable by appeal to the rule~$R$ itself for left and right hand side of the equation. In case~$\congrule{R}$ is a term equation, the type judgement arising as presupposition of the conclusion of~$R$ is derivable in~$T_R$ by presuppositivity of~$R$, and can be translated along~$\act \ell$ in the same way that we treated the left-hand copies of the premises.
\end{proof}

\begin{proposition}%
  \label{prop:structural-rules-acceptable}
  The raw structural rules, i.e., the equivalence relation rules and the conversion rules are acceptable for any type theory.
\end{proposition}

\begin{proof}
  Tightness is obvious. Presuppositivity is obvious for all but the conclusion of the equality conversion rule $\eqterm{}{\symb{s}}{\symb{t}}{\symB}$, which immediately follows from the ordinary conversion rule for term judgements.
\end{proof}


\subsection{Acceptable type theories}
\label{sec:acceptable-type-theories}

It may happen that a raw type theory is flawed, even though each of its rules is acceptable. For instance, we might simply forget to state a rule governing one of the symbols, or provide two contradicting rules for the same symbol. Thus we also need a notion of acceptability of a raw type theory.

\begin{definition}%
  \label{def:symbol-rule}%
  Suppose $\Sigma$ is a signature and $S \in \Sigma$ has arity $\arity{S}$.
  %
  The \defemph{generic application of~$S$} is the expression
  %
  \begin{equation*}
     \genapp{S} \defeq
     S(\fammap
         {\synmeta{i}(\tuple{\synvar{j}}{j \in \argbinder {\alpha_S} i})}
         {i \in \args \arity{S}}
     ).
  \end{equation*}
  %
  We say that an inference rule~$R$ is a \defemph{symbol rule for~$S$} when its arity is $\arity{S}$, the judgement form of the conclusion is the syntactic class of~$S$, and its head is $\genapp{S}$.
\end{definition}

\begin{definition}
  \label{def:theory-good-properties}%
  A raw type theory $T$ over~$\Sigma$ is:
  %
  \begin{enumerate}
  \item \defemph{tight} if its rules are tight and there is a bijection $\beta$ from the index set of~$\Sigma$ to the object rules of~$T$ such that, for every symbol $S$ of~$\Sigma$, $\beta(S)$ is a symbol rule for~$S$;
  \item \defemph{presuppositive} if all of its rules are presuppositive over $T$;
  \item \defemph{substitutive} if all its rules have empty conclusion context; and
  \item \defemph{congruous} if for every object rule of~$T$ the associated congruence rule (cf.~\cref{def:congruence-rule}) is a rule of~$T$.
  \end{enumerate}
  %
  A raw type theory is \defemph{acceptable} if it enjoys all of these properties.
\end{definition}

% NB: We require that the congruence rules be specific rules of T so that arguments
% by induction on the derivation may assume congruence rules are there directly.
% (I don't know if this actually matters.)

The definition omits a common criterion for being ``reasonable'', namely there being a well-founded order that prevents cyclic references between parts of the theory. We address well-foundedness separately in \cref{sec:well-founded-type-theories}, and provide a couple of examples showing how cyclic references may appear in an acceptable type theory.

\begin{example}
  \label{ex:cyclic-quantifier}%
  Let $\symb{Q}$ be a quantifier-like type symbol which takes a type and a term, and binds one variable in the term, with the raw rule
  %
  \begin{equation*}
    \infer{
      \istype{}{\symA}
      \\
      \isterm
        {\synvar{0} \of \symb{Q}(\symA, \symb{t}(\synvar{0}))}
        {\symb{t}(\synvar{0})}
        {\symA}
    }{
      \istype{}{\symb{Q}(\symA, \symb{t}(\synvar{0}))}
    }
  \end{equation*}
  %
  The context in the second premise is \emph{not} cyclic because $\symb{Q}$ binds $\synvar{0}$, but the premise itself is cyclic because the term metavariable $\symb{t}$ is introduced in a context that mentions it, and the rule is only presuppositive thanks to itself.
  %
  Even so, the rule can still be used to derive judgements. For example, for any
  $\isterm{}{t}{A}$ we can form the type $\symb{Q}(A, t)$.
  %
  It is not clear what one would do with such rules, but we have no reason to banish them outright.
\end{example}


\begin{example}
  \label{ex:type-in-type}%
  The second example of cyclic references is a Tarski-style universe that contains itself, formulated as follows.
  %
  Let $\symb{u}$ be a term constant and $\symb{El}$ a type symbol taking one term argument, with the raw rules
  %
  \begin{mathpar}
    \infer{
    }{
      \isterm{}{\symb{u}}{\symb{El}(\symb{u})}
    }

    \infer{
      \isterm{}{\symb{a}}{\symb{El}(\symb{u})}
    }{
      \istype{}{{\symb{El}(\symb{a})}}
    }
  \end{mathpar}
  %
  Think of $\symb{u}$ as the code of the universe $\symb{El}(\symb{u})$ that contains itself, and $\symb{El}$ as the constructor taking codes to types. The rules themselves are not cyclic,
  and the type theory comprising them and the associated congruence rules is acceptable. However, in order to derive $\istype{}{\symb{El}(\symb{u})}$, which is a presupposition for both rules, we need both rules.
  %
  In this case the cycles can be broken easily enough: introduce a type constant $\istype{}{\symb{U}}$ and the equation $\eqtype{}{\symb{U}}{\symb{El}(\symb{u})}$, then use $\symb{U}$ in place of $\symb{El}(\symb{u})$ in the above rules.
  %
  In \cref{sec:well-founded-replacement} we shall provide a general method for removing cyclic dependencies between rules by introduction of new symbols.
\end{example}


\subsection{Derivability of presuppositions}

Our first meta-theorem is a fairly easy one, giving a property that is always desired but not often explicitly discussed.

\begin{theorem}[Presuppositions theorem]
  \label{thm:presuppositions}%
  Let $T$ be a raw type theory with all rules weakly presuppositive.
  %
  If a judgement is derivable over $T$, then so are all its presuppositions.
\end{theorem}

\begin{proof}
  We proceed by induction on derivations $D$ over~$T$.

  If $D$ ends with a variable rule (\cref{def:variable-rule}), then the only presupposition appears directly as the premise of the rule, so we may re-use its subderivation.

  If $D$ ends with a substitution rule (\cref{def:substitution-rule}), then its conclusion must be $\Delta \types \tca{f} J$ for some substitution~$f : \Delta \to \Gamma$ and judgement~$\Gamma \types J$.
  %
  By \cref{prop:presuppositions-action-substitution}, each presupposition of the conclusion is $\Delta \types \tca{f} J'$ for some presupposition $\Gamma \types J'$ of $\Gamma \types J$.
  %
  But $\Gamma \types J$ is a premise of the last rule of $D$, so by induction we have a derivation $D'$ of $\Gamma \types J'$.
  %
  So we  can apply the substitution rule with $\Gamma \types J'$ (derived by $D'$) and the same substitution $f$ (with its premises derived as in $D$) to get the desired derivation of $\Delta \types \tca{f} J'$.
  
  Similarly, if $D$ ends with an equality substitution rule (\cref{def:equality-substitution-rule}), substituting an pair $f,g : \Delta \to \Gamma$ into a judgement $\Gamma \types J$, each presupposition of the conclusion can be derived by either a substitution (along $f$ or $g$ individually) or an equality substitution (along $f,g$) into some presupposition of $\Gamma \types J$.

  The equivalence and conversion rules (\cref{def:equivalence-relation-rule,def:conversion-rule}) are presuppositive by \cref{prop:structural-rules-acceptable}, so we treat them together with the specific raw rules of~$T$.

  If $D$ ends with an instance $\act{(I,\Gamma)} R$ of a raw rule $R$ (either specific or structural), then its conclusion is of the form $\act{(I,\Gamma)} \Delta \types \act{I} J$, where $\Delta \types J$ is the conclusion of~$R$.
  %
  Now \cref{prop:presuppositions-action-instantiation} tells us that each presupposition of the conclusion is an instantiation $\act{(I,\Gamma)} \Delta \types \act{I} J'$ of some presupposition $\Delta \types J'$ of $\Delta \types J$.
  %
  Since $R$ is weakly presuppositive, $\Delta \types J'$ is derivable from the premises of~$R$ plus their presuppositions.
  %
  So by \cref{cor:instantiation-of-derivations}, $\act{(I,\Gamma)} \Delta \types \act{I} J'$ is derivable from the premises of $\act{(I,\Gamma)} R$ plus their presuppositions, which in turn are derivable by induction.
\end{proof}

%%% Substitution elimination is long, so we put it in a separate file
\subsection{Elimination of substitution}
\label{sec:elimination-substitution}

In this section we show that over an acceptable type theory, the substitution rules (\cref{def:substitution-rule,def:equality-substitution-rule}) can be eliminated: anything derivable with them is derivable without.
%
At least, this will hold over a strict scope system;
%
for a general scope system, it can be almost eliminated but not quite entirely.

\begin{definition}
  An instance of the substitution rule (\cref{def:substitution-rule}) is a \defemph{trivial renaming}, or just \defemph{trivial}, if its substitution $f : \Delta \to \Gamma$ corresponds to a renaming on underlying scopes of the form $\inlscope^{-1} : \position{\Gamma} = \sumscope{\position{\Delta}}{\emptyscope} \to \position{\Delta}$, acting trivially at all positions.
\end{definition}

Typically these arise with $\Gamma = \act{(I,\Delta)}\emptycxt$, an instantiation of the empty context;
  %
a trivial renaming is therefore of the form
\[
  \infer{
      \act{(I,\Delta)}\emptycxt \typesjudgement J
    }{
      \Delta \typesjudgement \act{(\inlscope^{-1})} J.
    }
\]
%
In a \emph{strict} scope system, trivial renamings are identities and hence redundant.

To avoid ambiguity with variance, we will in this section distinguish more carefully than usual between a renaming function $r : \position{\Gamma} \to \position{\Delta}$ and its associated substitution $\bar{r} : \Delta \to \Gamma$.

\begin{definition}
  Call a derivation over a raw type theory~$T$ \defemph{substitution-free} if it uses only trivial instances of the substitution rule, and does not use the equality substitution rule.
  %
  Equivalently, it uses just the variable rule, equality rules, conversion rules, trivial renamings, and the specific rules of~$T$.
\end{definition}

The core of this section, \cref{lem:admissibility-substitution}, will be that substitution is admissible for substitution-free derivations; this can be seen as defining an action of substitution on such derivations.
%
We first need an analogous action of renaming, paralleling how substitution on expressions needed renaming to be defined first.

\begin{lemma}[Admissibility of renaming]
  \label{lem:admissibility-renaming}%
  Let $T$ be a substitutive type theory, with signature $\Sigma$.
  %
  Let $\Gamma$ and $\Gamma'$ be contexts over $\Sigma$, and $r : \position{\Gamma} \to \position{\Gamma'}$ a renaming acting trivially at all positions in the sense of~\cref{def:substitution-rule}, i.e.\ such that $\Gamma'_{r(i)} = \act{r}\Gamma_i$ for all $i \in \Gamma$.
  %
  Then given a substitution-free derivation $D$ of $\Gamma \types J$ in $T$, there is a substitution-free derivation $\act{r}D$ of $\Gamma' \types \act{r} J$.
\end{lemma}

\begin{proof}
  For this proof, we say a renaming \defemph{respects types} when it acts trivially at all positions; and say a derivation $D$ with conclusion $\Gamma \types J$ is \defemph{renameable} if we have an operation giving, for every $\Gamma'$ and renaming $r : \position{\Gamma} \to \position{\Gamma'}$ respecting types, a derivation $\act{r}D$ of $\Gamma' \types \act{r} J$.
  
  We show by induction that every derivation is renameable.
  %
  Call the derivation under consideration $D$, and suppose given in each case suitable $\Gamma'$, $r$.
  
  If $D$ concludes with a variable rule, giving $\isterm{\Gamma}{\synvar{i}}{\Gamma_i}$, then by induction, we can rename the derivation of the premise ${\istype{\Gamma}{\Gamma_i}}$ to a derivation of $\istype{\Gamma'}{\Gamma'_{r(i)}}$,
  %
  and then apply the variable rule to derive $\isterm{\Gamma'}{\synvar{r(i)}}{\Gamma'_{r(i)}}$, which is the desired judgement since~$r$ respects types.

  If $D$ concludes with a trivial renaming $s : \Gamma \to \Delta$,
  %
  then by induction, the derivation of the premise is renameable; so renaming it along $r \circ s$, we are done.

  Otherwise, $D$ concludes with an instantiation $\act{(I,\Theta)} R$, where $I \in \Inst{\Sigma}{\Theta}{\arity{R}}$ is an instantiation, and~$R$ is either an equality rule, a conversion rule, or a specific rule of~$T$.
  % R has conclusion ⊢ J' and a typical premise Δ ⊢ J''
  % I ∈ Inst Σ Θ α_R
  % I_* R has conclusion Θ + ∅ ⊢ I_* J', J', but this must be Γ ⊢ J.
  % So: I ∈ Inst Σ Γ α_R
  % A typical premise of I_* R is Γ ⊕ I_* Δ ⊢ I_* J''.
  %
  In each cases the conclusion of~$R$ is of the form $\typesjudgement J'$, with empty context; so $\Gamma$ is exactly $\act{(I,\Theta)}\emptycxt$, and $J$ is has the form $\act{I} J'$.
  %
  % r : |Γ| → |Γ'|
  % r_* I ∈ Inst Σ Γ' α_R
  % (r_* I)_* R has conclusion Γ' ⊢ (r_* I)_* J' and typical premise Γ' ⊕ (r_* I)_* Δ ⊢ (r_* I)_* J''.
  %
  So now to derive $\Gamma' \types \act{r}{(\act{I} J')}$, we will apply the same raw rule~$R$ with the instantiation $I' \defeq \act{(r \circ \inlscope)}{I}$.
  %
  Computing with renamings and instantiations according to (\cref{prop:instantiation-boilerplate}) shows that the conclusion of $\act{(I',\Gamma')} R$ is not quite $\Gamma' \typesjudgement \act{r}J$, but is the same modulo a trivial renaming, according to the following commutative square:
  \[
    \xymatrix@C=3em{
      \Gamma' \ar[r]^{\overline{\inlscope} \circ {\bar{r}}} \ar[dr]^{\bar{r}} & \Theta \\
      \act{(I',\Gamma')} \emptycxt \ar[r]  \ar[u]^{\overline{\inlscope}} & \Gamma \mathrlap{{} = \act{(I,\Theta)} \emptycxt} \ar[u]_{\overline{\inlscope}}
    }
    \phantom{\act{(\Theta)} } % to take above \mathrlap into account for centering
  \]
  We can therefore conclude the derivation of $\act{r}D$ by the rule $\act{(I',\Gamma')} R$ followed by a trivial renaming.

  It remains to derive the premises of $\act{(I',\Gamma')} R$.
  %
  Each such premise is linked to a corresponding premise of $\act{(I,\Theta)} R$ by a renaming repecting types --- specifically, a context extension of $r \circ \inlscope$.
  %
  But by induction, we have renameable derivations of the premises of $\act{(I,\Theta)} R$; so we are done.
\end{proof}

It is worthwhile to record a special case of admissibility of renaming.

\begin{corollary}[Admissibility of weakening]
  \label{cor:admissibility-weakening}%
  If a substitutive raw type theory derives $\Gamma \types J$ substitution-free, then it also derives $\Delta \types \act{w} J$ substitution-free for any \defemph{weakening} $w : \Gamma \to \Delta$, i.e., an injective variable renaming such that $\Delta_{w(i)} = \act{w} \Gamma_i$ for all $i \in \position{\Gamma}$. \qed
\end{corollary}

We can now give the action of substitution on derivations.

\begin{lemma}[Admissibility of substitution]%
  \label{lem:admissibility-substitution}%
  Let $T$ be a substitutive raw type theory over signature~$\Sigma$.
  %
  Let $f : \Delta \to \Gamma$ be a raw substitution over~$\Sigma$, and $K \subseteq \position{\Gamma}$ a complemented subset such that:
  %
  \begin{enumerate}
  %
  \item \label{item:subst-trivial-case} $f$ acts trivially at each $i \in K$ in the sense of~\cref{def:substitution-rule}, i.e., for some $j \in \Delta$, $f(i) = \synvar{j}$ and $\Delta_j = f^*\Gamma_i$
  %
  \item \label{item:subst-nontrivial-case} for each $i \in \position{\Gamma} \setminus K$, $T$ derives $\isterm{\Delta}{f(i)}{f^*\Gamma_i}$ without substitutions.
  \end{enumerate}
  %
  Given a substifution-free derivation $D$ of $\Gamma \types J$ over $T$, there is a substitution-free derivation $\tca{f}D$ of $\Delta \types \tca{f} J$.
\end{lemma}

Before proceeding with the proof, we take a moment to comment on the condition the lemma assumes on~$f$. It matches the condition used in the premises of the substitution rule, skipping type-checking on a set of indices~$K$ on which~$f$ acts trivially, and requiring derivability of $\isterm{\Delta}{f(i)}{\tca{f} \Gamma_i}$ only for $i \in \position{\Gamma} \setminus K$.
%
An alternative, maybe more conventional condition would be to require $\isterm{\Delta}{f(i)}{\tca{f} \Gamma_i}$ for all $i \in \position{\Gamma}$.

There are a couple of reasons to weaken the condition as we do, thus strengthening the statement of the lemma. The superficial one is its applicability to raw substitutions that potentially contain ill-formed type expressions.
%
The more essential one is that strengthened formulation is needed to keep the proof structurally inductive, allowing us to descend under a premise with a non-empty context, without needing to check that in the process the domain of $\tca{f}$ is extended with well-formed types.
%
Even if they are in fact well formed, we cannot show this by appealing to an induction hypothesis, because the derivations involved are not structural subderivations of the one we are recursing over.
%
What happens instead is that verification of well-formedness of types in contexts is deferred until their variables are accessed, at which point the variable rule provides the desired structural subderivations.
%
This phenomenon seems to be a genuine consequence of spelling out the proof for a \emph{general} class of type theories.
%
For any specific type theory, only certain concrete type-schemes will occur in contexts of premises of rules; and
these specific type-schemes are always designed by their authors in such a way that they can be shown well-formed individually, so that the inductive arguments do not break.

\begin{proof}[Proof of \cref{lem:admissibility-substitution}]
  %
  Within this proof, all derivations are assumed substitution-free, and a (substitution-free) derivation~$D$ of a judgement $\Gamma \types J$ is called \defemph{substitutable} when, for all~$\Delta$, $f$ and $K$ satisfying the condition of the lemma, we have a (substitution-free) derivation of $\Delta \types f^*J$.
  %
  We prove by induction that every derivation $D$ is substitutable.
  %
  Much of the proof parallels that of \cref{lem:admissibility-renaming}.

  Suppose $D$ concludes with a variable rule showing $\isterm{\Gamma}{\synvar{i}}{\Gamma_i}$, and $f : \Delta \to \Gamma$ is a suitable substitution, acting trivially on $K \subseteq \position{\Gamma}$.
  %
  When $i \in K$, we work just as in \cref{lem:admissibility-renaming}:
  %
  given a suitable substitution into the conclusion, we inductively substitute the premise derivation along the same substitution, and then conclude with the variable rule.
  %
  Otherwise, for $i \in \position{\Gamma}\setminus K$, we use the derivation of $\isterm{\Delta}{f(i)}{\tca{f}\Gamma_i}$ given by assumption. 
  
  Next, if $D$ concludes with a trivial renaming $\overline{\inlscope^{-1}} : \Gamma \to \Gamma'$, to conclude $\Gamma \typesjudgement J$, then suppose $f : \Delta \to \Gamma$ is a substitution acting trivially on $K$ and with derivations of $\isterm{\Delta}{f(i)}{\tca{f}\Gamma_i}$ for $i \in \position{\Gamma} \setminus K$.
  %
  Then the substitution $\overline{\inlscope^{-1}} \circ f : \Delta \to \Gamma'$ acts trivially on $\inlscope(K) \subseteq \position{\Gamma'}$, and the same derivations witness that $\isterm{\Delta}{f(\inlscope^{-1}i)}{\tca{f}\Gamma'_i}$ for $i \in \position{\Gamma'} \setminus \inlscope{(K)}$.
  %
  So by induction, we can substitute the derivation of the premise along $\overline{\inlscope^{-1}} \circ f $ to derive $\Delta \types \tca{f}J$as required.
  
  Otherwise, $D$ must conclude with an instantiation $\act{(I,\Theta)} R$, for some instantiation $I \in \Inst{\Sigma}{\Theta}{\arity{R}}$ and~$R$ a flat rule (structural or specific) with empty-context conclusion $\typesjudgement J$.
  %
  So, suppose given a suitable substitution $f : \Delta \to \act{(I,\Theta)}\emptycxt$, acting trivially on $K \subseteq \position{\act{(I,\Theta)}\emptycxt} = \sumscope{\Theta}{\emptyscope}$; we need to derive $\Delta \typesjudgement f^*\act{I}J$.

  Just as in \cref{lem:admissibility-renaming}, we substitute $I$ along $\overline{\inlscope} \circ f : \Delta \to \Theta$ to get another instantiation $I'$ of $\arity{R}$ over $\Delta$, such that the conclusion of $\act{(I',\Delta)}R$ is just a trivial renaming away from $\Delta \typesjudgement f^*\act{I}J$.
  %
  So it remains just to derive the premises of $\act{(I',\Delta)}R$.
  
  Again as in \cref{lem:admissibility-renaming}, by induction we have substitutable derivations of all premises of $\act{(I,\Theta)}R$.
  %
  So it suffices to give, for each premise $\act{(I',\Delta)} \Psi \typesjudgement \act{I'}J'$ of $\act{(I',\Delta)}R$, a substitution $g : \act{(I',\Delta)} \Psi  \to \act{(I,\Theta)} \Psi$ to the corresponding premise of $\act{(I,\Theta)}R$, with $g^* \act{I}J' = \act{I'}J'$, and with $g$ satisfying the conditions of the lemma.
  %
  (Recall that $\act{(I,\Theta)} \Psi$ is the context extension of $\Theta$ by the instantiations of types from $\Psi$, with positions $\sumscope{\position{\Theta}}{\position{\Psi}}$, and $\act{(I',\Delta)}$ similarly.)
  %
  We define:
  \[ g \defeq \sumscope{(\overline{\inlscope} \circ f)}{\position{\Psi}} : \act{(I',\Delta)} \to \act{(I,\Theta)} \Psi. \]
  
  Now $g^* \act{I}J' = \act{I'}J'$ follows directly from \cref{prop:instantiation-boilerplate} (which we will continue to use without further comment), the definitions of $I'$ and $g$, and the following commuting diagram.
  %
  \[
    \xymatrix@C=3em{
      \act{(I',\Delta)} \Psi \ar[d]_{\overline{\inlscope}} \ar[r]^{g} & \act{(I,\Theta)} \Psi \ar[d]^{\overline{\inlscope}} \\   
      \Delta \ar[r]^{\overline{\inlscope} \circ f} \ar[dr]^{f} & \Theta \\
      \act{(I',\Delta)} \emptycxt \ar[r] \ar[u]^{\overline{\inlscope}} & \act{(I,\Theta)} \emptycxt \ar[u]_{\overline{\inlscope}} 
    }
  \]
 
  Next, $g$ clearly acts trivially on $\act{\inrscope}\position{\Psi} \subseteq \position{\act{(I,\Theta)} \Psi}$.
  %
  It also acts trivially on the subset of $\act{\inlscope}{\position{\Theta}}$ corresponding to the given $K \subseteq \sumscope{\position{\Theta}}{\emptyscope}$ on which $f$ acts trivially.
  
  Taking the union of these as the trivial set for $g$, it remains to show that for $i$ in the subset of $\act{\inlscope}{\position{\Theta}}$ corresponding to $\sumscope{\position{\Theta}}{\emptyscope} \setminus K$, we have $\isterm{\act{(I',\Delta)}\Psi}{g(i)}{\tca{g}(\act{(I,\Theta)} \Psi)_i}$.
  % 
  But this judgement is just the renaming of $\isterm{\Delta}{f(j)}{\tca{f}(\act{(I,\Theta)}\emptyset)_i}$ along the evident map $\sumscope{\position{\Delta}}{\emptyscope} \to \sumscope{\position{\Delta}}{\position{\Psi}}$. 
  % 
  So using \cref{lem:admissibility-renaming} to rename the derivation of $\isterm{\Delta}{f(j)}{\tca{f}(\act{(I,\Theta)}\emptycxt)_i}$ supplied with $f$, we are done.
\end{proof}

Next, we show that substitution respects judgemental equality of raw substitutions.
%
For this, we introduce a handy notation: for raw substitutions $f, g : \Gamma' \to \Gamma$ and an object judgement $J$, with head expression $e$ and boundary $B$, we write $\tca{(f \equiv g)} J$ for the equality judgement asserting that $\tca{f} e$ and $\tca{g} e$ are equal over the boundary $\tca{f} B$. Thus $\Gamma' \types \tca{(f \equiv g)} (A \type)$ stands for $\eqtype{\Gamma'}{\tca{f} A}{\tca{g} A}$ and $\Gamma' \types \tca{(f \equiv g)} (e : A)$ stands for $\eqterm{\Gamma'}{\tca{f} e}{\tca{g} e}{\tca{f} A}$.

\begin{lemma}[Admissibility of equality substitution]
  \label{lem:admissibility-equality-substitution}
  Let $T$ be a substitutive and congruous raw type theory over~$\Sigma$.
  %
  Let $f, g : \Gamma' \to \Gamma$ be raw substitutions over~$\Sigma$,
  and $K \subseteq \position{\Gamma}$ a complemented subset such that:
  %
  \begin{enumerate}

  \item \label{item:adm-subst-eq-trivial-case}%
    for each $i \in K$ there exists some (necessarily unique) $j \in \position{\Gamma'}$ such that
    $f(i) = g(i) = \synvar{j}$, and $\Gamma'_j = \tca{f} \Gamma_i$ or $\Gamma'_j = \tca{g} \Gamma_i$,

  \item \label{item:adm-subst-eq-nontrivial-case}%
    for each $i \in \position{\Gamma} \setminus K$,
    $T$ derives $\isterm{\Gamma'}{f(i)}{ \tca{f} \Gamma_i}$ and $\isterm{\Gamma'}{g(i)}{\tca{g} \Gamma_i}$ and $\eqterm{\Gamma'}{f(i)}{g(i)}{\tca{f} \Gamma_i}$ without substitutions.
  \end{enumerate}
  %
  If $T$ derives $\Gamma \types J$ without substitutions, then $T$ derives $\Gamma' \types \tca{f} J$, $\Gamma' \types \tca{g} J$, and (if $J$ is an object judgement) $\Gamma' \types \tca{(f \equiv g)} J$, still without substitutions.
\end{lemma}

The assumption on~$f$ and~$g$ is perhaps a little surprising, especially the last ``or'' in case~\eqref{item:adm-subst-eq-trivial-case}. Another peculiarity is the fact that we include $\Gamma' \types \tca{f} J$ and $\Gamma' \types \tca{g} J$ in the conclusion rather than obtaining them by elimination of substitution.
%
The point is that the induction arguments need to work when we pass into extended contexts of premises, whereby types of the form $\tca{f} A$ \emph{or} $\tca{g} A$ are introduced; so we cannot assume either~$f$ or~$g$ satisfying the conditions of \cref{thm:elimination-substitution} individually, but need to give a condition on them together that is preserved.
%
And since this condition is too weak for applying elimination of substitution to~$f$ or~$g$, we carry the conclusion of that along as well.

\begin{proof}[Proof of \cref{lem:admissibility-equality-substitution}]
  %
  The proof proceeds by induction on the derivation $D$ of $\Gamma \types J$.
  %
  The details are closely analogous to elimination of substitution, so we spell out fewer.

  Consider the case when $D$ ends with a variable rule
  %
  \begin{equation*}
    \infer{
      \istype{\Gamma}{\Gamma_i}
    }{
      \isterm{\Gamma}{\synvar{i}}{\Gamma_i}
    }
  \end{equation*}
  %
  If case~\eqref{item:adm-subst-eq-nontrivial-case} applies for~$i$, we are done immediately.
  %
  If case~\eqref{item:adm-subst-eq-trivial-case} applies, we obtain derivations of $\istype{\Gamma'}{\act{f} \Gamma_i}$, $\istype{\Gamma'}{\act{g} \Gamma_i}$, and $\eqtype{\Gamma'}{\act{f} \Gamma_i}{\act{g} \Gamma_i}$ by induction hypothesis.
  If $\Gamma'_j = \act{f} \Gamma_i$ then $\isterm{\Gamma'}{\synvar{j}}{\act{f} \Gamma_i}$ follows by the variable rule and $\isterm{\Gamma'}{\synvar{j}}{\act{g} \Gamma_i}$ from it by conversion. And of course, $\eqterm{\Gamma'}{\synvar{j}}{\synvar{j}}{\Gamma'_j}$ is derivable by reflexivity. If $\Gamma'_j = \act{g} \Gamma_i$, the situation is symmetric.

  Otherwise, $D$ concludes with an instantiation $\act{I} R$ where $I \in \Inst{\Sigma}{\Gamma}{\arity{R}}$ and~$R$ is either an equality rule, a conversion rule, or a specific rule of~$T$. The conclusion of $\act{I} R$ has the form $\Gamma \types \act{I} J'$.
  %
  We need to derive $\Gamma' \types \tca{f} (\act{I} J')$, $\Gamma' \types \tca{g} (\act{I} J')$, and if $R$ is an object rule then also $\Gamma' \types \tca{(f \equiv g)} (\act{I} J')$.

  Let us first verify that the raw substitutions $f' \defeq \sumscope f {\position{\Delta}} : \ctxextend {\Gamma'}{\act{(\tca{f} I)} \Delta} \to \ctxextend \Gamma {\act{I} \Delta}$ and $g' \defeq \sumscope g {\position{\Delta}} : \ctxextend {\Gamma'}{\act{(\tca{f} I)} \Delta} \to \ctxextend \Gamma {\act{I} \Delta}$ satisfy the conditions~\eqref{item:adm-subst-eq-trivial-case} and~\eqref{item:adm-subst-eq-nontrivial-case} of the lemma for the complemented subset $K' \subseteq \position{\ctxextend{\Gamma}{\act{I} \Delta}} = \position{\Gamma} + \position{\Delta}$, given by
  %
  $
    K' = \set{\inl(i) \such i \in K} \cup \set{\inr(k) \such k \in \position{\Delta}}
  $:
  %
  \begin{enumerate}

  \item
    %
    For $\inl(i) \in K'$, there is $j \in \Gamma'$ such that $f(i) = g(i) = \synvar{j}$ and either $\Gamma'_j = \tca{f} \Gamma_i$ or $\Gamma'_j = \tca{g} \Gamma_i$.
    %
    Now $f'$ and $g'$ satisfy condition~\eqref{item:subst-trivial-case} because $f'(\inl(i)) = \synvar{\inl(j)} = g'(\inl(i))$ and
    $(\ctxextend{\Gamma'}{\act{(\tca{f} I) \Delta}})_{\inl(j)} = \act{{\inl}} \Gamma'_j$, which is equal either to $\act{{\inl}} (\tca{f} \Gamma_i) = \tca{f'} (\ctxextend \Gamma {\act{I} \Delta})$ or to $\act{{\inl}} (\tca{g} \Gamma_i) = \tca{g'} (\ctxextend \Gamma {\act{I} \Delta})$, as the case may be.

  \item
    %
    For $\inr(k) \in K'$, condition~\eqref{item:subst-trivial-case} is satisfied by $f'$ and $g'$: it is clear that
    %
    $
    f'(\inr(k)) = \synvar{\inr(k)} = g'(\inr(k))
    $,
    %
    while
    %
    $
    \tca{f'} (\ctxextend \Gamma {\act{I} \Delta})_{\inr(k)}
    = \tca{f'} (\act{I} \Delta_k)
    = \act{(\tca{f} I)} \Delta_k
    = (\ctxextend {\Gamma'}{\act{(\tca{f} I)} \Delta})_{\inr(k)}
    $
    %
    holds, where we used \cref{prop:instantiation-boilerplate} in the second step.

  \item
    %
    For $\inl(j) \in (\position{\Gamma} + \position{\Delta}) \setminus K'$, $f'$ and $g'$ satisfy \eqref{item:subst-nontrivial-case} because the desired judgements
    %
    \begin{align*}
      \isterm
        {\ctxextend {\Gamma'}{\act{(\tca{f} I)} \Delta} &}
        {f'(\inl(j))}
        {\tca{f'} (\ctxextend \Gamma {\act{I} \Delta})_{\inl(j)}}
      \\
      \isterm
        {\ctxextend {\Gamma'}{\act{(\tca{f} I)} \Delta} &}
        {g'(\inl(j))}
        {\tca{g'} (\ctxextend \Gamma {\act{I} \Delta})_{\inl(j)}}
      \\
      \eqterm
        {\ctxextend {\Gamma'}{\act{(\tca{f} I)} \Delta} &}
        {f'(\inl(j))}
        {g'(\inl(j))}
        {\tca{f'} (\ctxextend \Gamma {\act{I} \Delta})_{\inl(j)}}
      \\
      \intertext{are respectively equal to}
      \isterm
        {\ctxextend {\Gamma'}{\act{(\tca{f} I)} \Delta} &}
        {\act{{\inl}} (f(j))}
        {\act{{\inl}} (\tca{f} \Gamma_j)}
      \\
      \isterm
        {\ctxextend {\Gamma'}{\act{(\tca{f} I)} \Delta} &}
        {\act{{\inl}} (g(j))}
        {\act{{\inl}} (\tca{g} \Gamma_j)}
      \\
      \eqterm
        {\ctxextend {\Gamma'}{\act{(\tca{f} I)} \Delta} &}
        {\act{{\inl}} (f(j))}
        {\act{{\inl}} (g(j))}
        {\act{{\inl}} (\tca{f} \Gamma_j)}
    \end{align*}
    %
    and these are derivable by \cref{lem:admissibility-renaming} applied to the renaming $\inl$ and the assumptions~\eqref{item:subst-nontrivial-case} for~$f$ and~$g$.
  \end{enumerate}

  We similarly check that the raw substitutions $f'' \defeq \sumscope f {\position{\Delta}} : \ctxextend {\Gamma'} {\act{(\tca{g} I)} \Delta} \to \ctxextend \Gamma {\act{I} \Delta}$ and $g'' \defeq \sumscope g {\position{\Delta}} : \ctxextend {\Gamma'} {\act{(\tca{g} I)} \Delta} \to \ctxextend \Gamma {\act{I} \Delta}$ satisfy the conditions of the lemma with the same set $K'$, too. The verification is similar to the case of $f'$ and $g'$ above, and at this point the disjunction in~\eqref{item:subst-trivial-case} lets us exchange the role of~$g$ and~$f$.

  We may now derive $\Gamma' \types \tca{f} (\act{I} J')$ by the closure rule $\act{(\tca{f}I)} R$, as it has the correct conclusion by \cref{prop:instantiation-boilerplate}. To see that its premises are derivable, we verify that for any premise $\Delta \types J''$ of~$R$, the instantiation by $\tca{f} I$, namely
  %
  \begin{equation}
    \label{eq:adm-subst-eq-1}
    \ctxextend {\Gamma'}{\act{(\tca{f} I)} \Delta} \types \act{(\tca{f} I)} J''
  \end{equation}
  %
  is derivable. The corresponding premise of $\act{I} R$, which is
  %
  \begin{equation}
    \label{eq:adm-subst-eq-2}
    \ctxextend{\Gamma} {\act{I} \Delta} \types \act{I} J'',
  \end{equation}
  %
  is derivable by assumption, and so we obtain~\eqref{eq:adm-subst-eq-1} by the induction hypothesis for~\eqref{eq:adm-subst-eq-2} applied to $f'$ and $g'$.

  By a similar argument $\Gamma' \types \tca{g} (\act{I} J')$ is derivable by the closure rule $\act{(\tca{g} I)} R$, we only need to use $f''$ and $g''$ instead of $f'$ and $g'$ to derive the premise
  %
  \begin{equation}
    \label{eq:adm-subst-eq-3}
    \ctxextend {\Gamma'} {\act{(\tca{g} I)} \Delta} \types \act{(\tca{g} I)} J''.
  \end{equation}

  It remains to be checked that $\Gamma' \types \tca{(f \equiv g)} (\act{I} J')$ is derivable when~$R$ is an object rule. Because $T$ is congruous, the congruence rule~$C$ associated with~$R$ is a specific rule of~$T$. Let $\ell, r : \mvextend{\Sigma}{\arity{R}} \to \mvextend{\Sigma}{\arity{C}}$ be the signature maps from \cref{def:congruence-rule}
  and $I' \defeq \tca{f} I + \tca{g} I \in \Inst{\Sigma}{\Gamma'}{\arity{R} + \arity{R}}$. Note that $\act{I'} \circ \act{\ell} = \tca{f} I$ and $\act{I'} \circ \act{r} = \tca{g} I$.
  %
  The instantiation $\act{I'} C$ is a closure rule whose conclusion is precisely $\Gamma' \types \tca{(f \equiv g)} (\act{I} J')$, so we only have to establish that its premises are derivable, of which there are three kinds:
  %
  \begin{enumerate}
  \item For each premise $\Delta \types J''$ of $R$, there is a corresponding premise $\act{I'}(\act{\ell}(\Delta \types J''))$, which is equal to~\eqref{eq:adm-subst-eq-1}.
    We have already seen that it is derivable.
  \item For each premise $\Delta \types J''$ of $R$, there is a corresponding premise $\act{I'}(\act{r}(\Delta \types J''))$, which is equal to~\eqref{eq:adm-subst-eq-3}.
    Its derivability has been established, too.
  \item For each object premise $\Delta \types J''$ of $R$, there is a corresponding premise, namely the associated equality judgement (\cref{def:judgement-associated-congruence}) instantiated by~$I'$. A short calculation relying on \cref{prop:instantiation-boilerplate} shows that the judgement is
    %
    \begin{equation*}
      \ctxextend {\Gamma'}{\act{(\tca{f} I)} \Delta} \types \tca{(f' \equiv g')} J'',
    \end{equation*}
    %
    which is one of the consequences of the induction hypothesis for~\eqref{eq:adm-subst-eq-2} applied to $f'$ and $g'$. \qedhere
  \end{enumerate}
\end{proof}

We can now put these together into the main theorem of this section.

\begin{theorem}[Elimination of substitution]
  \label{thm:elimination-substitution}%
  Let $T$ be a substitutive and congruous raw type theory; then every derivable judgement over $T$ has a substitution-free derivation.
\end{theorem}

\begin{proof}
  Work by induction over the original derivation.
  %
  At substitution rules, apply \cref{lem:admissibility-substitution}; and at equality substitution rules, \cref{lem:admissibility-equality-substitution}.
\end{proof}

% In the presence of Π types, substitution can be replaced by λ-abstraction
% followed by application. However, this means that the β rule, concluding
% app(λs,t)≡s[t] : B[t]) violates presuppositivity in the absence of a
% substitution rule.


%%% Local Variables:
%%% mode: latex
%%% End:



\subsection{Uniqueness of typing}

Whether it is desirable for a term to have many types depends on one's motivations, but certainly in our setting, where the terms record detailed information about premises, we should expect a term to have at most one type, which we prove here.

\begin{theorem}
  \label{thm:tight-uniqueness-of-typing}
  %
  If a tight, substitutive raw type theory $T$ derives $\istype \Gamma A$, $\istype \Gamma B$, $\isterm \Gamma t A$ and $\isterm \Gamma t B$ then it also derives $\eqtype \Gamma A B$.
\end{theorem}

\begin{proof}
  By \cref{thm:elimination-substitution} it suffices to prove the claim for substitution-free derivations.
  %
  % The precise statement we prove is as follows:
  % %
  % \begin{quote}
  %   For all substitution-free derivations $D_A$, $D_B$, $D_1$ and $D_2$,
  %   and for all $\Gamma$, $t$, $A$, $B$,
  %   if the conclusion of $D_A$ is $\istype \Gamma A$
  %   the conclusion of $D_B$ is $\istype \Gamma B$,
  %   the conclusion of $D_1$ is $\isterm \Gamma t A$,
  %   and the conclusion of $D_2$ is $\isterm \Gamma t B$,
  %   then there exists a derivation of $\eqtype \Gamma A B$.
  % \end{quote}
  % %
  Suppose we have derivations $D_A$, $D_B$, $D_1$ and $D_2$:
  %
  \begin{mathpar}
    \infer
    {D_A}
    {\istype{\Gamma}{A}}

    \infer
    {D_B}
    {\istype{\Gamma}{B}}

    \infer
    {D_1}
    {\isterm{\Gamma}{t}{A}}

    \infer
    {D_2}
    {\isterm{\Gamma}{t}{B}}
  \end{mathpar}
  %
  The proof proceeds by a double induction on the derivations $D_1$
  and $D_2$.

  Consider the case where $D_1$ ends with a conversion:
  %
  \begin{equation*}
  \infer
    {    \infer {D_{1,A'}} {\istype \Gamma {A'}}
    \and \infer {D_{1,A}} {\istype \Gamma A}
    \and \infer {D_{1,t}} {\isterm \Gamma t {A'}}
    \and \infer {D_{1,\mathrm{eq}}} {\eqtype \Gamma {A'} A}}
    {\isterm \Gamma t A}
  \end{equation*}
  %
  We apply the induction hypothesis to $D_{1,t}$ and $D_2$ to derive $\eqtype \Gamma {A'} {B}$. The desired $\eqtype \Gamma A B$ now follows from $D_{1,\mathrm{eq}}$ by symmetry and transitivity of equality.
  %
  The case where $D_2$ ends with a conversion is symmetric, except that it does not require the use of symmetry.

  Consider the case where $D_1$ ends with a variable rule:
  %
  \[ \infer {D_1'}
    {\infer {\istype \Gamma {\Gamma_j}} {\isterm \Gamma {\synvar j}{\Gamma_j}}}
  \]
  %
  Because $T$ is tight $D_2$ must end with a variable or a conversion rule. We have already dealt with the latter one.
  If~$D_2$ ends with a variable rule, then $A = \Gamma_j = B$, and we may conclude $\eqtype{\Gamma}{A}{B}$ by reflexivity.

  In the remaining case $D_1$ and $D_2$ both end with instantiations of specific rules of~$T$.
  Let $\beta$ be the map which takes each symbol $S \in \Sigma$ to the corresponding symbol rule in~$T$.
  There is a unique symbol $S \in \Sigma$, such that $D_1$ and $D_2$ both end with instantiations of $\beta(S)$:
  %
  \begin{mathpar}
    \inferrule{
      D_1
    }{
      \isterm \Gamma {\act I S(\fammap{\synmeta{i}}{i \in \args S})} {\act I C}
    }

    \inferrule{
      D_2
    }{
      \isterm \Gamma {\act J S(\fammap{\synmeta{j}}{j \in \args S})} {\act J C}
    }
  \end{mathpar}
  %
  Of course, $\act I C$ is just $A$ and $\act J C$ is $B$, and both heads are equal to~$t$, from which it follows that
  %
  \begin{equation*}
    S(\fammap{\act I {\synmeta i}}{i \in \args S}) = S(\fammap{\act J {\synmeta i}}{i \in \args S}),
  \end{equation*}
  %
  and so $I$ and $J$ are equal because they match on every $i \in \args S$.
  %
  Thus $A = \act I C = \act J C = B$, and we may derive $\eqtype \Gamma A B$ by reflexivity.
\end{proof}

We record a more economical version of uniqueness of typing, which one can afford in reasonable situations.

\begin{corollary}
  \label{cor:accaptable-uniqueness}
  %
  If an acceptable type theory derives $\isterm{\Gamma}{e}{A}$ and $\isterm{\Gamma}{e}{B}$ then it also derives
  $\eqtype{\Gamma}{A}{B}$.
\end{corollary}

\begin{proof}
  Apply \cref{thm:presuppositions} to $\isterm{\Gamma}{e}{A}$ and $\isterm{\Gamma}{e}{B$} to obtain
  $\istype{\Gamma}{A}$ and $\istype{\Gamma}{B}$, and conclude by \cref{thm:tight-uniqueness-of-typing}.
\end{proof}

Acceptability also easily gives us uniqueness of typing for term equalities.

\begin{corollary}
  If an acceptable type theory derives $\eqterm{\Gamma}{s}{t}{A}$ and $\eqterm{\Gamma}{s}{t}{B}$ then it also derives $\eqtype{\Gamma}{A}{B}$.
\end{corollary}

\begin{proof}
  Again, apply \cref{thm:presuppositions} to get $\istype{\Gamma}{A}$ and $\istype{\Gamma}{B}$, and conclude by \cref{thm:tight-uniqueness-of-typing}.
\end{proof}


\subsection{An inversion principle}

Given the fact that a judgement $\Gamma \types J$ is derivable, to what extent can a derivation of it be constructed just from the information given in the judgement? We show in this section that, for sufficiently well-behaved type theories, one can read off the proof-relevant part of a derivation from the head of~$J$. The proof-irrelevant parts are the applications of conversion rules, and subderivations of equalities. The former may be arranged to always appear just once after variable and symbol rules, while the latter must be dealt with on a case-by-case basis, as a particular type theory may or may not possess an algorithm that checks derivability of equalities.

When we attempt to reconstruct a derivation from a judgement, the first obstacle we face is what types should be given to the subterms appearing in a judgement. For the variables the answer is clear, while for symbol expressions it is natural to use the types dictated by the corresponding rules, as follows.

\begin{definition}
  Let $T$ be a tight raw type theory over $\Sigma$ and $\beta$ the assignment of rules to the symbols of~$\Sigma$.
  %
  Thus for each term symbol $S \in \Sigma$, the conclusion of $\beta(S)$ takes the form
  %
  $
    \isterm{}
    {\genapp{S}}{A_S}
  $
  %
  for some $A_S \in \Expr{\Ty}{\mvextend{\Sigma}{\arity{R}}}{\emptyscope}$.
  %
  Given a term expression $t \in \Expr{\Tm}{\Sigma}{\Gamma}$, its \defemph{natural type} $\natty{\Gamma}{t} \in \Expr{\Ty}{\Sigma}{\Gamma}$ is defined by
  %
  \begin{equation*}
    \natty{\Gamma}{\synvar{i}} \defeq \Gamma_i
    \qquad\text{and}\qquad
    \natty{\Gamma}{S(e)} \defeq \act{e} A_S,
  \end{equation*}
  %
  % e ∈ Π(i : arg S) Expr (cl_S i) Σ (Γ ⊕ bind_S i)
  %
  where we used $e \in \prod_{i \in \args S} \Expr{\argclass{S}{i}}{\Sigma}{\sumscope{\gamma}{\argbinder{S}{i}}}$ as an instantiation, so that $\act{e} A_S$ is the expression in which each $\synmeta{i}(e')$ is replaced by $\tca{(e')} e_i$.
\end{definition}

To put it more simply, the natural type of $S(e)$ is the type one obtains by applying the symbol rule for~$S$ to the premises determined by~$e$.

\begin{theorem}[Inversion principle]
  \label{thm:inversion-principle}%
  Let $T$ be an acceptable type theory over~$\Sigma$.
  %
  \begin{enumerate}

  \item If $T$ derives $\isterm{\Gamma}{\synvar{i}}{A}$ then it does so by an application of a variable rule, followed by a conversion:
    %
    \begin{equation*}
      \infer{
        \infer{D'}{\isterm{\Gamma}{\synvar{i}}{\Gamma_i}}
        \\
       \infer{D''}{\eqtype{\Gamma}{\Gamma_i}{A}}
      }{
        \isterm{\Gamma}{\synvar{i}}{A}
      }
    \end{equation*}
    %
  \item If $T$ derives $\isterm{\Gamma}{S(e)}{A}$ then it does so by an application of the symbol rule for~$S$, followed by a conversion:
    %
    \begin{equation*}
      \infer{
        \infer{D'}{\isterm{\Gamma}{S(e)}{\natty{\Gamma}{S(e)}}}
        \\
       \infer{D''}{\eqtype{\Gamma}{\natty{\Gamma}{S(e)}}{A}}
      }{
        \isterm{\Gamma}{S(e)}{A}
      }
    \end{equation*}
    %
  \item If $T$ derives $\istype{\Gamma}{S(e)}$ then it does so by an application of the symbol rule for~$S$.
  \end{enumerate}
  %
\end{theorem}

\begin{proof}
  Let $T$ be an acceptable type theory over~$\Sigma$, and $\beta$ the assignment of symbol rules to the symbols of~$\Sigma$.
  %
  To establish the first two claims, we proceed by induction on a substitution-free derivation $D$, which exists by \cref{thm:elimination-substitution}.

  If $D$ ends with a variable rule,
  %
  \begin{equation*}
    \infer{
      \infer{D}{\istype{\Gamma}{\Gamma_i}}
    }{
      \isterm{\Gamma}{\synvar{i}}{\Gamma_i}
    }
  \end{equation*}
  %
  then we obtain the desired derivation by attaching a dummy conversion rule:
  %
  \begin{equation*}
    \infer{
      \infer{
        \infer{D}{\istype{\Gamma}{\Gamma_i}}
      }{
        \isterm{\Gamma}{\synvar{i}}{\Gamma_i}
      }
      \\
      \infer{
        \infer{D}{\istype{\Gamma}{\Gamma_i}}
      }{
        \eqtype{\Gamma}{\Gamma_i}{\Gamma_i}
      }
    }{
      \isterm{\Gamma}{\synvar{i}}{\Gamma_i}
    }
  \end{equation*}
  %
  Otherwise, $D$ ends with an application of the conversion rule
  %
  \begin{equation*}
    \infer{
      \infer{D'}{\isterm{\Gamma}{\synvar{i}}{B}}
      \\
      \infer{D''}{\eqtype{\Gamma}{B}{A}}
    }{
      \isterm{\Gamma}{\synvar{i}}{A}
    }
  \end{equation*}
  %
  We apply the induction hypothesis to $D'$ to obtain a derivation of the form
  %
  \begin{equation*}
    \infer{
      \infer{D^*}{\isterm{\Gamma}{\synvar{i}}{\Gamma_i}}
      \\
      \infer{D^{**}}{\eqtype{\Gamma}{\Gamma_i}{B}}
    }{
      \isterm{\Gamma}{\synvar{i}}{B}
    }
  \end{equation*}
  %
  Using the transitivity rule, we combine $D^{**}$ and $D''$ into a derivation of $\eqtype{\Gamma}{\Gamma_i}{A}$, which can then be used together with $D^{*}$ to get the desired form of derivation.

  If $D$ ends with an application of the symbol rule $\beta(S)$,
  %
  \begin{equation*}
    \infer{D'}{\isterm{\Gamma}{S(e)}{\natty{\Gamma}{S(e)}}},
  \end{equation*}
  %
  then by \cref{thm:presuppositions} there is a derivation $D''$ of the presupposition $\istype{\Gamma}{\natty{\Gamma}{S(e)}}$. We apply reflexivity to $D''$ to obtain $\eqtype{\Gamma}{\natty{\Gamma}{S(e)}}{\natty{\Gamma}{S(e)}}$, and then conversion to get the desired derivation.
  %
  Otherwise, $D$ ends with an application of a conversion rule, in which case we proceed as in the variable case.

  The third claim is trivial, because $\beta(S)$ is the only rule which can be instantiated to have the conclusion $\istype{\Gamma}{S(e)}$, apart from substitution rules, which we have dispensed with.
\end{proof}

The above theorem may be applied repeatedly to obtain a canonical form of the proof-relevant part of a derivation. The missing subderivations of equalities must be provided by other means. Also notice that it is easy enough to avoid insertion of unnecessary appeals to conversion rules along reflexivity.

A useful consequence of \cref{thm:inversion-principle} is the fact that a type of a term may be calculated directly from the term (and the symbol rules), as long as it has one.

\begin{corollary}
  In an acceptable type theory, a typeable term has its natural type.
\end{corollary}

\begin{proof}
   Whenever $\isterm{\Gamma}{e}{A}$ is derivable, then so is $\isterm{\Gamma}{e}{\natty{\Gamma}{e}}$ because its derivation appears as a subderivation in the statement of
   \cref{thm:inversion-principle}.
\end{proof}



% \subsection{Alternative forms of rules}

% \placeholder{Structural rules: Certain specific economical forms of structural rules are equivalent to our paranoid forms}

% \placeholder{Logical rules: “Economical forms” of logical rules (defined in some general way) are equivalent to paranoid forms.}

% \begin{theorem}
%   Suppose $\mathcal{T}$ is a presuppositive type theory. Let $R$ be a rule of $\mathcal{T}$ and let $P$ be a premise of $R$ which is already a presupposition of one of the other premises of~$R$. Let $R'$ be the rule which is like $R$, but with $P$ removed. Let $\mathcal{T}'$ be the type theory in which $R$ is replaced with~$R'$. Then $\mathcal{T}'$ is presuppositive and it derives the same judgements as~$\mathcal{T}$.
% \end{theorem}

% TODO Is the above theorem true in any sense? Is it about raw type theories, or some other level of type theories? Do we really need to roll the assumption that the theory is presuppositive with the rest of the induction? Yes, if the proof relies on the fact that the theory enjoys the presupposition theorem, or else we will not be able to iterate applications of the theorem to make multiple changes to a type theory.


% \subsection{Counterexamples}

% Possibly give some pathological theories here, showing some badly-behaved raw type theories failing the given conditions?

\section{Well-founded presentations}
\label{sec:well-founded-type-theories}

So far, our type theories have omitted one typical characteristic occurring in practice: the \emph{ordering} of the presentation of the theory.
%
This ordering appears, implicitly or explicitly, at three levels:
\begin{enumerate}
\item The \emph{positions of a context} usually form a finite sequence, and each type depends only on the preceding part of the context.
\item The \emph{premises of each rule} typically follow some well-founded order, usually simply a finite sequence, and the boundary of each premise depends only on the earlier ones.
\item The \emph{rules of the theory} are themselves well-founded, and each rule depends only on the earlier rules. This order is quite often infinite, in for instance theories with hierarchies of universes, and need not be total, as seen in the example below.
\end{enumerate}

At each of these three levels, the “depends only on” holds in two senses:
\begin{enumerate}
\item \emph{Raw expressions}: each type expression of a context uses only the preceding variables; in a rule, the expressions of each premise boundary only use previously-introduced metavariables; and in a theory, the raw premises and boundary of a rule only use previously-introduced symbols of the theory.
\item \emph{Derivations for presuppositivity}: each type expression in a context is a derivable type over just the preceding part of the context; each premise of a rule can be checked well-formed using just the preceding premises; and so can each rule using just the earlier rules.
\end{enumerate}

\begin{example}
The $\symPi$-formation rule uses no symbols of the signature in its premises or boundary, and relies only on structural rules for its well-formedness.
%
The rules for $\lambda$-abstraction and function application both use $\symPi$ in their raw expressions, and depend on the $\symPi$-formation rule for their reasonability, but not on any other earlier symbols or rules.
%
And the $\beta$-reduction rule in turn depends on all three of these.

Similarly, there is a natural order within the rules for $\symSigma$-types.
%
On the other hand, neither of the $\symSigma$- or $\symPi$-type groups naturally precedes the other, and it would be unnatural to force them into a total order.
\end{example}

Traditionally, well-foundedness is treated in two different ways, depending on the levels.
%
Since the class of all contexts of a theory is formally defined --- contexts are ``user-definable'' --- their well-foundedness must be explicitly mandated somehow; and so it is, usually by the context judgement $\iscxt{ \Gamma }$.

Rules and theories by contrast are not “user-definable”: each development usually presents a single theory, or a few, with a specific collection of rules.
%
The ordering on these can therefore be left entirely unstated,
%
but it is almost always clearly present.
%
The writer ensures it when setting up the theory; the reader follows it when understanding the theory, and convincing themself of its reasonableness; and it is respected in later proofs and constructions.

It would be jarring, for instance, and often logically impossible, to give the semantics of $\beta$-reduction before that of $\lambda$-abstraction.
%
On the other hand, it would be unsurprising if a writer introduced the rules for $\symPi$-types before those for $\symSigma$-types, but then gave semantics with $\symSigma$-types first.
%
Overall, the implicit partial order on rules is always respected, but no particular total extension of it is.

Defining what it means for a presentation to be ordered is a little subtler than one might expect;
%
we work up to it gradually, considering contexts first, then rules, and finally theories.

\subsection{Sequential contexts}
\label{sec:sequential-contexts}

In our setting, with scoped syntax, and with contexts as maps from positions to types (we henceforth refer to these as \emph{flat} contexts), traditional sequential contexts may be recovered in various ways.
%
They are all straightforwardly equivalent --- indeed, a sufficiently informal statement of the traditional definition could be read as any of them --- but explicitly comparing them provides a useful warmup for the less straightforward cases with rules and type theories later.

Recall that $\numscope{n}$ denotes the sum of $n \in \N$ copies of $\numscope{1}$.
%
When working with sequential contexts, we will identify the positions of $\numscope{n}$ with $\{0, \ldots, n-1\}$, and denote the evident “subscope inclusion” maps by $\numscopemap{i}{j} : \numscope{i} \to \numscope{j}$.

\begin{definition}[Sequential context I]
  \label{def:seq-cxt-as-wellpresentedness}
  A \defemph{raw sequential context} over a signature $\Sigma$ is a list $\Gamma
  = [\Gamma_0,\ldots,\Gamma_{n-1}]$, where $\Gamma_i \in \ExprTy{\Sigma}{\numscope{i}}$, for each $i \in \numscope{n}$.
  %
  We write $\Gamma_{<i}$ for the initial segment $[\Gamma_0, \ldots, \Gamma_{i-1}]$.
  %
  The \defemph{flattening} of a raw sequential context is the raw flat context of scope $\numscope{n}$ whose $i$-th type is the weakening $\rename{{\numscopemap{i}{n}}}(\Gamma_i)$.
  %
  We typically leave flattening implicit, writing $\Gamma$ both for a sequential context and its flattening.
  
  Given a signature $\Sigma$ and a raw type theory $T$ over it,
  a raw sequential context $\Gamma$ over $\Sigma$ is \defemph{well-formed over~$T$} if for each $i \in \Gamma$, the judgement $\istype{\Gamma_{<i}}{\Gamma_i}$ is derivable.
\end{definition}

Alternately, we can define sequentiality as a property of flat contexts:

\begin{definition}[Sequential context II]
  \label{def:seq-cxt-by-variable-occurrence}%
  A raw flat context $\Gamma$ of scope $\numscope{n}$ is \defemph{sequential} if for each $i \in \numscope{n}$, all variables $\synvar{j}$ occurring in $\Gamma_i$ have $j < i$.
  %
  Thus each $\Gamma_i$ is uniquely of the form $\rename{{\numscopemap{i}{n}}}(\overline{\Gamma_i})$,
  from which we define the initial segments $\Gamma_{<i}$ as sequential raw contexts of scope $\numscope{i}$.

  A sequential context $\Gamma$ over $\Sigma$ is \defemph{well-formed over $T$} if for each $i \in \Gamma$, the judgement $\istype{\Gamma_{<i}}{\Gamma_i}$ is derivable.
\end{definition}

Finally, we can define well-formed flat contexts via the traditional derivation rules,
without reference to raw sequential contexts.

\begin{definition}[Sequential context III]
  \label{def:seq-cxt-by-rules}%
  The property $\iscxt{\Gamma}$, read as “$\Gamma$ is \defemph{sequentially well-formed}” over a given theory, is the inductive predicate on flat contexts defined by the following closure conditions, the latter for all suitable $\Gamma$, $A$:
\begin{mathpar}
\infer{{}% do not remove the {} preceding this comment
}{\iscxt{\emptycxt}}
\and
\infer{\iscxt{\Gamma} \\ \istype{\Gamma}{A}}{\iscxt{\ctxextend{\Gamma}{A}}}
\end{mathpar}
% % I commented this out because I have no idea what it means, and I understand the definition without it.
% where by “$\iscxt{\ctxextend{\Gamma}{A}}$” we mean derivability of this judgement over $T$.
\end{definition}

Each of the above definitions moreover has two possible readings: \emph{proof-relevant}, where by derivability of a judgement $\istype{\Gamma}{A}$ we mean that a specific derivation is given, and \emph{proof-irrelevant}, where we merely mean that some derivation exists.
%
We take the proof-relevant reading in all cases.

\begin{proposition}  \Cref{def:seq-cxt-as-wellpresentedness,def:seq-cxt-by-variable-occurrence,def:seq-cxt-by-rules} are all equivalent, as predicates on flat contexts, in both their proof-relevant and -irrelevant forms.
\end{proposition}

\begin{proof}
Essentially straightforward, given the fact, already mentioned in \cref{def:seq-cxt-by-variable-occurrence}, that the variable-occurrence constraint there precisely characterises the images of the weakenings $\rename{{\numscopemap{i}{n}}} : \ExprTy{\Sigma}{\numscope{i}} \injto \ExprTy{\Sigma}{\numscope{n}}$.
\end{proof}

Of these definitions, \cref{def:seq-cxt-by-variable-occurrence} is the simplest to state, especially if one sweeps under the rug the inverse-weakening required for defining initial segments.
%
However, when spelling out details carefully, this inverse-weakening is tedious to keep track of.
%
When we bump these definitions up to sequential rules or type theories, therefore, we will focus on approaches based on \cref{def:seq-cxt-as-wellpresentedness,def:seq-cxt-by-rules}.

\subsection{Sequential rules}
\label{sec:sequential-rules}


Next, we wish to define sequential \emph{rules}, in which premises form a finite sequence, and each refers only to the previous ones.
%
Analogously to \cref{def:seq-cxt-by-variable-occurrence}, the easiest version to state is to start from an ordinary raw rule, and add desirable properties, together with restrictions on how earlier parts can be used in later parts:

\begin{definition}[Sequential rule, provisional]
  Let $R$ be a tight raw rule over a signature~$\Sigma$, with premises indexed by $\numscope{n}$, and $\beta$ the bijection witnessing its tightness.
  %
  We say that $R$ is \defemph{sequential} if for all $i \in \numscope{n}$ and $j \in \arity{R}$, if $\synmeta{j}$ appears in the $i$-th premise, then $\beta(j) < i$.

  Moreover, say that $R$ is \defemph{(sequentially) well-formed}
  over a raw type theory~$T$, also over~$\Sigma$,
  when for all $i \in [n]$, the presuppositions of the $i$-th premise of~$R$
  can be derived from the premises indexed by~$[i]$.
\end{definition}

This definition is adequate, but is in several regards somewhat unsatisfying:
%
\begin{enumerate}
\item
  We have said here, as in \cref{def:seq-cxt-by-variable-occurrence}, that the premises are formed over the extension by all the metavariables, and their presuppositions derived from all the premises, but use only the preceding ones.
  %
  When applying this condition, one typically wants to consider them as formed, or derived, over the extension by just the preceding initial segment. So rather than restricting them back there, and having to keep track of such restriction, it is simpler to say from the start, as in \cref{def:seq-cxt-as-wellpresentedness}, that they are formed, or derived, over those initial segments.
  
\item “Tightness” gives a redundancy of data in two ways.
  %
  Firstly, the heads of all object-judgement premises are redundant: each head must be the corresponding metavariable, applied to all variables of its scope.
  %
  Besides this, the arity itself is determined by the indexing family of the rule together with the scopes and forms of the premises.
\end{enumerate}

These issues can be remedied by defining sequential presentations inductively, analogously to \cref{def:seq-cxt-by-rules}, and adding each premise not as a full judgement but just its \emph{boundary}, whose head, if any, will be filled in automatically to ensure tightness by construction.

\begin{definition}
  \label{def:sequential-premise-family}%
  %
  Given a signature $\Sigma$ and a raw type theory $T$ over it, we define inductively the \defemph{sequential premise-families} $P$ with arities $\arity{P}$, and simultaneously their \defemph{flattenings} as families of judgements over $\mvextend{\Sigma}{\arity{P}}$, as follows.

  \begin{enumerate}
  \item
    %
    The \defemph{empty sequential premise-family} $\emptyfam$ has empty arity $\arity{\emptyfam}$, and its flattening is the empty family.
    
  \item
    %
    Let $P$ be a sequential premise-family, with arity $\arity{P}$ and flattening $F$.
    % 
    Let $\Delta \typesboundary B$ be a boundary over $\mvextend{\Sigma}{\arity{P}}$, such that all presuppositions of $\Delta \typesboundary B$ are derivable over $\mvextend{(T+F)}{\arity{P}}$ (i.e.\ the translation of $T+F$ from $\Sigma$ to $\mvextend{\Sigma}{\arity{P}}$), and $\Delta$ is a well-formed sequential context over the same theory.

    Then there is an \defemph{extension sequential premise-family} $P;(\Delta \typesboundary B)$.
    
    If $\Delta \typesboundary B$ is an object boundary of class~$c$, then the associated arity $\arity{P;(\Delta \typesboundary B)}$ is $\arity{P} + \singletonfamily{(c,\position{\Delta})}$; or if $\Delta \typesboundary B$ is an equality boundary, $\arity{P;(\Delta \typesboundary B)}$ is just $\arity{P}$.
    
    The flattening of $P;(\Delta \typesboundary B)$ is $\famtuple{\act{\iota}(\Gamma_i \typesjudgement J_i)}{i \in I} + \singletonfamily{(\act{\iota}{\Delta}) \typesjudgement J}$, where $\iota$ is the inclusion $\arity{P} \to \arity{P;(\Delta \typesboundary B)}$, and $J$ is $\act{\iota}B$
    with the head, if any, filled by the expression $\synmeta{\star}(\fammap{\synvar{i}}{i \in \delta})$, where $\star$ is the new argument adjoined to the arity.
  \end{enumerate}
\end{definition}

\noindent%
Notationally, we will not distinguish the flattening from the sequential premise-family itself.

\begin{definition}
  \label{def:sequential-rule}%
  A \defemph{sequential rule} $\SequentialRule{P}{J}$ over $\Sigma$ and $T$ is a sequential premise-family~$P$, together with a judgement $ \Gamma \typesjudgement J$ over $\mvextend{\Sigma}{\arity{P}}$,
  the \defemph{conclusion}, whose presuppositions are derivable over $T + P$ (with $T$ translated to the metavariable extension).
  %
  A sequential rule has an evident flattening as a raw rule.
\end{definition}

\noindent%
Again, we do not notate the flattening explicitly.
%
Note that as in the definition of raw rules the conclusion~$J$ has an empty context.

The addition of a boundary in the extension step of  \cref{def:sequential-premise-family} is precisely analogous to the traditional context extension rule, as in \cref{def:seq-cxt-by-rules}.
%
There, the extension is \emph{specified} just by a type $A$, but its \emph{effect} is to add a term-of-type judgement $\synvar{i} \of A$,
%
where the term is automatically determined to be a (fresh) variable, rather than specified as input to the extension.

Reading \cref{def:sequential-premise-family} with an eye towards computer-formalisation, one may note it can be formalized in several ways: as an inductive-recursive definition of a set with functions to arities and families of judgements; as an inductive family of sets indexed over pairs of an arity and a family of judgements; or an $\N$-indexed sequence of sets together with functions to arities and families, by induction on $n \in \N$, the length of the family.
%
These are all equivalent, by standard generalities about inductive definitions.

\begin{proposition}
  The flattening of a sequential rule with empty conclusion context is acceptable.
\end{proposition}

\begin{proof}
  Tightness is immediate, by construction of the arity of the premise-family and the heads of its object-judgement premises.
  %
  Presuppositivity is similarly by construction, from the well-formedness conditions in the definitions of sequential rules and premise-families, with the latter inductively translated along metavariable extensions as the premise-family is built up.
\end{proof}

As we defined premise-families using just boundaries rather than complete judgements, similarly when we define well-founded type theories we will specify them using sequential rules whose conclusions have no heads.
%
We will also (for substitutivity) restrict attention to empty conclusion contexts.

\begin{definition}
  \label{def:rule-boundary}%
  A \defemph{sequential rule-boundary} $\RuleBoundary{P}{B}$ over $\Sigma$ and $T$ is a sequential premise-family~$P$, together with a boundary $\Gamma \typesboundary B$ over $\mvextend{\Sigma}{\arity{P}}$ with empty context, whose presuppositions are derivable over $T + P$.
\end{definition}

Rule-boundaries can of course be completed to rules, by filling in a head if required.

\begin{definition}
  \label{def:rule-boundary-realisation}%
  The \defemph{realisation} of a sequential rule-boundary $R = (\RuleBoundary{P}{B})$ as a sequential rule (or, via flattening, a raw rule) is
  defined according to the form of $B$:
  %
  \begin{enumerate}
  \item
    %
    If $B$ is an object boundary of class~$c$, then given a symbol $\symS \in \Sigma$ with arity $\arity{P}$ and class~$c$, the \defemph{realisation $\plug{R}{\symS}$ of $R$ with $\symS$} is the sequential rule
    %
    \begin{equation*}
      \SequentialRule
      {P}
      {\plug{B}{\genapp{S}}}
    \end{equation*}
    %
    given by completing $B$ with the generic application of~$S$.

    \item
    %
    If $B$ is an equality, no further input is required: the \defemph{realisation of $R$} is just $\RuleBoundary{P}{B}$ with $B$ viewed as a judgement.
\end{enumerate}
\end{definition}

This gives, by construction:

\begin{propositionwithqed}
  The realisation of an object rule-boundary for~$\symS$ yields a symbol rule for~$\symS$.
\end{propositionwithqed}

\begin{example}
  The sequential rule-boundary
  %
  \begin{equation*}
    \RuleBoundary{
      (\istype{}{\symA}) ;
      (\istype{x \of \symA}{\symB(x)})
    }{
      (\istypebdry{})
    }
  \end{equation*}
  %
  realised with the symbol $\symPi$ gives the sequential rule
  %
  \begin{equation*}
    \SequentialRule{
      (\istype{}{\symA}) ;
      (\istype{x \of \symA}{\symB(x)})
    }{
      (\istype{}{\symPi(\symA, \symB(x))})
    }
  \end{equation*}
  %
  whose flattening is the usual formation rule for dependent products, as in \cref{ex:pi-congruence-rule}.
  %
  With $\symSigma$ instead of $\symPi$, it gives the formation rule for dependent sums.
\end{example}

\subsection{Well-presented rules}
\label{sec:well-presented-rules}

Sequential rules and rule-boundaries give a satisfactory treatment covering most example theories, and sufficing for many purposes, including implementation in proof assistants.
%
For instance, the Andromeda proof assistant~\citep{andromeda,bauer19} implements a variant of sequential rules and rule boundaries in the trusted nucleus.

Here we consider the generalisation from finite sequences to arbitrary well-founded orders, partly to encompass infinitary rules, but mainly as a warm-up for well-founded theories.

\Cref{def:well-founded-premises-shape,def:well-founded-premise-family,def:realisation-well-presented-rule} given below are rather long and pedantic, so we give first a guiding overview.
%
We follow the pattern first seen in \cref{def:seq-cxt-as-wellpresentedness}, where the components of the definitions must be stratified into several stages, with each stage making use of functions defined on earlier stages.

\begin{enumerate}
\item
  At the first stage, we can specify just the \emph{shape} of the family of premises.
  %
  This consists of a well-ordered set $(P,<)$, to index the premises,
  %
  along with for each $i \in P$, the judgement form~$\varphi_i$ and scope~$\gamma_i$ for the $i$-th premise.

  Form this data we can compute the arity of the rule, $\arity{P}$, and more generally the arity $\arity{P_{<i}}$, specifying what metavariables may occur in the $i$-th premise.
  
\item
  At the second stage, with the arities $\arity{P_{<i}}$ available, we can specify the \emph{raw syntax} of the premises.
  %
  The $i$-th premise $P_i$ is given by a boundary $\Gamma_i \typesboundary B_i$ of form~$\varphi_i$, scope~$\gamma_i$, and written over $\mvextend{\Sigma}{\arity{P_{<i}}}$.

  From these, filling in heads of object premises as required for tightness, we can construct the flattening of $P$ as family of judgements over $\mvextend{\Sigma}{\arity{P}}$, and more generally the flattening of $P_{<i}$ over $\mvextend{\Sigma}{\arity{P_{<i}}}$.

\item
  At the third stage, with the flattenings available, we can now specify the \emph{well-formedness} conditions.
  % 
  Derivations of presuppositions of $P_i$ should be given over the ambient theory~$T$, translated up to $\mvextend{\Sigma}{\arity{P_{<i}}}$, and with (the flattenings of) preceding premises $P_{<i}$ available as hypotheses.

\item
  We are now done with the hard part.
  %
  Having specified the premises, the conclusion is given as in the sequential case, as a well-formed boundary $B$ over $\mvextend{\Sigma}{\arity{P}}$, whose head (if $B$ is of object form) will later be filled in to yield a symbol rule.
\end{enumerate}

While the above explanation sounds plausible, it sweeps several technical subtleties under the rug.
%
Most importantly, since each premise is specified over its own signature $\mvextend{\Sigma}{\arity{P_{<i}}}$, we need to handle the translations between these extensions, and up to the overall extension $\mvextend{\Sigma}{\arity{P}}$.
%
Spelling out all details in full, we have the following definitions.

\begin{definition}
  \label{def:well-founded-premises-shape}%
  A \defemph{well-founded premises-shape} $(I, S)$ is given by
  %
  a well-founded set $(I, {<})$, and
  %
  a family $S = \fammap{(\varphi_i,\gamma_i)}{i \in I}$, where $\varphi_i$ is a judgement form and $\gamma_i$ a scope.
  %
  Given these, we define:
  %
  \begin{enumerate}
  \item
    %
    The arity $\arity{S}$ of $S$ is the subfamily $\fammap{(\varphi_i, \gamma_i)}{\set{i \in I \such \varphi_i \in \set{\Ty, \Tm}}}$ of the object forms of~$S$.

  \item
    %
    For each $i \in I$, the initial segment $S_{<i} \defeq \fammap{(\varphi_j, \gamma_j)}{j < i}$ is itself a well-founded premises-shape indexed by the initial segment $\initialSegment{i} \subseteq I$, and hence it also has an associated arity $\arity{S_{< i}}$.

  \item
    %
    For each $i < j \in I$, there are evident family maps $\arity{S_{<i}} \to \arity{S_{<j}}$ and hence $S_{<i} \to S_{<j}$, satisfying evident composition conditions with each other and with the subfamily inclusions $S_{<j} \to S$.
  \end{enumerate}
\end{definition}

\begin{definition}
  \label{def:well-founded-premise-family}%
  %
  Given a signature~$\Sigma$ and a well-founded premises-shape~$(I, S)$ as in \cref{def:well-founded-premises-shape}, a \defemph{well-founded premise-family~$P$} is given by
  %
  a family $B = \fammap{B_i}{i \in I}$ where $B_i$ is a boundary of form $\varphi_i$ in scope $\gamma_i$, and over $\mvextend{\Sigma}{\arity{S_{<i}}}$.
  %
  Given these, we define:
  %
  \begin{enumerate}
  \item
    % 
    The \defemph{flattening~$P^\flat$} is the family of judgements $\fammap{P_i}{i \in I}$ over $\mvextend{\Sigma}{\arity{S}}$, where $P_i$ is the boundary $B_i$ translated along the inclusion $\mvextend{\Sigma}{\arity{S_{<i}}} \to \mvextend{\Sigma}{\arity{S}}$, and when $\varphi_i \in \set{\Ty, \Tm}$ completed with the head expression $\synmeta{i}(\fammap{\synvar{j}}{j \in \gamma_i})$.

  \item
    %
    For each $i \in I$, the initial segment $B_{<i}$ yields a well-founded premise family $P_{<i}$ with respect to the well-founded premises-shape $S_{<i}$ indexed by the initial segment $\initialSegment{i}$. Thus it has its own flattening~$P_{<i}^\flat$.

  \item
    %
    For each $j < i$, the judgement $P_j$ as a member of the flattening $P_{<i}^\flat$ translated along the signature inclusion $\mvextend{\Sigma}{\arity{S_{<i}}} \to \mvextend{\Sigma}{\arity{S}}$ yields the same flattening~$P_j$, but as a member of~$P^\flat$.

    This exhibits the translation of $P_{<i}^\flat$ along the inclusion $\mvextend{\Sigma}{\arity{S_{<i}}} \to \mvextend{\Sigma}{\arity{S}}$ as a subfamily of~$P^\flat$.

  \item
    %
    Similarly, for all $k < j < i$, the judgement $P_k$ as a member of the flattening $P_{<j}^\flat$ translated along the signature inclusion $\mvextend{\Sigma}{\arity{S_{<j}}} \to \mvextend{\Sigma}{\arity{S_{<i}}}$ yields the same flattening~$P_k$, but as a member of~$P_{<i}^\flat$.

    This exhibits the translation of $P_{<j}^\flat$ along the inclusion $\mvextend{\Sigma}{\arity{S_{<j}}} \to \mvextend{\Sigma}{\arity{S_{<i}}}$ as a subfamily of $P_{<i}^\flat$.
  \end{enumerate}
\end{definition}

\begin{definition} \label{def:well-presented-premise-family}
  A well-founded premise-family $P$ as in \cref{def:well-founded-premise-family} is
  \emph{well-formed over $T$} if for each $i \in I$, there are derivations of all presuppositions of~$B_i$ from hypotheses $P_{<i}^\flat$, in the translation of $T$ to $\mvextend{\Sigma}{\arity{S_{<i}}}$.

  A \defemph{well-presented premise-family} is a well-formed well-founded premise-family~$P$.
  %
  Its arity $\arity{P}$ is the associated arity $\arity{S}$ of the underlying premises-shape~$S$.
\end{definition}

\noindent
%
When no confusion can occur, we will write the flattening of a well-founded premise-family~$P$ just as $P$, rather than $P^\flat$.

\begin{definition}
  \label{def:well-presented-rule-boundary}%
  %
  A \defemph{well-presented rule-boundary} $\RuleBoundary{P}{B}$ over $\Sigma$, $T$ consists of a well-presented premise-family~$P$ together with a boundary with empty context $\typesboundary B$ over $\Sigma + \arity{P}$, the \defemph{conclusion boundary}, such that all presuppositions of~$B$ derivable from~$P$ in the translation of~$T$ to $\Sigma + \arity{P}$.
  %
  The \defemph{arity} of such a rule-boundary is the arity~$\arity{P}$ of its premise-family.
\end{definition}

\begin{definition}
  \label{def:realisation-well-presented-rule}%
  The \defemph{realisation} of a well-presented rule-boundary $R = (\RuleBoundary{P}{B})$ as a raw rule is defined according to the form of $B$.
  %
  \begin{enumerate}
  \item
    %
    If $B$ is an object boundary of class~$c$, then given a symbol $\symS \in \Sigma$ of arity $\arity{P}$ and class~$c$, the \defemph{realisation $\plug{R}{\symS}$ of $R$ with $\symS$} has premises the flattening of $P$, and conclusion $\plug{B}{\genapp{S}}$.

  \item
    %
    If $B$ is an equality boundary, no extra input is required: the \defemph{realisation of $R$} has premises the flattening of $P$, and conclusion just $B$ viewed as an equality judgement.
  \end{enumerate}
\end{definition}


\subsection{Well-presented type theories}
\label{sec:well-presented-theories}

Finally, we reach well-foundedness for type theories.
%
Once again, a by now familiar pattern emerges.
%
It is fairly straightforward to define well-foundedness as an after-market property of acceptable type theories,
%
but a better definition is obtained by putting in a little more work.

We start with the simpler version.

\begin{definition}
  \label{def:well-founded-theory}%
  Let $T = \fammap{R_i}{i \in I}$ be an acceptable type theory over a signature $\Sigma$, and let $\beta : |\Sigma| \to I$ the bijection from symbols to their rules.
  %
  Then $T$ is \defemph{well-founded} when all its rules are well-founded, and the index set $I$ has a well-founded order~$<$, such that:
  %
  \begin{enumerate}
  \item If $\symS \in \Sigma$ appears in $R_i$ then $\beta(\symS) < i$.
  \item Each $R_j$ has derivations of presuppositions that only refer to symbols $\symS$ with $\beta(\symS) < j$ and rules~$R_i$ with $i < j$.
  \end{enumerate}
\end{definition}

For the more refined version, we follow a similar pattern to what we saw for well-presented rules in \cref{sec:well-presented-rules}, with the definition stratified into three stages:

\begin{enumerate}
  \item First, the \emph{shape}: a well-founded order (to index the rules), and the premises-shapes and judgement forms of all rules.
  %
  This suffices to compute the signature of the theory, and of its initial segments.
  
  \item Next, the \emph{raw} part: for each rule of the theory, a well-founded premise-family, and (raw) conclusion boundaries, of the shapes and forms specified in the first stage, and over the signature of the appropriate initial segment.
  %
  These suffice to compute the flattening of the theory as a raw type theory, and of its initial segments.

  \item Finally, the \emph{derivations} showing well-formedness of each rule over the preceding initial segment.
\end{enumerate}

Having previously given \cref{def:well-founded-premises-shape,def:well-founded-premise-family,def:well-presented-premise-family} in rather excruciating detail, we proceed here slightly more concisely, trusting the reader to be able to fill in the elided details along the lines spelled out in those definitions.

\begin{definition} \leavevmode
  \label{def:well-presented-type-theory}%
  \begin{enumerate}
  \item A \defemph{well-founded type theory shape $T$} consists of a well-founded order $(I,<)$, together with for each $i \in I$, a well-founded premises-shape $S_i$ and judgement form $\varphi_i$ (seen as the premises shape and conclusion form of the $i$th rule).

    From these, we can define the total signature $\Sigma_T$ of $T$: its symbols are just $\set{ \symS_i \in I \such \varphi_i \in \set{\Ty,\Tm}}$, with $\symS_i$ having arity $\arity{S_i}$ and class $\varphi_i$.
    %
    Similarly, we get signatures for initial segments $\Sigma_{T_{<i}}$, and signature maps between these, and from these to $\Sigma_T$.

  \item A \defemph{well-founded raw type theory $T$} consists of a well-founded type theory shape as above (which we also call $T$), together with for each $i \in I$, a well-founded premise-family $P_i$ of shape $S_i$ and a boundary $B_i$ of form $\varphi_i$ over the signature~$\Sigma_{T_{<i}}$.

    From these, we can define the flattening $T^\flat$ of $T$ as a raw type theory over $\Sigma_T$.
    %
    Its rules consist of the realisations of all rule-boundaries $\RuleBoundary{P_i}{B_i}$, using (when $i$ is of object form) the symbol $\symS_i$, together with the associated congruence rules of the object rules thus added.
    %
    Similarly, we obtain the flattening $T_{<i}^\flat$ of each initial segments of $T$, as a raw type theory over $\Sigma_{T_{<i}}$.

  \item A \defemph{well-presented type theory $T$} consists of a well-founded raw type theory $T$ as above that is additionally \defemph{well-formed}, in that it is equipped with, for each $i$, derivations exhibiting $\RuleBoundary{P_i}{B_i}$ as a well-formed rule-boundary over $T_{<i}^\flat$.
  \end{enumerate}
\end{definition}

\noindent As with well-presented rules, we will not notate flattening, when there is no ambiguity.

\begin{proposition}
  The flattening of a well-presented type theory $T$ is acceptable and well-founded.
\end{proposition}

\begin{proof}
  Well-foundedness, tightness, substitutivity, and congruousness are immediate by construction.
  %
  Presuppositivity is almost as direct, requiring just translation of well-formedness of rules from the signatures $\Sigma_{T_{<i}}$ and theories $T_{<i}$ up to the full signature $\Sigma$ and theory~$T$.
\end{proof}


\subsection{The well-founded replacement}
\label{sec:well-founded-replacement}

\Cref{def:theory-good-properties} of acceptable type theories allows cyclic references of three kinds: between types of a context, premises of a rule, or rules of a type theory.
%
We shall not concern ourselves with the former two, since type theories occurring in practice all avoid them by using sequential contexts and sequential rules from \cref{sec:sequential-contexts,sec:sequential-rules}. We address the latter one, to
vindicate our design choices from earlier sections, to demonstrate that our setup supports non-trivial meta-theoretic methods, and to give an interesting new construction that likely has further applications.

For the remainder of this section, all contexts, raw rules, and rule-boundaries are presumed to be sequential. Also, it will be convenient to speak of a raw theory~$T$ without explicitly displaying its underlying signature. When we need to refer to the signature, we do so by writing $\Sigma_T$.

In \cref{ex:type-in-type} an acceptable type theory was rectified to a well-founded one by the introduction of a new symbol and an equation. When one looks at other specific examples the same strategy works, possibly with the introduction of several symbols and equations.
%
In order to present a general method we first lay some category-theoretic groundwork. We save the adventure of spiralling into the depths of category theory for another day, and instead establish just enough structure to keep the syntactic constructions organized.

\begin{definition}
  \label{def:raw-syntax-map}%
  %
  A \defemph{raw syntax map} $f : \Sigma \to \Sigma'$ is given by a family of expressions
  $f_S \in \Expr{\class{S}}{\mvextend{\Sigma'}{\args{S}}}{\emptyscope}$, one for each
  $S \in \Sigma$.
  %
  Such a map acts on $e \in \Expr{c}{\Sigma}{\gamma}$ to give $\act{f} e \in \Expr{c}{\Sigma'}{\gamma}$ by
  %
  \begin{equation}
    \label{eq:raw-syntax-map}%
    \act{f} (\synvar{i}) \defeq \synvar{i}
    \qquad\text{and}\qquad
    \act{f} (S(e)) \defeq \act{(\act{f} \circ e)} f_S.
  \end{equation}
  %
  The \defemph{metavariable extension} $\mvextend{f}{\alpha} : \mvextend{\Sigma}{\alpha} \to \mvextend{\Sigma'}{\alpha}$ by arity~$\alpha$ is the raw syntax map defined by
  %
  \begin{equation*}
    (\mvextend{f}{\alpha})_S \defeq f_S
    \qquad\text{and}\qquad
    (\mvextend{f}{\alpha})_{\synmeta{i}} \defeq 
    \synmeta{i}(\fammap{\synvar{j}}{j \in \argbinder{\alpha}{i}})
  \end{equation*}
\end{definition}

In words, a raw syntax map $\Sigma \to \Sigma'$ interprets each symbol in~$\Sigma$ as a suitable compound expression over~$\Sigma'$, the interpretation extends compositionally to all expressions over~$\Sigma$, and metavariable extensions act on such a map by extending it trivially.

Let us unravel the second clause in~\eqref{eq:raw-syntax-map}, as it is a bit terse. Given a symbol~$S \in \Sigma$ and its arguments
%
$e \in
   \prod_{i \in \args S} 
   \Expr{\argclass{S}{i}}{\Sigma}{\sumscope{\gamma}{\argbinder{S}{i}}}
$,
the composition $\act{f} \circ e$ takes each $i \in \args{S}$ to
$\act{f} e_i \in 
   \Expr{\argclass{S}{i}}{\Sigma'}{\sumscope{\gamma}{\argbinder{S}{i}}}
$ -- it is an instantiation of arity $\args{S}$, which thus acts on~$f_S$ to yield an expression $\act{(\act{f} \circ e)} f_S \in \Expr{\class{S}}{\Sigma'}{\gamma}$, where we took into account that $\sumscope{\gamma}{\emptyscope} = \gamma$.

The action of a raw syntax map evidently extends from expressions to context, judgements,
and boundaries, and thanks to the metavariable extensions also to raw rules and rule-boundaries.

\begin{proposition}
  Signatures and raw syntax maps form a category:
  %
  \begin{itemize}

  \item
    The identity morphism $\idmap[\Sigma] : \Sigma \to \Sigma$ takes $S \in \Sigma$ to the generic application~$\genapp{S}$.

  \item
    The composition of $f : \Sigma \to \Sigma'$ and $g : \Sigma' \to \Sigma''$ is the
    map $g \circ f : \Sigma \to \Sigma''$ that interprets each $S \in \Sigma$ as $(g \circ f)_S \defeq \act{g} f_S$.
  \end{itemize}
  %
  Raw syntax map actions are functorial.
\end{proposition}

\begin{proof}
  A straightforward application of the basic properties of instantiations.
\end{proof}

Given raw theories $T$ and $T'$, a raw syntax map $f : \Sigma_T \to \Sigma'_T$ between the underlying signatures may be entirely unrelated to~$T$ and~$T'$. Requiring it to map derivable judgements in~$T$ to derivable judgements in~$T'$ helps, but ignores the fact that raw theories are families of raw rules, not derivable judgements. Here is a better definition.

\begin{definition}
  A \defemph{raw theory map} $f : T \to T'$ is a raw syntax map $f : \Sigma_T \to \Sigma'_T$ on the underlying signatures which maps each specific rule~$R$ of~$T$ to a derivation of $\act{f} R$ in~$T'$.
\end{definition}

\begin{proposition}
  Raw theories and raw theory maps form a category $\RawTh$.
  %
  A raw theory map $f : T \to T'$ acts functorially on a derivation $D$ of $\Gamma \typesjudgement J$ in~$T$ to give a derivation $\act{f} D$ of $\act{f} \Gamma \typesjudgement \act{f} J$ in $T'$.
\end{proposition}


\begin{proof}
  Let us first describe the action of $f$ on a derivation $D$ of $\Gamma \types J$.
  %
  We proceed by recursion on the structure of~$D$.

  Structural rules are mapped to the corresponding structural rules, e.g., if $D$ concludes with a variable rule $\isterm{\Gamma}{\synvar{i}}{\Gamma_i}$ then $\act{f} D$ concludes with the variable rule $\isterm{\act{f} \Gamma}{\synvar{i}}{\act{f} \Gamma_i}$, and similarly for other structural rules.

  Consider the case where $D$ ends with an instantiation $\act{I} R$ of a specific rule~$R$ of~$T$, whose premises are $\fammap{\Gamma_k \types J_k}{k \in K}$. Suppose~$f$ takes $R$ to the derivation $D_R$ of $\act{f} R$ in~$T'$.
  %
  First we recursively map each derivation $D_k$ of the $k$-th premise $\act{I} \Gamma_k \types \act{I} J_k$ to a derivation $\act{f} D_k$ of $\act{f} (\act{I} \Gamma_k \types \act{I} J_k)$ in~$T'$, which is the same as $\act{(\act{f} I)} \Gamma_k \types \act{(\act{f} I)} J_k$, because acting by~$I$ and then by~$f$ is the same as acting by the instantiation $\act{f} I$.
  %
  We then take $\act{f} D$ to be the derivation $\act{(\act{f} I)} R_D$ with the derivations $\act{f} D_k$ of its hypotheses grafted onto it, as in \cref{def:derivation-grafting}.

  It is straightforward to verify that the action on derivations so constructed satisfies functoriality, $\act{(g \circ f)} D = \act{g} (\act{f} D)$.

  The categorical structure of $\RawTh$ is inherited from the structure of raw syntax maps. Additionally, given composable raw theory maps $f$ and $g$, we let their composition $g \circ f$ take a specific rule $R$ to the derivation $\act{g} D_R$, where $D_R$ is the derivation of $\act{f} R$ provided by~$f$.
\end{proof}

It may happen that a raw theory map $f : T \to T'$ maps an uninhabited type to an inhabited one, or an underivable equality to a derivable one.
%
Let us make precise the sense in which such a map fails to be conservative,
%
and at the same time generalise inhabitation of types to general completion of boundaries in the presence of premises.

\begin{definition}
  Given a raw theory $T$,
  and an object rule-boundary $\RuleBoundary{P}{B}$ over~$\Sigma_T$ whose syntactic class is~$c_B$, say that $e \in \Expr{c_B}{\mvextend{\Sigma_T}{\arity{P}}}{\emptyscope}$ \defemph{realises} the rule-boundary when $\RuleBoundary{P}{\plug{B}{e}}$ is derivable in~$T$.
\end{definition}

\begin{example}
  Inhabitation of a closed type~$A$ corresponds to realisation of the rule-boundary
  $\RuleBoundary{\emptyfam}{\bdryhead : A}$.
  %
  We can also express more general inhabitation tasks, for instance
  $\RuleBoundary{(\istype{}{\symA})}{\bdryhead \type}$ asks for a construction of a type from a type parameter~$\symA$, and $\RuleBoundary{(\istype{}{\symA}); (\istype{[\synvar{0} \of A]}{\symB(\synvar{0})})}{\bdryhead \type}$ for a $\symPi$-like higher type constructor.
\end{example}


\begin{definition}
  A raw theory map $f : T \to T'$ is \defemph{conservative} when:
  %
  \begin{enumerate}
  \item $f$ reflects equations: if $T'$ derives the equational rule $\act{f} R$ then $T$ derives the equational rule~$R$, and

  \item $f$ reflects realisers: there is a map $r$ such that if $e$ realises the object rule-boundary $\act{f} (\RuleBoundary{P}{B})$ in~$T'$ then $r(R, e)$ realises $\RuleBoundary{P}{B}$ in~$T$.
  \end{enumerate}
\end{definition}

In the definition, we asked for a map $r$ to witness reflection of realisers in order to avoid spurious applications of the axiom of choice.

\begin{lemma}
  \label{lem:map-factor-conservative}
  Every raw theory map $f : T \to U$ factors through a
  conservative map $f^\dagger : T^\dagger \to U$
  %
  \begin{equation*}
    \xymatrix{
      {T} \ar[rr]^{f} \ar[rd] & & {U} \\
      & {T^\dagger} \ar[ur]_{f^\dagger} &
    }
  \end{equation*}
  %
  in a weakly universal fashion.
\end{lemma}

In the lemma, by weak universality we mean that whenever $f$ factors through a conservative map $g : V \to U$, there is a map $h : T^\dagger \to V$, not necessarily unique, such that the following diagram commutes:
%
\begin{equation}
  \label{eq:weakly-universal-ddagger}
  \xymatrix{
    {T} \ar[rr]^f \ar[rd] \ar@/_4ex/[rdd]_j & & {U} \\
    & {T^\dagger} \ar[ur]_{f^\dagger} \ar[d]^h & \\
    & {V} \ar@/_4ex/[ruu]_g &
  }
\end{equation}

\begin{proof}[Proof of \cref{lem:map-factor-conservative}]
  %
  We shall construct $T^\dagger$ by adjoining new symbols to~$T$, so that whenever $e$ realises $\act{f} R$ in~$U$, the corresponding new symbol realises~$R$ in~$T$. However, the new symbols generate new rule-boundaries, so the process needs to be iterated, and we have to adjoin equations as well. The construction of~$f^\dagger : T^\dagger \to U$ thus proceeds in an inductive fashion, as follows.

  Initially the signature $\Sigma_{T^\dagger}$ is just $\Sigma_T$, the specific rules of $T^\dagger$ are those of~$T$, and~$f^\dagger$ acts like~$f$.
  %
  We inductively extend $\Sigma_{T^\dagger}$ with new symbols, $T^\dagger$ with new specific rules, and~$f^\dagger$ with new values, as follows:
  %
  \begin{enumerate}

  \item
    %
    If $\RuleBoundary{P}{B}$ is an object rule-boundary in $T^\dagger$ of arity~$\alpha$ and $e$ realises $\RuleBoundary{\act{f^\dagger} P}{\act{f^\dagger} B}$ in $U$, then we extend $\Sigma_{T^\dagger}$ with a \defemph{new symbol} $\symb{c}_{(\RuleBoundary{P}{B},e)}$ of arity $\alpha$, and $T^\dagger$ with the associated symbol rule
    $
     \SequentialRule
       {P}
       {\plug{B}{\genapp{\symb{c}}_{(\RuleBoundary{P}{B},e)}}}
    $.
    %
    We extend $f^\dagger$ by letting it map~$\symb{c}_{(\RuleBoundary{P}{B},e)}$ to~$e$.

  \item
    %
    If $R$ is an equational rule in $T^\dagger$ such that $\act{f^\dagger} R$ is derivable in~$U$, we extend~$T^\dagger$ with~$R$ as a specific rule.
  \end{enumerate}
  %
  Because we assumed all contexts and premise-families to be sequential, the inductive definition is complete after countably many repetitions of the above process. Alternatively, $T^\dagger$ could be constructed as a suitable colimit.

  The map $f$ obviously factors as $f = f^\dagger \circ i$ where $i : T \to T^\dagger$ is induced by the inclusion $\Sigma_T \to \Sigma_{T^\dagger}$.

  The map $f^\dagger$ is conservative by construction. It obviously reflects equations, while an object rule-boundary~$R$
  in~$T^\dagger$, such that $e$ realises $\act{f^\dagger} R$ in~$U$, is realised by $\symb{c}_{(R, e)}$.

  It remains to be shown that the factorization is weakly universal. Consider a factorization $f = g \circ j$ through a conservative map $g : V \to U$, as in~\eqref{eq:weakly-universal-ddagger}. There exists a map~$r$, not necessarily unique, that witnesses conservativity of~$g$. The desired factor $h : T^\dagger \to V$ is defined inductively: it acts like~$j$ on symbols of~$\Sigma_T$, and takes $\symb{c}_{(R,e)}$ to $r(\act{h} R, e)$.
\end{proof}

\begin{corollary}
  \label{cor:wf-replacement}%
  A raw theory $T$ has a weakly universal conservative map $t : \wfreplace{T} \to T$
  where $\wfreplace{T}$ is well-founded.
\end{corollary}

\begin{proof}
  We take $\wfreplace{T} = \emptyfam^\dagger$ and $t = o^\dagger$,
  as in \cref{lem:map-factor-conservative},
  where $o : \emptyfam \to T$ is the unique map from the empty theory~$\emptyfam$.
  %
  Weak universality is immediate, and well-foundedness of $\emptyfam^\dagger$ is witnessed by the inductive nature of its construction.
\end{proof}

\begin{definition}
  \label{def:wf-replacement}%
  The map $t : \wfreplace{T} \to T$ from \cref{cor:wf-replacement} is called the \defemph{well-founded replacement of~$T$}.
\end{definition}

Here, finally is the main theorem of this section.

\begin{theorem}
  \label{thm:wf-replacement-equivalence}%
  If $T$ is acceptable, the map $t : \wfreplace{T} \to T$ has a section.
\end{theorem}

The theorem establishes a form of equivalence of $\wfreplace{T}$ and $T$, because
any conservative map $r : U \to V$ with a section $s : V \to U$ is an isomorphism up to judgmental equality. Indeed, $r \circ s = \idmap_V$ because~$s$ is a section of~$r$, while $s \circ r$ is identity up to judgemental equality: given any derivable rule $\SequentialRule{P}{A \type}$ in~$U$, the rule $\SequentialRule{\act{r} P}{\act{r}(\act{s}(\act{r} A)) \judgeq \act{r} A}$ is derivable in~$V$ by reflexivity, and hence $\SequentialRule{P}{\act{s}(\act{r} A) \judgeq A}$ in~$U$ by conservativity of~$r$. An analogous argument works for term judgements.

\begin{proof}[Proof of \cref{thm:wf-replacement-equivalence}]

  The theory~$\wfreplace{T}$ has symbols of the form $\symb{c}_{(\RuleBoundary{P}{B},e)}$, where~$B$ is a closed boundary, but in the proof we will have to deal with boundaries that refer to variables.

  For this purpose, define the \defemph{(variable-to-metavariable) promotion} of an expression $e \in \Expr{c}{\Sigma}{\gamma}$ to be the expression $\mvpromote{e} \in \Expr{c}{\mvextend{\Sigma}{\simplearity \gamma}}{\emptyscope}$, cf.\ \cref{def:simple-arity}, which is~$e$ with the variables replaced by metavariables,
  %
  \begin{equation*}
    \mvpromote{\synvar{i}} \defeq \synmeta{i},
    \qquad\text{and}\qquad
    \mvpromote{S(e)} \defeq S(\fammap{\mvpromote{e_i}}{i \in \args{S}}).
  \end{equation*}
  %
  The associated \defemph{demotion} is the instantiation $D_\gamma \in \Inst{\Sigma}{\gamma}{\simplearity \gamma}$ which takes the metavariables back to variables, $D_\gamma(\synmeta{i}) = \synvar{i}$. Thus we have $e = \act{{D_\gamma}}{\mvpromote{e}}$.

  One level up, given a premise-family~$P$ and a context $\Gamma$ over $\mvextend{\Sigma}{\arity{P}}$, the \defemph{promotion} of $\Gamma$ is the extension $P ; \mvpromote{\Gamma}$ of~$P$ in which the variables of~$\Gamma$ are promoted to metavariables of suitable types,
  %
  \begin{equation*}
    (P ; \mvpromote{\emptycxt}) \defeq P
    \qquad\text{and}\qquad
    (P ; \mvpromote{\ctxextend{\Gamma}{A}}) \defeq
    (P ; \mvpromote{\Gamma}) ; (\istype{}{\mvpromote{A}}).
  \end{equation*}

  We begin the construction of a section of~$t$ by defining a map $d$ which maps sequential rules and rule-boundaries from~$T$ to~$\wfreplace{T}$ by replacing compound expressions $e$ with suitable symbols~$\symb{c}_{(R,e)}$ from~$\wfreplace{T}$.
  %
  When acting on contexts and judgements, $d$ takes a sequential rule-family~$P$ from $T$ as an additional parameter, in which case we write~$d_P$.

  The map~$d$ recurses over a premise-family in~$T$ to give a premise-family in~$\wfreplace{T}$:
  %
  \begin{align*}
    d (\emptyfam) \defeq \emptyfam
    \qquad\text{and}\qquad
    d (P ; (\Gamma \types J)) \defeq (d(P) ; d_{P}(\Gamma \types J)).
  \end{align*}
  %
  Similarly, it takes a sequential context $\Gamma$ over $T + P$ to one over $\wfreplace{T} + d(P)$:
  %
  \begin{align*}
    d_P(\emptycxt) &\defeq \emptycxt, \\
    d_P(\ctxextend{\Gamma}{A}) &\defeq
    \ctxextend
      {d_P(\Gamma)}
      {(\act{{D_{\position{\Gamma}}}}
        \genapp{\symb{c}}_{((\RuleBoundary{d(P) ; \mvpromote{d_P(\Gamma)}}{\bdryhead \type}), \mvpromote{A})}
      )}.
  \end{align*}
  %
  In the second clause, $d_P$ recurses into~$\Gamma$ and extends it with the rather intimidating
  %
  \begin{equation*}
    \act{{D_{\position{\Gamma}}}}
        \genapp{\symb{c}}_{((\RuleBoundary{d(P) ; \mvpromote{d_P(\Gamma)}}{\bdryhead \type}), \mvpromote{A})},
  \end{equation*}
  %
  which is just the generic application of the $\symb{c}$-symbol for the promoted~$\mvpromote{A}$, demoted back to~$\Gamma$.
  Note that $(d(P) ; \mvpromote{d_P(\Gamma)}) = d(P ; \mvpromote{\Gamma})$ and hence $\act{t} (d_P(\ctxextend{\Gamma}{A})) = \ctxextend{\Gamma}{A}$.

  It remains to explain how~$d_P$ maps a judgement $\Gamma \types J$ over $T + P$ to one over $\wfreplace{T} + d(P)$. Here too we use the same method of demoting a generic application of a symbol for a promoted expression:
  %
  \begin{align*}
    %
    d_P (\istype{\Gamma}{A}) &\defeq (\istype{\Gamma'}{A'}),
    \\
    d_P (\isterm{\Gamma}{t}{A}) &\defeq (\isterm{\Gamma'}{t'}{A'}),
    \\
    d_P (\eqtype{\Gamma}{A}{B}) &\defeq (\eqtype{\Gamma'}{A'}{B'}),
    \\
    d_P (\eqterm{\Gamma}{s}{t}{A}) &\defeq (\eqterm{\Gamma'}{s'}{t'}{A'}),
    \\
    \shortintertext{where}
    %
    \Gamma' &\defeq d_P(\Gamma), \\
    A' &\defeq \act{{D_{\position{\Gamma}}}}
          \genapp{\symb{c}}_{((\RuleBoundary{d(P); \mvpromote{\Gamma'}}{\bdryhead \type}), \mvpromote{A})},
    \\
    B' &\defeq \act{{D_{\position{\Gamma}}}}
          \genapp{\symb{c}}_{((\RuleBoundary{d(P); \mvpromote{\Gamma'}}{\bdryhead \type}), \mvpromote{B})},
    \\
    t' &\defeq \act{{D_{\position{\Gamma}}}}
          \genapp{\symb{c}}_{((\RuleBoundary{d(P); \mvpromote{\Gamma'}}{\bdryhead : \mvpromote{A'}}), \mvpromote{t})},
    \\
    s' &\defeq \act{{D_{\position{\Gamma}}}}
          \genapp{\symb{c}}_{((\RuleBoundary{d(P); \mvpromote{\Gamma'}}{\bdryhead : \mvpromote{A'}}), \mvpromote{s})}.
  \end{align*}
  %
  By having $d$ map $\bdryhead$ to $\bdryhead$ the above clauses also provide the action of~$d$ on boundaries.
  %
  Finally, let $d$ map a sequential rule $\SequentialRule{P}{J}$ in~$T$ to the sequential rule $\SequentialRule{d(P)}{J'}$ where
  $d_P(\types J) = (\types J')$, and similarly for rule-boundaries.

  We have arranged~$d$ in such a way that $\act{t} (d(R)) = R$ for any sequential rule~$R$ over $T$.
  Moreover, if $R$ is derivable in~$T$, then $d(R)$ is derivable in~$\wfreplace{T}$, by an appeal to suitable symbol rules in~$\wfreplace{T}$. For instance, $d$ maps the rule $\SequentialRule{P}{A \type}$ to the rule $\SequentialRule{d(P)}{\genapp{\symb{c}}_{((\RuleBoundary{d(P)}{\bdryhead \type}), A)} \type}$. If the former is derivable in~$T$ then the latter is a symbol rule of~$\wfreplace{T}$.

  At last, let us define the section~$s$ of~$t$. Consider first a type symbol $S \in \Sigma_T$. Because~$T$ is acceptable, it has a unique symbol rule $\SequentialRule{P}{\genapp{S} \type}$. When we map it with $d$ we get
  %
  \begin{equation*}
    \SequentialRule{d(P)}{\genapp{\symb{c}}_{((\RuleBoundary{d(P)}{\bdryhead \type}), \genapp{S})} \type},
  \end{equation*}
  %
  which is a symbol rule in~$\wfreplace{T}$. We may therefore take $s_S \defeq \genapp{\symb{c}}_{((\RuleBoundary{d(P)}{\bdryhead \type}), \genapp{S})}$.

  A term symbol $S \in \Sigma_T$ is dealt with analogously. Its symbol rule takes the form $\SequentialRule{P}{\genapp{S} : A}$, which is mapped by $d$ to
  %
  \begin{equation*}
    \SequentialRule{d(P)}{\genapp{\symb{c}}_{((\RuleBoundary{d(P)}{\bdryhead : A'}), \genapp{S})} : A'},
  \end{equation*}
  %
  where $A' = \genapp{\symb{c}}_{((\RuleBoundary{d(P)}{\bdryhead \type}), A)}$,
  Again, this is a symbol rule in~$\wfreplace{T}$, so we may define $s_S \defeq \genapp{\symb{c}}_{((\RuleBoundary{d(P)}{\bdryhead : A'}), \genapp{S})}$.
\end{proof}

\begin{example}
  We revisit \cref{ex:type-in-type}, the type theory expressing type-in-type in a cyclic fashion as~$\isterm{}{\symb{u}}{\symb{El}(\symb{u})}$.
  %
  Earlier we pointed out that the theory can be made well-founded by using the defined type constant $\eqtype{}{\symb{U}}{\symb{El}(\symb{u})}$.
  %
  The well-founded replacement works much the same way, except that it is replete with many more defined symbols. The analogue of $\symb{U}$ appears already at the first stage of the construction. Indeed, the rule-boundary $\RuleBoundary{\emptyfam}{\bdryhead \type}$ is realised by $\symb{El}(\symb{u})$, hence the well-founded replacement contains the type constant $U = \symb{c}_{(\RuleBoundary{\emptyfam}{\bdryhead \type}, \symb{El}(\symb{u}))}$.
  %
  We also have the new symbols
  %
  \begin{equation*}
    \mathit{El} = \symb{c}_{(\RuleBoundary{(\isterm{}{\symb{a}}{U})}{\bdryhead \type}, \genapp{\symb{El}})}
    \qquad\text{and}\qquad
    \mathit{u} = \symb{c}_{(\RuleBoundary{}{\bdryhead : U}, \genapp{\symb{u}})},
  \end{equation*}
  %
  and these suffice to express the type equation $\eqtype{}{U}{\mathit{El}(\mathit{u})}$, which is a specific rule of the well-founded replacement because it is mapped to a valid equation in the original type theory.
\end{example}


%%% local Variables:
%%% mode: latex
%%% End:


%%% Local Variables:
%%% mode: latex
%%% End:

% !TEX root = ../lifeonbrane3.tex
%

We have described a holographic framework where quantum extremal surfaces and the island rule \reef{wonderA} can be examined in higher dimensions, \ie for gravity theories in $d\ge2$. In particular, the background is simple enough that the construction given in section \ref{sec:branegravity} is straightforward and purely analytic, in contrast to the numerical approach of \cite{Almheiri:2019psy}. In section \ref{face}, we were also able to describe the system from three different perspectives, analogous to the three descriptions of the two-dimensional system examined in \cite{Almheiri:2019hni}. In particular, we have the boundary perspective, where the system is described as a $d$-dimensional CFT coupled to a ($d-1$)-dimensional conformal defect;
the bulk gravity perspective, where ($d+1$)-dimensional gravity with a negative cosmological constant is coupled to a codimension-one brane; and the brane perspective, where the boundary CFT is coupled to an AdS$_d$ region which supports Einstein gravity and two copies of the same CFT, which are weakly coupled to each other. As we emphasized, this last perspective is an effective theory, as is made clear by the cut-off arising in this Randall-Sundrum braneworld scenario. As discussed and examined in some detail in section \ref{HEE}, this effective gravity theory lends itself to the appearance of quantum extremal islands in the brane perspective, although these have a conventional interpretation from the bulk gravity perspective, in terms of RT surfaces which cross the brane for certain of choices of the entangling geometry on the boundary.\\

\hd{Unconventional features:} Of course, the analysis presented in our paper is somewhat unusual in that we are finding quantum extremal islands but there are no black holes, no horizons and no Hawking radiation involved. Rather we simply considered the entanglement entropy of various entangling regions in the vacuum state of the boundary system. However, to favour the formation of these quantum extremal islands, and at the same time have the brane in the `Einstein gravity regime,' \ie $L/\leff\ll1$, we had to introduce somewhat unconventional couplings. That is, we considered a negative Newton's constant on the brane $\lamb<0$ and nonzero Gauss-Bonnet coupling $\lgb$ for a four-dimensional bulk. Both of these choices were enhancing the connected RT surfaces over the disconnected RT surfaces in calculating the holographic EE. Of course, an interesting question is the interpretation of these `exotic' bulk couplings in terms of data describing the boundary CFT (and the conformal defect). While we do not have a precise interpretation, some qualitative results can be stated.

As observed in section \ref{face}, using standard holographic techniques, one finds that the gravitational coupling in the DGP brane action \reef{newbran} affects the spectrum of defect operators in the boundary theory \cite{domino}. Now let us reiterate that there is no apriori reason not to consider $\lamb<0$. For example, integrating out quantum fields on the brane could produce either a positive or negative shift of Newton's constant. In particular, the shift can be negative for gauge fields or nonminimally coupled scalar fields, as was discussed in the context of EE in \cite{Larsen:1995ax,Kabat:1995eq} -- see also discussion is appendix \ref{bubble}. However, this scenario is not the one we are describing here. In particular, additional brane fields such as these would make significant contributions to the EE which are not accounted for in our calculations. Hence, implicitly, we simply assume that the gravitational coupling $1/\Gbr$ (either positive or negative) is induced by some unknown UV physics.

Introducing the Gauss-Bonnet term \reef{top2} does not modify the gravitational dynamics in the four-dimensional bulk, considered in section \ref{sec:examples}, and hence the correlators of the stress tensor are not modified in the dual three-dimensional boundary theory.\footnote{Of course, such modifications arise for holographic constructions in higher dimensions \cite{Buchel:2009sk}.} However, the topological coupling $\lgb$ affects the entanglement structure of the boundary CFT states. To see this, consider calculating the entanglement entropy holographically for two nearby regions in the boundary. The phase transition between connected and disconnected phase of the RT surfaces is sensitive to a Gauss-Bonnet term. For positive $\lgb$, the transition from disconnected to connected phase takes place earlier (and vice versa for negative $\lgb$). This means that with $\lgb>0$, the mutual information between these two regions remains of order $c_\mt{T}$ for larger separations, \eg \cite{Headrick:2010zt}. Note, however, that choosing positive $\lgb$ favours higher genus surfaces. A concern with this choice might be if higher genus extremal surfaces exist, they may produce unusual results. Finally, we note that the topological coupling appears directly in the expressions for the holographic EE, \eg see eq.~\reef{Sdisc}. Therefore to have an appreciable effect, we must choose this coupling to be of the order of the central charge of the boundary theory, \ie $\lgb\sim L^2/\Gbk\sim \cT$.

Let us add that in section \ref{sec:examples}, we focused on the example of $d=3$ with a four-dimensional bulk. In this case, the natural topological term to add to the bulk gravity is the Gauss-Bonnet term \reef{top2}. Of course, the scenario extends straightforwardly to any $d=2n-1$ for which there is a corresponding topological term which can be added to the bulk gravity action, \ie the Euler character for $2n$-dimensional manifolds, \eg see \cite{Hung:2011xb}. Similarly, for even boundary dimensions ($d=2n$), the analogous topological terms could be added to the brane action, where they would not modify the dynamics of gravity on the brane but they would modify the gravitational entropy associated with the boundary of the quantum extremal islands. 

In light of these unconventional features, a natural question therefore is whether we find quantum extremal islands in our analysis with both $\lamb =0= \lgb$. The answer is affirmative, however, one must reduce to the tension of the brane to reduce its backreaction and the extent of the additional geometry in the vicinity of the  brane's location. As a result, the connected RT surfaces will have a smaller (bulk) area contribution as they cross the brane. However, in this case, the curvature of the AdS geometry on the brane is also smaller, and hence the effective description of the brane theory in terms of Einstein gravity breaks down. That is, with $\leff\sim L$, the contributions of the higher curvature corrections in the induced action \reef{act3} are no longer suppressed relative to the Einstein term and these new interactions play an important role in the dynamics of gravity in the brane perspective. Furthermore, the cutoff of the corresponding CFT on the brane will be much lower. Alternatively, one could think about computing the EE in settings beyond the vacuum state that we studied here. In fact, in \cite{QEI}, we will explicitly show without additional Gauss-Bonnet or DGP couplings that quantum extremal islands appear for (nonextremal) eternal black holes in equilibrium with an external heat bath, \ie in a higher dimensional analog of the analysis in \cite{Almheiri:2019yqk}.

Let us conclude here by comparing our approach with the recent work \cite{Geng:2020qvw}, which appeared while the present paper was prepared for submission. The latter examines essentially the same model (with no DGP term) but concentrates on a very different regime. The authors of \cite{Geng:2020qvw} focused on the formation of islands for the case of a tensionless brane, where the brane gravity becomes very nonstandard, as explained above. Further, in the limit where the graviton becomes massless, \ie $\ell_\mt{eff}\to \infty$, they  observe that no islands form \cite{Geng:2020qvw}. On the other hand, the present work focuses the regime of large brane tension, where the theory on the brane can be well approximated by Einstein gravity (\ie the graviton mass and higher curvature interactions are negligible). We moreover show that by allowing either a topological term or a negative $\Gbr$, islands can appear even in the absence of horizons.\\ 

\hd{Resolving Puzzles:} Our construction clarifies certain conceptual puzzles that arose in early discussions of quantum extremal islands in a holographic framework, \eg for the two-dimensional gravity models introduced in \cite{Almheiri:2019hni} and studied in \cite{Almheiri:2019yqk, Chen:2019uhq}. For example in these models the Planck brane, which supports the JT gravity theory, appears at the boundary of the three-dimensional bulk spacetime. Hence one might have wondered if the brane degrees of freedom (including the JT gravity) are a part of the boundary theory or part of the bulk theory. In our construction, the Planck brane is in the middle of the spacetime geometry and so this question does not arise -- these degrees of freedom belong to the bulk. An important corrolary of this observation is that when a quantum extremal island appears on the brane, \eg see the lower panel in figure \ref{fig:RTPhases}, we are able to recover information about the island with data from the boundary CFT in the corresponding boundary subregion, by applying standard entanglement wedge reconstruction \cite{EW1,EW2,EW3,Jafferis:2015del,Dong:2016eik,Faulkner:2017vdd,Cotler:2017erl}. Of course, the latter would not apply if the brane degrees of freedom were a part of the boundary theory.

Further, our construction circumvents the question of whether RT surfaces are allowed to end on the Planck brane. Rather in our paper, the extremal surfaces just pass through the bulk and only end on the asymptotic boundary as usual. It is simply that in certain situations, the RT surfaces will pass through the brane, which of course, corresponds to the formation of a quantum extremal island.

Another `novel' feature of the two-dimensional JT gravity model of \cite{Almheiri:2019hni} was that the holographic entanglement entropy included an extra boundary term, \ie the gravitational entropy of the JT model, where the RT surface terminated on the Planck brane. That is, the holographic entanglement entropy was given by extremizing the sum of the bulk area of the RT surface and this additional boundary term. An analogous gravitational entropy term on the brane arises in our construction with a DGP brane -- see eq.~\reef{eq:sad}. In fact, our derivation in appendix \ref{generalE} suggests that if the brane supports intrinsic gravitational interactions then the corresponding Wald-Dong entropy on the brane is part of the holographic entanglement entropy formula, as shown in eq.~\reef{fish9}. Hence this general result agrees with the boundary term introduced in the two-dimensional JT gravity models, mentioned above. A shortcoming of the derivation in appendix \ref{generalE} is that the geometric configuration involved a high degree of symmetry, which precluded  finding the expected extrinsic curvature terms \cite{Dong:2013qoa}. Therefore it would be interesting to extend our construction there to more general configurations  along the lines of \cite{Lewkowycz:2013nqa,Dong:2016hjy}.

We want to emphasize the above discussion is distinct from finding in section \ref{sec:enzyme} that the leading contribution to the holographic EE where the RT surface crosses the brane matches the Wald-Dong entropy of the induced gravitational action on the brane\reef{act3}.\footnote{Recall that this analysis was general enough to see the extrinsic curvature contributions coming from the higher curvature interactions in eq.~\reef{act3}.} For example, the leading contribution is $\area(\sigma_\xR)/{4G_\mt{eff}}$, where $\sigma_\xR$ is the cross-section of the RT surface on the brane. As shown in eq.~\reef{eq:bazinga2}, the DGP term is one important contribution to this result, but the bulk area of the RT surface in the vicinty of the brane is also necessary. Of course, we still find the leading contributions reproduce the gravitational entropy of the induced gravity theory on the brane even without the DGP term, \ie with $1/\Gbr=0$. This must be closely related to the fact that the bulk Einstein equations combined with the Israel junction conditions are equivalent to the gravity equations of motion on the brane in the Randall-Sundrum scenario \cite{deHaro:2000wj}.

In passing we note here that $d=2$ is distinguished in the above discussion. In this case, the leading contribution corresponds to the Wald-Dong entropy for the the Polyakov-Liouville action \eqref{PolyAct2} and takes the form given in eq.~\reef{arc}. However, since it only depends on the curvature scalar which is constant across the AdS$_2$ geometry of the brane, this contribution takes the same value no matter where the RT surface  crosses the brane. This contrasts with the higher dimensional result $\area(\sigma_\xR)/{4G_\mt{eff}}$, which rapidly grows as the position of $\sigma_\xR$ moves to larger radii on the brane. That is, there is an enormous penalty against forming large quantum extremal islands for $d\ge3$. In contrast, no such penalty arises for $d=2$ facilitating the formation of islands, as discussed in detail in
\cite{Rozali:2019day}. Of course, if one adds JT gravity \reef{JTee} to the two-dimensional brane action, as in eq.~\reef{braneact2}, then the gravitational entropy on the brane includes $\(\Phi_0+
\Phi(x)\)/4\Gbr$, which will favour smaller quantum extremal islands because the dilaton profile grows with the radius on the brane \cite{Maldacena:2016upp}.

Of course, we can modify our higher dimensional construction to make it more analogous to the two-dimensional model introduced in \cite{Almheiri:2019hni} by taking a $\mathbb Z_2$ orbifold quotient across the brane. With this orbifold, the brane appears as the edge of the bulk geometry but clearly the association with the bulk degrees of freedom has not changed. The brane now only supports a a single copy of the boundary CFT and there are factors of 1/2 appearing in various expressions, \eg we make the following replacement in eq.~\reef{Newton2}: ${1}/{G_\mt{eff}}=L/((d-2)\Gbk)$. Similarly, the RT surfaces will now end on the orbifolded brane while satisfying the boundary condition,
\beq\label{ortho8}
 0  =  \tg_j{}^\nu\(g_{\mu\nu}\,\partial_{n}X^\mu
  + \frac{G_\bulk}{G_\brane}\,\inducedK_i \,\partial_\nu x^i\)\,,
\eeq
%\rcm{Vincent: please confirm}\vc{I'm not sure about the factor of 2: if there are really two identical bulks, then \eqref{ortho8} is just a special case of \eqref{ortho7} and the factor of 2 is correct; but, if there is only one bulk (as suggested by stripping off a factor of $2$ in \eqref{Newton2} to get ${1}/{G_\mt{eff}}=L/((d-2)\Gbk)$ mentioned above), then the $\partial_{n_L} X^\mu$ term in \eqref{ortho7} is just not present, so there should not be a factor of 2 in \eqref{ortho8}.}
which replaces eq.~\reef{ortho7}. Further, the conformal defect becomes a conformal boundary in the orbifolded theory, \ie the spatial geometry on which the CFT lives is now a ($d-1$)-dimensional hemisphere with the conformal boundary being the $S^{d-2}$ at the edge of the hemisphere. 

Other questions that may have arisen from the early discussions of quantum extremal islands which focussed on JT gravity might include the importance of having a low spacetime dimension, \ie $d=2$, or of the JT model itself. The early work of \cite{Penington:2019npb} considered black hole evaporation with Einstein gravity in higher dimensions, and the holographic model of \cite{Almheiri:2019hni} was extended to a holographic framework with $d=4$ in \cite{Almheiri:2019psy} using numerical calculations. Hence our paper reinforces these results by describing quantum extremal islands in a new setting, in particular, in higher dimensions and with Einstein gravity. Our construction is also simple enough that further investigations of the role of quantum extremal islands in higher dimensions are straightforward, \eg see \cite{QEI}. Let us add that JT gravity can be seen as the gravitational dual of the so-called SYK model \cite{Maldacena:2016hyu,Sachdev:1992fk,Sachdev:2010um,Ktalks}. This duality involves an ensemble average over the couplings in the boundary quantum mechanics and so one may expect that this averaging plays a role in the appearance of quantum extremal islands. However, it seems that this is not the case as our construction relies on the standard holographic rules of the AdS/CFT correspondence where there is no such averaging of the couplings in the boundary theory.

One other perplexing issue with the island rule \reef{rule1} is the appearance of the entanglement of the CFT degrees of freedom in the region $\CFTR$ on both sides of the equation \cite{Almheiri:2019hni}. As explained in \cite{Almheiri:2019yqk}, we should distinguish the ``full quantum description'' of, \eg the Hawking radiation in the presence of black holes on the left-hand side from the ``semiclassical description'' which includes the outgoing radiation and purifying partners on the quantum extremal island on the right-hand side. Our holographic construction makes clear that the description of quantum states with islands in the brane picture is on a different footing than that solely in terms of the boundary theory. In particular, referring to the three perspectives discussed in section \ref{face}, it is clear that the boundary perspective (with the boundary CFT coupled to a conformal defect) gives a complete description of quantum state.  By the standard rules of the AdS/CFT correspondence, the bulk perspective (where Einstein gravity with a negative cosmological constant is coupled to a codimension-one brane) gives an equivalent description.\footnote{In this paper, we modeled the CFT defect with a simple brane in the bulk. This bottom-up approach is neither sufficient, nor completely correct. For example, in the case of $\mathcal N=4$ SYM theory on $S^4$, the presence of an interface breaks at least half of the supersymmetry generators and the $R$ symmetry. In a complete description, this will result in a deformation of the bulk $S^5$. For top-down models, see \cite{Karch:2001cw,DeWolfe:2001pq, DHoker:2007hhe, DHoker:2008rje, Chiodaroli:2009yw, Chiodaroli:2011nr, Chiodaroli:2012vc}. }
However, the brane perspective has a different character. In particular, the description in terms of a CFT coupled to the dynamical AdS$_d$ region is only an effective one. Indeed, as emphasized in section \ref{face}, the Randall-Sundrum gravity is only valid down to  the short distance cutoff $\tilde\delta\sim L$, \ie see eqs.~\reef{ctoffplus} and \reef{ctoffminus}. Beyond this cutoff, gravity is no longer localized to the brane and the additional `Kaluza-Klein' modes of the graviton are strongly coupled to the brane and their contribution cannot be ignored. 

Further, this brane perspective also provides an effective description of the coupling to the defect CFT. That is, it only accounts for the couplings localized at the defect, which dominate at low energies, but ignores the subtle nonlocal couplings, which could be seen as coming through the bulk AdS geometry in the dual description. Of course, the quantum extremal islands in the effective description of the brane perspective are a clear example of this. These islands are a remnant of replica wormholes in the limit $n\to1$ \cite{Penington:2019kki,Hartman:2020swn}. However, in the replica trick construction of the corresponding Renyi entropies in the bath CFT, one can ask why the gravity on the different branes in the replica copies should connect with one another. However, these effective gravity theories are UV completed by a single theory of gravity in the bulk and so it is natural to consider geometries connecting the branes, \ie replica wormholes if the effective theory. Hence the connection of the brane and boundary through the bulk provides a simple explanation of these wormholes.  Given the simplicity of our construction, it may provide a useful framework in which to understand further subtleties in distinguishing the various expressions in the island rule.

As a final note here, we observe that the finite cutoff for the CFT on the brane has noticeable effects even for $d=2$, \eg see eq.~\reef{almost}. In contrast, the early discussions of \eg \cite{Almheiri:2019hni,Almheiri:2019psf, Almheiri:2019yqk, Chen:2019uhq, Penington:2019kki, Almheiri:2019qdq} assumed that one could use standard formulae for conformal transformations in the $d=2$ CFT in the gravitational region (\ie on the brane). It would be interesting to understand if the cutoff modifies any of this analysis in a significant way \cite{QEI}.\\

\hd{Brane geometry, Part I:} As described in section \ref{sec:branegravity}, we choose the brane tension to produce a negative cosmological constant in the gravity theory on the brane, in accord with eqs.~\reef{act3} and \reef{Newton2}. As a result, the $d$-dimensional geometry on the brane is AdS space. However, it is straightforward to consider the case where the brane tension takes its critical value, such that $1/\ell_\mt{eff}^2=0$, as is usually done in the Randall-Sundrum scenario \cite{Randall:1999ee,Randall:1999vf}. In this case, the analogous brane geometry is simply flat space, and the brane is easily embedded in the bulk AdS$_{d+1}$ geometry on a slice of constant radius (or constant $z$) in standard Poincar\'e coordinates. An interesting feature of this embedding is that the brane reaches the asymptotic AdS$_{d+1}$ boundary along the null boundaries of the flat space geometry (as well as a timelike and spacelike infinity) \eg see \cite{Karch:2001cw}. 
%\dn{I've added Karch-Randall as a ref, since they draw nice pictures (see figure 5 in ''locally localized gravity''. However, I don't think there is an explicit reference which discusses these constructions in detail. It seems to always have been common knowledge.} 
Hence we can naturally investigate quantum extremal surfaces and the island formula in flat space using the usual expressions for holographic entanglement entropy in this construction as long as we consider regions on null infinity. Notably this matches the approach pursued in \cite{Hartman:2020swn}, but contrasts with studies of \eg \cite{Gautason:2020tmk} which considered spacelike regions. It would, of course, be interesting to use this framework to study quantum extremal islands in the context of asymptotically flat braneworld black holes, \eg as described in \cite{Emparan:1999wa,Emparan:1999fd}. We should note however that there are undoubtedly subtleties with the proposed construction, \eg as the brane completely cuts out the asymptotic AdS$_{d+1}$ boundary (except for a single point) on constant time slices. 

Of course, one can also consider the case where the brane tension is chosen such that $1/\ell_\mt{eff}^2<0$. That is, the brane gravity theory would have a positive cosmological constant and the corresponding brane geometry becomes de Sitter space. In this case, one constructs a foliation of the bulk AdS$_{d+1}$ geometry in terms of $d$-dimensional de Sitter slices and the brane can be embedded along the slice with the appropriate curvature, \eg see \cite{Karch:2001cw}. In this case, the brane reaches the asymptotic AdS$_{d+1}$ boundary on the future and past timelike infinities of the de Sitter geometry. 
%\dn{Don't know of any reference, but again seems like common knowledge.} 
Hence, this construction provides a framework to use holographic entanglement entropy for investigating the island formula in de Sitter space as long as we consider regions on the timelike future of the latter geometry. Let us add that this would be similar to upcoming work of \cite{dSone,dStwo}, which studies related questions in the context of JT gravity with a positive cosmological constant 
\cite{Maldacena:2019cbz}. The de Sitter evolution of the Hartle-Hawking vacuum prepares a two-dimensional CFT state on circle and the entanglement entropy of various regions in the latter state are investigated, revealing new islands in the de Sitter geometry \cite{dSone,dStwo}.\\ 
%\dn{I'm not sure what the last sentence refers to.}\\

\hd{Brane geometry, Part II:}



The geometry of the setup presented in this paper might look unconventional. As seen from the brane perspective, we have the bath CFT on the asymptotic boundary with geometry $S^{d-1} \times \mathbb R$, and two copies of the same CFT on the brane with an AdS$_d$ geometry. These two geometries are joined by introducing a cutoff surface (with topology $S^{d-2} \times \mathbb R$) near the asymptotic boundary of the AdS$_d$ geometry and gluing it to the equator of the  $S^{d-1} \times \mathbb R$ geometry. In particular, the resulting geometry is not a manifold in the vicinity of the gluing region -- see the left panel of figure \ref{fig:no_mfld}. Of course, we can obtain a manifold by taking the $\mathbb Z_2$ quotient which identifies the two halves of the bath CFT, such that the theory is again defined on a manifold with topology $S^{d-1} \times \mathbb R$. However, we will ignore this simplification here. Rather, we want to comment on the theory before taking the $\mathbb Z_2$ quotient. 

\begin{figure}[t]
\centering
\includegraphics[scale = 0.7]{no_mfld}
\caption{Left: In the brane perspective, the bath CFT on the asymptotic boundary (blue) is connected to two copies of the effective CFT on the brane (green) but the resulting geometry is not a manifold. Right: For excitations below the effective CFT cutoff the system behaves as if it consists of two systems on a manifold which are weakly coupled in the gravitational region (green).}
\label{fig:no_mfld}
\end{figure}

First, we note that constructions where multiple CFTs are joined at a common defect are not rare. For example they appear in the study of boundary and interface CFTs (\eg see \cite{Chiodaroli:2012vc}), and sometimes seem to be required to remove anomalies \cite{Ooguri:2020sua}.

Second, we would like to argue that in the regime where the defect theory can be described by two copies of the boundary CFT coupled to Einstein gravity, we can approximately think of the full theory as two copies of the orbifolded theory (each living on a manifold), which are weakly coupled in the gravitational region -- see the right panel of figure \ref{fig:no_mfld}. This is particularly easy to see from the bulk perspective. For brevity we restrict ourselves to the discussion of graviton modes, but a similar story applies to all bulk fields. 

Let us begin by recalling that for $\veps \ll 1$, the spectrum of graviton fluctuations in the bulk is almost unchanged with respect to the modes in (two copies of) empty AdS space. Hence much of the corresponding physics should be very similar that of two copies of the the AdS$_{d+1}$, or to two copies of the dual CFT$_d$ on the boundaries of two independent AdS$_{d+1}$ geometries. Of course, one exception to the preceding is that upon gluing the two AdS$_{d+1}$ geometries together, a new set of very light graviton states localized in the vicinity of the brane \cite{Randall:1999vf,Randall:1999ee,Karch:2000ct,Karch:2001jb}, as discussed in section \ref{face}. For simplicity, we refer to the latter as the brane graviton modes, while we refer to the former as the standard normalizable modes.\footnote{These bulk modes are $\mathbb Z_2$ graded under reflection across the Planck brane, and the even modes survive the $\mathbb Z_2$ orbifold discussed above include the brane graviton states as well as half of the standard normalizable modes. However, this organization of the modes is not useful for the following discussion.}

On a fixed time slice, as shown in the right panel of figure \ref{fig:brane2}, the standard normalizable modes will describe stress energy excitations in the dual CFT on both the left and right halves of the asymptotic boundary. If we assume an approximate extrapolate dictionary \cite{Harlow:2011ke} for the brane theory as well, these normalizable modes will also describe analogous excitations for the effective CFT on the brane. However, there will be two sets of such excitations: those described by bulk excitations\footnote{We stress here that the localized excitations considered here do not correspond to individual energy eigenmodes, which were implicit in the previous paragraph. Rather they will consist of linear combinations of such eigenmodes evaluated on the fixed time slice being examined here. Of course, having superpositions of energy eigenmodes is what produces the complicated time evolution described below.} with support primarily in the right copy of the AdS$_{d+1}$ geometry, and those described by the analogous excitations primarily in the left AdS$_{d+1}$ geometry. Hence, the stress tensor on the brane can be decomposed into two pieces which correspond to subsectors of the brane theory, each of which is determined by bulk excitations which essentially live on one side of the brane. If these subsectors were truly superselection sectors (\eg as one might imagine arises in the limit $\veps\to0$), our brane theory would contain two independent copies of the boundary CFT  and each of these copies would only interact with the bath CFT on the corresponding half of the asymptotic boundary. That is, each of these systems would live on an independent manifold with topology $S^{d-1} \times \mathbb R$. 

However, this is not strictly correct and the two copies of the CFT on the brane are weakly coupled with $\veps\ll1$ but finite. In particular, localized stress energy excitations of the form considered above will not remain localized with time evolution. Rather they will eventually spread across the entire asymptotic boundary if time evolves for a sufficiently long time. For example, an excitation localized on the right asymptotic boundary will evolve to eventually produce excitations of the stress tensors on the left asymptotic boundary and on the brane as well. From the boundary perspective, excitations moving onto the brane correspond to excitations that are absorbed by the conformal defect (and remain there for a long time).

The spreading of the localized excitations can be seen to arise through two physical effects: First, the bulk excitations can tunnel between the two AdS$_{d+1}$ regions shown in figure \ref{fig:brane2}. Recall that (the radial part of) the linearized bulk equation of motion can be reduced to a Schroedinger equation with a double-well potential, where the height of the barrier is determined by the brane tension \cite{Karch:2000ct}. With $\veps\ll1$ but finite, the barrier height while large remains finite and there will be a finite probability for a bulk excitation on one side of the Planck brane to tunnel to the other. A second independent coupling comes because the stress tensors of the two copies of the CFT couple to the same gravity theory on the brane. From the bulk perspective, the nonlinear Einstein equation produces interactions between the brane graviton modes with excitations on either side of the brane. Hence bulk excitation excitations on one side can leak to the other side by scattering process involving the brane gravitons. However, we note that both effects become smaller as the brane tension approaches its critical value, \ie as $\veps$ approaches zero. Thus, to a good approximation, the brane theory can be treated at two copies of the boundary CFT which only interact weakly. \\


\hd{Entanglement wedge cross-sections:} Recent work \cite{Takayanagi:2017knl,Nguyen:2017yqw} has drawn attention to
the entanglement wedge cross-section, \ie for disconnected boundary regions, the codimension-two surfaces in the bulk which have minimal area and which split the entanglement wedge in two. In particular, there are a number of proposals relating these holographic surfaces to various entanglement measures: entanglement of purification \cite{Takayanagi:2017knl,Nguyen:2017yqw}, reflected entropy \cite{Dutta:2019gen},  odd entanglement entropy \cite{Tamaoka:2018ned,Kusuki:2019evw,Kusuki:2019rbk}, or entanglement negativity \cite{Kudler-Flam:2018qjo,Kusuki:2019zsp}. 

Turning to our model and examining figure \ref{fig:RTPhases}, we see that there are two such minimal surfaces in the connected phase, for which a quantum extremal island appears on the brane. These surfaces are simply disks of radius $P=P_0$ on either side of the brane, with area
\beq\label{reflw}
A=\frac{2\,L^{d-1}\, \Omega_{d-2}}{d-1} \ P_0^{d-1}\, {}_{2}F_1\left[ \frac{1}{2},\frac{d-1}{2},\frac{d+1}{2},-P_0^2 \right]\,,
\eeq
as can be seen from eq.~\reef{A_disc}. The fact that both disks have the same area results from the fact that the corresponding boundary regions are symmetric on either of the conformal defect -- see figure \ref{EEprob}. Of course, if one of the two caps comprising the boundary regions was smaller, the minimal area disk closer to this cap would provide the global minimum and hence become the entanglement wedge cross-section. It would be interesting to understand if the second minimal disk also plays an interesting role in characterizing the entanglement of the boundary state. In this vein, let us add that there are also two additional extremal disks which divide the entanglement wedge in two but their area is actually a local maximum. These disks again lie on either side of the brane but end on $\sigma_\xR$, the intersection of the RT surface with the brane. Again, it is natural to wonder if these surfaces have an interpretation in terms of the boundary entanglement. Let us note that similar surfaces appear in the following discussion.\\

\hd{RT Bubbles and Wormholes:} 

	In appendix \ref{bubble}, we consider a surprising class of RT surfaces with the topology of a sphere, \ie $S^{d-1}$ in the ($d+1$)-dimensional bulk. The appearance of these extremal `bubbles' is quite unusual as they are homologous to the entire boundary. Hence the standard RT prescription would assign an entropy to the ground state of the dual boundary system. Further, presence of a `zero mode' which allows the bubbles to be translated along the brane makes their interpretation even more puzzling. An essential feature for the appearance of the RT bubbles was that the gravitational coupling in the DGP term \reef{newbran} was negative, \ie $\lamb<0$. We also noted that the bubbles do not appear to be macroscopic objects in the brane theory. Rather, as shown in eq.~\reef{haiku2}, their size is always of order of the effective cutoff $\tilde\delta$.
	
Despite the unusual features of these RT bubbles, the discussion in appendix \ref{bubble} highlights a general feature of the quantum extremal islands in a simple way. In particular, as discussed below eq.~\reef{genbubble1}, there are two competing terms contributing to the generalized entropy of these surfaces: the bulk area which describes the entropy of the CFT fields on the brane enclosed by the bubble and the area of the boundary where they intersect the brane, which appears in the gravitational entropy of the DGP term. The bulk contribution naturally acts to contract the bubble but with $\lamb<0$, the brane contribution acts to expand the bubble. As described in the appendix, there is an equilibrium radius where these two effects balance one another. Of course, with $\lamb>0$, the brane contribution also acts to contract the boundary of the bubble and so no closed extremal surfaces appear, as expected.

As noted above, a similar competition is a general feature in the formation of quantum extremal islands. However, in this case as discussed in section \ref{sec:enzyme}, the bulk and brane contributions combine to produce a Bekenstein-Hawking term $\area(\sigma_\xR)/{4G_\mt{eff}}$ on the boundary of the island. This contribution, of course, imposes a large penalty to the formation of a large island and acts to contract the boundary towards a smaller (\ie vanishing) radius. For an island to appear, this contraction must be balanced by an expanding contribution. From the bulk perspective, this is simply coming from the remaining\footnote{We combined part of the bulk area into the Bekenstein-Hawking term above.} bulk area contribution of the RT surface, which we can ascribe to the quantum EE of the CFT state from the brane perspective. The point to be noted here is that for this to provide an expansion the RT surface must be anchored far from the island, \ie in the asymptotic (nongravitational) region associated with the boundary CFT. While perhaps self-evident, this discussion highlights the nonlocal nature of the physics producing the quantum extremal islands.

Let us add that the quantum extremal islands discussed here (as well as the RT bubbles) are remnants of replica wormholes in the limit $n\to1$. This follows from the fact that we are simply studying holographic EE with RT surfaces in a new bulk background, \ie with a back-reacted brane. Hence the analysis of \cite{Lewkowycz:2013nqa}\footnote{Following \cite{Dong:2016hjy,Faulkner:2017vdd}, the same applies for general time dependent situations.} introduces a smooth $n$-fold covering geometry for the corresponding Renyi entropies with positive integer indices. These covering geometries produce smooth wormhole geometries on brane analogous to those discussed in \cite{Almheiri:2019qdq,Penington:2019kki} for two dimensions. 

Now assuming replica symmetry, one can then take a $\mathbb Z_n$ orbifold quotient which leaves a single copy of the boundary geometry but the bulk solution now contains a codimension-two cosmic brane with tension $T_n=(n-1)/(4\Gbk\,n)$. In the presence of a DGP brane, we expect that there is an additional contribution where the two branes intersect, \ie the intersection surface carries an intrinsic tension $\widehat T_n=(n-1)/(4\Gbr\,n)$. In this setting, our discussion above for the formation of quantum extremal islands extends to the Renyi entropies in a relatively straightforward way. In particular, we expect that an area contribution associated with the boundary of the island now carries an effective tension $\tilde T_n=(n-1)/(4G_\mt{eff}\,n)$, which combines the intrinsic tension of this intersection surface and the contribution of the cosmic brane in the vicinity of the Planck brane. The contraction created by this term must be balance by the expansion provided by the remaining cosmic brane contributions. However, to provide an expansion the cosmic brane must be anchored by a twist operator in the asymptotic (nongravitational) boundary. Again, this highlights the nonlocal nature of the physics which implicitly supports the replica wormholes.

Of course, these dynamical considerations are emergent in the topological models considered in \cite{Marolf:2020xie,Penington:2019kki}. Hence it would be interesting to understand the implications of this dynamics to extend the new discussions of baby universes and ensembles to higher dimensions.\\



To conclude, let us comment that we will build on the holographic model constructed here to study the Page curve and the appearance of quantum extremal islands for higher dimensional black holes in \cite{QEI}. In particular, we study eternal black holes coming to equilibrium with an external heat bath (prepared at the same temperature) in a higher dimensional analog of the analysis appearing in \cite{Almheiri:2019yqk}. Let us reiterate that unconventional features (\ie Gauss-Bonnet and DGP couplings) introduced to favour quantum extremal islands here are unimportant in the discussion of higher dimensional black holes.\\





%%% Local Variables:
%%% mode: latex
%%% TeX-master: "../lifeonbrane3"
%%% End:



\bibliographystyle{plainnat}
\bibliography{general-type-theories.bib}

\appendix

\section{Formalisation in Coq}
\label{sec:formalisation-coq}

We have partially formalised our work in the Coq proof assistant~\citep{coq} on top of the HoTT library~\citep{bauer17:_hott}. The formalisation is publicly available at~\citep{lumsdaine:_formal}, wherein further instructions are given on how to compile and use the formalisation.
%
The formalisation will continue to evolve in future; the description here refers to the version tagged as \texttt{\small arXiv}.

The formalisation broadly follows the structure of the paper.
%
\Cref{tab:dict-paper-coq} lists selected major definitions and theorems from the paper, along with the names of the corresponding items in the formalisation, if any.
%
Almost all material of \cref{sec:preliminaries,sec:raw-syntax,sec:typing} has been formalised, as has some but not all of \cref{sec:well-behavedness}.
%
Versions of the main definitions of \cref{sec:well-founded-type-theories} are also formalised, but at time of writing, their treatment in the formalisation has non-trivial differences from the definitions here; such items are marked with an asterisk.
 
\begin{table}[htbp]
  \centering
  \footnotesize
  \begin{tabular}{ll}
    \toprule
    Paper & Formalisation \\ \midrule
    Family
    (\cref{def:family})
    & \coqident{Auxiliary.Family.family}
    \\
%    Family map
%    (\cref{def:family-map})
%    & \coqident{Auxiliary.Family.map}
%    \\
%    Family map over
%    (\cref{def:family-map-over})
%    & \coqident{Auxiliary.Family.over}
%    \\
    Closure rule
    (\cref{def:closure-rule})
    & \coqident{Auxiliary.Closure.rule}
    \\
    Closure system
    (\cref{def:closure-system})
    & \coqident{Auxiliary.Closure.system}
    \\
    Derivation
    (\cref{def:closure-system-derivation})
    & \coqident{Auxiliary.Closure.derivation}
    \\
%    Derivation grafting
%    (\cref{lem:hypotheses-grafting})
%    & \coqident{Auxiliary.Closure.graft}
%    \\
%    Well-founded order
%    (\cref{def:well-founded-order})
%    & \coqident{Auxiliary.WellFounded.well\_founded\_order}
%    \\
    Scope system
    (\cref{def:scope-system})
    & \coqident{Syntax.ScopeSystem.system}
    \\
    De Bruijn scope system
    (\cref{ex:de-bruijn-scope-systems})
    & \coqident{Examples.ScopeSystemExamples.DeBruijn}
    \\
    Syntactic class
    (\cref{def:syntactic-class})
    & \coqident{Syntax.SyntacticClass.class}
    \\
    Arity
    (\cref{def:arity})
    & \coqident{Syntax.Arity.arity}
    \\
    Signature
    (\cref{def:signature})
    & \coqident{Syntax.Signature.signature}
    \\
    Signature map
    (\cref{def:signature-map})
    & \coqident{Syntax.Signature.map}
    \\
    Raw expressions
    (\cref{def:raw-syntax})
    & \coqident{Syntax.Expression.expression}
    \\
    % Renaming action
    % (\cref{renaming-action})
    % & \coqident{Syntax.Expression.rename}
    % \\
    % Signature map action
    % (\cref{def:signature-map-action})
    % & \coqident{Syntax.Expression.rename}
    % \\
%    Renaming functoriality
%    (\cref{prop:commutation-renaming-signature-map})
%    & \coqident{Syntax.Expressions.Renaming}
%    \\
%    Signature map functoriality
%    (\cref{prop:commutation-renaming-signature-map})
%    & \coqident{Syntax.Expressions.Signature\_Maps}
%    \\
    Raw substitution
    (\cref{def:raw-substitution})
    & \coqident{Syntax.Substitution.raw\_substitution}
    \\
    Metavariable extension
    (\cref{def:metavariable-extensions})
    & \coqident{Syntax.Metavariable.extend}
    \\
    Instantiation of syntax
    (\cref{def:instantiation})
    & \coqident{Syntax.Metavariable.instantiate\_expression}
    \\
    Raw context
    (\cref{def:raw-context})
    & \coqident{Typing.Context.raw\_context}
    \\
    Raw rule
    (\cref{def:raw-rule})
    & \coqident{Typing.RawRule.raw\_rule}
    \\
    Instantiation of derivations
    (\cref{cor:instantiation-of-derivations})
    & \coqident{Typing.RawTypeTheory.instantiate\_derivation}
    \\
    Associated closure system
    (\cref{def:associated-closure-system})
    & \coqident{Typing.RawRule.closure\_system}
    \\
    Structural rules
    (\cref{def:structural-rules})
    & \coqident{Typing.StructuralRule.structural\_rule}
    \\
    Congruence rule
    (\cref{def:congruence-rule})
    & \coqident{Typing.RawRule.raw\_congruence\_rule}
    \\
    Raw type theory
    (\cref{def:raw-type-theory})
    & \coqident{Typing.RawTypeTheory.raw\_type\_theory}
    \\
    Acceptable rule
    (\cref{def:acceptable-rule})
    & (not formalised)
    \\
    Acceptable type theory
    (\cref{def:theory-good-properties})
    & \coqident{Metatheorem.Acceptability.acceptable}
    \\
    Presuppositions theorem
    (\cref{thm:presuppositions})
    & \coqident{Metatheorem.Presuppositions.presupposition}
    \\
    Admissibility of renaming
    (\cref{lem:admissibility-renaming})
    & \coqident{Metatheorem.Elimination.rename\_derivation}
    \\
    Admissibility of substitution
    (\cref{lem:admissibility-substitution})
    & \coqident{Metatheorem.Elimination.substitute\_derivation}
    \\
    Admissibility of equality substitution
    (\cref{lem:admissibility-equality-substitution})
    & \coqident{Metatheorem.Elimination.substitute\_equal\_derivation}
    \\
    Elimination of substitution
    (\cref{thm:elimination-substitution})
    & \coqident{Metatheorem.Elimination.elimination}
    \\
    Uniqueness of typing
    (\cref{thm:tight-uniqueness-of-typing})
    & (not formalised)
    \\
    Inversion principle
    (\cref{thm:inversion-principle})
    & (not formalised)
    \\
    Sequential context
    (\cref{def:seq-cxt-by-rules})
    & \coqident{ContextVariants.wf\_context\_derivation}$^{(*)}$
    \\
    Sequential rule
    (\cref{def:sequential-rule})
    & (not formalised)
    \\
    Well-presented rule
    (\cref{def:realisation-well-presented-rule})
    & (not formalised)
    \\
    Well-presented type theory
    (\cref{def:well-presented-type-theory})
    & \coqident{Presented.TypeTheory.type\_theory}$^{(*)}$
    \\
    Well-founded replacement
    (\cref{thm:wf-replacement-equivalence})
    & (not formalised)
    \\ \bottomrule
  \end{tabular}
  \caption{The correspondence between the paper and the formalisation~\citep{lumsdaine:_formal}. Items marked with~$(*)$ differ non-trivially from their counterparts in the paper.}
  \label{tab:dict-paper-coq}
\end{table}

Throughout the paper we worked rigorously but informally, and without discussing which mathematical foundation might be sufficient to carry out the constructions and proofs.
%
On this topic we may consult the formalisation.

Our formalisation is built on top of a homotopy type theory library with an eye towards future formalisation of the categorical semantics of type theories, but is so far agnostic with respect to commitments such as the Univalence axiom or the Uniqueness of identity proofs. The only axiom that we use is function extensionality. In other words, the code can be read in plain Coq.

The formalisation confirms that our development is constructive, there are no uses of excluded middle or the axiom of choice.

It is a bit harder to tell how many universes we have used, because Coq relieves the user from explicit handling of universes. Two seem to be enough, one to serve as a base and another to work with families over the base. The base universe can be very small, say consisting of the decidable finite types, if we limit attention to finitary syntax only.

We rely in many places on the ability to perform inductive constructions and carry out proofs by induction, and so we require some meta-theoretic support for these. Of course, there is no shortage of induction in Coq, and even a fairly weak set theory will have the capability to construct the necessary inductive structures, whereas the higher-order logic of toposes would have to be extended with $W$-types. Alternatively, we could restrict to finitary syntax, contexts and rules throughout to allow Gödelization of 
syntax and reliance on induction supplied by arithmetic.

%%% Local Variables:
%%% mode: latex
%%% End:

% \section{Thirteen ways of looking at a sequential context} \label{app:sequential-context-variants}

TODO This probably wants to remain internal-only, not included in the final submission?

A well-formed context is often defined as something like: a list $A_1, \ldots, A_n$, where for each $i$, $\istype{A_1, \ldots, A_{i}}{A_{i+1}}$.

When formalising this, there are several ways one might make it fully precise.
%
The following are all intended to be reasonable ways of reading the above definition, in a typed metatheory.

The aim here is to look over all such options in excruciatingly fine detail, and convince ourselfs which ones are equivalent, and how trivially/generally.

Most assume our setting of (a) scoped syntax, and (b) using flat raw contexts in the derivability judgemnets.
%
Exceptions are noted.

TODO We could add some more below on how this is affected by (a) unscoped syntax, (b) if the context judgements were mutual with the other judgements, and (c) the “well-formed syntax only” judgement.

\newcommand{\seqcxtlabel}[1]{(\emph{#1}). \label{seqcxt:#1}}
\newcommand{\seqcxtref}[1]{\ref{seqcxt:#1} (\emph{#1})}
\newcommand{\seqrawlabel}[1]{(\emph{#1}). \label{seqraw:#1}}
\newcommand{\seqwflabel}[1]{(\emph{#1}). \label{seqwf:#1}}
\newcommand{\seqrawref}[1]{\ref{seqraw:#1} (\emph{#1})}
\newcommand{\seqwfref}[1]{(\ref{seqwf:#1}) (\emph{#1})}
\newcommand{\seqrawwfref}[2]{\ref{seqraw:#1}(\ref{seqwf:#2}) (\emph{#1}, \emph{#2})}
  
Various ways of defining sequential well-formed contexts from raw flat contexts (without an intermediate stage “sequential raw contexts”):
\begin{enumerate}
\item Seq w-f contexts as an inductive-recursive type, with recursive flattening function \seqcxtlabel{ind-rec}
\item Seq w-f contexts as an inductive-recursive family over $\N$, with recursive flattening \seqcxtlabel{ind-rec-fam}
\item Seq w-f contexts by induction on length, mutual with flattening \seqcxtlabel{ind-by-length}
\item Seq w-f contexts as a (proof-relevant) inductive predicate on raw flat contexts (by the traditional rules) \seqcxtlabel{ind-pred-on-raw}
\item Seq w-f contexts as a list of pairs of a flat context and a well-formed type over it, such that the context of each entry equals the extension of the previous entry’s context by its type \seqcxtlabel{list-pairs}
\item Seq w-f contexts as a vector of types scoped over the length of the vector, such that each type is derivable over the earlier part (in whole scope, i.e.\ no variable-occurrence restrictions) \seqcxtlabel{flat-list-derivable}
\end{enumerate} \saveitem

(To make sense of the “earlier part” in \seqcxtref{flat-list-derivable}, it needs to be read with  a slightly more general idea of flat contexts and derivability: the context is a \emph{partial} map from a scope to types in that scope, and the variable rule has a side condition of definedness.  Or, relatedly, it can be read as working in an unscoped syntax.  The comments below apply to either reading.)

Next, there are various two-stage approaches: a well-formed context is a sequential raw context that is sequentially well-formed, where these are defined as…

\begin{enumerate} \restoreitem
\item seq raw contexts: as inductive type \seqrawlabel{raw-ind-type}
\item seq raw contexts: as inductive family over $\N$ \seqrawlabel{raw-ind-fam}
\item seq raw contexts: defined by induction on length, mutually with flattening \seqrawlabel{raw-ind-length}
\item seq raw contexts: as list of pairs of a scope and type in that scope, such that each scope is the successor of the previous one \seqrawlabel{raw-list-pairs}
\item seq raw contexts: as raw flat contexts of some scope $[n]$, such that the $i$th type is a weakening from scope $[i]$.  \seqrawlabel{raw-list-wkn}
\item seq raw contexts: as raw flat contexts satisfying variable-occurence constraint \seqrawlabel{raw-var-occ}
\end{enumerate}
(Here all except \seqrawref{raw-var-occ} admit a direct definition of the initial segment preceding any entry; for \seqrawref{raw-var-occ} this is non-trivial.)

…followed by:
\begin{enumerate}
\item seq w-f as inductive predicate \seqwflabel{ind-pred}
\item seq w-f by induction on length \seqwflabel{ind-length}
\item seq w-f as non-inductive predicate: each entry is well-formed over preceding initial segment \seqwflabel{non-ind-pred}
\end{enumerate}

(Another axis we could vary, but don’t for now, is proof-(ir)relevance.  All predicates are un-squashed, “derivable” is shorthand for “with a given derivation”.)

So we have $6 + (6 \times 3) = 24$ variants. How equivalent are they?  By equivalent, we always mean “isomorphic, preserving the realisation map to raw flat contexts”.  We assume throughout that the signature is a set, and hence so are expressions, and contexts of any given scope.

\begin{itemize}
\item For direct definitions of sec w-f contexts: \seqcxtref{ind-rec}, \seqcxtref{ind-rec-fam}, \seqcxtref{ind-by-length}, \seqcxtref{ind-pred-on-raw}, and \seqcxtref{list-pairs} are fairly trivially equivalent just by general and fairly standard manipulations of suitable inductive/ind-rec types.

  This leaves \seqcxtref{flat-list-derivable}, which we’ll return to below.
  
\item For seq raw contexts: \seqrawref{raw-ind-type}, \seqrawref{raw-ind-fam}, \seqrawref{raw-ind-length}, \seqrawref{raw-list-pairs}, and \seqrawref{raw-list-wkn} are again fairly trivially equivalent by general facts about inductive types.

  \seqrawref{raw-var-occ} is equivalent to \seqrawref{raw-list-wkn}; but this relies more on specifics of expressions, showing that constraints on variable occurrence are equivalent to being in the image of weakening (which is intuitively clear for renaming along injective scope maps, and I think also true if stated right for renaming along any map).

\item For the seq well-formedness predicate: \seqwfref{ind-pred}, \seqwfref{ind-length}, and \seqwfref{non-ind-pred} are all fairly straightforwardly equivalent.

\item \seqcxtref{ind-by-length} is fairly trivially equivalent to \seqrawwfref{raw-ind-length}{ind-length}.  Along with the above, completes the equivalence of all versions of the definition except \seqcxtref{flat-list-derivable}.

\item \seqcxtref{flat-list-derivable} is \emph{not} necessarily equivalent to the others, for arbitrary raw type theories. But consider type theories satisfying: whenever $\istype{\Gamma}{A}$, then $\istype{\Gamma}{\Gamma_i}$ for each variable $\synvar{i}$ occurring in $A$.  Over such type theories, \seqcxtref{flat-list-derivable} is equivalent to \seqrawwfref{raw-var-occ}{non-ind-pred}.  In particular, any type theory satisfying suitable inversion principles should satisfy this property; we certainly expect any acceptable type theory should satisfy it.  TODO Maybe tight is enough?  It would be nice to prove this, or some metatheorem from which it can be read off!
\end{itemize}

In summary: almost all variants considered here are equivalent for all raw type theories, mostly by (composites/inverses of) fairly general constructions on inductive types.  The exception is \seqcxtref{flat-list-derivable}; this should be equivalent to the others over reasonable type theories, but the equivalence requires a fairly non-trivial metatheorem about derivability.




\end{document}

%%% Local Variables:
%%% mode: latex
%%% TeX-master: t
%%% End:
