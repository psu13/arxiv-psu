%%%%% Comments and todos

\newif\ifshow
%change to \showfalse to hide todos and placeholders
\showtrue
\usepackage{comment}
% Notes of things to do in the paper
\ifshow
  \newcommand{\todo}[1]{\textcolor{darkred}{#1}}
  \newcommand{\placeholder}[1]{\textcolor{purple}{#1}}
  \newcommand{\internal}[1]{\textcolor{darkgreen}{#1}}
  \newcommand{\obsolete}[1]{\textcolor{gray}{#1}}
\else
  \newcommand{\todo}[1]{}
  \newcommand{\placeholder}[1]{}
  \newcommand{\internal}[1]{}
  \newcommand{\obsolete}[1]{}
\fi

\ifshow
%Provisional text: a rough version of content, but definitely not intended as final.
  \newenvironment{longplaceholder}{\leavevmode\color{purple}}{}
  \newenvironment{longtodo}{\color{darkred}}{}
  \newenvironment{longinternal}{\color{dark green}}{}
  \newenvironment{longobsolete}{\color{gray}}{}
\else
  \excludecomment{longplaceholder}
  \excludecomment{longtodo}
  \excludecomment{longinternal}
  \excludecomment{longobsolete}
\fi

%%%% Stylistic stuff

\newcommand{\defemph}[1]{\emph{\textbf{#1}}}
\newcommand{\defeq}{\coloneqq}

%%%% Letter-like macros

\newcommand{\D}{\mathcal{D}}
\newcommand{\N}{\mathbb{N}}

%%% Miscellaneous

\newcommand{\powfin}[1]{\mathcal{P}_{\mathrm{fin}}(#1)} % finite powerset
\newcommand{\injto}{\hookrightarrow} % a canonical inclusion
\makeatletter
\newcommand{\idmap}[1][]{\@ifmtarg{#1}{\operatorname{id}}{\operatorname{id}_{#1}}} % identity map or morphism
\makeatother

\newcommand{\tuple}[2]{\langle #1 \rangle_{#2}} % a tuple, element of a product

\newcommand{\set}[1]{\{#1\}}
\newcommand{\such}{\mid}

\newcommand{\initialSegment}[2][]{{\downarrow}_{#1}#2} % the initial segment of a well-founded order

% Logical quantifiers
\newcommand{\all}[1]{\forall #1 \,.\,}
\newcommand{\some}[1]{\exists #1 \,.\,}

% A shorter implication
\newcommand{\lthen}{\Rightarrow}


%%%% Arities & signatures

\DeclareMathOperator{\args}{arg} % The arguments of a symbol / the indexing sets of the arity of a symbol
\newcommand{\argclass}[2]{\operatorname{cl}_{#1} #2} % The syntactic class of an argument
\newcommand{\argbinder}[2]{\operatorname{bind}_{#1} #2} % The binder of an argument
\DeclareMathOperator{\simplearity}{simp} % the simple arity for a given scope

\newcommand{\class}[1]{\operatorname{c}_{#1}} % the syntactic class of an entity
\newcommand{\arity}[1]{\upalpha_{#1}} % the arity of an entity

\newcommand{\mto}{\colon} % to be used in mappings of deBruijn indices to types

%%%% Sets of expressions, judgements, etc.

\makeatletter
\newcommand{\Expr}[3]{\operatorname{Expr}^{#1}_{#2}\@ifmtarg{#3}{}{(#3)}} % expressions of syntactic class #1 over signature #2 and scope #3
\makeatother
\newcommand{\ExprTy}[2]{\Expr{\Ty}{#1}{#2}} % type expressions over signature #1 and scope #2
\newcommand{\ExprTm}[2]{\Expr{\Tm}{#1}{#2}} % term expressions over signature #1 and scope #2

\newcommand{\Context}[1]{\operatorname{Cxt}{#1}} % contexts over signature #1
\newcommand{\Judg}[1]{\operatorname{Judg}{#1}} % judgements over signature #1
% \newcommand{\JudgCxt}[1]{\operatorname{Judg}_{\mathrm{cxt}}{#1}} % judgements over signature #1 % NOT USED
\newcommand{\RawRule}[1]{\operatorname{RawRule}{#1}} % raw rules over signature #1
\newcommand{\RawTT}[1]{\operatorname{RawTT}{#1}} % raw type theories over signature #1

\newcommand{\ctxextend}[2]{#1 \mathbin{.} #2}  % Context extension

\newcommand{\SequentialRule}[2]{#1 \Longrightarrow #2} % Sequential rule
\newcommand{\RuleBoundary}[2]{#1 \Longrightarrow #2} % Rule boundary

\newcommand{\wfreplace}[1]{#1^{\mathrm{wf}}} % the well-founded replacement

%%%% Metavariables and related things

\newcommand{\mvextend}[2]{#1 + #2}  % metavariable extension of signature #1 by arity #2
\newcommand{\Inst}[3]{\operatorname{Inst}_{{#1},{#2}}(#3)} % instantiations of arity #1 over signature #2

\newcommand{\mvpromote}[1]{\widetilde{#1}} % the expression with variables promoted to metavariables

%%%% Combinatorics of syntax, judgements…

% \newcommand{\Class}{\mathsf{Cls}} % the set of syntactic classes (used only once without explanation)
% \renewcommand{\Form}{\mathsf{Form}} % the set of judgement forms (not used)

\newcommand{\Ty}{\mathsf{ty}} % the syntactic class of types
\newcommand{\Tm}{\mathsf{tm}} % the syntactic class of terms
\newcommand{\TyEq}{\mathsf{tyeq}} % the type equality judgement
\newcommand{\TmEq}{\mathsf{tmeq}} % the term equality judgement

\DeclareMathOperator{\ob}{ob} % for referring to object judgements
\DeclareMathOperator{\eq}{eq} % for referring to equality judgements

\newcommand{\Presup}[1]{\operatorname{Presup}{#1}} % presuppositions of judgement #1

\newcommand{\congrule}[1]{{#1}_{\equiv}} % associated congruence rule

%%%%  Judgements + syntax of type theories

%% General judgement macros

% Examples:
%    \types \nat \type
%   n \of \nat \types S(n) : \nat
%
% Note the spacing around colons is a bit tricky to get nice.
% One typically wants it not too wide within binders and context declarations, especially when the types involved are simple, but not too tight when it’s the head position of a typing judgement.
% Currently, “\of” gives a colon with fairly tight spacing, well-suited to binders and simple context declarations; LaTeX’s standard “:” (i.e. \mathrel{:}) works well in head position of a typing judgement; and then for complex binders/contexts when \of is a bit too tight, one can adjust it by hand as e.g. “\of[2]”, “\of[3]”, etc.  (The default “\of” is “\of[1]”.)
\newcommand{\types}{\mathrel{\vdash}} % the turnstile of a judgement
\newcommand{\of}[1][1]{\mspace{#1 mu plus 1.0mu}\mathord{:}\mspace{#1 mu plus 1mu}} % tighter colon for contexts/binders
\newcommand{\type}{\;\mathsf{type}} % for the judgement “Γ |– A type”
\newcommand{\cxt}{\mathrel{\textsf{cxt}}} % for the judgement “|– Γ cxt”
\newcommand{\emptycxt}{[\,]} % the empty context (but in judgements, usually just omit)
\newcommand{\judgeq}{\equiv}

% primitive judgement forms
\newcommand{\typesjudgement}{\types}
\newcommand{\istype}[2]{#1 \typesjudgement #2 \type} % is-a-type judgement
\newcommand{\isterm}[3]{#1 \typesjudgement #2 : #3} % is-a-type judgement
\newcommand{\eqtype}[3]{#1 \typesjudgement #2 \judgeq #3} % equal-types judgement
\newcommand{\eqterm}[4]{#1 \typesjudgement #2 \judgeq #3 : #4} % equal-terms judgement

% primitive boundary forms
\newcommand{\typesboundary}{\types} 
\newcommand{\bdryhead}{\Box} % the placeholder for heads of object boundaries
\newcommand{\qjudgeq}{\judgeq^{\scriptscriptstyle ?}}
\newcommand{\istypebdry}[1]{#1 \typesboundary \bdryhead \type} % is-a-type judgement
\newcommand{\istermbdry}[2]{#1 \typesboundary \bdryhead : #2} % is-a-type judgement
\newcommand{\eqtypebdry}[3]{#1 \typesboundary #2 \qjudgeq #3} % equal-types judgement
\newcommand{\eqtermbdry}[4]{#1 \typesboundary #2 \qjudgeq #3 : #4} % equal-terms judgement

% derived judgement, boundary forms
\newcommand{\iscxt}[1]{\types #1 \cxt} % is-a-context judgement
\newcommand{\eqcxt}[2]{\types #1 \judgeq #2} % equal-contexts judgement

% completing a boundary to a judgement
\newcommand{\plug}[2]{#1[#2]}

%%%% Category theory

\newcommand{\cat}[1]{\mathcal{#1}} % generic category
\newcommand{\Set}{\mathrm{Set}} % the category of sets
\newcommand{\RawTh}{\mathrm{RawTh}} % the category of raw theories and raw theory maps
\newcommand{\Sig}{\mathrm{Sig}} % the category of signatures

\newcommand{\Scope}{\mathrm{Scope}} % the category of scopes

\newcommand{\cod}{\operatorname{cod}} % the codomain function/functor
\newcommand{\arr}{\rightarrow} % for arrow categories, as e.g. “\Set^\arr”

\newcommand{\act}[1]{#1_{*}} % covariant functorial action, e.g., "\act{f} J" for the action of f on judgement J
\newcommand{\tca}[1]{#1^{*}} % contravarian functorial action

\newcommand{\rename}[1]{\act{#1}} % rename expression #1 along function #2, for now just uses \act

%%%% Closure systems

\DeclareMathOperator{\clos}{cl} % the closure system associated to a raw rule or raw type theory
\newcommand{\closCxt}{\clos_{\mathrm{cxt}}} % the closure system associated to a raw rule or raw type theory

\newcommand{\closureSystem}[1]{\mathcal{#1}} % font for closure systems
\newcommand{\csS}{\closureSystem S} % generic closure system S
\newcommand{\csT}{\closureSystem T} % generic closure system T
\DeclareMathOperator{\premises}{prems} % the premises of rule #1
\DeclareMathOperator{\conclusion}{concl} % the conclusion of rule #1
\newcommand{\derivation}[3]{\operatorname{Der}_{#1}(#2,#3)} % derivations w.r.t. closure system #1 from hypotheses #2 to conclusion #3
\DeclareMathOperator{\hyp}{hyp} % the derivation of a hypothesis
\newcommand{\der}[2]{\operatorname{der}(#1,#2)} % the derivation from premises #2, using rule #1

\DeclareMathOperator{\ClosureRule}{Rule} % \ClosureRule X : the set of closure rules on X
\DeclareMathOperator{\ClosureSystem}{Clos} % \ClosureSystems X : the set of closure systems on X

\newcommand{\StructuralRules}{\operatorname{Struct}} % the structural rules over a signature
\newcommand{\SubstFreeStructuralRules}{\operatorname{Struct}^{-}} % the same, without the subst rules

%%%% Scope systems

\newcommand{\position}[1]{|#1|} % the positions of scope #1
\newcommand{\emptyscope}{\mathsf{0}} % the empty scope
\newcommand{\sumscope}[2]{#1 \oplus #2} % the sum of two scopes
\DeclareMathOperator{\inlscope}{inl} % left inclusion of a scope into a sum
\DeclareMathOperator{\inrscope}{inr} % right inclusion of a scope into a sum
\DeclareMathOperator{\topscope}{top} % the new variable of a context extension
\newcommand{\singletonscope}{\mathsf{1}}
\newcommand{\numscope}[1]{[{#1}]}
\newcommand{\numscopemap}[2]{w_{{#1},{#2}}}

%%%% Families 

\DeclareMathOperator{\Fam}{Fam} % Families of elements
\DeclareMathOperator{\famindex}{ind} % indexing set of a family
\newcommand{\famev}[1]{\operatorname{ev}_{#1}} % evaluation map of a family

\newcommand{\family}[1]{\langle #1 \rangle} % an explicit family
\newcommand{\famtuple}[2]{\family{#1 \mid #2}} % a family displayed as a tuple
\newcommand{\fammap}[2]{\family{#1}_{#2}} % a family map, given as a mapping rule
\newcommand{\singletonfamily}[1]{\family{#1}} % a family with a single member, indexed by the unit
\newcommand{\emptyfam}{\langle \, \rangle} %empty family

% Coproduct injections (compare to \inlscope and \inrscope)
\newcommand{\inl}{\iota_0}
\newcommand{\inr}{\iota_1}

%%%% Type-theoretic constructs

% Concrete symbols (sans-serif fonts, please)
\newcommand{\synvar}[1]{\mathsf{var}_{#1}} % the variable #1
\newcommand{\synmeta}[1]{\mathsf{meta}_{#1}} % the metavariable symbol #1

\newcommand{\symb}[1]{\mathsf{#1}} % generic symbol in sans-serif font (\symbol and \sym are taken)
\newcommand{\symPi}{\Uppi} % sans-serif Π
\newcommand{\symSigma}{\Upsigma} % sans-serif Sigma
\newcommand{\symlambda}{\uplambda} % sans-serif λ
\newcommand{\symapp}{\symb{app}} % application

\newcommand{\synlambda}[1]{\symLambda(#1)} % object-level lambda abstraction
\newcommand{\synPi}[1][]{\symPi(#1)} % object-level product

\newcommand{\symA}{\symb{A}}
\newcommand{\symB}{\symb{B}}
\newcommand{\symC}{\symb{C}}
\newcommand{\symS}{\symb{S}}

\newcommand{\syma}{\symb{a}}
\newcommand{\symf}{\symb{f}}

\newcommand{\genapp}[1]{\widehat{#1}} % generic applicationof a symbol

%%%% Meta-theorems

\newcommand{\natty}[2]{\tau_{#1}(#2)} % the natural type

%%%%% Meta-level (informal) notations

\newcommand{\prd}[1]{\symPi (#1) \,.\,} % meta-level (informal) dependent type, use as "\prd{x \of A} B" for "Π (x:A) B"

% % Not used:
%\newcommand{\lam}[1]{\uplambda #1 \,.\,} % meta-level (informal) function abstraction, use it as "\lam{x \of A} e" for "λ x : A . e"

%%%% Coq identifier links

% see https://tex.stackexchange.com/a/35314/ for partial explanation of the following:
% \makeatletter
% \newcommand{\coqdocbaseurl}{https://github.com/peterlefanulumsdaine/general-type-theories/blob/master/}
% \newcommand{\urlhash}{\#}
% \newcommand{\coqdocurl}[2]{\@ifmtarg{#1}{\coqdocbaseurl #2}{\coqdocbaseurl #2\urlhash L#1}}
% \newcommand{\coqidentdisplay}[1]{\nolinkurl{#1}} % to configure font/style: see documentation for packages ‘hyperref’, ‘url’
% \newcommand{\coqident}{\begingroup\@makeother\#\@coqident}
% \newcommand{\@coqident}[3][]{%    Arguments: #2 display text; #3 filename; #1 line number (optional)
%   \@ifmtarg{#3}%
%   {\coqidentdisplay{#2}}%
%   {\href{\coqdocurl{#1}{#3}}{\coqidentdisplay{#2}}}%
% \endgroup}
% \makeatother

\newcommand{\coqident}[1]{#1} % Display a Coq identifier

% optional argument allows link text to differ from link url

% Local Variables:
% mode: latex
% End: