\section{Thirteen ways of looking at a sequential context} \label{app:sequential-context-variants}

TODO This probably wants to remain internal-only, not included in the final submission?

A well-formed context is often defined as something like: a list $A_1, \ldots, A_n$, where for each $i$, $\istype{A_1, \ldots, A_{i}}{A_{i+1}}$.

When formalising this, there are several ways one might make it fully precise.
%
The following are all intended to be reasonable ways of reading the above definition, in a typed metatheory.

The aim here is to look over all such options in excruciatingly fine detail, and convince ourselfs which ones are equivalent, and how trivially/generally.

Most assume our setting of (a) scoped syntax, and (b) using flat raw contexts in the derivability judgemnets.
%
Exceptions are noted.

TODO We could add some more below on how this is affected by (a) unscoped syntax, (b) if the context judgements were mutual with the other judgements, and (c) the “well-formed syntax only” judgement.

\newcommand{\seqcxtlabel}[1]{(\emph{#1}). \label{seqcxt:#1}}
\newcommand{\seqcxtref}[1]{\ref{seqcxt:#1} (\emph{#1})}
\newcommand{\seqrawlabel}[1]{(\emph{#1}). \label{seqraw:#1}}
\newcommand{\seqwflabel}[1]{(\emph{#1}). \label{seqwf:#1}}
\newcommand{\seqrawref}[1]{\ref{seqraw:#1} (\emph{#1})}
\newcommand{\seqwfref}[1]{(\ref{seqwf:#1}) (\emph{#1})}
\newcommand{\seqrawwfref}[2]{\ref{seqraw:#1}(\ref{seqwf:#2}) (\emph{#1}, \emph{#2})}
  
Various ways of defining sequential well-formed contexts from raw flat contexts (without an intermediate stage “sequential raw contexts”):
\begin{enumerate}
\item Seq w-f contexts as an inductive-recursive type, with recursive flattening function \seqcxtlabel{ind-rec}
\item Seq w-f contexts as an inductive-recursive family over $\N$, with recursive flattening \seqcxtlabel{ind-rec-fam}
\item Seq w-f contexts by induction on length, mutual with flattening \seqcxtlabel{ind-by-length}
\item Seq w-f contexts as a (proof-relevant) inductive predicate on raw flat contexts (by the traditional rules) \seqcxtlabel{ind-pred-on-raw}
\item Seq w-f contexts as a list of pairs of a flat context and a well-formed type over it, such that the context of each entry equals the extension of the previous entry’s context by its type \seqcxtlabel{list-pairs}
\item Seq w-f contexts as a vector of types scoped over the length of the vector, such that each type is derivable over the earlier part (in whole scope, i.e.\ no variable-occurrence restrictions) \seqcxtlabel{flat-list-derivable}
\end{enumerate} \saveitem

(To make sense of the “earlier part” in \seqcxtref{flat-list-derivable}, it needs to be read with  a slightly more general idea of flat contexts and derivability: the context is a \emph{partial} map from a scope to types in that scope, and the variable rule has a side condition of definedness.  Or, relatedly, it can be read as working in an unscoped syntax.  The comments below apply to either reading.)

Next, there are various two-stage approaches: a well-formed context is a sequential raw context that is sequentially well-formed, where these are defined as…

\begin{enumerate} \restoreitem
\item seq raw contexts: as inductive type \seqrawlabel{raw-ind-type}
\item seq raw contexts: as inductive family over $\N$ \seqrawlabel{raw-ind-fam}
\item seq raw contexts: defined by induction on length, mutually with flattening \seqrawlabel{raw-ind-length}
\item seq raw contexts: as list of pairs of a scope and type in that scope, such that each scope is the successor of the previous one \seqrawlabel{raw-list-pairs}
\item seq raw contexts: as raw flat contexts of some scope $[n]$, such that the $i$th type is a weakening from scope $[i]$.  \seqrawlabel{raw-list-wkn}
\item seq raw contexts: as raw flat contexts satisfying variable-occurence constraint \seqrawlabel{raw-var-occ}
\end{enumerate}
(Here all except \seqrawref{raw-var-occ} admit a direct definition of the initial segment preceding any entry; for \seqrawref{raw-var-occ} this is non-trivial.)

…followed by:
\begin{enumerate}
\item seq w-f as inductive predicate \seqwflabel{ind-pred}
\item seq w-f by induction on length \seqwflabel{ind-length}
\item seq w-f as non-inductive predicate: each entry is well-formed over preceding initial segment \seqwflabel{non-ind-pred}
\end{enumerate}

(Another axis we could vary, but don’t for now, is proof-(ir)relevance.  All predicates are un-squashed, “derivable” is shorthand for “with a given derivation”.)

So we have $6 + (6 \times 3) = 24$ variants. How equivalent are they?  By equivalent, we always mean “isomorphic, preserving the realisation map to raw flat contexts”.  We assume throughout that the signature is a set, and hence so are expressions, and contexts of any given scope.

\begin{itemize}
\item For direct definitions of sec w-f contexts: \seqcxtref{ind-rec}, \seqcxtref{ind-rec-fam}, \seqcxtref{ind-by-length}, \seqcxtref{ind-pred-on-raw}, and \seqcxtref{list-pairs} are fairly trivially equivalent just by general and fairly standard manipulations of suitable inductive/ind-rec types.

  This leaves \seqcxtref{flat-list-derivable}, which we’ll return to below.
  
\item For seq raw contexts: \seqrawref{raw-ind-type}, \seqrawref{raw-ind-fam}, \seqrawref{raw-ind-length}, \seqrawref{raw-list-pairs}, and \seqrawref{raw-list-wkn} are again fairly trivially equivalent by general facts about inductive types.

  \seqrawref{raw-var-occ} is equivalent to \seqrawref{raw-list-wkn}; but this relies more on specifics of expressions, showing that constraints on variable occurrence are equivalent to being in the image of weakening (which is intuitively clear for renaming along injective scope maps, and I think also true if stated right for renaming along any map).

\item For the seq well-formedness predicate: \seqwfref{ind-pred}, \seqwfref{ind-length}, and \seqwfref{non-ind-pred} are all fairly straightforwardly equivalent.

\item \seqcxtref{ind-by-length} is fairly trivially equivalent to \seqrawwfref{raw-ind-length}{ind-length}.  Along with the above, completes the equivalence of all versions of the definition except \seqcxtref{flat-list-derivable}.

\item \seqcxtref{flat-list-derivable} is \emph{not} necessarily equivalent to the others, for arbitrary raw type theories. But consider type theories satisfying: whenever $\istype{\Gamma}{A}$, then $\istype{\Gamma}{\Gamma_i}$ for each variable $\synvar{i}$ occurring in $A$.  Over such type theories, \seqcxtref{flat-list-derivable} is equivalent to \seqrawwfref{raw-var-occ}{non-ind-pred}.  In particular, any type theory satisfying suitable inversion principles should satisfy this property; we certainly expect any acceptable type theory should satisfy it.  TODO Maybe tight is enough?  It would be nice to prove this, or some metatheorem from which it can be read off!
\end{itemize}

In summary: almost all variants considered here are equivalent for all raw type theories, mostly by (composites/inverses of) fairly general constructions on inductive types.  The exception is \seqcxtref{flat-list-derivable}; this should be equivalent to the others over reasonable type theories, but the equivalence requires a fairly non-trivial metatheorem about derivability.
