%!TEX root = ../diffcoh.tex

\section{Loop groups and intertwining of positive-energy representations}
\textit{by Sanath Devalapurkar}
\label{loop_groups}

We will give an introduction to the representation theory of loop groups of
compact Lie groups: we will discuss what positive energy representations
are, why they exist, how to construct them (via a Schur--Weyl style
construction and a Borel--Weil style construction), and how to show that they
don't depend on choices. Motivation will come from both mathematics and
quantum mechanics.

The theory of positive-energy representations of loop groups is modeled
on the representation theory of compact Lie groups. Some parts of the talk will make more sense if you are familiar
with the compact Lie group story, but this is not a requirement: in this section, we try to emphasize the ``big
picture'' over details, and we hope that this choice makes it readable for you. Likewise, we will not assume any
familiarity with loop groups or infinite-dimensional topology, nor will we dig into those details.

In \cref{ssec_loop_overview}, we state the main theorem (\cref{main-thm}) and discuss some motivation for caring
about representations of loop groups. In \cref{rep_loop}, we begin thinking about projective representations of
loop groups and the corresponding central extensions. In \cref{PS_proof_sketch}, we provide an extended proof
sketch of \cref{main-thm}, and discuss some connections to physics. Finally, in \cref{PS_diffcoh}, we discuss how
this relates to differential cohomology. There are two ways to lift the construction of central extensions of loop
groups to differential cohomology; one follows the Chern--Weil story we've used several times already in this part,
and the other more closely resembles the story we told about off-diagonal Deligne cohomology and the Virasoro
algebra in \cref{VirasoroAlgebra}.

%-------------------------------------------------------------------%
%-------------------------------------------------------------------%
%  Introduction													 %
%-------------------------------------------------------------------%
%-------------------------------------------------------------------%

\subsection{Overview}
\label{ssec_loop_overview}

The objective of this chapter is to explain the following theorem of Pressley--Segal \cite[Theorem 13.4.2]{loop}:
\begin{theorem}\label{main-thm}
	Let $G$ be a simply connected compact Lie group. Then any positive energy representation $E$ of the loop group
	$\LG$ admits a projective intertwining action of $\Diffplus(\Circ)$.
\end{theorem}
If this means nothing to you, that's okay: the goal of this talk is to explain all the components of this theorem
(\cref{rep_loop}) and sketch a proof (\cref{PS_proof_sketch}). Then, in \cref{PS_diffcoh}, we discuss how the
representation theory of loop groups is related to differential cohomology.


Here's a rough sketch of what \Cref{main-thm} is about. The
representation theory of a semisimple compact Lie group $G$ is very well-behaved: the Peter--Weyl
theorem \cite{PW27} \index[terminology]{Peter--Weyl theorem} allows one to provide any finite-dimensional
$G$-representation with a $G$-invariant Hermitian inner product, and this inner product decomposes the
representation into a direct sum of irreducibles. Moreover, the irreducibles are in bijection with dominant
weights,\index[terminology]{dominant weight} where by the Borel--Weil theorem (see \cite{Ser54}), the
representation associated to a dominant weight is given  as the global sections of a line bundle associated to a
homogeneous space of $G$ (a particular flag variety). \index[terminology]{Borel--Weil
theorem}\index[terminology]{homogeneous space}\index[terminology]{flag variety}

Most representations of loop groups will not satisfy analogues of this property,
so we'd like to hone down on the ones which do. These are the ``positive energy
representations''; these essentially satisfy properties necessary to be able to
write down highest/lowest weight vectors. \Cref{main-thm} then states
that positive energy representations are preserved under reparametrizations of
the circle (which give automorphisms of the loop group $\LG$). One can therefore
think of \Cref{main-thm} as a consistency result.

Before proceeding, I'd like to give some motivation for caring about the
representation theory of loop groups.
\begin{enumerate}[(1)]
	\item One motivation comes from the connection between representation theory and homotopy theory. The
	Atiyah--Segal completion theorem \cites[Theorem 7.2]{Ati61}[\S 4.8]{AH61}[Theorem
	2.1]{AS69}\index[terminology]{Atiyah--Segal completion theorem} relates representations of a compact Lie group
	$G$ to $G$-equivariant $\Kup$-theory, and likewise the representation theory of the loop group $\LG$ is related
	to (twisted) $G$-equivariant elliptic cohomology. This has been explored in \cite{Bry90, Dev96, Liu96, And00,
	And03, Gro07, Lur09a, Gan14, Lau16, Kit19, Rez20, BET21}.\index[terminology]{elliptic cohomology}
	\item Another motivation comes from the hope that geometry on the free loop space $\LM$ of a manifold $M$ is
	supposed to correspond to correspond to ``higher-dimensional geometry'' over $M$.\index[terminology]{loop
	space!in terms of higher-dimensional geometry} For instance, if $M$ has a Riemannian metric, one can
	think of the scalar curvature of $\LM$ at a loop as the integral of the Ricci curvature of $g$ over the loop.
	Similarly, spin structures on $M$ are closely related to orientations on $\LM$ \cites{Wit85}[\S
	3]{Ati85}{Wit88}[\S 2]{McL92}[Theorem 9]{ST05}[Corollary E, \S 1.2]{Wal16}, and string structures on $M$ are
	closely related to spin structures on $\LM$ \cites{Kil87}[Theorem 6.9]{NW13}.\footnote{There are a number of
	other works providing additional proofs of this fact or pointing out subtleties in the definitions,
	including \cites{PW88}{CP89}[\S 3]{McL92}{KY98}{ST05}{KM13a}{Wal15}{Cap16}{Wal16a}{Kri20}.}
	\index[terminology]{string structure}
\end{enumerate}
In light of this hope, it is rather pacifying to
have a strong analogy between representation theory of compact Lie groups and of loop groups. In fact, all of
these motivations are related by a story that still seems to be mysterious at the moment.

There's also motivation from physics for studying the representation theory of
loop groups. The wavefunction of a free particle on the circle $\Circ$ must be an
$\Lup^2$-function on $\Circ$ (because the probability of finding the particle
somewhere on the circle is $1$). There is an action of the loop group $\LU_1$
on $\Lup^2(\Circ;\CC)$ given by pointwise multiplication (a pair $\gamma \colon \Circ\to
\Uup_1$ and $f\in \Lup^2(\Circ;\CC)$ is sent to the $\Lup^2$-function $f_\gamma(z) =
\gamma(z) f(z)$). In particular, $\LU_1$ gives a lot of automorphisms of the
Hilbert space $\Lup^2(\Circ;\CC)$; this is relevant to quantum mechanics, where
observables are (Hermitian) operators on the Hilbert space of states. Having a
particularly (mathematically) natural source of symmetries is useful. In
\cite{segal-survey}, Segal in fact says: ``In fact it is not much of an
exaggeration to say that the mathematics of two-dimensional quantum field theory
is almost the same thing as the representation theory of loop groups''.


%-------------------------------------------------------------------%
%-------------------------------------------------------------------%
%  Representations of loop groups								   %
%-------------------------------------------------------------------%
%-------------------------------------------------------------------%

\subsection{Representations of loop groups}
\label{rep_loop}

\begin{definition}
	Let $G$ be a compact connected Lie group. The loop group $ \LG \colonequals \Cinf(\Circ, G)$ is the group of smooth unbased loops in $G$.
\end{definition}
If $G$ is positive-dimensional, $\LG$ is not finite-dimensional. A fair amount of the theory of finite-dimensional
manifolds generalizes to infinite-dimensional spaces locally modeled by nice classes of topological vector spaces,
and in this sense $\LG$ is an infinite-dimensional Lie group, in fact quite a nice one. Reading this chapter does
not require any additional familiarity with infinite-dimensional topology, but if you're interested, you can learn
more in \cites{Ham82}{Mil84}[\S 3.1]{loop}

There will be a lot of circles floating around, and so we will distinguish these
by subscripts. Some of these will be denoted by $\TT$, for ``torus''.

\begin{remark}[(Classification of compact Lie groups)]
We quickly review the classification of compact Lie groups. This may clarify the generality in which some of the
results in this section hold.
\begin{itemize}
	\item Let $G$ be a compact Lie group and $G_0\subset G$ denote the connected component containing the identity.
	Then there is a short exact sequence $1\to G_0\to G\to \pi_0(G)\to 1$.
	\item Let $G$ be a compact, connected Lie group. Then there is a short exact sequence $1\to F \to\tilde G\to
	G\to 1$, where $F$ is finite and $\tilde G$ is a product of a torus $\TT^n$ and a simply connected group.
	\item Let $G$ be a compact, connected, simply connected Lie group. Then $G$ is a product of \emph{simple}
	simply connected Lie groups.\index[terminology]{simple Lie group}
	\item Let $G$ be a compact, simply connected, simple Lie group. Then $G$ is isomorphic to one of
	$\SU_n$, $\mathrm{Spin}_n$, $\mathrm{Sp}_n$, $\mathrm G_2$, $\mathrm F_4$, $\mathrm E_6$,
	$\mathrm E_7$, or $\mathrm
	E_8$.\index[notation]{SUn@$\SU_n$}\index[notation]{Spinn@$\mathrm{Spin}_n$}
	\index[notation]{Spn@$\mathrm{Sp}_n$}\index[notation]{G2@$\mathrm G_2$} \index[notation]{F4@$\mathrm F_4$}
	\index[notation]{E6@$\mathrm E_6$}\index[notation]{E7@$\mathrm E_7$}\index[notation]{E8@$\mathrm E_8$}
\end{itemize}
Most of the results in this section require $G$ to be connected and simply connected; a few will also require $G$
to be simple. In particular, when $G$ is simple, $\H^4(\BG;\ZZ)\cong\ZZ$.\footnote{This isomorphism can be made
canonical by specifying that under the Chern--Weil map, the Killing form $B\colon\g\times\g\to\RR$ defines a
positive element of $\HdR^4(\BG)\cong\RR$.\index[terminology]{Killing form}}
\end{remark}
\begin{remark}
	The loop group $\LG$ is an infinite-dimensional Lie group, and it has an
	action of $\Circ$ by rotation. We will denote this ``rotation'' circle by
	$\TTrot$.  This action will turn out to be very useful shortly.\index[notation]{Trot@$\TTrot$}
\end{remark}

The action of $\TTrot$ allows one to consider the semidirect product $\LG \rtimes \TTrot$. The following proposition is then an exercise in manipulating
symbols:
\begin{prop}
	An action of $\LG \rtimes \TTrot$ on a vector space $V$ is the same data as
	an action $R$ of $\TTrot$ on $V$ and an action $U$ of $\LG$ on $V$
	satisfying
	$$R_\theta U_\gamma R_\theta^{-1} = U_{R_\theta \gamma}.$$
\end{prop}
Most interesting representations $U$ of $\LG$ on a vector space $V$ are not, strictly speaking, representations:
instead of $U_\gamma U_{\gamma'} = U_{\gamma\gamma'}$, they satisfy the weaker condition that
\begin{equation}
	U_\gamma U_{\gamma'} = c(\gamma, \gamma') U_{\gamma \gamma'},
\end{equation}
where $c(\gamma, \gamma')\in \CC^\times$. This is precisely:
\begin{definition}
	A \emph{projective representation} of $\LG$ on a Hilbert space $V$ is a
	continuous homomorphism $\LG \to \PU(V)$.\index[terminology]{projective representation}
\end{definition}

\begin{remark}
	Why Hilbert spaces? From a mathematical perspective, this is because Hilbert
	spaces are well-behaved infinite-dimensional vector spaces. From a physical
	perspective, this is because Hilbert spaces are spaces of states. In fact,
	this also explains why most interesting representations are projective: the
	state of a quantum system is not a vector in the Hilbert space, but rather a
	vector in the projectivization of the Hilbert space. This corresponds to the
	statement that shifting the wavefunction by a phase does not affect physical
	observations.
\end{remark}

Assume $V$ is an infinite-dimensional, separable Hilbert space. Then $\PU(V)$ is a $\Kup(\ZZ, 2)$, so
projective representations determine cohomology classes in $\H^2(\LG;\ZZ)$.
\begin{lem}
\label{csc_loop}
When $G$ is compact and simply connected, $\H^2(\LG;\ZZ)\cong\H^3(G;\ZZ)$.
\end{lem}
\begin{proof}
Since $G$ is simply connected, $\pi_1(G) = 0$, and $\pi_2$ vanishes for any Lie group. Therefore the Hurewicz
theorem identifies $\pi_3(G)$ and $\H_3(G;\ZZ)$. Let $\Omega G$ denote the based loop space of
$G$\index[notation]{OmegaG@$\Omega G$}, i.e.\ the subspace of $\LG$ consisting of loops beginning and ending at the
identity. Essentially by definition, there is an isomorphism $\pi_k(G)\to\pi_{k-1}(\Omega G)$ for $k > 1$, so we
learn $\pi_1(\Omega G) = 0$ and $\pi_2(\Omega G)\cong\pi_3(G)$.

To get to $\LG$, we use that as topological spaces, $\LG\cong G\times\Omega G$ \cite[\S 4.4]{loop}. Thus
$\pi_1(\LG) = 0$ and $\pi_2(\LG) \cong \pi_3(G)$, and the Hurewicz and universal coefficient theorems allow us to
conclude.\index[terminology]{Hurewicz theorem}\index[terminology]{universal coefficient theorem}
\end{proof}
Another way to construct this isomorphism is as follows: there is an evaluation map $\ev\colon
\Circ\times\LG\to G$ sending $(x, \ell)\mapsto\ell(x)$; then the isomorphism in~\cref{csc_loop} is: pull back by
$\ev$, then integrate in the $\Circ$ direction.

It turns out that when $G$ is compact and simply connected, every class in $\H^2(\LG;\ZZ)$ arises from a projective
representation as above \cite[Theorem 4.4.1]{loop}. There is a central extension\footnote{This central extension is
also a fiber bundle, and by Kuiper's theorem \cite{Kui65}, the total space $\Uup(V)$ is contractible (see
also \cites[Lemme 3]{DD63}[Proposition A2.1]{AS04}).\index[terminology]{Kuiper's theorem} This fiber bundle is
homotopy equivalent to two other interesting fiber bundles: the universal principal $\Uup_1$-bundle
$\Uup_1\to\mathrm{EU}_1\to\mathrm{BU}_1$, and the loop space-path space bundle $\Omega \Kup(\ZZ, 2)\to P
\Kup(\ZZ, 2)\to \Kup(\ZZ, 2)$.}
\begin{equation}
\label{PUcentral_ext}
	1\to \TTce\to \Uup(V)\to \PU(V)\to 1,
\end{equation}
and so any projective representation $\rho$ of $\LG$ determines a central
extension by pulling \eqref{PUcentral_ext} back:
\begin{equation}
	1\to \TTce\to \LGtilde_\rho \to \LG \to 1.
\end{equation}
Conversely, any central extension of $\LG$ gives rise to a projective
representation of $\LG$. In particular:

%\begin{remark}\label{central}
%	Isomorphism classes of central extensions of $\LG$ are in bijection with
%	elements of $\H^2(\LG;\ZZ)$. If $G$ is simple and simply connected, then
%	$\H^2(\LG; \ZZ) \cong \H^3(G; \ZZ) \cong \ZZ$.
%\end{remark}

\begin{definition}
	Let $G$ be a simple and simply connected compact Lie group. The
	\emph{universal central extension} $\LGtilde$ of $\LG$ is the central
	extension corresponding to the generator of $\H^2(\LG; \ZZ) \cong \ZZ$.\index[terminology]{universal central
	extension}
\end{definition}
We first met universal central extensions in a different context, in \cref{diffcoh_virasoro}.


The following result is key.
\begin{theorem}[{\cite[Theorem 4.4.1]{loop}}]
Let $G$ be simply connected. Then there is a unique action of $\Diffplus(\TTrot)$ on $\LGtilde$ which covers
	the action on $\LG$. Moreover, $\LGtilde$ deserves to be called
	``universal'', because there is a unique map of extensions from $\LGtilde$ to
	any other central extension of $\LG$.
\end{theorem}

\begin{remark}
	As a consequence, the action of $\TTrot$ on $\LG$ lifts canonically to
	$\LGtilde$. Every projective unitary representation of $\LG$ with an
	intertwining action of $\TTrot$ is equivalently a unitary representation
	of $\LGtilde\rtimes \TTrot$. For the remainder of this talk, we will assume $G$ is simply connected and
	abusively say write ``representation of $\LG$'' to mean a representation of $\LGtilde \rtimes \TTrot$.
\end{remark}

\begin{notation}
	It is a little inconvenient to constantly keep writing $\LGtilde\rtimes
	\TTrot$, so we will henceforth denote it by $\LGtildeplus$. The subgroup
	$\TTrot$ of $\LGtildeplus$ is also known as the ``energy circle'' (for
	reasons to be explained below).
	\index[terminology]{energy circle}
	\index[notation]{LGplus@$\LGtildeplus$}
\end{notation}

One of the nice properties of tori is that their representations take on a
particularly simple form, thanks to the magic of Fourier series. The action of
$\Circ$ on a finite-dimensional vector space is the same data as a $\ZZ$-grading.
The case of topological vector spaces is slightly more subtle: if $\Circ$ acts on
a topological vector space $V$, then one can consider the closed ``weight''
subspace $V_n$ of $V$ where the action of $\Circ$ is by the
character\footnote{Some conventions are different: the action might be by
$z\mapsto z^n$. We're following \cite{loop}.} $z\mapsto z^{-n}$. Then the direct
sum $\bigoplus_{n\in \ZZ} V_n$ is a dense subspace of $V$; it is known as the
subspace of \emph{finite energy} vectors in $V$. This is simply the usual weight
decomposition adapted to the topological setting.

\begin{definition}
	The action of $\Circ$ on a topological vector space $V$ is said to satisfy the
	\emph{positive energy condition} if the weight subspace $V_n = 0$ for
	$n<0$. Equivalently, the action of $\Circ$ is represented by $e^{-iA\theta}$,
	where $A$ is an operator with positive spectrum.
\end{definition}

\begin{remark}
	The motivation for this definition comes from quantum mechanics: the
	wavefunction of a free particle on a circle is $e^{inx}$ (up to
	normalization), and requiring that the energy (which is essentially the
	weight $n$) to be positive is mandated by physics.
\end{remark}

\begin{definition}
\label{pos_en}
	A representation of $\LG$ (which, recall, means a representation of
	$\LGtildeplus$) is said to satisfy the \emph{positive energy condition} if it
	satisfies the positive energy condition when viewed as a representation of
	the energy/central circle $\TTrot$.
	\index[terminology]{positive energy condition}
\end{definition}

\begin{remark}
	It doesn't make sense for a representation of $\LG$ to be positive energy if
	you take ``representation of $\LG$'' to mean a literal representation of
	$\LG$; one needs to interpret that phrase as meaning a representation of
	$\LGtildeplus$.
\end{remark}

We can now see the utility of \Cref{main-thm}: the positive energy
condition involves the canonical parametrization of the circle, and to ensure that our definition would agree with
that of an alien civilization's, we should ensure that the pullback $f^\ast V$ of any positive energy
representation $V$ of $\LG$ along an orientation-preserving diffeomorphism $f\in \Diffplus(\TTrot)$ is another
positive energy representation. That is precisely the content of \Cref{main-thm}.

At the beginning of this chapter, we said that positive energy representations of loop groups satisfy analogues of
many properties of representations of compact Lie groups. To make that statement precise, we need to introduce some
definitions that impose sanity conditions on the representations we want to study.

\begin{definition}
	Let $V$ be a representation of a topological group $G$ (possibly
	infinite-dimensional). Then $V$ is said to be:
	\begin{itemize}
	\item \emph{irreducible} if it has no closed $G$-invariant subspace;\index[terminology]{irreducible
	representation}
	\item \emph{smooth} if the following condition is satisfied: let
		$V_\mathrm{sm}$ denote the subspace of vectors $v\in V$ such that
		the orbit map $G\to V$ sending $g$ to $gv$ is continuous; then
		$V_\mathrm{sm}$ is dense in $V$.\index[terminology]{smooth representation}
	\end{itemize}
	Two $G$-representations $V$ and $W$ are \emph{essentially equivalent} if
	there is a continuous $G$-equivariant map $V\to W$ which is injective and
	has dense image.\index[terminology]{essential equivalence}
\end{definition}

\begin{warning}
	Essential equivalence is \emph{not} an equivalence relation!
\end{warning}

The representation theory of compact Lie groups is really nice: every finite-dimensional complex representation of
a compact Lie group $G$ is semisimple (i.e.\ it is a direct sum of irreducible
representations),\index[terminology]{semisimple!representation} and unitary, and extends to a representation of the
complexification $G_\CC$ of $G$.\footnote{A complexification of a real Lie group $G$ is a complex Lie group,
generally noncompact, whose Lie algebra is isomorphic to $\g\otimes\CC$. When $G$ is compact, $G_\CC$ is unique up
to isomorphism.}\index[terminology]{complexification} These properties have analogues for positive energy
representations of loop groups.

\begin{theorem}[{\cite[Theorem 9.3.1]{loop}}]
\label{like_cpt_Lie}
	Let $V$ be a smooth positive energy representation of $\LG$. Then up to
	essential equivalence:
	\begin{itemize}
	\item $V$ is completely reducible into a discrete direct sum of
		irreducible representations,
	\item $V$ is unitary,
	\item $V$ extends to a holomorphic projective representation of
		$\mathrm L(G_\CC)$, and
	\item $V$ admits a projective intertwining action of $\Diffplus(\Circ)$,
		where this $\Circ$ is the energy/rotation circle. (This is \Cref{main-thm}.)
	\end{itemize}
\end{theorem}

The proof of this result takes up the bulk of the second part of Pressley--Segal.

\begin{remark}\label{pos-energy}
	The group $G$ includes into $\LG$ as the subgroup of constant loops. Let $G$
	be simple and simply connected. If $T$ is a maximal torus of $G$, then one
	has $\TTrot \times T\times \TTce \subseteq \LGtildeplus$. Consequently, if
	$V$ is a representation of $\LGtildeplus$, then $V$ can be decomposed (up to
	essential equivalence) as a $\TTrot \times T\times
	\TTce$-representation:
	\begin{equation}
		V = \bigoplus_{(n,\lambda, h) \in \TTrot^\vee \times T^\vee\times
	\TTce^\vee} V_{(n,\lambda,h)}\period
	\end{equation}
	Here, $n$ is the energy of $V$; $\lambda$ is a weight of $V$ (regarded
	as a representation of $T$); and $h$ is a character of $\TTce$. The notation $(\text{--})^\vee \colonequals
	\Hom(\text{--}, \CC^\times)$ denotes the character dual:\index[terminology]{character dual} because
	$\TTrot\times T\times\TTce$ is a compact abelian group, its unitary representations are direct sums of
	one-dimensional representations. Therefore as a $\TTrot\times T\times\TTce$-representation, $V$ splits as a
	direct sum of one-dimensional representations, which are indexed by the character dual $(\TTrot\times
	T\times\TTce)^\vee = \TTrot^\vee \times T^\vee\times
	\TTce^\vee$.
	
	If $V$ is
	irreducible, then $\TTce$ must act by scalars by Schur's lemma, and so
	only one value of $h$ can occur; this is called the \emph{level} of $V$. It\index[terminology]{level}
	turns out that if $V$ is a smooth positive energy representation, then each
	weight space $V_{n,\lambda,h}$ is finite-dimensional. In fact, a
	representation of $\LG$ of level $h$ is the same as a representation of
	$\LGtilde_h\rtimes \TTrot$, where $\LGtilde_h$ is the central extension of
	$\LG$ corresponding to $h\in \ZZ \cong \H^2(\LG;\ZZ)$.
\end{remark}

\begin{remark}
	By \Cref{pos-energy}, an irreducible positive energy representation
	$V$ of $\LG$ is uniquely determined by the level $h$ and its lowest energy
	subspace $V_0$: the representation $V$ is generated as a
	$\LGtildeplus$-representation by $V_0$.
\end{remark}

\begin{remark}
	Since $G$ is simply connected, there are transgression isomorphisms
	\begin{equation*}
		\H^4(\BG;\ZZ)\to\H^3(G;\ZZ)\to\H^2(\LG;\ZZ) \comma
	\end{equation*}
	meaning we can understand the level as (up to homotopy)
	a map $\BG\to \Kup(\ZZ,4)$. This $\Kup(\ZZ,4)$ is closely tied to the
	twisting $\Kup(\ZZ,4)\to \BGL_1(\tmf)$ of $\tmf$ constructed in \cite[Theorem 1.1]{ABG10}: see \cite{And00,
	Gro07, BET21}.
\end{remark}

As a side note, we observe the following:

\begin{prop}
	Let $V$ be a smooth positive energy representation of $\LG$. Then $V$ is
	irreducible as a representation of $\LGtilde$.
\end{prop}

\begin{proof}
	Assume $V$ is not irreducible as a $\LGtilde$-representation. Projection onto
	a proper $\LGtilde$-invariant summand defines a bounded self-adjoint operator
	$T:V\to V$ which commutes with $\LGtilde$, but (by hypothesis) not with the action
	of $\TTrot$. Choose $R\in\TTrot$; then define for each $n\in \ZZ$ the bounded operator
	\begin{equation}
		T_n = \int_{\TTrot} z^n R_z T R_z^{-1}\, \d z\period
	\end{equation}
	$T_n$ commutes with the action of $\LGtilde$, and $T_n$ sends the weight space
	$V_m$ to $V_{m+n}$. Because $T$ does not commute with $\TTrot$, the
	operator $T_n$ must be nontrivial for at least one $n<0$. Suppose that $m$
	is the lowest energy of $V$ (i.e., the smallest $m$ such that the weight
	space $V_m \neq 0$).\footnote{Because $V$ is positive energy, $m\geq 0$ --- but
	that doesn't matter for now.} Then $T_n(V_m) = 0$ if $n<0$. Since $V$ is
	irreducible as a representation of $\LGtildeplus$, it is generated as a
	representation by $V_m$.  But then $T_n(V) = 0$ for all $n<0$. The adjoint
	to $T_n$ is $T_{-n}$, and so $T_n(V) = 0$ for all $n\neq 0$.

	This implies that $T$ commutes with the action of $\TTrot$, which is a
	contradiction: the $T_n$ are the Fourier coefficients of the loop
	$\Circ\to \End(V)$ sending $z$ to $R_z T R_z^{-1}$, so we find that this loop
	must be constant.  Consequently, $T$ must commute with the action of
	$\TTrot$, as desired. 
\end{proof}

%-------------------------------------------------------------------%
%-------------------------------------------------------------------%
%  A proof sketch of Theorem \ref{main-thm}						    %
%-------------------------------------------------------------------%
%-------------------------------------------------------------------%

\subsection{A proof sketch of \texorpdfstring{\Cref{main-thm}}{Theorem \ref*{main-thm}}}
\label{PS_proof_sketch}

The goal of this section is to go through the proof of \Cref{main-thm}.
As with all proofs in representation theory, we may first reduce to
the irreducible case, thanks to the first part of \cref{like_cpt_Lie}.
%Before arguing the general case, I want to make an
%observation.
\begin{observation}\label{observe}
	Recall that Schur--Weyl duality sets up a one-to-one correspondence between
	representations of $\SU_n$ and representations of the symmetric groups, by
	studying the decomposition of the tensor power $V^{\otimes d}$ of the
	standard representation $V$ under the action of $\Sigma_d$.\index[terminology]{Schur--Weyl duality}
\end{observation}
One may hope that some analogue of \Cref{observe} is true for
representations of loop groups: suppose we could construct a giant
representation of $\LSU_n$ whose $h$-fold tensor product contains all the
irreducible positive energy representations of level $h$, such that this big
representation admits an intertwining action of $\Diffplus(\Circ)$. Then (with a
little bit of work), we would obtain an intertwining action of $\Diffplus(\Circ)$ on
all irreducible positive representations of $\LSU_n$, which would prove \Cref{main-thm} in this particular case. We would like to then reduce from the
case of a general $G$ to the case of $\SU_n$. The Peter--Weyl theorem says that
a simply connected $G$ is a closed subgroup of $\SU_n$ for some $n$, suggesting
that a technique like this might work.\index[terminology]{Peter--Weyl theorem}

Pressley--Segal's approach is similar, but not the same.
\begin{itemize}
	\item Their base case consists not just of $\LSU_n$, but the loop groups of all simply connected,
	simply laced compact Lie groups.\footnote{Recall that $G$ is simply laced if all its nonzero
	roots have the same length; in other words, if the Dynkin diagram\index[terminology]{Dynkin diagram} of $G$
	does not have multiple edges (so the Dynkin diagram is of ADE type). The simple, simply connected, simply laced
	Lie groups are $\SU_n$ for all $n$, $\mathrm{Spin}_n$ for $n$ even, $\mathrm E_6$, $\mathrm E_7$, and
	$\mathrm E_8$.}\index[terminology]{simply laced} In \cite[Lemma 13.4.4]{loop}, they extend from simply laced
	groups to all simply connected Lie groups; the reason they cannot just use an embedding $j\colon
	G\hookrightarrow \SU_n$ is that, given a representation $V$ of $\LGtilde$, Pressley--Segal need not
	just the embedding $j$, but also the condition that there is an irreducible representation $V'$ of the bigger
	group with $V$ a summand in $j^*V'$.
	\item Now assume $G$ is simply connected and simply laced. Instead of constructing a huge tensor product,
	Pressley--Segal reduce to the case of level $1$ representations in a different way. Let $m_n\colon\LG\to\LG$ be
	the map precomposing a loop $\Circ\to G$ with the $n^{\mathrm{th}}$-power map $\Circ\to\Circ$.
	Then \cite[Proposition 9.3.9]{loop} every irreducible representation $V$ of $\LGtilde$ is contained in $m_h^*F$
	for some level $1$ representation $F$. This allows Pressley--Segal to carry the $\Diffplus(\Circ)$-action from
	$F$ to $V$.
	\item Finally, when $G$ is simply laced and $F$ is level $1$, Pressley--Segal construct the
	$\Diffplus(\Circ)$-action directly using the ``blip construction'' \cite[\S 13.2, \S 13.3]{loop}.
\end{itemize}
\begin{remark}
Pressley--Segal write that ``one hopes that a more satisfactory proof of \cref{main-thm} can be
found,'' \cite[p.\ 271]{loop}, so perhaps there's a proof out there that more closely resembles the
Schur--Weyl-style argument.
\end{remark}

%This isn't exactly the
%approach that's taken in \cite{loop}, but something very close to it is in fact
%what is done. I haven't had time to flesh out the details of the approach
%outlined above (nor have I found a reference which takes this
%approach\footnote{Probably because it doesn't work for some obvious reason.} ---
%except for maybe \cite{loop-rep}, but I've gotten very confused over whether
%certain results in that paper are stated for a general compact Lie group or only
%for $\SU_n$), but I think it might be interesting.

Now we will see how the story goes for $\LSU_n$.
\begin{construction}
\label{Fock_constr}
	Let $G = \SU_n$. Define $H \colonequals \Lup^2(\Circ, V)$, where $V$ is the standard
	representation.\index[terminology]{standard representation} Let $\Har^2(\Circ, V)\subseteq H$ denote the \emph{Hardy
	space}\index[terminology]{Hardy space} of
	$\Lup^2$-functions on $\Circ$ with only nonnegative Fourier coefficients, and let
	$P$ denote orthogonal projection of $H$ onto $\Har^2(\Circ, V)$. Then $H = PH \oplus
	P^\perp H$. The \emph{Fock space}\index[terminology]{Fock space} $\Fock_P$ is the Hilbert
	space completion of the alternating algebra:
	\begin{equation}
	\label{Fock_P}
	\Fock_P = \exteriorhat (PH \oplus \overline{P^\perp H}) \cong
	\Directsumhat_{i,j \geq 0} \exterior^i(PH) \oplus \exterior^j(\overline{P^\perp H})\period
	\end{equation}
	Here $\overline V$ denotes the complex conjugate vector space to $V$, and $\exteriorhat$ and
	$\Directsumhat $ denote Hilbert space completions. The Fock space
	turns out to be the ``giant representation'' we were after: it's the
	fundamental representation of $\LSU_n$.
\end{construction}
\begin{remark}[(The Fock space in physics)]
The process of building a Fock space out of a Hilbert space $H$, as in \eqref{Fock_P}, has a quantum-mechanical
interpretation. Suppose that $H$ is the space of states describing the mechanics of a particle: for example,
$\Lup^2(\Circ, \CC)$ corresponds to a particle moving on a circle. The corresponding Fock space is the space of
states for systems with any number of particles. In \cref{Fock_constr}, we used the alternating algebra, which
means that the particles are fermions: the relation $f\wedge f = 0$ is the Pauli exclusion principle, imposing that
two fermions cannot be in the same state. For a bosonic many-body system, one would use the (Hilbert space
completion of the) symmetric algebra.%
\index[terminology]{fermion}%
\index[terminology]{boson}%
\index[terminology]{Pauli exclusion principle}
The process of building a Fock space from a single-particle Hilbert space is called second
quantization.\index[terminology]{second quantization}

In our setting, $\Lup^2(\Circ, V)$ corresponds to a system with a fermion moving on a circle, together with some
kind of $G$-symmetry. The subspace $\exterior^i(PH) \oplus \exterior^j(\overline{P^\perp H})$ consists of $i$ fermionic
particles and $j$ fermionic antiparticles. This explains why we take the conjugate space to $P^\perp H$: it is so
that the antiparticles have positive energy.\index[terminology]{antiparticle}
\end{remark}
A loop on $G$ acts on $H$ by pointwise multiplication, and $f\in \Diffplus(\Circ)$
acts on $H$ by sending $\xi \colon \Circ\to V$ to $\xi(f^{-1}(z)) \cdot
|(f^{-1})'(z)|^{1/2}$. (The square root factor is a normalization factor to
ensure unitarity of the action.) In fact, this gives an action of $\LG\rtimes
\Diffplus(\Circ)$ on $H$, and one can ask when this descends to a projective
representation of $\LG \rtimes \Diffplus(\Circ)$ on the Fock space $\Fock_P$. Segal
wrote down a \emph{quantization condition} for when a unitary operator on $H$ descends to a projective
transformation of $\Fock_P$: namely, $u$ descends to $\Fock_P$ if and only if the commutator $[u,P]$ is
Hilbert--Schmidt.\footnote{Recall that a bounded operator $A$ on a Hilbert space is \textit{Hilbert--Schmidt} if
$\Tr(A^\ast A)$ is finite.} One checks that the action of $\LG \rtimes \Diffplus(\Circ)$ on $H$ satisfies Segal's
quantization criterion, and so descends to a projective representation of $\LG \rtimes \Diffplus(\Circ)$ on the
Fock space $\Fock_P$.

Almost by definition, the action of $\Circ = \TTrot$ on $\Fock_P$ is of positive
energy, and so $\Fock_P$ is a representation of positive energy. 
It turns out that:

\begin{theorem}[{\cites[Section 10.6]{loop}[Chapter I.5]{loop-rep}}]\label{sun-summand}
	The irreducible summands of $\Fock_P^{\otimes h}$ give all the irreducible
	positive energy representations of $\LSU_n$ of level $h$.
\end{theorem}

%\begin{remark}
%	The space denoted $\cal{H}$ in that section of \cite{loop} is the Hilbert space
%	completion of the Fock space $\Fock_P$.
%\end{remark}
%
We will expand on this construction of the irreducible level $h$ representations of $\LSU_n$ in
\cref{segal_sugawara}, when we discuss the Segal--Sugawara construction.

%This approach isn't exactly the one taken in \cite{loop}, but something
%essentially like it is.
The first reduction comes from:
\begin{lemma}[{\cite[Lemma 13.4.3]{loop}}]
\label{summand}
	Let $V$ and $W$ be positive energy representations of $\LGtilde$. Suppose
	that $V$ is irreducible, and that $V\oplus W$ admits an intertwining action
	of $\Diffplus(\Circ)$. Then $V$ admits an intertwining action of $\Diffplus(\Circ)$.
\end{lemma}

We will prove this shortly; first, we will indicate how to use this to prove the
general case.

\begin{remark}\label{reduction}
	It suffices to prove by \cref{summand} that for every irreducible
	positive energy representation $V$ of $\LG$, there is some $G'$ and an
	embedding $i\colon \LG\to \LG'$ where \Cref{main-thm} is true for $G'$, and
	an irreducible representation $V'$ of $\LG'$ such that $V$ is a summand of
	$i^\ast V'$.
\end{remark}

To use this reduction, we first need to establish that \Cref{main-thm} is true for a
class of Lie groups $G$. In fact:

\begin{theorem}
	\Cref{main-thm} is true if $G$ is simple, simply connected, and
	simply laced.
\end{theorem}

The proof of this result is quite similar to that of \cref{sun-summand}:
one constructs the analogue of the Fock space for $\LG$ (which, like in the
$\SU_n$ case, has an intertwining action of $\Diffplus(\Circ)$), and then shows
that every irreducible positive energy representation is a summand of some twist
of this representation of $\LG$. See \cite[\S 13.4]{loop} for more details.

\begin{construction}
	Let $\Omega G$ denote the \emph{based} loop space of $G$, regarded as the
	homogeneous quotient $\LG/G \simeq \LG_\CC/\Lup^+ G_\CC$.
	Since $ G $ is simple any simply connected,
	\begin{equation*}
		\H^2(\Omega G;\ZZ) \cong \H^3(G; \ZZ) \cong \ZZ \comma
	\end{equation*}
	so every integer gives rise to a complex line bundle on
	$\Omega G$. The holomorphic sections $\Gamma$ of the line bundle
	corresponding to the generator is called the \emph{basic representation} of
	$\LG$.\footnote{Of course, the abelian group $\ZZ$ has two generators. Here we have a canonical one: as
	discussed above, we have a canonical generator for $\H^4(\BG;\ZZ)$, hence $\H^3(G;\ZZ)$ via transgression, and
	therefore also for $\H^2(\Omega G;\ZZ)$.}
	\index[terminology]{based loop space}
	\index[notation]{OmegaG@$\Omega G$}
	\index[terminology]{basic representation of $\LG$}
\end{construction}

\begin{example}
	If $G = \SU_n$, $\Gamma$ is the Fock space
	described above.
\end{example}

Then:

\begin{prop}[{\cite[Proposition 9.3.9]{loop}}]\label{basic}
	Let $G$ be a simple, simply connected, and simply laced Lie group. Then any
	irreducible positive energy representation of level $h$ of $\LG$ is a summand
	in $i_h^\ast \Gamma$, where $i_h:\LG\to \LG$ is the map induced by the degree
	$h$ map $\Circ\to \Circ$.
\end{prop}

The level $1$ representation $\Gamma$ admits an intertwining
action of $\Diffplus(\Circ)$ via the ``blip construction.''%
\index[terminology]{blip construction} We will not go into the details here; see \cite[\S 13.3]{loop}. Assuming
this, combining \cref{basic} with \cref{summand} shows that \cref{main-thm} is true for $\LG$ when $G$ is simply
laced (and simple and simply connected).

According to \cref{reduction}, it now suffices to show:
\begin{prop}\label{new-reduction}
	For every irreducible positive energy representation $V$ of $\LG$, there is a
	simply laced $G'$ and an embedding $i:\LG\to \LG'$, as well as an irreducible
	representation $V'$ of $\LG'$ such that $V$ is a summand of $i^\ast V'$. 
\end{prop}
This is proved in \cite[Lemma 13.4.4]{loop} in the following manner.

One first classifies all the irreducible representations of $\LG$. Using the loop
group analogue of Schur--Weyl duality worked well when $G = \SU_n$, but that
won't do in the general case. Instead, one utilizes a loop group analogue of
Borel--Weil (see \cite[Section 4.2]{segal-survey}). Recall how this works for finite-dimensional, compact Lie
groups: fix a maximal torus $T$ of $G$, and then, for every antidominant weight $\lambda$ of $T$ (i.e., $\langle
h_\alpha, \lambda\rangle \leq 0$ for every positive root $\alpha$), there is an associated line bundle
$\cal{L}_\lambda$ on $G/T \cong G_\CC/B^+$. The space of holomorphic sections of $\cal{L}_\lambda$ is an
irreducible representation of $G$ of lowest weight $\lambda$, and all irreducible representations of $G$ arise this
way.\index[terminology]{maximal torus}

In the loop group case, one again begins by fixing a maximal torus $T$ of $G$
(one should think of $\TTrot\times T \times \TTce$ as a maximal torus of
$\LG$). Consider the homogeneous space $\LG/T$. There is a fiber sequence
\begin{equation}
	G/T\to \LG/T \to \Omega G,
\end{equation}
and the set of isomorphism classes of complex line bundles on $\LG/T$ is
\begin{equation}
	\H^2(\LG/T;\ZZ) \cong \H^2(\Omega G;\ZZ) \oplus \H^2(G/T; \ZZ) = \ZZ \oplus
\widehat{T},
\end{equation}
where $\widehat{T}$ is the character group of $T$. You can prove this using the Serre spectral sequence, which as
usual is easier because $G$ is simple and simply connected. Anyways, we learn that line bundles on $\LG/T$ are
indexed by $(h,\lambda)\in \ZZ\oplus \widehat{T}$.

\begin{theorem}[{(Borel--Weil for loop groups \cite[Theorem 9.3.5]{loop})}]
	One has:
	\begin{itemize}
	\item The space $\Gamma(\cal{L}_{h,\lambda})$ of holomorphic sections is
		zero or irreducible of positive energy of level $h$; moreover, every
		projective irreducible representation of $\LG$ arises this way.
	\item The space $\Gamma(\cal{L}_{h,\lambda})$ is nonzero if and only if
		$(h,\lambda)$ is antidominant,\footnote{Recall that if $G$ is the
		simply laced group $\SU_n$, then the weight lattice is
		$\bigoplus_{1\leq i\leq n+1} \ZZ\chi_i/\ZZ\sum_i \chi_i$, and the
		roots are $\chi_i - \chi_j$ with $i\neq j$. The positive roots,
		corresponding to the usual Borel subgroup of upper-triangular matrices, are
		$\chi_i - \chi_j$ for $i<j$.  Therefore, $(h, \lambda = \lambda_1,
		\cdots, \lambda_n)$ is antidominant if $\lambda$ is antidominant,
		i.e., $\lambda_1 \leq \cdots \leq \lambda_n$, and if $\lambda_n -
		\lambda_1 \leq h$.} i.e.,
		$$0\geq \lambda(h_\alpha) \geq -\frac{h}{2}\langle h_\alpha,
		h_\alpha\rangle$$
		for each positive coroot $h_\alpha$ of $G$. (In particular,
		$\lambda$ is antidominant as a weight of $T\subseteq G$.)
		\index[terminology]{antidominant}%
		\index[terminology]{Borel subgroup}%
		\index[terminology]{coroot}
	\end{itemize}
\end{theorem}
The upshot is that irreducible representations correspond to antidominant
weights. To prove \Cref{new-reduction}, it suffices to show that
all antidominant weights of $\LG$ are restrictions of antidominant weights of
$\LG'$ for some simply laced $G'$. The argument now proceeds case-by-case, as $G$
ranges over all simple simply connected simply laced compact Lie groups. The
proof is not very enlightening, so we will not go into more detail here.

%-------------------------------------------------------------------%
%-------------------------------------------------------------------%
%  Appendix: Random thoughts										%
%-------------------------------------------------------------------%
%-------------------------------------------------------------------%

\begin{remark}[(Relationship with Wess--Zumino--Witten theory)]
Segal \cite{segal-book} studies the theory of positive energy representations of $\LG$ from a different
perspective, that of conformal field theory.\index[terminology]{conformal field theory} Specifically, the category
of level $h$ positive energy representations of $\LG$ has the structure of a \textit{modular tensor
category}\index[terminology]{modular tensor category} Given a modular tensor category $\mathsf C$, one can build
\begin{enumerate}[(1)]
	\item a $3$-dimensional topological field theory $Z_{\mathsf C}$ \cite{RT90, RT91, Wal91, BK01, KL01, BDSV15},
	and
	\item a $2$-dimensional conformal field theory \cite{MS89}.
\end{enumerate}
These two theories are related: the 2d CFT is a boundary theory for the 3d TFT \cite{Wit89, FT12}. When $\fC$ is
the category of level $h$ representations of $\LG$, the TFT is Chern--Simons theory (see \cref{quantum_CS}) and the
CFT is the Wess--Zumino--Witten model (see \cref{quantum_WZW}).\footnote{One might wonder if every modular tensor
category arises in this way, as a category of positive-energy representations of a loop group. This is the
Moore--Seiberg conjecture,\index[terminology]{Moore--Seiberg conjecture} and is open at the time of writing. See,
e.g., \cite{HRW08}.}

You do not need \cref{main-thm} to construct the modular tensor category structure on $\mathsf{Rep}_k(\LG)$, and
the TFT and CFT provide a very large amount of data associated to that structure. It may be possible to coax
\cref{main-thm} out of that extra structure. For example, Segal \cite[\S 12]{segal-book} discusses this for abelian
Lie groups.
\end{remark}
%When $G$ is a compact, connected, abelian Lie group (so a torus), \cref{main-thm} Segal \cite[\S 12]{segal-book} gives a different
%prof of the analogu\cref{main-thm} via conformal field theory, which we briefly discussed in \cref{virasoro_applications}, in
%\cite{segal-book}.
%
%It might also be possible to prove \cref{main-thm} using TFT, specifically a 3d TFT called Chern--Simons theory. Fix
%a compact Lie group $G$; associated to a $3$-manifold $M$, principal $G$-bundle $P\to M$, and connection $A$
%on $P$, we have the Chern--Simons $3$-form
%\begin{equation}
%	\Tr\left(A \wedge dA + \frac{2}{3} A \wedge A \wedge A\right),
%\end{equation}
%whose derivative is $\Tr(F \wedge F)$, where $F$ is the curvature of $A$ and $\Tr$ is
%induced from the (positive-definite) Killing form on $\g$ associated via the
%Chern--Weil homomorphism
%\begin{equation*}
%	\RR\cong \H^4(\BG;\RR) \to \Sym^2(\g^\ast)^G
%\end{equation*}
%to the generator. 
%The Chern--Simons action
%\begin{equation}
%\label{CS_action}
%	S = \frac{k}{4\pi} \int_{M^3} \Tr\left(A \wedge dA + \frac{2}{3} A \wedge A \wedge A\right)
%\end{equation}
%defines for every integer $k\in \ZZ$ a functional on the moduli space of principal $G$-bundles with connection on
%$M$.  The integer $k$ is the called the \emph{level}.
%
%%The functional \eqref{CS_action} can be interpreted as a Lagrangian action for a classical field theory, and
%Schwarz \cite{Sch77} and Witten \cite{Wit89} quantized it using the path integral and argued that the resulting
%quantum field theory is topological, in that it does not depend on the choice of Riemannian metric on
%$M$.\footnote{In general, the path integral is not entirely mathematically understood. For $G$ finite, this path
%integral construction is made rigorous by Freed-Quinn \cite{FQ93}.} Therefore we would like to describe this theory
%as a symmetric monoidal functor
%\[Z_{G,k}\colon\Bord_3^{\SO}\longrightarrow \fC\comma\]
%where $\fC$ is some symmetric monoidal $(\infty, 3)$-category. It is not known how to do this in
%general,\footnote{There are a few different perspectives on what $Z_{G,k}(\mathrm{pt}_+)$ should be. For $G$
%finite, the answer is known by work of Freed--Hopkins-Lurie-Teleman \cite[\S 4.2]{FHLT10} and Wray \cite[\S
%9]{Wra10}; for $G$ a torus, the answer is due to Freed--Hopkins--Lurie--Teleman (\textit{ibid}.). For general $G$, two
%different approaches are provided by Freed--Teleman (see \cite{432876}) and Henriques \cite{Hen17b, Hen17a}.}
%but it is known how to extend it to a theory of $1$-, $2$-, and $3$-manifolds, valued in the $2$-category of
%$\mathbb C$-linear categories, by work of Reshetikhin--Turaev \cite{RT90, RT91}, Walker \cite{Wal91},
%Bakalov--Kirillov \cite{BK01}, Kerler--Lyubashenko \cite{KL01}, and
%Bartlett--Douglas--Schommer--Pries--Vicary \cite{BDSV15}, \footnote{These constructions require some additional
%structure on our manifolds, such as a choice of trivialization of the first Pontryagin class. As theories of merely
%oriented manifolds, Chern--Simons theories are \emph{anomalous}. See \cites[\S 9.3]{FHLT10}[]{432876} for more
%information.  When we dimensionally reduce to the Verlinde TFT, below, the anomaly can be canonically trivialized.}
%and $Z_{G,k}(\Circ)\simeq\mathsf{Rep}^k(LG)$, at least when $G$ is simply connected. See Freed \cite{Fre09} for a
%%general survey on Chern--Simons theory.
%
%In other words, the reduction of Chern--Simons theory to a $2$-dimensional TQFT (i.e., the TFT
%$Z_{G, k}(-\times \Circ)$), which is called the \emph{Verlinde TFT}, is a fully extended two-dimensional TFT
%whose value on a point is $\mathsf{Rep}^k(LG)$. Freed--Hopkins--Teleman \cite{FHT10} also provide a direct construction
%of the Verlinde TFT.
%
%Realizing $\mathsf{Rep}^k(LG)$ as part of the data of a TFT gives it additional structure. For example, given
%$\varphi\in\Diffplus(\Circ)$ and let $T_\varphi$ denote its mapping cylinder: $[0, 1]\times \Circ$, where the incoming $\Circ$ is
%attached by the identity and the outgoing one is attached by $\varphi$. Evaluating the Verlinde TFT on these mapping
%cylinders defines a $\Diffplus(\Circ)$-action on the category $\mathsf{Rep}^k(LG)$, and perhaps this is related to the
%p%rojective intertwining action from \cref{main-thm}. There is also a close relationship between Chern--Simons theory
%and 2d CFTs, so this approach may be related to Segal's proof.
%
%
%
%%This action defines a \emph{topological} quantum field theory. It is quite
%well-behaved, and one expects it to be an extended topological quantum field
%theory. In particular, the cobordism hypothesis says that this should be
%deem by a dualizable object in some symmetric monoidal
%$(\infty,3)$-category. I don't think that the value of Chern--Simons theory on a
%point is known (but there are multiple candidates). However, a general rule of
%extended TQFTs is that the value of an extended TQFT on the circle is the center
%(i.e., Hochschild (co)homology) of the value on a point. In particular, the
%value of Chern--Simons theory on the circle is the center of its value on the
%point, and it seems to be the case that its value on the circle (if $G$ is
%simply connected, at least) is the category $\mathrm{Rep}^k(\LG)$ of positive
%energy representations of $\LG$ of level $k$.  I would guess that the flow of information goes
%the other way, but perhaps this perspective can be used to prove some of the
%statements above about the positive energy representations of loop groups.

\subsection{OK, but what does this have to do with differential cohomology?}
\label{PS_diffcoh}
There is differential cohomology hiding in the background of the story of central extensions of loop groups. There
are two ways in which it appears: one which is related to the story of on-diagonal differential characteristic
classes built from Chern--Weil theory, and another which relates central extensions to off-diagonal Deligne
cohomology similarly to the discussion of the Virasoro group in \cref{VirasoroAlgebra}. This, together with the
appearance of $\Diffplus(\Circ)$ in the representation theory of loop groups, suggests that loop groups and the
Virasoro group should interact somehow, as we will see in the next chapter.

% first: Chern--Weil, curvature, etc
\subsubsection{The on-diagonal story}
Suppose $G$ is simple and simply connected, so that $\H^4(\BG;\ZZ)$, $\H^3(G;\ZZ)$, and $\H^2(\LG;\ZZ)$ are all
isomorphic to $\ZZ$, and the transgression maps
\begin{equation*}
	\H^4(\BG;\ZZ)\to\H^3(G;\ZZ)\to\H^2(\LG;\ZZ)
\end{equation*}
are isomorphisms.%
\index[terminology]{transgression}
The level $h$ canonically refines to $\hat h\in\Hhat^4(\BnablaG;\ZZ)$
(\cref{differential_CW_lift}), and the transgression map refines to a map
$\Hhat^4(\BnablaG;\ZZ)\to\Hhat^3(G;\ZZ)$ \cite[\S 3]{CJMSW05}, as we discussed in \cref{transgression_detail}. Does
the story continue to a differential refinement $\Hhat^3(G;\ZZ)\to\Hhat^2(\LG;\ZZ)$? That is, a projective
representation $\LG\to\PU(V)$ determines a central extension $\LGtilde$ of $\LG$, which is a principal
$\TT$-bundle over $\LG$. Does this $\TT$-bundle come with a canonical connection?

Of course, this is a loaded question, and we'll see that the answer is yes. But first, a (relatively) down-to-Earth
plausibility argument. Given a central extension
% TODO\colon \shortexact?
\begin{subequations}
\begin{equation}
	1\to \TTce\to \LGtilde \to \LG \to 1,
\end{equation}
we can differentiate it to obtain a central extension of Lie algebras
\begin{equation}
\label{Loop_alg_cext}
	0\to \RR\to \Lgtildeno \to \Lg \to 0.
\end{equation}
\end{subequations}
Recall from \cref{cext_lie_alg} that the central extension \eqref{Loop_alg_cext} can be described by a cocycle for
the Lie algebra cohomology group\index[terminology]{Lie algebra!cohomology} $\HLie^2(\Lg; \RR)$. Cocycles
are alternating maps $\omega\colon \Lg\times\Lg\to\RR$ satisfying the cocycle condition \eqref{LA_cext_Jacobi}.
Choose a cocycle $\omega$; then, $\Lgtildeno$ is the vector space $\Lg\oplus\RR$ with the Lie bracket
\begin{equation}
	[(\xi, a), (\eta, b)] \colonequals ([\xi, \eta],\omega(\xi, \eta)).
\end{equation}
For example, an element of $\H^4(\BG;\RR)$ corresponds via the Chern--Weil machine to an invariant symmetric
bilinear form $\langle\text{--}, \text{--}\rangle\colon \g\times\g\to\RR$, and it defines a degree-$2$ Lie algebra
cocycle for $\Lg$ by \cite[\S 4.2]{loop}
\begin{equation}
	\omega(\xi, \eta) \colonequals \frac{1}{2\pi}\int_{\Circ} \langle \xi(\theta), \eta'(\theta)\rangle\,\d\theta.
\end{equation}
Suppose that $\omega$ comes from a central extension of $\LG$ which is a principal $\TT$-bundle $\pi\colon
\LGtilde\to \LG$. Then $T\LGtilde$ fits into a short exact sequence
\begin{equation}
\label{loop_ext_tangent_ses}
	0\to T\TT\to T\LGtilde\to \pi^*T\LG\to 0.
\end{equation}
At the identity of $\LGtilde$ this is \eqref{Loop_alg_cext}, and left translation carries this identification to
every tangent space. The data of $\omega$ includes a splitting of \eqref{Loop_alg_cext}, and left translation turns
this into a splitting of \eqref{loop_ext_tangent_ses}. A connection on $\pi\colon\LGtilde\to\LG$ is a
$\TT$-invariant splitting, and since $\TT$ acts trivially on its Lie algebra, we have just built a connection with
curvature $\omega$. Thus the class of \eqref{Loop_alg_cext} in $\H^2(\LG;\ZZ)$ refines to a class in
$\Hhat^2(\LG;\ZZ)$. Pressley--Segal \cite[Theorem 4.4.1]{loop} show that this is a necessary and sufficient
condition on $\omega$ for any compact, simply connected Lie group $G$, and that $\omega$ determines the
extension.\footnote{When $G$ is not simply connected, the theorem is not quite as nice: see \cite[Theorem
4.6.9]{loop} and \cite{Wal17}.}
\begin{remark}
It may be possible to do this ``all at once'' by finding a canonical connection $\conn$ on the principal
$\TT$-bundle $\pi\colon \Uup(V)\to \PU(V)$ where $V$ is an infinite-dimensional separable Hilbert space;
this would lift the tautological class $c_1(\Uup(V))\in\H^2(\PU(V);\ZZ) = \H^2(\Kup(\ZZ, 2); \ZZ)$ to
$\chat_1(\Uup(V), \conn) \in \Hhat^2(\PU(V);\ZZ)$. Then a projective representation would pull back
$\chat_1(\Uup(V), \conn)$ (and $\conn$) to $\LG$.
\end{remark}
To summarize a little differently, given $\hat h\in\Hhat^4(\BnablaG;\ZZ)$, we can obtain a Chern--Weil form
$\langle\text{--}, \text{--}\rangle$, hence a cocycle $\omega\in\HLie^2(\Lg;\RR)$. Because $\curv(\hat h)$
satisfies an integrality condition, so does $\omega$, which turns out to be the same condition needed to define a
central extension $\LGtilde\to\LG$ with a connection. That is, we built a map
$\Hhat^4(\BnablaG;\ZZ)\to\Hhat^2(\LG;\ZZ)$. We would like to describe it more directly.

The first step is the transgression map $\Hhat^4(\BnablaG;\ZZ)\to\Hhat^3(\BG;\ZZ)$ constructed by \cite[\S
3]{CJMSW05}. To get from $3$ to $2$, Gawędzki \cite[\S 3]{Gaw88} constructs for any closed manifold $M$ a
transgression map\index[terminology]{transgression}
\begin{equation}
	\Hhat^3(M;\ZZ)\to\Hhat^2(\mathrm LM;\ZZ)
\end{equation}
from the perspective that differential cohomology is isomorphic to the
hypercohomology\index[terminology]{hypercohomology} of the Deligne complex\index[terminology]{Deligne complex}%
\footnote{Gawędzki actually works with a different complex,
namely $0\to \TT\to i\Omega^1\to\dotsb\to i\Omega^{n-1}\to 0$, where the map $\TT\to i\Omega^1$ is $\d\circ{\log}$.
This is equivalent to $\Sigma\ZZ(n)$ \cite[Remark 3.6]{BML94}, and the proof is a straightforward generalization of
\cref{cext_lemma}.} 
\begin{equation*}
	0\to\ZZ\to\Omega^0\to\dotsb\to\Omega^{n-1}\to 0 \period
\end{equation*}
Another option is to construct the transgression as follows: first pull back by the evaluation
map $\Circ\times \mathrm LM\to M$, then integrate over the $\Circ$ factor using the map we constructed in
\cref{FiberIntegration}.

% second: Deligne-Beĭlinson
\subsubsection{The off-diagonal story}
In \cref{VirasoroAlgebra}, we saw in \cref{cext_corollary} that central extensions of a Lie group $\Gamma$
(possibly infinite-dimensional) which are principal $\TT$-bundles are classified by $\H^3(\BbulletGamma;\ZZ(1))$.
The central extensions of loop groups we constructed in this chapter are principal $\TT$-bundles. Therefore there
is in principle a way to start with a class $h\in \H^4(\BG;\ZZ)$ and obtain a class $\phi(h)\in\H^3(\mrm{B}_\bullet
\LG;\ZZ(1))$, and that is what we are going to do next.

Recall that truncating defines a map of complexes of sheaves of abelian groups $\ZZ(n)\to\ZZ$, inducing for us a
map
\begin{equation}
\label{4_to_2}
	\H^4(\BbulletG;\ZZ(2)) \to \H^4(\BbulletG;\ZZ)\isomorphism \H^4(\BG;\ZZ).
\end{equation}
%(TODO: make sure the equality in the second map is right. We don't say why this is anywhere, yet.)
\begin{lem}
For $G$ a compact Lie group, \eqref{4_to_2} is an isomorphism.
\end{lem}
\begin{proof}
Recall from \cref{2n_to_n} that \eqref{4_to_2} is part of the pullback square
\begin{equation}
\begin{gathered}
	\begin{tikzcd}[column sep=3.5em]
		\H^{4}(\BbulletG;\ZZ(2))\arrow["\eqref{4_to_2}", r]\arrow[d] & \H^{4}(\BG;\ZZ)\arrow[d]\\
		\Sym^2(\gdual)^G\arrow[r] & \H^{4}(\BG;\RR),
	\end{tikzcd}
\end{gathered}
\end{equation}
where the bottom map is the Chern--Weil map. Since $G$ is compact, the Chern--Weil map is an isomorphism,
so \eqref{4_to_2} is as well.
\end{proof}
Therefore our level $h\in \H^4(\BG;\ZZ)$ is equivalent data to an off-diagonal characteristic class $\tilde h\in
\H^4(\BbulletG;\ZZ(2))$. The next step is the construction of yet another transgression map, this time due to
Brylinski--McLaughlin \cite[\S 5, on p.\ 618]{BML94}:
\begin{equation}
	\H^4(\BbulletG;\ZZ(2))\longrightarrow \H^3(\mrm{B}_\bullet\LG;\ZZ(1)).
\end{equation}
Their construction models elements of these two differential cohomology groups simplicially: they identify
$\H^4(\BbulletG;\ZZ(2))$ as the abelian group of equivalence classes of gerbes with a connective structure over a
simplicial manifold model for $\BbulletG$, and $\H^3(\mrm{B}_\bullet\LG;\ZZ(1))$ as equivalence classes of line
bundles over a simplicial model for $\mrm{B}_\bullet\LG$ (\textit{ibid.}, Theorem 5.7).

We have obtained some class in $\H^3(\mrm{B}_\bullet\LG;\ZZ(1))$ from a level $h\in \H^4(\BG;\ZZ)$, hence some
central extension. That this coincides with the central extension obtained from $h$ by the other methods in this
chapter is due to Brylinski--McLaughlin (\textit{ibid.}, \S 5). See also Brylinski \cite[\S 6.5]{Bry08} for related
discussion and Waldorf \cite[\S 3.1]{Wal10} for another construction of this transgression map.


