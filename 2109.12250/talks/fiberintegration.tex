%!TEX root = ../diffcoh.tex

\section{Fiber Integration}\label{FiberIntegration}
\textit{by Araminta Amabel}

The goal of this section is to define a refinement of \textit{fiber integration} (along with its usual properties) in the setting of differential cohomology.
In ordinary cohomology, we get a fiber integration map from combining the Thom isomorphism and the suspension isomorphism. 
Let $E\to B$ be an oriented fiber bundle with fiber a compact manifold of dimension $k$. 
Let $ E\hookrightarrow\RR^{N}$ be an embedding with normal bundle $\nu$, and let $E^\nu$ denote the Thom space of $ \nu $. 
Then fiber integration is given by the composite
\begin{equation*}
	\begin{tikzcd}
		\H^{q+k}(E) \arrow[r, "\sim"{yshift=-0.25em}] & \H^{q+N}(E^\nu) \arrow[r, "\PT"] & \H^{q+N}(B_+\wedge \Sph{N}) \simeq \H^{q}(B_+) \comma
	\end{tikzcd}
\end{equation*}
where the first map is the Thom isomorphism, the second map is the Pontryagin--Thom collapse map, and the third map is the suspension isomorphism. Recall that the Thom isomorphism is given by taking the cup product with the Thom class. 

To do fiber integration in differential cohomology, we need to provide differential refinements of the following:
\begin{enumerate}[(1)]
	\item Thom classes/orientations.

	\item The suspension isomorphism.
\end{enumerate}
To do this, we combine fiber integration in ordinary cohomology with integration of forms. 

%-------------------------------------------------------------------%
%-------------------------------------------------------------------%
%  Differential Integration                                         %
%-------------------------------------------------------------------%
%-------------------------------------------------------------------%

\subsection{Differential Integration}\label{subsec:differentialintegration}

The input will be a fiber bundle of manifolds
\begin{equation*}
	M\to E\to X \comma
\end{equation*}
where $ M $ is a closed, smooth manifold of dimension $ d $.
The output will be a map of %ring? 
spectra
\begin{equation*}
	\ZZ(k)(E)\to\Sigma^{d}\ZZ(k-d)(X)
\end{equation*}
where $\ZZ(k)$ is the pullback 
\begin{equation*}
	\begin{tikzcd}
		\ZZ(k)\arrow[r]\arrow[d] & \Gamma^*\HZZ\arrow[d]\\
		\Sigma^{-k}\HOmegacl^k\arrow[r] & \Gamma^*\HRR
	\end{tikzcd}
\end{equation*}
in $\Sh(\Man;\Sp)$ and, similarly, $\ZZ(k-d)$ is the pullback
\begin{equation*}
	\begin{tikzcd}
		\ZZ(k-d)\arrow[r]\arrow[d] & \Gamma^*\HZZ\arrow[d]\\
	\Sigma^{d-k}\HOmegacl^{k-d}\arrow[r] & \Gamma^*\HRR \period
	\end{tikzcd}
\end{equation*}
To produce a map $\ZZ(k)\to\Sigma^d\ZZ(k-d)$, it therefore suffices to produce maps $\HZZ\to\Sigma^d \HZZ $ and $\Omegacl^k\to\Omegacl^{k-d}$
 together with a path between their images in $\Sigma^{d}\Gamma^*\HRR$.

%-------------------------------------------------------------------%
%  Differential Thom Classes and Orientations                       %
%-------------------------------------------------------------------%

\subsection{Differential Thom Classes and Orientations}

\begin{definition}
	Let $M$ be a smooth compact manifold and $V\to M$ a real vector bundle of dimension $k$. 
	A \emph{differential Thom cocycle} on $V$ is a cocycle
	\begin{equation*}
		U=(c,h,\omega)\in \check{Z}(k)^k_c(V)
	\end{equation*}
	such that, for each $m\in M$
	\begin{equation*}
		\int_{V_m}\omega =\pm 1
	\end{equation*}
\end{definition}

\begin{remark}
	A differential Thom class determines a ordinary Thom class in integral cohomology $\H^k_c(V;\ZZ)$.
\end{remark}

\begin{definition}[{\cite[Definition 2.9]{HopkinsSinger}}]
	An \emph{$\Hhat$-orientation} of $p\colon E\to B$ consists of the following data:
	\begin{enumerate}[(1)]
		\item a smooth embedding $E\subset B\times\RR^N$ for some $N$;

		\item a tubular neighborhood $W\subset B\times\RR^N$;

		\item a differential Thom cocycle $U$ on $W$.
	\end{enumerate}
\end{definition} 


%-------------------------------------------------------------------%
%  Differential Fiber Integration                                   %
%-------------------------------------------------------------------%

\subsection{Differential Fiber Integration}

Our hope is to get an analogue of the suspension isomorphism
\begin{equation*}
	\Hc^{q+N}(B\times\RR^N)\simeq \H^q(B) \period
\end{equation*}
To understand the correct analogue of the suspension isomorphism in the differential setting, let us consider the most simple case.

\begin{example}
	Consider the case when $B$ is a point and $N=1$.
	Then the ordinary suspension isomorphism says that 
	\begin{equation*}
		\H^1(\Circ;\ZZ)\cong  \H^0(\pt;\ZZ)\simeq \ZZ
	\end{equation*}
	The calculation $\H^1(\Circ;\ZZ) \isomorphic \ZZ$ is by degree:
	\begin{equation*}
		\begin{tikzcd}[sep=3em]
			\H^1(\Circ;\ZZ) = \uppi_0 \Map_{\Spc}(\Circ,\K(\ZZ,1)) = \uppi_0 \Map_{\Spc}(\Circ,\Circ) \arrow[r, "\sim"{yshift=-0.25em}, "\deg"'] & \ZZ \period 
		\end{tikzcd}
	\end{equation*}
	In differential cohomology, we have an isomorphism
	\begin{equation*}
		\Hhat^1(\Circ) \isomorphic \Mapsm(\Circ,\Circ) \period
	\end{equation*}
	We still have a degree map
	\begin{equation*}
		\deg \colon \Mapsm(\Circ,\Circ) \to \ZZ \comma
	\end{equation*}
	but it is no longer an isomorphism.
\end{example}

The upshot is that we are looking for a suspension \textit{map} not an isomorphism.

\begin{nul}
	We start by working with the trivial bundle $B\times\RR^N\to B$ and defining integration for compactly-supported forms. 
	This is \cite[\S 3.4]{HopkinsSinger}. 
	Define the map
	\begin{equation*}
		\int_{B\times\RR^N/B} \colon \Cech(p+N)^{q+N}_c(B\times\RR^N)\to\Cech(p)^q(B)
	\end{equation*}
	by the slant product with a fundamental cycle $Z_N\in \Cup_N(\RR^N;\ZZ)$,
	\begin{equation*}
		(c,h,\omega)\mapsto\left(c/Z_N,h/Z_N,\int_{B\times\RR^N/B}\omega\right)
	\end{equation*}
	Note that this is simply a map, \textit{not} an isomorphism.
\end{nul}

\begin{remark}
	Checking that the slant product goes through to differential cohomology seems to require some work. 
	See \cite[\S3.4]{HopkinsSinger}.
\end{remark}

\begin{definition}[{\cite[Definition 3.11]{HopkinsSinger}}]
	Let $p\colon E\to B$ be an $\Hhat$-oriented map of smooth manifolds with boundary of relative dimension $k$. 
	The \emph{integration map} is the map
	\begin{equation*}
		\int_{E/B}\colon\Cech(p+k)^{q+k}(E)\to \Cech(p)^q(B)
	\end{equation*}
	given by the composite
	\begin{equation*}
		\begin{tikzcd}[sep=3em]
			\Cech(p+k)^{q+k}(E) \arrow[r, "\cupprod U"] & \Cech(p+N)_c^{q+N}(B\times\RR^N) \arrow[r, "\int_{\RR^N}(-)"] & \Cech(p)_c^q(B) \period
		\end{tikzcd}
	\end{equation*}
\end{definition}

\begin{example}
	In dimension $ 1 $, the only closed manifold is $\Circ$. 
	If $E\to B$ is an oriented $\Circ$-bundle, then integration along the fibers defines a map
	\begin{equation*}
		\int_{E/B}\colon \Hhat^2(E)\to\Hhat^1(E)
	\end{equation*}
	If $ x \in \Hhat^2(E) $ corresponds to a line bundle with connection, then 
	\begin{equation*}
		\int_{E/B}x
	\end{equation*}
	represents the function $ B \to \Circ $ sending $ b \in B $ to the monodromy of $ x $ computed around the fiber $ E_b $.
\end{example}