%!TEX root = ../diffcoh.tex


\section{The Segal--Sugawara construction}
\textit{by Peter Haine}
\label{segal_sugawara}

Let $ G $ be a simply connected, simple, compact Lie group with Lie algebra $ \g $.
In \Cref{loop_groups}, we looked at central extensions
\begin{equation*}
	\begin{tikzcd}[sep=1.5em]
		1 \arrow[r] & \Circ \arrow[r] & \LGtilde \arrow[r] & \LG \arrow[r] & 1
	\end{tikzcd}
\end{equation*}
of the \textit{loop group} $ \LG \colonequals \Cinf(\Circ,G) $.
The group $ \Diffplus(\Circ) $ of orientation-preserving diffeomorphisms of the circle acts on $ \LG $ by precomposition.
So we might expect an action of the Virasoro group $ \Vir $ on $ \LGtilde $.
We saw that even though there is not an action of $ \Vir $ on $ \LGtilde $, roughly, the Virasoro group acts on any \textit{positive energy} representation of $ \LGtilde $.
However, the Virasoro action on positive energy representations of $ \LGtilde $ is very inexplicit, and we can
only guarantee the existence of the Virasoro action up to ``essential equivalence,'' which is not actually an
equivalence relation.\index[terminology]{essential equivalence}
In particular, the Pressley--Segal Theorem \cite[Theorem 13.4.3]{loop} (\cref{main-thm}) does not explicitly
explain how the central circle $ \Circ \subset \LGtilde $ acts.

The goal of this chapter is to explain the Lie algebra version of the Pressley--Segal Theorem, which gives an
explicit representation of the Virasoro algebra on any positive energy representation of the \textit{Kac--Moody
algebra} $ \Lgtildeno $ associated to a simple Lie algebra $ \g $ (over the complex numbers).  We'll be able to do
this by writing down explicit universal formulas for ``elements'' of the universal enveloping algebra $
\Univ(\Lgtildeno) $ that satisfy the Virasoro relations.  The catch is that these universal formulas involve
infinite sums, so they do not actually make sense as elements of $ \Univ(\Lgtildeno) $, but they do make sense
whenever we act on a representation where only finitely many of the terms don't act by zero; this is what the
positive energy condition guarantees.



Like in the previous chapter, we are not assuming you're familiar with all of these words. In
\cref{subsec:reminders}, we review some important definitions from \Cref{VirasoroAlgebra}. In \cref{sec:KacMoody},
we define the loop algebra of a Lie algebra, which up to regularity issues is the Lie-algebraic analogue of the
loop group of a Lie group. We also introduce \textit{Kac--Moody algebras}, the analogues of the central extensions
of loop groups we constructed in \cref{rep_loop}. In \cref{sec:SegalSugawara}, we introduce the Segal--Sugawara
construction, first at a high level, then digging into the details.


%-------------------------------------------------------------------%
%-------------------------------------------------------------------%
%  Introduction                                                     %
%-------------------------------------------------------------------%
%-------------------------------------------------------------------%

%-------------------------------------------------------------------%
%  Reminders on Virasoro \& Witt algebras                           %
%-------------------------------------------------------------------%

\subsection{Reminders on Virasoro \& Witt algebras}\label{subsec:reminders}

\begin{definition}
	The (complex) \textit{Witt algebra} is the complex Lie algebra $ \wittCC $ of polynomial vector fields on $
	\Circ $.  Explicitly, $ \wittCC $ has generators $ L_m \colonequals ie^{im\theta} \frac{\dup}{\dup\theta} $ for
	$ m \in \ZZ $ with Lie bracket
	\begin{equation*}
		[L_m,L_n] \colonequals (m-n) L_{m+n}
	\end{equation*}
	for all $ m,n \in \ZZ $.\index[terminology]{Witt algebra!complex}
\end{definition}
This is the complexification of the Witt algebra we discussed in \Cref{Witt_algebra}.

\begin{nul}
	Ignoring regularity issues, the Witt algebra is the complexification of the Lie algebra of the group $
	\Diffplus(\Circ) $ of orientation-preserving diffeomorphisms of the circle.\footnote{For the readers who care
	about regularity: the Lie algebra of $\Diffplus(\Circ)$ is the Lie algebra of all smooth vector fields on
	$\Circ$, and $\wittCC$ is a dense subset of the complexification. See \cites[\S 3.3]{loop}{MO:267249}.}
\end{nul}

\begin{nul}
Recall from \Cref{cext_lie_alg} that central extensions of Lie algebras are classified by Lie algebra
cohomology.\index[terminology]{Lie algebra cohomology} We have that $ \HLie^2(\wittCC;\CC) \isomorphic \CC $, so
there is a $ 1 $-dimensional space of central extensions of the Witt algebra.
\end{nul}

\begin{definition}
	The (complex) \textit{Virasoro algebra} $ \virCC $ is the central extension\index[terminology]{Virasoro
	algebra!complex}
	\begin{equation}
	\label{cpxVira}
		\begin{tikzcd}[sep=1.5em]
			1 \arrow[r] & \CC \chg \arrow[r] & \virCC \arrow[r] & \wittCC \arrow[r] & 1
		\end{tikzcd}
	\end{equation}
	of $ \wittCC $ with generators $ L_m $ for $ m \in \ZZ $ and a central element $ \chg $, and nontrivial Lie bracket given by
	\begin{equation}
	\label{Witt_cocycle}
		[L_m,L_n] \colonequals (m-n) L_{m+n} + \delta_{m,-n} \frac{m^3 - m}{12} \chg
	\end{equation}
	for all $ m,n \in \ZZ $.

	We call the central element $ \chg \in \virCC $ the \textit{central charge}.\index[terminology]{central
	charge}\index[notation]{chg@$\chg$}
\end{definition}
Said a little differently,~\eqref{Witt_cocycle} spells out a cocycle for $\HLie^2(\wittCC;\CC)$, which determines
the central extension~\eqref{cpxVira}.

\begin{nul}
	Again, ignoring regularity issues, the Virasoro algebra is the complexification of the Lie algebra of the Virasoro group $ \Vir $.
\end{nul}

%-------------------------------------------------------------------%
%  Talk overview                                                    %
%-------------------------------------------------------------------%

%\subsubsection{Talk overview}\label{subsec:overview}

%\begin{goal}
%	\end{goal}

%-------------------------------------------------------------------%
%-------------------------------------------------------------------%
%  Loop algebras & Kac–Moody algebras                               %
%-------------------------------------------------------------------%
%-------------------------------------------------------------------%

\subsection{Loop algebras and Kac--Moody algebras}\label{sec:KacMoody}

The first thing we need to explain in order to state the Segal--Sugawara construction is what the Kac--Moody algebra $ \Lgtildeno $ is.
As the notation suggests, $ \Lgtildeno $ is the Lie algebra analog of the central extension $ \LGtilde $ of the loop group $ \LG $ (with suitable finiteness hypotheses).
Before talking about Kac--Moody algebras, we need to talk about loop algebras.

%-------------------------------------------------------------------%
%  Loop algebras                                                    %
%-------------------------------------------------------------------%

\subsubsection{Loop algebras}\label{subsec:loops}

\begin{recollection}
	Let $ \g $ be a Lie algebra over a ring $ R $, and let $ S $ be an $ R $-algebra.
	The basechange $ \g \tensor_R S $ of $ \g $ to $ S $ is the Lie algebra over $ S $ with underlying $ S $-module the basechange $ \g \tensor_R S $ of the underlying $ R $-module of $ \g $ to $ S $ with Lie bracket extended from pure tensors from the formula
	\begin{equation*}
		[X_1 \tensor s_1, X_2 \tensor s_2]_{\g \tensor_R S} \colonequals [X_1,X_2]_{\g} \tensor s_1 s_2 \period 
	\end{equation*}
\end{recollection}

\begin{definition}
	Let $ \g $ be a complex Lie algebra.
	The \textit{loop algebra} $ \Lg $ of $ \g $ is the Lie algebra\index[terminology]{loop algebra}
	\index[notation]{Lg@$\Lg$}
	\begin{equation*}
		\Lg \colonequals \g \tensor_{\CC} \CC[t^{\pm 1}] \comma
	\end{equation*}
	regarded as a Lie algebra over $ \CC $ (rather than $ \CC[t^{\pm 1}] $).
\end{definition}

\begin{notation}
	Let $ \g $ be a complex Lie algebra, $ X \in \g $, and $ m $ an integer. 
	We write 
	\begin{equation*}
		X\ang{m} \colonequals X \tensor t^m \in \Lg \period
	\end{equation*}
	\index[notation]{Xangm@$X\ang m$}
\end{notation}

\begin{nul}
	If $ \{u_i\}_{i \in I} $ is a Lie algebra basis for $ \g $, then $ \{u_i\ang{m} \}_{(i,m) \in I \cross \ZZ} $ is a basis for $ \Lg $.
\end{nul}

\begin{remark}\label{remark:prodloopalg}
	The loop algebra functor $ \Lup \colon \LieAlg_{\CC} \to \LieAlg_{\CC} $ preserves finite products.
\end{remark}

\begin{recollection}
	A finite dimensional Lie algebra $ \g $ is \textit{simple} if $ \g $ is not abelian and the only ideals of $ \g
	$ are $ \g $ and $ 0 $.\index[terminology]{simple Lie algebra}
\end{recollection}

\begin{theorem}[{(Garland \cite[\S\S1 \& 2]{MR601519})}]
	If $ \g $ is a simple Lie algebra over $ \CC $, then
	\begin{equation*}
		\HLie^2(\Lg;\CC) \isomorphic \CC \period 
	\end{equation*}
\end{theorem}

In particular, if $ \g $ is simple there is a $ 1 $-dimensional space of central extensions of $ \Lg $.

%-------------------------------------------------------------------%
%  Recollection on bilinear forms & semisimplicity                  %
%-------------------------------------------------------------------%

\subsubsection{Recollection on bilinear forms \& semisimplicity}\label{subsec:recbilin}

\begin{notation}
	Let $ \g $ be a complex Lie algebra.
	We write $ \ad \colon \fromto{\g}{\End_{\CC}(\g)} $ for the adjoint representation, defined
	by\index[terminology]{adjoint representation!for Lie algebras}
	\begin{equation*}
		\ad(X) \colonequals [X,-] \period
	\end{equation*}
\end{notation}

\begin{example}
	A Lie algebra $ \g $ is abelian if and only if the adjoint representation of $ \g $ is trivial.
\end{example}

\begin{recollection}[(Killing form)]
	Let $ \g $ be a finite-dimensional Lie algebra.
	The \textit{Killing form} on $ \g $ is the bilinear form\index[terminology]{Killing form}
	\begin{align*}
		\Kil_{\g} \colon \g \cross \g &\to \CC \\
		(X,Y) &\mapsto \tr(\ad(X) \of \ad(Y)) \period
	\end{align*}
	The Killing form is symmetric and \textit{invariant} in the sense that
	\begin{equation*}
		\Kil_{\g}([X,Y],Z) = \Kil_{\g}(X,[Y,Z])
	\end{equation*}
	for all $ X,Y,Z \in \g $.
\end{recollection}

\begin{example}
	If $ \g $ is a simple Lie algebra, then every invariant symmetric bilinear form on $ \g $ is a $ \CC $-multiple of the Killing form $ \Kil_{\g} $.
	See \cite{MSE:298287} for a nice exposition of this fact. It is also related to Chern--Weil theory, which tells
	us that the space of invariant symmetric bilinear forms is isomorphic to $\H^4(\BG;\RR)$, and when $G$ is a
	compact, simple, simply connected Lie group, $\H^4(\BG;\RR)\cong\RR$. This is because $\H^4(\BG;\ZZ)\cong\ZZ$,
	which we have discussed and used in previous chapters.
\end{example}

\begin{example}
	Let $ \afrak $ be a finite-dimensional abelian Lie algebra over $ \CC $.
	Since the adjoint representation of $ \afrak $ is trivial, the Killing form of $ \afrak $ is identically zero.
	Also note that every bilinear form on the underlying vector space of $ \afrak $ is an invariant bilinear form on $ \afrak $. 
\end{example}

\begin{proposition}[{\cite[Chapter II, Theorems 2 \& 4]{MR1808366}}]\label{prop:semisimplicity}
	Let $ \g $ be a finite dimensional complex Lie algebra.
	The following conditions are equivalent:
	\begin{enumerate}[{\upshape (\ref*{prop:semisimplicity}.1)}]	
		\item\label{prop:semisimplicity.0} The center of $ \g $ is trivial.

		\item\label{prop:semisimplicity.1} The only abelian ideal in $ \g $ is $ 0 $.

		\item\label{prop:semisimplicity.2} The Lie algebra $ \g $ is isomorphic to a product of simple Lie algebras.

		\item\label{prop:semisimplicity.3} \emph{Cartan--Killing criterion:} the Killing form of $ \g $ is
		nondegenerate.\index[terminology]{Cartan--Killing criterion}
	\end{enumerate}
\end{proposition}

\begin{definition}
	Let $ \g $ be a finite-dimensional complex Lie algebra.
	If the equivalent conditions~\enumref{prop:semisimplicity}{0}--\enumref{prop:semisimplicity}{3} are satisfied,
	we say that $ \g $ is \textit{semisimple}.\index[terminology]{semisimple!Lie algebra}
\end{definition}

%-------------------------------------------------------------------%
%  Kac–Moody algebras                                               %
%-------------------------------------------------------------------%

\subsubsection{Kac--Moody algebras}\label{subsec:KacMoody}

Now we define the Lie algebra analogue of the central extensions $ \LGtilde $ of the loop group $ \LG $ that we
studied in \cref{loop_groups}. Those central extensions were parametrized by an element of $\H^4(\BG;\ZZ)$, and
these similarly require the additional data of an invariant symmetric bilinear form on $ \g $, i.e.\ an element of
$\H^4(\BG;\RR)$. The Killing form provides a canonical choice. The forms not in the image of
$\H^4(\BG;\ZZ)\to\H^4(\BG; \RR)$ correspond to loop algebra central extensions which do not lift to loop groups.

\begin{definition}[\cite{Kac68, Moo68}]
	Let $ \g $ be a Lie algebra over $ \CC $ with invariant symmetric bilinear form $ B \colon \fromto{\g \cross \g}{\CC} $.
	The \textit{Kac--Moody algebra} of $ \g $ with respect to the form $ B $ is the central extension
	\index[terminology]{Kac--Moody algebra}
	\begin{equation*}
		\begin{tikzcd}[sep=1.5em]
			1 \arrow[r] & \CC c \arrow[r] & \Lgtilde{B} \arrow[r] & \Lg \arrow[r] & 1
		\end{tikzcd}
	\end{equation*}
	with central element $ c $ and with Lie bracket extended from the relation
	\begin{align*}
		[X\ang{m},Y\ang{n}]_{\Lgtilde{B}} &\colonequals [X\ang{m},Y\ang{n}]_{\Lg} + \delta_{m,-n} m B(X,Y) c \\ 
		&= [X,Y]_{\g}\ang{m+n} + \delta_{m,-n} m B(X,Y) c
	\end{align*} 
	for all $ X,Y \in \g $.
\end{definition}

\begin{nul}
	If $ \{u_i\}_{i \in I} $ is a Lie algebra basis for $ \g $, then $ \{u_i\ang{m} \}_{(i,m) \in I \cross \ZZ} \union \{c\} $ is a basis for $ \Lgtilde{B} $.
\end{nul}

\begin{remark}
	The Kac--Moody algebra $ \Lgtildeno $ is usually denoted by $ \hat{\g} $ and is also known as the
	\textit{affine Lie algebra} of $ \g $.\index[terminology]{affine Lie algebra}
\end{remark}

\begin{remark}\label{remark:prodKacMoodyalg}
	Let $ \g_1 $ and $ \g_2 $ be complex Lie algebras equipped with invariant symmetric bilinear forms
	\begin{equation*}
		B_1 \colon \fromto{\g_1 \cross \g_1}{\CC} \andeq B_2 \colon \fromto{\g_2 \cross \g_2}{\CC} \period
	\end{equation*}
	Write $ B $ for the bilinear form on the product Lie algebra $ \g_1 \cross \g_2 $ defined by
	\begin{equation*}
		B((x_1,x_2),(y_1,y_2)) \colonequals B_1(x_1,y_1) + B(x_2,y_2) \period
	\end{equation*}
	Then we have a canonical isomorphism
	\begin{equation*}
		\widetilde{\Lup}_{B}(\g_1 \cross \g_2) \isomorphic \Lgtilde{B_1}_1 \cross \Lgtilde{B_2}_2 \period
	\end{equation*}
\end{remark}

%-------------------------------------------------------------------%
%-------------------------------------------------------------------%
%  The Segal–Sugawara construction                                  %
%-------------------------------------------------------------------%
%-------------------------------------------------------------------%

\subsection{The Segal--Sugawara construction}\label{sec:SegalSugawara}

We now have enough of of the background on Lie algebras to give a vague statement of the Segal--Sugawara construction.

\begin{definition}\label{def:posenergy}
	Let $ \g $ be a Lie algebra over $ \CC $ and $ B $ an invariant symmetric bilinear form on $ \g $.
	A representation $ \rho \colon \fromto{\Lgtilde{B}}{\End_{\CC}(V)} $ has \textit{positive energy} if for all $
	v \in V $ and $ X \in \g $ there exists an integer $ m > 0 $ such that\index[terminology]{positive energy!for
	Kac--Moody algebra representations}
	\begin{equation*}
		\rho(X\ang{m})v = 0 \period
	\end{equation*}
\end{definition}

\begin{remark}
	In the theory of Kac--Moody algebras, positive energy representations are more often called \textit{admissible}.
	We have chosen the term ``positive energy'' to align with the loop group
	terminology.\index[terminology]{admissible representation}
\end{remark}
Compare with the loop groups analogue, \Cref{pos_en}.

\begin{theorem}[(Segal--Sugawara construction, vague formulation)]\label{thm:SegalSugawaravague}
	Let $ \g $ be an abelian or simple Lie algebra over $ \CC $ and let $ B \colon \fromto{\g \cross \g}{\CC} $ be
	a nondegenerate invariant symmetric bilinear form on $ \g $.\index[terminology]{Segal--Sugawara
	construction!vague formulation}
	Write $ \Cas_B(\g) \in \Univ(\g) $ for the Casimir element of $ \g $ with respect to the bilinear form $ B
	$.\index[terminology]{Casimir element}\index[notation]{CasBg@$\Cas_B$}
	Let
	\begin{equation*}
		\rho \colon \fromto{\Lgtilde{B}}{\End_{\CC}(V)}
	\end{equation*} 
	be a positive energy representation of $ \Lgtilde{B} $ such that
	\begin{enumerate}[{\upshape (\ref*{thm:SegalSugawaravague}.1)}]
		\item the central element $ c \in \Lgtilde{B} $ acts by multiplication by a complex number $ \ell $,

		\item and the complex number $ -\ell $ is not equal to
		\begin{equation*}
			\lambda_B(G) \colonequals \frac{\tr(\ad(\Cas_{B}(\g)))}{2\dim(\g)} \period
		\end{equation*}
	\end{enumerate}
	Then there is an explicit action of the Virasoro algebra on $ V $ where the central charge $ \chg \in \virCC $
	acts by multiplication by\index[terminology]{Virasoro algebra}
	\begin{equation*}
		\frac{\ell \dim(\g)}{\ell + \lambda_B(\g)} \period
	\end{equation*}

	As special cases:
	\begin{enumerate}[{\upshape (\ref*{thm:SegalSugawaravague}.1)}]
		\setcounter{enumi}{2}
		\item If $ \g $ is abelian, then $ \lambda_B(\g) = 0 $ for any nondegenerate invariant symmetric bilinear form $ B $, and the central charge $ \chg \in \virCC $ acts by multiplication by $ \dim(\g) $.

		\item If $ \g $ is simple and $ B $ is the normalization of the Killing form such that the long roots of $
		\g $ have square length $ 2 $, then $ \lambda_B(\g) $ is a positive integer known as the \emph{dual Coxeter
		number} of $ \g $.\index[terminology]{dual Coxeter number}
	\end{enumerate}
\end{theorem}

\begin{nul}
	The complex number $ \ell $ in \Cref{thm:SegalSugawaravague} is known as the \textit{level} of the positive energy representation $ \rho $.
\end{nul}

\begin{goal}
	The goal for the rest of the talk is to explain this construction, the Casimir element $ \Cas_B(\g) $, and give a better description of the normalized trace $ \lambda_B(\g) $ as an eigenvalue of $ \ad(\Cas_B(\g)) $.
\end{goal}

%-------------------------------------------------------------------%
%  Motivating case: the Heisenberg algebra                          %
%-------------------------------------------------------------------%

\subsubsection{Motivating case: the Heisenberg algebra}\label{subsec:Heisenberg}

As motivation for the Segal--Sugawara construction, we start with the most simple case, where $ \g $ is the $ 1 $-dimensional abelian Lie algebra.
Since the constant $ \lambda_B(\g) $ will be zero in this case, we can do this without yet introducing the Casimir element.

\begin{definition}
	The \textit{Heisenberg algebra}\index[terminology]{Heisenberg!algebra}\index[terminology]{Lie algebra!Heisenberg} is the Kac--Moody algebra
	\begin{equation*}
		\heis \colonequals \widetilde{\Lup}\CC \index[notation]{Heis@$\heis$}
	\end{equation*}
	of the $ 1 $-dimensional abelian Lie algebra $ \CC $ with respect to the bilinear form $ \fromto{\CC \cross \CC}{\CC} $ given by multiplication.
\end{definition}

\begin{nul}
	Write $ u \in \CC $ for the element $ 1 $, which we regard as a basis for $ \CC $ as a $ 1 $-dimensional abelian Lie algebra.
	Then the Heisenberg algebra has generators $ \{c\} \union \{u\ang{m}\}_{m \in \ZZ} $, where $ c $ is central and the nontrivial bracket relation is given by
	\begin{equation*}
		[u\ang{m},u\ang{n}] \colonequals \delta_{m,-n}m c \period
	\end{equation*}
\end{nul}

\begin{definition}
	Let $ \mu, \hbar \in \CC $.
	Write $ u \in \CC $ for the element $ 1 $, which we regard as a basis for $ \CC $ as a $ 1 $-dimensional abelian Lie algebra.
	The \textit{Fock representation} $ \Fock(\mu,\hbar) $ is the representation of the Heisenberg algebra on the
	polynomial ring\index[terminology]{Fock representation}
	\begin{equation*}
		\Fock(\mu,\hbar) \colonequals \CC[x_1,x_2,\ldots]
	\end{equation*}
	in infinitely many variables, where
	\begin{align*}
		c &\mapsto \hbar \id{} \\ 
		u\ang{n} &\mapsto \begin{cases}
			\frac{\partial}{\partial x_n} \comma & n > 0 \\
			-\hbar x_{-n} \comma & n < 0 \\ 
			\mu \id{} \comma & n = 0 \period
		\end{cases} \\
	\end{align*}
\end{definition}

The following fact about the irreducibility of Fock representations is easy:

\begin{lemma}[{\cite[Lemma 2.1]{MR1021978}}]
	Let $ \mu,\hbar \in \CC $.
	If $ \hbar \neq 0 $, then the $ \heis $-representation $ \Fock(\mu,\hbar) $ is irreducible.
\end{lemma}

\begin{nul}
	If $ \hbar = 0 $, then the constants $ \CC \subset \Fock(\mu,0) $ are invariant.
\end{nul}

\begin{properties}\label{properties:Fock}
	The following are some important properties of the Fock representations of the Heisenberg algebra.
	\begin{enumerate}[(\ref*{properties:Fock}.1)]
		\item\label{properties:Fock.1} The elements $ u(0) $ and $ c $ of $ \heis $ act by multiplication.

		\item\label{properties:Fock.2} For every polynomial $ p \in \Fock(\mu,\hbar) $, there exists an integer $ n \gg 0 $ such that $ u\ang{n} p = 0 $: let $ n $ be any positive such that the variable $ x_n $ does not appear in $ p $.
		That is, the Fock representation $ \Fock(\mu,\hbar) $ is ``positive energy'' in the sense of \Cref{def:posenergy}.

		\item\label{properties:Fock.3} For each integer $ n > 0 $, the element $ u\ang{n} \in \heis $ acts locally nilpotently on $ \Fock(\mu,\hbar) $.
	\end{enumerate}
\end{properties}

Now we can give the Segal--Sugawara construction for the Fock representations of the Heisenberg algebra.

\begin{construction}[(Virasoro action of Fock representations)]
	For each integer $ m \in \ZZ $, define an infinite sum of elements of $ \Univ(\heis) $ by
	\begin{equation*}
		L_m^S \colonequals \frac{1}{2} \sum_{j \in \ZZ} \normord{u\ang{-j} u\ang{j+m}} \period
	\end{equation*}
	Here, $ \normord{u\ang{-j} u\ang{j+m}} $ denotes the \textit{normal ordering} on $ u\ang{-j} u\ang{j+m} $,
	defined by\index[terminology]{normal ordering}\index[notation]{uangj@$\normord{u\ang{-j}u\ang{j+m}}$}
	\begin{equation*}
		\normord{u\ang{-j} u\ang{j+m}} \colonequals \begin{cases}
			u\ang{-j} u\ang{j+m} \comma & -j \leq j + m \\ 
			u\ang{j+m} u\ang{-j} \comma & -j \geq j + m \period
		\end{cases}
	\end{equation*}
	Explicitly,
	\begin{equation*}
		L_m^S = \begin{cases}
			\displaystyle \frac{1}{2} u\ang{n}^2 +  \sum_{j > 0 } u\ang{n - j} u\ang{n + j} \comma & m = 2n \\ 
			\displaystyle \sum_{j > 0} u\ang{n + 1 - j} u\ang{n + j} \comma & m = 2n + 1 \period
		\end{cases}
	\end{equation*}
	
	The operators $ L_m^S $ are not well-defined elements of $ \Univ(\heis) $, but since the Fock representations of $ \heis $ are positive energy \enumref{properties:Fock}{2}, the operators $ L_m^S $ make sense as operators on $ \Fock(\mu,\hbar) $.
\end{construction}

\begin{theorem}[{(Segal--Sugawara for $ \Fock(\mu,1) $ \cite[Proposition 2.3]{MR1021978})}]
	Under the representation of $ \heis $ on the Fock space $ \Fock(\mu,1) $, the operators $ L_{m}^S $ on $ \Fock(\mu,1) $ satisfy the commutation relation
	\begin{equation*}
		[L_m^S,L_n^S] = (m - n)L_{m+n}^S + \delta_{m,-n} \frac{m^3 - m}{12} \period
	\end{equation*}
	Hence the assignment
	\begin{align*}
		\virCC &\to \End_{\CC}(\Fock(\mu,1)) \\ 
		L_m &\mapsto L_{m}^S \\ 
		\chg &\mapsto \id{}
	\end{align*}
	is a $ \virCC $-representation with central charge $ 1 $.
\end{theorem}

\begin{remark}
	To derive the Segal--Sugawara action on $ \Fock(\mu,\hbar) $ for $ \hbar \neq 0 $, let $ L_{m} $ act by $ \frac{1}{\hbar} L_{m}^S $.
\end{remark}

\begin{remark}
	Gordon's notes \cite{Gordon:InfiniteLie} give a nice exposition of the Segal--Sugawara construction for Fock representations and the representation theory of the Virasoro algebra.
\end{remark}

%-------------------------------------------------------------------%
%  The Casimir element                                              %
%-------------------------------------------------------------------%

\subsubsection{The Casimir element}\label{subsec:Casimir}

In the general case, the idea is to try to mimic the formulas that we wrote down defining the operators on the Fock representations that satisfy the Virasoro relations.
First, we need to explain the ``Casimir element'' and normalized trace $ \lambda_B(\g) $ appearing in \Cref{thm:SegalSugawaravague}.

\begin{definition}\label{rec:Casimir}
	Let $ \g $ be a finite-dimensional Lie algebra over $ \CC $ and let $ B $ be a nondegenerate invariant symmetric bilinear form on $ \g $.
	The \textit{Casimir element} $ \Cas_{B}(\g) $ of $ \g $ with respect to the form $ B $ is the element of the
	universal enveloping algebra $ \Univ(\g) $ given by the image of $ \id{\g} $ under the
	composite\index[terminology]{Casimir element}\index[notation]{Casb@$\Cas_B$}
	\begin{equation*}
		\begin{tikzcd}[sep=1.5em]
			\End_{\CC}(\g) \isomorphic \g \tensor_{\CC} \gdual \arrow[r, "\sim"{yshift=-0.25em}] & \g \tensor_{\CC} \g \arrow[r, hook] & \Tup_{\CC}(\g) \arrow[r, ->>] & \Univ(\g) \period
		\end{tikzcd}
	\end{equation*}
	Here the isomorphism $ \isomto{\g \tensor_{\CC} \gdual}{\g \tensor_{\CC} \g} $ is the identity on the first factor and the isomorphism $ \isomto{\gdual}{\g} $ induced by the form $ B $ on the second factor, and $ \Tup_{\CC}(\g) $ is the tensor algebra of $ \g $ over $ \CC $.
\end{definition}
The following are some key properties that we need to know about the Casimir element:
	\begin{enumerate}[(\ref*{rec:Casimir}.1)]
		\item\label{rec:Casimir.1} The Casimir element $ \Cas_{B}(\g) $ is a central element of $ \Univ(\g) $.

		\item\label{rec:Casimir.2} If $ \{u_1,\ldots,u_d\} $ and $ \{u^1,\ldots,u^d\} $ are bases of $ \g $ that
		are dual with respect to the bilinear form $ B $ in the sense that $B(u_i,u^j) = \delta_{i,j}$, then
		\[\Cas_{B}(\g) = \sum_{i=1}^d u_i u^i\period\]
		\item\label{rec:Casimir.3} Assume that $ \g $ is simple. 
		Then the Casimir element of the Killing form of $ \g $ acts by the identity in the adjoint representation. 
		Hence for any nondegenerate invariant symmetric bilinear form $ B $ on $ \g $, the Casimir element $ \Cas_{B}(\g) $ acts by scalar multiplication in the adjoint representation of $ \g $.
		If $ B $ is the normalization of the Killing form on $ \g $ such that long roots have square length $ 2 $, then in the adjoint representation $ \Cas_{B}(\g) $ acts by multiplication by an even positive integer.

		\item\label{rec:Casimir.4} If $ \g $ is abelian, then since the adjoint representation of $ \g $ is trivial, for any nondegenerate invariant symmetric bilinear form $ B $ on $ \g $ we have $ \ad(\Cas_{B}(\g)) = 0 $.
		In particular, in the adjoint representation $ \Cas_{B}(\g) $ acts by scalar multiplication. 
	\end{enumerate}

Even though there are no Lie algebras that are both abelian and simple, it is important for us that both types of Lie algebras have the property that the Casimir element associated to any nondegenerate invariant symmetric bilinear form acts by scalar multiplication in the adjoint representation.
In particular, if $ \g $ is abelian or simple, then $ \ad(\Cas_{B}(\g)) $ only has exactly one eigenvalue.

\begin{definition}
	Let $ \g $ be a finite dimensional abelian or simple Lie algebra over $ \CC $ and let $ B \colon \fromto{\g \cross \g}{\CC} $ be a nondegenerate invariant symmetric bilinear form on $ \g $.
	Define a complex number $ \lambda_B(\g) $ by 
	\begin{equation*}
		\lambda_B(\g) \colonequals \frac{1}{2} \Big(\text{eigenvalue of } \ad(\Cas_B(\g)) \Big) \period
	\end{equation*}
\end{definition}

\begin{nul}
	If $ \dim(\g) > 0 $, then 
	\begin{equation*}
		\lambda_B(\g) = \frac{\tr(\ad(\Cas_{B}(\g)))}{2\dim(\g)} \comma
	\end{equation*}
	which aligns with the vague formulation of the Segal--Sugawara construction (\Cref{thm:SegalSugawaravague}).
\end{nul}

\begin{example}
	If $ \g $ is simple and $ B $ is the normalization of the Killing form on $ \g $ such that long roots have square length $ 2 $, then $ \lambda_B(\g) $ is a positive integer \enumref{rec:Casimir}{3} known as the \textit{dual Coxeter number} of $ \g $.
\end{example}

\begin{example}
	If $ \afrak $ is an abelian Lie algebra, then for any nondegenerate invariant symmetric bilinear form $ B $ on $ \afrak $, we have $ \lambda_{B}(\afrak) = 0 $.
\end{example}


%-------------------------------------------------------------------%
%  The general case                                                 %
%-------------------------------------------------------------------%

\subsubsection{The general case}\label{subsec:SSgeneral}

Now let us try using ``the same'' formula to write down a Virasoro action on positive energy representations of $ \Lgtildeno $ as we did for the Heisenberg algebra.
The first modification is that we need to sum over a basis of $ \g $.

\begin{construction}
	Let $ \g $ be a finite-dimensional Lie algebra over $ \CC $ and let $ B $ be a nondegenerate invariant symmetric bilinear form on $ \g $.
	Given a positive energy representation $ \rho \colon \fromto{\Lgtilde{B}}{\End_{\CC}(V)} $, for each integer $ m \in \ZZ $ define
	\begin{equation*}
		T_m^{\rho} \colonequals \frac{1}{2} \sum_{i = 1}^d \sum_{j \in \ZZ} \normord{\rho(u_i\ang{-j}) \rho(u^i\ang{j+m})} \, \in \End_{\CC}(V) \period
	\end{equation*}
	Note that even though the formula defining $ T_m^{\rho} $ involves an infinite sum, since $ \rho $ is a positive energy representation, for each $ v \in V $, all but finitely many terms in the sum defining $ T_m^{\rho} $ annihilate $ v $.
	Hence $ T_m^{\rho} $ is well-defined as an element of $ \End_{\CC}(V) $. 
\end{construction}

We used the letter ``$ T $'' instead of ``$ L $'' because the commutation relation is not quite right:

\begin{lemma}[{\cite[Theorem 10.1]{MR1021978}}]
	Let $ \g $ be a finite dimensional abelian or simple Lie algebra over $ \CC $ and let $ B $ be a nondegenerate invariant symmetric bilinear form on $ \g $.
	For every positive energy representation $ \rho \colon \fromto{\Lgtilde{B}}{\End_{\CC}(V)} $, we have the following commutation relation in $ \End_{\CC}(V) $:
	\begin{align*}
		[T_m^{\rho},T_n^{\rho}] &= (\rho(c) + \lambda_B(\g))(m - n) T_{m+n}^{\rho} \\ 
		&\phantom{=} \qquad + \delta_{m,-n} \dim(\g) \frac{m^3 - m}{12} \rho(c) (\rho(c) + \lambda_B(\g))  \period 
	\end{align*}
\end{lemma}

\begin{idea}
	The naïve guess that the operators $ T_m^{\rho} $ satisfy the Virasoro relations is not correct. 
	However, if we could invert $ \rho(c) + \lambda_B(\g) $, then the operators
	\begin{equation*}
		\frac{1}{\rho(c) + \lambda_B(\g)} T_m^{\rho} 
	\end{equation*}
	would satisfy the Virasoro relations.
	We can do this provided that the central element $ c \in \Lgtilde{B} $ acts by a scalar $ \ell $ on $ V $, and $ \ell \neq -\lambda_B(\g) $.
\end{idea}

\begin{theorem}[{(Segal--Sugawara construction \cite[Corollary 10.1]{MR1021978})}]\label{thm:SegalSugawara}
	Let $ \g $ be a finite dimensional abelian or simple Lie algebra over $ \CC $ and let $ B \colon \fromto{\g
	\cross \g}{\CC} $ be a nondegenerate invariant symmetric bilinear form.\index[terminology]{Segal--Sugawara
	construction}
	Let
	\begin{equation*}
		\rho \colon \fromto{\Lgtilde{B}}{\End_{\CC}(V)}
	\end{equation*} 
	be a positive energy representation of $ \Lgtilde{B} $ such that
	\begin{enumerate}[{\upshape (\ref*{thm:SegalSugawara}.1)}]
		\item the central element $ c \in \Lgtilde{B} $ acts by multiplication by a complex number $ \ell $,

		\item and $ \ell \neq - \lambda_B(\g) $.  
	\end{enumerate}
	Choose bases $ \{u_1,\ldots,u_d\} $ and $ \{u^1,\ldots,u^d\} $ of $ \g $ that are dual with respect to the bilinear form $ B $.

	Then the assignment
	\begin{equation*}
		L_m \mapsto L_m^{\rho} \colonequals \frac{1}{2(\ell + \lambda_B(\g))} \sum_{i=1}^d \sum_{j \in \ZZ} \normord{\rho(u_i\ang{-j}) \rho(u^i\ang{j+m})}
	\end{equation*}
	extends to a $ \virCC $-representation on $ V $ with central charge
	\begin{equation*}
		\frac{\ell \dim(\g)}{\ell + \lambda_B(\g)} \period
	\end{equation*}
	That is, in $ \End_{\CC}(V) $, the operators $ L_m^{\rho} $ satisfy the commutation relation
	\begin{equation*}
		[L_m^{\rho},L_n^{\rho}] = (m - n) L_{m+n}^{\rho} + \delta_{m,-n} \frac{m^3 - m}{12} \frac{\ell \dim(\g)}{\ell + \lambda_B(\g)} \period 
	\end{equation*}
\end{theorem}

\begin{remark}
	For $ a,b \in \ZZ $, the sum $ \sum_{i=1}^d u_i(a)u^i(b) $ is independent of the choice of basis $ \{u_1,\ldots,u_d\} $ of $ \g $.
	In particular, the operators $ L_m^{\rho} $ are independent of the choice of basis.
\end{remark}

\begin{remark}
	If $ \ell = -\lambda_{B}(\g) $, then the formulas we wrote down for the Segal--Sugawara operators $ L_m^{\rho}
	$ do not make sense, and there is a fundamental difficulty in dealing with the ``critical level'' $ \ell =
	-\lambda_{B}(\g) $.\index[terminology]{critical level}
	At the critical level, the theory seems to resemble the positive characteristic situation rather than the classical one; see \cite{MO:25592} for some discussion of this point. 
\end{remark}

\begin{remark}
	In light of \Cref{remark:prodKacMoodyalg}, the Segal--Sugawara construction can be extended to the case where $ \g $ is \textit{reductive}, i.e., $ \g $ decomposes as a product
	\begin{equation*}
		\g \isomorphic \afrak \cross \g_1 \cross \cdots \cross \g_r \comma
	\end{equation*}
	where $ \afrak $ is an abelian Lie algebra and $ \g_1,\ldots,\g_r $ are simple Lie algebras.
	In this case, the central charge of the resulting $ \virCC $-representation is 
	\begin{equation*}
		\dim(\afrak) + \sum_{i=1}^r \frac{\ell_i \dim(\g_i)}{\ell_i + \lambda_{B_i}(\g_i)} \period
	\end{equation*}
	Here the central element of $ \widetilde{\Lup}\afrak $ acts by multiplication by a nonzero complex number and the central element of each $ \Lgtildeno_i $ acts by multiplication by $ \ell_i \in \CC \smallsetminus \{ - \lambda_{B_i}(\g_i) \} $. 
	This is rather useful as all of the classical Lie algebras are reductive \cite[Theorem 5.49]{MR2440737}; see \cite[Remark 10.3]{MR1021978} for details.
\end{remark}

\begin{remark}
	The Segal--Sugawara construction is usually stated with the assumptions that $ \g $ is simple and $ B $ is the normalization of the Killing form such that the long roots of $ \g $ have square length $ 2 $ (so that $ \lambda_{B}(\g) $ is the \textit{dual Coxeter number}, often denoted by $ h^{\vee} $).
	This is somewhat unfortunate; because the Killing form of an abelian Lie algebra is trivial, to include the
	abelian case (and the reductive extension) the ``usual'' statement needs to be modified to include arbitrary nondegenerate invariant symmetric bilinear forms as in \Cref{thm:SegalSugawaravague}.
\end{remark}

\begin{remark}
	One of the motivations for the formula for the Segal--Sugawara operators $ L_m^{\rho} $ comes from the theory of vertex algebras.
	See \cite[\S3]{MR2079371}, in particular \cite[Proposition 3.3.1]{MR2079371}, for more details on the relation
	to vertex algebras.\index[terminology]{vertex algebra}
\end{remark}
