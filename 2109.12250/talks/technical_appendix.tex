%!TEX root = ../diffcoh.tex

%-------------------------------------------------------------------%
%-------------------------------------------------------------------%
%  Appendix: technical details from topos theory                    %
%-------------------------------------------------------------------%
%-------------------------------------------------------------------%

\section{Technical details from topos theory}\label{app:technicaldeatails}

The purpose of this appendix is to prove a number of technical results used throughout the text.
We have relegated these proofs to this appendix because of one of the following reason:
\begin{enumerate}[(1)]
	\item They are lengthy and, while the result is important, the proof is not important to know.

	\item They require some knowledge from the theory of \topoi.
\end{enumerate}
In \cref{sec:Cvaluedsheaves}, we explain a formal procedure to get from sheaves of spaces to sheaves valued in another presentable \category $ C $.
This lets us deduce many results about sheaves on $ \Mfld $ valued in a general presentable \category $ C $ from the case $ C = \Spc $.
\Cref{sec:restriction} explains the important properties of the functor given by restricting a sheaf defined on $ \Mfld $ to a sheaf defined on only a single manifold.
\Cref{sec:sheafification} explains why this restriction procedure commutes with sheafification.
In \cref{sec:completeness}, we give some background on notions of ``completeness'' for \topoi.
\Cref{subsec:points} shows that equivalences in $ \Sh(\Man;\Spc) $ can be checked on stalks and uses this to show that $ \Sh(\Mfld;\Spc) $ satisfies the strongest of these completeness notions (\Cref{app.prop:postnikovcompleteness}).
This also implies that $ \Sh(\Mfld;C) $ is equivalent to the category of $ C $-valued sheaves on the subcategory $ \Euc \subset \Man $ spanned by the Euclidean spaces (\Cref{app.lem:hypersheavesonCart}).
We complete the section by using the fact that sheafification and restriction to a manifold commute to show that the sheafification of an $ \RR $-invariant presheaf is again $ \RR $-invariant (\cref{sec:sheafificationpreservesinvariance}).

Since we are mostly interested in sheaves of spaces in this appendix, we adopt the following notational convention.

\begin{notation}
	We write $ \Sh(\Man) \colonequals \Sh(\Man;\Spc) $ for the \topos of sheaves of spaces on $ \Man $.
\end{notation}

\begin{remark}
	For this appendix, it is sufficient to know that the \category of sheaves of spaces on a site is \atopos, and that a \textit{geometric morphism} of \topoi is a right adjoint functor $ \flowerstar \colon \fromto{\X}{\Y} $ whose left adjoint $ \fupperstar $ is left exact.
\end{remark}

%-------------------------------------------------------------------%
%  From sheaves of spaces to C-valued sheaves                       %
%-------------------------------------------------------------------%

\subsection{From sheaves of spaces to \texorpdfstring{$ C $}{C}-valued sheaves}\label{sec:Cvaluedsheaves}

Let $ C $ be a presentable \category.
In this section we explain a formal procedure that allows us to pass from the \category $ \Sh(\Man;\Spc) $ of sheaves of spaces on $ \Man $ to the \category $ \Sh(\Man;C) $ of $ C $-valued sheaves on $ \Man $.
We'll also recall the basics of tensor products of presentable \categories and explain how to describe $ \Sh(\Man;C) $ as the tensor product
\begin{equation*}
	\Sh(\Man;C) \equivalent \Sh(\Man;\Spc) \tensor C \period
\end{equation*} 

The first thing to observe is that if $ G \colon \fromto{\Sh(\Man;\Spc)^{\op}}{C} $ is a functor that preserves limits, then the restriction $ G \colon \fromto{\Manop}{C} $ is a sheaf. 
It turns out that all $ C $-valued sheaves arise in this way.

\begin{proposition}[{\SAG{Proposition}{1.3.1.7}}]
	Let $ (S,\tau) $ be \asite and $ C $ an \category with all limits.
	Write $ \yo_{\tau} \colon \fromto{S}{\Sh_{\tau}(S;\Spc)} $ for the $ \tau $-sheafification of the Yoneda embedding.
	Then pre-composition with $ \yo_{\tau} $ defines an equivalence
	\begin{equation*}
		\Funlim(\Sh_{\tau}(S;\Spc)^{\op},C) \equivalence \Sh_{\tau}(S;C) \period
	\end{equation*} 
\end{proposition}

Now we give the \category $ \Funlim(\Sh(\Man)^{\op},C) $ a description in terms of a universal property of presentable \categories.

\begin{recollection}[{\HA{Proposition}{4.8.1.17}}]
	Let $ C $ and $ D $ be presentable \categories.
	The \textit{tensor product} of presentable \categories $ C \tensor D $ along with the functor $ \tensor \colon \fromto{C \cross D}{C \tensor D} $ are characterized by the following universal property: for any presentable \category $ E $, restriction along $ \tensor $ defines an equivalence 
	\begin{equation*}
		\Fun^{\colim}(C \tensor D,E) \equivalence \Fun^{\colim,\colim}(C \cross D,E) \period
	\end{equation*}
	Here the right-hand side is the full subcategory of $ \Fun(C \cross D,E) $ spanned by those functors $ \fromto{C \cross D}{E} $ that preserve colimits separately in each variable.
	The tensor product of presentable \categories defines a functor
	\begin{equation*}
		\tensor \colon \fromto{\PrL \cross \PrL}{\PrL}
	\end{equation*}
	and can be used to equip $ \PrL $ with the structure of a symmetric monoidal \category.

	The tensor product $ C \tensor D $ admits the following useful (seemingly asymmetric) description:
	\begin{equation*}
		C \tensor D \equivalent \Fun^{\lim}(C^{\op},D) \period
	\end{equation*}
	If $ F \colon \fromto{D}{D'} $ is a right adjoint functor of presentable \categories, then the induced right adjoint
	\begin{equation*}
		\id{C} \tensor F \colon C \tensor D \equivalent \Fun^{\lim}(C^{\op},D) \to \Fun^{\lim}(C^{\op},D') \equivalent C \tensor D'
	\end{equation*}
	is given by post-composition with $ F $.
	Unfortunately, the left adjoint to $ \id{C} \tensor F $ does not generally admit a simple description.
	However, if $ C $ is compactly generated and the left adjoint to $ F $ is left exact, then the left adjoint to $ \id{C} \tensor F $ admits a simple description; see \cite[\S2.2]{arXiv:2108.03545}.
\end{recollection}

\begin{example}
	For any presentable \category $ C $, we have a natural equivalence
	\begin{equation*}
		\equivto{\Sh(\Man) \tensor C}{\Sh(\Man;C)} \period
	\end{equation*}
\end{example}
	
%-------------------------------------------------------------------%
%  Restriction to a manifold                                        %
%-------------------------------------------------------------------%

\subsection{Restriction to a manifold}\label{sec:restriction}

We now give an alternative description of the functor $ \fromto{\Sh(\Man;C)}{C} $ that sends a sheaf to its value on a manifold $ M $.

\begin{notation}
	Let $ T $ be a topological space and $ C $ a presentable \category.
	Write
	\begin{equation*}
		\PSh(T;C) \colonequals \Fun(\Open(T)^{\op},C)
	\end{equation*}
	and write $ \Sh(T;C) \subset \PSh(T;C) $ for the \category of $ C $-valued sheaves on $ T $.
	Write
	\begin{equation*}
		\Gamma_{T,*} \colon \fromto{\Sh(T;C)}{C}
	\end{equation*}
	for the \textit{global sections} functor, defined by $ \Gamma_{T,*}(F) \colonequals F(T) $, and write $ \Gammaupperstar_T \colon \fromto{C}{\Sh(T;C)} $ for the left adjoint to $ \Gamma_{T,*} $, i.e., the \textit{constant sheaf} functor.
\end{notation}

\begin{observation}\label{obs:restrictiontoamanifold}
	Let $ C $ be a presentable \category and $ M $ a manifold.
	The forgetful functor $ \fromto{\Open(M)}{\Man} $ preserves finite limits and is a morphism of sites.
	Moreover, the forgetful functor satisfies the \emph{covering lifting property} \cite[Definition A.12]{arXiv:1803.01804}.
	In particular: 
	\begin{enumerate}[(\ref*{obs:restrictiontoamanifold}.1)]
		\item The presheaf retriction functor $ \restrict{(-)}{M} \colon \fromto{\PSh(\Man;C)}{\PSh(M;C)} $ carries sheaves to sheaves.
		
		\item The functor $ \restrict{(-)}{M} \colon \fromto{\Sh(M;C)}{\Sh(\Man;C)} $ is both a left and right adjoint \cite[Proposition A.12]{arXiv:1803.01804}. 
	\end{enumerate}
\end{observation}

\begin{nul}
	Note that the functor given by sending a sheaf $ E $ on $ \Man $ to its value on $ M $ is given by the composite
	\begin{equation*}
		\begin{tikzcd}
			\Sh(\Man;C) \arrow[r, "\restrict{(-)}{M}"] & \Sh(M;C) \arrow[r, "{\Gamma_{M,*}}"] & C \period
		\end{tikzcd}
	\end{equation*} 
\end{nul}

\begin{nul}
	Moreover, if $ p \colon \fromto{N}{M} $ is a morphism in $ \Man $, then there is a canonical natural transformation fitting into the triangle 
	\begin{equation*}
		\begin{tikzcd}[column sep={12ex,between origins}, row sep={8ex,between origins}]
			\Sh(\Man;C) \arrow[rr, "\restrict{(-)}{M}"] \arrow[dr, "\restrict{(-)}{N}"'] & \phantom{\Sh(N;C)} & \Sh(M;C) \\
			& \Sh(N;C) \arrow[ur, "\plowerstar"'] \arrow[u, phantom, "\Longrightarrow"{sloped, near end}, "{\, \scriptstyle\can_p}"{xshift=1em, yshift=.25em}]
		\end{tikzcd}
	\end{equation*} 
	defined as follows: given a sheaf $ E $ on $ \Man $ and an open subset $ U \subset M $, the morphism
	\begin{equation*}
		\fromto{E(U)}{E(\pinverse(U))}
	\end{equation*}
	is induced by the projection $ \fromto{\pinverse(U)}{U} $ by the functoriality of $ E $.
	In particular, upon taking global sections, the morphism 
	\begin{equation*}
		\can_p \colon E(M) = \Gamma_{M,*}(\restrict{E}{M}) \to  \Gamma_{M,*}(\plowerstar(\restrict{E}{N})) = E(N)
	\end{equation*}
	is the morphism $ \fromto{E(M)}{E(N)} $ induced by $ p $ by the functoriality of $ E $.
\end{nul}

%-------------------------------------------------------------------%
%  Sheafification                                                   %
%-------------------------------------------------------------------%

\subsection{Sheafification}\label{sec:sheafification}

Next we show that restriction from $ \Sh(\Man;C) $ to $ \Sh(M;C) $ commutes with sheafification.

\begin{nul}
	Consider the commutative square
	\begin{equation*}
		\begin{tikzcd}
			\Sh(\Man;C) \arrow[d, "\restrict{(-)}{M}"'] \arrow[r, hooked] & \PSh(\Man;C) \arrow[d, "\restrict{(-)}{M}"] \\ 
			\Sh(M;C) \arrow[r, hooked] & \PSh(M;C) \period
		\end{tikzcd}
	\end{equation*} 
	Using the unit and counit of the sheafification-inclusion adjunctions for $ \Man $ and $ M $, one can define an \textit{exchange transformation}
	\begin{equation*}
		\Ex \colon \fromto{\Sup_{M} \of \restrict{(-)}{M}}{\restrict{(-)}{M} \of \SMan} \period
	\end{equation*}
	See \cites[\HAthm{Definition}{4.7.4.13}]{HA}[Definition 1.1]{arXiv:2108.03545}.
	The exchange morphism $ \Ex $ fits into a diagram
	\begin{equation*}
		\begin{tikzcd}
			\PSh(\Man;C) \arrow[d, "\restrict{(-)}{M}"'] \arrow[r, "\SMan"] & \Sh(\Man;C) \arrow[d, "\restrict{(-)}{M}"] \arrow[dl, phantom, "\scriptstyle \Ex" above left, "\Longrightarrow" sloped] \\ 
			\PSh(M;C) \arrow[r, "\Sup_{M}"'] & \Sh(M;C) \period
		\end{tikzcd}
	\end{equation*}
\end{nul}

\begin{lemma}\label{lem:rescommuteswithsheaf}
	Let $ C $ be a presentable \category and $ M $ a manifold.
	Then the exchange transformation $ \Ex \colon \fromto{\Sup_{M} \of \restrict{(-)}{M}}{\restrict{(-)}{M} \of \SMan} $ is an equivalence.
	That is, there is a commuative square of \categories
	\begin{equation*}
		\begin{tikzcd}
			\PSh(\Man;C) \arrow[d, "\restrict{(-)}{M}"'] \arrow[r, "\SMan"] & \Sh(\Man;C) \arrow[d, "\restrict{(-)}{M}"] \\ 
			\PSh(M;C) \arrow[r, "\Sup_{M}"'] & \Sh(M;C) \period
		\end{tikzcd}
	\end{equation*} 
\end{lemma}

\begin{proof}
	In the case $ C = \Spc $, the claim follows from the fact that the forgetful functor $ \fromto{\Open(M)}{\Man} $ satisfies the covering lifting property; see \cites[Proposition 7.1]{ClausenMathew:hyperdescent}[Proposition A.12]{arXiv:1803.01804}.
	The claim for general $ C $ follows from the claim for sheaves of spaces by applying the tensor product of presentable \categories and \cite[Lemma 1.18]{arXiv:2108.03545}.
\end{proof}

\begin{corollary}\label{cor:restrictionofconstantsheaf}
	Let $ C $ be a presentable \category, $ X \in C $, and $ M $ a manifold.
	Then we have a natural identification $ \restrict{\Gammaupperstar(X)}{M} = \Gammaupperstar_M(X) $ of the restriction of $ \Gammaupperstar(X) $ to $ M $ with the constant sheaf on $ M $ at $ X $.
\end{corollary}

\begin{proof}
	Note that by tensoring with the presentable \category $ C $, it suffices to prove the claim for $ C = \Spc $.
	In this case, note that by \Cref{lem:rescommuteswithsheaf} the functors
	\begin{equation*}
		\restrict{(-)}{M} \of \Gammaupperstar, \Gammaupperstar_M \colon \fromto{\Spc}{\Sh(M)}
	\end{equation*}
	are both left exact left adjoints.
	The claim follows from the fact that for \atopos $ \X $, the constant sheaf functor is the unique left exact left adjoint $ \fromto{\Spc}{\X} $ \HTT{Proposition}{6.3.4.1}.
\end{proof}

%-------------------------------------------------------------------%
%  Background on notions of completeness for higher topoi           %
%-------------------------------------------------------------------%

\subsection{Background on notions of completeness for higher topoi}\label{sec:completeness}

There are three notions of ``completeness'' for \atopos $ \X $:
\begin{enumerate}[(1)]
	\item \textit{Hypercompleteness:} Whitehead's Theorem holds in $ \X $.

	\item \textit{Convergence of Postnikov towers:} Every object of $ \X $ is the limit of its Postnikov tower.

	\item \textit{Postnikov completeness:} $ \X $ can be recovered as the limit $ \lim_{n} \X_{\leq n} $ of its subcategories $ \X_{\leq n} \subset \X $ of $ n $-truncated objects along the truncation functors $ \trun_{\leq n} \colon \fromto{\X_{\leq n+1}}{\X_{\leq n}} $.
\end{enumerate}
While all of these properties hold for the \topos $ \Spc $ of spaces, they need not hold for a general \topos.
We have implications (3) $ \Rightarrow $ (2) $ \Rightarrow $(1), and none of the implications are reversible in general.
In this section we give a brief overview of hypercompletness as it plays a role in relating the Freed--Hopkins approach to differential cohomology from \cite{MR3049871} to the \categorical approach we have taken here.
Detailed accounts of hypercompleteness and Postnikov completeness can be found in \cite[\HTTsec{6.5}]{HTT} and \cite[\SAGsec{A.7}]{SAG}, respectively.

\begin{definition}
	Let $ \X $ be \atopos.
	An object $ U \in \X $ is \textit{$ \infty $-connected} if for every integer $ n \geq -2 $ the $ n $-truncation $ \trun_{\leq n}(U) $ of $ U $ is the terminal object of $ \X $.
	A morphism $ f \colon \fromto{U}{V} $ is \textit{$ \infty $-connected} if $ f \colon \fromto{U}{V} $ is an $ \infty $-connected object of the \topos $ \X_{/V} $.
\end{definition}

\begin{definition}\label{def:hypercompleteness}
	Let $ \X $ be \atopos.
	An object $ U \in \X $ is \textit{hypercomplete} if $ U $ is local with respect to the class of $ \infty $-connected morphisms in $ \X $.
	We write $ \X^{\hyp} \subset \X $ for the full subcategory spanned by the hypercomplete objects of $ \X $.
	\Atopos is \textit{hypercomplete} if $ \X^{\hyp} = \X $.
\end{definition}

\begin{nul}
	The \category $ \X^{\hyp} \subset \X $ is a left exact localization of $ \X $, hence \atopos \cite[\HTTpage{699}]{HTT}.
	Moreover, the \topos $ \X^{\hyp} $ is hypercomplete \HTT{Lemma}{6.5.2.12}.
\end{nul}

\begin{nul}
	The \topos $ \X^{\hyp} $ is the universal hypercomplete \topos equipped with a geometric morphism to $ \X $ \HTT{Proposition}{6.5.2.13}.
	For this reason we call $ \X^{\hyp} $ the \textit{hypercompletion} of $ \X $.
\end{nul}

\begin{observation}\label{obs:hypercompletecons}
	Let $ \X $ be \atopos.
	Then $ \X $ is hypercomplete if and only if the pullback functor $ \pupperstar \colon \fromto{\X}{\X^{\post}} $ is conservative.
\end{observation}

The standard way of working with sheaves of spaces on a site $ (S,\tau) $ in the language of model-categories is to use the Brown--Joyal--Jardine model structure on simplicial presheaves \cites{MR341469}{MR906403}.
However, this model structure only presents the hypercompletion of the \topos of sheaves of spaces on $ (S,\tau) $.

\begin{proposition}[{\HTT{Proposition}{6.5.2.14}}]\label{app.prop:HTT.6.5.2.14}
	Let $ (S,\tau) $ be a site.
	Then the underlying \category of the category of simplicial presheaves on $ S $ in the Brown--Joyal--Jardine model structure is equivalent to the \topos $ \Sh_{\tau}(S;\Spc)^{\hyp} $ of hypercomplete sheaves of spaces on $ S $.
\end{proposition}

\begin{definition}
	Let $ \X $ be \atopos.
	A \textit{point} of $ \X $ is a left exact left adjoint $ \xupperstar \colon \fromto{\X}{\Spc} $.
	Given an object $ U \in \X $ and point $ \xupperstar $ of $ \X $, we call $ \xupperstar(U) $ the \textit{stalk} of $ U $ at $ \xupperstar $.
\end{definition}

\begin{example}
	Let $ T $ be a topological space and $ t \in T $. 
	Then the stalk functor
	\begin{equation*}
		(-)_t \colon \fromto{\Sh(T)}{\Spc}
	\end{equation*}
	defines a point of $ \Sh(T) $.
\end{example}

\begin{definition}
	\Atopos $ \X $ \textit{has enough points} if a morphism $ f $ in $ \X $ is an equivalence if and only if for every point $ \xupperstar $ of $ \X $, the stalk $ \xupperstar(f) $ is an equivalence in $ \Spc $.
\end{definition}

\begin{example}\label{exm:enoughpointsishypercomplete}
	\Atopos with enough points is hypercomplete.
\end{example}

\begin{remark}
	The existence of enough points is incomparable with the convergence of Postnikov towers and is also incomparable with Postnikov completeness (both of which imply hypercompleteness).
\end{remark}

\begin{example}\label{ex:ShMPostnikovcomplete}
	Let $ M $ be a manifold. 
	Then the \topos $ \Sh(M) $ is Postnikov complete \cite[\HTTthm{Proposition}{7.2.1.10} \& \HTTthm{Theorem}{7.2.3.6}]{HTT}.
\end{example}

%-------------------------------------------------------------------%
%  A conservative family of points                                  %
%-------------------------------------------------------------------%

\subsection{A conservative family of points}\label{subsec:points}

In this section we show the stalks at the origins in $ \RR^n $ for $ n \geq 0 $ form a conservative family of points for the \topos $ \Sh(\Man) $ (\Cref{app.prop:hypercompleteness}).
This implies that the model structure on simplicial presheaves on $ \Man $ considered by Freed--Hopkins in \cite[\S 5]{MR3049871} presents the \topos $ \Sh(\Man) $.
We also present an observation of Hoyois that shows that the \topos $ \Sh(\Man) $ is Postnikov complete (\Cref{app.prop:postnikovcompleteness}).

We begin by discussing the stalk of a sheaf on $ \Man $ at a point of a manifold.

\begin{construction}\label{cons:stalk}
	Let $ M $ be a manifold and $ x \in M $.
	In light of \Cref{lem:rescommuteswithsheaf}, the composition of the restriction to $ M $ with the stalk at $ x $ defines a left exact left adjoint
	\begin{equation*}
		\begin{tikzcd} 
			\Sh(\Man;C) \arrow[r, "\restrict{(-)}{M}"] & \Sh(M;C) \arrow[r, "{(-)_x}"] & C \comma
		\end{tikzcd}
	\end{equation*}
	which we denote by $ \xupperstar $.
	Given a sheaf $ E $ on $ \Man $, we call $ \xupperstar(E) $ the \textit{stalk} of $ E $ at $ x \in M $.
\end{construction}

\begin{observation}\label{obs:restricttoopen}
	Let $ M $ be a manifold and $ j \colon \incto{U}{M} $ an open embedding.
	Then, by definition, the triangle
	\begin{equation*}
		\begin{tikzcd}[sep=3em]
			\Sh(\Man;C) \arrow[r, "\restrict{(-)}{M}"] \arrow[dr, "\restrict{(-)}{U}"'] & \Sh(M;C) \arrow[d, "\jupperstar"] \\
			& \Sh(U;C) 
		\end{tikzcd}
	\end{equation*} 
	commutes.
	Thus for any $ x \in U $, then there is a canonical identification of the stalk functor $ \fromto{\Sh(\Man;C)}{C} $ at $ x \in U $ with the stalk functor at $ j(x) \in M $. 
\end{observation}

Recall that for each integer $ n \geq 0 $, write $ 0_n \in \RR^n $ for the origin (\Cref{ntn:origin}).

\begin{proposition}\label{app.prop:hypercompleteness}
	Let $ C $ be a compactly generated \category.
	Then the set of stalk functors
	\begin{equation*}
		\{0_n\upperstar \colon \Sh(\Man;C) \to C \}_{n \geq 0}
	\end{equation*}
	is jointly conservative. 
	In particular, the \topos $ \Sh(\Man) $ is hypercomplete.
\end{proposition}

\begin{proof}
	In light if \cite[Lemma 2.8]{arXiv:2108.03545}, it suffices to treat the case $ C = \Spc $.
	In this case, first note that the family of restriction functors
	\begin{equation*}
		\restrict{(-)}{M} \colon \fromto{\Sh(\Man)}{\Spc} 
	\end{equation*}
	for $ M \in \Man $ is jointly conservative (\Cref{obs:restrictiontoamanifold}).
	For each manifold $ M $, the \topos $ \Sh(M) $ is a hypercomplete \topos and the points of $ M $ provide conservative family of points for $ \Sh(M) $ \HTT{Corollary}{7.2.1.17}.
	Thus the stalk functors
	\begin{equation*}
		\xupperstar \colon \Sh(\Man) \to \Spc
	\end{equation*}
	for all $ M \in \Man $ and $ x \in M $ form a conservative family of points for $ \Sh(\Man) $.
	To conclude, note that for every manifold $ M $ and point $ x \in M $, there exists an open embedding $ j \colon \incto{\RR^n}{M} $ such that $ j(0_n) = x $ and apply \Cref{obs:restricttoopen}.
\end{proof}

We now give a quick argument showing that the \topos $ \Sh(\Man) $ is Postnikov complete.
We learned the following argument from Hoyois; it is a slight refinement of the argument for the convergence of Postnikov towers that Hoyois gave in \cite{MO:130999}.

\begin{proposition}\label{app.prop:postnikovcompleteness}
	The \topos $ \Sh(\Man) $ is Postnikov complete.
\end{proposition}

\begin{proof}
	Since $ \Sh(\Man) $ is hypercomplete, by \Cref{obs:hypercompletecons} it suffices to show that the right adjoint $ \plowerstar \colon \fromto{\Sh(\Man)^{\post}}{\Sh(\Man)} $ is fully faithful.
	That is, we need to show that for every collection of objects $ \{F_n\}_{n \geq -2} $ of $ \Sh(\Man)$ equipped with compatible equivalences $ \equivto{\trun_{\leq n}(F_{n+1})}{F_n} $, and integer $ k \geq -2 $, the natural morphism
	\begin{equation}\label{eq:truncationmor}
		\fromto{\trun_{\leq k}\paren{\lim_{n \geq -2} F_n}}{F_k}
	\end{equation}
	is an equivalence.
	To see this, note that since the restriction functors 
	\begin{equation*}
		\{\restrict{(-)}{M} \colon \fromto{\Sh(\Man)}{\Sh(M)}\}_{M \in \Man}
	\end{equation*}
	are jointly conservative and commute with limits and truncations, it suffices to show that the morphism \eqref{eq:truncationmor} becomes an equivalence after restriction to each manifold $ M $.
	This last claim follows from the fact that the \topos $ \Sh(M) $ is Postnikov complete (\Cref{ex:ShMPostnikovcomplete}).
\end{proof}

We finish this section by proving that $ \Sh(\Man) $ is equivalent to the \topos of sheaves on the smaller site $ \Euc \subset \Man $ spanned by the Euclidean spaces (\Cref{def:Euclideansite}).
Note that since every manifold admits a cover by Euclidean spaces, the Euclidean site is a \textit{basis} for the Grothendieck topology on $ \Man $ (see \cite[\S B.6]{Ultracategories} for more about bases for Grothendieck topologies).

\begin{corollary}\label{app.lem:hypersheavesonCart}
	Let $ C $ be a presentable \category.
	Then restriction of presheaves
	\begin{equation*}
		\restrict{(-)}{\Eucop} \colon \fromto{\Sh(\Man;C)}{\Sh(\Euc;C)}
	\end{equation*}
	is an equivalence of \categories.
	The inverse is given by right Kan extension along the inclusion $ \incto{\Eucop}{\Manop} $.
\end{corollary}

\begin{proof}
	Since $ \Sh(\Euc;C) $ and $ \Sh(\Man;C) $ are the tensor products of presentable \categories
	\begin{equation*}
		\Sh(\Euc;C) \equivalent \Sh(\Euc) \tensor C \andeq \Sh(\Man;C) \equivalent \Sh(\Man) \tensor C \comma
	\end{equation*}
	it suffices to treat the case where $ C = \Space $ is the \category of spaces.
	In this case, since the \topos $ \Sh(\Man) $ is hypercomplete (\Cref{app.prop:hypercompleteness}), the claim follows from the fact that $ \incto{\Euc}{\Man} $ is a basis for the topology on $ \Man $ \cite[Corollary 3.12.13]{exodromy}.
\end{proof}

%-------------------------------------------------------------------%
%-------------------------------------------------------------------%
%  The sheafification of an ℝ-invariant presheaf                    %
%-------------------------------------------------------------------%
%-------------------------------------------------------------------%

\subsection{The sheafification of an \texorpdfstring{$ \RR $}{ℝ}-invariant presheaf}\label{sec:sheafificationpreservesinvariance}

In this section we show that if $ F $ is an $ \RR $-invariant presheaf on $ \Man $, then the sheafification of $ F $ is $ \RR $-invariant (\Cref{app.prop:sheafificationofinvariantisinvariant}).
This provides a description of the homotopification functor $ \Lhi $.

\begin{recollection}\label{app.rec:stalksandsheafification}
	Let $ C $ be a compactly generated \category, $ T $ be a topological space, $ t \in T $, and $ F $ a $ C $-valued presheaf on $ T $.
	Then the morphism $ \fromto{F_t}{\Sup_{T}(F)_t} $ on stalks at $ t $ induced by the unit $ \fromto{F}{\Sup_{T} F} $ is an equivalence.
	See \cite[Proposition 4.1.4]{Broken}.
\end{recollection}

\begin{lemma}\label{app.lem:stalksbeforesheafification}
	Let $ C $ be a compactly generated \category, $ F \in \PSh(\Man;C) $, $ M $ a manifold, and $ x \in M $.
	Then the morphism
	\begin{equation*}
		\fromto{\xupperstar F}{\xupperstar \SMan F}
	\end{equation*}
	induced by the unit is an equivalence.
\end{lemma}

\begin{proof}
	By definition, if $ F' $ is a presheaf on $ \Man $, then $ \xupperstar F' \colonequals (\restrict{F'}{M})_x $.
	By \Cref{lem:rescommuteswithsheaf} we have a canonical identification $ \restrict{\SMan(F)}{M} = \Sup_{M}(\restrict{F}{M}) $.
	The claim now follows from \Cref{app.rec:stalksandsheafification}.
\end{proof}

\begin{proposition}\label{app.prop:sheafificationofinvariantisinvariant}
	Let $ C $ be a presentable \category and $ F \colon \fromto{\Manop}{C} $ an $ \RR $-invariant presheaf on $ \Man $.
	Then the counit $ \fromto{\Gammaupperstar \Gammalowerstar \SMan F}{\SMan F} $ is an equivalence.
	In particular, $ \SMan F $ is $ \RR $-invariant.
\end{proposition}

\begin{proof}
	Since the left adjoints
	\begin{equation*}
		\Gammaupperstar\Gammalowerstar \SMan, \SMan \colon \fromto{\PShhi(\Man;C)}{\Sh(\Man;C)}
	\end{equation*}
	are obtained by applying the tensor product of presentable \categories $ - \tensor C $ to the left adjoints
	\begin{equation*}
		\Gammaupperstar\Gammalowerstar \SMan, \SMan \colon \fromto{\PSh_{\RR}(\Man;\Spc)}{\Sh(\Man;\Spc)} \comma
	\end{equation*}
	it suffices to prove the claim in the case that $ C = \Spc $.
	In this case, we show that the counit
	\begin{equation*}
		\counit_F \colon \fromto{\Gammaupperstar \Gammalowerstar \SMan F}{\SMan F}\ 
	\end{equation*}
	is an equivalence by checking that $ \counit_F $ is an equivalence on stalks (\Cref{app.prop:hypercompleteness}).
	Let $ M $ be a manifold and $ x \in M $, and write $ \Gammaupperstar_{\pre} \colon \fromto{C}{\Fun(\Manop,C)} $ for the constant presheaf functor.
	By \Cref{app.lem:stalksbeforesheafification} it suffices to show that the counit
	\begin{equation}\label{eq:presheafred}
		\fromto{\xupperstar \Gammaupperstar_{\pre} F(*)}{\xupperstar F} \period
	\end{equation}

	By definition, $ \xupperstar \Gammaupperstar_{\pre} F(*) = F(*) $, and
	\begin{equation*}
		\xupperstar F = \colim_{U \in \Open_x(M)^{\op}} F(U) \comma
	\end{equation*}
	where $ \Open_x(M) \subset \Open(M) $ is the full subposet spanned by those opens containing $ x \in M $.
	Let $ \Open'_x(M) \subset \Open_x(M) $ denote the full subposet with elements those opens diffeomorphic to $ \RR^{\dim(M)} $.
	Note that the inclusion
	\begin{equation*}
		\Open'_x(M)^{\op} \subset \Open_x(M)^{\op}
	\end{equation*}
	is colimit-cofinal.
	Since $ F $ is $ \RR $-invariant, we see that 
	\begin{align*}
		\xupperstar F &\equivalent \colim_{U \in \Open'_x(M)^{\op}} F(U) \\
		&\equivalent \colim_{U \in \Open'_x(M)^{\op}} F(*) \\
		&\equivalent F(*) \period
	\end{align*}
	Unwinding the definitions we see that the counit morphism \eqref{eq:presheafred} is an equivalence.
\end{proof}

% \Cref{app.prop:sheafificationofinvariantisinvariant} provides another description of the functor
% \begin{equation*}
% 	\Lhi \colon \fromto{\Sh(\Man;C)}{\Shhi(\Man;C)} \period
% \end{equation*}

% \begin{notation}
% 	Write $ \LRR \colon \fromto{\Fun(\Manop,C)}{\PShhi(\Man;C)} $ for the left adjoint to the inclusion.
% \end{notation}

% \begin{corollary}\label{app.cor:hilowershriekasacomposite}
% 	Let $ C $ be a presentable \category.
% 	Then the composite
% 	\begin{equation*}
% 		\SMan \LRR \colon \fromto{\Sh(\Man;C)}{\Shhi(\Man;C)}
% 	\end{equation*}
% 	is left adjoint to the inclusion $ \incto{\Shhi(\Man;C)}{\Sh(\Man;C)} $.
% 	That is, $ \Lhi \equivalent \SMan \LRR $.
% \end{corollary}
