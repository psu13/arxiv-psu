%!TEX root = ../diffcoh.tex

\section{Examples}\label{sec:examples}
\textit{by Araminta Amabel}

The purpose of this section is to construct examples of differential cohomology theories, i.e., sheaves of spectra on the category $ \Mfld $.
We'll construct these examples by using the method of \textit{differential refinements} introduced in \cref{subsec:refinements}.
Note that given a spectrum $E$, there are possibly many differential refinements of $E$. 
We will construct differential cohomology theories refining the cohomology theory $ E $ by the following process:
\begin{enumerate}[(1)]
	\item Choose a pure sheaf $\Phat $ (\Cref{def:pursheaf}).

	\item Compute $\Gammalowershriek\Phat $ using the formula $ \Gammalowershriek\Phat  = \colim_{\Deltaop} \Phat (\Deltaalgdot) $ of \Cref{cor:Gammalowershriek}.

	\item Find a map of spectra $ f \colon \fromto{E}{\Gammalowershriek\Phat} $.

	\item Define $\Ehat$ as in the pullback
	\begin{equation*}
		\begin{tikzcd}[column sep={15ex,between origins}, row sep={11ex,between origins}]
			\Ehat \arrow[r] \arrow[d] \arrow[dr, phantom, "\square" description] & \Phat \arrow[d] & \\
			\Gammaupperstar(E) \arrow[r, "\Gammaupperstar(f)"'] & \Gammaupperstar\Gammalowershriek(\Phat) \period
		\end{tikzcd}
	\end{equation*} 
\end{enumerate}

We start in \cref{subsec:moststimple} with differential refinements of $ 0 $ and what the differential cohomology hexagon looks like in this case.
In \cref{subsec:simplefilt}, we refine this most simple example by adding a filtration.
\Cref{subsec:CheegerSimonsDiffchar} explains how the Cheeger--Simons theory of differential characters fits into this story, and \cref{subsec:diffKtheory} studies differential refinements of $ \Kup $-theory.

%-------------------------------------------------------------------%
%-------------------------------------------------------------------%
%  The most simple example                                          %
%-------------------------------------------------------------------%
%-------------------------------------------------------------------%

\subsection{The most simple example}\label{subsec:moststimple}

To start off, let's try to construct a differential refinement where the pure sheaf $ \Phat $ is zero.
That is, $ \Phat = 0 = \Gammaupperstar  0 $. 
In this case, since the functor $ \Gammalowershriek $ is exact, $ \Gammalowershriek(\Phat) = 0 $.
Any spectrum $E$ maps uniquely to $ 0 $.
Thus for any spectrum $ E $ we have a differential refinement $ \Ehat $ defined by the pullback
\begin{equation*}
	\begin{tikzcd}
		\Ehat\arrow[r]\arrow[d] & \Gammaupperstar E\arrow[d]\\
		\Gammaupperstar 0\arrow[r, equals] & \Gammaupperstar 0
	\end{tikzcd}
\end{equation*}
Since the bottom horizontal arrow is an equivalence, the top horizontal arrow is as well: $ \Ehat = \Gammaupperstar E $. 
The rest of the differential cohomology diagram looks as follows
\begin{equation*}
	\begin{tikzcd}[column sep={10ex,between origins}, row sep={8ex,between origins}]
		& \Gammaupperstar E\arrow[rr]\arrow[dr, equals] & & \Gammaupperstar E\arrow[dr] & \\
		0 \arrow[dr]\arrow[ur] & & \Gammaupperstar E\arrow[ur, equals]\arrow[dr] & & 0\\
		& \Def(\Gammaupperstar E)\arrow[rr]\arrow[ur] & & 0 \arrow[ur] & \phantom{\Gammaupperstar E} \period
	\end{tikzcd}
\end{equation*}
Since the upwards diagonal sequence is a fiber sequence, we also have $\Def(\Gammaupperstar E) = 0 $.

This example is just saying that that $ E $-cohmology is a special case of differential cohomology.
We're really just reformulating the fact that the constant sheaf functor $ \Gammaupperstar \colon \fromto{\Sp}{\Sh(\Man;\Sp)} $ is fully faithful with essential image the $ \RR $-invariant sheaves (\Cref{prop:Dugger}). 

%-------------------------------------------------------------------%
%-------------------------------------------------------------------%
%  The most simple example, but with a filtration                   %
%-------------------------------------------------------------------%
%-------------------------------------------------------------------%

\subsection{The most simple example, but with a filtration}\label{subsec:simplefilt} 

We give an alternative differential refinement of the zero spectrum which comes with a natural filtration.

\begin{nul}
	Let $\Omegabullet\in\Shv(\Man;\D(\RR))$ the sheaf of de Rham forms with cohomological grading; so $ \Omega^k $ is in degree $ -k $. 
	Consider the resulting functor of spectra, $\HOmegabullet$. 
	By the Poincaré Lemma, $\Omegabullet$ is quasi-isomorphic to the constant sheaf at $\RR[0]$. 
	Thus $ \equivto{\HOmegabullet}{\Gammaupperstar\HRR} $. 
	In particular, $\HOmegabullet$ is not pure. 
\end{nul}

However, since $ \HOmegabullet \equivalent \Gammaupperstar\HRR $ is $ \RR $-invariant, the purification $ \Cyc(\HOmegabullet) $ is equivalent zero.
Now $\Omegabullet$ has a filtration by degree. 
For $k\in\NN$, let $\Omega^{\geq k}$ denote the stunted piece of the chain complex $\Omegabullet$ where we have replaced everything in degrees $<k$ by $0$. 
We get induced filtrations of $\HOmegabullet$ and of $\Cyc(\HOmegabullet)\simeq\Gammaupperstar 0$.

For $k\geq 1$, there is an equivalence $\Omega^{\geq k}(*)\simeq 0$ of chain complexes. 
Thus the global sections of $\HOmega^{\geq k}$ is 0,
\begin{equation*}
	\Gammalowerstar \HOmega^{\geq k}= \HOmega^{\geq k}(*) = 0 \period
\end{equation*}
By definition, this means that $\HOmega^{\geq k}$ is a pure sheaf if (and only if) $k\geq 1$. 
The purification functor $\Cyc$ is the identity on pure sheaves, so we obtain a filtration of the pure sheaf $\Gammaupperstar 0$ by pure sheaves
\begin{equation*}
	\Gammaupperstar 0 \to \HOmega^{\geq 1}\to\cdots\to \HOmega^{\geq k}\to\cdots \period
\end{equation*}
Now for each $k\geq 1$, we can choose the pure sheaf $\HOmega^{\geq k}$ and follow our procedure.

We need to compute the homotopification of our chosen pure sheaf.

\begin{lemma}\label{2.1}
	For any $k\in\NN$, there is an equivalence $\Gammalowershriek \HOmega^{\geq k}\simeq \HRR$.
\end{lemma}

\begin{proof}
	For $k=0$ ,we have seen that $\HOmega^{\geq 0}\simeq \Gammaupperstar \HRR$, which is already homotopy invariant. Thus $\Gammalowershriek\Gammaupperstar \HRR\simeq \HRR$. 
	For $k\geq 1$, see \cite[Lemma 7.15]{MR3462099}.
\end{proof}

The following family of differential refinements was introduced by Hopkins and Singer, \cite{HopkinsSinger}.

\begin{definition}
	Let $ E $ be a spectrum and $ f \colon \fromto{E}{\HRR} $ a map of spectra.
	For each $ k \geq 1 $, write $ \Ehat(k) $ for the pullback 
	\begin{equation*}
		\begin{tikzcd}
			\Ehat(k)\arrow[r]\arrow[d] & \Gammaupperstar E\arrow[d, "f"]\\
			\HOmega^{\geq k}\arrow[r] & \HRR \period
		\end{tikzcd}
	\end{equation*}
\end{definition}

The differential cohomology diagram \eqref{diag:generaldiffcohhexGamma} for $\Ehat(k)$ looks like
\begin{equation*}
	\begin{tikzcd}[column sep={10ex,between origins}, row sep={8ex,between origins}]
		& \Gammaupperstar \Gammalowerstar\Ehat\arrow[rr]\arrow[dr] & & \Gammaupperstar E\arrow[dr] & \\
		\Sigma^{-1}\Gammaupperstar \HRR\arrow[dr]\arrow[ur] & & \Ehat(k)\arrow[dr]\arrow[ur] & & \Gammaupperstar \HRR\\
		& \HOmega^{\leq k-1}[-1]\arrow[rr] \arrow[ur] & & \HOmega^{\geq k}\arrow[ur] & \phantom{\Gammaupperstar \HRR} \period
	\end{tikzcd}
\end{equation*}

%-------------------------------------------------------------------%
%-------------------------------------------------------------------%
%  Cheeger–Simons Differential Characters                           %
%-------------------------------------------------------------------%
%-------------------------------------------------------------------%

\subsection{Ordinary Differential Cohomology}\label{subsec:CheegerSimonsDiffchar}

Take $E=\HZZ$ and the map $\HZZ\to \HRR$ induced from the inclusion $\ZZ\subset\RR$.

\begin{defn}
	The $k$-th \emph{ordinary differential cohomology group} of a manifold $M$, denoted $\Hhat^k(M)$ is the $(-k)$-th homotopy group
	\begin{equation*}
		\Hhat^k(M)=\pi_{-k}\HZZhat(k)(M)
	\end{equation*}
	where $\HZZhat(k)$ is defined by the homotopy pullback square
	\begin{equation*}
		\begin{tikzcd}
			\HZZhat(k)\arrow[r]\arrow[d] & \Gammaupperstar  \HZZ\arrow[d]\\
			\Cyc(\HOmega^{\geq k})\arrow[r] & \Gammaupperstar \HRR \period
		\end{tikzcd}
	\end{equation*}
\end{defn}

\begin{nul}
	Note that $\Cyc(\HOmega^{\geq k})\simeq \HOmega^{\geq k}$ if $k\geq 1$ and is $\HRR$ if $k=0$.
\end{nul}

\begin{remark}
	The group $\Hhat^k(M)$ is also known as the \emph{Cheeger--Simons differential characters}, or the \emph{smooth Deligne cohomology}. 
\end{remark}

The following gives an explicit complex computing ordinary differential cohomology.
This complex first appeared in the setting of complex manifolds in Deligne's work on Hodge theory (see \cites[\S2.2]{MR498551}[\S12.3]{MR2451566}), and is why differential cohomology is also called smooth Deligne cohomology.

\begin{lemma}\label{lem:Delignemodel}
	Let $k\geq 1$. 
	The sheaf of spectra $\HZZhat(k)$ is given by applying the Eilenberg--Mac Lane functor $ \EM \colon \fromto{\D(\ZZ)}{\Sp} $ (\Cref{rec:EMspectra}) pointwise to the sheaf of chain complexes
	\begin{equation*}
		(\Gammaupperstar \ZZ\to\Omega^0\to\Omega^1\to\cdots\to\Omega^{k-1}\to 0\to \cdots) \period
	\end{equation*}
	Here $\Omega^i$ is in degree $-i-1$. 
	Moreover, the group $\Hhat^k(M)$, for a manifold $M$, can be computed as the $k$-th sheaf cohomology group of this sheaf of chain complexes.
\end{lemma}

\begin{proof}
	By construction, $\HZZhat(k)$ comes from applying $\EM$ of the sheaf of chain complexes $F$ given by the homotopy pullback
	\begin{equation*}
		\begin{tikzcd}
			F\arrow[r]\arrow[d] & \Gammaupperstar \ZZ[0]\arrow[d]\\
			\Omega^{\geq k}\arrow[r] & \Omegabullet \period
		\end{tikzcd}
	\end{equation*}
	Since the bottom horizontal arrow is an inclusion, its cofiber is given by the cokernel. 
	We have a cofiber sequence in $ \D(\ZZ) $
	\begin{equation*}
		\Omega^{\geq k}\to\Omegabullet\to\Omega^{\leq k-1}
	\end{equation*}
	where $\Omega^{\leq k-1}$ has $\Omega^i$ in degree $-i$, and $0$ above $k-1$. 
	The cofiber of the top horizontal map is equivalent to the cofiber of the bottom horizontal map. 
	Since we are in a stable setting, these cofiber sequences are also fiber sequences. 
	Thus, we have a fiber sequence
	\begin{equation*}
		F\to\Gammaupperstar \ZZ[0]\to\Omega^{\leq k-1}
	\end{equation*}
	where $\ZZ[0]\to\Omega^{\leq k-1}$ includes $\ZZ$ and $\Omega^0$. 
	The fiber of this inclusion is a shift of the mapping cone, which is
	\begin{equation*}
		(\Gammaupperstar \ZZ\to\Omega^0\to\Omega^1\to\cdots\to\Omega^{k-1}\to 0\to\cdots)
	\end{equation*}
	Finally, note that $ \pi_{-k}(\EM F) = \H^k(F) $.
\end{proof}

\begin{example}
	Take $k=0$. 
	Then $\HZZhat(k)\simeq\Gammaupperstar \HZZ$ and 
	\begin{equation*}
		\Gammaupperstar \HZZ(M) = \Hom_{\Sp}(\Sigma_{+}^{\infty} \Piinf(M),\HZZ)
	\end{equation*}
	(\Cref{ex:Sptcotensor,lem:constantishi}).
	Hence $ \Gammaupperstar \HZZ(M) $ has $ 0 $-th homotopy group $\H^0(M;\ZZ)$.	
\end{example}

The following two computations from Kumar's notes \cite{Nilay}.

\begin{example}
	Take $k=1$. We compute $\Hhat^1(M)$. 
	By \Cref{lem:Delignemodel}, we can compute $\Hhat^1(M)$ as the $ 1 $-st sheaf cohomology group of the sheaf of chain complexes $(\Gammaupperstar \ZZ\to\Omega^0)$. 
	After choosing a good cover of $M$, we can compute this sheaf cohomology as Čech cohomology. 
	The Čech cohomology will be the cohomology of the total complex of the following bicomplex,
	\begin{equation*}
		\begin{tikzcd}
			\Cech^0(\Gammaupperstar \ZZ)\arrow[r]\arrow[d] & \Cech^0(\Omega^0)\arrow[d]\\
			\Cech^1(\Gammaupperstar \ZZ)\arrow[r]\arrow[d] & \Cech^1(\Omega^0)\arrow[d]\\
			\Cech^2(\Gammaupperstar \ZZ)\arrow[r]\arrow[d] & \Cech^1(\Omega^0)\arrow[d]\\
			\vdots & \vdots
		\end{tikzcd}
	\end{equation*}
	with $\Cech^i(\Gammaupperstar \ZZ)$ in bidegree $(0,-i)$ and $\Cech^i(\Omega^0)$ in bidgree $(-1,-i)$. 
	The differential on this bicomplex is $D=d^\mrm{hor}+(-1)^pd^\mrm{ver}$ where $p$ is the horizontal degree. 
	The piece of the total complex that we are interested looks like
	\begin{equation*}
		\begin{tikzcd}
			\Cech^0(\Gammaupperstar \ZZ) \arrow[r, "D_0"] & \Cech^0(\Omega^0)\oplus\Cech^1(\Gammaupperstar \ZZ) \arrow[r, "D_1"] & \Cech^1(\Omega^0)\oplus\Cech^2(\Gammaupperstar \ZZ) \period 
		\end{tikzcd}
	\end{equation*}
	If our good cover of $M$ is $\{U_\alpha\}$ with intersections $U_{\alpha\beta}$, then an element of $\Cech^0(\Omega^0)\oplus\Cech^1(\Gammaupperstar \ZZ)$ looks like a collection of smooth maps $f_\alpha\colon U_\alpha\to\RR$ and integers $n_{\alpha\beta}\in\ZZ$. The map $D_1$ sends
	\begin{equation*}D_1(f_\alpha, n_{\alpha\beta})=(f_\alpha-f_\beta+n_{\alpha\beta},n_{\beta\gamma}-n_{\alpha\gamma}+n_{\alpha\beta})\end{equation*}
	In particular, an element of $\ker D_1$ consists of maps $f_\alpha$ that agree on intersections up to an integer. These glue together to give a (smooth) map $f\colon M\to \Circ=\Uup_{1}$.

	The map $D_0$ sends a collection $(n_\alpha)$ to
	\begin{equation*}D_0(n_\alpha)= (c_{n_\alpha},n_\alpha-n_\beta)\end{equation*}
	where $c_{n_\alpha}$ is the constant function $U_\alpha\to\RR$ at the integer $n_\alpha$. As a map $M\to \Circ$, these glue together to the constant map at the base point.

	Thus we have an isomorphism
	\begin{equation*}
		\Hhat^1(M)\isomorphic\Mapsm(M,\Uup_{1}) \period
	\end{equation*}
	In ordinary cohomology, we have
	\begin{equation*}
		\H^1(M;\ZZ)= \uppi_{0} \Map_{\Spc}(M,\Kup(\ZZ,1)) = \uppi_{0} \Map_{\Spc}(M,\Uup_{1}) \period
	\end{equation*}
	In this sense, differential cohomology replaced homotopy maps with smooth maps.
\end{example}

\begin{example}
	Take $k=2$. 
	Then we have an isomorphism
	\begin{equation*}
		\Hhat^2(M) \isomorphic \{\text{line bundles on $M$ with connection}\}/\sim \period
	\end{equation*}
	In ordinary cohomology, we have
	\begin{equation*}
		\H^2(M;\ZZ) = \uppi_{0} \Map_{\Spc}(M,\Kup(\ZZ,2)) = \uppi_{0} \Map_{\Spc}(M,\BU(1)) = \{\text{line bundles on $M$}\}/\sim \period 
	\end{equation*}
	In this sense, the new geometric information encoded in differential cohomology is the connection.
\end{example}

%-------------------------------------------------------------------%
%-------------------------------------------------------------------%
%  Differential K-Theory                                            %
%-------------------------------------------------------------------%
%-------------------------------------------------------------------%

\subsection{Differential \texorpdfstring{$\Kup$}{K}-Theory}\label{subsec:diffKtheory}

\begin{nul}
	Consider de Rham forms with $\CC[u^{\pm 1}]$ coefficients, with $u$ in degree $2$. 
	We obtain a family of pure sheaves $\HOmega^{\geq k}(-;\CC[u^{\pm 1}])$. 
	As in \Cref{2.1}, we have an equivalence,
	\begin{equation*}
		\Gammalowershriek\HOmega^{\geq k}(-;\CC[u^{\pm 1}])\simeq \HCC[u^{\pm 1}]
	\end{equation*}
\end{nul}

\begin{nul}
	Take $E=\ku$ to be the spectrum defining connective complex $\Kup$-theory. 
	The \textit{Chern character} defines a map of spectra
	\begin{equation*}
		\ch \colon \ku \to \HCC[u^{\pm 1}] \period
	\end{equation*}
	The resulting family of differential cohomology theories defined by pullback squares,
	\begin{equation*}
		\begin{tikzcd}
			\kuhat(k) \arrow[r] \arrow[d] & \Gammaupperstar  \HOmega^{\geq k}(-;\CC[u^{\pm 1}]) \arrow[d]\\
			\Gammaupperstar(\ku) \arrow[r, "\Gammaupperstar(\ch)"'] & \Gammaupperstar \HCC[u^{\pm 1}] \period
		\end{tikzcd}
	\end{equation*}
	first studied by Hopkins and Singer in \cite{HopkinsSinger} is called \textit{differential $\Kup$-theory}.
\end{nul}

\begin{nul}
	There are other interesting differential refinements of $\ku$ that do not arise from the pure sheaves $\HOmega^{\geq k}(-;\CC[u^{\pm 1}])$.
\end{nul}
