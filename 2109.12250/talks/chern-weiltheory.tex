%!TEX root = ../diffcoh.tex

\section{Chern--Weil Theory}\label{ChernWeilTheory}
\textit{by Greg Parker}

As this talk is a review of standard material, 
many technical results are stated without proof. 
For more detailed review, including proofs, the reader should consult \cite[Appendix C]{MilnorStasheff} for a review of connections and Chern--Weil theory for vector bundles, 
\cite[Chapter II]{KobayashiNomizu} or \cite[Chapter 2]{JohnRoe} for the theory of connections on principal bundles, and 
\cite[Chapter XII]{KobayashiNomizuII} for Chern--Weil theory for principal bundles.
%-------------------------------------------------------------------%
%-------------------------------------------------------------------%
%  Motivation                                                       %
%-------------------------------------------------------------------%
%-------------------------------------------------------------------%

\subsection{Motivation}

To begin, let's recall
\begin{thm}[(Gauss--Bonnet)]
	Let $(\Sigma,g)$ be a compact, oriented, Riemannian $ 2 $-manifold without boundary. Let $\chi(\Sigma)$ be its Euler characteristic. 
	Then
	\[
		\int_\Sigma \kappa \d A=2\pi\chi(\Sigma) \period
	\]
\end{thm}
Here $\kappa$ is the Gaussian curvature defined as follows. 
If $R_{ij}\d x_i\d x_j$ is the Riemann curvature tensor, locally
\[R=\begin{pmatrix}
0 & R_{12}\\
-R_{21} & 0
\end{pmatrix} \d x_1\wedge \d x_2\]
and $\kappa= R_{12}$. So we can rewrite the above as
\[
	\langle[\sqrt{\det(R)}],[\Sigma]\rangle=\int_\Sigma\sqrt{\det(R)}=2\pi\chi(\Sigma)=\langle 2\pi e(T\Sigma),[\Sigma]\rangle \comma
\]
where $e(T\Sigma)$ is the Euler class of $\Sigma$ and the brackets on the right-hand side  denote the pairing 
\[
	\H^2(\Sigma;\RR)\otimes \H_2(\Sigma;\RR)\rta\RR\period
\]

Thus we observe $\sqrt{\det(R)}$, a polynomial in the curvature, captures information about the topology of $\Sigma$ and its tangent bundle $T\Sigma$. Chern--Weil theory (which was actually the original formulation/theory of characteristic classes) generalizes the above to higher dimension and arbitrary bundles.

%-------------------------------------------------------------------%
%-------------------------------------------------------------------%
%  Connections and Curvature                                        %
%-------------------------------------------------------------------%
%-------------------------------------------------------------------%

\subsection{Connections and Curvature}
In order to formulate things correctly, we will need to recall some facts about connections and curvature, 
both for vector bundles and for principal bundles. 

\begin{convention}
	Throughout this talk, 
	let $M$ be a closed $n$-manifold, $\pi\colon E\rta M$ a rank $k$ real or complex vector bundle with structure group $G = \Or_k $ or $ G = \Uup_k $. 
	Denote the real (or Hermitian) inner product by $\langle -, -\rangle$. 
	Let $\g$ denote the Lie algebra of $G$.

	Also, let $K$ be a Lie group and $p\colon Q\rta X$ be a principal $K$-bundle. 
	Let $\mfrk{k}$ denote the Lie algebra of $K$.
\end{convention}

\subsubsection{For Vector Bundles}
\label{subsubsec:connections_for_vbs}
We would like to differentiate sections $\psi\colon M\rta E$. 
The problem is $\psi_{x(t)}$ for a path $x(t)\subset M$ all live in different vector spaces: $E_{x(t)}$, respectively, so we must find a way to ``connect'' them. 

View $\psi_{x(t)}$ as a path in the total space. The derivative (intuitively) is the vertical component of $\frac{\partial\psi}{\d t}$. Think of $f\colon\RR\rta\RR$, then $\frac{\d f}{\d t}$ is the $y$-coordinate of the graph inside $\RR^2$. 
To define this precisely we need to choose a splitting
\[
	TE\simeq VE\oplus HE
\]
into the ``vertical'' and ``horizontal'' subbundles.
The vertical piece $VE=\ker \d\pi$ is canonical, and the horizontal piece $HE$ is not. Such a splitting is called a \emph{connection}. 
Once we choose a connection, 
we get an isomorphism $\d\pi\colon HE\rta TM$. 
So given $e\in E_{x(t)}$ we can lift $\dot{x}$ (a vector field along $x(t)$) to one $X^e_H\subset HE$. 
Then the flow is a path in $E$ projecting to $x(t)$, which is the \emph{parallel transport}, denoted $\varphi_te\in E_{x(t\times I)}$. Then
\[
	\varphi_{-t}\psi(t)\in E_{x(0)}
\]
for all $t$, so we can differentiate. 
The covariant derivative (with respect to our chosen connection) in the $\dot{x}(0)$ direction at $x(0)$ is 
$\frac{\d}{\d t}\vert_{t=0}\varphi_{-t}\psi_{x(t)}$. 
Thus we get an operator
\[
	\d_{A}\text{ or }\nabla^A\colon\Gamma(M,E)\rta\Gamma(M,T^*M\otimes E)
\]
associated to a connection $A$, called the \emph{covariant derivative}. 
Here, $\nabla^A$ eats a vector field $X\in\Gamma(M,TM)$ and gives the derivative in that direction at each point. It satisfies
\begin{itemize}
	\item $\nabla_{fX}^A\psi=f\nabla_X^A\psi$ ($\Cinf$-linear in direction of derivative), and 

	\item $\nabla^Af\psi=\d f\otimes\psi+f\nabla\psi$ (Leibniz rule).
\end{itemize}

The existence of connections is preserved under various bundle constructions.  

\begin{prop}\label{properties of connections}
Given $\nabla^A$ on $E$, $\nabla^B$ on $F$ we get connections
\begin{itemize}
	\item $\nabla^{A^*}$ on the dual bundle $E^*$

	\item $\nabla^{AB}$ on the tensor product $E\otimes F $ by the formula 
\[\nabla^{AB}(\varphi\otimes\psi)=\nabla^A\varphi\otimes\psi+\varphi\otimes\nabla^B\psi\]

	\item if $F\colon M\rta N$ and $E\rta N$ then $F^*(\nabla^A)$ is a connection on  $f^*E$ by
		\[
			(F^*\nabla^A)_X\psi(m)\colonequals \nabla^A_{F_*X}\psi(f(m))\in E_{F(m)}=F^*E_m \period 
		\]
\end{itemize}
\end{prop}

\begin{prop}\label{difference of connections}
	Two connections differ by a $1$-form valued in $\End(E)$. 
	In particular, the set of connections form an affine, and hence contractible, space.
\end{prop}

\begin{remark}
	Thus one might expect invariants defined using them (if discrete) to not depend on the choice of connection.
\end{remark}

\begin{proof}
Let $A$ and $A'$ be two connections on the bundle $\pi\colon E\rta M$. 
For $f\in\Cinf(X)$ and $\psi$ a section of $\pi$, we have
	\begin{align*}
		(\nabla^A-\nabla^{A'})(f\psi)&=\d f\otimes\psi+f\nabla^A\psi-\d f\otimes\psi-f\nabla^{A'}\psi \\
		&=f(\nabla^A-\nabla^{A'})\psi
	\end{align*}
	is $\Cinf$-linear with values in $\Gamma(T^*M\otimes E)$ so $\nabla^A-\nabla^{A'}\in\Omega^1(\End(E))$.
\end{proof}

\begin{ex}
	On the trivial rank $k$-bundle $\underline{\RR}_M$ on $M$, the exterior derivative 
	\[\d\colon\Gamma(M,\underline{\RR}_M)\rta\Omega^1(M)\]
	 is a connection.
\end{ex}

\begin{ex}
	In a local trivialization (by \Cref{difference of connections}) we can always write $\nabla=\d+A$, where $A\in\Omega^1(\End(E))$. 
	That is $A=A_1\d x_1+\cdots A_n\d x_n$ for $A_i$ matrices, and
	\[\nabla_i\psi=\frac{\del \psi}{\del x_i}+A_i\psi \period \]
\end{ex}

\begin{ex} 
	On $\End(E)=E^*\otimes E$, the induced $\nabla$ from \Cref{properties of connections} is 
	\[\nabla B=\d B+[A,B]\]
	in a trivialization.
\end{ex}

Define a connection as \emph{compatible with $\ang{-,-}$} if
\[
	\d\langle\psi,\varphi\rangle=\langle \nabla\psi,\varphi\rangle+\langle\psi,\nabla\varphi\rangle \period
\]
Note that for compatible $\nabla$, $A$ will be in $\Omega^1(\mathfrak{o}(E))$ or $\mathfrak{u}(E)$. 

\begin{remark}
	A fancy way of saying this is $\ang{-,-}\in E^*\otimes E^*$ has $\nabla =0$.
\end{remark}

\begin{lem}
	Every bundle $E$ has a connection compatible with $\ang{-,-}$.
\end{lem}

\begin{proof}
	Locally, connections of the form $ \d+A$ are compatible with $\ang{-,-}$ if $A$ is in $\Omega^1(\mathfrak{o}(E))$ or $\Omega^1(\mathfrak{u}(E))$. 
	This gives existence locally. Using a partition of unity, one obtains the desired connection globally.
	\end{proof}

\subsubsection{For Principal Bundles}\label{subsec-principal}
For a principal $K$-bundle $p\colon Q\rta X$, 
the space of vertical tangent vectors $\ker(\d p)$ of $Q$ gives a short exact sequence
\begin{align}\label{ses principal}
0\rta\ker(p_*)\rta TQ\rta p^*TX\rta 0
\end{align}
of vector bundles over $P$. 
As in the vector bundle situation, a connection will be a way of considering \emph{horizontal} tangent vectors.

\begin{defn}
A \emph{principal connection} on $p\colon Q\rta X$ is a splitting of the exact sequence \Cref{ses principal}. 
\end{defn}

The kernel $\ker(\d p)$ can be identified with the trivial bundle with fiber the tangent space of the fiber $K$ of $p$. 
That is, we have an equivalence $\ker(\d p)\simeq Q\times \mfrk{k}$. 
A splitting of \Cref{ses principal} is equivalent to a section of the map $\ker (\d p)\rta TQ$. 
Using the identification $\ker(\d p)\simeq Q\times \mfrk{k}$, 
a section $TQ\rta \ker(\d p)$ is equivalent to a section of $T^*Q\otimes( \mfrk{k}\times Q)$; i.e., a one form with coefficients in $\mfrk{k}$. 

\begin{defn}
Let $p\colon Q\rta X$ a principal $K$-bundle with principal connection. 
The \emph{connection 1-form} os the principal connection is the one form 
$\omega\in \Omega^1(Q;\mfrk{k})$ 
corresponding the splitting of \Cref{ses principal}.
\end{defn}

Note that $\mfrk{k}$ acts on $\mfrk{k}$ in two ways: 
by right $m_r$, 
and by conjugation $\mrm{Ad}_\mfrk{k}$. 

\begin{lem}\label{connection 1-form equivariant}
Let $p\colon Q\rta X$ a principal $K$-bundle with principal connection 1-form $\omega$. 
Then $\omega$ is $K$-equivariant,
\[\mrm{Ad}_\mfrk{k}(m_r(\omega))=\omega\]
and for $\xi\in\mfrk{k}$ with associated vector field $X_\xi$, we have $\omega(X_\xi)=\xi$. 
\end{lem}

\begin{rmk}
A connection on a principal bundle gives rise to a vector bundle connection on any associated vector bundle. 
Likewise, a $K$-compatible connection on a vector bundle $E$ gives rise to a connection on the $K$-frame bundle, 
and these operations are inverses. 
The horizontal distribution on $TQ$ complementing $\ker(p_*)$ in \Cref{ses principal} 
is obtained from a vector bundle connection as directions of the infinitesmal parallel transport at a point. 
In the opposite direction, the parallel transport of frames on $Q$ naturally gives a parallel transport of section of the vector bundle. 
Alternatively, in local coordinates the connection form is just $d+\omega$ for $\omega$ the ad-equivariant connection form on $Q$.
\end{rmk}

%-------------------------------------------------------------------%
%-------------------------------------------------------------------%
%  Curvature                                                        %
%-------------------------------------------------------------------%
%-------------------------------------------------------------------%

\subsection{Curvature}

\subsubsection{For Vector Bundles} 

Given two vector fields $X,Y\in\Gamma(M,TM)$, 
the maps $\nabla_X$ and $\nabla_Y$ need not commute; i.e.,
\[\nabla_X\nabla_Y-\nabla_Y\nabla_X\neq 0\]
Geometrically, since these were defined by flowing along horizontal lifts, $\tilde{X},\tilde{Y}$, this is a question about non-commuting flows; i.e., $[\tilde{X},\tilde{Y}]$. 
In particular, if the horizontal bundle $HE$ is integrable, then $[\tilde{X},\tilde{Y}]=0$ so the flows (and hence $\nabla_X,\nabla_Y$) commute. Thus the \emph{curvature} 
\[
	F_A(X,Y)(\psi)\colonequals[\nabla_X,\nabla_Y](\psi)-\nabla_{[X,Y]}(\psi)
\]
is a measure of the integrability of $HE\subseteq E$. 
Here $A$ is such that locally we have $\nabla=d+A$. 

We get a local description of the curvature by 
	\[
		F_A=\d A+A\wedge A\period 
	\]
	In other words, 
	\[
		F_A=F_A^{ij}\d x^i\wedge \d x^j
	\]
	with
	\[
		F^{ij}_A=\del_iA_j-\del_jA_i+A_iA_j-A_jA_i \period
	\]

\begin{claim}
The curvature $F_A$ defines a 2-form with values in the endomorphism bundle, \[F_A\in \Omega^2(M;\mrm{End}(E))\period\]
In particular, the curvature is $\Cinf$-linear, $F_A(f\psi)=fF_A\psi$. 
\end{claim}

\begin{proof}
This follows from the Leibniz rule for connections. 
\end{proof}

For $F_A\in\Omega^2(\End(E))$. 
We have 
\[
	\d_{A}\colon\Omega^2(\End(E))\rta\Omega^3(\End(E))
	\]
by $\alpha\otimes B\mapsto \d\alpha\otimes B+\alpha\otimes\nabla B$. 

\begin{thm}[Bianchi Identity]
The exterior derivative of the curvature vanishes, $\d_{A}F_A=0$.
\end{thm}

\subsubsection{For Principal Bundles}
\label{sssec:CW_principal}
The wedge product of $\omega\in \Omega^1(Q;\mfrk{k}\otimes\mfrk{k})$ with itself is an element of $\Omega^2(Q;\mfrk{k})$. 
The Lie bracket on $\g$ induces a map 
\[[-]\colon \Omega^2(Q;\mfrk{k}\otimes\mfrk{k})\rta\Omega^2(Q;\mfrk{k})\period\]

\begin{defn}
Let $p\colon Q\rta X$ a principal $K$-bundle with principal connection 1-form $\omega$. 
The \emph{curvature} of $\omega$ is 
\[\Omega=\d\omega+[\omega\wedge \omega]\]
in $\Omega^2(Q;\mfrk{k})$.
\end{defn}

Consider $\mfrk{k}$ as a $K$-module with the adjoint action. 
Let $\mfrk{k}_Q\rta X$ denote the adjoint bundle ${\mfrk{k}_Q=Q\times_K\mfrk{k}}$. 

\begin{lem}
Let $p\colon Q\rta X$ a principal $K$-bundle with connection. 
Let $\Omega$ be its curvature. 
Then $\Omega$ descends to a $2$-form $\tilde{\Omega}\in\Omega^2(X;\g_Q)$. 
\end{lem}

\begin{ex}
Take $K=\GL_n$ so that $Q$ has an associated rank $n$ vector bundle $V\rta X$. 
The adjoint bundle can be identified with the endomorphism bundle $\mrm{End}(V)$. 
Under this identification, 
a principal connection on $Q$ corresponds to a connection on the vector bundle $V\rta X$, and 
the curvature $\tilde{\Omega}$ from a principal connection on $Q$ corresponds to the curvature of $V\rta X$. 
\end{ex}

\begin{thm}[(Bianchi Identities)]
	We have $\d\Omega+[\omega\wedge\Omega]=0$ and $\d\Omega=0$. 
\end{thm}

%-------------------------------------------------------------------%
%-------------------------------------------------------------------%
%  Invariant Polynomials                                            %
%-------------------------------------------------------------------%
%-------------------------------------------------------------------%

\subsection{Invariant Polynomials}

\subsubsection{For Vector Bundles}

In Gauss--Bonnet we used $\sqrt{\det}$ to turn the $R\in\Omega^2(\so(T\Sigma))$ into an $\RR$-valued form to integrate. In general, since $F_A$ isn't basis-invariant we want a map $P\colon\g\rta\RR$ (for $G=\SO_k$ or $\SU_k$) invariant under $\Ad$. If $P$ is a polynomial, we say it is an \emph{invariant polynomial}. 
The space of $\Ad$-invariant polynomials on $\mathfrak{g}$ is $\Sym^\bullet(\gdual)^{\Ad}$. 

\begin{ex}
	Both $\Tr$ and $\Det$ are $\Ad$-invariant. 
\end{ex}

\noindent Thus given $P,A$ we get an $\RR$-valued form $P(F_A)\in\Omega^*(M;\RR)$. 


\begin{prop}
The form $P(F_A)$ is closed, $\d P(F_A)=0$, 
Hence we get a homomorphism
	\[
		\Sym^\bullet(\gdual)^{\Ad}\rta \HdR^*(M;\RR) \period  \index[terminology]{Chern--Weil homomorphism} 
	\]
\end{prop}
 The map above is called the Chern--Weil homomorphism, or sometimes just the Weil homomorphism.

\begin{proof}
Write $P(\xi)=\sum_I P_I\xi_{i_1},\dots,\xi_{i_N}$. 
Since $P$ is $\Ad$-invariant, for $g_t=\exp(t\eta)$, we have
		\[
			P(\xi)=P(\Ad_{g_t}\xi)
		\]
		so
		\begin{align*}
			0&=\frac{\d}{\d t}P(\xi)\\
			&=\frac{\d}{\d t}\sum_I P_I(\Ad_{g_t}\xi)_{i_1}\cdots(\Ad_{g_t}\xi)_{i_N}\\
			&=\sum_{I,k}P_I\xi_{i_1}\cdots\xi_{i_{k-1}}[\eta,\xi]_{i_k}\cdots\xi_{i_N} \period
		\end{align*}
Writing $F_A=\sum F_A^i$, we have
		\begin{align*}
			\d P(F_A)&=\d\paren{\sum_IP_IF^{i_1}\wedge\cdots\wedge F^{i_N}} \\
			&=\sum_{I,k}P_IF^{i_1}\wedge\cdots \wedge dF^{i_k}\wedge\cdots \wedge F^{i_N}+\sum_{I,k}P_IF^{i_1}\wedge\cdots\wedge [A,F_A]_{i_k} \wedge\cdots\\
			&=\sum_{I,k}P_IF^{i_1}\wedge\cdots\wedge (\d_{A}F_A)_{i_k}\wedge\cdots\wedge F_{i_N}\\
			&=0 \period \qedhere
		\end{align*}
\end{proof}

\begin{prop}[(invariance)]
	The class $[P(F_A)]$ satisfies the following properties.
	\begin{enumerate}[{\upshape (1)}]
		\item $[P(F_A)]$ is independent of $A$.

		\item $[P(F_A)]$ is independent of $\ang{-,-}$.

		\item If $E\simeq E'$ then $[P(F_A)]=[P(F_{A'})]$. 
		The \emph{characteristic class} of $E$ is $[P(F_A)]\in \H^*$.
	\end{enumerate}
\end{prop}

\begin{proof}[Proof Sketch]
For (1), take $A,A'$ and set $\nabla_A-\nabla_{A'}=B$. Define $A_t$ n $E\times I\rta M\times I$ by $\nabla_A+tB$. Then $P(F_{A_t})\in\Omega^\bullet(M\times I;\RR)$, and $i_0\colon M\rta M\times\{0\}$ has $i_0^*P(F_{A_t})=P(F_A)$ for some $i_1,A'$. But $i_0,i_1$ are homotopic.

The proof of (2) is similar. 

For (3), use the pullback connection plus the independence of $A$.
\end{proof}

\subsubsection{For Principal Bundles}

We have an analogous story for principal bundles, 
using the corresponding notions of curvature and Bianchi identities. 

\begin{prop}\label{Chern--Weil map for Principal}
	Let $Q\rta X$ be a principal $K$-bundle with curvature $\Omega$. 
	The assignment $P\mapsto P(\Omega)$ determines a map
	\[
		\Sym(\mfrk{k}^\vee)^\mrm{Ad}\rta\Omega_\mrm{dR}^\bullet(X)
	\]
	that descends to a map on cohomology. 
\end{prop}

%-------------------------------------------------------------------%
%-------------------------------------------------------------------%
%  Examples                                                         %
%-------------------------------------------------------------------%
%-------------------------------------------------------------------%

\subsection{Examples}

Now the fun part: choose different $P$ and see what we get.

%-------------------------------------------------------------------%
%  Chern Classes                                                    %
%-------------------------------------------------------------------%

\subsubsection{Chern Classes}
\label{chern_const}

Consider the polynomial $P=\Det(\lambda\id{}-\frac{1}{2\pi i}X)\colon\ufrak_k\rta\RR$. 
Then expanding out, we get 
\[P=\lambda^k-c_1(X)\lambda^{k-1}+c_2(X)\lambda^{k-2}+\cdots\]
for $c_k$ polynomials in $X$. Define the characteristic class $c_k$ in $\H^{2k}$ obtained from $P$ to be the $k$th Chern class. Explicitly
\begin{align*}
c_k(F_A)&=\frac{\Tr(F_A^{\wedge k})}{(2\pi i)^k}\\
&=1-\frac{1}{2\pi i}\Tr(F_A)+\frac{\Tr(F_A\wedge F_A)-\Tr(F_A)^2}{8\pi^2}-\cdots.
\end{align*}
\begin{remark}
It's immediate that $c_1=0$ for an $\SU_n$-bundle since $\su_n$ is traceless. In fact, one can show $c_k$ are a basis for $\Ad$-invariant polynomials so this is a complete list.
\end{remark}

%-------------------------------------------------------------------%
%  Pontryagin Classes                                               %
%-------------------------------------------------------------------%

\subsubsection{Pontryagin Classes}
\label{pontrjagin}
Consider the polynomial $P$ from 
\[
	\Det(\lambda\id{}-\frac{1}{2\pi}X)\colon\ofrak_k\rta\RR.
\]
Expanding out, we get
\[
	P=\lambda^k-g_1(X)\lambda^{k-1}+\cdots.
\]
Since $\ofrak_k$ is skew-symmetric $g_{\odd}=0$ and $g_{2k}=p_k(E)$ is the $k$th Pontryagin class. 
For example, we have
\[
	p_1=-\frac{\Tr(F_A\wedge F_A)}{8\pi^2}
\]
and
\[
	p_2=\frac{\Tr(F_A\wedge F_A)^2-2\Tr(F_A\wedge\cdots\wedge F_A)}{128\pi^4}.
\]

%-------------------------------------------------------------------%
%  Euler Class                                                      %
%-------------------------------------------------------------------%

\subsubsection{Euler Class}
\label{euler_const}

If $k$ is even, there is the Pfaffian $\Pf\colon\ofrak(2k)\rta\RR$ with $\Pf(X)^2=\det(X)$. Then the \emph{Euler class} is 
\[e(E)=\Pf(F_A)\period\]

%-------------------------------------------------------------------%
%  Other Classes                                                    %
%-------------------------------------------------------------------%

\subsubsection{Other Classes}

If $g(X)=a_0+a_1X+a_2X^2+\cdots$ is a power series, then $\det(g(X))$ is invariant. For example,
\begin{itemize}
\item we get the total Chern class from
\[g=1+\frac{z}{2\pi i}\]
\item we get the L-genus from
\[g=\frac{z}{\tanh(z)}\]
\item we get the Todd genus from
\[g=\frac{z^2}{1-\exp(-z^2)}\period\]
\end{itemize}

%-------------------------------------------------------------------%
%-------------------------------------------------------------------%
%  Axioms                                                           %
%-------------------------------------------------------------------%
%-------------------------------------------------------------------%

\subsection{Axioms}

There are a set of axioms that Chern classes satisfy. 
Moreover, these axioms uniquely determine the Chern classes. 
See, for example, \cite[\S 4]{MilnorStasheff} for a discussion of this perspective. 
The axioms are
\begin{enumerate}[(1)]
\item $c_0(E)=1$, $c_i(E)=0$ for $i>\rank(E)$
\subitem $c_i=\Tr(\wedge^iF_A)$ and $\wedge^i=0$ for $i\geq\rank(E)+1$.
\item Naturality with pullbacks
\item Whitney sum, $c(E\oplus F)=c(E)\cup c(F)$
\item Normalization $c_1(\cal{O}(1))=-1$ on $\CCP^1$.
\end{enumerate}
One can check that the Chern classes, as we have defined them above, 
satisfy these axioms, see \cite[Appendix C]{MilnorStasheff}. 
Thus, the Chern--Weil definition gives the same Chern classes as other definitions.
\begin{remark}
Although, a priori, $c_k$ has real coefficients $\HdR^{2k}(M;\RR)$, the normalization shows it is actually in the image of the map
\[\H^{2k}(M;\ZZ)\rta \H^{2k}(M;\RR)\period\]
\end{remark}

%-------------------------------------------------------------------%
%-------------------------------------------------------------------%
%  An Application                                                   %
%-------------------------------------------------------------------%
%-------------------------------------------------------------------%

\subsection{An Application}

Here's an application of Chern--Weil theory to something harder to see with other definitions of characteristic classes.
\begin{lem}
Let $E\rta M$ be a complex vector bundle that admits a reduction of structure group to locally constant transition functions (i.e., $E$ is a local system with group $\CC^n)$, then $c_k(E)\in \H^{2k}(M;\ZZ)$ is torsion.
\end{lem}
\begin{proof}
$E$ admits a flat connection 
\[A\mapsto gAg^{-1}+g^{-1}\d g\]
so we can take $ \d+A $  with $A=0$, and this is preserved by changing trivializations.
\end{proof}

