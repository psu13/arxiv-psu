%!TEX root = ../diffcoh.tex

\section{Invertible field theories}
\textit{by Arun Debray}
\label{invertible_field_theories}

Freed--Hopkins \cite[\S 5.4]{FH21} conjecture a different application of generalized differential cohomology to
field theory, describing reflection-positive invertible field theories which are not necessarily topological. In
this chapter we go over this conjecture. This story is similar to an established theorem, Freed--Hopkins'
classification of reflection-positive invertible \emph{topological} field theories \cite{FH21}, so we begin in
\cref{top_IFT} by going over that classification; then in \cref{non_top_field_theory} we generalize to the
nontopological setting.
\subsection{Topological invertible field theories}
\label{top_IFT}
\begin{defn}
Let $\rho(n)\colon H_n\to\Or_n$ be a Lie group homomorphism. An \textit{$H_n$-structure} on a smooth manifold
$M$ is a principal $H_n$-bundle $P\to M$ together with an isomorphism of principal $\Or_n$-bundles \[\theta\colon
P\times_{H_n}\Or_n\isomorphism \mathcal B_{\Or}(M),\] where $\mathcal B_{\Or}(M)$ is the frame
bundle of $M$.
\index[terminology]{Hn-structure@$H_n$-structure}
\index[terminology]{frame bundle}
\end{defn}
%
%
%Let $\rho\colon H\to\Or$ be a homomorphism of topological groups. We may then speak of ``manifolds with
%$H$-structure,'' meaning manifolds $M$ with a lift of the stable tangent bundle map $TM\colon M\to B\Or$ across
%$\rho$:
%\begin{equation}
%% https://q.uiver.app/?q=WzAsMyxbMCwxLCJNIl0sWzEsMSwiQlxcbWF0aHJtIE8iXSxbMSwwLCJCIl0sWzIsMSwiXFx4aSJdLFswLDEsIlRNIiwyXSxbMCwyLCIiLDEseyJzdHlsZSI6eyJib2R5Ijp7Im5hbWUiOiJkYXNoZWQifX19XV0=
%\begin{tikzcd}
%	& BH \\
%	M & {B\mathrm O}
%	\arrow["\rho", from=1-2, to=2-2]
%	\arrow["TM"', from=2-1, to=2-2]
%	\arrow[dashed, from=2-1, to=1-2]
%\end{tikzcd}
%\end{equation}
For example, an $\SO_n$-structure is equivalent data to an orientation, a $\Spin_n$-structure is equivalent to a
spin structure, and so forth.

An $H_n$-structure on a manifold $M$ induces an $H_n$-structure on $\partial M$, and we may therefore consider
bordism groups $\Omega_n^{H}$ of $H_n$-manifolds, as Lashof \cite{Las63} did, and their categorified analogues:
bordism $(\infty, n)$-categories $\Bord_n^H$ of $n$-manifolds with $H_n$-structure, such as the bordism categories
constructed by Lurie \cite{Lur09}, Schommer-Pries \cite{SP17}, and Calaque--Scheimbauer \cite{CS19}.
\index[terminology]{bordism groups}
\index[notation]{OmegaH@$\Omega_n^H$}
\index[terminology]{bordism category}
\index[notation]{BordnH@$\Bord_n^H$}

Recall that a topological field theory (TFT) is a symmetric monoidal functor
\index[terminology]{topological field theory}
\index[terminology]{TFT|see {topological field theory}}
\begin{equation}
	Z\colon\Bord_n^H\to\fC,
\end{equation}
where $\fC$ is some symmetric monoidal $(\infty, n)$-category. The $\infty$-category of TFTs is symmetric monoidal
under ``pointwise tensor product:''\index[terminology]{tensor product!of topological field theories}
\[(Z_1\otimes Z_2)(M) \colonequals Z_1(M)\otimes Z_2(M)\period\]

\begin{defn}[{(Freed--Moore \cite{FM06})}]
A TFT $Z\colon\Bord_n^H\to\fC$ is \textit{invertible} if there is some other TFT $Z^{-1}$ such that $Z\otimes
Z^{-1}$ is isomorphic to the trivial theory (i.e.\ the constant functor valued in $\mathbf{1}_\fC$).
\index[terminology]{topological field theory!invertible}
\index[terminology]{invertible topological field theory|see {topological field theory!invertible}}
\end{defn}

\noindent Equivalently, $Z$ carries objects of $M$ to $\otimes$-invertible objects in $\fC$ and $k$-morphisms to
composition-invertible $k$-morphisms in $\fC$ for all $k$. In many cases it suffices to check invertibility on a
subset of objects, such as certain spheres \cite{432876} or tori \cite{SP18}.

\begin{example}[(Euler theories)]
Let $\lambda\in\CC^\times$. The \emph{Euler theory}
\begin{equation*}
	Z_\lambda\colon\Bord_{n, n-1}^{\Or}\to\Vect_\CC
\end{equation*}
is an invertible TFT which to every object assigns the vector space $\CC$, and to every bordism $X\colon M_1\to M_2$ assigns
multiplication by $\lambda^{\chi(X, M_1)}$. These compose properly because the Euler characteristic satisfies a
gluing formula.\index[terminology]{Euler theory}
\end{example}
Freed--Hopkins--Teleman \cite{FHT10} classified invertible TFTs using work of
Galatius--Madsen--Tillmann--Weiss \cite{GMTW09} and Nguyen \cite{Ngu17}. Freed--Hopkins \cite{FH21} went further:
they studied \textit{reflection-positive invertible TFTs}, which have additional structure. This structure is
related to the notion of unitarity in quantum field theory, so invertible TFTs appearing in the study of unitary
QFTs should have reflection-positive structures.
\index[terminology]{unitarity!in quantum field theory}
\index[terminology]{topological field theory!reflection-positive invertible}

Let $\MTH$ denote the Thom spectrum of $-\mrm{B}\rho\colon \mrm{B}H\to \BO$.\footnote{There is an important
subtlety here: we started with $\rho(n)\colon H_n\to\Or_n$, not the stabilized version $\rho\colon H\to\Or$.
Freed--Hopkins \cite[Theorem 2.19]{FH21} show that the additional data associated to reflection positivity allows
one to define $\rho$ and $H$ such that $\rho(n)\colon H_n\to\Or_n$ is the pullback of $\rho\colon H\to\Or$ along
the inclusion $\Or_n\hookrightarrow\Or$.}
\index[terminology]{Thom spectrum}%
\index[notation]{MTH@$\MTH$}%
Thom's collapse map identifies the homotopy groups
of $\MTH$ with the bordism groups of manifolds with $H_n$-structure \cites[Théorème
IV.8]{ThomThesis}{Pon55}[Theorem C]{Las63}.\footnote{The use of $-\rho$ ensures that we obtain an $H$-structure on
the stable tangent bundle.  Homotopy theorists more traditionally study the Thom spectrum of $\rho$, denoted
$\mathrm{MH}$, which corresponds to bordism of manifolds with an $H$-structure on the stable \emph{normal} bundle.
Often $\MTH\simeq\mathrm{MH}$, as is the case for $\mathrm{MTO}$, $\mathrm{MTSO}$, $\mathrm{MTSpin}$,
$\mathrm{MTSpin}^c$, $\mathrm{MTString}$, and $\mathrm{MTU}$, but not always:
$\mathrm{MTPin}^+\not\simeq\mathrm{MPin}^+$.} Let $\IZ$ denote the \textit{Anderson dual of the sphere
spectrum} \cite{And69, Yos75}, which satisfies the universal property that there is a short exact sequence
\index[terminology]{Anderson dual of the sphere spectrum}
\index[notation]{IZ@$\IZ$}
\begin{equation}
	% https://q.uiver.app/?q=WzAsNSxbMCwwLCIwIl0sWzEsMCwiXFxidWxsZXQiXSxbMiwwLCJcXGJ1bGxldCJdLFszLDAsIlxcYnVsbGV0Il0sWzQsMCwiMCJdLFswLDFdLFsxLDJdLFsyLDNdLFszLDRdXQ==
\begin{tikzcd}
	0 & \Ext(\pi_{n-1}(X), \ZZ) & {[X, \Sigma^n \IZ]} & \Hom(\pi_n(X), \ZZ) & 0 \comma
	\arrow[from=1-1, to=1-2]
	\arrow[from=1-2, to=1-3]
	\arrow[from=1-3, to=1-4]
	\arrow[from=1-4, to=1-5]
\end{tikzcd}
\end{equation}
which noncanonically splits.

\begin{thm}[{(Freed--Hopkins \cite{FH21})}]
	There is an isomorphism of abelian groups from $\pi_0$ of the space reflection-positive, invertible,
	$n$-dimensional, topological field theories to the torsion subgroup of $[\MTH, \Sigma^{n+1}\IZ]$.
\end{thm}

\begin{remark}
	Any classification of TFTs $Z\colon\Bord_n^H\to\fC$ depends on what we take $\fC$ to be. For this theorem,
	Freed--Hopkins make an ansatz about the choice of $\fC$. Example $\fC$ meeting this ansatz are known in category
	number $2$ and below: see \cite[Theorem 1.52]{Vienna} and \cite[Proposition 4.21]{DG18}.
\end{remark}

If $B$ admits a CW structure with finitely many cells in each dimension, so that the homotopy groups of $\MTH$ are
finitely generated, then
\begin{equation*}
	\Tors([\MTH, \Sigma^{n+1}\IZ])\cong \Tors(\Hom(\pi_n(\MTH),\CC^\times)) \period
\end{equation*}
Thus we have identified $\Tors([\MTH, \Sigma^{n+1}\IZ])$ with the group of torsion
$\CC^\times$-valued bordism invariants for $n$-dimensional $H$-manifolds.  Given such a bordism invariant
$\varphi$, it is possible to choose a reflection-positive invertible TFT $Z$ in the component of
$\pi_0(\mathrm{ITFT}s)$ corresponding to $\varphi$ such that the partition function of $Z$ is equal to $\varphi$.

\begin{example}[{(Classical Dijkgraaf--Witten theory \cite{DW90, FQ93})}]
	\label{classical_DW}
	\index[terminology]{Dijkgraaf--Witten theory!classical}
	Let $G$ be a group and $\lambda\in \H^n(\BG;\QQ/\ZZ)$. Then $\lambda$ defines a bordism invariant of oriented
	$n$-manifolds $M$ with a principal $G$-bundle $P$ by integrating, then exponentiating:
	\begin{equation}
		(M, P)\longmapsto \exp\paren{2\pi i \int_M \lambda(P)}\in\CC^\times\comma
	\end{equation}
	where $\lambda(P)$ denotes the pullback of $\lambda$ along the map $M\to \BG$ defined by $P$. Stokes' theorem
	\index[terminology]{Stokes' theorem}
	implies this is a bordism invariant, and it is torsion; therefore~\eqref{classical_DW} is the partition
	function of a unique (up to isomorphism) reflection-positive invertible TFT. This TFT is called
	\textit{classical Dijkgraaf--Witten theory}. The state space assigned to any codimension-$1$ manifold is
	noncanonically isomorphic to $\CC$; see Freed--Quinn \cite[\S 1]{FQ93} for a fuller description and
	Yonekura \cite[\S 4]{Yon19} for another construction.
\end{example}

\begin{example}[(Arf theory)]
	\label{arf_TFT}
	\index[terminology]{Arf theory}
	\index[terminology]{Arf invariant}
	We have $\Omega_2^{\Spin}\cong\ZZ/2$, and the \textit{Arf invariant} is a complete invariant
	\begin{equation*}
		\Arf\colon \Omega_2^{\Spin}\to\{\pm 1\}
	\end{equation*}
	\cite[Proposition (4.1)]{Ati71}. 
	Using Freed--Hopkins' classification, there is a
	reflection-positive invertible TFT $Z_A\colon\Bord_2^{\Spin}\to\fC$, called the \textit{Arf theory}, whose
	partition function is the Arf invariant, and $Z_A$ is unique up to isomorphism. Gunningham \cite[Example
	2.19]{Gun16} showed that we can take $\fC$ to be $\categ{sAlg}_\CC$, the Morita bicategory of complex
	superalgebras.
	\index[notation]{sAlgC@$\categ{sAlg}_\CC$}
	\index[terminology]{Morita bicategory}

	As in \cref{classical_DW}, we can recast this example as integration, this time in generalized cohomology.
	Atiyah--Bott--Shapiro \cite{ABS64} showed that spin manifolds admit pushforward maps for $\KO$-theory. On a
	spin surface, the partition function of the Arf theory (i.e.\ the Arf invariant) is the pushforward
	\index[terminology]{Atiyah--Bott--Shapiro map}
	\begin{equation}
	\begin{aligned}
			\exp2 \pi i\int_\Sigma^{\KO}\colon \KO^0(\Sigma) &\longrightarrow \KO^{-2}(\pt)\cong\{\pm 1\}\\
		1 &\longmapsto Z_A(\Sigma)
	\end{aligned}
	\end{equation}
	That is, the $\KO$-theoretic pushforward lands in $\ZZ/2$, and exponentiation brings us to $\{\pm
	1\}\subset\CC^\times$.

	Something similar also works in positive codimension! Let $C$ be a closed spin $1$-manifold.
	\begin{equation}
	\label{curve_KO_int}
	\begin{aligned}
		\int_C^{\KO}\colon \KO^0(C) &\longrightarrow \KO^{-1}(\pt)\cong\ZZ/2\\
		1 &\longmapsto Z_A(C).
	\end{aligned}
	\end{equation}
	This $\ZZ/2$ is different --- we interpret it as the group of isomorphism classes of complex super lines
	$\{\CC, \Pi\CC\}$ under tensor product. That is, an invertible field theory valued in $\categ{sAlg}_\CC$
	assigns to a codimension-$1$ manifold a $\otimes$-invertible complex super vector space; up to isomorphism this
	is either the even line or the odd line, and~\eqref{curve_KO_int} tells us which one the Arf theory assigns to
	$C$. For example, the bounding spin circle is assigned an even line, and the nonbounding spin circle is
	assigned an odd line.\index[terminology]{super line}\index[terminology]{super vector
	space}\index[terminology]{bounding spin circle}\index[terminology]{nonbounding spin circle}
\end{example}

When we turn to non-topological invertible field theories, these integrals will use differential (generalized)
cohomology.
\subsection{Non-topological invertible field theories}
\label{non_top_field_theory}
Using reflection-positive invertible TFTs, we saw the torsion subgroup of $[\MTH, \Sigma^{n+1}\IZ]$.
Freed--Hopkins \cite[\S 5.4]{FH21} go further and conjecture that the entire group classifies reflection-positive
invertible field theories that are not necessarily topological. At present, it is not clear how to define these
field theories. But Freed--Hopkins predict what the partition functions of these theories should be, which is a
differential-cohomological lift of the topological story, where we had bordism invariants. We follow
Freed \cite[Lecture 9]{Fre19} and Freed--Hopkins \cite[\S 5.4]{FH21} in this section.
\begin{defn}
A \textit{differential $H_n$-structure} on a smooth manifold $M$ is
\index[terminology]{differential Hn-structure@differential $H_n$-structure}
\begin{enumerate}[(1)]
	\item a Riemannian metric on $M$,
	\item an $H$-structure in the sense above, i.e.\ a principal $H_n$-bundle $P\to M$ with an isomorphism
	$\theta\colon P\times_{H_n}\Or_n\isomorphism\mathcal B_{\Or}(M)$, and
	\item a connection $\conn$ on $P$ whose induced connection under $\theta$ is the Levi-Civita connection for
	the metric.
\end{enumerate}
\end{defn}
A differential $H_n$-structure on $M$ induces a differential $H_n$-structure on a collar neighborhood of $\partial
M$, so analogously to $\Bord_n^H$, there should be a ``geometric bordism category'' $\Bord_n^{H_n, \nabla}$. Then
one should be able to define field theories as symmetric monoidal functors from $\Bord_n^{H_n, \nabla}$ to
something like a category of topological vector spaces, and define invertibility as above.
\index[terminology]{bordism category!geometric}
\index[terminology]{geometric bordism category|see {bordism category!geometric}}
Following ideas of Atiyah, Kontsevich, and Segal \cite{Seg11}, various geometric versions of bordism categories
have been constructed or sketched by
Cheung \cite{Che07},
Ayala \cite{Aya09},
Hohnhold--Stolz--Teichner \cite[\S 6.2]{HST10},
Hohnhold--Kreck--Stolz--Teichner \cite[\S 5.2]{HKST11},
Stolz--Teichner \cite{ST11},
Tachikawa \cite[\S 1]{Tac13},
Schommer-Pries--Stapleton \cite[\S 7]{SS14},
Kandel \cite{Kan16},
Grady--Sati \cite[\S 5.2]{GS17},
Ulrickson \cite[\S 2.1.2]{Ulr17},
Müller--Szabo \cite[\S 2.1]{MS18},
Grady--Pavlov \cite[\S 4.2]{GP20},
Ludewig--Stoffel \cite[\S 3]{LS20}, and
Kontsevich--Segal \cite{KS21}; Müller--Szabo use their model to study examples of invertible,
non-topological field theories.
\begin{conjecture}[{(Freed--Hopkins \cite[Conjecture 8.37]{FH21})}]
\label{nontopinv}
	There is an isomorphism of abelian groups from $\pi_0$ of the space reflection-positive, invertible,
	$n$-dimensional field theories to $[\MTH, \Sigma^{n+1}\IZ]$.
\end{conjecture}
Key to this conjecture is formulating a good definition of invertible, non-topological field theory. In the rest of
this section, we assume the conjecture is true, which in particular means finding a definition.

% conjectural description of partition function: a differential H_n-structure means we can integrate classes in
% \check IZ(BH). maybe should phrase parallelly
This conjecture includes a prediction for the value of the partition function of an invertible field theory given
by $\varphi\in\Map(\MTH, \Sigma^{n+1}\IZ)$. An $H$-manifold $M$ gives a point in $\MTH$, i.e.\ a map
$M\colon \Sigma^n\mathbb S\to\MTH$. Composing with $\varphi$ and desuspending, we have a map $\mathbb
S\to\Sigma\IZ$; its homotopy class is an element of $\IZ^1(\pt) = \pi_{-1}\IZ = 0$, so this construction is not
very interesting. But conjecturally, a differential refinement of this procedure takes a manifold $M$ with a
differential $H_n$-structure and obtains an element $\varphi(M)\in \IZhat^1(\pt)\cong\RR/\ZZ$; then the partition
function of the corresponding invertible field theory is predicted to be $\exp(2\pi i\varphi(M))$. See
Hopkins--Singer \cite[\S 5.1]{HopkinsSinger} for a construction which adopts this perspective; they in particular
construct the differential refinement $\IZhat$ of $\IZ$, by using that $\HZZ\to\IZ$ is a rational equivalence.
Yamashita--Yonekura \cite{YY21} take another approach, directly constructing a differential refinement of
$\mathrm{Map}(\MTH, \Sigma^2\IZ)$ and using it to access the partition functions of these conjectured field
theories.
\index[terminology]{Anderson dual of the sphere spectrum!differential refinement}

Often there is a simpler description. Assume $\varphi$ can be identified with the element of the group $\Hom(\Omega_{n+1}^H,
\ZZ)$ given by integrating a (generalized) cohomology class $c$. Then the partition function of the theory
associated to $\varphi$ is the secondary invariant associated to $c$, as defined in \cref{secondary_invariants}.
\index[terminology]{secondary invariant}
\begin{example}[(Classical Chern--Simons theory)]
\label{classical_CS}
\index[terminology]{Chern--Simons theory!classical}
The Chern--Simons invariants we discussed above in \cref{cs_invariants} fit together into an invertible,
non-topological field theory which is a differential analogue of \cref{classical_DW}. Fix a compact Lie group and a
\textit{level} $\lambda\in\H^4(\BG;\ZZ)$. Assume $\lambda$ is not torsion. Since $G$ is compact, the Chern--Weil map
is an isomorphism, so as in \cref{DifferentialCharacteristicClasses}, $\lambda$ refines to a class
$\hat \lambda\in \Hhat^4(\BnablaG;\ZZ)$.

The level $\lambda$ defines an element of $\Hom(\Omega_4^{\SO}(\BG);\ZZ)$: send an oriented $4$-manifold $X$ with
principal $G$-bundle $P\to M$ to the integer $\int_M \lambda(P)$, where $\lambda(P)$ denotes the pullback of
$\lambda$ along the homotopy class of maps $M\to \BG$ defined by $P$. Again, Stokes' theorem is why this is a
bordism invariant. According to \cref{nontopinv}, this bordism invariant determines (up to isomorphism) an
invertible field theory for $3$-manifolds with a differential $\SO_3\times G$-structure. This field theory is
classical Chern--Simons theory \cite{Fre95, Fre02, Gom01}
\begin{equation}
	\alpha_{(G, \lambda)}\colon\Bord_3^{\SO\times G, \nabla}\longrightarrow \Line_\CC.
\end{equation}
Let $Y$ be a closed $3$-manifold with a differential $\SO\times G$-structure, which means an orientation, a
Riemannian metric, a principal $G$-bundle $P\to Y$, and a connection $\conn$ for $P$. The data of $(P,\conn)$
gives us a map $Y\to\BnablaG$, allowing us to pull $\hat \lambda$ back to $Y$, and the orientation allows us to
integrate differential cohomology classes, as in \cref{FiberIntegration}. The
partition function of $\alpha_{(G, \lambda)}$ is $\exp\paren{2\pi i \int_Y \hat \lambda(P, \conn)}$, which is
exactly the exponentiated Chern--Simons invariant of $(P, \conn)$, as we established in~\eqref{CS_secondary_CW}:
\begin{equation}
\begin{aligned}
	\exp2\pi i\int_Y\colon \Hhat^4(Y) &\longrightarrow \Hhat^1(\pt)\to \CC^\times\\
	\hat \lambda(P, \conn) &\longmapsto \exp\paren{2\pi i\CS_\lambda(P, \conn)}.
\end{aligned}
\end{equation}
That is, $\Hhat^1(\pt)\cong\RR/\ZZ$, and exponentiating gets us to $\CC^\times$.

On a closed, oriented surface $\Sigma$ with a Riemannian metric, principal $G$-bundle $P\to\Sigma$, and connection
$\conn$, $\alpha_{(G, \lambda)}$ again assigns the pushforward of $\hat \lambda(P, \conn)$, but this time the
pushforward map has signature
\begin{equation}
\int_\Sigma\colon \Hhat^4(\Sigma)\longrightarrow \Hhat^2(\pt)\cong\Line_\CC,
\end{equation}
which sends $\hat \lambda(P, \conn)$ to the Chern--Simons line constructed in, e.g., \cite[\S 4]{Fre95}. This
story continues in extended TFT, assigning higher-categorical objects to lower-dimensional manifolds, such as
in \cite{Gom01a}.

See also Fiorenza--Sati--Schreiber \cite{FSS15a} and Yamashita--Yonekura \cite[Example 4.81 and Proposition
6.3]{YY21} for additional constructions of classical Chern--Simons theory as an invertible field theory, and
Freed--Neitzke \cite{FN20} for an application to special functions.
\end{example}
\begin{remark}[(Quantizing Chern--Simons theory)]
\label{quantum_CS}
One of the interesting things you can do with the classical Chern--Simons theory is to quantize it. This amounts to
summing $\alpha_{(G, \lambda)}$ over the space of all principal $G$-bundles with connection on a given closed,
oriented $3$-manifold. This procedure, known as taking the path integral, is still only heuristically
defined,\footnote{When $G$ is finite, Freed--Quinn \cite{FQ93} define a path integral of \emph{topological} field
theories whose fields include a principal $G$-bundle. Applied to classical Dijkgraaf--Witten theory from
\cref{classical_DW}, the resulting TFT, called \textit{(quantum) Dijkgraaf--Witten theory}, is a commonly studied
model organism in topological field theory.} but enough is known about it in the physics literature that we can ask
mathematical questions about the quantized theory. In physics, this quantum Chern--Simons theory was first studied
by Schwarz \cite{Sch77} and Witten \cite{Wit89}.%
\index[terminology]{Chern--Simons theory}%
\index[terminology]{Dijkgraaf--Witten theory}%
\index[terminology]{path integral}

Something strange happens in this quantization procedure, though: Witten (\textit{ibid.}) gives a physical argument
that quantum Chern--Simons theory is in fact a topological field theory! Therefore it should be possible to
formalize it mathematically as a symmetric monoidal functor 
\begin{equation}
	Z_{G,k}\colon\Bord_3^{\SO}\longrightarrow \fC,
\end{equation}
where $\fC$ is some symmetric monoidal $(\infty, 3)$-category. It is not known how to do this in
general,\footnote{There are a few different perspectives on what $Z_{G,k}(\mathrm{pt}_+)$ should be. For $G$
finite, the answer is known by work of Freed--Hopkins--Lurie--Teleman \cite[\S 4.2]{FHLT10} and Wray \cite[\S
9]{Wra10}; for $G$ a torus, the answer is due to Freed--Hopkins--Lurie--Teleman (\textit{ibid}.). For general $G$,
two different approaches are provided by Freed--Teleman (see \cite{432876}) and Henriques \cite{Hen17b, Hen17a}.
See also \cite{FT20}.}
but it is known how to extend it to a theory of $1$-, $2$-, and $3$-manifolds, valued in the $2$-category of
$\mathbb C$-linear categories, by work of Reshetikhin--Turaev \cite{RT90, RT91}, Walker \cite{Wal91},
Bakalov--Kirillov \cite{BK01}, Kerler--Lyubashenko \cite{KL01}, and
Bartlett--Douglas--Schommer-Pries--Vicary \cite{BDSV15}.\footnote{These constructions require some additional
structure on our manifolds, such as a choice of trivialization of the first Pontryagin class. As theories of merely
oriented manifolds, Chern--Simons theories are \emph{anomalous}. See \cites[\S 9.3]{FHLT10}[]{432876} for more
information.} Much more can be said about this TFT and its connections to various parts of
geometry, topology, representation theory, and physics; see Freed \cite{Fre09} for a
general survey on Chern--Simons theory and the references therein for more information.
\end{remark}

\begin{example}[(Classical Wess--Zumino--Witten theory)]\label{ex-WZW}
	This example is related to the previous example, but with a slightly different flavor. Let $G$ be a compact Lie
	group and $\hat h\in \Hhat^3(G;\ZZ)$. If $h \colonequals \cc(\hat{h})$ (the image of $\hat{h}$ under the characteristic class map of \Cref{cons:charclassmap}),
	then $h$ defines a bordism invariant of oriented $3$-manifolds $M$ with a map $\psi\colon M\to G$:
	\begin{equation}
	\begin{aligned}
		\Omega_3^{\SO}(G) &\longrightarrow \ZZ\\
		(M,\psi) &\longmapsto\int_M \psi^*(h).
	\end{aligned}
	\end{equation}
	\Cref{nontopinv} therefore says there is a two-dimensional invertible field theory $\beta_{G, h}$ whose
	partition function is the secondary invariant associated to $\hat h$. This theory is called \textit{classical
	Wess--Zumino--Witten (WZW) theory}; it was originally studied by Witten \cite{Wit83}, following
	Wess--Zumino \cite{WZ71}. See Freed \cite[Appendix A]{Fre95} for a discussion of the classical theory
	specifically.\footnote{There are considerably more general objects studied in quantum physics under the name
	``Wess--Zumino--Witten theory'' or ``Wess--Zumino--Witten term.'' See \cites[\S 6]{DF99}{Fre08}[\S
	5.6]{Urs}{FSS15b}{LOT20}{Yon20} for some examples taking an algebro-topological viewpoint.}%
	\index[terminology]{secondary invariant}%
	\index[terminology]{Wess--Zumino--Witten model}%
	\index[terminology]{Wess--Zumino--Witten model!classical}

	As part of a trend you may have noticed by now, the original description of the classical WZW partition function
	$\int_M \psi^*(\hat h)$ was not phrased in this way; the connection with differential cohomology is due to
	Gawędzki \cite{Gaw88}. For a moment assume that $G$ is connected, simple, and simply connected, so that
	$\H^3(G;\ZZ)\cong\ZZ$. Let $\theta\in\Omega^1(G; \g)$ be the Maurer--Cartan form,
	\index[terminology]{Maurer--Cartan form}
	which is defined to assign to a tangent vector $v\in T_gG$ the Lie algebra element canonically identified to it. As
	mentioned in \cref{transgression_detail}, the transgression map $\tau^{-1}\colon \H^3(G;\ZZ)\to \H^4(\BG;\ZZ)$ is
	an isomorphism; since $G$ is compact, the Chern--Weil machine associates to $\tau^{-1}(h)$ (or rather, its image in
	$\RR$-valued cohomology) a degree-two invariant polynomial $f$. In this case, the Wess--Zumino--Witten action is%
	\index[terminology]{transgression}
	\begin{equation}
		\beta_{G,h}(M, \psi) = \int_M -\frac 16\psi^*(f(\theta\wedge [\theta, \theta])).
	\end{equation}
	The differential refinement of $\tau\colon \H^4(\BG;\ZZ)\to\H^3(G;\ZZ)$ constructed by
	Carey--Johnson--Mur\-ray--Stevenson--Wang \cite[\S 3]{CJMSW05} and Schreiber \cite[1.4.1.2]{Urs} can be thought of as
	starting with a classical Chern--Simons theory and obtaining a classical Wess--Zumino--Witten theory in one
	dimension lower.
\end{example}

\begin{remark}[(Quantizing the Wess--Zumino--Witten model)]
	\label{quantum_WZW}
	Just as in \cref{quantum_CS}, it is possible to quantize the classical WZW model, at least at a physical level of
	rigor: one sums over the space of maps to $G$. The result is called the quantum Wess--Zumino--Witten model, or just
	the Wess--Zumino--Witten or WZW model.\index[terminology]{Wess--Zumino--Witten model!quantum} This theory is a
	conformal field theory,~\index[terminology]{conformal field theory} meaning its value on a manifold depends only
	on the conformal class of the Riemannian metric. Some of what we do in the next two chapters, involving the
	representation theory of loop groups, is related to the WZW model.

	Given a level $h\in\Hhat^4(\BnablaG;\ZZ)$, there is a (quantum) Chern--Simons theory and a quantum WZW model
	(obtained by transgressing $h$ to $\Hhat^3(G;\ZZ)$), and the two are related: the WZW model is a boundary theory
	for the Chern--Simons theory. There are different ways of formulating this precisely: one uses \textit{relative
	field theory}\index[terminology]{relative field theory} \cite{FT12}. In this formalism, the bulk theory $\alpha$
	is a symmetric monoidal functor out of a bordism category, and its boundary theory $Z$ is a natural transformation
	from (a truncation of) $\alpha$ to the trivial field theory. Among other things, this implies that the partition
	function of $Z$ on an $(n-1)$-manifold $M$ is not a number, but an element of the state space $\alpha(M)$; when
	$\alpha$ is Chern--Simons theory and $Z$ is the WZW model, this fact was first noticed by Witten \cite{Wit89}.
	See Gwilliam--Rabinovich--Williams \cite{GRW20} for another approach to this bulk-boundary correspondence, in the
	language of factorization algebras.
\end{remark}


\begin{example}[(Exponentiated $\eta$-invariants)]
We give a differential analogue of \cref{arf_TFT}: in that example, we used the Atiyah--Bott--Shapiro
pushforward \cite{ABS64} in $\KO$-theory to produce a torsion bordism invariant, hence an invertible topological
field theory. Here we will use the same pushforward to produce a nontorsion bordism invariant, hence an invertible,
non-topological field theory. This theory is discussed by Freed \cite[Example 9.24]{Fre19}.
\index[terminology]{Atiyah--Bott--Shapiro map}

The bordism invariant in question is the \emph{$\Ahat$-genus} $\Ahat\colon
\Omega_4^{\Spin}\to\ZZ$,\footnote{$\Ahat$ is pronounced ``$A$-hat'' or ``$A$-roof.'' This gives rise to the
following joke: A man walks into a bar with a dog and says to the bartender, ``This is a talking dog. I'll bet you
a drink he can answer a question.''

The bartender says, ``Sure. Ok dog, what's your favorite spin bordism invariant?''

``Arf!''

``\dots''

``Clifford, how about a different one?''

``A-roof!''

(they get thrown out)

The dog looks at the man and says, ``Ok fine, next time I'll say `index of the Dirac operator.' ''}
\index[terminology]{A-genus@$\Ahat$-genus}
\index[notation]{Ahat@$\Ahat$}
which, like the Arf invariant, is a pushforward in $\KO$-theory: for a closed spin $4$-manifold $X$, we have
\begin{equation}
\begin{aligned}
	\int_X^{\KO}\colon \KO^0(X) &\longrightarrow \KO^{-4}(\pt)\cong\ZZ\\
	1 &\longmapsto \Ahat(X).
\end{aligned}
\end{equation}
This is nonvanishing on the K3 surface,\index[terminology]{K3 surface} hence nontorsion. By Freed--Hopkins'
conjecture, this bordism invariant corresponds to some invertible, non-topological field theory on $3$-dimensional
differential spin manifolds (i.e.\ $3$-manifolds with a spin structure and a Riemannian metric):
\begin{equation}
	\alpha'\colon\Bord_3^{\Spin, \nabla}\longrightarrow \sLine_\CC.
\end{equation}
And analogously to the Arf theory, we can describe the value of $\alpha'$ on closed $2$- and $3$-manifolds with
differential spin structure using the pushforward in differential $\KO$-theory. Grady--Sati \cite[\S 4.3]{GS21}
construct this pushforward for a closed spin manifold; using this, the partition function of $\alpha'$ on a closed
spin Riemannian $3$-manifold $Y$ is
\index[terminology]{differential KO-theory@differential $\KO$-theory}
\begin{equation}
\begin{aligned}
	\exp2\pi i\int_Y^{\KOhat}\colon \KOhat^0(Y) &\longrightarrow \KOhat^{-3}(\pt)\to
	\CC^\times\\
	1 &\longmapsto \alpha'(Y) \comma
\end{aligned}
\end{equation}
where as usual $\KOhat^{-3}(\pt)\cong\RR/\ZZ$, and we exponentiate to obtain the partition function in
$\CC^\times$. The isomorphism type of the state space assigned to a closed spin Riemannian $2$-manifold $\Sigma$ is
in a similar way the image of $1$ under the pushforward
$\KOhat^0(\Sigma)\to\KOhat^{-2}(\pt)\cong\ZZ/2$, corresponding to the two isomorphism classes of
complex super lines, $\CC$ and $\Pi\CC$.

Like in \cref{classical_CS}, the partition function of $Y$ also has a more geometric description. A differential
spin structure is the data needed to define the Dirac operator on the spinor bundle of $Y$, and index-theoretic
methods allow one to extract an \textit{exponentiated $\eta$-invariant} from this Dirac operator, as constructed by
Atiyah--Patodi--Singer \cite{APS1, APS2, APS3}. The Dai--Freed theorem \cite{DF94} proves this exponentiated
$\eta$-invariant satisfies a gluing law which can be interpreted as implying that $\alpha'$ is symmetric monoidal.
\index[terminology]{Dai--Freed theorem}
\index[terminology]{Dirac operator}
\index[terminology]{eta-invariant@$\eta$-invariant}
\end{example}
For more examples of invertible, non-topological field theories and their relationship to differential cohomology,
see Monnier \cites[\S 4]{Mon15}[\S 5]{Mon17}{Mon18}, Monnier--Moore \cite{MM19},
Córdova--Freed--Lam--Seiberg \cite[\S\S 6.2 \& 7]{CFLS20a}, and
Yamashita--Yonekura \cite[\S\S 4.2, 6]{YY21}.
%Invertible TFTs are very special examples of TFTs: they are almost trivial but not necessarily trivial. Their
%classification is correspondingly much easier. This classification is originally due to
%Freed--Hopkins--Teleman \cite{FHT10}.
%\begin{defn}
%A \textit{Picard $n$-groupoid} is a symmetric monoidal $n$-category such that all objects are $\otimes$-invertible
%and all $k$-morphisms are composition-invertible for all $k$.
%\end{defn}
%Inside a symmetric monoidal $(\infty, n)$-category $\fC$ one can consider the subcategory of invertible objects and
%(higher) morphisms. This is called the \textit{Picard groupoid of units} of $\fC$ and denoted $\fC^\times$.  In
%particular, a TFT $Z\colon\Bord_n^\xi\to\fC$ is invertible if and only if it factors through $\fC^\times\inj\fC$;
%in this case it can also be extended to the \textit{Picard groupoid completion} of $\Bord_n^\xi$, defined by adding
%formal inverses to all objects and (higher) morphisms in $\Bord_n^\xi$, then extending $Z$ by $Z(x^{-1})\colonequals
%(x)^{-1}$. Therefore to classify invertible TFTs it suffices to understand these morphisms of Picard
%$\infty$-groupoids.
%
%If $X$ is a Picard $n$-groupoid, the geometric realization of the nerve of $X$ is an $E_\infty$-space (using the
%symmetric monoidal structure) and is grouplike (because all objects in $X$ are $\otimes$-invertible). It therefore
%defines a connective spectrum which we denote $\abs X$ and call the \textit{classifying spectrum} of $X$.
%\todo[inline]{Figure out what category number I want!}
%\begin{thm}[Stable homotopy hypothesis \cite{MOPSV20}]
%Taking the classifying spectrum defines an equivalence of $\infty$-categories from Picard $\infty$-groupoids to
%connective spectra.
%\end{thm}
%TODO\colon \cite{MOPSV20} only prove this for $n$-categories, and as stated I want $\infty$-categories.
%
%So it suffices to identify the classifying spectra of $\fC^\times$ and of the Picard groupoid completion of
%$\Bord_n^\xi$.
%\begin{defn}
%Let $\xi_n\colon B_n\to B\Or_n$ be the pullback
%% https://q.uiver.app/?q=WzAsNCxbMCwwLCJCX24iXSxbMSwwLCJCIl0sWzAsMSwiQlxcbWF0aHJtIE9fbiJdLFsxLDEsIkJcXG1hdGhybSBPIl0sWzAsMV0sWzAsMiwiXFx4aV9uIl0sWzEsMywiXFx4aSJdLFsyLDNdXQ==
%\[\begin{tikzcd}
%	{B_n} & B \\
%	{B\mathrm O_n} & {B\mathrm O,}
%	\arrow[from=1-1, to=1-2]
%	\arrow["{\xi_n}", from=1-1, to=2-1]
%	\arrow["\xi", from=1-2, to=2-2]
%	\arrow[from=2-1, to=2-2]
%\end{tikzcd}\]
%and let $V_n\to B\Or_n$ denote the tautological bundle. The \textit{Madsen-Tillmann spectrum} (TODO: cite) $\MTxi_n$
%is the Thom spectrum of the virtual bundle $\xi_n^*(-V_n)\to B_n$.
%\end{defn}
%In general, $\MTxi_n$ is not connective; its homotopy groups are supported in $[-n, \infty)$. TODO: refs below.
%\begin{thm}[Galatius--Madsen--Tillmann--Weiss, Nguyen, Schommer-Pries]
%The classifying spectrum of the Picard groupoid completion of $\Bord_n^\xi$ is $\Sigma^n\MTxi_n$.
%\end{thm}
%Therefore the classification of invertible TFTs is $[\Sigma^n\MTxi_n, \abs{\fC^\times}]$. For common choices of
%$\fC$, $\abs{\fC^\times}$ is something like $\Sigma^n H\CC^\times$, so invertible TFTs are classified using
%cohomology, or the connective cover of $\Sigma^nI\CC^\times$.
