%!TEX root = ../diffcoh.tex

%-------------------------------------------------------------------%
%-------------------------------------------------------------------%
%  ℝ-localization                                                   %
%-------------------------------------------------------------------%
%-------------------------------------------------------------------% 

\section{\texorpdfstring{$ \RR $}{ℝ}-localization}\label{sec:localization}
\textit{by Peter Haine}

The purpose of this chapter is to provide formulas for the $ \RR $-localization functor
\begin{equation*}
	\LRR \colon \fromto{\PSh(\Man;C)}{\PShhi(\Man;C)}
\end{equation*}
(\Cref{homotopification_definition}) and left adjoint $ \Gammalowershriek \colon \fromto{\Sh(\Man;C)}{C} $ to the constant sheaf functor \cref{nul:Gammaadjunctions}.
Specifically, write 
\begin{equation*}
	\Deltaalg^n \colonequals \setbar{(t_0,\ldots,t_{n}) \in \RR^{n+1}}{t_0 + \cdots + t_{n} = 1} \subset \RR^{n+1} 
\end{equation*}
for the algebraic $ n $-simplex; the assignment $ \goesto{[n]}{\Deltaalg^n} $ defines a cosimplicial manifold.
We show that $ \LRR $ and $ \Gammalowershriek $ are computed by the geometric realizations
\begin{equation*}
	\LRR(F)(M) \equivalent \real{F(M \cross \Deltaalgdot)} \andeq \Gammalowershriek(E) \equivalent \real{E(\Deltaalgdot)}
\end{equation*}
(\Cref{prop:MorelVoevodsky,cor:Gammalowershriek}).

In \cref{sec:MVSconstruction}, we give a precise statement of the main result of this section (\Cref{prop:MorelVoevodsky}), but do not prove it.
We then explain some consequences of these formulas (\cref{subsubsec:consequencesofMSV}). 
Of particular interest is that given a Lie group $ G $, one can recover the classifying space $ \BG $ by applying $ \Gammalowershriek $ to the sheaf $ \BunG $ sending a manifold $ M $ to the groupoid of principal $ G $-bundles over $ M $ (\Cref{cor:GammalowershriekBunG}).
In \cref{sec:simplicialhomotopy}, we recall some background on simplicial homotopies in \categories that we need to prove the formula for $ \LRR $.
\Cref{subsec:proofofMVS} is dedicated to proving this formula.

%-------------------------------------------------------------------%
%-------------------------------------------------------------------%
%  The Morel–Suslin–Voevodsky construction                          %
%-------------------------------------------------------------------%
%-------------------------------------------------------------------%

\subsection{The Morel--Suslin--Voevodsky construction}\label{sec:MVSconstruction}

%-------------------------------------------------------------------%
%  The construction                                                 %
%-------------------------------------------------------------------%

\subsubsection{The construction}

\begin{notation}
	Let $ n \geq 0 $ be an integer.
	Write $ \Deltaalg^n $ for the hyperplane in $ \RR^{n+1} $ defined by
	\begin{equation*}
		\Deltaalg^n \colonequals \setbar{(t_0,\ldots,t_{n}) \in \RR^{n+1}}{t_0 + \cdots + t_{n} = 1} \subset \RR^{n+1} \comma
	\end{equation*}
	so that as a smooth manifold $ \Deltaalg^n $ is diffeomorphic to $ \RR^{n} $.
	We call $ \Deltaalg^n $ the \textit{algebraic $ n $-simplex}.

	In the usual way, the algebraic $ n $-simplices for $ n \geq 0 $ assemble into a cosimplicial manifold
	\begin{equation*}
		\Deltaalgdot \colon \fromto{\Deltabf}{\Man} \period
	\end{equation*}
\end{notation}

\begin{proposition}[(Morel--Suslin--Voevodsky construction)]\label{prop:MorelVoevodsky}
	Let $ C $ be a presentable \category. 
	The left adjoint
	\begin{equation*}
		\LRR \colon \fromto{\PSh(\Man;C)}{\PShhi(\Man;C)}
	\end{equation*}
	is given by the geometric realization
	\begin{equation*}
		\LRR(F)(M) \colonequals \real{F(M \cross \Deltaalgdot)} \period
	\end{equation*}
\end{proposition}

\begin{remark}
	We call the construction $ \goesto{F}{\real{F(- \cross \Deltaalgdot)}} $ the \textit{Morel--Suslin--Voevodsky} construction.
	Morel and Voevodsky provide a very general version of the Morel--Suslin--Voevodsky construction for ``sites
	with an interval object'' \cite[\S 2.3]{MR1813224}, which covers the site $ \Man $ with $ \RR $ as the interval object (see also \cites[\S 4.3]{MR3727503}[\S 4]{MR3679884}).
	They attribute this argument to Suslin.

	Their arguments are model category-theoretic and apply to a more specific coefficient \categories $ C $ than we're interested in.
	Hence we provide separate argument.
	So as to not take us too far afield, we settle for working with the site of manifolds rather than a general site with an interval object.
	Our proof of \Cref{prop:MorelVoevodsky} takes the approach used in Brazelton's notes on motivic homotopy theory \cite[\S 3]{Brazelton:A1}.
\end{remark}

%-------------------------------------------------------------------%
%  Consequences of the Morel–Suslin–Voevodsky construction          %
%-------------------------------------------------------------------%

\subsubsection{Consequences of the Morel--Suslin--Voevodsky construction}\label{subsubsec:consequencesofMSV}

We defer the proof of \Cref{prop:MorelVoevodsky} to \cref{subsec:proofofMVS,sec:simplicialhomotopy} and first explain why \Cref{prop:MorelVoevodsky} gives formulas for $ \Gammalowershriek $ and $ \Lhi $.
To do this, we need the following fact; its proof is a bit of a technical digression, so we defer it to \Cref{app:technicaldeatails}.

\begin{proposition}[(\Cref{app.prop:sheafificationofinvariantisinvariant})]\label{prop:sheafificationofinvariantisinvariant}
	Let $ C $ be a presentable \category.
	Then for every $ \RR $-invariant presheaf $ F \colon \fromto{\Manop}{C} $, the counit $ \fromto{\Gammaupperstar \Gammalowerstar \SMan F}{\SMan F} $ is an equivalence.
	In particular, $ \SMan F $ is $ \RR $-invariant.
\end{proposition}

\noindent \Cref{prop:sheafificationofinvariantisinvariant} immediately gives a description of the homotopification functor $ \Lhi $ in terms of the $ \RR $-localization functor for presheaves.

\begin{corollary}\label{cor:hilowershriekasacomposite}
	Let $ C $ be a presentable \category.
	Then the composite
	\begin{equation*}
		\SMan \LRR \colon \fromto{\Sh(\Man;C)}{\Sh(\Man;C)}
	\end{equation*}
	factors through $ \Shhi(\Man;C) $ and is left adjoint to the inclusion $ \incto{\Shhi(\Man;C)}{\Sh(\Man;C)} $.
	That is, $ \Lhi \equivalent \SMan \LRR $.
\end{corollary}

\begin{corollary}\label{cor:Gammalowershriek}
	Let $ C $ be a presentable \category.
	The left adjoint $ \Gammalowershriek \colon \fromto{\Sh(\Man;C)}{C} $ to the constant sheaf functor is given by
	\begin{equation*}
		\Gammalowershriek(E) \equivalent \real{E(\Deltaalgdot)} \period
	\end{equation*}
\end{corollary}

\begin{proof}
	By \Cref{cor:hilowershriekasacomposite} and the identification $ \Gammalowerstar \Lhi \equivalent \Gammalowershriek $, it suffices to show that for every sheaf $ E $ on $ \Man $, the global sections of $ \SMan \LRR E $ are given by the geometric realization $ \real{E(\Deltaalgdot)} $.
	Since the unit
	\begin{equation*}
		\fromto{\LRR E}{\SMan \LRR E}
	\end{equation*}
	of the sheafification adjunction induces an equivalence on global sections (\Cref{cor:sheafificationofglobalsections}), the claim follows from \Cref{prop:MorelVoevodsky}.
\end{proof}

\begin{nul}
	Since $ \Lhi \equivalent \Gammaupperstar \Gammalowershriek $, \Cref{lem:constantishi,cor:Gammalowershriek} show that $ \Lhi $ is given by the formula
	\begin{equation*}
		\Lhi(E)(M) \equivalent \real{E(\Deltaalgdot)}^{\Piinf(M)} \period
	\end{equation*}
	In particular, when $ C = \Spc $, the functor $ \Lhi $ is given by the formula
	\begin{equation*}
		\Lhi(E)(M) \equivalent \Map_{\Spc}(\Piinf(M),\real{E(\Deltaalgdot)}) \period
	\end{equation*}
\end{nul}

\begin{corollary}\label{cor:Gammalowershriekpreservesprod}
	Let $ C $ be a presentable \category.
	If geometric realizations commute with finite products in $ C $ (e.g., $ C $ is \atopos), then the functor $ \Gammalowershriek \colon \fromto{\Sh(\Man;C)}{C} $ preserves finite products.
\end{corollary}

\begin{remark}
	The functors $ \LRR $, $ \Lhi $, and $ \Gammalowershriek $ do not generally commute with finite limits.
	However, general category theory \cite[Proposition 3.4]{MR3570135} shows that the functor
	\begin{equation*}
		\LRR \colon \fromto{\PSh(\Man;\Spc)}{\PShhi(\Man;\Spc)}
	\end{equation*}
	is \emph{locally cartesian}: for any cospan $ E \to G \leftarrow F $ with $ E, G \in \PShhi(\Man;\Spc) $, the natural morphism
	\begin{equation*}
		\fromto{\LRR(E \cross_G F)}{E \cross_G \LRR(F)}
	\end{equation*}
	is an equivalence.
	Since the sheafification functor $ \SMan \colon \fromto{\PSh(\Man;\Spc)}{\Sh(\Man;\Spc)} $ is left exact, \Cref{cor:hilowershriekasacomposite} shows that $ \Lhi $ and $ \Gammalowershriek $ are locally cartesian as well.
\end{remark}

We conclude this section by explaining what the functor $ \Gammalowershriek $ does to manifolds.
Let $ M $ be a manifold. 
Recall that the underlying homotopy type $ \Piinf(M) $ can be computed as the geometric realization in the \category $ \Spc $ of (a version of) the singular simplicial set 
\begin{equation*}
	[n] \mapsto \Map_{\Top}(\Deltaalg^n,M) \period
\end{equation*}
The Whitehead Approximation Theorem implies that the inclusion of simplicial sets
\begin{equation*}
	\incto{\Map_{\Mfld}(\Deltaalgdot,M)}{\Map_{\Top}(\Deltaalgdot,M)}
\end{equation*}
induces an equivalence on geometric realizations in $ \Spc $.
Hence \Cref{cor:Gammalowershriek} implies:

\begin{corollary}\label{cor:Gammalowershriekofamanifold}
	Write $ \yo \colon \incto{\Man}{\Sh(\Man;\Spc)} $ for the Yoneda embedding, and let $ M $ be a manifold.
	There is a natural equivalence
	\begin{equation*}
		\Gammalowershriek(\yo(M)) \equivalent \Piinf(M) \period
	\end{equation*}
\end{corollary}

As an application of \Cref{cor:Gammalowershriekpreservesprod,cor:Gammalowershriekofamanifold}, given a Lie group $ G $, one can show that by applying $ \Gammalowershriek $ to the sheaf $ \BunG $ sending a manifold to the groupoid of principal $ G $-bundles over it (\Cref{ex:BunG}) we recover the classifying space of $ G $.

\begin{notation}\label{ntn:classifyingspaceBG}
	Let $ G $ be a Lie group.
	We write $ \BG \in \Spc $ for the \textit{classifying space} of $ G $.
	Explicitly, $ \BG $ can be defined as the geometric realization of the simplicial space
	\begin{equation*}
		\begin{tikzcd}[sep=1.5em]
		    \cdots \arrow[r, shift left=0.75ex] \arrow[r, shift right=0.75ex] \arrow[r, shift right=2.25ex] \arrow[r, shift left=2.25ex] & \Piinf(G) \cross \Piinf(G) \arrow[l] \arrow[l, shift left=1.5ex] \arrow[l, shift right=1.5ex] \arrow[r] \arrow[r, shift left=1.5ex] \arrow[r, shift right=1.5ex] & \Piinf(G) \arrow[l, shift left=0.75ex] \arrow[l, shift right=0.75ex] \arrow[r, shift left=0.75ex] \arrow[r, shift right=0.75ex] & * \arrow[l]
		\end{tikzcd}
	\end{equation*}
	obtained by applying the underlying homotopy type functor $ \Piinf \colon \fromto{\Mfld}{\Spc} $ to the bar construction of $ G $.
\end{notation}

\begin{corollary}[{\cite[Lemma 5.2]{MR3462099}}]\label{cor:GammalowershriekBunG}
	Let $ G $ be a Lie group.
	There is a natural equivalence of spaces $ \equivto{\Gammalowershriek(\BunG)}{\BG} $.
\end{corollary}

\begin{notation}\label{ntn:BbulletGBnablaG}
	In light of \Cref{cor:GammalowershriekBunG}, following Freed--Hopkins \cite{FreedHopkins} we also denote the sheaf $ \BunG $ by $ \BbulletG $.
	Similarly, we write $ \BnablaG $ for the sheaf $ \BunGnabla $ of \Cref{ex:BunG}.
\end{notation}

%-------------------------------------------------------------------%
%-------------------------------------------------------------------%
%  Background on simplicial homotopies in ∞-categories              %
%-------------------------------------------------------------------%
%-------------------------------------------------------------------%

\subsection{Background on simplicial homotopies in \texorpdfstring{$ \infty $}{∞}-categories}\label{sec:simplicialhomotopy}

In order to prove the Morel--Suslin--Voevodsky formula (\Cref{prop:MorelVoevodsky}), we need to use homotopies of simplicial objects in an arbitrary \category.
Since we're working natively to \categories and not in simplicial sets or simplicial presheaves, doing so requires a reformulation of the usual definition of a simplicial homotopy.

%-------------------------------------------------------------------%
%  Motivation from simplicial sets                                  %
%-------------------------------------------------------------------%

\subsubsection{Motivation from simplicial sets}

Recall that a \textit{simplicial homotopy} between morphisms of simplicial sets $ f_0,f_1 \colon \fromto{\Xdot}{\Ydot} $ consists of a morphism $ h \colon \fromto{\Xdot \cross \Delta^1}{\Ydot} $ along with identifications of the restriction of $ h $ to $ \Xdot \cross \{0\} $ with $ f_0 $ and the restriction of $ h $ to $ \Xdot \cross \{1\} $ with $ f_1 $.
First we reformulate this notion in terms of morphisms in the overcategory $ \sSet_{/\Delta^1} $.

\begin{notation}
	Write $ \uupperstar \colon \fromto{\sSet}{\sSet_{/\Delta^1}} $ for the functor $ \goesto{\Xdot}{\Xdot \cross \Delta^1} $. 
	Note that $ \uupperstar $ is right adjoint to the forgetful functor $ \ulowershriek \colon \fromto{\sSet_{/\Delta^1}}{\sSet} $.
\end{notation}

\begin{lemma}\label{lem:reformshtpy}
	Let $ \Xdot $ and $ \Ydot $ be simplicial sets.
	There is a natural bijection 
	\begin{equation*}
		\Map_{\sSet}(\Xdot \cross \Delta^1,\Ydot) \isomorphic \Map_{\sSet_{/\Delta^1}}(\uupperstar(\Xdot),\uupperstar(\Ydot)) \period 
	\end{equation*}
\end{lemma}

\begin{proof}
	Since $ \ulowershriek $ is left adjoint to $ \uupperstar $, we have natural bijections
	\begin{align*}
		\Map_{\sSet_{/\Delta^1}}(\uupperstar(\Xdot),\uupperstar(\Ydot)) &\isomorphic \Map_{\sSet}(\ulowershriek\uupperstar(\Xdot),\Ydot) \\
		&= \Map_{\sSet}(\Xdot \cross \Delta^1,\Ydot) \period \qedhere
	\end{align*}
\end{proof}

In order to use \Cref{lem:reformshtpy} to generalize simplicial homotopies to arbitrary \categories, notice that the functor $ \uupperstar $ admits an alternative interpretation that makes sense for simplicial objects in any \category.

\begin{observation}[(presheaf categories and slice categories)]
	Let $ S $ be a small category and $ s \in S $.
	Write $ \yo \colon \incto{S}{\Fun(S^{\op},\Set)} $ for the Yoneda embedding.
	The colimit-preserving extension of the ``sliced Yoneda embedding''
	\begin{align*}
		S_{/s} &\inclusion \Fun(S^{\op},\Set)_{/\yo(s)} \\ 
		[s' \to s] &\mapsto [\yo(s') \to \yo(s)] \\
		\intertext{defines an equivalence of categories}
		\Fun((S_{/s})^{\op},\Set) &\equivalence \Fun(S^{\op},\Set)_{/\yo(s)} \period
	\end{align*}
	Under this identification, the functor $ \fromto{\Fun(S^{\op},\Set)}{\Fun((S_{/s})^{\op},\Set)} $ given by precomposition with the forgetful functor $ \fromto{(S_{/s})^{\op}}{S^{\op}} $ is identified with the functor
	\begin{equation*}
		\yo(s) \cross (-) \colon \fromto{\Fun(S^{\op},\Set)}{\Fun(S^{\op},\Set)_{/\yo(s)}} \period
	\end{equation*}
	Moreover, the functor $ \yo(s) \cross (-) $ is right adjoint to the forgetful functor
	\begin{equation*}
		\fromto{\Fun(S^{\op},\Set)_{/\yo(s)}}{\Fun(S^{\op},\Set)} \period
	\end{equation*}
\end{observation}

\begin{nul}
	Specializing to the case $ S = \Deltabf $ and $ s = [1] $ shows that the functor $ \uupperstar \colon \fromto{\sSet}{\sSet_{/\Delta^1}} $ is identified with the functor
	\begin{equation*}
		\Fun(\Deltaop,\Set) \to \Fun((\Deltabf_{/[1]})^{\op},\Set)
	\end{equation*}
	given by precomposition with the forgetful functor $ \fromto{(\Deltabf_{/[1]})^{\op}}{\Deltaop} $.
	We also write
	\begin{equation*}
		\uupperstar \colon \Fun(\Deltaop,\Set) \to \Fun((\Deltabf_{/[1]})^{\op},\Set)
	\end{equation*}
	for this functor.
\end{nul}

Thus, we have a further reformulation of what a simplicial homotopy is:

\begin{corollary}\label{cor:reformshtpy}
	Let $ \Xdot $ and $ \Ydot $ be simplicial sets.
	There is a natural bijection 
	\begin{equation*}
		\Map_{\sSet}(\Xdot \cross \Delta^1,\Ydot) \isomorphic \Map_{\Fun((\Deltabf_{/[1]})^{\op},\Set)}(\uupperstar(\Xdot),\uupperstar(\Ydot)) \period 
	\end{equation*}
\end{corollary}

\noindent The benefit of \Cref{cor:reformshtpy} is that the right-hand side makes sense in any \category.

\begin{notation}
	Write $ u \colon \fromto{\Deltabf_{/[1]}}{\Deltabf} $ for the forgetful functor.
	For $ i \in [1] $, write $ j_i \colon \incto{\Deltabf}{\Deltabf_{/[1]}} $ for the fully faithful functor given on objects by the assignment
	\begin{equation*}
		\goesto{[n]}{\big[[n] \to \{i\} \inclusion [1]\big]} \comma
	\end{equation*}
	with the obvious assignment on morphisms.
	Given \acategory $ D $, write
	\begin{equation*}
		\uupperstar \colon \fromto{\Fun(\Deltaop,D)}{\Fun((\Deltabf_{/[1]})^{\op},D)} \andeq \jupperstar_i \colon \fromto{\Fun((\Deltabf_{/[1]})^{\op},D)}{\Fun(\Deltaop,D)}
	\end{equation*}
	for the functors given by precomposition with $ u $ and $ j_i $, respectively.
\end{notation}

\begin{observation}\label{obs:jadjoint}
	For each $ i \in [1] $, the fully faithful functor $ j_i \colon \incto{\Deltabf}{\Deltabf_{/[1]}} $ is left adjoint to the functor $ \fromto{\Deltabf_{/[1]}}{\Deltabf} $ that sends an object $ \sigma \colon \fromto{[m]}{[1]} $ to the fiber $ \sigma^{-1}(i) $ of $ \sigma $ over $ i $ (with the induced ordering), and the obvious assignment on morphisms.
\end{observation}

\begin{definition}[{\HA{Definition}{7.2.1.6}}]\label{def:simplicialhomotopy}
	Let $ D $ be \acategory and let
	\begin{equation*}
		f_0,f_1 \colon \fromto{\Xdot}{\Ydot}
	\end{equation*}
	be morphisms in the \category $ \Fun(\Deltaop,D) $ of simplicial objects in $ D $.
	A \textit{simplicial homotopy} from $ f_0 $ to $ f_1 $ consists of the following data:
	\begin{enumerate}[(\ref*{def:simplicialhomotopy}.1)]
		\item A morphism $ h \colon \fromto{\uupperstar(\Xdot)}{\uupperstar(\Ydot)} $ in $ \Fun((\Deltabf_{/[1]})^{\op},D) $.

		\item Equivalences $ \jupperstar_0(h) \equivalent f_0 $ and $ \jupperstar_1(h) \equivalent f_1 $ of morphisms $ \fromto{\Xdot}{\Ydot} $ in $ \Fun(\Deltaop,D) $.
	\end{enumerate}

	We often write $ h \colon \fromto{\uupperstar(\Xdot)}{\uupperstar(\Ydot)} $ for the entire data of a simplicial homotopy from $ f_0 $ to $ f_1 $.
\end{definition}


%-------------------------------------------------------------------%
%  Realizations of simplicial homotopies                            %
%-------------------------------------------------------------------%

\subsubsection{Realizations of simplicial homotopies}

The fact that we need about simplicial homotopies is that if $ h \colon \fromto{\uupperstar(\Xdot)}{\uupperstar(\Ydot)} $ is a simplicial homotopy from $ f_0 $ to $ f_1 $, then $ f_0 $ and $ f_1 $ induce the same map $ \fromto{\real{\Xdot}}{\real{\Ydot}} $ on geometric realizations.

\begin{lemma}\label{lem:simplicialhomotopyequivalentmaps}
	Let $ D $ be \acategory that admits geometric realizations of simplicial objects.
	Let $ f_0, f_1 \colon \fromto{\Xdot}{\Ydot} $ be morphisms of simplicial objects in $ D $ and let $ h $ be a simplicial homotopy from $ f_0 $ to $ f_1 $.
	Then the simplicial homotopy $ h $ induces an equivalence $ \real{f_0} \equivalent \real{f_1} $ between the induced morphisms
	\begin{equation*}
		\real{f_0}, \real{f_1} \colon \fromto{\real{\Xdot}}{\real{\Ydot}}
	\end{equation*}
	on geometric realizations.
\end{lemma}

\begin{proof}
	Since the functors $ j_0, j_1 \colon \incto{\Deltaop}{(\Deltabf_{/[1]})^{\op}} $ are right adjoints (\Cref{obs:jadjoint}), both $ j_0 $ and $ j_1 $ are colimit-cofinal.
	Hence the simplicial homotopy $ h $ provides equivalences
	\begin{equation*}
		\real{f_0} \equivalent \real{\jupperstar_0(h)} \equivalent \colim_{(\Deltabf_{/[1]})^{\op}} h \colon \real{\Xdot} \equivalent \colim_{(\Deltabf_{/[1]})^{\op}} \uupperstar(\Xdot) \to \colim_{(\Deltabf_{/[1]})^{\op}} \uupperstar(\Ydot) \equivalent \real{\Ydot} 
	\end{equation*}
	and 
	\begin{equation*}
		\real{f_1} \equivalent \real{\jupperstar_1(h)} \equivalent \colim_{(\Deltabf_{/[1]})^{\op}} h \colon \real{\Xdot} \equivalent \colim_{(\Deltabf_{/[1]})^{\op}} \uupperstar(\Xdot) \to \colim_{(\Deltabf_{/[1]})^{\op}} \uupperstar(\Ydot) \equivalent \real{\Ydot} \period
	\end{equation*}
	Hence
	\begin{equation*}
		\real{f_0} \equivalent \colim_{(\Deltabf_{/[1]})^{\op}} h \equivalent \real{f_1} \comma
	\end{equation*}
	as desired.
\end{proof}

%-------------------------------------------------------------------%
%-------------------------------------------------------------------%
%  Proof of the Morel–Suslin–Voevodsky formula                      %
%-------------------------------------------------------------------%
%-------------------------------------------------------------------%

\subsection{Proof of the Morel--Suslin--Voevodsky formula}\label{subsec:proofofMVS}

We prove \Cref{prop:MorelVoevodsky} by applying the following recognition principle for localization functors.

\begin{proposition}[{\HTT{Proposition}{5.2.7.4}}]\label{prop:HTT.5.2.7.4}
	Let $ C $ be \acategory and $ L \colon \fromto{D}{D} $ a functor with essential image $ LD \subset D $.
	Then the following are equivalent:
	\begin{enumerate}[{\upshape (\ref*{prop:HTT.5.2.7.4}.1)}]
		\item\label{prop:HTT.5.2.7.4.1} There exists a functor $ F \colon \fromto{D}{D'} $ with fully faithful right adjoint $ G \colon \incto{D'}{D} $ such that $ GF \equivalent L $.

		\item\label{prop:HTT.5.2.7.4.2} The functor $ L \colon \fromto{D}{LD} $ is left adjoint to the inclusion $ \incto{LD}{D} $.

		\item\label{prop:HTT.5.2.7.4.3} There is a natural transformation $ \unit \colon \fromto{\id{D}}{L} $ such that for all $ d \in D $, the morphisms
		\begin{equation*}
			\unit_{L(d)}, L(\unit_d) \colon \fromto{L(d)}{L(L(d))}
		\end{equation*}
		are equivalences.
	\end{enumerate}
\end{proposition}

\begin{notation}
	Let us temporarily write $ \Hup \colon \fromto{\PSh(\Man;C)}{\PSh(\Man;C)} $ for the Morel--Suslin--Voevodsky construction
	\begin{equation*}
		\Hup(F)(M) \colonequals \real{F(M \cross \Deltaalgdot)} \period
	\end{equation*}
\end{notation}

\begin{construction}
	Let $ C $ be a presentable \category.
	Define a natural transformation
	\begin{equation*}
		\unit \colon \fromto{\id{\PSh(\Man;C)}}{\Hup} 
	\end{equation*}
	as follows.
	Let $ M $ be a manifold, and also simply write $ M $ for the constant cosimplicial manifold at $ M $.
	Projection onto the first factor defines a morphism of cosimplicial manifolds
	\begin{equation*}
		\pr_M \colon \fromto{M \cross \Deltaalgdot}{M}
	\end{equation*}
	from the product cosimplicial manifold $ M \cross \Deltaalgdot $ to the constant cosimplicial manifold at $ M $.
	For each $ C $-valued presheaf $ F \in \PSh(\Man;C) $, the morphism $ \unit_F \colon \fromto{F}{\Hup(F)} $ is defined as the geometric realization
	\begin{equation*}
		\unit_F(M) \colonequals \real{\prupperstar_M} \colon F(M) \equivalence \real{F(M)} \to \real{F(M \cross \Deltaalgdot)} = \Hup(F)(M) \period
	\end{equation*}

	Equivalently, the morphism $ \unit_F(M) $ is the composite
	\begin{equation*}
		F(M) \equivalent F(M \cross \Deltaalg^0) \to \real{F(M \cross \Deltaalgdot)}
	\end{equation*}
	of the equivalence $ \equivto{F(M)}{F(M \cross \Deltaalg^0)} $ induced by the projection $ \isomto{M \cross \Deltaalg^0}{M} $ with the induced map
	\begin{equation*}
		\fromto{F(M \cross \Deltaalg^0)}{\real{F(M \cross \Deltaalgdot)}}
	\end{equation*}
	from the $ 0 $-simplices of the simplicial object $ F(M \cross \Deltaalgdot) $ to its geometric realization. 
\end{construction}

%-------------------------------------------------------------------%
%  Proof of ℝ-invariance                                            %
%-------------------------------------------------------------------%

\subsubsection{Proof of \texorpdfstring{$ \RR $}{ℝ}-invariance}

In order to apply \Cref{prop:HTT.5.2.7.4}, the we first check:

\begin{lemma}\label{lem:Hishi}
	Let $ C $ be a presentable \category.
	For any presheaf $ F \colon \fromto{\Manop}{C} $, the presheaf $ \Hup(F) $ is $ \RR $-invariant.
\end{lemma}

\noindent To prove \Cref{lem:Hishi}, we apply the technology of simplicial homotopies.

\begin{lemma}\label{lem:simplicialhomotopyindpr}
	Let $ M $ be a manifold.
	There is a natural simplicial homotopy in $ \Manop $ from the map
	\begin{equation*}
		i_{M \cross \Deltaalgdot,0}\, \of \pr_{M \cross \Deltaalgdot} \colon M \cross \Deltaalgdot \cross \RR \to M \cross \Deltaalgdot \cross \RR
	\end{equation*}
	to the identity.
\end{lemma}

\begin{proof}
	Define a simplicial homotopy
	\begin{equation*}
		h \colon \fromto{\uupperstar(M \cross \Deltaalgdot \cross \RR)}{\uupperstar(M \cross \Deltaalgdot \cross \RR)}
	\end{equation*}
	as follows.
	For each map $ \sigma \colon \fromto{[n]}{[1]} $ in $ \Deltabf $, write $ h'_{\sigma} \colon \fromto{\Deltaalg^n \cross \RR}{\Deltaalg^n \cross \RR} $ for the smooth map defined by the formula
	\begin{equation*}
		h'_{\sigma}(t_0,\ldots,t_n,x) \colonequals \paren{t_0,\ldots,t_n, \textstyle x\sum_{k \in \sigmainverse(1)} t_k} \period
	\end{equation*}
	Define $ h_{\sigma} \colon \fromto{M \cross \Deltaalg^n \cross \RR}{M \cross \Deltaalg^n \cross \RR} $ by setting $ h_{\sigma} \colonequals \id{M} \cross h'_{\sigma} $.
	It is immediate from the definitions that $ h $ defines a simplicial homotopy
	\begin{equation*}
		\fromto{\uupperstar(M \cross \Deltaalgdot \cross \RR)}{\uupperstar(M \cross \Deltaalgdot \cross \RR)} \comma
	\end{equation*}
	and, moreover,
	\begin{equation*}
		\jupperstar_0(h) = i_{M \cross \Deltaalgdot,0}\, \of \pr_{M \cross \Deltaalgdot} \andeq \jupperstar_1(h) = \id{M \cross \Deltaalgdot \cross \RR} \period \qedhere
	\end{equation*}
\end{proof}

\begin{proof}[Proof of \Cref{lem:Hishi}]
	Let $ M $ be a manifold.
	Since $ \pr_M i_{M,0} = \id{M} $, to see that
	\begin{equation*}
		\prupperstar_M \colon \fromto{\Hup(F)(M)}{\Hup(F)(M \cross \RR)}
	\end{equation*}
	is an equivalence, it suffices to show that $ \prupperstar_M \iupperstar_{M,0} \equivalent \id{\Hup(F)(M \cross \RR)} $.
	This follows from combining \Cref{lem:simplicialhomotopyindpr,lem:simplicialhomotopyequivalentmaps}.
\end{proof}

%-------------------------------------------------------------------%
%  Proof that the unit is an equivalence                            %
%-------------------------------------------------------------------%

\subsubsection{Proof that the unit is an equivalence}

The second thing to check is that $ \unit_{H(G)} $ is an equivalence for every presheaf $ G $.
Combined with \Cref{lem:Hishi} this guarantees that the essential image of the functor
\begin{equation*}
	\Hup \colon \fromto{\PSh(\Man;C)}{\PSh(\Man;C)}
\end{equation*}
is $ \PShhi(\Man;C) $.

\begin{lemma}\label{lem:Hofhiishi}
	Let $ C $ be a presentable \category.
	If $ F \colon \fromto{\Manop}{C} $ is $ \RR $-invariant, then the map $ \unit_F \colon \fromto{F}{\Hup(F)} $ is an equivalence.
\end{lemma}

\begin{proof}
	Let $ M $ be a manifold.
	Since $ F $ is $ \RR $-invariant and $ \Deltaalg^n \isomorphic \RR^n $ for each $ n \geq 0 $, the projection  $ \pr_M \colon \fromto{M \cross \Deltaalgdot}{M} $ from the cosimplicial manifold $ M \cross \Deltaalgdot $ to the constant cosimplicial manifold at $ M $ induces an equivalence
	\begin{equation*}
		\prupperstar_M \colon \equivto{F(M)}{F(M \cross \Deltaalgdot)} 
	\end{equation*}
	of simplicial objects in $ C $.
	The claim now follows by passing to geometric realizations.
\end{proof}

\begin{corollary}\label{cor:essentialimageofH}
	Let $ C $ be a presentable \category.
	The essential image of the functor
	\begin{equation*}
		 \Hup \colon \fromto{\PSh(\Man;C)}{\PSh(\Man;C)}
	\end{equation*}
	is $ \PShhi(\Man;C) $. 
\end{corollary}

Now we complete the proof of \Cref{prop:MorelVoevodsky} by showing that see that $ \Hup(\unit_F) $ is an equivalence.

\begin{lemma}\label{lem:alphaequivs}
	Let $ C $ be a presentable \category.
	For all $ F \in \PSh(\Man;C) $, the maps
	\begin{equation*}
		\unit_{\Hup(F)}, \Hup(\unit_F)  \colon \fromto{\Hup(F)}{\Hup(\Hup(F))}
	\end{equation*}
	are equivalences.
\end{lemma}

\begin{proof}
	By \Cref{lem:Hofhiishi,cor:essentialimageofH}, the morphism $ \unit_{\Hup(F)} $ is an equivalence.
	To see that $ \Hup(\unit_F) \colon \fromto{\Hup(F)}{\Hup(\Hup(F))} $ is an equivalence, note that for each manifold $ M $ we have
	\begin{align*}
		\Hup(F)(M) &= \colim_{[m] \in \Deltaop} F(M \cross \Deltaalg^m) \\
		\intertext{and} 
		\Hup(\Hup(F))(M) &= \colim_{[m] \in \Deltaop} \colim_{[n] \in \Deltaop} F(M \cross \Deltaalg^m \cross \Deltaalg^n) \\ 
		&\equivalent \colim_{([m],[n]) \in \Deltaop \cross \Deltaop} F(M \cross \Deltaalg^m \cross \Deltaalg^n) \period
	\end{align*} 
	Moreover, the map $ \Hup(\unit_F) \colon \fromto{\Hup(F)}{\Hup(\Hup(F))} $ is induced by restriction of diagrams along the fully faithful functor
	\begin{align*}
		\Deltaop &\inclusion \Deltaop \cross \Deltaop \\ 
		[m] &\mapsto ([m],[0]) \period
	\end{align*}
	First taking the colimit over the variable $ [m] \in \Deltaop $, we see that the map $ \Hup(\unit_F)(M) $ is induced by the map from the $ 0 $-simplices $ \Hup(F)(M) $ of the simplicial object $ \Hup(F)(M \cross \Deltaalgdot) $ to its geometric realization.
	Since $ \Hup(F) $ is $ \RR $-invariant (\Cref{lem:Hishi}), the simplicial object $ \Hup(F)(M \cross \Deltaalgdot) $ is equivalent to the constant simplicial object at $ \Hup(F)(M) $, hence the induced map 
	\begin{equation*}
		\Hup(F)(M) \to \colim_{[n] \in \Deltaop} \Hup(F)(M \cross \Deltaalg^n) 
	\end{equation*}
	from the $ 0 $-simplices is an equivalence.
\end{proof}

\begin{proof}[Proof of \Cref{prop:MorelVoevodsky}]
	Combine \Cref{cor:essentialimageofH,lem:alphaequivs,prop:HTT.5.2.7.4}.
\end{proof}
