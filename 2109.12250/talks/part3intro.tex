%!TEX root = ../diffcoh.tex
\label{part:applications}\label{applications_part}

In \cref{applications_part}, we survey some applications of differential cohomology to questions in geometry and
physics. Some of these applications belong to the pattern that what ordinary cohomology tells us about topological
objects, differential cohomology tells us about their geometric analogues: this includes both the use of
differential cohomology to obstruct conformal immersions as well as the classification of invertible field
theories, both of which we say more about below. For other applications, the analogy with ordinary cohomology is
subtler; some use the differential characteristic classes we built in \cref{char_class_part}, such as the study of
loop groups and the Virasoro group.

\subsection*{Chern--Simons invariants}
Chern--Simons invariants, which we define and study in \cref{config_spaces}, are the key to many of these
applications. Let $G$ be a compact Lie group; choose a class $\lambda\in \H^4(\BG;\ZZ)$ and let $\langle\text{--},
\text{--}\rangle$ be the degree-$2$ $G$-invariant symmetric polynomial on $\g$ associated to the image of $\lambda$
in de Rham cohomology. The Chern--Simons invariant associated to $\lambda$ is defined for a $3$-manifold $Y$, a
principal $G$-bundle $\pi\colon P\to Y$, and a connection $\conn$ on $P$ with curvature $\curvature{\conn}$. If we
assume that $\pi$ has a section, so that we can descend $\curvature{\conn}$ to a form on $Y$, the Chern--Simons
invariant is
\begin{equation}
\label{CS_3_intro}
	\CS_\lambda(P, \conn) = \int_Y \langle \conn\wedge\curvature{\conn}\rangle - \frac 16\langle \conn\wedge [\conn,
	\conn]\rangle \in \RR/\ZZ.
\end{equation}
We first met this invariant in a different guise in \cref{secondary_chern_simons}. In \cref{differential_CW_lift}
we showed $\lambda$ and $\langle\text{--}, \text{--}\rangle$ determine a differential refinement
$\hat\lambda\in\Hhat^4(\BnablaG; \ZZ)$, and said but did not prove that the Chern--Simons invariant is the
secondary invariant associated to $\hat\lambda$. We will prove the latter fact in \cref{config_spaces}.

Chern--Simons invariants and their generalizations play a central role in most of the applications of differential
cohomology which we survey: they bridge the geometry of connections with the algebraic topology of (differential)
characteristic classes, and therefore have something to say about both worlds.

% config spaces
For example, in \cref{config_spaces} we follow Evans-Lee--Saveliev \cite{deletedsquare} and use Chern--Simons
invariants as a tool to determine when two homotopy-equivalent lens spaces are not
diffeomorphic; to do so, we also spend time developing a little of the theory of Chern--Simons invariants. The
classification of lens spaces up to diffeomorphism or homotopy equivalence is classical \cites{Rei35}[\S
5]{Whi41}{Bro60}, which makes it a good testing ground to determine how powerful manifold invariants are. For
example, Longoni--Salvatore \cite{LS05} proved the surprising result that the homotopy type of the two-point
configuration space of a lens space can distinguish homotopy-equivalent lens spaces.  Evans-Lee--Saveliev build on
Longoni--Salvatore's work, providing more comprehensive tools for understanding when the homotopy type of the
two-point configuration space of $L(p, q)$ is a stronger invariant than the homotopy type of $L(p, q)$. They extend
Chern--Simons invariants to two-point configuration spaces and use them to give a numerical criterion
(\cref{f_h_restr}) for a map of configuration spaces to be a homotopy equivalence. They combine this criterion with
a few other tools, including Massey products, to provide many pairs of homotopy-equivalent lens spaces whose
two-point configuration spaces are not homotopy equivalent.

% conformal immersions
In \cref{conformal_immersions}, we use on-diagonal differential characteristic classes to obstruct conformal
immersions, following Chern--Simons \cite{cs}. Recall that characteristic classes in ordinary cohomology can
obstruct immersions into $\RR^n$ as follows: if $M$ is a smooth $m$-manifold that immerses into $\RR^n$ with normal
bundle $\nu$, then $TM\oplus\nu \cong T\RR^n|_M\cong\underline\RR^n$, and $\nu$ is rank $n-m$, so all of its
characteristic classes in degree greater than $n-m$ vanish. This places constraints on the characteristic classes
of $M$. For example, let $w_i$ denote the $i^{\mathrm{th}}$ Stiefel--Whitney
class;\index[terminology]{Stiefel--Whitney class} if $\CCP^2$ immersed in $\RR^5$, then the normal bundle $\nu$ is
one-dimensional, so
\begin{equation}
	w_2(T\CCP^2\oplus\nu) = w_2(T\CCP^2) + \underbracket{w_1(T\CCP^2)w_1(\nu) + w_2(\nu)}_{=0} =
	w_2(\underline\RR^5) = 0,
\end{equation}
but $w_2(T\CCP^2)\ne 0$, which prevents such an immersion. One can run the same argument using Cheeger--Simons'
differential characteristic classes, which we discussed in \cref{DifferentialCharacteristicClasses}: since these
characteristic classes are defined for vector bundles with connection, they can obstruct isometric embeddings of a
Riemannian manifold $M$ by placing constraints on $TM$ with its Levi-Civita
connection.\index[terminology]{Levi-Civita connection} Chern--Simons \cite{cs} improve on this argument in two
ways, giving it considerably more power: they prove that the on-diagonal differential Pontryagin classes of the
Levi-Civita connection only depend on the conformal class of the metric (\cref{diff_p_conformally_invariant}), so
can be used to obstruct conformal immersions. They then use the Chern--Simons form to obtain additional
obstructions: in some cases, the Chern--Simons form is closed, and conformal immersions restrict what its de Rham
class can be. The proofs of these obstructions make use of the close relationship between differential
characteristic classes and Chern--Simons forms. Chern--Simons' obstructions are strong enough to prove that
$\RRP^3$ with the round metric cannot conformally immerse in $\RR^4$ (\cref{nope_RP3}).

Our third application of Chern--Simons invariants is to physics: there is a classical field theory whose Lagrangian
is the Chern--Simons invariant~\eqref{CS_3_intro}. We discuss this theory in \cref{classical_CS}, focusing on how
various pieces of the theory can be described using differential cohomology. Schwarz \cite{Sch77} and
Witten \cite{Wit89} quantized this theory, producing a topological field theory called Chern--Simons
theory\index[terminology]{Chern--Simons theory} which has been a major object of study in both mathematics and
physics. See \cref{quantum_CS} for references and more information on the quantum theory.

\subsection*{Quantum physics}
Speaking of physics, several of the applications of differential cohomology that we survey are in physics or are
closely related to it. In these applications, differential cohomology tends to appear because quantization imposes
integrality conditions on objects in field theories; in many cases these can be lifted to integrality data,
allowing differential cohomology to enter the picture.

\Cref{field_theory} is dedicated to this idea, working with the example of electromagnetism. We first discuss
classical Maxwell theory, describing how information in this theory can be expressed with differential forms. Then
we walk through Dirac's argument \cite{Dir31} that the presence of magnetic monopoles forces electric and magnetic
charges to be quantized, i.e.\ valued in a discrete subgroup of $\RR$. As a consequence, the fields in the quantum
theory are cocycles for differential cohomology, and the action can be rewritten using the
differential-cohomological cup product and integration. For electromagnetism, the appearance of differential
cohomology is relatively explicit and simple, making it a good example, but the concept of quantization of abelian
gauge fields leading to differential cohomology appears in numerous other places in quantum physics, and can
involve fancier objects such as differential $\Kup$-theory.

The next chapter, \cref{invertible_field_theories}, is about a different application of differential (generalized)
cohomology to physics: the classification of invertible field theories. This is one of the applications which is a
geometric analogue of a use of ordinary (generalized) cohomology for something topological. Following Atiyah and
Segal, a topological field theory (TFT) is a symmetric monoidal functor
\[Z\colon\Bord_n\to\mathsf C\comma\]
where $\Bord_n$ is a bordism (higher) category and $\mathsf C$ is some symmetric monoidal (higher) category, often
$\mathsf{Vect}_{\CC}$. The simplest nontrivial TFTs are the invertible TFTs, which are the TFTs whose values on all
objects and morphisms in $\Bord_n$ are invertible in $\mathsf C$, meaning that objects are invertible under the
tensor product, and morphisms are invertible under composition.
%Invertible TFTs were classified by
%Freed--Hopkins--Teleman \cite{FHT10} using the work of Galatius--Madsen--Tillmann--Weiss \cite{GMTW09} and
%Nguyen \cite{Ngy17} characterizing the stable homotopy types of bordism categories.
We are interested in \emph{reflection-positive} invertible TFTs; this extra requirement is a physically motivated
version of unitarity. The classification of reflection-positive invertible TFTs is due to
Freed--Hopkins \cite{FH21}, who show that, up to isomorphism, reflection-positive invertible TFTs are classified by
the torsion subgroup of $[\mathrm{MTH}, \Sigma^n \mathrm I_\ZZ]$ (see \cref{top_IFT} for definitions of these
spectra). In typical examples, the partition functions of these theories are bordism invariants defined by
integrating characteristic classes in (generalized) cohomology. Freed--Hopkins (\textit{ibid.}) go further and
conjecture that the entirety of $[\mathrm{MTH}, \Sigma^n \mathrm I_\ZZ]$ classifies invertible field theories that
need not be topological, which would be defined on some yet-to-be-constructed geometric bordism category. Again,
partition functions can often be described by integrating characteristic classes, but this time in differential
(generalized) cohomology, and typically in one dimension lower, so as to obtain a secondary invariant. We discuss
this conjecture and several examples: classical Chern--Simons theory as mentioned above, the classical
Wess--Zumino--Witten model, and an example using differential $\mathrm{KO}$-theory.

\subsection*{Representations of loop groups}

In \cref{loop_groups}, we turn to the representation theory of loop groups. These are infinite-dimensional Lie
groups, but unusually nice ones: as long as you are careful about what you mean by a representation, their
representation theory closely resembles that of compact Lie groups! The representations we care about are
projective representations, so genuine representations of a central extension by the circle group
\begin{equation}
\label{Tcent_}
	1\to \TT\to \LGtilde\to \LG\to 1,
\end{equation}
satisfying a ``positive energy'' condition: restricting the representation to $\TT$, its weight subspaces for
negative weights are trivial. The reader may wonder how invariant this definition is, and is right to be concerned:
it is a significant theorem of Pressley--Segal \cite[Theorem 13.4.2]{loop} that when $G$ is simply connected and
compact, every positive energy representation of $\LG$ admits an intertwining projective $\Diffplus(\TT)$-action,
meaning that the notion of positive energy is preserved under reparametrizations of $\TT$. One of the major goals
of \cref{loop_groups} is to discuss the key ideas in this theorem and its proof: we introduce and motivate the
positive energy condition, we discuss the nice properties of positive energy representations, and we sketch the
proof of Pressley--Segal's theorem. Along the way, we discuss some connections with physics. In \cref{PS_diffcoh},
we discuss two different connections to differential cohomology: first, the central extensions of the sort we
consider are principal $\TT$-bundles over $\LG$, hence determine classes in $\H^2(\LG;\ZZ)$. It turns out that
every element of this cohomology group comes from a central extension, and moreover, as principal $\TT$-bundles
they carry canonical connections, allowing for a lift to $\Hhat^2(\LG;\ZZ)$. This class is related to the ``level''
that one starts with via transgression maps $\Hhat^4(\BnablaG;\ZZ)\to\Hhat^3(G;\ZZ)\to\Hhat^2(\LG;\ZZ)$. Central
extensions that are principal $\TT$-bundles correspond to off-diagonal classes in $\H^3(\mathrm B_\bullet \LG;
\ZZ(1))$, as in \cref{VirasoroAlgebra}, and we say a little about this perspective too.

Our final chapter, \cref{segal_sugawara}, takes the above story and makes it explicit, albeit at the level of Lie
algebras. The Lie algebra of a central extension $\LGtilde$, denoted $\Lgtilde{}$, is an example of a
\emph{Kac--Moody algebra}, and is a central extension of the loop algebra of the Lie algebra of $\g$. The
Pressley--Segal theorem cooks up an intertwining projective $\Diffplus(\TT)$-action on the representations of
$\LGtilde$, so at the level of Lie algebras we might expect a compatible Virasoro algebra action on the
representations of Kac--Moody algebras. This is true, and Segal--Sugawara show we can do better, explicitly
identifying how the central $\CC$ in the Virasoro algebra acts in terms of the level of the central
extension~\eqref{Tcent_}. Both this chapter and the previous chapter on loop groups are closely related to
two-dimensional conformal field theory: the data of the category of positive-energy representations of $\LGtilde$
can be used to build a two-dimensional conformal field theory called the Wess--Zumino--Witten model. This CFT is
further related to Chern--Simons theory, a 3d TFT. All of this data --- the central extension of $\LG$, the
specific Wess--Zumino--Witten model, the specific Chern--Simons theory --- is indexed by groups such as
$\H^2(\LG;\ZZ)$, $\H^3(G;\ZZ)$, and $\H^4(\BG;\ZZ)$, which when $G$ is simple and simply connected are all
canonically isomorphic to $\ZZ$. These groups are related to each other by transgression maps, and this corresponds
to the relationship between, e.g.\ loop groups and the WZW model, or the WZW model and Chern--Simons theory. These
cohomology classes have differential refinements, as do the transgression maps relating them.

These are not the only applications of differential cohomology to topology, geometry, or physics, but we hope they
illustrate the diversity of things that can be done with differential cohomology, and that they make for an
interesting and enjoyable read.

