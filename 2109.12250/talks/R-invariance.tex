%!TEX root = ../diffcoh.tex

%-------------------------------------------------------------------%
%-------------------------------------------------------------------%
%  ℝ-invariant sheaves                                              %
%-------------------------------------------------------------------%
%-------------------------------------------------------------------%

\section{\texorpdfstring{$ \RR $}{ℝ}-invariant sheaves}\label{sec:hisheaves}
\textit{by Peter Haine}

In this chapter, we investigate \textit{$ \RR $-invariant} (or \textit{homotopy invariant}) sheaves on $ \Man $.
These are the sheaves that invert homotopy equivalences of manifolds.
The main result of this chapter is Dugger's observation that the global sections functor induces an equivalence from the subcategory $ \Sh(\Man;C) $ of $ \RR $-invariant sheaves to $ C $ (\Cref{prop:Dugger}).
We prove this by showing that the constant sheaf functor $ \Gammaupperstar \colon \fromto{C}{\Sh(\Man;C)} $ is given by the assignment
\begin{equation*}
	X \mapsto [M \mapsto X^{\Piinf(M)}] \comma
\end{equation*} 
where $ X^{\Piinf(M)} $ denotes the cotensor of $ X \in C $ by the underlying homotopy type $ \Piinf(M) $ of the manifold $ M $ (\Cref{rec:cotensoring}).
These results imply that there exists a chain of four (explicit) adjoints
\begin{equation*}
	\begin{tikzcd}[sep=4em]
		\Sh(\Man;C) \arrow[r, shift left=2ex, "\Gammalowershriek"] \arrow[r, shift right=0.75ex, "\Gammalowerstar"{description, xshift=0.5em}] & C \arrow[l, shift left=2ex, "\Gammauppershriek", hooked'] \arrow[l, shift right=0.75ex, "\Gammaupperstar"{description, xshift=-0.5em}, hooked']
	\end{tikzcd}
\end{equation*}
relating $ \Sh(\Man;C) $ and $ C $ (see \Cref{obs:LhiandRhi,nul:Gammaadjunctions}).

Looking forward, in \Cref{sec:localization}, we give an explicit formula for $ \Gammalowershriek $ as a geometric realization.
In \Cref{sec:stable}, we use these adjoints and relations between them to construct a ``differential cohomology
diagram'' for sheaves on $ \Man $ with values in any presentable stable \category.

\Cref{sec:prelimsglobalsec} starts with some preliminary observations about the global sections and constant sheaf functors..
In \Cref{sec:RRinvarbasics}, we define $ \RR $-invariant sheaves and explore some of their basic properties.
\Cref{sec:constantsheafdescript} is dedicated to proving that the global sections functor restricts to an equivalence on $ \RR $-invariant sheaves.
\Cref{sec:digressionRRinvar} is a digression giving two alternative ways to check that a sheaf is $ \RR $-invariant.

%-------------------------------------------------------------------%
%-------------------------------------------------------------------%
%  Preliminaries on global sections and constant sheaves            %
%-------------------------------------------------------------------%
%-------------------------------------------------------------------%

\subsection{Preliminaries on global sections and constant sheaves}\label{sec:prelimsglobalsec}

We begin by fixing some notation that we use throughout the rest of this text.

\begin{notation}
	Write $ \Gammalowerstar \colon \fromto{\PSh(\Man;C)}{C} $ for the \textit{global sections} functor, defined by
	\begin{equation*}
		\Gammalowerstar(E) \colonequals E(*) \period
	\end{equation*}
	Write $ \Gammaupperstar \colon \fromto{C}{\Sh(\Man;C)} $ for the \textit{constant sheaf} functor.
	That is, $ \Gammaupperstar $ is left adjoint to the restriction $ \Gammalowerstar \colon \fromto{\Sh(\Man;C)}{C} $ of the global sections functor to sheaves.
\end{notation}

\noindent The global sections functor also has a right adjoint.

\begin{lemma}\label{lem:Gammauppershriekpsh}
	Let $ C $ be a presentable \category.
	Then the functor $ \Gammauppershriek \colon \fromto{C}{\PSh(\Man;C)} $ defined by the formula
	\begin{equation*}
		\Gammauppershriek(X)(M) \colonequals \prod_{m \in M} X 
	\end{equation*}
	is fully faithful and right adjoint to the global sections functor $ \Gammalowerstar \colon \fromto{\PSh(\Man;C)}{C} $.
	(Here the product is over the underlying set of the manifold $ M $.)
\end{lemma}

\begin{proof}
	We define the unit and counit of the adjunction.
	The unit $ \unit_{F} \colon \fromto{F}{\Gammauppershriek \Gammalowerstar(F)} $ is defined by the natural map
	\begin{equation*}
		F(M) \to \prod_{m \in M} F(\{m\}) \equivalence \Gammauppershriek \Gammalowerstar(F)(M)
	\end{equation*}
	induced by the inclusions $ \incto{\{m\}}{M} $ for all $ m \in M $.
	The counit $ \counit_X \colon \fromto{\Gammalowerstar \Gammauppershriek(X)}{X} $ is given by the natural identification $ \prod_{*} X \equivalent X $.
	The triangle identities are immediate from the definitions.

	To conclude, note that since the counit $ \counit $ is an equivalence, the functor $ \Gammauppershriek $ is fully faithful.
\end{proof}

Before recording the consequences of \Cref{lem:Gammauppershriekpsh} on the level of sheaves, we recall a basic fact from category theory.
For a proof see, for example, \cite[Chapter VII, \S4, Lemma 1]{MR1300636}.

\begin{lemma}\label{lem:ladjradj}
	Let $ \flowerstar \colon \fromto{A}{B} $ be a functor between \categories.
	Assume that $ \flowerstar $ admits a left adjoint $ \fupperstar $ and right adjoint $ \fuppershriek $.
	Then $ \fupperstar $ is fully faithful if and only if $ \fuppershriek $ is fully faithful.
\end{lemma}

\begin{corollary}\label{cor:GammauppershriekSh}
	Let $ C $ be a presentable \category.
	\begin{enumerate}[{\upshape (\ref*{cor:GammauppershriekSh}.1)}]
		\item\label{cor:GammauppershriekSh.1} The functor $ \Gammauppershriek $ factors through $ \Sh(\Man;C) $.

		\item\label{cor:GammauppershriekSh.2} The global sections functor $ \Gammalowerstar \colon \fromto{\Sh(\Man;C)}{C} $ is left adjoint to $ \Gammauppershriek $.
		
		\item\label{cor:GammauppershriekSh.3} The constant sheaf functor $ \Gammaupperstar $ is fully faithful.
	\end{enumerate}
\end{corollary}

\begin{proof}
	For \enumref{cor:GammauppershriekSh}{1}, note that is immediate from \Cref{def:sheafonMan} that for each $ X \in C $, the presheaf $ \Gammauppershriek(X) $ is a sheaf on $ \Man $.
	\Cref{lem:Gammauppershriekpsh} and \enumref{cor:GammauppershriekSh}{1} immediately imply \enumref{cor:GammauppershriekSh}{2}.
	Finally, \enumref{cor:GammauppershriekSh}{3} is a consequence of \Cref{lem:Gammauppershriekpsh}, \enumref{cor:GammauppershriekSh}{2}, and the full faithfulness of $ \Gammauppershriek $.
\end{proof}

One useful consequence of \Cref{cor:GammauppershriekSh} is that sheafification on $ \Man $ does not change global sections.
This result will be of great importance in \Cref{sec:localization}.

\begin{corollary}\label{cor:sheafificationofglobalsections}
	Let $ C $ be a presentable \category.
	For every $ F \in \PSh(\Man;C) $, the unit $ \fromto{F}{\SMan(F)} $ induces an equivalence on global sections.  
\end{corollary}

\begin{proof}
	We need to show that for each $ X \in C $, the morphism
	\begin{equation}\label{eq:gammalowerstarmap}
		\fromto{\Map_C(\Gammalowerstar\SMan(F),X)}{\Map_C(\Gammalowerstar(F),X)}
	\end{equation}
	induced by the unit is an equivalence.
	By adjunction, \eqref{eq:gammalowerstarmap} is an equivalence if and only if the 
	morphism 
	\begin{equation}\label{eq:gammauppershriekmap}
		\fromto{\Map_{\PSh(\Man;C)}(\SMan(F),\Gammauppershriek(X))}{\Map_{\PSh(\Man;C)}(F,\Gammauppershriek(X))}
	\end{equation}
	induced by the unit is an equivalence.
	To complete the proof, note that the fact that $ \Gammauppershriek(X) $ is a sheaf (\Cref{cor:GammauppershriekSh}) implies that the morphism \eqref{eq:gammauppershriekmap} is an equivalence.
\end{proof}

\begin{remark}
	The functor $ \Gammauppershriek $ does not play a significant role in the approach to differential cohomology presented here.
	Rather, it serves as a convenient way to see that $ \Gammalowerstar $ preserves colimits and $ \Gammaupperstar $ is fully faithful.
\end{remark}

%-------------------------------------------------------------------%
%-------------------------------------------------------------------%
%  Basics of ℝ-invariant sheaves                                    %
%-------------------------------------------------------------------%
%-------------------------------------------------------------------%

\subsection{Basics of \texorpdfstring{$ \RR $}{ℝ}-invariant sheaves}\label{sec:RRinvarbasics}

We start by introducing an important subcategory of $ \Sh(\Man;C) $.

\begin{definition}
	Let $ C $ be a presentable \category.
	We say that a $ C $-valued presheaf
	\begin{equation*}
		F \colon \fromto{\Manop}{C}
	\end{equation*}
	is \textit{$ \RR $-invariant}, \textit{homotopy-invariant}, or \textit{concordance-invariant} if for every manifold $ M $, the first projection $ \pr_M \colon \fromto{M \cross \RR}{M} $ induces an equivalence
	\begin{equation*}
		\prupperstar_M \colon \isomto{F(M)}{F(M \cross \RR)} \period
	\end{equation*}
	Write
	\begin{equation*}
		\Shhi(\Man;C) \subset \Sh(\Man;C) \andeq \PShhi(\Man;C) \subset \PSh(\Man;C)
	\end{equation*}
	for the full subcategories spanned by the $ \RR $-invariant $ C $-valued sheaves and presheaves, respectively.
\end{definition}

\begin{remark}
	Note that by the $ 2 $-of-$ 3 $ property, a presheaf $ F \colon \fromto{\Manop}{C} $ is $ \RR $-invariant if and only if for every \textit{homotopy equivalence} of manifolds $ \fromto{N}{M} $, the induced map $ \fromto{F(M)}{F(N)} $ is an equivalence in $ C $.
\end{remark}

\begin{lemma}\label{lem:checkequivforhionpt}
	Let $ C $ be a presentable \category.
	A morphism $ f \colon \fromto{E}{E'} $ in $ \Shhi(\Man;C) $ is an equivalence if and only if $ \Gammalowerstar(f) $ is an equivalence in $ C $.
\end{lemma}

\begin{proof}
	This follows from \Cref{lem:checkequivonRn} and the assumption that $ E $ and $ E' $ are $ \RR $-invariant.
\end{proof}

We conclude this section by noting that the inclusion of $ \RR $-invariant (pre)sheaves into all (pre)sheaves admits a left adjoint.

\begin{observation}\label{obs:LRRandLhi}
	Notice that the full subcategory $ \PShhi(\Man;C) \subset \PSh(\Man;C) $ is closed under both limits and colimits.
	Hence $ \PShhi(\Man;C) $ is presentable and by the Adjoint Functor Theorem, the inclusion
	\begin{equation*}
		\PShhi(\Man;C) \subset \PSh(\Man;C)
	\end{equation*}
	admits both a left and a right adjoint.
	We write $ \LRR \colon \fromto{\PSh(\Man;C)}{\PShhi(\Man;C)} $ for the left adjoint to the inclusion.
	By a general result of category theory \HTT{Lemma}{5.5.4.18}, the intersection 
	\begin{equation*}
		\Shhi(\Man;C) = \Sh(\Man;C) \intersect \PShhi(\Man;C)
	\end{equation*}
	is presentable and the inclusion $ \Shhi(\Man;C) \subset \Sh(\Man;C) $ admits a left adjoint $ \Lhi $.
\end{observation}
	
\begin{definition}\label{homotopification_definition}
	Let $ C $ be a presentable \category.
	We refer to the left adjoints
	\begin{equation*}
		\LRR \colon \fromto{\PSh(\Man;C)}{\PShhi(\Man;C)} \andeq \Lhi \colon \fromto{\Sh(\Man;C)}{\Shhi(\Man;C)}
	\end{equation*}
	as the \emph{$ \RR $-localization} and \emph{homotopification} functors, respectively.
\end{definition}

At the end of \cref{sec:constantsheafdescript}, we give another way of seeing that the inclusion $ \Shhi(\Man;C) \subset \Sh(\Man;C) $ admits a left adjoint, as well as show that it admits a right adjoint (see \Cref{obs:LhiandRhi,nul:Gammaadjunctions}).
\Cref{sec:localization} is dedicated to providing explicit formulas for the functors $ \LRR $ and $ \Lhi $.

We finish this section with some remarks on the difference between $ \LRR $ and $ \Lhi $ and the notations we have chosen.

\begin{remark}[{($ \LRR $ vs. $ \Lhi $)}]\label{remark:BE-BdB-P}
	For a general presentable \category $ C $ and $ C $-valued sheaf $ E $ on $ \Man $, the presheaf $ \LRR(E) $ need not be a sheaf.
	Hence $ \Lhi $ is \textit{not} given by simply restricting $ \LRR $ to sheaves.
	However, the main result of the work of Berwick-Evans--Boavida de Brito--Pavlov \cite{BEBdBP:Classifying} shows that when $ C = \Spc $, the functor $ \LRR $ \textit{does} preserve sheaves.
	That is, for each sheaf $ E \in \Sh(\Man;\Spc) $, the natural morphism $ \fromto{\LRR(E)}{\Lhi(E)} $ is an equivalence.
	The keys to their proof are the reformulation of the sheaf condition given in \Cref{thm:excision} and technical results about when geometric realizations commute with infinite products and pullbacks akin to the results in \cite[\SAGsubsec{A.5.4}]{SAG}.
	We do not have occasion to use Berwick-Evans, Boavida de Brito, and Pavlov's result in this text.
\end{remark}

\begin{remark}[{(notations)}]
	Our notations $ \LRR $ and $ \Lhi $ are chosen in analogy with the standard notations in \textit{unstable motivic homotopy theory} \cites[\S2.2]{MotivicNorms:BachmannHoyois}{MR2242284}{MR2275634}{MR2934577}{MR1648048}{MR1813224}.
	To explain this, let us give an overview of how motivic spaces are defined.

	Let $ S $ be a scheme.
	We say that a presheaf $ F $ on the category $ \Sm_S $ of smooth schemes of finite type over $ S $ is \emph{$ \AA^1 $-invariant} if for every $ X \in \Sm_S $, the projection $ \pr_X \colon X \times_S \AA_S^1 \to X $ induces an equivalence
    \begin{equation*}
       \prupperstar_X \colon \isomto{F(X)}{F(X \cross_S \AA_S^1)} \period
    \end{equation*}
    Write $ \PSh_{\AA^1}(\Sm_S) \subset \PSh(\Sm_S) $ for the full subcategory spanned by the $ \AA^1 $-invariant presheaves of spaces on $ \Sm_S $.
    The inclusion $ \PSh_{\AA^1}(\Sm_S) \subset \PSh(\Sm_S) $ admits a left adjoint, typically denoted by $ \LAA $ and called \emph{$ \AA^1 $-localization}.
    The \category of motivic spaces over $ S $ is defined as the \category
    \begin{equation*}
    	\Sh_{\Nis,\AA^1}(\Sm_S) \colonequals \Sh_{\Nis}(\Sm_S) \intersect \PSh_{\AA^1}(\Sm_S)
    \end{equation*}
    of presheaves of spaces on $ \Sm_S $ that are $ \AA^1 $-invariant as well as sheaves for the \emph{Nisnevich topology} on $ \Sm_S $.
    The inclusion
    \begin{equation*}
    	\Sh_{\Nis,\AA^1}(\Sm_S) \subset \Sh_{\Nis}(\Sm_S) 
    \end{equation*}
    of motivic spaces into Nisnevich sheaves on $ \Sm_S $ also admits a left adjoint, typically denoted by $ \Lmot $ and called \emph{motivic localization}.
    An important point is that the functor $ \LAA \colon \fromto{\PSh(\Sm_S)}{\PSh_{\AA^1}(\Sm_S)} $ does not carry Nisnevich sheaves to Nisnevich sheaves, so $ \Lmot $ is not given by simply restricting $ \LAA $ to Nisnevich sheaves.

    Here, we should think $ \Man $ as analogous to $ \Sm_S $ and $ \Sh(\Man;\Spc) $ as analogous to $ \Sh_{\Nis}(\Sm_S) $.
    In analogy with $ \LAA $, we have chosen to use the notation $ \LRR $ for the functor inverting $ \RR $ at the level of presheaves.
    Similarly, we have used letters for the sheaf version of inverting $ \RR $.
    The ``hi'' in $ \Lhi $ stands for ``homotopy invariant''.
\end{remark}

%-------------------------------------------------------------------%
%-------------------------------------------------------------------%
%  The constant sheaf functor                                       %
%-------------------------------------------------------------------%
%-------------------------------------------------------------------%

\subsection{The constant sheaf functor}\label{sec:constantsheafdescript}

The goal of this section is to prove the following result, originally sketched for sheaves of spaces by Dugger \cites[Theorem 3.4.3]{Duggersheaves}[Proposition 8.3]{MR1870515} and Morel--Voevodsky \cite[Proposition 3.3.3]{MR1813224}.

\begin{proposition}\label{prop:Dugger}
	Let $ C $ be a presentable \category.
	Then:
	\begin{enumerate}[{\upshape (\ref*{prop:Dugger}.1)}]
		\item The constant sheaf functor $ \Gammaupperstar \colon \fromto{C}{\Sh(\Man;C)} $ factors through $ \Shhi(\Man;C) $.

		\item The global sections functor
		\begin{equation*}
			\Gammalowerstar \colon \fromto{\Shhi(\Man;C)}{C}
		\end{equation*}
		is an equivalence with inverse given by $ \Gammaupperstar $.
	\end{enumerate}
\end{proposition}

\begin{remark}
	An analogue of \Cref{prop:Dugger} holds where the category of manifolds is replaced by the category of smooth complex analytic spaces, and $ \RR $ is replaced by the open unit disk in $ \CC $; see \cite[Remarque 1.9]{MR2602027}.
	Similarly, there are many variants of this result where $ \Man $ is replaced by any reasonable category of locally contractible spaces.
\end{remark}

%-------------------------------------------------------------------%
%  Background on cotensors                                          %
%-------------------------------------------------------------------%

\subsubsection{Background on cotensors}

In order to prove \Cref{prop:Dugger}, we give a concrete description of the constant sheaf functor. 
To do this, we first recall the natural cotensoring of a presentable \category over $ \Spc $.

\begin{recollection}[(cotensoring over $ \Spc $)]\label{rec:cotensoring}
	Every presentable \category $ C $ is naturally \textit{cotensored over} the \category $ \Spc $ of spaces \HTT{Remark}{5.5.2.6}.
	That is, there is a functor 
	\begin{align*}
		(-)^{(-)} \colon \Spc^{\op} \cross C &\to C \\
		(K,X) &\mapsto X^K \comma
	\end{align*}
	along with natural equivalences
	\begin{equation*}
		\Map_{C}(X',X^K) \equivalent \Map_{\Spc}(K,\Map_C(X',X)) \period
	\end{equation*}
\end{recollection}

\begin{example}
	If $ C = \Spc $ is the \category of spaces, then the cotensoring is given by
	\begin{equation*}
		X^K \colonequals \Map_{\Spc}(K,X) \period
	\end{equation*}
\end{example}

\begin{example}\label{ex:Sptcotensor}
	If $ C = \Sp $ is the \category of spectra, then the cotensoring is given by
	\begin{equation*}
		X^K \colonequals \Hom_{\Sp}(\Sigma_+^\infty K,X) \comma
	\end{equation*}
	where $ \Hom_{\Sp} $ denotes the mapping \textit{spectrum} in $ \Sp $.
\end{example}

\begin{example}
	If $ R $ is a ring and let $ C = \D(R) $ be the derived \category of $ R $, then the cotensoring is given by
	\begin{equation*}
		A_{*}^K \colonequals \RHom_{R}(\Cup_*(K;R),A_{*}) \period
	\end{equation*}
	Here $ \Cup_*(K;R) $ is the complex of singular chains on $ K $, and $ \RHom_{R} $ is the derived $ \Hom $
	functor of chain complexes of $ R $-modules.
\end{example}

%-------------------------------------------------------------------%
%  Description of the constant sheaf functor                        %
%-------------------------------------------------------------------%

\subsubsection{Description of the constant sheaf functor}

We now give an explicit formula for the constant sheaf functor.

\begin{notation}
	For a topological space $ T $, we write $ \Piinf(T) \in \Spc $ for the underlying homotopy type of $ T $.
\end{notation}

\begin{construction}
	Let $ C $ be a presentable \category.
	Using the cotensoring of $ C $ over $ \Spc $, define a functor $ \sm \colon \fromto{C}{\Shhi(\Man;C)} $ by the assignment
	\begin{equation*}
		X \mapsto \big[M \mapsto X^{\Piinf(M)}\big] \period 
	\end{equation*}
	Given $ X \in C $, the presheaf $ \sm(X) $ is obviously $ \RR $-invariant. 
	Moreover, the van Kampen Theorem \HAa{Proposition}{A.3.2} implies that $ \sm(X) $ is a sheaf on $ \Man $.

	% Construct a natural transformation $ \alpha \colon \fromto{\Gammaupperstar}{\sm} $ as follows. 
	% For each $ X \in C $, note that we have a canonical identification $ \Gammalowerstar\sm(X) = X $.
	% Then $ \alpha $ is the natural transformation whose component at $ X $ corresponds to $ \id{X} $ under the equivalence
	% \begin{equation*}
	% 	\Map_{\Sh(\Man;C)}(\Gammaupperstar(X),\sm(X)) \equivalent \Map_C(X,\Gammalowerstar\sm(X)) = \Map_C(X,X) \period
	% \end{equation*}
\end{construction}

\begin{proposition}\label{lem:constantishi}
	Let $ C $ be a presentable \category.
	There is a canonical identification
	\begin{equation*}
		\Gammaupperstar = \sm
	\end{equation*}
	of the constant sheaf functor $ \fromto{C}{\Sh(\Man;C)} $ with the functor $ \sm $.
	In particular, $ \Gammaupperstar $ factors through the full subcategory $ \Shhi(\Man;C) \subset \Sh(\Man;C) $
\end{proposition}

\begin{proof}
	Since the restriction functor
	\begin{equation*}
		\restrict{(-)}{\Eucop} \colon \fromto{\Sh(\Man;C)}{\Sh(\Euc;C)}
	\end{equation*}
	is an equivalence (\Cref{lem:hypersheavesonCart}), it suffices to prove that the composite
	\begin{equation}
		\begin{tikzcd}[sep=3.5em]
			C \arrow[r, "\sm"] & \Sh(\Man;C) \arrow[r, "\restrict{(-)}{\Eucop}", "\sim"'] & \Sh(\Euc;C)
		\end{tikzcd}
	\end{equation}
	is the constant sheaf functor.
	To see this, note that since Euclidean spaces are contractible, for each $ X \in C $, the sheaf $ \restrict{\sm(X)}{\Eucop} \colon \fromto{\Eucop}{C} $ is actually the constant functor at $ X $.
	Since $ \restrict{\sm(X)}{\Eucop} $ is also a sheaf, $ \restrict{\sm}{\Eucop} $ is the constant sheaf functor.
\end{proof}

\Cref{prop:Dugger} now follows with the facts that $ \Gammaupperstar $ is fully faithful and $ \Gammalowerstar $ is conservative when restricted to the $ \RR $-invariant sheaves (\Cref{lem:checkequivforhionpt}), combined with the following basic lemma from category theory.

\begin{lemma}\label{lem:conservativeradjffladj}
	Let $ \adjto{\fupperstar}{A}{B}{\flowerstar} $ be an adjunction between \categories, and assume that the left adjoint $ \fupperstar $ is fully faithful.
	Then $ \fupperstar $ is an equivalence if and only if $ \flowerstar $ is conservative.
\end{lemma}

\begin{proof}
	If $ \fupperstar $ is an equivalence, then $ \flowerstar $ is also an equivalence, hence conservative.

	On the other hand, assume that $ \flowerstar $ is conservative. 
	Since the left adjoint $ \fupperstar $ is fully faithful, the unit $ \fromto{\id{A}}{\flowerstar\fupperstar} $ is an equivalence.
	Hence $ \fupperstar $ is an equivalence if and only if for each object $ X \in B $, the counit $ \counit_{X} \colon \fromto{\fupperstar\flowerstar(X)}{X} $ is an equivalence.
	Since $ \flowerstar $ is conservative, the counit $ \counit_{X} $ is an equivalence if and only if
	\begin{equation*}
		\flowerstar(\counit_{X}) \colon \fromto{\flowerstar\fupperstar\flowerstar(X)}{\flowerstar(X)}
	\end{equation*}
	is an equivalence.
	The claim now follows from the fact that the unit $ \fromto{\id{A}}{\flowerstar\fupperstar} $ is an equivalence and the triangle identity.
\end{proof}

\begin{proof}[Proof of \Cref{prop:Dugger}]
	Since $ \Gammaupperstar \colon \incto{C}{\Shhi(\Man;C)} $ is fully faithful and $ \Gammalowerstar \colon \fromto{\Shhi(\Man;C)}{C} $ is conservative (\Cref{lem:checkequivforhionpt}), we conclude by \Cref{lem:conservativeradjffladj}.
\end{proof}

%-------------------------------------------------------------------%
%  Consequences of \Cref{prop:Dugger}                               %
%-------------------------------------------------------------------%

\subsubsection{Consequences of \texorpdfstring{\Cref{prop:Dugger}}{\Cref*{prop:Dugger}}}

We finish this section by observing that \Cref{prop:Dugger} gives us a chain of four adjoints relating $ \Sh(\Man;C) $ and $ C $.

\begin{observation}\label{obs:LhiandRhi}
	By \Cref{prop:Dugger}, the \category $ \Shhi(\Man;C) $ is presentable and is closed under colimits in $ \Sh(\Man;C) $.
	Moreover, since limits in $ \Sh(\Man;C) $ are computed pointwise, the full subcategory $ \Shhi(\Man;C) $ is also closed under limits.
	The Adjoint Functor Theorem \HTT{Corollary}{5.5.2.9} implies that the inclusion
	\begin{equation*}
		\incto{\Shhi(\Man;C)}{\Sh(\Man;C)}
	\end{equation*}
	admits both a left and right adjoint.
	This gives another way of seeing that that \textit{homotopification} functor $ \Lhi $ of \Cref{obs:LRRandLhi} exists.
	We denote the left and right adjoints to the inclusion
	\begin{equation*}
		\incto{\Shhi(\Man;C)}{\Sh(\Man;C)}
	\end{equation*}
	by $ \Lhi $ and $ \Rhi $, respectively.
\end{observation}

\begin{nul}\label{nul:Gammaadjunctions}
	As a consequence we have a chain of adjunctions
	\begin{equation*}
		\begin{tikzcd}
			\Sh(\Man;C) \arrow[r, shift left=1.25ex, "\Lhi"] \arrow[r, shift right=1.25ex, "\Rhi"'] & \Shhi(\Man;C) \arrow[l, hooked'] \arrow[r, shift right, "\Gammalowerstar"'] & C \comma \arrow[l, shift right, "\Gammaupperstar"', "\sim"{yshift=0.15ex}] 
		\end{tikzcd}
	\end{equation*}
	where the right-hand adjunction is an adjoint equivalence.
	Thus the functor $ \Gammalowerstar \Lhi $ is left adjoint to the constant sheaf functor $ \Gammaupperstar \colon \fromto{C}{\Sh(\Man;C)} $.
	We denote this left adjoint by
	\begin{equation*}
		\Gammalowershriek \colon \fromto{\Sh(\Man;C)}{C} \period
	\end{equation*}
	In this notation, we have equivalences $ \Lhi \equivalent \Gammaupperstar \Gammalowershriek $ and $ \Rhi \equivalent \Gammaupperstar \Gammalowerstar $.
	Thus we have a chain of four adjoints
	\begin{equation}\label{diag:4adjoints}
		\begin{tikzcd}[sep=4em]
			\Sh(\Man;C) \arrow[r, shift left=2ex, "\Gammalowershriek"] \arrow[r, shift right=0.75ex, "\Gammalowerstar"{description, xshift=0.5em}] & C \comma \arrow[l, shift left=2ex, "\Gammauppershriek", hooked'] \arrow[l, shift right=0.75ex, "\Gammaupperstar"{description, xshift=-0.5em}, hooked']
		\end{tikzcd}
	\end{equation}
	where functors lie above their right adjoints.
	\Cref{sec:localization} is dedicated to providing an explicit formula for $ \Gammalowershriek $ (see \Cref{cor:Gammalowershriek})
\end{nul}

\begin{remark}[(cohesion)]
	Much of the structure of sheaves on $ \Man $ that we are interested in for studying differential cohomology (particularly \Cref{sec:stable}) only depends on the existence of the chain of four adjoints \eqref{diag:4adjoints}.
	In the case where $ C = \Spc $, the existence of these extra adjoints for the global sections geometric morphism (along with the condition that the extreme left adjoint $ \Gammalowershriek $ preserve finite products; see \Cref{cor:Gammalowershriek}) is what Schreiber calls a \textit{cohesive \topos} \cite[Definition 3.4.1]{Schreiber:cohesive}.
	The primary examples of cohesive \topoi are \textit{global spaces} \cite{Rezk:global} and variants of sheaves on $ \Man $.
	Cohesive \topoi are a very general setting in which one can talk about a generalized form of ``differential
	cohomology''.

	Many of the ideas about cohesive \topoi go back to work of Lawvere \cites{MR1257681}{MR2125786}{MR2369017}[\S C.1]{MR1965482} as well as Simpson--Teleman \cite{SimpsonTeleman:deRham}.
\end{remark}

%-------------------------------------------------------------------%
%-------------------------------------------------------------------%
%  Digression: criteria for ℝ-invariance                            %
%-------------------------------------------------------------------%
%-------------------------------------------------------------------%

\subsection{Digression: criteria for \texorpdfstring{$ \RR $}{ℝ}-invariance}\label{sec:digressionRRinvar}

We conclude this chapter by collecting two reformulations of $ \RR $-invariance due to Voevodsky \cite[Lemma 2.16]{MR2242284}.
We do not have occasion to use these criteria in this text, but they are nonetheless quite useful.
To state these reformulations, we first need some notation.

\begin{notation}
	Let $ M $ be a manifold and $ t \in \RR $.
	We write $ i_{M,t} \colon \incto{M}{M \cross \RR} $ for the closed embedding defined by $ \goesto{x}{(x,t)} $.
\end{notation}

\begin{observation}\label{obs:iMcrossR}
	For each manifold $ M $ and $ t \in \RR $, the map $ i_{M \cross \RR,t} $ is given by the composite 
	\begin{equation*}
		\begin{tikzcd}[column sep=4.5em]
			M \cross \RR \arrow[r, hooked, "i_{M,t} \cross \id{\RR}"] & M \cross \RR \cross \RR \arrow[r, "\id{M} \cross \swap", "\sim"' ] & M \cross \RR \cross \RR \comma
		\end{tikzcd}
	\end{equation*}
	where $ \swap \colon \isomto{\RR \cross \RR}{\RR \cross \RR} $ is the map that swaps the two factors.
\end{observation}

\begin{proposition}\label{prop:differentRinvariance}
	Let $ C $ be a presentable \category.
	The following are equivalent for a presheaf $ F \colon \fromto{\Manop}{C} $:
	\begin{enumerate}[{\upshape (\ref*{prop:differentRinvariance}.1)}]
		\item\label{prop:differentRinvariance.1} The presheaf $ F $ is $ \RR $-invariant.

		\item\label{prop:differentRinvariance.2} For every manifold $ M $, the induced map
		\begin{equation*}
			\iupperstar_{M,0} \colon \fromto{F(M \cross \RR)}{F(M)} 
		\end{equation*}
		is an equivalence.

		\item\label{prop:differentRinvariance.3} For every manifold $ M $, the induced maps
		\begin{equation*}
			\iupperstar_{M,0}, \iupperstar_{M,1} \colon \fromto{F(M \cross \RR)}{F(M)} 
		\end{equation*}
		are equivalent.
	\end{enumerate}
\end{proposition}

\begin{proof}
	Since the embeddings
	\begin{equation*}
		i_{M,0}, i_{M,1} \colon \incto{M}{M \cross \RR}
	\end{equation*}
	are sections of the projection $ \pr_M \colon \fromto{M \cross \RR}{M} $, it is clear that \enumref{prop:differentRinvariance}{1} $ \Leftrightarrow $ \enumref{prop:differentRinvariance}{2} and \enumref{prop:differentRinvariance}{1} $ \Rightarrow $ \enumref{prop:differentRinvariance}{3}.

	To complete the proof, we show that \enumref{prop:differentRinvariance}{3} $ \Rightarrow $ \enumref{prop:differentRinvariance}{1}.
	Assuming \enumref{prop:differentRinvariance}{3}, since $ i_{M,0} $ is a section or the projection $ \pr_M \colon \fromto{M \cross \RR}{M} $, it suffices to show that we have an equivalence
	\begin{equation*}
		\prupperstar_M \iupperstar_{M,0} \equivalent \id{F(M \cross \RR)} \period
	\end{equation*}
	To see this, write $ \mult \colon \fromto{\RR \cross \RR}{\RR} $ for the multiplication map, and notice that we have a commutative diagram in $ \Man $
	\begin{equation}\label{diag:multcom}
		\begin{tikzcd}[row sep=3em, column sep=4em]
			M \cross \RR \arrow[r, hooked, "i_{M,1} \cross \id{\RR}"] \arrow[dr, equals] & M \cross \RR \cross \RR \arrow[d, "\id{M} \cross \mult" description] & M \cross \RR \arrow[l, hooked', "i_{M,0} \cross \id{\RR}"'] \arrow[d, "\pr_M"] \\
			 & M \cross \RR & M \arrow[l, hooked', "i_{M,0}"]
		\end{tikzcd}
	\end{equation}
	Combining the assumption that $ \iupperstar_{M \cross \RR,0} \equivalent \iupperstar_{M \cross \RR,1} $ with \Cref{obs:iMcrossR} shows that
	\begin{equation}\label{eq:prodequiv}
		(i_{M,0} \cross \id{\RR})\upperstar \equivalent (i_{M,1} \cross \id{\RR})\upperstar \period
	\end{equation}
	\Cref{eq:prodequiv} and the commutativity of the diagram \eqref{diag:multcom} now show that
	\begin{align*}
		\prupperstar_M \iupperstar_{M,0} &\equivalent (i_{M,0} \cross \id{\RR})\upperstar \of (\id{M} \cross \mult)\upperstar \\
		&\equivalent (i_{M,1} \cross \id{\RR})\upperstar \of (\id{M} \cross \mult)\upperstar \\
		&\equivalent \id{F(M \cross \RR)} \comma
	\end{align*}
	as desired.
\end{proof}
