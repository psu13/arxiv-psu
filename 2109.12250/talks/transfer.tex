%!TEX root = ../diffcoh.tex

\section{Digression: the Transfer Conjecture}\label{sec:digressiononTransferConjecture}
\textit{by Peter Haine}

%-------------------------------------------------------------------%
%-------------------------------------------------------------------%
%  Introduction                                                     %
%-------------------------------------------------------------------%
%-------------------------------------------------------------------%

\subsection{Introduction}

Let $ X $ be a space.
We have seen that the constant sheaf of spaces $ \Gammaupperstar(X) $ on $ \Man $ is given by the formula
\begin{equation*}
	 \Gammaupperstar(X) = \Map_{\Space}(\Piinf(M),X)
\end{equation*}
(\Cref{lem:constantishi}).
If $ X = \Omega^{\infty} E $ is the infinite loop space of a spectrum $ E $, then the sheaf $ \Gammaupperstar(X) $ acquires additional functoriality: for any finite covering map between manifolds $ f \colon \fromto{N}{M} $, the \textit{Becker--Gottlieb transfer}
\begin{equation*}
	\fromto{\Sigma_{+}^{\infty} \Piinf(M)}{\Sigma_{+}^{\infty} \Piinf(N)}
\end{equation*}
\cite[Definition 3.11]{Haugseng:Becker-Gottlieb} induces a \textit{transfer} map
\begin{equation*}
	\flowerstar \colon \fromto{\Gammaupperstar(X)(N)}{\Gammaupperstar(X)(M)} \period
\end{equation*} 
This enhanced functoriality can be used to make $ \Gammaupperstar(X) $ into a copresheaf on a $ 2 $-category $ \Corfcov $ with objects smooth manifolds and morphisms \textit{correspondences}
\begin{equation*}
	\begin{tikzcd}[row sep=1.5em, column sep=0.75em]
		& N \arrow[dr, "f"] \arrow[dl] & \\
		M_0 & & M_1 \comma
	\end{tikzcd}
\end{equation*}
where $ f $ is a finite covering map.
Composition in $ \Corfcov $ is given by pullback. 

For a sheaf $ F $ on $ \Man $, Quillen conjectured that an extension of $ F $ to $ \Corfcov $ is just another way of encoding an $ \Einf $-structure on $ F $.
However, when Quillen originally formulated this \textit{Transfer Conjecture}, the language to express the higher coherences necessary for the validity of the result was not available.
Moreover, Quillen's original formulation was \textit{dis}proven by Kraines and Lada \cites{MR557187}{MR3243401}.

The goal of this section is to explain how to deduce the following corrected version of the Transfer Conjecture from very general results of Bachmann--Hoyois on commutative algebras and \categories of spans \cite[Appendix C]{MotivicNorms:BachmannHoyois}.

\begin{theorem}[{(Transfer Conjecture; \Cref{cor:transferconjecture,cor:transferconjecturehi})}]\label{thm:transfer}
	Let $ C $ be a presentable \category.
	There is an equivalence of \categories
	\begin{equation*}
		\Fun_{\loc}(\Corfcov,C) \equivalence \Sh(\Man;\CMon(C))
	\end{equation*}
	between functors $ \fromto{\Corfcov}{C} $ whose restriction to $ \Manop $ is a sheaf and sheaves of commutative monoids in $ C $.
	This further restricts to an equivalence
	\begin{equation*}
		\Fun_{\loc,\RR}(\Corfcov,C) \equivalence \CMon(C)
	\end{equation*}
	between functors $ \fromto{\Corfcov}{C} $ whose restriction to $ \Manop $ is an $ \RR $-invariant sheaf and commutative monoids in $ C $.
\end{theorem}

\begin{example}
	Setting $ C = \Spc $ in \Cref{thm:transfer} gives an equivalence between functors
	\begin{equation*}
		\fromto{\Corfcov}{\Spc}
	\end{equation*}
	whose restriction to $ \Manop $ is an $ \RR $-invariant sheaf and $ \Einf $-spaces.
	Restricting to grouplike objects on both sides and applying the Segal's Recognition Principle for connective spectra \HA{Remark}{5.2.6.26} provides an equivalence between grouplike objects of $ \Fun_{\loc,\RR}(\Corfcov,\Space) $ and the \category $ \Sp_{\geq 0} $ of connective spectra. 
\end{example}

\begin{remark}
	The Becker--Gottlieb transfer is defined in more generality than finite covering maps; for example, for proper submersions.
	It is possible to modify \Cref{thm:transfer} to encode this additional generality.
	However, since pullbacks along proper submersions do not exist in the category of manifolds, in order for composition of correspondences where one leg is proper to be defined, one needs to work with \textit{derived} manifolds \cites{arXiv:1905.06195}{MR2641940}.
	For the sake of simplicity, we will satisfy ourselves with just working with manifolds and finite covering maps.
\end{remark}

In order to give a more precise formulation of \Cref{thm:transfer}, we'll first review constructing $ 2 $-categories of \textit{correspondences} or \textit{spans} from $ 1 $-categories (\cref{subsec:spans}). 
We then briefly recall the role that \categories of spans play in encoding $ \Einf $-structures (\cref{subsec:spansCMon}).
Finally, we walk through \cite[Appendix C]{MotivicNorms:BachmannHoyois} in the case of interest and explain how to deduce the Transfer Conjecture from their results (\cref{subsec:transferproof}).  

%-------------------------------------------------------------------%
%  Categories of spans                                              %
%-------------------------------------------------------------------%

\subsection{Categories of spans}\label{subsec:spans}

In this section we explain how to construct the $ 2 $-category $ \Corfcov $ of correspondences of manifolds appearing in the Transfer Conjecture.
This is a special case of a general construction for \categories due to Barwick \cite[§§3--5]{MR3558219}.
If $ D $ is an $ n $-category, then Barwick's \category of spans in $ D $ is an $ (n+1) $-category. 
In order to avoid explaining how to deal with the homotopy coherence problems that arise, we only present the $ 1 $-categorical case as we can give a simple definition as a $ 2 $-category.

\begin{construction}[($ 2 $-category of spans)]\label{def:spancat}
	Let $ D $ be a $ 1 $-category, and let $ L, R \subset \Mor(D) $ be two classes of morphisms in $ D $ satisfying the following properties:
	\begin{enumerate}[{\upshape (\ref*{def:spancat}.1)}]
		\item The classes $ L $ and $ R $ contain all isomorphisms.

		\item The classes $ L $ and $ R $ are each stable under composition.

		\item Given a morphism $ \el \colon \fromto{X}{Z} $ in $ L $ and morphism $ r \colon \fromto{Y}{Z} $ in $ R $, there exists a pullback diagram
		\begin{equation*}
			\begin{tikzcd}
				W \arrow[r, "\rbar"] \arrow[d, "\elbar"'] \arrow[dr, phantom, very near start, "\lrcorner"] & Y \arrow[d, "r"] \\
             	X \arrow[r, "\el"'] & Z 
			\end{tikzcd}
		\end{equation*}
		in $ D $ where $ \elbar \in L $ and $ \rbar \in R $.
	\end{enumerate}

	Define a $ 2 $-category $ \Span(D;L,R) $ as follows.
	The objects of $ \Span(D;L,R) $ are the objects of $ D $.
	Given objects $ X_0, X_1 \in D $, the groupoid $ \Map_{\Span(D;L,R)}(X_0,X_1) $ has objects diagrams 
	\begin{equation*}
		\begin{tikzcd}[row sep=1.5em, column sep=0.75em]
			& Y \arrow[dr, "r"] \arrow[dl, "\el"'] & \\
			X_0 & & X_1 \comma
		\end{tikzcd}
	\end{equation*}
	in $ D $ where $ \el \in L $ and $ r \in R $, and morphisms isomorphisms of diagrams.
	Composition is given by pullback of spans: given morphisms $ X_0 \to X_1 $ and $ X_1 \to X_2 $ corresponding to spans
	\begin{equation*}
		\begin{tikzcd}[row sep=1.5em, column sep=0.75em]
			& Y \arrow[dr] \arrow[dl] & \\
			X_0 & & X_1 
		\end{tikzcd} \andeq
		\begin{tikzcd}[row sep=1.5em, column sep=0.75em]
			& Z \arrow[dr] \arrow[dl] & \\
			X_1 & & X_2 \comma
		\end{tikzcd}
	\end{equation*}
	the composite morphism $ X_0 \to X_2 $ in $ \Span(D;L,R) $ is defined as the large pullback span
	\begin{equation*}
		\begin{tikzcd}[column sep={6ex,between origins}, row sep={6ex,between origins}]
			&  & \displaystyle Y \cross_{X_1} Z \arrow[dr] \arrow[dl] & & \\
			& Y \arrow[dr] \arrow[dl] &  & Z \arrow[dr] \arrow[dl] & \\
			X_0 & & X_1 & & X_2 \period
		\end{tikzcd}
	\end{equation*}
\end{construction}

\begin{notation}
	Let $ D $ be a $ 1 $-category.
	We write $ \all \colonequals \Mor(D) $ for the class of all morphisms in $ D $.
	If $ D $ has pullbacks, we write
	\begin{equation*}
		\Span(D) \colonequals \Span(D;\all,\all)
	\end{equation*}
	for the $ 2 $-category of spans of arbitrary morphisms in $ D $.
\end{notation}

\begin{observation}
	Let $ D $ be a category and $ R $ a class of morphisms in $ D $ such that the pullback of a morphism in $ R $ along an arbitrary morphism of $ D $ exists, and the class $ R $ is stable under pullback.
	Then there is a natural faithful functor
	\begin{equation*}
		\fromto{D^{\op}}{\Span(D;\all,R)}
	\end{equation*}
	given by the identity on objects, and on morphisms by sending a morphism $ f \colon \fromto{X}{Y} $ to the span
	\begin{equation*}
		\begin{tikzcd}[row sep=1.5em, column sep=0.75em]
			& X \arrow[dl, "f"'] \arrow[dr, equals] & \\
			Y & & X \period
		\end{tikzcd}
	\end{equation*}
\end{observation}

\begin{example}
	Write $ \fcov \subset \Mor(\Man) $ for the class of finite covering maps of manifolds.
	Note that the pullback of a finite covering map of manifolds along any morphism exists, and the class of finite covering maps is stable under pullback.
	We write
	\begin{equation*}
		\Corfcov \colonequals \Span(\Man;\all,\fcov)
	\end{equation*}
	for the $ 2 $-category with objects manifolds and morphisms \textit{correspondences}\footnote{The term
	``correspondence'' is just another name for a span; ``correspondence'' seems to be the more common term in geometry.} of manifolds 
	\begin{equation*}
		\begin{tikzcd}[row sep=1.5em, column sep=0.75em]
			& N \arrow[dr, "f"] \arrow[dl] & \\
			M_0 & & M_1 \comma
		\end{tikzcd}
	\end{equation*}
	where $ f $ is a finite covering map.
\end{example}

\begin{example}
	Write $ \foldtext \subset \Mor(\Man) $ for the class of maps that are finite coproducts of fold maps of manifolds, i.e., finite coproducts of fold maps $ \nabla \colon \fromto{M^{\coproduct i}}{M} $ from a finite disjoint union of copies of $ M $ to $ M $.
	Note that coproduct decompositions are stable under all pullbacks that exist in the category of manifolds, hence the class $ \foldtext $ is stable under pullback.
	We write
	\begin{equation*}
		\Corfold \colonequals \Span(\Man;\all,\foldtext)
	\end{equation*}
	for the $ 2 $-category with objects manifolds and morphisms correspondences of manifolds 
	\begin{equation*}
		\begin{tikzcd}[row sep=1.5em, column sep=0.75em]
			& N \arrow[dr, "f"] \arrow[dl] & \\
			M_0 & & M_1 \comma
		\end{tikzcd}
	\end{equation*}
	where $ f $ is a finite coproduct of fold maps.

	Note that $ \foldtext \subset \fcov $, so that $ \Corfold $ defines a subcategory of $ \Corfcov $ that contains all objects.
\end{example}

%-------------------------------------------------------------------%
%  Spans and commutative monoids                                    %
%-------------------------------------------------------------------%

\subsection{Spans and commutative monoids}\label{subsec:spansCMon}

In this section we briefly recall the role that \categories of spans play in encoding $ \Einf $-structures.
We begin by introducing the relevant $ 2 $-category of spans.

\begin{notation}
	Write $ \Fin $ for the category of finite sets.
	Given \acategory $ C $ with a terminal object, we write $ * $ for the terminal object.
\end{notation}

\begin{recollection}
	Let $ C $ be \acategory with finite products.
	A \textit{commutative monoid} or \textit{$ \Einf $-monoid} in $ C $ is a functor $ M \colon \fromto{\Finstar}{C} $ such that $ \equivto{M(*)}{*} $ and for each integer $ n \geq  1 $, the collapse maps $ \fromto{\{1,\ldots,n\}_+}{\{i\}_+} $ induce an equivalence
	\begin{equation*}
		\equivto{M(\{1,\ldots,n\}_+)}{\prod_{i=1}^n M(\{i\}_+)} \period
	\end{equation*}
	We write $ \CMon(C) \subset \Fun(\Finstar,C) $ for the full subcategory spanned by the commutative monoids.
\end{recollection}

\begin{observation}
	The $ 2 $-category $ \Span(\Fin) $ is \textit{semiadditive}: the direct sum in $ \Span(\Fin) $ is given by disjoint union of finite sets.
	See \cites[Lemma C.3]{MotivicNorms:BachmannHoyois}[Proposition 4.3]{MR3558219} for more general results on the semiadditivity of \categories of spans.
\end{observation}

\begin{observation}
	Write $ \inj $ for the class of injective maps in $ \Fin $.
	There the functor $ \equivto{\Finstar}{\Span(\Fin;\inj,\all)} $ given by sending $ \goesto{X_{+}}{X} $ and a morphism $ f \colon \fromto{X_{+}}{Y_{+}} $ to the span
	\begin{equation*}
		\begin{tikzcd}[row sep=1.5em, column sep=0.75em]
			& \finverse(Y) \arrow[dr, "f"] \arrow[dl, hooked'] & \\
			X & & Y 
		\end{tikzcd}
	\end{equation*}
	is an equivalence of categories.

	The category $ \Span(\Fin;\inj,\all) $ is often referred to as the category of finite sets and \textit{partially defined maps}.
\end{observation}

The importance of transfers in $ \Einf $-structures is explained by the following universal property of the $ 2 $-category $ \Span(\Fin) $ of spans of finite sets.

\begin{proposition}[{(Cranch \cites[Proposition C.1]{MotivicNorms:BachmannHoyois}[§5]{Cranch:Thesis})}]\label{prop:BH.C.1}
	Let $ C $ be \acategory with finite products.
	Then the restriction 
	\begin{equation*}
		\fromto{\Fun(\Span(\Fin),C)}{\Fun(\Finstar,C)}
	\end{equation*}
	along the inclusion $ \fromto{\Finstar}{\Span(\Fin)} $ restricts to an equivalence between:
	\begin{enumerate}[{\upshape (\ref*{prop:BH.C.1}.1)}]
		\item Functors $ M \colon \fromto{\Span(\Fin)}{C} $ that preserve finite products (equivalently, $ \restrict{M}{\Finop} $ preserves finite products).

		\item Commutative monoids in $ C $.
	\end{enumerate}
	The inverse is given by right Kan extension.
\end{proposition}

The $ 2 $-category $ \Span(\Fin) $ has a second (related) universal property: $ \Span(\Fin) $ is the free semiadditive \category generated by a single object.

\begin{proposition}[{(Harpaz \cite[Theorem 1.1]{Harpaz:Spans})}]
	Let $ C $ be a semiadditive \category.
	Then evaluation at $ * \in \Span(\Fin) $ defines an equivalence
	\begin{equation*}
		\equivto{\Fun^{\directsum}(\Span(\Fin),C)}{C} \period
	\end{equation*}
\end{proposition}

%-------------------------------------------------------------------%
%  The Transfer Conjecture after Bachmann–Hoyois                    %
%-------------------------------------------------------------------%

\subsection{The Transfer Conjecture after Bachmann--Hoyois}\label{subsec:transferproof}

In this section we outline work of Bachmann--Hoyois that implies the Transfer Conjecture \cite[Appendix C]{MotivicNorms:BachmannHoyois}. 
The perspective on commutative monoids in $ D $ as finite product-preserving functors $ \fromto{\Span(\Fin)}{D} $ (\Cref{prop:BH.C.1}) is fundamental to proving the Transfer Conjecture.

The first step is to relate finite product-preserving functors $ \fromto{\Corfold}{D} $ to presheaves of commutative monoids on $ \Man $.
Then we impose the sheaf condition to pass from $ \Corfold $ to $ \Corfcov $.

\begin{notation}
	Write $ \Theta \colon \fromto{\Manop \cross \Span(\Fin)}{\Corfold} $ for the functor given on objects by the assignment
	\begin{equation*}
		(M,I) \mapsto M^{\coproduct I}
	\end{equation*}
	and on morphisms by the assignment
	\begin{equation*}
		(M \to N, I_0 \leftarrow J \to I_1) \quad \mapsto
		\quad
		\begin{tikzcd}[row sep=1.5em, column sep=0.75em]
			& M^{\coproduct J} \arrow[dl] \arrow[dr] & \\
			N^{\coproduct I_0} & & M^{\coproduct I_1} \period
		\end{tikzcd} 
	\end{equation*}
\end{notation}

The functor $ \Theta $ is the universal functor that preserves finite products in each variable:

\begin{proposition}[{\cite[Proposition C.5]{MotivicNorms:BachmannHoyois}}]\label{prop:BH.C.5}
	Let $ C $ be \acategory with finite products.
	Then the restriction functor
	\begin{equation*}
		\Theta\upperstar \colon \Fun(\Corfold,C) \to \Fun(\Manop \cross \Span(\Fin), C)
	\end{equation*}
	restricts to an equivalence 
	\begin{equation*}
		\Funcross(\Corfold,C) \equivalence \Funcross(\Manop, \CMon(C)) \period
	\end{equation*}
	The inverse is given by right Kan extension along $ \Theta $.
\end{proposition}

Since every finite covering map is locally a fold map, we see:

\begin{proposition}[{\cite[Proposition C.11]{MotivicNorms:BachmannHoyois}}]\label{prop:BH.C.11}
	Let $ C $ be \acategory with finite products.
	Then the restriction functor
	\begin{equation*}
		\Fun(\Corfcov,C) \to \Fun(\Corfold,C) 
	\end{equation*}
	restricts to an equivalence between the full subcategories of those functors whose restrictions to $ \Manop $ are sheaves.
	The inverse is given by right Kan extension.
\end{proposition}

\begin{notation}
	Write
	\begin{equation*}
		\Fun_{\loc}(\Corfcov,C) \subset \Fun(\Corfcov,C)
	\end{equation*}
	for the full subcategory spanned by those functors $ F $ whose restrictions to $ \Manop $ are sheaves.
\end{notation}

We now arrive at Quillen's Transfer Conjecture:

\begin{corollary}[(Transfer Conjecture)]\label{cor:transferconjecture}
	Let $ C $ be \acategory with all limits.
	Restriction along the inclusion $ \incto{\Manop}{\Corfcov} $ defines an equivalence of \categories 
	\begin{equation*}
		\Fun_{\loc}(\Corfcov,C) \equivalence \Sh(\Man;\CMon(C)) \period
	\end{equation*}
\end{corollary}

\noindent Combining \Cref{prop:Dugger} and \Cref{cor:transferconjecture} shows:

\begin{corollary}\label{cor:transferconjecturehi}
	Let $ C $ be a presentable \category.
	Restriction along the inclusion
	\begin{equation*}
		\incto{\Manop}{\Corfcov}
	\end{equation*}
	defines an equivalence of \categories 
	\begin{equation*}
		\Fun_{\loc,\RR}(\Corfcov,C) \equivalence \Shhi(\Man;\CMon(C)) \period
	\end{equation*}
	Post-composing with the global sections functor $ \Gammalowerstar $ defines an equivalence
	\begin{equation*}
		\Fun_{\loc,\RR}(\Corfcov,C) \equivalence \CMon(C) \period
	\end{equation*} 
\end{corollary}

\begin{nul}
	Unwinding the definitions we see that restriction along the inclusion
	\begin{equation*}
		\Span(\Fin) \subset \Corfcov
	\end{equation*}
	defines an equivalence
	\begin{equation*}
		\Fun_{\loc,\RR}(\Corfcov,C) \equivalence \Funcross(\Span(\Fin),C) \equivalent \CMon(C) \period
	\end{equation*}
\end{nul}
