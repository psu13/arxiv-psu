%!TEX root = ../diffcoh.tex

\section{Chern--Simons invariants}
\textit{by YiYu (Adela) Zhang}
\label{config_spaces}

%-------------------------------------------------------------------%
%-------------------------------------------------------------------%
%  Motivation/Review                                                %
%-------------------------------------------------------------------%
%-------------------------------------------------------------------%

Our first application is to the theory of Chern--Simons forms and invariants, tools in geometry which are closely
tied to differential cohomology. We first mentioned these in \Cref{secondary_chern_simons}, where we said that
Chern--Simons invariants are defined to be the secondary invariants associated to the on-diagonal differential
characteristic classes constructed in Chern--Weil theory. But they also have a much more geometric description,
given by integrating a specific form built from the connection and curvature forms. These two descriptions are part
of the reason Chern--Simons invariants are so useful: one can use homotopy-theoretic methods in differential
cohomology to learn facts about geometry, and vice versa. This will be a common theme throughout this part of the
book, and Chern--Simons forms will appear several times.

We begin in \cref{ssec:lens_CS}, defining and discussing Chern--Simons forms associated to a principal bundle
$\pi\colon P\to M$ with connection, and relating them to the differential lifts of Chern--Weil characteristic
classes from \cref{DifferentialCharacteristicClasses}. In \cref{cs_invariants}, we focus on the case when $\pi$ is
a principal $\SU_2$-bundle over a $3$-manifold, where we can descend the Chern--Simons invariant from an integral
on $P$ to an integral on $M$.  Finally, in \cref{config_ssec}, we show an application of Chern--Simons invariants,
as a tool to determine when two-point configuration spaces of lens spaces are homotopy equivalent.

%Our first application of the ideas we have developed in this book is an idea of
%Evans-Lee--Saveliev \cite{deletedsquare} using Chern--Simons invariants of configuration spaces to distinguish lens
%spaces. Chern--Simons invariants are secondary invariants associated to Cheeger--Simons' differential Chern--Weil
%characteristic classes that we constructed in \cref{DifferentialCharacteristicClasses}. In this and the next
%chapter, we will get into the geometry of Chern--Simons invariants and apply them to geometric questions.
%
%\subsection{Motivation/Review}
%
%Let $P$ be a principal $G$-bundle over a manifold $M$ and $A$ a $G$-connection on $P$ with curvature $F_A$. 
%Recall the Chern--Weil homomorphism $$\CW:\Sym(\g^*)^{\Ad}\rightarrow \HdR^*(M;\RR),$$ which sends an homogeneous
%invariant polynomial $f $ to a real cohomology class $[f (F_A)]$.\index[terminology]{Chern--Weil homomorphism}
%
%For instance, the image $\hat{c}_k$ of the $k$-th Chern class $c_k$ in $\H^{2k}(M;\RR)$ is given by $\tr(\bigwedge^kF_A)$. Now suppose that $A$ is a flat connection on a vector bundle over $M$, then  $\hat{c}_k=0$ and there is a lifting of $c_k$ in the exact sequence
%\begin{equation*}
%  \cdots\rightarrow \H^{2k-1}(M;\RR/\ZZ)\rightarrow \H^{2k}(M;\ZZ)\rightarrow \H^{2k}(M;\RR)\rightarrow\cdots
%\end{equation*}
%A canonical way to produce the lifting was first given by the Chern--Simons form in \cite{cs}. 
%The Cheeger--Simons differential characters came out as a refinement of the Chern--Simons form.

%-------------------------------------------------------------------%
%-------------------------------------------------------------------%
%  Chern–Simons forms                                               %
%-------------------------------------------------------------------%
%-------------------------------------------------------------------%

\subsection{Chern--Simons forms}
\label{ssec:lens_CS}

Let $G$ be a compact Lie group and $\pi\colon P\rightarrow M$ a principal $G$-bundle. 
Fix a degree-$k$ invariant polynomial $f \in\Sym^k(\g^*)^G$. Given a connection $\conn$ on $P$ with
curvature $\curvature{\conn}$, we will write $f(\curvature{\conn})\in \HdR^{2k}(M)$ for the associated Chern--Weil form.
%We will write $f (A)=f (F_A)$ for a connection $A$.

\begin{recall}
\label{recall_connection}
  A \textit{connection} $\conn$ on the principal $G$-bundle $\pi\colon P\to M$ is a $\g$-valued $1$-form on $P$
  which is $G$-equivariant in the sense that $(R_g)^*\conn =\Ad_{g^{-1}}\conn$, and it
  is ``the identity'' on tangent vectors along the fiber, i.e.\ $\conn(X_{\xi})=\xi$ for $\xi\in\g$ and $X_{\xi}$ its
  fundamental vector field.

  The \textit{curvature} of $\conn$, which we usually denote $\curvature{\conn}$, is the form $\d\omega + [\omega,
  \omega]\in\Omega^2(M;\g)$.
\end{recall}
\index[terminology]{connection!on a principal $G$-bundle}
Analogous to connections on vector bundles, a $G$-connection corresponds to a splitting $TP\cong H\oplus V$, where
$V$ is the vertical tangent bundle (the kernel of $\pi_*\colon TP\to TM$), and $H$ is the horizontal tangent
bundle. A priori, there is only a short exact sequence
\index[terminology]{vertical tangent bundle}
\index[terminology]{horizontal tangent bundle}
\begin{equation}
	\begin{tikzcd}[ampersand replacement=\&]
        0 \& V
        \&  TP
        \& H
        \& 0;
        \arrow[from=1-1, to=1-2]
        \arrow[from=1-2, to=1-3]
        \arrow[from=1-3, to=1-4]
        \arrow[from=1-4, to=1-5]
\end{tikzcd}
\end{equation}
a connection is a $G$-equivariant splitting. Because the fibers of a principal $G$-bundle are $G$-torsors,
there is an isomorphism $V\cong\underline{\g}$, and the $G$-action is the fiberwise adjoint action, leading to the
definition of connection given in \ref{recall_connection}.

Recall from \cref{sssec:CW_principal} that the \textit{adjoint bundle} to a principal $G$-bundle $P\to M$, denoted
$\g_P$, is the associated vector bundle to the adjoint representation $G\to\Aut(\g)$. The affine space of
connections on $P$ can be identified with $\cA_P=\Omega^1(M;\g_P)$, i.e.\ 1-forms on $M$ with values in the
adjoint bundle. Given two connections $\conn_0$, $\conn_1\in\cA_P$, the straight-line path
$\conn_t:I\rightarrow \cA_P$ determines a connection $\overline\conn$ on the $G$-bundle $P\times [0,1]$ over
$M\times[0,1]$. Let $\curvature{\overline\conn}$ be the curvature of $\overline\conn$.
\begin{defn}
The \textit{Chern--Simons form} associated to $\conn_0,\conn_1\in\cA_P$ and $f $ is given by
\index[terminology]{Chern--Simons form}
\index[notation]{CSf@$\CS_f(\conn)$}
\[\CS_{f }(\conn_1, \conn_0)=\int_{[0,1]}f(\curvature{\overline\conn})\in\Omega^{2k-1}(M)\period\]
\end{defn}
Let $\curvature{\conn_i}$ denote the curvature of $\conn_i$; then, by Stokes' theorem,
\begin{equation}
\label{stokes_CS}
\d\CS_{f }(\conn_1, \conn_0)= f(\curvature{\conn_1})-f(\curvature{\conn_0}).
\end{equation}
That is, the de Rham class $[f(\curvature{\conn_1})]$ is independent of the choice of connection, a fact that we first saw in
\cref{ChernWeilTheory}.
\begin{remark}
The path from $\conn_0$ to $\conn_1$ matters --- if we choose a different path, the Chern--Simons form will
differ by an exact term. This is beyond the scope of this chapter.
\end{remark}
Suppose instead we take the $G$-bundle $\pi^*P\rightarrow P$, which has a tautological section and hence a
tautological (flat) connection $\conn_0$. Then we can define a Chern--Simons form on $P$ (not on $M$!) for a single
connection $\conn$:
\begin{equation}
\label{total_space_CS}
\CS_{f }(\conn)=\CS_{f }(\pi^*\conn, \conn_0)\in \Omega^{2k-1}(P).
\end{equation}
Since $\conn_0$ is flat,~\eqref{stokes_CS} implies
\begin{equation}
\label{chern_simons_differential}
	\d\CS_{f }(\conn )= f(\pi^*\curvature{\conn})=\pi^*f(\curvature{\conn}).
\end{equation}
At this point, we want you to recall the differential cohomology hexagon from \cref{thm:SimonsSullivanunique}.
%
%Chern and Simons (\cite{cs}) showed that when $[f (A)]$ is an integral class, then there is a $u\in C^{2k-1}(M,\RR/\ZZ)$ such that $\pi^*u$ is the reduction of $\CS_{f }(A)\mod \ZZ$.
%
%-------------------------------------------------------------------%
%-------------------------------------------------------------------%
%  Relation to Cheeger–Simons differential characters               %
%-------------------------------------------------------------------%
%-------------------------------------------------------------------%
%
%\subsubsection{Relation to Cheeger--Simons differential characters}
%
%Now we will briefly explain the relation between Chern--Simons forms and Cheeger--Simons differential cohomology, following \cite[Chapter 2]{b}.
%
%
%Recall the definition of the Cheeger--Simons differential cohomology
%$$\Hhat^{k}(M;\ZZ)=\big\{ \chi\colon \Zsm_{k-1}\rightarrow \RR/\ZZ\big| \exists \alpha\in\curvature{\conn}^{k}(M)_{\ZZ},
%\chi(\partial c)=\int_c\alpha\mod\ZZ\big\}$$
%and the differential cohomology diagram. (c.f. Peter's introductory talk.)
\index[terminology]{differential cohomology hexagon}
\begin{equation}
\begin{gathered}
    \begin{tikzcd}[column sep={13ex,between origins}, row sep={11ex,between origins}]
      0 \arrow[dr] & & & & 0 \\
      & \H^{*-1}(M;\RR/\ZZ) \arrow[rr, "-\Bock"] \arrow[dr] & & \H^*(M;\ZZ) \arrow[dr] \arrow[ur] & \\
      \HdR^{*-1}(M) \arrow[ur] \arrow[dr] & & \Hhat^*(M;\ZZ) \arrow[ur, "\ch" description] \arrow[dr, "\curv" description] & & \HdR^*(M) \\
      & \displaystyle\frac{\Omega^{*-1}(M)}{\Omegacl^{*-1}(M)_{\ZZ}} \arrow[rr, "\d"'] \arrow[ur,
	  "\iota" description] & & \Omegacl^*(M)_{\ZZ} \arrow[ur] \arrow[dr] & \\
      0 \arrow[ur] & & & & 0 
    \end{tikzcd}
\end{gathered}
\end{equation}
The squares and triangles are commutative, and the diagonals are short exact sequences.
%
%The curvature map\index[terminology]{curvature map}
%\begin{equation*}
%  \curv \colon \Hhat^{k}(M;\ZZ)\rightarrow \curvature{\conn}^{k}(M)
%\end{equation*}
%sends $\chi$ to $\alpha$. 
%The characteristic class map\index[terminology]{characteristic class map}
%\begin{equation*}
%  \ch\colon\Hhat^{k}(M;\ZZ)\rightarrow \H^{k}(M;\ZZ)
%\end{equation*}
%is obtained by lifting $\chi$ to $\tilde{\chi} \colon \Zsm_{k-1}\rightarrow\RR$ and sending $\chi$ to the integral class defined by
%\begin{equation*}
%  c\mapsto -\tilde{\chi}(\partial c)+\int_c \alpha \,, \quad \text{ for } c\in \Csm_{k}(M;\ZZ) \period
%\end{equation*}
%We specifically need $\iota$, which is the descent of the map $\curvature{\conn}^{k-1}(M)\rightarrow\Hhat^{k}(M,\ZZ)$ defined
%by
%\begin{equation*}
%  \iota(\omega)(z) \colonequals \exp(2\pi i\int_z\omega)
%\end{equation*}
%for $z\in \Zsm_{k-1}$. Since the Chern--Simons form is a closed $(2k-1)$-form on $P$, we can ask about its image
%under $\iota$.
\begin{prop}
\label{iota_chern_simons}
Suppose $c^\ZZ\in\H^{2k}(\BG;\ZZ)$ is an integral lift of the Chern--Weil characteristic class of $f$ and
$\chat\in\Hhat^{2k}(\BnablaG; \ZZ)$ is the differential refinement of $c^\ZZ$ and $f$ guaranteed by
\cref{differential_CW_lift}. Then for any principal $G$-bundle $\pi\colon P\to M$ with connection $\conn$,
\begin{equation}
	\iota(\CS_f(\conn)) = \pi^*\chat(P, \conn)\in \Hhat^{2k-1}(P;\ZZ).
\end{equation}
\end{prop}
\begin{proof}
As usual, we can prove this for all principal bundles with connection at once by working universally on $(\EnablaG,
\conn)\to \BnablaG$. By construction, if $\curvature{\conn}$ denotes the curvature of $\conn$, $\curv(\chat) =
f(\curvature{\conn})\in\Omega^{2k}(\BnablaG)$, so by~\eqref{chern_simons_differential},
\begin{equation}
	\d\CS_f(\conn) = \pi^*\curv(\chat)\in\Hhat^{2k}(\EnablaG;\ZZ).
\end{equation}
The hexagon does all the hard work for us: $\H^{2k-1}(\EnablaG; \RR/\ZZ) = 0$, so the curvature map is injective.
Since $\d = {\curv}\circ\iota$, we can conclude.
\end{proof}
%
%Here we describe a way to lift the Chern--Weil homomorphism $\CW=\CW_{\theta}$ to
%$\Hhat^{2k-1}(M;\ZZ)$.\index[terminology]{Chern--Weil homomorphism!lift to differential cohomology}
%Set
%\begin{align*}
%  K^{2k}(G;\ZZ) &\colonequals \{(f ,u)\in\Sym^k(\g^*)^{\Ad}\times \H^{2k}(\BG;\ZZ)\big|\ \CW(f )=u_{\RR}\} \comma \\
%  \intertext{and}
%  R^{2k}(M;\ZZ) &\colonequals \{(\omega,v)\in\curvature{\conn}^{2k}(M)_{\ZZ}\times \H^{2k}(M,\ZZ)\big| [\omega]_{dR}=v_{\RR}\} \period
%\end{align*}
%Here $u_{\RR}$ denotes the image of $u$ in real cohomology. Then there is a unique natural map $\CWhat_{\theta}$ that makes the diagram commutes.
%\begin{equation*}
%  \begin{tikzcd}
%    & & \Hhat^{2k}(M;\ZZ)\arrow[d, "{(\curv,\ch)}"] \\
%    K^{2k}(G;\ZZ)\arrow[rr, "{(\CW_\theta,f^*)}"'] \arrow[urr, dotted, "\CWhat_\theta"] & & R^{2k}(M;\ZZ)
%  \end{tikzcd}
%\end{equation*}
%
%The universal bundle $\pi_{G} \colon \EG\rightarrow \BG$ comes with a universal connection $\conn$. 
%Let $f \colon M\rightarrow \BG$ be the classifying map with lift $F \colon (P,A)=(f^*(\EG),f^*(\conn))\rightarrow (\EG,\conn)$. 
%Recall that $\Sym(\g^*)^{\Ad}\rightarrow \HdR^*(\BG;\RR)$ is an isomorphism.
%Fix a pair $(\lambda,u)\in K^{2k}(G;\ZZ)$. 
%Then
%\begin{equation*}
%  \d \CS_{f }(\conn)=\pi^*_{G}f (\conn)=\curv(\pi^*_{G}\CWhat_{\conn}(f ,u))) \period
%\end{equation*} 
%Since $\EG$ is contractible, we conclude that
%\begin{equation*}
%  \iota(\CS_{f }(\conn))=\pi^*_{G}\CWhat_{\conn}(f ,u) \period 
%\end{equation*}
%
%Pulling back to the bundle $P$ and the connection $\theta$, we obtain the desired relation between Chern--Simons forms and Cheeger--Simons differential characters:
%\begin{equation*}
%  \iota(\CS_{f }(A))=F^*\iota(\CS_{f }(\conn))=F^*\pi^*_{G}\CWhat_{\conn}(f ,u)=\pi^* f^*\CWhat_{\conn}(f ,u)=\pi^* \CWhat_{A}(f ,u) \period
%\end{equation*}

Now suppose that $\pi\colon P\rightarrow M$ admits a section $\sigma\colon M\rightarrow P$. 
Then we further deduce that
\begin{equation}
	\chat(P, \conn) = \sigma^*\pi^*\chat(P, \conn) = \iota(\sigma^*\CS_f(\conn)),
%  \CWhat_{A}(f ,u)=\sigma^*(\pi^*\CWhat_{A}(f ,u))=\iota(\sigma^*\CS_{A}(f )) \period
\end{equation}
meaning that
\begin{equation}
\label{CS_secondary_CW}
	\int_M \chat(P, \conn) = \int_M \sigma^*(\CS_f(\conn))\in\RR/\ZZ.
\end{equation}
That is, as promised in \cref{secondary_chern_simons}, this Chern--Simons invariant is the secondary invariant
associated to $\chat$.
\index[terminology]{secondary invariant}
%
%Then (the generalization of) the \textit{Chern--Simons functional} is the evaluation at the fundamental class
%\begin{equation}
%  \CWhat_{A}(f ,u)([M])=\exp(2\pi i\int_M\sigma^*\CS_{A}(f )) \comma
%\end{equation}
%or its more familiar form 
%\begin{equation*}
%  \log(\CWhat_{A}(f ,u)([M]))=\int_M\sigma^*\CS_{A}(f ) \mod \ZZ \period
%\end{equation*}
This is conceptually nice, but how do we obtain computable topological invariants from this formula?

%-------------------------------------------------------------------%
%-------------------------------------------------------------------%
%  Chern–Simons invariants for 3-manifolds                          %
%-------------------------------------------------------------------%
%-------------------------------------------------------------------%

\subsection{Chern--Simons invariants for 3-manifolds}
\label{cs_invariants}
\index[terminology]{Chern--Simons invariant!of a $3$-manifold}

As an example, we examine the case where $P$ is a principal $\SU_2$-bundle over a path-connected $3$-manifold $M$,
$f (\conn)=\frac{1}{8\pi^2}\tr(\curvature{\conn}\wedge \curvature{\conn})$, and $c^\ZZ\in\H^4(\BSU_2;\ZZ)$ is the
second Chern class.\index[terminology]{Chern class}
%is the second Chern class in $\H^4(M;\RR)$. This is the classical Chern--Simons theory.
We mostly follow the exposition in \cite{KK}.

The quaternionic projective space $\HHP^\infty$ is a $\BSU_2$, so $\BSU_2$ is $3$-connected; hence every principal
$\SU_2$-bundle over a $3$-manifold is trivializable. Fix a trivialization; then there is a trivial (flat)
connection $\conn_0$, which allows us to identify $\cA_P$ with $\curvature{\conn}^1(M; \mathfrak{su}_2)$. Recall
that $\SU_2$ acts on $\cA_P$ by
\[g\cdot \conn =g\conn g^{-1}-dg\ g^{-1}\period\]
This action preserves flatness: if $\curvature{\conn}$ is the curvature of $\conn$, then the curvature of $g\cdot\conn$ is
$g\curvature{\conn} g^{-1}$. The \textit{gauge group} of $P$ is the group of bundle automorphisms of $P$ which cover the
identity on $M$.\index[terminology]{gauge group!of a principal bundle} In this case, the gauge group is
$\mathcal{G}\cong\Mapsm(M,\SU_2)$ and it acts on $P\cong M\times \SU_2$ by left multiplication, so the
$\mathcal{G}$-action preserves flat connections.

On the other hand, each flat connection $A$  gives rise to a \textit{holonomy representation} $\pi_1(M)\rightarrow
G$: parallel transport along a loop $\gamma$ at $m_0$ gives an automorphism of the fiber $\SU_2$ at $m_0$, which
depends only on the homotopy class $[\gamma]\in\pi_1(M,m_0)$.\index[terminology]{holonomy
representation} With a bit of work, one can recover the well-known fact that 
\begin{equation}
	\{\text{Flat connections on }P\}/\mathcal{G}\hookrightarrow R(M) \colonequals\Hom(\pi_1(M),\SU_2)/\text{conjugation}.
\end{equation}
Since $P$ is trivial, this injection becomes a bijection. In fact, this can be upgraded to a homeomorphism, with
the right-hand side the character variety of $M$.\index[terminology]{character variety}

Now look at the $3$-form
\begin{equation*}
  \CS_{f }(\conn)=\CS_{f }(\conn, \conn_0)=\int_{[0,1]}\frac{1}{8\pi^2}\tr(\curvature{\conn}\wedge \curvature{\conn}),
\end{equation*}
where as usual $\curvature{\conn}$ is the curvature of $\conn$.
Integrating over $M$ gives us the \textit{Chern--Simons functional} on $\cA_P$:\index[terminology]{Chern--Simons functional}
\index[notation]{cs@$\tilde{\cs}$}
\begin{equation}
\label{CS_fun}
  \tilde{\cs}(\conn)=\int_{M\times[0,1]}\frac{1}{8\pi^2}\tr(\curvature{\conn}\wedge \curvature{\conn}) = \frac{1}{8\pi^2}\int_M 
  \tr\paren{\conn\wedge \d\conn +\frac{2}{3}\conn\wedge [\conn\wedge \conn]}.
\end{equation}
This map is smooth and functorial in $P\rightarrow M$, and up to $\ZZ$ factors, it is independent of the
trivialization of $P$. Therefore $\tilde{\cs}$ descends to a functional
\begin{equation}
	\cs \colon R(M)\cong\cA_P/\mathcal{G}\rightarrow\RR/\ZZ.
\end{equation}
\index[notation]{cs@$\cs$}
The reason is that if $\sigma\in\mathcal{G}$, there is a straight-line path in $\cA_P$ from $\conn$ to
$\sigma\cdot\conn$, which we can interpret as a connection $\overline\conn$ on $[0,1]\times P\to[0,1]\times M$
with curvature $\curvature{\overline\conn}$. When we quotient by $\mathcal G$, we obtain a loop in $\cA_P/\mathcal{G}$, or
a connection on $P\times \Circ\to M\times \Circ$. Then
\begin{equation*}
  \cs(\sigma\cdot \conn)-\cs(\conn)=\int_{M\times \Circ}\frac{1}{8\pi^2}\tr(\curvature{\overline\conn}\wedge \curvature{\overline\conn}) =
  \int_{M\times\Circ} c_2(P\times\Circ),
\end{equation*}
which is an integer because $c_2$ is an integer-valued characteristic class.

The function $\cs\colon R(M)\to\RR/\ZZ$ is a homotopy invariant of $M$. In practice, it is relatively computable,
as we will see for lens spaces.

%-------------------------------------------------------------------%
%  Chern–Simons invariants of Lens spaces                           %
%-------------------------------------------------------------------%

\subsubsection{Chern--Simons invariants of lens spaces}
Let $p$ and $q$ be coprime positive integers and $\zeta$ be a primitive $p^{\mathrm{th}}$ root of unity. Then
$\ZZ/p$ acts on $\CC^2$ by
\begin{equation}
	(z_1, z_2)\mapsto (\zeta z_1, \zeta^q z_2).
\end{equation}
Restricting to the unit $\Sph{3}\subset\CC^2$, this is a free action, and the quotient is called a \textit{lens space}
and denoted $L(p, q)$ \cite[\S 20]{Tie08}.
\index[terminology]{lens space}
\index[notation]{$L(p, q)$}

Lens spaces form a nice collection of examples of $3$-manifolds, and given an invariant of $3$-manifolds, one can
test how powerful it is by checking how well it distinguishes inequivalent lens spaces. For example, $L(5, 1)$ and
$L(5, 2)$ have the same homology and fundamental group, but are not homotopy equivalent \cite{Ale19}; and there are
homotopy-equivalent lens spaces which are not homeomorphic \cites{Rei35}[\S 5]{Whi41}{Bro60}. The full
classifications of lens spaces up to homotopy equivalence and homeomorphism are known, due to work of
Whitehead \cite[\S 5]{Whi41}, resp.\ Reidemeister \cite{Rei35} and Brody \cite{Bro60}.

Let's test the power of Chern--Simons invariants on lens spaces.
\begin{thm}[{\cite[Theorem 5.1]{KK}}]\label{theorem:kk}
  The image of $\cs\colon R(L(p, q))\to\RR/\ZZ$ is the set\index[terminology]{lens space}
  \[\left\{-\frac{n^2r}{p} \,\bigg|\, n=0,1,\ldots,\left\lfloor \frac{p}{2}\right\rfloor \right\}\comma\]
  where $r$ is an integer satisfying $qr\equiv -1\bmod p$.
\end{thm}

You can think of $\mathrm{Im}(\cs)$ as the set of Chern--Simons invariants of a $3$-manifold.

\begin{remark}  
  Two lens spaces $L(p,q)$ and $L(p',q')$ have the same set of Chern--Simons invariants if and only if $p=p'$ and
  $q'q^{-1}\equiv a^2 \mod p$ for some $a\in\ZZ$, i.e., there is an orientation preserving homotopy equivalence
  between the two \cite[\S 5]{Whi41}. Hence Chern--Simons invariants detect the homotopy type of lens spaces.
\end{remark}

\begin{proof}[Proof sketch of \cref{theorem:kk}]
  The lens space $L(p,q)$ can be obtained by gluing the boundary of two solid tori $X$, $K$ together via an element
  \begin{equation*}
    \begin{pmatrix}
      p&q \\
      r&s 
  \end{pmatrix}\in \SL_2(\ZZ)
  \end{equation*}
  Let $x=\Circ\times\{1\}$ represent a generator of $\pi_1(X)$ and $y$ a meridian of $\partial X$. Let $\mu, \lambda$ be the corresponding generators of $\partial K$, so $\mu=px+qy,\ \lambda=rx+sy$. 

  Now we utilize some general results about $3$-manifolds with a single torus boundary in \cite{KK}. 
  Suppose we have a path $f _t$ in $\Hom(\pi_1(X),\SU_2)$ with 
  \begin{equation*}
    f _t(\mu) =
    \begin{pmatrix}
      e^{2\pi i\alpha(t)}& \\
      &e^{-2\pi i\alpha(t)} 
    \end{pmatrix} 
    \andeq
    f _t(\lambda) =
    \begin{pmatrix}
      e^{2\pi i\beta(t)}& \\
      &e^{-2\pi i\beta(t)} 
    \end{pmatrix},
  \end{equation*}
  where $\alpha, \beta \colon [0,1] \rightarrow\RR$. 
  The corresponding path of flat connections takes the form
  \begin{equation*}
  A_t=\begin{pmatrix}
     i\alpha(t)& \\
      &-i\alpha(t) 
  \end{pmatrix}\, \d x+ \begin{pmatrix}
      i\beta(t)& \\
      &-i\beta(t) 
  \end{pmatrix}\, \d y
  \end{equation*}
  near the torus boundary. 
  If $f _0 $ and $ f _1$ send $\mu$ to $1$,
  then \cite[Theorem 4.2]{KK}
  \begin{equation}
  	\cs(f _1)-\cs(f _0)=-2\int_0^1\beta\alpha'\, dt \bmod \ZZ.
  \end{equation}
  On the other hand, a holonomy representation on $X$ extends to one on the Dehn filling $M$ (in our case, the lens
  space itself) if and only if it sends $\mu$ to 1 (\textit{ibid.}, proof of Theorem 4.2).\index[terminology]{Dehn
  filling}

  Back to the sketch. We take $\gamma_t$ to be a path sending $x$ to $e^{2\pi i\theta}$ with $\theta\in[0, 1/2]$.
  (Every representation of $\pi_1(X)$ is conjugate to a representation in the image of the path.)
  Then $\gamma_{t_1}$ extends to a representation $f _t$ of $\pi_1(L(p,q))=\ZZ/p$ if and only if $pt_1\in\ZZ$, so
  we can obtain $\lfloor p/2 \rfloor+1$ conjugacy classes of representations of $\ZZ/p$, which correspond to $t_1 =
  n/p$ for $0\le n\le \lfloor p/2\rfloor$.

  On the other hand, $\alpha(t)=pt$ and $\beta(t)=rt$, so
  \begin{equation*}
    \cs(f _{t_1})=-2\int_0^{t_1}\beta\alpha'dt=-rpt_1^2\period
  \end{equation*}
  Plug in $t_1 = n/p$ and conclude.
\end{proof}

%-------------------------------------------------------------------%
%  Application: confiuguration spaces of Lens spaces                %
%-------------------------------------------------------------------%

\subsection{Application: configuration spaces of lens spaces}
\label{config_ssec}
To strengthen our Chern--Simons invariants, let's use them to study a related invariant of lens spaces:
the homotopy type of $F_2(L(p, q))$, the space of two-point subsets of
$L(p, q)$. Longoni--Salvatore \cite{LS05} showed that this distinguishes $L(7, 1)$ and $L(7, 2)$, which are
homotopy equivalent; the fact that the homotopy type of $F_2(X)$ knows more than the homotopy type of $X$ was a
surprising result. Differential cohomology enters the story with work of Evans-Lee--Saveliev \cite{deletedsquare}
using Chern--Simons invariants to provide a more comprehensive way to test whether the two-point configuration
spaces of two homotopy-equivalent lens spaces are homotopy equivalent.
%In this subsection, we follow Evans-Lee--Saveliev \cite{deletedsquare}, extending Chern--Simons invariants to
%two-point configuration spaces of lens spaces. Recall that $L(7, 1)$ and $L(7, 2)$ are homotopy equivalent but not
%homeomorphic, and Longoni--Salvatore \cite{LS05} showed that the homotopy type of their two-point configuration
%spaces can tell them apart. Now the question is: does the homotopy type of two-point configuration space
%distinguish lens spaces up to homeomorphism?

Choose a lens space $L=L(p,q)$ and a CW structure on it with a single $i$-cell $e_i$ for $0\le i\le 3$. Let $X =
L\times L$. The \textit{two-point configuration space}\index[terminology]{configuration space} of $L$ is
\begin{equation}
  X_0 \colonequals \Conf_2(L) \cong X \smallsetminus \Delta,
\end{equation}
where $\Delta\subset X$ is the diagonal, i.e.\ the subspace of elements $(x, x)$ with $x\in L$. Taking the
product CW structure on $X$, $X_0\subset L$ is a subcomplex, and the inclusion $X_0\hookrightarrow L\times
L$ induces an isomorphism of fundamental groups.

Using this CW structure, one can compute that
\begin{equation*}
  \H_3(X)\cong\ZZ\oplus\ZZ\oplus\ZZ/p \semicolon
\end{equation*}
the classes $[e_0\times e_3]=[e_0\times L]$ and $[e_3\times e_0]=[L\times e_0]$ generate the two $\ZZ$ summands and $[e_1\times
e_2 + e_2\times e_1]$ generates the $\ZZ/p$ summand.

\begin{lemma}
  There is a closed, oriented $3$-manifold $S$ with a map $f\colon M\to X$ such that $f_*[M] = [e_1\times e_2
  + e_2\times e_1]$.
\end{lemma}

\begin{proof}
  This is a special case of the \textit{Steenrod realization problem} asking when a given degree-$n$ homology class
  can be represented as a map from a closed, oriented $n$-manifold. This can be reformulated as a question about
  oriented bordism $\Omega_n^{\SO}(X)$, a generalized homology
  theory,\index[notation]{OmeganSO@$\Omega_n^{\SO}$}\index[terminology]{bordism} and the natural
  transformation $\Omega_n^{\SO}\to\H_n$ sending $(M, f\colon M\to X)\mapsto f_*[M]$. In this
  form, the question was answered negatively in general by Thom \cite[Théorème III.9]{ThomThesis}, but when $X$ is a
  manifold, $\Omega_3^{\SO}(X)\to\H_3(X)$ is surjective (\textit{ibid.}, Théorème III.3).
\end{proof}

\noindent Evans-Lee--Saveliev \cite[\S 3]{deletedsquare} give an explicit example of such a representative manifold $S$.

With this choice of generators of $\H_3(X)$, the inclusion $\H_3(X_0)=\ZZ\oplus\ZZ/p \hookrightarrow \H_3(X)$
sends a generator of the free summand to $(1,1,0)$ and a generator of the torsion summand to $[S]=(0,0,1)$.

Given a representation $\alpha \colon \pi_1(X)=\ZZ/p\times\ZZ/p\rightarrow \SU_2$ and a closed, oriented
$3$-manifold $M$ with a map $f\colon M\to X$, we get a representation $f^*\alpha$ of $\pi_1(M)$. Hence we can define an extension of the Chern--Simons invariants 
\begin{subequations}
\begin{equation}
  \cs_X \colon R(X_0)\rightarrow \Hom(\H_3(X_0), \RR/\ZZ)
\end{equation}
by
\begin{equation}
  \cs_X(\alpha) = \cs_M(f^*\alpha) = \frac{1}{8\pi^2}\int_M \tr\paren{\conn\wedge \d\conn +\frac{2}{3}\conn
  \wedge [\conn \wedge \conn]}.
\end{equation}
\end{subequations}
A priori this depends on our choice of $(M, f)$, but it is actually independent of this choice, and is also
functorial in $X$. Thus we obtain a homotopy invariant for each pair of conjugacy class of representation and third
homology class.

Now we compute. Fix an $\SU_2$-representation $\alpha$, which is conjugate to one sending the generators of
$\pi_1(X)$ to $e^{2\pi i k/p}$ and $e^{2\pi i \ell/p}$; we will call this representation $\alpha(k, \ell)$. Under
the two maps $L\rightrightarrows X$ realizing our two nontorsion generators of $\H_3(X)$, $\alpha(k, \ell)$ pulls
back to the representations sending a generator of $\pi_1(L)$ to $e^{2\pi i k/p}$ and $e^{2\pi i\ell/p}$. By
\cref{theorem:kk}, the Chern--Simons invariants of these representations are $-k^2r/p$ and $-\ell^2r/p$, where $r$
can be any integer such that $qr\equiv -1\bmod p$.

Evaluating the Chern--Simons invariant for $S\to X$ is harder. Evans-Lee--Saveliev show that the choice of $S$
they constructed is Seifert fibered\index[terminology]{Seifert fiber space} over $S^2$ (\textit{ibid.}, Lemma 4.4),
allowing them to use a theorem of Auckly \cite[\S 2]{auckly} computing the Chern--Simons invariants of such
$3$-manifolds. The upshot is that the Chern--Simons invariant of $f^*\alpha$ on $S$ is $2k\ell/p$. Pulling back
along $X_0\hookrightarrow X$, our nontorsion generator of $\H_3(X_0)$ has Chern--Simons invariant $r(k^2 +
\ell^2)/p$, and our torsion generator has invariant $2k\ell/p$.

Now suppose that $f:X_0\rightarrow X'_0$ is a homotopy equivalence, where $X_0=\Conf_2(L(p,q))$ and
$X'_0=\Conf_2(L(p,q'))$. Then the induced isomorphism $\ZZ/p\times \ZZ/p\to
\ZZ/p\times\ZZ/p$ on fundamental groups corresponds to a matrix
\begin{equation}
	f_1=\begin{pmatrix}
   a& c\\
    b& d 
\end{pmatrix}\in\GL_2(\ZZ/p).\end{equation}
The induced isomorphism on $\H_3=\ZZ\oplus\ZZ/p$ has the form $h_3=\begin{pmatrix}
   \epsilon& 0\\
    a & b
\end{pmatrix}$, where $\epsilon=\pm 1$ and $b\in (\ZZ/p)^{\times}$. Using naturality of Chern--Simons
invariants, we can deduce the following numerical constraints: 

\begin{prop}[{\cite[Proposition 5.2]{deletedsquare}}]
\label{f_h_restr}
If $f$ is a homotopy equivalence, then $\epsilon q'\equiv qa^2 \bmod p$ and $$f_1=\begin{pmatrix}
   a &0 \\
    0& \pm a 
\end{pmatrix}, \begin{pmatrix}
  0 & a\\
     \pm a &0 
\end{pmatrix};\ h_3=\begin{pmatrix}
   \epsilon& 0\\
    0 & \pm a^2 
\end{pmatrix}.$$ 
\end{prop}
Composing with the swap map $(x, y)\mapsto (y, x)$ if necessary, we can and do make $f$ diagonal, rather than
antidiagonal.

To learn more information about lens spaces, we have to combine \cref{f_h_restr} with other invariants. These
invariants are further away from differential cohomology, so we will be terser and point the reader towards
references with more information. Specifically, we will combine the Chern--Simons invariants results from above
with information about Massey products\index[terminology]{Massey product} in the cohomology of the universal cover
$\tilde X_0$ of $X_0$.
% info abour H*(\tilde X_0)
\begin{prop}[{\cite[Lemma 6.1]{deletedsquare}}]
\label{deleted_square_coh}
$\H^*(\tilde X_0)\cong \ZZ[a_1,\dotsc,a_{p-1}, b]/(a_i^2, b^2)$, where $\lvert a_i\rvert = 2$ and $\lvert b
\rvert = 3$.
\end{prop}

\begin{proof}[Proof sketch]
  The universal cover of $X$ is $\Sph{3}\times \Sph{3}$; therefore the universal cover of $X_0$ is a subspace of $\Sph{3}\times
  \Sph{3}$, specifically the complement of the orbit of the diagonal of $\Sph{3}\times \Sph{3}$ under the $\pi_1(X)$-action.
  Therefore there is a map $\pi\colon \tilde X_0\hookrightarrow \Sph{3}\times \Sph{3}\to \Sph{3}$ given by inclusion followed by
  projection onto the first factor; it is a surjective submersion, and the fiber is a $(p-1)$-punctured $\Sph{3}$. 
  Set up the Serre spectral sequence;\index[terminology]{Serre spectral sequence} there are only a few differentials not
  zeroed out by degree considerations, and they vanish because $\pi$ has a section. 
  Thus the spectral sequence
  collapses. 
  There are no nontrivial extension questions, so the cohomology ring of $\tilde X_0$ is the tensor
  product of
  \begin{equation*}
    \H^*(\Sph{3};\ZZ/2)\cong\ZZ/2[b]/(b^2) \andeq \H^*(\Sph{3}\setminus \{x_1,\dotsc,x_{p-1})\}\cong\ZZ/2[a_1,\dotsc,a_{p-1}]/(a_i^2) \period \qedhere
  \end{equation*}
\end{proof}

Let $a_0 = -a_1 - \dotsb -a_{p-1}$. Miller \cite[\S 2.1]{Mil11} calculates the $\pi_1(X_0)$-action on $\H^2(\tilde
X_0)$. Specifically, for $k,\ell\in\ZZ/p$, let $\tau_{k,\ell}$ denote the element corresponding to $(k,\ell)$ under
the identification $\pi_1(X_0)\cong\ZZ/p\times\ZZ/p$ above. Then,
\begin{equation}
	\tau_{k,\ell}\cdot a_i = a_{i+k-\ell}.
\end{equation}
This puts an additional constraint on a homotopy equivalence $f\colon X_0\to X_0'$: $f$ must intertwine the action
map $\pi_1(X_0)\to\Aut(\H^2(\tilde X_0))$. With $\alpha,\epsilon$ as above, this implies $f =
\alpha\cdot\mathrm{id}$ and that the following diagram commutes \cite[Proposition 6.3]{deletedsquare}:
\begin{equation}
\begin{gathered}
% https://q.uiver.app/?q=WzAsNCxbMCwwLCJcXEheMihcXHRpbGRlIFhfMCkiXSxbMSwwLCJcXEheMihcXHRpbGRlIFhfMCcpIl0sWzAsMSwiXFxIXjIoXFx0aWxkZSBYXzApIl0sWzEsMSwiXFxIXjIoXFx0aWxkZSBYXzAnKSJdLFswLDEsIlxcdGlsZGUgZl4qIl0sWzIsMywiXFx0aWxkZSBmXioiXSxbMSwzLCJcXHRhdV97ayxcXGVsbH0iXSxbMCwyLCJcXHRhdV97XFxhbHBoYSBrLCBcXGFscGhhXFxlbGx9IiwyXV0=
\begin{tikzcd}
	{\H^2(\tilde X_0')} & {\H^2(\tilde X_0)} \\
	{\H^2(\tilde X_0')} & {\H^2(\tilde X_0)}.
	\arrow["{\tilde f^*}", from=1-1, to=1-2]
	\arrow["{\tilde f^*}", from=2-1, to=2-2]
	\arrow["{\tau_{k,\ell}}", from=1-2, to=2-2]
	\arrow["{\tau_{\alpha k, \alpha\ell}}"', from=1-1, to=2-1]
\end{tikzcd}
\end{gathered}
\end{equation}
This provides an additional constraint on $f$.

% what is a Massey product?
Next we need information about Massey products in $\H^*(\tilde X_0; \ZZ/2)$. The Massey product is a secondary
cohomology operation; the corresponding primary operation is the cup product.\index[terminology]{Massey
product}\index[terminology]{secondary cohomology operation} As a quick review, a Massey product \cites[\S
2]{UM57}{Mas58} is defined for $x,y,z\in \H^*(X; A)$ when $A$ is a ring, $x\cupprod y = 0$, and $y\cupprod z = 0$: one
chooses cocycles $\overline x$, $\overline y$, and $\overline z$ representing $x$, $y$, and $z$ respectively, and
chooses cochains $A$ and $B$ such that $\delta A = \overline x\cupprod\overline y$ and $\delta B = \overline
y\cupprod\overline z$. The Massey product $\ang{x, y, z}$ is defined to be the set of cohomology classes
$[A\cupprod\overline z - \overline x\cupprod B]$ for all possible choices of $A$ and $B$. Massey products are
functorial, which follows directly from their definition.

Assume $p$ is odd and $0 < q < p/2$. It follows from \cref{deleted_square_coh} that there are identifications of
abelian groups
\begin{equation}
\label{zeta_p}
	\FF_2(\zeta_p) \colonequals \FF_2[t]/(1 + t + \cdots + t^{p-1}) \isomorphism \H^m(\tilde X_0;\ZZ/2),\ m = 2,5;
\end{equation}
for $m = 2$, this map sends $t^k\mapsto a_k\bmod 2$, and for $m = 5$, $t^k\mapsto a_kb\bmod 2$.\footnote{We chose
the notation $\FF_2(\zeta_p)$ because this is the cyclotomic field associated to a primitive $p^{\mathrm{th}}$ root
of unity $\zeta_p$ over $\FF_2$.\index[terminology]{cyclotomic field}} If $x,y,z\in
\H^2(\tilde X_0;\ZZ/2)$ satisfy $xy = yz = 0$ (so that their Massey product is defined), then $\ang{x,y,z}\subset
\H^5(\tilde X_0;\ZZ/2)$, so we may describe these Massey products as (possibly multivalued) maps
\begin{equation}
\label{zeta_Massey}
	\ang{\text{--}, \text{--}, \text{--}}\colon \FF_2(\zeta_p)\times\FF_2(\zeta_p)\times\FF_2(\zeta_p)\to
	\FF_2(\zeta_p).
\end{equation}
Miller \cite[Theorem 3.33]{Mil11} calculates these Massey products. For example, $t^n\cdot \ang{t^k,t^\ell, t^j} =
\ang{t^{k+n}, t^{\ell+n}, t^{j+n}}$ and $\ang{t^k, t^\ell, t^j} = \ang{t^j, t^\ell, t^k}$. These two relations
allow us to inductively reduce to the case when at least one of $j$, $k$, or $\ell$ is $0$; the description of the
Massey products in that case is a little more complicated, and can be found in \cite[Theorem 7.1]{deletedsquare}.

This leads us to our last obstruction. The two different maps $\tilde f^*\colon \H^m(\tilde
X_0';\ZZ_2)\to\H^m(\tilde X_0;\ZZ/2)$, $m = 2,5$, become the same map $\tilde f^*\colon
\FF_2(\zeta_p)\to\FF_2(\zeta_p)$ under the identification~\eqref{zeta_p}. Therefore we obtain the constraint that
this $\tilde f^*$ must intertwine the Massey product map~\eqref{zeta_Massey}.

Our three constraints (coming from Chern--Simons invariants, cohomology of $\tilde X_0$, and Massey products) each
boil down to numerical constraints on $p$ and $q$, and these are amenable to computer calculation. This is how
Evans-Lee--Saveliev showed that these constraints can detect some homotopy-equivalent but not homeomorphic lens
spaces that Longoni--Salvatore's techniques miss. These pairs include $L(11, 2)$ and $L(11, 3)$; $L(13, 2)$ and
$L(13, 5)$; and $L(17, 3)$ and $L(17, 5)$.
