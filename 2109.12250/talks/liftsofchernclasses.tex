%!TEX root = ../diffcoh.tex

\section{Lifts of Chern Classes}\label{LiftsofChernClasses}
\textit{Talk by Mike Hopkins}\\
\textit{Notes by Araminta Amabel}
 %maybe should put in reminder that have lifts with connection all the way to Z(2n). use virasoro as motivation for getting them half way up? or add in explanation why they should only go half way up
%define B_\bullett G if haven't yet. say that can take \H^*(\BbulletG;\ZZ(n))$ tthings

%-------------------------------------------------------------------%
%-------------------------------------------------------------------%
%  Introduction                                                     %
%-------------------------------------------------------------------%
%-------------------------------------------------------------------%

\subsection{Introduction}

Let $\ZZ(n)$ be the Deligne complex  \index[notation]{Z(n)@$\ZZ(n)$}
\begin{equation*}
	\ZZ\rta\Omega^0\rta\cdots\rta \Omega^{n-1}
\end{equation*}
We'll also let $\ZZ(\infty)$ denote the untruncated complex,
\begin{equation*}
	\ZZ\rta\Omega^0\rta\cdots
\end{equation*}
Similarly, we define $\RR(n)$ where $n=1,\dots,\infty$ to be the complex  \index[notation]{R(n)@$\RR(n)$}
\begin{equation*}
	\RR\rta\Omega^0\rta\cdots\rta \Omega^{n-1}
\end{equation*}
and $\ZZ_\CC(n)$ to be the complex  \index[notation]{ZC(n)@$\ZZ_\CC(n)$}
\begin{equation*}
	\ZZ\rta\Omega^0_\CC\rta\Omega^1_\CC\rta\cdots\rta\Omega^{n-1}_\CC \period
\end{equation*}
One can also think of $\ZZ(n)$ as the homotopy pullback
\begin{equation*}
	\begin{tikzcd}
		\ZZ(n)\arrow[r]\arrow[d] & \ZZ\arrow[d]\\
		\Sigma^{-n}\Omegacl^n\arrow[r] & \RR \period
	\end{tikzcd}
\end{equation*}
One take away is that there are a lot more characteristic classes in differential cohomology than you would expect.

%-------------------------------------------------------------------%
%  Virasoro Group Motivation                                        %
%-------------------------------------------------------------------%

\subsubsection{Virasoro Group Motivation}

The Virasoro group is a certain central extension of $\Diffplus(\Circ)$ by $\Uup_1$,  \index[terminology]{Virasoro!group}
\begin{equation*}
	\Uup_1\rta\Vir\rta\Diffplus(\Circ) \period
\end{equation*}
Let $\Gamma=\Diffplus(\Circ)$ be the group of orientation preserving diffeomorphisms of $\Circ$. Central extensions of $\Gamma$ are classified by elements of $\H^3(\BGamma;\ZZ(1))$; i.e., by homotopy classes of maps $\BGamma\rta \K(\ZZ(1), 3)$. We have a fiber sequence
\begin{equation*}
	\K(\ZZ(1),2)\rta \EGamma\rta \BGamma\rta \K(\ZZ(1), 3) \period
\end{equation*}
Consider the fibration $\EGamma\times_\Gamma \Circ\rta \BGamma$ with fiber $\Circ$. Integration along the fibers gives a map
\begin{equation*}
	\H^4(\EGamma\times_\Gamma \Circ;\ZZ(2))\rta \H^3(\BGamma;\ZZ(1)) \period 
\end{equation*}
There is a map $\EGamma\times_\Gamma \Circ\rta \BSL(\RR)$. Thus given a class $\tilde{p}_1\in \H^4(\BSL(\RR);\ZZ(2))$, we can pull it back to get a class in $\H^4(\EGamma\times_\Gamma \Circ;\ZZ(2))$. 
Integrating along the fiber produces a class in $\H^3(\BGamma;\ZZ(1))$. 
Thus, classes in $\H^4(\BSL(\RR);\ZZ(2))$ produce central extensions of $\Diffplus(\Circ)$. 

%-------------------------------------------------------------------%
%  Hopes                                                            %
%-------------------------------------------------------------------%

\subsubsection{Hopes}

Let $G$ be a Lie group. 
Recall the sheaf of groupoids $\BbulletG$ from \Cref{ex:BunG,ntn:BbulletGBnablaG}.
\begin{enumerate}[(1)]
	\item If $V\rta X$ is a real vector bundle, we want lifted Pontryagin classes $\tilde{p}_n(V)\in \H^{4n}(X;\ZZ(2n))$. 

	\subitem To obtain such lifts, it suffices to construct $\tilde{p}_n\in \H^{4n}(\BbulletGL_n\RR;\ZZ(2n))$ such that $\tilde{p}_n$ maps to $p_n$ under the map
	\begin{equation*}
		\H^{4n}(\BbulletGL_n(\CC);\ZZ(2n))\rta \H^{4n}(\BGL_n(\CC);\ZZ)
	\end{equation*}

	\item If $W\rta X$ is a complex vector bundle, we want (off-diagonal) Chern classes $\tilde{c}_m(W)\in \H^{2m}(X;\ZZ_\CC(m))$. 

	\subitem To obtain such lifts, it suffices to construct $\tilde{c}_n\in \H^{2n}(\BbulletGL_n(\CC);\ZZ_\CC(n))$ such that $\tilde{c}_n$ maps to $c_n$ under the map
	\begin{equation*}
		\H^{2n}(\BbulletGL_n(\CC);\ZZ_\CC(n))\rta \H^{2n}(\BGL_n(\CC);\ZZ) \period
	\end{equation*}

	\item \textit{Cartan formula:} Given a short exact sequence of vector bundles
	\begin{equation*}
		0\rta V\rta W\rta U\rta 0
	\end{equation*}
	an expression of the differential characteristic classes of $W$ in terms of the differential characteristic classes for $U$ and $V$. Every short exact sequence of vector bundles is split, but this splitting might not be smooth. 
	Thus it's possible that such a formula exists for split short exact sequences, $V\oplus U$. %right? what lol

	\item \textit{Projective bundle formula:} More generally, higher characteristic classes being determined by those for line bundles.
\end{enumerate}

%-------------------------------------------------------------------%
%  Statement of Results                                             %
%-------------------------------------------------------------------%

\subsubsection{Statement of Results}

The following are things Hopkins has worked out and attributes to ideas found in papers of Bott, \cites{BottPaper,BottNotes}

\begin{theorem}
	There is a pullback square
	\begin{equation*}
		\begin{tikzcd}
			\H^{2n}(\BbulletGL_m(\CC); \ZZ_\CC(n))\arrow[rr]\arrow[d] & & \H^{2n}(\BU_m;\ZZ)\arrow[d] \\
			\H^n(\BU_m\times \BU_m;\CC)\arrow[rr, "\textup{diagonal}^*"'] & & \H^{2n}(\BU_m;\CC) \period
		\end{tikzcd}
	\end{equation*}
\end{theorem}

\noindent This is \Cref{cor-BGLC} below.

So if we wanted to lift the first Chern class $c_1$, we could take 
\begin{equation*}
	\frac{1}{2}(c_1\otimes 1+ 1\otimes c_1)\in \H^2(\BU_1\times \BU_1;\CC) \period
\end{equation*}
But, could also add to this any terms that are in the kernel of the diagonal map. So there are many possible off-diagonal lifts of $c_1$ to something with $\ZZ_\CC(1)$ coefficients. 

Using the $e^{2\pi i}$ induced isomorphism $\K(\ZZ_\CC(1);2)\similarrightarrow \BGL_1(\CC)$ produces the lift of $c_1$ corresponding to $\frac{1}{2}(c_1\otimes 1+1\otimes c_1)$. %explain this

\begin{remark}
This also works for products of copies of $\GL_n(\CC)$. For example, let 
\begin{equation*}
	G=\GL_n(\CC)\times\cdots\times \GL_n(\CC) \period
\end{equation*}
Then we have a pullback 
\begin{equation*}
	\begin{tikzcd}
		\H^{2n}(\BbulletG; \ZZ_\CC(n))\arrow[rr]\arrow[d] & & \H^{2n}(\BG;\ZZ)\arrow[d] \\
		\H^n(\BG\times \BG;\CC)\arrow[rr, "\textup{diagonal}^*"'] & & \H^{2n}(\BG;\CC) \period
	\end{tikzcd}
\end{equation*}
\end{remark}

\noindent Let $P_{a|b}\subset \GL_{a+b}(\CC)$ be the subset of matrices of the form 
\begin{equation*}
	\begin{pmatrix}
		A & B \\
		0 & C
	\end{pmatrix} \comma
\end{equation*}
where $A$ is an $(a\times a)$-matrix and $B$ is a $(b\times b)$-matrix. %this is some paragbolic?
Note that there is a map
\begin{equation*}
	\GL_a(\CC)\times \GL_b(\CC)\rta P_{a|b}
\end{equation*}
sending $(A,B)$ to the block matrix with $A$ and $B$ on the diagonal.

\begin{conjecture}
	The induced map
	\begin{equation*}
		\H^{2n}(P_{a|b};\ZZ_\CC(n))\rta \H^{2n}( \GL_a(\CC)\times \GL_b(\CC);\ZZ_\CC(n))
	\end{equation*}
	is an isomorphism.
\end{conjecture} 

\begin{proof}[Proof Outline]
	Completing \Cref{exercise-Pab} below, one should find that 
	$\H^{2n}(P_{a\vert b};\ZZ_\CC(n))$ fits into a pullback diagram
	\begin{equation*}
		\begin{tikzcd}
			\H^{2n}(B_\bullet P_{a\vert b}; \ZZ_\CC(n))\arrow[rr]\arrow[d] & & \H^{2n}(B(P_{a\vert b}\cap U_{a+b});\ZZ)\arrow[d]\\
			\H^n(B(P_{a\vert b}\cap U_{a+b})\times B(P_{a\vert b}\cap U_{a+b});\CC)\arrow[rr, "\textup{diagonal}^*"'] & & \H^{2n}(B(P_{a\vert b}\cap U_{a+b};\CC)
		\end{tikzcd}
	\end{equation*}
	and $\H^{2n}(\GL_a(\CC)\times \GL_b(\CC));\ZZ_\CC(n))$ fits into a pullback diagram
	\begin{equation*}
		\begin{tikzcd}
			\H^{2n}(B_\bullet (\GL_a(\CC)\times \GL_b(\CC)); \ZZ_\CC(n))\arrow[rr]\arrow[d] & & \H^{2n}(\BU_{a+b};\ZZ)\arrow[d]\\
			\H^n(\BU_{a+b}\times \BU_{a+b};\CC)\arrow[rr, "\textup{diagonal}^*"'] & & \H^{2n}(\BU_{a+b};\CC)
		\end{tikzcd}
	\end{equation*}
	Since every short exact sequence of vector bundles splits, the inclusion $\BU_{a+b}\hookrightarrow BP_{a\vert b}$ is a homotopy equivalence. Thus so is the inclusion $\BU_{a+b}\hookrightarrow B(P_{a\vert b}\cap U_{a+b})$. 
	Hence the lower left corners of the above two pullback diagrams are isomorphic.
\end{proof}

\noindent Thus if we have a Cartan-like formula for split short exact sequences, we can get a Cartan-like formula for any short exact sequence.

The following is an example of \Cref{cor-BGZ} below.
\begin{theorem}\label{thm-BGZ}
	There is a pullback square
	\begin{equation*}
		\begin{tikzcd}
			\H^{2n}(\BbulletGL_m(\RR);\ZZ(n))\arrow[r]\arrow[d] & \H^{2n}(\BO_m;\ZZ)\arrow[d]\\
			\H^{2n}(\BGL_m(\CC);\RR)\arrow[r] & \H^{n}(\BO_m;\RR) \period
		\end{tikzcd}
	\end{equation*}
\end{theorem}

\begin{ex}
	Take $n=1$ and choose $m$ large. 
	The first Pontryagin class $p_1$ lives in $\H^4(\BO_m;\ZZ)$. 
	By the theorem, off-diagonal differential lifts of $p_1$ are given by a choice of class in 
	\begin{equation*}
		\H^4(\BGL_m(\CC);\RR)\simeq \RR\oplus \RR
	\end{equation*}
	that agrees with the image of $p_1$ in
	\begin{equation*}
		\H^4(\BO_m;\RR)\simeq\RR \period
	\end{equation*}
	Pictorially, there is a pullback diagram
	\begin{equation*}
		\begin{tikzcd}
			\H^4(\BbulletGL_m(\RR);\ZZ(2)) \arrow[r, "f"] \arrow[d] & \H^4(\BO_m;\ZZ)\arrow[d] \\
			\RR\oplus \RR\arrow[r] & \RR \period
		\end{tikzcd}
	\end{equation*}
	Since this is pullback diagram, the kernel of $f$ is the same as the kernel of the bottom horizontal map. That is, $\ker(f)=\RR$. Thus there is a 1-parameter family of differential lifts of $p_1$. 

	One way to choose such a lift $\tilde{p}_1$ is to ask for $\tilde{p}_1$ to be primitive; i.e.,
	\begin{equation*}
		\tilde{p}_1(V\oplus U)=\tilde{p}_1(V)+\tilde{p}_1(U)
	\end{equation*}
	Up to a scalar $\lambda$, there is only one choice of primitive element of $\H^4(\BGL_m;\RR)$ that agrees with $p_1$ in $\H^4(\BO_m;\RR)$. That class is 
	\begin{equation*}
		\frac{1}{2}(\lambda c_1^2-2c_2) \period
	\end{equation*}
\end{ex}

%-------------------------------------------------------------------%
%-------------------------------------------------------------------%
%  Computations                                                     %
%-------------------------------------------------------------------%
%-------------------------------------------------------------------%

\subsection{Computations}

Suppose that $G$ is a finite-dimensional Lie group. 
We are interested in computing
\begin{equation*}
	\H^{2n}(\BbulletG;\ZZ(n)) \period
\end{equation*}
We start with $\H^{2n}(\BbulletG;\RR(n))$.

\begin{prop}
	For all $k$ one has $\H^k(\BbulletG;\RR(\infty))=0$.
\end{prop}

\begin{proof}
	By definition, $\RR(\infty)$ is the complex
	\begin{equation*}
		\RR\rta\Omega^0\rta\cdots
	\end{equation*}
	which is acyclic by the Poincaré Lemma.
\end{proof}

\begin{cor}\label{zero below 2n}
	For $k<2n$ one has $ \H^k(\BbulletG;\RR(n))=0 $.
\end{cor}

\begin{proof}
	We will show that for $k<2n$ the map
	\begin{equation*}
		\H^k(\BbulletG;\RR(n+1))\rta \H^k(\BbulletG;\RR(n))
	\end{equation*}
	is surjective. For this we have the long exact sequence associated to the short exact sequence
	\begin{equation*}
		0\rta\Sigma^{-(n+1)}\Omega^n\rta\RR(n+1)\rta\RR(n)\rta 0 \period
	\end{equation*}
	It gives us an exact sequence
	\begin{equation*}
		\H^k(\BbulletG;\RR(n+1))\rta \H^k(\BbulletG;\RR(n))\rta \H^{k-n}(\BbulletG;\Omega^n) \period
	\end{equation*}
	By Bott's theorem \cite[Theorem 1]{BottPaper}, we have %ref thm in these notes
	\begin{equation*}
		\H^{k-n}(\BbulletG;\Omega^n)=\Hcont^{k-2n}(G;\Sym^n(\gdual)) \comma
	\end{equation*}
	where the right-hand side is the continuous cohomology group, which is zero since $k-2n<0$.
\end{proof}

\begin{cor}\label{R to Omega}
	The map
	\begin{equation*}
		\H^{2n}(\BbulletG;\RR(n))\rta \H^n(\BbulletG;\Omega^n)
	\end{equation*}
	is an isomorphism.
\end{cor}

\begin{proof}
	This map is part of the long exact sequence
	\begin{equation*}
		\cdots\rta \H^{2n}(\BbulletG;\RR(n+1))\rta \H^{2n}(\BbulletG;\RR(n))\rta \H^{n}(\BbulletG;\Omega^n)\rta \H^{2n+1}(\BbulletG;\RR(n+1))\rta\cdots
	\end{equation*}
	and the two end terms are zero by \Cref{zero below 2n}.
\end{proof}

\begin{cor}\label{cor-CWoff}
	We have an isomorphism
	\begin{equation*}
		\H^{2n}(\BbulletG;\RR(n))\isomorphic\Sym^n(\gdual)^G \period
	\end{equation*}
\end{cor}

\begin{proof}
	By \Cref{R to Omega}, we have an isomorphism
	\begin{equation*}
		\H^{2n}(\BbulletG;\RR(n))\similarrightarrow \H^n(\BbulletG;\Omega^n) \period
	\end{equation*}
	Bott's theorem gives an isomorphism
	\begin{equation*}
		\H^{n}(\BbulletG;\Omega^n)\isomorphic \Hcont^{n-n}(G;\Sym^n(\gdual)) \period
	\end{equation*}
	One has
	\begin{equation*}
		\Hcont^{0}(G;\Sym^n(\gdual))\isomorphic (\Sym^n(\gdual))^G \period \qedhere
	\end{equation*}
\end{proof}

\begin{cor}\label{cor-BGZ}
\label{2n_to_n}
	For every $n$ there is a pullback square
	\begin{equation*}
		\begin{tikzcd}
			\H^{2n}(\BbulletG;\ZZ(n))\arrow[r]\arrow[d] & \H^{2n}(\BG;\ZZ)\arrow[d]\\
			\Sym^n(\gdual)^G\arrow[r] & \H^{2n}(\BG;\RR) \period
		\end{tikzcd}
	\end{equation*}
\end{cor}

\begin{proof}
	For this consider the pullback square
	\begin{equation*}
		\begin{tikzcd}
			\ZZ(n)\arrow[r]\arrow[d] & \ZZ\arrow[d]\\
			\RR(n)\arrow[r] & \RR
		\end{tikzcd}
	\end{equation*}
	The associated Mayer--Vietoris sequence shows that the kernel of the map from the upper left corner of
	\begin{equation*}
		\begin{tikzcd}
			\H^{2n}(\BbulletG;\ZZ(n))\arrow[r]\arrow[d] & \H^{2n}(\BG;\ZZ)\arrow[d]\\
			\H^{2n}(\BbulletG;\RR(n))\arrow[r] & \H^{2n}(\BG;\RR)
		\end{tikzcd}
	\end{equation*}
	to the pullback is $\H^{2n-1}(\BbulletG;\RR)$, which is zero by Chern--Weil. %cite earlier too
\end{proof}

Tensoring with $\CC$ gives:

\begin{cor}
	For every $n$ there is a pullback square
	\begin{equation*}
		\begin{tikzcd}
			\H^{2n}(\BbulletG;\ZZ_\CC(n))\arrow[r]\arrow[d] & \H^{2n}(\BG;\ZZ)\arrow[d]\\
			\Sym^n_\CC(\gdual\otimes\CC)^{G_\CC}\arrow[r] & \H^{2n}(\BG;\CC)
		\end{tikzcd}
	\end{equation*}
	where $G_\CC$ is the complexification of the Lie group $G$.
\end{cor}
%this needs a complex version of Bott's Theorem
%\begin{equation*}\H^p(\BbulletG;\Omega^q_\CC)\simeq \Hcont^{p-q}(\BG;\Sym_\CC^n(\gdual\otimes\CC)^{G_\CC}\end{equation*}
%I think that just needs
%\begin{equation*}\CC\otimes\Sym^n(\gdual)\simeq\Sym_\CC^n(\gdual\otimes\CC)^{G_\CC}\end{equation*}

\begin{remark}
	When $G$ is connected, the map
	\begin{equation*}
		\Sym^n(\gdual)^G\rta\Sym^n(\gdual)^\g
	\end{equation*}
	is an isomorphism. 
	Otherwise, there is a residual action of $\pi_0G$ and one has an isomorphism
	\begin{equation*}
		\Sym^n(\gdual)^G\rta\left(\Sym^n(\gdual)^\g\right)^{\pi_0G} \period
	\end{equation*}
\end{remark}

We now turn to evaluating these groups.

\begin{ex}
	Let's take $G = \GL_n(\CC)$. 
	Then since $ \GL_n(\CC) $ is connected, we have
	\begin{equation*}
		\Sym^n(\gdual)^G=\Sym^n(\gdual)^\g
	\end{equation*}
	which depends only on $\g$. 
	Since $\g$ is complex, we have
	\begin{equation*}
		\g\otimes\CC\isomorphic \g\oplus\g
	\end{equation*}
	and so 
	\begin{equation*}
		\CC\otimes\Sym^n(\gdual)^G = \Sym^n_\CC(\gdual\oplus\gdual)^{\g\oplus\g}
	\end{equation*}
	Now $\g$ is also the complexification of the Lie algebra $\ufrak_n$ of the unitary group $\Uup_n$. Thus the above is isomorphic to
	\begin{equation*}
		\CC\otimes(\Sym^n(\ufrak_n\oplus\ufrak_n))^{\Uup_n\times \Uup_n}
	\end{equation*}
	which, by Chern--Weil, is 
	\begin{equation*}
		\H^{2n}(\BU_n\times \BU_n;\CC) \period
	\end{equation*}
\end{ex}

\begin{cor}\label{cor-BGLC}
	There is a pullback diagram
	\begin{equation*}
		\begin{tikzcd}
			\H^{2n}(\BbulletGL_m(\CC);\ZZ_\CC(n))\arrow[r]\arrow[d] & \H^{2n}(\BU_m;\ZZ)\arrow[d] \\
			\H^{2n}(\BU_m\times \BU_m;\CC)\arrow[r] & \H^{2n}(\BU_m;\CC)
		\end{tikzcd}
	\end{equation*}
\end{cor}

\begin{ex}
	Let's now take the case $G = \GL_n(\RR)$. The main thing now is to compute
	\begin{equation*}
		\Sym^n(\gdual)^G=\left(\Sym^n(\gdual)^\g\right)^{\ZZ/2}
	\end{equation*}
	Using Weyl's unitary trick again, we can complexify and recognize
	\begin{equation*}
		\g_\CC\isomorphic (\ufrak_n)\otimes\CC
	\end{equation*}
	and we find by Chern--Weil that
	\begin{equation*}
		\left(\Sym^n(\gdual)^\g\right)_\CC\isomorphic \H^{2n}(\BU_m;\CC) \period 
	\end{equation*}
	The action of $\Gal(\CC/\RR)$ is complex conjugation on both $\CC$ and on $U_m$ so
	\begin{equation*}
		\H^{2n}(\BU_m;\CC)^{\Gal(\CC/\RR)} \isomorphic \begin{cases}
		\H^{2n}(\BU_m;i\RR), & n\text{ odd} \\
		\H^{2n}(\BU_m;\RR), & n \text{ even} \period
		\end{cases}
	\end{equation*}
	In this case, the action of $\pi_0\GL_m$ is trivial.
\end{ex}

\begin{remark}
	Maybe the easiest way to be convinced of the action of complex conjugation and of $\pi_0\GL_m$ is to remember the formula for the Chern classes in terms of $\Sym^\bullet(\gdual)$. 
	For $x\in\gl_n(\CC)$, the total Chern class
	\begin{equation*}
		1+c_1t+\cdots+ c_nt^n
	\end{equation*}
	is given by the homogeneous terms in the characteristic polynomial
	\begin{equation*}
		\det\frac{t}{2\pi i}\begin{pmatrix}
		e_{11} & \cdots & e_{1n}\\
		\vdots & \ddots & \vdots\\
		e_{n1} & \cdots & e_{nn}
		\end{pmatrix}-1
	\end{equation*}
	where $e_{ij}\in\gl_n\CC^*$ is the function associating to a matrix its $(ij)$ entry. 
	If we apply this to a matrix with real entries, we see that the $k$th chern class lies in $\frac{1}{(2\pi i)^k}\RR$ and that it is invariant under conjugation by any matrix.
\end{remark}

\begin{exercise}\label{exercise-Pab}
	Let $P_{a,b}\subset \GL_{a+b}(\CC)$ be the subgroup which sends vectors whose last $b$ coordinates are zero to vectors whose last $b$ coordinates are zero, as above. One may compute $
		\Sym^\bullet(\pfrak^\vee_{a,b})^{P_{a,b}}$
	by first computing $
		\Sym^\bullet(\pfrak^\vee_{a,b})^{G_{a,b}}$ 
	and appealing to the unitary trick. 
This is the relevant computation for working out a Cartan formula for an exact sequence which does not necessarily split.
\end{exercise}
