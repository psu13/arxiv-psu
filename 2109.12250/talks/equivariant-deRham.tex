%!TEX root = ../diffcoh.tex

\section{Equivariant de Rham Cohomology}\label{EquivariantdeRhamCohomology}
\textit{by Greg Parker}

%-------------------------------------------------------------------%
%-------------------------------------------------------------------%
%  Motivation                                                       %
%-------------------------------------------------------------------%
%-------------------------------------------------------------------%

\subsection{Motivation}
Let $G$ be a Lie group and $M$ be a smooth manifold with a $G$ action. We want a cohomology theory that takes into account the $G$-action. If the action is free, then we can take
\[\HG^\bullet(M)\colonequals \H^\bullet(M/G)\period\]
If the action is not free, we take the homotopy quotient $\EG\times_G M$ and set the \emph{equivariant cohomology of $M$} to be 
\[\HG^\bullet(M)\colonequals \H^\bullet(\EG\times_G M)\period\]
Here $\EG\rta \BG$ is the universal bundle, so that $\EG$ is a contractible space with a free $G$-action. 
\begin{quest}
How should one define equivariant cohomology using differential forms?
\end{quest}

\noindent To answer this question, we will roughly follow \cite[Chapter 1-4]{MR1689252}. 
The reader is encouraged to read \cite{MR1689252} for more details and applications. 

As motivation, again consider a free action. That is, take $P\rta X$ to be a principal $G$-bundle. 
We want to distinguish forms in $\Omegabullet(P)$ that pullback from $X=P/G$. 
Let $\g $ be the Lie algebra of $G$. 
%The linearized action gives $X_\xi\in\Gamma(P,TP)$ for $\xi\in\g$, which together span the vertical subbundle $\ker(\pi_*:TP\rta TM)$. 

For $\alpha\in\Omegabullet(P)$, we can locally write
\[\alpha=\sum_I\alpha_I\dup x_{i_1}\wedge\cdots\wedge\dup x_{i_N}\period\]
The form $\alpha$ is pulled back from $M$ if, for all $i$, 

(i) the form $\dup x_i$ is vertical: $i_\xi\alpha=0$ for all $\xi\in\g$, and 

(ii) $\alpha$ does not depend on vertical coordinates: $i_\xi\dup \alpha=0$ for all $\xi\in\g$. 

\noindent Forms satisfying these two conditions are called \emph{basic}. Let $\Omegabullet(P)_{\basic}$ denote the subcomplex of basic forms. Then, we have 
\[
	\H^\bullet(\Omega(P)_{\basic})\isomorphic\HdR^\bullet(X)\isomorphic\HdR^\bullet(P/G)=\HG^\bullet(P)\period
\]

%-------------------------------------------------------------------%
%-------------------------------------------------------------------%
%  G*-Algebras                                                      %
%-------------------------------------------------------------------%
%-------------------------------------------------------------------%

\subsection{\texorpdfstring{$G^*$}{G*}-Algebras}

Given an element $\xi\in \g$, there are multiple maps on $\Omegabullet(M)$:

\begin{itemize}

\item a degree $-1$ map by contraction, $\xi\mapsto i_\xi$ and 

\item a degree $0$ map by Lie derivative, $\xi\mapsto L_\xi$.

\end{itemize}

We can package these actions of $\g $, together with the differential $d$, on $\Omegabullet(M)$ as a representation of a certain Lie superalgebra $\gtilde$. 
Take 
\[\gtilde\colonequals \g_{-1}\oplus\g_0\oplus\RR\]
where, for each element $\xi\in\g$, we have corresponding elements of $\g_{-1}$ and $\g_0$ that we denote by their action on $\Omegabullet(M)$. That is, by $i_\xi$ and $L_\xi$, respectively. 
The generator of $\RR$ is denoted $d$. 
The bracket of the Lie superalgebra $\gtilde$ is defined by
\begin{align*}
	[i_\xi,i_\eta] &= 0\\
	[L_\xi,i_\eta] &= i_{[\xi,\eta]}\\
	[L_\xi,L_\eta] &= L_{[\xi,\eta]}\\
	[d,i_\xi] &= L_\xi\\
	[d,L_\xi] &= 0\\
	[d,d] &= 2d^2 = 0
\end{align*}
for all $\xi,\eta\in\g$.

The following is \cite[Definition 2.3.1]{MR1689252}. 
\begin{definition}
	A \emph{$G^*$-algebra} is a graded algebra $A$ with an action $G\rta\Aut(A)$ of $G$ and an action $\gtilde\rta\End(A)$ of $\gtilde$, so that 
	\begin{enumerate}[(1)]
		\item $\frac{d}{\dup t}\vert_{t=0}\exp(t_\xi)=L_\xi$,
		\item $gL_\xi g^{-1}=L_{\Ad_g\xi}$ and $gi_\xi g^{-1}=i_{\Ad_g\xi}$, and
		\item $gd=dg$.
	\end{enumerate}
\end{definition}

Note that the tensor product of two $G^*$-algebras is again a $G^*$-algebra. %prove?

\begin{ex}
	The complex $\Omegabullet(M)$ is a $G^*$-algebra with multiplication by the wedge product.
\end{ex}

\noindent Considering a $G^*$-algebra $A$ with its differential from the action of $d\in\gtilde$, 
we define ${\H^\bullet(A)\colonequals \H_\bullet(A,d)}$. 

\begin{definition}
Let $A$ be a $G^*$-algebra. 
A \emph{basic form} in $A$ is an element $\alpha\in A$ so that 
\[i_\xi\alpha=L_\xi=0\]
for all $\xi\in\g $. 
\end{definition}

We will need to add an assumption on our $G^*$-algebra, referred to as \emph{Condition C} in \cite[\S 2.3.4]{MR1689252}. 
Condition C will ensure the existence of a certain $G$-invariant subspace that acts like the vertical subbundle (\Cref{subsec-principal}) in the locally free case, see \cite[Definition 2.3.3]{MR1689252}. 

\begin{definition}
	Let $\xi_1,\dots,\xi_k$ be a basis for $\g $. 
	A $G^*$-algebra $A$ is \emph{satisfies Condition C} if there exists elements $\theta^1,\dots,\theta^k\in A$ of degree 1 so that for all $i,j=1,\dots, k$,
	\[\iota_{\xi_i}\theta^{j}=\delta_{ij}\]
	and the subspace spanned by $\{\theta^i\}$ is invariant under $G$.
\end{definition}

In particular, if the action of $G$ on $M$ is free, then $\Omegabullet(M)$ is a $G^*$-algebra satisfying Condition C.


We say a $G^*$-algebra $A$ is \emph{acyclic} if the chain complex $(A,d)$ is.

\begin{definition}\label{def-Conditionc}
Let $M$ be a manifold with $G$ action and 
let $E$ be a $G^*$-algebra that is acyclic and satisfies condition C. 
Define the equivariant de Rham cohomology by
\[
	\HGdR^\bullet(M)\colonequals \H^\bullet((\Omega(M)\otimes E)_{\basic}) \period
\]
\end{definition}

The following is \cite[Theorem 2.5.1]{MR1689252}. In particular, by \cite[Prop. 2.5.4]{MR1689252}, such $G^*$-algebras $E$ as in \Cref{def-Conditionc} exist in the context we care about. 
\begin{thm}[(Equivariant de Rham)]
	There is an isomorphism 
	\[
		\HGdR^*(M)\isomorphic\HG^*(M) \period
	\]
\end{thm}

We discuss the idea of the proof here. For a full proof, see \cite[\S 2.5]{MR1689252}. 
\begin{proof}[Proof Idea]
Approximate $\EG$ with a sequence of finite-dimensional manifolds $E_k$ and take \[E = \lim_k \Omega(E_k)\period\]
By the free case,
\[
	\H^*(M\times E_k/G)=\H^*(\Omega(M\times E_k)_{\basic})
\] for $* \ll k$. 
To finish, one shows that
\[
	\Omega(M\times E_k)_{\basic}=\Omega(M)\otimes\Omega(E_k)_{\basic}
\]
in the limit.
\end{proof}

\begin{rmk}
By \cite[\S 4.4]{MR1689252}, the definition of $\HGdR^*$ is independent of $E$ satisfying the assumptions (acyclic and Condition C).
\end{rmk}

%-------------------------------------------------------------------%
%-------------------------------------------------------------------%
%  Cartan Model                                                     %
%-------------------------------------------------------------------%
%-------------------------------------------------------------------%

\subsection{Cartan Model}

Now we can look for a specific $E$ that gives a nice algebraic structure, so it might be more computable.

For a vector space $V$, the \emph{Koszul algebra} is $(\exterior^\bullet(V)\otimes \Sym^\bullet(V),d)$ where $d(\alpha\otimes 1)=1\otimes \alpha$ and $d(1\otimes\alpha)=0$ extended as a derivation.  \index[terminology]{Koszul complex}The \emph{Weil Algebra} is the Koszul algebra of $\gdual$, 
\[W=\exterior^\bullet(\gdual)\otimes\Sym^\bullet(\gdual)  \index[terminology]{Weil Algebra}\]
as a $G^*$ algebra: For a basis (as an algebra) $\theta^i, z^j$ we have
\begin{align*}
	i_a\theta_b &= \delta_{ab}\\
	L_a\theta_b &= -[\theta_a,\theta_b]=-c_{ab}^k\theta_k\\
	L_az_b &= -c_{ab}^k z_k \\
	i_az_b &= -c_{ab}^k\theta_k \period
\end{align*}

The following is \cite[Theorem 3.2.1]{MR1689252}.
\begin{prop}
	The Weil algebra $W$ is acyclic and satisfies condition C.
\end{prop}

\begin{proof}[Proof of Acyclicity]
	Define a chain homotopy $ Q $ from $\id{}$ to $0$ by setting
	\begin{equation*}
		Q(\alpha\otimes 1) \colonequals 0 \andeq Q(1\otimes \alpha) \colonequals \alpha\otimes 1 \period \qedhere
	\end{equation*}
\end{proof}

\noindent In particular, we can use $W$ as a model for $E$.

The $G^*$-algebra $W$ has a rather nice subalgebra of basic forms. 
By \cite[Theorem 3.2.2]{MR1689252}, the basic cohomology ring of the Weil algebra $W$ is $\Sym^\bullet(\gdual)^G$. 
Thus 
\[\H_*((W\otimes \Omega^*(M))_{\basic},\dup \vert_{\basic})\]
 calculates $\HG^*(M)$. 
One can use this description of the equivariant de Rham cohomology of the Weil algebra to obtain a description, 
called the \emph{Cartan model}, 
of the equivariant de Rham cohomology of any $G^*$-algebra. 

\begin{thm}[(Cartan model)]  \index[terminology]{Cartan model}\label{thm-Cartan}
	For a $G^*$-algebra $A$, there is an isomorphism (the Mathai--Quillen isomorphism) 
	\begin{align*}
		\varphi\colon (W\otimes A)_{\basic} &\isomorphism (\Sym^\bullet(\gdual)\otimes A)^G \\ 
		\shortintertext{sending}
		\dup \vert_{\basic} &\mapsto\dup_G=1\otimes \dup_A-\mu^a\otimes i_a \period
	\end{align*}
\end{thm}

\noindent In particular, $\HG^*(M)$ can be computed from $(\Sym^\bullet(\gdual)\otimes\Omega^*(M))^G,\dup_G)$.
