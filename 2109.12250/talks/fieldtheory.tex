%!TEX root = ../diffcoh.tex

\section{Charge quantization}
\textit{Talk by Dan Freed}\\
\textit{Notes by Arun Debray}
%These notes were typeset by Arun Debray, following a lecture by Dan Freed. Any mistakes or typos are probably due to me (i.e. Arun).
\label{field_theory}

There are a few different applications of differential cohomology to quantum physics; today, we'll focus on charge
quantization, using Maxwell theory as an example. First, in \S\ref{classical_Maxwell}, we introduce classical
Maxwell theory, formulated in the language of differential forms. Then, in \S\ref{quantum_Maxwell}, we pass to the
quantum theory. This imposes integrality conditions on differential forms, leading to the appearance of
differential cohomology. This lecture is based on~\cite[Part 3]{Fre02a}.

The history of the use of differential cohomology to implement charge quantization is closely tied to the
development of the theory of differential cohomology itself. Alvarez~\cite{Alv85} was the first to use
differential cohomology in this context, though he does not use the words ``differential
cohomology.''\footnote{Alvarez also uses differential cohomology to characterize quantized topological terms. This
is a related but different application of differential cohomology to physics, and is more closely related to the
discussion of invertible field theories in the next chapter. See Deligne--Freed~\cite[Chapter 6]{DF99} for a
mathematical exposition of topological terms and their relationship to differential cohomology.}
Gawędzki~\cite{Gaw88} then explicitly brings in differential cohomology in the form of Deligne cohomology.

The original motivation to consider generalized differential cohomology came from charge quantization in string
theory: work of Minasian--Moore~\cite{MM97}, Sen~\cite{Sen98}, and Witten~\cite{Wit98} argued that D-brane charges
and Ramond--Ramond field strengths are valued in $\Kup$-theory,\footnote{See~\cite{FW99, MW00, DMW02} for some
related work.} leading to a search for a $\Kup$-theoretic analogue of differential
cohomology.\index[terminology]{D-brane charge}\index[terminology]{Ramond--Ramond field} Freed--Hopkins~\cite{FH00}
first provided a definition of differential $\Kup$-theory for this purpose, and Freed~\cite{Fre00} considers more
general differential generalized cohomology theories. Hopkins--Singer~\cite{HopkinsSinger}, who comprehensively
studied differential generalized cohomology theories, write that they originally began their project to investigate
string-theoretic phenomena.\footnote{Similarly, twisted differential cohomology was first motivated by the
appearance of examples of twisted differential $\Kup$-theory in string theory~\cites[\S 5.3]{Wit98}{BV00}{Fre00},
and has since become an object of study in its own right~\cites[\S
4.1.2]{Urs}{GS18}{BN19}{GS19c}{GS19a}{GS19b}{FSS20d}.}

%-------------------------------------------------------------------%
%-------------------------------------------------------------------%
%  Classical Maxwell theory                                         %
%-------------------------------------------------------------------%
%-------------------------------------------------------------------%

\subsection{Classical Maxwell theory}
\label{classical_Maxwell}
\index[terminology]{Maxwell theory!classical}

Let $(N, g_N)$ be a Riemannian $3$-manifold without boundary and $M=\RR\times N$. Let $t$ be the $\RR$
coordinate, so we give $M$ the Lorentz metric
\begin{equation}
	g_M= \d t^2 - g_N.
\end{equation}
Choose differential forms $E\in\Omega^1(N)$ and $B\in\Omega^2(N)$, respectively the electric and magnetic fields;
also choose the \emph{charge density} $\rho_E\in\Omegac^3(N)$, and the \emph{current}
\index[terminology]{charge density}
\index[terminology]{current}
$J_E\in\Omegac^2(N)$.% 
\footnote{Here $\Omegac^k(X)$ denotes the space of compactly supported $k$-forms on $X$.} 
If $\star_N$ denotes the Hodge star on $N$, then \emph{Maxwell's equations}, as you might see them on a t-shirt, are
\index[terminology]{Maxwell's equations}
\begin{align*}
	\d B &= 0 \\
	 \frac{\partial B}{\partial t} + \d E &= 0\\
	\d  {\star_N} E &= \rho_E \\
	 {\star_N} \frac{\partial E}{\partial t} - \d {\star_N} B &= J_E.
\end{align*}
Writing $F= B - \d t\wedge E\in\Omega^2(M)$ and $j_E= \rho_E + \d t\wedge J_E\in\Omega^3(M)$, we
obtain a more concise form of Maxwell's equations:
\begin{equation}
	\d F = 0,\qquad\qquad \d {\star_M} F = j_E.
\end{equation}
Now we include topology. We just saw that $j_E$ is exact, so it cannot define an interesting de Rham cohomology
class, but $F$ is closed, so may be interesting. Define the \emph{charge} at time $t$ to be the de Rham class
\begin{equation}
	Q_E = [j_E|_{\{t\}\times N}]\in \Hc^3(N;\RR).
\end{equation}
This is in the kernel of the map $\Hc^3(N;\RR)\rta \H^3(N;\RR)$; hence, on a compact manifold, $Q_E = 0$.

Let $W$ be the worldline of a charged particle with electric charge $q_E\in\RR$. Then $j_E = q_E\cdot\delta_W$,
where $\delta_W$ is the ``current sitting at $W$.'' We have two ways of making sense of this.
\begin{itemize}
	\item First, we could take $\delta_W$ to be a current in the de Rham sense, akin to a differential form but
	built with distributions instead of smooth functions. Amusingly, this is a current in both the Maxwell and de
	Rham senses. This is a typical example of a current in electromagnetism.
	\item Alternatively, we could take $\delta_W$ to be an honest $3$-form Poincaré dual to $W$. In this case we
	can choose $\delta_W$ to be supported in an arbitrary neighborhood of $W$.
\end{itemize}
One more ingredient in Maxwell theory, though not strictly necessary, is an action principle. This follows the
Lagrangian formulation of physics: we aim to find a variational problem whose solutions are the Maxwell equations.
We add an assumption from classical physics: that $[F] = 0$ in $\HdR^2(M)$; this means there are no magnetic
monopoles.
\index[terminology]{action principle}

This assumption also implies $F = \d A$ for some $1$-form $A$ called the \emph{electromagnetic potential}.
\index[terminology]{electromagnetic potential} This is not unique, but its class in $\Omega^1(M)/\Omegacl^1(M)$
(i.e.\ up to closed $1$-forms) is unique.  Then, the \emph{classical action} \index[terminology]{classical action}
of Maxwell theory is
\index[terminology]{electromagnetic potential}
\index[terminology]{Lagrangian!for Maxwell theory}
\begin{equation}
\label{classical_action}
	S = \int_M -\frac{1}{2} \d A\wedge {\star} \d A + A\wedge j_E.
\end{equation}
Since $M$ is noncompact, this could be infinite, but we're just interested in its first variation anyways, which is
well-behaved.
\begin{exercise}
Show that the Euler--Lagrange equation for \eqref{classical_action} is $\d{\star} F = j_E$. (We already assumed $\d
F = 0$, the other half of Maxwell's equations.)
\index[terminology]{Euler--Lagrange equation}
\end{exercise}
One caveat: defining the action requires $A$ to be in $\Omega^1(M)$, not
$\Omega^1(M)/\Omegacl^1(M)$. This ends up not a problem; adding a closed form to $A$ does not change
the Euler--Lagrange equation.

%-------------------------------------------------------------------%
%-------------------------------------------------------------------%
%  Quantum Maxwell theory                                           %
%-------------------------------------------------------------------%
%-------------------------------------------------------------------%

\subsection{Quantum Maxwell theory}
\label{quantum_Maxwell}
\index[terminology]{Maxwell theory!quantum}

In the quantum theory, we allow magnetic monopoles. Dirac \cite{Dir31} argues that this forces electric and
magnetic charges to be \emph{quantized}, i.e.\ taking values in a discrete subgroup of $\RR$. This is how
differential cohomology enters the picture.

So assume $N = \RR^3$ with the usual Euclidean metric, and introduce a magnetic monopole of charge $q_B\in\RR$ at
the origin. Then we have a \emph{magnetic current} \index[terminology]{magnetic current} $j_B = q_B\cdot\delta_0$.
The condition that $\d F = 0$ is modified to
\begin{equation}
\label{dFmon}
	\d F = q_B\cdot\delta_0.
\end{equation}

The input to the path integral is the exponentiated action $\exp(iS/\hbar)$ (where $S$ is
as in~\eqref{classical_action}. However, this is not quite consistent with~\eqref{dFmon} --- there is a problem at
the origin. On $\RR\times(\RR^3\smallsetminus 0)$, we can write $F = \d A$, and therefore realize $F$ as the curvature of
a connection $A$ on a principal $\RR/q_B\ZZ$-bundle $P$. The characteristic class of $P$ is
\begin{equation}
	[P]\in \H^2(\RR\times (\RR^3\smallsetminus 0); q_B\ZZ)\cong \H^2(S^2;q_B\ZZ) = q_B\ZZ,
\end{equation}
and $[P]$ is a generator of this abelian group.

The space of fields in the quantum theory is the groupoid of principal $\RR/q_B\ZZ$-bundles with connection. Now we
can revisit the action~\eqref{classical_action} --- it doesn't have to make sense as is (e.g.\ $A$ isn't exactly a
$1$-form), but we do want $\exp(iS/\hbar)$ to make sense.

Let's work on a general $4$-manifold $X$. To avoid causality issues, let's make $X$ a Riemannian manifold, rather
than a Lorentz one. Assume $j_E$ is Poincaré dual to some loop $\gamma\subset X$. If there is a $q_E$ charge moving
along this loop, then
\begin{equation}
	\int_M A\wedge j_E = \oint_\gamma q_EA = q_E\Hol_\gamma(A).
\end{equation}
Now $\Hol_\gamma(A)\in\RR/q_B\ZZ$, so the quantity
\begin{equation}
	\exp\left(\frac{i}{\hbar} q_E\Hol_\gamma(A)\right)
\end{equation}
is well-defined if and only if
\begin{equation}
	\frac{1}{\hbar} q_Eq_B\in 2\pi\ZZ.
\end{equation}
This is Dirac's quantization condition. Thus integrality enters a story told with differential forms; this is
already suggestive of differential cohomology!
\index[terminology]{Dirac quantization}

To say it more explicitly, the space of quantum fields is the stack $\Bunnabla_{\RR/q_B\ZZ}(X)$; the set of
isomorphism classes of objects is $\Hhat^2(X; q_B\ZZ)$. The curvature map lands in those $2$-forms with periods in
$q_B\ZZ$, giving us a short exact sequence we've seen before:
\index[terminology]{curvature map}
\begin{equation*}
	\begin{tikzcd}
		0 \arrow[r] & \H^1(X;\RR/q_B\ZZ) \arrow[r] & \Hhat^2(X;q_B\ZZ) \arrow[r, "\curv"] & \Omegacl^2(X)_{q_B\ZZ} \arrow[r] & 0 \period
	\end{tikzcd}
\end{equation*}
The classical fields $\Omega^1(X)/\Omegacl^1(X)$ sit as a subspace in $\Hhat^2(X; q_B\ZZ)$; the cokernel
is $\H^2(X;q_B\ZZ)$ modulo torsion, indicating the new information in the quantum theory.

Another interesting upshot is that since the kernel of the curvature map corresponds to the flat connections, i.e.\
those on which $F$ is boring, the electric flux really lives in $\Hhat^2(X; q_B\ZZ)$. This is new. The flat
connections are new, too --- even if you don't usually get to observe them, they manifest in the physics, e.g.\
through the Aharonov--Bohm effect. And all of this is still ``semiclassical,'' i.e.\ about the input to the path
integral, before we try to evaluate said path integral.
\index{Aharonov--Bohm effect}

\begin{remark}
	One important clarification: $F$ is not a differential cohomology class; it's the curvature of an actual bundle
	with connection, not an equivalence class. So really we need a cochain model: bundles and connections glue, but
	equivalence classes don't. 
	Cheeger--Simons characters aren't built in this way, so for physics applications one must
	do something different.
\end{remark}

Now we revisit the electric charge, a closed $3$-form.
%\footnote{You might be wondering where the compact support
%condition went. To work in Euclidean signature, rather than Minkowski signature, we must \emph{Wick-rotate} the
%theory, a nontrivial procedure which ultimately removes the requirement for compact
%supports.\index[terminology]{Wick rotation}}
Because
$(i/\hbar)j_Ej_B\in 2\pi\ZZ$, we'd like to impose that $[j_E]\in \HdR^3(X)$ is also in the image of the map
$\H^3(X;q_E\ZZ)\rta \H^3(X;\RR)$, i.e.\ that we're in the homotopy pullback, which is $\Hhat^3(X;q_E\ZZ)$. Again,
though, we want a local object in the end, not just its isomorphism class.

We can also rewrite one term in the exponentiated action in terms of differential cohomology, as
\begin{equation}
	\exp\left(\frac i\hbar \int_X \Fhat\cdot\jhat\right).
\end{equation}
Here $\Fhat$ and $\jhat$ are the differential cohomology refinements of $F$ and $j_E$, respectively.  The
product $\cdot$ is the cup product from \cref{sec:Delignecup}, which is a map
\begin{equation}\label{differential_action}
	\Hhat^2(X; q_B\ZZ)\otimes\Hhat^3(X;q_E\ZZ)\longrightarrow \Hhat^5(X; q_Eq_B\ZZ).
\end{equation}
Since $X$ is a $4$-manifold, the integration map has degree $-4$, so is of the form
\begin{equation}
	\int_X\colon \Hhat^5(X; q_Eq_B\ZZ)\longrightarrow \Hhat^1(\pt; q_Eq_B\ZZ)\cong \RR/q_Eq_B\ZZ.
\end{equation}

\begin{exercise}
	Show that if $\Fhat$ is topologically trivial, meaning that it comes from a connection on a trivial vector
	bundle, or equivalently that its image under the characteristic class map vanishes, then
	$\Fhat\cdot\jhat$ is also topologically trivial.
\end{exercise}

\begin{remark}
	There are many variations of this story in field theory and string theory, generally for abelian gauge fields.
	For example, $F$ might have some other degree, or even be inhomogeneous. Dirac charge quantization still
	applies, and will refine $F$ to an appropriate differential cohomology group.\index[terminology]{string theory}

	More recently, people realized that this story sometimes yields generalized differential cohomology theories.
	Understanding which cohomology theory one obtains is a bit of an art --- physics tells you some constraints,
	but not an algorithm. For example, this happens in superstring theory: the Ramond--Ramond
	field\index[terminology]{Ramond--Ramond field} is realized in differential $\Kup$-theory \cite{FH00,
	MW00}\index[terminology]{differential K-theory@differential $K$-theory}, and the $B$-field in a differential
	refinement of (a truncation of) $\ko$ \cite{DFM11a, DFM11b}. These and other refinements of Dirac
	quantization to generalized differential cohomology are also studied in \cite{BM06a, BM06b, DFM07, Fre08,
	Sat10, SZ10, Sat11, SSS12, KM13, KV14, DMDR14, FSS15, FR16, GS19, Sat19, FRRB20}. The choice of generalized
	cohomology theory is not always an exact science: for example, there are different proposals for the $C$-field
	in M-theory. Witten \cite[\S 2.3]{Wit97} argues that the $C$-field should be quantized in $w_1$-twisted
	degree-$4$ ordinary differential cohomology, which passes consistency checks for various possible
	anomalies \cites[\S 4]{Wit97}[\S 4]{Wit16}{FH21a}; there is also the ambitious ``hypothesis H'' of
	Fiorenza--Sati--Schreiber \cite{Sat18, FSS19, FSS20a} proposing that the $C$-field in M-theory is quantized
	using a differential refinement of $\mathrm{Im}(J)$-twisted stable cohomotopy instead. Work of Fiorenza, Sati,
	Schreiber, and their collaborators \cite{FSS19, SS19, FSS20a, FSS20b, FSS20c, GS20, SS20b, SS20a, SS20c, BSS21,
	SS21} and Roberts \cite{Rob20} recovers as consequences of hypothesis H several things physicists predicted to
	be true about M-theory.\index[terminology]{hypothesis
	H}\index[terminology]{C-field@$C$-field}\index[terminology]{M-theory}
\end{remark}

If we consider Maxwell theory with both electric and a magnetic currents, the theory has an
``anomaly,''\index[terminology]{anomaly} meaning that some quantity that we'd like to obtain as a complex number is
actually an element of a complex line that's not trivialized (and in some cases cannot be trivialized canonically
for all manifolds of a given dimension). Differential cohomology also provides a perspective on the anomaly. The
expression $\Fhat\cdot\jhat_E$ in \eqref{differential_action} is valid if there's electric current but
not magnetic current; if $\jhat_B\neq 0$, then $F$ isn't closed, hence isn't the curvature of a line bundle.
But $\jhat_B$ is also quantized, hence represents a differential cohomology class, and we can ask for
$\Fhat$ to trivialize $\jhat_B$. Now the action is
\begin{equation}
	\exp\left(\frac i\hbar \int_X \Fhat\cdot \jhat_E\jhat_B\right).
\end{equation}
Since $\Fhat\cdot\jhat_E\jhat_B\in\Hhat^6$, integrating brings us to $\Hhat^2(\pt;
q_Eq_B\ZZ)$, yielding the complex line which signals the anomaly. More on this anomaly can be found in
Freed--Moore--Segal \cite{FMS07a, FMS07b}.
% redundant in view of invertible field theories section
%\begin{ex}
%In the last few minutes, we'll discuss a different example of differential cohomology in physics. Suppose $M$ is an
%oriented Riemannian $3$-manifold and $P\rta M$ is a principal $SU_2$-bundle with connection $\Theta$. The second
%Chern class admits a differential refinement $\ccech_2(\Theta)\in \Hhat^4(M)$, and
%\begin{equation}
%	\int_M \ccech_2(\Theta) \in \Hhat^1(\pt)\cong\RR/\ZZ.
%\end{equation}
%Hence this is the sort of thing you can add to an action. It's an example of a \emph{Chern--Simons term} in 3d QFT.
%\index[terminology]{Chern--Simons term}
%The de Rham class underlying $\ccech_2$ is sometimes called the level of the theory, and the fact that it must
%refine to differential cohomology is saying the level is quantized. In general, quantization of coupling constants
%%provides another instance of differential cohomology in physics.
%
%Chern--Simons terms are usually described without differential cohomology, using the Chern--Simons form associated to
%$\Theta$, but writing that term on $M$, rather than on $P$, requires a choice of a section of $P$, and we don't
%always have that.
%\end{ex}
