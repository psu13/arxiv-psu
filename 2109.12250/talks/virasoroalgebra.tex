%!TEX root = ../diffcoh.tex

\section{Virasoro Algebra}\label{VirasoroAlgebra}
\textit{by Arun Debray}

The contents of this section can be summarized as follows:
\begin{itemize}
	\item The Virasoro group is a particular central extension of $\Diffplus(\Circ)$ by $\TT$.
	\item A theorem of Segal \cite[Corollary 7.5]{Seg81} proves that
	\begin{equation}
		\Cent_{\TT}(\Diffplus(\Circ)) \isomorphism \Cent_{\TT}(\PSL_2(\RR))\times\Cent_\RR(\wittRR) \comma
	\end{equation}
	where $\wittRR = \mathrm{Lie}(\Diffplus(\Circ))$ is the Witt algebra. The map is: restrict the central extension to
	$\PSL_2(\RR)\subset\Diffplus(\Circ)$ for the first component, and differentiate for the second component.
%%	\item The (isomorphism classes of) central extensions created from differential lifts of $p_1$ are expected to
%	be trivial when restricted to $\PSL_2(\RR)$. Proving this might be a good warm-up to studying $\Diffplus(\Circ)$.
%	\item The Virasoro extension is also trivial when restricted to $\PSL_2(\RR)$. Therefore, to identify which
%	off-diagonal differential lift $\tilde{p}_1$ of $p_1$ induces the Virasoro central extension, it suffices to look at the
%	induced central extension of Lie algebras of $\wittRR$. There is an $\RR$ worth of differential lifts of $p_1$, and
%	$\Cent_\RR(\wittRR)\cong\RR$.
\end{itemize}

%-------------------------------------------------------------------%
%-------------------------------------------------------------------%
%  Review of central extensions                                     %
%-------------------------------------------------------------------%
%-------------------------------------------------------------------%

\subsection{Review of central extensions}

\begin{defn} \index[terminology]{Central Extension}
	Let $G$ be a group and $A$ be an abelian group. A \emph{central extension} of $G$ by $A$ is a short exact sequence
	of groups
	\begin{equation}\label{centralext}
		\begin{tikzcd}[sep=1.5em]
			1\arrow[r] & A\arrow[r] & \widetilde G\arrow[r] & G\arrow[r] & 1,
		\end{tikzcd}
	\end{equation}
	such that $A\subset Z(\widetilde G)$. An equivalence of central extensions is a map of short exact sequences which
	is the identity on $G$ and on $A$. These form an abelian group we denote $\Cent_A(G)$.  \index[notation]{Cent@$\Cent$}
\end{defn}

When $G$ and $A$ have additional structure, we will ask that central extensions respect that structure: for
example, when both are Lie groups (possibly infinite-dimensional), we want~\eqref{centralext} to be a short exact
sequence of Lie groups. 

For discrete $G$ and $A$, central extensions are classified by $\H^2(G;A)$. Explicitly, given a cocycle $b\colon
G\times G\to A$, we build the central extension by setting $\widetilde G = G\times A$ as sets, with the twisted
multiplication
\begin{equation}
	(g_1, a_1)\cdot_b (g_2, a_2) \colonequals (g_1g_2, a_1 + a_2 + b(g_1, g_2)).
\end{equation}
Associativity follows from the cocycle condition; if two cocycles are related by a coboundary, their induced
central extensions are equivalent.

Generalizing this to Lie groups is not straightforward --- you can't just use smooth cochains unless $A$ is a
topological vector space. We are interested in central extensions by $\TT$, so we'll have to be craftier. The fix is
due to Segal \cite{Seg70}, and was later rediscovered by Brylinski \cite{Bry00}, following Blanc \cite{Bla85}. We
rephrase it in language familiar to this seminar.

Let $A$ be an abelian Lie group. Throughout today's talk, $\underline A$ denotes the simplicial sheaf on $\msf{Man}$
whose value on a test manifold $M$ is the space of \emph{smooth } maps $M\to A$.\footnote{By contrast, the
simplicial sheaf just denoted ``$A$'' treats $A$ as having the discrete topology. This is a little bit
counterintuitive but is standard notation.}
\begin{thm}[{(Segal \cite{Seg70}, Brylinski \cite{Bry00})}]
Let $G$ and $A$ be abelian Lie groups. Then, equivalences classes of central extensions in which $\widetilde G\to
G$ is a principal $A$-bundle are classified by $\H^2(\BbulletG; \underline A)$.
\end{thm}
The idea of the characterization is that $\BbulletG$ admits a simplicial resolution
% \[
% 	\BbulletG \simeq \left(\xymatrix{
% 		{*} & G\ar@<-0.4ex>[l]\ar@<0.4ex>[l] & G\times G\ar@<-0.6ex>[l]\ar[l]\ar@<0.6ex>[l] & G\times G\times
% 		G\ar@<-0.8ex>[l]\ar@<-0.4ex>[l]\ar@<0.4ex>[l]\ar@<0.8ex>[l] &\cdots}\right)
% \]
\begin{equation*}
	\BbulletG \simeq \left(
	\begin{tikzcd}[sep=1.5em]
	    \cdots \arrow[r, shift left=0.75ex] \arrow[r, shift right=0.75ex] \arrow[r, shift right=2.25ex] \arrow[r, shift left=2.25ex] & G \cross G \arrow[l] \arrow[l, shift left=1.5ex] \arrow[l, shift right=1.5ex] \arrow[r] \arrow[r, shift left=1.5ex] \arrow[r, shift right=1.5ex] & G \arrow[l, shift left=0.75ex] \arrow[l, shift right=0.75ex] \arrow[r, shift left=0.75ex] \arrow[r, shift right=0.75ex] & * \arrow[l]
	\end{tikzcd}\right)
\end{equation*}
which is the content of the bar construction, and we want to compute $\pi_0$ of the simplicial set of maps
\begin{equation}
	\begin{tikzcd}[sep=2.5em]
	    \cdots \arrow[r, shift left=0.75ex] \arrow[r, shift right=0.75ex] \arrow[r, shift right=2.25ex] \arrow[r, shift left=2.25ex] & G \cross G \arrow[d, blue] \arrow[l] \arrow[l, shift left=1.5ex] \arrow[l, shift right=1.5ex] \arrow[r] \arrow[r, shift left=1.5ex] \arrow[r, shift right=1.5ex] & G \arrow[d] \arrow[l, shift left=0.75ex] \arrow[l, shift right=0.75ex] \arrow[r, shift left=0.75ex] \arrow[r, shift right=0.75ex] & * \arrow[d] \arrow[l] \\
	    \cdots \arrow[r, shift left=0.75ex] \arrow[r, shift right=0.75ex] \arrow[r, shift right=2.25ex] \arrow[r, shift left=2.25ex] & A \arrow[l] \arrow[l, shift left=1.5ex] \arrow[l, shift right=1.5ex] \arrow[r] \arrow[r, shift left=1.5ex] \arrow[r, shift right=1.5ex] & * \arrow[l, shift left=0.75ex] \arrow[l, shift right=0.75ex] \arrow[r, shift left=0.75ex] \arrow[r, shift right=0.75ex] & * \arrow[l]
	\end{tikzcd}
% \xymatrix{
% 	{*}\ar[d] &G\ar@<-0.4ex>[l]\ar@<0.4ex>[l] & G\times G\ar@[blue][d]\ar@<-0.6ex>[l]\ar[l]\ar@<0.6ex>[l] &\dots
% 	\ar@<-0.8ex>[l]\ar@<-0.4ex>[l]\ar@<0.4ex>[l]\ar@<0.8ex>[l]\\
% 	{*} & {*}\ar[l] & A\ar[l] & \dots\ar[l]
% }
\end{equation}
The blue map corresponds to the $2$-cocycle for the extension in ordinary group cohomology.

\begin{remark}
\label{cext_lie_alg}
	Differentiating a central extension of Lie groups produces a \emph{central extension of Lie algebras}
	\[
	0\rta	\afrak \rta \gtilde \rta \g \rta 0 \comma 
	\]
	which is what you would expect ($\afrak$ is an abelian Lie algebra contained in the center of
	$\gtilde$).

	Central extensions of Lie algebras are classified by second \emph{Lie algebra cohomology} $\HLie^2(\g;\afrak)$.  \index[notation]{HLie@$\HLie^\bullet$} \index[terminology]{Lie algebra!cohomology}
	Cocycles are alternating bilinear maps $\omega\colon\exterior^2\g\to\afrak$ satisfying a version of the Jacobi
	identity,
	\begin{equation}
	\label{LA_cext_Jacobi}
		\omega(X, [Y, Z]) + \omega(Y, [Z, X]) + \omega(Z, [X, Y]) = 0\period
	\end{equation}
	From such an $\omega$, we build a central extension which, as a vector space, is $\g\oplus\afrak$, but with
	Lie bracket
	\begin{equation}
	\label{liealgext}
		[(X_1, A_1), (X_2, A_2)] \colonequals [X_1, X_2] + \omega(X_1, X_2)\period
	\end{equation}
	A $1$-cochain is a map $\lambda\colon\g\to\afrak$, and its differential is $d\lambda(X,
	Y)\colonequals \lambda([X, Y])$.

	So we have a map $\H^2(\BbulletG, \underline A)\to \H^2(\g; \afrak)$. The van Est theorem says this is an
	equivalence in certain nice situations (not ours, unfortunately).
\end{remark}

%-------------------------------------------------------------------%
%-------------------------------------------------------------------%
%  The Virasoro algebra and the Virasoro group                      %
%-------------------------------------------------------------------%
%-------------------------------------------------------------------%

\subsection{The Virasoro algebra and the Virasoro group}

Let $\Gamma\colonequals \Diffplus(\Circ)$,  \index[notation]{Diff Pluss@$\Diffplus(\Circ)$}
the group of orientation-preserving diffeomorphisms of the circle. This is an
infinite-dimensional \emph{Fréchet Lie group}, meaning it is locally modeled on a Fréchet space and has a group
structure in which multiplication and inversion are smooth.

\begin{defn} \index[terminology]{Witt algebra} \index[notation]{Witt@$\wittRR$}
\label{Witt_algebra}
	The \emph{Witt algebra} $\wittRR$ is the infinite-dimensional real Lie algebra of polynomial vector fields on $\Circ$.
	Explicitly, it is generated by $\xi_n\colonequals -x^{n+1}\frac{\partial}{\partial x}$ for $n\in\ZZ$, with bracket
	\begin{equation}
		[\xi_m, \xi_n] \colonequals (m-n)\xi_{m+n} \period
	\end{equation}
\end{defn}

\noindent Skating over issues of regularity, the Witt algebra is the Lie algebra of $\Gamma$.\footnote{If we were to treat
regularity more carefully, we would allow some infinite linear combinations of the $\xi_n$, corresponding to the
Fourier series of a smooth vector field.}

The Virasoro algebra $\virRR$ is a central extension of $\wittRR$ by $\RR$.  \index[notation]{Vir@$\virRR$} \index[terminology]{Virasoro!algebra}
There is also a Virasoro group
$\widetilde\Gamma$, a central extension of $\Gamma$; the Virasoro algebra is its Lie algebra, and is easier to
define (since Lie algebra $\HLie^2$ just works to produce central extensions, whereas we had to modify group
cohomology). Specifically, consider the $2$-cocycle $c\colon \exterior^2\wittRR\to\RR$ given by
\begin{equation}
	c(\xi_m, \xi_n) \colonequals \frac{1}{12}(m^3-m)\delta_{m+n, 0}c \comma
\end{equation}
where $c$ is a chosen basis for $\RR$. The $1/12$ is not there for any deep reason, just as a normalization
constant. Anyways, as in~\eqref{liealgext} this defines for us an extension
\begin{equation*}
	1\to\RR\to\virRR\to\wittRR\to 1 \comma
\end{equation*}
called the \emph{Virasoro algebra}. The element $c$ inside $\virRR$ is called the \emph{central charge}.  \index[terminology]{Central Charge}

The Virasoro group $\widetilde\Gamma$ is the extension of $\Gamma$ by $\TT$ which is, as a space,
$\TT\times\Gamma$, with multiplication
\begin{equation}
	(z_1, f)\cdot (z_2, g) \colonequals (z_1 + z_2 + B(f, g), f\circ g) \comma
\end{equation}
where $B\colon \Gamma\times\Gamma\to\TT$ is the \emph{Bott cocycle}  \index[terminology]{Bott cocycle}
\begin{equation}
	B(f, g)\colonequals \oint_{\Circ} \log(f\circ g)' \mrm{d}(\log g)' \period
\end{equation}

\begin{remark}
	The identification $\Circ\cong\RRP^1$ embeds $\PGL_2^+(\RR) = \PSL_2(\RR)\subset\Gamma$ as the real fractional linear
	transformations; hence also
	\begin{equation*}
		\psl_2(\RR) = \slfrak_2(\RR)\subset\wittRR \comma
	\end{equation*}
	as the Lie algebra generated by
	$\xi_{-1}$, $\xi_0$, and $\xi_1$. Restricted to $\PSL_2(\RR)$, the Virasoro central extension is trivializable,
	which will be useful later.
\end{remark}


\begin{remark}
	Some authors' definitions will differ. 
	For example, defining the Witt and Virasoro algebras as complex Lie algebras, or
	defining the Virasoro group as the universal cover of ours. 
	%In particular, I used a few different sources when preparing this talk, and though I think they were consistent, it's possible that I missed something and they used different conventions.
	%%I (Minta) checked these and think they are all ok!
\end{remark}

\begin{remark}[(Applications)]  \index[terminology]{Conformal Field Theory}
\label{virasoro_applications}
The Virasoro group and algebra appear in two-dimensional conformal field theory (CFT). Usually, in quantum field
theory, one specifies a (Riemannian or Lorentzian) metric on spacetime, and the information in the theory depends
on the metric. A \emph{conformal field theory} is a quantum field theory in which all information only depends on
the conformal class of the metric. Two-dimensional CFTs in particular connect to many areas of mathematics and
physics.
\begin{itemize}
	\item The mathematical formalization of 2d CFT, using vertex algebras, has connections to representation
	theory, and, famously, to monstrous moonshine.

	\item One way to think of string theory is as a 2d CFT on the worldsheet, one of whose fields is a map into
	(10- or 26-dimensional) spacetime.

	\item In condensed-matter physics, Wess--Zumino--Witten models (particular 2d CFTs) are used in modeling the
	quantum Hall effect. See also \Cref{ex-WZW}.

	\item Maybe closest to the hearts of the attendees of this seminar: the Stolz--Teichner conjecture suggests that
	cocycles for TMF on a space $X$ are given by families of 2d supersymmetric quantum field theories parametrized
	by $X$. Superconformal field theories are particularly nice examples of these, and have been used to shine
	light on this conjecture.\footnote{The appearances of SCFTs, rather than just CFTs, in superstring theory and
	in the Stolz--Teichner conjecture aren't as related to the Virasoro group and algebra; they have a larger
	symmetry algebra, though it's closely related.}
\end{itemize}
So how does the Virasoro appear in CFT? Let's suppose we're on a Riemann surface $\Sigma$ in a local holomorphic
coordinate $z$. If you write out commutators for the Lie algebra $\mathfrak c$ of infinitesimal conformal
transformations, you might notice they look like those for the Witt algebra --- in fact, if you complexify it, you
obtain precisely $\wittCC \oplus \wittCC $. So this acts on the system as a symmetry; you can think of it as two
different Witt group symmetries.

The fact that we obtain a central extension is standard lore from quantum mechanics. The state space in a quantum
system is a complex Hilbert space, but if $\lambda\in\bb{C}^\times$, the states $\vert\psi\rangle$ and $\lambda\vert\psi\rangle$ are
thought of as the same, in that measurements cannot distinguish them. Nonetheless, the formalism of quantum
mechanics uses the Hilbert space structure.

The takeaway, though, is that a symmetry of the system, as in acting on the states and all that, only has to be a
projective representation on the state space! So to describe an honest Lie group or Lie algebra acting on the state
space, we need to take a central extension of the symmetry group or Lie algebra. This leads us to the
(complexified) Virasoro algebra and Virasoro group. Thus, the symmetry algebra of conformal field theory is (at
least) a product of two copies of the Virasoro algebra, and the space of states is a representation of the Virasoro
algebra.
\end{remark}

%-------------------------------------------------------------------%
%-------------------------------------------------------------------%
%  Constructing the central extension with differential cohomology  %
%-------------------------------------------------------------------%
%-------------------------------------------------------------------%

\subsection{Constructing the central extension with differential cohomology}
\label{diffcoh_virasoro}

The key fact bridging differential cohomology and central extensions is:

\begin{lem}
\label{cext_lemma}
	There is an equivalence of simplicial sheaves $\ZZ(1)\simeq \Sigma^{-1}\underline{\TT}$.
\end{lem}

\begin{proof}
	By definition, $\ZZ(1)$ is the sheaf $0\to \ZZ\to\Omega^0\to 0$, and $\Omega^0 = \underline{\RR}$. The chain map
	\begin{equation}
		\begin{tikzcd}
			0\arrow[r] & \ZZ\arrow[d]\arrow[r] & \underline{\RR}\arrow[d, "\mod{\ZZ}"] \arrow[r] & 0\\
			0\arrow[r] & 0\arrow[r] & \underline{\TT}\arrow[r] & 0
		\end{tikzcd}
	\end{equation}
	is a quasi-isomorphism.
\end{proof}

\begin{cor}
\label{cext_corollary}
	For any Lie group $G$, possibly infinite-dimensional, we have an isomorphism
	\begin{equation*}
		\H^2(\BbulletG;\underline{\TT})\cong \H^3(\BbulletG;\ZZ(1)) \period
	\end{equation*}
	In particular, the group $ \H^3(\BbulletG;\ZZ(1)) $ classifies central extensions of $G$ by $\TT$ which are principal $\TT$-bundles
	over $G$.
\end{cor}

Thus, we would like to construct the Virasoro central extension via a differential cohomology class in
$\H^3(\BbulletGamma;\ZZ(1))$. 
This builds on the hard work of the previous few talks.
In \Cref{thm-BGZ}, Hopkins described how $\H^4(\BbulletGL_n(\RR);\ZZ(2))$ fits into a pullback square
\begin{equation}\label{diffp1square}
	\begin{tikzcd}
		\H^4(\BbulletGL_n(\RR);\ZZ(2)) \arrow[r, blue] \arrow[d] & \H^4(\BGL_n(\RR);\ZZ) \arrow[d]\\
		\H^4(\BbulletGL_n(\RR);\RR(2)) \arrow[r, red] & \H^4(\BGL_n(\RR);\RR) \period
	\end{tikzcd}
\end{equation}

\begin{defn}
	An \emph{off-diagonal differential lift of $p_1$} is a class $\tilde{p}_1\in \H^4(\BbulletGL_n(\RR);\ZZ(2))$ whose image under the blue map is the usual $p_1\in \H^4(\BGL_n(\RR);\ZZ)$.
\end{defn}

By \Cref{cor-CWoff}, we have an isomorphism 
$\H^4(\BbulletG; \RR(2))\cong\Sym^2(\gdual)^G$. 
For $\GL_n(\RR)$, this is
an $\RR^2$, spanned by the invariant polynomials $\Tr(A)^2$ and $\Tr(A)^2$, which we call $c_1^2$ and $c_2$,
respectively. 
The group $\H^4(\BG;\RR)$ can be dispatched with ordinary Chern--Weil theory: we repeat the same story, but
retracting $G$ onto its maximal compact. 
Here, we get
\begin{equation*}
	\H^4(\BO(n);\RR)\cong\RR \comma
\end{equation*}
spanned by $\Tr(A^2)$, as $\Tr(A)^2 = 0$. 
Accordingly, the red map in~\eqref{diffp1square} is a rank-$1$ map $\RR^2\to\RR$. Since~\eqref{diffp1square} is
a pullback square, there is an $\RR$ worth of differential lifts of $p_1$: explicitly, $\lambda\in\RR$ gives you the
lift of $p_1$ which maps in the lower left to $(1/2)(\lambda c_1^2 - 2c_2)$. However, if you want $
\tilde{p}_1(E_1\oplus E_2) = \tilde{p}_1(E_1) + \tilde{p}_1(E_2)$, you force $\lambda = 1$, which is a quick calculation with
the Whitney formula. 
(All this was in \Cref{sec:Delignecup}.)

In \cref{FiberIntegration}, we also discussed the fiber integration map for an $\Hhat $-oriented fiber bundle
\begin{equation*}
	F\to E\to B \comma  
\end{equation*}
which has the form
\begin{equation*}
	\H^k(E;\ZZ(\ell))\to \H^{k-\dim(F)}(B;\ZZ(\ell-\dim(F))) \period
\end{equation*}
Combining all this, consider the universal
oriented sphere bundle $\EbulletGamma\times_\Gamma \Circ\to \BbulletGamma$, which is an $\Hhat $-oriented fiber
bundle with fiber $\Circ$. Therefore, given a differential lift of $p_1$, we can apply it to the vertical tangent
bundle $V\to \EbulletGamma\times_\Gamma \Circ$, and get a class $\tilde{p}_1(V)\in \H^4(\EbulletGamma\times_\Gamma
\Circ;\ZZ(2))$. Then we can push it forward to a class in $\H^3(\BbulletGamma;\ZZ(1))$, which determines an
isomorphism class of central extensions of $\Gamma$ as above. The goal is to determine the choice of $\lambda$ such
that this central extension gives the Virasoro group. I'll suggest some ways forward.

The first thing we need is a way to get a handle on the group of extensions of $\Gamma$. Recall that
$\PSL_2(\RR)\subset\Gamma$ as the real fractional linear transformations of $\RRP^1 = \Circ$; hence a central extension
of $\Gamma$ restricts to a central extension of $\PSL_2(\RR)$.

\begin{thm}[{(Segal \cite[Corollary 7.5]{Seg81})}]
	A central extension of $\Gamma$ by $\TT$ is determined by the pair of (1) its restriction to $\PSL_2(\RR)$ and (2)
	the induced Lie algebra central extension of $\wittRR$ by $\RR$. 
	Said differently, there is an isomorphism of abelian
	groups
	\begin{equation*}
		\Cent_{\TT}(\Gamma) \isomorphism \Cent_{\TT}(\PSL_2(\RR))\times\Cent_{\RR}(\wittRR) \period
	\end{equation*}
\end{thm}

We can identify both of these groups. First, $\pi_1\PSL_2(\RR)\cong\ZZ$, and the universal cover
$\SLtilde_2(\RR)\to\PSL_2(\RR)$\footnote{The notation is because it's also the universal cover of $\SL_2(\RR)$,
which is the connected double cover of $\PSL_2(\RR)$.} is the \emph{universal central extension} of $\PSL_2(\RR)$:
for any abelian group $A$, central extensions of $\PSL_2(\RR)$ by $A$ are in bijection with maps $\varphi\colon \ZZ\to
A$, given by
\begin{equation}
	\begin{tikzcd}
		0\arrow[r] & \ZZ\arrow[r]\arrow[d, "\varphi"'] & \SLtilde_2(\RR)\arrow[d]\arrow[r] & \PSL_2(\RR)\arrow[r] \arrow[d, equals] & 0\\
		0\arrow[r] & A\arrow[r] & (\SLtilde_2(\RR))_\varphi\arrow[r] & \PSL_2(\RR)\arrow[r] & 0 \period
	\end{tikzcd}
\end{equation}
So $\Cent_{\TT}(\PSL_2(\RR))\cong\Hom(\ZZ, \TT) = \TT$. 
The computation that $\H^2(\wittRR;\RR)\cong\RR$ is standard, e.g. \cite[\S 6.2.1]{Obl17}.

Thus the map from off-diagonal differential lifts of $p_1$ to central extensions of $\Gamma$ is a map $\RR\to\RR\times\TT$. 

One can then ask the following question, 
which was posed to us by Dan Freed and Mike Hopkins.
\begin{quest}
Does there exist an off-diagonal differential lift $\tilde{p}_1$ of the first Pontryagin class that hits the Virasoro algebra central extension in $\RR\times\TT$?
\end{quest}

Note that the Virasoro central extension is in the first factor of $\RR\times\TT$.  
Indeed, it induces the Virasoro algebra central
extension, and hence is nontrivial on the first factor and trivial when restricted to $\PSL_2(\RR)$.

%%%%%removed proof ideas %%%%%%%

%The $\tilde{p}_1$-induced central extensions should also land purely in the $\RR$ factor, which Dan Freed suggested
%to me. His proof idea involved trivializing this characteristic class over $\BnablaPSL_2(\RR)$. Alternatively,
%this amounts to showing that for any differential lift $\tilde{p}_1$ of $p_1$, under the fiber integration map
%\begin{equation}
%	\H^4(\EbulletPSL_2(\RR)\times_{\PSL_2(\RR)} \Circ;\ZZ(2))\longrightarrow %\H^3(\BbulletPSL_2(\RR); \ZZ(1)),
%\end{equation}
%$\tilde{p}_1(V)\mapsto 0$. We're going to need to understand this map for $\BbulletGamma$ anyways, so studying
%this one might make for a good baby case to figure out first. After that, it will hopefully be clearer how to study
%the map for $\BbulletGamma$. We have a map between two real lines, though we can identify both of them with $\RR$:
%the line of differential lifts of $p_1$, through $\lambda$; the line of central extensions of $\Gamma$ trivialized
%on $\PSL_2(\RR)$, through the Virasoro extension (either of the group or the Lie algebra).

% include calculation of \H^2(\mfrk{v}; R), but don't go into it

% here possibly mention Segal's calculation that a central extension is determined by its restriction to PSL(2, R)
% and its induced Lie alg central extension


%\subsection{Central extensions from differential lifts of $p_1$}


% recall, there's an R worth of lifts of p_1
% only one satisfies additivity (include proof, but don't go into it)
%\begin{remark}
%Dan Freed suggested another, alternate way to obtain ``the right'' differential lift of $p_1$, though he hasn't
%thought about it in much detail: $\H^4(\BbulletGL_n(\RR);\ZZ(4))\cong\ZZ$, with generator given by a different lift
%of $p_1$ to differential cohomology, namely the one given by Chern--Weil theory. The truncation map $\ZZ(4)\to\ZZ(2)$
%nduces a map 
%\[\H^4(\BbulletGL_n(\RR);\ZZ(4))\to \H^4(\BbulletGL_n(\RR);\ZZ(2))\comma\]
% and this map is not surjective; it
%might send the generating lift of $p_1$ to something helpful.
%\end{remark}
