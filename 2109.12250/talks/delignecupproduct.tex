%!TEX root = ../diffcoh.tex

\section{Deligne Cup Product}\label{sec:Delignecup}
\textit{by Araminta Amabel}

Let $M$ be a manifold. 
Recall that the Deligne complex $\ZZ(k)$ is the homotopy pullback
\begin{equation*}
	\begin{tikzcd}
		\ZZ(k)\arrow[r]\arrow[d] & \ZZ\arrow[d] \\
		\Sigma^k\Omegacl^k\arrow[r] & \RR \period
	\end{tikzcd}
\end{equation*}
The goal of this section is to combine the cup product on $\HZZ$ and the wedge product on differential forms to put a ring structure on differential cohomology.

%-------------------------------------------------------------------%
%-------------------------------------------------------------------%
%  Combining the Cup and Wedge Products                             %
%-------------------------------------------------------------------%
%-------------------------------------------------------------------%

\subsection{Combining the Cup and Wedge Products}

\begin{nul}
	Notice that the cup product on $ \HZZ $ and $ \HRR $ and the wedge product on differential forms fit into a commutative digram.
	\begin{equation*}
		\begin{tikzcd}
			\ZZ(n)\otimes\ZZ(m)\arrow[r]\arrow[d] & \HZZ[n]\otimes \HZZ[m]\arrow[r, "\cupprod"] \arrow[d] & \HZZ[m+n]\arrow[dd]\\
			\Omegacl^n\otimes\Omegacl^m \arrow[d, "\wedge"'] \arrow[r] & \HRR[n]\otimes \HRR[m]\arrow[dr, "\cupprod" description] & \\
			\Omegacl^{n+m}\arrow[rr] & & \HRR[m+n] \period 
		\end{tikzcd}
	\end{equation*}
\end{nul}

\begin{nul}
	By the definition of $ \ZZ(k) $ as a pullback, we can represent $\ZZ(k)(M)$ as a triple $(c,h,\omega)$ where $c$ is an integral degree $k$ cocycle on $M$, $\omega$ is a closed $k$ form on $M$, and $h$ is a degree $k-1$ real cochain on $M$ so that $\d x=\omega-c$.
\end{nul}

\begin{nul}
	In particular, if we represent an element of $\Cup^n(M;\ZZ(n)) $ by a triple $(c_1,h_1,\omega_1)$ and an element of $\Cup^m(M;\ZZ(m))$ by a triple $(c_2,h_2,\omega_2)$ we would like the product to be a triple
	\[
		(c_1,h_1,\omega_1)\cupprod(c_2,h_2,\omega_2)=(c_3,h_3,\omega_3)\in \Cup^{m+n}(M;\ZZ(m+n)) \period
	\]
	Saying that this product comes from combining the cup product and the wedge product, means that $c_3=c_1\cupprod c_2$ and $\omega_3=\omega_1\wedge\omega_2$. We are only left with figuring out what $h_3$ should be. 
	Heuristically, $h_3$ should be a homotopy between $c_3$ and $\omega_3$; i.e., a homotopy between the cup product and the wedge product. 
\end{nul}

\begin{nul}
	Given forms $\omega\in\Omega^n(M)$ and $\eta\in\Omega^m(M)$, we can form the wedge product $\omega\wedge \eta\in\Omega^{n+m}(M)$ and view that as a real cochain under the map $\Omega^{n+m}(M)\rta \Cup^{n+m}(M;\RR)$.  
	We could also map the forms $\omega,\eta$ to real cochains on $M$ and then take their cup product. 
	Let $B(\omega,\eta)\in \Cup^{n+m-1}(M;\RR)$ be a choice of natural homotopy between these two cochains so that 
	\[
		\d B(\omega,\eta)+B(d\omega,\eta)+(-1)^{|\omega|} B(\omega,d\eta)=\omega\wedge\eta-\omega\cupprod \eta \period
	\]
	Note that we can take $B(\omega,0)=0$. 
\end{nul}

\begin{nul}
	Then the product of $(c_1,h_1,\omega_1)\in \Cup^n(M;\ZZ(n))$ and $(c_2,h_2,\omega_2)\in \Cup^m(M;\ZZ(m))$ is given by
	\[
		(c_3,h_3,\omega_3)=(c_1\cupprod c_2,(-1)^{|c_1|}c_1\cupprod h_2+h_1\cupprod\omega_2+B(\omega_1,\omega_2),\omega_1\wedge\omega_2) \period 
	\]
	For this to be a differential cocycle, we need to have 
	\[
		\d((-1)^{|c_1|}c_1\cupprod h_2+h_1\cupprod \omega_2+B(\omega_1,\omega_2)=\omega_1\wedge\omega_2-c_1\cupprod c_2 \period 
		\]
	This will only work if $(c_1,h_1,\omega_1)$ and $(c_2,h_2,\omega_2)$ are themselves cocycles; i.e., $\d c_i=0=\d\omega_i$. 
	In this case, we have
	\[
		\omega_1\wedge\omega_2-\omega_1\cupprod\omega_2=\d B(\omega_1,\omega_2)=B(0,\omega_2)+(-1)^{|\omega_1|}B(\omega_1,0)=\d B(\omega_1,\omega_2) \period
	\]
	Thus
	\begin{align*}
		\d\paren{(-1)^{|c_1|}c_1\cupprod h_2+h_1\cupprod \omega_2+B(\omega_1,\omega_2)} &= (-1)^{|c_1|}\d(c_1\cupprod h_2)+\d(h_1\cupprod\omega_2)+\d B(\omega_1,\omega_2) \\
		&=(-1)^{|c_1|}\left(\d c_1\cupprod h_2+(-1)^{|c_1|}c_1\cupprod \d h_2\right)+\d h_1\cupprod\omega_2 \\ 
		&\phantom{=} \qquad +(-1)^{|h_1|}h_1\cupprod \d\omega_2+\d B(\omega_1,\omega_2) \\
		&= c_1\cupprod \d h_2+\d h_1\cupprod\omega_2+dB(\omega_1,\omega_2) \\
		&= c_1\cupprod(\omega_2-c_2)+(\omega_1-c_1)\cupprod\omega_2+\omega_1\wedge\omega_2-\omega_1\cupprod\omega_2 \\
		&= c_1\cupprod\omega_2-c_1\cupprod c_2+\omega_1\cupprod\omega_2-c_1\cupprod\omega_2 \\ 
		&\phantom{=} \qquad +\omega_1\wedge\omega_2-\omega_1\cupprod\omega_2 \\ 
		&=\omega_1\wedge\omega_2-c_1\cupprod c_2 \period
	\end{align*}
\end{nul}

\begin{remark}
	In fact we can get $\Einf$-structure from the homotopy pullback diagram.
	% \todo{It might be worth expanding on this remark more.} 
	View $\HZZ$ as a (trivially) filtered $\Einf$-algebra. 
	View the de Rham complex $ \Omegabullet$ as a filtered $\Einf$-algebra with filtration $ \{\Omega^{\geq k}\}_{k \geq 0} $. 
	Then the homotopy pullback of two $\Einf$-algebras is again an $\Einf$-algebra.
\end{remark}

%-------------------------------------------------------------------%
%-------------------------------------------------------------------%
%  Deligne Cup Product                                              %
%-------------------------------------------------------------------%
%-------------------------------------------------------------------%

\subsection{Deligne Cup Product}

Recall that we have an identification of the homotopy pullback $\HZZhat(k)$ with the complex of sheaves $\ZZ(k)$,
\begin{equation*}
	\ZZ(k)=\bigg(
	\begin{tikzcd}[sep=1.5em]
		\Gamma^*\ZZ \arrow[r, "\iota"] & \Omega^0 \arrow[r, "\d"] & \Omega^1 \arrow[r, "\d"] & \cdots \arrow[r, "\d"] & \Omega^{k-1} 
	\end{tikzcd}\bigg) \period
\end{equation*}
Under this identification, we can describe the product in differential cohomology more explicitly. This is
sometimes called the ``Deligne cup product.''

Let $M$ be a manifold and $U\subset M$ an open set. 
Then $\ZZ(k)(U)$ is a chain complex that is $\Cup^0(U;\ZZ)$ in degree $ 0 $ and $\Omega^p(U)$ in degree $p+1$. 

\begin{proposition}\label{formula}
	The Deligne cup product
	\begin{equation*}
		\cupprod \colon \ZZ(k)(U)\otimes\ZZ(\ell)(U)\rta\ZZ(k+\ell)(U)
	\end{equation*}
	is given by 
	\begin{equation*}
		x \cupprod y =
		\begin{cases}
			x\cdot y \comma & \deg(x)=0\\
			x\wedge \iota y \comma  &\deg(x)>0, \deg(y)=0\\
			x\wedge \d y \comma & \deg(x)>0,\deg(y)=\ell>0\\
			0 \comma & \textup{otherwise } \phantom{0,\deg(y)=\ell>0} \period
		\end{cases}
	\end{equation*}
\end{proposition}

\begin{remark}
	This is only commutative up to homotopy.
\end{remark}

%-------------------------------------------------------------------%
%-------------------------------------------------------------------%
%  Examples                                                         %
%-------------------------------------------------------------------%
%-------------------------------------------------------------------%

\subsection{Examples}\label{subsec:Delignecupexamples}

We analyze the Deligne cup product in detail in the lowest dimensions. Let $M$ be a manifold. Recall the following computations.
\begin{itemize}
	\item $\ZZ(0)=\Gamma^*\ZZ[0]$ is the complex with $\Gamma^*\ZZ$ in degree zero. 
	Thus $\Hcech^0(M)=\H^0(M;\ZZ)$.

	\item $\Hcech^1(M)=\Mapsm(M,\Uup_{1})$.

	\item $\Hcech^2(M)=\{\text{line bundles on $M$ with connection}\}/\sim$.
\end{itemize}
Let $\ZZ(k)^\ell$ denote the degree $\ell$ term of the complex $\ZZ(k)$. 
For example, $\ZZ(3)^2=\Omega^1$. 
Let $\Ucal$ be a good cover for $M$. 
Using Čech cohomology for this good cover, the Deligne cup product gives a map
\begin{equation*}
	\paren{\bigoplus_{i+j=k}\Cech^i(\Ucal;\ZZ(k)^j)} \tensor \paren{\bigoplus_{i+j=l}\Cech^i(\Ucal;\ZZ(\ell)^j)} \longrightarrow \paren{\bigoplus_{i+j=k+\ell}\Cech^i(\Ucal;\ZZ(k+\ell)^j)} \period
\end{equation*}

\begin{example}
	The Deligne cup product
	\[
		\ZZ(0)\otimes\ZZ(0)\rta\ZZ(0)
	\]
	should give us a way of taking two locally constant functions of $M\rta\ZZ$ and producing a third.
	By \Cref{formula}, the Deligne cup product of two elements in degree 0 agrees with the ordinary cup product in $\H^0(M;\ZZ)$; i.e., the product of the two locally constant functions.
\end{example}

\begin{example}
	The Deligne cup product
	\[
		\ZZ(0)\otimes\ZZ(1)\rta\ZZ(1)
	\]
	should give us a way of taking a locally constant function $M\rta\ZZ$ and a smooth map $g\colon M\rta \Uup_{1}$ and producing a new smooth map $M\rta \Uup_{1}$.  
	In the Čech complex, we are looking at a map
	\[
		\Cech^0(\Ucal;\ZZ(0)^0)\otimes \left(\Cech^0(\Ucal;\ZZ(1)^1)\oplus \Cech^1(\Ucal;\ZZ(1)^0)\right)\rta \left(\Cech^0(\Ucal;\ZZ(1)^1)\oplus \Cech^1(\Ucal;\ZZ(1)^0)\right)
	\]
	Identifying these terms, we have
	\[
		\Cech^0(\Ucal;\ZZ)\otimes \left(\Cech^0(\Ucal;\Omega^0)\oplus \Cech^1(\Ucal;\ZZ)\right)\rta \left(\Cech^0(\Ucal;\Omega^0)\oplus \Cech^1(\Ucal;\ZZ)\right)
	\]
	This sends $n\otimes(f,m)$ to $(n\cdot f,n\cdot m)$.
\end{example}

\begin{example}
	The Deligne cup product
	\[
		\ZZ(1)\otimes\ZZ(0)\rta\ZZ(1)
	\]
	should give us a way of taking a locally constant function $M\rta \ZZ$ and a smooth map $g\colon M\rta \Uup_{1}$ and producing a new smooth map $M\rta \Uup_{1}$. In the Čech complex, we are looking at a map
	\[ 
		\left(\Cech^0(\Ucal;\Omega^0)\oplus \Cech^1(\Ucal;\ZZ)\right)\otimes\Cech^0(\Ucal;\ZZ)\rta \left(\Cech^0(\Ucal;\Omega^0)\oplus \Cech^1(\Ucal;\ZZ)\right)
	\]
	This map sends $(f,m)\otimes n)$ to $(f\cdot \iota n,m\cdot n)$.
\end{example}

More geometrically, we can describe the Deligne cup product as follows. 
Given a pair $(n,f)$ where $n\colon M\rta\ZZ$ is a locally constant function and $f\colon M\rta \Circ$ is a smooth map, the Deligne cup product of $n$ with $f$ is the smooth function $g\colon M\rta \Circ$ given by $g(x)=e^{2\pi i n(x)}f(x)$. 

\begin{remark}
	We can note that the Deligne cup product commutes up to homotopy,
	\begin{equation*}
		\begin{tikzcd}
			\ZZ(1)\otimes\ZZ(0)\arrow[r]\arrow[d] & \ZZ(1)\\
			\ZZ(0)\otimes\ZZ(1)\arrow[ur] & 
		\end{tikzcd}
	\end{equation*}
	since $(f\cdot \iota n=n\cdot f)$ as functions to $\RR$.
\end{remark}

\begin{example}
	The Deligne cup product
	\[
		\ZZ(1)\otimes\ZZ(1)\rta\ZZ(2)
	\]
	should give us a way of taking two smooth maps $M\rta \Uup_{1}$ and producing a line bundle on $M$ with connection. In the Čech complex, we are looking at a map
	\[
		\left(\Cech^0(\Ucal;\ZZ(1)^1)\oplus \Cech^1(\Ucal;\ZZ(1)^0)\right)^{\otimes 2}\rta\left(\Cech^0(\Ucal;\ZZ(2)^2)\oplus\Cech^1(\Ucal;\ZZ(2)^1)\oplus\Cech^2(\Ucal;\ZZ(2)^0)\right) \period
	\]
	Then the Deligne cup product sends 
	\[
		(f,n)\otimes(g,m)\mapsto(n_{\alpha\beta}\cdot m_{\beta\gamma},n_\alpha\beta\cdot g_\beta+0,f_\alpha \d g_\alpha) \period
	\]
	If we think of $(f,n)$ and $(g,m)$ as smooth maps $M\rta \Uup_{1}$, then $(n_{\alpha\beta}\cdot m_{\beta\gamma}, n_{\alpha\beta}\cdot g_\beta,f_\alpha \d g_\alpha)$ corresponds to the line bundle with transition function $n_{\alpha\beta}\cdot g_\beta$ and connection given by one form $(2\pi i)f_\alpha \d g_\alpha$.

	By \cite[Lemma 1.3.1]{Beilinson}, the curvature of $f\cupprod g$ is $\dlog (f)\wedge\dlog(g)$.
\end{example}
