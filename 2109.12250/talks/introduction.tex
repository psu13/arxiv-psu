%!TEX root = ../diffcoh.tex

\section{Introduction}\label{Introduction}
\textit{by Peter Haine}

The purpose of this chapter is to give some motivation for the perspective we take on differential cohomology.
We do this by giving an overview of the work of Cheeger--Simons \cite{MR827262}, Deligne \cites[\S2.2]{MR498551}[\S12.3]{MR2451566}, and Simons--Sullivan \cite{MR2365651} on differential cohomology.

%-------------------------------------------------------------------%
%-------------------------------------------------------------------%
%  Motivation for differential cohomology                           %
%-------------------------------------------------------------------%
%-------------------------------------------------------------------%

\subsection{Motivation for differential cohomology}

\begin{observation}[{(Simons--Sullivan \cite[\S1]{MR2365651})}]
	Let $ M $ be a manifold.
	Then we have exact sequences
	\begin{equation}\label{diag:diffcohpre}
		\begin{tikzcd}[column sep={10ex,between origins}, row sep={8ex,between origins}]
			& \H^{k-1}(M;\RR/\ZZ) \arrow[rr, "-\Bock"] & & \H^k(M;\ZZ) \arrow[dr] & \\
			\HdR^{k-1}(M) \arrow[ur] \arrow[dr] & & & & \HdR^k(M) \\
			& \Omega^{k-1}(M)/\im(\d) \arrow[rr, "\d"'] & & \Omegacl^k(M) \arrow[ur] & \phantom{\HdR^k(M)} \comma
		\end{tikzcd}
	\end{equation}
	where the top sequence is the Bockstein sequence associated to the short exact sequence
	\begin{equation*}
		\begin{tikzcd}[sep=1.5em]
			0 \arrow[r] & \ZZ \arrow[r, hook] & \RR \arrow[r, ->>] & \RR/\ZZ \arrow[r] & 0 \comma 
		\end{tikzcd}
	\end{equation*}
	and we are identifying singular and de Rham cohomology via the de Rham isomorphism
	\begin{equation*}
		 \HdR\upperstar(M) \isomorphic \H\upperstar(M;\RR) \period
	\end{equation*}

	The top sequence is ``purely homotopy-theoretic'' in nature, while the bottom sequence is ``purely geometric'' in nature (e.g., the functor $ \Omegacl^k $ is not homotopy-invariant).
\end{observation}

\begin{question}\label{qst:origin}
	Can we fill \eqref{diag:diffcoh} in with an invariant \textcolor{love}{$ \Hhat^k(M;\ZZ) $ in maroon} 
	\begin{equation}\label{diag:diffcoh}
		\begin{tikzcd}[column sep={10ex,between origins}, row sep={8ex,between origins}]
			& \H^{k-1}(M;\RR/\ZZ) \arrow[rr, "-\Bock"] \arrow[dr, love] & & \H^k(M;\ZZ) \arrow[dr] & \\
			\HdR^{k-1}(M) \arrow[ur] \arrow[dr] & & \textcolor{love}{\Hhat^k(M;\ZZ)} \arrow[ur, love] \arrow[dr, love] & & \HdR^k(M) \\
			& \Omega^{k-1}(M)/\im(\d) \arrow[rr, "\d"'] \arrow[ur, love] & & \Omegacl^k(M) \arrow[ur] & \phantom{\HdR^k(M)} \comma
		\end{tikzcd}
	\end{equation}
	that better blends homotopy theory and geometry, and makes the diagonals exact?
\end{question}

Now let us attempt to provide a satisfactory answer to \Cref{qst:origin} when $ k = 1 $.

\begin{attempt}[(for $ k = 1$)]
	Let $ M $ be a manifold.
	Consider the abelian group $ \Cinf(M,\RR/\ZZ) $ of smooth functions to the circle (with the group structure defined pointwise).
	We should really think of $ \Cinf(M,\RR/\ZZ) $ as an infinite-dimensional abelian Lie group.
	Recall that the inclusion
	\begin{equation*}
		 \Cinf(M,\RR/\ZZ) \subset \Map(M,\RR/\ZZ)
	\end{equation*}
	from the space of smooth maps to the space of all maps is a homotopy equivalence.
	Since the circle is $ 1 $-truncated,\footnote{I.e., only has nontrivial homotopy groups in degrees $ \leq 1$.} 
	this implies that 
	$ \Cinf(M,\RR/\ZZ) $ is also $ 1 $-truncated.

	Since $ \RR/\ZZ $ is a $ \K(\ZZ,1) $, we see that
	\begin{equation*}
		\uppi_0 \Cinf(M,\RR/\ZZ) \isomorphic \H^1(M;\ZZ) \period
	\end{equation*}
	In particular, we have a surjection $ \uppi_0 \colon \surjto{\Cinf(M,\RR/\ZZ)}{\H^1(M;\ZZ)} $.
	Also notice that
	\begin{align*}
		\uppi_1\Cinf(M,\RR/\ZZ) &\isomorphic \uppi_0\Map_{*}(\Circ,\Cinf(M,\RR/\ZZ)) \\
		&\isomorphic \uppi_0\Map_{*}(\Circ,\Map(M,\RR/\ZZ)) \\
		&\isomorphic \uppi_0\Map(M,\Map_{*}(\Circ,\RR/\ZZ)) \\
		&\isomorphic \uppi_0\Map(M,\Omega(\RR/\ZZ)) \\
		&\isomorphic \H^0(M;\ZZ) \period
	\end{align*}
\end{attempt}

\begin{construction}
	Let $ \vol $ denote the standard volume form on the circle $ \Circ \isomorphic \RR/\ZZ $.
	Define a \emph{curvature} map $ \curv \colon \fromto{\Cinf(M,\RR/\ZZ)}{\Omegacl^1(M)} $ by
	\begin{equation*}
		\curv(f) \colonequals \fupperstar(\vol) \period
	\end{equation*}
\end{construction}

\begin{nul}
	The kernel of $ \curv $ consists of the locally constant maps $ \fromto{M}{\RR/\ZZ} $, i.e.,
	\begin{equation*}
		\ker(\curv) \isomorphic \H^0(M;\RR/\ZZ) \period
	\end{equation*}
	Note that the curvature map is not surjective:
	\begin{equation*}
		\image(\curv) = \{ \alpha \in \Omegacl^1(M) \, | \, \textstyle\int_{\Circ} \alpha \in \ZZ \text{ for every embedding } \incto{\Circ}{M} \} \period
	\end{equation*}
	That is, the image of $ \curv $ is the group of \textit{closed $ 1 $-forms with integral periods}.
\end{nul}

\begin{definition}
	Let $ M $ be a manifold and $ k \geq 0 $ an integer. 
	A closed $ k $-form $ \omega $ on $ M $ \emph{has integral periods} if for every smooth $ k $-cycle $ c $ in $ M $ the integral $ \int_c \omega $ is an integer.
	We write
	\begin{equation*}
		\Omegacl^k(M)_{\ZZ} \subset \Omegacl^k(M)
	\end{equation*}
	for the subgroup of $ k $-forms with integral periods.
\end{definition}

\begin{remark}
	A closed $ k $-form $ \omega $ has integral periods if and only if the class of $ \omega $ lies in the image of the change-of-coefficients map
	\begin{equation*}
		\fromto{\H^k(M;\ZZ)}{\H^k(M;\RR) \isomorphic \HdR^k(M)} \period 
	\end{equation*}
\end{remark}

\begin{nul}
	We also have a map
	\begin{equation*}
		\iota \colon \fromto{\Omega^0(M) = \Cinf(M,\RR)}{\Cinf(M,\RR/\ZZ)}
	\end{equation*}
	given by post-composition with the quotient map $ \surjto{\RR}{\RR/\ZZ} $.
	The map $ \iota $ has kernel the integer-valued smooth functions $ \fromto{M}{\RR} $, i.e., the locally constant functions with integer values. 
	That is, $ \image(\iota) = \Omegacl^0(M)_{\ZZ} $.
\end{nul}

\begin{nul}
	These maps give rise to a commutative diagram with exact diagonals
	\begin{equation*}
		\begin{tikzcd}[column sep={10ex,between origins}, row sep={8ex,between origins}]
			& \H^{0}(M;\RR/\ZZ) \arrow[rr, "-\Bock"] \arrow[dr, hook] & & \H^1(M;\ZZ) \arrow[dr] & \\
			\HdR^{0}(M) \arrow[ur] \arrow[dr, hook] & & \Cinf(M,\RR/\ZZ) \arrow[ur, ->>, "\uppi_0" description] \arrow[dr, "\curv" description] & & \HdR^1(M) \\
			& \Omega^{0}(M) \arrow[rr, "\d"'] \arrow[ur, "\iota" description] & & \Omegacl^1(M) \arrow[ur] & \phantom{\HdR^1(M)} \period
		\end{tikzcd}
	\end{equation*}
	The diagonals become short exact sequences if we replace $ \Omega^0(M) $ by $ \Omega^0(M)/\Omegacl^0(M)_{\ZZ} $ and $ \Omegacl^1(M) $ by $ \Omegacl^1(M)_{\ZZ} $:
	\begin{equation*}
		\begin{tikzcd}[column sep={10ex,between origins}, row sep={8ex,between origins}]
			0 \arrow[dr] & & & & 0 \\
			& \H^{0}(M;\RR/\ZZ) \arrow[rr, "-\Bock"] \arrow[dr, hook] & & \H^1(M;\ZZ) \arrow[dr] \arrow[ur]  & \\
			\HdR^{0}(M) \arrow[ur] \arrow[dr, hook] & & \Cinf(M,\RR/\ZZ) \arrow[ur, ->>, "\uppi_0" description] \arrow[dr, "\curv" description, ->>] & & \HdR^1(M) \\
			& \Omega^0(M)/\Omegacl^0(M)_{\ZZ} \arrow[rr, "\d"'] \arrow[ur, >->, "\iota" description] & & \Omegacl^1(M)_{\ZZ} \arrow[ur] \arrow[dr] & \\
			0 \arrow[ur] & & & & 0 \period
		\end{tikzcd}
	\end{equation*}
\end{nul}

\begin{nul}
	The takeaway is that in \Cref{qst:origin}, we should really replace $ \Omega^{k-1}(M)/\im(d) $ by $ \Omega^{k-1}(M)/\Omegacl^{k-1}(M)_{\ZZ} $ and $ \Omegacl^k(M) $ by $ \Omegacl^0(M)_{\ZZ} $ and ask for the diagonal sequences to be short exact.
\end{nul}



% \begin{attempt}[for $ k = 2 $]
% 	Let $ M $ be a manifold and consider the group $ \Buncon(M) $ of isomorphism classes of principal $ \mathup{U}(1) $-bundles with connection under tensor product.
% \end{attempt}


%-------------------------------------------------------------------%
%-------------------------------------------------------------------%
%  Differential characters                                          %
%-------------------------------------------------------------------%
%-------------------------------------------------------------------%

\subsection{Differential characters}

We now present a unified approach to defining the ``differential cohomology'' groups $ \Hhat^{*}(M;\ZZ) $ due to Cheeger--Simons \cite{MR827262}.
We follow Bär and Becker's exposition on \textit{differential characters} \cite[Part I, \S5]{MR3237728}.

\begin{notation}
	Let $ M $ be a manifold and $ i \geq 0 $ an integer.
	We write $ \Csm_i(M;\ZZ) $ for the abelian group of smooth (integer-valued) chains on $ M $.
	We write $ \Zsm_i(M;\ZZ) \subset \Csm_i(M;\ZZ) $ for the subgroup of smooth cycles.
\end{notation}

\begin{definition}[{(Cheeger--Simons \cite[\S1]{MR827262})}]
	Let $ k \geq 1 $ be an integer and $ M $ a manifold.
	A \emph{degree $ k $ differential character} on $ M $ is a homomorphism $ \chi \colon \fromto{\Zsm_{k-1}(M;\ZZ)}{\RR/\ZZ} $ such that there exists a $ k $-form $ \omega(\chi) \in \Omega^k(M) $ with the property that for every $ c \in \Csm_k(M;\ZZ) $, 
	\begin{equation*}
		\chi(\partial c) = \int_c \omega(\chi) \quad \text{mod } \ZZ \period
	\end{equation*} 
	We write
	\begin{equation*}
		\Hhat^k(M;\ZZ) \subset \Hom_{\ZZ}(\Zsm_{k-1}(M;\ZZ),\RR/\ZZ)
	\end{equation*}
	for the abelian group of degree $ k $ differential characters on $ M $.
	
	It follows that $ \omega(\chi) $ is unique and closed.
	Moreover, $ \omega(\chi) $ has integral periods.
	The form $ \omega(\chi) $ is called the \emph{curvature} of $ \chi $, and we have a curvature map
	\begin{align*}
		\curv \colon \Hhat^k(M;\ZZ) &\to \Omega^k(M) \\
		\chi &\mapsto \omega(\chi)
	\end{align*} 
	with image $ \Omegacl^k(M)_{\ZZ} $ those closed $ k $-forms with integral periods.
\end{definition}

\begin{warning}
	The indexing convention used here is off by $ 1 $ from the indexing convention in \cite[\S1]{MR827262}.
	However, this indexing convention is what was later adopted by Simons--Sullivan \cite[\S1]{MR2365651}.
	See also \Cref{rem:Delignecohomology} for why $ k $ is the right index rather than $ k - 1 $.
\end{warning}

\begin{remark}
	When $ k = 0 $, the diagram \eqref{diag:diffcoh} is quite degenerate, and it will be convenient to define $ \Hhat^0(M;\ZZ) \colonequals \H^0(M;\ZZ) $.
\end{remark}

Now let us construct maps to fill in the ``differential cohomology'' diagram \eqref{diag:diffcoh}.

\begin{construction}\label{cons:charclassmap}
	There is a \emph{characteristic class} map $ \cc \colon \fromto{\Hhat^k(M;\ZZ)}{\H^k(M;\ZZ)} $ defined as follows.
	Since $ \Zsm_{k-1}(M;\ZZ) $ is a free $ \ZZ $-module and the quotient map $ \surjto{\RR}{\RR/\ZZ} $ is an epimorphism, any homomorphism $ \chi \colon \fromto{\Zsm_{k-1}(M;\ZZ)}{\RR/\ZZ} $ lifts to a homomorphism
	\begin{equation*}
		\chitilde \colon \fromto{\Zsm_{k-1}(M;\ZZ)}{\RR} \period
	\end{equation*}
	Now define a homomorphism $ I(\chitilde) \colon \fromto{\Csm_k(M;\ZZ)}{\ZZ} $ by the assignment
	\begin{equation*}
		c \mapsto - \chitilde(\partial c) + \int_{c} \curv(\chi)  \period
	\end{equation*}

	Since $ \curv(\chi) $ is closed, $ I(\chitilde) $ defines a cocycle.
	Moreover, $ I(\chitilde) $ takes integral values, and the cohomology class $ [I(\chitilde)] \in \H^k(M;\ZZ) $ does not depend on the choice of lift $ \chitilde $.
	We define the characteristic class map $ \cc $ by the assignment
	\begin{align*}
		\cc \colon \Hhat^k(M;\ZZ) &\to \H^k(M;\ZZ) \\
		\chi &\mapsto [I(\chitilde)] \phantom{;\ZZ)} \period
	\end{align*} 
\end{construction}

\begin{warning}
	Simons and Sullivan \cite{MR2365651} denote the characteristic class map $ \cc $ by `$ \ch $'.
\end{warning}

\begin{construction}
	Consider the universal coefficient sequence
	\begin{equation*}
		\begin{tikzcd}[sep=2em]
			0 \arrow[r] & \Ext_{\ZZ}^1(\H_{i-1}(M;\ZZ),\RR/\ZZ) \arrow[r] & \H^i(M;\RR/\ZZ) \arrow[r, "\ang{-,-}"] & \Hom_{\ZZ}(\H_i(M;\ZZ),\RR/\ZZ) \arrow[r] & 0 \comma 
		\end{tikzcd}
	\end{equation*}
	where the morphism $ \ang{-,-} $ is given by sending the class of a cocycle $ u $ to the homomorphism
	\begin{align*}
		\ang{u,-} \colon \H_i(M;\ZZ) &\to \RR/\ZZ \\
		[z] &\mapsto u(z) \period
	\end{align*}
	Since the circle $ \RR/\ZZ $ is an injective $ \ZZ $-module, for any $ \ZZ $-module $ A $ and integer $ j > 0 $, we have \smash{$ \Ext_{\ZZ}^j(A,\RR/\ZZ) = 0 $}.
	In particular, $ \ang{-,-} $ is an isomorphism.

	Setting $ i = k - 1 $, precomposition with the quotient map $ \surjto{\Zsm_{k-1}(M;\ZZ)}{\H_{k-1}(M;\ZZ)} $ defines an injection
	\begin{equation*}
		\begin{tikzcd}[sep=1.5em]
			\H^i(M;\RR/\ZZ) \arrow[r, "\sim"{yshift=-0.1em}] & \Hom_{\ZZ}(\H_i(M;\ZZ),\RR/\ZZ) \arrow[r, hook] & \Hom_{\ZZ}(\Zsm_{k-1}(M;\ZZ),\RR/\ZZ) \period 
		\end{tikzcd}
	\end{equation*}
	It follows from the definitions that this factors through $ \Hhat^k(M;\ZZ) $.
	We simply denote this composite by $ \ang{-,-} \colon \incto{\H^{k-1}(M;\RR/\ZZ)}{\Hhat^k(M;\ZZ)} $.
\end{construction}

\begin{construction}
	Define a map $ \iota \colon \fromto{\Omega^{k-1}(M)}{\Hhat^k(M;\ZZ)} $ by setting
	\begin{equation*}
		\iota(\omega)(z) \colonequals \exp\paren{\textstyle 2\pi i \int_z \omega} 
	\end{equation*}
	for every smooth $ (k-1) $-cycle $ z $.
	By Stokes' Theorem, we see that $ \curv(\iota(\omega)) = \d\omega $.

	We have an $ \RR $-valued lift of $ \iota(\omega) $ given by setting
	\begin{equation*}
		\iotatilde(\omega)(z) \colonequals \int_z \omega
	\end{equation*}
	for every smooth $ (k-1) $-cycle $ z $.
	So by Stokes' Theorem we have
	\begin{align*}
		I(\iotatilde(\omega))(c) &= -\,\iotatilde(\omega)(\partial c) + \int_c \curv(\iota(\omega)) \\ 
		&= -\int_{\partial c} \omega + \int_{c} \d\omega = 0
	\end{align*}
	for every smooth $ k $-chain $ c $.
	Hence $ \cc \of \iota = 0 $.

	We see that $ \iota \colon \fromto{\Omega^{k-1}(M)}{\Hhat^k(M;\ZZ)} $ has kernel those closed forms $ \omega $ such that $ \int_{z} \omega $ is an integer for all $ z \in \Zsm_{k-1}(M;\ZZ) $. 
	That is,
	\begin{equation*}
		\ker(\iota) = \Omegacl^{k-1}(M)_{\ZZ}
	\end{equation*}
	is the group of closed $ (k-1) $-forms with integral periods.
	Hence $ \iota $ descends to an injection
	\begin{equation*}
		\iota \colon \into{\Omega^{k-1}(M)/\Omegacl^{k-1}(M)_{\ZZ}}{\Hhat^{k}(-;\ZZ)} \period
	\end{equation*}
\end{construction}

%-------------------------------------------------------------------%
%-------------------------------------------------------------------%
%  The differential cohomology hexagon                              %
%-------------------------------------------------------------------%
%-------------------------------------------------------------------%

\subsection{The differential cohomology hexagon}


\begin{notation}
	Write $ \Man $ for the category of smooth manifolds and $ \GrAb $ for the category of graded abelian groups. 
\end{notation}

\begin{theorem}[{(Simons--Sullivan \cite[Theorem 1.1]{MR2365651})}]\label{thm:SimonsSullivanunique}
	There is an essentially unique functor
	\begin{equation*}
		\Hhat^{*}(-;\ZZ) \colon \fromto{\Manop}{\GrAb}
	\end{equation*}
	equipped with natural transformations
	\begin{enumerate}[{\upshape (\ref*{thm:SimonsSullivanunique}.1)}]
		\item $ \ang{-,-} \colon \fromto{\H^{*-1}(-;\RR/\ZZ)}{\Hhat^{*}(-;\ZZ)} $, 

		\item $ \iota \colon \fromto{\Omega^{*-1}(M)/\Omegacl^{*-1}(M)_{\ZZ}}{\Hhat^{*}(-;\ZZ)} $,

	 	\item $ \cc \colon \fromto{\Hhat^{*}(-;\ZZ)}{\H^{*}(-;\ZZ)} $, 

	 	\item and $ \curv \colon \fromto{\Hhat^{*}(-;\ZZ)}{\Omegacl^{*}(-)_{\ZZ}} $
	\end{enumerate}
	filling in the ``differential cohomology hexagon''
	\begin{equation*}
		\begin{tikzcd}[column sep={10ex,between origins}, row sep={8ex,between origins}]
			0 \arrow[dr] & & & & 0 \\
			& \H^{*-1}(M;\RR/\ZZ) \arrow[rr, "-\Bock"] \arrow[dr] & & \H^*(M;\ZZ) \arrow[dr] \arrow[ur] & \\
			\HdR^{*-1}(M) \arrow[ur] \arrow[dr] & & \Hhat^*(M;\ZZ) \arrow[ur] \arrow[dr] & & \HdR^*(M) \\
			& \frac{\Omega^{*-1}(M)}{\Omegacl^{*-1}(M)_{\ZZ}} \arrow[rr, "\d"'] \arrow[ur] & & \Omegacl^*(M)_{\ZZ} \arrow[ur] \arrow[dr] & \\
			0 \arrow[ur] & & & & 0 
		\end{tikzcd}
	\end{equation*}
	so that the diagonal sequences are exact.
\end{theorem}

\noindent Any functor $ \Hhat^{*}(-;\ZZ) \colon \fromto{\Manop}{\GrAb} $ satisfying these properties is called \emph{ordinary differential cohomology}. 

\begin{remark}[(Deligne's model)]\label{rem:Delignecohomology}
	Motivated by Deligne cohomology in Hodge theory \cites[\S2.2]{MR498551}[\S12.3]{MR2451566}, we can consider the smooth version of the Deligne complex on a manifold $ M $.
	Write $ \ZZ(k) $ for the complex of sheaves on $ M $
	\begin{equation*}
		\begin{tikzcd}[sep=1.5em]
			0 \arrow[r] & \ZZ \arrow[r, hook] & \Omega^0 \arrow[r, "\d"] & \Omega^1 \arrow[r, "\d"] & \cdots \arrow[r] & \Omega^{k-1} \arrow[r, ""] & 0 \comma
		\end{tikzcd}
	\end{equation*}
	where $ \Omega^i $ is in degree $ i + 1 $.
	The \emph{$ k $-th smooth Deligne cohomology group} of $ M $ is the sheaf cohomology (i.e., hypercohomology) group $ \H^k(M;\ZZ(k)) $.
	We will see later that smooth Deligne cohomology agrees with ordinary differential cohomology (see \Cref{lem:Delignemodel}).
\end{remark}

\begin{questions}\label{qst:extensions}
	There are a number of questions that naturally arise
	\begin{enumerate}[(\ref*{qst:extensions}.1)]
		\item Is there differential $ \K $-theory?

		Yes! Hopkins--Singer \cite{HopkinsSinger} define differential $\K$-theory. 
		Simons--Sullivan \cites{MR2732065}{MR3220448} tell a similar story, and define differential $ \K $-theory in terms of vector bundles with connection.
		We study this in \cref{subsec:diffKtheory}. 

		\item What about differential [favorite cohomology theory]? 

		Also yes, but the theory is more complicated. 
		The fundamental observation is that everything we've considered comes from a sheaf of abelian groups or chain complexes (which we regard as spectra) on the category of \textit{all} smooth manifolds.
		We begin to set up this theory in \Cref{sec:basicsetup}.

		Moreover, the \category $ \Sh(\Man;\Sp) $ of sheaves of spectra on the category of manifolds has rich structure that gives rise to a ``differential cohomology hexagon''
		associated to every object.
		We study this in \Cref{sec:stable}.
	\end{enumerate}
\end{questions}

\begin{remark}
	The category $ \Sh(\Man;\Set) $ is really the right place for moduli spaces of manifolds to live, and Fréchet manifolds embed as a full subcategory of $ \Sh(\Man;\Set) $.
	See \cref{subsec:infdimMan}.
\end{remark}

There are many applications of this perspective on differential cohomology that we study throughout this book.
See, in particular, \Cref{part:applications}.
