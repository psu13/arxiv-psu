%!TEX root = ../diffcoh.tex

\section{On-diagonal Differential Characteristic Classes}\label{DifferentialCharacteristicClasses}
\textit{by Arun Debray}

In \cref{ChernWeilTheory}, we constructed Chern, Pontryagin, and Euler classes of vector bundles in the
de Rham cohomology of manifolds $M$. The catalyst for this chapter is the observation that these classes are always in the image of the map $\H^*(M;\ZZ)\to\HdR^*(M)$. That is, we have the diagram
\begin{equation}
% https://q.uiver.app/?q=WzAsMyxbMCwwLCJcXEheayhNO1xcWlopIl0sWzEsMSwiXFxIZFJeayhNKSJdLFswLDIsIlxcT21lZ2FjbF5rKE0pIl0sWzAsMSwiY15cXFpaKFYpXFxtYXBzdG8gYyhWKSIsMCx7ImxhYmVsX3Bvc2l0aW9uIjo5MH1dLFsyLDEsIlAoRl9cXG5hYmxhKVxcbWFwc3RvIGMoVikiLDIseyJsYWJlbF9wb3NpdGlvbiI6OTB9XV0=
\begin{tikzcd}
	{\H^k(M;\ZZ)} \\
	& {\HdR^k(M)}, \\
	{\Omegacl^k(M)}
	\arrow["{c^\ZZ(V)\mapsto c(V)}"{pos=0.5}, from=1-1, to=2-2]
	\arrow["{P(\curvature{A})\mapsto c(V)}"'{pos=0.5}, from=3-1, to=2-2]
\end{tikzcd}
\end{equation}
which looks suspiciously like two sides of the differential cohomology hexagon. We therefore ask whether it is
possible to fill in the middle: can one choose a class $\chat\in \Hhat^*(M; \ZZ)$ whose image under the curvature
map is the Chern--Weil form, and whose image under the characteristic class map is the lift of the characteristic
class to $\ZZ$-valued cohomology?

The answer is yes, and in fact this was one of Cheeger--Simons' original applications of their theory of
differential characters \cite[\S 2]{MR827262}. In this section, we will follow the proof of
Bunke--Nikolaus--Völkl \cite[\S 5.2]{MR3462099}, who work universally on the classifying stack $\BnablaG$ from
\Cref{ex:BunG,ntn:BbulletGBnablaG}. After that, we review our examples, constructing differential lifts of Chern,
Pontryagin, and Euler classes, and discuss how the Whitney sum formula behaves in the differential context.
Finally, we use the differential refinement of Chern--Weil theory to give a clean general description of secondary
invariants. These invariants in particular include Chern--Simons invariants, which we will use again and again in
\cref{part:applications}.

%-------------------------------------------------------------------%
%-------------------------------------------------------------------%
%  Lifting the Chern–Weil map to differential cohomolog             %
%-------------------------------------------------------------------%
%-------------------------------------------------------------------%

\subsection{Lifting the Chern--Weil map to differential cohomology}

Begin with a Lie group $G$ and an invariant polynomial $P\in I^\bullet(G)$. From $P$, the Chern--Weil machine
constructs a closed form $P(\Omega)\in \Omegacl^\bullet(\BnablaG)$.\footnote{We end up with a form on the
universal object $\BnablaG$ because Chern--Weil forms are natural in the connection. For more information,
see \cite[(7.21)]{FreedHopkins} and the surrounding text.}

We next need to choose an integer lift $c^\ZZ$ of $c$. There is both an existence and a uniqueness question: an
arbitrary cohomology class need not be in the lattice $\operatorname{Im}(\H^k(\BG;\ZZ)\to \H^k(\BG;\RR))$, and if
there is torsion in $\H^k(\BG;\ZZ)$, the lift is not unique.\footnote{For example, there is torsion in
$\H^*(\BO_n; \ZZ)$ and $\H^*(\BSO_n;\ZZ)$ \cite{Bro82}.}

\begin{thm}[{(Cheeger--Simons \cite[Theorem 2.2]{MR827262}, Bunke--Nikolaus--Völkl \cite[\S 5.2]{MR3462099})}]
\label{differential_CW_lift}
Given this data, there is a unique natural class $\chat\in\Hhat^k(\BnablaG; \ZZ)$ whose image under the
characteristic class map is $c^\ZZ$ and whose image under the curvature map is $P(\Omega)$.
\end{thm}

\noindent Naturality is with respect to $G$, keeping track of the data $c^\ZZ$.

\begin{proof}
	The invariant polynomial $P$ gives us a map of sheaves of sets on $\Man$:
	\begin{equation}
	\label{first_sheaf_set}
	\begin{tikzcd}
		\Omega^1\otimes\g && \Omega^2\otimes\g & \Cyc^{2p}(\Omega),
		\arrow["{\omega\mapsto \d\omega + [\omega, \omega]}", from=1-1, to=1-3]
		\arrow["P", from=1-3, to=1-4]
	\end{tikzcd}
	\end{equation}
	where $\Cyc$ is the sheaf of differential cycles from \cref{differential_cycles}.
	\index[terminology]{differential cycles}
	If $\yo(G)$ denotes the sheaf of groups associated to $G$ by the Yoneda embedding, then the maps
	in~\eqref{first_sheaf_set} are $\yo(G)$-equivariant, where $\Cyc^{2p}(\Omega)$ is given the trivial
	$\yo(G)$-action. Take the groupoid quotient
	\begin{equation}
	\begin{tikzcd}
		(\Omega^1\otimes\g) \modmod \yo(G) & (\Omega^2\otimes\g) \modmod \yo(G) & \Cyc^{2p}(\Omega)\modmod \yo(G),
		\arrow[from=1-1, to=1-2]
		\arrow[from=1-2, to=1-3]
	\end{tikzcd}
	\end{equation}
	then take the nerve and sheafify, giving $\BnablaG$ and $\BbulletG$ as we discussed in
	\Cref{ex:BunG,ntn:BbulletGBnablaG}:
	\begin{equation}
	\label{BnablaBbullet}
		\BnablaG\longrightarrow \BbulletG\times i(\Cyc^{2p}(\Omega))\comma
	\end{equation}
	where $i\colon\Set\to\sSet$ builds the constant simplicial set out of a set. There is an equivalence of simplicial
	sheaves
	\begin{equation}
	\label{what_is_iz2p}
		i(\Cyc^{2p}(\Omega)) \isomorphism \Omega^\infty \EM(\Cyc^{2p}(\Omega)[0])\comma
	\end{equation}
	where $\EM\colon \D(\ZZ) \to\Sp$ is the Eilenberg--Mac Lane functor and $[0]$ means we regard the sheaf
	$\Cyc^{2p}(\Omega)$ of abelian groups as a sheaf of complexes concentrated in degree zero.

	Take~\eqref{BnablaBbullet}, compose with the projection onto $i(\Cyc^{2p}(\Omega))$, and apply~\eqref{what_is_iz2p}
	to obtain a map $\varphi_P\colon \BnablaG\to \Omega^\infty \EM(Z^{2p}(\Omega)[0])$. Let $\psi_P\colon
	\Sigma^\infty_+ \BnablaG\to \EM(\Cyc^{2p}(\Omega)[0])$ be the image of $\varphi_P$ under the $(\Sigma^\infty,
	\Omega^\infty)$ adjunction.

	Now apply the homotopification functor $\Lhi\colon\Sh(\Man, \Spc)\to\Sh_\RR(\Man, \Spc)$ from
	\cref{homotopification_definition} to $\psi_P$. We claim this produces a map
	\begin{equation}
		\Gamma^*(\Sigma_+^\infty\BG)\overset{\chi_P}{\longrightarrow} \Gamma^*(\HRR[2p])\period
	\end{equation}
	To see this, use the identifications $\Lhi\simeq\Gammaupperstar\Gammalowershriek$
	(\cref{homotopification_definition}) and $\Gammalowershriek(E)\simeq \real{E(\Deltaalgdot)}$
	(\cref{cor:Gammalowershriek}). The identification
	\begin{equation*}
		\Gammalowershriek(\EM(\Cyc^{2p}(\Omega)[0]))\simeq \HRR[2p]
	\end{equation*}
	is a dressed-up version of the de Rham theorem, and the equivalence
	\begin{equation*}
		\Gammalowershriek(\Sigma_+^\infty\BnablaG) \simeq \Sigma_+^\infty\BG
	\end{equation*}
	uses contractibility of the space of connections on a principal $G$-bundle on a space.

	Next we have to identify $\chi_P$. On cohomology, the Chern--Weil construction uses $P$ to naturally assign a
	degree-$2p$ real cohomology class to a principal $G$-bundle; this soups up to a map
	$\xi_P\colon\Sigma_+^\infty\BG\to \HRR[2p]$. Looking back at the definition of $\varphi_P$, we see that
	$\Gammalowershriek(\psi_P) = \xi_P$; therefore $\chi_P = \Gammaupperstar(\xi_P)$.

	Here is where $c^\ZZ$ comes in. It is data of a lift
	\begin{equation}
	\begin{gathered}
	\begin{tikzcd}
		& \HZZ[2p] \\
		\Sigma_+^\infty\BG & \HRR[2p],
		\arrow[from=1-2, to=2-2]
		\arrow["\xi_P"', from=2-1, to=2-2]
		\arrow["c^\ZZ", from=2-1, to=1-2]
	\end{tikzcd}
	\end{gathered}
	\end{equation}
	giving us a diagram
	\begin{equation}
	\begin{gathered}
	% https://q.uiver.app/?q=WzAsNSxbMCwwLCJcXFNpZ21hXyteXFxpbmZ0eSBCRyJdLFswLDEsIlxcR2FtbWFeKihcXFNpZ21hXyteXFxpbmZ0eSBCRykiXSxbMSwxLCJcXEdhbW1hXiooSFpbMnBdKSJdLFsyLDEsIlxcR2FtbWFeKihIUlsycF0pIl0sWzIsMCwiSChaXnsycH0oXFxPbWVnYSlbMF0pIl0sWzQsMywiPyJdLFswLDEsIj8iLDJdLFsxLDIsIlxcR2FtbWFeKihjXlopIl0sWzIsM10sWzEsMywiXFxHYW1tYV4qKFxceGlfUCkiLDIseyJjdXJ2ZSI6M31dLFswLDQsIlxccHNpX1AiLDAseyJjdXJ2ZSI6LTN9XV0=
	\begin{tikzcd}
		{\Sigma_+^\infty\BnablaG} && {\EM(\Cyc^{2p}(\Omega)[0])} \\
		{\Gamma^*(\Sigma_+^\infty\BG)} & {\Gammaupperstar(\HZZ[2p])} & {\Gammaupperstar(\HRR[2p]).}
		\arrow[from=1-3, to=2-3]
		\arrow[from=1-1, to=2-1]
		\arrow["{\Gammaupperstar(c^\ZZ)}", from=2-1, to=2-2]
		\arrow[from=2-2, to=2-3]
		\arrow["{\Gammaupperstar(\xi_P)}"', curve={height=18pt}, from=2-1, to=2-3]
		\arrow["{\psi_P}", curve={height=-18pt}, from=1-1, to=1-3]
	\end{tikzcd}
	\end{gathered}
	\end{equation}
	The vertical arrows are both of the form $F\to\Lhi(F)$, and are unit maps for the adjunction $(\Lhi,
	\text{inclusion})$ from \cref{nul:Gammaadjunctions}.

	The map from the upper left to the lower right factors through the pullback
	\begin{equation}
	\begin{gathered}
	% https://q.uiver.app/?q=WzAsNixbMCwwLCJcXFNpZ21hXyteXFxpbmZ0eSBCRyJdLFswLDEsIlxcR2FtbWFeKihcXFNpZ21hXyteXFxpbmZ0eSBCRykiXSxbMSwxLCJcXEdhbW1hXiooSFpbMnBdKSJdLFsyLDEsIlxcR2FtbWFeKihIUlsycF0pIl0sWzIsMCwiSChaXnsycH0oXFxPbWVnYSlbMF0pIl0sWzEsMCwiSFooMnApIl0sWzQsMywiPyJdLFswLDEsIj8iLDJdLFsxLDIsIlxcR2FtbWFeKihjXlopIl0sWzIsM10sWzEsMywiXFxHYW1tYV4qKFxceGlfUCkiLDIseyJjdXJ2ZSI6M31dLFswLDQsIlxccHNpX1AiLDAseyJjdXJ2ZSI6LTN9XSxbNSw0XSxbNSwyXSxbMCw1LCJjIl1d
	\begin{tikzcd}
		{\Sigma_+^\infty\BnablaG} & {\HZZhat(2p)} & {\EM(\Cyc^{2p}(\Omega)[0])} \\
		{\Gammaupperstar(\Sigma_+^\infty\BG)} & {\Gammaupperstar(\HZZ[2p])} & {\Gammaupperstar(\HRR[2p]),}
		\arrow[from=1-3, to=2-3]
		\arrow[from=1-1, to=2-1]
		\arrow["{\Gammaupperstar(c^\ZZ)}", from=2-1, to=2-2]
		\arrow[from=2-2, to=2-3]
		\arrow["{\Gammaupperstar(\xi_P)}"', curve={height=18pt}, from=2-1, to=2-3]
		\arrow["{\psi_P}", curve={height=-18pt}, from=1-1, to=1-3]
		\arrow[from=1-2, to=1-3]
		\arrow[from=1-2, to=2-2]
		\arrow["\chat"', from=1-1, to=1-2]
		\arrow[from=1-2, to=2-3, phantom, "\lrcorner" , very near start, color=black]
	\end{tikzcd}
	\end{gathered}
	\end{equation}
	and $\chat$ is the desired differential refinement.
\end{proof}

\begin{remark}
	Cheeger--Simons' original proof did not use this language: they did not have $\BnablaG$ available. Instead, they
	use \emph{$n$-classifying spaces} $\beta_\nabla^{(n)} G$. These are spaces such that all connections on principal
	$G$-bundles $P\to M$ pull back from $\beta_\nabla^{(n)} G$, provided $\dim M < n$, and the pullback need not be
	unique. Narasimhan--Ramanan \cite{NR61, NR63} proved $n$-classifying spaces exist for all $n$ and $G$, provided
	$\pi_0(G)$ is finite.
\end{remark}

\begin{example}[(Differential Chern classes)]
\label{differential_Chern}
\index[terminology]{Chern class!on-diagonal differential refinement}
\index[terminology]{differential Chern class}
Borel \cite[\S 29]{Bor53} shows that
\begin{equation*}
	\H^*(\BU_n;\ZZ)\cong \ZZ[c_1,\dotsc, c_n] \comma
\end{equation*}
so integer lifts are unique, and using Grothendieck's axioms, one can show that the images of these Chern classes in de Rham
cohomology are equal to the Chern classes we constructed in \S\ref{chern_const}. Therefore we obtain
\emph{on-diagonal differential Chern classes} $\chat_k(P, \conn)\in\Hhat^{2k}(M; \ZZ)$ associated to principal
$\Uup_n$-bundles $P\to M$ with connection $\conn$. 
See \Cref{rmk-OnOffDiagonal} for a note on terminology.

Several authors construct differential Chern classes by other methods, including
Brylinski--McLaughlin \cite{BM96},
Berthomieu \cite{Ber10},
Bunke \cite{Bun10,Bunke:notes}, and
Ho \cite{Ho15}. Schreiber \cite{Urs} constructs $\chat_1$.
\index[notation]{ccheckk@$\chat_k$}

\end{example}
\begin{example}[(Differential Pontryagin classes)]
\label{differential_Pontryagin}
\index[terminology]{Pontryagin class!on-diagonal differential refinement}
\index[terminology]{differential Pontryagin class}
	Brown \cite[Theorem 1.6]{Bro82} shows there is torsion in $\H^*(\BO_n; \ZZ)$, so choosing $p_k^\ZZ$ is not
	automatic. Let $c\colon\BO_n\to\BU_n$ be the complexification map, and for a principal
	$\Or_n$-bundle $P\to M$ define
	\begin{equation}
		p_k(P) \colonequals (-1)^k c_{2k}(c(P))\in \H^{2k}(M; \ZZ)\period
	\end{equation}
	The images of these classes in de Rham cohomology are equal to the Pontryagin classes we defined in
	\S\ref{pontrjagin},
	so \cref{differential_CW_lift} produces for us
	\emph{on-diagonal differential Pontryagin classes} $\phat_k(P,  \conn)\in\Hhat^{4k}(M;\ZZ)$ associated to
	principal $\Or_n$-bundles with connection $\conn$.

	Brylinski--McLaughlin \cite{BM96} and Grady--Sati \cite[Proposition 3.6]{GS21} construct $\phat_k$ in a
	different way.
	\index[notation]{pcheck@$\phat_k$}
\end{example}

\begin{example}[(Differential Euler classes)]
	\index[terminology]{Euler class!on-diagonal differential refinement}
	\index[terminology]{differential Euler class}
	Brown \cite[Theorem 1.6]{Bro82} shows that there is also torsion in $\H^*(\BSO_n; \ZZ)$, so we must
	choose a lift $e^\ZZ$ of the Euler class we constructed in \cref{euler_const}. There, we defined $e$ only for $n$
	even; for odd $n$, we set $e\colonequals 0$.

	Let $V\to\BSO_n$ denote the tautological bundle. Since $V$ is oriented, it has a $\ZZ$-cohomology
	Thom class $\tau(E)\in \widetilde{\H}{}^n(V, V\setminus 0; \ZZ)$. We let $e^\ZZ$ be the pullback of $\tau(E)$ by
	the zero section of $V$. The image of this class is $e$, so the class defined by the Pfaffian when $n$ is even, and
	$0$ when $n$ is odd. For all $n$, however, $e^\ZZ\ne 0$; it is $2$-torsion when $n$ is odd.

	Therefore we obtain a \emph{on-diagonal differential Euler class} $\ehat(P, \conn)\in\Hhat^n(M; \ZZ)$
	associated to a principal $\SO_n$-bundle with connection $\conn$, and it can be nonzero for all $n$, not just
	even $n$.

	Brylinski--McLaughlin \cite{BM96} and Bunke \cite[Example 3.85]{Bunke:notes} construct $\ehat$ in a different way.
\end{example}

\begin{remark}[(From principal bundles to vector bundles: an important nuance)]
\label{metric_differential_cc}
	We would like to use the characteristic classes we just constructed to define differential lifts of characteristic
	classes of vector bundles with connection. The way this usually works for characteristic classes is that a vector
	bundle has an associated principal $G$-bundle, and we consider characteristic classes for $G$. For example, a
	rank-$n$ complex vector bundle has a principal $\GL_n( \CC)$-bundle of frames. The maximal compact of $\GL_n( \CC)$
	is $\Uup_n$, so inclusion $\Uup_n\to\GL_n( \CC)$ induces a homotopy equivalence of classifying spaces, which
	means characteristic classes of principal $\Uup_n$-bundles give you characteristic classes of principal $\GL_n(
	\CC)$-bundles give you characteristic classes of complex vector bundles. Both of these steps are necessary: the
	Chern--Weil map is only guaranteed to be an isomorphism for compact groups, and without additional structure such
	as a metric, the structure group of a vector bundle is noncompact.

	In differential cohomology, this becomes a stumbling block: homotopy equivalences do not always induce isomorphisms
	on differential cohomology, so what we learn about principal $\Uup_n$-bundles does not necessarily help us with
	complex vector bundles. Therefore a priori, the differential characteristic classes we defined above only make
	sense for vector bundles with a metric and a compatible connection as in \S\ref{subsubsec:connections_for_vbs},
	because these correspond to connections on principal $\Uup_n$-bundles, rather than principal $\GL_n(
	\CC)$-bundles.
	\index[terminology]{connection!compatibility with the metric}

	In addition to complex vector bundles and $\Uup_n$ versus $\GL_n( \CC)$, which is about differential Chern
	classes, there are two more cases to worry about.
	\begin{enumerate}[(1)]
		\item Real vector bundles and $\Or_n$ and $\GL_n( \RR)$, and differential Pontryagin classes.
		
		\item Oriented real vector bundles, $\SO_n$, and $\GL_n( \RR)_0$ (i.e.\ the connected component of $\GL_n(
		\RR)$ containing the identity), for the differential Euler class.
	\end{enumerate}
	First, Chern classes. For $\GL_n( \CC)$ the Chern--Weil map is not an isomorphism, but it is surjective \cites[\S
	4]{MR827262}[\S 11.8.1]{Pro07}, so differential Chern classes can be defined in the absence of a metric.

	Next, Pontryagin classes. The construction in \cref{differential_Pontryagin} implies differential Pontryagin
	classes of $V,  \conn$ are equal to differential Chern classes of $V\otimes\CC$ with connection induced from
	$\conn$, so differential Pontryagin classes can be defined in the absence of a metric.

	But Euler classes are different! If $A\in\GL_n( \RR)$ and $X\in\so_n$, then
	\begin{equation}
		\mathrm{pf}(AXA^{-1}) = \det(A)\mathrm{pf}(X)\comma
	\end{equation}
	so the Pfaffian is not $\GL_n( \RR)_0$-invariant. Therefore the differential Euler class requires an oriented
	vector bundle, a Euclidean metric, and a compatible connection.
	\index[terminology]{differential Euler class!metric-compatibility requirement}
\end{remark}

We will use these classes in a few different ways in \cref{applications_part}, including obstructing conformal
immersions in \cref{conformal_immersions} and constructing non-topological invertible field theories in
\cref{invertible_field_theories}. Cheeger--Simons \cite{Cheeger-Simons} discuss some additional applications,
including characteristic classes associated to foliations and a geometric refinement of the Atiyah--Singer index
theorem. There are also differential refinements of the Todd genus, $\Ahat$-genus \cite[Definition
3.9]{GS21}, and so forth.
\index[terminology]{Todd genus!differential refinement}
\index[terminology]{Ahat genus@$\Ahat$-genus!differential refinement}

%-------------------------------------------------------------------%
%-------------------------------------------------------------------%
%  The Whitney sum formula for differential characteristic classes  %
%-------------------------------------------------------------------%
%-------------------------------------------------------------------%

\subsection{The Whitney sum formula for on-diagonal differential characteristic classes}

The Whitney sum formula expresses the Chern, Pontryagin, and Euler classes of a direct sum $E\oplus F$ of vector
bundles in terms of the respective characteristic classes of $E$ and of $F$.
\index[terminology]{Whitney sum formula!in ordinary cohomology}
Let $c \colonequals 1 + c_1 + c_2 + \cdots$ denote the total Chern class\index[terminology]{total Chern class} and $p \colonequals 1 +
p_1 + p_2 + \cdots$ denote the total Pontryagin class.\footnote{Though these appear to be infinite sums, they are
finite when evaluated on any vector bundle, because $c_k(E)$ and $p_k(E)$ vanish when $k >
\mathrm{rank}(E)$.}\index[terminology]{total Pontryagin class} For complex vector bundles $E,F\to X$,
\begin{subequations}
\label{ordinary_Whitney_sum}
\begin{equation}
\begin{aligned}
	c(E\oplus F) &= c(E)c(F)\\
	c_k(E\oplus F) &= \sum_{i+j=k} c_i(E) c_j(F).
\end{aligned}
\end{equation}
For oriented real vector bundles $E,F\to X$,
\begin{equation}
	e(E\oplus F) = e(E)e(F).
\end{equation}
Both of these equations take place in the ring $\H^*(X;\ZZ)$. However, for Pontryagin classes, the corresponding
formula only holds modulo $2$-torsion. That is, in the ring $\H^*(X;\ZZ[1/2])$,
\begin{equation}
\begin{aligned}
	p(E\oplus F) &= p(E)p(F)\\
	p_k(E\oplus F) &= \sum_{i+j=k} p_i(E) p_j(F).
\end{aligned}
\end{equation}
\end{subequations}
The formula for the Pontryagin classes of a direct sum with $\ZZ$ coefficients is known by work of
Thomas \cite{Tho62} and Brown \cite[Theorem 1.6]{Bro82}, but it is a little more complicated.

On to differential cohomology. Given vector bundles with connection $(E, A^E)$ and $(F, A^F)$ over a
space $X$, the direct sum $E\oplus F$ has an induced connection $A^E\oplus A^F$. One can prove the Whitney
sum formulas~\eqref{ordinary_Whitney_sum} by studying the effect of the maps 
\[\mathrm B(\GL_{n_1}(\CC)\times\GL_{n_2}(
\CC))\to \mathrm{BGL}_{n_1 + n_2}(\CC)\]
(resp.\ $\BSO_{n_i}$, $\mathrm{BGL}_{n_i}(\RR)$) on cohomology. Naturality of
\cref{differential_CW_lift} then implies
\begin{subequations}
\label{differential_Whitney}
\begin{align}
	\chat(E\oplus F,  \conn^E\oplus \conn^F) &= \chat(E,  \conn^E)\chat(F,  \conn^F)\\
	\label{differential_Whitney_Euler}
	\ehat(E\oplus F,  \conn^E\oplus \conn^F) &= \ehat(E,  \conn^E)\ehat(F,  \conn^F)\\
	\phat(E\oplus F,  \conn^E\oplus \conn^F) &= \phat(E,  \conn^E)\phat(F,  \conn^F),
	\label{differential_Whitney_Pontryagin}
\end{align}
\end{subequations}
where $E$ and $F$ are complex or oriented where needed. For~\eqref{differential_Whitney_Euler} we must assume $E$
and $F$ come with Euclidean metrics which $\conn^E$ and $\conn^F$ are compatible with, because of
\cref{metric_differential_cc}; and as usual~\eqref{differential_Whitney_Pontryagin} takes
place in $\Hhat^*(X;\ZZ[1/2])$.\index[terminology]{Whitney sum formula!in differential cohomology}

The formulas~\eqref{differential_Whitney} are less useful than they might seem: in some places you might want to
use it, the connection you care about on $E\oplus F$ is not a direct sum connection. This happens, for example, in
the proof of \cref{differential_conformal_immersion} in \cref{applications_part}. Fortunately, the differential
Whitney sum formula is true in more generality.
\begin{defn}
\index[terminology]{compatible connection}
\label{compatible_connection}
Choose connections $\conn^E$ on $E$, $\conn^F$ on $F$, and $\overline{\conn}$ on $E\oplus F$. The projections
$E\oplus F\rightrightarrows E, F$ induce connections $\overline{\conn}{}^E$, resp.\ $\overline{\conn}{}^F$ on $E$,
resp.\ $F$ from $\overline A$. 
Let $\curvature{\overline{\conn}}\in\Omega_X^2(\End(E\oplus F))$ be the curvature of
$\overline{\conn}$. We say $\conn$ is \textit{compatible} with $\conn^E\oplus\conn^F$ if
\begin{enumerate}[(1)]
	\item $\overline{\conn}{}^E =  \conn^E$ and $\overline{\conn}{}^F =  \conn^F$, and
	\item given vector fields $v,w$ on $X$, $\curvature{\overline{\conn}}(v, w)\in\Gamma(\End(E\oplus F))$ is block
	diagonal.
\end{enumerate}
\end{defn}
There are two notions of compatibility floating around: compatibility with a metric, and compatibility with the
direct-sum connection. They are different.

\begin{thm}[{(Cheeger--Simons \cite[Theorem 4.7]{Cheeger-Simons})}]
	If $\overline{\conn}$ is compatible with $\conn^E\oplus\conn^F$, then $\phat(E\oplus F, \overline{\conn}) =
	\phat(E\oplus F,  \conn^E\oplus \conn^F)$. If $E$ and $F$ are oriented and Euclidean, $\ehat(E\oplus F,
	\overline{\conn}) = \ehat(E\oplus F,  \conn^E\oplus \conn^F)$. If $E$ and $F$ are complex, $\chat(E\oplus F,
	\overline{\conn}) = \chat(E\oplus F,  \conn^E\oplus \conn^F)$. Therefore analogues of~\eqref{differential_Whitney}
	hold with $\overline A$ in place of $ \conn^E\oplus \conn^F$.
\end{thm}

\noindent The proof uses a variation formula for the Chern--Simons form similar to \cref{variation_Chern_Simons}.
\index[terminology]{Chern--Simons form!variation formula}

Recall that the Whitney sum formula can be used to show that the Euler class obstructs the existence of
a section of an oriented vector bundle. In the same way, the differential Euler class obstructs flat sections.

\begin{lem}
	Let $V\to M$ be an oriented Euclidean vector bundle with compatible connection $\conn$ admitting a flat section.
	Then $\ehat(V, \conn) = 0$.
\end{lem}

\begin{proof}
	The flat section splits $V = V'\oplus\underline{\RR}$ such that $\conn$ is compatible with the direct sum
	connection, where $\underline{\RR}$ carries the standard connection $\d$.  Because $\ehat(\underline{\RR}, \d) =
	0$, the Whitney sum formula finishes the proof for us.
\end{proof}

%-------------------------------------------------------------------%
%-------------------------------------------------------------------%
%  Secondary invariants and Chern–Simons forms                      %
%-------------------------------------------------------------------%
%-------------------------------------------------------------------%

\subsection{Secondary invariants and Chern--Simons forms}
\label{secondary_invariants}
% first, the motivating idea
Degree-$n$ characteristic classes provide invariants of closed, oriented $n$-manifolds by integration, and these
invariants provide useful topological information: integrating the Euler class produces the Euler characteristic,
and integrating products of Pontryagin classes produces oriented bordism invariants. In this section we discuss
the analogous invariants defined by integrating on-diagonal differential characteristic classes; since the
differential cohomology of a point is not concentrated in degree zero, we do not have to stick to $n$-manifolds.

Let $G$ be a compact Lie group and $c^\ZZ\in\H^n(\BG;\ZZ)$. \Cref{differential_CW_lift} gives us an on-diagonal differential lift
$\chat\in\Hhat^n(\BnablaG;\ZZ)$ of $c^\ZZ$. Let $M$ be a closed, oriented $(n-1)$-manifold, and let $P \to M$ be a
principal $G$-bundle with connection $\conn$. In \cref{FiberIntegration}, we constructed an integration map on
differential cohomology. Integration has degree $-(n-1)$, so if $\alpha_c(P,  \conn)$ denotes the integral of
$\chat(P,  A)$, then $\alpha_c(P,  \conn)$ is an element of $\RR/\ZZ$:
\begin{equation}
\begin{aligned}
	\int_M\colon \Hhat^n(M; \ZZ) &\longrightarrow \Hhat^1(\pt; \ZZ)\cong\RR/\ZZ\\
	\chat(P,  \conn) &\longmapsto \alpha_c(P,  \conn).
\end{aligned}
\end{equation}
The quantity $\alpha_c(P, \conn)$, as an $\RR/\ZZ$-valued invariant of principal bundles with connection, is
called the \emph{secondary invariant} associated to $c$. In this context, the $\ZZ$-valued purely topological
invariant $\int_M c^\ZZ(P)$ on $n$-manifolds is called the \emph{primary invariant}.
\index[terminology]{secondary invariant}
\index[terminology]{primary invariant}

In examples, secondary invariants tend to be very geometric, despite our general abstract definition.
\begin{example}[(Holonomy of a connection on a principal $\Uup_1$-bundle)]
Let $P\to M$ be a principal $\Uup_{1}$-bundle with connection $\conn$ and consider the differential
first Chern class $\chat_1(P, \conn)$, built from the curvature form of $\conn$. Given an embedded, oriented loop
$i\colon \Circ\hookrightarrow M$, we can pull back $\chat_1(P, \conn)$ to $\Circ$ and integrate, defining an element
of $\RR/\ZZ$. Cheeger--Simons \cite[Example 1.5]{MR827262} show that this $\RR/\ZZ$-valued quantity is the log of
the holonomy of $P$ around $\Circ$. That is, holonomy is the secondary invariant associated to the first Chern class
or the curvature for principal $\Uup_1$-bundles.
\index[terminology]{holonomy!as a secondary invariant}
\end{example}
\begin{example}[(Chern--Simons invariants)]
\label{secondary_chern_simons}
Chern--Simons invariants are important examples of secondary invariants: they will appear several times in several
different ways in \cref{applications_part}. In some settings, any secondary invariant constructed via Chern--Weil
theory is called a Chern--Simons invariant, but by far the most commonly considered example is in dimension $3$.

Choose a compact Lie group $G$ and an element $\lambda\in\H^4(\BG;\ZZ)$, which we call the
\emph{level}. Given a closed $3$-manifold $Y$, a principal $G$-bundle $P\to Y$, and a connection $\conn$ on $P$,
the \emph{Chern--Simons invariant} $\CS_\lambda(P,\conn)\in\RR/\ZZ$ \cite{cs} is defined to be value of
the secondary invariant associated to $\lambda$ on $(P,
\conn)$.\index[terminology]{level}\index[terminology]{Chern--Simons invariant}

The standard construction of $\CS_\lambda(P, \conn)$, which is the construction Chern--Simons gave, is more
geometric. We will discuss this in \cref{config_spaces}. The approach here, using differential cohomology, is due
to Cheeger--Simons \cite{MR827262}.
\end{example}
Chern \cite{Che44} defines a differential form in a sphere bundle related to the secondary invariant built from the
Euler class.

% then a digression about eta-invariants
\begin{remark}[(Secondary invariants and differential generalized cohomology)]
We can try to run the same story with a generalized cohomology theory $E$. To do so, we need a differential
refinement $\Ehat$ of $E$, an integration map for $\Ehat$-cohomology (possibly on manifolds with some additional
structure) and an on-diagonal differential characteristic class $\chat\in \Ehat^*(\BnablaG)$. Together these data are a lot to
ask for, but everything goes through in $\KU$-theory, for example.

Definitions of differential refinements of $\KU$ and $\KO$ were first sketched by Freed \cite[Examples 1.12 and
1.13]{Fre00} and Freed--Hopkins \cite{FH00}. Hopkins--Singer \cite[\S 4.4]{HopkinsSinger} first constructed
differential $\KU$-theory, and Grady--Sati \cite{GS21} first systematically study differential
$\KO$-theory. There are differential lifts of the Atiyah--Bott--Shapiro integration maps in $\KU$- and $\KO$-theory
on closed spin\textsuperscript{$c$}, resp.\ spin manifolds.

We can therefore study secondary invariants for $\KU$- and $\KO$-theories. The final piece of data we need is a
differential characteristic class, and we choose $1\in\KU^0(X)$ or $\KO^0(X)$. The primary invariant associated
with this data on a spin or spin\textsuperscript{$c$} manifold admits a geometric interpretation as the index of
the spinor Dirac operator \cite{AS68}. The secondary invariant has a related description \cite{Lot94}, as the
\emph{$\eta$-invariant} of the Dirac operator, defined and studied by Atiyah--Patodi--Singer \cite{APS1, APS2,
APS3}.

There are several additional models for differential $\KU$-theory constructed by
Klonoff \cite{Klo08},
Bunke--Schick \cite[\S 2]{BS09},
Simons--Sullivan \cite{MR2732065},
Bunke--Nikolaus--Völkl \cite[\S 6]{MR3462099},
Schlegel \cite[\S 4.2]{Sch13},
Tradler--Wilson--Zenalian \cite{TWZ13, TWZ16},
Hekmati--Murray--Schlegel--Vozzo \cite{HMSV15},
Park \cite{Par17},
Gorokhovski--Lott \cite{GL18},
Schlarmann \cite{Sch19},
and Park--Parzygnat--Redden--Stoffel \cite{PPRS21}.
%Several of these teams of researchers also show how to lift the Atiyah--Bott--Shapiro
%integration map on $K$-theory, defined on closed spin\textsuperscript{$c$} manifolds, to differential $K$-theory.
See Bunke--Schick \cite{BunksSchick} for a survey.
\end{remark}
