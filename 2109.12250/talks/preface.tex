%!TEX root = ../diffcoh.tex

\section{Preface}
% stuff Arun added, very much a draft for now
Differential cohomology begins with the observation that many naturally occurring differential forms have
integrality properties. One example is the curvature $\Omega$ of a connection on a complex vector bundle over a
closed manifold $M$; if $N\subset M$ is a closed, oriented, two-dimensional submanifold, then $\int_N \Omega$ is
an integer multiple of $2\pi$. Analogous statements are true, though with different normalization constants, for
other Chern--Weil forms of a vector bundle with connection. The first explanation given is typically that the
cohomology classes represented by these forms are in the image of the map $\H^*(\text{--};\ZZ)\to \H^*(\text{--};
\RR)$, but in a way this fails to capture the entire picture: that the de Rham class of the Chern--Weil form has a
canonical lift to $\H^*(\text{--};\ZZ)$. For example, $(1/2\pi) \Omega$ lifts to the first Chern class of a complex
vector bundle. Differential cohomology is built to house this kind of data: a closed differential form, an
integer-valued cohomology class, and an identification of their images in de Rham cohomology.

A similar situation can happen in quantum physics: abelian
gauge fields give rise to differential forms such as field strengths and currents, and quantization imposes strong
integrality properties on these objects. For example, in the classical theory of electromagnetism, the electric
field $E$ is a $1$-form, and the magnetic field $B$ is a $2$-form. Maxwell's equations on a closed $4$-manifold $M$
imply that the field strength\index[terminology]{field strength} $F = B - \d t\wedge E$ is a closed $2$-form. But
in the quantum theory, the possible values of electric and magnetic fluxes and charges are discretized; there is a
minimum magnetic charge $q_B$, and the integral of $F$ on a closed, oriented surface must be an integer multiple of
$2\pi q_B$. Again we have closed forms with integrality conditions, and so the field strength $B$ refines to a
cocycle representative of a differential cohomology class $\hat B\in\Hhat^2(M; q_B\ZZ)$.

Another perspective on differential cohomology is that it does for geometric objects what ordinary cohomology does
for their topological analogues. Vector bundles and principal bundles have characteristic classes in cohomology;
vector bundles with connection and principal bundles with connection have characteristic classes in differential
cohomology.
%An orientation induces a pushforward map in cohomology, and an orientation and a Riemannian metric
%induce a pushforward map in differential cohomology.
Analogously, topological $\Kup$-theory is built out of vector bundles, and differential
$\Kup$-theory is built out of vector bundles with connection. 

The goal of this book is to provide an introduction to differential cohomology, including both foundational aspects
of generalized differential cohomology theories and applications. We follow Bunke--Nikolaus--Völkl, defining
differential (generalized) cohomology theories as sheaves of spectra on the site of smooth manifolds. We go over the
basics of the theory, including defining the cup product and integration maps. We spend time with characteristic
classes: as hinted above, Chern--Weil forms refine to characteristic classes in differential cohomology, but there
are additional classes which have no topological counterparts. We also go over several applications of differential
cohomology. Often, these are geometric analogues of a well-known application of cohomology to topological
questions. For example, characteristic classes obstruct smooth embeddings of manifolds into $\RR^n$, and
differential characteristic classes can obstruct conformal embeddings into $\RR^n$. Some of these applications are
angled towards physics; for example, we revisit the idea above that differential cohomology has something to say
about quantization.

This book began as lectures given in a graduate student seminar joint between MIT and UT Austin in fall 2019,
initiated by Dan Freed and Mike Hopkins. Most chapters are notes from talks given by various speakers at the
seminar and a few chapters were written afterwards.

%-------------------------------------------------------------------%
%-------------------------------------------------------------------%
%  Assumed Background                                                  %
%-------------------------------------------------------------------%
%-------------------------------------------------------------------%

\subsection{Assumed Background}

We hope that these notes are accessible to readers with a wide range of background knowledge.
The talks included here were part of a topology seminar,
and are therefore biased toward the homotopy theoretic perspective. 
This is evidenced by the fact that we review the definition of a connection
and not that of an \category.
However, knowledge of \categories is not a prerequisite for making use of these notes.
Comfort with sheaves, spectra, and simplicial sets will make reading easier.
The reader will also benefit, both in motivation and understanding,
from a familiarity with basic differential geometry;
this includes connections, curvature, and de Rham cohomology.
\Cref{part:applications} of these notes includes talks on several different applications of differential cohomology.
Enjoyment of these sections should not require any background other than interest in the section title.



%-------------------------------------------------------------------%
%-------------------------------------------------------------------%
%  Linear Overview                                                  %
%-------------------------------------------------------------------%
%-------------------------------------------------------------------%

\subsection{Linear Overview}

We give a brief overview of the three parts of these notes. A more detailed introduction is given at the beginning of each part.

%-------------------------------------------------------------------%
%  Part I: Basics of the Theory                                     %
%-------------------------------------------------------------------%

\subsubsection{\texorpdfstring{\Cref{part:basics}}{Part \ref*{part:basics}}: Basics of the Theory}

The purpose of this part is to introduce the basics of and develop the general theory behind differential cohomology.
In \Cref{Introduction}, we start with some motivation to the approach we take to differential cohomology coming from work of Cheeger--Simons \cite{MR827262} and Simons--Sullivan \cite{MR2365651} on differential characters and ordinary differential cohomology.
The perspective we take on differential cohomology theories is as sheaves of spectra on the category $ \Mfld $ of manifolds; since we also want to consider sheaves that come from chain complexes, we'll work in the framework of sheaves with values in a general \category.
While this might sound somewhat daunting, there are many familiar examples: 
\begin{enumerate}[(1)]
	\item The functor sending a manifold $ M $ to the complex $ \Omegabullet(M) $ of de Rham cochains on $ M $.

	\item The functor sending a manifold $ M $ to the complex $ \Csing^\bullet(M) $ of singular cochains on $ M $.

	\item Given a Lie group $ G $, the functor sending a manifold $ M $ to the groupoid $ \BunG(M) $ (or $ \BunGnabla(M) $) of principal $ G $-bundles on $ M $ (with connection). 
\end{enumerate}
The new example of differential cohomology is essentially built from these ones in a nontrivial way.

In \Cref{sec:basicsetup}, we introduce the basics of sheaves on the category of manifolds, how to manipulate sheaves on $ \Mfld $, and any the category of sheaves (of sets) on $ \Mfld $ contains the standard category of infinite-dimensional manifolds (\textit{Fréchet manifolds}) as a full subcategory.
One important class of sheaves on $ \Mfld $ are those that invert all homotopy equivalences of manifolds.
\Cref{sec:hisheaves} is dedicated to explaining why all sheaves with this property have a very simple and concrete description.
In \Cref{sec:localization}, we explain how to resolve a sheaf by one that inverts all homotopy equivalences of manifolds.
This provides a way of decomposing a sheaf of spectra on $ \Mfld $ into one that inverts all homotopy equivalences
and another that ``comes from geometry''.
\Cref{sec:stable} explains this decomposition as well as how this gives rise to the Simons--Sullivan ``differential
cohomology hexagon'' \cite[\S1]{MR2365651}) relating ordinary cohomology, differential forms, and differential cohomology.

The remainder of this part is dedicated to important examples of differential cohomology theories and refining important constructions with ordinary cohomology.
\Cref{sec:examples} explains Cheeger--Simons differential characters, differential $ \K $-theory, and examples coming from $ G $-bundles in the framework of sheaves on $ \Mfld $.
\Cref{sec:Delignecup} refines the cup product to differential cohomology and explains how to calculate it in many examples.
\Cref{FiberIntegration} refines fiber integration to differential cohomology.
\Cref{sec:digressiononTransferConjecture} finishes the main text of this part with a digression proving Quillen's Transfer Conjecture.
Though not directly related to differential cohomology, this result states that connective spectra can be realized as homotopy-invariant sheaves on the category of correspondences of manifolds where the backwards maps are finite covering maps (i.e., connective spectra have natural \textit{transfers} along finite covering maps).
Our exposition follows work of Bachmann--Hoyois \cite[Appendix C]{MotivicNorms:BachmannHoyois}.

\Cref{part:basics} also has an appendix (\Cref{app:technicaldeatails}).
In this appendix, we prove a few technical category theory results that we need to get the foundations of sheaves on $ \Mfld $ on a solid framework in \Cref{sec:basicsetup,sec:hisheaves}.

%-------------------------------------------------------------------%
%  Part II: Characteristic Classes                                  %
%-------------------------------------------------------------------%

\subsubsection{\texorpdfstring{\Cref{part:charclasses}}{Part \ref*{part:charclasses}}: Characteristic Classes}

Just as one ordinary cohomology is a natural home for characteristic classes, differential cohomology offers its own invariants of bundles. These invariants, known as ``differential characteristic classes,'' are refinements of the classical characteristic classes in cohomology. More explicitly, we will investigate lifts of well-known characteristic classes, such as Chern classes, under the map from differential cohomology to ordinary cohomology.

This part begins be reviewing a few classical techniques and results that will be useful in studying differential characteristic classes, see \Cref{ChernWeilTheory} and \Cref{EquivariantdeRhamCohomology}.

Differential characteristic classes where first studied by Cheeger--Simons \cite{Cheeger-Simons}. %sentence about what these are
We discuss differential characters in \Cref{DifferentialCharacteristicClasses}. 
Building on work of Bott \cite{BottPaper}, Freed and Hopkins \cite{FreedHopkins} classified all differential characteristic classes for bundles equipped with a flat connection. 
This refines the classical Chern--Weil story, which we review in \Cref{ChernWeilTheory}. 
The contents of \cite{FreedHopkins} are covered in \Cref{WorkofFreedHopkins}.  
A closer look at the methods used in \cite{BottPaper} reveal that one can remove the connection data with some alterations. 
In \Cref{BottsMethod}, we delve into Bott's paper and the theorems it relies upon. In particular, we discuss van Est's theorem relating continuous cohomology to Lie algebra cohomology. 
Using the results of \cite{BottPaper}, Hopkins, in \Cref{LiftsofChernClasses}, discusses how to lift ordinary Chern classes to a form of differential cohomology, without the presence of a connection. 
The existence of a differential version of the Cartan formula is also considered. 

This part of the notes concludes with an interesting application of differential lifts of Chern classes to a possible construction of the Virasoro group. The Virasoro group is a certain central extension of orientation preserving diffeomorphisms $\Diffplus(\Circ)$ of $\Circ$. As Hopkins outlines in \Cref{LiftsofChernClasses}, one can obtain central extensions of $\Diffplus(\Circ)$ from a certain differential cohomology group. The details of this construction, as well as a review of the Virasoro algebra and group, appear in \Cref{VirasoroAlgebra}.

%-------------------------------------------------------------------%
%  Part III: Applications                                           %
%-------------------------------------------------------------------%

\subsubsection{\texorpdfstring{\Cref{part:applications}}{Part \ref*{part:applications}}: Applications}
In this part we discuss some uses of differential cohomology in topology, geometry, and physics. Some, but not all,
of these applications are part of the idea that what ordinary cohomology can do for topological questions,
differential cohomology can do for geometric ones, and many of these applications are related to various aspects of
quantum field theory.

One of the key links between differential cohomology and geometry is through Chern--Simons invariants, invariants
of connections which can be defined either in terms of integration of differential characteristic classes or
directly using geometric information. Because of this, several applications of differential cohomology to geometry
or physics pass through Chern--Simons theory. We introduce and apply Chern--Simons invariants in
\Cref{config_spaces} and also use them in \Cref{conformal_immersions}.

Our first two applications of differential cohomology are in geometry and topology. In \Cref{config_spaces}, we
discuss work of Evans-Lee--Saveliev \cite{deletedsquare}, who use Chern--Simons invariants to study the homotopy
types of two-point configuration spaces of lens spaces.
%Longoni--Salvatore \cite{LS05} had shown that the homotopy
%type of the two-point configuration space is in general a stronger invariant than the homotopy type of the
%underlying manifold, and using Chern--Simons invariants and a few other techniques, Evans-Lee--Saveliev are able to
%provide many more examples of this phenomenon.
Then in \Cref{conformal_immersions}, we use differential Pontryagin classes and Chern--Simons forms to obstruct
conformal immersions of conformal manifolds into Euclidean space, following Chern--Simons \cite{cs}; along the way
we spend some time getting to know the geometry of Chern--Weil and Chern--Simons forms.

The next two applications are to physics. \Cref{field_theory} applies differential cohomology to the quantization
of abelian gauge fields, using electromagnetism as an example. In classical physics, the field strength of an
abelian gauge field is a closed differential form; quantization lifts from closed forms to
cocycles for a differential cohomology group. The other physics
application we discuss, in \Cref{invertible_field_theories}, is quite different: a conjecture of
Freed--Hopkins \cite{FH21} using differential generalized cohomology to classify invertible, non-topological field
theories. This is a geometric conjecture modeled on a topological theorem of Freed--Hopkins (\textit{ibid.})
classifying invertible topological field theories using Madsen--Tillmann spectra.
We discuss this conjecture and several examples, including
classical Chern--Simons theory.

Our final two chapters are about the representation theory of loop groups. Loop groups are infinite-dimensional Lie
groups whose representation theory is strikingly similar to that of compact Lie groups, so long as one works with
what are called positive energy representations. In \Cref{loop_groups}, we survey this theory, defining and
motivating positive energy representations and sketching a proof of a theorem of Pressley--Segal \cite{loop}, which
says that positive energy representations admit projective intertwining actions of $\Diffplus(\Circ)$.
%We also
%discuss connections between the representation theory of loop groups and differential cohomology.
In
\Cref{segal_sugawara}, we study the Pressley--Segal theorem at the Lie algebra level, where this intertwining
projective action can be made more explicit. Since projective representations are equivalent to representations of
a central extension, the Virasoro algebra makes an appearance here.
%The math in
%\Cref{loop_groups,segal_sugawara} is related to aspects of two-dimensional conformal field theory, and we discuss
%some of the connections.

%-------------------------------------------------------------------%
%-------------------------------------------------------------------%
%  What's Not Included                                              %
%-------------------------------------------------------------------%
%-------------------------------------------------------------------%

\subsection{What's Not Included}

One %the?
original approach to differential cohomology is presented by Hopkins and Singer in \cite{HopkinsSinger}. 
While we look to this reference for motivation and intuition, 
we do not take this as our definition of a differential cohomology theory. 
Instead, we work with the more modern approach using sheaves on manifolds. 
We also make use of \cite{HopkinsSinger} for constructions of the cup product and fiber integration in differential cohomology, see \Cref{sec:Delignecup,FiberIntegration}. 

Several examples of differential cohomology theories, such as differential K-theory, are discussed in \Cref{sec:examples}; 
however, there are many more examples that we do not mention. %add reference to other examples in literature
Moreover, for most of these notes, we focus our attention on the specific example of the differential version of ordinary cohomology. 
This leaves several interesting areas of study, such as differential K-theory characteristic classes, untouched.

We do not present Schreiber's elegant and very general theory of \textit{differential cohomology in a cohesive \topos} \cite{Urs}.
Schreiber's work requires background that we do not assume; we decided to stick with the setting of sheaves on the category of manifolds to make the material accessible to the graduate students attending the seminar.

There are also many applications of differential cohomology to physics which we do not discuss in detail here. See \Cref{applications_part} for a discussion of related work. 

\subsection{Cover image}
One of the theses of this book is that differential cohomology has applications to physics. It therefore seems apt
to choose a cover image of another example of hexagons in the real world. Our cover image is a picture of Giant's
Causeway, a part of the coastline in Northern Ireland consisting of tens of thousands of tessellating hexagonal
basalt columns. This image is by Giuseppe Milo and can be found at
\href{https://www.flickr.com/photos/giuseppemilo/46587488041/in/photostream/}{\nolinkurl{flickr.com/photos/giuseppemilo/46587488041/in/photostream/}}; we cropped it slightly. It is
licensed under the \href{https://creativecommons.org/licenses/by/2.0/}{CC BY 2.0} license.

%-------------------------------------------------------------------%
%  Acknowledgements                                                 %
%-------------------------------------------------------------------%

\subsection{Acknowledgements}%i hate writing acknowledgements, so plz change this!!

These notes are a compilation of talks given in the Juvitop seminar at MIT in the fall of 2019. 
The seminar was run jointly with UT Austin and we thank the participants of both cities for their comments and discussion. 
%In particular, thanks to specific participants, 
%I don't remember who Arun said was particularly helpful, maybe it was Charlie?
%maybe something about people who talked in the slack
%maybe thanks to Dylan for discussions about organization
We would also like to thank the speakers, 
Dexter Chua, Sanath Devalapurkar, Dan Freed, Mike Hopkins, Greg Parker, Charlie Reid, and Adela Zhang,
both for volunteering to speak, as well as writing up notes to appear here. 

The seminar, and these notes, could not have existed without the help of Dan Freed and Mike Hopkins. 
We thank both Freed and Hopkins for their mathematical help, their organizational help, and for giving talks in the seminar. 
We also would like to express our appreciation for Freed and Hopkins %and teleman?
generosity in sharing new mathematical ideas and older insights along the way.

Extra thanks are due to Hopkins for buying us fancy video equipment to help with our half-virtual seminar.

PH gratefully acknowledges support from the MIT Dean of Science Fellowship, UC President's Postdoctoral Fellowship, and NSF Mathematical Sciences Postdoctoral Research Fellowship under Grant \#DMS-2102957. 
PH and AA acknowledge support from the National Science Foundation Graduate Research Fellowship under Grant \#112237.
This text is partially based upon work supported by the National Science Foundation under Grant \#DMS-1440140, while AD and PH were in residence at the Mathematical Sciences Research Institute in Berkeley, California, during the Spring 2020 semester.
