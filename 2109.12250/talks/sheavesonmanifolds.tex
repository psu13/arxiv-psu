%!TEX root = ../diffcoh.tex

\section{Basics of Sheaves on Manifolds}\label{sec:sheavesonMan}\label{sec:basicsetup}
\textit{by Peter Haine}

The purpose of this chapter is to begin to set up the basics of differential cohomology theories as sheaves on the category of all manifolds.
\Cref{subsec:ShManDef} starts with the basic definitions.
\Cref{subsec:DRreminder} gives a reminder on derived \categories and their relation to spectra so that we can give examples of sheaves on the category of manifolds in \cref{subsec:firstexamples}.
In \cref{subsec:checkonstalks}, we explain why in all situations of interest, we can check equivalences of
differential cohomology theories ``on stalks''.
\Cref{subsec:Cartsheaves} gives an alternative description of the \category of sheaves on manifolds in terms of sheaves on the smaller category of Euclidean spaces.
\Cref{subsec:excision} is a digression giving a reformulation of the sheaf condition in terms of an
\textit{excision} condition (or \textit{Mayer--Vietoris} property) and a ``finiteness'' condition.
We finish the chapter with a digression explaining Losik and Hain's results embedding infinite dimensional manifolds into sheaves of sets on the category of (finite dimensional) manifolds (\cref{subsec:infdimMan}).

%-------------------------------------------------------------------%
%  Definitions                                                      %
%-------------------------------------------------------------------%

\subsection{Definitions}\label{subsec:ShManDef}

\begin{notation}
	We write $ \Man $ for the (ordinary) category of smooth manifolds, including the empty manifold.
	The category $ \Man $ has a Grothendieck topology where the covering families are families of open embeddings
	\begin{equation*}
		\{ j_{\alpha} \colon \incto{U_{\alpha}}{M} \}_{\alpha \in A}
	\end{equation*}
	such that the family of open sets $ \{j_{\alpha}(U_{\alpha})\}_{\alpha \in A} $ is an open cover of $ M $. 
	Whenever we regard $ \Man $ as a site, we use this topology.
\end{notation}

\begin{remark}
	Since the category $ \Man $ is equivalent to the category of manifolds with a fixed embedding into $ \RR^\infty $, the category $ \Man $ is essentially small.
\end{remark}

\begin{definition}\label{def:sheafonMan}
	Let $ C $ be a presentable \category.
	We write 
	\begin{equation*}
		\PSh(\Man;C) \colonequals \Fun(\Manop,C)
	\end{equation*}
	and write
	\begin{equation*}
		\Sh(\Man;C) \subset \PSh(\Man;C)
	\end{equation*}
	for the full subcategory spanned by the $ C $-valued sheaves on the site $ \Man $ with respect to the Grothendieck topology given by open covers.

	Explicitly, a $ C $-valued presheaf $ E \colon \fromto{\Manop}{C} $ is a sheaf if and only if for each manifold $ M $, the restriction $ \restrict{E}{\Open(M)} $ of $ E $ to the site $ \Open(M) $ of open submanifolds of $ M $ is a sheaf on the topological space $ M $.
\end{definition}

\begin{notation}
	We write $ \SMan \colon \fromto{\PSh(\Man;C)}{\Sh(\Man;C)} $ for the left adjoint to the inclusion, that is, the \textit{sheafification} functor.
\end{notation}

\begin{notation}
	We write $ \Set $ for the category of sets, $ \Spc $ for the \category of spaces, $ \Sp $ for the \category of spectra, and $ \Cat_{\infty} $ for the \category of \categories.
\end{notation}

\begin{example}
	Let $ \yo \colon \incto{\Man}{\PSh(\Man;\Set)} $ denote the Yoneda embedding.
	For each manifold $ M $, the representable presheaf $ \yo(M) $ is a sheaf.
	Unless noted otherwise, we simply write $ M $ for the sheaf $ \yo(M) $.
\end{example}

The following is the fundamental definition of this text:

\begin{definition}\label{def:diffcoh}
	The \category of \textit{differential cohomology theories} is the \category $ \Sh(\Man;\Sp) $ of sheaves of spectra on $ \Man $. 
\end{definition}

\noindent For most of this text we work in the generality of sheaves with values in a general presentable \category, or stable presentable \category. 
The main reason for doing this is because we have reason to consider sheaves of spaces, sheaves of chain complexes, and sheaves of spectra, and want to treat them on the same footing.

\begin{remark}
	We take the approach of Freed--Hopkins \cite{MR3049871} and consider sheaves on the category of smooth manifolds.
	The general setup here is very robust, and one can take the basic objects to be manifolds with corners without
	essential change to how theory works; this is the approach taken by Hopkins--Singer \cite{MR2192936} and
	Bunke--Nikolaus--Völkl \cite{MR3462099}.
\end{remark}

The first basic property we prove about sheaves on $ \Man $ is that morphism is an equivalence if and only if it is when evaluated on each Euclidean space.
For this, we use the fact that manifolds admit \textit{good covers}.

\begin{recollection}[(good covers)]
	Let $ M $ be an $ n $-manifold.
	An open cover $ \Uu $ of $ M $ is \textit{good} if for every finite set $ U_1,\ldots,U_m \in \Uu $ of opens in $ \Uu $, the intersection $ U_1 \intersect \cdots \intersect U_m $ is either empty or diffeomorphic to $ \RR^n $.
\end{recollection}

\begin{notation}
	Let $ T $ be a topological space and $ U \subset T $ be open.
	For every open cover $ \Uu $ of $ U $, write $ \Iup(\Uu) \subset \Open(T) $ for the full subposet  consisting of all nonempty finite intersections of elements in $ \Uu $.
\end{notation}

\begin{lemma}\label{lem:checkequivonRn}
	Let $ C $ be a presentable \category.
	A morphism $ f \colon \fromto{E}{E'} $ in $ \Sh(\Man;C) $ is an equivalence if and only if for each integer $ n \geq 0 $, the morphism $ f(\RR^n) \colon \fromto{E(\RR^n)}{E'(\RR^n)} $ is an equivalence in $ C $.
\end{lemma}

\begin{proof}
	Let $ M $ be a manifold and $ \Uu $ a good cover of $ M $.
	The morphism $ f $ induces a commutative square
	\begin{equation*}\label{eq:Cechequivs}
		\begin{tikzcd}
			E(M) \arrow[d, "f(M)"'] \arrow[r, "\sim"{yshift=-0.2em}] & \lim_{U \in \Iup(\Uu)^{\op}} E(U) \arrow[d] \\
			E'(M) \arrow[r, "\sim"{yshift=-0.2em}] & \lim_{U \in \Iup(\Uu)^{\op}} E'(U) \comma
		\end{tikzcd}
	\end{equation*}
	where the horizontal morphisms are equivalences because $ E $ and $ E' $ are sheaves.
	Since the cover $ \Uu $ is good and $ f $ is an equivalence on Euclidean spaces, we see that the induced morphism
	\begin{equation*}
		f \colon \fromto{\restrict{E}{\Iup(\Uu)^{\op}}}{\restrict{E'}{\Iup(\Uu)^{\op}}}
	\end{equation*}
	of $ \Iup(\Uu)^{\op} $-indexed diagrams in $ C $ is an equivalence, which proves the claim.
\end{proof}

%-------------------------------------------------------------------%
%  Reminder on derived ∞-categories and Eilenberg–Mac Lane spectra  %
%-------------------------------------------------------------------%

\subsection{Reminder on derived \texorpdfstring{$\infty$}{∞}-categories and Eilenberg--Mac Lane spectra}\label{subsec:DRreminder}

In order to give some important examples of sheaves on $ \Mfld $, we review the basics of derived \categories of rings and their relation to spectra.

\begin{notation}[(derived \categories)]
	Let $ R $ be a ring.
	We write $ \Ch(R) $ for the category of chain complexes of $ R $-modules.
	We write $ \D(R) $ for the \emph{derived \category} of $ R $ obtained from the category $ \Ch(R) $ by formally inverting the quasi-isomorphisms \cite[\HAthm{Definition}{1.3.5.8}, \HAthm{Propositon}{1.3.5.15}, \& \HAthm{Remark}{7.1.1.16}]{HA}.
	That is, $ \D(R) $ is the universal \category equipped with a functor $ \fromto{\Ch(R)}{\D(R)} $ carrying quasi-isomorphisms in $ \Ch(R) $ to equivalences in the \category $ \D(R) $.
	Note that for every map of rings $ \fromto{R}{S} $, the forgetful functor $ \fromto{\Ch(R)}{\Ch(S)} $ preserves quasi-isomorphisms, hence induces a forgetful functor $ \fromto{\D(R)}{\D(S)} $.
\end{notation}

\begin{recollection}[(Eilenberg--Mac Lane spectra)]\label{rec:EMspectra}
	The inclusion $ \Ab \subset \Sp $ of the category of abelian groups into the category of spectra as those spectra with homotopy groups in degree $ 0 $ (i.e., ordinary cohomology theories) extends to a right adjoint functor
	\begin{equation*}
		\EM \colon \fromto{\D(\ZZ)}{\Sp} \period
	\end{equation*}
	The functor $ \EM $ is called the \emph{Eilenberg--Mac Lane} functor \HA{Example}{1.3.3.5}.
	For a ring $ R $, we also simply write $ \EM $ for the composite
	\begin{equation*}
		\begin{tikzcd}
			\D(R) \arrow[r] & \D(\ZZ) \arrow[r, "\EM"] & \Sp
		\end{tikzcd}
	\end{equation*}
	for the composite of the forgetful functor $ \fromto{\D(R)}{\D(\ZZ)} $ with the Eilenberg--Mac Lane functor.
	The spectrum $ \HR $ represents ordinary cohomology with coefficients in $ R $.
\end{recollection}

\begin{recollection}[{($ \HR $-modules)}]
	Every spectrum in the image of $ \EM \colon \fromto{\D(R)}{\Sp} $ is a module over the Eilenberg--Mac Lane spectrum $ \HR $ representing ordinary cohomology with coefficients in $ R $.
	Moreover, the Eilenberg--Mac Lane functor induces an equivalence
	\begin{equation*}
		\equivto{\D(R)}{\Mod(\HR)}
	\end{equation*}
	between the derived \category $ \D(R) $ and the \category $ \Mod(\HR) $ of $ \HR $-module spectra \HA{Proposition}{7.1.4.6}.
	Under this equivalence $ \equivto{\D(R)}{\Mod(\HR)} $, the functor $ \EM \colon \fromto{\D(R)}{\Sp} $ corresponds to the forgetful functor $ \fromto{\Mod(\HR)}{\Sp} $
\end{recollection}

%-------------------------------------------------------------------%
%  First examples                                                   %
%-------------------------------------------------------------------%

\subsection{First examples}\label{subsec:firstexamples}

Now we give some examples of sheaves on manifolds coming from topological spaces, complexes of differential forms, and bundles.

%-------------------------------------------------------------------%
%  Topological spaces                                               %
%-------------------------------------------------------------------%

\subsubsection{Topological spaces}

\begin{notation}
	Write $ \Top $ for the category of topological spaces.
\end{notation}

\begin{construction}\label{ex:Topyoneda}
	Define a restricted Yoneda functor $ y_{\Top} $ by
	\begin{align*}
		y_{\Top} \colon \Top &\to \PSh(\Man;\Set) \\
		T &\mapsto [M \mapsto \Map_{\Top}(M,T)] \period
	\end{align*}
	Since continuous functions glue over open covers, the assignment $ M \mapsto \Map_{\Fre}(M,T) $ is a sheaf on $ \Man $.
	That is, $ y_{\Top} $ factors through $ \Sh(\Man;\Set) $.
	Hence every topological space defines a sheaf on $ \Man $.
\end{construction}


%-------------------------------------------------------------------%
%  Differential forms                                               %
%-------------------------------------------------------------------%

\subsubsection{Differential forms}

\begin{example}[(differential forms)]
	Let $ i \geq 0 $ be an integer.
	The functor
	\begin{equation*}
		\Omega^i \colon \fromto{\Manop}{\Vect(\RR)}
	\end{equation*} 
	sending manifold $ M $ to vector space $ \Omega^i(M) $ of differential $ i $-forms on $ M $ with functoriality given by pullback of bundles is a sheaf.
	Note that by the Yoneda Lemma, there is a natural isomorphism
	\begin{equation*}
		\Map_{\Sh(\Man;\Set)}(M,\Omega^i) \isomorphic \Omega^i(M) \period
	\end{equation*}
\end{example}

\begin{example}[(de Rham complex)]
	Putting togther all $ i $ at once, the functor
	\begin{equation*}
		\Omegabullet \colon \fromto{\Manop}{\Ch(\RR)}
	\end{equation*} 
	sending manifold $ M $ to its de Rham complex $ \Omegabullet(M) $ is a sheaf of chain complexes on $ \Man $.

	Even better, $ \Omegabullet $ is a sheaf in the derived sense: the composite
	\begin{equation*}
		\begin{tikzcd}
			\Manop \arrow[r, "\Omegabullet"] & \Ch(\RR) \arrow[r] & \D(\RR)
		\end{tikzcd}
	\end{equation*}
	with the localization functor $ \fromto{\Ch(\RR)}{\D(\RR)} $ is a sheaf valued in the \textit{\category} $ \D(\RR) $.
\end{example}

%-------------------------------------------------------------------%
%  Bundles & sheaves                                                %
%-------------------------------------------------------------------%

\subsubsection{Bundles \& sheaves}

\begin{example}[(vector bundles)]
	Write
	\begin{equation*}
		\VectRR \colon \fromto{\Manop}{\Gpd}
	\end{equation*} 
	for the functor sending a manifold $ M $ to the groupoid of (finite dimensional) real vector bundles on $ M $ and bundle isomorphisms, with functoriality given by pullback of bundles.
	Again, the local nature of the definition of a vector bundle ensures that $ \VectRR  $ is a sheaf of groupoids on $ \Man $.
\end{example}

\begin{example}[(principal bundles)]\label{ex:BunG}
	Let $ G $ be a Lie group.
	Write
	\begin{equation*}
		\BunG \colon \fromto{\Manop}{\Gpd}
	\end{equation*} 
	for the functor sending a manifold $ M $ to the groupoid of (smooth) principal $ G $ bundles on $ M $ and bundle isomorphisms, with functoriality given by pullback of bundles.
	The locally triviality of principal bundles implies that $ \BunG $ is a sheaf of groupoids on $ \Man $.

	Similarly, write
	\begin{equation*}
		\BunGnabla \colon \fromto{\Manop}{\Gpd}
	\end{equation*} 
	for the functor sending a manifold $ M $ to the groupoid of (smooth) principal $ G $ bundles on $ M $ with connection and bundle isomorphisms respecting connections, with functoriality given by pullback of bundles.
	Again, the local nature of the definition of a bundle with connection ensures that $ \BunGnabla  $ is a sheaf of groupoids on $ \Man $.
\end{example}

\begin{example}[(sheaves)]\label{ex:ShLC}
	For each manifold $ M $, write $ \Sh(M) $ for the \category of sheaves of spaces on $ M $, and $ \LC(M) \subset \Sh(M) $ for the full subcategory spanned by the locally constant sheaves of spaces.
	The assignment $ \goesto{M}{\Sh(M)} $ extends to a functor
	\begin{equation*}
		\Sh \colon \fromto{\Manop}{\Cat_{\infty}}
	\end{equation*} 
	with functoriality given by pullback of sheaves.
	The functor $ \Sh $ is a sheaf of (large) \categories on $ \Man $ \HTT{Theorem}{6.1.3.9}.
	Since locally constant sheaves are preserved by sheaf pullback and local constancy is a local condition, the subfunctor $ \LC \subset \Sh $ is also a sheaf of (large) \categories on $ \Man $.
\end{example}

%-------------------------------------------------------------------%
%  Checking equivalences on stalks                                  %
%-------------------------------------------------------------------%

\subsection{Checking equivalences on stalks}\label{subsec:checkonstalks}

We now explain that equivalences of sheaves on $ \Man $ with values in a \textit{compactly generated} \category
(e.g., $ \Spc $, $ \Sp $, $ \D(R) $) can be checked on ``stalks'' at the origins in $ \RR^n $ for $ n \geq 0 $.
The proof of this requires a few technical detours which we defer to \Cref{subsec:points}.

\begin{notation}
	Let $ M $ be a manifold and $ x \in M $.
	We write $ \Open_x(M) \subset \Open(X) $ for the full subposet spanned by the open neighborhoods of $ x \in M $.
\end{notation}

\begin{definition}\label{def:stalkCG}
	Let $ C $ be a compactly generated \category, $ E \in \Sh(\Man;C) $ a $ C $-valued sheaf on $ \Man $, $ M $ a manifold, and $ x \in M $.
	The \textit{stalk} of $ E $ at $ x \in M $ is the filtered colimit
	\begin{equation}\label{eq:stalkfilteredcolim}
		\xupperstar (E) \colonequals \colim_{U \in \Open_x(M)^{\op}} E(U)
	\end{equation}
	in $ C $.
\end{definition}

\begin{warning}
	It is important that we have phrased \Cref{def:stalkCG} only for compactly generated coefficients.
	It is true that for any presentable \category $ C $, manifold $ M $, and point $ x \in M $, there is a stalk functor $ \xupperstar \colon \fromto{\Sh(\Man;C)}{C} $ (see \Cref{cons:stalk}).
	However, if $ C $ is not compactly generated then $ \xupperstar $ need not be computed by the filtered colimit \eqref{eq:stalkfilteredcolim}.
\end{warning}

\begin{notation}\label{ntn:origin}
	For each integer $ n \geq 0 $ and number $ r \in \RR_{>0} $, write $ 0_n \in \RR^n $ for the origin, and write
	\begin{equation*}
		\Bup_{\RR^n}(r) \subset \RR^n
	\end{equation*}
	for the open ball in $ \RR^n $ of radius $ r $ centered at the $ 0_n $.
\end{notation}

\begin{nul}
	Let $ E \colon \fromto{\Manop}{C} $ be a sheaf on $ \Man $.
	Note that the stalk $ 0_n\upperstar(E) $ can be computed as the colimit
	\begin{equation*}
		0_n\upperstar(E) \equivalent \colim_{k \in \NN} E(\Bup_{\RR^n}(1/k)) \period
	\end{equation*}
\end{nul}

The following result comes from the functoriality of a sheaf on $ \Man $ in \textit{all} manifolds, the fact that for ever $ n $-manifold $ M $ and point $ x \in M $, there exists an open embedding $ j \colon \incto{\RR^n}{M} $ such that $ j(0_n) = x $, and that equivalences in sheaves on $ M $ can be checked on stalks.
In \Cref{subsec:points} we provide a detailed proof.

\begin{proposition}[(\Cref{app.prop:hypercompleteness})]\label{prop:hypercompleteness}
	Let $ C $ be a compactly generated \category.
	A morphism $ f $ in $ \Sh(\Man;C) $ is an equivalence if and only if for each integer $ n \geq 0 $, the morphism $ 0_n\upperstar(f) $ is an equivalence in $ C $.
\end{proposition}

\begin{remark}\label{rem:FreedHopkinsmodel}
	\Cref{prop:hypercompleteness} is important from our perspective.
	Freed and Hopkins work with differential cohomology theories using the language of simplicial sheaves and model categories \cite{MR3049871}.  
	Combining \Cref{prop:hypercompleteness} with \cite[\HTTthm{Remark}{6.5.2.2} \& \HTTthm{Proposition}{6.5.2.14}]{HTT} shows that the model structure on simplicial presheaves on $ \Man $ considered in \cite[§5]{MR3049871} presents the \category $ \Sh(\Man;\Spc) $.
\end{remark}

\begin{warning}
	\Cref{prop:hypercompleteness} does \textit{not} hold when $ C $ is replaced by an arbitrary presentable \category.
\end{warning}

%-------------------------------------------------------------------%
%  Sheaves on the Euclidean site                                    %
%-------------------------------------------------------------------%

\subsection{Sheaves on the Euclidean site}\label{subsec:Cartsheaves}

In this section, we refine \Cref{lem:checkequivonRn} in the following manner.
Since every manifold admits an open cover by Euclidean spaces, the category of sheaves of \textit{sets} on $ \Man $ is equivalent to sheaves of sets on the full subcategory spanned by the Euclidean spaces.
We prove an analogous result for sheaves of \textit{spaces}; this is not immediate in the higher-categorical setting \SAG{Counterexample}{20.4.0.1}.
The reason for this subtlety is exactly the failure of Whitehead's Theorem to hold in an arbitrary \category of sheaves of spaces.
However, \Cref{prop:hypercompleteness} implies that Whitehead's Theorem holds in $ \Sh(\Man;\Spc) $; a general result \cite[Corollary 3.12.13]{exodromy} implies that sheaves on the site of Euclidean spaces and sheaves on $ \Man $ coincide.

 
\begin{definition}\label{def:Euclideansite}
	The \textit{Euclidean site} is the full subcategory $ \Euc \subset \Man $ spanned by the Euclidean spaces $ \RR^n $ for $ n \geq 0 $, with the induced Grothendieck topology.
\end{definition}

The proof of the following is quite short. 
However, it involves some technical tools we have not yet introduced, so we defer it to \cref{subsec:points}.

\begin{lemma}[(\Cref{app.lem:hypersheavesonCart})]\label{lem:hypersheavesonCart}
	Let $ C $ be a presentable \category.
	Then restriction of presheaves
	\begin{equation*}
		\restrict{(-)}{\Eucop} \colon \fromto{\Sh(\Man;C)}{\Sh(\Euc;C)}
	\end{equation*}
	is an equivalence of \categories.
	The inverse is given by right Kan extension along the inclusion $ \incto{\Eucop}{\Manop} $.
\end{lemma}


%-------------------------------------------------------------------%
%  Digression: Excision & the sheaf condition                       %
%-------------------------------------------------------------------%

\subsection{Digression: Excision \& the sheaf condition}\label{subsec:excision}

The goal of this section is to prove a convenient reformulation of the sheaf condition in terms of an \textit{excision} property.
We do not make use of the reformation in this text, but present it here because it is the manifold analogue of \textit{Nisnevich excision} from algebraic geometry \cites[\SAGthm{Proposition}{B.5.1.1}]{SAG}[\S3.2]{MR3679884}[\S3.1, Proposition 1.4]{MR1813224}.
Another way of explaining the following result is that it says that a presheaf on $ \Man $ is a sheaf if and only if it satisfies the \textit{Mayer--Vietoris} property and transforms countable increasing chains of open submanifolds to limits.

\begin{theorem}[{\cite[Theorem 5.1]{BEBdBP:Classifying}}]\label{thm:excision}
	Let $ C $ be a presentable \category.
	A $ C $-valued presheaf $ F \colon \fromto{\Manop}{C} $ on $ \Man $ is a sheaf if and only if $ F $ satisfies the following conditions:
	\begin{enumerate}[{\upshape (\ref*{thm:excision}.1)}]
		\item\label{thm:excision.1} The object $ F(\emptyset) $ is terminal in $ C $.

		\item\label{thm:excision.2} For every manifold $ M $ and pair of open subsets $ U, V \subset M $ such that $ U \union V = M $, the induced square 
		\begin{equation*}
			\begin{tikzcd}
				F(M) \arrow[r] \arrow[d] & F(V) \arrow[d] & \\
				F(U) \arrow[r] & F(U \intersect V) 
			\end{tikzcd}
		\end{equation*}
		is a pullback square in $ C $.

		\item\label{thm:excision.3} For every manifold $ M $ and $ \NN $-indexed sequence of open sets
		\begin{equation*}
			U_0 \subset U_1 \subset \cdots \subset M
		\end{equation*}
		such that $ \Union_{n \geq 0} U_n = M $, the induced morphism
		\begin{equation*}
			\fromto{F(M)}{\lim_{n \geq 0} F(U_n)}
		\end{equation*}
		is an equivalence in $ C $.
	\end{enumerate}
\end{theorem}

We do not have occasion to use \Cref{thm:excision} in this text, but include it for completeness and because it is useful.
For example, \Cref{thm:excision} is crucial to work of Berwick-Evans--Boavida de Brito--Pavlov \cite{BEBdBP:Classifying} extending results of Madsen--Weiss \cite[Appendix A]{MR2335797}.
See \Cref{remark:BE-BdB-P} for more details.

The idea of \Cref{thm:excision} is as follows.
Conditions \enumref{thm:excision}{1} and \enumref{thm:excision}{2} guarantee that $ F $ satisfies the sheaf condition with respect to finite open covers.
Given descent with respect to finite open covers, by writing a countable cover as a union of a sequence of finite covers of smaller subspaces, \enumref{thm:excision}{3} implies descent with respect to \textit{countable} open covers.
Note that implicit in \Cref{thm:excision} is the claim that descent with respect to countable open covers implies descent with respect to arbitrary open covers.

Since the sheaf condition on $ \Man $ is defined after restriction to each manifold, \Cref{thm:excision} follows from an analogous rephrasing of the sheaf condition for a presheaf on an individual manifold (\Cref{prop:Lindelofexcision}).
The manifold structure isn't really used here; all that is necessary is that an open cover of an open subset of a manifold admits a countable subcover.
Hence we work at this level of generality.

\begin{observation}\label{obs:descfinitecovers}
	Let $ T $ be a topological space and $ C $ a presentable \category.
	Since limits of finite cubes can be written as iterated pullbacks, the following are equivalent for a presheaf $ F \in \PSh(T;C) $ on $ T $:
	\begin{enumerate}[{\upshape (\ref*{obs:descfinitecovers}.1)}]
		\item The presheaf $ F $ satisfies descent with respect to nonempty finite covers.

		\item For all opens $ U, V \subset T $, the induced square 
		\begin{equation*}
			\begin{tikzcd}
				F(U \union V) \arrow[r] \arrow[d] & F(V) \arrow[d] & \\
				F(U) \arrow[r] & F(U \intersect V) 
			\end{tikzcd}
		\end{equation*}
		is a pullback square in $ C $.
	\end{enumerate}
\end{observation}

\begin{recollection}\label{rec:Lindelof}
	A topological space $ T $ is \textit{Lindelöf} if every open cover of $ T $ has a countable subcover.

	The following conditions are equivalent for a topological space $ T $:
	\begin{enumerate}[(\ref*{rec:Lindelof}.1)]
		\item\label{rec:Lindelof.1} Every open subspace of $ T $ is Lindelöf.

		\item\label{rec:Lindelof.2} Every subspace of $ T $ is Lindelöf.
	\end{enumerate}
	We say that $ T $ is \textit{hereditarily Lindelöf} if $ T $ satisfies the equivalent conditions \enumref{rec:Lindelof}{1}--\enumref{rec:Lindelof}{2}.

	Note that every second-countable topological space (e.g., manifold) is hereditarily Lindelöf.
\end{recollection}

\begin{lemma}\label{lem:sheafLindelof}
	Let $ T $ be a hereditarily Lindelöf topological space and $ C $ a presentable \category.
	The following are equivalent for a presheaf $ F \in \PSh(T;C) $ on $ T $:
	\begin{enumerate}[{\upshape(\ref*{lem:sheafLindelof}.1)}]
		\item\label{lem:sheafLindelof.1} The presheaf $ F $ is a sheaf on $ T $.

		\item\label{lem:sheafLindelof.2} The presheaf $ F $ satisfies descent with respect to countable open covers.
	\end{enumerate}
\end{lemma}

\begin{proof}
	Clearly \enumref{lem:sheafLindelof}{1} $ \Rightarrow $ \enumref{lem:sheafLindelof}{2}.
	To see that \enumref{lem:sheafLindelof}{2} $ \Rightarrow $ \enumref{lem:sheafLindelof}{1}, let $ U \subset T $ be open and let $ \Uu $ be an open cover of $ U $
	Since $ T $ is hereditarily Lindelöf, there exists a countable subset $ \Uu_0 \subset \Uu $ that also covers $ U $. 
	To conclude, note that have a commutative triangle
	\begin{equation*}
		\begin{tikzcd}[sep=1.5em]
			& F(U) \arrow[dl] \arrow[dr, "\sim"{sloped, yshift=-0.2em}] & \\
			\displaystyle\lim_{V \in \Iup(\Uu)^{\op}} F(V) \arrow[rr, "\sim"{yshift=-0.2em}] & & \displaystyle\lim_{V \in \Iup(\Uu_0)^{\op}} F(V) \comma
		\end{tikzcd}
	\end{equation*}
	where the right-hand diagonal morphism is an equivalence by \enumref{lem:sheafLindelof}{2} and the horizontal morphism is an equivalence because the inclusion $ \Iup(\Uu_0)^{\op} \subset \Iup(\Uu)^{\op} $ is limit-cofinal.
\end{proof}

Now we provide a characterization of sheaves on a hereditarily Lindelöf topological space in terms of an excision property.
This characterization immediately implies \Cref{thm:excision}.

\begin{proposition}\label{prop:Lindelofexcision}
	Let $ T $ be a hereditarily Lindelöf topological space and $ C $ a presentable \category.
	A $ C $-valued presheaf $ F \in \PSh(T;C) $ on $ T $ is a sheaf if and only if $ F $ satisfies the following conditions:
	\begin{enumerate}[{\upshape (\ref*{prop:Lindelofexcision}.1)}]
		\item\label{prop:Lindelofexcision.1} The object $ F(\emptyset) $ is terminal in $ C $.

		\item\label{prop:Lindelofexcision.2} For all opens $ U, V \subset T $, the induced square 
		\begin{equation*}
			\begin{tikzcd}
				F(U \union V) \arrow[r] \arrow[d] & F(V) \arrow[d] & \\
				F(U) \arrow[r] & F(U \intersect V) 
			\end{tikzcd}
		\end{equation*}
		is a pullback square in $ C $.

		\item\label{prop:Lindelofexcision.3} For every $ \NN $-indexed sequence of open sets $ U_0 \subset U_1 \subset \cdots \subset T $, the induced morphism
		\begin{equation*}
			\fromto{F\paren{\textstyle\Union_{n \geq 0} U_n}}{\lim_{n \geq 0} F(U_n)}
		\end{equation*}
		is an equivalence in $ C $.
	\end{enumerate}
\end{proposition}

\begin{proof}
	First note that \enumref{prop:Lindelofexcision}{1} and \enumref{prop:Lindelofexcision}{2} are equivalent to saying that $ F $ satisfies descent with respect to finite covers.
	By \Cref{lem:sheafLindelof}, it suffices to show that $ F $ satisfies descent with respect to countable covers.

	Let $ V \subset T $ be open and $ \Uu = \{V_i\}_{i \in \NN} $ a countable open cover of $ V $.
	For each $ n \in \NN $, define
	\begin{equation*}
		U_n \colonequals \Union_{i=0}^{n} V_i \andeq \Uu_n \colonequals \{V_0,\ldots,V_n\} \period
	\end{equation*}
	Then $ \Uu_n $ is a finite open cover of $ U_n $ and we have inclusions $ U_n \subset U_{n+1} $ and $ \Uu_n \subset \Uu_{n+1} $.
	Note that the poset $ \Iup(\Uu) $ is the filtered union
	\begin{equation*}
		\Iup(\Uu) = \colim_{n \geq 0} \Iup(\Uu_n) \period
	\end{equation*}
	Since $ F $ satisfies descent with respect to finite covers, by \enumref{prop:Lindelofexcision}{3} we see that we have natural equivalences
	\begin{align*}
		F(V) &\equivalence \lim_{n \geq 0} F(U_n) \\ 
		&\equivalence \lim_{n \geq 0} \lim_{U \in \Iup(U_n)^{\op}} F(U) \\ 
		&\equivalent \lim_{U \in \Iup(\Uu)^{\op}} F(U) \period
	\end{align*}
	Hence $ F $ satisfies descent with respect to the countable cover $ \Uu $, as desired.
\end{proof}

\begin{proof}[Proof of \Cref{thm:excision}]
	Since manifolds are second-countable and open subsets of manifolds are manifolds, the claim is immediate from \Cref{prop:Lindelofexcision} and the definition of what it means to be a sheaf on $ \Man $ (\Cref{def:sheafonMan}).
\end{proof}

%-------------------------------------------------------------------%
%  Digression: relation to infinite dimensional manifolds           %
%-------------------------------------------------------------------%

\subsection{Digression: relation to infinite dimensional manifolds}\label{subsec:infdimMan}

We finish this chapter by describing a ``Yoneda embedding'' of infinite dimensional manifolds into sheaves of \textit{sets} on $ \Man $.

\begin{recollection}[(infinite dimensional manifolds)]
	There are two classes of possibly infinite dimensional manifolds that are commonly considered: \emph{Banach manifolds} and \emph{Fréchet manifolds} \cites[Chapter III, \S1]{MR0341518}[\S I.4]{MR656198}.  
	Banach spaces are examples of Fréchet spaces, and the category of Banach manifolds is a full subcategory of the
	category of Fréchet manifolds. 
\end{recollection}

One reason to consider Fréchet manifolds is that the (smooth) free loop space of a manifold naturally has the structure of a Fréchet manifold:

\begin{example}\label{ex:mappingFrechet}
	If $ M $ and $ N $ are manifolds, and $ M $ is compact, then the topological space $ \Cinf(M,N) $ of smooth maps $ \fromto{M}{N} $ has a natural Fréchet manifold structure.  
	See \cite[Chapter III, §1]{MR0341518}, in particular \cite[Chapter III, Theorem 1.11]{MR0341518}, for details.
\end{example}

\begin{notation}
	We write $ \Fre $ for the category of Fréchet manifolds.
	Note that $ \Man $ is the full subcategory of $ \Fre $ spanned by the finite dimensional Fréchet manifolds.
\end{notation}

\begin{construction}\label{construction:embedBanFre}
	Define a restricted Yoneda functor $ y_{\Fre} $ by
	\begin{align*}
		y_{\Fre} \colon \Fre &\to \PSh(\Man;\Set) \\
		F &\mapsto [M \mapsto \Map_{\Fre}(M,F)] \period
	\end{align*}
	Notice that since morphisms of Fréchet manifolds are defined locally, the assignment $ M \mapsto \Map_{\Fre}(M,F) $ is a sheaf on $ \Man $.
	That is, $ y_{\Fre} $ factors through $ \Sh(\Man;\Set) $.
\end{construction}

\begin{theorem}[{(Hain \cite{MR539632}, Losik \cites{MR1213569}[Theorem 3.1.1]{MR1307261}[Theorem A.1.5]{MR2905777})}]
	The functor $ y_{\Fre} \colon \fromto{\Fre}{\Sh(\Man;\Set)} $ is fully faithful. 
\end{theorem}

The next result about infinite dimensional manifolds is that the embedding $ y_{\Fre} $ sends Fréchet manifold of smooth maps from a compact manifold to an arbitrary manifold (\Cref{ex:mappingFrechet}) to the internal-$ \Hom $ in $ \Sh(\Man;\Set) $.
In particular, free loop spaces are correctly represented in $ \Sh(\Man;\Set) $.
To state this result, let us first recall the internal-$ \Hom $ in sheaves on $ \Man $.

\begin{recollection}[(cartesian closedness)]
	Like any topos, the category $ \Sh(\Man;\Set) $ of sheaves of sets on $ \Man $ is \textit{cartesian closed}.
	In particular, $ \Sh(\Man;\Set) $ has an internal-$ \Hom $ defined by
	\begin{align*}
		\Hom_{\Sh(\Man;\Set)}(E,E') \colon \Manop &\to \Set \\
		M &\mapsto \Map_{\Sh(\Man;\Set)}(E \cross M, E') \period
	\end{align*}
\end{recollection}

\begin{theorem}[{(Waldorf \cite[Lemma A.1.7]{MR2905777})}]
	Let $ M $ and $ N $ be manifolds.
	If $ M $ is compact, then there is a natural isomorphism
	\begin{equation*}
		y_{\Fre}(\Cinf(M,N)) \isomorphic \Hom_{\Sh(\Man;\Set)}(M,N) \period
	\end{equation*}
\end{theorem}

We finish this section by explaining how a commonly used enlargement of the category of Fréchet manifolds fits into the category $ \Sh(\Man;\Set) $.

\begin{remark}[(diffeological spaces)]
	Souriau introduced \cite{MR607688} \emph{diffeological spaces} as generalization of manifolds to include infinite dimensional manifolds as well as manifold-like spaces appearing in mathematical physics.
	Diffeological spaces have been extensively studied by Iglesias-Zemmour and collaborators \cites{MR799609}{MR844156}{MR906897}{MR906897}{MR2346968}{MR2424313}{MR3025051}{MR2846342}{MR2592936}.

	To explain how diffeological spaces fit into sheaves on manifolds, write 
	\begin{equation*}
		\yo \colon \incto{\Euc}{\Sh(\Euc;\Set) \equivalent \Sh(\Man;\Set)}
	\end{equation*}
	for the Yoneda embedding.
	A \emph{diffeological space} is a sheaf $ E $ on $ \Euc $ such that for each $ n \geq 0 $, the natural map
	\begin{equation*}
		E(\RR^n) \isomorphic \Map_{\Sh(\Euc;\Set)}(\yo(\RR^n),E) \to \Map_{\Set}(\yo(\RR^n)(*),E(*)) = \Map_{\Set}(\RR^n,E(*))
	\end{equation*}
	is injective.
	This injectivity condition also allows a diffeological space to be described as a set $ X $ equipped with a
	collection of ``plots''
	\begin{equation*}
	 	\Cinf(\RR^n,X) \subset \Map_{\Set}(\RR^n,X)
	\end{equation*}
	for each $ n \geq 0 $, subject to a collection of explicit conditions that are equivalent to saying that the assignment 
	\begin{equation*}
		\goesto{\RR^n}{\Cinf(\RR^n,X)}
	\end{equation*}
	is a sheaf on $ \Euc $.
	(To match up notation, $ X = E(*) $ and $ \Cinf(\RR^n,X) = E(\RR^n) $.)
\end{remark}

