%!TEX root = ../diffcoh.tex

%-------------------------------------------------------------------%
%-------------------------------------------------------------------%
%  Structures in the stable case                                    %
%-------------------------------------------------------------------%
%-------------------------------------------------------------------%

\section{Structures in the stable case}\label{sec:stable}
\textit{by Peter Haine}

In ordinary differential cohomology, we had the Simons--Sullivan ``differential cohomology hexagon''
\begin{equation*}\label{diag:SimonsSullivan}
	\begin{tikzcd}[column sep={10ex,between origins}, row sep={8ex,between origins}]
		0 \arrow[dr] & & & & 0 \\
		& \H^{*-1}(M;\RR/\ZZ) \arrow[rr, "-\Bock"] \arrow[dr] & & \H^*(M;\ZZ) \arrow[dr] \arrow[ur] & \\
		\HdR^{*-1}(M) \arrow[ur] \arrow[dr] & & \Hhat^*(M;\ZZ) \arrow[ur] \arrow[dr] & & \HdR^*(M) \\
		& \frac{\Omega^{*-1}(M)}{\Omegacl^{*-1}(M)_{\ZZ}} \arrow[rr, "\d"'] \arrow[ur] & & \Omegacl^*(M)_{\ZZ} \arrow[ur] \arrow[dr] & \\
		0 \arrow[ur] & & & & 0 \comma
	\end{tikzcd}
\end{equation*}
which actually characterized ordinary differential cohomology (\Cref{thm:SimonsSullivanunique}).
We want to be able to reproduce an analogue of the differential cohomology hexagon for \textit{any} sheaf of spectra on $ \Man $.
To do this, we need to identify how cohomology with coefficients in $ \RR/\ZZ $, $ \ZZ $, and $ \RR $ as well as $ \Omega^{*-1}(M)/\Omegacl^{*-1}(M)_{\ZZ} $ and $ \Omegacl^*(M)_{\ZZ} $ fit into the story.

One general machine for producing diagrams aesthetically similar to the differential cohomology hexagon is the
theory of \textit{recollements}, or ways of ``gluing'' a category together out of two pieces.
It turns out that the differential cohomology hexagon falls exactly into this framework: one of the subcategories that we build $ \Sh(\Man;\Sp) $ from is the subcategory $ \Shhi(\Man;C) $ of $ \RR $-invariant sheaves, and the other piece is the subcategory of sheaves with vanishing global sections.
Since this whole story is a special case of the theory of recollements, the first half of the section (\cref{subsec:recollement}) gives a quick introduction to the theory of recollements and the key results.
In \cref{subsec:fracture}, we apply this general machinery to sheaves on manifolds to obtain the a version of differential cohomology hexagon for any sheaf of spectra on $ \Mfld $ (see \cref{nul:spectraldiffcohdiag}).
We finish the section by making precise what it means for a sheaf of spectra on $ \Mfld $ to ``refine'' a cohomology theory.


%-------------------------------------------------------------------%
%-------------------------------------------------------------------%
%  Background on recollements                                       %
%-------------------------------------------------------------------%
%-------------------------------------------------------------------%

\subsection{Background on recollements}\label{subsec:recollement} 

\textit{Recollements}%
\footnote{Roughly, the French verb \textit{recoller} means ``to glue back together''.} %
were introduced by Grothendieck and Verdier in the context of topoi \cites[Exposé IV, \S9]{MR50:7130} and by
Beĭlinson--Bernstein--Deligne in the context of triangulated categories \cite[\S1.4]{MR751966} to ``glue'' together sheaves over open-closed decompositions of a space.
However, there are many other situations in which \acategory can be ``glued together'' from two subcategories that are in some sense complementary.
For example, if $ R $ is a ring and $ I \subset R $ is a finitely generated ideal, then the derived \category of $ R $ can be clued together from its subcategories of $ I $-nilpotent and $ I $-local objects.

The goal of this section is to explain this general theory and how it can be applied to the context of sheaves of spectra on the category of manifolds.
The key insight is that given a stable \category $ \Xcat $ and a full subcategory $ \ilowerstar \colon \incto{\Zcat}{\Xcat} $ that is both localizing and colocalizing
\begin{equation*}
	\begin{tikzcd}[sep=3.5em]
		\Zcat \arrow[r, "\ilowerstar" description, hooked] & \Xcat \comma \arrow[l, shift left=1.25ex, "\iuppershriek"] \arrow[l, shift right=1.25ex, "\iupperstar"'] 
	\end{tikzcd}
\end{equation*}
the \category $ \Xcat $ can be glued together from the subcategory $ \Zcat $ and the subcategory $ \Zrorth \subset \Xcat $ \textit{right orthogonal} to $ \Xcat $ (\Cref{prop:orthogonaladjoints,cor:stablerecollement}).
That is, $ \Zrorth $ is the subcategory of objects of $ \Xcat $ that admit no nontrivial maps \textit{from} objects of $ \Zcat $.
This applies to the situation of interest because we have both a left and right adjoint
\begin{equation*}
	\begin{tikzcd}[sep=3.5em]
		\Shhi(\Man;\Sp) \arrow[r, hooked] & \Sh(\Man;\Sp) \arrow[l, shift left=1.25ex, "\Rhi"] \arrow[l, shift right=1.25ex, "\Lhi"']
	\end{tikzcd}
\end{equation*}
to the inclusion of $ \RR $-invariant sheaves on $ \Mfld $ into all sheaves \Cref{nul:Gammaadjunctions}.
We'll apply the general theory studied in this section to the context of sheaves on $ \Mfld $ in \cref{subsec:fracture}.
 
%-------------------------------------------------------------------%
%  Motivation                                                       %
%-------------------------------------------------------------------%

\subsubsection{Motivation}\label{subsec:recollementmotivation}

To explain the motivation for recollements, let $ X $ be a topological space and $ Z \subset X $ a closed subspace.
Write $ U \colonequals X \smallsetminus Z $ for the open complement of $ Z $ in $ X $, and write 
\begin{equation*}
	i \colon \incto{Z}{X} \andeq j\colon \incto{U}{X}
\end{equation*}
for the inclusions.
Any sheaf $ F $ of sets on $ X $ pulls back to sheaves
\begin{equation*}
	F_Z \colonequals \iupperstar(F) \andeq F_U \colonequals \jupperstar(F)
\end{equation*}
on $ Z $ and $ U $, respectively.
Moreover, the sheaf $ F $ is completely determined by the sheaves $ F_Z $ and $ F_U $ in the following sense.
Applying $ \iupperstar $ to the unit $ \unit \colon \fromto{F}{\jlowerstar\jupperstar(F)} $, we obtain a natural morphism
\begin{equation*}
	u \colon F_Z = \iupperstar(F) \to \iupperstar\jlowerstar\jupperstar(F) = \iupperstar\jlowerstar(F_U) \period
\end{equation*}
The triangle identities imply that there is a commutative square
\begin{equation}\label{sq:fracturemotivation}
	\begin{tikzcd}
		F \arrow[r] \arrow[d] & \jlowerstar(F_U) \arrow[d] \\
		\ilowerstar(F_Z) \arrow[r, "\ilowerstar(u)"'] & \ilowerstar\iupperstar\jlowerstar(F_U) \comma
	\end{tikzcd}
\end{equation}
where the three morphisms
\begin{equation*}
	F \to \ilowerstar\iupperstar(F) = \ilowerstar(F_Z) \, , \quad F \to \jlowerstar\jupperstar(F) = \jlowerstar(F_U) \, , \andeq \jlowerstar(F_U) \to \ilowerstar\iupperstar\jlowerstar(F_U)
\end{equation*}
are all unit morphisms.
One can show that the square \eqref{sq:fracturemotivation} is in a \textit{pullback square}.
This provides an explicit way to reconstruct $ F $ from the data of the sheaves $ F_Z $ and $ F_U $ along with the morphism $ u \colon \fromto{F_Z}{\iupperstar \jlowerstar(F_U)} $.

In fact, even more is true.
The whole \textit{category} $ \Sh(X;\Set) $ can be reconstructed from the categories $ \Sh(Z;\Set) $ and $ \Sh(U;\Set) $ together with the functor $ \iupperstar\jlowerstar \colon \fromto{\Sh(U;\Set)}{\Sh(Z;\Set)} $ in the following sense.
Write $ [1] $ for the ``walking arrow'' poset $ \{0 < 1\} $.
There is a pullback square of categories
\begin{equation}\label{sq:fractureCatmotivation}
	\begin{tikzcd}
		\Sh(X;\Set) \arrow[r] \arrow[d, "\jupperstar"'] & \Fun([1],\Sh(Z;\Set)) \arrow[d, "\target"] \\
		\Sh(U;\Set) \arrow[r, "\iupperstar\jlowerstar"'] & \Sh(Z;\Set) \period
	\end{tikzcd}
\end{equation}
Here the unlabeled top horizontal arrow sends a sheaf $ F \in \Sh(X;\Set) $ to the morphism given by applying $ \iupperstar $ to the unit $ \fromto{F}{\jlowerstar\jupperstar(F)} $.
More explicitly, an object of $ \Sh(X;\Set) $ is equivalent to the data of a sheaf $ F_{Z} $ on $ Z $, a sheaf $ F_{U} $ on $ U $, and a \textit{gluing morphism} $ \fromto{F_Z}{\iupperstar\jlowerstar(F_U)} $.
Morphisms are morphisms of sheaves on $ Z $ and $ U $ commuting with the specified gluing morphisms.

In the rest of this section, we explain the general categorical framework for decompositions of this form.
We do not explain the proofs of the results presented in this section; for those, the reader should consult \cites[\HAsec{A.8}]{HA}[\SAGsec{7.2}]{SAG}{arXiv:1607.02064}.


%-------------------------------------------------------------------%
%  Definitions and general results                                  %
%-------------------------------------------------------------------%

\subsubsection{Definitions and general results} 

Now we generalize the situation for sheaves explained in \cref{subsec:recollementmotivation}.
The following are the key features of the situation.

\begin{definition}\label{def:recollement}
	Let $ \Xcat $ be \acategory with finite limits.
	Fully faithful functors
	\begin{equation*}
		\ilowerstar \colon \incto{\Zcat}{\Xcat} \andeq \jlowerstar \colon \incto{\Ucat}{\Xcat}
	\end{equation*}
	exhibit $ \Xcat $ as the \textit{recollement} of $ \Zcat $ and $ \Ucat $ if:
	\begin{enumerate}[(\ref*{def:recollement}.1)]
		\item\label{def:recollement.1} The functors $ \ilowerstar $ and $ \jlowerstar $ admit left exact left adjoints $ \iupperstar $ and $ \jupperstar $, respectively.

		\item\label{def:recollement.2} The functor $ \jupperstar \ilowerstar \colon \fromto{\Zcat}{\Ucat} $ is constant at the terminal object of $ \Ucat $.

		\item\label{def:recollement.3} The functors $ \iupperstar \colon \fromto{\Xcat}{\Zcat} $ and $ \jupperstar \colon \fromto{\Xcat}{\Ucat} $ are jointly conservative.
		That is, a morphism $ f $ in $ \Xcat $ is an equivalence if and only if both $ \iupperstar(f) $ and $ \jupperstar(f) $ are equivalences.
	\end{enumerate}

	We refer to the subcategory $ \Zcat \subset \Xcat $ as the \textit{closed} subcategory, and $ \Ucat \subset \Xcat $ as the \textit{open} subcategory.
\end{definition}

\begin{remark}
	Note that \enumref{def:recollement}{2} in particular implies that the are \textit{no} nontrivial maps from objects in $ \Zcat \subset \Xcat $ to objects in $ \Ucat \subset \Xcat $.
\end{remark}

\begin{warning}
	Note that the condition that $ \Xcat $ be the recollement of $ \Zcat $ and $ \Ucat $ is \textit{not} symmetric: if $ \Xcat $ is the recollement of $ \Zcat $ and $ \Ucat $, then $ \Xcat $ need note be the recollement of $ \Ucat $ and $ \Zcat $.
	For example, the composite $ \iupperstar \jlowerstar $ is not usually constant at the terminal object of $ \Zcat $.  
\end{warning}

The two most important examples of recollements from topology and algebraic geometry are the following:

\begin{example}\label{ex:Shrecollement}
	Let $ X $ be a topological space, $ i \colon \incto{Z}{X} $ a closed subspace, and $ j \colon \incto{U}{X} $ the open complement of $ Z $ in $ X $.
	Let $ C $ be a presentable \category that is compactly generated or stable.
	Then the pushforward functors $ \ilowerstar \colon \incto{\Sh(Z;C)}{\Sh(X;C)} $ and $ \jlowerstar \colon \incto{\Sh(U;C)}{\Sh(X;C)} $ exhibit $ \Sh(X;C) $ as the recollement of $ \Sh(Z;C) $ and $ \Sh(U;C) $.
	See \cites[\HAappthm{Remark}{A.8.16}]{HA}[Corollaries 2.12 \& 2.23]{arXiv:2108.03545}
\end{example}

\begin{example}\label{ex:QCohrecollement}
	Let $ X $ be a scheme, $ \incto{Z}{X} $ a closed subscheme, and $ \incto{U}{X} $ the complementary open subscheme in $ X $.
	Assume that $ U $ is quasicompact.
	We write $ \QCoh(X) $ and $ \QCoh(U) $ for the stable \categories of quasicoherent sheaves on $ X $ and $ U $, respectively.
	We write $ \QCoh_{Z}(X) \subset \QCoh(X) $ for the full subcategory spanned by those quasicoherent sheaves that are set-theoretically supported on $ Z $.
	Then the pushforward $ \incto{\QCoh(U)}{\QCoh(X)} $ and the inclusion $ \QCoh_{Z}(X) \subset \QCoh(X) $ exhibit $ \QCoh(X) $ as the recollement of $ \QCoh(U) $ and $ \QCoh_{Z}(X) $.
	See, for example, \SAG{Proposition}{7.2.3.1}.
\end{example}

\begin{warning}
	In \Cref{ex:QCohrecollement}, note that the subcategory $ \QCoh(U) $ is the \textit{closed} subcategory, and the subcategory $ \QCoh_{Z}(X) $ is the \textit{open} subcategory.
	There are thus two competing naming conventions for the ``closed'' and ``open'' subcategories: one coming from the theory of sheaves on topological spaces (\Cref{ex:Shrecollement}), and one coming from quasicoherent sheaves on schemes (\Cref{ex:QCohrecollement}).
	Both are used in the literature, depending on whether one is working in a ``topological'' or
	``algebro-geometric'' context. 
	In this text we use the ``topological'' convention.
\end{warning}

The following result explains how to reconstruct a recollement from the closed and open subcategories together with \textit{gluing functor} $ \iupperstar\jlowerstar \colon \fromto{\Ucat}{\Zcat} $.

\begin{theorem}[{\cites[\HAappthm{Corollary}{A.8.13}, \HAappthm{Remark}{A.8.5}, \& \HAappthm{Proposition}{A.8.17}]{HA}[1.17]{arXiv:1909.03920}}]\label{thm:recollfracture}
	Let $ \ilowerstar \colon \incto{\Zcat}{\Xcat} $ and $ \jlowerstar \colon \incto{\Ucat}{\Xcat} $ be functors that exhibit $ \Xcat $ as the recollement of $ \Zcat $ and $ \Ucat $.
	There is a pullback square of \categories
	\begin{equation*}
		\begin{tikzcd}
			\Xcat \arrow[r] \arrow[d, "\jupperstar"'] & \Fun([1],\Zcat) \arrow[d, "\target"] \\
			\Ucat \arrow[r, "\iupperstar\jlowerstar"'] & \Zcat \period
		\end{tikzcd}
	\end{equation*}
	Here the unlabeled top horizontal arrow sends an object $ F \in \Xcat $ to the morphism given by applying $ \iupperstar $ to the unit $ \fromto{F}{\jlowerstar\jupperstar(F)} $.

	As a consequence, there is a pullback square of endofunctors of $ \Xcat $
	\begin{equation}\label{sq:fracturegeneral}
		\begin{tikzcd}
			\id{\Xcat} \arrow[r] \arrow[d] & \jlowerstar\jupperstar \arrow[d] \\
			\ilowerstar\iupperstar \arrow[r] & \ilowerstar\iupperstar\jlowerstar\jupperstar \period
		\end{tikzcd}
	\end{equation}
	Here the top horizontal and left vertical morphisms are the unit morphisms, the bottom horizontal morphism is obtained by applying $ \ilowerstar\iupperstar $ to the unit morphism $ \fromto{\id{X}}{\jlowerstar\jupperstar} $, and the right vertical morphism is obtained by precomposing the unit morphism $ \fromto{\id{X}}{\ilowerstar\iupperstar} $ with $ \jlowerstar\jupperstar $.
\end{theorem}

\begin{definition}
	Let $ \ilowerstar \colon \incto{\Zcat}{\Xcat} $ and $ \jlowerstar \colon \incto{\Ucat}{\Xcat} $ be functors that exhibit $ \Xcat $ as the recollement of $ \Zcat $ and $ \Ucat $.
	The pullback square \eqref{sq:fracturegeneral} is referred to as the \textit{fracture square} of the recollement.
\end{definition}

Often the functors $ \ilowerstar $ and $ \jupperstar $ admit further adjoints.

\begin{theorem}[{\cites[\HAappthm{Corollary}{A.8.7}, \HAappthm{Remark}{A.8.8}, \& \HAappthm{Proposition}{A.8.11}]{HA}[Corollary 1.10]{arXiv:1909.03920}}]\label{thm:stabrecollement}
	Let $ \ilowerstar \colon \incto{\Zcat}{\Xcat} $ and $ \jlowerstar \colon \incto{\Ucat}{\Xcat} $ be functors that exhibit $ \Xcat $ as the recollement of $ \Zcat $ and $ \Ucat $.
	\begin{enumerate}[{\upshape (\ref*{thm:stabrecollement}.1)}]
		\item\label{thm:stabrecollement.1} If the \category $ \Zcat $ has an initial object, then $ \jupperstar $ admits a fully faithful left adjoint $ \jlowershriek \colon \incto{\Ucat}{\Xcat} $.

		\item\label{thm:stabrecollement.2} If, moreover, $ \Xcat $ has a zero object, then $ \ilowerstar $ admits a right adjoint $ \iuppershriek \colon \fromto{\Xcat}{\Zcat} $ characterized by the property that 
		\begin{equation*}
			\ilowerstar\iuppershriek \equivalent \fib(\eta \colon \id{\Xcat} \to \jlowerstar\jupperstar) \period
		\end{equation*}
		In particular, applying $ \iupperstar $, there is a fiber sequence
		\begin{equation*}
			\begin{tikzcd}
				\iuppershriek \arrow[r] & \iupperstar \arrow[r, "\iupperstar\unit"] & \iupperstar\jlowerstar\jupperstar \period
			\end{tikzcd}
		\end{equation*}

		\item\label{thm:stabrecollement.3} If $ \Xcat $ is stable, then $ \Zcat $ and $ \Ucat $ are also stable.
		Moreover, there is a canonical fiber sequence
		\begin{equation*}
			\begin{tikzcd}
				\jlowershriek \jupperstar \arrow[r] & \id{\Xcat} \arrow[r] & \ilowerstar \iupperstar \comma
			\end{tikzcd}
		\end{equation*}
		where the first morphism is the counit and the second is the unit.

		\item\label{thm:stabrecollement.4} If $ \Xcat $ is presentable and the gluing functor $ \iupperstar \jlowerstar $ is accessible, then $ \Zcat $ and $ \Ucat $ are presentable.
	\end{enumerate}
\end{theorem}

\begin{nul}\label{nul:stabelrecadjunction}
	Thus, if $ \Xcat $ is stable, there is a chain of adjunctions
	\begin{equation*}
		\begin{tikzcd}[sep=3.5em]
			\Zcat \arrow[r, "\ilowerstar" description, hooked] & \Xcat \arrow[l, shift left=1.25ex, "\iuppershriek"] \arrow[l, shift right=1.25ex, "\iupperstar"'] \arrow[r, "\jupperstar" description]  & \Ucat \period \arrow[l, shift left=1.25ex, hooked', "\jlowerstar"] \arrow[l, shift right=1.25ex, hooked', "\jlowershriek"']
		\end{tikzcd}
	\end{equation*}
\end{nul}

We're interested in applying this to the situation where $ \ilowerstar $ is the inclusion of $ \Shhi(\Man;\Sp) $ into $ \Sh(\Man;\Sp) $, $ \iupperstar $ is $ \Lhi $, and $ \iuppershriek $ is $ \Rhi $.
To get an analogue of the ``differential cohomology hexagon'', we need to enlarge the fracture square \eqref{sq:fracturegeneral} using the fiber sequences from \enumref{thm:stabrecollement}{2} and \enumref{thm:stabrecollement}{3} along with one more.

\begin{construction}[(norm map)]
	Let $ \Xcat $ and $ \Ucat $ be \categories, and suppose we are given adjunctions
	\begin{equation*}
		\begin{tikzcd}[sep=3.5em]
			\Xcat \arrow[r, "\jupperstar" description] & \Ucat  \arrow[l, shift left=1.25ex, hooked', "\jlowerstar"] \arrow[l, shift right=1.25ex, hooked', "\jlowershriek"']
		\end{tikzcd}
	\end{equation*}
	where left adjoint $ \jlowershriek $ and right adjoint $ \jlowerstar $ are fully faithful.
	Write $ \counit \colon \fromto{\jupperstar\jlowerstar}{\id{\Ucat}} $ for the counit.
	Since $ \jlowershriek $ is left adjoint to $ \jupperstar $ and the counit $ \counit $ is an equivalence, we have equivalences
	\begin{equation}\label{eq:normequiv}
		\begin{tikzcd}[sep=3.5em]
			\Map(\jlowershriek,\jlowerstar) \equivalent \Map(\id{\Ucat},\jupperstar\jlowerstar) \arrow[r, "\counit \of -"', "\sim"] & \Map(\id{\Ucat},\id{\Ucat}) \period
		\end{tikzcd} 
	\end{equation}
	The \textit{norm} natural transformation $ \Nm \colon \fromto{\jlowershriek}{\jlowerstar} $ is the natural transformation corresponding to the identity $ \fromto{\id{\Ucat}}{\id{\Ucat}} $ under the equivalence \eqref{eq:normequiv}.
\end{construction}

\begin{theorem}\label{thm:fracturehexagon}
	Let $ \Xcat $ be a stable \category and let $ \ilowerstar \colon \incto{\Zcat}{\Xcat} $ and $ \jlowerstar \colon \incto{\Ucat}{\Xcat} $ be functors that exhibit $ \Xcat $ as the recollement of $ \Zcat $ and $ \Ucat $.
	Then the sequence
	\begin{equation*}
		\begin{tikzcd}
			\jlowershriek \jupperstar \arrow[r, "\Nm\jupperstar"] & \jlowerstar\jupperstar \arrow[r] & \ilowerstar \iupperstar \jlowerstar\jupperstar 
		\end{tikzcd}
	\end{equation*}
	is a fiber sequence.
	As a consequence, the fracture square fits into a commutative diagram
	\begin{equation}\label{sq:fracturehexagon}
		\begin{tikzcd}[sep=2.5em]
			 &  \jlowershriek \jupperstar \arrow[r, equals] \arrow[d] & \jlowershriek \jupperstar \arrow[d, "\Nm\jupperstar"] \\
			\ilowerstar\iuppershriek \arrow[r] \arrow[d, equals] & \id{\Xcat} \arrow[r] \arrow[d] & \jlowerstar\jupperstar \arrow[d] \\
			\ilowerstar\iuppershriek \arrow[r] & \ilowerstar\iupperstar \arrow[r] & \ilowerstar\iupperstar\jlowerstar\jupperstar 
		\end{tikzcd}
	\end{equation}
	where all rows and columns are fiber sequences.
\end{theorem}

Aside from the explicit identification of the first map in the lower horizontal fiber sequence of \eqref{sq:fracturegeneral} with the norm map, \Cref{thm:fracturehexagon} can be deduced by applying the following characterization of pullback squares of stable \categories horizontally and vertically to the fracture square \eqref{sq:fracturegeneral}.

\begin{recollection}\label{rec:stablepullbackviafibers}
	Let $ C $ be a pointed \category and 
	\begin{equation}\label{sq:stablepullback}
		\begin{tikzcd}
			W \arrow[r, "\fbar"] \arrow[d] & Y \arrow[d] \\
			X \arrow[r, "f"'] & Z  
		\end{tikzcd}
	\end{equation}
	a commutative square in $ C $.
	Then there is a natural equivalence
	\begin{equation*}
		\fib(W \to X \cross_Z Y) \equivalent \fib(\fib(\fbar) \to \fib(f)) \period
	\end{equation*}
	In particular, if $ C $ is stable, then $ \equivto{\fib(\fbar)}{\fib(f)} $ if and only if the square \eqref{sq:stablepullback} is a pullback square.
	See \cites[\S2]{Chromfracture:BarthelAntolin}{MO:333239} for more details.
\end{recollection}

%-------------------------------------------------------------------%
%  Orthogonal complements & the stable situation                    %
%-------------------------------------------------------------------%

\subsubsection{Orthogonal complements \& the stable situation} 

In the stable case, it turns out that the data of a recollement of $ \Xcat $ is equivalent to the data of the closed subcategory $ \Zcat \subset \Xcat $.
The open subcategory $ \Ucat \subset \Xcat $ can be recovered as an \textit{orthogonal complement} to $ \Zcat $ in the following sense.

\begin{definition}\label{def:orthogonal}
	Let $ \Xcat $ be \acategory and $ \Zcat \subset \Xcat $ a full subcategory.
	\begin{enumerate}[(\ref*{def:orthogonal}.1)]
		\item We say that an object $ X \in \Xcat $ is \textit{right orthogonal} to the subcategory $ \Zcat $ if for each $ Z \in \Zcat $, the mapping space 
		$ \Map_{\Xcat}(Z,X) $ is contractible.

		\item We say that an object $ X \in \Xcat $ is \textit{left orthogonal} to the subcategory $ \Zcat $ if for each $ Z \in \Zcat $, the mapping space 
		$ \Map_{\Xcat}(X,Z) $ is contractible.
	\end{enumerate} 
	The \textit{right orthogonal complement} of $ \Zcat $ is the full subcategory $ \Zrorth \subset \Xcat $ spanned by those objects right orthogonal to $ \Zcat $.
	The \textit{left orthogonal complement} of $ \Zcat $ is the full subcategory $ \Zlorth \subset \Xcat $ spanned by those objects right orthogonal to $ \Zcat $.
\end{definition}

\begin{proposition}[{\cites[\SAGthm{Proposition}{7.2.1.10}]{SAG}[Lemmas 2 \& 5 and Proposition 7]{arXiv:1607.02064}}]\label{prop:orthogonaladjoints}
	Let $ \Xcat $ be a stable \category, and $ \ilowerstar \colon \incto{\Zcat}{\Xcat} $ a full subcategory.
	Assume that the inclusion $ \ilowerstar $ admits a left adjoint $ \iupperstar $ and a right adjoint $ \iuppershriek $.
	Then:
	\begin{enumerate}[{\upshape (\ref*{prop:orthogonaladjoints}.1)}]
		\item\label{prop:orthogonaladjoints.1} The inclusion $ \Zrorth \subset \Xcat $ admits a left adjoint $ j^{\orthogonal} \colon \fromto{\Xcat}{\Zrorth} $ defined as the cofiber
		\begin{equation*}
			j^{\orthogonal} \colonequals \cofib(\counit \colon \fromto{\ilowerstar\iuppershriek}{\id{\Xcat}}) \period
		\end{equation*}

		\item\label{prop:orthogonaladjoints.2} The inclusion $ \Zlorth \subset \Xcat $ admits a right adjoint $ ^{\orthogonal}j \colon \fromto{\Xcat}{\Zlorth} $ defined as the fiber
		\begin{equation*}
			^{\orthogonal}j \colonequals \fib(\unit \colon \fromto{\id{\Xcat}}{\iupperstar\ilowerstar}) \period
		\end{equation*}

		\item\label{prop:orthogonaladjoints.3} The composite functors
		\begin{equation*}
			\begin{tikzcd}
				\Zrorth \arrow[r, hooked] & \Xcat \arrow[r, "^{\smallorthogonal}j"] & \Zlorth
			\end{tikzcd}
			\andeq
			\begin{tikzcd}
				\Zlorth \arrow[r, hooked] & \Xcat \arrow[r, "j^{\smallorthogonal}"] & \Zrorth
			\end{tikzcd}
		\end{equation*}
		are inverse equivalences of \categories.

		\item\label{prop:orthogonaladjoints.4} The stable \category $ \Xcat $ is the recollement of the stable subcategories $ \Zcat $ and $ \Zrorth $.
	\end{enumerate}
\end{proposition}

\begin{corollary}\label{cor:stablerecollement}
	Let $ \Xcat $ be a stable \category, and let $ \ilowerstar \colon \incto{\Zcat}{\Xcat} $ and $ \jlowerstar \colon \incto{\Ucat}{\Xcat} $ be functors that exhibit $ \Xcat $ as the recollement of $ \Zcat $ and $ \Ucat $.
	Then the essential image of the fully faithful functor$ \jlowerstar $ is the right orthogonal complement $ \Zrorth $ of $ \Zcat $.
\end{corollary}

\noindent Said differently, \textit{every} stable recollement arises via \Cref{prop:orthogonaladjoints}.

\begin{remark}[(semiorthognal decompositions)]
	\Cref{prop:orthogonaladjoints,cor:stablerecollement} say that recollements are special types of \textit{semiorthogonal decompositions} of \categories.
	Semiorthogonal decompositions were originally introduced (in the context of triangulated categories) by Bondal and Kapranov \cite{MR1039961} to break apart stable \categories arising in algebraic geometry into more simple pieces.
	There are many beautiful examples (namely, Beĭlinson's celebrated semiorthogonal decomposition of $ \Coh(\PP^n) $ \cites{MR509388}{MR863137}) and connections to other important algebraic structures such as $ \mathrm{t} $-structures.
	The interested reader is encouraged to consult \cite[\SAGsec{7.2}]{SAG} as well as Antieau and Elmanto's recent work \cite{MR4205113}.
\end{remark}


%-------------------------------------------------------------------%
%-------------------------------------------------------------------%
%  Decomposing sheaves on manifolds                                 %
%-------------------------------------------------------------------%
%-------------------------------------------------------------------%

\subsection{Decomposing sheaves on manifolds}\label{subsec:fracture}

We now apply the framework of recollements introduced in \cref{subsec:recollement} to the case where $ \Xcat = \Sh(\Man;\Sp) $ and $ \Zcat = \Shhi(\Man;\Sp) $.
Since we can do so at no extra cost, we'll work in the more general setting of sheaves valued in a presentable stable \category. 
First, let's align our notation with \Cref{prop:orthogonaladjoints}.

\begin{nul}
	Let $ C $ be a presentable stable \category.
	Writing $ \Xcat = \Sh(\Man;C) $ and $ \Zcat = \Shhi(\Man;C) $, in the notation of \Cref{prop:orthogonaladjoints} we have $ \iupperstar = \Lhi $ and $ \iuppershriek = \Rhi $.
\end{nul}

\begin{definition}\label{def:pursheaf}
	Let $ C $ be a stable presentable \category.
	A sheaf $ \Ehat \colon \fromto{\Manop}{C} $ is \textit{pure} if $ \Ehat $ is right orthogonal to $ \Shhi(\Man;C) $. 
	We write
	\begin{equation*}
		\Shpure(\Man;C) \colonequals \Shhi(\Man;C)^{\orthogonal} \subset \Sh(\Man;C)
	\end{equation*}
	for the full subcategory spanned by the pure sheaves.
\end{definition}

\begin{observation}
	Recall that the subcategory $ \Shhi(\Man;C) $ is the essential image of the constant sheaf functor $ \Gammaupperstar \colon \incto{C}{\Sh(\Man;C)} $ (\Cref{prop:Dugger}).
	Let $ X \in C $ and $ \Ehat \in \Sh(\Man;C) $.
	Then 
	\begin{equation*}
		\Map_{\Sh(\Man;C)}(\Gammaupperstar(X),\Ehat) \equivalent \Map_C(X,\Gammalowerstar(\Ehat)) \period
	\end{equation*}
	Thus $ \Ehat $ is right orthogonal to $ \Shhi(\Man;C) $ if and only if 
	\begin{equation*}
		\Gammalowerstar(\Ehat) = \Ehat(*) = 0 \period
	\end{equation*}
	Said differently, $ \Shpure(\Man;C) $ is the kernel of the constant sheaf functor $ \Gammalowerstar \colon \fromto{\Sh(\Man;C)}{C} $.

	Also note that since the global sections functor $ \Gammalowerstar $ preserves all limits and colimits, the subcategory of pure sheaves is stable under limits and colimits.
\end{observation}

Now we introduce the left adjoint to the inclusion $ \Shpure(\Man;C) \subset \Sh(\Man;C) $ following the prescription of \enumref{prop:orthogonaladjoints}{1}.
In the following, we think of $ \Rhi(\Ehat) $ as playing the role of cohomology with coefficients in $ \RR/\ZZ $ in the differential cohomology hexagon (\Cref{thm:SimonsSullivanunique}).

\begin{definition}\label{differential_cycles}
	Let $ C $ be a stable presentable \category.
	Define a functor
	\begin{equation*}
		\Cyc \colon \fromto{\Sh(\Man;C)}{\Sh(\Man;C)}
	\end{equation*}
	and a \textit{curvature} natural transformation $ \curv \colon \fromto{\id{}}{\Cyc} $ by the cofiber sequence
	\begin{equation*}
		\begin{tikzcd}[sep=2em]
			\Rhi \arrow[r, "\varepsilon"] & \id{} \arrow[r, "\curv"] & \Cyc \comma
		\end{tikzcd}
	\end{equation*} 
	where $ \varepsilon \colon \fromto{\Rhi}{\id{}} $ is the counit.
	For a $ C $-valued sheaf $ \Ehat $ on $ \Man $, we call $ \Cyc(\Ehat) $ the sheaf of \textit{differential cycles} associated to $ \Ehat $.
\end{definition}

\begin{nul}
	As a consequence of \Cref{prop:orthogonaladjoints}, $ \Cyc $ factors through $ \Shpure(\Man;C) $ and is left adjoint to the inclusion $ \Shpure(\Man;C) \subset \Sh(\Man;C) $.
\end{nul}

\begin{observation}
	Since the global sections functor $ \Gammalowerstar $ preserves all limits and colimits, the subcategory of pure sheaves is stable under \textit{both} limits and colimits.
	Since $ \Shpure(\Man;C) $ is presentable, the inclusion $ \incto{\Shpure(\Man;C)}{\Sh(\Man;C)} $ also admits a right adjoint.
\end{observation}

To do this, we identify the left adjoint to the functor $ \Cyc \colon \fromto{\Sh(\Man;C)}{\Shpure(\Man;C)} $.

\begin{definition}
	Let $ C $ be a stable presentable \category.
	Define a functor
	\begin{equation*}
		\Def \colon \fromto{\Sh(\Man;C)}{\Sh(\Man;C)}
	\end{equation*}
	by the fiber sequence
	\begin{equation*}
		\begin{tikzcd}[sep=1.5em]
			\Def \arrow[r] & \id{} \arrow[r, "\eta"] & \Lhi \comma
		\end{tikzcd}
	\end{equation*} 
	where $ \eta \colon \fromto{\id{}}{\Lhi} $ is the unit.
	For a $ C $-valued sheaf $ \Ehat $ on $ \Man $, we call $ \Def(\Ehat) $ the sheaf of \textit{differential deformations} associated to $ \Ehat $.
\end{definition}

\begin{observations}\label{obs:Acomposite}
	In light of \Cref{thm:stabrecollement}, the functor 
	\begin{equation*}
		\Def \colon \fromto{\Shpure(\Man;C)}{\Sh(\Man;C)}
	\end{equation*}
	is left adjoint to the functor $ \Cyc $.
	In particular, $ \Def \colon \fromto{\Shpure(\Man;C)}{\Sh(\Man;C)} $ is fully faithful (\Cref{lem:ladjradj}).
	% \hfill
	% \begin{enumerate}[(\ref*{obs:Acomposite}.1)]
	% 	\item Since $ \Lhi $ is idempotent and exact, we see that $ \Lhi \of \Def \equivalent 0 $.

	% 	\item Since $ \Lhi \Rhi \equivalent \Rhi $, for any $ C $-valued sheaf $ \Ehat $ on $ \Man $, the fiber sequence defining $ \Def $ gives a fiber sequence
	% 	\begin{equation*}
	% 		\begin{tikzcd}[sep=1.5em]
	% 			\Def\Rhi(\Ehat) \arrow[r] & \Rhi(\Ehat) \arrow[r, "\sim"{yshift=-0.25em}] & \Lhi\Rhi(\Ehat) \comma
	% 		\end{tikzcd}
	% 	\end{equation*} 
	% 	hence $ \Def \of \Rhi \equivalent 0 $.  
	% \end{enumerate}
\end{observations}

\begin{nul}\label{nul:diffcohadjunctions}
	We have chains of adjunctions
	\begin{equation*}
		\begin{tikzcd}[sep=3.5em]
			\Shhi(\Man;C) \arrow[r, description, hooked] & \Sh(\Man;C) \arrow[l, shift left=1.25ex, "\Rhi"] \arrow[l, shift right=1.25ex, "\Lhi"'] \arrow[r, "\Cyc" description]  & \Shpure(\Man;C) \period \arrow[l, shift left=1.25ex, hooked'] \arrow[l, shift right=1.25ex, hooked', "\Def"']
		\end{tikzcd}
	\end{equation*}
	To align notation with \cref{nul:stabelrecadjunction}, we have $ \Xcat = \Sh(\Man;C) $, $ \Zcat = \Shhi(\Man;C) $, and $ \Ucat = \Shpure(\Man;C) $.
	The functors $ \ilowerstar \colon \incto{\Zcat}{\Xcat} $ and $ \jlowerstar \colon \incto{\Ucat}{\Xcat} $ are the two unlabeled inclusions.
	We also have $ \iuppershriek = \Rhi $, $ \iupperstar = \Lhi $, $ \jupperstar = \Cyc $, and $ \jlowershriek = \Def $.
\end{nul}

%-------------------------------------------------------------------%
%  The differential cohomology hexagon                              %
%-------------------------------------------------------------------%

\subsubsection{The differential cohomology hexagon}\label{subsec:diffcohdiagram}

Now we explain how the extended fracture diagram of a stable recollement (\Cref{thm:fracturehexagon}) gives rise to
a ``differential cohomology hexagon''.

\begin{notation}
	We write $ \dup \colon \fromto{\Def}{\Cyc} $ for the composite 
	\begin{equation*}
		\begin{tikzcd}[sep=2em]
			\dup \colon \Def \arrow[r] & \id{} \arrow[r, "\curv"] & \Cyc \period
		\end{tikzcd}
	\end{equation*} 
\end{notation}

\begin{corollary}[(fracture square)]\label{thm:fracturesquare}
	Let $ C $ be a stable presentable \category.
	The \category $ \Sh(\Man;C) $ is the recollement of the subcategories $ \Shhi(\Man;C) $ and $ \Shpure(\Man;C) $.
	In particular, there is a commutative diagram
	\begin{equation}\label{diag:fiberseqs}
		\begin{tikzcd}[sep=2.5em]
			 &  \Def \arrow[r, equals] \arrow[d] & \Def \arrow[d, "\d"] \\
			\Rhi \arrow[r] \arrow[d, equals] & \id{\Sh(\Man;C)} \arrow[r] \arrow[d] \arrow[dr, phantom, "\square" description] & \Cyc \arrow[d] \\
			\Rhi \arrow[r] & \Lhi \arrow[r] & \Lhi\Cyc 
		\end{tikzcd}
	\end{equation}
	of functors $ \fromto{\Sh(\Man;C)}{\Sh(\Man;C)} $, where the square is a pullback and all rows and columns are fiber sequences.
\end{corollary}

\begin{nul}
	Informally, $ \Sh(\Man;C) $ is the \category of triples 
	\begin{equation*}
		(\Ehat_{\RR}, \Ehat_{\pure}, \phi \colon \fromto{\Ehat_{\RR}}{\Lhi\Ehat_{\pure}}) \comma
	\end{equation*}
	where $ \Ehat_{\RR} $ is a $ \RR $-invariant sheaf, $ \Ehat_{\pure} $ is a pure sheaf, and $ \phi $ is any morphism.
\end{nul}

\begin{nul}[(differential cohomology hexagon)]\label{nul:spectraldiffcohdiag}
	With some rearrangement, \Cref{thm:fracturesquare} and the fact that pullback squares compose, we see that there is a diagram of pullback squares
	\begin{equation}\label{diag:firstdiffcoh}
		\begin{tikzcd}[sep=2.5em]
			\Sigma^{-1} \Lhi\Cyc \arrow[r] \arrow[d] \arrow[dr, phantom, "\square" description] & \Rhi \arrow[r] \arrow[d] \arrow[dr, phantom, "\square" description] & 0 \arrow[d] \\
			\Def \arrow[r] \arrow[d] \arrow[dr, phantom, "\square" description] & \id{\Sh(\Man;C)} \arrow[r] \arrow[d] \arrow[dr, phantom, "\square" description] & \Cyc \arrow[d] \\
			0 \arrow[r] & \Lhi \arrow[r] & \Lhi\Cyc \period
		\end{tikzcd}
	\end{equation}
	Rearranging the diagram \eqref{diag:firstdiffcoh}, for each $ \Ehat \in \Sh(\Man;C) $ we get the following
	``differential cohomology hexagon''
	\begin{equation}\label{diag:generaldiffcohhex}
		\begin{tikzcd}[column sep={10ex,between origins}, row sep={8ex,between origins}]
			& \Rhi(\Ehat) \arrow[rr] \arrow[dr] & & \Lhi(\Ehat) \arrow[dr] & \\
			\Sigma^{-1} \Lhi \Cyc(\Ehat) \arrow[ur] \arrow[dr] & & \Ehat \arrow[ur] \arrow[dr, "\curv" description] & & \Lhi \Cyc(\Ehat) \\
			& \Def(\Ehat) \arrow[rr, "\dup"'] \arrow[ur] & & \Cyc(\Ehat) \arrow[ur]  & \phantom{\Lhi \Cyc(\Ehat)} \period
		\end{tikzcd}
	\end{equation}
	Here the diagonals are fiber sequences, the top and bottom rows are extensions of fiber sequences by one term, and both squares are pullback squares.
	The ``top row'' consists of $ \RR $-invariant sheaves, whereas the ``bottom row'' consists of sheaves that are, in some sense, more geometric.

	Since $ \Lhi \equivalent \Gammaupperstar \Gammalowershriek $ and $ \Rhi \equivalent \Gammaupperstar \Gammalowerstar $ \Cref{nul:Gammaadjunctions}, the differential cohomology hexagon \eqref{diag:generaldiffcohhex} can be rewritten as
	\begin{equation*}\label{diag:generaldiffcohhexGamma}
		\begin{tikzcd}[column sep={10ex,between origins}, row sep={8ex,between origins}]
			& \Gamma^*\Gamma_*\Ehat\arrow[rr]\arrow[dr] & & \Gamma^*E\arrow[dr] & \\
			\Sigma^{-1}\Gamma^*\Gamma_!\Cyc(\Ehat)\arrow[dr]\arrow[ur] & & \Ehat\arrow[dr]\arrow[ur] & & \Gamma^*\Gamma_! \Cyc(\Ehat)\\
			& \Def(\Ehat)\arrow[ur]\arrow[rr, "\d"'] & & \Cyc(\Ehat)\arrow[ur] & \phantom{\Gamma^*\Gamma_! \Cyc(\Ehat)} \period
		\end{tikzcd}
	\end{equation*}
\end{nul}

%-------------------------------------------------------------------%
%  Differential refinements                                         %
%-------------------------------------------------------------------%

\subsubsection{Differential refinements}\label{subsec:refinements}

We finish this section by making precise what it means for a differential cohomology theory $ \Ehat \in \Sh(\Man;\Sp) $ to refine a cohomology theory $ E \in \Sp $.

\begin{definition}\label{def:refinement}
	Let $ C $ be a presentable stable \category.
	A \textit{differential refinement} of a an object $ E \in C $ is pair $ (\Ehat,\phi) $ of a sheaf $ \Ehat \in \Sh(\Man;C) $ together with an equivalence $ \phi \colon \equivto{\Gammalowershriek(\Ehat)}{E} $ in $ C $.
\end{definition}

\begin{nul}\label{nul:altdiffrefinement}
	From the fracture square (\Cref{thm:fracturesquare}), a differential refinement of $ E \in C $ is equivalently the data of a pure sheaf $ \Phat \in \Shpure(\Man;C) $ along with a morphism $ \fromto{E}{\Gammalowershriek(\Phat)} $ in $ C $.
	Given this data, we can construct a differential refinement $ \Ehat $ in the sense of \Cref{def:refinement} as the pullback
	\begin{equation*}
		\begin{tikzcd}[column sep={11ex,between origins}, row sep={11ex,between origins}]
			\Ehat \arrow[r] \arrow[d] \arrow[dr, phantom, "\square" description] & \Phat \arrow[d] & \\
			\Gammaupperstar(E) \arrow[r] & \Gammaupperstar\Gammalowershriek(\Phat) \period
		\end{tikzcd}
	\end{equation*}

	In this case, we have:
	\begin{enumerate}[(\ref*{nul:altdiffrefinement}.1)]
		\item $ \equivto{\Def(\Ehat)}{\Def(\Phat)} $.

		\item $ \equivto{\Cyc(\Ehat)}{\Phat} $.

		\item $ \Gammalowerstar(\Ehat) $ fits into a fiber sequence
		\begin{equation*}
			\begin{tikzcd}[sep=1.5ex]
				\Gammalowerstar(\Ehat) \arrow[r] & E \arrow[r] & \Gammalowershriek(\Phat) \period
			\end{tikzcd}
		\end{equation*}
	\end{enumerate}
\end{nul}

\begin{construction}[(pullback of a differential refinement)]\label{constr:pullbackrefinement}
	Let $ C $ be a presentable stable \category, $ f \colon \fromto{E}{E'} $ a morphism in $ C $, and $ (\Ehat',\phi') $ a differential refinement of $ E' $. 
 	Form the pullback
	\begin{equation}\label{sq:differential}
		\begin{tikzcd}[column sep={11ex,between origins}, row sep={11ex,between origins}]
			\Ehat \arrow[r, "\fhat"] \arrow[d] \arrow[dr, phantom, "\square" description] & \Ehat' \arrow[d] & \\
			\Gammaupperstar(E) \arrow[r, "\Gammaupperstar(f)"'] & \Gammaupperstar(E') \comma
		\end{tikzcd}
	\end{equation}
	where the morphism $ \fromto{\Ehat'}{\Gammaupperstar(E')} $ is adjoint to the given equivalence $ \phi' \colon \equivto{\Gammalowershriek(\Ehat')}{E'} $.
	Since $ \Gammalowershriek $ is exact, applying $ \Gammalowershriek $ to the square \eqref{sq:differential} gives a pullback square
	\begin{equation*}
		\begin{tikzcd}[column sep={11ex,between origins}, row sep={11ex,between origins}]
			\Gammalowershriek(\Ehat) \arrow[r] \arrow[d, "\phi"'] \arrow[dr, phantom, "\square" description] & \Gammalowershriek(\Ehat') \arrow[d, "\phi'"', "\sim"{sloped, yshift=-0.2em}] & \\
			E \arrow[r, "f"'] & E' \comma
		\end{tikzcd}
	\end{equation*}
	which provides an equivalence $ \phi \colon \equivto{\Gammalowershriek(\Ehat)}{E} $.
	The \textit{pullback differential refinement} of $ (\Ehat',\phi') $ along $ f $ is the differential refinement $ (\Ehat,\phi) $ of $ E $.
\end{construction}

\begin{lemma}\label{lem:pullbackrefinements}
	In the notation of \Cref{constr:pullbackrefinement}, the following
	\begin{enumerate}[{\upshape (\ref*{lem:pullbackrefinements}.1)}]
		\item The morphism $ \Def(\fhat) \colon \fromto{\Def(\Ehat)}{\Def(\Ehat')} $ is an equivalence.

		\item The morphism $ \Cyc(\fhat) \colon \fromto{\Cyc(\Ehat)}{\Cyc(\Ehat')} $ is an equivalence.

		\item The global sections of $ \Ehat $ is given by the pullback
		\begin{equation*}
			\begin{tikzcd}[column sep={11ex,between origins}, row sep={11ex,between origins}]
				\Gammalowerstar(\Ehat) \arrow[r, "\Gammalowerstar(\fhat)"] \arrow[d] \arrow[dr, phantom, "\square" description] & \Gammalowerstar(\Ehat') \arrow[d] & \\
				E \arrow[r, "f"'] & E' \period
			\end{tikzcd}
		\end{equation*}
	\end{enumerate} 
\end{lemma}
