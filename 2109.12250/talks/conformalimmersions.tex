%!TEX root = ../diffcoh.tex

\section{Conformal Immersions}
\textit{by Charlie Reid}
\label{conformal_immersions}

Let $M$ be a smooth, $m$-dimensional manifold and suppose $M$ immerses in $\RR^n$ with normal bundle $NM$. Then
there is a short exact sequence
% https://q.uiver.app/?q=WzAsNSxbMCwwLCIwIl0sWzEsMCwiVE0iXSxbMiwwLCJUXFxSUl5uIl0sWzMsMCwiTk0iXSxbNCwwLCIwIl0sWzAsMV0sWzEsMl0sWzIsM10sWzMsNF1d
\begin{equation}
\begin{tikzcd}
	0 & TM & {T\RR^n|_M} & NM & 0,
	\arrow[from=1-1, to=1-2]
	\arrow[from=1-2, to=1-3]
	\arrow[from=1-3, to=1-4]
	\arrow[from=1-4, to=1-5]
\end{tikzcd}
\end{equation}
so the Pontryagin classes of $TM$ and $NM$ satisfy\footnote{Because the Whitney sum formula for Pontryagin classes
only holds up to $2$-torsion, this formula should be thought of as taking place in cohomology with $\ZZ[1/2]$ or
$\RR$ coefficients.}
\begin{equation}
	p(TM)p(NM) = p(T\RR^n|_M) = 1.
\end{equation}
The total Pontryagin class is the sum of $1$ and a nilpotent element ($p_1(M) + p_2(M) + \cdots$), hence is
invertible. This means $p(NM)$ is uniquely determined if it exists: there is a formula for $p_k(NM)$ in terms of
$p(TM)$. If $M$ immerses in $\RR^n$, then $NM$ is rank $n-m$, so $p_k(NM) = 0$ for $k > n-m$, and because of the
formula, this is actually a constraint on the Pontryagin classes of $TM$. Thus Pontryagin classes can be used to
prove nonimmersion results for smooth manifolds by showing this constraint is not met.

In \cref{DifferentialCharacteristicClasses}, we saw that given a connection on the tangent bundle, Pontryagin
classes lift to differential cohomology. It therefore seems worthwhile to imitate the above argument and use
on-diagonal differential Pontryagin classes given by the Levi-Civita connection to obstruct isometric immersions of
Riemannian manifolds. Chern and Simons \cite{cs} did this, though with a few key differences.
\begin{enumerate}[(1)]
	\item Chern and Simons were able to show (\textit{ibid.}, Theorem 4.5) that if $g$ and $g'$ are two conformally
	equivalent metrics on a manifold $M$, with Levi-Civita connections $\conn$, resp.\ $\conn'$, then $\phat(M,
	\conn) = \phat(M, \conn')$. Therefore the differential Pontryagin classes of $M$ are conformal invariants,
	and can be used to study conformal immersions.
	\item There is an additional integrality result which has no analogue in the purely topological case
	(\textit{ibid.}, Theorem 5.14): when a Pontryagin class' Chern--Weil form vanishes, the corresponding
	Chern--Simons form is closed, and one-half of its de Rham class is contained within the lattice
	$\mathrm{Im}(\H^*(\text{--};\ZZ)\to\H^*(\text{--};\RR))$. After some more work, this leads to another necessary
	condition for the existence of a conformal immersion.
\end{enumerate}
As an example, $\RRP^3$ smoothly immerses in $\RR^4$ \cite{Boy03}, and given the round metric,
$\RRP^3$ locally conformally immerses in $\RR^4$. But Chern--Simons show (\textit{ibid.}, \S 6) that there is
no conformal immersion $\RRP^3\hookrightarrow\RR^4$.

In \cref{ssec:conformal_invariance}, we prove that the on-diagonal differential Pontryagin classes of the
Levi-Civita connection are conformal invariants of the Riemannian metric. Then, in \cref{ssec:obstruct}, we use
on-diagonal differential Pontryagin classes to obstruct conformal immersions. Finally, in \cref{ssec:div2}, we
produce the integrality obstruction using the Chern--Simons form and use it to show $\RRP^3$ with the round metric
cannot conformally immerse in $\RR^4$.

\begin{remark}
	The story we just told is a little anachronistic: Chern--Simons' work came before Cheeger--Simons' paper on
	differential characters, and was not stated in this language. But Chern and Simons were aware that their ideas
	could be rephrased as calculations in the ring of differential characters, as they write in the introduction to
	their paper. In any case, the paper \cite{cs} is best known for an entirely different reason: for introducing the
	Chern--Simons form of a connection!
\end{remark}


%-------------------------------------------------------------------%
%-------------------------------------------------------------------%
%  Conformal Invariance                                             %
%-------------------------------------------------------------------%
%-------------------------------------------------------------------%

\subsection{Conformal invariance of differential Pontryagin classes}
\label{ssec:conformal_invariance}
Let $G$ be a compact Lie group. Recall that given a degree-$k$ invariant polynomial $f$ on $\g$ and a
characteristic class $c^\ZZ\in\H^{2k}(\BG;\ZZ)$, we obtain a differential characteristic class $\chat\in
\Hhat^{2k}(\BnablaG;\ZZ)$ (as proven in \cref{differential_CW_lift}) and a Chern--Simons form
$\CS_f(\conn)\in\Omega^{2k-1}(P)$ given a principal $G$-bundle $\pi\colon P\to M$ and a connection $\conn$ on $P$
(as defined in~\eqref{total_space_CS}). We are specifically interested in the Pontryagin polynomials $P_k$ from
\cref{pontrjagin}, which we lifted to on-diagonal differential Pontryagin classes $\phat_k$ in
\cref{differential_Pontryagin}.

Our aim in this section is to prove:
\begin{theorem}[{\cite[Theorem 4.5]{cs}}]
\label{diff_p_conformally_invariant}
Let $M$ be a manifold and $g_0,g_1$ be conformally equivalent Riemannian metrics on $M$. If $\conn_0$ and
$\conn_1$ denote the Levi-Civita connections for $g_0$ and $g_1$, then for all $k$, $\phat_k(M, \conn_0) =
\phat_k(M, \conn_1)$ and $\CS_{P_k}(\conn_0) - \CS_{P_k}(\conn_1)$ is exact.
\end{theorem}
The first ingredient in the proof is a variation formula.
\index[terminology]{variation formula!for Chern--Simons forms}
\index[terminology]{Chern--Simons form}
\begin{lemma}[{(Variation formula \cite[Proposition 3.8]{cs})}]
\label{variation_Chern_Simons}
Suppose $\conn_t$ is a smooth path of connections on a principal $G$-bundle $P\to M$ and $\curvature{\conn_t}$ is the
curvature of $\conn_t$. Then
\begin{equation}
	\label{var_form_CS}
	\left.\frac{\mathrm d}{\mathrm dt} \CS_f(\conn_t)\right|_{t=0} = k\cdot  f(\conn'\wedge
	\curvature{\conn_0}^{k-1}) + \omega,
\end{equation}
where $\omega$ is exact and $\conn' = \left.\frac{\mathrm d}{\mathrm dt}(\conn_t)\right|_{t=0}$.
\end{lemma}
\begin{proof}
It suffices to work universally in $\EnablaG$. The de Rham complex of $\EnablaG$ is acyclic \cite[Theorem
7.19]{FreedHopkins}, so it suffices to apply the de Rham differential to~\eqref{var_form_CS} and then show both
sides are equal.\footnote{One can avoid the use of the abstract object $\EnablaG$ by using Narasimhan--Ramanan's
$n$-classifying spaces \cite{NR61, NR63}.}

For the left-hand side, we know
\begin{align*}
	\d\left(\left.\frac{\mathrm d}{\mathrm dt} \CS_f(\conn_t)\right|_{t=0}\right) &= \left.\frac{\d}{\d t}\left(
	\d(\CS_f(\conn_t))\right)\right|_{t=0}\\
	&= \left.\frac{\d}{\d t}\left(f((\curvature{\conn_t})^k)\right)\right|_{t=0}\\
	&= k\cdot f(\curvature{\conn_0}'\wedge \curvature{\conn_0}^{k-1}),
\end{align*}
where $\curvature{\conn_0}' = \left.\frac{\mathrm d}{\mathrm dt}(\curvature{\conn_t})\right|_{t=0}$.

For the right-hand side,
\begin{align*}
	\d(k\cdot f(\conn' \wedge \curvature{\conn_0}^{k-1})) &= k\cdot f(\d\conn' \wedge \curvature{\conn_0}^{k-1}) -
	k(k-1) f(\conn' \wedge \d\curvature{\conn}\wedge\curvature{\conn}^{k-2})\\
	&= k f(\d\conn'\wedge\curvature{\conn_0}^{k-1}) - k(k-1) f(\conn'\wedge [\curvature{\conn_0}, \conn_0]\wedge
	\curvature{\conn}^{k-2})\\
	&= k f(\d\conn'\wedge \curvature{\conn_0}^{k-1}) + k f([\conn', \conn_0]\wedge \curvature{\conn_0}^{k-1}).
\end{align*}
This uses two important facts from Chern--Weil theory: that $\d\curvature{\conn_0} = [\curvature{\conn_0},
\conn_0]$ together with the value of the invariant polynomial for a commutator \cite[(2.9)]{cs}. Now
\begin{align*}
	\d\conn' &= \left.\frac{\d}{\d t}\left(\d\conn_t\right)\right|_{t=0}\\
	&= \left.\frac{\d }{\d t}\left(\curvature{\conn_t} - \frac 12[\conn_t, \conn_t]\right)\right|_{t=0}\\
	&= \curvature{\conn_0}' - [\conn', \conn_0],
\end{align*}
so $\d(k\cdot f(\conn'\wedge\curvature{\conn_t}^{k-1}))= k\cdot (\curvature{\conn_0}'\wedge
\curvature{\conn_t}^{k-1})$ and we are done.
\end{proof}
\begin{proof}[Proof of \cref{diff_p_conformally_invariant}]
Now for $f$ we take $P_k$, the invariant polynomial that we used in \cref{pontrjagin} to define the $k^{\mathrm{th}}$
Pontryagin class.  This is the pullback of the $2k^{\mathrm{th}}$ Chern polynomial under the complexification map
$\mathfrak o(n)\to\mathfrak u(n)$; we tend not to use the pullback of the $(2k+1)^{\mathrm{st}}$ Chern polynomial
as much because it is $2$-torsion and its Chern--Simons form is exact \cite[Proposition 4.3]{cs}.

It suffices to show that $\delta\colonequals \CS_{P_k}(\conn_0) - \CS_{P_k}(\conn_1)$ is exact; this implies
it is a closed form with integral periods, so the image $\overline\delta$ of $\delta$ in
$\Omega^{4k-1}(M)/\Omegacl^{4k-1}(M)_\ZZ$ vanishes. This is the lower-left corner of the differential cohomology
hexagon, and as we saw in \cref{iota_chern_simons}, applying
$\iota\colon\Omega^{4k-1}(M)/\Omegacl^{4k-1}(M)_\ZZ\to\Hhat^{4k}(M; \ZZ)$ sends $\overline\delta\mapsto\phat_k(P,
\conn_0) - \phat_k(P, \conn_1)$, so showing $\overline\delta = 0$ is good enough.

Now to show $\delta$ is exact. It is always possible to connect $g_0$ and $g_1$ by a path $g_t$,
$t\in(-\varepsilon, 1+\varepsilon)$ of conformally equivalent metrics. Moreover, this path may be chosen to satisfy
\[g_t = e^{2th}g_0\]
for some real-valued smooth function $h$. Choose such a path and let $\conn_t$ be the Levi-Civita connection of
$g_t$.  Differentiating in $t$ commutes with the de Rham differential, so is suffices to show that $\frac{\d}{\d
t}\CS_{P_k}(\conn_t)$ is exact; without loss of generality, we prove this for $t = 0$.
\Cref{variation_Chern_Simons} means we only have to show
\begin{equation}
	P_k(\conn_0'\wedge \curvature{\conn_0}^{2k-1}) = 0.
\end{equation}
For a little while we work locally on the bundle $\pi\colon B(M)\to M$ of frames: the fiber at $x\in M$ is the
$\GL_n(\RR)$-torsor of orthonormal bases $(e_1, \dotsc, e_n)$ of $T_xM$. There are canonical one-forms
$\omega_i\in\Omega^1(B(M))$ defined at a point $(x, (e_1, \dotsc, e_n))$ so that
\begin{equation}
	\d\pi = \sum_{i=1}^n \omega_i \cdot e_i.
\end{equation}
Let $E_i$ be the horizontal vector field dual to $\omega_i$; here ``horizontal'' is with respect to the connection
$\conn_0$. Then on frames orthogonal to $g_0$, there is a decomposition \cite[Lemma 4.4]{cs}
\begin{equation}
	\conn_{ij}' = \underbracket{\delta_{ij}\d(h\circ\pi)}_{\alpha} + \underbracket{E_i(h\circ\pi)\omega_j -
	E_j(h\circ\pi)\omega_i}_{\beta}.
\end{equation}
We will address each piece separately. First, one directly checks that for $\varphi =
(\varphi_{ij})\in\Omega^k(F(M))$,
\begin{equation}
\label{cycle_indices}
	P_k(\varphi\wedge \curvature{\conn}^{k-1}) = \sum_{i_1,\dotsc,i_k = 1}^n
	\varphi_{i_1i_2}\wedge (\curvature{\conn})_{i_2i_3}\wedge\dotsb\wedge (\curvature{\conn})_{i_ni_1}.
\end{equation}
Plugging in $\varphi = \alpha$, we obtain
\begin{equation}
	P_k(\alpha\wedge \curvature{\conn}^{2k-1}) = \d(f\circ\pi)\wedge P_{2k-1}(\curvature{\conn}^{2k-1}) = 0,
\end{equation}
because $A$ is compatible with the metric. Now plugging $\beta$ into~\eqref{cycle_indices},
\begin{equation}
\label{beta_qty}
	P_k(\beta\wedge\curvature{\conn}^{2k-1}) = \sum_{i_1,\dotsc,i_{2k} = 1}^n
	(E_{i_1}(f\circ\pi)\omega_{i_2} - E_{i_2}(f\circ\pi)\omega_{i_1})\wedge(\curvature{\conn})_{i_2i_3}\wedge\dotsb\wedge
	(\curvature\conn)_{i_{2k}i_1}.
\end{equation}
The Jacobi identity implies $\sum \omega_i\wedge(\curvature{\conn})_{ij} = 0$, so~\eqref{beta_qty} vanishes as
well. Lastly, we need to descend from $B(M)$ to $M$, and $0$ descends to $0$.
\end{proof}

\subsection{Obstructing conformal immersions with differential Pontryagin classes}
\label{ssec:obstruct}

%-------------------------------------------------------------------%
%  Differential Chern Classes                                       %
%-------------------------------------------------------------------%

%\subsubsection{Differential Chern Classes}
\index[terminology]{Chern class!differential refinement}
\index[terminology]{differential Chern class}
Recall the on-diagonal differential lifts of Chern classes we constructed in
\cref{DifferentialCharacteristicClasses}, specifically \cref{differential_Chern}, defined as follows: define the
\emph{Chern polynomials} $C_k\in I^k(\Uup_n)$ by 
\index[terminology]{Chern polynomial}
\begin{equation}
	\det(\lambda I - \frac{1}{2\pi i}A) = \sum_{k=0} ^n C_k(A)\lambda^{n-k};
\end{equation}
apply Chern--Weil theory to $C_k$, producing a characteristic class $c_k$. The integer
cohomology of $\Uup_n$ is torsion-free \cite[\S 29]{Bor53} and its image in de Rham cohomology contains $c_k$,
so there is a unique lift to $\chat_k$ to degree-$2k$ differential cohomology.

We will also need the \emph{inverse Chern polynomials} $C_k^\perp$, which are defined to satisfy
\index[terminology]{inverse Chern polynomials}
\begin{equation}
	(1+C_1+\cdots+C_n)(1+C_1^\perp + C_2^\perp + \cdots) = 1.
\end{equation}
For example, $C_1^\perp = -C_1$, $C_2^\perp = -C_2 - C_1C_1^\perp$, $C_3^\perp = -C_3 - C_2C_1^\perp -
C_1C_2^\perp$, and so on. Chern--Weil theory associates de Rham characteristic classes $c_k^\perp\in\HdR^{2k}$ to
these, and like ordinary Chern classes, these classes lift uniquely to differential cohomology classes
$\chat_k^\perp\in \Hhat^{2k}(\mathrm B_\nabla\!\Uup_n;\ZZ)$. They satisfy analogous formulas to the inverse Chern
polynomials: for example
\index[notation]{chatperp@$\chat_k^\perp$}
%
%We don't care about the $D$'s. Since $\BGL_n(\CC)$ has no torsion in it's cohomology, these polynomials are enough
%to define differential characteristic classes $\chat_k(V)$ of a complex vector bundle with connection of rank $n$,
%which refine the ordinary Chern classes $c_k(V)$. We also define the total Chern class
%$$\chat(V) = 1 + \chat_1(V) + \cdots + \chat_n(V)\in \Hhat^{\even}(M)$$
%We also define inverse Chern polynomials by . We can solve for them and write explicit inductive formulas:
%\begin{align*}
%	C_1^\perp &= -C_1 \comma \\ 
%	C_2^\perp &= -C_2 -C_1C_1^\perp \comma \\ 
%	C_3^\perp &= -C_3 -C_2C_1^\perp - C_1C_2^\perp \comma \\ 
%	&\vdots
%\end{align*}
%We then define differential characteristic classes $\chat_k^\perp$ using these polynomials. 
%I haven't mentioned it yet, but the Weil map and it's differential refinement are actually ring homomorphisms, so for example
\begin{equation}
\label{inverse_Chern_relation}
	\chat_2^\perp = -\chat_2 - \chat_1 \chat_1^\perp.
\end{equation} 
%\todo[inline]{This uses that the Chern--Weil map is a ring homomorphism. Did we say this anywhere?}

%-------------------------------------------------------------------%
%  Differential Pontryagin Classes                                  %
%-------------------------------------------------------------------%

%\subsubsection{Differential Pontryagin Classes}
\index[terminology]{Pontryagin class!differential refinement}
\index[terminology]{differential Pontryagin class}

In \cref{differential_Pontryagin}, we defined on-diagonal differential Pontryagin classes $\phat_k$ in much the
same way as we defined differential Chern classes. Using the \emph{inverse Pontryagin polynomials} $P_k^\perp$,
defined to satisfy \index[terminology]{inverse Pontryagin polynomials}
\begin{equation}
	(1+P_1+\cdots+P_n)(1+P_1^\perp + P_2^\perp + \cdots) = 1,
\end{equation}
we define on-diagonal inverse differential Pontryagin classes $\phat_k^\perp\in \Hhat^{4k}(\mathrm
B_\nabla\!\Or_n;\ZZ)$.  Because there is torsion in $\H^*(\BO(n);\ZZ)$, a priori the lift to differential cohomology
requires a choice, but there is a canonical way to do this: complexify to pass to on-diagonal inverse differential
Chern classes. This means that analogues of~\eqref{inverse_Chern_relation} and its higher-rank generalizations hold
for on-diagonal inverse Pontryagin classes. For example, $\phat_2^\perp = -\phat_2 - \phat_1\phat_1^\perp$.
\index[notation]{phatperp@$\phat_k^\perp$}

\begin{thm}
\label{differential_conformal_immersion}
Let $M$ be a Riemannian manifold and $\phi \colon M^n\to \RR^{n+k}$ be a conformal immersion of $M$ into Euclidean
space. Then the image of $\phat^\perp_i(M, \conn^{\mathrm{LC}})$ in $\Hhat^{4i}(M;\ZZ[1/2])$ vanishes for all
$i>k/2$.
\end{thm}
\index[terminology]{conformal immersion}
\begin{proof}
Since the classes $\phat_k$ are conformally invariant (\cref{diff_p_conformally_invariant}), so too are the classes
$\phat_k^\perp$. Therefore, without loss of generality, we can assume $\phi$ is isometric. Let $NM$ denote the
orthogonal normal bundle: there is an orthogonal direct sum
	$$TM \oplus NM = T\RR^{n+k} = \underline{\RR}^{n+k}.$$
	The Levi-Civita connection $\conn_{T\RR^{n+k}}^{\mathrm{LC}}$ on $\RR^{n+k}$ compresses to the Levi-Civita
	connection $\conn_{TM}^{\mathrm{LC}}$ on $M$, and to a connection $\conn_{NM}$ on $NM$. Since
	$\conn_{T\RR^{n+k}}^{\mathrm{LC}}$ is flat, it is compatible with
	$\conn_{TM}^{\mathrm{LC}}\oplus\conn_{NM}$ (\cref{compatible_connection}). Hence
	\begin{equation}
		\phat(TM, \conn_{TM}^{\mathrm{LC}})*\phat(NM, \conn_{NM})= \phat(T\RR^{n+k},
		\conn_{T\RR^{n+k}}^{\mathrm{LC}}) = 1\comma
	\end{equation}
	implying
	\begin{equation}
		\phat^\perp(TM, \conn_{TM}^{\mathrm{LC}}) = \phat(NM, \conn_{NM})\period
	\end{equation}
	Since $NM$ has rank $k$, $\phat_i(NM, \conn_{TM}^{\mathrm{LC}})$ vanishes for $i > k/2$.
\end{proof}
\begin{remark}
As always, we use $\ZZ[1/2]$ coefficients because the Whitney sum for Pontryagin classes is more complicated over
the integers. See Thomas \cite{Tho62} and Brown \cite[Theorem 1.6]{Bro82}. The extra factors ultimately come from
Chern classes, so they too admit differential refinements, and a $\ZZ$-valued differential Whitney sum formula
exists. Using this, it is possible to upgrade \cref{differential_conformal_immersion} to take place in
$\Hhat^*(M;\ZZ)$.
\index[terminology]{Whitney sum formula}
\end{remark}

\subsection{Dividing by \texorpdfstring{2}{$2$}}
\label{ssec:div2}
We foreshadowed that Chern--Simons theory will allow us to prove that $\RRP^3$ with the round metric does not
conformally immerse in $\RR^4$, but to actually prove this we need another obstruction. This one is an evenness
result: we will use the Chern--Simons form to define a de Rham cohomology class of on the frame bundle of $\RRP^3$,
and prove that a conformal immersion would imply this class is in the image of the map induced by the inclusion
$2\ZZ\to\RR$. A direct calculation shows this is not the case, and we conclude.
\begin{lem}[{\cite[Proposition 3.15]{cs}}]
\label{universal_pullback_coboundary}
If $\pi\colon P\to M$ is a principal $G$-bundle with connection $\conn$,
there is a cochain $u\in C^{2k-1}(M; \RR/\ZZ)$ such that $\delta(u) = f(\curvature\conn)\bmod\ZZ$ and in
$C^*(P; \RR/\ZZ)$, $\CS_f(\conn)\bmod\ZZ \pi^*(u)$ is a coboundary.
\end{lem}
\begin{proof}
Since $[f(\curvature\conn)]$ is in the image of the map from integer cohomology to de Rham cohomology,
$f(\curvature\conn)\bmod\ZZ$ is a coboundary, so choose $u\in C^{2k-1}(M; \RR/\ZZ)$ with $\delta u =
f(\curvature\conn)\bmod\ZZ$. Then
\begin{align*}
	\delta(\pi^*(u)) &= \pi^*(\delta u) = \pi^*(f(\curvature\conn))\bmod\ZZ\\
		&= \delta(\CS_f(\conn))\bmod\ZZ = \delta(\CS_f(\conn)\bmod\ZZ).
\end{align*}
That is, $\delta(\pi^*(u) - \CS_f(\conn)\bmod\ZZ)$ vanishes.
\end{proof}
Let $\pi\colon P\to M$ be a principal $G$-bundle with connection $\conn$. In the previous chapter,
specifically~\eqref{chern_simons_differential}, we showed that
$\d\CS_f(\conn) = \pi^*f(\curvature\conn)$. Therefore if $f(\curvature\conn) = 0$, $\CS_f(\conn)$ is
closed and defines a class $[\CS_f(\conn)]\in \H^{2k-1}(P;\RR)$.
\begin{cor}[{\cite[Theorem 3.16]{cs}}]
\label{when_CS_is_pulledback}
Assume $f(\curvature\conn) = 0$. Then there is a class $\overline u\in \H^{2k-1}(M;\RR/\ZZ)$ such that in
$\H^{2k-1}(P;\RR/\ZZ)$, $[\CS_f(\conn)] \bmod \ZZ = \pi^*(\overline u)$.
\end{cor}
\begin{proof}
By hypothesis of \cref{universal_pullback_coboundary}, $\delta(u) = f(\curvature\conn) = 0$, so we can choose
$\overline u$ to be the class of $u$ in cohomology.
\end{proof}
\begin{example}
\label{cpx_stiefel}
Let $\St_n(\CC^{n+k})$ denote the \textit{Stiefel manifold} of isometric immersions
$\CC^n\hookrightarrow\CC^{n+k}$. Sending an immersion to its image defines a map $\pi$ to the \textit{Grassmannian
manifold} $\Gr_n(\CC^{n+k})$ parametrizing codimension-$k$ subspaces of $\CC^{n+k}$, and this map is a principal
$\Uup_n$-bundle.
\index[terminology]{Stiefel manifold}
\index[terminology]{Grassmannian}
This bundle has a natural connection. It is equivalent to describe the connection on the associated rank-$n$
complex vector bundle $\pi'\colon S\to\Gr_n(\CC^{n+k})$, which is the tautological bundle. If $\rho\colon
(-\varepsilon, \varepsilon)\to S$ is a smooth curve, $\rho(t)$ is an element of the vector space
$\pi(\rho(t))\in\Gr_n(\CC^{n+k})$; we specify the connection by declaring the covariant derivative of $\rho(t)$
along $\pi\circ\rho$ to be the orthogonal projection of $\rho'(t)$ into the subspace $\pi(\rho(t))$. Call this
connection $\conn^{\mathrm{can}}$.

There is a canonically defined rank-$k$ complex vector bundle $Q\to\Gr_n(\CC^{n+k})$, whose fiber at an
$n$-dimensional subspace $V\subset\CC^{n+k}$ is $V^\perp\subset\CC^{n+k}$. Thus $S\oplus Q = \underline\CC^{n+k}$,
so in a similar manner as in the proof of \cref{differential_conformal_immersion}, $[C_i^\perp(\conn^{\mathrm{can}})] = 0$, i.e.\ $C_i^\perp(\conn^{\mathrm{can}})$ is exact. The Grassmannian is a compact,
irreducible Riemannian symmetric space, so since $C_i^\perp(\conn^{\mathrm{can}})$ is an invariant, exact
differential form, it must vanish. Therefore \Cref{when_CS_is_pulledback} tells us
$[\CS_{C_i^\perp}(\conn^{\mathrm{can}})]\bmod\ZZ$ pulls back from $\overline u\in
\H^{2k-1}(\Gr_n(\CC^{n+k});\RR/\ZZ)$. Because the cohomology of complex Grassmannians is concentrated in even
degrees, $\overline u = 0$, meaning $[\CS_{C_i^\perp}(\conn^{\mathrm{can}})]$ is in the image of the map
$\H^*(\St_n(\CC^{n+k});\ZZ)\to\H^*(\St_n(\CC^{n+k});\RR)$.
\end{example}
By passing to real vector bundles, we will gain an additional factor of $2$.
We will say a real-valued cohomology
class is \textit{contained in the even integer lattice} if it is in the image of the composite
\begin{equation}
	\H^*(\text{--};\ZZ)\overset{\cdot 2}{\longrightarrow} \H^*(\text{--}; \ZZ)\longrightarrow\H^*(\text{--};\RR).
\end{equation}

\begin{lem}[{\cite[Lemma 5.12]{cs}}]
	\label{double_cpxif}
	Let $c\colon\St_n(\RR^{n+k})\to\St_n(\CC^{n+k})$ be the complexification map. The image of $c^*\colon
	\H^\ell(\St_n(\CC^{n+k});\ZZ)\to\H^\ell(\St_n(\RR^{n+k});\ZZ)$ is contained in the even integer lattice
	for $\ell > 0$.
\end{lem}
\begin{proof}
First suppose $k = 0$, for which $\St_n(\CC^n)\cong\Uup_n$ and $\St_n(\RR^n)\cong\Or_n$; $c$ is the usual
complexification map. It suffices to show that the mod $2$ reductions of all positive-degree classes in the image
of $c^*$ vanish.

At this point we need a tool called the \textit{inverse transgression map}.\index[terminology]{inverse
transgression map} We will say more about this map in \cref{transgression_detail} at the end of this chapter; for
this proof, we need only that inverse transgression is a map $\tau\colon\H^\ell(\BG;\ZZ)\to\H^{\ell-1}(G;\ZZ)$
satisfying two key properties:
\begin{enumerate}[(1)]
	\item $\tau$ is natural in $G$, and
	\item for $A = \ZZ$ or $\ZZ/2$ and $x\in \H^*(\BG; A)$, $\tau(x^2) = 0$.
\end{enumerate}
Let $Bc\colon\mathrm{BO}(n)\to\mathrm{BU}(n)$ be the map induced from complexification on classifying spaces. We
know $(Bc)^*(c_i) \bmod 2 = w_i^2$ \cite[Theorem 1.5]{Bro82}, so
\begin{equation}
	c^*(\tau(c_i))\bmod 2 = \tau((Bc)^*(c_i))\bmod 2 = 0.
\end{equation}
This suffices because $\{\tau(c_i)\}$ generates $\H^*(\Uup_n;\ZZ)$ \cite[Théorèmes 8.2 et 8.3]{Bor54}.

	At this point we need a tool called the \textit{inverse transgression map}.\index[terminology]{inverse
	transgression map} We will say more about this map in \cref{transgression_detail} at the end of this chapter; for
	this proof, we need only that inverse transgression is a map $\tau\colon\H^\ell(\BG;\ZZ)\to\H^{\ell-1}(G;\ZZ)$
	satisfying two key properties:
	\begin{enumerate}[(1)]
		\item $\tau$ is natural in $G$, and
		\item for $x\in \H^*(\BG;\ZZ)$, $\tau(x^2) = 0$.
	\end{enumerate}
	Let $Bc\colon\mathrm{BO}(n)\to\mathrm{BU}(n)$ be the map induced from complexification on classifying spaces. We
	know $(Bc)^*(c_i) \bmod 2 = w_i^2$ \cite[Theorem 1.5]{Bro82}, so
	\begin{equation}
		c^*(\tau(c_i))\bmod 2 = \tau((Bc)^*(c_i))\bmod 2 = 0.
	\end{equation}
	This suffices because $\{\tau(c_i)\}$ generates $\H^*(\Uup_n;\ZZ)$ \cite[Théorèmes 8.2 et 8.3]{Bor54}.

	For more general $k$, recall that $\St_n(\RR^{n+k}) \cong \Or_{n+k}/\Or_k$, and likewise $\St_n(\CC^{n+k}) \cong
	\Uup_{n+k}/\Uup_k$. Let $\pi$ denote the quotient $\Or_{n+k}\to\St_n(\RR^{n+k})$ as well as its complex analogue.
	Then $\pi$ commutes with complexification, so it suffices to show that
	$\pi^*\colon\H^*(\St_n(\RR^{n+k};\ZZ/2)\to\H^*(\Or_{n+k};\ZZ/2)$ is injective, and this is due to Borel \cite[\S
	10]{Bor53}.
\end{proof}
	
This extra factor of two provides an additional obstruction to the existence of a conformal immersion, and this is
what we will use to show $\RRP^3$ cannot conformally immerse in $\RR^4$. 

\begin{thm}[{\cite[Theorem 5.14]{cs}}]
	\label{extra_factor_of_two}
	Let $M$ be an $n$-dimensional Riemannian manifold, $B(M)\to M$ be the principal $\Or_n$-bundle of 
	frames, and $\conn$ be the Levi-Civita connection on $B(M)$. Suppose $M$ conformally immerses in $\RR^{n+k}$;
	then, for $i\ge \lfloor k/2\rfloor$, $\CS_{P_i^\perp}(\conn)$ is contained in the even integer lattice.
\end{thm}

\begin{proof}
	Let $\varphi\colon M\to\RR^{n+k}$ be a conformal immersion. By \cref{diff_p_conformally_invariant}, we can assume
	$\varphi$ is an isometric immersion. We then have a Gauss map $\Phi\colon M\to\Gr_n(\RR^{n+k})$ sending $x\mapsto
	T_xM\subset T_x\RR^{n+k} = \RR^{n+k}$, as well as its analogue on total spaces $\Phi\colon B(M)\to\St_n(\RR^{n+k})$
	defined analogously.

	For $i > \lfloor k/2\rfloor$, we know by \cref{cpx_stiefel,double_cpxif} that
	\begin{equation*}
		[\CS_{P_i^\perp}(\conn^{\mathrm{can}})]\in \H^{2i-1}(\St_n(\RR^{n+k});\RR) 
	\end{equation*}
	is contained in the even integer
	lattice. This property is natural in principal bundles with a connection, and $\conn =
	\Phi^*(\conn^{\mathrm{can}})$, so this is also true for $\CS_{P_i^\perp}(\conn)$.
\end{proof}

We use this to define an $\RR/\ZZ$-valued invariant which obstructs conformal immersions of an orientable
Riemannian $3$-manifold $Y$ into $\RR^4$. The frame bundle $B(Y)\to Y$ admits a section $\chi$; define
\begin{equation}
	\Phi(Y) \colonequals \int_Y \frac 12 \chi^*\CS_{P_1}(\conn)\in\RR/\ZZ,
\end{equation}
where $\conn$ is the Levi-Civita connection. A priori this depends on the section, but one can calculate (e.g.\
\cite[\S 6]{cs}) that if $\chi$ and $\chi'$ are two sections, the difference of their pullbacks of the
Chern--Simons invariant consists of torsion and an integer number of copies of an integral cohomology class; the
torsion disappears when we integrate, and the integer-valued cohomology class does not affect the answer mod $\ZZ$.
\Cref{extra_factor_of_two} (and the fact that $P_1^\perp = -P_1$) implies that if $Y$ conformally immerses in
$\RR^4$, then $\Phi(Y) = 0$.

And now the moment we've all been waiting for.

\begin{thm}[{\cite[\S 6, Example 1]{cs}}]
	\label{nope_RP3}
	The manifold $\RRP^3$ with the round metric does not conformally immerse into $\RR^4$.
\end{thm}

\begin{proof}
	We will calculate $\CS_{P_1}(\conn)$ for $\conn$ the Levi-Civita connection on $\RRP^3$. The identification
	$\RRP^3 = \SO(3)$ gives us an orthonormal basis $\{v_1, v_2, v_3\}$ of $\mathfrak{so}(3)$, the space of
	left-invariant vector fields; in the Levi-Civita connection, $\nabla_{v_1}v_2 = v_3$, $\nabla_{v_2}v_3 = v_1$, and
	$\nabla_{v_1}v_3 = -v_2$. If $\pi\colon B_{\Or}(\RRP^3)\to\RRP^3$ denotes the bundle of orthonormal frames, the
	above basis gives us a section $\chi$ of $\pi$. We have a formula for $\chi^*\CS_{P_1}(\conn)$~\eqref{CS_fun};
	expanding in coordinates and using the covariant derivatives of the $v_i$s, and we obtain
	\begin{equation}
		\chi^*\left(\frac 12\CS_{P_1}(\conn)\right) = -\frac{1}{2\pi^2} \,\mathrm{vol}\comma
	\end{equation}
	where $\mathrm{vol}$ is the volume form on $\RRP^3$. As a Riemannian manifold, $\RRP^3$ with the round metric is
	the quotient of $\Sph{3}$ with the round metric under the antipodal map, so the volume of $\RRP^3$ is one-half that of
	$\Sph{3}$, i.e.\ $\mathrm{Vol}(\RRP^3) = \pi^2$. Thus $\Phi(\RRP^3) = 1/2$.
\end{proof}
%-------------------------------------------------------------------%
%  ℝP³                                                              %
%-------------------------------------------------------------------%
%
%\subsubsection{\texorpdfstring{$\RRP^3$}{ℝP³}}
%TODO: in this section, we will go into the additional integrality result (probably just sketching the proof) and
%then show $\RRP^3$ does not conformally immerse in$\RR^4$.
%
%In the Chern--Simons paper, they prove that $\RRP^3$ cannot conformally immerse in $\RR^4$. One would hope to show this by showing $\phat_1^\perp(T\RRP^3)$ is non vanishing. Note that $\phat_1^\perp = -\phat_1$. Also note that we are on a $3$ manifold so the characteristic class and curvature of $\phat_1$ both vanish, but there is a little more information in the differential character. One can compute this it by choosing a section of the orthogonal frame bundle of $\RRP^3$, and integrating the pullback of the Chern--Simons form by this section. Unfortunately this integral gives $1$, and we are supposed to consider it as an element of $\RR/\ZZ$. In the paper they go further and show that it is actually well defined mod $2\ZZ$. At least thats what I understood.
%
There are numerous examples in the literature of calculations of this sort to obtain conformal nonimmersion
results: see \cite{HL74, APS2, Mil75, Don77, Tsu81, Bac82, Tsu84, Ouy94, MM01, MZ10, PT10, Li15} for some examples.
\begin{remark}[(Transgression and inverse transgression)]
\label{transgression_detail}
Here we go into a little more detail about the transgression and inverse transgression maps, the latter of which
appeared in the proof of \cref{double_cpxif}. We follow \cites[\S 9]{Bor55}{cs}.
\begin{definition}
Let $F\overset i\to E\overset{\pi}{\to} B$ be a fiber bundle, $x\in \H^k(F; A)$, and $y\in \H^{k+1}(B; A)$. We say
that \textit{$x$ transgresses to $y$}\index[terminology]{transgression} when there is a cochain $c\in Z^k(F; A)$
such that $[i^*(c)] = x$ and $\delta c = \pi^*b$ for some cocycle $b$ in the cohomology class of $y$.
\end{definition}
Given $x$, $y$ may not exist, and may not be unique if it exists. Transgression is natural under pullback of fiber
bundles, so when studying transgression in principal $G$-bundles, it makes sense to work universally in $G\to
\EG\to\BG$.

Transgression has something to say about the Serre spectral sequence\index[terminology]{Serre
spectral sequence} for the fiber bundle $F\to E\to B$. We can identify $x$ and $y$ with their images on the
$E_2$-page, in $E_2^{0,k}$ and $E_2^{k+1,0}$ respectively. Transgression as defined above is equivalent to asking
that
\begin{enumerate}[(1)]
	\item no differential $d_r$ for $r < k+1$ kills $x$ or $y$, so that their images in the $E_{k+1}$-page are
	nonzero; and
	\item $d_{k+1}(x) = y$.
\end{enumerate}
The Serre spectral sequence is first-quadrant, so $d_{k+1}$ is the last differential that could kill $x$ or $y$.
In the bundle $G\to \EG\to \BG$, all positive-degree elements must be killed by differentials, because $\EG$ is
contractible; this is another indication that transgression is important here.\footnote{Similarly, when $A$ is an
abelian group, there is a fibration $\mathrm K(A, n)\to E\to \mathrm K(A, n+1)$, where $E$ is contractible, and a
theorem of Borel \cite[Theorem 13.1]{Bor53} on transgression is a crucial part of Serre's calculation \cite{Ser53}
of the cohomology of Eilenberg--Mac Lane spaces.\index[terminology]{Eilenberg--Mac Lane space}} When $G$ is a
connected Lie group, transgression is often as nice as it can be: $\H^*(G;A)$ is an exterior algebra on odd-degree
generators $x_1,\dotsc,x_n$, $\H^*(\BG;A)$ is a polynomial algebra on even-degree generators $y_1,\dotsc,y_n$, and
$x_i$ transgresses to $y_i$. Here $A$ may be $\QQ$, $\ZZ/p$, or $\ZZ$ depending on $G$; for example, when $G =
\Uup_n$, we can use $\ZZ$ coefficients. In these settings we can begin to see how to define the inverse
transgression map: ignoring gradings, the only differences between the rings $\H^*(\BG;A)$ and $\H^*(G;A)$ are the
relations $x_i^2 = 0$, so we can think of transgression as a map $\H^*(G;A)\to \H^{*+1}(\BG; A)$ whose image is
everything not containing terms of the form $y_i^m$ for $m > 1$. Thus we can define an inverse transgression
map\index[terminology]{inverse transgression map} $\tau$ by sending $y_i\mapsto x_i$ and $y_i^2 = 0$.

Chern--Simons \cite[\S 5]{cs} define $\tau$ differently, and more directly: given $y\in \H^{k+1}(\BG;A)$, let $b$ be
a cocycle representative for $y$ which vanishes when pulled back to any point of $\BG$; since $\EG$ is
contractible, $\pi^*(b) = \delta c$ for some $c\in Z^k(\EG; A)$. Then $\tau(y)$ is defined to be the cohomology
class of the restriction of $c$ to a fiber; one has to check this is well-defined, but it is. When $\H^*(G; A)$ is
an exterior algebra on odd-degree generators, this definition recovers the definition from the previous paragraph,
but this definition is more general. It is natural in $G$, and $\tau(y^2) = 0$ follows because if we choose $b,c$
as above, then $\delta(b\cupprod c) = \pi^*(b\cupprod b)$, and restricted to a fiber, $b\cupprod c$ vanishes.

From here it is natural to wonder whether the inverse transgression map admits a differential refinement
$\hat\tau\colon\Hhat^4(\BG;\ZZ)\to\Hhat^3(G;\ZZ)$. This is true, and there are constructions of this map due to
Carey--Johnson--Murray--Stevenson--Wang \cite[\S 3]{CJMSW05} and Schreiber \cite[1.4.1.2]{Urs}.

Chern--Simons (\textit{ibid.}, \S 3) also discuss transgression in the context of the Chern--Simons form and when
the fiber bundle is a principal $G$-bundle $P\to M$ with connection $\conn$. Fixing an invariant polynomial $f$,
they use the Maurer--Cartan form\index[terminology]{Maurer--Cartan form} on $G$ to define a class in $\HdR^*(G)$
which transgresses to $[f(\curvature\conn)]\in\HdR^*(M)$.
\end{remark}
%
%
%
%
%
%Let $M$ be a Riemannian 3 manifold. Let $\conn_{LC}$ be the Levi-Civita connection. We can use the variation formula to show that the Pontryagin Classes classes $\hat{p_i}(\conn_{LC})$ are invariant under conformal changes in metric. It is sufficient to check invariance along a path of conformally related metrics $e^{tf}g$ where $f\in C^{\infty}(M)$, $t$ is a parameter, and $g$ is a metric. The proof is a little hard for people like me who aren't super up on their Riemannian geometry, but an important part is the variation formula.
%Let $M$ be a Riemannian 3 manifold. Let $\Theta_{LC}$ be the Levi--Civita connection. We can use the variation formula to show that the Pontryagin Classes classes $\hat{p_i}(\Theta_{LC})$ are invariant under conformal changes in metric. It is sufficient to check invariance along a path of conformally related metrics $e^{tf}g$ where $f\in C^{\infty}(M)$, $t$ is a parameter, and $g$ is a metric. The proof is a little hard for people like me who aren't super up on their Riemannian geometry, but an important part is the variation formula.
%
%-------------------------------------------------------------------%
%  Variation Formula                                                %
%-------------------------------------------------------------------%
%
%\subsubsection{Variation Formula}
%
%Note that $\conn'_t$ is a Lie algebra valued one form which descends to $M$, so the variation in this Chern--Simons form is by the pullback of a $(2\ell-1)$-form on $M$. 
%
%Now, let $u$ be a lift of $[f(\Omega)]$ to integral cohomology of $\BG$, so we have the differential character $S_{f,u}$. 
%$S_{f,u}(\conn_t)$ will also be changing by this $(2\ell-1)$-form on $M$. 
%As a consequence, we get back the fact of Chern--Weil theory that the characteristic class of $S_{f,u}(\conn_t)$ in $\H^{2\ell}(M,\ZZ)$ remains constant if we change the connection. 
%In the case that the $(2\ell-1)$-form is exact, our differential character isn't changing at all. 
%This is what happens in the case of our path of conformally related Levi Civita connections.
%
%
%
%
%-------------------------------------------------------------------%
%-------------------------------------------------------------------%
%  Principal bundles and Chern–Weil Theory                          %
%-------------------------------------------------------------------%
%-------------------------------------------------------------------%
%{\color{gray}
%\subsection{Principal bundles and Chern--Weil Theory}
%\index[terminology]{Chern--Weil theory}
%
%Let $G$ be a Lie group and let $g$ be its Lie algebra. Given a principal $G$-bundle $\pi:P\to M$ with connection $\conn\in \Omega^1(P;\g)$ we define its curvature to be
%$$\tilde\Omega = d\conn + [\conn\wedge\conn]$$
%The expression $[-\wedge -]$ is defined to be the composition
% $$\Omega^1(P;\g) \otimes \Omega^1(P;\g) \to \Omega^2(P;\g\otimes \g) \to \Omega^2(P;\g)$$
%If we have a Lie algebra element $X\in \g$, it induces a vector field $\xi_X$ on $P$. We can compute that $\mathcal{L}_{\xi_X}\tilde\Omega = \ad_\xi(\tilde\Omega)$ and $\iota_{X_\xi}\tilde\Omega = 0$. Together these imply that $\tilde\Omega$ is the pullback of a form $\Omega$ on $M$ valued in the adjoint bundle. We also call $\Omega$ the curvature. Now suppose $f:\g^{\otimes l}\to \RR$ is a degree $l$ invariant polynomial. We define the corresponding Chern--Weil Form as follows.
%$$w_f(\conn) \colonequals f(\Omega\wedge\cdots\wedge\Omega)\in \Omega^{2\ell}(M)$$ 
%$w_f(\conn)$ is closed, thus represents a real cohomology class on $M$. These closed forms are natural in that if we have a map $\phi:N\to M$, then
%$$w_f(\phi^*\conn) = \phi^*w_f(\conn)$$
%In other words, $f$ gives us a natural transformation
%$$w_f:\BnablaG \to \Omegacl^{2\ell}$$
%Where $\BnablaG$ denotes the functor sending a manifold to its groupoid of principal $G$ bundles with connection. It turns out we can lift $w_f$ naturally to a Cheeger--Simons differential character.
%
%
%-------------------------------------------------------------------%
%-------------------------------------------------------------------%
%  Chern–Simons forms                                               %
%-------------------------------------------------------------------%
%-------------------------------------------------------------------%
%
%\subsection{Chern--Simons forms}
%\index[terminology]{Chern--Simons form!for a pair of connections}
%
%First we define the Chern--Simons form for a pair of connections.
%Let $\pi \colon P\to M$ be a principal bundle. Let $\conn_0,\conn_1$ be connections on $P$. Let $\conn_t$ denote the connection on the principal bundle $P\times I\to M\times I$ given by
%$$\conn_t = \conn_0 + t(\conn_1 - \conn_0)$$
%where $t$ is the coordinate on $I$. Let $\Omega_t$ be the curvature of $\conn_t$.%
%\begin{defn}
%$$\alpha_f(\conn_0,\conn_1) \colonequals \int_{M\times I/M} f(\Omega_t) \in \Omega^{2\ell-1}(M)$$
%\end{defn}
%By a families version of Stokes' theorem, we have
%$$d\alpha_f(\conn_0,\conn_1) = w_f(\conn_1) - w_f(\conn_0)$$
%Which is the main important property. In some sense we only care about $\alpha_f$ up to exact forms but I'm not sure exactly what sense this is, and anyway it is convenient that it is canonically defined on the nose. I'm not sure how this is useful, but note that we can easily extend this definition by taking the linear interpolation of $n+1$ connections on $M$ times the $n$-simplex.
%$$\alpha^n_f(\conn_0,\ldots,\conn_n) \colonequals \int_{M\times \Delta^n/M} w_f(\conn_t)$$
%These forms satisfy a nice relation
%$$d\alpha^n_f(\conn_0,\ldots,\conn_n) = \sum_i (-1)^i\alpha^{n-1}_f(\conn_0,\ldots,\widehat{\conn_i},\ldots,\conn_n)$$
%Anyway, if we have a single connection $\conn$, we can define a natural Chern--Simons form, but it lives on $P$ instead of $M$. $\pi^*P$ has a tautological section, thus a tautological flat connection $\tilde\conn$
%\begin{defn}
%$\alpha_f(\conn) = \alpha_f(\tilde\conn,\pi^*\conn)$
%\end{defn}
%This $(2\ell-1)$-form satisfies the relation
%$$d\alpha_f(\conn) = \pi^*w_f(\conn)$$
%
%-------------------------------------------------------------------%
%-------------------------------------------------------------------%
%  Cheeger–Simons Differential Character                            %
%-------------------------------------------------------------------%
%-------------------------------------------------------------------%
%\subsection{Cheeger--Simons Differential Character}
%
%A Cheeger--Simons differential character on $M$ of degree $n$ is a homomorphism 
%\begin{equation*}
%	\alpha\colon \Zsm_{n-1}(M)\to \RR/\ZZ
%\end{equation*}
%from smooth $(n-1)$-cycles on $M$ to $\RR/\ZZ$ such that there exists a closed $n$ form with integral periods which, when integrated on a smooth $n$ chain $c$ gives $\alpha(\partial c)$. 
%It turns out that this closed $n$ form is unique, and we can also extract an integral degree $n$ cohomology class from $\alpha$. 
%Cheeger and Simons construct characteristic differential characters $S_{f,u}$ of principal bundles with connection given the data of an invariant polynomial $f$ which defines a cohomology class of $\BG$ with integral periods, and a lift $u$ of this class to integral cohomology. 
%
%-------------------------------------------------------------------%
%  Trivializable Bundles                                            %
%-------------------------------------------------------------------%
%
%\subsubsection{Trivializable Bundles}
%
%For simplicity, assume $P\to M$ has sections. 
%Choose a section $s$. 
%The pullback $s^*\alpha_f(\conn)$ is a $(2\ell-1)$-form, thus gives us a $(2\ell-1)$-cochain $h$ on $M$ via integration. 
%Suppose we have a $2\ell$-dimensional smooth chain $c$ in $M$. 
%Then by stokes theorem, 
%\begin{equation*}
%	\int_c w_f = \int_{\partial c} s^* \alpha_f = h(\partial c) \period
%\end{equation*}
%Thus we have a Cheeger--Simons differential character of degree $2\ell$ on $M$. 
%We should check if it depended on the section $s$. If we have another section $s'$, than there is a gauge transformation $\phi:P\to P$ taking $s$ to $s'$.
%We can put the connection $\conn_t$ on $P\times I$ which linearly interpolates from $\conn$ to $\phi^*\conn$. Let $z$ be a cycle in $M$. By Stokes' theorem:
%\begin{equation*}
%	\int_z \alpha_f(\phi^*\conn) - \int_z \alpha_f(\conn) = \int_{z\times I} w_f(\conn_t) \period
%\end{equation*}
%
%We can form a principal bundle $\tilde P$ on $M\times S^1$ using $\phi$ as a clutching function. 
%This shows the difference is an integer, because $w_f$ represents an integral cohomology class. Thus $h$ is a well defined homomorphism from $(2\ell-1)$-cycles to $\RR/\ZZ$. 
%
%In slightly more generality, we could just assume $P$ restricted to any $(2\ell-1)$-cycle has sections. This covers the important case where $G$ is simply connected and $\ell=2$, which covers the first Pontryagin Class which is the thing we often care about in QFT.
%
%-------------------------------------------------------------------%
%  General Case                                                     %
%-------------------------------------------------------------------%
%
%\subsubsection{General Case}
%
%In the general case, Cheeger and Simons use $n$-classifying spaces of Narashiman and Ramanan. Probably this can,
%and maybe this has been translated into the language of simplicial sheaves by
%someone.\index[terminology]{n-classifying space@$n$-classifying space}
%A manifold $B_{\nabla,n} G$ with principal $G$ bundle with connection $(P,\conn)$ is n-classifying if any
%principal $G$ bundle with connection on any manifold $M$ of dimension less than or equal to $n$ is a pullback of
%$(P,\conn)$, and furthermore, any two maps $M\to B_{\nabla,n} G$ inducing the same $G$ bundle with connection are
%smoothly homotopic.\index[notation]{Bnabla,n G@$B_{\nabla, n} G$}
%
%To define a cheeger simons differential character, we need to choose a lift $u\in \H^{2\ell}(\BG,\ZZ)$ of $[f(\Omega)]$. For $n$ big enough, any $n$-classifying space $B_{\nabla,n} G$ will have vanishing cohomology in degree $2\ell-1$. 
%By one of the important short exact sequences, that means that the differential cohomology of degree $2\ell$ is just built out of closed forms and integral cohomology:
%\begin{equation*}
%	\begin{tikzcd}
%		0 \arrow[r] & \displaystyle \frac{\H^{2\ell-1}(M,\RR)}{\H^{2\ell-1}(M;\ZZ)} \arrow[r] & \Hhat^{2\ell}(M) \arrow[r] & \displaystyle \Omegacl^{2\ell}(M) \crosslimits_{\H^{2\ell}(M;\RR)} \H^{2\ell}(M,\ZZ) \arrow[r] & 0 \period
%	\end{tikzcd}
%\end{equation*}
%
%So the pullback of $u$ to $B_{\nabla,n}G$ along with $f$, defines unique differential cohomology classes on all $n$-classifying spaces for $n$ sufficiently large. 
%These together sort of form the universal Cheeger--Simons characteristic class.
%
%Now it is necessary to show that given two classifying maps $f:M\to B_{\nabla,n}G$ and $f:M\to B_{\nabla,n}G'$, the pullbacks of the universal differential characters are the same. $B_{\nabla,n}G$ and $f:M\to B_{\nabla,n}G'$ both have classifying maps to some huge $B_{\nabla,N}G$ so we assume our two classifying maps are to this bigger classifying space. Now, there exists a smooth homotopy between the two maps. Pulling back along this homotopy we get a principal bundle with connection on $M\times I$. Given some $2\ell-1$ cycle $Z$ in $M$, the difference in the values of the two differential characters we get can be computed by integrating $f(\Omega)$ on $Z\times I$. However, the principal bundle with connection descends to a connection on $Z\times S^1$, thus this integral must be an integer.
%
%-------------------------------------------------------------------%
%-------------------------------------------------------------------%
%  Examples                                                         %
%-------------------------------------------------------------------%
%-------------------------------------------------------------------%
%
%\subsection{Examples}
%
%Now we get lots of interesting characteristic classes to think about. For the rest of the paper Cheeger and Simons specialize to vector bundles. The differential characteristic classes tell us things about vector bundles with connection that are sort of reminiscent of what the old characteristic classes told us about just vector bundles.
%\begin{enumerate}[(1)]
%	\item Differential Euler class: If we have a real vector bundle with connection of even rank we get
%	$\hat{\chi}(V)$ defined by the Pfaffian.\index[terminology]{Pfaffian}\index[terminology]{Euler
%	class!differential refinement}\index[terminology]{differential Euler class}
%
%	\item Differential Chern classes: These are defined for complex vector bundles with connection and are denoted
%	$\chat_i$. They are defined by symmetric polynomials.\index[terminology]{Chern class!differential
%	refinement}\index[terminology]{differential Chern class}
%
%	\item Differential Pontryagin Classes classes: These are defined for real vector bundles and are denoted
%	$\hat{p_i}$. They can be defined by complexifying, taking Chern classes, and multiplying by some
%	signs.\index[terminology]{Pontryagin class!differential refinement}\index[terminology]{differential Pontryagin
%	class}
%\end{enumerate}
%}
%-------------------------------------------------------------------%
%-------------------------------------------------------------------%
%  Obstructions to Conformal Immersions                             %
%-------------------------------------------------------------------%
%-------------------------------------------------------------------%
%
%For a real vector bundle $V$ with connection, let $V^\CC$ be its complexification, and define
%\begin{equation*}
%	\phat_k(V) \colonequals (-1)^k \chat_{2k}(V^\CC) \period
%\end{equation*}
%(I'm not sure whats up with the sign.) 
%Note that $\phat_k$ is a class of degree $4k$. 
%Also define
%\begin{equation*}
%	\phat_k^\perp(V) \colonequals \chat^\perp_k(V^\CC) \period
%\end{equation*}
%This is what they wrote, but it looks weird to me...
%
%-------------------------------------------------------------------%
%  Direct Sums                                                      %
%-------------------------------------------------------------------%
%{\color{gray} (NOTE: this still has to be added to the differential Chern--Weil section!)
%\subsubsection{Direct Sums}
%
%As with ordinary Chern classes, we have the following formula.
%$$\chat(V\oplus W) = \chat(V)*\chat(W)$$
%Where $V\otimes W$ has the Whitney sum connection. $V\oplus W$ can be induced from a product of classifying spaces which we may assume to have vanishing odd cohomology, so it suffices to check the formula there. There it follows basically because determinants multiply under direct sums. In the case we are going to apply this formula to however, $V\otimes W$ will not have the Whitney sum connection. Instead $V\otimes W$ will have a connection which is compatible with the connections on $V$ and $W$ in the following sense. It compresses to the connections on $V$ and $W$ and has curvature tensor $R\in \Omega(M;\End(V\oplus W))$ which respects the direct sum decomposition. Cheeger and Simons claim that you can use the variation formula to show that differential Chern classes for a compatible connection on $V\oplus W$ are the same as those for the whitney sum connection.
%}
%We have the same formula for Pontryagin classes of real vector bundles:
%$$\phat(V\oplus W) = \phat(V)*\phat(W)$$
%\todo[inline]{This is false as stated. Replace with the correct statement.}
%-------------------------------------------------------------------%
%-------------------------------------------------------------------%
%  Differential Euler Class                                         %
%-------------------------------------------------------------------%
%-------------------------------------------------------------------%
%\todo[inline]{mention Chern's form $Q$ which is an avatar of the differential Euler class back when we defined the
%differential Euler class --Arun}
%
%{\color{gray}
%
%\subsection{Differential Euler Class}
%
%
%Suppose we have a rank $2n$ real vector bundle $V$ with connection on a manifold $M$. The differential Euler class $\hat{\chi}(V)$ has a nice interpretation as an obstruction to flat sections of the unit sphere bundle $\pi \colon S\to M$ of $V$. 
%
%In an early paper of Chern, he defined a $2n-1$ form $Q$ on $S$ which restricts to the unit volume form on each fiber, and such that $dQ = \pi^*Pf(\Omega)$. Here $Pf$ is the Pfaffian and $\Omega$ is the curvature. He used this to give a proof of the higher dimensional gauss bonnet theorem.
%
%It turns out that $Q$ is sort of an avatar of the differential Euler class. If we have any $2n-1$ cycle $Z
%\subset M$, we can evaluate $\hat{\chi}(Z)$ by lifting it to $S$ and integrating $Q$ along it. Actually such a lift might not exist but $Z$ is always homologous to a cycle $Z'$ which does have a lift $\tilde{Z'}$. This means there is a $2n$ chain $C$ with $\partial C = Z - Z'$. Then we have
%$$\hat{\chi}(Z) = \int_{\tilde{Z'}} Q + \int_C Pf(\Omega) $$
%There are similar descriptions of differential Chern classes in the Cheeger--Simons paper with sphere bundles replaced by Stiefel bundles. The recipie for making Q and its generalizations is similar to how we made Chern--Simons forms. Dan Freed has a nice description of it. Let me know if you're interested. 
%
%
%-------------------------------------------------------------------%
%%  Flat Connections                                                 %
%%-------------------------------------------------------------------%
%
%\subsubsection{Flat Connections}
%
%Now assume the connection on $V$ is flat. Suppose $M$ is oriented and $2n-1$ dimensional. Choose a triangulation of $M$ by smooth simplices $\sigma_1,\ldots,\sigma_r$ with orientations agreeing with $M$ so that they add up to the fudamental class:
%\begin{equation*}
%	[M] = \sum_{i = 1}^r \sigma_i \period
%\end{equation*}
%Let $v_1,\ldots,v_n$ be the vertices of the triangulation. Choose an element $s_i\in \pi^{-1}(v_i)$ of the fiber of $S$ over each vertex $v_i$. Also choose an interior point of each simplex $b_i\in \sigma_i$. Inside each fiber $\pi^{-1}(b_i)$ we consider the geodesic simplex $\Sigma_i$ who's vertices are the paralell transports of the $s_i$ over the vertices of $\sigma_i$. 
%According to Cheeger and Simons the sum of the volumes of these simplices is the differential euler class evaluated on $[M]$:
%\begin{equation*}
%	\hat{\chi}(V)([M]) = \sum_{i = 1}^r \vol(\Sigma_i) \period
%\end{equation*}
%}
%
