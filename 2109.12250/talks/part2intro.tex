%!TEX root = ../diffcoh.tex
\label{part:charclasses}\label{char_class_part}

The objective of this portion of the notes is to construct, study, and use refinements of standard characteristic classes to differential cohomology. 

Historically, differential characteristic classes were studied by Cheeger and Simons \cite{MR827262}. This view is covered in \cref{DifferentialCharacteristicClasses}.


The modern approach uses the machinery of sheaves on manifolds developed in \Cref{part:basics} of these notes. 
Given a Lie group $G$, we consider three different, but related, sheaves $\Manop \to \Spc $ on $\Man$:
\begin{enumerate}[(1)]
	\item The constant sheaf at the classifying space $\BG$ of $ G $ (\Cref{ntn:classifyingspaceBG}).
	We simply denote this sheaf by $ \BG $.

	\item The sheaf $ \BbulletG = \BunG $ sending a manifold $ M $ to the groupoid of principal $ G $-bundles on $ M $ (\Cref{ex:BunG,ntn:BbulletGBnablaG}).

	\item The sheaf $ \BnablaG = \BunGnabla $ sending a manifold $ M $ to the groupoid of principal $ G $-bundles on $ M $ with connection.
\end{enumerate}%this is all wrong lol
Characteristic classes live in the de Rham cohomology of these sheaves. 

\begin{definition}
	Let $S$ be a sheaf on manifolds. 
	The \emph{de Rham cohomology} of $S$ is $\Omegabullet(S)$.
\end{definition}

\noindent For example, the de Rham cohomology $\Omegabullet(\BG)$ of the constant sheaf $\BG$ is where ordinary characteristic classes live.

\begin{remark}
	Given a manifold $M$, one can recover the differential cohomology $\Hcech^k(M)$ by taking the $k$th de Rham cohomology %%%%%finish!
\end{remark}

The de Rham cohomology of $\BnablaG$ is studied in \cref{WorkofFreedHopkins}. The de Rham cohomology $\Omegabullet(\BnablaG)$ classifies characteristic classes for $G$-bundles with connections. In \cref{WorkofFreedHopkins}, we give a proof of the main theorem of \cite{FreedHopkins}. The theorem is as follows,

\begin{theorem}[(Freed--Hopkins)]\label{FreedHopkinsThm}
	The Chern--Weil homomorphism induces an isomorphism
	\[
		(\Sym^\bullet\gdual)^G \isomorphism \Omegabullet (\BnablaG) \period
	\]
\end{theorem}

\noindent Thus the Chern--Weil construction, reviewed in \cref{ChernWeilTheory}, produces all  characteristic classes for bundles with connection. The set up for the proof of Theorem \ref{FreedHopkinsThm} uses tools similar to the Cartan model for equivariant de Rham cohomology, which we review in \cref{EquivariantdeRhamCohomology}.

The de Rham cohomology of $\BbulletG$ is a bit more complicated. The tools we use to compute $\Omegabullet \BbulletG$ originate in Bott's paper \cite{BottsPaper}. In \cref{BottsMethod}, we review the techniques used in \cite{BottsPaper} including continuous cohomology and the van Est theorem. 
The takeaway of \cref{BottsMethod} is the following theorem of Bott:

\begin{theorem}[(Bott)]
	The continuous cohomology $\Hcont^{p-q}(G;\Sym^q(\g^*))$ is isomorphic to the de Rham cohomology group $\H^{p}(\BbulletG;\Omega^q)$:
	\[
		\H^{p}(\BbulletG;\Omega^q) \isomorphic \Hcont^{p-q}(G;\Sym^q(\g^*)) \period
	\]
\end{theorem}

\noindent We will really only use Bott's theorem in degrees $p-q\leq 0$. 

In \cref{LiftsofChernClasses}, the results of \cite{BottsPaper} are applied to provide lifts of Chern classes to differential cohomology. 
In particular, we will see there exists multiple lifts of each Chern class $c_i$ to $\H^{2n}(\BbulletGL_n(\CC); \ZZ_\CC(n))$. The collection of lifts is determined by the following result, credited by Hopkins to Bott:

\begin{theorem}
	There is a pullback square
	\begin{equation*}
		\begin{tikzcd}
			\H^{2n}(\BbulletGL_k(\CC); \ZZ_\CC(n))\arrow[rr]\arrow[d] & & \H^{2n}(\BU_k;\ZZ)\arrow[d] \\
			\H^n(\BU_k\times \BU_k;\CC)\arrow[rr, "\textup{diagonal}^*"'] & & \H^{2n}(\BU_k;\CC) \period
		\end{tikzcd}
	\end{equation*}
\end{theorem}

\noindent A real analogue of this theorem provides lifts of the Pontryagin classes.

\begin{rmk}\label{rmk-OnOffDiagonal}
Note that differential cohomology $\H^i(-;\bb{Z}(j))$ is bigraded. 
The differential lifts of characteristic classes discussed in \cref{WorkofFreedHopkins} live in bidegree where $i=j$. 
We refer to these classes as ``on-diagonal." 
The classes defined in \cref{LiftsofChernClasses} live in bidegree where $i=2j$, 
and we call these ``off-diagonal" classes. 
Notationally, for a class $c$, we use $\hat{c}$ to denote an on-diagonal differential lift 
and $\tilde{c}$ for an off-diagonal lift.
\end{rmk}

As an application of this construction, in \cref{VirasoroAlgebra} we explain how a differential lift of the first Pontryagin class $\tilde{p}_1\in\H^4(\BSL(\RR);\ZZ(2))$ can be used to produce the Virasoro group. The Virasoro group is a certain central extension of $\Diffplus(\Circ)$ by $\Uup_1$,
\[
	\Uup_1\to\Vir\to\Diffplus(\Circ) \period
\]
The construction of $\Vir$ uses the fiber integration for differential cohomology covered in \cref{FiberIntegration} and pullback along the classifying map of a certain bundle. 
This process is outlined in \cref{LiftsofChernClasses} and covered in depth in \cref{VirasoroAlgebra}. 
Note that there are multiple lifts of $p_1$ to differential cohomology. 
We obtain criterion for which lift $\tilde{p}_1$ could correspond to the Virasoro algebra central extension, but we do not pin down which lift works.

As far as we know, the material in \cref{LiftsofChernClasses} and \cref{VirasoroAlgebra} does not appear elsewhere in the literature, aside from the underpinning in \cite{BottsPaper}. 
The new ideas here are due to Dan Freed, Mike Hopkins, and Constantin Teleman.





