%!TEX root = ../diffcoh.tex

\section{Bott's Method}\label{BottsMethod}
\textit{by Araminta Amabel}

For $G$ a Lie group, recall the sheaf of groupoids $\BbulletG$ from \Cref{ex:BunG,ntn:BbulletGBnablaG}. 
The goal of this section is to prove Bott's theorem \cite[Theorem 1]{BottPaper}:

\begin{thm}
	There is an isomorphism
	\[
		\H^{p}(\BbulletG;\Omega^q)=\Hcont^{p-q}(G;\Sym^q(\g^*)) \comma
	\]
	where the right-hand side is the continuous cohomology group.
\end{thm}

\subsection{Motivation and Set Up}
Let $G$ be a Lie group. 
Recall the Chern--Weil homomorphism 
\[
	\phi \colon \Sym(\gdual)^G\rta \H^*(\BG;\RR) \period  \index[terminology]{Chern--Weil homomorphism}
\]
Here, $\gdual$ denotes the linear dual of $\g$. 
We view $\gdual$ as a $G$-module under the adjoint action. 
If $G$ is compact, then this map $\varphi$ is an isomorphism.

Given any principal $G$-bundle on $X$ with connection, we get an induced map
\[
	\Sym(\gdual)^G\rta \Omega^*(X) \period
\]
 Taking $X=\BG$ with principal $G$-bundle $\EG\rta \BG$, recovers the universal case, $\varphi$. Note that this construction depends on a choice of connection, but this dependence no longer matters once we descend to cohomology. Bott's method will allow us to construct a similar map with no mention of a connection.
 %fix

%-------------------------------------------------------------------%
%-------------------------------------------------------------------%
%  Continuous Cohomology                                            %
%-------------------------------------------------------------------%
%-------------------------------------------------------------------%

\subsection{Continuous Cohomology}

The following definition can be found in \cite[\S 2]{MR494071}. 

\begin{defn} \index[terminology]{Continuous Cohomology} \index[notation]{Continuous cohomology@$\Hcont^\bullet(G;W)$}
	Let $G$ be a topological group. 
	Let $W$ be a $G$-space. 
	Then the \emph{continuous cohomology} of $G$ with coefficients in $W$ is the cohomology $\Hcont^p(G;W)$ of the cochain complex
	\[\Mapcont(G^{\times p},W)\]
	of continuous maps, with differential 
	\[\del\colon \Mapcont(G^{\times p},W)\rta \Mapcont(G^{\times p+1},W)\]
	sending a map $f\colon G^{\times p}\rta W$ to the map $(\del f)\colon G^{\times p+1}\rta W$ by
	\begin{align*}
		(\del f)(g_1,\dots,g_{p+1}) \colonequals f(g_2,\dots,g_{p+1}) &+ \left(\sum_{i=1}^{p} (-1)^i f(g_1,\dots, g_ig_{i+1},\dots,g_{p+1})\right) \\ 
		&+(-1)^{p+1}f(g_1,\dots,g_p)\cdot g_{p+1} \period
	\end{align*}
\end{defn}

Note that on the third term in $(\del f)$, we are using the action of $G$ on $W$.

\begin{ex}
Let $G$ be a topological group and $W$ a $G$-module. 
The zeroeth continuous cohomology of $G$ with values in $W$ is the fixed points, 
\[\Hcont^{0}(G;W)\simeq W^G\period
\]
\end{ex}

The following theorem of van Est can be found in \cite{MR0059285}. 

\begin{thm}[(van Est)] %cite
Let $G$ be a connected Lie group and $K\subset G$ a maximal compact subgroup. Then there is an equivalence
\[ \Hcont^\bullet(G;A)\simeq \HLie^\bullet(\g,\kfrak;A)\]
for any $G$-space $A$.
\end{thm} \index[terminology]{van Est Theorem}

See \cite[\S 5]{MR494071} for a discussion of this result, and \cite{MR147577} for generalizations. 

\begin{cor}
	Let $G$ be a compact, connected Lie group. For $i>0$,
	\[
		\Hcont^i(G;A)=0 \period
	\]%this is corollary of Van Est theorem 
\end{cor}%i think can prove directly too?

%-------------------------------------------------------------------%
%-------------------------------------------------------------------%
% Relating Continuous Cohomology to Ordinary Cohomology             %
%-------------------------------------------------------------------%
%-------------------------------------------------------------------%

\subsection{Relating Continuous Cohomology to Ordinary Cohomology}

We would like to produce a map
\[ 
	\H^\bullet(\BG;\RR)\rta \Hcont^\bullet(G;\RR)
\]
when $G$ is a connected Lie group. We will produce this map as the edge map of a spectral sequence.

For $K$ a Lie group. Let $\Lie(K)=\kfrak$.

\begin{lem}\label{ss}
	Let $G$ be a connected Lie group with maximal compact subgroup $K$. 
	There is a spectral sequence whose $E_1$ term is 
	\[
		E_1^{p,q}=\left(\exterior^p((\g/\kfrak)^\vee)\otimes \Sym^q(\gdual)\right)^\kfrak
	\]
	converging to 
	\[
		E_\infty^{p,q}=\Sym^{q-p}(\kdual)\period
	\]
\end{lem}

\begin{proof}
	Note that $\g$ splits as
	\[
		\g\simeq \g/\kfrak\oplus\kfrak\period
	\]
	Thus we can rewrite the $E_1$ page as
	\[
		E_1^{p,q}=\left(\exterior^p((\g/\kfrak)^\vee)\otimes \bigoplus_{a+b=q}\Sym^a((\g/\kfrak))^\vee\otimes \Sym^b(\kfrak)^\vee)\right)^\kfrak\period
	\]

	Note that the terms $\exterior^p((\g/\kfrak)^\vee)$ and $\Sym^p((\g/\kfrak)^\vee)$ are Koszul dual. 
	During the course of the spectral sequence, these Koszul dual terms cancel each other. 
	The $E_\infty$ page is thus
	\[
		E_\infty^{p,q}=\Sym^{q-p}(\kdual)\period \qedhere
	\]
\end{proof}

We can compute the $E_2$ term of this spectral sequence directly. The $E_1$ page comes from the relative Chevalley--Eilenberg complex,
\[E_1^{p,q}=\HLie^p(\g,\kfrak;\Sym^q(\gdual))\period
\]

The $d_1$ differential is the Chevalley--Eilenberg differential. Thus the $E_2$ page is just relative Lie algebra cohomology,
\[E_2^{p,q}=\HLie^p(\g,\kfrak;\Sym^q(\gdual)) \period\]
By the van Est theorem, this relative Lie algebra cohomology can be recognized in terms of continuous cohomology,
\[
	\HLie^p(\g,\kfrak;\Sym^q(\gdual))\simeq \Hcont^p(G;\Sym^q(\gdual)) \period
\]

\begin{cor}
	Let $G$ be a connected Lie group with maximal compact subgroup $K$.
	There is a map $\H^\bullet(\BG;\RR)\rta \Hcont^\bullet(G;\RR)$.
\end{cor}

\begin{proof}
	One of the edge maps of the spectral sequence from \Cref{ss} goes from the $E_\infty$ term to the $E_2^{p,0}$ column. 
	Since $K$ is compact, the $E_\infty$ term can be identified with $\H^\bullet(\BK;\RR)$ be the Chern--Weil homomorphism. The $E_2^{p,0}$ column is  
	\[
		\Hcont^p(G;\Sym^0(\gdual))\simeq \Hcont^p(G;\RR) \period \qedhere
	\]
\end{proof}
