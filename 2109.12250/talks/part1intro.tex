%!TEX root = ../diffcoh.tex
\label{part:basics}

The goal of this first part of the text is to introduce and study \textit{differential cohomology theories}. 
The term ``differential cohomology" was first coined by Hopkins and Singer in \cite{HopkinsSinger}. 
In \cref{Introduction}, we introduce the ideas of differential cohomology theories following Cheeger--Simons \cite{MR827262} and Simons--Sullivan \cite{MR2365651}.
The basic point is that given a manifold $ M $, we can consider both the ``homotopy-theoretic'' complex of singular
cochains on $ M $, and the ``geometric'' complex of differential forms on $ M $.  These are related by the de Rham
isomorphism, and we would like to combine them together into a ``cohomology theory'' that captures both the
features of $ M $ as a homotopy type as well as the geometry of $ M $.  The thing to notice is that both the complex
of singular cochains and differential forms are \textit{sheaves} (in the homotopy-theoretic sense) on the category
of all manifolds.  So this category of sheaves on manifolds is the setting in which both these homotopy-theoretic
and geometric objects live.

Thus the perspective that we take in this text is that differential cohomology theories are \textit{sheaves of spectra on the category $ \Man $ of manifolds}.
It will also be useful to consider sheaves of spaces on $ \Man $ or sheaves with values in the derived \category of a ring; \Cref{sec:basicsetup} starts with introducing sheaves on the category of manifolds with values in any \category.
While the phrase ``sheaf on $ \Man $'' may sound somewhat daunting, it is surprisingly concrete: a sheaf $ F $ on $ \Man $ consists of a functor $ \fromto{\Man}{C} $ such that for each manifold $ M $, the restriction of $ F $ to open subsets of $ M $ defines a sheaf on $ M $. 

Let $ C $ be a presentable \category (e.g, spaces, spectra, or the derived \category of a ring).
One of the basic features of the category $ \Sh(\Man;C) $ of $ C $-valued sheaves on $ \Man $ is that the full subcategory $ \Shhi(\Man;C) $ spanned by those sheaves that invert homotopy equivalences is already familiar: 

\begin{theorem}[(\Cref{prop:Dugger})]\label{introthm:Dugger}
	Evaluation on the point defines an equivalence
	\begin{align*}
		\Gammalowerstar \colon \Shhi(\Man;C) &\equivalence C \\
		F &\mapsto F(*) \period
	\end{align*}
	Moreover, the inverse equivalence is given by the \emph{constant sheaf functor} $ \Gammaupperstar \colon \fromto{C}{\Sh(\Man;C)} $.
	That is, $ \Shhi(\Man;C) $ coincides with the full subcategory of $ \Sh(\Man;C) $ spanned by the constant sheaves. 
\end{theorem}

We call objects of $ \Shhi(\Man;C) $ \textit{$ \RR $-invariant} sheaves.
\Cref{sec:hisheaves} is dedicated to proving \Cref{introthm:Dugger}. 
In \Cref{sec:hisheaves} we also give an explicit formula for the constant sheaf functor $ \fromto{C}{\Sh(\Man;C)} $:

\begin{proposition}[(\Cref{lem:constantishi})]\label{introprop:constantishi}
	The constant sheaf functor 
	\begin{equation*}
		\Gammaupperstar \colon C \equivalence \Shhi(\Man;C) \subset \Sh(\Man;C)
	\end{equation*}
	is given by the assignment
	\begin{equation*}
		X \mapsto \big[M \mapsto X^{\Piinf(M)}\big] \period 
	\end{equation*}
	Here, $ X^{\Piinf(M)} $ denotes the \emph{cotensor} of the object $ X \in C $ by the underlying homotopy type $ \Piinf(M) $ of the manifold $ M $ (see \Cref{rec:cotensoring}).
\end{proposition}

The cotensor in \Cref{introprop:constantishi} might look a bit mystifying, but it is actually a familiar object in
the specific values of $ C $ that we're most interested in:
\begin{enumerate}[(1)]
	\item Let $ C = \Spc $ be the \category of spaces.
	In this case, the constant sheaf functor is given by
	\begin{equation*}
		X \mapsto \big[M \mapsto \Map_{\Spc}(\Piinf(M),X) \big] \period 
	\end{equation*}

	\item Let $ C = \Sp $ be the \category of spectra.
	In this case, the constant sheaf functor is given by
	\begin{equation*}
		E \mapsto \big[M \mapsto \Hom_{\Sp}(\Sigma_{+}^{\infty} \Piinf(M),E) \big] \comma 
	\end{equation*}
	where $ \Hom_{\Sp} $ is the mapping \textit{spectrum}.

	\item Let $ R $ be a ring and let $ C = \D(R) $ be the \textit{derived \category} of $ R $ obtained from the
	category of chain complexes of $ R $-modules by formally inverting the quasi-isomorphisms
	\cite[\HAthm{Definition}{1.3.5.8}, \HAthm{Proposition}{1.3.5.15}, \& \HAthm{Remark}{7.1.1.16}]{HA}.
	In this case, the constant sheaf functor is given by
	\begin{equation*}
		A_{*} \mapsto \big[M \mapsto \RHom_{R}(\Cup_*(M;R),A_{*}) \big] \period 
	\end{equation*}
	Here $ \Cup_*(M;R) $ is the complex of singular chains on $ M $, and $ \RHom_{R} $ is the derived $ \Hom $
	functor of chain complexes of $ R $-modules.
\end{enumerate}

As a consequence of \Cref{introprop:constantishi} (and some simple observations), we show that there is a chain of four adjoints
\begin{equation}\label{diag:intromanyadjoints}
	\begin{tikzcd}[sep=4em]
		\Sh(\Man;C) \arrow[r, shift left=2ex, "\Gammalowershriek"] \arrow[r, shift right=0.75ex, "\Gammalowerstar"{description, xshift=0.5em}] & C \period \arrow[l, shift left=2ex, "\Gammauppershriek", hooked'] \arrow[l, shift right=0.75ex, "\Gammaupperstar"{description, xshift=-0.5em}, hooked']
	\end{tikzcd}
\end{equation}
Here functors lie above their right adjoints.
The extreme right adjoint $ \Gammauppershriek $ has an explicit formula (see \Cref{lem:Gammauppershriekpsh}), but is not particularly useful.
On the other hand, under the identification
\begin{equation*}
	\Gammaupperstar \colon \equivto{C}{\Shhi(\Man;C)}
\end{equation*}
the extreme left adjoint $ \Gammalowershriek $ corresponds to the left adjoint to the inclusion $ \Shhi(\Man;C) \subset \break \Sh(\Man;C) $.
We initially construct the left adjoint $ \Gammalowershriek $ abstractly via the Adjoint Functor Theorem, but since it plays a very important role throughout this text, it is useful to have an explicit formula for $ \Gammalowershriek $.
\Cref{sec:localization} is dedicated to showing that $ \Gammalowershriek(F) $ is computed by a simple geometric realization.
Write $ \Deltaalg^n $ for the hyperplane
\begin{equation*}
	\Deltaalg^n \colonequals \setbar{(t_0,\ldots,t_{n}) \in \RR^{n+1}}{t_0 + \cdots + t_{n} = 1} \subset \RR^{n+1} \period
\end{equation*}

\begin{theorem}[(\Cref{cor:Gammalowershriek})]\label{introthm:Gammalowershriek}
	The left adjoint $ \Gammalowershriek \colon \fromto{\Sh(\Man;C)}{C} $ is given by the formula
	\begin{equation*}
		\Gammalowershriek(F) \equivalent \real{F(\Deltaalgdot)} \period
	\end{equation*}
\end{theorem}

\noindent \Cref{sec:localization} also explores some important consequences of \Cref{introthm:Gammalowershriek}.
For example, we give differential refinements of classifying spaces for $ G $-bundles (see \cref{subsubsec:consequencesofMSV}).

Some of the proofs in \Cref{sec:basicsetup,sec:hisheaves,sec:localization} rely on technical results about \topoi or presentable \categories.
To avoid distracting the reader from the main point of the text, we have relegated many of these details to \Cref{app:technicaldeatails}.

\Cref{sec:stable} specializes to sheaves with values in a presentable stable \category like spectra or the derived \category of a ring.
Using the many adjoint functors \eqref{diag:intromanyadjoints} constructed in \Cref{sec:hisheaves}, we prove the existence of a \textit{fracture square} that shows that every sheaf on $ \Man $ can be glued together from an $ \RR $-invariant sheaf and a sheaf with vanishing global sections (\cref{subsec:fracture}).
Using this fracture square, we provide a version of the Simons--Sullivan differential cohomology diagram (\Cref{thm:SimonsSullivanunique}) for any differential cohomology theory (\cref{subsec:diffcohdiagram}).
We also begin the study of differential refinements of spectra (\cref{subsec:refinements}).

With the basic foundations out of the way, \Cref{sec:examples} is dedicated to examples of differential cohomology theories.
These include \textit{ordinary differential cohomology} after Cheeger--Simons and Delgine (\cref{subsec:CheegerSimonsDiffchar}), and \textit{differential $ \Kup $-theory} after Hopkins--Singer (\cref{subsec:diffKtheory}).

In \Cref{sec:Delignecup} we further analyze ordinary differential cohomology by giving it a product structure called the \textit{Deligne cup product}.

\begin{definition}
	Let $ k \geq 0 $ be an integer.
	The \textit{Deligne complex} $ \ZZ(k) $ is the pullback
	\begin{equation*}
		\begin{tikzcd}
			\ZZ(k)\arrow[r]\arrow[d] \arrow[dr, phantom, "\lrcorner"{description, very near start}] & \ZZ\arrow[d] \\
			\Sigma^k\Omegacl^k\arrow[r] & \RR 
		\end{tikzcd}
	\end{equation*}
	in the \category $ \Sh(\Man;\D(\ZZ)) $ of sheaves on $ \Man $ with values in the derived \category of $ \ZZ $.
\end{definition}

The Deligne complex $ \ZZ(k) $ used to define ordinary differential cohomology.
The Deligne cup product
\begin{equation*}
	\fromto{\ZZ(m) \tensor_{\ZZ} \ZZ(n)}{\ZZ(m+n)}
\end{equation*}
is defined by combining the cup product on integral cohomology with the wedge product on differential forms.
We conclude \Cref{sec:Delignecup} with an analysis of the Deligne cup product in detail in the lowest dimensions (\cref{subsec:Delignecupexamples}).

\Cref{FiberIntegration} refines fiber integration to differential cohomology.
After reviewing fiber integration for ordinary cohomology, we introduce differential versions of Thom classes and orientations (\cref{subsec:differentialintegration}).
We then use these notions to define \textit{differential fiber integration} and explain how this works for $ \Circ $-bundles.

\Cref{sec:digressiononTransferConjecture} is a digression explaining Bachmann and Hoyois' proof of Quillen's \textit{Transfer Conjecture} \cite[Appendix C]{MotivicNorms:BachmannHoyois}.
This identifies the \category of $ \Einf $-spaces with $ \RR $-invariant sheaves on a $ 2 $-category of manifolds with morphisms \textit{correspondences}
\begin{equation*}
	\begin{tikzcd}[row sep=1.5em, column sep=0.75em]
		& N \arrow[dr] \arrow[dl] & \\
		M_0 & & M_1 \comma
	\end{tikzcd}
\end{equation*}
where the ``backwards'' maps are finite covering maps.
Restricting to grouplike objects on both sides gives a description of the \category of connective spectra in terms of sheaves on this $ 2 $-category of manifolds and correspondences.
\Cref{sec:digressiononTransferConjecture} is not used later in the text; we have included it because of its connection to \Cref{sec:basicsetup,sec:hisheaves,sec:localization}, but the uninterested reader can safely skip it.
