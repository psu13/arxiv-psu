% Blurb on back cover, gets included in lulu cover as well
% as regular version, so be careful with formatting

{
\parindent=0pt
\parskip=\baselineskip
{\OPTbacktitlefont
\textit{From the Introduction:}}
\OPTbackfont

\emph{Homotopy type theory} is a new branch of mathematics that combines aspects of several different fields in a surprising way. It is based on a recently discovered connection between \emph{homotopy theory} and \emph{type theory}.
It touches on topics as seemingly distant as the homotopy groups of spheres, the algorithms for type checking, and the definition of weak $\infty$-groupoids.

Homotopy type theory brings new ideas into the very foundation of mathematics.
On the one hand, there is Voevodsky's subtle and beautiful \emph{univalence axiom}.
The univalence axiom implies, in particular, that isomorphic structures can be identified, a principle that mathematicians have been happily using on workdays, despite its incompatibility with the ``official'' doctrines of conventional foundations.
On the other hand, we have \emph{higher inductive types}, which provide direct, logical descriptions of some of the basic spaces and constructions of homotopy theory: spheres, cylinders, truncations, localizations, etc.
Both ideas are impossible to capture directly in classical set-theoretic foundations, but when combined in homotopy type theory, they permit an entirely new kind of ``logic of homotopy types''.

This suggests a new conception of foundations of mathematics, with intrinsic homotopical content, an ``invariant'' conception of the objects of mathematics --- and convenient machine implementations, which can serve as a practical aid to the working mathematician.
This is the \emph{Univalent Foundations} program.

The present book is intended as a first systematic exposition of the basics of univalent foundations, and a collection of examples of this new style of reasoning --- but without requiring the reader to know or learn any formal logic, or to use any computer proof assistant.
We believe that univalent foundations will eventually become a viable alternative to set theory as the ``implicit foundation'' for the unformalized mathematics done by most mathematicians.

\bigskip

\begin{center}
  {\Large
  \textit{Get a free copy of the book at HomotopyTypeTheory.org.}}
\end{center}
}
