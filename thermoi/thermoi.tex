\documentclass[12pt]{extreport}

\usepackage[noBBpl,sc]{mathpazo}
\linespread{1.1}
\sloppy
\raggedbottom
\pagestyle{plain}

\usepackage[papersize={6.8in, 10.0in}, left=.4in, right=.4in, top=.6in, bottom=.9in]{geometry}

\def\a{|\psi_1\rangle}
\def\b{|\psi_2\rangle}
\def\la{\langle \psi_1}
\def\lb{\langle \psi_2}
\def\da{|M_{1}\rangle}
\def\db{|M_{2}\rangle}
\def\m0{| M_0 \rangle}
\def\M0{ M_0 }

\input{../helpers/cmrsum}



\begin{document}

\noindent
Finally, I have to state my position on the most controversial question in the whole of topos theory: how to spell the plural of topos. The reader will already have observed that I use the English plural; I do so because (in its mathematical sense) the word topos is not a direct derivative of its Greek root, but a back-formation from topology. I have nothing further to say on the matter, except to ask those toposophers who persist in talking about topoi whether, when they go out for a ramble on a cold day, they carry supplies of hot tea with them in thermoi.

\bigskip \hfill Peter Johnstone: Introduction to {\it Topos theory}

\end{document}
