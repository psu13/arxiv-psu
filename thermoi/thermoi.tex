\documentclass[12pt]{extreport}

\usepackage[noBBpl,sc]{mathpazo}
\linespread{1.1}
\sloppy
\raggedbottom
\pagestyle{plain}

\usepackage[papersize={6.8in, 10.0in}, left=.4in, right=.4in, top=.6in, bottom=.9in]{geometry}

\def\a{|\psi_1\rangle}
\def\b{|\psi_2\rangle}
\def\la{\langle \psi_1}
\def\lb{\langle \psi_2}
\def\da{|M_{1}\rangle}
\def\db{|M_{2}\rangle}
\def\m0{| M_0 \rangle}
\def\M0{ M_0 }

\makeatletter
\@ifpackageloaded{amsmath}{}{\RequirePackage{amsmath}}

\DeclareFontFamily{U}  {cmex}{}
\DeclareSymbolFont{Csymbols}       {U}  {cmex}{m}{n}
\DeclareFontShape{U}{cmex}{m}{n}{
    <-6>  cmex5
   <6-7>  cmex6
   <7-8>  cmex6
   <8-9>  cmex7
   <9-10> cmex8
  <10-12> cmex9
  <12->   cmex10}{}

\def\Set@Mn@Sym#1{\@tempcnta #1\relax}
\def\Next@Mn@Sym{\advance\@tempcnta 1\relax}
\def\Prev@Mn@Sym{\advance\@tempcnta-1\relax}
\def\@Decl@Mn@Sym#1#2#3#4{\DeclareMathSymbol{#2}{#3}{#4}{#1}}
\def\Decl@Mn@Sym#1#2#3{%
  \if\relax\noexpand#1%
    \let#1\undefined
  \fi
  \expandafter\@Decl@Mn@Sym\expandafter{\the\@tempcnta}{#1}{#3}{#2}%
  \Next@Mn@Sym}
\def\Decl@Mn@Alias#1#2#3{\Prev@Mn@Sym\Decl@Mn@Sym{#1}{#2}{#3}}
\let\Decl@Mn@Char\Decl@Mn@Sym
\def\Decl@Mn@Op#1#2#3{\def#1{\DOTSB#3\slimits@}}
\def\Decl@Mn@Int#1#2#3{\def#1{\DOTSI#3\ilimits@}}

\let\sum\undefined
\DeclareMathSymbol{\tsum}{\mathop}{Csymbols}{"50}
\DeclareMathSymbol{\dsum}{\mathop}{Csymbols}{"51}

\Decl@Mn@Op\sum\dsum\tsum

\makeatother




\begin{document}

\noindent
Finally, I have to state my position on the most controversial question in the whole of topos theory: how to spell the plural of topos. The reader will already have observed that I use the English plural; I do so because (in its mathematical sense) the word topos is not a direct derivative of its Greek root, but a back-formation from topology. I have nothing further to say on the matter, except to ask those toposophers who persist in talking about topoi whether, when they go out for a ramble on a cold day, they carry supplies of hot tea with them in thermoi.

\bigskip \hfill Peter Johnstone: Introduction to {\it Topos theory}

\end{document}
