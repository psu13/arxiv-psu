% !TEX root = clerical.tex
\section{Example: computation of \texorpdfstring{$\pi$}{π}}
\label{sec:example}

To showcase Clerical and its specification logic, we construct a program that computes~$\pi$ and prove that it really does so.
%
We carry out the task in three steps: the definition of absolute value, the definition of~$\sin$, and a computation of~$\pi$ by a zero-finding procedure. Because the definition of $\sin$ uses absolute values, and computation of~$\pi$  uses~$\sin$, we first define 
functions $\mathtt{abs}$ and $\mathtt{sin}$, using the function notation of \cref{sec:first-order-func}.


\paragraph*{Absolute value}
\label{sec:absolute-value}

As a warm-up exercise we prove that the function $\mathtt{abs}$ defined in \cref{fig:abs-def}, whose body we already discussed in \cref{sec:syntax}, computes absolute values.
%
\begin{figure}[htb]
  \centering
  \begin{mdframed}
  \small
\begin{lstlisting}
let $\mathtt{abs}$(x:$\dR$):$\dR$ :=
 lim n.
   case
      $x \rlt 2^{\iminus n\iminus  \numeral{1}}$ =>  $~\rminus x$ 
    | $\rminus 2^{\iminus n\iminus \numeral{1}} \rlt x$ => $~x$
   end
\end{lstlisting}
  \end{mdframed}
  \caption{Implementation of absolute value.}
  \label{fig:abs-def}
\end{figure}
%

\begin{figure}[htb]
{\centering\small
\begin{mdframed}
\begin{lstlisting}
[\claim{ \top }]
lim n .
  [\claim{ \top }]
  case
    [\claim{ \top }] $x \rlt 2^{\iminus n\iminus  \numeral{1}}$ [\claimp{ b :  \dB \such b  \lthen x < 2^{-n-1} }]  =>
    [\claim{ x < 2^{-n-1}}] $\rminus x$ [\claimt{ z  :  \dR \such \abs{z - \abs{x}} < 2^{-n} }]
  | [\claim{ \top }] $\rminus 2^{\iminus n\iminus \numeral{1}} \rlt x$ [\claimp{ b :  \dB \such b \lthen - 2^{-n-1} < x}] =>
    [\claim{\rminus 2^{-n-1} < x}] x [\claimt{ z  :  \dR \such \abs{z - \abs{x}} < 2^{-n} }]
  end
  [\claimt{ z  :  \dR \such \abs{z - \abs{x}} < 2^{-n} }]
[\claimt{ y  :  \dR \such y = \abs{x} }]
\end{lstlisting}
%
\end{mdframed}
}
  \caption{Specification of $\mathtt{abs}$.}
  \label{fig:abs-correct}
\end{figure}
%
The specification and proof of correctness of the body of~$\mathtt{abs}$ is shown in \cref{fig:abs-correct}.
As is customary, we interleaved code and assertions to give a sequence of reasoning steps leading from the initial precondition to the final postcondition. That is, each line of code is preceded by a precondition and succeeded by a postcondition, which doubles as the precondition for the next line code.

The outer assertion in \cref{fig:abs-correct} states that any value $y : \dR$ computed by the limit equals~$\abs{x}$.  This is established by an application of \rref{Ro-Lim}, which creates the obligation that the any value $z : \dR$ computed by the case expression is within $2^{-n}$ of~$\abs{x}$. This in turn is proved by an applying \rref{Rw-Tot-Case}, which generates obligations not shown in \cref{fig:abs-correct}, namely:
%
\begin{itemize}
\item $\top \lthen x < 2^{-n - 1} \lor -2^{-n-1} < x$,
\item $\{ x < 2^{-n-1} \}\; x < 2^{-n - \numeral{1}}\; \{ b : \dB \mid b \}^\tot$,
\item $\{ -2^{-n-1} < x \}\; -2^{-n - \numeral{1}} < x \; \{ b : \dB \mid b \}^\tot$.
\end{itemize}
%
These obviously hold.
%
Finally, there is an additional obligation due to \rref{Ro-Lim},
%
\begin{equation*}
  \top \lthen
   \some{y \in \IR}
   y = \abs{x} \land
   \all{n \in \IZ, z\in\IR} \abs{z - \abs{x}} < 2^{-n} \lthen \abs{z - y} < 2^{-n},
\end{equation*}
%
which holds because we may take $y = \abs{x}$.

\paragraph*{Sine function}

We compute the sine function by employing its Taylor expansion at~$0$:
%
\begin{equation*}
  \sin(x) = \sum_{i = 0}^\infty \frac{(-1)^i x^{2i+1}}{(2i+1)!}.
\end{equation*}
%
The method is inefficient for large~$x$, but is good enough for $3 < x < 4$, which is what we need.
%
Define the terms of the expansion and the partial sums
%
\begin{equation*}
  \mathbf{t}(i, x) \defeq \frac{(-1)^i x^{2i+1}}{(2i+1)!}
  \qquad\text{and}\qquad
  \mathbf{S}(j, x) \defeq \sum_{i=0}^j \mathbf{t}(i, x),
\end{equation*}
%
and recall the error bound for the $j$-th partial sum
%
\begin{equation}
  \label{eq:exa:sine-bound}
  \abs{\sin(x) - \mathbf{S}(j,x)} \leq \abs{\mathbf{t}(j+1, x)}.
\end{equation}
%
The recursive relations
%
\begin{align*}
  \mathbf{t}(j+1, x) &= -\mathbf{t}(j, x) \cdot x^2 / ((2j + 2)(2j + 3)), \\
  \mathbf{S}(j + 1, x) &= \mathbf{S}(j, x) + \mathbf{t}(j + 1, x)
\end{align*}
%
with initial conditions $\mathbf{t}(0, x) = x$ and $\mathbf{S}(0, x) = x$ can be converted to computation by while loops. We do so in the definition of $\mathtt{sin}$, shown in \cref{fig:sin-def}.
%
Notice how the stopping condition for the while loop uses nondeterminism to ensure totality. It may stop if the error bound $\mathtt{abs}(t)$ is smaller than $2^{-n}$, or continue if it is larger than $2^{-n-1}$.
%
\begin{figure}
  \centering
  \begin{mdframed}
  \small
\begin{lstlisting}
let $\mathtt{sin}$(x:$\dR$):$\dR$ :=
  lim $n$.
    var j := $\numeral{0}$ in
    var S := $x$ in
    var t := $-x \times x \times x/\ccoerce{\numeral{6}}$ in
    while
      (case $2^{\iminus n \iminus \numeral{1}} \rlt \mathtt{abs}(t)$=>true | $~\mathtt{abs}(t) \rlt 2^{\iminus n}$=>false)
    do
      j := j + $\numeral{1}$ ;
      S := S + t ;
      t := $-t \times x \times x/\ccoerce{\numeral{2} \imult j \iplus \numeral{3}}$
    end ;
    $S$
\end{lstlisting}
  \end{mdframed}
  \caption{The definition of $\mathtt{sin}$}
  \label{fig:sin-def}
\end{figure}

The correctness of the implementation is outlined in \cref{fig:sin-correct}.
%
The loop invariant, abbreviated as $\phi$, is
%
$j \geq 0 \land S = \mathbf{S}(j, x) \land t = \mathbf{t}(j + 1, x)$.
%
Upon exit from the loop, the invariant ensures that $S$ is the $j$-th partial sum, and the exit condition $|t| < 2^{-n}$ that the error is sufficiently small. Together they imply that $S$ is within $2^{-n}$ of $\sin(x)$, as required by \rref{Ro-Lim}.
%
We elide several side conditions, such us those imposed by \rref{Ro-Lim} and \rref{Rw-Tot-Case}.
%
The complete details of the formal proof are available in the Coq formalization, see \cref{sec:formalization}.

\begin{figure}[htb]
  \centering
  \begin{mdframed}
  \small
\begin{lstlisting}
[\claimt{\top}]
lim $n$.
  [\claimt{\top}]
  var j := $\numeral{0}$ in
  var S := $x$ in
  var t := $-x \times x \times x/\ccoerce{\numeral{6}}$ in
  [\claimt{\phi}]      [where $\phi$ is $j \geq 0 \land S = \mathbf{S}(j,x) \land t = \mathbf{t}(j + 1, x)$]
  while
    [\claimt{\phi}]
    (case $2^{\iminus n \iminus \numeral{1}} \rlt \mathtt{abs}(t)$=>true | $~\mathtt{abs}(t) \rlt 2^{\iminus n}$=>false)
    [\claimt{b : \dB\such \phi \land (b \lthen 2^{-n-1} < \abs{t}) \land (\neg b \lthen \abs{t} < 2^{-n})}]
  do
    [\claimt{\phi \land 2^{-n-1}< \abs{t}}]
    [\claimt{j \geq 0 \land S = \mathbf{S}(j,x) \land t = \mathbf{t}(j + 1, x)}]
    j := j + $\numeral{1}$ ;
    [\claimt{j - 1 \geq 0 \land S = \mathbf{S}(j-1,x) \land t = \mathbf{t}(j, x)}]
    S := S + t ;
    [\claimt{j - 1 \geq 0 \land S = \mathbf{S}(j,x) \land t = \mathbf{t}(j, x)}]
    t := $-t \times x \times x/\ccoerce{\numeral{2} \imult j \iplus \numeral{3}}$
    [\claimt{j - 1 \geq 0 \land S = \mathbf{S}(j,x) \land t = \mathbf{t}(j+1, x)}]
    [\claimt{\phi}]
  end ;
  [\claimt{\phi \land \abs{t} < 2^{-n}}]
  [\claimt{S = \mathbf{S}(j, x) \land \abs{\mathbf{t}(j + 1, x)} < 2^{-n}}]
  $S$
  [\claimt{z  :  \dR\such \abs{z - \sin(x)} < 2^{-n}}]
[\claimt{y  :  \dR\such y = \sin(x)}]
\end{lstlisting}
  \end{mdframed}
  \caption{Specification of $\mathtt{sin}$.}
  \label{fig:sin-correct}
\end{figure}


\paragraph*{Computation of $\pi$}


We compute $\pi$ by using a root-finding algorithm to find the unique root of $\sin$ in the closed interval $[3,4]$.
For this, we could use the general root-finding algorithm in~\cite{brausse2016semantics}, which repeatedly narrows the search interval by splitting it into two overlapping intervals each $2/3$ the width of the original interval.
However, the fact that  we are computing a root that is known to be irrational, allows us to instead use a more efficient bisection-based search.
%
Once again, we proceed by computing a limit whose $n$-th term approximates the root within $2^{-n}$.
%
The code is shown in \cref{fig:pi-def}.
%
The bisection method is initialized with the lower bound $l = 3$ and the upper bound $u = 4$.
At each step of the iteration, we compute the midpoint $m = \frac{l + u}{2}$ and narrow the search interval to either $[m, u]$ or $[l, m]$, depending on the sign of $\sin(m)$. The irrationality of the  root guarantees that the comparison $0 < \sin(m)$ always terminates, because $m$ is rational and so $\sin(m) \neq 0$.
%
The above considerations are incorporated into the loop invariant
%
\begin{equation*}
  l \in \IQ \land u \in \IQ \land 3 \leq l < \pi < u \leq 4 \land u - l= 2^{-k}.
\end{equation*}
%
We leave detailed verification of correctness as exercise for the reader, who can also avoid doing it by consulting the formalized proof~\cite{clerical_coq}.
%
\begin{figure}
  \centering
  \begin{mdframed}
  \small
\begin{lstlisting}
lim $p$.
  var k := 0 in
  var l := $\ccoerce{\numeral{3}}$ in
  var u := $\ccoerce{\numeral{4}}$ in
  while k < n do
    var m := $(l \rplus u) \rmult 2^{-\numeral{1}}$ in
    if $\ccoerce{\numeral{0}} \rlt \mathtt{sin}(m)$ then
      l := m
    else
      u := m
    end ;
    k := $k \iplus \numeral{1}$
  end ;
  $(l \rplus u) \rmult 2^{\iminus \numeral{1}}$
\end{lstlisting}
  \end{mdframed}
  \caption{Computation of $\pi$.}
  \label{fig:pi-def}
\end{figure}

