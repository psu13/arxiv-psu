% !TEX root = clerical.tex
\section{Specification logic}
\label{sec:specification-logic}

We devise a Hoare-style specification logic for proving the correctness of Clerical expressions.
As briefly discussed in Section~\ref{sec:overview}, 
it is necessary to have both forms of correctness: partial and total.
While both types of correctness guarantee the avoidance of erroneous execution, the difference between them is that
partial correctness allows non-termination and total correctness forbids it. 
Having both forms of correctness is necessary  for the logic's correctness and expressivity, as will become clear when we present the proof rules in \cref{sec:proof-rules}.

First, we  formulate the correctness specifications we will use, which are written in the style of Hoare-style triples. For each pure expression and general expression, we define the notions of partial and total correctness, yielding four variants, as follows.


Given a context $\Gamma = (x_1 \of \tau_1, \ldots, x_n \of \tau_n)$, a pure expression $\Gamma \rotypes e : \tau$, a \defemph{precondition} $A \subseteq \sem{\Gamma}$ and a \defemph{postcondition} $B \subseteq \sem{\Gamma} \times \sem{\tau}$, we define the \defemph{partial} and \defemph{total (read-only) correctness triples} respectively as
%
\begin{align*}
  \rotrip[\prt] {\Gamma} {A} {e} {B}
  &\iff
  \all{\gamma \in A} 
  (\sem{e} \, \gamma \neq \emptyset 
\,\land \,
  \all{v \in \sem{e} \, \gamma}
 (v \neq \bot \lthen (\gamma, v) \in B)),
  \\
  \rotrip[\tot] {\Gamma} {A} {e} {B}
  &\iff
   \all{\gamma \in A} 
  (\sem{e} \, \gamma \neq \emptyset 
\,\land \,
  \all{v \in \sem{e} \, \gamma}
 (v \neq \bot \land(\gamma, v) \in B)).
\end{align*}
%
Here, $\sem{e}$ is an abbreviation for $\sem{\Gamma \rotypes e : \tau}$.
Both forms of correctness state that the error state is not reached and that any value computed by~$e$ satisfies the postcondition. Note that partial correctness does not require termination, whereas total correctness guarantees it.

Given a logical formula $\phi$ in variables $\vec{x} = (x_1, \ldots, x_n) \in \sem{\Gamma}$ and a formula $\psi$ in variables $(\vec{x}, y) \in \sem{\Gamma} \times \sem{\tau}$, the triple
%
\begin{equation*}
  \rotrip[\prt]
  {\Gamma} {\{\vec{x} \in \sem{\Gamma} \mid \phi\}}
  {e}
  {\{(\vec{x}, y) \in \sem{\Gamma} \times \sem{\tau} \such \psi\}}
\end{equation*}
%
can be written more concisely as
%
\begin{equation}
  \label{eq:notation-rotrip}
  \rotrip[\prt]{\Gamma} {\phi} {e} {y \of \tau \such \psi}.
\end{equation}
%
Read the above as: if $\phi$ holds then $e$ does not err, and if it terminates then every value~$y$ computed by~$e$ satisfies~$\psi$. By replacing the symbol $\prt$ with $\tot$ we obtain the analogous notation for total correctness: if $\phi$ holds then~$e$ does not err, it terminates and every value~$y$ computed by~$e$ satisfies~$\psi$.

The analogous triples for general expressions are more complicated because they need to take state-change into account. Let $\Gamma; \Delta \rwtypes c : \tau$ be a general expression, $A \subseteq \sem{\Gamma} \times \sem{\Delta}$ and $B \subseteq \sem{\Gamma} \times  \sem{\Delta} \times \sem{\tau}$. Then we define the \defemph{partial} and \defemph{total (read-write) correctness triples} respectively as
%
\begin{multline*}
  \rwtrip[\prt] {\Gamma; \Delta} {A} {c} {B} \iff {} \\
  \all{(\gamma, \delta) \in A}
  (\sem{c} \, \gamma \, \delta \neq \emptyset  \, \land\,
  \all{w \in \sem{c} \, \gamma \, \delta} 
  (w \neq \bot \lthen (\gamma, \delta_w,v_w) \in B),
\end{multline*}
%
and
%
\begin{multline*}
  \rwtrip[\tot] {\Gamma; \Delta} {A} {c} {B} \iff {} \\
  \all{(\gamma, \delta) \in A}
  (\sem{c} \, \gamma \, \delta \neq \emptyset  \, \land\,
  \all{w \in \sem{c} \, \gamma \, \delta} \,
  (w \neq \bot \land (\gamma, \delta_w,v_w) \in B), 
\end{multline*}
%
where $\sem{c}$ is an abbreviation for $\sem{\Gamma; \Delta \rwtypes c : \tau}$, and we use the fact that
any element $w \in \sem{c} \, \gamma \, \delta$ which is not $\bot$ is of the form $w = (\delta_w,v_w)$, so the conclusion 
$(\gamma, \delta_w,v_w) \in B$ states that, in the new read-write state $(\gamma, \delta_w)$, the return value~$v_w$ satisfies the postcondition~$B$.

Given a formula $\phi$ in variables $(\vec{x}, \vec{y}) \in \sem{\Gamma} \times \sem{\Delta}$ and a formula $\psi$ in variables $(\vec{x}, \vec{y}, z) \in \sem{\Gamma} \times \sem{\Delta} \times \sem{\tau}$,
the triple
%
\begin{equation*}
  \rwtrip[\prt]
  {\Gamma; \Delta} {\{(\vec{x}, \vec{y}) \in \sem{\Gamma} \times \sem{\Delta} \such \phi\}}
  {c}
  {\{(\vec{x}, \vec{y}, z) \in \sem{\Gamma} \times \sem{\Delta} \times \sem{\tau} \such \psi\}}.
\end{equation*}
%
can again be written more concisely as
%
\begin{equation}
  \label{eq:notation-rwtrip}
  \rwtrip[\prt]{\Gamma; \Delta} {\phi} {c} {z \of \tau \such \psi}.
\end{equation}
%
Read the above as: if $\phi$ holds then~$c$ does not err, and if it terminates then it computes a value satisfying~$\psi$ in the new state.
By replacing~$\prt$ with~$\tot$ we get analogous notation for total correctness: if $\phi$ holds then~$c$ does not err, it terminates and computes a value satisfying~$\psi$ in the new state.

We introduce one further notational convention that streamlines reasoning about expressions of the trivial type~$\dU$: if~$\psi$ is a formula in which the variable~$y$ does not appear freely, then we write $\{ \psi \}$ instead of $\{ y \of \dU \such \psi \}$.


\subsection{Proof rules}
\label{sec:proof-rules}

The notation~\eqref{eq:notation-rotrip} is general in the sense that it can be used to express any correctness triple $\rotrip[\prt]{\Gamma}{A}{e}{B}$ by taking $\phi$ to be the formula $\vec{x} \in A$ and $\psi$ to be $(\vec{x}, y) \in B$.
%
A similar observation holds for~\eqref{eq:notation-rwtrip}, therefore, notations~\eqref{eq:notation-rotrip} and~\eqref{eq:notation-rwtrip} may be used freely in rules about correctness triples.

Many rules come in pairs differing only in the use of symbols~$\prt$ and~$\tot$. To avoid pointless duplication in such cases, we use the symbol~$\star$ to indicate that either~$\prt$ or~$\tot$ can be inserted in its place, consistently throughout a rule. For example, the rule \rref{Ro-Conj} in \cref{fig:rules-general} decompresses to the rules
%
\begin{mathpar}
  \inferenceRule{Ro-Part-Conj}{
    \rotrip[\prt] {\Gamma} {\phi} {e} {y \of \tau \such \psi_1} \\
    \rotrip[\prt] {\Gamma} {\phi} {e} {y \of \tau \such \psi_2}
  }{
    \rotrip[\prt] {\Gamma} {\phi} {e} {y \of \tau \such \psi_1 \land \psi_2}
  }

  \inferenceRule{Ro-Tot-Conj}{
    \rotrip[\tot] {\Gamma} {\phi} {e} {y \of \tau \such \psi_1} \\
    \rotrip[\tot] {\Gamma} {\phi} {e} {y \of \tau \such \psi_2}
  }{
    \rotrip[\tot] {\Gamma} {\phi} {e} {y \of \tau \such \psi_1 \land \psi_2}
  }
\end{mathpar}
%
We partition the rules into three groups:
the \emph{general rules} in \cref{fig:rules-general},
the \emph{arithmetical rules} in \cref{fig:rules-arithmetic}, and
the \emph{operational rules} in \cref{fig:rules-operational}.
%
In all of them, it is presupposed that the expressions appearing are well-typed in the indicated contexts and with the indicated type and that all the formulas are well-scoped in the given contexts.
%
Many rules have a logical side-condition for the rule to apply, written as an additional premise. 
When a formula $\phi$ in variables $\vec{x} = (x_1, \ldots, x_n) \in \sem{\Gamma}$  is written as a premise, it means that the rule only applies in the case that $\phi$ holds for all $\vec{x} \in \sem{\Gamma}$.

\begin{figure}
  \centering
  \begin{mdframed}
    \footnotesize
    \centering

    \textbf{Logical rules:}
    %
    \begin{mathpar}
      \inferenceRule{Ro-Imply}{
        \rotrip {\Gamma} {\phi'} {e} {y \of \tau \such \psi'} \\\\
        \phi \lthen \phi' \\
        \psi' \lthen \psi
      }{
        \rotrip {\Gamma} {\phi} {e} {y \of \tau \such \psi}
      }

      \inferenceRule{Rw-Imply}{
        \rwtrip {\Gamma; \Delta} {\phi'} {c} {z \of \tau \such \psi'} \\\\
        \phi \lthen \phi' \\
        \psi' \lthen \psi
      }{
        \rwtrip {\Gamma; \Delta} {\phi} {c} {z \of \tau \such \psi}
      } \\

      \inferenceRule{Ro-ExFalso}{
      }{
        \rotrip {\Gamma} { \bot } {e} { \psi }
      }

      \inferenceRule{Rw-ExFalso}{
      }{
        \rwtrip {\Gamma; \Delta} { \bot } {c} { \psi }
      }

      \inferenceRule{Ro-Conj}{
        \rotrip {\Gamma} {\phi} {e} {y \of \tau \such \psi_1} \\
        \rotrip {\Gamma} {\phi} {e} {y \of \tau \such \psi_2}
      }{
        \rotrip {\Gamma} {\phi} {e} {y \of \tau \such \psi_1 \land \psi_2}
      }

      \inferenceRule{Rw-Conj}{
        \rwtrip {\Gamma; \Delta} {\phi} {c} {z \of \tau \such \psi_1} \\
        \rwtrip {\Gamma; \Delta} {\phi} {c} {z \of \tau \such \psi_2}
      }{
        \rwtrip {\Gamma; \Delta} {\phi} {c} {z \of \tau \such \psi_1 \land \psi_2}
      }

      \inferenceRule{Ro-Disj}{
        \rotrip {\Gamma} {\phi_1} {e} {y \of \tau \such \psi} \\
        \rotrip {\Gamma} {\phi_2} {e} {y \of \tau \such \psi}
      }{
        \rotrip {\Gamma} {\phi_1 \lor \phi_2} {e} {y \of \tau \such \psi}
      }

      \inferenceRule{Rw-Disj}{
        \rwtrip {\Gamma; \Delta} {\phi_1} {c} {z \of \tau \such \psi} \\
        \rwtrip {\Gamma; \Delta} {\phi_2} {c} {z \of \tau \such \psi}
      }{
        \rwtrip {\Gamma; \Delta} {\phi_1 \lor \phi_2} {e} {z \of \tau \such \psi}
      }
    \end{mathpar}
    
    \medskip

    \textbf{Variables and constants:}
    %
    \begin{mathpar}
      \inferenceRule{Ro-Var}{
      }{
        \rotrip {\Gamma_1, x \of \tau, \Gamma_2} {\psi[x/y]} {x} {y \of \tau \such \psi}
      }

      \inferenceRule{Ro-Skip}{
      }{
        \rotrip {\Gamma} {\psi} \cskip {\psi}
      }

      \inferenceRule{Ro-Bool-Const}{
        b \in \{\cfalse, \ctrue\}
      }{
        \rotrip {\Gamma} {\psi[b/y]} b {y \of \dB \such \psi}
      }

      \inferenceRule{Ro-Int-Const}{
        k \in \IZ
      }{
        \rotrip {\Gamma} {\psi[k/y]} {\numeral{k}} {y \of \dZ \such \psi}
      }
    \end{mathpar}

\medskip

    \textbf{Passage between read-only and read-write correctness:}
    %
    \begin{mathpar}
      \inferenceRule{Rw-Ro}{
        \rwtrip {\Gamma; \emptyctx} {\phi} {c} {z \of \tau \such \psi}
      }{
        \rotrip {\Gamma} {\phi} {c} {z \of \tau \such \psi}
      }

      \inferenceRule{Ro-Rw}{
        \rotrip {\Gamma, \Delta} {\phi} {e} {y \of \tau \such \psi}
      }{
        \rwtrip {\Gamma; \Delta} {\phi} {e} {y \of \tau \such \psi}
      }
    \end{mathpar}

  \end{mdframed}
  \caption{General proof rules}
  \label{fig:rules-general}
\end{figure}


\begin{figure}
  \centering
  \begin{mdframed}
    \footnotesize
    \centering

    \textbf{Coercion and exponentiation:}
    %
    \begin{mathpar}
      \inferenceRule{Ro-Coerce}{
        \rotrip {\Gamma} {\phi} e {y \of \dZ \such \psi[\inclZ{y}/y]}
      }{
        \rotrip {\Gamma} {\phi} {\ccoerce{e}} {y \of \dR \such \psi}
      }

      \inferenceRule{Ro-Exp}{
        \rotrip {\Gamma} {\phi} {e} {x \of \dZ \such \psi[2^x/y]} 
      }{
        \rotrip {\Gamma} {\phi} {2^e} {y \of \dR \such \psi}
      }

    \end{mathpar}
    
    \medskip

    \textbf{Integer arithmetic:}
    \begin{mathpar}
      \inferenceRule{Ro-Int-Op}{
        \rotrip {\Gamma} {\phi} {e_1} {y_1 \of \dZ \such \psi_1} \\
        \rotrip {\Gamma} {\phi} {e_2} {y_2 \of \dZ \such \psi_2} \\\\
        \psi_1 \land \psi_2 \lthen \psi[(y_1 \op y_2)/y]
      }{
        \rotrip {\Gamma} {\phi} {e_1 \iop e_2} {y \of \dZ \such \psi}
      }
    \end{mathpar}

\medskip

    \textbf{Real arithmetic:}
    \begin{mathpar}
      \inferenceRule{Ro-Real-Op}{
        \rotrip {\Gamma} {\phi} {e_1} {y_1 \of \dR \such \psi_1} \\
        \rotrip {\Gamma} {\phi} {e_2} {y_2 \of \dR \such \psi_2} \\\\
        \psi_1 \land \psi_2 \lthen \psi[(y_1 \op y_2)/y]
      }{
        \rotrip {\Gamma} {\phi} {e_1 \rop e_2} {y \of \dR \such \psi}
      }
    \end{mathpar}

\medskip

    \textbf{Reciprocal:}
    %
    \begin{mathpar}
      \inferenceRule{Ro-Part-Recip}{
        \rotrip[\prt] {\Gamma} {\phi} {e} {x \of \dR \such \theta} \\\\
        \theta \land x \neq 0 \lthen \psi[x^{-1}/y]
      }{
        \rotrip[\prt] {\Gamma} {\phi} {\cinv{e}} {y \of \dR \such \psi}
      }

      \inferenceRule{Ro-Tot-Recip}{
        \rotrip[\tot] {\Gamma} {\phi} {e} {x \of \dR \such \theta} \\\\
        \theta \lthen x \neq 0 \land \psi[x^{-1}/y]
      }{
        \rotrip[\tot] {\Gamma} {\phi} {\cinv{e}} {y \of \dR \such \psi}
      }
    \end{mathpar}

\medskip

    \textbf{Integer comparison ${\sim} \in \{{\ieq}, {\ilt}\}$:}
    \begin{mathpar}
      \inferenceRule{Ro-Int-Comp}{
        \rotrip {\Gamma} {\phi} {e_1} {y_1 \of \dZ \such \psi_1} \\
        \rotrip {\Gamma} {\phi} {e_2} {y_2 \of \dZ \such \psi_2} \\\\
        \psi_1 \land \psi_2 \lthen \psi[(y_1 \sim y_2)/b]
      }{
        \rotrip {\Gamma} {\phi} {e_1 \sim e_2} {b \of \dB \such \psi}
      }
    \end{mathpar}

\medskip

    \textbf{Real comparison:}
    \begin{mathpar}
      \inferenceRule{Ro-Part-Real-Lt}{
        \rotrip[\prt] {\Gamma} {\phi} {e_1} {y_1 \of \dR \such \psi_1} \\
        \rotrip[\prt] {\Gamma} {\phi} {e_2} {y_2 \of \dR \such \psi_2} \\\\
        \psi_1 \land \psi_2 \land y_1 \neq y_2 \lthen \psi[(y_1 < y_2)/b]
      }{
        \rotrip[\prt] {\Gamma} {\phi} {e_1 \rlt e_2} {b \of \dB \such \psi}
      }

      \inferenceRule{Ro-Tot-Real-Lt}{
        \rotrip[\tot] {\Gamma} {\phi} {e_1} {y_1 \of \dR \such \psi_1} \\
        \rotrip[\tot] {\Gamma} {\phi} {e_2} {y_2 \of \dR \such \psi_2} \\\\
        \psi_1 \land \psi_2 \lthen y_1 \neq y_2 \land \psi[(y_1 < y_2)/b]
      }{
        \rotrip[\tot] {\Gamma} {\phi} {e_1 \rlt e_2} {b \of \dB \such \psi}
      }
    \end{mathpar}

\medskip

    \textbf{Limit:}
    \begin{mathpar}
      \inferenceRule{Ro-Lim}{
        \rotrip[t] {\Gamma, x \of \dZ} {\phi} {e} {z \of \dR \such \theta} \\
        \phi \lthen \some{y \in \IR} \psi \land \all{x \in \IZ} \all{z \in \IR} \theta \lthen |z - y| < 2^{-x}
      }{
        \rotrip {\Gamma} {\phi} {\clim{x} e} {y \of \dR \such \psi}
      }
    \end{mathpar}

  \end{mdframed}
  \caption{Arithmetical proof rules}
  \label{fig:rules-arithmetic}
\end{figure}


\begin{figure}
  \centering
  \begin{mdframed}
    \footnotesize
    \centering

    \begin{mathpar}
      \inferenceRule{Rw-Sequence}{
        \rwtrip {\Gamma; \Delta} {\phi} {c_1} {\theta} \\
        \rwtrip {\Gamma; \Delta} {\theta} {c_2} {z \of \tau \such \psi}
      }{
        \rwtrip {\Gamma; \Delta} {\phi} {c_1 ; c_2} {z \of \tau \such \psi}
      }

      \inferenceRule{Rw-New-Var}{
        \rotrip {\Gamma, \Delta} {\phi} {e} {x \of \sigma \such \theta} \\
        \rwtrip {\Gamma; \Delta, x \of \sigma} {\theta} {c} {z \of \tau \such \psi}
      }{
        \rwtrip {\Gamma; \Delta} {\phi} {\cvar{x}{e} c} {z \of \tau \such \psi}
      }

      \inferenceRule{Rw-Assign}{
        \rotrip {\Gamma, \Delta} {\phi} {e} {x \of \tau \such \theta} \\
        \theta \lthen \psi[x/y]
      }{
        \rwtrip {\Gamma; \Delta_1, y \of \sigma, \Delta_2} {\phi} {\clet{y}{e}} {\psi}
      }

      \inferenceRule{Rw-Cond}{
        \rotrip {\Gamma, \Delta} {\phi} {e} {b \of \dB \such \theta } \\
        \rwtrip {\Gamma; \Delta} {\theta[\semtt/b]} {c_1} {z \of \tau \such \psi} \\
        \rwtrip {\Gamma; \Delta} {\theta[\semff/b]} {c_2} {z \of \tau \such \psi} \\
      }{
        \rwtrip {\Gamma; \Delta} {\phi} {\cif e \cthen c_1 \celse c_2 \cend} {z \of \tau \such \psi}
      }

      \inferenceRule{Rw-Part-Case}{
        {\begin{aligned}
        &\rotrip[\prt] {\Gamma, \Delta} {\phi} {e_i} {b \of \dB \such \theta_i }
         &&\text{for $i = 1, \ldots, n$} \\[-0.25em]
        &\rwtrip[\prt] {\Gamma; \Delta} {\theta_i[\semtt/b]} {c_i} {z \of \tau \such \psi}
         &&\text{for $i = 1, \ldots, n$}
        \end{aligned}}
      }{
        \rwtrip[\prt]
          {\Gamma; \Delta}
          {\phi}
          {\ccase e_1 \To c_1 \mid \cdots \mid e_n \To c_n \cend}
          {z \of \tau \such \psi}
      }

      \inferenceRule{Rw-Tot-Case}{
        {\begin{aligned}
          &\rotrip[\prt] {\Gamma, \Delta} {\phi} {e_i} {b \of \dB \such \theta_i}
            &&\text{for $i = 1, \ldots, n$} \\[-0.25em]
          &\rwtrip[\tot] {\Gamma; \Delta} {\theta_i[\semtt/b]} {c_i} {z \of \tau \such \psi}
            &&\text{for $i = 1, \ldots, n$} \\[-0.25em]
          &\phi \lthen \phi_1 \lor \cdots \lor \phi_n \\[-0.25em]
          &\rotrip[\tot] {\Gamma, \Delta} {\phi_i} {e_i} {b \of \dB \such b = \semtt}
            &&\text{for $i = 1, \ldots, n$}
        \end{aligned}}
      }{
        \rwtrip[\tot]
          {\Gamma; \Delta}
          {\phi}
          {\ccase e_1 \To c_1 \mid \cdots \mid e_n \To c_n \cend}
          {z \of \tau \such \psi}
      }

      \inferenceRule{Rw-Part-While}{
        \rotrip[\prt] {\Gamma, \Delta} {\phi} {e} {b \of \dB \such \theta} \\
        \rwtrip[\prt] {\Gamma; \Delta} {\theta[\semtt/b]} {c} {\phi}
      }{
        \rwtrip[\prt] {\Gamma; \Delta} {\phi} {\cwhile e \cdo c \cend} {\phi \land \theta[\semff/b]}
      }


      \inferenceRule{Rw-Tot-While}{
        \rotrip[\tot] {\Gamma, \Delta} {\phi} {e} {b \of \dB \such \theta} \\
        \rwtrip[\tot]
           {\Gamma; \Delta}
           {\theta[\semtt/b]}
           {c}
           {\phi}
        \\\\
        \phi \lthen (\text{$\psi$ well-founded}) \\
        \rwtrip[\tot]
           {\vec{z} \of \Delta', \Gamma; \vec{y} \of \Delta}
           {\theta[\semtt/b] \land \vec{z} = \vec{y} }
           {c}
           {\psi}
      }{
        \rwtrip[\tot] {\Gamma; \Delta} {\phi} {\cwhile e \cdo c \cend} {\phi \land \theta[\semff/b] }
      }

    \end{mathpar}

  \end{mdframed}
  \caption{Operational proof rules}
  \label{fig:rules-operational}
\end{figure}

The general rules in \cref{fig:rules-general} postulate logical principles and govern variables and constants. The rules \rref{Ro-Rw} and \rref{Rw-Ro} allow passage between reasoning about pure and general expressions.

Among the arithmetical rules in \cref{fig:rules-arithmetic} we point out the difference between the partial reciprocal rule \rref{Ro-Part-Recip} and its total variant \rref{Ro-Tot-Recip}. In the former, we may assume the argument to be non-zero, while in the latter we have to prove it.
%
A similar phenomenon happens with the strict comparison~$x \rlt y$ on the reals, where the partial rule \rref{Ro-Part-Real-Lt} provides the assumption $x \neq y$, whereas its total variant \rref{Ro-Tot-Real-Lt} requires a proof of $x \neq y$. The condition $x \neq y$ appears in these rules because $x \rlt y$ diverges when $x = y$. 
% There was previously explanation of this divergence here in terms of semidecidability, but since the divergence has already been discussed in earlier sections 

The limit rule \rref{Ro-Lim} states that $\clim{x} e$ computes $y \in \IR$ when~$e$ denotes a sequence rapidly converging to~$t$, i.e., any value computed by~$e$ is within distance~$2^{-x}$ of~$y$.
%
Note that the first premise imposes \emph{total} correctness, even in the partial version of the rule. This is necessary because the denotation of $\clim{x} e$ is~$\PPerr$ (rather than~$\PPbot$) as soon~$e$ fails to be total.

The first three operational rules in \cref{fig:rules-operational}, which regulate sequencing, new variables and assignment, require no comment.
%
The remaining rules all handle the conditions and guards in a similar fashion,
so we only look at the rule for conditional \rref{Rw-Cond}, keeping in mind that the boolean expression~$e$ is nondeterministic. The postcondition~$\theta$ describes the \emph{possible} values~$b$ of~$e$. The preconditions $\theta[\semtt/b]$ and $\theta[\semff/b]$ of~$c_1$ and $c_2$ should respectively be read as ``if $e$ evaluates to true'' and ``if $e$ evaluates to false''. 
To see this formally, consider the triple in the premise
\[
\rotrip {\Gamma, \Delta} {\phi} {e} {b \of \dB \such \theta}
\]
and a state $\gamma \in\sem{\Gamma}$ that satisfies the precondition $\phi$.
If $\semtt \in \sem{\Gamma \rotypes e : \dB}\,\gamma$, by the definition of triples, $\gamma$ satisfies $\theta[\semtt/b]$.
Hence, $\theta[\semtt/b]$ is a necessary condition of the states satisfying $\phi$ to let $e$ admit a nondeterministic branch leading to $\semtt$,  allowing a possibility of having some branches fail to terminate when $\star = \prt$.
Due to nondeterminism, the conditions $\theta[\semtt/b]$ and $\theta[\semff/b]$ need not exclude each other, even when~$\theta$ is as informative as it can be.
%
Incidentally, if desired the original precondition $\phi$ may be incorporated into~$\theta$ using the admissible rule \rref{Ro-Refine}; see \cref{sec:admissible-rules}.

The rules for nondeterministic guarded cases employ a similar technique. The possible values of the guards~$e_i$ are described by the corresponding postconditions~$\theta_i$, after which each case is considered under the
precondition~$\theta_i[\semtt/b]$. Again, $\theta_i[\semtt/b]$ need not preclude $\theta_i[\semff/b]$, nor any of the other preconditions $\theta_j[\semtt/b]$ with $j \neq i$.
%
One notable point in the total variant \rref{Rw-Tot-Case} is that it is not obtained from \rref{Rw-Part-Case} by changing $\prt$ to $\tot$.
In \rref{Rw-Tot-Case}, each guard is required to be equipped with one partial correctness triple and one total correctness triple:
\[\rotrip[\prt] {\Gamma, \Delta} {\phi} {e_i} {b \of \dB \such \theta_i}
\quad\text{and}\quad
\rotrip[\tot] {\Gamma, \Delta} {\phi_i} {e_i} {b \of \dB \such b = \semtt}
\]
The triple for capturing the values of $e_i$ is allowed to stay \emph{partial}, 
though the guarded case requires to be total. 
The reason is that the total correctness of the overall case expression does not require each of the guards to be total and
we only need to make sure that there is a guard that can be chosen.
The extra implication in the premise 
$\phi \lthen \phi_1 \lor \cdots \lor \phi_n$ ensures that.
The precondition $\phi_i$ of the total correctness triple 
stands for the condition in which $e_i$ terminates and evaluates to and only to $\semtt$. 

Lastly, we explain the rules for the loop $\cwhile e \cdo c \cend$. A superficial reading of \rref{Rw-Part-While} looks suspect because~$c$ seemingly need not satisfy an invariant. One has to read both premises together:  the invariant~$\phi$ starts as the precondition for~$b$ passes to $c$ via an intermediate statement~$\theta$, and emerges as the postcondition for~$c$.

The total rule \rref{Rw-Tot-While} establishes an invariant the same way, and ensures termination by means of a well-founded relation, as follows.
%
The formula~$\psi$ involves the read-only variables~$\Gamma$, the mutable variables~$\vec{y} : \Delta$, and a read-only copy $\vec{z} : \Delta'$ of the mutable variables.  Given a read-only state $\gamma \in \sem{\Gamma}$, an \defemph{infinite $\psi$-chain} is a sequence $s : \IN \to \sem{\Delta}$ of states such that $\psi[s_i/\vec{z}, \gamma/\vec{x}, s_{i+1}/\vec{y}]$ holds for all $i \in \IN$. Say that~$\psi$ is \defemph{well-founded} under condition $\phi$ when, for every $\gamma \in \sem{\Gamma}$ and $\delta \in \sem{\Delta}$ satisfying $\phi[\gamma/\vec{x}, \delta/\vec{y}]$, there is no infinite $\psi$-chain starting from $\delta$.
%
In the rule \rref{Rw-Tot-While},
%
the precondition $\vec{z} = \vec{y}$ and the postcondition~$\psi$ together express the fact that the state just before the execution of~$c$ and the state just after forms a link in a $\psi$-chain. The loop must therefore terminate, or else it would yield an infinite $\psi$-chain.

\begin{theorem}\label{t:sound}
  The proof rules given in \cref{fig:rules-general,fig:rules-arithmetic,fig:rules-operational} are sound:
  a derivable correctness triple is valid.
\end{theorem}

\begin{proof}
  The proof proceeds by induction on the derivation of a triple, which amounts to checking for each rule that its conclusion is valid if the premises are.
  %
  Establishing the soundness of general rules in \cref{fig:rules-general} is just an easy exercise in logic.

  To check the soundness of the arithmetical rules in \cref{fig:rules-arithmetic} one has to unwind the semantics of the premises and conclusions, and compare them to the denotational semantics of expressions involved.
  %
  We spell out just the total version of the limit rule \rref{Ro-Lim}.
  %
  Consider any $\gamma \in \sem{\Gamma}$ and $k \in \IZ$ such that $\phi(\gamma, k)$. The first premise guarantees that $\sem{e}(\gamma, k) \subseteq \IR$, as well as $\theta(\gamma, k, u)$ for all $u \in \sem{e}(\gamma, k)$.
  The second premise further ensures the existence of $t \in \IR$ such that $\psi(\gamma, k, t)$ and $\all{u \in \sem{e} (\gamma, k)} |u - t| \leq 2^{-k}$. Thus, the conditions of the first clause in the semantics of $\clim{x} e$ in \cref{figure:ro-denotations} are met, hence $\sem{\clim{x} e}\,\gamma = \{t\}$, yielding the desired conclusion.
\end{proof}

\subsection{Admissible rules}
\label{sec:admissible-rules}

Some useful admissible rules are collected in \cref{fig:rules-admissible}. There are verified by structural induction on the derivation of the premises, see \cref{sec:formalization}.
%
The rules \rref{Ro-Tot-Part} and \rref{Rw-Tot-Part} allow us to pass from total to partial correctness.
%
The rules \rref{Ro-Refine} and \rref{Rw-Refine} are useful for eliding an statement~$\theta$ that does not play any role in a proof, where the latter rule requires on the side that~$\theta$ not mention the read-write variables.

\begin{figure}[htbp]
  \centering
  \begin{mdframed}
    \footnotesize
    \centering

\begin{mathpar}
  \inferenceRule{Ro-Tot-Part}{
    \rotrip[\tot] {\Gamma} {\phi} {e} {y \of \tau \such \psi}
  }{
    \rotrip[\prt] {\Gamma} {\phi} {e} {y \of \tau \such \psi}
  }

  \inferenceRule{Rw-Tot-Part}{
    \rwtrip[\tot] {\Gamma; \Delta} {\phi} {c} {z \of \tau \such \psi}
  }{
    \rwtrip[\prt] {\Gamma; \Delta} {\phi} {c} {z \of \tau \such \psi}
  }

  \\

  \inferenceRule{Ro-Refine}{
    \rotrip {\Gamma} {\phi} {e} {y \of \tau \such \psi}
  }{
    \rotrip {\Gamma} {\phi \land \theta} {e} {y \of \tau \such \psi \land \theta}
  }

  \inferenceRule{Rw-Refine}{
    \rwtrip {\Gamma;\Delta} {\phi} {e} {y \of \tau \such \psi} \\
    \fv{\theta} \cap \dom{\Delta} = \emptyset
  }{
    \rwtrip {\Gamma;\Delta} {\phi \land \theta} {e} {y \of \tau \such \psi \land \theta}}
\end{mathpar}

  \end{mdframed}
  \caption{Admissible rules}
  \label{fig:rules-admissible}
\end{figure}

\subsection{Specification of functions and function calls}
\label{sec:spec-first-order}

Given a function definition
%
\begin{equation*}
  \cfunction f {x_1 \of \tau_1, \ldots, x_n \of \tau_n} {\sigma} {e}
\end{equation*}
%
the rule governing function calls to~$f$ is
%
\begin{equation*}
  \inferenceRule{Ro-Call}{
    \rotrip {\Gamma} {\phi} {e_i} {x_i \of \tau_i \such \theta_i} \quad\text{for $i = 1, \ldots, n$}
    \\\\
    \rotrip {\Gamma, x_1 \of \tau_1, \ldots, x_n \of \tau_n}
            {\theta_1 \land \cdots \land \theta_n}
            {e}
            {y \of \sigma \such \psi}
  }{
    \rotrip {\Gamma} {\phi} {f(e_1, \ldots, e_n)} {y \of \sigma \such \psi}
  }
\end{equation*}
%
It is typically used indirectly, as follows. Upon defining~$f$, we prove an assertion
%
\begin{equation}
  \label{eq:f-assertion}
  \rotrip
  {x_1 \of \tau_1, \ldots, x_n \of \tau_n}
  {\phi_f}
  {e}
  {y \of \sigma \such \psi_f}
\end{equation}
%
that is deemed to usefully characterize the body of the definition~$e$. Then, to establish
%
\begin{equation}
  \label{eq:call-target}
  \rotrip {\Gamma} {\phi} {f(e_1, \ldots, e_n)} {y \of \sigma \such \psi}
\end{equation}
%
we prove for each $i = 1, \ldots, n$ an assertion
%
$\rotrip {\Gamma} {\phi} {e_i} {x_i \of \tau_i \such \theta_i}$
%
such that the implication $\theta_1 \land \cdots \land \theta_n \lthen \phi_f$ holds, we verify $\psi_f \lthen \psi$, and appeal to \rref{Ro-Call} and~\eqref{eq:f-assertion} to conclude~\eqref{eq:call-target}.
%
The method is demonstrated in the next section.


%%% Local Variables:
%%% mode: latex
%%% TeX-master: "clerical"
%%% End:
