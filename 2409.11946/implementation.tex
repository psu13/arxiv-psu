\section{Implementation}
\label{sec:implementation}

We turn attention to how Clerical, or a language based on it, might be implemented in practice.
A sensible implementation ought to work in such a way that an error-free well-typed \defemph{program} (a closed expression) $e$ of type~$\tau$ evaluates to one of its denotations, i.e., if $\sem{\emptyctx \rotypes e : \tau}() \neq \emptyset$ then $e$ evaluates to any $v \in \sem{\emptyctx \rotypes e : \tau}()$. More precisely, $e$ ought to evaluate to a \emph{representation} of~$v$, with the proviso that~$\bot$ corresponds to a non-terminating evaluation.

An implementation is certainly possible in principle. To see this, we can follow the approach of~\cite{brausse2016semantics} to show that Clerical programs are computable in the sense of Type Two Effectivity~\cite{w00}. We first endow each datatype~$\tau$ with a standard Baire space representation. In particular, reals are encoded by rapidly converging sequences of (encoded) rationals. We claim that whenever $\Gamma \rotypes e : \tau$ there is a type two Turing machine~$M$ which takes as input a representation of $\gamma \in \sem{\Gamma}$ and
%
\begin{itemize}
\item either $\sem{\Gamma \rotypes e : \tau} \, \gamma = \emptyset$, or
\item $M$ diverges and $\bot \in \sem{\Gamma \rotypes e : \tau} \, \gamma$, or
\item $M$ outputs a representation of an element of $\sem{\Gamma \rotypes e : \tau} \, \gamma$.
\end{itemize}
%
The construction of~$M$ proceeds recursively on the structure of~$e$. The most interesting is the $\ccase$ statement, which is implemented by combining the machines that compute the guards into a single machine that interleaves the guard computations and proceeds with the case whose guard first evaluates to~$\ctrue$.

More interesting and relevant is the question of an actual implementation of Clerical.
We implemented a proof-of-concept interpreter in OCaml, available at~\cite{clerical_ocaml}.
The connoisseurs will recognize the strong influence of the iRRAM package for exact-real arithmetic~\cite{muller2000irram}, which we gladly acknowledge.
%

We use the GNU MPFR library~\cite{mpfr} to compute with large integers and multiple-precision floating-point numbers.
%
During evaluation, real numbers are approximated by intervals whose endpoints are represented by multiple-precision floating-point numbers, rounded at the current \defemph{working precision}.
As the computation progresses, rounding, interval arithmetic, and limit approximations contribute to making the intervals ever wider.
If the intervals approximating $x$ and $y$ in a comparison $x \rlt y$ overlap, its Boolean value cannot be computed and evaluation is aborted due to \defemph{loss of precision}. The control returns to the top level, where the entire computation is restarted with higher working precision. The computation of a reciprocal $x^{-1}$ behaves analogously in case the interval approximating~$x$ overlaps with~$0$.
%
If the top level needs to output the result of a computation at a given precision, it keeps restarting it with ever higher working precision until the desired output precision is achieved.

The most interesting part of the interpreter is the implementation of guarded case
%
\begin{equation*}
  \ccase e_1 \To c_1 \mid \cdots \mid e_n \To c_n \cend.
\end{equation*}
%
Our semantics demands that one of the branches~$c_j$ must evaluate so long as one of the guards necessarily evaluates to $\ctrue$. Therefore, we cannot evaluate the guards $e_1, \ldots, e_n$ sequentially one after the other, lest the evaluation get stuck in a non-terminating guard.
%
We employed algebraic effects and handlers~\cite{plotkin09:_handl_algeb_effec,bauer15:_progr}, which are supported in OCaml~5, to implement cooperative threads that interleave the computations of the guards. The threads periodically yield control to the main scheduler, which enforces a round-robing evaluation strategy.
%
As soon as one of the threads~$e_i$ evaluates to~$\ctrue$, the other ones are aborted and the computation proceeds with the corresponding case~$c_i$.
%
We are looking forward to future experimentation with parallel execution of guards, which can be implemented using OCaml~5 multi-core features~\cite{sivaramakrishnan22:_retrof_concur}.

%%% Local Variables:
%%% mode: latex
%%% TeX-master: "clerical"
%%% End:
