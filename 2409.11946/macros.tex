% !TEX root = clerical.tex
%% PLEASE COMMENT ON EACH MACRO WHAT IT IS FOR, AND PUT IT IN THE CORRECT GROUP

%%%%%%%%%% GROUP: LaTeX

\newcommand{\psection}[1]{\par\textbf{\textsf{#1}} \newline} % Section in proof

\newcommand{\defemph}[1]{\textbf{\emph{#1}}} % Defined word

%%% Theorem-style environments

% the following environments switch to a slanted font:
\theoremstyle{plain}

\newtheorem{theorem}{Theorem}[section]
\newtheorem{corollary}[theorem]{Corollary}
\newtheorem{lemma}[theorem]{Lemma}
\newtheorem{proposition}[theorem]{Proposition}

% the following environments keep the roman font:
\theoremstyle{definition}

\newtheorem{remark}[theorem]{Remark}
\newtheorem{example}[theorem]{Example}
\newtheorem{definition}[theorem]{Definition}

%%%%%%%%%%%% GROUP: Miscellaneous math

\newcommand{\defeq}{\mathrel{{:}{=}}} % a defined concept
\newcommand{\eqdef}{\mathrel{{=}{:}}} % a defined concept, right-to-left
\newcommand{\defiff}{\mathrel{{:}{\Leftrightarrow}}} % a defined proposition

\newcommand{\dom}[1]{\mathsf{dom}(#1)} % domain of a function
\newcommand{\powerset}[1]{\mathcal{P}(#1)} % powerset

\newcommand{\partiality}{\overline{?}}
\newcommand{\mval}{\mathcal{M}}
\newcommand{\bigcase}{\mathsf{Case}}

\newcommand{\CB}{\Sigma^\IN} % cantor or baire space

\newcommand{\abs}[1]{\lvert #1 \rvert}

% Quantifiers
\newcommand{\all}[1]{\forall #1 \,.\,}
\newcommand{\some}[1]{\exists #1 \,.\,}

% Shorter implication
\newcommand{\lthen}{\Rightarrow}

% Shoter iff
\newcommand{\liff}{\Leftrightarrow}

% Domain subtraction
\newcommand{\dsubt}{\mathrel{%
     \ooalign{$\triangleleft$\cr\hidewidth\scalebox{.65}[1]{$-$}\hidewidth\cr}%
     }}

% Number sets
\newcommand{\IB}{\mathbb{B}}
\newcommand{\IN}{\mathbb{N}}
\newcommand{\IQ}{\mathbb{Q}}
\newcommand{\IR}{\mathbb{R}}
\newcommand{\IZ}{\mathbb{Z}}

\newcommand{\numeral}[1]{\overline{#1}} % integer numeral (syntax)

%%%%%%%%%%%% GROUP: Syntax and rules

\newcommand{\bnfis}{\mathrel{\;{:}{:}{=}\ }}
\newcommand{\bnfor}{\mathrel{\;\big|\ \ }}

% Types
\newcommand{\dR}{\mathsf{R}}
\newcommand{\dZ}{\mathsf{Z}}
\newcommand{\dB}{\mathsf{B}}
\newcommand{\dU}{\mathsf{U}}

% Expressions
\newcommand{\ccoerce}[1]{\iota(#1)} % integer-to-real coercion
\newcommand{\ccase}{\mathtt{case}\;}
\newcommand{\cif}{\mathtt{if}\;}
\newcommand{\cthen}{\;\mathtt{then}\;}
\newcommand{\celse}{\;\mathtt{else}\;}
\newcommand{\cend}{\;\mathtt{end}}
\newcommand{\cwhile}{\mathtt{while}\;}
\newcommand{\climx}{\mathtt{lim}} % the limit symbol
\newcommand{\clim}[2]{\climx \; #1 \,.\, #2}
\newcommand{\cdo}{\;\mathtt{do}\;}
\newcommand{\cin}{\;\mathtt{in}\;}
\newcommand{\cskip}{\mathtt{skip}}
\newcommand{\cnewvar}{\mathtt{var}\;}
\newcommand{\cletx}{\mathrel{{:}{=}}} % assignment symbol
\newcommand{\clet}[2]{#1 \cletx #2} % assignment statement
\newcommand{\cvar}[2]{\mathtt{var}\; #1 \cletx #2 \cin} % variable declaration
\newcommand{\cvarx}{\mathtt{var}\;} % variable declaration
\newcommand{\cinv}[1]{{#1}^{-1}} % inverse
\newcommand{\To}{\Rightarrow}
\newcommand{\cfunction}[4]{\mathtt{let}\; #1(#2) : #3 \cletx #4} % function definition

\newcommand{\ctrue}{\mathsf{true}}
\newcommand{\cfalse}{\mathsf{false}}

% typeset for abbreviations \ example programs


% operators for real numbers
\newcommand{\rgt}{\mathbin{\boldsymbol{>}}}
\newcommand{\rlt}{\mathbin{\boldsymbol{<}}}
\newcommand{\rplus}{\mathbin{\boldsymbol{+}}}
\newcommand{\rminus}{\mathop{\boldsymbol{-}}}
\newcommand{\rmult}{\mathbin{\boldsymbol{\times}}}

% operators for integers
\newcommand{\igt}{\mathbin{>}}
\newcommand{\ilt}{\mathbin{<}}
\newcommand{\ieq}{\mathbin{=}}
\newcommand{\iplus}{\mathbin{+}}
\newcommand{\iminus}{\mathbin{-}}
\newcommand{\imult}{\mathbin{\times}}

\newcommand{\op}{\mathbin{\ast}} % any operator
\newcommand{\iop}{\mathbin{\varoast}} % integer operator
\newcommand{\rop}{\mathbin{\boxast}} % real operator

%%% Abbreviations
\newcommand{\cand}{\mathrel{\bar{\land}}}
\newcommand{\cdisj}{\mathrel{\bar{\lor}}}
\newcommand{\cneg}[1]{\neg #1}

%% Typing judgements & rules
\newcommand{\emptyctx}{{\cdot}} % empty context
\newcommand{\typingrule}[2]{\infer{#1}{#2}}
\newcommand{\of}{{:}} % typing of a variable in a context
\newcommand{\rwtypes}{\Vdash} % read-write typing
\newcommand{\rotypes}{\vdash} % read-only typing

\newcommand{\isEnv}{\;\mathsf{env}} % is a well-formed top-level environment

%%%%%%% GROUP: named inference rules

% the style for rule names
\newcommand{\rulename}[1]{\textnormal{\textsc{#1}}}

% use \rref{...} to refer to a rule in text
\newcommand{\rref}[1]{\hyperlink{rule:#1}{\rulename{#1}}}

% the color of rule names
\definecolor{rulenameColor}{rgb}{0.5,0.5,0.5}

% named inference rule
\newcommand{\inferenceRule}[3]{\inferrule*[lab={\hypertarget{rule:#1}{\rulename{\footnotesize\color{rulenameColor}#1}}}]{#2}{#3}}


%%%%%%%%%%%% GROUP: Domain theory and denotational semantics

\newcommand{\semtt}{\mathsf{tt}} % semantic truth
\newcommand{\semff}{\mathsf{ff}} % semantic falsehood
\newcommand{\semuu}{\star} % semantic unit

\newcommand{\inclZ}[1]{\iota_{\IZ}(#1)} % inclusion of integers into reals

\newcommand{\sem}[1]{\llbracket #1 \rrbracket} % Semantic bracket

\newcommand{\lifterr}[1]{{#1}_\bot^\err} % The lifting-with-error
\newcommand{\liftnoerr}[1]{{#1}_\bot} % Lifting-without-error

\newcommand{\Pstar}{\mathcal{P}_{\!\star}}

\newcommand{\dsup}[1]{{\textstyle\bigsqcup_{#1}}}

\newcommand{\PP}[1]{\Pstar(#1)} % The action of \PStar
\newcommand{\PPleq}{\sqsubseteq} % the order on \PStar
%\newcommand{\PPjoin}[1]{{\textstyle\biguplus_{#1}}} % the join operator on \PStar
\newcommand{\pure}[1]{\{#1\}} % the embedding of S into \PP{S}
\newcommand{\lift}[1]{#1^\dagger} % The lifting of an operation
\newcommand{\liftop}[1]{\mathbin{{#1}^\dagger}} % The lifting of an operation
\newcommand{\liftinv}[1]{#1^{-1^\dagger}} % The lifting of inverse
\newcommand{\PPlet}[2]{\mathsf{let}\; #1 \shortleftarrow #2 \;\mathsf{in}\;} % bind
\newcommand{\PPletx}[1]{\mathsf{let}\; #1}
\newcommand{\PPinx}[1]{\shortleftarrow #1 \;\mathsf{in}\;} % bind
\newcommand{\PPtuple}[1]{\langle #1 \rangle_\star} % monadic tupling

\newcommand{\err}{\mathsf{e}} % The error element
%\newcommand{\PPerr}{\mathfrak{e}} % The error element of \PP{S}
%\newcommand{\PPbot}{\mathfrak{b}} % The bottom element of \PP{S}
\newcommand{\PPerr}{\emptyset} % The error element of \PP{S}
\newcommand{\PPbot}{\{\bot\}} % The bottom element of \PP{S}

\newcommand{\rectify}[1]{#1_{\star}} % rectify a set ot an element of \PP{S}

\newcommand{\semcond}[1]{\mathsf{cond}_{#1}} % the semantic conditional
\newcommand{\semguard}[1]{\mathsf{guard}_{#1}} % semantic guard
\newcommand{\semlim}[1]{\mathsf{lim}\,#1} % the semantic limit


\newcommand{\cF}{\mathcal{F}}
\newcommand{\pto}{\rightharpoonup} % partial map


\newcommand{\llangle}{\langle\!\langle}
\newcommand{\rrangle}{\rangle\!\rangle}

\newcommand{\psem}[1]{\llangle #1 \rrangle} % another semantic bracket


%%%%%%%%%%%% GROUP: Assertion language

\newcommand{\prt}{\mathsf{p}}
\newcommand{\tot}{\mathsf{t}}
\newcommand{\such}{\mid}
\newcommand{\fv}[1]{\mathsf{fv}(#1)} % free variables

\newcommand{\rotrip}[5][\star]{#2 \rotypes \{ #3\} \, #4 \, \{ #5 \}^{#1}}
\newcommand{\rwtrip}[5][\star]{#2 \rwtypes \{ #3\} \, #4 \, \{ #5 \}^{#1}}

\newcommand{\subst}[2]{#1[#2]}
\newcommand{\claimtext}[1]{{\color{colAssert}$\text{#1}$}}
\newcommand{\byrule}[1]{{\color{colAssert}$\text{ by }\rref{#1}$}}
\newcommand{\claim}[1]{{\color{colAssert}$\{\,#1\,\}$}}
\newcommand{\claims}[1]{{\color{colAssert}$\{\,#1\,\}^{\star}$}}
\newcommand{\claimp}[1]{{\color{colAssert}$\{\,#1\,\}^{\prt}$}}
\newcommand{\claimt}[1]{{\color{colAssert}$\{\,#1\,\}^{\tot}$}}

\newcommand{\passert}[1]{\{ #1\}}
\newcommand{\tassert}[1]{\downarrow\hspace{-.3em}\passert{#1}}
\newcommand{\aassert}[1]{?\hspace{-.1em}\passert{#1}}


% without context
\newcommand{\assert}[2][\star]{{\passert{#2}}^{ #1 }}


% partial correctness read-write
\newcommand{\arwptriple}[5]{#1\triangleright#2 \rwtypes \passert{#3}\;\;#4\;\;\passert{#5} }
\newcommand{\rwptriple}[4]{\arwptriple{\Xi}{#1}{#2}{#3}{#4}}

% total correctness read-write
\newcommand{\arwttriple}[5]{#1\triangleright#2 \rwtypes \passert{#3}\;\;#4\;\;\downarrow\hspace{-.3em}\passert{#5} }
\newcommand{\rwttriple}[4]{\arwttriple{\Xi}{#1}{#2}{#3}{#4}}

% partial correctness read-only
\newcommand{\aroptriple}[5]{#1\triangleright#2 \rotypes \passert{#3}\;\;#4\;\;\passert{#5} }
\newcommand{\roptriple}[4]{\aroptriple{\Xi}{#1}{#2}{#3}{#4}}

% total correctness read-only
\newcommand{\arottriple}[5]{#1\triangleright#2 \rotypes \passert{#3}\;\;#4\;\;\downarrow\hspace{-.3em}\passert{#5} }
\newcommand{\rottriple}[4]{\arottriple{\Xi}{#1}{#2}{#3}{#4}}

% partial/total correctness read-write
\newcommand{\arwtriple}[5]{#1\triangleright#2 \rwtypes \passert{#3}\;\;#4\;\;?\passert{#5} }
\newcommand{\rwtriple}[4]{\arwtriple{\Xi}{#1}{#2}{#3}{#4}}

% partial/total correctness read-only
\newcommand{\arotriple}[5]{#1\triangleright#2 \rotypes \passert{#3}\;\;#4\;\;?\passert{#5} }
\newcommand{\rotriple}[4]{\arotriple{\Xi}{#1}{#2}{#3}{#4}}

% Proof rule
\newcommand{\proofrule}[3][]{\infer[#1]{#3}{#2}}

% Semantic validity of partial correctness assertion
\newcommand{\pTriple}[4]{\Xi\triangleright#1 \vDash \passert{#2}\;\;#3\;\;\passert{#4} }

% Semantic validity of total correctness assertion
\newcommand{\tTriple}[4]{\Xi\triangleright#1 \vDash \passert{#2}\;\;#3\;\;\downarrow\hspace{-.3em}\passert{#4} }

\newcommand{\attriple}[5]{#1 \triangleright#2 \vdash \passert{#3}\;\;#4\;\;\downarrow\hspace{-.3em}\passert{#5} }
\newcommand{\aptriple}[5]{#1 \triangleright#2 \vdash \passert{#3}\;\;#4\;\;\passert{#5} }

\newcommand{\swp}{\mathbf{wp}}
\newcommand{\swlp}{\mathbf{wlp}}
\newcommand{\swlpe}{\exists\mathbf{wlp}}

\newcommand{\cwp}{\mathsf{wp}}
\newcommand{\cwlp}{\mathsf{wlp}}
\newcommand{\cwlpe}{\exists\mathsf{wlp}}

% the set of well-formed formulas
\newcommand{\wf}{\mathsf{wf}} 
% True and False in logical language
\newcommand{\sTrue}{\textsf{\textup{True}}}
\newcommand{\sFalse}{\textsf{\textup{False}}}

% well-formedness under context
\newcommand{\wellformed}{\Vdash}
\newcommand{\emptystate}{\epsilon}

% make equations left and center aligned
\makeatletter
\newcommand{\mathleft}{\@fleqntrue\@mathmargin0pt}
\newcommand{\mathcenter}{\@fleqnfalse}
\makeatother



%%% Local Variables:
%%% mode: latex
%%% TeX-master: "clerical"
%%% End:

