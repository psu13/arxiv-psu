\section{Future work}
\label{sec:future-work}

In conclusion we discuss several directions for further work.

One is to explore how Clerical could be extended to include higher-order functions and general recursion.
%
Incorporating higher-order function without recursion should be straightforward from a language-design viewpoint. However, generalising the denotational semantics to cover higher-order functions may not be so straightforward, since the powerdomain $\PP{S}$ from 
Section~\ref{sec:denotation} will have to be generalised to allow $S$ to range over denotations of arbitrary types. 
%
With regards to general recursion, the non-monotonicity phenomena associated with the guarded case construct are likely to make it very challenging to define denotational semantics; see~\cite{LEVY2007221} for a discussion of related issues.

Second, in this paper we have not presented any formal operational semantics for Clerical.
Having one would provide an alternative and direct
account of the computability of the language, as well as a
framework within which implementation-relevant information, such as
the scope for parallelism in the execution strategy, could be studied in a mathematical setting. Also, a formally specified operational semantics
could guide implementations of Clerical and Clerical-like languages, and help estaliblish their correctness.

Third, we could further experiment with our implementation, which is good enough to evaluate the~$\pi$ program but cannot compete with the mature libraries for exact-real numbers.
%
To speed it up, we should at least implement parallel execution of threads, which is supported by the latest OCaml version.
%
A more substantive improvements would explore better evaluation strategies for nondeterminism, and compilation to a more efficient low-level language.

Fourth, there is significant room for improvement in the Coq formalization. By implementing better automation and tactics for proving correctness assertions, we would obtain a workable environment for formal verification of exact real computation, supported by the formidable machinery of Coq.