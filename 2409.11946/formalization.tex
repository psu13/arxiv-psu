% !TEX root = clerical.tex
\section{Formalization}
\label{sec:formalization}

We formalized Clerical in the Coq proof assistant, and proved important properties presented in this paper. 
The formalization can also be used to formally type-check and verify Clerical expressions with respect to given specifications.
This has been done for the examples from \cref{sec:example}, their correctness proofs included.
%
Here we outline the organization of the formalization, which is freely available at~\cite{clerical_coq}:
%
\begin{description}[style=nextline,font=\normalfont]
\item[\coqpath{BaseAxioms}]
  %
  Coq's type of propositions \coqcode{Prop} is intuitionistic, but we work in a classical
  metatheory. Thus we assume excluded middle for \coqcode{Prop}, dependent choice, as well as function extensionality and propositional extensionality.

\item[\coqpath{Powerdomain}]
  This part formalizes the domain-theoretic aspects of the powerdomain, as well as auxiliary constructions
  required in the formalization of the denotational semantics.

\item[\coqpath{Syntax}, \coqpath{Typing}, \coqpath{TypingProperties}]
  These parts formalize the syntax and typing rules of Clerical from \cref{sec:syntax}.
  We use de Bruijn indices to implement variables. We also prove several properties of the
  type system, such as uniqueness of typing (an expression has at most one type).

\item[\coqpath{Semantics}, \coqpath{SemanticProperties}]
  The formalization of the denotational semantics relies on the formalization of the powerdomain and uses the real numbers from Coq's standard library as the denotation of the type of reals.
  %
  The denotation of a term proceeds by induction of its typing derivation. We prove that the denotation does not
  depend on the derivation, but only on the term.

\item[\coqpath{Specification}, \coqpath{ReasoningRules}, \coqpath{ReasoningSoundness}, \coqpath{ReasoningAdmissible}]
  These modules formalize Hoare triples, their reasoning rules, and show them to be sound with respect
  to the denotational semantics. We also show that the rules given in \cref{sec:admissible-rules} are admissible.

\item[\coqpath{examples}, \coqpath{Mathematics}, \coqpath{Arith}]
  We formalized the examples from \cref{sec:example}, where we assumed several familiar facts about~$\pi$ in \coqpath{Mathematics}, for example that if $3 < x < 4$ and $\sin x = 0$ then $x = \pi$.

\item[\coqpath{Arith}]
  %
  In the formalized syntax of Clerical, every expression must be well-typed, and consequently equipped with
  a formal typing derivation. Constructing these is straightforward, but creating them by hand is time-consuming.
  We therefore provided automation for constructing well-typed arithmetical expressions involving variables,
  constants, arithmetical operations, comparisons, and coercions.
  %
  We additionally automated proofs showing that such expressions satisfy the expected partial and total correctness specifications.
  %
  The automation is used extensively in the formalized examples.
\end{description}








 