
\section{Discussion}
\label{sec:discuss}

Recently, there has been a resurgence of interest in exploring the quantum foundations of the universe, accompanied by a notable acceleration in the advancement of practical quantum technologies---a phenomenon sometimes known as the ``second quantum revolution''~\cite{dowling2003quantum}. However, a sound understanding of quantum theory has yet to penetrate mainstream awareness, education, and the professional training pipeline.

To make quantum knowledge accessible to a broader audience, we must  present complex scientific phenomena in a way that is more readily suitable for learning. This requirement necessitates a deviation from traditional approaches, which rely too heavily on complex symbolic reasoning and highly technical language. Additionally, our presentation should be practically applicable, as opposed to approaches which rely on inaccurate metaphors to convey a simplified, intuitive understanding of a complex subject matter. The experiment detailed in this paper proposes QP \cite{ContPhys} as a viable candidate for such a presentation, in the form of a visual paradigm already adopted in the wild for lecturing \cite{CKbook, coecke2023basic}, reasoning \cite{van2020zx}, and research \cite{de2020fast, de2019techniques, kissinger2019reducing, de2020zx, kissinger2022phase, khesinGraphicalQuantumCliffordencoder2023}.

This experiment is unique in three aspects.
Firstly, it offers state-of-the-art educational material, tailored to empower participants to perform quantum calculations using the diagrammatic formalism. Rather than merely advocating a `shut-up-and-calculate' approach, the curriculum presents a new conceptual foundation for learning physics and science, both quantum and beyond, with primary focus on a holistic understanding of the relationship between events and processes.
This unique perspective not only simplifies the formalism, but also supercharges the users' reasoning skills about quantum systems and their otherwise puzzling behaviours.

Secondly, the experiment aims to assess whether QP can democratise access to a crucial segment of university-level education by extending its reach to high school students. For almost a decade, QP has been taught to a small selection of students at elite universities, most notably the University of Oxford. As part of our experiment, a more diverse group of students---in terms of academic background, gender, socioeconomic status, and prior mathematical knowledge---will reap the benefit of QP for the first time.

Thirdly, the experiment explores whether learning quantum theory falls within the zone of proximal development of high school students, i.e. whether adequate guidance and support can enhance their understanding of such a subject matter.
A successful experiment would settle the question in the positive, while an unsuccessful outcome leaves the possibility that the experimental setup did not provide sufficient guidance to fully bridge the gap between the high-school syllabus and the more specialised requirements of quantum disciplines.

More evidence will be needed to test against diverse control groups, and to explore the effects of QP on larger samples of the high-school student population.
The facilitating effect of QP on the acquisition of quantum technology skills will also require further evaluation, investigating how it assists novices in accurately organising their mental representations while minimising the cognitive resources needed to achieve proficiency in real-world tasks.

The insights gained from this and future work will undoubtedly pave the way for the development of more inclusive and innovative educational endeavors. These efforts will play a significant role in bridging the gap between high school and university level learning, particularly for those aspiring to pursue a STEM career. It will help them establish crucial conceptual and abstract frameworks earlier, significantly contributing to their academic growth and understanding. 

Finally, the findings have the potential to inspire further research and advancement in quantum science pedagogy, contributing to the ongoing evolution of educational methodologies and practices specific to quantum technology and beyond. 



\section*{Appendecies}

\section{Content and learning outcomes}

Below is the detailed week-by-week breakdown of course content and learning outcomes:

\begin{itemize}
\item In week 1, the episode ``Quantum in Pictures: Wires and Boxes'' introduces diagrams composed of wires and boxes to describe processes and perform mathematical operations. The episode also introduces the concepts of space and time in diagrams and the physics they represent.
\item In week 2, the episode ``Quantum Teleportation'' introduces the concept of the quantum lottery and the idea that wires and boxes can be used to carry out advanced mathematical reasoning. It then discusses quantum teleportation, a fundamental building block of quantum communication and quantum computing.
\item In week 3, the episode ``A World of Spiders'' introduces ``spiders'', a special kind of box with basic rules to work with them. Spiders---discussed in Section II on Quantum Picturalism---provide the fundamental building blocks for quantum computing.
\item In week 4, the episode ``Quantum Computing'' introduces quantum computing using spiders: logical operations (such as copying and adding), quantum gates (such as rotations, CNOT gates and Hadamard gates) and measurements.
It covers the use of diagrammatic substitution rules for circuit simplification and other relevant calculations.
% Classical computers use bits, and quantum computers use qubits. The episode covers the representation of bits and qubits using spiders "decorated" with phases. The episode also covers logical operations on those (qu)bits, like copying, adding, and the CNOT gate using spiders. For example, the CNOT gate is composed of a green spider and a red spider connected to each other---discussed in Section II on Quantum Picturalism. Participants also learn how the Hadamard gates (colour-changing boxes), introduced in the previous episode, affect a circuit of spiders. Learning how applying rules on spiders helps realise computations (\emph{i.e.} given quantum states at the beginning of a circuit, what are the quantum states at the end of the circuit, and the output of the computation) and simplify circuits, valuable to experimentalists with limited resources.
\item In week 5, the episode ``Quantum Teleportation with Spiders'' explores quantum teleportation further. It then introduces measurement-based quantum computing (MBQC), a fault-tolerant flavour of quantum computing which uses ideas from quantum teleportation to shift computation from gates to measurements. MBQC was also one of the motivating examples in developing the ZX-calculus.
\item In week 6, the episode ``Keeping Einstein Happy'' introduces notions of relativistic causality within the diagrammatic framework in the form of Sure-boxes and Maybe-boxes. 
\item In week 7, the episode ``Quantum vs Ordinary Particles'' introduces the concept of double wires to distinguish the quantum world from the classical world. Mathematically, doubling the wire gives new kinds of numbers (real probabilities, as opposed to complex amplitudes).
\item In week 8, the episode ``Everything Just in Pictures'' introduces the square-popping rule and the necessary tools to describe all quantum processes using pictures alone.
\item In week 9, participants receive a take-home exam, which they are requested to work on until the end of week 10.
\end{itemize}



\section{Marking Criteria}

The marking criteria are set out in Table \ref{tab:marking_criteria} (p.\pageref{tab:marking_criteria}).
They are designed to focus on specific assessment areas, ensuring alignment with the learning objectives and expected outcomes.
They are used in conjunction with discipline-specific criteria, and they should be viewed as guidance on the overall standards expected at different grade bands, aligning with the taught postgraduate generic marking criteria used at the University of Oxford.
The marking criteria have been reviewed and utilise a 0-100\% grading structure in line with the current university regulations.

\renewcommand{\arraystretch}{1.1}
\makeatletter
\renewcommand{\fnum@figure}{Table \thefigure}
\makeatother

\begin{figure*}[!t]
    \centering
\tabulinesep=1mm
\scriptsize
\begin{tabu}{|X[c]|X[c]|X[c]|X[1.5c]|}
\hline
\multicolumn4{|c|}{\textbf{Distinction $>$70}} \\
\hline
\textit{Understanding} & \textit{Use of knowledge} & \textit{Structure} & \textit{Grade bands} \\
\hline
Advanced, in-depth, authoritative, full understanding of key ideas. Originality of the solutions, legitimacy of chain of reasoning in the answers provided.
&
Complex work and key problems solved. Correct application of concepts and techniques (e.g. the appropriate use of diagrammatic rewriting rules), the ability to use proper terminology (e.g. ``square-popping'', ``leg-chopping'')
&
Coherent and compelling work.
Logical and concise presentation. The solution drawn/written in a clear and unambiguous way (e.g. the proper use of notations, the difference between ``quantum'' and ``ordinary'' diagrams).
&
\textbf{(90-100)} insightful work displaying in-depth knowledge. Outstanding work, independent thought, highest standards of problem solving

\textbf{(80-89)} insightful work displaying in-depth knowledge. Good quality of work, independent thought 

\textbf{(70-79)} thoughtful work displaying in-depth knowledge, good standards of problem solving 
\\
\hline
\multicolumn4{|c|}{\textbf{Merit 60-69}} \\
\hline
\textit{Understanding} & \textit{Use of knowledge} & \textit{Structure} & \textit{Grade bands} \\
\hline
In-depth understanding of key ideas with evidence of some originality
&
Key problems solved. Correct application of most concepts, techniques, and correct use of terminology
&
Coherent work, logically presented. Clear solutions
&
\textbf{(65-69)} thoughtful work displaying good knowledge and accuracy. Evidence for the ability to solve problems

\textbf{(60-64)} work displays good knowledge, some evidence for problem solving
\\
\hline
\multicolumn4{|c|}{\textbf{Pass 50-59}} \\
\hline
\textit{Understanding} & \textit{Use of knowledge} & \textit{Structure} & \textit{Grade bands} \\
\hline
Understanding of some key ideas with evidence of ability to reflect critically 
&
Some key problems solved. Correct application of some concepts, techniques, and terminology
&
Competent work in places but lacks coherence 
&
\textbf{(55-59)} work displays some understanding in most areas, but standard of work is variable

\textbf{(50-54)} work displays knowledge and understanding of some areas, but some key problems are not solved
\\
\hline
\multicolumn4{|c|}{\textbf{Fail 40-49}} \\
\hline
\textit{Understanding} & \textit{Use of knowledge} & \textit{Structure} & \textit{Grade bands} \\
\hline
Superficial understanding of some key ideas, lack of focus 
&
Key problems are not solved/understood, gaps in application of concepts, techniques, and terminology
&
Weaknesses in structure and/or coherence 
&
\textbf{(40-49)} work displays patchy knowledge and understanding, most key problems are not solved
\\
\hline
\multicolumn4{|c|}{\textbf{Fail 0-39}} \\
\hline
\textit{Understanding} & \textit{Use of knowledge} & \textit{Structure} & \textit{Grade bands} \\
\hline
Lack of understanding
&
Key problems misunderstood/unanswered, limited/incorrect application of concepts, techniques and terminology 
&
Work is confused and incoherent 
&
\textbf{(33-39)} incomplete answers with some superficial knowledge


\textbf{(20-32)} some attempt to write something relevant but many flaws

\textbf{(0-19)} serious errors, irrelevant answers
\\
\hline
\end{tabu}
    \caption{Marking Criteria}
    \label{tab:marking_criteria}
\end{figure*}

