
\section{Non-Leptonic Two-body Decays and Factorization}
\setcounter{equation}{0}
\subsection{Preliminaries}
We will begin the applications of the formalism developed in the
previous seven sections by discussing two-body non-leptonic decays.
Although our discussion will concentrate on two-body B-decays, it
can be generalized in a straightforward manner to D-decays.

I should state from the beginning that it is not my intention
to give here a review of two-body decays and present detailed
comparision with the available data. My intention is rather to 
reanalyze critically the concepts of the {\it factorization}
hypothesis and in particular of the {\it generalized factorization}
hypothesis discussed in the literature. As we will see soon, it
is an excellent battle field for the formalism developed in
previous sections.

Now comes a rather unfortunate move. In this and only this section
we have to modify slightly our notation by interchanging the 
indices 1 and 2 in current-current operators. This we have to do
 in order to conform to the notation used in the literature
on two--body non--leptonic decays. Thus we introduce the operators
\be\label{T1}
O_1=Q_2, \quad\quad O_2=Q_1,
\ee
and their respective coefficients
\be\label{T2}
\bar C_1(\mu)=C_2(\mu), \quad \quad  \bar C_2(\mu)=C_1(\mu).
\ee
Correspondingly
\be\label{T3}
O_\pm=\frac{O_1\pm O_2}{2}, \quad\quad 
z_\pm= \bar C_1(\mu)\pm \bar C_2(\mu),
\ee
and
\begin{equation}\label{10N}
\bar C_1(\mu)=\frac{z_+(\mu)+z_-(\mu)}{2},
\qquad\qquad
\bar C_2(\mu)=\frac{z_+(\mu)-z_-(\mu)}{2},
\end{equation}
with all formulae (\ref{B9})--(\ref{B15}) unchanged.
We have introduced a ``bar", omitted in the literature, in order
to avoid possible confusion. 

This section is based on \cite{AJB94a} and the recent collaboration
with Luca Silvestrini \cite{BUSI}. We do not cover here more dynamical
approaches to non-leptonic decays like QCD sum reles. 
A very nice review of the 
applications of QCD sum rules to non-leptonic decays has been presented
this year by Khodjamirian and R\"uckl \cite{KR98} and is strongly 
recommended.
\subsection{Factorization}
In the factorization approach to non-leptonic meson decays
\cite{FEYNMAN,STECHF} one can
distinguish three classes of decays for which the amplitudes have the
following general structure \cite{BAUER,NEUBERT}:
\begin{equation}\label{1}
A_{\rm I}=\frac{G_F}{\sqrt{2}} V_{CKM}a_1(\mu)\langle O_1\rangle_F 
\qquad {\rm (Class~I)},
\end{equation}
\begin{equation}\label{2}
A_{\rm II}=\frac{G_F}{\sqrt{2}} V_{CKM}a_2(\mu)\langle O_2\rangle_F 
\qquad {\rm (Class~II)},
\end{equation}
\begin{equation}\label{3N}
A_{\rm III}=
\frac{G_F}{\sqrt{2}} V_{CKM}[a_1(\mu)+x a_2(\mu)]\langle O_1\rangle_F
 \qquad {\rm (Class~III)}.
\end{equation}
Here $V_{CKM}$ denotes symbolically the CKM factor characteristic for a
given decay. 
$\langle O_i\rangle_F$ are factorized 
hadronic matrix
elements of the operators $O_i$ given as products of matrix elements of
quark currents and $x$ is a non-perturbative factor equal to unity in
the flavour symmetry limit. Finally $a_i(\mu)$ are QCD factors which
are given as follows
\begin{equation}\label{BS23}
a_1(\mu)=\bar C_1(\mu)+\frac{1}{N} \bar C_2(\mu), \qquad
a_2(\mu)=\bar C_2(\mu)+\frac{1}{N} \bar C_1(\mu).
\end{equation}
We will soon give explicit examples and we will rederive these
formulae as limiting cases of the generalized factorization 
hypothesis. First, however, we would like to make a few general
comments on the weak points of this approach.

At first sight the simplicity of this approach is very appealing.
Once the matrix elements $\langle O_i\rangle_F $
have been expressed in terms of various meson
decay constants and generally model dependent formfactors, predictions
for non-leptonic heavy meson decays can be made. 
Moreover relations between non-leptonic and semi-leptonic decays can
be found which allow to test factorization in a model independent
manner. An incomplete list of analyses of this type is given in 
\cite{LNF,NEUBERT} and will be extended below.
 
On the other hand,
it is well known that
non-factorizable contributions must be present in the hadronic matrix
elements of the current--current operators $O_1$ and $O_2$ in order
to cancel the $\mu$ dependence of $\bar C_i(\mu)$ or $a_i(\mu)$ so that
the physical amplitudes do not depend on the arbitrary renormalization
scale $\mu$. 
$\langle O_i\rangle_F$ being products of matrix elements of
conserved currents
are $\mu$--independent and the cancellation of the $\mu$ dependence
in (\ref{1})--(\ref{3N}) does not take place.
Consequently from the point of view of QCD
the factorization approach can be at best correct at a single value
of $\mu$, the so-called factorization scale $\mu_f$. Although the
approach itself does not provide the value of $\mu_f$, the  
proponents
of factorization expect $\mu_f=O(m_b)$ and $\mu_f=O(m_c)$ for
B-decays and D-decays respectively. 

Here we would like to point out that beyond the leading logarithmic
approximation for $\bar C_i(\mu)$ a new complication arises. 
As we have discussed in previous sections,
 at next to leading level in the renormalization
group improved perturbation theory the coefficients $\bar C_i(\mu)$
depend on the renormalization scheme for operators. Again only
the presence of non-factorizable contributions
 in $\langle O_i\rangle$ can
remove this scheme dependence in the physical amplitudes. 
However $\langle O_i\rangle_F$ are renormalization scheme
independent and the factorization approach is of course unable
to tell us whether it works better with an anti-commuting $\gamma_5$
in $D\not=4$ dimensions (NDR scheme) or with another definition 
of $\gamma_5$ such as used in HV or DRED schemes. Moreover there
are other renormalization schemes parametrized by $\kappa_\pm$
in (\ref{B12})--(\ref{B15}).
The renormalization scheme dependence emphasized here is rather
annoying from the factorization point of view as it precludes
a unique phenomenological determination of $\mu_f$ as we will
show explicitly below. 

On the other hand, arguments have been given
\cite{BJORKEN,DUGAN,NEUBERT}, that 
 factorization approach could be
approximately true in the case of two-body decays with high
energy release \cite{BJORKEN}, or in certain kinematic regions
\cite{DUGAN,ISGUR,ITALY}. We will not repeat here these arguments, which
can be found in the original papers.
Needless to say the issue of factorization does not only 
involve the short distance gluon corrections discussed here
but also final state interactions as stressed in particular
in \cite{ITALY}.

It is difficult to imagine that factorization
can hold even approximately in all circumstances. 
In spite of this, it
became fashonable these days to
test this idea, 
to some extent, by using certain set of formfactors to calculate
$ \langle O_i\rangle_F  $ and by making global fits of the 
formulae (\ref{1})--(\ref{3N})
to the data treating
$ a_1 $ and $ a_2 $ as free independent parameters. As an example we
give the result of a recent
analysis of this type for non-leptonic two-body B-decays
\cite{NS97}  
\begin{equation}\label{8N}
a_1\approx 1.08\pm0.04
\qquad
a_2\approx 0.21\pm0.05
\end{equation}
which is compatible with other analyses 
\cite{Cheng,Soares,LNF,GNF,AKL98}.
At the level of accuracy of the existing  experimental data and 
because of strong model
dependence in the relevant formfactors it is not yet possible  
to conclude on the basis of these analyses whether the
factorization approach is a useful approximation in general or not. 
It is
certainly  conceivable that factorization may apply better to some
non-leptonic decays than to others 
\cite{NEUBERT}-\cite{ISGUR}
and using all decays in a global fit may misrepresent the true
situation. 

The fact that $\langle O_i\rangle_F$ are $\mu$-independent but
$a_i(\mu)$ are $\mu$-dependent, which is clearly inconsistent,
inspired a number of authors \cite{Cheng,Soares, NS97,GNF,AKL98} to
generalize the concept of factorization. The presentation given
in the next subsection, done in collaboration with Silvestrini 
\cite{BUSI},
follows closely the generalization due to 
Neubert and Stech
\cite{NS97} which deals exclusively with the operators $O_1$ and
$O_2$. The generalization presented in \cite{Cheng,GNF,AKL98} are
similar in spirit but includes also the penguin contributions. I will
discuss it briefly at the end of this section. In particular
the very recent analysis of Ali, Kramer and L\"u \cite{AKL98} 
is very informative.
\subsection{Generalized Factorization}
In the generalized factorization framework the formulae 
(\ref{1})--(\ref{3N}) are simply replaced by
\begin{equation}\label{1a}
A_{\rm I}=\frac{G_F}{\sqrt{2}} V_{CKM}
a^{\rm eff}_1\langle O_1\rangle_F 
\qquad {\rm (Class~I)},
\end{equation}
\begin{equation}\label{2a}
A_{\rm II}=\frac{G_F}{\sqrt{2}} V_{CKM}
a^{\rm eff}_2\langle O_2\rangle_F 
\qquad {\rm (Class~II)},
\end{equation}
\begin{equation}\label{3a}
A_{\rm III}=
\frac{G_F}{\sqrt{2}} V_{CKM}
[a^{\rm eff}_1+x a^{\rm eff}_2]\langle O_1\rangle_F
 \qquad {\rm (Class~III)},
\end{equation}
where $a^{\rm eff}_i$ are $\mu$-independent and renormalization
scheme independent parameters to be extracted from experimental data.
From phenomenological point of view there is no change here
relative to the standard factorization as only $a_i(\mu)$ have
been replaced by $a^{\rm eff}_i$. On the other hand, as stressed 
in particular in \cite{NS97}, the new formulation should allow in 
principle some insight into the importance of non-factorizable
contributions.  

In this context I should remark that in the recent literature
mainly the $\mu$-dependence of the non-factorizable contributions
has been emphasized. Their scheme dependence has  been only discussed
in \cite{AJB94a}. It is the latter issue which will be important in
the discussion below. Let us then derive the formulae for
$a_i^{\rm eff}$ including NLO corrections. 

In order to describe generalized factorization in explicit terms
let us consider the decay
$\bar B^0\to D^+\pi^-$. Then the
relevant effective Hamiltonian is given by
\begin{equation}\label{BS1}
H_{eff}=\frac{G_F}{\sqrt{2}}V_{cb}V_{ud}^{*}
\lbrack \bar C_1(\mu) O_1+\bar C_2(\mu)O_2 \rbrack~,
\end{equation}
where
\begin{equation}\label{BS2}
O_1=(\bar d_\alpha u_\alpha)_{V-A} (\bar c_\beta b_\beta)_{V-A}
\qquad 
O_2=(\bar d_\alpha u_\beta)_{V-A} (\bar c_\beta b_\alpha)_{V-A}~.
\end{equation}
$\bar C_1(\mu)$ and $\bar C_2(\mu)$ are 
computed at the renormalization scale $\mu=O(m_b)$.
Since all four quark flavours entering the operators in (\ref{BS2})
are different from each other, no penguin operators contribute to
this decay. 

Using Fierz reordering and colour identities one can rewrite the
amplitude for $\bar B^0\to D^+\pi^-$ as
\begin{equation}\label{BS3}
A(\bar B^0\to D^+\pi^-)=\frac{G_F}{\sqrt{2}}V_{cb}V_{ud}^{*}
a^{\rm eff}_1 \langle O_1\rangle_F
\end{equation}
where
\be
\langle O_1\rangle_F=
\langle\pi^-\mid(\bar d u)_{V-A}\mid 0\rangle
\langle D^+\mid (\bar c b)_{V-A}\mid \bar B^0\rangle
\ee
is the factorized matrix element of the operator $O_1$ and
summation over colour indices in each current is understood.

The effective parameter $a^{\rm eff}_1$ is then given by \cite{NS97}
\be\label{BS4}
a^{\rm eff}_1=\left(\bar C_1(\mu)+\frac{1}{N} \bar C_2(\mu)\right)
[1+\varepsilon_1^{(BD,\pi)}(\mu)]
+\bar C_2(\mu)\varepsilon_8^{(BD,\pi)}(\mu).
\ee
$\varepsilon_1^{(BD,\pi)}(\mu)$ and $\varepsilon_8^{(BD,\pi)}(\mu)$
are two hadronic parameters defined by
\be\label{BS5}
\varepsilon_1^{(BD,\pi)}(\mu)\equiv
\frac{\langle \pi^-D^+|(\bar d u)_{V-A}(\bar c b)_{V-A}|\bar B^0\rangle}
{\langle O_1 \rangle_F}-1
\ee
and
\be\label{BS6}
\varepsilon_8^{(BD,\pi)}(\mu)\equiv 2
\frac{\langle \pi^-D^+|(\bar dT^a u)_{V-A}
(\bar cT^a b)_{V-A}|\bar B^0\rangle}
{\langle O_1\rangle_F}
\ee
with $T^a$ denoting the colour matrices in the standard Feynman
rules. $\varepsilon_i(\mu)$ parametrize the non-factorizable 
contributions to
the hadronic matrix elements of operators. In the case of strict
factorization $\varepsilon_i$ vanish and $a_1^{\rm eff}$ reduces
to $a_1(\mu)$.

It should be emphasized that no approximation has been made
in (\ref{BS3}). Since the matrix element $\langle O_1 \rangle_F$
is scale and renormalization scheme independent this must also
be the case for the effective coefficient $a^{\rm eff}_1$.
Indeed the scale and scheme dependences of the coefficients
$\bar C_1(\mu)$ and $\bar C_2(\mu)$ are cancelled by those present in
the hadronic parameters $\varepsilon_i(\mu)$. We will give
explicit formulae for the latter dependences below.

A similar exercise with the
amplitude for $\bar B^0\to D^0\pi^0$ gives
\begin{equation}\label{BS7}
A(\bar B^0\to D^0\pi^0)=\frac{G_F}{\sqrt{2}}V_{cb}V_{ud}^{*}
a^{\rm eff}_2 \langle O_2 \rangle_F,
\end{equation}
where
\be\label{fact2}
\langle O_2 \rangle_F=
\langle D^0\mid(\bar c u)_{V-A}\mid 0\rangle
\langle \pi^0\mid (\bar d b)_{V-A}\mid \bar B^0\rangle
\ee
is the factorized matrix element of the operator $O_2$.

The effective parameter $a^{\rm eff}_2$ is given by \cite{NS97}
\be\label{BS8}
a^{\rm eff}_2=\left(\bar C_2(\mu)+\frac{1}{N} \bar C_1(\mu)\right)
[1+\varepsilon_1^{(B\pi,D)}(\mu)]+
\bar C_1(\mu)\varepsilon_8^{(B\pi,D)}(\mu).
\ee
$\varepsilon_1^{(B\pi,D)}(\mu)$ and $\varepsilon_8^{(B\pi,D)}(\mu)$
are two hadronic parameters defined by
\be\label{BS9}
\varepsilon_1^{(B\pi,D)}(\mu)\equiv
\frac{\langle \pi^0D^0|(\bar c u)_{V-A}(\bar d b)_{V-A}|\bar B^0\rangle}
{\langle O_2 \rangle_F}-1
\ee
and
\be\label{BS10}
\varepsilon_8^{(B\pi,D)}(\mu)\equiv 2
\frac{\langle \pi^0D^0|(\bar c T^a u)_{V-A}
(\bar d T^a b)_{V-A}|\bar B^0\rangle}
{\langle O_2 \rangle_F}~.
\ee
Again the $\mu$ and scheme dependences of $\varepsilon_i$ in
(\ref{BS9}) and (\ref{BS10}) cancel the corresponding
dependences in $\bar C_i(\mu)$ so that the effective coefficient
$a^{\rm eff}_2$ is $\mu$ and scheme independent.
Similarly one can derive the formula (\ref{3a}) by using
$B^-\to D^0 K^-$ or other decay belonging to class III.

Following section 5.1 of \cite{BJLW} and using the experience
accumulated in previous sections it is straightforward to
find the explicit $\mu$ and scheme dependences of the hadronic
parameters $\varepsilon_i(\mu)$. To this end we note
that the $\mu$ dependence of the matrix elements of the
operators $O_\pm$ is given by  
\be\label{BS110}
\langle O_\pm(\mu)\rangle = U_\pm(\mb,\mu) \langle O_\pm(\mb)\rangle~,
\ee
where the evolution function $U_\pm(\mb,\mu)$ 
including NLO QCD corrections is given as in (\ref{B9P}) by 
\begin{equation}\label{BS11}
U_\pm(\mb,\mu)=\left[1+\frac{\alpha_s(\mb)}{4\pi}J_\pm\right]
      \left[\frac{\alpha_s(\mu)}{\alpha_s(\mb)}\right]^{d_\pm}
\left[1-\frac{\alpha_s(\mu)}{4\pi}J_\pm\right]
\end{equation}
with $J_\pm$ and $d_\pm$ in (\ref{B10}). Note the different ordering
of scales in (\ref{BS110}) from the one in the evolution of
Wilson coefficients in (\ref{EVOLC}).

Having these formulae at hand it is straightforward to show
that the $\mu$-dependence of $\varepsilon_1(\mu)$ and
$\varepsilon_8(\mu)$ is governed by the following equations:
\begin{eqnarray}\label{BS20}
1+\varepsilon_1(\mu)&=&
\frac{1}{2}
\left[\left(1+\frac{1}{N}\right)[1+\R1(\mb)]+\E8(\mb)\right]
U_+(\mb,\mu)\\
&+&
\frac{1}{2}
\left[\left(1-\frac{1}{N}\right)[1+\R1(\mb)]-\E8(\mb)\right]
U_-(\mb,\mu),   \nonumber
\end{eqnarray}

\begin{eqnarray}\label{BS21}
\varepsilon_8(\mu)&=&
\frac{1}{2}
\left[\left(1-\frac{1}{N}\right)\E8(\mb)+
\left(1-\frac{1}{N^2}\right)[1+\R1(\mb)]\right]
U_+(\mb,\mu)\\
&+&
\frac{1}{2}
\left[\left(1+\frac{1}{N}\right)\E8(\mb)-
\left(1-\frac{1}{N^2}\right)[1+\R1(\mb)]\right]
U_-(\mb,\mu).
 \nonumber
\end{eqnarray}
It is a very good exercise to derive these formulae and any student
who wants to test her (his) skills in this field should try it.

These formulae reduce to the ones given in \cite{NS97} when $J_\pm$ in
(\ref{BS11}) are set to zero. They give both the $\mu$-dependence
and renormalization scheme dependence of $\varepsilon_i$. The
latter dependence has not been considered in \cite{NS97}.
We will return to these expressions in a moment. First, however,
we would like to formulate the generalized factorization in a
more transparent manner.
\subsection{A Different Formulation}
In order to be able to discuss the relation of our presentation
\cite{BUSI} to the
one of \cite{NS97} we have used until now, as in \cite{NS97}, the
hadronic parameters $\R1(\mu)$ and $\E8(\mu)$ to describe
non-factorizable contributions. 
It appears to us that it is more convenient to work instead
with two other parameters defined simply by \cite{BUSI}
\be\label{BS22}
a^{\rm eff}_1=a_1(\mu)+\xi^{\rm NF}_1(\mu),
\quad\quad
a^{\rm eff}_2=a_2(\mu)+\xi^{\rm NF}_2(\mu),
\ee
where $a_i(\mu)$ are defined in (\ref{BS23}).
Comparison with (\ref{BS4}) and (\ref{BS8}) gives
\be\label{xi1}
\xi^{\rm NF}_1(\mu)=\R1(\mu) a_1(\mu)+\E8(\mu) \bar C_2(\mu),
\ee
\be\label{xi2}
\xi^{\rm NF}_2(\mu)=\bar\R1(\mu) a_2(\mu)+\bar\E8(\mu) \bar C_1(\mu),
\ee
where
\be\label{E18}
\R1(\mu)=\R1^{(BD,\pi)}, \qquad \E8(\mu)=\E8^{(BD,\pi)},
\ee
\be\label{E19}
\bar\R1(\mu)=\R1^{(B\pi,D)}, \qquad \bar\E8(\mu)=\E8^{(B\pi,D)}. 
\ee

In the framework of the
strict factorization hypothesis $\xi^{\rm NF}_i(\mu)$
are set to zero. Their
$\mu$ and scheme dependences can in principle
be found by using the dependences of $\bar C_i(\mu)$ given in
section 7
and of $\varepsilon_i(\mu)$ in (\ref{BS20}) and (\ref{BS21}).
To this end, however, one needs the determination of the
non-perturbative parameters $\varepsilon_i(\mu)$ and
$\bar\varepsilon_i(\mu)$ at a single
value of $\mu$. If, as done in \cite{NS97}, $a_i^{\rm eff}$
are universal parameters, the determination of $\varepsilon_i(\mu)$
and $\bar\varepsilon_i(\mu)$ is only possible if one also
makes the following {\it universality} assumptions:
\be\label{FACTU}
\R1(\mu)=\bar\R1(\mu), \qquad \E8(\mu)=\bar\E8(\mu).
\ee
In \cite{NS97} such an assumption was unnecessary as $\R1(\mu)$
has been set to zero and only $\E8(\mu)$ has been extracted
from the data.

With the assumptions in (\ref{FACTU}), $\R1(\mu)$ and $\E8(\mu)$
can indeed be found once the effective parameters $a_i^{\rm eff}$
have been determined experimentally. Using (\ref{BS4}) and
(\ref{BS8}) together with (\ref{FACTU}) we find
\be\label{E1MU}
\R1(\mu)=\frac{\bar C_1(\mu) a_1^{\rm eff}-\bar C_2(\mu)a_2^{\rm eff}}
          {\bar C^2_1(\mu)-\bar C^2_2(\mu)}-1~,
\ee
\be\label{E2MU}
\E8(\mu)=\frac{a_2^{\rm eff}}{\bar C_1(\mu)}-
\left(\frac{\bar C_2(\mu)}{\bar C_1(\mu)}+\frac{1}{N}\right) 
[1+\R1(\mu)]~.
\ee
On the other hand  $\xi_i^{\rm NF}(\mu)$ can be determined without
the universality assumption (\ref{FACTU}) from two decays simply
as follows
\be\label{BS24}
 \xi^{\rm NF}_1(\mu)=a^{\rm eff}_1-a_1(\mu)~,
\quad\quad
 \xi^{\rm NF}_2(\mu)=a^{\rm eff}_2-a_2(\mu)~.
\ee

Formulae in (\ref{BS24}) make it clear that the strict factorization
in which $\xi_i^{\rm NF}(\mu)$ vanish can be at best correct at a single
value of $\mu$, the so-called factorization scale $\mu_f$. In the
first studies of factorization $\mu_f=\mb$ 
has been assumed. It has been concluded that such a choice is
not in accord with the data.
The idea of the generalized factorization as formulated in
\cite{Cheng,Soares,NS97}
is to allow $\mu_f$ to be different from $\mb$ and to extract
first non-factorizable parameters $\varepsilon_i(\mb)$ from the
data. Subsequently factorization scale $\mu_f$ can be found
by requiring these parameters to vanish.

In the numerical analysis of this procedure done in \cite{NS97}
one additional assumption has been made. Using large $N$ arguments
it has been argued that $\R1(\mu)$ can be set to zero while
$\E8(\mu)$ can be sizable. The resulting expressions
for $a_i^{\rm eff}$ are then
\be\label{NSF}
a_1^{\rm eff}=\bar C_1(\mb), 
\qquad a_2^{\rm eff}=a_2(\mb)+\bar C_1(\mb)\E8(\mb)~,
\ee
where additional small terms have been dropped in order to obtain
the formula for $a_1^{\rm eff}$. Using subsequently
the extracted value $a_2^{\rm eff}=0.21\pm0.05$ together with
the  coefficients $\bar C_i(\mb)$ from \cite{WEISZ} one finds
$\E8(\mb)=0.12\pm0.05$ \cite{NS97}. Next assuming $\E8(\mu_f)=0$ one
can find the factorization scale $\mu_f$
by inverting the formula \cite{NS97}
\be\label{UF}
\E8(\mb)=-\frac{4\as(\mb)}{3\pi}\ln\frac{\mb}{\mu_f}~,
\ee
which follows from (\ref{BS21}) with $\E8(\mu_f)=0$ and $\R1(\mb)=0.$
Thus
\be\label{UF1}
 \mu_f=\mb \exp\left[\frac{3\pi\E8(\mb)}{4\as(\mb)}\right]~.
\ee
Taking $\mb=4.8~\gev$ and $\as(m_b)=0.21$ 
(corresponding to $\as(\mz)=0.118$) we find using $\E8(\mb)=0.12\pm0.05$
a rather large factorization scale $\mu_f=(15.9+11.3-6.6)~\gev$,
by roughly a factor of 3-4 higher than $\mb$. This implies that
non-factorizable contributions in hadronic matrix elements at scales
close to $\mb$ are sizable. This is also signalled by the 
value of $\E8(\mb)\approx0.12$ which is larger than the factorizable
contribution $a_2(\mb)=0.09$ to the effective parameter
$a_2^{\rm eff}=0.21\pm0.05$.

We would like to emphasize that such an interpretation of the
analysis of Neubert and Stech \cite{NS97} would be 
misleading. As stressed in \cite{AJB94a} the coefficient $a_2(\mu)$
is very strongly dependent on the renormalization scheme.
Consequently for a given value of $a_2^{\rm eff}$ also
$\xi_2^{NF}(\mb)$ and $\E8(\mb)$ are strongly scheme dependent.
This shows, that a meaningful analysis of the
$\mu$-dependences in non-leptonic decays, such as the search for the
factorization scale $\mu_f$, cannot be be made without simultaneously
considering the scheme dependence. This is evident if one recalls that
any variation of $\mu_f$ in the leading logarithm is equivalent to
a shift in constant non-logarithmic terms. The latter represent
NLO contributions in the renormalization group improved
perturbation theory and must be included for a meaningful extraction
of $\mu_f$ or any other scale like $\Lms$. However, once  the NLO
contributions are taken into account, the renormalization scheme
dependence enters the analysis and consequently the factorization
scale $\mu_f$ at which the non-factorizable hadronic parameters
$\xi_i^{NF}(\mu_f)$ or $\varepsilon_i(\mu_f)$ vanish is renormalization
scheme dependent. Formula (\ref{21}) exhibits all these statements
very clearly.
  
From this discussion it becomes clear that for any chosen scale
$\mu_f=\ord(\mb)$, it is always possible to find a renormalization
scheme for which
\be\label{xifac}
\xi_1^{NF}(\mu_f)=\xi_2^{NF}(\mu_f)=0~.
\ee
Indeed as seen in (\ref{BS24}) $\xi_i^{NF}(\mu)$ depend through
$a_i(\mu)$ on $\kappa_\pm$ (see section 7)
which characterize a given renormalization scheme. The choice
of $\kappa_\pm$
corresponds to a particular finite renormalization of the operators
$O_\pm$ in addition to the renormalization in the NDR scheme. It
is then straightforward to find the values of $\kappa_\pm$ which
assure that for a chosen scale $\mu_f$ the conditions in (\ref{xifac})
are satisfied. We find \cite{BUSI}
\be\label{kappa+}
\kappa_+=3
\left[\frac{3}{4}\frac{a_1^{\rm eff}+a_2^{\rm eff}}{W_+(\mu_f)}-1\right] 
\frac{4\pi}{\as(\mu_f)}-3 (J_+)_{\rm NDR}~,
\ee
\be\label{kappa-}
\kappa_-=\frac{3}{2}
\left[\frac{3}{2}\frac{a_1^{\rm eff}-a_2^{\rm eff}}{W_-(\mu_f)}-1\right] 
\frac{4\pi}{\as(\mu_f)}-\frac{3}{2} (J_-)_{\rm NDR}~,
\ee
where
\be\label{W+-}
W_\pm(\mu_f)=\left[\frac{\alpha_s(M_W)}{\alpha_s(\mu_f)}\right]^{d_\pm}
\left[1+\frac{\alpha_s(M_W)}{4\pi}(B_\pm-J_\pm)\right]
\ee
with $(J_\pm)_{\rm NDR}$ being the values of $J_\pm$ in the NDR scheme.
$W_\pm(\mu_f)$ are clearly renormalization scheme independent as
$B_\pm-J_\pm$ and $d_\pm$ are scheme independent.
\subsection{Numerical Analysis}
Before presenting the numerical analysis of the formulae derived in
the preceding subsections, it is important  to clarify the difference
between the Wilson coefficients in (\ref{10N}) used by
us and the ones employed in \cite{NS97}. In \cite{NS97} scheme
independent coefficients $\tilde z_\pm(\mu)$ of \cite{WEISZ}  
instead of $z_\pm(\mu)$ have been used. These are obtained by
multiplying $z_\pm(\mu)$ by $(1-B_\pm \alpha_s(\mu)/4\pi)$ 
so that
\begin{equation}\label{T11}
\tilde z_\pm(\mu)=
      \left[\frac{\alpha_s(M_W)}{\alpha_s(\mu)}\right]^{d_\pm}
\left[1+\frac{\alpha_s(M_W)-\alpha_s(\mu)}{4\pi}(B_\pm-J_\pm)\right].
\end{equation}
These coefficients are clearly not the coefficients of the operators
$O_\pm$. 
In order to be consistent, the matrix elements $\langle O_\pm \rangle$
should then be replaced by
\be\label{TOPM}
\langle \tilde O_\pm \rangle=
(1+B_\pm \alpha_s(\mu)/4\pi) \langle O_\pm \rangle.
\ee
This, however, has not been done in \cite{NS97}. This explains, to a large
extent, why our results for $\varepsilon_8(\mb)$  differ considerably from 
the ones quoted in \cite{NS97}.
We strongly advice the practitioners of
non-leptonic decays not to use the scheme independent coefficients
of \cite{WEISZ} in phenomenological applications. These coefficients
have been introduced to test the compatibility of different 
renormalization schemes and can only be used for phenomenology
together with $\langle \tilde O_\pm \rangle$. This would however
unnecessarily complicate the analysis and it is therefore advisable
to  work with the true coefficients $\bar C_i(\mu)$ of the operators $O_i$ 
as given in (\ref{10N}).

In \cite{NS97} the values of $a_i^{\rm eff}$ given in (\ref{8N}) 
have been
extracted from existing data on two-body B--decays.
In order to illustrate various points made until now,
we take the central values of $a_i^{\rm eff}$ in (\ref{8N})
and calculate $\varepsilon_i(\mu)$
and $\xi_i^{NF}(\mu)$ as  functions of $\mu$ in the range
$2.5~\gev \le \mu \le 10~\gev$ for the NDR and HV schemes. The
results are shown in fig.~\ref{SILV1}  and fig.~\ref{SILV2}.
We observe that $\varepsilon_1(\mu)$ and 
$\xi^{\rm NF}_1(\mu)$ are only weakly $\mu$ and scheme
dependent in accordance with the findings in \cite{AJB94a},
where these dependences have been studied for $a_i(\mu)$
defined in (\ref{BS23}). The strong $\mu$ and scheme dependences
of $a_2(\mu)$ found there translate into similar strong dependences
of $\varepsilon_8(\mu)$ and $\xi_2^{\rm NF}(\mu)$.

\begin{figure}   % produce figure here
    \begin{center}
% GNUPLOT: LaTeX picture with Postscript
\setlength{\unitlength}{0.1bp}
\begin{picture}(3600,2160)(0,0)
\special{psfile=figeps.ps llx=0 lly=0 urx=720 ury=504 rwi=7200}
\put(1428,870){\makebox(0,0)[l]{$\varepsilon_8^{\scriptscriptstyle NDR}$}}
\put(2580,1104){\makebox(0,0)[l]{$\varepsilon_8^{\scriptscriptstyle HV}$}}
\put(3156,1611){\makebox(0,0)[l]{$\varepsilon_1^{\scriptscriptstyle NDR}$}}
\put(3156,1338){\makebox(0,0)[l]{$\varepsilon_1^{\scriptscriptstyle HV}$}}
\put(2100,180){\makebox(0,0){$\mu$}}
\put(120,1260){%
\special{ps: gsave currentpoint currentpoint translate
270 rotate neg exch neg exch translate}%
\makebox(0,0)[b]{\shortstack{$\varepsilon_i$}}%
\special{ps: currentpoint grestore moveto}%
}
\put(3540,360){\makebox(0,0){10}}
\put(3156,360){\makebox(0,0){9}}
\put(2772,360){\makebox(0,0){8}}
\put(2388,360){\makebox(0,0){7}}
\put(2004,360){\makebox(0,0){6}}
\put(1620,360){\makebox(0,0){5}}
\put(1236,360){\makebox(0,0){4}}
\put(852,360){\makebox(0,0){3}}
\put(600,2040){\makebox(0,0)[r]{0.14}}
\put(600,1884){\makebox(0,0)[r]{0.12}}
\put(600,1728){\makebox(0,0)[r]{0.1}}
\put(600,1572){\makebox(0,0)[r]{0.08}}
\put(600,1416){\makebox(0,0)[r]{0.06}}
\put(600,1260){\makebox(0,0)[r]{0.04}}
\put(600,1104){\makebox(0,0)[r]{0.02}}
\put(600,948){\makebox(0,0)[r]{0}}
\put(600,792){\makebox(0,0)[r]{-0.02}}
\put(600,636){\makebox(0,0)[r]{-0.04}}
\put(600,480){\makebox(0,0)[r]{-0.06}}
\end{picture}





    \end{center}
    \caption[]{$\varepsilon_{1,8}(\mu)$ in the NDR and HV schemes.}
    \label{SILV1}
\end{figure}

\begin{figure}   % produce figure here
    \begin{center}
% GNUPLOT: LaTeX picture with Postscript
\setlength{\unitlength}{0.1bp}
\begin{picture}(3600,2160)(0,0)
\special{psfile=figcsivsmu.ps llx=0 lly=0 urx=720 ury=504 rwi=7200}
\put(1620,792){\makebox(0,0)[l]{$\xi_2^{\scriptscriptstyle NDR}$}}
\put(1044,1728){\makebox(0,0)[l]{$\xi_2^{\scriptscriptstyle HV}$}}
\put(3156,1494){\makebox(0,0)[l]{$\xi_1^{\scriptscriptstyle NDR}$}}
\put(3156,1221){\makebox(0,0)[l]{$\xi_1^{\scriptscriptstyle HV}$}}
\put(2100,180){\makebox(0,0){$\mu$}}
\put(120,1260){%
\special{ps: gsave currentpoint currentpoint translate
270 rotate neg exch neg exch translate}%
\makebox(0,0)[b]{\shortstack{$\xi^{\rm NF}_i$}}%
\special{ps: currentpoint grestore moveto}%
}
\put(3540,360){\makebox(0,0){10}}
\put(3156,360){\makebox(0,0){9}}
\put(2772,360){\makebox(0,0){8}}
\put(2388,360){\makebox(0,0){7}}
\put(2004,360){\makebox(0,0){6}}
\put(1620,360){\makebox(0,0){5}}
\put(1236,360){\makebox(0,0){4}}
\put(852,360){\makebox(0,0){3}}
\put(600,2040){\makebox(0,0)[r]{0.16}}
\put(600,1884){\makebox(0,0)[r]{0.14}}
\put(600,1728){\makebox(0,0)[r]{0.12}}
\put(600,1572){\makebox(0,0)[r]{0.1}}
\put(600,1416){\makebox(0,0)[r]{0.08}}
\put(600,1260){\makebox(0,0)[r]{0.06}}
\put(600,1104){\makebox(0,0)[r]{0.04}}
\put(600,948){\makebox(0,0)[r]{0.02}}
\put(600,792){\makebox(0,0)[r]{0}}
\put(600,636){\makebox(0,0)[r]{-0.02}}
\put(600,480){\makebox(0,0)[r]{-0.04}}
\end{picture}




    \end{center}
    \caption[]{$\xi^{\rm NF}_{1,2}(\mu)$ in the NDR and HV schemes.}
    \label{SILV2}
\end{figure}

We make the following observations:
\bi
\item
$\varepsilon_1(\mu)$ and $\xi_1^{\rm NF}(\mu)$ are non-zero in the
full range of $\mu$ considered.
\item
$\varepsilon_8(\mu)$ and $\xi_2^{\rm NF}(\mu)$ vary strongly with
$\mu$ and vanish in the NDR scheme for $\mu=5.5~\gev$ and 
$\mu=6.3~\gev$
respectively. The corresponding values in the HV scheme are
$\mu=7.5~\gev$ and $\mu=8.6~\gev$.
\item
There is no value of $\mu=\mu_f$ in the full range considered for
which $\varepsilon_1(\mu)$ and $\varepsilon_8(\mu)$ or equivalently
$\xi_1^{\rm NF}(\mu)$ and $\xi_2^{\rm NF}(\mu)$ simultaneously
vanish. We also observe contrary to expectations in \cite{NS97}
that $\varepsilon_1(\mu)$ is not necessarily smaller than
$\varepsilon_8(\mu)$. In fact the large $N$ arguments presented
in \cite{NS97} that $\varepsilon_1(\mu)=\ord(1/N^2)$ and
$\varepsilon_8(\mu)=\ord(1/N)$, imply strictly speaking only
that the $\mu$-dependence of $\varepsilon_8(\mu)$
is much stronger than that of $\varepsilon_1(\mu)$,
which we indeed see in figs.~\ref{SILV1}  and \ref{SILV2}.
 The hierarchy of their actual
values is a dynamical question. Even if the large $N$-counting-rules
$\varepsilon_1(\mu)=\ord(1/N^2)$ and
$\varepsilon_8(\mu)=\ord(1/N)$ are true independently of the
factorization hypothesis \cite{EW,BGR}, it follows from our analysis
that once the generalized factorization hypothesis is made, the
extracted values of $\varepsilon_i$ violate for some range of $\mu$
the large-N rule $\varepsilon_1\ll\varepsilon_8$. 
\ei

We can next investigate for which renormalization scheme characterized
by $\kappa_\pm$ the factorization is exact at $\mu_f=\mb=4.8~\gev$.
We call this choice the ``factorization scheme" (FS).
Using the central values in (\ref{8N}) and $\Lms^{(5)}=225\mev$
we find by means of
(\ref{kappa+}) and (\ref{kappa-})
\be\label{KPKM}
\kappa_+=13.5~, \qquad  \kappa_-=3.9~~~~~~~({\rm FS}).
\ee
These values deviate considerably from the NDR values $\kappa_\pm=0$
and the HV values $\kappa_\pm=\mp 4 $. Yet one can verify that for these
values $J_+=6.13 $ and $J_-=1.17$ and consequently in this scheme the NLO
corrections at $\mu=\mb$ remain perturbative.
In table \ref{tabf} we give the values of 
$\xi_i^{\rm NF}(\mu)$ for the NDR,HV and FS schemes. 

The discussion of this subsection casts some doubts on the
usefulness of the formulation in \cite{NS97} with respect to the study of
non-factorizable contributions to non-leptonic decays.

\begin{table}[htb]
\caption[]{$\xi^{\rm NF}_{1,2}(\mu)$ as functions of $\mu$
for different schemes and $\Lms^{(5)}=225\mev$.}
\label{tabf}
\begin{center}
\begin{tabular}{|c|c|c|c||c|c|c|}
\hline
& \multicolumn{3}{c||}{$\xi^{\rm NF}_1(\mu)$} &
  \multicolumn{3}{c| }{$\xi^{\rm NF}_2(\mu)$} \\
\hline
$\mu [{\rm GeV}]$ &NDR & HV & FS & NDR & HV & FS  \\
\hline
\hline
2.5 & 0.046 & 0.035 & --0.033 & 0.102 & 0.144 & 0.075 \\
\hline
5.0 & 0.065 & 0.059 &   0.001 & 0.022 & 0.055 &--0.004 \\
\hline
7.5 & 0.071 & 0.067 & 0.014 & --0.016 & 0.013 &--0.041 \\
\hline
10.0 & 0.074 & 0.071 & 0.021 &--0.039 & -0.013 &--0.064 \\
\hline
\end{tabular}
\end{center}
\end{table}

\subsection{Generalized Factorization and $N^{\rm eff}$}
The generalized factorization  presented in \cite{Cheng,GNF,AKL98} is
similar in spirit but includes more dynamics than the formulation
in \cite{NS97}. Unfortunately, as we will demonstrate below, also this 
approach has its weak points. Let us then briefly describe the basic
idea.

As pointed sometime ago in \cite{BJLW1,rome2} and recently
discussed in \cite{Cheng,GNF,AKL98},
it is always possible
to calculate the  scale and scheme dependence of the hadronic matrix 
elements in perturbation theory by simply calculating the matrix elements
of the relevant operators between the quark states. 
Combining these scheme and scale dependent contributions with the
Wilson coefficients $C_i(\mu)$ one obtaines the effective coefficients
$C_i^{\rm eff}$ which are free from these dependences. If one neglects
in addition final state interactions and other possible non-factorizable
contributions the decay amplitudes can be generally written as follows
\begin{equation}\label{ALI}
A=\langle H_{eff}\rangle =\frac{G_F}{\sqrt{2}} V_{CKM}
\lbrack C_1^{\rm eff}\langle O_1\rangle^{\rm tree} +C_2^{\rm eff}
\langle O_2\rangle^{\rm tree}  \rbrack~,
\end{equation}
where $\langle O_i\rangle^{\rm tree}$ denote tree level matrix elements.
The proposal in \cite{Cheng,GNF,AKL98} is to use (\ref{ALI}) and to apply 
the idea of the factorization to the tree level matrix elements.
In this approach then the effective parameters $a_{1,2}^{\rm eff}$
are given by 
\begin{equation}\label{BS23F}
a_1^{\rm eff}=C^{\rm eff}_1+\frac{1}{N^{\rm eff}} C^{\rm eff}_2 \qquad
a_2^{\rm eff}=C^{\rm eff}_2+\frac{1}{N^{\rm eff}} C^{\rm eff}_1
\end{equation}
with analogous expressions for $a_{i}^{\rm eff}$ ($i=3-10$) parametrizing
penguin contributions.
Here
$N^{\rm eff}$  is treated as a phenomenological parameter which
models those non-factorizable contributions to the hadronic matrix elements
which have not been included in $C_i^{\rm eff}$.
In particular it has been suggested in \cite{Cheng,GNF,AKL98} that
the values for $N^{\rm eff}$ extracted from the data on two-body
non-leptonic decays should teach us about the pattern of 
non-factorizable contributions.

In particular when calculating the effective coefficients $C_i^{\rm eff}$, 
the authors of  \cite{GNF,AKL98}  have included a
subset of contributions to the perturbative matrix elements, which is
sufficient to cancel the scale and scheme dependence of the Wilson
coefficients.
Unfortunately the results of such calculations are generally gauge
dependent and suffer from the dependence on the infrared regulator
and generally on the assumptions about the external momenta. We
have discussed this already in detail in section 6 but it is instructive
to discuss this briefly once more in the context of the analyses 
in \cite{Cheng,GNF,AKL98}.

The Green function of the renormalized operator $O$,
for a given choice of the ultraviolet regularization (NDR or HV for example), 
a choice of the external momenta $p$ and of the gauge parameter $\lambda$, 
is given by 
\begin{equation}
  \label{eq:matel}
  \Gamma_O^\lambda (p) = 1 + \frac{\alpha_s}{4 \pi} 
\left(-\frac{\gamma^{(0)}}{2}
  \ln(\frac{-p^2}{\mu^2}) + \hat{r} \right),
\end{equation}
with
\begin{equation}
\hat{r}= \hat r^{NDR,HV} + \lambda \hat r^\lambda.
\label{eq:rral}
\end{equation}
The matrices $\hat r^{NDR,HV}$ depend on the choice of the external momenta and
on the ultraviolet regularization, while $\hat r^\lambda$ is 
regularization- and
gauge-independent, but depends on the external momenta. 
It is clearly possible to define a renormalization scheme in which, 
for given external momenta and gauge parameter, $\Gamma_O^\lambda (p)
= 1$, or in other words $\langle O \rangle_{p,\lambda}=\langle O
\rangle^{\rm tree}$ (this corresponds to the RI scheme discussed in
\cite{rome2}). However, the definition of the renormalized 
operators will now depend on the choice of the gauge and of the
external momenta. If one were able, for example by means of lattice
QCD, to compute the matrix element of the operator using the same
renormalization prescription, the dependences on the gauge and on the
external momenta would cancel between the Wilson coefficient and the
matrix element. If, on the contrary, the matrix elements are estimated
using factorization, no trace is kept of the renormalization
prescription and the final result is gauge and infrared dependent. 

In  \cite{GNF,AKL98} scale- and scheme-independent effective
Wilson coefficients $C_i^{\rm eff}$ 
have been obtained by adding to $C_i(\mu)$ the
contributions coming from vertex-type quark matrix elements, denoted
by $\hat r_V$ and $\hat\gamma_V$. In particular
\begin{eqnarray}
  \label{eq:cali}
  C_1^{\rm eff}&=& 
C_1(\mu) + \frac{\alpha_s}{4 \pi}\left( r_V^T + \gamma_V^T
  \log \frac{m_b}{\mu}\right)_{1j} C_j(\mu),\nonumber \\
  C_2^{\rm eff}&=& 
C_2(\mu)  + \frac{\alpha_s}{4 \pi}\left( r_V^T + \gamma_V^T
  \log \frac{m_b}{\mu}\right)_{2j} C_j(\mu)
\end{eqnarray}
where the index $j$ runs through all contributing operators, also
penguin operators considered in \cite{Cheng,GNF,AKL98}.

 It is evident from the above discussion that $\hat r_V$  
depends not only on the
external momenta, but also on the gauge chosen.   
For example, in \cite{GNF,AKL98} the following result for $\hat r_V$ is
quoted:
\begin{equation}
  \label{eq:rvali}
  \hat r_V=\left(
    \begin{array}{cccccc}
      \frac{7}{3} & - 7 & 0 & 0 & 0 & 0\\
      - 7 & \frac{7}{3} & 0 & 0 & 0 & 0\\ 
      0 & 0 & \frac{7}{3} & - 7 & 0 & 0\\ 
      0 & 0 & - 7 & \frac{7}{3} & 0 & 0\\ 
      0 & 0 & 0 & 0 & - \frac{1}{3} & 1\\ 
      0 & 0 & 0 & 0 & - 3 & \frac{35}{3} 
    \end{array}
    \right).
\end{equation}
This result is valid in the Landau gauge ($\lambda=0$); 
in an arbitrary gauge, with the same choice of external momenta used
to obtain (\ref{eq:rvali}) one would get
\begin{equation}
  \label{eq:rvtot}
  \hat r_V = \hat r_V (\lambda=0) + \lambda r_V^\lambda,
\end{equation}
with $\hat r_V (\lambda=0)$ given in (\ref{eq:rvali}) and
\begin{equation}
  \label{eq:rvluca}
  r_V^\lambda=\left(
    \begin{array}{cccccc}
      - \frac{5}{6} & - \frac{3}{2} & 0 & 0 & 0 & 0\\
      - \frac{3}{2} & - \frac{5}{6} & 0 & 0 & 0 & 0\\
      0 & 0 & - \frac{5}{6} & - \frac{3}{2} & 0 & 0\\ 
      0 & 0 & - \frac{3}{2} & - \frac{5}{6} & 0 & 0\\ 
      0 & 0 & 0 & 0 & - \frac{11}{6} & \frac{3}{2}\\
      0 & 0 & 0 & 0 & 0 & \frac{8}{3}
    \end{array}
    \right).
\end{equation}
The expressions for the full 
$10 \times 10$ $\hat r$ matrices in the
NDR and HV schemes and in the Feynman and Landau gauges are given in
\cite{rome2}, for a different choice of the external momenta.
The results for the Landau gauge are given in \cite{BJLW1}, where
also penguin diagrams have been included.

Equation (\ref{eq:rvtot}) shows that the definition of the effective
coefficients advocated in \cite{Cheng,GNF,AKL98}
is gauge-dependent. In addition, it also depends on the choice of the
external momenta.
This implies that the effective number of colors extracted in
\cite{Cheng,GNF,AKL98} is also gauge-dependent, and therefore it
cannot have any physical meaning. 
This finding casts some doubts on the
usefulness of the formulation in \cite{Cheng,GNF,AKL98} with respect
to the study of non-factorizable contributions to non-leptonic decays.

The gauge dependences and infrared dependences discussed here 
appear in any calculation of matrix elements of operators
between quark states necessary in the process of matching of the
full theory onto an effective theory as we have seen in section 6.
 Another example can be
found in \cite{BJW90} where the full gauge dependence of the quark
matrix element of the operator $(\bar s d)_{V-A}(\bar s d)_{V-A}$
has been calculated. However, in the process of matching such
unphysical dependences in the effective theory are cancelled by
the corresponding contributions in the full theory so that the
Wilson coefficients are free of such dependences. Similarly
in the case of inclusive decays of heavy quarks, where the spectator
model can be used, they are cancelled by gluon
bremsstrahlung. In exclusive hadron decays there is no meaningful way to
include such effects in a perturbative framework and one is left
with the gauge and infrared dependences in question.

\subsection{Summary}
In this section we have critically analyzed the hypothesis of
the generalized factorization. While the parametrization of the data
in terms of a set of effective parameters discussed in
\cite{NS97,Cheng,GNF,AKL98},
 may appear to be useful,
we do not think that this approach offers convincing means to
analyze the physics of non-factorizable contributions to
non-leptonic decays. In particular:
\begin{itemize}
\item
The renormalization scheme dependence of the non-factorizable
contributions to hadronic matrix elements precludes the
determination of the factorization scale $\mu_f$.
\item
Consequently for any chosen value of $\mu=\ord(m_b)$ 
it is possible to find a renormalization
scheme for which the non-perturbative parameters $\varepsilon_{1,8}$
used in \cite{NS97} to characterize the size of non-factorizable
contributions vanish. The same applies to 
$\xi^{\rm NF}_{1,2}(\mu)$ introduced in (\ref{BS24}).
\item
We point out that the recent extractions of the effective number
of colours $N^{\rm eff}$ from two-body non-leptonic B-decays,
presented in \cite{Cheng,GNF,AKL98},
while $\mu$ and renormalization scheme independent suffer from
gauge dependences and infrared regulator dependences.
\end{itemize}

Our analysis \cite{BUSI} demonstrates clearly the need for
an approach to non-leptonic decays which goes
beyond the generalized factorization discussed recently in
the literature. Some possibilities are offered by dynamical approaches
like QCD sum rules as recently reviewed in \cite{KR98}. 
However, even a phenomenological approach which does not suffer
from the weak points of factorization discussed here, would
be a step forward. Some ideas in this direction will be presented in
\cite{BUSI2}.

\section{$\eps_K$, $B^0$-$\bar B^0$ Mixing and the Unitarity Triangle}
        \label{sec:epsBBUT}
\setcounter{equation}{0}
%\setcounter{figure}{0}
%\setcounter{table}{0}
\subsection{Preliminaries}
Let us next discuss particle--antiparticle mixing which in the past
 has been of fundamental
importance in testing the Standard Model and often has proven to be an
undefeatable challenge for suggested extensions of this model.
Particle--antiparticle mixing is responsible
for the small mass differences between the mass eigenstates of neutral
mesons. Being an FCNC process it involves heavy quarks in loops
and consequently it is a perfect 
testing ground for heavy flavour physics. Let us just recall that
 from the calculation of the
$K_{\rm L}-K_{\rm S}$ mass difference, Gaillard and Lee \cite{GALE} 
were able to estimate the
value of the charm quark mass before charm discovery. On the
other hand $B_d^0-\bar B_d^0$ mixing \cite{ARGUS} gave the first 
indication of a large top quark mass. 
Finally, particle--antiparticle mixing in the $K^0-\bar K^0$ system
offers within the Standard Model a plausible description of
CP violation in $K_L\to\pi\pi$ discovered in 1964 \cite{CRONIN}. 

In this section we will predominantly discuss  the parameter 
$\varepsilon$ 
describing the {\it indirect} CP violation in the $K$ system and  
the mass differences $\Delta M_{d,s}$  which
describe the size of $B_{d,s}^0-\bar B^0_{d,s}$ mixings. 
In the Standard Model all these phenomena
appear first at the one--loop level and as such they are
sensitive measures of the top quark couplings $V_{ti}(i=d,s,b)$ and 
of the top quark mass. 

We have seen in section 2 that tree level 
decays and the unitarity of the CKM
matrix give us already a good information about $V_{tb}$ and $V_{ts}$:
$V_{tb}\approx 1$ and $\mid V_{ts}\mid\;\approx\;\mid V_{cb}\mid$.
Similarly the value of the top quark mass measured by CDF and D0 (see below)
is known within $\pm 4\%$.
Consequently the
main new information to be gained from the quantities discussed here are 
the values of $|V_{td}|$ and of the phase $\delta=\gamma$ in the CKM matrix.
This will allow us to construct the unitarity triangle which has been 
introduced in subsection 2.3.

\begin{figure}[hbt]
\vspace{0.10in}
\centerline{
\epsfysize=1.5in
%\rotate[r]{
\epsffile{L9.ps}
}%}
\vspace{0.08in}
\caption[]{Box diagrams contributing to $K^0-\bar K^0$ mixing
in the Standard Model.
\label{L:9}}
\end{figure}

First, however, let us briefly recall the formalism of
particle--antiparticle mixing. We will begin with the $K$--system.
Subsequently we will give
some formulae for $B_{d,s}^0-\bar B^0_{d,s}$ mixings, necessary for the
analysis of the unitarity triangle.
A very detailed discussion of $B_{d,s}^0-\bar B^0_{d,s}$ mixings can
be found in section 8 of a review by Robert Fleischer and myself
\cite{BF97} and in his review \cite{RF97}. The following subsection
borrows a lot from \cite{CHAU83} and \cite{BSSII}.
\subsection{Express Review of $K^0-\bar K^0$ Mixing}
$K^0=(\bar s d)$ and $\bar K^0=(s\bar d)$ are flavour eigenstates which 
in the Standard Model
may mix via weak interactions through the box diagrams in fig.
\ref{L:9}.
We will choose the phase conventions so that 
\be
CP|K^0\rangle=-|\bar K^0\rangle, \qquad   CP|\bar K^0\rangle=-|K^0\rangle.
\ee

In the absence of mixing the time evolution of $|K^0(t)\rangle$ is
given by
\be
|K^0(t)\rangle=|K^0(0)\rangle \exp(-iHt)~, 
\qquad H=M-i\frac{\Gamma}{2}~,
\ee
where $M$ is the mass and $\Gamma$ the width of $K^0$. Similar formula
for $\bar K^0$ exists.

On the other hand, in the presence of flavour mixing the time evolution 
of the $K^0-\bar K^0$ system is described by
\be
i\frac{d\psi(t)}{dt}=\hat H \psi(t) \qquad  
\psi(t)=
\left(\begin{array}{c}
|K^0(t)\rangle\\
|\bar K^0(t)\rangle
\end{array}\right)
\ee
where
\be
\hat H=\hat M-i\frac{\hat\Gamma}{2}
= \left(\begin{array}{cc} 
M_{11}-i\frac{\Gamma_{11}}{2} & M_{12}-i\frac{\Gamma_{12}}{2} \\
M_{21}-i\frac{\Gamma_{21}}{2}  & M_{22}-i\frac{\Gamma_{22}}{2}
    \end{array}\right)
\ee
with $\hat M$ and $\hat\Gamma$ being hermitian matrices having positive
(real) eigenvalues in analogy with $M$ and $\Gamma$. $M_{ij}$ and
$\Gamma_{ij}$ are the transition matrix elements from virtual and physical
intermediate states respectively.
Using
\be
M_{21}=M^*_{12}~, \qquad 
\Gamma_{21}=\Gamma_{12}^*~,\quad\quad {\rm (hermiticity)}
\ee
\be
M_{11}=M_{22}\equiv M~, \qquad \Gamma_{11}=\Gamma_{22}\equiv\Gamma~,
\quad {\rm (CPT)}
\ee
we have
\be\label{MM12}
\hat H=
 \left(\begin{array}{cc} 
M-i\frac{\Gamma}{2} & M_{12}-i\frac{\Gamma_{12}}{2} \\
M^*_{12}-i\frac{\Gamma^*_{12}}{2}  & M-i\frac{\Gamma}{2}
    \end{array}\right)~.
\ee

We can next diagonalize the system to find:

{\bf Eigenstates:}
\be\label{KLS}
K_{L,S}=\frac{(1+\bar\varepsilon)K^0\pm (1-\bar\varepsilon)\bar K^0}
        {\sqrt{2(1+\mid\bar\varepsilon\mid^2)}}
\ee
where $\bar\varepsilon$ is a small complex parameter given by
\be\label{bare3}
\frac{1-\bar\varepsilon}{1+\bar\varepsilon}=
\sqrt{\frac{M^*_{12}-i\frac{1}{2}\Gamma^*_{12}}
{M_{12}-i\frac{1}{2}\Gamma_{12}}}~.
\ee

{\bf Eigenvalues:}
\be
M_{L,S}=M\pm \RE Q  \qquad \Gamma_{L,S}=\Gamma\mp 2 \IM Q
\ee
where
\be
Q=\sqrt{(M_{12}-i\frac{1}{2}\Gamma_{12})(M^*_{12}-i\frac{1}{2}\Gamma^*_{12})}.
\ee
Consequently we have
\be\label{deltak}
\Delta M= M_L-M_S = 2\RE Q
\quad\quad
\Delta\Gamma=\Gamma_L-\Gamma_S=-4 \IM Q.
\ee

It should be noted that the mass eigenstates $K_S$ and $K_L$ differ from 
CP eigenstates
\begin{equation}
K_1={1\over{\sqrt 2}}(K^0-\bar K^0),
  \qquad\qquad CP|K_1\rangle=|K_1\rangle~,
\end{equation}
\begin{equation}
K_2={1\over{\sqrt 2}}(K^0+\bar K^0),
  \qquad\qquad CP|K_2\rangle=-|K_2\rangle~,
\end{equation}
by 
a small admixture of the
other CP eigenstate:
\begin{equation}
K_{\rm S}={{K_1+\bar\varepsilon K_2}
\over{\sqrt{1+\mid\bar\varepsilon\mid^2}}},
\qquad
K_{\rm L}={{K_2+\bar\varepsilon K_1}
\over{\sqrt{1+\mid\bar\varepsilon\mid^2}}}\,
\end{equation}
with $\bar\varepsilon$ defined in (\ref{bare3}).
$\bar\varepsilon$ can also be written as
\be\label{bare}
\frac{1-\bar\varepsilon}{1+\bar\varepsilon}
=\frac{\Delta M-i\frac{1}{2}\Delta\Gamma}
{2 M_{12}-i\Gamma_{12}}\equiv r\exp(i\kappa)~.
\ee

It should be stressed that
the small parameter $\bar\varepsilon$  depends on the 
phase convention
chosen for $K^0$ and $\bar K^0$. Therefore it may not 
be taken as a physical measure of CP violation.
On the other hand $\RE\bar\varepsilon$ and $r$ are independent of
phase conventions. In particular the departure of $r$ from 1
measures CP violation in the $K^0-\bar K^0$ mixing:
\be
r=1+\frac{2 |\Gamma_{12}|^2}{4 |M_{12}|^2+|\Gamma_{12}|^2}
    \IM\left(\frac{M_{12}}{\Gamma_{12}}\right)~.
\ee
Since $\bar\varepsilon$ is $\ord(10^{-3})$, we find, using (\ref{bare3}),
that
\be
\IM M_{12}\ll\RE M_{12}, \quad\quad 
\IM \Gamma_{12}\ll\RE \Gamma_{12}~.
\ee
Consequently to a very good approximation:
\be\label{deltak1}
\Delta M_K = 2 \RE M_{12}, \qquad \Delta\Gamma_K=2 \RE \Gamma_{12}~,
\ee
where we have introduced the subscript K to stress that these formulae apply
only to the $K^0-\bar K^0$ system.

The 
$K_{\rm L}-K_{\rm S}$
mass difference is experimentally measured to be 
\begin{equation}\label{DMEXP}
\Delta M_K=M(K_{\rm L})-M(K_{\rm S}) = 
(3.491\pm 0.009) \cdot 10^{-15} \gev\,.
\end{equation}
In the Standard Model roughly $70\%$ of the measured $\Delta M_K$
is described by the real parts of the box diagrams with charm quark
and top quark exchanges, wherby the contribution of the charm exchanges
is by far dominant. This is related to the smallness of the real parts
of the CKM top quark couplings compared with the corresponding charm
quark couplings. Thus even if the function $S_0(x_t)$ is by a factor
of 1600 larger than $S_0(x_c)$, it cannot compensate for the smallness
of the real top quark couplings. 
Some non-negligible contribution comes from the box diagrams with
simultaneous charm and top exchanges.
The $u$-quark contribution is needed only
for GIM mechanism but otherwise can be neglected.
The remaining $30 \%$ of the measured $\Delta M_K$ is attributed to long 
distance contributions which are difficult to estimate \cite{GERAR}.
It is a useful exercise to check these statements by using 
$\Delta M_K$ in (\ref{deltak}) and the expression for $M_{12}$ given 
in (\ref{eq:M12K}).
Further information with the relevant references can be found in 
\cite{HNa}.

The situation with $\Delta \Gamma_K$ is rather different.
It is fully dominated by long distance effects. Experimentally
one has
\begin{equation}
\Delta \Gamma_K=\Gamma(K_{\rm L})-\Gamma(K_{\rm S}) = 
-7.4 \cdot 10^{-15} \gev\,
\end{equation}
and consequently $\Delta\Gamma_K\approx-2 \Delta M_K$.

With all this information at hand and using  the experimentally 
observed dominance of $\Delta I=1/2$ transitions in $K\to \pi\pi$,
it is possible to derive an important formula for $\bar\varepsilon$
\be\label{basice}
\bar\varepsilon=\frac{i}{1+i}\frac{\IM M_{12}}{\Delta M_K}+
\frac{\xi}{1+i}~, \quad\qquad \xi = \frac{\IM A_0}{\RE A_0}~,
\ee
with the isospin amplitude $A_0$ defined below.
An explicit derivation of (\ref{basice}) can be found in a review
by Chau \cite{CHAU83}. A recent review by Belusevic \cite{BELU} is
also useful in this respect. 
Finally I recommend strongly  excellent lectures
by Yossi Nir \cite{NIRSLAC}, where the issues of phase conventions
are discussed in detail.
\subsection{The First Look at $\varepsilon$ and $\varepsilon'$}
Let us next move to two important CP violating  parameters which 
can be measured experimentally. The route to them proceeds as follows.
It involves the decays $K\to\pi\pi$.

Since a two pion final state is CP even while a three pion final state is CP
odd, $K_{\rm S}$ and $K_{\rm L}$ preferably decay to $2\pi$ and $3\pi$, 
respectively
via the following CP conserving decay modes:
\begin{equation}
K_{\rm L}\to 3\pi {\rm ~~(via~K_2),}\qquad K_{\rm S}\to 2 
\pi {\rm ~~(via~K_1).}
\end{equation}
This difference is responsible for the large disparity in their
life-times. A factor of 579.
However, $K_{\rm L}$ and $K_{\rm S}$ are not CP eigenstates and 
may decay with small branching fractions as follows:
\begin{equation}
K_{\rm L}\to 2\pi {\rm ~~(via~K_1),}\qquad K_{\rm S}\to 3 
\pi {\rm ~~(via~K_2).}
\end{equation}
This violation of CP is called {\it indirect} as it
proceeds not via explicit breaking of the CP symmetry in 
the decay itself but via the admixture of the CP state with opposite 
CP parity to the dominant one.
 The measure for this
indirect CP violation is defined as
\begin{equation}\label{ek}
\varepsilon
={{A(K_{\rm L}\rightarrow(\pi\pi)_{I=0}})\over{A(K_{\rm 
S}\rightarrow(\pi\pi)_{I=0})}},
\end{equation}
where $\varepsilon$ is, contrary to $\bar\varepsilon$ in (\ref{basice}),
independent of the phase conventions.
Following the derivation in \cite{CHAU83} one finds
\be\label{basic2}
\varepsilon=\bar\varepsilon+i \xi~.
\ee
The phase convention dependence of the term $i \xi$ cancells
the convention dependence of $\bar\varepsilon$. We will 
write down a nicer formula for $\varepsilon$ below.

\begin{figure}[hbt]
\centerline{
\epsfysize=1.5in
\epsffile{fig13.ps}
}
\caption[]{
Indirect versus direct CP violation in $K_L \to \pi\pi$.
\label{fig:14}}
\end{figure}

While {\it indirect} CP violation reflects the fact that the mass
eigenstates are not CP eigenstates, so-called {\it direct}
CP violation is realized via a 
direct transition of a CP odd to a CP even state or vice versa (see
fig.~\ref{fig:14}). 
A measure of such a direct CP violation in $K_L\to \pi\pi$ is characterized
by a complex parameter $\varepsilon'$  defined as
\begin{equation}\label{eprime}
\varepsilon'={1\over {\sqrt 2}}\IM
\left({{A_2}\over{A_0}}\right) e^{i\Phi},\quad\quad
\Phi=\pi/2+\delta_2-\delta_0,
\end{equation}
where
the isospin amplitudes $A_I$ in $K\to\pi\pi$
decays are introduced through
\begin{equation} 
A(K^+\rightarrow\pi^+\pi^0)=\sqrt{3\over 2} A_2 e^{i\delta_2}
\end{equation}
\begin{equation} 
A(K^0\rightarrow\pi^+\pi^-)=\sqrt{2\over 3} A_0 e^{i\delta_0}+ \sqrt{1\over
3} A_2 e^{i\delta_2}
\end{equation}
\begin{equation}
A(K^0\rightarrow\pi^0\pi^0)=\sqrt{2\over 3} A_0 e^{i\delta_0}-2\sqrt{1\over
3} A_2 e^{i\delta_2}\,.
\end{equation} 
Here the subscript $I=0,2$ denotes states with isospin $0,2$
equivalent to $\Delta I=1/2$ and $\Delta I = 3/2$ transitions,
respectively, and $\delta_{0,2}$ are the corresponding strong phases. 
The weak CKM phases are contained in $A_0$ and $A_2$.
The strong phases $\delta_{0,2}$ cannot be calculated, at least, at present.
They can be extracted from $\pi\pi$ scattering. Then
$\Phi\approx \pi/4$.

The isospin amplitudes $A_I$ are complex quantities which depend on
phase conventions. On the other hand, $\varepsilon'$ measures the 
difference between the phases of $A_2$ and $A_0$ and is a physical
quantity.

Experimentally $\varepsilon$ and $\varepsilon'$
can be found by measuring the ratios
\begin{equation}
\eta_{00}={{A(K_{\rm L}\to\pi^0\pi^0)}\over{A(K_{\rm S}\to\pi^0\pi^0)}},
            \qquad
  \eta_{+-}={{A(K_{\rm L}\to\pi^+\pi^-)}\over{A(K_{\rm S}\to\pi^+\pi^-)}}.
\end{equation}
Indeed, assuming $\varepsilon$ and $\varepsilon'$ to be small numbers one
finds
\be
\eta_{00}=\varepsilon-{{2\varepsilon'}\over{1-\sqrt{\omega}}}
            \simeq \varepsilon-2\varepsilon',~~~~
  \eta_{+-}=\varepsilon+{{\varepsilon'}\over{1+\omega/\sqrt{2}}}
            \simeq \varepsilon+ \varepsilon'
\end{equation}
where experimentally $\omega=\RE A_2/\RE A_0=0.045$.

In the absence of direct CP violation $\eta_{00}=\eta_{+-}$.
The ratio ${\varepsilon'}/{\varepsilon}$  can then be measured through
\begin{equation}
\left|{{\eta_{00}}\over{\eta_{+-}}}\right|^2\simeq 1 -6\; 
\RE(\frac{\varepsilon'}{\varepsilon})\,.
\end{equation}
\subsection{Basic Formula for $\eps$}
            \label{subsec:epsformula}
With all this information at hand let us derive a formula for $\varepsilon$
which can be efficiently used in pheneomenological applications.
Using (\ref{basice}) and (\ref{basic2}) we
first find the general formula 
\begin{equation}
\eps = \frac{\exp(i \pi/4)}{\sqrt{2} \Delta M_K} \,
\left( \IM M_{12} + 2 \xi \RE M_{12} \right),
\quad\quad
\xi = \frac{\IM A_0}{\RE A_0}.
\label{eq:epsdef}
\end{equation}
The two terms in (\ref{eq:epsdef}) are separately phase convention
dependent but there sum is free from this dependence.
The off-diagonal 
element $M_{12}$ in
the neutral $K$-meson mass matrix represents $K^0$-$\bar K^0$
mixing. It is given by
\begin{equation}
2 m_K M^*_{12} = \langle \bar K^0| \Heff(\Delta S=2) |K^0\rangle\,,
\label{eq:M12Kdef}
\end{equation}
where $\Heff(\Delta S=2)$ is the effective Hamiltonian for the 
$\Delta S=2$ transitions.
That $ M^*_{12}$ and not $ M_{12}$ stands on the l.h.s of this formula,
is evident from (\ref{MM12}). The factor $2 m_K$ reflects our normalization
of external states.

To lowest order in electroweak interactions $\Delta S=2$ transitions 
are induced
through the box diagrams of fig. \ref{L:9}. Including
 QCD corrections in the manner analogous to the one already discussed for
$\Delta B=2 $ transitions in Section 8.3 one has \cite{BJW90}
\begin{eqnarray}\label{hds2}
{\cal H}^{\Delta S=2}_{\rm eff}&=&\frac{G^2_{\rm F}}{16\pi^2}M^2_W
 \left[\lambda^2_c\eta_1 S_0(x_c)+\lambda^2_t \eta_2 S_0(x_t)+
 2\lambda_c\lambda_t \eta_3 S_0(x_c, x_t)\right] \times
\nonumber\\
& & \times \left[\as^{(3)}(\mu)\right]^{-2/9}\left[
  1 + \frac{\as^{(3)}(\mu)}{4\pi} J_3\right]  Q(\Delta S=2) + h. c.
\end{eqnarray}
where
$\lambda_i = V_{is}^* V_{id}^{}$. Here
$\mu<\mu_c=\ord(m_c)$.
In (\ref{hds2}),
the relevant operator
\begin{equation}\label{qsdsd}
Q(\Delta S=2)=(\bar sd)_{V-A}(\bar sd)_{V-A},
\end{equation}
is multiplied by the corresponding coefficient function.
This function is decomposed into a
charm-, a top- and a mixed charm-top contribution.
This form is obtained upon eliminating $\lambda_u$
by means of the unitarity of the CKM matrix and setting $x_u=0$. 
The functions $S_0$  are given in (\ref{S0})--(\ref{BFF}).

Short-distance QCD effects are described through the correction
factors $\eta_1$, $\eta_2$, $\eta_3$ and the explicitly
$\alpha_s$-dependent terms in (\ref{hds2}). 
$\eta_2$ is the analogue of $\eta_B$ discussed in Section 8.3.
The calculation of $\eta_1$ and $\eta_3$ is more involved and
is discussed in \cite{HNa,HNb}.
$\eta_{1-3}$ are defined in analogy to (\ref{ETANLO}). This means that
in $\ord(\as)$ they are independent of the renormalization
scales and the renormalization scheme for the operator $Q(\Delta S)$.
The NLO values of $\eta_i$ are given as follows \cite{HNa,BJW90,HNb}:
\begin{equation}
\eta_1=1.38\pm 0.20,\qquad
\eta_2=0.57\pm 0.01,\qquad
  \eta_3=0.47\pm0.04~.
\end{equation}
The quoted errors reflect the remaining theoretical uncertainties due to
leftover $\mu$-dependences at $\ord(\as^2)$ and $\Lambda_{\overline{MS}}$.
The factor $\eta_1$ plays only a minor role in the analysis of
$\varepsilon$ but its enhanced value through NLO corrections 
is essential for the $K_{\rm L}-K_{\rm S}$ mass difference.
We refer to \cite{HNa} for the discussion of $\Delta M_K$.

Defining, in analogy to (\ref{Def-Bpar0}),  the renormalization group 
invariant parameter $\hat B_K$ by
\begin{equation}
\hat B_K = B_K(\mu) \left[ \alpha_s^{(3)}(\mu) \right]^{-2/9} \,
\left[ 1 + \frac{\alpha_s^{(3)}(\mu)}{4\pi} J_3 \right]
\label{eq:BKrenorm}
\end{equation}
\begin{equation}
\langle \bar K^0| (\bar s d)_{V-A} (\bar s d)_{V-A} |K^0\rangle
\equiv \frac{8}{3} B_K(\mu) F_K^2 m_K^2
\label{eq:KbarK}
\end{equation}
and using (\eqn{hds2}) one finds
\begin{equation}
M_{12} = \frac{G_{\rm F}^2}{12 \pi^2} F_K^2 \hat B_K m_K \mw^2
\left[ {\lambda_c^*}^2 \eta_1 S_0(x_c) + {\lambda_t^*}^2 \eta_2 S_0(x_t) +
2 {\lambda_c^*} {\lambda_t^*} \eta_3 S_0(x_c, x_t) \right],
\label{eq:M12K}
\end{equation}
where $F_K$ is the $K$-meson decay constant and $m_K$
the $K$-meson mass. 

To proceed further we neglect the last term in (\eqn{eq:epsdef}) as it
 constitutes at most a 2\,\% correction to $\eps$. This is justified
in view of other uncertainties, in particular those connected with
$B_K$.
Inserting (\eqn{eq:M12K}) into (\eqn{eq:epsdef}) we find
\begin{equation}
\eps=C_{\eps} \hat B_K \IM\lambda_t \left\{
\RE\lambda_c \left[ \eta_1 S_0(x_c) - \eta_3 S_0(x_c, x_t) \right] -
\RE\lambda_t \eta_2 S_0(x_t) \right\} \exp(i \pi/4)\,,
\label{eq:epsformula}
\end{equation}
where we have used the unitarity relation $\IM\lambda_c^* = {\rm
Im}\lambda_t$ and  have neglected $\RE\lambda_t/\RE\lambda_c
 = \ord(\lambda^4)$ in evaluating $\IM(\lambda_c^* \lambda_t^*)$.
The numerical constant $C_\eps$ is given by
\begin{equation}
C_\eps = \frac{G_{\rm F}^2 F_K^2 m_K \mw^2}{6 \sqrt{2} \pi^2 \Delta M_K}
       = 3.78 \cdot 10^4 \, .
\label{eq:Ceps}
\end{equation}
To this end we have used the experimental value of $\Delta M_K$ 
in (\ref{DMEXP}). In principle we could use the theoretical
value for $\Delta M_K$ but in view of the presence of long
distance contributions it is safer to use the experimental
value. In this context it should be stressed that 
the parameter $\varepsilon$ 
being related to CP violation and top quark physics should
be dominated by short distance contributions and well approximated
by the imaginary parts of the box diagrams.
Consequently the only non-perturbative uncertainty in 
(\ref{eq:epsformula}) resides in $\hat B_K$.

Using the standard parametrization of (\eqn{2.72}) to evaluate ${\rm
Im}\lambda_i$ and $\RE\lambda_i$, setting the values for $s_{12}$,
$s_{13}$, $s_{23}$ and $\mt$ in accordance with experiment
 and taking a value for $\hat B_K$ (see below), one can
determine the phase $\delta$ by comparing (\eqn{eq:epsformula}) with the
experimental value for $\eps$
\begin{equation}\label{eexp}
\varepsilon^{exp}
=(2.280\pm0.013)\cdot10^{-3}\;e^{i{\pi\over 4}}\,.
\end{equation}

Once $\delta$ has been determined in this manner one can find the
apex $(\bar\varrho,\bar\eta)$ of the unitarity triangle
in fig. \ref{fig:utriangle}   by using 
\begin{equation}\label{2.84a} 
\varrho=\frac{s_{13}}{s_{12}s_{23}}\cos\delta,
\qquad
\eta=\frac{s_{13}}{s_{12}s_{23}}\sin\delta
\end{equation}
and
\begin{equation}\label{2.88da}
\bar\varrho=\varrho (1-\frac{\lambda^2}{2}),
\qquad
\bar\eta=\eta (1-\frac{\lambda^2}{2}).
\end{equation}

For a given set ($s_{12}$, $s_{13}$, $s_{23}$,
$\mt$, $\hat B_K$) there are two solutions for $\delta$ and consequently two
solutions for $(\bar\varrho,\bar\eta)$. 
This will be evident from the analysis of the unitarity triangle discussed
in detail below.

Finally we have to say a few words about the non-perturbative
parameter $\hat B_K$. There is a long history of evaluating  this parameter
in various non-perturbative approaches. A short review of older results
can be found in \cite{BF97}.
The present status of quenched lattice calculations has been recently
reviewed by Gupta \cite{GUPTA98}.
The most accurate result for $B_K(2~\gev)$
using lattice method is obtained by JLQCD collaboration  \cite{JLQCD}:
$B_K(2~\gev)=0.628\pm0.042$. 
A similar result has been published by Gupta, Kilcup and
Sharpe \cite{GKS} last year.
The APE collaboration \cite{APE} 
finds $B_K(2~\gev)=0.66\pm0.11$ which is
consistent with JLQCD and GKS. In order to convert these values into
$\hat B_K$ by means of (\ref{eq:BKrenorm}) one has to face the issue
of the choice of the number of flavours $f$. Fortunately the values for
$\hat B_K$ for $f=0$ and $f=3$ corresponding to the JLQCD result,
turn out to be very similar: $\hat B_K=0.87\pm0.06$ and
$\hat B_K=0.84\pm0.06$, respectively. The final present lattice value
given by Gupta is then
\be
(\hat B_K)_{\rm Lattice}=0.86\pm0.06\pm0.06
\ee
where the second error is attributed to quenching. The
corresponding result from APE  is $\hat B_K=0.93\pm0.16$.
On the other hand a recent analysis in
the chiral quark model gives surprisingly a value as high as 
$\hat B_K=1.1\pm 0.2$ \cite{BERT97}. In our numerical analysis presented 
below we will use 
\begin{equation}\label{BKT}
\hat B_K=0.75\pm 0.15 \,.
\end{equation}
which is in the ball park of various lattice estimates and
$\hat B_K=0.70\pm 0.10$ from
the $1/N$ approach \cite{BBG0,Bijnens}.
These values are higher than those found using QCD Hadronic Duality
approach   ($\hat B_K=0.39\pm0.10$) \cite{Prades} 
or using the SU(3) symmetry and
PCAC ($\hat B_K=1/3$) \cite{Donoghue}.
  
As we will see below, $\hat B_K\le 0.75 $ requires simultaneously high
values of $|V_{ub}/V_{cb}|$ and $\vcb$ in order to be able to fit
the experimental value of $\varepsilon$.


\subsection{Basic Formula for $B^0$-$\bar B^0$ Mixing}
            \label{subsec:BBformula}
The strength of the $B^0_{d,s}-\bar B^0_{d,s}$ mixings
is described by the mass differences
\begin{equation}
\Delta M_{d,s}= M_H^{d,s}-M_L^{d,s}
\end{equation}
with ``H'' and ``L'' denoting {\it Heavy} and {\it Light} respectively. 
In contrast to $\Delta M_K$ , in this case the long distance contributions
are estimated to be very small and $\Delta M_{d,s}$ is very well
approximated by the relevant box diagrams. 
Moreover, due $m_{u,c}\ll m_t$ 
only the top sector can contribute significantly to 
$B_{d,s}^0-\bar B_{d,s}^0$ mixings.
The charm sector and the mixed top-charm contributions are
entirely negligible. This can be easily verified and is left as an useful
exercise.

 $\Delta M_{d,s}$ can be expressed
in terms of the off-diagonal element in the neutral B-meson mass matrix
by using the formulae developed previously for the K-meson system.
One finds
\begin{equation}
\Delta M_q= 2 |M_{12}^{(q)}|, \qquad q=d,s.
\label{eq:xdsdef}
\end{equation}
This formula differs from $\Delta M_K=2 \RE M_{12}$ because in the
B-system $\Gamma_{12}\ll M_{12}$.

Equivalently, the mixing can be described by
\begin{equation}
x_q \equiv \frac{\Delta M_q}{\Gamma_{B_q}},
\end{equation}
where  $\Gamma_{B_q} = 1/\tau_{B_q}$ with
$\tau_{B_q}$ being the corresponding lifetimes.
However, working with $\Delta M_q$ instead of $x_q$
avoids the experimental errors in lifetimes. 

The off-diagonal
term $M_{12}$ in the neutral $B$-meson mass matrix is then given by
a formula analogous to (\ref{eq:M12Kdef})
\begin{equation}
2 m_{B_q} |M_{12}^{(q)}| = 
|\langle \bar B^0_q| \Heff(\Delta B=2) |B^0_q\rangle|,
\label{eq:M12Bdef}
\end{equation}
where 
in the case of $B_d^0-\bar B_d^0$
mixing 
\begin{eqnarray}\label{hdb2}
{\cal H}^{\Delta B=2}_{\rm eff}&=&\frac{G^2_{\rm F}}{16\pi^2}M^2_W
 \left(V^\ast_{tb}V_{td}\right)^2 \eta_{B}
 S_0(x_t)\times
\nonumber\\
& &\times \left[\alpha^{(5)}_s(\mu_b)\right]^{-6/23}\left[
  1 + \frac{\alpha^{(5)}_s(\mu_b)}{4\pi} J_5\right]  Q(\Delta B=2) + h. c.
\end{eqnarray}
Here $\mu_b=\ord(m_b)$,
\begin{equation}\label{qbdbd}
Q(\Delta B=2)=(\bar bd)_{V-A}(\bar bd)_{V-A}
\end{equation}
and \cite{BJW90}
\begin{equation}
\eta_B=0.55\pm0.01.
\end{equation}
Finally $J_5=1.627$ in the NDR scheme. 
In the case of  $B_s^0-\bar B_s^0$ mixing one should simply replace
$d\to s$ in (\ref{hdb2}) and (\ref{qbdbd}) with all other quantities
unchanged.

We next reapeat what we have done already in Section 8.3.
Defining the renormalization group invariant parameters $\hat B_q$
by
\begin{equation}\label{Def-Bpar1}
\hat B_{B_q} = B_{B_q}(\mu) \left[ \as^{(5)}(\mu) \right]^{-6/23} \,
\left[ 1 + \frac{\as^{(5)}(\mu)}{4\pi} J_5 \right]
\label{eq:BBrenorm}
\end{equation}
\begin{equation}
\langle \bar B^0_q| (\bar b q)_{V-A} (\bar b q)_{V-A} |B^0_q\rangle
\equiv \frac{8}{3} B_{B_q}(\mu) F_{B_q}^2 m_{B_q}^2\,,
\label{eq:BbarB}
\end{equation}
where
$F_{B_q}$ is the $B_q$-meson decay constant
and using (\ref{hdb2}) one finds
\begin{equation}
\Delta M_q = \frac{G_{\rm F}^2}{6 \pi^2} \eta_B m_{B_q} 
(\hat B_{B_q} F_{B_q}^2 ) \mw^2 S_0(x_t) |V_{tq}|^2,
\label{eq:xds}
\end{equation}
which implies two useful formulae
\begin{equation}\label{DMD}
\Delta M_d=
0.50/{\rm ps}\cdot\left[ 
\frac{\sqrt{\hat B_{B_d}}F_{B_d}}{200\mev}\right]^2
\left[\frac{\mtb(\mt)}{170\gev}\right]^{1.52} 
\left[\frac{\vtd}{8.8\cdot10^{-3}} \right]^2 
\left[\frac{\eta_B}{0.55}\right]  
\end{equation}
and
\begin{equation}\label{DMS}
\Delta M_{s}=
15.1/{\rm ps}\cdot\left[ 
\frac{\sqrt{\hat B_{B_s}}F_{B_s}}{240\mev}\right]^2
\left[\frac{\mtb(\mt)}{170\gev}\right]^{1.52} 
\left[\frac{\vts}{0.040} \right]^2
\left[\frac{\eta_B}{0.55}\right] \,.
\end{equation}

There is a vast literature on the calculations of $F_{B_d}$ and
$\hat B_d$.
The most recent world averages from lattice are \cite{Flynn,Bernard}
\begin{equation}
F_{B_d}=(175\pm 25)\mev\,, \qquad
\hat B_{B_d}=1.31\pm 0.03\,.
\end{equation}
This result for $F_{B_d}$ is compatible with the results obtained 
with the help of QCD sum rules   \cite{QCDSF}.
In our numerical analysis we will use
\be
F_{B_d}\sqrt{\hat B_{B_d}}=(200\pm 40)\mev.
\ee
The experimental situation on
$\Delta M_d$ taken from Gibbons \cite{Gibbons}
 is given in table \ref{tab:inputparams}. 
\subsection{Standard Analysis of the Unitarity Triangle}\label{UT-Det}
With all these formulae at hand we can now summarize the standard
analysis of the unitarity triangle in fig. \ref{fig:utriangle}. 
It proceeds in five steps.

{\bf Step 1:}

{}From  $b\to c$ transition in inclusive and exclusive $B$ meson decays
one finds $\vcb$ and consequently the scale of the unitarity triangle:
\begin{equation}
\vcb\quad \Longrightarrow\quad\lambda \vcb= \lambda^3 A
\end{equation}

{\bf Step 2:}

{}From  $b\to u$ transition in inclusive and exclusive $B$ meson decays
one finds $\vub$ and consequently the side $CA=R_b$ of the unitarity
triangle:
\begin{equation}\label{rb}
\left| \frac{V_{ub}}{V_{cb}} \right|
 \quad\Longrightarrow \quad R_b=\sqrt{\bar\varrho^2+\bar\eta^2}=
4.44 \cdot \left| \frac{V_{ub}}{V_{cb}} \right|
\end{equation}

{\bf Step 3:}

{}From experimental value of $\varepsilon$ (\ref{eexp}) 
and the formula (\ref{eq:epsformula}) one 
derives, using the approximations (\ref{2.51})--(\ref{2.53}), 
the constraint
\begin{equation}\label{100}
\bar\eta \left[(1-\bar\varrho) A^2 \eta_2 S_0(x_t)
+ P_0(\varepsilon) \right] A^2 \hat B_K = 0.226,
\end{equation}
where
\begin{equation}\label{102}
P_0(\varepsilon) = 
\left[ \eta_3 S_0(x_c,x_t) - \eta_1 x_c \right] \frac{1}{\lambda^4},
\qquad
x_t=\frac{\mt^2}{\mw^2}.
\end{equation}
 $P_0(\varepsilon)=0.31\pm0.05$ summarizes the contributions
of box diagrams with two charm quark exchanges and the mixed 
charm-top exchanges. The error in $P_0(\varepsilon)$ is dominated by the
uncertainties in $\eta_3$ and $m_c$.
However, the $P_0(\varepsilon)$
term contributes only $25\%$ to (\ref{100}) and these uncertainties
constitute only  a few percent uncertainty in the constraint
(\ref{100}). 
Recalling that
$\mt$ and the relevant QCD factors $\eta_2$ and $\eta_3$ 
are rather precisely known, we conclude that
the main uncertainties in the constraint (\ref{100}) reside in
$\hat B_K$ and to some extent in $A^4$ which multiplies the leading term.

\begin{figure}[hbt]
\vspace{0.010in}
\centerline{
\epsfysize=4in
\rotate[r]{
\epsffile{L10.ps}
}}
\vspace{0.0108in}
\caption[]{Schematic determination of Unitarity Triangle.
\label{L:10}}
\end{figure}
Equation (\ref{100}) specifies 
a hyperbola in the $(\bar \varrho, \bar\eta)$
plane.
This hyperbola intersects the circle found in step 2
in two points which correspond to the two solutions for
$\delta$ mentioned earlier. This is illustrated in fig. \ref{L:10}.
The position of the hyperbola (\ref{100}) in the
$(\bar\varrho,\bar\eta)$ plane depends on $\mt$, $|V_{cb}|=A \lambda^2$
and $\hat B_K$. With decreasing $\mt$, $|V_{cb}|$ and $\hat B_K$ the
$\eps$-hyperbola moves away from the origin of the
$(\bar\varrho,\bar\eta)$ plane. When the hyperbola and the circle
(\ref{rb}) touch each other lower bounds consistent with $\eps_K^{\rm
exp}$  can be found \cite{Buras}:
\begin{eqnarray}
(\mt)_{\rm min} &=& \mw \left[ \frac{1}{2 A^2} \left( \frac{1}
{A^2 \hat B_K R_b} - 1.4 \right) \right]^{0.658}
\label{eq:mtmin} \\
\left| \frac{V_{ub}}{V_{cb}} \right|_{\rm min} &=&
\frac{\lambda}{1-\lambda^2/2} \,
\left[ A^2 \hat B_K \left( 2 x_t^{0.76} A^2 + 1.4 \right) \right]^{-1}
\label{eq:Vubcbmin} \\
(\hat B_K)_{\rm min} &=& 
\left[ A^2 R_b \left( 2 x_t^{0.76} A^2 + 1.4 \right)
                    \right]^{-1}.
\label{eq:BKmin}
\end{eqnarray}

{\bf Step 4:}
{}From the observed $B^0_d-\bar B^0_d$ mixing parametrized by $\Delta M_d$ 
the side $BA=R_t$ of the unitarity triangle can be determined:
\begin{equation}\label{106}
 R_t= \frac{1}{\lambda}\frac{|V_{td}|}{\vcb} = 1.0 \cdot
\left[\frac{|V_{td}|}{8.8\cdot 10^{-3}} \right] 
\left[ \frac{0.040}{\vcb} \right]
\end{equation}
with
\begin{equation}\label{VT}
\vtd=
8.8\cdot 10^{-3}\left[ 
\frac{200\mev}{\sqrt{\hat B_{B_d}}F_{B_d}}\right]
\left[\frac{170~GeV}{\mtb(\mt)} \right]^{0.76} 
\left[\frac{\Delta M_d}{0.50/{\rm ps}} \right ]^{0.5} 
\sqrt{\frac{0.55}{\eta_B}}.
\end{equation}

Since $\mt$, $\Delta M_d$ and $\eta_B$ are already rather precisely
known, the main uncertainty in the determination of $\vtd$ from
$B_d^0-\bar B_d^0$ mixing comes from $F_{B_d}\sqrt{B_{B_d}}$.
Note that $R_t$ suffers from additional uncertainty in $\vcb$,
which is absent in the determination of $\vtd$ this way. 
The constraint in the $(\bar\varrho,\bar\eta)$ plane coming from
this step is illustrated in fig.~\ref{L:10}.

{\bf Step 5:}

{}The measurement of $B^0_s-\bar B^0_s$ mixing parametrized by $\Delta M_s$
together with $\Delta M_d$  allows to determine $R_t$ in a different
way. Using (\ref{eq:xds}) and setting $\Delta M^{{\rm max}}_d= 0.482/
\mbox{ps}$ and 
$|V_{ts}/V_{cb}|^{{\rm max}}=0.993$  one finds a useful formula
\cite{ABWAR}:
\begin{equation}\label{107b}
(R_t)_{\rm max} = 1.0 \cdot \xi \sqrt{\frac{10.2/ps}{\Delta M_s}},
\qquad
\xi = 
\frac{F_{B_s} \sqrt{\hat B_{B_s}}}{F_{B_d} \sqrt{\hat B_{B_d}}},
\end{equation}
where $\xi=1$ in the  $SU(3)$--flavour limit.
One should 
note that $\mt$ and $|V_{cb}|$ dependences have been eliminated this way
 and that $\xi$ should in principle 
contain much smaller theoretical
uncertainties than the hadronic matrix elements in $\Delta M_d$ and 
$\Delta M_s$ separately.  
The most recent values relevant for (\ref{107b}) are:
\begin{equation}\label{107c}
\Delta M_s > 10.2/ ps ~(95\%~{\rm  C.L.})
\qquad\quad
\xi=1.15\pm 0.05
\end{equation}
The first number is the improved lower bound from ALEPH \cite{Drell}.
The second number comes from quenched lattice calculations summarized
in \cite{Flynn} and \cite{Bernard}.
A similar result has been obtained using QCD sum rules \cite{NAR}.

The fate of the usefulness of the bound (\ref{107b}) depends clearly
on both $\Delta M_s$ and $\xi$ as well as on the type of the error
analysis. We will return to this point soon.
For $\xi=1.2$ 
the lower bound on $\Delta M_s$ in (\ref{107c}) implies $R_t\le 1.20$
which, as we will see, has a moderate impact on the unitarity triangle
obtained using the scanning method and 
the first four steps alone. 

Finally, I would like to point out that whereas step 5 can give, 
in contrast
to step 4, the value for $R_t$ free of the $\vcb$ uncertainty, it does
not provide at present a more accurate value of $\vtd$ if the scanning
method, discussed below, is used. The point is, that 
having $R_t$, one determines $\vtd$ by means of the relation (\ref{106})
which, in contrast to (\ref{VT}), depends on $\vcb$. 
In fact as we will see below, the inclusion
of step 5 has, with $\xi=1.2$, a visible impact on $R_t$ without
essentially any impact on the range of $\vtd$ obtained using the scanning
method and the first four steps alone.
\subsection{Numerical Results}\label{sec:standard}
\subsubsection{Input Parameters}
 The input parameters needed to perform the
standard analysis using the first four steps alone
are given in table \ref{tab:inputparams}.
We list here the "present" errors based on what we have discussed
above, as well as the "future" errors. The latter are a mere guess,
but as we will see in sections 13 and 14, these are the errors
one should aim at, in order that the standard analysis could be
competitive in the CKM determination with the cleanest rare decays and 
the CP asymmetries in B-decays. 

 $\mt$ in table~\ref{tab:inputparams} 
 refers
to the running current top quark mass normalized at $\mu=\mt$:
$\mtb(\mt)$ and is obtained from the value 
$\mt^{Pole}=175\pm 6\gev$ measured by CDF and D0 by means of the
relation.
 \begin{equation}\label{POLE}
\mtb(\mt)=\mt^{{\rm Pole}}
\left[ 1-\frac{4}{3}\frac{\alpha_s(m_t)}{\pi}\right].
\end{equation}
Thus for $\mt={\cal O}(170\gev)$, $\mtb(\mt)$ is typically
by $8\gev$ smaller than $m_t^{\rm Pole}$. 
In principle known $\ord(\as^2)$ corrections to the relation
(\ref{POLE}) could also be included which would decrease the value
of $\mtb(\mt)$ by roughly $1~\gev$.
Yet this would not be really consistent with the rest of the
analysis which does not include the next--to--NLO corrections.

\begin{table}[thb]
\caption[]{Collection of input parameters.\label{tab:inputparams}}
\vspace{0.4cm}
\begin{center}
\begin{tabular}{|c|c|c|c|}
\hline
{\bf Quantity} & {\bf Central} & {\bf Present} & {\bf Future} \\
\hline
$|V_{cb}|$ & 0.040 & $\pm 0.003$ & $\pm 0.001 $\\
$|V_{ub}/V_{cb}|$ & 0.080 & $\pm 0.020$ & $\pm 0.005 $ \\
$\hat B_K$ & 0.75 & $\pm 0.15$ & $\pm 0.05$ \\
$\sqrt{\hat B_d} F_{B_{d}}$ & $200\mev$ & $\pm 40\mev$ &$\pm 10\mev$ \\
$\mt$ & $167\gev$ & $\pm 6\gev$ & $\pm 3\gev $\\
$\Delta M_d$ & $0.464~\mbox{ps}^{-1}$ & $\pm 0.018~\mbox{ps}^{-1}$ 
& $\pm 0.006~\mbox{ps}^{-1}$\\ 
\hline
\end{tabular}
\end{center}
\end{table}
\subsubsection{$\left| V_{ub}/V_{cb} \right|$,
$\left| V_{cb} \right|$ and $\varepsilon_K$}

The values for $\left| V_{ub}/V_{cb} \right|$ 
and $\left| V_{cb} \right|$ in table \ref{tab:inputparams}
are not correlated with
each other. On the other hand such a correlation is present in
the analysis of the CP violating parameter $\varepsilon$ which
is roughly proportional to the fourth power of $\left| V_{cb}\right|$
and linear in $\left|V_{ub}/V_{cb} \right|$. It follows
that not all values in table \ref{tab:inputparams} are simultaneously
consistent with the observed value of $\varepsilon$.
This 
has been emphasized in particular by
Herrlich and Nierste \cite{HNb} and in \cite{BBL}. 
Explicitly one has using (\ref{eq:Vubcbmin}):

\begin{equation}
\left| \frac{V_{ub}}{V_{cb}} \right|_{\rm min}=
\frac{0.225}{\hat B_K A^2(2 x_t^{0.76}A^2+1.4)}.
\end{equation}

\begin{figure}[hbt]
\vspace{0.10in}
\centerline{
\epsfysize=4.5in
\rotate[r]{\epsffile{vubcbmin.ps}}
}
\vspace{0.08in}
\caption{Lower bound on $\vub$ from $\varepsilon_K$.}\label{fig:bound}
\end{figure}

This bound is shown as a function of $\vcb$ for different
values of $\hat B_K$ and $\mt=173\gev$ in fig.\ \ref{fig:bound}. 
We observe that simultaneously
small values of $\left| V_{ub}/V_{cb} \right|$ and $\left| V_{cb} \right|$,
although still consistent with the ones given in 
table \ref{tab:inputparams}, are not allowed
by the size of indirect CP violation observed in $K \to \pi\pi$.


\begin{table}[thb]
\caption[]{Present output of the Standard Analysis. 
 $\lambda_t=V^*_{ts} V_{td}$.\label{TAB2}}
\vspace{0.4cm}
\begin{center}
\begin{tabular}{|c||c||c|}\hline
{\bf Quantity} & {\bf Scanning} & {\bf Gaussian} \\ \hline
$\mid V_{td}\mid/10^{-3}$ &$6.9 - 11.3$ &$ 8.6\pm 1.1$ \\ \hline
$\mid V_{ts}/V_{cb}\mid$ &$0.959 - 0.993$ &$0.976\pm 0.010$  \\ \hline
$\mid V_{td}/V_{ts}\mid$ &$0.16 - 0.31$ &$0.213\pm 0.034$  \\ \hline
$\sin(2\beta)$ &$0.36 - 0.80$ &$ 0.66\pm0.13 $ \\ \hline
$\sin(2\alpha)$ &$-0.76 - 1.0$ &$ 0.11\pm 0.55 $ \\ \hline
$\sin(\gamma)$ &$0.66 - 1.0 $ &$ 0.88\pm0.10 $ \\ \hline
$\IM \lambda_t/10^{-4}$ &$0.86 - 1.71 $ &$ 1.29\pm 0.22 $ \\ \hline
\end{tabular}
\end{center}
\end{table}

\subsubsection{Output of the Standard Analysis}
The output of the standard analysis depends to some extent on the
error analysis. This should be always remembered in view of the fact
that different authors use different procedures. In order to illustrate
this  I show in tables \ref{TAB2} ("present") and \ref{TAB3} ("future") 
the results for various quantities of interest
using two types of error analyses:

\begin{itemize}
\item
Scanning: Both the experimentally measured numbers and the theoretical input
parameters are scanned independently within the errors given in
table~\ref{tab:inputparams}. 
\item
Gaussian: The experimentally measured numbers and the theoretical input 
parameters are used with Gaussian errors.
\end{itemize}
Clearly the "scanning" method is a bit conservative. On the other
hand using Gaussian distributions for theoretical input parameters
can be certainly questioned. 
I think that
at present the conservative "scanning" method should be preferred,
although one certainly would like to have a better method. Interesting
new methods have been presented in \cite{FRENCH,PAGA}.
They provide more stringent bounds on the apex of the unitarity triangle
than presented here. I must admitt that I did not find time yet
to analyze these papers to the extend that I could say anything profound
about them here. I hope to do it soon.
The analysis discussed here has been done in collaboration with Matthias 
Jamin and Markus Lautenbacher \cite{BJL96b}.


In figs.~\ref{fig:utdata} and  \ref{fig:utdataf}  we show the ranges
 for the upper
corner A of the UT in the case of the "present" input and "future" input
respectively. The circles correspond to $R_t^{max}$ from 
(\ref{107b})
using $\xi=1.20$ and $(\Delta M)_s=10/ps,~15/ps$ and $25/ps$, respectively.
The present bound (\ref{107c}) is represented by the first
of these circles. 
This bound has not 
been used in
obtaining the results in tables \ref{TAB2} and \ref{TAB3}. 
Its impact will be analysed separately
below.
The circles from $B^0_d-\bar B^0_d$ mixing are not shown explicitly
for reasons to be explained below. The impact of $\Delta M_d$ can however
be easily seen by comparing the shaded area with the area one would find
by using the lower $\varepsilon$-hyperbola and the $R_b$-circles alone.
\begin{figure}[thb]
\vspace{0.10in}
\centerline{
\epsfysize=3.4in
%\rotate[r]{\epsffile{rhoeta.ps}}
\rotate[r]{\epsffile{ut97.ps}}
}
\vspace{0.08in}
\caption[]{
Unitarity Triangle 1998.
\label{fig:utdata}}
\end{figure}
The allowed region has a typical "banana" shape which can be found
in many other analyses \cite{BLO,ciuchini:95,HNb,ALUT,FRENCH,PAGA}. 
The size of
the banana and its position depends on the assumed input
parameters and on the error analysis which varies from paper
to paper. The results in figs. \ref{fig:utdata} and  \ref{fig:utdataf}
correspond to a simple independent 
scanning of all parameters within one standard deviation.
I should remark that the plots in \cite{PAGA} give substantially smaller
allowed ranges in the $(\bar\varrho,\bar\eta)$ plane and look more
like potatoes than bananas.


\begin{figure}[thb]
\vspace{0.10in}
\centerline{
\epsfysize=3.6in
%\rotate[r]{\epsffile{rhoetaf.ps}}
\rotate[r]{\epsffile{ut07.ps}}
}
\vspace{0.08in}
\caption[]{
Unitarity Triangle 2008.
\label{fig:utdataf}}
\end{figure}

As seen in fig.~\ref{fig:utdata} our present knowledge of
the unitarity triangle is still rather poor. Fig.~\ref{fig:utdataf}
demonstrates very clearly that this situation may change dramatically
in the future provided the errors in the input parameters will be decreased
as anticipated in our "future" scenario. 

Comparing the results for $\vtd$ given in table \ref{TAB2} 
with the ones obtained on the basis of unitarity alone (\ref{uni1}) 
we observe that
the inclusion of the constraints from $\varepsilon$ and $\Delta M_d$
had a considerable impact on the allowed range for this CKM matrix
element. This impact will be amplified in the future as seen in
table \ref{TAB3}. An inspection shows that with our input parameters
the lower bound on $\vtd$ is governed by  $\varepsilon_K$, whereas
the upper bound by $\Delta M_d$.

Next we observe that whereas the angle $\beta$ is rather
constrained, the uncertainties in 
$\alpha$ and $\gamma$ are  huge: 
\be\label{ap}
35^\circ\le \alpha \le 115^\circ~,
\quad
11^\circ\le \beta \le 27^\circ~,
\quad
41^\circ\le \gamma \le 134^\circ~.
\ee
The situation will improve when the "future" scenario
will be realized:
\be\label{af}
70^\circ\le \alpha \le 93^\circ~,
\quad
19^\circ\le \beta \le 22^\circ~,
\quad
65^\circ\le \gamma \le 90^\circ~.
\ee
Finally we would like to comment on the impact of the bound on 
$\Delta M_s$
given in (\ref{107c}) if the scanning method is used.
This impact is still
rather small except for the upper limits for $\vtd/\vts$ and $\gamma$
which are lowered in the "scanning'' version to $0.27$ and $129^\circ$
respectively. Larger impact of the bound on $\Delta M_s$ on various
parameters is
found by using the methods in \cite{FRENCH,PAGA}.
\subsubsection{Correlation between $\varepsilon_K$ and $\Delta M_d$}
Now, why did we omitt the explicit circles from $B^0_d-\bar B^0_d$ mixing 
in the plots of unitarity triangles above ? I have to answer this
question because some of my colleagues suspected that a plot similar
to the one in fig.~\ref{fig:utdata} and shown already at the Rochester
conference in Warsaw was wrong. At first  one would expect that the
left border of the allowed area coming from $B^0_d-\bar B^0_d$ mixing
should have a shape similar to the circles coming from 
$\Delta M_d/\Delta M_s$ and shown in the figures above. This expectation
is correct at fixed values of $m_t$ and $\vcb$. Yet once these
two parameters are varied in the allowed ranges, this is no longer
true. In fact one can easily convince oneself that the uncertainties
coming from $\mt$ and $\vcb$ in the analyses of $\varepsilon_K$ and
$\Delta M_d$ cannot be represented simultaneously in the 
$(\bar\varrho,\bar\eta)$ plane in terms of nice hyperbolas
and nice circles. This is simply related to the correlation between
$\varepsilon_K$ and $\Delta M_d$ due to $m_t$ and $\vcb$. Neglecting
this correlation one finds for instance that the most negative value of 
$\bar\varrho$ corresponds to the maximal values of $(m_t,\vcb)$ in the
case of $\varepsilon_K$ and to the minimal values of $(m_t,\vcb)$ in the
case of $B^0_d-\bar B^0_d$ which is of course inconsistent. In 
figs. \ref{fig:utdata} and  \ref{fig:utdataf} we have decided
to show the $\varepsilon_K$-hyperbolas. Consequently the impact
of $B^0_d-\bar B^0_d$ mixing had to be found numerically and as
seen it is not described by a circle. Since $m_t$ is already
very well known, this discussion mainly applies to the $\vcb$ dependence.
Finally it should be stressed that similar correlations have to
be taken into account in the future when various rare decays discussed in
subsequent sections will enter the game of the determination of the
unitarity triangle. Needless to say, the radius $R^{max}_t$ 
determined through
(\ref{107b}) and shown in the UT plots, being independent of
$(\mt,\vcb)$, is not subject to the correlation in question.

\begin{table}[thb]
\caption[]{Future output of the Standard Analysis. 
 $\lambda_t=V^*_{ts} V_{td}$.\label{TAB3}}
\vspace{0.4cm}
\begin{center}
\begin{tabular}{|c||c||c|}\hline
{\bf Quantity} & {\bf Scanning} & {\bf Gaussian} \\ \hline
$\mid V_{td}\mid/10^{-3}$ &$8.1 - 9.2$ &$ 8.6\pm 0.3$ \\ \hline
$\mid V_{ts}/V_{cb}\mid$ &$0.969 - 0.983$ &$0.976\pm 0.004$  \\ \hline
$\mid V_{td}/V_{ts}\mid$ &$0.20 - 0.24$ &$0.215\pm 0.010$  \\ \hline
$\sin(2\beta)$ &$0.61 - 0.70$ &$ 0.67\pm0.03 $ \\ \hline
$\sin(2\alpha)$ &$-0.11 - 0.66.0$ &$ 0.21\pm 0.21 $ \\ \hline
$\sin(\gamma)$ &$0.90 - 1.0 $ &$ 0.96\pm0.03 $ \\ \hline
$\IM \lambda_t/10^{-4}$ &$1.21 - 1.41 $ &$ 1.29\pm 0.06 $ \\ \hline
\end{tabular}
\end{center}
\end{table}
\subsection{Final Remarks}
In this section we have completed the determination of the CKM matrix.
It is given by the values of $|V_{us}|$, $\vcb$ and $|V_{ub}|$ in
(\ref{vcb}) and (\ref{v13}), the results in table~\ref{TAB2} and
the unitarity triangle shown in fig.~\ref{fig:utdata}. Clearly
the accuracy of this determination is not impressive. We have
stressed, however, that in ten years from now the standard analysis
may give the results shown in table~\ref{TAB3} and fig.~\ref{fig:utdataf}.
Moreover a single precise measurement of $\Delta M_s$ in the future
will have a very important impact on the allowed area in the 
$(\bar\varrho,\bar\eta)$ plane. Such a measurement should come from
SLD and later from LHC.

Having the values of CKM parameters at hand, we can use them to predict
various branching ratios of radiative, rare and CP-violating decays.
This we will do in the subsequent three sections. We will see there,
that the poor knowledge of CKM parameters precludes precise predictions of
a number of interesting branching ratios at present. This may change in
the next decade as stressed above.

\clearpage
\section{$\epe$ in the Standard Model}\label{EpsilonPrime}
\setcounter{equation}{0}
%\setcounter{figure}{0}
%\setcounter{table}{0}
\subsection{Preliminaries}
Direct CP violation remains one of the important targets 
of contemporary particle physics. In this respect the search 
for direct CP violation in $K\to\pi\pi$ decays plays a special
role as already sixteen years have been devoted to this enterprise.
In this case,
a non-vanishing value of the ratio Re($\epe$) defined in (\ref{eprime}) 
would give the first
signal for direct CP violation ruling out superweak models
\cite{wolfenstein:64}.
The experimental situation of Re($\varepsilon'/\varepsilon$) is,
however, unclear
at present:
\begin{equation}\label{eprime1}
\RE(\varepsilon'/\varepsilon) =\left\{ \begin{array}{ll}
(23 \pm 7)\cdot 10^{-4} & \cite{barr:93} \\
(7.4 \pm 5.9)\cdot 10^{-4} & \cite{gibbons:93}.\end{array} \right.
\end{equation}

While the result of the NA31 collaboration at CERN  \cite{barr:93}
clearly indicates direct CP violation, the value of E731 at Fermilab
\cite{gibbons:93} is compatible with superweak models
 in which $\varepsilon'/\varepsilon = 0$.
 Hopefully, during the next two years the experimental situation concerning
$\varepsilon'/\varepsilon$ will be clarified through the improved
measurements by the two collaborations at the $10^{-4}$ level and by
the KLOE experiment at  DA$\Phi$NE. A recent discussion of superweak
models can be found in \cite{HALL}. I will not consider them here.

There is no question about that direct CP violation is present in
the Standard Model. Yet accidentally it could turn out that it will be
difficult to see it in $K \to \pi\pi$ decays.  Indeed as we will
discuss in detail below, in the Standard
Model $\varepsilon'/\varepsilon $ is governed by QCD penguins and
electroweak (EW) penguins. We have met them already in connection
with B-decays in Section 8. In spite of being suppressed by
$\alpha/\alpha_s$ relative to QCD penguin contributions, 
electroweak penguin contributions have to be included because of the
additional enhancement factor ${\rm Re}A_0/{\rm Re}A_2=22$ 
(see (\ref{eq:epsprim})--(\ref{eq:ReA0data})) relative
to QCD penguins. With increasing $\mt$ the EW penguins become
increasingly important \cite{flynn:89,buchallaetal:90} and, entering
$\varepsilon'/\varepsilon$ with the opposite sign to QCD penguins,
suppress this ratio for large $\mt$. For $\mt\approx 200\,\gev$ the ratio
can even be zero \cite{buchallaetal:90}.  Because of this strong
cancellation between two dominant contributions and due to uncertainties
related to hadronic matrix elements of the relevant local operators, a
precise prediction of $\varepsilon'/\varepsilon$ is not possible at
present. We will discuss this in detail below.
\subsection{History of $\epe$}
The first calculations of $\epe$ for $\mt \ll \mw$ and in the leading
order approximation can be found in \cite{GW79}. 
For $\mt \ll \mw$ only QCD
penguins play a substantial role. Over the eighties these calculations
were refined through the inclusion of isospin braking in the
quark masses \cite{donoghueetal:86,burasgerard:87,lusignoli:89},
the inclusion of QED penguin effects for $\mt \ll \mw$
\cite{BW84,donoghueetal:86,burasgerard:87}, 
and through improved estimates of hadronic matrix elements in
the framework of the $1/N$ approach \cite{bardeen:87}. 
This era of $\epe$ culminated
in the analyses in \cite{flynn:89,buchallaetal:90}, where QCD
penguins, electroweak penguins ($\gamma$ and $Z^0$ penguins)
and the relevant box diagrams were included for arbitrary
top quark masses. The strong cancellation between QCD penguins
and electroweak penguins for $m_t > 150~\gev$ found in these
papers was confirmed by other authors \cite{PW91}.

All these calculations were done in the leading logarithmic
approximation (e.g.\ one-loop anomalous dimensions of the relevant
operators) with the exception of the $\mt$-dependence which in 
the analyses \cite{flynn:89,buchallaetal:90,PW91} has been already
included at the NLO level. While such a procedure is not fully
consistent, it allowed for the first time to exhibit the strong
$\mt$-dependence of the electroweak penguin contributions,
which is not seen in a strict leading logarithmic approximation.

During the nineties considerable progrees has been made by
calculating complete NLO corrections to $\varepsilon'$
\cite{BJLW1,BJLW2,BJLW,ROMA1,ROMA2}. Together with the NLO
corrections to $\varepsilon$ and $B^0-\bar B^0$ mixing
discussed in the previous section, this allows
a complete NLO analysis of $\varepsilon'/\varepsilon$ including
constraints from the observed indirect CP violation ($\varepsilon$)
and  $B_{d,s}^0-\bar B_{d,s}^0$ mixings ($\Delta M_{d,s}$). The improved
determination of the $V_{ub}$ and $V_{cb}$ elements of the CKM matrix,
the improved estimates of hadronic matrix elements using the lattice 
approach as well as other non-perturbative approaches 
and in particular the determination of the top quark mass
$\mt$ had of course also an important impact on
$\varepsilon'/\varepsilon$. 

After these general remarks let us discuss 
$\epe$ in explicit terms. Other reviews of $\epe$ can be found
in \cite{WW,BERT98}.
\subsection{Basic Formulae}
           \label{subsec:epeformulae}
The direct CP violation in $K \to \pi\pi$ is described by the parameter
$\varepsilon'$ defined in (\ref{eprime}).
The latter formula  can be rewritten 
in terms of the real and imaginary parts of 
the amplitudes $A_0 \equiv
A(K \to (\pi\pi)_{I=0})$ and $A_2 \equiv
A(K \to (\pi\pi)_{I=2})$ as follows:
\begin{equation}
\eps' = -\frac{\omega}{\sqrt{2}} \xi (1 - \Omega) \exp(i \Phi) \, ,
\label{eq:epsprim}
\end{equation}
where
\begin{equation}
\xi = \frac{\IM A_0}{\RE A_0} \, , \quad
\omega = \frac{\RE A_2}{\RE A_0} \, , \quad
\Omega = \frac{1}{\omega} \frac{\IM A_2}{\IM A_0}
\label{eq:xiomega}
\end{equation}
and $\Phi \approx \pi/4$. Let us immediately emphasize the most 
important features
of various terms in (\ref{eq:epsprim}):
\bi
\item
$\IM A_0$ is dominated by QCD penguins and is very weakly dependent
on $\mt$. 
\item
$\IM A_2$ increases strongly with $\mt$
and for large $\mt$ is dominated by electroweak penguins. It receives
also a sizable contribution from isospin braking $(m_u\not=m_d)$ which
conspires with electroweak penguins to cancel substantially the
QCD penguin contribution in $\IM A_0$. 
The factor $1/\omega\approx 22$
in $\Omega$ giving a large enhancement is to a large extend responsible
for this cancellation.
\ei

When using (\ref{eq:epsprim}) and (\ref{eq:xiomega}) in phenomenological
applications one usually takes $\RE A_0$ and $\omega$ from
experiment, i.e.
\begin{equation}
\RE A_0 = 3.33 \cdot 10^{-7}\gev,
\qquad
\RE A_2 = 1.50 \cdot 10^{-8}\gev,
\qquad
\omega = 0.045,
\label{eq:ReA0data}
\end{equation}
where the last relation reflects the so-called $\Delta I=1/2$ rule. The
main reason for this strategy is the unpleasant fact that until today
nobody succeded in fully explaining this rule which to a large extent is
believed to originate in the long-distance QCD contributions
\cite{DI12}. 
On the other hand the
imaginary parts of the amplitudes in (\ref{eq:xiomega}) being related to
CP violation and the top quark physics should be dominated by
short-distance contributions. Therefore $\IM A_0$ and $\IM A_2$ are
usually calculated using the effective Hamiltonian for $\Delta S=1$
transitions:
\begin{equation}
\Heff(\Delta S=1) = 
\frac{G_{\rm F}}{\sqrt{2}} V_{us}^* V_{ud}^{} \sum_{i=1}^{10}
\left( z_i(\mu) + \tau \; y_i(\mu) \right) Q_i(\mu) 
\label{eq:HeffdF1:1010}
\end{equation}
with $\tau=-V_{ts}^* V_{td}^{}/(V_{us}^* V_{ud}^{})$.

The operators $Q_i$ are the analogues of the ones given in 
(\ref{O1})-(\ref{O3}) and (\ref{O4})-(\ref{O6}).
They are given explicitly  as follows:

{\bf Current--Current :}
\begin{equation}\label{OS1} 
Q_1 = (\bar s_{\alpha} u_{\beta})_{V-A}\;(\bar u_{\beta} d_{\alpha})_{V-A}
~~~~~~Q_2 = (\bar s u)_{V-A}\;(\bar u d)_{V-A} 
\end{equation}

{\bf QCD--Penguins :}
\begin{equation}\label{OS2}
Q_3 = (\bar s d)_{V-A}\sum_{q=u,d,s}(\bar qq)_{V-A}~~~~~~   
 Q_4 = (\bar s_{\alpha} d_{\beta})_{V-A}\sum_{q=u,d,s}(\bar q_{\beta} 
       q_{\alpha})_{V-A} 
\end{equation}
\begin{equation}\label{OS3}
 Q_5 = (\bar s d)_{V-A} \sum_{q=u,d,s}(\bar qq)_{V+A}~~~~~  
 Q_6 = (\bar s_{\alpha} d_{\beta})_{V-A}\sum_{q=u,d,s}
       (\bar q_{\beta} q_{\alpha})_{V+A} 
\end{equation}

{\bf Electroweak--Penguins :}
\begin{equation}\label{OS4} 
Q_7 = {3\over 2}\;(\bar s d)_{V-A}\sum_{q=u,d,s}e_q\;(\bar qq)_{V+A} 
~~~~~ Q_8 = {3\over2}\;(\bar s_{\alpha} d_{\beta})_{V-A}\sum_{q=u,d,s}e_q
        (\bar q_{\beta} q_{\alpha})_{V+A}
\end{equation}
\begin{equation}\label{OS5} 
 Q_9 = {3\over 2}\;(\bar s d)_{V-A}\sum_{q=u,d,s}e_q(\bar q q)_{V-A}
~~~~~Q_{10} ={3\over 2}\;
(\bar s_{\alpha} d_{\beta})_{V-A}\sum_{q=u,d,s}e_q\;
       (\bar q_{\beta}q_{\alpha})_{V-A} \,.
\end{equation}
Here, $e_q$ denotes the electrical quark charges reflecting the
electroweak origin of $Q_7,\ldots,Q_{10}$. 

The Wilson coefficient functions $z_i(\mu)$ and $ y_i(\mu)$
were calculated including
the complete next-to-leading order (NLO) corrections in
\cite{BJLW1,BJLW2,BJLW,ROMA1,ROMA2}. The details
of these calculations can be found there and in the review
\cite{BBL}. Only the coefficients $ y_i(\mu)$ enter the evaluation
of $\epe$. Examples of their numerical values are given in table 
\ref{tab:wc10smu13}.
Extensive tables for $ y_i(\mu)$ can be found in \cite{BBL}.

\begin{table}[htb]
\caption[]{$\Delta S=1 $ Wilson coefficients at $\mu=\mc=1.3\gev$ for
$\mt=170\gev$ and $f=3$ effective flavours.
$|z_3|,\ldots,|z_{10}|$ are numerically irrelevant relative to
$|z_{1,2}|$. $y_1 = y_2 \equiv 0$.
\label{tab:wc10smu13}}
\begin{center}
\begin{tabular}{|c|c|c|c||c|c|c||c|c|c|}
\hline
& \multicolumn{3}{c||}{$\Lms^{(4)}=245\mev$} &
  \multicolumn{3}{c||}{$\Lms^{(4)}=325\mev$} &
  \multicolumn{3}{c| }{$\Lms^{(4)}=405\mev$} \\
\hline
Scheme & LO & NDR & HV & LO & 
NDR & HV & LO & NDR & HV \\
\hline
$z_1$ & -0.550 & -0.364 & -0.438 & -0.625 & 
-0.415 & -0.507 & -0.702 & -0.469 & -0.585 \\
$z_2$ & 1.294 & 1.184 & 1.230 & 1.345 & 
1.216 & 1.276 & 1.399 & 1.251 & 1.331 \\
\hline
$y_3$ & 0.029 & 0.024 & 0.027 & 0.034 & 
0.029 & 0.033 & 0.039 & 0.034 & 0.039 \\
$y_4$ & -0.054 & -0.050 & -0.052 & -0.061 & 
-0.057 & -0.060 & -0.068 & -0.065 & -0.068 \\
$y_5$ & 0.014 & 0.007 & 0.014 & 0.015 & 
0.005 & 0.016 & 0.016 & 0.002 & 0.018 \\
$y_6$ & -0.081 & -0.073 & -0.067 & -0.096 & 
-0.089 & -0.081 & -0.113 & -0.109 & -0.097 \\
\hline
$y_7/\aem$ & 0.032 & -0.031 & -0.030 & 0.039 & 
-0.030 & -0.028 & 0.045 & -0.029 & -0.026 \\
$y_8/\aem$ & 0.100 & 0.111 & 0.120 & 0.121 & 
0.136 & 0.145 & 0.145 & 0.166 & 0.176 \\
$y_9/\aem$ & -1.445 & -1.437 & -1.437 & -1.490 & 
-1.479 & -1.479 & -1.539 & -1.528 & -1.528 \\
$y_{10}/\aem$ & 0.588 & 0.477 & 0.482 & 0.668 & 
0.547 & 0.553 & 0.749 & 0.624 & 0.632 \\
\hline
\end{tabular}
\end{center}
\end{table}

Using the Hamiltonian in (\ref{eq:HeffdF1:1010}) and the experimental
values for $\varepsilon$, $\RE A_0$ and $\omega$ the ratio $\epe$ can be
written as follows:
\begin{equation}
\frac{\varepsilon'}{\varepsilon} = 
\IM \lambda_t\cdot \left[ P^{(1/2)} - P^{(3/2)} \right],
\label{eq:epe}
\end{equation}
where
\begin{eqnarray}
P^{(1/2)} & = & r \sum y_i \langle Q_i\rangle_0
(1-\Omega_{\eta+\eta'})~,
\label{eq:P12} \\
P^{(3/2)} & = &\frac{r}{\omega}
\sum y_i \langle Q_i\rangle_2~,~~~~~~
\label{eq:P32}
\end{eqnarray}
with
\begin{equation}
r = \frac{G_{\rm F} \omega}{2 |\eps| \RE A_0}~, 
\qquad
\langle Q_i\rangle_I \equiv \langle (\pi\pi)_I | Q_i | K \rangle .
\label{eq:repe}
\end{equation}
One should note that the overall strong phases in $\varepsilon'$ and 
$\varepsilon$ cancel
in the ratio to an excellent approximation.
The sum in (\ref{eq:P12}) and (\ref{eq:P32}) runs over all contributing
operators. $P^{(3/2)}$ is fully dominated by electroweak penguin
contributions. $P^{(1/2)}$ on the other hand is governed by QCD penguin
contributions which are suppressed by isospin breaking in the quark
masses ($m_u \not= m_d$). The latter effect is described by

\begin{equation}
\Omega_{\eta+\eta'} = \frac{1}{\omega} \frac{(\IM A_2)_{\rm
I.B.}}{\IM A_0}\,.
\label{eq:Omegaeta}
\end{equation}
For $\Omega_{\eta+\eta'}$ we will take
\begin{equation}
\Omega_{\eta+\eta'} = 0.25 \pm 0.05\,,
\label{eq:Omegaetadata}
\end{equation}
which is in the ball park of the values obtained in the $1/N$ approach
\cite{burasgerard:87} and in chiral perturbation theory
\cite{donoghueetal:86,lusignoli:89}. $\Omega_{\eta+\eta'}$ is
independent of $\mt$.

The main source of uncertainty in the calculation of
$\epe$ are the hadronic matrix elements $\langle Q_i \rangle_I$.
They depend generally
on the renormalization scale $\mu$ and on the scheme used to
renormalize the operators $Q_i$. These two dependences are canceled by
those present in the Wilson coefficients $y_i(\mu)$ so that the
resulting physical $\epe$ does not (in principle) depend on $\mu$ and on the
renormalization scheme of the operators.  Unfortunately the accuracy of
the present non-perturbative methods used to evalutate $\langle Q_i
\rangle_I$, like lattice methods, the $1/N$ expansion, chiral
quark models and
chiral effective lagrangians, is not
sufficient to obtain the required $\mu$ and scheme dependences of
$\langle Q_i \rangle_I$. A brief review of the existing methods 
including most recent developments will be given below.

In view of this situation it has been suggested in \cite{BJLW} to
determine as many matrix elements $\langle Q_i \rangle_I$ as possible
from the leading CP conserving $K \to \pi\pi$ decays, for which the
experimental data are summarized in (\ref{eq:ReA0data}). To this end it
turned out to be very convenient to determine $\langle Q_i \rangle_I$
at the scale $\mu = \mc$.  Using the renormalization group evolution one
can then find $\langle Q_i \rangle_I$ at any other scale $\mu \not=
\mc$. The details of this procedure can be found in
\cite{BJLW}. We will briefly summarize the most important results of this
work below.

\subsection{Hadronic Matrix Elements}
\subsubsection{Preliminaries}
It is customary to express the matrix elements
$\langle Q_i \rangle_I$ in terms of non-perturbative parameters
$B_i^{(1/2)}$ and $B_i^{(3/2)}$ as follows:
\begin{equation}
\langle Q_i \rangle_0 \equiv B_i^{(1/2)} \, \langle Q_i
\rangle_0^{\rm (vac)}\,,
\qquad
\langle Q_i\rangle_2 \equiv B_i^{(3/2)} \, \langle Q_i
\rangle_2^{\rm (vac)} \,.
\label{eq:1}
\end{equation}
The label ``vac'' stands for the vacuum
insertion estimate of the hadronic matrix elements in question. 
The full list of $\langle Q_i\rangle_I$ is given in \cite{BJLW}.
 It suffices to give here only a few examples:
\begin{eqnarray}
\langle Q_1 \rangle_0 &=& -\,\frac{1}{9} X B_1^{(1/2)} \, ,
\label{eq:Q10} \\
\langle Q_2 \rangle_0 &=&  \frac{5}{9} X B_2^{(1/2)} \, ,
\label{eq:Q20} \\
\langle Q_6 \rangle_0 &=&  -\,4 \sqrt{\frac{3}{2}} 
\left[ \frac{m_{\rm K}^2}{\ms(\mu) + \md(\mu)}\right]^2
\frac{F_\pi}{\kappa} \,B_6^{(1/2)} \, ,
\label{eq:Q60}\\ 
\langle Q_1 \rangle_2 &=& 
\langle Q_2 \rangle_2 = \frac{4 \sqrt{2}}{9} X B_1^{(3/2)} \, ,
\label{eq:Q122} \\
\langle Q_i \rangle_2 &=&  0 \, , \qquad i=3,\ldots,6 \, ,
\label{eq:Q362} \\
\langle Q_8 \rangle_2 &=& 
  -\left[ \frac{\kappa}{2 \sqrt{2}} \langle \overline{Q_6} \rangle_0
          + \frac{\sqrt{2}}{6} X
   \right] B_8^{(3/2)} \, ,
\label{eq:Q82} \\
\langle Q_9 \rangle_2 &=& 
   \langle Q_{10} \rangle_2 = \frac{3}{2} \langle Q_1 \rangle_2 \, ,
\label{eq:Q9102}
\end{eqnarray}
where
\begin{equation}
\kappa = 
         \frac{F_\pi}{F_{\rm K} - F_\pi} \, ,
\qquad
X = \sqrt{\frac{3}{2}} F_\pi \left( m_{\rm K}^2 - m_\pi^2 \right) \, ,
\label{eq:XQi}
\end{equation}
and
\begin{equation}
\langle \overline{Q_6} \rangle_0 =
   \frac{\langle Q_6 \rangle_0}{B_6^{(1/2)}} \, .
\label{eq:Q60bar}
\end{equation}
In the vacuum insertion method $B_i=1$ independent of $\mu$. In QCD,
however, the hadronic parameters $B_i$ generally depend on the
renormalization scale $\mu$ and the renormalization scheme considered.
\subsubsection{$(V-A) \otimes (V-A)$ Operators}
Let us now extract some matrix from the data on $\RE A_0$ and $\RE A_2$
in (\ref{eq:ReA0data}). To this end we follow \cite{BJLW}.
One notes first that in view of the smallness of 
$\tau=\ord(10^{-4})$ entering
(\ref{eq:HeffdF1:1010}), the
real amplitudes in (\ref{eq:ReA0data}) are governed by the coefficients
$z_i(\mu)$. The method of extracting some of the matrix elements
from the data as proposed in \cite{BJLW} relies then on the
fact that due to the GIM mechanism for $ \mu\ge m_c$ 
the coefficients $z_i(\mu)$ of
the penguin operators (i=3....10) vanish at the matching scale $\mu_c$
(between the four-quark and three-quark effective theories) in the HV
scheme and are negligible in the NDR scheme. 
However, it should be remembered that the smallness or even vanishing
of $z_i(\mu)$ for $ \mu\ge m_c$ is characteristic for mass independent
renormalization schemes. In other schemes, in which the disparity of
$m_u$ and $m_c$ is felt well above $\mu=m_c$, the GIM cancellation is
incomplete and $z_i(m_c)$ for penguin operators are larger than in the
HV and NDR schemes. Examples of the leading order calculations
of this type can be found in \cite{ANII}.   

Staying within the NDR and HV schemes, we can however set $z_i(m_c)=0$
for $i\not=1,2$ 
to find
\begin{equation}
\langle Q_1(m_c) \rangle_2 = \langle Q_2(m_c) \rangle_2 =
\frac{10^6\gev^2}{1.77} \frac{\RE A_2}{z_+(m_c)} =
\frac{8.47 \cdot 10^{-3}\gev^3}{z_+(m_c)}
\label{eq:Q122data}
\end{equation}
with $z_+=z_1+z_2$ and
\begin{equation}
\langle Q_1(\mc) \rangle_0 = \frac{10^6\gev^2}{1.77} \frac{\RE
A_0}{z_1(\mc)} - \frac{z_2(\mc)}{z_1(\mc)} \langle Q_2(\mc) \rangle_0\,.
\label{eq:Q10mc}
\end{equation}
These formulae are easy to derive and are left as a useful homework
problem.

Comparing next (\ref{eq:Q122data})  with (\ref{eq:Q122}) one 
finds immediately
\begin{equation}
B_1^{(3/2)}(m_c) = \frac{0.363}{z_+(m_c)}\,,
\label{eq:B321}
\end{equation}
which using table \ref{tab:wc10smu13} gives for $\mc=1.3\gev$ and 
$\Lms^{(4)}=325\mev$
\begin{equation}
B_{1,NDR}^{(3/2)}(\mc) =  0.453\,,
\qquad
B_{1,HV}^{(3/2)}(\mc) =  0.472 \, .
\label{eq:B321mc}
\end{equation}
The extracted values for $B_1^{(3/2)}$ are by more than a factor of two
smaller than the vacuum insertion estimate.
They are compatible with the $1/N_c$ value $B_1^{(3/2)}(1\gev) \approx
0.55$ \cite{bardeen:87} and are somewhat smaller than the lattice result
$B_1^{(3/2)}(2\gev) \approx 0.6$ \cite{ciuchini:95}.
As analyzed in \cite{BJLW},
$B_1^{(3/2)}(\mu)$ decreases slowly with increasing $\mu$.
As seen in (\ref{eq:Q9102}), this analysis gives also
$\langle Q_9(\mc) \rangle_2$ and $\langle Q_{10}(\mc) \rangle_2$.

In order to extract $B_1^{(1/2)}(\mc)$ and $B_2^{(1/2)}(\mc)$ from
(\ref{eq:Q10mc})
one can make the very plausible assumption,
valid in known non-perturbative approaches, that
 $\langle Q_-(\mc) \rangle_0 \ge
\langle Q_+(\mc) \rangle_0 \ge 0$, where $Q_\pm=(Q_2\pm Q_1)/2$.
This gives  for $\Lms^{(4)}=325\mev$
\begin{equation}
B_{2,NDR}^{(1/2)}(\mc) =  6.6 \pm 1.0,
\qquad
B_{2,HV}^{(1/2)}(\mc) =  6.2 \pm 1.0 \, .
\label{eq:B122mc}
\end{equation}
The extraction of $B_1^{(1/2)}(\mc)$ and of analogous parameters
$B_{3,4}^{(1/2)}(\mc)$ are presented in detail in \cite{BJLW}.
$B_1^{(1/2)}(\mc)$ depends very sensitively on $B_2^{(1/2)}(\mc)$ and
its central value is as high as 15. $B_4^{(1/2)}(\mc)$ is 
typically by (10--15)\,\% lower than $B_2^{(1/2)}(\mc)$. In any case
this analysis shows very large deviations from the results of the
vacuum insertion method.

\subsubsection{$(V-A) \otimes (V+A)$ Operators}
The matrix elements of the $(V-A) \otimes (V+A)$ operators $Q_5$--$Q_8$
cannot be constrained by CP conserving data and one has to rely on
existing non-perturbative methods to calculate them. 
This is rather unfortunate because the QCD penguin operator $Q_6$
and the electroweak penguin operator $Q_8$, having large Wilson
coefficients and large hadronic matrix elements, play the dominant
role in $\epe$. 

We will now review the present status of $B_i$ factors describing
the matrix elements of $Q_5-Q_8$ operators as obtained in various
non-perturbative approaches. We will pay particular attention to
the parameters $B^{(1/2)}_{6}$ and $B^{(3/2)}_{8}$ which are most
important for the evaluation of $\epe$. We recall that $B_i=1$
in the vacuum insertion method.

\subsubsection{$B^{(1/2)}_{6}$ and $B^{(3/2)}_{8}$ from Lattice}
We begin with lattice calculations. These have been reviewed
recently by Gupta \cite{GUPTA98} and the APE collaboration \cite{APE}. 
The most reliable
results are found for $B^{(3/2)}_{7,8}$. The ``modern" quenched
estimates for these parameters, which supercede all previously
reported values are collected in table \ref{tab:317}, which has been
taken from Gupta and quenched a bit. The first three calculations
use perturbative matching between lattice and continuum, the last
one uses non-perturbative matching. Since all three groups agree
within perturbative matching and the non-perturbative matching
should be preferred, I conclude (probably naively) that the best
quenched lattice values are
\be\label{LAT}
(B^{(3/2)}_{7})_{\rm lattice}(2~\gev)=0.72\pm 0.05,
\quad\quad
(B^{(3/2)}_{8})_{\rm lattice}(2~\gev)=1.03\pm 0.03
\ee
where the errors are purely statistical. Concerning the
lattice results for $B^{(1/2)}_{5,6}$ the situation is
 worse. The old results read
$B^{(1/2)}_{5,6}(2~\gev)=1.0 \pm 0.2$ \cite{kilcup:91,sharpe:91}.
More accurate estimates for $B^{(1/2)}_{6}$ have been recently
obtained in \cite{kilcup:98}: 
$B^{(1/2)}_{6}(2~\gev)=0.67 \pm 0.04\pm 0.05$
(quenched) and $B^{(1/2)}_{6}(2~\gev)=0.76 \pm 0.03\pm0.05$
($f=2$). However, as stressed by Gupta, the systematic
errors in this analysis are not really under control.
We have to conclude, that there are no solid predictions for
$B^{(1/2)}_{5,6}$ from the lattice at present.
\begin{table}[thb]
\caption[]{ Lattice results for $B^{(3/2)}_{7,8} (2~\gev)$ obtained
by various groups. 
\label{tab:317}}
\begin{center}
\begin{tabular}{|c|c|c|c|}\hline
  { Fermion type}& $B^{(3/2)}_7$& $B^{(3/2)}_8$ & Matching \\
 \hline
Staggered\cite{GKS}& $0.62(3)(6)$ &$0.77(4)(4)$ & 1-loop \\
Wilson\cite{G67}& $0.58(2)(7)$ &$0.81(3)(3)$ & 1-loop \\
Clover\cite{APE}& $0.58(2)$ &$0.83(2)$ & 1-loop \\
Clover\cite{APE}& $0.72(5)$ &$1.03(3)$ &  Non-pert. \\
\hline
\end{tabular}
\end{center}
\end{table}
\subsubsection{$B^{(1/2)}_{6}$ and $B^{(3/2)}_{8}$ from the 1/N Approach}
The 1/N approach to weak hadronic matrix elements was introduced
in \cite{bardeen:87}. 
In this approach the 1/N expansion becomes a loop expansion
in an effective meson theory. In the strict large N limit only
the tree level matrix elements of $Q_6$ and $Q_8$ contribute
and one finds (\ref{eq:Q60}) and (\ref{eq:Q82}) with
\be\label{LN}
B^{(1/2)}_{6}=B^{(3/2)}_{8}=1, \quad\quad{\rm (Large-N~Limit)} 
\ee
while $B^{(1/2)}_{5}=B^{(3/2)}_{8}=0$. The latter fact is not disturbing,
however, as the operators $Q_5$ and $Q_7$, having small Wilson
coefficients are immaterial for $\epe$.

Now, $B^{(1/2)}_{6}$ and $B^{(3/2)}_{8}$ as given in (\ref{LN}) are
clearly $\mu$-independent. At first sight this appears as a problem.
But in fact it is not! The point is that $Q_{6,8}$ are
density$\times$density operators as one can see by writing them
with the help of the Fierz reordering as follows
\be\label{DENSITY}
Q_6=-2 \sum_{q=u,d,s} \bar s(1+\gamma_5)q\bar q (1-\gamma_5)d,
\quad\quad
Q_8=-3 \sum_{q=u,d,s}e_q \bar s(1+\gamma_5)q\bar q (1-\gamma_5)d.
\ee
Consequently their $\mu$-dependences are related to 
the $\mu$-dependence of the quark masses and the tree level 
factorizable contributions to
$\langle Q_6\rangle_0$ and $\langle Q_8\rangle_2$ are
$\mu$-dependent through the factor $1/\ms^2(\mu)$ as seen in
(\ref{eq:Q60}) and (\ref{eq:Q82}). This should be contrasted with
the matrix elements of $(V-A)\otimes(V-A)$ operators, which are
$\mu$-independent in the large-N limit. The $\mu$-dependence
of $1/\ms^2(\mu)$ in $\langle Q_6\rangle_0$ and $\langle Q_8\rangle_2$
is exactly cancelled in the decay amplitude by the diagonal
evolution (no mixing) of the Wilson coefficients $y_6(\mu)$ and
$y_8(\mu)$ taken in the large-N limit.

Indeed, the $\mu$-dependence of $1/\ms^2(\mu)$ is governed in LO
by $2\gamma_m^{(0)}=12 C_F$. On the other hand, the one-loop
anomalous dimensions of $Q_{6,8}$, which govern the diagonal
evolution of $y_{6,8}(\mu)$ are given by
\be
\gamma_{66}^{(0)}=-2 \gamma_m^{(0)}+\frac{2f}{3},
\quad\quad
\gamma_{88}^{(0)}=-2 \gamma_m^{(0)}.
\ee
Since for large N, $\gamma_m^{(0)}\sim\ord(N)$, we find indeed
$\gamma_{66}^{(0)}= \gamma_{88}^{(0)}=-2 \gamma_m^{(0)}$
in the large-N limit \cite{burasgerard:87}. 
Going back to the respective evolutions
of $\ms(\mu)$ and $y_{7,8}(\mu)$ we indeed confirm the cancellation
of the $\mu$-dependence in question. This feature is preserved
at the two-loop level as discussed in \cite{BJLW1}. 
One can go even further
and demonstrate numerically for $N=3$ that the parameters
$B^{1/2}_{6}$ and $B^{3/2}_{8}$ depend only very weakly on $\mu$,
when $\mu\ge 1~\gev$. In such a numerical
renormalization study in \cite{BJLW} the
factors $B_{6}^{(1/2)}$ and $B_{8}^{(3/2)}$  have been set to unity 
at $\mu=\mc$.
Subsequently the evolution of the matrix elements in the range $1\gev
\le \mu \le 4\gev$ has been calculated showing that for the NDR scheme
$B_{5,6}^{(1/2)}$ and $B_{7,8}^{(3/2)}$ were $\mu$ independent within
an accuracy of (2--3)\,\%. The $\mu$ dependence in the HV scheme has
been found to be stronger but still below 10\,\%.

In view of the fact that for $B^{(1/2)}_{6}=B^{(3/2)}_{8}=1$ and the
known value of $\mt$, there is a strong cancellation between
gluon and electroweak penguin contributions to $\epe$, it is
important to investigate whether the $1/N$ corrections significantly
affect this cancellation. First attempt in this direction has been
made by the Dortmund group \cite{heinrichetal:92, paschos:96}, 
which incorporating in part chiral
loops found an enhancement of $B_{6}^{(1/2)}$ and a suppression
of $B_{8}^{(3/2)}$. From \cite{paschos:96} $B^{(1/2)}_6=1.3$ and 
$B^{(3/2)}_8=0.7$ can be extracted.
 
Recently another Dortmund team \cite{DORT98}, in collaboration with Bill
Bardeen, performed this time a complete investigation
of $\langle Q_6\rangle_0$ and $\langle Q_8\rangle_2$ in the
twofold expansion in powers of external momenta $p$, and in 
$1/N$. Their final result gives $\langle Q_6\rangle_0$ and 
$\langle Q_8\rangle_2$ including the orders $p^2$ and $p^0/N$.
For $\langle Q_8\rangle_2$ also the term $p^0$ contributes.
Of particular interest are the $\ord(p^0/N)$ contributions
resulting from non-factorizable chiral loops which are
important for the matching between long- and short-distance
contributions. The cut-off scale $\Lambda_c$ in these
non-factorizable diagrams is identified with the QCD renormalization
scale $\mu$ which enters the Wilson coefficients. In contrast
to the matrix elements of $Q_{1,2}$ in which the $\Lambda_c$
dependence was quadratic \cite{bardeen:87}, 
the $\Lambda_c$ dependence in the present
case is logarithmic which improves the matching considerably.
There are several technical and conceptual improvements in \cite{DORT98}
over the first attempt in \cite{heinrichetal:92, paschos:96} 
and also over the original approach \cite{bardeen:87}.
Therefore I strongly recommend to read \cite{DORT98}, which is clearly
written.

\begin{table}[thb]
\caption[]{ Results for $B^{(3/2)}_{7,8}$ obtained
in the $1/N$ approach. 
\label{tab:318}}
\begin{center}
\begin{tabular}{|c||c|c|c|c|}\hline
  & $\Lambda_c=0.6~\gev$ & $\Lambda_c=0.7~\gev$ & 
$\Lambda_c=0.8~\gev$ & $\Lambda_c=0.9~\gev$ \\
 \hline\hline
$B_{6}^{(1/2)}$ & $1.10$ &$0.96$ & $0.84$ & $ 0.72 $ \\
                & $(1.30)$ &$(1.19)$ & $(1.09)$ & $(0.99) $ \\
$B_{8}^{(3/2)}$ & $0.66$ &$0.59$ & $0.52$ & $ 0.46 $ \\
                & $(0.71)$ &$(0.65)$ & $(0.60)$ & $(0.54) $ \\
\hline
\end{tabular}
\end{center}
\end{table}
In table \ref{tab:318}, taken from \cite{DORT98}, we show the values of
$B_{6}^{(1/2)}$ and $B_{8}^{(3/2)}$ as functions of the cut-off
scale $\Lambda_c$. The results depend on whether $F_\pi$ or
$F_K$ is used in the calculation, the difference being of higher
order. The results using $F_K$ are shown in the parentheses.
The decrease of both B--factors with $\Lambda_c=\mu$ is qualitatively
consistent with their $\mu$-dependence found for $\mu\ge 1$ in
\cite{BJLW}, but it is much stronger. Clearly one could also
expect stronger $\mu$-dependence in the analysis of \cite{BJLW}
for $\mu\le 1~\gev$, but in view of large perturbative corrections
for such small scales a meaningfull test of the dependence in
table \ref{tab:318} cannot be made.
We note that for $\Lambda_c=0.7~\gev$  the value of $B_{6}^{(1/2)}$
is close to unity as in the large-N limit. However, $B_{8}^{(3/2)}$
is considerably suppressed.

It is difficult to decide which value should be used in phenomenology
of $\epe$. On the one hand, for $\Lambda_c\ge 0.6~\gev$ neglected 
contributions from vector mesons in the loops should be included.
On the other hand for $\Lambda_c=\mu= 0.6~\gev$ it is difficult
to make contact with short distance calculations and with the
lattice results which are obtained for $\mu=2~\gev$. As for
$\mu \ge 1~\gev$ the parameters in table \ref{tab:318} are expected to be
almost $\mu$-independent, let us take the values at $\Lambda_c=0.9~\gev$
as the main result of \cite{DORT98}. 
Averaging over the $F_\pi$- and $F_K$-choices
we find
\be\label{LNN}
B^{(1/2)}_{6}=0.85\pm0.13~, \quad\quad B^{(3/2)}_{8}=0.50\pm 0.04~,
\quad\quad(\Lambda_c=0.9~\gev)
\ee
where the errors should not be taken too seriously. The value
of $B^{(1/2)}_{6}$ is compatible with the corresponding lattice 
results, whereas $ B^{(3/2)}_{8}$ is found to be substantially
smaller in the $1/N$ approach. On the other hand, it will be
tempting later on to calculate $\epe$ for the choice 
$\Lambda_c=0.6~\gev$ which gives instead:
\be\label{LNN1}
B^{(1/2)}_{6}=1.2\pm0.1~, \quad\quad B^{(3/2)}_{8}=0.68\pm 0.03~,
\quad\quad (\Lambda_c=0.6~\gev).
\ee
In view of the large correction to
$B^{(3/2)}_{8}$ one might question the convergence of the $1/N$
expansion. However, the non-factorizable contributions
considered in \cite{DORT98} represent the first term in a new type 
of a series
absent in the large-N limit and consequently there are no strong
reasons for questioning the convergence of the $1/N$ expansion on
the basis of these results. In this context one should also
remark that the lattice studies discussed previously use tree level
chiral perturbation theory to relate the matrix elements
$\langle \pi\pi| Q_i|K\rangle $ to $\langle\pi| Q_i|K\rangle $
which are calculated on the lattice. It is conceivable that
including chiral loops in this relation would decrease the
value of $ B^{(3/2)}_{8}$.

Finally I would like to express one criticism of the approach
in \cite{DORT98} as well as in 
\cite{bardeen:87}. It is the lack of any reference to the
renormalization scheme dependence which is necessary for
a complete matching at the NLO level.

\subsubsection{$B^{(1/2)}_{6}$ and $B^{(3/2)}_{8}$ from 
the Chiral Quark Model}
Effective Quark Models of QCD can be derived in the framework of 
the extended Nambu-Jona-Lasinio model of chiral symmetry 
breaking \cite{NJL}.
For kaon decays and in particular for $\epe$, an extensive analysis
of this model inclusive chiral loops, gluon and $\ord(p^4)$ corrections
has been performed over the last years by the Trieste group
\cite{TR96,TR97}. The
crucial parameters in this approach is a mass parameter $M$ and
the condensates $\langle\bar q q\rangle $ and $\langle\as GG\rangle$.
They can be constrained by imposing the $\Delta I=1/2$ rule.

Since there exists a recent nice review \cite{BERT98} by the Trieste 
group of their
approach, I will only quote here their estimate of the relevant 
$B_i$ parameters. They are
\be\label{TRIESTE}
B^{(1/2)}_{6}=1.6\pm0.3 \quad\quad B^{(3/2)}_{8}=0.92\pm 0.02~.
\quad\quad{(\rm Chiral~QM)}
\ee
We observe a rather large enhancement of $B^{(1/2)}_{6}$, not observed
by other groups, and a
moderate suppression of $B^{(3/2)}_{8}$. These parameters correspond
roughly to the scale $\mu=0.8~\gev$. Looking at the table
\ref{tab:318} we may expect a $10\%$ reduction of these values,
had the scale $\mu=0.9~\gev$ been used.

\subsubsection{Strategy for $(V-A) \otimes (V+A)$ Operators}
We have seen that various approaches differ in their estimates
of the most important parameters $B^{(1/2)}_{6}$ and $B^{(3/2)}_{8}$.
In table \ref{tab:31739} we collect the central values from various
approaches discussed above. In the case of lattice we have chosen
various possible scenarios in view of different results obtained
by various groups. Similarly in the case of the $1/N$ approach
we have chosen two sets of B-values corresponding to two values
of $\Lambda_c$. 
Even if
the $B_i$ factors given in this table are all within say $50\%$ from the
vacuum insertion estimate, they give rather different results
for central values of $\epe$ as illustrated in the last two columns
of this table. How these values have been obtained will be discussed 
a few pages below.
\begin{table}[thb]
\caption[]{ Results for $\epe$ in units of $10^{-4}$ 
for three choices of $\ms(\mc)$
and the central values of  $B^{(1/2)}_6$ and $B^{(3/2)}_8$ 
obtained in various approaches. $\IM\lambda_t=1.29\cdot 10^{-4}$
and $\mt=167\gev$ have been used.
\label{tab:31739}}
\begin{center}
\begin{tabular}{|c|c|c||c|c|c|}\hline
  Approach & $B^{(1/2)}_6$& $B^{(3/2)}_8$ & $150~\mev$& 
 $125~\mev$ & $100~\mev$ \\ \hline
  VIA    & $1.0$ &$1.0$ & $3.2$ & $5.2$ & $8.8$ \\
\hline
Lattice 1    & $1.0$ &$0.81$ & $4.2$ & $6.6$ & $10.9$  \\
Lattice 2    & $1.0$ &$1.03$ & $3.0$ & $5.0$ & $8.4$ \\
Lattice 3    & $0.76$ &$0.81$ & $1.7$ & $3.1$ & $5.7$ \\
Lattice 4    & $0.76$ &$1.03$ & $0.6$ & $1.5$ & $3.2$  \\
\hline
1/N (I) & $0.85$ &$0.50$ & $4.3$ & $6.7$ & $11.1$  \\
1/N (II) & $1.2$ &$0.68$ & $6.9$ & $10.4$ &$ 16.7$\\
\hline
Chiral QM & $1.6$ &$0.92$ & $9.7$ & $14.4$ & $22.7$  \\
\hline
\end{tabular}
\end{center}
\end{table}

Concerning $B_{7,8}^{(1/2)}$ one can simply set $B_{7,8}^{(1/2)}=1$ as
the matrix elementes $\langle Q_{7,8} \rangle_0$ play only a minor role
in the $\epe$ analysis. I should however stress that whereas lattice
results are consistent with this choice, this is not the case for
the chiral quark model \cite{TR97} in which values as high 
as 2.5 are found.

Concerning $B_5^{(1/2)}$ and $B_7^{(3/2)}$ we will simply
set them equal to $B^{(1/2)}_{6}$ and $B^{(3/2)}_{8}$ respectively.
This is consistent with the lattice results and the chiral
quark model. There are no results for these parameters from
the $1/N$ approach beyond the large-N limit.

In summary the treatment of $\langle Q_i \rangle_{0,2}$, $i=5,\ldots 8$
in \cite{BJLW,BBL,BJL96a} is to set
\begin{equation}
B_{7,8}^{(1/2)}(\mc) = 1,
\qquad
B_5^{(1/2)}(\mc) = B_6^{(1/2)}(\mc),
\qquad
B_7^{(3/2)}(\mc) = B_8^{(3/2)}(\mc)
\label{eq:B1278mc}
\end{equation}
and to treat $B_6^{(1/2)}(\mc)$ and $B_8^{(3/2)}(\mc)$ as free
parameters. In particular, in addition to estimates obtained
by other groups, we will
show below the results for $\epe$ when these parameters are varied 
in the ranges
\begin{equation}
B_6^{(1/2)}(\mc)=1.0 \pm 0.2,
\qquad
B_8^{(3/2)}(\mc)=1.0\pm 0.2.
\label{eq:B78mc}
\end{equation}
and
\begin{equation}
B_6^{(1/2)}(\mc)=1.0 \pm 0.2,
\qquad
B_8^{(3/2)}(\mc)=0.7\pm 0.2.
\label{eq:B87mc}
\end{equation}
The range (\ref{eq:B78mc}) corresponds to the variation of the $B_i$ parameters
in the neighbourhood of the large--N limit. The range (\ref{eq:B87mc})
gives a rough description of the fact that in recent analyses 
most approaches find $B_8^{(3/2)}$ to be smaller than $B_6^{(1/2)}$.
This range will be analyzed at the end of this section.

After this long exposition of $B_i$ parameters let us then
incorporate the collected information in the formula for
$\epe$ in a manner useful for phenomenological applications.

\subsection{An Analytic Formula for $\epe$}
           \label{subsec:epeanalytic}
As shown in \cite{buraslauten:93}, it is possible to cast the formal
expression for $\epe$ in (\ref{eq:epe})
into an analytic formula which exhibits the $\mt$ dependence
together with the dependence on $\ms$, $\Lms^{(4)}$,
 $B_6^{(1/2)}$ and $B_8^{(3/2)}$.
Such an analytic formula should be useful for those phenomenologists
and experimentalists who are not interested in getting involved with
the technicalities discussed above.

In order to find an analytic expression for $\epe$, which exactly
reproduces the numerical results based on the formal OPE method,
one uses the PBE presented in Section 3.3. The updated  
analytic formula for $\epe$ of \cite{buraslauten:93} 
presented in \cite{BJL96a}
is given as follows:
\begin{equation}
\frac{\varepsilon'}{\varepsilon} = {\rm Im}\lambda_t \cdot F(x_t) \, ,
\label{eq:3}
\end{equation}
where
\begin{equation}
F(x_t) =
P_0 + P_X \, X_0(x_t) + P_Y \, Y_0(x_t) + P_Z \, Z_0(x_t) 
+ P_E \, E_0(x_t) 
\label{eq:3b}
\end{equation}
and
\begin{equation}
{\rm Im}\lambda_t = {\rm Im} V_{ts}^*V_{td} = |V_{\rm ub}| \, 
|V_{\rm cb}| \, \sin \delta = \eta \, \lambda^5 \, A^2
\label{eq:4}
\end{equation}
in the standard parameterization of the CKM matrix
(\ref{2.72}) and in the Wolfenstein parameterization
(\ref{2.75}), respectively. 

The $\mt$-dependent functions in (\ref{eq:3b}) are given in 
(\ref{E0}) and (\ref{X0})--(\ref{Z0}).
The coefficients $P_i$ are given in terms of $B_6^{(1/2)} \equiv
B_6^{(1/2)}(\mc)$, $B_8^{(3/2)} \equiv B_8^{(3/2)}(\mc)$ and $\ms(\mc)$
as follows:
\begin{equation}
P_i = r_i^{(0)} + R_s 
\left(r_i^{(6)} B_6^{(1/2)} + r_i^{(8)} B_8^{(3/2)} \right) \, ,
\label{eq:pbePi}
\end{equation}
where
\be\label{RS}
R_s=\left[ \frac{158\mev}{\ms(\mc)+\md(\mc)} \right]^2.
\ee
The $P_i$ are renormalization scale and scheme independent. They depend,
however, on $\Lms^{(4)}$. In table~\ref{tab:pbendr} we give the numerical
values of $r_i^{(0)}$, $r_i^{(6)}$ and $r_i^{(8)}$ for different values
of $\Lms^{(4)}$ at $\mu=\mc$ in the NDR renormalization scheme. 
The
coefficients $r_i^{(0)}$, $r_i^{(6)}$ and $r_i^{(8)}$ depend only very
weakly on
$\ms(\mc)$ as the dominant $\ms$ dependence has been factored out. The
numbers given in table~\ref{tab:pbendr} correspond to $\ms(\mc)=150\,\mev$.
However, even for $\ms(\mc)\approx100\mev$, the analytic expressions given
here reproduce the numerical calculations of $\epe$ given below 
to better than $4\%$.
For different scales $\mu$ the numerical values in the tables change
without modifying the values of the $P_i$'s as it should be. The values
of $B_6^{(1/2)}$ and $B_8^{(3/2)}$ should also be  modified, 
in principle, but in view of the comments made previously it
is a good approximation to keep them $\mu$-independent
for $\mu\ge 1~\gev$.

Concerning the scheme dependence only the $r_0$ coefficients
are scheme dependent at the NLO level. Their values in the HV
scheme are given in the last row of table~\ref{tab:pbendr}.
The coefficients $r_i$, 
$i=X, Y, Z, E$ are on the other hand scheme independent at NLO. 
This is related to the fact that the $\mt$
dependence in $\epe$ enters first at the NLO level and consequently all
coefficients $r_i$ in front of the $\mt$ dependent functions must be
scheme independent. 
Consequently, when changing the renormalization scheme, one is only
obliged to change appropriately $B_6^{(1/2)}$ and $B_8^{(3/2)}$ in the
formula for $P_0$ in order to obtain a scheme independence of $\epe$.
In calculating $P_i$ where $i \not= 0$, $B_6^{(1/2)}$ and $B_8^{(3/2)}$
can in fact remain unchanged, because their variation in this part
corresponds to higher order contributions to $\epe$ which would have to
be taken into account in the next order of perturbation theory.

For similar reasons the NLO analysis of $\epe$ is still insensitive to
the precise definition of $\mt$. In view of the fact that the NLO
calculations needed to extract $\IM \lambda_t$ (see previous section) 
have been done with $\mt=\overline{m}_t(\mt)$ we will also use  this 
definition in calculating $F(x_t)$. 

\begin{table}[thb]
\caption[]{PBE coefficients for $\epe$ for various $\Lms^{(4)}$ in 
the NDR scheme.
The last row gives the $r_0$ coefficients in the HV scheme.
\label{tab:pbendr}}
\begin{center}
\begin{tabular}{|c||c|c|c||c|c|c||c|c|c|}
\hline
& \multicolumn{3}{c||}{$\Lms^{(4)}=245\mev$} &
  \multicolumn{3}{c||}{$\Lms^{(4)}=325\mev$} &
  \multicolumn{3}{c| }{$\Lms^{(4)}=405\mev$} \\
\hline
$i$ & $r_i^{(0)}$ & $r_i^{(6)}$ & $r_i^{(8)}$ &
      $r_i^{(0)}$ & $r_i^{(6)}$ & $r_i^{(8)}$ &
      $r_i^{(0)}$ & $r_i^{(6)}$ & $r_i^{(8)}$ \\
\hline
0 &
   --2.674 &   6.537 &   1.111 &
   --2.747 &   8.043 &   0.933 &
   --2.814 &   9.929 &   0.710 \\
$X$ &
    0.541 &   0.011 &       0 &
    0.517 &   0.015 &       0 &
    0.498 &   0.019 &       0 \\
$Y$ &
    0.408 &   0.049 &       0 &
    0.383 &   0.058 &       0 &
    0.361 &   0.068 &       0 \\
$Z$ &
    0.178 &  --0.009 &  --6.468 &
    0.244 &  --0.011 &  --7.402 &
    0.320 &  --0.013 &  --8.525 \\
$E$ &
    0.197 &  --0.790 &   0.278 &
    0.176 &  --0.917 &   0.335 &
    0.154 &  --1.063 &   0.402 \\
\hline
0 &
   --2.658 &   5.818 &   0.839 &
   --2.729 &   6.998 &   0.639 &
   --2.795 &   8.415 &   0.398 \\
\hline
\end{tabular}
\end{center}
\end{table}

The analytic formulae given above are useful for numerical
calculations, but in order to identify the dominant terms in an
elegant manner,
we follow Gupta \cite{GUPTA98} and rewrite it as
\be\label{GUPT}
\frac{\varepsilon'}{\varepsilon} = {\rm Im}\lambda_t \cdot 
\left[c_0+(c_6 B_6^{(1/2)} + c_8 B_8^{(3/2)}) R_s \right].
\end{equation}
For $\mt=167~\gev$ the values of the coefficients $c_i$ are given in
table \ref{cen}.

\begin{table}[thb]
\caption[]{The coefficients $c_i$ for various $\Lms^{(4)}$ in 
the NDR and HV schemes and $\mt=167~\gev$.
\label{cen}}
\begin{center}
\begin{tabular}{|c||c|c||c|c||c|c|}
\hline
& \multicolumn{2}{c||}{$\Lms^{(4)}=245\mev$} &
  \multicolumn{2}{c||}{$\Lms^{(4)}=325\mev$} &
  \multicolumn{2}{c| }{$\Lms^{(4)}=405\mev$} \\
\hline
${\rm Scheme}$ & ${\rm NDR}$ & ${\rm HV}$ &
${\rm NDR}$ & ${\rm HV}$ & ${\rm NDR}$ & ${\rm HV}$ \\
\hline
$c_0$ &
   --1.264 &   --1.248 &
   --1.359 &   --1.341 &
   --1.430 &   --1.411   \\
$c_6$ &
    6.387 &   5.668 &
    7.873 &   6.828 &
    9.735 &   8.221  \\
$c_8$ &
    --3.259 & --3.531 &
    --4.063 & --4.357 &
    --5.041 & --5.353  \\
\hline
\end{tabular}
\end{center}
\end{table}

The inspection of tables~\ref{tab:pbendr} and \ref{cen} shows
that within a few percent
\be
c_6=r_0^{(6)},\quad\quad c_8=r_0^{(8)}+ r_Z^{(8)} Z_0(x_t),
\ee 
whereby $c_8$ is dominated by the second term.
Thus we conclude 
that the terms involving $r_0^{(6)}$ and $r_Z^{(8)}$ dominate the ratio
$\epe$. Moreover, the function $Z_0(x_t)$ representing a gauge invariant
combination of $Z^0$- and $\gamma$-penguins grows rapidly with $\mt$
and due to $r_Z^{(8)} < 0$ these contributions suppress $\epe$ strongly
for large $\mt$ \cite{flynn:89,buchallaetal:90} as stressed at the
beginning of this section. 

\subsection{The Status of the Strange Quark Mass}
It seems appropriate to summarize now the present status of 
the value of the strange quark mass.
In the case of quenched lattice QCD this has been recently done
by Gupta \cite{GUPTA98}. His final result based on 1997 world data is
\be
\ms(2\gev)=(110\pm25)~\mev.
\ee
It is expected that unquenching will lower this value but it is difficult
to tell by how much. 

Gupta summarized also the most recent values
for $\ms(2\gev)$ obtained using QCD sum rules. The older values (in $\mev$)
are $144\pm 21$ \cite{narison:95}, $137\pm23$ \cite{jaminmuenz:95}
$148\pm 15$ \cite{chetyrkinetal:95}, whereas the most recent ones are
found to be $91-116$ \cite{Paver} and $115\pm 22$ \cite{Jamin97}.
On the other hand the following {\it lower bounds} on $\ms(2\gev)$ have
been derived: $118-189$ \cite{Yndurain}, $88\pm 9$ \cite{Dosch},
$104-116$ \cite{DERAF}. We observe that the QCD sum rule results are
consistent with quenched lattice values although generally they are
somewhat higher.

We conclude that the error on $\ms$ is still rather large.
Therefore it will be 
useful to present, few pages below,
 the results for $\epe$ for two values of $\ms(\mc)$:
\begin{equation}\label{msvalues}
\ms(\mc)=(150\pm20)\,\mev
\quad
{\rm and}
\quad
\ms(\mc)=(125\pm20)\,\mev
\end{equation}
corresponding (see table \ref{tab:ms}) roughly to
$\ms(2~\gev)=(129\pm17)\,\mev$ and $\ms(2~\gev)=(107\pm17)\,\mev$,
respectively.

Finally one should remark that the decomposition of the relevant hadronic
matrix elements of penguin operators into a product of $B_i$ factors times
$1/m_s^2$, although useful in the $1/N_c$ approach, is in principle 
unnecessary in a brute
force method like the lattice approach and in certain methods using
effective lagrangians. It is to be expected that the
future lattice calculations will directly give the relevant hadronic 
matrix elements and the issue of $\ms$ in connection with $\epe$ will
effectively disappear.

\subsection{Numerical Results for $\epe$}
In order to complete the analysis of $\epe$ one needs the value of ${\rm
Im}\lambda_t$. Since this value has been already determined in 
section \ref{sec:standard} (see table \ref{TAB2}), 
we are ready to present the results for $\epe$. 
In order to gain some insight in what is going on, let us take
the formula (\ref{GUPT}) and insert the central value 
$\IM\lambda_t=1.29\cdot 10^{-4}$ together with the NDR-values in 
table \ref{cen} for  $\Lms^{(4)}=325~\mev$. We find then
\be\label{GUPT1}
\frac{\varepsilon'}{\varepsilon} =  
\left[-1.75+(10.15\cdot B_6^{(1/2)} -5.24\cdot B_8^{(3/2)}) R_s \right]
\cdot 10^{-4}
\end{equation}
with $R_s$ defined in (\ref{RS}).

Our ``central" formula (\ref{GUPT1}) gives then the values of $\epe$
collected in table \ref{tab:31739}. We observe that for higher values
of $\ms$ the lattice and the 1/N approach (I) give values of $\epe$ in
the ball park of a few $10^{-4}$. Higher values are obtained for
the 1/N approach (II) 
and in particular in the chiral quark
model which even in the first scenario for $m_s$ gives value of $\epe$
close to $\ord(10^{-3})$. For smaller values of $\ms$ all approaches
give higher values of $\epe$ although only the last two give results
consistent with the NA31 value.
The results for the $1/N$ approach (II) are only shown for illustration.
A proper analysis of this case would require the calculation of Wilson
coefficients for $\mu$ well below $1~\gev$, which we do not want to do.

When analyzing these numbers some caution is needed. 
Our ``central" formula (\ref{GUPT1}) includes certain inputs which
are not necessarily the same in all approaches. For instance our
value of $\hat B_K$ is lower than the values obtained in the
  lattice and chiral model approaches. Similarly the value $c_0$
is very much constrained by the incorporation of 
the $\Delta I=1/2$ rule which
cannot be obtained using VIA. In addition in a given approach
$c_6$ and $c_8$ may differ somewhat from the ones used. But since
they are dominated by the short distance Wilson coefficients these
changes cannot be large and
we belive that our formula is not too bad and gives some
insight in what is going on.

On the other hand, once one begins to vary all input parameters
the differences between various approaches wash out to some extend.
We note for instance that the coefficients in tables \ref{tab:pbendr}
and \ref{cen}
exhibit a sizable $\Lms^{(4)}$-dependence leading to almost linear
dependence of $\epe$ on this parameter as pointed out in \cite{BJLW}.  

Let me than present results of the Munich group based on the input
parameters of section 10 and the choice of $B_i$ parameters
summarized in (\ref{eq:B78mc}). To this end exact expressions for
$\epe$ have been used.

For $m_s(\mc)=150\pm20\mev$ one finds \cite{BJL96a}
\begin{equation}
-1.2 \cdot 10^{-4} \le \epe \le 16.0 \cdot 10^{-4}
\label{eq:eperangenew}
\end{equation}
and
\begin{equation}
\epe= ( 3.6\pm 3.4) \cdot 10^{-4}
\label{eq:eperangefinal}
\end{equation}
for the ``scanning'' method and the ``gaussian'' method discussed
in section \ref{sec:standard}, respectively.
Using on the other hand $\ms(\mc)=(125\pm20)\mev$ one finds 
respectively \cite{BJL96b}:
\begin{equation}
-0.5\cdot 10^{-4} \le \epe \le 25.2 \cdot 10^{-4}
\label{eq:eperangenewa}
\end{equation}
and
\begin{equation}
\epe= ( 6.1\pm 5.2) \cdot 10^{-4}
\label{eq:eperangefinala}
\end{equation}
In \cite{BJL96a} the choice $\ms(\mc)=(100\pm20)\mev$ has been
considered giving $0 \le \epe \le 43.0 \cdot 10^{-4}$ and
$\epe= ( 10.4\pm 8.3) \cdot 10^{-4}$ respectively, but such low
values of $\ms(\mc)$ seem now rather improbable.

In table \ref{tab:31738} we compare these results with the existing
results obtained by various groups. There exists no recent 
phenomenological analysis from the Dortmund group based on the $B_i$
parameters obtained in \cite{DORT98}. The older result 
$\epe=(9.9\pm4.1)\cdot 10^{-4}$ from this group will certainly be
superceded by a new analysis which hopefully will be available soon.

We observe that
the result for $m_s( \mc)=150\pm20\mev$ 
in (\ref{eq:eperangefinal}) agrees rather well with
the 1996 analysis of the Rome group \cite{ciuchini:96}.
On the other hand the range in (\ref{eq:eperangenew}) shows that for
particular choices of the input parameters, values for $\epe$ as high as
$16\cdot 10^{-4}$ cannot be excluded. Such high values are
found if simultaneously  $\vub=0.10$, $B_6^{(1/2)}=1.2$, 
$B_8^{(3/2)}=0.8$, $B_K=0.6$,
$\ms(\mc)=130$ MeV, $\Lms^{(4)}=405\mev$ and low values of $\mt$ still
consistent with $\varepsilon_K$ and the observed $B_d^0-\bar B_d^0$ 
mixing
are chosen. It is, however, evident from  the comparision of
(\ref{eq:eperangenew}) and (\ref{eq:eperangefinal})  that such 
high values of $\epe$ and generally values above $10^{-3}$ 
are very improbable for $\ms(\mc)={\cal O}(150\mev)$.


\begin{table}[thb]
\caption[]{ Results for $\epe$ in units of $10^{-4}$ obtained
by various groups. The labels (S) and (G) in the last column
stand for ``Scanning'' and ``Gaussian'' respectively, as discussed
in the text. 
\label{tab:31738}}
\begin{center}
\begin{tabular}{|c|c|c|c||c|}\hline
  {\bf Reference}& $B^{(1/2)}_6$& $B^{(3/2)}_8$ & $\ms(\mc)[\mev]$ &
 $\epe[10^{-4}]$ \\ \hline
Munich
\cite{BJL96a}& $1.0\pm 0.2$ &$1.0\pm0.2$ & $150\pm20$ & $-1.2\to 16.0$ (S) \\
Munich
\cite{BJL96a}& $1.0\pm 0.2$ &$1.0\pm0.2$ & $150\pm20$ & $3.6\pm 3.4$ (G) \\
Munich
\cite{BJL96b}& $1.0\pm 0.2$ &$1.0\pm0.2$ & $125\pm20$ & $-0.5\to 25.2$ (S) \\
Munich
\cite{BJL96b}& $1.0\pm 0.2$ &$1.0\pm0.2$ & $125\pm20$ & $6.1\pm 5.2$ (G) \\
\hline
Rome
\cite{ciuchini:96}& $1.0\pm 0.2$ &$1.0\pm0.2$ & $150\pm20$ & 
$4.6\pm 3.0$ (G) \\
\hline
Trieste
\cite{BERT98}& $1.6\pm 0.3$ &$0.92\pm0.02$ & $-$ & 
$7\to 31$ (S) \\
\hline
Dubna-DESY
\cite{BELKOV} 
& $1.0$ &$1.0$ & $-$ & $-3.0 \to 3.6$ (S) \\
\hline
\end{tabular}
\end{center}
\end{table}

We observe that our ``gaussian'' result for $\ms(\mc)=(125\pm20)\mev$
agrees well with the E731
value and,
as stressed in \cite{BJL96a}, the decrease of $\ms$
 even below $\ms(\mc)= 100$ MeV is insufficient to bring 
the Standard Model in agreement with
the NA31 result provided $B_6=B_8=1.$
However, for $B_6>B_8$, sufficiently large values of
$\IM\lambda_t$ and $\Lms^{(4)}$, and small values of $\ms$, the values
of $\epe$ in the Standard Model can be as large as $(1-2)\cdot 10^{-3}$
and consistent with the NA31 result.
In order to see this explicitly we present in table \ref{tab:31731} the
values of $\epe$ for three choices of $\ms(\mc)$ and for selective
sets of other input parameters keeping  $\mt=167\,\gev$
fixed. 

The Trieste group finds generally higher values of
$\epe$, with the central value around $17\cdot 10^{-4}$ and consequently
consistent with the NA31 result. On the basis of table \ref{tab:31739}
we expect the $\epe$ from the Dortmund group to be below the one from
Trieste but generally higher than the results from Munich and Rome
for the same value of $\ms$.  

Finally I should comment on the results of \cite{BELKOV} where
$\epe$ has been investigated in the framework of an effective chiral
lagrangian approach. In this approach the values of
$B_6^{(1/2)}$ and $B_8^{(3/2)}$ cannot be calculated and the authors
set them to unity in order to obtain the values quoted in table
\ref{tab:31738}. In spite of joined efforts with Bill Bardeen to
understand this work
and discussions with these authors I failed to appreciate fully this
approach. These authors find $\epe$ consistent
with zero.

\begin{table}[thb]
\caption[]{ Values of $\epe$ in units of $10^{-4}$ 
for specific values of various input parameters at $\mt=167\,\gev$. 
\label{tab:31731}}
\begin{center}
\begin{tabular}{|c|c|c|c|c||c|}\hline
  $\IM\lambda_t[10^{-4}]$& 
$\Lms^{(4)}[MeV]$& $B^{(1/2)}_6$& $B^{(3/2)}_8$ & 
$\ms(\mc)[\mev]$ &
 $\epe[10^{-4}]$ \\ \hline
       &       &      &     & $100$ & $~8.8$ \\
$1.3$ & $325$ & $1.0$&$1.0$& $125$ & $~5.2$ \\
       &       &      &     & $150$ & $~3.2$ \\
 \hline
       &       &      &     & $100$ & $11.2$ \\
$1.3$ & $405$ & $1.0$&$1.0$ & $125$ & $~6.8$ \\
       &       &      &     & $150$ & $~4.2$ \\
 \hline
       &       &      &     & $100$ & $13.8$ \\
$1.6$ & $405$ & $1.0$&$1.0$ & $125$ & $~8.3$ \\
       &       &      &     & $150$ & $~5.2$ \\
 \hline\hline
       &       &      &     & $100$ & $12.2$ \\
$1.3$ & $325$ & $1.0$&$0.7$ & $125$ & $~7.5$ \\
       &       &      &     & $150$ & $~4.8$ \\
 \hline
       &       &      &     & $100$ & $15.4$ \\
$1.3$ & $405$ &$1.0$&$0.7$ & $125$ & $~9.5$ \\
       &       &      &     & $150$ & $~6.2$ \\
 \hline
       &       &      &     & $100$ & $19.0$ \\
$1.6$ & $405$ &$1.0$&$0.7$ & $125$ & $11.7$ \\
       &       &      &     & $150$ & $~7.6$ \\
 \hline
\end{tabular}
\end{center}
\end{table}

\subsection{Summary}
The fate of $\epe$ in the Standard Model after the
improved measurement of $\mt$ and complete NLO calculations of
short distance coefficients, depends sensitively on the values of
$|V_{ub}/V_{cb}|$, $\Lms^{(4)}$ and in particular on 
$B_6^{(1/2)}$, $B_8^{(3/2)}$ and $\ms$.
The predictions for $\epe$ obtained by
various groups are summarized in table \ref{tab:31738}. 
This table and the table \ref{tab:31731} show very clearly that any 
value for $\epe$ in
the range
\begin{equation}
0 \le \epe \le 3 \cdot 10^{-3}
\label{eq:epera}
\end{equation}
is still possible within the Standard Model at present, although most
estimates lie below $10^{-3}$ and in the range of E731 result.
Time will show which of the groups came closest to the true prediction.
It appears that most calculations give values of $B_6^{(1/2)}$ rather
close to unity and $B_8^{(3/2)}$ below one so that the inequality
$B_6^{(1/2)}\ge B_8^{(3/2)}$ should be expected to be true. If this
feature will survive more precise calculations and $\ms(\mc)$
will be eventually found in the range 
$125~\mev\le\ms(\mc)\le 150~\mev$ then $\epe$ within the Standard Model
should be somewhere between $5\cdot 10^{-4}$ and $1\cdot 10^{-3}$.
As an example let us then finally take the range (\ref{eq:B87mc}):
$B_6^{(1/2)}=1.0\pm 0.2$ and $B_8^{(3/2)}=0.7\pm 0.2$. Then the gaussian
analysis gives \cite{BJL96b}
\begin{equation}\label{eprimef}
\varepsilon'/\varepsilon =\left\{ \begin{array}{ll}
(5.3 \pm 3.8)\cdot 10^{-4}~, &~~\ms(\mc)=150\pm20\mev \\
(8.5 \pm 5.9)\cdot 10^{-4}~, & ~~\ms(\mc)=125\pm20\mev .\end{array} \right.
\end{equation}
In my opinion these results give the best representation of the
present status of $\epe$ in the Standard Model. 

One prominent physicist once told me that a person who spent
fifteen years in a given field should have enough insight into the
matters to be able to make predictions even if this is impossible
from a scientific point of view. In 1983 I made the first encounter
with $\epe$ and if the above was true I should have by now in my head
a precise prediction for $\epe$ within the Standard Model.
Clearly I do not have it, but I like to bet. Here is my bet for
the $\epe$ in the Standard Model
\begin{equation}
\epe= ( 7\pm 1) \cdot 10^{-4}.
\label{eq:bet}
\end{equation}
It is rather close to the central value of the Fermilab result in
(\ref{eprime}).
The value in (\ref{eq:bet}) corresponds to the average of the values in
(\ref{eprimef}) and the error is the one expected from new experiments.
Whether the new data will find this value is not really important as there
could be new physics invalidating my expectations.

On a more scientific level,
let us hope that the future experimental and theoretical results will
be sufficiently accurate to be able to see whether $\epe\not=0$,
whether the Standard Model agrees with the data or
whether some new physics can be discovered in this ratio. In any case the
coming years should be very exciting. 

\section{$B\to X_s\gamma$} 
\setcounter{equation}{0}
\subsection{General Remarks}
The rare decay $B\to X_s\gamma$ plays an important role in 
present day phenomenology. 
The effective Hamiltonian for $B\to X_s\gamma$ at scales 
$\mu_b={\cal O}(m_b)$
is given by
\begin{equation} \label{Heff_at_mu}
{\cal H}_{\rm eff}(b\to s\gamma) = - \frac{G_{\rm F}}{\sqrt{2}} V_{ts}^* V_{tb}
\left[ \sum_{i=1}^6 C_i(\mu_b) Q_i + C_{7\gamma}(\mu_b) Q_{7\gamma}
+C_{8G}(\mu_b) Q_{8G} \right]\,,
\end{equation}
where in view of $\mid V_{us}^*V_{ub} / V_{ts}^* V_{tb}\mid < 0.02$
we have neglected the term proportional to $V_{us}^* V_{ub}$.
Here $Q_1....Q_6$ are the usual four-fermion operators whose
explicit form is given in (\ref{O1})--(\ref{O3}). 
The remaining two operators,
characteristic for this decay, are the {\it magnetic--penguins}
\begin{equation}\label{O6B}
Q_{7\gamma}  =  \frac{e}{8\pi^2} m_b \bar{s}_\alpha \sigma^{\mu\nu}
          (1+\gamma_5) b_\alpha F_{\mu\nu},\qquad            
Q_{8G}     =  \frac{g}{8\pi^2} m_b \bar{s}_\alpha \sigma^{\mu\nu}
   (1+\gamma_5)T^a_{\alpha\beta} b_\beta G^a_{\mu\nu}  
\end{equation}
originating in the diagrams of fig.~\ref{L:12}.
In order to derive the contribution of $Q_{7\gamma}$ to the
Hamiltonian in (\ref{Heff_at_mu}), in the absence of QCD corrections,
one multiplies the vertex in (\ref{MGP})
by ``i'' and makes the replacement 
\begin{equation}
2i\sigma_{\mu\nu}q^\nu\to-\sigma^{\mu\nu}F_{\mu\nu}.
\end{equation}
Analogous
procedure gives the contribution of $Q_{8G}$.

\begin{figure}[hbt]
\vspace{0.10in}
\centerline{
\epsfysize=1.5in
%\rotate[r]{
\epsffile{L12.ps}
}%}
\vspace{0.08in}
\caption[]{Magnetic Photon (a) and Gluon (b) Penguins.
\label{L:12}}
\end{figure}
It is the magnetic $\gamma$-penguin which plays the crucial role in
this decay. However, the role of the dominant current-current
operator $Q_2$ should not be underestimated.
Indeed the short distance QCD effects involving in particular the
mixing between $Q_2$  and $Q_{7\gamma}$ are very important in this decay.
They are known
\cite{Bert,Desh} to enhance $C_{7\gamma}(\mu_b)$ 
substantially, so that the resulting branching ratio
$Br(B\to X_s\gamma)$ turns out to be by a factor 
of 3 higher than it would be without QCD effects.
Since the first analyses
in \cite{Bert,Desh} a lot of progress has been made in calculating
these important  QCD effects beginning with the work in \cite{Grin,Odon}. 
We will briefly summarize this progress.

A peculiar feature of the renormalization group analysis 
in $B\to X_s\gamma$ is that the mixing under infinite renormalization 
between
the set $(Q_1...Q_6)$ and the operators $(Q_{7\gamma},Q_{8G})$ vanishes
at the one-loop level. Consequently in order to calculate 
the coefficients
$C_{7\gamma}(\mu_b)$ and $C_{8G}(\mu_b)$ in the leading logarithmic
approximation, two-loop calculations of ${\cal{O}}(e g^2_s)$ 
and ${\cal{O}}(g^3_s)$
are necessary. The corresponding NLO analysis requires the evaluation
of the mixing in question at the three-loop level. 
This peculiar feature caused
that the first fully correct calculation of the leading  anomalous
dimension matrix relevant for this decay
has been obtained only in 1993 \cite{CFMRS:93,CFRS:94}.
It has been
confirmed subsequently in \cite{CCRV:94a,CCRV:94b,Mis:94}.

As of 1998 also the NLO corrections to $B\to X_s\gamma$ 
have been completed.
It was a joint effort of many groups. Let us summarize this progress: 
\bi
\item
The $O(\alpha_s)$
corrections to $C_{7\gamma}(\mu_W)$ and $C_{8G}(\mu_W)$ have been first
calculated in \cite{Yao1} and recently confirmed by several groups
\cite{GH97,BKP2,GAMB}.
\item
The two-loop
mixing involving the operators
$Q_1.....Q_6$ and the two-loop mixing
in the sector $(Q_{7\gamma},Q_{8G})$ has been calculated in 
\cite{ACMP,WEISZ,BJLW1,BJLW,ROMA1,ROMA2} 
and \cite{MisMu:94}, respectively.  
Finally after a heroic effort  the three loop mixing between
the set $(Q_1...Q_6)$ and the operators $(Q_{7\gamma},Q_{8G})$
 has been completed at the end of 1996 \cite{CZMM}.
As a byproduct the authors of \cite{CZMM} confirmed the existing
two-loop anomalous dimension matrix in the $Q_1...Q_6$ sector.
\item
One-loop matrix elements 
$\langle s\gamma {\rm gluon}|Q_i| b\rangle$ have been calculated in 
\cite{AG2,Pott} and the very difficult two-loop corrections to 
$\langle s\gamma |Q_i| b\rangle$ have been presented in \cite{GREUB}.
\ei

We will now discuss all these achievements in explicit terms.
In order to appreciate the importance of NLO calculations for this
decay it is instructive to discuss first the leading logarithmic
approximation.
\subsection{The Decay $B\to X_s\gamma$ in the Leading Log Approximation}
         \label{sec:Heff:Bsgamma:lo}
\subsubsection{Anomalous Dimension Matrix}
It is instructive to to discuss first the mixing between 
the sets $Q_1,\ldots,Q_6$ and $Q_{7\gamma},Q_{8G}$ in $\hat\gamma_s^{(0)}$.
To this end I use the work done in colaboration with Misiak, M\"unz
and Pokorski \cite{BMMP:94}.
The point is that this mixing resulting from two-loop
diagrams is generally regularization scheme dependent. This is
certainly disturbing because the matrix $\hat\gamma_s^{(0)}$, being the
first term in the expansion for $\hat\gamma_s$, is usually scheme
independent.  As we will show below, there is a simple way to circumvent 
this difficulty \cite{BMMP:94}.

As noticed in \cite{CFMRS:93,CFRS:94} the regularization scheme
dependence of $\hat\gamma_s^{(0)}$ in the case of $b\to s\gamma$ and
$b\to s g$ is signaled in the finite parts of the one-loop matrix 
elements of $Q_1,\ldots,Q_6$
for on-shell photons or gluons.  They vanish in any 4-dimensional
regularization scheme and in the HV scheme but some of them are
non-zero in the NDR scheme.  One has
\begin{equation}
\langle Q_i \rangle_{\rm one-loop}^\gamma =
y_i \, \langle Q_{7\gamma} \rangle_{\rm tree},
\qquad i=1,\ldots,6
\label{eq:defy}
\end{equation}
and
\begin{equation}
\langle Q_i\rangle_{\rm one-loop}^G =
z_i \, \langle Q_{8G} \rangle_{\rm tree},
\qquad i=1,\ldots,6.
\end{equation}

In the HV scheme all the $y_i$'s and $z_i$'s vanish, while in the NDR
scheme $\vec{y} = (0,0,0,0,-\frac{1}{3},-1)$ and $\vec{z} =
(0,0,0,0,1,0)$.  This regularization scheme dependence is canceled by a
corresponding regularization scheme dependence in $\hat\gamma_s^{(0)}$
as first demonstrated in \cite{CFMRS:93,CFRS:94}. It should be
stressed that the numbers $y_i$ and $z_i$ come from divergent, i.e.
purely short-distance parts of the one-loop integrals. So no reference
to the spectator-model or to any other model for the matrix elements is
necessary here.

In view of all this  it is convenient in the leading order to introduce
the so-called ``effective coefficients'' \cite{BMMP:94} for the
operators $Q_{7\gamma}$ and $Q_{8G}$ which are regularization scheme
independent. They are given as follows:
\begin{equation} \label{eq:defc7eff}
C^{(0)eff}_{7\gamma}(\mu_b) =
C^{(0)}_{7\gamma}(\mu_b) + \sum_{i=1}^6 y_i C^{(0)}_i(\mu_b)
\end{equation}
and 
\begin{equation}\label{eq:defc8eff}
C^{(0)eff}_{8G}(\mu_b) = C^{(0)}_{8G}(\mu_b) 
+ \sum_{i=1}^6 z_i C^{(0)}_i(\mu_b).
\end{equation}
One can then introduce a scheme-independent vector
\begin{equation} 
\vec{C}^{(0)eff}(\mu_b) = \left( C^{(0)}_1(\mu_b),\ldots, C^{(0)}_6(\mu_b), 
C^{(0)eff}_{7\gamma}(\mu_b),C^{(0)eff}_{8G}(\mu_b) \right) \, .
\end{equation}
From the RGE for $\vec{C}^{(0)}(\mu)$ it is straightforward
to derive the RGE for $\vec{C}^{(0)eff}(\mu)$. It has the form
\begin{equation} \label{RGEeff}
\mu \frac{d}{d \mu} C^{(0)eff}_i(\mu) = 
\frac{\as}{4\pi} \gamma^{(0)eff}_{ji} C^{(0)eff}_j(\mu)
\end{equation}
where
\begin{equation} \label{def.geff}
\gamma^{(0)eff}_{ji} = \left\{ \begin{array}{ccl}
\gamma^{(0)}_{j7} +
\sum_{k=1}^6 y_k\gamma^{(0)}_{jk} -y_j\gamma^{(0)}_{77} -z_j\gamma^{(0)}_{87}
&\quad& $i=7$,\ $j=1,\ldots,6$ \\
\gamma^{(0)}_{j8} +
\sum_{k=1}^6 z_k\gamma^{(0)}_{jk} -z_j\gamma^{(0)}_{88}
&\quad& $i=8$,\ $j=1,\ldots,6$ \\
\gamma^{(0)}_{ji} &\quad& \mbox{otherwise.}
\end{array}
\right.
\end{equation}
The matrix $\hat\gamma^{(0)eff}$ is a scheme-independent quantity.
It equals the matrix which one would directly obtain from two-loop
diagrams in the HV scheme.  In order to simplify the notation we will
omit the label ``eff'' in the expressions for the elements of this
effective one loop anomalous dimension matrix given below and keep it
only in the Wilson coefficients of the operators $Q_{7\gamma}$ and
$Q_{8G}$.

We are now ready to give the leading anomalous dimension matrix 
relevant for the calculation of the $B\to X_s\gamma $ rate in the
LO approximation.
The
$6 \times 6$ submatrix of $\hat\gamma^{(0)}$ involving the operators
$Q_1,\ldots,Q_6$ is given in (\eqn{eq:gs0Kpp}). Here we only give the
remaining non-vanishing entries of $\hat\gamma^{(0)}$
\cite{CFMRS:93,CFRS:94}.

The elements $\gamma^{(0)}_{i7}$ with $i=1,\ldots,6$ are:
\begin{eqnarray}
\gamma^{(0)}_{17} = 0, &\qquad&  \gamma^{(0)}_{27} =
\frac{104}{27} C_F
\label{eq:g0127} \\
\gamma^{(0)}_{37} = -\frac{116}{27} C_F
 &\qquad&  \gamma^{(0)}_{47}  = \left(\frac{104}{27} u -\frac{58}{27}d
\right) C_F
\label{eq:g0347} \\
\gamma^{(0)}_{57} = \frac{8}{3} C_F &\qquad&
\gamma^{(0)}_{67} = \left( \frac{50}{27}d -\frac{112}{27}u \right) C_F
\label{eq:g0567}
\end{eqnarray}
The elements $\gamma^{(0)}_{i8}$ with $i=1,\ldots,6$ are:
\begin{eqnarray}
\gamma^{(0)}_{18} = 3, &\quad& \gamma^{(0)}_{28} =
\frac{11}{9} N-\frac{29}{9}\frac{1}{N}
\label{eq:g0128} \\
\gamma^{(0)}_{38} = \frac{22}{9} N-\frac{58}{9}\frac{1}{N}+3 f
 &\quad& \gamma^{(0)}_{48}  = 
6+\left(\frac{11}{9} N -\frac{29}{9}\frac{1}{N}\right) f
\label{eq:g0348} \\
\gamma^{(0)}_{58} = -2 N+\frac{4}{N} -3 f  &\quad&
\gamma^{(0)}_{68} = -4-\left( \frac{16}{9} N -
\frac{25}{9}\frac{1}{N}\right) f
\label{eq:g0568}
\end{eqnarray}

Finally the $2\times 2$ one-loop anomalous dimension matrix in the
sector $Q_{7\gamma},Q_{8G}$ is given by \cite{Grin}
\begin{eqnarray}
\gamma^{(0)}_{77} = 8 C_F
&\qquad&
\gamma^{(0)}_{78} = 0
\label{gammaB0} \\
\gamma^{(0)}_{87} = -\frac{8}{3} C_F
&\qquad&
\gamma^{(0)}_{88} = 16 C_F - 4 N
\nn
\end{eqnarray}
\subsubsection{Renormalization Group Evolution}
         \label{sec:Heff:BXsgamma:RGE}
The coefficients $C_i(\mu_b)$ in (\ref{Heff_at_mu}) can be calculated
by using
\begin{equation}
\vec C(\mu_b)= \hat U_5(\mu_b,\mu_W)\vec C(\mu_W)
\end{equation}
Here $ \hat U_5(\mu_b,\mu_W)$ is the $8\times 8$ evolution matrix which is
given in general terms in (\eqn{u0jj}) with $\hat\gamma$ being this
time an $8\times 8$ anomalous dimension matrix. In the leading order
$\hat U_5(\mu_b,\mu_W)$ is to be replaced by $\hat U_5^{(0)}(\mu_b,\mu_W)$ 
and
the initial conditions by $\vec C^{(0)}(\mu_W)$ with \cite{Grin}
\begin{equation}\label{c2}
C^{(0)}_2(\mu_W) = 1                               
\end{equation}
\begin{equation}\label{c7}
C^{(0)}_{7\gamma} (\mu_W) = \frac{3 x_t^3-2 x_t^2}{4(x_t-1)^4}\ln x_t + 
   \frac{-8 x_t^3 - 5 x_t^2 + 7 x_t}{24(x_t-1)^3}
   \equiv -\frac{1}{2} D'_0(x_t)
\end{equation}
\begin{equation}\label{c8}
C^{(0)}_{8G}(\mu_W) = \frac{-3 x_t^2}{4(x_t-1)^4}\ln x_t +
   \frac{-x_t^3 + 5 x_t^2 + 2 x_t}{8(x_t-1)^3}                               
   \equiv -\frac{1}{2} E'_0(x_t)
\end{equation}
In LO
all remaining coefficients are set to zero at $\mu=\mu_W$. 

Using the techniques developed in section 5, the leading order results for 
the Wilson coefficients of all operators
entering the effective Hamiltonian in (\ref{Heff_at_mu}) can be written
in an analytic form. They are \cite{BMMP:94}
\begin{eqnarray}
\label{coeffs}
C_j^{(0)}(\mu_b)    & = & \sum_{i=1}^8 k_{ji} \eta^{a_i}
  \qquad (j=1,\ldots,6)  \\
\label{C7eff}
C_{7\gamma}^{(0)eff}(\mu_b) & = & 
\eta^\frac{16}{23} C_{7\gamma}^{(0)}(\mu_W) + \frac{8}{3}
\left(\eta^\frac{14}{23} - \eta^\frac{16}{23}\right) C_{8G}^{(0)}(\mu_W) +
 C_2^{(0)}(\mu_W)\sum_{i=1}^8 h_i \eta^{a_i},
\\
\label{C7Geff}
C_{8G}^{(0)eff}(\mu_b) & = & 
\eta^\frac{14}{23} C_{8G}^{(0)}(\mu_W) 
   + C_2^{(0)}(\mu_W) \sum_{i=1}^8 \bar h_i \eta^{a_i},
\end{eqnarray}
with
\begin{eqnarray}
\eta & = & \frac{\as(\mu_W)}{\as(\mu_b)}, 
\end{eqnarray}
and $C_{7\gamma}^{(0)}(\mu_W)$
and $ C_{8G}^{(0)}(\mu_W)$ given in (\ref{c7}) and (\ref{c8}),
respectively. The numbers $a_i$ and $k_{ji}$ have been already given
in section 8.4. For convenience we give again the values 
of $a_i$ together with $h_i$ and $\bar h_i$ 
in table \ref{tab:akh}.

\begin{table}[htb]
\caption[]{Magic Numbers.
\label{tab:akh}}
\begin{center}
\begin{tabular}{|r|r|r|r|r|r|r|r|r|}
\hline
$i$ & 1 & 2 & 3 & 4 & 5 & 6 & 7 & 8 \\
\hline
$a_i $&$ \frac{14}{23} $&$ \frac{16}{23} $&$ \frac{6}{23} $&$
-\frac{12}{23} $&$
0.4086 $&$ -0.4230 $&$ -0.8994 $&$ 0.1456 $\\
$h_i $&$ 2.2996 $&$ - 1.0880 $&$ - \frac{3}{7} $&$ -
\frac{1}{14} $&$ -0.6494 $&$ -0.0380 $&$ -0.0185 $&$ -0.0057 $\\
$\bar h_i $&$ 0.8623 $&$ 0 $&$ 0 $&$ 0
 $&$ -0.9135 $&$ 0.0873 $&$ -0.0571 $&$ 0.0209 $\\
\hline
\end{tabular}
\end{center}
\end{table}

Let us perform a quick numerical analysis of (\ref{C7eff}) and
(\ref{C7Geff}).
Using the leading $\mu_b$-dependence of $\as$:
\begin{equation} 
\as(\mu_b) = \frac{\as(\mz)}{1 
- \beta_0 \frac{\as(\mz)}{2\pi} \, \ln(\mz/\mu_b)} 
\label{eq:asmumz}
\end{equation} 
one finds the results in table \ref{tab:c78effnum}.

\begin{table}[htb]
\caption[]{Wilson coefficients $C^{(0){\rm eff}}_{7\gamma}$ and 
$C^{(0){\rm eff}}_{8G}$
for $\mt = 170 \gev$ and various values of $\as^{(5)}(\mz)$ and $\mu$.
\label{tab:c78effnum}}
\begin{center}
\begin{tabular}{|c||c|c||c|c||c|c|}
\hline
& \multicolumn{2}{c||}{$\as^{(5)}(\mz) = 0.113$} &
  \multicolumn{2}{c||}{$\as^{(5)}(\mz) = 0.118$} &
  \multicolumn{2}{c| }{$\as^{(5)}(\mz) = 0.123$} \\
\hline
$\mu [\gev]$ & 
$C^{(0){\rm eff}}_{7\gamma}$ & $C^{(0){\rm eff}}_{8G}$ &
$C^{(0){\rm eff}}_{7\gamma}$ & $C^{(0){\rm eff}}_{8G}$ &
$C^{(0){\rm eff}}_{7\gamma}$ & $C^{(0){\rm eff}}_{8G}$ \\
\hline
 2.5 & --0.328 & --0.155 & --0.336 & --0.158 & --0.344 & --0.161 \\
 5.0 & --0.295 & --0.142 & --0.300 & --0.144 & --0.306 & --0.146 \\
 7.5 & --0.277 & --0.134 & --0.282 & --0.136 & --0.286 & --0.138 \\
10.0 & --0.265 & --0.130 & --0.269 & --0.131 & --0.273 & --0.133 \\
\hline
\end{tabular}
\end{center}
\end{table}

Two features of these results should be emphasised:
\begin{itemize}
\item
The strong enhancement of the
coefficient $C^{(0){\rm eff}}_{7\gamma}$ by short distance QCD effects which 
we illustrate by the relative numerical importance of the three terms in
expression (\ref{C7eff}).
For instance, for $\mt = 170\gev$, $\mu_b = 5\gev$ and $\as^{(5)}(\mz)
=0.118$ one obtains
\begin{eqnarray}
C^{(0){\rm eff}}_{7\gamma}(\mu_b) &=&
0.695 \; C^{(0)}_{7\gamma}(\mu_W) +
0.085 \; C^{(0)}_{8G}(\mu_W) - 0.158 \; C^{(0)}_2(\mu_W)
\nn\\
 &=& 0.695 \; (-0.193) + 0.085 \; (-0.096) - 0.158 = -0.300 \, .
\label{eq:C7geffnum}
\end{eqnarray}
In the absence of QCD we would have $C^{(0){\rm eff}}_{7\gamma}(\mu_b) =
C^{(0)}_{7\gamma}(\mu_W)$ (in that case one has $\eta = 1$). Therefore, the
dominant term in the above expression (the one proportional to
$C^{(0)}_2(\mu_W)$) is the additive QCD correction that causes the
enormous QCD enhancement of the \Bsg rate \cite{Bert,Desh}.
It originates solely from the two-loop diagrams. On the other hand, the
multiplicative QCD correction (the factor 0.695 above) tends to
suppress the rate, but fails in the competition with the additive
contributions.

In the case of $C^{(0){\rm eff}}_{8G}$ a similar enhancement is observed
\begin{eqnarray}
C^{(0){\rm eff}}_{8G}(\mu_b) &=&
0.727 \; C^{(0)}_{8G}(\mu_W) - 0.074 \; C^{(0)}_2(\mu_W)
\nn \\
 &=& 0.727 \; (-0.096) - 0.074 = -0.144 \, .
\label{eq:C8Geffnum}
\end{eqnarray}
\item
A strong $\mu_b$-dependence of both coefficients   
as first stressed by Ali and Greub \cite{AG1} and confirmed
in \cite{BMMP:94}. 
Since \Bsg is dominated by QCD effects, it is not 
surprising 
that this scale-uncertainty in the leading order 
is particularly large. We will investigate this scale
uncertainty in a moment.
\end{itemize}
\subsubsection{Scale Uncertainties at LO}
In calculating $Br(B\to X_s\gamma)$ it is customary to use the
spectator model in which the inclusive decay $B\to X_s\gamma$
is approximated by the partonic decay $b\to s\gamma$. That is
one uses the following approximate equality: 

\begin{equation}\label{ratios}
\frac{\Gamma(B \to X_s \gamma)}
     {\Gamma(B \to X_c e \bar{\nu}_e)}
 \simeq                                                     
\frac{\Gamma(b \to s \gamma)}
     {\Gamma(b \to c e \bar{\nu}_e)} \equiv R_{{\rm quark}},
\end{equation}
where the quantities on the r.h.s are calculated in the spectator model
corrected for short-distance QCD effects. The normalization to the
semileptonic rate is usually introduced in order to reduce the
uncertainties due to the CKM matrix
elements and factors of $\mb^5$ in the r.h.s. of (\ref{ratios}).
Additional support for the approximation given above comes from the
heavy quark expansions.  Indeed the spectator model has been shown to
correspond to the leading order approximation of an expansion in
$1/\mb$.  The first corrections appear at the ${\cal O}(1/\mb^2)$
level and will be discussed at the end of this section. 

The leading
logarithmic calculations 
\cite{Grin,CFRS:94,CCRV:94a,Mis:94,AG1,BMMP:94} 
can be summarized in a compact form
as follows:
\begin{equation}\label{main}
R_{{\rm quark}} =\frac{Br(B \to X_s \gamma)}
     {Br(B \to X_c e \bar{\nu}_e)}=
 \frac{|V_{ts}^* V_{tb}^{}|^2}{|V_{cb}|^2} 
\frac{6 \alpha}{\pi f(z)} |C^{(0){\rm eff}}_{7}(\mu_b)|^2\,,
\end{equation}
where
\begin{equation}\label{g}
f(z) = 1 - 8z + 8z^3 - z^4 - 12z^2 \ln z           
\quad\mbox{with}\quad
z =
\frac{m^2_{c,pole}}{m^2_{b,pole}}
\end{equation}
is the phase space factor in $Br(B \to X_c e \bar{\nu}_e)$ and
$\alpha=e^2/4\pi$. In order to find (\ref{main}) only the tree level
matrix element $<s\gamma|Q_{7\gamma}|B>$ has to be computed. 

\noindent
There are three scale uncertainties present in (\ref{main}):
\begin{itemize}
\item
The low energy scale $\mu_b=\ord(m_b)$ at which the Wilson
Coefficient $C_{7}^{(0){\rm eff}}(\mu_b)$ is evaluated.
\item
The high energy scale $\mu_W=\ord(\mw)$ at which 
the full theory is matched with the effective five-quark theory.
In LO this scale enters only $\eta$.
$C_{7}^{(0)}(\mu_W)$ and  $C_{8}^{(0)}(\mu_W)$
serve in LO as initial
conditions to the renormalization group evolution from $\mu_W$ down
to $\mu_b$. As seen explicitly in (\ref{c7}) and (\ref{c8}) they do
not depend on $\mu_W$.
\item
The scale $\mu_t=\ord(m_t)$ at which the running top quark mass is
defined. In LO it enters only $x_t$: 
\be\label{xt}
x_t=\f{\mtb^2(\mu_t)}{\mw^2}.
\ee
As we stressed in connection with $B^0-\bar B^0$ mixing
in section 8.3, $\mu_W$ and $\mu_t$ do not have to be
equal. Initially when the top quark and the W-boson are integrated
out, it is convenient in the process of matching to keep
$\mu_t=\mu_W$. Yet one has always the freedom to redefine the top
quark mass and to work with $\mtb(\mu_t)$ where $\mu_t\not=\mu_W$.
\end{itemize}

It is evident from the formulae above that in LO the variations of
$\mu_b$, $\mu_W$ and $\mu_t$ remain uncompensated which results
in potential theoretical uncertainties in the predicted branching
ratio.
  In the context of phenomenological analyses of $B \to X_s\gamma$,
the uncertainty due to $\mu_b$ has been discussed
\cite{AG1,BMMP:94,CZMM,GREUB,BKP1}. The
uncertainties due to $\mu_W$
and $\mu_t$ have been analyzed first in \cite{BKP1} and 
recently in \cite{BG98}. I will follow here my own work with
Axel Kwiatkowski and Nicolas Pott \cite{BKP1}.

\noindent
It is customary to estimate the uncertainties due to $\mu_b$ by
varying it in the range $\mb/2\le\mu_b\le 2\mb$. Similarly one
can vary $\mu_W$ and $\mu_t$ in the ranges $\mw/2\le\mu_W\le 2\mw$
and $\mt/2\le\mu_t\le 2\mt$ respectively. Specifically in our
numerical analysis we will consider the ranges
\be\label{ranges1}
2.5~\gev\le\mu_b\le 10~\gev
\ee
and
\be\label{ranges}
40~\gev\le\mu_W\le 160~\gev\qquad 80~\gev\le \mu_t\le 320\gev
\ee
In the LO analysis we use the leading order formula for
$\as(\mu_b)$ in (\ref{eq:asmumz})
with $\alpha_s(\mz)=0.118$ and
\be\label{mbar}
\mtb(\mu_t)=\mtb(\mt)
\left[\f{\as(\mu_t)}{\as(\mt)}\right]^{\f{4}{\beta_0}}.
\ee
Here $\beta_0=23/3$.
We set $\mtb(\mt)=168~\gev$
and $\mt\equiv m_{t,pole}=176~\gev$.

\noindent
Varying $\mu_b$, $\mu_W$ and $\mu_t$ in the ranges (\ref{ranges1})
and (\ref{ranges})  we
find the following  uncertainties in the branching
ratio \cite{BKP1}:
\begin{equation}\label{LOmu1}
\Delta Br(B\to X_s \gamma)=\left\{ \begin{array}{ll}
\pm 22\% & (\mu_b) \\
\pm 13\% & (\mu_W) \\
\pm 3 \% & (\mu_t) \end{array} \right.
\end{equation}
The fact that the $\mu_W$-uncertainty is smaller than
the $\mu_b$ uncertainty is entirely due to $\as(\mu_W)<\as(\mu_b)$. 
Still this uncertainty is rather disturbing as it introduces an error of
approximately $\pm 0.40\cdot 10^{-4}$ in the branching ratio.
The
smallness of the $\mu_t$-uncertainty is related to the weak $x_t$
dependence of $C_{7}^{(0)}(\mu_W)$ and  $C_{8}^{(0)}(\mu_W)$
which in the range of interest can be well approximated by
\be
C_{7}^{(0)}(\mu_W)=-0.122~ x_t^{0.30}
\qquad  C_{8}^{(0)}(\mu_W)=-0.072~ x_t^{0.19}.
\ee
Thus even if $161\gev\le\mtb(\mu_t)\le 178\gev$ for $\mu_t$ in 
(\ref{ranges}),
the $\mu_t$ uncertainty in  $Br(B\to X_s \gamma)$ is small.
This should be contrasted with  $B_s\to\mu\bar\mu$,
$K_L\to\pi^0\nu\bar\nu$ and $ B_{d,s}^0-\bar B_{d,s}^0$ mixings, 
where $\mu_t$ uncertainties in LO have been
found \cite{BB2,BJW90} to be $\pm 13\%$, $\pm 10\%$ and $\pm 9\%$ 
respectively.

A critical analysis of theoretical and
experimental
uncertainties present in the prediction for Br(\Bsg) based on the
formula (\ref{main}) has been made in \cite{BMMP:94} 
with the result that the error in the Standard Model prediction
in the LO approximation is dominated by
the scale ambiguities. 
The final result of the LO analysis in \cite{BMMP:94} which 
omitted the $\mu_t$ and $\mu_W$ uncertainties was
\be\label{LORES}
Br(B{\to}X_s \gamma)_{\rm LO} =
 (2.8 \pm 0.8)  \times 10^{-4}
\ee
Similar result has been found in \cite{AG1}.

These finding made 
it  clear already in 1993 that  a  complete
NLO analysis of \Bsg was very desirable. 
Such a complete next-to-leading
calculation of \Bsg was described in \cite{BMMP:94} in general terms. 
As demonstrated formally there, the cancellation of the dominant 
$\mu_b$-dependence in the leading
order can then be  achieved. While this formal NLO analysis was
very encouraging with respect to the reduction of the $\mu_b$-dependence,
it could obviously not provide the actual size of Br(\Bsg) after
the inclusion of NLO corrections. Fortunately four years later such
a complete NLO analysis exists and the impact of NLO corrections on
Br(\Bsg) can be analysed in explicit terms. This is precisely 
what we will do now.

\subsection{\Bsg Beyond Leading Logarithms}
         \label{sec:Heff:Bsgamma:nlo}
\subsubsection{Master Formulae}
The formula (\ref{main}) modifies after the inclusion of NLO
corrections as follows \cite{CZMM}:
\be \label{ration}
R_{{\rm quark}} = 
\frac{|V_{ts}^* V_{tb}|^2}{|V_{cb}|^2} 
\frac{6 \alpha}{\pi f(z)} F \left( |D|^2 + A \right)\,,
\ee
%
where
\be \label{factor}
F = \f{1}{\kappa(z)} 
    \left( \f{\overline{m}_b(\mu=m_b)}{m_{b,{\rm pole}}} \right)^2
    = 
    \f{1}{\kappa(z)} \left( 1 - \f{8}{3} \f{\as(m_b)}{\pi} \right),
\ee
 \be \label{Dvirt}
D = C_{7\gamma}^{(0){\rm eff}}(\mu_b) + \frac{\as(\mu_b)}{4 \pi} \left\{ 
C_{7\gamma}^{(1){\rm eff}}(\mu_b) + \sum_{i=1}^8 C_i^{(0){\rm eff}}(\mu_b) 
\left[ r_i + \gamma_{i7}^{(0){\rm eff}} \ln \frac{m_b}{\mu_b} 
\right] \right\}
\ee
and $A$ is discussed below. 

Let us explain the origin of various new contributions:
\begin{itemize}
\item
First $\kappa(z)$
is the QCD correction to the semileptonic decay
\cite{CM78}. To a good approximation it is given by \cite{KIMM}
\be \label{kap}
\kappa(z) = 1 - \frac{2 \as (\bar\mu_b)}{3 \pi}
\left[(\pi^2-\frac{31}{4})(1-z)^2+\frac{3}{2}\right] \,.
\ee
An exact analytic formula for $\kappa(z)$ can be found in \cite{N89}.
Here $\bar\mu_b=\ord(m_b)$ is a scale in the calculation of QCD corrections
to the semi-leptonic rate which is generally different from the one used
in the $b\to s\gamma$ transition. In this respect we differ from Greub et al.
\cite{GREUB} who set $\bar\mu_b=\mu_b$. 
\item
The second factor in (\ref{factor}) originates as follows.
The \Bsg rate is proportional to $m_{b,{\rm pole}}^3$ present in the
two body phase space and to $\overline{m}_b(\mu=m_b)^2$ present in 
$<s\gamma|Q_{7\gamma}|B>^2$. On the other hand the semileptonic
rate is is proportional to $m_{b,{\rm pole}}^5$ present in the
three body phase space. Thus the $m_b^5$ factors present in
both rates differ by a ${\cal O}(\as)$ correction which has
been consistently omitted in the leading logarithmic approximation
but has to be included now.
\item
For similar reason the variable $z$ entering $f(z)$ and $\kappa(z)$
can be more precisely specified at the NLO level to be 
\cite{GREUB,CZMM}:
\be \label{g(z)}
 z = \frac{m_{c,{\rm pole}}}{m_{b,{\rm pole}}}=0.29\pm0.02 
\ee
which is obtained from $m_{b,{\rm pole}} = 4.8 \pm 0.15$~GeV and
$m_{b,{\rm pole}}-m_{c,{\rm pole}}=3.40$~GeV. 
This gives
\be \label{kf}
\kappa(z)=0.879\pm0.002\approx 0.88\,,
\qquad
f(z)=0.54\pm 0.04\,.
\end{equation}
\item
The amplitude $D$ in (\ref{Dvirt}) includes two types of new
contributions. The first $\as$-correction originates in
the NLO correction to the Wilson coefficients of $Q_{7\gamma}$:
\be \label{C.expanded}
C^{\rm eff}_{7\gamma}(\mu_b) = C^{(0){\rm eff}}_{7\gamma}(\mu_b) + 
\frac{\as(\mu_b)}{4 \pi} C^{(1){\rm eff}}_{7\gamma}(\mu_b)\,. 
\ee
It is this correction which requires the calculation of the
three-loop anomalous dimensions \cite{CZMM}. An explicit formula for
$C^{(1){\rm eff}}_{7\gamma}(\mu_b)$ has been given for the first time
in \cite{CZMM}. We will give a generalization of this formula 
in a moment.

The two remaining corrections in (\ref{Dvirt}) come from one-loop
matrix elements $<s\gamma|Q_{7\gamma}|B>$ and 
$<s\gamma|Q_{8G}|B>$ and from two-loop matrix elements
$<s\gamma|Q_i|B>$ of the remaining operators. These two-loop
matrix elements have been calculated in \cite{GREUB}. The coefficients
of the logarithm are the relevant elements in the leading
anomalous dimension matrix. The explicit logarithmic 
$\mu_b$ dependence in the last term in $D$ will play an important role
few pages below.

Now $C^{(1){\rm eff}}_{7\gamma}(\mu_b)$ is renormalization scheme dependent.
This scheme dependence is cancelled by the one present in the
constant terms $r_i$. Actually ref. \cite{GREUB} does not provide
the matrix elements of the QCD-penguin operators and consequently
$r_i~(i=3-6)$ are unknown. However, the Wilson coefficients
of QCD-penguin operators are very small and this omission is 
most probably immaterial.
\item
The term $A$ in (\ref{ration}) originates from the bremsstrahlung
corrections and the necessary virtual corrections needed for the
cancellation of the infrared divergences. These have been
calculated in \cite{AG2,Pott} and are also considered in 
\cite{CZMM,GREUB} in the
context of the full analysis.  Since the virtual corrections
are also present in the terms $r_i$ in $D$, care must be taken
in order to avoid double counting. This is discussed in detail
in \cite{CZMM} where an explicit formula for $A$ can be found.
It is the equation (32) of \cite{CZMM}.

Actually $A$ depends on  an explicit lower cut on the
photon energy 
%
\be\label{phs}
E_{\gamma} > 
( 1 - \delta ) E_{\gamma}^{{\rm max}} \equiv ( 1 - \delta ) \frac{m_b}{2}.
\ee
Moreover $A$ is divergent in the limit $\delta \to 1$.
In order to cancel this divergence one would have to consider
the sum of \Bsg and $b{\to}X_s$ decay rates.
However, the divergence at $\delta{\to}1$ is very slow. 
In order to
allow an easy comparison with previous experimental and theoretical
publications the authors in \cite{CZMM} choose $\delta = 0.99$.
Further details on the $\delta$-dependence can be found in this paper.
\item
Finally the values of $\as(\mu_b)$ in all the
above formulae are calculated with the use of the NLO expression 
for the strong coupling constant:
%
\be \label{alphaNLL1}
\as(\mu) = \frac{\as(M_Z)}{v(\mu)} \left[1 - \f{\beta_1}{\beta_0} 
           \frac{\as(M_Z)}{4 \pi}    \f{\ln v(\mu)}{v(\mu)} \right],
\ee
%
where 
%
\be \label{v(mu1)}
v(\mu) = 1 - \beta_0 \frac{\as(M_Z)}{2 \pi} 
\ln \left( \frac{M_Z}{\mu} \right),
\ee
%
$\beta_0 = \frac{23}{3}$ and $\beta_1 = \frac{116}{3}$.
\end{itemize}

Generalizing the formula (21) of \cite{CZMM} to include $\mu_t$ and $\mu_W$
dependences one finds \cite{BKP1}
\bea     \label{c7eff1}
C^{(1)eff}_7(\mu_b) &=& 
\eta^{\f{39}{23}} C^{(1)eff}_7(\mu_W) + \f{8}{3} \left( \eta^{\f{37}{23}} 
- \eta^{\f{39}{23}} \right) C^{(1)eff}_8(\mu_W) 
\nonumber \\ &&
+\left( \f{297664}{14283} \eta^{\f{16}{23}}-\f{7164416}{357075} 
\eta^{\f{14}{23}} 
       +\f{256868}{14283} \eta^{\f{37}{23}} -\f{6698884}{357075} 
\eta^{\f{39}{23}} \right) C_8^{(0)}(\mu_W) 
\nonumber \\ &&
+\f{37208}{4761} \left( \eta^{\f{39}{23}} - 
\eta^{\f{16}{23}} \right) C_7^{(0)}(\mu_W) 
+ \sum_{i=1}^8 (e_i \eta E_0(x_t) + f_i + g_i \eta) \eta^{a_i}
\nonumber \\ &&
+\Delta C^{(1)eff}_7(\mu_b),
\eea
where
in the $\overline{MS}$ scheme 
\bea\label{GENC7}
C_7^{(1)eff}(\mu_W)&=& C_7^{(1)eff}(M_W)+
8 x_t \f{\partial C_7^{(0)}(\mu_W)}{\partial x_t}\ln\f{\mu_t^2}{\mw^2}
\nonumber \\ &&
+\left(\f{16}{3}C_7^{(0)}(\mu_W)-\f{16}{9} C_8^{(0)}(\mu_W)
+\f{\gamma_{27}^{(0){\rm eff}}}{2}\right) \ln \frac{\mu_W^2}{\mw^2} 
\eea
\bea\label{GENC8}
C_8^{(1)eff}(\mu_W)&=& C_8^{(1)eff}(M_W)+
8 x_t \f{\partial C_8^{(0)}(\mu_W)}{\partial x_t}\ln\f{\mu_t^2}{\mw^2} 
\nonumber \\ &&
+\left(\f{14}{3}C_8^{(0)}(\mu_W)
+\f{\gamma_{28}^{(0){\rm eff}}}{2}\right) \ln\frac{\mu_W^2}{\mw^2} 
\eea
\be\label{GB981}
\Delta C^{(1)eff}_7(\mu_b)=
\sum_{i=1}^8 \left(\frac{2}{3}e_i  + 6 l_i \right) \eta^{a_i+1}
\ln\frac{\mu_W^2}{\mw^2} 
\ee
Here ($x=x_t$)
\bea
C_7^{(1)eff}(M_W) &=& \f{-16 x^4 -122 x^3 + 80 x^2 -  8 x}{9 (x-1)^4} 
{\rm Li}_2 \left( 1 - \f{1}{x} \right)
                  +\f{6 x^4 + 46 x^3 - 28 x^2}{3 (x-1)^5} \ln^2 x 
\nonumber \\ &&
                  +\f{-102 x^5 - 588 x^4 - 2262 x^3 + 3244 x^2 - 1364 x +
208} {81 (x-1)^5} \ln x
\nonumber \\ &&
                  +\f{1646 x^4 + 12205 x^3 - 10740 x^2 + 2509 x - 436}
{486 (x-1)^4} 
\vspace{0.2cm} \\
C_8^{(1)eff}(M_W) &=& \f{-4 x^4 +40 x^3 + 41 x^2 + x}{6 (x-1)^4} 
{\rm Li}_2 \left( 1 - \f{1}{x} \right)
                  +\f{ -17 x^3 - 31 x^2}{2 (x-1)^5} \ln^2 x 
\nonumber \\ &&
                  +\f{ -210 x^5 + 1086 x^4 +4893 x^3 + 2857 x^2 - 1994 x
+280} {216 (x-1)^5} \ln x
\nonumber \\ &&
        +\f{737 x^4 -14102 x^3 - 28209 x^2 + 610 x - 508}{1296 (x-1)^4}
\eea
and
\be
E_0(x) = \frac{x (18 -11
x - x^2)}{12 (1-x)^3} + \frac{x^2 (15 - 16 x + 4 x^2)}{6 (1-x)^4} \ln
x-\frac{2}{3} \ln x.
\ee

The formulae for $C_{7,8}^{(1)eff}(M_W)$ given above and presented in 
\cite{CZMM} are obtained from
the results in \cite{Yao1,GH97,BKP2,GAMB} by using the general formulae 
for the effective coefficient functions in (\ref{eq:defc7eff}) and
(\ref{eq:defc8eff}). For $\mu_W=\mu_t=\mw$ the formulae above 
reduce to the ones given
in \cite{CZMM}. We have put back the superscript "eff" in
(\ref{GENC7}) and (\ref{GENC8}) to emphasize that the effective
anomalous dimensions should be used here.

\begin{table}[htb]
\caption[]{Magic Numbers.
\label{tab:akh1}}
\begin{center}
\begin{tabular}{|r|r|r|r|r|r|r|r|r|}
\hline
$i$ & 1 & 2 & 3 & 4 & 5 & 6 & 7 & 8 \\
\hline
$a_i $&$ \frac{14}{23} $&$ \frac{16}{23} $&$ \frac{6}{23} $&$
-\frac{12}{23} $&$
0.4086 $&$ -0.4230 $&$ -0.8994 $&$ 0.1456 $\\
$e_i$ &$\frac{4661194}{816831}$&$ -\frac{8516}{2217}$ &$  0$ &$  0$ 
        & $ -1.9043$  & $  -0.1008$ & $ 0.1216$  &$ 0.0183$\\
$f_i$ & $-17.3023$ & $8.5027 $ & $ 4.5508$  & $ 0.7519$
        & $  2.0040 $ & $  0.7476$  &$ -0.5385$  & $ 0.0914$\\
$g_i$ & $14.8088$ &  $ -10.8090$  &$ -0.8740$  & $ 0.4218$ 
        & $  -2.9347$   & $ 0.3971$  & $ 0.1600$  & $ 0.0225$ \\
$l_i$ & $0.5784$ &  $ -0.3921$  &$ -0.1429$  & $ 0.0476$ 
        & $  -0.1275$   & $ 0.0317$  & $ 0.0078$  & $ -0.0031$ \\
\hline
\end{tabular}
\end{center}
\end{table}

The numbers $e_i$--$g_i$ and $l_i$ are given in table \ref{tab:akh1}. 
These  numbers as well as the numerical coefficients
in (\ref{c7eff1}) can be confirmed easily by using the anomalous dimension 
matrices  in \cite{CZMM} and the techniques developed in section 5. 

For completeness we give here some information on the relevant
NLO anomalous dimension matrix $\gamma^{(1)}_s$.
The $6\times 6$ two-loop submatrix of $\gamma^{(1)}_s$ involving
the operators $Q_1,\ldots,Q_6$ is given in (\ref{eq:gs1ndrN3Kpp}).
The two-loop generalization of (\ref{gammaB0}) has been calculated 
in \cite{MisMu:94}. It is given for both NDR and HV
schemes as follows
\begin{eqnarray}
\gamma^{(1)}_{77} &=& 
   C_F \left(\frac{548}{9} N - 16 C_F - \frac{56}{9} f \right)
\nn \\
\gamma^{(1)}_{78} &=& 0
\label{gammaB1} \\
\gamma^{(1)}_{87} &=& 
   C_F \left(-\frac{404}{27} N +\frac{32}{3} C_F +\frac{56}{27} f \right)
\nn \\
\gamma^{(1)}_{88} &=& -\frac{458}{9} -\frac{12}{N^2}+ \frac{214}{9} N^2 +
   \frac{56}{9} \frac{f}{N} - \frac{13}{9} f N
\nn
\end{eqnarray}

The generalization of (\ref{eq:g0127})--(\ref{eq:g0568}) to next-to-leading
order requires three loop calculations . The result can be found in
\cite{CZMM}.

The constants $r_i$ resulting from the calculations of NLO corrections
to decay matrix elements \cite{GREUB} are collected in \cite{CZMM}.
It should be stressed that the basis of the operators with $i=1-6$ used
in \cite{CZMM} differs from the standard basis used in the literature
\cite{BBL,GREUB} and here. The basis used in \cite{CZMM} has been chosen in 
order to avoid $\gamma_5$ problems in the three-loop calculations
peformed in the NDR scheme. This has to be remembered when using
formulae of this paper.
In particular the constants $r_i$ calculated in \cite{GREUB} have
to be transformed to the basis of \cite{CZMM}. As pointed out
this year in \cite{GAMB} and in particular by Kagan and Neubert 
\cite{KN98}
this tranformation made originally in \cite{CZMM} contained some errors.
The corrected values of $r_i$ can be found in the hep--version of
\cite{CZMM} and in \cite{KN98}. 
The numerical analysis given below is based on these
new values.

For the discussion below it will be useful to have \cite{CFMRS:93}
\be
\gamma_{27}^{(0){\rm eff}}=\f{416}{81} \qquad
\gamma_{28}^{(0){\rm eff}}=\f{70}{27}
\ee
which enter (\ref{GENC7}) and (\ref{GENC8}) respectively.
They can be obtained from (\ref{eq:g0127}) and (\ref{eq:g0128}).

\subsubsection{Going Beyond the Spectator Model}
In order to calculate the final rate we
have to pass from the calculated $b$-quark decay rates to 
the $B$-meson decay rates. Relying on the
Heavy Quark Expansion (HQE) calculations one finds \cite{CZMM}
%
\be \label{BR} 
Br(B{\to}X_s \gamma) = Br(B{\to}X_c e \bar{\nu_e})
\cdot R_{{\rm quark}} 
\left( 1 - \frac{\delta^{NP}_{sl}}{m_b^2}
         + \frac{\delta^{NP}_{rad}}{m_b^2} \right),
\ee
%
where $\delta^{{\rm NP}}_{{\rm sl}}$ and $\delta^{{\rm NP}}_{{\rm rad}} $
parametrize nonperturbative corrections to the semileptonic and
radiative $B$-meson decay rates, respectively. 

Following \cite{FLS96}, one can express
$\delta^{{\rm NP}}_{{\rm sl}}$ and $\delta^{{\rm NP}}_{{\rm rad}} $
in terms of the HQET parameters $\lambda_1$ and $\lambda_2$

\be
\delta^{{\rm NP}}_{{\rm sl}}  = \f{1}{2} \lambda_1 
+\left(\frac{3}{2}- \frac{6 (1-z)^4}{f(z)}\right) \lambda_2.
\ee

\be
\delta^{{\rm NP}}_{{\rm rad}}  = \f{1}{2} \lambda_1 - \f{9}{2} \lambda_2.
\ee
where $f(z)$ is given (\ref{g}).

The value of $\lambda_2$ is known from $B^*$--$B$ mass splitting
%
\be
\lambda_2 = \f{1}{4} ( m_{B^*}^2 - m_B^2 ) \simeq 0.12\;{\rm GeV}^2.
\ee
%
The value of $\lambda_1$ is controversial. Fortunately it cancels out
in the r.h.s. of (\ref{BR}).

The two nonperturbative corrections in (\ref{BR}) are
both around $4\%$ in magnitude and tend to cancel each other. In
effect, they sum up to only around $1\%$. As stressed in \cite{CZMM},
such a small number has
to be taken with caution. 
Indeed, one has to remember that the four-quark operators 
$Q_1...... Q_6$ have not been included in the calculation of
$\delta^{{\rm NP}}_{{\rm rad}}$. Contributions from these operators could
potentially give one- or two-percent effects. Nevertheless, it seems
reasonable to conclude that the total nonperturbative $1/m_b^2$ 
correction to
(\ref{BR}) is well below 10\%, i.e.\ it is smaller than the
inaccuracy of the perturbative calculation of $R_{{\rm quark}}$.

In additions to the $1/m_b^2$ corrections one has to consider
long distance contributions to $B\to X_s\gamma$. These are not easy
to calculate and until recently most estimates were based on phenomenological
models. In these model estimates long distance contributions 
are expected to arise dominantly 
from transitions $B \to \sum_i V_i+X_s \to \gamma X_s$ where
$V_i=J/\psi,\psi^\prime,...$ and are found  to be below
$10\%$ \cite{LDGAMMA}. 

A more modern way to estimate these long distance corrections
is to use heavy quark expansions treating the charm quark as
a heavy quark. As pointed out
by Voloshin \cite{VOL96} and also discussed by other
authors \cite{LRW97}, 
these non-perturbative corrections
originate in the photon coupling to a virtual $c\bar c$ loop and
their general structure is given by 
$$
(\Lambda_{\rm QCD}^2/\mc^2)(\Lambda_{\rm QCD}\mb/\mc^2)^n
$$
with $(n=0,1..)$. The term $n=0$ can be estimated reliably.
Originally a 3\% suppression of the decay rate by this term has been
found  in \cite{VOL96} 
Subsequently, however, an overall sign error 
in this estimate has been pointed out in \cite{BUC97} so that this
$1/m_c^2$ correction  is positive. 

Since
$\Lambda_{\rm QCD}\mb/\mc^2\approx 0.6 $, the terms with $n>0$
are not necessarily much smaller. Although the presence of
unknown matrix elements in these contributions does not
allow a definite estimate of their actual size, the analyses in
\cite{VOL96,LRW97} find that these contributions are weighted by
very small calculable coefficients. Consequently these higher order 
contributions
are expected to be  substantially smaller than the $n=0$ term 
and  the $3\%$ {\it enhancement} from $1/m_c^2$ corrections found in
\cite{BUC97} appears to be a good estimate of the long distance 
contributions to the $B\rightarrow X_s\gamma$ decay rate. 
We will include this enhancement in the numerical analysis below.

\subsubsection{Numerical Analysis at NLO}
Let us investigate how much the uncertainties in
(\ref{LOmu1}) are reduced after including NLO corrections.
We begin this discussion by demonstrating analytically
that the $\mu_b$, $\mu_W$ and $\mu_t$ dependences present in 
$C^{(0){\rm eff}}_{7}(\mu_b)$ are indeed cancelled at $\ord(\as)$
by the explicit scale dependent terms in (\ref{Dvirt}) and (\ref{GENC7}). 
The scale dependent terms in (\ref{GENC8}) do not enter this cancellation
at this order in $\as$ in $B\to X_s \gamma$. On the other hand
they are responsible for the cancellation of the scale dependences
in $C^{(0){\rm eff}}_{8}(\mu_b)$ relevant for the $b \to s~{\rm gluon}$
transition. 

\noindent
Expanding the three terms in (\ref{C7eff}) in $\as$ and
keeping the leading logarithms we find:
\be\label{E1}
\eta^\frac{16}{23} C_{7}^{(0)}(\mu_W)=
\left (1+\f{\as}{4\pi}\f{16}{3}\ln\f{\mu_b^2}{\mu^2_W}\right)
C_{7}^{(0)}(\mu_W)
\ee

\be\label{E2}
 \frac{8}{3}
   \left(\eta^\frac{14}{23} - \eta^\frac{16}{23}\right) 
C_{8}^{(0)}(\mu_W)= 
-\f{\as}{4\pi}\f{16}{9}\ln\f{\mu_b^2}{\mu^2_W}
C_{8}^{(0)}(\mu_W)
\ee

\be\label{E3}
\sum_{i=1}^8 h_i \eta^{a_i}= \f{\as}{4\pi}\f{23}{3}\ln\f{\mu_b^2}{\mu^2_W}
\sum_{i=1}^8 h_i a_i =\f{208}{81} \f{\as}{4\pi}\ln\f{\mu_b^2}{\mu^2_W}
\ee
respectively. In (\ref{E3}) we have used $\sum h_i=0$. 
Inserting these expansions into (\ref{Dvirt}),
we observe that the $\mu_W$ dependences
in (\ref{E1}), (\ref{E2}) and (\ref{E3})
are precisely cancelled  by the three  explicit logarithms in
(\ref{GENC7}) involving $\mu_W$, respectively. Similarly one can convince
oneself that the $\mu_t$-dependence of $C^{(0){\rm eff}}_{7}(\mu_b)$
is cancelled at $\ord(\as)$ by 
the $\ln \mu_t^2/\mw^2$ term in (\ref{GENC7}).
Finally and most importantly the $\mu_b$ dependences in
(\ref{E1}), (\ref{E2}) and (\ref{E3}) are cancelled by the explicit
logarithms in (\ref{Dvirt}) which result from the calculation of
the one-loop
matrix elements $<s\gamma|Q_{7\gamma}|B>$ and 
$<s\gamma|Q_{8G}|B>$ and the two-loop matrix element
$<s\gamma|Q_2|B>$ as discussed previously. Interestingly
the scale dependent term in (\ref{GB981}) does not contribute to
any cancellation of the $\mu_W$ dependence
at this order in $\as$ due to the relation
\be
\sum_{i=1}^8 \left(\frac{2}{3}e_i  + 6 l_i \right)=0.
\ee
which can be verified by using the table \ref{tab:akh1}.

Clearly there remain small $\mu_b$, $\mu_W$ and $\mu_t$ dependences in
(\ref{ration}) which can only be reduced by going beyond the NLO
approximation. They constitute the theoretical uncertainty which
should be taken into account in estimating the error in the
prediction for $Br(B\to X_s\gamma)$. For this reason also the term
$\Delta C^{(1)eff}_7(\mu_b)$ in (\ref{GENC7}), originally omitted in
\cite{BKP1}), has to be kept as pointed out in \cite{BG98}.

\noindent
Using the two-loop generalization of $(\ref{mbar})$ from Section 4.7
and varying $\mu_b$, $\mu_W$ and $\mu_t$ in the ranges (\ref{ranges1})
and (\ref{ranges}) one
finds \cite{BKP1} the following respective uncertainties in the branching
ratio after the inclusion of NLO corrections:
\begin{equation}\label{NLOm}
\Delta Br(B\to X_s \gamma)=\left\{ \begin{array}{ll}
\pm 4.3\% & (\mu_b) \\
\pm 1.1\% & (\mu_W) \\
\pm 0.4\% & (\mu_t) \end{array} \right.
\end{equation}

This reduction of the $\mu_b$-uncertainty by roughly a factor of seven 
 relative to $\pm 22\%$ in LO is impressive.
The remaining $\mu_W$ and $\mu_t$ uncertainties are negligible.

Next we would like to comment on the uncertainty due to variation of
$\bar\mu_b$ in $\kappa(z)$ given in (\ref{kap}). In \cite{GREUB}
the choice $\bar\mu_b=\mu_b$ has been made. Yet in my opinion
such a treatment is not really correct, since the scale $\bar\mu_b$ in
the semi-leptonic decay has nothing to do with the scale $\mu_b$
in the renormalization group evolution in the $B\to X_s\gamma$
decay. 
Varying $\bar\mu_b$ in the range $2.5\gev\le\mu_b\le 10\gev$ we find
\begin{equation}\label{NLOm1}
\Delta Br(B\to X_s \gamma)=\pm 1.7\% \quad (\bar\mu_b)
\end{equation}
Since the $\mu_b$ and $\bar\mu_b$ uncertainties are uncorrelated we
can add them in quadrature finding $\pm 4.6\%$ for the total
scale uncertainty due to $\mu_b$ and $\bar\mu_b$. 
The addition of the uncertainties in $\mu_t$ and $\mu_W$ in
(\ref{NLOm}) modifies this result slightly and the total scale 
uncertainty in $Br(B \to X_s\gamma)$ amounts then to
\be\label{stheon}
\Delta Br(B{\to}X_s \gamma) = \pm 4.8\% \quad({\rm scale})
 \ee

It should be stressed that this pure theoretical 
uncertainty related to the truncation of the perturbative series
should be distinguished from parametric uncertainties related
to $\alpha_s$, the quark masses etc. discussed below.

In our numerical calculations we have included all corrections
in the NLO approximation. To work  consistently
in this order, we have in particular
expanded the various factors in (\ref{ration}) in  $\alpha_s$ and discarded
all NNLO terms of order $\alpha_s^2$ which resulted in the process
of multiplication. This treatment is different
from \cite{CZMM,GREUB}, where the $\alpha_s$ corrections in (\ref{factor}) 
have not been expanded in the evaluation of 
(\ref{ration}) and therefore some higher order corrections have been kept.
Different scenarios of partly incorporating higher order corrections
by expanding or not expanding various factors in (\ref{ration})
affect the branching ratio by $\Delta Br(B\rightarrow X_s\gamma)\approx
\pm 3.0 \%$. This number indicates that indeed the scale uncertainty
in (\ref{stheon}) realistically  estimates the magnitude of yet
unknown higher order corrections.   
The remaining uncertainties are due to the values of the various 
input parameters.
In order to obtain the final result for the branching ratio
we have used  the  parameters given in table \ref{tabbsg}.

\begin{table}[htb]
\caption[]{Input parameter values and their uncertainties.
The masses are given in GeV.
\label{tabbsg}}
\begin{center}
\begin{tabular}{|c|c|c|c|c|c|c|c|}
\hline
 & $\as(M_Z)$ & $m_{t,pole}$   & $m_{c,pole}/m_{b,pole}$ 
& $m_{b,pole}$ & $\alpha_{em}^{-1}$ 
& $|V_{ts}^{\star}V_{tb}|/V_{cb}$ & $Br(B\to X_c e\bar\nu_e)$ \\
\hline
{\rm Central} & $0.118$ & $176$  & $0.29$ & $4.8$  &  
  $130.3$  & $0.976$ &$0.104$ \\
\hline
{\rm Error} & $\pm 0.003$   & $\pm 6.0$  & $\pm 0.02$ & $\pm 0.15$  &  
  $\pm 2.3$  & $\pm 0.010$ &$\pm 0.004$ \\
\hline
\end{tabular} 
\end{center}
\end{table}

\begin{table}[htb]
\caption[]{Uncertainties in $Br(B \to X_s\gamma)$ due to various 
sources.\label{tab:akh2}}
\begin{center}
\begin{tabular}{|c|c|c|c|c|c|c|c|}
\hline
{\rm Scales} & $\as(M_Z)$ & $m_{t,pole}$  & $m_{c,pole}/m_{b,pole}$ 
& $m_{b,pole}$ & $\alpha_{em}$ & CKM angles & $B\to X_c e\bar\nu_e$ \\
\hline
$\pm 4.8\%$ & $\pm 2.9\%$   & $\pm 1.7\%$  & $\pm 5.4\%$ & $\pm 0.7\%$  &  
  $\pm 1.8\%$  & $\pm 2.0\%$ &$\pm  3.8\%$ \\
\hline
\end{tabular} 
\end{center}
\end{table}

Adding all the uncertainties 
in quadrature we find  
\be\label{sfin}
Br(B{\to}X_s \gamma) =(3.60 \pm 0.17~({\rm scale})~\pm 0.28~({\rm par})) 
  \times 10^{-4}
= (3.60 \pm 0.33)  \times 10^{-4}
\ee
where we show separately scale and parametric uncertainties.
The relative importance of various
uncertainties is shown in table \ref{tab:akh2}. Similar results
can be found in \cite{CZMM,BG98}. 
We observe that inclusion of NLO corrections
increased the value of the LO prediction in (\ref{LORES}) by 
roughly $25\%$. Simultaneously the total error has been decreased
by more than a factor of two. The shift upwards is mainly caused by the
$\ord(\alpha_s)$ corrections to the matrix elements of the
contributing operators calculated in \cite{GREUB}, 
not to the Wilson coefficients.
One has to remember, however, that this feature is valid in
the NDR scheme considered here and may not be true in another
renormalization scheme without changing the total result
for the decay rate.

We also observe that 
the parametric uncertainties dominate the theoretical
error at present. Once these parametric uncertainties will be reduced
in the future the smallness of the scale uncertainties achieved
through very involved QCD calculations, in particular in 
\cite{CZMM,GREUB,AG2,Pott,Yao1,GH97,BKP2}, can be better appreciated.
This reduction of the theoretical error in the Standard Model
prediction for $Br(B{\to}X_s \gamma)$ could turn out to be very
important in the searches for new physics when the experimental
data improve. 
\subsection{Recent Developments}
Very recently electroweak $\ord(\alpha)$ corrections to $R_{\rm quark}$
have been investigated in an interesting paper by a student of 
this school, Andrzej Czarnecki, and Bill Marciano \cite{CZMA}.
A study of $\ord(\alpha)$ corrections to $R_{\rm quark}$ must
entail two-loop electroweak contributions to $b \to s\gamma$
as well as one loop corrections to $b\to ce\nu$. A complete
calculation of all $\ord(\alpha)$ contributions would be a very
heroic task, but it is already valuable to identify potentially
dominant contributions.

One obvious question is the scale $\mu$ in 
$\alpha_{\rm em}\equiv e^2(\mu)/4\pi$ which is rather arbitrary
if corrections $\ord(\alpha)$ are not considered. In all recent
calculations $\mb\le \mu\le\mw$ has been used, giving
$1/\alpha_{\rm em}=130.3\pm 2.3$. The inclusion of fermion
loop contributions in the photon propagator indicates \cite{CZMA},
however, that $\alpha$ renormalized at $q^2=0$, i.e
$\alpha=1/137.036$ is more appropriate. This reduces the branching
ratio by roughly $5\%$. The fermion loops in the W-propagator
bring a reduction of $2\%$. Two other reductions, 
each of roughly $1\%$,
come from short distance photonic corrections to $b\to s\gamma$
and $b\to ce\nu$. The total reduction of $R_{\rm quark}$ 
quoted in \cite{CZMA} amounts then to $(9\pm2)\%$ where the
error is a guess-estimate of unknown corrections.
With this reduction
the branching ratio in (\ref{sfin}) becomes
\be\label{sfincm}
Br(B{\to}X_s \gamma) 
= (3.28 \pm 0.30)  \times 10^{-4}~.
\ee
The $\pm 2\%$ error in the estimate of $\ord(\alpha)$ corrections
is compensated by the fact that $\alpha$ has a negligible error
compared to $\alpha_{em}$ in table~\ref{tabbsg}.
Personally, I am not yet convinced that the $\ord(\alpha)$ 
reduction is as high as $9\%$. The reduction of $5\%$ through
the change $\alpha_{\rm em}\to \alpha$ appears rather plausible.
On the other hand the same sign of three smaller corrections
could turn out to be accidental and other corrections, not considered
yet, could well cancel them. 
Some indication for this is given by a very recent analysis of
Strumia \cite{STRUMIA}, who performed a complete calculation of
two--loop electroweak contributions to $B\to X_s\gamma$ in the
large $\mt$ limit, finding them below $1\%$.
In spite of this reservation, the
calculation of Czarnecki and Marciano certainly indicates that
a reduction of $Br(B{\to}X_s \gamma)$ through $\ord(\alpha)$ corrections
by $\ord(5\%)$ is certainly possible. A more detailed investigation
of this issue would be desirable at some stage in the future. 

Finally I would like to mention here a very recent paper of Kagan and
Neubert \cite{KN98} who also made an extensive analysis of $B\to X_s\gamma$.
Reanalyzing in detail the issue of the 
photon-spectrum and of $\delta$ in (\ref{phs}) and including also
QED corrections, Kagan and Neubert arrive
at the estimate of $Br(B{\to}X_s \gamma)$, 
which in spite of some differences at intermediate stages agrees very
well with (\ref{sfincm}). Since the analysis in \cite{KN98} is very
recent, I am not in a position to make any useful comments on it.
Certainly of interest is their reanalysis of the extraction of
the total decay rate $Br(B{\to}X_s \gamma)$ from the experimental 
photon spectrum, which I will briefly mention below.

\subsection{Experimental Status}
After all this theoretical exposition it is really time to summarize
the present data.
The branching ratio for $B \to X_s \gamma$  found 
by the CLEO collaboration already in 1994 \cite{CLEO2} is given by
\begin{equation}\label{EXP}
Br(B \to X_s\gamma) = (2.32 \pm 0.57 \pm 0.35) \times 10^{-4}
\end{equation}
and the very recent preliminary update from CLEO reads \cite{CLEO98}
\begin{equation}\label{EXP98}
Br(B \to X_s\gamma) = (2.50 \pm 0.47 \pm 0.39) \times 10^{-4}\,.
\end{equation}
On the other hand the recent ALEPH measurement of the corresponding
branching ratio for b--hadrons (mesons and baryons) produced
in $Z^0$ decays
reads \cite{ALEPH}
\begin{equation}\label{EXP2}
Br(H_b \to X_s\gamma) = (3.11 \pm 0.80 \pm 0.72) \times 10^{-4}.
\end{equation}
which is compatible with the CLEO result.
In (\ref{EXP})-(\ref{EXP2})
 the first error is statistical and the second is systematic.
As stressed already by several authors in the literature the
measurements in \cite{CLEO2} and \cite{ALEPH} are quite different
and should not be expected to give identical results.

Now, the experimental results given above, are obtained by
measuring the high-energy part of the photon spectrum and the
extrapolation to the total rate. This requires theoretical
understanding of the photon spectrum. Improving recently the
analysis of the photon spectrum, Kagan and Neubert \cite{KN98}
find that the result in (\ref{EXP}) should actually read
\begin{equation}\label{EXPKN}
Br(B \to X_s\gamma) = (2.66 \pm 0.56_{\rm exp} \pm 0.45_{\rm th})
\times 10^{-4}\,,
\end{equation}
and that the central value in (\ref{EXP98}) should be increased
to 2.8. It will be interesting to watch the further development
and to have a new official CLEO value including this new insight.

The theoretical estimates in (\ref{sfin}) and (\ref{sfincm}) 
are somewhat higher than experimental data.
However, within
 the remaining theoretical
and in particular experimental uncertainties, the Standard Model value
is compatible with experiment. 
\subsection{A Look Beyond the Standard Model}
The
inclusive  radiative \Bsg decay  plays an important role in the 
indirect searches 
for physics beyond the Standard Model and 
places already now rather strong constraints on some new physics
scenarios. 
The possible non-standard contributions can indeed be of the same order of
magnitude of the Standard Model loop effects discussed above.
This is well illustrated by the simplest of these extensions,
where the Higgs sector of the Standard Model is enlarged to include two
doublets (Two Higgs Doublet Models, or 2HDM), 
leading to three new  physical fields, two  neutral  scalars 
(CP even and odd) and one charged scalar.
 In this context, 
only the charged Higgs $H^\pm$ contributes to the Wilson 
coefficients $C_{7\gamma}$ and  $C_{8G}$.
 Its interaction with  quarks  is described by the Lagrangian
\be
{\cal L}= (2\sqrt{2}G_F)^{1/2}\sum_{i,j=1}^3\bar u_i
\left( A_u m_{u_i}V_{ij}\frac{1-\gamma_5}{2}-
A_d V_{ij}m_{d_i}\frac{1+\gamma_5}{2}\right) d_j H^++{\rm h.c.}
\ee
Here $i,j$ are generation indices, $m_{u,d}$ are quark masses, and $V$ is the
CKM matrix. 
The fermions may then acquire their masses in two ways:
the first possibility, referred to as Model I, is that both up and down quarks
get their masses from the same Higgs doublet $H_2$, and
\be
A_u=A_d=1/\tan \beta ~,
\ee
where $\tan \beta$ is the ratio of the v.e.v. of $H_2$ and $H_1$.
In the case of
Model II, up quarks get their masses from Yukawa couplings to $H_2$,
while down quarks get masses from couplings to $H_1$, and
\be
A_u=-1/A_d=1/\tan \beta ~.
\ee 
Model II is particularly interesting because it is realized in the minimal
supersymmetric extension of the SM.
The charged-Higgs contributions to $C_{7\gamma}$ and  $C_{8G}$
are functions of the top and charged Higgs masses and of $\tan\beta$
whose LO expressions are given in \cite{chwil}.
Recently, the complete NLO corrections to $Br(B\to X_s\gamma)$ 
in the 2HDM have been computed
\cite{GAMB,BG98}. Partial results can also be found in \cite{strum}.

With respect to the Standard Model, in Model II the branching ratio is
strongly enhanced for a light charged Higgs and the decoupling 
at large $M_H$ takes place very slowly.
 This leads to very stringent bounds on
${\rm M_H}$ for any 
particular value of $\tan\beta$.
Actually, for $\tan\beta>1$, the dependence on $\tan\beta$ is very mild
and practically saturates for $\tan \beta \ge 2$.
Using the current  CLEO 95\% CL upper bound 
$Br(B\to X_s\gamma) < 4.2 \times 10^{-4}$
and adopting a conservative approach to evaluate the theoretical uncertainty
(scanning), 
one obtains lower bounds on ${\rm M_H}$ of $\approx 250$GeV,
independent of $\tan\beta$ \cite{GAMB,BG98}.
On the other hand, 
adding different theoretical errors in quadrature leads to 
${\rm M_H}>370$GeV. 
Indeed, these bounds are quite sensitive to
the errors of the theoretical prediction and to the details of the
calculation \cite{GAMB}.
For instance including Czarnecki-Marciano $\ord({\alpha})$ corrections
would weaken them significantly.
 Improving the calculation to
the NLO has also had 
important effects on these bounds, since the theoretical error
is significantly reduced \cite{GAMB}.
Finally, it is  clear that
 one of the reasons we are  able to obtain such strong bounds on
${\rm M_H}$ is the poor agreement between the Standard Model prediction 
and CLEO measurement, and
that the situation may drastically change with new experimental results.
In the case of a heavy Higgs, 
a resummation of the leading logarithms of ${\rm M_H}/\mw$ has been 
performed in \cite{anl}.

For what concerns Model I, in that case 
the charged-Higgs contribution reduces the 
value of $Br(B\to X_s\gamma)$ and therefore no significant bound can be 
obtained. On
the other hand, it is interesting that new physics effects can
bring the prediction for \Bsg closer to the CLEO value. 
A significant effect can only be expected for small $\tan \beta$, since
in Model I all charged-Higgs contributions vanish in the limit of
large $\tan \beta$, as $\tan^{-2} \beta$. However, in that case the top Yukawa
coupling grows and strong limitations come from high energy measurements, in
particular of $R_b$. It can be concluded \cite{GAMB} that 
the reduction of $Br(B\to X_s\gamma)$  can be at best about 20{\%}. 

A more general class of multi-Higgs models, where only one charged Higgs does
not decouple and its couplings are arbitrary and may violate CP, 
 has been studied at LO in \cite{multiH}
and more recently at NLO in \cite{BG98}.

In the MSSM, the charged Higgs loops are accompanied by chargino-squark 
contributions which can partly compensate the
effect of the charged Higgs. Therefore the above bounds do not apply to the
MSSM, except in some scenarios, like gauge-mediated supersymmetric models 
\cite{rattazzi}, where the Higgs contribution is known to dominate over the
chargino loops, because the squarks are generally heavy.
 Indeed, in the supersymmetric limit, there
is an exact cancellation of different contributions \cite{io}. In the
realistic case of broken supersymmetry, this cancellation is spoiled but,
if charginos and stops are light, it may still be partially effective.
A complete analysis at LO in the MSSM can be found in 
\cite{berto}. Although no direct limit on ${\rm M_H}$ 
can be set, $b\to s\gamma$ has
helped in ruling out very large portions of the SUSY parameter space.
It can be expected that a NLO analysis would further enhance  this 
exclusion potential.
\subsection{Summary and Outlook}
The rare decay $B\to X_s\gamma$ plays at present together with
$B^0_{d,s}-\bar B^0_{d,s}$ mixing the central role
in loop induced transitions in the $B$-system. On the theoretical
side considerable progress has been made recently by calculating
NLO corrections, thereby reducing the large $\mu_b$ uncertainties
present in the leading order. This way the error in the
prediction for $Br(B\to X_s\gamma)$ as
given in (\ref{sfin}), and in  (\ref{sfincm}) 
after including QED corrections, 
has been decreased down to roughly
$\pm 10\%$ compared with $\pm (25-30)\%$ in the leading order.
Since during last two years the central value for 
$Br(B\to X_s\gamma)$ was changing constantly due to inclusion
of various small corrections and different error analyses, 
it is hard to imagine that the
result in  (\ref{sfincm}) is the final word
on this subject. It will be interesting to see how this value
will look like in five years from now. 

On the experimental side considerable progress has been made
by CLEO \cite{CLEO96} in the case of $Br (B^0_d\to K^*\gamma)$, which
we left out due to space limitations.
 It is very
desirable to obtain now an improved measurement of 
$Br(B\to X_s\gamma)$.
Indeed,
in the forthcoming years much more
precise measurements of $Br(B{\to}X_s \gamma)$ are expected from the
upgraded CLEO detector, as well as from the B-factories at SLAC and KEK.

Confrontation of these new
improved experimental values with the already rather precise theoretical
Standard Model estimate may shed some light on whether some
physics beyond this model is required to fit the improved data.

More
on $B\to X_s\gamma$, in particular on the photon spectrum and 
the determination
of $\vtd/\vts$ from $B\to X_{s,d}\gamma$, can be found in
\cite{ALUT,ALIB,Photon,KN98}. 
CP violation in $B \to K^* \gamma$ and $B \to \varrho \gamma$ 
is discussed in \cite{GSW95}.

\section{ Rare $K$- and $B$-Decays}
         \label{sec:HeffRareKB}
\setcounter{equation}{0}
%\setcounter{figure}{0}
%\setcounter{table}{0}
\subsection{General Remarks}
            \label{sec:HeffRareKB:overview}
We will now move to discuss
the semileptonic rare FCNC
transitions $\kpn$,  $K_{\rm L}\to\pi^0\nu\bar\nu$, $B\to X_{s,
d}\nu\bar\nu$ and $B_{s,d}\to l^+l^-$ paying particular attantion
to the first two decays.
The presentation given here overlaps considerably with the ones given
in the reviews \cite{BBL,BF97}, although there are some differences.
In particular the  decay $\klm$ will not be considered here in view of 
space limitations.
Some details on this decay, which is not as theoretically clean as the ones 
discussed here, can be found  in the latter reviews and in \cite{CPRARE}.
On the other hand we will provide certain derivations which cannot
be found in \cite{BBL,BF97}. 
Moreover we discuss briefly two--loop electroweak contributions and
make a few remarks on the physics beyond the Standard Model.

The decay modes considered here are very
similar in their structure which differs considerably from the one
encountered in
the decays $K \to \pi\pi$ and
 $B \to X_s \gamma$  discussed in previous
sections. In particular: 

\begin{itemize}
\item
Within the Standard Model all the decays listed above are loop-induced
semileptonic FCNC processes determined only 
by $Z^0$-penguin and box diagrams which we encountered already
on many occasions in these lectures.
Thus, a distinguishing feature of the present class of decays
is the absence of a photon penguin contribution. For the decay modes
with neutrinos in the final state this is obvious, since the photon
does not couple to neutrinos. For the mesons decaying into a charged
lepton pair the photon penguin amplitude vanishes due to vector current
conservation. Consequently the decays in question are governed by the
functions $X_0(x_t)$ and $Y_0(x_t)$ (see (\ref{X0}) and (\ref{Y0}))
which as seen in (\ref{PBE1}) and (\ref{PBE2}) exhibit strong 
$\mt$-dependences.
\item
A particular and very important advantage of these decays
is their clean theoretical character.
This is related to the fact that
the low energy hadronic
matrix elements required are just the matrix elements of quark currents
between hadron states, which can be extracted from the leading
(non-rare) semileptonic decays. Other long-distance contributions
are negligibly small. As a consequence of these features,
the scale ambiguities, inherent to perturbative QCD, essentially
constitute  the only theoretical uncertainties 
present in the analysis of these decays.
These theoretical uncertainties have been considerably reduced
through the inclusion of
the next-to-leading QCD corrections 
 \cite{BB1,BB2,BB3} as we will demonstrate below. 
\item
The investigation of these low energy rare decay processes in
conjunction with their theoretical cleanliness, allows to probe,
albeit indirectly, high energy scales of the theory and in particular
to measure the top quark couplings $V_{ts}$ and $V_{td}$.
Moreover $\klpn$  offers
a clean determination of the Wolfenstein parameter $\eta$ and 
as we will stress below offers the cleanest measurement
of $\IM\lambda_t= \IM V^*_{ts} V_{td}$ which governs all  
CP violating  $K$-decays. 
However, the very fact
that these processes are based on higher order electroweak effects
implies
that their branching ratios are expected to be very small and not easy to
access experimentally.
\end{itemize}

\begin{table}[htb]
\caption[]{
Order of magnitude of CKM parameters relevant for the various decays,
expressed in powers of the Wolfenstein parameter $\lambda=0.22$. In the
case of $K_{\rm L}\to\pi^0\nu\bar\nu$, which is CP-violating, only the
imaginary parts of $\lambda_{c, t}$ contribute.
\label{tab:lambdaexp}}
\begin{center}
\begin{tabular}{|r|c|c|c|c|}
\hline
&$\kpn$&$K_{\rm L}\to\pi^0\nu\bar\nu$&$B\to X_s\nu\bar\nu$&
$B\to X_d\nu\bar\nu$\\
&~&~&$B_s\to l^+l^-$&$B_d\to l^+l^-$\\  \hline
$\lambda_c$&$\sim\lambda$&(${\rm Im}\lambda_c\sim\lambda^5$)&
$\sim\lambda^2$&$\sim\lambda^3$\\  \hline
$\lambda_t$&$\sim\lambda^5$&(${\rm Im}\lambda_t\sim\lambda^5$)&
$\sim\lambda^2$&$\sim\lambda^3$ \\
\hline
\end{tabular}
\end{center}
\end{table}

The effective Hamiltonians governing the decays
$\kpn$, $K_{\rm L}\to\pi^0\nu\bar\nu$,
$B\to X_{s, d}\nu\bar\nu$ and $B\to l^+l^-$
resulting from the $Z^0$-penguin and box-type contributions can all be
written in the following general form:
\begin{equation}\label{hnr} 
{\cal H}_{\rm eff}={G_{\rm F} \over{\sqrt 2}}{\alpha\over 2\pi 
\sin^2\Theta_{\rm W}}
 \left( \lambda_c F(x_c) + \lambda_t F(x_t)\right)
 (\bar nn^\prime)_{V-A}(\bar rr)_{V-A}\,,  \end{equation}
where $n$, $n^\prime$ denote down-type quarks
($n, n^\prime=d, s, b$ but $n\not= n^\prime$) and $r$ leptons,
$r=l, \nu_l$ ($l=e, \mu, \tau$). The $\lambda_i$ are products of CKM elements,
in the general case $\lambda_i=V^*_{in}V_{in^\prime}^{}$. Furthermore
$x_i=m^2_i/M^2_W$.
The functions $F(x_i)$ describe the dependence on the internal
up-type quark masses $m_i$ (and on lepton masses if necessary)
and are understood to include QCD corrections.
They are increasing functions of the quark masses, a property that is
particularly important for the top contribution.
Since $F(x_c)/F(x_t)\approx
\ord(10^{-3})\ll 1$ the top contributions are by far dominant unless there
is a partial compensation through the CKM factors $\lambda_i$. 
 As seen in
table~\ref{tab:lambdaexp} such a partial compensation takes place in
$\kpn$  and consequently in this decay internal
charm contribution, albeit smaller than the top contribution,
has to be kept. On the other hand in the remaining decays the
charm contributions can be safely neglected. Since the charm contributions
involve QCD corrections with $\as(m_c)$, the scale uncertainties in 
$\kpn$  are found to be larger 
than in the remaining decays in which the QCD effects enter only
through $\as(m_t) < \as(m_c)$.
After these general remarks let us enter some details. Other reviews
of rare decays can be found in \cite{CPRARE,BF97}.
\subsection{The Decay \kpnn}
            \label{sec:HeffRareKB:kpnn}
\subsubsection{The effective Hamiltonian}
The effective Hamiltonian for $\kpn$  can
be written as
\begin{equation}\label{hkpn} 
{\cal H}_{\rm eff}={G_{\rm F} \over{\sqrt 2}}{\alpha\over 2\pi 
\sin^2\Theta_{\rm W}}
 \sum_{l=e,\mu,\tau}\left( V^{\ast}_{cs}V_{cd} X^l_{\rm NL}+
V^{\ast}_{ts}V_{td} X(x_t)\right)
 (\bar sd)_{V-A}(\bar\nu_l\nu_l)_{V-A} \, .
\end{equation}
The index $l$=$e$, $\mu$, $\tau$ denotes the lepton flavour.
The dependence on the charged lepton mass resulting from the box-graph
is negligible for the top contribution. In the charm sector this is the
case only for the electron and the muon but not for the $\tau$-lepton.

We have discussed the top quark contribution already in section 8.2
but there is no harm when we repeat certain things in order to
have the most important information about this decay in one place.

The function $X(x_t)$ relevant for the top part is given by
\begin{equation}\label{xx9} 
X(x_t)=X_0(x_t)+\aspi X_1(x_t) 
\end{equation}
with the leading contribution $X_0(x)$ given in (\ref{X0})
and the QCD correction \cite{BB2}
\begin{eqnarray}\label{xx1}
X_1(x_t)=&-&{23x_t+5x_t^2-4x_t^3\over 3(1-x_t)^2}
+{x_t-11x_t^2+x_t^3+x_t^4\over (1-x_t)^3}\ln x_t
\nonumber\\
&+&{8x_t+4x_t^2+x_t^3-x_t^4\over 2(1-x_t)^3}\ln^2 x_t
-{4x_t-x_t^3\over (1-x_t)^2}L_2(1-x_t)
\nonumber\\
&+&8x_t{\partial X_0(x_t)\over\partial x_t}\ln x_\mu\,,
\end{eqnarray}
where $x_\mu=\mu_t^2/M^2_W$ with $\mu_t=\ord(m_t)$ and
\begin{equation}\label{l2UU} 
L_2(1-x)=\int^x_1 dt {\ln t\over 1-t}   \,.
\end{equation}
The $\mu_t$-dependence of the last term in (\ref{xx1}) cancels to the
considered order the $\mu_t$-dependence of the leading term 
$X_0(x_t(\mu))$.
The leftover $\mu_t$-dependence in $X(x_t)$ is tiny and will be given
in connection with the discussion of the branching ratio below.

The function $X$ in (\ref{xx9})
can also be written as
\begin{equation}\label{xeta9}
X(x_t)=\eta_X\cdot X_0(x_t), \qquad\quad \eta_X=0.985,
\end{equation}
where $\eta_X$ summarizes the NLO corrections represented by the second
term in (\ref{xx9}).
With $\mt\equiv \mtb(\mt)$ the QCD factor $\eta_X$
is practically independent of $m_t$ and $\Lambda_{\overline{MS}}$.

The expression corresponding to $X(x_t)$ in the charm sector is the function
$X^l_{\rm NL}$. It results from the NLO calculation \cite{BB3} and is given
explicitly in \cite{BB3,BBL}.
The inclusion of NLO corrections reduced considerably the large
$\mu_c$ dependence
(with $\mu_c={\cal O}(m_c)$) present in the leading order expressions
for the charm contribution
 \cite{novikovetal:77,ellishagelin:83,dibetal:91,PBE0}.
Varying $\mu_c$ in the range $1\gev\le\mu_c\le 3\gev$ changes $X_{\rm NL}$
by roughly $24\%$ after the inclusion of NLO corrections to be compared
with $56\%$ in the leading order. Further details can be found in
\cite{BB3,BBL}. The impact of the $\mu_c$ uncertainties on the
resulting branching ratio $Br(\kpn)$ is discussed below.

The
numerical values for $X_{\rm NL}$ for $\mu = \mc$ and several values of
$\Lms^{(4)}$ and $\mc(\mc)$ are given in table \ref{tab:xnlnum}. 
The net effect of QCD corrections is to suppress the charm contribution
by roughly $30\%$.

\begin{table}[htb]
\caption[]{The functions $X^e_{\rm NL}$ and $X^\tau_{\rm NL}$
for various $\Lms^{(4)}$ and $\mc$.
\label{tab:xnlnum}}
\begin{center}
\begin{tabular}{|c|c|c|c|c|c|c|}
\hline
& \multicolumn{3}{c|}{$X^e_{\rm NL}/10^{-4}$} &
  \multicolumn{3}{c|}{$X^\tau_{\rm NL}/10^{-4}$} \\
\hline
$\Lms^{(4)}\ [\mev]\;\backslash\;\mc\ [\gev]$ &
1.25 & 1.30 & 1.35 & 1.25 & 1.30 & 1.35 \\
\hline
245 & 10.32  & 11.17  & 12.04 & 6.94 & 7.63 & 8.36 \\
285 & 10.02  & 10.86  & 11.73 & 6.64 & 7.32 & 8.04 \\
325 &  9.71  & 10.55  & 11.41 & 6.32 & 7.01 & 7.72 \\
365 &  9.38  & 10.22  & 11.08 & 6.00 & 6.68 & 7.39 \\
405 &  9.03  &  9.87  & 10.72 & 5.65 & 6.33 & 7.04 \\
\hline
\end{tabular}
\end{center}
\end{table}

\subsubsection{Deriving the Branching Ratio}
The relevant hadronic
matrix element of the weak current $(\bar sd)_{V-A}$ can be extracted,
with the help of isospin symmetry from
the leading decay $K^+\to\pi^0e^+\nu$.
Consequently the resulting theoretical
expression for  the branching fraction $Br(K^+\to\pi^+\nu\bar\nu)$ can
be related to the experimentally well known quantity
$Br(K^+\to\pi^0e^+\nu)$. Let us demonstrate this.

The effective Hamiltonian for the tree level decay $K^+\to\pi^0 e^+\nu$
is given by
\begin{equation}\label{kp0} 
{\cal H}_{\rm eff}(K^+\to\pi^0 e^+\nu)
={G_{\rm F} \over{\sqrt 2}}
 V^{\ast}_{us}
 (\bar su)_{V-A}(\bar\nu_e e)_{V-A} \, .
\end{equation}
Using isospin symmetry we have
\be\label{iso1}
\langle \pi^+|(\bar sd)_{V-A}|K^+\rangle=\sqrt{2}
\langle \pi^0|(\bar su)_{V-A}|K^+\rangle.
\ee
Consequently neglecting differences in the phase space of these two decays,
due to $m_{\pi^+}\not=m_{\pi^0}$ and $m_e\not=0$, we find 
\be\label{br1}
\frac{Br(\kpn)}{Br(K^+\to\pi^0 e^+\nu)}=
{\alpha^2\over |V_{us}|^2 2\pi^2 
\sin^4\Theta_{\rm W}}
 \sum_{l=e,\mu,\tau}\left| V^{\ast}_{cs}V_{cd} X^l_{\rm NL}+
V^{\ast}_{ts}V_{td} X(x_t)\right|^2~.
\end{equation}
\subsubsection{Basic Phenomenology}
We are now ready to present the expression for the branching fraction
$Br(\kpn)$ and to collect various formulae relevant for phenomenological
applications.
Using (\ref{br1}) 
and including isospin breaking corrections one finds
\begin{equation}\label{bkpn}
Br(\kpn)=\kappa_+\cdot\left[\left({\imlt\over\lambda^5}X(x_t)\right)^2+
\left({\relc\over\lambda}P_0(X)+{\relt\over\lambda^5}X(x_t)\right)^2
\right]~,
\end{equation}
\begin{equation}\label{kapp}
\kappa_+=r_{K^+}{3\alpha^2 Br(K^+\to\pi^0e^+\nu)\over 2\pi^2
\sin^4\Theta_{\rm W}}
 \lambda^8=4.11\cdot 10^{-11}\,,
\end{equation}
where we have used
\begin{equation}\label{alsinbr}
\alpha=\frac{1}{129},\qquad \sin^2\Theta_{\rm W}=0.23, \qquad
Br(K^+\to\pi^0e^+\nu)=4.82\cdot 10^{-2}\,.
\end{equation}
Here $\lambda_i=V^\ast_{is}V_{id}$ with $\lambda_c$ being
real to a very high accuracy. $r_{K^+}=0.901$ summarizes isospin
breaking corrections in relating $\kpn$ to $K^+\to\pi^0e^+\nu$.
These isospin breaking corrections are due to quark mass effects and 
electroweak radiative corrections and have been calculated in
\cite{MP}. Next
\begin{equation}\label{p0k}
P_0(X)=\frac{1}{\lambda^4}\left[\frac{2}{3} X^e_{\rm NL}+\frac{1}{3}
 X^\tau_{\rm NL}\right]
\end{equation}
with the numerical values for $X_{\rm NL}^l$ given in table \ref{tab:xnlnum}.
The corresponding values for $P_0(X)$ as a function of
$\Lambda^{(4)}_{\overline{MS}}$ and $m_c\equiv m_c(m_c)$ 
are collected in
table \ref{tab:P0Kplus}. We remark that a negligibly small term
$\sim (X^e_{\rm NL}-X^{\tau}_{\rm NL})^2 $ has been discarded in
(\ref{bkpn}).

\begin{table}[htb]
\caption[]{The function $P_0(X)$ for various $\Lms^{(4)}$ and $m_c$.
\label{tab:P0Kplus}}
\begin{center}
\begin{tabular}{|c|c|c|c|}
\hline
&\multicolumn{3}{c|}{$P_0(X)$}\\
\hline
$\Lms^{(4)}$ $\backslash$ $m_c$ & $1.25\gev$ & $1.30\gev$ & $1.35\gev$  \\
\hline
$245\mev$ & 0.393 & 0.426 & 0.462 \\
$285\mev$ & 0.380 & 0.413 & 0.448 \\
$325\mev$ & 0.366 & 0.400 & 0.435 \\
$365\mev$ & 0.352 & 0.386 & 0.420 \\
$405\mev$ & 0.337 & 0.371 & 0.405 \\ 
\hline
\end{tabular}
\end{center}
\end{table}

Using the improved Wolfenstein parametrization and the approximate
formulae (\ref{2.51}) -- (\ref{2.53}) we can next put 
(\ref{bkpn}) into a more transparent form \cite{BLO}:
\begin{equation}\label{108}
Br(K^{+} \to \pi^{+} \nu \bar\nu) = 4.11 \cdot 10^{-11} A^4 X^2(x_t)
\frac{1}{\sigma} \left[ (\sigma \bar\eta)^2 +
\left(\varrho_0 - \bar\varrho \right)^2 \right]\,,
\end{equation}
where
\begin{equation}\label{109}
\sigma = \left( \frac{1}{1- \frac{\lambda^2}{2}} \right)^2\,.
\end{equation}

The measured value of $Br(K^{+} \to \pi^{+} \nu \bar\nu)$ then
determines  an ellipse in the $(\bar\varrho,\bar\eta)$ plane  centered at
$(\varrho_0,0)$ with 
%
\begin{equation}\label{110}
\varrho_0 = 1 + \frac{P_0(X)}{A^2 X(x_t)}
\end{equation}
%
and having the squared axes
%
\begin{equation}\label{110a}
\bar\varrho_1^2 = r^2_0, \qquad \bar\eta_1^2 = \left( \frac{r_0}{\sigma}
\right)^2
\end{equation}
%
where
%
\begin{equation}\label{111}
r^2_0 = \frac{1}{A^4 X^2(x_t)} \left[
\frac{\sigma \cdot Br(K^{+} \to \pi^{+} \nu \bar\nu)}
{4.11 \cdot 10^{-11}} \right]\,.
\end{equation}
%
Note that $r_0$ depends only on the top contribution.
The departure of $\varrho_0$ from unity measures the relative importance
of the internal charm contributions.

The ellipse defined by $r_0$, $\varrho_0$ and $\sigma$ given above
intersects with the circle (\ref{2.94}).  This allows to determine
$\bar\varrho$ and $\bar\eta$  with 
\begin{equation}\label{113}
\bar\varrho = \frac{1}{1-\sigma^2} \left( \varrho_0 - \sqrt{\sigma^2
\varrho_0^2 +(1-\sigma^2)(r_0^2-\sigma^2 R_b^2)} \right), \qquad
\bar\eta = \sqrt{R_b^2 -\bar\varrho^2}
\end{equation}
%
and consequently
%
\begin{equation}\label{113aa}
R_t^2 = 1+R_b^2 - 2 \bar\varrho,
\end{equation}
%
where $\bar\eta$ is assumed to be positive.

In the leading order of the Wolfenstein parametrization
%
\begin{equation}\label{113ab}
\sigma \to 1, \qquad \bar\eta \to \eta, \qquad \bar\varrho \to \varrho
\end{equation}
%
and $Br(K^+ \to \pi^+ \nu \bar\nu)$ determines a circle in the
$(\varrho,\eta)$ plane centered at $(\varrho_0,0)$ and having the radius
$r_0$ of (\ref{111}) with $\sigma =1$. Formulae (\ref{113}) and
(\ref{113aa}) then simplify to \cite{BB3}
%
\begin{equation}\label{113a}
R_t^2 = 1 + R_b^2 + \frac{r_0^2 - R_b^2}{\varrho_0} - \varrho_0, \qquad
\varrho = \frac{1}{2} \left( \varrho_0 + \frac{R_b^2 - r_0^2}{\varrho_0}
\right).
\end{equation}
Given $\bar\varrho$ and $\bar\eta$ one can determine $V_{td}$:
\begin{equation}\label{vtdrhoeta}
V_{td}=A \lambda^3(1-\bar\varrho-i\bar\eta),\qquad
|V_{td}|=A \lambda^3 R_t.
\end{equation}
At this point a few remarks are in
order:
\begin{itemize}
\item
The long-distance contributions to $\kpn$ have been studied in
\cite{RS} and found to be
very small: a few percent of the charm contribution to the amplitude at
most, which is savely negligible.
\item
The determination of $|V_{td}|$ and of the unitarity triangle requires
the knowledge of $V_{cb}$ (or $A$) and of $|V_{ub}/V_{cb}|$. Both
values are subject to theoretical uncertainties present in the existing
analyses of tree level decays. Whereas the dependence on
$|V_{ub}/V_{cb}|$ is rather weak, the very strong dependence of
$Br(\kpn)$ on $A$ or $V_{cb}$ makes a precise prediction for this
branching ratio difficult at present. We will return to this below.
\item
The dependence of $Br(\kpn)$ on $\mt$ is also strong. However $\mt$
is known already  within $\pm 4\%$ and
consequently the related uncertainty in 
$Br(\kpn)$ is substantialy smaller than the corresponding uncertainty 
due to $V_{cb}$.
\item
Once $\varrho$ and $\eta$ are known precisely from CP asymmetries in
$B$ decays, some of the uncertainties present in (\ref{108}) related
to $|V_{ub}/V_{cb}|$ (but not to $V_{cb}$) will be removed.
\item
A very clean determination of $\sin 2\beta$ without essentially
any dependence on $m_t$ and $V_{cb}$ can be made by combining
$Br(\kpn)$ with $Br(\klpn)$ discussed below. 
\end{itemize}

\subsubsection{Numerical Analysis of \kpnn}
\label{sec:Kpnn:NumericalKp}
Let us begin the numerical analysis by  investigating the uncertainties 
in the prediction for $Br(\kpn)$ and in the determination of  $|V_{td}|$
related to the choice of the renormalization scales $\mu_t$
and $\mu_c$ in the top part and the charm part, respectively. To this end we
will fix the remaining parameters as follows:
\begin{equation}\label{mcmtnum}
\mc\equiv \mcb(\mc)=1.3\gev, \qquad \mt\equiv \mtb(\mt)=170\gev
\end{equation}
\begin{equation}\label{vcbubnum}
V_{cb}=0.040, \qquad |V_{ub}/V_{cb}|=0.08\,.
\end{equation}
In the case of $Br(\kpn)$ we need the values of both $\bar\varrho$
and $\bar\eta$. Therefore in this case we will work with
\begin{equation}\label{rhetnum}
\bar\varrho=0, \qquad\quad  \bar\eta=0.36
\end{equation}
rather than with $|V_{ub}/V_{cb}|$. Finally we will set
$\Lambda_{\overline{MS}}^{(4)}=0.325\gev$ and
$\Lambda_{\overline{MS}}^{(5)}=0.225\gev$ for the charm part and top
part, respectively.
We then vary the scales $\mu_c$ and $\mu_t$ entering $m_c(\mu_c)$
and $m_t(\mu_t)$, respectively, in the ranges
\begin{equation}\label{muctnum}
1\gev\leq\mu_c\leq 3\gev, \qquad 100\gev\leq\mu_t\leq 300\gev\,.
\end{equation}

The results of such an analysis are as follows \cite{BBL}:
The uncertainty in $Br(\kpn)$
\begin{equation}\label{varbkpnLO}
0.68\cdot 10^{-10}\leq Br(\kpn)\leq 1.08\cdot 10^{-10}
\end{equation}
present in the leading order is reduced to
\begin{equation}\label{varbkpnNLO}
0.79\cdot 10^{-10}\leq Br(\kpn)\leq 0.92\cdot 10^{-10}
\end{equation}
after including NLO corrections. 
The difference in the numerics compared to \cite{BBL} results
from $r_{K^+}=1$ used there.
Similarly one finds
\begin{equation}\label{varvtdLO}
8.24\cdot 10^{-3}\leq |V_{td}|\leq 10.97\cdot 10^{-3} \qquad {\rm LO}
\end{equation}
\begin{equation}\label{varvtdNLO}
9.23\cdot 10^{-3}\leq |V_{td}|\leq 10.10\cdot 10^{-3}  \qquad {\rm NLO}\,,
\end{equation}
where $Br(\kpn)=0.9\cdot 10^{-10}$ has been set. We observe that including
the full next-to-leading corrections reduces the uncertainty in the
determination of $|V_{td}|$ from $\pm 14\%$ (LO) to $\pm 4.6\%$ (NLO)
in the present example. The main bulk of this theoretical error stems
from the charm sector. Indeed, keeping $\mu_c=m_c$ fixed and varying
only $\mu_t$, the uncertainties in the determination of $|V_{td}|$
would shrink to $\pm 4.7\%$ (LO) and $\pm 0.6\%$ (NLO).
Similar comments apply to $Br(\kpn)$ where, as seen in
(\ref{varbkpnLO}) and (\ref{varbkpnNLO}), the theoretical uncertainty
due to $\mu_{c,t}$ is reduced from $\pm 22\%$ (LO) to $\pm 7\%$ (NLO).

Finally using the input parameters of table \ref{tab:inputparams}
(``present") and performing two
types of error analysis one finds \cite{BJL96b}
\begin{equation}\label{kpnr}
Br(\kpn)=\left\{ \begin{array}{ll}
(9.1 \pm 3.8)\cdot 10^{-11} & {\rm Scanning} \\
(8.0 \pm 1.6) \cdot 10^{-11} & {\rm Gaussian}\,, \end{array} \right.
\end{equation}
where the error comes dominantly from the uncertainties in the CKM
parameters.
The corresponding analysis with the ``future" input parameters gives
\begin{equation}\label{kpnr0}
Br(\kpn)=\left\{ \begin{array}{ll}
(8.0 \pm 1.6)\cdot 10^{-11} & {\rm Scanning} \\
(7.8 \pm 0.7) \cdot 10^{-11} & {\rm Gaussian}\,, \end{array} \right.
\end{equation}
 
\subsubsection{$\vtd$ from $K^+\to\pi^+\nu\bar\nu$}
Once $Br(K^+\to\pi^+\nu\bar\nu)\equiv Br(K^+)$ is measured, $\vtd$ can be
extracted subject to various uncertainties:
\be\label{vtda}
\frac{\sigma(\vtd)}{\vtd}=\pm 0.04_{scale}\pm \frac{\sigma(\vcb)}{\vcb}
\pm 0.7 \frac{\sigma(\bar\mc)}{\bar\mc}
\pm 0.65 \frac{\sigma( Br(K^+))}{Br(K^+)}~.
\ee
Taking $\sigma(\vcb)=0.002$, $\sigma(\bar\mc)=100\mev$ and
$\sigma( Br(K^+))=10\%$ and adding the errors in quadrature we find that
$\vtd$ can be determined with an accuracy of $\pm 10\%$ in the present
example. This number
is increased to $\pm 11\%$ once the uncertainties due to $\mt$,
$\alpha_s$ and $|V_{ub}|/\vcb$ are taken into account. Clearly this
determination can be improved although a determination of $\vtd$ with
an accuracy better than $\pm 5\%$ seems rather unrealistic.
\subsubsection{Summary and Outlook}
The accuracy of the Standard Model prediction for $Br(\kpn)$ has
improved considerably during the last five years. Indeed in 1992
 ranges like $(5-80)\cdot 10^{-11}$
could be found in the literature. This progress can be traced back to the
improved values of $\mt$ and $\vcb$ and to the inclusion of NLO
QCD corrections which considerably reduced the scale uncertainties
in the charm sector. I expect that further progress
in the determination of CKM parameters via the standard analysis of
section \ref{sec:standard} could reduce 
the errors in (\ref{kpnr}) by at least a
factor of two during the next five years.
A numerical example is given in (\ref{kpnr0}).

Now, what about the experimental status of this decay ?
Until August 97 the experimental lower bound on $Br(K^+\to \pi^+\nu\bar\nu)$
was \cite{Adler95}: $Br(\kpn)<2.4 \cdot 10^{-9}$.
One of the high-lights of August 97 was the observation by BNL787
collaboration at Brookhaven \cite{Adler97} 
of one event consistent with the signature expected for this decay.
The branching ratio:
\be\label{kp97}
Br(K^+ \rightarrow \pi^+ \nu \bar{\nu})=
(4.2+9.7-3.5)\cdot 10^{-10}
\end{equation}
has the central value  by a factor of 4 above the Standard Model
expectation but in view of large errors the result is compatible with the
Standard Model. This new result implies that $\vtd$ lies in the range
$0.006<\vtd< 0.06 $ which is substantially larger than the range
from the standard analysis of section 10. The analysis of additional
data on $K^+\to \pi^+\nu\bar\nu$ present on tape at BNL787 should narrow
this range in the near future considerably.
In view of the clean character of this decay a measurement of its
branching ratio at the level of $ 2 \cdot 10^{-10}$ 
would signal the presence of physics
beyond the Standard Model. The Standard Model sensitivity is
expected to be reached at AGS around the year 2000 \cite{AGS2}.
Also Fermilab with the Main Injector 
could measure this decay \cite{Cooper}.
\subsection{The Decay $K_{\rm L}\to\pi^0\nu\bar\nu$}
            \label{sec:HeffRareKB:klpinn1}
\subsubsection{The effective Hamiltonian}
The effective
Hamiltonian for $K_{\rm L}\to\pi^0\nu\bar\nu$
is given as follows:
\begin{equation}\label{hxnu}
{\cal H}_{\rm eff} = {G_{\rm F}\over \sqrt 2} {\alpha \over
2\pi \sin^2 \Theta_{\rm W}} V^\ast_{ts} V_{td}
X (x_t) (\bar sd)_{V-A} (\bar\nu\nu)_{V-A} + h.c.\,,   
\end{equation}
where the function $X(x_t)$, present already in $\kpn$,
includes NLO corrections and is given in (\ref{xx}). 

As we will demonstrate shortly, $\klpn$  proceeds in the Standard Model 
almost
entirely through direct CP violation \cite{littenberg:89}. Consequently it
is completely dominated by short-distance loop diagrams with top quark
exchanges. The charm contribution can be fully
neglected and the theoretical uncertainties present in $\kpn$ due to
$m_c$, $\mu_c$ and $\Lambda_{\overline{MS}}$ are absent here. 
Consequently the rare decay $\klpn$ is even cleaner than $\kpn$
and is very well suited for the determination of 
the Wolfenstein parameter $\eta$ and $\imlt$.

Before going into the details it is appropriate to clarify one point
\cite{NIR96,BUCH96}. It is usually stated in the literature that the
decay $\klpn$ is dominated by {\it direct} CP violation. Now
the standard definition of the direct CP violation (see section 8
of \cite{BF97}) requires the presence of strong phases which are
completely negligible in $\klpn$. Consequently the violation of
CP symmetry in $\klpn$ arises through the interference between
$K^0-\bar K^0$ mixing and the decay amplitude. This type of CP
violation is often called {\it mixing-induced} CP violation.
However, as already pointed out by Littenberg \cite{littenberg:89}
and demonstrated explictly in a moment,
the contribution of CP violation to $\klpn$ via $K^0-\bar K^0$ mixing 
alone is tiny. It gives $Br(\klpn) \approx 5\cdot 10^{-15}$.
Consequently, in this sence,  CP violation in $\klpn$ with
$Br(\klpn) = {\cal O}(10^{-11})$ is a manifestation of CP violation
in the decay and as such deserves the name of {\it direct} CP violation.
In other words the difference in the magnitude of CP violation in
$K_{\rm L}\to\pi\pi~(\varepsilon)$ and $\klpn$ is a signal of direct
CP violation and measuring $\klpn$ at the expected level would
rule out superweak scenarios. More details on this
issue can be found in \cite{NIR96,BUCH96,BB96}.
\subsubsection{Deriving the Branching Ratio}
Let us derive the basic formula for $Br(\klpn)$ in a manner analogous
to the one for  $Br(K^+ \to \pi^+ \nu \bar\nu)$. To this end we
consider one neutrino flavour and define the complex function:
\begin{equation}\label{hxnu1}
F = {G_{\rm F}\over \sqrt 2} {\alpha \over
2\pi \sin^2 \Theta_{\rm W}} V^\ast_{ts} V_{td}
X (x_t).   
\end{equation}
Then the effective Hamiltonian in (\ref{hxnu}) can be written as
\begin{equation}\label{hxnu2}
{\cal H}_{\rm eff} =  F (\bar sd)_{V-A} (\bar\nu\nu)_{V-A}+
F^\ast (\bar ds)_{V-A} (\bar\nu\nu)_{V-A}~.
\end{equation}
Now, from (\ref{KLS}) we have
\be\label{KLS1}
K_L=\frac{1}{\sqrt{2}}
[(1+\bar\varepsilon)K^0+ (1-\bar\varepsilon)\bar K^0]
\ee
where we have neglected
$\mid\bar\varepsilon\mid^2\ll 1$. Thus the amplitude
for $K_L\to\pi^0\nu\bar\nu$ is given by
\be\label{ampkl0}
A(K_L\to\pi^0\nu\bar\nu)=
\frac{1}{\sqrt{2}}
\left[F(1+\bar\varepsilon) \langle \pi^0|(\bar sd)_{V-A}|K^0\rangle
+ 
F^\ast (1-\bar\varepsilon) \langle \pi^0|(\bar ds)_{V-A}|\bar K^0\rangle
 \right] (\bar\nu\nu)_{V-A}.
\ee
Recalling
\be\label{DEF}
CP|K^0\rangle = - |\bar K^0\rangle, \quad\quad
C|K^0\rangle =  |\bar K^0\rangle
\ee
we have
\be
\langle \pi^0|(\bar ds)_{V-A}|\bar K^0\rangle=-
\langle \pi^0|(\bar sd)_{V-A}|K^0\rangle,
\ee
where the minus sign is crucial for the subsequent steps.

Thus we can write
\be\label{bmpkl0}
A(K_L\to\pi^0\nu\bar\nu)=
\frac{1}{\sqrt{2}}
\left[F(1+\bar\varepsilon) -F^\ast (1-\bar\varepsilon)\right]
 \langle \pi^0|(\bar sd)_{V-A}| K^0\rangle
 (\bar\nu\nu)_{V-A}.
\ee
Now the terms $\bar\varepsilon$ can be safely neglected in comparision
with unity, which implies that the indirect CP violation
(CP violation in the $K^0-\bar K^0$ mixing) is negligible in this decay.
We have then
\be
F(1+\bar\varepsilon) -F^\ast (1-\bar\varepsilon)=
{G_{\rm F}\over \sqrt 2} {\alpha \over
\pi \sin^2 \Theta_{\rm W}} \IM (V^\ast_{ts} V_{td})
\cdot X(x_t).   
\end{equation}
Consequently using isospin relation
\be
\langle \pi^0|(\bar ds)_{V-A}|\bar K^0\rangle=
\langle \pi^0|(\bar su)_{V-A}|K^+\rangle
\ee
together with (\ref{kp0}) and taking into account the difference
in the lifetimes of $K_L$ and $K^+$ we have after summation over three
neutrino flavours
\be\label{br2}
\frac{Br(K_L\to\pi^0\nu\bar\nu)}{Br(K^+\to\pi^0 e^+\nu)}=
3\frac{\tau(K_L)}{\tau(K^+)}
{\alpha^2\over |V_{us}|^2 2\pi^2 
\sin^4\Theta_{\rm W}}
 \left(\IM \lambda_t \cdot X(x_t)\right)^2
\end{equation}
where $\lambda_t=V^{\ast}_{ts}V_{td}$.
\subsubsection{Master Formulae for $Br(\klpn)$}
\label{sec:Kpnn:MasterKL}
Using (\ref{br2}) we can write $Br(\klpn)$ simply as
follows
\begin{equation}\label{bklpn}
Br(K_{\rm L}\to\pi^0\nu\bar\nu)=\kappa_{\rm L}\cdot
\left({\imlt\over\lambda^5}X(x_t)\right)^2
\end{equation}
\begin{equation}\label{kapl}
\kappa_{\rm L}=\frac{r_{K_{\rm L}}}{r_{K^+}}
 {\tau(K_{\rm L})\over\tau(K^+)}\kappa_+ =1.80\cdot 10^{-10}
\end{equation}
with $\kappa_+$ given in (\ref{kapp}) and
$r_{K_{\rm L}}=0.944$ summarizing isospin
breaking corrections in relating $\klpn$ to $K^+\to\pi^0e^+\nu$
\cite{MP}.

Using the Wolfenstein
parametrization we can rewrite (\ref{bklpn}) as
\begin{equation}\label{bklpnwol1}
Br(\klpn)=1.80\cdot 10^{-10} \eta^2 A^4 X^2(x_t)
\end{equation}
or
\begin{equation}\label{bklpnwol2}
Br(\klpn)=3.29\cdot 10^{-5} \eta^2 |V_{cb}|^4 X^2(x_t)
\end{equation}
or using 
\begin{equation}\label{xxappr}
X(x_t)=0.65\cdot x_t^{0.575}
\end{equation}
as
\begin{equation}
Br(K_{\rm L}\to\pi^0\nu\bar\nu)=
3.0\cdot 10^{-11}
\left [ \frac{\eta}{0.39}\right ]^2
\left [\frac{\mtb(\mt)}{170~GeV} \right ]^{2.3} 
\left [\frac{\mid V_{cb}\mid}{0.040} \right ]^4 \,.
\label{bklpn1}
\end{equation}

The determination of $\eta$ using $Br(\klpn)$ requires the knowledge
of $V_{cb}$ and $\mt$. The very strong dependence on $V_{cb}$ or $A$
makes a precise prediction for this branching ratio difficult at
present.

\subsubsection{$\vcb$ and $\IM \lambda_t$  from $K_L\to\pi^0\nu\bar\nu$}
It was pointed out in \cite{AJB94} that the strong
dependence of $Br(\klpn)$ on $V_{cb}$, together with the clean nature of
this decay, can be used to determine this element without any hadronic
uncertainties. To this end $\eta$ and $m_t$ have to be known with
sufficient precision in addition to $Br(\klpn)$. 
Inverting (\ref{bklpn1})
one finds
\begin{equation}\label{vcbklpn}
|V_{cb}|=40.0\cdot 10^{-3} \sqrt{\frac{0.39}{\eta}}
\left[\frac{170\gev}{\mtb(\mt)}\right]^{0.575}
\left[\frac{Br(\klpn)}{3\cdot 10^{-11}}\right]^{1/4}\,.
\end{equation}
We note that the weak dependence of $V_{cb}$ on $Br(\klpn)$ allows
to achieve a high precision for this CKM element even when $Br(\klpn)$
is known with only relatively moderate accuracy, e.g.\ 10--15\%.

With $\eta$ 
determined one day from CP asymmetries in B-decays
and $\mt$ measured very precisely at LHC and NLC,
a measurement of $Br(\klpn)$ with an accuracy of $10\%$
would determine $\vcb$ with an error of $\pm 0.001$.
A comparision of
this determination of $|V_{cb}|$ with the usual one in tree level
B-decays would offer an excellent test of the Standard Model
and in the case of discrepancy would signal physics beyond 
it.

On the other hand inverting (\ref{bklpn}) and using (\ref{xxappr})
 one finds \cite{BB96}:
\begin{equation}\label{imlta}
\IM\lambda_t=1.36\cdot 10^{-4} 
\left[\frac{170\gev}{\mtb(\mt)}\right]^{1.15}
\left[\frac{Br(\klpn)}{3\cdot 10^{-11}}\right]^{1/2}\,.
\end{equation}
(\ref{imlta}) offers
 the cleanest method to measure $\IM\lambda_t$;
even better than the CP asymmetries
in $B$ decays discussed briefly in the next section.
\subsubsection{Numerical Analysis of \klpnn}
\label{sec:Kpnn:NumericalKL}
The $\mu_t$-uncertainties present in the function $X(x_t)$ have 
already been
discussed in connection with $\kpn$. After the inclusion of NLO
corrections they are so small that they can be neglected for all
practical purposes. 
At the level of $Br(\klpn)$ the ambiguity in the choice of $\mu_t$ is
reduced from $\pm 10\%$ (LO) down to $\pm 1\%$ (NLO), which
considerably increases the predictive power of the theory. Varying
$\mu_t$ according to (\ref{muctnum}) and using the input parameters
as in the case of $\kpn$ we find that the uncertainty
in $Br(\klpn)$
\begin{equation}\label{varbklpnLO}
2.53\cdot 10^{-11}\leq Br(\klpn)\leq 3.08\cdot 10^{-11}
\end{equation}
present in the leading order is reduced to
\begin{equation}\label{varbklpnNLO}
2.64\cdot 10^{-11}\leq Br(\klpn)\leq 2.72\cdot 10^{-11}
\end{equation}
after including NLO corrections. This means that the theoretical
uncertainty in the determination of $\eta$ amounts to only $\pm 0.7\%$
which is safely negligible.

Using the input parameters of table \ref{tab:inputparams}
one finds \cite{BJL96b}
\begin{equation}\label{klpnr4}
Br(\klpn)=\left\{ \begin{array}{ll}
(2.8 \pm 1.7)\cdot 10^{-11} & {\rm Scanning} \\
(2.6 \pm 0.9) \cdot 10^{-11} & {\rm Gaussian} \end{array} \right.
\end{equation}
where the error comes dominantly from the uncertainties in the CKM
parameters. The corresponding analysis with the ``future" input
parameters gives
\begin{equation}\label{klpnr5}
Br(\klpn)=\left\{ \begin{array}{ll}
(2.7 \pm 0.5)\cdot 10^{-11} & {\rm Scanning} \\
(2.6 \pm 0.3) \cdot 10^{-11} & {\rm Gaussian} \end{array} \right.
\end{equation}
\subsubsection{Summary and Outlook}
The accuracy of the Standard Model prediction for $Br(\klpn)$ has
improved considerably during the last five years. Indeed in 1992
values as high as $15\cdot 10^{-11}$ could be found in the
literature. This progress can be traced back mainly to the
improved values of $\mt$ and $\vcb$ and to some extent to 
the inclusion of NLO QCD corrections.
I expect that further progress
in the determination of CKM parameters via the standard analysis of
section 9 could reduce the errors in (\ref{klpnr4}) by at least a
factor of two during the next five years.
A numerical example is given in (\ref{klpnr5}).

The present upper bound on $Br(K_{\rm L}\to \pi^0\nu\bar\nu)$ from
FNAL experiment E799 \cite{XX97} is 
\begin{equation}\label{KLD}
Br(\klpn)<1.8 \cdot 10^{-6}\,.
\end{equation}
This is about five orders of magnitude above the Standard Model expectation
(\ref{klpnr4}).

How large could $Br(\klpn)$ really be? As shown  in \cite{NIR96}
one can easily derive by means of isospin symmetry the following 
{\it model independent} bound:
\begin{equation}
Br(\klpn) < 4.4 \cdot Br(\kpn)
\end{equation}
which through (\ref{kp97})  gives
\begin{equation}\label{B108}
Br(\klpn) < 6.1 \cdot 10^{-9}
\end{equation}
This bound is much stronger than the direct experimental bound in
(\ref{KLD}).

Now FNAL-E799 expects to reach
the accuracy ${\cal O}(10^{-8})$ and
a very interesting new experiment
at Brookhaven (BNL E926) \cite{AGS2} 
expects to reach the single event sensitivity $2\cdot 10^{-12}$
allowing a $10\%$ measurement of the expected branching ratio. 
There are furthermore plans
to measure this gold-plated  decay with comparable sensitivity
at Fermilab \cite{FNALKL} and KEK \cite{KEKKL}.
\begin{figure}[hbt]
\vspace{0.10in}
\centerline{
\epsfysize=2.7in
\epsffile{utfig1.eps}
}
\vspace{0.08in}
\caption{Unitarity triangle from $K\to\pi\nu\bar\nu$.}\label{fig:KPKL}
\end{figure}

\subsection{Unitarity Triangle and $\sin 2\beta$ from $K\to\pi\nu\bar\nu$}
\label{sec:Kpnn:Triangle}
The measurement of $Br(\kpn)$ and $Br(\klpn)$ can determine the
unitarity triangle completely, (see fig.~\ref{fig:KPKL}), 
provided $\mt$ and $V_{cb}$ are known \cite{BH}.
Using these two branching ratios simultaneously allows to eliminate
$|V_{ub}/V_{cb}|$ from the analysis which removes a considerable
uncertainty. Indeed it is evident from (\ref{bkpn}) and
(\ref{bklpn}) that, given $Br(\kpn)$ and $Br(\klpn)$, one can extract
both $\imlt$ and $\relt$. One finds \cite{BB4,BBL}
\begin{equation}\label{imre}
\imlt=\lambda^5{\sqrt{B_2}\over X(x_t)}\qquad
\relt=-\lambda^5{{\relc\over\lambda}P_0(X)+\sqrt{B_1-B_2}\over X(x_t)}\,,
\end{equation}
where we have defined the ``reduced'' branching ratios
\begin{equation}\label{b1b2}
B_1={Br(\kpn)\over 4.11\cdot 10^{-11}}\qquad
B_2={Br(\klpn)\over 1.80\cdot 10^{-10}}\,.
\end{equation}
Using next the expressions for $\imlt$, $\relt$ and $\relc$ given
in (\ref{2.51})--(\ref{2.53}) we find
\begin{equation}\label{rhetb}
\bar\varrho=1+{P_0(X)-\sqrt{\sigma(B_1-B_2)}\over A^2 X(x_t)}\,,\qquad
\bar\eta={\sqrt{B_2}\over\sqrt{\sigma} A^2 X(x_t)}
\end{equation}
with $\sigma$ defined in (\ref{109}). An exact treatment of the CKM
matrix shows that the formulae (\ref{rhetb}) are rather precise
\cite{BB4}. The error in $\bar\eta$ is below 0.1\% and
$\bar\varrho$ may deviate from the exact expression by at most
$\Delta\bar\varrho=0.02$ with essentially negligible error for
$0\leq\bar\varrho\leq 0.25$.

Using (\ref{rhetb}) one finds subsequently \cite{BB4}
\begin{equation}\label{sin}
r_s=r_s(B_1, B_2)\equiv{1-\bar\varrho\over\bar\eta}=\cot\beta\,, \qquad
\sin 2\beta=\frac{2 r_s}{1+r^2_s}
\end{equation}
with
\begin{equation}\label{cbb}
r_s(B_1, B_2)=\sqrt{\sigma}{\sqrt{\sigma(B_1-B_2)}-P_0(X)\over\sqrt{B_2}}\,.
\end{equation}
Thus within the approximation of (\ref{rhetb}) $\sin 2\beta$ is
independent of $V_{cb}$ (or $A$) and $m_t$. An exact treatment of
the CKM matrix confirms this finding to a high accuracy. The
dependence on $V_{cb}$ and $m_t$ enters only at order
$\ord(\lambda^2)$ and as a numerical analysis shows this
dependence can be fully neglected.

It should be stressed that $\sin 2\beta$ determined this way depends
only on two measurable branching ratios and on the function
$P_0(X)$ which is completely calculable in perturbation theory.
Consequently this determination is free from any hadronic
uncertainties and its accuracy can be estimated with a high degree
of confidence. 

An extensive numerical analysis of the formulae above has been presented
in \cite{BB4,BB96}. We summarize the results of the latter paper.
Assuming that the branching ratios are known to within $\pm 10\%$
\begin{equation}\label{bkpkl}
Br(\kpn)=(1.0\pm 0.1)\cdot 10^{-10}\,,\qquad
Br(\klpn)=(3.0\pm 0.30)\cdot 10^{-11}
\end{equation}
and choosing 
\begin{equation}\label{mtcv}
\mt=(170\pm 3)\gev,\quad P_0(X)=0.40\pm0.06,\quad
|V_{cb}|=0.040\pm 0.002,
\end{equation}
one finds the results given in the second column of table 
\ref{tabkb1}.
In the third column the results for the choice
$|V_{cb}|=0.040\pm 0.001$ are shown.
It should be remarked that the quoted errors for the input parameter
are quite reasonable if one keeps in mind
that it will take  five years to achieve the accuracy
assumed in (\ref{bkpkl}). The error in $P_0(X)$ in (\ref{mtcv}) 
results from the errors (see table \ref{tab:P0Kplus} and (\ref{muctnum})) 
in $\Lms^{(4)}$, $m_c$ and $\mu_c$ added quadratically.
Doubling the error in $m_c$ would give $P_0(X)=0.40\pm 0.09$ and
an increase of the errors in $|V_{td}|/10^{-3}$, $\bar\varrho$ and
$\sin 2\beta$ by at most $\pm 0.2$, $\pm 0.02$ and $\pm 0.01$
respectively, without any changes in $\bar\eta$ and 
${\rm Im}\lambda_t$.

\begin{table}
\caption[]{Illustrative example of the determination of CKM
parameters from $K\to\pi\nu\bar\nu$ for two choices of
$V_{cb}$ and other parameters given in the text.
\label{tabkb1}}
\begin{center}
\begin{tabular}{|c||c|c|}\hline
&$|V_{cb}|=0.040\pm 0.002$&$|V_{cb}|=0.040\pm 0.001$.\\ 
\hline
\hline
$|V_{td}|/10^{-3}$&$10.3\pm 1.1$&$10.3\pm 0.9$\\ 
\hline
$|V_{ub}/V_{cb}|$&$0.089\pm 0.017$
&$0.089\pm 0.011$ \\
\hline 
$\bar\varrho$&$-0.10\pm 0.16$ &$-0.10\pm 0.12$\\
\hline
$\bar\eta$&$0.38\pm 0.04$&$0.38\pm 0.03$\\
\hline
$\sin 2\beta$&$0.62\pm 0.05$&$0.62\pm 0.05$\\
\hline
${\rm Im}\lambda_t/10^{-4}$&$1.37\pm 0.07$
&$1.37\pm 0.07$ \\
\hline
\end{tabular}
\end{center}
\end{table}


We observe that respectable determinations of all considered 
quantities except for 
$\bar\varrho$ can be obtained.
Of particular interest are the accurate determinations of
$\sin 2\beta$ and of ${\rm Im}\lambda_t$.
The latter quantity as seen in (\ref{imre}) 
can be obtained from
$K_{\rm L}\to\pi^0\nu\bar\nu$ alone and does not require knowledge
of $V_{cb}$.

As pointed out in \cite{BB96},
$K_{\rm L}\to\pi^0\nu\bar\nu$ appears to be the best decay to 
measure ${\rm Im}\lambda_t$; even better than the CP asymmetries
in $B$ decays discussed in the next section.
The importance of measuring accurately  ${\rm Im}\lambda_t$ is evident.
It plays a central role in the phenomenology of CP violation
in $K$ decays and is furthermore equivalent to the 
Jarlskog parameter $J_{\rm CP}$ \cite{CJ}, 
the invariant measure of CP violation in the Standard Model, 
$J_{\rm CP}=\lambda(1-\lambda^2/2){\rm Im}\lambda_t$.

The accuracy to which $\sin 2\beta$ can be obtained from
$K\to\pi\nu\bar\nu$ is, in the  example discussed above, 
comparable to the one expected
in determining $\sin 2\beta$ from CP asymmetries in $B$ decays prior to
LHC experiments.  In this case $\sin 2\beta$ is determined best by
measuring CP violation in $B_d\to J/\psi K_{\rm S}$.
Using the formula  for the corresponding time-integrated 
CP asymmetry one finds an
interesting connection between rare $K$ decays and $B$ physics \cite{BB4}
\begin{equation}\label{kbcon}
{2 r_s(B_1,B_2)\over 1+r^2_s(B_1,B_2)}=
-a_{\mbox{{\scriptsize CP}}}(B_d\to J/\psi K_{\mbox{{\scriptsize S}}})
{1+x^2_d\over x_d}
\end{equation}
which must be satisfied in the Standard Model. We stress that except
for $P_0(X)$ given in table \ref{tab:P0Kplus} all quantities in
(\ref{kbcon}) can be directly measured in experiment and that this
relationship is essentially independent of $m_t$ and $V_{cb}$.
Due to very small theoretical uncertainties in (\ref{kbcon}), this
relation is particularly suited for tests of CP violation in the
Standard Model and offers a powerful tool to probe the physics
beyond it.
Further comparision between the potential of $K \to \pi \nu\bar\nu$ and
CP asymmetries in $B$ decays will be given in section 14.

Finally we  compare the determination of the unitarity triangle by
means of $K\to\pi\nu\bar\nu$ with
the one by means of the standard analysis of the unitarity triangle.
The results obtained from $K\to\pi\nu\bar\nu$ corresponding to table
\ref{tabkb1}
are given in the second and the third column of table 
\ref{tabkb2}. In the fourth and fifth column the corresponding results
of the standard analysis of the unitarity triangle are shown.
We observe that a considerable progress,
when compared with the present analysis of the unitarity triangle,
can be achieved through the measurements of $K\to\pi\nu\bar\nu$
decays.
\begin{table}
\caption[]{Illustrative example of the determination of CKM
parameters from $K\to\pi\nu\bar\nu$ and from the standard
analysis of the unitarity triangle.
\label{tabkb2}}
\vspace{0.4cm}
\begin{center}
\begin{tabular}{|c||c|c||c|c|}\hline
&$\sigma(|V_{cb}|)=\pm 0.002$ & $\sigma(|V_{cb}|)=\pm 0.001$
& {\rm Present} & {\rm Future}
\\ 
\hline
\hline
$\sigma(|V_{td}|) $& $\pm 10\% $ & $ \pm 9\% $
& $\pm 24\%$ & $\pm 7\%$\\ 
\hline 
$\sigma(\bar\varrho) $ & $\pm 0.16$ &$\pm 0.12$
& $\pm 0.32$  & $\pm 0.08$\\
\hline
$\sigma(\bar\eta)$ & $\pm 0.04$&$\pm 0.03$
&$\pm 0.12 $ & $\pm 0.03 $\\
\hline
$\sigma(\sin 2\beta)$ & $\pm 0.05$&$\pm 0.05$
& $\pm 0.22 $ & $\pm 0.05$\\
\hline
$\sigma({\rm Im}\lambda_t)$&$\pm 5\%$ &$\pm 5\%$ 
& $\pm 33\%$ & $\pm 8\%$\\
\hline
\end{tabular}
\end{center}
\end{table}
\subsection{$K\to\pi\nu\bar\nu$ Beyond the Standard Model}
In view of the very clean character of $K\to\pi\nu\bar\nu$,
these decays are very suitable for the study of new physics
effects. One example is the relation (\ref{kbcon}). Recently
several extensive analyses of supersymmetry effects in general
supersymmetric models have been presented in \cite{NIR96,GN1,BRS}
where further references can be found. In the MSSM these
effects are found to be very small but in certain more 
general scenarios of supersymmetry enhancements or
suppressions of $Br(K^+\to\pi^+\nu\bar\nu)$ and
$Br(K_L\to\pi^0\nu\bar\nu)$ by factors 2-3 cannot be
excluded. 
Model independent studies of these decays
can be found in \cite{NIR96,BRS}. The corresponding analyses in
various no--supersymmetric extensions of the Standard Model are 
listed in \cite{KLBSM}.
In particular, enhancement of $Br(K_L\to\pi^0\nu\bar\nu)$
by 1--2 orders of magnitude above the Standard Model
expectations is according to \cite{HHW98} 
still possible in four-generation models.
\subsection{The Decays $B\to X_{s,d}\nu\bar\nu$}
            \label{sec:HeffRareKB:klpinn2}
\subsubsection{Effective Hamiltonian}
The decays $B\to X_{s,d}\nu\bar\nu$ are the theoretically
cleanest decays in the field of rare $B$-decays.
They are dominated by the same $Z^0$-penguin and box diagrams
involving top quark exchanges which we encountered already
in the case of $\kpn$ and $\klpn$ except for the appropriate
change of the external quark flavours. Since the change of external
quark flavours has no impact on the $m_t$ dependence,
the latter is fully described by the function $X(x_t)$ in
(\ref{xx}) which includes
the NLO corrections \cite{BB2}. The charm contribution as
discussed at the beginning of this section is fully neglegible
here and the resulting effective Hamiltonian is very similar to
the one for $\klpn$ given in (\ref{hxnu}). 
For the decay $B\to X_s\nu\bar\nu$ it reads
\begin{equation}\label{bxnu}
{\cal H}_{\rm eff} = {G_{\rm F}\over \sqrt 2} {\alpha \over
2\pi \sin^2 \Theta_{\rm W}} V^\ast_{tb} V_{ts}
X (x_t) (\bar bs)_{V-A} (\bar\nu\nu)_{V-A} + h.c.   
\end{equation}
with $s$ replaced by $d$ in the
case of $B\to X_d\nu\bar\nu$.
 
The theoretical uncertainties related to the renormalization
scale dependence are as in $\klpn$ and 
can be essentially neglected. The same applies to long distance
contributions considered in \cite{BUC97}.
On the other hand $B\to X_{s,d}\nu\bar\nu$ are CP conserving and
consequently the relevant branching ratios are sensitive to 
$\vtd$ and $\vts$ as opposed to $Br(\klpn)$ in which
$\IM (V^\ast_{ts} V_{td})$ enters. As we will stress below the
measurement of both
$B\to X_{s}\nu\bar\nu$ and $B\to X_{d}\nu\bar\nu$ offers the
cleanest determination of the ratio $\vtd/\vts$.

\subsubsection{The Branching Ratios}
The calculation of the branching fractions for $B\to X_{s,d}\nu\bar\nu$ 
can be done similarly to $B\to X_s \gamma$ 
in the spectator model corrected for short distance QCD effects.
Normalizing as in these latter decays 
to $Br(B\to X_c e\bar\nu)$ and summing over three neutrino 
flavours one finds

\begin{equation}\label{bbxnn}
\frac{Br(B\to X_s\nu\bar\nu)}{Br(B\to X_c e\bar\nu)}=
\frac{3 \alpha^2}{4\pi^2\sin^4\Theta_{\rm W}}
\frac{|V_{ts}|^2}{|V_{cb}|^2}\frac{X^2(x_t)}{f(z)}
\frac{\bar\eta}{\kappa(z)}\,.
\end{equation}
Here $f(z)$ is the phase-space factor for $B\to X_c
e\bar\nu$ defined already in (\eqn{g}) and $\kappa(z)$ is the
corresponding QCD correction given in (\eqn{kap}). The
factor $\bar\eta$ represents the QCD correction to the matrix element
of the $b\to s\nu\bar\nu$ transition due to virtual and bremsstrahlung
contributions and is given by the well known expression
\begin{equation}\label{etabar}
\bar\eta=\kappa(0)=
1+\frac{2\alpha_s(m_b)}{3\pi}\left(\frac{25}{4}-\pi^2\right)
\approx 0.83\,.
\end{equation}
In the case of $B\to X_d\nu\bar\nu$ one has to replace $V_{ts}$ by
$V_{td}$ which results in a decrease of the branching ratio by
roughly an order of magnitude.

It should be noted that $Br(B \to X_s \nu\bar\nu)$ as given in
(\eqn{bbxnn}) is in view of $|V_{ts}/V_{cb}|^2 \approx 0.95 \pm 0.03$
essentially independent of the CKM parameters and the main uncertainty
resides in the value of $\mt$ which is already rather precisely
known. Setting $Br(B\to X_ce\bar\nu)=10.4\%$, $f(z)=0.54$,
$\kappa(z)=0.88$ and using the values in (\ref{alsinbr})
 we have
\begin{equation}
Br(B \to X_s \nu\bar\nu) = 3.7 \cdot 10^{-5} \,
\frac{|V_{ts}|^2}{|V_{cb}|^2} \,
\left[ \frac{\mtb(\mt)}{170\gev} \right]^{2.30} \, .
\label{eq:bxsnnnum}
\end{equation}

Taking next, in accordance with (\ref{kf}), $\kappa(z)=0.88$,
$f(z)=0.54\pm 0.04$ and 
$Br(B\to X_ce\bar\nu)=(10.4\pm 0.4)\%$
and using the input parameters of table \ref{tab:inputparams}
one finds \cite{BJL96b}
\begin{equation}\label{klpnr3}
Br(B \to X_s \nu\bar\nu)=\left\{ \begin{array}{ll}
(3.4 \pm 0.7)\cdot 10^{-5} & {\rm Scanning} \\
(3.2 \pm 0.4) \cdot 10^{-5} & {\rm Gaussian}\,. \end{array} \right.
\end{equation}
 
What about the data? 
One of the high-lights of FCNC-1996 was the upper bound:
\begin{equation}\label{124}
Br(B\to X_s \nu\bar\nu) < 7.7\cdot 10^{-4} 
\quad
(90\%\,\,\mbox{C.L.})
\end{equation}
obtained for the first time by ALEPH \cite{Aleph96}.
This is only a factor of 20 above the Standard Model expectation.
Even if the actual measurement of this decay is extremly difficult,
all efforts should be made to measure it. One should also 
make attempts to measure $Br(B\to X_d \nu\bar\nu)$. Indeed 

\begin{equation}\label{bratio}
\frac{Br(B\to X_d\nu\bar\nu)}{Br(B\to X_s\nu\bar\nu)}=
\frac{|V_{td}|^2}{|V_{ts}|^2}
\end{equation} 
offers the
cleanest direct determination of $\vtd/\vts$ as all uncertainties related
to $\mt$, $f(z)$ and $Br(B\to X_ce\bar\nu)$ cancel out.
\subsection{The Decays $B_{s,d}\to l^+l^-$}
\subsubsection{The Effective Hamiltonian}
The decays $B_{s,d}\to l^+l^-$ are after $B\to X_{s,d}\nu\bar\nu$ 
the theoretically cleanest decays in the field of rare $B$-decays.
They are dominated by the $Z^0$-penguin and box diagrams
involving top quark exchanges which we encountered already
in the case of $B\to X_{s,d}\nu\bar\nu$   except that due to
charged leptons in the final state the charge flow in the
internal lepton line present in the box diagram is reversed.
This results in a different $\mt$ dependence summarized
by the function  $Y(x_t)$, the NLO generalization \cite{BB2}
of the function $Y_0(x_t)$ given in (\ref{Y0}).
The charm contributions as
discussed at the beginning of this section are fully negligible
here and the resulting effective Hamiltonian is given 
for $B_s\to l^+l^-$ as follows:

\begin{equation}\label{hyll}
{\cal H}_{\rm eff} = -{G_{\rm F}\over \sqrt 2} {\alpha \over
2\pi \sin^2 \Theta_{\rm W}} V^\ast_{tb} V_{ts}
Y (x_t) (\bar bs)_{V-A} (\bar ll)_{V-A} + h.c.   \end{equation}
with $s$ replaced by $d$ in the
case of $B_d\to l^+l^-$.

The function $Y(x)$ is given by
\begin{equation}\label{yyx}
Y(x_t) = Y_0(x_t) + \aspi Y_1(x_t)\,,
\end{equation}
where $Y_0(x_t)$ can be found in (\ref{Y0})
and $Y_1(x_t)$ in (\ref{yy1}).
The leftover $\mu_t$-dependence in $Y(x_t)$ is tiny and amounts to
an uncertainty of $\pm 1\%$ at the level of the branching ratio.
We recall that $Y(x_t)$ can also be written as
\begin{equation}\label{yeta2}
Y(x_t)=\eta_Y\cdot Y_0(x_t)\,, \qquad\quad \eta_Y=1.026\pm 0.006\,,
\end{equation}
where $\eta_Y$ summarizes the NLO corrections.
With $\mt\equiv \mtb(\mt)$ this QCD factor
depends only very weakly on $m_t$. The range in (\ref{yeta2})
corresponds to $150\gev\leq m_t\leq 190\gev$. The dependence on
$\Lambda_{\overline{MS}}$ can be neglected. 

\subsubsection{The Branching Ratios}
The branching ratio for $B_s\to l^+l^-$ is given by \cite{BB2}
\begin{equation}\label{bbll}
Br(B_s\to l^+l^-)=\tau(B_s)\frac{G^2_{\rm F}}{\pi}
\left(\frac{\alpha}{4\pi\sin^2\Theta_{\rm W}}\right)^2 F^2_{B_s}m^2_l m_{B_s}
\sqrt{1-4\frac{m^2_l}{m^2_{B_s}}} |V^\ast_{tb}V_{ts}|^2 Y^2(x_t)
\end{equation}
where $B_s$ denotes the flavour eigenstate $(\bar bs)$ and $F_{B_s}$ is
the corresponding decay constant. Using
(\ref{alsinbr}), (\ref{yeta2}) and (\ref{PBE2}) we find in the
case of $B_s\to\mu^+\mu^-$
\begin{equation}\label{bbmmnum}
Br(B_s\to\mu^+\mu^-)=3.5\cdot 10^{-9}\left[\frac{\tau(B_s)}{1.6
\mbox{ps}}\right]
\left[\frac{F_{B_s}}{210\mev}\right]^2 
\left[\frac{|V_{ts}|}{0.040}\right]^2 
\left[\frac{\mtb(\mt)}{170\gev}\right]^{3.12}.
\end{equation}

The main uncertainty in this branching ratio results from
the uncertainty in $F_{B_s}$.
Using the input parameters of table \ref{tab:inputparams}
together with $\tau(B_s)=1.6$ ps and $F_{B_s}=(210\pm 30)\mev$ 
one finds \cite{BJL96b}
\begin{equation}\label{klpnr1}
Br(B_s\to\mu^+\mu^-)=\left\{ \begin{array}{ll}
(3.6 \pm 1.9)\cdot 10^{-9} & {\rm Scanning} \\
(3.4 \pm 1.2) \cdot 10^{-9} & {\rm Gaussian.} \end{array} \right.
\end{equation}

For $B_d\to\mu^+\mu^-$ a similar formula holds with obvious
replacements of labels $(s\to d)$. Provided the decay constants
$F_{B_s}$ and $F_{B_d}$ will have been calculated reliably by
non-perturbative methods or measured in leading leptonic decays one
day, the rare processes $B_{s}\to\mu^+\mu^-$ and $B_{d}\to\mu^+\mu^-$
should offer clean determinations of $|V_{ts}|$ and $|V_{td}|$. 
In particular the ratio
\begin{equation}
\frac{Br(B_d\to\mu^+\mu^-)}{Br(B_s\to\mu^+\mu^-)}
=\frac{\tau(B_d)}{\tau(B_s)}
\frac{m_{B_d}}{m_{B_s}}
\frac{F^2_{B_d}}{F^2_{B_s}}
\frac{|V_{td}|^2}{|V_{ts}|^2}
\end{equation}
having smaller theoretical uncertainties than the separate
branching ratios should offer a useful measurement of
$\vtd/\vts$. Since $Br(B_d\to\mu^+\mu^-)= {\cal O}(10^{-10})$
this is, however, a very difficult task. For $B_s \to \tau^+\tau^-$
and $B_s\to e^+e^-$ one expects branching ratios ${\cal O}(10^{-6})$
and ${\cal O}(10^{-13})$, respectively, with the corresponding branching 
ratios for $B_d$-decays by one order of magnitude smaller.

We should also remark that in conjunction with a future measurement of 
$x_s$, the branching
ratio $Br(B_s\to \mu\bar\mu)$ could help to determine 
the non-perturbative parameter $B_{B_s}$ and consequently allow
a test of existing non-perturbative methods \cite{B95}:
\begin{equation}
 B_{B_s}=
\left[\frac{x_s}{22.1}\right]
\left[\frac{\mtb(\mt)}{170~\mbox{GeV}} \right]^{1.6} 
\left[\frac{4.2\cdot 10^{-9}}{Br(B_s\to \mu\bar\mu)} \right] \,.
\end{equation}
This test could be of course affected by new physics contributions.
\subsubsection{Outlook}
What about the data?

The bounds on $B_{s,d}\to l\bar l$ are still
many orders of magnitude away from Standard Model expectations.
The best bounds come from CDF \cite{CDFMU}. One has:
\begin{equation}\label{MUBOUND}
Br(B_s\to\mu^+\mu^-)\le 
2.6\cdot 10^{-6}~~~~~(95\% C.L.)
\end{equation}
and $Br(B_d\to\mu^+\mu^-)\le 8.6\cdot 10^{-7}$.
CDF should reach in Run II the
sensitivity of $1\cdot 10^{-8}$ and $4\cdot 10^{-8}$ for
$B_d\to \mu\bar\mu$ and $B_s\to \mu\bar\mu$, respectively.
It is hoped that these decays will be observed at
LHC-B. The experimental status of $B\to\tau^+\tau^-$ and its
usefulness in tests of the physics beyond the Standard Model
is discussed in \cite{GLN96}.
\subsection{Higher Order Electroweak Effects in Rare Decays}
Until now we have considered various penguin and box diagrams
contributing to rare decays together with QCD corrections.
In none of these contributions the role of the neutral Higgs boson 
$H^0$ has been felt. Since the couplings of $H^0$ to fermions
are proportional to fermion masses, contributions of internal
$H^0$ are very strongly suppressed unless $H^0$ couples at both
ends of its propagator to the top. This situation appears first
at two-loop level in electroweak interactions. 
Examples of such diagrams can be constructed from diagrams
(a)--(c) in fig. \ref{L:7} by replacing there the gluon propagator
by the $H^0$-propagator. Even more important diagrams are obtained
by replacing $W^\pm$ and the gluon by the fictitious $\phi^\pm$
Higgs exchanges with the appropriate change in internal fermion
propagators.

Once the higher order electroweak contributions are considered and
one  recalls the extensive precision electroweak studies at 
$Z^0$-factories, an obvious question arises. What about the
ambiguities in rare meson decays stemming from various possible 
definitions of electroweak parameters? We have seen in this section
that the branching ratios $Br(K_L\to\pi^0\nu\bar\nu)$,
$Br(K^+\to\pi^+\nu\bar\nu)$, $Br(B\to X_{d,s}\nu\bar\nu)$ and
$Br(B\to l^-l^+)$ all had the following generic structure
\be\label{HO1}
Br \sim \frac{G^2_F \alpha^2(\mz)}{\sin^4\Theta_W}
\lbrack F(x_t)\rbrack^2,
\ee
where we have suppressed the charm contribution to 
$Br(K^+\to\pi^+\nu\bar\nu)$.

Now, there are several definitions of $\sin^2\Theta_W$. For
instance, $\sin^2\Theta_W=0.224$ in the on-shell scheme,
whereas the effective $\sin^2\hat\Theta_W|_{\rm eff}=0.230$.
These two choices result in branching ratios which differ by
$5.6\%$ to be compared with uncertainties of $1-2\%$ from
QCD after NLO corrections have been taken into account.
There is of course also the question of the scale in $\alpha$.
This is analogous to the recent discussion of two--loop
electroweak effects in $B\to X_s\gamma$ presented in section
12.4 and the related issue of $\alpha(\mu)$ there.

Clearly, in order to reduce such uncertainties,
one has to consider two-loop electroweak contributions
to the rare decays in question. Such an analysis has been performed
in \cite{BB97} in the large $\mt$-limit. Schematically the formula
(\ref{HO1}) reads now
\be\label{HO2}
Br \sim \frac{G^2_F \alpha^2(\mz)}{\sin^4\Theta_W}
\left[ F(x_t)+c G_F \mt^2 \frac{\mt^2}{\mw^2}\right]
\ee
where the second term represents  two-loop electroweak corrections
for large $\mt$. The scheme dependence of this term cancels in the 
large $\mt$ limit, the scheme dependence of $\sin^2\Theta_W$. 
Moreover the proper scale in $\alpha$ turns out to be $\mz$ as
anticipated (\ref{HO1}) and in all our calculations before.
Evidently the decays in question being governed by short distance
penguin and box contributions involve $\alpha(\mz)$, as opposed
to $B\to X_s\gamma$, where due to the on-shell photons $\alpha(m_e)$
matters.

The
large $\mt$ estimate of the full two-loop electroweak corrections
can be only trusted within a factor of two. Yet the residual parameter
uncertainties after the inclusion of these corrections turns out
to be less than $2\%$, which is well below the experimental
sensitivity in the forseeable future. Similarly for
$\sin^2\hat\Theta_W|_{\rm eff}=0.230$, used previously in our numerical
estimates, there is an enhancement of various branching ratios by
$1-2\%$ which can also be neglected. It should be stressed that all
these effects cancel in the determination of $\sin 2\beta$ from
$K\to\pi\nu\bar\nu$. Further details can be found in \cite{BB97}. 
\section{Future Visions}
\setcounter{equation}{0}
%\setcounter{figure}{0}
%\setcounter{table}{0}
\subsection{Preliminaries}
Let us next have a look in the future and ask the question how well various
parameters of the Standard Model can be determined provided the
cleanest decays  have been measured
to some respectable precision. We have made already such an exercise in
section \ref{sec:Kpnn:Triangle}
using the decays $\klpn$ and $\kpn$. Now we want to make
an analogous analysis using CP-asymmetries in $B$-decays. This way we
will be able to compare the potentials of the CP asymmetries in
determining the parameters of the Standard Model with those
of the cleanest rare $K$-decays: $K_{\rm L}\to\pi^0\nu\bar\nu$ and
$K^+\to\pi^+\nu\bar\nu$. This section is based on 
\cite{BLO,AJB94,BB96,B95}.
\subsection{CP-Asymmetries in B-Decays}
CP violation in B-decays is certainly one of the most important 
targets of B-factories and of dedicated B-experiments at hadron 
facilities. It is well known that CP violating effects are expected
to occur in a large number of channels at a level attainable at 
forthcoming experiments. Moreover there exist channels which
offer the determination of CKM phases essentially without any hadronic
uncertainties. Since extensive reviews on CP violation in B decays can 
be found in the literature \cite{BF97,NQ,RF97}, 
let me concentrate only on the most important points.

The classic determination of $\alpha$ by means of the
time dependent CP  asymmetry in the decay
$B_d^0 \rightarrow \pi^+ \pi^-$ 
is affected by the "QCD penguin pollution" which has to be
taken care of in order to extract $\alpha$. 
The recent CLEO results for penguin dominated decays indicate that
this pollution could be substantial as stressed recently in particular
in \cite{ITAL}.
The most popular strategy to deal with this "penguin problem''
is the isospin analysis of Gronau and London \cite{CPASYM}. It
requires however the measurement of $Br(B^0\to \pi^0\pi^0)$ which is
expected to be below $10^{-6}$: a very difficult experimental task.
For this reason several, rather involved, strategies \cite{SNYD} 
have been proposed which
avoid the use of $B_d \to \pi^0\pi^0$ in conjunction with
$a_{CP}(\pi^+\pi^-,t)$. They are reviewed in \cite{BF97}. 
 It is to be seen which of these methods
will eventually allow us to measure $\alpha$ with a respectable precision.
It is however clear that the determination of this angle is a real
challenge for both theorists and experimentalists.

The CP-asymmetry in the decay $B_d \rightarrow \psi K_S$ allows
 in the Standard Model
a direct measurement of the angle $\beta$ in the unitarity triangle
without any theoretical uncertainties \cite {BSANDA}.
Of considerable interest \cite{RF97,PHI} is also the pure penguin decay
$B_d \rightarrow \phi K_S$, which is expected to be sensitive
to physics beyond the Standard Model. Comparision of $\beta$
extracted from $B_d \rightarrow \phi K_S$ with the one from
$B_d \rightarrow \psi K_S$ should be important in this
respect. An analogue of $B_d \rightarrow \psi K_S$ in $B_s$-decays
is $B_s \rightarrow \psi \phi$. The CP asymmetry measures here
$\eta$ \cite{B95} in the Wolfenstein parametrization. It is very
small, however, and this fact makes it a good place to look for the 
physics beyond the Standard Model. In particular the CP violation
in $B^0_s-\bar B^0_s$ mixing from new sources beyond the Standard
Model should be probed in this decay.

The two theoretically cleanest methods for the determination of $\gamma$
are: i) the full time dependent analysis of 
$B_s\to D^+_s K^{-}$ and $\bar B_s\to D^-_s K^{+}$  \cite{adk}
and ii) the well known triangle construction due to Gronau and Wyler 
\cite{Wyler}
which uses six decay rates $B^{\pm}\to D^0_{CP} K^{\pm}$,
$B^+ \to D^0 K^+,~ \bar D^0 K^+$ and  $B^- \to D^0 K^-,~ \bar D^0 K^-$.
Both methods are  unaffected by penguin contributions. 
The first method is experimentally very
challenging because of the
expected large $B^0_s-\bar B^0_s$ mixing. The second method is problematic
because of the small
branching ratios of the colour supressed channel $B^{+}\to D^0 K^{+}$
and its charge conjugate,
giving a rather squashed triangle and thereby
making
the extraction of $\gamma$ very difficult. Variants of the latter method
which could be more promising have been proposed in \cite{DUN2,V97}.
It appears that these methods will give useful results at later stages
of CP-B investigations. In particular the first method will be feasible
only at LHC-B.

All this has been known already for some time and is well documented
in the literature \cite{BF97,RF97}. Let us now be more explicit on
the most recent developments which deal with the extraction of
the angle $\gamma$ from the decays $B^0_d\to\pi^-K^+$, 
$B^+\to\pi^+K^0$ and their charge conjugates  
\cite{PAPIII}--\cite{defan}. These modes, which have recently been observed 
by the CLEO collaboration \cite{cleo}, should allow us to obtain direct 
information on $\gamma$ at future $B$-factories (BaBar, 
BELLE, CLEO III) (for interesting feasibility studies, see 
\cite{groro,wuegai,babar}). At present, there are only experimental results 
available for the combined branching ratios of these modes, i.e.\ averaged 
over decay and its charge conjugate, suffering from large hadronic
uncertainties. 

In order to determine the CKM angle $\gamma$ by using the strategy proposed
in \cite{PAPIII} (see also \cite{groro}), the separate branching ratios 
for $B^0_d\to\pi^-K^+$, $B^+\to\pi^+K^0$ and their charge conjugates are 
needed, i.e.\ the combined branching ratios are not sufficient, and an 
additional input is required to fix the magnitude of a certain decay 
amplitude $T$, which is usually referred to as a ``tree'' amplitude. Using 
arguments based on the factorization discussed in section 9, one expects 
that a future theoretical 
uncertainty of $|T|$ as small as ${\cal O}(10\%)$ may be achievable 
\cite{groro,wuegai}. Unfortunately detailed studies show, that the properly 
defined 
amplitude $T$ is actually not just a colour-allowed ``tree'' amplitude, 
where factorization may work reasonably well \cite{bjorken}. 
It receives also contributions from penguin and annihilation topologies 
due to certain rescattering effects \cite{defan,bfm} and consequently
the expectations in \cite{groro,wuegai}
appear too optimistic. In any case, some model dependence enters in the 
extracted value of $\gamma$ by means of these decays.

In this context
an interesting method for constraining $\gamma$,
which  
does not suffer from a model 
dependence related to $|T|$, is the method of 
Fleischer and Mannel \cite{fm2}. 
This method
uses only the combined rates for $B^{\pm}\to\pi^{\pm}K$ and 
$B_d\to\pi^{\mp}K^{\pm}$.
Assuming that the final state interactions and electroweak penguin 
contributions are small,
one finds  the bound:
\be\label{FMBOUND}
\sin^2\gamma \le 
\frac{Br(B_d\to\pi^{\mp}K^{\pm})}{Br(B^{\pm}\to\pi^{\pm}K)}\equiv R~.
\ee
The Fleischer-Mannel bound is of particular interest because the most 
recent CLEO data give $R=0.65\pm 0.40$ \cite{cleo}.
If true,
the FM--bound with $R<1$
would exclude the region around $\bar\varrho=0$ in the 
$(\bar\varrho,\bar\eta)$ space  putting 
the "$\gamma=90^\circ$ club" \cite{BjSt} into serious difficulties. It
should be stressed that excluding the region around $\bar\varrho=0$
would have a profound impact on the unitarity triangle dividing the
allowed region for its apex into well separated regions with
$\bar\varrho<0$ and $\bar\varrho>0$. The former could then probably
be eliminated by improving the lower bound on $\Delta M_s$ leaving
only a small allowed area with $\bar\varrho>0$. 
More details on
the implications of the FM--bound can be found in 
\cite{fm2,GNF,FRENCH}.

The crucial questions
then are, whether R is indeed smaller than unity  and whether
the assumptions used to obtain the FM bound can be justified.
The first question will hopefully be answered by CLEO and future
B factories. Here we concentrate on the second question.
Indeed,
the theoretical accuracy of the FM bound on $\gamma$ is limited by 
rescattering processes of the kind $B^+\to\{\pi^0K^+,\,\pi^0K^{\ast +},\,
\rho^0K^{\ast +},\,\ldots\,\}\to\pi^+K^0$ \cite{gewe}--\cite{atso} (for
earlier references, see \cite{FSI}), and by contributions from electroweak 
penguins \cite{groro,neubert,fm3}, which led to considerable interest in 
the recent literature. 

In order to gain some insight into this issue,
a completely general 
parametrization of the $B^+\to\pi^+K^0$ and $B^0_d\to\pi^-K^+$ decay 
amplitudes was presented in \cite{defan}, 
relying only on the isospin symmetry of strong 
interactions and the phase structure of the Standard Model. This 
parametrization leads to the following transparent expression for the 
minimal value of $R$:
\begin{equation}\label{Rmin}
R_{\rm min}=\kappa\,\sin^2\gamma\,+\,
\frac{1}{\kappa}\left(\frac{A_0}{2\,\sin\gamma}\right)^2,
\end{equation}
where the ``pseudo-asymmetry'' $A_0$ is defined by
\begin{equation}
A_0\equiv\frac{Br(B^0_d\to\pi^-K^+)-
Br(\overline{B^0_d}\to\pi^+K^-)}{Br(B^+\to\pi^+K^0)+
Br(B^-\to\pi^-\overline{K^0})}=
A_{\rm CP}(B_d\to\pi^\mp K^\pm)\,R~.
\end{equation}
Rescattering and electroweak penguin effects are included through the 
parameter $\kappa$, which is given by 
\begin{equation}
\kappa=\frac{1}{w^2}\left[\,1+2\,(\epsilon\,w)\cos\Delta+
(\epsilon\,w)^2\,\right]
\end{equation}
with
\begin{equation}\label{w-def}
w\equiv\sqrt{1+2\,\rho\,\cos\theta\cos\gamma+\rho^2}\,.
\end{equation}
The parameters $\rho$ and $\epsilon$ measure the ``strengths'' of the
rescattering processes and electroweak penguin contributions, respectively,
and $\theta$ and $\Delta$ are CP-conserving strong phases. Simple model 
estimates typically give values of $\rho$ and $\epsilon$ at the level of 
$1\%$. However, in a recent attempt to evaluate rescattering processes such 
as $B^+\to\{\pi^0K^+\}\to\pi^+K^0$, it is found that $\rho$ may be as large 
as ${\cal O}(10\%)$ \cite{fknp}. A similar feature arises also in a simple 
model to describe final-state interactions, which assumes elastic 
rescattering processes and has been proposed in \cite{gewe,neubert}. 
Also electroweak penguins may play a more important role than naively 
expected \cite{groro,neubert,fm3}, so that $\epsilon$ may actually be of 
${\cal O}(10\%)$. 

A detailed study of the impact of these effects on the generalized
bound on $\gamma$ 
related to (\ref{Rmin}) was performed in \cite{defan}. The ``original''
bound derived in \cite{fm2} corresponds to $\kappa=1$ and sets effectively
the asymmetry $A_0$ to zero. As soon as a non-vanishing experimental 
result for 
$A_0$ has been established, also an interval around $\gamma=0^\circ$ and 
$180^\circ$ can be ruled out, while the impact on the excluded region around 
$90^\circ$ is rather small \cite{defan}. 

An interesting feature of the rescattering effects is that they may lead to 
sizeable CP violation in the decay $B^+\to\pi^+K^0$ 
\cite{gewe}--\cite{atso}, 
in contrast to simple 
quark-level estimates, from which at most a few percent for this CP 
asymmetry~\cite{pert-pens} could be expected. 
This CP asymmetry provides a first step towards 
the experimental control of rescattering processes \cite{defan}. The 
rescattering effects can be included in the generalized bounds on 
$\gamma$ completely 
by using additional experimental information on the decay $B^+\to K^+
\overline{K^0}$ and its charge conjugate \cite{defan,rf-FSI}. 
Different strategies to constrain rescattering effects have also been
considered in \cite{fknp}.

At first sight, an experimental study of $B^+\to K^+ \overline{K^0}$ appears 
to be challenging, since model estimates performed at the perturbative quark 
level give a combined branching ratio 
$Br(B^\pm\to K^\pm K)={\cal O}(10^{-6})$, which is one order of 
magnitude below the present upper limit $2.1\times10^{-5}$ obtained by the 
CLEO collaboration. However, as was pointed out in \cite{defan,rf-FSI}, 
rescattering processes may well enhance this branching ratio by 
${\cal O}(10)$, so that it may be possible to study this 
mode to obtain insights into final state interactions at future $B$-factories.
Also electroweak penguins can be constrained by using additional information
\cite{defan}, and certainly experiment will tell us one day how important
rescattering processes and electroweak penguins in $B\to\pi K$ decays really
are. An interesting probe of $\gamma$ is also provided by $B_s\to K
\overline{K}$ decays, which can be combined with their $B_{u,d}
\to\pi K$ counterparts through the $SU(3)$ flavour symmetry \cite{bskk}. 

Finally I would like to mention a recent interesting paper of Lenz,
Nierste and Ostermaier \cite{LNO}, 
where inclusive direct CP-asymmetries in 
charmless $B^{\pm}$-decays including QCD effects have been studied.
These asymmetries should offer useful means to constrain the unitarity
triangle.

\subsection{CP-Asymmetries in $B$-Decays versus $K \to \pi \nu\bar\nu$}
Let us next compare the potentials of the CP asymmetries in
determining the parameters of the Standard Model with those
of the cleanest rare $K$-decays: $K_{\rm L}\to\pi^0\nu\bar\nu$ and
$K^+\to\pi^+\nu\bar\nu$.

To this end let us assume that the problems with the determination
of $\alpha$ will be solved somehow. Since in the usual rescaled 
unitarity triangle  one side is known, it suffices to measure
two angles to determine the triangle completely. This means that
the measurements of $\sin 2\alpha$ and $\sin 2\beta$ can determine
the parameters $\varrho$ and $\eta$.
As the standard analysis of the unitarity triangle of section 10
shows, $\sin 2\beta$ is expected to be large: $\sin 2\beta=0.58\pm 0.22$
implying the time-integrated CP asymmetry  
$a_{\rm CP}(B_d\to J/\psi K_{\rm S})$
as high as $(30 \pm 10)\%$.
The prediction for $\sin 2\alpha$ is very
uncertain on the other hand $(0.1\pm0.9)$ and even a rough measurement
of $\alpha$ would have a considerable impact on our knowledge of
the unitarity triangle as stressed in \cite{BLO,BB96}.

Measuring then $\sin 2\alpha$ and $\sin 2\beta$ from CP asymmetries in
$B$ decays allows, in principle, to fix the 
parameters $\bar\eta$ and $\bar\varrho$, which can be expressed as
\cite{AJB94}
\begin{equation}\label{ersab}
\bar\eta=\frac{r_-(\sin 2\alpha)+r_+(\sin 2\beta)}{1+
r^2_+(\sin 2\beta)}\,,\qquad
\bar\varrho=1-\bar\eta r_+(\sin 2\beta)\,,
\end{equation}
where $r_\pm(z)=(1\pm\sqrt{1-z^2})/z$.
In general the calculation of $\bar\varrho$ and $\bar\eta$ from
$\sin 2\alpha$ and $\sin 2\beta$ involves discrete ambiguities.
As described in \cite{AJB94}
they can be resolved by using further information, e.g.\ bounds on
$|V_{ub}/V_{cb}|$, so that eventually the solution (\ref{ersab})
is singled out.

Let us then consider two scenarios of the measurements of CP asymmetries 
in $B_d\to\pi^+\pi^-$ and $B_d\to J/\psi K_{\rm S}$, expressed in terms 
of $\sin 2\alpha$ and
$\sin 2\beta$:
\begin{equation}\label{sin2a2bI}
\sin 2\alpha=0.40\pm 0.10\,, \qquad \sin 2\beta=0.70\pm 0.06
\qquad ({\rm scenario\ I})
\end{equation}
\begin{equation}\label{sin2a2bII}
\sin 2\alpha=0.40\pm 0.04\,, \qquad \sin 2\beta=0.70\pm 0.02
\qquad ({\rm scenario\ II})\,.
\end{equation}
Scenario I corresponds to the accuracy being aimed for at $B$-factories
and HERA-B prior to the LHC era. An improved precision can be anticipated from
LHC experiments, which we illustrate with the scenario II.

In table \ref{tabkb} this way of the determination of
the Standard Model parameters is compared with the analogous analysis
using $\klpn$ and $\kpn$ which has been presented in section 13. We
recall that in the latter analysis
the following input has been used:
\begin{equation}\label{vcbmt}
|V_{cb}|=0.040\pm 0.002(0.001)\,, \qquad m_t=(170\pm 3) \mbox{GeV}
\end{equation}
\begin{equation}\label{bklkp}
Br(K_{\rm L}\to\pi^0\nu\bar\nu)=(3.0\pm 0.3)\cdot 10^{-11}\,,\qquad
Br(K^+\to\pi^+\nu\bar\nu)=(1.0\pm 0.1)\cdot 10^{-10}\,.
\end{equation}
The value  $|V_{cb}|=0.040\pm 0.002(0.001)$ is also used in B physics
scenarios I and II respectively.

\begin{table}
\caption[]{Illustrative example of the determination of CKM
parameters from $K\to\pi\nu\bar\nu$ and B-decays.
\label{tabkb}}
\vspace{0.4cm}
\begin{center}
\begin{tabular}{|c||c||c|c|}\hline
&$K\to\pi\nu\bar\nu$ 
& {\rm Scenario I} & {\rm Scenario II}
\\ 
\hline
\hline
$\sigma(|V_{td}|) $& $\pm 10\% (9\% )$
& $\pm 5.5\% (3.5\%)$ & $\pm 5.0\% (2.5\%)$\\ 
\hline 
$\sigma(\bar\varrho) $ & $\pm 0.16 (0.12)$
& $\pm 0.03$  & $\pm 0.01$\\
\hline
$\sigma(\bar\eta)$ & $\pm 0.04(0.03)$
&$\pm 0.04 $ & $\pm 0.01 $\\
\hline
$\sigma(\sin 2\beta)$ & $\pm 0.05$
& $\pm 0.06 $ & $\pm 0.02$\\
\hline
$\sigma({\rm Im}\lambda_t)$&$\pm 5\%$ 
& $\pm 14\%(11\%)$ & $\pm 10\%(6\%)$\\
\hline
\end{tabular}
\end{center}
\end{table}
As can be seen in table \ref{tabkb}, the CKM determination
using $K\to\pi\nu\bar\nu$ is competitive with the one based
on CP violation in $B$ decays in scenario I, except for $\bar\varrho$ which
is less constrained by the rare kaon processes.
On the other hand as advertised previously ${\rm Im}\lambda_t$ 
is better determined
in $K\to\pi\nu\bar\nu$ even if scenario II is considered.
The virtue of the comparision of the determinations
of various parameters using CP-B asymmetries with the determinations
in very clean decays $K\to\pi\nu\bar\nu$ is that any substantial deviations
from these two determinations would signal new physics beyond the
Standard Model.
 Formula (\ref{kbcon}) is an example of such a comparison.

\subsection{Unitarity Triangle from $\klpn$ and $\sin 2\alpha$}
Next, results from CP asymmetries in $B$ decays could also be
combined with measurements of $K\to\pi\nu\bar\nu$.
As an illustration we would like to present a scenario \cite{BB96}
where
the unitarity triangle is determined by $\lambda$, $V_{cb}$,
$\sin 2\alpha$ and $Br(K_{\rm L}\to\pi^0\nu\bar\nu)$.
In this case $\bar\eta$ follows directly from 
$Br(K_{\rm L}\to\pi^0\nu\bar\nu)$ (\ref{bklpn1}) and $\bar\varrho$ is
obtained using \cite{AJB94}
\begin{equation}\label{rhoalpha}
\bar\varrho=\frac{1}{2}-\sqrt{\frac{1}{4}-\bar\eta^2+
\bar\eta r_-(\sin 2\alpha)}\,,
\end{equation}
where $r_-(z)$ is defined after (\ref{ersab}).
The advantage of this strategy is that most CKM quantities are
not very sensitive to the precise value of $\sin 2\alpha$.
Moreover a high accuracy in 
${\rm Im}\lambda_t$ is automatically guaranteed. As shown in
table \ref{tabkl2a}, very respectable results can be expected
for other quantities as well with only modest requirements
on the accuracy of $\sin 2\alpha$. 
It is conceivable that theoretical uncertainties due to penguin
contributions could eventually be brought under control at least
to the level assumed in table \ref{tabkl2a}. 
\begin{table}
\caption[]{Determination of the CKM matrix from $\lambda$, $V_{cb}$,
$K_{\rm L}\to\pi^0\nu\bar\nu$ and $\sin 2\alpha$ from the CP asymmetry
in $B_d\to\pi^+\pi^-$ \cite{BB96}. Scenario A (B) assumes
$V_{cb}=0.040\pm 0.002 (\pm 0.001)$
and $\sin 2\alpha=0.4\pm 0.2 (\pm 0.1)$. In both cases we take
$Br(K_{\rm L}\to\pi^0\nu\bar\nu)\cdot 10^{11}=3.0\pm 0.3$ and
$\mt=(170\pm 3)\gev$. 
\label{tabkl2a}}
\begin{center}
\begin{tabular}{|c||c|c|c|}\hline
&&A&B \\
\hline
\hline
$\bar\eta$&$0.380$&$\pm 0.043$&$\pm 0.028$ \\
\hline
$\bar\varrho$&$0.070$&$\pm 0.058$&$\pm 0.031$ \\
\hline
$\sin 2\beta$&$0.700$&$\pm 0.077$&$\pm 0.049$ \\
\hline
$|V_{td}|/10^{-3}$&$8.84$&$\pm 0.67$&$\pm 0.34$ \\
\hline
$|V_{ub}/V_{cb}|$&$0.087$&$\pm 0.012$&$\pm 0.007$ \\
\hline 
\end{tabular}
\end{center}
\end{table}
As an alternative, $\sin 2\beta$ from $B_d\to J/\psi K_{\rm S}$ 
could be used as an independent input instead of $\sin 2\alpha$.
Unfortunately the combination of $K_{\rm L}\to\pi^0\nu\bar\nu$ and
$\sin 2\beta$ tends to yield somewhat less restrictive constraints
on the unitarity triangle \cite{BB96}. 
On the other hand it has of course the
advantage of being practically free of any theoretical uncertainties.   

\subsection{Unitarity Triangle and $\vcb$ from $\sin 2\alpha$,
$\sin 2\beta$ and $\klpn$}
As proposed in \cite{AJB94},
unprecedented precision for all basic CKM
parameters could be achieved by combining the cleanest $K$ and 
$B$ decays. 
While $\lambda$ is obtained as usual from
$K\to\pi e\nu$, $\bar\varrho$ and $\bar\eta$ could be determined
from $\sin 2\alpha$ and $\sin 2\beta$ as measured in CP
violating asymmetries in $B$ decays. Given $\eta$, one could
take advantage of the very clean nature of $K_{\rm L}\to\pi^0\nu\bar\nu$
to extract $A$ or, equivalently $|V_{cb}|$. As seen in (\ref{vcbklpn}),
this determination
benefits further from the very weak dependence of $|V_{cb}|$ on
the $K_{\rm L}\to\pi^0\nu\bar\nu$ branching ratio, which is only with
a power of $0.25$. Moderate accuracy in $Br(K_{\rm L}\to\pi^0\nu\bar\nu)$
would thus still give a high precision in $|V_{cb}|$.
As an example we take $\sin 2\alpha=0.40\pm 0.04$,
$\sin 2\beta=0.70\pm 0.02$ and 
$Br(K_{\rm L}\to\pi^0\nu\bar\nu)=(3.0\pm 0.3)\cdot 10^{-11}$,
$m_t=(170\pm 3)$ GeV. 
This yields \cite{BB96}:
\begin{equation}\label{rhetvcb}
\bar\varrho=0.07\pm 0.01\,,\qquad
\bar\eta=0.38\pm 0.01\,,\qquad
|V_{cb}|=0.0400\pm 0.0013\,,
\end{equation}
which would be a truly remarkable result. Again the comparision of
this determination of $|V_{cb}|$ with the usual one in tree level
$B$-decays would offer an excellent test of the Standard Model
and in the case of discrepancy would signal physics beyond 
it.  

\subsection{Unitarity Triangle from $R_t$ and $\sin 2\beta$} 
Another strategy is to use the measured value of $R_t$ together with
$\sin 2\beta$. Useful measurements of $R_t$ can be achieved
using the ratios $Br(B\to X_d \nu\bar\nu)/Br(B\to X_s \nu\bar\nu)$,
$\Delta M_d/\Delta M_s$,
$Br(B_d\to l^+l^-)/Br(B_s \to l^+l^-)$
 and $Br(\kpn)$. Then (\ref{ersab})
is replaced by \cite{B95}
\begin{equation}\label{5a}
\bar\eta=\frac{R_t}{\sqrt{2}}\sqrt{\sin 2\beta \cdot r_{-}(\sin 2\beta)}\,,
\quad\quad
\bar\varrho = 1-\bar\eta r_{+}(\sin 2\beta)\,.
\end{equation}
The numerical results of this exercise can be found in \cite{B95}.
Additional strategies involving the angle $\gamma$ 
can be found in \cite{BLO}.
\section{Summary and Outlook}
\setcounter{equation}{0}
%\setcounter{figure}{0}
%\setcounter{table}{0}
We are approaching the end of our tour. I hope that some of you
enjoyed reading these  lectures as much as I did preparing,
delivering and finally writing them.
The collection of many techniques and formulae should be useful
in various phenomenological applications and constitutes hopefully
a good introduction to future research.
I hope that I have convinced the students that the field of weak decays
plays an important 
role in the deeper understanding of the Standard Model 
and particle physics in general.
Indeed the field of weak decays and of CP violation is one of the least
understood sectors of the Standard Model.
Even if the Standard Model is fully consistent with the existing data for
weak decay processes, the near future could change 
this picture
dramatically through the advances in experiment and theory.
In particular the experimental work
done in the next ten
years at BNL, CERN, CORNELL, DA$\Phi$NE, DESY, 
FNAL, KEK, SLAC and eventually LHC will certainly 
have considerable impact on this field.

Before closing these lectures with a few final messages, I would
like to make a list of things we could expect in the next ten years.
This list is certainly very biased by my own interests but could
be useful anyway. Here we go:

\begin{itemize}
\item
The error on the CKM elements $\vcb$ and $\vub$ could be decreased 
below 0.002 and 0.01, respectively. This progress should come mainly from
Cornell, $B$-factories and new theoretical efforts. It would have
considerable impact on the unitarity triangle and would improve
theoretical predictions for rare and CP-violating decays sensitive
to these elements.
\item
The error on $\mt$ should be decreased down to $\pm 3\gev$
at Tevatron in the Main Injector era and to $\pm 1\gev$ at LHC.
\item
The improved measurements of $\epe$ with the accuraccy of
 $\pm (1-2) \cdot 10^{-4}$ 
from CERN, FNAL and DA$\Phi$NE should give some insight into the 
physics of 
direct CP violation inspite of large theoretical uncertainties. 
Excluding confidently the superweak models would be an important result. 
In this respect measurements of CP-violating asymmetries in charged $B$
decays will also play an outstanding role. These experiments can be
performed e.g.\ at CLEO since no time-dependences
are needed. The situation concerning hadronic uncertainties is quite similar
to $\epe$. Although these CP asymmetries cannot be calculated
reliably, any measured non-vanishing values would unambiguously rule out 
superweak scenarios. Simultaneously one should hope 
that some definite progress in calculating relevant hadronic matrix elements 
will be made. 
\item
More events for $K^+\to\pi^+\nu\bar\nu$ could in principle
be seen at BNL already this or next year. In view of the theoretical 
cleanliness of this decay an observation of events at the $2\cdot 10^{-10}$
level would signal physics beyond the Standard Model.
A detailed study of this very
important decay requires, however, new experimental ideas and
new efforts. The new efforts \cite{AGS2,Cooper} in this direction allow 
to hope that
a measurement of $Br(\kpn)$ with an accuracy of $\pm 10 \%$ should
be possible before 2005. This would have a very important impact
on the unitarity triangle and would constitute an important test of
the Standard Model.
\item
The future improved inclusive $B \to X_{s,d} \gamma$ measurements
confronted with improved Standard Model predictions could
give the first signals of new physics. It appears that the errors
on the input parameters could be lowered further and the
theoretical error on $Br(B\to X_s\gamma)$ could be decreased
confidently down to $\pm 8 \%$ in the next years. The same
accuracy in the experimental branching ratio will hopefully
come soon from CLEO II and later from KEK and SLAC. 
This may, however, be insufficient to
disentangle new physics contributions although such an accuracy
should put important constraints on the physics beyond the Standard
Model. It would also be desirable to look for $B \to X_d \gamma$,
but this is clearly a much harder task.
\item
Similar comments apply to transitions $B \to X_s l^+l^-$ (not discussed
here)
which appear to be even  more sensitive to new physics contributions
than $ B \to X_{s,d} \gamma$. An observation of
$B \to X_s \mu\bar\mu$ is expected from D0 and $B$-physics dedicated
experiments at the beginning of the next 
decade. The distributions of various kind when measured should
be very useful in the tests of the Standard Model and its extensions.
\item
The theoretical status of $K_{\rm L}\to \pi^0 e^+ e^-$ and of 
$K_{\rm L}\to \mu\bar\mu$, which we did not cover here, 
should be improved to confront future
data. Experiments at DA$\Phi$NE should be very helpful in this
respect. The first events of $K_{\rm L}\to \pi^0 e^+ e^-$ should
come in the first years of the next decade from KAMI at FNAL.
The experimental status of $K_{\rm L}\to \mu\bar\mu$, with the 
experimental error of $\pm 7\%$ to be decreased soon down to $\pm 1\%$,
is truly impressive.
\item
The newly approved experiment at BNL to
measure $Br(\klpn)$ at the $\pm 10\%$ level before 2005 may make a decisive
impact on the field of CP violation. 
In particular $\klpn$ seems to allow the
cleanest determination of $\imlt$. Taken together with $\kpn$
a very clean determination of $\sin 2 \beta$ can be obtained.
\item
The measurement of the $B^0_s-\bar B^0_s$ mixing and in particular of
$B \to X_{s,d}\nu\bar\nu$ and 
$B_{s,d}\to \mu\bar\mu$ will take most probably longer time but
as stressed in these lectures all efforts should be made to measure
these transitions. Considerable progress on $B^0_s-\bar B^0_s$ mixing
should be expected from HERA-B, SLAC and TEVATRON in the first years
of the next decade. LHC-B should measure it to a high precision.
With the improved calculations of $\xi$ in (\ref{107b}) this will have
important impact on the determination of $\vtd$ and on the
unitarity triangle. 
\item
Clearly future precise studies of CP violation at SLAC-B, KEK-B, 
HERA-B, CORNELL, FNAL and  LHC-B providing first
direct measurements of $\alpha$, $\beta$ and $\gamma$ may totally
revolutionize our field. In particular the first signals
of new physics could be found in the $(\bar\varrho,\bar\eta)$ plane.
During the recent years several, in some cases quite sophisticated and
involved, strategies have been developed to extract these angles with
small or even no hadronic uncertainties. Certainly the future will bring
additional methods to determine $\alpha$, $\beta$ and $\gamma$. 
Obviously it is very desirable to have as many such strategies as possible
available in order to overconstrain the unitarity triangle and to resolve 
certain discrete ambiguities which are a characteristic feature of these 
methods.
\item
The forbidden or strongly suppressed transitions such as
$D^0-\bar D^0$ mixing and $K_{\rm L}\to \mu e$ are also very
important in this respect. Considerable progress in this area
should come from the experiments at BNL, FNAL and KEK.
\item
On the theoretical side,
one should hope that the non-perturbative
methods will be considerably improved so that various $B_i$ parameters
will be calculated with sufficient precision. It is very important
that simultaneously with advances in lattice QCD, further efforts
are being made in finding efficient analytical tools for calculating
QCD effects in the long distance regime. This is, in particular very
important in the field of non-leptonic decays, where one should
not expect too much from our lattice friends in the coming ten years
unless somebody will get a brilliant idea which will revolutionize
lattice calculations. The accumulation of data for non-leptonic $B$
and $D$
decays at Cornell, SLAC, KEK and FNAL should teach us more 
about the role of non-factorizable contributions and in particular
about the final state interactions. 
In this context, in the case of K-decays, important
lessons will come from DA$\Phi$NE which is an excellent machine
for testing chiral perturbation theory and other non-perturbative
methods. 
\end{itemize}

In any case the field of weak decays and in particular of the FCNC 
transitions and of CP violation have a great future and
one should expect that they could dominate particle physics in the first 
part of the next decade. 
Clearly the next ten years should be very exciting in this field
and it is advisable
to buy shares before it is too late.

\section{Final Messages}

The two weeks I have spent in Les Houches in August 1997 will remain in my
memory for ever. Therefore I would like to close these lectures by thanking
those who contributed most to this happening.

First of all I would like to thank Rajan Gupta and Francois David for
inviting me to this school and keeping me busy. In particular I would like to
thank Rajan for creating such a pleasent atmosphere and his persistent
e-mails reminding me that it is time to finish writing up these lectures.

However my warmest thanks go to the students of this school 
who made the sixteen hours of my presence in front of the blackboard
and the remaining time a real joy. In particular:
\bi
\item
 Many thanks to the magnificant seven: Fabien Motsch, Markus Peter, Solveig
Skadhauge, Thomas Teubner, Anja Werthenbach, Joerg Westphalen and Stefan
Wienzerl for keeping me alive during a two day mountain expedition.
Champagne offered after this tour by a very special
student of this school, Leung Ka Chun, will never be forgotten.
\item
 The results of our expedition appeared in hep-ph/9708777 under the title
``No Loops beyond the Trees in the Splittorff Renormalization Scheme", where
further details can be found. Splittorff, the youngest student of the school
was the only one of this Les Houches session to climb Mont Blanc. There is
nothing exciting in hep-ph/9708777 except one thing: 
this work will go down in history as yet another 
Buras et al. paper.
\item
 Many thanks to Luca Girlanda, Nicos Irges and Leszek Motyka 
for arranging table tennis
championships and to Andrzej Czarnecki for giving me Polysporin which allowed
me to reach quarter finals where I was slaughtered by a spanish
matador (Francisco Guerrero).
\item
 From all these remarks it is clear that I had rather close contacts with
the students of this school. 
Yet my closest companions, day and night, were
the washing machine and the dryer both placed next to my room.
The lively discussions, in particular at night, in front of my door
forced me to work hard on my lectures, except for the last night of my stay
when following the advice of the sole experimentalist
of the school (Fabien Motsch) I switched off these two important
inventions of this century.
\ei

I hope that these final comments made it clear why I have enjoyed this school
so much. Many thanks to all of you.    

Particular thanks go to Markus Lautenbacher for creating many figures
and a number of numerical calculations. 
I would also like to thank Robert Fleischer, Paolo Gambino, 
Axel Kwiatkowski, Mikolaj Misiak, Nicolas Pott
and Luca Silvestrini for helpful discussions during  the preparation of
these lectures. 

This work has been supported by the
German Bundesministerium f{\"u}r Bildung and Forschung under contract 
06 TM 874  and DFG Project Li 519/2-2.

\begin{thebibliography}{999}
\bibitem{CAB}
N. Cabibbo, Phys. Rev. Lett. {\bf 10} (1963) 531.
\bibitem{KM}
{ M. Kobayashi and K. Maskawa},
 { Prog. Theor. Phys.} {\bf 49} (1973) 652.
\bibitem{OPE}
K.G. Wilson, { Phys. Rev.} {\bf 179} (1969) 1499;
K.G. Wilson and W. Zimmermann, { Comm. Math. Phys.} 
{\bf 24} (1972) 87.
\bibitem{ZIMM}
W. Zimmermann, in Proc. 1970 Brandeis Summer Institute in
Theor. Phys, (eds. S. Deser, M. Grisaru and H. Pendleton),
MIT Press, 1971, p.396; 
{ Ann. Phys.} {\bf 77} (1973) 570.
\bibitem{SUMA}
{ E.C.G. Sudarshan and R.E. Marshak}, Proc. Padua-Venice Conf. on 
Mesons and Recently Discovered Particles (1957).
\bibitem{GF}
{ R.P. Feynman and M. Gell-Mann,}
 { Phys. Rev.} {\bf 109} (1958) 193.
\bibitem{WIT}
E. Witten, { Nucl. Phys.} {\bf B 120} (1977) 189.
\bibitem{REGM}
E.C.G. Stueckelberg and A. Petermann, Helv. Phys. Acta {\bf 26} (1953)
499;
M. Gell--Mann and F.E. Low, { Phys. Rev.} {\bf 95} (1954) 1300;
L.V. Ovsyannikov, Dokl. Acad. Nauk SSSR {\bf 109} (1956) 1112;
K. Symanzik, Comm. Math. Phys. {\bf 18} (1970) 227;
C.G. Callan Jr, { Phys. Rev.} {\bf D 2} (1970) 1541.
\bibitem{HV1}
G. 't Hooft,
{ Nucl. Phys.} {\bf B 61} (1973) 455.
\bibitem{Weinberg}
S. Weinberg, { Phys. Rev.} {\bf D 8} (1973) 3497.
\bibitem{DIAG}
D. Zeppenfeld, Z. Phys. {\bf C 8} (1981) 77;
L.L. Chau, { Phys. Rev.} {\bf D 43} (1991) 2176;
M. Gronau, J.L. Rosner and D. London, Phys. Rev. Lett. {\bf 73} (1994) 21;
O.F. Hernandez, M. Gronau, J.L. Rosner and D. London,
{ Phys. Lett.} {\bf B 333} (1994) 500, { Phys. Rev.} {\bf D 50} (1994) 4529.
\bibitem{HQE1}
J. Chay, H. Georgi and B. Grinstein,
{ Phys. Lett.} {\bf B 247} (1990) 399.
\bibitem{HQE2}
I.I. Bigi, N.G. Uraltsev and A.I. Vainshtein,
{ Phys. Lett.} {\bf B 293} (1992) 430
[E: {\bf B 297} (1993) 477].
I.I. Bigi, M.A. Shifman, N.G. Uraltsev and A.I. Vainshtein,
Phys. Rev. Lett. {\bf 71} (1993) 496;
B. Blok, L. Koyrakh, M.A. Shifman and A.I. Vainshtein,
{ Phys. Rev.} {\bf D 49} (1994) 3356 [E: {\bf D 50} (1994) 3572].
\bibitem{HQE3}
A.V. Manohar and M.B. Wise,
{ Phys. Rev.} {\bf D 49} (1994) 1310.
\bibitem{GIM1}
{ S.L. Glashow, J. Iliopoulos and L. Maiani}
{ Phys. Rev.} {\bf D 2} (1970) 1285.
\bibitem{PBE0}
{ G. Buchalla, A.J. Buras and M.K. Harlander,} { Nucl. Phys.}
 {\bf B 349} (1991) 1.
\bibitem{BBL}
{ G. Buchalla, A.J. Buras and M. Lautenbacher,} 
{ Rev. Mod. Phys} {\bf 68} (1996) 1125.
\bibitem{BF97}
{ A.J. Buras and R. Fleischer,} hep-ph/9704376, to appear in \cite{HFII}.
\bibitem{HFII}
A.J. Buras and M. Lindner, Heavy Flavours II, World Scientific,
1998.
\bibitem{MUTA}
T. Muta, Foundations of Chromodynamics, World Scientific, 1987.
\bibitem{BOOK1}
M.E. Peskin and D.V. Schroeder, An Introduction to Quantum Field
Theory, Addison-Wesley Publishing Company.
\bibitem{BOOK1b}
F. Mandl and G. Shaw, Quantum Field Theory, John Wiley $\&$ Sons.
\bibitem{BOOK1c}
T.-P. Cheng and L.-F. Li, Gauge Theory of Elementary Particle
Physics, Clarendon Press, Oxford. 
\bibitem{BOOK1a}
L.H. Ryder, Quantum Field Theory, Cambridge University Press.
\bibitem{BOOK2}
J.F. Donoghue, E. Golowich and B.R. Holstein, Dynamics of the
Standard Model, Cambridge Monographs.
\bibitem{BOOK3}
D. Bailin and A. Love, Introduction to Gauge Field Theory,
Adam Hilger, Bristol and Boston.
\bibitem{BOOK4}
S. Pokorski, Gauge Field Theory, Cambridge Monographs.
\bibitem{BOOK5}
S. Weinberg, The Quantum Theory of Fields, Cambridge University
Press.
\bibitem{Collins}
J.C. Collins, Renormalization, Cambridge University Press.
\bibitem{CHAU}
{ L.L. Chau and W.-Y. Keung}, 
{ Phys. Rev. Lett.} {\bf 53} (1984) 1802.
\bibitem{PDG}
{ Particle Data Group,} { Phys. Rev.} {\bf D 54} (1996) 1.
\bibitem{WO}
{ L. Wolfenstein}, { Phys. Rev. Lett.} {\bf 51} (1983) 1945.
\bibitem{HALE}
{ H. Harari and M. Leurer,} { Phys. Lett.} {\bf B 181} (1986) 123.
\bibitem{BLO}
{ A.J. Buras, M.E. Lautenbacher and G. Ostermaier,}
{ Phys. Rev.} {\bf D 50} (1994) 3433.
\bibitem{schubert}
{ M. Schmidtler and K.R. Schubert}, { Z. Phys.} {\bf C 53}
(1992) 347.
\bibitem{js}
{ C. Jarlskog and R. Stora},
{ Phys. Lett.} {\bf B 208} (1988) 268.
\bibitem{LER1}
{ H. Leutwyler and M. Roos}, { Z. Physik} {\bf C25} (1984) 91.
\bibitem{DHK}
{ J.F. Donoghue, B.R. Holstein and S.W. Klimt,}
{ Phys. Rev.} {\bf D35} (1987) 934.
\bibitem{Gibbons}
{ L. Gibbons}, in proceedings of the 28th International Conference
on High Energy Physics, July 1996, Warsaw, Poland, page 183.
\bibitem{SUV}
{ M. Shifman, N.G. Uraltsev and A. Vainshtein,}
{ Phys. Rev.} {\bf D51} (1995) 2217;
I. Bigi, M. Shifman and N. Uraltsev, Ann. Rev. Nucl. Part. Sci.
47 (1997) 591.
\bibitem{Neubert}
{  M. Neubert,} { Phys. Lett.} {\bf B338} (1994) 84;
{ Int. J. Mod. Phys.} {\bf A11} (1996) 4173.
\bibitem{Braun}
{P. Ball, M. Beneke and V.M. Braun,} 
{ Phys. Rev.} {\bf D52} (1995) 3929.
\bibitem{CZMI}
A. Czarnecki, { Phys. Rev. Lett.} {\bf 76} (1996) 4124.
A. Czarnecki and K. Melnikov, 
{ Phys. Rev. Lett.} {\bf 78} (1997) 3630.
\bibitem{CLEOU}
{J.P. Alexander} et al. (CLEO), CLNS 96/1419, CLEO 96-9 (1996).
\bibitem{IL}
{ T. Inami and C.S. Lim,}
{ Progr. Theor. Phys.} {\bf 65} (1981) 297.
\bibitem{DDD}
G. Burdman, hep-ph/9407378, hep-ph/9508349.
\bibitem{AB80}
A.J. Buras,
{ Rev. Mod. Phys} {\bf 52} (1980) 199.
\bibitem{WEISZ}
{ A.J. Buras and P.H. Weisz,}
{ Nucl. Phys.} {\bf B 333} (1990) 66.
\bibitem{HV}
{ G. 't Hooft and M. Veltman},
{ Nucl. Phys.} {\bf B 44} (1972) 189.
\bibitem{BM}
{ P. Breitenlohner and D. Maison,} { Comm. Math. Phys.}
{\bf 52} (1977) 11, 39, 55.
\bibitem{Bo}
G. Bonneau, { Phys. Lett.} {\bf B 94} (1980) 147;
{ Nucl. Phys.} {\bf B 177} (1981) 523.
\bibitem{Si}
W. Siegel, { Phys. Lett.} {\bf B 84} (1979) 193.
\bibitem{ACMP}
{ G. Altarelli, G. Curci, G. Martinelli and S. Petrarca,}
{ Nucl. Phys.} {\bf B 187} (1981) 461.
\bibitem{Ma}
D. Maison, { Phys. Lett.} {\bf B 150} (1985) 39.
\bibitem{NT}
H. Nicolai and P.K. Townsend, { Phys. Lett.} {\bf B 93} (1980) 111;
P. Majumdar, E.C. Poggio and H.J. Schnitzer, 
{ Phys. Rev.} {\bf D 21} (1980) 2203.
\bibitem{CAND}
{ R.~Grigjanis, P.J.~O'Donnell, M.~Sutherland and H.~Navelet,} 
{ Phys.~Lett.} {\bf B213} (1988) 355;
{ Phys.~Lett.} {\bf B286} (1992) 413 E.
\bibitem{MISD}
M. Misiak, { Phys.~Lett.} {\bf B321} (1994) 113.
\bibitem{AD}
{ D.A. Akyeampong and R. Delbourgo}, { Nuovo Cim.}
{\bf 17A} (1973) 578, {\bf 18A} (1973) 94, {\bf 19A} (1974) 219.
\bibitem{KNS}
J.G. K\"orner, N. Nasrallah and K. Schilcher, 
{ Phys. Rev.} {\bf D 41} (1990) 888.
\bibitem{JaLau}
M. Jamin and M. Lautenbacher, Comput. Phys. Commun. 74 (1993) 265.
\bibitem{BBDM}
{ W.A. Bardeen, A.J. Buras, D.W. Duke and T. Muta},
{ Phys. Rev.} {\bf D 18} (1978) 3998.
\bibitem{Gross}
D.J. Gross,  Methods in Fleld Theory, (eds. R. Balian and J. Zinn-Justin),
North-Holland, 1976, p. 141.
\bibitem{Schmelling}
{ M. Schmelling}, in proceedings of the 28th International Conference
on High Energy Physics, July 1996, Warsaw, Poland, page 91.
\bibitem{MAR80}
W.J. Marciano, { Phys. Rev.} {\bf D 12} (1975) 3861.
\bibitem{BB1}
{ G. Buchalla and A.J. Buras,}
{ Nucl. Phys.} {\bf B 398} (1993) 285.
\bibitem{GH97}
{ C. Greub and T. Hurth,} { Phys. Rev.} {\bf D 56} (1997) 2934.
\bibitem{BKP2}
A.J. Buras, A. Kwiatkowski and N. Pott,
{ Nucl. Phys.} {\bf B 517} (1998) 353. 
\bibitem{BJLW1}
{ A.J. Buras, M. Jamin, M.E. Lautenbacher and P.H. Weisz,}
{ Nucl. Phys.} {\bf B 370} (1992) 69;
{ Nucl. Phys.} {\bf B 400} (1993) 37.
\bibitem{BJLW2}
{ A.J. Buras, M. Jamin and M.E. Lautenbacher,}
{ Nucl. Phys.} {\bf B 400} (1993) 75.
\bibitem{CURCI}
C. Curci and G. Ricciardi, { Phys. Rev.} {\bf D 47} (1993) 2991.
\bibitem{MISTRIK}
K. Chetyrkin, M. Misiak and M{\"u}nz, hep-ph/9711280; hep-ph/9711266.
\bibitem{MAIANI}
G. Altarelli and L. Maiani, { Phys. Lett.} {\bf B 52} (1974) 351;
M.K. Gaillard and B.W. Lee, { Phys. Rev. Lett.} {\bf 33} (1974) 108.  
\bibitem{BJLW}
{ A.J. Buras, M. Jamin and M.E. Lautenbacher,}
{ Nucl. Phys.} {\bf B 408} (1993) 209.
\bibitem{ROMA1}
{ M. Ciuchini, E. Franco, G. Martinelli and L. Reina,}
{ Phys. Lett.} {\bf B 301} (1993) 263.
\bibitem{ROMA2}
{ M. Ciuchini, E. Franco, G. Martinelli and L. Reina,}
{ Nucl. Phys.} {\bf B 415} (1994) 403.
\bibitem{GREEK}
N. Tracas and N. Vlachos, { Phys. Lett.} {\bf B 115} (1982) 419.
\bibitem{BKP1}
A.J. Buras, A. Kwiatkowski and N. Pott, 
{ Phys. Lett.} {\bf B 414} (1997) 157.
\bibitem{BuMu:94}
{ A.J. Buras and M. M{\"u}nz,}
{ Phys. Rev.} {\bf D 52} (1995) 186.
\bibitem{DuGr}
M.J. Dugan and B. Grinstein,
 { Phys. Lett.} {\bf B 256} (1991) 239.
\bibitem{HNE}
S. Herrlich and U. Nierste, { Nucl. Phys.} {\bf B 445} (1995) 39.
\bibitem{SH94}
S. Herrlich, Technical Univesity, PhD Thesis 1994 (in German).
\bibitem{UN95}
U. Nierste, Technical Univesity, PhD Thesis 1995.
\bibitem{FLEISCHP}
R. Fleischer, { Zeit. Phys.} {\bf C 62} (1994) 81.
\bibitem{PENGUIN}
A.I. Vainshtein, V.I. Zakharov and M.A. Shifman, JEPT {\bf 45} (1977) 670.
\bibitem{BH}
{ A.J. Buras and M.K. Harlander,} {\it A Top Quark Story, in
Heavy Flavours,} eds. A.J. Buras and M. Lindner, World Scientific,
1992, p.58.
\bibitem{BBHLS}
{ G. Buchalla, A.J. Buras, M.K. Harlander, M.E. Lautenbacher and C. Salazar,}
{ Nucl. Phys.} {\bf B 355} (1991) 305.
\bibitem{MW96}
P. Cho, M. Misiak and D. Wyler, { Phys. Rev.} {\bf D 54} (1996) 3329.
\bibitem{AAA}
A. Ali, Th. Mannel and Ch. Greub, { Zeit. Phys.} {\bf C 67} (1995) 417.
\bibitem{AJB94a}
A.J. Buras, { Nucl. Phys.} {\bf B 434} (1995) 606.
\bibitem{BJW90}
{ A.J. Buras, M. Jamin, and P.H. Weisz,}
{ Nucl. Phys.} {\bf B347} (1990) 491;\\
J. Urban, F. Krauss, U.Jentschura and G. Soff, 
{ Nucl. Phys.} {\bf B523} (1998) 40. 
\bibitem{BB3}
{ G. Buchalla and A.J. Buras,}
{ Nucl. Phys.} {\bf B 412} (1994) 106.
\bibitem{HNa}
{ S. Herrlich and U. Nierste,}
{ Nucl. Phys.} {\bf B419} (1994) 292. 
\bibitem{HNb}
{ S.~Herrlich and  U.~Nierste},
{ Phys. Rev.} {\bf D52} (1995) 6505; 
{ Nucl. Phys.} {\bf B476} (1996) 27. 
\bibitem{MisMu:94}
{ M.Misiak and M. M{\"u}nz,}
{ Phys. Lett.} {\bf B344} (1995) 308.
\bibitem{Buch:93}
{ G. Buchalla,} { Nucl. Phys.} {\bf B 391} (1993) 501.
\bibitem{Bagan}
{ E. Bagan, P.Ball, V.M. Braun and P.Gosdzinsky,}
{ Nucl. Phys.} {\bf B 432} (1994) 3;
{ E. Bagan} { et al.,} { Phys. Lett.} {\bf B 342} (1995) 362;
{\bf B 351} (1995) 546.
\bibitem{JP}
{ M. Jamin and A. Pich,}
{ Nucl.~Phys.} {\bf B425} (1994) 15.
\bibitem{BB2}
{ G. Buchalla and A.J. Buras,}
{ Nucl. Phys.} {\bf B 400} (1993) 225.
\bibitem{BB5}
{ G. Buchalla and A.J. Buras,}
{ Phys. Lett.} {\bf B 336} (1994) 263.
\bibitem{BLMM}
{ A. J. Buras, M. E. Lautenbacher, M. Misiak and M. M{\"u}nz,}
{ Nucl.~Phys.} {\bf B423} (1994) 349.
\bibitem{Mis:94}
{ M. Misiak,}
{ Nucl.~Phys.} {\bf B393} (1993) 23;
{ Erratum}, { Nucl.~Phys.} {\bf B439} (1995) 461.
\bibitem{Potte}
N. Pott, hep-ph/9710503.
\bibitem{Krakauer}
J. Krakauer, "Into Thin Air", Villard Books, New York, 1997.
\bibitem{BF95}
A.J. Buras and R. Fleischer, { Phys. Lett.} {\bf B 341} (1995) 379.
\bibitem{ITAL}
M. Ciuchini, E. Franco, G. Martinelli, and  L. Silvestrini,
{ Nucl. Phys.} {\bf B501} (1997) 271;
M. Ciuchini, R. Contino, E. Franco, G. Martinelli, and  L. Silvestrini,
{ Nucl. Phys.} {\bf B512} (1998) 3.
\bibitem{AG2} 
{  A.~Ali, and  C.~Greub,} { Z.Phys.} {\bf C49} (1991) 431;  
{ Phys.~Lett.} {\bf B259} (1991) 182;
{ Phys.~Lett.} {\bf B361} (1995) 146.
\bibitem{Yao1} {  K.~Adel and Y.P.~Yao,} 
{ Modern Physics Letters} {\bf A8} (1993) 1679;
{ Phys. Rev.} {\bf D 49} (1994) 4945.
\bibitem{Pott} 
{ N. Pott,} { Phys. Rev.} {\bf D 54} (1996) 938.
\bibitem{GREUB}
{ C. Greub, T. Hurth and D. Wyler,} { Phys.~Lett.} {\bf B380} 
(1996) 385; { Phys. Rev.} {\bf D 54} (1996) 3350; 
{ C. Greub and T. Hurth,} hep-ph/9608449.
\bibitem{CZMM}
{ K.G. Chetyrkin, M. Misiak and M. M{\"u}nz,} 
{ Phys. Lett.} {\bf B400} (1997) 206; hep-ph/9612313. 
\bibitem{GAMB}
M. Ciuchini, G. Degrassi, P. Gambino and G.F. Giudice, 
hep-ph/9710335.
\bibitem{strum} 
P. Ciafaloni, A. Romanino, and A. Strumia, 
hep-ph/9710312.
\bibitem{BUSI}
A.J. Buras and L. Silvestrini, TUM-HEP-315/98, hep-ph/9806278.
\bibitem{KR98}
A. Khodjamirian and R. R\"uckl, hep-ph/9801443, to appear in \cite{HFII}.
\bibitem{FEYNMAN}
J. Schwinger, { Phys. Rev. Lett.} {\bf 12} (1964) 630; 
R.P. Feynman, in {\it Symmetries in Particle Physics}, ed. A. Zichichi,
Acad. Press 1965, p.167; O. Haan and B. Stech, 
{ Nucl. Phys.} {\bf B 22}  (1970) 448.  
\bibitem{STECHF}
D. Fakirov and B. Stech, { Nucl. Phys.} {\bf B 133}  (1978) 315;
L.L. Chau, Phys. Rep. {\bf 95} (1983) 1.  
\bibitem{BAUER}
M. Wirbel, B. Stech and M. Bauer, { Z. Phys.}{\bf C 29} (1985) 637.
M. Bauer, B. Stech and M. Wirbel, { Z. Phys.}{\bf C 34} (1987) 103.
\bibitem{NEUBERT}
M. Neubert, V. Rieckert, B. Stech and Q.P. Xu, in ``Heavy Flavours",
 eds. A.J. Buras and M. Lindner (World Scientific, Singapore, 1992),
p. 286.
\bibitem{BJORKEN}
J.D. Bjorken,
{ Nucl. Phys.} {\bf B } (Proc. Suppl.) 11 (1989) 325;
SLAC-PUB-5389.
\bibitem{DUGAN}
M.J. Dugan and B. Grinstein,
{ Phys. Lett.} {\bf B 255} (1991) 583.
\bibitem{NS97}
M. Neubert and B. Stech,
[hep-ph/9705292], to appear in \cite{HFII};
B. Stech [hep-ph/9706384];
M.Neubert, Nucl. Phys. {\bf B } (Proc. Suppl.) 64 (1998) 474, 
[hep-ph/9801269].
\bibitem{ISGUR}
C. Reader and N. Isgur,
{ Phys. Rev.} {\bf D 47} (1993) 1007.
\bibitem{ITALY}
M. Ciuchini, R. Contino, E. Franco, G. Martinelli, L. Silvestrini,
hep-ph/9801420.
\bibitem{LNF}
D. Du and Z. Xing, { Phys. Lett.} {\bf B 312} (1993) 199;
A. Deandrea et al., { Phys. Lett.} {\bf B 318} (1993) 549,
{ Phys. Lett.} {\bf B 320} (1994) 170;
N.G. Deshpande, B. Dutta, S. Oh, { Phys. Rev.} {\bf D 57} (1998) 5723,
hep-ph/9712445.
\bibitem{Cheng}
H.-Y. Cheng, { Phys. Lett.} {\bf B 335} (1994) 428,
{ Phys. Lett.} {\bf B 395} (1997) 345;
H.-Y. Cheng and B. Tseng, [hep-ph/9708211], [hep-ph/9803457].
\bibitem{Soares}
J.M. Soares, { Phys. Rev.} {\bf D 51} (1995) 3518.
\bibitem{GNF}
A. Ali and C. Greub, { Phys. Rev.} {\bf D57} (1998) 2996;
A. Ali, J. Chay, C. Greub and P. Ko, 
{ Phys. Lett.} {\bf B 424} (1998) 161.
\bibitem{AKL98}
A. Ali, G. Kramer and C.-D. L\"u, hep-ph/9804363.
\bibitem{EW}
 E. Witten,
{ Nucl. Phys.} {\bf B 160} (1979) 57.
\bibitem{BGR}
 A.J. Buras, J.M. G{\'e}rard and R. R\"uckl,
{ Nucl. Phys.} {\bf B 268} (1986) 16.
\bibitem{rome2}
M. Ciuchini, E. Franco, G. Martinelli, L. Reina and L. Silvestrini, 
Z.Phys. {\bf C68} (1995) 239.
\bibitem{BUSI2}
A.J. Buras and L. Silvestrini, work in progress.
\bibitem{GALE}
{ M.K. Gaillard and B.W. Lee,} 
{ Phys. Rev.} {\bf D10} (1974) 897.
\bibitem{ARGUS}
{ H. Albrecht et al. (ARGUS)}, { Phys. Lett.} {\bf B192} (1987) 245;
{ M. Artuso et al. (CLEO)}, { Phys. Rev. Lett.} {\bf 62} (1989) 2233.
\bibitem{CRONIN}
{ J.H. Christenson, J.W. Cronin, V.L. Fitch and R. Turlay},
{ Phys. Rev. Lett.} {\bf 13} (1964) 128. 
\bibitem{RF97}
{ R. Fleischer}, { Int. J. of Mod. Phys.}
 {\bf A12} (1997) 2459.
\bibitem{CHAU83}
L.L. Chau, { Physics Reports}, {\bf 95} (1983) 1.
\bibitem{BSSII}
A.J. Buras, W. Slominski and H. Steger,
{ Nucl. Phys.} {\bf B245} (1984) 369.
\bibitem{GERAR}
J. Bijnens, J.-M. G{\'e}rard and G. Klein, 
{ Phys. Lett.} {\bf B257} (1991) 191.
\bibitem{BELU}
R. Belusevic, KEK preprint 97--264 (1998).
\bibitem{NIRSLAC}
Y. Nir, SLAC-PUB-5874 (1992).
\bibitem{GUPTA98}
R. Gupta, hep-ph/9801412.
\bibitem{JLQCD}
S. Aoki et al., JLQCD collaboration, 
{ Phys. Rev. Lett.} {\bf 80} (1998) 5271. 
\bibitem{GKS}
G. Kilcup, R. Gupta and S.R. Sharpe, 
{ Phys. Rev.} {\bf D57} (1998) 1654.
\bibitem{G67}
R. Gupta, T. Bhattacharaya, and S.R. Sharpe, 
{ Phys. Rev.} {\bf D55} (1997) 4036.
\bibitem{APE}
L. Conti, A. Donini, V. Gimenez, G.Martinelli, M. Talevi and
A. Vladikas, hep-lat/9711053.
\bibitem{BERT97}
S. Bertolini, J.O. Eeg, M. Fabbrichesi and E.I. Lashin,
{ Nucl. Phys.} {\bf B514} (1998) 63.
\bibitem{BBG0}
{W.A. Bardeen, A.J. Buras and J.-M. G\'erard,}
{ Phys. Lett.} {\bf B211} (1988) 343;
 {J-M. G\'erard,} { Acta Physica Polonica} {\bf B21} (1990) 257. 
\bibitem{Bijnens}
{ J. Bijnens and J. Prades,} { Nucl. Phys.} {\bf B444} (1995) 523. 
\bibitem{Prades}
{ A. Pich and E. de Rafael,} { Phys. Lett.} {\bf B158} (1985) 477;
{ J. Prades} {  et al,} { Z. Phys.}  {\bf C51} (1991) 287.
\bibitem{Donoghue}
{ J.F. Donoghue, E. Golowich and B.R. Holstein,}
{ Phys. Lett.} {\bf B119} (1982) 412.
\bibitem{Flynn}
{ J. Flynn}, in proceedings of the 28th International Conference
on High Energy Physics, July 1996, Warsaw, Poland, page 335; 
 J.M. Flynn and C.T. Sachrajda, hep-lat/9710057, 
to appear in \cite{HFII}.
\bibitem{Bernard}
C. Bernard, hep-ph/9709460.
\bibitem{QCDSF}
{ E. Bagan, P. Ball, V.M. Braun and H.G. Dosch},
{ Phys. Lett.} {\bf B278} (1992) 457;
{ M. Neubert}, { Phys. Rev.} {\bf D45} (1992) 2451 and references
therein.
\bibitem{Buras}
A.J. Buras, { Phys. Lett.} {\bf B317} (1993) 449.
\bibitem{ABWAR}
A.J. Buras, hep-ph/9610461.
\bibitem{Drell}
The LEP B Oscillations Working Group, LEPBOSC 97/002.3 (August 14, 1997).
\bibitem{NAR}
{S. Narison,}
{ Phys. Lett.} {\bf B322} (1994) 247.
\bibitem{FRENCH}
Y. Grossman, Y. Nir, S. Plaszczynski and M. Schune,
{ Nucl.~Phys.} {\bf B511} (1998) 69.
\bibitem{PAGA}
P. Paganini, F. Parodi, P. Roudeau and A. Stocchi, hep-ph/9711261;
F. Parodi, P. Roudeau and A. Stocchi, hep-ph/9802289.
\bibitem{BJL96b}
{ A.J. Buras, M.Jamin and M.E. Lautenbacher,} 1997, unpublished;
A.J. Buras, hep-ph/9711217.
\bibitem{ciuchini:95}
{ M.~Ciuchini}, { E.~Franco}, { G.~Martinelli}, {L.~Reina
 and   L.~Silvestrini},
 { Z. Phys.} {\bf C68} (1995) 239.
\bibitem{ALUT}
{ A. Ali and D. London,}
{ Z. Phys.} {\bf C65} (1995) 431; 
{ Nucl. Phys. B} (proc. Suppl.) {\bf 54A} (1997) 297;
A. Ali, hep-ph/9801270.
\bibitem{barr:93}
{ G.~D. Barr} { et~al.},
{ Phys. Lett.} {\bf B317} (1993) 233.
\bibitem{gibbons:93}
{ L.~K. Gibbons} { et~al.},
{ Phys. Rev. Lett.} {\bf 70} (1993) 1203.
\bibitem{wolfenstein:64}
{ L.~Wolfenstein},
 { Phys. Rev. Lett.} {\bf 13} (1964) 562.
\bibitem{HALL}
R. Barbieri, L. Hall, A. Stocchi, and  N. Weiner, hep-ph/9712252.
\bibitem{flynn:89}
{ J.~M. Flynn} and { L.~Randall},
{ Phys. Lett.} {\bf B224} (1989) 221; erratum ibid.\ { Phys.
  Lett.} {\bf B235} (1990) 412.
\bibitem{buchallaetal:90}
{ G.~Buchalla}, { A.~J. Buras}, and { M.~K. Harlander},
{ Nucl. Phys.} {\bf B337} (1990) 313.
\bibitem{GW79}
{ F.J. Gilman and M.B. Wise,} { Phys. Lett.} {\bf B83} (1979) 83;
{ B. Guberina and R.D. Peccei,} { Nucl. Phys.} {\bf B163} (1980) 289.
\bibitem{donoghueetal:86} 
{ J.F. Donoghue, E. Golowich, B.R. Holstein and J. Trampetic,}
{ Phys. Lett.} {\bf B179} (1986) 361. 
\bibitem{burasgerard:87}
{ A.~J. Buras} and { J.-M. G{\'e}rard},
{ Phys. Lett.} {\bf B192} (1987) 156.
\bibitem{lusignoli:89}
{ M. Lusignoli,} { Nucl. Phys.} {\bf B325} (1989) 33. 
\bibitem{ANII}
M. Ciuchini, E. Franco and R. Onforio,
{ Mod. Phys. Lett.} {\bf A5} (1990) 2173;
W.A. Bardeen,  A.~J. Buras and  J.-M. G{\'e}rard,
{ Phys. Lett.} {\bf B180} (1986) 133;
{ Nucl. Phys.} {\bf B293} (1987) 787;
H.-Y. Cheng, { Phys. Rev.} {\bf D37} (1988) 1908.
\bibitem{BW84}
{ J. Bijnens and M.B. Wise,} { Phys. Lett.} {\bf B137} (1984) 245.
\bibitem{bardeen:87}
{ W. A. Bardeen}, { A. J. Buras} and { J.-M. G{\'e}rard},
 { Phys. Lett.} {\bf B180} (1986) 133;
{ Nucl. Phys.} {\bf B293} (1987) 787;
{ Phys. Lett.} {\bf B192} (1987) 138.
\bibitem{PW91}
{ E.A. Paschos and Y.L. Wu,} { Mod. Phys. Lett.} {\bf A6} (1991) 93;
{ M. Lusignoli, L. Maiani, G. Martinelli and L. Reina,} 
{ Nucl. Phys.} {\bf B369} (1992) 139.
\bibitem{WW}
{ B. Winstein and L. Wolfenstein,} { Rev. Mod. Phys.} {\bf 65} (1993)
1113.
\bibitem{BERT98}
S. Bertolini, M. Fabbrichesi and J.O. Eeg, hep-ph/9802405.
\bibitem{DI12}
{ W. A. Bardeen}, { A. J. Buras} and { J.-M. G{\'e}rard},
{ Phys. Lett.} {\bf B192} (1987) 138;
{ A. Pich and E. de Rafael}, { Nucl. Phys.} {\bf B358} (1991) 311;
{ M. Neubert and B. Stech}, { Phys. Rev.} {\bf D 44} (1991) 775;
{ M. Jamin and A. Pich}, { Nucl. Phys.} {\bf B425} (1994) 15; 
{ J. Kambor, J. Missimer and D. Wyler},
{ Nucl. Phys.} {\bf B346} (1990) 17;
{ Phys. Lett.} {\bf B261} (1991) 496;
{ V. Antonelli, S. Bertolini, M. Fabrichesi, and E.I. Lashin},
{ Nucl. Phys.} {\bf B469} (1996) 181.
\bibitem{kilcup:91}
{ G.~W. Kilcup},
 { Nucl. Phys. (Proc. Suppl.)} {\bf B20} (1991) 417.
\bibitem{sharpe:91}
{ S.~R. Sharpe},
 { Nucl. Phys. (Proc. Suppl.)} {\bf B20} (1991) 429.
\bibitem{kilcup:98}
D. Pekurovsky and G. Kilcup, hep-lat/9709146.
\bibitem{heinrichetal:92}
{ J.~Heinrich}, { E.~A. Paschos}, { J.-M. Schwarz}, and { Y.~L.
  Wu},
{ Phys. Lett.} {\bf B279} (1992) 140.
\bibitem{paschos:96}
{ E.~A. Paschos},
 review presented at the 27th Lepton-Photon Symposium,
  Beijing, China (August 1995).
  \bibitem{DORT98}
T. Hambye, G.O. K\"ohler, E.A. Paschos, P.H. Soldan and W.A. Bardeen,
hep-ph/9802300.
\bibitem{NJL}
D. Espriu, E. de Rafael and J. Taron, { Nucl. Phys.} {\bf B345} (1990) 22;
J. Bijnens, Phys. Rept. {\bf 265} (1996) 369.
\bibitem{TR96}
S. Bertolini, J.O. Eeg and  M. Fabbrichesi,
{ Nucl. Phys.} {\bf B449} (1995) 197;
{ Nucl. Phys.} {\bf B476} (1996) 225.
\bibitem{TR97}
S. Bertolini, J.O. Eeg, M. Fabbrichesi and E.I. Lashin,
{ Nucl. Phys.} {\bf B514} (1998) 93.
\bibitem{BJL96a}
{ A.~J. Buras}, { M.~Jamin}, and { M.~E. Lautenbacher},
{ Phys. Lett.} {\bf B389} (1996) 749.
\bibitem{buraslauten:93}
{ A.~J. Buras} and { M.~E. Lautenbacher},
{ Phys. Lett.} {\bf B318} (1993) 212.
\bibitem{narison:95}
{ S.~Narison},
{ Phys. Lett.} {\bf B358} (1995) 113.
\bibitem{jaminmuenz:95}
{ M.~Jamin} and { M.~M{\"u}nz},
 { Z. Phys.} {\bf C66} (1995) 633.
\bibitem{chetyrkinetal:95}
K.~G. Chetyrkin, C.~A. Dominguez,  D.~Pirjol, and 
  K.~Schilcher,
{ Phys. Rev.} {\bf D51} (1995) 5090;
K.~G. Chetyrkin, D.~Pirjol, and 
  K.~Schilcher,
{ Phys. Lett.} {\bf B404} (1997) 337.
\bibitem{Paver}
P. Colangelo, F. De Fazio, G. Nardulli, and N. Paver,
{ Phys. Lett.} {\bf B408} (1997) 340.
\bibitem{Jamin97}
M. Jamin, Nucl. Phys. B. Proc. Suppl. {\bf 64} (1998) 250.
\bibitem{Yndurain}
F.J. Yndurain, hep-ph/9708300.
\bibitem{Dosch}
H.G. Dosch and S. Narison, { Phys. Lett.} {\bf B417} (1998) 173.
\bibitem{DERAF}
L. Lellouch, E. de Rafael, and J. Taron, 
{ Phys. Lett.} {\bf B414} (1997) 195.
\bibitem{ciuchini:96}
{ M. Ciuchini}, Nucl. Phys. B. Proc. Suppl. {\bf 59} (1997) 149.
\bibitem{BELKOV}
A.A. Belkov, G. Bohm, A.V. Lanyov, A.A. Moshkin, hep-ph/9704354.
\bibitem{Bert} 
{ S.~Bertolini, F.~Borzumati and A.~Masiero,} 
{ Phys. Rev. Lett.} {\bf 59} (1987) 180.
\bibitem{Desh} 
{ N.~G.~Deshpande, P.~Lo, J.~Trampetic, G.~Eilam and P. Singer}
{ Phys. Rev. Lett.} {\bf 59} (1987) 183.
\bibitem{Grin} 
{ B.~Grinstein, R.~Springer and M.B.~Wise,} 
{ Nucl.~Phys.} {\bf B339} (1990) 269.
\bibitem{Odon} 
{ R.~Grigjanis, P.J.~O'Donnell, M.~Sutherland and H.~Navelet,} 
{ Phys.~Lett.} {\bf B213} (1988) 355;
{ Phys.~Lett.} {\bf B286} (1992) 413 E.
\bibitem{CFMRS:93}
{ M. Ciuchini, E. Franco, G. Martinelli, L. Reina and L. Silvestrini,}
 { Phys.~Lett.} {\bf B316} (1993) 127.
\bibitem{CFRS:94}
{ M. Ciuchini, E. Franco, L. Reina and L. Silvestrini,}
{ Nucl.~Phys.} {\bf B421} (1994) 41.
\bibitem{CCRV:94a}
{ G.~Cella, G.~Curci, G.~Ricciardi and  A.~Vicer{\'e},}
{ Phys.~Lett.} {\bf B325} (1994) 227.
\bibitem{CCRV:94b}
{ G.~Cella, G.~Curci, G.~Ricciardi and A.~Vicer{\'e},}
{ Nucl.~Phys.} {\bf B431} (1994) 417.
\bibitem{AG1} 
{ A.~Ali, and  C.~Greub,} { Z.Phys.} {\bf C60} (1993) 433.  
\bibitem{BMMP:94}
{ A. J. Buras, M. Misiak, M. M{\"u}nz and S. Pokorski,}
{ Nucl.~Phys.} {\bf B424} (1994) 374.
\bibitem{BG98}
F.M. Borzumati and Ch. Greub, hep-ph/9802391.
\bibitem{CM78} 
{ N. Cabibbo and L. Maiani}, 
{ Phys.~Lett.} {\bf B79} (1978) 109.
\bibitem{KIMM}
{ C.S. Kim and A.D. Martin},
{ Phys.~Lett.} {\bf B225} (1989) 186.
\bibitem{N89} 
{ Y. Nir,}
{ Phys.~Lett.} {\bf B221} (1989) 184.
\bibitem{KN98}
A.L. Kagan and M. Neubert, hep-ph/9805303.
\bibitem{FLS96} 
{ A.F.~Falk, M.~Luke and M.~Savage,}
{ Phys. Rev.} {\bf D53} (1996) 2491.
\bibitem{LDGAMMA}
{ D. Atwood, B. Blok, and A. Soni}, 
{ Int. J. Mod. Phys.} {\bf A11} (1996) 3743;
{ H.-Y. Cheng,} { Phys. Rev.} {\bf D51} (1995) 6228;
{ E. Golowich and S. Pakvasa,} { Phys. Rev.} {\bf D51} (1995) 1215;
{ G. Ricciardi}, { Phys.~Lett.} {\bf B355} (1995) 313;
{ A. Khodjamirian, G. Stoll and D. Wyler},
{ Phys.~Lett.} {\bf B358} (1995) 129;
{ G. Eilam, A. Ioannissian and R.R. Mendel}, 
{ Z. Phys.} {\bf C71} (1996) 95;
{ G. Eilam, A. Ioannissian, R.R. Mendel and P. Singer},
 { Phys. Rev.} {\bf D53} (1996) 3629;
{ J.M. Soares,} { Phys. Rev.} {\bf D53} (1996) 241;
{ J. Milana,} { Phys. Rev.} {\bf D53} (1996) 1403;
{ N.G. Deshpande, X.-G. He and J. Trampetic,}
{ Phys.~Lett.} {\bf B367} (1996) 362.
\bibitem{VOL96}
{ M.B. Voloshin}, { Phys.~Lett.} {\bf B397} (1997) 275.
\bibitem{LRW97}
{ A. Khodjamirian, R. R\"uckl, G. Stoll and D. Wyler},
{ Phys.~Lett.} {\bf B402} (1997) 167;
{ Z. Ligeti, L. Randall and M.B. Wise}, 
{ Phys.~Lett.} {\bf B402} (1997) 178;
{ A.K. Grant, A.G. Morgan, S. Nussinov and R.D. Peccei},
 { Phys. Rev.} {\bf D56} (1997) 3151.
\bibitem{BUC97}
G. Buchalla, G. Isidori and S.-J. Rey, 
{ Nucl.~Phys.} {\bf B511} (1998) 594.
\bibitem{CZMA}
A. Czarnecki and W.J. Marciano, hep-ph/9804252.
\bibitem{STRUMIA}
A. Strumia, hep-ph/9804274. 
\bibitem{CLEO2} { M.S. Alam} { et. al} (CLEO), 
{ Phys. Rev. Lett.} {\bf 74} (1995) 2885.
\bibitem{CLEO98}
S. Glenn (CLEO), talk presented at the Meeting of the American Physics
Society, Columbus, Ohio, 18-21 March 1998.
\bibitem{ALEPH}
R. Barate et al., (ALEPH), CERN-EP/98-044.
\bibitem{chwil} 
W.S. Hou and R.S. Willey, { Phys.~Lett.} {\bf B202} (1988) 591; 
B. Grinstein, R. Springer, and M. Wise, 
{ Nucl.~Phys.} {\bf B339} (1990) 269. 
\bibitem{anl}
H. Anlauf, { Nucl.~Phys.} {\bf B430} (1994) 245.
\bibitem{multiH} 
P. Krawczyk and S. Pokorski, 
{ Nucl.~Phys.} {\bf B364} (1991) 10;
  Y. Grossmann, Y. Nir, R. Rattazzi, in \cite{HFII}.
\bibitem{rattazzi} 
See for instance G.F. Giudice, R. Rattazzi, hep-ph/9801271.
\bibitem{io} R. Barbieri and G.F. Giudice, 
{ Phys.~Lett.} {\bf B309} (1993) 86.
\bibitem{berto} S. Bertolini, F. Borzumati, A. Masiero, and G. Ridolfi, 
{ Nucl. Phys.} {\bf B353} (1991) 591;
N. Oshimo, { Nucl.~Phys.} {\bf B404} (1993) 20; 
R. Garisto, J.N. Ng, { Phys.~Lett.} {\bf B315} (1993) 372; 
M.A. Diaz, { Phys.~Lett.} {\bf B304} (1993) 278; 
Y. Okada, { Phys.~Lett.} {\bf B315} (1993) 119; 
F. Borzumati, { Z. Phys.} {\bf C63} (1994) 291; 
P. Nath and R. Arnowitt, { Phys.~Lett.} {\bf B336} (1994) 395; 
S. Bertolini and F. Vissani, { Z. Phys.} {\bf C67} (1995) 513; 
J. Lopez et al., { Phys. Rev.} {\bf D51} (1995) 147. 
\bibitem{CLEO96}
CLEO II, Contribution (PA05-093) to the 28th International Conference
on High Energy Physics, July 1996, Warsaw, Poland.
\bibitem{ALIB}
{ A. Ali}, hep-ph/9606324, hep-ph/9610333.
\bibitem{Photon}
{ I. Bigi et al.,} { Phys.~Rev.~Lett.} {\bf 71} (1993) 496;
{ Int. J. Mod. Phys.} {\bf A9} (1994) 2467;
{ G. Korchemsky and G. Sterman,} { Phys.~Lett.} {\bf B340} (1994) 96;
{ M. Neubert}, { Phys. Rev.} {\bf D49} (1994) 4623;
{ A. Ali and C. Greub}, { Phys.~Lett.} {\bf B361} (1995) 146;
{ A. Kapustin, Z. Ligeti and H.D. Politzer,}
{ Phys.~Lett.} {\bf B357} (1995) 653;
{ R.D. Dikeman, M. Shifman and R.G. Uraltsev},
{ Int. J. Mod. Phys.} {\bf A11} (1996) 571;
{ N. Pott,} { Phys. Rev.} {\bf D 54} (1996) 938.
\bibitem{GSW95}
{ C. Greub, H. Simma and D. Wyler},
{ Nucl.~Phys.} {\bf B434} (1995) 39; Erratum-ibid,
{\bf B444} (1995) 447.
\bibitem{CPRARE}
{ L. Littenberg and G. Valencia,}
{ Ann. Rev. Nucl. Part. Sci.} {\bf 43} (1993) 729;
{ J.L. Ritchie and S.G. Wojcicki,} { Rev. Mod. Phys.} {\bf 65} (1993)
1149; A. Pich, hep-ph/9610243; G. D'Ambrosio and G. Isidori,
hep-ph/9611284.
\bibitem{novikovetal:77}
{ V.A. Novikov, A.I. Vainshtein, V.I. Zakharov and M.A. Shifman,}
Phys. Rev. {\bf D16}, (1977) 223.
\bibitem{ellishagelin:83}
{ J. Ellis and J.S. Hagelin,} { Nucl.~Phys.} {\bf B217} (1983) 189.
\bibitem{dibetal:91}
{ C.O. Dib, I. Dunietz and F.J. Gilman,} { Mod. Phys. Lett.}
{\bf A6} (1991) 3573.
\bibitem{MP}
{ W. Marciano and Z. Parsa}, Phys. Rev. {\bf D53}, R1 (1996).
\bibitem{RS}
{ D. Rein and L.M. Sehgal,} { Phys. Rev.} {\bf D39} (1989) 3325;
{ J.S. Hagelin and L.S. Littenberg,} { Prog. Part. Nucl. Phys.}
{\bf 23} (1989) 1;
{ M. Lu and M.B. Wise,} { Phys. Lett.} {\bf B324} (1994) 461;
{ S. Fajfer}, [hep-ph/9602322]; { C.Q. Geng, I.J. Hsu and Y.C. Lin},
{ Phys. Rev.} {\bf D54} (1996) 877.
\bibitem{Adler95}
{ S. Adler} et al., { Phys. Rev. Lett.} {\bf 76} (1996) 1421.
\bibitem{AGS2}
{ L. Littenberg and J. Sandweiss}, eds., AGS2000, Experiments for the
21st Century, BNL 52512.
\bibitem{Cooper}
{ P. Cooper, M. Crisler, B. Tschirhart and J. Ritchie}
(CKM collaboration), 
EOI for measuring $Br(K^+\to\pi^+\nu\bar\nu)$ at the Main Injector,
Fermilab EOI 14, 1996.
\bibitem{littenberg:89}
{ L. Littenberg,} { Phys. Rev.} {\bf D39} (1989) 3322.
\bibitem{NIR96}
{ Y. Grossman, Y. Nir and R. Rattazzi}, [hep-ph/9701231] in \cite{HFII}.
\bibitem{BUCH96}
{ G. Buchalla}, hep-ph/9612307.
\bibitem{BB96}
{ G. Buchalla} and { A.J. Buras},
 { Phys. Rev.} {\bf D54} (1996) 6782.
\bibitem{AJB94}
{ A.J. Buras}, { Phys. Lett.} {\bf B333} (1994) 476.
\bibitem{XX97}
T. Nakaya, in proceedings of FCNC97, page 105.
\bibitem{Adler97}
S. Adler et al., { Phys. Rev. Lett.} {\bf 79}, (1997) 2204.
\bibitem{AGS2000}
L. Littenberg and J. Sandweiss, eds., AGS2000, Experiments for the 21st 
Century, BNL 52512.
\bibitem{FNALKL}
{ K. Arisaka et al.,} KAMI conceptual design report, FNAL, June 1991.
\bibitem{KEKKL}
{ T. Inagaki, T. Sato and T. Shinkawa,} Experiment to search for the
decay \klpnn\, at KEK 12 GeV proton synchrotron, 30 Nov. 1991.
\bibitem{KLBSM}
Y. Grossman and Y. Nir, { Phys. Lett.} {\bf B398} (1997) 163;
C.E. Carlson, G.D. Dorada and M. Sher,
{ Phys. Rev.} {\bf D54} (1996) 4393; G. Burdman, hep-ph/9705400;
A. Berera, T.W. Kephart and M. Sher, 
{ Phys. Rev.} {\bf D56} (1997) 7457;
Gi-Chol Cho, hep-ph/9804327.
\bibitem{HHW98}
T. Hattori, T. Hasuike and S. Wakaizumi, hep-ph/9804412.
\bibitem{BRS}
A.J. Buras, A. Romanino and L. Silvestrini, 
{ Nucl. Phys.} {\bf B520} (1998) 3.
\bibitem{GN1}
Y. Nir and M.P. Worah, hep-ph/9711215.
\bibitem{BB4}
{ G. Buchalla and A.J. Buras}, 
{ Phys. Lett.} {\bf B333} (1994) 221.
\bibitem{CJ}
{ C. Jarlskog,} { Phys. Rev. Lett.} {\bf 55}, (1985) 1039;
{ Z. Phys.} {\bf C29} (1985) 491.
\bibitem{Aleph96}
ALEPH Collaboration, Contribution (PA10-019) to 
the 28th International Conference
on High Energy Physics, July 1996, Warsaw, Poland.
\bibitem{B95}
{ A.J. Buras,} { Nucl. Instr. Meth.} {\bf A368} (1995) 1.
\bibitem{CDFMU}
F. Abe et al. (CDF), { Phys. Rev.} {\bf D57} (1998) R3811.
\bibitem{GLN96}
Y. Grossman, Z. Ligeti and E. Nardi,
{ Phys. Rev.} {\bf D55} (1997) 2768.
\bibitem{BB97}
{ G. Buchalla} and { A.J. Buras}, 
{ Phys. Rev.} {\bf D57} (1998) 216.
\bibitem{NQ}
{Y. Nir and H.R. Quinn}
{ Ann. Rev. Nucl. Part. Sci.} {\bf 42}
(1992) 211 and
 in " B Decays ", ed S. Stone
 (World Scientific, 1994),
p. 520; {I. Dunietz,} ibid p.550 and refs. therein.
\bibitem{CPASYM}
{ M. Gronau and D. London,} { Phys. Rev. Lett.}
 {\bf 65} (1990) 3381.
\bibitem{SNYD}
A. Snyder and H.R. Quinn, { Phys. Rev.} {\bf D48} (1993) 2139;
{ A.J. Buras and R. Fleischer,}
{ Phys. Lett.} {\bf B360} (1995) 138;
J.P. Silva and L. Wolfenstein, 
{ Phys. Rev.} {\bf D49} (1995) R1151; 
{ A.S. Dighe, M. Gronau and J. Rosner}, 
{ Phys. Rev.} {\bf D54} (1996) 3309; 
R. Fleischer and T. Mannel, 
{ Phys. Lett.} {\bf B397} (1997) 269;
C.S. Kim, D. London and T. Yoshikawa, 
{ Phys. Rev.} {\bf D57} (1998) 4010. 
\bibitem{BSANDA}
{ I.I.Y. Bigi and A.I. Sanda,}
{ Nucl. Phys.} {\bf B193} (1981) 85.
\bibitem{PHI}
D. London and A. Soni, { Phys. Lett.} {\bf B407} (1997) 61;
Y. Grossman and M.P. Worah, { Phys. Lett.} {\bf B395} (1997) 241;
M. Ciuchini et al., { Phys. Rev. Lett.} {\bf B79} (1997) 978;
R. Barbieri and and A. Strumia, { Nucl. Phys.} {\bf B508} (1997) 3.
\bibitem{adk}
{ R. Aleksan, I. Dunietz and B. Kayser,}
 { Z.Phys.} {\bf C54} (1992) 653;\\
R. Fleischer and I. Dunietz, { Phys. Lett.} {\bf B387} (1996) 361.
\bibitem{Wyler}
{ M. Gronau and D. Wyler,} { Phys. Lett.} {\bf B265} (1991) 172.
\bibitem{DUN2}
M. Gronau and D. London, { Phys. Lett.} {\bf B253} (1991) 483.
{ I. Dunietz}, { Phys. Lett.} {\bf B270} (1991) 75.
\bibitem{V97}
D. Atwood, I. Dunietz and A. Soni, 
{ Phys. Rev. Lett.} {\bf B78} (1997) 3257.
\bibitem{PAPIII}
R. Fleischer, { Phys.\ Lett.} {\bf B365} (1996) 399.
 \bibitem{fm2}
R. Fleischer and T. Mannel, {Phys.\ Rev.} {\bf D57}
(1998) 2752.
\bibitem{groro}
M. Gronau and J.L. Rosner, hep-ph/9711246, hep-ph/9712287.
\bibitem{wuegai}
F. W{\"u}rthwein and P. Gaidarev, hep-ph/9712531.
\bibitem{defan}
R. Fleischer, hep-ph/9802433.
\bibitem{cleo}
R. Godang  et al., hep-ex/9711010.
\bibitem{babar}
The BaBar Physics Book, preprint SLAC-R-504, in preparation.
\bibitem{bjorken}
J.D. Bjorken, { Nucl.\ Phys.}~{\bf B} (Proc.\ Suppl.)
{\bf 11} (1989) 325; SLAC-PUB-5389 (1990), published in the proceedings
of the SLAC Summer Institute 1990, p.\ 167.
\bibitem{bfm}
A.J. Buras, R. Fleischer and T. Mannel, hep-ph/9711262.
\bibitem{gewe}
J.-M. G{\'e}rard and J. Weyers, hep-ph/9711469; D. Del{\'e}pine,
J.-M. G{\'e}rard, J. Pestieau and J. Weyers, hep-ph/9802361;
J.-M. G{\'e}rard, J. Pestieau and J. Weyers, hep-ph/9803328.
\bibitem{neubert}
M. Neubert, hep-ph/9712224.
\bibitem{fknp}
A.F. Falk, A.L. Kagan, Y. Nir and A.A. Petrov, 
{Phys.\ Rev.} {\bf D57} (1998) 4290.
\bibitem{atso}
D. Atwood and A. Soni (1997), hep-ph/9712287, hep-ph/9712252.
\bibitem{FSI}
L. Wolfenstein, {Phys.\ Rev.} {\bf D52} (1995) 537; 
J. Donoghue, E. Golowich, A.~Petrov and J. Soares, 
{ Phys. Rev. Lett.} {\bf 77} (1996) 2178; 
B. Blok and I. Halperin, { Phys. Lett.} {\bf B385} (1996) 324; 
B. Blok, M. Gronau and J.L.~Rosner,
{ Phys.\ Rev.\ Lett.} {\bf 78} (1997) 3999.
\bibitem{BjSt}
B. Stech, { Phys. Lett.} {\bf B130} (1983) 189; 
J. Bjorken, hep-ph/9706524.
\bibitem{fm3}
R. Fleischer and T. Mannel, hep-ph/9706261.
\bibitem{pert-pens}
R. Fleischer, Z. Phys. {\bf C58} (1993) 483 and 
{\bf C62} (1994) 81; G. Kramer, W.F. Palmer and H. Simma,
 Z. Phys. {\bf C66} (1995) 429.
\bibitem{rf-FSI}
R. Fleischer, hep-ph/9804319.
\bibitem{bskk}
R. Fleischer, hep-ph/9710331.
\bibitem{LNO}
A. Lenz, U. Nierste and G. Ostermaier, hep-ph/9802202; 
U. Nierste, hep-ph/9805388.
\end{thebibliography}



