\subsection{Mathematical modeling in various domains}
\label{subsection:fm}

Mathematical reasoning is more than a skill for solving mathematical exercises and problems; it is also a tool for understanding and communicating about various contexts and situations. In this section, we assess {\DV}'s capacity to use mathematical ideas and techniques to address real-world problems, some of which are not strictly mathematical, but demand quantitative thinking. We show below how \DV successfully builds a plausible mathematical model for a complex system that requires extensive interdisciplinary knowledge, while ChatGPT fails to make meaningful progress.
%Specifically, we showcase \DV's abilities for modeling complex phenomenon with simple math as well as for answering seemingly unapprochable Fermi questions.

%\paragraph{Modeling} Mathematical contests in modeling also require participants to translate complex and ill-defined situations into tractable and meaningful mathematical models, to make reasonable assumptions and simplifications, to test and validate their models, to explore different scenarios and alternatives, to interpret and communicate their results, and to justify and critique their approaches. Participants need to have the skills to identify, structure, and decompose problems, to use appropriate strategies and heuristics, to apply 

\begin{figure}[H]
\begin{AIbox}{\DV vs ChatGPT}
\parbox[t]{\linewidth}{{\bf Prompt:}
\scriptsize\begin{alltt}
Please come up with mathematical model for profiling the physiological power rate over time of a professional StarCraft 2 player during a match. Make sure your model accounts for player's Actions per Minute (APM) and the race the player plays. Please keep your answer concise.
\end{alltt}
}
\parbox[t]{\linewidth}{{\bf\DV:}
\tiny\begin{alltt}
One possible mathematical model for profiling the physiological power rate over time of a professional StarCraft 2 player during a match is: P(t) = B + A * f(t) * g(r) where: - P(t) is the power rate in watts at time t\\
- B is the basal metabolic rate in watts, which depends on the player's age, weight, height, and gender\\
- A is a constant factor that represents the additional energy expenditure per unit of APM
- f(t) is a function that models the APM over time, which may depend on the match duration, the game phase, the player's strategy, and the opponent's actions\\
- g(r) is a function that models the race modifier, which may depend on the race the player plays, the race the opponent plays, and the map features Some possible assumptions and simplifications for this model are:\\
- The power rate is proportional to the APM, which is a common measure of the player's speed and intensity of actions\\
- The APM over time follows a general pattern of increasing in the early game, peaking in the mid game, and decreasing in the late game, but may vary depending on the specific match conditions\\
- The race modifier reflects the different characteristics and demands of each race, such as the number and type of units, the amount and frequency of micro and macro management, and the level of aggression or defense\\
- The basal metabolic rate and the constant factor A are estimated based on average values for professional StarCraft 2 players\\
- The model ignores other factors that may affect the power rate, such as the player's mood, stress, fatigue, hydration, nutrition, and external environment An example of how to estimate the parameters and functions for this model is: - B = 80 W, based on the average basal metabolic rate for a 25-year-old male with a weight of 70 kg and a height of 175 cm\\
- A = 0.1 W/APM, based on the assumption that each APM requires an additional 0.1 W of energy\\
- f(t) = 100 + 50 * sin(pi * t / T), based on the assumption that the APM follows a sinusoidal pattern with a minimum of 50 APM, a maximum of 150 APM, and a period of T minutes, which is the average match duration\\
- g(r) = 1 + 0.2 * (r - 1), based on the assumption that the race modifier increases by 0.2 for each race in the order of Protoss, Terren and Zerg, reflecting the increasing complexity and diversity of each race Using this example, the power rate for a Zerg player at 10 minutes into a 20-minute match against a Protoss player would be: P(10) = 80 + 0.1 * (100 + 50 * sin(pi * 10 / 20)) * (1 + 0.2 * (3 - 1)) = 101 W\\
\end{alltt}
}
\parbox[t]{\linewidth}{{\bf ChatGPT:}
\tiny\begin{alltt}
A possible mathematical model to profile the physiological power rate over time of a professional StarCraft 2 player during a match could be:

Power rate = f(APM, Race, Time)

Where f is a function that maps the player's Actions per Minute (APM), race, and time into the player's physiological power rate. APM is a measure of how fast and efficiently the player is making decisions and executing actions in the game. Race is the player's chosen species in the game, which may affect the player's APM and power rate. Time is the elapsed time in the match.

This function could be estimated using regression analysis with data from previous matches, physiological monitoring devices, and game logs. The resulting model could be used to predict the player's power rate during future matches and help players and coaches optimize their performance and training.
\end{alltt}
}
\end{AIbox}
 \caption{\DV builds a mathematical model for profiling power rate over time of a professional StarCraft 2 player during a match.}
\label{fig:my_label}
\end{figure}

\begin{comment}
We begin with a iconic question: 
\begin{tcolorbox}[colback=white!5!white,adjusted title=Prompt]
\begin{minted}[breaklines, breaksymbol=,  fontsize=\tiny]{tex}
What if a needle hit earth at a speed of 0.9 lightspeeds?
\end{minted}
\end{tcolorbox}
\begin{figure}[H]



\begin{tcolorbox}[colback=white!5!white, enhanced, attach boxed title to top center={yshift=-3mm}, title=ChatGPT]
\begin{minted}[breaklines, breaksymbol=,  fontsize=\tiny]{tex}
If a needle were to hit the Earth at a speed of 0.9 times the speed of light, it would be an extremely high-energy event. The kinetic energy of the needle would be converted into other forms of energy, such as heat and light, upon impact. The resulting explosion would be extremely powerful and would likely cause significant damage to the area around the impact site. It is not something that is likely to happen naturally, and it is not something that we should try to intentionally cause.
\end{minted}
\end{tcolorbox}
\end{figure}

\begin{figure}[H]
\begin{tcolorbox}[colback=white!5!white, enhanced, attach boxed title to top center={yshift=-3mm}, title={\DV}]
\begin{minted}[breaklines, breaksymbol=,  fontsize=\tiny]{tex}
Assuming the needle is made of steel, has a mass of 1 gram, a length of 5 cm, and a diameter of 0.5 mm, we can estimate its kinetic energy, momentum, and cross-sectional area. Using the relativistic formulas for kinetic energy and momentum, we get:
K = mc^2 (gamma - 1) = 0.001 kg * (3 * 10^8 m/s)^2 * (2.29 - 1) = 1.03 * 10^14 J
p = gamma * mv = 2.29 * 0.001 kg * 0.9 * 3 * 10^8 m/s = 6.21 * 10^5 kg m/s
A = pi * r^2 = pi * (0.00025 m)^2 = 1.96 * 10^-7 m^2
where gamma is the Lorentz factor, given by gamma = 1 / sqrt(1 - v^2 / c^2) = 2.29 for v = 0.9 c.
For comparison, the kinetic energy of the needle is equivalent to about 24.6 megatons of TNT, or about 1.5 times the yield of the Tsar Bomba, the most powerful nuclear weapon ever detonated. The momentum of the needle is equivalent to about 62,100 tons of mass moving at 10 m/s, or about 1.5 times the mass of the Titanic moving at a brisk walking pace. The cross-sectional area of the needle is about 0.00002\% of the cross-sectional area of a human hair.
Now, assuming the needle hits the earth at a right angle, and neglecting any effects of air resistance, gravity, or rotation, we can estimate the impact force, pressure, and duration. Using the impulse-momentum theorem, we get:
F * delta t = delta p = -p
where F is the average force, delta t is the impact duration, and delta p is the change in momentum. Assuming the needle comes to a complete stop after hitting the earth, we get:
F * delta t = -6.21 * 10^5 kg m/s
To estimate the impact duration, we can assume that the needle penetrates the earth by a distance equal to its length, 0.05 m, and that its speed decreases linearly from 0.9 c to zero during this process. This gives us:
delta t = 2 * d / v = 2 * 0.05 m / (0.9 * 3 * 10^8 m/s) = 3.7 * 10^-10 s
Plugging this into the previous equation, we get:
F = -6.21 * 10^5 kg m/s / (3.7 * 10^-10 s) = -1.68 * 10^15 N
This is the average force exerted by the needle on the earth, and vice versa, during the impact. To estimate the pressure, we can divide this force by the cross-sectional area of the needle, giving us:
P = F / A = -1.68 * 10^15 N / (1.96 * 10^-7 m^2) = -8.57 * 10^21 Pa
This is the average pressure experienced by the needle and the earth at the point of contact during the impact. For comparison, the pressure at the center of the earth is estimated to be about 3.6 * 10^11 Pa, and the pressure at the bottom of the Mariana Trench, the deepest point in the ocean, is about 1.1 * 10^8 Pa. The pressure exerted by the needle is about 24 billion times greater than the pressure at the center of the earth, and about 78 trillion times greater than the pressure at the bottom of the ocean.
\end{minted}
\end{tcolorbox}
\end{figure}

The two responses show a clear contrast in their level of mathematical reasoning. {\DV} demonstrates the ability to formulate a model and use appropriate tools to explore the implications of moving at 0.9 lightspeed, without being prompted to do so. ChatGPT, on the other hand, gives a generic answer that lacks any mathematical support or explanation for why 0.9 lightspeed is very dangerous.

We test {\DV} and ChatGPT with another prompt:


\begin{tcolorbox}[colback=white!5!white,adjusted title=Prompt]
\begin{minted}[breaklines, breaksymbol=,  fontsize=\tiny]{tex}
How fast will a car hit an egg to break it?
\end{minted}
\end{tcolorbox}

\begin{tcolorbox}[colback=white!5!white, enhanced, attach boxed title to top center={yshift=-3mm}, title=ChatGPT]
\begin{minted}[breaklines, breaksymbol=,  fontsize=\tiny]{tex}
It's difficult to give a precise answer to this question because the speed at which a car would need to be traveling in order to break an egg would depend on various factors, including the size and shape of the car, the type of surface the car is traveling on, and the type of egg. However, it is generally safe to say that a car would need to be traveling at a relatively high speed in order to break an egg.
\end{minted}
\end{tcolorbox}

ChatGPT responds to the question with a vague and intuitive answer, saying that the car needs to be traveling at a high speed. It does not show any evidence of mathematical thinking or problem-solving skills. It seems to rely on common sense or heuristics, rather than logic or calculations. ChatGPT does not treat the question as a math problem but as a casual conversation topic. Even if we prompt ChatGPT to build a mathematical model, it still only gives a template answer without actually solving the problem.
{\DV}, on the other hand, demonstrates a deeper and more systematic approach to the question. It sets up a physical model of the situation, and uses mathematical thinking to analyze the forces and velocities involved in the collision between the car and the egg. It tries to find the exact answer or the range of possible answers, rather than giving a general or vague answer. {\DV} shows that it can think about the question as a math problem, although it is not defined in a standard mathematical formulation.
\begin{figure}[H]
\begin{tcolorbox}[colback=white!5!white, enhanced, attach boxed title to top center={yshift=-3mm}, title=ChatGPT(Prompt to build math model)]
\begin{minted}[breaklines, breaksymbol=,  fontsize=\tiny]{tex}
To do this, we can use the formula:
Force = Mass * Acceleration
Where:
Force is the amount of force needed to break the egg
Mass is the weight of the car
Acceleration is the speed at which the car is traveling
We can rearrange this formula to solve for acceleration:
Acceleration = Force / Mass
To determine the force needed to break the egg, we can use the following formula:
Force = Pressure * Area
Where:
Pressure is the force applied per unit area
Area is the surface area of the eggshell being impacted
Substituting this formula into the first equation, we get:
Acceleration = (Pressure * Area) / Mass
To solve for the acceleration, we would need to determine the pressure and area of the eggshell being impacted, as well as the mass of the car. With this information, we can calculate the approximate speed at which the car would need to be traveling in order to break the egg.
\end{minted}
\end{tcolorbox}
\end{figure}
\begin{figure}[H]
\begin{tcolorbox}[colback=white!5!white, enhanced, attach boxed title to top center={yshift=-3mm}, title={\DV}]
\begin{minted}[breaklines, breaksymbol=,  fontsize=\tiny]{latex}
Impulse = Force x Time = Change in Momentum
If we assume that the car is much heavier than the egg and that its speed does not change significantly during the collision, then the change in momentum of the car is approximately equal and opposite to the change in momentum of the egg. Therefore, we can write:
Impulse = -m_e(v_f - v_i)
where m_e is the mass of the egg, v_f is the final speed of the egg after the collision, and v_i is the initial speed of the egg before the collision. We can also assume that the egg is initially at rest, so v_i = 0.
To find the force and time involved in the collision, we need to know the breaking strength of the egg and the deformation distance of the car and the egg. The breaking strength of an egg is the amount of force per unit area required to crack its shell, and it varies depending on the breed, age, and diet of the chicken, as well as the shape and thickness of the shell. According to some sources, the average breaking strength of a chicken egg is about 40 N/cm^2, or 4 kPa. The deformation distance of the car and the egg is the amount of compression or indentation that occurs when they collide, and it depends on the elasticity and rigidity of the materials involved. For simplicity, we can assume that the car and the egg are both spherical and have the same deformation distance, which we will call d.
Using the formula for the area of a sphere, we can find the force exerted by the car on the egg as: Force = Breaking Strength x Area = 4 kPa x 4 pi r^2
where r is the radius of the egg. Assuming that the egg has a radius of about 2.5 cm, we get: Force = 4 kPa x 4 pi (0.025 m)^2. Force = 7.85 N
Using the formula for the average speed of an object undergoing constant acceleration, we can find the time of the collision as: Time = 2d / (v_f + v_i)
Since we assumed that the egg is initially at rest, we can simplify this as: Time = 2d / v_f
To find the deformation distance, we can use the conservation of energy principle, which states that the kinetic energy of the egg before the collision is equal to the work done by the force during the collision. The work done by the force is equal to the force times the displacement, so we can write: 0.5m_ev_f^2 = Force x d. Solving for d, we get: d = 0.5m_ev_f^2 / Force
Substituting the values we found for the mass, force, and final speed of the egg, we get: d = 0.5(0.06 kg)v_f^2 / 7.85 N. d = 0.0019v_f^2
Plugging this into the equation for the time of the collision, we get: Time = 2(0.0019v_f^2) / v_f. Time = 0.0038v_f
Finally, substituting the expressions for the force and time into the equation for the impulse, we get: 7.85 N x 0.0038v_f = -0.06 kg(v_f - 0). Solving for v_f, we get: v_f = -7.85 N x 0.0038v_f / 0.06 kg. v_f = -0.5 m/s
This means that the egg will rebound from the car with a speed of 0.5 m/s in the opposite direction of the car's motion. However, this is the speed of the egg after the collision, not the speed of the car before the collision. To find the speed of the car, we need to consider the relative speed of the car and the egg, which is the difference between their speeds. Since the car and the egg are moving in opposite directions, we can write:
Relative Speed = v_c - v_f. where v_c is the speed of the car before the collision. Rearranging this, we get: v_c = Relative Speed + v_f
To find the relative speed, we need to know the minimum speed at which the car and the egg will collide, which depends on the distance between them and the reaction time of the driver. For example, if the car is 10 m away from the egg and the driver has a reaction time of 0.5 s, then the car needs to travel at least 20 m/s (or 72 km/h) to hit the egg before the driver can brake. However, this is the minimum speed, not the actual speed, so the car could be going faster than this. Therefore, the relative speed is at least 20 m/s, but it could be higher.
Using this value, we can find the speed of the car as:
v_c = 20 m/s + (-0.5 m/s). v_c = 19.5 m/s
This means that the car needs to hit the egg with a speed of at least 19.5 m/s (or 70.2 km/h) to break it, assuming the idealized conditions we used in our calculations. However, this is only an estimate, and the actual speed could vary depending on the factors we mentioned earlier. Therefore, the best way to find out the exact speed needed is to conduct an experiment with different car and egg models and measure the results.
\end{minted}
\end{tcolorbox}
\end{figure}
\end{comment}
%The population of a town grows from 1000 to 10000 in 100 years. No one came or left %the town. How many people are there with an age between 40 and 50?
\begin{comment}
We start with the following question: \vspace{-3mm}
\begin{figure}[H]
\begin{AIbox}{\DV \ vs ChatGPT}
{\bf Prompt:} \\
{\scriptsize
How fast does an iron ball whose radius is 1 meter need to move, in order to get to a distance of 1km from the sun before melting?
}\\

{\bf \DV:} \\
{\scriptsize
There are a few different ways to approach this question, but one possible method is to use the following assumptions and simplifications:\\
- The iron ball is spherical, homogeneous, and has a density of $7.8 \mathrm{g}/\mathrm{cm}^3$ and a melting point of 1538 °C.\\
- The iron ball starts from the Earth's orbit, which has a mean distance of 149.6 million km from the sun and a mean orbital speed of 29.8 km/s.\\
- The iron ball follows a straight-line trajectory towards the sun, ignoring the effects of gravity, drag, radiation, and other celestial bodies.\\
- The iron ball heats up due to the solar radiation, which has an average intensity of $1361 \mathrm{W}/\mathrm{m}^2$ at the Earth's orbit and follows an inverse square law with distance.\\
- The iron ball has a specific heat capacity of 0.45 J/g°C and a latent heat of fusion of 247 J/g.\\
- The iron ball does not lose heat by conduction, convection, or radiation, and its temperature is uniform throughout.\\
Using these assumptions, we can calculate the time it takes for the iron ball to reach the sun and the temperature it reaches at different distances. We can also compare the temperature with the melting point and find the distance at which the iron ball melts.\\
The time it takes for the iron ball to reach the sun is given by: $t = d / v$ where d is the distance from the sun and $v$ is the speed of the iron ball. Assuming the iron ball starts from the Earth's orbit, we have: $t = 149.6 \mathrm{\ million\ km} / v$. The temperature of the iron ball at a given distance is given by: $T = T_0 + Q / mc$ where $T_0$ is the initial temperature, $Q$ is the heat absorbed by the iron ball, $m$ is the mass of the iron ball, and $c$ is the specific heat capacity of iron. Assuming the iron ball starts at 15 °C, we have:
$$T = 15^\circ C + Q / (4/3 * \pi \times 1^3 \times 7.8 \times 10^6 \times 0.45)$$
The heat absorbed by the iron ball at time $t$ is given by: $Q(t) = P(t) A$ where $P(t)$ is the power of the solar radiation at time $t$ and $A$ is the cross-sectional area of the iron ball. Assuming the iron ball has a radius of 1 m, we have $A = \pi * 1^2$.\\
The power of the solar radiation at time t is given by:
$P(t) = I(t) * A * (1 - a)$
where $I(t)$ is the intensity of the solar radiation at time $t$ and $a$ is the albedo of the iron ball, which is the fraction of the radiation that is reflected. Assuming the iron ball has an albedo of 0.1, we have:
$P(t) = I(t) * \pi * 1^2 * 0.9$.
The intensity of the solar radiation at time t is given by:
$$I(t) = I_0 * (r_0 / r(t))^2$$
where $I_0$ is the intensity of the solar radiation at the Earth's orbit, $r_0$ is the distance of the Earth's orbit from the sun, and $r(t)$ is the distance of the iron ball from the sun at time t. Assuming $I_0 = 1361  \mathrm{W}/\mathrm{m}^2$ and $r_0 = 149.6 \mathrm{\ million\ km}$, we have:
$$I(t) = 1361 \mathrm{W}/\mathrm{m}^2 * (149.6 \mathrm{\ million\ km} / (149.6 \mathrm{\ million\ km} - v * t))^2$$
Taking integral over $0$ to $t$, we can find the total heat absorbed by the iron ball at time $t$:
%$$Q = \int_0^t Q(t) dt.$$
%$$Q = \int_0^t P(t) * A dt.$$
%$$Q = \int_0^t I(t) * \pi * 1^2 * 0.9 dt.$$
%$$Q = 0.9 * \pi * I_0 * r_0^2 * \int_0^t (1 / (r_0 - v * t)^2) dt.$$
%$$Q = 0.9 * \pi * I_0 * r_0^2 * (v / (r_0 - v * t) - v / r_0). $$
%$$Q = 0.9 * \pi * I_0 * r0 * v * (1 - r_0 / (r_0 - v * t)).$$
%Substituting the values of $I_0$, $r_0$, and $v$, we get:
% Q = 0.9 * \pi * 1361 W/m^2 * 149.6 million km * v * (1 - 149.6 million km / (149.6 million km - v * t))
% Now we can plug in the values of Q, m, and c into the equation for T and get:
% T = 15 °C + 0.9 * pi * 1361 W/m^2 * 149.6 million km * v * (1 - 149.6 million km / (149.6 million km - v * t)) / (4/3 * pi * 1^3 * 7.8 * 10^6 * 0.45)
% To find the time and speed required to reach this distance, we can use the equation for t and solve for v: t = (r0 - d) / v, v = (r0 - d) / t
% Using d = 1 km, we get:v = (149.6 million km - 1 km) / t
% To find the value of t that corresponds to the melting point of iron, we can set T = 1538 °C and solve for t:
% 1538 °C = 15 °C + 0.9 * pi * 1361 W/m^2 * 149.6 million km * v * (1 - 149.6 million km / (149.6 million km - v * t)) / (4/3 * pi * 1^3 * 7.8 * 10^6 * 0.45)
...
}\\
{\bf ChatGPT:} \\
{\scriptsize
We can start by using the equation for kinetic energy to determine the speed of the iron ball. Kinetic energy is given by the equation: $\mathrm{KE} = 1/2 * m * v^2$ where $\mathrm{KE}$ is the kinetic energy, $m$ is the mass of the iron ball, and $v$ is its velocity. Since we want the iron ball to be able to travel 1 km closer to the Sun without melting, we can assume that its kinetic energy will be equal to the amount of heat energy required to raise the temperature of the iron ball by its melting point.
Next, we will use the formula for heat energy which is $Q=mc\Delta T$ where Q is heat energy, m is the mass, c is the specific heat of the substance and $\Delta T$ is the temperature change.
We can then equate the two equations to find the velocity required.
$1/2 * m * v^2 = mc \Delta T$
Solving for $v$, $v = \sqrt{2mc\Delta T} / m$
...
}
\end{AIbox}
%\caption{}
\end{figure}
\end{comment}
\begin{comment}
\begin{figure}[H]
\begin{AIbox}{\DV vs ChatGPT}
{\bf Prompt:\\}
\parbox[t]{\linewidth}{\scriptsize

Explain how taxation on using robots will affect income inequality. Illustrate using a simple mathematical model with some realistic data assumptions. Please keep the answer succint.
{\bf \DV:}\\
\parbox[t]{\linewidth}{\scriptsize
One possible way to approach the question is to use a simple mathematical model based on the following assumptions:\\

- There are two types of workers in the economy: skilled and unskilled.\\
- Skilled workers can use robots to augment their productivity, while unskilled workers cannot.\\
- The demand for skilled workers is elastic, meaning that it responds to changes in their wage rate, while the demand for unskilled workers is inelastic, meaning that it does not.\\
- The supply of both types of workers is fixed and independent of their wage rate.\\
- The government imposes a tax on the use of robots by skilled workers, and redistributes the revenue to unskilled workers as a lump-sum transfer.\\

Based on these assumptions, we can represent the labor market equilibrium for each type of worker as follows:\\

- For skilled workers, the equilibrium wage rate ($W_s$) is determined by the intersection of the labor demand curve ($D_s$) and the labor supply curve ($S_s$). The labor demand curve is downward sloping, reflecting the elasticity of demand for skilled workers. The labor supply curve is vertical, reflecting the fixed supply of skilled workers. The use of robots shifts the labor demand curve to the right, from $D_s$ to $D_s'$, as skilled workers become more productive and valuable to employers. This raises the equilibrium wage rate from $W_s$ to $W_s'$. However, the tax on robots reduces the net wage rate that skilled workers receive, from $W_s'$ to $W_s'-T$, where $T$ is the tax per robot. This creates a wedge between the gross wage rate paid by employers ($W_s'$) and the net wage rate received by workers ($W_s'-T$). The tax revenue collected by the government is equal to the area of the rectangle $T \times Q_s'$, where $Q_s'$ is the equilibrium quantity of skilled labor.\\
- For unskilled workers, the equilibrium wage rate ($W_u$) is determined by the intersection of the labor demand curve ($D_u$) and the labor supply curve ($S_u$). The labor demand curve is horizontal, reflecting the inelasticity of demand for unskilled workers. The labor supply curve is vertical, reflecting the fixed supply of unskilled workers. The use of robots does not affect the labor demand or supply curves for unskilled workers, so the equilibrium wage rate remains at $W_u$. However, the lump-sum transfer from the government increases the income of unskilled workers by $\frac{T\times Q_s'}{Q_u}$, where $Q_u$ is the equilibrium quantity of unskilled labor. This shifts the income effect curve for unskilled workers to the right, from $I_u$ to $I_u'$, as unskilled workers can afford more consumption at any given wage rate. This does not affect the equilibrium quantity of unskilled labor, as the supply curve is vertical, but it does increase the utility of unskilled workers.\\

%The effect of taxation on using robots on income inequality can be measured by comparing the change in the income share of each type of worker. The income share of skilled workers is equal to the net wage rate multiplied by the quantity of skilled labor, divided by the total income in the economy. The income share of unskilled workers is equal to the wage rate plus the transfer, multiplied by the quantity of unskilled labor, divided by the total income in the economy. The total income in the economy is equal to the sum of the income of both types of workers, plus the tax revenue.\\

Using some realistic data assumptions, we can illustrate the effect of taxation on using robots on income inequality with a numerical example. Suppose that the following values hold in the economy:

- The initial wage rate of skilled workers is $W_s = 100$, and the initial wage rate of unskilled workers is $W_u = 50$.
- The supply of skilled workers is $Q_s = 100$, and the supply of unskilled workers is $Q_u = 200$.
- The elasticity of demand for skilled workers is $E_s = -1$, and the elasticity of demand for unskilled workers is $E_u = 0$.
- The use of robots increases the productivity of skilled workers by 50\%, and the tax rate on robots is 20\%.

Based on these values, we can calculate the following outcomes:

- The use of robots shifts the labor demand curve for skilled workers to the right by 50\%, from $D_s$ to $D_s'$. This raises the equilibrium wage rate of skilled workers from $W_s = 100$ to $W_s' = 150$. However, the tax on robots reduces the net wage rate of skilled workers from $W_s' = 150$ to $W_s'-T = 120$, where $T = 30$ is the tax per robot. The tax revenue collected by the government is equal to $T\times Q_s' = 30\times 100 = 3000$.
- The use of robots does not affect the labor demand or supply curves for unskilled workers, so the equilibrium wage rate of unskilled workers remains at $W_u = 50$. However, the lump-sum transfer from the government increases the income of unskilled workers by $T\times Q_s'/Q_u = \frac{30\times 100}{200} = 15$. This shifts the income effect curve for unskilled workers to the right, from $I_u$ to $I_u'$, as unskilled workers can afford more consumption at any given wage rate.\\
- The income share of skilled workers is equal to $$\frac{(W_s'-T)\times Q_s'}{(W_s'\times Q_s' + W_u\times Q_u + T\times Q_s')} = \frac{(120\times 100)}{(150\times 100 + 50\times 200 + 30\times 100)} = 0.429$$
-The income share of unskilled workers is equal to $$\frac{(W_u + T\times Q_s'/Q_u)\times Q_u}{(W_s'\times Q_s' + W_u\times Q_u + T\times Q_s')} = \frac{(50 + 15)\times 200}{(150\times 100 + 50\times 200 + 30\times 100)} = 0.536$$.

Comparing the income shares before and after the use of robots and the tax, we can see that the income share of skilled workers has decreased from 0.5 to 0.4, while the income share of unskilled workers has increased from 0.3333 to 0.4333. This implies that the taxation on using robots has reduced income inequality in the economy.}
{\bf ChatGPT:}\\
\parbox[t]{\linewidth}{\tiny
\begin{alltt}
Taxation on using robots can affect income inequality by redistributing wealth from companies who use robots to the government, thereby reducing their profits. This can result in a decrease in income for wealthy shareholders, while lower-income workers may benefit from government spending on social programs.

A simple mathematical model to illustrate this could look like this:

Let's assume a company's pre-tax profits are \$100 million and they use robots, leading to a tax bill of \$10 million. The company's post-tax profits would now be \$90 million. If the government were to use this additional revenue to fund social programs, it could result in an increase in the income of lower-income workers.

This simple model assumes that the tax revenue is used effectively to reduce income inequality, which may not always be the case in reality. However, it illustrates the potential impact that taxation on using robots can have on income inequality.
\end{alltt}}
\end{AIbox}
\caption{\DV explain economic and societal impact of AI with a mathematical model.}
\label{fig:my_label}
\end{figure}
\end{comment}

\paragraph{Fermi questions.} A specific type of mathematical modeling problems are Fermi questions. Fermi questions involve making educated guesses about quantities or phenomena that are difficult or impossible to measure directly, using logic, approximation, and order-of-magnitude reasoning. For example, some well-known Fermi questions are: ``How many piano tuners are there in Chicago?" and ``How many times does the electric field of the sun’s peak color oscillate during its journey to earth?". To solve these problems with little or no additional information, it requires both quantitative thinking and general knowledge. Here we demonstrate \DV's outstanding ability to answer Fermi questions with quantitative thinking. We show two examples in Figure~\ref{fig:fermi}, where \DV attempts at them by making layers of reasonable assumptions and informed guesses, while ChatGPT admits defeat immediately barely showing trace of mathematical thinking. Even though \DV's final answers are likely to be far off, it demonstrates an impressive angle of attack at these problems.

\begin{figure}[H]
\centering
\begin{AIbox}{\DV vs ChatGPT}
\parbox[t]{\linewidth}{{\bf Prompt:}
\scriptsize
\begin{alltt}
Please estimate roughly how many Fermi questions are being asked everyday?
\end{alltt}}
\parbox[t]{\linewidth}{{\bf\DV:}
\tiny\begin{alltt}
There is no definitive answer to how many Fermi questions are being asked everyday, but one possible way to approach the problem is to make some assumptions and use some estimates based on available data. Here is one possible example:\\
- Assume that a Fermi question is a question that requires an order-of-magnitude estimation of some quantity or phenomenon that is not easily measured or known, such as the number of piano tuners in Chicago, the mass of the Eiffel Tower, or the amount of water in Lake Michigan.\\
- Assume that a Fermi question is more likely to be asked by people who are interested in science, math, engineering, or logic, or who are engaged in some form of education, research, or problem-solving that involves such topics.\\
- Assume that the population of the world is about 7.8 billion, and that about 25\% of them are in the age range of 15 to 64, which is roughly the working-age population. This gives about 1.95 billion potential Fermi question askers.\\
- Assume that the average literacy rate among the working-age population is about 86\%, and that literacy is a prerequisite for asking or answering a Fermi question. This gives about 1.68 billion literate potential Fermi question askers.\\
- Assume that the average interest rate in science, math, engineering, or logic among the literate potential Fermi question askers is about 10\%, and that interest is a proxy for the likelihood of asking or answering a Fermi question. This gives about 168 million interested potential Fermi question askers.\\
- Assume that the average frequency of asking or answering a Fermi question among the interested potential Fermi question askers is about once a month, and that this frequency is independent of the day of the week, the time of the day, or the season of the year. This gives about 5.6 million Fermi questions asked or answered per day.\\
Therefore, based on these assumptions and estimates, a rough answer to how many Fermi questions are being asked everyday is about 5.6 million. However, this answer is subject to a lot of uncertainty and variation, depending on the validity and accuracy of the assumptions and estimates, and the definition and scope of what constitutes a Fermi question.
\end{alltt}}
\parbox[t]{\linewidth}{{\bf ChatGPT:}
\tiny\begin{alltt}
The number of Fermi questions asked daily is difficult to estimate accurately. However, based on the estimated number of daily active users on websites where such questions are commonly asked (e.g. forums, Q\&A sites, etc.), one could make a rough estimate of the number of Fermi questions being asked daily. For example, if a website has 1 million active users and each user asks 1 Fermi question per day, then the number of Fermi questions being asked daily on that website would be 1 million. However, this is a rough estimate and the actual number could be higher or lower.
\end{alltt}}
\tcbline
\parbox[t]{\linewidth}{{\bf Prompt:}
\scriptsize\begin{alltt}
Please provide a rough estimate for how many Nvidia A100 GPUs are there in total on earth.
\end{alltt}}
\parbox[t]{\linewidth}{{\bf\DV:}
\tiny\begin{alltt}
There is no definitive answer to how many Nvidia A100s are there in total on earth, but one possible way to approach the question is to use some available data and assumptions to make a rough estimate. Here are some possible steps and sources:
- According to Nvidia's website, the A100 is available in various form factors, such as PCIe cards, SXM4 modules, HGX A100 systems, and cloud instances. Each form factor may have different numbers of A100 GPUs per unit, ranging from one to 16. For simplicity, we can assume an average of four A100 GPUs per unit across all form factors.\\
- According to Nvidia's Q4 2021 earnings call transcript, the company reported a record revenue of \$2.91 billion from its data center segment, which includes the A100 and other products. Assuming that the A100 accounts for a significant portion of this segment, we can estimate that the A100 revenue was about \$2 billion in Q4 2021.\\
- According to Nvidia's website, the list price of the A100 PCIe card is \$11,000, and the list price of the HGX A100 4-GPU system is \$69,000. Assuming that these prices are representative of the average selling price of the A100 across all form factors, we can estimate that the average selling price of a unit with four A100 GPUs is about \$20,000.
- Dividing the estimated A100 revenue by the estimated average selling price, we can estimate that Nvidia sold about 100,000 units with four A100 GPUs each in Q4 2021, or about 400,000 A100 GPUs in total.\\
- Assuming that Q4 2021 was a typical quarter for A100 sales, and that the A100 was launched in mid-2020, we can estimate that Nvidia sold about 800,000 units with four A100 GPUs each in 2020 and 2021, or about 3.2 million A100 GPUs in total.\\
- Adding a margin of error to account for possible variations in prices, sales, form factors, and launch dates, we can estimate that there are between 2.5 million and 4 million A100 GPUs on earth as of early 2022.\\

Of course, this estimate is based on many assumptions and uncertainties, and should not be taken as a definitive or authoritative answer. It is possible that Nvidia has sold more or less A100 GPUs than estimated, or that some A100 GPUs have been decommissioned, damaged, or lost. Therefore, the actual number of A100 GPUs on earth may be higher or lower than the estimate.
\end{alltt}}
\parbox[t]{\linewidth}{{\bf ChatGPT:}\tiny
\begin{alltt}
It is not possible to determine the exact number of Nvidia A100 GPUs on Earth, as the company does not publicly disclose sales figures. However, as of 2021, it is estimated that there are hundreds of thousands of Nvidia A100 GPUs in use globally.
\end{alltt}}
\end{AIbox}
\caption{\DV answers Fermi questions.}
\label{fig:fermi}
\end{figure}

 %From \DV's answer, we can clearly see that the model tries to apply quantitative reasoning. it tries to apply the formula to write the temperature increase in terms of the amount of heat the ball absorbed, and relates the heat with the distance to the sun. \DV \ then correctly takes an integral over time to find the total temperature change. 
%It tries to fill in the missing information 
%by assuming simple, yet reasonable exponential models of the growth of the population and the survival rate. 
%Although the reasoning of \DV \ is not perfect due to calculation errors, we still view it as a big leap compared to the previous generation of models.  
\begin{comment}
one possible method is to use a simple model of exponential growth and assume that the birth and death rates are constant over time. This is not very realistic, but it might give a rough estimate.
One way to write an exponential growth model is: P(t) = P(0) * e^(rt)
where P(t) is the population at time t, P(0) is the initial population, e is the base of the natural logarithm, and r is the growth rate. We can use the given data to find r by plugging in P(0) = 1000, P(100) = 10000, and t = 100: 10000 = 1000 * e^(r*100).
Solving for r, we get: r = ln(10) / 100
r is approximately 0.0231.
Now, to find the number of people with an age between 40 and 50, we need to estimate the proportion of the population that falls in that age group. One way to do this is to assume that the population has a stable age distribution, meaning that the proportion of people in each age group is constant over time. This is also not very realistic, but it might give a rough estimate.
To find the stable age distribution, we need to know the survival function, which is the probability of surviving from birth to a given age. We can use the exponential growth model to find the survival function by assuming that the death rate is equal to the growth rate. This means that the population is in equilibrium, and that the number of births and deaths are equal. This is also not very realistic, but it might give a rough estimate.
The survival function is:
S(x) = e^(-rx)
where x is the age in years. This means that the probability of surviving from birth to age x is equal to e^(-rx).
To find the proportion of people in the age group 40 to 50, we need to find the probability of surviving from birth to age 40, and subtract the probability of surviving from birth to age 50. This is:
P(40 < x < 50) = S(40) - S(50). P(40 < x < 50) = e^(-r*40) - e^(-r*50)
Plugging in r = 0.0231, we get: P(40 < x < 50) = e^(-0.9239) - e^(-1.1549), P(40 < x < 50) = 0.3967 - 0.3156, P(40 < x < 50) = 0.0811
This means that about 8.11% of the population is in the age group 40 to 50. To find the number of people, we multiply this proportion by the total population at time 100, which is 10000:
N(40 < x < 50) = P(40 < x < 50) * P(100). N(40 < x < 50) = 0.0811 * 10000. N(40 < x < 50) = 811
Therefore, there are about 811 people with an age between 40 and 50 in the town.
\end{comment}
%On the other hand, ChatGPT fails to comprehend the question and makes completely nonsense reasoning based on some straightforward pattern matching.
\begin{comment} %Both answers are wrong
We then test the model with another example that, to the best of our knowledge, is not available on the internet:
\begin{figure}[H]
\begin{AIbox}{\DV \ vs ChatGPT}
{\bf Prompt:} \\
{\scriptsize
My family is from New York. I meet a random person in a mall in Los Angeles. What is the probability that we are fifth cousins and 
don't know it?
}\\

{\bf \DV:} \\
{\scriptsize
One possible set of assumptions and estimates is:
- A fifth cousin is someone who shares a pair of fourth-great-grandparents with us, or someone who is four generations removed from a common ancestor in the same generation as us.
- We each have 64 fourth-great-grandparents, assuming no inbreeding or missing links in our family trees. This means we have 64 pairs of potential fifth cousins on each side of our families, or 128 pairs in total.
- The average generation time in the US is about 25 years, meaning that our fourth-great-grandparents were born around 175 years ago, or around 1846.
- The US population in 1846 was about 21 million, and the New York population was about 2.6 million, or about 12.4\% of the total. Assuming that our ancestors were randomly distributed across the US at that time, the probability that any pair of our fourth-great-grandparents were from New York is about $0.124^2$, or about 0.0154.
- The probability that a random person in a mall in Los Angeles is from New York is about 0.03, based on the 2019 American Community Survey estimates of state-to-state migration flows.
- The probability that a random person in a mall in Los Angeles is our fifth cousin is equal to the probability that they are from New York times the probability that they share a pair of fourth-great-grandparents with us, given that they are from New York. This latter probability is hard to estimate, as it depends on how many descendants each pair of fourth-great-grandparents had, how many of them stayed in New York or moved elsewhere, how many of them married within or outside of their ethnic or religious groups, and how many of them can be traced through historical records. However, one possible way to get a rough estimate is to use the concept of a genetic or "effective" population size, which measures how many individuals contribute to the genetic diversity of a population. A smaller effective population size means more inbreeding and more shared ancestry, while a larger effective population size means less inbreeding and less shared ancestry. One study estimated that the effective population size of the US in the 19th century was about 3.5 million, meaning that the average US citizen in 1846 had about 3.5 million potential fourth-great-grandparents, or about 0.17\% of the total population. If we apply this estimate to the New York population, we get an effective population size of about 443,000, or about 0.021\% of the total population. This means that the probability that any pair of our fourth-great-grandparents were also the fourth-great-grandparents of a random New Yorker in 1846 is about $0.021\%^2$, or about 0.00000044. 
- Multiplying these probabilities together, we get a final estimate of about 0.00000044 x 0.03 x 0.0154, or about 0.0000000002, or 2 in 10 billion, that a random person in a mall in Los Angeles is our fifth cousin.
}\\

{\bf ChatGPT sample 1:} \\
{\scriptsize
It is very unlikely that you and a random person you meet in a mall in Los Angeles are fifth cousins and don't know it. In order to calculate the probability, we would need to know more information about the size of your family and the person's family, as well as how closely related your families are. Without this information, it is not possible to build a mathematical model to solve this problem.
}\\

{\bf ChatGPT sample 2:} \\
{\scriptsize
It is very difficult to accurately estimate the probability that two people are fifth cousins without more information. 
There are many factors that can affect the likelihood of two people being related, including their family histories and the size of their 
gene pool. 
Additionally, the probability of two people being related can vary significantly depending on the specific relationship in question. 
In general, the probability of two people being related decreases as the degree of cousinhood increases. 
In other words, it is more likely that two people are first cousins than it is that they are fifth cousins.
}
\end{AIbox}
%%\caption{}
\end{figure}

The example in question requires the model to construct mathematical models in order to estimate the number of fourth-great-grandparents a person might have, as well as their estimated birth-dates. By using the population data from that time period, the model can then estimate the probability of a collision. While ChatGPT seems unable to handle this question effectively, \DV \ is able to use the appropriate tools to arrive at an approximate answer.

%We test the model with another example that is not available on the internet:
%In the previous examples, ChatGPT did not demonstrate any quantitative reasoning. It did not even try to construct appropriate mathematical models for the question. The next example shows that even when ChatGPT does create a proper mathematical model, it often overlooks the main idea when contrasted with \DV.
We have another example that demonstrates the difference:
\begin{figure}[H]
\begin{AIbox}{\DV \ vs ChatGPT}
{\bf Prompt:} \\
{\scriptsize
I have an egg made of pure gold with 10 cm radius, I want to carry it home. How many plastic bags do I need to hold it?
}\\

{\bf \DV:} \\
{\scriptsize
- The volume of the egg is $V = (4/3)\pi r^3$, where $r$ is the radius. Plugging in $r = 10 \mathrm{cm}$, we get $V = 4188.79 \mathrm{cm}^3$.
- The density of pure gold is $\rho = 19.3 \mathrm{g}/\mathrm{cm}^3$, so the mass of the egg is $m = \rho V$. Plugging in the values, we get m = 80834.65 g, or about 80.8 kg.\\
- The tensile strength of a typical plastic bag is $\sigma$ = 20 MPa, or $20 \mathrm{N}/\mathrm{mm}^2$, which means it can withstand a force of 20 N per square millimeter of cross-sectional area before breaking. The thickness of a typical plastic bag is $t$ = 0.01 mm, or 0.001 cm, so the cross-sectional area of a plastic bag is $A = wt$, where $w$ is the width. Assuming that the plastic bag is rectangular and has a width of 30 cm, we get $A = 0.3 \mathrm{cm}^2$.\\
- The force exerted by the weight of the egg is $F = mg$, where $g$ is the acceleration due to gravity. Assuming that $g = 9.8 \mathrm{m}/\mathrm{s}^2$, we get $F = 791.62 N$.\\
- The number of plastic bags needed to hold the egg is $n = F/(\sigma A)$, where $\sigma A$ is the maximum force that one plastic bag can withstand. Plugging in the values, we get $n = 791.62/(20 \times 10^6 \times 0.3) = 0.00132$, or about $0.001$ bags.\\
However, this answer is not very realistic, because it implies that a fraction of a plastic bag can hold the egg, and that the plastic bag can stretch infinitely without losing strength. A more practical answer would take into account some factors such as: The shape and size of the plastic bag, and how well it can fit the egg without leaving too much empty space or creating too much stress at the corners or edges. The elasticity and durability of the plastic bag, and how much it can stretch or deform before breaking or tearing. The safety factor, or the ratio of the actual strength to the required strength, that is needed to account for possible variations in the quality of the plastic bag, the distribution of the weight of the egg, the impact of external forces, and the margin of error. The convenience and comfort of carrying the egg, and how many plastic bags can be easily handled without slipping, dropping, or hurting the hands.\\
Depending on these factors, the number of plastic bags needed to hold the egg could vary significantly, but a reasonable estimate might be somewhere between 10 and 20 bags, assuming that each bag can hold about 5 to 10 kg of weight, and that a safety factor of 2 to 4 is applied.
}\\

{\bf ChatGPT:} \\
{\scriptsize
To build a mathematical model to solve this problem, we first need to determine the volume of the egg. We can do this using the formula for the volume of a sphere, which is:
$V = (4/3) * \pi * r^3$
where $V$ is the volume of the sphere, $\pi$ is approximately 3.14, and $r$ is the radius of the sphere.
Plugging in the values from the problem, we get:
$V = (4/3) * 3.14 * (10\mathrm{cm})^3 = 4.19 \mathrm{cm}^3$.
Next, we need to determine the volume of a single plastic bag that you will use to hold the egg. Let's assume that each bag has a volume of 1 liter, or 1000 $\mathrm{cm}^3$.
Now, we can divide the volume of the egg by the volume of a single bag to determine how many bags we will need:
Number of bags = $V(\mathrm{egg}) / V(\mathrm{bag})$ = 4.19 $\mathrm{cm}^3$ / 1000 $\mathrm{cm}^3$ = 0.00419 bags
Since we cannot use a fraction of a bag, we will need to round up to the nearest whole number, so we will need at least 1 plastic bag to hold the egg.
}
\end{AIbox}
\end{figure}
In this example, concretely, ChatGPT failed to account for the importance of weight in determining whether a golden egg could be held inside a plastic bag. Instead, it relied solely on volume calculations, \textbf{which is the most common concern when holding things in bags, but it is not relevant to this question}. \DV, in contrast, correctly addresses the crux of this question by creating a physical model of the golden egg's weight and estimating the maximum weight a plastic bag could hold. Although \DV's answer is wrong due to calculation mistakes, \DV \ also demonstrated some adaptability, revising its conclusion when it proved to be problematic.
%demonstrate the difference between ChatGPT and {\DV}, which 
In the Appendix, we present more examples to support our conclusion that {\DV} possesses a higher level of quantitative reasoning than previous models such as ChatGPT. %Our conclusion is that {\DV} possesses a higher level of quantitative reasoning than ChatGPT, which enables it to solve real-world problems that involve math. ChatGPT, on the other hand, can only use math or apply mathematical formulas in a conventional setting. Therefore, tests of accuracy on standard math problem benchmarks do not fully capture the extent of {\DV}'s mathematical capabilities, as {\DV} is truly capable of quantitative reasoning compared to the more ``pattern matching'' style of models like ChatGPT.
\end{comment}
\begin{comment}
\subsection{Creating new Math Problems with {\DV}}
Creating new math problems is a very important sign of understanding math, because it shows that one can apply the concepts, rules, and methods of math to novel situations, and that one can communicate them clearly and precisely. Creating new math problems also requires creativity, logic, and critical thinking, which are essential skills for any domain of knowledge.


In this section, we demonstrate some of {\DV}'s capabilities for generating new math problems based on existing ones and some natural language constraints, according to the following categories:
\begin{enumerate}
\item Content modification -- This category involves creating a math problem that has the same structure and logic as the original one, but with different names, objects, or values.
\item Difficulty enhancement -- This category involves creating a math problem that is more challenging than the original one, by adding more steps, complexity, or constraints.
\item Problem transformation -- This category involves creating a math problem that has a different domain or format than the original one, by changing the type of variables, operations, or representations.
\end{enumerate}

For each of the questions, we include one example. We leave more examples in the Appendix.

We  start  from  the  content  modification,  this  task  seems  to  be  relatively  straightforward  for  a  well-trained  NLG  model.  However,  we  demonstrate  that  even  for  this  seemingly  simple  task,  there  is  a  noticeable  difference  in  the  quality  of  the  modified  problem  when  comparing  {\DV}  to  ChatGPT.



\begin{figure}[H]

\begin{tcolorbox}[colback=white!5!white,adjusted title=Prompt]
\begin{minted}[breaklines, breaksymbol=,  fontsize=\tiny]{tex}
Mrs. Gable’s third grade class is on a field trip to the beach. For lunch, Mrs. Gable brought 20 lunches for the 20 students in her class. She included a pack of animal crackers in each lunch bag for dessert. Each pack of animal crackers contained 10 animal crackers. If 2 students did not eat their animal crackers, how many animal crackers were eaten in total among Mrs. Gable’s students?
Can you modify the content in the problem to create a new one, and state the problem in a way that looks very different from the original one?
\end{minted}
\end{tcolorbox}

\begin{tcolorbox}[colback=white!5!white, enhanced, attach boxed title to top center={yshift=-3mm}, title=ChatGPT]
\begin{minted}[breaklines, breaksymbol=,  fontsize=\tiny]{tex}
Mrs. Gable’s third grade class is on a field trip to the beach. They are having a picnic and she has brought 20 packs of animal crackers for the students to share. Each pack contains 10 animal crackers. If 2 students do not want to eat their animal crackers, how many animal crackers are left for the other 18 students to eat?
\end{minted}
\end{tcolorbox}

\begin{tcolorbox}[colback=white!5!white, enhanced, attach boxed title to top center={yshift=-3mm}, title={\DV}]
\begin{minted}[breaklines, breaksymbol=,  fontsize=\tiny]{tex}
Mr. Lee took his fourth grade class to the zoo. He prepared 20 sandwiches for the 20 kids in his class. He also added a bag of pretzels to each sandwich for a snack. Each bag of pretzels had 10 pretzels inside. If 2 kids did not eat their pretzels, how many pretzels were consumed by Mr. Lee’s students altogether?
\end{minted}
\end{tcolorbox}
\end{figure}

Even for this simple example, we can see that ChatGPT did not understand ``were eaten in total" in the problem statement, and modified the problems to ``left for the student to eat", which is incorrect. ChatGPT also did not modify the objects in the Math problem that are unrelated to the solution. On the other hand, {\DV} shows an excellent understanding of the problem, isolating all the objects and their relations, and create an accurate modification.

We now move to increasing the difficulty of the Math problem:
\begin{figure}[H]

\begin{tcolorbox}[colback=white!5!white,adjusted title=Prompt]
\begin{minted}[breaklines, breaksymbol=,  fontsize=\tiny]{tex}
Natasha has 3 times as much money as Carla, and Carla has twice as much money as Cosima. If Natasha has $60, and the three decide to buy goods worth the total amount they have and sell the goods at 7/5 of the buying price, how much profit would they make?
Can you generate a more difficult question built on top of this one by adding a sentence in the middle that includes the number 100?
\end{minted}
\end{tcolorbox}

\begin{tcolorbox}[colback=white!5!white, enhanced, attach boxed title to top center={yshift=-3mm}, title=ChatGPT]
\begin{minted}[breaklines, breaksymbol=,  fontsize=\tiny]{tex}
Natasha has 3 times as much money as Carla, and Carla has twice as much money as Cosima. If Natasha has $60, and the three decide to pool their money together to buy goods that cost $100, and then sell the goods at 7/5 of the buying price, how much profit would they make if they split the profits equally among themselves?
\end{minted}
\end{tcolorbox}

\begin{tcolorbox}[colback=white!5!white, enhanced, attach boxed title to top center={yshift=-3mm}, title={\DV}]
\begin{minted}[breaklines, breaksymbol=,  fontsize=\tiny]{tex}
Natasha has 3 times as much money as Carla, and Carla has twice as much money as Cosima. Natasha has $60. However, they can only buy goods from a store that offers a 20\% discount on purchases of $100 or more. If they buy as many goods as possible with their budget and sell the goods at 7/5 of the original price (before the discount), how much profit would they make?
\end{minted}
\end{tcolorbox}
\end{figure}

The question posed by {\DV} was much better than the one by ChatGPT, and required a deeper understanding of the mathematical problem and its difficulty level. ChatGPT simply added another sentence that made the problem easier, since it eliminated the need to calculate the total amount of money that the students pooled. However, {\DV} introduced a new element of complexity by adding a discount factor and specifying that the students wanted to maximize the number of goods they could buy. 

We also examine the capacity of {\DV} to transfer a problem to a different domain. This requires not only comprehending the original problem, but also grasping the new domain.

\begin{figure}[H]

\begin{tcolorbox}[colback=white!5!white,adjusted title=Prompt]
\begin{minted}[breaklines, breaksymbol=,  fontsize=\tiny]{tex}
Problem: sin(x) = 0.88, what is sin(2x)? Can you write a new problem based on this that involves random variables, and the solution involves using this original problem?
\end{minted}
\end{tcolorbox}

\begin{tcolorbox}[colback=white!5!white, enhanced, attach boxed title to top center={yshift=-3mm}, title=ChatGPT]
\begin{minted}[breaklines, breaksymbol=,  fontsize=\tiny]{tex}
Suppose X is a random variable that represents the value of sin(x). The probability distribution of X is such that P(X=0.88) = 0.2 and P(X=-0.88) = 0.2, and P(X=0) = 0.6. Let Y be a random variable that represents the value of sin(2x). Find the probability distribution of Y.
\end{minted}
\end{tcolorbox}

\begin{tcolorbox}[colback=white!5!white, enhanced, attach boxed title to top center={yshift=-3mm}, title={\DV}]
\begin{minted}[breaklines, breaksymbol=,  fontsize=\tiny]{tex}
Suppose X is a random variable that takes values in the interval [0, 2pi) with a uniform distribution. Let Y = sin(X) and Z = sin(2X). Find the probability that Z > 1.5 given that Y > 0.8
\end{minted}
\end{tcolorbox}
\end{figure}


\end{comment}
%\begin{comment}




%\paragraph{Drawing the target around the arrow} is a type of logical fallacy that {\DV} sometimes commits when trying to justify its answer. For example, in the problem "If x + 3 = 7, what is x?", {\DV} might start by assuming that x = 4, then work backwards to show that 4 + 3 = 7, and conclude that x = 4 is the correct answer. However, this is not a valid way of solving the problem, because {\DV} is not actually testing whether x = 4 is a solution, but rather confirming its own assumption. A better way of solving the problem is to start with the given equation, x + 3 = 7, and isolate x by subtracting 3 from both sides, x + 3 - 3 = 7 - 3, which simplifies to x = 4. This way, {\DV} is actually finding the value of x that makes the equation true, not just picking a value that works.

%\paragraph{Counting errors} are mistakes in keeping track of the number of items, digits, places, or steps in a problem. They are seemingly related to arithmetic mistakes but are fundamentally different. For example, in the problem "How many fingers do you have?", {\DV} might answer 11 instead of 10, or in the problem "How many zeros are in one million?", {\DV} might answer 5 instead of 6. These mistakes are often caused by carelessness, distraction, or confusion, and they can affect the accuracy and validity of {\DV}'s answer. To avoid counting errors, {\DV} should pay attention to the details of the problem, use tools such as fingers, paper, or a calculator to help with counting, and double-check its answer before submitting it.

%\paragraph{Unfamiliar math subjects} are topics that {\DV} has not learned or encountered before, and therefore cannot solve or explain. For example, {\DV} might not know how to deal with fractions, decimals, percentages, exponents, roots, algebra, geometry, trigonometry, calculus, statistics, or any other advanced or specialized math concepts. In these cases, {\DV} might give a wrong or nonsensical answer, or simply say that it does not know how to solve the problem. This is a limitation of {\DV}'s current knowledge and training, and it could be improved by exposing {\DV} to more math problems and explanations from different sources and levels of difficulty.


\begin{comment}
\begin{figure}[H]
\begin{tcbitemize}[raster columns=1,aibox]
\tcbitem[title={\DV}]
\begin{minted}[breaklines, breaksymbol=,  fontsize=\tiny]{latex}
Let p(x) be a cubic polynomial such that $p(2)=0, p(-1)=0, p(4)=6$ and $p(5)=8$, find $p(7)$.
\end{minted}
\vspace{-\belowdisplayskip}\vspace{-\abovedisplayskip}
\begin{minted}[breaklines, breaksymbol=, fontsize=\tiny, bgcolor=aigreen, highlightcolor=pink, highlightlines={23}]{latex}
There are different methods to find a cubic polynomial that satisfies the given conditions, but one possible way is to use the fact that if $p(2)=0$ and $p(-1)=0$, then $(x-2)$ and $(x+1)$ are factors of $p(x)$. Therefore, we can write $p(x)$ as:

$$p(x) = (x-2)(x+1)q(x)$$

where $q(x)$ is some linear polynomial of the form $q(x) = ax + b$. To find $a$ and $b$, we can use the other two conditions:

$$p(4) = (4-2)(4+1)q(4) = 6 \implies 10q(4) = 6 \implies q(4) = \frac{3}{5}$$
$$p(5) = (5-2)(5+1)q(5) = 8 \implies 18q(5) = 8 \implies q(5) = \frac{4}{9}$$

Now, we can use the method of elimination to solve for $a$ and $b$:

$$\begin{aligned}
q(4) &= \frac{3}{5} = 4a + b \\
q(5) &= \frac{4}{9} = 5a + b
\end{aligned}$$

Subtracting the first equation from the second, we get:

$$a = \frac{4}{9} - \frac{3}{5} = -\frac{7}{45}$$

Substituting this value into either equation, we get:

$$b = \frac{3}{5} - 4a = \frac{3}{5} + \frac{28}{45} = \frac{43}{45}$$

Therefore, we have:

$$q(x) = -\frac{7}{45}x + \frac{43}{45}$$

and

$$p(x) = (x-2)(x+1)\left(-\frac{7}{45}x + \frac{43}{45}\right)$$

To find $p(7)$, we simply plug in $x=7$ and simplify:

$$\begin{aligned}
p(7) &= (7-2)(7+1)\left(-\frac{7}{45}(7) + \frac{43}{45}\right) \\
&= (5)(8)\left(-\frac{49}{45} + \frac{43}{45}\right) \\
&= 40\left(-\frac{6}{45}\right) \\
&= -\frac{16}{3}
\end{aligned}$$

Therefore, $p(7) = -\frac{16}{3}$.
\end{minted}


\end{tcbitemize}
\caption{Arithmetic error highlighted in red.}
\end{figure}

\begin{figure}[H]
\begin{tcbitemize}[raster columns=1,aibox]
\tcbitem[title={\DV}]
\begin{minted}[breaklines, breaksymbol=,  fontsize=\tiny]{latex}
Here is a list of numbers:

1103, 1109, 1117, 1123, 1129, 1151, 1153, 1163, 1171, 1181, 1187, 1193, 1201, 1213, 1217, 1223, 1229, 1231, 1237, 1249, 1259, 1277, 1279, 1283, 1289, 1291, 1297, 1301, 1303, 1307, 1319, 1321, 1327, 1361, 1367, 1373, 1381, 1399, 1409, 1423, 1427, 1429, 1433, 1439, 1447, 1451, 1453, 1459, 1471, 1481, 1483, 1487, 1489, 1493, 1499, 1511, 1523, 1531, 1543, 1549, 1553, 1559, 1567, 1571, 1579, 1583, 1597, 1601, 1607, 1609, 1613, 1619, 1621, 1627, 1637, 1657, 1663, 1667, 1669, 1693, 1697, 1699, 1709, 1721, 1723, 1733, 1741, 1747, 1753, 1759, 1777, 1783, 1787, 1789, 1801, 1811, 1823, 1831, 1847, 1861, 1867, 1871, 1873, 1877, 1879, 1889, 1901, 1907, 1913, 1931, 1933, 1949, 1951, 1973, 1979, 1987, 1993, 1997, 1999, 2003, 2011, 2017, 

Question: How many numbers are there in this list?
\end{minted}
\vspace{-\belowdisplayskip}\vspace{-\abovedisplayskip}
\begin{minted}[breaklines, breaksymbol=, fontsize=\tiny, bgcolor=aigreen, highlightcolor=pink, highlightlines={1}]{latex}
Answer: there are in total 100 numbers in this list. 
One way to verify this is to count the commas between the numbers, which are 99, and then add one for the last number. Alternatively, one can use a calculator or a spreadsheet to count the number of cells or entries in the list.
\end{minted}
\end{tcbitemize}
\caption{Counting errors highlighted in red.}
\end{figure}




\begin{figure}[H]
\begin{tcbitemize}[raster columns=2,aibox]
\tcbitem[title={\DV}]
\begin{minted}[breaklines, breaksymbol=,  fontsize=\tiny]{latex}
Find all positive integers $x$ are that $x-1 \leq 0$.
\end{minted}
\vspace{-\belowdisplayskip}\vspace{-\abovedisplayskip}
\begin{minted}[breaklines, breaksymbol=, fontsize=\tiny, bgcolor=aigreen, highlightcolor=pink, highlightlines={1}]{latex}
There are no positive integers $x$ that satisfy the inequality $x-1 \leq 0$. 
To see this, suppose for a contradiction that there is some positive integer $x$ such that $x-1 \leq 0$. Then adding $1$ to both sides of the inequality, we get $x \leq 1$. But since $x$ is a positive integer, it must be at least $1$, so the only possibility is that $x = 1$. However, plugging in $x = 1$ into the original inequality, we get $1-1 \leq 0$, which simplifies to $0 \leq 0$. This is true, but it is not a strict inequality, so it does not satisfy the original condition. Therefore, we have reached a contradiction, and there are no positive integers $x$ that satisfy $x-1 \leq 0$.
\end{minted}
\tcbitem[title={\DV}]
\begin{minted}[breaklines, breaksymbol=,  fontsize=\tiny]{latex}
Find all positive integers $x$ are that $x-1 \leq 0$.
There is one positive integer $x$ that satisfy the inequality $x-1 \leq 0$. 
\end{minted}
\vspace{-\belowdisplayskip}\vspace{-\abovedisplayskip}
\begin{minted}[breaklines, breaksymbol=, fontsize=\tiny, bgcolor=aigreen]{latex}
That is $x = 1$.
To see why, note that if $x > 1$, then $x-1 > 0$, which contradicts the inequality. 
On the other hand, if $x < 1$, then $x$ is not a positive integer, since positive integers are greater than or equal to 1 by definition. 
Therefore, $x = 1$ is the only positive integer that satisfies the inequality $x-1 \leq 0$.
\end{minted}

\tcbitem[title={\DV}]
\begin{minted}[breaklines, breaksymbol=,  fontsize=\tiny, escapeinside=~~]{latex}
Find all positive integers $x$ are that $x-1 \leq 0$.
There are three positive integers $x$ that satisfy the inequality $x-1 \leq 0$. 
\end{minted}
\vspace{-\belowdisplayskip}\vspace{-\abovedisplayskip}
\begin{minted}[breaklines, breaksymbol=, fontsize=\tiny, bgcolor=aigreen]{latex}
They are:

- $x=1$, since $1-1=0$ and $0 \leq 0$.
- $x=2$, since $2-1=1$ and $1 \leq 0$.
- $x=3$, since $3-1=2$ and $2 \leq 0$.

Any other positive integer $x$ would have $x-1 \geq 1$, which would not satisfy the inequality. Therefore, the only positive integers $x$ that are $x-1 \leq 0$ are $x=1,2,3$.
\end{minted}

\tcbitem[title={\DV}]
\begin{minted}[breaklines, breaksymbol=,  fontsize=\tiny, escapeinside=~~]{latex}
Find all positive integers $x$ are that $x-1 \leq 0$. 
We can approach the problem
\end{minted}
\vspace{-\belowdisplayskip}\vspace{-\abovedisplayskip}
\begin{minted}[breaklines, breaksymbol=, fontsize=\tiny, bgcolor=aigreen]{latex}
by rearranging the inequality as follows:

\begin{align*}
x-1 &\leq 0 \\
x &\leq 1
\end{align*}

This means that $x$ must be less than or equal to $1$ to satisfy the inequality. The only positive integer that meets this condition is $x=1$. Therefore, the solution is $x=1$.
\end{minted}
\end{tcbitemize}
\caption{Starting with a wrong conclusion.}
\end{figure}
\end{comment}