\subsection{Higher level mathematics}

We conclude this section with a few examples that demonstrate the model's potential performance on more advanced mathematical topics. These examples have been intentionally selected to demonstrate the model's capabilities, and it is important to note that the model does not always succeed with questions of this level of difficulty. Instead, they are meant to showcase the extent of the model's capabilities, suggesting what future models might be able to achieve. \\

We begin with a simplification of a question which appeared in the 2022 International Mathematics Olympiad (IMO).

\begin{figure}[H]
\begin{AIbox}{\DV}
{\bf Prompt:} \\
{\scriptsize
Suppose $g$ is a continuous smooth function such that for every $x > 0$, there is \textbf{one and only one} $y > 0$ such that $g(x) + g(y) \leq 2xy$. Can you prove that $g(x) = x^2$? 
}\\

{\bf \DV:} \\
{\scriptsize
There are a few possible ways to approach this problem, but one possible method is as follows:\\
- Assume, for the sake of contradiction, that $g(x)$ is not equal to $x^2$ for some $x > 0$.\\
- Then, either $g(x) < x^2$ or $g(x) > x^2$ for that $x$.\\
- If $g(x) < x^2$, then let $y = x$. Then $g(x) + g(y) = 2g(x) < 2x^2 = 2xy$.\\
- If $g(x) + g(y) < 2xy$, then $g(x) + g(y)$ is strictly less than the upper bound of $2xy$ for positive $x$ and $y$.\\
- This means that there is some room to increase $y$ slightly and still satisfy the inequality $g(x) + g(y) \leq 2xy$.\\
- For example, let $y' = y + \epsilon$, where $\epsilon$ is a small positive number such that $g(x) + g(y') < 2xy$. This is possible because $g$ is continuous and smooth, so it does not have any jumps or kinks that would prevent such a choice of epsilon.\\
- Then, $y'$ is another positive number such that $g(x) + g(y') \leq 2xy'$, which contradicts the assumption that $y$ is the one and only one positive number that satisfies this condition given $x$.\\
- Therefore, if $g(x) < x^2$, then $g(x) + g(y)$ cannot be strictly less than $2xy$, and it must be equal to $2xy$. \\
- If $g(x) > x^2$, then let $y^*$ be the unique positive solution of $g(x) + g(y) = 2xy$. Then $g(y^*) < (y^*)^2$, since otherwise we would have $g(x) + g(y^*) > 2xy$, which contradicts the existence of $y^*$ given $x$.\\
- But then, by the same argument as before, we can let $x = y = y^*$ and obtain a contradiction, since $g(y) + g(y) = 2g(y) < 2y^2 = 2yy$, which contradicts the uniqueness of $y$ given $y$.\\
- Therefore, $g(x)$ must be equal to $x^2$ for all $x > 0$, as required. 
}
\end{AIbox}
\end{figure}
What distinguishes this question from those that typically appear in undergraduate calculus exams in STEM subjects is that it does not conform to a structured template. Solving it requires a more creative approach, as there is no clear strategy for beginning the proof. For example, the decision to split the argument into two cases ($g(x) > x^2$ and $g(x) < x^2$) is not an obvious one, nor is the choice of $y^*$ (its reason only becomes clear later on in the argument). Furthermore, the solution demands knowledge of calculus at the undergraduate level. Nevertheless, \DV \ manages to produce a correct proof. \\

The second example is in the form of a discussion which concerns algorithms and graph theory, topics typically covered in the first or second year of an undergraduate computer science degree. The discussion is comparable to that of a graduate-level interview.
\begin{figure}[H]
\begin{AIbox}{\DV}\vspace{-5px}
\parbox{1\textwidth}{\scriptsize
\begin{alltt}
Human: Can you remind me what a $k$-SAT problem is?

AI: \hl{A $k$-SAT problem is a type of logical problem that involves finding a way to assign truth values (true or false) to a set of variables that appear in a collection of clauses, each of which consists of a disjunction (logical OR) of exactly $k$ literals. A literal is either a variable or its negation. For example, ($x_1$ OR NOT $x_2$ OR $x_3$) is a clause with $k = 3$ and three literals.

The problem is to find an assignment of truth values to the variables that satisfies (makes true) all the clauses in the collection, or to determine that no such assignment exists. This is called the satisfiability problem, or SAT for short.}

Human: What's the most obvious way (the first thing you would refer to when teaching a class) to solve a k-SAT problem?

AI: \hl{There are different ways to approach the k-SAT problem, but one of the most obvious and straightforward methods is to use brute force, or exhaustive search. This means trying all possible combinations of truth values for the variables, and checking if any of them satisfies all the clauses. For example, if there are n variables, there are $2^n$ possible combinations to try.}

Human: Ok, great. Now, given the instance of a k-SAT problem suppose that I define a graph where each vertex refers to a variable in the instance, and two vertices are connected by an edge if and only if the two variables appear in the same clause. Suppose that for a certain instance, the graph I just described has two connected components of size $n/2$ each. In this case, is there a more efficient way to solve the k-SAT instance?

AI: \hl{Yes, there is a more efficient way to solve the k-SAT instance in this case. Since the graph has two connected components, this means that the clauses can be partitioned into two subsets, such that each subset only involves the variables in one component. Therefore, the k-SAT instance is equivalent to solving two independent $k$-SAT instances, each with $n/2$ variables. This reduces the number of possible combinations to try from $2^n$ to $2^{n/2} + 2^{n/2}$, which is much smaller for large n. In general, the more disconnected the graph is, the easier the $k$-SAT instance becomes, because it can be decomposed into smaller subproblems.}

Human: Now suppose that the graph has only one connected component, but if you were to remove one single edge, then it would have two connected components. Can you suggest an algorithm in this case?

AI: \hl{One possible algorithm in this case is to use a divide and conquer strategy. First, identify the edge that, if removed, would split the graph into two components. Then, assign a truth value to one of the variables that are connected by that edge, and solve the resulting $k$-SAT instance recursively. If the instance is satisfiable, then return the solution. If the instance is unsatisfiable, then assign the opposite truth value to the same variable, and solve the $k$-SAT instance recursively again. If the instance is still unsatisfiable, then return that the original $k$-SAT instance is unsatisfiable. This algorithm reduces the number of possible combinations to try from $2^n$ to $2^{n-1} + 2^{n-1}$, which is still better than brute force. However, this algorithm depends on finding the edge that splits the graph, which may not be easy in general.}

Human: Can you generalize this principle to the case that the graph has $q$ edges which, if removed, would split the graph into $p$ connected components?

AI: \hl{The general principle is similar, but more complicated. The idea is to assign truth values to the variables that are connected by the q edges that split the graph, and solve the resulting $k$-SAT instance recursively for each possible assignment. If any of the assignments leads to a satisfiable $k$-SAT instance, then return the solution. If none of the assignments leads to a satisfiable $k$-SAT instance, then return that the original k-SAT instance is unsatisfiable. This algorithm reduces the number of possible combinations to try from $2^n$ to $2^q \times 2^{(n-q)/p}$, which is better than brute force if $q$ is small and $p$ is large. However, this algorithm also depends on finding the $q$ edges that split the graph, which may be hard in general.}

Human: Note that the complexity that you wrote is only correct in the case that all connected components are of the same size. Can you find a more general formula?

AI: \hl{Yes, you are right. The more general formula is $2^q \times \sum_i 2^{n_i}$, where $n_i$ is the number of variables in the $i$-th connected component, and the sum is over all $p$ components. This is because each component has $2^{n_i}$ possible combinations to try, and the $q$ edges that split the graph have $2^q$ possible assignments. This formula is equal to the previous one if all $n_i$ are equal to $(n-q)/p$.}
\end{alltt}
}
\end{AIbox}
\caption{A discussion regarding $k$-SAT problems and graph theory.}
\end{figure}
{\DV} demonstrates understanding of the concepts from graph theory and algorithms. It is able to reason about an abstract graph construction, which relates to a constraint satisfaction problem, and deduce correct conclusions about the SAT problem (to the best of our knowledge, this construction does not appear in the mathematical literature). The conversation reflects profound understanding of the undergraduate-level mathematical concepts discussed, as well as a significant extent of creativity. Although {\DV} makes a mistake in one instance (writing $2^{n-1}$ rather than $2^{n/2}$), the subsequent conversation demonstrates that this error does not reflect a lack of understanding. Rather, {\DV} seems to have made an error in a manner similar to a human typo, as it later provides the correct generalization of the formula.

Our last example requires the grasp of a new mathematical definition, and combines knowledge in both number theory and probability. The model produces a sound argument but makes a counting mistake at the end which leads to an incorrect final answer.
\begin{figure}[H]
\begin{AIbox}{\DV}
{\bf Prompt:} \\
{\scriptsize
A good number is a 5-digit number where the 1,3,5-th digits are odd numbers and they form an increasing arithmetic progression, and the number is divisible by 3. If I randomly sample a good number, what is the probability that its 2nd digit is 4?
}\\

{\bf \DV:} \\
{\scriptsize
There are a few ways to approach this problem, but one possible method is to use some modular arithmetic and enumeration. 
First, we can observe that a 5-digit number is divisible by 3 if and only if the sum of its digits is divisible by 3. So, if we let the 5-digit number be $abcde$, where $a$, $c$, and $e$ are odd digits forming an arithmetic progression, we have the equation:
$$a + b + c + d + e = 0 \quad (\textrm{mod } 3)$$
We can also observe that the difference between any two consecutive terms in an arithmetic progression is constant, so we can let that difference be $f$,  where $f$ is a positive integer. Then we have:
$$c = a + f, e = a + 2f$$
Substituting these into the equation above, we get:
$$a + b + a + f + d + a + 2f = 0 \quad (\textrm{mod } 3)$$
$$3a + b + d + 3f = 0 \quad (\textrm{mod } 3)$$
Since $3a$ and 3f are always divisible by 3, we can ignore them and focus on the equation:
$$b + d = 0 \quad (\textrm{mod } 3)$$
The choice of $a$, $f$ and $e$ does not affect this equation, so we can fix them and vary $b$ and $d$. For example, let $a$ = 1, $f$ = 2, and $e$ = 5. Then we have the possible values of $b$ and $d$ as:
$$b = 0, d = 0, 3, 6, 9; b = 1, d = 2, 5, 8; b = 2, d = 1, 4, 7; b = 3, d = 0, 3, 6, 9; b = 4, d = 2, 5, 8$$
$$b = 5, d = 1, 4, 7; b = 6, d = 0, 3, 6, 9; b = 7, d = 2, 5, 8; b = 8, d = 1, 4, 7; b = 9, d = 0, 3, 6, 9$$
We can see that there are 10 possible values of $b$, and for each value of $b$, there are 4 possible values of $d$. So, there are $10 \times 4 = 40$ possible pairs of b and d that satisfy the equation. Out of these 40 pairs, only 3 have $b = 4$, namely (4, 2), (4, 5), and (4, 8). So, the probability that $b = 4$ is $3/40$.
}
\end{AIbox}
\end{figure}

