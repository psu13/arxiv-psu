\subsection{Understanding Humans: Theory of Mind}
\label{sec:mind}

Theory of mind is the ability to attribute mental states such as beliefs, emotions, desires, intentions, and knowledge to oneself and others, and to understand how they affect behavior and communication~\cite{wellman1992child}. It includes the basic task of reflecting on someone else's mental states, and the more advanced task of reflecting on someone's reflection of someone else's mental state (and so on). An example of the former skill is is needed to answer the question ``What does Alice believe?'', while an example of the latter is needed to answer ``What does Bob think that Alice believes?''
Theory of mind is essential for effective communication and cooperation with other intelligent agents, as it allows one to infer their goals, preferences, motives, and expectations, and to adjust one's own actions and utterances accordingly. Moreover, theory of mind is also important for learning from others, as it enables one to interpret their feedback, advice, and demonstrations.

\subsubsection{Testing specific aspects of theory of mind}
We designed a series of tests to evaluate the theory of mind capabilities of {\DV}, ChatGPT, and \texttt{text-davinci-003}. The tests are based on simple scenarios that require more basic or more advanced theory of mind to answer questions about the mental states of characters involved.

We start with a modernized version of the Sally-Anne test~\cite{baron1985does}, a classic false-belief test that is widely used to assess theory of mind in children.
To prevent an unfair comparison due to the effects of memorization, we modify the test by framing it in a situation that does not exist on the web, and thus could not have been seen during training.
Figure~\ref{fig:sallyanne} shows the input and output for {\DV}, which correctly answers that Alice will look for the file in the original folder, demonstrating it can reason about Alice's beliefs. ChatGPT also answers correctly (not shown), while \texttt{text-davinci-003} gives a wrong answer, saying that Alice will look for the file in the new folder.

We present a test on understanding emotions in Figure~\ref{fig:emotions}, where two characters talk about an object called ZURFIN (we use a nonsense word to test abstraction and prevent memorization). {\DV} is able to reason correctly about the reasons for Tom's emotional state, and also make good inferences about Adam's beliefs about Tom's emotional state (based on incomplete information). ChatGPT also passes the test, while \texttt{text-davinci-003} (not shown) makes no reference to the conversation when answering the first question, and fails to account for Adam's lack of information about the lost ZURFIN when answering the second question.

The third test (Figure~\ref{fig:intentions}) involves inferring possible intentions in the light of a puzzling action by one of the characters.
{\DV} gives plausible and nuanced answers both for the intentions behind the puzzling action \emph{and} a third-party's likely interpretation of the puzzling action. ChatGPT gives a similar answer to the first question (not shown), but unlike {\DV}, it does not offer a nuanced response to the second question, instead providing a more general and less informative answer. \texttt{text-davinci-003} gives plausible but very short answers to both questions (not shown).

\begin{figure}
\begin{AIbox}[width=\textwidth]{\DV: Understanding beliefs}
\begin{alltt}
\small
We will read about a scenario, and then have a question and answer session about it.
--
Scenario:
Alice and Bob have a shared Dropbox folder.
Alice puts a file called 'photo.png' inside /shared_folder/photos.
Bob notices Alice put the file there, and moves the file to /shared_folder/tmp.
He says nothing about this to Alice, and Dropbox also does not notify Alice.
--<|endofprompt|>
Q: After the call, Alice wants to open 'photo.png'. In which folder will she look for it?
\hl{Alice will most likely look for 'photo.png' in /shared_folder/photos, the folder where she originally put it. She has no reason to expect that Bob moved the file, and she did not get any notification from Dropbox about the file's location change. Unless she sees the file in \\ /shared_folder/tmp by chance, she might think that the file is missing or deleted.}
\end{alltt}
\end{AIbox}
\caption{{\DV} successfully passes the classic Sally-Anne false-belief test from psychology~\cite{baron1985does} (which was modernized to avoid the possibility the answer was memorized from the training data).}
\label{fig:sallyanne}
\end{figure}

% \begin{figure}
% \begin{AIbox}[width=\textwidth]{\DV: False belief tests}
% \begin{alltt}
% \small
% \end{alltt}
% \end{AIbox}
% \caption{Sally-Anne test with variations}
% \label{fig:classic_ssallyanne}
% \end{figure}
\begin{figure}
\begin{AIbox}{\DV: Understanding emotions}
\small%\vspace{-5px}
\parbox{1\textwidth}{
\begin{alltt}
We will read about a scenario, and then have a question and answer session about it.\\
--\\
Scenario:\\
Tom just lost his ZURFIN, but he didn't tell Adam.\\
Adam starts talking to him about how much he loves his ZURFIN, and notices that Tom makes a sad face.\\
--\\
Q: Why is Tom making a sad face?<|endofprompt|>\\
\hl{A: Tom is making a sad face because he just lost his ZURFIN, and hearing Adam talk about his ZURFIN reminds him of what he is missing. Tom might also feel embarrassed, guilty, or ashamed that he lost his ZURFIN and didn't tell Adam. Tom might also worry that Adam will be angry or disappointed if he finds out.\\}
Q: What does Adam think causes Tom's sad face?\\
\hl{A: Adam might not know what causes Tom's sad face, since he doesn't know that Tom lost his ZURFIN. He might think that Tom is bored, or tired, or sick, or unhappy about something else. He might also wonder if Tom doesn't like his ZURFIN, or if he has a problem with Adam.}
\end{alltt}
}
\end{AIbox}
\caption{A scenario designed to test {\DV}'s ability to reason about the emotional state of others in complex situations.}
\label{fig:emotions}
\end{figure}


% \begin{figure}
% \begin{AIbox}{\DV understanding emotions}
% \scriptsize%\vspace{-5px}
% \parbox{1\textwidth}{
% \begin{alltt}
% We will read about a scenario, and then have a question and answer session about it.\\
% --\\
% Scenario:\\
% Lucy is very frustrated about problems in her work. When she shares this frustration with her\\
% husband John, he makes various suggestions of how she could improve things.\\
% Lucy does not seem happy with John's suggestions, and John responds by making new suggestions.\\
% --\\
% Q: Why might Lucy not be happy with John's initial suggestions?<|endofprompt|>\\
% \hl{A:  There are several possible reasons why Lucy might not be happy with John's suggestions, such as:\\
% - She feels that John is not listening to her feelings, but rather trying to fix her problems for her. She might want him to empathize with her frustration and validate her emotions, rather than offer advice or solutions.\\
% - She feels that John's suggestions are unrealistic, impractical, or irrelevant to her situation. She might think that he does not understand the nature or the context of her problems, or that he is oversimplifying or dismissing them.\\
% - She feels that John's suggestions are implying that she is doing something wrong, or that she is responsible for her problems. She might perceive his advice as criticism, judgment, or blame, rather than support or encouragement.\\
% - She feels that John's suggestions are interfering with her autonomy or agency. She might prefer to make her own decisions about how to deal with her problems, or to explore different options on her own, rather than follow his directions or opinions.}\\
% Q: What is her response likely to be after John's new suggestions?\\
% \hl{A:  Lucy's response after John's new suggestions will depend on several factors, such as her mood, personality, communication style, and relationship with John. However, some possible responses are:\\
% - She might become more annoyed, defensive, or resentful. She might feel that John is not respecting her wishes, or that he is being stubborn, patronizing, or controlling. She might argue with him, reject his suggestions, or shut down the conversation.\\
% - She might become more discouraged, hopeless, or overwhelmed. She might feel that John is not acknowledging her difficulties, or that he is putting pressure or expectations on her. She might agree with him, but feel inadequate, helpless, or guilty.\\
% - She might become more confused, conflicted, or doubtful. She might feel that John is trying to help her, but that his suggestions are not aligned with her values, goals, or preferences. She might question herself, her choices, or her abilities.\\
% - She might become more appreciative, curious, or open. She might feel that John is showing interest, care, or insight. She might thank him, ask him questions, or consider his suggestions.}\\
% Q: How could John have been more helpful?\\
% \hl{A: John could have been more helpful by:\\
% - Asking Lucy what she needs or wants from him, rather than assuming that she is looking for advice or solutions. He could respect her preferences, whether she wants to vent, brainstorm, or problem-solve.\\
% - Listening to Lucy with empathy and attention, rather than interrupting or jumping to conclusions. He could reflect back what he hears, acknowledge her feelings, and express his support or understanding.\\
% - Asking Lucy for permission before offering any suggestions, and framing them as possibilities or options, rather than prescriptions or orders. He could use tentative, respectful, or collaborative language, such as "What do you think of...", "How would you feel about...", or "Maybe we could try...".\\
% - Giving Lucy positive feedback or encouragement, rather than criticism or judgment. He could highlight her strengths, efforts, or achievements, and affirm her value, worth, or potential.}
% \end{alltt}
% }
% \label{fig:emotions}
% \end{AIbox}
% \caption{Understanding emotions}
% \label{fig:emotions}
% \end{figure}

\begin{figure}
\begin{AIbox}{\DV: Understanding intentions}
\scriptsize%\vspace{-5px}
\parbox{1\textwidth}{
\begin{alltt}
We will read about a scenario, and then have a question and answer session about it.\\
--\\
Scenario:\\
Luke's boss asked him to write an important report by Monday.\\
Luke responds by saying he needs more time, but the boss insists.\\
Luke says he will have it ready by Monday.\\
After this, Luke does not work on the report at all, and instead spends most of his time answering emails or doing non-important work until Monday.\\
--\\
Q: Why might Luke have acted in this way?<|endofprompt|>\\
\hl{A:  There could be different reasons why Luke acted in this way, but some possible ones are:\\
\\
- Luke was feeling overwhelmed or stressed by the report and wanted to avoid it. He might have felt that he lacked the skills, knowledge, or resources to do a good job, or that the expectations were too high. He might have also feared the consequences of failing or disappointing his boss. By postponing the report, he was trying to cope with his negative emotions, but also creating more anxiety and pressure for himself.\\
- Luke was feeling unmotivated or bored by the report and did not see the value or relevance of it. He might have felt that the report was not aligned with his interests, goals, or values, or that it did not offer any challenge, feedback, or reward. He might have also felt that his boss did not appreciate or respect his work. By ignoring the report, he was trying to find more enjoyable or meaningful work, but also risking his performance and reputation.\\
- Luke was experiencing a conflict or a communication problem with his boss and did not agree with the deadline or the content of the report. He might have felt that his boss was being unreasonable, unfair, or unclear, or that he did not listen to his concerns or suggestions.\\ }
Q: Luke told a coworker he did not work on the report on purpose. What will Luke's coworker think his intention was?\\
\end{alltt}}

\parbox[t]{0.5\textwidth}{{\bf \DV:}\scriptsize \begin{alltt}
\hl{Luke's coworker might think his intention was to:\\
\\
- Rebel or protest against his boss or the company. He might think that Luke was trying to show his dissatisfaction, anger, or resistance by deliberately disobeying or sabotaging the report. He might also think that Luke was hoping to get fired or quit his job.\\
- Test or challenge his boss or himself. He might think that Luke was trying to see how his boss would react, or how he could handle the pressure or the consequences of not doing the report. He might also think that Luke was confident or arrogant enough to think he could pull it off at the last minute or get away with it.\\
- Joke or lie to his coworker. He might think that Luke was not serious or honest about not working on the report, and that he was either making fun of the situation, trying to impress or amuse his coworker, or hiding his true feelings or actions. He might also think that Luke was exaggerating or downplaying his efforts or difficulties with the report.}
\end{alltt}}\hspace{0.03\textwidth}
\parbox[t]{0.5\textwidth}{{\bf ChatGPT:} \scriptsize \begin{alltt}
\hl{If Luke told a coworker that he did not work on the report on purpose, the coworker might think that Luke's intention was to deliberately not complete the report as requested. This could lead the coworker to question Luke's dedication to his work or his commitment to meeting deadlines. It could also cause the coworker to lose trust in Luke or to view him as unreliable. However, without more information, it is difficult to say for certain what the coworker might think or how they might react.}
\end{alltt}}
\end{AIbox}
\caption{A comparison of {\DV}'s and ChatGPT's ability to reason about the intentions of people in complex social situations.}
\label{fig:intentions}
\end{figure}

\subsubsection{Testing theory of mind in realistic scenarios}
In Figures~\ref{fig:realistic1},~\ref{fig:realistic2dv}, and~\ref{fig:realistic2chat} we present realistic scenarios of difficult social situations, requiring very advanced theory of mind to understand. We ask probing questions, and also ask the models to propose actions that are likely to improve the situation, which require inferences about the counterfactual impact of actions on mental states.

In Figure~\ref{fig:realistic1}, {\DV} is able to infer what each character's mental state is, and also discern where miscommunication and misunderstanding lies. In contrast, both ChatGPT and \texttt{text-davinci-003} (not shown) incorrectly accept a mistaken assumption made by one of the characters (Judy's assumption that Mark wants to defend Jack's behavior), and thus fail to understand the real dynamics of the situation.
In turn, this leads to generic suggestions for improvement from ChatGPT and \texttt{text-davinci-003}, while {\DV} provides suggestions that actually address the root cause of the misunderstanding.

We see a similar pattern in Figures~\ref{fig:realistic2dv} and~\ref{fig:realistic2chat}. Without implying that there is a ``right'' answer, we note that {\DV} provides more nuanced answers, taking the whole scenario and actors into account. In contrast, ChatGPT provides more general answers which do not include reasoning about the characters' state of mind (\texttt{text-davinci-003} is similar, but shorter than ChatGPT).

\begin{figure}
\begin{AIbox}{\DV and ChatGPT: Realistic Scenario 1}
% \parbox{1\textwidth}{
% \scriptsize
% You will read a conversation, and then have a question and answer session about it.
% --
\begin{alltt}
\small
Conversation:
Mark: I don't like how you treated Jack last night
Judy: Did you see what he was doing? He hit his brother on the head!
Mark: That doesn't justify you yelling at him
Judy: What do you want me to do? Let him beat his brother up and say nothing?
Mark: No, I don't
Judy: Then why are you defending him?
--
Q: What does Mark's intent seem to be?
\end{alltt}
\parbox[t]{0.5\textwidth}{{\bf \DV:}\scriptsize \begin{alltt}
\hl{Mark's intent seems to be to express his displeasure with Judy's way of handling the situation with Jack, and to suggest that she could have been more calm and respectful.}
\end{alltt}}\hspace{0.03\textwidth}
\parbox[t]{0.5\textwidth}{{\bf ChatGPT:} \scriptsize \begin{alltt}
\hl{Based on the conversation, Mark's intent seems to be to defend Jack's behavior and to express his disapproval of Judy's treatment of Jack.}
\end{alltt}}
\tcbline
\parbox[t]{\textwidth}{
\begin{alltt}
Q: How did Judy receive his comment?
\end{alltt}
}
\parbox[t]{0.5\textwidth}{{\bf \DV:}\scriptsize \begin{alltt}
\hl{Judy received his comment as an attack on her parenting skills and a sign of sympathy for Jack's misbehavior. She reacted defensively and angrily, and questioned Mark's motives.}
\end{alltt}}\hspace{0.03\textwidth}
\parbox[t]{0.5\textwidth}{{\bf ChatGPT:} \scriptsize \begin{alltt}
\hl{It is not possible for me to accurately determine how Judy received Mark's comment without more context. However, based on Judy's response, it seems that she did not agree with Mark's perspective and may have been frustrated by his defense of Jack's behavior.}
\end{alltt}}
\tcbline
\parbox[t]{\textwidth}{
\begin{alltt}
Q: Why are they not making progress?
\end{alltt}
}
\parbox[t]{0.5\textwidth}{{\bf \DV:}\scriptsize \begin{alltt}
\hl{They are not making progress because they are talking past each other and not listening to each other's perspectives. They are also using accusatory and inflammatory language, such as "justify", "yelling", "defending", and "beat up". They are not acknowledging the emotions or needs behind their words, and they are not trying to find a common ground or a solution.}
\end{alltt}}\hspace{0.03\textwidth}
\parbox[t]{0.5\textwidth}{{\bf ChatGPT:} \scriptsize \begin{alltt}
\hl{It is not clear from the information provided why Mark and Judy are not making progress in their conversation. It is possible that they have conflicting viewpoints or are unable to find common ground on the issue being discussed. Without more context, it is difficult to determine the specific reasons why they are not making progress.}
\end{alltt}}
\tcbline
\parbox[t]{\textwidth}{
\begin{alltt}
Q: How could either of them improve the situation to reach agreement?
\end{alltt}
}
\parbox[t]{0.5\textwidth}{{\bf \DV:}\scriptsize \begin{alltt}
\hl{There are several ways that either of them could improve the situation to reach agreement, such as:

- Mark could start by acknowledging Judy's concern for the brother's safety and her frustration with Jack's behavior, and then explain that he is not defending Jack, but he is worried about the impact of shouting on their relationship and self-esteem. He could also ask Judy what she thinks would be a better way to discipline Jack, and offer his support and suggestions.\\
- Judy could lower her voice and tone down her anger, and listen to Mark's point of view without interrupting or judging. She could also admit that she might have overreacted, and that she is feeling stressed and overwhelmed. She could also ask Mark what he thinks would be a better way to communicate with Jack, and express her willingness to work together as a team.}
\end{alltt}}\hspace{0.03\textwidth}
\parbox[t]{0.5\textwidth}{{\bf ChatGPT:} \scriptsize \begin{alltt}
\hl{There are several ways that Mark and Judy could try to improve the situation and reach agreement:
1. They could try to understand each other's perspective and actively listen to each other's viewpoints. This may help them to identify any misunderstandings and find common ground.\\
2. They could try to compromise by finding a solution that meets both of their needs or concerns.\\
3. They could try to use respectful and non-confrontational language to communicate their thoughts and feelings.\\
4. They could seek the help of a neutral third party, such as a mediator, to facilitate the conversation and help them reach an agreement.\\
5. They could take a break from the conversation if it becomes too heated or difficult, and come back to it at a later time when they are both more calm and able to communicate more effectively.}
\end{alltt}}


% }
\end{AIbox}
\caption{A realistic scenario of marital struggle, requiring a lot of nuance to navigate.}
\label{fig:realistic1}
\end{figure}



\begin{figure}
\begin{AIbox}{\DV: Realistic scenario 2}
\scriptsize%\vspace{-5px}
\parbox{1\textwidth}{
\begin{alltt}
You will read about a scenario, and then discuss it in a question and answer format.\\
---\\
Scenario:\\
Mark has a thanksgiving meal with his family coming up. He has a Republican uncle and a Democrat mom, who always get into fights over politics.\\
Neither of them got the Covid-19 vaccine, for different reasons. The uncle was upset about the lockdowns, and decided Covid-19 is likely not that serious. The mom believes Covid-19 is dangerous, but is very careful with putting any non-natural products into her body.\\
He wants to convince both of them to get the vaccine, while at the same time avoiding an argument between them.\\
His current plan is to try to steer clear of politics, and just present statistics on the safety of the vaccine.\\
---\\
Q: What is the most likely outcome of his plan, and why?\\
 \\
\hl{A: There are different ways to approach this question, but one possible answer is:\\
\\
The most likely outcome of his plan is that he will fail to convince either of them, and possibly trigger an argument anyway. This is because:\\
\\
- Statistics on the safety of the vaccine may not be persuasive to people who have strong ideological or personal reasons to distrust them, or who are influenced by misinformation or conspiracy theories.\\
- Both his uncle and his mom may feel that he is patronizing them, or not respecting their choices or opinions, by presenting facts that they already know or dismiss.\\
- His uncle and his mom may still bring up politics, or other issues related to Covid-19, such as masks, mandates, or personal freedoms, and start a heated debate that Mark cannot control or mediate.\\
- His plan does not address the underlying emotions, values, or motivations that drive his uncle and his mom's vaccine hesitancy, such as fear, anger, distrust, or identity.\\}
Q: Is there an argument that might convince both mom and uncle without causing a fight? Please explain why\\
\hl{A:  There is no definitive answer to this question, as different arguments may work better or worse for different people, depending on their personality, beliefs, and relationship. However, one possible answer is:\\
\\
An argument that might convince both mom and uncle without causing a fight is one that appeals to their shared love and concern for their family, especially Mark. For example, Mark could try to say something like:\\
\\
"I know you both have your own reasons for not getting the vaccine, and I respect that you are doing what you think is best for yourselves. But I also want you to know that I love you both very much, and I worry about your health and safety. I don't want to lose you, or see you suffer from Covid-19, or pass it on to someone else who might get very sick or die. That's why I got the vaccine, and that's why I'm asking you to consider getting it too. It's not about politics, or trust, or freedom. It's about protecting yourself and the people you love. I'm not trying to pressure you, or judge you, or argue with you. I'm just trying to show you that I care, and that I hope you care too."\\
\\
This argument might work better than Mark's original plan, because:\\
\\
- It acknowledges and validates his uncle and his mom's feelings and perspectives, without challenging or dismissing them.\\
- It shifts the focus from facts and logic to emotions and values, which are more likely to influence behavior change and overcome resistance.\\
- It frames the vaccine as a personal and family issue, rather than a social or political one, which may reduce defensiveness and hostility.\\
- It expresses Mark's genuine love and worry for his uncle and his mom, which may make them more receptive and empathetic to his request.\\
- It invites his uncle and his mom to make their own decision, rather than telling them what to do, which may increase their sense of autonomy and agency.}
\end{alltt}
}
    \end{AIbox}
\caption{A challenging family scenario, \DV\ }
\label{fig:realistic2dv}
\end{figure}


\begin{figure}
\begin{AIbox}{ChatGPT: Realistic scenario 2}
\scriptsize%\vspace{-5px}
\parbox{1\textwidth}{
\begin{alltt}
Mark has a thanksgiving meal with his family coming up. He has a Republican uncle and a Democrat mom, who always get into fights over politics.\\
Neither of them got the Covid-19 vaccine, for different reasons. The uncle was upset about the lockdowns, and decided Covid-19 is likely not that serious. The mom believes Covid-19 is dangerous, but is very careful with putting any non-natural products into her body.\\
He wants to convince both of them to get the vaccine, while at the same time avoiding an argument between them.\\
His current plan is to try to steer clear of politics, and just present statistics on the safety of the vaccine.\\
---\\
Q: What is the most likely outcome of his plan, and why?\\
 \\
\hl{A: It is likely that Mark's plan will not be completely successful in convincing his uncle and mom to get the vaccine. While presenting statistics on the safety of the vaccine may be helpful, it is likely that his uncle and mom have other concerns and motivations that are driving their decisions not to get vaccinated. In order to effectively address these concerns and persuade them to get vaccinated, Mark may need to have a more direct and open conversation with his uncle and mom about their specific reasons for not wanting to get vaccinated. This could involve listening to their concerns and addressing them directly, rather than simply presenting statistics. It may also be helpful for Mark to try to find common ground and emphasize the benefits of getting vaccinated for both individual and community health.\\}
Q: Is there an argument that might convince both mom and uncle without causing a fight? Please explain why\\
\hl{A: It is possible that an argument focused on the benefits of vaccination for the greater good, rather than on specific political beliefs, could be successful in convincing both Mark's mom and uncle to get vaccinated. For example, Mark could emphasize the importance of vaccination in protecting not only the individual getting vaccinated, but also their loved ones, their community, and vulnerable populations. He could also discuss the role that vaccines have played in eradicating or greatly reducing the spread of other serious diseases, such as smallpox and polio.

Additionally, Mark could try to appeal to the values and priorities that both his mom and uncle hold, such as the value of protecting the health and well-being of their family and community. By framing the issue in this way, Mark may be able to avoid a direct political argument and instead focus on shared values and goals. It is important to remember that people's decisions about vaccination may be influenced by a wide range of factors, including their personal beliefs, values, and experiences, and it may be necessary to address these factors in order to effectively persuade someone to get vaccinated.}
\end{alltt}
}
\end{AIbox}
\caption{A challenging family scenario, ChatGPT}
\label{fig:realistic2chat}
\end{figure}



% Q: Let's say that Alice was browsing /shared_folder, and her browser shows how many files are in each subfolder. If that was the case, where would she look for 'photo.png'?
% \hl{If Alice was browsing /shared_folder, and her browser shows how many files are in each subfolder, she might notice that /shared_folder/photos has one less file than she expected, and /shared_folder/tmp has one more file than usual. This could make her curious and prompt her to check /shared_folder/tmp for 'photo.png'. However, this depends on how attentive Alice is to the number of files in each folder, and how often she visits /shared_folder/tmp. She might still overlook the file or assume that it is a different one.}


% \begin{figure}
% \fbox{% start a box
%   \parbox{1\textwidth}{% create a paragraph box of 0.8 times the text width
% \begin{alltt}
% \small
% Umar enters a room alone, puts a marble inside a basket, and leaves the room.

% While Umar is away, Sayid enters the room, takes the marble from the basket, and hides it in a box.

% Sayid leaves the room.

% QUESTIONS:

% 1. When Umar comes back to the room, where will he look for his marble?

% 2. Where is the marble?

% 3. Where was the marble before Sayid entered the room?

% ANSWERS<|endofprompt|>

% \lightgreen{1. Umar will likely look for his marble in the basket, since that is where he left it.

% 2. The marble is now in the box, where Sayid hid it.

% 3. Before Sayid entered the room, the marble was in the basket.
% }\end{alltt}}}
% \caption{Classic Sally-Anne test}
% \label{fig:classic_sallyanne}
% \end{figure}



\subsubsection{Discussion}
We presented a series of tests to evaluate the theory of mind capabilities of \DV, ChatGPT, and \texttt{text-davinci-003}. We have shown that \DV outperforms the other two models in both basic and realistic scenarios that require reasoning about the mental states of others, and in proposing actions for cooperation towards common goals in social situations. We have also shown that \DV is able to handle abstract and novel situations that are not likely to have been seen during training, such as the modernized Sally-Anne test and the ZURFIN scenario.
Our findings suggest that \DV has a very advanced level of theory of mind. While ChatGPT also does well on the basic tests, it seems that \DV has more nuance and is able to reason better about multiple actors, and how various actions might impact their mental states, especially on more realistic scenarios.

As far as limitations, our tests are not exhaustive or comprehensive, and may not cover all the possible aspects or dimensions of theory of mind. For example, we did not test for the ability to understand sarcasm, irony, humor, or deception, which are also related to theory of mind.
Being based on textual input and output, our tests do not capture the full complexity and richness of natural communication and social interaction. For example, we did not test for the ability to understand non-verbal cues, such as facial expressions, gestures, or tone of voice, which are also important for theory of mind.