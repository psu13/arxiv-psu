\section{Interaction with the world}
\label{sec:environment1}
\label{sec:interact}

One of the key aspects of intelligence is interactivity, which we define as the ability to communicate and respond to feedback from other agents, tools, and environments. Interactivity is important for intelligence, as it enables agents to acquire and apply knowledge, solve problems, adapt to changing situations, and achieve goals that are beyond their individual capabilities.
For example, humans interact with one another and with environments to collaborate, learn, teach, negotiate, create, etc. Interactivity requires an agent to comprehend complex ideas, learn quickly, and learn from experience, and thus it is closely tied to our definition of intelligence. 

In this section, we explore two dimensions of interactivity: tool use and embodied interaction. Tool use involves the use of external resources, such as search engines, calculators, or other APIs, to perform tasks that are difficult or impossible for the agent alone. Embodied interaction involves the use of natural language as a text interface to interact with simulated or real-world environments, and to receive feedback from them. 

