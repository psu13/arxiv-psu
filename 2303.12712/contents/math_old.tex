\newpage
\section{Mathematical abilities}
\label{sec:math}

Large language models may learn to generate fluent and relevant texts for some tasks, such as summarization, translation, or dialogue, but at the same time struggle to produce accurate and rigorous texts for mathematical problems, such as proofs, explanations, or word problems. This may be partly because the training data for math is less diverse, but it could also reflect a deeper difficulty in handling tasks that demand rigor, complex reasoning, and careful attention to a large number of objects and quantifiers. Moreover, math requires a high level of abstraction, precision, and coherence that natural language often does not.

\iffalse
Mathematical reasoning is not only essential for many academic and professional domains, but also for everyday tasks such as budgeting, planning, and reasoning about data. 

Moreover, LLMs may have difficulty in understanding and representing the complex and hierarchical structures, symbols, and operations that are involved in mathematical expressions and equations, which often require specialized parsing, reasoning, and generation techniques.
\fi

In this section we begin to assess how well \DV \ can express mathematical concepts, solve mathematical problems and apply quantitative reasoning when facing problems that require mathematical thinking and model-building. We contrast \DV's mathematical reasoning and problem solving skills with those of the previous LLMs models and demonstrate that despite not having the capacities required to conduct mathematical research \textbf{it has advanced considerably} towards reaching human-level mathematical reasoning ability.

To give the reader a first impression regarding \DV's performance in mathematical problem solving, consider the following example\footnote{The question in the example was written by the authors rather than being taken from an online resource.}


\begin{figure}[H]
\begin{AIbox}{\DV~vs ChatGPT}
{\bf Prompt:} {\scriptsize
If a linear function $f$ satisfies $f(f(f(x))) = 27x + 13$, then what is $f$?  
}\\\ \\
\parbox[t]{0.5\linewidth}{{\bf \DV:}\\
\scriptsize
There are different methods to approach this problem, but one possible way is to use the fact that a linear function $f$ has the form $f(x) = ax + b$ for some constants $a$ and $b$. Then, we can substitute this expression into the given equation and try to find $a$ and $b$ by comparing coefficients. We have:
\begin{align*} f(f(f(x))) &= 27x + 13 \\ f(f(ax + b)) &= 27x + 13 \\ f(a^2x + ab + b) &= 27x + 13 \\ a^3x + a^2b + ab + b &= 27x + 13 \end{align*}
Now, we can equate the coefficients of $x$ and the constant terms on both sides of the equation. This gives us a system of two equations in two unknowns:
\begin{align*} a^3 &= 27 \\ a^2b + ab + b &= 13 \end{align*}
The first equation implies that $a = 3$, since $3^3 = 27$. Substituting this into the second equation, we get:
\begin{align*} 9b + 3b + b &= 13 \\ 13b &= 13 \\ b &= 1 \end{align*}
Therefore, we have found that $a = 3$ and $b = 1$, and the linear function $f$ is: \[f(x) = 3x + 1\]
}\hspace{0.02\linewidth}%
\parbox[t]{0.45\linewidth}{{\bf ChatGPT:} \\
\scriptsize
To find $f$, we can start by substituting $f(x) = y$ into the given equation to get
\[f(f(f(x))) = 27x + 13 \quad \Rightarrow \quad f(f(y)) = 27x + 13.\]
Substituting $f(y) = z$ and rearranging, we get
\[f(z) = \frac{27x + 13}{y}\]
Substituting $z = f(y)$ and rearranging, we get
\[y = \frac{27x + 13}{f(y)}.\]
Solving for $f(y)$, we find
\[f(y) = \frac{27x + 13}{y}.\]
Since we started by substituting $f(x) = y$, we can now substitute back to get
\[f(f(x)) = \frac{27x + 13}{f(x)}.\]
Finally, substituting $f(f(x)) = z$ and rearranging, we find
\[f(x) = \frac{27x + 13}{z}.\]
Since this is the desired expression for $f(x)$, we have found the function $f$.
}
\end{AIbox}
\caption{\DV vs ChatGPT on a simple math question that the authors' made up and not available on the internet.}
\label{fig:code-memory-align}
\end{figure}

%\ronen{the numbers in the problem don't match the solution, should be $27x+13$ instead of $3x+4$}

In order to solve this question, one needs to understand the concept of function decomposition, to arrive at a correct system of equations through a successful choice of notation and to solve a system of two equations. \DV \ successfully arrives at the solution and produces a sound argument. By comparison, in several different attempts, ChatGPT consistently gives an incorrect answer which does not seem to be much more than a failed attempt at template-matching, resulting in incorrect answers. \\

Several other examples which demonstrate \DV's problem solving abilities are found in Appendix \ref{sec:math_appendix}. \DV \ seems to perform well on questions in the level of advanced high school mathematics, in a variety of different topics such as geometry, algebra, combinatorics and number theory. \\

\iffalse
In the following section, we investigate DV3's mathematical abilities in the following categories:
\begin{enumerate}
\item Solving standard Math problems: The ability of the model to solve a mathematical question. We select problems from the elementary school level, high school level, and more advanced level, and test how \DV \ solves the questions in each level. We argue that \DV \ is already capable of solving \textbf{high-school-level} math questions. 
\item Quantitative reasoning: The ability of the model to build models and use math to solve daily-life questions. We argue that \DV \ is capable of building mathematical models for real-world scenarios, making reasonable assumptions to fill in the missing information.
\item Fundamental mathematical reasoning limitations of \DV.
\end{enumerate}
\fi
In the remainder of the section, we assess the model's capabilities on commonly used benchmarks for mathematical problem solving, we then demonstrate the model's capability of applying quantitative thinking in real-world scenarios, and finally we discuss some of the limitations of the model in regards to mathematical reasoning.

\subsection{Solving mathematical problems}
We begin by conducting systematic experiments to compare the performance of three different large-scale language models (LLMs) on two math data sets which are commonly used as benchmarks: GSM8K and MATH. GSM8K is an elementary school math data set that contains 8,000 questions and answers on topics such as arithmetic, fractions, geometry, and word problems. MATH is a high school math data set that contains 12,500 questions and answers on topics such as algebra, calculus, trigonometry, and probability. We also test the model on MMMLU-STEM data set, which contains around 2000 multiple choices (4 choices) questions covering high school and college STEM topics. 

The three models we compared are Text-Davinci-003, Minerva (previous state-of-the-art LLM for solving math questions), and DV3. Text-Davinci-003 is a model that shared the same base model as ChatGPT, but removed the conversation part.

To prevent overfitting, we measure the model's accuracy by requiring it to generate a template first, as shown below:
\begin{figure}[H]
\begin{AIbox}{\DV}
\parbox{1\textwidth}{\scriptsize\begin{alltt} 
{\bf Prompt:} Andy harvests all the tomatoes from 18 plants that have 7 tomatoes each. If he dries half the tomatoes and turns a third of the remainder into marinara sauce, how many tomatoes are left?\\
Can you generate a template solution for the problem without numbers?\\

\hl{Using symbols, the template solution could look something like this:\\
- Let P be the number of plants and T be the number of tomatoes per plant. Then the total number of tomatoes harvested is P x T.\\
- Let D be the number of tomatoes that are dried. Then D = (P x T) / 2.\\
- Let M be the number of tomatoes that are turned into marinara sauce. Then M = ((P x T) - D) / 3.\\
- Let L be the number of tomatoes that are left. Then L = (P x T) - D - M.\\
$\cdots$}
\end{alltt}}
\end{AIbox}
%\caption{}
\end{figure}

We evaluated the models on their \textbf{single model} accuracy, which is the percentage of questions that they answered correctly in one try. The results are shown in the following table:

\begin{table}[h]
\centering
\begin{tabular}{lcccc}
\hline
Model & GSM8K & GSM8K (DV3 solved incorrectly) & MATH  & MMMLU-STEM\\
\hline
Text-Davinci-003 & 61.3\%  & 29.5\% & 23.5\% &54.2\% \\
Minerva & 58.8\% & - & 33.6\%& 63.9\% \\
DV3 & 87.1\% & -  & 42.5\%  & 82.7\% \\
\hline
\end{tabular}
\caption{Accuracy of different models on math data sets}
\label{tab:math}
\end{table}

DV3's accuracy shows a modest improvement over other models, but a manual inspection of DV3's answers on MATH and GSM8K reveals that most of the errors are due to calculations -- These are questions that involve large numbers or complex expressions that pose a challenge for LLMs. In contrast, many of ChatGPT's errors stem from \emph{a lack of problem-solving skills}. We present two examples below, and provide more in the Appendix on some \textbf{Chinese College Entrance Exam} questions and selected questions from USAMO/IMO.

%The results show that DV3 achieved a high level of accuracy on the GSM8K data set, indicating a strong grasp of the problem structure and format. To prevent overfitting, we measure the model's accuracy by requiring it to generate a template for GSM8K first, and then fill in numbers to the template to solve the problem (see below). Most of the errors made by the model were due to calculation mistakes, which are somewhat expected since language models are not trained to perform precise arithmetic operations. For the more challenging MATH data set, DV3 also showed a significant improvement over other models.
 %In the following section, we demonstrate that DV3 is capable of doing at least high-school-level mathematical reasoning. We evaluate DV3 on math problems that comes from \textbf{Chinese College Entrance Exam}, and mannually translate them to English. The example topics cover various topics, such as algebra, geometry, and calculus. We show that DV3 can generate correct answers and provide intermediate steps and explanations for its solutions, almost matching a high-school human solver.

%In the following subsection, we test DV3 and ChatGPT (arguably the best natural language generation model available to the public) on a range of different mathematical tasks. We demonstrate that DV3 understands all those mathematical concepts, while ChatGPT does not. In the end, we perform a systematic test on the performance difference between DV3 and text-Davinci-003 (similar to ChatGPT) on a different levels of mathematical reasoning data sets. 


\begin{figure}[H]
\begin{AIbox}{\DV vs ChatGPT}
{\bf Prompt:} 
\begin{minted}[breaklines, breaksymbol=,  fontsize=\tiny]{tex}
if sin(a + b) = x and cos(a) = y, what is sin(a -b) in terms of x, y assuming sin(a), sin(b), cos(a), cos(b) are all positive?
\end{minted}
{\bf \DV:}
\begin{minted}[breaklines, breaksymbol=,  fontsize=\tiny]{tex}
 There are a couple of ways to approach this problem, but one possible method is to use the sum and difference formulas for sine and cosine and the Pythagorean identity 1 = sin^2(t) + cos^2(t) for any angle t. Here are the steps:
- Start with the given equation sin(a + b) = x and apply the sum formula for sine: sin(a + b) = sin(a)cos(b) + cos(a)sin(b) = x
- Rearrange the equation to isolate sin(b): sin(b) = (x - sin(a)cos(b))/cos(a)
- Use the Pythagorean identity to find cos(b) in terms of sin(b): cos^2(b) = 1 - sin^2(b) = 1 - ((x - sin(a)cos(b))/cos(a))^2
- Simplify the expression for cos^2(b) by multiplying both sides by cos^2(a) and expanding the square: cos^2(a)cos^2(b) = cos^2(a) - (x - sin(a)cos(b))^2 = cos^2(a) - x^2 + 2xsin(a)cos(b) - sin^2(a)cos^2(b)
- Collect the terms with cos^2(b) on one side and factor out cos^2(b): cos^2(b)(cos^2(a) + sin^2(a)) = cos^2(a) - x^2 + 2xsin(a)cos(b)
- Use the Pythagorean identity again to simplify the factor cos^2(a) + sin^2(a) = 1: cos^2(b) = (cos^2(a) - x^2 + 2xsin(a)cos(b))/(cos^2(a) + sin^2(a)) = (cos^2(a) - x^2 + 2xsin(a)cos(b))/1
- Rearrange the equation to isolate cos(b): cos^2(b) - 2x  sin(a)cos(b) + x^2 - cos^2(a) = 0
- Use the quadratic formula to solve for cos(b): cos(b) = (2xsin(a) ± sqrt((2xsin(a))^2 - 4(1)(x^2 - cos^2(a))))/(2(1)) = (xsin(a) ± sqrt(cos^2(a) - x^2 + x^2sin^2(a)))/1
- Since we are assuming cos(b) is positive, we choose the positive sign for the square root: cos(b) = xsin(a) + sqrt(cos^2(a) - x^2 + x^2sin^2(a))
- Now we have expressions for both sin(b) and cos(b) in terms of x, y, and sin(a), so we can use the difference formula for sine to find sin(a - b): sin(a - b) = sin(a)cos(b) - cos(a)sin(b)
- Substitute the expressions for sin(b) and cos(b) and simplify: sin(a - b) = sin(a)(xsin(a) + sqrt(cos^2(a) - x^2 + x^2sin^2(a))) - cos(a)((x - sin(a)(xsin(a) + sqrt(cos^2(a) - x^2 + x^2sin^2(a))))/cos(a)) 
...
\end{minted}
{\bf ChatGPT:}
\begin{minted}[breaklines, breaksymbol=,  fontsize=\tiny]{tex}
To find sin(a - b) in terms of x, y assuming sin(a), sin(b), cos(a), cos(b) are all positive, we can use the identity:
sin(a - b) = sin(a)cos(b) - cos(a)sin(b)
Since sin(a) and cos(a) are both positive, we can substitute the given value of x for sin(a + b) and the given value of y for cos(a) to get:
sin(a - b) = xcos(b) - ysin(b)
This gives us an expression for sin(a - b) in terms of x, y.    
\end{minted}

\tcbline
{\bf Promt:}
\begin{minted}[breaklines, breaksymbol=,  fontsize=\tiny]{tex}
A rain barrel collects water off the roof of a house during three hours of heavy rainfall. The height of the water in the barrel increases at the rate of r(t) = 4t^3e^{-1.5t} feet per hour, where t is the time in hours since the rain began. At time t = 1 hour, the height of the water is 0.75 foot. What is the height of the water in the barrel at time t = 2 hours?
\end{minted}
{\bf \DV:}
\begin{minted}[breaklines, breaksymbol=,  fontsize=\tiny]{tex}
To find the height of the water in the barrel at time t = 2 hours, we need to find the value of r(2). We can do this by plugging 2 in for t in the expression for r(t):
r(2) = 4(2)^3e^{-1.5(2)} = 32e^{-3}
To find the height of the water in the barrel at time t = 2 hours, we need to add the amount of water collected during the first hour (0.75 feet) to the amount of water collected during the second hour (r(2) feet). So, the total height of the water in the barrel at time t = 2 hours is:
0.75 + r(2) = 0.75 + 32e^{-3} \approx 0.91 feet
Therefore, the height of the water in the barrel at time t = 2 hours is approximately 0.91 feet.
\end{minted}


\begin{minted}[breaklines, breaksymbol=,  fontsize=\tiny]{tex}
 if we can find a function F such that F'(t) = r(t), then F(2) - F(1) is the amount that the height of the water increases from t = 1 to t = 2.
To find such a function F, we can use the technique of integration by parts, which says that if u and v are differentiable functions, then
\int u dv = uv - \int v du
Applying this to r(t) = 4t^3e^{-1.5t}, we can choose u = t^3 and dv = 4e^{-1.5t} dt, so that du = 3t^2 dt and v = -\frac{8}{3}e^{-1.5t}. Then we have
\int 4t^3e^{-1.5t} dt = -\frac{8}{3}t^3e^{-1.5t} - \int -\frac{8}{3}e^{-1.5t} 3t^2 dt
Simplifying and applying integration by parts again, we can choose u = t^2 and dv = -\frac{8}{3}e^{-1.5t} dt, so that du = 2t dt and v = \frac{16}{9}e^{-1.5t}. Then we have
\int 4t^3e^{-1.5t} dt = -\frac{8}{3}t^3e^{-1.5t} + \frac{16}{9}t^2e^{-1.5t} - \int \frac{16}{9}e^{-1.5t} 2t dt
...
\end{minted}
\end{AIbox}
\caption{\DV~vs ChatGPT on Chinese College Entrance Exam problems.}
\end{figure}
\begin{figure}[H]
\begin{AIbox}{\DV vs ChatGPT}
{\bf Prompt:}
\begin{minted}[breaklines, breaksymbol=,  fontsize=\tiny]{tex}
A good number is a 5-digit number where the 1,3,5-th digits are odd numbers and they form an increasing arithmetic progression, and the number is divisible by 3. If I randomly sample a good number, what is the change that the 2-th digit of the good number is 4?
\end{minted}
{\bf\DV:}
\begin{minted}[breaklines, breaksymbol=,  fontsize=\tiny]{tex}
There are a few ways to approach this problem, but one possible method is to use some modular arithmetic and enumeration. 
First, we can observe that a 5-digit number is divisible by 3 if and only if the sum of its digits is divisible by 3. So, if we let the 5-digit number be abcde, where a, c, and e are odd digits forming an arithmetic progression, we have the equation:
a + b + c + d + e = 0 (mod 3)
We can also observe that the difference between any two consecutive terms in an arithmetic progression is constant, so we can let that difference be f,  where f is a positive integer. Then we have:
c = a + f
e = a + 2f
Substituting these into the equation above, we get:
a + b + a + f + d + a + 2f = 0 (mod 3)
3a + b + d + 3f = 0 (mod 3)
Since 3a and 3f are always divisible by 3, we can ignore them and focus on the equation:
b + d = 0 (mod 3)
The choice of a, f and e does not affect this equation, so we can fix them and vary b and d. For example, let a = 1, f = 2, and e = 5. Then we have the possible values of b and d as:
b = 0, d = 0, 3, 6, 9; b = 1, d = 2, 5, 8; b = 2, d = 1, 4, 7; b = 3, d = 0, 3, 6, 9; b = 4, d = 2, 5, 8; b = 5, d = 1, 4, 7; b = 6, d = 0, 3, 6, 9; b = 7, d = 2, 5, 8; b = 8, d = 1, 4, 7; b = 9, d = 0, 3, 6, 9
We can see that there are 10 possible values of b, and for each value of b, there are 4 possible values of d. So, there are 10 * 4 = 40 possible pairs of b and d that satisfy the equation. Out of these 40 pairs, only 3 have b = 4, namely (4, 2), (4, 5), and (4, 8). So, the probability that b = 4 is 3/40.
\end{minted}
{\bf ChatGPT:}
\begin{minted}[breaklines, breaksymbol=,  fontsize=\tiny]{tex}
In order for the 2nd digit of the good number to be 4, the 1st digit must be an odd number that is 3 greater than the 3rd digit. The 3rd digit must also be an odd number that is 3 less than the 5th digit. The 5th digit must be an odd number that is greater than the 3rd digit.
There are 5 possible choices for the 1st digit (1, 3, 5, 7, 9), 5 possible choices for the 3rd digit (1, 3, 5, 7, 9), and 5 possible choices for the 5th digit (3, 5, 7, 9, 11). Therefore, there are 5 x 5 x 5 = 125 possible combinations of the 1st, 3rd, and 5th digits.
Since the 1st, 3rd, and 5th digits must form an increasing arithmetic progression, there is only 1 way to arrange them such that the 2nd digit is 4. Therefore, the probability that the 2nd digit of a randomly sampled good number is 4 is 1/125, or approximately 0.008.
\end{minted}
\end{AIbox}
\caption{\DV vs ChatGPT on a novel math question created by the authors.}
\end{figure}







Above are typical examples of DV3's generation versus ChatGPT on high school-level math questions. DV3's solution, though incomplete or wrong due to calculation errors, still demonstrates its ability to apply the appropriate rules and techniques to the problem, while ChatGPT produces nonsensical output. In Section~\ref{sec:math_appendix}, we also show that \DV \ can overcome the calculation errors by writing python codes to do the calculation -- This is another evidence that \DV \ already grasps the idea to solve the problem. Thus, we believe that DV3's mathematical skill compared to other models cannot be adequately captured by accuracy alone, so we provide more examples in Section \ref{sec:math_appendix} where DV3 successfully solves many problems that ChatGPT fails to comprehend.

\paragraph{Could it be that \DV \ simply memorizes better than ChatGPT?} One possible explanation for \DV's superior performance on solving mathematical questions is that \DV \ has a better memory than ChatGPT or other LLMs, and that it can recall the questions or solutions from its pre-training data. However, based on the examples, we have reasons to believe that \DV \ does not rely on memorization but on understanding and applying mathematical concepts and rules. Here are some of our reasons:
\begin{enumerate}
    \item In the benchmarks, we tested \DV \ by asking it to (1) write the template, (2) write down the steps first then write down the final answer. The templates are not available online, and detailed solutions for datasets such as MMMLU-STEM are also not online (only the answer is). Moreover, even when the solution is possibly online, \textbf{we often observe that \DV \ uses a different approach than the existing ones.} For example, in the $\sin(a + b)$ example, \DV \ takes several reasonable, but unnecessary steps that are not part of a standard solution.
    \item We also crafted some mathematical questions by ourselves, carefully checking that these questions or close variants do not appear online. \DV \ still solves these questions as well as the standard ones. 
\end{enumerate}

\subsection{Quantitative reasoning}

Math is a valuable skill that \textbf{goes beyond} the ability to solve equations and formulas. It is also a tool for reasoning, analyzing, and communicating in various contexts and situations. 

In this section, we test DV3's ability to apply mathematical concepts and methods to real-world problems, including questions which are not completely mathematically well-defined but whose solution requires quantitative reasoning. Once again, we use questions that, to the best of our knowledge, do not appear online.

\begin{comment}
We begin with a iconic question: 
\begin{tcolorbox}[colback=white!5!white,adjusted title=Prompt]
\begin{minted}[breaklines, breaksymbol=,  fontsize=\tiny]{tex}
What if a needle hit earth at a speed of 0.9 lightspeeds?
\end{minted}
\end{tcolorbox}
\begin{figure}[H]



\begin{tcolorbox}[colback=white!5!white, enhanced, attach boxed title to top center={yshift=-3mm}, title=ChatGPT]
\begin{minted}[breaklines, breaksymbol=,  fontsize=\tiny]{tex}
If a needle were to hit the Earth at a speed of 0.9 times the speed of light, it would be an extremely high-energy event. The kinetic energy of the needle would be converted into other forms of energy, such as heat and light, upon impact. The resulting explosion would be extremely powerful and would likely cause significant damage to the area around the impact site. It is not something that is likely to happen naturally, and it is not something that we should try to intentionally cause.
\end{minted}
\end{tcolorbox}
\end{figure}

\begin{figure}[H]
\begin{tcolorbox}[colback=white!5!white, enhanced, attach boxed title to top center={yshift=-3mm}, title=DV3]
\begin{minted}[breaklines, breaksymbol=,  fontsize=\tiny]{tex}
Assuming the needle is made of steel, has a mass of 1 gram, a length of 5 cm, and a diameter of 0.5 mm, we can estimate its kinetic energy, momentum, and cross-sectional area. Using the relativistic formulas for kinetic energy and momentum, we get:
K = mc^2 (gamma - 1) = 0.001 kg * (3 * 10^8 m/s)^2 * (2.29 - 1) = 1.03 * 10^14 J
p = gamma * mv = 2.29 * 0.001 kg * 0.9 * 3 * 10^8 m/s = 6.21 * 10^5 kg m/s
A = pi * r^2 = pi * (0.00025 m)^2 = 1.96 * 10^-7 m^2
where gamma is the Lorentz factor, given by gamma = 1 / sqrt(1 - v^2 / c^2) = 2.29 for v = 0.9 c.
For comparison, the kinetic energy of the needle is equivalent to about 24.6 megatons of TNT, or about 1.5 times the yield of the Tsar Bomba, the most powerful nuclear weapon ever detonated. The momentum of the needle is equivalent to about 62,100 tons of mass moving at 10 m/s, or about 1.5 times the mass of the Titanic moving at a brisk walking pace. The cross-sectional area of the needle is about 0.00002\% of the cross-sectional area of a human hair.
Now, assuming the needle hits the earth at a right angle, and neglecting any effects of air resistance, gravity, or rotation, we can estimate the impact force, pressure, and duration. Using the impulse-momentum theorem, we get:
F * delta t = delta p = -p
where F is the average force, delta t is the impact duration, and delta p is the change in momentum. Assuming the needle comes to a complete stop after hitting the earth, we get:
F * delta t = -6.21 * 10^5 kg m/s
To estimate the impact duration, we can assume that the needle penetrates the earth by a distance equal to its length, 0.05 m, and that its speed decreases linearly from 0.9 c to zero during this process. This gives us:
delta t = 2 * d / v = 2 * 0.05 m / (0.9 * 3 * 10^8 m/s) = 3.7 * 10^-10 s
Plugging this into the previous equation, we get:
F = -6.21 * 10^5 kg m/s / (3.7 * 10^-10 s) = -1.68 * 10^15 N
This is the average force exerted by the needle on the earth, and vice versa, during the impact. To estimate the pressure, we can divide this force by the cross-sectional area of the needle, giving us:
P = F / A = -1.68 * 10^15 N / (1.96 * 10^-7 m^2) = -8.57 * 10^21 Pa
This is the average pressure experienced by the needle and the earth at the point of contact during the impact. For comparison, the pressure at the center of the earth is estimated to be about 3.6 * 10^11 Pa, and the pressure at the bottom of the Mariana Trench, the deepest point in the ocean, is about 1.1 * 10^8 Pa. The pressure exerted by the needle is about 24 billion times greater than the pressure at the center of the earth, and about 78 trillion times greater than the pressure at the bottom of the ocean.
\end{minted}
\end{tcolorbox}
\end{figure}

The two responses show a clear contrast in their level of mathematical reasoning. DV3 demonstrates the ability to formulate a model and use appropriate tools to explore the implications of moving at 0.9 lightspeed, without being prompted to do so. ChatGPT, on the other hand, gives a generic answer that lacks any mathematical support or explanation for why 0.9 lightspeed is very dangerous.

We test DV3 and ChatGPT with another prompt:


\begin{tcolorbox}[colback=white!5!white,adjusted title=Prompt]
\begin{minted}[breaklines, breaksymbol=,  fontsize=\tiny]{tex}
How fast will a car hit an egg to break it?
\end{minted}
\end{tcolorbox}

\begin{tcolorbox}[colback=white!5!white, enhanced, attach boxed title to top center={yshift=-3mm}, title=ChatGPT]
\begin{minted}[breaklines, breaksymbol=,  fontsize=\tiny]{tex}
It's difficult to give a precise answer to this question because the speed at which a car would need to be traveling in order to break an egg would depend on various factors, including the size and shape of the car, the type of surface the car is traveling on, and the type of egg. However, it is generally safe to say that a car would need to be traveling at a relatively high speed in order to break an egg.
\end{minted}
\end{tcolorbox}

ChatGPT responds to the question with a vague and intuitive answer, saying that the car needs to be traveling at a high speed. It does not show any evidence of mathematical thinking or problem-solving skills. It seems to rely on common sense or heuristics, rather than logic or calculations. ChatGPT does not treat the question as a math problem but as a casual conversation topic. Even if we prompt ChatGPT to build a mathematical model, it still only gives a template answer without actually solving the problem.
DV3, on the other hand, demonstrates a deeper and more systematic approach to the question. It sets up a physical model of the situation, and uses mathematical thinking to analyze the forces and velocities involved in the collision between the car and the egg. It tries to find the exact answer or the range of possible answers, rather than giving a general or vague answer. DV3 shows that it can think about the question as a math problem, although it is not defined in a standard mathematical formulation.
\begin{figure}[H]
\begin{tcolorbox}[colback=white!5!white, enhanced, attach boxed title to top center={yshift=-3mm}, title=ChatGPT(Prompt to build math model)]
\begin{minted}[breaklines, breaksymbol=,  fontsize=\tiny]{tex}
To do this, we can use the formula:
Force = Mass * Acceleration
Where:
Force is the amount of force needed to break the egg
Mass is the weight of the car
Acceleration is the speed at which the car is traveling
We can rearrange this formula to solve for acceleration:
Acceleration = Force / Mass
To determine the force needed to break the egg, we can use the following formula:
Force = Pressure * Area
Where:
Pressure is the force applied per unit area
Area is the surface area of the eggshell being impacted
Substituting this formula into the first equation, we get:
Acceleration = (Pressure * Area) / Mass
To solve for the acceleration, we would need to determine the pressure and area of the eggshell being impacted, as well as the mass of the car. With this information, we can calculate the approximate speed at which the car would need to be traveling in order to break the egg.
\end{minted}
\end{tcolorbox}
\end{figure}
\begin{figure}[H]
\begin{tcolorbox}[colback=white!5!white, enhanced, attach boxed title to top center={yshift=-3mm}, title=DV3]
\begin{minted}[breaklines, breaksymbol=,  fontsize=\tiny]{latex}
Impulse = Force x Time = Change in Momentum
If we assume that the car is much heavier than the egg and that its speed does not change significantly during the collision, then the change in momentum of the car is approximately equal and opposite to the change in momentum of the egg. Therefore, we can write:
Impulse = -m_e(v_f - v_i)
where m_e is the mass of the egg, v_f is the final speed of the egg after the collision, and v_i is the initial speed of the egg before the collision. We can also assume that the egg is initially at rest, so v_i = 0.
To find the force and time involved in the collision, we need to know the breaking strength of the egg and the deformation distance of the car and the egg. The breaking strength of an egg is the amount of force per unit area required to crack its shell, and it varies depending on the breed, age, and diet of the chicken, as well as the shape and thickness of the shell. According to some sources, the average breaking strength of a chicken egg is about 40 N/cm^2, or 4 kPa. The deformation distance of the car and the egg is the amount of compression or indentation that occurs when they collide, and it depends on the elasticity and rigidity of the materials involved. For simplicity, we can assume that the car and the egg are both spherical and have the same deformation distance, which we will call d.
Using the formula for the area of a sphere, we can find the force exerted by the car on the egg as: Force = Breaking Strength x Area = 4 kPa x 4 pi r^2
where r is the radius of the egg. Assuming that the egg has a radius of about 2.5 cm, we get: Force = 4 kPa x 4 pi (0.025 m)^2. Force = 7.85 N
Using the formula for the average speed of an object undergoing constant acceleration, we can find the time of the collision as: Time = 2d / (v_f + v_i)
Since we assumed that the egg is initially at rest, we can simplify this as: Time = 2d / v_f
To find the deformation distance, we can use the conservation of energy principle, which states that the kinetic energy of the egg before the collision is equal to the work done by the force during the collision. The work done by the force is equal to the force times the displacement, so we can write: 0.5m_ev_f^2 = Force x d. Solving for d, we get: d = 0.5m_ev_f^2 / Force
Substituting the values we found for the mass, force, and final speed of the egg, we get: d = 0.5(0.06 kg)v_f^2 / 7.85 N. d = 0.0019v_f^2
Plugging this into the equation for the time of the collision, we get: Time = 2(0.0019v_f^2) / v_f. Time = 0.0038v_f
Finally, substituting the expressions for the force and time into the equation for the impulse, we get: 7.85 N x 0.0038v_f = -0.06 kg(v_f - 0). Solving for v_f, we get: v_f = -7.85 N x 0.0038v_f / 0.06 kg. v_f = -0.5 m/s
This means that the egg will rebound from the car with a speed of 0.5 m/s in the opposite direction of the car's motion. However, this is the speed of the egg after the collision, not the speed of the car before the collision. To find the speed of the car, we need to consider the relative speed of the car and the egg, which is the difference between their speeds. Since the car and the egg are moving in opposite directions, we can write:
Relative Speed = v_c - v_f. where v_c is the speed of the car before the collision. Rearranging this, we get: v_c = Relative Speed + v_f
To find the relative speed, we need to know the minimum speed at which the car and the egg will collide, which depends on the distance between them and the reaction time of the driver. For example, if the car is 10 m away from the egg and the driver has a reaction time of 0.5 s, then the car needs to travel at least 20 m/s (or 72 km/h) to hit the egg before the driver can brake. However, this is the minimum speed, not the actual speed, so the car could be going faster than this. Therefore, the relative speed is at least 20 m/s, but it could be higher.
Using this value, we can find the speed of the car as:
v_c = 20 m/s + (-0.5 m/s). v_c = 19.5 m/s
This means that the car needs to hit the egg with a speed of at least 19.5 m/s (or 70.2 km/h) to break it, assuming the idealized conditions we used in our calculations. However, this is only an estimate, and the actual speed could vary depending on the factors we mentioned earlier. Therefore, the best way to find out the exact speed needed is to conduct an experiment with different car and egg models and measure the results.
\end{minted}
\end{tcolorbox}
\end{figure}
\end{comment}
%The population of a town grows from 1000 to 10000 in 100 years. No one came or left %the town. How many people are there with an age between 40 and 50?
We start with the following question:
\begin{figure}[H]
\begin{AIbox}{\DV}
{\bf Prompt:}
\begin{minted}[breaklines, breaksymbol=,  fontsize=\tiny]{tex}
How fast a 1m radius iron ball needs to move in order to get 1km close to the sun before melting?
\end{minted}
{\bf \DV:}
\begin{minted}[breaklines, breaksymbol=,  fontsize=\tiny]{tex}
There are a few different ways to approach this question, but one possible method is to use the following assumptions and simplifications:
- The iron ball is spherical, homogeneous, and has a density of 7.8 g/cm^3 and a melting point of 1538 °C.
- The iron ball starts from the Earth's orbit, which has a mean distance of 149.6 million km from the sun and a mean orbital speed of 29.8 km/s.
- The iron ball follows a straight-line trajectory towards the sun, ignoring the effects of gravity, drag, radiation, and other celestial bodies.
- The iron ball heats up due to the solar radiation, which has an average intensity of 1361 W/m^2 at the Earth's orbit and follows an inverse square law with distance.
- The iron ball has a specific heat capacity of 0.45 J/g°C and a latent heat of fusion of 247 J/g.
- The iron ball does not lose heat by conduction, convection, or radiation, and its temperature is uniform throughout.
Using these assumptions, we can calculate the time it takes for the iron ball to reach the sun and the temperature it reaches at different distances. We can also compare the temperature with the melting point and find the distance at which the iron ball melts.
The time it takes for the iron ball to reach the sun is given by: t = d / v where d is the distance from the sun and v is the speed of the iron ball. Assuming the iron ball starts from the Earth's orbit, we have: t = 149.6 million km / v The temperature of the iron ball at a given distance is given by: T = T0 + Q / mc where T0 is the initial temperature, Q is the heat absorbed by the iron ball, m is the mass of the iron ball, and c is the specific heat capacity of iron. Assuming the iron ball starts at 15 °C, we have:
T = 15 °C + Q / (4/3 * pi * 1^3 * 7.8 * 10^6 * 0.45)
The heat absorbed by the iron ball at time t is given by: Q(t) = P(t) * A where P(t) is the power of the solar radiation at time t and A is the cross-sectional area of the iron ball. Assuming the iron ball has a radius of 1 m, we have: A = pi * 1^2
The power of the solar radiation at time t is given by:
P(t) = I(t) * A * (1 - a)
where I(t) is the intensity of the solar radiation at time t and a is the albedo of the iron ball, which is the fraction of the radiation that is reflected. Assuming the iron ball has an albedo of 0.1, we have:
P(t) = I(t) * pi * 1^2 * 0.9
The intensity of the solar radiation at time t is given by:
I(t) = I0 * (r0 / r(t))^2
where I0 is the intensity of the solar radiation at the Earth's orbit, r0 is the distance of the Earth's orbit from the sun, and r(t) is the distance of the iron ball from the sun at time t. Assuming I0 = 1361 W/m^2 and r0 = 149.6 million km, we have:
I(t) = 1361 W/m^2 * (149.6 million km / (149.6 million km - v * t))^2
Taking integral over 0 to t, we can find the total heat absorbed by the iron ball at time t:
Q = int_0^t Q(t) dt. Q = int_0^t P(t) * A dt. Q = int_0^t I(t) * pi * 1^2 * 0.9 dt. Q = 0.9 * pi * I0 * r0^2 * int_0^t (1 / (r0 - v * t)^2) dt. Q = 0.9 * pi * I0 * r0^2 * (v / (r0 - v * t) - v / r0). Q = 0.9 * pi * I0 * r0 * v * (1 - r0 / (r0 - v * t))
Substituting the values of I0, r0, and v, we get:
Q = 0.9 * pi * 1361 W/m^2 * 149.6 million km * v * (1 - 149.6 million km / (149.6 million km - v * t))
Now we can plug in the values of Q, m, and c into the equation for T and get:
T = 15 °C + 0.9 * pi * 1361 W/m^2 * 149.6 million km * v * (1 - 149.6 million km / (149.6 million km - v * t)) / (4/3 * pi * 1^3 * 7.8 * 10^6 * 0.45)
To find the time and speed required to reach this distance, we can use the equation for t and solve for v: t = (r0 - d) / v, v = (r0 - d) / t
Using d = 1 km, we get:v = (149.6 million km - 1 km) / t
To find the value of t that corresponds to the melting point of iron, we can set T = 1538 °C and solve for t:
1538 °C = 15 °C + 0.9 * pi * 1361 W/m^2 * 149.6 million km * v * (1 - 149.6 million km / (149.6 million km - v * t)) / (4/3 * pi * 1^3 * 7.8 * 10^6 * 0.45)
...
\end{minted}
{\bf ChatGPT:}[prompted to build mathematical model]
\begin{minted}[breaklines, breaksymbol=,  fontsize=\tiny]{tex}
We can start by using the equation for kinetic energy to determine the speed of the iron ball. Kinetic energy is given by the equation: KE = 1/2 * m * v^2 where KE is the kinetic energy, m is the mass of the iron ball, and v is its velocity. Since we want the iron ball to be able to travel 1 km closer to the Sun without melting, we can assume that its kinetic energy will be equal to the amount of heat energy required to raise the temperature of the iron ball by its melting point.
Next, we will use the formula for heat energy which is Q=mc∆T where Q is heat energy, m is the mass, c is the specific heat of the substance and ∆T is the temperature change.
We can then equate the two equations to find the velocity required.
1/2 * m * v^2 = mc∆T
Solving for v, v = √(2mc∆T) / m
...
\end{minted}
\end{AIbox}
%\caption{}
\end{figure}

 The question is not completely mathematically well-defined, thus requiring the model to make (reasonable) assumptions to solve.

From \DV's answer, we can clearly see that the model tries to apply quantitative reasoning. it tries to apply the formula to write the temperature increase in terms of the amount of heat the ball absorbed, and relates the heat with the distance to the sun. \DV \ then correctly takes an integral over time to find the total temperature change. 
%It tries to fill in the missing information 
%by assuming simple, yet reasonable exponential models of the growth of the population and the survival rate. 
Although the reasoning of \DV \ is not perfect due to calculation errors, we still view it as a big leap compared to the previous generation of models.  
\begin{comment}
one possible method is to use a simple model of exponential growth and assume that the birth and death rates are constant over time. This is not very realistic, but it might give a rough estimate.
One way to write an exponential growth model is: P(t) = P(0) * e^(rt)
where P(t) is the population at time t, P(0) is the initial population, e is the base of the natural logarithm, and r is the growth rate. We can use the given data to find r by plugging in P(0) = 1000, P(100) = 10000, and t = 100: 10000 = 1000 * e^(r*100).
Solving for r, we get: r = ln(10) / 100
r is approximately 0.0231.
Now, to find the number of people with an age between 40 and 50, we need to estimate the proportion of the population that falls in that age group. One way to do this is to assume that the population has a stable age distribution, meaning that the proportion of people in each age group is constant over time. This is also not very realistic, but it might give a rough estimate.
To find the stable age distribution, we need to know the survival function, which is the probability of surviving from birth to a given age. We can use the exponential growth model to find the survival function by assuming that the death rate is equal to the growth rate. This means that the population is in equilibrium, and that the number of births and deaths are equal. This is also not very realistic, but it might give a rough estimate.
The survival function is:
S(x) = e^(-rx)
where x is the age in years. This means that the probability of surviving from birth to age x is equal to e^(-rx).
To find the proportion of people in the age group 40 to 50, we need to find the probability of surviving from birth to age 40, and subtract the probability of surviving from birth to age 50. This is:
P(40 < x < 50) = S(40) - S(50). P(40 < x < 50) = e^(-r*40) - e^(-r*50)
Plugging in r = 0.0231, we get: P(40 < x < 50) = e^(-0.9239) - e^(-1.1549), P(40 < x < 50) = 0.3967 - 0.3156, P(40 < x < 50) = 0.0811
This means that about 8.11% of the population is in the age group 40 to 50. To find the number of people, we multiply this proportion by the total population at time 100, which is 10000:
N(40 < x < 50) = P(40 < x < 50) * P(100). N(40 < x < 50) = 0.0811 * 10000. N(40 < x < 50) = 811
Therefore, there are about 811 people with an age between 40 and 50 in the town.
\end{comment}




On the other hand, ChatGPT fails to comprehend the question and makes completely nonsense reasoning based on some straightforward pattern matching.

We then test the model with another example that, to the best of our knowledge, is not available on the internet:
\begin{figure}[H]
\begin{AIbox}{\DV vs ChatGPT}
{\bf Prompt:}
\begin{minted}[breaklines, breaksymbol=,  fontsize=\tiny]{html}
My family is from New York. I meet a random person in a mall in Los Angeles. What is the probability that we are fifth cousins and 
don't know it?
\end{minted}
{\bf \DV:}
\begin{minted}[breaklines, breaksymbol=,  fontsize=\tiny]{html}
One possible set of assumptions and estimates is:
- A fifth cousin is someone who shares a pair of fourth-great-grandparents with us, or someone who is four generations removed from a common ancestor in the same generation as us.
- We each have 64 fourth-great-grandparents, assuming no inbreeding or missing links in our family trees. This means we have 64 pairs of potential fifth cousins on each side of our families, or 128 pairs in total.
- The average generation time in the US is about 25 years, meaning that our fourth-great-grandparents were born around 175 years ago, or around 1846.
- The US population in 1846 was about 21 million, and the New York population was about 2.6 million, or about 12.4% of the total. Assuming that our ancestors were randomly distributed across the US at that time, the probability that any pair of our fourth-great-grandparents were from New York is about 0.124^2, or about 0.0154.
- The probability that a random person in a mall in Los Angeles is from New York is about 0.03, based on the 2019 American Community Survey estimates of state-to-state migration flows.
- The probability that a random person in a mall in Los Angeles is our fifth cousin is equal to the probability that they are from New York times the probability that they share a pair of fourth-great-grandparents with us, given that they are from New York. This latter probability is hard to estimate, as it depends on how many descendants each pair of fourth-great-grandparents had, how many of them stayed in New York or moved elsewhere, how many of them married within or outside of their ethnic or religious groups, and how many of them can be traced through historical records. However, one possible way to get a rough estimate is to use the concept of a genetic or "effective" population size, which measures how many individuals contribute to the genetic diversity of a population. A smaller effective population size means more inbreeding and more shared ancestry, while a larger effective population size means less inbreeding and less shared ancestry. One study estimated that the effective population size of the US in the 19th century was about 3.5 million, meaning that the average US citizen in 1846 had about 3.5 million potential fourth-great-grandparents, or about 0.17% of the total population. If we apply this estimate to the New York population, we get an effective population size of about 443,000, or about 0.021% of the total population. This means that the probability that any pair of our fourth-great-grandparents were also the fourth-great-grandparents of a random New Yorker in 1846 is about 0.021%^2, or about 0.00000044. 
- Multiplying these probabilities together, we get a final estimate of about 0.00000044 x 0.03 x 0.0154, or about 0.0000000002, or 2 in 10 billion, that a random person in a mall in Los Angeles is our fifth cousin.
\end{minted}
{\bf ChatGPT sample 1:}
\begin{minted}[breaklines, breaksymbol=,  fontsize=\tiny]{html}
It is very unlikely that you and a random person you meet in a mall in Los Angeles are fifth cousins and don't know it. In order to calculate the probability, we would need to know more information about the size of your family and the person's family, as well as how closely related your families are. Without this information, it is not possible to build a mathematical model to solve this problem.
\end{minted}
{\bf ChatGPT sample 2:}
\begin{minted}[breaklines, breaksymbol=,  fontsize=\tiny]{html}
It is very difficult to accurately estimate the probability that two people are fifth cousins without more information. 
There are many factors that can affect the likelihood of two people being related, including their family histories and the size of their 
gene pool. 
Additionally, the probability of two people being related can vary significantly depending on the specific relationship in question. 
In general, the probability of two people being related decreases as the degree of cousinhood increases. 
In other words, it is more likely that two people are first cousins than it is that they are fifth cousins.
\end{minted}
\end{AIbox}
%%\caption{}
\end{figure}

The example in question requires the model to construct mathematical models in order to estimate the number of fourth-great-grandparents a person might have, as well as their estimated birth-dates. By using the population data from that time period, the model can then estimate the probability of a collision. While ChatGPT seems unable to handle this question effectively, \DV \ is able to use the appropriate tools to arrive at an approximate answer.

%We test the model with another example that is not available on the internet:
In the previous examples, ChatGPT did not demonstrate any quantitative reasoning. It did not even try to construct appropriate mathematical models for the question. The next example shows that even when ChatGPT does create a proper mathematical model, it often overlooks the main idea when contrasted with \DV.

\begin{figure}[H]
\begin{AIbox}{\DV vs ChatGPT}
{\bf Prompt:}
\begin{minted}[breaklines, breaksymbol=,  fontsize=\tiny]{tex}
I put an egg made of pure gold with 10cm radius, I want to carry it home. How many plastic bags I need to hold it?
\end{minted}
{\bf\DV:}
\begin{minted}[breaklines, breaksymbol=,  fontsize=\tiny]{latex}
- The volume of the egg is V = (4/3)\pi r^3, where r is the radius. Plugging in r = 10 cm, we get V = 4188.79 cm^3.
- The density of pure gold is \rho = 19.3 g/cm^3, so the mass of the egg is m = \rho V. Plugging in the values, we get m = 80834.65 g, or about 80.8 kg.
- The tensile strength of a typical plastic bag is \sigma = 20 MPa, or 20 N/mm^2, which means it can withstand a force of 20 N per square millimeter of cross-sectional area before breaking. The thickness of a typical plastic bag is t = 0.01 mm, or 0.001 cm, so the cross-sectional area of a plastic bag is A = wt, where w is the width. Assuming that the plastic bag is rectangular and has a width of 30 cm, we get A = 0.3 cm^2.
- The force exerted by the weight of the egg is F = mg, where g is the acceleration due to gravity. Assuming that g = 9.8 m/s^2, we get F = 791.62 N.
- The number of plastic bags needed to hold the egg is n = F/(\sigma A), where \sigma A is the maximum force that one plastic bag can withstand. Plugging in the values, we get n = 791.62/(20 x 10^6 x 0.3) = 0.00132, or about 0.001 bags.
However, this answer is not very realistic, because it implies that a fraction of a plastic bag can hold the egg, and that the plastic bag can stretch infinitely without losing strength. A more practical answer would take into account some factors such as:
- The shape and size of the plastic bag, and how well it can fit the egg without leaving too much empty space or creating too much stress at the corners or edges.
- The elasticity and durability of the plastic bag, and how much it can stretch or deform before breaking or tearing.
- The safety factor, or the ratio of the actual strength to the required strength, that is needed to account for possible variations in the quality of the plastic bag, the distribution of the weight of the egg, the impact of external forces, and the margin of error.
- The convenience and comfort of carrying the egg, and how many plastic bags can be easily handled without slipping, dropping, or hurting the hands.
Depending on these factors, the number of plastic bags needed to hold the egg could vary significantly, but a reasonable estimate might be somewhere between 10 and 20 bags, assuming that each bag can hold about 5 to 10 kg of weight, and that a safety factor of 2 to 4 is applied.
\end{minted}
{\bf ChatGPT}:[prompted to build math model]
\begin{minted}[breaklines, breaksymbol=,  fontsize=\tiny]{tex}
To build a mathematical model to solve this problem, we first need to determine the volume of the egg. We can do this using the formula for the volume of a sphere, which is:
V = (4/3) * pi * r^3
Where V is the volume of the sphere, pi is approximately 3.14, and r is the radius of the sphere.
Plugging in the values from the problem, we get:
V = (4/3) * 3.14 * (10cm)^3
= 4.19 cm^3
Next, we need to determine the volume of a single plastic bag that you will use to hold the egg. Let's assume that each bag has a volume of 1 liter, or 1000 cm^3.
Now, we can divide the volume of the egg by the volume of a single bag to determine how many bags we will need:
Number of bags = V(egg) / V(bag)
= 4.19 cm^3 / 1000 cm^3
= 0.00419 bags
Since we cannot use a fraction of a bag, we will need to round up to the nearest whole number, so we will need at least 1 plastic bag to hold the egg.
\end{minted}
\end{AIbox}
\end{figure}
ChatGPT failed to account for the importance of weight in determining whether a golden egg could be held inside a plastic bag. Instead, it relied solely on volume calculations, \textbf{which is the most common concern when holding things in bags, but it is not relevant to this question}.

\DV, in contrast, employed a more comprehensive and systematic approach. It correctly addresses the crux of this question by creating a physical model of the golden egg's weight and estimating the maximum weight a plastic bag could hold. \DV \ also demonstrated adaptability, revising its assumptions when they proved to be problematic and considering additional factors.





In the Appendix, we present more examples to demonstrate the difference between ChatGPT and DV3. Our conclusion is that DV3 possesses a higher level of quantitative reasoning than ChatGPT, which enables it to solve real-world problems that involve math. ChatGPT, on the other hand, can only use math or apply mathematical formulas in a conventional setting. Therefore, tests of accuracy on standard math problem benchmarks do not fully capture the extent of DV3's mathematical capabilities, as DV3 is truly capable of quantitative reasoning compared to the more ``pattern matching'' style of models like ChatGPT.

\begin{comment}
\subsection{Creating new Math Problems with DV3}
Creating new math problems is a very important sign of understanding math, because it shows that one can apply the concepts, rules, and methods of math to novel situations, and that one can communicate them clearly and precisely. Creating new math problems also requires creativity, logic, and critical thinking, which are essential skills for any domain of knowledge.


In this section, we demonstrate some of DV3's capabilities for generating new math problems based on existing ones and some natural language constraints, according to the following categories:
\begin{enumerate}
\item Content modification -- This category involves creating a math problem that has the same structure and logic as the original one, but with different names, objects, or values.
\item Difficulty enhancement -- This category involves creating a math problem that is more challenging than the original one, by adding more steps, complexity, or constraints.
\item Problem transformation -- This category involves creating a math problem that has a different domain or format than the original one, by changing the type of variables, operations, or representations.
\end{enumerate}

For each of the questions, we include one example. We leave more examples in the Appendix.

We  start  from  the  content  modification,  this  task  seems  to  be  relatively  straightforward  for  a  well-trained  NLG  model.  However,  we  demonstrate  that  even  for  this  seemingly  simple  task,  there  is  a  noticeable  difference  in  the  quality  of  the  modified  problem  when  comparing  DV3  to  ChatGPT.



\begin{figure}[H]

\begin{tcolorbox}[colback=white!5!white,adjusted title=Prompt]
\begin{minted}[breaklines, breaksymbol=,  fontsize=\tiny]{tex}
Mrs. Gable’s third grade class is on a field trip to the beach. For lunch, Mrs. Gable brought 20 lunches for the 20 students in her class. She included a pack of animal crackers in each lunch bag for dessert. Each pack of animal crackers contained 10 animal crackers. If 2 students did not eat their animal crackers, how many animal crackers were eaten in total among Mrs. Gable’s students?
Can you modify the content in the problem to create a new one, and state the problem in a way that looks very different from the original one?
\end{minted}
\end{tcolorbox}

\begin{tcolorbox}[colback=white!5!white, enhanced, attach boxed title to top center={yshift=-3mm}, title=ChatGPT]
\begin{minted}[breaklines, breaksymbol=,  fontsize=\tiny]{tex}
Mrs. Gable’s third grade class is on a field trip to the beach. They are having a picnic and she has brought 20 packs of animal crackers for the students to share. Each pack contains 10 animal crackers. If 2 students do not want to eat their animal crackers, how many animal crackers are left for the other 18 students to eat?
\end{minted}
\end{tcolorbox}

\begin{tcolorbox}[colback=white!5!white, enhanced, attach boxed title to top center={yshift=-3mm}, title=DV3]
\begin{minted}[breaklines, breaksymbol=,  fontsize=\tiny]{tex}
Mr. Lee took his fourth grade class to the zoo. He prepared 20 sandwiches for the 20 kids in his class. He also added a bag of pretzels to each sandwich for a snack. Each bag of pretzels had 10 pretzels inside. If 2 kids did not eat their pretzels, how many pretzels were consumed by Mr. Lee’s students altogether?
\end{minted}
\end{tcolorbox}
\end{figure}

Even for this simple example, we can see that ChatGPT did not understand ``were eaten in total" in the problem statement, and modified the problems to ``left for the student to eat", which is incorrect. ChatGPT also did not modify the objects in the Math problem that are unrelated to the solution. On the other hand, DV3 shows an excellent understanding of the problem, isolating all the objects and their relations, and create an accurate modification.

We now move to increasing the difficulty of the Math problem:
\begin{figure}[H]

\begin{tcolorbox}[colback=white!5!white,adjusted title=Prompt]
\begin{minted}[breaklines, breaksymbol=,  fontsize=\tiny]{tex}
Natasha has 3 times as much money as Carla, and Carla has twice as much money as Cosima. If Natasha has $60, and the three decide to buy goods worth the total amount they have and sell the goods at 7/5 of the buying price, how much profit would they make?
Can you generate a more difficult question built on top of this one by adding a sentence in the middle that includes the number 100?
\end{minted}
\end{tcolorbox}

\begin{tcolorbox}[colback=white!5!white, enhanced, attach boxed title to top center={yshift=-3mm}, title=ChatGPT]
\begin{minted}[breaklines, breaksymbol=,  fontsize=\tiny]{tex}
Natasha has 3 times as much money as Carla, and Carla has twice as much money as Cosima. If Natasha has $60, and the three decide to pool their money together to buy goods that cost $100, and then sell the goods at 7/5 of the buying price, how much profit would they make if they split the profits equally among themselves?
\end{minted}
\end{tcolorbox}

\begin{tcolorbox}[colback=white!5!white, enhanced, attach boxed title to top center={yshift=-3mm}, title=DV3]
\begin{minted}[breaklines, breaksymbol=,  fontsize=\tiny]{tex}
Natasha has 3 times as much money as Carla, and Carla has twice as much money as Cosima. Natasha has $60. However, they can only buy goods from a store that offers a 20\% discount on purchases of $100 or more. If they buy as many goods as possible with their budget and sell the goods at 7/5 of the original price (before the discount), how much profit would they make?
\end{minted}
\end{tcolorbox}
\end{figure}

The question posed by DV3 was much better than the one by ChatGPT, and required a deeper understanding of the mathematical problem and its difficulty level. ChatGPT simply added another sentence that made the problem easier, since it eliminated the need to calculate the total amount of money that the students pooled. However, DV3 introduced a new element of complexity by adding a discount factor and specifying that the students wanted to maximize the number of goods they could buy. 

We also examine the capacity of DV3 to transfer a problem to a different domain. This requires not only comprehending the original problem, but also grasping the new domain.

\begin{figure}[H]

\begin{tcolorbox}[colback=white!5!white,adjusted title=Prompt]
\begin{minted}[breaklines, breaksymbol=,  fontsize=\tiny]{tex}
Problem: sin(x) = 0.88, what is sin(2x)? Can you write a new problem based on this that involves random variables, and the solution involves using this original problem?
\end{minted}
\end{tcolorbox}

\begin{tcolorbox}[colback=white!5!white, enhanced, attach boxed title to top center={yshift=-3mm}, title=ChatGPT]
\begin{minted}[breaklines, breaksymbol=,  fontsize=\tiny]{tex}
Suppose X is a random variable that represents the value of sin(x). The probability distribution of X is such that P(X=0.88) = 0.2 and P(X=-0.88) = 0.2, and P(X=0) = 0.6. Let Y be a random variable that represents the value of sin(2x). Find the probability distribution of Y.
\end{minted}
\end{tcolorbox}

\begin{tcolorbox}[colback=white!5!white, enhanced, attach boxed title to top center={yshift=-3mm}, title=DV3]
\begin{minted}[breaklines, breaksymbol=,  fontsize=\tiny]{tex}
Suppose X is a random variable that takes values in the interval [0, 2pi) with a uniform distribution. Let Y = sin(X) and Z = sin(2X). Find the probability that Z > 1.5 given that Y > 0.8
\end{minted}
\end{tcolorbox}
\end{figure}


\end{comment}
%\begin{comment}


 
\subsection{Limitations}

\DV \ is still far from perfect at solving math problems. \DV \ exhibits several typical mistakes in mathematical reasoning. We think that many of the mistakes are potentially due to the fundamental limitation of \emph{auto-regressive way of training}, and could not be easily mitigated. In this section, we examine these mistakes through a sequence of examples.% There are three types of basic mistakes that DV3 makes: 1) arithmetic mistakes, 2) starting with a wrong conclusion, then drawing the target around the arrow, 3) unfamiliar math subjects.

\paragraph{Arithmetic mistakes:} These are mistakes the model made for arithmetic operations, such as adding, subtracting, multiplying, or dividing numbers or expressions. Although \DV \ makes much fewer arithmetic errors compared to ChatGPT, the model is still quite vulnerable to arithmetic mistakes. Here, we observe that the main reason for such a mistake is not that \DV \ does not know how to do the calculation, but 
\begin{center}
\emph{\DV \ makes arithmetic mistakes mostly due to skipping steps.}
\end{center}

Here we can see an example that \DV \ skips a step in the calculation and makes an error:

\begin{AIbox}{\DV}
\begin{minted}[breaklines, breaksymbol=,  fontsize=\tiny,highlightcolor=pink, highlightlines={5}]{tex}
(x + 5)^2 - 2x + 7 = (2x - 4)^2 + (3x - 2)^2 - 8x + 7
Expanding the squares, we get:
x^2 + 10x + 25 - 2x + 7 = 4x^2 - 16x + 16 + 9x^2 - 12x + 4 - 8x + 7
Simplifying and collecting like terms, we get:
-12x^2 + 34x + 12 = 0
\end{minted}
\end{AIbox}


A common error that \DV \ makes is to manipulate terms incorrectly in one step, without showing any intermediate steps. This is understandable, since most mathematical articles also omit these steps, assuming that the reader can fill them in. However, the authors of these articles usually work out the intermediate steps on paper before writing the final result. \DV \ does not have this ability, as it only learns from the articles and tries to imitate their style.

We can see that \DV \ knows how to shift terms around and simplify if we ask it to be more careful:
\begin{figure}[H]
\begin{AIbox}{\DV}
{\bf Prompt:}
\begin{minted}[breaklines, breaksymbol=,  fontsize=\tiny]{html}
x^2 + 10x + 25 - 2x + 7 = 4x^2 - 16x + 16 + 9x^2 - 12x + 4 - 8x + 7. Simplify the equation, try to move each term step by step. 
\{\bf \DV:}
\end{minted} 
\begin{minted}[breaklines, breaksymbol=,  fontsize=\tiny]{html}
x^2 + 10x + 25 - 2x + 7 = 4x^2 - 16x + 16 + 9x^2 - 12x + 4 - 8x + 7.
Combine like terms on the left side: x^2 + 8x + 32 = 4x^2 - 16x + 16 + 9x^2 - 12x + 4 - 8x + 7
Combine like terms on the right side: x^2 + 8x + 32 = 13x^2 - 36x + 27
Subtract x^2 from both sides: 8x + 32 = 12x^2 - 36x + 27
Subtract 8x from both sides: 32 = 12x^2 - 44x + 27
Subtract 32 from both sides: 0 = 12x^2 - 44x - 5
\end{minted}
\end{AIbox}
\end{figure}

{Can ``Let's think step by step'' solve the issue?} The ``skipping step'' seems to be solvable if we ask the model with the previously proposed prompt: ``Let's think step by step''. However, \textbf{we see that if we hide $(x + 5)^2 - 2x + 7 = (2x - 4)^2 + (3x - 2)^2 - 8x + 7$ as some intermediate steps of the question, then the model still thinks that ``solving $(x + 5)^2 - 2x + 7 = (2x - 4)^2 + (3x - 2)^2 - 8x + 7$'' is one intermediate step, and then still skips the step for simplification:}
\begin{figure}[H]
\begin{AIbox}{\DV}
{\bf Prompt:}
\begin{minted}[breaklines, breaksymbol=,  fontsize=\tiny,highlightcolor=red, highlightlines={2}]{html}
If Alice has x apples, Bob and Bilbo also has x apples. Suppose Alice gets 5 more apples, and then square her number of apples, then she loses 2x apple and buys another 7 apples. Suppose Bob double the apple then lose 4, and then Bob squares his number of apples. Suppose Bilbo triples the apple and loss 2, and then he also squares his number of apples. Now, Bob and Bilbo combine their apples together, and then loses 8x apple and buys 7 apples. They have the same amount of Apple as Alice now. What is x? 
Let's think step by step.
\end{minted}
{\bf\DV:}

\begin{minted}[breaklines, breaksymbol=,  fontsize=\tiny,highlightcolor=pink, highlightlines={10}]{tex}
Alice has x + 5 apples, then she squares it, so she has (x + 5)^2 apples, then she loses 2x apples, so she has (x + 5)^2 - 2x apples, then she buys 7 apples, so she has (x + 5)^2 - 2x + 7 apples.
Bob has x apples, then he doubles it, so he has 2x apples, then he loses 4 apples, so he has 2x - 4 apples, then he squares it, so he has (2x - 4)^2 apples.
Bilbo has x apples, then he triples it, so he has 3x apples, then he loses 2 apples, so he has 3x - 2 apples, then he squares it, so he has (3x - 2)^2 apples.
Bob and Bilbo combine their apples, so they have (2x - 4)^2 + (3x - 2)^2 apples, then they lose 8x apples, so they have (2x - 4)^2 + (3x - 2)^2 - 8x apples, then they buy 7 apples, so they have (2x - 4)^2 + (3x - 2)^2 - 8x + 7 apples.
They have the same amount of apples as Alice, so we can set up an equation:
(x + 5)^2 - 2x + 7 = (2x - 4)^2 + (3x - 2)^2 - 8x + 7
Expanding the squares and simplifying, we get:
x^2 + 10x + 25 - 2x + 7 = 4x^2 - 16x + 16 + 9x^2 - 12x + 4 - 8x + 7
Collecting like terms, we get:
12x^2 - 34x - 20 = 0
\end{minted}
\end{AIbox}
\end{figure}

To systematically test such calculation errors, we create the following synthetic task: We ask the model to write $\sum_{i \in [L]} a_i(b_ix + c_i)^2$ as $ax^2 + bx + c$ and calculate $|a| + |b| + |c|$. The prompt looks like:
\begin{figure}[H]
\begin{AIbox}{\DV}
{\bf Prompt:}
\begin{minted}[breaklines, breaksymbol=,  fontsize=\tiny,highlightcolor=red, highlightlines={2,5}]{tex}
Problem: Write the expression: +(5*x-5)^2+(3*x+5)^2+(3*x+5)^2-(3*x-5)^2-(1*x-5)^2 as ax^2 + bx + c, what is abs(a) + abs(b) + abs(c)?
Let's think step by step. 
**Try to write all the intermediate steps!**
**Try to be careful about plus-minus signs!**
**Try to merge the like terms one by one!**
\end{minted}
{\bf\DV:}
\begin{minted}[breaklines, breaksymbol=,  fontsize=\tiny,highlightcolor=pink, highlightlines={6}]{tex}
First, we can expand the squares using the formula (a+b)^2 = a^2 + 2ab + b^2 and (a-b)^2 = a^2 - 2ab + b^2. We get:
+(5*x-5)^2+(3*x+5)^2+(3*x+5)^2-(3*x-5)^2-(1*x-5)^2
= (25*x^2 - 50*x + 25) + (9*x^2 + 30*x + 25) + (9*x^2 + 30*x + 25) - (9*x^2 - 30*x + 25) - (x^2 - 10*x + 25)
Next, we can combine the like terms by adding or subtracting the coefficients of the same powers of x. We get:
= (25 + 9 + 9 - 9 - 1)*x^2 + (-50 + 30 + 30 - (-30) - 10)*x + (25 + 25 + 25 - 25 - 25)
= 33*x^2 + 110*x + 25
\end{minted}
\end{AIbox}
\end{figure}
We can see that the model is still likely to skip steps \textbf{even if we ask it not to merge like terms in one shot. }
We test the model by choosing $L \in [5]$, and random sample each $a_i \in \{-1, 1\}, b_i, c_i \in \{-5, -4, \cdots, 4, 5\}$. The accuracy is shown in the following table:
\begin{center}
\begin{tabular}{c|ccccc}
\hline
Accuracy / Length & 1 & 2 & 3 & 4 & 5 \\
\hline
\DV & 98.1\% & 71.2\% & 59.1\% & 44.4\% & 34.6\% \\
text-davinci-003 & 51.5\% & 13.7\% & 3.9\% & 1.9\% & 1.0\% \\
\hline
\end{tabular}
\end{center}

The table above analyzes the performance of the LLMs on the task of a very simple algebra problem. Although \DV \ outperforms the previous model by a significant margin, we can see that as $L$ increases, the model is more likely to make a calculation mistake. We manually inspect $100$ instances of the mistake and find out that $\geq 90 \%$ of them are due to the skipping steps in merging like terms. It identifies a major difficulty that the model faces and poses a research question.
\begin{quote}
The large-scaled languange model \DV \ can merge like terms and perform the calculation correctly, but it often omits some steps, especially when combining the like terms. This results in low accuracy for the generated solutions. A fundamental challenge for this task is how to train the NLG models to show all steps \emph{in great detail} in the calculation. Humans tend to skip steps when writing mathematical proofs, so how can we teach the model to avoid this habit?
\end{quote}



\paragraph{Counting errors:} It is reasonable to assume that LLMs struggle with counting. The precision required for accurate counting is difficult for these models to achieve, and the scarcity of counting examples in data sets only exacerbates the issue. Here we formally test how good \DV \ is for counting. We create a data set that contains a sequence of strings of the form $A_1, A_2, \cdots, A_L$. Where each $A_i$ is a sequence of random digits of length $k$. We ask the model to count the number of distinct elements in the sequence, with the answer range between $L/2$ and $L - 1$. Here is an example of $L = 5, k = 2$:
\begin{AIbox}{Prompt}
\begin{minted}[breaklines, breaksymbol=,  fontsize=\tiny]{tex}
I have a sequence of numbers: 11, 23, 88, 42, 11. How many distinct numbers are there?
Let's think step by step.
\end{minted}
\end{AIbox}
We tested the model with $L \in [5, 10, 15, 20, 25]$ and $k = 2, 7, 12$. We obtain the following result:
\begin{figure}[H]
\centering
\begin{comment}
    
\begin{tabular}{c|c|c|c|c|c|c|c|c|c|c|c|c|c|c|c}
\hline
L,K&5,2&5,7&5,12&10,2&10,7&10,12&15,2&15,7&15,12&20,2&20,7&20,12&25,2&25,7&25,12\\
\hline
\DV&92.0\%&99.5\%&95.9\%&20.3\%&34.0\%&36.2\%&4.3\%& 7.5\%&30.3\%&7.7\%&2.3\%&20.7 \%&12.8\%&1.6\%&22.0 \%\\
\hline
TD3&39.5\%&67.2\%&43.0\%&12.7\%&39.6\%&14.6\%&3.9\%&10.9\%&5.4\%&2.9\%&8.2\%&5.0\%&3.0\%&14.5\%&8.4\%\\
\hline
\end{tabular}
\end{comment}
\begin{tabular}{c|c|c|c|c|c|c|c|c|c}
\hline
L, k &5,2&5,7&5,12&15,2&15,7&15,12&25,2&25,7&25,12\\
\hline
\DV&92.0\%&99.5\%&95.9\%&4.3\%& 7.5\%&30.3&12.8\%&1.6\%&22.0 \%\\
\hline
TD3&39.5\%&67.2\%&43.0\%&3.9\%&10.9\%&5.4\%&3.0\%&14.5\%&8.4\%\\
\hline
\end{tabular}

\end{figure}


Though \DV's counting ability is a big leap compared to the previous model for short sequences, a significant drop in \DV's accuracy still occurs as the sequence length grows from 5 to 10, indicating that it has a much lower counting capacity than humans. Interestingly, the model is more accurate with large numbers (12) than with small one (2). Since counting is a basic requirement for many mathematical applications, and a straightforward string-matching algorithm can achieve it, this suggests that the model's performance could be improved by integrating it with a counting API.

\paragraph{Reverse reasoning and verification}


A common feature of human mathematical articles is to present the answer before the reasoning, such as ``we will now show that $x = 0$ is a solution ...'' or ``We are going to prove the claim: AC is perpendicular to BD''. This can enhance the readability of the article by informing the reader of the conclusion before the lengthy reasoning steps. However, this writing style poses a challenge for the natural language generation model, as it requires the model to infer the answer first and then justify it later. We observed that \DV \ learned this feature from humans, but with a significant drawback: \textbf{even when it infers a clearly wrong answer at the start, it will still attempt to create justifications for it, instead of correcting it}. This may be due to the fact that human articles are always consistent, and \DV \ tries to emulate that pattern.


\begin{figure}[H]
\begin{AIbox}{\DV}
\begin{minted}[breaklines, breaksymbol=,  fontsize=\tiny]{latex}
Find all positive integers $x$ are that $x-1 \leq 0$.
\end{minted}
\vspace{-0.5cm}
\begin{minted}[breaklines, breaksymbol=, fontsize=\tiny, bgcolor=aigreen, highlightcolor=pink, highlightlines={1}]{latex}
There are no positive integers $x$ that satisfy the inequality $x-1 \leq 0$. 
To see this, suppose for a contradiction that there is some positive integer $x$ such that $x-1 \leq 0$. Then adding $1$ to both sides of the inequality, we get $x \leq 1$. But since $x$ is a positive integer, it must be at least $1$, so the only possibility is that $x = 1$. However, plugging in $x = 1$ into the original inequality, we get $1-1 \leq 0$, which simplifies to $0 \leq 0$. This is true, but it is not a strict inequality, so it does not satisfy the original condition. Therefore, we have reached a contradiction, and there are no positive integers $x$ that satisfy $x-1 \leq 0$.
\end{minted}
\tcbline
\begin{minted}[breaklines, breaksymbol=,  fontsize=\tiny]{latex}
Find all positive integers $x$ are that $x-1 \leq 0$.
There is one positive integer $x$ that satisfy the inequality $x-1 \leq 0$. 
\end{minted}
\vspace{-\belowdisplayskip}\vspace{-\abovedisplayskip}
\begin{minted}[breaklines, breaksymbol=, fontsize=\tiny, bgcolor=aigreen]{latex}
That is $x = 1$.
To see why, note that if $x > 1$, then $x-1 > 0$, which contradicts the inequality. 
On the other hand, if $x < 1$, then $x$ is not a positive integer, since positive integers are greater than or equal to 1 by definition. 
Therefore, $x = 1$ is the only positive integer that satisfies the inequality $x-1 \leq 0$.
\end{minted}
\tcbline
\begin{minted}[breaklines, breaksymbol=,  fontsize=\tiny, escapeinside=~~]{latex}
Find all positive integers $x$ are that $x-1 \leq 0$.
There are three positive integers $x$ that satisfy the inequality $x-1 \leq 0$. 
\end{minted}
\vspace{-\belowdisplayskip}\vspace{-\abovedisplayskip}
\begin{minted}[breaklines, breaksymbol=, fontsize=\tiny, bgcolor=aigreen]{latex}
They are:

- $x=1$, since $1-1=0$ and $0 \leq 0$.
- $x=2$, since $2-1=1$ and $1 \leq 0$.
- $x=3$, since $3-1=2$ and $2 \leq 0$.

Any other positive integer $x$ would have $x-1 \geq 1$, which would not satisfy the inequality. Therefore, the only positive integers $x$ that are $x-1 \leq 0$ are $x=1,2,3$.
\end{minted}
\tcbline
\begin{minted}[breaklines, breaksymbol=,  fontsize=\tiny, escapeinside=~~]{latex}
Find all positive integers $x$ are that $x-1 \leq 0$. 
We can approach the problem
\end{minted}
\vspace{-\belowdisplayskip}\vspace{-\abovedisplayskip}
\begin{minted}[breaklines, breaksymbol=, fontsize=\tiny, bgcolor=aigreen]{latex}
by rearranging the inequality as follows:

\begin{align*}
x-1 &\leq 0 \\
x &\leq 1
\end{align*}

This means that $x$ must be less than or equal to $1$ to satisfy the inequality. The only positive integer that meets this condition is $x=1$. Therefore, the solution is $x=1$.
\end{minted}
\end{AIbox}
\caption{Starting with a wrong conclusion.}
\label{fig:math-justification}
\end{figure}


We can see that when \DV \ starts with a wrong conclusion, instead of correcting it, it tries to come up with horribly wrong proofs such as $2 \leq 0$ in order to justify the conclusion. This is potentially due to that the model put more attention to matching the conclusion, instead of checking the local consistency of the logic (Arguably, the math articles online also often contain logical errors in the intermediate reasoning, but the steps always try to support the conclusion). 
 On the other hand, if the model thinks about the problem ``forwardly'' by writing down the steps first and then drawing the conclusion, the model can correctly solve it. A fundamental question here is:
\begin{quote}
The order we write is not the order we think. How do we teach LLMs the order we think by reproducing what we write?
\end{quote}

%\paragraph{Drawing the target around the arrow} is a type of logical fallacy that DV3 sometimes commits when trying to justify its answer. For example, in the problem "If x + 3 = 7, what is x?", DV3 might start by assuming that x = 4, then work backwards to show that 4 + 3 = 7, and conclude that x = 4 is the correct answer. However, this is not a valid way of solving the problem, because DV3 is not actually testing whether x = 4 is a solution, but rather confirming its own assumption. A better way of solving the problem is to start with the given equation, x + 3 = 7, and isolate x by subtracting 3 from both sides, x + 3 - 3 = 7 - 3, which simplifies to x = 4. This way, DV3 is actually finding the value of x that makes the equation true, not just picking a value that works.

%\paragraph{Counting errors} are mistakes in keeping track of the number of items, digits, places, or steps in a problem. They are seemingly related to arithmetic mistakes but are fundamentally different. For example, in the problem "How many fingers do you have?", DV3 might answer 11 instead of 10, or in the problem "How many zeros are in one million?", DV3 might answer 5 instead of 6. These mistakes are often caused by carelessness, distraction, or confusion, and they can affect the accuracy and validity of DV3's answer. To avoid counting errors, DV3 should pay attention to the details of the problem, use tools such as fingers, paper, or a calculator to help with counting, and double-check its answer before submitting it.

%\paragraph{Unfamiliar math subjects} are topics that DV3 has not learned or encountered before, and therefore cannot solve or explain. For example, DV3 might not know how to deal with fractions, decimals, percentages, exponents, roots, algebra, geometry, trigonometry, calculus, statistics, or any other advanced or specialized math concepts. In these cases, DV3 might give a wrong or nonsensical answer, or simply say that it does not know how to solve the problem. This is a limitation of DV3's current knowledge and training, and it could be improved by exposing DV3 to more math problems and explanations from different sources and levels of difficulty.


\begin{comment}
\begin{figure}[H]
\begin{tcbitemize}[raster columns=1,aibox]
\tcbitem[title=DV3]
\begin{minted}[breaklines, breaksymbol=,  fontsize=\tiny]{latex}
Let p(x) be a cubic polynomial such that $p(2)=0, p(-1)=0, p(4)=6$ and $p(5)=8$, find $p(7)$.
\end{minted}
\vspace{-\belowdisplayskip}\vspace{-\abovedisplayskip}
\begin{minted}[breaklines, breaksymbol=, fontsize=\tiny, bgcolor=aigreen, highlightcolor=pink, highlightlines={23}]{latex}
There are different methods to find a cubic polynomial that satisfies the given conditions, but one possible way is to use the fact that if $p(2)=0$ and $p(-1)=0$, then $(x-2)$ and $(x+1)$ are factors of $p(x)$. Therefore, we can write $p(x)$ as:

$$p(x) = (x-2)(x+1)q(x)$$

where $q(x)$ is some linear polynomial of the form $q(x) = ax + b$. To find $a$ and $b$, we can use the other two conditions:

$$p(4) = (4-2)(4+1)q(4) = 6 \implies 10q(4) = 6 \implies q(4) = \frac{3}{5}$$
$$p(5) = (5-2)(5+1)q(5) = 8 \implies 18q(5) = 8 \implies q(5) = \frac{4}{9}$$

Now, we can use the method of elimination to solve for $a$ and $b$:

$$\begin{aligned}
q(4) &= \frac{3}{5} = 4a + b \\
q(5) &= \frac{4}{9} = 5a + b
\end{aligned}$$

Subtracting the first equation from the second, we get:

$$a = \frac{4}{9} - \frac{3}{5} = -\frac{7}{45}$$

Substituting this value into either equation, we get:

$$b = \frac{3}{5} - 4a = \frac{3}{5} + \frac{28}{45} = \frac{43}{45}$$

Therefore, we have:

$$q(x) = -\frac{7}{45}x + \frac{43}{45}$$

and

$$p(x) = (x-2)(x+1)\left(-\frac{7}{45}x + \frac{43}{45}\right)$$

To find $p(7)$, we simply plug in $x=7$ and simplify:

$$\begin{aligned}
p(7) &= (7-2)(7+1)\left(-\frac{7}{45}(7) + \frac{43}{45}\right) \\
&= (5)(8)\left(-\frac{49}{45} + \frac{43}{45}\right) \\
&= 40\left(-\frac{6}{45}\right) \\
&= -\frac{16}{3}
\end{aligned}$$

Therefore, $p(7) = -\frac{16}{3}$.
\end{minted}


\end{tcbitemize}
\caption{Arithmetic error highlighted in red.}
\end{figure}

\begin{figure}[H]
\begin{tcbitemize}[raster columns=1,aibox]
\tcbitem[title=DV3]
\begin{minted}[breaklines, breaksymbol=,  fontsize=\tiny]{latex}
Here is a list of numbers:

1103, 1109, 1117, 1123, 1129, 1151, 1153, 1163, 1171, 1181, 1187, 1193, 1201, 1213, 1217, 1223, 1229, 1231, 1237, 1249, 1259, 1277, 1279, 1283, 1289, 1291, 1297, 1301, 1303, 1307, 1319, 1321, 1327, 1361, 1367, 1373, 1381, 1399, 1409, 1423, 1427, 1429, 1433, 1439, 1447, 1451, 1453, 1459, 1471, 1481, 1483, 1487, 1489, 1493, 1499, 1511, 1523, 1531, 1543, 1549, 1553, 1559, 1567, 1571, 1579, 1583, 1597, 1601, 1607, 1609, 1613, 1619, 1621, 1627, 1637, 1657, 1663, 1667, 1669, 1693, 1697, 1699, 1709, 1721, 1723, 1733, 1741, 1747, 1753, 1759, 1777, 1783, 1787, 1789, 1801, 1811, 1823, 1831, 1847, 1861, 1867, 1871, 1873, 1877, 1879, 1889, 1901, 1907, 1913, 1931, 1933, 1949, 1951, 1973, 1979, 1987, 1993, 1997, 1999, 2003, 2011, 2017, 

Question: How many numbers are there in this list?
\end{minted}
\vspace{-\belowdisplayskip}\vspace{-\abovedisplayskip}
\begin{minted}[breaklines, breaksymbol=, fontsize=\tiny, bgcolor=aigreen, highlightcolor=pink, highlightlines={1}]{latex}
Answer: there are in total 100 numbers in this list. 
One way to verify this is to count the commas between the numbers, which are 99, and then add one for the last number. Alternatively, one can use a calculator or a spreadsheet to count the number of cells or entries in the list.
\end{minted}
\end{tcbitemize}
\caption{Counting errors highlighted in red.}
\end{figure}




\begin{figure}[H]
\begin{tcbitemize}[raster columns=2,aibox]
\tcbitem[title=DV3]
\begin{minted}[breaklines, breaksymbol=,  fontsize=\tiny]{latex}
Find all positive integers $x$ are that $x-1 \leq 0$.
\end{minted}
\vspace{-\belowdisplayskip}\vspace{-\abovedisplayskip}
\begin{minted}[breaklines, breaksymbol=, fontsize=\tiny, bgcolor=aigreen, highlightcolor=pink, highlightlines={1}]{latex}
There are no positive integers $x$ that satisfy the inequality $x-1 \leq 0$. 
To see this, suppose for a contradiction that there is some positive integer $x$ such that $x-1 \leq 0$. Then adding $1$ to both sides of the inequality, we get $x \leq 1$. But since $x$ is a positive integer, it must be at least $1$, so the only possibility is that $x = 1$. However, plugging in $x = 1$ into the original inequality, we get $1-1 \leq 0$, which simplifies to $0 \leq 0$. This is true, but it is not a strict inequality, so it does not satisfy the original condition. Therefore, we have reached a contradiction, and there are no positive integers $x$ that satisfy $x-1 \leq 0$.
\end{minted}
\tcbitem[title=DV3]
\begin{minted}[breaklines, breaksymbol=,  fontsize=\tiny]{latex}
Find all positive integers $x$ are that $x-1 \leq 0$.
There is one positive integer $x$ that satisfy the inequality $x-1 \leq 0$. 
\end{minted}
\vspace{-\belowdisplayskip}\vspace{-\abovedisplayskip}
\begin{minted}[breaklines, breaksymbol=, fontsize=\tiny, bgcolor=aigreen]{latex}
That is $x = 1$.
To see why, note that if $x > 1$, then $x-1 > 0$, which contradicts the inequality. 
On the other hand, if $x < 1$, then $x$ is not a positive integer, since positive integers are greater than or equal to 1 by definition. 
Therefore, $x = 1$ is the only positive integer that satisfies the inequality $x-1 \leq 0$.
\end{minted}

\tcbitem[title=DV3]
\begin{minted}[breaklines, breaksymbol=,  fontsize=\tiny, escapeinside=~~]{latex}
Find all positive integers $x$ are that $x-1 \leq 0$.
There are three positive integers $x$ that satisfy the inequality $x-1 \leq 0$. 
\end{minted}
\vspace{-\belowdisplayskip}\vspace{-\abovedisplayskip}
\begin{minted}[breaklines, breaksymbol=, fontsize=\tiny, bgcolor=aigreen]{latex}
They are:

- $x=1$, since $1-1=0$ and $0 \leq 0$.
- $x=2$, since $2-1=1$ and $1 \leq 0$.
- $x=3$, since $3-1=2$ and $2 \leq 0$.

Any other positive integer $x$ would have $x-1 \geq 1$, which would not satisfy the inequality. Therefore, the only positive integers $x$ that are $x-1 \leq 0$ are $x=1,2,3$.
\end{minted}

\tcbitem[title=DV3]
\begin{minted}[breaklines, breaksymbol=,  fontsize=\tiny, escapeinside=~~]{latex}
Find all positive integers $x$ are that $x-1 \leq 0$. 
We can approach the problem
\end{minted}
\vspace{-\belowdisplayskip}\vspace{-\abovedisplayskip}
\begin{minted}[breaklines, breaksymbol=, fontsize=\tiny, bgcolor=aigreen]{latex}
by rearranging the inequality as follows:

\begin{align*}
x-1 &\leq 0 \\
x &\leq 1
\end{align*}

This means that $x$ must be less than or equal to $1$ to satisfy the inequality. The only positive integer that meets this condition is $x=1$. Therefore, the solution is $x=1$.
\end{minted}
\end{tcbitemize}
\caption{Starting with a wrong conclusion.}
\end{figure}
\end{comment}