% !TEX root = scott-1979-cmr.tex
%\usepackage[dotinlabels]{titletoc}
%\titlelabel{{\thetitle}.\quad}
%\usepackage{titletoc}
\usepackage[small]{titlesec}

\titleformat{\section}[block]
  {\fillast\medskip}
  {\bfseries{\thesection. }}
  {1ex minus .1ex}
  {\bfseries}
 
\titleformat*{\subsection}{\itshape}
\titleformat*{\subsubsection}{\itshape}

\setcounter{tocdepth}{2}

\titlecontents{section}
              [2.3em] 
              {\bigskip}
              {{\contentslabel{2.3em}}}
              {\hspace*{-2.3em}}
              {\titlerule*[1pc]{}\contentspage}
              
\titlecontents{subsection}
              [4.7em] 
              {}
              {{\contentslabel{2.3em}}}
              {\hspace*{-2.3em}}
              {\titlerule*[.5pc]{}\contentspage}

% hopefully not used.           
\titlecontents{subsubsection}
              [7.9em]
              {}
              {{\contentslabel{3.3em}}}
              {\hspace*{-3.3em}}
              {\titlerule*[.5pc]{}\contentspage}
%\makeatletter
\renewcommand\tableofcontents{%
    \section*{\contentsname
        \@mkboth{%
           \MakeLowercase\contentsname}{\MakeLowercase\contentsname}}%
    \@starttoc{toc}%
    }
\def\@oddhead{{\scshape\rightmark}\hfil{\small\scshape\thepage}}%
\def\sectionmark#1{%
      \markright{\MakeLowercase{%
        \ifnum \c@secnumdepth >\m@ne
          \thesection\quad
        \fi
        #1}}}
        
\makeatother

%\makeatletter

 \def\small{%
  \@setfontsize\small\@xipt{13pt}%
  \abovedisplayskip 8\p@ \@plus3\p@ \@minus6\p@
  \belowdisplayskip \abovedisplayskip
  \abovedisplayshortskip \z@ \@plus3\p@
  \belowdisplayshortskip 6.5\p@ \@plus3.5\p@ \@minus3\p@
  \def\@listi{%
    \leftmargin\leftmargini
    \topsep 9\p@ \@plus3\p@ \@minus5\p@
    \parsep 4.5\p@ \@plus2\p@ \@minus\p@
    \itemsep \parsep
  }%
}%
 \def\footnotesize{%
  \@setfontsize\footnotesize\@xpt{12pt}%
  \abovedisplayskip 10\p@ \@plus2\p@ \@minus5\p@
  \belowdisplayskip \abovedisplayskip
  \abovedisplayshortskip \z@ \@plus3\p@
  \belowdisplayshortskip 6\p@ \@plus3\p@ \@minus3\p@
  \def\@listi{%
    \leftmargin\leftmargini
    \topsep 6\p@ \@plus2\p@ \@minus2\p@
    \parsep 3\p@ \@plus2\p@ \@minus\p@
    \itemsep \parsep
  }%
}%
\def\open@column@one#1{%
 \ltxgrid@info@sw{\class@info{\string\open@column@one\string#1}}{}%
 \unvbox\pagesofar
 \@ifvoid{\footsofar}{}{%
  \insert\footins\bgroup\unvbox\footsofar\egroup
  \penalty\z@
 }%
 \gdef\thepagegrid{one}%
 \global\pagegrid@col#1%
 \global\pagegrid@cur\@ne
 \global\count\footins\@m
 \set@column@hsize\pagegrid@col
 \set@colht
}%

\def\frontmatter@abstractheading{%
\bigskip
 \begingroup
  \centering\large
  \abstractname
  \par\bigskip
 \endgroup
}%

\makeatother

%\DeclareSymbolFont{CMlargesymbols}{OMX}{cmex}{m}{n}
%\DeclareMathSymbol{\sum}{\mathop}{CMlargesymbols}{"50}

\usepackage[papersize={6.7in, 10.0in}, left=.5in, right=.5in, top=1in, bottom=.9in]{geometry}
\linespread{1.1}
%\sloppy
\raggedbottom
\pagestyle{plain}
\usepackage{mathpartir}
\usepackage{stmaryrd}
\usepackage{mathtools}
\usepackage{tikz-cd}
\usepackage{microtype}
\usepackage{amssymb}
%\usepackage{fdsymbol}

% these include amsmath and that can cause trouble in older docs.
\makeatletter
\@ifpackageloaded{amsmath}{}{\RequirePackage{amsmath}}

\DeclareFontFamily{U}  {cmex}{}
\DeclareSymbolFont{Csymbols}       {U}  {cmex}{m}{n}
\DeclareFontShape{U}{cmex}{m}{n}{
    <-6>  cmex5
   <6-7>  cmex6
   <7-8>  cmex6
   <8-9>  cmex7
   <9-10> cmex8
  <10-12> cmex9
  <12->   cmex10}{}

\def\Set@Mn@Sym#1{\@tempcnta #1\relax}
\def\Next@Mn@Sym{\advance\@tempcnta 1\relax}
\def\Prev@Mn@Sym{\advance\@tempcnta-1\relax}
\def\@Decl@Mn@Sym#1#2#3#4{\DeclareMathSymbol{#2}{#3}{#4}{#1}}
\def\Decl@Mn@Sym#1#2#3{%
  \if\relax\noexpand#1%
    \let#1\undefined
  \fi
  \expandafter\@Decl@Mn@Sym\expandafter{\the\@tempcnta}{#1}{#3}{#2}%
  \Next@Mn@Sym}
\def\Decl@Mn@Alias#1#2#3{\Prev@Mn@Sym\Decl@Mn@Sym{#1}{#2}{#3}}
\let\Decl@Mn@Char\Decl@Mn@Sym
\def\Decl@Mn@Op#1#2#3{\def#1{\DOTSB#3\slimits@}}
\def\Decl@Mn@Int#1#2#3{\def#1{\DOTSI#3\ilimits@}}

\let\sum\undefined
\DeclareMathSymbol{\tsum}{\mathop}{Csymbols}{"50}
\DeclareMathSymbol{\dsum}{\mathop}{Csymbols}{"51}

\Decl@Mn@Op\sum\dsum\tsum

\makeatother

\makeatletter
\@ifpackageloaded{amsmath}{}{\RequirePackage{amsmath}}

\DeclareFontFamily{OMX}{MnSymbolE}{}
\DeclareSymbolFont{largesymbolsX}{OMX}{MnSymbolE}{m}{n}
\DeclareFontShape{OMX}{MnSymbolE}{m}{n}{
    <-6>  MnSymbolE5
   <6-7>  MnSymbolE6
   <7-8>  MnSymbolE7
   <8-9>  MnSymbolE8
   <9-10> MnSymbolE9
  <10-12> MnSymbolE10
  <12->   MnSymbolE12}{}

\DeclareMathSymbol{\downbrace}    {\mathord}{largesymbolsX}{'251}
\DeclareMathSymbol{\downbraceg}   {\mathord}{largesymbolsX}{'252}
\DeclareMathSymbol{\downbracegg}  {\mathord}{largesymbolsX}{'253}
\DeclareMathSymbol{\downbraceggg} {\mathord}{largesymbolsX}{'254}
\DeclareMathSymbol{\downbracegggg}{\mathord}{largesymbolsX}{'255}
\DeclareMathSymbol{\upbrace}      {\mathord}{largesymbolsX}{'256}
\DeclareMathSymbol{\upbraceg}     {\mathord}{largesymbolsX}{'257}
\DeclareMathSymbol{\upbracegg}    {\mathord}{largesymbolsX}{'260}
\DeclareMathSymbol{\upbraceggg}   {\mathord}{largesymbolsX}{'261}
\DeclareMathSymbol{\upbracegggg}  {\mathord}{largesymbolsX}{'262}
\DeclareMathSymbol{\braceld}      {\mathord}{largesymbolsX}{'263}
\DeclareMathSymbol{\bracelu}      {\mathord}{largesymbolsX}{'264}
\DeclareMathSymbol{\bracerd}      {\mathord}{largesymbolsX}{'265}
\DeclareMathSymbol{\braceru}      {\mathord}{largesymbolsX}{'266}
\DeclareMathSymbol{\bracemd}      {\mathord}{largesymbolsX}{'267}
\DeclareMathSymbol{\bracemu}      {\mathord}{largesymbolsX}{'270}
\DeclareMathSymbol{\bracemid}     {\mathord}{largesymbolsX}{'271}

\def\horiz@expandable#1#2#3#4#5#6#7#8{%
  \@mathmeasure\z@#7{#8}%
  \@tempdima=\wd\z@
  \@mathmeasure\z@#7{#1}%
  \ifdim\noexpand\wd\z@>\@tempdima
    $\m@th#7#1$%
  \else
    \@mathmeasure\z@#7{#2}%
    \ifdim\noexpand\wd\z@>\@tempdima
      $\m@th#7#2$%
    \else
      \@mathmeasure\z@#7{#3}%
      \ifdim\noexpand\wd\z@>\@tempdima
        $\m@th#7#3$%
      \else
        \@mathmeasure\z@#7{#4}%
        \ifdim\noexpand\wd\z@>\@tempdima
          $\m@th#7#4$%
        \else
          \@mathmeasure\z@#7{#5}%
          \ifdim\noexpand\wd\z@>\@tempdima
            $\m@th#7#5$%
          \else
           #6#7%
          \fi
        \fi
      \fi
    \fi
  \fi}

\def\overbrace@expandable#1#2#3{\vbox{\m@th\ialign{##\crcr
  #1#2{#3}\crcr\noalign{\kern2\p@\nointerlineskip}%
  $\m@th\hfil#2#3\hfil$\crcr}}}
\def\underbrace@expandable#1#2#3{\vtop{\m@th\ialign{##\crcr
  $\m@th\hfil#2#3\hfil$\crcr
  \noalign{\kern2\p@\nointerlineskip}%
  #1#2{#3}\crcr}}}

\def\overbrace@#1#2#3{\vbox{\m@th\ialign{##\crcr
  #1#2\crcr\noalign{\kern2\p@\nointerlineskip}%
  $\m@th\hfil#2#3\hfil$\crcr}}}
\def\underbrace@#1#2#3{\vtop{\m@th\ialign{##\crcr
  $\m@th\hfil#2#3\hfil$\crcr
  \noalign{\kern2\p@\nointerlineskip}%
  #1#2\crcr}}}

\def\bracefill@#1#2#3#4#5{$\m@th#5#1\leaders\hbox{$#4$}\hfill#2\leaders\hbox{$#4$}\hfill#3$}

\def\downbracefill@{\bracefill@\braceld\bracemd\bracerd\bracemid}
\def\upbracefill@{\bracefill@\bracelu\bracemu\braceru\bracemid}

\DeclareRobustCommand{\downbracefill}{\downbracefill@\textstyle}
\DeclareRobustCommand{\upbracefill}{\upbracefill@\textstyle}

\def\upbrace@expandable{%
  \horiz@expandable
    \upbrace
    \upbraceg
    \upbracegg
    \upbraceggg
    \upbracegggg
    \upbracefill@}
\def\downbrace@expandable{%
  \horiz@expandable
    \downbrace
    \downbraceg
    \downbracegg
    \downbraceggg
    \downbracegggg
    \downbracefill@}

\DeclareRobustCommand{\overbrace}[1]{\mathop{\mathpalette{\overbrace@expandable\downbrace@expandable}{#1}}\limits}
\DeclareRobustCommand{\underbrace}[1]{\mathop{\mathpalette{\underbrace@expandable\upbrace@expandable}{#1}}\limits}

\makeatother


\usepackage[small]{titlesec}
\usepackage{cite}

% make sure there is enough TOC for reasonable pdf bookmarks.
\setcounter{tocdepth}{3}

\def\to{\rightarrow}
\def\imp{\shortrightarrow}
\def\iff{\leftrightarrow}
\def\union{\cup}
\def\inc{\subseteq}
\def\dom{\mathop{\rm dom}}
\def\cod{\mathop{\rm cod}}
\def\id{{\mathrm 1}}
\def\res{\!\upharpoonleft\!}
\def\ffam{\varphi}
\def\comp{\circ}
\def\bbone{\mathbb 1}
\def\zeromap{0}
\def\bbzero{{\mathbb O}}
\def\ccc{{c.c.c.}}
\def\ev{\varepsilon}
\def\ebc{\varepsilon_{BC}}
\def\L{\Lambda}
\def\l{\lambda}
\def\lm#1.#2{\lambda#1.\, #2}
\def\br#1{[\, #1 \, ]}
\def\V{V}
\def\U{U}
\def\D{D}
\def\C{\mathcal C}
\def\S{\mathcal S}
\def\lxy{\l x\, \l y . \,}
\def\lmm#1#2.#3{\l #1\, \l #2 . \, #3}
\def\sss{(*\!*\!*)}
\def\ss{(**)}
\def\ssn{(**_n)}
\def\scop{\S^{\C^{op}}}
\def\PU{\mathcal P U}
\def\P{\mathcal P}
\def\UU{(U\to U)}
\def\BA{B \to A}
\def\AB{A \to B}

\makeatletter
\newcommand*\dotop{\mathpalette\bigcdot@{.6}}
\newcommand*\bigcdot@[2]{\mathbin{\vcenter{\hbox{\scalebox{#2}{$\m@th#1\bullet$}}}}}
\makeatother

\title{\large Identity and Existence in Intuitionistic Logic\footnote{This is a retyping of Scott D. (1979) Identity and existence in intuitionistic logic. In: Fourman M., Mulvey C., Scott D. (eds) Applications of Sheaves. Lecture Notes in Mathematics, vol 753. Springer, Berlin, Heidelberg. https://doi-org.cmu.idm.oclc.org/10.1007/BFb0061839. This version was typed out in \LaTeX\ in November of 2021}
}
\author{\normalsize Dana Scott \\
{\small\it Merton College}\\
{\small\it Oxford}}
\date{\small 1979}

\begin{document}

\maketitle

Standard formulations of intuitionistic logic, whether by logicians or by category theorists, generally do not take into account partially defined elements. (For a recent reference see Makkai and Reyes [18] , esp. pp. 144--163.) Perhaps
there is a simple psychological reason: we dislike talking of those things not
already proved to exist. Certainly we should not assume that things exist without making this assumption explicit. In classical logic the problem is not important, because it is always possible to split the definition (or theorem) into cases according as the object in question does or does not exist. In intuitionistic logic this way is not open to us, and the circumstance complicates many constructions, the
theory of descriptions, for example. Many people I find do not agree with me, but
I should like to advocate in a mild way in this paper what I consider a simple extension of the usual formulation of logic allowing reference to partial elements. The discussion will be entirely formal here, but for the model theory of the system the reader should consult Fourman and Scott [10] for interpretations over a complete Heyting algebra (and this includes the so-called Kripke models) and Fourman [8]
(the paper was written in 1975) for the interpretation in an arbitrary topos.

Technically the idea is to permit a wider interpretation of {\it free} variables.
All bound variables retain their usual existential import (when we say something exists it does exist), but free variables behave in a more ``schematic'' way. Thus there will be no restrictions on the use of {\it modus ponens} or on the rule of {\it substitution} involving free variables and their occurrences. The laws of quantifiers require some modification, however, to make the existential assumptions explicit. The modification is very straightforward, and I shall argue that what has to be done is
simply what is done naturally in making a {\it relativization} of quantifiers from a
larger domain to a subdomain. Again in intuitionstic logic we have So take care over relativization, because we cannot say that either the subdomain is empty or not -- thus a given element may be only ``partially'' in the subdomain.

In Section \ref{sec1}, I discuss the idea of allowing existence as a predicate and
the laws of quantifiers. Section 2 treats the theory of identity and the connections with existence. Questions of strictness and extensionality of relations and functions are discussed in Section 3 along with some examples of first-order theories. As further examples of the use of the system, the familiar theories of apartness and ordering in intuitionistic logic are presented in Section 4 . Section 5 goes briefly into relativization, and Section 6 details the principles of descriptions. Finally, Section 7 reviews the axioms for higher-order intuitionistic logic from this general viewpoint.
The idea of schematic free variables is not new for classical logic, and the literature on ``free'' logic (or logic without existence assumptions) is extensive.
(For some earlier references see Scott [21].) All I have done in this essay is
to make what seems to me to be the obvious carryover to intuitionistic logic,
because I think it is necessary and convenient. For those who do not like this formulation, some comfort can be taken from the fact that in topos theory both kinds of systems are completely equivalent, and the domains of partial elements can be defined at higher types (this is analogous to passing from a sheaf to its ``flabbyfication'', which is a subsheaf of the power sheaf). However, in first-order logic something is lost in not allowing partial elements, as I shall try to argue along
the way.

\section{Existence and the Laws of Quantifiers}\label{sec1}

It has often been suggested that identity is a trivial relation, since to say
``$a = b$'' is trivially true in case $a$ and $b$ are the same and otherwise trivially false. 
If ``$a$'' and ``$b$'' are ``constant'' names, this criticism may be reasonable; but when the expressions depend on parameters, it is obviously useful to express properties by equations. If an example is needed, take the equation:

$$
x^2 = x + 1.
$$
Whether this is true or false depends on x , and such equations (generally) define a whole class of solutions. We can, of course, in this case investigate by well- known methods exactly which x make the equation true; but with only the most superficial knowledge of the laws of algebra, we can easily assert a {\it conditional} like:

$$
x^2 = x + 1 \imp x^6 = 8x + 5.
$$
Indeed, all the values of $x^n$ can be simplified under the assumption that $x^2 =
x + 1$. Passing to the many examples we are familiar with in several variables,
we see that conditional equations may often be verified even when a complete analysis
of the solution set corresponding to the hypothesis is lacking. The assumption
is used as if it were true even though by itself it has no determinate truth value 
owing to the occurrence of parameters.

If we are willing to employ complex equations in this way, why should we not feel 
free to use complex expressions (terms) without demanding that they always denote? Just as we have to make certain equations conditional on the truth of
other equations in order that they be valid, we may also have to make some 
statements conditional on the existence of certain complex terms. In algebra (say, in ring theory), the implication:
$$
\forall x.\, \phi(x) \imp \phi(0)
$$
is unconditionally valid because the constant 0 is taken as always denoting in
all rings. However, the statement:
$$
\forall x.\, \phi(x) \imp \phi(1/a)
$$
cannot be valid in general because not every element a has an inverse. We can circumvent the difficulty by rephrasing the statement:
$$
\forall x.\, \phi(x) \imp \forall y. [ a \cdot y = 1 \imp \phi(y) ] \quad,
$$
but though correct this seems clumsy. Why not say more directly:
$$
\forall x.\, \phi(x) \land E(1/a) \to \phi(1/a)
$$
where ``$E(1/a)$'' is to be read as ``$1/a$ exists''? Even if we agree that
$$
E(1/a) \iff \exists y. a \cdot y = 1
$$
(which avoids the notation $1/a$ on the right-hand side of the equivalence), we still want to use $1/a$ in the conclusion. 
The desire to keep to fractional notation will become even more urgent when more complex rational functions 
(say, $3x + (4 / 2x^2) + x + 1$) are to be manipulated.
\begin{thebibliography}{86}


\end{thebibliography}
