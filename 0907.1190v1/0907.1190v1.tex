\documentclass[12pt,aps,prl]{revtex4}


%\usepackage{amsmath,amsthm,amscd,amssymb}
\usepackage[noBBpl,sc]{mathpazo}
\usepackage[papersize={6.6in, 10.0in}, left=.5in, right=.5in, top=.6in, bottom=.9in]{geometry}
\linespread{1.05}
\sloppy
\raggedbottom
\pagestyle{plain}
\usepackage{MnSymbol}

% these include amsmath and that can cause trouble in older docs.
\makeatletter
\@ifpackageloaded{amsmath}{}{\RequirePackage{amsmath}}

\DeclareFontFamily{U}  {cmex}{}
\DeclareSymbolFont{Csymbols}       {U}  {cmex}{m}{n}
\DeclareFontShape{U}{cmex}{m}{n}{
    <-6>  cmex5
   <6-7>  cmex6
   <7-8>  cmex6
   <8-9>  cmex7
   <9-10> cmex8
  <10-12> cmex9
  <12->   cmex10}{}

\def\Set@Mn@Sym#1{\@tempcnta #1\relax}
\def\Next@Mn@Sym{\advance\@tempcnta 1\relax}
\def\Prev@Mn@Sym{\advance\@tempcnta-1\relax}
\def\@Decl@Mn@Sym#1#2#3#4{\DeclareMathSymbol{#2}{#3}{#4}{#1}}
\def\Decl@Mn@Sym#1#2#3{%
  \if\relax\noexpand#1%
    \let#1\undefined
  \fi
  \expandafter\@Decl@Mn@Sym\expandafter{\the\@tempcnta}{#1}{#3}{#2}%
  \Next@Mn@Sym}
\def\Decl@Mn@Alias#1#2#3{\Prev@Mn@Sym\Decl@Mn@Sym{#1}{#2}{#3}}
\let\Decl@Mn@Char\Decl@Mn@Sym
\def\Decl@Mn@Op#1#2#3{\def#1{\DOTSB#3\slimits@}}
\def\Decl@Mn@Int#1#2#3{\def#1{\DOTSI#3\ilimits@}}

\let\sum\undefined
\DeclareMathSymbol{\tsum}{\mathop}{Csymbols}{"50}
\DeclareMathSymbol{\dsum}{\mathop}{Csymbols}{"51}

\Decl@Mn@Op\sum\dsum\tsum

\makeatother

\makeatletter
\@ifpackageloaded{amsmath}{}{\RequirePackage{amsmath}}

\DeclareFontFamily{OMX}{MnSymbolE}{}
\DeclareSymbolFont{largesymbolsX}{OMX}{MnSymbolE}{m}{n}
\DeclareFontShape{OMX}{MnSymbolE}{m}{n}{
    <-6>  MnSymbolE5
   <6-7>  MnSymbolE6
   <7-8>  MnSymbolE7
   <8-9>  MnSymbolE8
   <9-10> MnSymbolE9
  <10-12> MnSymbolE10
  <12->   MnSymbolE12}{}

\DeclareMathSymbol{\downbrace}    {\mathord}{largesymbolsX}{'251}
\DeclareMathSymbol{\downbraceg}   {\mathord}{largesymbolsX}{'252}
\DeclareMathSymbol{\downbracegg}  {\mathord}{largesymbolsX}{'253}
\DeclareMathSymbol{\downbraceggg} {\mathord}{largesymbolsX}{'254}
\DeclareMathSymbol{\downbracegggg}{\mathord}{largesymbolsX}{'255}
\DeclareMathSymbol{\upbrace}      {\mathord}{largesymbolsX}{'256}
\DeclareMathSymbol{\upbraceg}     {\mathord}{largesymbolsX}{'257}
\DeclareMathSymbol{\upbracegg}    {\mathord}{largesymbolsX}{'260}
\DeclareMathSymbol{\upbraceggg}   {\mathord}{largesymbolsX}{'261}
\DeclareMathSymbol{\upbracegggg}  {\mathord}{largesymbolsX}{'262}
\DeclareMathSymbol{\braceld}      {\mathord}{largesymbolsX}{'263}
\DeclareMathSymbol{\bracelu}      {\mathord}{largesymbolsX}{'264}
\DeclareMathSymbol{\bracerd}      {\mathord}{largesymbolsX}{'265}
\DeclareMathSymbol{\braceru}      {\mathord}{largesymbolsX}{'266}
\DeclareMathSymbol{\bracemd}      {\mathord}{largesymbolsX}{'267}
\DeclareMathSymbol{\bracemu}      {\mathord}{largesymbolsX}{'270}
\DeclareMathSymbol{\bracemid}     {\mathord}{largesymbolsX}{'271}

\def\horiz@expandable#1#2#3#4#5#6#7#8{%
  \@mathmeasure\z@#7{#8}%
  \@tempdima=\wd\z@
  \@mathmeasure\z@#7{#1}%
  \ifdim\noexpand\wd\z@>\@tempdima
    $\m@th#7#1$%
  \else
    \@mathmeasure\z@#7{#2}%
    \ifdim\noexpand\wd\z@>\@tempdima
      $\m@th#7#2$%
    \else
      \@mathmeasure\z@#7{#3}%
      \ifdim\noexpand\wd\z@>\@tempdima
        $\m@th#7#3$%
      \else
        \@mathmeasure\z@#7{#4}%
        \ifdim\noexpand\wd\z@>\@tempdima
          $\m@th#7#4$%
        \else
          \@mathmeasure\z@#7{#5}%
          \ifdim\noexpand\wd\z@>\@tempdima
            $\m@th#7#5$%
          \else
           #6#7%
          \fi
        \fi
      \fi
    \fi
  \fi}

\def\overbrace@expandable#1#2#3{\vbox{\m@th\ialign{##\crcr
  #1#2{#3}\crcr\noalign{\kern2\p@\nointerlineskip}%
  $\m@th\hfil#2#3\hfil$\crcr}}}
\def\underbrace@expandable#1#2#3{\vtop{\m@th\ialign{##\crcr
  $\m@th\hfil#2#3\hfil$\crcr
  \noalign{\kern2\p@\nointerlineskip}%
  #1#2{#3}\crcr}}}

\def\overbrace@#1#2#3{\vbox{\m@th\ialign{##\crcr
  #1#2\crcr\noalign{\kern2\p@\nointerlineskip}%
  $\m@th\hfil#2#3\hfil$\crcr}}}
\def\underbrace@#1#2#3{\vtop{\m@th\ialign{##\crcr
  $\m@th\hfil#2#3\hfil$\crcr
  \noalign{\kern2\p@\nointerlineskip}%
  #1#2\crcr}}}

\def\bracefill@#1#2#3#4#5{$\m@th#5#1\leaders\hbox{$#4$}\hfill#2\leaders\hbox{$#4$}\hfill#3$}

\def\downbracefill@{\bracefill@\braceld\bracemd\bracerd\bracemid}
\def\upbracefill@{\bracefill@\bracelu\bracemu\braceru\bracemid}

\DeclareRobustCommand{\downbracefill}{\downbracefill@\textstyle}
\DeclareRobustCommand{\upbracefill}{\upbracefill@\textstyle}

\def\upbrace@expandable{%
  \horiz@expandable
    \upbrace
    \upbraceg
    \upbracegg
    \upbraceggg
    \upbracegggg
    \upbracefill@}
\def\downbrace@expandable{%
  \horiz@expandable
    \downbrace
    \downbraceg
    \downbracegg
    \downbraceggg
    \downbracegggg
    \downbracefill@}

\DeclareRobustCommand{\overbrace}[1]{\mathop{\mathpalette{\overbrace@expandable\downbrace@expandable}{#1}}\limits}
\DeclareRobustCommand{\underbrace}[1]{\mathop{\mathpalette{\underbrace@expandable\upbrace@expandable}{#1}}\limits}

\makeatother


\usepackage[small]{titlesec}
\usepackage{microtype}

% hyperref last because otherwise some things go wrong.
\usepackage[colorlinks=true
,breaklinks=true
,urlcolor=blue
,anchorcolor=blue
,citecolor=blue
,filecolor=blue
,linkcolor=blue
,menucolor=blue
,linktocpage=true]{hyperref}
\hypersetup{
bookmarksopen=true,
bookmarksnumbered=true,
bookmarksopenlevel=10
}

% make sure there is enough TOC for reasonable pdf bookmarks.
\setcounter{tocdepth}{3}

%\usepackage[dotinlabels]{titletoc}
%\titlelabel{{\thetitle}.\quad}
%\usepackage{titletoc}
\usepackage[small]{titlesec}

\titleformat{\section}[block]
  {\fillast\medskip}
  {\bfseries{\thesection. }}
  {1ex minus .1ex}
  {\bfseries}
 
\titleformat*{\subsection}{\itshape}
\titleformat*{\subsubsection}{\itshape}

\setcounter{tocdepth}{2}

\titlecontents{section}
              [2.3em] 
              {\bigskip}
              {{\contentslabel{2.3em}}}
              {\hspace*{-2.3em}}
              {\titlerule*[1pc]{}\contentspage}
              
\titlecontents{subsection}
              [4.7em] 
              {}
              {{\contentslabel{2.3em}}}
              {\hspace*{-2.3em}}
              {\titlerule*[.5pc]{}\contentspage}

% hopefully not used.           
\titlecontents{subsubsection}
              [7.9em]
              {}
              {{\contentslabel{3.3em}}}
              {\hspace*{-3.3em}}
              {\titlerule*[.5pc]{}\contentspage}
%\makeatletter
\renewcommand\tableofcontents{%
    \section*{\contentsname
        \@mkboth{%
           \MakeLowercase\contentsname}{\MakeLowercase\contentsname}}%
    \@starttoc{toc}%
    }
\def\@oddhead{{\scshape\rightmark}\hfil{\small\scshape\thepage}}%
\def\sectionmark#1{%
      \markright{\MakeLowercase{%
        \ifnum \c@secnumdepth >\m@ne
          \thesection\quad
        \fi
        #1}}}
        
\makeatother

%\makeatletter

 \def\small{%
  \@setfontsize\small\@xipt{13pt}%
  \abovedisplayskip 8\p@ \@plus3\p@ \@minus6\p@
  \belowdisplayskip \abovedisplayskip
  \abovedisplayshortskip \z@ \@plus3\p@
  \belowdisplayshortskip 6.5\p@ \@plus3.5\p@ \@minus3\p@
  \def\@listi{%
    \leftmargin\leftmargini
    \topsep 9\p@ \@plus3\p@ \@minus5\p@
    \parsep 4.5\p@ \@plus2\p@ \@minus\p@
    \itemsep \parsep
  }%
}%
 \def\footnotesize{%
  \@setfontsize\footnotesize\@xpt{12pt}%
  \abovedisplayskip 10\p@ \@plus2\p@ \@minus5\p@
  \belowdisplayskip \abovedisplayskip
  \abovedisplayshortskip \z@ \@plus3\p@
  \belowdisplayshortskip 6\p@ \@plus3\p@ \@minus3\p@
  \def\@listi{%
    \leftmargin\leftmargini
    \topsep 6\p@ \@plus2\p@ \@minus2\p@
    \parsep 3\p@ \@plus2\p@ \@minus\p@
    \itemsep \parsep
  }%
}%
\def\open@column@one#1{%
 \ltxgrid@info@sw{\class@info{\string\open@column@one\string#1}}{}%
 \unvbox\pagesofar
 \@ifvoid{\footsofar}{}{%
  \insert\footins\bgroup\unvbox\footsofar\egroup
  \penalty\z@
 }%
 \gdef\thepagegrid{one}%
 \global\pagegrid@col#1%
 \global\pagegrid@cur\@ne
 \global\count\footins\@m
 \set@column@hsize\pagegrid@col
 \set@colht
}%

\def\frontmatter@abstractheading{%
\bigskip
 \begingroup
  \centering\large
  \abstractname
  \par\bigskip
 \endgroup
}%

\makeatother

%\DeclareSymbolFont{CMlargesymbols}{OMX}{cmex}{m}{n}
%\DeclareMathSymbol{\sum}{\mathop}{CMlargesymbols}{"50}
%\pdfbookmark[1]{Introduction}{Introduction}

\begin{document}
\title{Black hole entropy as entropy of entanglement,\\
or it's curtains for the equivalence principle}

\author{Samuel L.\ Braunstein}
\affiliation{Computer Science, University of York, York YO10 5DD, UK}

\date{July 2009}

\begin{abstract}
The equivalence principle provides an important tenet of black hole
physics: that a sufficiently small observer freely falling into a
black hole should experience nothing special as she passes the event
horizon --- the boundary of no return. Similarly, quantum fields
impinging on a black hole should exhibit no special behavior at the
event horizon. Indeed quantum fields should be entangled across the
event horizon just as they would be across the boundary to any volume
in flat space. We study this claim using random subsystems as models
of black hole evaporation. We find that unless the Bekenstein-Hawking
entropy of a black hole is almost entirely entropy of entanglement,
then the trans-event horizon entanglement vanishes long before the
black hole has evaporated to the Planck scale. This would force quantum
fields across the event horizon to be arbitrarily far from the vacuum
state; an energetic curtain would have descended around the black hole.
\end{abstract}

\maketitle

Cutting out a volume of space across which a quantum field propagates
yields entanglement between the subsystems so formed. For fields (or
condensed matter systems) in or close to the ground state, the
entropy of this entanglement scales as the surface area of the
boundary \cite{Eisert09}. One can mentally grow or shrink the volume,
and the entanglement responds accordingly, but in a manner which is
essentially decoupled from any underlying dynamics. For a black hole,
a distinguished boundary, namely the event horizon, is chosen by
causality. Classically, the internal degrees of freedom of trans-event
horizon entanglement are forbidden from escaping even as the black
hole's surface area shrinks due to quantum evaporation \cite{Hawking75}.
The only way for this entanglement to scale down with the black hole's
thermodynamic entropy is to leave through some quantum mechanical
tunneling process \cite{Parikh00}. We claim that unlike trans-boundary
entanglement in flat space, trans-event horizon entanglement within a
black hole {\it must\/} participate within the underlying (evaporative)
dynamics. 

A powerful tool for studying unitary dynamics in high dimensional systems
is random matrix theory. Indeed, numerical simulations show that for the
quantities studied here individual randomly selected unitaries yield
almost identical results as analytically computed averages. Thus,
unless the black hole dynamics is from a vanishingly small
set, averages over random unitaries should give an excellent
approximation to the actual dynamics.

In particular, Don Page \cite{Page93} pioneered this approach for modeling
black hole evaporation as the `budding off' of random subsystems from
the interior Hilbert space of a black hole to represent outgoing
radiation \cite{Page93,Hayden07,B09}. Page argued that the Hilbert
space dimensionality of this interior space should be well approximated
by $N=e^{S_{\text{BH}}}$, in terms of the {\it thermodynamic\/} entropy
$S_{\text{BH}}={\cal A}/4$ of a black hole of area ${\cal A}$. He
assumed an initially pure state $|i\rangle_{\text{int}}$ for the black
hole interior (int), so evaporation becomes 
$|i\rangle_{\text{int}}\rightarrow (U|i\rangle)_{\text{RB}}$. Here a
random unitary $U\in U(N)$ is followed by the `emission' of radiation
into subsystem $R$ with the remaining interior subsystem $B$.

When incorporating trans-event horizon entanglement into the
evaporative dynamics, Page's model becomes
$$
\sum_{i}\sqrt{p_i}\, |i\rangle_{\text{ext}}\otimes|i\rangle_{\text{int}}
\rightarrow
\sum_{i} \sqrt{p_i}\,
|i\rangle_{\text{ext}}\otimes(U|i\rangle)_{\text{RB}}.
$$
Here $\sum_i p_i |i\rangle%_{\text{ext}}\,{}_{\text{ext}}\!
\langle i|$
is the reduced density matrix of the external (ext) modes neighboring
the event horizon.  The entropy of entanglement may be quantified with
a R\'enyi entropy $H_{\text{ext}}$ ($\le S_{\text{BH}}$) for this state.
As the logarithm of the dimension of the radiation subsystem grows to
$\ln[{\text{dim}}(R)]=\frac{1}{2}S_{\text{BH}}+\frac{1}{2}H_{\text{ext}}+c$
the initial trans-event horizon entanglement between the external
neighborhood and the interior subsystems has virtually vanished, with
it appearing instead (with a fidelity of at least $1-e^{-c}$) as
entanglement between external neighborhood modes and the outgoing
radiation (see Appendix~B).

In an arbitrary system where trans-boundary entanglement has vanished,
the quantum field cannot be in or anywhere near its ground state.
Applied to black holes, a loss of trans-event horizon entanglement
implies fields far from the vacuum state in the vicinity of the event
horizon. Were this to happen before the black hole had evaporated to
the Planck scale, at $\ln[{\text{dim}}(R)]\approx S_{\text{BH}}$, there
would be a manifest failure of the equivalence principle. Therefore
either the thermodynamic entropy of black holes is due primarily to
entropy of entanglement, i.e., $S_{\text{BH}}\approx H_{\text{ext}}$,
or the randomly selected subsystem model of black hole evaporation is
missing an important subtlety.

We are not the first to conjecture that a black hole's thermodynamic
entropy is entropy of entanglement
\cite{tHooft85,Bombelli86,Srednicki93,Hawking01,Brustein06,Emparan06}.
Indeed, it unavoidably holds for some models of eternal black
holes \cite{Hawking01,Brustein06} and even resolves some difficulties
associated with computing their entropy at the microscopic
level \cite{Emparan06}. The argument above demonstrates that in models
of dynamically evolving black holes, unless this conjecture holds at
least approximately, it is time to draw a curtain on some of the
equivalence principle's cherished predictions.

\vskip 0.07truein
\noindent
The author gratefully acknowledges the hospitality of the Frank Graham
Research House.

\section{Appendix A}

Before considering the general entangled state described by the manuscript
let us understand it for a simpler model, one for which the behavior
is available in the literature. In particular, consider the
model for black hole evaporation with trans-event horizon entanglement as
\begin{equation}
\frac{1}{\sqrt{E}}\sum_{i=1}^{E}
|i\rangle_{\text{ext}}\otimes|i\rangle_{\text{int}}\rightarrow
\frac{1}{\sqrt{E}}\sum_{i=1}^{E}
|i\rangle_{\text{ext}}\otimes(U|i\rangle)_{\text{RB}}.
\end{equation}
Here $\ln E$ is the entropy of entanglement between the external (ext)
modes neighboring the event horizon and the interior of the black
hole. Except for the interpretation of the source of entanglement,
this model has been recently analyzed by Hayden and Preskill \cite{Hayden07}.
We may therefore quote their key result in our terms: As the logarithm
of the dimension of the radiation subsystem grows to
$\frac{1}{2} S_{\text{BH}}+\frac{1}{2} \ln E + c$ the initial trans-event
horizon entanglement between the external neighborhood and the interior
subsystems has virtually vanished, with it appearing instead (with a
fidelity of at least $1-e^{-c}$) as entanglement between external
neighborhood modes and the outgoing radiation. Here (as in the manuscript)
$c$ is a free parameter, but will be dwarfed by any of the entropies
involved.

Below we shall see that when the uniform entanglement of the above analysis
is replaced with general trans-event horizon entanglement, the measure
of entanglement $\ln E$ is replaced by the R\'enyi entropy $H_{\text{ext}}$.

\section{Appendix B}
\label{decoupling}

The R\'enyi entropy of a state is defined
\begin{equation}
H^{(q)}(\rho)=\frac{1}{1-q} \ln \,({\text{tr}}\; \rho^q).
\end{equation}
Note that all R\'enyi entropies are bounded above by the logarithm of
the Hilbert space dimension, so
 $0\le H^{(q)}(\rho_{\text{ext}})\le S_{\text{BH}}$
for the state we study. Of particular interest to us here will be the
fractional R\'enyi entropy for $q=\frac{1}{2}$, so
\begin{equation}
H_{\text{ext}} \equiv H^{(1/2)}(\rho_{\text{ext}})
=\ln \,\bigl[({\text{tr}}\;\sqrt{\rho_{\text{ext}}}\,)^2\bigr].
\end{equation}

Our key result is based on a generalization of the decoupling theorem of
Ref.~\onlinecite{Abey06}. Consider now the tripartite state
\begin{equation}
\rho_{XYZ}^{\text{~}}=
\rho_{XY_1Y_2Z}^{\text{~}},
\end{equation}
where the joint subsystems $Y=Y_1Y_2$ will be decomposed as either
the radiation modes and interior black holes modes $RB$ or vice-versa
$BR$. This allows us to define
\begin{equation}
\sigma_{XY_2Z}^U\equiv
\text{tr}_{Y_1}^{\text{~}} \bigl(U_{Y}\;
\rho_{XYZ}^{\text{~}}\; U_{Y}^\dagger\bigr).
\end{equation}

\noindent
% {\bf Generalized decoupling theorem:}\\
{\bf The mother-in-law of all decoupling theorems:}
\begin{eqnarray}
&&\biggl(\int_{U\in U(Y)}dU\,\bigl\| \sigma_{XY_2Z}^U-
\sigma_{X}^U\otimes \sigma_{Y_2Z}^U\bigr\|_1\biggr)^2\nonumber\\
&\le& {\text{tr}}\; \rho_{X}^{2\nu}\; {\text{tr}}\; \rho_{Z}^{2\mu}
\biggl\{\Bigl[ {\text{tr}}\; \rho_{XZ}^{2}
        ( \rho_{X}^{-2\nu}\otimes \rho_{Z}^{-2\mu})\nonumber\\
&&\phantom{{\text{tr}}\; \rho_{X}^{2\nu}\; {\text{tr}}\; \rho_{Z}^{2\mu}}
\;-2\, {\text{tr}}\; \rho_{XZ}^{\text{~}}
   (\rho_{X}^{1-2\nu}\otimes \rho_{Z}^{1-2\mu})
+{\text{tr}}\; \rho_{X}^{2-2\nu}\; {\text{tr}}\; \rho_{Z}^{2-2\mu}\Bigr]
\nonumber\\
&&\phantom{{\text{tr}}\; \rho_{X}^{2\nu}\; {\text{tr}} }
+\frac{Y_2}{Y_1}\Bigl[
{\text{tr}}\; \rho_{XYZ}^{2}
        ( \rho_{X}^{-2\nu}\otimes \rho_{Z}^{-2\mu}) \nonumber \\
&&\phantom{{\text{tr}}\; \rho_{X}^{2\nu}\; {\text{tr}}~~~~~~~~}
+{\text{tr}}\; \rho_{X}^{2-2\nu}\;
{\text{tr}}\; \rho_{YZ}^{2}\;\rho_{Z}^{-2\mu}\Bigr]
\biggr\} \label{step1}\\
&\le & \frac{Y_2}{Y_1}\, {\text{tr}}\; \rho_{X}^{2\nu}\;
{\text{tr}}\; \rho_{Z}^{2\mu}\,
\Bigl[
{\text{tr}}\; \rho_{XYZ}^{2}
        ( \rho_{X}^{-2\nu}\otimes \rho_{Z}^{-2\mu}) \nonumber \\
&&\phantom{\frac{Y_2}{Y_1}\, {\text{tr}}\; \rho_{X}^{2\nu}\; 
{\text{tr}}\; \rho_{Z}^{2\mu}\Bigl[\,}
+{\text{tr}}\; \rho_{X}^{2-2\nu}\; 
{\text{tr}}\; \rho_{YZ}^{2}\;\rho_{Z}^{-2\mu}\Bigr] \label{step2} \\
&\le & 2\, \frac{Y_2}{Y_1}\,
e^{H_X^{\text{~}}+H_Z^{\text{~}}},
\label{step3}
\end{eqnarray}
where $H_A^{\text{~}}\equiv H^{(1/2)}(\rho_A^{\text{~}})$
and $0\le 2\nu,2\mu\le 1$. Here, to go from Eq.~(\ref{step1})
to Eq.~(\ref{step2}), we assume
$\rho_{XZ}^{\text{~}}=\rho_X^{\text{~}}\otimes \rho_Z^{\text{~}}$; and
to go from Eq.~(\ref{step2}) to Eq.~(\ref{step3}), we assume
$\rho_{XYZ}^{\text{~}}$ is pure and we take $2\nu=2\mu=\frac{1}{2}$.
\\

\noindent
{\bf Proof:}
Using the Cauchy-Schwarz inequality we may write
\begin{eqnarray}
&&\bigl\| \sigma_{XY_2Z}^U-
\sigma_{X}^U\otimes \sigma_{Y_2Z}^U\bigr\|_1 \\
&\le& \bigl\| \rho_{X}^{\nu}\otimes\openone_{Y_2}^{\text{~}}\otimes
\rho_{Z}^{\mu} \bigr\|_2\;
\bigl\| \rho_{X}^{-\nu}\otimes \rho_{Z}^{-\mu}(\sigma_{XY_2Z}^U-
\sigma_{X}^U\otimes \sigma_{Y_2Z}^U)\bigr\|_2, \nonumber
\end{eqnarray}
where without loss of generality we may assume that $\rho_{X}^{\nu}$
and $\rho_{Z}^{\mu}$ are invertible; then using the methods already
outlined in Ref.~\onlinecite{Abey06} the results are easily obtained.
$\hfill \ensuremath{\filledmedsquare}$\\
We note that the statement of the result reduces to the conventional
decoupling theorem for the choice $\nu=0$ and subsystem $Z$ is
one-dimensional.

Of particular interest here is the case where $2\nu=\frac{1}{2}$ and
$\rho_{\text{ext},Y}$ is pure, which gives
\begin{equation}
\int_{U\in U(Y)}dU\,\bigl\| \sigma_{\text{ext},Y_2}^U-
\sigma_{\text{ext}}^U\otimes \sigma_{Y_2}^U\bigr\|_1
\le \Bigl(2\,\frac{Y_2}{Y_1}\,e^{H_{\text{ext}}}\Bigr)^{\frac{1}{2}}.
\end{equation}

Now
$1-F(\rho,\sigma) \le \frac{1}{2}\|\rho-\sigma\|_1$, where the trace
norm is defined by $\|X\|_1\equiv \text{tr}\, |X|$ and the fidelity
by $F(\rho,\sigma)\equiv \|\sqrt{\rho}\sqrt{\sigma}\|_1$.
As a consequence, the fidelity with which the initial trans-event
horizon entanglement is encoded within the combined $\text{ext},Y_1$
subsystem is bounded below by \cite{B09}
$1-\sqrt{e^{H_{\text{ext}}} Y_2/Y_1}$.
Now allowing this in turn to be bounded from below by $1-e^{-c}$ and
choosing $Y_1=R$ and $Y_2=B$ gives the result quoted in the manuscript.

Interestingly, the opposite choice $Y_1=B$ and $Y_2=R$ tells us that
for the logarithm of the dimension of the radiation subsystem less than
$\frac{1}{2}(S_{\text{BH}}-H_{\text{ext}})-c$ the initial trans-event
horizon entanglement remains encoded between the external neighborhood and
the interior subsystems with fidelity of at least $1-e^{-c}$. This 
effectively gives the number of qunats ($\ln 2$ times the number of
qubits) that must be radiated before trans-event horizon entanglement
begins to be depleted; for $H_{\text{ext}}\approx S_{\text{BH}}$ this
occurs almost immediately.

\vskip 0.1truein
\noindent
{\bf Including matter:} A recent paper \cite{B09} considers matter 
(entangled with some distant reference, ref, subsystem) which collapses
to form a black hole, which itself exhibits trans-event horizon
entanglement, via
\begin{eqnarray}
&&\frac{1}{\sqrt{K}}\sum_{i=1}^K |i\rangle_{\text{ref}}\otimes
\sum_{j}\sqrt{p_j}\,(|i\rangle\otimes |j\rangle\oplus 0)_{\text{int}}
\otimes|j\rangle_{\text{ext}}\\
&\rightarrow&
\frac{1}{\sqrt{K}}\sum_{i=1}^K |i\rangle_{\text{ref}}\otimes
\!\sum_{j}\sqrt{p_j}\,[U(|i\rangle\otimes |j\rangle\oplus 0)]_{RB}
\otimes|j\rangle_{\text{ext}}, \nonumber
\end{eqnarray}
where here $k=\ln K$ is the number of qunats of quantum information 
in the in-fallen matter and where again we assume the initial
dimensionality of the black hole
interior is well approximated by its thermodynamic entropy, so
$N={\text{dim}}({\text{int}})=RB=e^{S_{\text{BH}}}$.
It will also be convenient to define
\begin{equation}
\chi \equiv S_{\text{BH}} - H_{\text{ext}} - k,
\end{equation}
which we shall here interpret as approximating the number of unentangled
qunats initially within the interior of the black hole. Note, that
$0\le H_{\text{ext}}\le S_{\text{BH}}-k$ for this model, so
$0\le \chi\le S_{\text{BH}}-k$.

It is now straightforward to apply the generalized decoupling
theorem above to show that the when the logarithm of the size
of the radiation subsystem reaches $S_{\text{BH}}-\frac{1}{2}\chi +c$,
the trans-event horizon entanglement has effectively vanished
and instead has been transferred to entanglement between the
external neighborhood modes and the outgoing radiation, with
a fidelity of at least $1-e^{-c}$. 
% Further, since the entropy of the in-fallen matter is tiny in
% comparison to the thermodynamic entropy of the black hole
% itself${}^{14}$,
Thus, unless the black hole entropy is primarily entropy of entanglement,
i.e., $\chi \lll S_{\text{BH}}$, there would be a manifest failure of
the equivalence principle long before the black hole evaporated to
the Planck scale.

We may also apply our generalized decoupling theorem to learn more
details about encoding and decoding into the quantum one-time pad
described in Ref.~\onlinecite{B09}. In particular, prior to the first
$\frac{1}{2}\chi -c$ qunats radiated, the information
about the in-fallen matter is still encoded solely within the
black hole interior, with a fidelity of at least $1-e^{-c}$. Similarly,
within the final $\frac{1}{2}\chi -c$ qunats radiated, the information
about the in-fallen matter is encoded within the out-going radiation,
with a fidelity of at least $1-e^{-c}$.

Now, with these results and those of Ref.~\onlinecite{B09}, we can
determine the encoding (and decoding) time for the black hole to
transform all the information about the in-fallen matter into its
quantum one-time pad. Both the encoding and decoding occur during the
radiation of $k+\frac{1}{2}(H_{\text{ext}}-\tilde H_{\text{ext}}) + 2c$
qunats, where $\tilde H_{\text{ext}}\equiv H^{(2)}(\rho_{\text{ext}})$
is another R\'enyi entropy. Since typically
$H_{\text{ext}}-\tilde H_{\text{ext}}\lesssim O(1)$, and this quantity
cannot become negative, this implies that the encoding (and decoding)
of the black hole's quantum one-time pad occurs at roughly the
radiation emission rate.

\vskip 0.1truein
\noindent
{\bf Black hole entropy sum rule?} We recall (see, e.g., T.\ Nishioka et al.
arXiv:0905.0932) that the entropy of entanglement of quantum fields
piercing a black hole's event horizon is expected to be proportional
to the number of matter fields, however, the black hole's thermodynamic
entropy is purely geometric. To preserve the equivalence principle,
therefore, our result would appear to imply that black hole entropy
provides a ``sum rule'' quantifying the number of matter fields. The
tools necessary to calculate such a sum rule in a cut-off independent
manner do not yet appear to be available.

\begin{thebibliography}{99}

\bibitem{Eisert09} J.\ Eisert, M.\ Cramer and M.\ B.\ Plenio,
Area laws for the entanglement entropy -- a review.
Rev.\ Mod.\ Phys.\ to appear, arXiv/0808.3773.

\bibitem{Hawking75} S.\ W.\ Hawking,
Particle creation by black holes.
Commun.\ Math.\ Phys.\ {\bf 43}, 199-220 (1975).

\bibitem{Parikh00} M.\ K.\ Parikh and F.\ Wilczek,
Hawking radiation as tunneling.
Phys.\ Rev.\ Lett.\ {\bf 85}, 5042-5045 (2000).

\bibitem{Page93} D.\ N.\ Page,
Information in black hole radiation.
Phys.\ Rev.\ Lett.\ {\bf 71}, 3743-3746 (1993);

\bibitem{Hayden07} P.\ Hayden and J.\ Preskill,
Black holes as mirrors: quantum information in random subsystems.
JHEP {\bf 2007}(09), 120 (2007).

\bibitem{B09} S.\ L.\ Braunstein, H.-J.\ Sommers and K.\ \.{Z}yczkowski,
Entangled black holes as ciphers of hidden information.
arXiv:0907.0739.

\bibitem{tHooft85} G.\ 't Hooft,
On the quantum structure of a black hole.
Nucl.\ Phys.\ B {\bf 256}, 727-745 (1985).

\bibitem{Bombelli86} L.\ Bombelli, R.\ K.\ Koul, J.\ Lee and R.\ D.\ Sorkin,
A quantum source of entropy for black holes.
Phys.\ Rev.\ D {\bf 34}, 373-383 (1986).

\bibitem{Srednicki93} M.\ Srednicki,
Entropy and area.
Phys.\ Rev.\ Lett.\ {\bf 71}, 666-669 (1993).

\bibitem{Hawking01} S.\ Hawking, J.\ Maldacena and A.\ Strominger,
DeSitter entropy, quantum entanglement and AdS/CFT.
JHEP {\bf 2001}(05), 001 (2001).

\bibitem{Brustein06} R.\ Brustein, M.\ B.\ Einhorn and A.\ Yarom,
Entanglement interpretation of black hole entropy in string theory,
JHEP {\bf 2006}(01), 098 (2006).

\bibitem{Emparan06} R.\ Emparan,
Black hole entropy as entanglement entropy: a holographic derivation.
JHEP {\bf 2006}(06), 012 (2006).

\bibitem{Abey06} A.\ Abeyesinghe, I.\ Devetak, P.\ Hayden and A.\ Winter,
The mother of all protocols: Restructuring quantum information's family
tree.
arXiv:quant-ph/0606225.

\end{thebibliography}

\end{document}

