% !TEX root = ReviewDraft.tex

\section{Comments on the AMPS paradox } 

In \cite{Almheiri:2012rt}  a  problem or paradox  was found,  and a proposal was made for its resolution.   Briefly stated, the paradox was the impossible quantum state appearing after the Page time, where the newly outgoing Hawking quantum needs to be maximally entangled with two seemingly separate systems: its interior partner and the early Hawking radiation. The proposed resolution was to declare the former entanglement broken, forming a ``firewall'' at the horizon.
A related problem was discussed in \cite{Marolf:2012xe}. 

The paradox involved the central dogma plus one extra   implicit assumption. 
The extra assumption is that the black hole interior can also be described by the {\it same} 
degrees of freedom that describe the black hole from the outside, the degrees of freedom that appear in the central dogma.
 We have not made this assumption in this review. 

According to this review,  the  paradox is resolved by dropping the assumption that the interior is also described by the same degrees of freedom that describe it as viewed from outside.     Instead,  we assume that only a portion of the interior is described by the black hole degrees of freedom appearing in the central dogma $-$   only the portion in the entanglement wedge, see figure \ref{EWfig}(b).  This leaves the interior as part of the radiation, and the resolution of the apparently impossible quantum state is that the interior partner is identified with part of the early radiation that the new Hawking quantum is entangled with.
This is different than the resolution proposed in AMPS. With this resolution, the horizon is smooth. 
 

\section{Glossary}  

{\bf Causal diamond}: The spacetime region that can be determined by evolution (into the future or the past) of initial data on any spatial region.  See figure \ref{Diamond}. \\
\\
{\bf Central dogma}: A black hole -- as viewed from the outside -- is simply a quantum system with a number of degrees of freedom equal to Area$/4G_N$. Being a quantum system, it evolves unitarily under time evolution. See section \ref{central}. \\
\\
{\bf Fine-grained entropy}: Also called the von Neumann entropy or quantum entropy. Given a density matrix $\rho$, the fine-grained entropy is given as $S = -Tr[\rho \log \rho]$. See section \ref{finecoarse}.\\
\\
{\bf Coarse-grained entropy}: Given a density matrix $\rho$ for our system, we measure a subset of simple observables $A_i$ and consider all $\tilde{\rho}$ consistent with the outcome of our measurements, $Tr[\tilde{\rho} A_i] = Tr[\rho A_i]$. We then maximize the entropy $S(\tilde{\rho}) = -Tr[\tilde{\rho} \log \tilde{\rho}]$ over all possible choices of $\tilde{\rho}$. See section \ref{finecoarse}. \\
\\
{\bf Semiclassical entropy}: The fine-grained entropy of matter and gravitons on a fixed background geometry. See section \ref{semiclassical}. \\
\\
{\bf Generalized entropy}: The sum of an area term and the semi-classical entropy.  See \eqref{sgendef}. When evaluated at an event horizon soon after it forms, for example in \eqref{sgen}, the generalized entropy is coarse grained. When evaluated at the extremum, as in \eqref{RT} or \eqref{island}, the generalized entropy is fine grained. \\
\\
{\bf Gravitational fine-grained entropy}: Entropy given by the formulas \nref{RT} and \nref{island}. They give the fine-grained entropy through a formula that involves a geometric part, the area term, and the semiclassical entropy of the quantum fields.  \\
\\
{\bf Page curve}: Consider a spacetime with a black hole formed by the collapse of a pure state. Surround the black hole by an imaginary sphere whose radius is a few Schwarzschild radii. The Page curve is a plot of the fine-grained entropy outside of this imaginary sphere, where we subtract the contribution of the vacuum. Since the black hole Hawking radiates and the Hawking quanta enter this faraway region, this computes the fine-grained entropy of Hawking radiation as a function of time. Notice that the regions inside and outside the imaginary sphere are open systems. The curve begins at zero when no Hawking quanta have entered the exterior region, and ends at zero when the black hole has completely evaporated and all of the Hawking quanta are in the exterior region. The ``Page time" corresponds to the turnover point of the curve. See figure \ref{HawkingPageCurves}.\\
\\
{\bf Quantum extremal surface}: The surface $X$ that results from extremizing (and if necessary minimizing) the generalized entropy as in \eqref{RT}. This same surface appears as a boundary of the island region in \eqref{island}. \\
\\
{\bf Island}: Any  disconnected codimension-one regions found by the extremization procedure \eqref{island}. Its boundary is the quantum extremal surface. The causal diamond of an island region is a part of the entanglement wedge of the radiation.\\
\\
{\bf Entanglement wedge}: For a given system (in our case either the radiation or the black hole), the entanglement wedge is a region of the semiclassical spacetime that is described by the system.  It is defined at a moment in time and has nontrivial time dependence. Notice that language is not a good guide: the transition in the Page curve from increasing entropy to decreasing entropy corresponds to when most of the interior of the black hole becomes described by the radiation, i.e. the entanglement wedge of the black hole degrees of freedom does not include most of the black hole interior. See section \ref{wedge} and figure \ref{EWfig}. \\
\\
{\bf Replica trick}: A mathematical technique used to compute $-Tr[\rho \log \rho]$ in a situation where we do not have direct access to the matrix $\rho_{ij}$. See section \ref{replicas}.\\
\\



  