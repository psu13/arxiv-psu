% !TEX root = ReviewDraft.tex


 
In this review we discuss some recent progress on aspects of the black hole information paradox. 

Before delving into it let us discuss a big picture motivation. 
One of the main motivations to study quantum gravity is to understand the 
earliest moments of the universe, where we expect that quantum effects are dominant. 
In the search for this theory, it is better to consider simpler problems. A simpler problem involves black holes. They also contain a singularity in  their interior. It is an anisotropic big crunch singularity, but it is also a situation where quantum gravity is necessary, making it difficult to analyze. Black holes, however, afford us the opportunity to study them as seen from the outside. This is simpler because far from the black hole we can neglect the effects of gravity and we can imagine asking sharp questions probing the black hole from far away. One of these questions will be the subject of this review. We hope that, by studying these questions, we will eventually understand the black hole singularity and learn some lessons for the big bang, but we will not do that here. 

Studies of black holes in the '70s showed that black holes behave as thermal objects. They have a temperature that leads to Hawking radiation. They also have an entropy given by the area of the horizon. This suggested that, from the point of view of the outside, they could be viewed as an ordinary quantum system.   
Hawking objected to this idea through what we now know as the ``Hawking information paradox." He argued that a black hole would destroy quantum information, and that the von Neumann entropy of the universe would increase by the process of black hole formation and evaporation. 
 Results from the '90s using string theory, a theory of quantum gravity, provided some precise ways to study this problem for very specific gravity theories. These results strongly suggest that information does indeed come out. However, the current understanding requires certain dualities to quantum systems  where the geometry of spacetime is not manifest. 
 
 During the past 15 years, a better understanding of the von Neumann entropy for gravitational systems was developed.   The computation of the entropy involves also an area of a surface, but the surface is not the horizon. It is a surface that minimizes the generalized entropy. This formula is almost as simple as the Bekenstein formula for black hole entropy \cite{Bekenstein:1972tm,Bekenstein:1973ur}. 
  More recently, this formula was applied to the black hole information problem, giving a new way to compute the entropy of Hawking radiation \cite{Penington:2019npb,Almheiri:2019psf}. The final result differs from Hawking's result and is consistent with unitary evolution.  

  


The first version of the fine-grained entropy formula was discovered by Ryu and Takayanagi \cite{Ryu:2006bv}. It was subsequently refined and generalized by a number of authors \cite{Hubeny:2007xt,Lewkowycz:2013nqa,Barrella:2013wja,Faulkner:2013ana,Engelhardt:2014gca,Almheiri:2019psf,Penington:2019npb,Almheiri:2019hni}. Originally, the Ryu-Takayanagi formula was proposed to calculate holographic entanglement entropy in anti-de Sitter spacetime, but the present understanding of the formula is much more general. It requires neither holography, nor entanglement, nor anti-de Sitter spacetime. Rather it is a general formula for the fine-grained entropy of quantum systems coupled to gravity.
  
  
    
  Our objective is to review these results for people with minimal background in this problem. We will not follow a historical route but rather try to go directly to the final formulas and explain how to use them. For that reason,  we will not discuss many related ideas that served as motivation, or that are also very useful for the general study of quantum aspects of black holes. A sampling of related work includes  \cite{Zhao:2019nxk, Akers:2019nfi, Rozali:2019day, Chen:2019uhq, Bousso:2019ykv, Almheiri:2019psy, Chen:2019iro, Laddha:2020kvp, Mousatov:2020ics, Kim:2020cds, Saraswat:2020zzf, Chen:2020wiq, Marolf:2020xie, Verlinde:2020upt, Giddings:2020yes, Liu:2020gnp, Pollack:2020gfa, Balasubramanian:2020hfs, Gautason:2020tmk,Anegawa:2020ezn, Hollowood:2020cou, Krishnan:2020oun, Banks:2020zrt, Geng:2020qvw}.   

Many details and caveats will necessarily be swept under the rug, although we will discuss some potential technical issues in the discussion section. We believe the caveats are only technical and unlikely to change the basic picture.
 
 
