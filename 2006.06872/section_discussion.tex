% !TEX root = ReviewDraft.tex

\section{Discussion} 
\la{Discussion} 

\subsection{Short summary} 

Let us summarize some of the points we made in the review. 
First we discussed classic results in black hole thermodynamics, including Hawking radiation and black hole entropy. The entropy of the black hole is given by the area of the horizon plus the entropy of the quantum fields outside. 
We discussed how these results inspired a central dogma which says that a black hole from the outside can be described in terms of a quantum system with a number of degrees of freedom set by the entropy. 
Next we discussed a formula for the fine-grained entropy of the black hole which involves finding a surface that minimizes the area plus the entropy of quantum fields outside.
Using this formula,  we computed the entropy for an evaporating black hole and found that it follows the Page curve. Then we discussed how to compute the entropy of radiation. The gravitational fine-grained entropy formula tells us that we should include the black hole interior and it gives a result that follows the Page curve, too. 
These results suggest that the black hole degrees of freedom describe a portion of the interior, the region inside the entanglement wedge. 
Finally we discussed how replica wormholes explain why the interior should be included in the computation of the entropy of radiation.  

\subsection{Comments and problems for the future } 

It is important to point out the following important feature of the gravitational entropy formulas, both the coarse-grained and the fine-grained one. Both formulas involve a geometric piece, the area term, which does not obviously come from taking a trace over some explicit microstates. The interpretation of these quantities as arising from sums over microstates is an assumption, a part of the ``central dogma," which is the simplest way to explain the emergence of black hole thermodynamics, and  has strong evidence from string theory.

For this reason, the success in reproducing the Page curve does not translate into a formula for individual matrix elements of the density matrix.  The geometry is giving us the correct entropy, which involves a trace of (a function of) the density matrix. 
Similarly we do not presently know how to extract 
  individual matrix elements of the black hole S-matrix, which describes individual transition amplitudes for each microstate. Therefore the current discussion leaves an important problem unresolved. Namely, how do we compute individual matrix elements of the S-matrix, or $\rho$, directly from the gravity description (without using a holographic duality)? In other words, we have discussed how to compute the entropy of Hawking radiation, but not how to compute its precise quantum  state. This is an important aspect of the black hole information problem, since one way of stating the problem is: Why do different initial states lead to the same final state? In this description the different initial states correspond to different interiors. In gravity, we find that the final state for radiation also includes the interior.  
  The idea is that very complex computations in the radiation can create wormholes that reach into that interior and pull out the information stored there \cite{Penington:2019kki}, 
  see also \cite{Maldacena:2013xja,Gao:2016bin}.

 
The present derivations for the 
  gravitational fine-grained entropy formulas discussed in this paper rely on the Euclidean path integral. It is not clear how this is defined precisely in gravity. For example, which saddle points should we include? What is the precise integration contour? It is possible that some theories of gravity include replica wormhole saddles and black holes evaporate unitarily, while in other theories of gravity they do not contribute to the path integral, the central dogma fails, and Hawking's picture is accurate. (We suspect that the latter would not be fully consistent theories.)

 
Another aspect of the formulas which is not yet fully understood is the imaginary cutoff surface, beyond which we treated spacetime as fixed. This is an important element in the derivation of the formula \eqref{island}  as discussed in section \ref{replicas}.   
 A more complete understanding will require allowing gravity to fluctuate everywhere throughout spacetime. For example, we do not know whether the central dogma applies when the cutoff is at a finite distance from the black hole, or precisely how far we should go in order to apply these formulas. The case that is best understood is when this cutoff is at the boundary of an AdS space.   On the other hand, the imaginary cutoff surface is not as drastic as it sounds because the same procedure is required to make sense of the ordinary Gibbons-Hawking entropy in asymptotically flat spacetime.



 
Note that when we discussed the radiation, we had  two quantum states in mind. First we had the semiclassical state, the state of radiation that appears when we use the semiclassical geometry of the evaporating black hole. Then we had the exact quantum state of radiation. This is the state that would be produced by the exact and complete theory of quantum gravity. Presumably, to obtain this state we will need to sum over all geometries, including non-perturbative corrections. This is something that we do not know how to do in any theory of gravity complicated enough to contain quantum fields describing Hawking radiation. (See however \cite{Saad:2019lba,Penington:2019kki} for some toy models.) The magic of the gravitational fine-grained entropy formula is that it gives us the entropy of the exact state in terms of quantities that can be computed using the semiclassical state. One could ask, if you are an observer in the radiation region, which of these two states should you use? If you do simple observations, the semiclassical state is good enough. But if you consider very complex observables, then you need to use the exact quantum state. One way to understand this is that very complex operations on the radiation weave their own spacetime, and this spacetime can develop a connection to the black hole interior. See \cite{Susskind:2018pmk} for more discussion. 
  

  This review has focused on novel physics in the presence of black hole event horizons. In our universe, we also have a cosmological event horizon due to accelerated expansion. This horizon is similar to a black hole horizon in that it has an associated Gibbons-Hawking entropy and it Hawking radiates at a characteristic temperature 
\cite{Figari:1975km,Gibbons:1977mu}. 
However, it is unclear whether we should think of the cosmological horizon as a quantum system in the sense of the central dogma for black holes. Applying the ideas developed in the previous sections to cosmology may shed light on the nature of these horizons and the quantum nature of cosmological spacetimes. 

   
   There is a variant of the black hole information problem where one perturbs the black hole and then looks at the response at a very late time in the future \cite{Maldacena:2001kr}.  For recent progress in that front see \cite{Saad:2018bqo,Saad:2019lba,Saad:2019pqd}.  

   
     
      Wormholes similar to the ones discussed here were considered in the context of theories with random couplings \cite{Coleman:1988cy,Giddings:1988cx,Polchinski:1994zs}. Recently, random couplings played an important role in the solution of a simple two dimensional gravity theory \cite{Saad:2018bqo,Saad:2019lba}. 
      We do not know to what extent random couplings are important for the issues we discussed in this review. See also \cite{Marolf:2020xie}.
        
  

 
 
 We should emphasize one point. In this review,  we have presented results that can be understood purely in terms of gravity as an effective theory. However, string theory and holographic dualities played an  instrumental role in inspiring and checking these results. They provided concrete examples where these ideas were tested and developed,  before they were applied to the study of black holes in general. 
 Also, as we explained in the beginning, we have not followed a  historical route and we have not reviewed ideas that have led to the present status of understanding.    
   
  Finally, we should finish with a bit of a cautionary tale. Black holes are very confusing and many researchers who have written papers on them have gotten some things right and some wrong.  What we have discussed in this review is an {\it interpretation} of some geometric gravity computations. We interpreted them in terms of entropies of quantum systems. It could well be that our interpretation will have to be revised in the future, but we have strived to be conservative and to present something that is likely to stand the test of time. 

    
    

 
 
 
A goal of quantum gravity is to understand what spacetime is made of. The fine-grained entropy formula is giving us very valuable  information on how the fundamental quantum degrees of freedom are building the spacetime geometry. These studies have involved the merger and ringdown of several different fields of physics over the last few decades: high energy theory, gravitation, quantum information, condensed matter theory, etc.,  creating connections beyond their horizons. This has not only provided exciting insights into the quantum mechanics of black holes, but also turned black holes into a light that illuminates many questions of these other fields. Black holes have become a veritable source of information!
    
 
 
