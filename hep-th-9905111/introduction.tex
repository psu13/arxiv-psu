
\section{General Introduction and Overview}
\label{introduction}

The microscopic description of nature as presently
understood and verified by experiment involves 
quantum field theories. All particles are excitations of 
some field. These particles are pointlike and they 
interact locally with other particles. 
Even though quantum field theories 
 describe nature at the distance scales we observe,
there are strong indications that new
elements will be involved 
 at very short distances (or very high energies), distances
of the order of the
Planck scale. 
The reason is that at those distances (or energies) quantum gravity 
effects become important. It has not been possible to quantize
gravity following the usual perturbative methods. 
Nevertheless, one
 can incorporate quantum gravity in a consistent quantum 
theory by giving up 
the notion that particles are pointlike and assuming that 
the fundamental objects in the theory are strings, namely one-dimensional
extended objects \cite{Green:1987sp,joebook}.
 These strings can oscillate, and there is 
a spectrum of energies, or masses, for these oscillating strings.
The oscillating strings  look like localized, particle-like
excitations to a low energy observer.  
So, a single oscillating string can effectively give rise to many
types of
particles, depending on its state of oscillation. All string theories
include a particle with zero mass and spin two. 
Strings can interact by splitting and joining interactions.
The only consistent interaction for massless spin two 
particles is that  of gravity. Therefore, 
any string theory will contain gravity. 
The structure of string theory is highly constrained. String theories
do not make sense in an arbitrary number of dimensions or on any 
arbitrary geometry. Flat space string theory exists (at least in
perturbation theory) only in 
ten dimensions. Actually, 10-dimensional string theory is described
by a string which also has fermionic excitations and gives rise
to a supersymmetric theory.\footnote{One could consider a  string with no
fermionic excitations, the so called ``bosonic'' string. It  lives in
26 dimensions and contains tachyons, signaling an instability of the
theory.}
String theory is then a candidate for a quantum theory of gravity.
One can get down to four dimensions by considering string theory on 
$\IR^4\times M_6$ where $M_6$ is some six dimensional compact manifold. 
Then, low energy interactions are determined by the geometry
of $M_6$. 

Even though this is the motivation usually given for string theory
nowadays, it is not how string theory was originally discovered. 
String theory was discovered in an attempt to describe 
the large number of mesons and hadrons that were experimentally 
discovered in the 1960's. The idea was to view all these particles 
as different oscillation modes of a string. 
The string idea described well some features of the 
hadron spectrum. For example,
 the mass of the lightest hadron with a given spin 
obeys a  relation like $m^2 \sim T J^2 + const $. This is explained 
simply by assuming that the mass and angular momentum come
from a rotating, relativistic string of tension $T$. 
It was later discovered that hadrons and mesons  are actually
made of quarks
and that they are described by QCD. 

QCD is a gauge theory based
on the group $SU(3)$. This is sometimes stated by saying that quarks
have three  colors. QCD is asymptotically free, meaning that the 
effective
coupling constant decreases as the energy increases. At low energies
QCD becomes strongly coupled and it is not easy to perform calculations. 
One possible approach is to use numerical simulations on the lattice.
This is at present the best available tool to do calculations in 
QCD at low energies. It was suggested by 't Hooft that 
the theory might simplify when the number of colors $N$ is large
 \cite{'tHooft:1974jz}.
The hope was that one could solve exactly the theory with 
$N = \infty$, and then one could do an expansion in $1/N = 1/3$. 
Furthermore, as explained in the next section, the diagrammatic expansion
of the field theory suggests that the large $N$  theory 
is a free string theory and that the string coupling constant is $1/N$.  
If the case with $N=3$ is similar to the case with $N=\infty$
then this explains why the string model gave the correct relation 
between the mass and the angular momentum. In this way 
the large $N$ limit connects gauge  theories with string theories. 
The 't Hooft argument, reviewed below, is very general, so it suggests
that different kinds of gauge theories will correspond to different 
string theories. In this review we will study this correspondence between
string theories and the large $N$ limit of field theories. We 
will see that the strings arising in the large $N$ limit of field 
theories are the same as the strings describing quantum gravity.
Namely, string theory in some backgrounds, including quantum gravity,
is equivalent (dual) to a field theory.

We said above that  strings are not consistent in four flat
dimensions. Indeed, if one wants to quantize a four dimensional
string theory an anomaly appears that forces the introduction 
of an extra field, sometimes called the 
``Liouville'' field \cite{Polyakov:1981rd}. 
This field on the string worldsheet may be interpreted as an extra 
dimension, so that the
strings effectively move in five dimensions. One might
qualitatively think of this new field as the ``thickness'' of the string.
If this is the case, why do we say that the string  moves in 
five dimensions? The reason is that, like any string theory, this 
theory will
contain gravity, and the gravitational theory will live in as
many dimensions as the number of fields we have on the string. 
It is crucial then that the five dimensional geometry is curved, so that
it can correspond to a four dimensional field theory, as described in
detail below. 

The argument that gauge theories are related to string theories in the 
large $N$ limit is very general and is
valid for basically any gauge theory.
In particular we could consider a gauge theory where the coupling
does not run (as a function of the energy scale). 
Then, the theory is  conformally invariant.
It is quite hard to find quantum field theories that are conformally
invariant. In supersymmetric theories it is sometimes possible to 
prove exact conformal invariance. A simple example, which will 
be the main example
 in this review, is the supersymmetric $SU(N)$ (or $U(N)$)
gauge theory in four
dimensions
with four spinor supercharges ($\cn=4$).  
Four is the maximal possible number of supercharges for a field theory
in four dimensions. Besides the gauge fields (gluons) 
this theory contains also four
fermions and six scalar fields in the adjoint representation of the
gauge group. 
The Lagrangian of such theories is completely 
determined by supersymmetry. There is a global $SU(4)$ $R$-symmetry that
rotates the six scalar fields and the four fermions.   
 The conformal group in four dimensions is $SO(4,2)$, including 
the usual \Poincare transformations as well as scale
transformations and special conformal transformations (which include
the inversion symmetry
$x^\mu \to x^\mu/x^2$). 
These symmetries of the field theory should be reflected in the dual
string theory. The simplest way for this to happen is if 
the five dimensional geometry has
these symmetries. Locally there is only one space 
with $SO(4,2)$ isometries: five dimensional
Anti-de-Sitter space, or $AdS_5$. Anti-de Sitter space is the maximally 
symmetric solution of Einstein's equations with a negative
cosmological constant.  
In this supersymmetric case we expect the strings to also be
 supersymmetric. We said that 
 superstrings move in ten dimensions. Now that
we have added one more dimension it is not surprising any more
to add five more to get to a ten dimensional space. 
Since the gauge theory has an $SU(4)\simeq SO(6)$ global symmetry 
 it is rather natural that the extra five dimensional space
should be a five sphere, $S^5$. So, we conclude that 
${\cal N}=4$ $U(N)$ Yang-Mills theory
could be the same as ten dimensional superstring
theory on $AdS_5 \times S^5 $ \cite{Maldacena:1997re}.
Here we have presented a very 
heuristic argument for this equivalence; later we will be more precise
and give more evidence for this correspondence. 

The relationship we described between gauge theories and string theory 
on Anti-de-Sitter
spaces was motivated by studies of D-branes and black holes in
strings theory. D-branes are solitons in string theory
 \cite{Polchinski:1995mt}. 
They come in various dimensionalities. If they have zero spatial
dimensions they are like ordinary localized, particle-type soliton
solutions, analogous to the 't Hooft-Polyakov 
\cite{'tHooft:1974qc,Polyakov:1974ek} 
monopole in gauge theories. These are called D-zero-branes. 
If they have one extended dimension they are called D-one-branes
or D-strings. They are much heavier than ordinary fundamental strings
when the string coupling is small. In fact, the tension of all D-branes
is proportional to $1/g_s$, where $g_s$ is the string coupling constant.
D-branes are defined in string perturbation theory in a very simple
way: they are surfaces where open strings can end. These open
strings have some massless modes, which describe the oscillations 
of the branes, a gauge field living on the brane, and their
fermionic partners. If we have $N$ coincident branes the open strings
can start and end on different branes, so they carry two indices
that run from one to $N$. 
This in turn implies that the low energy dynamics is 
described by a $U(N)$ gauge theory. 
 D-$p$-branes are charged under $p+1$-form gauge potentials, in the
same way that a 0-brane (particle) can be charged under 
a one-form gauge potential (as in electromagnetism). 
These $p+1$-form gauge potentials have $p+2$-form field strengths, and
they are part of the massless closed
string modes, which belong to the supergravity (SUGRA) multiplet 
containing the massless fields in flat
space string theory (before
we put in any D-branes). If we now add D-branes they generate a 
flux of the corresponding field strength, and this flux in turn 
contributes to the stress energy tensor so the geometry 
becomes curved. Indeed it is possible to find solutions of
the supergravity equations carrying these fluxes. 
Supergravity is the low-energy limit of string theory, and it is
believed that these solutions may be extended to solutions of the full
string theory. These solutions
are very similar to extremal charged black hole solutions in general
relativity, except that in this case they are black branes
with $p$ extended spatial dimensions. Like black holes they contain
event horizons. 

If we consider a set of $N$ coincident 
D-3-branes the near horizon geometry turns out to be
$AdS_5\times S^5$. On the other hand, the low energy 
dynamics on their worldvolume is governed by a $U(N)$ gauge theory
with ${\cal N} =4$ supersymmetry \cite{Witten:1996im}. 
These two pictures of D-branes are perturbatively valid for
different regimes in the space of possible coupling constants. Perturbative
field theory is valid when $g_s N$ is small, 
while the low-energy gravitational description is perturbatively 
valid when
the radius of curvature is much larger than the string scale,
which turns out to imply that $g_s N$ should be very large. 
As an object is brought closer and closer to the 
 black brane  horizon
its energy measured by an outside observer is redshifted, due to the
large gravitational potential, and the energy seems to be very small. 
On the other hand low energy excitations on the branes are governed
by the Yang-Mills theory. So, it becomes natural to conjecture that 
Yang-Mills theory at strong coupling is describing
the near horizon region of the black brane, whose geometry
is $AdS_5\times S^5$.
The first indications that this is the case came from calculations
of low energy graviton absorption cross sections 
\cite{Klebanov:1997kc,Gubser:1997yh,Gubser:1997se}. It was
noticed there that the calculation done using gravity and the 
calculation done using super Yang-Mills theory agreed. 
These calculations, in turn, were inspired by similar calculations
 for coincident D1-D5 branes. In this case the
near horizon geometry involves $AdS_3\times S^3$ and the low energy
field theory living on the D-branes is a 1+1 dimensional conformal
field theory.
In this D1-D5 case 
there were numerous calculations that agreed between the field theory
and gravity. First black hole entropy for extremal black holes was
calculated in terms of the field theory in \cite{Strominger:1996sh},
and then 
agreement was shown for near extremal black holes 
\cite{Callan:1996dv,Horowitz:1996fn} and
for absorption cross sections 
\cite{Das:1996wn,Dhar:1996vu,Maldacena:1997ix}. 
More generally, we will see that
correlation functions in the gauge theory can be calculated 
using the string theory (or gravity for large $g_s N$)
description, by considering the propagation
of particles between different points in the boundary of $AdS$, 
the points where operators are inserted 
\cite{Gubser:1998bc,Witten:1998qj}. 

Supergravities on $AdS$ spaces  were 
 studied very  extensively,
see 
\cite{Salam:1989fm,Duff:1986hr} for reviews. See also
\cite{Boonstra:1997dy,Sfetsos:1998xs} for earlier hints of the
 correspondence.
 

One of the main points of this review 
will be that the strings coming from gauge theories
are very much like the ordinary superstrings that have been 
studied during the last 20 years. The only particular feature is that
they are moving on a curved geometry 
(anti-de Sitter space) which has a boundary at spatial infinity. 
The boundary is at an infinite spatial distance, but a light ray 
can go to the boundary and come back in finite time. Massive particles
can never get to the boundary. The radius of curvature of 
Anti-de Sitter space 
 depends on $N$  so  that large $N$ corresponds to a large radius
of curvature. Thus, by taking $N$ to be large 
we can make the curvature as small as we want. 
The theory in $AdS$ includes gravity, since any string theory 
includes gravity. So in the end we claim that there is an equivalence
between a gravitational 
theory and a field theory. However, the mapping between the gravitational
and field theory degrees of freedom is quite non-trivial since
the field theory lives in a lower dimension. In some sense the 
field theory (or at least the set of 
local observables in the field theory)
lives on the boundary of spacetime. 
One could argue that in general any quantum gravity theory in $AdS$
defines a conformal field theory (CFT) ``on the boundary''. 
In some sense the situation is similar to the correspondence
between three dimensional Chern-Simons theory and a WZW model on
the boundary \cite{Witten:1989hf}. This is a topological theory in three 
dimensions that induces a normal (non-topological) 
 field theory on the boundary.
A theory which includes gravity
is in some sense topological since one is integrating
over all metrics and therefore the theory does not depend on the
metric. Similarly, in a quantum gravity theory we do not
have any local observables. 
Notice that when we say that the theory includes
``gravity on $AdS$'' we are considering
any finite energy excitation, even black holes in $AdS$. 
So this is really a sum over all spacetimes that are asymptotic to 
$AdS$ at the boundary. This is analogous to the usual 
flat space discussion of quantum gravity,
where asymptotic flatness is required, but
the spacetime could have any topology as long as it is asymptotically 
flat. The asymptotically $AdS$ case as well as the asymptotically 
flat cases are special in the sense that one can choose a natural
time and an associated Hamiltonian to define the quantum theory. 
Since black holes might be present this time coordinate is not 
necessarily
globally well-defined, but it is certainly well-defined at infinity. 
If we assume that the conjecture we made above is valid, then the $U(N)$ 
Yang-Mills theory 
gives a non-perturbative definition of string theory on $AdS$.
And, by taking the limit $N\to \infty$, we can extract the (ten
dimensional string theory) flat space
physics, a procedure  which is in principle (but not in detail) 
similar to the one used in matrix theory \cite{Banks:1997vh}. 

The fact that the field theory lives in a lower dimensional space 
blends in perfectly with some previous speculations about quantum 
gravity. It was suggested \cite{'tHooft:1993gx,Susskind:1995vu}
 that quantum gravity theories
should be holographic, in the sense that physics in some region 
can be described by a theory at the boundary with no more than 
one degree of freedom per Planck area. This ``holographic'' 
principle comes from thinking about the Bekenstein bound which 
states that the maximum amount of entropy in some region 
is given by the area of the region in Planck units 
\cite{Bekenstein:1994dz}. 
The reason for this bound is that otherwise black hole 
formation could violate the second law of thermodynamics.
We will see that the correspondence between field theories
and string theory on $AdS$ space (including gravity) 
is a concrete realization of this holographic principle. 

The review is organized as follows. 

In the rest of the introductory chapter, we present background material.
In section 1.2, we present the   
't Hooft large $N$ limit and its indication that gauge theories may be 
dual to string theories.
In section 1.3, we review the $p$-brane supergravity solutions.
We discuss D-branes, their worldvolume theory and their relation to the p-branes.
We discuss greybody factors and their calculation for
black holes built out of D-branes.  

In chapter 2, we review conformal field theories and $AdS$ spaces.
In section 2.1, we give a brief description of conformal field theories.
In section 2.2, we summarize
the geometry of $AdS$ spaces and gauged supergravities.

In chapter 3,
we ``derive'' the correspondence between supersymmetric
Yang Mills theory and string theory on 
$AdS_5 \times S^5$ from the physics of D3-branes in
string theory. We define, in section 3.1, the correspondence 
between fields in 
the string theory and operators of the conformal field theory and
the prescription for the computation of correlation functions.
We also point out that the correspondence gives an explicit
holographic description of gravity. 
In section 3.2,
we review the direct tests of the duality, including
matching the spectrum of chiral primary operators and some correlation functions
and anomalies. Computation
of correlation functions is reviewed in section 3.3.
The isomorphism of the Hilbert spaces of string theory on $AdS$ spaces and 
of CFTs is decribed in section 3.4.
We describe
how to introduce Wilson loop operators in section 3.5. 
In section 3.6, we analyze  finite temperature theories 
and the thermal phase transition.

In chapter 4, we review other topics involving $AdS_5$. 
In section 4.1, we consider some
other gauge theories that arise from D-branes at orbifolds, orientifolds,
or conifold points. In section 4.2,
we review  how baryons and instantons arise in the string theory
description. In section 4.3,
we study  some deformations of the CFT and how they arise in the
string theory description. 

In chapter 5, we describe a similar correspondence involving
1+1 dimensional CFTs and $AdS_3$ spaces. We also describe 
the relation of these results to black holes in five dimensions. 

In chapter 6, we consider  other examples of the 
AdS/CFT correspondence as well as non conformal and non supersymmetric
cases. In section 6.1, we analyse the M2 and M5 branes theories, 
and go on to describe situations that are not conformal, 
realized on the worldvolume
of various Dp-branes, and the ``little string theories'' on the
worldvolume of NS 5-branes.
In section 6.2, we describe an approach to studying theories
that are  confining and have a behavior similar
to QCD in three and four dimensions. We discuss confinement, $\theta$-vacua, 
the mass spectrum and other dynamical aspects of these theories. 

Finally, the last chapter is devoted to a summary and discussion.

Other reviews  of this subject are 
\cite{DiVecchia:1999du,Douglas:1999ww,Petersen:1999zh,Klebanov:1999ku}.







% \end{document}
