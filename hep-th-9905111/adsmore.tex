
\section{Other Branes} 
\label{adsmore}

\subsection{M5 Branes}
\label{m5branes}

There exist six dimensional $\cn=(2,0)$ SCFTs, which have sixteen
supercharges, and are expected to be non-trivial isolated fixed points
of the renormalization group in six dimensions (see
\cite{Seiberg:1997ax} and references therein).  As a consequence, they
have neither dimensionful nor dimensionless parameters.  These
theories have an $Sp(2) \simeq SO(5)$ R-symmetry group.

The $A_{N-1}$ $(2,0)$ theory is realized as the low-energy theory on
the worldvolume of $N$ coincident M5 branes (five branes of M theory).
The $\cn=(2,0)$ supersymmetry algebra includes four real spinors of the
same chirality, in the {\bf 4} of $SO(5)$.  Its only irreducible
massless matter representation consists of a 2-form $B_{\mu\nu}$ with
a self-dual field strength, five real scalars and fermions. It is
called a tensor multiplet. 
For a single 5-brane the five real scalars in the tensor multiplet
define the embedding of the M5 brane in eleven dimensions.  The
R-symmetry group is the rotation group in the five dimensions
transverse to the M5 worldvolume, and it rotates the five scalars.
The low-energy theory on the moduli space
of flat directions includes $r$ tensor multiplets (where for the
$A_{N-1}$ theories $r=N-1$). The moduli space is parametrized by the
scalars in the tensor multiplets.  It has orbifold singularities (for
the $A_{N-1}$ theory it is $\IR^{5(N-1)}/S_N$) and the theory at the
singularities is superconformal.  The self-dual 2-form $B_{\mu\nu}$
couples to self-dual strings. At generic points on the moduli space
these strings are BPS saturated, and at the superconformal point their
tension goes to zero.

The $A_{N-1}$ $(2,0)$ superconformal theory has a Matrix-like DLCQ
description as quantum mechanics on the moduli space of $A_{N-1}$
instantons \cite{Aharony:1997md}.  In this description the chiral
primary operators are identified with the cohomology with compact
support of the resolved moduli space of instantons, which is localized
at the origin
\cite{Aharony:1998lc}.  Their lowest components are scalars 
in the symmetric traceless
representations of the $SO(5)$ R-symmetry group.

The eleven dimensional supergravity metric describing $N$ M5 branes is
given by\footnote{Our conventions are such that the tension of the 
M2 brane is $T_2 = 1 / (2 \pi)^2 l_p^3$. }
\beqar
ds^2 &=& f^{-1/3}(-dt^2 + \sum_{i=1}^5 dx_i^2) + 
f^{2/3}(dr^2 + r^2 d \Omega_4^2) \ , \nonumber\\
f &=& 1 + \frac{\pi N l_p^3}{r^3} \ ,
\label{M5}
\eeqar
and there is a 4-form flux of $N$ units on the $S^4$.

The near horizon geometry of (\ref{M5}) is of the form $AdS_7 \times
S^4$ with the radii of curvature $R_{AdS} = 2R_{S^4} = 2 l_p (\pi
N)^{1/3}$. Note that since $R_{AdS} \neq R_{S^4}$ this background is
not conformally flat, unlike the $AdS_5\times S^5$ background
discussed above. Following similar arguments to those of 
section \ref{correspondence} 
leads to the conjecture that the $A_{N-1}$ $(2,0)$ SCFT is dual to M
theory on $AdS_7 \times S^4$ with $N$ units of 4-form flux on $S^4$
\cite{Maldacena:1997re}.

The eleven dimensional supergravity description is applicable for
large $N$, since then the curvature is small in Planck units.
Corrections to supergravity will go like positive powers of
$l_p/R_{AdS} \sim N^{-1/3}$; the supergravity action itself is of
order $M_p^9 \sim N^3$ (instead of $N^2$ in the $AdS_5\times S^5$
case). The known corrections in M theory are all positive powers of
$l_p^3 \sim 1/N$, suggesting that the $(2,0)$ theories have a $1/N$
expansion at large $N$.  The bosonic symmetry of the supergravity
compactification is $SO(6,2)
\times SO(5)$. The $SO(6,2)$ part is the conformal group of the SCFT,
and the $SO(5)$ part is its R-symmetry.
  
The Kaluza-Klein excitations of supergravity contain particles with
spin less than two, so they fall into small representations of
supersymmetry.  Therefore, their masses are protected from quantum
(M theory) corrections.  As in the other examples of the duality, these
excitations correspond to chiral primary operators of the $A_{N-1}$
$(2,0)$ SCFT, whose scaling dimensions are protected from quantum
corrections.  The spectrum of Kaluza-Klein harmonics of supergravity
on $AdS_7 \times S^4$ was computed in \cite{Nieuwenhuizen:1985tc}.
The lowest components of the SUSY multiplets are scalar fields with
%re is one family of
%scalar excitations that contains states also with negative and zero
%mass squared corresponding to relevant and marginal operators of the
%SCFT
\beq
m^2 R_{AdS}^2 = 4k(k-3),~~~~k=2,3,\cdots \ .
\label{massfor}
\eeq
They fall into the $k$-th order symmetric traceless representation of
$SO(5)$ with unit multiplicity.  The $k=1$ excitation is the singleton
that can be gauged away except on the boundary of $AdS$.  It decouples
from the other operators and can be identified with the free ``center
of mass'' tensor multiplet on the field theory side.

Using the relation between the dimensions of the operators $\Delta$
and the masses $m$ of the Kaluza-Klein excitations $m^2 R_{AdS}^2 =
\Delta(\Delta-6)$, the dimensions of the corresponding operators in
the SCFT are $\Delta=2k,~~k=2,3,\cdots$
\cite{Aharony:1998mt, Minwalla:1998po,Leigh:1998tl,Halyo:1998so}.
These are the dimensions of the chiral primary operators of the
$A_{N-1}$ $(2,0)$ theory as found from the DLCQ
description\footnote{The DLCQ description corresponded to the theory
including the free tensor multiplet, so it included also the $k=1$
operator.}. The expectation values of these operators parametrize the
space of flat directions of the theory, $(\IR^5)^{N-1}/S_N$. The
dimensions of these operators are the same as the naive dimension of
the product of $k$ free tensor multiplets, though there is no good
reason for this to be true (unlike the $d=4$ $\cn=4$ theory, where the
dimension had to be similar to the free field dimension for small
$\lambda$, and then for the chiral operators it could not change as we
vary $\lambda$). For large $N$, the $k=2$ scalar field with $\Delta=4$
is the only relevant deformation of the SCFT and it breaks the
supersymmetry.  All the non-chiral fields appear to have large masses
in the large $N$ limit, implying that the corresponding operators have
large dimensions in the field theory.

The spectrum includes also a family of spin one Kaluza-Klein
excitations that couple to 1-form operators of the SCFT.  The massless
vectors in this family couple to the dimension five R-symmetry
currents of the SCFT.  The massless graviton couples to the
stress-energy tensor of the SCFT.  As in the $d=4$ $\cN=4$ case, the
chiral fields corresponding to the different towers of Kaluza-Klein
harmonics are related to the scalar operators associated with the
Kaluza-Klein tower (\ref{massfor}) by the supersymmetry algebra.  For
each value of (large enough) $k$, the SUSY multiplets include one
field in each tower of Kaluza-Klein states.  Its $SO(5)$
representation is determined by the representation of the scalar
field.  For instance, the R-symmetry currents and the energy-momentum
tensor are in the same supersymmetry multiplet as the scalar
field corresponding to $k=2$ in equation (\ref{massfor}).

As we did for the D3 branes in section \ref{other_backgrounds}, we can place
the M5 branes at singularities and obtain other dual models.  If we
place the M5 branes at the origin of $\IR^6 \times \IR^5/\Gamma$ where
$\Gamma$ is a discrete subgroup of the $SO(5)$ R-symmetry group, we get
$AdS_7 \times S^4/\Gamma$ as the near horizon geometry. With $\Gamma
\subset SU(2) \subset SO(5)$ which is an ADE group we obtain 
theories with $(1,0)$ supersymmetry.  The analysis of these models
parallels that of section \ref{orbifolds}.  In particular, the
matching of the $\Gamma$-invariant supergravity Kaluza-Klein modes and
the field theory operators has been discussed in \cite{Ahn:1998oa}.

Another example is the $D_N$ $(2,0)$ SCFT.  It is realized as the
low-energy theory on the worldvolume of $N$ M5 branes at an
$\IR^5/\IZ_2$ orientifold singularity.  The $\IZ_2$ reflects the five
coordinates transverse to the M5 branes and changes the sign of the
3-form field $C$ of eleven dimensional supergravity.  The near horizon
geometry is the smooth space $AdS_7
\times \RP^4$ \cite{Aharony:1998mt}.  
In the supergravity solution we identify the fields at points on the
sphere with the fields at antipodal points, with a change of the sign
of the $C$ field.  This identification projects out half of the
Kaluza-Klein spectrum and only the even $k$ harmonics remain.  An
additional chiral field arises from a M2 brane wrapped on the
2-cycle in $\RP^4$, which is non-trivial due to the orientifolding;
this is analogous to the Pfaffian of the $SO(2N)$ $d=4$ $\cn=4$ SYM
theories which is identified with a wrapped 3-brane
\cite{Witten:1998xy} (as discussed in section \ref{orientifolds}). The
dimension of this operator is $\Delta=2N$. To leading order in $1/N$
the correlation functions of the other chiral operators are similar to
those of the $A_{N-1}$ SCFT.  The $D_N$ theories also have a DLCQ
Matrix description as quantum mechanics on the moduli space of $D_N$
instantons \cite{Aharony:1997md}. This moduli space is singular.  One
would expect to associate the spectrum of chiral primary operators
with the cohomology with compact support of some resolution of this
space, but such a resolution has not been constructed yet.

A different example is the $(1,0)$ six dimensional SCFT with
$E_8$ global symmetry, which is realized on the worldvolume of M5 branes
placed on top of the nine brane in the Ho\v rava-Witten
\cite{Horava:1996qa} compactification of M theory on $\IR^{10} \times
S^1/\IZ_2$.  The conjectured dual description is in terms of M theory on
$AdS_7 \times S^4/\IZ_2$ \cite{Berkooz:1998as}.  The $\IZ_2$ action has a
fixed locus $AdS_7 \times S^3$ on which a ten dimensional $\cN=1$
$E_8$ vector multiplet propagates.
%The difference between
% this example and the $(1,0)$ SCFTs obtained via the orbifolds $AdS_7
% \times S^4/\Gamma$ discussed previously is that in those models the
% M5 branes were placed at different locations of the eleventh
% dimension.  
The chiral operators fall into short representations of the supergroup
$OSp(6,2|2)$. In \cite{Gimon:1999to} $E_8$ neutral and charged
operators of the $(1,0)$ theory were matched with Kaluza-Klein modes
of bulk fields and fields living on the singular locus, respectively.


Correlation functions of chiral primary operators of the large $N$
$(2,0)$ theory can be computed by solving classical differential
equations for the supergravity fields that correspond to the field
theory operators.  Two and three point functions of the chiral primary
operators have been computed in
\cite{Corrado:1999cf}.

The $(2,0)$ SCFT has Wilson surface observables \cite{Ganor:1997nf}, which are
generalizations of the operator given by $W(\Sigma) =
exp (i \int_{\Sigma} B_{\mu\nu} d\sigma^{\mu\nu})$ in the theory of
a free tensor multiplet, where $\Sigma$ is a
two dimensional surface.  A prescription for computing the Wilson
surface in the dual M theory picture has been given in
\cite{Maldacena:1998im}.  It amounts, in the supergravity
approximation, to the computation of the minimal volume of a membrane
bounded at the boundary of $AdS_7$ by $\Sigma$.  The reasoning is
analogous to that discussed in section \ref{wilsonloops}, but here
instead of the strings stretched between D-branes, M2 branes are
stretched between M5 branes.  Such an M2 brane behaves as a string on
the M5 branes worldvolume, with a tension proportional to the distance
between the M5 branes.  By separating one M5 brane from $N$ M5 branes
this string can be used as a probe of the SCFT on the worldvolume of
the $N$ M5 branes, analogous to the external quarks discussed in
section \ref{wilsonloops}.
If we consider two such parallel strings with length $l$ and distance
$L$ and of opposite orientation, the resulting potential per unit length is
\cite{Maldacena:1998im}
\beq
\frac{V}{l} = -c\frac{N}{L^2} \ ,
\eeq
where $c$ is a positive numerical constant.
The dependence on $L$ is as expected from conformal invariance. 
The procedure for Wilson surface computations
has been applied also to the computation of the operator product expansion
of Wilson surfaces, and the extraction of the OPE coefficients of
the chiral primary operators \cite{Corrado:1999cf}.

The six dimensional $A_{N-1}$ theory can be wrapped on various two
dimensional manifolds. At energies lower than the inverse size of the
manifolds, the low-energy effective description is in terms of four
dimensional $SU(N)$ gauge theories.  The two dimensional manifold and
its embedding in eleven dimensions determine the amount of
supersymmetry of the gauge theory.  The simplest case is a wrapping on
$T^2$ which preserves all the supersymmetry.  This results in the $\cN
=4$ $SU(N)$ SCFT, with the complex gauge coupling being the complex
structure $\tau$ of the torus.  In general, when the two dimensional
manifold is a holomorphic curve (Riemann surface), called a
supersymmetric cycle, the four dimensional theory is supersymmetric.
For $\cN=2$ supersymmetric gauge theories the Riemann surface is the
Seiberg-Witten curve and its period matrix gives the low energy
holomorphic gauge couplings $\tau_{ij}$ ($i,j=1,\cdots,N-1$)
\cite{Kachru:1995wm,Kachru:1996fv,Klemm:1996bj,Witten:1997so}.  For $\cN=1$ supersymmetric gauge theories the
Riemann surface has genus zero and it encodes holomorphic properties
of the supersymmetric gauge theory, namely the structure of its moduli
space of vacua \cite{Hori:1998sc}.  For a generic real two dimensional
manifold the four dimensional theory is not supersymmetric.  Some
qualitative properties of the QCD string
\cite{Witten:1997ba} and the $\theta$ vacua follow from the wrapping
procedure.  Of course, in the non-supersymmetric cases the subtle
issue of stability has to be addressed as discussed in section
\ref{other_backgrounds}. In general it is not known how to compute the
near-horizon limit of 5-branes wrapped on a general manifold. At any
rate, it seems that the theory on M5 branes is very relevant to the
study of four dimensional gauge theories.  The M5 branes theory will
be one starting point for an approach to studying pure QCD in section
\ref{adsqcd}.

Other works on M5 branes in the context of the $AdS/$CFT correspondence are 
\cite{Castellani:1998nz,Russo:1998ze,Kallosh:1998qs,Grojean:1998zt,Ahn:1999qe,
Claus:1998mw,Claus:1999yw,
Awata:1998qy,Gutowski:1999iu,Forste:1999yj,
Fayyazuddin:1999zu, Bastianelli:1999bm}.

\subsection{M2 Branes}
\label{m2branes}

$\cN=8$ supersymmetric gauge theories in three dimensions can be
obtained by a dimensional reduction of the four dimensional $\cN=4$
gauge theory.  The automorphism group of the $\cN=8$ supersymmetry
algebra is $SO(8)$.  The fermionic generators of the $\cN=8$
supersymmetry algebra transform in the real two dimensional
representation of the $SO(2,1)$ Lorentz group, and in the ${\bf 8_s}$
representation of the $SO(8)$ automorphism algebra.  The massless
matter representation of the algebra consists of eight bosons in the
${\bf 8_v}$ and eight fermions in the ${\bf 8_c}$ of $SO(8)$.  Viewed
as a dimensional reduction of the vector multiplet of the four
dimensional $\cN=4$ theory which has six real scalars, one extra
scalar is the component of the gauge field in the reduced dimension
and the second extra scalar is the dual to the vector in three
dimensions.

An $\cN=8$ supersymmetric Yang-Mills Lagrangian does not posses the
full $SO(8)$ symmetry.  It is only invariant under an $SO(7)$
subgroup. At long distances it is expected to flow to a superconformal
theory that exhibits the $SO(8)$ R-symmetry (see \cite{Seiberg:1997ax}
and references therein). The flow will be discussed in the next
section.  This IR conformal theory is realized as the low-energy theory
on the worldvolume of $N$ overlapping M2 branes. For a single M2
brane, the eight real scalars define its embedding in eleven
dimensions.  The R-symmetry group is the rotation group in the eight
transverse dimensions to the M2 worldvolume, which rotates the eight
scalars.

The eleven dimensional supergravity metric describing $N$ M2 branes
is given by
\beqar
ds^2 &=& f^{-2/3}(-dt^2 + dx_1^2 + dx_2^2 ) + f^{1/3}(dr^2 + r^2 d
\Omega_7^2) \ , \nonumber\\ f &=& 1 + \frac{32 \pi^2 N l_p^6}{r^6} \ ,
\label{M2}
\eeqar
and there are $N$ units of flux of the dual to the 4-form field on $S^7$. 

The near horizon geometry of (\ref{M2}) is of the form $AdS_4 \times
S^7$ with the radii of curvature $2R_{AdS} = R_{S^4} = l_p (32 \pi^2
N)^{1/6}$.  One conjectures that the three dimensional $\cN=8$ SCFT
on the worldvolume of $N$ M2 branes is dual to M theory on $AdS_4
\times S^7$ with $N$ units of flux of the dual to the 4-form field on
$S^7$
\cite{Maldacena:1997re}.

The supergravity description is applicable for large $N$.  Corrections
to supergravity will be proportional to positive powers of
$l_p/R_{AdS} \sim N^{-1/6}$; the known corrections are all
proportional to powers of $l_p^3 \sim N^{-1/2}$. The supergravity
action itself is in this case proportional to $M_p^9 \sim N^{3/2}$, so
this will be the leading behavior of all correlation functions in the
large $N$ limit.  The bosonic symmetry of the supergravity
compactification is $SO(3,2) \times SO(8)$.  As is standard by now,
the $SO(3,2)$ part is identified with the conformal group of the three
dimensional SCFT, and the $SO(8)$ part is its R-symmetry.  The
fermionic symmetries may also be identified.  We can relate the chiral
fields of the SCFT with the Kaluza-Klein excitations of supergravity
whose spectrum was analyzed in \cite{Biran:1984tf,Castellani:1984tb}.

The lowest component of the supersymmetry multiplets is a
family of scalar excitations with
%There is one family of scalar  excitations
%that contains states also with negative and zero mass squared
\beq
m^2 R_{AdS}^2 = \frac{1}{4}k(k-6),~~~~k=2,3,\cdots \ .
\label{massfor2}
\eeq
They fall into the $k$-th order symmetric traceless representation of
$SO(8)$ with unit multiplicity.  The dimensions of the corresponding
operators in the $\cn=8$ SCFT are $\Delta=k/2,~~k=2,3,\cdots$ 
\cite{Aharony:1998mt, Minwalla:1998po, Halyo:1998so}.  Their
expectation values parametrize the space of flat directions of the
theory, $(\IR^8)^{N-1}/S_N$.  When viewed as the IR limit of the three
dimensional $\cN=8$ Yang-Mills theory, some of these operators can be
identified as $\tr(\phi^{I_1}...\phi^{I_k})$ where $\phi^I$ are the seven
scalars of the vector multiplet.  As noted above, the eighth scalar
arises upon dualizing the vector field, which we can perform
explicitly only in the abelian case. The other chiral fields are all
obtained by the action of the supersymmetry generators on the fields
of (\ref{massfor2}).

%There is also a family of pseudoscalar excitations
%that contains states also with negative and zero mass squared
%\beq
%m^2 = \frac{1}{4}((k-1)(k+1) -8),~~~~k=1,2,\cdots \ .
%\label{massfor3}
%\eeq
%The $k$'th state transforms in symmetric traceless product of 
%${\bf 35_c}$ with $(k-1)$
%${\bf 8_v}'s$ of $SO(8)$.
%The dimensions of the corresponding operators in the SCFT are 
%$\frac{k+3}{2},~~k=1,2,\cdots$.
%In the three dimensional $\cN=8$ Yang-Mills theory at the UV 
%we identify some of these operators
%with a product of two fermions and $k-1$ scalars.

Unlike the $(2,0)$ SCFTs, the $d=3$ $\cn=8$ theories do not have a
simple DLCQ description (see \cite{Ganor:1998jx}), and the spectrum of
their chiral operators is not known.  The above spectrum is the
prediction of the conjectured duality, for large $N$.
 
We can place the M2 branes at singularities and obtain other dual
models, as in section \ref{other_backgrounds}.  If we place the M2 branes at
the origin of $\IR^3 \times
\IR^8/\Gamma$ with $\Gamma$ a discrete subgroup of the $SO(8)$
R-symmetry group, we get $AdS_4 \times S^7/\Gamma$ as the near horizon
geometry.  One class of models is when $\Gamma \subset SU(2) \times
SU(2)$ is a cyclic group.  It is generated by multiplying the complex
coordinates $z_{1,2,3,4}$ of $\IC^4 \simeq 
\IR^8$ by $diag(e^{2 \pi i/k},e^{-2 \pi
i/k},e^{2 \pi i a/k},e^{-2 \pi i a/k})$ for relatively prime integers
$a,k$.  When $a=1, k=2$ the near horizon geometry is $AdS_4 \times
\RP^7$ with a dual $\cN=8$ theory, which is the IR limit of the 
$SO(2N)$ gauge theory \cite{Aharony:1998mt}. As in section
\ref{orientifolds}, one can add a discrete theta angle to get
additional theories \cite{Sethi:1998zk,Berkooz:1999sn}. When $a=\pm 1,
k > 2$ one gets $\cN=6$ supersymmetry, while for $a\neq \pm 1$ the
supersymmetry is reduced to $\cN=4$.  Other models are obtained by non
cyclic $\Gamma$.  As for the D3 branes \cite{Oz:1998of} and the M5
branes \cite{Ahn:1998oa}, the $\Gamma$-invariant supergravity
Kaluza-Klein modes and the field theory operators of some of these
models have been analyzed in
\cite{Entin:1998so}.

Another class of models is obtained by putting the M2 branes at hypersurface singularities
defined by the complex equation
\beq
x^2 + y^2 + z^2 + v^3 + w^{6k-1} =0  \ ,
\eeq
where $k$ is an integer.  The near horizon geometry is of the form
$AdS_4 \times H$, where $H$ is topologically equivalent to $S^7$ but in
general not diffeomorphic to it.  Some of these examples,
$k=1,\cdots,28$, correspond to the known exotic seven-spheres.  The
expected supersymmetry is at least $\cN=2$ and may be $\cN=3$, depending
on whether the R-symmetry group corresponding to the isometry
group of the metric on the exotic seven spheres is $SO(2)$ or $SO(3)$.
An example with $\cN=1$ supersymmetry is when $H$ is the squashed
seven sphere which is the homogeneous space $(Sp(2) \times Sp(1))/(Sp(1)
\times Sp(1))$.  In this case the R-symmetry group is trivial ($SO(1)$).

A general classification of possible near horizon geometries of
the form $AdS_4 \times H$ and related SCFTs in three dimensions is
given in \cite{Morrison:1998cs, Acharya:1998db}.  Most of these SCFTs
have not been explored yet.

Other works on M2 branes in the context of the $AdS/$CFT correspondence are 
\cite{Ferrara:1998vf,Gomis:1998xj,Ahn:1998sv,deWit:1998yu,Claus:1999fh,
Oh:1999qi,Ahn:1998vm,Duff:1999gh,
Furuuchi:1999tn,Fabbri:1999mk,Berkooz:1999ji,
Dall'Agata:1999hh}.

\subsection{Dp Branes}
\label{dpbranes}

Next, we discuss the near-horizon limits of other Dp~branes. They give
spaces which are different from AdS, corresponding to the fact that
the low-energy field theories on the Dp~branes are not conformal.

The Dp branes of the type II string are charged under the
Ramond-Ramond $p+1$-form potential.  Their tension is given by $T_p
\simeq 1/g_sl_s^{p+1}$ and is equal to their Ramond-Ramond charge.
They are BPS saturated objects preserving half of the 32 supercharges
of Type II string theories.  The low energy worldvolume theory of $N$
flat coinciding Dp branes is thus invariant under sixteen
supercharges. It is the maximally supersymmetric $p+1$ dimensional
Yang-Mills theory with $U(N)$ gauge group.  Its symmetry group is
$ISO(1,p) \times SO(9-p)$, where the first factor is the $p+1$
dimensional \Poincare group and the second factor is the R-symmetry
group.  The theory can be obtained as a dimensional reduction of
$\cN=1$ SYM in ten dimensions to $p+1$ dimensions.  Its bosonic fields
are the gauge fields and $9-p$ scalars in the adjoint
representation of the gauge group.  The scalars parametrize the
embedding of the Dp branes in the $9-p$ transverse dimensions.  The
$SO(9-p)$ R-symmetry group is the rotation group in these dimensions,
and the scalars transform in its vector representation.  In the
following we will discuss the decoupling limit of the brane
worldvolume theory from the bulk and the regions of validity of
different descriptions.

The Yang-Mills gauge coupling in the Dp~brane theory is given 
by
\beq
g_{YM}^2 =2  (2 \pi)^{p-2} g_s l_s^{p-3} \ .
\label{dp_coupling}
\eeq
The decoupling from the bulk (field theory) limit is the limit $l_s
\rightarrow 0$ where we keep the Yang-Mills coupling constant and the
energies fixed.  For $p \leq 3$ this
implies that the theory decouples from the bulk and that the higher
$g_s$ and $\alpha'$ corrections to the Dp brane action are suppressed.
For $p > 3$, as seen from (\ref{dp_coupling}), the string coupling goes
to infinity and we need to use a dual description to analyze this
issue.

Let $u \equiv r/\alpha'$ be a fixed expectation value of a scalar. 
At an energy scale $u$, the dimensionless effective coupling constant
of the Yang-Mills theory is 
\beq
g_{eff}^2 \sim g_{YM}^2Nu^{p-3} \ .
\label{effective}
\eeq
The perturbative Yang-Mills description is applicable when $g_{eff}^2 \ll 1$.


The ten dimensional supergravity background describing $N$ Dp branes
is given by the string frame metric
\beqar
ds^2 &=& f^{-1/2}(-dt^2 + \sum_{i=1}^p dx_i^2) + 
f^{1/2}\sum_{i=p+1}^9 dx_i^2 \ , \nonumber\\
f &=& 1 + \frac{c_p g_{YM}^2 N }{l_s^4 u^{7-p}} \ ,
\label{Dp}
\eeqar
with a constant $c_p=2^{6-2p}\pi^{(9-3p)/2}\Gamma((7-p)/2)$.
The background has
a Ramond-Ramond $p+1$-form potential $A_{0...p} = (1-f^{-1})/2$, and
a dilaton
\beq
e^{-2(\phi-\phi_{\infty})} = f^{(p-3)/2} \ .
\eeq


After a variable redefinition 
\beq 
 z = {2 \sqrt{c_p g^2_{YM} N}  \over (5-p)  u^{ 5-p \over 2 } },
\label{ztou}
\eeq
the field theory limit of the metric (\ref{Dp}) for $p < 5$ takes the form 
\cite{Itzhaki:1998sa,Boonstra:1999mp}
\beq
ds^2 = \alpha' \left( 2 \over 5-p\right)^{7-p \over 5-p}
\left(c_p g^2_{YM} N \right)^{1\over 5-p} z^{3-p \over 5-p} 
\left\{ {-dt^2 + d{\vec x}^2 + dz^2 \over z^2 } + { (5-p)^2 \over 4 } 
d \Omega_{8-p}^2 \right\} ~,
%
%\left( \frac{u^{\frac{7-p}{2}}}
%{\sqrt{c_p g_{YM}^2 N}} 
%(-dt^2 + \sum_{i=1}^p dx_i^2)  
%+ \frac{\sqrt{c_p g_{YM}^2 N}}{u^{\frac{7-p}{2}}}du^2 + \sqrt{c_p g_{YM}^2 N}u^{\frac{p-3}{2}}
%d \Omega_{8-p}^2 \right ) \ ,
\label{Dpnear}
\eeq
with the dilaton
\beq
e^{\phi} \sim \frac{g_{eff}^{\frac{7-p}{2}}}{N} \ .
\label{effectivephi}
\eeq
The curvature associated with the metric (\ref{Dpnear})
is
\beq
{\cal R} \sim \frac{1}{l_s^2 g_{eff}} \ .
\label{curvature}
\eeq
In the form of the metric (\ref{Dpnear}) it is easy to see that
the UV/IR correspondence, as described in section \ref{holography},
leads to the relationship $\lambda \sim z$ between wavelengths
in the dual field theories and distances in the gravity solution. 
Through (\ref{ztou}) we can then relate energies in the field theory
to distances in the $u$ variable. 

In the limit of infinite $u$ the effective string coupling
(\ref{effectivephi}) vanishes for $p < 3$. This corresponds to the UV
freedom of the Yang-Mills theory.  For $p > 3$ the coupling increases
and we have to use a dual description.  This corresponds to the fact
that the Yang-Mills theory is non renormalizable and new degrees of
freedom are required at short distances to define the theory.  The
isometry group of the metric (\ref{Dpnear}) is $ISO(1,p) \times
SO(9-p)$. The first factor corresponds to the
\Poincare symmetry group of the Yang-Mills theory
and the second factor corresponds to its R-symmetry group.

For each Dp brane we can plot a phase diagram as a function
of the two dimensionless parameters $g_{eff}$ and $N$ \cite{Itzhaki:1998sa}.
Different regions in the phase diagram have a good description
in terms of different variables.  
As an example consider the D2 branes in Type IIA string theory.   
The dimensionless effective gauge coupling 
(\ref{effective})
is now $g_{eff}^2 \sim g_{YM}^2N/u$.
The perturbative Yang-Mills description is valid for $g_{eff}\ll 1$.
When  $g_{eff} \sim 1$ we have a transition from the perturbative Yang-Mills
description to the Type IIA supergravity description.
The Type IIA supergravity description is valid when both the curvature 
is string units (\ref{curvature})
and the effective string coupling (\ref{effectivephi})
are small. This implies that $N$ must be large.

When $g_{eff} > N^{2/5}$ the effective string coupling becomes large.
In this region we grow the eleventh dimension $x_{11}$ and the good
description is in terms of an eleven dimensional theory.  We can
uplift the D2 brane solution (\ref{Dpnear}) and (\ref{effectivephi})
to an eleven dimensional background that reduces to the ten
dimensional background upon Kaluza-Klein reduction on $x_{11}$.  This
can be done using the relation between the ten dimensional Type IIA
string metric $ds_{10}^2$ and the eleven dimensional metric
$ds_{11}^2$,
\beq
ds_{11}^2 = e^{4\phi/3} (dx_{11}^2 + A^{\mu}dx_{\mu})^2 +
e^{-2\phi/3}ds_{10}^2 \ .
\label{11d}
\eeq
$\phi$ and $A_{\mu}$ are the Type IIA dilaton and RR gauge field.
The 4-form field strength is independent of $x_{11}$.

The curvature of the eleven dimensional metric 
in eleven dimensional Planck units $l_p$
is given by 
\beq
{\cal R} \sim \frac{e^{2\phi/3}}{l_p^2 g_{eff}} \sim
\frac{g_{eff}^{2/3}}{l_p^2 N^{2/3}} \ .
\label{R11}
\eeq
When the curvature (\ref{R11})
is small we can use the eleven dimensional
supergravity description.    

The metric (\ref{11d}) corresponds to the M2 branes solution smeared
over the transverse direction $x_{11}$.  The near-horizon limit of the
supergravity solution describing M2 branes localized in the compact
dimension $x_{11}$ has the form (\ref{M2}), but with a harmonic
function $f$ of the form
\beq
f = \sum_{n=-\infty}^{\infty} \frac{32 \pi^2 l_p^6 N}{\left(r^2 + (x_{11}-x_{11}^0 + 
2 \pi n R_{11})^2 \right)^3} \ ,
\label{f}
\eeq
where $r$ is the radial distance in the seven non-compact transverse
directions and $x_{11} \sim x_{11} + 2 \pi R_{11}$.  $x_{11}^0$
corresponds to the expectation value of the scalar dual to the vector
in the three dimensional gauge theory.  The expression for the
harmonic function (\ref{f}) can be Poisson resummed at distances much
larger than $R_{11}= g_{YM}^2 l_s^2$, leading to
\beq
f = \frac{6 \pi^2 N g_{YM}^2}{l_s^4 u^5} + O( e^{-u/g_{YM}^2}) \ .
\label{local}
\eeq 
The difference between the localized M2 branes solution and the
smeared one is the exponential corrections in (\ref{local}). They can
be neglected at distances  $u \gg g_{YM}^2$,  or in terms of the
dimensionless parameters when $g_{eff} \ll N^{1/2}$. 
According to (\ref{ztou}) this corresponds to distance scales in 
the field theory of order $\sqrt{N}/g^2_{YM}$. In this region we
can still use the up lifted D2 brane solution since it is the same
as the one coming from (\ref{f}) up to exponentially small corrections.
%(even though we should
%really use the localized solution (\ref{f}), since the 2+1 dimensional
%theory naturally sits at a specific point in its moduli space $(\IR^7
%\times S^1)^{N}/S_N$). 
 When $g_{eff} \gg N^{1/2}$, which
corresponds to very low energies $u \ll g_{YM}^2$, the sum in
(\ref{f}) is dominated by the $n=0$ contribution.  This background is
of the form (\ref{M2}) (with $f=32\pi^2 Nl_p^6/r^6$), 
namely the near-horizon limit of M2 branes in
eleven non-compact dimensions.  This is the superconformal theory
which we discussed in the previous section.  In figure \ref{regions}
we plot the transition between the different descriptions as a
function of the energy scale $u$.  We see the flow from the high
energy $\cN=8$ super Yang-Mills theory realized on the worldvolume of
D2 branes to the low energy $\cN=8$ SCFT realized on the worldvolume
on M2 branes.

\begin{figure}[htb]
\begin{center}
\epsfxsize=6in\leavevmode\epsfbox{regions.eps}
\end{center}
\caption{The different descriptions of the D2 brane theory as
a function of the energy scale $u$. 
We see the flow from the high energy $\cN=8$
super Yang-Mills theory to
the low energy $\cN=8$ SCFT. 
}
\label{regions}
\end{figure} 



A similar analysis can be done for the other Dp branes of the Type II
string theories.  In the D0 branes case one starts at high energies
with a perturbative super quantum mechanics description. At
intermediate energies the good description is in terms of the Type IIA
D0 brane solution. At low energies the theory is expected to describe
matrix black holes \cite{Banks:1998hz}. 
 In the D1 branes case one starts in
the UV with a perturbative super Yang-Mills theory in two dimensions.
In the intermediate region the good description is in terms of the
Type IIB D1 brane solution. The IR limit is described by the $Sym^N
(\IR^8)$ orbifold SCFT.  The D3 branes correspond to the $\cn=4$ SCFT
discussed extensively above.

In the D4 branes case, the UV definition of the theory is obtained by
starting with the six dimensional $(2,0)$ SCFT discussed in section
\ref{m5branes}, and compactifying it on a circle. At high energies,
higher than the inverse size of the circle, we have a good description
in terms of the $(2,0)$ SCFT (or the $AdS_7\times S^4$ background of M
theory). The intermediate description is via the background of the
Type IIA D4 brane. Finally at low energies we have a description in
terms of perturbative super Yang-Mills theory in five dimensions.  In
the D5 branes case we have a good description in the IR region in
terms of super Yang-Mills theory. At intermediate energies the system
is described by the near-horizon background of the Type IIB D5 brane,
and in the UV in terms of the solution of the Type IIB NS5~branes.
We will discuss the NS5~brane theories in the next section.

Consider now the system of $N$ D6 branes of Type IIA string theory.
As before, we can attempt at a decoupling of the seven dimensional
theory on the D6 branes worldvolume from the bulk by taking the string
scale to zero and keeping the energies and the seven dimensional
Yang-Mills coupling fixed.  The effective Yang-Mills coupling
(\ref{effective}) is small at low energies $u \ll
(g_{YM}^2N)^{-1/3}$ and super Yang-Mills is a good
description in this regime.  The curvature in string units
(\ref{curvature}) is small when $u \gg (g_{YM}^2N)^{-1/3}$
while the effective string coupling (\ref{effectivephi}) is small when
$u \ll N/g_{YM}^{2/3}$. In between these limits we can use the
Type IIA supergravity solution.

When $u \sim N/g_{YM}^{2/3}$ the effective string coupling is
large and we should use the description of D6~branes in terms of
eleven dimensional supergravity compactified on a circle with $N$
Kaluza-Klein monopoles. Equivalently, the description is in terms of
eleven dimensional supergravity on an ALE space with an $A_{N-1}$
singularity.  When $u \gg N/g_{YM}^{2/3}$ the curvature of the
eleven dimensional space vanishes and, unlike the lower dimensional
branes, there does not not exist a seven dimensional field theory that
describes the UV.  In fact, the D6 brane worldvolume theory does not
decouple from the bulk.

A simple way to see that the D6 brane worldvolume theory does not
decouple from the bulk is to note that now in the decoupling limit we
keep $g_{YM}^2 \sim g_s l_s^3$ fixed.  When we lift the D6 branes
solution to M theory, this means that the eleven dimensional Planck
length $l_p^3 = g_s l_s^3$ remains fixed, and therefore gravity does not
decouple.  Another way to see that gravity does not decouple is to
consider the system of D6 branes at finite temperature in the
decoupling limit. For large energy densities above extremality, $E/V
\gg N/l_p^7$, we need the eleven dimensional description.  This is
given by an uncharged Schwarzschild black hole at the ALE
singularity. The associated Hawking temperature is $T_H \sim
1/\sqrt{Nl_p^9E/V}$ and there is  Hawking radiation to the asymptotic
region of the bulk eleven dimensional supergravity.  Generally, the
worldvolume theories of D$p$~branes with $p >5$ do not decouple from
the bulk.

The supergravity computation of the Wilson loop, discussed in section
\ref{wilsonloops}, can be carried out for the Dp brane theories.  For
instance for the $N$ D2 branes theory one gets for the quark antiquark
potential, using the type IIA SUGRA D2~brane solution
\cite{Maldacena:1998im},
\beq
V = - c \frac{(g_{YM}^2N)^{1/3}}{L^{2/3}} \ ,
\label{VV}
\eeq
where $c$ is a positive numerical constant.  In view of the discussion
above, this result should be trusted only for loops with sizes
$  1/g_{YM}^2N \ll L \ll \sqrt{N}/g^2_{YM}$. For smaller 
loops the computation fails because we go into the perturbative regime, 
where the potential becomes logarithmic. For larger loops we 
get into the $AdS_4 \times S^7$ region. 


Other works on Dp branes in the context of the $AdS/$CFT correspondence are 
\cite{Itzhaki:1998uz,Boonstra:1999mp,Itzhaki:1998ka,Pelc:1999ms,
Hashimoto:1999xx,Youm:1999zs,Lu:1999uc,Lu:1999uv,Barbon:1999zp}.

\subsection{NS5 Branes}
\label{ns5branes}

The NS5 branes of Type II string theories couple magnetically to the
NS-NS $B_{\mu\nu}$ field, and they are magnetically dual to the
fundamental string.  Their tension is given by $T_{NS} \simeq
1/g_s^2l_s^6$.  Like the Dp branes, they are BPS objects that preserve
half of the supersymmetry of Type II theories.  A fundamental string
propagating in the background of $N$ parallel NS5 branes is described
far from the branes by a conformal field theory with non trivial
metric, $B$ field and dilaton, constructed in
\cite{Callan:1991ss}. The string coupling grows as the string
approaches the NS5~branes.  At low energies the six dimensional theory
on the worldvolume of $N$ Type IIB NS5 branes is a $U(N)$ $\cn=(1,1)$
super Yang-Mills theory, which is free in the IR. However, it is an
interacting theory at intermediate energies.  At low energies the
theory on the worldvolume of $N$ Type IIA NS5 branes is the $A_{N-1}$
$(2,0)$ SCFT discussed above.

The six dimensional theories on the worldvolume of NS5 branes of Type
II string theories were argued \cite{Seiberg:1997md} to decouple from
the bulk in the limit
\beq
g_s\rightarrow 0,~~~~~l_s=fixed \ .
\label{decoupling}
\eeq
This is because the effective coupling on the NS5 branes (e.g. the
low-energy Yang-Mills coupling in the type IIB case) is $1/l_s$, while
the coupling to the bulk modes goes like $g_s$.  However, the
computation of \cite{Maldacena:1997sd} showed that in this limit there
is still Hawking radiation to the tube region of the NS5~brane
solution, suggesting a non decoupling of the worldvolume theory from
the bulk.  In the spirit of the other correspondences discussed
previously, one can reconcile the two by conjecturing
\cite{Aharony:1998ub} that string or M theory in the NS5 brane
background in the limit (\ref{decoupling}), which includes the tube
region, is dual to the decoupled NS5 brane worldvolume theory
(``little string theory'').  In particular, the fields in the tube
which are excited in the Hawking radiation correspond to objects in
the decoupled NS5 brane theory.  In the following we will mainly
discuss the Type IIA NS5 brane theory\footnote{Type IIB NS5~branes at 
orbifold singularities are discussed in    \cite{Diaconescu:1998pj}.}.

The Type IIA NS5 brane may be considered as the M5 brane localized on
the eleven dimensional circle.  Therefore its metric is that of an
M5 brane at a point on a transverse circle.  In such a configuration
the near horizon metric of $N$ NS5 branes can be written as
\cite{Itzhaki:1998sa,Aharony:1998ub}
\beqar
ds^2 &=& l_p^2 \left(f^{-1/3}(-dt^2 + \sum_{i=1}^5 dx_i^2) + 
f^{2/3}(dx_{11}^2  + du^2 + u^2 d\Omega_3^2) \right) \ , \nonumber\\
f &=& \sum_{n=-\infty}^{\infty}
\frac{\pi N}{(u^2+(x_{11}-2\pi n/l_s^2)^2)^{3/2}} \ .
\label{NS5}
\eeqar
The $x_{11}$ coordinate is periodic and has been rescaled by $l_p^3$
($x_{11} \equiv x_{11} + 2\pi/l_s^2$).  The background also has a 4-form
flux of $N$ units on $S^1 \times S^3$.

At distances larger than $l_s\sqrt{N}$ the NS5 brane theory is
described by the $A_{N-1}$ $(2,0)$ SCFT.  Indeed, in the extreme low
energy limit $l_s \rightarrow 0$ the sum in (\ref{NS5}) is dominated
by the $n=0$ term and the background is of the form $AdS_7 \times
S^4$.  This reduces to the conjectured duality between M theory on
$AdS_7 \times S^4$ and the $(2,0)$ SCFT, discussed previously.
However, the NS5 brane theory is not a local quantum field theory at
all energy scales since at short distances it is not described by a UV
fixed point.  To see this one can take $l_s$ to infinity (or $u$ to
infinity) in (\ref{NS5}) and get a Type IIA background with a linear
dilaton.  It has the topology of $\IR^{1,5} \times \IR \times S^3$
with $g_s^2(\phi) = e^{-2\phi/l_s\sqrt{N}}$, where $\phi$ is the $\IR$
coordinate.
%\beq
%ds^2 = -dt^2 + \sum_{i=1}^5 dx_i^2 + 
%d\phi^2 + N l_s^2d\Omega_3^2),~~~~~
%g_s^2(\phi) = e^{-2\phi/{l_s\sqrt{N}}} \ .
%\label{linear}
%\eeqar
This is in accord with the fact that the NS5 brane theory exhibits a
T-duality property upon compactification on tori (note that in this
background a finite radius in field theory units corresponds to a
finite radius in string theory units on the string theory side of the
correspondence, unlike the previous cases we discussed).
%The theory has a stress energy tensor. However it is not unique.  

The NS5 brane theories have an A-D-E classification. This can be seen
by viewing them as Type II string theory
on K3 with A-D-E singularities in the
decoupling limit (\ref{decoupling}).  The NS5 brane theories have an
$SO(4)$ R-symmetry which we identify with the $SO(4)$ isometry of $S^3$.
The IIA NS5~brane theories have a moduli space of vacua of the form
$(\IR^4 \times S^1)^r/\cW$ where $r$ is the rank of the A-D-E gauge
group and $\cW$ is the corresponding Weyl group.  It is parametrized
by the $\cW$-invariant products of the $5r$ scalars in the $r$ tensor
multiplets.  They fall into short representations of the supersymmetry
algebra.  
%Like in a local quantum field theory the observables may be
%defined in the UV region corresponding to large $\phi$ coordinate.  In
%order for the duality to hold they should be matched with the string
We can match these chiral operators with the string excitations in the
linear dilaton geometry describing the large $u$ region of
(\ref{NS5}). The string excitations, in short representations of the
supersymmetry algebra, in the linear dilaton geometry were analyzed in
\cite{Aharony:1998ub}. Indeed, they match the spectrum of the chiral
operators in short representations of the NS5 brane theories. Actually,
due to the fact that the string coupling goes to zero at the boundary
of the linear dilaton solution, one can compute here the precise
spectrum of chiral fields in the string theory, and find an agreement
with the field theory even for finite $N$ (stronger than the large $N$
agreement that we described in section \ref{tests}). 

As in the dualities with local quantum field theories, also here one
can compute correlation functions by solving differential equations on
the NS5 branes background (\ref{NS5}). Since in this case the boundary
is infinitely far away, it is more natural to compute correlation
functions in momentum space, which correspond to the S-matrix in the
background (\ref{NS5}). The computation of two point functions 
of a scalar field was sketched in \cite{Aharony:1998ub} and described
more rigorously in \cite{Minwalla:1999xi}.
The NS5~brane theories are non-local, and this
causes some differences in the matching between M theory and the
non-gravitational NS5~brane theory in this case. One difference
from the previous cases we discussed is that in the linear dilaton
backgrounds if we put a cutoff at some value of the radial coordinate
(generalizing the discussion of \cite{Susskind:1998dq} which we
reviewed in section \ref{holography}), the volume enclosed by the
cutoff is not proportional to the area of the boundary (which it is in
AdS space). Thus, if holography is valid in these backgrounds (in the
sense of having a number of degrees of freedom proportional to the
boundary area) it is more remarkable than holography in AdS space.


