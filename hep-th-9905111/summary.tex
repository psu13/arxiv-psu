%\section{Summary}
\label{summary}

We conclude by summarizing some of the successes and remaining 
open problems  of the AdS/CFT correspondence.

From the field theory point of view we have learned and understood
better many properties
of the large $N$ limit. Since 't Hooft's work \cite{'tHooft:1974jz}
 we knew 
that the large $N$ limit of gauge theories should be described
by strings, if the parameter $g_{YM}^2 N$ is kept fixed. 
Through the correspondence we have learned that not only does this
picture really work (beyond perturbation theory where it was first
derived), but that the Yang-Mills strings (made from gluons) are the same
as the fundamental strings. 
 Moreover, these strings move in higher
dimensions, as was argued in \cite{Polyakov:1997tj}. 
These extra dimensions arise dynamically in the gauge theory. 
For some field theories the curvatures in the higher dimensional space
could be small. The prototypical example is ${\cal N} = 4$ 
super-Yang-Mills with large $N, g_{YM}^2 N$. 
From this example we can obtain others by taking 
quotients, placing branes at various singularities, etc. (section 
\ref{other_backgrounds}). 
In all cases for which we can find a low-curvature gravity description we 
can do numerous calculations in the large $N$ limit. 
We can calculate the spectrum of operators
and states (sections \ref{tests}, \ref{isom}).
We can calculate correlation functions of operators and of 
Wilson loops (sections \ref{correlators}, \ref{wilsonloops}). 
We can calculate thermal properties, like the equation of state 
(section \ref{FiniteT}), and so on.

If the field theory is conformal the gravity solution will include
an $AdS$ factor.   
It is possible, in principle, to deform the theory by any
relevant operators.
In some cases fairly explicit solutions have been found for 
flows between different conformal field theories 
(section \ref{deformations}). A ``$c$-theorem'' for field theories in
more than two dimensions was
proven within the gravity approximation. It would be very interesting
to generalize this beyond this approximation. 
It would also be interesting to understand better 
exactly what it is the class of field
theories which have a gravity approximation. One constraint on such
four dimensional conformal
field theories, described in section \ref{anomalies},
is that they must have $a=c$.

It is possible to give a field theory 
interpretation to various branes that one 
can have in the AdS description (section \ref{baryons}).
 Some correspond to baryons in the 
field theory, others to various defects like domain walls, etc. 
In the $AdS_5$ case D-instantons in the string theory correspond 
to gauge theory instantons in the field theory. 

In general, the large $N$ limit of a gauge theory should have 
a string theory
description. Whether it also has a gravity description depends on how
large the curvatures in this string theory
are. If the curvatures are small, we can 
have an approximate classical gravity description. 
Otherwise, we should consider all string 
modes on the same footing. This involves solving the worldsheet
theories for strings in 
Ramond-Ramond backgrounds. This is a problem that only now is beginning
to be elucidated \cite{Berkovits:1999im,Metsaev:1998it,%
Pesando:1998fv,Kallosh:1998nx,Kallosh:1998ji,Polyakov:1998pm,%
Dolan:1999pi,Rajaraman:1999rc,Berenstein:1999jq}.
For non-supersymmetric QCD, or other theories which are weakly coupled
(as QCD is at high energies), we expect to 
have curvatures at least
of the order of the string scale, so that a proper understanding of
strings on highly curved spaces seems crucial. 

It is also possible to deform the $\cn=4$ field theory,
breaking supersymmetry and conformal invariance, by giving  a mass
to the fermions or by compactifying the theory on a circle with
supersymmetry breaking boundary conditions.
Then, we have a theory that should describe pure
 Yang-Mills theory at low energies (sections \ref{deformations},
\ref{FiniteT}, \ref{adsqcd}). 
In the case of field theories compactified on a circle with
supersymmetry breaking boundary conditions and  
 large $g_{YM}^2 N$ at the compactification scale,
one can show that the theory is confining, has a 
mass gap, has $\theta$-vacua with the right qualitative properties and
has a confinement-deconfinement transition at finite temperature.
However, in the regime where the analysis
can be done (small curvature) this theory includes many
additional degrees of freedom beyond those in the standard bosonic
Yang-Mills theory.
In order to do quantitative calculations in bosonic Yang-Mills
one would have to do calculations when 
the curvatures are large, which goes beyond the gravity approximation
and requires understanding the propagation of strings in Ramond-Ramond
backgrounds. Unfortunately, this is proving to be very difficult, and
so far we have not obtained new results in QCD from the correspondence.
As discussed in section \ref{adsqcd}, the gravity approximation
resembles the strong coupling lattice QCD description 
\cite{Wilson:1974co},
where the $\alpha'$ expansion of string theory corresponds to the strong
coupling expansion.
The gravity description has an advantage over the  strong  
coupling lattice QCD description
by being fully Lorentz invariant. This allows, for instance, 
the analysis of topological  
properties 
of the vacuum which is a difficult task in the lattice description.
The AdS/CFT correspondence does provide direct
evidence that QCD is describable as some sort of string theory (to the
extent that we can use the name string theory for strings propagating
on spaces whose radius of curvature is of the order of  the string scale
or smaller).

One of the surprising things we learned about field theory is that 
there are various new large $N$ limits which had not been considered
before. For instance, 
we can take $N\to \infty$ keeping $g_{YM}$ fixed, and the AdS/CFT
correspondence implies that many properties of the field theory (like
correlation functions of chiral primary operators) have a
reasonable limiting behavior in this limit, though there is no good
field theory argument for this. Similarly, we find that there exist
large $N$ limits for theories which are not gauge theories, like the 
$d=3,\cn=8$ and $d=6,\cn=(2,0)$ superconformal field theories, and for
various theories with less supersymmetry. The existence of these
limits cannot be
derived directly in field theory.

The correspondence has also been used to learn about the properties of
field theories which were previously only poorly understood. For
instance, it has been used \cite{Seiberg:1999xz} to understand
properties of two dimensional field theories with singular target
spaces, and to learn properties of ``little string theories'', like
the fact that they have a Hagedorn behavior at high energies. The
correspondence has also been used to construct many new conformal
field theories, both in the large $N$ limit and at finite $N$.

Another interesting case is topological Chern-Simons 
theory in three dimensions, which is related to a topological
string theory in six dimensions \cite{Gopakumar:1998ki}. In this
case one can solve exactly both sides of the correspondence and
see explicitly that it works.

The correspondence is also useful for studying non-conformal 
gauge theories,
as we discussed in section \ref{dpbranes}. A particularly interesting
case is the maximally supersymmetric quantum mechanical $SU(N)$ gauge
theory, which is related
to Matrix theory \cite{Banks:1997vh,Balasubramanian:1997kd,
Hyun:1998bi,Itzhaki:1998sa,McCarthy:1998uw,
deAlwis:1999ki,Silva:1998nk,Chepelev:1999pm,
Yoneya:1999zi,Townsend:1998qp,
Polchinski:1999br}.

From the quantum gravity point of view we have now an explicit
holographic description for gravity in many backgrounds involving 
an asymptotically AdS space. 
The field theory effectively sums over all geometries which are 
asymptotic to $AdS$. This defines the theory non-perturbatively. 
This also implies that gravity in these spaces is unitary, giving the
first explicit non-perturbative
construction of a unitary theory of quantum gravity,\footnote{In
the context of Matrix theory \cite{Banks:1997vh} we need to take a
large $N$ limit which is not well understood in order 
to describe a theory of gravity in a space with no
closed light-like curves.} albeit in a curved space background. 
Black holes are some  mixed states in the field theory Hilbert space.
Explicit microscopic 
calculations of black hole entropy and greybody factors
can be done in the $AdS_3$ case (chapter \ref{ChapAdS3}). 

Basic properties of quantum gravity, such as approximate causality and
locality at low energies, 
are far from clear in this description \cite{Horowitz:1998pq,Banks:1998dd,
Das:1999fx,Horowitz:1999gf,Kabat:1999yq,Bak:1999iq,Lowe:1999pk}, and it would
be interesting to understand them better. We are also still far from
having a precise mapping between general configurations in the
gravitational theory and in the field theory (see \cite{Berkooz:1998wv,
Balasubramanian:1999ri}
for some attempts to go in this direction).

In principle one can extract the physics of quantum gravity 
in flat space by taking
the large radius limit of physics in $AdS$ space. 
Since we have not discussed this yet in the review, let us expand on
this here, following
\cite{Polchinski:1999ry,Susskind:1998vk,Giddings:1999qu,Polchinski:1999yd}
(see also \cite{Balasubramanian:1999ri,Aref'eva:1999mi,Lee:1999ua,
Li:1999vi}). 
We would like to be able to describe processes in flat space which
occur, for instance, at some fixed string coupling, with the energies and
the size of the interaction region kept fixed in string (or Planck)
units. Computations on AdS space are necessarily done with some finite
radius of curvature; however, we can
view this radius of curvature as a 
regulator, and take it to infinity at the end of any calculation, 
in such a way that the local physics remains the same.
Let us discuss what this means for the $AdS_5\times S^5$ case (the
discussion is similar for other cases). We need to keep the string
coupling fixed, and take $N \to \infty$ since the radius of curvature
in Planck units is proportional to $N^{1/4}$. Note that this is
different from the 't Hooft limit, and involves taking $\lambda \to
\infty$. In order to describe a scattering process in space-time which
has finite energies in this limit, it turns out that the energies in
the field theory must scale as $N^{1/4}$ (measured in units of
the scale of the
$S^3$ which the field theory is compactified on; we need to work in
global AdS coordinates to describe flat-space scattering). In this
limit the field theory is very strongly coupled and the energies are
also very high, and there are no known ways to do any computations on
the field theory side.
It would be interesting to compute anything explicitly in this limit. 
For example, it would be interesting to compute the entropy of a small
Schwarzschild black hole, much smaller than the radius of $AdS$, to
see flat-space Hawking radiation, and so on. If we start with
$AdS_5\times S^5$ this limit gives us the physics in flat ten
dimensional space, and similarly starting with $AdS_4\times S^7$ or
$AdS_7\times S^4$ we can get the physics in flat eleven dimensional
space. It would be interesting to understand how the correspondence
can be used to learn about theories with lower dimension, where some
of the dimensions are compactified. A limit of string theory on
$AdS_3\times S^3\times M^4$ may be used to give string theory on
$\IR^{5,1}\times M^4$, but it is not clear how to get four dimensional
physics out of the correspondence. 

One could, in principle, get four dimensional flat space by
starting from   $AdS_2\times
S^2$ compactifications.  However, the correspondence in the case of
$AdS_2$ spaces is not well understood. 
$AdS_2$ spaces 
arise as the near horizon geometry of extremal
charged Reissner-Nordstrom black holes. Even though fields propagating in 
$AdS_2$ behave similarly to  the higher dimensional cases 
\cite{Strominger:1999yg},
the problem is that any finite energy excitation seems to 
destroy the $AdS_2$ boundary conditions \cite{Maldacena:1999uz}. 
This is related to the fact that black holes (as opposed to black
$p$-branes, $p>0$) have an energy gap 
(see section \ref{fivedbh}), so that in the extreme low energy limit
we seem to have no excitations. 
One possibility is that the correspondence works only for the ground
states. Even then, there are instantons that can lead to a fragmentation
of the spacetime into several pieces 
\cite{Brill:1992rw}. 
Some conformal quantum mechanics systems that are, or 
could be,  related to 
$AdS_2$ were studied in
\cite{Kumar:1999fx,Kallosh:1999mi,Gibbons:1998fa,Townsend:1998qp}. 
Aspects of Hawking radiation in $AdS_2$ were studied in 
\cite{Spradlin:1999bn}.

In all the known cases of the correspondence the gravity solution 
has a timelike boundary\footnote{This is not precisely true in the
linear dilaton backgrounds described in section \ref{ns5branes}
\cite{Aharony:1998ub}.}. It would be interesting to understand how the
correspondence works when the boundary is light-like, as in Minkowski
space. It seems that holography must work quite differently in these
cases (see \cite{Aharony:1999tt,Hashimoto:1999yc} for discussions of
some of the issues involved).
In the cases we understand, the asymptotic space close to the boundary 
has a well defined notion of time, which is the one that is associated
to the gauge theory.  It would be interesting to understand how 
holography works in other spacetimes, where we do not have this
notion of time. Interesting examples are spatially closed universes, 
expanding universes, de-Sitter spacetimes, etc. See
 \cite{Horowitz:1998xk,Hull:1998vg,Hull:1998ym,Hull:1998fh}
 for some
attempts in this direction.
The precise meaning of holography in the cosmological context is still
not clear \cite{Fischler:1998st,Bak:1998vj,%
Dawid:1998ip,Rama:1998pk,Easther:1999gk,%
Bak:1999hd,Kaloper:1999tt}.

To summarize, the past 18 months have seen much progress in our
understanding of string/M theory compactifications on AdS and related
spaces, and in our understanding of large $N$ field theories. However,
the correspondence is still far from realizing the hopes that it
initially raised, and much work still remains to be done. The
correspondence gives us implicit ways to describe QCD and related
interesting field theories in a dual ``stringy'' description, but so
far we are unable to do any explicit computations in the field
theories that we are really interested in. The main hope for progress
in this direction lies in a better understanding of string theory in
RR backgrounds. The correspondence also gives us an explicit example
of a unitary and holographic theory of quantum gravity. We hope this
example can be used to better understand quantum gravity in flat
space, where the issues of unitarity (the ``information problem'') and
holography are still quite obscure. Even better, one could hope that
the correspondence would hint at a way to formulate string/M theory
independently of the background. These questions will apparently have
to wait until the next millennium.






















