
\section{Black $p$-Branes} 
\label{black_pbranes}

The recent insight into the connection between
large $N$ field theories and string theory has emerged
from the study of $p$-branes in string theory. The $p$-branes 
were originally found as classical solutions to supergravity,
which is the low energy limit of string theory. 
Later it was pointed out by Polchinski that D-branes give
their full string theoretical description. Various comparisons 
of the two descriptions led to the discovery of the AdS/CFT 
correspondence.

\subsection{Classical Solutions}

String theory has a variety of classical solutions corresponding
to extended black holes
\cite{Gibbons:1982ih,Gibbons:1988ps,Callan:1989hs,Dabholkar:1989jt,
Dabholkar:1990yf,Duff:1991wv,Garfinkle:1991qj,Callan:1991dj,Callan:1991ky,
Horowitz:1991cd,Duff:1992pe}. Complete descriptions of all
possible black hole solutions would be beyond the scope of this 
review, and we will discuss here only illustrative examples
corresponding to parallel D$p$ branes. 
For a more extensive review of extended objects in
string theory, see \cite{Duff:1996zn,Peet:1997es}. 

Let us consider type II string theory in ten dimensions,
and look for a black hole solution carrying 
 electric charge with respect 
to the Ramond-Ramond (R-R) $(p+1)$-form $A_{p+1}$ 
\cite{Gibbons:1988ps,Garfinkle:1991qj,Horowitz:1991cd}.
In type IIA (IIB) theory, $p$ is even (odd). 
The theory contains also magnetically charged 
$(6-p)$-branes, which are  electrically charged under the dual 
$  dA_{7-p}= * dA_{p+1}$ potential.
Therefore, 
%Because of the duality constraint on the $(p+1)$-form potentials, 
%$^* dA_{7-p}=dA_{p+1}$, such a solution 
%is also magnetically charged for $A_{7-p}$. This means that the
R-R charges have to be quantized according to the Dirac quantization
condition. 
To find the solution, we start with the low energy effective
action in the string frame,
\beq
  S =\frac{1}{(2\pi)^7 l_s^8} \int d^{10} x \sqrt{-g} \left( e^{-2\phi}
\left({\cal R} + 4 (\nabla \phi)^2 \right) 
- \frac{2}{(8-p)!} F_{p+2}^2 \right),
\label{action}
\eeq
where $l_s$ is the string length, related to the string tension
$(2\pi \alpha')^{-1}$ as $\alpha' = l_s^2$, and 
$F_{p+2}$ is the field strength of the $(p+1)$-form
potential,  $F_{p+2} = dA_{p+1}$. In the self-dual case of $p=3$
we work directly with the equations of motion.  
We then look for a solution corresponding to
a $p$-dimensional electric source of charge $N$ for $A_{p+1}$,
by requiring the Euclidean symmetry $ISO(p)$ in $p$-dimensions:
\beq
  ds^2 = ds_{10-p}^2 + e^\alpha \sum_{i=1}^p dx^i dx^i.
\eeq
Here $ds_{10-p}^2$ is a Lorentzian-signature metric 
in $(10-p)$-dimensions. We also assume that the metric is
spherically symmetric in $(10-p)$ dimensions with the  R-R source
at the origin,
\beq
   \int_{S^{8-p}} ~^* F_{p+2} = N ,
\eeq
where $S^{8-p}$ is the $(8-p)$-sphere surrounding the source. 
By using the Euclidean symmetry $ISO(p)$, we can reduce
the problem to the one of finding a spherically symmetric
charged black hole solution in $(10-p)$ dimensions \cite{Gibbons:1988ps,
Garfinkle:1991qj,Horowitz:1991cd}. The resulting metric, in the string
frame, is given by
\beq
 ds^2 = - \frac{f_+(\rho)}{\sqrt{f_-(\rho)}} dt^2
+  \sqrt{f_-(\rho)}
   \sum_{i=1}^p dx^i dx^i 
+  \frac{f_-(\rho)^{-\frac{1}{2} -\frac{5-p}{7-p}}}{f_+(\rho)} d\rho^2
 + r^2  f_-(\rho)^{\frac{1}{2}-\frac{5-p}{7-p}} d\Omega_{8-p}^2 ,
\label{solution}
\eeq
with the dilaton field,
\beq
e^{-2\phi} = g_s^{-2} f_-(\rho)^{-\frac{p-3}{2}},
\eeq
where
\beq
  f_{\pm}(\rho) = 1 - \left(\frac{r_\pm}{\rho} \right)^{7-p},
\eeq
and $g_s$ is the asymptotic string coupling constant. 
The parameters $r_+$ and $r_-$ are related to the mass $M$ 
(per unit volume) and 
the RR charge $N$ of the solution by
\beq
M =  \frac{1}{(7-p) (2 \pi)^7 d_p l_P^8}
  \left((8-p) r_+^{7-p} - r_-^{7-p} \right),~~~
N = \frac{1}{d_pg_s l_s^{7-p}}(r_+ r_-)^{\frac{7-p}{2}} ,
\label{masscharge}
\eeq
where $l_P = g_s^{\frac{1}{4}} l_s$ is the 10-dimensional Planck
length and $d_p$ is a numerical factor,
\beq
  d_p = 2^{5-p} \pi^{\frac{5-p}{2}} \Gamma\left( \frac{7-p}{2}
 \right).
\eeq

The metric in the Einstein frame, $(g_{\cal E})_{\mu\nu}$, 
is defined by multiplying
the string frame metric $g_{\mu\nu}$ by
$\sqrt{ g_se^{-\phi}}$
in (\ref{action}), so that the action takes the standard
Einstein-Hilbert form, 
\beq
S= \frac{1}{(2\pi)^7 l_P^8}\int d^{10} x 
 \sqrt{-g_{\cal E}} ( {\cal R}_{\cal E} - {1 \over 2} (\nabla \phi)^2 +
\cdots ).
\eeq
The Einstein frame metric has a horizon at $\rho=r_+$. For
$p\leq 6$, there is also a curvature singularity at $\rho=r_-$.
When $r_+ > r_-$, the singularity is covered by the horizon
and the solution can be regarded as a black hole. 
When $r_+ < r_-$, there is a timelike naked singularity
and the Cauchy problem is not well-posed. 

The situation is subtle in the critical case $r_+ = r_-$. 
If $p \neq 3$, the horizon
and the singularity coincide and there is a ``null'' singularity\footnote{
This is the case for $p < 6$. For $p=6$, the singularity is timelike
as one can see from the fact that it can be lifted to the Kaluza-Klein 
monopole in 11 dimensions.}. 
Moreover, the dilaton either diverges or vanishes at $\rho=r_+$. 
This singularity, however, is milder than in the case of $r_+ < r_-$,
and the supergravity description is still valid up to 
a certain distance from the singularity. The situation is
much better for $p=3$. In this case, the dilaton is constant.
Moreover, the $\rho=r_+$ surface is regular even when $r_+=r_-$,
allowing a smooth analytic extension beyond $\rho=r_+$  \cite{Gibbons:1995vm}.


According to (\ref{masscharge}), for a fixed value of $N$, 
the mass $M$ is an increasing
function of $r_+$. The condition $r_+ \geq r_-$ for 
the absence of the timelike naked singularity therefore translates into
an inequality between the mass $M$ and 
the R-R charge $N$, of the form
\beq
   M \geq  \frac{N}{(2 \pi)^p g_s l_s^{p+1}} .
\label{bps}
\eeq
The solution whose mass $M$ is at the lower bound of this inequality
is called an {\it extremal $p$-brane}. On the other hand, 
when $M$ is strictly greater than that, we have
a {\it non-extremal black $p$-brane}. It is called {\it black}
since there is an event horizon for $r_+ > r_-$. 
The area of the black hole horizon goes 
to zero in the extremal limit $r_+ = r_-$. Since
the extremal solution with $p \neq 3$ 
has a  singularity,
the supergravity description breaks down
near $\rho = r_+$ and we need to use the full string
theory.  The D-brane construction discussed below
will give exactly such a description. 
The inequality (\ref{bps}) is 
also the BPS bound with respect to the 10-dimensional 
supersymmetry, and the extremal solution $r_+=r_-$ 
preserves one half of the supersymmetry
in the regime where we can trust the supergravity
description. This suggests that the extremal $p$-brane
is a ground state of the black $p$-brane for a given charge
$N$. 

The extremal limit $r_+ = r_-$ of the solution (\ref{solution}) is
given by
\beq
  ds^2 = \sqrt{f_+(\rho)} \left( -dt^2 + \sum_{i=1}^p dx^i dx^i \right)
 + f_+(\rho)^{-\frac{3}{2} - \frac{5-p}{7-p}} d\rho^2
 + \rho^2 f_+(\rho)^{\frac{1}{2} - \frac{5-p}{7-p}} d\Omega_{8-p}^2.
\label{extremal}
\eeq
In this limit, the symmetry of the metric is enhanced from
the Euclidean group $ISO(p)$ to the Poincar\'e group
$ISO(p,1)$. This fits well with the interpretation that
the extremal solution corresponds to the ground state 
of the black $p$-brane.
To describe the geometry of the extremal solution outside of
the horizon,  it is often useful to define a new coordinate $r$ by
\beq
  r^{7-p} \equiv \rho^{7-p} - r_+^{7-p},
\eeq
and introduce the isotropic coordinates, $r^a = r \theta^a$
($a=1,...,9-p; ~\sum_a (\theta^a)^2 = 1$). 
The metric and the dilaton for the extremal $p$-brane are
then written as
\beq
ds^2 = \frac{1}{\sqrt{H(r)}}
 \left(-dt^2 + \sum_{i=1}^p dx^idx^i \right)
 + \sqrt{H(r)}
   \sum_{a=1}^{9-p} dr^a dr^a ,
\label{another}
\eeq
\beq
  e^{\phi} = g_s H(r)^{\frac{3-p}{4}},
\label{dilaton}
\eeq
where
\beq
  H(r) = \frac{1}{f_+(\rho)} = 1 + \frac{r_+^{7-p}}{r^{7-p}},~~~~~~~
r_+^{7-p} = d_p g_s N l_s^{7-p}.
\eeq
The horizon is now located at $r=0$. 

In general, (\ref{another}) and (\ref{dilaton}) give
a solution to the supergravity equations of motion for any
function $H(\vec{r})$ which is a harmonic function in the
$(9-p)$ dimensions which are transverse to the $p$-brane.  
For example, we may consider a more general solution, of the form
\beq
   H(\vec{r}) = 1 + \sum_{i=1}^k 
\frac{r_{(i)+}^{7-p}}{|\vec{r}-\vec{r}_i|^{7-p}},~~~~~~~r_{(i)+}^{7-p}
= d_p g_s N_i l_s^{7-p}.
\label{multicentered}
\eeq
This is called a multi-centered solution and represents
parallel extremal $p$-branes located at $k$ different locations,
$\vec{r}=\vec{r}_i$ ($i=1,\cdots,k$), each of which carries
$N_i$ units of the R-R charge. 

So far we have discussed the black $p$-brane using the classical
supergravity. This description is appropriate when the curvature
of the $p$-brane geometry is small compared to the string scale,
so that stringy corrections are negligible. Since the strength 
of the curvature is characterized by $r_+$, this requires
 $r_+ \gg l_s$. 
To suppress string loop corrections, 
the effective string coupling $e^{\phi}$ also needs
to be kept small. When $p=3$,
the dilaton is constant and we can make it small
everywhere in the $3$-brane geometry by setting
$g_s < 1$, namely $l_P <l_s$. If $g_s>1$ we might need to do an 
$S$-duality, $g_s \rightarrow 1/g_s$, first. 
 Moreover, in this case it is known that 
the metric (\ref{another}) can be analytically
extended beyond the horizon $r=0$, and that
the maximally extended metric is
geodesically complete and without a singularity
\cite{Gibbons:1995vm}. 
The strength of the curvature is then
uniformly bounded by $r_+^{-2}$. 
To summarize, for $p=3$, the supergravity approximation is valid
when
\beq
   l_P < l_s \ll r_+.
\eeq
Since $r_+$ is related to the R-R charge $N$ as
\beq
     r_+^{7-p} = d_p g_s N l_s^{7-p},
\eeq
this can also be expressed
as
\beq
    1 \ll g_s N < N.
\label{hier1}
\eeq
For  $p \neq 3$, the metric is singular at $r=0$,
and the supergravity description is valid only 
in a limited region of the spacetime. 

\subsection{D-Branes}

Alternatively, the extremal $p$-brane can be
described as a D-brane. For a review of D-branes,
see \cite{Polchinski:1996na}.
The D$p$-brane is a $(p+1)$-dimensional hyperplane
in spacetime where an open string can end. 
By the worldsheet duality, this means that 
the D-brane is also a source of closed strings (see Fig. \ref{F9}).
In particular, it can carry the R-R charges.  
It was shown in  \cite{Polchinski:1995mt} that,
if we put $N$ D$p$-branes on top of each other,
the resulting $(p+1)$-dimensional hyperplane carries 
exactly $N$ units of the $(p+1)$-form charge. 
On the worldsheet of a type II string, the left-moving degrees
of freedom 
and the right-moving degrees of freedom 
carry separate spacetime supercharges.
Since the open string boundary condition identifies
the left and right movers,  
%the open string ending on 
the D-brane breaks at least one half of the spacetime 
supercharges. In type IIA (IIB) string theory, 
precisely one half
of the supersymmetry is preserved if $p$ is even
(odd). This is consistent with the types of R-R charges
that appear in the theory. Thus, the D$p$-brane is a BPS 
object in string theory which carries 
exactly the same charge as the black $p$-brane
solution in supergravity. 
  
\begin{figure}[htb]
\begin{center}
\epsfxsize=4.5in\leavevmode\epsfbox{hfig9.eps}
\end{center}
\caption{(a) The D-brane is where open strings can end.
(b) The D-brane is a source
of closed strings.}
\label{F9}
\end{figure} 

It is believed that the extremal $p$-brane in supergravity
and the D$p$-brane are two different descriptions of the same
object. The D-brane uses the string worldsheet and, therefore,
is a good description in string perturbation theory. 
When there are $N$ D-branes on top of each other,
the effective loop expansion parameter for the open strings is
$g_s N$ rather than $g_s$, since each open string boundary loop 
ending on the D-branes
comes with the Chan-Paton factor $N$ as well as the
string coupling $g_s$. Thus, the D-brane description
is good when $g_s N \ll 1$. 
This is complementary to the regime (\ref{hier1}) where
the supergravity description is appropriate. 

The low energy effective theory of open strings on 
the D$p$-brane is the $U(N)$ gauge theory in $(p+1)$ dimensions
with $16$ supercharges \cite{Witten:1996im}. 
The theory has $(9-p)$ scalar fields
$\vec{\Phi}$ in the adjoint representation of $U(N)$. 
If the vacuum expectation value $\langle \vec{\Phi} \rangle$ 
has $k$ distinct eigenvalues\footnote{There is a potential
$\sum_{I,J} \tr [\Phi^I, \Phi^J]^2$ for the scalar fields,
so expectation values of the matrices
$\Phi^I$ ($I=1,\cdots,9-p$) minimizing the potential
are simultaneously diagonalizable.}, with $N_1$ identical 
eigenvalues $\vec{\phi}_1$, $N_2$ identical eigenvalues $\vec{\phi}_2$
and so on, the gauge group $U(N)$ is broken 
to $U(N_1) \times \cdots \times U(N_k)$. 
This corresponds to the situation when $N_1$ D-branes are at 
$\vec{r}_1 = \vec{\phi}_1 l_s^2$, $N_2$ D$p$-branes are
at $\vec{r}_2= \vec{\phi}_2 l_s^2$, and so on. 
In this case, there are massive
$W$-bosons for the broken gauge groups. The 
$W$-boson in the bi-fundamental representation of
$U(N_i) \times U(N_j)$ comes from the open string stretching 
between the D-branes at $\vec{r}_i$ and $\vec{r}_j$, 
and the mass of the W-boson
is proportional to the Euclidean distance $|\vec{r}_i - \vec{r}_j|$
between the $D$-branes. It is important to note that the same result 
is obtained if we use the supergravity solution for the multi-centered 
$p$-brane (\ref{multicentered}) and compute the mass of
the string going from $\vec{r}_i$ to $\vec{r}_j$, since
the factor $H(\vec{r})^{\frac{1}{4}}$ from
 the metric in the $\vec{r}$-space (\ref{another})
is cancelled by the redshift factor 
$H(\vec{r})^{-\frac{1}{4}}$ when converting
the string tension into energy. Both
the D-brane description and the supergravity solution give
the same value of the W-boson mass, since it is determined
by the BPS condition. 







