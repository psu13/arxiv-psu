\chapter{Introduction}

\section{Motivation}

Quantum mechanics in its modern form is now more than 80 years old.
It is probably the most successful physical theory that was ever proposed. 
It started as an attempt to understand the structure of the atom and the interactions of matter and light at the atomic scale, and it became quickly the general physical framework, valid from  the presently accessible high energy scales (10 TeV$\simeq 10^{-19} $m) -- and possibly from the Planck scale ($10^{-35}$ m) -- up to  macroscopic scales (from $\ell\sim 1$ nm up to  $\ell\sim 10^5$ m depending of physical systems and experiments). Beyond these scales,  classical mechanics takes over as an effective theory, valid when quantum interferences and non-local correlations effects can be neglected.

Quantum mechanics has fully revolutionized physics (as a whole, from particle and nuclear physics to atomic an molecular physics, optics, condensed matter physics and material science), chemistry (again as a whole), astrophysics, etc. with a big impact on mathematics and of course a huge impact on modern technology, the whole communication technology, computers, energy, weaponry (unfortunately) etc.
In all these domains, and despites the huge experimental and technical progresses of the last decades, quantum mechanics has never been seriously challenged by experiments, and its mathematical foundations are very solid.

Quantum information  has become a important and very active field (both theoretically and experimentally) in the last decades. It has enriched our points of view on the quantum theory, and on its applications (quantum computing). 
Quantum information, together with the experimental tests of quantum mechanics, the theoretical advances in quantum gravity and cosmology, the slow diffusion of the concepts from quantum theory in the general public, etc. have led to a revival of the discussions about the principles of quantum mechanics and its seemingly paradoxical aspects.

Thus one sometimes gets the feeling that quantum mechanics is both: (i) the unchallenged and dominant paradigm of modern physical sciences and technologies, (ii) still (often presented as) mysterious and poorly understood, and waiting for some revolution.

\medskip

These lecture notes present a brief and  introductory (but hopefully coherent) view of the main formalizations of quantum mechanics (and of its version compatible with special relativity, quantum field theory), of their interrelations and of their theoretical foundation. 

The ``standard'' formulation of quantum mechanics (involving the Hilbert space of pure states, self-adjoint operators as physical observables, and the probabilistic interpretation given by the Born rule), and the path integral and functional integral representations of probabilities amplitudes are the standard tools used in most applications of quantum theory in physics and chemistry.
It is important to be aware that there are other formulations of quantum mechanics, i.e. other representations (in the mathematical sense) of quantum mechanics, which allow a better comprehension and justification of the quantum theory. This course will focus on two of them, algebraic QM and the so called ``quantum logic'' approach, that I find the most interesting and that I think I managed to understand (somehow...). 
I shall insist on the algebraic aspects of the quantum formalism.

In my opinion discussing and comparing the various formulations is useful in order to get a  better understanding of the coherence and the strength of the quantum formalism. 
This is important when discussing  which features of quantum mechanics are  basic principles and which ones are just natural consequences of the former. Indeed this depends on the different formulations. 
For instance the Born rule or the projection postulate are postulates in the standard formulation, while in some other formulations they are mere consequences of the postulates.
This is also important for understanding the relation between quantum physics and special relativity through their common roots, causality, locality and reversibility.

Discussing the different formulations is useful to discuss these issues, in particular when considering the relations between quantum theory, information theory and quantum gravity.

\medskip
These notes started from: (i) a spin-off of more standard lecture notes for a master course in quantum field theory and its applications to statistical physics, (ii) a growing interest
\footnote{A standard syndrome for the physicist over 50... encouraged (for useful purpose) by the European Research Council}
 in understanding what was going on in the fields of quantum information, of quantum measurements and of the foundational studies of the quantum formalism, (iii) a course that I was kindly asked to give at the Institut de Physique Th�orique (my lab) and at the graduate school of physics of the Paris Area (ED107) in May-June 2012, (iv)  a short Letter \cite{PhysRevLett.107.180401} about reversibility in quantum mechanics that I published last year. These notes can be considered partly as a very extended version of this letter.

\section{Organization}

After this introductory section, the second section is a reminder of the basic concepts of classical physics, of probabilities  and of the standard (canonical) and path integral formulations of quantum physics.
I tried to introduce in a  consistent way the important classical concepts of states, observables and probabilities, which are of course crucial in the formulations of quantum mechanics.  
I discuss in particular the concept of quantum probabilities and the issue of reversibility in quantum mechanics in the last subsection.

The third section is devoted to a presentation and a discussion of the algebraic formulation of quantum mechanics and of quantum field theory, based on operator algebras. Several aspects of the discussion are original.
Firstly I justify the appearance of abstract C$^*$-algebras of observables using arguments based on causality and  reversibility. In particular the existence of a $^*$-involution (corresponding to conjugation) is argued to follow from the assumption of reversibility for the quantum probabilities.
Secondly, the formulation is based on real  algebras, not complex ones as usually done, and I explain why this is more natural. I give the mathematical references which justify that the GNS theorem, which ensures that complex abstract C$^*$-algebras are always representable as algebras of operators on a Hilbert space, is also valid for real algebras. 
The standard physical arguments for the use of complex algebras are only given after the general construction.
The rest of the presentation is shorter and quite standard.

The fourth section is devoted to one of the formulations of the so-called quantum logic formalism. This formalism is much less  popular outside the community interested in the foundational basis of quantum mechanics, and in mathematics, but deserves to be better known.
Indeed, it provides a  convincing justification of the algebraic structure of quantum mechanics, which for an important part is still postulated in the algebraic formalism. Again, if the global content is not original, I try to present the quantum logic formalism in a similar light than the algebraic formalism, pointing out which aspects are linked to causality, which ones to reversibility, and which ones to locality and separability. This way to present the quantum logic formalism is  original, I think.
Finally, I discuss in much more details than is usually done Gleason's theorem, a very important theorem of Hilbert space geometry and operator algebras, which justify the Born rule and is also very important when discussing hidden variable theories.

The final section contains short, introductory and more standard discussions of some other questions about the quantum formalism.
I present  some recent approaches based on quantum information. I discuss  some features of quantum correlations: entanglement, entropic inequalities, the Tisrelson bound for bipartite systems.
The problems with hidden variables, contextuality, non-locality, are  reviewed. Some very basic features of quantum measurements are recalled. Then I stress the difference between 
\begin{itemize}
  \item the various formalizations (representations) of quantum mechanics;
  \item the various possible interpretations of this formalism;
\end{itemize}
I finish this section with a few very standard remarks on the problem of quantum gravity.


\section{What this course is not!}
These notes are (tentatively) aimed at a non specialized audience: graduate students and more advanced researchers. The mathematical formalism is the main subject of the course, but it will be presented and discussed at a not too abstract, rigorous or advanced level.
Therefore these notes \textbf{do not intend to be}:
\begin{itemize}
  \item a real course of mathematics  or of mathematical physics;
  \item a real physics course on high energy quantum physics,  on atomic physics and quantum optics, of quantum condensed matter, discussing the physics of specific systems and their applications;
 \item a course on what is \emph{not} quantum mechanics;
  \item a course on the history of quantum physics;
  \item a course on the present sociology of quantum physics;
\item a course on the  philosophical and epistemological aspects of quantum physics.
\end{itemize} 
But I hope that it could be useful as an introduction to these topics. Please keep in mind that this is not a course made by a specialist, it is rather a course made by an amateur, for amateurs!


\section{Acknowledgements}

I thank Roger Balian, Michel Bauer, Marie-Claude David, Kirone Mallick and Vincent Pasquier for their interest and their advices.
