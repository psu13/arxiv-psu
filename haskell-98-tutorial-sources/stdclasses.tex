%**<title>A Gentle Introduction to Haskell: Standard Classes</title>

%**~header

\section{Standard Haskell Classes}

In this section we introduce the predefined standard type
classes in Haskell.  We have simplified these classes somewhat by
omitting some of the less interesting methods in these classes; the
Haskell report contains a more complete description.  Also, some of
the standard classes are part of the standard Haskell libraries; these
are described in the Haskell Library Report.

\subsection{Equality and Ordered Classes}

The classes \mbox{\tt Eq} and \mbox{\tt Ord} have already been discussed.  The
definition of \mbox{\tt Ord} in the Prelude is somewhat more complex than the
simplified version of \mbox{\tt Ord} presented earlier.  In particular, note the
\mbox{\tt compare} method:
\bprog
\mbox{\tt data\ Ordering\ \ \ \ \ \ \ \ \ \ \ =\ \ EQ\ |\ LT\ |\ GT\ }\\
\mbox{\tt compare\ \ \ \ \ \ \ \ \ \ \ \ \ \ \ \ \ ::\ Ord\ a\ =>\ a\ ->\ a\ ->\ Ordering}
\eprog
The \mbox{\tt compare} method is sufficient to define all other
methods (via defaults) in this class and is the best way to create
\mbox{\tt Ord} instances. 

\subsection{The Enumeration Class}
\label{tut-enum-classes}

Class \mbox{\tt Enum} has a set of operations that underlie the syntactic sugar of
arithmetic sequences; for example, the arithmetic sequence expression
\mbox{\tt [1,3..]} stands for \mbox{\tt enumFromThen\ 1\ 3} (see
\see{arithmetic-sequences} for the formal translation).  
We can now see that arithmetic sequence expressions can be used to
generate lists of any type that is an instance of \mbox{\tt Enum}.  This
includes not only most numeric types, but also \mbox{\tt Char}, so that, for
instance, \mbox{\tt ['a'..'z']} denotes the list of lower-case letters in
alphabetical order.  Furthermore, user-defined enumerated types like
\mbox{\tt Color} can easily be given \mbox{\tt Enum} instance declarations.  If so:
\[ \mbox{\tt [Red\ ..\ Violet]}\ \ \ \ \red\ \ \ \ \mbox{\tt [Red,\ Green,\ Blue,\ Indigo,\ Violet]}
\]
Note that such a sequence is {\em arithmetic} in the sense that the
increment between values is constant, even though the values are not
numbers.  Most types in \mbox{\tt Enum} can be mapped onto fixed precision
integers; for these, 
the \mbox{\tt fromEnum} and \mbox{\tt toEnum} convert between \mbox{\tt Int} and a type in \mbox{\tt Enum}.

\subsection{The Read and Show Classes}
\label{tut-text-class}

The instances of class \mbox{\tt Show} are those types that can be converted
to character strings (typically for I/O).  The class \mbox{\tt Read} 
provides operations for parsing character strings to obtain the values
they may represent.  The simplest function in the class \mbox{\tt Show} is
\mbox{\tt show}: 
\bprog
\mbox{\tt show\ \ \ \ \ \ \ \ \ \ \ \ \ \ \ \ \ \ \ \ ::\ (Show\ a)\ =>\ a\ ->\ String}
\eprog
Naturally enough, \mbox{\tt show} takes any value of an appropriate type and
returns its representation as a character string (list of characters),
as in \mbox{\tt show\ (2+2)}, which results in \mbox{\tt "4"}.  This is fine as far as
it goes, but we typically need to produce more complex strings
that may have the representations of many values in them, as in
\bprog
\mbox{\tt "The\ sum\ of\ "\ ++\ show\ x\ ++\ "\ and\ "\ ++\ show\ y\ ++\ "\ is\ "\ ++\ show\ (x+y)\ ++\ "."}
\eprog
and after a while, all that concatenation gets to be a bit
inefficient.  Specifically, let's consider a function to represent
the binary trees of Section \ref{tut-recursive-types} as a string,
with suitable markings to show the nesting of subtrees and the
separation of left and right branches (provided the element type is
representable as a string):
\bprog
\mbox{\tt showTree\ \ \ \ \ \ \ \ \ \ \ \ \ \ \ \ ::\ (Show\ a)\ =>\ Tree\ a\ ->\ String}\\
\mbox{\tt showTree\ (Leaf\ x)\ \ \ \ \ \ \ =\ \ show\ x}\\
\mbox{\tt showTree\ (Branch\ l\ r)\ \ \ =\ \ "<"\ ++\ showTree\ l\ ++\ "|"\ ++\ showTree\ r\ ++\ ">"}
\eprog
Because \mbox{\tt (++)} has time complexity linear in the length of its
left argument, \mbox{\tt showTree} is potentially quadratic in the size of the
tree. 

To restore linear complexity, the function \mbox{\tt shows} is provided:
\bprog
\mbox{\tt shows\ \ \ \ \ \ \ \ \ \ \ \ \ \ \ \ \ \ \ ::\ (Show\ a)\ =>\ a\ ->\ String\ ->\ String}
\eprog
\mbox{\tt shows} takes a printable value and a string and returns
that string with the value's representation concatenated
at the front.  The second argument serves as a sort of string
accumulator, and \mbox{\tt show} can now be defined as \mbox{\tt shows} with
the null accumulator.  This is the default definition of
\mbox{\tt show} in the \mbox{\tt Show} class definition:
\bprog
\mbox{\tt show\ x\ \ \ \ \ \ \ \ \ \ \ \ \ \ \ \ \ \ =\ \ shows\ x\ ""}
\eprog
We can use \mbox{\tt shows} to define a more efficient version of \mbox{\tt showTree},
which also has a string accumulator argument:
\bprog
\mbox{\tt showsTree\ \ \ \ \ \ \ \ \ \ \ \ \ \ \ ::\ (Show\ a)\ =>\ Tree\ a\ ->\ String\ ->\ String}\\
\mbox{\tt showsTree\ (Leaf\ x)\ s\ \ \ \ =\ \ shows\ x\ s}\\
\mbox{\tt showsTree\ (Branch\ l\ r)\ s=\ \ '<'\ :\ showsTree\ l\ ('|'\ :\ showsTree\ r\ ('>'\ :\ s))}
\eprog
This solves our efficiency problem (\mbox{\tt showsTree} has linear complexity),
but the presentation of this function (and others like it) can be
improved.  First, let's create a type synonym:
\bprog
\mbox{\tt type\ ShowS\ \ \ \ \ \ \ \ \ \ \ \ \ \ =\ \ String\ ->\ String}
\eprog
This is the type of a function that returns a string representation of
something followed by an accumulator string.
Second, we can avoid carrying accumulators around, and also avoid
amassing parentheses at the right end of long constructions, by using
functional composition:
\bprog
\mbox{\tt showsTree\ \ \ \ \ \ \ \ \ \ \ \ \ \ \ ::\ (Show\ a)\ =>\ Tree\ a\ ->\ ShowS}\\
\mbox{\tt showsTree\ (Leaf\ x)\ \ \ \ \ \ =\ \ shows\ x}\\
\mbox{\tt showsTree\ (Branch\ l\ r)\ \ =\ \ ('<':)\ .\ showsTree\ l\ .\ ('|':)\ .\ showsTree\ r\ .\ ('>':)}
\eprog
Something more important than just tidying up the code has come about
by this transformation:  we have raised the presentation from an
{\em object level} (in this case, strings) to a {\em function level.}
We can think of the typing as saying that \mbox{\tt showsTree} maps a tree
into a {\em showing function}.  Functions like \mbox{\tt ('<'\ :)} or
\mbox{\tt ("a\ string"\ ++)} are primitive showing functions, and we build up
more complex functions by function composition.

Now that we can turn trees into strings, let's turn to the inverse
problem.  The basic idea is a parser for a type \mbox{\tt a}, which
is a function that takes a string and returns a list of \mbox{\tt (a,\ String)}
pairs~\cite{wadler:list-of-successes}.  The Prelude provides
a type synonym for such functions:
\bprog
\mbox{\tt type\ ReadS\ a\ \ \ \ \ \ \ \ \ \ \ \ =\ \ String\ ->\ [(a,String)]}
\eprog
Normally, a parser returns a singleton list, containing a value
of type \mbox{\tt a} that was read from the input string and the remaining
string that follows what was parsed.  If no parse was possible, however,
the result is the empty list, and if there is more than one possible
parse (an ambiguity), the resulting list contains more than one pair.
The standard function \mbox{\tt reads} is a parser for any instance of \mbox{\tt Read}:
\bprog
\mbox{\tt reads\ \ \ \ \ \ \ \ \ \ \ \ \ \ \ \ \ \ \ ::\ (Read\ a)\ =>\ ReadS\ a}
\eprog
We can use this function to define a parsing function for the string
representation of binary trees produced by \mbox{\tt showsTree}.  List comprehensions
give us a convenient idiom for constructing such parsers:\footnote{An
even more elegant approach to parsing uses monads and parser
combinators.  These are part of a standard parsing library distributed
with most Haskell systems.}
\bprog
\mbox{\tt readsTree\ \ \ \ \ \ \ \ \ \ \ \ \ \ \ ::\ (Read\ a)\ =>\ ReadS\ (Tree\ a)}\\
\mbox{\tt readsTree\ ('<':s)\ \ \ \ \ \ \ =\ \ [(Branch\ l\ r,\ u)\ |\ (l,\ '|':t)\ <-\ readsTree\ s,}\\
\mbox{\tt \ \ \ \ \ \ \ \ \ \ \ \ \ \ \ \ \ \ \ \ \ \ \ \ \ \ \ \ \ \ \ \ \ \ \ \ \ \ \ \ \ \ \ \ \ \ (r,\ '>':u)\ <-\ readsTree\ t\ ]}\\
\mbox{\tt readsTree\ s\ \ \ \ \ \ \ \ \ \ \ \ \ =\ \ [(Leaf\ x,\ t)\ \ \ \ \ |\ (x,t)\ \ \ \ \ \ <-\ reads\ s]}
\eprog
Let's take a moment to examine this function definition in detail.
There are two main cases to consider:  If the first character of the
string to be parsed is \mbox{\tt '<'}, we should have the representation of
a branch; otherwise, we have a leaf.  In the first case, calling the
rest of the input string following the opening angle bracket \mbox{\tt s},
any possible parse must be a tree \mbox{\tt Branch\ l\ r} with remaining string \mbox{\tt u},
subject to the following conditions:
\begin{enumerate}
\item
The tree \mbox{\tt l} can be parsed from the beginning of the string \mbox{\tt s}.
\item
The string remaining (following the representation of \mbox{\tt l}) begins
with \mbox{\tt '|'}.  Call the tail of this string \mbox{\tt t}.
\item
The tree \mbox{\tt r} can be parsed from the beginning of \mbox{\tt t}.
\item
The string remaining from that parse begins with \mbox{\tt '>'}, and
\mbox{\tt u} is the tail.
\end{enumerate}
Notice the expressive power we get from the combination of pattern
matching with list comprehension: the form of a resulting parse is
given by the main expression of the list comprehension, the first
two conditions above are expressed by the first generator
(``\mbox{\tt (l,\ '|':t)} is drawn from the list of parses of \mbox{\tt s}''), and the
remaining conditions are expressed by the second generator.

The second defining equation above just says that to parse the
representation of a leaf, we parse a representation of the element
type of the tree and apply the constructor \mbox{\tt Leaf} to the value thus
obtained.

We'll accept on faith for the moment that there is a \mbox{\tt Read} (and
\mbox{\tt Show}) instance
of \mbox{\tt Integer} (among many other types), providing a \mbox{\tt reads} that behaves
as one would expect, e.g.:
\[ \mbox{\tt (reads\ "5\ golden\ rings")\ ::\ [(Integer,String)]}\ \ \ \ \red\ \ \ \ %
\mbox{\tt [(5,\ "\ golden\ rings")]} \]
With this understanding, the reader should verify the following evaluations:
\[\ba{lcl}
  \mbox{\tt readsTree\ "<1|<2|3>>"}&\ \ \ \ \red\ \ \ \ &
    \mbox{\tt [(Branch\ (Leaf\ 1)\ (Branch\ (Leaf\ 2)\ (Leaf\ 3)),\ "")]}\\
  \mbox{\tt readsTree\ "<1|2"}     &\ \ \ \ \red\ \ \ \ & \mbox{\tt []}
\ea\]

There are a couple of shortcomings in our definition of \mbox{\tt readsTree}.
One is that the parser is quite rigid, allowing no white space before
or between the elements of the tree representation; the other is that
the way we parse our punctuation symbols is quite different from the
way we parse leaf values and subtrees, this lack of uniformity making
the function definition harder to read.  We can address both of these
problems by using the lexical analyzer provided by the Prelude:
\bprog
\mbox{\tt lex\ \ \ \ \ \ \ \ \ \ \ \ \ \ \ \ \ \ \ \ \ ::\ ReadS\ String}
\eprog
\mbox{\tt lex} normally returns a singleton list containing a
pair of strings: the first lexeme in the input string and the remainder
of the input.  The lexical rules are those of Haskell programs,
including comments, which \mbox{\tt lex} skips, along with whitespace.
If the input string is empty or contains only whitespace and comments,
\mbox{\tt lex} returns \mbox{\tt [("","")]}; if the input is not empty in this sense,
but also does not begin with a valid lexeme after any leading whitespace
and comments, \mbox{\tt lex} returns \mbox{\tt []}.

Using the lexical analyzer, our tree parser now looks like this:
\bprog
\mbox{\tt readsTree\ \ \ \ \ \ \ \ \ \ \ \ \ \ \ ::\ (Read\ a)\ =>\ ReadS\ (Tree\ a)}\\
\mbox{\tt readsTree\ s\ \ \ \ \ \ \ \ \ \ \ \ \ =\ \ [(Branch\ l\ r,\ x)\ |\ ("<",\ t)\ <-\ lex\ s,}\\
\mbox{\tt \ \ \ \ \ \ \ \ \ \ \ \ \ \ \ \ \ \ \ \ \ \ \ \ \ \ \ \ \ \ \ \ \ \ \ \ \ \ \ \ \ \ \ \ \ \ (l,\ \ \ u)\ <-\ readsTree\ t,}\\
\mbox{\tt \ \ \ \ \ \ \ \ \ \ \ \ \ \ \ \ \ \ \ \ \ \ \ \ \ \ \ \ \ \ \ \ \ \ \ \ \ \ \ \ \ \ \ \ \ \ ("|",\ v)\ <-\ lex\ u,}\\
\mbox{\tt \ \ \ \ \ \ \ \ \ \ \ \ \ \ \ \ \ \ \ \ \ \ \ \ \ \ \ \ \ \ \ \ \ \ \ \ \ \ \ \ \ \ \ \ \ \ (r,\ \ \ w)\ <-\ readsTree\ v,}\\
\mbox{\tt \ \ \ \ \ \ \ \ \ \ \ \ \ \ \ \ \ \ \ \ \ \ \ \ \ \ \ \ \ \ \ \ \ \ \ \ \ \ \ \ \ \ \ \ \ \ (">",\ x)\ <-\ lex\ w\ \ \ \ \ \ \ \ \ ]}\\
\mbox{\tt \ \ \ \ \ \ \ \ \ \ \ \ \ \ \ \ \ \ \ \ \ \ \ \ \ \ \ ++}\\
\mbox{\tt \ \ \ \ \ \ \ \ \ \ \ \ \ \ \ \ \ \ \ \ \ \ \ \ \ \ \ [(Leaf\ x,\ t)\ \ \ \ \ |\ (x,\ \ \ t)\ <-\ reads\ s\ \ \ \ \ \ \ ]}
\eprog

We may now wish to use \mbox{\tt readsTree} and \mbox{\tt showsTree} to declare 
\mbox{\tt (Read\ a)\ =>\ Tree\ a} an instance of \mbox{\tt Read} and \mbox{\tt (Show\ a)\ =>\ Tree\ a} an
instance of \mbox{\tt Show}.  This would allow us to
use the generic overloaded functions from the Prelude to parse and
display trees.  Moreover, we would automatically then be able to parse
and display many other types containing trees as components, for
example, \mbox{\tt [Tree\ Integer]}.  As it turns out, \mbox{\tt readsTree} and \mbox{\tt showsTree}
are of almost the right types to be \mbox{\tt Show} and \mbox{\tt Read} methods
% , needing
% only the addition of an extra parameter each that has do do with
% parenthesization of forms with infix constructors.
The \mbox{\tt showsPrec}
and \mbox{\tt readsPrec} methods are parameterized versions of \mbox{\tt shows} and
\mbox{\tt reads}.  The extra parameter is a precedence level, used to properly
parenthesize expressions containing infix constructors.  For types
such as \mbox{\tt Tree}, the precedence can be ignored.  The \mbox{\tt Show} and \mbox{\tt Read}
instances for \mbox{\tt Tree} are:
\bprog
\mbox{\tt instance\ Show\ a\ =>\ Show\ (Tree\ a)\ where}\\
\mbox{\tt \ \ \ \ showsPrec\ {\char'137}\ x\ =\ showsTree\ x}\\
\mbox{\tt }\\[-8pt]
\mbox{\tt instance\ Read\ a\ =>\ Read\ (Tree\ a)\ where}\\
\mbox{\tt \ \ \ \ readsPrec\ {\char'137}\ s\ =\ readsTree\ s}
\eprog
Alternatively, the \mbox{\tt Show} instance could be defined in terms of
\mbox{\tt showTree}: 
\bprog
\mbox{\tt instance\ Show\ a\ =>\ Show\ (Tree\ a)\ where}\\
\mbox{\tt \ \ \ show\ t\ =\ showTree\ t}
\eprog
This, however, will be less efficient than the \mbox{\tt ShowS} version.  Note
that the \mbox{\tt Show} class defines default methods for both \mbox{\tt showsPrec} and
\mbox{\tt show}, allowing the user to define either one of these in an instance
declaration.  Since these defaults are mutually recursive, an instance
declaration that defines neither of these functions will loop when
called.  Other classes such as \mbox{\tt Num} also have these ``interlocking
defaults''. 

We refer the interested reader to \see{derived-appendix} for details
of the \mbox{\tt Read} and \mbox{\tt Show} classes.

We can test the \mbox{\tt Read} and \mbox{\tt Show} instances by applying \mbox{\tt (read\ .\ show)}
(which should be the identity) to some trees, where \mbox{\tt read} is a
specialization of \mbox{\tt reads}:
\bprog
\mbox{\tt read\ \ \ \ \ \ \ \ \ \ \ \ \ \ \ \ \ \ \ \ ::\ (Read\ a)\ =>\ String\ ->\ a}
\eprog
This function fails if there is not a unique parse or if the input
contains anything more than a representation of one value of type \mbox{\tt a}
(and possibly, comments and whitespace).

\subsection{Derived Instances}
\label{tut-derived-instances}

Recall the \mbox{\tt Eq} instance for trees we presented in Section
\ref{tut-type-classes}; such a declaration is
simple---and boring---to produce: we require that the
element type in the leaves be an equality type; then, two leaves are
equal iff they contain equal elements, and two branches are equal iff
their left and right subtrees are equal, respectively.  Any other two
trees are unequal:
\bprog
\mbox{\tt instance\ \ (Eq\ a)\ =>\ Eq\ (Tree\ a)\ \ where}\\
\mbox{\tt \ \ \ \ (Leaf\ x)\ \ \ \ \ ==\ (Leaf\ y)\ \ \ \ \ \ \ \ =\ \ x\ ==\ y}\\
\mbox{\tt \ \ \ \ (Branch\ l\ r)\ ==\ (Branch\ l'\ r')\ \ =\ \ l\ ==\ l'\ {\char'46}{\char'46}\ r\ ==\ r'}\\
\mbox{\tt \ \ \ \ {\char'137}\ \ \ \ \ \ \ \ \ \ \ \ ==\ {\char'137}\ \ \ \ \ \ \ \ \ \ \ \ \ \ \ =\ \ False}
\eprog

Fortunately, we don't need to go through this tedium every time we
need equality operators for a new type; the \mbox{\tt Eq} instance can be {\em
derived automatically} from the \mbox{\tt data} declaration if we so specify:
\bprog
\mbox{\tt data\ \ Tree\ a\ \ \ \ \ \ \ \ \ \ \ \ =\ \ Leaf\ a\ |\ Branch\ (Tree\ a)\ (Tree\ a)\ \ deriving\ Eq}
\eprog
The \mbox{\tt deriving} clause implicitly produces an \mbox{\tt Eq} instance declaration
just like the one in Section \ref{tut-type-classes}.  Instances of \mbox{\tt Ord},
\mbox{\tt Enum}, \mbox{\tt Ix}, \mbox{\tt Read}, and \mbox{\tt Show} can also be generated by the
\mbox{\tt deriving} clause. 
\syn{More than one class name can be specified, in which case the list
of names must be parenthesized and the names separated by commas.}

The derived \mbox{\tt Ord} instance for \mbox{\tt Tree} is slightly more complicated than
the \mbox{\tt Eq} instance:
\bprog
\mbox{\tt instance\ \ (Ord\ a)\ =>\ Ord\ (Tree\ a)\ \ where}\\
\mbox{\tt \ \ \ \ (Leaf\ {\char'137})\ \ \ \ \ <=\ (Branch\ {\char'137})\ \ \ \ \ \ =\ \ True}\\
\mbox{\tt \ \ \ \ (Leaf\ x)\ \ \ \ \ <=\ (Leaf\ y)\ \ \ \ \ \ \ \ =\ \ x\ <=\ y}\\
\mbox{\tt \ \ \ \ (Branch\ {\char'137})\ \ \ <=\ (Leaf\ {\char'137})\ \ \ \ \ \ \ \ =\ \ False}\\
\mbox{\tt \ \ \ \ (Branch\ l\ r)\ <=\ (Branch\ l'\ r')\ \ =\ \ l\ ==\ l'\ {\char'46}{\char'46}\ r\ <=\ r'\ ||\ l\ <=\ l'}
\eprog
This specifies a {\em lexicographic} order:  Constructors are ordered
by the order of their appearance in the \mbox{\tt data} declaration, and the
arguments of a constructor are compared from left to right.  Recall
that the built-in list type is semantically equivalent to an ordinary
two-constructor type.  In fact, this is the full declaration:
\bprog
\mbox{\tt data\ [a]\ \ \ \ \ \ \ \ =\ []\ |\ a\ :\ [a]\ deriving\ (Eq,\ Ord)\ \ \ \ \ --\ pseudo-code}
\eprog
(Lists also have \mbox{\tt Show} and \mbox{\tt Read} instances, which are not derived.)
The derived \mbox{\tt Eq} and \mbox{\tt Ord} instances for lists are the usual ones; in
particular, character strings, as lists of characters, are ordered as
determined by the underlying \mbox{\tt Char} type, with an initial substring
comparing less than a longer string; for example, \mbox{\tt "cat"\ <\ "catalog"}.

In practice, \mbox{\tt Eq} and \mbox{\tt Ord} instances are almost always derived,
rather than user-defined.  In fact, we should provide our own
definitions of equality and ordering predicates only with some
trepidation, being careful to maintain the expected algebraic
properties of equivalence relations and total orders.
An intransitive \mbox{\tt (==)} predicate, for example, could be disastrous,
confusing readers of the program and confounding manual or
automatic program transformations that rely on the \mbox{\tt (==)} predicate's
being an approximation to definitional equality.  Nevertheless,
it is sometimes necessary to provide \mbox{\tt Eq} or \mbox{\tt Ord} instances 
different from those that would be derived; probably the most
important example is that of an abstract data type in which different
concrete values may represent the same abstract value.

An enumerated type can have a derived \mbox{\tt Enum} instance, and here again,
the ordering is that of the constructors in the \mbox{\tt data} declaration.
For example:
\bprog
\mbox{\tt data\ Day\ =\ Sunday\ |\ Monday\ |\ Tuesday\ |\ Wednesday}\\
\mbox{\tt \ \ \ \ \ \ \ \ \ |\ Thursday\ |\ Friday\ |\ Saturday\ \ \ \ \ \ \ \ \ deriving\ (Enum)}
\eprog 
Here are some simple examples using the derived instances for this type:
\[\ba{lcl}
\mbox{\tt [Wednesday\ ..\ Friday]}   \ \ &\red&\ \ \mbox{\tt [Wednesday,\ Thursday,\ Friday]}\\
\mbox{\tt [Monday,\ Wednesday\ ..]}\ \ &\red&\ \ \mbox{\tt [Monday,\ Wednesday,\ Friday]}
\ea\]

Derived \mbox{\tt Read} (\mbox{\tt Show}) instances are possible for all types
whose component types also have \mbox{\tt Read} (\mbox{\tt Show}) instances.
(\mbox{\tt Read} and \mbox{\tt Show} instances for most of the standard types
are provided by the Prelude.  Some types, such as the function type \mbox{\tt (->)},
have a \mbox{\tt Show} instance but not a corresponding \mbox{\tt Read}.)  The textual
representation 
defined by a derived \mbox{\tt Show} instance is consistent with the
appearance of constant Haskell expressions of the type in question.
For example, if we add \mbox{\tt Show} and \mbox{\tt Read} to the \mbox{\tt deriving} clause for type
\mbox{\tt Day}, above, we obtain
\[ \mbox{\tt show\ [Monday\ ..\ Wednesday]}\ \ \ \ \red\ \ \ \ 
   \mbox{\tt "[Monday,Tuesday,Wednesday]"}
\]

%**~footer
