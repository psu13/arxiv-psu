%**<title>A Gentle Introduction to Haskell: Classes</title>
%**~header
\section{Type Classes and Overloading}
\label{tut-type-classes}

There is one final feature of Haskell's type system that sets it apart
from other programming languages.  The kind of polymorphism that we
have talked about so far is commonly called {\em parametric}
polymorphism.  There is another kind called {\em ad hoc} polymorphism,
better known as {\em overloading}.  Here are some examples of {\em ad hoc}
polymorphism:
\begin{itemize}
\item The literals \mbox{\tt 1}, \mbox{\tt 2}, etc. are often used to represent both
fixed and arbitrary precision integers.

\item Numeric operators such as \mbox{\tt +} are often defined to work on
many different kinds of numbers.

\item The equality operator (\mbox{\tt ==} in Haskell) usually works on
numbers and many other (but not all) types.

\end{itemize}
Note that these overloaded behaviors are different for each type
(in fact the behavior is sometimes undefined, or error), whereas in
parametric polymorphism the type truly does not matter (\mbox{\tt fringe}, for
example, really doesn't care what kind of elements are found in the
leaves of a tree).  In Haskell, {\em type classes} provide a
structured way to control {\em ad hoc} polymorphism, or overloading.

Let's start with a simple, but important, example: equality.
There are many types for which we would like equality defined, but
some for which we would not.  For example, comparing the equality of
functions is generally considered computationally intractable, whereas
we often want to compare two lists for equality.\footnote{The kind of
equality we are referring to here is ``value equality,'' and opposed
to the ``pointer equality'' found, for example, with Java's \mbox{\tt ==}.
Pointer equality is not referentially transparent, and thus does not
sit well in a purely functional language.} To highlight the issue,
consider this definition of the function \mbox{\tt elem} which tests for
membership in a list:
\bprog
\mbox{\tt x\ `elem`\ \ []\ \ \ \ \ \ \ \ \ \ \ \ =\ False}\\
\mbox{\tt x\ `elem`\ (y:ys)\ \ \ \ \ \ \ \ \ =\ x==y\ ||\ (x\ `elem`\ ys)}
\eprog
\syn{For the stylistic reason we discussed in Section \ref{tut-lambda},
we have chosen to define \mbox{\tt elem} in infix form.  \mbox{\tt ==} and \mbox{\tt ||} are the
infix operators for equality and logical or, respectively.}

\noindent Intuitively speaking, the type of \mbox{\tt elem} ``ought'' to be:
\mbox{\tt a->[a]->Bool}.  But this would imply that \mbox{\tt ==} has type \mbox{\tt a->a->Bool},
even though we just said that we don't expect \mbox{\tt ==} to be defined for
all types.

Furthermore, as we have noted earlier, even if \mbox{\tt ==} were 
defined on all types, comparing two lists for equality is very
different from comparing two integers.  In this sense, we expect \mbox{\tt ==}
to be {\em overloaded} to carry on these various tasks.

{\em Type classes} conveniently solve both of these problems.  They
allow us to declare which types are {\em instances} of which class,
and to provide definitions of the overloaded {\em operations}
associated with a class.  For example, let's define a type class
containing an equality operator:
\bprog
\mbox{\tt class\ Eq\ a\ where\ }\\
\mbox{\tt \ \ (==)\ \ \ \ \ \ \ \ \ \ \ \ \ \ \ \ \ \ ::\ a\ ->\ a\ ->\ Bool}
\eprog
Here \mbox{\tt Eq} is the name of the class being defined, and \mbox{\tt ==} is the
single operation in the class.  This declaration may be read ``a type
\mbox{\tt a} is an instance of the class \mbox{\tt Eq} if there is an (overloaded)
operation \mbox{\tt ==}, of the appropriate type, defined on it.''  (Note that
\mbox{\tt ==} is only defined on pairs of objects of the same type.)

The constraint that a type \mbox{\tt a} must be an instance of the class \mbox{\tt Eq}
is written \mbox{\tt Eq\ a}.  Thus \mbox{\tt Eq\ a} is not a type expression, but rather
it expresses a constraint on a type, and is called a {\em context}.
Contexts are placed at the front of type expressions.  For example,
the effect of the above class declaration is to assign the following
type to \mbox{\tt ==}:
\bprog
\mbox{\tt (==)\ \ \ \ \ \ \ \ \ \ \ \ \ \ \ \ \ \ \ \ ::\ (Eq\ a)\ =>\ a\ ->\ a\ ->\ Bool}
\eprog
This should be read, ``For every type \mbox{\tt a} that is an instance of the
class \mbox{\tt Eq}, \mbox{\tt ==} has type \mbox{\tt a->a->Bool}''.  This is the type that would
be used for \mbox{\tt ==} in the \mbox{\tt elem} example, and indeed the constraint
imposed by the context propagates to the principal type for \mbox{\tt elem}:
\bprog
\mbox{\tt elem\ \ \ \ \ \ \ \ \ \ \ \ \ \ \ \ \ \ \ \ ::\ (Eq\ a)\ =>\ a\ ->\ [a]\ ->\ Bool}
\eprog
This is read, ``For every type \mbox{\tt a} that is an instance of the
class \mbox{\tt Eq}, \mbox{\tt elem} has type \mbox{\tt a->[a]->Bool}''.  This is just what we
want---it expresses the fact that \mbox{\tt elem} is not defined on all
types, just those for which we know how to compare elements for
equality.

So far so good.  But how do we specify which types are instances of
the class \mbox{\tt Eq}, and the actual behavior of \mbox{\tt ==} on each of those
types?  This is done with an {\em instance declaration}.  For example:
\bprog
\mbox{\tt instance\ Eq\ Integer\ where\ }\\
\mbox{\tt \ \ x\ ==\ y\ \ \ \ \ \ \ \ \ \ \ \ \ \ \ \ =\ \ x\ `integerEq`\ y}
\eprog
The definition of \mbox{\tt ==} is called a {\em method}.  The function \mbox{\tt integerEq}
happens to 
be the primitive function that compares integers for equality, but in
general any valid expression is allowed on the right-hand side, just
as for any other function definition.  The overall declaration is
essentially saying: ``The type \mbox{\tt Integer} is an instance of the class \mbox{\tt Eq},
and here is the definition of the method corresponding to the
operation \mbox{\tt ==}.''  Given this declaration, we can now compare fixed
precision integers for equality using \mbox{\tt ==}.  Similarly:
\bprog
\mbox{\tt instance\ Eq\ Float\ where}\\
\mbox{\tt \ \ x\ ==\ y\ \ \ \ \ \ \ \ \ \ \ \ \ \ \ \ =\ \ x\ `floatEq`\ y}
\eprog
allows us to compare floating point numbers using \mbox{\tt ==}.

Recursive types such as \mbox{\tt Tree} defined earlier can also be handled:
\bprog
\mbox{\tt instance\ (Eq\ a)\ =>\ Eq\ (Tree\ a)\ where\ }\\
\mbox{\tt \ \ Leaf\ a\ \ \ \ \ \ \ \ \ ==\ Leaf\ b\ \ \ \ \ \ \ \ \ \ =\ \ a\ ==\ b}\\
\mbox{\tt \ \ (Branch\ l1\ r1)\ ==\ (Branch\ l2\ r2)\ \ =\ \ (l1==l2)\ {\char'46}{\char'46}\ (r1==r2)}\\
\mbox{\tt \ \ {\char'137}\ \ \ \ \ \ \ \ \ \ \ \ \ \ ==\ {\char'137}\ \ \ \ \ \ \ \ \ \ \ \ \ \ \ =\ \ False}
\eprog
Note the context \mbox{\tt Eq\ a} in the first line---this is necessary because
the elements in the leaves (of type \mbox{\tt a}) are compared for equality in
the second line.  The additional constraint is essentially saying that we can
compare trees of \mbox{\tt a}'s for equality as long as we know how to compare
\mbox{\tt a}'s for equality.  If the context were omitted from the instance
declaration, a static type error would result.

The Haskell Report, especially the Prelude, contains a wealth
of useful examples of type classes.  
Indeed, a class \mbox{\tt Eq} is defined
that is slightly larger than the one defined earlier:
\bprog
\mbox{\tt class\ \ Eq\ a\ \ where}\\
\mbox{\tt \ \ (==),\ (/=)\ \ \ \ \ \ \ \ \ \ \ \ ::\ a\ ->\ a\ ->\ Bool}\\
\mbox{\tt \ \ x\ /=\ y\ \ \ \ \ \ \ \ \ \ \ \ \ \ \ \ =\ \ not\ (x\ ==\ y)}
\eprog
This is an example of a class with two operations, one for
equality, the other for inequality.  It also demonstrates the use of a
{\em default method}, in this case for the inequality operation \mbox{\tt /=}.
If a method for a particular operation is omitted in an instance
declaration, then the default one defined in the class declaration, if
it exists, is used instead.  For example, the three instances of \mbox{\tt Eq}
defined earlier will work perfectly well with the above class
declaration, yielding just the right definition of inequality that we
want: the logical negation of equality.

Haskell also supports a notion of {\em class extension}.  For example,
we may wish to define a class \mbox{\tt Ord} which {\em inherits} all of the
operations in \mbox{\tt Eq}, but in addition has a set of comparison operations
and minimum and maximum functions:
\bprog
\mbox{\tt class\ \ (Eq\ a)\ =>\ Ord\ a\ \ where}\\
\mbox{\tt \ \ (<),\ (<=),\ (>=),\ (>)\ \ ::\ a\ ->\ a\ ->\ Bool}\\
\mbox{\tt \ \ max,\ min\ \ \ \ \ \ \ \ \ \ \ \ \ \ ::\ a\ ->\ a\ ->\ a}
\eprog 
Note the context in the \mbox{\tt class} declaration.  We say that \mbox{\tt Eq} is a
{\em superclass} of \mbox{\tt Ord} (conversely, \mbox{\tt Ord} is a {\em subclass} of
\mbox{\tt Eq}), and any type which is an instance of \mbox{\tt Ord} must also be an
instance of \mbox{\tt Eq}. 
(In the next Section we give a fuller definition of \mbox{\tt Ord} taken from
the Prelude.)

One benefit of such class inclusions is shorter contexts: a type
expression for a function that uses operations from both the \mbox{\tt Eq} and
\mbox{\tt Ord} classes can use the context \mbox{\tt (Ord\ a)}, rather than 
\mbox{\tt (Eq\ a,\ Ord\ a)}, since \mbox{\tt Ord} ``implies'' \mbox{\tt Eq}.  More importantly,
methods for subclass operations can assume the existence of methods
for superclass operations.  For example, the \mbox{\tt Ord} declaration in the
Standard Prelude contains this default method for \mbox{\tt (<)}: 
\bprog
\mbox{\tt \ \ \ \ x\ <\ y\ \ \ \ \ \ \ \ \ \ \ \ \ \ \ =\ \ x\ <=\ y\ {\char'46}{\char'46}\ x\ /=\ y}
\eprog
As an example of the use of \mbox{\tt Ord}, the principal typing of \mbox{\tt quicksort}
defined in Section \ref{tut-list-comps} is:
\bprog
\mbox{\tt quicksort\ \ \ \ \ \ \ \ \ \ \ \ \ \ \ ::\ \ (Ord\ a)\ =>\ [a]\ ->\ [a]}
\eprog
In other words, \mbox{\tt quicksort} only operates on lists of values of 
ordered types.  This typing for \mbox{\tt quicksort} arises because of the use
of the comparison operators \mbox{\tt <} and \mbox{\tt >=} in its definition.

Haskell also permits {\em multiple inheritance}, since classes may
have more than one superclass.  For example, the declaration
\bprog
\mbox{\tt class\ (Eq\ a,\ Show\ a)\ =>\ C\ a\ where\ ...}
\eprog
creates a class \mbox{\tt C} which inherits operations from both \mbox{\tt Eq} and \mbox{\tt Show}.

Class methods are treated as top level declarations in
Haskell.  They share the same namespace as ordinary variables; a name
cannot be used to denote both a class method and a variable or methods
in different classes.

Contexts are also allowed in \mbox{\tt data} declarations; see \see{datatype-decls}.

Class methods may have additional class constraints on any type
variable except the one defining the current class.  For example, in
this class:
\bprog
\mbox{\tt class\ C\ a\ where}\\
\mbox{\tt \ \ m\ \ \ \ \ \ \ \ \ \ \ \ \ \ \ \ \ \ \ \ \ ::\ Show\ b\ =>\ a\ ->\ b}
\eprog
the method \mbox{\tt m} requires that type \mbox{\tt b} is in class \mbox{\tt Show}.  However, the
method \mbox{\tt m} could not place any additional class constraints on type
\mbox{\tt a}.  These would instead have to be part of the context in the class
declaration. 

So far, we have been using ``first-order'' types.  For example, the
type constructor \mbox{\tt Tree} has so far always been paired with an
argument, as in \mbox{\tt Tree\ Integer} (a tree containing \mbox{\tt Integer} values) or
\mbox{\tt Tree\ a} 
(representing the family of trees containing \mbox{\tt a} values).  But \mbox{\tt Tree}
by itself is a type constructor, and as such takes a type as an
argument and returns a type as a result.  There are no values in
Haskell that have this type, but such ``higher-order'' types can be
used in \mbox{\tt class} declarations.

To begin, consider the following \mbox{\tt Functor} class (taken from the Prelude):
\bprog
\mbox{\tt class\ Functor\ f\ where}\\
\mbox{\tt \ \ fmap\ \ \ \ \ \ \ \ \ \ \ \ \ \ \ \ \ \ ::\ (a\ ->\ b)\ ->\ f\ a\ ->\ f\ b}
\eprog
The \mbox{\tt fmap} function generalizes the \mbox{\tt map} function used previously.
Note that the type variable \mbox{\tt f} is applied to other types in \mbox{\tt f\ a} and
\mbox{\tt f\ b}.  Thus we would expect it to be bound to a type such as \mbox{\tt Tree}
which can be applied to an argument.  An instance of \mbox{\tt Functor}
for type \mbox{\tt Tree} would be:
\bprog
\mbox{\tt instance\ Functor\ Tree\ where}\\
\mbox{\tt \ \ fmap\ f\ (Leaf\ x)\ \ \ \ \ \ \ =\ Leaf\ \ \ (f\ x)}\\
\mbox{\tt \ \ fmap\ f\ (Branch\ t1\ t2)\ =\ Branch\ (fmap\ f\ t1)\ (fmap\ f\ t2)}
\eprog
This instance declaration declares that \mbox{\tt Tree}, rather than \mbox{\tt Tree\ a},
is an instance of \mbox{\tt Functor}.  This capability is quite useful, and
here demonstrates the ability to describe generic ``container'' types,
allowing functions such as \mbox{\tt fmap} to work uniformly over arbitrary
trees, lists, and other data types.

\syn{Type applications are written in the same manner as
function applications.  The type \mbox{\tt T\ a\ b} is parsed as \mbox{\tt (T\ a)\ b}.
Types such as tuples which use special syntax can be written in an
alternative style which allows currying.  For functions, \mbox{\tt (->)} is a
type constructor; the types \mbox{\tt f\ ->\ g} and \mbox{\tt (->)\ f\ g} are the same.
Similarly, the types \mbox{\tt [a]} and \mbox{\tt []\ a} are the same.  For tuples, the
type constructors (as well as the data constructors) are \mbox{\tt (,)},
\mbox{\tt (,,)}, and so on.}

As we know, the type system detects typing errors in expressions.  But
what about errors due to malformed type expressions?  The expression
\mbox{\tt (+)\ 1\ 2\ 3} results in a type error since \mbox{\tt (+)} takes only two arguments.
Similarly, the type \mbox{\tt Tree\ Int\ Int} should produce some sort of an
error since the \mbox{\tt Tree} type takes only a single argument.  So, how
does Haskell detect malformed type expressions?  The answer is a second
type system which ensures the correctness of types!  Each
type has an associated \mbox{$\it kind$} which ensures that the type is used
correctly.

Type expressions are classified into different {\em kinds} which take
one of two possible forms:
\begin{itemize}
\item The symbol $\ast$ represents the kind of type associated with
concrete data objects.  That is, if the value \mbox{$\it v$} has type \mbox{$\it t$}, the
kind of \mbox{$\it v$} must be $\ast$.

\item If $\kappa_1$ and $\kappa_2$ are kinds, then
$\kappa_1\rightarrow\kappa_2$ is the kind of types that take a type of
kind $\kappa_1$ and return a type of kind $\kappa_2$.
\end{itemize}
The type constructor \mbox{\tt Tree} has the kind $\ast\rightarrow\ast$; the
type \mbox{\tt Tree\ Int} has the kind $\ast$.  Members of the \mbox{\tt Functor} class
must all have the kind $\ast\rightarrow\ast$; a kinding error would
result from an declaration such as 
\bprog
\mbox{\tt instance\ Functor\ Integer\ where\ ...}
\eprog
since \mbox{\tt Integer} has the kind $\ast$.  

Kinds do not appear directly in Haskell programs.
The compiler infers kinds before doing type checking without any need
for `kind declarations'.  Kinds stay in the background of a Haskell
program except when an erroneous  type signature leads to a kind
error.  Kinds are 
simple enough that compilers should be able to provide descriptive
error messages when kind conflicts occur.  See 
\see{type-syntax} and \see{kindinference} for more information about
kinds.

\paragraph*{A Different Perspective.}
Before going on to further examples of the use of type classes, it is
worth pointing out two other views of Haskell's type classes.
The first is by analogy with object-oriented programming (OOP).  In the
following general statement about OOP, simply substituting {\em type
class} for {\em class}, and {\em type} for {\em object}, yields a valid
summary of Haskell's type class mechanism:

``{\em Classes} capture common sets of operations.  A particular
{\em object} may be an instance of a {\em class}, and will have a
method corresponding to each operation.  {\em Classes} may be arranged
hierarchically, forming notions of super{\em classes} and sub{\em
classes}, and permitting inheritance of operations/methods.
A default method may also be associated with an operation.''

In contrast to OOP, it should be clear that types are not
objects, and in particular there is no notion of an object's or type's
internal mutable state.  An advantage over some OOP languages is that
methods in 
Haskell are completely type-safe: any attempt to apply a method to a
value whose type is not in the required class will be detected at
compile time instead of at runtime.  In other words, methods are not
``looked up'' at runtime but are simply passed as higher-order
functions.

A different perspective can be gotten by considering the relationship
between parametric and {\em ad hoc} polymorphism.  We have shown how
parametric polymorphism is useful in defining families of types by
universally quantifying over all types.  Sometimes, however,
that universal quantification is too broad---we wish to quantify over
some smaller set of types, such as those types whose elements can be
compared for equality.  Type classes can be seen as providing a
structured way to do just this.  Indeed, we can think of parametric
polymorphism as a kind of overloading too!  It's just that the
overloading occurs implicitly over all types instead of a constrained
set of types (i.e.~a type class).

\paragraph*{Comparison to Other Languages.}

The classes used by Haskell are similar to those used in other
object-oriented languages such as C++ and Java.  However, there are
some significant differences:
\begin{itemize}
\item Haskell separates the definition of a type from the definition
of the methods associated with that type.  A class in C++ or Java
usually defines both a data structure (the member variables) and the
functions associated with the structure (the methods).  In Haskell,
these definitions are separated.
\item The class methods defined by a Haskell class correspond to
virtual functions in a C++ class.  Each instance of a class provides
its own definition for each method; class defaults correspond to
default definitions for a virtual function in the base class. 
\item Haskell classes are roughly similar to a Java interface.  Like
an interface declaration, a Haskell class declaration defines a
protocol for using an object rather than defining an object itself.
\item Haskell does not support the C++ overloading style in which
functions with different types share a common name. 
\item The type of a Haskell object cannot be implicitly coerced; there
is no universal base class such as \mbox{\tt Object} which values can be
projected into or out of.
\item C++ and Java attach identifying information (such as a VTable)
to the runtime representation of an object.  In Haskell, such
information is attached logically instead of physically to values,
through the type system.
\item There is no access control (such as public or private class
constituents) built into the Haskell class system. Instead, the module
system must be used to hide or reveal components of a class.
\end{itemize}

%**~footer
