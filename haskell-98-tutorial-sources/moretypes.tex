%**<title>A Gentle Introduction to Haskell: Types, Again</title>
%**~header
\section{Types, Again}

Here we examine some of the more advanced aspects of type
declarations.  

\subsection{The Newtype Declaration}

A common programming practice is to define a type whose representation
is identical to an existing one but which has a separate identity in
the type system.  
In Haskell, the \mbox{\tt newtype} declaration creates a new type from an
existing one.  For example, natural numbers can be represented by
the type \mbox{\tt Integer} using the following declaration:
\bprog
\mbox{\tt newtype\ Natural\ =\ MakeNatural\ Integer}
\eprog
This creates an entirely new type, \mbox{\tt Natural}, whose only
constructor contains a single \mbox{\tt Integer}.  The constructor \mbox{\tt MakeNatural} 
converts between an \mbox{\tt Natural} and an \mbox{\tt Integer}:
\bprog
\mbox{\tt toNatural\ \ \ \ \ \ \ \ \ \ \ \ \ \ \ ::\ Integer\ ->\ Natural}\\
\mbox{\tt toNatural\ x\ |\ x\ <\ 0\ \ \ \ \ =\ error\ "Can't\ create\ negative\ naturals!"\ }\\
\mbox{\tt \ \ \ \ \ \ \ \ \ \ \ \ |\ otherwise\ =\ MakeNatural\ x}\\
\mbox{\tt }\\[-8pt]
\mbox{\tt fromNatural\ \ \ \ \ \ \ \ \ \ \ \ \ ::\ Natural\ ->\ Integer}\\
\mbox{\tt fromNatural\ (MakeNatural\ i)\ =\ i}
\eprog
The 
following instance declaration admits \mbox{\tt Natural} to the \mbox{\tt Num} class:
\bprog
\mbox{\tt instance\ Num\ Natural\ where}\\
\mbox{\tt \ \ \ \ fromInteger\ \ \ \ \ \ \ \ \ =\ toNatural}\\
\mbox{\tt \ \ \ \ x\ +\ y\ \ \ \ \ \ \ \ \ \ \ \ \ \ \ =\ toNatural\ (fromNatural\ x\ +\ fromNatural\ y)}\\
\mbox{\tt \ \ \ \ x\ -\ y\ \ \ \ \ \ \ \ \ \ \ \ \ \ \ =\ let\ r\ =\ fromNatural\ x\ -\ fromNatural\ y\ in}\\
\mbox{\tt \ \ \ \ \ \ \ \ \ \ \ \ \ \ \ \ \ \ \ \ \ \ \ \ \ \ \ \ if\ r\ <\ 0\ then\ error\ "Unnatural\ subtraction"}\\
\mbox{\tt \ \ \ \ \ \ \ \ \ \ \ \ \ \ \ \ \ \ \ \ \ \ \ \ \ \ \ \ \ \ \ \ \ \ \ \ \ else\ toNatural\ r}\\
\mbox{\tt \ \ \ \ x\ *\ y\ \ \ \ \ \ \ \ \ \ \ \ \ \ \ =\ toNatural\ (fromNatural\ x\ *\ fromNatural\ y)}
\eprog
Without this declaration, \mbox{\tt Natural} would not be in \mbox{\tt Num}.  Instances
declared for the old type do not carry over to the new one.  Indeed,
the whole purpose of this type is to introduce a different \mbox{\tt Num}
instance.  This would not be possible if \mbox{\tt Natural} were
defined as a type synonym of \mbox{\tt Integer}.  

All of this works using a \mbox{\tt data} declaration instead of a
\mbox{\tt newtype} declaration.  However, the \mbox{\tt data} declaration 
incurs extra overhead in the representation of \mbox{\tt Natural} values.  The
use of \mbox{\tt newtype} avoids the extra level of indirection (caused by
laziness) that the \mbox{\tt data} declaration would introduce.  
See section 
\ref{datatype-renaming} of the report for a more discussion of the
relation between \mbox{\tt newtype}, \mbox{\tt data}, and \mbox{\tt type} declarations.
 
\syn{Except for the keyword, the \mbox{\tt newtype} declaration uses the same
syntax as a \mbox{\tt data} declaration with a single constructor containing a
single field.  This is appropriate since types defined using \mbox{\tt newtype}
are nearly identical to those created by an ordinary \mbox{\tt data}
declaration.}

\subsection{Field Labels}

The fields within a Haskell data type can be accessed either
positionally or by name using \mbox{$\it field\ labels$}.
Consider a data type for a two-dimensional point:
\bprog
\mbox{\tt data\ Point\ =\ Pt\ Float\ Float}
\eprog
The two components of a \mbox{\tt Point} are the first and second arguments to the
constructor \mbox{\tt Pt}.  A function such as
\bprog
\mbox{\tt pointx\ \ \ \ \ \ \ \ \ \ \ \ \ \ \ \ \ \ ::\ Point\ ->\ Float}\\
\mbox{\tt pointx\ (Pt\ x\ {\char'137})\ \ \ \ \ \ \ \ \ =\ \ x}
\eprog
may be used to refer to the first component of a point in a more
descriptive way, but, for large structures, it becomes tedious to
create such functions by hand.

Constructors in a \mbox{\tt data} declaration may be declared
with associated \mbox{$\it field\ names$}, enclosed in braces.  These field names
identify the components of constructor by name rather than by position.
This is an alternative way to define \mbox{\tt Point}:
\bprog
\mbox{\tt data\ Point\ =\ Pt\ {\char'173}pointx,\ pointy\ ::\ Float{\char'175}}
\eprog
This data type is identical to the earlier definition
of \mbox{\tt Point}.  The constructor \mbox{\tt Pt} is the same in both cases.  However,
this declaration also defines two field names, \mbox{\tt pointx}
and \mbox{\tt pointy}.  These field names can be used as \mbox{$\it selector\ functions$} to
extract a component from a structure.  In this example, the selectors
are:
\bprog
\mbox{\tt pointx\ \ \ \ \ \ \ \ \ \ \ \ \ \ \ \ \ \ ::\ \ \ Point\ ->\ Float\ }\\
\mbox{\tt pointy\ \ \ \ \ \ \ \ \ \ \ \ \ \ \ \ \ \ ::\ \ \ Point\ ->\ Float\ }
\eprog
This is a function using these selectors: 
\bprog
\mbox{\tt absPoint\ \ \ \ \ \ \ \ \ \ \ \ \ \ \ \ ::\ Point\ ->\ Float}\\
\mbox{\tt absPoint\ p\ \ \ \ \ \ \ \ \ \ \ \ \ \ =\ \ sqrt\ (pointx\ p\ *\ pointx\ p\ +\ }\\
\mbox{\tt \ \ \ \ \ \ \ \ \ \ \ \ \ \ \ \ \ \ \ \ \ \ \ \ \ \ \ \ \ \ \ \ \ pointy\ p\ *\ pointy\ p)}
\eprog

Field labels can also be used to construct new values.  The expression
\mbox{\tt Pt\ {\char'173}pointx=1,\ pointy=2{\char'175}} is identical to \mbox{\tt Pt\ 1\ 2}.  The use of field
names in the declaration of a data constructor does not preclude the
positional style of field access; both
\mbox{\tt Pt\ {\char'173}pointx=1,\ pointy=2{\char'175}} and \mbox{\tt Pt\ 1\ 2} are allowed.  
When constructing a value using field names, some fields may be
omitted; these absent fields are undefined.   

Pattern matching using field names uses a similar syntax for the
constructor \mbox{\tt Pt}:
\bprog
\mbox{\tt absPoint\ (Pt\ {\char'173}pointx\ =\ x,\ pointy\ =\ y{\char'175})\ =\ sqrt\ (x*x\ +\ y*y)\ }
\eprog

An update function uses field values in an existing structure to fill
in components of a new structure.  If \mbox{\tt p} is a \mbox{\tt Point}, then 
\mbox{\tt p\ {\char'173}pointx=2{\char'175}} is a point with the same \mbox{\tt pointy} as \mbox{\tt p} but with
\mbox{\tt pointx} replaced by \mbox{\tt 2}.  This is not a destructive update:
the update function merely creates a new copy of the object, filling
in the specified fields with new values.

\syn{The braces used in conjunction with field labels are somewhat
special: Haskell syntax usually allows braces to be omitted using the
\mbox{$\it layout\ rule$} (described in Section \ref{tut-layout}).  However, the
braces associated with field names must be explicit.}

%Within the braces used to name the fields of a structure, {\em
%punning}---using the same word in two different ways---can be used to
%abbreviate bindings which associate a field name with a local variable
%of the same name.  That is, \mbox{\tt {\char'173}pointx{\char'175}} abbreviates \mbox{\tt {\char'173}pointx\ =\ pointx{\char'175}}
%in a field list.  Thus, the \mbox{\tt abs} function could also be written
%\bprog
%
%abs (Pt {pointx, pointy}) = sqrt (pointx*pointx + pointy*pointy)
%
%\eprog

Field names are not restricted to types with a single constructor
(commonly called `record' types).  In a type with multiple
constructors, selection or update operations using field names may
fail at runtime.  This is similar to the behavior of the \mbox{\tt head}
function when applied to an empty list.

Field labels share the top level namespace with ordinary variables and
class methods.
A field name cannot be used in more than one data type in scope.
However, within a data type, the same field
name can be used in more than one of the constructors so long as it
has the same typing in all cases.  For example, in this data type
\bprog
\mbox{\tt data\ T\ =\ C1\ {\char'173}f\ ::\ Int,\ g\ ::\ Float{\char'175}}\\
\mbox{\tt \ \ \ \ \ \ \ |\ C2\ {\char'173}f\ ::\ Int,\ h\ ::\ Bool{\char'175}}
\eprog
the field name \mbox{\tt f} applies to both constructors in \mbox{\tt T}.  Thus if
\mbox{\tt x} is of type \mbox{\tt T}, then \mbox{\tt x\ {\char'173}f=5{\char'175}} will work for values created by
either of the constructors in \mbox{\tt T}.

Field names does not change the basic nature of an algebraic
data type; they are simply a convenient syntax for accessing the
components of a data structure by name rather than by position.  They
make constructors with many components 
more manageable since fields can be added or removed without changing
every reference to the constructor.  For full details of field labels
and their semantics, see Section~\see{field-labels}.

\subsection{Strict Data Constructors}
\label{tut-strict-flag}
Data structures in Haskell are generally \mbox{$\it lazy$}: the
components are not evaluated until 
needed.  This permits structures that contain elements which, if
evaluated, would lead to an error or fail to terminate.  Lazy data
structures enhance the expressiveness of Haskell and are an
essential aspect of the Haskell programming style.

Internally, each field of a lazy data object is wrapped up in a structure
commonly referred to as a \mbox{$\it thunk$} that encapsulates the computation
defining the field value.  This thunk is not entered until
the value is needed; thunks which contain errors (\mbox{$\it \bot$}) do not affect other
elements of a data structure.   For 
example, the tuple \mbox{\tt ('a',}\mbox{$\it \bot$}\mbox{\tt )} is a perfectly legal Haskell
value.  The
\mbox{\tt 'a'} may be used without disturbing the other component of the tuple.
Most programming languages are 
\mbox{$\it strict$} instead of lazy: that is, all components of a data structure
are reduced to values before being placed in the structure.

There are a number of overheads associated with thunks: they take time to
construct and evaluate, they occupy space in the heap, and they cause
the garbage collector to retain other structures needed for the
evaluation of the thunk.  
To avoid these overheads, {\em strictness flags} in \mbox{\tt data} declarations
allow specific fields of a constructor to be evaluated immediately,
selectively suppressing laziness.  A field
marked by \mbox{$\it \makebox{\tt !}$} in a \mbox{\tt data} declaration is evaluated when
the structure is created instead of delayed in a thunk.
 
There are a number of situations where it may be appropriate to use
strictness flags:
\begin{itemize}
\item Structure components that are sure to be evaluated at some
point during program execution.  
\item Structure components that are simple to evaluate and never
cause errors. 
\item Types in which partially undefined values are not meaningful.
\end{itemize}
For example, the complex number library defines the \mbox{\tt Complex} type as:
\bprog
\mbox{\tt data\ RealFloat\ a\ =>\ Complex\ a\ =\ !a\ :+\ !a}
\eprog
\syn{note the infix definition of the constructor \mbox{\tt :+}.} This definition
marks the two components, the real and imaginary parts, of the complex
number as being strict.  This is a more compact representation of
complex numbers but this comes at the expense of making a complex
number with an undefined component, \mbox{\tt 1\ :+\ } \mbox{$\it \bot$}  for example,
totally undefined (\mbox{$\it \bot$}).  As there is no real need for partially 
defined complex numbers, it makes sense to use strictness flags to
achieve a more efficient representation.

Strictness flags may be used to address memory leaks: structures
retained by the garbage collector but no longer necessary for computation.

The strictness flag, \mbox{\tt !}, can only appear in \mbox{\tt data} declarations.
It cannot be used in other type
signatures or in any other type definitions.  There is no
corresponding way to mark function arguments as being strict, although
the same effect can be obtained using the \mbox{\tt seq} or \mbox{\tt !{\char'44}} functions.  See
~\see{strictness-flags} for further details. 

It is difficult to present exact guidelines for the use of strictness
flags.  They should be used with caution: laziness is one of the
fundamental properties of Haskell and adding strictness flags may lead
to hard to find infinite loops or have other unexpected consequences.

%**~footer


