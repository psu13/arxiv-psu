\section{Presentable $\infty$-Categories}\label{c5s6}

Our final object of study in this chapter is the theory of {\em presentable} $\infty$-categories.

\begin{definition}\label{presdef}\index{gen}{presentable!$\infty$-category}\index{gen}{$\infty$-category!presentable}
An $\infty$-category $\calC$ is {\it presentable} if $\calC$ is accessible and admits small colimits.
\end{definition}

We will begin in \S \ref{presint} by giving a number of equivalent reformulations of Definition \ref{presdef}. The main result, Theorem \ref{pretop}, is due to Carlos Simpson: an $\infty$-category $\calC$ is presentable if and only if it arises as an (accessible) localization of an $\infty$-category of presheaves. 

Let $\calC$ be an $\infty$-category, and let $F: \calC \rightarrow \SSet^{op}$ be a functor. If $F$ is representable by an object of $\calC$, then $F$ preserves colimits (Proposition \ref{yonedaprop}). In \S \ref{aftt}, we will prove that the converse holds when $\calC$ is presentable. This representability criterion has a number of consequences: it implies that $\calC$ admits (small) limits (Corollary \ref{preslim}), and leads to an $\infty$-categorical analogue of the adjoint functor theorem (Corollary \ref{adjointfunctor}).

In \S \ref{colpres}, we will see that the collection of all presentable $\infty$-categories can be organized into an $\infty$-category $\LPres$. Moreover, we will explain how to compute limits and colimits in $\LPres$. In the course of doing so, we will prove that the class of presentable $\infty$-categories is stable under most of the basic constructions of higher category theory.

In view of Theorem \ref{pretop}, the theory of localizations plays a central role in the study of presentable $\infty$-categories. In \S \ref{invloc}, we will show that the collection of all (accessible) localizations of a presentable $\infty$-category $\calC$ can be parametrized in a very simple way. Moreover, there is a good supply of localizations of $\calC$: given any (small) collection of morphisms $S$ of $\calC$, one can construct a corresponding localization functor
$$\calC \stackrel{L}{\rightarrow} S^{-1} \calC \subseteq \calC,$$
where $S^{-1} \calC$ is a the full subcategory of $\calC$ spanned by the {\it $S$-local} objects.
These ideas are due to Bousfield, who works in the setting of model categories; we will give an exposition here in the language of $\infty$-categories. In \S \ref{factgen2}, we will employ the same techniques to produce examples of factorization systems on the $\infty$-category $\calC$.

Let $\calC$ be an $\infty$-category, and let $C \in \calC$ be an object. We will say that
$C \in \calC$ is {\em discrete} if, for every $D \in \calC$, the nonzero homotopy groups of the mapping space $\bHom_{\calC}(D,C)$ vanish. If we let $\tau_{\leq 0} \calC$ denote the full subcategory of $\calC$ spanned by the discrete objects, then $\tau_{\leq 0} \calC$ is (equivalent to) an ordinary category. If $\calC$ is the $\infty$-category of spaces, then we can identify the discrete objects of $\calC$ with the ordinary category of sets. Moreover, the inclusion $\tau_{\leq 0} \SSet \subseteq \SSet$ has a left adjoint, given by
$$ X \mapsto \pi_0 X.$$
In \S \ref{truncintro}, we will show that the preceding remark generalizes to an arbitrary presentable $\infty$-category $\calC$: the discrete objects of $\calC$ constitute an (accessible) localization of $\calC$. We will also consider a more general condition of $k$-truncatedness (which specializes to the condition of discreteness when $k=0$). The truncation functors which we construct will play an important role throughout \S \ref{chap6}. 

In \S \ref{compactgen}, we will study the theory of {\em compactly generated} $\infty$-categories: $\infty$-categories which are generated (under colimits) by their compact objects. This class of $\infty$-categories includes some of the most important examples, such as $\SSet$ and $\Cat_{\infty}$. 
In fact, the $\infty$-category $\SSet$ satisfies an even stronger condition: it is generated by
compact {\em projective} objects (see Definition \ref{humber}). The presence of enough compact projective objects in an $\infty$-category allows us to construct projective resolutions, which gives rise to the theory of nonabelian homological algebra (or ``homotopical algebra''). We will review the rudiments of this theory in \S \ref{stable11}. Finally, in \S \ref{stable12} we will present the same ideas in a more classical form, following Quillen's manuscript \cite{homotopicalalgebra}. The comparison of these two perspectives is based on a rectification result (Proposition \ref{trent}) which is of some independent interest.

\begin{remark}
We refer the reader to \cite{adamek} for a study of presentability in the setting of ordinary
category theory. Note that \cite{adamek} uses the term {\it locally presentable categories} for what we have chosen to call {\it presentable categories}.
\end{remark}

\setcounter{theorem}{0}
\subsection{Presentability}\label{presint}

Our main goal in this section is to establish the following characterization of presentable $\infty$-categories:

\begin{theorem}[Simpson \cite{simpson}]\label{pretop}\index{gen}{presentable!$\infty$-category}\index{gen}{$\infty$-category!presentable}
Let $\calC$ be an $\infty$-category. The following conditions are
equivalent:

\begin{itemize}
\item[$(1)$] The $\infty$-category $\calC$ is presentable.

\item[$(2)$] The $\infty$-category $\calC$ is accessible, and for every
regular cardinal $\kappa$, the full subcategory $\calC^{\kappa}$ admits $\kappa$-small colimits.

\item[$(3)$] There exists a regular cardinal $\kappa$ such $\calC$ is $\kappa$-accessible
and $\calC^{\kappa}$ admits $\kappa$-small colimits.

\item[$(4)$] There exists a regular cardinal $\kappa$, a small $\infty$-category $\calD$ which
admits $\kappa$-small colimits, and an equivalence $\Ind_{\kappa} \calD \rightarrow \calC$.

\item[$(5)$] There exists a small $\infty$-category $\calD$ such that $\calC$
is an accessible localization of $\calP(\calD)$.

\item[$(6)$] The $\infty$-category $\calC$ is locally small, admits small colimits, and there
exists a regular cardinal $\kappa$ and a (small) set $S$ of $\kappa$-compact objects of $\calC$ such that every object of $\calC$ is a colimit of a small diagram taking values in the full subcategory of $\calC$ spanned by $S$.
\end{itemize}
\end{theorem}

Before giving the proof, we need a few preliminaries remarks. We first observe that condition $(5)$ is potentially ambiguous: it is unclear whether the accessibility hypothesis is on $\calC$ or on the associated localization functor $L: \calP(\calD) \rightarrow \calP(\calD)$. The distinction turns out to be irrelevant, by virtue of the following:

\begin{proposition}\label{accloc}\index{gen}{localization!accessible}\index{gen}{accessible!localization}
Let $\calC$ be an accessible $\infty$-category, and let $L: \calC \rightarrow \calC$ be a functor satisfying the equivalent conditions of Proposition \ref{recloc}. The following conditions are equivalent:
\begin{itemize}
\item[$(1)$] The essential image $L \calC$ of $L$ is accessible.
\item[$(2)$] There exists a localization $f: \calC \rightarrow \calD$, where $\calD$ is accessible,
and an equivalence $L \simeq g \circ f$.
\item[$(3)$] The functor $L$ is accessible $($when regarded as a functor from $\calC$ to itself$)$.
\end{itemize}
\end{proposition}

\begin{proof}
Suppose $(1)$ is satisfied. Then we may take $\calD = L\calC$, $f = L$, and $g$ to be the inclusion $L\calC \subseteq \calC$; this proves $(2)$. If $(2)$ is satisfied, then Proposition \ref{adjoints} shows that $f$ and $g$ are accessible functors, so their composite $g \circ f \simeq L$ is
also accessible; this proves $(3)$. Now suppose that $(3)$ is satisfied. Choose a regular
cardinal $\kappa$ such that $\calC$ is $\kappa$-accessible and $L$ is $\kappa$-continuous.
The full subcategory $\calC^{\kappa}$ consisting of $\kappa$-compact objects of
$\calC$ is essentially small, so there exists a regular cardinal $\tau \gg \kappa$ such that
$LC$ is $\tau$-compact for every $C \in \calC^{\kappa}$. Let $\calC'$ denote the full subcategory of $\calC$ spanned by the colimits of all $\tau$-small, $\kappa$-filtered diagrams in $\calC^{\kappa}$, and let $L\calC'$ denote the essential image of $\calC'$ under $L$. We note that
$L\calC'$ is essentially small. Since $L$ is $\kappa$-continuous, $L \calC$ is stable under
small $\kappa$-filtered colimits in $\calC$. It follows that any $\tau$-compact object
of $\calC$ which belongs to $L \calC$ is also $\tau$-compact when viewed as an object
of $L \calC$, so that $L \calC'$ consists of $\tau$-compact objects of $L \calC$. According to Proposition \ref{clear}, to complete the proof that $L \calC$ is accessible it will suffice to
show that $L \calC'$ generates $L \calC$ under small, $\tau$-filtered colimits.

Let $X$ be an object of $\calC$. Then $X$ can be written as a small $\kappa$-filtered
colimit of objects of $\calC^{\kappa}$. The proof of Proposition \ref{enacc} shows that
we can also write $X$ as the colimit of a small $\tau$-filtered diagram in $\calC'$.
Since $L$ is preserves colimits, it follows that $LX$ can be obtaines as the colimit of a small $\tau$-filtered diagram in $L \calC'$.
\end{proof}

The proof of Theorem \ref{pretop} will require a few easy lemmas:

\begin{lemma}\label{idc}
Let $f: \calC \rightarrow \calD$ be a functor between small $\infty$-category which exhibits
$\calD$ as an idempotent completion of $\calC$, and let $\kappa$ be a regular cardinal. Then
$\Ind_{\kappa}(f): \Ind_{\kappa}(\calC) \rightarrow \Ind_{\kappa}(\calD)$ is an equivalence of $\infty$-categories.
\end{lemma}

\begin{proof}
We first apply Proposition \ref{uterr} to conclude that $\Ind_{\kappa}(f)$ is fully faithful. To prove
that $\Ind_{\kappa}(f)$ is an equivalence, we must show that it generates $\Ind_{\kappa}(\calD)$ under $\kappa$-filtered colimits. Since $\Ind_{\kappa}(\calD)$ is generated under $\kappa$-filtered colimits by the essential image of the Yoneda embedding $j_{\calD}: \calD \rightarrow \Ind_{\kappa}(\calD)$. Let $D$ be an object of $\calD$. Then $D$ is a retract of $f(C)$
for some object $C \in \calC$. Then $j_{\calD}(D)$ is a retract of
$(\Ind_{\kappa}(f) \circ j_{\calC})(C)$. Since $\Ind_{\kappa}(\calC)$ is idempotent complete
(Corollary \ref{swwe}), we conclude that $j_{\calD}(D)$ belongs to the essential image
of $\Ind_{\kappa}(f)$.
\end{proof}

\begin{lemma}\label{easybumb}
Let $F: \calC \rightarrow \calD$ be a functor between $\infty$-categories which admit
small, $\kappa$-filtered colimits, and let $G$ be a right adjoint to $F$. Suppose
that $G$ is $\kappa$-continuous. Then $F$ carries $\kappa$-compact objects of $\calC$ to $\kappa$-compact objects of $\calD$.
\end{lemma}

\begin{proof}
Let $C$ be a $\kappa$-compact object of $\calC$, $e_{C}: \calC \rightarrow \hat{\SSet}$ the functor corepresented by $C$, and $e_{F(C)}: \calD \rightarrow \hat{\SSet}$ the functor corepresented by $F(C)$. Since $G$ is a right adjoint to $F$, we have an equivalence $e_{F(C)} = e_{C} \circ G$.
Since $e_{C}$ and $G$ are both $\kappa$-continuous, $e_{FC}$ is $\kappa$-continuous.
It follows that $F(C)$ is $\kappa$-compact, as desired.
\end{proof}

\begin{proof}[Proof of Theorem \ref{pretop}]
Corollary \ref{tyrmyrr} asserts that the full subcategory $\calC^{\kappa}$ is stable under all
$\kappa$-small colimits which exist in $\calC$. This proves that $(1)$ implies $(2)$.
The implications $(2) \Rightarrow (3) \Rightarrow (4)$ are obvious. We next prove that $(4)$ implies
$(5)$. According to Lemma \ref{idc}, we may suppose without loss of generality that $\calD$ is idempotent complete. Let
$\calP^{\kappa}(\calD)$ denote the full subcategory of $\calP(\calD)$ spanned by the $\kappa$-compact objects, let $\calD'$ be a minimal model for $\calP^{\kappa}(\calD)$, and let $g$ denote the composition
$$ \calD \stackrel{j}{\rightarrow} \calP^{\kappa}(\calD) \rightarrow \calD'$$
where the second map is a homotopy inverse to the inclusion $\calD' \subseteq \calP^{\kappa}(\calD)$. Proposition \ref{fulfaith} implies that $g$ is fully faithful and Proposition \ref{kcolim} implies that $g$ admits a left adjoint $f$. It follows that $F = \Ind_{\kappa}(f)$ and $G = \Ind_{\kappa}(g)$ are adjoint functors, and Proposition \ref{uterr} implies that $G$ is fully faithful. Moreover,
Proposition \ref{precst} implies that $\Ind_{\kappa} \calD'$ is equivalent to $\calP(\calD)$,
so that $\calC$ is equivalent to an accessible localization of $\calP(\calD')$.

We now prove that $(5)$ implies $(6)$. Let $\calD$ be a small $\infty$-category and
$L: \calP(\calD) \rightarrow \calC$ an accessible localization. Remark \ref{localcolim} implies
that $\calC$ admits small colimits and that $\calC$ is generated under colimits by
the essential image of the composition
$$ T: \calD \stackrel{j}{\rightarrow} \calP(\calD) \stackrel{L}{\rightarrow} \calC.$$
To complete the proof of $(6)$, it will suffice to show that there exists a regular cardinal $\kappa$ such that the essential image of $T$ consists of $\kappa$-compact objects.
Let $G$ denote a left adjoint to $L$. By assumption, $G$ is an accessible functor so that there exists a regular cardinal $\kappa$ such that $G$ is $\kappa$-continuous. For each
object $D \in \calD$, the Yoneda image $j(D)$ is a completely compact object of
$\calP(\calD)$, and in particular $\kappa$-compact. Lemma \ref{easybumb} implies that
$T(D)$ is a $\kappa$-compact object of $\calC$. 

We now complete the proof by showing that $(6)
\Rightarrow (1)$.  Assume that there exists a regular cardinal
$\kappa$ and a set $S$ of $\kappa$-compact objects of $\calC$ such
that every object of $\calC$ is a colimit of objects in $S$. Let $\calC' \subseteq \calC$ be the full subcategory of $\calC$ spanned by $S$, and let $\calC'' \subseteq \calC$ be the full subcategory
of $\calC$ spanned by the colimits of all $\kappa$-small diagrams with values in $\calC''$.
Since $\calC'$ is essentially small, there is only a bounded number of such diagrams up to equivalence, so that $\calC''$ is essentially small. Moreover, since every object of $\calC$
is a colimit of a small diagram with values in $\calC'$, the proof of Corollary \ref{uterrr} shows that
every object of $\calC$ can also be obtained as the colimit of a small $\kappa$-filtered diagram with values in $\calC''$. Corollary \ref{tyrmyrr} implies that $\calC''$ consists of $\kappa$-compact objects of $\calC$ (a slightly more refined argument shows that, if $\kappa > \omega$, then $\calC''$ consists of {\em precisely} the $\kappa$-compact objects of $\calC$). We may therefore apply Proposition \ref{clear} to deduce that $\calC$ is accessible.
\end{proof}

\begin{remark}\label{modelcatpresentcat}
The characterization of presentable $\infty$-categories as localizations of presheaf $\infty$-categories was established by Simpson in \cite{simpson} (using a somewhat different language). 
The theory of presentable $\infty$-categories is essentially equivalent to the theory of {\em combinatorial} model categories (see \S \ref{turka} and Proposition \ref{notthereyet}).
Since most of the $\infty$-categories we will meet are presentable, our study
could also be phrased in the language of model categories. However, we will try to avoid this language, since for many purposes the restriction to presentable
$\infty$-categories seems unnatural and is often technically
inconvenient.
\end{remark}

\begin{remark}
Let $\calC$ be a presentable $\infty$-category, and let $\calD$ be an accessible localization of $\calC$. Then $\calD$ is presentable: this follows immediately from characterization $(5)$ of Proposition \ref{pretop}.
\end{remark}

\begin{remark}\label{tensored}
Let $\calC$ be a presentable $\infty$-category. Since $\calC$ admits arbitrary
colimits, it is ``tensored over spaces'', as we explained in 
\S \ref{quasilimit7}. In particular, the homotopy category of $\calC$ is naturally tensored over the homotopy category $\calH$: for each object $C$ of $\calC$ and every simplicial set $S$, there exists an object $C \otimes S$ of $\calC$, well defined up to equivalence, equipped with isomorphisms $$ \bHom_{\calC}(C \otimes S, C') \simeq
\bHom_{\calC}(C,C')^{S}$$
in the homotopy category $\calH$.
\end{remark}

\begin{example}\label{spacesarepresentable}
The $\infty$-category $\SSet$ of spaces is presentable. This follows from characterization
$(1)$ of Theorem \ref{pretop}, since $\SSet$ is accessible (Example \ref{spacesareaccessible}) and admits (small) colimits by Corollary \ref{limitsinmodel}.
\end{example}

According to Theorem \ref{pretop}, if $\calC$ is $\kappa$-accessible, then
$\calC$ admits small colimits if and only if the full subcategory $\calC^{\kappa} \subseteq \calC$
admits $\kappa$-small colimits. Roughly speaking, this is because arbitrary colimits in $\calC$ can be rewritten in terms of $\kappa$-filtered colimits and $\kappa$-small colimits of $\kappa$-compact objects. Our next result is another variation on this idea; it may also be regarded as an 
analogue of Theorem \ref{pretop} (which describes functors, rather than $\infty$-categories):

\begin{proposition}\label{sumatch}
Let $f: \calC \rightarrow \calD$ be a functor between presentable $\infty$-categories. Suppose that
$\calC$ is $\kappa$-accessible. The following conditions are equivalent:
\begin{itemize}
\item[$(1)$] The functor $f$ preserves small colimits.
\item[$(2)$] The functor $f$ is $\kappa$-continuous, and the restriction
$f| \calC^{\kappa}$ preserves $\kappa$-small colimits.
\end{itemize}
\end{proposition}

\begin{proof}
Without loss of generality, we may suppose $\calC = \Ind_{\kappa}( \calC')$, where $\calC'$ is a small, idempotent complete $\infty$-category which admits $\kappa$-small colimits. The proof of Theorem \ref{pretop} shows that the inclusion $\Ind_{\kappa}(\calC') \subseteq \calP(\calC')$ admits a left adjoint $L$.
Let $\alpha: \id_{\calP(\calC')} \rightarrow L$ be a unit for the adjunction, and let
$f': \calC' \rightarrow \calD$ denote the composition of $f$ with the Yoneda embedding
$j: \Ind_{\kappa}(\calC')$. According to Theorem \ref{charpresheaf}, there exists a
colimit-preserving functor $F: \calP(\calC') \rightarrow \calD$ and an equivalence
of $f'$ with $F \circ j$. Proposition \ref{intprop} implies that $f$ and $F| \Ind_{\kappa}(\calC)$
are equivalent; we may therefore assume without loss of generality that $f = F | \Ind_{\kappa}(\calC)$. Let $F' = f \circ L$, so that $\alpha$ induces a natural transformation 
$\beta: F \rightarrow F'$ of functors from $\calP(\calC')$ to $\calD$. We will show that
$\beta$ is an equivalence. Consequently, we deduce that the functor $F'$ is colimit preserving. 
It then follows that $f$ is colimit preserving. To see this, we consider an arbitrary diagram
$p: K \rightarrow \Ind_{\kappa}(\calC')$ and choose a colimit $\overline{p}: K^{\triangleright} \rightarrow \calP(\calC')$. Then $\overline{q} = L \circ \overline{p}$ is a colimit diagram in $\Ind_{\kappa}(\calC')$, and $f \circ \overline{q} = F' \circ \overline{p}$ is a colimit diagram in $\calD$. Since $q = \overline{q} | K$ is equivalent (via $\alpha$) to the original diagram $p$, we conclude that $f$ preserves the colimit of $p$ in $\Ind_{\kappa}(\calC')$, as well.

It remains to prove that $\beta$ is an equivalence of functors. Let $\calE \subseteq \calP(\calC')$ denote the full subcategory spanned by those objects $X \in \calP(\calC')$ for which
$\beta(X): F(X) \rightarrow F'(X)$ is an equivalence in $\calD$. We wish to prove that
$\calE = \calP(\calC')$. Since $F$ and $F'$
are both $\kappa$-continuous functors, $\calE$ is stable under $\kappa$-filtered colimits
in $\calP(\calC')$. It will therefore suffice to prove that $\calE$ contains $\calP^{\kappa}(\calC')$. 

It is clear that $\calE$ contains $\Ind_{\kappa}(\calC')$; in particular, $\calE$ contains
the essential image $\calE'$ of the Yoneda embedding $j: \calC' \rightarrow \calP(\calC')$. 
According to Proposition \ref{charsmallpre}, every object of $\calP^{\kappa}(\calC')$ is a retract of the colimit of a $\kappa$-small diagram $p: K \rightarrow \calE'$. Since $\calC'$ is idempotent complete, we may identify $\calE'$ with the full subcategory of $\Ind_{\kappa}(\calC')$ consisting of $\kappa$-compact objects. In particular, $\calE'$ is stable under $\kappa$-small colimits and retracts in $\Ind_{\kappa}(\calC')$. It follows that $L$ restricts to a functor $L': \calP^{\kappa}(\calC) \rightarrow \calE'$
which preserves $\kappa$-small colimits.

To complete the proof that $\calP^{\kappa}(\calC') \subseteq \calE$, it will suffice to prove that
$F' | \calP^{\kappa}(\calC)$ preserves $\kappa$-small colimits. 
To see this, we write
$F| \calP^{\kappa}(\calC')$ as a composition
$$ \calP^{\kappa}(\calC') \stackrel{L'}{\rightarrow} \calE' \stackrel{F|\calE'}{\rightarrow} \calC,$$
where $L'$ preserves $\kappa$-small colimits (as noted above) and $F|\calE' = f|\calC^{\kappa}$ preserves $\kappa$-small colimits by assumption. 
\end{proof}

\subsection{Representable Functors and the Adjoint Functor Theorem}\label{aftt}

An object $F$ of the $\infty$-category $\calP(\calC)$ of presheaves on $\calC$ is\index{gen}{representable!functor}\index{gen}{functor!representable}
{\it representable} if it lies in the essential image of the Yoneda embedding $j: \calC \rightarrow \calP(\calC)$. If $F: \calC^{op} \rightarrow \SSet$ is representable, then $F$ preserves limits:
this follows from the fact that $F$ is equivalent to the composite map
$$ \calC^{op} \stackrel{j}{\rightarrow} \calP( \calC^{op} ) \rightarrow \SSet $$
where $j$ denotes the Yoneda embedding for $\calC^{op}$ (which is limit-preserving by Proposition \ref{yonedaprop}) and the right map is given by evaluation at $C$ (which is 
limit-preserving by Proposition \ref{limiteval}). If $\calC$ is presentable, then the converse holds.

\begin{lemma}\label{stewgood}
Let $S$ be a small simplicial set, let $f: S \rightarrow \SSet$ be an object
of $\calP(S^{op})$, and let $F: \calP(S^{op}) \rightarrow \hat{\SSet}$ be the functor corepresented by $f$. Then the composition
$$ S \stackrel{j}{\rightarrow} \calP(S^{op}) \stackrel{F}{\rightarrow} \hat{\SSet}$$
is equivalent to $f$.
\end{lemma}

\begin{proof}
According to Corollary \ref{strictify}, we may can choose a (small) fibrant simplicial category $\calC$ and a categorical equivalence $\phi: S \rightarrow \sNerve(\calC^{op})$ such that $f$ is equivalent to the composition of $\psi^{op}$ with the nerve of a simplicial functor $f': \calC \rightarrow \Kan$.
Without loss of generality, we may suppose that $f' \in \Set_{\Delta}^{\calC}$ is projectively cofibrant.
Using Proposition \ref{gumby444}, we have an equivalence of $\infty$-categories
$$ \psi: \sNerve ( \Set_{\Delta}^{\calC} )^{\degree} ) \rightarrow \calP(S).$$
We observe that the composition $F \circ \psi$ can be identified with
the simplicial nerve of the functor $G: (\Set_{\Delta}^{\calC})^{\degree} \rightarrow \Kan$
corepresented by $f'$. The Yoneda embedding factors through $\psi$, via the adjoint of the composition
$$ j': \sCoNerve[S] \rightarrow \calC^{op} \rightarrow (\Set_{\Delta}^{\calC})^{\degree}.$$
It follows that $F \circ j$ can be identified with the adjoint of the composition
$$ \sCoNerve[S] \stackrel{j'}{\rightarrow} ( \Set_{\Delta}^{\calC} )^{\degree} \stackrel{G}{\rightarrow} \Kan.$$
This composition is equal to the functor $f'$, so its simplicial nerve coincides with the original functor $f$.
\end{proof}

\begin{proposition}\label{representable}
Let $\calC$ be a presentable $\infty$-category, and let 
$F: \calC^{op} \rightarrow \SSet$ be a functor. The following are equivalent:
\begin{itemize}
\item[$(1)$] The functor $F$ is representable by an object $C \in \calC$.
\item[$(2)$] The functor $F$ preserves small limits.
\end{itemize}
\end{proposition}

\begin{proof}
The implication $(1) \Rightarrow (2)$ was proven above (for an arbitrary $\infty$-category $\calC$).
For the converse, we first treat the case where $\calC = \calP(\calD)$, for some small $\infty$-category $\calD$. Let $f: \calD^{op} \rightarrow \SSet$ denote the composition
of $F$ with the (opposite) Yoneda embedding $j^{op}: \calD^{op} \rightarrow \calP(\calD)^{op}$, and let $F': \calP(\calD)^{op} \rightarrow \hat{\SSet}$ denote the functor represented by
$f \in \calP(\calD)$. We will prove that $F$ and $F'$ are equivalent. We observe that
$F$ and $F'$ both preserve small limits; consequently, according to Theorem \ref{charpresheaf}, it will suffice to show that the compositions $f = F \circ j^{op}$ and $f' = F' \circ j^{op}$ are equivalent.
This follows immediately from Lemma \ref{stewgood}.

We now consider the case where $\calC$ is an arbitrary presentable $\infty$-category.
According to Theorem \ref{pretop}, we may suppose that $\calC$ is an accessible localization of a presentable $\infty$-category $\calC'$ which has the form $\calP(\calD)$, so that the assertion for $\calC'$ has already been established. Let $L: \calC' \rightarrow \calC$ denote the localization functor. The functor $F \circ L^{op}: (\calC')^{op} \rightarrow \SSet$ preserves small limits, and is therefore representable by an object
$C \in \calC'$. Let $S$ denote the set of all morphisms $\phi$ in $\calC'$ such that
$L(\phi)$ is an equivalence in $\calC$. Without loss of generality, we may identify
$\calC$ with the full subcategory of $\calC'$ consisting of $S$-local objects. By construction,
$C \in \calC'$ is $S$-local and therefore belongs to $\calC$. It follows that
$C$ represents the functor $(F \circ L^{op})| \calC$, which is equivalent to $F$.
\end{proof}

The representability criterion of Proposition \ref{representable}
has many consequences, as now explain.

\begin{lemma}\label{limitscommute}
Let $X$ and $Y$ be simplicial sets, let $\calC$ be an $\infty$-category, and let
$p: X^{\triangleright} \times Y^{\triangleright} \rightarrow \calC$ be a diagram.
Suppose that:
\begin{itemize}
\item[$(1)$] For every vertex $x$ of $X^{\triangleright}$, the associated map
$p_{x}: Y^{\triangleright} \rightarrow \calC$ is a colimit diagram.
\item[$(2)$] For every vertex $y$ of $Y$, the associated map
$p_{y}: X^{\triangleright} \rightarrow \calC$ is a colimit diagram.
\end{itemize}
Let $\infty$ denote the cone point of $Y^{\triangleright}$. Then the restriction
$p_{\infty}: X^{\triangleright} \rightarrow \calC$ is a colimit diagram. 
\end{lemma}

\begin{proof}
Without loss of generality, we can suppose that $X$ and $Y$ are $\infty$-categories.
Since the inclusion $X \times \{\infty\} \subseteq X \times Y^{\triangleright}$ is cofinal, it will suffice to show that the restriction $p| (X \times Y^{\triangleright})^{\triangleright}$ is a colimit diagram.
According to Proposition \ref{stormus} $p|(X \times Y^{\triangleright})$ is a left
Kan extension of $p|(X \times Y)$. By transitivity, it suffices to show that 
$p| (X \times Y)^{\triangleright}$ is a colimit diagram. For this, it will suffice to prove the stronger assertion that $p| (X^{\triangleright} \times Y)^{\triangleright}$ is a left Kan extension of
$p| (X \times Y)$. Since Proposition \ref{stormus} also implies that
$p| (X^{\triangleright} \times Y)$ is a left Kan extension of $p|(X \times Y)$, we may
again apply transitivity and reduce to the problem of showing that
$p| (X^{\triangleright} \times Y)^{\triangleright}$ is a colimit diagram. Let $\infty'$
denote the cone point of $X^{\triangleright}$. Since
the inclusion $\{\infty'\} \times Y \subseteq X^{\triangleright} \times Y$ is cofinal, we
are reduced to proving that $p_{\infty'}: Y^{\triangleright} \rightarrow \calC$ is a colimit diagram, which follows from $(1)$.
\end{proof}

\begin{corollary}\label{preslim}\index{gen}{limit!in a presentable $\infty$-category}
A presentable $\infty$-category $\calC$ admits all (small) limits.
\end{corollary}

\begin{proof}
Let $\hat{\calP}(\calC) = \Fun(\calC^{op}, \hat{ \SSet })$, where $\hat{\SSet}$ denotes the
$\infty$-category of spaces which are not necessarily small, and let $j: \calC \rightarrow \hat{\calP}(\calC)$ be the Yoneda embedding. Since $j$ is fully faithful, it will suffice to show that the essential image of $j$ admits small limits. The $\infty$-category
$\hat{\calP}(\calC)$ admits all small limits (in fact, even limits which are not necessarily small); it therefore suffices to show that the essential image of $j$ is stable under small limits.
This follows immediately from Proposition \ref{representable} and Lemma \ref{limitscommute}.
\end{proof}

\begin{remark}
Let $A$ be a (small) partially ordered set. The $\infty$-category $\Nerve(A)$ is presentable if and only if every subset of $A$ has a least upper bound. Corollary \ref{preslim} can then be regarded as a generalization of the following classical observation: if every subset of $A$ has a least upper bound, then every subset of $A$ has a greatest lower bound (namely, a least upper bound for the collection of all lower bounds). 
\end{remark}

\begin{remark}\label{coten}
Now that we know that every presentable $\infty$-category $\calC$ has arbitrary limits, we can apply
an argument dual to that of Remark \ref{tensored} to show that
$\calC$ is {\it cotensored over $\SSet$}. In other words, for any
$C \in \calC$ and every simplicial set $X$, there exists an object $C^X
\in \calC$ (well defined up to equivalence) together with a collection of
natural isomorphisms
$$\bHom_{\calC}(C',C^X) \simeq
\bHom_{\calC}(C',C)^{X}$$
in the homotopy category $\calH$.
\end{remark}

We can now formulate a ``dual'' version of Proposition \ref{representable}, which requires a slightly stronger hypothesis.

\begin{proposition}\label{representableprime}\index{gen}{corepresentable!functor}\index{gen}{functor!corepresentable}
Let $\calC$ be a presentable $\infty$-category, and let 
$F: \calC \rightarrow \SSet$ be a functor. Then $F$ is {\em corepresentable} by an object
of $\calC$ if and only if $F$ is accessible and preserves small limits.
\end{proposition}

\begin{proof}
The ``only if'' direction is clear, since every object of $\calC$ is $\kappa$-compact for 
$\kappa \gg 0$. We will prove the converse.
Without loss of generality we may suppose that $\calC$ is minimal (this assumption is a technical convenience which will guarantee that various constructions below stay in the world of small $\infty$-categories).
Let $\widetilde{\calC} \rightarrow \calC$ denote the left fibration represented by $F$. 
Choose a regular cardinal $\kappa$ such that $\calC$ is $\kappa$-accessible and $F$ is
$\kappa$-continuous, and let $\widetilde{\calC}^{\kappa}$ denote the fiber product
$\widetilde{\calC} \times_{\calC} \calC^{\kappa}$, where $\calC^{\kappa} \subseteq \calC$ denotes the full subcategory spanned by the $\kappa$-compact objects of $\calC$. The $\infty$-category
$\widetilde{\calC}^{\kappa}$ is small (since $\calC$ is assumed minimal). Corollary \ref{preslim} implies that the diagram $p: \widetilde{\calC}^{\kappa} \rightarrow \calC$ admits a limit
$\overline{p}: (\widetilde{\calC}^{\kappa})^{\triangleleft} \rightarrow \calC$. Since the functor $F$
preserves small limits, Corollary \ref{charspacelimit} implies that there exists a map
$\overline{q}: (\widetilde{\calC}^{\kappa})^{\triangleleft} \rightarrow \widetilde{\calC}$ which extends
the inclusion $q: \widetilde{\calC}^{\kappa} \subseteq \widetilde{\calC}$ and covers $\overline{p}$.
Let $\widetilde{X}_0 \in \widetilde{\calC}$ denote the image of the cone point under $\overline{q}$ and $X_0$ its image in $\calC$. Then $\widetilde{X}_0$ determines a connected component of the space $F(X_0)$. Since $\calC$ is $\kappa$-accessible, we can write
$X_0$ as a $\kappa$-filtered colimit of $\kappa$-compact objects $\{ X_{\alpha} \}$ of $\calC$. Since $F$ is $\kappa$-continuous, there exists a $\kappa$-compact object $X \in \calC$ such that the induced map $F(X) \rightarrow F(X_0)$ has nontrivial image in the connected component classified by $\widetilde{X}_0$. It follows that there exists an object $\widetilde{X} \in \widetilde{\calC}$ lying
over $X_{\alpha}$, and a morphism $f: \widetilde{X} \rightarrow \widetilde{X}_0$ in
$\widetilde{\calC}$. Since $\widetilde{\calC}_{/q} \rightarrow \widetilde{\calC}$ is a right fibration,
we can pull $\overline{q}$ back to obtain a map
$\overline{q}': (\widetilde{\calC}^{\kappa})^{\triangleleft} \rightarrow \widetilde{\calC}$ 
which extends $q$ and carries the cone point to $\widetilde{X}$. It follows that
$\overline{q}'$ factors through $\widetilde{\calC}^{\kappa}$. We have a commutative diagram
$$ \xymatrix{ & \widetilde{\calC} \ar[dr] & \\
\{ \widetilde{X} \} \ar[ur] \ar[rr]^{i} & & \widetilde{\calC}^{\triangleleft} }$$
where $i$ denotes the inclusion of the cone point. The map $i$ is left anodyne, and therefore
a covariant equivalence in $(\sSet)_{/\calC}$. 
It follows that $\widetilde{\calC}^{\kappa}$ is a retract of $\{ X \}$ in
the homotopy category of the covariant model category $(\sSet)_{/\calC^{\kappa}}$. Proposition \ref{othermod} implies that $F | \calC^{\kappa}$ is a retract of the Yoneda image
$j(X)$ in $\calP(\calC^{\kappa})$. Since the $\infty$-category $\calC^{\kappa}$
is idempotent complete and the Yoneda embedding $j: \calC^{\kappa} \rightarrow
\calP(\calC^{\kappa})$ is fully faithful, we deduce that $F| \calC^{\kappa}$ is
equivalent to $j(X')$, where $X' \in \calC^{\kappa}$ is a retract of $X$. Let $F': \calC \rightarrow \SSet$ denote the functor co-represented by $X'$. We note that $F|\calC^{\kappa}$ and
$F'| \calC^{\kappa}$ are equivalent, and that both $F$ and $F'$ are $\kappa$-continuous.
Since $\calC$ is equivalent to $\Ind_{\kappa}(\calC^{\kappa})$, Proposition \ref{intprop} guarantees that $F$ and $F'$ are equivalent, so that $F$ is representable by $X'$.
\end{proof}

\begin{remark}
It is not difficult to adapt our proof of Proposition \ref{representableprime} to obtain an alternative proof of Proposition \ref{representable}.
\end{remark}

From Propositions \ref{representable} and \ref{representableprime} we can deduce a version of the
adjoint functor theorem:

\begin{corollary}[Adjoint Functor Theorem]\label{adjointfunctor}\index{gen}{adjoint functor theorem}
Let $F: \calC \rightarrow \calD$ be functor between presentable $\infty$-categories.
\begin{itemize}
\item[$(1)$] The functor $F$ has a right adjoint if and only if it preserves small colimits.
\item[$(2)$] The functor $F$ has a left adjoint if and only if it is accessible and preserves small limits.
\end{itemize}
\end{corollary}

\begin{proof}
The ``only if'' directions follow from Propositions \ref{adjointcol} and \ref{adjoints}. We now prove the converse direction of $(2)$; the proof of $(1)$ is similar but easier. Suppose that
$F$ is accessible and preserves small limits. Let $F': \calD \rightarrow \SSet$ be a corepresentable functor. Then $F'$ is accessible and preserves small limits, by Proposition \ref{representableprime}. It follows that the composition $F' \circ F: \calC \rightarrow \SSet$ is accessible and preserves small limits. Invoking Proposition \ref{representableprime} again, we deduce that $F' \circ F$ is representable. We now apply Proposition \ref{adjfuncbaby} to deduce that $F$ has a left adjoint.
\end{proof}

\begin{remark}\label{afi}
The proof of $(1)$ in Corollary \ref{adjointfunctor} does not require that $\calD$ is presentable, but only that $\calD$ is (essentially) locally small.
\end{remark}

\subsection{Limits and Colimits of Presentable $\infty$-Categories}\label{colpres}

In this section, we will introduce an $\infty$-category whose objects are presentable $\infty$-categories, and study its properties. In fact, we will introduce two such $\infty$-categories, which are (canonically) anti-equivalent to one another. The basic observation is the following:
given a pair of presentable $\infty$-categories $\calC$ and $\calD$, the proper notion of ``morphism'' between them is a pair of adjoint functors
$$ \Adjoint{F}{\calC}{\calD}{G} $$
Of course, either one of $F$ and $G$ determines the other up to canonical equivalence. We may therefore think of either one as encoding the data of a morphism.

\begin{definition}\index{not}{PresR@$\RPres$}\index{not}{PresL@$\LPres$}
Let $\widehat{\Cat}_{\infty}$ denote the $\infty$-category of (not necessarily small) $\infty$-categories.
We define subcategories $\RPres, \LPres \subseteq \widehat{\Cat}_{\infty}$ as follows:

\begin{itemize}
\item[$(1)$] The objects of both $\RPres$ and $\LPres$ are the presentable $\infty$-categories.
\item[$(2)$] A functor $F: \calC \rightarrow \calD$ between presentable $\infty$-categories is a morphism in $\LPres$ if and only if $F$ preserves small colimits.
\item[$(3)$] A functor $G: \calC \rightarrow \calD$ between presentable $\infty$-categories is a morphism in $\RPres$ if and only if $G$ is accessible and preserves small limits.
\end{itemize}\index{gen}{$\infty$-category!of presentable $\infty$-categories}
\end{definition}

As indicated above, the $\infty$-categories $\RPres$ and $\LPres$ are anti-equivalent to one another. To prove this, it is convenient to introduce the following definition:

\begin{definition}\label{urtus}\index{gen}{presentable!fibration}\index{gen}{fibration!presentable}
A map of simplicial sets $p: X \rightarrow S$ is a {\it presentable fibration}
if it is both a Cartesian fibration and a coCartesian fibration, and each fiber
$X_{s} = X \times_{S} \{s\}$ is a presentable $\infty$-category.
\end{definition}

The following result is simply a reformulation of Corollary \ref{adjointfunctor}:

\begin{proposition}\label{surtog}
\begin{itemize}
\item[$(1)$] Let $p: X \rightarrow S$ be a Cartesian fibration of simplicial sets, classified
by a map $\chi: S^{op} \rightarrow \widehat{\Cat}_{\infty}$. Then $p$ is a presentable fibration if and only if $\chi$ factors through $\RPres \subseteq \widehat{\Cat}_{\infty}$. 

\item[$(2)$] Let $p: X \rightarrow S$ be a coCartesian fibration of simplicial sets, classified by a map $\chi: S \rightarrow \widehat{\Cat}_{\infty}$. Then $p$ is a presentable fibration if and only if
$\chi$ factors through $\LPres \subseteq \widehat{\Cat}_{\infty}$.

\end{itemize}
\end{proposition}

\begin{corollary}\label{warhog}
For every simplicial set $S$, there is a canonical bijection
$$ [ S, \LPres ] \simeq [ S^{op}, \RPres ]$$
where $[S, \calC]$ denotes the collection of equivalence classes of objects of $\Fun(S,\calC)$.
In particular, there is a canonical isomorphism $\LPres \simeq ( \RPres )^{op}$ in the homotopy category of $\infty$-categories.
\end{corollary}

\begin{proof}
According to Proposition \ref{surtog}, both $[S, \LPres]$ and $[S^{op}, \RPres]$ can be identified with the collection of equivalence classes of presentable fibrations $X \rightarrow S$.
\end{proof}

We now commence our study of the $\infty$-category $\LPres$ (or, equivalently, the anti-equivalent $\infty$-category $\RPres$). The next few results express the idea that $\LPres \subseteq \widehat{\Cat}_{\infty}$ is stable under a variety of categorical constructions.

\begin{proposition}\label{complexhorse2}
Let $\{ \calC_{\alpha} \}_{ \alpha \in A}$ be a family of $\infty$-categories indexed by a small
set $A$, and let $\calC = \prod_{\alpha \in A} \calC_{\alpha}$ be their product. If each
$\calC_{\alpha}$ is presentable, then $\calC$ is presentable.
\end{proposition}

\begin{proof}
It follows from Lemma \ref{complexhorse} that $\calC$ is accessible. Let $p: K \rightarrow \calC$ be a diagram indexed by a small simplicial set $K$, corresponding to a family of diagrams
$\{ p_{\alpha}: K \rightarrow \calC_{\alpha} \}_{\alpha \in A}$. Since each $\calC_{\alpha}$ is presentable, each $p_{\alpha}$ has a colimit $\overline{p_{\alpha}}: K^{\triangleright} \rightarrow \calC_{\alpha}$. These colimits determine a map $\overline{p}: K^{\triangleright} \rightarrow \calC$ which is a colimit of $p$.
\end{proof}

\begin{proposition}\label{presexp}\index{gen}{presentable!functor categories}
Let $\calC$ be an presentable $\infty$-category, and let $K$ be a small simplicial set. Then
$\Fun(K,\calC)$ is presentable.
\end{proposition}

\begin{proof}
According to Proposition \ref{horse1}, $\Fun(K,\calC)$ is accessible. It follows from Proposition \ref{limiteval} that if $\calC$ admits small colimits, then $\Fun(K,\calC)$ admits small colimits.
\end{proof}

\begin{remark}
Let $S$ be a (small) simplicial set.
It follows from Example \ref{spacesareaccessible} and Corollary \ref{storum} that $\calP(S)$
is a presentable $\infty$-category. Moreover, Theorem \ref{charpresheaf} has a natural interpretation in the language of presentable $\infty$-categories: informally speaking, it asserts that the construction $$ S \mapsto \calP(S)$$ is left adjoint to the inclusion functor from presentable $\infty$-categories to all $\infty$-categories.
\end{remark}

The following is a variant on Proposition \ref{presexp}:

\begin{proposition}\label{intmap}
Let $\calC$ and $\calD$ be presentable $\infty$-categories. The $\infty$-category
$\LFun(\calC, \calD)$ is presentable.
\end{proposition}

\begin{proof}
Since $\calD$ admits small colimits, the $\infty$-category $\Fun(\calC, \calD)$ admits small colimits (Proposition \ref{limiteval}). Using Lemma \ref{limitscommute}, we conclude that 
$\LFun(\calC, \calD) \subseteq \Fun(\calC, \calD)$ is stable under small colimits. To complete the proof, it will suffice to show that $\LFun(\calC, \calD)$ is accessible. 

Choose a regular cardinal $\kappa$ such that $\calC$ is $\kappa$-accessible, and let
$\calC^{\kappa}$ be the full subcategory of $\calC$ spanned by the $\kappa$-compact objects. 
Propositions \ref{sumatch} and \ref{intprop} imply that the restriction functor $$ \LFun(\calC, \calD) \rightarrow \Fun(\calC^{\kappa}, \calD)$$
is fully faithful, and its essential image is the full subcategory 
$ \calE \subseteq \Fun(\calC^{\kappa}, \calD)$ spanned by those functors which preserve $\kappa$-small colimits. 

Since $\calC^{\kappa}$ is essentially small. the $\infty$-category $\Fun(\calC^{\kappa}, \calD)$
is accessible (Proposition \ref{horse1}). To complete the proof, we will show that $\calE$
is an accessible subcategory of $\Fun(\calC^{\kappa}, \calD)$. For each $\kappa$-small diagram $p: K \rightarrow \calC^{\kappa}$, let $\calE(p)$ denote the full subcategory of
$\Fun(\calC^{\kappa}, \calD)$ which preserve the colimit of $p$. Then $\calE = \bigcap_{p} \calE(p)$, where the intersection is taken over a set of representatives for all equivalence classes of $\kappa$-small diagrams in $\calC^{\kappa}$. According to Proposition \ref{boundint}, it will suffice to show that each $\calE(p)$ is an accessible subcategory of $\Fun(\calC^{\kappa}, \calD)$. 
We now observe that there is a (homotopy) pullback diagram of $\infty$-categories
$$ \xymatrix{ \calE(p) \ar@{^{(}->}[d] \ar[r] & \calE'(p) \ar@{^{(}->}[d] \\
\Fun( \calC^{\kappa}, \calD) \ar[r] & \Fun( K^{\triangleright}, \calD) }$$
where $\calE'$ denotes the full subcategory of $\Fun(K^{\triangleright}, \calD)$ spanned by the colimit diagrams. According to Proposition \ref{horse1}, it will suffice to prove that
$\calE'(p)$ is an accessible subcategory of $\Fun(K^{\triangleright}, \calD)$, which follows from Example \ref{colexam}.
\end{proof}

\begin{remark}
In the situation of Proposition \ref{intmap}, the presentable $\infty$-category
$\LFun(\calC, \calD)$ can be regarded as an {\em internal mapping object} in
$\LPres$. For every presentable $\infty$-category $\calC'$, a colimit-preserving functor
$\calC' \rightarrow \LFun(\calC, \calD)$ can be identified with a bifunctor
$\calC \times \calC' \rightarrow \calD$, which is colimit-preserving separately in each variable. 
There exists a universal recipient for such a bifunctor: a presentable category which we may denote by $\calC \otimes \calC'$. The operation $\otimes$ endows $\LPres$ with the structure of a {\it symmetric monoidal $\infty$-category}. Proposition \ref{intmap} can be interpreted as asserting that this monoidal structure is {\em closed}.
\end{remark}

\begin{proposition}\label{slicstab}\index{gen}{presentable!overcategories}\index{gen}{overcategory!of presentable $\infty$-categories}
Let $\calC$ be an $\infty$-category, and let $p: K \rightarrow \calC$ be a diagram
in $\calC$ indexed by a (small) simplicial set $K$. If $\calC$ is presentable, then the $\infty$-category $\calC_{/p}$ is also presentable.
\end{proposition}

\begin{proof}
According to Corollary \ref{horsemuun}, $\calC_{/p}$ is accessible. The existence of small colimits in $\calC_{/p}$ follows from Proposition \ref{needed17}.
\end{proof}

\begin{proposition}\label{stabslic}\index{gen}{presentable!undercategories}\index{gen}{undercategory!of a presentable $\infty$-categories}
Let $\calC$ be an $\infty$-category, and let $p: K \rightarrow \calC$ be a diagram
in $\calC$ indexed by a small simplicial set $K$. If $\calC$ is presentable, then the $\infty$-category $\calC_{p/}$ is also presentable.
\end{proposition}

\begin{proof}
It follows from Corollary \ref{horsemn} that $\calC_{p/}$ is accessible. It therefore suffices to prove that every diagram $q: K' \rightarrow \calC_{p/}$ has a colimit in $\calC$. We now observe that
$(\calC_{p/})_{q/} \simeq \calC_{q'/}$ where $q': K \star K' \rightarrow \calC$ is the map classified by $q$. Since $\calC$ admits small colimits, $\calC_{q'/}$ has an initial object.
\end{proof}

\begin{proposition}\label{horse22}
Let $$ \xymatrix{ \calX' \ar[r]^{q'} \ar[d]^{p'} & \calX \ar[d]^{p} \\
\calY' \ar[r]^{q} & \calY }$$
be a diagram of $\infty$-categories which is homotopy Cartesian $($with respect to the Joyal model structure$)$. Suppose further that $\calX$, $\calY$, and $\calY'$ are presentable, and that
$p$ and $q$ are presentable functors. Then $\calX'$ is presentable. Moreover, for any presentable $\infty$-category $\calC$ and any functor $f: \calC \rightarrow \calX$, $f$ is presentable if and only if the compositions $p' \circ f$ and $q' \circ f$ are presentable. In particular $($taking $f = \id_{\calX}${}$)$, $p'$ and $q'$ are presentable functors.
\end{proposition}

\begin{proof}
Proposition \ref{horse2} implies that $\calX'$ is accessible. It therefore suffices to prove that
any diagram $f: K \rightarrow \calX'$ indexed by a small simplicial set $K$ has a colimit in $\calX'$. 
Without loss of generality, we may suppose that $p$ and $q$ are categorical fibrations, and that
$\calX' = \calX \times_{ \calY} \calY'$. Let $X$ be an initial object of $\calX_{q' \circ f/}$ and let
$Y'$ be an initial object of $\calY'_{p' f/}$. Since $p$ and $q$ preserve colimits, 
the images $p(X)$ and $q(Y')$ are initial objects in $\calY_{p q' f/}$, and therefore equivalent to one another. Choose an equivalence $\eta: p(X) \rightarrow q(Y')$. Since $q$ is a categorical fibration, $\eta$ lifts to an equivalence $\overline{\eta}: Y \rightarrow Y'$ in $\calY'_{p' f/}$ such that $q(\overline{\eta}) = \eta$. Replacing $Y'$ by $Y$, we may suppose that
$p(X) = q(Y)$ so that the pair $(X,Y)$ may be considered as an object of
$ \calX'_{f/} = \calY'_{ p' f/} \times_{ \calY_{p q f/}} \calX_{q f/}$. 
According to Lemma \ref{bird1}, it is an initial object of $\calX'_{f/}$, so that $f$ has a colimit in $\calX'$. This completes the proof that $\calX'$ is accessible. The second assertion follows immediately from Lemma \ref{bird3}. 
\end{proof}

\begin{proposition}\label{limitpres1}\index{gen}{limit!of presentable $\infty$-categories}
The $\infty$-category $\LPres$ admits all small limits, and the inclusion functor
$\LPres \subseteq \widehat{\Cat}_{\infty}$ preserves all small limits.
\end{proposition}

\begin{proof}
The proof of Proposition \ref{alllimits} shows that it will suffice to consider the case of pullbacks and small products. The desired result now follows by combining Propositions \ref{horse22} and
\ref{complexhorse2}.
\end{proof}

\begin{corollary}\label{kevinet}
Let $p: X \rightarrow S$ be a presentable fibration of simplicial sets, where $S$ is small. Then the $\infty$-category $\calC$ of {\em coCartesian} sections of $p$ is presentable.
\end{corollary}

\begin{proof}
According to Proposition \ref{surtog}, $p$ is classified by a functor
$\chi: S \rightarrow \LPres$. Using Proposition \ref{limitpres1}, we deduce that
the limit of the composite diagram
$$ S \rightarrow \LPres \rightarrow \widehat{\Cat}_{\infty}$$
is presentable. Corollary \ref{blurt} allows us to identify this limit with the $\infty$-category $\calC$.
\end{proof}

Our goal, in the remainder of this section, is to prove the analogue of Proposition \ref{limitpres1} for the $\infty$-category $\RPres$ (which will show that $\LPres$ is equipped with all small {\em colimits} as well as all small limits). The main step is to prove that for every small diagram
$S \rightarrow \RPres$, the limit of the composite functor
$$ S \rightarrow \RPres \rightarrow \widehat{\Cat}_{\infty}$$
is presentable. As in the proof of Corollary \ref{kevinet}, this is equivalent to the assertion
that for any presentable fibration $p: X \rightarrow S$, the $\infty$-category $\calC$ of {\em Cartesian} sections of $p$ is presentable. To prove this, we will show that the $\infty$-category $\bHom_{S}(S,X)$ is presentable, and that $\calC$ is an accessible localization of $\bHom_{S}(S,X)$.

\begin{lemma}\label{doghold}
Let $p: \calM \rightarrow \Delta^1$ be a Cartesian fibration, let $\calC$
denote the $\infty$-category of sections of $p$, and let
$e: X \rightarrow Y$ and $e': X' \rightarrow Y'$ be objects of $\calC$. If
$e'$ is $p$-Cartesian, then the evaluation map
$\bHom_{\calC}( e,e') \rightarrow \bHom_{\calM}(Y,Y')$
is a homotopy equivalence.
\end{lemma}

\begin{proof}
There is a homotopy pullback diagram of simplicial sets, whose image in the homotopy
category $\calH$ is isomorphic to
$$ \xymatrix{ \bHom_{\calC}(e,e') \ar[r] \ar[d] & \bHom_{\calM}(Y,Y') \ar[d] \\
\bHom_{\calM}(X,X') \ar[r] & \bHom_{\calM}(X,Y'). }$$
If $e'$ is $p$-Cartesian, then the lower horizonal map is a homotopy equivalence, so the upper horizonal map is a homotopy equivalence as well.
\end{proof}

\begin{lemma}\label{cathold}
Let $p: \calM \rightarrow \Delta^1$ be a Cartesian fibration. Let
$\calC$ denote the $\infty$-category of sections of $p$, and $\calC' \subseteq \calC$ the full subcategory spanned by Cartesian sections of $p$. Then $\calC'$ is a reflective subcategory of $\calC$.
\end{lemma}

\begin{proof}
Let $e: X \rightarrow Y$ be an arbitrary section of $p$, and choose a Cartesian section
$e': X' \rightarrow Y$ with the same target. Since $e'$ is Cartesian, there exists a diagram
$$ \xymatrix{ X \ar[r] \ar[d] & Y \ar[d]^{\id_{Y}} \\
X' \ar[r] & Y }$$
in $\calM$, which we may regard as a morphism $\phi$ from $e \in \calC$ to $e' \in \calC'$. 
In view of Proposition \ref{testreflect}, it will suffice to show that $\phi$ exhibits
$e'$ as a $\calC'$-localization of $\calC$. In other words, we must show that for {\em any}
Cartesian section $e'': X'' \rightarrow Y''$, composition with $\phi$ induces a
homotopy equivalence
$\bHom_{\calC}( e', e'') \rightarrow \bHom_{\calC}(e,e'').$
This follows immediately from Lemma \ref{doghold}.
\end{proof}

\begin{proposition}\label{seccatdog}
Let $p: X \rightarrow S$ be a presentable fibration, where $S$ is a small simplicial set. 
Then: 
\begin{itemize}
\item[$(1)$] The $\infty$-category $\calC = \bHom_{S}(S,X)$ of sections of $p$ is presentable.
\item[$(2)$] The full subcategory $\calC' \subseteq \calC$ spanned by Cartesian sections of $p$ is an accessible localization of $\calC$.
\end{itemize}
\end{proposition}

\begin{proof}
The accessibility of $\calC$ follows from Corollary \ref{storkus1}. Since $p$ is a Cartesian fibration and the fibers of $p$ admit small colimits, $\calC$ admits small colimits by Proposition \ref{limiteval}. This proves $(1)$.

For each edge $e$ of $S$, let $\calC(e)$ denote the full subcategory of $\calC$ spanned by those maps $S \rightarrow X$ which carry $e$ to a $p$-Cartesian edge of $X$. By definition,
$\calC' = \bigcap \calC(e)$. According to Lemma \ref{stur3}, it will suffice to show that each
$\calC(e)$ is an accessible localization of $\calC$. Consider the map
$$ \theta_e: \calC \rightarrow \bHom_{S}( \Delta^1, X).$$
Proposition \ref{limiteval} implies that $\theta_{e}$ preserves all limits and colimits. 
Moreover, $\calC(e) = \theta_{e}^{-1} \bHom_{S}'(\Delta^1,X)$, where
$\bHom_{S}'(\Delta^1,X)$ denotes the full subcategory of $\bHom_{S}(\Delta^1,X)$ spanned by $p$-Cartesian edges. According to Lemma \ref{stur2}, it will suffice to show that
$\bHom_{S}'(\Delta^1,X) \subseteq \bHom_{S}(\Delta^1,X)$ is an accessible localization
of $\bHom_{S}(\Delta^1,X)$.
In other words, we may suppose $S = \Delta^1$. It then follows that evaluation at 
$\{1\}$ induces a trivial fibration $\calC' \rightarrow X \times_{S} \{1\}$, so that
$\calC'$ is presentable. It therefore suffices to show that $\calC'$ is a reflective subcategory of $\calC$, which follows from Lemma \ref{cathold}.
\end{proof}

\begin{theorem}\label{surbus}\index{gen}{colimit!of presentable $\infty$-categories}
The $\infty$-category $\RPres$ admits small limits, and the inclusion functor
$\RPres \subseteq \widehat{\Cat}_{\infty}$ preserves small limits.
\end{theorem}

\begin{proof}
Let $\chi: S^{op} \rightarrow \RPres$ be a small diagram, and let
$\overline{\chi}: (S^{\triangleright})^{op} \rightarrow \widehat{\Cat}_{\infty}$
be a limit of $\chi$ in $\Cat_{\infty}$. We will show that $\overline{\chi}$ factors
through $\RPres \subseteq \widehat{\Cat}_{\infty}$ and that $\overline{\chi}$ is a limit when regarded as a diagram in $\RPres$.

We first show that $\overline{\chi}$ carries each vertex to a presentable $\infty$-category.
This is clear with the exception of the cone point of $(S^{\triangleright})^{op}$. Let $p: X \rightarrow S$ be a presentable fibration classified by $\chi$. According to Corollary \ref{blurt}, we may identify the image of the cone point under $\overline{\chi}$ with the $\infty$-category $\calC$ of Cartesian sections of $p$. Proposition \ref{seccatdog} implies that this $\infty$-category is presentable.

We next show that $\overline{\chi}$ carries each edge of $(S^{\triangleright})^{op}$ to
an accessible, limit-preserving functor. This is clear for edges which are degenerate or belong
to $S^{op}$. The remaining edges are in bijection with the vertices of $s$, and connect those vertices to the cone point. The corresponding functors can be identified with the composition
$$ \calC \subseteq \bHom_{S}(S,X) \rightarrow X_{s},$$
where the second functor is given by evaluation at $s$. Proposition \ref{seccatdog} implies
that the inclusion $i: \calC \subseteq \bHom_{S}(S,X)$ is accessible and preserves small limits, and
Proposition \ref{limiteval} implies that the evaluation map $\bHom_{S}(S,X) \rightarrow X_{s}$
preserves all limits and colimits. This completes the proof that $\overline{\chi}$ factors through $\RPres$.

We now show that $\overline{\chi}$ is a limit diagram in $\RPres$. Since $\RPres$ is a subcategory of $\widehat{\Cat}_{\infty}$ and $\overline{\chi}$ is already a limit diagram in $\widehat{\Cat}_{\infty}$, it will suffice to verify the following assertion:

\begin{itemize}
\item If $\calD$ is a presentable $\infty$-category, and $F: \calD \rightarrow \calC$ has the property that each of the composite functors
$$ \calD \stackrel{F}{\rightarrow} \calC \stackrel{i}{\subseteq} \bHom_{S}(S,X) \rightarrow X_{s}$$
is accessible and limit-preserving, then $F$ is accessible and limit-preserving.
\end{itemize}

Applying Proposition \ref{seccatdog}, we see that $F$ is accessible and limit preserving if and only if $i \circ F$ is accessible and limit preserving. We now conclude by applying Proposition \ref{limiteval}.
\end{proof}

\subsection{Local Objects}\label{invloc}

According to Theorem \ref{pretop}, every presentable $\infty$-category arises as an (accessible) localization of some presheaf $\infty$-category $\calP(X)$. Consequently, understanding the process of localization is of paramount importance in the study of presentable $\infty$-categories. In this section, we will classify the accessible localizations of an arbitrary presentable $\infty$-category $\calC$. The basic observation is that a localization functor $L: \calC \rightarrow \calC$ is determined, up to equivalence, by the collection $S$ of all morphisms $f$ such that $Lf$ is an equivalence. Moreover, a collection of morphisms $S$ arises from an accessible localization functor if and only if $S$ is {\em strongly saturated} (Definition \ref{saturated2}) and {\em of small generation} (Remark \ref{sat2}). Given any small collection of morphisms $S$ in $\calC$, 
there is a smallest strongly saturated collection containing $S$: this permits us to define a localization
$S^{-1} \calC \subseteq \calC$. The ideas presented in this section go back (at least) to Bousfield; we refer the reader to \cite{bousfield} for a discussion in a more classical setting.

\begin{definition}\index{gen}{$S$-local}\index{gen}{local!object}
Let $\calC$ be an $\infty$-category and $S$ a collection of morphisms of $\calC$. We say that
an object $Z$ of $\calC$ is {\it $S$-local} if, for every morphism $s: X \rightarrow Y$ belonging to $S$, composition with $s$ induces an isomorphism
$\bHom_{\calC}(Y,Z) \rightarrow \bHom_{\calC}(X,Z)$
in the homotopy category $\calH$ of spaces.

A morphism $f: X \rightarrow Y$ of $\calC$ is an {\it $S$-equivalence} if, for every $S$-local
object $Z$, composition with $f$ induces a homotopy equivalence
$\bHom_{\calC}(Y,Z) \rightarrow \bHom_{\calC}(X,Z)$
is an isomorphism in $\calH$.\index{gen}{$S$-equivalence}
\end{definition}

The following result provides a dictionary for relating localization functors to classes of morphisms:

\begin{proposition}\label{localloc}
Let $\calC$ be an $\infty$-category, and let $L: \calC \rightarrow \calC$ be a localization functor.
Let $S$ denote the collection of all morphisms $f$ in $\calC$ such that $Lf$ is an equivalence.
Then:
\begin{itemize}
\item[$(1)$] An object $C$ of $\calC$ is $S$-local if and only if it belongs to $L\calC$.
\item[$(2)$] Every $S$-equivalence in $\calC$ belongs to $S$.
\item[$(3)$] Suppose that $\calC$ is accessible. The following conditions are equivalent:
\begin{itemize}
\item[$(i)$] The $\infty$-category $L \calC$ is accessible.
\item[$(ii)$] The functor $L: \calC \rightarrow \calC$ is accessible.
\item[$(iii)$] There exists a $($small$)$ subset $S_0 \subseteq S$ such that every $S_0$-local object is $S$-local. 

\end{itemize}

\end{itemize}

\end{proposition}

\begin{proof}[Proof of $(1)$ and $(2)$]
By assumption, $L$ is left adjoint to the inclusion $L \calC \subseteq \calC$; let
$\alpha: \id_{\calC} \rightarrow L$ be a unit map for the adjunction. We begin by proving $(1)$.
Suppose that $X \in L \calC$. Let $f: Y \rightarrow Z$ belong to $S$. Then we have a commutative diagram
$$ \xymatrix{ \bHom_{\calC}(LZ,X) \ar[r] \ar[d] & \bHom_{\calC}(LY, X) \ar[d] \\
\bHom_{\calC}(Z,X) \ar[r] & \bHom_{\calC}(Y,X) }$$
in the homotopy category $\calH$, where the vertical maps are given by composition with $\alpha$ and are homotopy equivalences by assumption. Since $Lf$ is an equivalence, the top horizontal map is also a homotopy equivalence. It follows that the bottom horizontal map is a homotopy equivalence, so that $X$ is $S$-local. Conversely, suppose that $X$ is $S$-local. According to Proposition \ref{recloc}, the map $\alpha(X): X \rightarrow LX$ belongs to $S$, so that composition with $\alpha(X)$ induces a homotopy equivalence $\bHom_{\calC}(LX,X) \rightarrow \bHom_{\calC}(X,X)$. In particular, there exists a map $LX \rightarrow X$ whose composition
with $\alpha(X)$ is homotopic to $\id_{X}$. Thus $X$ is a retract of $LX$. Since $\alpha(LX)$
is an equivalence, we conclude that $\alpha(X)$ is an equivalence, so that $X \simeq LX$ and
therefore $X$ belongs to the essential image of $L$, as desired. This proves $(1)$.

Suppose that $f: X \rightarrow Y$ is an $S$-equivalence. 
We have a commutative diagram $$ \xymatrix{ X \ar[r]^{f} \ar[d]^{\alpha(X)} & Y \ar[d]^{\alpha(Y)} \\
LX \ar[r]^{Lf} & LY }$$
where the vertical maps are $S$-equivalences (by Proposition \ref{recloc}), so that $Lf$
is also an $S$-equivalence. Therefore $LX$ and $LY$ corepresent the same functor
on the homotopy category $\h{L \calC}$. Yoneda's lemma implies that $Lf$ is an equivalence, so that $f \in S$. This proves $(2)$.
\end{proof}

The proof of $(3)$ is somewhat more involved, and will require a few preliminaries.

\begin{lemma}\label{neelda}
Let $\tau \gg \kappa$ be regular cardinals, and suppose that $\tau$ is uncountable. Let $A$ be a $\kappa$-filtered partially ordered set, $A' \subseteq A$ a $\tau$-small subset, and
$$ \{ f_{\gamma}: X_{\gamma} \rightarrow Y_{\gamma} \}_{ \gamma \in C}$$ a $\tau$-small collection of natural transformations of diagrams in $\Kan^{A}$. Suppose that for each
$\alpha \in A$, $\gamma \in C$, the Kan complexes $X_{\gamma}(\alpha)$ and
$Y_{\gamma}(\alpha)$ are essentially $\tau$-small.
Suppose further that, for each $\gamma \in C$, the map of Kan complexes $\colim_{A} f_{\gamma}$
is a homotopy equivalence. Then there exists a $\tau$-small, $\kappa$-filtered subset
$A'' \subseteq A$ such that $A' \subseteq A''$, and $\colim_{A''} f_{\gamma}|A''$ 
is a homotopy equivalence for each $\gamma \in C$.
\end{lemma}

\begin{proof}
Replacing each $f_{\gamma}$ by an equivalent transformation if necessary, we may suppose for each $\gamma \in C$, $\alpha \in A$, the map $f_{\gamma}(\alpha)$ is a Kan fibration.

Let $\alpha \in A$, $\gamma \in C$, and let $\sigma(\alpha,\gamma)$ be a diagram
$$ \xymatrix{ \bd \Delta^n \ar[r] \ar@{^{(}->}[d] & X_{\gamma}(\alpha) \ar[d] \\
\Delta^n \ar[r] & Y_{\gamma}(\alpha). }$$
We will say that $\alpha' \geq \alpha$ {\it trivializes} $\sigma(\alpha,\gamma)$ if the lifting problem
depicted in the induced diagram
$$ \xymatrix{ \bd \Delta^n \ar[r] \ar@{^{(}->}[d] & X_{\gamma}(\alpha') \ar[d] \\
\Delta^n \ar[r] \ar@{-->}[ur] & Y_{\gamma'}(\alpha). }$$
admits a solution. Observe that, if $B \subseteq A$ is filtered, then
$\colim_{B} f_{\gamma}|B$ is a Kan fibration, which is trivial if and only if
for every diagram $\sigma(\alpha,\gamma)$ as above, where $\alpha \in B$, there
exists $\alpha' \in B$ such that $\alpha' \geq \alpha$, and $\alpha'$ trivializes
$\sigma(\alpha, \gamma)$. In particular, since $\colim_{A} f_{\gamma}$ is a homotopy equivalence,
every such diagram $\sigma(\alpha,\gamma)$ is trivialized by some $\alpha' \geq \alpha$. 

We now define a sequence of $\tau$-small subsets $A(\lambda) \subseteq A$, indexed by
ordinals $\lambda \leq \kappa$. Let $A(0) = A'$, and let $A(\lambda) = \bigcup_{\lambda' < \lambda} A(\lambda')$ when $\lambda$ is a limit ordinal. Supposing that $\lambda < \kappa$ and that $A(\lambda)$ has been defined, we choose a set of representatives
$\Sigma = \{ \sigma(\alpha, \gamma) \}$ for all homotopy classes of diagrams as above, where
$\alpha \in A(\lambda)$ and $\gamma \in C$. Since the Kan complexes $X_{\gamma}(\alpha)$,
$Y_{\gamma}(\alpha)$ are essentially $\tau$-small, we may choose the set $\Sigma$ to be $\tau$-small. Each $\sigma \in \Sigma$ is trivialized by some $\alpha'_{\sigma} \in A$; let $B = \{ \alpha'_{\sigma} \}_{\sigma \in \Sigma}$. Then $B$ is $\tau$-small. Now choose a $\tau$-small, $\kappa$-filtered subset $A(\lambda+1) \subseteq A$ containing $A(\lambda) \cup B$ (the existence of $A(\lambda+1)$ is guaranteed by Lemma \ref{estimate}).

We now define $A''$ to be $A(\kappa)$; it is easy to see that $A''$ has the desired properties.
\end{proof}

\begin{lemma}\label{neeld}
Let $\tau \gg \kappa$ be regular cardinals, and suppose that $\tau$ is uncountable. Let $A$ be a $\kappa$-filtered partially ordered set, and for every subset $B \subseteq A$, let
$$ \colim_{B}: \Fun(\Nerve(B), \SSet) \rightarrow \SSet$$
denote a left adjoint to the diagonal functor. Let $A' \subseteq A$ be a $\tau$-small subset
and $\{ f_{\gamma}: X_{\gamma} \rightarrow Y_{\gamma} \}_{\gamma \in C}$ a $\tau$-small collection of morphisms in the $\infty$-category $\Fun(\Nerve(A), \SSet^{\tau})$ of
diagrams $\Nerve(A) \rightarrow \SSet^{\tau}$. Suppose that $\colim_A(f_{\gamma})$ is an
equivalence, for each $\gamma \in C$. Then there exists a $\tau$-small, $\kappa$-filtered subset
$A'' \subseteq A$ which contains $A'$, such that each of the morphisms
$\colim_{A''}(f_{\gamma} | \Nerve(A'' ))$ is an equivalence in $\SSet$.
\end{lemma}

\begin{proof}
Using Proposition \ref{gumby444}, we may assume without loss of generality that
each $f_{\gamma}$ is the simplicial nerve of a natural transformation of functors
from $A$ to $\Kan$. According to Theorem \ref{colimcomparee}, we can identify
$\colim_{B}( X_{\gamma} | \Nerve(B))$ and
$\colim_{B}( Y_{\gamma} | \Nerve(B))$ with homotopy colimits in $\Kan$. If $B$ is filtered, then these homotopy colimits reduce to ordinary colimits (since the class of weak homotopy equivalences in $\Kan$ is stable under filtered colimits), and we may apply Lemma \ref{neelda}.
\end{proof}

\begin{proof}[Proof of part $(3)$ of Proposition \ref{localloc}]
If $L \calC$ is accessible, then Proposition \ref{adjoints} implies that both the inclusion $L \calC \rightarrow \calC$ and $L: \calC \rightarrow L \calC$ are accessible functors, so that their composition is accessible. Thus $(i) \Rightarrow (ii)$. Suppose next that $(ii)$ is satisfied.
Let $\alpha: \id_{\calC} \rightarrow L$ denote a unit for the adjunction between $L$ and
the inclusion $L \calC \subseteq \calC$, and let $\kappa$ be a regular cardinal such that
$\calC$ is $\kappa$-accessible and $L$ is $\kappa$-continuous. Without loss of generality,
we may suppose that $\calC$ is minimal, so that $\calC^{\kappa}$ is a small $\infty$-category.
Let $S_0 = \{ \alpha(X): X \in \calC^{\kappa} \}$, and let
$Y \in \calC$ be $S_0$-local. We wish to prove that $Y$ is $S$-local. Let 
$F_Y: \calC \rightarrow \SSet^{op}$ denote the functor represented by $Y$.
Then $\alpha$ induces a natural transformation $F_Y \rightarrow F_Y \circ L$. The functors $F_Y$ and $F_Y \circ L$ are both $\kappa$-continuous, and by assumption $\alpha$ induces
an equivalence of functors $F_Y | \calC^{\kappa} \rightarrow (F_Y \circ L)| \calC^{\kappa}$ when both sides are restricted to $\kappa$-compact objects. Proposition \ref{intprop} now implies that
$F_Y$ and $F_Y \circ L$ are equivalent, so that $Y$ is $S$-local. This proves $(iii)$.

We complete the proof by showing that $(iii)$ implies $(i)$. Let $\kappa$ be a regular cardinal such that $\calC$ is $\kappa$-accessible and $S_0$ is a set of morphisms between $\kappa$-compact objects of $\calC$. We claim that $L \calC$ is stable under $\kappa$-filtered colimits
in $\calC$. To prove this, let $\overline{p}: K^{\triangleright} \rightarrow \calC$ be a colimit diagram, where $K$ is small and $\kappa$-filtered, and $p = \overline{p}|K$ factors through
$L\calC \subseteq \calC$. Let $s: X \rightarrow Y$ be a morphism which belongs to $S_0$, and let
$s': F_X \rightarrow F_Y$ be the corresponding map of co-representable functors
$\calC \rightarrow \hat{\SSet}$. Since $X$ and $Y$ are $\kappa$-compact by assumption, both
$\overline{p}_{X}: F_X \circ \overline{p}$ and $\overline{p}_Y: F_Y \circ \overline{p}$ are colimit diagrams in $\hat{\SSet}$. The map $s'$ induces a transformation
$\overline{p}_{X} \rightarrow \overline{p}_{Y}$, which is an equivalence when restricted to
$K$, and is therefore an equivalence in general. It follows that
$\bHom_{\calC}(Y, \overline{p}(\infty)) \simeq \bHom_{\calC}(X, \overline{p}(\infty))$, where
$\infty$ denotes the cone point of $K^{\triangleright}$. Thus $\overline{p}(\infty)$
is $S_0$-local as desired.

Now choose an uncountable regular cardinal $\tau \gg \kappa$ such that $\calC^{\kappa}$ is essentially $\tau$-small. According to Proposition \ref{clear}, to complete the proof that $\calC$ is accessible it will suffice to show that $L \calC$ is generated by $\tau$-compact objects under $\tau$-filtered colimits. Let $X$ be an object of $L \calC$. Lemma \ref{longwait0} implies
that $X$ can be written as the colimit of a small diagram $p: \calI \rightarrow \calC^{\kappa}$,
where $\calI$ is $\kappa$-filtered. Using Proposition \ref{rot}, we may suppose that
$\calI$ is the nerve of a $\kappa$-filtered partially ordered set $A$. Let $B$ denote the
collection of all $\kappa$-filtered, $\tau$-small subsets $A_{\beta} \subseteq A$ for which
the colimit of $p| \Nerve(A_{\beta})$ is $S_0$-local. Lemma \ref{neeld} asserts that every $\tau$-small subset of $A$ is contained in $A_{\beta}$, for some $\beta \in B$. It follows that
$B$ is $\tau$-filtered, when regarded as partially ordered by inclusion, and that
$A = \bigcup_{\beta \in B} A_{\beta}$. Using Proposition \ref{utl} and Corollary \ref{util}, we can
obtain $X$ as the colimit of a diagram $q: \Nerve(B) \rightarrow \calC$, where each
$q(\beta)$ is a colimit $X_{\beta}$ of $p | \Nerve(A_{\beta})$. The objects $\{ X_{\beta} \}_{\beta \in B}$ are $S_0$-local and $\tau$-compact by construction.
\end{proof}

According to Proposition \ref{localloc}, every localization $L$ of an $\infty$-category $\calC$ is determined by the class $S$ of morphisms $f$ such that $Lf$ is an equivalence. This raises the question: which classes of morphisms $S$ arise in this way? To answer this question, we will begin by isolating some of the most obvious properties enjoyed by $S$.

\begin{definition}\label{saturated2}\index{gen}{strongly saturated}\index{gen}{saturated!strongly}
Let $\calC$ be a $\infty$-category which admits small colimits, and let $S$ be a collection of morphisms of $\calC$. We will say that $S$ is {\it strongly saturated} if it satisfies the following conditions:

\begin{itemize}
\item[$(1)$] Given a pushout diagram 
$$ \xymatrix{ C \ar[r]^{f} \ar[d] & D \ar[d] \\
C' \ar[r]^{f'} \ar[r] & D' }$$
in $\calC$, if $f$ belongs to $S$, then so does $f'$.

\item[$(2)$] The full subcategory of $\Fun(\Delta^1, \calC)$ spanned by $S$ is stable under small  colimits.

\item[$(3)$] Suppose given a $2$-simplex of $\calC$, corresponding to a diagram
$$ \xymatrix{ X \ar[rr]^{f} \ar[dr]^{g} & & Y \ar[dl]_{h} \\
& Z. & }$$
If any two of $f$, $g$, and $h$ belong to $S$, then so does the third.
\end{itemize}
\end{definition}

\begin{remark}\label{simpcons}
Let $\calC$ be an $\infty$-category which admits small colimits, and let
$S$ be a strongly saturated class of morphisms of $\calC$. Let $\emptyset$ be
an initial object of $\calC$. Condition $(2)$ of Definition
\ref{saturated2} implies that $\id_{\emptyset}: \emptyset \rightarrow \emptyset$ belongs to $S$,
since it is an initial object of $\Fun(\Delta^1,\calC)$. Any equivalence in $\calC$ is a pushout of
$\id_{\emptyset}$, so condition $(1)$ implies that $S$ contains all equivalences in $\calC$.
It also follows from condition $(1)$ that if $f: C \rightarrow D$ belongs to $S$ and
$f': C \rightarrow D$ is homotopic to $f$, then $f'$ belongs to $S$ (since $f'$ is a pushout of $f$).
Note also that condition $(2)$ implies that $S$ is stable under retracts, since any retract
of a morphism $f$ can be written as a colimit of copies of $f$ (Proposition \ref{autokan}).
\end{remark}

\begin{remark}\label{sat2}
Let $\calC$ be an $\infty$-category which admits colimits. Given any collection $\{ S_{\alpha} \}_{\alpha \in A}$ of strongly saturated classes of morphisms of $\calC$, the intersection
$S = \bigcap_{\alpha \in A} S_{\alpha}$ is also strongly saturated. It follows that {\em any} collection $S_0$ of morphisms in $\calC$ is contained in a smallest strongly saturated class of morphisms $S$. In this case we will also write $S = \overline{S_0}$; we refer to it as the strongly saturated class of morphisms {\em generated} by $S_0$. We will say that $S$ is {\it of small generation} if $S = \overline{S_0}$, where $S_0 \subseteq S$ is small.\index{gen}{small generation}
\end{remark}

\begin{remark}
Let $\calC$ be an $\infty$-category which admits small colimits. Let $S$ be a strongly saturated
class of morphisms of $\calC$.
If $f: X \rightarrow Y$ lies in $S$ and $K$ is a simplicial set, then
the induced map $X \otimes K \rightarrow Y \otimes K$ (which is well-defined up to equivalence) lies in $S$. This follows from the closure of $S$ under colimits.
We will use this observation in the proof of Proposition \ref{local}, in the case where $K = \bd \Delta^n$ is a (simplicial) sphere.
\end{remark}

\begin{example}
Let $\calC$ be an $\infty$-category which admits small colimits, and let $S$ denote the class of all equivalences in $\calC$. Then $S$ is strongly saturated; it is clearly the smallest strongly saturated class of morphisms of $\calC$.
\end{example}

\begin{remark}\label{prebluse}
Let $F: \calC' \rightarrow \calC$ be a functor between $\infty$-categories. Suppose that $\calC$ and $\calC'$ admit small colimits and that $F$ preserves small colimits. Let $S$ be a strongly saturated class of morphisms in $\calC'$. Then $F^{-1} S$ is a strongly saturated class of morphisms of $\calC$. In particular, if we let $S$ denote the collection of all morphisms $f$ of $\calC'$ such that $F(f)$ is an equivalence, then $S$ is strongly saturated.
\end{remark}

\begin{lemma}\label{bluh}
Let $\calC$ be an $\infty$-category which admits small colimits, let $S_0$ be a class of morphisms
in $\calC$, and let $S$ denote the collection of all $S$-equivalences. Then $S$ is strongly saturated.
\end{lemma}

\begin{proof}
For each object $X \in \calC$, let $F_X: \calC \rightarrow \SSet^{op}$ denote the functor
represented by $X$, and let $S(X)$ denote the collection of all morphisms $f$ such
that $F_X(f)$ is an equivalence. Since $F_X$ preserves small colimits, Remark \ref{prebluse} implies that $S(X)$ is strongly saturated. We now observe that $S$ is the intersection $\bigcap S(X)$, where $X$ ranges over the class of all $S_0$-local objects of $\calC$.
\end{proof}

\begin{lemma}\label{yorkan}
Let $\calC$ be an $\infty$-category which admits small colimits, let $S$ be a strongly saturated collection of morphisms of $\calC$, and let $C \in \calC$ be an object. Let $\calD \subseteq \calC^{C/}$ be the full subcategory of $\calC^{C/}$ spanned by those objects $C \rightarrow C'$ which belong to $S$. Then $\calD$ is stable under small colimits in $\calC^{C/}$.
\end{lemma}

\begin{proof}
The proofs of Corollary \ref{uterrr} and \ref{allfin} show that it will suffice to prove that $\calD$ is stable under filtered colimits, pushouts, and contains the initial objects of $\calC^{C/}$. The last condition is equivalent to the requirement that $S$ contains all equivalences, which follows from Remark \ref{simpcons}. Now suppose that $\overline{p}: K^{\triangleright} \rightarrow \calC^{C/}$
is a colimit of $p = \overline{p} | K$, where $K$ is either filtered or equivalent to $\Lambda^2_0$,
and that $p(K) \subseteq \calD$. We can identify $\overline{p}$ with a map
$P: K^{\triangleright} \times \Delta^1 \rightarrow \calC$ such that
$P | K^{\triangleright} \times \{0\}$ is the constant map taking the value $C \in \calC$.
Since $K$ is weakly contractible, $P | K^{\triangleright} \times \{0\}$ is a colimit diagram in $\calC$.
The map $P | K^{\triangleright} \times \{1\}$ is the image of a colimit diagram under the left
fibration $\calC^{C/} \rightarrow \calC$; since $K$ is weakly contractible, Proposition \ref{goeselse} implies that $P | K^{\triangleright} \times \{1\}$ is a colimit diagram. We now apply
Proposition \ref{limiteval} to deduce that
$P: K^{\triangleright} \rightarrow \calC^{\Delta^1}$ is a colimit diagram. Since $S$ is stable under colimits in $\calC^{\Delta^1}$, we conclude that $P$
carries the cone point of $K^{\triangleright}$ to a morphism belonging to $S$, as we wished to show.
\end{proof}

\begin{lemma}\label{walter}
Let $\calC$ be an $\infty$-category which admits small filtered colimits, let $\kappa$ be an uncountable regular cardinal, let $A$ and $B$ be $\kappa$-filtered partially ordered sets, and
let $p_0: \Nerve(A_0) \rightarrow \calC^{\kappa}$ and $p_1: A_1 \rightarrow \calC^{\kappa}$ be two diagrams which have the same colimit. Let $A''_0 \subseteq A_0$, $A''_1 \subseteq A_1$ be $\kappa$-small subsets. Then
there exist $\kappa$-small, filtered subsets $A'_0 \subseteq A_0$, $A'_1 \subseteq A_1$ such that
$A''_0 \subseteq A'_0$, $A''_1 \subseteq A'_1$, and the diagrams $p_0|\Nerve(A'_0)$, $p_1|\Nerve (A'_1)$ have the same colimit in $\calC$.
\end{lemma}

\begin{proof}
Let $\overline{p}_0$, $\overline{p}_1$ be colimits of $p_0$ and $p_1$, respectively, which carry the cone points to the same object $C \in \calC$. Let
$B = A''_0 \cup A''_1 \cup \Z_{\geq 0} \cup \{ \infty\}$, which we regard as a partially ordered set so that $$\Nerve(B) \simeq (( \Nerve(A''_0) \coprod \Nerve (A''_1)) \star \Nerve(\Z_{\geq 0}))^{\triangleright}.$$
We will construct sequences of elements
$$ \{ a_0 \leq a_2 \leq \ldots \} \subseteq \{ a \in A_0: (\forall a'' \in A''_0) [a'' \leq a] \}$$
$$ \{ a_1 \leq a_3 \leq \ldots \} \subseteq \{ a \in A_1: (\forall a'' \in A''_1) [a'' \leq a] \}$$
and a diagram $\overline{q}: \Nerve(B) \rightarrow \calC$ such that
$$ \overline{q} | (\Nerve(A''_0) \cup \Nerve \{ 0,2,4, \ldots \})^{\triangleright} = \overline{p}_0 | \Nerve ( A''_0 \cup \{ a_0, a_2, \ldots \})^{\triangleright}$$
$$ \overline{q} | (\Nerve(A''_1) \cup \Nerve \{ 1,3,5, \ldots \})^{\triangleright} = \overline{p}_1 | \Nerve ( A''_1 \cup \{ a_1, a_3, \ldots \})^{\triangleright}.$$
Supposing that this has been done, we take $A'_0 = A''_0 \cup \{ a_0, a_2, \ldots \}$,
$A'_1 = A''_1 \cup \{a_1, a_3, \ldots \}$, and observe that the colimits of $p_0 | \Nerve(A'_0)$ 
and $p_1 | \Nerve(A'_1)$ are both equivalent to the colimit of $\overline{q} | \Nerve(\Z_{\geq 0})$.

The construction is by recursion; let us suppose that the sequence 
$a_0, a_1, \ldots, a_{n-1}$ and the map $\overline{q}_{n} = \overline{q}| ((\Nerve(A''_0) \coprod \Nerve(A''_1)) \star (\Nerve\{0, \ldots, n-1\})^{\triangleright}$ have already been constructed (when $n=0$, we observe that $\overline{q}_{0}$ is uniquely determined by $\overline{p}_0$ and $\overline{p}_1$). For simplicity we will treat only the case where $n$ is even; the case where $n$ is odd can be handled by a similar argument.

Let $q_n = \overline{q}_n | ( \Nerve(A''_0) \coprod \Nerve(A''_1)) \star \Nerve \{0, \ldots, n-1\}$ and
$q'_n = \overline{q}_n | \Nerve(A''_0) \star \Nerve \{0, 2, \ldots, n-2\}$.
According to Corollary \ref{jurman}, the left fibrations $\calC_{q_n/} \rightarrow \calC$
and $\calC_{q'_n/} \rightarrow \calC$ are $\kappa$-compact. Let 
$A_0(n) = \{ a \in A_0: (\forall a'' \in A''_0 \cup \{ a_0, \ldots, a_{n-2} \}) [a'' \leq a] \}$, and let
$X = \calC_{q_n/} \times_{\calC} \Nerve(A_0(n))^{\triangleright}$, $X' = \calC_{q'_n/} \times_{\calC} \Nerve(A_0(n))^{\triangleright}$, so that $X$ and $X'$ are left fibrations classified by colimit diagrams $\Nerve(A_0(n))^{\triangleright} \rightarrow \SSet$. Form a pullback diagram
$$ \xymatrix{ Y \ar[r] \ar[d] & X \ar[d] \\
\Nerve(A_0(n))^{\triangleright} \ar[r] & X' }$$
where the left vertical map is a left fibration (by Proposition \ref{sharpen}) and the bottom horizontal map is determined by $\overline{p} |  \Nerve(A''_0 \cup \{0,\ldots, n-2\}) \star \Nerve (A_0(n))^{\triangleright}$. It follows that the diagram is a homotopy pullback, so that 
$Y \rightarrow \Nerve(A_0(n))^{\triangleright}$ is also a left fibration classified by
a colimit diagram $\Nerve(A_0(n))^{\triangleright} \rightarrow \SSet$. The map
$\overline{q}_{n}$ determines a vertex $v$ of $Y$ lying over the cone point of $\Nerve (A_0(n))^{\triangleright}$. According to Proposition \ref{charspacecolimit}, the inclusion $Y \times_{ \Nerve(A_0(n))^{\triangleright} } \Nerve(A_0(n)) \subseteq Y$ is a weak homotopy equivalence of simplicial sets. It follows that there exists an edge $e: v' \rightarrow v$ of $Y$ which joins
$v$ to some vertex $v'$ lying over an element $a \in A_0(n)$. We now define $a_n = a$, and
observe that the edge $e$ corresponds to the desired extension $\overline{q}_{n+1}$ of
$\overline{q}_n$.
\end{proof}

\begin{lemma}\label{perry}
Let $\calC$ be a presentable $\infty$-category, let $S$ be a strongly saturated collection of morphisms in $\calC$, and let $\calD \subseteq \Fun(\Delta^1,\calC)$ be the full subcategory spanned by $S$. The following conditions are equivalent:
\begin{itemize}
\item[$(1)$] The $\infty$-category $\calD$ is accessible.
\item[$(2)$] The $\infty$-category $\calD$ is presentable.
\item[$(3)$] The collection $S$ is of small generation (as a strongly saturated class of morphisms).
\end{itemize}
\end{lemma}

\begin{proof}
We observe that $\calD$ is stable under small colimits in $\Fun(\Delta^1,\calC)$, and therefore admits small colimits; thus $(1) \Rightarrow (2)$. To see that $(2)$ implies $(3)$, we choose a small collection $S_0$ of morphisms in $\calC$ which generates $\calD$ under colimits; it is then obvious that $S_0$ generates $S$ as a strongly saturated class of morphisms. 

Now suppose that $(3)$ is satisfied. 
Choose a small collection of morphisms $\{ f_{\beta} : X_{\beta} \rightarrow Y_{\beta} \}$ which generates $S$ and an uncountable regular cardinal $\kappa$ such that $\calC$ is $\kappa$-accessible and each of the objects $X_{\beta}$, $Y_{\beta}$ is $\kappa$-compact. We will prove that $\calD$ is $\kappa$-accessible.

It is clear that $\calD$ is locally small and admits $\kappa$-filtered colimits. Let $\calD' \subseteq \calD$ be the collection of all morphisms $f: X \rightarrow Y$ such that $f$ belongs to $S$, where both $X$ and $Y$ are $\kappa$-compact. Lemma \ref{hardstuff1} implies that each $f \in \calD'$ is a $\kappa$-compact object of $\Fun(\Delta^1,\calC)$, and in particular a $\kappa$-compact object of $\calD$.
Assume for simplicity that $\calC$ is a minimal $\infty$-category, so that $\calD'$ is small. According to Proposition \ref{intprop}, the inclusion $\calD' \subseteq \calD$ is equivalent to
$j \circ F$, where $j: \calD' \rightarrow \Ind_{\kappa} \calD'$ is the Yoneda embedding and
$F: \Ind_{\kappa} \calD' \rightarrow \calD$ is $\kappa$-continuous. Proposition \ref{uterr} implies that $F$ is fully faithful; let $\calD''$ denote its essential image. To complete the proof, it will suffice to show that $\calD'' = \calD$. Let $S'' \subseteq S$ denote the collection of objects of $\calD''$ (which we may identify with morphisms in $\calC$). By construction, $S''$ contains the collection of
morphisms $\{ f_{\beta} \}$ which generates $S$. Consequently, to prove that $S'' = S$, it will suffice to show that $S''$ is strongly saturated.

It follows from Proposition \ref{sumatch} that $\calD'' \subseteq \Fun(\Delta^1,\calC)$ is stable under small colimits. We next verify that $S''$ is stable under pushouts. Let $K = \Lambda^2_0$, and let
$\overline{p}: K^{\triangleright} \rightarrow \calC$ be a colimit of $p=\overline{p} | K$, 
$$ \xymatrix{ X \ar[r]^{f} \ar[d] & Y \ar[d] \\
X' \ar[r]^{f'} & Y' }$$
such that $f$ belongs to $S''$. The proof of Proposition \ref{horse1} shows that we can write
$p$ as a colimit of a diagram $q: \Nerve(A) \rightarrow \Fun(\Lambda^2_0,\calC^{\kappa})$,
where $A$ is a $\kappa$-filtered partially ordered set. For $\alpha \in A$, we let
$p_{\alpha}$ denote the corresponding diagram, which we may depict as
$$ X'_{\alpha} \leftarrow X_{\alpha} \stackrel{ f_{\alpha} }{\rightarrow} Y_{\alpha}.$$
For each $A' \subseteq A$, we let $p_{A'}$ denote a colimit of $q| \Nerve(A')$, which we will denote by
$$ X'_{A'} \leftarrow X_{A'} \stackrel{ f_{A'} }{\rightarrow} Y_{A'}. $$
Let $B$ denote the collection of $\kappa$-small, filtered subsets $A' \subseteq A$ such that
the $f_{A'}$ belongs to $S''$. Since $f \in S''$, we conclude that $f$ can be 
obtained as the colimit of a $\kappa$-filtered diagram $\Nerve(A') \rightarrow \calD'$,
Applying Lemma \ref{walter}, we deduce that $B$ is $\kappa$-filtered, and that $A = \bigcup_{A' \in B} A'$. Using Proposition \ref{extet} and Corollary \ref{util}, we deduce that $p$ is the colimit of a
diagram $q': \Nerve(B) \rightarrow \Fun(\Lambda^2_0,\calC)$, where $q'(A') = p_{A'}$. Replacing
$A$ by $B$, we may suppose that each $f_{\alpha}$ belongs to $S'$.

Let $\colim: \Fun(\Lambda^2_0,\calC) \rightarrow \Fun(\Delta^1 \times \Delta^1,\calC)$ be a colimit functor
(that is, a left adjoint to the restriction functor). Lemma \ref{limitscommute} implies
that we may identify $\overline{p}$ with a colimit of the diagram $\colim \circ q$. 
Consequently, the morphism $f'$ can be written as a colimit of morphisms $f'_{\alpha}$ which
fit into pushout diagrams
$$ \xymatrix{ X_{\alpha} \ar[r]^{f_{\alpha}} \ar[d] & Y_{\alpha} \ar[d] \\
X'_{\alpha} \ar[r]^{f'_{\alpha}} & Y'_{\alpha}. }$$
Since $f_{\alpha} \in S'' \subseteq S$, we conclude that $f'_{\alpha} \in S$. Since
$X'_{\alpha}$ and $Y'_{\alpha}$ are $\kappa$-compact, we deduce that
$f'_{\alpha} \in S''$. Since $\calD''$ is stable under colimits, we deduce that $f' \in S''$, as desired.

We now complete the proof by showing that $S''$ has the two-out-of-three property, using the same style of argument. 
Let $\sigma: \Delta^2 \rightarrow \calC$ be a simplex corresponding to a diagram
$$ \xymatrix{ X \ar[rr]^{f} \ar[dr]^{g} & & Y \ar[dl]^{h} \\
& Z & }$$
in $\calC$. We will show that if $f,g \in S''$, then $h \in S''$: the argument in the other two cases is the same. The proof of Proposition \ref{horse1} shows that we can write $\sigma$
as the colimit of a diagram $q: \Nerve(A) \rightarrow \Fun(\Delta^2,\calC^{\kappa})$, where
$A$ is a $\kappa$-filtered partially ordered set. For each $\alpha \in A$, we will denote the
corresponding diagram by
$$ \xymatrix{ X_{\alpha} \ar[rr]^{f_{\alpha}} \ar[dr]^{g_{\alpha}} & & Y_{\alpha} \ar[dl]^{h_{\alpha}} \\
& Z_{\alpha}. & }$$
Arguing as above, we may assume (possibly after changing $A$ and $q$) that each
$f_{\alpha}$ belongs to $S''$. Repeating the same argument, we may suppose that
$g_{\alpha}$ belongs to $S''$. Since $S$ has the two-out-of-three property, we conclude that each $h_{\alpha}$ belongs to $S$. Since $X_{\alpha}$ and $Z_{\alpha}$ are $\kappa$-compact, we then have $h_{\alpha} \in S''$. The stability of $\calD''$ under colimits now implies that $h \in S''$, as desired.
\end{proof}

\begin{proposition}\label{local}
Let $\calC$ be a presentable $\infty$-category, and let $S$ be a $($small$)$ collection of morphisms of $\calC$. Let $\overline{S}$ denote the strongly saturated class of morphisms generated by $S$.
Let $\calC' \subseteq \calC$ denote the full subcategory of $\calC$ consisting of $S$-local objects. Then:

\begin{itemize}
\item[$(1)$] For each object $C \in \calC$, there exists a morphism $s: C \rightarrow C'$ such that
$C'$ is $S$-local and $s$ belongs to $\overline{S}$.
\item[$(2)$] The $\infty$-category $\calC'$ is presentable.
\item[$(3)$] The inclusion $\calC' \subseteq \calC$ has a left adjoint $L$.
\item[$(4)$] For every morphism $f$ of $\calC$, the following are equivalent:
\begin{itemize}
\item[$(i)$] The morphism $f$ is an $S$-equivalence.
\item[$(ii)$] The morphism $f$ belongs to $\overline{S}$.
\item[$(iii)$] The induced morphism $Lf$ is an equivalence.
\end{itemize}
\end{itemize}
\end{proposition}

\begin{proof}
%To prove $(1)$, we take $C$ to be an arbitrary object of $\calC$, let $\calD$ be the full subcategory of $\Fun(\Delta^1,\calC)$ spanned by the elements of $S$, and form a fiber diagram
%$$ \xymatrix{ \calD_{C} \ar[r] \ar[d] & \calD \ar[d] \\
%\{C\} \ar[r] & \Fun(\{0\}, \calC). }$$
%Since $\overline{S}$ is stable under pushouts, the right vertical map is a coCartesian fibration, so that the above diagram is homotopy Cartesian by Proposition \ref{basechangefunky}. Lemma 
%\ref{perry} asserts that $\calD$ is accessible, so that $\calD_{C}$ is accessible by Proposition \ref{horse2}. Lemma \ref{yorkan} now implies that $\calD_{C}$ is presentable, so that
%$\calD_{C}$ has a final object $s: C \rightarrow C'$ by Corollary \ref{preslim}. The morphism $s$ belongs to $\overline{S}$ by construction; we will complete the proof of $(1)$ by showing that $C'$ is $S$-local.

%Let $t: X \rightarrow Y$ be an arbitrary morphism in $\calC$ which belongs to $S$. We wish to show that composition with $t$ induces a homotopy equivalence $\phi: \bHom_{\calC}(Y,C') \rightarrow \bHom_{\calC}(X,C')$. Let $g: X \rightarrow C'$ be an arbitrary morphism; using Lemma \ref{sugarplace} we may identify $\bHom_{\calC_{X/}}(t,g)$ with the homotopy fiber of
%$\phi$ over the base point $g$ of $\bHom_{\calC}(X,C')$. We wish to show that this space
%is contractible. Form a pushout diagram
%$$ \xymatrix{ X \ar[d]^{g} \ar[r]^{t} & Y \ar[d] \\
%C' \ar[r]^{t'} & C''}$$
%in the $\infty$-category $\calC$. 
%Lemma \ref{sugarplaces} implies the existence of a homotopy equivalence
%$\bHom_{\calC_{X/}}(t,g) \simeq \bHom_{\calC_{C'/}}(t', \id_{C'})$. It will therefore suffice to prove that $\bHom_{\calC_{C'/}}(t', \id_{C'})$ is contractible. Since $t'$ is a pushout of $t$, it belongs to $\overline{S}$. Let
%$\sigma$ be a $2$-simplex of $\calC$ classifying a diagram
%$$ \xymatrix{ & C' \ar[dr]^{t'} & \\
%C \ar[ur]^{s} \ar[rr]^{s'} & & C'', }$$
%so that $s'$ is a composition of the morphisms $s$ and $t'$ in $\calC$, and therefore also belongs to $\overline{S}$.
%Applying Lemma \ref{sugarplace} again, we may identify
%$$\bHom_{\calC_{C'/}}(t', \id_{C'}) \simeq \bHom_{\calC_{s/}}(\sigma, s_1(s))$$ with the homotopy fiber of the map
%$$ \bHom_{ \calC_{C/}}(s',s) \rightarrow \bHom_{ \calC_{C/}}(s,s).$$
%given by composition with $\sigma$.
%By construction, $\calD_{C}$ is a full subcategory of $\calC^{C/}$ which contains
%$s$ and $s'$, and $s$ is a final object of $\calD_{C}$.
%In view of the equivalence of $\calC_{C/}$ with $\calC^{C/}$, we conclude that the spaces
%$\bHom_{\calC_{C/}}(s',s)$ and $\bHom_{\calC_{C/}}(s,s)$ are contractible, so that
%$\phi$ is a homotopy equivalence as desired. This completes the proof of $(1)$.
Assertion $(1)$ is a consequence of Lemma \ref{superlocal}, which we will prove in \S \ref{factgen2}. The equivalence $(1) \Leftrightarrow (3)$ follows immediately from Proposition \ref{testreflect}. We now prove $(4)$. Lemma \ref{bluh} implies that the collection of $S$-equivalences
is a strongly saturated class of morphisms containing $S$; it therefore contains $\overline{S}$, so that $(ii) \Rightarrow (i)$. Now suppose that $f: X \rightarrow Y$ is such that $Lf$ is an equivalence,
and consider the diagram 
$$ \xymatrix{ X \ar[r] \ar[dr] \ar[d] & Y \ar[d] \\
LX \ar[r] & LY. }$$
Our proof of $(1)$ shows that the vertical morphisms belong to $\overline{S}$, and the
lower horizontal arrow belongs to $\overline{S}$ by Remark \ref{simpcons}. Two applications of the two-out-of-three property now show that $f \in \overline{S}$, so that $(iii) \Rightarrow (ii)$. If $f$ is an $S$-equivalence, then
we may again use the above diagram and the two-out-of-three property to conclude that $Lf$ is an equivalence. It follows that $LX$ and $LY$ co-represent the same functor on the homotopy category $h\calC'$, so that Yoneda's lemma implies that $Lf$ is an equivalence.
Thus $(i) \Rightarrow (iii)$ and the proof of $(4)$ is complete.

It remains to prove $(2)$. Remark \ref{localcolim} implies that $L \calC$ admits small colimits, so
it will suffice to prove that $L \calC$ is accessible. According to Proposition \ref{localloc}, 
this follows from the implication $(iii) \Rightarrow (i)$ of assertion $(4)$.
\end{proof}

Proposition \ref{local} gives a clear picture of the collection of all accessible localizations of a presentable $\infty$-category $\calC$. For any (small) set of morphisms $S$ in $\calC$, the full subcategory $S^{-1} \calC \subseteq \calC$ consisting of $S$-local objects is a localization of $\calC$, and every localization arises in this way. Moreover, the subcategories $S^{-1} \calC$ and $T^{-1} \calC$\index{not}{S-C@$S^{-1} \calC$}
coincide if and only if $S$ and $T$ generate the same strongly saturated class of morphisms.
We will also write $S^{-1} \calC$ for the class of $S$-local objects of $\calC$ in the case where $S$ is {\em not} small; however, this is generally only a well-behaved object in the case where there is a (small) subset $S_0 \subseteq S$ which generates the same strongly saturated class of morphisms.

\begin{proposition}\label{postbluse}
Let $f: \calC \rightarrow \calD$ be a presentable functor between presentable $\infty$-categories, let $S$ be strongly saturated class of morphisms of $\calD$ which is of small generation. Then $f^{-1} S$ is of small generation $($as a strongly saturated class of morphisms of $\calC${}$)$.
\end{proposition}

\begin{proof}
Replacing $\calD$ by $S^{-1} \calD$ if necessary, we may suppose that $S$ consists
of precisely the equivalences in $\calD$. Let $\calE_{\calD} \subseteq \Fun(\Delta^1,\calD)$ denote the full subcategory spanned by those morphisms which are equivalences in $\calD$, and let
$\calE_{\calC} \subseteq \Fun(\Delta^1,\calC)$ denote the full subcategory spanned by
those morphisms which belong to $f^{-1} S$. We have a homotopy Cartesian diagram
of $\infty$-categories
$$ \xymatrix{ \calE_{\calC} \ar[r] \ar[d] & \calE_{\calD} \ar[d] \\
\Fun(\Delta^1,\calC) \ar[r] & \Fun(\Delta^1,\calD). }$$
The $\infty$-category $\calE_{\calD}$ is equivalent to $\calD$, and therefore presentable.
The $\infty$-categories $\Fun(\Delta^1,\calC)$ and $\Fun(\Delta^1,\calD)$ are presentable by 
Proposition \ref{presexp}. It follows from Proposition \ref{horse22} that $\calE_{\calC}$
is presentable. In particular, there is a small collection of objects of $\calE_{\calC}$ which generates $\calE_{\calC}$ under colimits, as desired.
\end{proof}

Let $\calC$ be a presentable $\infty$-category. We will say that a full subcategory
$\calC^0 \subseteq \calC$ is {\it strongly reflective} if it is
the essential image of an accessible localization functor. Equivalently, $\calC^0$ is strongly reflective if it is presentable, stable under equivalence in $\calC$, and the inclusion
$\calC^0 \subseteq \calC$ admits a left adjoint. According to Proposition \ref{local}, $\calC^0$ is strongly reflective if and only if there exists a (small) set $S$ of morphisms of $\calC$ such that $\calC^0$ is the full subcategory of $\calC$ spanned by the $S$-local objects.
For later use, we record a few easy stability properties enjoyed by the collection of strongly reflective subcategories of $\calC$:\index{gen}{strongly reflective}\index{gen}{reflective subcategory!strongly}

\begin{lemma}\label{stur2}
Let $f: \calC \rightarrow \calD$ be a presentable functor between presentable $\infty$-categories, and let $\calC^0 \subseteq \calC$ be a strongly reflective subcategory. Let $f^{\ast}$ denote a right adjoint of $f$, and let
$\calD^{0} \subseteq \calD$ be the full subcategory spanned by those objects
$D \in \calD$ such that $f^{\ast} D \in \calC^0$. Then $\calD^{0}$ is a strongly reflective subcategory of $\calD$.
\end{lemma}

\begin{proof}
Let $S$ be a (small) set of morphisms of $\calC$ such that $\calC^0$ is the full subcategory
of $\calC$ spanned by the $S$-local objects. Then $\calD^0$ is the full subcategory of $\calD$ spanned by the $f(S)$-local objects.
\end{proof}

\begin{lemma}\label{stur3}
Let $\calC$ be a presentable $\infty$-category, and let $\{ \calC_{\alpha} \}_{\alpha \in A}$ be a family of full subcategories of $\calC$ indexed by a $($small$)$ set $A$. Suppose that each
$\calC_{\alpha}$ is strongly reflective. Then
$\bigcap_{\alpha \in A} \calC_{\alpha}$ is strongly reflective.
\end{lemma}

\begin{proof}
For each $\alpha \in A$, choose a (small) set $S(\alpha)$ of morphisms of $\calC$ such that $\calC_{\alpha}$ is the full subcategory of $\calC$ spanned by the $S(\alpha)$-local objects.
Then $\bigcap_{\alpha \in A} \calC_{\alpha}$ is the full subcategory of $\calC$ spanned by the
$\bigcup_{\alpha \in A} S(\alpha)$-local objects.
\end{proof}

\begin{lemma}\label{stur1}
Let $\calC$ be a presentable $\infty$-category and $K$ a small simplicial set.
Let $\calD$ denote the full subcategory of $\Fun(K^{\triangleleft}, \calC)$ spanned by those diagrams $\overline{p}: K^{\triangleleft} \rightarrow \calC$ which are limits of $p = \overline{p}|K$.
Then $\calD$ is a strongly reflective subcategory of $\calC$.
\end{lemma}

\begin{proof}
The restriction functor $\calD \rightarrow \Fun(K,\calC)$ is an equivalence of $\infty$-categories. This proves that $\calD$ is accessible. Let $s: \Fun(K,\calC) \rightarrow \calD$ be a homotopy inverse to the restriction map. Then the composition
$$ \Fun(K^{\triangleright}, \calC) \rightarrow \Fun(K,\calC) \stackrel{s}{\rightarrow} \calD$$
is left adjoint to the inclusion.
\end{proof}

We conclude this section by giving a universal property which characterizes the localization $S^{-1} \calC$.

\begin{proposition}\label{unichar}
Let $\calC$ be a presentable $\infty$-category, and $\calD$ an arbitrary $\infty$-category.
Let $S$ be a $($small$)$ set of morphisms of $\calC$, and $L: \calC \rightarrow S^{-1} \calC \subseteq \calC$ an associated (accessible) localization functor. Composition with $L$ induces
a functor
$$ \eta: \LFun( S^{-1} \calC, \calD) \rightarrow \LFun( \calC, \calD).$$
The functor $\eta$ is fully faithful, and the essential image of $\eta$ consists of those functors
$f: \calC \rightarrow \calD$ such that $f(s)$ is an equivalence in $\calD$, for each $s \in S$.
\end{proposition}

\begin{proof}
Let $\alpha: \id_{\calC} \rightarrow L$ be a unit for the adjunction between $L$ and
the inclusion $S^{-1} \calC \subseteq \calC$.
We first observe that every functor $f_0: S^{-1} \calC \rightarrow \calD$ admits a right
Kan extension $f: \calC \rightarrow \calD$. To prove this, we may first replace $f_0$ by
the equivalent diagram $g_0 = f_0 \circ (L| S^{-1} \calC)$, and define $g= f_0 \circ L$. To prove
that $g$ is a right Kan extension of $g_0$, it suffices to show that for each
object $X \in \calC$, the diagram
$$ \overline{p}: (S^{-1} \calC)_{X/}^{\triangleleft} \rightarrow \calC \stackrel{L}{\rightarrow} 
S^{-1} \calC \stackrel{f_0}{\rightarrow} \calD$$
exhibits $f_0(LX)$ as a limit of $p = \overline{p} | (S^{-1} \calC)_{/X})$. For this, we note
that an $S$-localization map $\alpha(X): X \rightarrow LX$ is an initial object of $(S^{-1} \calC)_{X/}$ (Remark \ref{initrem}), and that $f_0( L \alpha(X))$ is an equivalence by Proposition \ref{recloc}.

Let $\calX$ denote the full subcategory of $\calD^{\calC}$ spanned by those functors
$f: \calC \rightarrow \calD$ which are right Kan extensions of $f| S^{-1} \calC$. According to 
Proposition \ref{lklk}, the restriction map $\calX \rightarrow \Fun(S^{-1} \calC,\calD)$ is a
trivial fibration. Let $\overline{\eta}: \Fun(S^{-1} \calC, \calD) \rightarrow \Fun(\calC, \calD)$ be given
by composition with $L$. The above argument shows that $\overline{\eta}$ factors through $\calX$.
Moreover, the composition of $\overline{\eta}$ with the restriction map is homotopic to the identity on $\Fun(S^{-1} \calC, \calD)$. It follows that $\overline{\eta}$ is an equivalence of $\infty$-categories.

We have a commutative diagram
$$ \xymatrix{ \LFun(S^{-1} \calC, \calD) \ar[r]^{\eta} \ar[d] & \LFun(\calC, \calD) \ar[d] \\
\Fun(S^{-1} \calC,\calD) \ar[r]^{\overline{\eta}} & \Fun(\calC, \calD) }$$
where the vertical maps are inclusions of full subcategories, and the lower horizontal map is fully faithful. It follows that $\eta$ is fully faithful. To complete the proof, we must show that a functor
$f: \calC \rightarrow \calD$ belongs to the essential image of $\eta$ if and only if
$f(s)$ is an equivalence for each $s \in S$. The ``only if'' direction is clear, since
the functor $L$ carries each element of $S$ to an equivalence in $\calC$. Conversely, suppose that
$f$ carries each $s \in S$ to an equivalence. The natural transformation $\alpha$
gives a map of functors $\alpha(f): f \rightarrow f \circ L$; we wish to show that $\alpha(f)$ is an equivalence. Equivalently, we wish to show that for each object $X \in \calC$, $f$ carries
the map $\alpha(X): X \rightarrow LX$ to an equivalence in $\calD$. Let $S'$ denote the
class of all morphisms $\phi$ in $\calC$ such that $f(\phi)$ is an equivalence in $\calD$.
By assumption, $S \subseteq S'$. Lemma \ref{bluh} implies that $S'$ is strongly saturated, so
that Proposition \ref{local} asserts that $\alpha(X) \in S'$, as desired.
\end{proof}

\subsection{Factorization Systems on Presentable $\infty$-Categories}\label{factgen2}

Let $\calC$ be a presentable $\infty$-category. In \S \ref{invloc}, we saw that it is easy to produce localizations of $\calC$: for any small collection of morphisms $S$ in $\calC$, the full subcategory $S^{-1} \calC$ of $S$-local objects of $\calC$ is a presentable localization of $\calC$, which depends only on the strongly saturated class of morphisms $\overline{S}$ generated by $S$. Our goal in this section is to prove a similar result for factorization systems on
$\calC$. The first step is to introduce the analogue of the notion of ``strongly saturated'':

\begin{definition}\index{gen}{saturated}
Let $S$ be a collection of morphisms in a presentable $\infty$-category $\calC$. We will say that
$S$ is {\it saturated} if the following conditions are satisfied:
\begin{itemize}
\item[$(1)$] The collection $S$ is closed under small colimits in $\Fun( \Delta^1, \calC)$.
\item[$(2)$] The collection $S$ contains all equivalences and is stable under composition.
\item[$(3)$] The collection $S$ is closed under the formation of pushouts. That is, given a pushout diagram
$$ \xymatrix{ X \ar[r] \ar[d]^{f} & X' \ar[d]^{f'} \\
Y \ar[r] & Y' }$$
in $\calC$, if $f$ belongs to $S$ then $f'$ also belongs to $S$.
\end{itemize}
\end{definition}

\begin{remark}\index{gen}{of small generation}
Let $\calC$ be a presentable $\infty$-category. Then any intersection of saturated collections of morphisms in $\calC$ is again saturated. It follows that for {\em any} class of morphisms $S$ of $\calC$, there exists a smallest saturated collection of morphisms $\overline{S}$ containing $S$. We will refer to $\overline{S}$ as the {\it saturated collection of morphisms generated by $S$}. We will say that a saturated collection of morphisms $\overline{S}$ is {\it of small generation} if it
is generated by some (small) subset $S \subseteq \overline{S}$.
\end{remark}

\begin{remark}
If $S$ is a saturated collection of morphisms of $\calC$, then $S$ is closed under retracts.
\end{remark}

\begin{remark}
Let $\calC$ be (the nerve of) a presentable category, and let $S$ be a saturated class of morphisms
in $\calC$. Then $S$ is also weakly saturated, in the sense of Definition \ref{saturated}.
\end{remark}

\begin{example}
Let $\calC$ be a presentable $\infty$-category. Then every strongly saturated class of morphisms in $\calC$ is also saturated.
\end{example}

\begin{example}\label{cobblet}
Let $\calC$ be a presentable $\infty$-category, and $S$ any collection of morphisms of $\calC$.
Then $^{\perp}S$ is saturated; this follows immediately from Proposition \ref{swimmm}. 
In particular, if $(S_L, S_R)$ is a factorization system on $\calC$, then $S_L$ is saturated.
\end{example}

The main result of this section is the following converse to Example \ref{cobblet}:

\begin{proposition}\label{nir}
Let $\calC$ be a presentable $\infty$-category, and $S$ a saturated collection of morphisms in $\calC$ which is of small generation. Then $(S, S^{\perp})$ is a factorization system on $\calC$.
\end{proposition}

\begin{corollary}\label{wugg}
Let $\calC$ be a presentable $\infty$-category, let $S$ be a saturated collection of morphisms of $\calC$, and suppose that $S$ is of small generation. Let
$$ \xymatrix{ & Y \ar[dr]^{g} & \\
X \ar[ur]^{f} \ar[rr]^{h} & & Z} $$
be a commutative diagram in $\calC$. If $f$ and $h$ belong to $S$, then $g$ belongs to $S$.
\end{corollary}

\begin{proof}
Combine Propositions \ref{nir}, \ref{swin}, and \ref{swimmm}. 
\end{proof}

In the situation of Proposition \ref{nir}, we will refer to the elements of $S^{\perp}$ as {\it $S$-local morphisms of $\calC$}. Note that an object $X \in \calC$ is $S$-local if and only if a
morphism $X \rightarrow 1_{\calC}$ is $S$-local, where $1_{\calC}$ denotes a final object of $\calC$. 

The proof of Proposition \ref{nir} will be given at the end of this section, after we have established a series of technical lemmas.

\begin{lemma}\label{sweener}
Let $\calC$ be a presentable $\infty$-category, and $S$ a saturated collection of morphisms
of $\calC$. The following conditions are equivalent:
\begin{itemize}
\item[$(1)$] The collection $S$ is of small generation.
\item[$(2)$] The full subcategory $\calD \subseteq \Fun( \Delta^1, \calC)$ spanned by the elements of
$S$ is presentable.
\end{itemize}
\end{lemma}

\begin{proof}
If $\calD$ is presentable, then $\calD$ is generated under small colimits by a small set of objects; these objects clearly generate $S$ as a saturated collection of morphisms. This proves that $(2) \Rightarrow (1)$. To prove the reverse implication, choose a small collection of morphisms $\{ f_{\beta} : X_{\beta} \rightarrow Y_{\beta} \}$ which generates $S$ as a semiaturated class of morphisms and an uncountable regular cardinal $\kappa$ such that $\calC$ is $\kappa$-accessible and each of the objects $X_{\beta}$, $Y_{\beta}$ is $\kappa$-compact. Let $\calD' \subseteq \calD$ be the collection of all morphisms $f: X \rightarrow Y$ such that $f$ belongs to $S$, where both $X$ and $Y$ are $\kappa$-compact. Lemma \ref{hardstuff1} implies that each $f \in \calD'$ is a $\kappa$-compact object of $\Fun(\Delta^1,\calC)$, and in particular a $\kappa$-compact object of $\calD$. 
Assume for simplicity that $\calC$ is a minimal $\infty$-category, so that $\calD'$ is small. According to Proposition \ref{intprop}, the inclusion $\calD' \subseteq \calD$ is equivalent to
$j \circ F$, where $j: \calD' \rightarrow \Ind_{\kappa} \calD'$ is the Yoneda embedding and
$F: \Ind_{\kappa} \calD' \rightarrow \calD$ is $\kappa$-continuous. Proposition \ref{uterr} implies that $F$ is fully faithful; let $\calD''$ denote its essential image. To complete the proof, it will suffice to show that $\calD'' = \calD$. 

Let $S' \subseteq S$ denote the collection of objects of $\calD''$ (which we may identify with morphisms in $\calC$). By construction, $S'$ contains the collection of morphisms $\{ f_{\beta} \}$ which generates $S$. Consequently, to prove that $S' = S$, it will suffice to show that $S'$ is saturated.

It follows from Proposition \ref{sumatch} that $\calD'' \subseteq \Fun(\Delta^1,\calC)$ is stable under small colimits. We next verify that $S'$ is stable under pushouts. Let $K = \Lambda^2_0$, and let
$\overline{p}: K^{\triangleright} \rightarrow \calC$ be a colimit of $p=\overline{p} | K$, 
$$ \xymatrix{ X \ar[r]^{f} \ar[d] & Y \ar[d] \\
X' \ar[r]^{f'} & Y' }$$
such that $f$ belongs to $S'$. Using Proposition \ref{urgh1}, we can write $p$ as the colimit of a diagram $q: \Nerve(A) \rightarrow \Fun(\Lambda^2_0,\calC^{\kappa})$,
where $A$ is a $\kappa$-filtered partially ordered set. For $\alpha \in A$, we let
$p_{\alpha}$ denote the corresponding diagram, which we may depict as
$$ X'_{\alpha} \leftarrow X_{\alpha} \stackrel{ f_{\alpha} }{\rightarrow} Y_{\alpha}.$$
For each $A' \subseteq A$, we let $p_{A'}$ denote a colimit of $q| \Nerve(A')$, which we will denote by
$$ X'_{A'} \leftarrow X_{A'} \stackrel{ f_{A'} }{\rightarrow} Y_{A'}. $$
Let $B$ denote the collection of $\kappa$-small, filtered subsets $A' \subseteq A$ such that
the $f_{A'}$ belongs to $S'$. Since $f \in S'$, we conclude that $f$ can be 
obtained as the colimit of a $\kappa$-filtered diagram in $\calD'$,
Applying Lemma \ref{walter}, we deduce that $B$ is $\kappa$-filtered, and that $A = \bigcup_{A' \in B} A'$. Using Proposition \ref{extet} and Corollary \ref{util}, we deduce that $p$ is the colimit of a diagram $q': \Nerve(B) \rightarrow \Fun(\Lambda^2_0,\calC)$, where $q'(A') = p_{A'}$. Replacing $A$ by $B$, we may suppose that each $f_{\alpha}$ belongs to $S'$.

Let $\colim: \Fun(\Lambda^2_0,\calC) \rightarrow \Fun(\Delta^1 \times \Delta^1,\calC)$ be a colimit functor (that is, a left adjoint to the restriction functor). Lemma \ref{limitscommute} implies
that we may identify $\overline{p}$ with a colimit of the diagram $\colim \circ q$. 
Consequently, the morphism $f'$ can be written as a colimit of morphisms $f'_{\alpha}$ which
fit into pushout diagrams
$$ \xymatrix{ X_{\alpha} \ar[r]^{f_{\alpha}} \ar[d] & Y_{\alpha} \ar[d] \\
X'_{\alpha} \ar[r]^{f'_{\alpha}} & Y'_{\alpha}. }$$
Since $f_{\alpha} \in S' \subseteq S$, we conclude that $f'_{\alpha} \in S$. Since
$X'_{\alpha}$ and $Y'_{\alpha}$ are $\kappa$-compact, we deduce that
$f'_{\alpha} \in S'$. Since $\calD''$ is stable under colimits, we deduce that $f' \in S'$, as desired.

We now complete the proof by showing that $S'$ is stable under composition.
Let $\sigma: \Delta^2 \rightarrow \calC$ be a simplex corresponding to a diagram
$$ \xymatrix{ X \ar[rr]^{f} \ar[dr]^{g} & & Y \ar[dl]^{h} \\
& Z & }$$
in $\calC$. We will show that if $f,g \in S'$, then $h \in S'$. Using Proposition \ref{urgh1}, we can write $\sigma$ as the colimit of a diagram $q: \Nerve(A) \rightarrow \Fun(\Delta^2,\calC^{\kappa})$, where
$A$ is a $\kappa$-filtered partially ordered set. For each $\alpha \in A$, we will denote the
corresponding diagram by
$$ \xymatrix{ X_{\alpha} \ar[rr]^{f_{\alpha}} \ar[dr]^{g_{\alpha}} & & Y_{\alpha} \ar[dl]^{h_{\alpha}} \\
& Z_{\alpha}. & }$$
Arguing as above, we may assume (possibly after changing $A$ and $q$) that each
$f_{\alpha}$ belongs to $S'$. Repeating the same argument, we may suppose that
$g_{\alpha}$ belongs to $S'$. Since $S$ is stable under composition, we conclude that each
$h_{\alpha}$ belongs to $S$. Since each $X_{\alpha}$ and $Z_{\alpha}$ are $\kappa$-compact, we have $h_{\alpha} \in S'$. The stability of $\calD''$ under colimits now implies that $h \in S'$, as desired.
\end{proof}

\begin{lemma}\label{swunl}
Let $\calC$ be a presentable $\infty$-category, and let $S$ be a saturated collection of morphisms in $\calC$. For every object $X \in \calC$, let
$S_X$ denote the collection of all morphisms of $\calC_{/X}$ whose image in $\calC$ belongs to $S$. Then each $S_{X}$ is strongly saturated in $\calC_{/X}$. Moreover, if $S$ is of small generation, then each $S_{X}$ is also of small generation.
\end{lemma}

\begin{proof}
The first assertion follows immediately from the definitions and Proposition \ref{needed17}.
To prove the second, let $\calD$ be the full subcategory of $\Fun( \Delta^1, \calC)$ spanned by the elements of $S$, and $\calD'$ the full subcategory of $\Fun( \Delta^1, \calC_{/X})$ spanned by the elements of $S_X$. We have a (homotopy) pullback diagram of $\infty$-categories
$$ \xymatrix{ \calD' \ar[d] \ar[r] & \Fun( \Delta^1, \calC_{/X} ) \ar[d]^{\psi} \\
\calD \ar[r]^-{\phi} & \Fun(\Delta^1, \calC). }$$
The functors $\phi$ and $\psi$ preserve small colimits, and the $\infty$-categories
$\Fun( \Delta^1, \calC_{/X})$, $\Fun( \Delta^1, \calC)$, and $\calD$ are all presentable
(the last in view of Lemma \ref{sweener}). Using Proposition \ref{horse22}, we deduce that $\calD'$ is presentable, and therefore generated under small colimits by a (small) set of elements of $S_{X}$. This proves that $S_{X}$ is of small generation, as desired.
\end{proof}

\begin{lemma}\label{sweetyork}
Let $\calC$ be a presentable $\infty$-category, $S$ a saturated collection of morphisms of $\calC$, and $X$ an object of $\calC$. Then the full subcategory $\calD \subseteq \calC^{X/}$ spanned by the elements which belong to $S$ is closed under small colimits.
\end{lemma}

\begin{proof}
In view of Corollary \ref{uterrr} and \ref{allfin}, it will suffice to show that
$\calD$ is closed under small filtered colimits, pushouts, and contains the initial objects
of $\calC^{X/}$. The last condition follows from the fact that $S$ contains all equivalences.
Now suppose that $\overline{p}: K^{\triangleright} \rightarrow \calC^{C/}$
is a colimit of $p = \overline{p} | K$, where $K$ is either filtered or equivalent to $\Lambda^2_0$,
and that $p(K) \subseteq \calD$. We can identify $\overline{p}$ with a map
$P: K^{\triangleright} \times \Delta^1 \rightarrow \calC$ such that
$P | K^{\triangleright} \times \{0\}$ is the constant map taking the value $C \in \calC$.
Since $K$ is weakly contractible, $P | K^{\triangleright} \times \{0\}$ is a colimit diagram in $\calC$.
The map $P | K^{\triangleright} \times \{1\}$ is the image of a colimit diagram under the left
fibration $\calC^{C/} \rightarrow \calC$; since $K$ is weakly contractible, Proposition \ref{goeselse} implies that $P | K^{\triangleright} \times \{1\}$ is a colimit diagram. We now apply
Proposition \ref{limiteval} to deduce that
$P: K^{\triangleright} \rightarrow \calC^{\Delta^1}$ is a colimit diagram. Since $S$ is stable under colimits in $\calC^{\Delta^1}$, we conclude that $P$
carries the cone point of $K^{\triangleright}$ to a morphism belonging to $S$, as we wished to show.
\end{proof}

\begin{lemma}\label{sugarplace}
Let $\calC$ be an $\infty$-category, and let $f: C \rightarrow D$, $g: C \rightarrow E$ be morphisms in $\calC$. Then there is a natural identification of $\bHom_{\calC_{C/}}(f,g)$ with
the homotopy fiber of the map
$$ \bHom_{\calC}(D,E) \rightarrow \bHom_{\calC}(C,E)$$
induced by composition with $f$, where the fiber is taken over the point corresponding to $g$.
\end{lemma}

\begin{proof}
We have a commutative diagram of simplicial sets
$$ \xymatrix{ \calC_{f/} \times_{\calC_{C/}} \{g\} \ar[d]^{\phi} \ar[r] & \calC_{f/} \times_{\calC} \{E\} \ar[d]^{\phi'} \ar[r] & \calC_{f/} \ar[d]^{\phi''} \\
\{g\} \ar[r] & \calC_{C/} \times_{\calC} \{E\}  \ar[r] & \calC_{C/} }$$
where both squares are pullbacks. Proposition \ref{sharpen} asserts that $\phi''$ is a left fibration, so that $\phi'$ and $\phi$ are left fibrations as well. 
Since $\calC_{C/} \times_{\calC} \{E\} = \Hom^{\lft}_{\calC}(C,E)$ is a Kan complex, the map
$\phi'$ is actually a Kan fibration (Lemma \ref{toothie2}), so that the square on the left is homotopy pullback, and identifies
$$ \calC_{f/} \times_{ \calC_{C/} } \{g\} \simeq \bHom_{\calC_{C/}}(f,g)$$
with the homotopy fiber of $\phi'$ over $g$; we conclude by observing that $\phi'$ is a model for the map $$ \bHom_{\calC}(D,E) \rightarrow \bHom_{\calC}(C,E)$$ given by composition with $f$. 
\end{proof}

\begin{lemma}\label{sugarplaces}
Let $$ \xymatrix{ X \ar[r]^{f} \ar[d]^{g} & X' \ar[d] \\
Y \ar[r]^{f'} & Y' }$$ be a pushout diagram in an $\infty$-category $\calC$.
Then there exists an isomorphism
$$ \bHom_{\calC_{X/}}(f,g) \simeq \bHom_{\calC_{Y/}}(f', \id_{Y})$$
in the homotopy category $\calH$. 
\end{lemma}

\begin{proof}
According to Corollary \ref{strictify}, we can assume without loss of generality
that $\calC$ is the nerve of a fibrant simplicial category $\calD$, and that the diagram in question is the nerve of a commutative diagram
 $$ \xymatrix{ X \ar[r]^{f} \ar[d]^{g} & X' \ar[d] \\
Y \ar[r]^{f'} & Y' }$$
in $\calD$. Theorem \ref{colimcomparee} implies that this diagram is homotopy coCartesian in $\calD$, so that we have a homotopy pullback diagram
$$ \xymatrix{ \bHom_{\calD}(Y', Y) \ar[r]^{\phi} \ar[d] & \bHom_{\calD}(Y,Y) \ar[d] \\
\bHom_{\calD}(X',Y) \ar[r]^{\phi'} & \bHom_{\calD}(X,Y) }$$
of Kan complexes. Consequently, we obtain an isomorphism in $\calH$ between  the homotopy fiber of $\phi$ over $\id_{Y}$ and the homotopy fiber of $\phi'$ over $g$.
According to Lemma \ref{sugarplace}, these homotopy fibers may be identified with
$\bHom_{\calC_{Y/}}(f', \id_{Y})$ and $ \bHom_{\calC_{X/}}(f,g)$, respectively. 
\end{proof}

\begin{lemma}\label{superlocal}
Let $\calC$ be a presentable $\infty$-category, and let $S$ be a saturated collection of morphisms of $\calC$ which is of small generation. Then, for every object $X \in \calC$, there exists a morphism
$f: X \rightarrow Y$ in $\calC$, such that $f \in S$ and $Y$ is $S$-local.
\end{lemma}

\begin{proof}
Let $\calD$ be the full subcategory of $\Fun(\Delta^1,\calC)$ spanned by the elements of $S$, and form a fiber diagram
$$ \xymatrix{ \calD_{X} \ar[r] \ar[d] & \calD \ar[d] \\
\{X\} \ar[r] & \Fun(\{0\}, \calC). }$$
Since $S$ is stable under pushouts, the right vertical map is a coCartesian fibration, so that the above diagram is homotopy Cartesian by Proposition \ref{basechangefunky}. Lemma 
\ref{sweener} asserts that $\calD$ is accessible, so that $\calD_{X}$ is accessible by Proposition \ref{horse2}. Using Lemma \ref{sweetyork}, we conclude that $\calD_{X}$ is presentable, so that
$\calD_{X}$ has a final object $f: X \rightarrow Y$. To complete the proof, it will suffice to show that $Y$ is $S$-local.

Let $t: A \rightarrow B$ be an arbitrary morphism in $\calC$ which belongs to $S$. We wish to show that composition with $t$ induces a homotopy equivalence $\phi: \bHom_{\calC}(B,Y) \rightarrow \bHom_{\calC}(A,Y)$. Let $g: A \rightarrow Y$ be an arbitrary morphism; using Lemma \ref{sugarplace} we may identify $\bHom_{\calC_{A/}}(t,g)$ with the homotopy fiber of
$\phi$ over the base point $g$ of $\bHom_{\calC}(A,Y)$. We wish to show that this space
is contractible. Form a pushout diagram
$$ \xymatrix{ A \ar[d]^{g} \ar[r]^{t} & B \ar[d] \\
Y \ar[r]^{t'} & Z}$$
in the $\infty$-category $\calC$. 
Lemma \ref{sugarplaces} implies the existence of a homotopy equivalence
$\bHom_{\calC_{A/}}(t,g) \simeq \bHom_{\calC_{Y/}}(t', \id_{Y})$. It will therefore suffice to prove that $\bHom_{\calC_{Y/}}(t', \id_{Y})$ is contractible. Since $t'$ is a pushout of $t$, it belongs to $S$. Let
$\sigma$ be a $2$-simplex of $\calC$ classifying a diagram
$$ \xymatrix{ & Y \ar[dr]^{t'} & \\
X \ar[ur]^{s} \ar[rr]^{s'} & & Z, }$$
so that $s'$ is a composition of the morphisms $s$ and $t'$ in $\calC$, and therefore also belongs to $S$.
Applying Lemma \ref{sugarplace} again, we may identify
$$\bHom_{\calC_{Y/}}(t', \id_{C'}) \simeq \bHom_{\calC_{s/}}(\sigma, s_1(s))$$ with the homotopy fiber of the map
$$ \bHom_{ \calC_{Y/}}(s',s) \rightarrow \bHom_{ \calC_{Y/}}(s,s).$$
given by composition with $\sigma$.
By construction, $\calD_{X}$ is a full subcategory of $\calC^{X/}$ which contains
$s$ and $s'$, and $s$ is a final object of $\calD_{X}$.
In view of the equivalence of $\calC_{X/}$ with $\calC^{X/}$, we conclude that the spaces
$\bHom_{\calC_{X/}}(s',s)$ and $\bHom_{\calC_{X/}}(s,s)$ are contractible, so that
$\phi$ is a homotopy equivalence as desired.
\end{proof}

\begin{proof}[Proof of Proposition \ref{nir}]
Let $h: X \rightarrow Z$ be a morphism in $\calC$; we wish to show that $h$ admits a factorization
$$ \xymatrix{ & Y \ar[dr]^{g} & \\
X \ar[ur]^{f} \ar[rr]^{h} & & Z }$$
where $f \in S$ and $g \in S^{\perp}$. Using Remark \ref{spack}, we deduce that a morphism
$g: Y \rightarrow Z$ belongs to $S^{\perp}$ if and only if it is an $S_{Z}$-local object of
$\calC_{/Z}$, where $S_Z$ is defined as in Lemma \ref{swunl}. The existence of $h$ then follows from
Lemma \ref{superlocal}.
\end{proof}


\subsection{Truncated Objects}\label{truncintro}

Let $X$ be a topological space. The first step in the homotopy-theoretic analysis of the space $X$ is to divide $X$ into path components. The situation can be described as follows: we associate to $X$ a set $\pi_0 X$, which we may view as a discrete topological space. There is a map $f: X \rightarrow \pi_0 X$ which collapses each component of $X$ to a point. If $X$ is a sufficiently nice space (for example, a CW complex), then the path components of $X$ are open, so $f$ is continuous. Moreover, $f$ is universal
among continuous maps from $X$ into a discrete topological space.

The next step in the analysis of $X$ is to consider its fundamental group $\pi_1 X$, which (provided that $X$ is sufficiently nice) may be studied by means of a universal cover $\widetilde{X}$ of $X$.
However, it is important to realize that neither $\pi_1 X$ nor $\widetilde{X}$ is invariantly associated to $X$: both require a choice of base point. The situation can be described more canonically as follows: to $X$ we can associate a {\em fundamental groupoid} $\pi(X)$, and a map $\phi$ from
$X$ to the classifying space $B \pi(X)$. The universal cover $\widetilde{X}$ of $X$ can be identified
(up to homotopy equivalence) with the homotopy fibers of the map $\phi$. The classifying space $B \pi(X)$ can be regarded as a ``quotient'' of $X$, obtained by killing all of the higher homotopy groups of $X$. Like $\pi_0 X$, it can be described by a universal mapping property. 

To continue the analysis, we first recall that a space $Y$ is said to be {\it $k$-truncated} if the homotopy groups of $Y$ vanish in dimensions larger than $k$ (see Definition \ref{trunckan}). 
Every (sufficiently nice) topological space $X$ admits an essentially unique {\em Postnikov tower}
$$ X \rightarrow \ldots \rightarrow \tau_{\leq n} X \rightarrow \ldots \rightarrow \tau_{\leq -1} X$$
where $\tau_{\leq i} X$ is $i$-truncated, and is universal (in a suitable homotopy-theoretic sense) among $i$-truncated spaces which admit a map from $X$. For example, we can take
$\tau_{\leq 0} X = \pi_0 X$, considered as a discrete space, and $\tau_{\leq 1} X = B \pi(X)$.
Moreover, we can recover the space $X$ (up to weak homotopy equivalence) by
taking the homotopy limit of the tower.\index{gen}{Postnikov tower}\index{gen}{tower!Postnikov}

The objective of this section is to construct an analogous theory in the case where $X$ is not a space, but an object of some (abstract) $\infty$-category $\calC$. We begin by observing that the condition that a space $X$ is $k$-truncated can be reformulated in more categorical terms:
a Kan complex $X$ is $k$-truncated if and only if, for every simplicial set $S$, the mapping space
$\bHom_{\sSet}(S,X)$ is $k$-truncated. This motivates the following:

\begin{definition}\label{tooka}
Let $\calC$ be an $\infty$-category and $k \geq -1$ an integer. We will say that an object
$C$ of $\calC$ is {\it $k$-truncated} if, for every object $D \in \calC$, the
space $\bHom_{\calC}(D,C)$ is $k$-truncated. By convention, we will say that $C$ is {\it $(-2)$-truncated} if it is a final object of $\calC$. We will say that an object of $\calC$ is {\it discrete} if it is $0$-truncated. We will generally let $\tau_{\leq k} \calC$ denote the full subcategory of $\calC$ spanned by the $k$-truncated objects.\index{gen}{truncated!object of an $\infty$-category}\index{gen}{$k$-truncated!object of an $\infty$-category}\index{gen}{discrete}
\end{definition}

\begin{notation}
Let $\calC$ be an $\infty$-category. Using Propositions \ref{tokerp} and \ref{huka}, we conclude that the full subcategory $\tau_{\leq 0} \calC$ is equivalent to the nerve of its homotopy category. We will denote this homotopy category by $\Disc(\calC)$, and refer to it as the {\it category of discrete objects of $\calC$}.\index{not}{DiscC@$\Disc(\calC)$}
\end{notation}

\begin{lemma}
Let $C$ be an object of an $\infty$-category $\calC$, and let $k \geq -2$. The following conditions
are equivalent:
\begin{itemize}
\item[$(1)$] The object $C$ is $k$-truncated.
\item[$(2)$] For every $n \geq k+3$ and every diagram
$$ \xymatrix{ \bd \Delta^n \ar@{^{(}->}[d] \ar[r]^{f} & \calC \\
\Delta^n \ar@{-->}[ur], & }$$ for which $f$ carries the final vertex of $\Delta^n$ to
$C$, there exists a dotted arrow rendering the diagram commutative.
\end{itemize}
\end{lemma}

\begin{proof}
Suppose first that $(2)$ is satisfied. Then for every object $D \in \calD$, the Kan complex
$\Hom^{\rght}_{\calC}(D,C)$ and every $n \geq k+3$
has the extension property with respect to $\bd \Delta^{n-1} \subseteq \Delta^{n-1}$, and is therefore $k$-truncated. For the converse, suppose that $(1)$ is satisfied and choose a categorical equivalence $g: \calC \rightarrow \tNerve \calD$, where $\calD$ is a topological category.
According to Proposition \ref{princex}, it will suffice to show that for every $n \geq k+3$ and every diagram $$ \xymatrix{ | \sCoNerve[\bd \Delta^n] | \ar@{^{(}->}[d] \ar[r]^{F} & \calD \\
| \sCoNerve[\Delta^n] | \ar@{-->}[ur], & }$$
having the property that $F$ carries the final object of $| \sCoNerve[\Delta^n] |$ to
$g(C)$, there exists a dotted arrow as indicated, rendering the diagram commutative. Let $D \in \calD$ denote the image of the initial object of $\bd \Delta^n$ under $F$. Then constructing the desired extension is equivalent to extending a map $\bd [0,1]^{n-1} \rightarrow \bHom_{\calD}( D, g(C))$ to a map defined on all of $[0,1]^{n-1}$, which is possibly in virtue of the assumption $(1)$.
\end{proof}

\begin{remark}\label{humpter}
A Kan complex $X$ is $k$-truncated if and only if it is $k$-truncated when regarded as an object in the $\infty$-category $\SSet$ of spaces.
\end{remark}

\begin{proposition}\label{altum}
Let $\calC$ be an $\infty$-category, and $k \geq -2$ an integer. The full subcategory
$\tau_{\leq k} \calC \subseteq \calC$ of $k$-truncated objects is stable under all limits
which exist in $\calC$.\index{not}{taukC@$\tau_{\leq k} \calC$}
\end{proposition}

\begin{proof}
Let $j: \calC \rightarrow \calP(\calC)$ be the Yoneda embedding. By definition,
$\tau_{\leq k} \calC$ is the preimage of $\Fun(\calC^{op}, \tau_{\leq k} \SSet)$ under
$j$. Since $j$ preserves all limits which exist in $\calC$, it will suffice to prove that
$\Fun(\calC^{op},\tau_{\leq k} \SSet) \subseteq \Fun(\calC^{op},\SSet)$ is stable under limits.
Using Proposition \ref{limiteval}, it suffices to prove that the inclusion $i: \tau_{\leq k} \SSet \subseteq \SSet$ is stable under limits. In other words, we must show that $\tau_{\leq k} \SSet$
admits small limits, and that $i$ preserves small limits. According to Propositions \ref{alllimits} and \ref{allimits}, it will suffice to show that $\tau_{\leq k} \SSet \subseteq \SSet$ is stable
under the formation of pullbacks and small products. According to Theorem \ref{colimcomparee}, this is equivalent to the assertion that the full subcategory of $\Kan$ spanned by the $k$-truncated Kan complexes is stable under homotopy products and the formation of homotopy pullback squares. Both assertions can be verified easily by computing homotopy groups.
\end{proof}

\begin{remark}\label{socal}
Let $p: \calC \rightarrow \calD$ be a coCartesian fibration of $\infty$-categories.
Let $C$ and $C'$ be objects of $\calC$, let $f: p(C') \rightarrow p(C)$ be a morphism in $\calC$,
and let $\overline{f}: C' \rightarrow C''$ be a $p$-coCartesian morphism lifting $f$.
According to Proposition \ref{compspaces}, we may identify $\bHom_{\calC_{p(C)}}(C'',C)$ with the homotopy fiber of $\bHom_{\calC}(C',C) \rightarrow \bHom_{\calD}(p(C'), p(C))$ over the base point determined by $f$.
By examining the associated long exact sequences of homotopy groups (as $f$ varies), we conclude that if $C$ is a $k$-truncated object of the fiber $\calC_{p(C)}$ and 
$p(C)$ is a $k$-truncated object of $\calD$, then $C$ is a $k$-truncated object of $\calC$. This can be considered as a generalization of Lemma \ref{sabreto} (which treats the case $k=-2$).
\end{remark}

\begin{remark}\label{trumble}
Let $p: \calM \rightarrow \Delta^1$ be a coCartesian fibration of simplicial sets, which we regard as
a correspondence from the $\infty$-category $\calC = p^{-1} \{0\}$ to $\calD = p^{-1} \{1\}$. 
Suppose that $D$ is a $k$-truncated object of $\calD$. Remark \ref{socal} implies that $D$ is a $k$-truncated object of $\calM$. Let $C,C' \in \calC$, and let $f: C \rightarrow D$ be a $p$-Cartesian morphism of $\calM$. Then composition with $f$ induces a homotopy equivalence
$ \bHom_{\calC}(C',C) \rightarrow \bHom_{\calM}(C',D)$; we conclude that $C$ is a $k$-truncated object of $\calM$.
\end{remark}

\begin{definition}\label{sugarcoat}\index{gen}{$k$-truncated!map between Kan complexes}\index{gen}{truncated!map between Kan complexes}\index{gen}{$k$-truncated!morphism in an $\infty$-category}\index{gen}{truncated!morphism in an $\infty$-category}
We will say that a map $f: X \rightarrow Y$ of Kan complexes is {\it $k$-truncated} if the homotopy
fibers of $f$ (taken over any base point of $Y$) are $k$-truncated. We will say that a morphism
$f: C \rightarrow D$ in an arbitrary $\infty$-category $\calC$ is {\it $k$-truncated}
if composition with $f$ induces a $k$-truncated map
$\bHom_{\calC}(E,C) \rightarrow \bHom_{\calC}(E,D)$
for every object $E \in \calC$.
\end{definition}

\begin{remark}
There is an apparent potential for ambiguity in Definition \ref{sugarcoat} in the case where
$\calC$ is an $\infty$-category whose objects are Kan complexes. However, there is no cause for concern: a map $f: X \rightarrow Y$ of Kan complexes is $k$-truncated if and only if it is $k$-truncated as a morphism in the $\infty$-category $\SSet$.
\end{remark}

\begin{remark}
Let $f: C \rightarrow D$ and $g: E \rightarrow D$ be morphisms in an $\infty$-category $\calC$,
and let $\phi: \bHom_{\calC}(E,C) \rightarrow \bHom_{\calC}(E,D)$ be the map (in the homotopy category $\calH$) given by compostion with $f$. Lemma \ref{sugarplace} implies that the homotopy fiber of $\phi$ over $g$ is homotopy equivalent to 
$\bHom_{\calC_{/D}}(f,g)$. Consequently, we deduce that $f: C \rightarrow D$ is $k$-truncated in the sense of Definition \ref{sugarcoat} if and only if it is $k$-truncated when viewed as an object of the $\infty$-category $\calD_{/D}$.
\end{remark}

\begin{lemma}\label{truncslice}
Let $p: \calC \rightarrow \calD$ be a right fibration of $\infty$-categories, and let
$f: X \rightarrow Y$ be a morphism in $\calC$. Then $f$ is $n$-truncated if and only if
$p(f): p(X) \rightarrow p(Y)$ is $n$-truncated.
\end{lemma}

\begin{proof}
The map $\calC_{/Y} \rightarrow \calD_{/p(Y)}$ is
a trivial fibration, and therefore an equivalence of $\infty$-categories.
\end{proof}

\begin{remark}\label{tunc}
A morphism $f: C \rightarrow D$ in an $\infty$-category $\calC$ is  $k$-truncated if and only if it is $k$-truncated when regarded as an object of the $\infty$-category $\calC^{/D}$ (since the natural
map $\calC_{/D} \rightarrow \calC^{/D}$ is an equivalence of $\infty$-categories). We may identify $\calC^{/D}$ with $p^{-1} \{D \}$, where $p$ denotes the evaluation map
$\calC^{\Delta^1} \rightarrow \calC^{\{1\}}$. Corollary \ref{tweezegork} implies that $p$ is a coCartesian fibration. Consequently, Remark \ref{trumble} translates into the following assertion: if
$$ \xymatrix{ C' \ar[r]^{f'} \ar[d] & D' \ar[d] \\
C \ar[r]^{f} & D }$$
is a pullback diagram in $\calC$, and $f$ is $k$-truncated, then $f'$ is $k$-truncated.
\end{remark}

\begin{example}
A morphism $f: C \rightarrow D$ in an $\infty$-category $\calC$ is $(-2)$-truncated if and only if it is an equivalence.
\end{example}

We will say that a morphism $f: C \rightarrow D$ is a {\it monomorphism} if it is $(-1)$-truncated; this is equivalent to the assertion that the functor $\calC_{/f} \rightarrow \calC_{/D}$ is fully faithful.\index{gen}{monomorphism}

\begin{lemma}\label{trunccomp}
Let $\calC$ be an $\infty$-category and $f: X \rightarrow Y$ a morphism in $\calC$.
Suppose that $Y$ is $n$-truncated. Then $X$ is $n$-truncated if and only if $f$ is $n$-truncated.
\end{lemma}

\begin{proof}
Unwinding the definitions, we reduce immediately to the following statement in classical homotopy theory: given a map $f: X \rightarrow Y$ of Kan complexes, where $Y$ is $n$-truncated, $X$ is $n$-truncated if and only if the homotopy fibers of $f$ are $n$-truncated. This can be established easily, by examining the long exact sequence of homotopy groups associated to $f$.
\end{proof}

The following lemma gives a recursive characterization of the class of $n$-truncated morphisms:

\begin{lemma}\label{trunc}
Let $\calC$ be an $\infty$-category which admits finite limits and let $k
\geq -1$ be an integer. A morphism $f: C \rightarrow C'$ is $k$-truncated if and
only if the diagonal $\delta: C \rightarrow C \times_{C'} C$ (which is well-defined up to homotopy)
is $(k-1)$-truncated.
\end{lemma}

\begin{proof}
For each object $D \in \calC$, let $F_{D}: \calC \rightarrow \SSet$ denote the 
functor co-represented by $D$. Then each $F_{D}$ preserves finite limits, and
a morphism $f$ in $\calC$ is $k$-truncated if and only if each $F_D(f)$ is a $k$-truncated morphism in $\SSet$. We may therefore reduceo to the case where $\calC = \SSet$. Without loss of generality, we may suppose that $f: C \rightarrow C'$ is a Kan fibration. Then Theorem \ref{colimcomparee} allows us to identify the fiber product $C \times_{C'} C$ in $\SSet$ with the same fiber product, formed in the ordinary category $\Kan$. We now reduce to the following assertion in classical homotopy theory (applied to the fibers of $f$): if $X$ is a Kan complex, then $X$ is $k$-truncated if and only if the homotopy fibers of the diagonal map $X \rightarrow X \times X$ are $(k-1)$-truncated. This can be proven readily by examining homotopy groups.
\end{proof}

We immediately deduce the following:

\begin{proposition}\label{eaa}
Let $F: \calC \rightarrow \calC'$ be a left-exact functor between
$\infty$-categories which admit finite limits. Then $F$ carries
$k$-truncated objects into $k$-truncated objects and $k$-truncated
morphisms into $k$-truncated morphisms.
\end{proposition}

\begin{proof}
An object $C$ is $k$-truncated if and only if the morphism $C
\rightarrow 1$ to the final object is $k$-truncated. Since $F$
preserves final object, it suffices to prove the assertion
concerning morphisms. Since $F$ commutes with fiber products,
Lemma \ref{trunc} allows us to use induction on $k$, thereby
reducing to the case where $k=-2$. But the $(-2)$-truncated
morphisms are precisely the equivalences, and these are preserved
by any functor.
\end{proof}

We now specialize to the case of a {\em presentable} $\infty$-category $\calC$. In this setting, we can construct an analogue of the Postnikov tower.

\begin{lemma}\label{slurm}
Let $X$ be a Kan complex, and let $k \geq -2$. The following conditions are equivalent:
\begin{itemize}
\item[$(1)$] The Kan complex $X$ is $k$-truncated.
\item[$(2)$] The diagonal map $\delta: X \rightarrow X^{ \bd \Delta^{k+2} }$ is a homotopy equivalence.
\end{itemize}
\end{lemma}

\begin{proof}
If $k=-2$, then $X^{ \bd \Delta^{k+2} }$ is a point and the assertion is obvious. Assuming $k > -2$,
we can choose a vertex $v$ of $\bd \Delta^{k+2}$, which gives rise to an evaluation map
$e: X^{ \bd \Delta^{k+2} } \rightarrow X$. Since $e \circ \delta = \id_{X}$, $(2)$ is equivalent
to the assertion that $e$ is a homotopy equivalence. We observe that $e$ is a Kan fibration. For each $x$, let $Y_{x} = X^{ \bd \Delta^{k+2} } \times_{X} \{x\}$ denote the fiber of $e$ over the vertex $x$. Then $Y_x$ has a canonical base point, given by the constant map
$\delta(x)$. Moreover, we have a natural isomorphism
$$ \pi_{i}(Y_x, \delta(x) ) \simeq \pi_{i+k+1}(X,x).$$
Condition $(1)$ is equivalent to the assertion that $\pi_{i+k+1}(X,x)$ vanishes for
all $i \geq 0$ and all $x \in X$. In view of the above isomorphism, this is equivalent to the assertion that each $Y_{x}$ is contractible, which is true if and only if the Kan fibration $e$ is trivial.
\end{proof}

\begin{proposition}\label{maketrunc}
Let $\calC$ be a presentable $\infty$-category, $k \geq -2$. Let $\tau_{\leq k} \calC$
denote the full subcategory of $\calC$ spanned by the $k$-truncated objects.
Then the inclusion $\tau_{\leq k} \calC \subseteq \calC$ has an accessible left adjoint, which we will denote by $\tau_{\leq k}$.\index{not}{tauk@$\tau_{\leq k}$}
\end{proposition}

\begin{proof}
Let $f: \bd \Delta^{k+2} \rightarrow \Fun(\calC,\calC)$ denote the constant diagram
taking the value $\id_{\calC}$. Let $\overline{f}: (\bd \Delta^{k+2})^{\triangleright} \rightarrow \Fun(\calC,\calC)$ be a colimit of $f$, and let $F: \calC \rightarrow \calC$ be the image of the cone point under $\overline{f}$. Informally, $F$ is given by the formula
$$ C \mapsto C \otimes S^{k+1},$$
where $S^{k+1}$ denotes the $(k+1)$-sphere and we regard $\calC$ as tensored over spaces (see Remark \ref{tensored}).

Let $\overline{f}': (\bd \Delta^{k+2})^{\triangleright} \rightarrow \calC^{\calC}$ be the constant diagram taking the value $\id_{\calC}$. It follows that there exists an essentially unique map $\overline{f} \rightarrow \overline{f'}$ in $(\calC^{\calC})_{f/}$, which induces a natural transformation of functors $\alpha: F \rightarrow \id_{\calC}$. Let $S = \{ \alpha(C): C \in \calC \}$.
Since $F$ is a colimit of functors which preserve small colimits, $F$ itself preserves small colimits (Lemma \ref{limitscommute}). Applying Proposition \ref{limiteval}, we conclude that
$\alpha: \calC \rightarrow \Fun(\Delta^1,\calC)$ also preserves small colimits. Consequently, there
exists a small subset $S_0 \subseteq S$ which generates $S$ under colimits in $\Fun(\Delta^1,\calC)$.
According to Proposition \ref{local}, the collection of $S$-local objects of $\calC$ is an accessible localization of $\calC$. It therefore suffices to prove that an object $X \in \calC$ is $S$-local if and only if $X$ is $k$-truncated.

According to Proposition \ref{limiteval}, for each $C \in \calC$ we may identify $F(C)$
with the colimit of the constant diagram $\bd \Delta^{k+2} \rightarrow \calC$ taking the value $C$. Corollary \ref{charext} implies that we have a homotopy equivalence
$$ \bHom_{\calC}( F(C), X) \simeq \bHom_{\calC}(C,X)^{ \bd \Delta^{k+2} }.$$
The map $\alpha(C)$ induces a map
$$ \alpha(C)_{X}: \bHom_{\calC}(C,X) \rightarrow \bHom_{\calC}(C,X)^{ \bd \Delta^{k+2} }$$ which can be identified with the inclusion of $\bHom_{\calC}(C,X)$ as the space of constant maps from
$\bd \Delta^{k+2}$ into $\bHom_{\calC}(C,X)$. According to Lemma \ref{slurm}, the map
$\alpha(C)_{X}$ is an equivalence if and only if $\bHom_{\calC}(C,X)$ is $k$-truncated.
Thus $X$ is $k$-truncated if and only if $X$ is $S$-local. 
\end{proof}

\begin{remark}
The notation of Proposition \ref{maketrunc} is self-consistent, in the sense that the existence
of the localization functor $\tau_{\leq k}$ implies that the collection of $k$-truncated objects of $\calC$ may be identified with the essential image of $\tau_{\leq k}$.
\end{remark}

\begin{remark}
If the $\infty$-category $\calC$ is potentially unclear in
context, then we will write $\tau^{\calC}_{\leq k}$ for the truncation
functor in $\calC$. Note also that $\tau^{\calC}_{\leq k}$ is well-defined up to equivalence (in fact, up to a contractible ambiguity).\index{not}{tau^Ck@$\tau^{\calC}_{\leq k}$}
\end{remark}

\begin{remark}
It follows from Proposition \ref{maketrunc} that if $\calC$ is a presentable $\infty$-category, then the full subcategory $\tau_{\leq k} \calC$ of $k$-truncated objects is also presentable. In particular,
the ordinary category $\Disc(\calC)$ of discrete objects of $\calC$ is a presentable
category in the sense of Definition \ref{catpor}.
\end{remark}

Recall that, if $\calC$ and $\calD$ are $\infty$-categories, then $\LFun(\calC, \calD)$ denotes the full subcategory of $\Fun(\calC,\calD)$ spanned by those functors which are left adjoints.
The following result gives a characterization of $\tau_{\leq n} \calC$ by a universal mapping property:

\begin{corollary}\label{truncprop}
Let $\calC$ and $\calD$ be presentable $\infty$-categories. Suppose that $\calD$ is equivalent to an $(n+1)$-category. Then composition with $\tau_{\leq n}$ induces an equivalence
$s: \LFun( \tau_{\leq n} \calC, \calD) \rightarrow \LFun(\calC, \calD).$
\end{corollary}

\begin{proof}
According to Proposition \ref{unichar}, the functor $s$ is fully faithful. A functor $f: \calC \rightarrow \calD$ belongs to the essential image of $s$ if and only if $f$ has a right adjoint $g$ which
factors through $\tau_{\leq n} \calC$. Since $g$ preserves limits, it automatically carries
$\calD = \tau_{\leq n} \calD$ into $\tau_{\leq n} \calC$ (Proposition \ref{eaa}).
\end{proof}

In classical homotopy theory, every space $X$ can be recovered (up to weak homotopy equivalence) as the homotopy inverse limit of its Postnikov tower $\{ \tau_{\leq n} X \}_{n \geq 0}$.
The analogous statement is not true in an arbitrary presentable $\infty$-category, but often holds in specific examples. We now make a general study of this phenomenon.

\begin{definition}\label{postit1}\index{gen}{tower}\index{gen}{pretower}\index{gen}{Postnikov tower}\index{gen}{Postnikov pretower}
Let $\Z_{\geq 0}^{\infty}$ denote the union $\Z_{\geq 0} \cup \{ \infty \}$, regarded as a linearly ordered
set with largest element $\infty$. Let $\calC$ be a presentable $\infty$-category. Recall that
a {\it tower} in $\calC$ is a functor $\Nerve( \Z_{\geq 0}^{\infty})^{op} \rightarrow \calC$, which we view as a diagram
$$ X_{\infty} \rightarrow \ldots \rightarrow X_{2} \rightarrow X_{1} \rightarrow X_{0}.$$
A {\it Postnikov tower} is a tower with the property that for each $n \geq 0$, the map
$X_{\infty} \rightarrow X_{n}$ exhibits $X_{n}$ as an $n$-truncation of $X_{\infty}$. 
We define a {\it pretower} to be a functor from $\Nerve( \Z_{\geq 0})^{op} \rightarrow \calC$.
A {\it Postnikov pretower} is a pretower
$$ \ldots \rightarrow X_{2} \rightarrow X_{1} \rightarrow X_0$$
which exhibits each $X_{n}$ as an $n$-truncation of $X_{n+1}$.
We let $\Post^{+}(\calC)$ denote the full subcategory of $\Fun( \Nerve( \Z_{\geq 0}^{\infty})^{op}, \calC)$\index{not}{Post+C@$\Post^{+}(\calC)$}\index{not}{PostC@$\Post(\calC)$}
spanned by the Postnikov towers, and $\Post(\calC)$ the full subcategory of
$\Fun( \Nerve(\Z_{\geq 0})^{op}, \calC)$ spanned by the Postnikov pretowers.
We have an evident forgetful functor
$\phi: \Post^{+}(\calC) \rightarrow \Post(\calC).$ We will say that
{\it Postnikov towers in $\calC$ are convergent} if $\phi$ is an equivalence of $\infty$-categories.
\end{definition}

\begin{remark}\label{sumptime}
Let $\calC$ be a presentable $\infty$-category, and let
$\calE$ denote the full subcategory of $\calC \times \Nerve( \Z^{\infty}_{\geq 0})^{op}$
spanned by those pairs $(C, n)$ where $C \in \calC$ is $n$-truncated (by convention, we agree that
this condition is always satisfied where $C = \infty$). Then we have a coCartesian fibration
$p: \calE \rightarrow \Nerve( \Z^{\infty}_{\geq 0})^{op}$, which classifies a tower of $\infty$-categories
$$ \calC \rightarrow \ldots \rightarrow \tau_{\leq 2} \calC \stackrel{\tau_{\leq 1}}{\rightarrow} \tau_{\leq 1} \calC
\stackrel{\tau_{\leq 0}}{\rightarrow} \tau_{\leq 0} \calC.$$
We can identify Postnikov towers with coCartesian sections of $p$ (and Postnikov pretowers
with coCartesian sections of the induced fibration
$\calE \times_{ \Nerve( \Z^{\infty}_{\geq 0})^{op}} \Nerve(\Z_{\geq 0})^{op}
\rightarrow \Nerve( \Z_{\geq 0})^{op})$.
According to Proposition \ref{charcatlimit}, Postnikov towers in $\calC$ converge if and only if
the tower above exhibits $\calC$ as the homotopy limit of the sequence of $\infty$-categories
$$ \ldots \rightarrow \tau_{\leq 2} \calC \rightarrow \tau_{\leq 1} \calC \rightarrow
\tau_{\leq 0} \calC.$$
\end{remark}

\begin{remark}\label{swit}
Let $\calC$ be a presentable $\infty$-category, and assume that Postnikov towers in $\calC$
are convergent. Then every Postnikov tower in $\calC$ is a limit diagram. Indeed, given
objects $X, Y \in \calC$, we have natural homotopy equivalences
$$ \bHom_{\calC}( X,Y) \simeq \holim \bHom_{\calC}( \tau_{\leq n} X, \tau_{\leq n} Y)
\simeq \holim \bHom_{\calC}( X, \tau_{\leq n} Y)$$
so that $Y$ is the limit of the pretower $\{ \tau_{\leq n} Y\}$.
\end{remark}

\begin{proposition}\label{slibe1}
Let $\calC$ be a presentable $\infty$-category. Then Postnikov towers in $\calC$ are convergent if and only if, for every tower $X: \Nerve( \Z^{\infty}_{\geq 0})^{op} \rightarrow \calC$, 
the following conditions are equivalent:
\begin{itemize}
\item[$(1)$] The diagram $X$ is a Postnikov tower.
\item[$(2)$] The diagram $X$ is a limit in $\calC$, and the restriction
$X| \Nerve( \Z_{\geq 0})^{op}$ is a Postnikov pretower.
\end{itemize}
\end{proposition}

\begin{proof}
Let $\Post'(\calC)$ be the full subcategory of $\Fun( \Nerve( \Z^{\infty}_{\geq 0})^{op}, \calC)$ spanned by those towers which satisfy condition $(2)$. Using Proposition \ref{lklk}, we deduce that the restriction functor $\Post'(\calC) \rightarrow \Post(\calC)$ is a trivial Kan fibration. If
conditions $(1)$ and $(2)$ are equivalent, then $\Post'(\calC) = \Post^{+}(\calC)$, so that
Postnikov towers in $\calC$ are convergent. Conversely, suppose that
Postnikov towers in $\calC$ are convergent. Using Remark \ref{swit}, we deduce that
$\Post^{+}(\calC) \subseteq \Post'(\calC)$, so we have a commutative diagram
$$ \xymatrix{ \Post^{+}(\calC) \ar[dr] \ar[rr] & & \Post'(\calC) \ar[dl] \\
& \Post(\calC). & }$$
Since both of the vertical arrows are trivial Kan fibrations, we conclude that the inclusion
$\Post^{+}(\calC) \subseteq \Post'(\calC)$ is an equivalence, so that
$\Post^{+}(\calC) = \Post'(\calC)$. This proves that $(1) \Leftrightarrow (2)$.
\end{proof}

\begin{remark}\label{urkan}\index{gen}{tower!highly connected}
Let $\calC$ be a presentable $\infty$-category. We will say that a tower
$X: \Nerve( \Z^{\infty}_{\geq 0})^{op} \rightarrow \calX$ is {\it highly connected} if, for every $n \geq 0$, there exists an integer $k$ such that the induced map $\tau_{\leq n} X(\infty) \rightarrow \tau_{\leq n} X(k')$ is an equivalence for $k' \geq k$. We will say that a pretower 
$Y: \Nerve( \Z_{\geq 0}) \rightarrow
\calX$ is {\it highly connected} if, for every $n \geq 0$, there exists an integer $k$ such that
the map $\tau_{\leq n} Y(k'') \rightarrow \tau_{\leq n} Y(k')$ is an equivalence for $k'' \geq k' \geq k$.\index{gen}{pretower!highly connected}
It is clear that every Postnikov (pre)tower is highly connected. Conversely, if $X$ is a highly connected tower and its underlying pretower is highly connected, then $X$ is a Postnikov tower. Indeed, for each $n \geq 0$ we can choose $k \geq n$ such that the map
$\tau_{\leq n} X(\infty) \rightarrow \tau_{\leq n} X(k)$ is an equivalence. Since $X$ is a Postnikov pretower, this induces an equivalence $\tau_{\leq n} X(\infty) \simeq X(n)$. Consequently, to establish
the implication $(2) \Rightarrow (1)$ in the criterion of Proposition \ref{slibe1}, it suffices to verify the following:
\begin{itemize}
\item[$(\ast)$] Let $X: \Nerve( \Z^{\infty}_{\geq} ) \rightarrow \calC$ be a tower in $\calC$. Assume that $X$ is a limit diagram, and that the underlying pretower is highly connected. Then $X$ is highly connected.
\end{itemize}
In \S \ref{homdim}, we will apply this criterion to prove that Postnikov towers are convergent in a large class of $\infty$-topoi.
\end{remark}

We conclude this section with a useful compatibility property between truncation functors in different $\infty$-categories:

\begin{proposition}\label{compattrunc}
Let $\calC$ and $\calD$ be presentable $\infty$-categories, and let $F: \calC \rightarrow \calD$
be a left-exact presentable functor. Then there is an equivalence of functors
$F \circ \tau_{\leq k}^{\calC} \simeq \tau_{\leq k}^{\calD} \circ F$.
\end{proposition}

\begin{proof}
Since $F$ is left exact, it restricts to a functor from $\tau_{\leq k} \calC$ to $\tau_{\leq k} \calD$ by Proposition \ref{eaa}. We therefore have a diagram
$$ \xymatrix{ \calC \ar[r]^{F} \ar[d]^{\tau_{\leq k}^{\calC} } & \calD \ar[d]^{ \tau_{\leq k}^{\calD}} \\
\tau_{\leq k} \calC \ar[r]^{F} & \tau_{\leq k} \calD }$$
which we wish to prove is commutative up to homotopy. Let $G$ denote a right adjoint to
$F$; then $G$ is left exact and so induces a functor $\tau_{\leq k} \calD
\rightarrow \tau_{\leq k} \calC$. Using Proposition \ref{compadjoint}, we can reduce to proving
that the associated diagram of right adjoints 
$$ \xymatrix{ \calC  & \calD \ar[l]_{G} \\
\tau_{\leq k} \calC \ar[u] & \tau_{\leq k} \calD \ar[u] \ar[l]_{G} }$$
commutes up to homotopy, which is obvious (since the diagram strictly commutes).
\end{proof}


\subsection{Compactly Generated $\infty$-Categories}\label{compactgen}

\begin{definition}\index{gen}{compactly generated}\index{gen}{$\kappa$-compactly generated}\label{compgen}
Let $\kappa$ be a regular cardinal. We will say that an
$\infty$-category $\calC$ is {\it $\kappa$-compactly generated} if it is presentable and $\kappa$-accessible. When $\kappa = \omega$, we will simply say that $\calC$ is {\it compactly generated}.
\end{definition}

The proof of Theorem \ref{pretop} shows that an $\infty$-category $\calC$ is $\kappa$-compactly generated if and only if there exists a small $\infty$-category $\calD$ which admits $\kappa$-small colimits, and an equivalence $\calC \simeq \Ind_{\kappa}(\calD)$. In fact, we can choose $\calD$ to be (a minimal model of)
the $\infty$-category of $\kappa$-compact objects of $\calC$. We would like to assert that this construction establishes an equivalence between two sorts of $\infty$-categories. In order to make this precise, we need to introduce the appropriate notion of functor between $\kappa$-compactly generated $\infty$-categories.

\begin{proposition}\label{comppress}
Let $\kappa$ be a regular cardinal, and let $\Adjoint{F}{\calC}{\calD}{G}$ be a pair of adjoint functors, where $\calC$ and $\calD$ admit small, $\kappa$-filtered colimits.
\begin{itemize}
\item[$(1)$] If $G$ is $\kappa$-continuous, then $F$ carries $\kappa$-compact objects of $\calC$ to $\kappa$-compact objects of $\calD$.
\item[$(2)$] Conversely, if $\calC$ is $\kappa$-accessible and $F$ preserves $\kappa$-compactness, then $G$ is $\kappa$-continuous.
\end{itemize}
\end{proposition}

\begin{proof}
Suppose first that $G$ is $\kappa$-continuous, and let $C \in \calC$ be a $\kappa$-compact object. Let $e: \calC \rightarrow \widehat{\SSet}$ be a functor corepresented by $C$. Then
$e \circ G: \calD \rightarrow \widehat{\SSet}$ is corepresented by $F(C)$. Since $e$ and $G$
are $\kappa$-continuous, so is $e \circ G$; this proves $(1)$.

Conversely, suppose that $F$ preserves $\kappa$-compact objects and that $\calC$ is $\kappa$-accessible. Without loss of generality, we may suppose that there is a small $\infty$-category $\calC'$ such that $\calC = \Ind_{\kappa}(\calC') \subseteq \calP(\calC')$. We wish to prove
that $G$ is $\kappa$-continuous. Since $\Ind_{\kappa}(\calC')$ is stable under $\kappa$-filtered colimits in $\calP(\calC')$, it will suffice to prove that the composite map
$$ \theta: \calD \stackrel{G}{\rightarrow} \calC \subseteq \calP(\calC') $$
is $\kappa$-continuous. In view of Proposition \ref{limiteval}, it will suffice to prove that
for every object $C \in \calC'$, the composition of $\theta$ with evaluation at $C$ is
a $\kappa$-continuous functor. We conclude by observing that this functor is corepresentable
by the image under $F$ of $j(C) \in \calC$ (here $j: \calC' \rightarrow \Ind_{\kappa}(\calC)$ denotes the Yoneda embedding). 
\end{proof}

\begin{corollary}\label{starmin}
Let $\calC$ be a $\kappa$-compactly generated $\infty$-category, and let
$L: \calC \rightarrow \calC$ be a localization functor. The following conditions are equivalent:
\begin{itemize}
\item[$(1)$] The functor $L$ is $\kappa$-continuous.
\item[$(2)$] The full subcategory $L \calC \subseteq \calC$ is stable under $\kappa$-filtered colimits.
\end{itemize}
Suppose that these conditions are satisfied. Then:
\begin{itemize}
\item[$(3)$] The functor $L$ carries $\kappa$-compact objects of $\calC$ to $\kappa$-compact objects
of $L \calC$.

\item[$(4)$] The $\infty$-category $L \calC$ is $\kappa$-compactly generated.

\item[$(5)$] An object $D \in L \calC$ is $\kappa$-compact (in $L \calC$) if and only if
there exists a compact object $C \in \calC$ such that $D$ is a retract of $LC$.
\end{itemize}

\end{corollary}

\begin{proof}
Suppose that $(1)$ is satisfied. Let $p: K \rightarrow L \calC$ be a $\kappa$-filtered diagram. Then the natural transformation $p \rightarrow Lp$ is an equivalence. Using $(1)$, we conclude that
the induced map $\varinjlim(p) \rightarrow L \varinjlim(p)$ is an equivalence, so that
$\varinjlim(p) \in L \calC$. This proves $(2)$.

Conversely, if $(2)$ is satisfied, then the inclusion $L \calC \subseteq \calC$ is $\kappa$-continuous, so that $L: \calC \rightarrow \calC$ is a composition of $\kappa$-continuous functors
$$ \calC \stackrel{L}{\rightarrow} L \calC \rightarrow \calC,$$
which proves $(1)$.

Assume that $(1)$ and $(2)$ are satisfied. Then $L$ is accessible, so that
$L \calC$ is a presentable $\infty$-category. 
Assertion $(3)$ follows from Proposition \ref{comppress}.
Let $D \in L \calC$. Since $\calC$ is $\kappa$-compactly generated, 
$D$ can be written as the colimit of a $\kappa$-filtered diagram $p: K \rightarrow \calC$ taking values in the $\kappa$-compact objects of $\calC$. Then $D \simeq LD$ can be written
as the colimit of $L \circ p$, which takes values $\kappa$-compact objects of $L \calC$. This proves $(4)$. If $D$ is a $\kappa$-compact object of $\calD$, then we deduce that the identity map
$\id_{D}: D \rightarrow D$ factors through $(L \circ p)(k)$ for some vertex $k \in K$, which proves $(5)$.
\end{proof}

\begin{corollary}\label{hunterygreen}
Let $\calC$ be a $\kappa$-compactly generated $\infty$-category, and let
$n \geq -2$. Then:
\begin{itemize}
\item[$(1)$] The full subcategory $\tau_{\leq n} \calC$ is stable under $\kappa$-filtered colimits in $\calC$.
\item[$(2)$] The truncation functor $\tau_{\leq n}: \calC \rightarrow \calC$ is $\kappa$-continuous.
\item[$(3)$] The truncation functor $\tau_{\leq n}$ carries compact objects of $\calC$ to compact objects of $\calC_{\leq n}$. 
\item[$(4)$] The full subcategory $\tau_{\leq n} \calC$ is $\kappa$-compactly generated.
\item[$(5)$] An object $C \in \tau_{\leq n} \calC$ is compact (in $\tau_{ \leq n} \calC$) if and only if there exists a compact object $C' \in \calC$ such that $C$ is a retract of $\tau_{\leq n} C'$.
\end{itemize}
\end{corollary}

\begin{proof}
Corollary \ref{starmin} shows that condition $(1)$ implies $(2)$, $(3)$, $(4)$, and $(5)$.
Consequently, it will suffice to prove that $(1)$ is satisfied. 

Let $C$ be an object of $\calC$. We will show that $C$ is $n$-truncated if and only if the space
$\bHom_{\calC}(D, C)$ is $n$-truncated, for every $\kappa$-compact object $D \in \calC$. 
The ``only if'' direction is obvious. For the converse, let $F_{C}: \calC^{op} \rightarrow \SSet$ be the functor represented by $C$, and let $\calC' \subseteq \calC$ be the full subcategory of $\calC$ spanned by those objects $D$ such that $F_{C}(D)$ is $n$-truncated. Since $F_{C}$ preserves limits, $\calC'$ is stable under colimits in $\calC$. If $\calC'$ contains every $\kappa$-compact object of $\calC$, then $\calC' = \calC$ (since $\calC$ is $\kappa$-compactly generated).

Now suppose that $D$ is a $\kappa$-compact object of $\calC$, let
$G_{D}: \calC \rightarrow \SSet$ be the functor co-represented by $D$, and let
$\calC(D) \subseteq \calC$ be the full subcategory of $\calC$ spanned by those objects
$C$ for which $G_{D}(C)$ is $n$-truncated. Then $\tau_{\leq n} \calC = \bigcap_{D} \calC(D)$. To complete the proof, it will suffice to show that each $\calC(D)$ is stable under $\kappa$-filtered colimits. Since $G_{D}$ is $\kappa$-continuous, it suffices to observe that
$\tau_{\leq n} \SSet$ is stable under $\kappa$-filtered colimits in $\SSet$.
\end{proof}


\begin{definition}\index{not}{PresRkappa@$\RRPres{\kappa}$}
If $\kappa$ is a regular cardinal, we let $\RRPres{\kappa}$ denote the full subcategory
of $\widehat{\Cat}_{\infty}$ whose objects are $\kappa$-compactly generated $\infty$-categories, and whose morphisms are $\kappa$-continuous, limit-preserving functors.
\end{definition}

\begin{proposition}\label{cnote}\index{gen}{limit!of compactly generated $\infty$-categories}
The $\infty$-category $\RRPres{\kappa}$ admits small limits, and the inclusion
$\RRPres{\kappa} \subseteq \widehat{\Cat}_{\infty}$ preserves small limits.
\end{proposition}

\begin{proof}
In view of Theorem \ref{surbus}, the only nontrivial point is to verify that if $p: K \rightarrow \RRPres{\kappa}$ is a diagram of $\kappa$-compactly generated $\infty$-categories $\{ \calC_{\alpha} \}$, then the limit $\calC = \varprojlim(p)$ in $\widehat{\Cat}_{\infty}$ is $\kappa$-compactly generated. In other words, we must show that $\calC$ is generated under colimits by its $\kappa$-compact objects.

For each vertex $\alpha$ of $K$, let 
$$ \Adjoint{ F_{\alpha} }{\calC_{\alpha}}{\calC}{G_{\alpha} }$$ denote the corresponding adjunction. Lemma \ref{steakknife} implies that the identity functor $\id_{\calC}$ can be obtained as the colimit of a diagram $q: K \rightarrow \Fun(\calC, \calC)$, where $q(\alpha) \simeq F_{\alpha} \circ G_{\alpha}$. In particular, $\calC$ is generated (under small colimits) by the essential images of the functors $F_{\alpha}$. Since each $\calC_{\alpha}$ is generated under colimits by $\kappa$-compact objects, and the functors $F_{\alpha}$ preserve colimits and $\kappa$-compact objects (Proposition \ref{comppress}), we conclude that $\calC$ is generated under colimits by its $\kappa$-compact objects, as desired.
\end{proof} 

\begin{notation}\label{funnote}\index{not}{PresLkappa@$\LLPres{\kappa}$}
Let $\kappa$ be a regular cardinal. We let $\LLPres{\kappa}$ denote the full subcategory of
$\widehat{\Cat}_{\infty}$ whose objects are $\kappa$-compactly generated $\infty$-categories, and whose morphisms are functors which preserve small colimits and $\kappa$-compact objects. In view of Proposition \ref{comppress}, the equivalence
$ \LPres \simeq (\RPres)^{op}$ of Corollary \ref{warhog} restricts to an equivalence
$ \LLPres{\kappa} \simeq (\RRPres{\kappa})^{op}$.

Let $\widehat{\Cat}_{\infty}^{\Rex{\kappa}}$ denote the subcategory of $\widehat{\Cat}_{\infty}$ whose objects are $\infty$-categories which admit $\kappa$-small colimits, and whose morphisms are functors which preserve $\kappa$-small colimits, and let $\Cat_{\infty}^{\Rex{\kappa}} = \widehat{\Cat}_{\infty}^{\Rex{\kappa}} \cap \Cat_{\infty}$.\index{not}{catrexhat@$\widehat{\Cat}_{\infty}^{\Rex{\kappa}}$}\index{not}{catrex@$\Cat_{\infty}^{\Rex{\kappa}}$}
\end{notation}

\begin{proposition}\label{suchy}
Let $\kappa$ be a regular cardinal, and let
$$\theta: \LLPres{\kappa} \rightarrow \widehat{\Cat}_{\infty}^{\Rex{\kappa}}$$ be the nerve of the simplicial functor which associates to a $\kappa$-compactly generated $\infty$-category 
$\calC$ the full subcategory $\calC^{\kappa} \subseteq \calC$ spanned by the $\kappa$-compact objects of $\calC$. Then:
\begin{itemize}
\item[$(1)$] The functor $\theta$ is fully faithful. 
\item[$(2)$] The essential image of $\theta$ consists precisely of those objects of $\widehat{\Cat}_{\infty}$ which are essentially small and idempotent complete. 
\end{itemize}
\end{proposition}

\begin{proof}
Combine Propositions \ref{humatch} and \ref{sumatch}.
\end{proof}

\begin{remark}
If $\kappa > \omega$, then Corollary \ref{swwe} shows that the hypothesis of idempotent completeness in $(2)$ is superfluous.
\end{remark}

The proof of Proposition \ref{lockap} yields the following analogue:

\begin{proposition}\label{sumer}
Let $\kappa$ be a regular cardinal. The functor $\Ind_{\kappa}: \Cat_{\infty} \rightarrow \Acc_{\kappa}$ exhibits $\LLPres{\kappa}$ as a localization of $\Cat_{\infty}^{\Rex{\kappa}}$. 
If $\kappa > \omega$, then $\Ind_{\kappa}$ induces an equivalence of $\infty$-categories
$\Cat_{\infty}^{\Rex{\kappa}} \rightarrow \LLPres{\kappa}$.
\end{proposition}

\begin{proof}
The only additional ingredient needed is the following observation: if $\calC$ is an $\infty$-category which admits $\kappa$-small colimits, then the idempotent completion $\calC'$ of $\calC$ also admits $\kappa$-small colimits. To prove this, we observe that $\calC'$ can be identified with the collection of $\kappa$-compact objects of $\Ind_{\kappa}(\calC)$ (Lemma \ref{stylus}). Since $\calC$ admits all small colimits (Theorem \ref{pretop}), we conclude that $\calC'$ admits $\kappa$-small colimits.
\end{proof}

We conclude with a remark about the structure of the $\infty$-category
$\Cat_{\infty}^{\Rex{\kappa}}$.

\begin{proposition}\label{unrose}\index{gen}{filtered colimit!of colimit-preserving functors}
Let $\kappa$ be a regular cardinal. Then the $\infty$-category
$\Cat_{\infty}^{\Rex{\kappa}}$ admits small, $\kappa$-filtered colimits, and the inclusion 
$\Cat_{\infty}^{\Rex{\kappa}} \subseteq \Cat_{\infty}$ preserves small $\kappa$-filtered colimits.
\end{proposition}

\begin{proof}
Let $\calI$ be a small, $\kappa$-filtered $\infty$-category, and let
$p: \calI \rightarrow \Cat_{\infty}^{\Rex{\kappa}}$ be a diagram. Let $\calC$ be a colimit of the induced diagram $\calI \rightarrow \Cat_{\infty}$. 
To complete the proof we must prove the following:
\begin{itemize}
\item[$(i)$] The $\infty$-category $\calC$ admits $\kappa$-small colimits.
\item[$(ii)$] For each $I \in \calI$, the associated functor $p(I) \rightarrow \calC$ preserves $\kappa$-small colimits.
\item[$(iii)$] Let $f: \calC \rightarrow \calD$ be an arbitrary functor. If each of the compositions
$p(I) \rightarrow \calC \rightarrow \calD$ preserves $\kappa$-small colimits, then $f$ preserves $\kappa$-small colimits.
\end{itemize}

Since $\calI$ is $\kappa$-filtered, any $\kappa$-small diagram in $\calC$ factors through one of the maps $p(I) \rightarrow \calC$ ( Proposition \ref{grapeape} ). Thus $(ii) \Rightarrow (i)$ and $(ii) \Rightarrow (iii)$. To prove $(ii)$, we first use Proposition \ref{rot} to reduce to the case where $\calI \simeq \Nerve(A)$, where $A$ is a $\kappa$-filtered partially ordered set. Using Proposition \ref{gumby444}, we can reduce to the case where $p$ is the nerve of a functor from $q: A \rightarrow \sSet$. In view of Theorem \ref{colimcomparee}, we can identify $\calC$ with a homotopy colimit of $q$. Since the collection of categorical equivalences is stable under filtered colimits, we can assume that $\calC$ is actually the filtered colimit of a family of $\infty$-categories $\{ \calC_{\alpha} \}_{\alpha \in A}$. 

Let $K$ be a $\kappa$-small simplicial set, and let $\overline{g}_{\alpha}: K^{\triangleright} \rightarrow \calC_{\alpha}$ be a colimit diagram. We wish to show that the induced map 
$\overline{g}: K^{\triangleright} \rightarrow \calC$ is a colimit diagram. Let
$g = \overline{g} |K$; we need to show that the map $\theta: \calC_{\overline{g}/} \rightarrow \calC_{g/}$ is a trivial Kan fibration. We observe that $\theta$ is a filtered colimit of maps
$\theta_{\beta}: (\calC_{\beta})_{\overline{g}_{\beta}/} \rightarrow (\calC_{\beta})_{g_{\beta}/}$, where $\beta$ ranges over the set $\{ \beta \in A : \beta \geq \alpha \}$. Using the fact that each of the associated maps $\calC_{\alpha} \rightarrow \calC_{\beta}$ preserves $\kappa$-small colimits, we conclude that each $\theta_{\beta}$ is a trivial fibration, so that $\theta$ is a trivial fibration as desired.
\end{proof}

\subsection{Nonabelian Derived Categories}\label{stable11}

According to Corollary \ref{uterrr}, we can analyze arbitrary colimits in an $\infty$-category $\calC$ in terms of finite colimits and filtered colimits. In particular, suppose that $\calC$ admits finite colimits and that we construct new $\infty$-category $\Ind(\calC)$ by formally adjoining filtered colimits to $\calC$. Then $\Ind(\calC)$ admits all small colimits (Theorem \ref{pretop}), and the Yoneda embedding $\calC \rightarrow \Ind(\calC)$ preserves finite colimits (Proposition \ref{turnke}). 
Moreover, we can identify $\Ind(\calC)$ with the $\infty$-category of functors $\calC^{op} \rightarrow \SSet$ which carry finite colimits in $\calC$ to finite limits in $\SSet$. In this section, we will introduce a variation on the same theme. Instead of assuming $\calC$ admits {\em all} finite colimits, we will only assume that $\calC$ admits finite coproducts. We will construct a coproduct-preserving embedding of $\calC$ into a larger $\infty$-category $\calP_{\Sigma}(\calC)$ which admits all small colimits. Moreover, we can characterize $\calP_{\Sigma}(\calC)$ as the $\infty$-category obtained from $\calC$ by formally adjoining colimits of {\em sifted diagrams} (Proposition \ref{surottt}). 

Our first goal in this section is to introduce the notion of a {\em sifted} simplicial set. We begin with a bit of motivation. Let $\calC$ denote the (ordinary) category of groups. Then $\calC$ admits arbitrary colimits. However, colimits of diagrams in $\calC$ can be very difficult to analyze, even if the diagram itself is quite simple. For example, the coproduct of a pair of groups $G$ and $H$ is the {\it amalgamated product} $G \star H$. The group $G \star H$ is typically very complicated, even when $G$ and $H$ are not. For example, the amalgamated product
$\Z / 2 \Z \star \Z / 3 \Z$ is isomorphic to arithmetic group $\PSL_2(\Z)$. 
In general, $G \star H$ is much larger than the coproduct
$G \coprod H$ of the underlying sets of $G$ and $H$. In other words, the forgetful functor $U: \calC \rightarrow \Set$ does not preserve coproducts. However, $U$ does preserve {\em some} colimits: for example, the colimit of a sequence of groups
$$ G_0 \rightarrow G_1 \rightarrow \ldots $$
can be obtained by taking the colimit of the underlying sets, and equipping the result with an appropriate group structure.

The forgetful functor $U$ from groups to sets preserves another important type of colimit: namely, the formation of quotients by equivalence relations. If $G$ is a group, then a subgroup
$R \subseteq G \times G$ is an equivalence relation on $G$ if and only if there exists
a normal subgroup $H \subseteq G$ such that $R = \{ (g,g'): g^{-1} g' \in H \}$. In this case,
the set of $R$-equivalence classes in $G$ is in bijection with the quotient $G/H$, which inherits a group structure from $G$. In other words, the quotient of $G$ by the equivalence relation $R$ can be computed either in the category of groups or the category of sets; the result is the same.

Each of the examples given above admits a generalization: the colimit of a sequence is a special case of a {\it filtered colimit}, and the quotient by an equivalence relation is a special case of a {\it reflexive coequalizer}. The forgetful functor $\calC \rightarrow \Set$ preserves filtered colimits and reflexive coequalizers; moreover, the same is true if we replace the category of groups by any other category of sets with some sort of finitary algebraic structure (for example, abelian groups, or commutative rings). The following definition, which is taken from \cite{homotopyvarieties}, is an attempt to axiomatize the essence of the situation:\index{gen}{coequalizer!reflexive}\index{gen}{reflexive coequalizer}

\begin{definition}[\cite{homotopyvarieties}]\label{siftdef}\index{gen}{sifted}\index{gen}{simplicial set!sifted}
A simplicial set $K$ is {\it sifted} if it satisfies the following conditions:
\begin{itemize}
\item[$(1)$] The simplicial set $K$ is nonempty.
\item[$(2)$] The diagonal map $K \rightarrow K \times K$ is cofinal.
\end{itemize}
\end{definition}

\begin{warning}
In \cite{homotopyvarieties}, Rosicki uses the term {\it homotopy sifted} to describe the analogue of Definition \ref{siftdef} for simplicial categories, and reserves the term {\it sifted} for analogous notion in the setting of ordinary categories. There is some danger of confusion with our terminology:
if $\calC$ is an ordinary category and $\Nerve(\calC)$ is sifted (in the sense of Definition \ref{siftdef}), then $\calC$ is sifted in the sense of \cite{homotopyvarieties}. However, the converse is false in general.
\end{warning}

\begin{example}\label{bin1}
Every filtered $\infty$-category is sifted (Proposition \ref{undertruck}). 
\end{example}

\begin{lemma}\label{bball3}
The simplicial set $\Nerve( \cDelta)^{op}$ is sifted.
\end{lemma}

\begin{proof}
Since $\Nerve(\cDelta)^{op}$ is clearly nonempty, it will suffice to show that the diagonal map
$\Nerve( \cDelta)^{op} \rightarrow \Nerve( \cDelta)^{op} \times \Nerve(\cDelta)^{op}$ is cofinal.
According to Theorem \ref{hollowtt}, this is equivalent to the assertion that for every object
$([m], [n]) \in \cDelta \times \cDelta$, the category
$$ \calC = \cDelta_{/ [m]} \times_{ \cDelta} \cDelta_{/ [n] }$$
has weakly contractible nerve. Let $\calC^{0}$ be the full subcategory of $\calC$
spanned by those objects which correspond to {\em monomorphisms} of partially ordered sets $J \rightarrow [m] \times [n]$. The inclusion of $\calC^{0}$ into $\calC$
has a left adjoint, so the inclusion $\Nerve(\calC^{0}) \subseteq \Nerve(\calC)$ is a weak homotopy equivalence. It will therefore suffice to show that $\Nerve(\calC^{0})$ is weakly contractible. We now observe that $\Nerve(\calC^{0})$ can be identified with the first barycentric subdivision of
$\Delta^m \times \Delta^n$, and is therefore weakly homotopy equivalent to 
$\Delta^m \times \Delta^n$ and so weakly contractible.
\end{proof}

\begin{remark}
The formation of geometric realization of simplicial objects should be thought of as the $\infty$-categorical analogue of the formation of reflexive coequalizers.
\end{remark}

Our next pair of results captures some of the essential features of the theory of sifted simplicial sets:

\begin{proposition}\label{urbil}
Let $K$ be a sifted simplicial set, let
$\calC$, $\calD$, and $\calE$ be $\infty$-categories which admit $K$-indexed colimits, and let
$f: \calC \times \calD \rightarrow \calE$ be a map which preserves $K$-indexed colimits separately in each variable. Then $f$ preserves $K$-indexed colimits. 
\end{proposition}

\begin{proof}
Let $p: K \rightarrow \calC$ and $q: K \rightarrow \calD$ be diagrams indexed by a small simplicial set $K$, and let $\delta: K \rightarrow K \times K$ be the diagonal map. Using the fact that 
$f$ preserves $K$-indexed colimits separately in each variable and Lemma \ref{limitscommute}, we conclude that $\colim( f \circ (p \times q) )$ is a colimit for the diagram $f \circ (p \times q) \circ \delta$. Consequently, $f$ preserves $K$-indexed colimits provided that the diagonal $\delta$ is cofinal. We conclude by invoking the assumption that $K$ is sifted.
\end{proof}

\begin{proposition}\label{siftcont}
Let $K$ be a sifted simplicial set. Then $K$ is weakly contractible.
\end{proposition}

\begin{proof}
Choose a vertex $x$ in $K$. According to Whitehead's theorem, it will suffice to show that for each
$n \geq 0$, the homotopy set $\pi_n( |K|,x)$ consists of a single element. Let $\delta: K \rightarrow K \times K$ be the diagonal map. Since $\delta$ is cofinal, Proposition \ref{cofbasic} implies that
the induced map
$$ \pi_n( |K|, x) \rightarrow \pi_n( |K \times K|, \delta(x) ) \simeq \pi_n(|K|,x) \times \pi_n(|K|,x)$$
is bijective. Since $\pi_n( |K|,x)$ is nonempty, we conclude that it is a singleton.
\end{proof}

We now return to the problem introduced in the beginning of this section.

\begin{definition}\label{vardef}
Let $\calC$ be a small $\infty$-category which admits finite coproducts. We let
$\calP_{\Sigma}(\calC)$ denote the full subcategory of $\calP(\calC)$ spanned by
those functors $\calC^{op} \rightarrow \SSet$ which preserve finite products.
\end{definition}\index{not}{PSigmaC@$\calP_{\Sigma}(\calC)$}

\begin{remark}
The $\infty$-categories of the form $\calP_{\Sigma}(\calC)$ have been studied in
\cite{homotopyvarieties}, where they are called {\it homotopy varieties}. Many of the results proven below can also be found in \cite{homotopyvarieties}.\index{gen}{homotopy varieties}
\end{remark}

\begin{proposition}\label{utut}
Let $\calC$ be a small $\infty$-category which admits finite coproducts. Then:
\begin{itemize}
\item[$(1)$] The $\infty$-category $\calP_{\Sigma}(\calC)$ is an accessible localization
of $\calP(\calC)$. % In particular, $\calP_{\Sigma}(\calC)$ is presentable.
\item[$(2)$] The Yoneda embedding $j: \calC \rightarrow \calP(\calC)$ factors
through $\calP_{\Sigma}(\calC)$. Moreover, $j$ carries finite coproducts in $\calC$
to finite coproducts in $\calP_{\Sigma}(\calC)$. 
\item[$(3)$] Let $\calD$ be a presentable $\infty$-category, and let
$$ \Adjoint{ F}{\calP(\calC)}{\calD}{G}$$
be a pair of adjoint functors. Then $G$ factors through $\calP_{\Sigma}(\calC)$ if and only if
$f = F \circ j: \calC \rightarrow \calD$ preserves finite coproducts.
\item[$(4)$] The full subcategory $\calP_{\Sigma}(\calC) \subseteq \calP(\calC)$ is stable
under sifted colimits. 
\item[$(5)$] Let $L: \calP(\calC) \rightarrow \calP_{\Sigma}(\calC)$ be a left adjoint to the inclusion. Then $L$ preserves sifted colimits $($when regarded as a functor from $\calP(\calC)$ to itself$)$.
\item[$(6)$] The $\infty$-category $\calP_{\Sigma}(\calC)$ is compactly generated.
\end{itemize}
\end{proposition}

Before giving the proof, we need a preliminary result concerning the interactions between products sifted colimits.

\begin{lemma}\label{bale2}
Let $K$ be a sifted simplicial set. Let $\calX$ be an $\infty$-category which admits finite products and
$K$-indexed colimits, and suppose that the formation of products in $\calX$ preserves $K$-indexed colimits separately in each variable. Then the colimit functor $\colim: \Fun(K, \calX) \rightarrow \calX$
preserves finite products.
\end{lemma}

\begin{remark}\label{bale3}
The hypotheses of Lemma \ref{bale2} are satisfied when $\calX$ is the $\infty$-category $\SSet$ of spaces: see Lemma \ref{sugartime}. More generally, Lemma \ref{bale2} applies whenever
the $\infty$-category $\calX$ is an $\infty$-topos (see Definition \ref{def1topos}).
\end{remark}

\begin{proof}
Since the simplicial set $K$ is weakly contractible (Proposition \ref{siftcont}), Corollary \ref{charext} implies that the functor $\colim$ preserves final objects. To complete the proof, it will suffice to show that the functor $\colim$ preserves pairwise products. Let $X$ and $Y$ be objects of $\Fun(K, \calX)$. We wish to prove that the canonical map
$$ \colim( X \times Y) \rightarrow \colim(X) \times \colim(Y)$$ is an equivalence.
In other words, we must show that the formation of products commutes with $K$-indexed colimits, which follows immediately by applying Proposition \ref{urbil} to the Cartesian product functor
$\calX \times \calX \rightarrow \calX$.
\end{proof}

\begin{proof}[Proof of Proposition \ref{utut}]
Assertion $(1)$ is an immediate consequence of Lemmas \ref{stur2}, \ref{stur3}, and \ref{stur1}. To prove $(2)$, it will suffice to show that for every representable functor
$e: \calP_{\Sigma}(\calC)^{op} \rightarrow \SSet$, the composition
$$ \calC^{op} \stackrel{j^{op}}{\rightarrow} \calP_{\Sigma}(\calC)^{op} \stackrel{e}{\rightarrow} \SSet$$
preserves finite products (Proposition \ref{yonedaprop}). This is obvious, since the composition can be identified with the object of $\calP_{\Sigma}(\calC) \subseteq
\Fun( \calC^{op}, \SSet)$ representing $e$.

We next prove $(3)$. We note that $f$ preserves finite coproducts if and only if, for
every object $D \in \calD$, the composition
$$ \calC^{op} \stackrel{f^{op}}{\rightarrow} \calD^{op} \stackrel{e_D}{\rightarrow} \SSet$$
preserves finite products, where $e_D$ denotes the functor represented by $D$. 
This composition can be identified with $G(D)$, so that $f$ preserves finite coproducts if and only if
$G$ factors through $\calP_{\Sigma}(\calC)$.

Assertion $(4)$ is an immediate consequence of Lemma \ref{bale2} and Remark \ref{bale3}, and $(5)$ follows formally from $(4)$. To prove $(6)$, we first observe that
$\calP(\calC)$ is compactly generated (Proposition \ref{precst}). Let $\calE
\subseteq \calP(\calC)$ be the full subcategory spanned by the compact objects, and let
$L: \calP(\calC) \rightarrow \calP_{\Sigma}(\calC)$ be a localization functor. Since
$\calE$ generates $\calP(\calC)$ under filtered colimits, $L(\calD)$ generates
$\calP_{\Sigma}(\calC)$ under filtered colimits. Consequently, it will suffice to
show that for each $E \in \calE$, the object $LE \in \calP_{\Sigma}(\calC)$ is compact. Let
$f: \calP_{\Sigma}(\calC) \rightarrow \SSet$ be the functor corepresented by $LE$, and let
$f': \calP(\calC) \rightarrow \SSet$ be the functor corepresented by $E$. Then
the map $E \rightarrow LE$ induces an equivalence
$f \rightarrow f' | \calP_{\Sigma}(\calC)$. Since $f'$ is continuous and
$\calP_{\Sigma}(\calC)$ is stable under filtered colimits in $\calP(\calC)$, we conclude
that $f$ is continuous, so that $LE$ is a compact object of $\calP_{\Sigma}(\calC)$ as desired.
\end{proof}

Our next goal is to prove a converse to part $(4)$ of Proposition \ref{utut}. Namely, we will show that $\calP_{\Sigma}(\calC)$ is generated by the essential image of the Yoneda embedding
under sifted colimits. In fact, we will only need to use special types of sifted colimits: namely, filtered colimits and geometric realizations (Lemma \ref{subato}). The proof is based on the following technical result:

\begin{lemma}\label{presubato}
Let $\calC$ be a small $\infty$-category, and let $X$ be an object of $\calP(\calC)$. Then
there exists a simplicial object $Y_{\bigdot}: \Nerve(\cDelta)^{op} \rightarrow \calP(\calC)$ with the following properties:
\begin{itemize}
\item[$(1)$] The colimit of $Y_{\bigdot}$ is equivalent to $X$.
\item[$(2)$] For each $n \geq 0$, the object $Y_n \in \calP(\calC)$ is equivalent to a small coproduct of objects lying in the image of the Yoneda embedding $j: \calC \rightarrow \calP(\calC)$. 
\end{itemize}
\end{lemma}

We will defer the proof until the end of this section.

\begin{lemma}\label{subato}
Let $\calC$ be a small $\infty$-category which admits finite coproducts, and let
$X \in \calP(\calC)$. The following conditions are equivalent:
\begin{itemize}
\item[$(1)$] The object $X$ belongs to $\calP_{\Sigma}(\calC)$.
\item[$(2)$] There exists a simplicial object $U_{\bigdot}: \Nerve( \cDelta)^{op} \rightarrow \Ind(\calC)$ whose colimit in $\calP(\calC)$ is $X$.
\end{itemize}
\end{lemma}

\begin{proof}
The full subcategory $\calP_{\Sigma}(\calC)$ contains the essential image of the Yoneda embedding and is stable under filtered colimits and geometric realizations (Proposition \ref{utut}); thus $(2) \Rightarrow (1)$. We will prove that $(1) \Rightarrow (2)$. 

We first choose a simplicial object $Y_{\bigdot}$ of $\calP(\calC)$ which satisfies the conclusions of
Lemma \ref{presubato}. Let $L$ be a left adjoint to the inclusion $\calP_{\Sigma}(\calC) \subseteq \calP(\calC)$. Since $X$ is a colimit of $Y_{\bigdot}$, $LX \simeq X$ is a colimit of $LY_{\bigdot}$ (part $(5)$ of Proposition \ref{utut}). It will therefore suffice to prove that each $LY_{n}$ belongs to $\Ind(\calC)$. By hypothesis, each $Y_{n}$ can be written as a small coproduct $\coprod_{\alpha \in A} j(C_{\alpha} )$, where
$j: \calC \rightarrow \calP(\calC)$ denotes the Yoneda embedding. 
Using the results of \S \ref{quasilimit1}, we see that $Y_{n}$ can be obtained also as a filtered colimit of coproducts $\coprod_{ \alpha \in A_0} j(C_{\alpha})$, where $A_0$ ranges over the finite subsets of $A$. Since $L$ preserves filtered colimits (Proposition \ref{utut}), it will suffice to show that each of the objects
$$L(\coprod_{\alpha \in A_0} j(C_{\alpha} ))$$
belongs to $\Ind(\calC)$. We now invoke part $(2)$ of Proposition \ref{utut} to identify this object with $j( \coprod_{\alpha \in A_0} C_{\alpha} )$. 
\end{proof}

\begin{proposition}\label{surottt}
Let $\calC$ be a small $\infty$-category which admits finite coproducts, and let
$\calD$ be an $\infty$-category which admits filtered colimits and geometric realizations.
Let $\Fun_{\Sigma}( \calP_{\Sigma}(\calC), \calD)$ denote the full subcategory spanned by those functors $\calP_{\Sigma}(\calC) \rightarrow \calD$ which preserve filtered colimits and geometric realizations. Then:
\begin{itemize}
\item[$(1)$] Composition with the Yoneda embedding $j: \calC \rightarrow \calP_{\Sigma}(\calC)$ induces an equivalence of categories
$$ \theta: \Fun_{\Sigma}( \calP_{\Sigma}(\calC), \calD) \rightarrow \Fun(\calC, \calD).$$

\item[$(2)$] Any functor $g \in \Fun_{\Sigma}( \calP_{\Sigma}(\calC), \calD)$ preserves sifted colimits.

\item[$(3)$] Assume that $\calD$ admits finite coproducts. A functor $g \in \Fun_{\Sigma}( \calP_{\Sigma}(\calC), \calD)$ preserves small colimits if and only if $g \circ j$ preserves finite coproducts.\index{not}{FunSigma@$\Fun_{\Sigma}( \calC, \calD)$}
\end{itemize}
\end{proposition}

\begin{proof}
Lemma \ref{subato} and Proposition \ref{utut} imply that $\calP_{\Sigma}(\calC)$ is the smallest full subcategory of $\calP(\calC)$ which is closed under filtered colimits, closed under geometric realizations, and contains the essential image of the Yoneda embedding. Consequently, assertion $(1)$ follows from Remark \ref{poweryoga} and Proposition \ref{lklk}. 

We now prove $(2)$. Let $g \in \Fun_{\Sigma}( \calP_{\Sigma}(\calC), \calD)$; we wish to show that $g$ preserves sifted colimits. It will suffice to show that for every representable functor
$e: \calD \rightarrow \SSet^{op}$, the composition $e \circ g$ preserves sifted colimits. In other words, we may replace $\calD$ by $\SSet^{op}$, and thereby reduce to the case where
$\calD$ itself admits sifted colimits. Let $\Fun'_{\Sigma}( \calP_{\Sigma}(\calC), \calD)$ denote the full subcategory of $\Fun_{\Sigma}( \calP_{\Sigma}(\calC), \calD)$ spanned by those functors which preserve sifted colimits. Since $\calP_{\Sigma}(\calC)$ is also the smallest full subcategory of $\calP(\calC)$ which contains the essential image of the Yoneda embedding and is stable under sifted colimits, Remark \ref{poweryoga} implies that $\theta$ induces an equivalence
$$ \Fun'_{\Sigma}( \calP_{\Sigma}(\calC), \calD ) \rightarrow \Fun(\calC, \calD).$$
Combining this observation with $(1)$, we deduce that the inclusion
$\Fun'_{\Sigma}( \calP_{\Sigma}(\calC), \calD) \subseteq \Fun_{\Sigma}( \calP_{\Sigma}(\calC), \calD)$
is an equivalence of $\infty$-categories, and therefore an equality.

The ``only if'' direction of $(3)$ is immediate, since the Yoneda embedding $j: \calC \rightarrow \calP_{\Sigma}(\calC)$ preserves finite coproducts (Proposition \ref{utut}). To prove the converse,
we first apply Lemma \ref{diverti} to reduce to the case where $\calD$ is a full subcategory of an $\infty$-category $\calD'$, with the following properties:
\begin{itemize}
\item[$(i)$] The $\infty$-category $\calD'$ admits small colimits.
\item[$(ii)$] A small diagram $K^{\triangleright} \rightarrow \calD$ is a colimit if and only if the induced diagram $K^{\triangleright} \rightarrow \calD'$ is a colimit.
\end{itemize}
Let $\calC'$ denote the essential image of the Yoneda embedding $j: \calC \rightarrow \calP(\calC)$.
Using Lemma \ref{longwait1}, we conclude that there exists functor $G: \calP(\calC) \rightarrow \calD'$ which is a left Kan extension of $G | \calC' = g | \calC'$, and that $G$ preserves small colimits. 
Let $G_0 = G | \calP_{\Sigma}(\calC)$. Then $G_0$ is a left Kan extension of $g | \calC'$, so there
is a canonical natural transformation $G_0 \rightarrow g$. Let $\calC''$ denote the full subcategory
of $\calP_{\Sigma}(\calC)$ spanned by those objects $C$ for which the map $G_0(C) \rightarrow g(C)$
is an equivalence. Then $\calC''$ contains $\calC'$ and is stable under filtered colimits and geometric realizations, and therefore contains all of $\calP_{\Sigma}(\calC)$. We may therefore replace
$g$ by $G_0$ and thereby assume that $G | \calP_{\Sigma}(\calC) = g$.
Since $G \circ j = g \circ j$ preserves finite coproducts, the right adjoint to $G$ factors through
$\calP_{\Sigma}(\calC)$ (Proposition \ref{utut}), so that $G$ is equivalent to the composition
$$ \calP(\calC) \stackrel{L}{\rightarrow} \calP_{\Sigma}(\calC) \stackrel{G'}{\rightarrow} \calD'$$
for some colimit-preserving functor $G': \calP_{\Sigma}(\calC) \rightarrow \calD'$. Restricting to
the subcategory $\calP_{\Sigma}(\calC) \subseteq \calP(\calC)$, we deduce that
$G'$ is equivalent to $g$, so that $g$ preserves small colimits as desired.
\end{proof}

\begin{remark}\label{spuduse}
Let $\calC$ be a small $\infty$-category which admits finite coproducts. It follows from
Proposition \ref{surottt} that we can identify $\calP_{\Sigma}(\calC)$ with
$\calP_{\calK}^{\calK'}(\calC)$ in each of the following three cases (for
an explanation of this notation, we refer the reader to \S \ref{agileco}):
\begin{itemize}
\item[$(1)$] The collection $\calK$ is empty, and the collection $\calK'$ consists of all
small filtered simplicial sets together with $\Nerve(\cDelta)^{op}$.
\item[$(2)$] The collection $\calK$ is empty, and the collection $\calK'$ consists of all
small sifted simplicial sets.
\item[$(3)$] The collection $\calK$ consists of all finite discrete simplicial sets, and
the collection $\calK'$ consists of all small simplicial sets.
\end{itemize}
\end{remark}

\begin{corollary}\label{swillt}
Let $f: \calC \rightarrow \calD$ be a functor between $\infty$-categories. Assume that $\calC$ admits small colimits. Then $f$ preserves sifted colimits if and only if $f$ preserves filtered colimits and geometric realizations.
\end{corollary}

\begin{proof}
The ``only if'' direction is clear. For the converse, suppose that $f$ preserves filtered colimits and geometric realizations. Let $\calI$ be a small sifted $\infty$-category and $\overline{p}: \calI^{\triangleright} \rightarrow \calC$ a colimit diagram; we wish to prove that $f \circ \overline{p}$ is also a colimit diagram. Let $p = \overline{p} | \calI$. Let $\calJ \subseteq \calP(\calI)$ denote a small full subcategory which contains the essential image of the Yoneda embedding $j: \calI \rightarrow \calP(\calI)$ and is closed under finite coproducts. It follows from Remark \ref{poweryoga} that
the functor $p$ is homotopic to a composition $q \circ j$, where $q: \calJ \rightarrow \calC$ is a functor which preserves finite coproducts. Proposition \ref{surottt} implies that
$q$ is homotopic to a composition
$$ \calJ \stackrel{j'}{\rightarrow} \calP_{\Sigma}(\calJ) \stackrel{q'}{\rightarrow} \calC,$$
where $j'$ denotes the Yoneda embedding and $q'$ preserves small colimits. 
The composition $f \circ q'$ preserves filtered colimits and geometric realization, and therefore
preserves sifted colimits (Proposition \ref{surottt}). 

Let $\overline{p}': \calI^{\triangleright} \rightarrow \calP_{\Sigma}(\calJ)$ be a colimit of the diagram
$j' \circ j$. Since $q'$ preserves colimits, the composition $q' \circ \overline{p}'$ is a colimit of
$q' \circ j' \circ j \simeq p$, and is therefore equivalent to $\overline{p}$. Consequently, it will suffice to show that $f \circ q' \circ \overline{p}'$ is a colimit diagram. Since $\calI$ is sifted, we need only
verify that $f \circ q'$ preserves sifted colimits. By Proposition \ref{surottt}, it will suffice to show that
$f \circ q'$ preserves filtered colimits and geometric realizations. Since $q'$ preserves all colimits, this follows from our assumption that $f$ preserves filtered colimits and geometric realizations.
\end{proof}

In the situation of Proposition \ref{surottt}, every functor $f: \calC \rightarrow \calD$ extends (up to homotopy) to a functor $F: \calP_{\Sigma}(\calC) \rightarrow \calD$, which preserves sifted colimits. We will sometimes refer to $F$ as the {\it left derived functor} of $f$\index{gen}{derived functor}\index{gen}{left derived functor}. In \S \ref{stable12} we will explain the connection of this notion of derived functor with the more classical definition provided by Quillen's theory of homotopical algebra.\index{gen}{derived functor}\index{gen}{functor!derived}

Our next goal is to characterize those $\infty$-categories which have the form $\calP_{\Sigma}(\calC)$.

\begin{definition}\label{humber}
Let $\calC$ be an $\infty$-category which admits geometric realizations of simplicial objects. We will say that an object $P \in \calC$ is {\it projective} if the functor $\calC \rightarrow \SSet$ co-represented by $P$ commutes with geometric realizations.\index{gen}{projective object}\index{gen}{object!projective}
\end{definition}

\begin{remark}\label{untine}
Let $\calC$ be an $\infty$-category which admits geometric realizations of simplicial objects. Then
the collection of projective objects of $\calC$ is stable under all finite coproducts which exist in $\calC$.
This follows immediately from Lemma \ref{bale2} and Remark \ref{bale3}. 
\end{remark}

\begin{remark}\label{comproj}
Let $\calC$ be an $\infty$-category which admits small colimits, and let $X$ be an object of
$\calC$. Then $X$ is compact and projective if and only if $X$ corepresents a functor
$\calC \rightarrow \sSet$ which preserves sifted colimits. The ``only if'' direction is obvious, and
the converse follows from Corollary \ref{swillt}. 
\end{remark}

\begin{example}
Let $\calA$ be an abelian category. Then an object $P \in \calA$ is projective in the sense of classical homological algebra (that is, the functor $\Hom_{\calA}(P, \bigdot)$ is exact) if and only if $P$ corepresents a functor $\calA \rightarrow \Set$ which commutes with geometric realizations of simplicial objects. This is {\em not} equivalent to the condition of Definition \ref{humber}, since the fully faithful embedding $\Set \rightarrow \SSet$ does not preserve geometric realizations. However, it is equivalent to the requirement that $P$ be a projective object (in the sense of Definition \ref{humber}) in the $\infty$-category underlying the homotopy theory of simplicial objects
of $\calA$ (equivalently, the theory of nonpositively graded chain complexes with values in $\calA$;
we will discuss this example in greater detail in \cite{DAG}).
\end{example}

\begin{proposition}\label{smearof}
Let $\calC$ be a small $\infty$-category which admits finite coproducts, 
$\calD$ an $\infty$-category which admits filtered colimits and geometric realizations,
and $F: \calP_{\Sigma}(\calC) \rightarrow \calD$ a left derived functor of
$f = F \circ j: \calC \rightarrow \calD$, where $j: \calC \rightarrow \calP_{\Sigma}(\calC)$ denotes the Yoneda embedding. Consider the following conditions:
\begin{itemize}
\item[$(i)$] The functor $f$ is fully faithful.
\item[$(ii)$] The essential image of $f$ consists of compact projective objects of $\calD$.
\item[$(iii)$] The $\infty$-category $\calD$ is generated by the essential image of $f$ under filtered colimits and geometric realizations.
\end{itemize}
If $(i)$ and $(ii)$ are satisfied, then $F$ is fully faithful. Moreover, $F$ is an equivalence if
and only if $(i)$, $(ii)$, and $(iii)$ are satisfied.
\end{proposition}

\begin{proof}
If $F$ is an equivalence of $\infty$-categories, then $(i)$ follows from Proposition \ref{fulfaith}, and $(iii)$ from Lemma \ref{subato}. To prove $(ii)$, it suffices to show that for
each $C \in \calC$, the functor $e: \calP_{\Sigma}(\calC) \rightarrow \SSet$ corepresented by $C$ preserves filtered colimits and geometric realizations. We can identify $e$ with the composition
$$ \calP_{\Sigma}(\calC) \stackrel{e'}{\subseteq} \calP(\calC) \stackrel{e''}{\rightarrow} \SSet,$$
where $e''$ denotes evaluation at $C$. It now suffices to observe that $e'$ and $e''$
preserve filtered colimits and geometric realizations (Lemma \ref{subato} and Proposition \ref{limiteval}).

For the converse, let us suppose that $(i)$ and $(ii)$ are satisfied. We will show that $F$ is fully faithful. First fix an object $C \in \calC$, and let $\calP'_{\sigma}(\calC)$ be the full subcategory
of $\calP_{\Sigma}(\calC)$ spanned by those objects $M$ for which the map
$$ \bHom_{ \calP_{\Sigma}(\calC)}( j(C), M) \rightarrow \bHom_{\calD}(f(C), F(M) )$$
is an equivalence. Condition $(i)$ implies that $\calP'_{\sigma}(\calC)$ contains the essential image of $j$, and condition $(ii)$ implies that $\calP'_{\sigma}(\calC)$ is stable under filtered colimits and geometric realizations. Lemma \ref{subato} now implies that $\calP'_{\Sigma}(\calC) = \calP_{\Sigma}(\calC)$.

We now define $\calP''_{\Sigma}(\calC)$ to be the full subcategory of $\calP_{\Sigma}(\calC)$ spanned by those objects $M$ such that for all $N \in \calP_{\Sigma}(\calC)$, the map
$$ \bHom_{ \calP_{\Sigma}(\calC)}(M,N) \rightarrow \bHom_{\calD}( F(M), F(N) )$$
is a homotopy equivalence. The above argument shows that $\calP''_{\Sigma}(\calC)$ contains
the essential image of $j$. Since $F$ preserves filtered colimits and geometric realizations,
$\calP''_{\Sigma}(\calC)$ is stable under filtered colimits and geometric realizations. Applying Lemma \ref{subato}, we conclude that $\calP''_{\Sigma}(\calC) = \calP_{\Sigma}(\calC)$; this proves that $F$ is fully faithful.

If $F$ is fully faithful, then the essential image of $F$ contains $f(\calC)$ and is stable under filtered colimits and geometric realizations. If $(iii)$ is satisfied, it follows that $F$ is an equivalence of $\infty$-categories.
\end{proof}

\begin{definition}\label{defpro}
Let $\calC$ be an $\infty$-category which admits small colimits, and let $S$ be a collection of objects of $\calC$. We will say that $S$ is a {\it set of compact projective generators for $\calC$} if the following conditions are satisfied:
\begin{itemize}
\item[$(1)$] Each element of $S$ is a compact projective object of $\calC$.
\item[$(2)$] The full subcategory of $\calC$ spanned by the elements of $S$ is essentially small.
\item[$(3)$] The set $S$ generates $\calC$ under small colimits.
\end{itemize}
We will say that $\calC$ is {\it projectively generated} if there exists a set $S$ of compact projective generators for $\calC$.\index{gen}{projectively generated}\index{gen}{generator!projective} 
\end{definition}

\begin{example}\label{swine}
The $\infty$-category $\SSet$ of spaces is projectively generated. The compact projective objects of $\SSet$ are precisely those spaces which are homotopy equivalent to finite sets (endowed with the discrete topology).
\end{example}

\begin{proposition}\label{protus}
Let $\calC$ be an $\infty$-category which admits small colimits, and let $S$ be a set of compact projective generators for $\calC$. Then:
\begin{itemize}
\item[$(1)$] Let $\calC^{0} \subseteq \calC$ be the full subcategory spanned by finite coproducts of the objects $S$, let $\calD \subseteq \calC^{0}$ be a minimal model for $\calC^{0}$, and let
$F: \calP_{\Sigma}( \calD ) \rightarrow \calC$ be a left derived functor of the inclusion. Then $F$ is an equivalence of $\infty$-categories. In particular, $\calC$ is a compactly generated presentable $\infty$-category.
\item[$(2)$] Let $C \in \calC$ be an object. The following conditions are equivalent:
\begin{itemize}
\item[$(i)$] The object $C$ is compact and projective.
\item[$(ii)$] The functor $e: \calC \rightarrow \widehat{\SSet}$ corepresented by $C$
preserves sifted colimits.
\item[$(iii)$] There exists an object $C' \in \calC^{0}$ such that $C$ is a retract of $C'$.
\end{itemize}
\end{itemize}
\end{proposition}

\begin{proof}
Remark \ref{untine} implies that $\calC^{0}$ consists of compact projective objects of $\calC$.
Assertion $(1)$ now follows immediately from Proposition \ref{smearof}. We now prove $(2)$.
The implications $(iii) \Rightarrow (i)$ and $(ii) \Rightarrow (i)$ are obvious. To complete the proof, we will show that
$(i) \Rightarrow (iii)$. Using $(1)$, we are free to assume $\calC = \calP_{\Sigma}(\calD)$.
Let $C \in \calC$ be a compact projective object.
Using Lemma \ref{subato}, we conclude that there exists a simplicial object
$X_{\bigdot}$ of $\Ind(\calD)$ and an equivalence $C \simeq | X_{\bigdot} |$. Since
$C$ is projective, we deduce $\bHom_{\calC}(C,C)$ is equivalent to the geometric realization of the simplicial space $\bHom_{\calC}(C, X_{\bigdot})$. In particular, $\id_{C} \in \bHom_{\calC}(C,C)$ is homotopic to the image of the some map $f: C \rightarrow X_0$. Using our assumption that
$C$ is compact, we conclude that $f$ factors as a composition
$$ C \stackrel{f_0}{\rightarrow} j(D) \rightarrow X_0,$$
where $j: \calD \rightarrow \Ind(\calD)$ denotes the Yoneda embedding. It follows that
$C$ is a retract of $j(D)$ in $\calC$, as desired.
\end{proof}

\begin{remark}\label{parei}
Let $\calC$ be a small $\infty$-category which admits finite coproducts. Since
the truncation functor $\tau_{\leq n}: \SSet \rightarrow \SSet$ preserves finite products,
it induces a map $\tau: \calP_{\Sigma}(\calC) \rightarrow \calP_{\Sigma}(\calC)$, which is easily seen to be a localization functor. The essential image of $\tau$ consists of those functors
$F \in \calP_{\Sigma}(\calC)$ which take $n$-truncated values. We claim that these
are precisely the $n$-truncated object of $\calP_{\Sigma}(\calC)$. Consequently, we can
identify $\tau$ with the $n$-truncation functor on $\calP_{\Sigma}(\calC)$.

One direction is clear: if $F \in \calP_{\Sigma}(\calC)$ is $n$-truncated, then for each $C \in \calC$ the space $\bHom_{ \calP_{\Sigma}(\calC)}( j(C), F) \simeq F(C)$ must be $n$-truncated. Conversely,
suppose that $F: \calC^{op} \rightarrow \SSet$ takes $n$-truncated values. We wish to prove that
the space $\bHom_{ \calP_{\Sigma}(\calC)}(F', F)$ is $n$-truncated, for each $F' \in \calP_{\Sigma}(\calC)$. The collection of all objects $F'$ which satisfy this condition is stable under small colimits in $\calP_{\Sigma}(\calC)$ and contains the essential image of the Yoneda embedding. It therefore contains the entirety of $\calP_{\Sigma}(\calC)$, as desired.
\end{remark} 

We conclude this section by giving the proof of Lemma \ref{presubato}. Our argument uses some
concepts and results from \S \ref{chap6}, and may be omitted at first reading.
%&&&This should be moved to appendix on Reedy categories, fix below
%\begin{remark}[Anatomy of a Simplicial Object]\label{huggrus}
%Let $\calX$ be an $\infty$-category, and suppose we wish to construct a simplicial object
%$\psi: \Nerve( \cDelta)^{op} \rightarrow \calX$. For each $n \geq -1$, we let
%$\cDelta^{\leq n}$ denote the full subcategory of $\cDelta$ spanned by the objects
%$\{ [k] \}_{k \leq n}$. Suppose that we have already constructed a functor
%$\psi(n-1): \Nerve( \cDelta^{ \leq n-1} )^{op} \rightarrow \calX$ and we would like to find a compatible extension
%$\psi(n): \Nerve( \cDelta^{ \leq n})^{op} \rightarrow \calX$. 

%Let $K^{(k)}$ denote the simplicial subset of $\Nerve( \cDelta^{ \leq n} )^{op}$ spanned by those nondegenerate simplices which
%include the vertex $[n] \in \cDelta^{\leq n}$ no more than $k$ times. Then we have a filtration
%$$ \Nerve( \cDelta^{ \leq n-1})^{op} \simeq K^{(0)} \subseteq K^{(1)} \subseteq \ldots$$ 
%which exhausts $\Nerve( \cDelta^{\leq n})^{op}$. An easy calculation shows that the inclusions $K^{(k)} \subseteq K^{(k+1)}$ are categorical equivalences for $k \geq 1$. It follows that $\psi(n)$ is determined, up to equivalence, by the restriction $\psi(n) | K^{(1)}$. Moreover, we have a pushout diagram of simplicial sets
%$$ \xymatrix{ \Nerve( \cDelta^{\leq n-1})^{op}_{/[n] } \star \Nerve( \cDelta^{\leq n-1})^{op}_{[n]/} \ar[r] \ar@{^{(}->}[d] & \Nerve( \cDelta^{\leq n-1} ) \ar[d] \\
%\Nerve( \cDelta^{\leq n-1})^{op}_{/[n] } \star \{ [n] \} \star \Nerve( \cDelta^{\leq n-1})^{op}_{[n]/} \ar[r] & K^{(1)}. }$$
%If $\calX$ admits finite limits and colimits, then we can define a {\it matching object}
%$M_n$ to be a limit of the composition 
%$$\Nerve( \cDelta^{\leq n-1})_{[n]/}
%\rightarrow \Nerve( \cDelta^{\leq n-1}) \stackrel{\psi(n-1)}{\rightarrow} \calX$$ and a {\it latching object} $L_n$ to be a colimit of the composition $$\Nerve( \cDelta^{\leq n-1})_{/[n]}
%\rightarrow \Nerve( \cDelta^{\leq n-1}) \stackrel{\psi(n-1)}{\rightarrow} \calX.$$
%The diagram $\psi(n-1)$ determines a morphism $\phi: L_n \rightarrow M_n$ (well-defined up to homotopy), and constructing $\psi(n)$ is equivalent to producing a factorization
%$$ \xymatrix{ & Y_n \ar[dr] & \\
%L_n \ar[ur] \ar[rr]^{\phi} & & M_n. }$$\index{gen}{matching object}\index{gen}{latching object}
%\end{remark}

\begin{proof}[Proof of Lemma \ref{presubato}]
For $n \geq 0$, let $\cDelta^{\leq n}$ denote the full subcategory of $\cDelta$ spanned by the objects
$\{ [k] \}_{k \leq n}$. We will construct a compatible sequence of functors
$f_{n}: \Nerve( \cDelta^{\leq n})^{op} \rightarrow \calP(\calC)_{/X}$
with the following properties:
\begin{itemize}
\item[$(A)$] For $n \geq 0$, let $L_n$ denote a colimit of the composite diagram
$$ \Nerve( \cDelta^{\leq n-1})^{op} \times_{ \Nerve(\cDelta)^{op} }
\Nerve( \cDelta_{[n]/})^{op} \rightarrow \Nerve( \cDelta^{\leq n-1})^{op}
\stackrel{ f_{n-1}}{\rightarrow} \calP(\calC)_{/X} \rightarrow \calP(\calC).$$
(the $n$th {\it latching object}). Then there exists an object $Z_{n} \in \calP(\calC)$ which
is a small coproduct of objects in the essential image of the Yoneda embedding
$\calC \rightarrow \calP(\calC)$, and a map $Z_{n} \rightarrow f_{n}([n])$ which,
together with the canonical map $L_n \rightarrow f_{n}([n])$, determines an equivalence
$L_n \coprod Z_{n} \simeq f_{n}([n])$. 

\item[$(B)$] For $n \geq 0$, let $\overline{M}_n$ denote the limit of the diagram
$$ \Nerve( \cDelta^{\leq n-1})^{op} \times_{ \Nerve(\cDelta)^{op} }
\Nerve( \cDelta_{/[n]})^{op} \rightarrow \Nerve( \cDelta^{\leq n-1})^{op}
\stackrel{ f_{n-1}}{\rightarrow} \calP(\calC)_{/X}$$
(the $n$th {\it matching object}), and let $M_n$ denote its image in $\calP(\calC)$.
Then the canonical map $f_{n}([n]) \rightarrow M_n$ is an effective epimorphism
in $\calP(\calC)$ (see \S \ref{surjsurj}).
\end{itemize}
The construction of the functors $f_{n}$ proceeds by induction on $n$,
the case $n < 0$ being trivial. For $n \geq 0$, we invoke Remark \ref{stapler2}:
to extend $f_{n-1}$ to a functor $f_{n}$ satisfying $(A)$ and $(B)$, it suffices to produce
an object $Z_{n}$ and a morphism $\psi: Z_{n} \rightarrow M_n$ in $\calP(\calC)$, such that
the coproduct $L_n \coprod Z_{n} \rightarrow M_n$ is an effective epimorphism.
This is satisfied in particular if $\psi$ itself is an effective epimorphism.

The maps $f_{n}$ together determine a simplicial object $\overline{Y}_{\bigdot}$ of
$\calP(\calC)_{/X}$, which we can identify with a simplicial object $Y_{\bigdot}$
in $\calP(\calC)$ equipped with a map $\theta: \varinjlim Y_{\bigdot} \rightarrow X$.
Assumption $(B)$ guarantees that $\theta$ is a hypercovering of $X$ (see \S \ref{hcovh}),
so that the map $\theta$ is $\infty$-connective (Lemma \ref{fierminus}).
The $\infty$-topos $\calP(\calC)$ has enough points (given by evaluation at objects of $\calC$), and is therefore hypercomplete (Remark \ref{notenough}). It follows that $\theta$ is an equivalence.
We now complete the proof by observing that for $n \geq 0$, we have an equivalence
$Y_{n} \simeq \coprod_{ [n] \rightarrow [k] } Z_{k}$
where the coproduct is taken over all surjective maps of linearly ordered sets $[n] \rightarrow [k]$, so that
$Y_{n}$ is itself a small coproduct of objects lying in the essential image of the Yoneda embedding
$j: \calC \rightarrow \calP(\calC)$.
\end{proof}
  
\subsection{Quillen's Model for $\calP_{\Sigma}(\calC)$}\label{stable12} 
 
Let $\calC$ be a small category which admits finite products. Then $\Nerve(\calC)^{op}$ is an $\infty$-category which admits finite coproducts. In \S \ref{stable11}, we studied the $\infty$-category $\calP_{\Sigma}( \Nerve(\calC)^{op} )$, which we can view as the full subcategory of
$\Fun( \Nerve(\calC), \SSet)$ spanned by those functors which preserve finite products. According to Proposition \ref{gumby444}, $\Fun( \Nerve(\calC), \SSet)$ can be identified with the $\infty$-category underlying the simplicial model category of diagrams $\Set_{\Delta}^{\calC}$ (which we will endow with the {\em projective} model structure described in \S \ref{compp4}). It follows that every functor
$f: \Nerve(\calC) \rightarrow \SSet$ is equivalent to the (simplicial) nerve of a functor $F: \calC \rightarrow \Kan$. Moreover, $f$ belongs to $\calP_{\Sigma}( \Nerve(\calC)^{op})$ if and only if the functor
$F$ is {\em weakly} product preserving, in the sense that for any finite collection of objects $\{ C_i \in \calC \}_{1 \leq i \leq n}$, the natural map $$F( C_1 \times \ldots C_n ) \rightarrow F(C_1) \times \ldots \times F(C_n)$$ is a homotopy equivalence of Kan complexes. Our goal in this section is to prove a refinement of Proposition \ref{gumby444}: if $f$ preserves finite products, then it is possible to arrange that $F$ preserves finite products (up to isomorphism, rather than up to homotopy equivalence). This result is most naturally phrased as an equivalence between model categories (Proposition \ref{trent}), and is due to Bergner (see \cite{bergner3}). We begin by recalling the following result of Quillen (for a proof, we refer the reader to
\cite{homotopicalalgebra}):

\begin{proposition}[Quillen]\label{sutcoat}
Let $\calC$ be a category which admits finite products, and let
$\bfA$ denote the category of functors $F: \calC \rightarrow \sSet$
which preserve finite products. Then $\bfA$ has the structure of a simplicial model category, where:
\begin{itemize}
\item[$(W)$] A natural transformation $\alpha: F \rightarrow F'$ of functors
is a weak equivalence in $\bfA$ if and only if $\alpha(C): F(C) \rightarrow F'(C)$ is a weak
homotopy equivalence of simplicial sets, for every $C \in \calC$. 
\item[$(F)$] A natural transformation $\alpha: F \rightarrow F'$ of functors
is a fibration in $\bfA$ if and only if $\alpha(C): F(C) \rightarrow F'(C)$ is a Kan fibration of simplicial sets, for every $C \in \calC$. 
%\item[$(C)$] A natural transformation $\alpha: F \rightarrow F'$ of functors
%is a cofibration in $\bfA$ if and only if it has the left lifting property with respect to every
%natural transformation $\beta: G \rightarrow G'$ which is both a fibration and a weak equivalence.
\end{itemize}
\end{proposition}

%\begin{proof}
%Using the small object argument, we can show that for any map $X \rightarrow Z$ in
%$\bfA$, there exist factorizations
%$$ X \stackrel{u}{\rightarrow} Y \stackrel{v}{\rightarrow} Z$$
%$$ X \stackrel{u'}{\rightarrow} Y' \stackrel{v'}{\rightarrow} Z$$
%where $v'$ is a trivial fibration, $u'$ is a cofibration, $v$ is a fibration, and $u$ has the left lifting property with respect to all fibrations. The only nontrivial point is to check that a map $u: X \rightarrow Y$ which has the left lifting property with respect to all fibrations is a trivial cofibration. 
%Clearly $u$ is cofibration. To prove that $u$ is a weak equivalence, we first introduce a functor
%$T: \Set_{\Delta}^{\calD} \rightarrow \Set_{\Delta}^{\calD}$, given by the formula 
%$$T(F)(D) = \Sing | F(D) |.$$
%Since the functors $\Sing$ and $||$ commute with products, the functor $T$ carries
%$\bfA$ into itself. Moreover, for each $F \in \bfA$, $T(F)$ is a fibrant object of $\bfA$ equipped with a weak equivalence $\theta_{F}: F \rightarrow T(F)$.

%Now suppose that $u: X \rightarrow Y$ has a the left lifting property with respect to all fibrations. Applying this to the diagram
%$$ \xymatrix{ X \ar[d]^{u} \ar[r]^{\theta_{X}} & T(X) \ar[d] \\
%Y \ar[r] \ar@{-->}[ur]^{\phi} & \ast, }$$
%we deduce the existence of a map $\phi: Y \rightarrow T(X)$ such that $u \circ \phi = \theta_{X}$.
%Now consider the diagram
%$$ \xymatrix{ X \ar[d]^{u} \ar[r]^{\theta_{X}^{\Delta^1}} & T(Y)^{\Delta^1} \ar[d] \\
%Y \ar[r]^{ T(u) \circ \phi, \theta_{Y} } \ar@{-->}[ur]^{\psi} & T(Y) \times T(Y). }$$
%Applying the lifting property once more, we deduce the existence of a homotopy
%$\psi$ from $T(u) \circ \phi$ to $\theta_{Y}$. It follows that, for each $D \in \calD$, we have a homotopy commutative diagram
%$$ \xymatrix{ X(D) \ar[d]^{u(D)} \ar[r]^{\theta_{X}(D)} & \Sing | X(D) | \ar[d] \\
%Y(D) \ar[r] \ar[ur]^{\phi(D)} \ar[r] & \Sing | Y(D) | }$$
%in the homotopy category $\calH$. Since the horizontal arrows are isomorphisms in $\calH$, we conclude that $u(D)$ is an isomorphism in $\calH$ as well (with inverse given by
%$\theta_{X}(D)^{-1} \circ \phi(D)$.

%The fact that $\bfA$ is a {\it simplicial} model category follows immediately from the descriptions of the fibrations and trivial fibrations in $\bfA$ ( see criterion $(2'')$ of Remark \ref{cyclor} ).
%\end{proof}

Suppose that $\calC$ and $\bfA$ are as in the statement of Proposition \ref{sutcoat}. Then we may regard $\bfA$ as a full subcategory of the category $\Set_{\Delta}^{\calC}$ of {\em all} functors
from $\calC$ to $\sSet$, which we regard as endowed with the projective model structure (so that fibrations and weak equivalences are given pointwise). The inclusion $G:\bfA \subseteq \Set_{\Delta}^{\calC}$ preserves fibrations and trivial fibrations, and therefore determines a Quillen adjunction
$$\Adjoint{F}{ \Set_{\Delta}^{\calC}}{\bfA}{G}.$$
(A more explicit description of the adjoint functor $F$ will be given below.) Our goal in this section is to prove the following result:

\begin{proposition}[Bergner]\label{trent}
Let $\calC$ be a small category which admits finite products, and let
$$ \Adjoint{ F }{ \Set_{\Delta}^{\calC} }{ \bfA }{ G}$$
be as above. Then the right derived functor
$$ RG: \h{\bfA} \rightarrow \h{\Set_{\Delta}^{\calC}}$$
is fully faithful, and an object $f \in \h{\Set_{\Delta}^{\calC}}$ belongs to the essential image of $RG$ if and only if $f$ preserves finite products up to weak homotopy equivalence.
\end{proposition}

\begin{corollary}\label{smokerr}
Let $\calC$ be a small category which admits finite products, and let $\bfA$ be as in Proposition \ref{trent}. Then the natural map $\Nerve( \bfA^{\degree} ) \rightarrow \calP_{\Sigma}( \Nerve(\calC)^{op} )$
is an equivalence of $\infty$-categories.
\end{corollary}

The proof of Proposition \ref{trent} is somewhat technical and will occupy the rest of this section. We begin by introducing some preliminaries.

\begin{notation}\label{bignote}
Let $\calC$ be a small category. We define a pair of categories $\Env(\calC) \subseteq \Env^{+}(\calC)$ as follows:\index{not}{EnvC@$\Env(\calC)$}\index{not}{EnvpC@$\Env^{+}(\calC)$}

\begin{itemize}
\item[$(i)$] An object of $\Env^{+}(\calC)$ is a pair $C = (J, \{ C_j\}_{j \in J} )$, where $J$ is a finite set and each $C_j$ is an object of $\calC$. The object $C$ belongs to $\Env(\calC)$ if and only if $J$ is nonempty. 

\item[$(ii)$] Given objects $C = (J, \{ C_j \}_{j \in J})$ and $C' = (J', \{ C'_{j'} \}_{j' \in J' } )$
of $\Env^{+}(\calC)$, a morphism $C \rightarrow C'$ consists of the following data:
\begin{itemize}
\item[$(a)$] A map $f: J' \rightarrow J$ of finite sets.
\item[$(b)$] For each $j' \in J'$, a morphism $C_{f(j')} \rightarrow C'_{j'}$ in the category $\calC$.
\end{itemize}
Such a morphism belongs to $\Env(\calC)$ if and only if $J$ and $J'$ are nonempty, and $f$ is surjective.
\end{itemize}

There is a fully faithful embedding functor $\theta: \calC \rightarrow \Env(\calC)$, given by
$C \mapsto ( \ast, \{ C \} )$. We can view $\Env^{+}(\calC)$ as the category
obtained from $\calC$ by freely adjoining finite products. In particular, if $\calC$ admits finite products, then $\theta$ admits a (product-preserving) left inverse $\phi^{+}_{\calC}$, given by the formula $( J, \{ C_j \}_{j \in J} ) \mapsto \prod_{j \in J} C_j.$. We let $\phi_{\calC}$ denote the restricton $\phi^{+}_{\calC} | \Env(\calC)$. 

Given a functor $\calF \in \Set_{\Delta}^{\calC}$, we let $E^{+}(\calF) \in
\Set_{\Delta}^{\Env^{+}(\calC)}$ denote the composition
$$ \Env^{+}(\calC) \stackrel{ \Env^{+}(\calF) }{\rightarrow} \Env^{+}(\sSet) \stackrel{ \phi^{+}_{\sSet}}{\rightarrow} \sSet$$
$$ (J, \{ C_j\}_{j \in J} ) \mapsto \prod f(C_j).$$
We let $E(\calF)$ denote the restriction $E^{+}(\calF) | \Env(\calC) \in \Set_{\Delta}^{\Env(\calC)}$.

If the category $\calC$ admits finite products, then we let
$L, L^{+}: \Set_{\Delta}^{\calC} \rightarrow \Set_{\Delta}^{\calC}$ denote the compositions 
$$ \Set_{\Delta}^{\calC} \stackrel{E}{\rightarrow} \Set_{\Delta}^{\Env(\calC)}
\stackrel{ (\phi_{\calC})_{!} }{\rightarrow} \Set_{\Delta}^{\calC}$$
$$ \Set_{\Delta}^{\calC} \stackrel{E^{+}}{\rightarrow} \Set_{\Delta}^{\Env^{+}(\calC)}
\stackrel{ (\phi_{\calC}^{+})_{!} }{\rightarrow} \Set_{\Delta}^{\calC},$$
where $(\phi_{\calC})_{!}$ and $(\phi_{\calC}^{+})_{!}$ indicate left Kan extension functors. 
There is a canonical isomorphism $\theta^{\ast} \circ E \simeq \id$, which induces a natural transformation $\alpha: \id \rightarrow L$. Let $\beta: L \rightarrow L^{+}$ indicate the natural transformation induced by the inclusion $\Env(\calC) \subseteq \Env^{+}(\calC)$.
\end{notation}

\begin{remark}\label{suppper}
Let $\calC$ be a small category. The functor $E^{+}: \Set_{\Delta}^{\calC} \rightarrow
\Set_{\Delta}^{\Env^{+}(\calC)}$ is fully faithful, and has a left adjoint given by $\theta^{\ast}$. 
\end{remark}

We begin by constructing the left adjoint which appears in the statement of Proposition \ref{trent}.

\begin{lemma}\label{toughluff}
Let $\calC$ be a simplicial category which admits finite products, and let $\calF \in \Set_{\Delta}^{\calC}$. Then:
\begin{itemize}
\item[$(1)$] The object $L^{+}(\calF) \in \Set_{\Delta}^{\calC}$ is product-preserving.
\item[$(2)$] If $\calF' \in \Set_{\Delta}^{\calC}$ is product-preserving, then composition with $\beta \circ \alpha$ induces an isomorphism of simplicial sets
$$ \bHom_{ \Set_{\Delta}^{\calC} }(L^{+}(\calF), \calF') \rightarrow \bHom_{\Set_{\Delta}^{\calC}}( \calF, \calF').$$
\end{itemize}
\end{lemma}

\begin{proof}
Suppose given a finite collection of objects $\{ C_1, \ldots, C_n \}$ in $\calC$, and let
$$u: L^{+}(\calF)( C_1 \times \ldots \times C_n) \rightarrow L^{+}(\calF)(C_1) \times \ldots \times L^{+}(\calF)(C_n)$$ be the product of the projection maps. We wish to show that $u$ is an isomorphism of simplicial sets. We will give an explicit construction of an inverse to $u$. For
$C \in \calC$, we let $\Env^{+}(\calC)_{/C}$ denote the fiber product
$\Env^{+}(\calC) \times_{\calD} \calC_{/C}$. For $1 \leq i \leq n$, let $\calG_i$ denote the restriction of $E^{+}(\calF)$ to $\Env^{+}(\calC)_{/C_i}$, and let 
$$ \calG: \prod \Env^{+}(\calD)_{/C_i} \rightarrow \sSet$$
be the product of the functors $\calG_i$. We observe that $L^{+}(\calF)(C_i) \simeq \varinjlim( \calG_i )$, so that the product $\prod L^{+}(\calF)(C_i) \simeq \varinjlim( \calG)$. We now observe that the formation of products in $\calE^{+}(\calC)$ gives an identification of $\calG$ with the composition
$$ \prod \Env^{+}(\calC)_{/C_i} \rightarrow \Env^{+}(\calD)_{/ C_1 \times \ldots \times C_n }
\stackrel{ E^{+}(\calF) }{\rightarrow} \sSet.$$
We thereby obtain a morphism $$v: \varinjlim(\calG) \rightarrow \varinjlim( E^{+}(\calF) | \Env^{+}(\calD)_{/  C_1 \times \ldots \times C_n} \simeq L^{+}(\calF)( C_1 \times \ldots \times C_n).$$
It is not difficult to check that $v$ is an inverse to $u$. 

We observe that $(2)$ is equivalence to the assertion that composition with $\theta^{\ast}$ induces an isomorphism
$$ \bHom_{ \Set_{\Delta}^{\Env^{+}(\calC)} }( E^{+}(\calF), (\phi_{\calC}^{+})^{\ast}(\calF'))
\rightarrow \bHom_{\Set_{\Delta}^{\calC}}( \calF, \calF').$$
Because $\calG$ is product-preserving, there is a natural isomorphism
$( \phi_{\calC}^{+})^{\ast}(\calF') \simeq E^{+}(\calF')$. The desired result now follows from Remark \ref{suppper}.
\end{proof}

It follows that the functor $L^{+}: \Set_{\Delta}^{\calC} \rightarrow \Set_{\Delta}^{\calC}$ factors through $\bfA$, and can be identified with a left adjoint to the inclusion
$\bfA \subseteq \Set_{\Delta}^{\calC}$. In order to prove Proposition \ref{trent}, we need to be able to compute the functor $L^{+}$. We will do this in two steps: first, we show that (under mild hypotheses), the natural transformation $L \rightarrow L^{+}$ is a weak equivalence. Second, we will see that the colimit defining $L$ is actually a homotopy colimit, and therefore has good properties. More precisely, we have the following pair of lemmas, whose proofs will be given at the end of this section.

\begin{lemma}\label{toughstuff}
Let $\calC$ be a small category which admits finite products, and let $\calF \in \Set_{\Delta}^{\calC}$ be a functor which carries the final object of $\calC$ to a contractible Kan complex $K$. Then the canonical map $\beta: L(\calF) \rightarrow L^{+}(\calF)$ is a weak equivalence in $\Set_{\Delta}^{\calC}$.
\end{lemma}

\begin{lemma}\label{trentlem}
Let $\calC$ be a small simplicial category. If $\calF$ is a projectively cofibrant object of $\Set_{\Delta}^{\calC}$, then $E(\calF)$ is a projectively cofibrant object of $\Set_{\Delta}^{\Env(\calC)}$. 
\end{lemma}

We are now almost ready to give the proof of Proposition \ref{trent}. The essential step is contained in the following result:

\begin{lemma}\label{presut}
Let $\calC$ be a simplicial category finite admits finite products, and let
$$ \Adjoint{ F }{ \Set_{\Delta}^{\calC} }{ \bfA }{ G}$$
be as in the statement of Proposition \ref{trent}. Then:
\begin{itemize}
\item[$(1)$] The functors $F$ and $G$ are Quillen adjoints.
\item[$(2)$] If $\calF \in \Set_{\Delta}^{\calC}$ is projectively cofibrant and weakly product preserving, then the unit map $\calF \rightarrow (G \circ F)(\calF)$ is a weak equivalence.
\end{itemize}
\end{lemma}

\begin{proof}
Assertion $(1)$ is obvious, since $G$ preserves fibrations and trivial cofibrations.
It follows that $F$ preserves weak equivalences between projectively cofibrant objects. 
Let $K \in \sSet$ denote the image under $\calF$ of the final object of $\calD$. 
In proving $(2)$, we are free to replace $\calF$ by any weakly equivalent diagram which is also projectively cofibrant. Choosing a fibrant replacement for $\calF$, we may suppose that $K$ is a Kan complex. Since $\calF$ is weakly product preserving, $K$ is contractible.

In view of Lemma \ref{toughluff}, we can identify the composition $G \circ F$ with $L^{+}$ and the unit map with the composition
$$ \calF \stackrel{\alpha}{\rightarrow} L(\calF) \stackrel{\beta}{\rightarrow} L^{+}(\calF).$$
Lemma \ref{toughstuff} implies that $\beta$ is a weak equivalence. Consequently, it will suffice to show that $\alpha$ is a weak equivalence.

We recall the construction of $\alpha$. Let $\theta: \calC \rightarrow \Env(\calC)$ be as in Notation \ref{bignote}, so that there is a canonical isomorphism $\calF \simeq \theta^{\ast} E(\calF)$. This
isomorphism induces a natural transformation $\overline{\alpha}: \theta_{!} \calF \rightarrow E(\calF)$. The functor $\alpha$ is obtained from $\overline{\alpha}$ by applying the functor
$(\phi_{\calC})_{!}$, and identifying $((\phi_{\calC})_{!} \circ \theta_{!} )(\calF)$ with $\calF$. 
We observe that $(\phi_{\calC})_{!}$ preserves weak equivalences between projectively cofibrant objects. Since $\theta_{!}$ preserves projective cofibrations, $\theta_{!} \calF$ is projectively cofibrant. Lemma \ref{trentlem} asserts that $E(\calF)$ is projectively cofibrant. Consequently, it will suffice to prove that $\overline{\alpha}$ is a weak equivalence in $\Set_{\Delta}^{\Env(\calC)}$. Unwinding the definitions, this reduces to the condition that $\calF$ be weakly compatible with (nonempty) products.
\end{proof}

\begin{proof}[Proof of Proposition \ref{trent}]
Lemma \ref{presut} shows that $(F,G)$ is a Quillen adjunction. To complete the proof, we must show:

\begin{itemize}
\item[$(i)$] The counit transformation $LF \circ RG \rightarrow \id$ is an isomorphism of functors
from the homotopy category $\h{\bfA}$ to itself.
\item[$(ii)$] The essential image of $RG: \h{\bfA} \rightarrow \h{ \Set_{\Delta}^{\calC} }$ consists
precisely of those functors which are weakly product-preserving.
\end{itemize}

We observe that $G$ preserves weak equivalences, so we can identify $RG$ with $G$. Since $G$ also detects weak equivalences, $(i)$ will follow if we can show that the induced transformation
$\theta: G \circ LF \circ G \rightarrow G$ is an isomorphism of functors from the homotopy
category $\h{\bfA}$ to itself. This transformation has a right inverse, given by composing the unit transformation $\id \rightarrow G \circ LF$ with $G$. Consequently, $(i)$ follows immediately from Lemma \ref{presut}. 

The image of $G$ consists precisely of the product-preserving diagrams $\calC \rightarrow \sSet$; it follows immediately that every diagram in the essential image of $G$ is weakly product preserving. Lemma \ref{presut} implies the converse: every weakly product preserving functor belongs to the essential image of $G$. This proves $(ii)$.
\end{proof}

\begin{remark}
Proposition \ref{trent} can be generalized to the situation where $\calC$ is a {\em simplicial} category which admits finite products. We leave the necessary modifications to the reader.
\end{remark}

It remains to prove Lemmas \ref{trentlem} and \ref{toughstuff}.

\begin{proof}[Proof of Lemma \ref{trentlem}]
For every object $C \in \calC$ and every simplicial set $K$, we let 
$\calF^{K}_{C} \in \Set_{\Delta}^{\calC}$ denote the functor given by the formula
$\calF^{K}_{C}(D) = \bHom_{\calC}( C, D) \times K$. A cofibration $K \rightarrow K'$ induces
a projective cofibration $\calF^{K}_{C} \rightarrow \calF^{K'}_{C}$. We will refer to a projective cofibration of this form as a {\em generating projective cofibration}.

The small object argument implies that if $\calF \in \Set_{\Delta}^{\calC}$, then there is a transfinite sequence
$$ \calF_0 \subseteq \calF_{1} \subseteq \ldots \subseteq \calF_{\alpha}$$
with the following properties:
\begin{itemize}
\item[$(a)$] The functor $\calF_0: \calD \rightarrow \sSet$ is constant, with value $\emptyset$.
\item[$(b)$] If $\lambda \leq \alpha$ is a limit ordinal, then $\calF_{\lambda} = \bigcup_{ \beta < \lambda} \calF_{\beta}$.
\item[$(c)$] For each $\beta < \alpha$, the inclusion $\calF_{\beta} \subseteq \calF_{\beta+1}$ is a pushout of a generating projective cofibration.
\item[$(d)$] The functor $\calF$ is a retract of $\calF_{\alpha}$.
\end{itemize}

The functor $\calG \mapsto E(\calG)$ preserves initial objects, filtered colimits, and retracts. Consequently, to show that $E(\calF)$ is projectively cofibrant, it will suffice to prove the following 
assertion:
\begin{itemize}
\item[$(\ast)$] Suppose given a cofibration $K \rightarrow K'$ of simplicial sets and a pushout diagram
$$ \xymatrix{ \calF^{K}_{C} \ar[d] \ar[r] & \calG \ar[d] \\
\calF^{K'}_{C} \ar[r] & \calG' }$$
in $\Set_{\Delta}^{\calC}$. If $E(\calG)$ is projectively cofibrant, then $E(\calG')$ is projectively cofibrant.
\end{itemize}

To prove this, we will need to analyze the structure of $E(\calG')$. Given an object
$C' = ( J, \{ C'_{j} \}_{j \in J} )$ of $\Env(\calC)$, we have
$$E(\calG')( C' ) = \prod_{j \in J} ( \calG( C'_{j} ) \coprod_{ K \times \bHom_{\calC}(C,C'_j) }
( K' \times \bHom_{\calC}(C, C'_j) ). $$
Let $\sigma: \Delta^n \rightarrow E(\calG')(C')$ be a simplex, and let $J_{\sigma} \subseteq J$
be the collection of all indices $j$ for which the corresponding simplex 
$\sigma(j): \Delta^n \rightarrow \calG'(C'_j)$ does not factor through $\calG(C'_j)$. 
In this case, we can identify $\sigma(j)$ with an $n$-simplex of $K'$ which does not belong to
$K$. We will say that $\sigma$ is of {\it index $\leq k$} if the set $\{ \sigma(j) : j \in J_{\sigma} \}$ has cardinality $\leq k$. Note that $\sigma$ can be of index smaller than the cardinality of
$J_{\sigma}$, since it is possible for $\sigma(j) = \sigma(j') \in \bHom_{\sSet}(\Delta^n, K')$ even if $j \neq j'$.

Let $E(\calG')^{(k)}(C')$ be the full simplicial subset of $E(\calG')(C')$ spanned by those simplices which are of index $\leq k$. It is easy to see that that $E(\calG')^{(k)}(C')$ depends functorially in $C'$, so we can view $E(\calG')^{(k)}$ as an object of $\Set_{\Delta}^{\Env(\calC)}$. We observe that
$$ E(\calG) \simeq E(\calG')^{(0)} \subseteq E(\calG')^{(1)} \subseteq \ldots $$
and that the union of this sequence is $E(\calG')$. Consequently, it will suffice to prove that
each of the inclusions $E(\calG')^{(k-1)} \subseteq E(\calG')^{(k)}$ is a projective cofibration.

First, we need a bit of notation. Let us say that a simplex of ${K'}^{k}$ is {\it new} if it consists of $k$ distinct simplices of $K'$, none of which belong to $K$. We will say that a simplex of ${K'}^{k}$ is {\it old} if it is not new. The collection of old simplices of ${K'}^{k}$ determine a simplicial subset which we will denote by ${K'}^{(k)}$. We define a functor $\psi: \Env(\calC) \rightarrow \Env(\calC)$ by the formula
$$\psi( J, \{ C'_{j} \}_{j \in J} ) = ( J \cup \{ 1, \ldots, k \}, \{ C'_{j} \}_{j \in J} \cup \{ C \}_{\{1 \ldots k \}} ).$$
Let $\psi^{\ast}: \Set_{\Delta}^{\Env(\calC)} \rightarrow \Set_{\Delta}^{\Env(\calC)}$ be given by composition with $\psi$, and let $\psi_{!}: \Set_{\Delta}^{\Env(\calC)} \rightarrow \Set_{\Delta}^{\Env(\calC)}$ be a left adjoint to $\psi^{\ast}$ (a functor of left Kan extension).
Since $\psi^{\ast}$ preserves projective fibrations and weak equivalences, $\psi_{!}$ preserves projective cofibrations.

Recall that $\Set_{\Delta}^{\Env(\calC)}$ is tensored over the category of simplicial sets:
Given an object $\calM \in \Set_{\Delta}^{\Env(\calC)}$ and a simplicial set $A$, we let $\calM \otimes A \in \Set_{\Delta}^{\Env(\calC)}$ be defined by the formula
$( \calM \otimes A)(D') = \calM(D') \times A$. If $\calM$ is projectively cofibrant, then the operation
$\calM \mapsto \calM \otimes A$ preserves cofibrations in $A$.

There is an obvious map $E(\calG) \otimes {K'}^k \rightarrow \psi^{\ast} E(\calG')^{(k)}$, which restricts to a map $E(\calG) \otimes {K'}^{(k)} \rightarrow \psi^{\ast} E(\calG')^{(k-1)}$. Passing to adjoints, we obtain a commutative diagram
$$ \xymatrix{ \psi_{!}( E(\calG) \otimes {K'}^{(k)} ) \ar[d] \ar[r] & E(\calG')^{(k-1)} \ar[d] \\
\psi_{!}( E(\calG) \otimes {K'}^k ) \ar[r] & E(\calG')^{(k)}. }$$
An easy computation shows that this diagram is coCartesian.
Since $E(\calG)$ is projectively cofibrant, the above remarks imply that the left vertical map is a projective cofibration. It follows that the right vertical map is a projective cofibration as well, which completes the proof.
\end{proof}

The proof of Lemma \ref{toughstuff} is somewhat more difficult, and will require some preliminaries.

\begin{notation}
Let $\Mult: \Set \rightarrow \Set$ be the covariant functor which associates to each set
$S$ the collection of $\Mult(S)$ of nonempty finite subsets of $S$. If $K$ is a simplicial set, we let
$\Multi(K)$ denote the composition of $K$ with $\Mult$, so that an $m$-simplex of $\Multi(K)$ is
a finite nonempty collection of $m$-simplices of $K$.\index{not}{MultiK@$\Multi(K)$}
\end{notation}

\begin{lemma}\label{stuffem}
Let $K$ be a finite simplicial set, let $X \subseteq \Multi(K^{\triangleright}) \times \Delta^n$ be a simplicial subset with the following properties:
\begin{itemize}
\item[$(i)$] The projection $X \rightarrow \Delta^n$ is surjective. 
\item[$(ii)$] If $\tau = (\tau', \tau''): \Delta^m \rightarrow \Multi(K^{\triangleright}) \times \Delta^n$ belongs to $X$, and $\tau' \subseteq \overline{\tau}'$ as subsets of $\Hom_{\sSet}( \Delta^m, K^{\triangleright})$, then $(\overline{\tau}',\tau''): \Delta^m \rightarrow \Multi(K^{\triangleright}) \times \Delta^n$ belongs to $X$.
\end{itemize}
Then $X$ is weakly contractible.
\end{lemma}

\begin{proof}
Let $X' \subseteq X$ be the simplicial subset spanned by those simplices
$\tau = (\tau', \tau''): \Delta^m \rightarrow \Multi(K^{\triangleright}) \times \Delta^n$ which factor through $X$, and for which $\tau' \subseteq \Hom_{\sSet}( \Delta^m, K^{\triangleright})$ includes the constant simplex at the cone point of $K^{\triangleright}$. Our first step is to show that $X'$ is a deformation retract of $X$. More precisely, we will construct a map
$$ h: \Multi(K^{\triangleright}) \times \Delta^n \times \Delta^1 \rightarrow 
\Multi(K^{\triangleright}) \times \Delta^n$$ with the following properties:
\begin{itemize}
\item[$(a)$] The map $h$ carries $X \times \Delta^1$ into $X$ and $X' \times \Delta^1$ into $X'$.
\item[$(b)$] The restriction $h | \Multi(K^{\triangleright}) \times \Delta^n \times \{0\}$ is the identity map.
\item[$(c)$] The restriction $h| X \times \{1\}$ factors through $X'$.
\end{itemize}
The map $h$ will be the product of a map $h': \Multi(K^{\triangleright}) \times \Delta^1 \rightarrow \Multi(K^{\triangleright} )$ with the identity map on $\Delta^n$. To define $h'$, we consider an arbitrary simplex $\tau: \Delta^m \rightarrow \Multi(K^{\triangleright}) \times \Delta^1$, corresponding to a subset $S \subseteq \Hom_{\sSet}( \Delta^m, K^{\triangleright} )$ and a decomposition
$[m] = \{ 0, \ldots, i\} \cup \{i+1, \ldots, m\}$. The subset $h'(\tau) \subseteq \Hom_{\sSet}( \Delta^m, K^{\triangleright})$ is defined as follows: an arbitrary simplex $\sigma: \Delta^m \rightarrow K^{\triangleright}$ belongs to $h'(\tau)$ if there exists $\sigma' \in S$, $i < j \leq n$ such that
$\sigma' | \Delta^{ \{0, \ldots, j-1 \} } = \sigma| \Delta^{ \{0, \ldots, j-1 \} }$, and $\sigma | \Delta^{ \{j, \ldots, m \} }$ is constant at the cone point of $K^{\triangleright}$. It is easy to check that $h'$ has the desired properties.

It remains to prove that $X'$ is weakly contractible. At this point, it is convenient to work in the setting of {\em semisimplicial sets}: that is, we will ignore the degeneracy operations. Let
$X''$ be the semisimplicial subset of $\Multi(K^{\triangleright}) \times \Delta^n$ spanned by those maps $\tau = (\tau', \tau''): \Delta^m \rightarrow \Multi(K^{\triangleright}) \times \Delta^n$ for which
$\tau' = \Hom_{\sSet}( \Delta^m, K^{\triangleright} )$ (we observe that $X''$ is not stable under the degeneracy operators on $\Multi(K^{\triangleright}) \times \Delta^n$). Assumptions $(i)$ and $(ii)$ guarantee that $X'' \subseteq X'$. Moreover, the projection $X \rightarrow \Delta^n$ induces an isomorphism of semisimplicial sets $X'' \rightarrow \Delta^n$. Consequently, it will suffice to prove that $X''$ is a deformation retract of $X'$. 

The proof now proceeds by a variation on our earlier construction. Namely, we will define a map 
of semisimplicial sets
$$ g: \Multi(K^{\triangleright}) \times \Delta^n \times \Delta^1 \rightarrow 
\Multi(K^{\triangleright}) \times \Delta^n$$ with the following properties:
\begin{itemize}
\item[$(a)$] The map $g$ carries $X' \times \Delta^1$ into $X'$ and $X'' \times \Delta^1$ into $X''$.
\item[$(b)$] The restriction $g |X' \times \{1\}$ is the identity map.
\item[$(c)$] The restriction $g| \Multi(K^{\triangleright}) \times \Delta^n \times \{0\}$ factors through $X'$.
\end{itemize}
As before, $g$ is the product of a map $g': \Multi(K^{\triangleright}) \times \Delta^1 \rightarrow \Multi(K^{\triangleright})$ with the identity map on $\Delta^n$. To define $g'$, we consider an arbitrary simplex $\tau: \Delta^m \rightarrow \Multi(K^{\triangleright}) \times \Delta^1$, corresponding to a subset $S \subseteq \Hom_{\sSet}( \Delta^m, K^{\triangleright} )$ and a decomposition
$[m] = \{ 0, \ldots, i\} \cup \{i+1, \ldots, m\}$. We let $g'(\tau) \subseteq \Hom_{\sSet}(\Delta^m, K^{\triangleright} ) = S \cup S'$, where $S'$ is the collection of all simplices $\sigma: \Delta^m \rightarrow K^{\triangleright}$ such that $\sigma | \Delta^{ \{i+1, \ldots, m\} }$ is the constant map at the cone poine of $K^{\triangleright}$. It is readily checked that $g'$ has the desired properties.
\end{proof}

\begin{lemma}\label{toughfluff}
Let $K$ be a contractible Kan complex, and let $X \subseteq \Multi(K) \times \Delta^n$ be a simplicial subset with the following properties:
\begin{itemize}
\item[$(i)$] The projection $X \rightarrow \Delta^n$ is surjective.
\item[$(ii)$] If $\tau = (\tau', \tau''): \Delta^m \rightarrow \Multi(K) \times \Delta^n$ belongs to
$X$, and $\tau' \subseteq \overline{\tau}'$ as subsets of $\Hom_{\sSet}( \Delta^m, K)$, then
$(\overline{\tau}',\tau''): \Delta^m \rightarrow \Multi(K) \times \Delta^n$ belongs to $X$.
\end{itemize}
Then $X$ is weakly contractible.
\end{lemma}

\begin{proof}
It will suffice to show that for every finite simplicial subset $X' \subseteq X$, the inclusion of $X'$ into $X$ is weakly nullhomotopic. Enlarging $X'$ if necessary, we may assume that
$X' = (\Multi(K') \times \Delta^n) \cap X$, where $K'$ is a finite simplicial subset of $K$. By further enlargement, we may suppose that the map $X' \rightarrow \Delta^n$ is surjective. Since $K$ is a contractible Kan complex, the inclusion $K' \subseteq K$ extends to a map $i: {K'}^{\triangleright} \rightarrow K$. Let $\overline{X} \subseteq \Multi( {K'}^{\triangleright} ) \times \Delta^n$ denote
the inverse image of $X$. Then the inclusion $X' \subseteq X$ factors through $\overline{X}$, and Lemma \ref{stuffem} implies that $\overline{X}$ is weakly contractible.
\end{proof}

\begin{proof}[Proof of Lemma \ref{toughstuff}]
Fix an object $C \in \calC$. The simplicial set $L(\calF)(C)$ can be described as follows:
\begin{itemize}
\item[$(\ast)$] For every $n \geq 1$, every map $f: C_1 \times \ldots \times C_n \rightarrow C$ in $\calC$, and every collection of simplices $\{ \sigma_i: \Delta^k \rightarrow \calF(C_i) \}$, there is a simplex $f( \{ \sigma_i \}): \Delta^k \rightarrow L(\calF)(C)$.
\end{itemize}
The simplices $f( \{ \sigma_i \} )$ satisfy relations which are determined by morphisms in the simplicial category $\Env(\calC)$.

To every $k$-simplex $\tau: \Delta^k \rightarrow L(\calF)(D)$, we can associate a nonempty finite subset $S_{\tau} \subseteq \Hom_{\sSet}( \Delta^k, K)$. If $\tau = f( \{ \sigma_i \})$, we assign the set of images of the simplices $\sigma_i$ under the canonical maps $\calF(C_i) \rightarrow \calF(1) = K$. It is easy to see that $S_{\tau}$ is independent of the representation $f( \{ \sigma_i \})$ chosen for $\tau$, and depends functorially on $\tau$. Consequently, we obtain a map of simplicial sets
$L(\calF)(C) \rightarrow \Multi(K)$. Moreover, this map has the following properties:
\begin{itemize}
\item[$(i)$] The product map $\beta': L(\calF)(C) \rightarrow \Multi(K) \times L^{+}(\calF)(C)$ is a monomorphism of simplicial sets. 
\item[$(ii)$] If a $k$-simplex $\tau = (\tau', \tau'') : \Delta^k \rightarrow \Multi(K) \rightarrow L^{+}(\calF)(C)$ belongs to the image of $\beta$, and 
$\tau' \subseteq \overline{\tau}'$ as finite subsets of $\Hom_{\sSet}(\Delta^k, K)$, then
$(\overline{\tau}', \tau''): \Delta^k \rightarrow \Multi(K) \times L^{+}(\calF)(C)$ belongs to the image
of $\beta'$. 
\end{itemize}

We wish to show that $\beta: L(\calF)(C) \rightarrow L^{+}(\calF)(C)$ is a weak homotopy equivalence. It will suffice to show that for every simplex $\Delta^k \rightarrow L^{+}(\calF)(C)$, the fiber product $L(\calF)(C) \times_{ L^{+}(\calF)(C) } \Delta^k$ is weakly contractible. In view of $(i)$, we can identify this fiber product with a simplicial subset $X \subseteq \Delta^k \times \Multi(K)$.
The surjectivity of $\beta$ and condition $(ii)$ imply that $X$ satisfies the hypotheses of Lemma \ref{toughfluff}, so that $X$ is weakly contractible as desired.
\end{proof}

The model category $\bfA$ appearing in Proposition \ref{sutcoat} is very well suited to certain calculations, such as the formation of homotopy colimits of simplicial objects. The following result provides a precise formulation of this idea:

\begin{proposition}\label{eggers}
Let $\calC$ be a category which admits finite products, and $\bfA \subseteq \Set_{\Delta}^{\calC}$,
$\calA \subseteq \Set^{\calC}$ the full subcategories spanned by the product-preserving functors. 
Let $\calF: \cDelta^{op} \rightarrow \bfA$ be a simplicial object of $\bfA$, which we can identify with a {\em bisimplicial} object $F: \cDelta^{op} \times \cDelta^{op} \rightarrow \calA$. Composition with the diagonal $$ \cDelta^{op} \rightarrow \cDelta^{op} \times \cDelta^{op} \stackrel{F}{\rightarrow} \calA$$ gives a simplicial object of $\calA$, which we can identify with an object $| \calF| \in \bfA$. Then the homotopy colimit of $\calF$ is canonically isomorphic to $|\calF|$ in the homotopy category $\h{\bfA}$.
\end{proposition}

The proof requires the following lemma:

\begin{lemma}\label{urplus}
Let $\calC$ be a category which admits finite products, and $\bfA \subseteq \Set_{\Delta}^{\calC}$
be the full subcategory spanned by the product preserving functors. For every object $C \in \calC$, the evaluation map $\bfA \rightarrow \sSet$ preserves homotopy colimits of simplicial objects.
\end{lemma}

\begin{proof}
In view of Corollary \ref{smokerr} and Theorem \ref{colimcomparee}, it will suffice to show that the evaluation functor $\calP_{\Sigma}( \Nerve(\calC)^{op} ) \rightarrow \sSet$ preserves
$\Nerve(\cDelta)^{op}$-indexed colimits. This follows from Proposition \ref{utut}, since
$\Nerve(\cDelta)^{op}$ is sifted (Lemma \ref{bball3}).
\end{proof}

\begin{proof}[Proof of Proposition \ref{eggers}]
Since $\bfA$ is a combinatorial simplicial model category, Corollary \ref{twinner} implies the existence of a canonical map $\gamma: \hocolim \calF \rightarrow | \calF |$ in the homotopy category $\h{\bfA}$; we wish to prove that $\gamma$ is an isomorphism. To prove this, it will suffice to show that
the induced map $\gamma_{C}: ( \hocolim \calF)(C) \rightarrow | \calF|(C)$ is an isomorphism in the homotopy category of simplicial sets, for each object $C \in \calC$. This map fits into a commutative diagram
$$ \xymatrix{ \hocolim( \calF(C) ) \ar[r]^{\gamma'_{C}} \ar[d] & | \calF(C) | \ar[d] \\
\hocolim(\calF)(C) \ar[r] & | \calF|(C).}$$
The left vertical map is an isomorphism in the homotopy category of simplicial sets by Lemma \ref{urplus}, the right vertical map is evidently an isomorphism, and the map $\gamma'_{C}$ is an isomorphism in the homotopy category by Example \ref{swupt}; it follows that $\gamma_{C}$ is also an isomorphism, as desired.
\end{proof}

