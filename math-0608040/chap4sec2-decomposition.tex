\section{Techniques for Computing Colimits}\label{c4s2}

\setcounter{theorem}{0}

In this section, we will introduce various techniques for computing, analyzing, and manipulating colimits. We begin in \S \ref{quasilimit2} by introducing a variant on the join construction of \S \ref{langur}. The new join construction is (categorically) equivalent to the version we are already familiar with, but has better formal behavior in some contexts. For example, they permit us to define a {\em parametrized} version of overcategories and undercategories, which we will analyze in \S \ref{consweet}.

In \S \ref{quasilimit1}, we address the following question: given a diagram $p: K \rightarrow \calC$ and a decomposition of $K$ into ``pieces'', how is the colimit
$\injlim(p)$ related to the colimits of those pieces? For example, if $K = A \cup B$, then it seems reasonable expect an equation of the form
$$\colim(p) = ( \colim p|A ) \coprod_{ \colim(p| A \cap B)} (\colim p|B).$$ 
Of course there are many variations on this theme; we will lay out a general framework in \S \ref{quasilimit1}, and apply it to specific situations in \S \ref{coexample}.

Although the $\infty$-categorical theory of colimits is elegant and powerful, it can be be difficult to work with because the colimit $\colim(p)$ of a diagram $p$ is only well-defined up to equivalence.
This problem can sometimes be remedied by working in the more rigid theory of model categories, where the notion of $\infty$-categorical colimit should be replaced by the notion of {\it homotopy colimit} (see \S \ref{quasilimit3}). In order to pass smoothly between these two settings, we need to know that the $\infty$-categorical theory of colimits agrees with the more classical theory of homotopy colimits. A precise statement of this result (Theorem \ref{colimcomparee}) will be formulated and proven in \S \ref{quasilimit4}. 

\subsection{Alternative Join and Slice Constructions}\label{quasilimit2}

In \S \ref{join}, we introduced the {\em join} functor $\star$ on simplicial sets. In \cite{joyalnotpub}, Joyal introduces a closely related operation $\diamond$ on simplicial sets. This operation is equivalent to $\star$ (Proposition \ref{rub3}) but is more technically convenient in certain contexts. In this section we will review the definition of the operation $\diamond$ and to establish some of its basic properties (see also \cite{joyalnotpub} for a discussion).

\begin{definition}[\cite{joyalnotpub}]\index{not}{diamond@$\diamond$}
Let $X$ and $Y$ be simplicial sets. The simplicial set $X \diamond Y$ is defined to be pushout
$$ X \coprod_{ X \times Y \times \{0\} } (X \times Y \times \Delta^1) \coprod_{X \times Y \times \{1\} } Y.$$
\end{definition}

We note that since $X \times Y \times (\bd \Delta^1) \rightarrow X \times Y \times \Delta^1$ is a monomorphism, the pushout diagram defining $X \diamond Y$ is a homotopy pushout in $\sSet$ (with respect to the Joyal model structure). Consequently, we deduce that categorical equivalences $X \rightarrow X'$, $Y \rightarrow Y'$ induce a categorical equivalence $X \diamond Y \rightarrow X' \diamond Y'$.

The simplicial set $X \diamond Y$ admits a map $p: X \diamond Y \rightarrow \Delta^1$, with
$X \simeq p^{-1} \{0\}$ and $Y \simeq p^{-1} \{1\}$. Consequently, there is a unique map
$X \diamond Y \rightarrow X \star Y$ which is compatible with the projection to $\Delta^1$ and induces the identity maps on $X$ and $Y$.

\begin{proposition}\label{rub3}
For any simplicial sets $X$ and $Y$, the natural map $\phi: X
\diamond Y \rightarrow X \star Y$ is a categorical equivalence.
\end{proposition}

\begin{proof}
Since both sides are compatible with the formation of filtered
colimits in $X$, we may suppose that $X$ contains only finitely
many nondegenerate simplices. If $X$ is empty, then $\phi$ is an isomorphism and the result is
obvious. Working by induction on the dimension of $X$ and the
number of nondegenerate simplices in $X$, we may write
$$ X = X' \coprod_{\bd \Delta^{n}} \Delta^n,$$ and we may assume
that the statement is known for the pairs $(X',Y)$ and $(\bd
\Delta^n, Y)$. Since the Joyal model structure on $\sSet$ is
left-proper, we have a map of homotopy pushouts
$$ (X' \diamond Y) \coprod_{ \bd \Delta^n \diamond Y } (\Delta^n
\diamond Y) \rightarrow (X' \star Y) \coprod_{ \bd \Delta^n
\star Y} ( \Delta^n \star Y),$$ and we are therefore reduced
to proving the assertion in the case where $X = \Delta^n$.
The inclusion $$ \Delta^{ \{0,1\} } \coprod_{ \{1\} } \ldots \coprod_{ \{n-1\} }
\Delta^{ \{n-1,n\} } \subseteq \Delta^n$$ is inner anodyne. Thus, if
$n > 1$, we can conclude by induction. Thus we may suppose that
$X = \Delta^0$ or $X = \Delta^1$. By a similar argument, we may reduce to the case where
$Y = \Delta^0$ or $Y = \Delta^1$. The desired result now follows from an explicit calculation.
\end{proof}

\begin{corollary}\label{gyyyt}
Let $S \rightarrow T$ and $S' \rightarrow T'$ be categorical
equivalences of simplicial sets. Then the induced map
$$ S \star S' \rightarrow T \star T'$$ is a categorical
equivalence.
\end{corollary}

\begin{proof}
This follows immediately from Proposition \ref{rub3}, since the operation $\diamond$
has the desired property.
\end{proof}

\begin{corollary}\label{diamond3}
Let $X$ and $Y$ be simplicial sets. Then the natural map
$$ \sCoNerve[X \star Y] \rightarrow \sCoNerve[X] \star
\sCoNerve[Y]$$ is an equivalence of simplicial categories.
\end{corollary}

\begin{proof}
Using Corollary \ref{gyyyt}, we may reduce to the case where $X$ and $Y$ are $\infty$-categories.
We note that $\sCoNerve[X \star Y]$ is a correspondence from $\sCoNerve[X]$ to
$\sCoNerve[Y]$. To complete the proof, it suffices to show that 
$\bHom_{\sCoNerve[X \star Y]}(x,y)$ is weakly contractible, for any
pair of objects $x \in X$, $y \in Y$. Since $X \star Y$ is an $\infty$-category, we can apply Theorem \ref{biggie} to deduce that the mapping space $\bHom_{\sCoNerve[X \star Y]}(x,y)$ is weakly homotopy equivalent to $\Hom^{\rght}_{X \star Y}(x,y)$, which consists of a single point.
\end{proof}

For fixed $X$, the functor
$$ Y \mapsto X \diamond Y$$
$$ \sSet \rightarrow (\sSet)_{X/}$$
preserves all colimits. By the adjoint functor theorem (or by direct construction), this functor has a right adjoint $$ ( p: X \rightarrow \calC) \mapsto \calC^{p/}. $$
Since the functor $Y \mapsto X \diamond Y$ preserves cofibrations and categorical equivalences, we deduce that
the passage from $\calC$ to $\calC^{p/}$ preserves categorical fibrations and categorical equivalences between $\infty$-categories. Moreover, Proposition \ref{rub3} has the following consequence:\index{not}{calC^p/@$\calC^{p/}$}

\begin{proposition}\label{certs}
Let $\calC$ be an $\infty$-category, and let $p: X \rightarrow \calC$ be a diagram. Then the natural map $$ \calC_{p/} \rightarrow \calC^{p/}$$ is an equivalence of $\infty$-categories.
\end{proposition}

According to Definition \ref{defcolim}, a colimit for a diagram $p: X \rightarrow \calC$ is an initial object of the $\infty$-category $\calC_{p/}$. In view of the above remarks, an object of $\calC_{p/}$ is a colimit for $p$ if and only if its image in $\calC^{p/}$ is an initial object; in other words, we can replace $\calC_{p/}$ by $\calC^{p/}$ (and $\star$ by $\diamond$) in Definition \ref{defcolim}.

By Proposition \ref{sharpen}, for any $\infty$-category $\calC$ and any
map $p: X \rightarrow \calC$, the induced map $\calC_{p/} \rightarrow \calC$
is a left fibration. We now show that $\calC^{p/}$ has the same
property:

\begin{proposition}\label{sharpenn}
Suppose given a diagram of simplicial sets
$$ K_0 \subseteq K \stackrel{p}{\rightarrow} X \stackrel{q}{\rightarrow} S$$
where $q$ is a categorical fibration. Let $r = q \circ p: K \rightarrow S$,
$p_0 = p|K_0$, and $r_0 = r|K_0$. Then the induced map
$$ \phi: X^{p/} \rightarrow X^{p_0/} \times_{ S^{r_0/} } S^{r/}$$
is a left fibration.
\end{proposition}

\begin{proof}
We must show that $q$ has the right lifting property with respect to every left-anodyne inclusion
$A_0 \subseteq A$. Unwinding the definition, this amounts to proving that $q$ has the right
lifting property with respect to the inclusion
$$ i: (A_0 \diamond K) \coprod_{ A_0 \diamond K_0} (A \diamond K_0) \subseteq
A \diamond K.$$
Since $q$ is a categorical fibration, it suffices to show that $i$ is a categorical equivalence.
The above pushout is a homotopy pushout, so it will suffice to prove the analogous statement for the weakly equivalent inclusion
$$ (A_0 \star K) \coprod_{ A_0 \star K_0 } (A \star K_0) \subseteq A \star K.$$
But this map is inner anodyne (Lemma \ref{precough}).
\end{proof}

\begin{corollary}\label{measles}
Let $\calC$ be an $\infty$-category, and let $p: K \rightarrow \calC$ be any
diagram. For every vertex $v$ of $\calC$, the map $\calC_{p/} \times_{\calC}
\{v\} \rightarrow \calC^{p/} \times_{\calC} \{v\}$ is a homotopy
equivalence of Kan complexes.
\end{corollary}

\begin{proof}
The map $\calC_{p/} \rightarrow \calC^{p/}$ is a categorical equivalence
of left fibrations over $\calC$; now apply Proposition
\ref{apple1}.
\end{proof}

\begin{corollary}\label{homsetsagree}
Let $\calC$ be an $\infty$-category containing vertices $x$ and $y$. The maps
$$ \Hom^{\rght}_{\calC}(x,y) \rightarrow \Hom_{\calC}(x,y) \leftarrow \Hom^{\lft}_{\calC}(x,y)$$
are homotopy equivalences of Kan complexes (see \S \ref{prereq1} for an explanation of this notation).
\end{corollary}

\begin{proof}
Apply Corollary \ref{measles} (the dual of Corollary \ref{measles}) to the case where $p$ is the
inclusion $\{x \} \subseteq \calC$ (the inclusion $\{y\} \subseteq \calC$).
\end{proof}

\begin{remark}
The above ideas dualize in an evident way; given a map of simplicial sets
$p: K \rightarrow X$, we can define a simplicial set $X^{/p}$ with the universal
mapping property
$$ \Hom_{\sSet}(K', X^{/p} ) = \Hom_{(\sSet)_{K/}}( K' \diamond K, X).$$\index{not}{calC^/p@$\calC^{/p}$}
\end{remark}

\subsection{Parametrized Colimits}\label{consweet}

Let $p: K \rightarrow \calC$ be a diagram in an $\infty$-category $\calC$. The goal of this section is to make precise the idea that the colimit $\injlim(p)$ depends functorially on $p$ (provided that $\injlim(p)$ exists). We will prove this in a very general context, in which not only the diagram $p$ but also the simplicial set $K$ is allowed to vary. We begin by introducing a {\em relative} version of the $\diamond$-operation.

\begin{definition}
Let $S$ be a simplicial set, and let $X,Y \in (\sSet)_{/S}$. We define
$$X \diamond_S Y = X \coprod_{X \times_{S} Y \times \{0\} } (X \times_{S} Y \times \Delta^1) \coprod_{ X \times_{S} Y \times \{1\}} Y \in (\sSet)_{/S}.$$\index{not}{diamondS@$\diamond_{S}$}
\end{definition}

We observe that the operation $\diamond_S$ is compatible with base change in the following sense: for any map $T \rightarrow S$ of simplicial sets and any objects $X,Y \in (\sSet)_{/S}$, there is a natural isomorphism
$$ (X_T \diamond_T Y_T) \simeq (X \diamond_S Y)_T,$$
where we let $Z_T$ denote the fiber product $Z \times_{S} T$. 
We also note that in the case where $S$ is a point, the operation $\diamond_S$ coincides
with the operation $\diamond$ introduced in \S \ref{quasilimit2}. 

Fix $K \in (\sSet)_{/S}$. We note that functor $(\sSet)_{/S} \rightarrow ( (\sSet)_{/S})_{K/}$ defined by
$$ X \mapsto K \diamond_S S$$
has a right adjoint; this right adjoint associates to a diagram
$$ \xymatrix{ K \ar[dr] \ar[rr]^{p_S} & & Y \ar[dl] \\
& S }$$
the simplicial set $Y^{p_S/}$, defined by the property that
$ \Hom_{S}(X, Y^{p_S/})$ classifies commutative diagrams
$$ \xymatrix{ K  \ar[r]^{p_S} \ar@{^{(}->}[d] & Y \ar[d] \\  
K \diamond_S X \ar[ur] \ar[r] & S.}$$

The base-change properties of the operation $\diamond_S$ imply similar base-change properties for the relative slice construction: given a map $p_S: K \rightarrow Y$ in $(\sSet)_{/S}$ and any map $T \rightarrow S$, we have a natural isomorphism
$$ Y^{p_S/} \times_{S} T \simeq (Y \times_{S} T)^{p_T/}$$
where $p_T$ denotes the induced map $K_{T} \rightarrow Y_{T}$. 
In particular, the fiber of $Y^{p_S/}$ over a vertex $s$ of $S$ can be identified with the absolute slice construction $Y_s^{p_s/}$ studied in \S \ref{quasilimit2}.\index{not}{X^p_S/@$X^{p_{S}/}$}

\begin{remark}
Our notation is somewhat abusive: the simplicial set $Y^{p_S/}$ depends not only on the map
$p_S: K \rightarrow Y$, but also on the simplicial set $S$. We will attempt to avoid confusion by always indicating the simplicial set $S$ by a subscript in the notation for the map in question; we will only omit this subscript in the case $S = \Delta^0$, in which case the functor described above
coincides with the definition given in \S \ref{quasilimit2}.
\end{remark}

\begin{lemma}\label{stumble}
Let $n > 0$, and let 
$$B = ( \Lambda^n_n \times \Delta^1 ) \coprod_{ \Lambda^n_n \times \bd \Delta^1 }
( \Delta^n \times \bd \Delta^1 ) \subseteq \Delta^n \times \Delta^1.$$
Suppose given a diagram of simplicial sets
$$ \xymatrix{ A \times B \ar[r]^{f_0} \ar@{^{(}->}[d] & Y \ar[d]^{q} \\
A \times \Delta^n \times \Delta^1 \ar[r] \ar@{-->}[ur]^{f} & S}$$
in which $q$ is a Cartesian fibration, and that $f_0$ carries $\{ a\} \times \Delta^{ \{n-1,n\} } \times \{1\}$ to a
$q$-Cartesian edge of $Y$, for each vertex $a$ of $A$. Then there exists a morphism $f$
rendering the diagram commutative. 
\end{lemma}

\begin{proof}
Invoking Proposition \ref{doog}, we may replace $q: Y \rightarrow S$ by the induced map
$Y^A \rightarrow S^A$, and thereby reduce to the case where $A = \Delta^0$. 
We now recall the notation introduced in the proof of Proposition \ref{usejoyal}: more specifically, the family $\{ \sigma_i\}_{ 0 \leq i \leq n}$ of nondegenerate simplices of $\Delta^n \times \Delta^1$.
Let $B(0) = B$, and more generally set $B(n) = B \cup \sigma_n \cup \ldots \cup \sigma_{n+1-i}$
so that we we have a filtration
$$ B(0) \subseteq \ldots \subseteq B(n+1) = \Delta^n \times \Delta^1.$$
A map $f_0: B(0) \rightarrow Y$ has been prescribed for us already; we construct extensions $f_i: B(i) \rightarrow Y$ by induction on $i$. For $i < n$, there is a pushout diagram
$$ \xymatrix{ \Lambda^{n+1}_{n-i} \ar[r] \ar@{^{(}->}[d] & B(i) \ar@{^{(}->}[d] \\
\Delta^{n+1} \ar[r] & B(i+1) }.$$
Thus, the extension $f_{i+1}$ can be found in virtue of the assumption that $q$ is an inner fibration. For $i = n$, we obtain instead a pushout diagram
$$ \xymatrix{ \Lambda^{n+1}_{n+1} \ar[r] \ar@{^{(}->}[d] & B(n) \ar@{^{(}->}[d] \\
\Delta^{n+1} \ar[r] & B(n+1) },$$
and the desired extension can be found in virtue of the assumption that $f_0$ carries the edge $\Delta^{ \{n-1,n\}} \times \{1\}$ to a $q$-Cartesian edge of $Y$.
\end{proof}

\begin{proposition}\label{colimfam}
Suppose given a diagram of simplicial sets
$$ \xymatrix{ K \ar[drr]^{t} \ar[r]^{p_S} & X \ar[r]^{q} \ar[dr] & Y \ar[d] \\
& & S. }$$
Let $p'_{S} = q \circ p_{S}$. Suppose further that:
\begin{itemize}
\item[$(1)$] The map $q$ is a Cartesian fibration.
\item[$(2)$] The map $t$ is a coCartesian fibration.
\end{itemize}
Then the induced map $r: X^{p_S/} \rightarrow Y^{p'_S/}$ is a Cartesian fibration; moreover an edge of $X^{p_S/}$ is $r$-Cartesian if and only if its image in $X$ is $q$-Cartesian.
\end{proposition}

\begin{proof}
We first show that $r$ is an inner fibration. Suppose given $0 < i < n$ and a diagram
$$ \xymatrix{ \Lambda^n_i \ar[r] \ar@{^{(}->}[d] & X^{p_S/} \ar[d] \\
\Delta^n \ar[r] \ar@{-->}[ur] & Y^{p'_{S}/},}$$
we must show that it is possible to provide the dotted arrow. Unwinding the definitions, we see that it suffices to produce the indicated arrow in the diagram
$$ \xymatrix{ K \diamond_S \Lambda^n_i \ar[r] \ar@{^{(}->}[d] & X \ar[d]^{q} \\
K \diamond_S \Delta^n \ar[r] \ar@{-->}[ur] & Y.}$$
Since $q$ is a Cartesian fibration, it is a categorical fibration by Proposition \ref{funkyfibcatfib}.
Consequently, it suffices to show that the inclusion
$$ K \diamond_S \Lambda^n_i \subseteq K \diamond_S \Delta^n$$ is a categorical equivalence. In view of the definition of $K \diamond_S M$ as a (homotopy) pushout
$$ K \coprod_{ K \times_S M \times \{0\}} (K \times_S M \times \Delta^1) \coprod_{ K \times_S M \times \{1\} } M,$$ it suffices to verify that the inclusions
$$ \Lambda^n_i \subseteq \Delta^n$$
$$ K \times_S \Lambda^n_i \subseteq K \times_S \Delta^n$$ 
are categorical equivalences. The first statement is obvious; the second follows from (the dual of) Proposition \ref{basechangefunky}.

Let us say that an edge of $X^{p_S/}$ is {\it special} if its image in $X$ is $q$-Cartesian. To complete the proof, it will suffice to show that every special edge of $X^{p_S/}$ is $r$-Cartesian, and that there are sufficiently many special edges of $X^{p_S/}$. More precisely, consider any $n \geq 1$ and any diagram
$$ \xymatrix{ \Lambda^n_n \ar[r]^{h} \ar@{^{(}->}[d] & X^{p_S/} \ar[d] \\
\Delta^n \ar[r] \ar@{-->}[ur] & Y^{p'_{S}/}.}$$
We must show that:
\begin{itemize}
\item If $n = 1$, then there exists a dotted arrow rendering the diagram commutative, classifying a special edge of $X^{p_S/}$. 
\item If $n > 1$ and $h| \Delta^{ \{n-1,n\} }$ classifies a special edge of $X^{p_S/}$, then there exists a dotted arrow rendering the diagram commutative.
\end{itemize}
Unwinding the definitions,  we have a diagram
$$ \xymatrix{ K \diamond_S \Lambda^n_n \ar[r]^{f_0} \ar@{^{(}->}[d] & X \ar[d]^{q} \\
K \diamond_S \Delta^n \ar[r] \ar@{-->}[ur]^{f} & Y}$$
and we wish to prove the existence of the indicated arrow $f$. As a first step, we consider the restricted diagram
$$ \xymatrix{ \Lambda^n_n \ar[r]^{f_0| \Lambda^n_n} \ar@{^{(}->}[d] & X \ar[d]^{q} \\
\Delta^n \ar[r] \ar@{-->}[ur]^{f_1} & Y}.$$
By assumption, $f_0 | \Lambda^n_n$ carries $\Delta^{ \{n-1,n\} }$ to a $q$-Cartesian edge of $X$ (if $n > 1$), so there exists a map $f_1$ rendering the diagram commutative (and classifying a $q$-Cartesian edge of $X$ if $n = 1$).
It now suffices to produce the dotted arrow in the diagram
$$ \xymatrix{ (K \diamond_S \Lambda^n_n) \coprod_{ \Lambda^n_n } \Delta^n  \ar[r] \ar@{^{(}->}[d]^{i} & X \ar[d]^{q} \\
K \diamond_S \Delta^n \ar[r] \ar@{-->}[ur]^{f} & Y,}$$
where the top horizontal arrow is the result of amalgamating $f_0$ and $f_1$.

Without loss of generality, we may replace $S$ by $\Delta^n$. By (the dual of) Proposition \ref{simplexplay}, there exists a composable sequence of maps
$$ \phi: A^0 \rightarrow \ldots \rightarrow A^n$$ and a quasi-equivalence
$M^{op}(\phi) \rightarrow K$. We have a commutative diagram
$$ \xymatrix{ 
(M^{op}(\phi) \diamond_S \Lambda^n_n) \coprod_{ \Lambda^n_n } \Delta^n \ar@{^{(}->}[d]^{i'} \ar[r] & (K \diamond_S \Lambda^n_n) \coprod_{ \Lambda^n_n } \Delta^n \ar[d]^{i} \\
 M^{op}(\phi) \diamond_S \Delta^n \ar[r] & K \diamond_S \Delta^n}.$$
Since $q$ is a categorical fibration, Proposition \ref{princex} shows that it suffices to produce a dotted arrow $f'$ in the induced diagram
$$ \xymatrix{ (M^{op}(\phi) \diamond_S \Lambda^n_n) \coprod_{ \Lambda^n_n } \Delta^n  \ar[r] \ar@{^{(}->}[d]^{i} & X \ar[d]^{q} \\
M^{op}(\phi) \diamond_S \Delta^n \ar[r] \ar@{-->}[ur]^{f'} & Y}.$$
Let $B$ be as the statement of Lemma \ref{stumble}; then we have a pushout diagram
$$ \xymatrix{ 
A^0 \times B \ar[r] \ar@{^{(}->}[d]^{i''} & (M^{op}(\phi) \diamond_S \Lambda^n_n) \coprod_{ \Lambda^n_n} \Delta^n \ar[d] \\
A^0 \times \Delta^n \times \Delta^1 \ar[r] & M^{op}(\phi) \diamond_S \Delta^n.}$$
Consequently, it suffices to prove the existence of the map $f''$ in the diagram
$$ \xymatrix{ A^0 \times B \ar[r]^{g} \ar@{^{(}->}[d]^{i''} & X \ar[d]^{q} \\
A^0 \times \Delta^n \times \Delta^1 \ar[r] \ar@{-->}[ur]^{f''} & Y}.$$
Here the map $g$ carries $\{a\} \times \Delta^{ \{n-1,n\} } \times \{1\}$ to a $q$-Cartesian edge of $Y$, for each vertex $a$ of $A^0$. The existence of $f''$ now follows from Lemma \ref{stumble}.
\end{proof}

\begin{remark}\label{superfam}
In most applications of Proposition \ref{colimfam}, we will have $Y = S$. In that case,
$Y^{p'_{S}/}$ can be identified with $S$, and the conclusion is that the projection
$X^{p_S/} \rightarrow S$ is a Cartesian fibration.
\end{remark}

\begin{remark}\label{notnec}
The hypothesis on $s$ in Proposition \ref{colimfam} can be weakened: all we need in the proof is existence of maps $M^{op}(\phi) \rightarrow K \times_{S} \Delta^n$ which are universal categorical equivalences (that is, induce categorical equivalences $M^{op}(\phi) \times_{ \Delta^n } T \rightarrow K \times_{S} T$ for any $T \rightarrow \Delta^n$). Consequently, Proposition \ref{colimfam} remains valid when $K \simeq S \times K^0$, for {\em any} simplicial set $K^0$ (not necessarily an $\infty$-category). It seems likely that Proposition \ref{colimfam} remains valid whenever $s$ is a smooth map of simplicial sets, but we have not been able to prove this.
\end{remark}

We can now express the idea that the colimit a diagram should depend functorially on the diagram (at least for ``smoothly parametrized'' families of diagrams):

\begin{proposition}\label{familycolimit}\index{gen}{colimit!in families}
Let $q: Y \rightarrow S$ be a Cartesian fibration, let
$p_S: K \rightarrow Y$ be a diagram. Suppose that:
\begin{itemize}
\item[$(1)$] For each vertex $s$ of $S$, the restricted diagram $p_s: K_s \rightarrow Y_s$
has a colimit in the $\infty$-category $Y_s$.
\item[$(2)$] The composition $q \circ p_S$ is a coCartesian fibration.
\end{itemize}

There exists a map $p'_S$ rendering the diagram
$$ \xymatrix{ K \ar@{^{(}->}[d] \ar[r]^{p_S} & Y \ar[d]^{q} \\
K \diamond_S S \ar[ur]^{p'_S} \ar[r] & S }$$
commutative, and having the property that for each vertex $s$ of $S$, the restriction $p'_s: K_s \diamond \{s\} \rightarrow Y_s$ is a colimit of $p_s$.
Moreover, the collection of all such maps is parametrized by a contractible Kan complex.
\end{proposition}

\begin{proof}
Apply Proposition \ref{topaz} to the Cartesian fibration $Y^{p_S/}$, and observe that the collection of sections of a trivial fibration constitutes a contractible Kan complex.
\end{proof}

\subsection{Decomposition of Diagrams}\label{quasilimit1}
 
Let $\calC$ be an $\infty$-category, and $p: K \rightarrow \calC$ a diagram indexed by a simplicial set $K$. In this section, we will try to analyze the colimit $\colim(p)$ (if it exists) in terms of the colimits $\{ \colim(p | K_{I}) \}$, where $\{ K_{I} \}$ is some family of simplicial subsets of $K$. In fact, it will be useful to work in slightly more generality: we will allow each $K_{I}$ to be an arbitrary simplicial set mapping to $K$ (not necessarily via a monomorphism).

Throughout this section, we will fix a simplicial set $K$, an ordinary category $\calI$, and a functor
$F: \calI \rightarrow (\sSet)_{/K}$. It may be helpful to imagine that $\calI$
is a partially ordered set and that $F$ is an order-preserving map
from $\calI$ to the collection of simplicial subsets of $K$; this
will suffice for many but not all of our applications. We will
denote $F(I)$ by $K_I$, and the tautological map $K_I \rightarrow
K$ by $\pi_{I}$.

Our goal is to show that, under appropriate hypotheses, we can recover the colimit
of a diagram $p: K \rightarrow \calC$ in terms of the colimits of diagrams
$p \circ \pi_I: K_I \rightarrow \calC$. Our first goal is to show that the construction of these colimits is suitably functorial in $I$. For this, we need an auxiliary construction.

\begin{notation}\label{nixx}\index{not}{KF@$K_{F}$}
We define a simplicial set $K_{F}$ as follows. A map $\Delta^n \rightarrow K_{F}$
is determined by the following data:

\begin{itemize}
\item[$(i)$] A map $\Delta^{n} \rightarrow \Delta^{1}$, corresponding to a decomposition
$[n] = \{ 0, \ldots, i \} \cup \{ i+1, \ldots, n\}$.

\item[$(ii)$] A map $e_{-}: \Delta^{ \{ 0, \ldots, i\} } \rightarrow K$.

\item[$(iii)$] A map $e_{+}: \Delta^{\{ i+1, \ldots, n\} } \rightarrow \Nerve(\calI)$, which we may view as a chain of composable morphisms
$$ I(i+1) \rightarrow \ldots \rightarrow I(n)$$
in the category $\calI$.

\item[$(iv)$] For each $j \in \{i+1, \ldots, n\}$, a map
$e_{j}$ which fits into a commutative diagram
$$ \xymatrix{ & K_{I(j)} \ar[d]^{\pi_{I(j)}} \\
\Delta^{ \{0, \ldots, i\} } \ar[r]^{e_{-}} \ar[ur]^{e_j} & K. }$$
Moreover, for $j \leq k$ we require that
$e_k$ is given by the composition
$$ \Delta^{ \{0, \ldots, i\} } \stackrel{ e_j }{\rightarrow} 
K_{I(j)} \rightarrow K_{I(k)}.$$
\end{itemize}
\end{notation}

\begin{remark}
In the case where $i < n$, the maps $e_{-}$ and $\{ e_j \}_{j > i}$ are completely
determined by $e_{i+1}$, which can be arbitrary.
\end{remark}

The simplicial set $K_{F}$ is equipped with a map $K_{F} \rightarrow \Delta^1$. Under this map, the preimage of the vertex $\{0\}$ is $K \subseteq K_{F}$, and the preimage of the vertex $\{1\}$
is $\Nerve(\calI) \subseteq K_{F}$. For $I \in \calI$, we will denote the
corresponding vertex of $\Nerve(\calI) \subseteq K_F$ by $X_I$. We
note that, for each $I \in \calI$, there is a commutative diagram
$$ \xymatrix{ K_I \ar[r]^{\pi_I} \ar@{^{(}->}[d] & K \ar@{^{(}->}[d] \\
K_I^{\triangleright} \ar[r]^{\pi'_I} & K_F }$$
where $\pi'_I$ carries the cone point of $K_I^{\triangleright}$ to
the vertex $X_I$ of $K_F$.

Let us now suppose that $p: K \rightarrow \calC$ is a diagram in an $\infty$-category $\calC$. Our next goal is to prove Proposition \ref{extet}, which will allow us to extend $p$ to a larger diagram $K_{F} \rightarrow \calC$ which carries each vertex $X_I$ to a colimit of $p \circ \pi_I: K_I \rightarrow \calC$. First, we need a lemma.

\begin{lemma}\label{chort}
Let $\calC$ be an $\infty$-category, and let
$\sigma: \Delta^n \rightarrow \calC$ be a simplex having the property that
$\sigma(0)$ is an initial object of $\calC$. Let $\bd \sigma = \sigma | \bd \Delta^n$. 
The natural map $\calC_{\sigma/} \rightarrow \calC_{\bd \sigma/}$ is a trivial fibration.
\end{lemma}

\begin{proof}
Unwinding the definition, we are reduced to solving the extension problem depicted in the diagram
$$ \xymatrix{ (\bd \Delta^n \star \Delta^m) \coprod_{
\bd \Delta^n \star \bd \Delta^m } (\Delta^n \star \bd \Delta^m) \ar[r]^-{f_0} 
\ar@{^{(}->}[d] & \calC \\
\Delta^n \star \Delta^m. \ar@{-->}[ur]^-{f} & }$$
We can identify the domain of $f_0$ with $\bd \Delta^{n+m+1}$. Our hypothesis
guarantees that $f_0(0)$ is an initial object of $\calC$, which in turn guarantees the existence of $f$.
\end{proof}

\begin{proposition}\label{extet}
Let $p: K \rightarrow \calC$ be a diagram in an $\infty$-category $\calC$, let $\calI$ be an ordinary category, and let $F: \calI \rightarrow (\sSet)_{/K}$ be a functor. Suppose that, for each $I \in \calI$, the induced diagram $p_I = p \circ \pi_I: K_I \rightarrow \calC$ has a colimit
$q_I: K_I^{\triangleright} \rightarrow \calC$.

There exists a map $q: K_F \rightarrow \calC$ such that $q \circ \pi'_I = q_{I}$ and $q|K = p$. 
Furthermore, for any such $q$, the induced map $\calC_{q/} \rightarrow \calC_{p/}$ is a trivial fibration.
\end{proposition}

\begin{proof}
For each $X \subseteq \Nerve(\calI)$, we let $K_{X}$ denote the
simplicial subset of $K_F$ consisting of all simplices $\sigma \in
K_{F}$ such that $\sigma \cap \Nerve(\calI) \subseteq X$. We note that
$K_{\emptyset} = K$ and that $K_{ \Nerve(\calI) } = K_{F}$.

Define a transfinite sequence $Y_{\alpha}$ of simplicial subsets
of $\Nerve(\calI)$ as follows. Let $Y_{0} = \emptyset$, and let
$Y_{\lambda} = \bigcup_{ \gamma < \lambda } Y_{\gamma}$ when
$\lambda$ is a limit ordinal. Finally, let $Y_{\alpha+1}$ be
obtained from $Y_{\alpha}$ by adjoining a single nondegenerate
simplex, provided that such a simplex exists. We note that for
$\alpha$ sufficiently large, such a simplex will not exist and we
set $Y_{\beta} = Y_{\alpha}$ for all $\beta > \alpha$.

We define a sequence of maps $q_{\beta}: K_{Y_{\beta}} \rightarrow
\calC$ so that the following conditions are satisfied:

\begin{itemize}
\item[$(1)$] We have $q_{0} = p: K_{\emptyset} = K \rightarrow \calC$.

\item[$(2)$] If $\alpha < \beta$, then $q_{\alpha} = q_{\beta} |
K_{Y_{\alpha}}$.

\item[$(3)$] If $\{X_I\} \subseteq Y_{\alpha}$, then $q_{\alpha} \circ \pi'_I = q_I: K_I^{\triangleright} \rightarrow \calC$.

\end{itemize}
Provided that such a sequence can be constructed, we may conclude
the proof by setting $q = q_{\alpha}$ for $\alpha$ sufficiently
large.

The construction of $q_{\alpha}$ goes by induction on $\alpha$. If
$\alpha = 0$, then $q_{\alpha}$ is determined by condition $(1)$;
if $\alpha$ is a (nonzero) limit ordinal, then $q_{\alpha}$ is
determined by condition $(2)$. Suppose that $q_{\alpha}$ has been
constructed; we give a construction of $q_{\alpha+1}$.

There are two cases to consider. Suppose first that $Y_{\alpha+1}$
is obtained from $Y_{\alpha}$ by adjoining a vertex $X_I$. In this
case, $q_{\alpha+1}$ is uniquely determined by conditions $(2)$
and $(3)$.

Now suppose that $X_{\alpha+1}$ is obtained from $X_{\alpha}$ by
adjoining a nondegenerate simplex $\sigma$ of positive dimension, corresponding
to a sequence of composable maps
$$ I_0 \rightarrow \ldots \rightarrow I_n$$
in the category $\calI$. We
note that the inclusion $K_{Y_{\alpha}} \subseteq
K_{Y_{\alpha+1}}$ is a pushout of the inclusion
$$ K_{I_0} \star \bd \sigma \subseteq K_{I_0} \star \sigma.$$
Consequently, constructing the map $q_{\alpha+1}$ is tantamount to
finding an extension of a certain map $s_0: \bd \sigma \rightarrow \calC_{p_I/}$ to the whole of the simplex $\sigma$. By assumption, $s_0$ carries the initial vertex of $\sigma$ to an initial
object of $\calC_{p_I/}$, so that the desired extension $s$ can be found. For use below, we record
a further property of our construction: the projection $\calC_{q_{\alpha+1}/} \rightarrow \calC_{q_{\alpha}/}$ is a pullback of the map $( \calC_{ p_I/} )_{s/} \rightarrow (\calC_{p_I/})_{s_0/}$, which is a trivial fibration.

We now wish to prove that for any extension $q$ with the above properties, the induced map
$\calC_{q/} \rightarrow \calC_{p/}$ is a trivial fibration. We first observe that the map $q$ can be obtained by the inductive construction given above: namely, we take $q_{\alpha}$ to be
the restriction of $q$ to $K_{Y_{\alpha}}$. It will therefore suffice to show that, for every pair of ordinals $\alpha \leq \beta$, the induced map $\calC_{q_{\beta}/} \rightarrow \calC_{q_{\alpha}/}$ is a trivial fibration. The proof of this goes by induction on $\beta$: the case $\beta = 0$ is clear, and if $\beta$ is a limit ordinal we observe that the inverse limit of transfinite tower of trivial fibrations is itself a trivial fibration. We may therefore suppose that $\beta = \gamma + 1$ is a successor ordinal.
Using the factorization
$$ \calC_{q_{\beta}/} \rightarrow \calC_{q_{\gamma}/} \rightarrow \calC_{q_{\alpha}/}$$
and the inductive hypothesis, we are reduced to proving this in the case where $\beta$ is the successor of $\alpha$, which was treated above.
\end{proof}

Let us now suppose that we are given diagrams $p: K \rightarrow \calC$, $F: \calI \rightarrow (\sSet)_{/K}$ as in the statement of Proposition \ref{extet}, and let $q: K_{F} \rightarrow \calC$ be a map which satisfies the conclusions of the Proposition. Since $\calC_{q/} \rightarrow \calC_{p/}$ is a trivial fibration, we may identify colimits of the diagram $q$ with colimits of the diagram $p$ (up to equivalence). Of course, this is not useful in itself,
since the diagram $q$ is {\em more} complicated than $p$. Our objective
now is to show that, under the appropriate hypotheses, we may
identify the colimits of $q$ with the colimits of $q|\Nerve(\calI)$.
First, we need a few lemmas.

\begin{lemma}[Joyal \cite{joyalnotpub}]\label{chotle}
Let $f: A_0 \subseteq A$ and $g: B_0 \subseteq B$ be inclusions of simplicial sets, and suppose
that $g$ is a weak homotopy equivalence. Then the induced map
$$h: (A_0 \star B) \coprod_{ A_0 \star B_0} (A \star B_0) \subseteq A \star B$$ is right anodyne.
\end{lemma}

\begin{proof}
Our proof follows the pattern of Lemma \ref{precough}. The collection of all maps $f$ which
satisfy the conclusion (for {\em any} choice of $g$) forms a weakly saturated class of morphisms. It will therefore suffice to prove that the $h$ is right anodyne when $f$ is the inclusion $\bd \Delta^n \subseteq \bd \Delta^n$. Similarly, the collection of all maps $g$ which satisfy the conclusion (for fixed $f$) forms a weakly saturated class. We may therefore reduce to the case where $g$ is a horn inclusion $\Lambda^m_i \subseteq \Delta^m$. In this case, we may identify $h$ with the horn inclusion $\Lambda^{m+n+1}_{i+n+1} \subseteq \Delta^{m+n+1}$, which is clearly right-anodyne.
\end{proof}

\begin{lemma}\label{chotle2}
Let $A_0 \subseteq A$ be an inclusion of simplicial sets, and let
$B$ be weakly contractible. Then the inclusion $A_0 \star B
\subseteq A \star B$ is right anodyne.
\end{lemma}

\begin{proof}
As above, we may suppose that the inclusion $A_0 \subseteq A$ is identified with
$ \bd \Delta^n \subseteq \Delta^n$. If $K$ is a point, then the inclusion $A_0 \times B \subseteq A \times B$ is 
isomorphic to $\Lambda^{n+1}_{n+1}
\subseteq \Delta^{n+1}$, which is clearly right-anodyne.

In the general case, $B$ is nonempty, so we may choose a vertex
$b$ of $B$. Since $B$ is weakly contractible, the inclusion $\{b\}
\subseteq B$ is a weak homotopy equivalence. We have already shown
that $A_0 \star \{b\} \subseteq A \star \{b\}$ is right anodyne. It
follows that the pushout inclusion
$$A_0 \star B \subseteq (A \star \{b\}) \coprod_{ A_0 \star \{b\} } (A_0
\star B)$$ is right anodyne. To complete the proof, we apply Lemma
\ref{chotle} to deduce that the inclusion
$$  (A \star \{b\}) \coprod_{ A_0 \star \{b\} } (A_0
\star B) \subseteq A \star B$$ is right anodyne.
\end{proof}

\begin{notation}
Let $\sigma \in K_n$ be a simplex of $K$. We define a category
$\calI_{\sigma}$ as follows. The objects of $\calI_{z}$ are pairs $(I,
\sigma')$, where $I \in \calI$, $\sigma' \in (K_I)_n$, and $\pi_I(\sigma') = \sigma$. A
morphism from $(I',\sigma')$ to $(I'',\sigma'')$ in $\calI_{\sigma}$ consists of
a morphism $\alpha: I' \rightarrow I''$ in $\calI$ with the
property that $F(\alpha)(\sigma') = \sigma''$. We let $\calI'_{\sigma} \subseteq
\calI_{\sigma}$ denote the full subcategory consisting of pairs
$(I,\sigma')$ where $\sigma'$ is a degenerate simplex in $K_{I}$. Note that
if $\sigma$ is nondegenerate, $\calI'_{\sigma}$ is empty.
\end{notation}

\begin{proposition}\label{utl}
Let $K$ be a simplicial set, $\calI$ an ordinary category, and $F: \calI \rightarrow (\sSet)_{/K}$ a functor.
Suppose further that:
\begin{itemize}
\item[$(1)$] For each nondegenerate simplex $\sigma$ of $K$, the category
$\calI_{\sigma}$ is acyclic (that is, the simplicial set $\Nerve(\calI_{\sigma})$
is weakly contractible).

\item[$(2)$] For each degenerate simplex $\sigma$ of $K$, the inclusion $\Nerve
(\calI'_{\sigma}) \subseteq \Nerve(\calI_{\sigma})$ is a weak homotopy equivalence.
\end{itemize}

Then the inclusion $\Nerve(\calI) \subseteq K_{F}$ is right anodyne.
\end{proposition}

\begin{proof}
Consider any family of subsets $\{ L_{n} \subseteq K_{n}\}$ which is
stable under the ``face maps'' $d_i$ on $K$ (but not necessarily
the degeneracy maps $s_i$, so that the family $\{ L_{n} \}$ does
not necessarily have the structure of a simplicial set). We define
a simplicial subset $L_{F} \subseteq K_F$ as follows: a {\em nondegenerate} simplex
$\Delta^n \rightarrow K_{F}$ belongs to $L_{F}$ if and only if the
corresponding (possibly degenerate) simplex $\Delta^{ \{0, \ldots, i\} } \rightarrow K$ belongs to
$L_i \subseteq K_i$ (see Notation \ref{nixx}). 

We note that if $L = \emptyset$, then $L_{F} = \Nerve(\calI)$. If $L = K$, then $L_{F}= K_{F}$ (so that our notation is unambiguous).
Consequently, it will suffice to prove that for any $L \subseteq
L'$, the inclusion $L_{F} \subseteq L'_{F}$ is right-anodyne. By
general nonsense, we may reduce to the case where $L'$ is obtained
from $L$ by adding a single simplex $\sigma \in K_n$.

We now have two cases to consider. Suppose first that the simplex
$\sigma$ is nondegenerate. In this case, it is not difficult to see
that the inclusion $L_{F} \subseteq L'_{F}$ is a pushout of $ \bd
\sigma \star \Nerve(\calI_{\sigma}) \subseteq \sigma \star \Nerve(\calI_{\sigma})$. By hypothesis,
$N \calI_{z}$ is weakly contractible, so that the inclusion $L_{F}
\subseteq L'_{F}$ is right anodyne by Lemma \ref{chotle2}.

In the case where $\sigma$ is degenerate, we observe that $L_{F}
\subseteq L'_{F}$ is a pushout of the inclusion
$$ (\bd \sigma \star \Nerve( \calI_{\sigma} )) \coprod_{ \bd \sigma \star \Nerve(\calI'_{\sigma})} (\sigma \star \Nerve(\calI'_{\sigma})) \subseteq \sigma
\star \Nerve( \calI_{\sigma} ),$$ which is right anodyne by Lemma
\ref{chotle}.
\end{proof}

\begin{remark}
Suppose that $\calI$ is a partially ordered set, and that $F$ is
an order-preserving map from $\calI$ to the collection of
simplicial subsets of $K$. In this case, we observe that
$\calI'_{\sigma} = \calI_{\sigma}$ whenever $\sigma$ is a degenerate simplex of $K$, and that
$\calI_{\sigma} = \{ I \in \calI: \sigma \in K_I \}$ for any $\sigma$. Consequently, the
conditions of Proposition \ref{utl} hold if and only if each of
the partially ordered subsets $\calI_{\sigma} \subseteq \calI$ has a
contractible nerve. This holds automatically if $\calI$ is
directed and $K = \bigcup_{I \in \calI} K_I$.
\end{remark}

\begin{corollary}\label{util}
Let $K$ be a simplicial set, $\calI$ a category, and $F: \calI \rightarrow (\sSet)_{/K}$ a functor which satisfies the hypotheses of Proposition \ref{utl}. Let $\calC$ be an $\infty$-category, $p: K \rightarrow \calC$ any diagram, and let $q: K_{F} \rightarrow \calC$ be an extension of $p$ which satisfied the conclusions of Proposition \ref{extet}. The natural maps
$$ \calC_{p/} \leftarrow \calC_{q/} \rightarrow \calC_{q | \Nerve(\calI)/}$$ are trivial fibrations.
In particular, we may identify colimits of $p$ with colimits of $q| \Nerve(\calI)$.
\end{corollary}

\begin{proof}
This follows immediately from Proposition \ref{utl}, since the right anodyne inclusion
$\Nerve \calI \subseteq K_{F}$ is cofinal and therefore induces a trivial fibration $\calC_{q/} \rightarrow \calC_{q|\Nerve(\calI)/}$ by Proposition \ref{gute}.
\end{proof}

We now illustrate the usefulness of Corollary \ref{util} by giving a sample application. First, a bit of terminology. If $\kappa$ and $\tau$ are regular cardinals, we will write $\tau \ll \kappa$ if, for any cardinals $\tau_0 < \tau$, $\kappa_0 < \kappa$, we have $\kappa_0^{\tau_0} < \kappa$ (we refer the reader to Definition \ref{ineq} and the surrounding discussion for more details concerning this condition).\index{not}{kappalltau@$\kappa \ll \tau$}

\begin{corollary}\label{uterrr}
Let $\calC$ be an $\infty$-category and $\tau \ll \kappa$ regular cardinals. Then
$\calC$ admits $\kappa$-small colimits if and only if $\calC$ admits $\tau$-small
colimits and colimits indexed by (the nerves of) $\kappa$-small, $\tau$-filtered partially ordered sets.
\end{corollary}

\begin{proof}
The ``only if'' direction is obvious. Conversely, let $p: K
\rightarrow \calC$ be any $\kappa$-small diagram. Let $\calI$ denote the partially
ordered set of $\tau$-small simplicial subsets of $K$. Then
$\calI$ is directed and $\bigcup_{I \in \calI} K_I = K$, so that the hypotheses of
Proposition \ref{utl} are satisfied. Since each $p_I = p \circ \pi_I$ has a colimit
in $\calC$, there exists a map $q: K_{F} \rightarrow \calC$ satisfying the Proposition \ref{extet}.
Since $\calC_{q/} \rightarrow \calC_{p/}$ is an equivalence of $\infty$-categories, $p$ has a colimit if and only if $q$ has a colimit. By Corollary \ref{util}, $q$ has a colimit if and only if
$q| \Nerve(\calI)$ has a colimit. It is clear that $\calI$ is a $\tau$-filtered partially ordered set.
Furthermore, it is $\kappa$-small provided that $\tau \ll \kappa$.
\end{proof}

Similarly, we have:

\begin{corollary}
Let $f: \calC \rightarrow \calC'$ be a functor between $\infty$-categories, and let
$\tau \ll \kappa$ be regular cardinals. Suppose that
$\calC$ admits $\kappa$-small colimits. Then $f$ preserves $\kappa$-small colimits if and only if it preserves $\tau$-small colimits, and all colimits indexed by (the nerves of) $\kappa$-small, $\tau$-filtered partially ordered sets.
\end{corollary}

We will conclude this section with another application of Proposition
\ref{utl}, in which $\calI$ is not a partially ordered
set, and the maps $\pi_I: K_I \rightarrow K$ are not (necessarily)
injective. Instead, we take $\calI$ to be the {\it category of\index{gen}{category!of simplices}
simplices of $K$}. In other words, an object of $I \in \calI$ consists
of a map $\sigma_I: \Delta^n \rightarrow K$, and a morphism
from $I$ to $I'$ is given by a commutative diagram
$$ \xymatrix{ \Delta^n \ar[dr]^{\sigma_I} \ar[rr] & & \Delta^{n'} \ar[dl]_{\sigma'_{I'}} \\
& K. & }$$
For each $I \in \calI$, we let $K_I$ denote the domain $\Delta^n$ of $\sigma_I$, and we
let $\pi_{I} = \sigma_I: K_I \rightarrow K$.

\begin{lemma}\label{snick}
Let $K$ be a simplicial set, and let $\calI$ denote the
category of simplices of $K$ (as defined above). Then there is a
retraction $r: K_{F} \rightarrow K$ which fixes $K \subseteq K_F$.
\end{lemma}

\begin{proof}
Given a map $e: \Delta^n \rightarrow K_{F}$, we will describe the
composite map $r \circ e: \Delta^n \rightarrow K$. The map $e$ classifies the following data:
\begin{itemize}
\item[$(i)$] A decomposition $[n] = \{0, \ldots, i\} \cup \{i+1, \ldots, n \}$. 
\item[$(ii)$] A map $e_{-}: \Delta^{i} \rightarrow K$.
\item[$(iii)$] A string of morphisms
$$ \Delta^{ m_{i+1} } \rightarrow \ldots \rightarrow \Delta^{ m_n } \rightarrow K.$$
\item[$(iv)$] A compatible family of maps
$\{ e_j: \Delta^{i} \rightarrow \Delta^{m_j} \}_{ j > i }$, having the property that each composition $\Delta^i \stackrel{e_j}{\rightarrow} \Delta^{m_j} \rightarrow K$
coincide with $e_{-}$.
\end{itemize}

If $i = n$, we set $r \circ e = e_{-}$. Otherwise, we let $r \circ e$ denote the composition
$$ \Delta^n \stackrel{f}{\rightarrow} \Delta^{m_n} \rightarrow K$$
where $f: \Delta^n \rightarrow \Delta^{m_n}$ is defined as follows:

\begin{itemize}
\item The restriction $f | \Delta^{i}$ coincides with $e_n$.

\item For $i < j \leq n$, we let $f(j)$ denote the image in
$\Delta^{m_n}$ of the final vertex of $\Delta^{m_j}$.
\end{itemize}
\end{proof}

\begin{proposition}\label{cofinalcategories}
For every simplicial set $K$, there exists a category
$\calI$ and a cofinal map $f: \Nerve(\calI) \rightarrow K$.
\end{proposition}

\begin{proof}
We take $\calI$ to be the category of simplices of $K$, as defined
above, and $f$ to the composition of the inclusion $\Nerve(\calI)
\subseteq K_{F}$ with the retraction $r$ of Lemma \ref{snick}. 
To prove that $f$ is cofinal, it suffices to show that the inclusion
$\Nerve(\calI) \subseteq K_{F}$ is right anodyne, and that the retraction $r$ is cofinal.

To show that $\Nerve( \calI) \subseteq K_{F}$ is right anodyne, it suffices to show that the hypotheses of Proposition \ref{utl} are satisfied.  Let $\sigma: \Delta^J \rightarrow K$ be a simplex of $K$. We observe
that the category $\calI_{\sigma}$ may be described as follows: its
objects consist of pairs of maps $(s: \Delta^J \rightarrow
\Delta^{M}, t: \Delta^{M} \rightarrow K)$ with $t \circ s = \sigma$. A
morphism from $(s,t)$ to $(s',t')$ consists of a map
$$ \alpha: \Delta^M \rightarrow \Delta^{M'}$$
with $s' = \alpha \circ s$, $t = t' \circ \alpha$. In particular,
we note that $\calI_{\sigma}$ has an initial object $(\id_{\Delta^J},
\sigma)$. It also has a final object: namely, a pair $(s,t)$ such
that $s$ is surjective and $t: \Delta^M \rightarrow K$ is
nondegenerate. It follows that $\Nerve(\calI_{\sigma})$ is weakly
contractible for {\em any} simplex $\sigma$ of $K$. Moreover, if $z$ is
degenerate, then any final object of $\calI_{\sigma}$ belongs to
$\calI'_{\sigma}$ (and is therefore a final object of $\calI'_{\sigma}$). We conclude that $\Nerve(\calI'_{\sigma})$ is weakly contractible when $\sigma$ is
degenerate, so that the inclusion $\Nerve(\calI'_{\sigma}) \subseteq
\Nerve(\calI_{\sigma})$ is a weak homotopy equivalence. This completes the verification of the hypotheses of Proposition \ref{utl}.

We now show that $r$ is cofinal. According to Proposition \ref{gute}, it suffices to show that
for any $\infty$-category $\calC$ and any map $p: K \rightarrow \calC$, the induced map
$\calC_{q/} \rightarrow \calC_{p/}$ is a categorical equivalence, where $q = p \circ r$.
This follows from Proposition \ref{extet}.
\end{proof}

\begin{variant}\label{baryvar}\index{gen}{barycentric subdivision}
Let $K$ be a simplicial set, and let $\calI$ be the category of simplices of $K$ as above.
Let $\calI'$ be the full subcategory of $\calI$ spanned by the nondegenerate simplices of $K$. 
The inclusion $\calI' \subseteq \calI$ has a left adjoint $L$. It follows immediately from Theorem \ref{hollowtt} that the inclusion $\Nerve(\calI') \subseteq \Nerve(\calI)$ is cofinal. Consequently,
we obtain also a cofinal map $f: \Nerve(\calI') \rightarrow K$. The simplicial set
$\Nerve(\calI')$ can be identified with the {\it barycentric subdivision} of $K$. The assertion that $f$ is cofinal can be regarded as a generalization of the classical fact that barycentric subdivision does not change the weak homotopy type of a simplicial set. 

Note the category of nondegenerate simplices of $\Nerve(\calI')$ can be identified with a partially ordered set. The nerve of this partially ordered set can be identified with the {\it second barycentric subdivision $K^{(2)}$ of $K$}. Applying the above argument twice, we conclude that
there is a cofinal map $K^{(2)} \rightarrow K$. Consequently, we obtain the following refinement of 
Proposition \ref{cofinalcategories}: for every simplicial set $K$, there exists a partially ordered set $A$ and a cofinal map $\Nerve(A) \rightarrow K$.
\end{variant}

\subsection{Homotopy Colimits}\label{quasilimit4}

Our goal in this section is to compare the $\infty$-categorical theory of colimits with
the more classical theory of homotopy colimits in simplicial categories (see
Remark \ref{curble}). Our main result is the following:

\begin{theorem}\label{colimcomparee}\index{gen}{colimit!homotopy}\index{gen}{homotopy colimit}
Let $\calC$ and $\calI$ be fibrant simplicial categories and
$F: \calI \rightarrow \calC$ a simplicial functor. Suppose given an object $C \in \calC$ and a compatible
family of maps $\{ \eta_{I}: F(I) \rightarrow C \}_{I \in \calI}$. The following conditions are
equivalent:
\begin{itemize}
\item[$(1)$] The maps $\eta_I$ exhibit $C$ as a homotopy colimit of the diagram $F$.
\item[$(2)$] Let $f: \Nerve(\calI) \rightarrow \Nerve(\calC)$ be the simplicial nerve of $F$,
and $\overline{f}: \Nerve(\calI)^{\triangleright} \rightarrow \Nerve(\calC)$ the
extension of $f$ determined by the maps $\{ \eta_I \}$. Then
$\overline{f}$ is a colimit diagram in $\Nerve(\calC)$.
\end{itemize}
\end{theorem}

\begin{remark}
For an analogous result (in a slightly different setting), we refer
the reader to \cite{hirschowitz}.
\end{remark}

The proof of Theorem \ref{colimcomparee} will occupy the remainder of this section. 
We begin with a convenient criterion for detecting colimits in $\infty$-categories:

\begin{lemma}\label{kamma}
Let $\calC$ be an $\infty$-category, $K$ a simplicial set, and
$\overline{p}: K^{\triangleright} \rightarrow \calC$ a diagram. The following conditions are equivalent:
\begin{itemize}
\item[$(i)$] The diagram $\overline{p}$ is a colimit of $p = \overline{p} | K$.
\item[$(ii)$] Let $X \in \calC$ denote the image under $\overline{p}$ of the cone point
of $K^{\triangleright}$, let $\delta: \calC \rightarrow \Fun(K,\calC)$ denote the diagonal
embedding, and let $\alpha: p \rightarrow \delta(X)$ denote the natural transformation determined by $\overline{p}$. Then, for every object $Y \in \calC$, composition with $\alpha$ induces
a homotopy equivalence
$$ \phi_{Y}: \bHom_{\calC}(X, Y) \rightarrow \bHom_{ \Fun(K,\calC) }( p, \delta(Y) ).$$
\end{itemize}
\end{lemma}

\begin{proof}
Using Corollary \ref{homsetsagree}, we can identify the mapping space
$\bHom_{\Fun(K,\calC) }( p, \delta(Y) )$ with the fiber
$\calC^{p/} \times_{\calC} \{Y\}$, for each object $Y \in \calC$.
Under this identification, the map $\phi_{Y}$ can be identified with
the fiber over $Y$ of the composition
$$ \calC^{X/} \stackrel{\phi'}{\rightarrow} \calC^{ \overline{p}/ } \stackrel{\phi''}{\rightarrow} \calC^{p/},$$
where $\phi'$ is a section to the trivial fibration $\calC^{\overline{p}/} \rightarrow \calC^{X/}$. 
The map $\phi''$ is a left fibration (Proposition \ref{sharpenn}). Condition $(i)$ is equivalent to the requirement that $\phi''$ be a trivial Kan fibration, and condition $(ii)$ is equivalent to the
requirement that each of the maps
$$ \phi''_{Y}: \calC^{ \overline{p}/} \times_{\calC} \{Y \} \rightarrow \calC^{p/} \times_{\calC} \{Y\}.$$
is a homotopy equivalence of Kan compexes (which, in view of Lemma \ref{toothie2}, is equivalent to the requirement that $\phi''_{Y}$ be a trivial Kan fibration). The equivalence of these two conditions now follows from Lemma \ref{toothie}.
\end{proof}

The key to Theorem \ref{colimcomparee} is the following result, which compares
the construction of diagram categories in the $\infty$-categorical and simplicial settings:

\begin{proposition}\label{gumby444}\index{gen}{straightening of diagrams}
Let $S$ be a small simplicial set, $\calC$ a small simplicial category, and 
$u: \sCoNerve[S] \rightarrow \calC$ an equivalence. Suppose that
$\bfA$ is a combinatorial simplicial model category, and let
$\calU$ be a $\calC$-chunk of $\bfA$ (see Definition \ref{cattusi}). 
Then the induced map
$$ \sNerve ( \calU^{\calC} )^{\degree}  \rightarrow \Fun(S,
\sNerve(\calU^{\degree}) )$$ is a categorical equivalence of
simplicial sets.
\end{proposition}

\begin{remark}
In the statement of Proposition \ref{gumby444}, it makes no difference whether we regard
$\bfA^{\calC}$ as endowed with the projective or injective model structure.
\end{remark}

\begin{remark}
An analogous result was proved by Hirschowitz and Simpson; see \cite{hirschowitz}.
\end{remark}

\begin{proof}
Choose a regular cardinal $\kappa$ such that $S$ and $\calC$ are $\kappa$-small.
Using Lemma \ref{exchunk}, we can write $\calU$ as a $\kappa$-filtered colimit
of small $\calC$-chunks $\calU'$ contained in $\calU$. Since the collection of
categorical equivalences is stable under filtered colimits, it will suffice to prove the
result after replacing $\calU$ by each $\calU'$; in other words, we may suppose that
$\calU$ is small. 

According to Theorem \ref{biggier}, we may identify
the homotopy category of $\sSet$ (with respect to the Joyal model
structure) with the homotopy category of $\sCat$. We now observe that, because
$\sNerve( \calU^{\degree})$ is an $\infty$-category, the simplicial set
$\Fun(S,\sNerve( \calU^{\degree}))$ can be identified with an exponential
$[ \Nerve(\calU^{\degree}) ]^{ [S] }$ in the homotopy category $\h{\sSet}$. We now conclude by applying Corollary \ref{sniffle}.
\end{proof}

One consequence of Proposition \ref{gumby444} is that every homotopy coherent diagram
in a suitable model category $\bfA$ can be ``straightened'', as we indicated in Remark \ref{remmt}.

\begin{corollary}\label{strictify}
Let $\calI$ be a fibrant simplicial category, $S$ a simplicial
set, and $p: \sNerve(\calI) \rightarrow S$ a map. Then it is possible to find the following:
\begin{itemize}
\item[$(1)$] A fibrant simplicial category $\calC$.
\item[$(2)$] A simplicial functor $P: \calI \rightarrow \calC$.
\item[$(3)$] A categorical equivalence of simplicial sets
$j: S \rightarrow \sNerve(\calC)$.
\item[$(4)$] An equivalence between $j \circ p$ and $\sNerve(P)$, as objects of the
$\infty$-category $\Fun( \Nerve(\calI), \sNerve(\calC))$.
\end{itemize}
\end{corollary}

\begin{proof}
Choose an equivalence $i: \sCoNerve[S] \rightarrow \calC_0$, where
$\calC_0$ is fibrant; let $\bfA$ denote the model category of
simplicial presheaves on $\calC_0$ (endowed with the {\em injective} model structure). Composing $i$ with the Yoneda
embedding of $\calC_0$, we obtain a fully faithful simplicial
functor $\sCoNerve[S] \rightarrow \bfA^{\degree}$, which we may
alternatively view as a morphism $j_0: S \rightarrow \sNerve
(\bfA^{\degree})$.

We now apply Proposition \ref{gumby444} to the case where $u$ is the
counit map $\sCoNerve[\sNerve(\calI)] \rightarrow
\calI$. We deduce that the natural map
$$ \sNerve (\bfA^{\calI})^{\degree} \rightarrow \Fun( \Nerve(\calI), \sNerve(\bfA^{\degree}
) )$$ is an equivalence. From the essential
surjectivity, we deduce that $j_0 \circ p$ is equivalent to
$\sNerve(P_0)$, where $P_0: \calI \rightarrow \bfA^{\degree}$ is a
simplicial functor.

We now take $\calC$ to be the essential image of $\sCoNerve[S]$ in
$\bfA^{\degree}$, and note that $j_0$ and $P_0$ factor uniquely
through maps $j: S \rightarrow \sNerve(\calC)$, $P: \calI
\rightarrow \calC$ which possess the desired properties.
\end{proof}

We now return to our main result.

\begin{proof}[Proof of Theorem \ref{colimcomparee}:]
Let $\bfA$ denote the category $\Set_{\Delta}^{\calC}$, endowed with the projective
model structure. Let $j: \calC^{op} \rightarrow \bfA$ denote the Yoneda embedding, and let $\calU$ denote the full subcategory of $\bfA$ spanned by those objects which are weakly equivalent
to $j(C)$ for some $C \in \calC$, so that $j$ induces an equivalence of simplicial categories
$\calC^{op} \rightarrow \calU^{\degree}$. Choose a trivial injective cofibration
$j \circ F \rightarrow F'$, where $F'$ is a injectively fibrant object of $\bfA^{\calI^{op}}$. 
Let $f': \Nerve(\calI)^{op} \rightarrow \Nerve( \calU^{\degree} )$
be the nerve of $F'$, and let $C' = j(C)$, so that the maps 
$\{ \eta_{I}: F(I) \rightarrow C \}_{I \in \calI}$ induce a natural transformation $\alpha: \delta(C') \rightarrow f'$, where $\delta: \Nerve( \calU^{\degree}) \rightarrow \Fun( \Nerve(\calI)^{op}, \Nerve(\calU^{\degree}) )$ denotes the diagonal embedding. In view of Lemma \ref{kamma}, condition
$(1)$ admits the following reformulation:

\begin{itemize}
\item[$(1')$] For every object $A \in \calU^{\degree}$, composition with
$\alpha$ induces a homotopy equivalence 
$$ \bHom_{ \Nerve(\calU^{\degree}) }( A,C') \rightarrow \bHom_{ \Fun( \Nerve(\calI)^{op}, \Nerve(\calU^{\degree})) }( \delta(A),f' ).$$
\end{itemize}

Using Proposition \ref{gumby444}, we can reformulate this condition again:

\begin{itemize}
\item[$(1'')$] For every object $A \in \calU^{\degree}$, the canonical map
$$ \bHom_{ \bfA }( A,C') \rightarrow \bHom_{ \bfA^{ \calI^{op}}}( \delta'(A), F')$$
is a homotopy equivalence, where $\delta': \bfA \rightarrow \bfA^{\calI^{op}}$ denotes
the diagonal embedding. 
\end{itemize}

Let $B \in \bfA$ be a limit of the diagram $F'$, so we have a canonical map
$\beta: C' \rightarrow B$ between fibrant objects of $\bfA$. Condition $(2)$ is
equivalent to the assertion that $\beta$ is a weak equivalence in $\bfA$, while
condition $(1'')$ is equivalent to the assertion that composition with
$\beta$ induces a homotopy equivalence
$$ \bHom_{\bfA}(A, C') \rightarrow \bHom_{\bfA}(A, B)$$
for each $A \in \calU^{\degree}$. The implication $(2) \Rightarrow (1'')$ is clear.
Conversely, suppose that $(1'')$ is satisfied. For each $X \in \calC$, the
object $j(X)$ belongs to $\calU^{\degree}$, so that $\beta$ induces a
homotopy equivalence
$$ C'(X) \simeq \bHom_{\bfA}( j(X), C') \rightarrow \bHom_{\bfA}( j(X), B)
\simeq B(X).$$
It follows that $\beta$ is a weak equivalence in $\bfA$ as desired.
\end{proof}

%\begin{remark}
%Theorem \ref{colimcompare} can be generalized to the case where indexing category $\calI$ is a {\em simplicial} category. However, we will not need this generalization: our primary interest in homotopy colimits is as a tool for establishing properties of $\infty$-categorical colimits. In view of Proposition \ref{cofinalcategories}, questions about arbitrary colimits can usually be reduced to questions about colimits indexed by ordinary categories.
%\end{remark}

%Suppose
%given a fibrant simplicial category $\calC$ and a diagram $p: \calI \rightarrow \calC$, where
%$\calI$ is an ordinary category. Let $S = \sNerve(\calC)$, and let $q: \Nerve(\calI) \rightarrow S$ be the induced map. Every object $Z \in \calC$ determines a functor $F_Z: \calI^{op} \rightarrow \sSet$, given by the formula
%$$ F_Z(I) = \bHom_{\calC}( p(I), Z ).$$
%The main step of the proof of Theorem \ref{colimcompare} is to reconstruct the functor
%$F_Z$ using only the $\infty$-category $S$. For each
%object $I \in \calI$, $Z \in \calC$, we may regard $q(I)$ and $Z$ as objects of $S$, and form
%the mapping space $\Hom^{\lft}_{S}( q(I), Z).$ 
%In virtue of Theorem \ref{biggie}, this space is homotopy equivalent to $F_Z(I)$. Unfortunately, $\Hom^{\lft}_{S}(q(I), Z)$ does not depend functorially on $I$.

%To obtain a simplicial set which {\em is} functorial in $I$, we
%make two observations. First of all, $\Hom^{\lft}_{S}(q(I), Z)$ is
%the fiber of the left fibration $S_{q(I)/} \rightarrow S$ over the
%point $Z$. For each $I \in \calI$, let $q_I$ denote the composition
%$$ \Nerve(\calI_{/I}) \rightarrow \Nerve(\calI) \stackrel{q}{\rightarrow} S.$$
%Since $\calI_{/I}$ contains $\id_{I}$ as a final object, the natural map
%$$ S_{q_I/} \rightarrow S_{q(I)/}$$ is a trivial fibration of simplicial sets.
%We now define $G_Z(I)$ to be the fiber product
%$$ \{ Z \} \times_{S} S_{q_I/},$$ and we observe that $G_{Z}$
%is functor from $\calI^{op}$ to the category of Kan complexes.

%Our proof of Theorem \ref{colimcompare} hinges on a comparison between the
%functors $F_Z$ and
%$G_Z$. Morally, they are the same: $G_Z(I)$ is homotopy equivalent to
%the fiber of $S_{q(I)/} \rightarrow S$ over $Z$, which may be
%identified with the simplicial set $\Hom^{\lft}_{S}(q(I),Z)$. However, this equivalence
%is not functorial in $I$. Consequently, we cannot use it to compare the functors $F_Z$
%and $G_Z$ directly. Nevertheless:

%\begin{lemma}\label{techytech}
%Let $\calC$ be a fibrant simplicial category, $\calI$ an ordinary category,
%$p: \calI \rightarrow \calC$ a functor, and $Z \in \calC$ an object.
%Let $$F_{Z}, G_{Z}: \calI^{op} \rightarrow \Kan$$ be defined as above.

%There exists a functor $H$ from $\calI^{op}$ to the category of
%compactly generated Hausdorff spaces which is equipped with
%natural transformations
%$$ |F_Z(I)| \stackrel{\alpha_I}{\leftarrow} H(I)
%\stackrel{\beta_I}{\rightarrow} |G_Z(I)|$$ having the property that
%for each $I \in \calI$, the maps $\alpha_I$ and $\beta_I$ are homotopy equivalences.
%\end{lemma}

%The proof of Lemma \ref{techytech} is somewhat technical, and will be given in \S \ref{techtolem}. In this section, we will show that Lemma \ref{techytech} implies Theorem \ref{colimcompare}.

%It follows from Lemma \ref{techytech} that
%$F_Z$ and $G_Z$ are weakly equivalent as objects of $(\sSet)^{
%\calI^{op}}$. Here we regard $(\sSet)^{\calI^{op}}$ as endowed with the projective
%model structure described in \S \ref{quasilimit3}. Our next step is to show that
%the functor $G_Z$ is (strongly) fibrant, and therefore suitable for use in recognizing homotopy colimits. First, we need a lemma.

%\begin{lemma}\label{rub5}
%Let $X$ be a simplicial set, and let $j: A \rightarrow B$ be a weak homotopy equivalence
%of simplicial sets. The induced map
%$$ (X \times \bd \Delta^1) \coprod_{ X \times A \times \bd \Delta^1}
%(X \times A \times \Delta^1) \rightarrow (X \times \bd \Delta^1)
%\coprod_{ X \times B \times \bd \Delta^1} (X \times B \times
%\Delta^1)$$ is a categorical equivalence.
%\end{lemma}

%\begin{proof}
%For every map $j: A \rightarrow B$ of simplicial sets, let
%$$F(j): \bd \Delta^1 \coprod_{ A \times \bd \Delta^1 } (A \times \Delta^1) \rightarrow
%\bd \Delta^1 \coprod_{ B \times \bd \Delta^1 } (B \times \Delta^1).$$
%We need to show that if $j$ is a weak homotopy equivalence, then $F(j)$ is a categorical equivalence. This will imply the desired result, since forming the product with $X$ preserves categorical equivalences.

%First suppose that $j$ is the inclusion $\{0\} \subseteq \Delta^1$. In this case, the result follows by a simple explicit computation.

%Now suppose that $j: A \rightarrow B$ is a cofibration such that $F(j)$ is a categorical equivalence, and let $j': A' \rightarrow B'$ be another cofibration of simplicial sets. Let 
%$$j \wedge j': (A \times B') \coprod_{A \times A'} (B \times A') \rightarrow (B \times B')$$
%denote the smash product of $j$ with $j'$. We now observe that there is a homotopy
%pushout diagram
%$$ \xymatrix{ \id_{ B' \times \bd \Delta^1} \ar[r] \ar[d] & \id_{\bd \Delta^1} \ar[d] \\
%F(j) \wedge j' \ar[r] & F(j \wedge j') }$$
%in the category of arrows of $\sSet$. It follows that $F(j \wedge j')$ is a categorical
%equivalence.

%The class of all cofibrations $j$ such that $F(j)$ is a categorical equivalence is saturated. The above arguments show that it contains the smash product of $\{0\} \subseteq \Delta^1$ with any other cofibration. By Proposition \ref{usejoyal}, we deduce that $F(j)$ is a categorical equivalence whenever $j$ is a left anodyne. A dual argument shows that $F(j)$ is a categorical equivalence whenever $j$ is right anodyne. It follows that $F(j)$ is a categorical equivalence whenever $j$
%is anodyne: that is, whenever $j$ is simultaneously a cofibration and a weak homotopy equivalence of simplicial sets.

%We now treat the general case, when $j$ is not assumed to be a cofibration.
%Choose a cofibration of simplicial sets $i: A \rightarrow K$, where $K$ is a contractible Kan complex. Consider the diagram
%$$ \xymatrix{ & K \times B \ar[d]^{\pi_{B}} \ar[d] \\
%A \ar[ur]^{i \times j} \ar[r]^{j} & B. }$$
%Since $i \times j$ is a cofibration and a weak homotopy equivalence, the above arguments
%show that $F(i \times j)$ is a categorical equivalence. Consequently, to prove that $F(j)$ is a categorical equivalence, it suffices to show that $F(\pi_{B})$ is a categorical equivalence.
%But $\pi_{B}$ admits a section $s: B \rightarrow K \times B$. The map $s$ is a cofibration
%and a weak homotopy equivalence, so that $F(s)$ is a categorical equivalence. Since
%$F(s)$ is a right inverse of $F( \pi_{B})$, we conclude that $F( \pi_{B} )$ is a categorical equivalence.
%\end{proof}

%Before giving the next proof, we recall the notion of a {\em coend}.\index{gen}{coend}
%Suppose given a pair of functors $T: \calI \rightarrow \calC$, $T': \calI^{op} \rightarrow \calC$,
%where $\calC$ is any category with finite products and (small) colimits. The coend $$\int_{I \in \calI} T(I) \times T'(I)$$ is defined to be the coequalier of the evident pair of maps
%$$ \xymatrix{ \coprod_{f: I \rightarrow I'} T(I) \times T'(I') \ar@<.4ex>[r] \ar@<-1ex>[r] &
%\coprod_{I} T(I) \times T'(I) }$$

%\begin{proposition}\label{obser}
%Let $\calC$ be a fibrant simplicial category, $\calI$ an ordinary category,
%$p: \calI \rightarrow \calC$ a functor, and $Z \in \calC$ an object.
%The functor $G_Z: \calI^{op} \rightarrow \sSet$ is strongly fibrant $($Definition \ref{projinj}$)$
%when considered as an object of $(\sSet)^{\calI^{op}}$.
%\end{proposition}

%\begin{proof}
%Let $\calF \subseteq \calF'$ be an inclusion of functors
%$\calI^{op} \rightarrow \sSet$, which induces a weak equivalence
%when evaluated at each $I \in \calI$. We need to show that $G_Z$ has
%the right extension property with respect to the inclusion $\calF
%\subseteq \calF'$. Translating this into the language of
%$\infty$-categories, we see that it suffices to prove that $S = \sNerve(\calC)$ has
%the right extension property with respect to the inclusion
%$$j: \Nerve(\calI) \coprod_{M(\calF)} M(\calF)^{\triangleright} \subseteq
%\Nerve(\calI) \coprod_{M(\calF')} M(\calF')^{\triangleright},$$ where
%$M(\calF')$ is defined to be the coend $\int_{I \in
%\calI} \calF'(I) \times \Nerve(\calI_{/I})$, and $M(\calF) \subseteq M(\calF')$ is
%defined similarly. Since $S$ is an $\infty$-category, it will suffice to
%prove that $j$ is a categorical equivalence. Working cell-by-cell
%on $\Nerve(\calI)$, we may reduce to the problem of showing that the inclusions
%$$ j_I: \Delta^n \coprod_{ \Delta^n \times \calF(I) } ( \Delta^n \times
%\calF(I)^{\triangleright} ) \subseteq \Delta^n \coprod_{ \Delta^n
%\times \calF'(I) } (\Delta^n \times \calF'(I)^{\triangleright} )$$
%are categorical equivalences. 

%In view of Proposition \ref{rub3},
%we are free to replace $\calF(I)^{\triangleright} = \calF(I) \star \Delta^0$ by
%$\calF(I) \diamond \Delta^0$ and $\calF'(I)^{\triangleright}$ by $\calF'(I) \diamond \Delta^0$.
%After this replacement, the relevant map is a pushout of the inclusion
%$$ (\Delta^n \times \bd \Delta^1) \coprod_{ \calF(I) \times \Delta^n
%\times \bd \Delta^1 } (\calF(I) \times \Delta^n \times \Delta^1)
%\subseteq (\Delta^n \times \bd \Delta^1) \coprod_{ \calF'(I) \times
%\Delta^n \times \bd \Delta^1} (\calF'(I) \times \Delta^n \times
%\Delta^1).$$ This map is a cofibration, and furthermore a categorical equivalence by Lemma \ref{rub5}.
%\end{proof}

%\begin{proof}[Proof of Theorem \ref{colimcompare}]
%Let $\calC$ be a fibrant simplicial category, $\calI$ an ordinary category, and
%$\overline{p}: \calI \star \{x\} \rightarrow \calC$ a functor; let $\overline{q}: \Nerve(\calI)^{\triangleright} \rightarrow S = \sNerve(\calC)$ be the induced map of $\infty$-categories. We wish to show that 
%$\overline{p}$ is a homotopy colimit of $p = \overline{p}| \calI$ if and only if $\overline{q}$ is a colimit of $q = \overline{q} | \Nerve(\calI)$. 

%By definition, $\overline{p}$ is a homotopy colimit of $p$ if and only if, for each $Z \in \calC$,
%the associated functor $F_{Z}$ exhibits $F_Z(x)$ as a homotopy limit of the diagram
%$F_Z | \calI$. In view of Proposition \ref{techytech}, this is equivalent to the
%assertion that $G_Z$ exhibits $G_Z(x)$ as a homotopy limit of the
%diagram $G|\calI$. Applying Proposition \ref{obser} to
%$\calI$, we deduce that the homotopy limit of $G_Z|\calI$ is
%simply the limit of the diagram $G_Z|\calI$. Consequently, $\overline{p}$ is
%a homotopy colimit of $p$ if and only if
%$$ \phi_Z: G_Z(x) \rightarrow \projlim_{I \in \calI} G_Z(I)$$
%is a weak homotopy equivalence (for each $Z \in \calC$).

%We now observe that $G_Z(x)$ is isomorphic to the fiber of
%$S_{\overline{q}/} \rightarrow S$ over $Z$, while the limit $\projlim_{I \in \calI} G_Z(I)$ is isomorphic to the fiber of $S_{q/} \rightarrow S$ over $Z$. Thus, $\overline{p}$ is a homotopy colimit of
%$p$ if and only if the projection $\phi: S_{\overline{q}/} \rightarrow S_{q/}$ induces a homotopy equivalence of Kan complexes after taking the fiber over each vertex $Z$ of $S$. If $\overline{q}$
%is a colimit of $q$, then $\phi$ is a trivial fibration, so that $\overline{p}$ is a homotopy colimit of $p$. For the converse, we observe that $\phi$ is a left fibration. Consequently, $\phi$ is a trivial fibration if and only if the fibers of $\phi$ are contractible Kan complexes. If $\overline{p}$ is a homotopy colimit of $p$, then for each $Z \in S$, the induced map
%$$ \phi_Z: S_{\overline{q}/} \times_{S} \{Z\} \rightarrow S_{q/} \times_{S} \{Z\}$$ is a left fibration 
%and a homotopy equivalence of Kan complexes, hence a (trivial) Kan fibration (Lemma \ref{toothie2}). It follows that the fibers of $\phi_Z$ are contractible, as desired.
%\end{proof}

\begin{corollary}\label{limitsinmodel}
Let $\bfA$ be a combinatorial simplicial model category. The associated $\infty$-category
$S = \sNerve( \bfA^{\degree})$ admits $($small$)$ limits and colimits.
\end{corollary}

\begin{proof}
We give the argument for colimits; the case of limits follows by a dual argument. Let $p: K \rightarrow S$ be a (small) diagram in $S$. 
By Proposition \ref{cofinalcategories}, there exists a (small) category
$\calI$ and a cofinal map $q: \Nerve(\calI) \rightarrow K$. Since $q$ is cofinal, $p$ has a colimit in $S$ if and only if $p \circ q$ has a colimit in $S$; thus we may reduce to the case where $K = \Nerve(\calI)$. 

Using Proposition \ref{gumby444}, we may suppose that $p$ is the nerve of a injectively fibrant diagram $p': \calI \rightarrow \bfA^{\degree}$. Let $\overline{p'}: \calI \star \{x\} \rightarrow \bfA^{\calI}$ be a limit of $p'$, so that $\overline{p}'$ is a homotopy limit diagram in $\bfA$. Now choose a trivial fibration $\overline{p}'' \rightarrow \overline{p}'$ in $\bfA^{\calI}$, where $\overline{p}''$ is cofibrant. The simplicial nerve of $\overline{p}''$ determines a colimit diagram
$\overline{f}: \Nerve(\calI)^{\triangleright} \rightarrow S$, by Theorem \ref{colimcomparee}. We now
observe that $f = \overline{f} | \Nerve(\calI)$ is equivalent to $p$, so that $p$ also admits a colimit in $S$.
\end{proof}

%\subsection{Completion of the Proof}\label{techtolem}

%In this section, we will finish the proof of Theorem \ref{colimcompare} by establishing Lemma \ref{techytech}. Throughout this section, we will fix a fibrant simplicial category $\calC$, an ordinary category $\calI$, a functor $p: \calI \rightarrow \calC$, and an object $Z \in \calC$. We let
%$F_{Z},G_{Z}: \calI^{op} \rightarrow \Kan$ denote the functors constructed in \S \ref{quasilimit4}.
%We will construct a functor $H: \calI^{op} \rightarrow \CG$ and natural transformations
%$$ |F_{Z}| \stackrel{\alpha}{\leftarrow} H \stackrel{\beta}{\rightarrow} |G_{Z}|$$
%which induce homotopy equivalences for each object $I \in \calI$; here
%$\CG$ denotes the category of compactly generated Hausdorff spaces. 

%\begin{notation}
%We let $S$ denote the nerve of the simplicial category $\calC$, and
%$q: \Nerve(\calI) \rightarrow S$ the induced map. For $I \in \calI$, we let
%$q_I$ denote the composition
%$$ \Nerve(\calI_{/I}) \rightarrow \Nerve(\calI) \stackrel{q}{\rightarrow} S,$$
%and $q^I: \Nerve(\calI_{I/}) \rightarrow S_{q_I/}$
%the induced map.
%\end{notation}

%\begin{definition}\label{defHt}
%For each $I \in \calI$, the topological space $H(I)$ is defined as follows.

%\begin{itemize}
%\item[$(1)$] Let $\sigma$ denote a commutative diagram
%$$ \xymatrix{ \Delta^{ \{0,\ldots, m\} } \ar@{^{(}->}[r] \ar[d]^{\sigma_0} & \Delta^n \ar[d]^{\sigma_1} & &
%\Delta^{ \{m+1, \ldots, n\} } \ar@{_{(}->}[ll] \ar[d] \\
%\Nerve( \calI_{I/} ) \ar[r]^{q^I} & S_{q_I/} \ar[r] & S & \{Z \}. \ar@{_{(}->}[l], }$$
%and let 
%$C_{\sigma} = \{ k: [n] \rightarrow [0,1]: (k(n) =
%1) \wedge (\exists i \leq m) [k(i) = 1] \} \subseteq [0,1]^{n+1}$.
%Then there exists a continuous map $$C_{\sigma} \rightarrow H(I)$$
%$$ k \mapsto \sigma[[k]].$$
%
%\item[$(2)$] Let $\sigma$ be as in $(1)$, and let
%$k,k' \in C_{\sigma}$ be such that 
%$$(\exists i > m) [k(i) = 1 \wedge (\forall j \leq i) [k(j) = k'(j)] ].$$
%Then $\sigma[[k]] = \sigma[[k']] \in H(I)$.

%\item[$(3)$] Let $\sigma$ be as in $(1)$, let $k \in C_{\sigma}$, and suppose
%that $k(i) = 0$ for some $0 \leq i < n$. Let $\sigma'$ denote the commutative diagram obtained
%from $\sigma$ by deleting the $i$th vertex of $\Delta^n$ (and $\Delta^m$, if $i \leq m$), and let
%$k' \in C_{\sigma'}$ be the function obtained by omitting the value of $k$ on $i$. Then
%$$ \sigma[[k]] = \sigma'[[k']] \in H(I).$$

%\item[$(4)$] The topological space $H(I)$ is the quotient
%of the disjoint union $\coprod_{\sigma} C_{\sigma}$ obtained by imposing
%the relations indicated in $(2)$ and $(3)$.
%\end{itemize}
%\end{definition}

%We observe that $H(I)$ is contravariantly functorial in $I$. More precisely,
%suppose that $\gamma: I' \rightarrow I$ is a morphism in $I$, and let
%$\sigma$ be a commutative diagram in $(1)$. We can form a composite diagram
%$$ \xymatrix{ \Delta^{ \{0,\ldots, m\} } \ar@{^{(}->}[r] \ar[d]^{\sigma_0} & \Delta^n \ar[d]^{\sigma_1} & &
%\Delta^{ \{m+1, \ldots, n\} } \ar@{_{(}->}[ll] \ar[d] \\
%\Nerve( \calI_{I/} ) \ar[r]^{q^I} \ar[d] & S_{q_I/} \ar[r] \ar[d] & S \ar@{=}[d]& \{Z \} \ar@{=}[d] \ar@{_{(}->}[l] \\
%\Nerve( \calI_{I'/} ) \ar[r]^{q^{I'}} & S_{q_{I'}/} \ar[r] & S & \{Z\}. \ar@{_{(}->}[l] }$$
%Let $\sigma'$ denote the commutative diagram obtained by omitting the middle line, so that
%we have a map $C_{\sigma'} \rightarrow H(I')$. There is a uniquely determined map $H(\gamma)$
%with the property that each of the diagrams
%$$ \xymatrix{ C_{\sigma} \ar@{=}[r] \ar[d] & C_{\sigma'} \ar[d] \\
%H(I) \ar[r]^{H(\gamma)} & H(I') }$$ 
%is commutative.

%\begin{definition}
%If $I$ is an object of $\calI$, then the subspace $H'(I) \subseteq
%H(I)$ is defined to be the images of all maps
%$$ C_{\sigma} \rightarrow H(I) $$
%where $\sigma$ is a commutative diagram of the form
%$$ \xymatrix{ \Delta^{ 0 } \ar@{^{(}->}[r] \ar[d]^{ \id_{I} } & \Delta^n \ar[d]^{\sigma_1} & &
%\Delta^{ \{1, \ldots, n\} } \ar@{_{(}->}[ll] \ar[d] \\
%\Nerve( \calI_{I/} ) \ar[r]^{q^I} & S_{q_I/} \ar[r] & S & \{Z \}. \ar@{_{(}->}[l], }$$
%where $\sigma_0$ classifies a map
%$$ (\Nerve( \calI_{/I} ) \star \{I \}) \star \Delta^{ \{1, \ldots, n \} }$$
%which factors through $\Nerve(\calI_{/I}) \star \Delta^{ \{1, \ldots, n\} }$.
%\end{definition}

%\begin{warning}
%The subspaces $H'(I) \subseteq H(I)$ are not stable under the maps
%$H(\gamma)$ defined above. In other words, $H'(I)$ does not depend functorially on $I$.
%\end{warning}

%\begin{lemma}\label{apul1}
%For each object $I \in \calI$, the inclusion $H'(I) \rightarrow
%H(I)$ is a homotopy equivalence.
%\end{lemma}

%\begin{proof}
%We will define a map $h: H(I) \times [0,1] \rightarrow H(I)$
%having the property that $h| H(I) \times \{0\}$ is the identity on
%$H(I)$, and $h|H(I) \times \{1\}$ is a retraction $H(I)
%\rightarrow H'(I)$ (which is therefore a homotopy inverse to the inclusion $H'(I) \subseteq H(I)$).
%( The map $h$ will {\em} not have the property that it induces the identity $H'(I) \times \{t\}
%\rightarrow H'(I)$ for $0 < t < 1$, but this is not important. )

%Let $\sigma$ be a diagram as in Definition \ref{defHt}, so that the map
%$C_{\sigma} \rightarrow H(I)$ is defined. We can identify the map
%$\sigma_0: \Delta^m \rightarrow \Nerve( \calI_{I/})$ with a diagram
%$$ I \rightarrow \sigma_0(0) \rightarrow \sigma_0(1) \rightarrow \ldots
%\rightarrow \sigma_0(m)$$ in the category $\calI$. We define $\sigma'_0:
%\Delta^{m+1} \rightarrow \Nerve(\calI_{I/})$ to classify the diagram
%$$ I \stackrel{\id_I}{\rightarrow} I \rightarrow \sigma_0(0) \rightarrow
%ldots \rightarrow \sigma_0(m).$$ 
%Similarly, if $\sigma_1: \Delta^n \rightarrow S_{q_I/}$ classifies a map
%$\Nerve(\calI_{/I}) \star \Delta^n \rightarrow S$, then we let
%$\sigma'_1$ classify the induced map
%$$ (\Nerve(\calI_{/I}) \star \{I\} ) \star \Delta^n \rightarrow S,$$
%obtained by composing with the retraction
%$$ \Nerve(\calI_{/I}) \star \{ I \} \rightarrow \Nerve(\calI_{/I}).$$
%Together, $\sigma'_0$ and $\sigma'_1$ determine a diagram
%$$ \xymatrix{ \Delta^{ \{0,\ldots, m+1\} } \ar@{^{(}->}[r] \ar[d]^{\sigma'_0} & \Delta^{n+1} \ar[d]^{\sigma'_1} & &
%\Delta^{ \{m+1, \ldots, n\} } \ar@{_{(}->}[ll] \ar[d] \\
%\Nerve( \calI_{I/} ) \ar[r]^{q^I} & S_{q_I/} \ar[r] & S & \{Z \}. \ar@{_{(}->}[l]. }$$
%We define a map $h_{\sigma}: C_{\sigma} \times [0,1] \rightarrow C_{\sigma'}$ as follows:
%$$ h_{\sigma}(k)(0) = \begin{cases} 2t & \text{if } t \leq \frac{1}{2} \\
%1 & \text{if } t \geq \frac{1}{2} \end{cases} $$
%$$ h_{\sigma}(k)(i) = \begin{cases} k(i-1) & \text{if } t \leq \frac{1}{2}, i
%> 0 \\
%k(i-1)(2-2t) & \text{if } t \geq \frac{1}{2}, i > 0. \\
%\end{cases}$$
%The desired map $h: H(I) \times [0,1] \rightarrow H(I)$ is uniquely determined by
%the requirement that the diagrams
%$$ \xymatrix{ C_{\sigma} \times [0,1] \ar[r]^{h_{\sigma}} \ar[d] & C_{\sigma'} \ar[d] \\
%H(I) \times [0,1] \ar[r]^{h} & H(I) }$$
%commute.
%\end{proof}
%
%We now construct the natural transformation
%$$ \alpha_{I}: H(I) \rightarrow |F_Z(I)|.$$
%First, we need to introduce a bit of notation. Let $\sigma$ be as in Definition \ref{defHt}, and let
%$0 \leq i \leq m$. We let $C_{\sigma}[i]$ denote the closed subset of $C_{\sigma}$ consisting
%of those points $k \in C_{\sigma}$ such that $k(i) = 1$. In this case, 
%$\sigma_0(i)$ determines a morphism $\eta: I \rightarrow I'$ in $\calI$. Let
%$\tau = \sigma_1 | \Delta^{ \{i, \ldots, n\} }$, and let $q = k' | \{ i, \ldots, n \}$. Using the notation
%of \S \ref{compp2}, we have a morphism $\tau[q] \in | \bHom_{ \sCoNerve[S] }( p(I'), Z) |$. 
%Composing with $\eta$ and with the counit map, we obtain a point
%$\psi_{i}[k] \in | \bHom_{\calC}( p(I), Z) | = |F_Z(I)|.$ Allowing $k$ to vary, we obtain a continuous map
%$$ \psi_{i}: C_{\sigma}[i] \rightarrow |F_{Z}(I)|.$$
%The map $\alpha_{I}$ is determined by the requirement that the diagrams
%$$ \xymatrix{ C_{\sigma}[i] \ar[drr]^{\psi_{i}} \ar@{^{(}->}[r] & 
%C_{\sigma} \ar[r] & H(I) \ar[d]^{\alpha_{I}} \\
%& & |F_{Z}(I)| }$$
%commute, for all $\sigma$ and $i$ as above. The uniqueness of
%$\alpha_{I}$ follows from the observation that 
%$C_{\sigma} = \bigcup_{0 \leq i \leq m} C_{\sigma}[i]$ (by construction). 
%The existence follows readily by examining the relations in Definition
%\ref{defHt}. It is not difficult to check that this construction is functorial in $I$.
%The following claim is somewhat less obvious:

%\begin{lemma}
%Let $I \in \calI$ be an object. Then the map $\alpha_I: H(I)
%\rightarrow |F_Z(I)|$ is a homotopy equivalence.
%\end{lemma}

%\begin{proof}
%For each $\sigma$ as in Definition \ref{defHt} and each
%$i \leq m$, we observe that the composition
%$$C_{\sigma}[i] \rightarrow H(I) \stackrel{\alpha_I}{\rightarrow} |F_{Z}(I)|$$
%factors through $\bHom_{| \sCoNerve[S] |}( p(I), Z)$ (by construction). These maps are
%not strictly compatible with one another, so the map $\alpha_I$ does not itself factor
%through $\bHom_{ | \sCoNerve[S] |}(p(I),Z)$. However, it is easy to see that
%these maps {\em are} compatible when restricted to $H'(I)$, so we have a commutative
%diagram of topological spaces
%$$ \xymatrix{ H'(I) \ar@{^{(}->}[d] \ar[r]^-{\alpha'_I} & \bHom_{| \sCoNerve[S]|}(p(I),Z) \ar[d] \\
%H(I) \ar[r]^{\alpha_I} & |F_Z(I)|. }$$
%The left vertical map is a homotopy equivalence by Lemma \ref{apul1}, and the
%right vertical map is a homotopy equivalence by Theorem \ref{biggie}. Consequently,
%it will suffice to prove that $\alpha'_{I}$ is a homotopy equivalence.

%We observe that $\alpha_I'$ factors as a composition $$ H'(I) \stackrel{\theta_1}{\rightarrow}
%|S_{q_I/} \times_{S}
 %\{Z\}|_{\calQ^{\bigdot}} \stackrel{\theta_2}{\rightarrow} |S_{p(I)/} \times_{S}
 %\{Z\}|_{\calQ^{\bigdot}} \stackrel{\theta_3}{\rightarrow}
% \bHom_{| \sCoNerve[S] |}(p(I),Z).$$
%We can show that $\theta_1$ is a homotopy equivalence using the
%the proof of Lemma \ref{oof}. The map $\theta_2$ is a homotopy equivalence
%because the projection $S_{q_I/} \rightarrow S_{p(I)/}$ is a trivial fibration of simplicial sets.
%Proposition \ref{wiretrack} implies that $\theta_3$ is a homotopy equivalence. It follows that $\alpha'_I$ is a homotopy equivalence, as desired.
%\end{proof}

%To complete the proof of Lemma \ref{techytech}, it remainss to construct the natural transformation
%$$ \beta_{I}: H(I) \rightarrow | G_Z(I) |.$$ We will obtain $\beta_{I}$ as a composition
%$$ H(I) \stackrel{\beta'_{I}}{\rightarrow} | G_{Z}(I) |_{\calQ^{\bigdot}} \rightarrow | G_{Z}(I) |.$$
%The second map is a homotopy equivalence (Proposition \ref{baby}), so it will suffice to construct $\beta'_{I}$ and to prove that $\beta'_{I}$ is a homotopy equivalence. We first construct an auxiliary space $X_{I}$.

%\begin{definition}
%For $I \in \calI$, the topological space $X_I$ is defined as follows:

%\begin{itemize}
%\item[$(1)$] Let $\tau: \Delta^n \rightarrow S_{q_I/}$ fit in a commutative diagram
%$$ \xymatrix{ \Delta^n \ar[d]^{\tau} \ar@{=}[rr] & & \Delta^n \ar[d] \\
%S_{q_I/} \ar[r] & S & \{Z \}. \ar@{_{(}->}[l], }$$
%and let 
%$D_{\tau} = \{ k: [n] \rightarrow [0,1]: (k(n) =
%1) \} \subseteq [0,1]^{n+1}$.
%Then there exists a continuous map $$D_{\tau} \rightarrow X_{I}$$
%$$ k \mapsto \tau[[k]].$$

%\item[$(2)$] Let $\tau$ be as in $(1)$, and let
%$k,k' \in D_{\tau}$ be such that 
%$$(\exists i > m) [k(i) = 1 \wedge (\forall j \leq i) [k(j) = k'(j)] ].$$
%Then $\tau[[k]] = \tau[[k']] \in X_I$.

%\item[$(3)$] Let $\tau$ be as in $(1)$, let $k \in D_{\tau}$, and suppose
%that $k(i) = 0$ for some $0 \leq i < n$. Let $\tau' = d_i \tau$, and let
%$k' \in D_{\tau'}$ be obtained from $k$ by omitting the $i$th value. Then
%$$ \tau[[k]] = \tau'[[k']] \in X_I.$$

%\item[$(4)$] The topological space $X_I$ is the quotient
%of the disjoint union $\coprod_{\tau} D_{\tau}$ obtained by imposing
%the relations indicated in $(2)$ and $(3)$.
%\end{itemize}
%\end{definition}

%The argument of Proposition \ref{wiretrack} shows that there is a canonical homotopy equivalence
%$\theta: X_{I} \rightarrow | G_{Z}(I) |_{\calQ^{\bigdot} }$. We define the map $\beta'_I$ to be
%the composition of $\theta$ with a map $\beta''_I: H(I) \rightarrow X_I$, which is uniquely
%determined by the requirement that if $\sigma$ is as Definition \ref{defHt} and
%$k \in C_{\sigma}$, then
%$$ \beta_{I}''(\sigma[[k]]) = \tau[[k']] \in X_I, $$
%where $\tau = \sigma | \Delta^{ \{ m+1, \ldots, n\} }$ and
%$k'$ is obtained by restricting $k$ to the range $\{m+1, \ldots, n\}$. 
%It is not difficult to check that $\beta''_I$ is well-defined and
%functorial in $I$. To show that it is a homotopy equivalence, 
%we observe that the restriction $\beta''_I | H'(I)$ is a homeomorphism and apply
%Lemma \ref{apul1}.

