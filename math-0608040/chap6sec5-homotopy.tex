 
\section{Homotopy Theory in an $\infty$-Topos}\label{chap6sec5}

\setcounter{theorem}{0}

In classical homotopy theory, the most important invariants of a (pointed) space
$X$ are its homotopy groups $\pi_{i}(X,x)$. Our first objective in this section is to define analogous invariants in the case where $X$ is an object of an arbitrary $\infty$-topos $\calX$. In this setting, the homotopy groups are not ordinary groups but are instead {\em sheaves} of groups on the underlying topos $\Disc(\calX)$. In \S \ref{homotopysheaves}, we will study these homotopy groups
and the closely related theory of {\it $n$-connectivity}. The main theme is that the internal homotopy theory of a general $\infty$-topos $\calX$ behaves much like the classical case $\calX = \SSet$.

One important classical fact which does {\em not} hold in general for an $\infty$-topos is Whitehead's theorem. If $f: X \rightarrow Y$ is a map of CW-complexes, then $f$ is a homotopy equivalence if and only if $f$ induces bijective maps $\pi_{i}(X,x) \rightarrow \pi_i(Y, f(x))$ for
any $i \geq 0$ and any base point $x \in X$. If $f: X \rightarrow Y$ is a map in an arbitrary $\infty$-topos $\calX$ satisfying an analogous condition on (sheaves of) homotopy groups, then we say that
$f$ is {\it $\infty$-connective}. We will say that an $\infty$-topos $\calX$ is {\it hypercomplete}
if every $\infty$-connective morphism in $\calX$ is an equivalence. Whitehead's theorem may be interpreted as saying that the $\infty$-topos $\SSet$ is hypercomplete. An arbitrary $\infty$-topos $\calX$ need not be hypercomplete. We will survey the situation in \S \ref{hyperstacks}, where we also give some reformulations of the notion of hypercompleteness and show that every topos $\calX$ has a {\it hypercompletion} $\calX^{\hyp}$. In \S \ref{hcovh}, we will show that an $\infty$-topos $\calX$ is hypercomplete if and only if $\calX$ satisfies a descent condition with respect to hypercoverings (other versions of this result can be found in \cite{hollander} and \cite{toen}).

\begin{remark}
The Brown-Joyal-Jardine theory of simplicial (pre)sheaves on a topological space $X$ is a model for the hypercomplete $\infty$-topos $\Shv(X)^{\hyp}$. In many respects, the $\infty$-topos $\Shv(X)$ of sheaves of spaces on $X$ is better behaved {\em before} hypercompletion. We will outline some of the advantages of $\Shv(X)$ in \S \ref{versus} and in \S \ref{chap7}.
\end{remark}

\subsection{Homotopy Groups}\label{homotopysheaves}

Let $\calX$ be an $\infty$-topos, and let $X$ be an object of
$\calX$. We will refer to a discrete object of $\calX_{/X}$ as a {\it sheaf of sets on $X$}.
Since $\calX$ is presentable, it is automatically {\em cotensored} over
spaces, as explained in Remark \ref{coten}. Consequently, for any object $X$ of
$\calX$ and any simplicial set $K$, there exists an object $X^K$ of $\calX$ equipped with natural isomorphisms
$$ \bHom_{\calX}( Y, X^K ) \rightarrow \bHom_{\calH}( K, \bHom_{\calX}(Y,X) )$$
in the homotopy category $\calH$ of spaces.

\begin{definition}\index{not}{pinX@$\pi_n(X)$}\index{gen}{homotopy groups!in an $\infty$-topos}
Let $S^n  = \bd \Delta^{n+1} \in \calH$ denote the (simplicial) $n$-sphere, and fix a base point $\ast \in S^n$. Then evaluation at $\ast$ induces a morphism $s: X^{S^n} \rightarrow X$ in $\calX$.
We may regard $s$ as an object of $\calX_{/X}$, and we define $\pi_n(X) = \tau_{\leq 0} s \in \calX_{/X}$ to be the associated discrete object of $\calX_{/X}$.
\end{definition}

 We will generally identify $\pi_n(X)$ with its image in the underlying topos $\Disc(\calX_{/X})$ (where it is well-defined up to canonical isomorphism). The constant map $S^n \rightarrow \ast$ induces
a map $X \rightarrow X^{S^n}$, which determines a base point of $\pi_n(X)$. 

Suppose that $K$ and $K'$ are pointed simplicial sets, and let $K \vee K'$ denote the coproduct $K \amalg_{\ast} K'$. There is a pullback diagram
$$ \xymatrix{ & X^{K \vee K'} \ar[dr] \ar[dl] & \\
X^K \ar[dr] & & X^{K'} \ar[dl] \\
& X & }$$
in $\calX$, so that $X^{K \vee K'}$ may be identified with a product
of $X^K$ and $X^{K'}$ in the $\infty$-topos $\calX_{/X}$. We now make the following
general observation:

\begin{lemma}\label{slurpy}
Let $\calX$ be an $\infty$-topos. The truncation functor $\tau_{\leq n}: \calX \rightarrow \calX$
preserves finite products.
\end{lemma}

\begin{proof}
We must show that for any finite collection of objects $\{ X_{\alpha} \}_{\alpha \in A}$
having product $X$, the induced map
$$ \tau_{\leq n} X \rightarrow \prod_{\alpha \in A} \tau_{\leq n} X_{\alpha} $$
is an equivalence. If $\calX$ is the $\infty$-category of spaces, then this follows from Whitehead's theorem: simply compute homotopy groups (and sets) on both sides. If $\calX = \calP(\calC)$, then to prove that a map in $\calX$ is an equivalence, it suffices to show that it remains an equivalence after evaluation at any object $C \in \calC$; thus we may reduce to the case where $\calX = \SSet$
considered above. In the general case, $\calX$ is equivalent to the essential image of a left-exact localization functor $L: \calP(\calC) \rightarrow \calP(\calC)$ for some small $\infty$-category $\calC$. Without loss of generality, we may identify $\calX$ with a full subcategory of $\calP(\calC)$. Then
$\calX \subseteq \calX' = \calP(\calC)$ is stable under limits, so that $X$ may be identified with
a product of the family $\{ X_{\alpha} \}_{\alpha \in A}$ in $\calX'$. It follows from the case treated above that the natural map
$$ \tau_{\leq n}^{\calX'} X \rightarrow \prod_{\alpha \in A} \tau_{\leq n}^{\calX'} X_{\alpha}$$
is an equivalence. But Proposition \ref{compattrunc} implies that $L \circ \tau_{\leq n}^{\calX'} | \calX$ is an $n$-truncation functor for $\calX$. The desired result now follows by applying the functor $L$ to both sides of the above equivalence, and invoking the assumption that $L$ is left-exact (here we must require the finiteness of $A$).
\end{proof}

It follows from Lemma \ref{slurpy} that there is a canonical isomorphism
$$\tau^{\calX_{/X}}_{\leq 0} (X^{K \vee K'}) \simeq \tau^{\calX_{/X}}_{\leq 0}(X^K) \times \tau^{\calX_{/X}}_{\leq 0}(X^{K'})$$ in the topos $\Disc(\calX_{/X})$. 
In particular, for $n > 0$, the usual comultiplication $S^n \rightarrow S^n \vee S^n$
(a well-defined map in the homotopy category $\calH$)
induces a multiplication map $\pi_n(X) \times \pi_n(X) \rightarrow \pi_n(X)$. As in ordinary homotopy theory, we conclude that $\pi_n(X)$ is a group object of $\Disc(\calX_{/X})$ for $n > 0$, which is commutative for $n > 1$.

In order to work effectively with homotopy sets, it is convenient
to define the homotopy sets $\pi_n(f)$ of a morphism $f: X
\rightarrow Y$ to be the homotopy sets of $f$ considered as an object
of the $\infty$-topos $\calX_{/Y}$. In view of the equivalences
$\calX_{/f} \rightarrow \calX_{/X}$, we may identify $\pi_n(f)$ with an object
of $\Disc(\calX_{/X})$, which is again a sheaf of groups if
$n \geq 1$, and abelian groups if $n\geq 2$.
The intuition is that the stalk of these sheaves at a point $p$ of $X$ is the $n$th homotopy group of the homotopy fiber of $f$, taken with respect to the base point $p$.

\begin{remark}\label{recgroup}
It is useful to have the following recursive definition for homotopy groups.
Let $f: X \rightarrow Y$ be a morphism in an $\infty$-topos $\calX$. 
Regarding $f$ as an object of the topos
$\calX_{/Y}$, we may take its $0$th truncation
$\tau_{\leq 0}^{\calX_{/Y}} f$. This is a discrete object of $\calX_{/Y}$, and by definition
we have $\pi_0(f) \simeq f^{\ast} \tau_{\leq 0}^{\calX_{/Y}}(X) \simeq
X \times_{Y} \tau_{\leq 0}^{\calX_{/Y}}(f)$. The natural map $X
\rightarrow \tau_{\leq 0}^{\calX_{/Y}}(f)$ gives a global section of
$\pi_0(f)$. Note that in this case, $\pi_0(f)$ is the pullback of
a discrete object of $\calX_{/Y}$: this is because the definition of $\pi_0$ does not
require a base point. 

If $n >0$, then we have a natural isomorphism $\pi_{n}(f) \simeq
\pi_{n-1}(\delta)$ in $\Disc(\calX_{/X})$, where $\delta: X \rightarrow X \times_{Y} X$ is the associated diagonal map. \end{remark}

\begin{remark}\label{emmy}
Let $f: \calX \rightarrow \calY$ be a geometric morphism of
$\infty$-topoi, and let $g: Y \rightarrow Y'$ be a morphism in
$\calY$. Then there is a canonical isomorphism $f^{\ast}( \pi_n(g) ) \simeq \pi_n(f^{\ast}(g))$
in $\Disc(\calX_{/f^{\ast} Y})$. This follows immediately from Proposition \ref{compattrunc}.
\end{remark}

\begin{remark}\label{sequence}\index{gen}{long exact sequence of homotopy groups}
Given a pair of composable morphisms $X \stackrel{f}{\rightarrow}
Y \stackrel{g}{\rightarrow} Z$, there is an associated sequence of pointed objects
$$\ldots \rightarrow f^{\ast} \pi_{n+1}(g) \stackrel{\delta_{n+1}}{\rightarrow} \pi_n(f) \rightarrow \pi_n(g \circ f) \rightarrow f^{\ast}
\pi_n(g) \stackrel{\delta_{n}}{\rightarrow} \pi_{n-1}(f) \rightarrow \ldots$$
in the ordinary topos $\Disc(\calX_{/X})$, with the
usual exactness properties. To construct the boundary map $\delta_n$, we observe
that the $n$-sphere $S^n$ can be written as a (homotopy) pushout
$D^{-} \amalg_{ S^{n-1} } D^{+}$ of two hemispheres along the equator. By construction,
$f^{\ast} \pi_n(g)$ can be identified with the $0$-truncation of
$$X \times_{Y} Y^{S^n} \times_{ Z^{S^n} } Z \simeq
X^{D^{-}} \times_{ Y^{D^{-}} } Y^{S^n} \times_{ Z^{S^{n}} } Z,$$
which maps by restriction to
$$ X^{S^{n-1}} \times_{ Y^{S^{n-1}} } Y^{D^{+}} \simeq X^{ S^{n-1} } \times_{Y^{S^{n-1}} } Y.$$
We now observe that the $0$-truncation of the latter object is naturally isomorphic to
$\pi_{n-1}(f) \in \Disc(\calX_{/X})$.

To prove the exactness of the above sequence in an $\infty$-topos $\calX$, we
first choose an accessible left exact localization $L: \calP(\calC) \rightarrow \calX$.
Without loss of generality, we may suppose that the diagram
$X \stackrel{f}{\rightarrow} Y \stackrel{g}{\rightarrow} Z$ is the image under $L$ of
a diagram in $\calP(\calC)$. Using Remark \ref{emmy}, we conclude that the sequence constructed above is equivalent to the image under $L$ of an analogous sequence in the $\infty$-topos
$\calP(\calC)$. Since $L$ is left exact, it will suffice to prove that this second sequence is exact; in other words, we may reduce to the case $\calX = \calP(\calC)$. Working componentwise, we
can reduce further to the case where $\calX = \SSet$. The desired result now follows from classical homotopy theory. (Special care should be taken regarding the exactness of the above sequence
at $\pi_0(f)$: really this should be interpreted in terms of an action of the group $f^{\ast} \pi_1(g)$ on
$\pi_0(f)$. We leave the details of the construction of this action to the reader. )
\end{remark}

\begin{remark}
If $\calX = \SSet$, and $\eta: \ast \rightarrow X$ is a pointed
space, then $\eta^{\ast} \pi_{n}(X)$ can be identified with the $n$th homotopy group
of $X$ with base point $\eta$.
\end{remark}

We now study the implications of the vanishing of homotopy groups.

\begin{proposition}\label{ditz}\index{gen}{truncated!and homotopy groups}
Let $f: X \rightarrow Y$ be an $n$-truncated morphism in an $\infty$-topos $\calX$. Then
$\pi_{k}(f) \simeq \ast$ for all $k > n$. If $n \geq 0$ and $\pi_{n}(f)
\simeq \ast$, then $f$ is $(n-1)$-truncated.
\end{proposition}

\begin{proof}
The proof goes by induction on $n$. If $n = -2$, then $f$ is an
equivalence and there is nothing to prove. Otherwise, the diagonal map
$\delta: X \rightarrow X \times_Y X$ is $(n-1)$-truncated (Lemma \ref{trunc}). The
inductive hypothesis and Remark \ref{recgroup}
allow us to deduce that $\pi_{k}(f) \simeq \pi_{k-1}(\delta) \simeq \ast$ whenever $k > n$ and $k > 0$. Similarly, if $n \geq 1$ and $\pi_{n}(f) \simeq \pi_{n-1}(\delta) \simeq \ast$, then $\delta$ is
$(n-2)$-truncated by the inductive hypothesis, so that $f$ is
$(n-1)$-truncated (Lemma \ref{trunc}). 

The case of small $k$ and $n$ requires special attention: we must
show that if $f$ is $0$-truncated, then $f$ is $(-1)$-truncated if
and only if $\pi_0(f) \simeq \ast$. Because $f$ is $0$-truncated, we have
an equivalence $\tau_{\leq 0}^{\calX_{/Y}}(f) \simeq f$, so that $\pi_0(f) \simeq X
\times_Y X$. To say $\pi_0(f) \simeq \ast$ is to assert that the diagonal map
$\delta: X \rightarrow X \times_{Y} X$ is an equivalence, which is equivalent
to the assertion that $f$ is $(-1)$-truncated (Lemma \ref{trunc}). 
\end{proof}

\begin{remark}
The Proposition \ref{ditz} implies that if $f$ is $n$-truncated
for {\em some} $n \gg 0$, then we can test whether or not $f$ is
$m$-truncated for any particular value of $m$ by computing the
homotopy groups of $f$. In contrast to the classical situation, it
is not possible to drop the assumption that $f$ is
$n$-truncated for $n \gg 0$. 
\end{remark}

\begin{lemma}\label{truncatepin}
Let $X$ be an object in an $\infty$-topos $\calX$, and let
$p: X \rightarrow Y$ be an $n$-truncation of $X$. Then $p$ induces isomorphisms
$\pi_{k}(X) \simeq p^{\ast} \pi_{k}(Y)$ for all $k
\leq n$.
\end{lemma}

\begin{proof}
Let $\phi: \calX \rightarrow \calY$ be a geometric morphism such
that $\phi_{\ast}$ is fully faithful. By Proposition
\ref{compattrunc} and Remark \ref{emmy}, it will suffice to prove
the lemma in the case where $\calX = \calY$. We may therefore assume that $\calY$ is an $\infty$-category of presheaves. In this case, homotopy groups and truncations are
computed pointwise. Thus we may reduce to the case $\calX =
\SSet$, where the conclusion follows from classical homotopy theory. 
\end{proof}

\begin{definition}\label{stooog}\index{gen}{connective!$n$-connective object}\index{gen}{connective!$n$-connective morphism}
Let $f: X \rightarrow Y$ be a morphism in an $\infty$-topos
$\calX$, and let $0 \leq n \leq \infty$. We will say that $f$ is
{\it $n$-connective} if it is an effective epimorphism and $\pi_k(f) = \ast$ for
$0 \leq k < n$. We shall say that the object $X$ is {\it
$n$-connective} if $f: X \rightarrow 1_{\calX}$ is $n$-connective, where $1_{\calX}$ denotes the final object of $\calX$. By convention, we will say that every morphism $f$ in $\calX$ is
$(-1)$-connective.
\end{definition}

\begin{definition}\index{gen}{connected!object of an $\infty$-topos}
Let $X$ be an object of an $\infty$-topos $\calX$. We will say that $X$ is {\it connected} if
it is $1$-connective: that is, if the truncation $\tau_{\leq 0} X$ is a final object in $\calX$.
\end{definition}

%\begin{warning}
%Our terminology does {\em not} coincide with the classical homotopy-theoretic terminology in the case where $\calX$ is the $\infty$-topos of spaces. A map $f: X \rightarrow Y$ of spaces is
%usually said to be {$n$-connected} if the mapping cone (that is, the homotopy pushout
%$Y \amalg_{X} \ast$) is $n$-connected. Definition \ref{stooog} places this condition instead on the mapping fibers. The notions differ only by a change of indexing: for $n \geq 0$, $f$ is $n$-connected in the classical sense if and only if it is $(n-1)$-connected in the sense of Definition \ref{stooog}.
%\end{warning}

\begin{proposition}\label{goober}
Let $X$ be an object in an $\infty$-topos $\calX$ and let $n \geq -1$. Then $X$ is
$n$-connective if and only if $\tau_{\leq n-1} X$ is a final object of $\calX$.
\end{proposition}

\begin{proof}
The case $n=-1$ is trivial. The proof in general goes by induction on $n \geq 0$. If $n=0$, then the
conclusion follows from Proposition \ref{slurpme}. 
Suppose $n > 0$. Let $p: X \rightarrow \tau_{n-1} X$ be an $(n-1)$-truncation of
$X$. If $\tau_{\leq n-1} X$ is a final object of $\calX$, then 
$$\pi_{k} X \simeq p^{\ast} \pi_k(\tau_{n-1} X) \simeq \ast$$ 
for $k < n$ by Lemma \ref{truncatepin}. Since 
the map $p: X \rightarrow \tau_{\leq n-1} X \simeq 1_{\calX}$ is an effective epimorphism
(Proposition \ref{pi00detects}), it follows that $X$ is $n$-connective.

Conversely, suppose that $X$ is $n$-connective. Then $p^{\ast}
\pi_{n-1}(\tau_{\leq n-1} X) \simeq \ast$. Since $p$ is an effective epimorphism, Lemma \ref{hint0} implies that
$\pi_{n-1}(\tau_{\leq n-1} X) = \ast$. Using Proposition \ref{ditz}, we conclude that $\tau_{\leq n-1} X$ is
$(n-2)$-truncated, so that $\tau_{\leq n-1} X \simeq \tau_{\leq n-2} X$.
Repeating this argument, we reduce to the case where $n=0$ which was handled above.
\end{proof}

\begin{corollary}\label{togoto}
The class of $n$-connective objects of an $\infty$-topos $\calX$ is stable under finite products.
\end{corollary}

\begin{proof}
Combine Proposition \ref{goober} with Lemma \ref{slurpy}. 
\end{proof}

Let $\calX$ be an $\infty$-topos and $X$ an object of $\calX$. 
Since $\bHom_{\calX}(X,Y) \simeq \bHom_{\calX}(\tau_{\leq n} X, Y)$ whenever
$Y$ is $n$-truncated, we deduce that $X$ is $(n+1)$-connective if and
only if the natural map $\bHom_{\calX}(1_{\calX},Y) \rightarrow \bHom_{\calX}(X,Y)$ is an
equivalence for all $n$-truncated $Y$. From this, we can
immediately deduce the following relative version of Proposition
\ref{goober}:

\begin{corollary}\label{goober2}
Let $f: X \rightarrow X'$ be a morphism in an $\infty$-topos
$\calX$. Then $f$ is $(n+1)$-connective if and only if composition with $f$ induces a homotopy equivalence $$\bHom_{\calX_{/X'}}(\id_{X'},Y)
\rightarrow \bHom_{\calX_{/X'}}(f,Y)$$ 
for every $n$-truncated object $Y \in \calX_{/X'}$.
\end{corollary}

\begin{remark}\label{nconn}
Let $L: \calX \rightarrow \calY$ be a left exact localization of $\infty$-topoi, and let
$f: Y \rightarrow Y'$ be an $n$-connective morphism in $\calY$. Then $f$ is equivalent (in $\Fun(\Delta^1,\calY)$) to $Lf_0$, where $f_0$ is an $n$-connective morphism in $\calX$.
To see this, we choose a (fully faithful) right adjoint $G$ to $L$, and a factorization
$$ \xymatrix{ & X \ar[dr]^{f''} & \\
G(Y) \ar[ur]^{f'} \ar[rr]^{ G(f_0)} & & G(Y') }$$
where $f'$ is $n$-connective and $f''$ is $(n-1)$-truncated. Then
$Lf'' \circ Lf'$ is equivalent to $f$, and is therefore $n$-connective. It follows that $Lf''$ is
an equivalence, so that $Lf'$ is equivalent to $f$.
\end{remark}

We conclude by noting the following stability properties of the
class of $n$-connective morphisms:

\begin{proposition}\label{inftychange}
Let $\calX$ be an $\infty$-topos.
\begin{itemize}
\item[$(1)$] Let $f: X \rightarrow Y$ be a morphism in $\calX$. If $f$
is $n$-connective, then it is $m$-connective for all $m \leq n$. Conversely, if
$f$ is $n$-connective for all $n < \infty$, then $f$ is $\infty$-connective.

\item[$(2)$] Any equivalence in $\calX$ is $\infty$-connective.

\item[$(3)$] Let $f,g: X \rightarrow Y$ be a homotopic morphisms in $\calX$. Then
$f$ is $n$-connective if and only if $g$ is $n$-connective.

\item[$(4)$] Let $\pi^{\ast}: \calX \rightarrow \calY$ be left adjoint to a geometric morphism
from $\pi_{\ast}: \calY \rightarrow \calX$, and let $f: X \rightarrow X'$ be an $n$-connective
morphism in $\calX$. Then $\pi^{\ast} f$ is an $n$-connective morphism in $\calY$.

\item[$(5)$] Suppose given a diagram
$$ \xymatrix{ & Y \ar[dr]^{g} & \\
X \ar[ur]^{f} \ar[rr]^{h} & & Z }$$
in $\calX$, where $f$ is $n$-connective. Then $g$ is $n$-connective if and only if
$h$ is $n$-connective. 

\item[$(6)$] Suppose given a pullback diagram
$$ \xymatrix{ X' \ar[d]^{f'} \ar[r]^{q'} & X \ar[d]^{f} \\
Y' \ar[r]^{q} & Y }$$
in $\calX$. If $f$ is $n$-connective, then so is $f'$. The converse holds if $q$ is an effective epimorphism.
\end{itemize}
\end{proposition}

\begin{proof}
The first three assertions are obvious. Claim $(4)$ follows from Propositions \ref{goober} and \ref{compattrunc}. To prove $(5)$, we first observe that Corollary \ref{composite} implies that
$g$ is an effective epimorphism if and only if $h$ is an effective epimorphism. According to
Remark \ref{sequence}, we have a long exact sequence
$$\ldots \rightarrow f^{\ast} \pi_{i+1}(g) {\rightarrow} \pi_i(f) \rightarrow \pi_i(h) \rightarrow f^{\ast}
\pi_i(g) \rightarrow \pi_{i-1}(f) \rightarrow \ldots$$ 
of pointed objects in the topos $\Disc(\calX_{/X})$. It is then clear that if $f$ and $g$
are $n$-connective, then so is $h$. Conversely, if $f$ and $h$ are $n$-connective, then
$f^{\ast} \pi_i(g) \simeq \ast$ for $i \leq n$. Since $f$ is an effective epimorphism, Lemma \ref{hint0} implies that $\pi_i(g) \simeq \ast$ for $i \leq n$, so that $g$ is also $n$-connective.

The first assertion of $(6)$ follows from $(4)$, since a pullback functor
$q^{\ast}: \calX_{/Y} \rightarrow \calX_{/Y'}$ is left adjoint to a geometric morphism.
For the converse, let us suppose that $q$ is an effective epimorphism and that $f'$ is $n$-connective. According to Lemma \ref{hintdescent1}, the maps $f$ and $q'$ are effective epimorphisms. Applying Remark \ref{emmy}, we conclude that there are canonical isomorphisms
${q'}^{\ast} \pi_k(f) \simeq \pi_k(f')$ in the topos $\Disc( \calX_{/X'})$, so that
${q'}^{\ast} \pi_k(f) \simeq \ast$ for $k < n$. Applying Lemma \ref{hint0}, we conclude
that $\pi_k(f) \simeq \ast$ for  $k < n$, so that $f$ is $n$-connective as desired.
\end{proof}

\begin{corollary}\label{pusherr}
Let 
$$\xymatrix{ X' \ar[r]^{g} \ar[d]^{f'} & X \ar[d]^{f} \\
Y' \ar[r] & Y }$$
be a pushout diagram in an $\infty$-topos $\calX$. Suppose that $f'$ is
$n$-connective. Then $f$ is $n$-connective.
\end{corollary}

\begin{proof}
Choose an accessible, left exact localization functor $L: \calP(\calC) \rightarrow \calX$.
Using Remark \ref{nconn}, we can assume without loss of generality that
$f' = Lf'_0$, where $f'_0: X'_0 \rightarrow Y'_0$ is a morphism in $\calP(\calC)$.
Similarly, we may assume $g = Lg_0$, for some morphism $g_0: X'_0 \rightarrow X_0$.
Form a pushout diagram
$$\xymatrix{ X'_0 \ar[r]^{g_0} \ar[d]^{f'_0} & X_0 \ar[d]^{f_0} \\
Y'_0 \ar[r] & Y_0 }$$
in $\calP(\calC)$. Then the original diagram is equivalent to the image (under $L$) of the diagram above. In view of Proposition \ref{inftychange}, it will suffice to show that $f_0$ is $n$-connective.
Using Propositions \ref{goober} and \ref{compattrunc}, we see that $f_0$ is $n$-connective if and only if its image under the evaluation map $\calP(\calC) \rightarrow \SSet$ associated to any object $C \in \calC$ is $n$-connective. In other words, we can reduce to the case where $\calX = \SSet$, and the result now follows from classical homotopy theory.
\end{proof}

We conclude by establishing a few results which will be needed in \S \ref{dimension}:

\begin{proposition}\label{trowler}
Let $f: X \rightarrow Y$ be a morphism in an $\infty$-topos $\calX$, 
$\delta: X \rightarrow X \times_{Y} X$ the associated diagonal morphism, and $n \geq 0$.
The following conditions are equivalent:
\begin{itemize}
\item[$(1)$] The morphism $f$ is $n$-connective.

\item[$(2)$] The diagonal map $\delta: X \rightarrow X \times_{Y} X$ is $(n-1)$-connective, and
$f$ is an effective epimorphism.
\end{itemize}
\end{proposition}

\begin{proof}
Immediate from Definition \ref{stooog} and Remark \ref{recgroup}.
\end{proof}

\begin{proposition}\label{conslice}
Let $\calX$ be an $\infty$-topos containing an object $X$, and let $\sigma: \Delta^2 \rightarrow \calX$ be a $2$-simplex corresponding to a diagram
$$ \xymatrix{ Y \ar[dr] \ar[rr]^{f} & & Z \ar[dl]^{g} \\
& X. & }$$
Then $f$ is an $n$-connective morphism in $\calX$ if and only if $\sigma$ is an
$n$-connective morphism in $\calX_{/X}$.
\end{proposition}

\begin{proof}
We observe that $\calX_{/g} \rightarrow \calX_{/Z}$ is a
trivial fibration, so that an object of $\calX_{/g}$ is $n$-connective if and only if its
image in $\calX_{/Z}$ is $n$-connective. 
\end{proof}

\begin{proposition}\label{sectcon}
Let $f: X \rightarrow Y$ be a morphism in an $\infty$-topos $\calX$, let
$s:Y \rightarrow X$ be a section of $f$ (so that $f \circ s$ is homotopic to $\id_{Y}$), and
let $n \geq 0$. Then $f$ is $n$-connective if and only if $s$ is $(n-1)$-connective.
\end{proposition}

\begin{proof}
We have a $2$-simplex $\sigma: \Delta^2 \rightarrow \calX$ which we may depict as follows:
$$ \xymatrix{ & X \ar[dr]^{f} & \\
Y \ar[ur]^{s} \ar[rr]^{\id_Y} & & Y. }$$
Corollary \ref{composite} implies that $f$ is an effective epimorphism; this completes the proof in the case $n = 0$. Suppose that $n > 0$, and that $s$ is $(n-1)$-connective.
In particular, $s$ is an effective epimorphism. The long exact sequence of Remark \ref{sequence} gives an isomorphism $\pi_{i}(s) \simeq s^{\ast} \pi_{i+1}(f)$, so that $s^{\ast} \pi_k(f)$ vanishes for
$1 \leq k < n$. Applying Lemma \ref{hint0}, we conclude that $\pi_k(f) \simeq \ast$ for $1 \leq k < n$. Moreover, since $s$ is an effective epimorphism it induces an effective epimorphism
$\pi_0( \id_{Y} ) \rightarrow \pi_0(f)$ in the ordinary topos $\Disc( \calX_{/Y} )$, so that
$\pi_0(f) \simeq \ast$ as well. This proves that $f$ is $n$-connective.

Conversely, if $f$ is $n$-connective, then $\pi_i(s) \simeq \ast$ for
$i < n-1$; the only nontrivial point is to verify that $s$ is an effective epimorphism. According to Proposition \ref{conslice}, it will suffice to prove that $\sigma$ is an effective epimorphism when viewed as a morphism in $\calX_{/Y}$. Using Proposition \ref{pi00detects}, we may reduce to proving that $\sigma' = \tau_{\leq 0}^{\calX_{/Y}}(\sigma)$ is an equivalence in $\calX_{/Y}$.
This is clear, since the source and target of $\sigma'$ are both final objects of $\calX_{/Y}$ (in virtue of our assumption that $f$ is $1$-connective).
\end{proof}

\subsection{$\infty$-Connectedness}\label{hyperstacks}

Let $\calC$ be an ordinary category equipped with a Grothendieck topology, and let
$\bfA = \Set_{\Delta}^{\calC^{op}}$ be the category of simplicial presheaves on $\calC$.

\begin{proposition}[Jardine \cite{jardine}]\label{jardinesardine}\index{gen}{model category!local}
There exists a left proper, combinatorial, simplicial model structure on the category $\bfA$, which admits the following description:

\begin{itemize}
\item[$(C)$] A map $f: F_{\bigdot} \rightarrow G_{\bigdot}$ of simplicial presheaves on $\calC$
is a {\it local cofibration} if it is a injective cofibration: that is, if and only if the induced map
$F_{\bigdot}(C) \rightarrow G_{\bigdot}(C)$ is a cofibration of simplicial sets for each object $C \in \calC$.

\item[$(W)$] A map $f: F_{\bigdot} \rightarrow G_{\bigdot}$ of simplicial presheaves on $\calC$
is a {\it local equivalence} if and only if, for any object $C \in \calC$ and any commutative diagram of topological spaces
$$ \xymatrix{ S^{n-1} \ar[r] \ar@{^{(}->}[d] & |F_{\bigdot}(C)| \ar[d] \\
D^n \ar[r] & |G_{\bigdot}(C)|, } $$
there exists a collection of morphisms $\{ C_{\alpha} \rightarrow C\}$ which generate a covering
sieve on $C$, such that in each of the induced diagrams
$$ \xymatrix{ S^{n-1} \ar[r] \ar@{^{(}->}[d] & |F_{\bigdot}(C_{\alpha})| \ar[d] \\
D^n \ar[r] \ar@{-->}[ur] & |G_{\bigdot}(C_{\alpha})|, } $$
one can produce a dotted arrow so that the upper triangle commutes and the lower triangle
commutes up to a homotopy which is fixed on $S^{n-1}$.
\end{itemize}
\end{proposition}

We refer the reader to \cite{jardine} for a proof (one can also deduce Proposition \ref{jardinesardine} from Proposition \ref{goot}). We will refer to the model structure of Proposition \ref{jardinesardine} as the {\it local} model structure on $\bfA$.

\begin{remark}\label{pointeddesc}
In the case where the topos $\toposX$ of sheaves of sets on $\calC$ has enough points, there is a simpler description of the class $(W)$ of local equivalences: a map $F \rightarrow G$ of simplicial presheaves is a local equivalence if and only if it induces weak homotopy equivalences 
$F_{x} \rightarrow G_{x}$ of simplicial sets, after passing to the stalk at any point $x$ of $\toposX$.
We refer the reader to \cite{jardine} for details.
\end{remark}

Let $\bfA^{\degree}$ denote the full subcategory of $\bfA$ consisting of fibrant-cofibrant objects (with respect to the local model structure), and let $\calX = \sNerve(\bfA^{\degree})$ be the associated $\infty$-category. We observe that the local model structure on $\bfA$ is a localization of the injective model structure on $\bfA$. Consequently, the $\infty$-category $\calX$ is a localization of the $\infty$-category associated to the injective model structure on $\bfA$, which (in view of 
Proposition \ref{othermod}) is equivalent to $\calP( \Nerve(\calC))$. It is tempting to guess
that $\calX$ is equivalent to the left exact localization $\Shv( \Nerve(\calC))$ constructed in \S \ref{cough}. This is not true in general; however, as we will explain below, we can always recover $\calX$ as an accessible left-exact localization of $\Shv( \Nerve(\calC))$. In particular, $\calX$
is itself an $\infty$-topos.

In general, the difference between $\calX$ and $\Shv( \Nerve(\calC))$ is measured by the failure of Whitehead's theorem. Essentially by construction, the equivalences in $\bfA$ are those maps which induce isomorphisms on homotopy sheaves. In general, this assumption is not strong enough to guarantee that a morphism in $\Shv( \Nerve(\calC) )$ is an equivalence. However, this is the only difference: the $\infty$-category $\calX$ can be obtained from
$\Shv(\Nerve(\calC))$ by inverting the class of $\infty$-connective morphisms (Proposition \ref{suga}). Before proving this, we study the class of $\infty$-connective morphisms in an arbitrary
$\infty$-topos.

\begin{lemma}\label{swarp}
Let $p: \calC \rightarrow \calD$ be a Cartesian fibration of $\infty$-categories, let
$\calC'$ be a full subcategory of $\calC$, and suppose that for every $p$-Cartesian morphism
$f: C \rightarrow C'$ in $\calC$, if $C' \in \calC'$, then $C \in \calC'$. Let
$D$ be an object of $\calD$, and let $f: C \rightarrow C'$ be a morphism in the fiber
$\calC_{D} = \calC \times_{\calD} \{D\}$ which exhibits $C'$ as a $\calC^{0}_{D}$-localization
of $C$ (see Definition \ref{locaobj}). Then $f$ exhibits $C'$ as a $\calC$-localization of $\calC$.
\end{lemma}

\begin{proof}
According to Proposition \ref{verylonger}, $p$ induces a Cartesian fibration
$\calC_{C/} \rightarrow \calD_{D/}$, which restricts to give a Cartesian fibration
$p': \calC'_{C/} \rightarrow \calD_{D/}$. We observe that $f$ is an object of
$\calC'_{C/}$ which is an initial object of $(p')^{-1} \{ \id_{D} \}$ (Remark \ref{initrem}), 
and that $\id_{D}$ is an initial object of $\calD_{D/}$. Lemma \ref{sabreto} implies that $f$ is an initial object of $\calC'_{C/}$, so that $f$ exhibits $C'$ as $\calC'$-localization of $C$ (Remark \ref{initrem}) as desired.
\end{proof}

\begin{lemma}\label{swarp2}
Let $p: \calC \rightarrow \calD$ be a Cartesian fibration of $\infty$-categories, let
$\calC'$ be a full subcategory of $\calC$, and suppose that for every $p$-Cartesian morphism
$f: C \rightarrow C'$ in $\calC$, if $C' \in \calC'$, then $C \in \calC'$. Suppose that for each
object $D \in \calD$, the fiber $\calC'_{D} = \calC' \times_{\calD} \{D\}$ is a reflective subcategory
of $\calC_{D} = \calC \times_{\calD} \{D\}$ (see Remark \ref{reflective}). Then $\calC'$ is a reflective subcategory of $\calC$.
\end{lemma}

\begin{proof}
Combine Lemma \ref{swarp} with Proposition \ref{testreflect}.
\end{proof}

\begin{lemma}\label{swarp3}
Let $\calX$ be a presentable $\infty$-category, let $\calC$ be an accessible $\infty$-category,
and let $\alpha: F \rightarrow G$ be a natural transformation between accessible functors
$F,G: \calC \rightarrow \calX$. Let $\calC(n)$ be the full subcategory of $\calC$ spanned by those objects $C$ such that $\alpha(C): F(C) \rightarrow G(C)$ is $n$-truncated. Then $\calC(n)$ is an accessible subcategory of $\calC$ (see Definition \ref{defaccsub}).
\end{lemma}

\begin{proof}
We work by induction on $n$. If $n=-2$, then we have a (homotopy) pullback diagram
$$ \xymatrix{ \calC(n) \ar[r] \ar[d] & \calC \ar[d]^{\alpha} \\
\calE \ar[r] & \Fun(\Delta^1,\calX) }$$
where $\calE$ is the full subcategory of $\Fun(\Delta^1,\calX)$ spanned by
equivalences. The inclusion of $\calE$ into $\Fun(\Delta^1,\calX)$ is equivalent
to the diagonal map $\calX \rightarrow \Fun(\Delta^1, \calX)$, and therefore accessible.
Proposition \ref{horse2} implies that $\calC(n)$ is an accessible subcategory of $\calC$, as desired.

If $n \geq -1$, we apply the the inductive hypothesis to the diagonal functor
$\delta: F \rightarrow F \times_{G} F$, using Lemma \ref{trunc}.
\end{proof}

\begin{lemma}\label{tur}
Let $\calX$ be a presentable $\infty$-category, and let $-2 \leq n < \infty$. Let
$\calC$ be the full subcategory of $\Fun(\Delta^1,\calX)$ spanned by the $n$-truncated
morphisms. Then $\calC$ is a strongly reflective subcategory of $\Fun(\Delta^1,\calX)$.
\end{lemma}

\begin{proof}
Applying Lemma \ref{swarp2} to the projection $\Fun(\Delta^1,\calX) \rightarrow \Fun( \{1\}, \calX)$, we conclude that $\calC$ is a reflective subcategory of $\Fun(\Delta^1,\calX)$. The accessibility
of $\calC$ follows from Lemma \ref{swarp3}.
\end{proof}

\begin{lemma}\label{swarp4}
Let $\calX$ be an $\infty$-topos, let $0 \leq n \leq \infty$, and let $\calD(n)$ be the full subcategory of $\Fun(\Delta^1,\calX)$ spanned by the $n$-connective morphisms of $\calX$. Then
$\calD(n)$ is an accessible subcategory of $\calX$ and is stable under colimits in 
$\calX$.
\end{lemma}

\begin{proof}
Suppose first that $n < \infty$. Let $\calC(n) \subseteq \Fun(\Delta^1,\calX)$ be the full subcategory spanned by the $n$-truncated morphisms in $\calX$. According to Lemma \ref{tur}, the
inclusion $\calC(n) \subseteq \Fun(\Delta^1,\calX)$ has a left adjoint $L: \Fun(\Delta^1,\calX) \rightarrow \calC(n)$. Moreover, the proof of Lemma \ref{swarp} shows that $f$ is $n$-connective if and only if $Lf$ is an equivalence. It is easy to see that the full subcategory $\calE \subseteq \calC(n)$ spanned by the equivalences is stable under colimits in $\calC(n)$, so that $\calD(n)$ is stable under colimits in $\Fun(\Delta^1,\calX)$. The accessibility of $\calD(n)$ follows from the existence of the (homotopy) pullback diagram
$$ \xymatrix{ \calD(n) \ar[r] \ar[d] & \Fun(\Delta^1,\calX) \ar[d]^{L} \\
\calE \ar[r] & \calC(n) }$$
and Proposition \ref{horse2}.

If $n=\infty$, we observe that $\calD(n) = \cup_{m < \infty} \calD(m)$, which is manifestly stable under colimits, and is an accessible subcategory of $\calX^{\Delta^1}$ by Proposition \ref{boundint}.
\end{proof}

\begin{proposition}\label{goober3}
Let $\calX$ be an $\infty$-topos, and let $S$ denote the
collection of $\infty$-connective morphisms of $\calX$. Then $S$ is strongly 
saturated and of small generation $($see Definition \ref{saturated2}$)$. 
\end{proposition}

\begin{proof}
Lemma \ref{swarp4} implies that $S$ is stable under colimits in $\Fun(\Delta^1,\calX)$, and
Corollary \ref{pusherr} shows that $S$ is stable under pushouts. To prove that $S$ has the two-out-of-three property, we consider a diagram $\sigma: \Delta^2 \rightarrow \calX$, which we depict as
$$ \xymatrix{ & Y \ar[dr]^{g} & \\
X \ar[ur]^{f} \ar[rr]^{h} & & Z. }$$
If $f$ is $\infty$-connective, then Proposition \ref{inftychange} implies that $g$
is $\infty$-connective if and only if $h$ is $\infty$-connective. Suppose that $g$ and $h$
are $\infty$-connective. The long exact sequence
$$\ldots \rightarrow f^{\ast} \pi_{n+1}(g) \rightarrow \pi_n(f) \rightarrow \pi_n(h) \rightarrow f^{\ast}
\pi_n(g) \rightarrow \pi_{n-1}(f) \rightarrow \ldots$$ 
of Remark \ref{sequence} shows that $\pi_n(f) \simeq \ast$ for all $n \geq 0$. It
will therefore suffice to prove that $f$ is an effective epimorphism. According to Proposition \ref{conslice}, it will suffice to show that $\sigma$ is an effective epimorphism in 
$\calX_{/Z}$. According to Proposition \ref{pi00detects}, it suffices to show that
$\tau_{\leq 0}^{\calX_{/Z}}(h)$ and $\tau_{\leq 0}^{\calX_{/Z}}(g)$ are both final objects of
$\calX_{/Z}$, which follows from the $0$-connectivity of $g$ and $h$ (Proposition \ref{goober}). 

To show that $S$ is of small generation, it suffices (in view of Lemma \ref{perry}) to show that
the full subcategory of $\Fun(\Delta^1,\calX)$ spanned by $S$ is accessible. This follows from Lemma \ref{swarp4}.
\end{proof}

Let $\calX$ be an $\infty$-topos. We will say that an object $X$ of $\calX$ is
{\it hypercomplete}\index{gen}{hypercomplete!object} if it is local with respect to the class of $\infty$-connective morphisms. 
Let $\calX^{\hyp}$\index{gen}{hypercomplete!$\infty$-topos}\index{not}{Xhyp@$\calX^{\hyp}$} denote the full subcategory of $\calX$ spanned by the hypercomplete objects of $\calX$. Combining Propositions \ref{goober3} and \ref{local}, we deduce that
$\calX^{\hyp}$ is an accessible localization of $\calX$. Moreover, since Proposition \ref{inftychange} implies that the class of $\infty$-connective morphisms is stable under pullback, we deduce from Proposition \ref{charleftloc} that $\calX^{\hyp}$ is a {\em left exact} localization
of $\calX$. It follows that $\calX^{\hyp}$ is itself an $\infty$-topos. We will show in a moment that
$\calX^{\hyp}$ can be described by a universal property.

\begin{lemma}\label{sshock1}
Let $\calX$ be an $\infty$-topos, and let $n < \infty$. Then
$\tau_{\leq n} \calX \subseteq \calX^{\hyp}$.
\end{lemma}

\begin{proof}
Corollary \ref{goober2} implies that an $n$-truncated object of $\calX$ is local with respect to every $n$-connective morphism of $\calX$, and therefore with respect to every $\infty$-connective morphism of $\calX$. 
\end{proof}

\begin{lemma}\label{sshock2}
Let $\calX$ be an $\infty$-topos, let $L: \calX \rightarrow \calX^{\hyp}$ be a left adjoint to the inclusion, and let $X \in \calX$ be such that $LX$ is an $\infty$-connective object of
$\calX^{\hyp}$. Then $LX$ is a final object of $\calX^{\hyp}$.
\end{lemma}

\begin{proof}
For each $n  < \infty$, we have equivalences
$$1_{\calX} \simeq \tau_{\leq n}^{\calX^{\hyp}} LX \simeq L \tau_{\leq n}^{\calX} X\simeq \tau_{\leq n}^{\calX} X$$
where the first is because of our hypothesis that $LX$ is $\infty$-connective, the second
is given by Proposition \ref{compattrunc}, and the third by Lemma \ref{sshock1}. It follows
that $X$ is an $\infty$-connective object of $\calX$, so that $LX$ is a final object of
$\calX^{\hyp}$ by construction.
\end{proof}

We will say that an $\infty$-topos $\calX$ is {\it hypercomplete} if $\calX^{\hyp} = \calX$; in
other words, $\calX$ is hypercomplete if every $\infty$-connective morphism of $\calX$ is an equivalence, so that Whitehead's theorem holds in $\calX$.

\begin{remark}
In \cite{toen}, the authors use the term {\it $t$-completeness} to refer to the property that we have called hypercompleteness.
\end{remark}

\begin{lemma}
Let $\calX$ be an $\infty$-topos. Then the $\infty$-topos $\calX^{\hyp}$ is hypercomplete.
\end{lemma}

\begin{proof}
Let $f: X \rightarrow Y$ be an $\infty$-connective morphism in $\calX^{\hyp}$. Applying
Lemma \ref{sshock2} to the $\infty$-topos $(\calX^{\hyp})_{/Y} \simeq (\calX_{/Y})^{\hyp}$, we deduce that $f$ is an equivalence.
\end{proof}

We are now prepared to characterize $\calX^{\hyp}$ by a universal property:

\begin{proposition}
Let $\calX$ and $\calY$ be $\infty$-topoi. Suppose that $\calY$ is hypercomplete.
Then composition with the inclusion $\calX^{\hyp} \subseteq \calX$ induces an isomorphism
$$ \Fun_{\ast}(\calY, \calX^{\hyp} ) \rightarrow \Fun_{\ast}(\calY, \calX).$$
\end{proposition}

\begin{proof}
Let $f_{\ast}: \calY \rightarrow \calX$ be a geometric morphism; we wish to prove that
$f_{\ast}$ factors through $\calX^{\hyp}$. Let $f^{\ast}$ denote a left adjoint to $f_{\ast}$; it will suffice to show that $f^{\ast}$ carries each $\infty$-connective morphism $u$ of
$\calX$ to an equivalence in $\calY$. Proposition \ref{inftychange} implies that
$f^{\ast}(u)$ is $\infty$-connective, and the hypothesis that $\calY$ is hypercomplete guarantees that $u$ is an equivalence.
\end{proof}

The following result establishes the relationship between our theory of hypercompleteness
and the Brown-Joyal-Jardine theory of simplicial presheaves.

\begin{proposition}\label{suga}
Let $\calC$ be a small category equipped with a Grothendieck topology, and let $\bfA$
denote the category of simplicial presheaves on $\calC$, endowed with the local model structure $($ see Proposition \ref{jardinesardine} $)$.
Let $\bfA^{\degree}$ denote the full subcategory consisting of fibrant-cofibrant objects, and let
$\calA = \sNerve( \bfA^{\degree} )$ be the corresponding $\infty$-category. Then $\calA$
is equivalent to $\Shv(\calC)^{\hyp}$; in particular, it is a hypercomplete $\infty$-topos.
\end{proposition}

\begin{proof}
Let $\calP(\calC)$ denote the $\infty$-category $\calP(\Nerve(\calC))$ of presheaves on
$\Nerve(\calC)$, and let
$\bfA'$ denote the model category of simplicial presheaves on $\calC$, endowed with the {\em injective} model structure of \S \ref{quasilimit3}. According to Proposition \ref{gumby444}, 
the simplicial nerve functor induces an equivalence
$$ \theta: \sNerve ({\bfA'}^{\degree}) \rightarrow \calP(\calC).$$
We may identify $\sNerve(\bfA^{\degree})$ with the full subcategory of 
$\sNerve({\bfA'}^{\degree})$ spanned by the $S$-local objects, where $S$ is the class
of local equivalences (Proposition \ref{suritu}). 

We first claim that $\theta | \sNerve(\bfA^{\degree})$ factors
through $\Shv(\calC)$. Consider an object $C \in \calC$ and a sieve
$\calC^{(0)}_{/C} \subseteq \calC_{/C}$. Let $\chi_{C}: \calC \rightarrow \Set$ be the functor
$D \mapsto \Hom_{\calC}(D,C)$ represented by $\calC$, let $\chi_{C}^{(0)}$ be the subfunctor of $\chi_{C}$ determined by the sieve $\calC^{(0)}_{/C}$, and let $i: \chi_{C}^{(0)} \rightarrow \chi_{C}$ be the inclusion. We regard $\chi_{C}$ and $\chi_{C}^{(0)}$ as objects of
simplicial presheaves on $\calC$, which take values in the full subcategory 
of $\sSet$ spanned by the {\em constant} simplicial sets. We observe that
every simplicial presheaf on $\calC$ which is valued in constant simplicial sets
is automatically fibrant, and every object of $\bfA'$ is cofibrant. Consequently, we
may regard $i$ as a morphism in the $\infty$-category $\sNerve (\bfA')^{\degree}$.
It is easy to see that $\theta(i)$ represents the monomorphism
$U \rightarrow j(C)$ classified by the sieve $\calC^{(0)}_{/C}$. If $\calC^{(0)}_{/C}$ is a covering sieve on $C$, then $i$ is a local equivalence. Consequently, every object $X \in \sNerve (\bfA^{\degree})$ is $i$-local, so that $\theta(X)$ is $\theta(i)$-local. By construction, $\Shv(\calC)$ 
is the full subcategory of $\calP(\calC)$ spanned by those objects which are $\theta(i)$-local
for every covering sieve $\calC^{(0)}_{/C}$ on every object $C \in \calC$. We conclude
that $\theta | \sNerve(\bfA^{\degree})$ factors through $\Shv(\calC)$. 

Let $\calX = \theta^{-1} \Shv(\calC)$, so that $\sNerve(\bfA^{\degree})$ can be identified
with the collection of $S'$-local objects of $\calX$, where $S'$ is the collection of all morphisms
in $\calX$ which belong to $S$. Then $\theta$ induces an equivalence
$\sNerve (\bfA^{\degree}) \rightarrow \theta(S')^{-1} \Shv(\calC)$. We now observe
that a morphism $f$ in $\calX$ belongs to $S'$ if and only if $\theta(f)$ is an $\infty$-connective morphism in $\Shv(\calC)$ (since the condition of being a local equivalence can be tested on homotopy sheaves). It follows that $\theta(S')^{-1} \Shv(\calC) = \Shv(\calC)^{\hyp}$, as desired. 
\end{proof}

\begin{remark}
In \cite{toen}, the authors discuss a generalization of Jardine's construction, in which
the category $\calC$ is replaced by a simplicial category. Proposition \ref{suga} holds in this more general situation as well.
\end{remark}

We conclude this section with a few remarks about localizations of an $\infty$-topos
$\calX$. In \S \ref{leloc} we introduced the class of topological localizations of $\calX$, consisting of those left exact localizations which can be obtained by inverting monomorphisms in $\calX$.
The hypercompletion $\calX^{\hyp}$ is, in some sense, at the other extreme: it is obtained by
inverting the $\infty$-connective morphisms in $\calX$, which are never monomorphisms unless they are already equivalences. In fact, $\calX^{\hyp}$ is the {\em maximal} (left exact) localization of $\calX$ which can be obtained without inverting monomorphisms:

\begin{proposition}\label{antitopchar}
Let $\calX$ and $\calY$ be $\infty$-topoi, and let $f^{\ast}: \calX \rightarrow \calY$ be a left exact, colimit preserving functor.
The following conditions are equivalent:
\begin{itemize}
\item[$(1)$] For every monomorphism $u$ in $\calX$, if $f^{\ast} u$ is an equivalence in $\calY$, then
$u$ is an equivalence in $\calX$.
\item[$(2)$] For every morphism $u \in \calX$, if $f^{\ast} u$ is an equivalence in $\calY$, then $f$ is $\infty$-connective.
\end{itemize}
\end{proposition}

\begin{proof}
Suppose first that $(2)$ is satisfied. If $u$ is a monomorphism and $f^{\ast} u$ is an equivalence in $\calY$, then $u$ is $\infty$-connective. In particular, $u$ is both a monomorphism and an effective epimorphism, and therefore an equivalence in $\calX$. This proves $(1)$. Conversely, suppose that $(1)$ is satisfied, and let $u: X \rightarrow Z$ be an arbitrary morphism in $\calX$ such
that $f^{\ast}(u)$ is an equivalence. We will prove by induction on $n$ that $u$ is $n$-connective.

We first consider the case $n=0$. Choose a factorization
$$ \xymatrix{ & Y \ar[dr]^{u''} & \\
X \ar[ur]^{u'} \ar[rr]^{u} & & Z }$$
where $u'$ is an effective epimorphism, and $u''$ is a monomorphism. Since
$f^{\ast} u$ is an equivalence, Corollary \ref{composite} implies that $f^{\ast} u''$
is an effective epimorphism. Since $f^{\ast} u''$ is also a monomorphism (in virtue of our
assumption that $f$ is left exact), we conclude that $f^{\ast} u''$ is an equivalence.
Applying $(1)$, we deduce that $u''$ is an equivalence, so that $u$ is an effective epimorphism as desired.

Now suppose $n > 0$. According to Proposition \ref{trowler}, it will suffice to show that the
diagonal map $\delta: X \rightarrow X \times_{Z} X$ is $(n-1)$-connective. By the inductive hypothesis, it will suffice to prove that $f^{\ast}(\delta)$ is an equivalence in $\calY$. We conclude by observing that 
$f^{\ast}$ is left exact, so we can identify $\delta$ with the diagonal map associated to
the equivalence $f^{\ast}(u): f^{\ast} X \rightarrow f^{\ast} Z$.
\end{proof}

\begin{definition}\index{gen}{localization!cotopological}
Let $\calX$ be an $\infty$-topos, and let $\calY \subseteq \calX$ be an accessible
left exact localization of $\calX$. We will say that $\calY$ is an {\em cotopological} localization
of $\calX$ if the left adjoint $L: \calX \rightarrow \calY$ to the inclusion of $\calY$ in $\calX$
satisfies the equivalent conditions of Proposition \ref{antitopchar}.
\end{definition}

\begin{remark}\label{sorkum}
Let $f^{\ast}: \calX \rightarrow \calY$ be the left adjoint of a geometric morphism between $\infty$-topoi, and suppose that the equivalent conditions of Proposition \ref{antitopchar} are satisfied.
Let $u: X \rightarrow Z$ be a morphism in $\calX$, and choose a factorization
$$ \xymatrix{ & Y \ar[dr]^{u''} & \\
X \ar[ur]^{u'} \ar[rr]^{u} & & Z }$$
where $u'$ is an effective epimorphism and $u''$ is a monomorphism. Then
$u''$ is an equivalence if and only if $f^{\ast}(u'')$ is an equivalence. Applying
Corollary \ref{composite}, we conclude that $u$ is an effective epimorphism if and only if
$f^{\ast}(u)$ is an effective epimorphism.
\end{remark}

The hypercompletion $\calX^{\hyp}$ of an $\infty$-topos $\calX$ can be characterized as the {\em maximal} cotopological localization of $\calX$ (that is, the cotopological localization which
is obtained by inverting as many morphisms as possible). According to our next result, every localization can be obtained by combining topological and cotopological localizations:

\begin{proposition}\label{factanti}
Let $\calX$ be an $\infty$-topos, and let $\calX'' \subseteq \calX$ be an accessible, left
exact localization of $\calX$. Then there exists a topological localization
$\calX' \subseteq \calX$ such that $\calX'' \subseteq \calX'$ is a cotopological localization
of $\calX'$.
\end{proposition}

\begin{proof}
Let $L: \calX \rightarrow \calX''$ be a left adjoint to the inclusion, let
$S$ be the collection of all monomorphisms $u$ in $\calX$ such that $Lu$ is an equivalence,
and let $\calX' = S^{-1} \calX$ be the collection of $S$-local objects of $\calX$. Since $L$ is left exact, $S$ is stable under pullbacks and therefore determines a topological localization of $\calX$. By construction, we have $\calX'' \subseteq \calX'$. The restriction $L| \calX'$ exhibits
$\calX''$ as an accessible left exact localization of $\calX'$. Let $u$ be a monomorphism
in $\calX'$ such that $Lu$ is an equivalence. Then $u$ is a monomorphism in $\calX$, so
that $u \in S$. Since $\calX'$ consists of $S$-local objects, we conclude that $u$ is an equivalence. It follows that $\calX''$ is a cotopological localization of $\calX'$, as desired.
\end{proof}

\begin{remark}
It is easy to see that the factorization of Proposition \ref{factanti} is essentially uniquely determined: more precisely, $\calX'$ is unique provided that we assume that it is stable under equivalences in $\calX$.
\end{remark}

Combining Proposition \ref{factanti} with Remark \ref{charnice}, we see that every $\infty$-topos $\calX$ can be obtained in following way:
\begin{itemize}
\item[$(1)$] Begin with the $\infty$-category $\calP(\calC)$ of presheaves on some small $\infty$-category $\calC$.

\item[$(2)$] Choose a Grothendieck topology on $\calC$: this is equivalent to choosing a 
left exact localization of the underlying topos $\Disc(\calP(\calC)) = \Set^{ \h{\calC^{op}}}$.

\item[$(3)$] Form the associated topological localization $\Shv(\calC) \subseteq \calP(\calC)$, which can be described as the pullback 
$$ \calP(\calC) \times_{ \calP( \Nerve ( \h{\calC}) )} \Shv( \Nerve ( \h{\calC} ) )$$
in $\RGeom$.

\item[$(4)$] Form a cotopological localization of $\Shv(\calC)$ by inverting some class
of $\infty$-connective morphisms of $\Shv(\calC)$.
\end{itemize}

\begin{remark}\label{comk1}
Let $\calX$ be an $\infty$-topos. The collection of all $\infty$-connective morphisms
in $\calX$ is saturated. It follows from Proposition \ref{nir} that there exists a factorization system
$(S_L, S_R)$ on $\calX$, where $S_L$ is the collection of all $\infty$-connective morphisms in
$\calX$. We will say that a morphism in $\calX$ is {\it hypercomplete} if it belongs to $S_R$.
Unwinding the definitions (and using the fact that a morphism in $\calX_{/Y}$ is $\infty$-connective if and only if its image in $\calX$ is $\infty$-connective), we conclude that a morphism
$f: X \rightarrow Y$ is hypercomplete if and only if it is hypercomplete when viewed as an object of the $\infty$-topos $\calX_{/Y}$ (see \S \ref{hyperstacks}). 

Using Proposition \ref{swimmm}, we deduce that the collection of hypercomplete morphisms in $\calX$ is stable under limits and the formation of pullback squares.
\end{remark}

\begin{remark}\label{suchlike}
Let $\calX$ be an $\infty$-topos. The condition that a morphism $f: X \rightarrow Y$ be hypercomplete is {\em local}: that is, if $\{ Y_{\alpha} \rightarrow Y \}$ is a collection of morphisms which
determine an effective epimorphism $\coprod Y_{\alpha} \rightarrow Y$, and each of the induced maps
$f_{\alpha}: X \times_{Y} Y_{\alpha} \rightarrow Y_{\alpha}$ is hypercomplete, then $f$ is hypercomplete. To prove this, we set $Y_0 = \coprod_{\alpha} Y_{\alpha}$; then
$\calX_{/Y_0} \simeq \prod_{ \alpha} \calX_{/Y_{\alpha}}$ (since coproducts in $\calX$ are disjoint),
so it is easy to see that the induced map $f': X \times_{Y} Y_0 \rightarrow Y_0$ is hypercomplete.
Let $Y_{\bigdot}$ be the simplicial object of $\calX$ given by the \Cech nerve of the effective epimorphism $Y_0 \rightarrow Y$. For every map $Z \rightarrow Y$, let $Z_{\bigdot}$ be
the simplicial object described by the formula
$Z_{n} = Y_{n} \times_{Y} Z$
(equivalently, $Z_{\bigdot}$ is the \Cech nerve of the effective epimorphism $Z \times_{Y} Y_0 \rightarrow Z$). Using Remark \ref{comk1}, we conclude that each of the maps $X_{n} \rightarrow Y_{n}$ is hypercomplete.

For every map $A \rightarrow Y$, the mapping space
$\bHom_{ \calX_{/Y}}( A, X)$ can be obtained as the totalization of a cosimplicial space
$$ n \mapsto \bHom_{ \calX_{/Y_n} }( A_{n}, X_n).$$
If $g: A \rightarrow B$ is an $\infty$-connective morphism in $\calX_{/Y}$, then
each of the induced maps $A_{n} \rightarrow B_{n}$ is $\infty$-connective, so the induced map
$$ \bHom_{ \calX_{/Y_n}}( B_{n}, X_{n} ) \rightarrow \bHom_{ \calX_{/Y_{n}}}( A_{n}, X_n)$$
is a homotopy equivalence. Passing to the totalization, we obtain a homotopy equivalence
$\bHom_{\calX_{/Y}}(B,X) \rightarrow \bHom_{\calX_{/Y}}( A, X)$. Thus $f$ is hypercomplete, as desired.
\end{remark}


\subsection{Hypercoverings}\label{hcovh}

Let $\calX$ be an $\infty$-topos. In \S \ref{hyperstacks}, we defined the {\em hypercompletion}
$\calX^{\hyp} \subseteq \calX$ to be the left exact localization of $\calX$ obtained by inverting the $\infty$-connective morphisms. In this section, we will give an alternative description of the hypercomplete objects $X \in \calX^{\hyp}$: they are precisely those objects of $\calX$ which satisfy a descent condition with respect to hypercoverings (Theorem \ref{surp}). We begin by reviewing the definition of a hypercovering.

Let $X$ be a topological space, and let $\calF$ be a presheaf of sets on $X$. To construct the sheaf associated to $\calF$, it is natural to consider the presheaf $\calF^{+}$, defined by
$$ \calF^{+} = \varinjlim_{ \calU } \varprojlim_{ V \in \calU } \calF(V).$$ 
Here the direct limit is taken over all sieves $\calU$ which cover $U$. There is an obvious map
$ \calF \rightarrow \calF^{+}$, which is an isomorphism whenever $\calF$ is a sheaf. Moreover, $\calF^{+}$ is ``closer'' to being a sheaf than $\calF$ is. More precisely, $\calF^{+}$ is always a separated presheaf: two sections of $\calF^{+}$ which agree locally automatically coincide.
If $\calF$ is itself a separated presheaf, then $\calF^{+}$ is a sheaf.\index{not}{Fcal+@$\calF^{+}$}

For a general presheaf $\calF$, we need to apply the above construction twice to construct the associated sheaf $(\calF^{+})^{+}$. To understand the problem, let us try to prove that $\calF^{+}$ is a sheaf (to see where the argument breaks down). Suppose given an open covering
$X = \bigcup U_{\alpha}$, and a collection of sections $s_{\alpha} \in \calF^{+}(U_{\alpha})$ such that
$$ s_{\alpha} | U_{\alpha} \cap U_{\beta} = s_{\beta} | U_{\alpha} \cap U_{\beta}. $$
Refining the covering $U_{\alpha}$ if necessary, we may assume that each $s_{\alpha}$
is the image of some section $t_{\alpha} \in \calF(U_{\alpha})$. However, the equation
$$ t_{\alpha} | U_{\alpha} \cap U_{\beta} = t_{\beta} | U_{\alpha} \cap U_{\beta} $$
only holds locally on $U_{\alpha} \cap U_{\beta}$, so the sections $t_{\alpha}$ do not necessarily determine a global section of $\calF^{+}$. To summarize: the freedom to consider
arbitrarily fine open covers $\calU = \{ U_{\alpha} \}$ is not enough; we also need to be able to refine the intersections $U_{\alpha} \cap U_{\beta}$. This leads very naturally to the notion of a {\it hypercovering}. Roughly speaking, a hypercovering of $X$ consists of an open covering
$\{ U_{\alpha} \}$ of $X$, an open covering of $\{ V_{\alpha\beta\gamma} \}$ of each intersection $U_{\alpha} \cap U_{\beta}$, and analogous data associated to more complicated intersections (see Definition \ref{worum} for a more precise formulation).\index{gen}{hypercovering}

In classical sheaf theory, there are two ways to construct the sheaf associated to a presheaf $\calF$:
\begin{itemize}
\item[$(1)$] One can apply the construction $\calF \mapsto \calF^{+}$ twice.
\item[$(2)$] Using the theory of hypercoverings, one can proceed directly by defining
$$\calF^{\dagger}(U) = \varinjlim_{\calU} \varprojlim \calF(V)$$
where the direct limit is now taken over arbitrary {\em hypercoverings} $\calU$.
\end{itemize}

In higher category theory, the difference between these two approaches becomes more prominent. For example, suppose that $\calF$ is not a presheaf of sets, but a presheaf of {\em groupoids} on
$X$. In this case, one can construct the associated sheaf of groupoids using either approach. However, in the case of approach $(1)$, it is necessary to apply the construction
$\calF \mapsto \calF^{+}$ {\em three} times: the first application guarantees that the automorphism groups of sections of $\calF$ are separated presheaves, the second guarantees that they are sheaves, and the third guarantees that $\calF$ itself satisfies descent. More generally, if $\calF$ is a sheaf of $n$-truncated spaces, then the sheafification of $\calF$ via approach $(1)$ takes place in $(n+2)$-stages. 

When we pass to the case $n= \infty$, the situation becomes more complicated. If $\calF$ is a presheaf of spaces on $X$, then it is not reasonable to expect to obtain a sheaf by applying the construction $\calF \mapsto \calF^{+}$ any finite number of times. In fact, it is not obvious that $\calF^{+}$ is any closer than $\calF$ to being a sheaf. Nevertheless, this is true: we can construct the sheafification of $\calF$ via a {\em transfinite iteration} of the construction $\calF \mapsto \calF^{+}$. More precisely, we define a transfinite sequence of presheaves
$$ \calF(0) \rightarrow \calF(1) \rightarrow \ldots $$
as follows:
\begin{itemize}
\item[$(i)$] Let $\calF(0) = \calF$.
\item[$(ii)$] For every ordinary $\alpha$, let $\calF(\alpha+1) = \calF(\alpha)^{+}$.
\item[$(iii)$] For every limit ordinal $\lambda$, let $\calF(\lambda) = \colim_{\alpha} \calF(\alpha)$, where $\alpha$ ranges over ordinals less than $\lambda$.
\end{itemize}

One can show that the above construction {\em converges}, in the sense that $\calF(\alpha)$
is a sheaf for $\alpha \gg 0$ (and therefore $\calF(\alpha) \simeq \calF(\beta)$ for $\beta \geq \alpha)$. Moreover, $\calF(\alpha)$ is universal among sheaves of spaces which admits a map from $\calF$.

Alternatively, one use the construction $\calF \mapsto \calF^{\dagger}$ to construct a sheaf of spaces from $\calF$ in a single step. The universal property asserted above guarantees the existence of a morphism of sheaves $\theta: \calF(\alpha) \rightarrow \calF^{\dagger}$. However, the morphism $\theta$ is generally {\em not} an equivalence. Instead, $\theta$ realizes
$\calF^{\dagger}$ as the {\em hypercompletion} of $\calF(\alpha)$ in the $\infty$-topos $\Shv(X)$. 
We will not prove this statement directly, but will instead establish a reformulation (Corollary \ref{charhyp}) which does not make reference to the sheafification constructions outlined above.

Before we can introduce the definition of a hypercovering, we need to review some simplicial terminology.

\begin{notation}\index{not}{Deltaleqn@$\cDelta^{\leq n}$}
For each $n \geq 0$, let $\cDelta^{\leq n}$ denote the full
subcategory of $\cDelta$ spanned by the set of objects
$\{ [0], \ldots, [n] \}$. 
If $\calX$ is a presentable $\infty$-category, the 
restriction functor
$$ \sk_n: \calX_{\Delta} \rightarrow \Fun(\Nerve (\cDelta^{\leq n})^{op}, \calX)$$
has a right adjoint, given by right Kan extension along the inclusion $\Nerve (\cDelta^{\leq n})^{op} \subseteq \Nerve(\cDelta)^{op}$. Let $\cosk_n: \calX_{\Delta} \rightarrow \calX_{\Delta}$ be the composition of $\sk_n$ with its right adjoint. We will refer to $\cosk_{n}$ as the {\it $n$-coskeleton functor}.\index{gen}{skeleton}\index{gen}{coskeleton}\index{not}{skn@$\sk_n$}\index{not}{coskn@$\cosk_n$}
\end{notation}

\begin{definition}\label{worum}\index{gen}{hypercovering}\index{gen}{hypercovering!effective}
Let $\calX$ be an $\infty$-topos. A simplicial object $U_{\bigdot} \in \calX_{\Delta}$ is a {\it hypercovering of $\calX$} if, for each $n \geq 0$, the unit map
$$ U_{n} \rightarrow ( \cosk_{n-1} U_{\bigdot} )_n $$
is an effective epimorphism. We will say that $U_{\bigdot}$ is an {\em effective hypercovering of $\calX$}
if the colimit of $U_{\bigdot}$ is a final object of $\calX$.
\end{definition}

\begin{remark}
More informally, a simplicial object $U_{\bigdot} \in \calX_{\Delta}$ is a hypercovering of $\calX$ if each of the associated maps
$$ U_0 \rightarrow 1_{\calX}$$
$$ U_1 \rightarrow U_0 \times U_0 $$
$$ U_2 \rightarrow \ldots $$
is an effective epimorphism.
\end{remark}

\begin{lemma}\label{fier1}
Let $\calX$ be an $\infty$-topos, and let $U_{\bigdot}$ be a simplicial object
in $\calX$.  Let $L: \calX \rightarrow \calX^{\hyp}$ be a left adjoint to the inclusion. The following conditions are equivalent:
\begin{itemize}
\item[$(1)$] The simplicial object $U_{\bigdot}$ is a hypercovering of $\calX$.
\item[$(2)$] The simplicial object $L \circ U_{\bigdot}$ is a hypercovering
of $\calX^{\hyp}$.
\end{itemize}
\end{lemma}

\begin{proof}
Since $L$ is left exact, we can identify $L \circ \cosk_{n} U_{\bigdot}$ with
$\cosk_{n} (L \circ U_{\bigdot})$. The desired result now follows from Remark \ref{sorkum}.
\end{proof}

\begin{lemma}\label{fier0}
Let $\calX$ be an $\infty$-topos, and let $U$ be an $\infty$-connective object of $\calX$.
Let $U_{\bigdot}$ be the constant simplicial object with value $U$.
Then $U_{\bigdot}$ is a hypercovering of $\calX$.
\end{lemma}

\begin{proof}
Using Lemma \ref{fier1}, we can reduce to the case where $\calX$ is hypercomplete.
Then $U \simeq 1_{\calX}$, so that $U_{\bigdot}$ is equivalent to the constant functor with value $1_{\calX}$, and is therefore a final object of $\calX_{\Delta}$. For each $n \geq 0$, the coskeleton functor $\cosk_{n-1}$ preserves small limits, so $\cosk_{n-1} U_{\bigdot}$ is also a final object of $U_{\bigdot}$. It follows that the unit map $U_{\bigdot} \rightarrow \cosk_{n-1} U_{\bigdot}$ is an equivalence.
\end{proof}



\begin{notation}
Let $\cDelta_{s}$ be the subcategory of $\cDelta$ with the same objects, but where the morphisms are given by {\em injective} order preserving maps between nonempty linearly ordered sets.
If $\calX$ is an $\infty$-category, we will refer to a diagram $\Nerve(\cDelta_{s})^{op} \rightarrow \calX$ as a {\it semisimplicial object of $\calX$}.
\end{notation}

\begin{lemma}\label{bball4}
The inclusion $\Nerve(\cDelta^{op}_{s}) \subseteq \Nerve(\cDelta^{op})$ is cofinal. 
\end{lemma}

\begin{proof}
According to Theorem \ref{hollowtt}, it will suffice to prove that for every $n \geq 0$, the 
category $\calC = \cDelta_{s} \times_{\cDelta} \cDelta_{/ [n]}$ has weakly contractible nerve.
To prove this, we let $F: \calC \rightarrow \calC$ be the constant functor taking value given by the inclusion $[0] \subseteq [n]$, and $G: \calC \rightarrow \calC$ the functor which carries
an arbitrary map $[m] \rightarrow [n]$ to the induced map
$[0] \amalg [m] \rightarrow [n]$. We have natural transformations of functors
$$ F \rightarrow G \leftarrow \id_{\calC}. $$
Let $X$ be the topological space $| \Nerve(\calC) |$. The natural transformations above show that the identity map $\id_{X}$ is homotopic to a constant, so that $X$ is contractible as desired.
\end{proof}

Consequently, if $U_{\bigdot}$ is a simplicial object in an $\infty$-category $\calX$, and $U_{\bigdot}^{s} = U_{\bigdot} | \Nerve(\cDelta^{op}_{s})$ is the associated semisimplicial object, then we can identify colimits of $U_{\bigdot}$ with colimits of $U_{\bigdot}^{s}$.

We will say that a simplicial object $U_{\bigdot}$ in an $\infty$-category
$\calX$ is {\it $n$-coskeletal}\index{gen}{coskeletal} if it is a right Kan extension of its restriction to
$\Nerve(\cDelta^{op}_{\leq n})$. Similarly, we will say that a semisimplicial
object of $U_{\bigdot}$ of $\calX$ is {\it $n$-coskeletal} if it is a right Kan extension
of its restriction to $\Nerve(\cDelta^{op}_{s,\leq n})$, where
$\cDelta_{s,\leq n} = \cDelta_{s} \times_{\cDelta} \cDelta_{\leq n}$.

\begin{lemma}\label{bball5}
Let $\calX$ be an $\infty$-category, let $U_{\bigdot}$ be a simplicial object of $\calX$, and let $U^{s}_{\bigdot} = U_{\bigdot} | \Nerve(\cDelta_{s}^{op})$ the associated semisimplicial object. Then $U_{\bigdot}$ is $n$-coskeletal if and only if $U^{s}_{\bigdot}$ is $n$-coskeletal.
\end{lemma}

\begin{proof}
It will suffice to show that, for each $\Delta^m \in \cDelta$, the nerve of the inclusion
$$ (\cDelta_{s})_{/[m]} \times_{ \cDelta_{s} } \cDelta_{s, \leq n} \subseteq 
\cDelta_{ / [m] } \times_{\cDelta} \cDelta_{\leq n}$$
is cofinal. Let $\theta: [m'] \rightarrow [m]$ be an object of 
$\cDelta_{ / [m] } \times_{\cDelta} \cDelta_{\leq n}$. We let
$\calC$ denote the category of all factorizations 
$$ [m'] \stackrel{\theta'}{\rightarrow} [m''] \stackrel{\theta''}{\rightarrow} [m]$$
for $\theta$ such that $\theta''$ is a monomorphism and $m'' \leq n$. According to Theorem \ref{hollowtt}, it will suffice to prove that $\Nerve(\calC)$ is weakly contractible (for every choice of $\theta$). We now
simply observe that $\calC$ has an initial object (given by the unique a factorization where
$\theta'$ is an epimorphism). 
\end{proof}

\begin{lemma}[\cite{hollander}]\label{bball1}
Let $\calX$ be an $\infty$-topos, and let $U_{\bigdot}$ be an $n$-coskeletal hypercovering of $\calX$. Then $U_{\bigdot}$ is effective.
\end{lemma}

\begin{proof}
We will prove this result by induction on $n$. If $n=0$, then $U_{\bigdot}$ can be identified with the underlying groupoid of the \Cech nerve of the map $\theta: U_0 \rightarrow 1_{\calX}$, where $1_{\calX}$ is a final object of $\calX$. Since $U_{\bigdot}$ is a hypercovering, $\theta$ is an effective epimorphism, so the \Cech nerve of $\theta$ is a colimit diagram and the desired result follows. Let us therefore assume that $n > 0$. Let $V_{\bigdot} = \cosk_{n-1} U_{\bigdot}$, and let
$f_{\bigdot}: U_{\bigdot} \rightarrow V_{\bigdot}$ be the adjunction map. 
For each $m \geq 0$, the map $f_{m}: U_{m} \rightarrow V_{m}$ is a composition of finitely many pullbacks of $f_{n}$. Since $U_{\bigdot}$ is a hypercovering, $f_n$ is an effective epimorphism, so each $f_m$ is also an effective epimorphism. We also observe that $f_m$ is an equivalence for $m < n$.

Let $W_{+}: \Nerve (\cDelta_{+} \times \cDelta)^{op} \rightarrow \calX$
be a \Cech nerve of $f_{\bigdot}$ (formed in the $\infty$-category $\calX_{\Delta}$ of simplicial objects of $\calX$). We observe that
$W_{+} | \Nerve ( \{ \emptyset \} \times \cDelta)^{op}$ can be identified with
$V_{\bigdot}$. Since $V_{\bigdot}$
is an $(n-1)$-coskeletal hypercovering of $\calX$, the inductive hypothesis implies that any colimit $|V_{\bigdot}|$ is a final object of $\calX$. The inclusion $\Nerve ( \{ \emptyset \} \times \cDelta)^{op}
\subseteq \Nerve( \cDelta_{+} \times \cDelta)^{op}$ is cofinal (being a product
of $\Nerve(\cDelta)^{op}$ with the inclusion of a final object into $\Nerve(\cDelta_{+})^{op}$ ), so
we may identify colimits of $W_{+}$ with colimits of $V_{\bigdot}$. It follows that
any colimit of $W_{+}$ is a final object of $\calX$.
We next observe that each of the augmented simplicial objects
$W_{+}| \Nerve( \cDelta_{+} \times \{ [m] \})^{op}$ is a \Cech nerve of $f_{m}$, and therefore a colimit diagram (since $f_m$ is an effective epimorphism). Applying Lemma \ref{longerwait}, we conclude that $W_{+}$ is a left Kan extension of the bisimplicial object
$W = W_{+}| \Nerve( \cDelta \times \cDelta)^{op}$. According to Lemma \ref{kan0}, we can
identify colimits of $W_{+}$ with colimits of $W$, so any colimit of $W$ is a final object of $\calX$.

Let $D_{\bigdot}: \Nerve(\cDelta^{op}) \rightarrow \calX$ be the simplicial object of $\calX$
obtained by composing $W$ with the diagonal map $\delta: \Nerve(\cDelta^{op}) \rightarrow \Nerve (\cDelta \times \cDelta)^{op}$. According to Lemma \ref{bball3}, $\delta$ is cofinal.
We may therefore identify colimits of $W$ with colimits of $D_{\bigdot}$, so that
any colimit $|D_{\bigdot}|$ of $D_{\bigdot}$ is a final object of $\calX$.

Let $U_{\bigdot}^{s} = U_{\bigdot} | \Nerve(\cDelta_{s}^{op})$, and let
$D_{\bigdot}^{s} = D_{\bigdot} | \Nerve(\cDelta_{s}^{op})$. We will prove that
$U_{\bigdot}^{s}$ is a retract of $D_{\bigdot}^{s}$ in the $\infty$-category of semisimplicial
objects of $\calX$. According to Lemma \ref{bball4}, we can identify colimits of
$D_{\bigdot}^{s}$ with colimits of $D_{\bigdot}$. It will follow that any colimit of
$U_{\bigdot}^{s}$ is a retract of a final object of $\calX$, and therefore itself final.
Applying Lemma \ref{bball4} again, we will conclude that any colimit of $U_{\bigdot}$ is
a final object of $\calX$, and the proof will be complete.

We observe that $D^{s}_{\bigdot}$ is the result of composing $W$ with the (opposite of the nerve of the) diagonal functor
$$\delta^{s}: \cDelta_{s} \rightarrow \cDelta \times \cDelta.$$
Similarly, the semisimplicial object $U^{s}_{\bigdot}$ is obtained from $W$ via the composition
$$ \epsilon: \cDelta_{s} \subseteq \cDelta \simeq \{ [0] \} \times \cDelta \subseteq
\cDelta \times \cDelta.$$
There is a obvious natural transformation of functors $\delta^{s} \rightarrow \epsilon$, which
yields a map of semisimplicial objects $\theta: U^{s}_{\bigdot} \rightarrow D^{s}_{\bigdot}$.
To complete the proof, it will suffice to show that there exists a map
$$ \theta': D^{s}_{\bigdot} \rightarrow U^{s}_{\bigdot}$$
such that $\theta' \circ \theta$ is homotopic to the identity on $U^{s}_{\bigdot}$.

According to Lemma \ref{bball5}, $U^{s}_{\bigdot}$ is $n$-coskeletal as a {\em semisimplicial} object of $\calX$. Let $D^{s}_{\leq n}$ and $U^{s}_{\leq n}$ denote
restrictions of $D^{s}_{\bigdot}$ and $U^{s}_{\bigdot}$ to $\Nerve(\cDelta_{s, \leq n}^{op})$, and $\theta_{\leq n}:
U^{s}_{\leq n} \rightarrow D^{s}_{\leq n}$ the morphism induced by $\theta$.
We have canonical homotopy equivalences
$$\bHom_{ \Fun(\Nerve(\cDelta^{op}_{s}), \calX)} ( D^{s}_{\bigdot}, U^{s}_{\bigdot})
\simeq \bHom_{ \Fun(\Nerve(\cDelta^{op}_{s, \leq n}),\calX)}( D^{s}_{\leq n}, U^{s}_{\leq n})$$
$$\bHom_{ \Fun(\Nerve(\cDelta^{op}_{s}),\calX)} ( U^{s}_{\bigdot}, U^{s}_{\bigdot})
\simeq \bHom_{ \Fun(\Nerve(\cDelta^{op}_{s, \leq n}),\calX)}( U^{s}_{\leq n}, U^{s}_{\leq n}).$$
It will therefore suffice to prove that there exists a map
$$ \theta'_{\leq n}: D^{s}_{\leq n} \rightarrow U^{s}_{\leq n}$$
such that $\theta'_{\leq n} \circ \theta_{\leq n}$ is homotopic to the identity on $U^{s}_{\leq n}$.

Consider the functors 
$$\overline{\delta}^{s}: \cDelta_{s, \leq n} \rightarrow \cDelta_{+} \times \cDelta$$
$$\overline{\epsilon}: \cDelta_{s, \leq n} \rightarrow \cDelta_{+} \times \cDelta$$
defined as follows:
$$ \overline{\delta}^{s}( [m]) = 
\begin{cases} ( \emptyset, [m] ) & \text{if } m < n \\
([n], [n] ) & \text{if } m = n \end{cases}$$
$$ \overline{\epsilon}( [m] ) = 
\begin{cases} ( \emptyset, [m] ) & \text{if } m < n \\
( [0] , [n] ) & \text{if } m = n. \end{cases}$$
We have a commutative diagram of natural transformations
$$ \xymatrix{ \overline{\delta}^{s} \ar[r] \ar[d] & \delta^{s} \ar[d] \\
\overline{\epsilon} \ar[r] & \epsilon }$$
which gives rise to a diagram
$$ \xymatrix{ \overline{D}^{s}_{\leq n} & D^{s}_{\leq n} \ar[l] \\
\overline{U}^{s}_{\leq n} \ar[u]^{\psi_{\leq n}} & U^{s}_{\leq n} \ar[u]^{\theta_{\leq n}} \ar[l] }$$
in the $\infty$-category $\Fun(\Nerve(\cDelta^{op}_{s, \leq n}),\calX)$. The
vertical arrows are equivalences. Consequently, it will suffice to produce
a (homotopy) left inverse to $\psi_{\leq n}$. 

For $m \geq 0$, let $V^{s}_{\leq m} = V_{\bigdot} | \cDelta_{s, \leq m}$. We can identify
$\overline{D}^{s}_{\leq n}$ and $\overline{U}^{s}_{\leq n}$ with objects
$X,Y \in \calX_{/V^{s}_{\leq n-1}}$, and $\psi_{\leq n}$ with a morphism
$f: X \rightarrow Y$. To complete the proof, it will suffice to produce a left
inverse to $f$ in the $\infty$-category $\calX_{/V^{s}_{\leq n-1}}$. 
We observe that, since $V_{\bigdot}$ is $(n-1)$-coskeletal, we have a diagram of trivial fibrations
$$ \calX_{/V_{n}} \leftarrow \calX_{/ V^{s}_{\leq n} } \rightarrow \calX_{ / V^{s}_{\leq n-1}}.$$
Using this diagram (and the construction of $W$), we conclude that $Y$ can be identified
with a product of $(n+1)$ copies of $X$ in $\calX_{/V^{s}_{\leq n-1}}$, and that
$f$ can be identified with the identity map. The existence of a left homotopy inverse to $f$ is now obvious (choose any of the $(n+1)$-projections from $Y$ onto $X$).
\end{proof}

\begin{lemma}\label{bball2}
Let $\calX$ be an $\infty$-topos, and let $f_{\bigdot}: U_{\bigdot} \rightarrow V_{\bigdot}$ be a natural transformation between simplicial objects of $\calX$. Suppose that, for each
$k \leq n$, the map $f_{k}: U_k \rightarrow V_{k}$ is an equivalence. Then the induced map
$|f_{\bigdot}| : | U_{\bigdot} | \rightarrow | V_{\bigdot} |$ of colimits is $n$-connective.
\end{lemma}

\begin{proof}
Choose a left exact localization functor $L: \calP(\calC) \rightarrow \calX$. Without loss of generality, we may suppose that $f_{\bigdot} = L \circ \overline{f}_{\bigdot}$, where
$\overline{f}_{\bigdot}: \overline{U}_{\bigdot} \rightarrow \overline{V}_{\bigdot}$ is
a transformation between simplicial objects of $\calP(\calC)$, where $\overline{f}_{k}$ is an equivalence for $k \leq n$. Since $L$ preserves colimits and $n$-connectivity (Proposition \ref{inftychange}), it will suffice to prove that $|f_{\bigdot}|$ is $n$-connective. Using Propositions \ref{goober} and \ref{compattrunc}, we see that $|f_{\bigdot}|$ is $n$-connective if and only if, for each object $C \in \calC$, the induced morphism in $\SSet$ is $n$-connective. In other words, we may assume without loss of generality that $\calX = \SSet$.

According to Proposition \ref{gumby444}, we may assume that
$f_{\bigdot}$ is obtained by taking the simplicial nerve of a map $f'_{\bigdot}: U'_{\bigdot} \rightarrow V'_{\bigdot}$ between simplicial objects in the ordinary category $\Kan$.
Without loss of generality, we may suppose that
 $U'_{\bigdot}$ and $V'_{\bigdot}$ are projectively cofibrant (as diagrams in the model category $\sSet$). According to Theorem \ref{colimcomparee}, it will suffice to prove that the induced map from the (homotopy) colimit of $U'_{\bigdot}$ to the (homotopy) colimit of $V'_{\bigdot}$ has $n$-connective homotopy fibers, which follows from classical homotopy theory.
\end{proof}

\begin{lemma}\label{fierminus}
Let $\calX$ be an $\infty$-topos, let $U_{\bigdot}$ be a hypercovering of $\calX$.
Then the colimit $| U_{\bigdot} |$ is $\infty$-connective.
\end{lemma}

\begin{proof}
We will prove that $\theta$ is $n$-connective for every $n \geq 0$. Let
$V_{\bigdot} = \cosk_{n+1} U_{\bigdot}$, and let
$u: U_{\bigdot} \rightarrow V_{\bigdot}$ be the adjunction map.
Lemma \ref{bball2} asserts that the induced map $|U_{\bigdot}| \rightarrow |V_{\bigdot}|$
is $n$-connective, and Lemma \ref{bball1} asserts that $| V_{\bigdot} |$ is a final object of $\calX$. It follows that $| U_{\bigdot} | \in \calX$ is $n$-connective, as desired.
\end{proof}

The preceding results lead to an easy characterization of the class of hypercomplete $\infty$-topoi:

\begin{theorem}\label{surp}
Let $\calX$ be an $\infty$-topos. The following conditions are equivalent:
\begin{itemize}
\item[$(1)$] For every $X \in \calX$, every hypercovering $U_{\bigdot}$ of
$\calX_{/X}$ is effective.
\item[$(2)$] The $\infty$-topos $\calX$ is hypercomplete.
\end{itemize}
\end{theorem}

\begin{proof}
Suppose that $(1)$ is satisfied. Let $f: U \rightarrow X$ be an $\infty$-connective morphism in $\calX$, and let $f_{\bigdot}$ be the constant simplicial object of $\calX_{/X}$ with value $f$.
According to Lemma \ref{fier0}, $f$ is a hypercovering of $\calX_{/X}$. Invoking $(1)$, we conclude that $f \simeq | f_{\bigdot} |$ is a final object of $\calX_{/X}$; in other words, $f$ is an equivalence. This proves that $(1) \Rightarrow (2)$.

Conversely, suppose that $\calX$ is hypercomplete. Let $X \in \calX$ be an object and
$U_{\bigdot}$ a hypercovering of $\calX_{/X}$. Then Lemma \ref{fierminus} implies that
$| U_{\bigdot} |$ is an $\infty$-connective object of $\calX_{/X}$. Since $\calX$ is hypercomplete, we conclude that $|U_{\bigdot}|$ is a final object of $\calX_{/X}$, so that $U_{\bigdot}$ is effective.
\end{proof}

\begin{corollary}[Dugger-Hollander-Isaksen \cite{hollander}, To\"{e}n-Vezzosi \cite{toen}]\label{charhyp}
Let $\calX$ be an $\infty$-topos. For each $X \in \calX$ and each hypercovering
$U_{\bigdot}$ of $\calX_{/X}$, let $|U_{\bigdot}|$ be the associated morphism of $\calX$
$(${}which has target $X${}$)$. 
Let $S$ denote the collection of all such morphisms $|U_{\bigdot}|$. Then
$\calX^{\hyp} = S^{-1} \calX$. In other words, an object of
$\calX$ is hypercomplete if and only if it is $S$-local.
\end{corollary}

\begin{remark}
One can generalize Corollary \ref{charhyp} as follows: let $L: \calX \rightarrow \calY$ be an arbitrary left exact localization of $\infty$-topoi, and let $S$ be the collection of all morphisms of the form
$|U_{\bigdot}|$, where $U_{\bigdot}$ is a simplicial object of $\calX_{/X}$ such that
$L \circ U_{\bigdot}$ is an effective hypercovering of $\calY_{/LX}$. 
Then $L$ induces an equivalence $S^{-1} \calX \rightarrow \calY$.

It follows that every $\infty$-topos can be obtained by starting with an $\infty$-category of presheaves $\calP(\calC)$, selecting a collection of augmented simplicial objects
$U^{+}_{\bigdot}$, and inverting the corresponding
maps $|U_{\bigdot}| \rightarrow U_{-1}$. The specification of the desired class of augmented simplicial objects can be viewed as a kind of ``generalized topology'' on $\calC$, in which one specifies not only the covering sieves but also the collection of hypercoverings which are to become effective after localization. It seems plausible that this notion of topology can be described more directly in terms of the $\infty$-category $\calC$, but we will not pursue the matter further.
\end{remark}

\subsection{Descent versus Hyperdescent}\label{versus}

Let $X$ be a topological space, and let $\calU(X)$ denote the category of open subsets of $X$.\index{not}{Ucal(X)@$\calU(X)$} The category $\calU(X)$ is equipped with a Grothendieck topology in which the covering sieves
on $U$ are those sieves $\{ U_{\alpha} \subseteq U \}$ such that $U = \bigcup_{\alpha} U_{\alpha}$. We may therefore consider the $\infty$-topos $\Shv(\Nerve(\calU(X)))$, which we will 
call the $\infty$-topos of {\it sheaves on $X$}\index{gen}{$\infty$-topos!of sheaves on a topological space} and denote by $\Shv(X)$\index{not}{ShvX@$\Shv(X)$}
In \S \ref{hyperstacks} we discussed an alternative theory of sheaves on $X$, which can be obtained either through Jardine's local model structure on the category of simplicial presheaves or by passing to the hypercompletion $\Shv(X)^{\hyp}$ of $\Shv(X)$. According to Theorem \ref{surp}, $\Shv(X)^{\hyp}$ is distinguished from $\Shv(X)$ in that objects of $\Shv(X)^{\hyp}$ are required to satisfy a descent condition for arbitrary hypercoverings of $X$, while objects of $\Shv(X)$ are required to satisfy a descent condition only for ordinary coverings.

\begin{warning}
The notation $\Shv(X)$ will always represent the $\infty$-category of $\SSet$-valued
sheaves on $X$, rather than the ordinary category of set-valued sheaves. If we need to
indicate the latter, we will denote it by $\Shv_{\Set}(X)$.
\end{warning}

The $\infty$-topos $\Shv(X)^{\hyp}$ seems to have received more attention than $\Shv(X)$ in the literature (though there is some discussion of $\Shv(X)$ in \cite{hollander} and \cite{toen}).
We would like to make the case that for most purposes, $\Shv(X)$ has better properties.
A large part of \S \ref{chap7} will be devoted to justifying some of the claims made below.

\begin{itemize}

\item[$(1)$] In \S \ref{nlocalic}, we saw that the construction
$$ X \mapsto \Shv(X)$$
could be interpreted as a {\em right} adjoint to the functor which associates to every
$\infty$-topos $\calY$ the underlying locale of subobjects of the final object of $\calY$.
In other words, $\Shv(X)$ occupies a universal position among $\infty$-topoi which are related to the original space $X$.

\item[$(2)$] Suppose given a Cartesian square
$$ \xymatrix{ X' \ar[r]^{\psi'} \ar[d]^{\pi'} & X \ar[d]^{\pi} \\
S' \ar[r]^{\psi} & S }$$
in the category of locally compact topological spaces.
In classical sheaf theory, there is a {\it base change} transformation
$$\psi^{\ast} \pi_{\ast} \rightarrow \pi'_{\ast} \psi'^{\ast}$$
of functors between the derived categories of (left-bounded) complexes of (abelian) sheaves
on $X$ and on $S'$. The proper base change theorem
asserts that this transformation is an equivalence whenever the
map $\pi$ is proper.\index{gen}{proper!base change theorem}

The functors $\psi^{\ast}$, ${\psi'}^{\ast}$, $\pi_{\ast}$, and $\pi'_{\ast}$
can be defined on the $\infty$-topoi $\Shv(X), \Shv(X'), \Shv(S),$ and $\Shv(S')$, and on their hypercompletions. Moreover, one has a base change map
$$\psi^{\ast} \pi_{\ast} \rightarrow \pi'^{\ast} {\psi'}^{\ast}$$
in this nonabelian situation as well. 

It is natural to ask if the
base change transformation is an equivalence when $\pi$ is proper.
It turns out that this is {\em false} if we work with hypercomplete $\infty$-topoi.
Let us sketch a counterexample:

\begin{counterexample}\label{thrust}\index{gen}{Hilbert cube}
Let $Q$ denote the Hilbert cube $[0,1] \times [0,1] \times
\ldots$. For each $i$, we let $Q_i \simeq Q$ denote ``all but the
first $i$'' factors of $Q$, so that $Q = [0,1]^i \times Q_i$.

We construct a sheaf of spaces $\calF$ on $X = Q \times [0,1]$ as follows.
Begin with the empty stack. Adjoin to it two sections, defined
over the open sets $[0,1) \times Q_1 \times [0,1)$ and $(0,1]
\times Q_1 \times [0,1)$. These sections both restrict to give
sections of $\calF$ over the open set $(0,1) \times Q_1 \times
[0,1)$. We next adjoin paths between these sections, defined over
the smaller open sets $(0,1) \times [0,1) \times Q_2 \times
[0,\frac{1}{2})$ and $(0,1) \times (0,1] \times Q_2 \times [0,
\frac{1}{2})$. These paths are both defined on the smaller open
set $(0,1) \times (0,1) \times Q_2 \times [0, \frac{1}{2})$, so we
next adjoin two homotopies between these paths over the open sets
$(0,1) \times (0,1) \times [0,1) \times Q_3 \times [0,
\frac{1}{3})$ and $(0,1) \times (0,1) \times (0,1] \times Q_3
\times [0, \frac{1}{3})$. Continuing in this way, we obtain a
sheaf $\calF$. On the closed subset $Q \times \{0\} \subset X$,
the sheaf $\calF$ is $\infty$-connective by construction, and
therefore its hypercompletion admits a global section.
However, the hypercompletion of $\calF$ does not admit a
global section in any neighborhood of $Q \times \{0\}$, since such
a neighborhood must contain $Q \times [ 0, \frac{1}{n})$ for $n
\gg 0$ and the higher homotopies required for the construction of
a section are eventually not globally defined.
\end{counterexample}

However, in the case where $\pi$ is a proper map, the base-change map
$$\psi^{\ast} \pi_{\ast} \rightarrow \pi'_{\ast} \psi'^{\ast}$$
{\em is} an equivalence of functors from $\Shv(X)$ to $\Shv(S')$. One may regard this
fact as a nonabelian generalization of the classical proper-base change theorem.
We refer the reader to \S \ref{chap7sec3} for a precise statement and proof.

\begin{remark}
A similar issue arises in classical sheaf theory if one chooses to
work with unbounded complexes. In \cite{spaltenstein},
Spaltenstein defines a derived category of unbounded complexes of
sheaves on $X$, where $X$ is a topological space. His definition
forces all quasi-isomorphisms to become invertible, which is
analogous to procedure of obtaining $\calX^{\hyp}$ from $\calX$ by
inverting the $\infty$-connective morphisms. Spaltenstein's work
shows that one can extend the {\it definitions} of all of the
basic objects and functors. However, it turns out that the {\it
theorems} do not all extend: in particular, one does not have the
proper base change theorem in Spaltenstein's setting
(Counterexample \ref{thrust} can be adapted to the setting
of complexes of abelian sheaves). The problem may be rectified by
imposing weaker descent conditions, which do not invert all
quasi-isomorphisms; we will give a more detailed discussion in \cite{DAG}.
\end{remark}

\item[$(3)$] The $\infty$-topos $\Shv(X)$ often has better finiteness properties
than $\Shv(X)^{\hyp}$. Recall that a topological space $X$ is {\it coherent}
if the collection of compact open subsets of $X$ is stable under finite intersections, and forms a basis for the topology of $X$.\index{gen}{coherent topological space}

\begin{proposition}\label{cohcomp}
Let $X$ be a coherent topological space. Then the $\infty$-category
$\Shv(X)$ is compactly generated: that is, $\Shv(X)$ is generated under filtered colimits
by its compact objects.\index{gen}{compactly generated}
\end{proposition}

\begin{proof}
Let $\calU_c(X)$ be the partially ordered set of {\em compact} open subsets of $X$, let
$\calP_c(X) = \calP( \Nerve(\calU_c(X)) )$, and let $\Shv_c(X)$ be the full subcategory of 
$\calP_c(X)$ spanned by those presheaves $\calF$ with the following properties:
\begin{itemize}
\item[$(1)$] The object $\calF(\emptyset) \in \calC$ is final.
\item[$(2)$] For every pair of compact open sets $U, V \subseteq X$, the associated diagram
$$ \xymatrix{ \calF( U \cap V) \ar[r] \ar[d] & \calF(U) \ar[d] \\
\calF(V) \ar[r] & \calF(U \cup V) }$$
is a pullback.
\end{itemize}

In \S \ref{cohthm}, we will prove that the restriction functor 
$\Shv(X) \rightarrow \Shv_c(X)$ is an equivalence of $\infty$-categories (Theorem \ref{surm}). 
It will therefore suffice to prove that $\Shv_c(X)$ is compactly generated. 

Using Lemmas \ref{stur1}, \ref{stur2}, and \ref{stur3}, we conclude that $\Shv_c(X)$ is an accessible localization of $\calP_c(X)$. 
Let $X$ be a compact object of $\calP_{c}(X)$. We observe that $X$ and $LX$ co-represent the same functor on $\Shv_{c}(X)$. Proposition \ref{frent} implies that the subcategory $\Shv_c(X) \subseteq \calP_c(X)$ is stable under filtered colimits in $\calP_c(X)$. It follows that
$LX$ is a compact object of $\Shv_0(X)$. Since $\calP_{c}(X)$ is generated under filtered colimits by its compact objects (Proposition \ref{precst}), we conclude that $\Shv_{c}(X)$ has the same property.
\end{proof}

\begin{remark}
In the situation of Proposition \ref{cohcomp}, we can give an explicit description of the class of compact objects of $\Shv(X)$. Namely, they are precisely those sheaves $\calF$ whose stalks are compact objects of $\SSet$, and which are {\em locally constant} along a suitable stratification of $X$. In other words, we may interpret Proposition \ref{cohcomp} as asserting that there is a good theory of {\em constructible} sheaves on $X$.
\end{remark}

It is not possible to replace to $\Shv(X)$ by $\Shv(X)^{\hyp}$ in the statement of Proposition \ref{cohcomp}.

\begin{counterexample}\label{trust}
Let $S = \{ x,y,z \}$ be a topological space consisting of three points, with topology generated
by the open subsets $S^{+} = \{x,y\} \subset S$ and $S^{-} = \{x,z\} \subset S$.
Let $X = S \times S \times \ldots$ be a product of infinitely many copies of $S$.
Then $X$ is a coherent topological space. We will show that the global sections functor
$\Gamma: \Shv(X)^{\hyp} \rightarrow \SSet$ does not commute with filtered colimits, so that
the final object of $\Shv(X)^{\hyp}$ is not compact. A more elaborate version of the same argument shows that $\Shv(X)^{\hyp}$ contains no compact objects other than its initial object.

To show that $\Gamma$ does not commute with filtered colimits, we use a variant on the
construction of Counterexample \ref{thrust}. We define a sequence of
sheaves $$\calF_0 \rightarrow \calF_1 \rightarrow \ldots $$
as follows. Let $\calF_0$ be generated by sections
$$ \eta^0_{+} \in \calF(S^{+} \times S \times \ldots) $$
$$ \eta^0_{-} \in \calF(S^{-} \times S \times \ldots). $$
Let $\calF_1$ be the sheaf obtained from $\calF_0$ by adjoining
paths 
$$ \eta^{1}_{+}: \Delta^1 \rightarrow \calF( \{x\} \times S^{+} \times S \times \ldots )$$
$$ \eta^{1}_{-}: \Delta^1 \rightarrow \calF( \{x\} \times S^{-} \times S \times \ldots )$$
from $\eta^0_{+}$ to $\eta^{0}_{-}$.
Similarly, let $\calF_2$ be obtained from $\calF_1$ by adjoining homotopies
$$ \eta^2_{+}: (\Delta^1)^2 \rightarrow \calF( \{x\} \times \{x\} \times S^{+} \times S \times \ldots )$$
$$ \eta^2_{-}: (\Delta^1)^2 \rightarrow \calF( \{x \} \times \{x\} \times S^{-} \times S \times \ldots ),$$
from $\eta^1_{+}$ to $\eta^{1}_{-}$. Continuing this procedure, we obtain a
sequence of sheaves $$\calF_0 \rightarrow \calF_1 \rightarrow \calF_2 \rightarrow \ldots $$
whose colimit $\calF_{\infty} \in \Shv(X)^{\hyp}$ admits a section (since we allow descent
with respect to hypercoverings). However, none of the individual sheaves $\calF_n$ admits a global section.
\end{counterexample}

\begin{remark}
The analogue of Proposition \ref{cohcomp} fails, in general, if we replace the coherent topological space $X$ by a coherent topos. For example, we cannot take $X$ to be the topos of \'{e}tale sheaves on an algebraic variety. However, it turns out that the analogue Proposition \ref{cohcomp} {\em is} true for the topos of {\em Nisnevich} sheaves on an algebraic variety; we refer the reader to \cite{DAG} for details.
\end{remark}

\begin{remark}\label{notenough}
A {\it point}\index{gen}{point!of an $\infty$-topos} of an $\infty$-topos $\calX$ is a geometric morphism $p_{\ast}: \SSet \rightarrow \calX$, where $\SSet$ denotes the $\infty$-category of spaces (which is a final object of $\RGeom$, in virtue of Proposition \ref{spacefinall}). We say that $\calX$ has {\it enough points}\index{gen}{enough points} if, for every morphism $f: X \rightarrow Y$ in $\calX$ having the property that $p^{\ast}(f)$ is an equivalence for {\em every} point $p$ of $\calX$, $f$ is itself an equivalence in $\calX$. If $f$ is $\infty$-connective, then every stalk $p^{\ast}(f)$ is $\infty$-connective, hence an equivalence by Whitehead's theorem. Consequently, if $\calX$ has enough points, then it is hypercomplete.

In classical topos theory, Deligne's version of the G\"{o}del completeness theorem (see \cite{where}) asserts that every coherent topos has enough points.  Counterexample \ref{trust} shows that there exist coherent topological spaces with
$\Shv(X)^{\hyp} \neq \Shv(X)$, so that $\Shv(X)$ does not necessarily have enough points. 
Consequently, Deligne's theorem does not hold in the $\infty$-categorical context.
\end{remark}

\item[$(4)$] Let $k$ be a field, and let $\calC$ denote the category of chain complexes of $k$-vector spaces. Via the Dold-Kan correspondence we may regard $\calC$ as a simplicial category. We let $\Mod(k) = \sNerve(\calC)$ denote the simplicial nerve. We will refer to $\Mod(k)$ as the {\it $\infty$-category of $k$-modules}; it is a presentable $\infty$-category which we will discuss at greater length in \cite{DAG}.\index{not}{Modk@$\Mod(k)$}

Let $X$ be a compact topological space, and choose a functorial injective resolution
$$ \calF \rightarrow I^0( \calF) \rightarrow I^1( \calF ) \rightarrow \ldots $$
on the category of sheaves $\calF$ of $k$-vector spaces on $X$. For every open subset
$U$ on $X$, we let $k_U$ denote the constant sheaf on $U$ with value $k$, extended by zero to $X$. Let $\HH^{BM}(U) = \Gamma(X, I^{\bigdot}( k_U) )^{\vee}$, the {\it dual} of the complex
of global sections of the injective resolution $I^{\bigdot}(k_U)$. Then $\HH^{BM}(U)$ is a 
complex of $k$-vector spaces, whose homologies are precisely the Borel-Moore homology of $U$ with coefficients in $k$ (in other words, they are the dual spaces of the compactly supported cohomology groups of $U$). The assignment
$$ U \mapsto \HH^{BM}(U)$$
determines a presheaf on $X$ with values in the $\infty$-category $\Mod(k)$.

In view of the existence of excision exact sequences for Borel-Moore homology, it is natural to suppose that $\HH^{BM}(U)$ is actually a {\it sheaf} on $X$ with values in $\Mod(k)$. This is true provided that the notion of ``sheaf'' is suitably interpreted: namely, 
$\HH^{BM}$ extends (in an essentially unique fashion) to a colimit-preserving functor
$$ \phi: \Shv(X) \rightarrow \Mod(k)^{op}.$$
(In other words, the functor $U \mapsto \HH^{BM}(U)$ determines a $\Mod(k)$-valued sheaf
on $X$ in the sense of Definition \ref{valsheaf}.)
However, the sheaf $\HH^{BM}$ is not necessarily hypercomplete, in the sense that $\phi$ does not necessarily factor through $\Shv(X)^{\hyp}$.

\begin{counterexample}\label{spacerk}
There exists a compact Hausdorff space $X$ and a hypercovering $U_{\bigdot}$ of $X$ such that the natural map $\HH^{BM}(X) \rightarrow \varprojlim \HH^{BM}(U_{\bigdot})$ is not an equivalence. Let
$X$ be the Hilbert cube $Q = [0,1] \times [0,1] \times \ldots$ (more generally, we could take $X$ to be any nonempty Hilbert cube manifold). It is proven in \cite{chapman} that every point of $X$ has arbitrarily small neighborhoods which are homeomorphic to $Q \times [0,1)$. Consequently, 
there exists a hypercovering $U_{\bigdot}$ of $X$, where each $U_n$ is a disjoint union
of open subsets of $X$ homeomorphic to $Q \times [0,1)$. The Borel-Moore homology of
every $U_n$ vanishes; consequently, $\varprojlim \HH^{BM}(U_{\bigdot})$ is zero. However,
the (degree zero) Borel-Moore homology of $X$ itself does not vanish, since $X$ is nonempty and compact.
\end{counterexample}

Borel-Moore homology is a very useful tool in the study of a locally compact space $X$, and its descent properties (in other words, the existence of various Mayer-Vietoris sequences) is very naturally encoded in the statement that $\HH^{BM}$ is a {\it $k$-module in the $\infty$-topos $\Shv(X)$} (in other words, a sheaf on $X$ with values in $\Mod(k)$); however, this $k$-module generally does not lie in $\Shv(X)^{\hyp}$. We see from this example that non-hypercomplete sheaves (with values in $\Mod(k)$, in this case) on $X$ often arise naturally in the study of infinite-dimensional spaces.

\item[$(5)$] Let $X$ be a topological space, and $f: \Shv(X) \rightarrow \Shv(\ast) \simeq \SSet$ the geometric morphism induced by the projection $X \rightarrow \ast$. Let $K$ be a Kan complex, regarded as an object of $\SSet$. Then $\pi_0 f_{\ast} f^{\ast} K$
is a natural definition of the sheaf cohomology of
$X$ with coefficients in $K$. If $X$ is paracompact, then the cohomology set defined above is naturally isomorphic to the set $[X,|K|]$ of homotopy classes of maps from $X$ into the geometric realization $|K|$; we will give a proof of this statement in \S \ref{paracompactness}.
The analogous statement fails if we replace $\Shv(X)$ by $\Shv(X)^{\hyp}$.

\item[$(6)$] Let $X$ be a topological space. Combining Remark \ref{pointeddesc} with Proposition \ref{suga}, we deduce that $\Shv(X)^{\hyp}$ has enough points, and that $\Shv(X)^{\hyp} = \Shv(X)$ if and only if $\Shv(X)$ has enough points. The possible failure of Whitehead's theorem in $\Shv(X)$ may be viewed either as a bug or a feature. The existence of enough points for $\Shv(X)$ is extremely convenient; it allows us to reduce many statements about the $\infty$-topos $\Shv(X)$
to statements about the $\infty$-topos $\SSet$ of spaces, where we can apply classical homotopy theory. On the other hand, if $\Shv(X)$ does {\em not} have enough points, then there is the possibility that it detects certain global phenomena 
which cannot be properly understood by restricting to points. Let us consider an example from geometric topology. A map $f: X \rightarrow Y$ of compact metric spaces is called {\it cell-like} if each fiber $X_{y} = X \times_{Y} \{y\}$ has trivial shape (see \cite{cellmap})\index{gen}{cell-like}. This notion has good formal properties provided that we restrict our attention to metric spaces which are {\em absolute neighborhood retracts}. In the general case, the theory of cell-like maps can be badly behaved: for example, a composition of cell-like maps need not be cell-like. 

The language of $\infty$-topoi provides a convenient formalism for discussing the problem. 
In \S \ref{celluj}, we will introduce the notion of a {\em cell-like} morphism $p_{\ast}: \calX \rightarrow \calY$ between $\infty$-topoi. By definition, $p_{\ast}$ is cell-like if it is proper and if the unit map $u: \calF \rightarrow p_{\ast} p^{\ast} \calF$ is an equivalence for each $\calF \in \calY$. A cell-like map $p: X \rightarrow Y$ of compact metric spaces {\em need not} give rise to a cell-like morphism $p_{\ast}: \Shv(X) \rightarrow \Shv(Y)$. The hypothesis
that each fiber $X_{y}$ has trivial shape ensures that the unit $u: \calF \rightarrow p_{\ast} p^{\ast} \calF$ is an equivalence after passing to stalks at each point $y \in Y$. This implies only that
$u$ is $\infty$-connective, and in general $u$ need not be an equivalence.

\begin{remark}
It is tempting to try to evade the problem described above by working instead with the hypercomplete $\infty$-topoi $\Shv(X)^{\hyp}$ and $\Shv(Y)^{\hyp}$. In this case, we {\em can} test whether or not $u: \calF \rightarrow p_{\ast} p^{\ast} \calF$ is an equivalence by passing to stalks. However, since the proper base change theorem does not hold in the hypercomplete context, the stalk $(p_{\ast} p^{\ast} \calF)_{y}$ is not generally equivalent to the global sections of
$p^{\ast} \calF | X_{y}$. Thus, we still encounter difficulties if we want to deduce global consequences from information about the individual fibers $X_{y}$. 
\end{remark}

\item[$(7)$] The counterexamples described in this section have one feature in common: the underlying space $X$ is infinite-dimensional. In fact, this is necessary: if the space $X$ is finite-dimensional (in a suitable sense), then the $\infty$-topos $\Shv(X)$ is hypercomplete (Corollary \ref{fdfd}). This finite-dimensionality condition on $X$ is satisfied in many of the situations to which the theory of simplicial presheaves is commonly applied, such as the Nisnevich
topology on a scheme of finite Krull dimension.

\end{itemize}
