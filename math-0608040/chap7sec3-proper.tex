\section{The Proper Base Change Theorem}\label{chap7sec3}

\setcounter{theorem}{0}

Let 
$$ \xymatrix{ X' \ar[r]^{q'} \ar[d]^{p'} & X \ar[d]^{p} \\
Y' \ar[r]^{q} & Y }$$
be a pullback diagram in the category of locally compact Hausdorff spaces. One has a natural isomorphism of pushforward functors $$  q_{\ast} p'_{\ast} \simeq p_{\ast} q'_{\ast}$$
from the category of sheaves of sets on $Y$ to the category of sheaves of sets on $X'$. This isomorphism induces a natural transformation
$$ \eta: q^{\ast} p_{\ast} \rightarrow p'_{\ast} {q'}^{\ast}.$$
If $p$ (and therefore also $p'$) is a proper map, then $\eta$ is an isomorphism: this is a simple version of the classical {\it proper base change theorem}.\index{gen}{proper base change theorem!for sheaves of sets}

The purpose of this section is to generalize the above result, allowing sheaves which take values in the $\infty$-category $\SSet$ of spaces rather than in the ordinary category of sets. Our generalization can be viewed as a proper base change theorem for nonabelian cohomology. 

We will begin in \S \ref{propertopoi} by defining the notion of a {\em proper morphism} of $\infty$-topoi. Roughly speaking, a geometric morphism $\pi_{\ast}: \calX \rightarrow \calY$ of $\infty$-topoi is proper if and only if it satisfies the conclusion of the proper base change theorem. Using this language, our job is to prove that a proper map of topological spaces $p: X \rightarrow Y$ induces a proper morphism $p_{\ast}: \Shv(X) \rightarrow \Shv(Y)$ of $\infty$-topoi. We will outline the proof of this result in \S \ref{propertopoi} by reducing to two special cases: the case where $p$ is a closed embedding, and the case where $Y$ is a point. We will treat the first case in \S \ref{closedsub},
after introducing a general theory of {\em closed immersions} of $\infty$-topoi. This allows us to reduce to the case where $Y$ is a point and $X$ a compact Hausdorff space. Our approach is now in two parts:

\begin{itemize}
\item[$(1)$] In \S \ref{products}, we will show that we can identify the $\infty$-category 
$\Shv(X') = \Shv(X \times Y')$ with an $\infty$-category of sheaves on $X$, taking values in 
$\Shv(Y')$.
\item[$(2)$] In \S \ref{properproper}, we give an analysis of the category of sheaves
on a compact Hausdorff space $X$, taking values in a general $\infty$-category
$\calC$. Combining this analysis with $(1)$, we will deduce the desired base change theorem.
\end{itemize}

The techniques used in \S \ref{properproper} to analyze $\Shv(X)$ can be applied also in the (easier) setting of coherent topological spaces, as we explain in \S \ref{cohthm}. Finally, we conclude in \S \ref{celluj} by reformulating the classical theory of {\em cell-like} maps in the language of $\infty$-topoi.

\subsection{Proper Maps of $\infty$-Topoi}\label{propertopoi}

In this section, we introduce the notion of a {\em proper} geometric morphism between $\infty$-topoi. Here we follow the ideas of \cite{moerdijk}, and turn the conclusion of the proper base change theorem into a definition. First, we require a bit of terminology.

Suppose given a diagram of categories and functors
$$ \xymatrix{ \calC' \ar[r]^{q'_{\ast}} \ar[d]^{p'_{\ast}} & \calD' \ar[d]^{p_{\ast}} \\
\calC \ar[r]^{q_{\ast}} & \calD }$$
which commutes up to a specified isomorphism 
$\eta: p_{\ast} q'_{\ast} \rightarrow
q_{\ast} p'_{\ast}$. Suppose furthermore that the functors $q_{\ast}$ and $q'_{\ast}$ admit
left adjoints, which we will denote by $q^{\ast}$ and ${q'}^{\ast}$. Consider the composition
$$ \gamma: q^{\ast} p_{\ast} \stackrel{u}{\rightarrow} q^{\ast} p_{\ast} q'_{\ast} {q'}^{\ast}
\stackrel{\eta}{\rightarrow} q^{\ast} q_{\ast} p'_{\ast} {q'}^{\ast} \stackrel{v}{\rightarrow}
p'_{\ast} {q'}^{\ast},$$
where $u$ denotes a unit for the adjunction $( {q'}^{\ast}, q'_{\ast})$ and $v$ a counit for the adjunction $( q^{\ast}, q_{\ast})$. We will refer to $\gamma$ as the {\it push-pull} transformation
associated to the above diagram.\index{gen}{push-pull transformation}

\begin{definition}
A diagram of categories $$ \xymatrix{ \calC' \ar[r]^{q'_{\ast}} \ar[d]^{p'_{\ast}} & \calD' \ar[d]^{p_{\ast}} \\ \calC \ar[r]^{q_{\ast}} & \calD }$$ which commutes up to specified isomorphism is
{\it left adjointable} if the functors $q_{\ast}$ and $q'_{\ast}$ admit left adjoints
$q^{\ast}$ and ${q'}^{\ast}$, and the associated push-pull transformation
$$ \gamma: q^{\ast} p_{\ast} \rightarrow p'_{\ast} {q'}^{\ast}$$
is an isomorphism of functors.\index{gen}{left adjointable}
\end{definition}

\begin{definition}
A diagram of $\infty$-categories
$$ \xymatrix{ \calC' \ar[r]^{q'_{\ast}} \ar[d]^{p'_{\ast}} & \calD' \ar[d]^{p_{\ast}} \\ \calC \ar[r]^{q_{\ast}} & \calD }$$ which commutes up to (specified) homotopy is {\it left adjointable} if the associated
diagram of homotopy categories is left adjointable.
\end{definition}

\begin{remark}\label{toadcatcher}
Suppose given a diagram of simplicial sets
$$ \calM' \stackrel{P}{\rightarrow} \calM \stackrel{f}{\rightarrow} \Delta^1,$$
where both $f$ and $f \circ P$ are Cartesian fibrations. Then we may view
$\calM$ as a correspondence from $\calD = f^{-1} \{0\}$ to $\calC = f^{-1} \{1\}$, associated
to some functor $q_{\ast}: \calC \rightarrow \calD$. Similarly, we may view
$\calM'$ as a correspondence from $\calD' = (f \circ P)^{-1} \{0\}$ to
$\calC' = (f \circ P)^{-1} \{1\}$, associated to some functor $q'_{\ast}: \calC' \rightarrow \calD'$.
The map $P$ determines functors $p'_{\ast}: \calC' \rightarrow \calC$, $q'_{\ast}: \calD' \rightarrow \calD$, and (up to homotopy) a natural transformation $\alpha: p_{\ast} q'_{\ast} \rightarrow
q_{\ast} p'_{\ast}$, which is an equivalence if and only if the map $P$ carries
$(f \circ P)$-Cartesian edges of $\calM'$ to $f$-Cartesian edges of $\calM$. In this case,
we obtain a diagram of homotopy categories
$$ \xymatrix{ \h{\calC'} \ar[r]^{q'_{\ast}} \ar[d]^{p'_{\ast}} & \h{\calD'} \ar[d]^{p_{\ast}} \\ \h{\calC} \ar[r]^{q_{\ast}} & \h{\calD} }$$
which commutes up to canonical isomorphism. 

Now suppose that the functors $q_{\ast}$ and $q'_{\ast}$ admit left adjoints, which we will denote by $q^{\ast}$ and ${q'}^{\ast}$, respectively. Then the maps $f$ and $f \circ P$ are coCartesian fibrations. Moreover, the associated push-pull transformation can be described as follows. 
Choose an object $D' \in \calD'$, and a $(f \circ P)$-coCartesian morphism
$\phi: D' \rightarrow C'$, where $C' \in \calC$. Let $D = P(D')$, and choose an $f$-coCartesian morphism $\psi: D \rightarrow C$ in $\calM$, where $C \in \calC$. Using the fact that
$\psi$ is $f$-coCartesian, we can choose a $2$-simplex in $\calM$ depicted as follows:
$$ \xymatrix{ & C \ar[dr]^{\theta} & \\
D \ar[ur]^{\psi} \ar[rr]^{P(\phi)} & & P(C'). }$$ 
We may then identify $C$ with $q^{\ast} p_{\ast} D'$, $P(C')$ with $p'_{\ast} {q'}^{\ast} D'$, and
$\theta$ with the value of the push-pull transformation $q^{\ast} p_{\ast} \rightarrow p'_{\ast} {q'}^{\ast} D'$ on the object $D' \in \calD'$. The morphism $\theta$ is an equivalence if and only if $P(\phi)$ is $f$-coCartesian. Consequently, we deduce that the original diagram
$$ \xymatrix{ \h{\calC'} \ar[r]^{q'_{\ast}} \ar[d]^{p'_{\ast}} & \h{\calD'} \ar[d]^{p_{\ast}} \\ \h{\calC} \ar[r]^{q_{\ast}} & \h{\calD} }$$
is left adjointable if and only if $P$ carries $(f \circ P)$-coCartesian edges to $f$-coCartesian edges. We will make use of this criterion in \S \ref{properproper}.
\end{remark}

\begin{definition}\label{smurk}\index{gen}{proper!morphism of $\infty$-topoi}
Let $p_{\ast}: \calX \rightarrow \calY$ be a geometric morphism of $\infty$-topoi. We will say that
$p_{\ast}$ is {\em proper} if the following condition is satisfied:
\begin{itemize}
%\item[$(1)$] For every geometric morphism $q_{\ast}: \calY' \rightarrow \calY$, there exists
%a pullback diagram 
%$$ \xymatrix{ \calX' \ar[d] \ar[r] & \calX \ar[d]^{p_{\ast}} \\
%\calY' \ar[r]^{q_{\ast}} & \calY }$$ in the $\infty$-category $\RGeom$ of $\infty$-topoi.
\item[$(\ast)$] For every Cartesian rectangle 
$$ \xymatrix{ \calX'' \ar[d] \ar[r] & \calX' \ar[d] \ar[r] & \calX \ar[d]^{p_{\ast}} \\
\calY'' \ar[r] & \calY' \ar[r] & \calY }$$
of $\infty$-topoi, the left square is left adjointable.
\end{itemize}
\end{definition}

%\begin{remark}
%Condition $(1)$ of Definition \ref{smurk} is actually satisfied for {\em every} geometric
%morphism $p_{\ast}: \calX \rightarrow \calY$ of $\infty$-topoi. However, the construction
%of fiber products in general requires ideas that we have not developed in this book (higher category theory {\em within} an $\infty$-topos). We will discuss the case of ordinary products in \S \ref{products}; this will be sufficiently general for our applications.
%\end{remark}

\begin{remark}\label{swurk}
Let $\calX$ be an $\infty$-topos, and let $\calJ$ be a small $\infty$-category.
The diagonal functor $\delta: \calX \rightarrow \Fun(\calJ, \calX)$ preserves all (small) limits and colimits, by Proposition \ref{limiteval}, and therefore admits both a left adjoint $\delta_{!}$ and a right adjoint $\delta_{\ast}$. If $\calJ$ is filtered, then $\delta_{!}$ is left exact (Proposition \ref{frent}). Consequently, we have a diagram of geometric morphisms
$$ \calX \stackrel{ \delta }{\rightarrow} \Fun(\calJ,\calX) \stackrel{ \delta_{\ast} }{\rightarrow} \calX. $$

Now suppose that $p_{\ast}: \calX \rightarrow \calY$ is a proper geometric morphism of $\infty$-topoi. We obtain a rectangle
$$ \xymatrix{ \calX \ar[r] \ar[d]^{p_{\ast}} & \Fun(\calJ, \calX) \ar[d]^{p_{\ast}^{\calJ}} \ar[r] & \calX \ar[d] \\
\calY \ar[r] & \Fun(\calJ, \calY) \ar[r] & \calY }$$ 
which commutes up to (specified) homotopy. One can show that this is a Cartesian
rectangle in $\RGeom$, so that the square on the left is left adjointable.
Unwinding the definitions, we conclude that $p_{\ast}$ commutes with filtered colimits. 
Conversely, if
$p_{\ast}: \calX \rightarrow \calY$ is an arbitrary geometric morphism of $\infty$-topoi which
commutes with colimits indexed by {\em filtered $\calY$-stacks} (over each object of $\calY$), then $p_{\ast}$ is proper. To give a proof (or even a precise formulation) of this statement would require ideas from relative category theory which we will not develop in this book. We refer the reader to \cite{moerdijk}, where the analogous result is established for proper maps between ordinary topoi.
\end{remark}

The following properties of the class of proper morphisms follow immediately from Definition \ref{smurk}:

\begin{proposition}\label{properties}
\begin{itemize}
\item[$(1)$] Every equivalence of $\infty$-topoi is proper.
\item[$(2)$] If $p_{\ast}$ and $p'_{\ast}$ are equivalent geometric morphisms from
an $\infty$-topos $\calX$ to another $\infty$-topos $\calY$, then $p_{\ast}$ is proper if and only if $p'_{\ast}$ is proper.
\item[$(3)$] Let 
$$ \xymatrix{ \calX' \ar[d]^{p'_{\ast}} \ar[r] & \calX \ar[d]^{p_{\ast}} \\
\calY' \ar[r] & \calY }$$ be a pullback diagram of $\infty$-topoi. If
$p_{\ast}$ is proper, then so is $p'_{\ast}$.
\item[$(4)$] Let 
$$\calX \stackrel{p_{\ast}}{\rightarrow} \calY \stackrel{q_{\ast}}{\rightarrow} \calZ$$
be proper geometric morphisms between $\infty$-topoi. Then $q_{\ast} \circ p_{\ast}$ is a 
proper geometric morphism.
\end{itemize}
\end{proposition}

In order to relate Definition \ref{smurk} to the classical statement of the proper base change theorem, we need to understand the relationship between products in the category of topological spaces and products in the $\infty$-category of $\infty$-topoi. A basic result asserts that these are compatible, provided that a certain local compactness condition is met.

\begin{definition}\index{gen}{locally compact}\index{gen}{topological space!locally compact}
Let $X$ be a topological space which is not assumed to be Hausdorff. We say that $X$ is {\it locally compact} if, for every open set $U \subseteq X$ and every point $x \in U$, there
exists a (not necessarily closed) compact set $K \subseteq U$, where $K$ contains an open neighborhood of $x$.
\end{definition}

\begin{example}
If $X$ is Hausdorff space, then $X$ is locally compact in the sense defined above if and only if
$X$ is locally compact in the usual sense.
\end{example}

\begin{example}
Let $X$ be a topological space for which the compact open subsets of $X$ form a basis for the topology of $X$. Then $X$ is locally compact.
\end{example}

\begin{remark}
Local compactness of $X$ is precisely the condition which is needed for function
spaces $Y^X$, endowed with the compact-open topology, to represent the functor
$Z \mapsto \Hom( Z \times X, Y)$.
\end{remark}

\begin{proposition}\label{cartmun}
Let $X$ and $Y$ be topological spaces, and assume that $X$ is locally compact.
The diagram
$$ \xymatrix{ \Shv(X \times Y) \ar[r] \ar[d] & \Shv(X) \ar[d] \\
\Shv(Y) \ar[r] & \Shv(\ast) }$$
is a pullback square in the $\infty$-category $\RGeom$ of $\infty$-topoi.
\end{proposition}

\begin{proof}
Let $\calC \subseteq \RGeom$ be the full subcategory spanned by the $0$-localic $\infty$-topoi. Since $\calC$ is a localization of $\RGeom$, the inclusion $\calC \subseteq \RGeom$ preserves limits. It therefore suffices to prove that
$$ \xymatrix{ \Shv(X \times Y) \ar[r] \ar[d] & \Shv(X) \ar[d] \\
\Shv(Y) \ar[r] & \Shv(\ast) }$$
gives a pullback diagram in $\calC$. Note that $\calC^{op}$ is equivalent to the (nerve of the) ordinary category of locales. For each topological space $M$, let $\calU(M)$ denote the locale of open subsets of $M$. Let
$$ \calU(X) \stackrel{\psi_X}{\rightarrow} \calP \stackrel{\psi_{Y}}{\leftarrow} \calU(Y)$$
be a diagram which exhibits $\calP$ as a coproduct of $\calU(X)$ and $\calU(Y)$ in the category of locales, and let $\phi: \calP \rightarrow \calU(X \times Y)$ be the induced map. We wish to prove that $\phi$ is an isomorphism. This is a standard result in the theory of locales; we will include a proof for completeness.

Given open subsets $U \subseteq X$ and $V \subseteq Y$, let
$U \otimes V = ( \psi_{X} U) \cap (\psi_Y V) \in \calP$, so that
$\phi(U \otimes V) = U \times V \in \calU(X \times Y)$. We define a map
$\theta: \calU(X \times Y) \rightarrow \calP$ by the formula
$$ \theta(W) = \bigcup_{ U \times V \subseteq W } U \otimes V.$$
Since every open subset of $X \times Y$ can be written as a union of products $U \times V$, where $U$ is an open subset of $X$ and $V$ an open subset of $Y$, it is clear that
$\phi \circ \theta: \calU( X \times Y) \rightarrow \calU(X \times Y)$ is the identity.
To complete the proof, it will suffice to show that $\theta \circ \phi: \calP \rightarrow \calP$ is the identity. Every element of $\calP$ can be written as $\bigcup_{\alpha} U_{\alpha} \otimes V_{\alpha}$ for $U_{\alpha} \subseteq X$ and $V_{\alpha} \subseteq Y$ appropriately chosen.
We therefore wish to show that
$$ \bigcup_{ U \times V \subseteq \bigcup_{\alpha} U_{\alpha} \otimes V_{\alpha}} U \times V = \bigcup_{\alpha} U_{\alpha} \otimes V_{\alpha}.$$
It is clear that the right hand side is contained in the left hand side. The reverse containment is equivalent to the assertion that if $U \times V \subseteq \bigcup_{\alpha} U_{\alpha} \times V_{\alpha}$, then $U \otimes V \subseteq \bigcup_{\alpha} U_{\alpha} \otimes V_{\alpha}$.

We now invoke the local compactness of $X$.
Write $U = \bigcup K_{\beta}$, where each $K_{\beta}$ is a compact subset of $U$
and the interiors $\{ K^{\degree}_{\beta} \}$ cover $U$.
Then $U \otimes V = \bigcup_{\beta} K^{\degree}_{\beta} \otimes V$; it therefore suffices to prove
that $K^{\degree}_{\beta} \otimes V \subseteq \bigcup_{\alpha} U_{\alpha} \otimes V_{\alpha}$.
Let $v$ be a point of $V$. Then $K_{\beta} \times \{v\}$ is a compact subset of
$\bigcup_{ \alpha} U_{\alpha} \times V_{\alpha}$. Consequently, there exists a finite
set of indices $\{ \alpha_1, \ldots, \alpha_n \}$ such that $v \in V_{v,\beta} = V_{\alpha_1} \cap \ldots \cap V_{\alpha_n}$ and $K_{\beta} \subseteq U_{\alpha_1} \cup \ldots \cup U_{\alpha_n}$. 
It follows that $K^{\degree}_{\beta} \otimes V_{v,\beta} \subseteq \bigcup_{\alpha} U_{\alpha} \otimes V_{\alpha}$. Taking a union over all $v \in V$, we deduce the desired result.
\end{proof}

Let us now return to the subject of the proper base change theorem. We have essentially defined a proper morphism of $\infty$-topoi to be one for which the proper base change theorem holds. The challenge, then, is to produce examples of proper geometric morphisms.
The following results will be proven in \S \ref{closedsub} and \S \ref{properproper}, respectively:

\begin{itemize}
\item[$(1)$] If $p: X \rightarrow Y$ is a closed embedding of topological spaces, then
$p_{\ast}: \Shv(X) \rightarrow \Shv(Y)$ is proper.
\item[$(2)$] If $X$ is a compact Hausdorff space, then the global sections functor
$\Gamma: \Shv(X) \rightarrow \Shv(\ast)$ is proper.
\end{itemize}

Granting these statements for the moment, we can deduce the main result of this section.
First, we must recall a bit of point-set topology:

\begin{definition}\index{gen}{completely regular}\index{gen}{topological space!completely regular}
A topological space $X$ is said to be {\it completely regular} if every point of $X$ is closed in $X$, and for every closed subset $Y \subseteq X$ and every point $x \in X-Y$ there is a continuous function $f: X \rightarrow [0,1]$ such that $f(x) = 0$ and $f|Y$ takes the constant value $1$. 
\end{definition}

\begin{remark}
A topological space $X$ is completely regular if and only if it is homeomorphic to a subspace of a compact Hausdorff space $\overline{X}$ (see \cite{munkres}).
\end{remark}

\begin{definition}\index{gen}{proper!map of topological spaces}
A map $p: X \rightarrow Y$ of (arbitrary) topological spaces is said to be {\it proper} if it is universally closed. In other words, $p$ is proper if and only if for every pullback diagram of topological spaces
$$ \xymatrix{ X' \ar[d]^{p'} \ar[r] & X \ar[d]^{p} \\
Y' \ar[r] & Y }$$
the map $p'$ is closed.
\end{definition}

\begin{remark}
A map $p: X \rightarrow Y$ of topological spaces is proper 
if and only if it is closed and each of the fibers of $p$ is compact (though not necessarily Hausdorff).
\end{remark}

\begin{theorem}\label{pbct}
Let $p: X \rightarrow Y$ be a proper map of topological spaces, where $X$ is completely regular. Then $p_{\ast}: \Shv(X) \rightarrow \Shv(Y)$ is proper.
\end{theorem}

\begin{proof}
Let $q: X \rightarrow \overline{X}$ be an identification of $X$ with a subspace of a compact Hausdorff space $\overline{X}$. The map $p$ admits a factorization
$$ X \stackrel{q \times p}{\rightarrow}\overline{X} \times Y \stackrel{\pi_{Y}}{\rightarrow} Y.$$
Using Proposition \ref{properties}, we can reduce to proving that $(q \times p)_{\ast}$ and $(\pi_{Y})_{\ast}$ are proper.

Because $q$ identifies $X$ with a subspace of $\overline{X}$, $q \times p$ identifies
$X$ with a subspace over $\overline{X} \times Y$. Moreover, $q \times p$ factors
as a composition 
$$ X \rightarrow \overline{X} \times X \rightarrow \overline{X} \times Y$$
where the first map is a closed immersion (since $\overline{X}$ is Hausdorff) and
the second map is closed (since $p$ is proper). It follows that $q \times p$ is a closed immersion,
so that $(q \times p)_{\ast}$ is a proper geometric morphism by Proposition \ref{closeduse2}.

Proposition \ref{cartmun} implies that the geometric morphism
$(\pi_{Y})_{\ast}$ is a pullback of the global sections functor $\Gamma: \Shv( \overline{X} ) \rightarrow \Shv(\ast)$ in the $\infty$-category $\RGeom$. Using Proposition \ref{properties}, we may reduce to proving that $\Gamma$ is proper, which follows from Corollary \ref{compactprop}.
\end{proof}

\begin{remark}\label{pbct2}
The converse to Theorem \ref{pbct} holds as well (and does not require the assumption that $X$ is completely regular): if $p_{\ast}: \Shv(X) \rightarrow \Shv(Y)$ is a proper geometric morphism, then
$p$ is a proper map of topological spaces. This can be proven easily, using the characterization of properness described in Remark \ref{swurk}.
\end{remark}

\begin{corollary}[Nonabelian Proper Base Change Theorem]\index{gen}{proper base change theorem!nonabelian}\label{suman}
Let
$$ \xymatrix{ X' \ar[r]^{q'} \ar[d]^{p'} & X \ar[d]^{p} \\
Y' \ar[r]^{q} & Y }$$
be a pullback diagram of locally compact Hausdorff spaces, and suppose that
$p$ is proper. Then the associated diagram
$$ \xymatrix{ \Shv(X') \ar[r]^{q'_{\ast}} \ar[d]^{p'_{\ast}} & \Shv(X) \ar[d]^{p_{\ast}} \\
\Shv(Y') \ar[r]^{q_{\ast}} & \Shv(Y) }$$ is left adjointable.
\end{corollary}

\begin{proof}
In view of Theorem \ref{pbct}, it suffices to show that 
$$ \xymatrix{ \Shv(X') \ar[r]^{q'_{\ast}} \ar[d]^{p'_{\ast}} & \Shv(X) \ar[d]^{p_{\ast}} \\
\Shv(Y') \ar[r]^{q_{\ast}} & \Shv(Y) }$$
is a pullback diagram of $\infty$-topoi. Let $\overline{X}$ denote a compactification of $X$ (for example, the one-point compactification) and consider the larger diagram of $\infty$-topoi
$$ \xymatrix{ \Shv(X') \ar[r] \ar[d] & \Shv(X) \ar[d] & \\
\Shv(\overline{X} \times Y' ) \ar[r] \ar[d] & \Shv( \overline{X} \times Y) \ar[r] \ar[d] & \Shv( \overline{X} ) \ar[d] \\
\Shv(Y') \ar[r] & \Shv(Y) \ar[r] & \Shv(\ast). }$$
The upper square is a (homotopy) pullback by Proposition \ref{closeduse2} and Corollary \ref{closeduse1}. The lower right square and the lower rectangle are (homotopy) Cartesian
by Proposition \ref{cartmun}, so that the lower left square is (homotopy) Cartesian as well.
It follows that the vertical rectangle is also (homotopy) Cartesian, as desired.
\end{proof}

\begin{remark}\label{classicpbct}
The classical proper base change theorem, for sheaves of abelian groups on locally compact
topological spaces, is a formal consequence of Corollary \ref{suman}. We give a brief sketch.
The usual formulation of the proper base change theorem (see, for example, \cite{kashiwara}) is equivalent to the statement that
if
$$ \xymatrix{ X' \ar[r]^{q'} \ar[d]^{p'} & X \ar[d]^{p} \\
Y' \ar[r]^{q} & Y }$$
is a pullback diagram of locally compact topological spaces, and $p$ is proper, then
the associated diagram
$$ \xymatrix{ D^{-}(X') \ar[r]^{q'_{\ast}} \ar[d]^{p'_{\ast}} & D^{-}(X) \ar[d]^{p_{\ast}} \\
D^{-}(Y') \ar[r]^{q_{\ast}} & D^{-}(Y) }$$
is left adjointable. Here $D^{-}(Z)$ denotes the (bounded below) derived category
of abelian sheaves on a topological space $Z$. 

Let $\bfA$ denote the category whose objects are chain complexes
$$ \ldots \rightarrow A^{-1} \rightarrow A^0 \rightarrow A^1 \rightarrow \ldots $$
of abelian groups. Then $\bfA$ admits the structure of a combinatorial model category, in which the weak equivalences are given by quasi-isomorphisms. Let $\calC = \sNerve( \bfA^{\degree})$ be the underlying $\infty$-category.
For any topological space $Z$, one can define an $\infty$-category
$\Shv(Z;\calC)$ of sheaves on $Z$ with values in $\calC$; see \S \ref{products}. The homotopy
category $\h{\Shv(Z;\calC)}$ is an unbounded version of the derived category
$D^{-}(Z)$; in particular, it contains $D^{-}(Z)$ as a full subcategory. Consequently, we obtain a natural generalization of the proper base change theorem where boundedness hypotheses have been removed, which asserts that the diagram
$$ \xymatrix{ \Shv(X'; \calC) \ar[r]^{q'_{\ast}} \ar[d]^{p'_{\ast}} & \Shv(X;\calC) \ar[d]^{p_{\ast}} \\
\Shv(Y'; \calC) \ar[r]^{q_{\ast}} & \Shv(Y;\calC) }$$
is left adjointable. Using the fact that $\calC$ has enough compact objects, one can deduce this statement formally from Corollary \ref{suman}.
\end{remark}

\subsection{Closed Subtopoi}\label{closedsub}

If $X$ is a topological space and $U \subseteq X$ is an open subset, then we may view
the closed complement $X-U \subseteq X$ as a topological space in its own right. Moreover, the inclusion $(X-U) \hookrightarrow X$ is a proper map of topological spaces (that is, a closed map whose fibers are compact). The purpose of this section is to present an analogous construction in the case where $\calX$ is an $\infty$-topos.

\begin{lemma}\label{cst}
Let $\calX$ be an $\infty$-topos and $\emptyset$ an initial object of $\calX$. Then $\emptyset$ is $(-1)$-truncated.
\end{lemma}

\begin{proof}
Let $X$ be an object of $\calX$. The space $\bHom_{\calX}(X, \emptyset)$ is contractible if $X$ is an initial object of $\calX$, and empty otherwise (by Lemma \ref{sumoto}). In either case,
$\bHom_{\calX}(X, \emptyset)$ is $(-1)$-truncated.
\end{proof}

\begin{lemma}\label{drupe}
Let $\calX$ be an $\infty$-topos and let $f: \emptyset \rightarrow X$ be a morphism in $\calX$, where $\emptyset$ is an initial object. Then $f$ is a monomorphism.
\end{lemma}

\begin{proof}
Apply Lemma \ref{cst} to the $\infty$-topos $\calX_{/X}$.
\end{proof}

\begin{proposition}\label{turb}
Let $\calX$ be an $\infty$-topos and let $U$ be an object of $\calX$. Let
$S_{U}$ be the smallest strongly saturated class of morphisms of $\calX$ which is stable under pullbacks and contains a morphism $f: \emptyset \rightarrow U$, where $\emptyset$ is an initial object of $\calX$. Then $S_{U}$ is topological (in the sense of Definition \ref{deftoploc}).
\end{proposition}

\begin{proof}
For each morphism $g: X \rightarrow U$ in $\calC$, form a pullback square
$$ \xymatrix{ \emptyset' \ar[r]^{f_{Y}} \ar[d] & Y \ar[d]^{g} \\
\emptyset \ar[r]^{f} & U. }$$
Let $S = \{ f_{X} \}_{ g: X \rightarrow U}$ and let $\overline{S}$ be the strongly saturated class of morphisms generated by $S$. We note that each $f_{X}$ is a pullback of $f$, and therefore a monomorphism (by Lemma \ref{drupe}). Let $S'$ be the collection of all morphisms $h: V \rightarrow W$ with the property that for every pullback diagram
$$ \xymatrix{ V' \ar[r] \ar[d]^{h'} & V \ar[d]^{h} \\
W' \ar[r] & W }$$
in $\calX$, the morphism $h$ belongs to $\overline{S}$. Since colimits in $\calX$ are universal, we deduce that $S'$ is strongly saturated, and $S \subseteq S' \subseteq \overline{S}$ by construction. Therefore $S' = \overline{S}$, so that $\overline{S}$ is stable under pullbacks. Since
$f \in \overline{S}$, we deduce that $S_{U} \subseteq \overline{S}$. On the other hand,
$S \subseteq S_{U}$ and $S_{U}$ is strongly saturated, so $\overline{S} \subseteq S_U$.
Therefore $S_{U} = \overline{S}$. Since $S$ consists of monomorphisms, we conclude that $S_{U}$ is topological.
\end{proof}

In the situation of Proposition \ref{turb}, we will say that a morphism of $\calX$ is an {\it equivalence away from $U$} if it belongs to $S_{U}$.\index{gen}{equivalence!away from $U$}

\begin{lemma}\label{charclosedtub}
Let $\calX$ be an $\infty$-topos containing a pair of objects $U,X \in \calX$, and let $S_{U}$ denote the class of morphism in $\calX$ which are equivalences away from $U$. The following are equivalent:
\begin{itemize}
\item[$(1)$] The object $X$ is $S_{U}$-local.
\item[$(2)$] For every map $\widetilde{U} \rightarrow U$ in $\calX$, the space
$\bHom_{\calX}(\widetilde{U}, X)$ is contractible.
\item[$(3)$] There exists a morphism $g: U \rightarrow X$ such that the diagram
$$ \xymatrix{ & U \ar[dl]^{\id_{U}} \ar[dr]^{g} & \\
U & & X }$$
exhibits $U$ as a product of $U$ with $X$ in $\calX$.
\end{itemize}
\end{lemma}

\begin{proof}
Let $S$ be the collection of all morphisms $f_{\widetilde{U}}$ which come from pullback diagrams
$$ \xymatrix{ \emptyset' \ar[r]^{f_{\widetilde{U}}} \ar[d] & \widetilde{U} \ar[d] \\
\emptyset \ar[r] & U }$$
where $\emptyset$ and therefore also $\emptyset'$ are initial objects of $\calX$. We saw in the proof of Proposition \ref{turb} that $S$ generates $S_{U}$ as a strongly saturated class of morphisms. Therefore, $X$ is $S_{U}$-local if and only if each $f_{\widetilde{U}}$ induces
an isomorphism
$$ \bHom_{\calX}( \widetilde{U} , X) \rightarrow \bHom_{\calX}( \emptyset', X) \simeq \ast$$
in the homotopy category $\calH$. This proves that $(1) \Leftrightarrow (2)$.

Now suppose that $(2)$ is satisfied. Taking $\widetilde{U}  = U$, we deduce that there exists a morphism $g: U \rightarrow X$. We will prove that $g$ and $\id_{X}$ exhibit $U$ as a product of $U$ with $X$. As explained in \S \ref{quasilimit5}, this is equivalent to the assertion that for every $Z \in \calX$, the map
$$ \bHom_{\calX}(Z,U) \rightarrow \bHom_{\calX}(Z,U) \times \bHom_{\calX}(Z,X)$$ is an isomorphism in $\calH$. If there are no morphisms from $Z$ to $U$ in $\calX$, then both sides are empty and the result is obvious. Otherwise, we may invoke $(2)$ to deduce that
$\bHom_{\calX}(Z,X)$ is contractible, and the desired result follows. This completes the proof that $(2) \Rightarrow (3)$.

Suppose now that $(3)$ is satisfied for some morphism $g: U \rightarrow X$. For any object
$Z \in \calX$, we have a homotopy equivalence
$$ \bHom_{\calX}(Z,U) \rightarrow \bHom_{\calX}(Z,U) \times \bHom_{\calX}(Z,X).$$
If $\bHom_{\calX}(Z,U)$ is nonempty, then we may pass to the fiber over a point
of $\bHom_{\calX}(Z,U)$ to obtain a homotopy equivalence $\ast \rightarrow \bHom_{\calX}(Z,X)$, so that $\bHom_{\calX}(Z,X)$ is contractible. This proves $(2)$.
\end{proof}

If $\calX$ is an $\infty$-topos and $U \in \calX$, then we will say that an object
$X \in \calX$ is {\it trivial on $U$}\index{gen}{trivial on $U$}\index{not}{X/U@$\calX/U$}
if it satisfies the equivalent conditions of Lemma \ref{charclosedtub}. We let $\calX/U$ denote the full subcategory of $\calX$ spanned by the objects $X$ which are trivial on $U$. It follows from Proposition \ref{turb} that $\calX/U$ is a topological localization of $\calX$, and in particular $\calX/U$ is an $\infty$-topos. We next show that
$\calX/U$ depends only on the support of $U$.

\begin{lemma}\label{yurba}
Let $\calX$ be an $\infty$-topos, and let $g: U \rightarrow V$ be a morphism in
$\calX$. Then $\calX/V \subseteq \calX/U$. Moreover, if $g$ is an effective epimorphism, then
$\calX/U = \calX/V$. 
\end{lemma}

\begin{proof}
The first assertion follows immediately from Lemma \ref{charclosedtub}. To prove the
second, it will suffice to prove that if $g$ is strongly saturated then $S_{V} \subseteq S_{U}$. Since $S_{U}$ is strongly saturated and stable under pullbacks, it will suffice to prove that $S_{U}$ contains a morphism
$f: \emptyset \rightarrow V$, where $\emptyset$ is an initial object of $\calX$.

Form a pullback diagram $\sigma: \Delta^1 \times \Delta^1 \rightarrow \calX$:
$$ \xymatrix{ \emptyset' \ar[r]^{f'} \ar[d] & U \ar[d]^{g} \\
\emptyset \ar[r]^{f} & V.}$$ We may view $\sigma$ as an effective epimorphism
from $f'$ to $f$ in the $\infty$-topos $\calX^{\Delta^1}$. Let $f_{\bigdot} = \mCech(\sigma): \cDelta_{+} \rightarrow \calX^{\Delta^1}$ be a \Cech nerve of $\sigma: f' \rightarrow f$. We note that
for $n \geq 0$, the map $f_n$ is a pullback of $f'$, and therefore belongs to $S_{U}$. Since
$f_{\bigdot}$ is a colimit diagram, we deduce that $f$ belongs to $S_{U}$ as desired.
\end{proof}

If $\calX$ is an $\infty$-topos, we let $\Sub(1_{\calX})$ denote the partially ordered set of equivalence classes of $(-1)$-truncated objects of $\calX$. We note that this set is independent of the choice of a final object $1_{\calX} \in \calX$, up to canonical isomorphism.
Any $U \in \Sub(1_{\calX})$ can be represented by a $(-1)$-truncated object $\widetilde{U} \in \calX$. We define $\calX/U = \calX/ \widetilde{U} \subseteq \calX$. It follows from Lemma \ref{yurba} that $\calX/U$ is independent of the choice of $\widetilde{U}$ representing $U$, and that for any object $X \in \calX$, we
have $\calX/X = \calX/U$ where $U \in \Sub(1_{\calX})$ is the ``support'' of $X$ (namely, the equivalence class of the truncation $\tau_{-1} X$).\index{gen}{support}\index{not}{X/U@$\calX/U$}

\begin{definition}\index{gen}{closed!subtopos}\index{gen}{closed!immersion}
If $\calX$ is an $\infty$-topos and $U \in \Sub(1_{\calX})$, then we will refer to 
 $\calX/U$ as the {\it closed subtopos of $\calX$ complementary to $U$}. More generally, we will say that a geometric morphism $\pi: \calY \rightarrow \calX$ is a {\it closed immersion} if
there exists $U \in \Sub(1_{\calX})$ such that $\pi_{\ast}$ induces an equivalence of $\infty$-categories from $\calY$ to $\calX/U$.
\end{definition}

\begin{proposition}\label{slurppp}
Let $\calX$ be an $\infty$-topos, and let $U \in \Sub(1_{\calX})$. Then the closed immersion
$$ \pi: \calX/U \rightarrow \calX$$ induces an isomorphism of partially ordered sets from
$\Sub( 1_{\calX/U})$ to $\{ V \in \Sub(1_{\calX}): U \subseteq V\} )$.
\end{proposition}

\begin{proof}
Choose a $(-1)$-truncated object $\widetilde{U} \in \calX$ representing $U$.
Since $\pi^{\ast}$ is left exact, an object $X$ of $\calX/U$ is $(-1)$-truncated as an object of $\calX/U$ if and only if it is $(-1)$-truncated as an object of $\calX$. It therefore suffices to prove that
if $\widetilde{V}$ is a $(-1)$-truncated object of $\calX$ representing an element $V \in \Sub(1_{\calX})$, then $\widetilde{V}$ is $S_{U}$-local if and only if $U \subseteq V$. One direction is clear: if $\widetilde{V}$ is $S_U$-local, then we have an isomorphism
$$ \bHom_{\calX}( \widetilde{U}, \widetilde{V} ) \rightarrow \bHom_{\calX}( \emptyset, \widetilde{V}) = \ast$$
in the homotopy category $\calH$, so that $U \subseteq V$. The converse follows from characterization $(3)$ given in Lemma \ref{charclosedtub}.
\end{proof}

\begin{corollary}
Let $\calX$ be an $\infty$-topos, and let $U,V \in \Sub(1_{\calX})$. Then $S_{U} \subseteq S_{V}$ if and only if $U \subseteq V$.
\end{corollary}

\begin{proof}
The ``if'' direction follows from Lemma \ref{yurba} and the converse from Proposition \ref{slurppp}.
\end{proof}

\begin{corollary}\label{glad1}
Let $\calX$ be a $0$-localic $\infty$-topos, associated to the locale $\calU$, and let
$U \in \calU$. Then $\calX/U$ is a $0$-localic $\infty$-topos
associated to the locale $\{ V \in \calU: U \subseteq V \}$.
\end{corollary}

\begin{proof}
The $\infty$-topos $\calX/U$ is a topological localization of a $0$-localic $\infty$-topos, and therefore also $0$-localic (Proposition \ref{useiron}). The identification of the underlying locale follows from Proposition \ref{slurppp}.
\end{proof}

\begin{corollary}\label{closeduse1}
Let $X$ be a topological space, $U \subseteq X$ an open subset and $Y = X - U$. The inclusion
of $Y$ in $X$ induces a closed immersion of $\infty$-topoi $\Shv(Y) \rightarrow \Shv(X)$ and
an equivalence $\Shv(Y) \rightarrow \Shv(X)/U$.
\end{corollary}

\begin{lemma}\label{unipropclose}
Let $\calX$ and $\calY$ be $\infty$-topoi, and let $U \in \calY$ be an object. The map
$$ \Fun_{\ast}(\calX, \calY/U) \rightarrow \Fun_{\ast}(\calX, \calY)$$ identifies
$\Fun_{\ast}(\calX, \calY/U)$ with the full subcategory of $\Fun_{\ast}(\calX,\calY)$ spanned by those geometric morphisms $\pi_{\ast}: \calX \rightarrow \calY$ such that $\pi^{\ast} U$ is an initial object of $\calX$ $($here $\pi^{\ast}$ denotes a left adjoint to $\pi_{\ast}${}$)$.
\end{lemma}

\begin{proof}
Let $\pi_{\ast}: \calX \rightarrow \calY$ be a geometric morphism. Using the adjointness of
$\pi_{\ast}$ and $\pi^{\ast}$, it is easy to see that $\pi_{\ast} X$ is $S_U$-local if and onyl if
$X$ is $\pi^{\ast}(S_U)$-local. In particular, $\pi_{\ast}$ factors through $\calY/U$ if and only if
$\pi^{\ast}(S_U)$ consists of equivalences in $\calX$. 
Choosing $f \in S_{U}$ of the form $f: \emptyset \rightarrow U$, where
$\emptyset$ is an initial object of $\calX$, we deduce that $\pi^{\ast} f$ is an equivalence
so that $\pi^{\ast} U \simeq \pi^{\ast} \emptyset$ is an initial object of $\calX$. 
Conversely, suppose that $\pi^{\ast} U$ is an initial object of $\calX$. Then
$\pi^{\ast} f$ is a morphism between two initial objects of $\calX$, and therefore an equivalence.
Since $\pi^{\ast}$ is left exact and colimit-preserving, the collection of all morphisms
$g$ such that $\pi^{\ast} g$ is an equivalence is strongly saturated, stable under pullbacks, and contains $f$; it therefore contains $S_{U}$, so that $\pi_{\ast}$ factors through $\calY/U$ as desired.
\end{proof}

\begin{proposition}\label{closeduse2}
Let $\pi_{\ast}: \calX \rightarrow \calY$ be a geometric morphism of $\infty$-topoi, and 
let $\pi^{\ast}: \Sub(1_{\calX}) \rightarrow \Sub(1_{\calY})$ denote the induced map of
partially ordered sets. Let $U \in \Sub(1_{\calX})$. There is a commutative diagram
$$ \xymatrix{ \calX/\pi^{\ast}U \ar[rrr]^{\pi_{\ast} | (\calX/\pi^{\ast}U)} \ar[d] & & & \calY/U \ar[d] \\
\calX \ar[rrr]^{\pi_{\ast}} & & &  \calY }$$
of $\infty$-topoi and geometric morphisms, where the vertical maps are given by the natural inclusions. This diagram is left adjointable, and exhibits $\calX/ (\pi^{\ast}U)$ as a fiber product of
$\calX$ and $\calY/U$ over $\calY$ in the $\infty$-category $\RGeom$.
\end{proposition}

\begin{proof}
Let $\pi^{\ast}$ denote a left adjoint to $\pi_{\ast}$.
Our first step is to show that the upper horizontal map $\pi_{\ast} |(\calX/\pi^{\ast}U)$ is well-defined. In other words, we must show that if $X \in \calX$ is trivial on $\pi^{\ast} U$, then
$\pi_{\ast} X \in \calY$ is trivial on $U$. Suppose that $Y \in \calY$ has support contained in $U$; we must show that $\bHom_{\calY}(Y, \pi_{\ast} X)$ is contractible. But this space is homotopy equivalent to $\bHom_{\calX}( \pi^{\ast} Y, X) \simeq \ast$, since $\pi^{\ast} Y$ has support contained in $\pi^{\ast} U$ and $X$ is trivial on $\pi^{\ast} U$.

We note also that $\pi^{\ast}$ carries $\calY/U$ into $\calX/\pi^{\ast}U$. This follows immediately from characterization $(3)$ of Lemma \ref{charclosedtub}, since $\pi^{\ast}$ is left exact. Therefore $\pi^{\ast}| \calY/U$ is a left adjoint of $\pi_{\ast} | \calX/\pi^{\ast}U$. From the fact that $\pi^{\ast}$ is left-exact we easily deduce that $\pi^{\ast}| \calY/U$ is left exact. It follows that $\pi_{\ast}|\calX/\pi^{\ast}U$ has a left-exact left adjoint, and is therefore a geometric morphism of $\infty$-topoi.
Moreover, the diagram
$$ \xymatrix{ \calX/\pi^{\ast}U \ar[d] & & & \calY/U \ar[d] \ar[lll]_{\pi^{\ast} | \calY/Y} \\
\calX & & & \calY \ar[lll]_{\pi^{\ast}} }$$
is (strictly) commutative, which proves that the diagram of pushforward functors is left adjointable.

We now claim that the diagram 
$$ \xymatrix{ \calX/\pi^{\ast}U \ar[rrr]^{\pi_{\ast} | \calX/\pi^{\ast}U} \ar[d] & &  & \calY/U \ar[d] \\
\calX \ar[rrr] & &  & \calY }$$
is a pullback diagram of $\infty$-topoi. For every pair of $\infty$-topoi $\calA$ and $\calB$,
let $[\calA, \calB]$ denote the largest Kan complex contained in $\Fun_{\ast}(\calA, \calB)$. According to Theorem \ref{colimcomparee}, it will suffice to show that for any $\infty$-topos $\calZ$, the associated diagram of Kan complexes
$$ \xymatrix{ [\calZ, \calX/\pi^{\ast}U] \ar[r] \ar[d] & [\calZ, \calY/U] \ar[d] \\
[\calZ, \calX] \ar[r] & [\calZ, \calY] }$$
is homotopy Cartesian. Lemma \ref{unipropclose} implies that the vertical maps are inclusions of full simplicial subsets. 
It therefore suffices to show that if $\phi_{\ast}: \calZ \rightarrow \calY$ is a geometric morphism
such that $\pi_{\ast} \circ \phi_{\ast}$ factors through $\calY/U$, then
$\phi_{\ast}$ factors through $\calX/\pi^{\ast}U$. This follows immediately from the characterization given in Lemma \ref{unipropclose}.
\end{proof}

\begin{corollary}
Let $$ \xymatrix{ \calX' \ar[d]^{p'_{\ast}} \ar[r] & \calX \ar[d]^{p_{\ast}} \\
\calY' \ar[r] & \calY }$$
be a pullback diagram in the $\infty$-category $\RGeom$ of $\infty$-topoi. If
$p_{\ast}$ is a closed immersion, then $p'_{\ast}$ is a closed immersion.
\end{corollary}

\subsection{Products of $\infty$-Topoi}\label{products}\index{gen}{product!of $\infty$-topoi}

In \S \ref{genlim}, we showed that the $\infty$-category $\RGeom$ of $\infty$-topoi admits all
(small) limits. Unfortunately, the construction of general limits was rather inexplicit. Our goal in this section is to give a very concrete description of the product of two $\infty$-topoi, at least in a special case.

\begin{definition}\label{valsheaf}\index{gen}{presheaf!with values in $\calC$}
Let $X$ be a topological space, and $\calC$ an $\infty$-category. We let
$\calU(X)$ denote the collection of all open subsets of $X$, partially ordered by inclusion.
A {\it presheaf on $X$ with values in $\calC$} is a functor
$\calU(X)^{op} \rightarrow \calC$. 

Let $\calF: \calU(X)^{op} \rightarrow \calC$ be a presheaf on $X$ with values in $\calC$. We will say that $\calF$ is a {\it sheaf} with values in $\calC$ if, for every $U \subseteq X$ and every
covering sieve $\calU(X)_{/U}^{(0)} \subseteq \calU(X)_{/U}$, the composition
$$ \Nerve(\calU(X)_{/U}^{(0)})^{\triangleright}
\subseteq \Nerve(\calU(X)_{/U})^{\triangleright} \rightarrow
\Nerve(\calU(X)) \stackrel{\calF}{\rightarrow} \calC^{op}$$
is a colimit diagram.

We let $\calP(X; \calC)$ denote the $\infty$-category
$\Fun( \calU(X)^{op}, \calC)$ consisting of all presheaves on $X$ with values in $\calC$, and
$\Shv(X;\calC)$ the full subcategory of $\calP(X;\calC)$ spanned by the sheaves on
$X$ with values in $\calC$.
\end{definition}

\begin{remark}
We can phrase the sheaf condition informally as follows: a $\calC$-valued presheaf $\calF$ on
a topological space $X$ is a sheaf if, for every open subset $U \subseteq X$ and every
covering sieve $\{ U_{\alpha} \subseteq U \}$, the natural map
$\calF(U) \rightarrow \projlim_{\alpha} \calF(U_{\alpha})$
is an equivalence in $\calC$.
\end{remark}

\begin{remark}
If $X$ is a topological space, then $\Shv(X) = \Shv(X, \SSet)$, where $\SSet$ denotes the $\infty$-category of spaces.
\end{remark}

\begin{lemma}\label{strippus}
Let $\calC$, $\calD$, and $\calE$ be $\infty$-categories which admit finite limits, 
let $\calC^{0} \subseteq \calC$ and $\calD^{0} \subseteq \calD$ be the full subcategories of $\calC$ and $\calD$ consisting of final objects. Let $F: \calC \times \calD \rightarrow \calE$ be a functor.
The following conditions are equivalent:

\begin{itemize}
\item[$(1)$] The functor $F$ preserves finite limits.

\item[$(2)$] 
The functors $F | \calC^{0} \times \calD$ and
$F| \calC \times \calD^{0}$ preserve finite limits, and for every pair of morphisms
$C \rightarrow 1_{\calC}$, $D \rightarrow 1_{\calD}$ where $1_{\calC} \in \calC$
and $1_{\calD} \in \calD$ are final objects, the associated diagram
$$ F(1_{\calC},D) \leftarrow F(C,D) \rightarrow F(C, 1_{\calD})$$ exhibits
$F(C,D)$ as a product of $F(1_{\calC},D)$ and $F(C,1_{\calD})$ in $\calE$.

\item[$(3)$] The functors $F | \calC^{0} \times \calD$ and
$F| \calC \times \calD^{0}$ preserve finite limits, and $F$ is a right
Kan extension of the restriction
$$F^0=  F | (\calC \times \calD^{0}) \coprod_{ \calC^{0} \times \calD^{0} } (\calC^{0} \times \calD).$$
\end{itemize}
\end{lemma}

\begin{proof}
The implication $(1) \Rightarrow (2)$ is obvious. To see that $(2) \Rightarrow (1)$, we
choose final objects $1_{\calC} \in \calC$, $1_{\calD} \in \calD$, and natural transformations
$\alpha: \id_{\calC} \rightarrow \underline{1}_{\calC}$, $\beta: \id_{\calD} \rightarrow \underline{1}_{\calD}$ (where $\underline{X}$ denotes the constant functor with value $X$).
Let $F_{\calC}: \calC \rightarrow \calE$ denote the composition
$$ \calC \simeq \calC \times \{1_{\calD} \} \subseteq \calC \times \calD \stackrel{F}{\rightarrow} \calE,$$ and define $F_{\calD}$ similarly. Then $\alpha$ and $\beta$ induce natural transformations
$$ F_{\calC} \circ \pi_{\calC} \leftarrow F \rightarrow F_{\calD} \circ \pi_{\calD}.$$
Assumption $(2)$ implies that the functors $F_{\calC}$, $F_{\calD}$ preserve finite limits, and that the above diagram exhibits $F$ as a product of $F_{\calC} \circ \pi_{\calC}$ with
$F_{\calD} \circ \pi_{\calD}$ in the $\infty$-category $\calE^{\calC \times \calD}$. We now apply Lemma \ref{limitscommute} to deduce that $F$ preserves finite limits as well.

We now show that $(2) \Leftrightarrow (3)$. Assume that $F| \calC^{0} \times \calD$
and $F| \calC \times \calD^{0}$ preserve finite limits, so that in particular 
$F | \calC^{0} \times \calD^{0}$ takes values in the full subcategory $\calE^0 \subseteq \calE$ spanned by the final objects. Fix morphisms $u: C \rightarrow 1_{\calC}$, $v: D \rightarrow 1_{\calD}$, where $1_{\calC} \in \calC$ and $1_{\calD} \in \calD$ are final obejcts. We will show that
the diagram $$ F(1_{\calC},D) \leftarrow F(C,D) \rightarrow F(C, 1_{\calD})$$ exhibits
$F(C,D)$ as a product of $F(1_{\calC},D)$ and $F(C,1_{\calD})$ if and only if
$F$ is a right Kan extension of $F^0$ at $(C,D)$.

The morphisms $u$ and $v$ determine a map $u \times v: \Delta^1 \times \Delta^1 \rightarrow \calC \times \calD$, which we may identify with a map
$$ w: \Lambda^2_2 \rightarrow ((\calC^{0} \times \calD) \coprod_{ \calC^0 \times \calD^0} (\calC \times \calD^{0}))_{(C,D)/}.$$
Using Theorem \ref{hollowtt}, it is easy to see that $w^{op}$ is cofinal. Consequently,
$F$ is a right Kan extension of $F^{0}$ at $(C,D)$ if and only if the diagram
$$ \xymatrix{ F(C,D) \ar[r] \ar[d] & F(C, 1_{\calD}) \ar[d] \\
F(1_{\calC}, D) \ar[r] & F(1_{\calC}, 1_{\calD}) }$$
is a pullback square. Since $F(1_{\calC}, 1_{\calD})$ is a final object of $\calE$, this is
equivalent to assertion $(2)$.
\end{proof}

\begin{lemma}\label{sablesilk}
Let $\calC$ and $\calD$ be small $\infty$-categories which admit finite limits, and let
$1_{\calC} \in \calC$, $1_{\calD} \in \calD$ be final objects, and let $\calX$ be
an $\infty$-topos. The projections
$$ \calP(\calC \times \{1_{\calD} \}) \stackrel{p_{\ast}}{\leftarrow} \calP(\calC \times \calD) \stackrel{q_{\ast}}{\rightarrow} \calP(
\{1_{\calC} \} \times \calD)$$
induce a categorical equivalence
$$ \Fun_{\ast}(\calX, \calP(\calC \times \calD)) \rightarrow \Fun_{\ast}(\calX, \calP(\calC)) \times
\Fun_{\ast}(\calX, \calP(\calD)).$$
In particular, $\calP(\calC \times \calD)$ is a product of $\calP(\calC)$ with
$\calP(\calD)$ in the $\infty$-category $\RGeom$ of $\infty$-topoi.
\end{lemma}

\begin{proof}
For every $\infty$-category $\calY$ which admits finite limits, let
$[\calY,\calX]$ denote the full subcategory of $\Fun(\calY,\calX)$ spanned by the left
exact functors $\calY \rightarrow \calX$. If $\calY$ is an $\infty$-topos, we let
$[\calY, \calX]_0$ denote the full subcategory of $[\calY, \calX]$ spanned by
the {\em colimit-preserving} left exact functors $\calY \rightarrow \calX$. In view of
Proposition \ref{switcheroo} and Remark \ref{switcheroo2}, it will suffice to prove that composition with the left adjoints to
$p_{\ast}$ and $q_{\ast}$ induces an equivalence of $\infty$-categories
$$ [\calP(\calC \times \calD),\calX]_0 \rightarrow [\calP(C),\calX]_0 \times [\calP(\calD),\calX]_0.$$
Applying Proposition \ref{igrute}, we may reduce to the problem of showing that the map
$$ [ \calC \times \calD, \calX] \rightarrow [\calC, \calX] \times [\calD, \calX]$$
is an equivalence of $\infty$-categories. 

Let $\calC^{0} \subseteq \calC$ and $\calD^{0} \subseteq \calD$ denote the full subcategories
consisting of final objects of $\calC$ and $\calD$, respectively. Proposition \ref{initunique}
implies that $\calC^{0}$ and $\calD^{0}$ are contractible. It will therefore suffice to prove
that the restriction map
$$ \phi: [ \calC \times \calD, \calX] \rightarrow [\calC \times \calD^{0}, \calX]
\times_{[ \calC^0 \times \calD^{0}, \calX] } [\calC^{0} \times \calD, \calX]$$
is a trivial fibration of simplicial sets. This follows immediately from
Lemma \ref{strippus} and Proposition \ref{lklk}.
\end{proof}

\begin{notation}\index{not}{timesO@$\otimes$}\index{not}{timesOC@$\otimes^{\calC}$}
Let $\calX$ be an $\infty$-topos, and $p^{\ast}: \SSet \rightarrow \calX$ a geometric
morphism (essentially unique in view of Proposition \ref{spacefinall}). Let $\pi_{\calX}: \calX \times \SSet \rightarrow \calX$ and $\pi_{\SSet}: \calX \times \SSet \rightarrow \SSet$ denote the projection functors. Let $\otimes$ be a product of $\pi_{\calX}$ with $p^{\ast} \circ \pi_{\SSet}$ in the
$\infty$-category of functors from $\calX \times \SSet$ to $\calX$. Then $\otimes$
is uniquely defined up to equivalence, and we have natural transformations
$$ X \leftarrow X \otimes S \rightarrow p^{\ast} S$$
which exhibit $X \otimes S$ as product of $X$ with $p^{\ast} S$ for all $X \in \calX$, $S \in \SSet$.
We observe that $\otimes$ preserves colimits separately in each variable.

If $\calC$ is a small $\infty$-category, we let 
$\otimes^{\calC}$ denote the composition
$$ \calP(\calC; \calX) \times \calP(\calC) \simeq \calP(\calC; \calX \times \SSet)
\stackrel{\circ \otimes}{\rightarrow} \calP(\calC, \calX).$$
We observe that if $F \in \calP(\calC; \calX)$ and $G \in \calP(\calC)$, then
$F \otimes^{\calC} G$ can be identified with a product of $F$ with $p^{\ast} \circ G$ in
$\calP(\calC; \calX)$.
\end{notation}

\begin{lemma}\label{goldenboy}
Let $\calC$ be a small $\infty$-category, $\calX$ an $\infty$-topos. Let
$g: \calX \rightarrow \SSet$ a functor corepresented by an object $X \in \calX$,
and $G: \calP(\calC; \calX) \rightarrow \calP(\calC)$ the induced functor.
Let $\underline{X} \in \calP(\calC; \calX)$ denote the constant functor with the value $X$. Then
the functor
$$ F = \underline{X} \otimes^{\calC} \id_{ \calP(\calC)}.$$
is a left adjoint to $G$.
\end{lemma}

\begin{proof}
Since adjoints and $\otimes^{\calC}$ can both be computed pointwise on $\calC$, it suffices
to treat the case where $\calC = \Delta^0$. In this case, we deduce the existence of a left adjoint
$F'$ to $G$ using Corollary \ref{adjointfunctor} (the accessibility of $G$ follows from the fact
that $X$ is $\kappa$-compact for sufficiently large $\kappa$, since $\calX$ is accessible). Now
$F$ and $F'$ are both colimit-preserving functors $\SSet \rightarrow \calX$. In virtue of Theorem \ref{charpresheaf}, to prove that $F$ and $F'$ are equivalent, it will suffice to show that the objects
$F(\ast), F'(\ast) \in \calX$ are equivalent. In other words, we must prove that $F'(\ast) \simeq X$. By adjointness, we have natural isomorphisms
$$ \bHom_{\calX}( F'(\ast), Y) \simeq \bHom_{\calH}( \ast, G(Y)) \simeq \bHom_{\calX}(X,Y)$$
in $\calH$ for each $Y \in \calX$, so that $F'(\ast)$ and $X$ corepresent the same functor on
the homotopy category $\h{\calX}$, and are therefore equivalent by Yoneda's lemma.
\end{proof}

\begin{lemma}\label{goldenrod}
Let $\calC$ be a small $\infty$-category which admits finite limits and contains
a final object $1_{\calC}$, let $\calX$ and $\calY$ be $\infty$-topoi, and let 
$p_{\ast}: \calX \rightarrow \SSet$ be a geometric morphism $($essentially unique, in virtue of Proposition \ref{spacefinall}{}$)$. Then the maps 
$$ \calP(\calC) \stackrel{P_{\ast}}{\leftarrow} \calP(\calC; \calX) \stackrel{e_{1_{\calC}}}{\rightarrow} \calX$$
induce equivalences of $\infty$-categories
$$ \Fun_{\ast}( \calY, \calP(\calC; \calX)) \rightarrow \Fun_{\ast}(\calY, \calX) \times \Fun_{\ast}(\calY, \calP(\calC)).$$
In particular, $\calP(\calC; \calX)$ is a product of $\calP(\calC)$ and $\calX$ in the 
$\infty$-category $\RGeom$ of $\infty$-topoi.
Here $e_{1_{\calC}}$ denote the evaluation map at the object $1_{\calC} \in \calC$, 
and $P_{\ast}: \calP(\calC; \calX) \rightarrow \calP(\calC)$ is given by composition with
$p_{\ast}$. 
\end{lemma}

\begin{proof}
According to Proposition \ref{precisechar}, we may assume without loss of generality
that there exists a small $\infty$-category $\calD$ such that $\calX$ is the essential
image of an accessible left exact localization functor $L: \calP(\calD) \rightarrow \calP(\calD)$, 
and that $p_{\ast}$ is given by evaluation at a final object $1_{\calD} \in \calD$.
We have a commutative diagram
$$ \xymatrix{ \Fun_{\ast}(\calY, \calP(\calC; \calX)) \ar[r] \ar[d] & \Fun_{\ast}(\calY, \calP(\calC))
\times \Fun_{\ast}(\calY, \calX) \ar[d] \\
\Fun_{\ast}(\calY, \calP(\calC \times \calD)) \ar[r] & \Fun_{\ast}(\calY, \calP(\calC))
\times \Fun_{\ast}(\calY, \calP(\calD))}$$
where the vertical arrows are inclusions of full subcategories, and the bottom arrow is an equivalence of $\infty$-categories by Lemma \ref{sablesilk}. Consequently, it will suffice
to show that if $q_{\ast}: \calY \rightarrow \calP(\calC \times \calD)$ is a geometric morphism with the property that the composition
$$ r_{\ast}: \calY \rightarrow \calP(\calC \times \calD) \rightarrow \calP(\calD)$$
factors through $\calX$, then $q_{\ast}$ factors through $\calP(\calC; \calX)$. 

Let $Y \in \calY$ and $C \in \calC$; we wish to show that
$q_{\ast}(Y)(C) \in \calX$. It will suffice to show that if $s: D \rightarrow D'$ is
a morphism in $\calP(\calD)$ such that $L(s)$ is an equivalence in $\calX$, then
$q_{\ast}(Y)(C)$ is $s$-local. Let $F: \calP(\calD) \rightarrow \calP(\calC \times \calD)$
be a left adjoint to the functor given by evaluation at $C$. 
We have a commutative diagram
$$ \xymatrix{  \bHom_{\calP(\calD)}( D', q_{\ast}(Y)(C)) \ar[r] \ar[d] & \bHom_{\calY}(q^{\ast} F(D'), Y) \ar[d] \\
\bHom_{\calP(\calD)}( D, q_{\ast}(Y)(C) ) \ar[r] & \bHom_{\calY}(
q^{\ast} F(D) , Y) }$$
where the horizontal arrows are homotopy equivalences. Consequently, to prove that the left vertical map is an equivalence, it will suffice to prove that $q^{\ast} F(s)$ is an equivalence in $\calY$. According to Lemma \ref{goldenboy}, the functor
$F$ can be identified with a product of a left adjoint $r^{\ast}$ to the projection
$r_{\ast}: \calP(\calC \times \calD) \rightarrow \calP(\calD)$ with a constant functor. Since $q^{\ast}$ preserves finite products,
it will suffice to show that $(q^{\ast} \circ r^{\ast})(s)$ is an equivalence in $\calY$. This follows immediately
from our assumption that $r_{\ast} \circ q_{\ast}: \calY \rightarrow \calP(\calD)$ factors through
$\calX$.
\end{proof}

The main result of this section is the following:

\begin{theorem}\label{conprod}
Let $X$ be a topological space, $\calX$ an $\infty$-topos, and
$\pi_{\ast}: \calX \rightarrow \SSet$ a geometric morphism $($which is essentially unique,
by virtue of Proposition \ref{spacefinall}{}$)$. Then $\Shv(X; \calX)$ is an $\infty$-topos, and the
diagram
$$ \calX \stackrel{\Gamma}{\leftarrow} \Shv(X; \calX) \stackrel{\pi_{\ast}}{\rightarrow} \Shv(X)$$
exhibits $\Shv(X;\calX)$ as a product of $\Shv(X)$ and $\calX$ in the $\infty$-category
$\RGeom$ of $\infty$-topoi. Here $\Gamma$ denotes the global sections functor, given by evaluation at $X \in \calU(X)$.
\end{theorem}

\begin{proof}
We first show that $\Shv(X; \calX)$ is an $\infty$-topos. Let $\calP(X; \calX)$ be the $\infty$-category
$\Fun(\Nerve(\calU(X))^{op},\calX)$ of $\calX$-valued presheaves on $\calX$. For each object
$Y \in \calX$, choose a morphism $e_Y: \emptyset_{\calX} \rightarrow Y$ in $\calX$, whose source is
an initial object of $\calX$.
For each sieve $\calV$ on $\calX$, let $\chi^Y_{\calV}: \calU(X)^{op} \rightarrow \calX$
be the composition
$$ \calU(X)^{op} \rightarrow \Delta^1 \stackrel{e_Y}{\rightarrow} \calX,$$
so that $$\chi^Y_{\calV}(U) \begin{cases} Y & \text{if } U \in \calV \\
\emptyset_{\calX} & \text{if } U \notin \calV.
\end{cases}$$
so that we have a natural map $\chi^Y_{\calV} \rightarrow \chi^Y_{\calV'}$ if
$\calV \subseteq \calV'$. For each open subset $U \subseteq X$, let
$\chi^{Y}_{U}= \chi^{Y}_{\calV}$, where $\calV = \{ V \subseteq U \}$. 
Let $S$ be the set of all morphisms
$f^{Y}_{\calV}: \chi^{Y}_{\calV} \rightarrow \chi^{Y}_{U}$, where $\calV$ is a sieve covering $U$,
and let $\overline{S}$ be the strongly saturated class of morphisms generated by $\calX$. We first claim that $\overline{S}$ is setwise generated. To see this, we observe that the passage from
$Y$ to $f^{Y}_{\calV}$ is a colimit-preserving functor of $Y$, so it suffices to consider a set of objects $Y \in \calX$ which generates $\calX$ under colimits.

We next claim that $\overline{S}$ is topological, in the sense of Definition \ref{deftoploc}.
By a standard argument, it will suffice to show that there is a class of objects
$F_{\alpha} \in \calP(X; \calX)$ which generates $\calP(X; \calX)$ under colimits, such that
for every pullback diagram
$$ \xymatrix{ F'_{\alpha} \ar[d]^{f'} \ar[r] & \chi^{Y}_{\calV} \ar[d]^{f^{Y}_{\calV}} \\
F_{\alpha} \ar[r] & \chi^{Y}_{U}, }$$
the morphism $f'$ belongs to $\overline{S}$. We observe that if
$\calX$ is a left exact localization of $\calP(\calD)$, then $\calP(X; \calX)$ is a left
exact localization of $\calP( \calU(X) \times \calD)$ and is therefore generated under colimits
by the Yoneda image of $\calU(X) \times \calD$. In other words, it will suffice to consider 
$F_{\alpha}$ of the form $\chi^{Y'}_{U'}$, where $Y' \in \calX$ and $U' \subseteq X$.
If $Y'$ is an initial object of $\calX$, then $g$ is an equivalence and there is nothing to prove. Otherwise, the existence of the lower horizontal map implies that $U' \subseteq U$. 
Let $\calV' = \{ V \in \calV: V \subseteq U' \}$; then it is easy to see that $f'$ is
equivalent to $\chi^{Y'}_{\calV'}$, and therefore belongs to $\overline{S}$.

We next claim that $\Shv(X; \calX)$ consists precisely of the $S$-local objects of
$\calP(X; \calX)$. To see this, let $Y \in \calX$ be an arbitrary object, and consider
the functor $G_Y: \calX \rightarrow \SSet$ corepresented by $Y$. It follows from Proposition \ref{yonedaprop} that an arbitrary $F \in \calP(X; \calX)$ is a $\calX$-valued sheaf on $X$ if and only if, for each $Y \in \calX$, the composition $G_Y \circ F \in \calP(X)$ belongs to $\Shv(X)$.
This is equivalent to the assertion that, for every sieve $\calV$ which covers $U \subseteq X$,
the presheaf $G_Y \circ F$ is $s_{\calV}$-local, where $s_{\calV}: \chi_{\calV} \rightarrow \chi_{U}$ is the associated monomorphism of presheaves. Let $G^{\ast}_Y$ denote a left adjoint to $G_{Y}$; then $G_Y \circ F$ is $s_{\calV}$-local if and only if $F$ is $G^{\ast}_{Y}(s_{\calV})$-local. We now apply Lemma \ref{goldenboy} to identify $G^{\ast}_{Y}(s_{\calV})$ with $f_{\calV}^{Y}$.

We now have an identification $\Shv(X; \calX) \simeq \overline{S}^{-1} \calP(X; \calX)$, so that
$\Shv(X; \calX)$ is a topological localization of $\calP(X; \calX)$ and in particular an $\infty$-topos. 
We now consider an arbitrary $\infty$-topos $\calY$. We have a commutative diagram
$$ \xymatrix{ \Fun_{\ast}(\calY, \Shv(X; \calX)) \ar[r] \ar[d] & \Fun_{\ast}(\calY, \Shv(X)) \times \Fun_{\ast}(\calY; \calX) \ar[d] \\
\Fun_{\ast}(\calY, \calP(X; \calX)) \ar[r] & \Fun_{\ast}(\calY,\calP(X)) \times \Fun_{\ast}(\calY, \calX), } $$
where the vertical arrows are inclusions of full subcategories and the lower horizontal arrow is an equivalence by Lemma \ref{goldenrod}. To complete the proof, it will suffice to show that the upper horizontal arrow is also an equivalence. In other words, we must show that if
$g_{\ast}: \calY \rightarrow \calP(X; \calX)$ is a geometric morphism with the property that
the composition 
$$ \calY \stackrel{g_{\ast}}{\rightarrow} \calP(X; \calX) \stackrel{h_{\ast}}{\rightarrow} \calP(X) $$
factors through $\Shv(X)$, then $g_{\ast}$ factors through $\Shv(X; \calX)$.
Let $g^{\ast}$ and $h^{\ast}$ denote left adjoints to $g_{\ast}$ and $h_{\ast}$, respectively. It will suffice to show that for every
morphism $f_{\calV}^{Y} \in S$, the pullback $g^{\ast} f_{\calV}^{Y}$ is an equivalence in $\calY$.
We now observe that $f_{\calV}^{Y}$ is a pullback of $f_{\calV}^{1_{\calX}}$; since
$g^{\ast}$ is left exact, it will suffice to show that $g^{\ast} f_{\calV}^{1_{\calX}}$
is an equivalence in $\calY$. We have an equivalence $f_{\calV}^{1_{\calX}} \simeq h^{\ast} s_{\calV}$, where $s_{\calV}$ is the monomorphism in $\calP(X)$ associated to the sieve $\calV$.
The composition $(g^{\ast} \circ h^{\ast})( s_{\calV} )$ is an equivalence because 
$h_{\ast} \circ g_{\ast}$ factors through $\Shv(X)$, which consists of $s_{\calV}$-local objects
of $\calP(X)$.
\end{proof}

\begin{remark}\index{gen}{fiber product!of $\infty$-topoi}
It is not difficult to extend the proof of Theorem \ref{conprod} to the case where
$\Shv(X)$ is replaced by an arbitrary $\infty$-topos $\calY$. In this case, one must replace
$\Shv(X; \calX)$ by the $\infty$-category of all limit-preserving functors $\calY^{op} \rightarrow \calX$. Using these ideas, one can construct the fiber product
$$ \calX \times_{ \calZ} \calY$$ in $\RGeom$ where
$\calZ = \SSet$ is the final object in $\RGeom$. To give a construction which works in general,  one needs to repeat all of the above arguments in a relative setting over the $\infty$-topos $\calZ$. We will not pursue the subject any further in this book.
\end{remark}

\subsection{Sheaves on Locally Compact Spaces}\label{properproper}

By definition, a sheaf of sets $\calF$ on a topological space $X$ is determined by the sets
$\calF(U)$ as $U$ ranges over the open subsets of $X$. If $X$ is a locally compact Hausdorff space, then there is an alternative collection of data which determines $X$: the values
$\calF(K)$, where $K$ ranges over the compact subsets of $X$. Here $\calF(K)$ denotes the direct limit $\colim_{K \subseteq U} \calF(U)$ taken over all open neighborhoods of $K$ (or, equivalently, the collection of global sections of the restriction $\calF|K$). The goal of this section is to prove a generalization of this result, where the sheaf $\calF$ is allowed to take values in a more general $\infty$-category $\calC$.

\begin{definition}\index{not}{Kcal(X)@$\calK(X)$}
Let $X$ be a locally compact Hausdorff space. We let $\calK(X)$ denote the collection of all compact subsets of $X$.
If $K, K' \subseteq X$, we write $K \Subset K'$ if there exists an open subset $U \subseteq X$ such that $K \subseteq U \subseteq K'$. If $K \in \calK(X)$, we let $\calK_{K \Subset}(X) = \{ K' \in \calK(X): K \Subset K' \}$.\index{not}{KSubsetK'@$K \Subset K'$}\index{not}{KcalKSubsetX@$\calK_{K \Subset}(X)$}

Let $\calF: \Nerve(\calK(X))^{op} \rightarrow \calC$ be a presheaf on $\Nerve(\calK(X))$ (here $\calK(X)$ is viewed as a partially ordered set with respect to inclusion) with values in $\calC$.
We will say that
$\calF$ is a {\it $\calK$-sheaf} if the following conditions are satisfied:\index{gen}{$\calK$-sheaf}
\begin{itemize}
\item[$(1)$] The object $\calF(\emptyset) \in \calC$ is final.
\item[$(2)$] For every pair $K, K' \in \calK(X)$, the associated diagram
$$ \xymatrix{ \calF( K \cup K' ) \ar[r] \ar[d] & \calF(K) \ar[d] \\
\calF(K') \ar[r] & \calF(K \cap K') }$$
is a pullback square in $\calC$.
\item[$(3)$] For each $K \in \calK(X)$, the restriction of $\calF$ exhibits
$\calF(K)$ as a colimit of $\calF | \Nerve(\calK_{K \Subset}(X))^{op}$.
\end{itemize}

We let $\Shv_{\calK}(X;\calC)$ denote the full subcategory of $\Fun(\Nerve(\calK(X))^{op},\calC)$ spanned by the $\calK$-sheaves. In the case where $\calC = \SSet$, we will write
$\Shv_{\calK}(X)$ instead of $\Shv_{\calK}(X;\calC)$.\index{not}{ShvcalKXC@$\Shv_{\calK}(X;\calC)$}\index{not}{ShvcalKX@$\Shv_{\calK}(X)$}
\end{definition}

\begin{definition}\label{leftexactcolim}\index{gen}{filtered colimit!left exactness of}
Let $\calC$ be a presentable $\infty$-category. We will say that {\it filtered colimits
in $\calC$ are left exact} if the following condition is satisfied: for every small filtered $\infty$-category $\calI$, the colimit functor $\Fun(\calI,\calC) \rightarrow \calC$
is left exact.
\end{definition}

\begin{example}\index{gen}{Grothendieck abelian category}\index{gen}{abelian category!Grothendieck}
A {\it Grothendieck abelian category} is an abelian category $\calA$ whose nerve 
$\Nerve(\calA)$ is a presentable $\infty$-category with left exact filtered colimits, in the sense of Definition \ref{leftexactcolim}. We refer the reader to \cite{tohoku} for further discussion.
\end{example}

\begin{example}\label{sumta1}
Filtered colimits are left exact in the $\infty$-category $\SSet$ of spaces; this follows immediately from Proposition \ref{frent}. It follows that filtered colimits in $\tau_{\leq n} \SSet$ are left exact for each $n \geq -2$, since the full subcategory $\tau_{ \leq n} \SSet \subseteq \SSet$ is stable under filtered colimits and finite limits (in fact, under all limits).
\end{example}

\begin{example}\label{sumta2}
Let $\calC$ be a presentable $\infty$-category in which filtered colimits are left exact, and let $X$ be an arbitrary simplicial set. Then filtered colimits are left exact in $\Fun(X,\calC)$. This follows
immediately from Proposition \ref{limiteval}, which asserts that the relevant limits and colimits can be computed pointwise.
\end{example}

\begin{example}\label{sumta3}
Let $\calC$ be a presentable $\infty$-category in which filtered colimits are left exact, and let
$\calD \subseteq \calC$ be the essential image of an (accessible) left exact localization functor $L$. Then filtered colimits in $\calD$ are left exact. To prove this, we consider an arbitrary filtered $\infty$-category $\calI$, and observe that the colimit functor $\varinjlim: \Fun(\calI, \calD) \rightarrow \calD$ is equivalent to the composition
$$ \Fun(\calI, \calD) \subseteq \Fun(\calI,\calC) \rightarrow \calC \stackrel{L}{\rightarrow} \calD,$$
where the second arrow is given by the colimit functor $\varinjlim \Fun(\calI,\calC) \rightarrow \calC$.
\end{example}

\begin{example}\label{tucka}
Let $\calX$ be an $n$-topos, $0 \leq n \leq \infty$. Then filtered colimits in $\calX$ are left exact. This follows immediately from Examples \ref{sumta1}, \ref{sumta2}, and \ref{sumta3}.
\end{example}

Our goal is to prove that if $X$ is a locally compact Hausdorff space and $\calC$ is a presentable $\infty$-category, then the $\infty$-categories $\Shv(X)$ and $\Shv_{\calK}(X)$ are equivalent. As a first step, we prove that a $\calK$-sheaf on $X$ is determined ``locally''.

\begin{lemma}\label{noodlesoup}
Let $X$ be a locally compact Hausdorff space and $\calC$ a presentable $\infty$-category in which filtered colimits are left exact.
Let $\calW$ be a collection of open
subsets of $X$ which covers $X$, and let $\calK_{\calW}(X) = \{ K \in \calK(X): (\exists W \in \calW) [ K \subseteq W] \}$. Suppose that $\calF \in \Shv_{\calK}(X;\calC)$. Then
$\calF$ is a right Kan extension of $\calF | \Nerve(\calK_{\calW}(X))^{op}$. 
\end{lemma}

\begin{proof}
Let us say that an open covering $\calW$ of a locally compact Hausdorff space $X$ is {\it good} if it satisfies the conclusion of the Lemma. 
Note that $\calW$ is a good covering of $X$ if and only if, for every compact subset $K \subseteq X$, the open sets $\{ K \cap W: W \in \calW \}$ 
form a good covering of $K$. We wish to prove that {\em every} covering $\calW$ of a locally compact topological space $X$ is good. In virtue of the preceding remarks, we can reduce to the case where $X$ is compact, and thereby assume that $\calW$ has a finite subcover.

We will prove, by induction on $n \geq 0$, that if $\calW$ is collection of open subsets of a locally compact Hausdorff space $X$ such that there exist $W_1, \ldots, W_n \in \calW$ with
$W_1 \cup \ldots \cup W_n = X$, then $\calW$ is a good covering of $X$. If $n = 0$, then
$X = \emptyset$. In this case, we must prove that $\calF(\emptyset)$ is final, which is part of the definition of $\calK$-sheaf.

Suppose that $\calW \subseteq \calW'$ are coverings of $X$, and that for every $W' \in \calW'$ the induced covering $\{ W \cap W': W \in \calW \}$ is a good covering of $W'$. It then follows from Proposition \ref{acekan} that $\calW'$ is a good covering of $X$ if and only if $\calW$ is a good covering of $X$.

Now suppose $n > 0$. Let $V = W_2 \cup \ldots \cup W_n$, and let $\calW' = \calW \cup \{V\}$. Using the above remark and the inductive hypothesis, it will suffice to show that $\calW'$ is a good covering of $X$. Now $\calW'$ contains a pair of open sets $W_1$ and $V$ which cover $X$. We thereby reduce to the case $n=2$; using the above remark we can furthermore suppose that
$\calW = \{ W_1, W_2 \}$. 

We now wish to show that for every compact $K \subseteq X$, $\calF$ exhibits
$\calF(K)$ as the limit of $\calF | \Nerve(\calK_{\calW}(X))^{op}$. Let $P$ be the collection
of all pairs $K_1, K_2 \in \calK(X)$ such that $K_1 \subseteq W_1$, $K_2 \subseteq W_2$, and $K_1 \cup K_2 = K$. We observe that $P$ is a filtered when ordered by inclusion.
For $\alpha = (K_1, K_2) \in P$, let $\calK_{\alpha} = \{ K' \in \calK(X): (K' \subseteq K_1) \vee (K' \subseteq K_2) \}$. We note that $\calK_{\calW}(X) = \bigcup_{\alpha \in P} \calK_{\alpha}$.
Moreover, Theorem \ref{hollowtt} implies that for $\alpha = (K_1, K_2) \in P$, the inclusion
$ \Nerve \{ K_1, K_2, K_1 \cap K_2 \} \subseteq \Nerve(\calK_{\alpha})$ is cofinal.
Since $\calF$ is a $\calK$-sheaf, we deduce that $\calF$ exhibits
$\calF(K)$ as a limit of the diagram $\calF | \Nerve(\calK_{\alpha})^{op}$ for each
$\alpha \in P$. Using Proposition \ref{extet}, we deduce that $\calF(K)$ is a limit of
$\calF | \Nerve(\calK_{\calW}(X))^{op}$ if and only if $\calF(K)$ is a limit of the constant
diagram $\Nerve(P)^{op} \rightarrow \SSet$ taking the value $\calF(K)$. This is clear, since
$P$ is filtered so that the map $\Nerve(P) \rightarrow \Delta^0$ is cofinal by Theorem \ref{hollowtt}.
\end{proof}

\begin{theorem}\label{kuku}
Let $X$ be a locally compact Hausdorff space and $\calC$ a presentable $\infty$-category in which filtered colimits are left exact.
Let $\calF: \Nerve (\calK(X) \cup \calU(X))^{op} \rightarrow \calC$ be a presheaf on the partially ordered set $\calK(X) \cup \calU(X)$.  
The following conditions are equivalent:
\begin{itemize}
\item[$(1)$] The presheaf $\calF_{\calK} = \calF| \Nerve(\calK(X))^{op}$ is a $\calK$-sheaf and $\calF$ is a right Kan extension of $\calF_{\calK}$.
\item[$(2)$] The presheaf $\calF_{\calU} = \calF | \Nerve(\calU(X))^{op}$ is a sheaf and $\calF$ is a left Kan extension of $\calF_{\calU}$.
\end{itemize}
\end{theorem}

\begin{proof}
Suppose first that $(1)$ is satisfied. We first prove that $\calF$ is a left Kan extension of $\calF_{\calU}$. Let $K$ be a compact subset of $X$, and let $\calU_{K \subseteq}(X) = \{ U \in \calU(X): K \subseteq U \}$. Consider the diagram
$$ \xymatrix{ \Nerve(\calU_{K \subseteq}(X))^{op} \ar[r]^-{p} \ar[d] & \Nerve (\calU_{K \subseteq}(X) \cup \calK_{K \Subset}(X))^{op} \ar[d] & \Nerve(\calK_{K \Subset}(X))^{op} \ar[d] \ar[l]_-{p'} \\
\Nerve(\calU_{K \subseteq}(X)^{op})^{\triangleright} \ar[r] \ar[ddr]^{\psi} & 
\Nerve (\calU_{K \subseteq}(X)) \cup \calK_{K \Subset}(X))^{op})^{\triangleright} \ar[d] & \Nerve (\calK_{K \Subset}^{op})^{\triangleright} \ar[ddl]_{\psi'} \ar[l] \\
& \Nerve (\calU(X) \cup \calK(X))^{op} \ar[d]^{\calF} & \\
& \calC. & }$$
We wish to prove that $\psi$ is a colimit diagram. Since $\calF_{\calK}$ is a $\calK$-sheaf, we deduce that $\psi'$ is a colimit diagram. It therefore suffices to check that $p$ and $p'$ are cofinal. 
According to Theorem \ref{hollowtt}, it suffices to show that for every $Y \in \calU_{K \subseteq}(X) \cup \calK_{K \Subset}(X)$, the partially ordered sets $\{ K' \in \calK(X): K \Subset K' \subseteq Y \}$ and $\{ U \in \calU(X): K \subseteq U \subseteq Y \}$ have contractible nerves. We now observe that both of these partially ordered sets is filtered, since they are nonempty and stable under finite unions.

We now show that $\calF_{\calU}$ is a sheaf. Let $U$ be an open subset of $X$ and let $\calW$ be a sieve which covers $U$. Let $\calK_{\subseteq U}(X) = \{ K \in \calK(X): K \subseteq U \}$ and let
$\calK_{\calW}(X) = \{ K \in \calK(X): (\exists W \in \calW) [ K \subseteq W ] \}$. 
We wish to prove that the diagram 
$$ \Nerve(\calW^{op})^{\triangleleft} \rightarrow \Nerve(\calU(X))^{op} \stackrel{\calF_{\calU}} \rightarrow \SSet$$ is a limit. Using Theorem \ref{hollowtt}, we deduce that the inclusion
$\Nerve(\calW) \subseteq \Nerve (\calW \cup \calK_{\calW}(X) )$ is cofinal. It therefore suffices to prove that $ \calF | (\calW \cup \calK_{\calW}(X) \cup \{U \})^{op}$ is a right Kan extension of
$\calF | (\calW \cup \calK_{\calW}(X))^{op}$. Since $\calF|(\calW \cup \calK_{\calW}(X))^{op}$ is a right Kan extension of $\calF|\calK_{\calW}(X)^{op}$ by assumption, it suffices to prove that $\calF| (\calW \cup \calK_{\calW}(X) \cup \{U\})^{op}$ is a right Kan extension of $\calF| \calK_{\calW}(X)^{op}$. This is clear at every object distinct from $U$; it will therefore suffice
to prove that $\calF| (\calK_{\calW}(X) \cup \{U\})^{op}$ is a right Kan extension of
$\calF| \calK_{\calW}(X)^{op}$.

By assumption, $\calF | \Nerve (\calK_{\subseteq U}(X) \cup \{U\})^{op}$ is a right Kan extension of
$\calF| \Nerve(\calK_{\subseteq U}(X))^{op}$ and Lemma \ref{noodlesoup} implies that
$\calF| \Nerve(\calK_{\subseteq U}(X))^{op}$ is a right Kan extension of $\calF| \Nerve (\calK_{\calW}(X))^{op}$. Invoking Proposition \ref{acekan}, we deduce that
$\calF| \Nerve (\calK_{\calW}(X) \cup \{U\})^{op}$ is a right Kan extension of 
$\calF| \Nerve (\calK_{\calW}(X))^{op}$. This shows that $\calF_{\calU}$ is a sheaf, and completes the proof that $(1) \Rightarrow (2)$.

Now suppose that $\calF$ satisfies $(2)$. We first verify that $\calF_{\calK}$ is a $\calK$-sheaf. 
The space $\calF_{\calK}(\emptyset) = \calF_{\calU}(\emptyset)$ is contractible because $\calF_{\calU}$ is a sheaf (and because the empty sieve is a covering sieve on $\emptyset \subseteq X$). Suppose next that $K$ and $K'$ are compact subsets of $X$. We wish to prove that the diagram 
$$ \xymatrix{ \calF( K \cup K' ) \ar[r] \ar[d] & \calF(K) \ar[d] \\
\calF(K') \ar[r] & \calF(K \cap K') }$$
is a pullback in $\SSet$. Let us denote this diagram by $\sigma: \Delta^1 \times \Delta^1 \rightarrow \SSet$. Let $P$ be the set of all pairs $U,U' \in \calU(X)$ such that $K \subseteq U$ and $K' \subseteq U'$. The functor $\calF$ induces a map $\sigma_{P}: \Nerve(P^{op})^{\triangleright} \rightarrow \SSet^{\Delta^1 \times \Delta^1}$, which carries each pair $(U,U')$ to the diagram
$$ \xymatrix{ \calF( U \cup U' ) \ar[r] \ar[d] & \calF(U) \ar[d] \\
\calF(U') \ar[r] & \calF(U \cap U') }$$
and carries the cone point to $\sigma$. Since $\calF_{U}$ is a sheaf, each $\sigma_P(U,U')$ is
a pullback diagram in $\calC$. Since filtered colimits in $\calC$ are left exact, it will suffice to show that $\sigma_P$ is a colimit diagram. By Proposition \ref{limiteval}, it suffices to show that each of the four maps
$$ \Nerve(P^{op})^{\triangleright} \rightarrow \SSet$$, given by evaluating $\sigma_P$ at the four vertices of $\Delta^1 \times \Delta^1$, is a colimit diagram. We will treat the case of the final vertex; the other cases are handled in the same way. Let $Q = \{ U \in \calU(X): K \cap K' \subseteq U\}$. T
We are given a map $g: \Nerve(P^{op})^{\triangleright} \rightarrow \SSet$ which admits a factorization
$$ \Nerve(P^{op})^{\triangleright} \stackrel{g''}{\rightarrow} \Nerve(Q^{op})^{\triangleright} \stackrel{g'}{\rightarrow}
\Nerve (\calU(X) \cup \calK(X))^{op} \stackrel{\calF}{\rightarrow} \calC.$$
Since $\calF$ is a left Kan extension of $\calF_{\calU}$, the diagram $\calF \circ g''$ is a colimit.
It therefore suffices to show that $g''$ induces a cofinal map $\Nerve(P)^{op} \rightarrow \Nerve (Q)^{op}$. Using Theorem \ref{hollowtt}, it suffices to prove that for every $U'' \in Q$, the
partially ordered set $P_{U''} = \{ (U,U') \in P: U \cap U' \subseteq U'' \}$ has contractible nerve. It now suffices to observe that $P^{op}_{U''}$ is filtered (since $P_{U''}$ is nonempty and stable under intersections). 

We next show that for any compact subset $K \subseteq X$, the map
$$ \Nerve(\calK_{K \Subset}(X)^{op})^{\triangleright} \rightarrow \Nerve (\calK(X) \cup \calU(X))^{op} \stackrel{\calF}{\rightarrow} \calC$$
is a colimit diagram. Let $\calV = \calU(X) \cup \calK_{K \Subset}(X)$, and let $\calV' = \calV \cup \{K\}$. It follows from Proposition \ref{acekan} that $\calF | \Nerve(\calV)^{op}$ and
$\calF | \Nerve(\calV')^{op}$ are left Kan extensions of $\calF | \Nerve (\calU(X))^{op}$, so
that $\calF | \Nerve (\calV')^{op}$ is a left Kan extension of $\calF| \Nerve(\calV)^{op}$. Therefore the diagram
$$ ( \Nerve ( \calK_{K \Subset}(X) \cup \{ U \in \calU(X): K \subseteq U \} )^{op})^{\triangleright}
\rightarrow \Nerve( \calK(X) \cup \calU(X))^{op} \stackrel{\calF}{\rightarrow} \calC $$
is a colimit. It therefore suffices to show that the inclusion
$$ \Nerve (\calK_{K \Subset}(X))^{op} \subseteq \Nerve (\calK_{K \Subset}(X) \cup \{ U \in \calU(X): K\subseteq U\} )^{op}$$
is cofinal. Using Theorem \ref{hollowtt}, we are reduced to showing that if 
$Y \in \calK_{K \Subset}(X) \cup \{ U \in \calU(X): K \subseteq U\}$, then the nerve of the partially ordered set $R = \{ K' \in \calK(X): K \Subset K' \subset Y \}$ is weakly contractible. It now suffices to observe that $R^{op}$ is filtered, since $R$ is nonempty and stable under intersections. This completes the proof that $\calF_{\calK}$ is a $\calK$-sheaf.

We now show that $\calF$ is a right Kan extension of $\calF_{\calK}$. Let $U$ be an open subset of $X$, and for $V \in \calU(X)$ write $V \Subset U$ if the closure $\overline{V}$ is compact and contained in $U$. Let $\calU_{\Subset U}(X) = \{ V \in \calU(X): V \Subset U\}$,
and consider the diagram
$$ \xymatrix{ \Nerve (\calU_{\Subset U}(X))^{op} \ar[r]^-{f} \ar[d] & \Nerve ( \calU_{\Subset U}(X) \cup \calK_{\subseteq U}(X))^{op} \ar[d] & \Nerve (\calK_{\subseteq U}(X))^{op} \ar[d] \ar[l]_-{f'} \\
\Nerve (\calU_{\Subset U}(X)^{op})^{\triangleleft} \ar[ddr]^{\phi} & 
\Nerve ( \calU_{\Subset U}(X) \cup \calK_{\subseteq U}(X))^{op})^{\triangleleft} \ar[d] & 
\Nerve (\calK_{\subseteq U}(X)^{op})^{\triangleleft} \ar[ddl]_{\phi'} \\
& \Nerve (\calK(X) \cup \calU(X))^{op} \ar[d]^{\calF} & \\
& \calC. & }$$
We wish to prove that $\phi'$ is a limit diagram. Since the sieve $\calU_{\Subset U}(X)$ covers $U$ and $\calF_{\calU}$ is a sheaf, we conclude that $\phi$ is a limit diagram. It therefore suffices to prove that $f^{op}$ and $(f')^{op}$ are cofinal maps of simplicial sets. According to Theorem \ref{hollowtt}, it suffices to prove that if $Y \in \calK_{\subseteq U}(X) \cup \calU_{\Subset U}(X)$, then the partially ordered sets $\{ V \in \calU(X): Y \subseteq V \Subset U \}$ and
$\{ K \in \calK(X): Y \subseteq K \subseteq U \}$ have weakly contractible nerves. We now observe that both of these partially ordered sets are filtered (since they are nonempty and stable under unions). This completes the proof that $\calF$ is a right Kan extension of $\calF_{\calK}$.
\end{proof}

\begin{corollary}\label{streeem}
Let $X$ be a locally compact topological space and $\calC$ a presentable $\infty$-category in which filtered colimits are left exact. Let
$$\Shv_{\calKU}(X;\calC) \subseteq \Fun(\Nerve (\calK(X) \cup \calU(X))^{op}, \calC)$$
be the full subcategory spanned by those presheaves which satisfy the equivalent conditions of Theorem \ref{kuku}. Then the restriction functors $$ \Shv(X;\calC) \leftarrow \Shv_{\calKU}(X;\calC) \rightarrow \Shv_{\calK}(X;\calC)$$\index{not}{ShvcalKUX@$\Shv_{\calKU}(X;\calC)$}
are equivalences of $\infty$-categories.
\end{corollary}

\begin{corollary}\label{compactprop}
Let $X$ be a compact Hausdorff space. Then the global sections functor
$\Gamma: \Shv(X) \rightarrow \SSet$ is a proper morphism of $\infty$-topoi.
\end{corollary}

\begin{proof}
The existence of fiber products $\Shv(X) \times_{\SSet} \calY$ in
$\RGeom$ follows from Theorem \ref{conprod}. It will therefore suffice to prove that for any (homotopy) Cartesian rectangle
$$ \xymatrix{ \calX'' \ar[r] \ar[d] & \calX' \ar[r] \ar[d] & \Shv(X) \ar[d] \\
\calY'' \ar[r]^{f_{\ast}} & \calY' \ar[r] & \SSet,}$$
the square on the left is left adjointable. Using Theorem \ref{conprod}, we can
identify the square on the left with
$$ \xymatrix{ \Shv(X; \calY'') \ar[r] \ar[d] & \Shv(X; \calY') \ar[d] \\
\calY'' \ar[r]^{f_{\ast}} & \calY', }$$ where the vertical morphisms are given by taking global sections.

Choose a correspondence $\calM$ from $\calY'$ to $\calY''$ which is associated to the functor $f_{\ast}$. Since $f_{\ast}$ admits a left adjoint $f^{\ast}$, the projection $\calM \rightarrow \Delta^1$ is both a Cartesian fibration and a coCartesian fibration. For every simplicial
set $K$, let $\calM_{K} = \Fun(K,\calM) \times_{ \Fun(K,\Delta^1) } \Delta^1$. Then $\calM_{K}$ determines a correspondence from $\Fun(K,\calY')$ to $\Fun(K,\calY'')$. Using Proposition \ref{doog}, we
conclude that $\calM_{K} \rightarrow \Delta^1$ is both a Cartesian and a coCartesian fibration,
and that it is associated to the functors given by composition with $f_{\ast}$ and $f^{\ast}$.

Before proceeding further, let us adopt the following convention for the remainder of the proof:
given a simplicial set $Z$ with a map $q: Z \rightarrow \Delta^1$, we will say that an edge
of $Z$ is {\it Cartesian} or {\it coCartesian} if it is $q$-Cartesian or $q$-coCartesian, respectively.
The map $q$ to which we are referring should be clear from context.

Let $\calM_{\calU}$ denote the full subcategory of $\calM_{ \Nerve(\calU(X))^{op}}$ whose objects
correspond to {\em sheaves} on $X$ (with values in either $\calY'$ or $\calY''$). Since
$f_{\ast}$ preserves limits, composition with $f_{\ast}$ carries $\Shv(X; \calY'')$ into
$\Shv(X; \calY')$. We conclude that the projection $\calM_{\calU} \rightarrow \Delta^1$
is a Cartesian fibration, and that the inclusion $\calM_{\calU} \subseteq \calM_{ \Nerve (\calU(X))^{op}}$
preserves Cartesian edges.

Similarly, we define $\calM_{\calK}$ to be the full subcategory of $\calM_{ \Nerve(\calK(X))^{op} }$
whose objects correspond to $\calK$-sheaves on $X$ (with values in either $\calY'$ or
$\calY''$). Since $f^{\ast}$ preserves finite limits and filtered colimits, composition with $f^{\ast}$ carries $\Shv_{\calK}(X; \calY')$ into $\Shv_{\calK}(X; \calY'')$. It follows that the projection
$\calM_{\calK} \rightarrow \Delta^1$ is a coCartesian fibration, and that the inclusion
$\calM_{\calK} \subseteq \calM_{ \Nerve(\calU(X))^{op}}$ preserves coCartesian edges.

Now let $\calM'_{\calKU} = \calM_{ \Nerve (\calK(X) \cup \calU(X))^{op}}$ and let
$\calM_{\calKU}$ be the full subcategory of $\calM'_{\calKU}$
spanned by the objects of $\Shv_{\calKU}(X; \calY')$ and $\Shv_{\calKU}(X; \calY'')$. 
We have a commutative diagram
$$ \xymatrix{ & \calM_{\calKU} \ar[dr]^{\phi_{\calU}} \ar[dl]^{\phi_{\calK}} & \\
\calM_{\calU} \ar[dr]^{\Gamma_{\calU}} & & \calM_{\calK} \ar[dl]^{\Gamma_{\calK}} \\
& \calM & }$$
where $\Gamma_{\calU}$ and $\Gamma_{\calK}$ denote the global sections functors (given by evaluation at $X \in \calU(X) \cap \calK(X)$). According to Remark \ref{toadcatcher}, to complete the proof it will suffice to show that $\calM_{\calU} \rightarrow \Delta^1$ is a coCartesian fibration, and that $\Gamma_{\calU}$ preserves both Cartesian and coCartesian edges. It is clear that
$\Gamma_{\calU}$ preserves Cartesian edges, since it is a composition of maps
$$ \calM_{\calU} \subseteq \calM_{ \Nerve( \calU(X))^{op} } \rightarrow \calM$$
which preserve Cartesian edges. Similarly, we already know that $\calM_{\calK} \rightarrow \Delta^1$ is a coCartesian fibration, and that $\Gamma_{\calK}$ preserves coCartesian edges.
To complete the proof, it will therefore suffice to show that $\phi_{\calU}$ and $\phi_{\calK}$ are
equivalences of $\infty$-categories. We will give the argument for $\phi_{\calU}$; the proof in the case of $\phi_{\calK}$ is identical and left to the reader.

According to Corollary \ref{streeem}, the map $\phi_{\calU}$ induces equivalences
$$ \Shv_{\calKU}(X; \calY') \rightarrow \Shv(X; \calY')$$
$$ \Shv_{\calKU}(X; \calY'') \rightarrow \Shv(X; \calY'')$$ 
after passing to the fibers over either vertex of $\Delta^1$.
We will complete the proof by applying Corollary \ref{usefir}. In order to do so, we must verify that
$p: \calM_{\calKU} \rightarrow \Delta^1$ is a Cartesian fibration, and that $\phi_{\calU}$ preserves Cartesian edges.

To show that $p$ is a Cartesian fibration, we begin with an arbitrary
$\calF \in \Shv_{\calKU}(X; \calY'')$. Using Proposition \ref{doog}, we conclude the existence of a $p'$-Cartesian morphism $\alpha: \calF' \rightarrow \calF$, where $p'$ denotes the projection
$\calM'_{\calKU}$ and $\calF' = \calF \circ p_{\ast} \in
\Fun( \Nerve(\calK(X) \cup \calU(X))^{op}, \calY')$. Since $p_{\ast}$ preserves limits, we conclude that
$\calF' | \Nerve(\calU(X))^{op}$ is a sheaf on $\calX$ with values in $\calY'$; however,
$\calF'$ is not necessarily a left Kan extension of $\calF' | \Nerve(\calU(X))^{op}$. 
Let $\calC$ denote the full subcategory of $\Fun(\Nerve (\calK(X) \cup \calU(X))^{op}, \calY')$
spanned by those functors $\calG: \Nerve (\calK(X) \cup \calU(X))^{op}$ which are left Kan extensions of $\calG | \Nerve(\calU(X))^{op}$, and $s$ a section of the trivial fibration
$\calC \rightarrow (\calY')^{ \Nerve(\calU(X))^{op}}$, so that $s$ is a left adjoint to the
restriction map $r: \calM'_{\calKU} \rightarrow (\calY')^{ \Nerve(\calU(X))^{op} }$. 
Let $\calF'' = (s \circ r) \calF'$ be a left Kan extension of $\calF' | \Nerve(\calU(X))^{op}$. Then
$\calF''$ is an initial object of the fiber $\calM'_{\calKU} \times_{ \Fun(\Nerve(\calU(X))^{op}, \calY')}
\{ \calF' | \Nerve(\calU(X))^{op} \}$, so that there exists a map $\beta: \calF'' \rightarrow \calF'$
which induces the identity on $\calF'' | \Nerve(\calU(X))^{op} = \calF' | \Nerve(\calU(X))^{op}$. 

Let
$\sigma: \Delta^2 \rightarrow \calM'_{\calKU}$ classify
a diagram
$$ \xymatrix{ & \calF' \ar[dr]^{\alpha} & \\
\calF'' \ar[ur]^{\beta} \ar[rr]^{\gamma} & & \calF, }$$
so that $\gamma$ is a composition of $\alpha$ and $\beta$. It is easy
to see that $\phi_{\calU}(\gamma)$ is a Cartesian edge of $\calM_{\calU}$ (since it is a composition of a Cartesian edge with an equivalence in $\Shv(X; \calY')$). We claim that $\gamma$ is $p$-Cartesian. To prove this, consider the diagram
$$ \xymatrix{ \Shv_{\calKU}(X; \calY') \times_{ \calM'_{\calKU}} (\calM'_{\calKU})_{/\sigma} \ar[d]^{\theta_0} \ar[r]^-{\eta'} & (\calM_{\calKU})_{\gamma} \ar[dd]^{\eta} \\
(\Shv_{\calKU}(X; \calY'))_{/\beta}
\times_{ \Shv_{\calKU}(X; \calY')_{/ \calF'}} (\calM'_{\calKU})_{/\alpha}
\ar[d]^{\theta_1} & \\ 
\Shv_{\calKU}(X; \calY') \times_{ \calM'_{\calKU} } (\calM'_{\calKU})_{/\alpha} \ar[r]^-{\theta_2} &
\Shv_{\calKU}(X; \calY') \times_{ \calM_{\calKU} }
(\calM_{\calKU})_{/\calF}. }$$
We wish to show that $\eta$ is a trivial fibration. Since $\eta$ is a right fibration, it suffices
to show that the fibers of $\eta$ are contractible. The map $\eta'$ is a trivial fibration
(since the inclusion $\Delta^{ \{0,2\} } \subseteq \Delta^2$ is right anodyne), so it will suffice to prove that $\eta \circ \eta'$ is a trivial fibration. In view of the commutativity of the diagram, it will suffice to show that $\theta_0$, $\theta_1$, and $\theta_2$ are trivial fibrations. The triviality of $\theta_0$ follows from the fact that the horn inclusion $\Lambda^2_1 \subseteq \Delta^2$ is right anodyne.
The triviality of $\theta_2$ follows from the fact that $\alpha$ is $p'$-Cartesian. Finally, we observe that $\theta_1$ is a pullback of the map $\theta'_1: \Shv_{\calKU}(X; \calY')_{/\beta} \rightarrow
\Shv_{\calKU}(X; \calY')_{/\calF'}$. Let $\calC = (\calY')^{ \Nerve (\calK(X) \cup \calU(X))^{op} }$.
To prove that $\theta'_1$ is a trivial fibration, we must show that
for every $\calG \in \Shv_{\calKU}$, composition with $\beta$ induces a homotopy equivalence
$$ \bHom_{ \calC}( \calG, \calF'' ) \rightarrow \bHom_{\calC}(\calG, \calF').$$
Without loss of generality, we may suppose that $\calG = s( \calG')$, where
$\calG' \in \Shv(X; \calY')$; now we simply invoke the adjointness of $s$ with the
restriction functor $r$ and the observation that $r(\beta)$ is an equivalence.
\end{proof}

\begin{corollary}\label{compactcase}
Let $X$ be a compact Hausdorff space.
The global sections functor $\Gamma: \Shv(X) \rightarrow \SSet$ preserves filtered colimits.
\end{corollary}

\begin{proof}
Applying Theorem \ref{kuku}, we can replace $\Shv(X)$ by $\Shv_{\calK}(X)$. Now observe that the full subcategory $\Shv_{\calK}(X) \subseteq \calP( \Nerve(\calK(X))^{op})$ is stable under filtered colimits. We thereby reduce to proving that the evaluation functor $\calP( \Nerve(\calK(X))^{op} ) \rightarrow \SSet$ commutes with filtered colimits, which follows from Proposition \ref{limiteval}.
Alternatively, one can apply Corollary \ref{streeem} and Remark \ref{swurk}.
\end{proof}

\begin{remark}
One can also deduce Corollary \ref{compactcase} using the geometric model for $\Shv(X)$ introduced in \S \ref{paracompactness}. Using the characterization of properness in terms of filtered colimits described in Remark \ref{swurk}, one can formally deduce Corollary \ref{compactprop} from
Corollary \ref{compactcase}. This leads to another proof of the proper base change theorem, which does not make use of Theorem \ref{kuku} or the other ideas of this section. However, this alternative proof is considerably more difficult than the one described here, since it requires a rigorous justification of Remark \ref{swurk}. We also note that Theorem \ref{kuku} and Corollary \ref{streeem} are interesting in their own right, and could conceivably be applied in other contexts.
\end{remark}

\subsection{Sheaves on Coherent Spaces}\label{cohthm}\index{gen}{topological space!coherent}\index{gen}{coherent topological space}

Theorem \ref{kuku} has an analogue in the setting of coherent topological spaces which is somewhat easier to prove. First, we need the analogue of Lemma \ref{noodlesoup}:

\begin{lemma}\label{oldtime}\index{not}{Ucal0X@$\calU_0(X)$}
Let $X$ be a coherent topological space, let $\calU_0(X)$ denote the collection
of compact open subsets of $X$, and let $\calF: \Nerve(\calU_0(X))^{op} \rightarrow \calC$
be a presheaf taking values in an $\infty$-category $\calC$, having the following properties:
\begin{itemize}
\item[$(1)$] The object $\calF(\emptyset) \in \calC$ is final.
\item[$(2)$] For every pair of compact open sets $U, V \subseteq X$, the diagram
$$ \xymatrix{ \calF( U \cap V) \ar[r] \ar[d] & \calF(U) \ar[d] \\
\calF(V) \ar[r] & \calF(U \cup V) }$$
is a pullback.
\end{itemize}
Let $\calW$ be a covering of $X$ by compact open subsets, and let $\calU_{1}(X) \subseteq \calU_{0}(X)$ be collection of all compact open subsets of $X$ which are contained in some element of $\calW$. Then $\calF$ is a right Kan extension of $\calF | \Nerve(\calU_1(X))^{op}$. 
\end{lemma}

\begin{proof}
The proof is similar to that of Lemma \ref{noodlesoup}, but slightly easier.
Let us say that a covering $\calW$ of a coherent topological space $X$ by compact open subsets
is {\it good} if it satisfies the conclusions of the Lemma. We observe that $\calW$ automatically has a finite subcover. We will prove, by induction on $n \geq 0$, that if $\calW$ is collection of open subsets of a locally coherent topological space $X$ such that there exist $W_1, \ldots, W_n \in \calW$ with
$W_1 \cup \ldots \cup W_n = X$, then $\calW$ is a good covering of $X$. If $n = 0$, then
$X = \emptyset$. In this case, we must prove that 
$\calF(\emptyset)$ is final, which is one of our assumptions.

Suppose that $\calW \subseteq \calW'$ are coverings of $X$ by compact open sets, and that for every $W' \in \calW'$ the induced covering $\{ W \cap W': W \in \calW \}$ is a good covering of $W'$. It then follows from Proposition \ref{acekan} that $\calW'$ is a good covering of $X$ if and only if $\calW$ is a good covering of $X$.

Now suppose $n > 0$. Let $V = W_2 \cup \ldots \cup W_n$, and let $\calW' = \calW \cup \{V\}$. Using the above remark and the inductive hypothesis, it will suffice to show that $\calW'$ is a good covering of $X$. Now $\calW'$ contains a pair of open sets $W_1$ and $V$ which cover $X$. 
We thereby reduce to the case $n=2$; using the above remark we can furthermore suppose that
$\calW = \{ W_1, W_2 \}$. 

We now wish to show that for every compact $U \subseteq X$, $\calF$ exhibits
$\calF(U)$ as the limit of $\calF | \Nerve(\calU_1(X)_{/U})^{op}$. Without loss of generality,
we may replace $X$ by $U$ and thereby reduce to the case $U=X$. Let
$\calU_2(X) = \{ W_1, W_2, W_1 \cap W_2 \} \subseteq \calU_1(X)$. Using Theorem
\ref{hollowtt}, we deduce that the inclusion $\Nerve(\calU_2(X)) \subseteq \Nerve(\calU_1(X))$ is cofinal. Consequently, it suffices to prove that $\calF(X)$ is the limit of the diagram
$\calF | \Nerve(\calU_2(X) )^{op}$. In other words, we must show that the diagram
$$ \xymatrix{ \calF(X) \ar[r] \ar[d] & \calF(W_1) \ar[d] \\
\calF(W_2) \ar[r] & \calF(W_1 \cap W_2) }$$
is a pullback in $\calC$, which is true by assumption.
\end{proof}

\begin{theorem}\label{surm}
Let $X$ be a coherent topological space, and let $\calU_0(X) \subseteq \calU(X)$ denote the collection of compact open subsets of $X$. Let $\calC$ be an $\infty$-category which admits small limits.
The restriction map 
$$ \Shv(\calX; \calC) \rightarrow \Fun( \Nerve (\calU_0(X))^{op}, \calC)$$
is fully faithful, and its essential image consists of precisely those functors
$\calF_0: \Nerve( \calU_0(X))^{op} \rightarrow \calC$ satisfying the following conditions:
\begin{itemize}
\item[$(1)$] The object $\calF_0(\emptyset) \in \calC$ is final.
\item[$(2)$] For every pair of compact open sets $U, V \subseteq X$, the diagram
$$ \xymatrix{ \calF_0( U \cap V) \ar[r] \ar[d] & \calF_0(U) \ar[d] \\
\calF_0(V) \ar[r] & \calF_0(U \cup V) }$$
is a pullback.
\end{itemize}
\end{theorem}

\begin{proof}
Let $\calD \subseteq \calC^{ \Nerve(\calU(X))^{op} }$ be the full subcategory spanned by those presheaves $\calF: \Nerve( \calU(X))^{op} \rightarrow \calC$ which are right Kan extensions of
$\calF_0 = \calF | \Nerve(\calU_0(X))^{op}$, and such that $\calF_0$ satisfies conditions $(1)$ and $(2)$. According to Proposition \ref{lklk}, it will suffice to show that $\calD$ coincides with
$\Shv(X; \calC)$. 

Suppose that $\calF: \Nerve(\calU(X))^{op} \rightarrow \calC$ is a sheaf. We first show that $\calF$ is a right Kan extension of $\calF_0 = \calF| \Nerve(\calU_0(X))^{op}$. Let $U$ be an open subset of $X$, let $\calU(X)_{/U}^{(0)}$ denote the collection of compact open subsets of $U$, and let
$\calU(X)_{/U}^{(1)}$ denote the sieve generated by $\calU(X)_{/U}^{(0)}$. Consider the diagram

$$\xymatrix{ \Nerve(\calU(X)_{/U}^{(0)})^{\triangleright} \ar[drrr]^{f} \ar[r]^{i} &
\Nerve(\calU(X)_{/U}^{(1)})^{\triangleright} \ar[drr]^{f'} \ar[r] & 
\Nerve(\calU(X)_{/U})^{\triangleright} \ar[dr] \ar[r] & 
\Nerve(\calU(X)) \ar[d]^{\calF} \\
& & & \calC^{op}. }$$
We wish to prove that $f$ is a colimit diagram. Using Theorem \ref{hollowtt}, we deduce that the inclusion $\Nerve(\calU(X))^{(0)}_{/U} \subseteq \Nerve(\calU(X))^{(1)}_{/U}$ is cofinal.
It therefore suffices to prove that $f'$ is a colimit diagram. Since $\calF$ is a sheaf, it suffices to prove that $\calU(X)_{/U}^{(1)}$ is a covering sieve. In other words, we need to prove that $U$ is
a union of compact open subsets of $X$, which follows immediately from our assumption that
$X$ is coherent.

We next prove that $\calF_0$ satisfies $(1)$ and $(2)$. To prove $(1)$, we simply observe that
the empty sieve is a cover of $\emptyset$ and apply the sheaf condition. To prove $(2)$, we may assume without loss of generality that neither $U$ nor $V$ is contained in the other (otherwise the result is obvious). 
Let $\calU(X)^{(0)}_{/U \cup V}$ be the full subcategory spanned by $U$, $V$, and $U \cap V$, and let $\calU(X)^{(1)}_{/U \cup V}$ be the sieve on $U \cup V$ generated by $\calU(X)^{(0)}_{/U \cup V}$. As above, we have a diagram 
$$\xymatrix{ (\Nerve(\calU(X))_{/U \cup V}^{(0)})^{\triangleright} \ar[drrr]^{f} \ar[r]^{i} &
\Nerve(\calU(X)_{/U \cup V}^{(1)})^{\triangleright} \ar[drr]^{f'} \ar[r] & 
\Nerve(\calU(X)_{/U \cup V})^{\triangleright} \ar[dr] \ar[r] & 
\Nerve(\calU(X)) \ar[d]^{\calF} \\
& & & \calC^{op}, }$$
and we wish to show that $f$ is a colimit diagram. Theorem \ref{hollowtt} implies that
the inclusion $\Nerve(\calU(X))_{/U \cup V}^{(0)} \subseteq \Nerve(\calU(X))_{/U \cup V}^{(1)}$ is cofinal. It therefore suffices to prove that $f'$ is a colimit diagram, which follows from the sheaf condition since $\calU(X)^{(1)}_{/U \cup V}$ is a covering sieve. This completes the proof 
that $\Shv(X;\calC) \subseteq \calD$.

It remains to prove that $\calD \subseteq \Shv(X;\calC)$. In other words, we must show that
if $\calF$ is a right Kan extension of $\calF_0 = \calF | \Nerve(\calU_0(X))^{op}$, and
$\calF_0$ satisfies conditions $(1)$ and $(2)$, then $\calF$ is a sheaf. Let $U$ be an open subset of $X$, $\calU(X)_{/U}^{(0)}$ a sieve which covers $U$. Let $\calU_0(X)_{/U}$ denote the category of compact open subsets of $U$ and $\calU_0(X)^{(0)}_{/U}$ the category of compact
open subsets of $U$ which belong to the sieve $\calU(X)_{/U}^{(0)}$. We wish to prove
that $\calF(U)$ is a limit of $\calF | \Nerve(\calU(X)_{/U}^{(0)})^{op}$. We will in fact prove the slightly stronger assertion that $\calF | \Nerve (\calU(X)_{/U})^{op}$ is a right Kan extension of 
$\calF | \Nerve( \calU(X)_{/U}^{(0)})^{op}$.

We have a commutative diagram
$$ \xymatrix{ \calU_0(X)^{(0)}_{/U} \ar[r] \ar[d] & \calU_0(X)_{/U} \ar[d] \\
\calU(X)_{/U}^{(0)} \ar[r] & \calU(X)_{/U}. }$$
By assumption, $\calF$ is a right Kan extension of $\calF_0$. It follows that
$\calF | \Nerve (\calU(X)_{/U}^{(0)})^{op}$ is a right Kan extension of
$\calF | \Nerve (\calU_0(X)_{/U}^{(0)})^{op}$ and that
$\calF | \Nerve (\calU(X)_{/U})^{op}$ is a right Kan extension of
$\calF | \Nerve (\calU_0(X)_{/U})^{op}$. By the transitivity of Kan extensions
(Proposition \ref{acekan}), it will suffice to prove that $\calF | \Nerve(\calU_0(X)_{/U})^{op}$ is a right
Kan extension of $\calF | \Nerve (\calU_0(X)^{(0)}_{/U})^{op}$. This follows immediately
from Lemma \ref{oldtime}.
\end{proof}

\begin{corollary}\label{applyZR}
Let $X$ be a coherent topological space. Then the global sections functor
$\Gamma: \Shv(X) \rightarrow \SSet$ is a proper map of $\infty$-topoi.
\end{corollary}

\begin{proof}
Identical to the proof of Corollary \ref{compactprop}, using
Theorem \ref{surm} in place of Corollary \ref{streeem}.
\end{proof} 

\begin{corollary}
Let $X$ be a coherent topological space. Then the global sections functor
$$ \Gamma: \Shv(X) \rightarrow \SSet$$ commutes with filtered colimits.
\end{corollary}

\subsection{Cell-Like Maps}\label{celluj}

Recall that a topological space $X$ is an {\it absolute neighborhood retract} if $X$ is\index{gen}{topological space!absolute neighborhood retract}\index{gen}{absolute neighborhood retract}
metrizable and if for any closed immersion $X \hookrightarrow Y$ of $X$ in a metric space $Y$, there exists an open set $U \subseteq Y$ containing the image of $X$, such that the inclusion
$X \hookrightarrow U$ has a left inverse (in other words, $X$ is a {\em retract} of $U$).

Let $p: X \rightarrow Y$ be a continuous map between locally compact absolute neighborhood retracts. The map $p$ is said to be {\it cell-like} if $p$ is proper and each fiber $X_{y} = X \times_{Y} \{y\}$ has trivial shape (in the sense of Borsuk; see \cite{shapetheory} and \S \ref{shapesec}). The theory of cell-like maps plays an important role in geometric topology: we refer the reader to \cite{cellmap} for a discussion (and for several equivalent formulations of the condition that a map be cell-like).\index{gen}{cell-like!map of topological spaces}

The purpose of this section is to describe a class of geometric morphisms between $\infty$-topoi, which we will call {\it cell-like} morphisms. We will then compare our theory of cell-like morphisms with the classical theory of cell-like maps. We will also give a ``nonclassical'' example which arises in the theory of rigid analytic geometry.

\begin{definition}\label{celldef}\index{gen}{cell-like!map of $\infty$-topoi}
Let $p_{\ast}: \calX \rightarrow \calY$ be a geometric morphism of $\infty$-topoi. We will say that
$p_{\ast}$ is {\it cell-like} if it is proper and if the right adjoint $p^{\ast}$ (which is well-defined up to equivalence) is fully faithful.
\end{definition}

\begin{warning}
Many authors refer to a map $p: X \rightarrow Y$ of {\em arbitrary} compact metric spaces
as {\it cell-like} if each fiber $X_{y} = X \times_{Y} \{y\}$ has trivial shape. This condition
is generally {\em weaker} than the condition that $p_{\ast}: \Shv(X) \rightarrow \Shv(Y)$ be cell-like in the sense of Definition \ref{celldef}. However, the two definitions are equivalent provided that $X$ and $Y$ are sufficiently nice (for example, if they are locally compact absolute neighborhood retracts). Our departure from the classical terminology is perhaps justified by the 
fact that the class of morphisms introduced in Definition \ref{celldef} has good formal properties: for example, stability under composition.
\end{warning}

\begin{remark}\label{urjk}
Let $p_{\ast}: \calX \rightarrow \calY$ be a cell-like geometric morphism between $\infty$-topoi.
Then the unit map $\id_{\calY} \rightarrow p_{\ast} p^{\ast}$ is an equivalence of functors. It follows immediately that $p_{\ast}$ induces an equivalence of shapes $\Sh(\calX) \rightarrow \Sh(\calY)$ (see \S \ref{shapesec}.
\end{remark}

\begin{proposition}\label{fibersilly}
Let $p_{\ast}: \calX \rightarrow \calY$ be a proper morphism of $\infty$-topoi. Suppose that
$\calY$ has enough points. Then $p_{\ast}$ is cell-like if and only if, for every pullback diagram
$$ \xymatrix{ \calX' \ar[r] \ar[d] & \calX \ar[d]^{p_{\ast}} \\
\SSet \ar[r] & \calY }$$
in $\RGeom$, the $\infty$-topos $\calX'$ has trivial shape.
\end{proposition}

\begin{proof}
Suppose first that each fiber $\calX'$ has trivial shape. Let $\calF \in \calY$. We wish to show that the unit map $u: \calF \rightarrow p_{\ast} p^{\ast} \calF$ is an equivalence. Since $\calY$ has enough points, it suffices to show that for each point $q_{\ast}: \SSet \rightarrow \calY$, the map $q^{\ast} u$ is an equivalence in $\SSet$, where $q^{\ast}$ denotes a left adjoint to $q_{\ast}$.
Form a pullback diagram of $\infty$-topoi
$$ \xymatrix{ \calX' \ar[r] \ar[d]^{s_{\ast}} & \calX \ar[d]^{p_{\ast}} \\
\SSet \ar[r]^{q_{\ast}} & \calY. }$$
Since $p_{\ast}$ is proper, this diagram is left-adjointable. Consequently,
$q^{\ast} u$ can be identified with the unit map
$$ K \rightarrow s_{\ast} s^{\ast} K,$$
where $K = q^{\ast} \calF \in \SSet$. If $\calX'$ has trivial shape, then this map is an equivalence.

Conversely, if $p_{\ast}$ is cell-like, then the above argument shows that for every diagram
$$ \xymatrix{ \calX' \ar[r] \ar[d]^{s_{\ast}} & \calX \ar[d]^{p_{\ast}} \\
\SSet \ar[r]^{q_{\ast}} & \calY }$$
as above and every $\calF \in \calY$, the adjunction map
$$ K \rightarrow s_{\ast} s^{\ast} K$$ is an equivalence, where
$K = q^{\ast} \calF$. To prove that $\calX'$ has trivial shape, it will suffice to show that $q^{\ast}$ is essentially surjective. For this, we observe that since $\SSet$ is a final object in the $\infty$-category of $\infty$-topoi, there exists a geometric morphism
$r_{\ast}: \calY \rightarrow \SSet$ such that $r_{\ast} \circ q_{\ast}$ is homotopic to $\id_{\SSet}$.
It follows that $q^{\ast} \circ r^{\ast} \simeq \id_{\SSet}$. Since $\id_{\SSet}$ is essentially surjective, we conclude that $q^{\ast}$ is essentially surjective.
\end{proof}

\begin{corollary}\label{comb0}
Let $p: X \rightarrow Y$ be a map of paracompact topological spaces. Assume that $p_{\ast}$ is proper, and that $Y$ has finite covering dimension. Then $p_{\ast}: \Shv(X) \rightarrow \Shv(Y)$ is cell-like if and only if each fiber $X_y = X \times_{Y} \{y\}$ has trivial shape.
\end{corollary}

\begin{proof}
Combine Proposition \ref{fibersilly} with Corollary \ref{enuff}.
\end{proof}

\begin{proposition}\label{cell}
Let $p: X \rightarrow Y$ be a proper map of locally compact ANRs. The following conditions are equivalent:

\begin{itemize}
\item[$(1)$] The geometric morphism $p_{\ast}: \Shv(X) \rightarrow \Shv(Y)$ is cell-like.
\item[$(2)$] For every open subset $U \subseteq Y$, the restriction map
$X \times_{Y} U \rightarrow U$ is a homotopy equivalence.
\item[$(3)$] Each fiber $X_{y} = X \times_{Y} \{y\}$ has trivial shape.
\end{itemize}
\end{proposition}

\begin{proof}
It is easy to see that if $p_{\ast}$ is cell-like, then each of the restrictions $p': X \times_{Y} U \rightarrow U$ induces a cell-like geometric morphism. According to Remark \ref{urjk}, 
$p'_{\ast}$ is a shape equivalence, and therefore a homotopy equivalence by Proposition \ref{parashape}. Thus $(1) \Rightarrow (2)$.

We next prove that $(2) \Rightarrow (1)$. 
Let $\calF \in \Shv(Y)$, and let $u: \calF \rightarrow p_{\ast} p^{\ast} \calF$ be a unit map; we wish to show that $u$ is an equivalence. It will suffice to show that the induced map
$\calF(U) \rightarrow (p_{\ast} p^{\ast} \calF)(U)$ is an equivalence in $\SSet$ for each paracompact open subset $U \subseteq Y$. Replacing $Y$ by $u$, we may reduce to the problem of showing that the map $\calF(Y) \rightarrow (p^{\ast} \calF)(X)$ is a homotopy equivalence. According
to Corollary \ref{wamain}, we may assume that $\calF$ is the simplicial nerve of $\Sing_{Y} \widetilde{Y}$, where $\widetilde{Y}$ is a fibrant-cofibrant object of $\Top_{/Y}$. According to
Proposition \ref{basechang}, we may identify $p^{\ast} \calF$ with $\Sing_{X} \widetilde{X}$, where
$\widetilde{X} = X \times_{Y} \widetilde{Y}$. It therefore suffices to prove that the induced map of simplicial function spaces
$$ \bHom_{Y}( Y, \widetilde{Y}) \rightarrow \bHom_{X}(X, \widetilde{X}) \simeq \bHom_{Y}(X, \widetilde{Y})$$
is a homotopy equivalence, which follows immediately from $(2)$. 

The implication $(1) \Rightarrow (3)$ follows from the proof of Proposition \ref{cell}, and the implication $(3) \Rightarrow (2)$ is classical (see \cite{haver}).
\end{proof}

\begin{remark}
It is possible to prove the following generalization of Proposition \ref{cell}: a proper geometric morphism $p_{\ast}: \calX \rightarrow \calY$ is cell-like if and only if, for each
object $U \in \calY$, the associated geometric morphism
$\calX_{/p^{\ast} U} \rightarrow \calY_{/U}$ is a shape equivalence (and, in fact, it is only necessary to check this on a collection of objects $U \in \calY$ which generates $\calY$ under colimits). 
\end{remark}

\begin{remark}
Another useful property of the class of cell-like morphisms, which we will not prove here, is stability under base change: given a pullback diagram
$$ \xymatrix{ \calX' \ar[d]^{p'_{\ast}} \ar[r] & \calX \ar[d]^{p_{\ast}} \\
\calY' \ar[r] & \calY }$$
where $p_{\ast}$ is cell-like, $p'_{\ast}$ is also cell-like.
\end{remark}

If $p_{\ast}: \calX \rightarrow \calY$ is a cell-like morphism of $\infty$-topoi, then many properties of $\calY$ are controlled by the analogous properties of $\calX$. For example:

\begin{proposition}
Let $p_{\ast}: \calX \rightarrow \calY$ be a cell-like morphism of $\infty$-topoi. If
$\calX$ has homotopy dimension $\leq n$, then $\calY$ also has homotopy dimension $\leq n$.
\end{proposition}

\begin{proof}
Let $1_{\calY}$ be a final object of $\calY$, $U$ an $n$-connective object of $\calY$, and
$p^{\ast}$ a left adjoint to $p_{\ast}$. We wish to prove that $\Hom_{h \calY}(1_{\calY},U)$ is nonempty. Since $p^{\ast}$ is fully faithful, it will suffice to prove that
$\Hom_{\h{\calX}}( p^{\ast} 1_{\calY}, p^{\ast} U)$. We now observe that $p^{\ast} 1_{\calY}$ is a final object of $\calX$ (since $p$ is left exact), $p^{\ast}U$ is $n$-connective (Proposition \ref{inftychange}), and $\calX$ has homotopy dimension $\leq n$, so that
$\Hom_{\h{\calX}}( p^{\ast} 1_{\calY}, p^{\ast} U)$ is nonempty as desired.
\end{proof}

We conclude with a different example of a class of cell-like maps. We will assume in the following discussion that the reader is familiar with the basic ideas of rigid analytic geometry; for an account of this theory we refer the reader to \cite{rigidgeom}. Let
$K$ be field which is complete with respect to a non-Archimedean absolute value
$||_K: K \rightarrow \R$. Let $A$ be an affinoid algebra over $K$: that is, a quotient of an algebra of convergent power series (in several variables) with values in $K$. Let $X$ be
the rigid space associated to $A$. One can associate to $X$ two different ``underlying'' topological spaces:

\begin{itemize}
\item[$(ZR1)$] The category $\calC$ of rational open subsets of $X$ has a Grothendieck topology, given by admissible affine covers. The topos of sheaves of sets on $\calC$ is localic, and the underlying locale has enough points: it is therefore isomorphic to the locale of open subsets of a (canonically determined) topological space $X_{ZR}$, the {\it Zariski-Riemann} space of $X$.
 
\item[$(ZR2)$] In the case where $K$ is a {\em discretely} valued field with ring of integers $R$, one may define $X_{ZR}$ to be the inverse limit of the underlying spaces of all formal schemes
$\hat{X} \rightarrow \Spf R$ which have generic fiber $X$.
\item[$(ZR3)$] Concretely, $X_{ZR}$ can be identified with the set of all isomorphism classes of continuous multiplicative seminorms $||_A: A \rightarrow M \cup \{\infty\}$, where $M$ is an ordered abelian group containing the value group $|K^{\ast}|_{K} \subseteq \R^{\ast}$, and the restriction
of $||_A$ to $K$ is $||_K$.

\item[$(B1)$] The category of sheaves of sets on $\calC$ contains a full subcategory, consisting of {\em overconvergent} sheaves. This category is also a localic topos, and the underyling locale is isomorphic to the lattice of open subsets of a (canonically determined) topological space $X_{B}$, the {\it Berkovich space} of $X$. The category of overconvergent sheaves is a localization of the category of all sheaves on $\calC$, and there is an associated map of topological spaces
$p: X_{ZR} \rightarrow X_B$.\index{gen}{Berkovich space}

\item[$(B2)$] Concretely, $X_{B}$ can be identified with the set of all continuous multiplicative seminorms $||_A: A \rightarrow \R \cup \{\infty\}$ which extend $||_K$. It is equipped with the topology of pointwise convergence, and is a compact Hausdorff space. 
\end{itemize}

The relationship between the Zariski-Riemann space $X_{ZR}$ and the Berkovich space $X_{B}$ (or, more conceptually, the relationship between the category of {\em all} sheaves on $X$ and the category of {\em overconvergent} sheaves on $X$) is neatly summarized by the following result.

\begin{proposition}\label{rigidex}
Let $K$ be a field which is complete with respect to a non-Archimedean absolute value $||_K$, let $A$ be an affinoid algebra over $K$, let $X$ be the associated rigid space, and $p: X_{ZR} \rightarrow X_{B}$ the natural map. Then $p$ induces a
cell-like morphism of $\infty$-topoi $p_{\ast}: \Shv(X_{ZR}) \rightarrow \Shv(X_{B})$. 
\end{proposition}

Before giving the proof, we need an easy lemma.
Recall that a topological space $X$ is {\it irreducible} if every finite collection of nonempty open subsets of $X$ has nonempty intersections.\index{gen}{irreducible!topological space}\index{gen}{topological space!irreducible}

\begin{lemma}\label{silshape}
Let $X$ be an irreducible topological space. Then $\Shv(X)$ has trivial shape.
\end{lemma}

\begin{proof}
Let $\pi: X \rightarrow \ast$ be the projection from $X$ to a point, $\pi_{\ast}: \Shv(X) \rightarrow \Shv(\ast)$ the induced geometric morphism. We will construct a left adjoint $\pi^{\ast}$ to
$\pi_{\ast}$ such that the unit map $\id \rightarrow \pi_{\ast} \pi^{\ast}$ is an equivalence. 

We begin by defining $G: \calP(X) \rightarrow \calP(\ast)$ to be the functor
given by composition with $\pi^{-1}$, so that $G | \Shv(X) = \pi_{\ast}$. Let
$$i: \Nerve (\calU(X))^{op} \rightarrow \Nerve(\calU(\ast))^{op}$$
be defined so that
$$ i(U) = \begin{cases} \emptyset & \text{if } U = \emptyset \\
\{ \ast \} & \text{if } U \neq \emptyset, \end{cases}$$
and let $F: \calP(\ast) \rightarrow \calP(U)$ be given by composition with $i$.
We observe that $F$ is a left Kan extension functor, so that the identity map
$$ \id_{ \calP(\ast)} \rightarrow G \circ F$$
exhibits $F$ as a left adjoint to $G$. We will show that $F( \Shv(\ast) ) \subseteq \Shv(X)$.
Setting $\pi^{\ast} = F | \Shv(\ast)$, we conclude that the identity map
$$ \id_{ \Shv(\ast)} \rightarrow \pi_{\ast} \pi^{\ast} $$
is the unit of an adjunction between $\pi_{\ast}$ and $\pi^{\ast}$, which will complete the proof.

Let $\calU \subseteq \calU(X)$ be a sieve which covers the open set $U \subseteq X$.
We wish to prove that the diagram
$$p: \Nerve(\calU^{op})^{\triangleleft} \rightarrow \Nerve(\calU(X))^{op}
\stackrel{i}{\rightarrow} \Nerve(\calU(\ast))^{op} \stackrel{\calF}{\rightarrow} \SSet$$
is a limit. Let $\calU_0 = \{ V \in \calU: V \neq \emptyset \}$. Since
$\calF(\emptyset)$ is a final object of $\SSet$, $p$ is a limit if and only if
$p | \Nerve(\calU_0^{op})^{\triangleleft}$ is a limit diagram. If $U = \emptyset$,
then this follows from the fact that $\calF(\emptyset)$ is final in $\SSet$.
If $U \neq \emptyset$, then $p | \Nerve(\calU_0^{op})^{\triangleleft}$ is
a constant diagram, so it will suffice to prove that the simplicial set
$\Nerve(\calU_0)^{op}$ is weakly contractible. This follows from the observation
that $\calU_0^{op}$ is a filtered partially ordered set, since $\calU_0$ is nonempty and stable under finite intersections (because $X$ is irreducible). 
\end{proof}

\begin{proof}[Proof of Proposition \ref{rigidex}]
We first show that $p_{\ast}$ is a proper map of $\infty$-topoi. We note that $p$ factors as a composition
$$ X_{ZR} \stackrel{p'}{\rightarrow} X_{ZR} \times X_{B} \stackrel{p''}{\rightarrow} X_{B}.$$
The map $p'$ is a pullback of the diagonal map $X_{B} \rightarrow X_{B} \times X_{B}$. Since
$X_{B}$ is Hausdorff, $p'$ is a closed immersion. It follows $p'_{\ast}$ is a closed immersion of $\infty$-topoi (Corollary \ref{glad1}) and therefore a proper morphism (Proposition \ref{closeduse2}). It therefore suffices to prove that $p''$ is a proper map of $\infty$-topoi. We note the existence of a commutative diagram
$$ \xymatrix{ \Shv(X_{ZR} \times X_{B}) \ar[d]^{p''_{\ast}} \ar[r] & \Shv(X_{ZR}) \ar[d]^{g_{\ast}} \\
\Shv( X_{B}) \ar[r] & \Shv(\ast). }$$
Using Proposition \ref{cartmun}, we deduce that this is a homotopy Cartesian diagram of $\infty$-topoi. It therefore suffices to show that the global sections functor $g_{\ast}: \Shv(X_{ZR}) \rightarrow \Shv(\ast)$ is proper, which follows from Corollary \ref{applyZR}.

We now observe that the topological space $X_{B}$ is paracompact and has finite covering dimension (\cite{berkovich}, Corollary $3.2.8$), so that $\Shv(X_{B})$ has enough points (Corollary \ref{enuff}). According to Proposition \ref{fibersilly}, it suffices to show that for every fiber diagram
$$ \xymatrix{ \calX' \ar[r] \ar[d] & \Shv(X_{ZR}) \ar[d] \\
\Shv(\ast) \ar[r]^-{q_\ast} & \Shv(X_{B}), }$$
the $\infty$-topos $\calX'$ has trivial shape. Using Lemma \ref{eoi2}, we conclude that $q_{\ast}$ is necessarily induced by a homomorphism of locales $\calU(X_{B}) \rightarrow \calU(\ast)$, which corresponds to an irreducible closed subset of $X_{B}$. Since $X_{B}$ is Hausdorff, this subset consists of a single (closed) point $x$. Using Proposition \ref{closeduse2} and Corollary \ref{glad1}, we can identify $\calX'$ with the $\infty$-topos $\Shv(Y)$, where $Y = X_{ZR} \times_{X_B} \{x\}$.
We now observe that the topological space $Y$ is coherent and irreducible (it contains a unique ``generic'' point), so that $\Shv(Y)$ has trivial shape by Lemma \ref{silshape}.
\end{proof}

\begin{remark}
Let $p_{\ast}: \Shv(X_{ZR}) \rightarrow \Shv(X_{B})$ be as in Proposition \ref{rigidex}.
Then $p_{\ast}$ has a fully faithful left adjoint $p^{\ast}$. We might say that an object of
$\Shv(X_{ZR})$ is {\it overconvergent} if it belongs to the essential image of $p^{\ast}$; for sheaves of sets, this agrees with the classical terminology.\index{gen}{overconvergent sheaf}
\end{remark}

\begin{remark}
One can generalize Proposition \ref{rigidex} to rigid spaces which are not affinoid; we leave the details to the reader.
\end{remark}

