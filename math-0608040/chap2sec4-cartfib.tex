\section{Cartesian Fibrations}\label{cartfibsec}

\setcounter{theorem}{0}

Let $p: X \rightarrow S$ be an inner fibration of simplicial sets. Each fiber of $p$ is an $\infty$-category, and each edge $f: s \rightarrow s'$ of $S$ determines a correspondence between
the fibers $X_{s}$ and $X_{s'}$.
In this section, we would like to study the case in which each of these correspondences is associated to a functor $f^{\ast}: X_{s'} \rightarrow X_{s}$. 
Roughly speaking, we can attempt to construct $f^{\ast}$ as follows: for each vertex $y \in X_{s'}$, we choose an edge $\widetilde{f}: x \rightarrow y$ lifting $f$, and set $f^{\ast} y = x$. However, this recipe does not uniquely determine $x$, even up to equivalence, since there might be many different choices for $\widetilde{f}$. To get a good theory, we need to make a good choice of $\widetilde{f}$. More precisely, we should require that
$\widetilde{f}$ be a {\it $p$-Cartesian} edge of $X$. In \S \ref{universalmorphisms}, we will introduce the definition of $p$-Cartesian edges and study their basic properties. In particular, we will see that a $p$-Cartesian edge $\widetilde{f}$ is determined up to equivalence by its target $y$ and its image in $S$. Consequently, if there is a sufficient supply of $p$-Cartesian edges of $X$, then we can use the above prescription to define the functor $f^{\ast}: X_{s'} \rightarrow X_{s}$.
This leads us to the notion of a {\it Cartesian fibration}, which we will study in \S \ref{funkymid}. 

In \S \ref{slib}, we will establish a few basic stability properties of the class of Cartesian fibrations (we will discuss other results of this type in \S \ref{chap4}, after we have developed the language of marked simplicial sets). In \S \ref{slik} we will show that if $p: \calC \rightarrow \calD$ is a Cartesian fibration of $\infty$-categories, then we can reduce many questions about $\calC$ to similar questions about the base $\calD$ and about the fibers of $p$. This technique has many applications, which we will discuss in \S \ref{slim} and \S \ref{slin}. Finally, in \S \ref{bifib}, we will study the theory of {\it bifibrations}, which is useful for constructing examples of Cartesian fibrations.

\subsection{Cartesian Morphisms}\label{universalmorphisms}

Let $\calC$ and $\calC'$ be ordinary categories, and let $M: \calC^{op} \times \calC' \rightarrow \Set$ be a correspondence between them. Suppose that we wish to know whether or not $M$ arises as the correspondence associated to some functor $g: \calC' \rightarrow \calC$. This is the case if and only if, for each object $C' \in \calC'$, we can find an object $C \in \calC$ and a point $\eta \in M(C,C')$
having the property that the ``composition with $\eta$'' map 
$$\psi: \Hom_{\calC}(D,C) \rightarrow M(D,C')$$ is bijective, for all $D \in \calC$. Note that
$\eta$ may be regarded as a morphism in the category
$ \calC \star^{M} \calC' $. We will say that $\eta$ is a {\it Cartesian} morphism in $\calC \star^{M} \calC'$ if $\psi$ is bijective for each $D \in \calC$. The purpose of this section is to generalize this notion to the $\infty$-categorical setting and to establish its basic properties.

\begin{definition}\label{univedge}\index{gen}{morphism!$p$-Cartesian}\index{gen}{Cartesian edge}
Let $p: X \rightarrow S$ be an inner fibration of simplicial sets.
Let $f: x \rightarrow y$ be an edge in $X$. We shall say that $f$
is {\it $p$-Cartesian} if the induced map
$$ X_{/f} \rightarrow X_{/y} \times_{ S_{/p(y)} } S_{/p(f)}$$ 
is a trivial Kan fibration.
\end{definition}

\begin{remark}
Let $\calM$ be an ordinary category, and let $p: \Nerve(\calM) \rightarrow \Delta^1$ be a map (automatically an inner fibration), and let $f: x \rightarrow y$ be a morphism in $\calM$ which projects isomorphically onto $\Delta^1$. Then $f$ is $p$-Cartesian in the sense of Definition \ref{univedge} if and only if it is Cartesian in the classical sense.
\end{remark}

We now summarize a few of the formal properties of Definition \ref{univedge}:

\begin{proposition}\label{stuch}
\begin{itemize}
\item[$(1)$] Let $p: X \rightarrow S$ be an isomorphism of simplicial sets. Then every edge of $X$ is $p$-Cartesian.

\item[$(2)$] Suppose given a pullback diagram
$$ \xymatrix{ X' \ar[d]^{p'} \ar[r]^{q} & X \ar[d]^{p} \\
S' \ar[r] & S }$$
of simplicial sets, where $p$ $($and therefore also $p'${}$)$ is an inner fibration. Let
$f$ be an edge of $X'$. If $q(f)$ is $p$-Cartesian, then $f$ is $p'$-Cartesian.

\item[$(3)$] Let $p: X \rightarrow Y$ and $q: Y \rightarrow Z$ be
inner fibrations, and let $f: x' \rightarrow x$ be an edge of $X$ such that $p(f)$ is $q$-Cartesian. Then
$f$ is $p$-Cartesian if and only if $f$ is $(q \circ p)$-Cartesian.
\end{itemize}
\end{proposition}

\begin{proof}
Assertions $(1)$ and $(2)$ follow immediately from the definition. To prove $(3)$, we consider the commutative diagram
$$ \xymatrix{ X_{/f} \ar[rr]^{\psi} \ar[dr]^{\psi'} & & X_{/x} \times_{ Z_{/ (q \circ p)(x)} } Z_{/ (q \circ p)(f)} \\
& X_{/x} \times_{ Y_{/p(x)} } Y_{/p(f)}. \ar[ur]^{\psi''} & }$$
The map $\psi''$ is a pullback of 
$$Y_{/p(f)} \rightarrow Y_{/p(x)} \times_{ Z_{/(q \circ p)(x)} } Z_{/(q \circ p)(f)}$$
and therefore a trivial fibration, in view of our assumption that $p(f)$ is $q$-Cartesian. If $\psi'$ is a trivial fibration, it follows that $\psi$ is a trivial fibration as well, which proves the ``only if'' direction of $(3)$.

For the converse, suppose that $\psi$ is a trivial fibration. Proposition \ref{sharpen} implies
that $\psi'$ is a right fibration. According to Lemma \ref{toothie}, it will suffice to prove that the fibers of $\psi'$ are contractible. Let $t$ be a vertex of $X_{/x} \times_{ Y_{/p(x)} } Y_{/p(f)}$, and let
$K = (\psi'')^{-1} \{ \psi''(t) \}$. Since $\psi''$ is a trivial fibration, $K$ is a contractible Kan complex.
Since $\psi$ is a trivial fibration, the simplicial set
$ (\psi')^{-1} K = \psi^{-1} \{ \psi''(t) \}$ is also a contractible Kan complex. It follows that the
fiber of $\psi'$ over the point $t$ is weakly contractible, as desired.
\end{proof}

\begin{remark}\label{univsay}
Let $p: X \rightarrow S$ be an inner fibration of simplicial sets. Unwinding the definition, we see that an edge $f: \Delta^1 \rightarrow X$ is $p$-Cartesian if and only if for every $n \geq 2$ and every commutative diagram
$$ \xymatrix{ \Delta^{ \{n-1, n\} } \ar[dr]^{f} \ar@{^{(}->}[d] & \\
\Lambda^n_n \ar[r] \ar@{^{(}->}[d] & X \ar[d]^{p} \\
\Delta^n \ar[r] \ar@{-->}[ur] & S, }$$
there exists a dotted arrow as indicated, rendering the diagram commutative.
\end{remark}

In particular, we note that Proposition \ref{greenlem} may be restated as follows:

\begin{itemize}
\item[$(\ast)$] Let $\calC$ be a $\infty$-category, and let $p: \calC \rightarrow \Delta^0$ denote the projection from $\calC$ to a point. A morphism $\phi$ of $\calC$ is $p$-Cartesian if and only if $\phi$ is an equivalence.
\end{itemize}

In fact, it is possible to strengthen assertion $(\ast)$ as follows:

\begin{proposition}\label{universalequiv}
Let $p: \calC \rightarrow \calD$ be an inner fibration between $\infty$-categories, and let $f: C \rightarrow C'$ be a morphism in $\calC$. The following conditions are equivalent:
\begin{itemize}
\item[$(1)$] The morphism $f$ is an equivalence in $\calC$.
\item[$(2)$] The morphism $f$ is $p$-Cartesian and $p(f)$ is an equivalence in $\calD$.
\end{itemize}
\end{proposition}

\begin{proof}
Let $q$ denote the projection from $\calD$ to a point. We note that both $(1)$ and $(2)$
imply that $p(f)$ is an equivalence in $\calD$, and therefore $q$-Cartesian by $(\ast)$.
The equivalence of $(1)$ and $(2)$ now follows from $(\ast)$ and the third part of Proposition \ref{stuch}.
\end{proof}

\begin{corollary}\label{corpal}
Let $p: \calC \rightarrow \calD$ be an inner fibration between $\infty$-categories. Every identity morphism of $\calC$ $($in other words, every degenerate edge of $\calC${}$)$ is $p$-Cartesian.
\end{corollary}

We now study the behavior of Cartesian edges under composition.

\begin{proposition}\label{protohermes}\index{gen}{Cartesian edge!and composition}
Let $p: \calC \rightarrow \calD$ be an inner fibration between simplicial sets, and let
$\sigma: \Delta^2 \rightarrow \calC$ be a $2$-simplex of $\calC$, which we will depict as
a diagram
$$ \xymatrix{ & C_1 \ar[dr]^{g} & \\
C_0 \ar[ur]^{f} \ar[rr]^{h} & & C_2. }$$
Suppose that $g$ is $p$-Cartesian. Then $f$ is $p$-Cartesian if and only if $h$ is $p$-Cartesian.
\end{proposition}

\begin{proof}
We wish to show that the map
$$ i_0: \calC_{/h} \rightarrow \calC_{/C_2} \times_{ \calD_{/p(C_2)}} \calD_{/p(h)}$$ is a trivial fibration
if and only if $$ i_1: \calC_{ /f } \rightarrow \calC_{/C_1} \times_{ \calD_{/p(C_1)}} \calD_{/p(f)}$$ is a trivial fibration. The dual of Proposition \ref{sharpen} implies that both maps are right fibrations. Consequently, by (the dual of) Lemma \ref{toothie}, it suffices to show that the fibers of $i_0$ are contractible if and only if the fibers of $i_1$ are contractible. 

For any simplicial subset $B \subseteq \Delta^2$, let $X_B = \calC_{/\sigma|B} \times_{ \calD_{\sigma|B}} \calD_{/\sigma}$. We note that $X_B$ is functorial in $B$, in the sense that an inclusion 
$A \subseteq B$ induces a map $j_{A,B}: X_B \rightarrow X_A$ (which is a right fibration, again by Proposition \ref{sharpen}). We note that $j_{ \Delta^{ \{2\} }, \Delta^{ \{0,2\} }}$
is the base change of $i_0$ by the map $\calD_{/p(\sigma)} \rightarrow \calD_{/p(h)}$, and that $j_{ \Delta^{ \{1\} }, \Delta^{ \{0,1\} }}$ is the base change of $i_1$ by the map
$\calD_{/\sigma} \rightarrow \calD_{/p(f)}$.  The maps $$ \calD_{ /p(f)} \leftarrow \calD_{/p(\sigma)} \rightarrow \calD_{/p(h)}$$ are both surjective on objects (in fact, both maps have sections).
Consequently, it suffices to prove that
$j_{ \Delta^{ \{1\}}, \Delta^{ \{0,1\} }}$ has contractible fibers if and only if $j_{ \Delta^{ \{2\}}, \Delta^{ \{0,2\} }}$ has contractible fibers. Now we observe that the compositions

$$ X_{\Delta^2} \rightarrow X_{ \Delta^{ \{0,2\} }} \rightarrow X_{ \Delta^{ \{2\} }}$$
$$ X_{\Delta^2} \rightarrow X_{ \Lambda^{2}_{1} } \rightarrow X_{\Delta^{ \{1,2\} }}  \rightarrow X_{ \Delta^{ \{2\} }}$$ 
coincide. By Proposition \ref{sharpen2}, $j_{A,B}$ is a trivial fibration whenever the inclusion $A \subseteq B$ is left anodyne. we deduce that $j_{ \Delta^{ \{2\} }, \Delta^{ \{0,2\} } }$ is a trivial fibration if and only if
$j_{\Delta^{ \{1,2\} }, \Lambda^2_1}$ is a trivial fibration. Consequently, it suffices to show that $j_{ \Delta^{ \{1,2\} }, \Lambda^2_1}$ is a trivial fibration if and only if $j_{ \Delta^{ \{1\} }, \Delta^{ \{0,1\} }}$ is a trivial fibration.

Since $j_{ \Delta^{ \{1,2\} }, \Lambda^2_1}$ is a pullback of $j_{ \Delta^{ \{1\} }, \Delta^{ \{0,1\} }}$, the ``if'' direction is obvious. For the converse, it suffices to show that the natural map
$$\calC_{/g} \times_{ \calD_{ /p(g)} } \calD_{/ p(\sigma)} \rightarrow \calC_{/ C_1} \times_{ \calD_{/ p(C_1)} } \calD_{/p(\sigma)}$$ is surjective on vertices. But this map is a trivial fibration, since the inclusion $\{1\} \subseteq \Delta^{ \{1,2\} }$ is left anodyne.
\end{proof}

Our next goal is to reformulate the notion of a Cartesian morphism in a form which will be useful later.
For convenience of notation, we will prove this result in a dual form. If $p: X \rightarrow S$ is an inner fibration and $f$ an edge of $X$, we will say that $f$ is {\it $p$-coCartesian} if is Cartesian with respect to the morphism $p^{op}: X^{op} \rightarrow S^{op}$.\index{gen}{coCartesian edge}\index{gen}{morphism!$p$-coCartesian}

\begin{proposition}\label{goouse}\index{gen}{Cartesian edge}
Let $p: Y \rightarrow S$ be an inner fibration of simplicial sets, and $e: \Delta^1 \rightarrow
Y$ an edge. Then $e$ is $p$-coCartesian if and only if for each $n \geq 1$ and each diagram
$$ \xymatrix{ \{0\} \times \Delta^1 \ar[drr]^{e} \ar@{^{(}->}[d] & & \\
(\Delta^n \times \{0\}) \coprod_{ \bd \Delta^n \times \{0\} } ( \bd \Delta^n \times \Delta^1) \ar[rr]^-{f} \ar@{^{(}->}[d] & & Y \ar[d]^{p} \\
\Delta^n \times \Delta^1 \ar@{-->}[urr]^{h} \ar[rr]^{g} & & S }$$
there exists a map $h$ as indicated, rendering the diagram commutative.
\end{proposition}

\begin{proof}
Let us first prove the ``only if'' direction. We recall a bit of the
notation used in the proof of Proposition \ref{usejoyal}; in particular, the filtration
$$X(n+1) \subseteq \ldots \subseteq X(0) = \Delta^n \times
\Delta^1$$ of $\Delta^n \times \Delta^1$. We construct $h|X(m)$
by descending induction on $m$. To begin, we set $h|X(n+1) = f$.
Now, for each $m$ the space $X(m)$ is obtained from $X(m+1)$ by
pushout along a horn inclusion $\Lambda^{n+1}_m \subseteq
\Delta^{m+1}$. If $m > 0$, the desired extension exists because
$p$ is an inner fibration. If $m = 0$, the desired extension exists
because of the hypothesis that $e$ is a $p$-coCartesian edge.

We now prove the ``if'' direction. Suppose that $e$ satisfies the
condition in the statement of the Proposition. We wish to show
that $e$ is $p$-coCartesian. In other words, we must show that for every $n \geq 2$ and
every diagram
$$ \xymatrix{ \Delta^{ \{0,1\} } \ar[dr]^{e} \ar@{^{(}->}[d] & \\
\Lambda^n_0 \ar[r] \ar@{^{(}->}[d] & X \ar[d]^{p} \\
\Delta^n \ar[r] \ar@{-->}[ur] & S }$$
there exists a dotted arrow as indicated, rendering the diagram commutative. 
Replacing $S$ by $\Delta^n$ and $Y$ by
$Y \times_{S} \Delta^n$, we may reduce to the case where $S$ is a $\infty$-category. We
again make use of the notation (and argument) employed in the proof
of Proposition \ref{usejoyal}. Namely, the inclusion $\Lambda^n_0
\subseteq \Delta^n$ is a retract of the inclusion
$$ (\Lambda^n_0 \times \Delta^1) \coprod_{ \Lambda^n_0 \times
\{0\} } (\Delta^n \times \{0\}) \subseteq \Delta^n \times
\Delta^1.$$ The retraction is implemented by maps
$$ \Delta^n \stackrel{j}{\rightarrow} \Delta^n \times \Delta^1
\stackrel{r}{\rightarrow} \Delta^n$$ which were defined in the
proof of Proposition \ref{usejoyal}. We now set $F = f \circ r$,
$G = g \circ r$.

Let $K = \Delta^{ \{1, 2, \ldots, n \} } \subseteq \Delta^n$.
Then $$F| (\bd K \times \Delta^1) \coprod_{ \bd K \times \{0\}} (K \times
\Delta^1)$$ carries $\{1\} \times \Delta^1$ into $e$. By assumption, there exists an
extension of $F$ to $K \times \Delta^1$ which is compatible with
$G$. In other words, there exists a compatible extension $F'$ of
$F$ to $$ \bd \Delta^n \times \Delta^1 \coprod_{ \bd \Delta^n
\times \{0\} } \Delta^n \times \{0 \}.$$ Moreover, $F'$ carries
$\{0\} \times \Delta^1$ to a degenerate edge; such an edge is
automatically coCartesian (by Corollary \ref{corpal}, since $S$ is an $\infty$-category), and therefore there exists an extension of $F'$ to all of $\Delta^n \times \Delta^1$ by the first part of
the proof.
\end{proof}

\begin{remark}\label{kermy}
Let $p: X \rightarrow S$ be an inner fibration of simplicial sets, $x$ a vertex of $X$, and 
$\overline{f}: \overline{x}' \rightarrow p(x)$ an edge of $S$ ending at $p(x)$. There may exist many $p$-Cartesian edges $f: x' \rightarrow x$ of $X$ with $p(f) = \overline{f}$. However, there is a sense in which any two such edges having the same target $x$ are equivalent to one another.
Namely, any $p$-Cartesian edge $f: x' \rightarrow x$ lifting $\overline{f}$ can be regarded as a final object of the $\infty$-category $X_{/x} \times_{ S_{/p(x)} } \{ \overline{f} \}$, and is therefore determined up to equivalence by $\overline{f}$ and $x$.
\end{remark}

We now spell out the meaning of Definition \ref{univedge} in the setting of simplicial categories.

\begin{proposition}\label{trainedg}\index{gen}{Cartesian edge!and simplicial categories}
Let $F: \calC \rightarrow \calD$ be a functor between simplicial categories.
Suppose that $\calC$ and $\calD$ are fibrant, and that for every pair of objects
$C,C' \in \calC$, the associated map
$$ \bHom_{\calC}(C,C') \rightarrow \bHom_{\calD}(F(C),F(C'))$$ is a Kan fibration.
Then:
\begin{itemize}
\item[$(1)$] The associated map $q: \sNerve(\calC) \rightarrow \sNerve(\calD)$ is an inner fibration between $\infty$-categories.

\item[$(2)$] A morphism $f: C' \rightarrow C''$ in $\calC$ is $q$-Cartesian if and only if,
for every object $C \in \calC$, the diagram of simplicial sets
$$ \xymatrix{ \bHom_{\calC}(C, C') \ar[r] \ar[d] & \bHom_{\calC}(C,C'') \ar[d] \\
\bHom_{\calD}(F(C), F(C')) \ar[r] & \bHom_{\calD}(F(C), F(C''))}$$
is homotopy Cartesian.
\end{itemize}

\end{proposition}

\begin{proof}
Assertion $(1)$ follows from Remark \ref{goobrem}. Let $f$ be a morphism in $\calC$. By definition, $f: C' \rightarrow C''$ is $q$-Cartesian if and only if 
$$\theta: \sNerve(\calC)_{/f} \rightarrow \sNerve(\calC)_{/C''} \times_{ \sNerve(\calD)_{/F(C'')}} \sNerve(\calD)_{/F(f)}$$ is a trivial fibration. Since $\theta$ is a right fibration between
right fibrations over $\calC$, $f$ is $q$-Cartesian if and only if for every object $C \in \calC$, 
the induced map
$$\theta_{C}: \{C\} \times_{\sNerve(\calC)} \sNerve(\calC)_{/f} \rightarrow \{C\} \times_{\sNerve(\calC)} \sNerve(\calC)_{/C''} \times_{ \sNerve(\calD)_{/F(C'')}} \sNerve(\calD)_{/F(f)} $$
is a homotopy equivalence of Kan complexes. This is equivalent to the assertion that the diagram
$$ \xymatrix{ \sNerve(\calC)_{/f} \times_{\calC} \{C\} \ar[r] \ar[d] & \sNerve(\calC)_{/C''} \times_{\sNerve(\calC)} \{C\} \ar[d] \\
\sNerve(\calD)_{/F(f)} \times_{\sNerve(\calD)} \{F(C)\} \ar[r] & \sNerve(\calD)_{/F(C'')} \times_{ \sNerve(\calD)} \{ F(C) \} }$$
is homotopy Cartesian. In view of Theorem \ref{biggie}, this diagram is equivalent to the diagram of simplicial sets
$$ \xymatrix{ \bHom_{\calC}(C, C') \ar[r] \ar[d] & \bHom_{\calC}(C,C'') \ar[d] \\
\bHom_{\calD}(F(C), F(C')) \ar[r] & \bHom_{\calD}(F(C), F(C'')).}$$
This proves $(2)$.
\end{proof}

In some contexts, it will be convenient to introduce a slightly larger class of edges:

\begin{definition}\index{gen}{Cartesian!locally}\index{gen}{locally Cartesian!edge}
Let $p: X \rightarrow S$ be an inner fibration, and let $e: \Delta^1 \rightarrow X$ be an edge.
We will say that $e$ is {\it locally $p$-Cartesian} if it is a $p'$-Cartesian edge of the fiber product
$X \times_{S} \Delta^1$, where $p': X \times_{S} \Delta^1 \rightarrow \Delta^1$ denotes the projection.
\end{definition}

\begin{remark}\label{intin}
Suppose given a pullback diagram
$$ \xymatrix{ X' \ar[r]^{f} \ar[d]^{p'} & X \ar[d]^{p} \\
S' \ar[r] & S }$$
of simplicial sets, where $p$ (and therefore also $p'$) is an inner fibration. An edge
$e$ of $X'$ is locally $p'$-Cartesian if and only if its image $f(e)$ is locally $p$-Cartesian.
\end{remark}

We conclude with a somewhat technical result which will be needed in \S \ref{bicat1}:

\begin{proposition}\label{sworkk}
Let $p: X \rightarrow S$ be an inner fibration of simplicial sets. Let
$f: x \rightarrow y$ be an edge of $X$ Suppose that there is a $3$-simplex
$\sigma: \Delta^3 \rightarrow X$ such that $d_1 \sigma = s_0 f$ and $d_2 \sigma = s_1 f$.
Suppose furthermore that there exists a $p$-Cartesian edge $\widetilde{f}:
\widetilde{x} \rightarrow y$ such that $p(\widetilde{f}) = p(f)$.
Then $f$ is $p$-Cartesian.
\end{proposition}

\begin{proof}
We have a diagram of simplicial sets
$$ \xymatrix{ \Lambda^2_2 \ar[rr]^{(\widetilde{f}, f, \bigdot)} \ar@{^{(}->}[d] & & X \ar[d]^{p} \\
\Delta^2 \ar[rr]^{s_0 p(f)} \ar@{-->}[urr]^{\tau} & & S.}$$
Because $\widetilde{f}$ is $p$-Cartesian, there exists a map $\tau$ rendering the diagram commutative. Let $g = d_2(\tau)$, which we regard as a morphism $x \rightarrow \widetilde{x}$ in the $\infty$-category $X_{p(x)} = X \times_{S} \{p(x)\}$. We will show that $g$ is an equivalence in $X_{p(x)}$. It will follow that $g$ is $p$-Cartesian and that $f$, being a composition of $p$-Cartesian edges, is $p$-Cartesian (Proposition \ref{protohermes}).

Now consider the diagram
$$ \xymatrix{ \Lambda^2_1 \ar[rr]^{(d_0 d_3 \sigma, \bigdot, g)} \ar@{^{(}->}[d] & & X \ar[d]^{p} \\
\Delta^2 \ar[rr]^{ d_3 p(\sigma) } \ar@{-->}[urr]^{\tau'} & & S.}$$
The map $\tau'$ exists since $p$ is an inner fibration. Let $g' = d_1 \tau'$. We will show that
$g': \widetilde{x} \rightarrow x$ is a homotopy inverse to $g$ in the $\infty$-category
$X_{p(x)}$. 

Using $\tau$ and $\tau'$, we construct a new diagram
$$ \xymatrix{ \Lambda^3_2 \ar[rr]^{ (\tau', d_3 \sigma, \bigdot, \tau) }\ar@{^{(}->}[d] & &  X \ar[d]^{p} \\
\Delta^3 \ar[rr]^{ s_0 d_3 p(\sigma) } \ar@{-->}[urr]^{\theta} & & S.}$$
Since $p$ is an inner fibration, we deduce the existence of $\theta: \Delta^3 \rightarrow X$ rendering the
diagram commutative. The simplex $d_2(\theta)$ exhibits $\id_{x}$ as a composition
$g' \circ g$ in the $\infty$-category $X_{p(s)}$. It follows that $g'$ is a left homotopy inverse to $g$.

We now have a diagram
$$ \xymatrix{ \Lambda^2_1 \ar[rr]^{(g, \bigdot, g')} \ar@{^{(}->}[d] & & X_{p(x)} \\
\Delta^2. \ar@{-->}[urr]^{\tau''} } $$
The indicated $2$-simplex $\tau''$ exists since $X_{p(x)}$ is an $\infty$-category, and
exhibits $d_1(\tau'')$ as a composition $g \circ g'$. To complete the proof, it will suffice to show that
$d_1(\tau'')$ is an equivalence in $X_{p(x)}$. 

Consider the diagrams
$$ \xymatrix{ \Lambda^3_1 \ar[rr]^{ (d_0 \sigma, \bigdot, s_1 \widetilde{f}, \tau')} \ar@{^{(}->}[d] & & X \ar[d]^{p} &  \Lambda^3_1 \ar[rr]^{ (\tau, \bigdot, d_1 \theta', \tau'') } \ar@{^{(}->}[d] & & X \ar[d]^{p} \\
\Delta^3 \ar@{-->}[urr]^{\theta'} \ar[rr]^{\sigma} & & S & \Delta^3 \ar[rr]^{ s_0 s_0 p(f)} \ar@{-->}[urr]^{\theta''} & & S. }$$
Since $p$ is an inner fibration, there exist $3$-simplices $\theta', \theta'': \Delta^3 \rightarrow X$ with the inducated properties. The $2$-simplex $d_1(\theta'')$ identifies $d_1(\tau'')$ as a map between two
$p$-Cartesian lifts of $p(f)$; it follows that $d_1(\tau'')$ is an equivalence, which completes the proof.
\end{proof}

\subsection{Cartesian Fibrations}\label{funkymid}

In this section, we will introduce the study of {\it Cartesian fibrations} between simplicial sets. The theory of Cartesian fibrations is a generalization of the theory of right fibrations studied in \S \ref{leftfibsec}. Recall that if $f: X \rightarrow S$
is a right fibration of simplicial sets, then the fibers $\{ X_{s} \}_{s \in S}$
are Kan complexes, which depend in a (contravariantly) functorial fashion on
the choice of vertex $s \in S$. The condition that $f$ be a Cartesian fibration has a similar flavor: we still require that $X_{s}$ depend functorially on $s$, but weaken the requirement that $X_{s}$ be a Kan complex; instead, we merely require that it is an $\infty$-category.

\begin{definition}\label{defcart}\index{gen}{Cartesian fibration}\index{gen}{fibration!Cartesian}
We will say that a map $p: X \rightarrow S$ of simplicial sets is a {\it Cartesian fibration} if the following conditions are satisfied:
\begin{itemize}
\item[$(1)$] The map $p$ is an inner fibration.
\item[$(2)$] For every edge $f: x \rightarrow y$ of $S$ and every vertex $\widetilde{y}$
of $X$ with $p(\widetilde{y}) = y$, there exists a $p$-Cartesian edge $\widetilde{f}: \widetilde{x} \rightarrow \widetilde{y}$ with $p(\widetilde{f}) = f$.
\end{itemize}

We say that $p$ is a {\it coCartesian fibration} if the opposite map $p^{op}: X^{op} \rightarrow S^{op}$ is a Cartesian fibration.\index{gen}{fibration!coCartesian}\index{gen}{coCartesian fibration}
\end{definition}

If a general inner fibration $p: X \rightarrow S$ associates to each vertex $s \in S$ an $\infty$-category $X_{s}$ and to each edge $s \rightarrow s'$ a correspondence from $X_{s}$ to $X_{s'}$, then $p$ is Cartesian if each of these correspondences arise from an (canonically determined) functor $X_{s'} \rightarrow X_{s}$. In other words, a Cartesian fibration with base $S$ ought to be roughly the same thing as a contravariant functor from $S$ into an $\infty$-category of $\infty$-categories, where the morphisms are given by {\em functors}. 
One of the main goals of \S \ref{chap4} is to give a precise formulation
(and proof) of this assertion.

\begin{remark}\label{gcart}
Let $F: \calC \rightarrow \calC'$ be a functor between (ordinary) categories. The induced map of simplicial sets $\Nerve(F): \Nerve(\calC) \rightarrow \Nerve(\calC')$ of simplicial sets is automatically an inner fibration; it is Cartesian if and only if $F$ is a {\it fibration} of categories in the sense of Grothendieck.
\end{remark}

The following formal properties follow immediately from the definition:

\begin{proposition}
\begin{itemize}
\item[$(1)$] Any isomorphism of simplicial sets is a Cartesian fibration.

\item[$(2)$] The class of Cartesian fibrations between simplicial sets is stable under base change.

\item[$(3)$] A composition of Cartesian fibrations is a Cartesian fibration.
\end{itemize}
\end{proposition}


Recall that an $\infty$-category $\calC$ is a Kan complex if and only if every morphism in $\calC$ is an equivalence. We now establish a relative version of this statement:

\begin{proposition}\label{goey}\index{gen}{Cartesian fibration!and right fibrations}
Let $p: X \rightarrow S$ be an inner fibration of simplicial sets. The following
conditions are equivalent:
\begin{itemize}
\item[$(1)$] The map $p$ is a Cartesian fibration and every edge in $X$ is
$p$-Cartesian.

\item[$(2)$] The map $p$ is a right fibration.

\item[$(3)$] The map $p$ is a Cartesian fibration and every fiber of $p$ is a Kan
complex.
\end{itemize}
\end{proposition}

\begin{proof}
In view of Remark \ref{univsay}, the assertion that every edge of $X$ is $p$-Cartesian is equivalent to the assertion that $p$ has the right lifting property with respect to $\Lambda^n_n \subseteq \Delta^n$ for all $n \geq 2$. The requirement that $p$ be a Cartesian fibration further imposes the right lifting property with respect to $\Lambda^1_1 \subseteq \Delta^1$. This proves that $(1) \Leftrightarrow (2)$.

Suppose that $(2)$ holds. Since we have established that $(2)$ implies $(1)$, we know that $p$ is Cartesian. Furthermore, we have already seen that the fibers of a right fibration are Kan complexes. Thus $(2)$ implies $(3)$.

We complete the proof by showing that $(3)$ implies that every edge $f: x \rightarrow y$
of $X$ is $p$-Cartesian. Since $p$ is a Cartesian fibration, there exists a $p$-Cartesian edge $f': x' \rightarrow y$
with $p(f') = p(f)$. Since $f'$ is $p$-Cartesian, 
there exists a $2$-simplex $\sigma: \Delta^2 \rightarrow X$ which we may depict as a diagram
$$ \xymatrix{ & x' \ar[dr]^{f'} & \\
x \ar[ur]^{g} \ar[rr]^{f} & & y, }$$
where $p(\sigma) = s_0 p(f)$. Then $g$ lies in the fiber $X_{p(x)}$, and is therefore an equivalence (since $X_{p(x)}$ is a Kan complex). It follows that $f$ is equivalent to $f'$ as objects of
$X_{/y} \times_{ S_{/p(y)} } \{p(f) \}$, so that $f$ is $p$-Cartesian as desired.
\end{proof}

\begin{corollary}\label{relativeKan}
Let $p: X \rightarrow S$ be a Cartesian fibration. Let $X'
\subseteq X$ consist of all those simplices $\sigma$ of $X$ such that
every edge of $\sigma$ is $p$-Cartesian. Then $p|X'$ is a right fibration.
\end{corollary}

\begin{proof}
We first show that $p|X'$ is an inner fibration. It suffices to show
that $p|X'$ has the right lifting property with respect to every
horn inclusion $\Lambda^n_i$, $0 < i < n$. If $n > 2$, then
this follows immediately from the fact the fact that $p$ has the
appropriate lifting property. If $n = 2$, then we must show that
if $f: \Delta^2 \rightarrow X$ is such that $f|\Lambda^2_1$
factors through $X'$, then $f$ factors through $X'$. This follows immediately from Proposition \ref{protohermes}.

We now wish to complete the proof by showing that $p$ is a right fibration. According to
Proposition \ref{goey}, it suffices to prove that every edge of $X'$ is $p|X'$-Cartesian. This follows immediately from the characterization given in Remark \ref{univsay}, since every edge of $X'$ is $p$-Cartesian when regarded as an edge of $X$.
\end{proof}

In order to verify that certain maps are Cartesian fibrations, it often convenient to work in a slightly more general setting. 

\begin{definition}\index{gen}{locally Cartesian!fibration}\index{gen}{fibration!locally Cartesian} A map $p: X \rightarrow S$ of simplicial sets is a {\it locally Cartesian fibration} if it is an inner fibration and, for every edge $\Delta^1 \rightarrow S$, the pullback $X \times_{S} \Delta^1 \rightarrow \Delta^1$ is a Cartesian fibration.
\end{definition}

In other words, an inner fibration $p: X \rightarrow S$ is a locally Cartesian fibration if and only if, for every vertex $x \in X$ and every edge $e: s \rightarrow p(x)$ in $S$, there exists a locally $p$-Cartesian edge $\overline{s} \rightarrow x$ which lifts $e$.

Let $p: X \rightarrow S$ be an inner fibration of simplicial sets. It is clear that every $p$-Cartesian morphism of $X$ is locally $p$-Cartesian. Moreover, Proposition \ref{protohermes} implies that the class of $p$-Cartesian edges of $X$ is stable under composition. Then following result can be regarded as a sort of converse:

\begin{lemma}\label{charloccart}
Let $p: X \rightarrow S$ be a locally Cartesian fibration of simplicial sets, and let
$f: x' \rightarrow x$ be an edge of $X$. The following conditions are equivalent:
\begin{itemize}
\item[$(1)$] The edge $e$ is $p$-Cartesian.
\item[$(2)$] For every $2$-simplex $\sigma$
$$ \xymatrix{ & x' \ar[dr]^{f} & \\
x'' \ar[ur]^{g} \ar[rr]^{h} & & x }$$
in $X$, the edge $g$ is locally $p$-Cartesian if and only if the edge $h$ is locally $p$-Cartesian.
\item[$(3)$] For every $2$-simplex $\sigma$
$$ \xymatrix{ & x' \ar[dr]^{f} & \\
x'' \ar[ur]^{g} \ar[rr]^{h} & & x }$$
in $X$, if $g$ is locally $p$-Cartesian, then $h$ is locally $p$-Cartesian.
\end{itemize}
\end{lemma}

\begin{proof}
We first show that $(1) \Rightarrow (2)$. Pulling back via the composition $p \circ \sigma: \Delta^2 \rightarrow S$, we can reduce to the case where $S = \Delta^2$. In this case, $g$ is locally $p$-Cartesian if and only if it is $p$-Cartesian, and likewise for $h$. We now conclude by applying Proposition \ref{protohermes}.

The implication $(2) \Rightarrow (3)$ is obvious. We conclude by showing that $(3) \Rightarrow (1)$. We must show that $\eta: X_{/f} \rightarrow
X_{/x} \times_{ S_{/p(x)} } S_{/p(f)}$ is a trivial fibration.
Since $\eta$ is a right fibration, it will suffice to
show that the fiber of $\eta$ over any vertex is contractible. Any such vertex determines a map
$\sigma: \Delta^2 \rightarrow S$ with $\sigma| \Delta^{ \{1,2\} } = p(f)$. Pulling back via
$\sigma$, we may suppose that $S= \Delta^2$. 

It will be convenient to introduce a bit of notation: for every map $q: K \rightarrow X$, let
$Y_{/q} \subseteq X_{/q}$ denote the full simplicial subset spanned by those vertices of
$X_{/q}$ which map to the initial vertex of $S$. 
We wish to show that the natural map
$Y_{/f} \rightarrow Y_{/x}$ is a trivial
fibration. By assumption, there exists a locally $p$-Cartesian morphism
$g: x'' \rightarrow x'$ in $X$ covering the edge $\Delta^{ \{0,1\} } \subseteq S$.
Since $X$ is an $\infty$-category, there exists a $2$-simplex $\tau: \Delta^2 \rightarrow X$ with
$d_2(\tau)=g$ and $d_0(\tau)=f$. Then $h = d_1(\tau)$ is a composite of $f$
and $g$, and assumption $(3)$ guarantees that $h$ is locally $p$-Cartesian. We have a commutative diagram
$$ \xymatrix{ & & Y_{/h} \ar[drr]  & & \\
Y_{/\tau} \ar[urr] \ar[dr] & & & & Y_{/x} \\
& Y_{/\tau| \Lambda^2_1} \ar[rr] & & Y_{/f}. \ar[ur]^{\zeta} & }$$
Moreover, all of these maps in this diagram are trivial fibrations except possibly
$\zeta$, which is known to be a right fibration. It follows that $\zeta$ is a trivial fibration as well, which completes the proof.
\end{proof}

In fact, we have the following:

\begin{proposition}\label{gotta}
Let $p: X \rightarrow S$ be a locally Cartesian fibration. The following conditions are equivalent:
\begin{itemize}
\item[$(1)$] The map $p$ is a Cartesian fibration.
\item[$(2)$] Given a $2$-simplex
$$ \xymatrix{ x \ar[rr]^{f} \ar[dr]^{h} & & x' \ar[dl]^{g} \\
& z, & }$$
if $f$ and $g$ are locally $p$-Cartesian, then $h$ is locally $p$-Cartesian.
\item[$(3)$] Every locally $p$-Cartesian edge of $X$ is $p$-Cartesian.
\end{itemize}
\end{proposition} 
 
\begin{proof}
The equivalence $(2) \Leftrightarrow (3)$ follows from Lemma \ref{charloccart}, and the implication $(3) \Rightarrow (1)$ is obvious. To prove that $(1) \Rightarrow (3)$, let us suppose that
$e: x \rightarrow y$ is a locally $p$-Cartesian edge of $X$. Choose a $p$-Cartesian edge
$e': x' \rightarrow y$ lifting $p(e)$. The edges $e$ and $e'$ are both $p'$-Cartesian in
$X' = X \times_{S} \Delta^1$, where $p': X' \rightarrow \Delta^1$ denotes the projection. It follows that $e$ and $e'$ are equivalent in $X'$, and therefore also equivalent in $X$. Since $e'$ is $p$-Cartesian, we deduce that $e$ is $p$-Cartesian as well.
\end{proof}

\begin{remark}
If $p: X \rightarrow S$ is a locally Cartesian fibration, then we can associate to every edge
$s \rightarrow s'$ of $S$ a functor $X_{s'} \rightarrow X_{s}$, which is well-defined up to homotopy.
A $2$-simplex
$$ \xymatrix{ s \ar[rr] \ar[dr] & & s' \ar[dl] \\
& s'' & }$$
determines a triangle of $\infty$-categories
$$ \xymatrix{ X_{s} & & X_{s'} \ar[ll]^{F} \\
& X_{s''} \ar[ul]^{H} \ar[ur]^{G} & }$$
which commutes up to a (generally noninvertible) natural transformation $\alpha: F \circ G \rightarrow H$. Proposition \ref{gotta} implies that $p$ is a Cartesian fibration if and only if every such natural transformation is an equivalence of functors.
\end{remark}
 
\begin{corollary}
Let $p: X \rightarrow S$ be an inner fibration of simplicial sets. Then $p$ is Cartesian if and only if every pullback $X \times_{S} \Delta^n \rightarrow \Delta^n$ is
a Cartesian fibration, for $n \leq 2$.
\end{corollary}
 
One advantage the theory of locally Cartesian fibrations holds over the theory of Cartesian fibrations is the following ``fiberwise'' existence criterion:

\begin{proposition}\label{fibertest}
Suppose given a commutative diagram of simplicial sets.
$$ \xymatrix{ X \ar[dr]^{p} \ar[rr]^{r} & & Y \ar[dl]^{q} \\
& S & }$$
Suppose that:
\begin{itemize}
\item[$(1)$] The maps $p$ and $q$ are locally Cartesian fibrations, and $r$ is an inner fibration.
\item[$(2)$] The map $r$ carries locally $p$-Cartesian edges of $X$ to locally $q$-Cartesian edges of $Y$.
\item[$(3)$] For every vertex $s$ of $S$, the induced map $r_{s}: X_{s} \rightarrow Y_{s}$ is a
locally Cartesian fibration.
\end{itemize}
Then $r$ is a locally Cartesian fibration. Moreover, an edge $e$ of $X$ is locally $r$-Cartesian if and only if there exists a $2$-simplex $\sigma$
$$ \xymatrix{ & x' \ar[dr]^{e''} & \\
x \ar[ur]^{e'} \ar[rr]^{e} & & x'' }$$
with the following properties:
\begin{itemize}
\item[$(i)$] In the simplicial set $S$, we have $p( \sigma) = s^0( p(e))$.
\item[$(ii)$] The edge $e''$ is locally $p$-Cartesian.
\item[$(iii)$] The edge $e'$ is locally $r_{p(x)}$-Cartesian.
\end{itemize}
\end{proposition}

\begin{proof}
Suppose given a vertex $x'' \in X$ and an edge $e_0: y \rightarrow p(x'')$ in $Y$. It is clear that we can construct a $2$-simplex $\sigma$ in $X$ satisfying $(i)$ through $(iii)$, with
$p(e) = q(e_0)$. Moreover, $\sigma$ is uniquely determined up to equivalence. We will prove that $e$ is locally $r$-Cartesian. This will prove that $r$ is a locally Cartesian fibration, and the
``if'' direction of the final assertion. The converse will then follow from the uniqueness (up to equivalence) of locally $r$-Cartesian lifts of a given edge (with specified terminal vertex).
 
To prove that $e$ is locally $r$-Cartesian, we are free to pull back by the edge
$p(e): \Delta^1 \rightarrow S$, and thereby reduce to the case $S = \Delta^1$. Then
$p$ and $q$ are Cartesian fibrations. Since $e''$ is $p$-Cartesian and $r(e'')$ is $q$-Cartesian, Proposition \ref{stuch} implies that $e''$ is $r$-Cartesian. Remark \ref{intin} implies that
$e'$ is locally $p$-Cartesian. It follows from Lemma \ref{charloccart} that $e$ is locally $p$-Cartesian as well.
\end{proof}

\begin{remark}
The analogue of Proposition \ref{fibertest} for Cartesian fibrations is false.
\end{remark}

\subsection{Stability Properties of Cartesian Fibrations}\label{slib}

In this section, we will prove the class of Cartesian fibrations is stable under the formation of overcategories and undercategories. Since the definition of a Cartesian fibration is not self-dual, we must treat these results separately, using slightly different arguments (Propositions \ref{werylonger} and \ref{verylonger}). We begin with the following simple lemma.

\begin{lemma}\label{doweneed}
Let $A \subseteq B$ be an inclusion of simplicial sets. Then the inclusion
$$ (\{1\} \star B) \coprod_{ \{1\} \star A } (\Delta^1 \star A ) \subseteq \Delta^1 \star B$$ is inner anodyne.
\end{lemma}

\begin{proof}
Working by transfinite induction, we may reduce to the case where $B$ is obtained from $A$ by adjoining a single non-degenerate simplex, and therefore to the universal case
$B = \Delta^n$, $A = \bd \Delta^n$. Now the inclusion in question is isomorphic to $\Lambda^{n+2}_{1} \subseteq \Delta^{n+2}$.
\end{proof}

\begin{proposition}\label{werylonger}\index{gen}{Cartesian fibration!and overcategories}
Let $p: \calC \rightarrow \calD$ be a Cartesian fibration of simplicial sets, and let
$q: K \rightarrow \calC$ be a diagram. Then:
\begin{itemize}
\item[$(1)$] The induced map $p': \calC_{/q} \rightarrow \calD_{/pq}$ is a Cartesian fibration.
\item[$(2)$] An edge $f$ of $\calC_{/q}$ is $p'$-Cartesian if and only if the image
of $f$ in $\calC$ is $p$-Cartesian.
\end{itemize}
\end{proposition}

\begin{proof}
Proposition \ref{sharpen2} implies that $p'$ is an inner fibration. Let us call an edge
$f$ of $\calC_{q/}$ {\it special} if its image in $\calC$ is $p$Cartesian. To complete the proof, it will suffice to show that:
\begin{itemize}
\item[$(i)$] Given a vertex $\overline{q} \in \calC_{/q}$ and an edge
$\widetilde{f}: \overline{r}' \rightarrow p'( \overline{q} )$, there exists a special edge
$f: \overline{r} \rightarrow \overline{q}$ with $p'(f) = \widetilde{f}$. 
\item[$(ii)$] Every special edge of $\calC_{/q}$ is $p'$-Cartesian.
\end{itemize}

To prove $(i)$, let $\widetilde{f}'$ denote the image of $\widetilde{f}$ in $\calD$ and
$c$ the image of $\overline{q}$ in $\calC$. Using the assumption that $p$ is a coCartesian fibration, we can choose a $p$-coCartesian edge $f': c \rightarrow d$ lifting $\widetilde{f}'$. To extend this data to the desired edge $f$ of $\calC_{/q}$, it suffices to solve the lifting problem depicted in the diagram
$$ \xymatrix{ (\{1\} \star K) \coprod_{ \{1\} } \Delta^1 \ar[r] \ar@{^{(}->}[d]^{i} & \calC \ar[d]^{p} \\
\Delta^1 \star K  \ar[r] \ar@{-->}[ur] & \calD }$$
This lifting problem has a solution, since $p$ is an inner fibration and $i$ is inner anodyne (Lemma \ref{doweneed}).

To prove $(ii)$, it will suffice to show that if $n \geq 2$, then any lifting problem of the form
$$ \xymatrix{ \Lambda^n_n \star K \ar[r]^{g} \ar@{^{(}->}[d] & \calC \ar[d]^{p} \\
\Delta^n \star K \ar[r] \ar@{-->}[ur]^{G} & \calD }$$
has a solution, provided that $e=g( \Delta^{ \{n-1,n\} })$ is a $p$-Cartesian edge of $\calC$.
Consider the set $P$ of pairs $(K', G_{K'})$, where $K' \subseteq K$ and $G_{K'}$ fits in a commutative diagram 
$$ \xymatrix{ (\Lambda^n_n \star K) \coprod_{ \Lambda^n_n \star K'} (\Delta^n \star K') \ar[rrr]^-{G_{K'}} \ar@{^{(}->}[d] & & & \calC \ar[d]^{p} \\
\Delta^n \star K \ar[rrr] & & & \calD. }$$
Because $e$ is $p$-Cartesian, there exists an element
$(\emptyset, G_{\emptyset} ) \in P$. We regard $P$ as partially ordered, where
$(K', G_{K'} ) \leq (K'', G_{K''})$ if $K' \subseteq K''$ and $G_{K'}$ is a restriction of $G_{K''}$.
Invoking Zorn's lemma, we deduce the existence of a maximal element $(K', G_{K'})$ of $P$.
If $K' = K$, then the proof is complete. Otherwise, it is possible to enlarge $K'$ by adjoining a single nondegenerate $m$-simplex of $K$. Since $(K', G_{K''})$ is maximal, we conclude that the associated lifting problem
$$ \xymatrix{ (\Lambda^n_n \star \Delta^m) \coprod_{\Lambda^n_n \star \bd \Delta^m} (\Delta^n \star \bd \Delta^m) \ar[r] \ar@{^{(}->}[d] & \calC \ar[d]^{p} \\
\Delta^n \star \Delta^m \ar[r] \ar@{-->}[ur]^{\sigma} & \calD. }$$
has no solution.
The left vertical map is equivalent to the inclusion $\Lambda^{n+m+1}_{n+1} \subseteq \Delta^{n+m+1}$, which is inner anodyne. Since $p$ is an inner fibration by assumption, we obtain a contradiction.
\end{proof}

\begin{proposition}\label{verylonger}\index{gen}{coCartesian fibration!and overcategories}
Let $p: \calC \rightarrow \calD$ be a coCartesian fibration of simplicial sets, and let
$q: K \rightarrow \calC$ be a diagram. Then:
\begin{itemize}
\item[$(1)$] The induced map $p': \calC_{/q} \rightarrow \calD_{/pq}$ is a coCartesian fibration.
\item[$(2)$] An edge $f$ of $\calC_{/q}$ is $p'$-coCartesian if and only if the image
of $f$ in $\calC$ is $p$-coCartesian.
\end{itemize}
\end{proposition}

\begin{proof}
Proposition \ref{sharpen2} implies that $p'$ is an inner fibration. Let us call an edge
$f$ of $\calC_{/q}$ {\it special} if its image in $\calC$ is $p$-coCartesian. To complete the proof, it will suffice to show that:
\begin{itemize}
\item[$(i)$] Given a vertex $\overline{q} \in \calC_{/q}$ and an edge
$\widetilde{f}: p'( \overline{q} ) \rightarrow \overline{r}'$, there exists a special edge
$f: \overline{q} \rightarrow \overline{r}$ with $p'(f) = \widetilde{f}$. 
\item[$(ii)$] Every special edge of $\calC_{/q}$ is $p'$-coCartesian.
\end{itemize}

To prove $(i)$, we begin a commutative diagram
$$ \xymatrix{ \Delta^0 \star K \ar[r]^{\overline{q}} \ar@{^{(}->}[d] & \calC \ar[d] \\
\Delta^1 \star K \ar[r]^{\widetilde{f}} & \calD }. $$
Let $C \in \calC$ denote the image under $\overline{q}$ of the cone point of
$\Delta^0 \star K$, and choose a $p$-coCartesian morphism
$u: C \rightarrow C'$ lifting $\widetilde{f}| \Delta^1$. We now consider the
collection $P$ of all pairs $(L, f_L)$, where $L$ is a simplicial subset of $K$ and $f_L$ is a map fitting into a commutative diagram
$$ \xymatrix{ (\Delta^0 \star K) \coprod_{ \Delta^0 \star L} (\Delta^1 \star L) \ar[rrr]^-{f_L} \ar@{^{(}->}[d] & & & \calC \ar[d] \\
\Delta^1 \star K \ar[rrr]^{\widetilde{f}} & &  & \calD } $$
where $f_L | \Delta^1 = u$ and $f_L | \Delta^0 \star K = \overline{q}$. We partially order
the set $P$ as follows: $(L, f_L) \leq (L', f_{L'})$ if $L \subseteq L'$ and $f_{L}$ is equal
to the restriction of $f_{L'}$. The partially ordered set $P$ satisfies the hypotheses of Zorn's lemma, and therefore contains a maximal element $(L,f_L)$. If $L \neq K$, then we can choose a simplex
$\sigma: \Delta^n \rightarrow K$ of minimal dimension which does not belong to $L$. By maximality, we obtain a diagram
$$ \xymatrix{ \Lambda^{n+2}_{0} \ar[r] \ar@{^{(}->}[d] & \calC \ar[d] \\
\Delta^{n+2} \ar[r] \ar@{-->}[ur] & \calD }$$
in which the indicated dotted arrow cannot be supplied. This is a contradiction, since
the upper horizontal map carries the initial edge of $\Lambda^{n+2}_0$ to a $p$-coCartesian
edge of $\calC$. It follows that $L = K$, and we may take $f = f_L$. This completes the proof of $(i)$.

The proof of $(ii)$ is similar. Suppose given $n \geq 2$ and a diagram
$$ \xymatrix{ \Lambda^n_0 \star K \ar[r]^{f_0} \ar@{^{(}->}[d] & \calC \ar[d] \\
\Delta^n \star K \ar[r]^{g} \ar@{-->}[ur]^{f} & \calD }$$
be a commutative diagram, where $f_0 | K = q$ and $f_0 | \Delta^{ \{0,1\} }$ is a $p$-coCartesian edge of $\calC$. We wish to prove the existence of the dotted arrow $f$, indicated in the diagram.
As above, we consider the collection $P$ of all pairs $(L, f_L)$, where $L$ is a simplicial subset of $K$ and $f_L$ extends $f_0$ and fits into a commutative diagram
$$ \xymatrix{ (\Lambda^n_0 \star K) \coprod_{ \Lambda^n_0 \star L} (\Delta^n \star L) \ar[rrr]^-{f_L} \ar@{^{(}->}[d] & & &  \calC \ar[d] \\
\Delta^n \star K \ar[rrr]^{g} & & & \calD. }$$
We partially order $P$ as follows: $(L, f_L) \leq (L', f_{L'})$ if $L \subseteq L'$ and $f_L$ is a restriction of $f_{L'}$. Using Zorn's lemma, we conclude that $P$ contains a maximal element
$(L, f_L)$. If $L \neq K$, then we can choose a simplex $\sigma: \Delta^m \rightarrow K$ which does not belong to $L$, where $m$ is as small as possible. Invoking the maximality of $(L,f_L)$, we obtain a diagram
$$ \xymatrix{ \Lambda^{n+m+1}_{0} \ar[r]^{h} \ar@{^{(}->}[d] & \calC \ar[d] \\
\Delta^{n+m+1} \ar[r] \ar@{-->}[ur] & \calD }$$
where the indicated dotted arrow cannot be supplied. However, the map $h$ carries the initial edge of $\Delta^{n+m+1}$ to a $p$-coCartesian edge of $\calC$, so we obtain a contradiction. It follows that $L = K$, so that we can take $f = f_L$ to complete the proof.
\end{proof}

\subsection{Mapping Spaces and Cartesian Fibrations}\label{slik}

Let $p: \calC \rightarrow \calD$ be a functor between $\infty$-categories, and let
$X$ and $Y$ be objects of $\calC$. Then $p$ induces a map
$$ \phi: \bHom_{\calC}(X,Y) \rightarrow \bHom_{\calD}(p(X),p(Y)).$$
Our goal in this section is to understand the relationship between the fibers of $p$ and the {\em homotopy} fibers of $\phi$.

\begin{lemma}\label{sharpy}
Let $p: \calC \rightarrow \calD$ be an inner fibration of $\infty$-categories, and let $X,Y \in \calC$. The induced map $\phi: \Hom^{\rght}_{\calC}(X,Y) \rightarrow \Hom^{\rght}_{\calD}(p(X),p(Y))$ is a Kan fibration.
\end{lemma}

\begin{proof}
Since $p$ is an inner fibration, the induced map $\widetilde{\phi}: \calC_{/X} \rightarrow \calD_{/p(X)} \times_{\calD} \calC$ is a right fibration by Proposition \ref{sharpen}. We note that $\phi$ is obtained from $\widetilde{\phi}$ by restricting to the fiber over the vertex $Y$ of $\calC$. Thus $\phi$ is a right fibration; since the target of $\phi$ is a Kan complex, $\phi$ is a Kan fibration by Lemma \ref{toothie2}.
\end{proof}

Suppose the conditions of Lemma \ref{sharpy} are satisfied. Let us consider the problem of computing the fiber of $\phi$ over a vertex $\overline{e}: p(X) \rightarrow p(Y)$ of $\Hom^{\rght}_{\calD}(X,Y)$. 
Suppose that there is a $p$-Cartesian edge $e: X' \rightarrow Y$ lifting $\overline{e}$. By definition, we have
a trivial fibration
$$ \psi: \calC_{/e} \rightarrow \calC_{/Y} \times_{ \calD_{/p(Y)}} \calD_{/\overline{e}}.$$
Consider the $2$-simplex $\sigma = s_1(\overline{e})$, regarded as a vertex of $\calD_{/\overline{e}}$. Passing to the fiber, we obtain a trivial fibration
$$ F \rightarrow \phi^{-1}(e),$$ where $F$ denotes the fiber of $\calC_{/e} \rightarrow \calD_{/\overline{e}} \times_{\calD} \calC$ over the point $(\sigma,X)$.
On the other hand, we have a trivial fibration
$\calC_{/e} \rightarrow \calD_{/\overline{e}} \times_{ \calD_{/p(X)} } \calC_{/X'}$ by Proposition \ref{sharpen2}. Passing to the fiber again, we obtain a trivial fibration $F \rightarrow \Hom^{\rght}_{\calC_{p(X)}}(X,X')$. We may summarize the situation as follows:

\begin{proposition}\label{compspaces}
Let $p: \calC \rightarrow \calD$ be an inner fibration of $\infty$-categories. Let $X,Y \in \calC$, let
$\overline{e}: p(X) \rightarrow p(Y)$ be a morphism in $\calD$, and let $e: X' \rightarrow Y$ be a locally $p$-Cartesian morphism of $\calC$ lifting $\overline{e}$. Then in the homotopy category $\calH$ of spaces, there is
a fiber sequence
$$ \bHom_{\calC_{p(X)}}(X,X') \rightarrow \bHom_{\calC}(X,Y) \rightarrow \bHom_{\calD}(p(X),p(Y)).$$
Here the fiber is taken over the point classified by $\overline{e}: p(X) \rightarrow p(Y)$. 
\end{proposition}

\begin{proof}
The edge $\overline{e}$ defines a map $\Delta^1 \rightarrow \calD$. Note that the fiber
of the Kan fibration $\Hom^{\rght}_{\calC}(X,Y) \rightarrow \Hom^{\rght}_{\calD}(pX, pY)$ does not change if we replace $p$ by the induced projection $\calC \times_{\calD} \Delta^1 \rightarrow \Delta^1$.
We may therefore assume without loss of generality that $e$ is $p$-Cartesian, and the desired result follows from the above analysis.
\end{proof}

A similar assertion can be taken as a characterization of Cartesian morphisms:

\begin{proposition}\label{charCart}
Let $p: \calC \rightarrow \calD$ be an inner fibration of $\infty$-categories, and let
$f: Y \rightarrow Z$ be a morphism in $\calC$. The following are equivalent:
\begin{itemize}
\item[$(1)$] The morphism $f$ is $p$-Cartesian.
\item[$(2)$] For every object $X$ of $\calC$, composition with $f$ gives rise to a homotopy
Cartesian diagram
$$ \xymatrix{ \bHom_{\calC}(X,Y) \ar[r] \ar[d] & \bHom_{\calC}(X,Z) \ar[d] \\
\bHom_{\calD}( p(X), p(Y) ) \ar[r] & \bHom_{\calD}(p(X), p(Z)).} $$
\end{itemize}
\end{proposition}

\begin{proof}
Let $\phi: \calC_{/f} \rightarrow \calC_{/Z} \times_{\calD_{/p(Z)}} \calD_{/p(f)}$ be the canonical map; then $(1)$ is equivalent to the assertion that $\phi$ is a trivial fibration. According to Proposition \ref{sharpen}, $\phi$ is a right fibration. Thus, $\phi$ is a trivial fibration if and only if the fibers of $\phi$ are contractible Kan complexes. For each object $X \in \calC$, let 
$$\phi_X:  \calC_{/f} \times_{\calC} \{X \} \rightarrow \calC_{/Z} \times_{\calD_{/p(Z)}} \calD_{/p(f)}
\times_{\calC} \{X\}$$ be the induced map. Then $\phi_{X}$ is a right fibration between Kan complexes, and therefore a Kan fibration; it has contractible fibers if and only if it is a homotopy equivalence. Thus, $(1)$ is equivalent to the assertion that $\phi_{X}$ is a homotopy equivalence for every object $X$ of $\calC$.

We remark that $(2)$ is somewhat imprecise: although all the maps in the diagram are well defined in the homotopy category $\calH$ of spaces, we need to represent this by a commutative diagram in the category of simplicial sets before we can ask whether or not the diagram is homotopy Cartesian. We therefore rephrase $(2)$ more precisely: it asserts that the diagram of Kan complexes
$$ \xymatrix{ \calC_{/f} \times_{\calC} \{X\} \ar[r] \ar[d] & \calC_{/Z} \times_{\calC} \{X\} \ar[d] \\
\calD_{/p(f)} \times_{\calD} \{p(X)\} \ar[r] & \calD_{/p(Z)} \times_{\calD} \{p(X)\} }$$
is homotopy Cartesian. Lemma \ref{sharpy} implies that the right vertical map is a Kan fibration, so the homotopy limit in question is given by the fiber product $$\calC_{/Z} \times_{\calD_{/p(Z)}} \calD_{/p(f)} \times_{\calC} \{X\}.$$ 
Consequently, assertion $(2)$ is also equivalent to the condition that
$\phi_{X}$ be a homotopy equivalence for every object $X \in \calC$.
\end{proof}

\begin{corollary}\label{usefir}
Suppose given maps $\calC \stackrel{p}{\rightarrow} \calD \stackrel{q}{\rightarrow} \calE$ of $\infty$-categories, such that both $q$ and $q \circ p$ are locally Cartesian fibrations. Suppose that $p$ carries
locally $(q \circ p)$-Cartesian edges of $\calC$ to locally $q$-Cartesian edges of $\calD$, and that for every object $Z \in \calE$, the induced map $\calC_{Z} \rightarrow \calD_{Z}$ is a categorical equivalence. Then $p$ is a categorical equivalence.
\end{corollary}

\begin{proof}
Proposition \ref{compspaces} implies that $p$ is fully faithful. If $Y$ is any object of $\calD$, then $Y$ is equivalent in the fiber $\calD_{q(Y)}$ to the image under $p$ of some vertex of $\calC_{q(Y)}$. Thus $p$ is essentially surjective and the proof is complete.
\end{proof}

\begin{corollary}\label{usesec}
Let $p: \calC \rightarrow \calD$ be a Cartesian fibration of $\infty$-categories. Let $q: \calD' \rightarrow \calD$ be a categorical equivalence of $\infty$-categories. Then the induced map $q': \calC' = \calD' \times_{\calD} \calC \rightarrow \calC$ is a categorical equivalence.
\end{corollary}

\begin{proof}
Proposition \ref{compspaces} immediately implies that $q'$ is fully faithful. We claim that $q'$ is essentially surjective. Let $X$ be any object of $\calC$. Since $q$ is fully faithful, there exists an object $y$ of $T'$ and an equivalence $\overline{e}: q(Y) \rightarrow p(X)$. Since $p$ is Cartesian, we can choose a $p$-Cartesian edge $e: Y' \rightarrow X$ lifting $\overline{e}$. Since $e$ is $p$-Cartesian and $p(e)$ is an equivalence, $e$ is an equivalence. By construction, the object $Y'$ of $S$ lies in the image of $q'$.
\end{proof}

\begin{corollary}\label{heath}\index{gen}{Cartesian fibration!and trivial fibrations}
Let $p: \calC \rightarrow \calD$ be a Cartesian fibration of $\infty$-categories. Then $p$ is a categorical equivalence if and only if $p$ is a trivial fibration.
\end{corollary}

\begin{proof}
The ``if'' direction is clear. Suppose then that $p$ is a categorical equivalence. We first claim
that $p$ is surjective on objects. The essential surjectivity of $p$ implies that for each $Y \in \calD$ there is an equivalence $Y \rightarrow p(X)$, for some object $X$ of $\calC$. Since $p$ is Cartesian, this equivalence lifts to a $p$-Cartesian edge $\widetilde{Y} \rightarrow X$ of $S$, so that $p(\widetilde{Y}) = Y$.

Since $p$ is fully faithful, the map $\bHom_{\calC}(X,X') \rightarrow \bHom_{\calD}(p(X),p(X'))$ is a homotopy equivalence
for any pair of objects $X,X' \in \calC$. Suppose that $p(X) = p(X')$. Then, applying Proposition \ref{compspaces}, we deduce that $\bHom_{\calC_{p(X)}}(X,X')$ is contractible.
It follows that the $\infty$-category $\calC_{p(X)}$ is nonempty with contractible morphism spaces; it is therefore a contractible Kan complex. Proposition \ref{goey} now implies that $p$ is a right fibration. Since $p$ has contractible fibers, it is a trivial fibration by Lemma \ref{toothie}.
\end{proof}

We have already seen that if a $\infty$-category $S$ has an initial
object, then that initial object is essentially unique. We now establish a relative version of this
result. 

% I think this is preproved somewhere later; should reference to here 

\begin{lemma}\label{sabreto}
Let $p: \calC \rightarrow \calD$ be a Cartesian fibration of $\infty$-categories, and let
$C$ be an object of $\calC$. Suppose that $D = p(C)$ is an initial object of $\calD$, and that
$C$ is an initial object of the $\infty$-category $\calC_{D} = \calC \times_{\calD} \{D\}$. 
Then $C$ is an initial object of $\calC$.
\end{lemma}

\begin{proof}
Let $C'$ be any object of $\calC$, and let $D' = p(C')$. Since $D$ is an initial object of
$\calD$, the space $\bHom_{\calD}(D,D')$ is contractible. In particular, there
exists a morphism $f: D \rightarrow D'$ in $\calD$. Let $\widetilde{f}: \widetilde{D} \rightarrow C'$
be a $p$-Cartesian lift of $f$. According to Proposition \ref{compspaces}, there exists a fiber sequence in the homotopy category $\calH$:
$$ \bHom_{\calC_{D}}(C, \widetilde{D}) \rightarrow \bHom_{\calC}(C,C') \rightarrow
\bHom_{\calD}(D,D').$$
Since the first and last space in the sequence are contractible, we deduce that $\bHom_{\calC}(C,C')$ is contractible as well, so that $C$ is an initial object of $\calC$. 
\end{proof}

\begin{lemma}\label{sabretooth}
Suppose given a diagram of simplicial sets
$$ \xymatrix{ \bd \Delta^n \ar[r]^{f_0} \ar@{^{(}->}[d] & X \ar[d]^{p} \\
\Delta^n \ar@{-->}[ur]^{f} \ar[r]^{g} & S }$$
where $p$ is a Cartesian fibration and $n > 0$. Suppose that
$f_0(0)$ is an initial object of the $\infty$-category $X_{g(0)} = X \times_{S} \{ g(0) \}$.
Then there exists a map $f: \Delta^n \rightarrow S$ as indicated by the dotted arrow in the diagram, which renders the diagram commutative.
\end{lemma}

\begin{proof}
Pulling back via $g$, we may replace $S$ by $\Delta^n$ and thereby reduce to the case where $S$ is an $\infty$-category and $g(0)$ is an initial object of $S$. It follows from Lemma \ref{sabreto} that 
$f_0(v)$ is an initial object of $S$, which implies the existence of the desired extension $f$.
\end{proof}

\begin{proposition}\label{topaz}
Let $p: X \rightarrow S$ be a Cartesian fibration of
simplicial sets. Assume that, for each vertex $s$ of $S$, the
$\infty$-category $X_{s} = X \times_{S} \{s\}$ has an initial object. 
\begin{itemize}
\item[$(1)$] Let $X' \subseteq X$
denote the full simplicial subset of $X$ spanned by those vertices $x$ which are initial objects of $X_{p(x)}$. Then $p|X'$ is a trivial fibration of simplicial sets.
\item[$(2)$] Let $\calC = \bHom_{S}(S, X)$ be the $\infty$-category of sections of $p$.
An arbitrary section $q: S \rightarrow X$ is an initial object of $\calC$ if and only if
$q$ factors through $X'$.
\end{itemize}
\end{proposition}

\begin{proof}
Since every fiber $X_{s}$ has an initial object, the map $p|X'$ has the right lifting
property with respect to the inclusion $\emptyset \subseteq \Delta^0$. If $n > 0$,
then Lemma \ref{sabretooth} shows that $p|X'$ has the right lifting property with
respect to $\bd \Delta^n \subseteq \Delta^n$. This proves $(1)$. In particular, we deduce
that there exists a map $q: S \rightarrow X'$ which is a section of $p$. In view of the uniqueness of initial objects, $(2)$ will follow if we can show that $q$ is an initial object of $\calC$.
Unwinding the definitions, we must show that for $n > 0$, any lifting problem
$$ \xymatrix{ S \times \bd \Delta^n \ar[r]^{f} \ar@{^{(}->}[d] & X \ar[d]^{q} \\
S \times \Delta^n \ar[r] \ar@{-->}[ur] & S}$$
can be solved, provided that $f | S \times \{0\} = q$. The desired extension can be constructed simplex-by-simplex, using Lemma \ref{sabretooth}. 
\end{proof}

\subsection{Application: Invariance of  Undercategories}\label{slim}

Our goal in this section is to complete the proof of Proposition \ref{gorban3} by proving the
following assertion:

\begin{itemize}\index{gen}{undercategory}
\item[$(\ast)$] Let $p: \calC \rightarrow \calD$ be an equivalence of $\infty$-categories, and let
$j: K \rightarrow \calC$ be a diagram. Then the induced map
$$ \calC_{j/} \rightarrow \calD_{p j/}$$
is a categorical equivalence.
\end{itemize}

We will need a lemma.

\begin{lemma}\label{blaha}
Let $p: \calC \rightarrow \calD$ be a fully faithful map of $\infty$-categories, and let $j: K \rightarrow \calC$
be any diagram in $\calC$. Then, for any object $x$ of $\calC$, the map of Kan complexes
$$ \calC_{j/} \times_{\calC} \{x\} \rightarrow \calD_{p j/ } \times_{\calD} \{p(x) \}$$ is a homotopy equivalence.
\end{lemma}

\begin{proof}
For any map $r: K' \rightarrow K$ of simplicial sets, let $C_{r} = \calC_{j r/} \times_{\calC} \{x\}$ and $D_{r} = \calD_{pjr/} \times_{\calD} \{p(x)\}$. 

Choose a transfinite sequence of simplicial subsets $K_{\alpha}$ of $K$, such that $K_{\alpha+1}$ is the result of adjoining a single nondegenerate simplex to $K_{\alpha}$, and $K_{\lambda} = \bigcup_{\alpha<\lambda} K_{\alpha}$ whenever $\lambda$ is a limit ordinal (we include the case where $\lambda = 0$, so that $K_0 = \emptyset$). Let $i_{\alpha}: K_{\alpha} \rightarrow K$ denote the inclusion. We claim the following:
\begin{itemize}
\item[$(1)$] For every ordinal $\alpha$, the map $\phi_{\alpha}: C_{i_{\alpha}} \rightarrow D_{i_{\alpha}}$ is a homotopy equivalence of simplicial sets.
\item[$(2)$] For every pair of ordinals $\beta \leq \alpha$, the maps $C_{i_{\alpha}} \rightarrow C_{i_{\beta}}$ and $D_{i_{\alpha}} \rightarrow D_{i_{\beta}}$ are Kan fibrations of simplicial sets.
\end{itemize}

We prove both of these claims by induction on $\alpha$. When $\alpha = 0$, $(2)$ is obvious and $(1)$ follows since both sides are isomorphic to $\Delta^0$. If $\alpha$ is a limit ordinal, $(2)$ is again obvious, while $(1)$ follows from the fact that both $C_{i_{\alpha}}$ and $D_{i_{\alpha}}$
are obtained as the inverse limit of a transfinite sequence of fibrations, and the map $\phi_{\alpha}$ is an inverse limit of maps which are individually homotopy equivalences.

Assume that $\alpha=\beta+1$ is a successor ordinal, so that $K_{\alpha} \simeq K_{\beta} \coprod_{ \bd \Delta^n } \Delta^n$. Let $f: \Delta^n \rightarrow K_{\alpha}$ be the induced map, so that
$$ C_{i_\alpha} = C_{i_{\beta}} \times_{ C_{ f | \bd \Delta^n } } C_{f} $$
$$ D_{i_\alpha} = D_{i_{\beta}} \times_{ D_{ f| \bd \Delta^n }} D_{f}.$$
We note that the projections $C_{f} \rightarrow D_{f | \bd \Delta^n }$ and $C_{f} \rightarrow D_{f| \bd \Delta^n}$ are left fibrations by Proposition \ref{sharpen}, and therefore Kan fibrations by Lemma \ref{toothie2}. This proves $(2)$, since the class of Kan fibrations is stable under pullback.
We also note that the pullback diagrams defining $X_{i_{\alpha}}$ and $Y_{i_{\alpha}}$ are also  homotopy pullback diagrams. Thus, to prove that $\phi_{\alpha}$ is a homotopy equivalence, it suffices to show that $\phi_{\beta}$ and the maps
$$ C_{f| \bd \Delta^n } \rightarrow D_{ f| \bd \Delta^n }$$
$$ C_{f} \rightarrow D_{f}$$
are homotopy equivalences. In other words, we may reduce to the case where $K$ is a {\em finite} complex. 

We now work by induction on the dimension of $K$. Suppose that the dimension of $K$ is $n$, and that the result is known for all simplicial sets having smaller dimension. Running through the above argument again, we can reduce to the case where $K = \Delta^n$. Let $v$ denote the final vertex of $\Delta^n$. By Proposition \ref{sharpen2}, the maps
$$ C_{j} \rightarrow C_{ j| \{v\} }$$
$$ D_{j} \rightarrow D_{ j| \{v\} }$$
are trivial fibrations. Thus, it suffices to consider the case where $K$ is a single point $\{v\}$. 
In this case, we have $C_{j} = \Hom^{\lft}_{\calC}( j(v), x)$ and $Y_{j} = \Hom^{\lft}_{\calD}( p(j(v)), p(x))$.
It follows that the map $\phi$ is a homotopy equivalence, since $p$ is assumed fully faithful.
\end{proof}

\begin{proof}[Proof of $(\ast)$]
Let $p: \calC \rightarrow \calD$ be a categorical equivalence of $\infty$-categories, and $j: K \rightarrow \calC$ any diagram.
We have a factorization
$$ \calC_{j/} \stackrel{f}{\rightarrow} \calD_{p j/} \times_{\calD} \calC \stackrel{g}{\rightarrow} \calD_{pj/}.$$
Lemma \ref{blaha} implies that $\calC_{j/}$ and $\calD_{p j/} \times_{\calD} \calC$ are fiberwise equivalent left-fibrations over $\calC$, so that $f$ is a categorical equivalence by Corollary \ref{usefir} (we note that the map $f$
automatically carries coCartesian edges to coCartesian edges, since {\em all} edges of the target $\calD_{p j/} \times_{\calD} \calC$ are coCartesian). The map $g$ is a categorical equivalence by Corollary \ref{usesec}. It follows that $g \circ f$ is a categorical equivalence, as desired.
\end{proof} 

\subsection{Application: Categorical Fibrations over a Point}\label{slin}

Our main goal in this section is to prove the following result:

\begin{theorem}\label{joyalcharacterization}\index{gen}{$\infty$-category!as a fibrant object of $\sSet$}
Let $\calC$ be a simplicial set. Then $\calC$ is fibrant for the Joyal model structure if and only if $\calC$ is an $\infty$-category.
\end{theorem}

The proof will require a few technical preliminaries.

\begin{lemma}\label{gorban2}
Let $p: \calC \rightarrow \calD$ be a categorical equivalence of $\infty$-categories and $m \geq 2$ an integer. Suppose given maps $f_0: \bd \Delta^{ \{1, \ldots, m\} } \rightarrow \calC$ and $h_0: \Lambda^m_0 \rightarrow \calD$ with
$h_0 | \bd \Delta^{ \{1, \ldots, m\} } = p \circ f_0$. Suppose further that the restriction of $h$ to $\Delta^{ \{0,1\} }$ is an equivalence in $\calD$. Then there exist maps $f: \Delta^{ \{1, \ldots, m\} } \rightarrow \calC$, $h: \Delta^{m} \rightarrow \calD$, with $h| \Delta^{ \{1, \ldots, n\} }= p \circ f$, $f_0 = f| \bd \Delta^{ \{1, \ldots, m\} }$, $h_0 = h| \Lambda^m_0$.
\end{lemma}

\begin{proof}
We may regard $h_0$ as a point of the simplicial set $\calD_{/ p \circ f_0}$. Since $p$ is a categorical equivalence, Proposition \ref{gorban3} implies that $p': \calC_{/f_0} \rightarrow \calD_{/p \circ f_0}$ is a categorical equivalence. It follows that $h_0$ lies in the essential image of $p'$. Consider the linearly ordered set
$\{ 0 < 0' < 1 < \ldots < n\}$ and the corresponding simplex $\Delta^{ \{0, 0', \ldots, n\} }$. By hypothesis, we can extend
$f_0$ to a map $f'_0: \Lambda^{ \{ 0', \ldots, m \} }_{0'} \rightarrow \calC$ and $h_0$ to a map $h'_0: \Delta^{ \{0,0'\} } \star \bd \Delta^{ \{1, \ldots, m \} } \rightarrow \calD$ such that $h'_0| \Delta^{ \{0,0'\} }$ is an equivalence
and $h'_0| \Lambda^{ \{0', \ldots, m\} }_0 = p \circ f'_0$. 

Since $h'_0| \Delta^{ \{0,0'\} }$ and $h'_0| \Delta^{ \{0,1\} }$ are both equivalences in $\calD$, we deduce that $h'_0| \Delta^{ \{0',1\} }$ is an equivalence in $\calD$. Since $p$ is a categorical equivalence, it follows that
$f'_0| \Delta^{ \{0',1\} }$ is an equivalence in $\calC$. Proposition \ref{greenlem} implies that $f'_0$ extends to a map $f': \Delta^{ \{0', \ldots, m\} } \rightarrow \calC$. The union of $p \circ f'$ and $h'_0$ determines a map
$\Lambda^{ \{0, 0', \ldots, m \} }_{0'} \rightarrow \calD$; since $\calD$ is an $\infty$-category, this extends to a map
$h': \Delta^{ \{0,0', \ldots, m\} } \rightarrow \calD$. We may now take $f = f' | \Delta^{ \{1, \ldots, m \} }$ and
$h = h' | \Delta^{m}$.
\end{proof}

\begin{lemma}\label{gorban}
Let $p: \calC \rightarrow \calD$ be a categorical equivalence of $\infty$-categories and $A \subseteq B$ any inclusion of simplicial sets. Let $f_0: A \rightarrow \calC$, $g: B \rightarrow \calD$ be any maps, and let $h_0: A \times \Delta^1 \rightarrow \calD$ be an equivalence from $g|A$  to $p \circ f_0$. Then there exists a map $f: B \rightarrow \calC$ and an equivalence $h: B \times \Delta^1 \rightarrow \calD$ from $g$ to $p \circ f$, such that
$f_0 = f|A$ and $h_0 = h| A \times \Delta^1$.
\end{lemma}

\begin{proof}
Working cell-by-cell with the inclusion $A \subseteq B$, we may reduce to the case where $B = \Delta^n$, $A = \bd \Delta^n$. If $n = 0$, the existence of the desired extensions is a reformulation of the assumption that $p$ is essentially surjective. Let us assume therefore that $n \geq 1$.

We consider the task of constructing $h: \Delta^n \times \Delta^1 \rightarrow \calD$. Consider the filtration
$$ X(n+1) \subseteq \ldots \subseteq X(0) = \Delta^n \times \Delta^1 $$
described in the proof of Proposition \ref{usejoyal}. We note that the value of $h$ on $X(n+1)$ is uniquely prescribed by $h_0$ and $g$. We extend the definition of $h$ to $X(i)$ by descending induction on $i$. We note that $X(i) \simeq X(i+1) \coprod_{ \Lambda^{n+1}_k} \Delta^{n+1}$. For $i > 0$, the existence of the required extension is guaranteed by the assumption that $\calD$ is an $\infty$-category. Since $n \geq 1$, Lemma \ref{gorban2}  allows us to extend $h$ over the simplex $\sigma_0$ and to define $f$ so that the desired conditions are satisfied.
\end{proof}

\begin{lemma}\label{gorbann}
Let $\calC \subseteq \calD$ be an inclusion of simplicial sets which is also a categorical equivalence. Suppose further that $\calC$ is an $\infty$-category. Then $\calC$ is a retract of $\calD$.
\end{lemma}

\begin{proof}
Enlarging $\calD$ by an inner anodyne extension if necessary, we may suppose that $\calD$ is an $\infty$-category. We now apply Lemma \ref{gorban} in the case
where $A = \calC$, $B = \calD$.
\end{proof}

\begin{proof}[Proof of Theorem \ref{joyalcharacterization}]
The ``only if'' direction has already been established (Remark \ref{tokenn}). 
For the converse, we must show that if $\calC$ is an $\infty$-category, then $\calC$ has the extension property with respect to every inclusion of simplicial sets
$A \subseteq B$ which is a categorical equivalence. Fix any map $A \rightarrow \calC$. Since the Joyal model structure is left-proper, the inclusion
$\calC \subseteq \calC \coprod_{A} B$ is a categorical equivalence. We now apply Lemma \ref{gorbann} to conclude that $\calC$ is a retract of $\calC \coprod_{A} B$.
\end{proof}

We can state Theorem \ref{joyalcharacterization} as follows: if $S$ is a point, then
$p: X \rightarrow S$ is a categorical fibration (in other words, a fibration with respect to the Joyal model structure on $\SSet$) if and only if it is an inner fibration. However, the class of inner fibrations does {\em not} coincide with the class of categorical fibrations in general. The following result describes the situation when $T$ is an $\infty$-category:

\begin{corollary}[Joyal]\index{gen}{categorical fibration!of $\infty$-categories}\label{gottaput}
Let $p: \calC \rightarrow \calD$ be a map of simpicial sets, where $\calD$ is an $\infty$-category.
Then $p$ is a categorical fibration if and only if the following conditions are satisfied:
\begin{itemize}
\item[$(1)$] The map $p$ is an inner fibration.
\item[$(2)$] For every equivalence $f: D \rightarrow D'$ in $\calD$, and every
object $C \in \calC$ with $p(C) = D$, there exists an equivalence
$\overline{f}: C \rightarrow C'$ in $\calC$ with $p( \overline{f}) = f$.
\end{itemize}
\end{corollary}

\begin{proof}
Suppose first that $p$ is a categorical fibration. Then $(1)$ follows immediately (since the inclusions $\Lambda^n_i \subseteq \Delta^n$ are categorical equivalences for $0 < i < n$).
To prove $(2)$, we let $\calD^{0}$ denote the largest Kan complex contained in $\calD$, so that
the edge $f$ belongs to $\calD$. There exists a contractible Kan complex $K$ containing
an edge $\widetilde{f}: \widetilde{D} \rightarrow \widetilde{D}'$ and a map $q: K \rightarrow \calD$ such that $q( \widetilde{f} ) = f$. Since the inclusion $\{ \widetilde{\calD} \} \subseteq K$ is a categorical equivalence, our assumption that $p$ is a categorical fibration allows us to lift
$q$ to a map $\widetilde{q}: K \rightarrow \calC$ such that $\widetilde{q}(\widetilde{D})=C$.
We can now take $\overline{f} = \widetilde{q}( \widetilde{f} )$; since $\widetilde{f}$ is an equivalence in $K$, $\overline{f}$ is an equivalence in $\calC$.

Now suppose that $(1)$ and $(2)$ are satisfied. We wish to show that $p$ is a categorical fibration. Consider a lifting problem
$$ \xymatrix{ A \ar@{^{(}->}[d]^{i} \ar[r]^{g_0} & \calC \ar[d]^{p} \\
B \ar[r]^{h} \ar@{-->}[ur]^{g} & \calD }$$
where $i$ is a cofibration and a categorical equivalence; we wish to show that 
there exists a morphism $g$ as indicated which renders the diagram commutative.
We first observe that condition $(1)$, together with our assumption that $\calD$ is an $\infty$-category, guarantee that $\calC$ is an $\infty$-category.
Applying Theorem \ref{joyalcharacterization}, we can extend $g_0$ to a map
$g': B \rightarrow \calC$, not necessarily satisfying $h = p \circ g'$.
Nevertheless, the maps $h$ and $p \circ g'$ have the same restriction to $A$.
Let 
$$H_0: (B \times \bd \Delta^1) \coprod_{ A \times \bd \Delta^1 } (A \times \Delta^1) \rightarrow \calD$$
be given by $(p \circ g',h)$ on $B \times \bd \Delta^1$, and by the composition
$$ A \times \Delta^1 \rightarrow A \subseteq B \stackrel{h}{\rightarrow} \calD$$
on $A \times \Delta^1$. Applying Theorem \ref{joyalcharacterization} once more, we deduce
that $H_0$ extends to a map $H: B \times \Delta^1 \rightarrow \calD$. The map $H$ carries
$\{a \} \times \Delta^1$ to an equivalence in $\calD$, for every vertex $a$ of $A$. Since
the inclusion $A \subseteq B$ is a categorical equivalence, we deduce that
$H$ carries $\{b \} \times \Delta^1$ to an equivalence, for every $b \in B$.

Let $$G_0: (B \times \{0\}) \coprod_{ A \times \{0\} } (A \times \Delta^1) \rightarrow \calC$$
be the composition of the projection to $B$ with the map $g'$. We have a commutative diagram
$$ \xymatrix{  (B \times \{0\}) \coprod_{ A \times \{0\} } (A \times \Delta^1) \ar[r]^-{G_0} \ar[d] & \calC \ar[d]^{p} \\
B \times \Delta^1 \ar[r]^{H} \ar@{-->}[ur]^{G} & \calD. }$$
To complete the proof, it will suffice to show that we can supply a map $G$ as indicated, rendering the diagram commutative; in this case, we can solve the original lifting problem by defining
$g = G | B \times \{1\}$.

We construct the desired extension $G$ working cell-by-cell on $B$. We start by applying assumption $(2)$ to construct the map $G| \{b\} \times \Delta^1$ for every vertex $b$ of $B$ (that does not already belong to $A$); moreover, we ensure that $G| \{b\} \times \Delta^1$ is an equivalence in $\calC$.

To extend $G_0$ to simplices of higher dimension, we encounter lifting problems of the type
$$ \xymatrix{ ( \Delta^n \times \{0\} ) \coprod_{ \bd \Delta^n \times \{0\} } ( \bd \Delta^n \times \Delta^1 ) \ar[r]^-{e} \ar@{^{(}->}[d] & \calC \ar[d]^{p} \\
\Delta^n \times \Delta^1 \ar[r] \ar@{-->}[ur] & \calD. }$$
According to Proposition \ref{goouse}, these lifting problems can be solved provided that
$e$ carries $\{ 0\} \times \Delta^1$ to a $p$-coCartesian edge of $\calC$. This follows immediately from Proposition \ref{universalequiv}.
\end{proof}

\subsection{Bifibrations}\label{bifib}

As we explained in \S \ref{leftfib}, left fibrations $p: X \rightarrow S$ can be thought of as {\em covariant} functors from $S$ into an $\infty$-category of spaces. Similarly, right fibrations $q: Y \rightarrow T$ can be thought of as {\em contravariant} functors from $T$ into an $\infty$-category of spaces. The purpose of this section is to introduce a convenient formalism which encodes covariant and contravariant functoriality simultaneously.

\begin{remark}
The theory of bifibrations will not play an important role in the remainder of the book. In fact, the only result from this section that we will actually use is Corollary \ref{tweezegork}, whose statement makes no mention of bifibrations. A reader who is willing to take Corollary \ref{tweezegork} on faith, or supply an alternative proof, may safely omit the material covered in this section.
\end{remark}

\begin{definition}\label{biv}\index{gen}{bifibration}
Let $S$, $T$, and $X$ be simplicial sets, and $p: X \rightarrow S
\times T$ a map. We shall say that $p$ is a {\it bifibration} if it is an inner fibration having the following
properties:

\begin{itemize}
\item For every $n\geq 1$ and every diagram of solid arrows
$$ \xymatrix{ \Lambda^n_0 \ar@{^{(}->}[d] \ar[r] & X \ar[d] \\
\Delta^n \ar@{-->}[ur] \ar[r]^{f} & S \times T}$$
such that $\pi_T \circ f$ maps $\Delta^{ \{0,1\} } \subseteq \Delta^n$ to a degenerate edge
of $T$, there exists a dotted arrow as indicated, rendering the diagram commutative. 
Here $\pi_T$ denotes the projection $S \times T \rightarrow T$.

\item For every $n\geq 1$ and every diagram of solid arrows
$$ \xymatrix{ \Lambda^n_n \ar@{^{(}->}[d] \ar[r] & X \ar[d] \\
\Delta^n \ar@{-->}[ur] \ar[r]^{f} & S \times T}$$
such that $\pi_S \circ f$ maps $\Delta^{ \{n-1,n\} } \subseteq \Delta^n$ to a degenerate edge
of $T$, there exists a dotted arrow as indicated, rendering the diagram commutative. 
Here $\pi_S$ denotes the projection $S \times T \rightarrow S$.

\end{itemize}
\end{definition}

\begin{remark}
The condition that $p$ be a bifibration is not a condition on $p$ alone, but refers also to a decomposition of the codomain of $p$ as a product $S \times T$. We note also that the definition is not symmetric in $S$ and $T$: instead, $p: X \rightarrow S \times T$ is a bifibration if and only if $p^{op}: X^{op} \rightarrow T^{op} \times S^{op}$ is a bifibration.
\end{remark}

\begin{remark}
Let $p: X \rightarrow S \times T$ be a map of simplicial sets. If $T = \ast$, then $p$ is a bifibration if and only if it is a {\em left} fibration. If $S = \ast$, then $p$ is a bifibration if and only if it is a {\em right} fibration.
\end{remark}

Roughly speaking, we can think of a bifibration $p: X \rightarrow S \times T$ as a bifunctor from $S \times T$ to an $\infty$-category of spaces; the functoriality is covariant in $S$ and contravariant in $T$.

\begin{lemma}\label{gork}
Let $p: X \rightarrow S \times T$ be a bifibration of simplicial sets. Suppose that
$S$ is an $\infty$-category. Then the composition $q=\pi_{T} \circ p$ is a Cartesian fibration of simplicial sets. Furthermore, an edge $e$ of $X$ is $q$-Cartesian if and only if
$\pi_{S}(p(e))$ is an equivalence.
\end{lemma}

\begin{proof}
The map $q$ is an inner fibration, since it is a composition of inner fibrations. Let us say that an edge $e: x \rightarrow y$ of $X$ is {\it quasi-Cartesian} if $\pi_S(p(e))$ is degenerate in $S$. 
Let $y \in X_0$ be any vertex of $X$, and $\overline{e}: \overline{x} \rightarrow q(y)$ an edge of $S$. The pair $(\overline{e}, s_0 q(y))$ is an edge of $S \times T$ whose projection to $T$ is degenerate; consequently, it lifts to a (quasi-Cartesian) edge $e: x \rightarrow y$ in $X$. It is immediate from Definition \ref{biv} that any quasi-Cartesian edge of $X$ is $q$-Cartesian. Thus, $q$ is a Cartesian fibration.

Now suppose that $e$ is a $q$-Cartesian edge of $X$. Then $e$ is equivalent to a quasi-Cartesian edge of $X$; it follows easily that $\pi_{S}(p(e))$ is an equivalence. Conversely, suppose that $e: x \rightarrow y$ is an edge of $X$ and that $\pi_{S}(p(e))$ is an equivalence. We wish to show that $e$ is $q$-Cartesian. Choose a quasi-Cartesian edge
$e': x' \rightarrow y$ with $q(e')=q(e)$. Since $e'$ is $q$-Cartesian, there exists a simplex $\sigma \in X_2$ with $d_0 \sigma = e'$, $d_1 \sigma = e$, and $q(\sigma) = s_0 q(e)$. Let $f = d_2(\sigma)$, so that $\pi_S (p(e')) \circ  \pi_{S} (p(f)) \simeq \pi_S p(e)$ in the $\infty$-category $S$. We note that $f$ lies in the fiber $X_{q(x)}$, which is left-fibered over $S$; since $f$ maps to an equivalence in $S$, it is an equivalence in $X_{q(x)}$. Consequently, $f$ is $q$-Cartesian, so that
$e = e' \circ f$ is $q$-Cartesian as well.
\end{proof}

\begin{proposition}\label{equivbifib}
Let $X \stackrel{p}{\rightarrow} Y \stackrel{q}{\rightarrow} S \times T$ be a diagram of simplicial sets.
Suppose that $q$ and $q \circ p$ are bifibrations, and that $p$ induces a homotopy equivalence $X_{(s,t)} \rightarrow Y_{(s,t)}$ of fibers over each vertex $(s,t)$ of $S \times T$. Then $p$ is a categorical equivalence.
\end{proposition}

\begin{proof}
By means of a standard argument (see the proof of Proposition \ref{babyy}) we may reduce to the case where $S$ and $T$ are simplices; in particular, we may suppose that $S$ and $T$ are $\infty$-categories.
Fix $t \in T_0$, and consider the map of fibers $p_{t}: X_{t} \rightarrow Y_{t}$. Both sides are left-fibered over $S \times \{t\}$, so that $p_{t}$ is a categorical equivalence by (the dual of) Corollary \ref{usefir}. We may then apply Corollary \ref{usefir} again (along with the characterization of Cartesian edges given in Lemma \ref{gork}) to deduce that $p$ is a categorical equivalence.
\end{proof}

\begin{proposition}\label{equivbifib2}
Let $p: X \rightarrow S \times T$ be a bifibration, let $f: S' \rightarrow S$,
$g: T' \rightarrow T$ be categorical equivalences {\em between $\infty$-categories}, and let $X' = X \times_{S \times T} (S' \times T')$.
Then the induced map $X' \rightarrow X$ is a categorical equivalence.
\end{proposition}

\begin{proof}
We will prove the result assuming that $f$ is an isomorphism. A dual argument will establish the result when $g$ is an isomorphism, and applying the result twice we will deduce the desired statement for arbitrary $f$ and $g$.

Given a map $i: A \rightarrow S$, let us say that $i$ is {\it good} if the induced map
$X \times_{S \times T} (A \times T') \rightarrow X \times_{S \times T} (A \times T')$
is a categorical equivalence. We wish to show that the identity map $S \rightarrow S$ is good; it will suffice to show that {\em all} maps $A \rightarrow S$ are good. Using the argument of Proposition \ref{babyy}, we can reduce to showing that every map $\Delta^n \rightarrow S$ is good. In other words, we may assume that $S = \Delta^n$, and in particular that $S$ is an $\infty$-category.
By Lemma \ref{gork}, the projection $X \rightarrow T$ is a Cartesian fibration. The desired result now follows from Corollary \ref{usesec}.
\end{proof}

We next prove an analogue of Lemma \ref{gorban}.

\begin{lemma}\label{gorbaniz}
Let $X \stackrel{p}{\rightarrow} Y \stackrel{q}{\rightarrow} S \times T$ satisfy the hypotheses of Proposition \ref{equivbifib}. Let $A \subseteq B$ be a cofibration of simplicial sets {\em over} $S \times T$. 
Let $f_0: A \rightarrow X$, $g: B \rightarrow Y$ be morphisms in $(\sSet)_{/ S \times T}$ 
and let $h_0: A \times \Delta^1 \rightarrow Y$ be a homotopy (again over $S \times T$) from $g|A$  to $p \circ f_0$.

Then there exists a map $f: B \rightarrow X$ (of simplicial sets over $S \times T$) and a homotopy $h: B \times \Delta^1 \rightarrow T$ (over $S \times T$) from $g$ to $p \circ f$, such that
$f_0 = f|A$ and $h_0 = h| A \times \Delta^1$.
\end{lemma}

\begin{proof}
Working cell-by-cell with the inclusion $A \subseteq B$, we may reduce to the case where $B = \Delta^n$, $A = \bd \Delta^n$. If $n = 0$, we may invoke the fact that $p$ induces a surjection
$\pi_0 X_{(s,t)} \rightarrow \pi_0 Y_{(s,t)}$ on each fiber. Let us assume therefore that $n \geq 1$.
Without loss of generality, we may pull back along the maps $B \rightarrow S$, $B \rightarrow T$, and reduce to the case where $S$ and $T$ are simplices.

We consider the task of constructing $h: \Delta^n \times \Delta^1 \rightarrow T$. We 
now employ the filtration
$$ X(n+1) \subseteq \ldots \subseteq X(0) $$
described in the proof of Proposition \ref{usejoyal}. We note that the value of $h$ on $X(n+1)$ is uniquely prescribed by $h_0$ and $g$. We extend the definition of $h$ to $X(i)$ by descending induction on $i$. We note that $X(i) \simeq X(i+1) \coprod_{ \Lambda^{n+1}_k} \Delta^{n+1}$. For $i > 0$, the existence of the required extension is guaranteed by the assumption that $Y$ is inner-fibered over $S \times T$.

We note that, in view of the assumption that $S$ and $T$ are simplices, any extension of of $h$ over the simplex $\sigma_0$ is automatically a map {\em over} $S \times T$. Since $S$ and $T$ are $\infty$-categories, Proposition \ref{equivbifib} implies that $p$ is a categorical equivalence of $\infty$-categories; the existence of the desired extension of $h$ (and the map $f$ now follows from Lemma \ref{gorban2}.
\end{proof}

\begin{proposition}\label{lemba}
Let $X \stackrel{p}{\rightarrow} Y \stackrel{q}{\rightarrow} S \times T$ satisfy the hypotheses of Proposition \ref{equivbifib}. Suppose that $p$ is a cofibration. Then there exists a retraction $r: Y \rightarrow X$ $($as a map of simplicial sets {\em over} $S \times T${}$)$ such that $r \circ p = \id_X$.
\end{proposition}

\begin{proof}
Apply Lemma \ref{gorbaniz} in the case $A = X$, $B = Y$.
\end{proof}

Let $q: \calM \rightarrow \Delta^1$ be an inner fibration, which we view as a correspondence from
$\calC = q^{-1} \{0\}$ to $\calD = q^{-1} \{1\}$. Evaluation at the endpoints of
$\Delta^1$ induces maps $\bHom_{\Delta^1}(\Delta^1, \calM) \rightarrow \calC$, $\bHom_{\Delta^1}(\Delta^1,\calM) \rightarrow \calD$.

\begin{proposition}\label{tweez}\index{gen}{bifibration!associated to a correspondence}
For every inner fibration $q: \calM \rightarrow \Delta^1$ as above, the map
$p: \bHom_{\Delta^1}(\Delta^1, \calM) \rightarrow \calC \times \calD$ is a bifibration.
\end{proposition}

\begin{proof}
We first show that $p$ is an inner fibration. It suffices to prove
that $q$ has the right-lifting property with respect to
$$ ( \Lambda^n_i \times \Delta^1 ) \coprod_{ \Lambda^n_i \times
\bd \Delta^1 } (\Delta^n \times \bd \Delta^1) \subseteq \Delta^n
\times \Delta^1, $$ for any $0 < i < n$. But this is a smash
product of $\bd \Delta^1 \subseteq \Delta^1$ with the inner anodyne
inclusion $\Lambda^n_i \subseteq \Delta^n$.

To complete the proof that $p$ is a bifibration, we verify that
every $n\geq 1$, $f_0: \Lambda^0_n \rightarrow X$ and
$g: \Delta^n \rightarrow S \times T$ with $g|\Lambda^n_0 = p \circ f_0$, if 
$(\pi_S \circ g) | \Delta^{ \{0,1\} }$ is degenerate, then there exists
$f: \Delta^n \rightarrow X$ with $g= p \circ f$ and $f_0 = f | \Lambda^n_0$. (The dual assertion,
regarding extensions of maps $\Lambda^n_n \rightarrow X$, is verified in the same way.)
The pair $(f_0,g)$ may be regarded as a map
$$h_{0}: (\Delta^n \times \{0,1\}) \coprod_{ \Lambda^n_0 \times \{0,1\} } (\Lambda^n_0 \times \Delta^1)
\rightarrow \calM$$
and our goal is to prove that $h_0$ extends to a map $h: \Delta^n \times \Delta^1 \rightarrow \calM$. 

Let $\{ \sigma_i \}_{0 \leq i \leq n}$ be the maximal-dimensional simplices of $\Delta^n \times \Delta^1$, as in the proof of Proposition \ref{usejoyal}. We set $$K(0) = (\Delta^n \times \{0,1\}) \coprod_{ \Lambda^n_0 \times \{0,1\} } (\Lambda^n_0 \times \Delta^1)$$ and, for $0 \leq i \leq n$, let $K(i+1) = K(i) \bigcup \sigma_i$. We construct maps $h_i: K_i \rightarrow \calM$, with $h_{i} = h_{i+1} | K_{i}$, by induction on $i$.  We note that for $i < n$, $K(i+1) \simeq K(i) \coprod_{ \Lambda^{n+1}_{i+1} } \Delta^{n+1}$, so that the desired extension exists in virtue of the assumption that $\calM$ is an $\infty$-category. If $i = n$, we have instead an isomorphism $\Delta^n \times \Delta^1 = K(n+1) \simeq K(n) \coprod_{ \Lambda^{n+1}_0 } \Delta^{n+1}$. The desired extension of $h_n$ can be found by Proposition \ref{greenlem}, since $h_0 | \Delta^{ \{0,1\} } \times \{0\}$ is an equivalence in $\calC \subseteq \calM$ by assumption.
\end{proof}

\begin{corollary}\label{tweeze}
Let $\calC$ be an $\infty$-category. Evaluation at the endpoints gives a bifibration
$\Fun(\Delta^1,\calC) \rightarrow \calC \times \calC$.
\end{corollary}

\begin{proof}
Apply Proposition \ref{tweez} to the correspondence $\calC \times \Delta^1$.
\end{proof}

\begin{corollary}\label{tweezegork}
Let $f: \calC \rightarrow \calD$ be a functor between $\infty$-categories. The projection
$$ \Fun(\Delta^1, \calD) \times_{\Fun( \{1\}, \calD) } \calC \rightarrow \Fun(\{0\}, \calD)$$
is a Cartesian fibration.
\end{corollary}

\begin{proof}
Combine Corollary \ref{tweeze} with Proposition \ref{gork}.
\end{proof}
