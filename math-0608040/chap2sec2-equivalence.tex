\section{Simplicial Categories and $\infty$-Categories}\label{valencequi}


\setcounter{theorem}{0}

For every topological category $\calC$ and every pair of objects $X,Y \in \calC$, Theorem \ref{biggie} asserts that the counit map
$$u: |\bHom_{\sCoNerve[\tNerve(\calC)] }(X,Y)| \rightarrow \bHom_{\calC}(X,Y)$$
is a weak homotopy equivalence of topological spaces. This result is the main ingredient needed to establish the equivalence between the theory of topological categories and the theory of $\infty$-categories. The goal of this section is to give a proof of Theorem \ref{biggie} and to develop some of its consequences. 

We first replace Theorem \ref{biggie} by a statement about {\em simplicial} categories. Consider the composition
$$ \bHom_{\sCoNerve[ \tNerve(\calC) ]}(X,Y) \stackrel{v}{\rightarrow}
\Sing \bHom_{| \sCoNerve[ \tNerve(\calC)] |}(X,Y) \stackrel{ \Sing(u)}{\rightarrow} \Sing \bHom_{\calC}(X,Y).$$
Classical homotopy theory ensures that $v$ is a weak homotopy equivalence. Moreover, $u$ is a weak homotopy equivalence of topological spaces if and only if $\Sing(u)$ is a weak homotopy equivalence of simplicial sets. Consequently, $u$ is a weak homotopy equivalence of topological spaces if and only if $\Sing(u) \circ v$ is a weak homotopy equivalence of simplicial sets.
It will therefore suffice to prove the following {\em simplicial} analogue of Theorem \ref{biggie}:

\begin{theorem}\label{biggiesimp}
Let $\calC$ be a fibrant simplicial category $($that is, a simplicial category in which each mapping
space $\bHom_{\calC}(x,y)$ is a Kan complex$)$, and let $x, y \in \calC$ be a pair of objects. The counit map
$$ u: \bHom_{ \sCoNerve[\sNerve(\calC)]}(x,y) \rightarrow \bHom_{\calC}(x,y)$$ is a weak
homotopy equivalence of simplicial sets. 
\end{theorem}

The proof will be given in \S \ref{compp2} (see Proposition \ref{wiretrack}).
Our strategy is as follows:

\begin{itemize}
\item[$(1)$] We will show that, for every simplicial set $S$, there is a close relationship between
{\it right fibrations} $S' \rightarrow S$ and {\it simplicial presheaves} $\calF: \sCoNerve[S]^{op} \rightarrow \sSet$. This relationship is controlled by the straightening and unstraightening
functors which we introduce in \S \ref{rightstraight}.
\item[$(2)$] Suppose that $S$ is an $\infty$-category. Then, for each object
$y \in S$, the projection $S_{/y} \rightarrow S$ is a right fibration, which corresponds to a simplicial presheaf $\calF: \sCoNerve[S]^{op} \rightarrow \sSet$. This simplicial presheaf
$\calF$ is related to $S_{/y}$ in two different ways:
\begin{itemize}
\item[$(i)$] As a simplicial presheaf, $\calF$ is weakly equivalent to the functor
$x \mapsto \bHom_{ \sCoNerve[S]}(x,y)$.
\item[$(ii)$] For each object $x$ of $S$, there is a canonical homotopy equivalence
$\calF(x) \rightarrow S_{/y} \times_{S} \{x\} \simeq \Hom_{S}^{\rght}(x,y)$.
Here the Kan complex $\Hom_{S}^{\rght}(x,y)$ is defined as in \S \ref{prereq1}.
\end{itemize}
\item[$(3)$] Combining observations $(i)$ and $(ii)$, we will conclude that
the mapping spaces $\Hom_{S}^{\rght}(x,y)$ are homotopy equivalent to the correpsonding
mapping spaces $\Hom_{\sCoNerve[S]}(x,y)$.
\item[$(4)$] In the special case where $S$ is the nerve of a fibrant simplicial category $\calC$,
there is a canonical map $\Hom_{\calC}(x,y) \rightarrow \Hom_{S}^{\rght}(x,y)$, which we will
show to be a homotopy equivalence in \S \ref{twistt}.
\item[$(5)$] Combining $(3)$ and $(4)$, we will obtain a canonical isomorphism
$\bHom_{\calC}(x,y) \simeq \bHom_{ \sCoNerve[ \sNerve(\calC) ] }(x,y)$ in the homotopy category of spaces. We will then show that this isomorphism is induced by the unit map appearing in the statement of Theorem \ref{biggiesimp}.
\end{itemize}

We will conclude this section with \S \ref{compp3}, where we apply Theorem \ref{biggiesimp} to 
construct the {\it Joyal model structure} on $\sSet$ and to establish a more refined version of the equivalence between $\infty$-categories and simplicial categories. 

%\subsection{Composition Laws on $\infty$-Categories}\label{prereq2}

%In an ordinary category, if $f: X \rightarrow Y$ and $g: Y
%\rightarrow Z$ are two morphisms, then one has a specified
%composition $g \circ f$. In an $\infty$-category, this is not quite
%true: in general there exist many candidates for the composition $g \circ f$ (though all of these candidates are homotopic to one another). To prove Theorem \ref{biggiesimp} for a $\infty$-category $\calC$, it will be convenient for us to {\em choose} a composition for every composable pair of morphisms. In fact, we will need something slightly more general: a chosen composition for every ``composable'' pair of simplices, of arbitrary dimension. 

%\begin{definition}\index{gen}{composition law}
%Let $\calC$ be a simplicial set. A {\it composition law} on $\calC$
%consists of specifying, for every pair of simplices $\sigma \in \calC_m$,
%$\tau \in \calC_n$ with $\sigma | \Delta^{\{ m\} } = \tau | \Delta^{ \{ 0\} }$, a
%composite $\tau \circ \sigma \in \calC_{n+m}$. This composition law is
%required to satisfy the following conditions:
%\begin{itemize}
%\item[$(1)$] The diagram
%$$ \xymatrix{ \Delta^{ \{ 0, \ldots , m\} } \ar[dr]^{\sigma} \ar@{^{(}->}[r] & 
%\Delta^{n+m} \ar[d]^{\tau \circ \sigma} & \Delta^{ \{m, \ldots, m+n\} } \ar@{_{(}->}[l] \ar[dl]_{\tau} \\
%& \calC & }$$
%commutes.

%\item[$(2)$] If $0 \leq i \leq m$, then $(\tau \circ s_i \sigma) = s_i (\tau \circ
%\sigma)$. If $0 \leq i \leq n$, then $(s_i \tau \circ \sigma) = s_{m+i} ( \tau
%\circ \sigma)$.

%\item[$(3)$] If $0 \leq i < m$, then $(\tau \circ d_i \sigma) = d_i ( \tau \circ
%\sigma)$. If $0 < i \leq n$, then $(d_i \tau \circ \sigma) = d_{n+i}(\tau \circ
%\sigma)$.
%\end{itemize}
%\end{definition}

%The existence of a composition law on an $\infty$-category $\calC$ is guaranteed by the following result:

%\begin{proposition}\label{compexist}
%Suppose that $\calC$ is an $\infty$-category. Then $\calC$ admits a
%composition law.
%\end{proposition}

%\begin{proof}
%Let $\sigma \in \calC_m$; we will define the composition law $\tau \mapsto \tau
%\circ \sigma$ by induction on $m$. If $m =0$, we simply set $\tau \circ \sigma
%= \tau$. If $\sigma = s_i \sigma'$ is a degenerate simplex, then we set $\tau
%\circ \sigma = s_i(\tau \circ \sigma')$. If $m > 0$ and $\sigma$ is nondegenerate,
%then we define $\tau \circ \sigma$ for $\tau \in \calC_n$ by induction on $n$. If
%$n=0$, then we set $\tau \circ \sigma = \sigma$. If $\tau = s_i \tau'$ is degenerate,
%then we set $\tau \circ \sigma = s_{i+m} ( \tau' \circ \sigma)$. Finally, in the
%case where $n > 0$ and $\tau$ is also nondegenerate, we note that the
%definition of a composition law prescribes the face $d_i( \tau \circ
%\sigma)$ for $i \neq m$. These faces assemble to determine a map
%$\Lambda^{n+m}_m \rightarrow \calC$. Since $\calC$ is an $\infty$-category and
%$0 < m < n+m$, this map extends to an $(n+m)$-simplex $\tau \circ \sigma
%\in \calC_{n+m}$. One readily checks that this construction has the
%desired properties.
%\end{proof}

%There is generally no canonical choice of composition law on an $\infty$-category $\calC$.
%Moreover, we cannot generally ensure that the composition law given by Proposition \ref{compexist} is associative. However, the compatibility of composition with degeneracy maps
%can be interpreted as asserting that composition is unital.

\subsection{The Straightening and Unstraightening Constructions (Unmarked Case)}\label{rightstraight}

In \S \ref{scgp}, we asserted that a left fibration $X \rightarrow S$ can be viewed as a functor
from $S$ into a suitable $\infty$-category of Kan complexes. Our goal in this section is to make this idea precise. For technical reasons, it will be somewhat more convenient to phrase our results
in terms of the dual theory of {\em right} fibrations $X \rightarrow S$.
Given any functor $\phi: \sCoNerve[S]^{op} \rightarrow \calC$ between simplicial categories,
we will define an {\it unstraightening functor} $\Un_{\phi}: \Set_{\Delta}^{\calC} \rightarrow
(\sSet)_{/S}$. If $\calF: \calC \rightarrow \sSet$ is a diagram taking values in Kan complexes, then
the associated map $\Un_{\phi} \calF \rightarrow S$ is a right fibration, whose fiber at
a point $s \in S$ is homotopy equivalent to the Kan complex $\calF( \phi(s) )$.

Fix a simplicial set $S$, a simplicial category $\calC$ and a functor $\phi: \sCoNerve[S] \rightarrow \calC^{op}$. Given an object $X \in (\sSet)_{/S}$, let $v$ denote the cone point of $X^{\triangleright}$. We can view the simplicial category $$\calM = \sCoNerve[X^{\triangleright}] \coprod_{ \sCoNerve[X]} \calC^{op}$$
as a correspondence from $\calC^{op}$ to $\{v\}$, which we can identify with a simplicial functor
$$ \St_{\phi} X: \calC \rightarrow \sSet.$$\index{gen}{straightening functor}\index{gen}{unstraightening functor}
\index{not}{Stphi@$\St_{\phi}$}\index{not}{Unphi@$\Un_{\phi}$}\index{not}{StS@$\St_{S}$}\index{not}{UnS@$\Un_{S}$}
This functor is described by the formula
$$ (\St_{\phi} X)(C) = \bHom_{\calM}(C,v).$$
We may regard $\St_{\phi}$ as a functor from $(\sSet)_{/S}$ to $(\sSet)^{\calC}$. We refer to
$\St_{\phi}$ as the {\it straightening functor} associated to $\phi$. In the special case where
$\calC = \sCoNerve[S]^{op}$ and $\phi$ is the identity map, we will write
$\St_{S}$ instead of $\St_{\phi}$.

By the adjoint functor theorem (or by direct construction), the straightening functor $\St_{\phi}$
associated to $\phi: \sCoNerve[S] \rightarrow \calC^{op}$ has a right adjoint, which we will denote by $\Un_{\phi}$ and refer to as the {\it unstraightening functor}. We now record the obvious functoriality properties of this construction.

\begin{proposition}\label{straightchange}
\begin{itemize}
\item[$(1)$] Let $p: S' \rightarrow S$ be a map of simplicial sets, $\calC$ a simplicial category, and
$\phi: \sCoNerve[S] \rightarrow \calC^{op}$ a simplicial functor, and let $\phi': \sCoNerve[S'] \rightarrow \calC^{op}$ denote the composition $\phi \circ \sCoNerve[p]$. 
Let $p_{!}: (\sSet)_{/S'} \rightarrow (\sSet)_{/S}$ denote the forgetful functor, given by composition with $p$. There is a natural isomorphism of functors
$$ \St_{\phi} \circ p_{!} \simeq \St_{\phi'}$$
from $(\sSet)_{/S'}$ to $\Set_{\Delta}^{\calC}$.

\item[$(2)$] Let $S$ be a simplicial set, $\pi: \calC \rightarrow \calC'$ a simplicial functor between simplicial categories, and $\phi: \sCoNerve[S] \rightarrow
\calC^{op}$ a simplicial functor. Then there is a natural isomorphism of functors $$\St_{\pi^{op} \circ \phi} \simeq \pi_{!} \circ \St_{\phi}$$
from $(\sSet)_{/S}$ to $\Set_{\Delta}^{\calC'}$. Here $\pi_{!}: \Set_{\Delta}^{\calC} \rightarrow \Set_{\Delta}^{\calC'}$
is the left adjoint to the functor $\pi^{\ast}: \Set_{\Delta}^{\calC'} \rightarrow \Set_{\Delta}^{\calC}$ given by composition with $\pi$.
\end{itemize}
\end{proposition}

%\begin{proposition}\label{spekk3}
%Let $S$ and $S'$ be simplicial sets, $\calC$ and $\calC'$ simplicial categories, and $\phi: \sCoNerve[S] \rightarrow \calC$, $\phi': \sCoNerve[S'] \rightarrow \calC'$ simplicial functors; let $\phi \boxtimes \phi'$ denote the induced functor $\sCoNerve[S \times S'] \rightarrow \calC \times \calC'$. For every $X \in (\sSet)_{/S}$, $X' \in (\sSet)_{/S'}$, the natural map
%$$ s_{X,X'}: \St_{\phi \boxtimes \phi'}(M \times M') \rightarrow \St_{\phi}(X) \boxtimes \St_{\phi'}(X')$$
%is a weak equivalence of functors $\calC \times \calC' \rightarrow \sSet$. 
%\end{proposition}

%\begin{proof}
%Since both sides are compatible with the formations of filtered colimits in $X$, we may suppose that $X$ has only finitely many nondegenerate simplices. 
%We work by induction on the dimension $n$ of $X$ and the number of $n$-dimensional simplices of $X$. If $X = \emptyset$ there is nothing to prove. If $n \neq 1$, we may choose a nondegenerate simplex of $X$ having maximal dimension and thereby write 
%$X = Y \coprod_{ \bd \Delta^n} \Delta^n$. By the inductive hypothesis we may suppose that the result is known
%for $Y$ and $\bd \Delta^n$. The map $s_{X,X'}$ is a pushout of the maps
%$s_{Y,X'}$ and $s_{ \Delta^n, X'}$ over $s_{ \bd \Delta^n, X' }$.
%Since the usual model structure on $\sSet$ is left-proper, this pushout is a homotopy pushout; it therefore suffices to prove the result after replacing $X$ by $Y$, $\bd \Delta^n$, or $\Delta^n$. In the first two cases, the inductive hypothesis implies that $s_{X,X'}$ is an equivalence; we are therefore reduced to the case $X = \Delta^n$. Let $K = \{0\} \subseteq X$. The inclusion
%$K \subseteq X$ is left anodyne, so that inclusion $K \times X' \subseteq X \times X'$ is
%also left anodyne (Corollary \ref{prodprod1}). Every left anodyne morphism is a covariant
%equivalence (Proposition \ref{onehalf}), and the straightening functors carry covariant equivalences to equivalences of simplicial functors. We may therefore reduce to the case $X = K \simeq \Delta^0$.
%In this case, $s_{X,X'}$ is an isomorphism and there is nothing to prove.
%\end{proof}

%\begin{corollary}\label{spekt}
%Let $S$ be a simplicial set, $\calC$ a simplicial category, $\phi: \sCoNerve[S] \rightarrow \calC$ a simplicial functor, and $X \in (\sSet)_{/S}$ an object. For every simplicial set $K$, there is a natural equivalence 
%$$ \St_{\phi}(X \times K) \rightarrow \St_{\phi}(X) \boxtimes K$$
%of simplicial functors from $\calC$ to $\sSet$.
%\end{corollary}

%\begin{proof}
%Combine Proposition \ref{spekk3} with Proposition \ref{babyy}.
%\end{proof}

Our main result is the following:

\begin{theorem}\label{struns}
Let $S$ be a simplicial set, $\calC$ a simplicial category, and 
$\phi: \sCoNerve[S] \rightarrow \calC^{op}$ a simplicial functor. The straightening and unstraightening functors determine a Quillen adjunction
$$ \Adjoint{\St_{\phi}}{ (\sSet)_{/S} }{ \Set_{\Delta}^{\calC} }{ \Un_{\phi} },$$
where $(\sSet)_{/S}$ is endowed with the contravariant model structure and
$\Set_{\Delta}^{\calC}$ with the projective model structure.
If $\phi$ is an equivalence of simplicial categories, then $(\St_{\phi}, \Un_{\phi})$ is a Quillen equivalence.
\end{theorem}

\begin{proof}
It is easy to see that $\St_{\phi}$ preserves cofibrations and weak equivalences, so that
the pair $( \St_{\phi}, \Un_{\phi})$ is a Quillen adjunction. The real content of
Theorem \ref{struns} is the final assertion. Suppose that $\phi$ is an equivalence of simplicial categories; then we wish to show that $( \St_{\phi}, \Un_{\phi})$ is a Quillen equivalence.
We will prove this result in \S \ref{fullun} as a consequence of Proposition \ref{fullmeal}.
\end{proof}

\subsection{Straightening Over a Point}\label{twistt}

In this section, we will study behavior of the straightening functor
$\St_{X}$ in the case where the simplicial set $X = \{x\}$ consists of a single vertex.
In this case, we can view $\St_{X}$ as a colimit-preserving functor
from the category of simplicial sets to itself. We begin with a few general remarks about such functors.

Let $\cDelta$ denote the category of combinatorial simplices and
$\sSet$ the category of simplicial sets, so that $\sSet$ may be
identified with the category of presheaves of sets on $\cDelta$.
If $\calC$ is {\em any} category which admits small colimits, then
any functor $f: \cDelta \rightarrow \calC$ extends to a
colimit-preserving functor $F: \sSet \rightarrow \calC$ (which is unique up to unique isomorphism). We may regard $f$ as a cosimplicial
object $C^{\bigdot}$ of $\calC$. In this case, we shall denote the\index{gen}{geometric realization}
functor $F$ by
$$ S \mapsto |S|_{C^{\bigdot}}.$$\index{not}{|S|_C@$|S|_{C^{\bigdot}}$}

\begin{remark}
Concretely, one constructs $|S|_{C^{\bigdot}}$ by taking the
disjoint union of $S_n \times C^n$ and making the appropriate
identifications along the ``boundaries''. In the language of category theory, the geometric realization is given by
the {\it coend} $$\int_{[n] \in \cDelta} S_n \times C^{n}.$$\index{gen}{coend}
\end{remark}

The functor $S \mapsto |S|_{C^{\bigdot}}$ has a right adjoint
which we shall denote by $\Sing_{C^{\bigdot}}$. It may be described
by the formula
$$ \Sing_{C^{\bigdot}}(X)_n = \Hom_{\calC}( C^n, X).$$\index{not}{Sing_C@$\Sing_{C^{\bigdot}}(X)$}

\begin{example}
Let $\calC$ be the category $\CG$ of compactly generated Hausdorff
spaces, and let $C^{\bigdot}$ be the cosimplicial space defined by
$$ C^n = \{ ( x_0, \ldots, x_n) \in [0,1]^{n+1} : x_0 + \ldots +
x_n = 1 \}.$$ Then $|S|_{C^{\bigdot}}$ is the usual {\it geometric
realization} $|S|$ of the simplicial set $S$ and
$\Sing_{C^{\bigdot}} = \Sing$ is the functor which assigns to each topological space $X$ its its singular complex.
\end{example}

\begin{example}\index{gen}{standard simplex}
Let $\calC$ be the category $\sSet$, and let $C^{\bigdot}$ be the {\it standard simplex} (the cosimplicial object of $\sSet$ given by the Yoneda embedding):
$$ C^n = \Delta^n.$$ Then $||_{C^{\bigdot}}$ and $\Sing_{C^{\bigdot}}$ are
both (isomorphic to) the identity functor on $\sSet$.
\end{example}

\begin{example}
Let $\calC = \Cat$, and let $f: \cDelta \rightarrow \Cat$ be the functor which associates
to each finite nonempty linearly ordered set $J$ the corresponding category.
Then $\Sing_{C^{\bigdot}}= \Nerve$ is the functor which associates to each category its nerve, and $||_{C^{\bigdot}}$ associates, to each simplicial set $S$, the homotopy category $\h{S}$ as defined in \S \ref{hcat}.
\end{example}

\begin{example}
Let $\calC = \sCat$, and let $C^{\bigdot}$ be the cosimplicial
object of $\calC$ given in Definitions \ref{csimp1} and
\ref{csimp2}. Then 
$\Sing_{C^{\bigdot}}$ is the simplicial nerve functor, and
$||_{C^{\bigdot}}$ is its left adjoint
$$ S \mapsto \sCoNerve[S]. $$
\end{example}

Let us now return to the case of the straightening functor
$\St_{X}$, where $X = \{x\}$ consists of a single vertex. The above remarks show that
we can identify $\St_{X}$ with the geometric realization functor
$| |_{Q^{\bigdot}}: \sSet \rightarrow \sSet$, for some cosimplicial object
$Q^{\bigdot}$ in $\sSet$. To describe $Q^{\bigdot}$ more explicitly, let
us first define a cosimplicial simplicial set $J^{\bigdot}$ by the formula
$$ J^{n} = (\Delta^n \star \{y\}) \coprod_{ \Delta^n } \{x\}.$$
The cosimplical simplicial set $Q^{\bigdot}$ can then be described by the formula
$Q^{n} = \bHom_{ \sCoNerve[J^{n}]}(x,y)$. 

In order to proceed with our analysis, we need to understand better the cosimplicial object $Q^{\bigdot}$ of $\sSet$. It admits the following description:\index{not}{Qdot@$Q^{\bigdot}$}

\begin{itemize}
\item For each $n \geq 0$, let $P_{[n]}$ denote the partially ordered set of
{\em nonempty} subsets of $[n]$, and $K_{[n]}$ the simplicial set $\Nerve(P)$ (which may be identified with a simplicial subset of the $(n+1)$-cube $(\Delta^1)^{n+1}$).
The simplicial set $Q^{n}$ is obtained by collapsing, for
each $0 \leq i \leq n$, the subset
$$ ( \Delta^1 )^{ \{ j: 0 \leq j < i \} } \times
\{1\} \times (\Delta^1)^{ \{ j: i < j \leq n \} } \subseteq K_{[n]}$$
to its quotient $( \Delta^1)^{ \{ j: i < j \leq n \} }$. 

\item A map $f: [n] \rightarrow [m]$ determines a map $P_f: P_{[n]}
\rightarrow P_{[m]}$, by setting $P_f(I)=f(I)$. The map $P_f$ in turn
induces a map of simplicial sets $K_{[n]} \rightarrow K_{[m]}$, which
determines a map of quotients $Q^n \rightarrow Q^m$
when $f$ is order-preserving.
\end{itemize}

\begin{remark}\index{not}{Qcaldot@$\calQ^{\bigdot}$}
Let $\calQ^{\bigdot} = |Q^{\bigdot}|$ denote the cosimplicial space obtained by
applying the (usual) geometric realization functor to
$Q^{\bigdot}$. The space $\calQ^n$ may be described as a
quotient of the cube of all functions $p: [n] \rightarrow [0,1]$ satisfying $p(0)=1$. This cube is to be divided by the following equivalence relation: $p \simeq p'$ if there
exists a nonnegative integer $i \leq n$ such that $p| \{i, \ldots
n\} = p'| \{i, \ldots, n\}$ and $p(i) = p'(i) = 1$.

Each $\calQ^n$ is homeomorphic to an $n$-simplex, and these
homeomorphisms may be chosen to be compatible with the face maps
of the cosimplicial space $\calQ^{\bigdot}$. However,
$\calQ^{\bigdot}$ is not isomorphic to the standard simplex
because it has very different degeneracies. For example, the
product of the degeneracy mappings $\calQ^n \rightarrow
(\calQ^1)^n$ is not injective for $n \geq 2$.
\end{remark}

Our goal for the remainder of this section is to study the functors
$\Sing_{Q^{\bigdot}}$ and $||_{Q^{\bigdot}}$ and to prove
that they are ``close'' to the identity functor. More precisely, there is a map $\pi: Q^{\bigdot} \rightarrow \Delta^{\bigdot}$ of
cosimplicial objects of $\sSet$. It is induced by a map $K_{[n]}
\rightarrow \Delta^n$, which the nerve of the map of partially ordered sets
$P_{[n]} \rightarrow [n]$ which carries each nonempty subset of
$[n]$ to its largest element.

\begin{proposition}\label{babyy}
Let $S$ be a simplicial set. Then the map $p_S: |S|_{Q^{\bigdot}}
\rightarrow S$ induced by $\pi$ is a weak homotopy equivalence.
\end{proposition}

\begin{proof}
Consider the collection $A$ of simplicial sets $S$ for which the assertion of Proposition \ref{babyy} holds. 
Since $A$ is stable under filtered colimits, it will suffice to prove that every simplicial set $S$ having only finitely many nondegenerate simplices belongs to $A$. We prove this by induction on the dimension $n$ of $S$, and the number of nondegenerate simplices of $S$ of dimension $n$. If $S = \emptyset$, there is nothing to prove; otherwise we may write
$$S \simeq S' \coprod_{ \bd \Delta^n } \Delta^n $$
$$ |S|_{Q^{\bigdot}} \simeq |S'|_{Q^{\bigdot}} \coprod_{ |\bd \Delta^n|_{Q^{\bigdot}} } |\Delta^n|_{Q^{\bigdot}}.$$
Since both of these pushouts are homotopy pushouts, it suffices to show that $p_{S'}$, $p_{\bd \Delta^n}$, and $p_{\Delta^n}$ are weak homotopy equivalences. For $p_{S'}$ and $p_{\bd \Delta^n}$, this follows from the inductive hypothesis; for $p_{\Delta^n}$, we need only observe that both $\Delta^n$ and $|\Delta^n|_{Q^{\bigdot}} = Q^n$ are weakly contractible.
\end{proof}

\begin{remark}
The strategy used to prove Proposition \ref{babyy} will reappear frequently throughout this book: it allows us to prove theorems about arbitrary simplicial sets by reducing to the case of simplices.
\end{remark}

\begin{proposition}\label{realremmy}
The adjoint functors
$\Adjoint{ | |_{Q^{\bigdot}}}{\sSet}{\sSet}{ \Sing_{Q^{\bigdot} } }$
determine a Quillen equivalence from the category $\sSet$
$($endowed with the Kan model structure$)$ to itself.
\end{proposition}

\begin{proof}
We first show that the functors $( | |_{Q^{\bigdot}}, \Sing_{Q^{\bigdot}} )$ determine a Quillen adjunction from
$\sSet$ to itself. For this, it suffices to prove that the functor
$S \mapsto |S|_{Q^{\bigdot}}$
preserves cofibrations and weak equivalences. The case of cofibrations is easy, and the second case follows from Proposition \ref{babyy}.
To complete the proof, it will suffice to show that the left derived functor
$L | |_{Q^{\bigdot}}$ determines an equivalence from the homotopy category
$\calH$ to itself. This follows immediately from Proposition \ref{babyy}, which implies
that $L | |_{Q^{\bigdot}}$ is equivalent to the identity functor.
\end{proof}

\begin{corollary}\label{remmy33}
Let $X$ be a Kan complex. Then the counit map
$$v: | \Sing_{Q^{\bigdot}} X|_{Q^{\bigdot}} \rightarrow X$$ is a weak
homotopy equivalence.
\end{corollary}

\begin{remark}\label{undef}
Let $S$ be a simplicial set containing a vertex $s$. Let $\calC$ be a simplicial category,
$\phi: \sCoNerve[S]^{op} \rightarrow \calC$ a simplicial functor, and $C = \phi(s) \in \calC$.
For every simplicial functor $\calF: \calC \rightarrow \sSet$, there is a canonical isomorphism
$$ (\Un_{\phi} \calF) \times_{S} \{s\} \simeq \Sing_{Q^{\bigdot}} \calF(C).$$
In particular, we have a canonical map from $\calF(C)$ to the fiber $(\Un_{\phi} \calF)_{s}$, which is a homotopy equivalence if $\calF(C)$ is fibrant.
\end{remark}

\begin{remark}\label{simfun}
Let $\calC$ and $\calC'$ be simplicial categories. Given a pair of simplicial functors
$\calF: \calC \rightarrow \sSet$, $\calF': \calC' \rightarrow \sSet$, we let $\calF \boxtimes \calF': \calC \times \calC' \rightarrow \sSet$ denote the functor described by the formula
$$(\calF \boxtimes \calF')(C,C') = \calF(C) \times \calF'(C').$$
Given a pair of simplicial functors $\phi: \sCoNerve[S]^{op} \rightarrow \calC$, $\phi': \sCoNerve[S']^{op} \rightarrow \calC'$, we let $\phi \boxtimes \phi'$ denote the induced map
$\sCoNerve[S \times S'] \rightarrow \calC \times \calC'$. We observe that there is a canonical isomorphism of functors
$$ \Un_{\phi \boxtimes \phi'}( \calF \boxtimes \calF') \simeq \Un_{\phi}(\calF) \times \Un_{\phi'}(\calF').$$
Restricting our attention to the case where $S' = \Delta^0$ and $\phi'$ is an isomorphism,
we obtain an isomorphism
$$ \Un_{\phi}( \calF \boxtimes K) \simeq \Un_{\phi}(\calF) \times \Sing_{Q^{\bigdot}} K,$$
for every simplicial set $K$. In particular, for every pair of functors
$\calF, \calG \in \Set_{\Delta}^{\calC}$, we have a chain of maps
\begin{eqnarray*}
\Hom_{\sSet}(K, \bHom_{ \Set^{\calC}_{\Delta}}( \calF, \calG) )
& \simeq & \Hom_{ \Set^{\calC}_{\Delta}}( \calF \boxtimes K, \calG ) \\
& \rightarrow & \Hom_{ (\sSet)_{/S} }( \Un_{\phi}( \calF \boxtimes K), \Un_{\phi} \calG ) \\
& \simeq & \Hom_{ (\sSet)_{/S} }( \Un_{\phi}(\calF) \times \Sing_{Q^{\bigdot}} K, \Un_{\phi} \calG) \\
& \rightarrow & \Hom_{ (\sSet)_{/S} }( \Un_{\phi}(\calF) \times K, \Un_{\phi} \calG) \\
& \simeq & \Hom_{\sSet}(K, \bHom_{ (\sSet)_{/S}}( \Un_{\phi}(\calF), \Un_{\phi}(\calG) ).
\end{eqnarray*}
This construction is natural in $K$, and therefore determines a map of simplicial sets
$$ \bHom_{ \Set^{\calC}_{\Delta}}(\calF, \calG) \rightarrow
\bHom_{ (\sSet)_{/S}}( \Un_{\phi} \calF, \Un_{\phi} \calG ).$$
Together, these maps endow the unstraightening functor
$\Un_{\phi}$ with the structure of a {\em simplicial} functor from
$\Set^{\calC}_{\Delta}$ to $(\sSet)_{/S}$. 
\end{remark}

The cosimplicial object $Q^{\bigdot}$ of $\sSet$ will play an important role in our proof
of Theorem \ref{biggie}. To explain this, let us suppose that $\calC$ is a simplicial category and
$S = \Nerve(\calC)$ is its simplicial nerve.
For every pair of vertices $\overline{x}, \overline{y} \in S$, we can consider the right mapping space
$\Hom^{\rght}_{S}(\overline{x},\overline{y})$. By definition, giving an $n$-simplex of $\Hom^{\rght}_{S}(\overline{x},\overline{y})$ is equivalent to giving a map of simplicial sets $J^{n} \rightarrow S$, which carries $x$ to $\overline{x}$ and $y$ to $\overline{y}$. Using the identification
$S \simeq \Nerve(\calC)$, we see that this is equivalent to giving a map
$\sCoNerve[ J^{n} ]$ into $\calC$, which again carries $x$ to $\overline{x}$ and
$y$ to $\overline{y}$. This is simply the data of a map of simplicial sets
$Q^{n} \rightarrow \bHom_{\calC}( \overline{x}, \overline{y} )$. Moreover, this identification
is natural with respect to $[n]$; we therefore have the following result:

\begin{proposition}\label{remmy22}
Let $\calC$ be a simplicial category, and let $X,Y \in \calC$ be
two objects. There is a natural isomorphism of simplicial sets
$\Hom^{\rght}_{\sNerve(\calC)}(X,Y) \simeq
\Sing_{Q^{\bigdot}} \bHom_{\calC}(X,Y)$.
\end{proposition}

\subsection{Straightening of Right Fibrations}\label{fullun}

Our goal in this section is to prove Theorem \ref{struns}, which asserts that
the Quillen adjunction $$\Adjoint{ \St_{\phi} }{ (\sSet)_{/S} }{\Set_{\Delta}^{\calC}}{ \Un_{\phi} }$$
is a Quillen equivalence when $\phi: \sCoNerve[S] \rightarrow \calC^{op}$ is an equivalence
of simplicial categories. We first treat the case where $S$ is a simplex.

\begin{lemma}\label{sticx}
Let $n$ be a nonnegative integer, let $[n]$ denote the linearly ordered set
$\{0, \ldots, n\}$, regarded as a $($discrete$)$ simplicial category, and let
$\phi: \sCoNerve[ \Delta^n ] \rightarrow [n]$ be the canonical functor. Then
the Quillen adjunction
$$\Adjoint{\St_{\phi}}{ (\sSet)_{/ \Delta^n } }{ \sSet^{[n]} }{\Un_{\phi} }$$
is a Quillen equivalence.
\end{lemma}

\begin{proof}
It follows from the definition of the contravariant model structure that the left derived
functor $L \St_{\phi}: \h{(\sSet)_{/ \Delta^n } } \rightarrow \h{ \Set_{\Delta}^{[n] }}$
is conservative. It will therefore suffice to show that the counit map
$L \St_{\phi} \circ R \Un_{\phi} \rightarrow \id$ is an isomorphism of functors
from $\h{ \Set_{\Delta}^{[n]} }$ to itself. For this, we must show that if
$\calF: [n] \rightarrow \sSet$ is projectively fibrant, then the counit map
$$ \St_{\phi} \Un_{\phi} \calF \rightarrow \calF$$
is an equivalence in $\Set_{\Delta}^{[n]}$. In other words, we may assume that
$\calF(i)$ is a Kan complex for $i \in [n]$, and we wish to prove that
each of the induced maps
$$ v_i: (\St_{\phi} \Un_{\phi} \calF)(i) \rightarrow \calF(i)$$
is a weak homotopy equivalence of simplicial sets. 

Let $\psi: [n] \rightarrow [1]$ be defined by the formula 
$$ \psi'(j) = \begin{cases} 0 & \text{if } 0 \leq j \leq i \\
1& \text{otherwise.} \end{cases}$$
Then, for every object $X \in (\sSet)_{/ \Delta^n }$, we have isomorphisms
$$ (\St_{\phi} X)(i) \simeq ( \St_{\psi \circ \phi} X)(0)
\simeq | X \times_{ \Delta^n } \Delta^{ \{n-i, \ldots, n \} } |_{Q^{\bigdot}},$$ where
the twisted geometric realization functor $| |_{Q^{\bigdot}}$ is as defined in \S \ref{twistt}.
Taking $X = \Un_{\phi} \calF$, we see that $v_i$ fits into a commutative diagram
$$ \xymatrix{ | X \times_{\Delta^n} \{n-i\} |_{Q^{\bigdot} } \ar[r]^{\sim} \ar@{^{(}->}[d]
& | \Sing_{Q^{\bigdot} } \calF(i) |_{Q^{\bigdot}} \ar[d] \\
| X \times_{\Delta^n} \Delta^{ \{n-i, \ldots, n\} } |_{Q^{\bigdot}} \ar[r]^{v_i} &
\calF(i). }$$
Here the upper horizontal map is an isomorphism supplied by Corollary \ref{undef}, 
and the right vertical map is a weak homotopy equivalence by Proposition \ref{remmy33}. Consequently, to prove that the map $v_i$ is a weak homotopy equivalence, it will suffice
to show that the left vertical map is a weak homotopy equivalence. In view of Proposition
\ref{babyy}, it will suffice to show that the inclusion
$$ X \times_{ \Delta^n } \{n-i\} \subseteq X \times_{ \Delta^n} \Delta^{ \{n-i, \ldots, n \} }$$
is a weak homotopy equivalence. In fact, $X \times_{ \Delta^n } \{n-i\}$ is a deformation
retract of $X \times_{ \Delta^n} \Delta^{ \{n-i, \ldots, n \} }$: this follows from
the observation that the projection $X \rightarrow \Delta^n$ is a right fibration
(Proposition \ref{onehalf}).
\end{proof}

It will be convenient to restate Lemma \ref{sticx} in a slightly modified form. First, we need to introduce a bit of terminology.

\begin{definition}\index{gen}{pointwise equivalence}\index{gen}{equivalence!pointwise}\label{scatterbrain}
Suppose given a commutative diagram of simplicial sets
$$ \xymatrix{ X \ar[rr]^{f} \ar[dr]^{p} & & Y \ar[dl]^{q} \\
& S, & }$$
where $p$ and $q$ are right fibrations. We will say that $f$ is a
{\it pointwise equivalence} if, for each vertex $s \in S$, the induced map
$X_{s} \rightarrow Y_{s}$ is a homotopy equivalence of Kan complexes.
\end{definition}

\begin{remark}
In the situation of Definition \ref{scatterbrain}, the following conditions are equivalent:
\begin{itemize}
\item[$(a)$] The map $f$ is a pointwise equivalence of right fibrations over $S$.
\item[$(b)$] The map $f$ is a contravariant equivalence in $(\sSet)_{/S}$. 
\item[$(c)$] The map $f$ is a categorical equivalence of simplicial sets.
\end{itemize}
The equivalence $(a) \Leftrightarrow (b)$ follows from Corollary \ref{prefibchar}
(see below), and the equivalence $(a) \Leftrightarrow (c)$ from Proposition \ref{apple1}.
\end{remark}

\begin{lemma}\label{gottapruve2}
Let $S' \subseteq S$ be simplicial sets. Let
$p: X \rightarrow S$ be any map, and let $q: Y \rightarrow S$ be a right fibration.
Let $X' = X \times_{S} S'$ and $Y' = Y \times_{S} S'$. The restriction map
$$ \phi: \bHom_{ (\sSet)_{/S}} (X, Y) \rightarrow \bHom_{ (\sSet)_{/S'}}( X', Y')$$ is a Kan fibration.
\end{lemma}

\begin{proof}
We first show that $\phi$ is a right fibration. It will suffice to show that
$\phi$ has the right lifting property with respect to every right anodyne inclusion
$A \subseteq B$. This follows from the fact that $q$ has the right lifting property with respect
to the induced inclusion
$$i: (B \times S') \coprod_{ A \times S'} ( A \times S) \subseteq B \times S,$$
since $i$ is again right anodyne (Corollary \ref{prodprod1}). 

Applying the preceding argument to the inclusion $\emptyset \subseteq S'$, we deduce
that the projection map $$\bHom_{ (\sSet)_{/S'}}(X',Y') \rightarrow \Delta^0$$ is a right fibration.
Proposition \ref{greenwich} implies that $\bHom_{ (\sSet)_{/S'}}(X',Y')$ is a Kan complex.
Lemma \ref{toothie2} now implies that $\phi$ is a Kan fibration as desired.
\end{proof}

\begin{lemma}\label{blem}
Let $\calU$ be a collection of simplicial sets. Suppose that:
\begin{itemize}
\item[$(i)$] The collection $\calU$ is stable under isomorphism. That is, if
$S \in \calU$ and $S' \simeq S$, then $S' \in \calU$.
\item[$(ii)$] The collection $\calU$ is stable under the formation of disjoint unions.
\item[$(iii)$] Every simplex $\Delta^n$ belongs to $\calU$.
\item[$(iv)$] Given a pushout diagram
$$ \xymatrix{ X \ar[r] \ar[d]^{f} & X' \ar[d] \\
Y \ar[r] & Y' }$$
in which $X$, $X'$, and $Y$ belong to $\calU$. If the map $f$ is a monomorphism, then
$Y'$ belongs to $\calU$.
\item[$(v)$] Suppose given a sequence of monomorphisms of simplicial sets
$$ X(0) \rightarrow X(1) \rightarrow \ldots $$
If each $X(i)$ belongs to $\calU$, then the colimit $\colim X(i)$ belongs to $\calU$.
\end{itemize}
Then every simplicial set belongs to $\calU$.
\end{lemma}

\begin{proof}
Let $S$ be a simplicial set; we wish to show that $S \in \calU$. In view of $(v)$, it will suffice
to show that each skeleton $\sk^{n} S$ belongs to $\calU$. We may therefore assume that
$S$ is finite dimensional. We now proceed by induction on the dimension $n$ of $S$.
Let $A$ denote the set of nondegenerate $n$-simplexes of $S$, so that we have a pushout
diagram
$$ \xymatrix{ \coprod_{\alpha \in A} \bd \Delta^n \ar[r] \ar[d] & \sk^{n-1} S \ar[d] \\
\coprod_{\alpha \in A} \Delta^n \ar[r] & S. }$$
Invoking assumption $(iv)$, we are reduced to proving that
$\sk^{n-1} S$, $\coprod_{ \alpha \in A} \bd \Delta^n$, and $\coprod_{\alpha \in A} \Delta^n$
belong to $\calU$. For the first two this follows from the inductive hypothesis, and for the
last it follows from assumptions $(ii)$ and $(iii)$.
\end{proof}

\begin{lemma}\label{postcuse}
Suppose given a commutative diagram of simplicial sets
$$ \xymatrix{ X \ar[rr]^{f} \ar[dr]^{p} & & Y \ar[dl]^{q} \\
& S, & }$$
where $p$ and $q$ are right fibrations. The following conditions are equivalent:
\begin{itemize}
\item[$(a)$] The map $f$ is a pointwise equivalence.
\item[$(b)$] The map $f$ is an equivalence in the simplicial category $( \sSet)_{/S}$
$($that is, $f$ admits a homotopy inverse{}$)$.
\item[$(c)$] For every object $A \in (\sSet)_{/S}$, composition with $f$ induces a homotopy
equivalence of Kan complexes $\bHom_{ (\sSet)_{/S}}( A, X) \rightarrow
\bHom_{ (\sSet)_{/S}}(A, Y)$.
\end{itemize}
\end{lemma}

\begin{proof}
The implication $(b) \Rightarrow (a)$ is clear (any homotopy inverse to $f$ determines
homotopy inverses for the maps $f_{s}: X_{s} \rightarrow Y_{s}$, for each vertex $s \in S$),
and the implication $(c) \Rightarrow (b)$ follows from Proposition \ref{rooot}.
We will prove that $(a) \Rightarrow (c)$. Let $\calU$ denote the collection of all simplicial
sets $A$ such that, for {\em every} map $A \rightarrow S$, composition with $f$ induces a homotopy equivalence of Kan complexes
$$\bHom_{ (\sSet)_{/S}}( A, X) \rightarrow \bHom_{ (\sSet)_{/S}}(A, Y).$$
We will show that $\calU$ satisfies the hypotheses of Lemma \ref{blem}, and therefore
contains {\em all} simplicial sets. Conditions $(i)$ and $(ii)$ are obvious, and
conditions $(iv)$ and $(v)$ follow from Lemma \ref{gottapruve2}. It will therefore suffice
to show that every simplex $\Delta^n$ belongs to $\calU$. For every
map $\Delta^n \rightarrow S$, we have a commutative diagram
$$ \xymatrix{ \bHom_{ (\sSet)_{/S}}( \Delta^n, X) \ar[r] \ar[d] & \bHom_{ (\sSet)_{/S}}( \Delta^n, Y) \ar[d] \\
\bHom_{ (\sSet)_{/S} }( \{n\}, X) \ar[r] & \bHom_{ ( \sSet)_{/S} }( \{n\}, Y). }$$
Since the inclusion $\{n\} \subseteq \Delta^n$ is right anodyne, the vertical maps are trivial
Kan fibrations. It will therefore suffice to show that the bottom horizontal map is
a homotopy equivalence, which follows immediately from $(a)$.
\end{proof}

\begin{lemma}\label{prestix}
Let $\phi: \sCoNerve[ \Delta^n] \rightarrow [n]$ be as in Lemma \ref{sticx}.
Suppose given a right fibration $X \rightarrow \Delta^n$, a
projectively fibrant diagram $\calF \in \Set_{\Delta}^{[n]}$, and a weak
equivalence of diagrams $\alpha: \St_{\phi} X \rightarrow \calF$. Then the adjoint map
$X \rightarrow \Un_{\phi} \calF$ is a pointwise equivalence of left fibrations
over $\Delta^n$.
\end{lemma}

\begin{proof}
For $0 \leq i \leq n$, let $X(i) = X \times_{\Delta^n} \Delta^{ \{n-i, \ldots, n \} } \subseteq X$.
We observe that $(\St_{\phi} X)(i)$ is canonically isomorphic to the twisted geometric realization
$| X(i) |_{Q^{\bigdot}}$, where $Q^{\bigdot}$ is defined as in \S \ref{twistt}. 
Since $X \rightarrow \Delta^n$ is a right fibration, the fiber
$X \times_{ \Delta^n } \{i\}$ is a deformation retract of $X(i)$. Using Proposition \ref{babyy},
we conclude that the induced inclusion
$| X \times_{ \Delta^n} \{n-i\} |_{Q^{\bigdot}} \rightarrow | X(i) |_{Q^{\bigdot}}$ is a
weak homotopy equivalence. Since $\alpha$ is a weak equivalence,
we get weak equivalences $| X \times_{ \Delta^n} \{n-i\} |_{Q^{\bigdot}} \rightarrow \calF(i)$
for each $0 \leq i \leq n$. Using Proposition \ref{realremmy}, we deduce that
the adjoint maps $X \times_{ \Delta^n } \{n-i\} \rightarrow \Sing_{Q^{\bigdot}} \calF(i)$
are again weak homotopy equivalences. The desired result now follows from
the observation that $\Sing_{Q^{\bigdot}} \calF(i) \simeq (\Un_{\phi} \calF) \times_{ \Delta^n} \{n-i\}$
(Remark \ref{undef}). 
\end{proof}

\begin{notation}\index{not}{RFibS@$\RFib(S)$}
For every simplicial set $S$, we let $\RFib(S)$ denote the full subcategory
of $( \sSet)_{/S}$ spanned by those maps $X \rightarrow S$ which are right fibrations.
\end{notation}

Proposition \ref{onehalf} implies that if $p: X \rightarrow S$ exhibits $X$ as a fibrant
object of the contravariant model category $(\sSet)_{/S}$, then $p$ is a right fibration. We will
prove the converse below (Corollary \ref{usewhere1}). For the moment, we will be content with the following weaker result:

\begin{lemma}\label{camine}
For every integer $n \geq 0$, the inclusion
$i: (\sSet)_{/\Delta^n}^{\degree} \subseteq \RFib( \Delta^n)$ is an equivalence of simplicial
categories.
\end{lemma}

\begin{proof}
It is clear that $i$ is fully faithful. To prove that $i$ is essentially surjective, consider
any left fibration $X \rightarrow \Delta^n$. Let $\phi: \sCoNerve[ \Delta^n ] \rightarrow [n]$
be defined as in Lemma \ref{sticx}, and choose a weak equivalence
$\St_{\phi} X \rightarrow \calF$, where $\calF \in \Set_{\Delta}^{[n]}$ is a projectively fibrant
diagram. Lemma \ref{prestix} implies that the adjoint map
$X \rightarrow \Un_{\phi} \calF$ is a pointwise equivalence of right fibrations in $\Delta^n$, and
therefore a homotopy equivalence in $\RFib( \Delta^n)$ (Lemma \ref{postcuse}). 
It now suffices to observe that $\Un_{\phi} \calF \in ( \sSet)_{/\Delta^n}^{\degree}$.
\end{proof}

\begin{lemma}\label{stucks}
For each integer $n \geq 0$, the unstraightening functor
$\Un_{\Delta^n}: (\Set_{\Delta}^{\sCoNerve[\Delta^n]})^{\degree} \rightarrow \RFib(\Delta^n)$ is
an equivalence of simplicial categories.
\end{lemma}

\begin{proof}
In view of Lemma \ref{camine} and Proposition \ref{weakcompatequiv}, it will suffice to show that
the Quillen adjunction $( \St_{\Delta^n}, \Un_{\Delta^n} )$ is a Quillen equivalence. This follows
immediately from Lemma \ref{sticx} and Proposition \ref{lesstrick}.
\end{proof}

\begin{proposition}\label{fullmeal}
For every simplicial set $S$, the unstraightening functor 
$\Un_{S}$ induces an equivalence of simplicial categories
$(\Set_{\Delta}^{\sCoNerve[S]^{op}})^{\degree} \rightarrow \RFib(S)$.
\end{proposition}

\begin{proof}
For each simplicial set $S$, let $( \Set^{\sCoNerve[S]^{op}}_{\Delta} )_{f}$ denote the
category of projectively fibrant objects of $\Set^{\sCoNerve[S]^{op}}_{\Delta}$, and let
$W_{S}$ be the class of weak equivalences in $( \Set^{\sCoNerve[S]^{op}}_{\Delta})_{f}$.
Let $W'_{S}$ be the collection of pointwise equivalences in $\RFib(S)$.
We have a commutative diagram of simplicial categories
$$ \xymatrix{ ( \Set^{\sCoNerve[S]^{op}}_{\Delta})^{\degree} \ar[r]^{ \Un_{S} } \ar[d] &
\RFib(S) \ar[d]^{\psi_{S}} \\
( \Set^{\sCoNerve[S]^{op}}_{\Delta})_{f}[W_{S}^{-1} ] \ar[r]^{\phi_S} & \RFib[ {W'}^{-1}_{S} ] }$$
(see Notation \ref{localdef}). Lemma \ref{kur} implies that the left vertical map is an equivalence.
Using Lemma \ref{postcuse} and Remark \ref{uppa}, we deduce that the right vertical
map is also an equivalence. It will therefore suffice to show that $\phi_{S}$ is an equivalence. 

Let $\calU$ denote the collection of simplicial sets $S$ for which $\phi_{S}$ is an equivalence.
We will show that $\calU$ satisfies the hypotheses of Lemma \ref{blem}, and therefore contains every simplicial set $S$. Conditions $(i)$ and $(ii)$ are obviously satisfied, and
condition $(iii)$ follows from Lemma \ref{stucks} and Proposition \ref{weakcompatequiv}.
We will verify condition $(iv)$; the proof of $(v)$ is similar.

Applying Corollary \ref{uspin}, we deduce:
\begin{itemize}
\item[$(\ast)$] The functor $S \mapsto ( \Set^{\sCoNerve[S]^{op}}_{\Delta})_{f} [W_S^{-1}]$ carries homotopy colimit diagrams indexed by a partially ordered set to homotopy limit diagrams in $\sCat$.
\end{itemize}

Suppose given a pushout diagram
$$ \xymatrix{ X \ar[r] \ar[d]^{f} & X' \ar[d] \\
Y \ar[r] & Y' }$$
in which $X, X', Y \in \calU$, where $f$ is a cofibration. We wish to prove that $Y' \in \calU$.
We have a commutative diagram
$$ \xymatrix{ ( \Set^{\sCoNerve[Y']^{op}}_{\Delta} )_{f}[W_{Y'}^{-1}] \ar[r]^{\phi_{Y'}} &
\RFib(Y')[ {W'}_{Y'}^{-1}] \ar[r]^{u} \ar[d]^{v} \ar[dr]^{w} & \RFib(Y)[ {W'}_{Y}^{-1}] \ar[d] \\
& \RFib(X')[ {W'}_{X'}^{-1}] \ar[r] & \RFib(X)[ {W'}_{X}^{-1} ]. }$$
Using $(\ast)$ and Corollary \ref{wspin}, we deduce that $\phi_{Y'}$ is an equivalence if and only if,
for every pair of objects $x,y \in \RFib(Y')[ {W'}_{Y'}^{-1}]$, the diagram
of simplicial sets
$$ \xymatrix{ \bHom_{ \RFib(Y')[ {W'}_{Y'}^{-1}] }( x, y) \ar[r] \ar[d] 
& \bHom_{ \RFib(Y)[ {W'}_{Y}^{-1}] }( u(x), u(y) ) \ar[d] \\
\bHom_{ \RFib(X')[ {W'}_{X'}^{-1}] }( v(x), v(y) ) \ar[r] &
\bHom_{ \RFib(X)[ {W'}_{X}^{-1}] }( w(x), w(y) ) }$$ 
is homotopy Cartesian. Since $\psi_{Y'}$ is a weak equivalence of simplicial categories, 
we may assume without loss of generality that $x = \psi_{Y'}( \overline{x} )$ and
$y = \psi_{Y'}( \overline{y} )$, for some $ \overline{x}, \overline{y} \in (\sSet)^{\degree}_{/Y'}$. 
It will therefore suffice to prove that the equivalent diagram
$$ \xymatrix{ \bHom_{ \RFib(Y') }( \overline{x}, \overline{y}) \ar[r] \ar[d] 
& \bHom_{ \RFib(Y) }( \overline{u}(\overline{x}), \overline{u}(\overline{y}) ) \ar[d] \\
\bHom_{ \RFib(X') }( \overline{v}(\overline{x}), \overline{v}(\overline{y}) ) \ar[r]^{g} &
\bHom_{ \RFib(X) }( \overline{w}(\overline{x}), \overline{w}(\overline{y}) ) }$$ 
is homotopy Cartesian. But this diagram is a pullback square, and the map $g$ is a Kan
fibration by Lemma \ref{gottapruve2}.
\end{proof}

We can now complete the proof of Theorem \ref{struns}.
Suppose that $\phi: \sCoNerve[S] \rightarrow \calC^{op}$
is an equivalence of simplicial categories; we wish to show that the adjoint functors
$( \St_{\phi}, \Un_{\phi})$ determine a Quillen equivalence between
$(\sSet)_{/S}$ and $\Set_{\Delta}^{\calC}$. Using Proposition \ref{lesstrick}, we can reduce to
the case where $\phi$ is an isomorphism. In view of Proposition \ref{weakcompatequiv}, it will
suffice to show that $\Un_{\phi}$ induces an equivalence of simplicial categories
$(\Set^{\sCoNerve[S]^{op}}_{\Delta})^{\degree} \rightarrow (\sSet)_{/S}^{\degree}$, which
follows immediately from Proposition \ref{fullmeal}.

\begin{corollary}\label{usewhere1}
Let $p: X \rightarrow S$ be a map of simplicial sets. The following conditions
are equivalent:
\begin{itemize}
\item[$(1)$] The map $p$ is a right fibration.
\item[$(2)$] The map $p$ exhibits $X$ as a fibrant object
of $(\sSet)_{/S}$ $($with respect to the contravariant model structure$)$.
\end{itemize}
\end{corollary}

\begin{proof}
The implication $(2) \Rightarrow (1)$ follows from Proposition \ref{onehalf}.
For the converse, let us suppose that $p$ is a right fibration. 
Proposition \ref{fullmeal} implies that the unstraightening functor
$\Un_{S}: ( \Set_{\Delta}^{\sCoNerve[S]^{op}})^{\degree} \rightarrow \RFun(S)$
is essentially surjective. Since $\Un_{S}$ factors through the inclusion
$i: ( \sSet)_{/S}^{\degree} \subseteq \RFun(S)$, we deduce that $i$ is essentially surjective.
Consequently, we can choose a simplicial homotopy equivalence $f: X \rightarrow Y$ in $(\sSet)_{/S}$, where $Y$ is fibrant. Let $g$ be a homotopy inverse to $X$, so that there exists a homotopy
$h: X \times \Delta^1 \rightarrow X$ from $\id_{X}$ to $g \circ f$.

To prove that $X$ is fibrant, we must show that every lifting problem
$$ \xymatrix{ A \ar@{^{(}->}[d]^{j} \ar[r]^{e_0} & X \ar[d]^{p} \\
B \ar@{-->}[ur]^{e} \ar[r] & S }$$
has a solution, provided that $j$ is a trivial cofibration in the contravariant model category
$(\sSet)_{/S}$. Since $Y$ is fibrant, the map $f \circ e_0$ can be extended to a map
$\overline{e}: B \rightarrow Y$ in $(\sSet)_{/S}$. Let $e' = g \circ \overline{e}$. The maps
$\overline{e}$ and $h \circ (e_0 \times \id_{\Delta^1})$ determine another lifting problem
$$ \xymatrix{ (A \times \Delta^1) \coprod_{ A \times \{1\} } ( B \times \{1\} ) \ar@{^{(}->}[d]^{j'} \ar[r] & X \ar[d]^{p} \\
B \times \Delta^1 \ar[r] \ar@{-->}[ur]^{E} & S. }$$
Proposition \ref{usejoyal} implies that $j'$ is right anodyne. Since $p$ is a right fibration,
there exists an extension $E$ as indicated in the diagram. The restriction
$e = E | B \times \{0\}$ is then a solution the original problem. 
\end{proof}

\begin{corollary}\label{prefibchar}\index{gen}{covariant!equivalence}
Suppose given a diagram of simplicial sets
$$ \xymatrix{ X \ar[dr]^{p} \ar[rr]^{f} & & Y \ar[dl]^{q} \\
& S & }$$ 
where $p$ and $q$ are right fibrations. Then $f$ is a contravariant equivalence
in $(\sSet)_{/S}$ if and only if $f$ is a pointwise equivalence.
\end{corollary}

\begin{proof}
Since $(\sSet)_{/S}$ is a simplicial model category, this follows immediately
from Corollary \ref{usewhere1} and Lemma \ref{postcuse}.
\end{proof}

Corollary \ref{usewhere1} admits the following generalization:

\begin{corollary}\label{usewhere2}
Suppose given a diagram of simplicial sets
$$ \xymatrix{ X \ar[rr]^{f} \ar[dr]^{p} & & Y \ar[dl]^{q} \\
& S, & }$$
where $p$ and $q$ are right fibrations. Then $f$ is a contravariant fibration in
$(\sSet)_{/S}$ if and only if $f$ is a right fibration.
\end{corollary}

\begin{proof}
The map $f$ admits a factorization
$$ X \stackrel{f'}{\rightarrow} X' \stackrel{f''}{\rightarrow} Y$$
where $f'$ is a contravariant equivalence and $f''$ is a contravariant fibration (in $(\sSet)_{/S}$). 
Proposition \ref{onehalf} implies that $f''$ is a right fibration, so the composition
$q \circ f''$ is a right fibration. Invoking Corollary \ref{prefibchar}, we conclude that
for every vertex $s \in S$, the map $f'$ induces a homotopy equivalence of fibers
$X_{s} \rightarrow X'_{s}$. Consider the diagram
$$ \xymatrix{ X_{s} \ar[rr]^{f'_s} \ar[dr] & & X'_{s} \ar[dl] \\
& Y_{s}. & }$$
The vertical maps in this diagram are right fibrations between Kan complexes, and therefore
Kan fibrations (Lemma \ref{toothie2}). Since $f_{s}$ is a homotopy equivalence, we conclude
that the induced map of fibers $f'_{y}: X_{y} \rightarrow X'_{y}$ is a homotopy equivalence
for each vertex $y \in Y$. Invoking Lemma \ref{postcuse}, we deduce that
$f'$ is an equivalence in the simplicial category $(\sSet)_{/Y}$.

We can now repeat the proof of Corollary \ref{usewhere1}. Let $g$ be a homotopy
inverse to $f'$ in the simplicial category $(\sSet)_{/Y}$, and let
$h: X \times \Delta^1 \rightarrow X$ be a homotopy from $\id_{X}$ to $g \circ f'$
(which projects to the identity on $Y$). 
To prove that $f$ is a covariant fibration, we must show that every lifting problem
$$ \xymatrix{ A \ar@{^{(}->}[d]^{j} \ar[r]^{e_0} & X \ar[d]^{f} \\
B \ar@{-->}[ur]^{e} \ar[r] & Y }$$
has a solution, provided that $j$ is a trivial cofibration in the contravariant model category
$(\sSet)_{/S}$. Since $f''$ is a contravariant fibration, the map $f' \circ e_0$ can be extended to a map
$\overline{e}: B \rightarrow X'$ in $(\sSet)_{/Y}$. Let $e' = g \circ \overline{e}$. The maps
$\overline{e}$ and $h \circ (e_0 \times \id_{\Delta^1})$ determine another lifting problem
$$ \xymatrix{ (A \times \Delta^1) \coprod_{ A \times \{1\} } ( B \times \{1\} ) \ar@{^{(}->}[d]^{j'} \ar[r] & X \ar[d]^{f} \\
B \times \Delta^1 \ar[r] \ar@{-->}[ur]^{E} & Y. }$$
Proposition \ref{usejoyal} implies that $j'$ is right anodyne. Since $f$ is a right fibration,
there exists an extension $E$ as indicated in the diagram. The restriction
$e = E | B \times \{0\}$ is then a solution the original problem. 
\end{proof}

We conclude this section with one more result which will be useful
in studying the Joyal model structure on $\sSet$. 
Suppose that $f: X \rightarrow S$ is any map of simplicial sets, and $\{s\}$ is a vertex of
$S$. Let $Q^{\bigdot}$ denote the cosimplicial object of $\sSet$ defined in
\S \ref{twistt}. Then we have a canonical map
$$| X_{s} |_{Q^{\bigdot}} \simeq (\St_{ \{s\} } X_{s})(s)
\rightarrow (\St_{S} X)(s).$$

\begin{proposition}\label{canuble}
Suppose that $f: X \rightarrow S$ is a right fibration of simplicial sets. Then for each vertex
$s$ of $S$, the canonical map
$\phi: | X_{s} |_{ Q^{\bigdot} } \rightarrow (\St_{S} X)(s)$ is a weak homotopy equivalence of simplicial sets.
\end{proposition}

\begin{proof}
Choose a weak equivalence $\St_{S} X \rightarrow \calF$, where
$\calF: \sCoNerve[S]^{op} \rightarrow \sSet$ is a projectively fibrant diagram.
Theorem \ref{struns} implies that the adjoint map
$X \rightarrow \Un_{S}( \calF)$ is a contravariant equivalence in
$(\sSet)_{/S}$. Applying the ``only if'' direction of Corollary \ref{usewhere1}, we conclude
that each of the induced maps
$$X_{s} \rightarrow (\Un_{S} \calF)_{s} \simeq \Sing_{ Q^{\bigdot} } \calF(s)$$ 
is a homotopy equivalence of Kan complexes. Using Proposition \ref{realremmy}, we deduce that the adjoint map $| X_{s} |_{Q^{\bigdot} } \rightarrow \calF(s)$ is a weak homotopy equivalence. 
It follows from the two-out-of-three property that $\phi$ is also a weak homotopy equivalence.
\end{proof}

\subsection{The Comparison Theorem}\label{compp2}

Let $S$ be an $\infty$-category containing a pair of objects $x$ and $y$, and let
$Q^{\bigdot}$ denote the cosimplicial object of $\sSet$ described in \S \ref{twistt}. 
We have a canonical map of simplicial sets
$$ f: | \Hom^{\rght}_{S}(x,y) |_{Q^{\bigdot}} \rightarrow \bHom_{\sCoNerve[S]}(x,y).$$
Moreover, in the special case where $S$ is the nerve of a fibrant simplicial category $\calC$, the composition
$$ | \Hom^{\rght}_{S}(x,y) |_{Q^{\bigdot}} \stackrel{f}{\rightarrow} \bHom_{\sCoNerve[S]}(x,y) \rightarrow \bHom_{\calC}(x,y)$$ can be identified with
the counit map 
$$ | \Sing_{Q^{\bigdot}} \bHom_{\calC}(x,y) |_{Q^{\bigdot}} \rightarrow \bHom_{\calC}(X,Y),$$
and is therefore a weak equivalence (Proposition \ref{remmy33}). 
Consequently, we may reformulate Theorem \ref{biggiesimp} in the following way:

\begin{proposition}\label{wiretrack}
Let $S$ be an $\infty$-category containing a pair of objects $x$ and $y$. Then the natural map
$$ f: | \Hom^{\rght}_{S}(x,y) |_{Q^{\bigdot}} \rightarrow \bHom_{\sCoNerve[S]}(x,y)$$
is a weak homotopy equivalence of simplicial sets. 
\end{proposition}

\begin{proof}
Let $C = S_{/y}^{\triangleright} \coprod_{ S_{/y} } S$, and let $v$ denote the image in $C$ of the cone point of $S_{/y}^{\triangleright}$. There is a canonical projection $\pi: C \rightarrow S$, which induces a map of simplicial sets 
$$f'': (\St_{S} S_{/y})(x) \rightarrow \bHom_{ \sCoNerve[S]}(x,y).$$
The map $f$ can be identified with the composition $f'' \circ f'$, where
$f'$ is the map
$$ | \Hom^{\rght}_{S}(x,y)|_{Q^{\bigdot}} \simeq 
(\St_{ \{x\} } S_{/y} \times_{S} \{x\})(x) \rightarrow (\St_{S} S_{/y})(y).$$
Since the projection $S_{/y} \rightarrow S$ is a right fibration, the map
$f'$ is a weak homotopy equivalence (Proposition \ref{canuble}). It will therefore suffice to show that
$f''$ is a weak homotopy equivalence. To see this, we consider the commutative diagram
$$ \xymatrix{ & (\St_{S} S_{/y})(x) \ar[dr]^{f''} & \\
(\St_{S} \{y\})(x) \ar[ur]^{g} \ar[rr]^{h} & & \bHom_{ \sCoNerve[S]}(x,y). }$$
The inclusion $i: \{ y\} \subseteq S_{/y}$ is a retract of the inclusion
$$ (S_{/y} \times \{1\} ) \coprod_{ \{ y \} \times \{1\} }
( \{ \id_{y} \} \times \Delta^1) \subseteq S_{/y} \times \Delta^1,$$
which is right anodyne by Corollary \ref{prodprod1}. It follows that $i$ is a contravariant equivalence
in $(\sSet)_{/S}$ (Proposition \ref{onehalf}), so the map $g$ is a weak homotopy equivalence of simplicial sets. Since the map $h$ is an isomorphism, the map $f''$ is also a weak homotopy equivalence by virtue of the two-out-of-three property.
\end{proof}

%The proof is based on the following lemma, whose proof we postpone for the moment.

%\begin{lemma}\label{snackwise}
%Let $S$ be a simplicial set containing vertices $x$ and $y$. Then the diagram
%$$ \xymatrix{ | \Hom^{\rght}_{S}(x,y) |_{Q^{\bigdot}} \ar[r]^{f} \ar[d]^{p} & \bHom_{ \sCoNerve[S]}(x,y) \ar[d]^{\sim} \\ 
%\St_{S}( S_{/y} )(x) & (\St_{S} \{y\})(x) \ar[l]^{q} }$$
%commutes up to homotopy, where $f$ is defined as in
%Proposition \ref{wiretrack} and $p$ is defined as in Proposition \ref{canuble}.
%\end{lemma}

%\begin{proof}[Proof of Proposition \ref{wiretrack}]
%By the two-out-of-three property, it will suffice to show that the maps $p$ and
%$q$ appearing in the statement of Lemma \ref{snackwise} are weak homotopy equivalences.
%To show that $p$ is a weak equivalence, it will suffice to show that the projection
%$S_{/y} \rightarrow S$ is a right fibration (Proposition \ref{canuble}). This follows from
%Corollary \ref{gorban4}, since we have assumed that $S$ is an $\infty$-category.
%To prove that $q$ is a weak homotopy equivalence, it will suffice to show that
%the inclusion $i: \{ \id_{y} \} \subseteq S_{/y}$ is a contravariant equivalence
%in $(\sSet)_{/S}$. According to Proposition \ref{onehalf}, it will suffice to show that
%$i$ is right anodyne. But $i$ is a retract of the inclusion
%$$ (S_{/y} \times \{1\} ) \coprod_{ \{ \id_{x} \} \times \{1\} }
%( \{ \id_{y} \} \times \Delta^1) \subseteq S_{/y} \times \Delta^1,$$
%which is left anodyne by Corollary \ref{prodprod1}.
%\end{proof}

%\begin{proof}[Proof of Lemma \ref{snackwise}].
%We will define a map
%$h: \Delta^1 \times | \Hom^{\rght}_{S}(x,y) |_{Q^{\bigdot}} \rightarrow \St_{S}(S_{/y})(x)$
%such that $h | \{0\} \times | \Hom^{\rght}_{S}(x,y)|_{Q^{\bigdot}} = p$ and
%$h | \{1\} \times | \Hom^{\rght}_{S}(x,y)|_{Q^{\bigdot}} = q \circ f$. 

%Let $J^{\bigdot}$ denote the cosimplicial object of $\sSet$ defined by the formula
%$J^{n} = (\Delta^n \star \{ \overline{y} \} ) \coprod_{ \Delta^n } \{ \overline{x} \}$, and
%let $I^{\bigdot}$ be defined by the formula $I^{n} = J^{n} \star \{v\}$. 
%We have a canonical isomorphism of cosimplicial objects of $\sSet$
%$$ \bHom_{ \sCoNerve[I^{\bigdot}]}( \overline{x}, \overline{v} ) \simeq
%\Delta^1 \times Q^{\bigdot}.$$
%Let $Z = S_{/y}^{\triangleright} \coprod_{ S_{/y} } S$, let $v$ denote the cone point
%of $Z$, and set $\calC = \sCoNerve[Z]$. 
%Let $h'$ denote the composite map
%\begin{eqnarray*}
%\Hom^{\rght}_{S}(x,y) & \subseteq & \Sing_{J^{\bigdot}} S \\
%& \subseteq & \Sing_{I^{\bigdot}} S^{\triangleright} \\
%& \rightarrow & \Sing_{I^{\bigdot}} Z \\
%& \rightarrow & \Sing_{ \sCoNerve[ I^{\bigdot} ] } \calC 
%\end{eqnarray*}
%The image of this map is contained in the simplicial subset
%$$ \Sing_{ \Delta^1 \times Q^{\bigdot}} \bHom_{\calC}( x, v)
%\subseteq \Sing_{ \sCoNerve[ I^{\bigdot} ] } \calC.$$
%We now define $h$ to be the composition
%\begin{eqnarray*}
 %\Delta^1 \times | \Hom^{\rght}_{S}(x,y) |_{Q^{\bigdot}} & \simeq &
 %| \Hom^{\rght}_{S}(x,y) |_{\Delta^1 \times Q^{\bigdot}} \\
% & \stackrel{h''}{\rightarrow} & \bHom_{\calC}(x,v) \\
% & \simeq & \St_{S}( S_{/y})(x),
% \end{eqnarray*}
%where $h''$ is adjoint to $h'$. It is easy to check that $h'$ has the desired properties.
%\end{proof}

%The proof of Proposition \ref{wiretrack} will occupy the remainder of this section. Our argument is
%quite technical and can be safely omitted by the reader who does not want to become bogged down in details.

%To show that $f$ is a weak homotopy equivalence, it will be convenient for us to return
%to the topological setting and show instead that
%$|f|: | \Hom^{\rght}_{\calD}(X,Y) |_{\calQ^{\bigdot}} \rightarrow |\bHom_{\sCoNerve[\calD] }(X,Y)|$
%is a homotopy equivalence of topological spaces. To prove this, we will need an explicit description of the topological category $| \sCoNerve[\calD] |$.
%The objects of $| \sCoNerve[\calD] |$ may be identified with the
%vertices of $\calD$. If $\sigma: \Delta^k \rightarrow \calD$ is a simplex of $\calD$, and
%$p: [k] \rightarrow [0,1]$ is a function satisfying $p(0) = p(k) = 1$, then there is a corresponding
%morphism $\sigma[p]: \sigma(0) \rightarrow \sigma(k)$ in $| \sCoNerve[\calD] |$.
%Moreover, $\sigma[p] \in | \bHom_{ \sCoNerve[\calD] }( \sigma(0), \sigma(k)) |$ depends continuously on $p$ (which ranges over a cube of dimension $k-1$).\index{not}{sigma[p]@$\sigma[p]$}

%Every morphism in $| \sCoNerve[\calD] |$ can be written as a
%composition $\sigma_0[ p_0] \circ \ldots \circ \sigma_n[p_n]$. These morphisms
%are subject only to the following relations:

%\begin{itemize}
%\item[$(P1)$] Let $X: \Delta^0 \rightarrow \calD$ be a vertex of $\calD$, and let
%$p: \{0\} \rightarrow [0,1]$ be such that $p(0) = 1$. Then 
%$X[p] = \id_{X} \in | \bHom_{ \sCoNerve[\calD] }(X,X) |.$

%\item[$(P2)$] If $\sigma: \Delta^k \rightarrow \calD$ is a $k$-simplex of $\calD$,
%$0 < i < k$, and $p: [k] \rightarrow [0,1]$ satisfies $p(0) = p(k) = 1$, $p(i) = 0$, then
%$$\sigma[p] = (d_i \sigma)[q] \in | \bHom_{\sCoNerve[\calD] }( \sigma(0), \sigma(k))|,$$
%where $$q(j) = \begin{cases} p(j) & \text{if
%} j < i\\ p(j+1) & \text{if } j \geq i \end{cases}.$$

%\item[$(P3)$] If $\sigma: \Delta^k \rightarrow \calD$ is a $k$-simplex of $\calD$, 
%$0 \leq i \leq k$, and $p: [k+1] \rightarrow [0,1]$ satisfies $p(0) = p(k+1) = 1$, then
%$( s_i \sigma )[p] = \sigma[q]$, where
%$$q(j) = \begin{cases} p(j) & \text{if } j < i
%\\ \sup \{ p(j), p(j+1) \} & \text{if } j=i \\ p(j+1) & \text{if }
%j > i \end{cases}.$$

%\item[$(P4)$] If $\sigma: \Delta^k \rightarrow \calD$ is a $k$-simplex of
%$\calD$, $0 < i < k$, and $p: [d] \rightarrow [0,1]$ satisfies $p(0)=p(d)=p(i)=1$, then
%$$\sigma[p]=\sigma''[p''] \circ \sigma'[p'].$$ Here $\sigma' = \sigma | \Delta^{ \{0, \ldots, i \} }$, 
%$\sigma'' = \sigma | \Delta^{ \{i, \ldots, k \} }$, $p'(j) = p(j)$ for $0 \leq j \leq i$, and
%$p''(j) = p(j+i)$ for $0 \leq j \leq k-i$.
%\end{itemize}

%Using these properties, we deduce that every morphism in
%the topological category $| \sCoNerve[\calD] |$ admits a {\em unique} representation
%as a composition $\sigma_0[p_0] \circ \ldots \circ \sigma_n[p_n]$ having the property that each
%of the simplices $\sigma_i$ is nondegenerate of some positive dimension $k_i$, and
%the functions $p_i| \{1, \ldots, k_i-1 \}$ do not assume the value $0$ or $1$.
%In this representation, composition is simply given by concatenation. For our purposes, this presentation is too unwieldy, since it requires us to consider compositions of arbitrarily long strings of the generating morphisms $\sigma[p]$. In the case where $\calD$ is an $\infty$-category, a much simpler description is available.

%\begin{lemma}\label{tgr}
%Let $\calD$ be an $\infty$-category containing objects $X$ and $Y$. Every morphism $\phi: X \rightarrow Y$ in the topological category $|\sCoNerve[\calD]|$ has the
%form $\sigma[p]$, where $\sigma: \Delta^k \rightarrow \calD$ is a simplex of positive dimension and $p: [k] \rightarrow [0,1]$ satisfies $p(0) = p(k) = 1$.
%\end{lemma}

%\begin{proof}
%Using relation $(P1)$, we may assume that $\phi = \sigma_0[p_0] \circ \ldots \circ \sigma_n[p_n]$
%for $n \geq 0$. Choose a composition law on $\calD$ (Proposition \ref{compexist}).
%Using relation $(P4)$, we deduce that
%$\sigma''[p''] \circ \sigma'[p'] = ( \sigma'' \circ \sigma')[p]$, where
%$p$ is obtained by concatenating $p'$ and $p''$. Applying this relation repeatedly, we deduce that
%$\phi = \sigma[p]$, where 
%$\sigma: \Delta^k \rightarrow \calD$ and $p: [k] \rightarrow [0,1]$ is appropriately chosen.
%Replacing $\sigma$ by $s_0 \sigma$ if necessary (and applying relation $(P3)$), we may ensure
%that $k > 0$.
%\end{proof}

%Let us now fix an $\infty$-category $\calD$ and a pair of objects $X,Y \in \calD$. 
%Our goal is to analyze the topological space $M=| \bHom_{\sCoNerve[\calD] }(X,Y) |$. 
%Lemma \ref{tgr} implies that every point of this space admits a particularly simple representative.
%However, this representation is {\em not} unique. For each positive dimensional
%simplex $\sigma: \Delta^{k} \rightarrow \calD$ with $\sigma(0) = X$, $\sigma(n) = Y$, the
%collection of functions $p: [k] \rightarrow [0,1]$ with $p(0) = p(k) = 1$ is homeomorphic
%to an $(n-1)$-dimensional cube $[0,1]^{k-1}$. We have a continuous map
%$$ [0,1]^{k-1} \rightarrow M$$
%$$ p \mapsto \sigma[p].$$
%Lemma \ref{tgr} implies that $M$ is spanned by the images of these cubes, as $\sigma$ is allowed to vary. We may therefore view $M$ as a quotient of the disjoint union
%$$ \coprod_{ \sigma} [0,1]^{k-1}$$
%where the following relations have been imposed:

%\begin{itemize}
%\item[$(R1)$] Let $\sigma: \Delta^{k} \rightarrow \calD$ satisfy
%$\sigma(0) = X$, $\sigma(k) = Y$, let $0 < i < k$, and let $p: [k] \rightarrow [0,1]$ satisfy
%$p(0) = p(k) = 1$, $p(i) = 0$. Then $\sigma[p] = (d_i \sigma)[q]$, where
%$$q(j) = \begin{cases} p(j) & \text{if
%} j < i\\ p(j+1) & \text{if } j \geq i \end{cases}.$$

%\item[$(R2)$] Let $\sigma, \sigma': \Delta^k \rightarrow \calD$ be simplices
%with $\sigma(0) = \sigma'(0) = X$, $\sigma(k) = \sigma'(k) = Y$, let
%$0 = i_0 < i_1 < \ldots < i_n = k$, and let $p: [k] \rightarrow [0,1]$ satisfy
%$p(i_0) = \ldots = p(i_n) = 1$. If 
%$$\sigma | \Delta^{ \{i_j, \ldots, i_{j+1} \} }= \sigma' | \Delta^{
%\{ i_j, \ldots, i_{j+1} \}}$$ for $0 \leq j < n$, then $\sigma[p] =
%\sigma'[p]$.

%\item[$(R3)$] Let $\sigma: \Delta^k \rightarrow \calD$ satisfy
%$\sigma(0) = X$ and $\sigma(k) = Y$, let $0 \leq i \leq k$, and let
%$p: [k+1] \rightarrow [0,1]$ satisfy $p(0) = p(k+1) = 1$. Then
%$( s_i \sigma )[p] = \sigma[q]$, where $q: [d] \rightarrow [0,1]$ is
%given by 
%$$q(j) = \begin{cases} p(j) & \text{if } j < i
%\\ \sup \{ p(j), p(j+1) \} & \text{if } j=i \\ p(j+1) & \text{if }
%j > i \end{cases}.$$
%\end{itemize}

%Using relation $(R3)$, we see that $M$ is actually spanned by the images of the maps
%$p \mapsto \sigma[p]$, where $\sigma$ is a {\em nondegenerate} simplex of $\calD$
%(or, in the case $X = Y$, $\sigma = s_0(X) \in \calD_1)$. However,
%relation $(R3)$ will be very inconvenient for us in the
%arguments which follow. We therefore define $\widetilde{M}$ to be
%the quotient of the topological space $\coprod_{ \sigma} [0,1]^{k-1}$
%where the coproduct is taken over simplices $\sigma$ of {\em positive} dimension, and only the relations $(R1)$ and $(R2)$ have been imposed. 
%If $\sigma: \Delta^k \rightarrow \calD$ satisfies $\sigma(0) = X$, $\sigma(k) = Y$, and
%$p: [k] \rightarrow [0,1]$ satisfies $p(0) = p(k) = 1$, then we let $\widetilde{\sigma}[p]$ denote the corresponding point of $\widetilde{M}$. There is an obvious quotient map
%$\pi: \widetilde{M} \rightarrow M$, having the property that $\pi( \widetilde{\sigma}[p] ) = \sigma[p]$.

%\begin{lemma}\label{oof}
%The map $\pi: \widetilde{M} \rightarrow M$ is a
%homotopy equivalence.
%\end{lemma}

%\begin{proof}
%We first observe that the map $\pi$ may be obtained as the geometric
%realization of a map of simplicial sets. Consequently, it will
%suffice to show that the fibers of $\pi$ are contractible. Choose a
%point $\sigma[p] \in M$. We will assume that $\sigma: \Delta^k \rightarrow \calD$ is
%nondegenerate simplex of positive dimension, and that $p: [k] \rightarrow [0,1]$ is strictly
%positive. Such a representation always exists, unless $X = Y$ and $\sigma[p] = \id_{X}$; the proof in this exceptional case can be given using a slightly easier version of the argument below, and is left to the reader.

%Every point of $\widetilde{M}$ which lies in $\pi^{-1} \{ \sigma[p] \}$ can
%be written in the form $\widetilde{\sigma'}[p']$, where $\sigma': \Delta^m \rightarrow \calD$ 
%is the pullback of $\sigma$ under a surjective map
%$f: [m] \rightarrow [k]$, and $p(i) = \sup_{f(j) = i}
%p'(j)$. It follows that $\pi^{-1} \{ \sigma[p] \}$ is homomorphic to the product
%$$ F'_{1} \times \prod_{ 0 < i < n} F_{ p(i) } \times F''_{1},$$
%where:
%\begin{itemize}
%\item[$(1)$] For $t \in (0, 1]$, the space $F_{t}$ is described as follows: 
%points of $F_{t}$ may be represented by finite sequences of real numbers
%$( r_1, \ldots, r_n)$ lying in the interval $[0,t]$, which assume the value $t$ at least
%once. Two such sequences are identified if they differ from one
%another by inserting or deleting zeroes. 
%\item[$(2)$] The space $F'_{t}$ is defined in the same way as $F_{t}$, except that we consider only sequences which begin with $r_1 = t$.
%\item[$(3)$] The space $F''_{t}$ is defined in the same way as $F_{t}$, except that we consider only sequences which end with $r_n=t$.
%\end{itemize}

%We now observe that each of the spaces $F'_{t}$ is contractible: in fact, we have a homotopy
%$$ h': [0,1] \times F'_{t} \rightarrow F'_{t}$$
%from a constant map to the identity, given by
%$$ h'_{s}( r_1, \ldots, r_n ) = ( r_1, s r_2, \ldots, s r_n ).$$
%A similar argument shows that each $F''_{t}$ is contractible. Finally, we show that
%$F_{t}$ is contractible by showing that the identity map $F_{t} \rightarrow F_{t}$ is homotopic
%to a map which factors through the contractible subspace $F'_{t} \subseteq F_{t}$. To see this, we consider the homotopy
%$$ h: [0,1] \times F_{t} \rightarrow F_{t}$$
%given by
%$$ h_{s}( r_1, \ldots, r_n) = ( st, r_1, \ldots, r_n).$$
%\end{proof}

%Our next step is to perform a similar analysis of the topological space
%$N =  | \Hom^{\rght}_{\calD}(X,Y) |_{\calQ^{\bigdot}}$. For each
%$\sigma: \Delta^{k+1} \rightarrow \calD$ such that
%$\sigma(k+1) = Y$ and $\sigma| \Delta^k$ is the constant map with value $X$, and for
%each $p: [k+1] \rightarrow [0,1]$ satisfies $p(0) = p(k+1) = 1$, there
%is a corresponding point $\sigma[[p]] \in N$. Moreover, 
%$N$ is obtained by gluing together these cubes using the following relations:\index{not}{sigma[[p]]@$\sigma[[p]]$}

%\begin{itemize}
%\item[$(R1')$] Let $\sigma: \Delta^{k+1} \rightarrow \calD$ be such that
%$\sigma(k+1) = Y$ and $\sigma | \Delta^k$ is constant at $X$, let
%$1 \leq i \leq k$, and let $p: [k+1] \rightarrow [0,1]$ be such that
%$p(0) = p(k+1) = 1$, $p(i) = 0$. Then
%$\sigma[[p]] = (d_i \sigma)[[q]] \in N$, where
% $$q(j) = \begin{cases} p(j) & \text{if } j < i\\
%p(j+1) & \text{if } j \geq i \end{cases}.$$

%\item[$(R2')$] Let $\sigma: \Delta^{k+1} \rightarrow \calD$ be such that
%$\sigma(k+1) = Y$ and $\sigma | \Delta^k$ is constant at $X$, let
%$0 \leq i \leq k$, and let $p,p': [k+1] \rightarrow [0,1]$ be such that
%$p(0) = p(i) = p(k+1) = 1$, $p'(0) = p'(i) = p'(k+1) = 1$. If
%$p(j) = p'(j)$ for all $i \leq j \leq k+1$, then $\sigma[[p]] = \sigma[[p']]$.

%\item[$(R3')$] Let $\sigma: \Delta^{k+1} \rightarrow \calD$ be such that
%$\sigma(k+1) = Y$ and $\sigma | \Delta^k$ is constant at $X$, let
%$0 \leq i \leq k$, and let $p: [k+2] \rightarrow [0,1]$ be such that
%$p(0) = p(k+2) = 1$. Then $(s_i \sigma)[[p]] = \sigma[[q]]$, where
%$$q(j) = \begin{cases} p(j) & \text{if } j < i
%\\ \sup \{ p(i), p(i+1) \} & \text{if } j=i \\ p(j+1) & \text{if }
%j > i \end{cases}.$$
%\end{itemize}

%Our objective is to prove that the map
%$|f|: N \rightarrow M$ is a homotopy equivalence of topological spaces.
%In terms of the above presentations, the map $|f|$ is given by
%$$ \sigma[[p]] \mapsto \sigma[p].$$
%Once again, the relation $(R3')$ is actually somewhat inconvenient for us. We therefore define a $\widetilde{N}$ to be the topological space obtained using the above presentation, but omitting the relations of the form $(R3')$. By definition, there is a canonical projection map
%$\widetilde{N} \rightarrow N$. Repeating the argument of Lemma \ref{oof}, we deduce the following:

%\begin{lemma}\label{oof2}
%The projection $\widetilde{N} \rightarrow N$ is a homotopy equivalence of topological spaces.
%\end{lemma}

%We also observe that $\widetilde{N}$ can be identified with a closed subspace
%of $\widetilde{M}$ (even when $X =Y$).
%We now have a commutative diagram
%$$ \xymatrix{ \widetilde{N} \ar@{^{(}->}[r] \ar[d] & \widetilde{M} \ar[d] \\
%N \ar[r]^{|f|} & M }$$
%of topological spaces, in which the vertical arrows are weak homotopy equivalences.
%In view of Lemmas  \ref{oof} and \ref{oof2}, we are reduced to proving the following assertion:

%\begin{proposition}
%Let $\calD$ be an $\infty$-category containing a pair of objects $X$ and $Y$, and let
%$\widetilde{N} \subseteq \widetilde{M}$ be defined as above. Then $\widetilde{N}$ is a deformation retract of $\widetilde{M}$.
%\end{proposition}

%\begin{proof}
%Choose a composition law on $\calD$ (Proposition \ref{compexist}). 
%Let $0 < i \leq n$, and let $\sigma: \Delta^n \rightarrow \calD$ be a simplex such that
%$\sigma(n) = Y$, and $\sigma | \Delta^{ \{0, \ldots, i-1 \} }$ is constant at the vertex $X$. 
%We will define a new simplex $A_i(\sigma) \in \calD_{n+1}$. Our
%construction will possess the following properties:

%\begin{itemize}
%\item[(i)] If $i = 1$, then $d_0 A_i(\sigma) = \sigma| \Delta^{ \{1, \ldots, n\} }
%\circ \sigma | \Delta^{ \{0,1\} }$.

%\item[(ii)] If $i = n$, then $A_i(\sigma) = s_{n-1} \sigma$.

%\item[(iii)] If $\sigma| \Delta^{ \{0, \ldots, i\} }$ is constant at the
%vertex $x$, then $A_i(\sigma) = s_i \sigma$.

%\item[(iv)] If $i > 1$ and $j < i$, then $d_j A_i(\sigma) = A_{i-1} ( d_j
%\sigma)$.

%\item[(v)] If $j=i$, then $d_{j} A_i(\sigma) = \sigma$.

%\item[(vi)] If $i+1 < j \leq n+1$, $d_{j} A_i(\sigma ) = A_i( d_{j-1} \sigma)$.
%\end{itemize}

%The construction proceeds by induction on $i \geq 1$. For a fixed
%value of $i$, we work by induction on $n \geq i$. When $n=i$, we
%set $A_i(\sigma) = s_{n-1} \sigma$ in accordance with condition $(ii)$. For
%$n > i$, we note that the desired properties uniquely prescribe
%the restriction of $A_i(\sigma)$ to the horn $\Lambda^{n+1}_{i+1}$.
%Since $\calD$ is an $\infty$-category, this horn may be filled to an
%$(n+1)$-simplex which permits us to define $A_i(\sigma)$. Finally, we
%note that if $\sigma | \Delta^{ \{0, \ldots, i\} }$ is constant at the
%vertex $X$, then we may choose this filler to be $s_i \sigma$ and
%thereby satisfy condition $(iii)$.

%Now choose an arbitrary $\sigma: \Delta^n \rightarrow \calD$
%with $\sigma(0) = X$, $\sigma(n) = Y$. We will define a 
%sequence of simplices $( \sigma_0, \ldots, \sigma_{n-1})$ in
%$\calD_{n}$ and another sequence $(\tau_1, \ldots, \tau_{n-1}) \in
%\calD_{n+1}$. Let $\sigma_0 = \sigma$. For $i > 0$, we set
%$\tau_i = A_i( \sigma_{i-1})$ and $\sigma_i = d_{i+1} \tau_{i}$. 
%Let $p: [n] \rightarrow [0,1]$ satisfy $p(0) = p(n) = 1$. We
%will constructing a path in $\widetilde{M}$ of length $\sum_{0 < i < n}
%2p(i)$ which begins at the point $\widetilde{\sigma}[p]$ and ends at the point
%$\widetilde{\sigma_{n-1}}[p]$. This path is obtained by concatenating a sequence
%of paths $h_i: [0, 2p(i)]$ which join $\widetilde{\sigma}_{i-1}[p]$ to 
%$\widetilde{\sigma}_{i}[p]$.
%More specifically, we set $ h_i(t) = \widetilde{ \tau}_i[q_t]$, where
%$$q_t(j) = \begin{cases} p(j) & \text{if } j < i \text{ or } j > i+1\\
%t & \text{if } j=i \text{ and } 0 \leq t \leq p(i) \\
%p(i) & \text{if } j=i \text{ and } p(i) \leq t \leq 2p(i) \\
%p(i) & \text{if } j=i+1 \text{ and } 0 \leq t \leq p(i) \\
%2p(i) - t & \text{if } j = i+1 \text{ and } p(i) \leq t \leq 2p(i)
%\end{cases}.$$

%One readily checks that this path depends continuously on $p$ and
%is independent of the representation of the point $\widetilde{\sigma}[p] \in
%\widetilde{M}$. Consequently, we obtain a homotopy from the
%identity map of $\widetilde{M}$ to itself to a map $\widetilde{M}
%\rightarrow \widetilde{N}$. Moreover, condition $(iii)$ ensures that this homotopy leaves
%$\widetilde{N}$ {\em setwise} fixed. It follows that
%$\widetilde{N}$ is a deformation retract of $\widetilde{M}$, as desired.
%\end{proof}

\subsection{The Joyal Model Structure}\label{compp3}

The category of simplicial sets can be endowed with a model structure for which the fibrant objects are precisely the $\infty$-categories. The original construction of this model structure is due to Joyal, who uses purely combinatorial arguments (\cite{joyalnotpub}). In this section, 
we will exploit the relationship between simplicial categories and
$\infty$-categories to give an alternative description of this model structure. 
Our discussion will make use of a model structure on the category $\sCat$ of simplicial categories, which we review in \S \ref{compp4}.

\begin{theorem}\label{biggier}\index{gen}{model category!Joyal}\index{gen}{simplicial set!Joyal model structure}
There exists a left proper, combinatorial model structure on
the category of simplicial sets with the following properties:
\begin{itemize}
\item[$(C)$] A map $p: S \rightarrow S'$ of simplicial sets is a {\it
cofibration} if and only if it is a monomorphism.

\item[$(W)$] A map $p: S \rightarrow S'$ is a {\it categorical
equivalence} if and only if the induced simplicial functor
$\sCoNerve[S] \rightarrow \sCoNerve[S']$ is an equivalence of simplicial categories.
\end{itemize}

Moreover, the adjoint functors $(\sCoNerve, \sNerve)$
determine a Quillen equivalence between $\sSet$ $($with the model structure defined above$)$ and $\sCat$.
\end{theorem}

Our proof will make use of the theory of {\em inner anodyne} maps of simplicial sets,
which we will study in detail in \S \ref{midfibsec}. We first establish a simple Lemma.

\begin{lemma}\label{rugg}
Every inner anodyne map $f: A \rightarrow B$ of simplicial sets is a categorical equivalence.
\end{lemma}

\begin{proof}
It will suffice to prove that if $f$ is inner anodyne, then the associated map
$\sCoNerve[f]$ is a trivial cofibration of simplicial categories. The collection of all
morphisms $f$ for which this statement holds is weakly saturated (Definition \ref{saturated}). Consequently, we may assume that $f$ is an inner horn inclusion $\Lambda^n_i \subseteq \Delta^n$, $0 < i < n$.
We now explicitly describe the map $\sCoNerve[f]$:

\begin{itemize}
\item The objects of $\sCoNerve[\bd \Lambda^n_i]$ are the objects of
$\sCoNerve[\Delta^n]$: namely, elements of the linearly ordered
set $[n] = \{0, \ldots, n\}$. 

\item For $0 \leq j \leq k \leq n$, the simplicial set
$\bHom_{\sCoNerve[\Lambda^n_i]}(j,k)$ is equal to
$\bHom_{\sCoNerve[\Delta^n]}(j,k)$ unless $j=0$ and $k=n$. In the
latter case, $$\bHom_{\sCoNerve[\Lambda^n_i]}(j,k) = K \subseteq
(\Delta^1)^{n-1} \simeq \bHom_{\sCoNerve[\Delta^n]}(j,k),$$
where $K$ is the simplicial subset of the cube $(\Delta^1)^{n-1}$ obtained
by removing the interior and a single face.
\end{itemize}

We observe that $\sCoNerve[f]$ is a pushout of the inclusion $\calE_{K} \subseteq
\calE_{ (\Delta^1)^{n-1} }$ (see \S \ref{compp4} for an explanation of this notation).
It now suffices to observe that the inclusion $K \subseteq (\Delta^1)^{n-1}$ is trivial fibration of simplicial sets (with respect to the usual model structure on $\sSet$).
\end{proof}

\begin{proof}[Proof of Theorem \ref{biggier}]
We first show that $\sCoNerve$ carries cofibrations of simplicial sets to cofibrations of simplicial categories. Since the class of all cofibrations of simplicial
sets is generated by the inclusions $\bd \Delta^n \subseteq
\Delta^n$, it suffices to show that each map $\sCoNerve[\bd
\Delta^n] \rightarrow \sCoNerve[\Delta^n]$ is a cofibration of
simplicial categories. If $n = 0$, then the inclusion $\sCoNerve[\bd \Delta^n] \subseteq
\sCoNerve[\Delta^n]$ is isomorphic to the inclusion $\emptyset
\subseteq \ast$ of simplicial categories, which is a
cofibration. In the case where $n
> 0$, we make use of the following explicit description of
$\sCoNerve[\bd \Delta^n]$ as a subcategory of
$\sCoNerve[\Delta^n]$:

\begin{itemize}
\item The objects of $\sCoNerve[\bd \Delta^n]$ are the objects of
$\sCoNerve[\Delta^n]$: namely, elements of the linearly ordered
set $[n] = \{0, \ldots, n\}$. 

\item For $0 \leq j \leq k \leq n$, the simplicial set
$\Hom_{\sCoNerve[\bd \Delta^n]}(j,k)$ is equal to
$\Hom_{\sCoNerve[\Delta^n]}(j,k)$ unless $j=0$ and $k=n$. In the
latter case, $\Hom_{\sCoNerve[\bd \Delta^n]}(j,k)$ consists of the
boundary of the cube 
$$(\Delta^1)^{n-1} \simeq
\Hom_{\sCoNerve[\Delta^n]}(j,k).$$
\end{itemize}

In particular, the inclusion $\sCoNerve[\bd \Delta^n] \subseteq
\sCoNerve[\Delta^n]$ is a pushout of the inclusion $\calE_{ \bd
(\Delta^1)^{n-1} } \subseteq \calE_{ (\Delta^1)^{n-1} }$, which is
a cofibration of simplicial categories (see \S \ref{compp4} for an explanation of our notation).

We now declare that a map $p: S \rightarrow S'$ of simplicial sets is a {\it categorical
fibration}\index{gen}{categorical fibration}\index{gen}{fibration!categorical} if it has the right lifting property with respect to
all maps which are cofibrations and categorical equivalences. We now claim that the cofibrations, categorical equivalences, and categorical fibrations determine a left proper, combinatorial model structure on $\sSet$. To prove this, it will suffice to show that the hypotheses of Proposition \ref{goot} are satisfied:

\begin{itemize}
\item[$(1)$] The class of categorical equivalences in $\sSet$ is perfect. This follows from Corollary \ref{perfpull}, since the functor $\sCoNerve$ preserves filtered colimits, and the class of equivalences between simplicial categories is perfect. 
\item[$(2)$] The class of categorical equivalences is stable under pushouts by cofibrations.
Since $\sCoNerve$ preserves cofibrations, this follows immediately from the left-properness of $\sCat$.
\item[$(3)$] A map of simplicial sets which has the right lifting property with respect to {\em all} cofibrations is a categorical equivalence. In other words, we must show that if $p: S \rightarrow S'$ is a trivial fibration of simplicial sets, then
the induced functor $\sCoNerve[p]: \sCoNerve[S] \rightarrow
\sCoNerve[S']$ is an equivalence of simplicial categories.

Since $p$ is a trivial fibration, it admits a section $s: S'
\rightarrow S$. It is clear that $\sCoNerve[p] \circ \sCoNerve[s]$
is the identity; it therefore suffices to show that $$\sCoNerve[s]
\circ \sCoNerve[p]: \sCoNerve[S] \rightarrow \sCoNerve[S]$$ is
homotopic to the identity.

Let $K$ denote the simplicial set $\bHom_{S'}(S,S)$. Then $K$ is a
contractible Kan complex, containing points $x$ and $y$ which
classify $s \circ p$ and $\id_S$. We note the existence of a
natural ``evaluation map'' $e: K \times S \rightarrow S$, such
that $s \circ p = e \circ (\{x\} \times \id_S)$, $\id_S = e \circ
( \{y\} \times \id_S )$. It therefore suffices to show that the
functor $\sCoNerve$ carries $\{x\} \times \id_{S}$ and $\{y\}
\times \id_{S}$ into homotopic morphisms. Since both of these maps
section the projection $K \times S \rightarrow S$, it suffices to
show that the projection $\sCoNerve[K \times S] \rightarrow
\sCoNerve[S]$ is an equivalence of simplicial categories. Replacing $S$ by $S \times K$
and $S'$ by $S$, we are reduced to the
special case where $S = S' \times K$ and $K$ is a contractible
Kan complex.

By the small object argument, we can find an inner anodyne map $S' \rightarrow V$, where $V$ is an $\infty$-category. The corresponding map $S' \times K \rightarrow V
\times K$ is also inner anodyne (Proposition \ref{usejoyal2}), so the maps
$\sCoNerve[S'] \rightarrow \sCoNerve[V]$ and $\sCoNerve[S' \times
K] \rightarrow \sCoNerve[V \times K]$ are both trivial
cofibrations (Lemma \ref{rugg}). It follows that we are free to replace $S'$ by $V$
and $S$ by $V \times K$. In other words, we may suppose that $S'$
is an $\infty$-category (and now we will have no further need of the
assumption that $S$ is isomorphic to the product $S' \times K$).

Since $p$ is surjective on vertices, it is clear that
$\sCoNerve[p]$ is essentially surjective. It therefore suffices to
show that for every pair of vertices $x,y \in S_0$, the induced map of simplicial
sets $\bHom_{\sCoNerve[S]}(x,y) \rightarrow
\bHom_{\sCoNerve[S']}(p(x),p(y))$ is a weak homotopy equivalence. Using Propositions
\ref{wiretrack} and \ref{babyy}, it suffices to show that the map
$\Hom^{\rght}_{S}(x,y) \rightarrow \Hom^{\rght}_{S'}(p(x),p(y))$ is a weak homotopy equivalence. 
This map is obviously a trivial fibration if $p$ is a trivial fibration. 
\end{itemize}

By construction, the functor $\sCoNerve$ preserves weak equivalences. We verified above that $\sCoNerve$ preserves cofibrations as well. It follows that the adjoint functors $(\sCoNerve, \Nerve)$ determine a Quillen adjunction
$$ \Adjoint{\sCoNerve}{\sSet}{\sCat}{\Nerve}.$$
To complete the proof, we  wish to show that this Quillen adjunction is a Quillen equivalence. According to Proposition \ref{quilleq}, we must show that for every simplicial set $S$ and every {\em fibrant} simplicial category $\calC$, a map
$$ u: S \rightarrow \Nerve(\calC)$$
is a categorical equivalence if and only if the adjoint map
$$ v: \sCoNerve[S] \rightarrow \calC$$
is an equivalence of simplicial categories. We observe that $v$ factors as a composition
$$ \sCoNerve[S] \stackrel{ \sCoNerve[u] }{\rightarrow} \sCoNerve[ \Nerve(\calC)]
\stackrel{w}{\rightarrow} \calC.$$
By definition, $u$ is a categorical equivalence if and only if $\sCoNerve[u]$ is an
equivalence of simplicial categories. We now conclude by observing that the counit
map $w$ is an equivalence of simplicial categories (Theorem \ref{biggiesimp}).
\end{proof}

We now establish a few pleasant properties enjoyed by the Joyal model structure on $\sSet$.
We first note that every object of $\sSet$ is cofibrant; in particular, the Joyal model structure is {\em left proper} (Proposition \ref{propob}).

\begin{remark}\label{rightprop}
The Joyal model structure is {\em not} right proper. To see this,
we note that the inclusion $\Lambda^2_1 \subseteq \Delta^2$ is a
categorical equivalence, but it does not remain so after pulling
back via the fibration $\Delta^{ \{0,2\} } \subseteq \Delta^2$.
\end{remark}

\begin{corollary}\label{equivstable}\index{gen}{categorical equivalence!and products}
Let $f: A \rightarrow B$ be a categorical equivalence of simplicial sets, and $K$ an arbitrary simplicial set. Then the induced map $A \times K \rightarrow B \times K$ is a categorical equivalence.
\end{corollary}

\begin{proof}
Choose an inner anodyne map $B \rightarrow Q$, where $Q$ is an $\infty$-category. Then $B \times K \rightarrow Q \times K$ is also inner anodyne, hence a categorical equivalence (Lemma \ref{rugg}). 
It therefore suffices to prove that $A \times K \rightarrow Q \times K$ is a categorical equivalence. In other words, we may suppose to begin with that $B$ is an $\infty$-category.

Now choose a factorization $A \stackrel{f'}{\rightarrow} R \stackrel{f''}{\rightarrow} B$ where $f'$ is an inner anodyne map and $f''$ is an inner fibration. Since $B$ is an $\infty$-category, $R$ is an $\infty$-category.
The map $A \times K \rightarrow R \times K$ is inner anodyne (since $f'$ is), and therefore a categorical equivalence; consequently, it suffices to show that $R \times K \rightarrow B \times K$ is a categorical equivalence. In other words, we may reduce to the case where $A$ is also an $\infty$-category.

Choose an inner anodyne map $K \rightarrow S$, where $S$ is an $\infty$-category. Then $A \times K \rightarrow A \times S$ and $B \times K \rightarrow B \times S$ are both inner anodyne, and therefore categorical equivalences. Thus, to prove that $A \times K \rightarrow B \times K$ is a categorical equivalence, it suffices to show that $A \times S \rightarrow B \times S$ is a categorical equivalence. In other words, we may suppose that $K$ is an $\infty$-category.

Since $A$ and $K$ are $\infty$-categories, $\h{(A \times K)} \simeq \h{A} \times \h{K}$; similarly
$\h{(B \times K)} \simeq \h{B} \times \h{K}$. It follows that $A \times K \rightarrow B \times K$ is essentially surjective, provided that $f$ is essentially surjective. Furthermore, for any pair of vertices $(a,k), (a',k') \in (A \times K)_0$, we have 
$$\Hom^{\rght}_{A \times K}( (a,k), (a',k') ) \simeq \Hom^{\rght}_{A}(a,a') \times \Hom^{\rght}_K(k,k')$$
$$\Hom^{\rght}_{B \times K}( (f(a),k), (f(a'),k')) \simeq \Hom^{\rght}_{B}(f(a),f(a')) \times \Hom^{\rght}_K(k,k').$$
It follows that $A \times K \rightarrow B \times K$ is fully faithful, provided that $f$ is fully faithful, which completes the proof.
\end{proof}

\begin{remark}\label{tokenn}
Since every inner anodyne map is a categorical equivalence, it follows that every categorical fibration $p: X \rightarrow S$ is a inner fibration (see Definition \ref{fibdeff}). The converse is false in general; however, it is true when $S$ is a point. In other words, the fibrant objects for the Joyal model structure on $\sSet$ are precisely the $\infty$-categories. The proof will be given in \S \ref{slin}, as Theorem \ref{joyalcharacterization}. We will assume this result for the remainder of the section. No circularity will result from this, since the proof of Theorem \ref{joyalcharacterization} will not use any of the results proven below.
\end{remark}

The functor $\sCoNerve[\bigdot]$ does not generally commute with products.
However, Corollary \ref{equivstable} implies that $\sCoNerve$ commutes with products in the following weak sense:

\begin{corollary}\label{prodcom}
Let $S$ and $S'$ be simplicial sets. The natural map
$$\sCoNerve[S \times S'] \rightarrow \sCoNerve[S] \times
\sCoNerve[S']$$ is an equivalence of simplicial categories.
\end{corollary}

\begin{proof}
Suppose first that there are fibrant simplicial categories
$\calC$, $\calC'$ with $S = \sNerve(\calC)$, $S' =
\sNerve(\calC')$. In this case, we have a diagram
$$ \sCoNerve[S \times S'] \stackrel{f}{\rightarrow} \sCoNerve[S] \times
\sCoNerve[S'] \stackrel{g}{\rightarrow} \calC \times \calC'.$$
By the two-out-of-three property, it suffices to show that $g$ and
$g \circ f$ are equivalences. Both of these assertions follow
immediately from the fact that the counit map
$\sCoNerve[\sNerve(\calD)] \rightarrow \calD$ is an equivalence for
{\em any} fibrant simplicial category $\calD$ (Theorem \ref{biggier}).

In the general case, we may choose categorical equivalences $S
\rightarrow T$, $S' \rightarrow T'$, where $T$ and $T'$ are nerves of fibrant simplicial categories. Since $S \times S'
\rightarrow T \times T'$ is a categorical equivalence, we reduce
to the case treated above.
\end{proof}

Let $K$ be a fixed simplicial set, and let $\calC$ be a simplicial set which is fibrant with respect to the Joyal model structure. 
Then $\calC$ has the extension property with respect to all inner anodyne maps, and is therefore a
$\infty$-category. It follows that $\Fun(K,\calC)$ is also an $\infty$-category.
We might call two morphisms $f,g: K
\rightarrow \calC$ {\it homotopic} if they are equivalent when viewed
as objects of $\Fun(K,\calC)$. On the other hand, the general
theory of model categories furnishes another notion of homotopy:
$f$ and $g$ are {\it left homotopic} if the map
$$ f \coprod g: K \coprod K \rightarrow \calC$$
can be extended over a mapping cylinder $I$ for $K$.

\begin{proposition}\label{fruity!}
Let $\calC$ be a $\infty$-category and $K$ an arbitrary simplicial set. A pair of morphisms $f,g: K \rightarrow \calC$ are homotopic if and only if they are left-homotopic.
\end{proposition}

\begin{proof}
Choose a contractible Kan complex $S$ containing a pair of distinct vertices, $x$ and $y$.
We note that the inclusion
$$ K \coprod K \simeq K \times \{x,y\} \subseteq K \times S$$
exhibits $K \times S$ as a mapping cylinder for $K$. It follows that $f$ and $g$ are left homotopic
if and only if the map $f \coprod g: K \coprod K \rightarrow \calC$ admits an extension to $K \times S$. 
In other words, $f$ and $g$ are left homotopic if and only if there exists a map $h: S \rightarrow \calC^K$ such that $h(x)=f$ and $h(y)=g$. We note that any such map factors through $Z$, where $Z \subseteq \Fun(K,\calC)$ is the largest Kan complex contained in $\calC^K$. Now, by classical homotopy theory, the map $h$ exists if and only if $f$ and $g$ belong to the same path component of $Z$. It is clear that this holds if and only if $f$ and $g$ are equivalent when viewed as objects of the $\infty$-category $\Fun(K,\calC)$. 
\end{proof}

We are now in a position to prove Proposition \ref{tyty}, which was asserted without proof in
\S \ref{funcback}. We first recall the statement.

\begin{proposition2}\index{gen}{$\infty$-category!of functors}
Let $K$ be an arbitrary simplicial set.
\begin{itemize}
\item[$(1)$] For every $\infty$-category $\calC$, the simplicial set $\Fun(K,\calC)$ is an $\infty$-category.

\item[$(2)$] Let $\calC \rightarrow \calD$ be a categorical equivalence of $\infty$-categories. Then the induced map $\Fun(K,\calC) \rightarrow \Fun(K,\calD)$ is a categorical equivalence.

\item[$(3)$] Let $\calC$ be an $\infty$-category, and $K \rightarrow K'$ a categorical equivalence of simplicial sets. Then the induced map $\Fun(K',\calC) \rightarrow \Fun(K,\calC)$ is a categorical equivalence.
\end{itemize}
\end{proposition2}

\begin{proof}
We first prove $(1)$. To show that $\Fun(K,\calC)$ is an $\infty$-category, it
suffices to show that it has the extension property with respect
to every inner anodyne inclusion $A \subseteq B$. This is equivalent
to the assertion that $\calC$ has the right lifting property with
respect to the inclusion $A \times K \subseteq B \times K$. But
$\calC$ is an $\infty$-category and $A \times K \subseteq B \times K$ is
inner anodyne (Corollary \ref{prodprod2}).

Let $\h{\sSet}$ denote the homotopy category of $\sSet$, taken with respect to the Joyal model structure. For each simplicial set $X$, we let $[X]$ denote the same simplicial set, considered
as an object of $\h{\sSet}$. For every pair of objects $X, Y \in \sSet$, $[X \times Y]$ is a product
for $[X]$ and $[Y]$ in $\h{\sSet}$. This is a general fact when $X$
and $Y$ are fibrant; in the general case, we choose fibrant
replacements $X \rightarrow X'$, $Y \rightarrow Y'$, and apply the fact that
the canonical map $X \times Y \rightarrow X' \times Y'$ is a categorical equivalence (Proposition \ref{fruity!}). 

If $\calC$ is an $\infty$-category, then $\calC$ is a fibrant object
of $\sSet$ (Theorem \ref{joyalcharacterization}). Proposition \ref{fruity!} allows us to identify
$ \Hom_{ \h{\sSet}}( [X], [\calC] )$ with the set of equivalence classes of objects in
the $\infty$-category $\Fun(X, \calC)$. In particular, we have a canonical bijections
$$ \Hom_{\h{\sSet}}( [X] \times [K], [\calC] ) 
\simeq \Hom_{ \h{\sSet}}( [X \times K] , [\calC] )
\simeq \Hom_{\h{ \sSet}}( [X], [\Fun(K,\calC)]).$$ 
It follows that $[ \Fun(K, \calC) ]$ is determined up to canonical isomorphism by
$[K]$ and $[\calC]$ (more precisely, it is an {\em exponential} $[\calC]^{[K]}$ in the homotopy
category $\h{\sSet}$), which proves $(2)$ and $(3)$.
\end{proof}

Our description of the Joyal model structure on $\sSet$ is different from the definition given in \cite{joyalnotpub}. Namely, Joyal defines a map $f: A \rightarrow B$ to be a
{\it weak categorical equivalence} if, for every $\infty$-category $\calC$,
the induced map
$$ \h{\Fun(B, \calC)} \rightarrow \h{\Fun(A, \calC)}$$
is an equivalence (of ordinary categories). To prove that our definition agrees with his, it will suffice to prove the following.\index{gen}{categorical equivalence!weak}

\begin{proposition}\label{joyaldef}
Let $f: A \rightarrow B$ be a map of simplicial sets. Then $f$ is a categorical equivalence if and only if it is a weak categorical equivalence.
\end{proposition}

\begin{proof}
Suppose first that $f$ is a categorical equivalence. If $\calC$ is an arbitrary $\infty$-category, Proposition \ref{tyty} implies that the induced map $\Fun(B,\calC) \rightarrow \Fun(A,\calC)$ is a categorical equivalence, so that $\h{\Fun(B,\calC)}\rightarrow \h{\Fun(A, \calC)}$ is an equivalence of categories. This proves that $f$ is a weak categorical equivalence.

Conversely, suppose that $f$ is a weak categorical equivalence. We wish to show that $f$ induces an isomorphism in the homotopy category of $\sSet$ with respect to the Joyal model structure. It will suffice to show that for any fibrant object $\calC$, $f$ induces a bijection $[B,\calC] \rightarrow [A,\calC]$, where $[X,\calC]$ denotes the set of homotopy classes of maps from $X$ to $\calC$. By Proposition \ref{fruity!}, $[X,\calC]$ may be identified with the set of isomorphism classes of objects in the category $\h{\Fun(X, \calC)}$. By assumption, $f$ induces an equivalence of categories 
$\h{\Fun(B, \calC)} \rightarrow \h{\Fun(A, \calC)}$, and therefore a bijection on isomorphism classes of objects.
\end{proof}

\begin{remark}
The proof of Proposition \ref{tyty} makes use of Theorem \ref{joyalcharacterization}, which asserts that the (categorically) fibrant objects of $\sSet$ are precisely the $\infty$-categories. Joyal proves the analogous assertion for his model structure in \cite{joyalnotpub}. We remark that one cannot formally deduce Theorem \ref{joyalcharacterization} from Joyal's result, since we {\em need} Theorem  \ref{joyalcharacterization} to prove that Joyal's model structure coincides with the one we have defined above. On the other hand, our approach {\em does} give a new proof of Joyal's theorem.
\end{remark}

\begin{remark}
Proposition \ref{joyaldef} permits us to define the Joyal model structure without reference to the theory of simplicial categories (this is Joyal's original point of view \cite{joyalnotpub}). Our approach 
is less elegant, but allows us to easily compare the theory of $\infty$-categories with other models of higher category theory, such as simplicial categories. There is another approach to obtaining comparison results, due to To\"{e}n. In \cite{toenchar}, he shows that if $\calC$ is a model category equipped with a cosimplicial object  $C^{\bigdot}$ satisfying certain conditions, then $\calC$ is (canonically) Quillen equivalent to Rezk's category of complete Segal spaces.
To\"{e}n's theorem applies in particular when $\calC$ is the category of simplicial sets, and
$C^{\bigdot}$ is the ``standard simplex'' $C^n = \Delta^n$.
In fact, $\sSet$ is in some sense universal with respect to this property, since it is generated by $C^{\bigdot}$ under colimits and the class of categorical equivalences is dictated by To\"{e}n's axioms. We refer the reader to \cite{toenchar} for details.
\end{remark}

