\section{Examples of Colimits}\label{coexample}

\setcounter{theorem}{0}


In this section, we will analyze in detail the colimits of some very simple diagrams.
Our first three examples are familiar from classical category theory: coproducts (\S \ref{quasilimit5}), 
pushouts (\S \ref{quasilimit6}), and coequalizers (\S \ref{quasilimit8}). 

Our fourth example is slightly more unfamiliar. Let $\calC$ be an ordinary category which admits coproducts. Then $\calC$ is naturally {\em tensored} over the category of sets. Namely, for each
$C \in \calC$ and $S \in \Set$, we can define $C \otimes S$ to be the coproduct of a collection
of copies of $C$, indexed by the set $S$. The object $C \otimes S$ is characterized by the
following universal mapping property:
$$ \Hom_{\calC}(C \otimes S, D) \simeq \Hom_{\Set}( S, \Hom_{\calC}(C,D) ).$$
In the $\infty$-categorical setting, it is natural to try to generalize this definition by allowing
$S$ to an object of $\SSet$. In this case, $C \otimes S$ can again be viewed as a kind of colimit, but cannot be written as a {\em coproduct} unless $S$ is discrete. We will study the situation in \S \ref{quasilimit7}.

Our final objective in this section is to study the theory of {\em retracts} in an $\infty$-category $\calC$. In \S \ref{retrus}, we will see that there is a close relationship between retracts in $\calC$ and idempotent endomorphisms, just as in classical homotopy theory. Namely, any retract of an object $C \in \calC$ determines an idempotent endomorphism of $C$; conversely, if $\calC$ is {\it idempotent complete}, then every idempotent endomorphism of $C$ determines a retract of $C$.
We will return to this idea in \S \ref{surot}.

\subsection{Coproducts}\label{quasilimit5}

In this section, we discuss the simplest type of colimit: namely, coproducts.
Let $A$ be a set; we may regard $A$ as a category with
$$ \Hom_{A}(I,J) = \begin{cases} \ast & \text{if } I=J \\ \emptyset & \text{if } I \neq J. \end{cases}$$
We will also identify $A$ with the (constant) simplicial set which is the nerve of this category.
We note a functor $G: A \rightarrow \sSet$ is injectively fibrant if and only if it takes values in the category $\Kan$ of Kan complexes. If this condition is satisfied, then the product
$\prod_{\alpha \in A} G(\alpha)$ is a homotopy limit for $G$.

Let $F: A \rightarrow \calC$ be a functor from $A$ to a fibrant simplicial category; in other words, $F$ specifies a collection $\{ X_{\alpha} \}_{\alpha \in A}$ of objects in $\calC$. A homotopy colimit for $F$ will be referred to as a {\it homotopy coproduct} of the objects $\{ X_{\alpha} \}_{\alpha \in A}$. Unwinding the definition, we see that a homotopy coproduct is an object $X \in \calC$\index{gen}{coproduct!homotopy}\index{gen}{homotopy!coproduct} equipped with morphisms
$\phi_{\alpha}: X_{\alpha} \rightarrow X$ such that the induced map
$$ \bHom_{\calC}(X,Y) \rightarrow \prod_{\alpha \in A} \bHom_{\calC}(X_{\alpha},Y)$$ 
is a homotopy equivalence for every object $Y \in \calC$. 
Consequently, we recover the description given in Example \ref{examprod}. As we noted earlier, this characterization can be stated entirely in terms of the enriched homotopy category $\h{\calC}$: the maps $\{ \phi_{\alpha} \} $ exhibit $X$ as a homotopy coproduct of the family $\{ X_{\alpha} \}_{\alpha \in A}$
if and only if the induced map
$$ \bHom_{\calC}(X,Y) \rightarrow \prod_{\alpha \in A} \bHom_{\calC}(X_{\alpha},Y)$$ is an isomorphism in the homotopy category $\calH$ of spaces, for each $Y \in \calC$.

Now suppose that $\calC$ is an $\infty$-category, and let $p: A \rightarrow \calC$ be a map. 
As above, we may identify this with a collection of objects $\{ X_{\alpha} \}_{\alpha \in A}$ of $\calC$. 
To specify an object of $\calC_{p/}$ is to give an object $X \in \calC$ together with morphisms
$\phi_{\alpha}: X_{\alpha} \rightarrow X$ for each $\alpha \in A$. Using Theorem \ref{colimcomparee}, we deduce that
$X$ is a colimit of the diagram $p$ if and only if the induced map
$$ \bHom_{\calC}(X,Y) \rightarrow \prod_{\alpha \in A} \bHom_{\calC}(X_{\alpha},Y)$$
is an isomorphism in $\calH$, for each object $Y \in \calC$. In this case, we say that $X$ is a {\it coproduct} of the family $\{ X_{\alpha} \}_{\alpha \in A}$.

In either setting, we will denote the (homotopy) coproduct of a family of objects $\{ X_\alpha \}_{\alpha \in A}$ by
$$ \coprod_{\alpha \in A} X_I.$$
It is well-defined up to (essentially unique) equivalence.

Using Corollary \ref{util}, we deduce the following:

\begin{proposition}\index{gen}{coproduct}\index{gen}{product}\label{makerus}
Let $\calC$ be an $\infty$-category, and let $\{ p_{\alpha}: K_{\alpha} \rightarrow \calC\}_{\alpha \in A}$ be a family of diagrams in $\calC$ indexed by a set $A$. Suppose that each $p_{\alpha}$ has a colimit $X_{\alpha}$ in $\calC$. Let $K = \coprod K_{\alpha}$, and let $p: K \rightarrow \calC$ be the result of amalgamating the maps $p_{\alpha}$. Then $p$ has a colimit in $\calC$ if and only if the family $\{ X_{\alpha} \}_{\alpha \in A}$ has a coproduct in $\calC$; in this case, one may identify colimits of $p$ with coproducts $ \coprod_{\alpha \in A} X_{\alpha}.$
\end{proposition}

\subsection{Pushouts}\label{quasilimit6}

Let $\calC$ be an $\infty$-category. A {\it square} in $\calC$ is a map
$\Delta^1 \times \Delta^1 \rightarrow \calC$. We will typically denote
squares in $\calC$ by diagrams
$$ \xymatrix{ X' \ar[r]^{p'} \ar[d]^{q'} & X \ar[d]^{q} \\
Y' \ar[r]^{p} & Y, }$$
with the ``diagonal'' morphism $r: X' \rightarrow Y$ and homotopies
$r \simeq q \circ p'$, $r \simeq p \circ q'$ being implicit.\index{gen}{square}

We have isomorphisms of simplicial sets
$$ (\Lambda^2_0)^{\triangleright} \simeq \Delta^1 \times \Delta^1 \simeq (\Lambda^2_2)^{\triangleleft}.$$
Consequently, given a square $\sigma: \Delta^1 \times \Delta^1 \rightarrow \calC$, it makes sense to ask whether or not $\sigma$ is a limit or colimit diagram. If $\sigma$ is a limit diagram, we will also say that $\sigma$ is a {\it pullback square} or a {\it Cartesian square}, and we will informally write $X' = X \times_{Y} Y'$. Dually, if $\sigma$ is a colimit diagram, we will say that $\sigma$ is a {\it pushout square} or a {\it coCartesian square}, and write $Y = X \coprod_{X'} Y'$.\index{gen}{square!pullback}\index{gen}{square!pushout}\index{gen}{pullback}\index{gen}{pushout}\index{gen}{square!Cartesian}\index{gen}{square!coCartesian}

Now suppose that $\calC$ is a (fibrant) simplicial category. By definition, a commutative diagram
$$ \xymatrix{ X' \ar[r]^{p'} \ar[d]^{q'} & X \ar[d]^{q} \\
Y' \ar[r]^{p} & Y }$$
is a homotopy pushout square if, for every object $Z \in \calC$, the diagram
$$ \xymatrix{ \bHom_{\calC}(Y,Z) \ar[r] \ar[d] & \bHom_{\calC}(Y',Z) \ar[d] \\
\bHom_{\calC}(X,Z) \ar[r] & \bHom_{\calC}(X',Z) }$$
is a homotopy pullback square in $\Kan$. Using Theorem \ref{colimcomparee}, we can reduce questions about pushout diagrams in an arbitrary $\infty$-category to questions about homotopy pullback squares in $\Kan$.\index{gen}{pushout!homotopy}

The following basic transitivity property for pushout squares will be used repeatedly throughout this book: 

\begin{lemma}\label{transplantt}
Let $\calC$ be an $\infty$-category, and suppose given a map 
$\sigma: \Delta^2 \times \Delta^1 \rightarrow \calC$ which we will depict as a diagram
$$ \xymatrix{ X \ar[r] \ar[d] & Y \ar[d] \ar[r] & Z \ar[d] \\
X' \ar[r] & Y' \ar[r] & Z'. }$$
Suppose that the left square is a pushout in $\calC$. Then the right square is a pushout
if and only if the outer square is a pushout.
\end{lemma}

\begin{proof}
For every subset $A$ of $\{ x,y,z, x', y', z'\}$, let $\calD(A)$ denote the corresponding
full subcategory of $\Delta^2 \times \Delta^1$, and let $\sigma(A)$ denote the restriction of $\sigma$ to $\calD(A)$. We may regard $\sigma$ as determining an object
$\widetilde{\sigma} \in \calC_{\sigma(\{ y,z,x',y',z' \})/}$. Consider the maps
$$ \calC_{ \sigma(\{ z,x',z' \} ) /} \stackrel{\phi}{\leftarrow}
\calC_{ \sigma(\{ y,z, x',y',z'\})/} \stackrel{\psi}{\rightarrow} \calC_{\sigma(\{ y,x',y' \})/} $$
The map on $\phi$ is the composition of the trivial fibration
$$ \calC_{\sigma(\{ z,x',y',z' \} )/} \rightarrow \calC_{\sigma(\{ z,x',z' \}) /}$$ with a pullback of 
$$\calC_{\sigma( \{y,z,y',z'\} ) /} \rightarrow \calC_{ \sigma( \{z,y',z' \}) /},$$ also a trivial fibration in virtue of our assumption that the square
$$ \xymatrix{ Y \ar[d] \ar[r] & Z \ar[d] \\
Y' \ar[r] & Z' }$$
is a pullback in $\calC$. The map $\psi$ is a trivial fibration because the inclusion
$\calD( \{ y,x',y' \}) \subseteq \calD( \{y,z,x',y',z' \})$ is left anodyne. It follows that
$\phi(\widetilde{\sigma})$ is an initial object of $ \calC_{ \sigma(\{ z,x',z' \} ) /}$
if and only if $\psi(\widetilde{\sigma})$ is an initial object of 
$\calC_{\sigma(\{ y,x',y' \}) /}$, as desired.
\end{proof}

Our next objective is to apply Proposition \ref{utl} to show that
in many cases, complicated colimits may be decomposed as pushouts
of simpler colimits. Suppose given a pushout diagram of simplicial sets
$$ \xymatrix{ L' \ar[r]^{i} \ar[d] & L \ar[d] \\
K' \ar[r] & K }$$
and a diagram $p: K \rightarrow \calC$, where $\calC$ is an $\infty$-category.
Suppose furthermore that $p|K'$, $p|L'$, and $p|L$ admit colimits in $\calC$, which we will denote by $X$, $Y$, and $Z$, respectively. If we suppose further that the map $i$ is a cofibration of simplicial sets, then the hypotheses of Proposition \ref{extet} are satisfied. 
Consequently, we deduce:

\begin{proposition}\label{train}
Let $\calC$ be an $\infty$-category, and let $p: K \rightarrow \calC$ be a map of simplicial
sets. Suppose given a decomposition $K = K' \coprod_{L'} L$, where $L' \rightarrow L$ is a monomorphism of simplicial sets. Suppose further that
$p|K'$ has a colimit $X \in \calC$, $p|L'$ has a colimit $Y \in \calC$, and $p|L$ has
a colimit $Z \in \calC$. Then one may identify colimits for $p$ with
pushouts $X \coprod_{Y} Z$.
\end{proposition}

\begin{remark}
The statement of Proposition \ref{train} is slightly vague.
Implicit in the discussion is that identifications of $X$ with the
colimit of $p|K'$ and $Y$ with the colimit of $p|L'$ induce a
morphism $Y \rightarrow X$ in $\calC$ (and similarly for $Y$ and $Z$).
This morphism is not uniquely determined, but it is determined up
to a contractible space of choices: see the proof of Proposition
\ref{extet}.
\end{remark}

It follows from Proposition \ref{train} that any finite colimit
can be built using initial objects and pushout squares. For example,
we have the following:

\begin{corollary}\label{allfin}\index{gen}{colimit!finite}
Let $\calC$ be an $\infty$-category. Then $\calC$ admits all finite colimits if
and only if $\calC$ admits pushouts and has an initial object.
\end{corollary}

\begin{proof}
The ``only if'' direction is clear. For the converse, let us
suppose that $\calC$ has pushouts and an initial object. Let $p: K
\rightarrow \calC$ be any diagram, where $K$ is a finite simplicial
set: that is, $K$ has only finitely many nondegenerate simplices.
We will prove that $p$ has a colimit. The proof goes by induction:
first on the dimension of $K$, then on the number of simplices of
$K$ having the maximal dimension.

If $K$ is empty, then an initial object of $\calC$ is a colimit for
$p$. Otherwise, we may fix a nondegenerate simplex of $K$ having
the maximal dimension, and thereby decompose $K \simeq K_0
\coprod_{ \bd \Delta^n } \Delta^n$. By the inductive hypothesis,
$p|K_0$ has a colimit $X$ and $p| \bd \Delta^n$ has a colimit $Y$.
The $\infty$-category $\Delta^n$ has a final object, so $p| \Delta^n$
has a colimit $Z$ (which we may take to be $p(v)$, where $v$ is
the final vertex of $\Delta^n$). Now we simply apply Proposition
\ref{train} to deduce that $X \coprod_Y Z$ is a colimit for $p$.
\end{proof}

Using the same argument, one can show:

\begin{corollary}\label{allfinn}
Let $f: \calC \rightarrow \calC'$ be a functor between $\infty$-categories. Assume
that $\calC$ has all finite colimits. Then $f$ preserves all finite
colimits if and only if $f$ preserves initial objects and pushouts.
\end{corollary}

We conclude by showing how {\em all} colimits can be constructed
out of simple ones.

\begin{proposition}\label{alllimits}
Let $\calC$ be an $\infty$-category. Suppose that $\calC$ admits pushouts and
$\kappa$-small coproducts. Then $\calC$ admits colimits for all $\kappa$-small diagrams.
\end{proposition}

\begin{proof}
If $\kappa = \omega$, we have already shown this as Corollary
\ref{allfin}. Let us therefore suppose that $\kappa > \omega$, and
that $\calC$ has pushouts and $\kappa$-small sums.

Let $p: K \rightarrow \calC$ be a diagram, where $K$ is
$\kappa$-small. We first suppose that the dimension $n$ of $K$ is
finite: that is, $K$ has no nondegenerate simplices of dimension
$> n$. We prove that $p$ has a colimit, working by induction on
$n$.

If $n=0$, then $K$ consists of a finite disjoint union of fewer
than $\kappa$ vertices. The colimit of $p$ exists by the
assumption that $\calC$ has $\kappa$-small sums.

Now suppose that every diagram indexed by a $\kappa$-small
simplicial set of dimension $n$ has a colimit. Let $p: K
\rightarrow \calC$ be a diagram, with the dimension of $K$ equal to
$n+1$. Let $K^n$ denote the $n$-skeleton of $K$, and $K'_{n+1} \subseteq K_{n+1}$ the
set of all nondegenerate $(n+1)$-simplices of $K$, so that there is a pushout
diagram of simplicial sets
$$ K^n \coprod_{ K'_{n+1} \times \bd \Delta^{n+1}} (K'_{n+1} \times
\Delta^{n+1}) \simeq K.$$ By Proposition \ref{train}, we can
construct a colimit of $p$ as a pushout, using colimits for
$p|K^n$, $p|(K'_{n+1} \times \bd \Delta^{n+1})$, and $p|(K'_{n+1}
\times \Delta^{n+1})$. The first two exist by the inductive
hypothesis; the last, because it is a sum of diagrams which
possess colimits.

Now let us suppose that $K$ is not necessarily finite dimensional.
In this case, we can filter $K$ by its skeleta $\{ K^n \}$. This
is a family of simplicial subsets of $K$ indexed by the set
$\Z_{\geq 0}$ of nonnegative integers. By what we have shown
above, each $p|K^n$ has a colimit $x_n$ in $\calC$. Since this family
is directed and covers $K$, Corollary \ref{util} shows that we may identify colimits of
$p$ with colimits of a diagram $\Nerve(\Z_{\geq 0}) \rightarrow \calC$ which we may write informally as
$$ x_0 \rightarrow x_1 \rightarrow \ldots $$

Let $L$ be the simplicial subset of
$\Nerve(\Z_{\geq 0})$ which consists of all vertices, together with the
edges which join consecutive integers. A simple computation shows
that the inclusion $L \subseteq \Nerve(\Z_{\geq 0})$ is a categorical
equivalence, and therefore cofinal. Consequently, it suffices to
construct the colimit of a diagram $L \rightarrow \calC$. But $L$ is
$1$-dimensional, and is $\kappa$-small since $\kappa > \omega$.
\end{proof}

The same argument proves also the following:

\begin{proposition}\label{allimits}
Let $\kappa$ be a regular cardinal, and let $f: \calC \rightarrow \calD$ be a functor between $\infty$-categories, where $\calC$ admits $\kappa$-small colimits. Then
$f$ preserves $\kappa$-small colimits if and only if $f$ preserves pushout squares
and $\kappa$-small coproducts.
\end{proposition}

Let $\calD$ be an $\infty$-category containing an object $X$, and suppose that $\calD$
admits pushouts. Then $\calD_{X/}$ admits pushouts, and these pushouts map be computed in $\calD$. In other words, the projection $f: \calD_{X/} \rightarrow \calD$ preserves pushouts. In fact, this is a special case of a very general result; it requires only that $f$ is a left fibration and the
simplicial set $\Lambda^2_0$ is weakly contractible.

\begin{lemma}\label{pregoes}
Let $f: \calC \rightarrow \calD$ be a left fibration of $\infty$-categories, and let
$K$ be a weakly contractible simplicial set. Then any map
$\overline{p}: K^{\triangleright} \rightarrow \calC$ is an $f$-colimit diagram.
\end{lemma}

\begin{proof}
Let $p = \overline{p} | K$. We must show that the map
$$\phi:  \calC_{\overline{p} /} \rightarrow \calC_{p/} \times_{\calD_{f \circ p/}} \calD_{f \circ \overline{p}/}$$ is a trivial fibration of simplicial sets. In other words, we must show that
we can solve any lifting problem of the form
$$ \xymatrix{ (K \star A) \coprod_{ K \star A_0} (K^{\triangleright} \star A) \ar[r] \ar@{^{(}->}[d] & \calC \ar[d]^{f} \\
K^{\triangleright} \star A \ar[r] \ar@{-->}[ur] & \calD. }$$
Since $f$ is a left fibration, it will suffice to prove that the left vertical map is left anodyne, which follows immediately from Lemma \ref{chotle}. 
\end{proof}

\begin{proposition}\label{goeselse}
Let $f: \calC \rightarrow \calD$ be a left fibration of $\infty$-categories, and let
$p: K \rightarrow \calC$ be a diagram. Suppose that
$K$ is weakly contractible. Then:
\begin{itemize}
\item[$(1)$] Let $\overline{p}: K^{\triangleright} \rightarrow \calC$ be an extension of $p$.
Then $\overline{p}$ is a colimit of $p$ if and only if $f \circ \overline{p}$ is a colimit of $f \circ p$.

\item[$(2)$] Let $\overline{q}: K^{\triangleright} \rightarrow \calD$ be a colimit of $f \circ p$.
Then $\overline{q} = f \circ \overline{p}$, where $\overline{p}$ is an extension $($ automatically a colimit, in virtue of $(1)$ $)$ of $p$.
\end{itemize}
\end{proposition}

\begin{proof}
To prove $(1)$, fix an extension $\overline{p}: K^{\triangleright} \rightarrow \calC$. 
We have a commutative diagram
$$ \xymatrix{ \calC_{\overline{p}/} \ar[r]^-{\phi} & \calC_{p/} \times_{\calD_{fp/}} \calD_{f\overline{p}/} \ar[r]^-{\psi'} \ar[d] & \calC_{p/} \ar[d]^{\theta} \\
& \calD_{f \overline{p}/} \ar[r]^-{\psi} & \calD_{fp/}. }$$
Lemma \ref{pregoes} implies that $\phi$ is a trivial Kan fibration.
If $f \circ \overline{p}$ is a colimit diagram, the map $\psi$ is a trivial fibration. Since $\psi'$ is a pullback of $\psi$, we conclude that $\psi'$ is a trivial fibration. It follows that $\psi' \circ \phi$ is a trivial fibration, so that $\overline{p}$ is a colimit diagram. This proves the ``if'' direction of $(1)$.

To prove the converse, let us suppose that $\overline{p}$ is a colimit diagram. The maps
$\phi$ and $\psi' \circ \phi$ are both trivial fibrations. It follows that the fibers of $\psi'$ are contractible. Using Lemma \ref{chotle2}, we conclude that the map $\theta$ is a trivial fibration, and therefore surjective on vertices. It follows that the fibers of $\psi$ are contractible. Since $\psi$ is a left fibration with contractible fibers, it is a trivial fibration (Lemma \ref{toothie}). Thus $f \circ \overline{p}$ is a colimit diagram and the proof is complete.

To prove $(2)$, it suffices to show that $f$ has the right lifting property with respect to the inclusion
$i: K \subseteq K^{\triangleright}$. Since $f$ is a left fibration, it will suffice to show that $i$ is left anodyne, which follows immediately from Lemma \ref{chotle2}.
\end{proof}

\subsection{Coequalizers}\label{quasilimit8}

Let $\calI$ denote the category depicted by the diagram
$$\xymatrix{ X \ar@<.4ex>[r]^{F} \ar@<-.4ex>[r]_{G} & Y}.$$
In other words, $\calI$ has two objects, $X$ and $Y$, with
$ \Hom_{\calI}(X,X) = \Hom_{\calI}(Y,Y) = \ast$, $ \Hom_{\calI}(Y,X) = \emptyset$, and
$ \Hom_{\calI}(X,Y) = \{ F,G \}.$

To give a diagram $p: \Nerve(\calI) \rightarrow \calC$ in an $\infty$-category $\calC$, one must give a pair of morphisms $f = p(F)$, $g = p(G)$ in $\calC$, having the same domain $x=p(X)$ and the same codomain $y = p(Y)$. A colimit for the diagram $p$ is said to be a {\em coequalizer} of $f$ and $g$.\index{gen}{coequalizer}\index{gen}{equalizer}

Applying Corollary \ref{util}, we deduce the following:

\begin{proposition}\label{uty}
Let $K$ and $A$ be a simplicial sets, and let $i_0, i_1: A \rightarrow K$ be embeddings having disjoint images in $K$. Let $K'$ denote the coequalizer of $i_0$ and $i_1$; in other words, the simplicial set obtained from $K$ by identifying the image of $i_0$ with the image of $i_1$.
Let $p: K' \rightarrow \calC$ be a diagram in an $\infty$-category $S$, and let $q: K \rightarrow \calC$
be the composition
$$ K \rightarrow K' \stackrel{p}{\rightarrow} S.$$
Suppose that the diagrams $q \circ i_0= q \circ i_1$ and $q$ possess colimits $x$ and $y$ in $S$. 
Then $i_0$ and $i_1$ induce maps $j_0, j_1: x \rightarrow y$ $($well-defined up to homotopy$)$; 
colimits for $p$ may be identified with coequalizers of $j_0$ and $j_1$.
\end{proposition}

Like pushouts, coequalizers are a basic construction out of which other colimits can be built.
More specifically, we have the following:

\begin{proposition}\label{appendixdiagram}
Let $\calC$ be an $\infty$-category and $\kappa$ a regular cardinal. Then $\calC$ has all $\kappa$-small colimits if and only if $\calC$ has coequalizers and $\kappa$-small coproducts.
\end{proposition}

\begin{proof}
The ``only if'' direction is obvious. For the converse, suppose that $\calC$ has coequalizers and $\kappa$-small coproducts. In view of Proposition \ref{alllimits}, it suffices to show that $\calC$ has pushouts.
Let $p: \Lambda^2_0$ be a pushout diagram in $\calC$. We note that $\Lambda^2_0$ is the quotient of $\Delta^{ \{0,1\} } \coprod \Delta^{ \{0,2\} }$ obtained by identifying the initial vertex of $\Delta^{ \{0,1\} }$ with the initial vertex of $\Delta^{ \{0,2\} }$. In view of Proposition \ref{uty}, it suffices to show that $p| ( \Delta^{ \{0,1\} } \coprod \Delta^{ \{0,2\} } )$ and $p| \{0\}$ possess colimits in $\calC$. The second assertion is obvious. Since $\calC$ has finite sums, to prove that there exists a colimit for
$p| ( \Delta^{ \{0,1\} } \coprod \Delta^{ \{0,2\} } )$, it suffices to prove that $p| \Delta^{ \{0,1\}}$ and
$p|\Delta^{ \{0,2\} }$ possess colimits in $\calC$. This is immediate, since $\Delta^{ \{0,1\} }$ and $\Delta^{ \{0,2\} }$ both have final objects.
\end{proof}

Using the same argument, we deduce:

\begin{proposition}\label{appendicites}
Let $\kappa$ be a regular cardinal and $\calC$ be an $\infty$-category which admits $\kappa$-small colimits. A full subcategory $\calD \subseteq \calC$ is stable under $\kappa$-small colimits in $\calC$ if and only if $\calD$ is stable under coequalizers and under $\kappa$-small sums.
\end{proposition}

\subsection{Tensoring with Spaces}\label{quasilimit7}

Every ordinary category $\calC$ can be regarded as a category enriched over $\Set$.
Moreover, if $\calC$ admits coproducts, then $\calC$ can be regarded as {\em tensored} over $\Set$, in an essentially unique way. In the $\infty$-categorical setting, one has a similar situation: if $\calC$ is an $\infty$-category which admits all small limits, then $\calC$ may be regarded as tensored over the $\infty$-category $\SSet$ of spaces. To make this idea precise, we would need a good theory of {\em enriched} $\infty$-categories, which lies outside the scope of this book. We will instead settle for a slightly ad-hoc point of view which nevertheless allows us to construct the relevant tensor products. We begin with a few remarks concerning representable functors in the $\infty$-categorical setting.

\begin{definition}\label{repfunc}
Let $\calD$ be a closed monoidal category, and let $\calC$ be a category enriched over $\calD$.
We will say that a $\calD$-enriched functor $G: \calC^{op} \rightarrow \calD$ is {\it representable} if there exists an object $C \in \calC$ and a map $\eta: 1_{\calD} \rightarrow G(C)$ such that the induced map
$$ \bHom_{\calC}(X,C) \simeq \bHom_{\calC}(X,C) \otimes 1_{\calD}
\rightarrow \bHom_{\calC}(X,C) \otimes G(C) \rightarrow G(X)$$
is an isomorphism, for every object $X \in \calC$. In this case, we will say that 
$(C,\eta)$ {\it represents} the functor $F$. 
\end{definition}

\begin{remark}
In the situation of Definition \ref{repfunc}, we will sometimes abuse terminology and simply say that the functor $F$ is {\it represented} by the object $C$.\index{gen}{representable!functor}
\end{remark}

\begin{remark}
The dual notion of a {\it corepresentable functor} is may be defined in an obvious way.
\end{remark}

\begin{definition}\index{gen}{functor!representable by an object}
Let $\calC$ be an $\infty$-category, and let $\SSet$ denote the $\infty$-category of spaces.
We will say that a functor $F: \calC^{op} \rightarrow \SSet$ is {\it representable} if
the underlying functor
$$ \h{F}: \h{\calC}^{op} \rightarrow \h{\SSet} \simeq \calH$$
of $($ $\calH$-enriched $)$ homotopy categories is representable. We will say that a pair
$C \in \calC$, $\eta \in \pi_0 F(C)$ {\it represents} $F$ if the pair $(C,\eta)$ represents
$h F$. 
\end{definition}

\begin{proposition}\label{reppfunc}
Let $f: \widetilde{\calC} \rightarrow \calC$ be a right fibration of $\infty$-categories, let
$\widetilde{C}$ be an object of $\widetilde{\calC}$, $C = f(\widetilde{C}) \in \calC$, and let
$F: \calC^{op} \rightarrow \SSet$ be a functor which classifies $f$ (\S \ref{universalfib}).
The following conditions are equivalent:
\begin{itemize}
\item[$(1)$] Let $\eta \in \pi_0 F(C) \simeq \pi_0 
( \widetilde{\calC} \times_{\calC} \{C\} )$ be the connected component containing
$\widetilde{C}$. Then the pair $(C, \eta)$ represents the functor $F$. 

\item[$(2)$] The object $\widetilde{C} \in \widetilde{\calC}$ is final.

\item[$(3)$] The inclusion $\{ \widetilde{\calC} \} \subseteq \widetilde{\calC}$ is a contravariant equivalence in $(\sSet)_{/\calC}$. 
\end{itemize}
\end{proposition}

\begin{proof}
We have a commutative diagram of right fibrations
$$ \xymatrix{ \widetilde{\calC}_{/ \widetilde{C}} \ar[r]^{\phi} \ar[d] & \widetilde{\calC} \ar[d] \\
\calC_{/C} \ar[r] & \calC. }$$
Observe that the left vertical map is actually a trivial fibration.
Fix an object $D \in \calC$. The fiber of the upper horizontal map
$$ \phi_{D}: \widetilde{\calC}_{/ \widetilde{C}} \times_{\calC} \{D\} \rightarrow \widetilde{\calC} \times_{\calC} \{D\}$$ can be identified, in the homotopy category $\calH$, with the map
$\bHom_{\calC}(D,C) \rightarrow F(C).$ The map $\phi_{D}$ is a right fibration of Kan complexes, and therefore a Kan fibration. If $(1)$ is satisfied, then $\phi_{D}$ is a homotopy equivalence, and therefore a trivial fibration. It follows that the fibers of $\phi$ are contractible. Since $\phi$ is a right fibration, it is a trivial fibration (Lemma \ref{toothie}). This proves that $\widetilde{C}$ is a final object of $\widetilde{\calC}$. Conversely, if $(2)$ is satisfied, then $\phi_{D}$ is a trivial Kan fibration and therefore a weak homotopy equivalence. Thus $(1) \Leftrightarrow (2)$. 

If $(2)$ is satisfied, then the inclusion $\{ \widetilde{C} \} \subseteq \widetilde{\calC}$ is right anodyne, and therefore a contravariant equivalence by Proposition \ref{hunef}. Thus $(2) \Rightarrow (3)$. Conversely, suppose that $(3)$ is satisfied. The inclusion
$\{\id_{C} \} \subseteq \calC_{/C}$ is right anodyne, and therefore a contravariant equivalence. It follows that the lifting problem
$$ \xymatrix{ \{\id_{C} \} \ar[r]^{\widetilde{C}} \ar@{^{(}->}[d] & \widetilde{\calC} \ar[d]^{f} \\
\calC_{/C} \ar[r] \ar@{-->}[ur]^{e} & \calC }$$
has a solution. We observe that $e$ is a contravariant equivalence of right fibrations over $\calC$, and therefore a categorical equivalence. By construction, $e$ carries a final object
of $\calC_{/C}$ to $\widetilde{C}$, so that $\widetilde{C}$ is a final object of $\widetilde{\calC}$. 
\end{proof}

We will say that a right fibration $\widetilde{\calC} \rightarrow \calC$ is {\it representable} if
$\widetilde{\calC}$ has a final object.\index{gen}{representable!right fibration}

\begin{remark}
Let $\calC$ be an $\infty$-category, and let $p: K \rightarrow \calC$ be a diagram. Then
the right fibration $\calC_{/p} \rightarrow \calC$ is representable if and only if $p$ has a limit in $\calC$.
\end{remark}

\begin{remark}\index{gen}{corepresentable!functor}\index{gen}{corepresentable!left fibration}\index{gen}{functor!corepresentable}
All of the above ideas dualize in an evident way, so that we may speak of {\it corepresentable functors} and {\it corepresentable left fibrations} in the setting of $\infty$-categories.
\end{remark}

\begin{notation}
For each diagram $p: K \rightarrow \calC$ in an $\infty$-category $\calC$, we let
$\calF_{p}: \h{\calC} \rightarrow \calH$ denote the $\calH$-enriched functor
corresponding to the left fibration $\calC^{p/} \rightarrow \calC$.

If $p: \ast \rightarrow \calC$ is the inclusion of an object $X$ of $\calC$, then we write $\calF_{X}$ for $\calF_{p}$. We note that $\calF_{X}$ is the functor corepresented by $X$:
$$ \calF_X(Y) = \bHom_{\calC}(X,Y).$$
\end{notation}

Now suppose that $X$ is an object in an $\infty$-category $\calC$, and let $p: K \rightarrow \calC$
be a constant map taking the value $X$. For every object $Y$ of $\calC$, we have an isomorphism of simplicial sets
$(\calC^{p/}) \times_{\calC} \{Y\} \simeq ( \calC^{X/} \times_{\calC} \{Y\} )^K$. This identification is functorial up to homotopy, so we actually obtain an equivalence
$$\calF_{p}(Y) \simeq \bHom_{\calC}(X,Y)^{[K]}$$
in the homotopy category $\calH$ of spaces, where $[K]$ denotes the simplicial set $K$ regarded as an object of $\calH$. Applying Proposition \ref{reppfunc}, we deduce the following:

\begin{corollary}\label{charext}
Let $\calC$ be an $\infty$-category, $X$ and object of $\calC$, and $K$ a simplicial set.
Let $p: K \rightarrow \calC$ be the constant map taking the value $X$. The objects of the fiber
$\calC^{p/} \times_{\calC} \{Y\}$ are classified, up to equivalence, by maps
$\psi: [K] \rightarrow \bHom_{\calC}(X,Y)$
in the homotopy category $\calH$. Such a map $\psi$ classifies a colimit for $p$ if and only if
it induces isomorphisms
$$ \bHom_{\calC}(Y,Z) \simeq \bHom_{\calC}(X,Z)^{[K]} $$ 
in the homotopy category $\calH$, for every object $Z$ of $\calC$.
\end{corollary}

In the situation of Corollary \ref{charext}, we will denote a colimit for $p$ by $X \otimes K$, if such a colimit exists. We note that $X \otimes K$ is well defined up to (essentially unique) equivalence, and that it depends (up to equivalence) only on the weak homotopy type of the simplicial set $K$.\index{not}{XotimesK@$X \otimes K$}\index{gen}{tensor product with spaces}

\begin{corollary}\label{silt}
Let $\calC$ be an $\infty$-category, let $K$ be a weakly contractible simplicial set, and let
$p: K \rightarrow \calC$ be a diagram which carries each edge of $K$ to an equivalence in $\calC$.
Then:
\begin{itemize}
\item[$(1)$] The diagram $p$ has a colimit in $\calC$.
\item[$(2)$] An arbitrary extension $\overline{p}: K^{\triangleright} \rightarrow \calC$
is a colimit for $\calC$ if and only if $\overline{p}$ carries each edge of
$K^{\triangleright} \rightarrow \calC$ to an equivalence in $\calC$.
\end{itemize}
\end{corollary}

\begin{proof}
Let $\calC' \subseteq \calC$ be the largest Kan complex contained in $\calC$. By assumption, $p$ factors through $\calC'$. Since $K$ is weakly contractible, we conclude that $p: K \rightarrow \calC'$ is homotopic to a constant map $p': K \rightarrow \calC'$. Replacing $p$ by $p'$ if necessary, we may reduce to the case where $p$ is constant, taking value equal to some
fixed object $C \in \calC$.

Let $\overline{p}: K^{\triangleright} \rightarrow \calC$ be the constant map with value $C$. Using the characterization of colimits in Corollary \ref{charext}, we deduce that $\overline{p}$ is a colimit diagram in $\calC$. This proves $(1)$, and (in view of the uniqueness of colimits up to equivalence) the ``only if'' direction of $(2)$. To prove the converse, we suppose that $\overline{p}'$ is an arbitrary extension of $p$ which carries each edge of $K^{\triangleright}$ to an equivalence in $\calC$. Then $\overline{p}'$ factors through $\calC'$. Since $K^{\triangleright}$ is weakly contractible, we conclude as above that $\overline{p}'$ is homotopic to a constant map, and is therefore a colimit diagram.
\end{proof}

\subsection{Retracts and Idempotents}\label{retrus}

Let $\calC$ be a category. An object $Y \in \calC$ is said to be a {\it retract} of an object\index{gen}{retract} $X \in \calC$ if there is a commutative diagram
$$ \xymatrix{ & X \ar[dr]^{r} & \\
Y \ar[rr]^{\id_{Y}} \ar[ur]^{i} & & Y}$$
in $\calC$. In this case we can identify $Y$ with a subobject of $X$ via the monomorphism $i$, and think of $r$ as a retraction from $X$ onto $Y \subseteq X$. We observe also that the map
$i \circ r: X \rightarrow X$ is idempotent. Moreover, this idempotent determines $Y$ up to canonical isomorphism: we can recover $Y$ as the equalizer of the pair of maps $(\id_{X}, i \circ r): X \rightarrow X$ (or, dually, as the coequalizer of the same pair of maps). Consequently, we obtain an injective map from the collection of isomorphism classes of retracts of $X$ to the set of idempotent maps $f: X \rightarrow X$. We will say that $\calC$ is {\it idempotent complete} if this correspondence is bijective for every $X \in \calC$: that is, if every idempotent map $f: X \rightarrow X$ comes from a (uniquely determined) retract of $X$. If $\calC$ admits equalizers (or coequalizers), then $\calC$ is 
idempotent complete.\index{gen}{idempotent!completeness}\index{gen}{idempotent!in an ordinary category}

These ideas can be adapted to the $\infty$-categorical setting in a straightforward way. If $X$ and $Y$ are objects of an $\infty$-category $\calC$, then we say that $Y$ is a {\it retract} of $X$ if it is a retract of $X$ in the homotopy category $\h{\calC}$. Equivalently, $Y$ is a retract of $X$ if there exists a $2$-simplex $\Delta^2 \rightarrow \calC$ corresponding to a diagram
$$ \xymatrix{ & X \ar[dr]^{r} & \\
Y \ar[rr]^{\id_{Y}} \ar[ur]^{i} & & Y.}$$
As in the classical case, there is a correspondence between retracts $Y$ of $X$ and idempotent maps $f: X \rightarrow X$. However, there are two important differences: first, the notion of an idempotent map needs to be interpreted in an $\infty$-categorical sense. It is not enough to require that $f = f \circ f$ in the homotopy category $\h{\calC}$. This would correspond to the condition that there is a path $p$ joining $f$ to $f \circ f$ in the endomorphism space of $X$, which would give rise to {\em two} paths from $f$ to $f \circ f \circ f$. In order to have a hope of recovering $Y$, we need these paths to be homotopic. This condition does not even make sense unless $p$ is specified; thus we must take $p$ as part of the data of an idempotent map. In other words, in the $\infty$-categorical setting, idempotence is not merely a condition, but involves additional data (see Definition \ref{diags}).

The second important difference between the classical and $\infty$-categorical theory of retracts is that in the $\infty$-categorical case, one cannot recover a retract $Y$ of $X$ as the limit (or colimit) of a {\em finite} diagram involving $X$.

\begin{example}
Let $R$ be a commutative ring, and let $C_{\bigdot}(R)$ be the category of complexes of finite free $R$-modules, so that an object of $C_{\bigdot}(R)$ is a chain complex
$$ \ldots \rightarrow M_{1} \rightarrow M_0 \rightarrow M_{-1} \rightarrow \ldots $$
such that each $M_{i}$ is a finite free $R$-module, and $M_{i} = 0$ for $|i| \gg 0$; morphisms in $C_{\bigdot}(R)$ are given by morphisms of chain complexes. There is a natural simplicial structure on the category $C_{\bigdot}(R)$, for which the mapping spaces are Kan complexes; let $\calC = \sNerve( C_{\bigdot}(R) )$ be the associated $\infty$-category. Then $\calC$ admits all finite limits and colimits ($\calC$ is an example of a {\em stable} $\infty$-category; see \cite{DAG}). However, $\calC$ is idempotent complete if and only if every finitely generated projective $R$-module is stably free.
\end{example}

The purpose of this section is to define the notion of an {\it idempotent} in an $\infty$-category $\calC$, and to obtain a correspondence between idempotents and retracts in $\calC$.

\begin{definition}\index{not}{Idem+@$\Idem^{+}$}\index{not}{Idem@$\Idem$}\index{not}{Ret@$\Ret$}
The simplicial set $\Idem^{+}$ is defined as follows: 
for every nonempty, finite, linearly ordered set $J$, $\Hom_{\sSet}(\Delta^{J}, \Idem^{+})$ can be identified with the set of pairs $(J_0, \sim)$, where $J_0 \subseteq J$
and $\sim$ is an equivalence relation on $J_0$ which satisfies the following condition:
\begin{itemize}
\item[$(\ast)$] Let $i \leq j \leq k$ be elements of $J$ such that $i,k \in J_0$, and
$i \sim k$. Then $j \in J_0$, and $i \sim j \sim k$.
\end{itemize}

Let $\Idem$ denote the simplicial subset of $\Idem^{+}$, corresponding to those pairs
$(J_0, \sim)$ such that $J=J_0$. Let $\Ret \subseteq \Idem^{+}$ denote the simplicial subset corresponding to those pairs $(J_0, \sim)$ such that the quotient $J_0 / \sim$ has at most one element.
\end{definition}

\begin{remark}
The simplicial set $\Idem$ has exactly one nondegenerate simplex in each dimension $n$ (corresponding to the equivalence relation $\sim$ on $\{0, 1, \ldots, n\}$ given by
$( i \sim j) \Leftrightarrow (i = j)$ ), and the set of nondegenerate simplices of $\Idem$ is stable under passage to faces. In fact, $\Idem$ is characterized up to unique isomorphism by these two properties.
\end{remark}

\begin{definition}\label{diags}\index{gen}{idempotent!in an $\infty$-category}\index{gen}{retraction diagram!weak}\index{gen}{retraction diagram!small}
Let $\calC$ be an $\infty$-category.
\begin{itemize}
\item[$(1)$] An {\it idempotent} in $\calC$ is a map of simplicial sets
$\Idem \rightarrow \calC$. We will refer to $\Fun(\Idem,\calC)$ as the {\it $\infty$-category of
idempotents in $\calC$}.

\item[$(2)$] A {\it weak retraction diagram} in $\calC$ is a map of simplicial sets
$\Ret \rightarrow \calC$. We will refer to $\Fun(\Ret,\calC)$ as the {\it $\infty$-category of weak retraction diagrams in $\calC$}.

\item[$(3)$] A {\it strong retraction diagram} in $\calC$ is a map of simplicial sets
$\Idem^{+} \rightarrow \calC$. We will refer to $\Fun(\Idem^{+},\calC)$ as the {\it $\infty$-category of strong retraction diagrams in $\calC$}. 
\end{itemize}
\end{definition}

We now spell out Definition \ref{diags} in more concrete terms. We first observe that
$\Idem^{+}$ has precisely two vertices. Once of these vertices, which we will denote by $x$, belongs to $\Idem$, and the other, which we will denote by $y$, does not. The simplicial
set $\Ret$ can be identified with the quotient of $\Delta^{2}$ obtained by collapsing
$\Delta^{ \{0,2\} }$ to the vertex $y$. A weak retraction diagram $F: \Ret \rightarrow \calC$
in an $\infty$-category $\calC$ can therefore be identified with a $2$-simplex
$$ \xymatrix{ & X \ar[dr] & \\
Y \ar[ur] \ar[rr]^{\id_{Y}} & & Y }$$
where $X = F(x)$ and $Y = F(y)$. In other words, it is precisely the datum that we need in order to exhibit $Y$ as a retract of $X$ in the homotopy category $\h{\calC}$.

To give an idempotent $F: \Idem \rightarrow \calC$ in $\calC$, it suffices to specify the image under $F$ of each nondegenerate simplex of $\Idem$ in each dimension $n \geq 0$. Taking $n=0$, we obtain an object $X = F(x) \in \calC$.
Taking $n=1$, we get a morphism $f: X \rightarrow X$. Taking $n=2$, we get a $2$-simplex of $\calC$ corresponding to a diagram
$$ \xymatrix{ & X \ar[dr]^{f} & \\
X \ar[ur]^{f} \ar[rr]^{f} & & X }$$
which verifies the equation $f = f \circ f$ in the homotopy category $\h{\calC}$. Taking $n > 2$, we
get higher-dimensional diagrams which express the idea that $f$ is not only idempotent ``up to homotopy'', but ``up to coherent homotopy''.

The simplicial set $\Idem^{+}$ can be thought of as ``interweaving'' its simplicial subsets
$\Idem$ and $\Ret$, so that giving a strong retraction diagram $F: \Idem^{+} \rightarrow \calC$
is equivalent to giving a weak retraction diagram
$$ \xymatrix{ & X \ar[dr]^{r} & \\
Y \ar[ur]^{i} \ar[rr]^{\id_{Y}} & & Y }$$
together with a coherently idempotent map $f = i \circ r: X \rightarrow X$. Our next result makes precise the sense in which $f$ really is ``determined'' by $Y$.

\begin{lemma}\label{sturbie}
Let $J \subseteq \{ 0, \ldots, n\}$, and let $K \subseteq \Delta^n$ be the simplicial subset spanned by the nondegenerate simplices of $\Delta^n$ which do not contain $\Delta^{J}$. Suppose
that there exist $0 \leq i < j < k \leq n$ such that $i,k \in J$, $j \notin J$. Then the inclusion
$K \subseteq \Delta^n$ is inner anodyne.
\end{lemma}

\begin{proof}
Let $P$ denote the collection of all subset $J' \subseteq \{0, \ldots, n\}$ which contain $J \cup \{j\}$. Choose a linear ordering
$$ \{ J(1) \leq \ldots \leq J(m) \}$$
of $P$, with the property that if $J(i) \subseteq J(j)$, then $i \leq j$. Let
$$K(k) = K \cup \bigcup_{ 1\leq i \leq k} \Delta^{ J(i) }.$$
Note that there are pushout diagrams
$$ \xymatrix{ \Lambda^{J(i)}_{j} \ar[r] \ar[d] & \Delta^{J(i)} \ar[d] \\
K(i-1) \ar[r] & K(i). }$$
It follows that the inclusions $K(i-1) \subseteq K(i)$ are inner anodyne. Therefore
the composite inclusion $K = K(0) \subseteq K(m) = \Delta^n$ is also inner anodyne.
\end{proof}

\begin{proposition}
The inclusion $\Ret \subseteq \Idem^{+}$ is an inner anodyne map of simplicial sets.
\end{proposition}

\begin{proof}
Let $\Ret_{m} \subseteq \Idem^{+}$ be the simplicial subset defined so that
$(J_0, \sim): \Delta^{J} \rightarrow \Idem^{+}$ factors through $\Ret_m$ if and only if
the quotient $J_0 / \sim$ has cardinality $\leq m$. We observe that there is a pushout diagram
$$ \xymatrix{ K \ar[r] \ar[d] & \Delta^{2m} \ar[d] \\
\Ret_{m-1} \ar[r] & \Ret_{m} }$$
where $K \subseteq \Delta^{2m}$ denote the simplicial subset spanned by those faces
which do not contain $\Delta^{ \{1, 3, \ldots, 2m-1\} }$. If $m \geq 2$, Lemma \ref{sturbie} implies that the upper horizontal arrow is inner anodyne, so that the inclusion
$\Ret_{m-1} \subseteq \Ret_{m}$ is inner anodyne. The inclusion
$\Ret \subseteq \Idem^{+}$ can be identified with an infinite composition
$$ \Ret = \Ret_{1} \subseteq \Ret_{2} \subseteq \ldots $$
of inner anodyne maps, and is therefore inner anodyne.
\end{proof}

\begin{corollary}\label{gurgh}
Let $\calC$ be an $\infty$-category. Then the restriction map
$$ \Fun(\Idem^{+}, \calC) \rightarrow \Fun(\Ret, \calC)$$ from strong retraction diagrams
to weak retraction diagrams is a trivial fibration of simplicial sets. In particular,
every weak retraction diagram in $\calC$ can be extended to a strong retraction diagram.
\end{corollary}

We now study the relationship between strong retraction diagrams and idempotents in an $\infty$-category $\calC$. We will need the following lemma, whose proof is somewhat tedious.

\begin{lemma}
The simplicial set $\Idem^{+}$ is an $\infty$-category.
\end{lemma}

\begin{proof}
Suppose given $0 < i < n$ and a map $\Lambda^n_i \rightarrow \Idem^{+}$, corresponding
to a compatible family of pairs $\{ ( J_{k}, \sim_{k}) \}_{k \neq i}$, where $J_k \subseteq \{ 0, \ldots, k-1, k+1, \ldots, n \}$ and $\sim_k$ is an equivalence relation $J_{k}$ defining an element of
$\Hom_{\sSet}( \Delta^{ \{0, \ldots, k-1, k+1, \ldots, n\} }, \Idem^{+})$. Let $J = \bigcup J_{k}$, and
define a relation $\sim$ on $J$ as follows: if $a,b \in J$, then $a \sim b$ if and only if either
$$( \exists k \neq i) [ (a, b \in J_{k}) \wedge (a \sim_{k} b) ]$$
or
$$ ( a \neq b \neq i \neq a) \wedge (\exists c \in J_{a} \cap J_{b}) [ (a \sim_{b} c) \wedge (b \sim_{a} c)].$$
We must prove two things: that $(J, \sim) \in \Hom_{\sSet}(\Delta^n, \Idem^{+})$, and that the restriction of $(J, \sim)$ to $\{ 0, \ldots, k-1, k+1, \ldots, n\}$ coincides with $(J_k, \sim_k)$ for $k \neq i$.

We first check that $\sim$ is an equivalence relation. It is obvious that $\sim$ is reflexive and symmetric. Suppose that $a \sim b$ and that $b \sim c$; we wish to prove that $a \sim c$. There are several cases to consider:

\begin{itemize}
\item Suppose that there exists $j \neq i$, $k \neq i$ such that $a,b \in J_{j}$, 
$b,c \in J_{k}$, and $a \sim_{j} b \sim_{k} c$. If $a \neq k$, then also $a \in J_{k}$
and $a \sim_{k} b$, and we may conclude that $a \sim c$ by invoking the transitivity
of $\sim_{k}$. Therefore we may suppose that $a = k$. By the same argument, we may suppose
that $b = j$; we therefore conclude that $a \sim c$.

\item Suppose that there exists $k \neq i$ with $a,b \in J_k$, that $b \neq c \neq i \neq b$
and there exists $d \in J_{b} \cap J_{c}$ with $a \sim_{k} b \sim_{c} d \sim_{b} c$. If $a = b$ or $a=c$ there is nothing to prove; assume therefore that $a \neq b$ and $a \neq c$. Then $a \in J_{c}$
and $a \sim_{c} b$, so by transitivity $a \sim_{c} d$. Similarly, $a \in J_{b}$ and $a \sim_{b} d$
so that $a \sim_{b} c$ by transitivity. 

\item Suppose that $a \neq b \neq i \neq a$, $b \neq c \neq i \neq b$, and that there
exist $d \in J_{a} \cap J_{b}$ and $e \in J_{b} \cap J_{c}$ such that
$a \sim_{b} d \sim_{a} b \sim_{c} e \sim_{b} c$. It will suffice to prove that $a \sim_{b} c$. If $c = d$, this is clear; let us therefore assume that $c \neq d$.
By transitivity, it suffices to show that $d \sim_{b} e$. Since $c \neq d$, we have
$d \in J_{c}$ and $d \sim_{c} b$, so that $d \sim_{c} e$ by transitivity, and therefore
$d \sim_{b} e$.
\end{itemize}

To complete the proof that $(J, \sim)$ belongs to $\Hom_{\sSet}(\Delta^n, \Idem^{+})$, we must
show that if $a < b < c$, $a \in J$, $c \in J$, and $a \sim c$, then also $b \in J$ and
$a \sim b \sim c$. There are two cases to consider. Suppose first that there exists $k \neq j$ such that $a,c \in J_{k}$ and $a \sim_{k} c$. These relations hold for any $k \notin \{i,a,c\}$. If it is possible to choose $k \neq b$, then we conclude that $b \in J_{k}$ and $a \sim_{k} b \sim_{k} c$ as desired. 
Otherwise, we may suppose that the choices $k=0$ and $k=n$ are impossible, so that
$a = 0$ and $c = n$. Then $a < i < c$, so that $i \in J_{b}$ and $a \sim_{b} i \sim_{b} c$.
Without loss of generality we may suppose $b < i$. Then $a \sim_{c} i$, so that $b \in J_{c}$
and $a \sim_{c} b \sim_{c} i$ as desired. 

We now claim that $(J,\sim): \Delta^n \rightarrow \Idem^{+}$ is an extension of the original
map $\Lambda^n_i \rightarrow \Idem^{+}$. In other words, we claim that for $k \neq i$, 
$J_{k} = J \cap \{0, \ldots, k-1, k+1, \ldots, n\}$ and $\sim_{k}$ is the restriction of $\sim$ to
$J_{k}$. The first claim is obvious. For the second, let us suppose that $a,b \in J_{k}$ and
$a \sim b$. We wish to prove that $a \sim_{k} b$. It will suffice to prove that $a \sim_{j} b$ for any
$j \notin \{i, a, b\}$. Since $a \sim b$, either such a $j$ exists, or $a \neq b \neq i \neq a$
and there exists $c \in J_{a} \cap J_{b}$ such that $a \sim_{b} c \sim_{a} b$. If there exists
$j \notin \{a,b,c,i\}$, then we conclude that $a \sim_{j} c \sim_{j} b$ and hence $a \sim_{j} b$
by transitivity. Otherwise, we conclude that $c = k \neq i$ and that $0,n \in \{a,b,c\}$. 
Without loss of generality, $i < c$; thus $0 \in \{a,b\}$ and we may suppose without loss of generality that $a < i$. Since $a \sim_{b} c$, we conclude that $i \in J_{b}$ 
and $a \sim_{b} i \sim_{b} c$. Consequently,
$i \in J_{a}$ and $i \sim_{a} c \sim_{a} b$, so that $i \sim_{a} b$ by transitivity
and therefore $i \sim_{c} b$. We now have $a \sim_{c} i \sim_{c} b$ so that $a \sim_{c} b$
as desired.
\end{proof}

\begin{remark}
It is clear that $\Idem \subseteq \Idem^{+}$ is the full simplicial subset spanned by the vertex $x$, and therefore an $\infty$-category as well.
\end{remark}

According to Corollary \ref{gurgh}, every weak retraction diagram
$$ \xymatrix{ & X \ar[dr] & \\
Y \ar[ur] \ar[rr]^{\id_{Y}} & & Y }$$
in an $\infty$-category $\calC$ can be extended to a strong retraction diagram $F: \Idem^{+} \rightarrow \calC$, which restricts to give an idempotent in $\calC$. Our next goal is to show that
$F$ is canonically determined by the restriction $F| \Idem$.

Our next result expresses the idea that if an idempotent in $\calC$ arises in this manner, then $F$ is essentially unique.

\begin{lemma}\label{streaka}
The $\infty$-category $\Idem$ is weakly contractible.
\end{lemma}

\begin{proof}
An explicit computation shows that the topological space $|\Idem|$ is connected, simply connected, and has vanishing homology in degrees greater than zero. (Alternatively, we can deduce this from Proposition \ref{slanger} below.) 
\end{proof}

\begin{lemma}\label{linkdink}
The inclusion $\Idem \subseteq \Idem^{+}$ is a cofinal map of simplicial sets.
\end{lemma}

\begin{proof}
According to Theorem \ref{hollowtt}, it will suffice to prove that the simplicial sets
$\Idem_{x/}$ and $\Idem_{y/}$ are weakly contractible. The simplicial set $\Idem_{x/}$ is an $\infty$-category with an initial object, and therefore weakly contractible. The projection
$\Idem_{y/} \rightarrow \Idem$ is an isomorphism, and $\Idem$ is weakly contractible by Lemma \ref{streaka}.
\end{proof}

\begin{proposition}\label{autokan}
Let $\calC$ be an $\infty$-category, and let $F: \Idem^{+} \rightarrow \calC$ be a strong retraction diagram. Then $F$ is a left Kan extension of $F| \Idem$.
\end{proposition}

\begin{remark}
Passing to opposite $\infty$-categories, it follows that a strong retraction diagram $F: \Idem^{+} \rightarrow \calC$ is also a {\em right} Kan extension of $F|\Idem$.
\end{remark}

\begin{proof}
We must show that the induced map
$$ (\Idem_{/y})^{\triangleright} \rightarrow (\Idem^{+}_{/y})^{\triangleright} \stackrel{G}{\rightarrow} \Idem^{+}
\stackrel{F}{\rightarrow} \calC$$
is a colimit diagram. Consider the commutative diagram
$$ \xymatrix{ \Idem_{/y} \ar[r] \ar[d] & \Idem^{+}_{/y} \ar[d] \\
\Idem \ar[r] & \Idem^{+}. }$$
The lower horizontal map is cofinal by Lemma \ref{linkdink}, and the vertical maps are isomorphisms: therefore the upper horizontal map is also cofinal. Consequently, it will suffice to prove that $F \circ G$ is a colimit diagram, which is obvious.
\end{proof}

We will say that an idempotent $F: \Idem \rightarrow \calC$ in an $\infty$-category
$\calC$ is {\it effective} if it extends to a map $\Idem^{+} \rightarrow \calC$. According to Lemma \ref{kan2}, $F$ is effective if and only if it has a colimit in $\calC$. We will say that $\calC$ is {\it idempotent complete} if every idempotent in $\calC$ is effective.\index{gen}{idempotent complete}\index{gen}{idempotent!effective}

\begin{corollary}
Let $\calC$ be an $\infty$-category, and let $\calD \subseteq \Fun(\Idem,\calC)$ be the full subcategory spanned by the effective idempotents in $\calC$. The restriction map
$\Fun(\Idem^{+},\calC) \rightarrow \calD$ is a trivial fibration. In particular, if
$\calC$ is idempotent complete, then we have a diagram
$$ \Fun(\Ret,\calC) \leftarrow \Fun(\Idem^{+}, \calC) \rightarrow \Fun(\Idem,\calC)$$
of trivial fibrations.
\end{corollary}

\begin{proof}
Combine Proposition \ref{autokan} with Proposition \ref{lklk}.
\end{proof}

By definition, an $\infty$-category $\calC$ is idempotent complete if and only if every idempotent
$\Idem \rightarrow \calC$ has a colimit. In particular, if $\calC$ admits all small colimits, then it is idempotent complete. As we noted above, this is not necessarily true if $\calC$ admits only finite colimits. However, it turns out that filtered colimits do suffice: this assertion is not entirely obvious, since the $\infty$-category $\Idem$ itself is not filtered.

\begin{proposition}\label{slanger}
Let $A$ be a linearly ordered set with no largest element. Then there exists a cofinal map
$p: \Nerve(A) \rightarrow \Idem$.
\end{proposition}

\begin{proof}
Let $p: \Nerve(A) \rightarrow \Idem$ be the unique map which carries nondegenerate simplices to nondegenerate simplices. Explicitly, this map carries a simplex $\Delta^{J} \rightarrow \Nerve(A)$ corresponding to a map $s: J \rightarrow A$ of linearly ordered sets to the equivalence relation
$( i \sim j) \Leftrightarrow ( s(i) = s(j) )$. We claim that $p$ is cofinal. According to Theorem \ref{hollowtt}, it will suffice to show that the fiber product $\Nerve(A) \times_{\Idem} \Idem_{x/}$ is weakly contractible. We observe that $\Nerve(A) \times_{ \Idem} \Idem_{x} \simeq \Nerve(A')$, where
$A'$ denote the set $A \times \{0,1\}$ equipped with the partial ordering
$$ (\alpha, i) < (\alpha', j) \Leftrightarrow ( j = 1) \wedge ( \alpha < \alpha' ).$$

For each $\alpha \in A$, let $A_{< \alpha} = \{ \alpha' \in A: \alpha' < \alpha \}$ and let
$$A'_{\alpha} = \{ (\alpha',i) \in A' : (\alpha' < \alpha) \vee ( (\alpha',i) = (\alpha,1) ) \}.$$
By hypothesis, we can write
$A$ as a filtered union $\bigcup_{\alpha \in A} A_{< \alpha}$. It therefore suffices to prove
that for each $\alpha \in A$, the map
$$f: \Nerve(A_{< \alpha}) \times_{\Idem} \Idem_{x/} \rightarrow \Nerve(A) \times_{\Idem} \Idem_{x/}$$
has a nullhomotopic geometric realization $|f|$. But this map factors through
$\Nerve(A'_{\alpha})$, and $|\Nerve(A'_{\alpha})|$ is contractible because $A'_{\alpha}$ has a largest element.
\end{proof}

\begin{corollary}\label{swwe}
Let $\kappa$ be a regular cardinal, and suppose that $\calC$ is an $\infty$-category which admits $\kappa$-filtered colimits. Then $\calC$ is idempotent complete.
\end{corollary}

\begin{proof}
Apply Proposition \ref{slanger} to the linearly ordered set consisting of all ordinals 
less than $\kappa$ (and observe that this linearly ordered set is $\kappa$-filtered).
\end{proof}
