
\section{Paracompact Spaces}\label{paracompactness}

\setcounter{theorem}{0}

Let $X$ be a topological space and $G$ an abelian group. There are
many different definitions for the cohomology group $\HH^n(X;G)$;
we will single out three of them for discussion here. First of
all, we have the singular cohomology groups $\HH^n_{\text{sing}}(X;G)$,
which are defined to be cohomology of a chain complex of $G$-valued
singular cochains on $X$. An alternative is to regard $\HH^n( \bigdot,
G)$ as a representable functor on the homotopy category of
topological spaces, so that $\HH^n_{\text{rep}}(X;G)$ can be identified with the
set of homotopy classes of maps from $X$ into an Eilenberg-MacLane
space $K(G,n)$. A third possibility is to use the sheaf cohomology
$\HH^n_{\text{sheaf}}(X; \underline{G})$ of $X$ with coefficients
in the constant sheaf $\underline{G}$ on $X$.

If $X$ is a sufficiently nice space (for example, a CW complex),
then these three definitions give the same result. In general, however,
all three give different answers. The singular cohomology of $X$ is defined using continuous maps from $\Delta^k$ into $X$, and is useful only when there is a good supply of such maps.
Similarly, the cohomology group $\HH^{n}_{\text{rep}}(X;G)$ is defined using continuous maps
from $X$ to a simplicial complex, and is useful only when there is a good supply of
real-valued functions on $X$. However, the sheaf cohomology of $X$ seems to be
a good invariant for arbitrary spaces: it has excellent formal
properties and gives sensible answers in situations where the other definitions break down (such as the \'{e}tale topology of algebraic varieties).

We will take the position that the sheaf cohomology of a
space $X$ is the correct answer in all cases. It is then natural to ask
for conditions under which the other definitions of cohomology
give the same answer. We should expect this to be true for
singular cohomology when there are many continuous functions {\em
into $X$}, and for Eilenberg-MacLane cohomology when there are
many continuous functions {\em out of} $X$. It seems that the
latter class of spaces is much larger than the former: it
includes, for example, all paracompact spaces, and consequently
for paracompact spaces one can show that the sheaf cohomology
$\HH^n_{\text{sheaf}}(X;G)$ coincides with the Eilenberg-MacLane
cohomology $\HH^n_{\text{rep}}(X;G)$. Our goal in this section is to prove a generalization of the preceding statement to the setting of nonabelian cohomology (Theorem \ref{nice} below; see also Theorem \ref{main} for the case where the coefficient system $G$ it not assumed to be constant).

As we saw in \S \ref{versus}, we can associate to every topological space $X$ an $\infty$-topos $\Shv(X)$ of sheaves (of spaces) on $X$. Moreover, given a continuous map $p: X \rightarrow Y$ of topological spaces, $p^{-1}$ induces a map from the category of open subsets of $Y$ to the category of open subsets of $X$. Composition with $p^{-1}$ induces a geometric morphism $p_{\ast}: \Shv(X) \rightarrow \Shv(Y)$. 

Fix now a topological space $X$ and let $p: X \rightarrow \ast$ denote the projection from $X$ to a point. Let $K$ be a Kan complex, which we may identify with an object of $\SSet \simeq \Shv(\ast)$. 
Then $p^{\ast} K \in \Shv(X)$ may be regarded as the constant sheaf on $X$ having
value $K$, and $p_{\ast} p^{\ast} K \in \SSet$ as the space of global sections of $p^{\ast} K$. 
Let $|K|$ denote the geometric realization of $K$ (a topological space), and let
$[ X, |K| ]$ denote the {\em set} of homotopy classes of maps from $X$ into $|K|$. The main goal of this section is to prove the following:

\begin{theorem}\label{nice}
If $X$ is paracompact, then there is a canonical bijection
$$ \phi: [X, |K| ] \rightarrow \pi_0(p_{\ast} p^{\ast} K).$$
\end{theorem}

\begin{remark}
In fact, the map $\phi$ exists without the assumption that $X$ is paracompact: the construction in general can be formally reduced to the paracompact case, since the universal example $X = |K|$ is paracompact. However, in the case where $X$ is not paracompact, the map $\phi$ is not necessarily bijective.
\end{remark}

Our first step in proving Theorem \ref{nice} is to realize the space of maps
from $X$ into $|K|$ as a mapping space in an appropriate simplicial category
of {\it spaces over $X$}. In \S \ref{para1}, we define this category
and endow it with a (simplicial) model structure. We may therefore extract an underlying $\infty$-category $\Nerve(\Top^{\degree}_{/X})$. 

Our next goal is to construct an equivalence between $\Nerve(\Top_{/X}^{\degree})$ and the $\infty$-topos $\Shv(X)$ of sheaves of spaces on $X$ (a very similar comparison result has been obtained by 
To\"{e}n; see \cite{toen2}). To prove this, we will attempt to realize $\Nerve(\Top_{/X}^{\degree})$ as a localization of a certain $\infty$-category of presheaves. We will give an explicit description of the relevant localization in \S \ref{para2}, and show that it is equivalent to $\Nerve(\Top_{/X}^{\degree})$ in \S \ref{para3}. 
In \S \ref{dooky}, we will deduce Theorem \ref{nice} as a corollary of this more general comparison result. We conclude with \S \ref{shapesec}, in which we apply our results to obtain a reformulation of classical shape theory in the language of $\infty$-topoi.

\subsection{Some Point-Set Topology}

Let $X$ be a paracompact topological space. In order to prove
Theorem \ref{nice}, we will need to understand the homotopy
theory of presheaves on $X$. We then encounter the following
technical obstacle: an open subset of a paracompact space need not
be paracompact. Because we wish to deal only with paracompact
spaces, it will be convenient to restrict our attention to
presheaves which are defined only with respect to a particular
basis $\calB$ for $X$ consisting of paracompact open sets. The existence of a well-behaved basis is guaranteed by the following result:

\begin{proposition}\label{gooffy}
Let $X$ be a paracompact topological space and $U$ an open subset of $X$.
The following conditions are equivalent:
\begin{itemize}
\item[$(i)$] There exists a continuous function $f: X \rightarrow [0,1]$ such that
$U = \{ x \in X: f(x) > 0 \}$.
\item[$(ii)$] There exists a sequence of closed subsets 
$\{ K_{n} \subseteq X \}_{n \geq 0}$ such that each $K_{n+1}$ contains an open neighborhood of
$K_{n}$, and $U = \bigcup_{n \geq 0} K_n$.
\item[$(iii)$] There exists a sequence of closed subsets $\{ K_{n} \subseteq X \}_{n \geq 0}$ such that
$U = \bigcup_{n \geq 0} K_n$.
\end{itemize}
Let $\calB$ denote the collection of all open subsets of $X$ which satisfy these conditions. Then:
\begin{itemize}
\item[$(1)$] The elements of $\calB$ form a basis for the topology of
$X$. 

\item[$(2)$] Each element of $\calB$ is paracompact. 

\item[$(3)$] The
collection $\calB$ is stable under finite intersections (in particular, $X \in \calB$).

\item[$(4)$] The empty set $\emptyset$ belongs to $\calB$.
\end{itemize}
\end{proposition}

\begin{remark}
A subset of $X$ which can be written as a countable union of closed subsets of $X$ is called an\index{not}{Fsigma@$F_{\sigma}$}
{\it $F_{\sigma}$}-subset of $X$. Consequently, the basis $\calB$ for the topology of $X$ appearing in Proposition \ref{gooffy} can be characterized as the collection of open $F_{\sigma}$-subsets of $X$.
\end{remark}

\begin{remark}
If the topological space $X$ admits a metric $d$, then {\em every} open subset $U \subseteq X$
belongs to the basis $\calB$ of Proposition \ref{gooffy}. Indeed, we may assume without loss of generality that the diameter of $X$ is at most $1$ (adjusting the metric if necessary), in which case the function
$$ f(x) = d(x, X-U) = \inf_{y \notin U} d(x,y)$$
satisfies condition $(i)$.
\end{remark}

\begin{proof}
We first show that $(i)$ and $(ii)$ are equivalent. If $(i)$ is satisfied, then the closed subsets
$K_n = \{ x \in X: f(x) \geq \frac{1}{n} \}$ satisfy the demands of $(ii)$. Suppose next that $(ii)$ is satisfied.
For each $n \geq 0$, let $G_{n}$ denote the closure of $X - K_{n+1}$, so that $G_n \cap K_{n} = \emptyset$. It follows that there exists a continuous function $f_{n}: X \rightarrow [0, 1]$
such that that $f_n$ vanishes on $G_n$ and the restriction of $f$ to $K_{n}$ is the constant function taking the value $1$. Then the function $f = \Sum_{n > 0} \frac{f_n}{2^n}$ has the property required by $(i)$.

We now prove that $(ii) \Leftrightarrow (iii)$. The implication $(ii) \Rightarrow (iii)$ is obvious.
For the converse, suppose that $U = \bigcup_{n} K_n$, where the $K_n$ are closed subsets
of $X$. We define a new sequence of closed subsets $\{ K'_{n} \}_{n \geq 0}$ by induction as follows.
Let $K'_0 = K_0$. Assuming that $K'_n$ has already been defined, let $V$ and $W$ be
disjoint open neighborhoods of the closed sets $K'_{n} \cup K_{n+1}$ and $X-U$, respectively (the existence of such neighborhoods follows from the assumption that $X$ is paracompact; in fact, it would suffice to assume that $X$ is normal), and define $K'_{n+1}$ to be the closure of $V$. It is then
easy to see that the sequence of closed sets $\{ K'_{n} \}_{n \geq 0}$ satisfies the requirements of $(ii)$.

We now verify properties $(1)$ through $(4)$ of the collection of open sets $\calB$. Assertions $(3)$ and $(4)$ are obvious. To prove $(1)$, consider an arbitrary point $x \in X$ and an open set $U$ containing $x$. Then the closed sets $\{x\}$ and $X - U$ are disjoint, so there exists a continuous function
$f: X \rightarrow [0,1]$ supported on $U$ such that $f(x) = 1$. Then $U' = \{ y \in X: f(y) > 0$ is
an open neighborhood of $x$ contained in $U$, and $U' \in \calB$.

It remains to prove $(2)$. Let $U \in \calB$; we wish to prove that $U$ is paracompact. Write
$U = \bigcup_{n \geq 0} K_n$, where each $K_{n}$ is a closed subset of $X$ containing a neighborhood of $K_{n-1}$ (by convention, we set $K_{n} = \emptyset$ for $n < 0$). 
Let $\{ U_{\alpha} \}$ be an open covering of $X$. Since
each $K_n$ is paracompact, we can choose a locally finite covering
$\{ V_{\alpha, n} \}$ of $K_n$ which refines $\{ U_{\alpha} \cap K_n \}$. Let
$V^{0}_{\alpha,n}$ denote the intersection of $V_{\alpha,n}$ with 
the interior of $K_n$, and let $W_{\alpha,n} = V^{0}_{\alpha,n} \cap (X - K_{n-2})$. 
Then $\{ W_{\alpha, n} \}$ is a locally finite open covering of $X$ which refines
$\{ U_{\alpha} \}$.
\end{proof}

Let $X$ be a paracompact topological space, and let $\calB$ be the basis constructed in
Proposition \ref{gooffy}. Then $\calB$ can be viewed as a category with finite limits, and
is equipped with a natural Grothendieck topology. To simplify the notation, we will let
$\Shv(\calB)$ denote the $\infty$-topos $\Shv( \Nerve(\calB))$. Note that because
$\Nerve(\calB)$ is the nerve of a partially ordered set, the $\infty$-topos $\Shv(\calB)$ is
$0$-localic. Moreover, the corresponding locale $\Sub(\bf{1})$ of subobjects of
the final object ${\bf 1} \in \Shv(\calB)$ is isomorphic to the lattice of open subsets of $X$. It follows that the restriction map $\Shv(X) \rightarrow \Shv(\calB)$ is an equivalence of $\infty$-topoi.

\begin{warning}
Let $X$ be a topological space and $\calB$ a basis of $X$, regarded as a partially ordered set
with respect to inclusions. Then $\calB$ inherits a Grothendieck topology, and we can define
$\Shv(\calB)$ as above. However, the induced map $\Shv(X) \rightarrow \Shv(\calB)$ is generally not an equivalence of $\infty$-categories: this requires the assumption that $\calB$ is stable under finite intersections. In other words, a sheaf (of spaces) on $X$ generally {\em cannot} be recovered by knowing its sections on a basis for the topology of $X$; see Counterexample \ref{spacerk}.
\end{warning}

\subsection{Spaces over $X$}\label{para1}

Let $X$ be a topological space with a specified basis $\calB$, fixed throughout this section.
We wish to study the homotopy theory of {\it spaces over $X$}; that is, spaces $Y$ equipped with a map $p: Y \rightarrow X$.
We should emphasize that we do not wish to assume that the map $p$ is a fibration, or that $p$ is equivalent to a fibration in any reasonable sense: we are imagining that $p$ encodes a {\it sheaf of spaces} on $X$, and we do not wish to impose any condition of local triviality on this sheaf.

Let $\Top$\index{not}{Top@$\Top$} denote the category of topological spaces, and $\Top_{/X}$ the category of topological spaces mapping to $X$. For each $p: Y \rightarrow X$ and every open subset $U \subseteq X$,
we define a simplicial set $\Sing_{X}(Y,U)$ by the formula
$$ \Sing_{X}(Y,U)_{n} = \Hom_{X} ( U \times |\Delta^n|, Y )\index{not}{Sing_X@$\Sing_X$}.$$
Face and degeneracy maps are defined in the obvious way. We note the simplicial set $\Sing_X(Y,U)$ is {\em always} a Kan complex. We will simply write $\Sing_{X}(Y)$ to denote the simplicial presheaf on $X$ given by
$$ U \mapsto \Sing_{X}(Y,U).$$

\begin{proposition}\label{qur}\index{gen}{model category!of spaces over $X$}
There exists a model structure on the category $\Top_{/X}$, uniquely determined by the following properties:
\begin{itemize}
\item[$(W)$] A morphism $$ \xymatrix{ Y \ar[rr] \ar[dr]^{p} & & Z \ar[dl]^{q} \\
& X & }$$ is a {\it weak equivalence} if and only if, for every $U \subseteq X$ belonging to $\calB$, the induced map $\Sing_X(Z,U)_{\bigdot} \rightarrow \Sing_X(Y,U)_{\bigdot}$ is a homotopy equivalence of Kan complexes.
\item[$(F)$] A morphism $$ \xymatrix{ Y \ar[rr] \ar[dr]^{p} & & Z \ar[dl]^{q} \\
& X & }$$ is a {\it fibration} if and only if, for every $U \subseteq X$ belonging to $\calB$, the induced map $\Sing_X(Z,U)_{\bigdot} \rightarrow \Sing_X(Y,U)_{\bigdot}$ is a Kan fibration.
\end{itemize}
\end{proposition}

\begin{remark}
The model structure on $\Top_{/X}$ described in Proposition \ref{qur} depends on the chosen basis $\calB$ for $X$, and not only on the topological space $X$ itself.
\end{remark}

\begin{proof}[Proof of Proposition \ref{qur}]
The proof uses the theory of {\em cofibrantly generated} model categories; we give a sketch and refer the reader to \cite{hirschhorn} for more details. We will say that a morphism $Y \rightarrow Z$ in $\Top_{/X}$ is a cofibration if it has the left lifting property with respect to every trivial fibration in $\Top_{/X}$. 

We begin by observing that a map $Y \rightarrow Z$ in $\Top_{/X}$ is a fibration if and only if it has the right lifting property with respect to every inclusion $U \times \Lambda^n_i \subseteq U \times \Delta^n$, where $0 \leq i \leq n$ and $U$ is in $\calB$. Let $\calI$ denote the weakly saturated class of morphisms in $\Top_{/X}$ generated by these inclusions. Using the small object argument, one can show that every morphism $Y \rightarrow Z$ in $\Top_{/X}$ admits a factorization
$$ Y \stackrel{f}{\rightarrow} Y' \stackrel{g}{\rightarrow} Z$$
where $f$ belongs to $\calI$ and $g$ is a fibration. (Although the objects in $\Top_{/X}$ are not generally small, one can still apply the small object argument since they are small {\it relative} to the class $\calI$ of morphisms: see \cite{hirschhorn}).

Similarly, a map $Y \rightarrow Z$ is a trivial fibration if and only if it has the right lifting property with respect to every inclusion $U \times |\bd \Delta^n| \subseteq U \times |\Delta^n|$, where $U \in \calB$. 
Let $\calJ$ denote the weakly saturated class of morphisms generated by these inclusions: then every
morphism $Y \rightarrow Z$ admits a factorization 
$$ Y \stackrel{f}{\rightarrow} Y' \stackrel{g}{\rightarrow} Z$$
where $f$ belongs to $\calJ$ and $g$ is a trivial fibration.

The only nontrivial point to verify is that every morphism which belongs to $\calI$ is a trivial cofibration; once this is established, the axioms for a model category follow formally. Since it is clear that $\calI$ is contained in $\calJ$, and that $\calJ$ consists of cofibrations, it suffices to show that every morphism in $\calI$ is a weak equivalence. To prove this, let us consider the class $\calK$ of all closed immersions $k: Y \rightarrow Z$ in $\Top_{/X}$ such that there exist functions $\lambda: Z \rightarrow [0, \infty)$ and $h: Z \times [0, \infty) \rightarrow Z$ such that $k(Y) = \lambda^{-1} \{0\}$, $h(z,0)=z$, and $h(z, \lambda(z)) \in k(Y)$. Now we make the following observations:
\begin{itemize}
\item[$(1)$] Every inclusion $U \times |\Lambda^n_i| \subseteq U \times |\Delta^n|$ belongs to $\calK$.
\item[$(2)$] The class $\calK$ is weakly saturated; consequently, $\calI \subseteq \calK$.
\item[$(3)$] Every morphism $k: Y \rightarrow Z$ which belongs to $\calK$ is a homotopy equivalence in $\Top_{/X}$, and is therefore a weak equivalence.
\end{itemize}
\end{proof}

The category $\Top_{/X}$ is naturally tensored over simplicial sets, if we define $Y \otimes \Delta^n = Y \times |\Delta^n|$ for $Y \in \Top_{/X}$. This induces a simplicial structure on $\Top_{/X}$, which is obviously compatible with the model structure of Proposition \ref{qur}.

We note that $\Sing_X$ is a (simplicial) functor from $\Top_{/X}$ to the category of simplicial presheaves on $\calB$ (here we regard $\calB$ as a category whose morphisms are given by inclusions of open subsets of $X$). We regard $\Set_{\Delta}^{\calB^{op}}$ as a simplicial model category, via the {\it projective} model structure described in \S \ref{quasilimit3}.
By construction, $\Sing_X$ preserves fibrations and trivial fibrations. Moreover, the functor $\Sing_X$ has a left adjoint $$ F \mapsto |F|_{X};$$
we will refer to this left adjoint as {\it geometric realization} (in the case where $X$ is a point, it coincides with the usual geometric realization functor from $\sSet$ to the category of topological spaces).
The functor $|F|_{X}$ is determined by the property that $|F_U|_{X} \simeq U$ if $F_U$ denotes the presheaf (of sets) represented by $U$, and the requirement that geometric realization commutes with colimits and with tensor products by simplicial sets.

We may summarize the situation as follows:

\begin{proposition}\label{exquill}
The adjoint functors $(||_{X}, \Sing_X)$ determine a $($simplicial$)$ Quillen adjunction between
$\Top_{/X}$ $($with the model structure of Proposition \ref{qur} $)$ and $\Set_{\Delta}^{\calB^{op}}$ $($with the projective model structure$)$.
\end{proposition}

\subsection{The Sheaf Condition}\label{para2}

Let $X$ be a topological space and $\calB$ a basis for the topology
of $X$ which is stable under finite intersections. Let $\bfA$ denote the category
of $\Set_{\Delta}^{\calB^{op}}$ of simplicial presheaves on $\calB$; we regard
$\bfA$ as a model category with respect to the {\em projective} model structure defined in \S \ref{quasilimit3}. According to Proposition \ref{othermod}, the $\infty$-category $\sNerve (\bfA^{\degree})$ associated to $\bfA$ is equivalent to the $\infty$-category 
$\calP(\calB) = \calP( \Nerve(\calB))$\index{not}{PcalcalB@$\calP(\calB)$} of presheaves on $\calB$. In particular, the homotopy category
$\h{\calP( \Nerve(\calB))}$ is equivalent to the homotopy category $\h{\bfA}$ ( the category obtained from $\bfA$ by formally inverting all weak equivalences of simplicial presheaves). The $\infty$-category $\Shv(\calB)$ is a reflective subcategory of $\calP(\Nerve(\calB))$. Consequently, we may identify the homotopy category $\h{\Shv(\calB)}$ with a reflective subcategory of $\h{\bfA}$. We will say that a simplicial presheaf $F: \calB^{op} \rightarrow \sSet$ is a {\it sheaf} if it belongs to this reflective subcategory. The purpose of this section is to obtain an explicit criterion which will allow us to test whether or not a given simplicial presheaf $F: \calB^{op} \rightarrow \sSet$ is a sheaf.

\begin{warning}
The condition that a simplicial presheaf $F: \calB^{op} \rightarrow \sSet$ be a sheaf, in the sense defined above, is generally unrelated to the condition that $F$ be a simplicial object in the category of sheaves of sets on $X$ (though these two notions do agree in the special case where the simplicial presheaf $F$ takes values in {\em constant} simplicial sets).
\end{warning}

Let $j: \Nerve(\calB) \rightarrow \calP(\calB)$ be the Yoneda embedding. By definition, an object
$F \in \calP(\calB)$ belongs to $\Shv(\calB)$ if and only if, for every $U \in \calB$ and every
monomorphism $i: U^{0} \rightarrow j(U)$ which corresponds to a {\em covering} sieve $\calU$ on $U$, the induced map
$$ \bHom_{ \calP(\calB)}( j(U), F) \rightarrow \bHom_{ \calP(\calB)}( U^0, F)$$
is an isomorphism in the homotopy category $\calH$. In order to make this condition explicit
in terms of simplicial presheaves, we note that $i: U^{0} \rightarrow j(U)$ can be identified with the inclusion $\chi_{\calU} \subseteq \chi_{U}$ of simplicial presheaves, where
$$\chi_{U}(V) = \begin{cases} \ast & \text{if } V \subseteq U \\
\emptyset & \text{otherwise.} \end{cases}$$
$$\chi_{\calU}(V) = \begin{cases} \ast & \text{if } V \in \calU \\
\emptyset & \text{otherwise.} \end{cases}$$
However, we encounter a technical issue: in order to extract the correct space of maps
$\bHom_{\calP(\calB)}( U^0, F)$, we need to select a {\em projectively cofibrant} model for $U^0$ in $\bfA$. In general, the simplicial presheaf $\chi_{\calU}$ defined above is not projectively cofibrant.
To address this problem, we will construct a new simplicial presheaf, equivalent to $\chi_{\calU}$, which has better mapping properties.

\begin{definition}
Let $\calU$ be a linearly ordered set equipped with a map $s: \calU \rightarrow \calB$.\index{not}{NCalU@$N_{\calU}$}
We define a simplicial presheaf $N_{\calU}: \calB^{op} \rightarrow \sSet$ as follows: for each $V \in \calB$, let $N_{\calU}(V)$ be the nerve of the linearly ordered set $\{ U \in \calU: V \subseteq s(U) \}$. 
$N_{\calU}$ may be viewed as a subobject of the constant presheaf $\underline{ \Delta^{\calU} }$ taking the value $\Nerve(\calU) = \Delta^{\calU}$. 
\end{definition}

\begin{remark}
The above notation is slightly abusive, in that $N_{\calU}$ depends not only on $\calU$, but on the map $s$ and on the linear ordering of $\calU$.  If the map $s$ is injective (as it will be in most applications), we will frequently simply identify $\calU$ with its image in $\calB$. In practice, $\calU$ will usually be a covering sieve on some object $U \in \calB$.
\end{remark}

\begin{remark}
The linear ordering of $\calU$ is {\em unrelated} to the partial ordering of $\calB$ by inclusion.
We will write the former as $\leq$ and the latter as $\subseteq$.
\end{remark}

\begin{example}
Let $\calU = \emptyset$. Then $N_{\calU} = \emptyset$.
\end{example}

\begin{example}
Let $\calU = \{ U\}$ for some $U \in \calB$, and let $s: \calU \rightarrow \calB$ be the inclusion. Then $N_{\calU} \simeq \chi_{U}$.
\end{example}

\begin{proposition}\label{murkminus}
Let $$ \xymatrix{ \calU \ar[dd]^{p} \ar[dr]^{s} & \\
& \calB \\
\calU' \ar[ur]^{s'} & }$$
be a commutative diagram, where $p$ is an order-preserving injection between linearly ordered sets. Then the induced map $N_{\calU} \rightarrow N_{\calU'}$ is a projective cofibration of simplicial presheaves.
\end{proposition}

\begin{proof}
Without loss of generality, we may identify $\calU$ with a linearly ordered subset of $\calU'$ via $p$. Choose a transfinite sequence of simplicial subsets of $N \calU'$ 
$$ K_0 \subseteq K_1 \subseteq \ldots$$
where $K_0 = N \calU$, $K_{\lambda} = \bigcup_{\alpha < \lambda} K_{\alpha}$ if $\lambda$ is a nonzero limit ordinal, and $K_{\alpha+1}$ is obtained from $K_{\alpha}$ by adjoining a single nondegenerate simplex (if such a simplex exists). For each ordinal $\alpha$, let
$F_{\alpha} \subseteq N_{\calU'}$ be defined by
$$ F_{\alpha}(V) = N_{\calU'}(V) \cap K_{\alpha} \subseteq \Nerve(\calU'). $$
Then $F_0 = N_{\calU}$, $F_{\lambda} = \colim_{\alpha < \lambda} F_{\alpha}$ when $\alpha$ is a nonzero limit ordinal, and $F_{\alpha} \simeq N_{\calU'}$ for $\alpha \gg 0$. It therefore suffices to show that each map $F_{\alpha} \rightarrow F_{\alpha+1}$ is a projective cofibration. If
$K_{\alpha} = K_{\alpha+1}$, this is clear; otherwise, we may suppose that $K_{\alpha+1}$ is obtained from $K_{\alpha}$ by adjoining a single nondegenerate simplex $\{ U_0 < U_1 < \ldots U_{n} \}$ of $\Nerve(\calU')$. Let $U = s'(U_0) \cap \ldots \cap s'(U_n) \in \calB$. 

Then there is a coCartesian square
$$ \xymatrix{ \chi_{U} \otimes \bd \Delta^n \ar[r] \ar[d] & \chi_U \otimes \Delta^n \ar[d] \\
F_{\alpha} \ar[r] & F_{\alpha+1}, }$$
The desired result now follows, since the upper horizontal arrow is clearly a projective cofibration.
\end{proof}

\begin{corollary}
Let $\calU$ be a linearly ordered set and $s: \calU \rightarrow \calB$ a map. Then the simplicial presheaf $N_{\calU} \in \Set_{\Delta}^{\calB^{op}}$ is projectively cofibrant.
\end{corollary}

Note that $N_{\calU}(V)$ is contractible if $V \subseteq s(U)$ for some $U \in \calU$, and empty otherwise. Consequently, we deduce:

\begin{corollary}
Let $\calU \subseteq \calB$ be a sieve, equipped with a linear ordering.
The unique map $N_{\calU} \rightarrow \chi_{\calU}$ is a weak equivalence of simplicial presheaves.
\end{corollary}

\begin{notation}
Let $\calU$ be a linearly ordered set equipped with a map $s: \calU \rightarrow \calB$.
For any simplicial presheaf $F: \calB^{op} \rightarrow \sSet$, we let
$F(\calU)$ denote the simplicial set $\bHom_{\bfA}( N_{\calU}, F)$.
\end{notation}

\begin{remark}
Let $U \in \calB$, $\calU = \{ U\}$ and $s: \calU \rightarrow \calB$ is the inclusion. Then $F( \calU ) = F(U)$. In general, we can think of $F( \calU)$ as a {\it homotopy limit} of $F(V)$ taken over
$V$ in the sieve generated by $s: \calU \rightarrow \calB$. To give a vertex of $F(\calU)$, we must give for each $U \in \calU$ a point of $F( sU)$; for every pair of objects $U, V \in \calU$ a path between the corresponding points in $F( sU \cap sV)$, and so forth.
\end{remark}

\begin{corollary}\label{critsheaf}
Let $F: \calB^{op} \rightarrow \Kan$ be a $($projectively fibrant$)$ simplicial presheaf on $\calB$. Then $F$ is a sheaf if and only if, for every $U \in \calB$ and every sieve $\calU$ that covers $U$, there
exists a linearly ordered set $\calU_0$ equipped with a map $\calU_0 \rightarrow \calU$, which generates $\calU$ as a sieve, such that the induced map
$F(U) \rightarrow F(\calU_0)$ is a weak homotopy equivalence of simplicial sets.
\end{corollary}

\begin{lemma}\label{partit}
Suppose that $U \subseteq X$ is paracompact, and let $\calU \subseteq \calB$ be a covering
of $U$. Choose a linear ordering of $\calU$. Then the natural map 
$\pi: |N_{\calU}|_{X} \rightarrow U$ is a homotopy equivalence in $\Top_{/X}$.
$($In other words, there exists a section $s: U \rightarrow N_{\calU}$ of $\pi$, such that
$s \circ \pi$ is fiberwise homotopic to the identity.$)$
\end{lemma}

\begin{proof}
Any partition of unity subordinate to the open cover $\calU$ gives rise to a section of $\pi$. 
To check that $s \circ \pi$ is fiberwise homotopic to the identity, use a ``straight line'' homotopy.
\end{proof}

\begin{proposition}\label{aese}
Let $X$ be a topological space, $\calB$ a basis for the topology of $X$. Assume that $\calB$ is stable under finite intersections and that each element of $\calB$ is paracompact. For
every continuous map of topological spaces $p: Y \rightarrow X$, the simplicial presheaf $\Sing_{X}(Y)$ of sections of $p$ is sheaf.
\end{proposition}

\begin{proof}
Let $F = \Sing_{X}(Y)$. We note that $F$ is a projectively fibrant simplicial presheaf on $\calB$.
By Corollary \ref{critsheaf}, it suffices to show that for every $U \in \calB$,
every covering $\calU$ of $U$, and every linear ordering on $\calU$, 
the natural map $F(U) \rightarrow F(\calU)$ is a homotopy equivalence of simplicial sets. In other words, it suffices to show that composition with the projection $\pi: N_{\calU} \rightarrow U$ induces a homotopy equivalence
$$ \bHom_{/X}( U, Y ) \rightarrow \bHom_{/X}( |N_{\calU}|_{X}, Y)$$
of simplicial sets. This follows immediately from Lemma \ref{partit}.
\end{proof}

\begin{remark}
Under the hypotheses of Proposition \ref{aese}, the object of $\Shv(X)$ corresponding
to the simplicial presheaf $\Sing_{X}(Y)$ is {\em not necessarily hypercomplete}. 
\end{remark}

\subsection{The Main Result}\label{para3}

Suppose that $X$ is a paracompact topological space and $\calB$ is the basis for the topology of $X$ described in Proposition \ref{gooffy}. Our main goal is to show that the composition of the adjoint functors
$$ F \mapsto \Sing_{X} |F|_{X}$$ may be identified with a ``sheafification'' of $F$, at least in the case where $F$ is a projectively cofibrant simplicial presheaf on $\calB$. 

In proving this, we have some flexibility regarding the choice of $F$: it will suffice to treat the question after replacing $F$ by a weakly equivalent simplicial presheaf $F'$, provided that $F'$ is also projectively cofibrant. Our first step is to make a particularly convenient choice for $F'$.

\begin{lemma}\label{grumppp}
Let $\calB$ be a partially ordered set $($via $\subseteq${}$)$ with a least element $\emptyset$, and $F: \calB^{op} \rightarrow \sSet$ be an arbitrary simplicial presheaf such that $F(\emptyset)$ is weakly contractible.

There
exists a $($linearly ordered$)$ set $V$ and a simplicial presheaf $F': \calB^{op} \rightarrow \sSet$ with the following properties:
\begin{itemize}
\item[$(1)$] There exists a monomorphism $F' \rightarrow \underline{\Delta^{V}}$ from $F'$ to the (constant)
simplicial presheaf $\underline{\Delta^{V}}$ on $\calB$ taking the value $\Delta^V$. (Recall that
$\Delta^V$ denotes the nerve of the linearly ordered set $V$.)

\item[$(2)$] For every finite subset $V_0 \subseteq V$, there exists $U \in \calB$ such that 
$U' \subseteq U$ if and only if $\Delta^{V_0} \subseteq F'(U') \subseteq \Delta^{V}$.

\item[$(3)$] As a simplicial presheaf on $\calB$, $F'$ is projectively cofibrant.

\item[$(4)$] In the homotopy category of $\Set_{\Delta}^{\calB^{op}}$, $F'$ and $F$ are equivalent to one another.

\end{itemize}

\end{lemma}

\begin{proof}
Without loss of generality, we may suppose that $F$ is (weakly) fibrant. We now build a ``cellular model'' of $F$. More precisely, we construct the following data:

\begin{itemize}
\item[$(A)$] A transfinite sequence of simplicial sets
$$ Y_0 \rightarrow Y_1 \rightarrow \ldots, $$
where $Y_{\alpha}$ is defined for all ordinals $< \alpha_0$.

\item[$(B)$] For each $\alpha < \alpha_0$, a subsheaf $F_{\alpha}$ of the constant presheaf
on $\calB$ taking the value $Y_{\alpha}$.

\item[$(C)$] A compatible family of maps $F_{\alpha} \rightarrow F$, so that we may regard $\{F_{\alpha} \}$ as a functor from the linearly ordered set $\{ \alpha: \alpha < \alpha_0 \}$ to
$(\sSet)^{\calB^{op}}_{/F}$. 

\item[$(D)$] For each $\alpha < \alpha_0$, there exists $U \in \calB$, $n \geq 0$, and compatible pushout diagrams
$$ \xymatrix{ \bd \Delta^n \ar[d] \ar@{^{(}->}[r] & \Delta^n \ar[d] \\
\colim_{ \beta < \alpha} Y_{\beta} \ar@{^{(}->}[r] & Y_{\alpha}, }$$
$$ \xymatrix{ \chi_{U} \times \bd \Delta^n \ar[d] \ar@{^{(}->}[r] & \chi_{U} \times \Delta^n \ar[d] \\
\colim_{ \beta < \alpha} F_{\beta} \ar@{^{(}->}[r] & F_{\alpha}, }$$
where $$ \chi_U(W) = \begin{cases} \ast & \text{if } W \subseteq U \\
\emptyset & \text{otherwise.} \end{cases} $$

\item[$(E)$] The canonical map $\colim_{\beta < \alpha_0} F_{\beta} \rightarrow F$
is a weak equivalence in $\Set_{\Delta}^{\calB^{op}}$.

\end{itemize}

The construction of this data is reasonably standard, and left to the reader.
Let $Y = \colim_{ \beta < \alpha_0} Y_{\beta}$. Let $Y''$ be the second barycentric subdivision of $Y$, so that $Y''$ may be identified with a simplicial complex (that is, a simplicial subset of $\Delta^{V}$ for some linearly ordered set $V$. For each $\alpha$, let $F''_{\alpha}$ denote
the result of applying the second barycentric subdivision functor to $F_{\alpha}$ termwise.
Let $F'' = \colim_{ \beta < \alpha_0 } F''_{\beta}$. Finally, we define $F'$ by the coCartesian square
$$ \xymatrix{ F''(\emptyset) \times \chi_{\emptyset} \ar[r] \ar[d] & \Delta^V \times \chi_{\emptyset} \ar[d] \\ F'' \ar[r] & F'. }$$

The simplicial presheaf $F'$ satisfies $(1)$ by construction. Properties $(2)$ and $(3)$ are reasonably clear (in fact, $(3)$ is a formal consequence of $(2)$). Condition $(4)$ holds for the simplicial presheaf $F''$ as a consequence of $(E)$, and the fact that there is a canonical
weak homotopy equivalence $K'' \rightarrow K$, for any simplicial set $K$ (see Variant \ref{baryvar}).
Moreover, the assumption that $F(\emptyset)$ is weakly contractible ensures that $(4)$ remains valid for the pushout $F'$.
\end{proof}

Before we can state the next lemma, let us introduce a bit of notation. Let $F: \calB^{op} \rightarrow \sSet$ be a simplicial presheaf. Then we let $|F|$ denote the presheaf of topological spaces on $\calB$ obtained by composing $F$ with the geometric realization functor; similarly, if 
$G$ is a presheaf of topological spaces on $\calB$, then we let $\Sing G$ denote the presheaf of simplicial sets obtained by composing $G$ with the functor $\Sing$. We note that there is a natural transformation
$$ \Sing |F| \rightarrow \Sing_{X} |F|_X.$$

\begin{lemma}\label{hardd}
Let $X$ be a topological space, $\calB$ the collection of open $F_{\sigma}$ subsets of 
$X$ (see Proposition \ref{gooffy}). Let $F: \calB^{op} \rightarrow \sSet$ be a 
projectively cofibrant simplicial presheaf, which is a sheaf $($that is, a fibrant model for $F$ satisfies
the criterion of Corollary \ref{critsheaf}$)$. 
Then the unit map $F \rightarrow \Sing_{X} |F|_{X}$ is an equivalence.
\end{lemma}

\begin{proof}
We note that the functor $F \mapsto \Sing |F|$ preserves weak equivalences in $F$, and the functor $F \mapsto \Sing_{X} |F|_{X}$ preserves weak equivalences between {\em projectively cofibrant}
presheaves $F$. Consequently, by Lemma \ref{grumppp}, we may suppose without loss of generality that there is a linearly ordered set $V$ and that $F$ is a subsheaf of the constant simplicial presheaf taking the value $\Delta^{V}$, such that $F(\emptyset) = \Delta^V$.

It will be sufficient to prove that
$$\Sing |F| \rightarrow \Sing_{X} |F|_X$$ is an equivalence: in other words, we wish to show that
$(\Sing |F|)(U) \rightarrow (\Sing_{X} |F|_X)(U)$ is a homotopy equivalence of Kan complexes, for every $U \in \calB$. Replacing $X$ by $U$, we can reduce to the problem of showing that
$$ p: (\Sing |F|)(X) \rightarrow (\Sing_X |F|_X)(X)$$
is a homotopy equivalence. 
It now suffices to show that
for every inclusion $K' \subseteq K$ of {\em finite} simplicial sets (that is, simplicial sets with only finitely many nondegenerate simplices), a commutative diagram
$$ \xymatrix{ K' \times \{0\}  \ar@{^{(}->}[d] \ar[r] & (\Sing |F|)(X) \ar[d] \\
K \times \{0\} \ar[r]^-{g} & ( \Sing_{X} |F|_{X} )(X) }$$
can be expanded to a commutative diagram
$$ \xymatrix{ (K' \times \Delta^1) \coprod_{ K' \times \{1\} } (K \times \{1\})  \ar@{^{(}->}[d] \ar[rr] & & (\Sing |F|)(X) \ar[d] \\
K \times \Delta^1 \ar[rr] & & ( \Sing_{X} |F|_{X} )(X). }$$
(In fact, it suffices to treat the case where $K' \subseteq K$ is the inclusion $\bd \Delta^n \subseteq \Delta^n$; however, this will result in no simplification in the following arguments.) 

Now let $\calB = \{ U_{\alpha} \}_{\alpha \in A}$, where $A$ is a linearly ordered set.
Since $F$ is assumed to be a sheaf on $\calB$, the equivalent presheaf $\Sing |F|$ is also a sheaf. 
Consequently, for any covering $\calU \subseteq \calB$ (and any linear ordering of $\calU$), the natural map $(\Sing |F|)(X) \rightarrow (\Sing |F|)(\calU)$ is an equivalence.
Likewise, by Proposition \ref{aese}, the map
$(\Sing_{X} |F|_X)(X) \rightarrow ( \Sing_{X} |F|_X )(\calU)$ is an equivalence. Consequently, it suffices to find a covering $\calU \subseteq \calB$ of $X$ and a diagram
$$ \xymatrix{ (K' \times \Delta^1) \coprod_{ K' \times \{1\} } (K \times \{1\})  \ar[d] \ar[rr] & & (\Sing |F|)(\calU) \ar[d] \\
K \times \Delta^1 \ar[rr]^{G} & & ( \Sing_{X} |F|_{X} )(\calU) }$$
which extends $g$.

Since $K$ is finite, the map $g: K \rightarrow (\Sing_X |F|_X)(X)$ may be identified with a continuous, fiber-preserving map $X \times |K| \rightarrow |F|_X$, which we will also denote by $g$.
By assumption, $F$ is a subsheaf of the constant presheaf taking the value $\Delta^{V}$; constantly, we may identify $|F|_X$ with a subspace of $\Delta^V_X = X \otimes \Delta^V$.
(We may identify $\Delta^V_{X}$ with the product
$X \times | \Delta^V |$ {\em as a set}, though it generally has a finer topology.) We may represent a point of $\Delta^V_{X}$ by an ordered pair $(x, q)$, where $x \in X$ and $q: V \rightarrow [0,1]$
has the property that $\{ v \in V: q(v) \neq 0\}$ is finite and 
$\Sum_{v \in V} q(v) = 1$. For each $v \in V$, we let $\Delta^V_{X,v}$ denote the {\em open} subset of $\Delta^V_{X}$ consisting of all pairs $(x,g)$ such that $q(v) > 0$; note that the sets
$\{ \Delta^V_{X,v}\}_{ v \in V}$ form an open cover of cover $\Delta^V_{X}$.
Consequently, the open sets $\{ g^{-1} \Delta^V_{X,v} \}_{v \in V}$ form an open cover of $X \times |K|$. Let $x$ be a point of $X$. The compactness of $|K|$ implies that there is a finite subset
$V_0 \subseteq V$, an open neighborhood $U_x$ of $X$ containing $x$, and an open covering
$\{ W_{x,v}: v \in V_0 \}$ of $|K|$, such that $g( U_x \times W_{x,v} ) \subseteq \Delta^V_{X,v}$.
Choose a partition of unity subordinate to the covering $\{ W_{x,v} \}$, thereby determining a map $f_{x}: |K| \rightarrow |\Delta^{V_0}|$. The open sets $\{ U_x \}$ cover $X$; since $X$ is paracompact, this covering has a locally finite refinement. Shrinking the $U_x$ if necessary, we may suppose that this refinement is given by $\{ U_{x} \}_{x \in X_0}$, and that each $U_{x}$ belongs to $\calB$. Let $\calU = \{ U_{x} \}_{x \in X_0}$, and choose a linear ordering of $\calU$.

We now define a new map
$g': K \rightarrow (\Sing |F|)(\calU)$. To do so, we must give, for every finite
$\calU_0 = \{ U_{x_0} < \ldots < U_{x_n} \} \subseteq \calU$, a map
$$ g'_{A_0}: |\Delta^{ \{x_0, \ldots, x_n\} }| \times |K| \rightarrow |F(U_{x_0} \cap \ldots \cap U_{x_n})| \subseteq |\Delta^V|,$$
which are required to satisfy some obvious compatibilities. 
Define $g'_{\calU_0}$ by the formula
$$ g'_{\calU_0}( \Sum \lambda_i x_i, z) = \Sum \lambda_i f_{x_i}(z).$$
It is clear that $g'_{\calU_0}$ is well-defined as a map from $| \Delta^{ \{x_0, \ldots, x_n \} } | \times |K|$ to
$| \Delta^{V} |$. We claim that, in fact, this map factors through $| F( U_{x_0} \cap \ldots \cal U_{x_n} )|$. Let $z \in |K|$, and consider the set $V' = \{ v \in V: (\exists 0 \leq i \leq n) [f_{x_i}(z)(v) \neq 0] \} \subseteq V$. Condition $(2)$ of Lemma \ref{grumppp} ensures that there exists $U \in \calB$ such that $\Delta^{V'} \subseteq F(U')$ if and only if
$U' \subseteq U$. We note that, for each $y \in U_{x_0} \cap \ldots \cap U_{x_n}$, we have $g(y,z)(v) \neq 0$ for $v \in V'$; it follows that $y \in U$. Consequently, we deduce that
$U_{\alpha_0} \cap \ldots \cap U_{\alpha_n} \subseteq U$, so that
$\Delta^{V'} \subseteq F( U_{x_0} \cap \ldots \cap U_{x_n} ).$ It follows that
$g'_{\calU_0}| \{z\} \times |\Delta^{\calU_0}|$ factors through $| F(U_{x_0} \cap \ldots \cap U_{x_n} )|$. Since this holds for every $z \in |K|$, it follows that $g'_{A_0}$ is well-defined; evidently these maps are compatible with one another and give the desired map $g': K \rightarrow (\Sing |F|)(\calU)$.

We now observe that the composite maps
$$ K \stackrel{g'}{\rightarrow } (\Sing |F|)(\calU) \rightarrow (\Sing_{X} |F|_X)(\calU)$$
$$ K \stackrel{g}{\rightarrow } (\Sing_X |F|_X)(X) \rightarrow (\Sing_X |F|_X)(\calU)$$
are homotopic via a ``straight-line'' homotopy 
$G: K \times \Delta^1 \rightarrow (\Sing_X |F|_X)(\calU)$, which has the desired properties.
\end{proof}

Now, the hard work is done and we are ready to enjoy the fruits of our labors.

\begin{theorem}\label{main}
Let $X$ be a paracompact topological space and $\calB$ the collection of open
$F_{\sigma}$ subsets of $X$ (see Proposition \ref{gooffy}). Then, for any
projectively cofibrant $F: \calB^{op} \rightarrow \sSet$, the natural map
$$ F \rightarrow \Sing_{X} |F|_X$$ exhibits $\Sing_{X} |F|_X$ as a sheafification of $F$.
\end{theorem}

\begin{proof}
Let $\h{\Top_{/X}}$ be the homotopy category of the model category $\Top_{/X}$ (the category obtained by inverting all of the weak equivalences defined in Proposition \ref{qur}) and
$\h{\Set_{\Delta}^{\calB^{op}}}$ the homotopy category of the category of simplicial presheaves on $\calB$. It follows from Proposition \ref{exquill} that 
the adjoint functors $\Sing_X$ and $||_X$ induce adjoint functors
$$ \Adjoint{||_X^L}{\h{\Set_{\Delta}^{\calB^{op}}}}{\h{\Top_{/X}}}{\Sing_{X}}$$
Here $||^{L}_{X}$ denotes the left-derived functor of the geometric realization (since
every object of $\Top_{/X}$ is fibrant, $\Sing_X$ may be identified with its right-derived functor).

We first claim that for any $Y \in \Top_{/X}$, the counit map
$ | \Sing_X Y |_{X}^{L} \rightarrow Y$ is a weak equivalence. To see this, choose a projectively cofibrant model $F \rightarrow \Sing_X Y$ for $\Sing_X Y$; we wish to show that
the induced map $|F|_{X} \rightarrow Y$ is a weak equivalence. By definition, this is
equivalent to the assertion that $\Sing_X |F|_{X} \rightarrow \Sing_X Y$ is a weak equivalence. But we have a commutative triangle
$$ \xymatrix{ F \ar[rr] \ar[dr] & & \Sing_X Y \ar[dl] \\
& \Sing_X |F|_X & }$$
where left-diagonal map is a weak equivalence by Lemma \ref{hardd} and Proposition \ref{aese}, and the top horizontal map is a weak equivalence by construction; the desired result now follows from the two-out-of-three property.

It follows that we may identify $\h{\Top_{/X}}$ with a full subcategory $\calC \subseteq \h{\Set_{\Delta}^{\calB^{op}}}$. By Proposition \ref{aese}, the objects of this subcategory are sheaves on $\calB$; by Lemma \ref{hardd}, every sheaf on $\calB$ is equivalent to $\Sing_X Y$ for an appropriately chosen $Y$; thus $\calC$ consists of precisely the sheaves on $\calB$.

The composite functor $F \mapsto \Sing_{X} |F|^{L}_{X}$ may be identified with a localization
functor from $\h{\Set_{\Delta}^{\calB^{op}}}$ to the subcategory $\calC$. In particular, when $F$ is projectively cofibrant, the unit of the adjunction $F \rightarrow \Sing_{X} |F|_{X}$ is a localization of $F$.
\end{proof}

\begin{corollary}\label{wamain}
Under the hypotheses of Theorem \ref{main}, the functor $\Sing_{X}$ induces an equivalence of $\infty$-categories
$$ \sNerve( \Top_{/X}^{\degree} ) \rightarrow \Shv(X).$$
In particular, the $\infty$-category $\sNerve ( \Top_{/X}^{\degree} )$ is an $\infty$-topos.
\end{corollary}

\begin{remark}\label{qurk}
In the language of model categories, we may interpret Corollary \ref{wamain} as asserting
that $\Sing_X$, $||_X$ furnish a Quillen equivalence between $\Top_{/X}$ (with the model structure of Proposition \ref{qur}) and $\Set_{\Delta}^{\calB^{op}}$ where the latter is equipped with the following {\em localization} of the projective model structure:
\begin{itemize}
\item[$(1)$] A map $F \rightarrow F'$ in $\Set_{\Delta}^{\calB^{op}}$ is a {\it cofibration} if it is a projective cofibration (in the sense of Definition \ref{projinj}).
\item[$(2)$] A map $F \rightarrow F'$ in $\Set_{\Delta}^{\calB^{op}}$ is a {\it weak equivalence} if it induces an
equivalence in the $\infty$-category $\Shv(X)$.
\end{itemize}
\end{remark}

\subsection{Base Change}\label{dooky}

With Corollary \ref{wamain} in hand, we are {\em almost} ready to deduce Theorem \ref{nice}.
Suppose given a paracompact space $X$, and let $\calB$ denote the collection of all
open $F_{\sigma}$ subsets of $X$. Let $p: \Shv(X) \rightarrow \Shv(\ast) \simeq \SSet$ be the geometric morphism induced by the projection $X \rightarrow \ast$.

For any simplicial set $K$, let $F_{K}$ denote the constant simplicial presheaf
on $\calB$ taking the value $K$. Then, if we endow $\Set_{\Delta}^{\calB^{op}}$ with the localized model structure of Remark \ref{qurk}, then $F_K$ is a model for the sheaf $p^{\ast} K$. Consequently, the space $p_{\ast} p^{\ast} K$ may be identified up to homotopy with the mapping space
$$ \bHom_{\Shv(X)}( F_{\ast}, F_{K} )$$
which, in virtue of Corollary \ref{wamain}, is equivalent to
$$ \bHom_{\Top_{/X}}( X, X \otimes K ) = (\Sing_X (X \otimes K))(X)$$
However, at this point, a technical wrinkle appears: $X \otimes K$ agrees with $X \times |K|$ as a set, but it is equipped with a finer topology (given by the direct limit of the product topologies $X \times |K_0|$, where $K_0 \subseteq K$ is a finite simplicial subset). In general, we have only
an {\em inclusion} of simplicial presheaves
$$ \eta: \Sing_X (X \otimes K) \subseteq \Sing_{X} (X \times |K|),$$
which need not be an isomorphism. However, we will complete the proof of Theorem \ref{nice}
by showing that $\eta$ is an equivalence of simplicial presheaves.

We consider a slightly more general situation. Let $p: X \rightarrow Y$ be a continuous map between paracompact spaces, and let $\calB_{X}$ and $\calB_Y$ the collections of open $F_{\sigma}$ subsets in $X$ and $Y$, respectively. Note that the inverse image along $p$ determines a map
$q: \calB_Y \rightarrow \calB_X$. Composition with $q$ induces a
``pushforward'' functor $q_{\ast}: \Set_{\Delta}^{\calB_X^{op}} \rightarrow \Set_{\Delta}^{\calB_Y^{op}}$, which
has a left adjoint which we will denote by $q^{\ast}$. Similarly, there is a ``pullback'' functor
$p^{\ast}: \Top_{/Y} \rightarrow \Top_{/X}$; however, $p^{\ast}$ generally does not possess a right adjoint. Consider the square
$$ \xymatrix{ \Set_{\Delta}^{\calB_Y^{op}} \ar[r]^{||_Y} \ar[d]^{q^{\ast}} & \Top_{/Y} \ar[d]^{p^{\ast}} \\
\Set_{\Delta}^{\calB_X^{op}} \ar[r]^{||_X} & \Top_{/X}. }$$
This square is ``lax commutative'', in the sense that there exists a natural transformation of functors
$$\eta_{F}: | q^{\ast} F |_{X} \rightarrow p^{\ast} |F|_Y = |F|_Y \times_Y X.$$
The map $\eta_{F}$ is always a bijection of topological spaces, but is generally not a homeomorphism. Nevertheless, we have the following:

\begin{proposition}\label{basechang}
Under the hypotheses above, if $F: \calB_Y^{op} \rightarrow \sSet$ is a projectively cofibrant simplicial presheaf on $Y$, then the map $\eta_{F}: | q^{\ast} F|_{X} \rightarrow |F|_Y \times_Y X$ is a weak equivalence in $\Top_{/X}$.
\end{proposition}

The proof is based on the following lemma:

\begin{lemma}\label{prechange}
Let $Y$ be a paracompact topological space and let $\calB$ be the collection of open $F_{\sigma}$ subsets of $Y$ (see Proposition \ref{gooffy}). Let $V$ be a linearly ordered set. Suppose that for
every nonempty finite subset $V_0 \subseteq V$, we are given an basic open set
$U(V_0) \in \calB$ satisfying the following conditions:
\begin{itemize}
\item[$(a)$] If $V_0 \subseteq V_1$, then $U(V_1) \subseteq U(V_0)$.
\item[$(b)$] The open set $U( \emptyset)$ concides with $X$.
\end{itemize}
Let $F: \calB^{op} \rightarrow \sSet$ be the simplicial presheaf which assigns to
each $U \in \calB$ the simplicial subset $F(U) \subseteq \Delta^{V}$ spanned by those nondegenerate simplices $\sigma$ corresponding to finite subsets $V_0 \subseteq V$ such that $U \subseteq U(V_0)$
(see Lemma \ref{grumppp}). 

For every object $X \in \Top_{/Y}$, an $n$-simplex $\tau$ of 
$\bHom_{ \Top_{/Y}}(Y, |F|_Y)$ determines a map of topological spaces from
$X \times | \Delta^n |$ to $| \Delta^{V} |$, which in turn determines a collection of maps
$\phi_{v}: X \times | \Delta^n | \rightarrow [0,1]$ such that for every $x \in X \times | \Delta^n |$, the 
sum $\Sigma_{v \in V} \phi_v(x)$ is equal to $1$. Let $\bHom^{0}_{\Top_{/Y}}(X, |F|_Y)$ denote the simplicial subset of $\bHom_{ \Top_{/Y}}(X, |F|_Y)$ spanned by those simplices $\tau$ which satisfy the following condition, where $K = \Delta^n$: 
\begin{itemize}
\item[$(\ast)$] There exists a locally finite collection of open sets $\{ U_{v} \subseteq X \times |K| \}_{v \in V}$
such that each $U_{v}$ contains the closure of the support of the function $\phi_{v}$, and
$\bigcap_{v \in V_0} U_{v}$ is contained in the inverse image of $U(V_0)$ for every finite subset $V_0 \subseteq V$.
\end{itemize}
If the topological space $X$ is paracompact, then the inclusion
$i: \bHom^{0}_{ \Top_{/Y}}( X, |F|_Y) \subseteq \bHom_{ \Top_{/Y}}( X, |F|_Y)$ is a homotopy equivalence of Kan complexes.
\end{lemma}

\begin{proof}
Note that for any finite simplicial set $K$, we can identify
$\Hom_{\sSet}(K, \bHom^{0}_{ \Top_{/Y}}( X, |F|_Y)$ with the set of all collections
of continuous maps $\{ \phi_{v}: X \times |K| \rightarrow [0,1] \}$ satisfying the condition $(\ast)$. Composition with a retraction of $| \Delta^n |$ onto a horn $| \Lambda^n_i |$ determines a section of the restriction map 
$$\Hom_{\sSet}( \Delta^n, \bHom^{0}_{ \Top_{/Y}}( X, |F|_Y))
\rightarrow \Hom_{\sSet}( \Lambda^n_i, \bHom^{0}_{ \Top_{/Y}}( X, |F|_Y)),$$
from which it follows that $\bHom^{0}_{ \Top_{/Y}}( X, |F|_Y)$ is a Kan complex. 

To prove that $i$ is a homotopy equivalence, we argue as in the proof of Lemma \ref{hardd}: it will suffice to show that for every  inclusion $K' \subseteq K$ of finite simplicial sets, every commutative diagram $$ \xymatrix{ K' \times \{0\}  \ar@{^{(}->}[d] \ar[r] & \bHom^{0}_{\Top_{/Y}}(X, |F|_Y) \ar[d] \\
K \times \{0\} \ar[r]^-{g} & \bHom_{ \Top_{/Y}}(X, |F|_Y) }$$
can be expanded to a commutative diagram
$$ \xymatrix{ (K' \times \Delta^1) \coprod_{ K' \times \{1\} } (K \times \{1\})  \ar@{^{(}->}[d] \ar[rr] & &  \bHom^{0}_{\Top_{/Y}}(X, |F|_Y) \ar[d] \\
K \times \Delta^1 \ar[rr]^{G} & & \bHom_{\Top_{/Y}}(X, |F|_Y). }$$
The map $g$ is classified by a collection of continuous maps
$\{ g_v: X \times |K| \rightarrow [0,1] \}_{v \in V}$ such that
$\Sigma_{v \in V} g_v(x) = 1$. Let $\{ U_{v} \}_{v \in V}$ be a collection of open subsets of
$X \times |K'|$ satisfying condition $(\ast)$ for the functions $\{ g_{v} | X \times |K'| \}$.
For each $v \in V$, let $W_v = \{ x \in X \times |K|: g_v(x) \neq 0 \}$. Choose a locally finite open
covering $\{ U'_{v} \}_{v \in V}$ of $X \times |K|$ which refines $\{ W_{v} \}$. Let
$\{ g'_{v} \}_{v \in V}$ be a partition of unity such that the closure of the support of each
$g'_{v}$ is contained in $U'_{v}$. We define maps
$\{ G_{v}: X \times |K| \times [0,1] \rightarrow [0,1] \}_{v \in V}$ by the formula
$$ G_{v}(x, t) = \begin{cases} (2t) g'_{v}(x) + (1-2t) g_{v}(x) & \text{ if } t \leq \frac{1}{2} \\
g'_{v}(x) & \text{ if  } t \geq \frac{1}{2}. \end{cases}$$
Then the maps $\{ G_v \}$ determine a continuous map
$G: X \times |K| \times [0,1] \rightarrow |F|_Y$, which we can identify with a map of simplicial
sets $K \times \Delta^1 \rightarrow \bHom_{ \Top_{/Y}}(X, |F|_Y)$. The restriction of
this map to $(K' \times \Delta^1) \coprod_{ K' \times \{1\} } (K \times \{1\})$ factors
through $\bHom^{0}_{ \Top_{/Y}}(X, |F|_Y)$, since the open subsets
$\{ (U_{v} \times [0,1]) \cup (U'_{v} \times \{1\} \}$ satisfy condition $(\ast)$.
\end{proof}

\begin{remark}\label{postchan}
In the situation of Lemma \ref{prechange}, suppose we are given a Kan complex
$$\bHom^{1}_{ \Top_{/Y}}( X, |F|_Y) \subseteq \bHom_{ \Top_{/Y}}( X, |F|_Y)$$ which contains $\bHom^{0}_{ \Top_{/Y}}(X, |F|_Y)$ and is closed
under the formation of ``straight line'' homotopies. More precisely, suppose that any map
$G: \Delta^n \times \Delta^1 \rightarrow \bHom_{ \Top_{/Y}}( X, |F|_Y)$ factors through
$\bHom_{ \Top_{/Y}}^{1}(X, |F|_Y)$, provided that it satisfies the properties listed below:
\begin{itemize} 

\item[$(i)$] The map $G$ is classified by a collection of continuous functions $\{ G_{v}: X \times | \Delta^n| \times [0,1] \rightarrow [0,1] \}_{v \in V}$.

\item[$(ii)$] Each $G_{v}$ can be described by the formula
$$ G_{v}(x, t) = \begin{cases} (2t) g'_{v}(x) + (1-2t) g_{v}(x) & \text{ if } t \leq \frac{1}{2} \\
g'_{v}(x) & \text{ if  } t \geq \frac{1}{2}. \end{cases}$$

\item[$(iii)$] The closure of the support of each $g'_v$ is contained in the open set
$\{ x \in X \times | \Delta^n |: g_v(x) \neq 0 \}$.

\item[$(iv)$] The restriction of $G$ to $\Delta^n \times \{0\}$ belongs to
$\bHom_{ \Top_{/Y}}^{1}(X, |F|_Y)$ (by virtue of $(iii)$, this implies that
$G| \Delta^n \times \{1\}$ factors through
$ \bHom_{ \Top_{/Y}}^{0}(X, |F|_Y) \subseteq \bHom_{ \Top_{/Y}}^{1}(X, |F|_Y)$). 
\end{itemize}
Then the proof of Lemma \ref{prechange} shows that the inclusions
$$ \bHom^{0}_{ \Top_{/Y}}( X, |F|_Y)
\subseteq \bHom^{1}_{ \Top_{/Y}}(X, |F|_Y) \subseteq \bHom_{ \Top_{/Y}}( X, |F|_Y)$$
are homotopy equivalences.
\end{remark}

\begin{proof}[Proof of Proposition \ref{basechang}]
Suppose given a weak equivalence $F \rightarrow F'$ between projectively cofibrant simplicial presheaves $F,F': \calB_Y^{op} \rightarrow \sSet$. Both $q^{\ast}$ and $||_X$ are left Qullen functors, and therefore preserve weak equivalences between cofibrant objects; it follows that $| q^{\ast} F|_{X} \rightarrow | q^{\ast} F'|_{X}$ is a weak equivalence. Similarly, 
$|F|_{Y} \rightarrow |F'|_{Y}$ is a weak equivalence between cofibrant objects of $\Top_{/Y}$.
Since {\em every} object of $\Top_{/Y}$ is fibrant, we conclude that $|F|_{Y} \rightarrow |F'|_{Y}$
is a homotopy equivalence in $\Top_{/Y}$; thus $|F|_{Y} \times_{Y} X \rightarrow |F'|_{Y} \times_{Y} X$ is a homotopy equivalence in $\Top_{/X}$. Consequently, we deduce that $\eta_{F}$ is a weak equivalence if and only if $\eta_{F'}$ is a weak equivalence.

Let $F$ be an arbitrary projectively cofibrant simplicial presheaf; we wish to show that $\eta_{F}$
is a weak equivalence. There exists a trivial projective cofibration $F \rightarrow F'$, where $F'$ is projectively fibrant. It now suffices to show that $\eta_{F'}$ is a weak equivalence. Replacing $F$ by $F'$, we reduce to the case where $F$ is projectively fibrant.

Let $F'$ be a simplicial presheaf on $\calB_Y$ satisfying the conditions of Lemma \ref{grumppp}.
Then $F'$ and $F$ are equivalent in the homotopy category of simplicial presheaves on $\calB_Y$.
Since $F'$ is projectively cofibrant and $F$ is projectively fibrant, there exists a weak equivalence
$F' \rightarrow F$. We may therefore once again reduce to proving that $\eta_{F'}$ is a weak equivalence. Replacing $F$ by $F'$, we may suppose that $F$ satisfies the conditions of
Lemma \ref{grumppp}, for some linearly ordered set $V$.

For each $U \in \calB_Y$, we have a commutative diagram of Kan complexes
$$ \xymatrix{  \bHom^{0}_{ \Top_{/X} }( U, | q^{\ast} F|_{X} ) \ar[r]^{\phi_0} \ar[d] & 
\bHom^{0}_{ \Top_{/Y} }( U, |F|_Y) \ar[d] \\
\bHom_{ \Top_{/X} }( U, | q^{\ast} F|_{X} ) \ar[r]^{\phi} & 
\bHom_{ \Top_{/Y} }( U, |F|_Y) }$$ 
where the vertical maps are defined as in Lemma \ref{prechange}. We wish to show that $\phi$ is a homotopy equivalence. This follows from the observation that $\phi_0$ is an isomorphism, and the vertical arrows are homotopy equivalences by Lemma \ref{prechange}.
%Let $\Delta^V_{Y} = Y \otimes \Delta^V$ and let $\Delta^V_{X} = X \otimes \Delta^V$. Then
%$|F|_{Y}$ may be identified with a subspace of $Z \subseteq \Delta^V_{Y}$; likewise
%$| q^{\ast} F|_{X}$ may be identified with a subspace of $Z' \subseteq \Delta^V_{X}$.
%The map $\eta_{F}$ induces a bijection
%$$ Z' \rightarrow Z \times_{Y} X \subseteq \Delta^V_{Y} \times_{Y} X$$
%which is not necessarily a homeomorphism.

%We wish to show that $\eta_{F}$ is a weak equivalence. This is equivalent to the assertion that
%the inclusion $\Sing_{X} Z' \subseteq \Sing_{X}(Z \times_{Y} X)$ is an equivalence of simplicial presheaves on $\calB_X$; in other words, we must show that
%$$ (\Sing_{X} Z')(U) \subseteq (\Sing_{X}( Z \times_{Y} X))(U)$$
%is a homotopy equivalence of Kan complexes, for each $U \in \calB_X$. Replacing $X$ by $U$,
%it suffices to show that for any inclusion $K' \subseteq K$ of {\em finite} simplicial sets, any map of pairs $$G_0: (K,K') \rightarrow ( ( \Sing_{X} (Z \times_Y X))(X) , (\Sing_X Z')(X))$$
%can be extended to a map defined on the pair
%$$G: (K \times \Delta^1, (K' \times \Delta^1) \coprod_{ K' \times \{1\} } (K \times \{1\})).$$
%We may identify the map $G_0$ with a continuous, fiber-preserving map
%$$ g_0: |K| \times X \rightarrow Z \times_{Y} X \subseteq \Delta^V_{Y} \times_Y X.$$
%For each $v \in \Delta^V_{Y}$, let $\Delta^V_{Y,v}$ denote the open star of the vertex $v$, as in the proof of Lemma \ref{hardd}; then the open sets $U_{v} = g_0^{-1} ( \Delta^V_{Y,v} \times_Y X)$
%form an open cover of $|K| \times X$. Choose a (locally finite) partition of unity $\{ f_{v} \}$ subordinate to the cover $\{ U_{v} \}$, and define
%$g_1: |K| \times X \rightarrow X \times | \Delta^{V} |$ by the formula
%$$ g_1(k,x) = (x, \Sum f_{v}(k,x) v).$$
%Since $ \{ f_{v} \}$ is locally finite, it is clear that $g_1$ factors through
%$\Delta^V_{X}$, and by construction its image is contained in $Z'$. Now let 
%$$g: |K| \times [0,1] \times X \rightarrow X \times |\Delta^V|$$ be a ``straight-line''
%homotopy between $g_0$ and $g_1$. Then $g$ determines a map
%$$G: (K \times \Delta^1, (K' \times \Delta^1) \coprod_{ K' \times \{1\} } (K \times \{1\})) \rightarrow
%( ( \Sing_{X} (Z \times_Y X))(X) , (\Sing_X Z')(X))$$
%which extends $G_0$, as desired.
\end{proof}

Theorem \ref{nice} now follows immediately from Proposition \ref{basechang}, applied in the case where $Y = \ast$ and $F$ is the constant simplicial presheaf $\calB_{X} \rightarrow \sSet$ taking the value $K$.

\begin{remark}
There is another solution to the technical difficulty presented by the fact that the bijection 
$X \otimes K \rightarrow X \times |K|$ is not necessarily a homeomorphism: one can work in a suitable category of compactly generated topological spaces, where the base change functor
$Z \mapsto X \times_Y Z$ has a right adjoint and therefore automatically commutes with all colimits.
This is perhaps a more conceptually satisfying approach; however, it leads to a proof of
Theorem \ref{nice} only in the special case where the space $X$ is itself compactly generated.
\end{remark}

We close this section by describing a few applications of Proposition \ref{basechang} and its proof to the theory of sheaves (of spaces) on a paracompact topological space $X$.

\begin{corollary}\label{hidebound}
Let $X$ be a paracompact topological space, $Y$ a closed subset of $X$, and $i: Y \rightarrow X$
the inclusion map. Let $\calF$ be an object of $\Shv(X)$, and let $\eta_0$ be a global section of
$i^{\ast} \calF$. Then there exists an open subset $U$ of $X$ which contains $Y$ and a section
$\eta \in \calF(U)$ whose image under the restriction map $\calF( U ) \rightarrow (i^{\ast} \calF)(U \cap Y) = (i^{\ast} \calF)(Y)$ lies in the path component of $\eta_0$.
\end{corollary}

\begin{proof}
Let $\calB$ denote the collection of open $F_{\sigma}$ subsets of $X$. Without loss of generality, we may assume that $\calF$ is represented by a projectively cofibrant simplicial presheaf
$F \subseteq \underline{ \Delta^{V} }$ satisfying the conditions of Lemma \ref{grumppp},
where $V$ is a linearly ordered set. Using Proposition \ref{basechang}, Corollary \ref{wamain}, and 
Lemma \ref{prechange}, it will suffice to prove the following assertion:
\begin{itemize}
\item[$(a)$] For every vertex $\eta_0$ of $\bHom^{0}_{\Top_{/X}}( Y, |F|_X)$ can be
lifted to a vertex of $\bHom^{0}_{ \Top_{/X}}(U, |F|_X)$, for some sufficiently small
paracompact open neighborhood $U$ of $Y$.
\end{itemize}
To prove $(a)$, suppose we are given a vertex of
$\eta_0 \in \bHom^{0}_{\Top_{/X}}( Y, |F|_X)$, corresponding to a collection of functions
$\{ \phi_v: Y \rightarrow [0,1] \}$. Since $\eta_0$ belongs to
$\bHom^{0}_{\Top_{/X}}(Y, |F|_X)$, there exist open sets
$U_v \subseteq Y$ satisfying condition $(\ast)$ of Lemma \ref{prechange}.

For each $y \in Y$, there exists an open neighborhood $W_y$ of $y$ in $X$ for which the
set $V(y) = \{ v \in V: W_y \cap U_v \}$ is finite. Let $W = \bigcup_{y \in Y} W_y$.
Shrinking $W$ if necessary, we may suppose that $W$ is a paracompact open neighborhood of $Y$
in $X$. Since $W$ is paracompact, there exists a locally finite open covering
$\{ W'_{\alpha} \}_{\alpha}$ of $W$, so that for each index $\alpha$ there exists a point
$y_{\alpha} \in Y$ such that $W'_{\alpha} \subseteq W_{y}$. For $v \in V$, let
$U'_{v} = \bigcup_{v \in V(y_{\alpha})} W'_{\alpha}$. The open sets
$U'_{v}$ form a locally finite open covering of $W$, and each intersection
$U'_{v} \cap Y$ is an open subset of $U_{v}$ which contains the closure of the support
of $\phi_{v}$.

For each $v \in V$, choose a continuous function $\phi'_{v}: X \rightarrow [0,1]$ such that
$\phi'_{v} | Y = \phi_{v}$, and the closure of the support of $\phi'_{v}$ is contained in
$U'_{v}$. There exists another open set $U''_{v}$ whose closure is contained in $U'_{v}$, which again contains the closure of the support of $\phi'_{v}$. For every finite subset $V_0 \subseteq V$, let
$K_{V_0}$ denote the intersection $\bigcap_{v \in V_0} \overline{ U}''_{v}$, and let
$K^{0}_{V_0}$ denote the open subset of $K_{V_0}$ given by the inverse image of the
open set $U(V_0) \subseteq X$ (the largest open subset for which
$\Delta^{V_0}$ belongs to $F( U(V_0) ) \subseteq \Delta^V$). Then
$\{ K_{V_0} - K^{0}_{V_0} \}$ is a locally finite collection of closed subsets of $X$, none of which intersects $Y$. Let $K = \bigcup_{V_0}(K_{V_0} - K^{0}_{V_0})$; then $K$ is a closed subset
of $X$. Let $W'$ be an open $F_{\sigma}$-subset of $W$ which contains $Y$ and does not intersect $K$. Replacing $X$ by $W'$, we may assume that
$W = X$ and that $K = \emptyset$.

Since the collection of functions $\{ \phi'_{v} \}_{v \in V}$ have locally finite support,
the function $\phi' = \Sigma_{v \in V} \phi'_{v}$ is well-defined, and takes the value
$1$ on $Y$. The open set $\{ x \in X: \phi'(x) > 0 \}$ is a paracompact open subset of $X$ (Proposition \ref{gooffy}). Shrinking $X$ further, we may suppose
that $\phi'$ is everywhere nonzero on $X$. Set $\phi''_{v} = \frac{ \phi_{v} }{\phi}$ for each
$v \in V$. Then the functions $\phi''_{v}$ determine a vertex
$\eta \in \bHom_{ \Top_{/X}}(X, |F|_X)$. Moreover, the open sets
$\{ U''_{v} \}$ satisfy condition $(\ast)$ appearing in the statement of Lemma \ref{prechange}, so that
$\eta$ belongs to $\bHom^{0}_{\Top_{/X}}( X, |F|_X)$ as desired.
\end{proof}

Corollary \ref{hidebound} admits the following refinement:

\begin{corollary}\label{snottle}
Let $X$ be a paracompact topological space, $Y$ a closed subset of $X$, and $i: Y \rightarrow X$
the inclusion map. Let $\calF$ be an object of $\Shv(X)$. Then the canonical map
$$ \alpha_{\calF}: \varinjlim_{Y \subseteq U} \calF(U) \rightarrow
\varinjlim_{Y \subseteq U} (i^{\ast} \calF)(U \cap Y) \simeq (i^{\ast} \calF)(Y)$$
is a homotopy equivalence. Here the colimit is taken over the filtered partially ordered set of
all open subsets of $X$ which contain $Y$.
\end{corollary}

\begin{proof}
We will prove by induction on $n \geq 0$ that the map $\alpha_{\calF}$ is $n$-connective. The case
$n=0$ follows from Corollary \ref{hidebound}. Suppose that $n > 0$. We must show that,
for every pair of points $\eta, \eta' \in \varinjlim_{Y \subseteq U} \calF(U)$, the induced map
of fiber products
$$ \alpha'_{\calF}: \ast \times_{ \varinjlim_{ Y \subseteq U} \calF(U) } \ast
\rightarrow \ast \times_{ (i^{\ast} \calF)(Y)} \ast$$
is $(n-1)$-connective. Without loss of generality, we may assume that $\eta$ and $\eta'$
arise from sections of $\calF$ over some $U \subseteq X$ containing $Y$. Shrinking $U$ if necessary, we may assume that $U$ is paracompact. Replacing $X$ by $U$, we may assume that
$\eta$ and $\eta'$ arise from global sections $f,f': 1 \rightarrow \calF$, where $1$ denotes the final object of $\Shv(X)$. Let $\calG = 1 \times_{\calF} 1 \in \Shv(X)$. Using the left exactness of
$i^{\ast}$ and Proposition \ref{frent}, we can identify $\alpha'_{\calF}$ with $\alpha_{\calG}$.
We now invoke the inductive hypothesis to deduce that $\alpha_{\calG}$ is $(n-1)$-connective, as desired.
\end{proof}

\begin{lemma}\label{agint}
Let $Y$ be a paracompact topological and $\calB$ the collection of open $F_{\sigma}$ subsets of
$Y$ (see Proposition \ref{gooffy}). Let $V$ be a linearly ordered set, and let $F: \calB^{op} \rightarrow \sSet$ be as in the statement of Lemma \ref{prechange}. Suppose given a paracompact space
$X \in \Top_{/Y}$ and a closed subspace $X' \subseteq X$. Then the map
$$ \bHom^{0}_{ \Top_{/Y}}( X, |F|_Y) \rightarrow \bHom^{0}_{ \Top_{/Y}}( X', |F|_{Y} )$$
is a Kan fibration.
\end{lemma}

\begin{proof}
We must show that every lifting problem of the form
$$ \xymatrix{ \Lambda^{n}_i \ar[r] \ar@{^{(}->}[d] &  \bHom^{0}_{ \Top_{/Y}}( X, |F|_Y ) \ar[d] \\
\Delta^n \ar[r] & \bHom^{0}_{ \Top_{/Y}}( X', |F|_Y ) }$$
admits a solution. Since the pair $( |\Delta^m|, |\Lambda^m_i|)$ is homeomorphic to
$( |\Delta^{m-1}| \times [0,1], | \Delta^{m-1} | \times \{0\})$, we can replace $X$ by
$X \times | \Delta^{m-1} |$ and thereby reduce to the case $m=1$.

Let $Z = ( X \times \{0\} ) \coprod_{ X' \times \{0\} } (X' \times [0,1] )$, and let
$\eta_0 \in \bHom^{0}_{ \Top_{/Y}}( Z, |F|_Y)$; we wish to show that 
$\eta_0$ can be lifted to a point in $\bHom^{0}_{ \Top_{/Y}}( X \times [0,1], |F|_Y)$.
The proof of Corollary \ref{hidebound} shows that we can lift
$\eta_0$ to a point $\eta_1 \in \bHom^{0}_{ \Top_{/Y}}( U, |F|_Y)$, for some open set
$U \subseteq X \times [0,1]$ containing $Z$. 

For each $x \in X$, there exists a real number $\epsilon_x > 0$ and an open neighborhood
$V_x \subseteq X$ such that $V_x \times [0, \epsilon_x) \subseteq U$. 
Since $X$ is paracompact, the open covering $\{ V_x \}_{x \in X}$ admits a locally
finite refinement $\{ W_{\alpha} \}$, so that for each index $\alpha$ there
exists a point $x(\alpha) \in X$ such that $W_{\alpha} \subseteq V_{x(\alpha)}$.
Let $\{ \phi_{\alpha} \}$ be a partition of unity subordinate the covering $W_{\alpha}$, and let
$$ \psi = \Sigma_{ \alpha} \epsilon_{x(\alpha)} \phi_{\alpha}.$$
Since the interval $[0,1]$ is compact, there exists an open neighborhood
$V \subseteq X$ containing $X'$ such that $V \times [0,1] \subseteq U$.
Choose a function $\psi': X \rightarrow [0,1]$ such that
$\psi' | (X-V) = \psi|(X-V)$ and $\psi' | X'$ is equal to $1$. Set
$K = \{ (x,t) \in X \times [0,1]: t \leq \phi(x) \}$, so that
$Z \subseteq K \subseteq U$, and let $\eta_2 \in \bHom^{0}_{ \Top_{/Y}}(K, |F|_X)$ be the 
restriction of $\eta_1$. Since $K$ is a retract of $X \times [0,1]$ in the category
$\Top_{/Y}$, we can lift $\eta_2$ to a point $\eta in \bHom^{0}_{ \Top_{/Y}}( X \times [0,1], |F|_X)$ as desired.
\end{proof}

\begin{proposition}\label{siegland}
Let $X$ be a paracompact topological space. Suppose given a sequence of closed subspaces
$$ X_0 \subseteq X_1 \subseteq X_2 \subseteq \ldots \subseteq X$$
with the following properties:
\begin{itemize}
\item[$(1)$] The union $\bigcup X_{i}$ coincides with $X$.
\item[$(2)$] A subset $U \subseteq X$ is open if and only if each of the intersections
$U \cap X_i$ is an open subset of $X_i$.
\end{itemize}
Then the induced diagram
$$ \Shv(X_0) \rightarrow
\Shv(X_1) \rightarrow \ldots
\rightarrow \Shv(X)$$
exhibits $\Shv(X)$ as the colimit of the sequence $\{ \Shv(X_i) \}_{i \geq 0}$ in the 
$\infty$-category $\RGeom$ of $\infty$-topoi.
\end{proposition}

\begin{remark}\label{stasis}
Hypotheses $(1)$ and $(2)$ of Proposition \ref{siegland} can be summarized by saying that
$X$ is the direct limit of the sequence $\{ X_i \}$ in the category of topological spaces.
It follows from this condition that for any locally compact space $Y$, the product
$X \times Y$ is also the direct limit of the sequence $\{ X_i \times Y \}$.
To prove this, we observe that for any topological space $Z$, we have bijections
$$ \Hom(X \times Y, Z)
\simeq \Hom(X, Z^Y) \simeq \varprojlim \Hom(X_i, Z^Y) \simeq \varprojlim \Hom(X_i \times Y, Z),$$
where $Z^Y$ is endowed with the compact-open topology. In particular, we deduce that
$X \times \Delta^n$ is the direct limit of the topological spaces $X_i \times \Delta^n$, for each $n \geq 0$.
\end{remark}

\begin{proof}
For each nonnegative integer $n$, let $i(n)$ denote the inclusion from
$X_n$ to $X_{n+1}$, and $j(n)$ the inclusion of $X_{n}$ into $X$. 
These functors induce geometric morphisms
$$ \Adjoint{ i(n)^{\ast} }{ \Shv(X_{n+1}) }{ \Shv(X_n)}{i(n)_{\ast}} $$
$$ \Adjoint{ j(n)^{\ast} }{ \Shv(X)}{ \Shv(X_n)}{ j(n)_{\ast}}.$$
Let $\calC$ denote a homotopy inverse limit of the tower of $\infty$-categories
$$ \ldots \rightarrow \Shv( X_2) \stackrel{ i(1)^{\ast}}{\rightarrow} \Shv(X_1)
\stackrel{ i(0)^{\ast}}{\rightarrow} \Shv(X_0).$$
In view of Proposition \ref{colimtopoi}, we can also identify
$\calC$ with the direct limit of the sequence $\{ \Shv(X_i) \}_{i \geq 0}$ in $\RGeom$.
The maps $j(n)$ determine a geometric morphism
$$ \Adjoint{ j^{\ast} }{ \Shv(X)}{\calC}{j_{\ast}}.$$
To complete the proof, it will suffice to show that the functor $j^{\ast}$ is an equivalence of
$\infty$-categories.

We first show that the unit map $u: \id_{\Shv(X)} \rightarrow j_{\ast} j^{\ast}$ is an equivalence of functors.
Let $\calF \in \Shv(X)$; we wish to show that the map
$$u_{\calF}: \calF \rightarrow j_{\ast} j^{\ast} \calF
\simeq \varprojlim j(n)_{\ast} j(n)^{\ast} \calF$$
is an equivalence in $\Shv(X)$. It will suffice to prove the analogous assertion after evaluating
both sides on every open $F_{\sigma}$ subset $U \subseteq X$. Replacing $X$ by $U$, we are reduced to proving that the induced map
$$ \alpha_{\calF}: \calF(X) \rightarrow (j_{\ast} j^{\ast} \calF)(X) \simeq \varprojlim (j(n)^{\ast} \calF)(X_n)$$
is a homotopy equivalence. Let $\calB$ be the collection of all open $F_{\sigma}$ subsets of $X$. Without loss of generality, we may assume that $\calF$ is represented by a projectively cofibrant simplicial presheaf $F: \calB^{op} \rightarrow \SSet$ satisfying the conditions
of Lemma \ref{grumppp}. Using Theorem \ref{main}, Corollary \ref{wamain}, and Proposition \ref{basechang}, we can identify
$\calF(X)$ with the Kan complex of sections $K = \bHom^{0}_{ \Top_{/X}}( X, \widetilde{X})$ and
each $( j(n)^{\ast} \calF)(X_n)$ with the Kan complex of sections $K(n) = \bHom_{ \Top_{/X}}(X_n, \widetilde{X})$. It will therefore suffice to show that the canonical map
$K \rightarrow \varprojlim K(n)$ exhibits $K$ as a homotopy inverse limit of the tower
$\{ K(n) \}$. 

It follows from Remark \ref{stasis} that the map $K \rightarrow \varprojlim K(n)$ is
an isomorphism of simplicial sets. For each $n \geq 0$, let 
$K(n)^{0} = \bHom^{0}_{ \Top_{/X}}( X_n, \widetilde{X} ) \subseteq K(n)$
(with notation as in Lemma \ref{prechange}), and let $K^{0} = \varprojlim K(n)^{0} \subseteq K$.
Lemma \ref{prechange} implies that each inclusion $K(n)^{0} \subseteq K(n)$ is a homotopy equivalence. Lemma \ref{agint} implies that the restriction maps
$K(n+1)^{0} \rightarrow K(n)^{0}$ are Kan fibrations. It follows that the inverse limit $K^0$ of the tower
$\{ K(n)^{0} \}$ is a Kan complex, and that the map
$K^0 \simeq \varprojlim \{ K(n)^{0} \}$ exhibits $K^0$ as the homotopy inverse limit of
$\{ K(n)^{0} \}$. Invoking Remark \ref{postchan}, we deduce that the inclusion $K^0 \subseteq K$
is a homotopy equivalence, so that the equivalent diagram $K \simeq \varprojlim \{ K(n) \}$ exhibits
$K$ as a homotopy inverse limit of $\{ K(n) \}$ as desired.

We now argue that the counit map $v: j^{\ast} j_{\ast} \rightarrow \id$ is an equivalence of functors.
Unwinding the definitions, we must prove the following: given a collection of sheaves
$\calF_{n} \in \Shv(X_n)$ and equivalences $\calF_{n} \simeq i(n)^{\ast} \calF_{n+1}$, 
the canonical map
$$ j(n)^{\ast}( \varprojlim j(n+k)_{\ast} \calF_{n+k}) \rightarrow \calF_n$$
is an equivalence of sheaves on $X_{n}$, for each $n \geq 0$. It will suffice to show that this map induces a homotopy equivalence after passing to the global sections over every open
$F_{\sigma}$ subset $U \subseteq X_n$. There exists a function $\phi_0: X_n \rightarrow [0,1]$ such that
$U = \{ x \in X_n: \phi_0(x) > 0 \}$. Choose a map $\phi: X \rightarrow [0,1]$ such that
$\phi_0 = \phi| X_n$. Replacing $X$ by the paracompact open subset
$\{ x \in X: \phi(x) > 0 \}$, we can reduce to the case where $U = X_n$. 

We will prove by induction on $k$ that, for any compatible collection of sheaves
$\{ \calF_{n} \in \Shv(X_n), \calF_n \simeq i(n)^{\ast} \calF_{n+1} \}$, the map 
$$ \psi: (j(n)^{\ast} \calF)(X_n) \rightarrow \calF_n(X_n)$$
is $k$-connective, where $\calF = \varprojlim j(m)_{\ast} \calF_m$.
If $k > 0$, then it will suffice to show that for any pair of points
$\eta, \eta' \in (j(n)^{\ast} \calF)(X_n)$, the induced map
$$ \psi': \ast \times_{ (j(n)^{\ast} \calF)(X_n) } \ast \rightarrow
\ast \times_{ \calF_n(X_n)} \ast$$
is $(k-1)$-connective. Using Corollary \ref{hidebound}, we may assume that
$\eta$ and $\eta'$ arise from sections $\overline{\eta}, \overline{\eta}' \in
\calF(U)$, for some open neighborhood $U$ of $X_n$. Shrinking $U$ if necessary, we may assume that $U$ is paracompact. Replacing $X$ by $U$, we may assume that $\overline{\eta}$ and
$\overline{\eta}'$ are global sections of $\calF$. Since $j(n)^{\ast}$ is left exact, we can identify
$\psi'$ with the map
$$ j(n)^{\ast} ( \ast \times_{\calF} \ast)(X_n) \rightarrow ( \ast \times_{\calF_n} \ast)(X_n).$$
The $(k-1)$-connectivity of this map now follows from the inductive hypothesis.

It remains to treat the case $k=0$. Fix an element $\eta_n \in \calF_n(X_n)$; we wish to show that
$\eta_n$ lies in the image of $\pi_0 \psi$. For every open set $U \subseteq X$,
the map
$$ \pi_0 \calF(U) = \pi_0 \varprojlim (j(m)_{\ast} \calF_m)(U)
\rightarrow \varprojlim (\pi_0 j(m)_{\ast} \calF_m)(U)
\simeq \varprojlim \pi_0 \calF_m( U \cap X_m)$$
is surjective. Consequently, to prove that $\eta_n$ lies in the image of
$\pi_0 \psi$, it will suffice to show that there exists an open set $U$ containing $X_n$
such that $\eta_n$ can be lifted to $\varprojlim \pi_0 \calF_m(U \cap X_m)$. 
By virtue of assumption $(2)$, it will suffice to construct a sequence of
open $F_{\sigma}$ subsets $\{ U_{m} \subseteq X_m \}_{m \geq n}$ and a sequence of compatible
sections $\gamma_{m} \in \pi_0 \calF_m(U_m)$, such that $U_n = X_n$ and
$\gamma_m = \eta_m$. The construction goes by induction on $m$. Assuming that
$(U_m, \eta_m)$ has already been constructed, we invoke the assumption that
$U_m$ is an $F_{\sigma}$ to choose a continuous function $f: X_m \rightarrow [0,1]$ such that
$U_m = \{ x \in X_m: f(x) > 0 \}$. Let $f': X_{m+1} \rightarrow [0,1]$ be a continuous extension of
$f$, and let $V = \{ x \in X_{m+1}: f'(x) > 0 \}$. Then $V$ is a paracompact open subset of $X_{m+1}$, and $U_{m}$ can be identified with a closed subset of $V$. Applying Corollary \ref{hidebound} to the restriction $\calF_{n+1} | V$, we deduce the existence of an open set $U_{m+1} \subseteq V$
such that $\eta_{k}$ can be extended to a section $\eta_{k+1} \in \pi_0 \calF_{m+1}(U_{m+1})$. 
Shrinking $U_{m+1}$ if necessary, we may assume that $U_{m+1}$ is itself an $F_{\sigma}$, which completes the induction.
\end{proof}

\begin{remark}
Suppose given a given a sequence of closed embeddings of topological spaces
$$ X_0 \subseteq X_1 \subseteq X_2 \subseteq \ldots,$$
and let $X$ be the direct limit of the sequence. Suppose further that:
\begin{itemize}
\item[$(a)$] For each $n \geq 0$, the space $X_n$ is paracompact.
\item[$(b)$] For each $n \geq 0$, there exists an open neighborhood
$Y_n$ of $X_n$ in $X_{n+1}$ and a retraction $r_n$ of $Y_{n}$ onto $X_n$.
\end{itemize}
Then $X$ is itself paracompact, so that the hypotheses of Proposition \ref{siegland} are satisfied and $\Shv(X)$ is the direct limit of the sequence of $\infty$-topoi $\{ \Shv(X_n) \}_{n \geq 0}$. To prove this, it will suffice every open covering $\{ U_{\alpha} \}_{\alpha \in A}$ of $X$ admits a refinement
$\{ V_{\beta} \}_{\beta \in B}$ which is {\em countably locally finite}: that is, there exists a decomposition
$B = \bigcup_{n \geq 0} B_{n}$ such that each of the collections
$\{ V_{\beta} \}_{\beta \in B_n }$ is a locally finite collection of open sets, each of which is contained in some $U_{\alpha}$ (see \cite{munkres}). To construct this locally finite open covering, we choose for
each $n \geq 0$ a locally finite open covering $\{ W_{\beta} \}_{\beta \in B_n}$ of $X_n$ which refines the covering $\{ U_{\alpha} \cap X_n \}_{\alpha \in A}$. For each $\beta \in B_n$, we have
$W_{\beta} \subseteq U_{\alpha}$ for some $\alpha \in A$. We now define
$V_{\beta}$ to be the union of a collection of open subsets 
$\{ V_{\beta}(m) \subseteq X_m \}_{m \geq n}$, which are constructed as follows:
\begin{itemize}
\item If $m = n$, we set $V_{\beta}(m) = W_{\beta}$.
\item Let $m > n$, and let $Z_{m-1}$ be an open neighborhood of $X_{m-1}$ in
$X_m$ whose closure is contained in $Y_{m-1}$. We then set
$V_{\beta}(m)= \{ z \in Z_{m-1}: r_{m-1}(z) \in V_{\beta}(m-1) \} \cap U_{\alpha}.$ 
\end{itemize}
It is clear that each $V_{\beta}(m)$ is an open subset of $X_m$ contained in $U_{\alpha}$, and that
$V_{\beta}(m+1) \cap X_m = V_{\beta}(m)$. Since $X$ is equipped with the direct limit topology,
the union $V_{\beta} = \bigcup_{m} V_{\beta}(m)$ is open in $X$. The only nontrivial point is to
verify that the collection $\{ V_{\beta} \}_{\beta \in B_n}$ is locally finite. 

Pick a point $x \in X$; we wish to prove the existence of a neighborhood $S_x$ of $x$ such that
$\{ \beta \in B_n: S \cap V_{\beta} \neq \emptyset \}$ is finite. Then there exists some
$m \geq n$ such that $x \in X_{m}$; we will construct $S_{x}$ using induction on $m$. 
If $m > n$ and $x \in \overline{Z}_{m-1}$, then let $x' = r_{m-1}(x)$, and set
$S_{x} = S_{x'}$. If $m > n$ and $x \notin \overline{Z}_{m-1}$, or if
$m=n$, then we define $S_{x} = \bigcup_{k \geq m} S_{x}(k)$, where
$S_x(k)$ is an open subset of $X_{k}$ containing $x$, defined as follows.
If $m > n$, let $S_x(m) = X_{m} - \overline{Z}_{m-1}$, and if
$m=n$ let $S_x(m)$ be an open subset of $X_n$ which intersects only finitely many of the sets
$\{ W_{\beta} \}_{ \beta \in B_n }$. If $k > m$, we let 
$S_{x}(k) =  \{ z \in Y_{k-1}: r_{k-1}(z) \in S_x(k-1) \}.$
It is not difficult to verify that the open set $S_x$ has the desired properties.
\end{remark}

\subsection{Higher Topoi and Shape Theory}\label{shapesec}

If $X$ is a sufficiently nice topological space (for example, an absolute neighborhood retract), then there exists a homotopy equivalence $Y \rightarrow X$, where $Y$ is a CW complex.
If $X$ is merely assumed to be paracompact, then it is generally not possible to approximate $X$ well by means of a CW-complex $Y$ equipped with a map to $X$. However, in view of Theorem \ref{main}, one can still extract a substantial amount of information by considering maps
from $X$ to CW complexes. {\it Shape theory} is an attempt to summarize all of this information in a single invariant, called the {\it shape} of $X$. In this section, we will sketch a generalization of shape theory to the setting of $\infty$-topoi.

\begin{definition}\label{defshape}\index{gen}{pro-space}\index{gen}{shape}\index{not}{ProSSet@$\Pro(\SSet)$}
We let $\Pro(\SSet)$ denote the full subcategory of $\Fun(\SSet, \SSet)^{op}$ spanned by
left exact, accessible functors $f: \SSet \rightarrow \SSet$. We will refer to $\Pro(\SSet)$
as the $\infty$-category of {\em pro-spaces}, or as the $\infty$-category of {\it shapes}.
\end{definition}

\begin{remark}
If $\calC$ is a small $\infty$-category which admits finite limits, then any functor
$f: \calC \rightarrow \SSet$ is accessible and may be viewed as an object of $\calP(\calC^{op})$. 
The left exactness of $f$ is then equivalent to the condition that $f$ belongs to
$\Ind(\calC^{op}) = \Pro(\calC)^{op}$. Definition \ref{defshape} constitutes a natural extension
of this terminology to a case where $\calC$ is not necessarily small; here it is convenient to add a hypothesis of accessibility for technical reasons (which will not play any role in the discussion below).
\end{remark}

\begin{definition}\label{deshape}
Let $\calX$ be an $\infty$-topos. According to Proposition \ref{spacefinall}, there exists a
geometric morphism $q_{\ast}: \calX \rightarrow \SSet$, which is unique up to homotopy.
Let $q^{\ast}$ be a left adjoint to $q_{\ast}$ (also unique up to homotopy).
The composition $q_{\ast} q^{\ast}: \SSet \rightarrow \SSet$ is an accessible left-exact functor, which we will refer to as the {\it shape of $\calX$} and denote by $\Sh(\calX) \in \Pro(\SSet)$.
\index{gen}{shape!of an $\infty$-topos}\index{not}{ShcalX@$\Sh(\calX)$}
\end{definition}

\begin{remark}
This definition of the shape of an $\infty$-topos appears also in \cite{toen}.
\end{remark}

\begin{remark}
Let $p_{\ast}: \calY \rightarrow \calX$ be a geometric morphism of $\infty$-topoi and $p^{\ast}$ a left adjoint to $p_{\ast}$. Let $q_{\ast}: \calX \rightarrow \SSet$ and $q^{\ast}$ be as in Definition \ref{deshape}. The unit map $\id_{\calX} \rightarrow p_{\ast} p^{\ast}$ induces a transformation 
$$q_{\ast} q^{\ast} \rightarrow q_{\ast} p_{\ast} p^{\ast} q^{\ast} \simeq (q \circ p)_{\ast} (q \circ p)^{\ast},$$
which we may view as a map $\Sh(\calX) \rightarrow \Sh(\calY)$ in $\Pro(\SSet)$. Via this construction, we may view $\Sh$ as a functor from the homotopy category $\h{\RGeom}$ of $\infty$-topoi to the homotopy category $\h{\Pro(\SSet)}$. We will say that a geometric morphism $p_{\ast}: \calY \rightarrow \calX$ is a {\it shape equivalence} if it induces an equivalence $\Sh(\calY) \rightarrow \Sh(\calX)$ of pro-spaces.\index{gen}{shape!equivalence}
\end{remark}

\begin{remark}
By construction, the shape of an $\infty$-topos $\calX$ is well-defined up to equivalence
in $\Pro(\SSet)$. By refining the above construction, it is possible construct a shape functor from $\RGeom$ to the $\infty$-category $\Pro(\SSet)$, rather than on the level of homotopy.
\end{remark}

\begin{remark}\label{struke}
Our terminology does not quite conform to the usage in classical topology. Recall that if $X$ is a compact metric space, the {\it shape} of $X$ is defined as a pro-object in the {\em homotopy} category of spaces. There is a refinement of shape, known as {\it strong shape}, which takes values in the homotopy category of pro-spaces. Definition \ref{deshape} is a generalization of strong shape, rather than shape. We refer the reader to \cite{shapetheory} for a discussion of classical shape theory.
\end{remark}

\begin{proposition}\label{parashape}
Let $p: X \rightarrow Y$ be a continuous map of paracompact topological spaces.
Then $p_{\ast}: \Shv(X) \rightarrow \Shv(Y)$ is a shape equivalence if and only if,
for every Kan complex $K$, the induced map of Kan complexes 
$\bHom_{\Top}(Y, |K|)  \rightarrow \bHom_{\Top}(X, |K|)$ is a homotopy equivalence.
$($Here $\bHom_{\Top}(Y, |K| )$ denotes the simplicial set whose $n$-simplices are
given by continuous maps $Y \times | \Delta^n | \rightarrow |K|$, and
$\bHom_{\Top}(X, |K| )$ is defined likewise.$)$
\end{proposition}

\begin{proof}
Corollary \ref{wamain} and Proposition \ref{basechang} imply that for any paracompact topological space $Z$ and any Kan complex $K$, there is a natural isomorphism
$$ \Sh( \Shv(Z) ) ( K ) \simeq \bHom_{\Top}(Z, |K|)$$ in the homotopy category $\calH$.
\end{proof}

\begin{example}
Let $X$ be a scheme, let $\toposX$ be the topos of \'{e}tale sheaves on $X$, and let
$\calX$ be the associated $1$-localic $\infty$-topos (see \S \ref{nlocalic}). The shape $\Sh(\calX)$ defined above is closely related to the \'{e}tale homotopy type introduced by Artin and Mazur (see \cite{artinmazur}). There are three important differences:
\begin{itemize}
\item[$(1)$] Artin and Mazur work with pro-objects in the homotopy category $\calH$, rather than with actual pro-objects of $\SSet$. Our definition is closer in spirit to that of Friedlander, who works instead in the homotopy category of pro-objects in $\sSet$ (see \cite{fried}).
\item[$(2)$] The \'{e}tale homotopy type of \cite{artinmazur} is constructed by considering \'{e}tale hypercoverings of $X$; it is therefore more closely related to the shape of the hypercompletion
$\calX^{\hyp}$.
\item[$(3)$] Artin and Mazur generally study a certain completion of $\Sh(\calX^{\hyp})$ with respect to the class of truncated spaces, which has the effect of erasing the distinction between $\calX$ and $\calX^{\hyp}$ and discarding a bit of (generally irrelevant) information.
\end{itemize}
\end{example}

\begin{remark}\label{surety}
Let $\ast$ denote a topological space consisting of a single point. By definition,
$\Shv(\ast)$ is the full subcategory of $\Fun(\Delta^1,\SSet)$ spanned by those morphisms
$f: X \rightarrow Y$ where $Y$ is a final object of $\SSet$. We observe that $\Shv(\ast)$ is equivalent to the full subcategory spanned by those morphisms $f$ as above where
$Y = \Delta^0 \in \SSet$, and that this full subcategory is {\em isomorphic} to $\SSet$.
\end{remark}
\begin{definition}
We will say that an $\infty$-topos $\calX$ has {\it trivial shape} if $\Sh(\calX)$ is equivalent
to the identity functor $\SSet \rightarrow \SSet$.\index{gen}{shape!trivial}
\end{definition}

\begin{remark}
Let $q_{\ast}: \calX \rightarrow \SSet$ be a geometric morphism. Then the unit map
$u: \id_{\SSet} \rightarrow q_{\ast} q^{\ast}$ induces a map of pro-spaces
$\Sh(X) \rightarrow \id_{\SSet}$. Since $\id_{\SSet}$ is a final object in $\Pro(\SSet)$, 
we observe that $\calX$ has trivial shape if and only if $u$ is an equivalence; in other words, if and only if the pullback functor $q^{\ast}$ is fully faithful.
\end{remark}

We now sketch another interpretation of shape theory, based on the $\infty$-topoi associated
to {\em pro-spaces}. Let $X = \SSet$, let $\pi: \SSet \times \SSet \rightarrow \SSet$ be the projection onto the first factor, let $\delta: \SSet \rightarrow \SSet \times \SSet$ denote the diagonal map, and let $\phi: (\SSet \times \SSet)^{/\delta_{\SSet}} \rightarrow \SSet$ be defined as in \S \ref{consweet}. Proposition \ref{colimfam} implies that $\phi$ is a coCartesian fibration. We may identify
the fiber of $\phi$ over an object $X \in \SSet$ with the $\infty$-category $\SSet^{/X}$. For each morphism $f: X \rightarrow Y$ in $\SSet$, $\phi$ associates a functor
$f_{!}: \SSet^{/X} \rightarrow \SSet^{/Y}$, given by composition with $f$.
Since $\SSet$ admits pullbacks, each $f_{!}$ admits a right adjoint $f^{\ast}$, so that
$\phi$ is also a Cartesian fibration, associated to some functor
$\psi: \SSet^{\op} \rightarrow \LGeom$.

Let $\hat{X}: \SSet \rightarrow \SSet$ be a pro-space. Then $\hat{X}$ classifies a left fibration 
$M^{op} \rightarrow \SSet$, where $M$ is a filtered $\infty$-category. 
Let $\theta$ denote the composition
$$ M^{op} \rightarrow \SSet \stackrel{\psi^{op}}{\rightarrow} (\LGeom)^{op}.$$
Although $M$ is generally not small, the accessibility condition on $F$ guarantees the existence of a cofinal map $M' \rightarrow M$, where $M'$ is a small, filtered $\infty$-category. Theorem \ref{sutcar} implies that the diagram $\theta$ has a limit, which we will denote by
$$ \SSet_{/ \hat{X} }$$ and refer to as the {\it $\infty$-topos of local systems on $\hat{X}$}.\index{gen}{$\infty$-topos!of local systems}

\begin{remark}
If $\hat{X}$ is a pro-space, then Proposition \ref{charproet} implies that
the associated geometric morphism $\SSet_{/ \hat{X} } \rightarrow \SSet$ is pro-\'{e}tale. However, the converse is false in general.
\end{remark}

\begin{remark}
Let $G$ be a profinite group, which we may identify with a $\Pro$-object in the category of finite groups. We let $BG$ denote the corresponding $\Pro$-object of $\SSet$, obtained by applying the classifying space functor objectwise. Then $\SSet_{/ BG}$ can be identified with
the $1$-localic $\infty$-topos associated to the ordinary topos of sets with a continuous $G$-action.
It follows from the construction of filtered limits in $\RGeom$ (see \S \ref{inftyfiltlim}) that
we can describe objects $Y \in \SSet_{/BG}$ informally as follows: $Y$ associates to
each open subgroup $U \subseteq G$ a space $Y^{U}$ of {\it $U$-fixed points}, which depends functorially on the finite $G$-space $G/U$. Moreover, if $U$ is a normal subgroup of $V$, then
the natural map from $Y^{V}$ to the (homotopy) fixed point space $(Y^{U})^{V/U}$ should be a homotopy equivalence.
\end{remark}

\begin{remark}
By refining the construction above, it is possible to construct a functor
$$ \Pro(\SSet) \rightarrow \RGeom$$
$$ \hat{X} \mapsto \SSet_{/ \hat{X} }.$$
This functor has a left adjoint, given by
$$ \calX \mapsto \Sh( \calX ).$$
\end{remark}

\begin{warning}
If $\hat{X}$ is a pro-space, then the shape of $\SSet_{/\hat{X}}$ is not necessarily equivalent to $\hat{X}$. In general we have only a counit morphism
$$ \Sh ( \SSet_{/ \hat{X} } ) \rightarrow \hat{X}.$$
\end{warning}
