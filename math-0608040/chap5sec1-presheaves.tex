\section{$\infty$-Categories of Presheaves}\label{c5s1}

\setcounter{theorem}{0}

The category of sets plays a central role in classical category theory. The primary reason for this is Yoneda's lemma, which asserts that for any category $\calC$, the ``Yoneda embedding''
$$j: \calC \rightarrow \Set^{\calC^{op}}$$
$$ C \mapsto \Hom_{\calC}(\bigdot, C)$$\index{gen}{Yoneda embedding!classical}
is fully faithful. Consequently, objects in $\calC$ can be thought of as a kind of ``generalized sets'', and various questions about the category $\calC$ can be reduced to questions about the category of sets.

If $\calC$ is an $\infty$-category, then the mapping {\em sets} of the above discussion should be replaced by mapping {\em spaces}. Consequently, one should expect the Yoneda embedding to take values in presheaves of {\em spaces}, rather than presheaves of sets. To formalize this, 
we introduce the following notation:

\begin{definition}\index{not}{PcalC@$\calP(\calC)$}
Let $S$ be a simplicial set. We let $\calP(S)$ denote the simplicial set
$\Fun(S^{op}, \SSet)$; here $\SSet$ denotes the $\infty$-category of spaces defined in \S \ref{introducingspaces}. We will refer
to $\calP(S)$ as the {\it $\infty$-category of presheaves on $S$}.\index{gen}{presheaf}
\end{definition}

\begin{remark}
More generally, for any $\infty$-category $\calC$, we might refer to
$\Fun( S^{op}, \calC)$ as the {\it $\infty$-category of $\calC$-valued presheaves on $S$}. 
Unless otherwise specified, the word ``presheaf'' will always refer to a $\SSet$-valued presheaf.
This is somewhat nonstandard terminology: one usually understands
the term ``presheaf'' to refer to a presheaf of sets, rather than
a presheaf of spaces. The shift in terminology is justified by the
fact that the important role of $\Set$ in ordinary category theory
is taken on by $\SSet$ in the $\infty$-categorical setting.
\end{remark}

Our goal in this section is to establish the basic properties of $\calP(S)$. We begin in \S \ref{presheaf3} by reviewing two other possible definitions of $\calP(S)$: one via the theory of right fibrations over $S$, another via simplicial presheaves on the category $\sCoNerve[S]$. Using the ``straightening'' results of \S \ref{fullun} and \S \ref{quasilimit4}, we will show that all three of these definitions are equivalent.

The presheaf $\infty$-categories $\calP(S)$ are examples of {\em presentable} $\infty$-categories (see \S \ref{c5s6}). In particular, each $\calP(S)$ admits small limits and colimits. We will give a proof of this assertion in \S \ref{presheaf2}, by reducing to the case where $S$ is a point.

The main question regarding the $\infty$-category $\calP(S)$ is how it relates to the original simplicial set $S$. In \S \ref{presheaf1} we will construct a map $j: S \rightarrow \calP(S)$, which is an $\infty$-categorical analogue of the usual Yoneda embedding. Just as in classical category theory, the Yoneda embedding is fully faithful. In particular, we note that any $\infty$-category $\calC$ can be embedded in a larger $\infty$-category which admits limits and colimits; this observation allows us to construct an {\it idempotent completion} of $\calC$, which we will study in \S \ref{surot}.

In \S \ref{presheaf4}, we will characterize the $\infty$-category $\calP(S)$ in terms of the Yoneda embedding $j: S \rightarrow \calP(S)$. Roughly speaking, we will show that $\calP(S)$ is freely generated by $S$ under colimits (Theorem \ref{charpresheaf}). In particular, if $\calC$ is a category which admits colimits, then any diagram $f: S \rightarrow \calC$ extends uniquely (up to homotopy) to a functor $F: \calP(S) \rightarrow \calC$. In \S \ref{completecomp}, we will give a criterion for determining whether or not $F$ is an equivalence.

\subsection{Other Models for $\calP(S)$}\label{presheaf3}

Let $S$ be a simplicial set. We have defined the $\infty$-category $\calP(S)$ of presheaves on $S$ to be the mapping space $\Fun(S^{op}, \SSet)$. However, there are several equivalent models which would serve equally well; we discuss two of them in this section.

Let $\calP'_{\Delta}(S)$ denote the full subcategory of
$(\sSet)_{/S}$ spanned by the right fibrations $X \rightarrow S$. We define $\calP'(S)$ to be the simplicial nerve $\sNerve(\calP'_{\Delta}(S))$. Because $\calP'_{\Delta}(S)$ is a fibrant simplicial category, $\calP'(S)$ is an $\infty$-category. We will see in a moment that $\calP'(S)$ is (naturally) equivalent to $\calP(S)$. In order to do this, we need to introduce a third model.

Let $\phi: \sCoNerve[S]^{op} \rightarrow \calC$ be an equivalence of simplicial categories.
Let $\Set_{\Delta}^{\calC}$ denote the category of simplicial functors $\calC \rightarrow \sSet$ (which we may view as simplicial presheaves on $\calC^{op}$). We regard $\Set_{\Delta}^{\calC}$ as endowed with the {\em projective} model structure defined in \S \ref{quasilimit3}. With respect to this structure,
$\Set_{\Delta}^{\calC}$ is a simplicial model category; we let $\calP''_{\Delta}(\phi) =  (\Set_{\Delta}^{\calC})^{\degree}$ denote the full simplicial subcategory consisting of fibrant-cofibrant objects, and we define $\calP''(\phi)$ to be the simplicial nerve $\sNerve(\calP''_{\Delta}(\phi))$.

We are now ready to describe the relationship between these different models:

\begin{proposition}\label{othermod}
Let $S$ be a simplicial set, and let $\phi: \sCoNerve[S]^{op} \rightarrow \calC$ be
an equivalence of simplicial categories. Then there are $($canonical$)$ equivalences
of $\infty$-categories
$$ \calP(S) \stackrel{f}{\leftarrow} \calP''(\phi) \stackrel{g}{\rightarrow} \calP'(S).$$
\end{proposition}

\begin{proof}
The map $f$ was constructed in Proposition \ref{gumby444}; it therefore suffices to give a construction of $g$. 

Recall that the category $(\sSet)_{/S}$ of simplicial sets over $S$ is endowed model structure, the {\it contravariant} model structure defined in \S \ref{contrasec}. Moreover, this model structure is simplicial (Proposition \ref{natsim}) and the fibrant objects are precisely the right fibrations over $S$ (Corollary \ref{usewhere1}). Thus, we may identify
$\calP'_{\Delta}(S)$ with the simplicial category $(\sSet)_{/S}^{\degree}$ of fibrant-cofibrant objects of $(\sSet)_{/S}$. 

According to Theorem \ref{struns}, the straightening and unstraightening functors
$(\St_{\phi}, \Un_{\phi})$ determine a Quillen equivalence between
$(\sSet)^{\calC}$ and $(\sSet)_{/S}$. Moreover, for any $X \in (\sSet)_{/S}$ and any simplicial set $K$, there is a natural chain of equivalences
$$ \St_{\phi} (X \times K) \rightarrow (\St_{\phi} X) \otimes |K|_{Q^{\bigdot}} \rightarrow
(\St_{\phi} X) \otimes K.$$
(The fact that the first map is an equivalence follows easily from Proposition \ref{spek3}.)
It follows from Proposition \ref{weakcompatequiv} that $\Un_{\phi}$ is endowed with the structure of a simplicial functor, and induces an equivalence of simplicial categories
$$ (\Set_{\Delta}^{\calC})^{\degree} \rightarrow  (\sSet)_{/S}^{\degree}.$$
We obtain the desired equivalence $g$ by passing to the simplicial nerve.
\end{proof}

\subsection{Colimits in $\infty$-Categories of Functors}\label{presheaf2}

Let $S$ be an arbitrary simplicial set. Our goal in this section is to prove that the $\infty$-category $\calP(S)$ of presheaves on $S$ admits small limits and colimits. There are (at least) three approaches to proving this:

\begin{itemize}
\item[$(1)$] According to Proposition \ref{othermod}, we may identify $\calP(S)$ with the
$\infty$-category underlying the simplicial model category $\Set_{\Delta}^{\sCoNerve[S]^{op}}$. We can then deduce the existence of limits and colimits in $\calP(S)$ by invoking Corollary \ref{limitsinmodel}.

\item[$(2)$] Since the $\infty$-category $\SSet$ classifies left fibrations, the $\infty$-category
$\calP(S)$ classifies left fibrations over $S^{op}$: in other words, homotopy classes of maps
$K \rightarrow \calP(S)$ can be identified with equivalence classes of left fibrations
$X \rightarrow K \times S^{op}$. It is possible to generalize Proposition \ref{charspacecolimit} and
Corollary \ref{charspacelimit} to describe limits and colimits in $\calP(S)$ entirely in the language of left fibrations. The existence problem can then be solved by exhibiting explicit constructions of left fibrations.

\item[$(3)$] Applying either $(1)$ or $(2)$ in the case where $S$ is a point, we can deduce that
the $\infty$-category $\SSet \simeq \calP( \ast)$ admits limits and colimits. We can then attempt to deduce the same result for $\calP(S) = \Fun( S^{op}, \SSet)$ using a general result about (co)limits in functor categories (Proposition \ref{limiteval}). 
\end{itemize}

Although approach $(1)$ is probably the quickest, we will adopt approach $(3)$ because it gives additional information: our proof will show that the formation of limits and colimits in $\calP(S)$ are computed pointwise. The same proof will also apply to $\infty$-categories of $\calC$-valued presheaves in the case where $\calC$ is not necessarily the $\infty$-category $\SSet$ of spaces.

\begin{lemma}\label{topaz2}
Let $q: Y \rightarrow S$ be a Cartesian fibration of simplicial sets, and let
$\calC = \bHom_{S}(S,Y)$ denote the $\infty$-category of sections of $q$. Let
$p: S \rightarrow Y$ be an object of $\calC$ having the property that $p(s)$ is an initial object of the 
fiber $Y_{s}$ for each vertex $s$ of $S$. Then $p$ is an initial object of $\calC$.
\end{lemma}

\begin{proof}
By Proposition \ref{colimfam}, the map $Y^{p_S/} \rightarrow S$ is a Cartesian fibration. By hypothesis, for each vertex $s$ of $S$, the map $Y^{p_S/} \times_{S} \{s\}  \rightarrow  Y_s$
is a trivial fibration. It follows that the projection $Y^{p_S/} \rightarrow Y$ is an equivalence of Cartesian fibrations over $S$, and therefore a categorical equivalence; taking sections over $S$ we obtain another categorical equivalence
$$ \bHom_{S}( S, Y^{p_S/} ) \rightarrow \bHom_{S}(Y,S).$$
But this map is just the left fibration $j: \calC^{p/} \rightarrow \calC$; it follows that $j$ is a categorical equivalence. Applying Propostion \ref{apple1} to the diagram
$$ \xymatrix{ \calC^{p/} \ar[dr]^{j} \ar[rr]^{j} & & \calC \ar[dl]^{\id_{\calC}} \\
& \calC,}$$
we deduce that $j$ induces categorical equivalences $\calC_{p/} \times_{\calC} \{t\} \rightarrow \{t\}$ for each vertex $t$ of $Q$. Thus the fibers of $j$ are contractible Kan complexes, so that $j$ is a trivial fibration (by Lemma \ref{toothie}) and $p$ is an initial object of $\calC$, as desired.
\end{proof}

\begin{proposition}\label{limiteval}\index{gen}{colimit!in a functor category}\index{gen}{limit!in a functor category}
Let $K$ be a simplicial set, $q: X \rightarrow S$ a Cartesian fibration, and
$p: K \rightarrow \bHom_{S}(S,X)$ a diagram.
For each vertex $s$ of $S$, 
we let $p_s: K \rightarrow X_{s}$ be the induced map. Suppose,
furthermore, that each $p_s$ has a colimit in the $\infty$-category $X_{s}$. Then:

\begin{itemize}
\item[$(1)$] There exists a map $\overline{p}: K \diamond \Delta^0 \rightarrow \bHom_{S}(S,X)$
which extends $p$ and induces a colimit diagram $\overline{p}: K \diamond
\Delta^0 \rightarrow X_{s}$, for each vertex $s \in S$.
                                                                                                                                                                                                                                                                                                                                                                                                                                                                                                       \item[$(2)$] An arbitrary extension $\overline{p}: K \diamond \Delta^0 \rightarrow \bHom_{S}(S,X)$
of $p$ is a colimit for $p$ if and only if each $\overline{p}_s: K \diamond
\Delta^0 \rightarrow X_{s}$ is a colimit for $p_s$.
\end{itemize}
\end{proposition}

\begin{proof}
Choose a factorization $K \rightarrow K' \rightarrow \bHom_{S}(S,X)$ of $p$,
where $K \rightarrow K'$ is inner anodyne (and therefore a
categorical equivalence) and $K' \rightarrow \calC^S$ is an inner fibration (so that $K'$ is an $\infty$-category). The map $K \rightarrow K'$ is a categorical equivalence, and therefore cofinal. We are free to replace $K$ by $K'$, and may thereby assume 
that $K$ is an $\infty$-category.

We apply Proposition \ref{familycolimit} to the Cartesian
fibration $X \rightarrow S$ and the diagram $p_S: K
\times S \rightarrow X$ determined by the map $p$. We
deduce that there exists a map $$\overline{p}_S: (K \times S) \diamond_S S = (K
\diamond \Delta^0) \times S \rightarrow X$$ having the property that
its restriction to the fiber over each $s \in S$ is a colimit of
$p_s$; this proves $(1)$.

The ``if'' direction of $(2)$ follows immediately from Lemma \ref{topaz2}. The ``only if'' follows
from $(1)$ and the fact that colimits, when they exist, are unique up to equivalence.
\end{proof}

\begin{corollary}
Let $K$ and $S$ be simplicial sets, and let $\calC$ be an $\infty$-category which admits $K$-indexed colimits. Then:
\begin{itemize}
\item[$(1)$] The $\infty$-category $\Fun(S, \calC)$ admits $K$-indexed colimits.
\item[$(2)$] A map $K^{\triangleright} \rightarrow \Fun(S,\calC)$ is a colimit diagram if and only if,
for each vertex $s \in S$, the induced map $K^{\triangleright} \rightarrow \calC$ is a colimit diagram.
\end{itemize}
\end{corollary}

\begin{proof}
Apply Proposition \ref{limiteval} to the projection $\calC \times S \rightarrow S$.
\end{proof}

\begin{corollary}\label{storum}
Let $S$ be a simplicial set. The $\infty$-category $\calP(S)$ of presheaves on $S$ admits all small limits and colimits.
\end{corollary}

\subsection{Yoneda's Lemma}\label{presheaf1}

In this section, we will construct the $\infty$-categorical analogue of the Yoneda embedding, and prove that it is fully faithful. We begin with a somewhat naive approach, based on the formalism of simplicial categories. We note that an analogoue of Yoneda's Lemma is valid in enriched category
theory (with the usual proof). Namely, suppose that $\calC$ is a category enriched over another category $\calE$. Then there is an ``enriched Yoneda embedding"
$$ i: \calC \rightarrow \calE^{\calC^{op}}$$
$$ X \mapsto \bHom_{\calC}( \bigdot, X).$$\index{gen}{Yoneda embedding!simplicial}

Consequently, for any simplicial
category $\calC$, one obtains a fully faithful embedding $i$ of
$\calC$ into the simplicial category $\bHom_{\sCat}(\calC^{op}, \sSet)$ of
simplicial functors from $\calC^{op}$ into $\sSet$. In fact, $i$
is fully faithful in the strong sense that it induces {\em
isomorphisms} of simplicial sets$$ \bHom_{\calC}(X,Y) \rightarrow
\bHom_{\Set_{\Delta}^{\calC^{op}}}(i(X), i(Y)),$$ rather than merely
weak homotopy equivalences. Unfortunately, this assertion does not
necessarily have any $\infty$-categorical content, because the 
simplicial category $\Set_{\Delta}^{\calC^{op}}$ does not generally
represent the correct $\infty$-category of functors from
$\calC^{op}$ to $\sSet$.

Let us describe an analogous construction in the setting of $\infty$-categories. Let $K$ be a simplicial set, and let
$\calC = \sCoNerve[K]$. Then $\calC$ is a simplicial category, so
$$ (X,Y) \mapsto \Sing|\Hom_{\calC}(X,Y)|$$
determines a simplicial functor from
$ \calC^{op} \times \calC$ to the category $\Kan$.
The functor $\sCoNerve$ does not commute with products, but there
exists a natural map $\sCoNerve[K^{op} \times K] \rightarrow
\calC^{op} \times \calC$. Composing with this map, we obtain a map
of simplicial sets
$$ \sCoNerve[K^{op} \times K] \rightarrow \Kan.$$
Passing to the adjoint, we obtain a map of simplicial sets
$K^{op} \times K \rightarrow \SSet,$ which we can identify
with $$j: K \rightarrow
\Fun(K^{op}, \SSet) = \calP(K).$$
We shall refer to $j$ (or, more generally, any map equivalent to
$j$) as the {\it Yoneda embedding}.\index{gen}{Yoneda embedding}

\begin{proposition}[$\infty$-Categorical Yoneda Lemma]\label{fulfaith}\index{gen}{Yoneda's Lemma}
Let $K$ be a simplicial set. Then the Yoneda embedding $j: K
\rightarrow \calP(K)$ is fully faithful.
\end{proposition}

\begin{proof}
Let $\calC' = \Sing | \sCoNerve[K^{op}] |$ be the ``fibrant
replacement'' for $\calC=\sCoNerve[K^{op}]$. We endow
$\Set_{\Delta}^{\calC'}$ with the {\em projective} model structure described in \S \ref{quasilimit3}.

We note that the Yoneda embedding factors as a composition
$$ K \stackrel{j'}{\rightarrow} \sNerve( (\Set_{\Delta}^{\calC'})^{\degree} )
\stackrel{j''}{\rightarrow} \Fun( K^{op}, \SSet),$$ where $j''$ is the map of Proposition \ref{gumby444} and consequently a categorical equivalence. It therefore suffices to
prove that $j'$ is fully faithful. For this, we need only show
that the adjoint map
$$ J: \sCoNerve[K] \rightarrow \Set_{\Delta}^{\calC'}.$$
is a fully-faithful functor between simplicial categories. We now observe that
$J$ is the composition of an equivalence $\sCoNerve[K] \rightarrow (\calC')^{op}$
with the (simplicial enriched) Yoneda embedding 
$(\calC')^{op} \rightarrow \Set_{\Delta}^{\calC'}$, which is fully faithful
in virtue of the classical (simplicially enriched) version of Yoneda's Lemma.
\end{proof}

We conclude by establishing another pleasant property of the Yoneda
embedding:

\begin{proposition}\label{yonedaprop}\index{gen}{Yoneda embedding!and limits}
Let $\calC$ be a small $\infty$-category, and $j: \calC \rightarrow \calP(\calC)$ the Yoneda embedding. Then $j$ preserves all limits which exist in $\calC$.
\end{proposition}

\begin{proof}
Let $p: K \rightarrow \calC$ be a diagram having a limit in $\calC$. We
wish to show that $j$ carries any limit for $p$ to a limit of
$j \circ p$. Choose a category $\calI$ and a cofinal map
$N(\calI^{op}) \rightarrow K^{op}$ (the existence of which is guaranteed by Proposition \ref{cofinalcategories}) Replacing $K$ by $\Nerve(\calI)$, we may suppose that $K$ is
the nerve of a category. Let $\overline{p}: \Nerve(\calI)^{\triangleleft} \rightarrow
\calC$ be a limit for $p$.

We recall the definition of the Yoneda embedding. It involves the
choice of an equivalence $\sCoNerve[\calC] \rightarrow
\calD$, where $\calD$ is a fibrant simplicial category. For
definiteness, we took $\calD$ to be $\Sing |\sCoNerve[\calC]|$. However, we could just as well
choose some other fibrant simplicial category $\calD'$ equivalent to $\sCoNerve[\calC]$ and
obtain a ``modified Yoneda embedding'' $j': \calC \rightarrow \calP(\calC)$; it is easy to see that
$j'$ and $j$ are equivalent functors, so it suffices to show that $j'$ preserves the limit of $p$.
Using Corollary \ref{strictify}, we may suppose that $\overline{p}$ is obtained
from a functor between simplicial categories
$\overline{q}: \{x\} \star \calI \rightarrow \calD$ by passing to the nerve. According to Theorem \ref{colimcomparee}, $\overline{q}$ is a homotopy limit of $q = \overline{q} | \calI$.
Consequently, for each object $Z \in \calD$, the induced functor
$$ \overline{q}_Z: I \mapsto \Hom_{\calD}(Z, \overline{q}(I))$$
is a homotopy limit of $q_Z = \overline{q}_Z|\calI$. Taking $Z$ to be the image of an object
$C$ of $\calC$, we deduce that
$$ \Nerve(\calI)^{\triangleleft} \rightarrow \calC \stackrel{j'}{\rightarrow} \calP(\calC) \rightarrow \SSet$$
is a limit for its restriction to $\Nerve(\calI)$, where the map on the right is given by ``evaluation at $C$''. Proposition \ref{limiteval} now implies that $j' \circ \overline{p}$ is a limit for $j' \circ p$, as desired.
\end{proof}

\subsection{Idempotent Completions}\label{surot}

Recall that an $\infty$-category $\calC$ is said to be {\it idempotent complete} if every
functor $\Idem \rightarrow \calC$ admits a colimit in $\calC$ (see \S \ref{retrus}).
If an $\infty$-category $\calC$ is not idempotent complete, then we can attempt to correct the situation by passing to a larger $\infty$-category.

\begin{definition}\index{gen}{idempotent completion}
Let $f: \calC \rightarrow \calD$ be a functor between $\infty$-categories. We will say that $f$ {\it exhibits $\calD$ as an idempotent completion of $\calC$} if $\calD$ is idempotent complete, $f$ is fully faithful, and every object of $\calD$ is a retract of $f(C)$, for some object $C \in \calC$.
\end{definition}

Our goal in this section is to show that $\infty$-category $\calC$ has an idempotent completion $\calD$, which is unique up to equivalence. The uniqueness is a consequence of Proposition \ref{charidemcomp}, proven below. The existence question is much easier to address.


\begin{proposition}\label{idmcoo}
Let $\calC$ be an $\infty$-category. Then $\calC$ admits an idempotent completion.
\end{proposition}

\begin{proof}
Enlarging the universe if necessary, we may suppose that $\calC$ is small. Let
$\calC'$ denote the full subcategory of $\calP(\calC)$ spanned by those objects which are retracts of objects which belong to the image of the Yoneda embedding $j: \calC \rightarrow \calP(\calC)$.
Then $\calC'$ is stable under retracts in $\calP(\calC)$. Since $\calP(\calC)$ admits all small colimits, Corollary \ref{swwe} implies that
$\calP(\calC)$ is idempotent complete. It follows that $\calC'$ is idempotent complete. Proposition \ref{fulfaith} implies that the Yoneda embedding $j: \calC \rightarrow \calC'$ is fully faithful, and therefore exhibits $\calC'$ as an idempotent completion of $\calC$.
\end{proof}

We now address the question of uniqueness for idempotent completions. First, we need a few preliminary results.

\begin{lemma}\label{sweeble}
Let $\calC$ be an $\infty$-category which is idempotent complete, and let $p: K \rightarrow \calC$ be a diagram. Then $\calC_{/p}$ and $\calC_{p/}$ are also idempotent complete.
\end{lemma}

\begin{proof}
By symmetry, it will suffice to prove that $\calC_{/p}$ is idempotent complete. Let
$q: \calC_{/p} \rightarrow \calC$ be the associated right fibration, and let
$F: \Idem \rightarrow \calC_{/p}$ be an idempotent. We will show that $F$ has a limit.
Since $\calC$ is idempotent complete,
$q \circ F$ has a limit $\overline{q \circ F}: \Idem^{\triangleleft} \rightarrow \calC$. Consider the lifting problem
$$ \xymatrix{ \Idem \ar@{^{(}->}[d] \ar[r]^{F} & \calC_{/p} \ar[d]^{q} \\
\Idem^{\triangleleft} \ar[r]^{ \overline{q \circ F} } \ar@{-->}[ur]^{\overline{F}} & \calC. }$$
The right vertical map is a right fibration, and the left vertical map is right anodyne (Lemma \ref{chotle2}), so that there exists a dotted arrow $\overline{F}$ as indicated. Using Proposition \ref{goeselse}, we deduce that $\overline{F}$ is a limit of $F$. 
\end{proof}

\begin{lemma}\label{sweerum}
Let $f: \calC \rightarrow \calD$ be a functor between $\infty$-categories which exhibits $\calD$ as an idempotent completion of $\calC$, and let $p: K \rightarrow \calD$ be a diagram. Then the induced map $f_{/p}: \calC \times_{\calD} \calD_{/p} \rightarrow \calD_{/p}$ exhibits $\calD_{/p}$ as an idempotent completion of $\calC \times_{\calD} \calD_{/p}$.
\end{lemma}

\begin{proof}
Lemma \ref{sweeble} asserts that $\calD_{/p}$ is idempotent complete. We must show that every object $\overline{D} \in \calD_{/p}$ is a retract of $f_{/p}( \overline{C})$, for some
$\overline{C} \in \calC \times_{\calD} \calD_{/p}$. Let $q: \calD_{/p} \rightarrow \calD$ be the projection, and let $D = q( \overline{D})$. Since $f$ exhibits $\calD$ as an idempotent completion of $\calC$, there is a diagram
$$ \xymatrix{ & f(C) \ar[dr] & \\
D' \ar[rr]^{g} \ar[ur] & & D }$$
in $\calD$, where $g$ is an equivalence. Since $q$ is a right fibration, we can lift this to a diagram
$$ \xymatrix{ & \overline{f(C)} \ar[dr] & \\
\overline{D}' \ar[rr]^{\overline{g}} \ar[ur] & & \overline{D} }$$
in $\calD_{/q}$. Since $\overline{g}$ is $q$-Cartesian and $g$ is an equivalence,
$\overline{g}$ is ann equivalence. It follows that $\overline{D}$ is a retract of
$\overline{f(C)}$. By construction, $\overline{f(C)} = f_{/p}(\overline{C})$ for 
an appropriately chosen object $\overline{C} \in \calC \times_{\calD} \calD_{/p}$.  
\end{proof}

\begin{lemma}\label{wequivvv}
Let $f: \calC \rightarrow \calD$ be a functor between $\infty$-categories which exhibits $\calD$ as an idempotent completion of $\calC$. Suppose that $\calD$ has an initial object $\emptyset$. Then
$\calC$ is weakly contractible as a simplicial set.
\end{lemma}

\begin{proof}
Without loss of generality, we may suppose that $\calC$ is a full subcategory of $\calD$ and that $f$ is the inclusion. Since $f$ exhibits $\calD$ as an idempotent completion of $\calC$, the initial object $\emptyset$ of $\calD$ admits a map $f: C \rightarrow \emptyset$, where $C \in \calC$.
The $\infty$-category $\calC_{C/}$ has an initial object, and
is therefore weakly contractible. Since composition
$$ \calC_{f/} \rightarrow \calC_{C/} \rightarrow \calC$$
is both a weak homotopy equivalence (in fact, a trivial fibration) and weakly nullhomotopic, we
conclude that $\calC$ is weakly contractible.
\end{proof}

\begin{lemma}\label{sweetrum}
Let $f: \calC \rightarrow \calD$ be a functor between $\infty$-categories which exhibits $\calD$ as an idempotent completion of $\calC$. Then $f$ is cofinal.
\end{lemma}

\begin{proof}
According to Theorem \ref{hollowtt}, it suffices to prove that for every object $D \in \calD$, 
simplicial set $\calC \times_{\calD} \calD_{D/}$ is weakly contractible. Lemma \ref{sweerum} asserts that $f_{D/}$ is also an idempotent completion, and Lemma \ref{wequivvv} completes the proof.
\end{proof}

\begin{lemma}\label{honeybeen}
Let $F: \calC \rightarrow \calD$ be a functor between $\infty$-categories, and let
$\calC^{0} \subseteq \calC$ be a full subcategory such that the inclusion exhibits
$\calC$ as an idempotent completion of $\calC^{0}$. Then $F$ is a left Kan extension
of $F|\calC^{0}$. 
\end{lemma}

\begin{proof}
We must show that for every object $C \in \calC$, the composite map
$$ (\calC^0_{/C})^{\triangleright} \rightarrow (\calC_{/C})^{\triangleright}
\stackrel{G}{\rightarrow} \calC \stackrel{F}{\rightarrow} \calD$$
is a colimit diagram in $\calD$. Lemma \ref{sweerum} guarantees that 
$\calC^0_{/C} \subseteq \calC_{/C}$ is an idempotent completion, and therefore cofinal by Lemma \ref{sweetrum}. Consequently, it suffices prove that $F \circ G$ is a colimit diagram, which is obvious.
\end{proof}

\begin{lemma}\label{beenhoney}
Let $\calC$ and $\calD$ be $\infty$-categories which are idempotent complete, and let
$\calC^{0} \subseteq \calC$ be a full subcategory such that the inclusion exhibits
$\calC$ as an idempotent completion of $\calC^{0}$. Then any functor
$F_0: \calC^{0} \rightarrow \calD$ has an extension $F: \calC \rightarrow \calD$.
\end{lemma}

\begin{proof}
We will suppose that the $\infty$-categories $\calC$ and $\calD$ are small. Let $\calP(\calD)$ be the $\infty$-category of presheaves on $\calD$ (see \S \ref{c5s1}), $j: \calD \rightarrow \calP(\calD)$ the Yoneda embedding, and $\calD'$ the essential image of $j$. According to Proposition \ref{princex}, it will suffice to prove that $j \circ F_0$ can be extended to a functor $F': \calC \rightarrow \calD'$.
Since $\calP(\calD)$ admits small colimits, we can choose $F': \calC \rightarrow \calP(\calD)$
to be a left Kan extension of $j \circ F_0$. Every object of $\calC$ is a retract of an object of $\calC^{0}$, so that every object in the essential image of $F'$ is a retract of the Yoneda image of an object of $\calD$. Since $\calD$ is idempotent complete, it follows that the $F'$ factors through $\calD'$.
\end{proof}

\begin{proposition}\label{charidemcomp}\index{gen}{idempotent completion!universal property of}
Let $f: \calC \rightarrow \calD$ be a functor which exhibits $\calD$ as the idempotent completion of $\calC$, and let $\calE$ be an $\infty$-category which is idempotent complete. Then composition with $f$ induces an equivalence of $\infty$-categories
 $f^{\ast}: \Fun(\calD, \calE) \rightarrow \Fun(\calC, \calE)$.
\end{proposition}

\begin{proof}
Without loss of generality, we may suppose that $f$ is the inclusion of a full subcategory. In this case, we combine Lemmas \ref{honeybeen}, \ref{beenhoney}, and Proposition \ref{lklk} to deduce that $f^{\ast}$ is a trivial fibration.
\end{proof}

\begin{remark}
Let $\calC$ be a small $\infty$-category, and let $f: \calC \rightarrow \calC'$ be an idempotent completion of $\calC$. The proof of Proposition \ref{idmcoo} shows that $\calC'$ is equivalent to a full subcategory of $\calP(\calC)$, and therefore locally small (see \S \ref{locbrend}). Moreover, every object of $\h{\calC'}$ 
is the image of some retraction map in $\h{\calC}$; it follows that the set of equivalence classes of objects in $\calC'$ is bounded in size. It follows that $\calC'$ is essentially small.
\end{remark}

\subsection{The Universal Property of $\calP(S)$}\label{presheaf4}

Let $S$ be a (small) simplicial set. We have defined $\calP(S)$ to be the $\infty$-category of maps from $S^{op}$ into the $\infty$-category $\SSet$ of spaces. Informally, we may view $\calP(S)$ as the limit of a diagram in the $\infty$-bicategory of (large) $\infty$-categories: namely, the constant diagram carrying $S^{op}$ to $\SSet$. In more concrete terms, our definition of $\calP(S)$ leads immediately to a characaterization of $\calP(S)$ by a universal mapping property: for every $\infty$-category $\calC$, there is an equivalence of $\infty$-categories (in fact an isomorphism of simplicial sets)
$$ \Fun(\calC, \calP(S)) \simeq \Fun(\calC \times S^{op}, \SSet).$$
The goal of this section is to give a dual characterization of $\calP(S)$: it may also be viewed
as a {\em colimit} of copies of $\SSet$, indexed by $S$. However, this colimit needs to be understood in an appropriate $\infty$-bicategory of $\infty$-categories where the morphisms are given by {\em colimit preserving} functors. In other words, we will show that $\calP(S)$ is in some sense ``freely generated'' by $S$ under small colimits (Theorem \ref{charpresheaf}). First, we need to introduce a bit of notation.

\begin{notation}
Let $\calC$ be an $\infty$-category and $S$ a simplicial set. We will let
$\LFun( \calP(S), \calC)$ denote the full subcategory of
$\Fun( \calP(S), \calC)$ spanned by those functors $\calP(S) \rightarrow \calC$
which preserve small colimits. 

The motivation for this notation is as follows: in \S \ref{afunc5}, we will use the notation
$\LFun( \calD, \calC)$ to denote the full subcategory of $\Fun( \calD, \calC)$ spanned by those functors which are {\em left adjoints}. In \S \ref{aftt}, we will see that when $\calD = \calP(S)$ (or, more generally, when $\calD$ is presentable), then a functor $\calD \rightarrow \calC$ is a left adjoint if and only if it preserves small colimits (see Corollary \ref{adjointfunctor} and Remark \ref{afi}). 
\end{notation}

We wish to prove that if $\calC$ is an $\infty$-category which admits small colimits, then any map $S \rightarrow \calC$ extends in an essentially unique fashion to a colimit-preserving functor $\calP(S) \rightarrow \calC$. To prove this, we need a second characterization of the colimit-preserving functors $f: \calP(S) \rightarrow \calC$: they are precisely those functors which are left Kan extensions of their restriction to the essential image of the Yoneda embedding.

\begin{lemma}\label{repco}\index{gen}{corepresentable!functor}
Let $S$ be a small simplicial set, let $s$ be a vertex of $S$, let
$e: \calP(S) \rightarrow \SSet$ be the map given by evaluation at $s$, and let
$f: \calC \rightarrow \calP(S)$ be the associated left fibration (see \S \ref{universalfib}). Then
$f$ is corepresentable by the object $j(s) \in \calP(S)$, where $j: S \rightarrow \calP(S)$ denotes the Yoneda embedding.
\end{lemma}

\begin{proof}
Without loss of generality, we may suppose that $S$ is an $\infty$-category.
We make use of the equivalent model $\calP'(S)$ of \S \ref{presheaf3}. Observe that the functor
$f: \calP(S) \rightarrow \SSet$ is equivalent to $f': \calP'(S) \rightarrow \SSet$, where
$f'$ is the nerve of the simplicial functor $\calP'_{\Delta}(S) \rightarrow \Kan$ which
associates to each left fibration $Y \rightarrow S$ the fiber $Y_{s} = Y \times_{S} \{s\}$. 
Furthermore, under the equivalence of $\calP(S)$ with $\calP'(S)$, the object $j(s)$
corresponds to a left fibration $X(s) \rightarrow S$ which is corepresented by $s$. Then
$X(s)$ contains an initial object $x$ lying over $s$. The choice of $x$ determines a point 
$\eta \in \pi_0 f'(X(s))$. According to Proposition \ref{reppfunc}, to show that $X(s)$ corepresents $f'$, it suffices to show that for every left fibration $X \rightarrow S$, the map
$$ \bHom_{S}( X(s), Y) \rightarrow Y_{s}, $$
given by evaluation at $x$, is a homotopy equivalence of Kan complexes. 
We may rewrite the space on the right hand side as $\bHom_{S}( \{x\}, Y)$. According
to Proposition \ref{natsim}, the covariant model structure on $(\sSet)_{/S}$ is compatible with the simplicial structure. It therefore suffices to prove that the inclusion $i: \{ x\} \subseteq X(s)$ is a
covariant equivalence. But this is clear, since $i$ is the inclusion of an initial object and therefore left anodyne.
\end{proof}

\begin{lemma}\label{longwait0}
Let $S$ be a small simplicial set, and let $j: S \rightarrow \calP(S)$ denote the Yoneda embedding. Then $\id_{\calP(S)}$ is a left Kan extension of $j$ along itself.
\end{lemma}

\begin{proof}
Let $\calC \subseteq \calP(S)$ denote the essential image of $j$. According to Proposition \ref{fulfaith}, $j$ induces an equivalence $S \rightarrow \calC$. It therefore suffices to prove that
$\id_{\calP(S)}$ is a left Kan extension of its restriction to $\calC$. Let $X$ be an object of $\calP(S)$; we must show that the natural map
$$ \phi: \calC_{/X}^{\triangleright} \subseteq \calP(S)_{/X}^{\triangleright} \rightarrow \calP(S)$$
is a colimit diagram.

According to Proposition \ref{limiteval}, it will suffice to prove that for each vertex $s$ of $S$, the map
$$ \phi_{s}: \calC_{/X}^{\triangleright} \rightarrow \SSet$$ given by composing $\phi$ with the evaluation map is a colimit diagram in $\SSet$. Let 
$\calD \rightarrow \calC_{/X}^{\triangleright}$ be the pullback of the universal left fibration along $\phi_{s}$, and let $\calD^0 \subseteq \calD$ be the preimage in $\calD$ of
$\calC_{/X} \subseteq \calC_{/X}^{\triangleright}$. According to Proposition \ref{charspacecolimit}, it will suffice to prove that the inclusion $\calD^{0} \subseteq \calD$ is a weak homotopy equivalence of simplicial sets.

Let $C = j(s)$. Let $\calE = \calC_{/X}^{\triangleright} \times_{\calP(S)} \calP(S)_{C/}$,
let $\calE^{0} = \calC_{/X} \times_{\calC} \calC_{C/} \subseteq \calE$, and let
$\calE^{1} = \calC_{/X} \times_{\calC} \{ \id_{C} \} \subseteq \calE^{0}$.
Lemma \ref{repco} implies that the left fibrations
$$ \calD \rightarrow \calC_{/X}^{\triangleright} \leftarrow \calE$$ are
equivalent. It therefore suffices to show that the inclusion $\calE^{0} \subseteq \calE$ is a weak homotopy equivalence. To prove this, we observe that both $\calE$ and $\calE^{0}$ contain
$\calE^{1}$ as a deformation retract (that is, there is a retraction $r: \calE \rightarrow \calE^{1}$ and a homotopy $\calE \times \Delta^1 \rightarrow \calE$ from $r$ to $\id_{\calE}$, so that
the inclusion $\calE^{1} \subseteq \calE$ is a homotopy equivalence; the situation for
$\calE^0$ is similar).
\end{proof}

\begin{lemma}\label{natee}
Let $$\xymatrix{ A \ar[rr]^{f} \ar[dr] & & B \ar[dl]^{g} \\
& S & }$$
be a diagram of simplicial sets. The following conditions are equivalent:

\begin{itemize}
\item[$(1)$] The map $f$ is a covariant equivalence in $(\sSet)_{/S}$.

\item[$(2)$] For every diagram $p: S \rightarrow \calC$ taking values in an $\infty$-category $\calC$, and every limit $\overline{p \circ g}: B^{\triangleleft} \rightarrow \calC$ of
$p \circ g$, the composition $\overline{p \circ g} \circ f^{\triangleleft}: A^{\triangleleft} \rightarrow \calC$ is a limit diagram.

\item[$(3)$] For every diagram $p: S \rightarrow \SSet$ taking values in the $\infty$-category
$\SSet$ of spaces, and every limit $\overline{p \circ g}: B^{\triangleleft} \rightarrow \SSet$ of
$p \circ g$, the composition $\overline{p \circ g} \circ f^{\triangleleft}: A^{\triangleleft} \rightarrow \SSet$ is a limit diagram.
\end{itemize}
\end{lemma}

\begin{proof}
The equivalence of $(1)$ and $(3)$ follows from Corollary \ref{needta} (and the definition of a contravariant equivalence). The implication $(2) \Rightarrow (3)$ is obvious. We show that $(3) \Rightarrow (2)$. Let $p: S \rightarrow \calC$ and $\overline{p \circ g}$ be as in $(2)$. 
Passing to a larger universe if necessary, we may suppose that $\calC$ is small.
For each object $C \in \calC$, let $j_C: \calC \rightarrow \SSet$ denote the composition of the Yoneda embedding $j: \calC \rightarrow \calP(\calC)$ with the map $\calP(\calC) \rightarrow \SSet$ given by evaluation at $C$. Combining Proposition \ref{yonedaprop} with Proposition \ref{limiteval}, we deduce that each $j_{C} \circ \overline{p \circ g}$ is a limit diagram. Applying $(3)$, we conclude that each $j_{C} \circ \overline{p \circ g} \circ f^{\triangleleft}$ is a limit diagram. We now apply Propositions \ref{yonedaprop} and \ref{limiteval} to conclude that $\overline{p \circ g} \circ f^{\triangleleft}$ is a limit diagram, as desired.
\end{proof}

\begin{lemma}\label{longwait1}\index{gen}{Yoneda embedding!and left Kan extensions}
Let $S$ be a small simplicial set, $j: S \rightarrow \calP(S)$ the Yoneda embedding,
let $\calC$ denote the full subcategory of $\calP(S)$ spanned by the objects $j(s)$, where
$s$ is a vertex of $S$, and let $\calD$ be an arbitrary $\infty$-category.

\begin{itemize}
\item[$(1)$] Let $f: \calP(S) \rightarrow \calD$ be a functor. Then $f$ is a left Kan extension of $f|\calC$ if and only if $f$ preserves small colimits.
\item[$(2)$] Suppose that $\calD$ admits small colimits, and let $f_0: \calC \rightarrow \calD$
be an arbitrary functor. There exists an extension $f: \calP(S) \rightarrow \calD$ which is a left Kan extension of $f_0 = f|\calC$.
\end{itemize}
\end{lemma}

\begin{proof}
Assertion $(2)$ follows immediately from Lemma \ref{kan2}, since the $\infty$-category
$\calC_{/X}$ is small for each object $X \in \calP(S)$. We will prove $(1)$. Suppose first that
$f$ preserves small colimits. We must show that for each $X \in \calP(S)$, the composition
$$ \calC_{/X}^{\triangleright} \stackrel{\delta}{\rightarrow} \calP(S) \stackrel{f}{\rightarrow} \calD$$
is a colimit diagram. Lemma \ref{longwait0} implies that $\delta$ is a colimit diagram; if $f$ preserves small colimits, then $f \circ \delta$ is also a colimit diagram.

Now suppose that $f$ is a left Kan extension of $f_0 = f | \calC$. We wish to prove that $f$
preserves small colimits. Let $K$ be a small simplicial set, and let
$\overline{p}: K^{\triangleright} \rightarrow \calP(S)$ be a colimit diagram. We must show that
$f \circ \overline{p}$ is also a colimit diagram.

Let $$\overline{\calE} = \calC \times_{ \Fun( \{0\}, \calP(S) } \Fun(\Delta^1,\calP(S) ) \times_{\Fun( \{1\}, \calP(S))} K^{\triangleright},$$
and let $\calE = \overline{\calE} \times_{ K^{\triangleright} } K \subseteq \overline{\calE}$. 
We have a commutative diagram
$$ \xymatrix{ \calE \ar[d] \ar[r] & \overline{\calE} \ar[d] \\
K \ar[r] & K^{\triangleright}. }$$
where the vertical arrows are coCartesian fibrations (Corollary \ref{tweezegork}). 
Let $\overline{\eta}: \overline{\calE} \diamond_{ K^{\triangleright} } K^{\triangleright} 
\rightarrow \calP(S)$ be the natural map, and $\eta = \overline{\eta} | \calE \diamond_{K} K$. 
Proposition \ref{longwait2} implies that $f \circ \eta$ exhibits $f \circ p$ as a left Kan extension of 
$f \circ (\eta | \calE)$ along $q|\calE$. Similarly, $f \circ \overline{\eta}$ exhibits $f \circ \overline{p}$ as a left Kan extension of $f \circ (\overline{\eta} | \overline{\calE})$. It will therefore suffice to prove
that every colimit of $f \circ (\overline{\eta} | \overline{\calE})$ is also a colimit of
$f \circ (\eta | \calE)$. According to Lemma \ref{natee}, it suffices to show that the inclusion
$\calE \subseteq \overline{\calE}$ is a contravariant equivalence in $(\sSet)_{/\calC}$.

Since the map $\overline{\calE} \rightarrow K^{\triangleright} \times \calC$ is a bivariant fibration, 
we can apply Proposition \ref{longwait5} to deduce that the map
$\overline{\calE}^{op} \rightarrow \calC^{op}$ is smooth. Similarly, $\calE^{op} \rightarrow \calC^{op}$ is smooth. According to Proposition \ref{longwait44}, the inclusion
$\calE \subseteq \overline{\calE}$ is a contravariant equivalence if and only if, for every
object $C \in \calC$, the inclusion of fibers $\calE_{C} \subseteq \overline{\calE}_{C}$ 
is a weak homotopy equivalence. Lemma \ref{repco} implies that
$\overline{\calE}_{C} \rightarrow K^{\triangleright}$
is equivalent to the left fibration given by the pullback of the universal left fibration
along the map
$$ K^{\triangleright} \stackrel{\overline{p}}{\rightarrow} \calP(S) \stackrel{s}{\rightarrow} \SSet.$$
We now conclude by applying Proposition \ref{charspacecolimit}, noting that
$\overline{p}$ is a colimit diagram by assumption and that $s$ preserves colimits by
Proposition \ref{limiteval}.
\end{proof}

\begin{theorem}\label{charpresheaf}\index{gen}{presheaf!universal property of $\calP(S)$}
Let $S$ be a small simplicial set, and let $\calC$ be an $\infty$-category which admits small colimits. Composition with the Yoneda embedding $j: S \rightarrow \calP(S)$ induces
an equivalence of $\infty$-categories
$$ \LFun( \calP(S), \calC) \rightarrow \Fun(S,\calC).$$
\end{theorem}

\begin{proof}
Combine Corollary \ref{leftkanextdef} with Lemma \ref{longwait1}.
\end{proof}

\begin{definition}
Let $\calC$ be an $\infty$-category. A full subcategory $\calC' \subseteq \calC$ is {\it stable under colimits} if, for any small diagram $p: K \rightarrow \calC'$ which has a colimit
$\overline{p}: K^{\triangleright} \rightarrow \calC$ in $\calC$, the map $\overline{p}$ factors through $\calC'$.

Let $\calC$ be an $\infty$-category which admits all small colimits. Let $A$ be a collection of objects of $\calC$. We will say that $A$ {\it generates $\calC$ under colimits} if the following condition is satisfied: for any full subcategory $\calC' \subseteq \calC$ containing every element of $A$, if $\calC'$ is stable under colimits, then $\calC = \calC'$. 

We say that a map $f: S \rightarrow \calC$ {\it generates $\calC$ under colimits} if the image
$f(S_0)$ generates $\calC$ under colimits.\index{gen}{generation under colimits}
\end{definition}

\begin{corollary}\label{gencolcot}
Let $S$ be a small simplicial set. Then the Yoneda embedding $j: S \rightarrow \calP(S)$ generates $\calP(S)$ under small colimits.
\end{corollary}

\begin{proof}
Let $\calC$ be the smallest full subcategory of $\calP(S)$ which contains the essential image of
$j$ and is stable under small colimits. Applying Theorem \ref{charpresheaf}, we deduce that
the diagram $j: S \rightarrow \calC$ is equivalent to $F \circ j$, for some colimit-preserving
functor $F: \calP(S) \rightarrow \calC$. We may regard $F$ as a colimit preserving functor
from $\calP(S)$ to itself. Applying Theorem \ref{charpresheaf} again, we deduce that $F$ is
equivalent to the identity functor from $\calP(S)$ to itself. It follows that every object
of $\calP(S)$ is equivalent to an object which lies in $\calC$, so that $\calC = \calP(S)$ as desired.
\end{proof}

\subsection{Complete Compactness}\label{completecomp}

Let $S$ be a small simplicial set, and $f: S \rightarrow \calC$ a diagram in an $\infty$-category $\calC$. Our goal in this section is to analyze the following question: when is the diagram $f: S \rightarrow \calC$ equivalent to the Yoneda embedding $j: S \rightarrow \calP(S)$?
An obvious necessary condition is that $\calC$ admit small colimits (Corollary \ref{storum}). 
Conversely, if $\calC$ admits small colimits, then Theorem \ref{charpresheaf} implies that $f$ is equivalent to $F \circ j$, where $F: \calP(S) \rightarrow \calC$ is a colimit-preserving functor.
We are now reduced to the question of deciding whether or not the functor $F$ is an equivalence.
There are two obvious necessary conditions for this to be so: $f$ must be fully faithful (Proposition \ref{fulfaith}), and $f$ must generate $\calC$ under colimits (Corollary \ref{gencolcot}). We will show that the converse holds, provided that the essential image of $f$ consists of {\em completely compact} objects of $\calC$ (see Definition \ref{complcompdef} below).

We begin by considering an arbitrary simplicial set $S$ and a vertex $s$ of $S$.
Composing the Yoneda embedding $j: S \rightarrow \calP(S)$ with the ``evaluation map''
$$\calP(S)  = \Fun(S^{op}, \SSet) \rightarrow
\Fun( \{s\}, \SSet) \simeq \SSet,$$
we obtain a map $j_{s}: S \rightarrow \SSet$. We will refer to $j_{s}$
as the {\it functor corepresented by $s$}.\index{gen}{corepresentable!functor}

\begin{remark}
The above definition makes sense even when the simplicial set $S$ is not small. However, in
this case we need to replace $\SSet$ (the simplicial nerve of the category of {\em small} Kan complexes) by the (very large) $\infty$-category $\widehat{\SSet}$, where $\hat{\SSet}$ is the simplicial nerve of the category of {\em all} Kan complexes (not necessarily small).
\end{remark}

\begin{definition}\label{complcompdef}\index{gen}{compact object!completely}\index{gen}{completely compact}
Let $\calC$ be an $\infty$-category which admits small colimits. We will say that an object
$C \in \calC$ is {\it completely compact} if the functor $j_{C}: \calC \rightarrow \widehat{\SSet}$
corepresented by $C$ preserves small colimits.
\end{definition}

The requirement that an object $C$ of an $\infty$-category $\calC$ be completely compact
is {\em very} restrictive (see Example \ref{tryu} below). We introduce this notion not because it is a generally useful one, but because it is relevant for the purpose of characterizing $\infty$-categories of presheaves.

Our first goal is to establish that the class of completely compact objects of $\calC$ is stable under retracts.

\begin{lemma}\label{compcompcomp}
Let $\calC$ be an $\infty$-category, $K$ a simplicial set, and $\overline{p}, \overline{q}: K^{\triangleright} \rightarrow \calC$ a pair of diagrams. Suppose $\overline{q}$ is a colimit diagram, and $\overline{p}$ is a retract of $\overline{q}$ in the $\infty$-category $\Fun(K^{\triangleright}, \calC)$. Then $\overline{p}$ is a colimit diagram.
\end{lemma}

\begin{proof}
Choose a map $\sigma: \Delta^2 \times K^{\triangleright} \rightarrow \calC$ such that
$\sigma | \{1\} \times K^{\triangleright} = \overline{q}$ and
$\sigma | \Delta^{ \{0,2\} } \times K^{\triangleright} = \overline{p} \circ \pi_{K^{\triangleright}}$. 
We have a commutative diagram of simplicial sets:
$$ \xymatrix{ \calC_{\sigma/} \ar[r] \ar[d] & \calC_{\sigma| \Delta^2 \times K/} \ar[d] \\
\calC_{\sigma | \Delta^{ \{1,2\} /} \times K^{\triangleright}} \ar[r]^{f} \ar[d] & \calC_{\sigma| \Delta^{ \{1,2\} } \times K/} \ar[d] \\
\calC_{\sigma| \{2\} \times K^{\triangleright}/} \ar[r]^{f'} & \calC_{\sigma| \{2\} \times K/}.}$$

We first claim that both vertical compositions are categorical equivalences. We give the argument for the right vertical composition; the other case is similar. We have a factorization
$$ \calC_{\sigma | \Delta^2 \times K/} \stackrel{g'}{\rightarrow}
\calC_{\sigma| \Delta^{ \{0,2\}} \times K/} \stackrel{g''}{\rightarrow} \calC_{\sigma | \{2\} \times K/}$$
where the $g'$ is a trivial fibration, and $g''$ admits a section $s$, where $s$ is also a section
of the trivial fibration $\calC_{/ \sigma| \Delta^{ \{0,2\} \times K}} \rightarrow \calC_{/ \sigma| \{0\} \times K}$. Consequently, $s$ and therefore also $g''$ are categorical equivalences.
It follows that the map $f'$ is a retract of $f$ in the homotopy category of $\sSet$ (taken with respect to the Joyal model structure). 

The map $f$ sits in a commutative diagram
$$ \xymatrix{ \calC_{\sigma | \Delta^{ \{1,2\}/ } \times K^{\triangleright}} \ar[r]^{f} \ar[d] & \calC_{ \sigma| \Delta^{ \{1,2\}/ } \times K} \ar[d] \\
\calC_{\overline{q}/} \ar[r] & \calC_{q/} }$$
where the vertical maps and the lower horizontal map are trivial fibrations. It follows that
$f$ is a categorical equivalence. Since $f'$ is a retract of $f$, $f'$ is also a categorical equivalence. Since $f'$ is a left fibration, we deduce that $f'$ is a trivial fibration (Corollary \ref{heath}), so that $\overline{p}$ is a colimit diagram as desired.
\end{proof}

\begin{lemma}\label{retcompact}
Let $\calC$ be an $\infty$-category which admits small colimits. Let $C$ and
$D$ be objects of $\calC$. Suppose that $C$ is completely compact, and that $D$
is a retract of $C$ (that is, there exist maps $f: D \rightarrow C$ and $r: C \rightarrow D$
with $r \circ f \simeq \id_{D}$. Then $D$ is completely compact. In particular, if $C$ and
$D$ are equivalent, then $D$ is completely compact.
\end{lemma}

\begin{proof}
Let $j: \calC^{op} \rightarrow \SSet^{\calC}$ denote the Yoneda embedding (for $\calC^{op})$. Since $D$ is a retract of $C$, $j(D)$ is a retract of $j(C)$. Let $\overline{p}: K^{\triangleright} \rightarrow \calC$ be a diagram. Then $j(D) \circ \overline{p}: K^{\triangleright} \rightarrow \SSet$
is a retract of $j(C) \circ \overline{p}: K^{\triangleright} \rightarrow \SSet$ in the $\infty$-category
$\Fun(K^{\triangleright}, \SSet)$. If $\overline{p}$ is a colimit diagram, then $j(C) \circ \overline{p}$ is a colimit diagram (since $C$ is completely compact). Lemma \ref{compcompcomp} now implies
that $j(D) \circ \overline{p}$ is a colimit diagram as well.
\end{proof}

In order to study the condition of complete compactness in more detail, it is convenient to introduce a slightly more general notion.

\begin{definition}\index{gen}{completely compact!left fibration}
Let $\calC$ be an $\infty$-category which admits small colimits, and let
$\phi: \widetilde{\calC} \rightarrow \calC$ be a left fibration. We will say that $\phi$ is
{\it completely compact} if it is classified by a functor $\calC \rightarrow \widehat{\SSet}$ that preserves small colimits.
\end{definition}

\begin{lemma}\label{bstick}
Let $\calC$ be an $\infty$-category which admits small colimits, $f: X' \rightarrow X$
a map of Kan complexes, and
$$ \xymatrix{ \calF' \ar[r] \ar[d] & \calF \ar[d] \\
X' \times \calC \ar[r]^{f \times \id_{\calC}} & X \times \calC }$$
be a diagram of left fibrations over $\calC$, which is a homotopy pullback square
$($ with respect to the covariant model structure on $(\sSet)_{/\calC}$ $)$. 
If $\calF \rightarrow \calC$ is completely compact, then $\calF' \rightarrow \calC$ is completely compact.
\end{lemma}

\begin{proof}
Replacing the diagram by an equivalent one if necessary, we may suppose that 
it is Cartesian and that $f$ is a Kan fibration. Let $\overline{p}: K^{\triangleright} \rightarrow \calC$ be a colimit diagram, and let $F: \calC \rightarrow \hat{\SSet}$ be a functor which classifies
the left fibration $\calF'$. We wish to show that $F \circ \overline{p}$ is a colimit diagram in
$\hat{\SSet}$.

We have a pullback diagram
$$ \xymatrix{ K \times_{\calC} \calF' \ar[r] \ar[d]^{\psi'} & K \times_{\calC} \calF \ar[d]^{\psi} \\
K^{\triangleright} \times_{\calC} \calF' \ar[r] & K^{\triangleright} \times_{\calC} \calF }$$
of simplicial sets, which is homotopy Cartesian (with respect to the usual model structure on $\sSet$) since the horizontal maps are pullbacks of $f$.
Since $\calF$ is completely compact, Proposition \ref{charspacecolimit} implies that the inclusion $\psi$ is a weak homotopy equivalence. It follows that $\psi'$ is also a weak homotopy equivalence. Applying Proposition \ref{charspacecolimit} again, we deduce that $F \circ \overline{p}$ is a colimit diagram as desired.
\end{proof}

\begin{lemma}\label{compactslice}
Let $\calC$ be a presentable $\infty$-category, $p: K \rightarrow \calC$
be a small diagram, and let $X \in \calC_{/p}$ be an object whose image in $\calC$ is completely compact. Then $X$ is completely compact.
\end{lemma}

\begin{proof}
Let $\overline{p}: K^{\triangleleft} \rightarrow \calC$ be a limit of $p$, carrying the cone point to an object $Z \in \calC$. Then we have trivial fibrations
$$ \calC_{/Z} \leftarrow \calC_{/\overline{p} } \rightarrow \calC_{/p}.$$
Consequently, we may replace the diagram $p: K \rightarrow \calC$ with the inclusion
$\{Z\} \rightarrow \calC$.

We may identify the object $X \in \calC_{/Z}$ with a morphism $f: Y \rightarrow Z$ in $\calC$.
We have a commutative diagram of simplicial sets
$$ \xymatrix{ (\calC_{/Z})_{f/} \ar[dr]^{\psi} \ar[r]^{\theta} & (\calC_{/Y})_{f/} \ar[d] \ar[r]^{\theta'} & (\calC_{/Y})^{f/} \ar[d]^{\psi'} \\
& \calC_{/Z} \ar[r]^{\theta'_0} & \calC^{/Z} }$$
where $\theta$ is an isomorphism, the maps $\theta'$ and $\theta'_0$ are categorical equivalences (see \S \ref{quasilimit2}), and the vertical maps are left fibrations. We wish to prove
that $\psi$ is a completely compact left fibration. It will therefore suffice to prove that
$\psi'$ is completely compact. We have a (homotopy) pullback diagram
$$ \xymatrix{ \calC_{Y/}^{/f} \ar[r] \ar[d] & \calC_{Y/}^{\Delta^1} \times_{\calC^{\{1\}}} \{Z\} \ar[d] \\
\calC^{/Z} \ar[r] & (\calC_{Y/} \times_{\calC} \{Z\}) \times \calC^{/Z} }$$
of left fibrations over $\calC^{/Z}$. We observe that the left fibrations in the lower part of the diagram are constant. According to Lemma \ref{bstick}, to prove that $\psi'$ is completely compact, it will suffice to prove that the left fibration $\calC_{Y/}^{\Delta^1} \times_{ \calC^{ \{1\} } } \{Z\} \stackrel{\psi''}{\rightarrow} \calC^{/Z}$ is completely compact. We observe that $\psi''$ admits a factorization
$$ \calC_{Y/}^{\Delta^1} \times_{ \calC^{ \{1\} }} \{Z\} \stackrel{\phi}{\rightarrow}
\calC_{Y/} \times_{ \calC^{ \{0\} } } \calC^{/Z} \stackrel{\phi'}{\rightarrow} \calC^{/Z}$$
where $\phi$ is a trivial fibration, and $\phi'$ is a pullback of the left fibration
left fibration $\phi'': \calC_{Y/} \rightarrow \calC$. Since $Y$ is completely compact, $\phi''$ is completely compact. The projection $\calC^{/Z} \rightarrow \calC$ is equivalent to
$\calC_{/Z} \rightarrow \calC$, and therefore commutes with colimits by Proposition \ref{needed17}.
It follows that $\phi'$ is completely compact, which completes the proof.
\end{proof}

\begin{proposition}\label{dda}
Let $S$ be a small simplicial set, and let $j: S \rightarrow \calP(S)$ denote the Yoneda embedding.
Let $C$ be an object of $\calP(S)$. The following conditions are equivalent:
\begin{itemize}
\item[$(1)$] The object $C \in \calP(S)$ is completely compact.
\item[$(2)$] There exists a vertex $s$ of $S$ such that $C$ is a retract of $j(s)$.
\end{itemize}
\end{proposition}

\begin{proof}
Suppose first that $(1)$ is satisfied.
Let $S_{/C} = S \times_{\calP(S)} \calP(S)_{/C}$. According to Lemma \ref{longwait0}, 
the natural map
$$ S^{\triangleright}_{/C} \stackrel{j'}{\rightarrow} \calP(S)_{/C}^{\triangleright} \rightarrow \calP(S)$$
is a colimit diagram. Let $f: \calP(S) \rightarrow \SSet$ be the functor corepresented by $C$.
Since $C$ is completely compact. $f(C)$ can be identified with a colimit of the diagram $f | S_{/C}$. 
The space $f(C)$ is homotopy equivalent to $\bHom_{\calP(S)}(C,C)$, and therefore contains a point corresponding to $\id_{C}$. It follows that $\id_{C}$ lies in the image of
$\bHom_{\calP(S)}(C, j'(\widetilde{s}) ) \rightarrow \bHom_{\calP(S)}(C,C)$, for some
vertex $\widetilde{s}$ of $S_{/C}$. The vertex $\widetilde{s}$ classifies a vertex
$s \in S$ equipped with a morphism $\alpha: j(s) \rightarrow C$. It follows that there
is a commutative triangle
$$ \xymatrix{ & j(s) \ar[dr]^{\alpha} & \\
C \ar[ur] \ar[rr]^{\id_{C}} & & C }$$
in the $\infty$-category $\calP(S)$, so that $C$ is a retract of $j(s)$.

Now suppose that $(2)$ is satisfied. According to Lemma \ref{retcompact}, it suffices to prove that $j(s)$ is completely compact. Using Lemma \ref{repco}, we may identify the functor
$\calP(S) \rightarrow \SSet$ co-represented by $j(s)$ with the functor given by evaluation at $s$.
Proposition \ref{limiteval} implies that this functor preserves all limits and colimits that exist in
$\calP(S)$.
\end{proof}

\begin{example}\label{tryu}
Let $\calC$ be the $\infty$-category $\SSet$ of spaces. Then an object
$C \in \SSet$ is completely compact if and only if it is equivalent to $\ast$, the final
object of $\SSet$.
\end{example}

We now use the theory of completely compact objects to give a characterization of
presheaf $\infty$-categories.

\begin{proposition}\label{trumptow}\index{gen}{completely compact!and presheaf $\infty$-categories}
Let $S$ be a small simplicial set and $\calC$ an $\infty$-category which admits small colimits. Let $F: \calP(S) \rightarrow \calC$ be functor which preserves small colimits, and
$f = F \circ j$ its composition with the Yoneda embedding $j: S \rightarrow \calP(S)$. 
Suppose further that:
\begin{itemize}
\item[$(1)$] The functor $f$ is fully faithful.

\item[$(2)$] For every vertex $s$ of $S$, the object $f(s) \in \calC$ is
completely compact.
\end{itemize}

Then $F$ is fully faithful.
\end{proposition}

\begin{proof}
Let $C$ and $D$ be objects of $\calP(S)$. We wish to prove that the natural map
$$ \eta_{C,D}: \bHom_{\calP(S)}(C,D) \rightarrow \bHom_{\calC}(F(C), F(D))$$ is an isomorphism in the homotopy category $\calH$.
Suppose first that $C$ belongs to the essential image
of $j$. Let $G: \calP(S) \rightarrow \SSet$ be a functor co-represented by $C$, and let
$G': \calC \rightarrow \SSet$ be a functor co-represented by $F(C)$. Then we have a natural transformation of functors $G \rightarrow G' \circ F$. Assumption $(2)$ implies that $G'$ preserves small colimits, so that $G' \circ F$ preserves small colimits. Proposition \ref{dda} implies that
$G$ preserves small colimits. It follows that the collection of objects $D \in \calP(S)$ such
that $\eta_{C,D}$ is an equivalence is stable under small colimits. If $D$ belongs to the essential image of $j$, then assumption $(1)$ implies that $\eta_{C,D}$ is an equivalence. It follows from Lemma \ref{longwait0} that the essential image of $j$ generates $\calP(S)$ under small colimits; thus $\eta_{C,D}$ is an isomorphism in $\calH$ for every object $D \in \calP(S)$.

We now prove the result in general. Fix $D \in \calP(S)$. Let $H: \calP(S)^{op} \rightarrow \SSet$
be a functor represented by $D$, and let $H': \calC^{op} \rightarrow \SSet$ be a functor represented by $FD$. Then we have a natural transformation of functors $H \rightarrow H' \circ F^{op}$, which we wish to prove is an equivalence. By assumption, $F^{op}$ preserves small limits. Proposition \ref{yonedaprop} implies that $H$ and $H'$ preserve small limits. It follows that the collection $P$ of objects $C \in \calP(S)$ such that $\eta_{C,D}$ is an equivalence is stable under small colimits.
The special case above established that $P$ contains the essential image of the Yoneda embedding. We once again invoke Lemma \ref{longwait0} to deduce every object of $\calP(S)$ belongs to $P$, as desired.
\end{proof}

\begin{corollary}\label{charpr}\index{gen}{equivalence!of presheaf $\infty$-categories}
Let $\calC$ be an $\infty$-category which admits small colimits. Let $S$ be a small
simplicial set and $F: \calP(S) \rightarrow \calC$ a colimit preserving functor. Then
$F$ is an equivalence if and only if the following conditions are satisfied:
\begin{itemize}
\item[$(1)$] The composition $f =F \circ j: S \rightarrow \calC$ is fully faithful.
\item[$(2)$] For every vertex $s \in S$, the object $f(s) \in \calC$ is completely compact.
\item[$(3)$] The set of objects $\{ f(s): s \in S_0\}$ generates $\calC$ under colimits.
\end{itemize}
\end{corollary}

\begin{proof}
If $(1)$, $(2)$, and $(3)$ are satisfied, then $F$ is fully faithful (Proposition \ref{trumptow}). 
Since $\calP(S)$ is admits small colimits, and $F$ preserves small colimits, the essential image of $F$ is stable under small colimits. Using $(3)$, we conclude that $F$ is essentially surjective and therefore an equivalence of $\infty$-categories. For the converse, it suffices to check that
$\id_{\calP(S)}: \calP(S) \rightarrow \calP(S)$ satisfies $(1)$, $(2)$, and $(3)$. For this, we invoke Propsition \ref{fulfaith}, Proposition \ref{dda}, and Lemma \ref{longwait0}, respectively.
\end{proof}

\begin{corollary}\label{swapKK}\index{gen}{presheaf!and overcategories}
Let $\calC$ be a small $\infty$-category, and let $p: K \rightarrow \calC$ be a diagram, and
let $p': K \rightarrow \calP(\calC)$ be the composition of $p$ with the Yoneda embedding
$j: \calC \rightarrow \calP(\calC)$, and let $f: \calC_{/p} \rightarrow \calP(\calC)_{/p'}$ be the induced map. Let $F: \calP(\calC_{/p}) \rightarrow \calP(\calC)_{/p'}$ be a colimit-preserving functor such that $F \circ j'$ is equivalent to $f$, where $j': \calC_{/p} \rightarrow \calP(\calC_{/p})$ denotes the Yoneda embedding for $\calC_{/p}$ $($according to Theorem \ref{charpresheaf}, $F$ exists and is unique up to equivalence$)$. Then $F$ is an equivalence of $\infty$-categories.
\end{corollary}

\begin{proof}
We will show that the $f$ satisfies conditions $(1)$ through $(3)$ of Corollary \ref{charpr}.
The assertion that $f$ is fully faithful follows immediately from the assertion that $j$ is fully faithful (Proposition \ref{fulfaith}). To prove that the essential image of $f$ consists of completely compact objects, we use Lemma \ref{compactslice} to reduce to proving that the essential image of $j$ consists of completely compact objects of $\calP(\calC)$, which follows from Proposition \ref{dda}.
It remains to prove that $\calP(\calC)_{/p'}$ is generated under colimits by $f$. Let
$\overline{X}$ be an object of $\calP(\calC)_{/p'}$ and $X$ its image in $\calP(\calC)$. 
Let $\calD \subseteq \calP(\calC)$ be the essential image of $j$, and $\overline{\calD}$ the inverse image of $\calD$ in $\calP(\calC)_{/p'}$, so that $\overline{\calD}$ is the essential image of $f$.
Using Lemma \ref{longwait0}, we can choose a colimit diagram $\overline{q}: L^{\triangleright} \rightarrow \calP(\calC)$ which carries the cone point to $X$ such that $q = \overline{q}|L$ factors through $\calD$. Since the inclusion of the cone point into $L^{\triangleright}$ is right anodyne,
there exists a map $\overline{q}': L^{\triangleright} \rightarrow \calP(\calC)_{/p'}$ lifting
$\overline{q}$, which carries the cone point of $L^{\triangleright}$ to $\overline{X}$. 
Proposition \ref{needed17} implies that $\overline{q}'$ is a colimit diagram, so that 
$\overline{X}$ can be written as the colimit of a diagram $L \rightarrow \overline{\calD}$.
\end{proof}
