\section{Dimension Theory}\label{dimension}

\setcounter{theorem}{0}


In this section, we will discuss the dimension theory of
topological spaces from the point of view of higher topos theory. We
begin in \S \ref{homdim} by introducing the {\it homotopy dimension} of an $\infty$-topos. We will show that the finiteness
of the homotopy dimension of an $\infty$-topos $\calX$ has pleasant
consequences: it implies that every object is the inverse limit of
its Postnikov tower, and in particular that $\calX$ is hypercomplete.

In \S \ref{chmdim}, we define the {\it cohomology groups} of an $\infty$-topos $\calX$. These cohomology groups have a natural interpretation in terms of the classification of higher gerbes on $\calX$. Using this interpretation, we will show that the cohomology dimension of an $\infty$-topos $\calX$ {\em almost} coincides with its homotopy dimension.

In \S \ref{covdim}, we review the classical theory of {\it covering dimension} for paracompact topological spaces. Using the results of \S \ref{paracompactness}, we will show that the covering dimension of a paracompact space $X$ coincides with the homotopy dimension of the $\infty$-topos $\Shv(X)$.

We conclude in \S \ref{heyt} by introducing a dimension theory for {\em Heyting spaces}, which generalizes the classical theory of Krull dimension for Noetherian topological spaces. Using this theory, we will prove an upper bound for the homotopy dimension of $\Shv(X)$, for suitable Heyting spaces $X$. This result can be regarded as a generalization of Grothendieck's vanishing theorem for the cohomology of Noetherian topological spaces.

\subsection{Homotopy Dimension}\label{homdim}

Throughout this section, we will use the symbol $1_{\calX}$ to denote the final object of an $\infty$-topos
$\calX$.

\begin{definition}\label{sugarpie}\index{gen}{dimension!homotopy}\index{gen}{homotopy dimension}
\index{gen}{homotopy dimension!finite}
Let $\calX$ be an $\infty$-topos. We shall say that $\calX$ has
{\it homotopy dimension $\leq n$} if every $n$-connective
object $U \in \calX$ admits a global section $1_{\calX} \rightarrow U$. We
say that $\calX$ has {\it finite homotopy dimension} if there
exists $n \geq 0$ such that $\calX$ has homotopy dimension $\leq
n$.
\end{definition}

\begin{example}
An $\infty$-topos $\calX$ is of homotopy
dimension $\leq -1$ if and only if $\calX$ is equivalent to
the trivial $\infty$-category $\ast$ (the $\infty$-topos of sheaves on the empty space
$\emptyset$). The ``if'' direction is obvious. Conversely, if $\calX$ has homotopy dimension
$\leq -1$, then the initial object $\emptyset$ of $\calX$ admits a global section
$1_{\calX} \rightarrow \emptyset$. For every object $X \in \calX$, we have a map
$X \rightarrow 1_{\calX} \rightarrow \emptyset$, so that $X$ is also initial (Lemma \ref{sumoto}).
Since the collection of initial objects of $\calX$ span a contractible Kan complex (Proposition \ref{initunique}), we deduce that $\calX$ is itself a contractible Kan complex.
\end{example}

\begin{example}\label{honeypie}
The $\infty$-topos $\SSet$ has homotopy dimension $0$. 
More generally, if $\calC$ is an $\infty$-category with a final object $1_{\calC}$, then
$\calP(\calC)$ has homotopy dimension $\leq 0$. To see this, we first observe that the Yoneda embedding $j: \calC \rightarrow \calP(\calC)$ preserves limits, so that $j(1_{\calC})$ is a final object of $\calP(\calC)$. To prove that $\calP(\calC)$ has homotopy dimension $\leq 0$, we need to show that the functor $\calP(\calC) \rightarrow \SSet$ corepresented by $j(1_{\calC})$ preserves
effective epimorphisms. This functor can be identified with evaluation at $1_{\calC}$. It therefore preserves all limits and colimits, and so carries effective epimorphisms to effective epimorphisms by Proposition \ref{sinn}.
\end{example}

\begin{example}\label{honepie}
Let $X$ be a Kan complex, and let $n \geq -1$. The following conditions are equivalent:
\begin{itemize}
\item[$(1)$] The $\infty$-topos $\SSet_{/X}$ has homotopy dimension $\leq n$.
\item[$(2)$] The geometric realization $|X|$ is a retract (in the homotopy category $\calH$) of a CW complex $K$ of dimension $\leq n$.
\end{itemize}
To prove that $(2) \Rightarrow (1)$, let us choose an $n$-connective
object of $\calX_{/X}$ corresponding to a Kan fibration $p: Y \rightarrow X$, whose homotopy fibers are $n$-connective. Choose a map $K \rightarrow |X|$ which admits a right homotopy inverse. 
To prove that $p$ admits a section up to homotopy, tt will suffice to show that there exists a dotted arrow $$ \xymatrix{ & |Y| \ar[d]^{p} \\
K \ar@{-->}[ur]^{f} \ar[r] & |X| }$$
in the category of topological spaces, rendering the diagram commutative. The construction
of $f$ proceeds cell-by-cell on $K$, using the $n$-connectivity of $p$ to solve lifting problems of the form
$$ \xymatrix{ S^{k-1} \ar@{^{(}->}[d] \ar[r] & |Y| \ar[d]^{p} \\
D^{k} \ar[r] \ar@{-->}[ur] & |X| }$$
for $k \leq n$.

To prove that $(1) \Rightarrow (2)$, we choose any $n$-connective map $q: K \rightarrow |X|$, where
$K$ is an $n$-dimensional CW complex. Condition $(1)$ guarantees that $q$ admits a right homotopy inverse, so that $|X|$ is a retract of $K$ in the homotopy category $\calH$.
\end{example}

\begin{remark}\label{inter}
If $\calX$ is a coproduct (in the $\infty$-category $\RGeom$) of $\infty$-topoi $\calX_{\alpha}$, then
$\calX$ is of homotopy dimension $\leq n$ if and
only if each $\calX_{\alpha}$ is of homotopy dimension $\leq n$.
\end{remark}

It is convenient to introduce a relative version of Definition \ref{sugarpie}.

\begin{definition}\index{gen}{homotopy dimension!of a geometric morphism}
Let $f: \calX \rightarrow \calY$ be a geometric morphism of $\infty$-topoi.
We will say that $f$ is of {\it homotopy dimension $\leq n$} if, for every $k \geq n$ and every
$k$-connective morphism $X \rightarrow X'$ in $\calX$, the induced map
$f_{\ast} X \rightarrow f_{\ast} X'$ is a $(k-n)$-connective morphism in $\calY$ (since $f_{\ast}$ is well-defined up to equivalence, this condition is independent of the choice of $f_{\ast}$).
\end{definition}

\begin{lemma}\label{pie}
Let $\calX$ be an $\infty$-topos, and let $F_{\ast}: \calX \rightarrow \SSet$ be a geometric morphism (which is unique up to equivalence). The following are equivalent:
\begin{itemize}
\item[$(1)$] The $\infty$-topos $\calX$ is of homotopy dimension $\leq n$.
\item[$(2)$] The geometric morphism $F_{\ast}$ is of homotopy dimension $\leq n$.
\end{itemize}
\end{lemma}

\begin{proof}
Suppose first that $(2)$ is satisfied, and let $X$ be an $n$-connective object
of $\calX$. Then $F_{\ast} X$ is a $0$-connective object of $\SSet$: that is, it is a nonempty Kan complex. It therefore has a point $1_{\SSet} \rightarrow F_{\ast} X$. By adjointness, we see that there exists a map $1_{\calX} \rightarrow X$ in $\calX$, where
$1_{\calX} = F^{\ast} 1_{\SSet}$ is a final object of $\calX$ because $F^{\ast}$ is left exact. This proves $(1)$.

Now assume $(1)$, and let
$s: X \rightarrow Y$ be an $k$-connective morphism in $\calX$; we wish to show that
$F_{\ast}s$ is $(k-n)$-connective. The proof goes by induction on $k \geq n$. If $k = n$, then 
are reduced to proving the surjectivity of the horizontal maps in the diagram
$$ \xymatrix{ \pi_0 \bHom_{\calX}( 1_{\calX} , X) \ar[r] \ar@{=}[d] & \pi_0 \bHom_{\calX}( 1_{\calX}, Y) \ar@{=}[d] \\
\pi_0 \bHom_{\SSet}( 1_{\SSet}, F_{\ast} X) \ar[r] & \pi_0 \bHom_{\SSet}(1_{\SSet}, F_{\ast} Y) }$$
of sets. Let $p: 1_{\calX} \rightarrow Y$ be any morphism in $\calX$, and form a pullback diagram
$$ \xymatrix{ Z \ar[d]^-{s'} \ar[r] & X \ar[d]^-{s} \\
1_{\calX} \ar[r]^-{p} & Y. }$$
The map $s'$ is a pullback of $s$, and therefore $n$-connective by Proposition \ref{inftychange}. Using $(1)$, we deduce the existence of a map $1_{\calX} \rightarrow Z$, and a composite map
$$1_{\calX} \rightarrow Z \rightarrow X$$ is a lifting of $p$ up to homotopy.

We now treat the case where $k > n$. Form a diagram
$$ \xymatrix{ X \ar[drr]^-{s'} & & & \\
& & X \times_{Y} X \ar[r] \ar[d] & X \ar[d]^-{s} \\
& & X \ar[r]^-{s} & Y}$$ 
where the square on the bottom-right is a pullback in $\calX$. 
According to Proposition \ref{trowler}, $s'$ is $(k-1)$-connective. Using the inductive hypothesis, we deduce that $F_{\ast}(s')$ is $(k-n-1)$-connective. We now invoke Proposition \ref{trowler} in the $\infty$-topos $\SSet$ deduce that $F_{\ast}(s)$ is $(k-n)$-connective, as desired.
\end{proof}

\begin{definition}\index{gen}{homotopy dimension!locally finite}
We will say that an $\infty$-topos $\calX$ is {\it locally of homotopy dimension $\leq n$} if there exists a collection $\{U_{\alpha} \}$ of objects of $\calX$ which generate $\calX$ under colimits, such that each $\calX_{/U_{\alpha} }$ is of homotopy dimension $\leq n$.
\end{definition}

\begin{example}
Let $\calC$ be a small $\infty$-category. Then $\calP(\calC)$ is locally of homotopy dimension $\leq 0$. To prove this, we first observe that $\calP(\calC)$ is generated under colimits by the Yoneda embedding $j: \calC \rightarrow \calP(\calC)$. It therefore suffices to prove that
each of the $\infty$-topoi $\calP(\calC)_{/j(C)}$ has finite homotopy dimension. According to Corollary \ref{swapKK}, the $\infty$-topos $\calP(\calC)_{/j(C)}$ is equivalent to $\calP( \calC_{/C})$, which is of homotopy dimension $0$ (see Example \ref{honeypie}). 
\end{example}

Our next goal is to prove the following result:

\begin{proposition}\label{cumba}
Let $\calX$ be an $\infty$-topos which is locally of homotopy dimension $\leq n$ for some integer $n$. Then Postnikov towers in $\calX$ are convergent.
\end{proposition}

%We begin by treating the case where $\calX = \SSet$. In view of Remark \ref{urkan}, it will suffice to prove the following result:

%\begin{lemma}\label{highcon}
%Let $X: \Nerve ( \Z_{\geq 0}^{\infty} )^{op} \rightarrow \SSet$ be a tower in the $\infty$-category of spaces. Assume that $X$ is a limit tower, and that the underlying pretower
%$X| \Nerve( \Z_{\geq 0})^{op}$ is highly connected. Then $X$ is highly connected. (See Remark \ref{urkan}.)
%\end{lemma}

%\begin{remark}\label{swarmus}
%In proving Lemma \ref{highcon}, we will make use of the following facts concerning a %commutative diagram
%$$ \xymatrix{ X \ar[rr]^-{h} \ar[dr]^-{f} & & Z \\
%& Y \ar[ur]^-{g} & }$$
%of Kan complexes:
%\begin{itemize}
%\item[$(1)$] If $f$ and $g$ are $m$-connective, then so is $h$.
%\item[$(2)$] If $f$ is $(m-1)$-connective and $h$ is $m$-connective, then $g$ is $m$-connective.
%\item[$(3)$] If $g$ is $(m+1)$-connective and $h$ is $m$-connective, then $f$ is $m$-connective.
%\end{itemize}
%This is a simple exercise in the use of long exact sequences of homotopy groups.
%\end{remark}

%\begin{proof}[Proof of Lemma \ref{highcon}]
%Let $X: \Nerve( \Z_{\geq 0}^{\infty} )^{op} \rightarrow \SSet$ be a tower in the $\infty$-category of %spaces. According to Proposition \ref{gumby444}, $X$ is equivalent to the tower associated to a %diagram
%$$ Y(\infty) \rightarrow \ldots \rightarrow Y(1) \rightarrow Y(0) $$
%in the ordinary category $\Kan$. Replacing $Y$ by an equivalent diagram if necessary, we
%may suppose that each of the maps $Y(i+1) \rightarrow Y(i)$ is a Kan fibration (which
%is equivalent to the assertion that the diagram $\Z_{\geq 0}^{op} \rightarrow \sSet$ is 
%strongly fibrant). Since $X$ is a limit tower, Theorem \ref{colimcomparee} implies that
%the map $Y(\infty) \rightarrow \lim \{ Y(n) \}_{n \geq 0}$ is a homotopy equivalence.
%Without loss of generality, we may replace $Y(\infty)$ by the limit
%$\lim \{Y(n) \}_{n \geq 0}$. 

%Fix $m \geq 0$. We wish to show that the map $Y(\infty) \rightarrow Y(n)$ is
%$m$-connective for $n \gg 0$. Since the underlying pretower of $X$ is highly connected, there %exists $k \geq 0$ such that for all $k'' \geq k' \geq k$, the map $Y(k'') \rightarrow Y(k')$ is
%$(m+1)$-connective. We will prove that the map $f: Y(\infty) \rightarrow Y(n)$ is $m$-connective %for $n \geq k$.

%The fiber of $f$ over a vertex $y \in Y(n)$ can be identified with the inverse limit of the
%sequence $\{ Z(n') \}_{n' \geq n}$, where $Z(n') = Y(n') \times_{Y(n)} \{y\}$. By assumption,
%each of the spaces $Z(n')$ is $(m+1)$-connective. It follows that $Z$ is $m$-connective, as %desired.
%\end{proof}

%\begin{remark}
%It follows from Lemma \ref{highcon} that every Postnikov tower in $\SSet$ is a limit tower. This observation plays an important role in classical homotopy theory:
%it allows us to reduce many problems about an arbitrary space $X(\infty)$ to problems concerning the truncated spaces $X(n) \simeq \tau_{\leq n} X(\infty)$, which are often easier to analyze.
%\end{remark}

\begin{proof}
We will show that $\calX$ satisfies the criterion of Remark \ref{urkan}. Let $X: \Nerve( \Z^{\infty}_{\geq 0} )^{op} \rightarrow \calX$ be a limit tower, and assume that the underlying pretower is highly connected. We wish to show that $X$ is highly connected. Choose $m \geq -1$; we wish to show that the map $X(\infty) \rightarrow X(k)$ is $m$-connective for $k \gg 0$.
Reindexing the tower if necessary, we may suppose that for every $p \geq q$, the map
$X(p) \rightarrow X(q)$ is $(m+q)$-connective. We claim that, in this case, we can take $k = 0$. 
The proof goes by induction on $m$. If $m > 0$, we can deduce the desired result by applying the inductive hypothesis to the tower
$$ X(\infty) \rightarrow \ldots X(\infty) \times_{ X(2) } X(\infty) \rightarrow X(\infty) \times_{ X(1) } X(\infty)
\rightarrow X(\infty) \times_{ X(0) } X(\infty). $$
Let us therefore assume that $m = 0$; we wish to show that the map $X(\infty) \rightarrow X(0)$ is an effective epimorphism. Since the objects $\{ U_{\alpha} \}$ generate $\calX$ under colimits, there
is an effective epimorphism $\phi: U \rightarrow X(0)$, where $U$ is a coproduct of objects of the form
$\{ U_{\alpha} \}$. Using Remark \ref{inter}, we deduce that $\calX_{/U}$ has homotopy dimension $\leq n$. Let $F: \calX \rightarrow \SSet$ denote the functor corepresented by $U$. Then
$F$ factors as a composition
$$ \calX \stackrel{f^{\ast}}{\rightarrow} \calX_{/U} \stackrel{\Gamma}{\rightarrow} \SSet,$$
where $f^{\ast}$ is the left adjoint to the geometric morphism $\calX_{/U} \rightarrow \calX$
and $\Gamma$ is the global sections functor. It follows that $F$ carries $n$-connective morphisms to effective epimorphisms (Lemma \ref{pie}). The map $\phi$ determines a point of
$F( X(0) )$. Since each of the maps $F( X(k+1)  ) \rightarrow F( X(k) )$ induces a surjection on
connected components, we can lift this point successively to each $F( X(k) )$ and thereby obtain
a point in $F( X(\infty) ) \simeq \holim \{ F( X(n) ) \}$. This point determines a diagram
$$ \xymatrix{ & X(\infty) \ar[dr]^{\psi} & \\
U \ar[ur] \ar[rr]^{\phi} & & X(0) }$$
which commutes up to homotopy. Since $\phi$ is an effective epimorphism, we deduce that
the map $\psi$ is an effective epimorphism, as desired.
\end{proof}

%\begin{lemma}\label{slurpies}
%Let $f: \calX \rightarrow \calY$ be a geometric morphism of $\infty$-topoi.
%Suppose that $f$ is of homotopy dimension $\leq n$. Let
%$X: \Nerve ( \Z_{\geq 0}^{\infty} )^{op} \rightarrow \calX$ be a highly connected tower
%in $\calX$. Then $f_{\ast} \circ X$ is a highly connected tower in $\calY$.
%\end{lemma}

%\begin{proof}
%This follows immediately from the definitions.
%\end{proof}

%

%\begin{proposition}[Jardine \cite{jardine}]
%Let $\calX$ be an $\infty$-topos which is locally of finite homotopy dimension.
%Then every highly connected tower in $\calX$ is a limit tower.
%\end{proposition}

%\begin{proof}
%By hypothesis, there exists a small $\infty$-category $\calC$ and a fully faithful
%functor $f: \calC \rightarrow \calX$ which generates $\calX$ under colimits, with the property
%that $\calX_{/f(C)}$ is of finite homotopy dimension for each object $C \in \calC$. 
%According to Proposition \ref{charpresheaf}, we may assume without loss of generality that
%$f = F \circ j$, where $j: \calC \rightarrow \calP(\calC)$ is the Yoneda embedding and
%$F: \calP(\calC) \rightarrow \calX$ is a presentable functor. Since $f$ generates $\calX$ under colimits, the functor $F$ admits a fully faithful right adjoint $G$. 

%Let $X: \Nerve ( \Z_{\geq 0}^{\infty} )^{op} \rightarrow \calX$ be a highly connected tower
%in $\calX$. We wish to show that $X$ is a limit tower. Since $G$ is fully faithful and preserves limits, it will suffice to show that $G \circ X$ is a limit tower in $\calP(\calC)$. According to Proposition \ref{limiteval}, it will suffice to to show for each object $C \in \calC$, the tower
%$g_{C} \circ X: \Nerve(\Z_{\geq 0}^{\infty} )^{op} \rightarrow \calX$ is a limit, where
%$g_{C}$ denotes the composition of $G$ with evaluation at $C$. The functor $g_{C}$ is equivalent to the composition
%$$ \calX \stackrel{q^{\ast}}{\rightarrow} \calX_{/f(C)} \stackrel{p_{\ast}}{\rightarrow} \SSet$$
%where $q: \calX_{/f(C)} \rightarrow \calX$ and $p: \calX_{/f(C)} \rightarrow \SSet$ are
%the natural geometric morphisms. It is clear that $q^{\ast}$ preserves highly connected towers. Since $\calX_{/f(C)}$ is of finite homotopy dimension, Lemma \ref{slurpies} implies that $p_{\ast}$ preserves highly connected towers. We may therefore reduce to the case where $\calX = \SSet$, which was handled in Lemma \ref{highcon}.
%\end{proof}

\begin{lemma}\label{tow}
Let $\calX$ be a presentable $\infty$-category, let $\Fun(\Nerve (\Z_{\geq 0}^{\infty})^{op}, \calX)$ be the $\infty$-category of towers in $\calX$, and let $\calX_{\tau} \subseteq 
\Fun( \Nerve (\Z_{\geq 0}^{\infty})^{op}, \calX)$ denote the full subcategory spanned by the Postnikov towers. Evaluation at $\infty$ induces a trivial fibration of simplicial sets
$ \calX_{\tau} \rightarrow \calX$. In particular, every object $X(\infty) \in \calX$ can be extended to a Postnikov tower 
$$ X(\infty) \rightarrow \ldots \rightarrow X(1) \rightarrow X(0).$$
\end{lemma}

\begin{proof}
Let $\calC$ be the full subcategory of $\calX \times \Nerve(\Z_{\geq 0}^{\infty})^{op}$
spanned by the pairs $(X, n)$ where $X$ is an object of $\calX$, $n \in \Z_{\geq 0}^{\infty}$,
and $X$ is $n$-truncated, and let $p: \calC \rightarrow \calX$ denote the natural projection.
Since every $m$-truncated object of $\calX$ is also $n$-truncated for $m \geq n$, it is easy to see that $p$ is a Cartesian fibration. Proposition \ref{maketrunc} implies that each of the inclusion
functors $\tau_{\leq m} \calX \subseteq \tau_{\leq n} \calX$ has a left adjoint, so that $p$ is also a coCartesian fibration (Corollary \ref{getcocart}). By definition, $\calX_{\tau}$ can be identified
with the simplicial set
$$ \bHom^{\flat}_{\Nerve(\Z_{\geq 0}^{\infty})}( \Nerve(\Z_{\geq 0}^{\infty})^{\sharp}, 
(\calC^{op})^{\natural} )^{op}$$
and $\calX$ itself can be identified with
$$ \bHom^{\flat}_{\Nerve(\Z_{\geq 0}^{\infty})}( \{\infty\}^{\sharp}, 
(\calC^{op})^{\natural} )^{op}.$$
It now suffices to observe that the inclusion
$\{ \infty\}^{\sharp} \subseteq \Nerve(\Z_{\geq 0}^{\infty})^{\sharp}$ is marked anodyne.
\end{proof}

\begin{corollary}[Jardine]\label{fdfd}
Let $\calX$ be an $\infty$-topos which is locally of homotopy dimension $\leq n$ for some integer $n$. Then $\calX$ is hypercomplete.
\end{corollary}

\begin{proof}
Let $X(\infty)$ be an arbitrary object of $\calX$. By Lemma \ref{tow} we can find a Postnikov tower
$$ X(\infty) \rightarrow \ldots \rightarrow X(1) \rightarrow X(0).$$
Since $X(n)$ is $n$-truncated, it belongs to $\calX^{\hyp}$ by Corollary \ref{goober2}.
By Proposition \ref{cumba}, the tower exhibits $X(\infty)$ as a limit of objects
of $\calX^{\hyp}$, so that $X(\infty)$ belongs to $\calX^{\hyp}$ as well since
the full subcategory $\calX^{\hyp} \subseteq \calX$ is stable under limits.
\end{proof}

\begin{lemma}\label{nicelemma}
Let $\calX$ be an $\infty$-topos, $n \geq 0$, $X$ an
$(n+1)$-connective object of $\calX$, and $f^{\ast}: \calX \rightarrow \calX_{/X}$ a right adjoint to the projection $\calX_{/X} \rightarrow \calX$.
Then $f^{\ast}$ induces a fully faithful functor
$\tau_{\leq n} \calX \rightarrow \tau_{\leq n} \calX_{/X}$ which restricts to an equivalence from $\tau_{\leq n-1} \calX$ to $\tau_{\leq n-1} \calX_{/X}$
\end{lemma}

\begin{proof}
We first prove that $f^{\ast}$ is fully faithful when restricted to the
$\infty$-category of $n$-truncated objects of $\calX$. Let $Y,
Z \in \calX$ be objects, where $Y$ is $n$-truncated. We have a commutative
diagram 
$$ \xymatrix{ \bHom_{\calX_{/X}}(f^{\ast} Y, f^{\ast} Z) \ar@{=}[r] & 
 \bHom_{\calX}( X \times Y, Z) &  \bHom_{\calX}( \tau_{\leq n}(X \times Y), Z) \ar[l] \\
 \bHom_{\calX}(Y,Z) \ar[u] & & \bHom_{\calX}( \tau_{\leq n} Y, Z) \ar[u] \ar[ll] }$$
in the homotopy category $\calH$, where the horizontal arrows are homotopy equivalences.
Consequently, to prove that the left vertical map is a homotopy equivalence, it suffices to show that the projection $\tau_{\leq n} (X \times Y) \rightarrow \tau_{\leq n} Y$ is an equivalence.
This follows immediately from Lemma \ref{slurpy} and our assumption that
$X$ is $(n+1)$-connective.

Now suppose that $\overline{Y}$ is an $(n-1)$-truncated object of $\calX_{/X}$. We wish to show that $\overline{Y}$ lies in the essential image of $f^{\ast} | \tau_{\leq n-1} \calX$. Let $Y$
denote the image of $\overline{Y}$ in $\calX$, and let $Y \rightarrow Z$ exhibit $Z$ as
an $(n-1)$-truncation of $Y$ in $\calX$. To complete the proof, it will suffice to show that the composition
$$ u: \overline{Y} \stackrel{u'}{\rightarrow} f^{\ast} Y \stackrel{u''}{\rightarrow} f^{\ast} Z$$
is an equivalence in $\calX_{/X}$. Since both $\overline{Y}$ and $f^{\ast} Z$
are $(n-1)$-truncated, it suffices to prove that $u$ is $n$-connective. According to Proposition \ref{inftychange},
it suffices to prove that $u'$ and $u''$ are $n$-connective. Proposition \ref{compattrunc}
implies that $u''$ exhibits $f^{\ast} Z$ as an $(n-1)$-truncation of $f^{\ast} Y$, and is therefore
$n$-connective. 

We now complete the proof by showing that $u'$ is $n$-connective.
Let $v'$ denote the image of image of $u'$ in the $\infty$-topos $\calX$. According to
Proposition \ref{conslice}, it will suffice to show that $v'$ is $n$-connective.
We observe that $v'$ is a section of the projection $q: Y \times X \rightarrow Y$.
$q: Y \times X \rightarrow Y$. According to Proposition \ref{sectcon}, it will suffice to prove that
$q$ is $(n+1)$-connective. Since $q$ is a pullback of the projection $X \rightarrow 1_{\calX}$, Proposition \ref{inftychange} allows us to conclude the proof (since $X$ is $(n+1)$-connective, by assumption).
\end{proof}

Lemma \ref{nicelemma} has some pleasant consequences.

\begin{proposition}\label{pi00detects}
Let $\calX$ be an $\infty$-topos and let $\tau_{\leq 0} : \calX \rightarrow \tau_{\leq 0} \calX$
denote a left adjoint to the inclusion. A morphism $\phi: U \rightarrow X$ in $\calX$ is an effective epimorphism if and only if $\tau_{\leq 0}(\phi)$ is an effective epimorphism in the ordinary topos
$\h{(\tau_{\leq 0} \calX)}$.
\end{proposition}

\begin{proof}
Suppose first that $\phi$ is an effective epimorphism. Let $U_{\bigdot}: \Nerve \cDelta_{+}^{op} \rightarrow \calX$ be a \Cech nerve of $\phi$, so that $U_{\bigdot}$ is a colimit diagram.
Since $\tau_{\leq 0}$ is a left adjoint,
$\tau_{\leq 0} U_{\bigdot}$ is a colimit diagram in $\tau_{\leq 0} \calX$. 
Using Proposition \ref{charsurj}, we deduce easily that $\tau_{\leq 0} \phi$ is an effective epimorphism.

For the converse, choose a factorization of $\phi$ as a composition
$$ U \stackrel{\phi'}{\rightarrow} V \stackrel{\phi''}{\rightarrow} X$$
where $\phi'$ is an effective epimorphism and $\phi''$ is a monomorphism. Applying Lemma \ref{nicelemma} to the $\infty$-topos $\calX_{/ \tau_{\leq 0} X}$, we conclude that
$\phi''$ is the pullback of a monomorphism $i: \overline{V} \rightarrow \tau_{\leq 0} X$.
Since the effective epimorphism $\tau_{\leq 0}(\phi)$ factors through $i$, we conclude
that $i$ is an equivalence, so that $\phi''$ is likewise an equivalence. It follows that
$\phi$ is an effective epimorphism as desired.
\end{proof}

Proposition \ref{pi00detects} can be regarded as a generalization of the following well-known property of the $\infty$-category of spaces, which can itself be regarded as the $\infty$-categorical analogue of the second part of Fact \ref{factoid}:

\begin{corollary}
Let $f: X \rightarrow Y$ be a map of Kan complexes. Then $f$ is an effective epimorphism
in the $\infty$-category $\SSet$ if and only if the induced map $\pi_0 X \rightarrow \pi_0 Y$ is surjective.
\end{corollary}

\begin{remark}\label{charnice}
It follows from Proposition \ref{pi00detects} that the class of $\infty$-topoi having the form
$\Shv(\calC)$, where $\calC$ is a small $\infty$-category, is not substantially larger than the class of ordinary topoi. More precisely, every topological localization of $\calP(\calC)$ can be obtained by inverting
morphisms between {\em discrete} objects of $\calP(\calC)$. It follows that there exists a pullback diagram of $\infty$-topoi
$$ \xymatrix{ \Shv(\calC) \ar[r] \ar[d] & \calP(\calC) \ar[d] \\
\Shv(\Nerve( \h{\calC})) \ar[r] & \calP(\Nerve(\h{\calC})) }$$
where the $\infty$-topoi on the bottom line are $1$-localic, and therefore determined by the
ordinary topoi of presheaves of sets on the homotopy category $h\calC$ and sheaves of sets on $h\calC$, respectively.
\end{remark}

\begin{corollary}\label{enuff}
Let $X$ be a topological space. Suppose that $\Shv(X)$ is locally of homotopy dimension $\leq n$ for some integer $n$. Then $\Shv(X)$ has enough points.
\end{corollary}

\begin{proof}
Note that every point $x \in X$ gives rise to a point $x_{\ast}: \Shv(\ast) \rightarrow \Shv(X)$
of the $\infty$-topos $\Shv(X)$. Let $f: \calF \rightarrow \calF'$ be a morphism in
$\Shv(X)$ such that $x^{\ast}(f)$ is an equivalence in $\SSet$ for each $x \in X$. We wish to prove that $f$ is an equivalence. According to Corollary \ref{fdfd}, it will suffice to prove that $f$ is $\infty$-connective. We will prove by induction on $n$ that $f$ is $n$-connective. If
$n > 0$, we simply apply the inductive hypothesis to the diagonal morphism
$\delta: \calF \rightarrow \calF \times_{\calF'} \calF$. We may therefore reduce to the case
$n=0$; we wish to show that $f$ is an effective epimorphism. Since $\Shv(X)$ is generated under colimits by the sheaves $\chi_U$ associated to open subsets $U \subseteq X$,  we may assume without loss of generality that $\calF' = \chi_{U}$. We may now invoke Proposition \ref{pi00detects} to reduce to the case where $\calF$ is an object of $\tau_{\leq 0} \Shv(X)_{/\chi_U}$. This $\infty$-category is equivalent to the nerve of the category of sheaves of {\em sets} on $U$. We are therefore reduced to proving that if $\calF$ is a sheaf of sets on $U$ whose stalk
$\calF_{x}$ is a singleton at each point $x \in U$, then $\calF$ has a global section, which is clear.
\end{proof}

\subsection{Cohomological Dimension}\label{chmdim}

In classical homotopy theory, one can analyze a space $X$ by means of its Postnikov tower
$$ \ldots \tau_{\leq n} X \stackrel{\phi_n}{\rightarrow} \tau_{\leq n-1} X \rightarrow \ldots.$$
In this diagram, the homotopy fiber $F$ of $\phi_n$ $(n \geq 1)$ is a space which has only a single nonvanishing homotopy group, in dimension $n$. The space $F$ is determined up to homotopy equivalence by $\pi_n F$: in fact $F$ is homotopy equivalent to an Eilenberg-MacLane space
$K( \pi_n F, n)$ which can be functorially constructed from the group $\pi_n F$. The study of these Eilenberg-MacLane spaces is of central interest, because (according to the above analysis) they constitute basic building blocks out of which any arbitrary space can be constructed.
Our goal in this section is to generalize the theory of Eilenberg-MacLane spaces to the setting of 
an arbitrary $\infty$-topos $\calX$.

\begin{definition}\label{gropab}\index{gen}{object!pointed}\index{gen}{pointed object}\index{not}{Xast@$\calX_{\ast}$}
Let $\calX$ be an $\infty$-category. A {\it pointed object} is a morphism
$X_{\ast}: 1 \rightarrow X$ in $\calX$, where $1$ is a final object of $\calX$.
We let $\calX_{\ast}$ denote the full subcategory of $\Fun(\Delta^1,\calX)$ spanned by the pointed objects of
$\calX$.

A {\it group object} of $\calX$ is a groupoid object $U_{\bigdot}: \Nerve \cDelta^{op} \rightarrow \calX$ for which $U_0$ is a final object of $\calX$. Let $\Group(\calX)$ denote the full subcategory
of $\calX_{\Delta}$ spanned by the group objects of $\calX$.\index{gen}{object!group}\index{gen}{group object}\index{not}{groupcalX@$\Group(\calX)$}

We will say that a pointed object $1 \rightarrow X$ of an $\infty$-topos $\calX$ is an {\it Eilenberg-MacLane object of degree $n$} if $X$ is $n$-truncated and $n$-connective. We let $\EM_n(\calX)$ denote the full subcategory of $\calX_{\ast}$ spanned by the Eilenberg-MacLane objects of degree $n$.\index{gen}{object!Eilenberg-MacLane}\index{gen}{Eilenberg-MacLane object}\index{not}{EMncalX@$\EM_n(\calX)$}
\end{definition}

\begin{example}
Let $\calC$ be an ordinary category which admits finite limits. A {\it group object}
of $\calC$ is an object $X \in \calC$ which is equipped with an identity section
$1_{\calC} \rightarrow X$, an inversion map $X \rightarrow X$, and
a multiplication $m: X \times X \rightarrow X$, which satisfy the usual group axioms. Equivalently, a group object of $\calC$ is an object $X$ together with a group structure on each morphism
space $\Hom_{\calC}(Y,X)$, which depends functorially on $Y$. We will denote the category
of group objects of $\calC$ by $\Group(\calC)$. The $\infty$-category
$\Nerve(\Group(\calC))$ is equivalent to the $\infty$-category of group objects of
$\Nerve(\calC)$, in the sense of Definition \ref{gropab}. Thus, the notion of a group object
of an $\infty$-category can be regarded as a generalization of the notion of a group object of an ordinary category.
\end{example}

\begin{remark}\label{prodem}
Let $\calX$ be an $\infty$-topos and $n \geq 0$ an integer. Then the full subcategory of
$\Fun(\Delta^1,\calX)$ consisting of Eilenberg-MacLane objects $p: 1 \rightarrow X$
is stable under finite products. 
This is clear, since:
\begin{itemize}
\item[$(1)$] A finite product of $n$-connective objects of $\calX$ is $n$-connective (Corollary \ref{togoto}).
\item[$(2)$] {\em Any} limit of $n$-truncated objects of $\calX$ is $n$-truncated (since 
$\tau_{\leq n} \calX$ is a localization of $\calX$).
\end{itemize}
\end{remark}

\begin{proposition}\label{tinner}
Let $\calX$ be an $\infty$-category, and let $U_{\bigdot}$ be a simplicial object of $\calX$. 
Then $U_{\bigdot}$ is a group object of $\calX$ if and only if the following conditions are satisfied:
\begin{itemize}
\item[$(1)$] The object $U_0$ is final in $\calX$.
\item[$(2)$] For every decomposition $[n] = S \cup S'$, where
$S \cap S' = \{s\}$, the maps
$$ U(S) \leftarrow U_n \rightarrow U(S')$$
exhibit $U_n$ as a product of $U(S)$ and $U(S)$ in $\calX$.
\end{itemize}
\end{proposition}

\begin{proof}
This follows immediately from characterization $(4'')$ of Proposition \ref{grpobjdef}.
\end{proof}

\begin{corollary}\label{grpstable}
Let $\calX$ and $\calY$ be $\infty$-categories which admit finite products, and let
$f: \calX \rightarrow \calY$ be a functor which preserves finite products. Then
the induced functor $\calX_{\Delta} \rightarrow \calY_{\Delta}$ carries group objects
of $\calX$ to group objects of $\calY$.
\end{corollary}

\begin{corollary}
Let $\calX$ be an $\infty$-category which admits finite products, and let $\calY \subseteq \calX$ be a full subcategory which is stable under finite products. Let $Y_{\bigdot}$ be a simplicial
object of $\calY$. Then $Y_{\bigdot}$ is a group object of $\calY$ if and only if it is a group object of $\calX$. 
\end{corollary}

\begin{definition}
Let $\calX$ be an $\infty$-category. A {\it zero object} of $\calX$ is an object which is both initial and final.
\end{definition}

\begin{lemma}\label{pointer}
Let $\calX$ be an $\infty$-category with a final object $1_{\calX}$. Then the inclusion
$i: \calX^{1_{\calX}/} \subseteq \calX_{\ast}$ is an equivalence of $\infty$-categories.
\end{lemma}

\begin{proof}
Let $K$ be the full subcategory of $\calX$ spanned by the final objects, and let
$1_{\calX}$ be an object of $K$. Proposition \ref{initunique} implies that $K$ is a contractible Kan complex, so that the inclusion $\{1_{\calX} \} \subseteq K$ is an equivalence of $\infty$-categories.
Corollary \ref{tweezegork} implies that the projection 
$\calX_{\ast} \rightarrow K$ is a coCartesian fibration. We now apply Proposition \ref{basechangefunky} to deduce the desired result.
\end{proof}

\begin{lemma}\label{pointerprime}
Let $\calX$ be an $\infty$-category with a final object. Then the $\infty$-category
$\calX_{\ast}$ has a zero object. If $\calX$ already has a zero object, then the forgetful functor
$\calX_{\ast} \rightarrow \calX$ is an equivalence of $\infty$-categories.
\end{lemma}

\begin{proof}
Let $1_{\calX}$ be a final object of $\calX$, and let $U = \id_{1_{\calX}} \in \calX_{\ast}$. 
We wish to show that $U$ is a zero object of $\calX_{\ast}$. According to Lemma \ref{pointer}, 
it will suffice to show that $U$ is a zero object
of $\calX_{1_{\calX}/}$. It is clear that $U$ is initial, and the finality of $U$ follows from
Proposition \ref{needed17}. 

For the second assertion, let us suppose that $1_{\calX}$ is also an initial object of $\calX$.
We wish to show that the forgetful functor $\calX_{\ast} \rightarrow \calX$ is an equivalence of $\infty$-categories. Applying Lemma \ref{pointer}, it will suffice to show that the projection
$f: \calX^{1_{\calX}/} \rightarrow \calX$ is an equivalence of $\infty$-categories. 
But $f$ is a trivial fibration of simplicial sets.
\end{proof}

\begin{lemma}\label{postEM}
Let $\calX$ be an $\infty$-category, and let $f: \calX_{\ast} \rightarrow \calX$
be the forgetful functor $($ which carries a pointed object $1 \rightarrow X$ to $X$ $)$.
Then $f$ induces an equivalence of $\infty$-categories
$$ \Group( \calX_{\ast}) \rightarrow \Group( \calX).$$
\end{lemma}

\begin{proof}
The functor $f$ factors as a composition
$$ \calX_{\ast} \subseteq \Fun(\Delta^1,\calX) \rightarrow \calX$$
where the first map is the inclusion of a full subcategory which is stable under limits,
and the second map preserves all limits (Proposition \ref{limiteval}). It follows that $f$ preserves limits, and therefore composition with $f$ induces a functor $F: \Group(\calX_{\ast}) \rightarrow \Group(\calX)$ by Corollary \ref{grpstable}. 

Observe that the $0$-simplex $\Delta^0$ is an initial object of $\cDelta^{op}$. Consequently, there exists a functor $T: \Delta^1 \times \Nerve(\cDelta)^{op} \rightarrow \Nerve(\cDelta)^{op}$, which is a natural transformation from the constant functor taking the value $\Delta^0$ to the identity functor. Composition with $T$ induces a functor
$$ \calX_{\Delta} \rightarrow \Fun(\Delta^1,\calX)_{\Delta}.$$
Restricting to group objects, we get a functor $s: \Group(\calX) \rightarrow \Group(\calX_{\ast})$.
It is clear that $F \circ s$ is the identity. 

We observe that if $\calX$ has a zero object, then
$f$ is an equivalence of $\infty$-categories (Lemma \ref{pointerprime}). It follows immediately that $F$ is an equivalence of $\infty$-categories. Since $s$ is a right inverse to $F$, we conclude that $s$ is an equivalence of $\infty$-categories as well.

To complete the proof in the general case, it will suffice to show that the composition $s \circ F$ is an equivalence of $\infty$-categories. To prove this, we set $\calY = \calX_{\ast}$, and let
$F': \Group(\calY_{\ast}) \rightarrow \Group(\calX_{\ast})$ and $s': \Group(\calY) \rightarrow \Group(\calY_{\ast})$ be defined as above. We then have a commutative diagram
$$ \xymatrix{ \Group(\calY) \ar[r]^-{F} \ar[d]^-{s'} & \Group(\calX) \ar[d]^-{s} \\
\Group(\calY_{\ast}) \ar[r]^-{F'} & \Group(\calX_{\ast}) }$$ 
so that $s \circ F = F' \circ s'$. Lemma \ref{pointerprime} implies that $\calY$ has a zero object, so that $F'$ and $s'$ are equivalences of $\infty$-categories. Therefore $F' \circ s' = s \circ F$ is an equivalence of $\infty$-categories, and the proof is complete.
\end{proof}

The following Proposition guarantees a good supply of Eilenberg-MacLane objects in an $\infty$-topos $\calX$.

\begin{lemma}\label{preEM}
Let $\calX$ be an $\infty$-topos containing a final object $1_{\calX}$ and let $n \geq 1$.
Let $p$ denote the composition
$$ \Fun(\Delta^1,\calX) \stackrel{\mCech}{\rightarrow} \calX_{\cDelta_{+}} \rightarrow \calX_{\cDelta}$$
which associates to each morphism $U \rightarrow X$ the underlying groupoid of its \Cech nerve.
Then:
\begin{itemize}
\item[$(1)$] Let $\calX'$ denote the full subcategory of $\Fun(\Delta^1,\calX)$ consisting
of {\em connected} pointed objects of $\calX$. Then the restriction of $p$ induces an equivalence of $\infty$-categories from $\calX'$ to the $\infty$-category $\Group(\calX)$.

\item[$(2)$] The essential image of $p| \EM_{n}(\calX)$ coincides with
the essential image of the composition
$$ \Group( \EM_{n-1}(\calX)) \subseteq \Group( \calX_{\ast}) \rightarrow \Group(\calX).$$
\end{itemize}
\end{lemma}

\begin{proof}
Let $\calX''$ be the full subcategory of $\Fun(\Delta^1,\calX)$ spanned by the effective epimorphisms
$u: U \rightarrow X$. Since $\calX$ is an $\infty$-topos, $p$ induces an equivalence
from $\calX''$ to the $\infty$-category of groupoid objects of $\calX$. Consequently, to prove
$(1)$, it will suffice to show that if $u: 1_{\calX} \rightarrow X$ is a morphism in $\calX$ and
$1_{\calX}$ is a final object, then $u$ is an effective epimorphism if and only if $X$ is connected.
We note that $X$ is connected if and only if the map $\tau_{\leq 0}(u):
\tau_{\leq 0} 1_{\calX} \rightarrow \tau_{\leq 0} X$ is an isomorphism in the ordinary
topos $\Disc(\calX)$. According to Proposition \ref{pi00detects}, $u$
is an effective epimorphism if and only if $\tau_{\leq 0}(u)$ is an effective epimorphism.
We now observe that in any ordinary category $\calC$, an effective epimorphism
$u': 1_{\calC} \rightarrow X'$ whose source is a final object of $\calC$ is automatically an isomorphism, since the equivalence relation $1_{\calC} \times_{X'} 1_{\calC} \subseteq 1_{\calC} \times 1_{\calC}$ automatically consists of the whole of $1_{\calC} \times 1_{\calC} \simeq 1_{\calC}$.

To prove $(2)$, we consider an augmented simplicial object $X_{\bigdot}$ of $\calX$
which is a \Cech nerve, having the property that $X_0$ is a final object of $\calX$.
We wish to show that the pointed object $X_0 \rightarrow X_{-1}$ belongs to
$\EM_{n}(\calX)$ if and only if each $X_{k}$ is $(n-1)$-truncated and $(n-1)$-connective,
for $k \geq 0$. We conclude by making the following observations:
\begin{itemize}
\item[$(a)$] Since $X_{k}$ is equivalent to a $k$-fold product of copies of $X_{1}$, the
objects $X_{k}$ are $(n-1)$-truncated ($(n-1)$-connective) for all $k \geq 0$ if and only if
$X_{1}$ is $(n-1)$-truncated ($(n-1)$-connective).

\item[$(b)$] We have a pullback diagram
$$ \xymatrix{ X_{1} \ar[r]^-{f} \ar[d] & X_0 \ar[d]^-{g} \\
X_0 \ar[r]^-{g} & X_{-1}. }$$
The object $X_{1}$ is $(n-1)$-truncated if and only if $f$ is $(n-1)$-truncated.
Since $g$ is an effective epimorphism, $f$ is $(n-1)$-truncated if and only if $g$ is $(n-1)$-truncated (Proposition \ref{hintdescent0}). Using the long exact sequence of Remark \ref{sequence}, we conclude that this is equivalent to the vanishing of $g^{\ast} \pi_k X_{-1}$ for $k > n$.
Since $g$ is an effective epimorphism, this is equivalent to the vanishing of
$\pi_k X_{-1}$ for $k > n$; in other words, to the requirement that $X_{-1}$ is $n$-truncated.

\item[$(c)$] The object $X_{1}$ is $(n-1)$-connective if and only if
$f$ is $(n-1)$-connective. Arguing as above, we conclude that $f$ is $(n-1)$-connective if and only if $g$ is $(n-1)$-connective (Proposition \ref{inftychange}). Using the long exact sequence of Remark \ref{sequence}, this is equivalent to the vanishing of the homotopy sheaf $g^{\ast} \pi_k X_{-1}$ for $k < n$. Since $g$ is an effective epimorphism, this is equivalent to the vanishing of $\pi_k X_{-1}$ for $k < n$; in other words, to the condition that $X_{-1}$ is $(n-1)$-truncated.
\end{itemize}

\end{proof}

\begin{proposition}\label{EM}
Let $\calX$ be an $\infty$-topos and $n \geq 0$ a nonnegative integer, and let 
$\pi_n: \calX_{\ast} \rightarrow \Nerve(\Disc(\calX))$ denote the associated homotopy group functor.

Then:
\begin{itemize}
\item[$(1)$] If $n = 0$, then $\pi_n$ determines an equivalence from the $\infty$-category $\EM_{0}(\calX)$ to the $($nerve of the$)$ category of pointed objects of $\Disc(\calX)$.
\item[$(2)$] If $n = 1$, then $\pi_n$ determines an equivalence from the $\infty$-category $\EM_{1}(\calX)$ to the $($nerve of the$)$ category of group objects of $\Disc(\calX)$.
\item[$(3)$] If $n \geq 2$, then $\pi_n$ determines an equivalence from the $\infty$-category $\EM_{n}(\calX)$ to the $($nerve of the$)$ category of commutative group objects of $\Disc(\calX)$.
\end{itemize}
\end{proposition}

\begin{proof}
We use induction on $n$. The case $n=0$ follows immediately from the definitions. The
case $n=1$ follow from the case $n=0$, by combining Lemmas \ref{preEM} and \ref{postEM}.
If $n=2$, we apply the inductive hypothesis, together with Lemma \ref{preEM} and the observation
that if $\calC$ is an ordinary category which admits finite products, then $\Group( \Group(\calC))$ is equivalent to category $\Ab(\calC)$ of {\em commutative} group objects of $\calC$. The argument
in the case $n > 2$ makes use of the inductive hypothesis, Lemma \ref{preEM}, and the observation that $\Group( \Ab(\calC) )$ is equivalent to $\Ab(\calC)$ for any ordinary category $\calC$ which admits finite products.
\end{proof}

Fix an $\infty$-topos $\calX$, a final object $1_{\calX} \in \calX$, and an integer $n \geq 0$. According to Proposition \ref{EM}, there exists a homotopy inverse to the functor $\pi$. We will denote this functor by
$$ A \mapsto (p: 1_{\calX} \rightarrow K(A,n) )$$
where $A$ is a pointed object of the topos $\Disc(\calX)$ if $n = 0$, a group object
if $n =1$, and an abelian group object if $n \geq 2$.\index{not}{K(A,n)@$K(A,n)$}

\begin{remark}\label{produm}
The functor $A \mapsto K(A,n)$ preserves finite products. This is clear, since the class of Eilenberg-MacLane objects is stable under finite products (Remark \ref{prodem}) and the homotopy inverse functor $\pi$ commutes with finite products (since homotopy groups are constructed using pullback and truncation functors, each of which commutes with finite products).
\end{remark}

\begin{definition}
Let $\calX$ be an $\infty$-topos, $n \geq 0$ an integer, and $A$ an abelian group object of the topos $\Disc(\calX)$.
We define 
$$ \HH^{n}(\calX; A) = \pi_0 \bHom_{\calX}( 1_{\calX}, K(A,n) ); $$
we refer to $\HH^n(\calX; A)$ as the {\it $n$th cohomology group of $\calX$ with coefficients in $A$}.\index{not}{HncalXA@$\HH^{n}(\calX;A)$}\index{gen}{cohomology group!of an $\infty$-topos}
\end{definition}

\begin{remark}
It is clear that we can also make sense of $\HH^{1}(\calX; G)$ when $G$ is a sheaf of nonabelian groups, or $\HH^0(\calX; E)$ when $E$ is only a sheaf of (pointed) sets.
\end{remark}

\begin{remark}
It is clear from the definition that $\HH^{n}(\calX; A)$ is functorial in $A$. Moreover, this functor
commutes with finite products by Remark \ref{produm} (and the fact that products in $\calX$
are products in the homotopy category $h \calX$). If $A$ is an abelian group, then the multiplication
map $A \times A \rightarrow A$ induces a (commutative) group structure on $\HH^{n}(\calX;A)$. This justifies our terminology in referring to $\HH^{n}(\calX;A)$ as a cohomology {\em group}.
\end{remark}

\begin{remark}\label{compfood}
Let $\calC$ be a small category equipped with a Grothendieck topology, and let
$\calX$ be the $\infty$-topos $\Shv( \Nerve \calC)$ of sheaves of {\em spaces} on
$\calC$, so that the underlying topos $\Disc(\calX)$ is equivalent to the category of sheaves of {\em sets} on $\calC$. Let $A$ be a sheaf of abelian groups on $\calC$.
Then $\HH^{n}(\calX;A)$ may be identified with the $n$th cohomology group of $\Disc(\calX)$ with coefficients in $A$, in the sense of ordinary sheaf theory. To see this, choose a resolution
$$ A \rightarrow I^0 \rightarrow I^1 \rightarrow \ldots \rightarrow I^{n-1} \rightarrow J$$
of $A$ by abelian group objects of $\Disc(\calX)$, where each $I^{k}$ is injective. The complex
$$ I^0 \rightarrow \ldots \rightarrow J$$
may be identified, via the Dold-Kan correspondence, with a simplicial abelian group object
$C_{\bigdot}$ of $\Disc(\calX)$. Regard $C_{\bigdot}$ as a presheaf on $\calC$ with values in $\sSet$. Then:
\begin{itemize}
\item[$(1)$] The induced presheaf $F: \Nerve(\calC)^{op} \rightarrow \SSet$ belongs to
$\calX = \Shv(\Nerve(\calC)) \subseteq \calP( \Nerve(\calC))$ (this uses the injectivity
of the objects $I^k$) and is equipped with a canonical basepoint $p: 1_{\calX} \rightarrow F$.
\item[$(2)$] The pointed object $p: 1_{\calX} \rightarrow F$ is an Eilenberg-MacLane object of $\calX$, and there is a canonical identification $A \simeq p^{\ast}(\pi_n F)$. We may therefore identify
$F$ with $K(A,n)$.
\item[$(3)$] The set of homotopy classes of maps from $1_{\calX}$ to $F$ in $\calX$ may be identified with
the cokernel of the map $\Gamma( \Disc(\calX); I^{n-1} ) \rightarrow \Gamma( \Disc(\calX); J)$, which is
also the $n$th cohomology group of $\Disc(\calX)$ with coefficients in $A$ in the sense of classical sheaf theory.
\end{itemize} 
For further discussion of this point, we refer the reader to \cite{jardine}.
\end{remark}

We are ready to define the cohomological dimension of an $\infty$-topos.

\begin{definition}\label{codimmm}\index{gen}{dimension!cohomological}\index{gen}{cohomoogical dimension}
Let $\calX$ be an $\infty$-topos. We will say that $\calX$ has
{\it cohomological dimension $\leq n$} if, for any sheaf of
abelian groups $A$ on $\calX$, the cohomology group $\HH^k(\calX,A)$ vanishes for
$k > n$.
\end{definition}

\begin{remark}
For small values of $n$, some authors prefer to require a stronger
vanishing condition which applies also when $A$ is a non-abelian
coefficient system. The appropriate definition requires the
vanishing of cohomology for coefficient systems which are defined
only up to inner automorphisms, as in \cite{giraud}. With the
appropriate modifications, Theorem \ref{cohdim} below remains
valid for $n < 2$.
\end{remark}

The cohomological dimension of an $\infty$-topos $\calX$ is closely related to the homotopy dimension of $\calX$. If $\calX$ has homotopy dimension $\leq n$, then
$$ \HH^m (\calX; A)  = \pi_0 \bHom_{\calX}(1_{\calX}, K(A,m)) = \ast $$
for $m > n$ by Lemma \ref{pie}, so that $\calX$ is also of cohomological dimension
$\leq n$. We will establish a partial converse to this result. 

\begin{definition}\index{gen}{gerbe}\index{gen}{$n$-gerbe}
Let $\calX$ be an $\infty$-topos. An {\it $n$-gerbe} on $\calX$ is an object
$X \in \calX$ which is $n$-connective and $n$-truncated.
\end{definition}

Let $\calX$ be an $\infty$-topos containing an $n$-gerbe $X$, and let $f: \calX_{/X} \rightarrow \calX$ denote the associated geometric morphism. If $X$ is equipped with a base point $p: 1_{\calX} \rightarrow X$, then $X$
is canonically determined (as a pointed object) by $p^{\ast} \pi_n X$, by Proposition \ref{EM}.
We now wish to consider the case in which $X$ is {\em not} pointed. If $n \geq 2$, 
then $\pi_n X$ can be regarded as an abelian group object in the topos
$\Disc(\calX_{/X})$. 
Proposition \ref{nicelemma} implies that $\pi_n X \simeq f^{\ast} A$, where $A$ is a sheaf
of abelian groups on $\calX$, which is determined up to canonical isomorphism.
(In concrete terms, this boils down the observation that the $1$-connectivity of $X$ allows us
to extract higher homotopy groups without specifying a basepoint on $X$. )
In this situation, we will say that $X$ is {\it banded by $A$}.\index{gen}{gerbe!banded}

\begin{remark}
For $n < 2$, the situation is more complicated. We refer the reader to \cite{giraud} for a discussion.
\end{remark}

Our next goal is to show that the cohomology groups of an $\infty$-topos $\calX$ can be interpreted as classifying equivalence classes of $n$-gerbes over $\calX$. Before we can prove this, we need to establish some terminology.

\begin{notation}\index{not}{BandcalX@$\Band(\calX)$}
Let $\calX$ be an $\infty$-topos. We define a category $\Band(\calX)$ as follows:
\begin{itemize}
\item[$(1)$] The objects of $\Band(\calX)$ are pairs $(U,A)$, where $U$ is an object of $\calX$
and $A$ is an abelian group object of the homotopy category $\Disc( \calX_{/U} )$. 
\item[$(2)$] Morphisms from $(U,A)$ to $(U',A')$ are given by pairs
$(\eta, f)$, where $\eta \in \pi_0 \bHom_{\calX}(U,U')$ and $f: A \rightarrow A'$ is a map
which induces an isomorphism $A \simeq \eta^{\ast} A'$ of abelian group objects. Composition of morphisms is defined in the obvious way.
\end{itemize}

For $n \geq 2$, let $\Gerb_n(\calX)$ denote the subcategory of $\Fun(\Delta^1,\calX)$ spanned by those objects $f: X \rightarrow S$ which are $n$-gerbes in $\calX_{/S}$ and those morphisms which correspond to pullback diagrams
$$ \xymatrix{ X' \ar[r] \ar[d]^-{f} & X \ar[d]^-{f} \\
S' \ar[r] & S. }$$\index{not}{GerbncalX@$\Gerb_n(\calX)$}

\begin{remark}\label{sumh}
Since the class of morphisms $f: X \rightarrow S$ which belong to $\calX^{\Delta^1}$ is stable under pullback, we can apply Corollary \ref{tweezegork} (which asserts that
$p: \Fun(\Delta^1,\calX) \rightarrow \Fun( \{1\}, \calX)$ is a Cartesian fibration), Lemma \ref{charpull} (which characterizes the $p$-Cartesian morphisms of $\Fun(\Delta^1,\calX)$), and Corollary \ref{relativeKan} to deduce that the projection $\Gerb_{n}(\calX) \rightarrow \calX$ is a right fibration.
\end{remark}

If $f: X \rightarrow U$ belongs to $\Gerb_n(\calX)$, then 
there exists an abelian group object $A$ of $\Disc( \calX_{/U} )$ such that $X$ is banded by $A$. The construction
$$ (f: X \rightarrow U) \mapsto (U,A)$$
determines a functor
$$ \chi: \Gerb_n(\calX) \rightarrow \Nerve(\Band(\calX)).$$

Let $A$ be an abelian group object of $\Disc( \calX)$. We let $\Band^{A}(\calX)$\index{not}{BandAcalX@$\Band^{A}(\calX)$}
be the category whose objects are triples $(X, A_{X}, \phi)$, where $X \in \h{\calX}$,
$A_{X}$ is an abelian group object of $\Disc( \calX_{/X})$, and $\phi$ is a map
$A_{X} \rightarrow A$ which induces an isomorphism $A_{X} \simeq A \times X$ of abelian
group objects of $\Disc( \calX_{/X})$. We have forgetful functors
$$ \Band^{A}(\calX) \stackrel{\phi}{\rightarrow} \Band(\calX) \rightarrow \h{\calX},$$
both of which are Grothendieck fibrations and whose composition is an equivalence of categories.
We define $\Gerb^{A}_n(\calX)$ by the following pullback diagram:
$$ \xymatrix{ \Gerb_n^{A}(\calX) \ar[r] \ar[d] & \Gerb_n(\calX) \ar[d]^-{\chi} \\
\Nerve(\Band^{A}(\calX)) \ar[r] & \Nerve(\Band(\calX)). }$$
Note that since $\phi$ is a Grothendieck fibration, $\Nerve \phi$ is 
a Cartesian fibration (Remark \ref{gcart}), so that the diagram above is homotopy Cartesian
( Proposition \ref{basechangefunky} ). We will refer to $\Gerb_n^{A}(\calX)$ as the {\it sheaf of gerbes over $\calX$ banded by $A$}.\index{not}{GerbnAcalX@$\Gerb_n^{A}(\calX)$}
\end{notation}

More informally: an object of $\Gerb_n^{A}(\calX)$ is an $n$-gerbe $f: X \rightarrow U$
in $\calX_{/U}$ {\em together with} an isomorphism $\phi_{X}: \pi_{n} X \simeq X \times A$
of abelian group objects of $\Disc( \calX_{/X})$. Morphisms in $\Gerb_n^{A}$
are given by pullback squares
$$ \xymatrix{ X' \ar[d] \ar[r]^-{f} & X \ar[d] \\
U' \ar[r] & U }$$
such that the associated diagram of abelian group objects of $\Disc( \calX_{/X'})$
$$ \xymatrix{ & f^{\ast}(\pi_n X) \ar[dr]^-{ f^{\ast} \phi_{X} } & \\
\pi_n X' \ar[ur]^-{\pi_n f} \ar[rr]^-{\phi_{X'}} & & A \times X' }$$
is commutative.

\begin{lemma}\label{stareye}
Let $\calX$ be an $\infty$-topos, $n \geq 1$, and $A$ an abelian group object in the topos
$\Disc( \calX)$. Let $X$ be an $n$-gerbe in $\calX$ equipped with a fixed isomorphism
$\phi: \pi_n X \simeq X \times A$ of abelian group objects of $\Disc( \calX_{/X})$,
and let $u: 1_{\calX} \rightarrow K(A,n)$ be an Eilenberg-MacLane object of $\calX$ classified by $A$. Let $\bHom_{\calX}^{\phi}( K(A,n), X)$ be the summand of $\bHom_{\calX}( K(A,n), X)$
corresponding to those maps $f: K(A,n) \rightarrow X$ for which the composition
$$ A \times K(A,n) \simeq \pi_n K(A,n) \rightarrow f^{\ast}(\pi_n X) \stackrel{f^{\ast} \phi}{\rightarrow}
A \times K(A,n)$$
is the identity $($ in the category of abelian group objects of $\calX_{/K(A,n)}$ $)$. Then
composition with $u$ induces a homotopy equivalence
$$ \theta^{\phi}: \bHom_{\calX}^{\phi}( K(A,n), X) \rightarrow \bHom_{\calX}( 1_{\calX}, X).$$
\end{lemma}

\begin{proof}
Let $\theta: \bHom_{\calX}( K(A,n), X) \rightarrow \bHom_{\calX}(1_{\calX}, X)$, and let
$f: 1_{\calX} \rightarrow X$ be any map (which we may identify with an Eilenberg-MacLane object of $\calX$. The homotopy fiber of $\theta$ over the point represented by $f$ can be identified
with $\bHom_{\calX_{1_{\calX}/}}( u, f)$. In view of the equivalence between 
$\calX_{1_{\calX}/}$ and $\calX_{\ast}$, we can identify this mapping space with
$\bHom_{\calX_{\ast}}( u, f)$. Applying Proposition \ref{EM}, we deduce that
the homotopy fiber of $\theta$ is equivalent to the (discrete) set of all endomorphisms $v: A \rightarrow A$ (in the category of group objects of $\Disc( \calX)$). We now observe that
the homotopy fiber of $\theta^{\phi}$ over $f$ is a summand of the homotopy fiber of $\theta$ over $f$, corresponding to those components for which $v = \id_{A}$. It follows that the homotopy fibers
of $\theta^{\phi}$ are contractible, so that $\theta^{\phi}$ is a homotopy equivalence as desired.
\end{proof}

\begin{lemma}\label{starsky}
Let $\calX$ be an $\infty$-topos, $n \geq 1$, and $A$ an abelian group object of 
$\Disc( \calX)$. Let $f: K(A,n) \times X \rightarrow X$ be a trivial $n$-gerbe over
$X$ banded by $A$, and $g: \widetilde{Y} \rightarrow Y$ any $n$-gerbe over $Y$ banded by $A$.
Then there is a canonical homotopy equivalence
$$ \bHom_{ \Gerb^A_{n} }(f,g) \simeq \bHom_{ \calX } (X, \widetilde{Y} ).$$
\end{lemma}

\begin{proof}
Choose a morphism $\alpha: \id_{X} \rightarrow f$ in $\calX_{/X}$ as depicted below:
$$ \xymatrix{ X \ar[d] \ar[r]^-{s} & X \times K(A,n) \ar[d]^-{f} \\
X \ar[r]^-{\id_{X} } & X }$$
which exhibits $f$ as an Eilenberg-MacLane object of $\calX_{/X}$. We observe that
evaluation at $\{0\} \subseteq \Delta^1$ induces a trivial fibration
$$ \Hom^{\lft}_{\calX^{\Delta^1}}( \id_{X}, g) \rightarrow \Hom^{\lft}_{\calX}(X, \widetilde{Y}).$$
Consequently, we may identify $\bHom_{\calX}(X, \widetilde{Y})$ with the Kan complex
$$ Z = \Fun(\Delta^1,\calX)_{\id_{X}/} \times_{ \Fun(\Delta^1,\calX) } \{ g \}. $$
Similarly, the trivial fibration $\Fun(\Delta^1,\calX)_{\alpha/} \rightarrow \Fun(\Delta^1,\calX)_{f/}$
allows us to identify $\bHom_{\Gerb_n}(f,g)$ with the Kan complex
$$Z' = \Fun(\Delta^1,\calX)_{\alpha/} \times_{\Fun(\Delta^1,\calX) } \{ g\},$$
and $\bHom_{\Gerb_n}(f,g)$ with the summand $Z''$ of $Z'$ corresponding to those maps
which induce the identity isomorphism of $A \times (K(A,n) \times X)$ (in the category of group
objects of $\Disc( \calX_{/ K(A,n) \times X})$). We now observe that evaluation at
$\{1\} \subseteq \Delta^1$ gives a commutative diagram
$$ \xymatrix{ Z'' \ar[r] \ar[dr]^-{\psi''} & Z' \ar[d]^-{\psi'} \ar[r] & Z \ar[d]^-{\psi} \\
& \calX_{\id_{X}/} \times_{\calX} \{Y \} \ar[r] & \calX_{X/} \times_{\calX} \{Y\} }.$$
where the vertical maps are Kan fibrations. If we fix a pullback square
$$ \xymatrix{ \widetilde{X} \ar[r] \ar[d]^-{g'} & \widetilde{Y} \ar[d] \\
X \ar[r]^-{h} & Y,} $$
then we can identify $\psi^{-1} \{h\}$ with $\bHom_{\calX^{/X}}( \id_{X}, g')$,
${\psi'}^{-1} \{ s^0 h \}$ with $\bHom_{\calX^{/X}}( X \times K(A,n), g')$, 
${\psi'})^{-1} \{ s^0 h\}$ with the summand of $\bHom_{\calX^{/X}}( X \times K(A,n), g')$
corresponding to those maps which induce the identity on $A \times (K(A,n) \times X)$ (in the category of group objects of $\Disc( \calX_{/ K(A,n) \times X})$), and $\theta$ with
the map given by composition with $s$. Invoking Lemma \ref{stareye} in the $\infty$-topos
$\calX^{/X}$, we deduce that the map $\theta$ in the diagram
$$ \xymatrix{
Z'' \ar[r]^-{\theta} \ar[d]^-{\psi''} & Z \ar[d]^-{\psi} \\
 \calX_{\id_{X}/} \times_{\calX} \{Y \} \ar[r] & \calX_{X/} \times_{\calX} \{Y\} }$$
induces homotopy equivalences from the fibers of $\psi''$ to the fibers of $\psi$. Since the 
lower horizontal map is a trivial fibration of simplicial sets, we conclude that $\theta$ is itself a homotopy equivalence, as desired.
\end{proof}

\begin{theorem}\label{starthm}
Let $\calX$ be an $\infty$-topos, $n \geq 1$, and $A$ an abelian group object of
$\Disc( \calX)$. Then:

\begin{itemize}
\item[$(1)$] The composite map
$$ \theta: \Gerb_n^{A}(\calX) \rightarrow \Gerb_n(\calX) \subseteq \Fun(\Delta^1,\calX) \rightarrow
\Fun(\{1\}, \calX) \simeq \calX $$ is a right fibration.

\item[$(2)$] The right fibration $\theta$ is representable by an Eilenberg-MacLane object
$K(A,n+1)$. 
\end{itemize}
\end{theorem}

\begin{proof}
For each object $X \in \calX$, we let $A_{X}$ denote the projection $A \times X \rightarrow X$, viewed as an abelian group object of $\Disc( \calX_{/X})$.  
The functor $\phi: \Band^{A}(\calX) \rightarrow \Band(\calX)$
is a fibration in groupoids, so that $\Nerve \phi$ is a right fibration (Proposition \ref{stinkyer}). 
The functor $\theta$ admits a factorization
$$ \Gerb_n^{A}(\calX) \stackrel{\theta'}{\rightarrow} \Gerb_n(\calX) \stackrel{\theta''}{\rightarrow} \calX$$
where $\theta''$ is a right fibration (Remark \ref{sumh}) and $\theta'$ is a pullback of
$\Nerve \phi$, and therefore also a right fibration. It follows that $\theta$, being a composition of right fibrations, is a right fibration; this proves $(1)$.

To prove $(2)$, we consider an Eilenberg-MacLane object $u: 1_{\calX} \rightarrow K(A,n+1)$.
Since $K(A,n+1)$ is $(n+1)$-truncated and $1_{\calX}$ is $n$-truncated (in fact, $(-2)$-truncated), 
Lemma \ref{trunccomp} implies that $u$ is $n$-truncated. 
The long exact sequence
$$\ldots \rightarrow u^{\ast} \pi_{i+1} K(A,n+1) \rightarrow \pi_i u \rightarrow \pi_i(1_{\calX}) \rightarrow i^{\ast}
\pi_i( K(A,n+1) ) \rightarrow \pi_{i-1}(u) \rightarrow \ldots$$
of Remark \ref{sequence} shows that $u$ is $n$-connective, and provides an
isomorphism $\phi: A \simeq \pi_n(u)$ in the category of group objects of $\Disc( \calX)$,
so that we may view the pair $(u,\phi)$ as an object of $\Gerb_n^{A}(\calX)$. Since
$1_{\calX}$ is a final object of $\calX$, Lemma \ref{starsky} implies that $(u,\phi)$ 
is a final object of $\Gerb_n^{A}(\calX)$, so that the right fibration $\theta$ is representable
by $\theta(u,\phi) = K(A,n+1)$.
\end{proof}

\begin{corollary}\label{coclassify}
Let $\calX$ be an $\infty$-topos, $n \geq 2$, and $A$ an abelian group object
of $\Disc( \calX)$. There is a canonical bijection of
$\HH^{n+1}(\calX; A)$ with the set of equivalence classes of $n$-gerbes on $\calX$ banded by $A$.
\end{corollary}

\begin{remark}
Under the correspondence of Proposition \ref{coclassify}, an $n$-gerbe $X$ on $\calX$
admits a global section $1_{\calX} \rightarrow X$ if and only if the associated cohomology
class in $\HH^{n+1}(\calX;A)$ vanishes.
\end{remark}

\begin{theorem}\label{cohdim}
Let $\calX$ be an $\infty$-topos and $n \geq 2$. Then $\calX$ has
cohomological dimension $\leq n$ if and only if it satisfies the
following condition: any $n$-connective, truncated object of
$\calX$ admits a global section.
\end{theorem}

\begin{proof}
Suppose that $\calX$ has the property that every
$n$-connective, truncated object $X \in \calX$ admits a global
section. As in the proof of Lemma \ref{pie}, we deduce that for any
truncated, $(n+1)$-connective object $X \in \calX$, the space of global sections 
$\bHom_{\calX}(1,X)$ is connected.
Let $k > n$, and let $G$ be a sheaf of abelian groups on $\calX$. Then $K(G,k)$ is
$(n+1)$-connective, so that $\HH^k(\calX,G) = \ast$. Thus $\calX$ has
cohomological dimension $\leq n$.

For the converse, let us assume that $\calX$ has cohomological
dimension $\leq n$ and let $X$ denote an $n$-connective,
$k$-truncated object of $\calX$. We will show that $X$ admits a
global section by descending induction on $k$. If $k \leq n-1$, then $X$ is a final
object of $\calX$, so there is nothing to prove. In the general case, choose a truncation
$X \rightarrow \tau_{\leq k-1} X$; we may assume
by the inductive hypothesis that $\tau_{\leq k-1} X$ has a global section $s: 1
\rightarrow \tau_{\leq k-1} X$.
Form a pullback square
$$ \xymatrix{ X' \ar[r] \ar[d] & X \ar[d] \\
1 \ar[r]^-{s} & \tau_{\leq k-1} X. }$$ 
It now suffices to prove that $X'$ has a global section. We note that $X'$ is $k$-connective, where $k \geq n \geq 2$. It follows that $X'$ is a $k$-gerbe on $\calX$; suppose it is banded by
an abelian group object $A \in \Disc(\calX)$. 
According to Corollary \ref{coclassify}, $X'$ is classified up to equivalence by an element in $\HH^{k+1}(\calX, A)$, which vanishes in virtue of the fact that
$k+1 > n$ and the cohomological dimension of $\calX$ is $\leq n$. Consequently, $X'$ is equivalent to $K(A,k)$ and therefore admits a global section.
\end{proof}

\begin{corollary}\label{confusion}
Let $\calX$ be an $\infty$-topos. If $\calX$ has homotopy
dimension $\leq n$, then $\calX$ has cohomological dimension $\leq
n$. The converse holds provided that $\calX$ has finite homotopy
dimension and $n \geq 2$.
\end{corollary}

\begin{proof}
Only the last claim requires proof. Suppose that $\calX$ has
cohomological dimension $\leq n$ and homotopy dimension $\leq k$.
We must show that every $n$-connective object $X$ of $\calX$
has a global section. Choose a truncation $X \rightarrow \tau_{\leq k-1} X$.
Then $\tau_{\leq k-1} X$ is truncated and
$n$-connective, so it admits a global section by Theorem
\ref{cohdim}. Form a pullback square
$$ \xymatrix{ X' \ar[r] \ar[d] & X \ar[d] \\
1 \ar[r] & \tau_{\leq k-1} X.}$$
It now suffices to prove that $X'$ has a global section. But $X'$ is
$k$-connective, and therefore has a global section in virtue of the assumption that
$\calX$ has homotopy dimension $\leq k$.
\end{proof}

\begin{warning}\label{injur}[Weiland]
The converse to Corollary \ref{confusion} is false if we do not assume that
$\calX$ has finite homotopy dimension. To see this, we discuss the following example, which
we learned from Ben Wieland. Let $G$ denote the group $\Z_{p}$ of $p$-adic integers
(viewed as a profinite group). Let $\calC$ denote the category whose objects are
the finite quotients $\{ \Z_{p} / p^{n} \Z_{p} \}_{ n \geq 0}$, and whose morphisms
are given by $G$-equivariant maps. We regard $\calC$ as endowed with a Grothendieck topology in which every nonempty sieve is a covering. The $\infty$-topos $\Shv(\Nerve \calC)$
is $1$-localic, and the underlying ordinary topos $\h{ \tau_{\leq 0} \Shv(\Nerve \calC) }$ can be identified
with the category $BG$ of continuous $G$-sets (that is, sets $C$ equipped with an action of $G$
such that the stabilizer of each element $x \in C$ is an open subgroup of $G$). Since
the profinite group $G$ has cohomology dimension $2$ (see \cite{serre}), we deduce that $\calX$ is
of cohomological dimension $2$. However, we will show that $\calX$ is not hypercomplete, and therefore cannot be of finite homotopy dimension.

Let $K$ be a finite CW complex whose homotopy groups consist entirely of $p$-torsion
(for example, we could take $K$ to be a Moore space $M( \Z/p\Z)$), and let
$X = \Sing K \in \SSet$. Let $F: \Nerve(\calC)^{op} \rightarrow \SSet$ denote the constant functor taking the value $X$. We claim that $F$ belongs to
$\Shv(\calC)$. Unwinding the definitions, we must show that for each $m \leq n$, 
the diagram $F$ exhibits $F( \Z_{p} / p^{m} \Z_p)$ as equivalent to the homotopy
invariants for the trivial action of $p^{m} \Z_{p}/ p^{n} \Z_p$ on $F( \Z_{p} / p^{n} \Z_p)$. In
other words, we must show that the diagonal embedding
$$ \alpha: X \rightarrow \Fun( BH, X)$$
is a homotopy equivalence, where $H$ denotes the quotient group
$p^{m} \Z_{p}/ p^{n} \Z_p$. Since both sides are $p$-adically complete, it will suffice
to show that $\alpha$ is a $p$-adic homotopy equivalence, which follows from a suitable version of the Sullivan conjecture (see, for example, \cite{schwartz}). 

We define another functor $F': \Nerve(\calC)^{op} \rightarrow \SSet$, which is obtained as the simplicial nerve of the functor described by the formula
$$ \Z_{p}/ p^{n} \Z_{p} \mapsto \Sing (K^{\R / p^{n} \Z}).$$
For $m \leq n$, the loop space $K^{ \R/ p^{m} \Z}$ can be identified with the homotopy
fixed points of the (nontrivial) action of $H = p^{m} \Z_{p} / p^{n} \Z_{p} \simeq p^{m} \Z / p^{n} \Z$
on the loop space $K^{ \R/ p^{n} \Z}$: this follows from the observation that $H$ acts freely on
$\R / p^{n} \Z$, with quotient $\R / p^{m} \Z$. Consequently, $F'$ belongs to $\Shv( \Nerve(\calC))$.

The inclusion of $K$ into each loop space $K^{ \R / p^{n} \Z}$ induces a morphism
$\alpha: F \rightarrow F'$ in the $\infty$-topos $\Shv( \Nerve(\calC) )$. Using the fact
that the homotopy groups of $K$ are $p$-torsion, we deduce that the morphism
$\alpha$ is $\infty$-connective (this follows from the observation that the map
$$X \simeq \varinjlim F( \Z_p / p^{n} \Z_p) \rightarrow
\varinjlim F'( \Z / p^{n} \Z_P)$$ is a homotopy equivalence). However,
the morphism $\alpha$ is not an equivalence in $\Shv( \Nerve(\calC) )$ unless $K$ is essentially discrete. Consequently, $\Shv( \Nerve(\calC) )$ is not hypercomplete, and therefore cannot be of finite homotopy dimension.
\end{warning}

In spite of Warning \ref{injur}, many situations which guarantee that a topological space (or topos)
$X$ is of bounded cohomological dimension also guarantee that the associated $\infty$-topos is
of bounded homotopy dimension. We will see some examples in the next two sections.

\subsection{Covering Dimension}\label{covdim}

In this section, we will review the classical theory of covering
dimension for paracompact spaces, and then show that the covering
dimension of a paracompact space $X$ coincides with its homotopy
dimension.

\begin{definition}\label{paradim}\index{gen}{dimension!covering}\index{gen}{covering dimension}
A paracompact topological space $X$ has {\it covering
dimension $\leq n$} if the following condition is satisfied: for
any open covering $\{ U_{\alpha} \}$ of $X$, there exists an open
refinement $\{ V_{\alpha} \}$ of $X$ such that each intersection
$V_{\alpha_0} \cap \ldots \cap V_{\alpha_{n+1}} = \emptyset$
provided the $\alpha_i$ are pairwise distinct.
\end{definition}

\begin{remark}
When $X$ is paracompact, the condition of Definition \ref{paradim}
is equivalent to the (a priori weaker) requirement that such a refinement exist whenever
$\{U_{\alpha} \}$ is a finite covering of $X$. This weaker
condition gives a good notion whenever $X$ is a normal topological
space. Moreover, if $X$ is normal, then the covering dimension of
$X$ (by this second definition) coincides with the covering dimension of
the Stone-\Cech compactification of $X$. Thus, the dimension
theory of normal spaces is controlled by the dimension theory of
compact Hausdorff spaces.
\end{remark}

\begin{remark}
Suppose that $X$ is a compact Hausdorff space, which is written as
a filtered inverse limit of compact Hausdorff spaces $\{
X_{\alpha} \}$, each of which has dimension $\leq n$. Then $X$ has
dimension $\leq n$. Conversely, any compact Hausdorff space of
dimension $\leq n$ can be written as a filtered inverse limit of
finite simplicial complexes having dimension $\leq n$. Thus, the
dimension theory of compact Hausdorff spaces is controlled by the
(completely straightforward) dimension theory of finite simplicial
complexes.
\end{remark}

\begin{remark}
There are other approaches to classical dimension theory. For
example, a topological space $X$ is said to have {\it small
$($ large $)$ inductive dimension $\leq n$} if every point of $X$ (every
closed subset of $X$) has arbitrarily small open neighborhoods $U$
such that $\bd U$ has small inductive dimension $\leq n-1$. These
notions are well-behaved for separable metric spaces, where they
coincide with the covering dimension (and with each other). In
general, the covering dimension has better formal properties.
\end{remark}

Our goal in this section is to prove that the covering dimension of a paracompact topological space $X$ coincides with the homotopy dimension of $\Shv(X)$.
First, we need a technical
lemma.

\begin{lemma}\label{core}
Let $X$ be a paracompact space, $k \geq 0$, $\{U_{\alpha}
\}_{\alpha \in A}$ be a covering of $X$. Suppose that for every
$A_0 \subseteq A$ of size $k+1$, we are given a covering
$\{V_{\beta} \}_{\beta \in B(A_0)}$ of the intersection $U_{A_0} =
\bigcap_{\alpha \in A_0} U_{\alpha}$. Then there exists a covering
$\{ W_{\alpha} \}_{\alpha \in \widetilde{A}}$ of $X$ and a map
$\pi: \widetilde{A} \rightarrow A$ with the following properties:
\begin{itemize}
\item[$(1)$] For $\widetilde{\alpha} \in \widetilde{A}$ with $\pi(\widetilde{\alpha}) = \alpha$, we have $W_{\widetilde{\alpha}} \subseteq U_{\alpha}$. 

\item[$(2)$] Suppose that $\widetilde{\alpha}_0, \ldots, \widetilde{\alpha}_k$ is a collection
of elements of $\widetilde{A}$, with $\pi( \widetilde{\alpha}_i ) = \alpha_i$. Suppose
further that $A_0 = \{ \alpha_0, \ldots, \alpha_k \}$ has cardinality $(k+1)$ (in other words, the $\alpha_i$ are all disjoint from one another). 
Then there exists
$\beta \in B(A_0)$ such that $W_{\widetilde{\alpha}_0} \cap \ldots \cap
W_{\widetilde{\alpha}_k} \subseteq V_{\beta}$.
\end{itemize}
\end{lemma}

\begin{proof}
Since $X$ is paracompact, we may find a locally finite covering
$\{ U'_{\alpha} \}_{\alpha \in A}$ of $X$, such that the
each closure $\overline{ U'_{\alpha} }$ is contained in
$U_{\alpha}$. Let $S$ denote the set of all subsets $A_0 \subseteq
A$ having size $k+1$. For $A_0 \in S$, let $K(A_0) = \bigcap_{\alpha
\in A_0} \overline{U_{\alpha}}$. Now set 
$$\widetilde{A} = \{ (\alpha, A_0, \beta): \alpha \in A_0 \in S, \beta \in
B(A_0) \} \cup A.$$ 
For $\widetilde{\alpha} = (\alpha, A_0, \beta) \in \widetilde{A}$, we
set $\pi(\widetilde{\alpha}) = \alpha$ and 
$$W_{\widetilde{\alpha}} =
(U'_{\alpha} - \bigcup_{\alpha \in A'_0 \in S} K(A'_0) ) \cup (V_{\beta}
\cap U'_{\alpha}).$$
If $\alpha \in A \subseteq \widetilde{A}$, we let $\pi(\alpha) =
\alpha$ and $W_{\alpha} = U'_{\alpha} - \bigcup_{\alpha \in A_0 \in S}
K(A_0)$. The local finiteness of the cover $\{ U'_{\alpha} \}$ ensures that each
$W_{\widetilde{\alpha}}$ is an open set. It is now easy to check that the covering $\{ W_{ \widetilde{\alpha} } \}_{\widetilde{\alpha} \in \widetilde{A}}$ has the desired properties.
\end{proof}

\begin{theorem}\label{paradimension}
Let $X$ be a paracompact topological space of covering dimension
$\leq n$. Then the $\infty$-topos $\Shv(X)$ of sheaves on $X$ has
homotopy dimension $\leq n$.
\end{theorem}

\begin{proof}
We make use of the results and notations of \S \ref{paracompactness}.
Let $\calB$ denote the collection of all open $F_{\sigma}$ subsets of $X$, and fix a linear ordering on $\calB$. We may identify $\Shv(X)$ with the simplicial nerve of the category
of all functors $F: \calB^{op} \rightarrow \Kan$ which have the property that for any
$\calU \subseteq \calB$ with $U = \bigcup_{V \in \calU} V$, the natural map
$F(U) \rightarrow F(\calU)$ is a homotopy equivalence.

Suppose that $F: \calB^{op} \rightarrow \sSet$ represents an $n$-connective
sheaf; we wish to show that the simplicial set $F(X)$ is nonempty. It suffices to prove that
$F(\calU)$ is nonempty, for some covering $\calU$ of $X$; in other words, it suffices to produce a map $N_{\calU} \rightarrow F$. The idea is that since $X$ has finite covering dimension, we can choose arbitrarily fine covers $\calU$ such that $N_{\calU}$ is {\it $n$-dimensional}; that is, equal to its $n$-skeleton. 

For every simplicial set $K$, let $K^{(i)}$ denote the {\it $i$-skeleton} of $K$ (the union of all
nondegenerate simplices of $K$ of dimension $\leq i$). If $G: \calB^{op} \rightarrow \sSet$ is a simplicial presheaf, we let $G^{(i)}$ denote the simplicial presheaf given by the formula
$$ G^{(i)}(U) = (G(U))^{(i)}.$$

We will prove the following statement by induction on $i$, $-1
\leq i \leq n$:

\begin{itemize}
\item There exists an open cover $\calU_{i} \subseteq \calB$ of $X$
and a map $\eta_i: N_{\calU_i}^{(i)} \rightarrow
F$.
\end{itemize}

Assume that this statement holds for $i = n$. Passing to a
refinement, we may assume that the cover $\calU_{n}$ has the
property that no more than $n+1$ of its members intersect (this is
the step where we shall use the assumption on the covering
dimension of $X$). It follows that $N_{\calU_{n}}^{(n)} = N_{\calU_{n}}$, and the proof
will be complete.

To begin the induction in the case $i = -1$, we let $\calU_{-1} = \{ X \}$; the $(-1)$-skeleton of 
$N_{\calU_{-1}}$ is empty, so that $\eta_{-1}$ exists (and is unique).

Now suppose that $\calU_{i} = \{ U_{\alpha} \}_{ \alpha \in A} $ and $\eta_i$ have been constructed, $i < n$. Let $A_0 \subseteq A$ have cardinality $(i+2)$, and set $U(A_0) = \bigcap_{\alpha \in A_0} U_{\alpha}$; then $A_0$ determines an $n$-simplex of $N_{\calU_{i}}(U(A_0))$, so that
$\eta_i$ restricts to give a map
$$\eta_{i, A_0}: \bd \Delta^{i+1} \rightarrow F(U(A_0)).$$
By assumption, $F$ is $n$-connective; it follows that there is an open covering
$$ \{ V_{\beta} \}_{ \beta \in B(A_0) }$$
of $U(A_0)$, such for each $V_{\beta}$ there is a commutative diagram
$$ \xymatrix{ \bd \Delta^{i+1} \ar@{^{(}->}[d] \ar[r] & F(U(A_0)) \ar[d] \\
\Delta^{i+1} \ar[r] & F(V_{\beta} ). } $$

We apply Lemma \ref{core} to this data, to
obtain an new open cover $\calU_{i+1} = \{ W_{\widetilde{\alpha}} \}_{\widetilde{\alpha} \in
\widetilde{A} }$ which refines $\{ U_{\alpha} \}_{\alpha \in A}$.
Refining the cover further if necessary, we may assume that each
of its members belongs to $\calB$. By functoriality, we obtain a
map
$$ N^{(i)}_{\calU_{i+1} } \rightarrow F.$$
To complete the proof,
it will suffice to extend $f$ to the $(i+1)$-skeleton of the nerve
of $\{ W_{\alpha} \}_{\alpha \in \widetilde{A}}$. Let $\widetilde{A_0} \subseteq \widetilde{A}$
have cardinality $i+2$, and let $W( \widetilde{A_0}) = \bigcap_{\widetilde{\alpha} \in \widetilde{A_0}} W_{\widetilde{\alpha}}$; then we must solve a lifting problem
$$ \xymatrix{ \bd \Delta^{i+1} \ar@{^{(}->}[d] \ar[r] & F(W) \\
\Delta^{i+1}. \ar@{-->}[ur] & } $$
Let $\pi: \widetilde{A} \rightarrow A$ denote the map of Lemma \ref{core}.
If $A_0 = \pi( \widetilde{A}_0 )$ has cardinality smaller than $i+2$, then there is a canonical extension, given by applying $\pi$ and using $\eta_{i}$. Otherwise, Lemma \ref{core} guarantees that $W( \widetilde{A_0} ) \subseteq V_{\beta}$ for some $\beta \in B(A_0)$, so that the
desired extension exists by construction.
\end{proof}

\begin{corollary}\label{corub}
Let $X$ be a paracompact topological space. The following conditions are equivalent:
\begin{itemize}
\item[$(1)$] The covering dimension of $X$ is $\leq n$.
\item[$(2)$] The homotopy dimension of $\Shv(X)$ is $\leq n$.
\item[$(3)$] For every closed subset $A \subseteq X$, every $m \geq n$, and every
continuous map $f_0: A \rightarrow S^m$, there exists $f: X \rightarrow S^m$ extending
$f_0$.
\end{itemize}
\end{corollary}

\begin{proof}
The implication $(1) \Rightarrow (2)$ is Theorem \ref{paradimension}.
The equivalence $(1) \Leftrightarrow (3)$ follows from classical dimension theory (see, for example,  \cite{dimtheory}).
We will complete the proof by showing that $(2) \Rightarrow (3)$.
Let $A$ be a closed subset of $X$, $m \geq n$, and $f_0: A \rightarrow S^m$ a continuous map.
Let $\calB$ be the collection of all open $F_{\sigma}$ subsets of $X$.
We define a simplicial presheaf $F: \calB \rightarrow \Kan$, so that an $n$-simplex of $F(U)$
is a map $f$ rendering the diagram
$$ \xymatrix{ (U \cap A) \times | \Delta^n | \ar[r] \ar[d] & A \ar[d]^-{f_0} \\
U \times | \Delta^n | \ar[r]^-{f} \ar[r] & S^m}$$ commutative. 
To prove $(3)$, it will suffice to show that $F(X)$ is nonempty. In virtue of the assumption
that $\Shv(X)$ has homotopy dimension $\leq n$, it will suffice to show that $\calF$ is an $n$-connective sheaf on $X$.

We first show that $F$ is a sheaf. Choose a linear ordering on $\calB$. We must show that
for every open covering $\calU$ of $U \in \calB$, the natural map
$\calF(U) \rightarrow \calF(\calU)$ is a homotopy equivalence. The proof is similar to that of Proposition \ref{aese}. Let $\pi: |N_{\calU}|_{X} \rightarrow U$ be the projection; then
we may identify $F(\calU)$ with the simplicial set parametrizing
continuous maps $|N_{\calU}|_{X} \rightarrow S^m$, whose restriction to $\pi^{-1}(A)$ is
given by $f_0$. The desired equivalence now follows from the fact that $|N_{\calU}|_{X}$ is fiberwise homotopy equivalent to $U$ (Lemma \ref{partit}).

Now we claim that $\calF$ is $n$-connective as an object of $\Shv(X)$. In other words, we must show that for any $U \in \calB$, any $k \leq n$, and any map $g: \bd \Delta^k \rightarrow F(U)$, there is an open covering $\{ U_{\alpha} \}$ of $U$ and a family of commutative diagrams
$$ \xymatrix{ \bd \Delta^k \ar@{^{(}->}[d] \ar[r]^-{g} & F(U) \ar[d] \\
\Delta^k \ar[r]^-{g_{\alpha}} & F(U_{\alpha} ).} $$

We may identify $g$ with a continuous map
$$g: S^{k-1} \times U \rightarrow S^{m}$$
such that $g( z, a) = f_0(a)$ for $a \in A$. Choose a point $x \in U$.
Consider the map $g| S^{k-1} \times \{x\}$. Since $k-1 < n \leq m$, this map is nullhomotopic;
therefore it admits an extension $g'_{x}: D^k \times \{x\} \rightarrow S^m$. Moreover, if
$x \in A$, then we may choose $g'_{x}$ to be the constant map with value $f_0(x)$.
Amalgamating $g$, $g'_{x}$, and $f_0$, we obtain a continuous map
$$ g'_{0}: (S^{k-1} \times U) \cup (D^k \times ( A \cup \{x\} ) ) \rightarrow S^m.$$
Since $(S^{k-1} \times U) \cup (D^k \times (A \cup \{x\}))$ is a closed subset of
the paracompact space $U \times D^k$, and the sphere $S^m$ is an absolute
neighborhood retract, the map $g'_0$ extends continuously to a map $g'': W \rightarrow S^m$,
where $W$ is an open neighborhood of $(S^{k-1} \times U) \cup (D^k \times (A \cup \{x\}))$
in $U \times D^k$. The compactness of $D^k$ implies that $W$ contains
$D^k \times U_x$, where $U_x \subseteq U$ is an open neighborhood of $x$.
Shrinking $U_x$ if necessary, we may suppose that $U_x$ belongs to $\calB$;
these open sets $U_x$ form an open cover of $U$, with the required
extension $\Delta^k \rightarrow F(U_x)$ supplied by the map $g'' | D^k \times U_x$.
\end{proof}

\subsection{Heyting Dimension}\label{heyt}

For the purposes of studying paracompact topological spaces, Definition \ref{paradim}
gives a perfectly adequate theory of dimension. However, there are other situations in which
Definition \ref{paradim} is not really appropriate. For example, in algebraic geometry one often considers the Zariski topology on an algebraic variety $X$. This topology is generally not Hausdorff, and is typically of infinite covering dimension. In this setting, there is a better dimension theory: the theory of Krull dimension. In this section, we will introduce a mild generalization of the theory of Krull dimension, which we will call the {\it Heyting dimension} of a topological space $X$. We will then study the relationship between the Heyting dimension of $X$ and the homotopy dimension of the associated $\infty$-topos $\Shv(X)$.

Recall that a topological space $X$ is said to be {\it Noetherian}\index{gen}{topological space!Noetherian} if the collection of closed subsets of $X$ satisfies the descending chain condition. A closed subset $K \subseteq X$ is said to be
{\it irreducible}\index{gen}{irreducible!closed set} if it cannot be written as a finite union of proper closed subsets of $K$ (in particular, the empty set is {\em not} irreducible, since it can be written as an empty union).
The collection of irreducible closed subsets of $X$ forms a well-founded partially ordered set, therefore it has a unique ordinal rank function $\rk$, which may be characterized as follows:

\begin{itemize}
\item If $K$ is an irreducible closed subset of $X$, then $\rk(K)$ is the smallest ordinal which is larger than $\rk(K')$, for all proper irreducible closed subsets $K' \subset K$.
\end{itemize}

We call $\rk(K)$ the {\it Krull dimension} of $K$; the {\it Krull dimension} of $X$ is the supremum of $\rk(K)$, as $K$ ranges over all irreducible closed subsets of $X$.\index{gen}{dimension!Krull}\index{gen}{Krull dimension}

We next introduce a generalization of the Krull dimension to a suitable class of non-Noetherian spaces. We shall say that a topological space $X$ is a {\it
Heyting space} if satisfies the following conditions:\index{gen}{topological space!Heyting}\index{gen}{Heyting!space}

\begin{itemize}
\item[$(1)$] The compact open subsets of $X$ form a basis for the topology of $X$.

\item[$(2)$] A finite intersection of compact open subsets of $X$ is compact (in particular, $X$ is compact).

\item[$(3)$] If $U$ and $V$ are compact open subsets of
$X$, then the interior of $U \cup (X-V)$ is compact. 

\end{itemize}

\begin{remark}\index{gen}{Heyting!algebra}
Recall that a {\it Heyting algebra} is a distributive lattice $L$
with the property that for any $x,y \in L$, there exists a maximal
element $z$ with the property that $x \wedge z \subseteq y$. It
follows immediately from our definition that the lattice of
compact open subsets of a Heyting space forms a Heyting algebra.
Conversely, given any Heyting algebra one may form its spectrum,
which is a Heyting space. This sets up a duality between the
category of {\em sober} Heyting spaces (Heyting spaces in which every irreducible
closed subset has a unique generic point) and the category of Heyting algebras. This
duality is a special case of a more general duality between coherent
topological spaces and distributive lattices. We refer the reader
to \cite{johnstone} for further details.
\end{remark}

\begin{remark}
Suppose that $X$ is a Noetherian topological space. Then $X$ is
a Heyting space, since every open subset of $X$ is compact.
\end{remark}

\begin{remark}
If $X$ is a Heyting space and $U \subseteq X$ is a compact open
subset, then $X$ and $X-U$ are also Heyting spaces. In this case,
we say that $X-U$ is a {\it cocompact} closed subset of $X$.
\end{remark}

We next define the dimension of a Heyting space. The definition is
recursive. Let $\alpha$ be an ordinal. A Heyting space $X$ has
{\it Heyting dimension $\leq \alpha$} if and only if, for any
compact open subset $U \subseteq X$, the boundary of $U$ has
Heyting dimension $< \alpha$ (we note that the boundary of $U$ is
also a Heyting space); a Heyting space has {\it Heyting dimension} $< 0$ if and
only if it is empty.\index{gen}{Heyting!dimension}\index{gen}{dimension!Heyting}

\begin{remark}
A Heyting space has dimension $\leq 0$ if and only if it is
Hausdorff. The Heyting spaces of dimension $\leq 0$ are precisely
the compact, totally disconnected Hausdorff spaces. In particular,
they are also paracompact spaces and their Heyting dimension
coincides with their covering dimension.
\end{remark}

\begin{proposition}\label{closs}

\begin{itemize}
\item[$(1)$] Let $X$ be a Heyting space of dimension $\leq \alpha$. Then
for any compact open subset $U \subseteq X$, both $U$ and $X-U$
have Heyting dimension $\leq \alpha$.

\item[$(2)$] Let $X$ be a Heyting space which is a union of finitely many
compact open subsets $U_{\alpha}$ of dimension $\leq \alpha$. Then
$X$ has dimension $\leq \alpha$.

\item[$(3)$] Let $X$ be a Heyting space which is a union of finitely many
cocompact closed subsets $K_{\alpha}$ of Heyting dimension $\leq
\alpha$. Then $X$ has Heyting dimension $\leq \alpha$.
\end{itemize}
\end{proposition}

\begin{proof}
All three assertions are proven by induction on $\alpha$. The
first two are easy, so we restrict our attention to $(3)$. Let $U$
be a compact open subset of $X$, having boundary $B$. Then $U \cap
K_{\alpha}$ is a compact open subset of $K_{\alpha}$, so that the
boundary $B_{\alpha}$ of $U \cap K_{\alpha}$ in $K_{\alpha}$ has
dimension $\leq \alpha$. We see immediately that $B_{\alpha}
\subseteq B \cap K_{\alpha}$, so that $\bigcup B_{\alpha}
\subseteq B$. Conversely, if $b \notin \bigcup B_{\alpha}$ then,
for every $\beta$ such that $b \in K_{\beta}$, there exists a
neighborhood $V_{\beta}$ containing $b$ such that $V_{\beta} \cap
K_{\beta} \cap U = \emptyset$. Let $V$ be the intersection of the
$V_{\beta}$, and let $W = V - \bigcup_{b \notin K_{\gamma}}
K_{\gamma}$. Then by construction, $b \in W$ and $W \cap U =
\emptyset$, so that $b \in B$. Consequently, $B = \bigcup
B_{\alpha}$. Each $B_{\alpha}$ is closed in $K_{\alpha}$, thus in
$X$ and also in $B$. The hypothesis implies that $B_{\alpha}$ has
dimension $< \alpha$. Thus the inductive hypothesis guarantees
that $B$ has dimension $< \alpha$, as desired.
\end{proof}

\begin{remark}\label{DVR}
It is not necessarily true that a Heyting space which is a union
of finitely many {\em locally closed} subsets of dimension $\leq
\alpha$ is also of dimension $\leq \alpha$. For example, a
topological space with $2$ points and a nondiscrete, nontrivial
topology has Heyting dimension $1$, but is a union of two locally
closed subsets of Heyting dimension $0$.
\end{remark}

\begin{proposition}\label{krullheyt}
If $X$ is a Noetherian topological space, then the Krull
dimension of $X$ coincides with the Heyting dimension of $X$.
\end{proposition}

\begin{proof}
We first prove, by induction on $\alpha$, that if the Krull
dimension of a Noetherian space $X$ is $\leq \alpha$, then the
Heyting dimension of $X$ is $\leq \alpha$. Since $X$ is Noetherian,
it is a union of finitely many closed irreducible subspaces, each of which automatically has Krull
dimension $\leq \alpha$. Using Proposition \ref{closs}, we may
reduce to the case where $X$ is irreducible. Consider any open subset $U \subseteq X$, and let $Y$ be its boundary. We must show that $Y$ has Heyting dimension $\leq
\alpha$. Using Proposition \ref{closs} again, it suffices to prove
this for each irreducible component of $Y$. Now we simply apply
the inductive hypothesis and the definition of the Krull
dimension.

For the reverse inequality, we again use induction on $\alpha$.
Assume that $X$ has Heyting dimension $\leq \alpha$. To show that
$X$ has Krull dimension $\leq \alpha$, we must show that every
irreducible closed subset of $X$ has Krull dimension $\leq
\alpha$. Without loss of generality we may assume that $X$ is
irreducible. Now, to show that $X$ has Krull dimension $\leq
\alpha$, it will suffice to show that any {\em proper} closed
subset $K \subseteq X$ has Krull dimension $< \alpha$. By the
inductive hypothesis, it will suffice to show that $K$ has Heyting
dimension $< \alpha$. By the definition of the Heyting dimension,
it will suffice to show that $K$ is the boundary of $X - K$. In
other words, we must show that $X - K$ is dense in $X$. This
follows immediately from the irreducibility of $X$.
\end{proof}

We now prepare the way for our vanishing theorem. First, we
introduce a modified notion of connectivity:

\begin{definition}\label{strongcon}\index{gen}{connective!strongly}\index{gen}{strongly $k$-connective}
Let $X$ be a Heyting space and $k$ any integer. Let $\calF \in \Shv(V)$ 
be a sheaf of spaces
on a compact open subset $V \subseteq X$. We will say that $\calF$ is {\it strongly $k$-connective}
if the following condition is satisfied: for every compact open subset $U \subseteq V$
and every map $\zeta: \bd \Delta^m \rightarrow \calF(U)$, there exists a cocompact
closed subset $K \subseteq U$ such that $\overline{K} \subseteq X$ has Heyting dimension
$< m-k$, an open cover $\{ U_{\alpha} \}$ of $U-K$, and a collection of commutative diagrams
$$ \xymatrix{ \bd \Delta^m \ar[r]^-{\zeta} \ar@{^{(}->}[d] & \calF(U) \ar[d] \\
\Delta^m \ar[r]^-{\eta_{\alpha}} \ar[r] & \calF( U_{\alpha} ). } $$
\end{definition}

\begin{remark}
There is a slight risk of confusion with the terminology of Definition \ref{strongcon}.
The condition that a sheaf $\calF$ on $V \subseteq X$ be strongly $k$-connective
depends not only on $V$ and $\calF$, but also on $X$: this is because the Heyting dimension of 
a cocompact closed subset $K \subseteq U$ can increase when we take its closure $\overline{K}$ in $X$.
\end{remark}

\begin{remark}
Strong $k$-connectivity is an unstable analogue of the
connectivity conditions on complexes of sheaves, associated to the dual of the standard perversity. For a discussion of perverse sheaves in the
abelian context we refer the reader to \cite{deligne}.
\end{remark}

\begin{remark}
It is clear from the definition that a strongly $k$-connective sheaf $\calF$ on $V \subseteq X$
is $k$-connective. Conversely, suppose that $X$ has Heyting dimension $\leq n$ and that
$\calF$ is $k$-connective, then $\calF$ is strongly $(k-n)$-connected (if $\bd \Delta^{m} \rightarrow \calF(U)$ is any map, then we may take $K=U$ for $m > n$ and $K = \emptyset$ for $m \leq n$).
\end{remark}

The strong $k$-connectivity of a sheaf $\calF$ is, by definition, a
local property. The key to our vanishing result is that this is
equivalent to an apparently stronger {\it global} property.

\begin{lemma}\label{precheesit}
Let $X$ be a Heyting space, $V$ a compact open subset of $X$, and
$\calF: \calU(V)^{op} \rightarrow \Kan$ a strongly $k$-connective sheaf on $V$.
Let $A \subseteq B$ be an inclusion of finite simplicial sets of dimension $\leq m$, 
let $U \subseteq V$, and let $\zeta: A \rightarrow \calF(U)$ be a map of simplicial sets.

There exists a cocompact closed subset $K \subseteq U$ whose closure
$\overline{K} \subseteq X$ has Heyting dimension $< m-1-k$, an open covering
$\{ U_{\alpha} \} $ of $U - K$, and a collection of commutative diagrams
$$ \xymatrix{ A \ar@{^{(}->}[d] \ar[r]^-{\zeta} & \calF(U) \ar[d] \\
B \ar[r]^-{\eta_{\alpha}} & \calF( U_{\alpha} ). }$$
\end{lemma}

\begin{proof}
Induct on the number of simplices of $B$ which do not belong to $A$, and invoke Definition \ref{strongcon}.
\end{proof}

\begin{lemma}\label{cheezit}
Let $X$ be a Heyting space, $V$ a compact open subset of $X$, let 
$\calF: \calU(V)^{op} \rightarrow \Kan$ be a sheaf on $X$, let
$\eta: \bd \Delta^m \rightarrow \calF(V)$ be a map, and form a pullback square
$$ \xymatrix{ \calF' \ar[r] \ar[d] & \calF^{\Delta^m} \ar[d] \\
\ast \ar[r]^-{\eta} & \calF^{\bd \Delta^m}. }$$
Suppose that $\calF$ is strongly $k$-connective. Then $\calF'$ is strongly $(k-m)$-connective.
\end{lemma}

\begin{proof}
Unwinding the definitions, we must show that for every compact $U \subset V$ and every
map 
$$\zeta: ( \bd \Delta^m \times \Delta^n ) \coprod_{ \bd \Delta^m \times \bd \Delta^n }
( \Delta^m \times \bd \Delta^n ) \rightarrow \calF(U)$$
whose restriction $\zeta | \bd \Delta^m \times \Delta^n$ is given by $\eta$, there
exists a cocompact closed subset $K \subseteq U$ such that $\overline{K} \subseteq X$ has Heyting dimension $< n+m -k$, an open covering $\{ U_{\alpha} \}$ of $U - K$, and a collection of maps
$$ \zeta_{\alpha}: \Delta^m \times \Delta^n \rightarrow \calF( U_{\alpha})$$ which extend $\zeta$. 
This follows immediately from Lemma \ref{precheesit}.
\end{proof}

\begin{theorem}\label{vanishing}
Let $X$ be a Heyting space of dimension $\leq n$, let $W \subseteq
X$ be a compact open set, and let $\calF \in \Shv(W)$. The
following conditions are equivalent:

\begin{itemize}

\item[$(1)$] For any compact open sets $U \subseteq V \subseteq W$
and any commutative diagram
$$ \xymatrix{ \bd \Delta^{m} \ar[r]^-{\zeta} \ar@{^{(}->}[d] & \calF(V) \ar[d] \\
\Delta^m \ar[r]^-{\eta} & \calF(U), }$$
there exists a cocompact closed subset $K \subseteq V-U$
such that $\overline{K} \subseteq X$ has dimension $< m-k$
and a commutative diagram
$$ \xymatrix{ \bd \Delta^{m} \ar[r]^-{\zeta} \ar@{^{(}->}[d] & \calF(V) \ar[d] \\
\Delta^m \ar[r]^-{\eta'} & \calF(V-K), }$$
such that the composition
$ \Delta^m \stackrel{\eta'}{\rightarrow} \calF(V-K) \rightarrow \calF(U)$
is homotopic to $\eta$ relative to $\bd \Delta^m$.

\item[$(2)$] For any compact open sets $V \subseteq W$ and
any map $\zeta: \bd \Delta^m \rightarrow \calF(V)$, there
exists a commutative diagram
$$ \xymatrix{ \bd \Delta^{m} \ar[r]^-{\zeta} \ar@{^{(}->}[d] & \calF(V) \ar[d] \\
\Delta^m \ar[r]^-{\eta'} & \calF(V-K), }$$
where $K \subseteq V$ is a cocompact closed subset
and $\overline{K} \subseteq X$ has dimension $< m-k$.

\item[$(3)$] The sheaf $\calF$ is strongly $k$-connective.
\end{itemize}
\end{theorem}

\begin{proof}
It is clear that $(1)$ implies $(2)$ (take $U$ to be empty) and
that $(2)$ implies $(3)$ (by definition). We must show that $(3)$
implies $(1)$. So let $\calF$ be a strongly $k$-connective sheaf on $W$
and $$ \xymatrix{ \bd \Delta^{m} \ar[r]^-{\zeta} \ar@{^{(}->}[d]  & \calF(V) \ar[d] \\
\Delta^m \ar[r]^-{\eta} & \calF(U) }$$
a commutative diagram as above. Without loss of generality, we may replace $W$ by $V$ and $\calF$ by $\calF|V$.

We may identify $\calF$ with a functor from $\calU(V)^{op}$ into the category $\Kan$ of Kan complexes. Form a pullback square 
$$ \xymatrix{ \calF' \ar[r] \ar[d] & \calF^{\Delta^m} \ar[d] \\
\ast \ar[r]^-{\zeta} & \calF^{\bd \Delta^m} }$$
in $\Set_{\Delta}^{ \calU(V)^{op} }$. The right vertical map is a projective fibration, so that
the diagram is homotopy Cartesian (with respect to the projective model structure).
It follows that $\calF'$ is also a sheaf on $V$, which is strongly $(k-m)$-connective by Lemma \ref{cheezit}. Replacing $\calF$ by $\calF'$, we may reduce to the case $m=0$.

The proof now proceeds by induction on $k$. For our base case, we take
$k=-n-1$, so that there is no connectivity assumption on the stack
$\calF$. We are then free to choose $K = V-U$ (it is clear that
$\overline{K}$ has dimension $\leq n$).

Now suppose that the theorem is known for strongly
$(k-1)$-connective stacks on any compact open subset of $X$; we
must show that for any strongly $k$-connective $\calF$ on $V$ and
any $\eta \in \calF(U)$, there exists a closed subset
$K \subseteq V-U$ such that $\overline{K} \subseteq X$ has Heyting dimension
$ < -k$, and a point $\eta' \in \calF(V-K)$ whose restriction to $U$ lies in
the same component of $\calF(U)$ as $\eta$.

Since $\calF$ is strongly $k$-connective, we deduce that there
exists an open cover $\{V_{\alpha} \}$ of some open subset
$V-K_0$, where $K_0$ has dimension $< -k$ in $X$, together with
points $\psi_{\alpha} \in \calF( V_{\alpha})$. Adjoining
the open set $U$ and the point $\eta$ if necessary, we may suppose
that $K_0 \cap U = \emptyset$. Replacing $V$ by
$V- K_0$ we may reduce to the case $K_0 = \emptyset$.

Since $V$ is compact, we may assume that there exist only finitely
many indices $\alpha$. Proceeding by induction on the number of
indices, we may reduce to the case where $V = U \cup V_{\alpha}$ for some $\alpha$. 
Let $\eta'$ and $\psi'$ denote the images of $\eta$ and $\psi$ in $U \cap V_{\alpha}$,
and form a pullback
diagram $$ \xymatrix{ \calF' \ar[rr] \ar[d] & & (\calF|(U \cap V_{\alpha}))^{\Delta^1} \ar[d] \\
\ast \ar[rr]^-{(\eta', \psi')} & & (\calF|(U \cap V_{\alpha}))^{\bd \Delta^1}.}$$
Again, this diagram is a homotopy pullback, so that $\calF'$ is a sheaf on
$U \cap V_{\alpha}$ which is strongly $(k-1)$-connective by Lemma \ref{cheezit}.
According to the inductive hypothesis, there exists a closed subset 
$K \subset U \cap V_{\alpha}$ such that $\overline{K} \subseteq X$ has dimension
$< -k+1$, such that the images of $\psi_{\alpha}$ and $\eta$ belong to the same
component of $\calF( (U \cap V_{\alpha}) - K )$. Replacing $V_{\alpha}$ by
Since $\overline{K}$ has dimension $< -k+1$ in $X$, the
boundary $\bd K$ of $K$ has codimension $< -k$ in $X$. Let $V' =
V_{\alpha} - (V_{\alpha} \cap \overline{K})$. Since $\calF$ is a sheaf, we have a 
homotopy pullback diagram
$$ \xymatrix{ \calF( V' \cup U) \ar[r] \ar[d] & \calF(U) \ar[d] \\
\calF(V') \ar[r] & \calF( V' \cap U ). }$$
We observe that there is a path joining the images of $\eta$ and
$\psi_{\alpha}$ in $\calF(V' \cap U) = \calF( (U \cap V_{\alpha}) - K)$, so that there
is a vertex $\widetilde{\eta} \in \calF(V' \cup U)$ whose image in $\calF(U)$ lies
in the same component as $\eta$. We now observe that $V' \cup U = V - (V \cap \bd K)$,
and that $\overline{ V \cap \bd K }$ is contained in $\bd K$ and therefore has Heyting dimension
$\leq -k$.
\end{proof}

\begin{corollary}\label{hphp}
Let $\pi: X \rightarrow Y$ be a continuous map between Heyting
spaces of finite dimension. Suppose that $\pi$ has the property
that for any cocompact closed subset $K \subseteq X$ of dimension
$\leq n$, $\pi(K)$ is contained in a cocompact closed subset of
dimension $\leq n$. Then the functor $\pi_{\ast}: \Shv(X) \rightarrow \Shv(Y)$ 
carries strongly $k$-connective sheaves on $X$ to strongly $k$-connective sheaves on $Y$.
\end{corollary}

\begin{proof}
This is clear from the characterization $(2)$ of Theorem
\ref{vanishing}.
\end{proof}

\begin{corollary}\label{gra}
Let $X$ be a Heyting space of finite Heyting dimension, and let $\calF$ be
a strongly $k$-connective sheaf on $X$. Then $\calF(X)$ is
$k$-connective.
\end{corollary}

\begin{proof}
Apply Corollary \ref{hphp} in the case where $Y$ is a point.
\end{proof}

\begin{corollary}\label{turkant}\index{gen}{Grothendieck's vanishing theorem!nonabelian version}
Let $X$ be a Heyting space of Heyting dimension $\leq n$, and let $\calF$ be an
$n$-connective sheaf on $X$. Then for any compact open $U \subseteq X$, the
map $\pi_0 \calF(X) \rightarrow \pi_0 \calF(U)$ is surjective. In particular, $\Shv(X)$ has homotopy dimension $\leq n$.
\end{corollary}

\begin{proof}
Suppose first that $(1)$ is satisfied. Let $\calF$ be an $n$-connective sheaf on $X$.
Then $\calF$ is strongly $0$-connective; by characterization $(2)$ of Theorem \ref{vanishing}, we deduce that $\calF(X) \rightarrow \calF(U)$ is surjective. The last claim follows by taking
$U = \emptyset$.
\end{proof}

\begin{remark}
Let $X$ be a Heyting space of Heyting dimension $\leq n$. Then any compact open subset
of $X$ also has Heyting dimension $\leq n$. It follows that $\Shv(X)$ is locally of homotopy dimension $\leq n$, and therefore hypercomplete by Corollary \ref{fdfd}.
\end{remark}

\begin{remark}
It is not necessarily true that a Heyting space $X$ such that $\Shv(X)$ has homotopy dimension $\leq n$ is itself of Heyting dimension $\leq n$. For example, if $X$ is the Zariski spectrum of a discrete valuation ring (that is, a two point space with a nontrivial topology), then $X$ has homotopy dimension zero (see Example \ref{honeypie}).
\end{remark}

In particular, we obtain Grothendieck's vanishing theorem (see \cite{tohoku} for the original, quite different proof):

\begin{corollary}\index{gen}{Grothendieck's vanishing theorem}
Let $X$ be a Noetherian topological space of Krull dimension $\leq
n$. Then $X$ has cohomological dimension $\leq n$.
\end{corollary}

\begin{proof}
Combine Proposition \ref{krullheyt}, Corollary \ref{turkant}, and Corollary \ref{confusion}.
\end{proof}

\begin{example}
Let $V$ be a real algebraic variety (defined over the real
numbers, say). Then the lattice of open subsets of $V$ that can be
defined by polynomial equations and inequalities is a Heyting
algebra, and the spectrum of this Heyting algebra is a Heyting
space $X$ having dimension at most equal to the dimension of $V$.
The results of this section therefore apply to $X$.

More generally, let $T$ be an o-minimal theory (see for example
\cite{lou}), and let $S_n$ denote the set of complete $n$-types of
$T$. We endow $S_n$ with the topology generated by the
sets $U_{\phi} = \{p: \phi \in p\}$, where $\phi$ ranges over
formula with $n$ free variables such that the openness of the set of points satisfying
$\phi$ is provable in $T$. Then $S_n$ is a Heyting
space of Heyting dimension $\leq n$.
\end{example}

\begin{remark}
The methods of this section can be adapted to slightly more
general situations, such as the Nisnevich topology on a Noetherian
scheme of finite Krull dimension. It follows that the
$\infty$-topoi associated to such sites have (locally) finite homotopy
dimension and are therefore hypercomplete.
We will discuss this matter in more detail in \cite{DAG}.
\end{remark}