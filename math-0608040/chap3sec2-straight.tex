\section{Straightening and Unstraightening}\label{strsec}

\setcounter{theorem}{0}

Let $\calC$ be a category, and let $\chi: \calC^{op} \rightarrow \Cat$
be a functor from $\calC$ to the category $\Cat$ of small categories.
To this data, we can associate (by means of the {\it Grothendieck construction} discussed in \S \ref{scgp}) a new category $\widetilde{\calC}$ which may be described as follows:
\begin{itemize}
\item The objects of $\widetilde{\calC}$ are pairs $(C, \eta)$ where
$C \in \calC$ and $\eta \in \chi(C)$.
\item Given a pair of objects $(C, \eta), (C', \eta') \in \widetilde{\calC})$, a morphism
from $(C, \eta)$ to $(C', \eta')$ in $\widetilde{\calC}$ is a pair $(f, \alpha)$, where
$f: C \rightarrow C'$ is a morphism in the category $\calC$ and $\alpha: \eta \rightarrow \chi(f)(\eta')$ is a morphism in the category $\chi(C)$.
\item Composition is defined in the obvious way.
\end{itemize}
This construction establishes an equivalence between
$\Cat$-valued functors on $\calC^{op}$ and categories which are {\it fibered over
$\calC$}. (To formulate the equivalence precisely, it is best to view $\Cat$ as
a {\it bicategory}, but we will not dwell on this technical point here.)

The goal of this section is to establish an $\infty$-categorical version of the equivalence described above. We will replace the category $\calC$ by a simplicial set $S$, the category $\Cat$ by the $\infty$-category
$\Cat_{\infty}$, and the notion of ``fibered category'' with the notion of
``Cartesian fibration''. In this setting, we will obtain an equivalence of $\infty$-categories, which arises from a Quillen equivalence of simplicial model categories. On one side, we have the category
$(\mSet)_{/S}$, equipped with the Cartesian model structure (a simplicial model category whose fibrant objects are precisely the Cartesian fibrations $X \rightarrow S$; see \S \ref{markprop}). On the other, we have the category of simplicial functors
$$ \sCoNerve[S]^{op} \rightarrow \mSet,$$
equipped with the projective model structure (see \S \ref{quasilimit3}), whose underlying $\infty$-category is equivalent to $\Fun(S^{op}, \Cat_{\infty})$ (Proposition \ref{gumby444}).
The situation may be summarized as follows:

\begin{theorem}\label{straightthm}
Let $S$ be a simplicial set, $\calC$ a simplicial category, and $\phi: \sCoNerve[S] \rightarrow \calC^{op}$ a functor between simplicial categories. Then there exists a pair of adjoint functors
$$ \Adjoint{ \St^{+}_{\phi}}{(\mSet)_{/S}}{(\mSet)^{\calC}}{\Un^{+}_{\phi}} $$
with the following properties:
\begin{itemize}
\item[$(1)$] The functors $(\St^{+}_{\phi}, \Un^{+}_{\phi})$ determine a Quillen adjunction between
$(\mSet)_{/S}$ $($with the Cartesian model structure$)$ and $(\mSet)^{\calC}$ $($with the projective model structure$)$. 

\item[$(2)$] If $\phi$ is an equivalence of simplicial categories, then $(\St^{+}_{\phi}, \Un^{+}_{\phi})$ is a
Quillen equivalence.
\end{itemize}
\end{theorem}

We will refer to $\St^{+}_{\phi}$ and $\Un^{+}_{\phi}$ as the {\it straightening} and {\it unstraightening} functors, respectively. We will give a construct these functors in \S \ref{markmodel2}, and establish part $(1)$ of Theorem \ref{straightthm}. Part $(2)$ is more difficult and requires some preliminary work; we will begin in \S \ref{funkystructure} by analyzing the structure of Cartesian fibrations $X \rightarrow \Delta^n$.
We will apply these analyses in \S \ref{markmodel24} to complete the proof of Theorem \ref{straightthm} in the case where $S$ is a simplex. In \S \ref{markmodel25}, we will deduce the general result, using formal arguments to reduce to the special case of a simplex. 

In the case where $\calC$ is an ordinary category, the straightening and unstraightening procedures
of \S \ref{markmodel2} can be substantially simplified. We will discuss the situation in
\S \ref{altstr}, where we provide an analogue of Theorem \ref{straightthm} (see Propositions \ref{kudd} and \ref{sulken}).

\subsection{The Straightening Functor}\label{markmodel2}

Let $S$ be a simplicial set, and let $\phi: \sCoNerve[S] \rightarrow \calC^{op}$ be a functor between simplicial categories, which we regard as fixed throughout this section. Our objective is
to define the {\it straightening functor} $\St_{\phi}^{+}: (\mSet)_{/S} \rightarrow (\mSet)^{\calC}$ and its right adjoint $\Un_{\phi}^{+}$. 
The intuition is that an object $X$ of $(\mSet)_{/S}$ associates $\infty$-categories to vertices of $S$ in a homotopy coherent fashion, and the functor $\St^{+}_{\phi}$ ``straightens'' this diagram to obtain an $\infty$-category valued functor on $\calC$. The right adjoint $\Un^{+}_{\phi}$ should be viewed as a forgetful functor, which
takes a strictly commutative diagram and retains the underlying homotopy coherent diagram.

The functors $\St^{+}_{\phi}$ and $\Un^{+}_{\phi}$ are more elaborate versions of the straightening and unstraightening functors introduced in \S \ref{rightstraight}. We begin by recalling the unmarked version of the construction. For each object $X \in (\sSet)_{/S}$, form a pushout diagram of simplicial categories
$$ \xymatrix{ \sCoNerve[X] \ar[r] \ar[d]^{\phi} & \sCoNerve[X^{\triangleright}] \ar[d] \\
\calC^{op} \ar[r] & \calC^{op}_X }$$
where the left vertical map is given by composing $\phi$ with the map
$\sCoNerve[X] \rightarrow \sCoNerve[S]$. The functor
$\St_{\phi} X: \calC \rightarrow \sSet$ is defined by the formula
$$ (\St_{\phi}X)(C) = \bHom_{ \calC^{op}_X }(C, \ast)$$
where $\ast$ denotes the cone point of $X^{\triangleright}$.

We will define $\St^{+}_{\phi}$ by designating certain marked edges on the simplicial sets
$(\St_{\phi}X)(C)$, which depend in a natural way on the marked edges  of $X$.
In order to describe this dependence, we need to introduce a bit of notation.

\begin{notation}
Let $X$ be an object of $(\sSet)_{/S}$. Given an $n$-simplex $\sigma$ of the simplicial set
$\bHom_{\calC^{op}}(C,D)$, we let $\sigma^{\ast}: (\St_{\phi}X)(D)_n \rightarrow
(\St_{\phi}X)(C)_n$ denote the associated map on $n$-simplices.

Let $c$ be a vertex of $X$, and $C = \phi(c) \in \calC$.
We may identify $c$ with a map $c: \Delta^0 \rightarrow X$.
Then $c \star \id_{\Delta^0}: \Delta^1 \rightarrow X^{\triangleright}$ is an edge of
$X^{\triangleright}$, which determines a morphism $C \rightarrow \ast$ in
$\calC^{op}_X$, which we may identify with a vertex $\widetilde{c} \in (\St_{\phi}X)(C)$.

Similarly, suppose that $f: c \rightarrow d$ is an edge of $X$, corresponding to a morphism
$$ C \stackrel{F}{\rightarrow} D$$ in the simplicial category $\calC^{op}$.
We may identify $f$ with a map $f: \Delta^1\rightarrow X$. Then $f \star \id_{\Delta^1}: \Delta^2 \rightarrow X^{\triangleright}$ determines a map $\sCoNerve[\Delta^2] \rightarrow \calC_X$, which we may identify with a diagram (not strictly commutative)
$$ \xymatrix{ C \ar[rr]^{F} \ar[dr]^{\widetilde{c}} & & D \ar[dl]^{\widetilde{d}} \\
& \ast & }$$
together with an edge $$\widetilde{f}: \widetilde{c} \rightarrow \widetilde{d} \circ F = F^{\ast} 
\widetilde{d}$$
in the simplicial set $\bHom_{\calC^{op}_X}(C, \ast) = (\St_{\phi}X)(C)$.
\end{notation}

\begin{definition}\index{gen}{straightening functor}\index{not}{Stphi+@$\St_{\phi}^{+}$}
Let $S$ be a simplicial set, $\calC$ a simplicial category, and $\phi: \sCoNerve[S] \rightarrow \calC^{op}$ a simplicial functor. Let $(X, \calE)$ be an object of $(\mSet)_{/S}$. Then
$$ \St^{+}_{\phi}(X, \calE): \calC \rightarrow \mSet$$
is defined by the formula
$$ \St^{+}_{\phi}(X, \calE)(C) = ((\St_{\phi}X)(C), \calE_{\phi}(C))$$
where $\calE_{\phi}(C)$ is the set of all edges of $(\St_{\phi}X)(C)$ having the form
$$ G^{\ast} \widetilde{f},$$ where $f: d \rightarrow e$ is a marked edge of $X$, giving rise to an
edge $\widetilde{f}: \widetilde{d} \rightarrow F^{\ast} \widetilde{e}$ in $(\St_{\phi}X)(D)$, and $G$ belongs to $\bHom_{\calC^{op}}(C,D)_1$.
\end{definition}

\begin{remark}
The construction
$$ (X, \calE) \mapsto \St^{+}_{\phi}(X, \calE) = (\St_{\phi}X, \calE_{\phi})$$
is obviously functorial in $X$. Note that we may characterize the subsets
$\{ \calE_{\phi}(C) \subseteq (\St_{\phi} X)(C)_1 \}$ as the smallest collection of sets which contain
$\widetilde{f}$, for every $f \in \calE$, and depend functorially on $C$.
\end{remark}

The following formal properties of the straightening functor follow immediately from the definition:

\begin{proposition}\label{formall}

\begin{itemize}
\item[$(1)$] Let $S$ be a simplicial set, $\calC$ a simplicial category, and $\phi: \sCoNerve[S] \rightarrow
\calC^{op}$ a simplicial functor; then the associated straightening functor
$$\St^{+}_{\phi}: (\mSet)_{/S} \rightarrow (\mSet)^{\calC}$$ preserves colimits.

\item[$(2)$] Let $p: S' \rightarrow S$ be a map of simplicial sets, $\calC$ a simplicial category, and
$\phi: \sCoNerve[S] \rightarrow \calC^{op}$ a simplicial functor, and let $\phi': \sCoNerve[S'] \rightarrow \calC^{op}$ denote the composition $\phi \circ \sCoNerve[p]$. 
Let $p_{!}: (\mSet)_{/S'} \rightarrow (\mSet)_{/S}$ denote the forgetful functor, given by composition with $p$. There is a natural isomorphism of functors
$$ \St^{+}_{\phi} \circ p_{!} \simeq \St^{+}_{\phi'}$$
from $(\mSet)_{/S'}$ to $(\mSet)^{\calC}$.

\item[$(3)$] Let $S$ be a simplicial set, $\pi: \calC \rightarrow \calC'$ a simplicial functor between simplicial categories, and $\phi: \sCoNerve[S] \rightarrow
\calC^{op}$ a simplicial functor. Then there is a natural isomorphism of functors $$\St^{+}_{\pi \circ \phi} \simeq \pi_{!} \circ \St^{+}_{\phi}$$
from $(\mSet)_{/S}$ to $(\mSet)^{\calC'}$. Here $\pi_{!}: (\mSet)^{\calC} \rightarrow (\mSet)^{\calC'}$
is the left adjoint to the functor $\pi^{\ast}: (\mSet)^{\calC'} \rightarrow (\mSet)^{\calC}$ given by composition with $\pi$: see \S \ref{quasilimit3}.
\end{itemize}
\end{proposition}

\begin{corollary}\label{giraf}\index{gen}{unstraightening functor}\index{not}{Unphi+@$\Un^{+}_{\phi}$}
Let $S$ be a simplicial set, $\calC$ a simplicial category, and $\phi: \sCoNerve[S] \rightarrow \calC^{op}$ any simplicial functor. The straightening functor $\St^{+}_{\phi}$ has a right adjoint 
$$\Un^{+}_{\phi}: (\mSet)^{\calC} \rightarrow (\mSet)_{/S}.$$
\end{corollary}

\begin{proof}
This follows from part $(1)$ of Proposition \ref{formall} and the adjoint functor theorem. (Alternatively, one can construct $\Un^{+}_{\phi}$ directly; we leave details to the reader.)
\end{proof}

\begin{notation}\index{not}{StS+@$\St_{S}^{+}$}\index{not}{UnS+@$\Un_{S}^{+}$}
Let $S$ be a simplicial set, let $\calC = \sCoNerve[S]^{op}$, and let $\phi: \sCoNerve[S] \rightarrow \calC^{op}$ be the identity map. In this case, we will denote $\St^{+}_{\phi}$ by $\St^{+}_{S}$ and $\Un^{+}_{\phi}$ by $\Un^{+}_{S}$. 
\end{notation}

Our next goal is to show that the straightening and unstraightening functors $(\St^{+}_{\phi}, \Un^{+}_{\phi})$ give a {\em Quillen} adjunction between the model categories $(\mSet)_{/S}$ and
$(\mSet)^{\calC}$. The first step is to show that $\St^{+}_{\phi}$ preserves cofibrations.

\begin{proposition}\label{cougherup}
Let $S$ be a simplicial set, $\calC$ a simplicial category, and $\phi: \sCoNerve[S] \rightarrow \calC^{op}$ a simplicial functor. The functor $\St^{+}_{\phi}$ carries cofibrations $($with respect to the Cartesian model structure on $(\mSet)_{/S}${}$)$ to cofibrations $($with respect to the projective model structure on $(\mSet)^{\calC})${}$)$.
\end{proposition}

\begin{proof}
Let $j: A \rightarrow B$ be a cofibration in $(\mSet)_{/S}$; we wish to show that
$\St^{+}_{\phi}(j)$ is a cofibration. By general nonsense, we may suppose that $j$ is a
generating cofibration, either having the form $(\bd \Delta^n)^{\flat} \subseteq (\Delta^n)^{\flat}$
or $(\Delta^1)^{\flat} \rightarrow (\Delta^1)^{\sharp}$. Using Proposition \ref{formall}, we may reduce to the case where $S=B$, $\calC = \sCoNerve[S]$, and $\phi$ is the identity map. The result now follows from a straightforward computation.
\end{proof}

To complete the proof that $(\St^{+}_{\phi}, \Un^{+}_{\phi})$ is a Quillen adjunction, it suffices to show that
$\St^{+}_{\phi}$ preserves trivial cofibrations. Since every object of $(\mSet)_{/S}$ is cofibrant, this is equivalent to the apparently stronger claim that if $f: X \rightarrow Y$ is a Cartesian equivalence
in $(\mSet)_{/S}$, then $\St^{+}_{\phi}(f)$ is a weak equivalence in $(\mSet)^{\calC}$. The main step
is to establish this in the case where $f$ is marked anodyne. First, we need a few lemmas.

\begin{lemma}\label{blurgh}
Let $\calE$ be the set of all degenerate edges of $\Delta^n \times \Delta^1$, together with the edge
$\{n\} \times \Delta^1$. Let $B \subseteq \Delta^n \times \Delta^1$ be the coproduct
$$ ( \Delta^n \times \{1\} ) \coprod_{ \bd \Delta^n \times \{1\} } (\bd \Delta^n \times \Delta^1).$$
Then the map
$$i: ( B, \calE \cap B_1 ) \subseteq ( \Delta^n \times \Delta^1, \calE)$$ is marked anodyne.
\end{lemma}

\begin{proof}
We must show that $i$ has the left lifting property with respect to every map $p: X \rightarrow S$ satisfying the hypotheses of Proposition \ref{dubudu}. This is simply a reformulation of Proposition \ref{goouse}.
\end{proof}

\begin{lemma}\label{blughel}
Let $K$ be a simplicial set, $K' \subseteq K$ a simplicial subset, and $A$ a set of vertices of $K$.
Let $\calE$ denote the set of all degenerate edges of $K \times \Delta^1$, together with the edges
$\{a\} \times \Delta^1$ where $a \in A$. Let $B = (K' \times \Delta^1) \coprod_{ K' \times \{1\} } (K \times \{1\}) \subseteq K \times \Delta^1$. Suppose that, for every nondegenerate simplex $\sigma$ of $K$, either $\sigma$ belongs to $K'$, or the final vertex of $\sigma$ belongs to $A$. Then the inclusion
$$ (B, \calE \cap B_1) \subseteq (K \times \Delta^1, \calE)$$ is marked anodyne.
\end{lemma}

\begin{proof}
Working cell-by-cell, we reduce to Lemma \ref{blurgh}.
\end{proof}

\begin{lemma}\label{brend}
Let $X$ be a simplicial set, and let $\calE \subseteq \calE'$ be sets of edges of $X$ containing all degenerate edges. The following conditions are equivalent:
\begin{itemize}
\item[$(1)$] The inclusion $(X, \calE) \rightarrow (X, \calE')$ is trivial cofibration in $\mSet$ (with respect to the Cartesian model structure). 
\item[$(2)$] For every $\infty$-category $\calC$ and every map $f: X \rightarrow \calC$ which carries each edge of
$\calE$ to an equivalence in $\calC$, $f$ also carries each edge of $\calE'$ to an equivalence in $\calC$.
\end{itemize}
\end{lemma}

\begin{proof}
By definition, $(1)$ holds if and only if for every $\infty$-category $\calC$, the inclusion
$$j: \bHom^{\flat}( (X, \calE'), \calC^{\natural}) \rightarrow \bHom^{\flat}( (X, \calE), \calC^{\natural} )$$ is a categorical equivalence. Condition $(2)$ is the assertion that $j$ is an isomorphism. Thus $(2)$ implies $(1)$. Suppose that $(1)$ is satisfied, and let $f: X \rightarrow \calC$ be a vertex of
$\bHom^{\flat}((X,\calE), \calC^{\natural})$. By hypothesis, there exists an equivalence $f \simeq f'$, where $f'$ belongs to the image of $j$. Let $e \in \calE'$; then $f'(e)$ is an equivalence in $\calC$. Since $f$ and $f'$ are equivalent, $f(e)$ is also an equivalence in $\calC$. Consequently, $f$ also belongs to the image of $j$, and the proof is complete.
\end{proof}

\begin{proposition}\label{spec2}
Let $S$ be a simplicial set, $\calC$ a simplicial category, and $\phi: \sCoNerve[S] \rightarrow \calC^{op}$ a simplicial functor. The functor $\St^{+}_{\phi}$ carries marked anodyne maps in $(\mSet)_{/S}$ (with respect to the Cartesian model structure) to trivial cofibrations in 
$(\mSet)^{\calC}$ (with respect to the projective model structure).
\end{proposition}

\begin{proof}
Let $f: A \rightarrow B$ be a marked anodyne map in $(\mSet)_{/S}$. We wish to prove that
$\St^{+}_{\phi}(f)$ is a trivial cofibration. It will suffice to prove this under the assumption that $f$ is one of the generators for the class of marked anodyne maps, as given in Definition \ref{markanod}.
Using Proposition \ref{formall}, we may reduce to the case where $S$ is the underyling simplicial set of $B$, $\calC = \sCoNerve[S]^{op}$, and $\phi$ is the identity. There are four cases to consider:

\begin{itemize}
\item[(1)]  Suppose first that $f$ is among the morphisms listed in $(1)$ of Definition \ref{markanod}; that is, $f$ is an inclusion $(\Lambda^n_i)^{\flat} \subseteq (\Delta^n)^{\flat}$, where $0 < i < n$. Let $v_k$ denote the $k$th vertex of $\Delta^n$, which we may also think of as an object of the simplicial category $\calC$. 
We note that $\St^{+}_{\phi}(f)$ is an isomorphism when evaluated at $v_k$ for $k \neq 0$. Let $K$ denote the cube $(\Delta^1)^{ \{ j: 0 < j \leq n, j \neq i \} }$, let $K' = \bd K$, let $A$ denote the set of all vertices of $K$ corresponding to subsets of $\{ j: 0 < j \leq n, j \neq i\}$ which contain an element $> i$, and let
$\calE$ denote the set of all degenerate edges of $K \times \Delta^1$ together with all edges of the form
$\{a\} \times \Delta^1$, where $a \in A$. Finally, let $B = (K \times \{1\} ) \coprod_{ K' \times \{1\} } (K' \times \Delta^1)$. The morphism
$\St^{+}_{\phi}(f)(v_n)$ is a pushout of $g: (B, \calE \cap B_1) \subseteq (K \times \Delta^1, \calE)$. 
Since $i > 0$, we may apply Lemma \ref{blughel} to deduce that $g$ is marked-anodyne, and therefore a trivial cofibration in $\mSet$.

\item[(2)]
Suppose that $f$ is among the morphisms of part $(2)$ in Definition \ref{markanod}; that is, $f$
is an inclusion $$( \Lambda^n_n, \calE \cap (\Lambda^n_n)_{1} ) \subseteq ( \Delta^n, \calF ),$$
where $\calF$ denotes the set of all degenerate edges of $\Delta^n$, together with the final edge $\Delta^{ \{n-1,n\} }$. If $n > 1$, then one can repeat the argument given above in case $(1)$, except that the set of vertices $A$ needs to be replaced by the set of all vertices of $K$ which correspond to subsets of $\{j: 0 < j < n\}$ which contain $n-1$. If $n=1$, then we observe that $\St^{+}_{\phi}(f)(v_n)$ is
isomorphic to the inclusion $\{1\}^{\sharp} \subseteq (\Delta^1)^{\sharp}$, which is again a marked anodyne map and therefore a trivial cofibration in $\mSet$.

\item[(3)]
Suppose next that $f$ is the morphism $$ (\Lambda^2_1)^{\sharp} \coprod_{ (\Lambda^2_1)^{\flat} } (\Delta^2)^{\flat} \rightarrow (\Delta^2)^{\sharp}$$
specified in $(3)$ of Definition \ref{markanod}. Simple computation shows that
$\St^{+}_{\phi}(f)(v_n)$ is an isomorphism for $n \neq 0$, and $\St^{+}_{\phi}(f)(v_0)$ is may be identified with the inclusion $$(\Delta^1 \times \Delta^1, \calE) \subseteq (\Delta^1 \times \Delta^1)^{\sharp},$$ where $\calE$ denotes the set of all degenerate edges of $\Delta^1 \times \Delta^1$ together with
$\Delta^1 \times \{0\}$, $\Delta^1 \times \{1\}$, and $\{1\} \times \Delta^1$. This inclusion may be obtained as a pushout of  $$(\Lambda^2_1)^{\sharp} \coprod_{ (\Lambda^2_1)^{\flat} } (\Delta^2)^{\flat} \rightarrow (\Delta^2)^{\sharp}$$ followed by a pushout of
 $$ (\Lambda^2_2)^{\sharp} \coprod_{ (\Lambda^2_2)^{\flat} } (\Delta^2)^{\flat} \rightarrow (\Delta^2)^{\sharp}.$$ The first of these maps is marked-anodyne by definition; the second is marked anodyne by Corollary \ref{hermes}.

\item[(4)] 
Suppose that $f$ is the morphism $K^{\flat} \rightarrow K^{\sharp}$, where $K$ is a Kan complex, as in $(4)$ of Definition \ref{markanod}. For each vertex $v$ of $K$, let $\St^{+}_{\phi}(K^{\flat})(v)=(X_v, \calE_v)$, so that $\St^{+}_{\phi}(K^{\sharp})= X_v^{\sharp}$. Given a morphism $g \in \bHom_{ \sCoNerve[K]}(v,v')_n$, we let $g^{\ast}: X_v \times \Delta^n \rightarrow X_{v'}$ denote the induced map. We wish to show that the natural map $(X_v, \calE_v) \rightarrow X_{v}^{\sharp}$ is an equivalence in $\mSet$. 
By Lemma \ref{brend}, it suffices to show that for every $\infty$-category $Z$, if $h: X_v \rightarrow Z$
carries each edge belonging to $\calE_{v}$ into an equivalence, then $h$ carries {\em every} edge of $X_v$ to an equivalence. 

We first show that $h$ carries $\widetilde{e}$ to an equivalence, for every edge
$e: v \rightarrow v'$ in $K$. Let $m_{e}: \Delta^1 \rightarrow \bHom_{\calC^{op}}(v,v')$ denote the degenerate edge at the vertex corresponding to $e$.
Since $K$ is a Kan complex, the edge $e: \Delta^1 \rightarrow K$
extends to a $2$-simplex $\sigma: \Delta^2 \rightarrow K$ depicted as follows
$$ \xymatrix{ & v' \ar[dr]^{e'} & \\
v \ar[ur]^{e} \ar[rr]^{\id_{v}} & & v. }$$ 
Let $m_{e'}: \Delta^1 \rightarrow \bHom_{\calC}(v',v)$ denote the degenerate edge corresponding to $e'$. The map $\sigma$ gives rise to a diagram a diagram
$$ \xymatrix{ \widetilde{v} \ar[r]^{ \widetilde{e} } \ar[d]^{\id_{\widetilde{v}}} & e^{\ast} \widetilde{v}' \ar[d]^{ m_e^{\ast} \widetilde{e}' } \\
\widetilde{v} \ar[r] & e^{\ast} (e')^{\ast} \widetilde{v} }$$
in the simplicial set $X_{v}$. Since $h$ carries the left vertical arrow and the bottom horizontal arrow into equivalences, it follows that $h$ carries the composition
$(m_e^{\ast} \widetilde{e'}) \circ \widetilde{e}$ to an equivalence in $Z$; thus
$h(\widetilde{e})$ has a left homotopy inverse. A similar argument shows that $h(\widetilde{e})$ has a right homotopy inverse, so that $h(\widetilde{e})$ is an equivalence.

We observe that every edge of $X_{v}$ has the form
$g^{\ast} \widetilde{e}$, where $g$ is an edge of $\bHom_{\calC^{op}}(v,v')$ and
$e: v' \rightarrow v''$ is an edge of $K$. We wish to show that $h( g^{\ast} \widetilde{e} )$
is an equivalence in $Z$. Above, we have shown that this is true if $v=v'$ and $g$ is the identity.
We now consider the more general case where $g$ is not necessarily the identity, but is a degenerate edge corresponding to some map $v' \rightarrow v$ in $\calC$. Let $h'$ denote the composition
$$ X_{v'} \rightarrow X_{v} \stackrel{h}{\rightarrow} Z.$$
Then $h( g^{\ast} \widetilde{e}) = h'( \widetilde{e} )$ is an equivalence in $Z$ by the argument given above.

Now consider the case where $g: \Delta^1 \rightarrow \bHom_{\calC^{op}}(v,v')$ is nondegenerate.
In this case, there is a simplicial homotopy $G: \Delta^1 \times \Delta^1 \rightarrow \bHom_{\calC}(v,v')$ with $g = G| \Delta^1 \times \{0\}$ and $g' = G | \Delta^1 \times \{1\}$ a degenerate edge of $\bHom_{\calC^{op}}(v,v')$ (for example, we can arrange that $g'$ is
the constant edge at an endpoint of $g$). The map $G$ induces a simplicial homotopy 
$G(e)$ from $g^{\ast} \widetilde{e}$ to $(g')^{\ast} \widetilde{e}$. Moreover, the edges $G(e)| \{0\} \times \Delta^1$ and $G(e)| \{1\} \times \Delta^1$ belong to $\calE_{v}$, and are therefore carried by $h$ into equivalences in $Z$. Since $h$ carries $(g')^{\ast} \widetilde{e}$ into an equivalence of
$Z$, it carries $g^{\ast} \widetilde{e}$ into an equivalence of $Z$, as desired.
\end{itemize}
\end{proof}

We now study the behavior of straightening functors with respect to products.

\begin{notation}
Given two simplicial functors $\calF: \calC \rightarrow \mSet$, $\calF': \calC' \rightarrow \mSet$, we let $\calF \boxtimes \calF': \calC \times \calC' \rightarrow \mSet$ denote the functor described by the formula
$$(\calF \boxtimes \calF')(C,C') = \calF(C) \times \calF'(C').$$
\end{notation}

\begin{proposition}\label{spek3}
Let $S$ and $S'$ be simplicial sets, $\calC$ and $\calC'$ simplicial categories, and $\phi: \sCoNerve[S] \rightarrow \calC^{op}$, $\phi': \sCoNerve[S'] \rightarrow (\calC')^{op}$ simplicial functors; let $\phi \boxtimes \phi'$ denote the induced functor $\sCoNerve[S \times S'] \rightarrow (\calC \times \calC')^{op}$. For every $M \in (\mSet)_{/S}$, $M' \in (\mSet)_{/S'}$, the natural map
$$ s_{M,M'}: \St^{+}_{\phi \boxtimes \phi'}(M \times M') \rightarrow \St^{+}_{\phi}(M) \boxtimes \St^{+}_{\phi'}(M')$$
is a weak equivalence of functors $\calC \times \calC' \rightarrow \mSet$. 
\end{proposition}

\begin{proof}
Since both sides are compatible with the formations of filtered colimits in $M$, we may suppose that $M$ has only finitely many nondegenerate simplices. 
We work by induction on the dimension $n$ of $M$ and the number of $n$-dimensional simplices of $M$. If $M = \emptyset$ there is nothing to prove. If $n \neq 1$, we may choose a nondegenerate simplex of $M$ having maximal dimension and thereby write 
$M = N \coprod_{ (\bd \Delta^n)^{\flat} } (\Delta^n)^{\flat}$. By the inductive hypothesis we may suppose that the result is known
for $N$ and $(\bd \Delta^n)^{\flat}$. The map $s_{M,M'}$ is a pushout of the maps
$s_{N,M'}$ and $s_{ (\Delta^n)^{\flat}, M'}$ over $s_{ (\bd \Delta^n)^{\flat}, M' }$.
Since $\mSet$ is left-proper, this pushout is a homotopy pushout; it therefore suffices to prove the result after replacing $M$ by $N$, $(\bd \Delta^n)^{\flat}$, or $(\Delta^n)^{\flat}$. In the first two cases, the inductive hypothesis implies that $s_{M,M'}$ is an equivalence; we are therefore reduced to the case $M = (\Delta^n)^{\flat}$. If $n=0$, the result is obvious. If $n>2$, we set $$K = \Delta^{ \{0,1\} } \coprod_{ \{1\} } \Delta^{ \{1,2\} } \coprod_{ \{2\} } \ldots \coprod_{ \{n-1\} } \Delta^{ \{n-1, n\} } \subseteq \Delta^n.$$
The inclusion $K \subseteq \Delta^n$ is inner anodyne, so that $K^{\flat} \subseteq M$ is marked-anodyne. By Proposition \ref{spec2}, we deduce that $s_{M,M'}$ is an equivalence if and only if $s_{K^{\flat}, M'}$ is an equivalence, which follows from the inductive hypothesis since $K$ is $1$-dimensional.

We may therefore suppose that $n=1$. Using the above argument, we may reduce to the case where $M$ consists of a single edge, either marked or unmarked. Repeating the above argument with the roles of $M$ and $M'$ interchanged, we may suppose that $M'$ also consists of a single edge. Applying Proposition \ref{formall}, we may reduce to the case where $S = M$, $S' = M'$, $\calC = \sCoNerve[S]^{op}$, and $\calC' = \sCoNerve[S']^{op}$.

Let us denote the vertices of $M$ by $x$ and $y$, and the unique edge joining them by
$e: x \rightarrow y$. Similarly, we let $x'$ and $y'$ denote the vertices of $M'$, and $e': x' \rightarrow y'$ the edge which joins them. We note that the map
$s_{M,M'}$ induces an isomorphism when evaluated on any object of $\calC \times \calC'$
{\em except} $(x,x')$. Moreover, the map
$$s_{M,M'}(x,x'): \St^{+}_{\phi \boxtimes \phi'}(M \times M')(x,x') \rightarrow \St^{+}_{\phi}(M)(x) \times \St^{+}_{\phi'}(M')(x')$$
obtained from $s_{ (\Delta^1)^{\flat}, (\Delta^1)^{\flat}}$ by successive pushouts
along cofibrations of the form $(\Delta^1)^{\flat} \subseteq (\Delta^1)^{\sharp}$. Since $\mSet$ is left proper, we may reduce to the case where $M = M' = (\Delta^1)^{\flat}$. The result now follows from a simple explicit computation.
\end{proof}

We now study the situation in which $S = \Delta^0$, $\calC = \sCoNerve[S]$, and $\phi$ is the identity map. In this case, $\St^{+}_{\phi}$ may be regarded as a functor $T: \mSet \rightarrow \mSet$. The underlying functor of simplicial sets
is familiar: we have
$$ T( X, \calE) = ( | X |_{Q^{\bigdot} } , \calE' ),$$ where $Q$ denotes the cosimplicial object of $\sSet$ considered in \S \ref{twistt}. In that section, we exhibited a natural map $|X|_{Q^{\bigdot}} \rightarrow X$ which we proved to be a weak homotopy equivalence. We now prove a stronger version of that result:

\begin{proposition}\label{spek4}
For any marked simplicial set $M = (X, \calE)$, the natural map
$|X|_{Q^{\bigdot}} \rightarrow X$ induces a Cartesian equivalence
$$ T(M) \rightarrow M.$$
\end{proposition} 

\begin{proof}
As in the proof of Proposition \ref{spek3}, we may reduce to the case where $M$ consists of a simplex of dimension $\leq 1$ (either marked or unmarked). In these cases, the map $T(M) \rightarrow M$ is an isomorphism in $\mSet$.
\end{proof}

\begin{corollary}\label{spek5}
Let $S$ be a simplicial set, $\calC$ a simplicial category, $\phi: \sCoNerve[S] \rightarrow \calC^{op}$ a simplicial
functor, and $X \in (\mSet)_{/S}$ an object. For every $K \in \mSet$, there is a natural equivalence 
$$ \St^{+}_{\phi}(M \times K) \rightarrow \St^{+}_{\phi}(M) \boxtimes K$$
of functors from $\calC$ to $\mSet$.
\end{corollary}

\begin{proof}
Combine the equivalences of Proposition \ref{spek4} (in the case where
$S' = \Delta^0$, $\calC' = \sCoNerve[S']^{op}$, and $\phi'$ is the identity ) and Proposition \ref{spek5}.
\end{proof}

We can now complete the proof that $(\St^{+}_{\phi}, \Un^{+}_{\phi})$ is a Quillen adjunction:

\begin{corollary}\label{spek6}
Let $S$ be a simplicial set, $\calC$ a simplicial category, and $\phi: \sCoNerve[S]^{op} \rightarrow \calC$ a simplicial functor. The straightening functor $\St^{+}_{\phi}$ carries Cartesian equivalences in $(\mSet)_{/S}$ to $($objectwise$)$ Cartesian equivalences in $(\mSet)^{\calC}$.
\end{corollary}

\begin{proof}
Let $f: M \rightarrow N$ be a Cartesian equivalence in $(\mSet)_{/S}$. Choose
a marked anodyne map $M \rightarrow M'$, where $M'$ is fibrant; then choose a marked anodyne map $M' \coprod_M N \rightarrow N'$, with $N'$ fibrant. Since $\St^{+}_{\phi}$ carries marked anodyne maps to equivalences by Proposition \ref{spec2}, it suffices to prove that the induced map $\St^{+}_{\phi}(M') \rightarrow \St^{+}_{\phi}(N')$ is an equivalence. In other words, we may replace $M$ by $M'$ and $N$ by $N'$, thereby reducing to the case where $M$ and $N$ are fibrant.

Since $f$ is an Cartesian equivalence of fibrant objects, it has a homotopy inverse
$g$. We claim that $\St^{+}_{\phi}(g)$ is an inverse to $\St^{+}_{\phi}(f)$ in the homotopy category of $( \mSet )^{\calC}$. We will show that $\St^{+}_{\phi}(f) \circ \St^{+}_{\phi}(g)$ is homotopic to the identity; applying the same argument with the roles of $f$ and $g$ reversed will then establish the desired result.

Since $f \circ g$ is homotopic to the identity, there is a map $h: N \times K^{\sharp} \rightarrow N$, where $K$ is a contractible Kan complex containing vertices $x$ and $y$, such that
$f \circ g = h| N \times \{x\}$ and $\id_N = h|N \times \{y\}$. The map $\St^{+}_{\phi}(h)$ factors as
$$ \St^{+}_{\phi}(N \times K^{\sharp}) \rightarrow \St^{+}_{\phi}(N) \boxtimes K^{\sharp} \rightarrow \St^{+}_{\phi}(N)$$
where the left map is an equivalence by Corollary \ref{spek5} and the right map because $K$ is contractible. Since $\St^{+}_{\phi}(f \circ g)$ and $\St^{+}_{\phi}( \id_N)$ are both sections of $\St^{+}_{\phi}(h)$, they represent the same morphism in the homotopy category of $(\mSet)^{\calC}$.
\end{proof}


\subsection{Cartesian Fibrations over a Simplex}\label{funkystructure}

A map of simplicial sets $p: X \rightarrow S$ is a Cartesian fibration if and only if the pullback map $X \times_S \Delta^n \rightarrow \Delta^n$ is a Cartesian fibration, for each simplex of $S$.
Consequently, we might imagine that Cartesian fibrations $X \rightarrow \Delta^n$ are the ``primitive building blocks'' out of which other Cartesian fibrations are built. The goal of this section is to prove a structure theorem for these building blocks. This result has a number of consequences, and will play a vital role in the proof of Theorem \ref{straightthm}.

Note that $\Delta^n$ is the nerve of the category associated to the linearly ordered set
$$[n] = \{ 0 < 1 < \ldots < n\} .$$
Since a Cartesian fibration $p: X \rightarrow S$ can be thought of as giving a (contravariant) functor from $S$ to $\infty$-categories, it is natural to expect a close relationship between Cartesian fibrations $X \rightarrow \Delta^n$ and composable sequences of maps between $\infty$-categories $$ A^0 \leftarrow A^1 \leftarrow \ldots \leftarrow A^n.$$
In order to establish this relationship, we need to introduce a few definitions.

Suppose given a composable sequence of maps
$$ \phi: A^0 \leftarrow A^1 \leftarrow \ldots \leftarrow A^n$$
of simplicial sets. The {\it mapping simplex} $M(\phi)$ of $\phi$
is defined as follows. If $J$ is a nonempty finite linearly
ordered set with greatest element $j$, then to specify a map
$\Delta^J \rightarrow M(\phi)$ one must specify an
order-preserving map $f: J \rightarrow [n]$ together
with a map $\sigma: \Delta^J \rightarrow A^{f(j)}$. Given an order-preserving map
$p: J \rightarrow J'$ of partially ordered sets containing largest elements $j$ and $j'$,
there is natural map $M(\phi)(\Delta^{J'}) \rightarrow M(\phi)(\Delta^J)$ which carries
$(f,\sigma)$ to $(f \circ p, e \circ \sigma)$, where $e: A^{f(j')} \rightarrow A^{f(p(j))}$ is obtained
from $\phi$ in the obvious way.\index{gen}{mapping simplex}

\begin{remark}\label{megveg}\index{not}{Mphi@$M(\phi)$}
The mapping simplex $M(\phi)$ is equipped with a natural map $p:
M(\phi) \rightarrow \Delta^n$; the fiber of $p$ over the vertex
$j$ is isomorphic to the simplicial set $A^j$.
\end{remark}

\begin{remark}\label{megvegg}
More generally, let $f: [m] \rightarrow [n]$ be an
order-preserving map, inducing a map $\Delta^m \rightarrow
\Delta^n$. Then $M(\phi) \times_{\Delta^n} \Delta^m$ is naturally
isomorphic to $M(\phi')$, where the sequence $\phi'$ is given by
$$ A^{f(0)} \leftarrow \ldots \leftarrow A^{f(m)}.$$
\end{remark}

\begin{notation}\label{conf2}
Let $\phi: A^0 \leftarrow \ldots \leftarrow A^n$ be a composable sequence of maps of simplicial sets. To give an edge $e$ of $M(\phi)$, one must give a pair of integers
$0 \leq i \leq j \leq n$ and an edge $\overline{e} \in A^{j}$. We will say that
$e$ is {\em marked} if $\overline{e}$ is degenerate; let $\calE$ denote the set of all marked edges of $M(\phi)$. Then the pair $(M(\phi), \calE)$ is a marked simplicial set which we will denote by
$M^{\natural}(\phi)$.\index{gen}{mapping simplex!marked}\index{not}{Mnatphi@$M^{\natural}(\phi)$}
\end{notation}

\begin{remark}
There is a potential ambiguity between the terminology of Definition \ref{conf1} and that of
Notation \ref{conf2}. Suppose that 
$ \phi: A^0 \leftarrow \ldots \leftarrow A^n$ is a composable sequence of maps and that
$p: M(\phi) \rightarrow \Delta^n$ is a Cartesian fibration. Then $M(\phi)^{\natural}$ (Definition \ref{conf1})
and $M^{\natural}(\phi)$ (Notation \ref{conf2}) do not generally coincide as marked simplicial sets. We feel that there is little danger of confusion, since it is very rare that $p$ is a Cartesian fibration.
\end{remark}

\begin{remark}\label{funkytok}
The construction of the mapping simplex is functorial, in the sense that a commutative ladder
$$ \xymatrix{ \phi:A^0 \ar[d]^{f_0} & \ldots \ar[l] \ar[d] & A^n \ar[l] \ar[d]^{f_n} \\
\psi:B^0 & \ldots \ar[l] & B^n \ar[l] }$$
induces a map $M(f): M(\phi) \rightarrow M(\psi)$. Moreover, if each $f_i$ is a categorical equivalence, then $f$ is a categorical equivalence (this follows by induction on $n$, using the fact that the Joyal model structure is left proper).
\end{remark}

\begin{definition}
Let $p: X \rightarrow \Delta^n$ be a Cartesian fibration, and let
$$\phi: A^0 \leftarrow \ldots \leftarrow A^n$$ be a composable sequence of maps.\index{gen}{quasi-equivalence}
A map $q: M(\phi) \rightarrow X$ is a {\it quasi-equivalence} if it has the following properties:
\begin{itemize}
\item[$(1)$] The diagram 
$$ \xymatrix{ M(\phi) \ar[rr]^{q} \ar[dr] & & X \ar[dl]^{p} \\
& \Delta^n & }$$ is commutative.

\item[$(2)$] The map $q$ carries marked edges of $M(\phi)$ to $p$-Cartesian edges of $S$; in
other words, $q$ induces a map $M^{\natural}(\phi) \rightarrow X^{\natural}$ of marked simplicial sets.

\item[$(3)$] For $0 \leq i \leq n$, the induced map
$A^i \rightarrow p^{-1} \{i\}$ is a categorical equivalence.
\end{itemize}
\end{definition}

The goal of this section is to prove the following:

\begin{proposition}\label{simplexplay}
Let $p: X \rightarrow \Delta^n$ be a Cartesian fibration. 
\begin{itemize}
\item[$(1)$] There exists a composable sequence of maps
$$ \phi: A^0 \leftarrow A^1 \leftarrow \ldots \leftarrow A^n $$
and a quasi-equivalence $q: M(\phi) \rightarrow X$.

\item[$(2)$] Let 
$$ \phi: A^0 \leftarrow A^1 \leftarrow \ldots \leftarrow A^n $$
be a composable sequence of maps and $q: M(\phi) \rightarrow X$ a quasi-equivalence.
For any map $T \rightarrow \Delta^n$, the induced map
$$ M(\phi) \times_{\Delta^n} T \rightarrow X \times_{\Delta^n} T$$
is a categorical equivalence.
\end{itemize}
\end{proposition}

We first show that, to establish $(2)$ of Proposition \ref{simplexplay}, it suffices to consider
the case where $T$ is a simplex:

\begin{proposition}\label{tulky}
Suppose given a diagram $$ X \rightarrow Y \rightarrow Z$$ of
simplicial sets. For any map $T \rightarrow Z$, we let $X_T$
denote $X \times_Z T$ and $Y_T$ denote $Y \times_Z T$. The
following statements are equivalent:
\begin{itemize}
\item[$(1)$] For any map $T \rightarrow Z$, the induced map $X_T
\rightarrow Y_T$ is a categorical equivalence.

\item[$(2)$] For any $n \geq 0$ and any map $\Delta^n \rightarrow Z$, the
induced map $X_{\Delta^n} \rightarrow Y_{\Delta^n}$ is a
categorical equivalence.
\end{itemize}

\end{proposition}

\begin{proof}
It is clear that $(1)$ implies $(2)$. Let us prove the converse.
Since the class of categorical equivalences is stable under
filtered colimits, it suffices to consider the case where $T$ has
only finitely many nondegenerate simplices. We now work by
induction on the dimension of $T$, and the number of nondegenerate
simplices contained in $T$. If $T$ is empty, there is nothing to
prove. Otherwise, we may write $T = T' \coprod_{\bd \Delta^n}
\Delta^n$. By the inductive hypothesis, the maps
$$ X_{T'} \rightarrow Y_{T'}$$
$$ X_{\bd \Delta^n} \rightarrow Y_{\bd \Delta^n}$$
are categorical equivalences, and by assumption $X_{\Delta^n}
\rightarrow Y_{\Delta^n}$ is a categorical equivalence as well. We
note that $$X_{T} = X_{T'} \coprod_{ X_{\bd \Delta^n} }
X_{\Delta^n}$$ $$Y_{T} = Y_{T'} \coprod_{ Y_{\bd \Delta^n} }
Y_{\Delta^n}.$$ Since the Joyal model structure is left-proper,
these pushouts are homotopy pushouts, and therefore categorically
equivalent to one another.
\end{proof}

Suppose $p: X \rightarrow \Delta^n$ is a Cartesian fibration,
and $q: M(\phi) \rightarrow X$ is a quasi-equivalence. 
Let $f: \Delta^m \rightarrow \Delta^n$ be any
map. We note (see Remark \ref{funkytok}) that $M(\phi)
\times_{\Delta^n} \Delta^m$ may be identified with a mapping
simplex $M(\phi')$, and that the induced map
$$ M(\phi') \rightarrow X \times_{\Delta^n} \Delta^m$$ is again a quasi-equivalence.
Consequently, to establish $(2)$ of Proposition \ref{simplexplay}, it suffices to prove
that every quasi-equivalence is a categorical equivalence. First, we need the following lemma.

\begin{lemma}\label{coraveg}
Let $$ \phi: A^0 \leftarrow \ldots \leftarrow A^n$$
be a composable sequence of maps between simplicial sets, where $n > 0$. Let $y$ be
a vertex of $A^n$, and let the edge $e: y' \rightarrow y$ be the image of
$\Delta^{ \{n-1, n\} } \times \{y\}$ under the map $\Delta^n \times A^n \rightarrow M(\phi)$.
Let $x$ be any vertex of $M(\phi)$ which does not belong to the fiber $A^n$. Then
composition with $e$ induces a weak homotopy equivalence of simplicial sets
$$ \bHom_{ \sCoNerve[M(\phi)] }(x,y') \rightarrow \bHom_{ \sCoNerve[M(\phi)] }(x,y).$$
\end{lemma}

\begin{proof}
Replacing $\phi$ by an equivalent diagram if necessary (using Remark \ref{funkytok}), we may suppose that the map $A^n \rightarrow A^{n-1}$ is a cofibration. Let $\phi'$ denote the composable subsequence
$$ A^0 \leftarrow \ldots \leftarrow A^{n-1}.$$
Let $\calC = \sCoNerve[M(\phi)]$ and let
$\calC_{-} = \sCoNerve[M(\phi')] \subseteq \calC$. There is a pushout diagram in $\sCat$
$$ \xymatrix{ \sCoNerve[A^n \times \Delta^{n-1}] \ar[r] \ar[d] & \sCoNerve[ A^n \times \Delta^n] \ar[d] \\
\calC_{-} \ar[r] & \calC. }$$
This diagram is actually a homotopy pushout, since $\sCat$ is a left proper model category
and the top horizontal map is a cofibration. Form now the pushout
$$ \xymatrix{ \sCoNerve[ A^n \times \Delta^{n-1}] \ar[d] \ar[r] & \sCoNerve[ A^n \times ( \Delta^{n-1} 
\coprod_{ \{n-1\} } \Delta^{ \{n-1, n\} } )] \ar[d] \\
\calC_{-} \ar[r] & \calC_0. }$$
This diagram is also a homotopy pushout. Since the diagram of simplicial sets
$$ \xymatrix{  \{n-1\} \ar[r] \ar[d] & \Delta^{ \{n-1,n\} } \ar[d] \\
\Delta^{n-1} \ar[r] & \Delta^n } $$
is homotopy coCartesian (with respect to the Joyal model structure), we deduce that the natural
map $\calC_0 \rightarrow \calC$ is an equivalence of simplicial categories. It therefore suffices
to prove that composition with $e$ induces a weak homotopy equivalence
$$ \bHom_{\calC_0}(x,y') \rightarrow \bHom_{\calC}(x,y).$$

Form a pushout square
$$ \xymatrix{ \sCoNerve[ A^{n} \times \{n-1,n\} ] \ar[r] \ar[d] & \sCoNerve[ A^n] \times \sCoNerve[ \Delta^{ \{n-1,n\} } ] \ar[d] \\
\calC_0 \ar[r]^{F} & \calC'. }$$
The left vertical map is a cofibration (since $A^n \rightarrow A^{n-1}$ is a cofibration of simplicial sets), and the upper horizontal map is an equivalence of simplicial categories (Corollary \ref{prodcom}). Invoking the left-properness of $\sCat$, we conclude that $F$ is an equivalence of simplicial categories. Consequently, it will suffice to prove
that $\bHom_{\calC'}( F(x), F(y')) \rightarrow \bHom_{\calC'}(F(x),F(y))$ is a weak homotopy equivalence. We now observe that this map is an isomorphism of simplicial sets.
\end{proof}

\begin{proposition}\label{qequiv}
Let $p: X \rightarrow \Delta^n$ be a Cartesian fibration, let
$$ \phi: A^0 \leftarrow \ldots \leftarrow A^n$$ be a composable
sequence of maps of simplicial sets, and let $q: M(\phi)
\rightarrow X$ be a quasi-equivalence. Then $q$ is a categorical
equivalence.
\end{proposition}

\begin{proof}
We proceed by induction on $n$. The result is obvious if $n = 0$,
so let us assume that $n > 0$. Let $\phi'$ denote the composable
sequence of maps
$$ A^0 \leftarrow A^1 \leftarrow \ldots \leftarrow A^{n-1}$$
which is obtained from $\phi$ by omitting $A^n$. Let $v$ denote
the final vertex of $\Delta^n$, and let $T = \Delta^{ \{0,
\ldots, n-1\} }$ denote the face of $\Delta^n$ which is opposite
$v$. Let $X_{v} = X \times_{ \Delta^n} \{v\}$ and $X_T = X
\times_{\Delta^n} T$.

We note that $M(\phi) = M(\phi')\coprod_{ A^n \times
T } (A^n \times \Delta^n) $. We wish to show that the simplicial functor
$$F: \calC \simeq \sCoNerve[M(\phi)] \simeq  \sCoNerve[M(\phi')] \coprod_{
\sCoNerve[A^n \times T]} \sCoNerve[A^n \times \Delta^n] \rightarrow
\sCoNerve[X]$$ is an equivalence of simplicial categories. We note that $\calC$ decomposes
naturally into full subcategories $\calC_{+} = \sCoNerve[ A^n
\times \{v\} ]$ and $\calC_{-} = \sCoNerve[M(\phi')]$, having
the property that $\bHom_{\calC}(X,Y) = \emptyset$ if $x \in
\calC_{+}$, $y \in \calC_{-}$.

Similarly, $\calD = \sCoNerve[X]$ decomposes into full
subcategories $\calD_{+}= \sCoNerve[X_v]$ and $\calD_{-} =
\sCoNerve[X_T]$, satisfying $\bHom_{\calD}(x,y) = \emptyset$ if $x
\in \calD_{+}$ and $y \in \calD_{-}$. We observe that $F$
restricts to give an equivalence between $\calC_{-}$ and
$\calD_{-}$ by assumption, and gives an equivalence between
$\calC_{+}$ and $\calD_{+}$ by the inductive hypothesis. To
complete the proof, it will suffice to show that if $x \in
\calC_{-}$ and $y \in \calC_{+}$, then $F$ induces a homotopy
equivalence
$$ \bHom_{\calC}(x,y) \rightarrow \bHom_{\calD}(F(x),F(y)).$$

We may identify the object $y \in \calC_{+}$ with a vertex of $A^n$.
Let $e$ denote the edge of $M(\phi)$ which is the image of $\{y\} \times
\Delta^{ \{n-1,n\} }$ under the map $A^n \times \Delta^n \rightarrow
M(\phi)$. We let $[e]: y' \rightarrow y$ denote the corresponding
morphism in $\calC$. We have a commutative diagram
$$ \xymatrix{ \bHom_{\calC_{-}}(x,y') \ar[r] \ar[d] & \bHom_{\calC}(x,y) \ar[d] \\
\bHom_{\calD_{-}}(F(x),F(y')) \ar[r] & \bHom_{\calD}(F(x),F(y)). }$$
Here the left vertical arrow is a weak homotopy equivalence by the inductive hypothesis,
and the bottom horizontal arrow (which is given by composition with $[e]$) is a weak homotopy equivalence because $q(e)$ is $p$-Cartesian. Consequently, to complete the proof, it suffices to show that the top horizontal arrow (given by composition with $e$) is a weak homotopy equivalence. This follows immediately from Lemma \ref{coraveg}.
\end{proof}

To complete the proof of Proposition \ref{simplexplay}, it now suffices to show that for any Cartesian fibration $p: X \rightarrow \Delta^n$, there exists a quasi-equivalence $M(\phi) \rightarrow X$.
In fact, we will prove something slightly stronger (in order to make
our induction work):

\begin{proposition}\label{sharpsimplex}
Let $p: X \rightarrow \Delta^n$ be a Cartesian fibration of simplicial sets and $A$ another simplicial set.
Suppose given a commutative diagram of marked simplicial sets
$$ \xymatrix{ A^{\flat} \times (\Delta^n)^{\sharp} \ar[dr] \ar[rr]^{s} & & X^{\natural} \ar[dl] \\
& (\Delta^n)^{\sharp}. & }$$

Then there exists a sequence of composable
morphisms
$$ \phi: A^0 \leftarrow \ldots \leftarrow A^n,$$
a map $A \rightarrow A^n$, and an extension
$$ \xymatrix{ A^{\flat} \times (\Delta^n)^{\sharp} \ar[dr] \ar[r] & M^{\natural}(\phi) \ar[r]^{f} \ar[d] & X^{\natural} \ar[dl] \\
& (\Delta^n)^{\sharp}. & }$$
of the previous diagram, such that $f$ is a quasi-equivalence.
\end{proposition}

\begin{proof}
The proof goes by induction on $n$. We begin by considering the
fiber $s$ over the final vertex $v$ of $\Delta^n$. The map
$s_v: A \rightarrow X_v = X \times_{\Delta^n} \{v\}$ admits a
factorization
$$ A \stackrel{g}{\rightarrow} A^n \stackrel{h}{\rightarrow} S_v$$
where $g$ is a cofibration and $h$ is a trivial Kan fibration. The smash product inclusion
$$ (\{v\}^{\sharp} \times (A^n)^{\flat}) \coprod_{ \{v\}^{\sharp} \times A^{\flat}} ((\Delta^n)^{\sharp} \times A^{\flat}) \subseteq (\Delta^n)^{\sharp} \times (A^n)^{\flat}$$
is marked anodyne (Proposition \ref{markanodprod}). 
Consequently, we deduce the existence of a dotted arrow $f_0$ as indicated in the diagram
$$ \xymatrix{ A^{\flat} \times (\Delta^n)^{\sharp} \ar@{^{(}->}[d] \ar[r] & X^{\natural} \ar[d] \\
(A^n)^{\flat} \times (\Delta^n)^{\sharp} \ar@{-->}[ur]^{f_0} \ar[r] & (\Delta^n)^{\sharp} }$$
of marked simplicial sets, where $f_0 |(A^n \times \{n\}) = h$.

If $n=0$, we are now done. If $n > 0$, then we apply the inductive hypothesis to the diagram
$$ \xymatrix{ (A^n)^{\flat} \times (\Delta^{n-1})^{\sharp} \ar[dr] \ar[rr]^{f_0 | A^n \times \Delta^{n-1}} & & (X \times_{\Delta^n} \Delta^{n-1})^{\natural} \ar[dl] \\
& (\Delta^{n-1})^{\sharp} & }$$
to deduce the existence of a 
composable sequence of maps
$$ \phi': A^0 \leftarrow \ldots \leftarrow A^{n-1},$$ a map
$A^n \rightarrow A^{n-1}$, and a commutative diagram
$$ \xymatrix{ (A^n)^{\flat} \times (\Delta^{n-1})^{\sharp} \ar[dr] \ar[r] & M^{\natural}(\phi') \ar[r]^{f'} & (X \times_{\Delta^n} \Delta^{n-1})^{\natural} \ar[dl] \\
& (\Delta^{n-1})^{\sharp} & }$$
where $f'$ is a quasi-equivalence. We now define
$\phi$ to be the result of appending the map $A^n
\rightarrow A^{n-1}$ to the beginning of $\phi'$, and let $f: M(\phi) \rightarrow X$ be the
map obtained by amalgamating $f_0$ and $f'$.
\end{proof}

\begin{corollary}\label{presalad}
Let $p: X \rightarrow S$ be a Cartesian fibration of simplicial sets, and let
$q: Y \rightarrow Z$ be a coCartesian fibration. Define new simplicial sets
$Y'$ and $Z'$ equipped with maps $Y' \rightarrow S$, $Z' \rightarrow S$ via the formulas
$$ \Hom_{S}(K, Y') \simeq \Hom( X \times_{S} K, Y)$$
$$ \Hom_{S}(K,Z') \simeq \Hom( X \times_{S} K, Z).$$
Then:
\begin{itemize}
\item[$(1)$] Composition with $q$ determines a coCartesian fibration
$q': Y' \rightarrow Z'$.
\item[$(2)$] An edge $\Delta^1 \rightarrow Y'$ is $q'$-coCartesian if and only if 
the induced map $\Delta^1 \times_{S} X \rightarrow Y$
carries $p$-Cartesian edges to $q$-coCartesian edges.
\end{itemize}
\end{corollary}

\begin{proof}
Let us say that an edge of $Y'$ is {\it special} if it satisfies the hypothesis of $(2)$. Our first goal is to show that there is a sufficient supply of special edges in $Y'$. More precisely, we claim that given any edge $e: z \rightarrow z'$ in $Z'$ and any vertex $\widetilde{z} \in Y'$ covering $z$, there exists a special edge $\widetilde{e}: \widetilde{z} \rightarrow \widetilde{z}'$ of $Y'$ which covers $e$. 

Suppose that the edge $e$ covers an edge $e_0: s \rightarrow s'$ in $S$. We can identify
$\widetilde{z}$ with a map from $X_{s}$ to $Y$.
Using Proposition \ref{simplexplay}, we can choose a morphism $\phi: X'_{s} \leftarrow X'_{s'}$ and a quasi-equivalence $M(\phi) \rightarrow X \times_{S} \Delta^1$. Composing with $\widetilde{z}$, we obtain a map $X'_{s} \rightarrow Y$. Using Propositions \ref{funkyfibcatfib} and \ref{princex}, we may reduce to the problem of providing a dotted arrow in the diagram
$$ \xymatrix{ X'_{s} \ar@{^{(}->}[d] \ar[r] & Y \ar[d]_{q} \\
M(\phi) \ar@{-->}[ur] \ar[r] & Z }$$
which carries the marked edges of $M^{\natural}(\phi)$ to $q$-coCartesian edges of $Y$.
This follows from the the fact that $q^{X_{s}}: Y^{X_{s}} \rightarrow Z^{X_{s}}$ is a coCartesian fibration, and the description of the $q^{X_{s}}$-coCartesian edges (Proposition \ref{doog}).

To complete the proofs of $(1)$ and $(2)$, it will suffice to show that $q'$ is an inner fibration and that every special edge of $Y'$ is $q'$-coCartesian. For this, we must show that every lifting problem
$$ \xymatrix{ \Lambda^n_i \ar[r]^{\sigma_0} \ar@{^{(}->}[d] & Y' \ar[d]^{q'} \\
\Delta^n \ar[r] \ar@{-->}[ur] & Z' }$$
has a solution, provided that either $0 < i < n$, or $i =0$, $n \geq 2$, and
$\sigma_0 | \Delta^{ \{0,1\} }$ is special. We can reformulate this lifting problem
using the diagram
$$ \xymatrix{ X \times_{S} \Lambda^n_i \ar[r] \ar@{^{(}->}[d] & Y \ar[d]^{q} \\
X \times_{S} \Delta^n \ar[r] \ar@{-->}[ur] & Z. }$$
Using Proposition \ref{simplexplay}, we can choose a composable sequence of morphisms
$$ \psi: X'_0 \leftarrow \ldots \leftarrow X'_{n} $$
and a quasi-equivalence $M(\psi) \rightarrow X \times_{S} \Delta^n$. Invoking
Propositions \ref{funkyfibcatfib} and \ref{princex}, we may reduce to the associated mapping problem
$$ \xymatrix{ M(\psi) \times_{ \Delta^n} \Lambda^n_i \ar[r] \ar[d] & Y \ar[d]^{q} \\
M(\psi) \ar[r] \ar@{-->}[ur] & Z.}$$
Since $i < n$, this is equivalent to the mapping problem
$$ \xymatrix{ X'_{n} \times \Lambda^n_i \ar[r] \ar@{^{(}->}[d] & Y \ar[d]^{q} \\
X'_{n} \times \Delta^n \ar[r] & Z, }$$
which admits a solution in virtue of Proposition \ref{doog}.
\end{proof}

\begin{corollary}\label{skinnysalad}
Let $p: X \rightarrow S$ be a Cartesian fibration of simplicial sets, and let
$q: Y \rightarrow S$ be a coCartesian fibration. Define a new simplicial set $T$
equipped with a map $T \rightarrow S$ by the formula
$$ \Hom_{S}(K, T) \simeq \Hom_{S}( X \times_{S} K, Y).$$
Then:
\begin{itemize}
\item[$(1)$] The projection $r: T \rightarrow S$ is a coCartesian fibration.
\item[$(2)$] An edge $\Delta^1 \rightarrow Z$ is $r$-coCartesian if and only if 
the induced map $\Delta^1 \times_{S} X \rightarrow \Delta^1 \times_{S} Y$
carries $p$-Cartesian edges to $q$-coCartesian edges.
\end{itemize}
\end{corollary}

\begin{proof}
Apply Corollary \ref{presalad} in the case where $Z = S$.
\end{proof}

We conclude by noting the following additional property of quasi-equivalences, using the terminology of \S \ref{markmodel}:

\begin{proposition}\label{halfy}
Let $S = \Delta^n$, let $p: X \rightarrow S$ be a Cartesian fibration, let
$$\phi: A^0 \leftarrow \ldots \leftarrow A^n$$ be a composable sequence of maps, and let
$q: M(\phi) \rightarrow X$ be a quasi-equivalence. The induced map $M^{\natural}(\phi) \rightarrow X^{\natural}$ is a Cartesian equivalence in $(\mSet)_{/S}$.
\end{proposition}

\begin{proof}
We must show that for any Cartesian fibration $Y \rightarrow S$, the induced map
of $\infty$-categories $$\bHom^{\flat}_{S}(X^{\natural}, Y^{\natural}) \rightarrow \bHom^{\flat}_S( M^{\natural}(\phi), Y^{\natural} )$$
is a categorical equivalence. Because $S$ is a simplex, the left side may be identified with a full subcategory of $Y^X$ and the right side with a full subcategory of $Y^{M(\phi)}$. Since $q$ is a categorical equivalence, the natural map
$Y^X \rightarrow Y^{M(\phi)}$ is a categorical equivalence; thus, to complete the proof, it suffices to observe that a map of simplicial sets $f: X \rightarrow Y$ is compatible with the projection to $S$ and preserves marked edges if and only if $q \circ f$ has the same properties.
\end{proof}

\subsection{Straightening over a Simplex}\label{markmodel24}

Let $S$ be a simplicial set, $\calC$ a simplicial category, and $\phi: \sCoNerve[S]^{op} \rightarrow \calC$ a simplicial functor. In \S \ref{markmodel2}, we introduced the straightening and unstraightening functors
$$ \Adjoint{ \St^{+}_{\phi} }{ (\mSet)_{/S}}{(\mSet)^{\calC}}{ \Un^{+}_{\phi}}.$$
In this section, we will prove that $( \St^{+}_{\phi}, \Un^{+}_{\phi} )$ is a Quillen equivalence provided that $\phi$ is a categorical equivalence and $S$ is a simplex (the case of a general simplicial set $S$ will be treated in \S \ref{markmodel25}). 

Our first step is to prove the result in the case where $S$ is a point and $\phi$ is an isomorphism
of simplicial categories. We can identify the functor $\St^{+}_{\Delta^0}$ with the functor
$T: \mSet \rightarrow \mSet$ studied in \S \ref{markmodel2}. Consequently, Theorem \ref{straightthm} is an immediate consequence of Proposition \ref{spek4}:

\begin{lemma}\label{utest}
The functor $T: \mSet \rightarrow \mSet$ has a right adjoint $U$, and the pair
$(T,U)$ is a Quillen equivalence from $\mSet$ to itself.
\end{lemma}

\begin{proof}
We have already established the existence of the unstraightening functor $U$ in
\S \ref{markmodel2}, and proved that $(T,U)$ is a Quillen adjunction. To complete the proof,
it suffices to show that the left derived functor of $T$ (which we may identify with $T$, since
every object of $\mSet$ is cofibrant) is an equivalence from the homotopy category
of $\mSet$ to itself. But Proposition \ref{spek4} asserts that $T$ is isomorphic to the identity functor on the homotopy category of $\mSet$.
\end{proof}

Let us now return to the case of a general equivalence $\phi: \sCoNerve[S] \rightarrow \calC^{op}$.
Since we know that $(\St^{+}_{\phi}, \Un^{+}_{\phi})$ give a Quillen adjunction between $(\mSet)_{/S}$ and
$(\mSet)^{\calC}$, it will suffice to prove that the unit and counit
$$ u: \id \rightarrow R \Un^{+}_{\phi} \circ L \St^{+}_{\phi}$$
$$ v: L \St^{+}_{\phi} \circ R \Un^{+}_{\phi} \rightarrow \id$$
are weak equivalences. Our first step is to show that $R \Un^{+}_{\phi}$ detects weak equivalences: this reduces the problem of proving that $v$ is an equivalence to the problem of proving that $u$ is an equivalence. 

\begin{lemma}\label{garbz}
Let $S$ be a simplicial set, $\calC$ a simplicial category, and $\phi: \sCoNerve[S] \rightarrow \calC^{op}$ an essentially surjective functor. Let $p: \calF \rightarrow \calG$ be a map between (weakly) fibrant objects
of $(\mSet)^{\calC}$. Suppose that $\Un^{+}_{\phi}(p): \Un^{+}_{\phi} \calF \rightarrow \Un^{+}_{\phi} \calG$ is a Cartesian equivalence. Then $p$ is an equivalence.
\end{lemma}

\begin{proof}
Since $\phi$ is essentially surjective, it suffices to prove that $\calF(C) \rightarrow \calF(D)$
is a Cartesian equivalence for every object $C \in \calC$ which lies in the image of $\phi$. 
Let $s$ be a vertex of $S$ with $\psi(s) = C$. Let $i: \{s\} \rightarrow S$ denote the inclusion, and
$i^{\ast}: (\mSet)_{/S} \rightarrow \mSet$ denote the functor of passing to the fiber over $s$:
$$ i^{\ast} X = X_{s} = X \times_{ S^{\sharp} } \{s\}^{\sharp}.$$
Let $i_{!}$ denote the left adjoint to $i^{\ast}$. Let $\{C\}$ denote the trivial
category with one object (and only the identity morphism), and let $j: \{C\} \rightarrow \calC$ be the simplicial functor corresponding to the inclusion of $C$ as an object of $\calC$. According to Proposition \ref{formall}, we have a natural identification of functors
$$ \St^{+}_{\phi} \circ i_{!} \simeq j_{!} \circ T.$$
Passing to adjoints, we get another identification
$$ i^{\ast} \circ \Un^{+}_{\phi} \simeq U \circ j^{\ast}$$
from $(\mSet)^{\calC}$ to $\mSet$. Here $U$ denotes the right adjoint of $T$.

According to Lemma \ref{utest}, the functor $U$ detects equivalences between fibrant objects
of $\mSet$. Thus, it suffices to prove that $U( j^{\ast} \calF) \rightarrow U( j^{\ast} \calG)$ is a Cartesian equivalence. Using the identification above, we are reduced to proving that
$$ \Un^{+}_{\phi}(\calF)_{s} \rightarrow \Un^{+}_{\phi}(\calG)_{s}$$ is a Cartesian equivalence.
But $\Un^{+}_{\phi}(\calF)$ and $\Un^{+}_{\phi}(\calG)$ are fibrant objects of $(\mSet)_{/S}$, and therefore correspond to Cartesian fibrations over $S$: the desired result now follows from
Proposition \ref{crispy}.
\end{proof}

We have now reduced the proof of Theorem \ref{straightthm} to the problem of showing that
if $\phi: \sCoNerve[S] \rightarrow \calC^{op}$ is an equivalence of simplicial categories, then
the unit transformation
$$ u: \id \rightarrow R \Un^{+}_{\phi} \circ \St^{+}_{\phi}$$
is an isomorphism of functors from the homotopy category $\h{(\mSet)_{/S}}$ to itself. 

Our first step is to analyze the effect of the straightening functor $\St^{+}_{\phi}$ on a mapping simplex.
We will need a bit of notation. For any $X \in (\mSet)_{/S}$ and any vertex $s$ of $S$,
we let $X_{s}$ denote the fiber $X \times_{ S^{\sharp} } \{s\}^{\sharp}$, and let $i^{s}$
denote the composite functor
$$ \{s\} \hookrightarrow \sCoNerve[S] \stackrel{\phi}{\rightarrow} \calC^{op}$$
of simplicial categories. According to Proposition \ref{formall}, there is a natural identification
$$ \St^{+}_{\phi}(X_s) \simeq i^{s}_{!} T(X_s),$$ and consequently an induced map
$$ \psi^X_{s}: T(X_s) \rightarrow \St^{+}_{\phi}(X)(s).$$

\begin{lemma}\label{piecemeal}
Let $$\theta: A^0 \leftarrow \ldots \leftarrow A^n$$ 
be a composable sequence of maps of simplicial sets, and let
$M^{\natural}(\theta) \in (\mSet)_{\Delta^n}$ be its mapping simplex.
For each $0 \leq i \leq n$, the map
$$ \psi^{M^{\natural}(\theta)}_{i}: T(A^i)^{\flat} \rightarrow \St^{+}_{\Delta^n}(M^{\natural}(\theta))(i) $$  
is a Cartesian equivalence in $\mSet$.
\end{lemma}

\begin{proof}
The proof goes by induction on $n$. We first observe that $\psi^{M^{\natural}(\theta)}_{n}$
is an isomorphism; we may therefore restrict our attention to $i < n$. 
Let $\theta'$ be the composable sequence
$$ A^0 \leftarrow \ldots \leftarrow A^{n-1},$$
and $M^{\natural}(\theta')$ its mapping simplex, which we may regard either as an object
of $(\mSet)_{/\Delta^n}$ or $(\mSet)_{/\Delta^{n-1}}$.

For $i < n$, we have a commutative diagram
$$ \xymatrix{ & \St^{+}_{\Delta^n}(M^{\natural}(\theta'))(i) \ar[dr]^{f_{i}} & \\
T((A^i)^{\flat}) \ar[ur]^{ \psi^{M^{\natural}(\theta')}_i} \ar[rr]
 & & \St^{+}_{\Delta^n}(M^{\natural}(\theta))(i).}$$
By Proposition \ref{formall}, $\St^{+}_{\Delta^n} M^{\natural}(\theta') \simeq j_! \St^{+}_{\Delta^{n-1}} M^{\natural}(\theta')$, where $j: \sCoNerve[\Delta^{n-1}] \rightarrow \sCoNerve[\Delta^n]$ denotes the inclusion. Consequently, the inductive hypothesis implies that the maps
$$ T(A^i)^{\flat} \rightarrow \St^{+}_{\Delta^{n-1}}(M^{\natural}(\theta'))(i) $$ are Cartesian equivalences for $i < n$. It now suffices to prove that $f_i$ is a Cartesian equivalence, for $i < n$.

We observe that there is a (homotopy) pushout diagram
$$ \xymatrix{ (A^n)^{\flat} \times (\Delta^{n-1})^{\sharp} \ar[r] \ar[d] & (A^n)^{\flat} \times (\Delta^n)^{\sharp} \ar[d] \\
M^\natural(\theta') \ar[r] & M^\natural(\theta) }.$$
Since $\St^{+}_{\Delta^n}$ is a left Quillen functor, it induces a homotopy pushout diagram
$$ \xymatrix{ \St^{+}_{\Delta^n} ( (A^n)^{\flat} \times (\Delta^{n-1})^\sharp) \ar[r]^{g} \ar[d] &
\St^{+}_{\Delta^n} ( (A^n)^{\flat} \times (\Delta^n)^{\sharp} ) \ar[d] \\
\St^{+}_{\Delta^n} M^{\natural}(\theta') \ar[r] & \St^{+}_{\Delta^n} M^\natural(\theta).}$$
in $(\mSet)^{\calC}$. We are therefore reduced to proving that $g$ induces a Cartesian equivalence 
after evaluation at any $i < n$.

According to Proposition \ref{spek3}, the vertical maps of the diagram
$$ \xymatrix{ 
\St^{+}_{\Delta^n} ((A^n)^{\flat} \times (\Delta^{n-1})^{\sharp}) \ar[r] \ar[d] & \St^{+}_{\Delta^n} ((A^n)^{\flat} \times (\Delta^n)^{\sharp}) \ar[d] \\ T(A^n)^{\flat} \boxtimes
\St^{+}_{\Delta^n} (\Delta^{n-1})^{\sharp} \ar[r] & T(A^n)^{\flat} \boxtimes \St^{+}_{\Delta^n} (\Delta^n)^{\sharp}}$$
are Cartesian equivalences. To complete the proof we must show that
$$\St^{+}_{\Delta^n}( \Delta^{n-1})^{\sharp} \rightarrow \St^{+}_{\Delta^n}( \Delta^n )^{\sharp}$$ induces a Cartesian equivalence when evaluated at any $i < n$. Consider the diagram
$$ \xymatrix{ \{ n-1 \}^{\sharp} \ar[r] \ar[d] & ( \Delta^{n-1} )^{\sharp} \ar[d] \\
 ( \Delta^{ \{n-1,n\}} )^{\sharp} \ar[r] & (\Delta^n)^{\sharp}.}$$
The horizontal arrows are marked anodyne. It therefore suffices to show that
$$\St^{+}_{\Delta^n} \{n-1\}^{\sharp} \rightarrow \St^{+}_{\Delta^n} (\Delta^{ \{n-1,n\} })^{\sharp}$$
induces Cartesian equivalences when evaluated at any $i < n$. This follows from an easy computation.
\end{proof}

\begin{proposition}\label{speccase}
Let $n \geq 0$. Then the Quillen adjunction 
$$\Adjoint{\St^{+}_{\Delta^n}}{ (\mSet)_{/ \Delta^n} }{ (\mSet)^{\sCoNerve[\Delta^n]}}{ \Un^{+}_{\Delta^n}}$$ is a Quillen equivalence.
\end{proposition}

\begin{proof}
As we have argued above, it suffices to show that the unit
$$ \id \rightarrow R \Un^{+}_{\phi} \circ \St^{+}_{\Delta^n}$$
is an isomorphism of functors from $\h{ (\mSet)_{\Delta^n}}$ to itself. In other words, we must show that given an object
$X \in (\mSet)_{/ \Delta^n}$ and a weak equivalence
$$ \St^{+}_{\Delta^n} X \rightarrow \calF,$$
where $\calF \in (\mSet)^{\sCoNerve[\Delta^n]}$ is fibrant, the adjoint map
$$ j: X \rightarrow \Un^{+}_{\Delta^n} \calF$$ is a Cartesian equivalence in $(\mSet)_{/ \Delta^n}$.

Choose a fibrant replacement for $X$: that is, a Cartesian equivalence  $X \rightarrow Y^{\natural}$ where $Y \rightarrow \Delta^n$ is a Cartesian fibration. According to Proposition \ref{simplexplay}, there exists a composable sequence of maps
$$ \theta: A^0 \leftarrow \ldots \leftarrow A^n$$
and a quasi-equivalence $M^{\natural}(\theta) \rightarrow Y^{\natural}$. Proposition \ref{halfy}
implies that $M^{\natural}(\theta) \rightarrow Y^{\natural}$ is a Cartesian equivalence. Thus,
$X$ is equivalent to $M^{\natural}(\theta)$ in the homotopy category of $(\mSet)_{/\Delta^n}$ and we are free to replace $X$ by $M^{\natural}(\theta)$, thereby reducing to the case where $X$ is a mapping simplex.

We wish to prove that $j$ is a Cartesian equivalence. Since $\Un^{+}_{\Delta^n} \calF$ is fibrant,
Proposition \ref{halfy} implies that it suffices to show that $j$ is a quasi-equivalence: in other words, 
we need to show that the induced map of fibers 
$j_s: X_{s} \rightarrow (\Un^{+}_{\Delta^n} \calF)_s$ is a Cartesian equivalence, for each vertex $s$ of $\Delta^n$. 
As in the proof of Lemma \ref{garbz}, we may identify $(\Un^{+}_{\Delta^n} \calF)_s$ with
$U(\calF(s))$, where $U$ is the right adjoint to $T$. By Lemma \ref{utest}, $X_{s} \rightarrow U (\calF(s))$ is a Cartesian equivalence if and only if the adjoint map
$T(X_{s}) \rightarrow \calF(s)$ is a Cartesian equivalence. This map factors as a composition
$$ T(X_{s}) \rightarrow \St^{+}_{\Delta^n}(X)(s) \rightarrow \calF(s).$$
The map on the left is a Cartesian equivalence by Lemma \ref{piecemeal}, and the map on
the right in virtue of the assumption that $\St^{+}_{\Delta^n} X \rightarrow \calF$ is a weak equivalence.
\end{proof}

\subsection{Straightening in the General Case}\label{markmodel25}

Let $S$ be a simplicial set and $\phi: \sCoNerve[S] \rightarrow \calC^{op}$ an equivalence of simplicial categories. Our goal in this section is to complete the proof of Theorem \ref{straightthm} by showing that $( \St_{\phi}^{+}, \Un_{\phi}^{+})$ is a Quillen equivalence between
$(\mSet)_{/S}$ and $(\mSet)^{\calC}$. In \S \ref{markmodel24}, we handled the case where $S$ was a simplex (and $\phi$ an isomorphism), by verifying that the unit map
$\id \rightarrow R \Un_{\phi}^{+} \circ \St_{\phi}^{+}$ is an isomorphism of functors
from $\h{(\mSet)_{/S}}$ to itself. 

Here is the idea of the proof.  Without loss of generality, we may suppose that $\phi$ is an isomorphism (since the pair $(\phi_{!}, \phi^{\ast})$ is a Quillen equivalence between $(\mSet)^{ \sCoNerve[S]^{op}}$ and $(\mSet)^{ \calC }$, by Proposition \ref{lesstrick}). We wish to show that $\Un^{+}_{\phi}$
induces an equivalence from the homotopy category of $(\mSet)^{\calC}$ to
the homotopy category of $(\mSet)_{/S}$. According to Proposition \ref{speccase}, this is true
whenever $S$ is a simplex. In the general case, we would like to regard $(\mSet)^{\calC}$
and $(\mSet)_{/S}$ as somehow built out of pieces which are associated to simplices, and deduce that $\Un^{+}_{\phi}$ is an equivalence because it is an equivalence on each piece. In order to make this argument work, it is necessary to work not just with the homotopy categories of
$(\mSet)^{\calC}$ and $(\mSet)_{/S}$, but with the simplicial categories which give rise to them.

We recall that both $(\mSet)^{\calC}$ and $(\mSet)_{/S}$ are {\em simplicial} model categories with respect to the simplicial mapping spaces defined by
$$ \Hom_{\sSet}(K, \bHom_{ (\mSet)^{\calC}}(\calF, \calG)) = \Hom_{(\mSet)^{\calC}}(
\calF \boxtimes K^{\sharp}, \calG)$$
$$ \Hom_{\sSet}(K, \bHom_{(\mSet)_S}(X,Y)) = \Hom_{\sSet}(K, \bHom_{S}^{\sharp}(X,Y))
= \Hom_{(\mSet)_{/S}}(X \times K^{\sharp}, Y).$$
The functor $\St^{+}_{\phi}$ is not a simplicial functor. However, it is {\em weakly} compatible with the simplicial structure in the sense that there is a natural map
$$ \St^{+}_{\phi} (X \boxtimes K^{\sharp}) \rightarrow (\St^{+}_{\phi} X) \boxtimes K^{\sharp}$$
for any $X \in (\mSet)_{/S}$, $K \in \sSet$ (according to Corollary \ref{spek5}, this map is a weak equivalence in $(\mSet)^{\calC}$). Passing to adjoints, we get natural maps
$$ \bHom_{ (\mSet)^{\calC} }( \calF, \calG) \rightarrow \bHom_{S}^{\sharp}( \Un^{+}_{\phi} \calF, \Un^{+}_{\phi} \calG ).$$ In other words, $\Un^{+}_{\phi}$ {\em does} have the structure of a simplicial functor.
We now invoke Proposition \ref{weakcompatequiv} to deduce the following:

\begin{lemma}\label{gottaprove0}
Let $S$ be a simplicial set, $\calC$ a simplicial category, and $\phi: \sCoNerve[S] \rightarrow \calC^{op}$ a simplicial functor. The following are equivalent:
\begin{itemize}
\item[$(1)$] The Quillen adjunction $( \St^{+}_{\phi}, \Un^{+}_{\phi} )$ is a Quillen equivalence.

\item[$(2)$] The functor $\Un^{+}_{\phi}$ induces an equivalence of simplicial categories
$$ (\Un^{+}_{\phi})^{\degree} : ((\mSet)^{\calC})^{\degree} \rightarrow ((\mSet)_{/S})^{\degree},$$
where $((\mSet)^{\calC})^{\degree}$ denotes the full (simplicial) subcategory of
$( ( \mSet)^{\calC})$ consisting of fibrant-cofibrant objects, and $((\mSet)_{/S})^{\degree}$ denotes the full (simplicial) subcategory of $(\mSet)_{/S}$ consisting of fibrant-cofibrant objects.
\end{itemize}
\end{lemma}

Consequently, to complete the proof of Theorem \ref{straightthm}, it will suffice to show that
if $\phi$ is an equivalence of simplicial categories, then $(\Un^{+}_{\phi})^{\degree}$ is an equivalence of simplicial categories. The first step is to prove that $(\Un^{+}_{\phi})^{\degree}$ is fully faithful.

\begin{lemma}\label{gottaprove2}
Let $S' \subseteq S$ be simplicial sets, and let
$p: X \rightarrow S$, $q: Y \rightarrow S$ be Cartesian fibrations.
Let $X' = X \times_{S} S'$ and $Y' = Y \times_{S} S'$. The restriction map
$$ \bHom_{S}^{\sharp}(X^{\natural}, Y^{\natural}) \rightarrow \bHom_{S'}^{\sharp}( {X'}^{\natural}, {Y'}^{\natural})$$ is a Kan fibration.
\end{lemma}

\begin{proof}
It suffices to show that the map $Y^{\natural} \rightarrow S$ has the right lifting property with
respect to the inclusion
$$ ( {X'}^{\natural} \times B^{\sharp} ) \coprod_{ {X'}^{\natural} \times A^{\sharp} } (X^{\natural} \times A^{\sharp}) \subseteq X^{\natural} \times B^{\sharp},$$
for any anodyne inclusion of simplicial sets $A \subseteq B$.

But this is a smash product of a marked cofibration ${X'}^{\natural} \rightarrow X^{\natural}$ 
(in $(\mSet)_{/S}$) and a trivial marked cofibration $A^{\sharp} \rightarrow B^{\sharp}$
( in $\mSet$), and is therefore a trivial marked cofibration. We conclude by observing that $Y^{\natural}$ is a fibrant object of $(\mSet)_{/S}$ (Proposition \ref{markedfibrant}).
\end{proof}

\begin{proof}[Proof of Theorem \ref{straightthm}]
For each simplicial set $S$, let $(\mSet)^{\sCoNerve[S]^{op}}_{f}$ denote the
category of projectively fibrant objects of $(\mSet)^{\sCoNerve[S]^{op}}$, and let
$W_{S}$ be the class of weak equivalences in $( \mSet)^{\sCoNerve[S]^{op}}_{f}$.
Let $W'_{S}$ be the collection of pointwise equivalences in $(\mSet)_{/S}^{\degree}$.
We have a commutative diagram of simplicial categories
$$ \xymatrix{ ((\mSet)^{\sCoNerve[S]^{op}})^{\degree} \ar[r]^{ \Un^{+}_{S} } \ar[d] &
(\mSet)_{/S}^{\degree} \ar[d]^{\psi_{S}} \\
( \mSet)^{\sCoNerve[S]^{op}}_{f}[W_{S}^{-1} ] \ar[r]^{\phi_S} & ( \mSet)_{/S}^{\degree}[ {W'}^{-1}_{S} ] }$$ (see Notation \ref{localdef}). In view of Lemma \ref{gottaprove0}, it will suffice to show
that the upper horizontal map is an equivalence of simplicial categories.
Lemma \ref{kur} implies that the left vertical map is an equivalence.
Using Lemma \ref{postcuse} and Remark \ref{uppa}, we deduce that the right vertical
map is also an equivalence. It will therefore suffice to show that $\phi_{S}$ is an equivalence. 

Let $\calU$ denote the collection of simplicial sets $S$ for which $\phi_{S}$ is an equivalence.
We will show that $\calU$ satisfies the hypotheses of Lemma \ref{blem}, and therefore contains every simplicial set $S$. Conditions $(i)$ and $(ii)$ are obviously satisfied, and
condition $(iii)$ follows from Lemma \ref{gottaprove0} and Proposition \ref{speccase}.
We will verify condition $(iv)$; the proof of $(v)$ is similar.

Applying Corollary \ref{uspin}, we deduce:
\begin{itemize}
\item[$(\ast)$] The functor $S \mapsto ( \mSet)^{\sCoNerve[S]^{op}}_{f} [W_S^{-1}]$ carries homotopy colimit diagrams indexed by a partially ordered set to homotopy limit diagrams in $\sCat$.
\end{itemize}

Suppose given a pushout diagram
$$ \xymatrix{ X \ar[r] \ar[d]^{f} & X' \ar[d] \\
Y \ar[r] & Y' }$$
in which $X, X', Y \in \calU$, where $f$ is a cofibration. We wish to prove that $Y' \in \calU$.
We have a commutative diagram
$$ \xymatrix{ ( \mSet)^{\sCoNerve[Y']^{op}}_{f}[W_{Y'}^{-1}] \ar[dr]^{\phi_{Y'}} & & \\
& (\mSet)_{/Y'}^{\degree}[ {W'}_{Y'}^{-1}] \ar[r]^{u} \ar[d]^{v} \ar[dr]^{w} & (\mSet)_{/Y}^{\degree}[ {W'}_{Y}^{-1}] \ar[d] \\
& (\mSet)_{/X'}^{\degree}[ {W'}_{X'}^{-1}] \ar[r] & (\mSet)_{/X}^{\degree}[ {W'}_{X}^{-1} ]. }$$
Using $(\ast)$ and Corollary \ref{wspin}, we deduce that $\phi_{Y'}$ is an equivalence if and only if,
for every pair of objects $x,y \in (\mSet)_{/Y'}^{\degree}[ {W'}_{Y'}^{-1}]$, the diagram
of simplicial sets
$$ \xymatrix{ \bHom_{ (\mSet)_{/Y'}^{\degree}[ {W'}_{Y'}^{-1}] }( x, y) \ar[r] \ar[d] 
& \bHom_{ (\mSet)_{/Y}^{\degree}[ {W'}_{Y}^{-1}] }( u(x), u(y) ) \ar[d] \\
\bHom_{ (\mSet)_{/X'}^{\degree}[ {W'}_{X'}^{-1}] }( v(x), v(y) ) \ar[r] &
\bHom_{ (\mSet)_{/X}^{\degree}[ {W'}_{X}^{-1}] }( w(x), w(y) ) }$$ 
is homotopy Cartesian. Since $\psi_{Y'}$ is a weak equivalence of simplicial categories, 
we may assume without loss of generality that $x = \psi_{Y'}( \overline{x} )$ and
$y = \psi_{Y'}( \overline{y} )$, for some $ \overline{x}, \overline{y} \in (\mSet)^{\degree}_{/Y'}$. 
It will therefore suffice to prove that the equivalent diagram
$$ \xymatrix{ \bHom^{\sharp}_{ Y' }( \overline{x}, \overline{y}) \ar[r] \ar[d] 
& \bHom^{\sharp}_{Y }( \overline{u}(\overline{x}), \overline{u}(\overline{y}) ) \ar[d] \\
\bHom^{\sharp}_{ X' }( \overline{v}(\overline{x}), \overline{v}(\overline{y}) ) \ar[r]^{g} &
\bHom^{\sharp}_{X }( \overline{w}(\overline{x}), \overline{w}(\overline{y}) ) }$$ 
is homotopy Cartesian. But this diagram is a pullback square, and the map $g$ is a Kan
fibration by Lemma \ref{gottaprove2}.
\end{proof}


%\begin{proposition}\label{fulfat}
%For every simplicial set $S$, the functor 
%$$(\Un^{+}_{S})^{\degree}: (\mSet)^{ \sCoNerve[S]^{op}} \rightarrow (\mSet)_{/S}$$
%is fully faithful (as a functor between simplicial categories).
%\end{proposition}

%\begin{proof}
%Let $\calC = \sCoNerve[S]^{op}$, and choose
%fibrant-cofibrant objects $\calF, \calG \in (\mSet)^{\calC}$. We wish to show that the natural map $$ \bHom_{ (\mSet)^{\calC}}(\calF, \calG) \rightarrow \bHom^{\sharp}_{ S }(
%\Un^{+}_{S} \calF, \Un^{+}_{S} \calG)$$ is a homotopy equivalence of Kan complexes.

%According to Proposition \ref{gumby44}, there is an equivalence of $\infty$-categories
%$$ p: \sNerve ((\mSet)^{\calC})^{\degree} \rightarrow \Fun(S^{op}, \Cat_{\infty}).$$
%Let $F=p(\calF)$ and $G=p(\calG)$. Let $X_{S} = \Hom^{\rght}_{\Fun(S^{op}, \Cat_{\infty})}(F,G)$,
%$Y_S = | \Hom^{\rght}_{\sNerve((\mSet)^{\calC})^{\degree}} (\calF,\calG)|_{Q^{\bigdot}}$, and 
%$Z_S =  \bHom_{ (\mSet)_{/S}}( \Un^{+}_{S} \calF, \Un^{+}_{S} \calG)$.
%We have a chain of maps
%$$ X_S \stackrel{f_0}{\leftarrow} | \Hom^{\rght}_{\Fun(S^{op},\Cat_{\infty})}(F,G)|_{Q^{\bigdot}}
%\stackrel{f_1}{\leftarrow} Y_S \stackrel{f_2}{\rightarrow}
%\bHom_{ (\mSet)^{\calC}}(\calF,\calG) \stackrel{f_3}{\rightarrow} Z_S.$$ 

%Note that $f_0$ and $f_1$ are weak homotopy equivalences, so their composition
%$i_{S} = f_0 \circ f_1: Y_{S} \rightarrow X_{S}$ is a weak homotopy equivalence.
%Since $f_2$ is weak homotopy equivalence, the map $f_3$ is a weak homotopy equivalence if and only if the composition $j_{S} = f_3 \circ f_2: Y_{S} \rightarrow Z_{S}$ is a weak homotopy equivalence. Consequently, our goal is to prove that $j_{S}$ is a weak homotopy equivalence.

%Let us first assume that $S$ is finite-dimensional, and work by induction on the dimension $n$ of $S$.
%Let $S'$ denote the $(n-1)$-skeleton of $S$, and let $A$ denote the set of all nondegenerate $n$-simplices of $S$. Then $S \simeq S' \coprod_{ \bd \Delta^n \times A} (\Delta^n \times A)$. By Lemma \ref{sharpy}, the square
%$$ \xymatrix{ X_{S} \ar[d] \ar[r] & X_{S'} \ar[d] \\
%X_{\Delta^n \times A} \ar[r] & X_{ \bd \Delta^n \times A} }$$
%is homotopy Cartesian. It follows that the equivalent square
%$$ \xymatrix{ Y_{S} \ar[d] \ar[r] & Y_{S'} \ar[d] \\
%Y_{\Delta^n \times A} \ar[r] & Y_{ \bd \Delta^n \times A} }$$
%is homotopy Cartesian. By Lemma \ref{gottaprove2}, the square
%$$ \xymatrix{ Z_{S} \ar[d] \ar[r] & Z_{S'} \ar[d] \\
%Z_{\Delta^n \times A} \ar[r] & Z_{ \bd \Delta^n \times A} }$$
%is homotopy Cartesian. Consequently, to prove that $j_{S}$ is a weak equivalence, it suffices
%to show that $j_{S'}$, $j_{ \bd \Delta^n \times A}$, and $j_{ \Delta^n \times A}$ are weak equivalences.
%The first two of these assertions follow from the inductive hypothesis. The third follows from Lemma \ref{gottaprove0} and Proposition \ref{speccase}.

%Now suppose that $S$ is not finite dimensional. Let $S(i)$ denote the $i$-skeleton of $S$.
%By Lemma \ref{sharpy}, each of the maps $X_{S(i+1)} \rightarrow X_{S(i)}$ is a Kan fibration.
%It follows that $X_{S} = \lim_{i} X_{S(i)}$ is also the homotopy limit of the tower
%$$ \ldots \rightarrow X_{S(2)} \rightarrow X_{S(1)} \rightarrow X_{S(0)}.$$ 
%Since each of the maps $i_{K}$ is a homotopy equivalence, we deduce that
%$Y_{S}$ is a homotopy limit of the tower
%$$ \ldots \rightarrow Y_{S(2)} \rightarrow Y_{S(1)} \rightarrow Y_{S(0)}.$$ 
%(This is not obvious, since the maps in this tower are not fibrations of simplicial sets,
%and $Y_{S}$ is not generally the limit of the $Y_{S(i)}$.)

%By Lemma \ref{gottaprove2}, the tower
%$$ \ldots \rightarrow Z_{S(2)} \rightarrow Z_{S(1)} \rightarrow Z_{S(0)}$$ is a tower of fibrations, so that $Z_S = \varprojlim { Z_{S(i)} }$ is a homotopy limit. Since each $j_{S(i)}$ is an equivalence by the arguments given above, we see that $j_{S}: Y_S \rightarrow Z_S$ is an equivalence, since it is a map between homotopy limits of equivalent towers.
%\end{proof}

%To complete the proof of Theorem \ref{straightthm}, we need to show that the unstraightening functor $\Un^{+}_{\phi}$ is essentially surjective (on homotopy categories).

%\begin{proposition}\label{yomik}
%Let $S$ be a simplicial set, and $\calC = \sCoNerve[S]^{op}$. For every object $X \in (\mSet)_{/S}$, there exists a (weakly) fibrant object $\calF \in (\mSet)^{\calC}$ and a Cartesian equivalence
%$X \rightarrow \Un^{+}_{S}(\calF)$. 
%\end{proposition} 

%\begin{proof}
%Choose a Cartesian equivalence $X \rightarrow Z^{\natural}$, where $p: Z \rightarrow S$ is a Cartesian fibration. Without loss of generality, we may replace $X$ by $Z^{\natural}$. 
%Choose simplicial subsets $S(\alpha) \subseteq S$, where each $S(\alpha)$ is obtained by adjoining a single nondegenerate simplex to 
%$$S(< \! \alpha) = \bigcup_{\beta < \alpha} S(\beta),$$
%provided that such a simplex exists. Let $\calC_{\alpha} = \sCoNerve[S(\alpha)]^{op}$ and $Z(\alpha) = Z \times_{S} S(\alpha)$. The objects $\calC_{< \alpha}$, $\phi_{< \alpha}$, and $Z(< \! \alpha)$ are defined similarly.

%We will construct, by induction on $\alpha$, a compatible family
%of pairs $\calF_{\alpha}: \calC^{op} \rightarrow \mSet$, $f_{\alpha}: Z(\alpha)^{\natural} \rightarrow \Un^{+}_{\phi_{\alpha}} \calF_{\alpha}$, where $f_{\alpha}$ is a Cartesian equivalence in
%$(\mSet)_{/S(\alpha)}$. Let us suppose, then, that $\calF_{< \alpha}$ and $f_{< \alpha}$ have already been constructed. Supposing that $S(< \! \alpha) \neq S$, we may write
%$$ S(\alpha) = S(< \! \alpha) \coprod_{ \bd \Delta^n } \Delta^n. $$
%Let $\calC' = \sCoNerve[\Delta^n]^{op}$ and $\calC'_0 = \sCoNerve[\bd \Delta^n]^{op}
%\subseteq \calC'$.

%By Proposition \ref{speccase}, there exists a fibrant object
%$\calG \in (\mSet)^{\calC'}$ and an equivalence
%$g: Z^{\natural} \times_{S} \Delta^n \rightarrow \Un^{+}_{\Delta^n} \calG$. Let $\calG_0 = \calG|\calC'_0$
%The objects $\Un^{+}_{\bd \Delta^n} \calG_0$ and $\Un^{+}_{\bd \Delta^n}( \calF_{< \alpha} | \calC'_0)$ are equivalent
%in the homotopy category of $(\mSet)_{/\bd \Delta^n}$ (since they are both equivalent to
%$Z^{\natural} \times_{S} \bd \Delta^n$). Proposition \ref{fulfat} implies that 
%$\calG_0$ and $\calF_{< \alpha}| \calC'_0$ are equivalent in the homotopy category
%of $(\mSet)^{\calC'_0}$. 

%We may identify $\calF_{< \alpha}$ with a map $F_{< \alpha}: S(< \! \alpha) \rightarrow \Cat_{\infty}^{op}$, and
%$\calG$ with a map $G: \Delta^n \rightarrow \Cat_{\infty}^{op}$. The above argument shows that the maps
%$F_{<\alpha} | \bd \Delta^n$ and $G | \bd \Delta^n$ are homotopic. 
%In other words, there exists a map 
%$$H_0: (\bd \Delta^n \times \Delta^1) \coprod_{ \bd \Delta^n \times \{1\} } (\Delta^n \times \{1\}) \rightarrow \Cat_{\infty}^{op}$$ such that 
%$H_0| \bd \Delta^n \times \{0\} = F_{<\alpha} | \bd \Delta^n$, $H_0| \Delta^n \times \{1\} = G$,
%and $H_0 | \{v\} \times \Delta^1$ is an equivalence in $\Cat_{\infty}$, for every vertex
%$v$ of $\Delta^n$. It follows that there exists an extension 
%$$ H: \Delta^n \times \Delta^1 \rightarrow (\Cat_{\infty}^{op})^{\natural}$$
%of $H_0$; moreover, if $n=0$ we may furthermore choose $H$ to be an equivalence (for
%example, a degenerate edge at the vertex $H_0$).

%Now define $F_{\alpha}$ to be the result of amalgamating  $F_{<\alpha}$ with $H| \Delta^n \times \{0\}$, and let $\calF_{\alpha}: \calC_{\alpha} \rightarrow \mSet$ be the associated presheaf. Finally, let $Y = \Un^{+}_{S_{\alpha}} \calF_{\alpha}$. To complete the proof, it will suffice to show that the map
%$f_{<\alpha}: Z(< \! \alpha)^{\natural} \rightarrow Y$ can be extended to an equivalence $f_{\alpha}: Z(\alpha)^{\natural} \rightarrow Y$.

%Let $f'_{<\alpha}$ denote the pullback of $f_{<\alpha}$ to $\Delta^n$:
%$$ f'_{<\alpha}: Z \times_{S} \bd \Delta^n \rightarrow Y \times_{S} \Delta^n.$$
%By hypothesis, the map $f'_{<\alpha}$ is equivalent, in the homotopy category, to the restriction
%of $g$. It follows from Proposition \ref{princex} that $f'_{<\alpha}$ extends to a map
%$$f'_{\alpha}: Z \times_{S} \Delta^n \rightarrow Y \times_{S} \Delta^n$$
%which is isomorphic to $g$ in the homotopy category $(\mSet)_{/\Delta^n}$. Then
%$f'_{\alpha}$ is an equivalence, since $g$ is an equivalence. We now define
%$f_{\alpha}$ to be the result of amalgamating $f_{<\alpha}$ and $f'_{\alpha}$. It is clear that $f_{\alpha}$ has the desired properties, so the proof is complete.
%\end{proof}

%We are now ready to complete the proof of our main result.

%\begin{proof}[Proof of Theorem \ref{straightthm}]
%Let $\phi: \sCoNerve[S] \rightarrow \calC^{op}$ be an equivalence of simplicial categories. We wish to prove that $( \St_{\phi}^{+}, \Un_{\phi}^{+})$ is a Quillen equivalence. 
%Using Proposition \ref{lesstrick}, we may reduce to the case where $\phi$ is an isomorphism. According to Lemma \ref{gottaprove0}, it will suffices to show that $(\Un^{+}_{\phi})^{\degree}$ is an equivalence of simplicial categories. Proposition \ref{fulfat} guarantees that $(\Un^{+}_{\phi})^{\degree}$ is fully faithful, and Proposition \ref{yomik} guarantees that $\Un_{\phi}^{\degree}$ is essentially surjective.
%\end{proof}

\subsection{The Relative Nerve}\label{altstr}

In \S \ref{markmodel}, we defined {\em straightening} and {\em unstraightening} functors, which
give rise to a Quillen equivalence of model categories
$$ \Adjoint{\St^{+}_{\phi}}{(\mSet)_{/S}}{(\mSet)^{\calC}}{\Un^{+}_{\phi}}$$
whenever $\phi: \sCoNerve[S] \rightarrow \calC^{op}$ is a weak equivalence of simplicial categories.
For many purposes, these constructions are unnecessarily complicated. For example, suppose that
$\calF: \calC \rightarrow \mSet$ is a (weakly) fibrant diagram, so that $\Un^{+}_{\phi}(\calF)$ is a
fibrant object of $(\mSet)_{/S}$ corresponding to a Cartesian fibration of simplicial sets $X \rightarrow S$. For every vertex $s \in S$, the fiber $X_{s}$ is an $\infty$-category which is equivalent
to $\calF( \phi(s) )$, but usually not isomorphic to $\calF( \phi(s) )$. In the special case where
$\calC$ is an ordinary category and $\phi: \sCoNerve[ \Nerve(\calC)^{op} ] \rightarrow \calC^{op}$ is the counit map, there is another version of unstraightening construction $\Un^{+}_{\phi}$ which does not share this defect. Our goal in this section is to introduce this simpler construction, which we call
the {\it marked relative nerve} $\calF \mapsto \Nerve^{+}_{\calF}(\calC)$, and to study its basic properties.

\begin{remark}
To simplify the exposition which follows, the relative nerve functor introduced below will actually be
an alternative to the {\em opposite} of the unstraightening functor
$$ \calF \mapsto ( \Un^{+}_{\phi} \calF^{op} )^{op},$$
which is a right Quillen functor from the projective model structure on
$(\mSet)^{\calC}$ to the coCartesian model structure on $\mset{ \Nerve(\calC)}$. 
\end{remark}

%We begin with a few generalities. Let $\calI$ be an ordinary category, and let $f: \Nerve(\calI) \rightarrow \Cat_{\infty}^{op}$ be a diagram. We can associate to $f$ a Cartesian fibration $X \rightarrow \Nerve(\calI)$. Let us briefly review the construction (see \S \ref{universalfib}). We first identify $f$ with a simplicial functor $F: \sCoNerve[ \Nerve(\calI) ]^{op} \rightarrow \mSet$, where $\mSet$ denotes the category of marked simplicial sets (\S \ref{twuf}). The functor $F$ is a weakly fibrant object of
%$(\mSet)^{\sCoNerve[ \Nerve(\calI) ]^{op}}$, so that after applying the unstraightening functor
%$\Un^{+}_{\Nerve(\calI)}$ we obtain a fibrant object of $(\mSet)_{/\Nerve(\calI)}$, which we can identify with the desired Cartesian fibration $p: X \rightarrow \Nerve(\calI)$. For many purposes, this construction is unnecessarily complicated. For example, the fiber of $p$ over an object $I \in \calI$ is equivalent to $f(I)$, but not isomorphic to $f(I)$. In the case where $f$ arises as the nerve of a functor $\calI \rightarrow \Set_{\Delta}^{op}$, there is an equivalent construction which is quite a bit simpler. We will present this construction in a dual form (since our primary interest is in {\em coCartesian} fibrations).

\begin{definition}\label{sulke}\index{gen}{relative nerve}\index{gen}{nerve!relative}\index{not}{NervefI@$\Nerve^{R}_{\calF}(\calC)$}
Let $\calC$ be a small category, and let $f: \calC \rightarrow \sSet$ be a functor. We define a simplicial set $\Nerve_{f}(\calC)$, the {\it nerve of $\calC$ relative to $f$}, as follows. For every nonempty finite linearly ordered set $J$, a map $\Delta^J \rightarrow \Nerve_{f}(\calC)$ consists of the following data:
\begin{itemize}
\item[$(1)$] A functor $\sigma$ from $J$ to $\calC$. 
\item[$(2)$] For every nonempty subset $J' \subseteq J$ having a maximal element $j'$, a
map $\tau(J'): \Delta^{J'} \rightarrow \calF( \sigma(j'))$. 
\item[$(3)$] For nonempty subsets $J'' \subseteq J' \subseteq J$, with maximal elements $j'' \in J''$, $j' \in J'$, the diagram $$ \xymatrix{ \Delta^{J''} \ar[r]^{\tau(J'')} \ar@{^{(}->}[d] & f( \sigma(j'') ) \ar[d] \\
\Delta^{J'} \ar[r]^{\tau(J')} & f( \sigma(j') ) }$$
is required to commute.
\end{itemize}
\end{definition}

\begin{remark}
Let $\calI$ be denote the linearly ordered set $[n]$, regarded as a category, and let
$f: \calI \rightarrow \sSet$ correspond to a composable sequence of morphisms
$\phi: X_0 \rightarrow \ldots \rightarrow X_n$.
Then $\Nerve_{f}(\calI)$ is closely related to the mapping simplex $M^{op}(\phi)$ introduced in \S \ref{funkystructure}. More precisely, there is a canonical map $\Nerve_{f}(\calI) \rightarrow M^{op}(\phi)$ compatible with the projection to $\Delta^{n}$, which induces an isomorphism on each fiber.
\end{remark}

\begin{remark}\label{staplur}
The simplicial set $\Nerve_{f}(\calC)$ of Definition \ref{sulke} depends functorially on $f$. When
$f$ takes the constant value $\Delta^0$, there is a canonical isomorphism
$\Nerve_{f}(\calC) \simeq \Nerve(\calC)$. In particular, for {\em any} functor $f$, there is a canonical map $\Nerve_{f}(\calC) \rightarrow \Nerve(\calC)$; the fiber of this map over an object $C \in \calC$ can be identified with the simpicial set $f(C)$. 
\end{remark}

\begin{remark}\label{pear1}\index{not}{FF@$\lNerve_{X}(\calC)$}
Let $\calC$ be a small $\infty$-category. The construction $f \mapsto \Nerve_{f}(\calC)$ determines a functor from $(\sSet)^{\calC}$ to $(\sSet)_{/ \Nerve(\calC)}$. This functor admits a left adjoint, which we will denote by $X \mapsto \lNerve_X(\calC)$ (the existence of this functor follows from the adjoint functor theorem). If $X \rightarrow \Nerve(\calC)$ is a left fibration, then $\calF_{X}(\calC)$ is a functor
$\calC \rightarrow \sSet$ which assigns to each $C \in \calC$ a simplicial set which is weakly equivalent to the fiber $X_{C} = X \times_{ \Nerve(\calC)} \{C\}$; this follows from Proposition \ref{kudd} below. 
\end{remark}

\begin{example}\label{spacek}
Let $\calC$ be a small category, and regard $\Nerve(\calC)$ as an object of
$(\sSet)_{/ \Nerve(\calC)}$ via the identity map. Then $\lNerve_{\Nerve(\calC)}(\calC) \in (\sSet)^{\calC}$ can be identified with the functor $C \mapsto \Nerve(\calC_{/C})$. 
\end{example}

\begin{remark}\label{pear2}
Let $g: \calC \rightarrow \calD$ be a functor between small categories and let $f: \calD \rightarrow \sSet$ be a diagram. There is a canonical isomorphism of simplicial sets
$\Nerve_{f \circ g}(\calC) \simeq \Nerve_{f}(\calD) \times_{ \Nerve(\calD) } \Nerve(\calC).$
In other words, the diagram of categories
$$ \xymatrix{ (\sSet)^{\calD} \ar[r]^{g^{\ast}} \ar[d]^{\Nerve_{\bigdot}(\calD)} & (\sSet)^{\calC} \ar[d]^{ \Nerve_{\bigdot}(\calC) } \\
(\sSet)_{/ \Nerve(\calD)} \ar[r]^{ \Nerve(g)^{\ast}} & (\sSet)_{/ \Nerve(\calC)} }$$
commutes up to canonical isomorphism. Here $g^{\ast}$ denotes the functor given by composition with
$g$, and $\Nerve(g)^{\ast}$ the functor given by pullback along the map of simplicial sets
$\Nerve(g): \Nerve(\calC) \rightarrow \Nerve(\calD)$.
\end{remark}

\begin{remark}\label{saltine}
Combining Remarks \ref{pear1} and \ref{pear2}, we deduce that for any functor
$g: \calC \rightarrow \calD$ between small categories, the diagram of left adjoints
$$ \xymatrix{ (\sSet)^{\calD} & (\sSet)^{\calC} \ar[l]_{g_{!} } \\
(\sSet)_{/ \Nerve(\calD)} \ar[u]_{ \lNerve_{\bigdot}(\calD)} & (\sSet)_{/\Nerve(\calC)} \ar[u]_{\lNerve_{\bigdot}(\calC)} \ar[l] }$$
commutes up to canonical isomorphism; here $g_{!}$ denotes the functor of left Kan extension along $g$, and the bottom arrow is the forgetful functor given by composition with
$\Nerve(g): \Nerve(\calC) \rightarrow \Nerve(\calD)$.
\end{remark}

\begin{notation}
Let $\calC$ be a small category, and let $f: \calC \rightarrow \sSet$ be a functor. We let
$f^{op}$ denote the functor $\calC \rightarrow \sSet$ described by the formula
$f^{op}(C) = f(C)^{op}$. We will use a similar notation in the case where
$f$ is a functor from $\calC$ to the category $\mSet$ of marked simplicial sets.
\end{notation}

\begin{remark}\label{scuz}
Let $\calC$ be a small category, let $S = \Nerve(\calC)^{op}$, and let $\phi: \sCoNerve[S] \rightarrow \calC^{op}$ be the counit map. For each $X \in (\sSet)_{/ \Nerve(\calC)}$, there is a canonical map
$$\alpha_{\calC}(X): \St_{\phi} X^{op} \rightarrow \lNerve_{X}(\calC)^{op}.$$
The collection of maps $\{ \alpha_{\calC}(X) \}$ is uniquely determined by the following requirements:

\begin{itemize}
\item[$(1)$] The morphism $\alpha_{\calC}(X)$ depends functorially on $X$. More precisely, suppose
given a commutative diagram of simplicial sets
$$ \xymatrix{ X \ar[rr]^{f} \ar[dr] & & Y \ar[dl] \\
& \Nerve(\calC). & }$$
Then the diagram
$$ \xymatrix{ \St_{\phi} X^{op} \ar[r]^{\alpha_{\calC}(X)} \ar[d]^{ \St_{\phi} f^{op}} & \lNerve_{X}(\calC)^{op} \ar[d]^{ \lNerve_{f}(\calC)^{op}} \\
\St_{\phi} Y^{op} \ar[r]^{ \alpha_{\calC}(Y) } & \lNerve_{Y}(\calC)^{op} }$$
commutes. 

\item[$(2)$] The transformation $\alpha_{\calC}$ depends functorially on $\calC$ in the following sense: for every functor $g: \calC \rightarrow \calD$,
if $\phi': \sCoNerve[(\Nerve \calD)^{op}] \rightarrow \calD^{op}$ denotes the counit map and
$X \in (\sSet)_{/ \Nerve(\calC)}$, then the diagram
$$ \xymatrix{ \St_{\phi} X^{op} \ar[d] \ar[r]^{g_! \alpha_{\calC}} & g_! \lNerve_{X}(\calC)^{op} \ar[d] \\
\St_{\phi'} X^{op} \ar[r]^{ \alpha_{\calD}} & \lNerve_{X}(\calD)^{op} }$$
commutes, where the vertical arrows are the isomorphisms provided by Remark \ref{saltine} and
Proposition \ref{straightchange}.

\item[$(3)$] Let $\calC$ be the category associated to a partially ordered set $P$, and let
$X = \Nerve(\calC)$, regarded as an object of
$(\sSet)_{/ \Nerve(\calC)}$ via the identity map. Then $(\St_{\phi} X^{op}) \in (\sSet)^{\calC}$
can be identified with the functor $p \mapsto \Nerve X_{p}$, where for each $p \in P$ we let
$X_{p}$ denote the collection of nonempty finite chains in $P$ having largest element $p$. Similarly, Example \ref{spacek} allows us to identify $\lNerve_{X}(\calC) \in (\sSet)^{\calC}$ with the functor $p \mapsto \Nerve \{ q \in P: q \leq p \}$. The map
$\alpha_{\calC}(X): (\St_{\phi} X^{op}) \rightarrow \lNerve_{X}(\calC)^{op}$ is induced by the map
of partially ordered sets $X_p \rightarrow \{ q \in P: q \leq p \}$ which carries every chain to its
smallest element.
\end{itemize}

To see that the collection of maps $\{ \alpha_{\calC}(X) \}_{ X \in (\sSet)_{/ \Nerve(\calC)}}$ is determined by these properties, we first note that because the functors $\St_{\phi}$ and $\lNerve_{\bigdot}(\calC)$ commute with colimits, any natural transformation $\beta_{\calC}: \St_{\phi}(\bigdot^{op}) \rightarrow \lNerve_{\bigdot}(\calC)^{op}$ is determined by its values $\beta_{\calC}(X): \St_{\phi}(X^{op}) \rightarrow \lNerve_{X}(\calC)^{op}$ in the case where $X = \Delta^n$ is a simplex. In this case, any map $X \rightarrow \Nerve \calC$ factors through the isomorphism $X \simeq \Nerve [n]$, so we can use property $(2)$ to reduce to the case where the category $\calC$ is a partially ordered set and the map $X \rightarrow \Nerve(\calC)$ is an isomorphism. The behavior of the natural transformation $\alpha_{\calC}$ is then dictated by property $(3)$. This proves the uniqueness of the natural transformations $\alpha_{\calC}$; the existence follows by a similar argument.
\end{remark}

The following result summarizes some of the basic properties of the relative nerve functor:

\begin{lemma}\label{sulken2}
Let $\calI$ be a category and
let $\alpha: f \rightarrow f'$ be a natural transformation of functors
$f,f': \calC \rightarrow \sSet$.
\begin{itemize}
\item[$(1)$] Suppose that, for each $I \in \calC$, the map $\alpha(I): f(I) \rightarrow f'(I)$ is
an inner fibration of simplicial sets. Then the induced map $\Nerve_{f}(\calC) \rightarrow
\Nerve_{f'}(\calC)$ is an inner fibration.
\item[$(2)$] Suppose that, for each $I \in \calI$, the simplicial set $f(I)$ is an $\infty$-category. Then $\Nerve_{f}(\calC)$ is an $\infty$-category.
\item[$(3)$] Suppose that, for each $I \in \calC$, the map $\alpha(I): f(I) \rightarrow f'(I)$ is
a categorical fibration of $\infty$-categories. Then the induced map
$\Nerve_{f}(\calC) \rightarrow \Nerve_{f'}(\calC)$ is a categorical fibration of $\infty$-categories.
\end{itemize}
\end{lemma}

\begin{proof}
Consider a commutative diagram
$$ \xymatrix{ \Lambda^n_i \ar[r] \ar@{^{(}->}[d] & \Nerve_{f}(\calI) \ar[d]^{p} \\
\Delta^n \ar@{-->}[ur] \ar[r] & \Nerve_{f'}(\calC), }$$
and let $I$ be the image of $\{n\} \subseteq \Delta^n$ under the
bottom map. If $0 \leq i < n$, then the lifting problem depicted in the diagram above is equivalent to the existence of a dotted arrow in an associated diagram
$$ \xymatrix{ \Lambda^n_i \ar[r]^{g} \ar@{^{(}->}[d] & f(I) \ar[d]^{\alpha(I)} \\
\Delta^n \ar@{-->}[ur] \ar[r] & f'(I).}$$
If $\alpha(I)$ is an inner fibration and $0 < i < n$, then we conclude that this lifting problem admits a solution. This proves $(1)$. To prove $(2)$, we apply $(1)$ in the special case where $f'$ is the constant functor taking the value $\Delta^0$. It follows that $\Nerve_{f}(\calC) \rightarrow \Nerve(\calC)$ is an inner fibration, so that $\Nerve_{f}(\calC)$ is an $\infty$-category.

We now prove $(3)$. According to Corollary \ref{gottaput}, an inner fibration
$\calD \rightarrow \calE$ of $\infty$-categories is a categorical fibration if and only if the following condition is satisfied:
\begin{itemize}
\item[$(\ast)$] For every equivalence $e: E \rightarrow E'$ in $\calE$, and every
object $D \in \calD$ lifting $E$, there exists an equivalence $\overline{e}: D \rightarrow D'$
in $\calD$ lifting $e$.
\end{itemize}

We can identify equivalences in $\Nerve_{f'}(\calC)$ with triples
$(g: I \rightarrow I', X, e: X' \rightarrow Y)$ where $g$ is an isomorphism in $\calC$, $X$ is an object
of $f'(I)$, $X'$ is the image of $X$ in $f'(I')$, and $e: X' \rightarrow Y$ is an equivalence in
$f'(I')$. Given a lifting $\overline{X}$ of $X$ to $f(I)$, we can apply the assumption that
$\alpha(I')$ is a categorical fibration (and Corollary \ref{gottaput}) to lift $e$ to an equivalence $\overline{e}: \overline{X}' \rightarrow \overline{Y}$ in $f(I')$. This produces the desired equivalence
$(g: I \rightarrow I', \overline{X}, \overline{e}: \overline{X}' \rightarrow \overline{Y})$ in
$\Nerve_{f}(\calC)$.
\end{proof}

We now introduce a slightly more elaborate version of the relative nerve construction.

\begin{definition}\index{not}{Nervefplus@$\Nerve^{+}_{\calF}(\calC)$}
Let $\calC$ be a small category and $\calF: \calC \rightarrow \mSet$ a functor. We let
$\Nerve^{+}_{\calF}(\calC)$ denote the marked simplicial set
$( \Nerve_{f}(\calC), M)$, where $f$ denotes the composition
$\calC \stackrel{\calF}{\rightarrow} \mSet \rightarrow \sSet$ and
$M$ denotes the collection of all edges $\overline{e}$ of $\Nerve_{f}(\calC)$ with the following property:
if $e: C \rightarrow C'$ is the image of $\overline{e}$ in $\Nerve(\calC)$, and $\sigma$ denotes the
edge of $f(C')$ determined by $\overline{e}$, then $\sigma$ is a marked edge of $\calF(C')$.
We will refer to $\Nerve^{+}_{\calF}(\calC)$ as the {\it marked relative nerve} functor.\index{gen}{relative nerve!marked}\index{gen}{nerve!marked relative}\index{gen}{marked relative nerve}
\end{definition}

\begin{remark}
Let $\calC$ be a small category. We will regard the construction $\calF \mapsto \Nerve^{+}_{\calF}(\calC)$ as determining a functor from $(\mSet)^{\calC}$ to $\mset{\Nerve(\calC) }$ (see Remark \ref{staplur}). This functor admits a left adjoint, which we will denote by $\overline{X} \mapsto \lNerve^{+}_{\overline{X}}(\calC)$.\index{not}{FFplus@$\lNerve^{+}_{\overline{X}}(\calC)$}
\end{remark}

\begin{remark}\label{saltine2}
Remark \ref{saltine} has an evident analogue for the functors $\lNerve^{+}$: 
for any functor $g: \calC \rightarrow \calD$ between small categories, the diagram of left adjoints
$$ \xymatrix{ (\mSet)^{\calD} & (\mSet)^{\calC} \ar[l]_{g_{!} } \\
\mset{\Nerve(\calD)} \ar[u]_{ \lNerve^{+}_{\bigdot}(\calD)} & \mset{\calC} \ar[u]_{\lNerve^{+}_{\bigdot}(\calC)} \ar[l] }$$
commutes up to canonical isomorphism.
\end{remark}

\begin{lemma}\label{coughup}
Let $\calC$ be a small category. Then:
\begin{itemize}
\item[$(1)$] The functor $X \mapsto \lNerve_{X}(\calC)$ carries
cofibrations in $(\sSet)_{/ \Nerve(\calC)}$
to cofibrations in $(\sSet)^{\calC}$ $($with respect to the projective model structure$)$.
\item[$(2)$] The functor $\overline{X} \mapsto \lNerve^{+}_{\overline{X}}(\calC)$ carries
cofibrations $($with respect to the coCartesian monoidal structure on  $\mset{\Nerve(\calC)}${}$)$ to cofibrations in $(\mSet)^{\calC}$ $($with respect to the projective model structure$)$.
\end{itemize}
\end{lemma}

\begin{proof}
We will give the proof of $(2)$; the proof of $(1)$ is similar. It will suffice to show that the right adjoint functor $\Nerve^{+}_{\bigdot}(\calC): (\mSet)^{\calC} \rightarrow \mSet{ \Nerve(\calC) }$ preserves trivial fibrations. Let $\calF \rightarrow \calF'$ be a trivial fibration in $(\mSet)^{\calC}$ with respect to the projective model structure, so that for each $C \in \calC$ the induced map $\calF(C) \rightarrow \calF'(C)$ is a trivial fibration of marked simplicial sets. We wish to prove that the induced map
$\Nerve^{+}_{\calF}(\calC) \rightarrow \Nerve^{+}_{\calF'}(\calC)$ is also a trivial fibration of marked simplicial sets. Let $f$ denote the composition $\calC \stackrel{\calF}{\rightarrow} \mSet \rightarrow \sSet$, and let $f'$ be defined likewise. We must verify two things:
\begin{itemize}
\item[$(1)$] Every lifting problem of the form
$$ \xymatrix{ \bd \Delta^n \ar[r] \ar@{^{(}->}[d] & \Nerve_{f}(\calC) \ar[d] \\
\Delta^n \ar[r]^{u} & \Nerve_{f'}(\calC) }$$
admits a solution. Let $C \in \calC$ denote the image of the final vertex of $\Delta^n$ under the map $u$.
Then it suffices to solve a lifting problem of the form
$$ \xymatrix{ \bd \Delta^n \ar[r] \ar@{^{(}->}[d] & f(C) \ar[d] \\
\Delta^n \ar[r] & f'(C), }$$
which is possible since the right vertical map is a trivial fibration of simplicial sets.

\item[$(2)$] If $\overline{e}$ is an edge of $\Nerve^{+}_{\calF}(\calC)$ whose image $\overline{e}'$ in
$\Nerve^{+}_{\calF'}(\calC)$ is marked, then $\overline{e}$ is itself marked. Let
$e: C \rightarrow C'$ be the image of $\overline{e}$ in $\Nerve(\calC)$, and let
$\sigma$ denote the edge of $\calF(C')$ determined by $\overline{e}$. Since
$\overline{e}'$ is a marked edge of $\Nerve^{+}_{\calF'}(\calC)$, the image of
$\sigma$ in $\calF'(C')$ is marked. Since the map $\calF(C') \rightarrow \calF'(C')$ is a trivial fibration of marked simplicial sets, we deduce that $\sigma$ is a marked edge of $\calF(C')$, so that
$\overline{e}$ is a marked edge of $\Nerve^{+}_{\calF}(\calC)$ as desired.
\end{itemize}
\end{proof}

\begin{remark}\label{scuz2}
Let $\calC$ be a small category, let $S = \Nerve(\calC)^{op}$, and let $\phi: \sCoNerve[S] \rightarrow \calC^{op}$ be the counit map. For every $\overline{X} = (X,M) \in \mset{\Nerve(\calC)}$, the 
morphism $\alpha_{\calC}(X): \St_{\phi}(X^{op}) \rightarrow \lNerve_{X}(\calC)^{op}$ of Remark \ref{scuz} induces a natural transformation $\St^{+}_{\phi} \overline{X}^{op} \rightarrow \lNerve^{+}_{ \overline{X}}(\calC)^{op}$, which we will denote by $\alpha^{+}_{\calC}( \overline{X})$. We will regard the collection
of morphisms $\{ \alpha^{+}_{\calC}( \overline{X}) \}_{ \overline{X} \in \mset{\Nerve(\calC)}}$ as determining a natural transformation of functors 
$$\alpha_{\calC}: \St^{+}_{\phi}( \bigdot^{op}) \rightarrow \lNerve^{+}_{\bigdot}(\calC)^{op}.$$
\end{remark}

\begin{lemma}\label{standrum}
Let $\calC$ be small category, let $S = \Nerve(\calC)^{op}$, and let $\phi: \sCoNerve[S] \rightarrow \calC^{op}$ be the counit map, and let $C \in \calC$ be an object. Then:
\begin{itemize}
\item[$(1)$] For every $X \in (\sSet)_{/ \Nerve(\calC)}$, the map 
$\alpha_{\calC}(X): \St_{\phi}(X^{op}) \rightarrow \lNerve_{X}(\calC)^{op}$ of Remark \ref{scuz} induces
a weak homotopy equivalence of simplicial sets $\St_{\phi}(X^{op})(C) \rightarrow \lNerve_{X}(\calC)(C)^{op}$.
\item[$(2)$] For every $\overline{X} \in \mset{\Nerve(\calC)}$, the map
$\alpha^{+}_{\calC}( \overline{X}): \St^{+}_{\phi}( \overline{X}^{op}) \rightarrow \lNerve^{+}_{\overline{X}}(\calC)^{op}$ of Remark \ref{scuz2} induces a Cartesian equivalence $\St^{+}_{\phi}( \overline{X}^{op})(C) \rightarrow \lNerve^{+}_{\overline{X}}(\calC)(C)^{op}$. 
\end{itemize}
\end{lemma}

\begin{proof}
We will give the proof of $(2)$; the proof of $(1)$ is similar but easier. Let us say that an object
$\overline{X} \in \mset{\Nerve(\calC)}$ is {\it good} if the map $\alpha^{+}_{\calC}( \overline{X})$ is a weak equivalence. We wish to prove that every object $\overline{X} = (X,M) \in \mset{ \Nerve(\calC)}$ is good. The proof proceeds in several steps.
\begin{itemize}
\item[$(A)$] Since the functors $\St^{+}_{\phi}$ and $\lNerve^{+}_{\bigdot}(\calC)$ both commute with filtered colimits, the collection of good objects of $\mset{ \Nerve(\calC)}$ is stable under filtered colimits.
We may therefore reduce to the case where the simplicial set $X$ has only finitely many nondegenerate simplices.
\item[$(B)$] Suppose given a pushout diagram
$$ \xymatrix{ \overline{X} \ar[r]^{f} \ar[d]^{g} & \overline{X}' \ar[d] \\
\overline{Y} \ar[r] & \overline{Y}' }$$
in the category $\mset{ \Nerve(\calC) }$. Suppose that either $f$ or $g$ is a cofibration, and that
the objects $\overline{X}, \overline{X}'$, and $\overline{Y}$ are good. Then $\overline{Y}'$ is good.
This follows from the fact that the functors $\St^{+}_{\phi}$ and $\lNerve^{+}_{\bigdot}(\calC)$ preserve cofibrations (Proposition \ref{cougherup} and Lemma \ref{coughup}), together with the observation that the projective model structure on $(\mSet)^{\calC}$ is left proper. 
\item[$(C)$] Suppose that $X \simeq \Delta^n$ for $n \leq 1$. In this case,
the map $\alpha^{+}_{\calC}( \overline{X})$ is an isomorphism (by direct calculation), so that $\overline{X}$ is good.
\item[$(D)$] We now work by induction on the number of nondegenerate marked edges of $\overline{X}$.
If this number is nonzero, then there exists a pushout diagram
$$ \xymatrix{ (\Delta^1)^{\flat} \ar[r] \ar[d] & (\Delta^1)^{\sharp} \ar[d] \\
\overline{Y} \ar[r] & \overline{X} }$$
where $\overline{Y}$ has fewer nondegenerate marked edges than $\overline{X}$, so that
$\overline{Y}$ is good by the inductive hypothesis. The marked simplicial sets
$(\Delta^1)^{\flat}$ and $(\Delta^1)^{\sharp}$ are good by virtue of $(C)$, so that $(B)$ implies that
$\overline{X}$ is good. We may therefore reduce to the case where $\overline{X}$ contains no nondegenerate marked edges, so that $\overline{X} \simeq X^{\flat}$.
\item[$(E)$] We now argue by induction on the dimension $n$ of $X$ and the number of nondegenerate $n$-simplices of $X$. If $X$ is empty, there is nothing to prove; otherwise, we have a pushout diagram
$$ \xymatrix{ \bd \Delta^n \ar[r] \ar[d] & \Delta^n \ar[d] \\
Y \ar[r] & X. }$$
The inductive hypothesis implies that $( \bd \Delta^n)^{\flat}$ and $Y^{\flat}$ are good.
Invoking step $(B)$, we can reduce to the case where $X$ is an $n$-simplex. In view of
$(C)$, we may assume that $n \geq 2$.

Let $Z = \Delta^{ \{0,1\} } \coprod_{ \{ 1\} } \Delta^{ \{1,2\} } \coprod_{ \{1\} } \ldots
\coprod_{ \{n-1\}} \Delta^{ \{n-1, n \}}$ so that $Z \subseteq X$ is an inner anodyne inclusion.
We have a commutative diagram
$$ \xymatrix{ \St^{+}_{\phi} (Z^{op})^{\flat} \ar[r]^{u} \ar[d]^{v} & \St^{+}_{\phi}(X^{op})^{\flat} \ar[d] \\
\lNerve^{+}_{ Z^{\flat}}(\calC)^{op} \ar[r]^{w} & \lNerve^{+}_{ X^{\flat}}(\calC)^{op}. }$$
The inductive hypothesis implies that $v$ is a weak equivalence, and Proposition \ref{spec2} implies that $u$ is a weak equivalence. To complete the proof, it will suffice to show that
$w$ is a weak equivalence. 

\item[$(F)$] The map $X \rightarrow \Nerve(\calC)$ factors as a composition
$$ \Delta^n \simeq \Nerve( [n] ) \stackrel{g}{\rightarrow} \Nerve(\calC).$$
Using Remark \ref{saltine2} (together with the fact that the left
Kan extension functor $g_!$ preserves weak equivalences between projectively cofibrant objects),
we can reduce to the case where $\calC = [n]$ and the map $X \rightarrow \Nerve(\calC)$ is an isomorphism.

\item[$(G)$] Fix an object $i \in [n]$. A direct computation shows that the map
$\lNerve^{+}_{Z^{\flat}}(\calC)(i) \rightarrow \lNerve^{+}_{X^{\flat}}(\calC)(i)$ can be identified with the inclusion 
$$ (  \Delta^{ \{0,1\} } \coprod_{ \{ 1\} } \Delta^{ \{1,2\} } \coprod_{ \{1\} } \ldots
\coprod_{ \{i-1\}} \Delta^{ \{i-1, i \}})^{op, \flat} \subseteq (\Delta^{i})^{op, \flat}.$$
This inclusion is marked anodyne, and therefore an equivalence of marked simplicial sets as desired.
\end{itemize}
\end{proof}

\begin{proposition}\label{kudd}
Let $\calC$ be a small category. Then:
\begin{itemize}
\item[$(1)$] The functors $\lNerve_{\bigdot}(\calC)$ and $\Nerve_{\bigdot}(\calC)$ determine
a Quillen equivalence between $( \sSet)_{/ \Nerve(\calC)}$ $($endowed with the covariant model structure$)$
and $(\sSet)^{\calC}$ $($endowed with the projective model structure$)$.
\item[$(2)$] The functors $\lNerve^{+}_{\bigdot}(\calC)$ and $\Nerve^{+}_{\bigdot}(\calC)$ determine
a Quillen equivalence between $\mset{ \Nerve(\calC)}$ $($endowed with the coCartesian model structure) and $(\mSet)^{\calC}$ $($endowed with the projective model structure$)$.
\end{itemize}
\end{proposition}

\begin{proof}
We will give the proof of $(2)$; the proof of $(1)$ is similar but easier. We first show that
the adjoint pair $( \lNerve^{+}_{\bigdot}(\calC), \Nerve^{+}_{\bigdot}(\calC) )$ is a Quillen adjunction.
It will suffice to show that the functor $\lNerve^{+}_{\bigdot}(\calC)$ preserves cofibrations and weak equivalences. The case of cofibrations follows from Lemma \ref{coughup}, and the case of weak
equivalences from Lemma \ref{standrum} and Corollary \ref{spek6}. To prove that
$( \lNerve^{+}_{\bigdot}(\calC), \Nerve^{+}_{\bigdot}(\calC) )$ is a Quillen equivalence, it will suffice to show that the left derived functor $L \lNerve^{+}_{\bigdot}(\calC)$ induces an equivalence from the homotopy category $\h{ \mset{\Nerve(\calC)}}$ to the homotopy category $\h{ (\mSet)^{\calC}}$. In view of Lemma \ref{standrum}, it will suffice to prove an analogous result for the straightening functor
$\St^{+}_{\phi}$, where $\phi$ denotes the counit map
$\sCoNerve[ \Nerve(\calC)^{op}] \rightarrow \calC^{op}$. We now invoke Theorem \ref{straightthm}.
\end{proof}

\begin{corollary}
Let $\calC$ be a small category, and let $\alpha: f \rightarrow f'$ be a natural transformation of functors
$f,f': \calC \rightarrow \sSet$. Suppose that, for each $C \in \calC$, the induced map
$f(C) \rightarrow f'(C)$ is a Kan fibration. Then the induced map
$\Nerve_{f}(\calC) \rightarrow \Nerve_{f'}(\calC)$ is a covariant fibration in
$(\sSet)_{/ \Nerve(\calC)}$. In particular, if each $f(C)$ is Kan complex, then the map
$\Nerve_{f}(\calC) \rightarrow \Nerve(\calC)$ is a left fibration of simplicial sets.
\end{corollary}

\begin{corollary}\label{sandcor}
Let $\calC$ be a small category and $\calF: \calC \rightarrow \mSet$ a fibrant object of
$(\mSet)^{\calC}$. Let $S = \Nerve(\calC)$, and let $\phi: \sCoNerve[S^{op}] \rightarrow \calC^{op}$ denote the counit map. Then the natural transformation $\alpha_{\calC}^{+}$ of Remark \ref{scuz2}
induces a weak equivalence
$\Nerve_{\calF}(\calC)^{op} \rightarrow (\Un^{+}_{\phi} \calF^{op})$
$($with respect to the Cartesian model structure on $\mset{ S^{op}}${}$)$.
\end{corollary}

\begin{proof}
It suffices to show that $\alpha_{\calC}^{+}$ induces an isomorphism of right derived functors
$R \Nerve_{\bigdot}(\calC)^{op} \rightarrow R (\Un^{+}_{\phi} \bigdot^{op})$, which follows immediately from Lemma \ref{standrum}. 
\end{proof}

\begin{proposition}\label{sulken}
Let $\calC$ be a category, and let $f: \calC \rightarrow \sSet$ be a functor such that
$f(C)$ is an $\infty$-category for each $C \in \calC$. Then:
\begin{itemize}
\item[$(1)$] The projection $p: \Nerve_{f}(\calC) \rightarrow \Nerve(\calC)$ is a coCartesian fibration of simplicial sets. 

\item[$(2)$] Let $e$ be an edge of $\Nerve_{f}(\calC)$, covering a morphism $C \rightarrow C'$
in $\calC$. Then $e$ is $p$-coCartesian if and only if the corresponding edge of $f(C')$ is an equivalence.

\item[$(3)$] The coCartesian fibration $p$ is associated to the functor $\Nerve(f): \Nerve(\calC) \rightarrow \Cat_{\infty}$ $($see \S \ref{universalfib}$)$. 
\end{itemize}
\end{proposition}

\begin{proof}
Let $\calF: \calC \rightarrow \mSet$ be the functor described by the formula
$\calF(C) = f(C)^{\natural}$. Then $\calF$ is a projectively fibrant object of $(\mSet)^{\calC}$.
Invoking Proposition \ref{kudd}, we deduce that $\Nerve^{+}_{\calF}(\calC)$ is a fibrant object
of $\mset{ \Nerve(\calC)}$. Invoking Proposition \ref{markedfibrant}, we deduce that the underlying
map $p: \Nerve_{f}(\calC) \rightarrow \Nerve(\calC)$ is a coCartesian fibration of simplicial sets, and that
the $p$-coCartesian morphisms of $\Nerve_{f}(\calC)$ are precisely the marked wedges of
$\Nerve^{+}_{\calF}(\calC)$. This proves $(1)$ and $(2)$. To prove $(3)$, we let
$S = \Nerve(\calC)$ and $\phi: \sCoNerve[S]^{op} \rightarrow \calC^{op}$ be the counit map.
By definition, a coCartesian fibration $X \rightarrow \Nerve(\calC)$ is associated to $f$ if and only if
it is equivalent to $(\Un_{\phi} f^{op})^{op}$; the desired equivalence is furnished by Corollary \ref{sandcor}.

%Lemma \ref{sulken2} implies that $p$ is an inner fibration.
%Consider a commutative diagram
%$$ \xymatrix{ \Lambda^n_0 \ar[r] \ar@{^{(}->}[d] & \Nerve_{f}(\calI) \ar[d]^{p} \\
%\Delta^n \ar@{-->}[ur] \ar[r] & \Nerve(\calI), }$$
%and let $I$ be the image of $\{n\} \subseteq \Delta^n$ under the
%bottom map. Then the lifting problem depicted in the diagram above is equivalent to the existence of a dotted arrow in an associated diagram
%$$ \xymatrix{ \Lambda^n_0 \ar[r]^{g} \ar@{^{(}->}[d] & f(I) \\
%\Delta^n. \ar@{-->}[ur] & }$$
%If $n > 1$, an extension exists provided
%that $g$ carries the initial edge of $\Lambda^n_0$ to an equivalence in $f(I)$. This proves
%the ``if'' direction of $(2)$.

%We next observe that for every morphism $h: I \rightarrow I'$ in $\calI$ and every object $x \in f(I)$, there exists morphism $\overline{h}: x \rightarrow x'$ in $\Nerve_{f}(\calI)$ which lifts
%$h$ and classifies an equivalence in $f(I')$; in fact, we can choose $\overline{h}$ to correspond to the identity map from $h_{!}(x) \in f(I')$ to itself. The above argument shows that 
%$\overline{h}$ is $p$-coCartesian. This completes the proof of $(1)$. The ``only if'' direction of $(2)$ now follows from the fact that $p$-coCartesian lifts of morphisms in $\Nerve(\calI)$ are unique up to equivalence.

%To prove $(3)$, we use the formalism of marked simplicial sets (see \S \ref{twuf}). 
%Let $f^{op}: \Nerve(\calI) \rightarrow \mSet$ denote the functor given by the formula 
%$f^{op}(I) = ( f(I)^{op} )^{\natural}$, and let $Z$ be the result of applying the unstraightening functor 
%$\Un^{+}_{ \Nerve(\calI)^{op}}$ to $f^{op}$. Then $Z$ is a fibrant object of $(\mSet)_{/\Nerve(\calI)^{op}}$, and is therefore of the form $(X^{op})^{\natural}$, where
%$q: X \rightarrow \Nerve(\calI)$ is a coCartesian fibration. Unwinding the definitions, we see
%that there is a commutative diagram
%$$ \xymatrix{ \Nerve_{f}(\calI) \ar[dr]^{p} \ar[rr]^{r} & & X \ar[dl]^{q} \\
%& \Nerve(\calI) & }$$
%where $r$ carries $p$-coCartesian edges to $q$-coCartesian edges. It follows from 
%Theorem \ref{straightthm} (applied over a point) that $r$ induces an equivalence
%of $\infty$-categories after passing to the fiber over each object $I \in \calI$. 
%The desired result now follows from Corollary \ref{usefir}.
\end{proof}

