\section{Accessible $\infty$-Categories}\label{c5s5}

\setcounter{theorem}{0}

Many of the categories which commonly arise in mathematics can be realized as categories of $\Ind$-objects. For example, the category of
sets is equivalent to $\Ind(\calC)$, where $\calC$ is the category of finite sets; the category
of rings is equivalent to $\Ind(\calC)$, where $\calC$ is the category of finitely presented rings.
The theory of {\it accessible} categories is an axiomatization of this situation. We refer the reader to \cite{adamek} for an exposition of the theory of accessible categories. In this section, we will describe an $\infty$-categorical generalization of the theory of accessible categories.

We will begin in \S \ref{locbrend} by introducing the notion of a {\em locally small} $\infty$-category. A locally small $\infty$-category $\calC$ need not be small, but has small morphism spaces
$\bHom_{\calC}(X,Y)$ for any fixed pair of objects $X,Y \in \calC$. This is analogous to the usual set-theoretic conventions taken in category theory: one allows categories which have a proper class of objects, but requires that morphisms between any pair of objects form a {\em set}. 

In \S \ref{accessible}, we will introduce the definition of an {\em accessible} $\infty$-category.
An $\infty$-category $\calC$ is accessible if it is locally small and has a good supply of filtered colimits and compact objects. Equivalently, $\calC$ is accessible if it is equivalent to
$\Ind_{\kappa}(\calC^{0})$, for some small $\infty$-category $\calC^0$ and some regular cardinal $\kappa$ (Proposition \ref{clear}).

The theory of accessible $\infty$-categories will play an important technical role throughout the remainder of this book. To understand the usefulness of the hypothesis of accessibility, let us consider the following example. Suppose that $\calC$ is an ordinary category, $F: \calC \rightarrow \Set$ is a functor, and we would like to prove that $F$ is representable by an object $C \in \calC$. The functor
$F$ determines a category $\widetilde{\calC} = \{ (C, \eta): C \in \calC, \eta \in F(C) \}$, which
is fibered over $\calC$ in sets. We would like to prove that $\widetilde{\calC}$ is equivalent
to $\calC_{/C}$, for some $C \in \calC$. The object $C$ can then be characterized as the colimit
of the diagram $p: \widetilde{\calC} \rightarrow \calC$. If $\calC$ admits colimits, then we can attempt to construct $C$ by forming the colimit $\varinjlim(p)$. 

We now encounter a set-theoretic difficulty. Suppose that we try to ensure the existence of $\varinjlim(p)$ by assuming that $\calC$ admits {\em all} small colimits. In this case, it is not reasonable to expect $\calC$ itself to be small. The category $\widetilde{\calC}$ is roughly the same size as $\calC$ (or larger), so our assumption will not allow us to construct $\varinjlim(p)$. On
the other hand, if we assume $\calC$ and $\widetilde{\calC}$ are small, then it is not reasonable
to expect $\calC$ to admit colimits of arbitrary small diagrams.

An accessibility hypothesis can be used to circumvent the difficulty described above. 
An accessible category $\calC$ is generally not small, but is ``controlled'' by a small subcategory $\calC^{0} \subseteq \calC$: it therefore enjoys the best features of both the ``small'' and ``large'' worlds. More precisely, the fiber product $\widetilde{\calC} \times_{\calC} \calC^{0}$ is small enough that we might expect the colimit $\varinjlim(p | \widetilde{\calC} \times_{\calC} \calC^{0})$ to exist on general grounds, yet large enough to expect a natural isomorphism
$$ \varinjlim(p) \simeq \varinjlim( p | \widetilde{\calC} \times_{\calC} \calC^{0}).$$
We refer the reader to \S \ref{aftt} for a detailed account of this argument, which we will use to prove an $\infty$-categorical version of the adjoint functor theorem.

The discussion above can be summarized as follows: the theory of accessible $\infty$-categories is a tool which allows us to manipulate large $\infty$-categories as if they were small, without fear of encountering any set-theoretic paradoxes. This theory is quite useful because the condition of accessibility is very robust: the class of accessible $\infty$-categories is stable under most of the basic constructions of higher category theory. To illustrate this, we will prove the following results:

\begin{itemize}
\item[$(1)$] A small $\infty$-category $\calC$ is accessible if and only if $\calC$ is idempotent complete (\S \ref{accessidem}). 
\item[$(2)$] If $\calC$ is an accessible $\infty$-category and $K$ is a small simplicial set, then
$\Fun(K,\calC)$ is accessible (\S \ref{accessfunk}).
\item[$(3)$] If $\calC$ is an accessible $\infty$-category and $p:K \rightarrow \calC$ is a small diagram, then $\calC_{p/}$ and $\calC_{/p}$ are accessible (\S \ref{accessprime} and \S \ref{accessfiber}). 
\item[$(4)$] The collection of accessible $\infty$-categories is stable under homotopy fiber products (\S \ref{accessfiber}).
\end{itemize}

We will apply these facts in \S \ref{accessstable} to deduce a miscellany of further stability results, which will be needed throughout \S \ref{c5s6} and \S \ref{chap6}.

\subsection{Locally Small $\infty$-Categories}\label{locbrend}

In mathematical practice, it is very common to encounter categories $\calC$ for which the collection of all objects is large (too big to be form a set), but the collection of morphisms $\Hom_{\calC}(X,Y)$ is small for every $X,Y \in \calC$. The same situation arises frequently in higher category theory.
However, it is a slightly trickier to describe, because the formalism of $\infty$-categories blurs the distinction between objects and morphisms. Nevertheless, there is an adequate notion of ``local smallness'' in the $\infty$-categorical setting, which we will describe in this section.

Our first step is to give a characterization of the class of essentially small $\infty$-categories. We will need the following lemma.

\begin{lemma}\label{soirt}
Let $\calC$ be a simplicial category, $n$ a positive integer, and
$f_0: \bd \Delta^n \rightarrow \sNerve(\calC)$ a map. Let
$X = f_0 ( \{0 \})$, $Y = f_0 ( \{n\} )$, and $g_0$ denote the induced map
$$\bd (\Delta^1)^{n-1} \rightarrow \bHom_{\calC}(X,Y).$$
Let $f,f': \Delta^n \rightarrow \sNerve(\calC)$ be extensions of $f_0$, and
$g,g': (\Delta^1)^{n-1} \rightarrow \bHom_{\calC}(X,Y)$ the corresponding extensions
of $g_0$. The following conditions are equivalent:
\begin{itemize}
\item[$(1)$] The maps $f$ and $f'$ are homotopic relative to $\bd \Delta^n$.
\item[$(2)$] The maps $g$ and $g'$ are homotopic relative to $\bd (\Delta^1)^{n-1}$.
\end{itemize}
\end{lemma}

\begin{proof}
It is not difficult to show that $(1)$ is equivalent to the assertion that $f$ and $f'$ are left homotopic in the model category $(\sSet)_{\bd \Delta^n /}$ (with the Joyal model structure), and that $(2)$ equivalent to the assertion that $\sCoNerve[f]$ and $\sCoNerve[f']$ are left homotopic in the model category $(\sCat)_{ \sCoNerve[ \bd \Delta^n] /}$. We now invoke the Quillen equivalence of Theorem \ref{biggier} to complete the proof.
\end{proof}

\begin{proposition}\label{grapeape}
Let $\calC$ be an $\infty$-category, and $\kappa$ an uncountable regular cardinal.
The following conditions are equivalent:
\begin{itemize}
\item[$(1)$] The collection of equivalence classes of objects of $\calC$ is $\kappa$-small, and for every morphism $f: C \rightarrow D$ in $\calC$ and every $n \geq 0$, the homotopy set
$\pi_i( \Hom^{\rght}_{\calC}(C,D), f)$ is $\kappa$-small. 

\item[$(2)$] If $\calC' \subseteq \calC$ is a minimal model for $\calC$, then $\calC'$ is $\kappa$-small.

\item[$(3)$] There exists a $\kappa$-small $\infty$-category $\calC'$ and an equivalence
$\calC' \rightarrow \calC$ of $\infty$-categories.

\item[$(4)$] There exists a $\kappa$-small simplicial set $K$ and a categorical equivalences
$K \rightarrow \calC$.

\item[$(5)$] The $\infty$-category $\calC$ is $\kappa$-compact, when regarded as an object of
$\Cat_{\infty}$.

\end{itemize}
\end{proposition}

\begin{proof}
We begin by proving that $(1) \Rightarrow (2)$. Without loss of generality, we may suppose that
$\calC = \sNerve(\calD)$, where $\calD$ is a topological category. Let $\calC' \subseteq \calC$ be a minimal model for $\calC$. We will prove by induction on $n \geq 0$ that the set
$\Hom_{\sSet}(\Delta^n, \calC')$ is $\kappa$-small. If $n=0$, this reduces to the assertion that
$\calC$ has fewer than $\kappa$ equivalence classes of objects. Suppose therefore that $n > 0$. By the inductive hypothesis, the set $\Hom_{\sSet}(\bd \Delta^n, \calC')$ is $\kappa$-small.
Since $\kappa$ is regular, it will suffice to prove that for each map $f_0: \bd \Delta^n \rightarrow \calC'$, the set $S = \{ f \in \Hom_{\sSet}(\Delta^n, \calC'): f|\bd \Delta^n = f_0 \}$ is $\kappa$-small.
Let $C = f_0(\{0\})$, $D = f_0( \{n\})$, and let $g_0: \bd (\Delta^1)^{n-1} \rightarrow \bHom_{\calD}(C,D)$
be the corresponding map. Assumption $(1)$ ensures that there are fewer than $\kappa$
extensions $g: (\Delta^1)^{n-1} \rightarrow \bHom_{\calD}(C,D)$ modulo homotopy relative to $\bd (\Delta^1)^{n-1}$. Invoking Lemma \ref{soirt}, we deduce that there are fewer than $\kappa$ maps
$f: \Delta^n \rightarrow \calC$ modulo homotopy relative to $\bd \Delta^n$. Since $\calC'$ is minimal, no two distinct elements of $S$ are homotopic in $\calC$ relative to $\bd \Delta^n$; therefore $S$ is $\kappa$-small as desired.

It is clear that $(2) \Rightarrow (3) \Rightarrow (4)$. 
We next show that $(4) \Rightarrow (3)$. Let $K \rightarrow \calC$ be a categorical equivalence, where $K$ is $\kappa$-small.
We construct a sequence of inner anodyne inclusions
$$ K = K(0) \subseteq K(1) \subseteq \ldots $$
Supposing that $K(n)$ has been defined, we form a pushout diagram
$$ \xymatrix{ \coprod \Lambda^n_i \ar@{^{(}->}[r] \ar[d] & \coprod \Delta^n \ar[d] \\
K(n) \ar@{^{(}->}[r] & K(n+1) }$$
where the coproduct is taken over all $0 < i < n$ and all maps $\Lambda^n_i \rightarrow K(n)$.
It follows by induction on $n$ that each $K(n)$ is $\kappa$-small. Since $\kappa$ is regular and uncountable, the limit $K(\infty) = \bigcup_{n} K(n)$ is $\kappa$-small.The inclusion
$K \subseteq K(\infty)$ is inner anodyne; therefore the map $K \rightarrow \calC$ factors through an equivalence $K(\infty) \rightarrow \calC$ of $\infty$-categories; thus $(3)$ is satisfied. 

We next show that $(3) \Rightarrow (5)$. Suppose that $(3)$ is satisfied. Without loss of generality, we may replace $\calC$ by $\calC'$ and thereby suppose that $\calC$ is itself $\kappa$-small. Let
$F: \Cat_{\infty} \rightarrow \SSet$ denote the functor co-represented by $\calC$. According
to Lemma \ref{repco}, we may identify $F$ with the simplicial nerve of the functor
$f: \Cat_{\infty}^{\Delta} \rightarrow \Kan$, which carries an $\infty$-category
$\calD$ to the largest Kan complex contained in $\calD^{\calC}$. Let $\calI$ be a $\kappa$-filtered $\infty$-category and $p: \calI \rightarrow \Cat_{\infty}$ a diagram. We wish to prove that
$p$ has a colimit $\overline{p}: \calI^{\triangleright} \rightarrow \Cat_{\infty}$ such that
$F \circ \overline{p}$ is a colimit diagram in $\SSet$. According to Proposition \ref{rot}, we may suppose that $\calI$ is the nerve of a $\kappa$-filtered partially ordered set $A$. Using Proposition \ref{gumby444}, we may further reduce to the case where $p$ is the simplicial nerve of a diagram $P: A \rightarrow \Cat_{\infty}^{\Delta} \subseteq \mSet$ taking values in the {\em ordinary} category of marked simplicial sets. Let $\overline{P}$ be a colimit of $P$. Since the class of weak equivalences in $\mSet$ is stable under filtered colimits, $\overline{P}$ is a homotopy colimit. Theorem \ref{colimcomparee} implies that $\overline{p} = \sNerve(\overline{P})$ is a colimit of $p$.
It therefore suffices to show that $F \circ \overline{p} = \sNerve( f \circ \overline{P})$ is a colimit diagram. Using Theorem \ref{colimcomparee}, it suffices to show that $f \circ \overline{P}$ is a homotopy colimit diagram in $\sSet$. Since the class of weak homotopy equivalences in $\sSet$ is stable under filtered colimits, it will suffice to prove that $f \circ \overline{P}$ is a colimit diagram in the ordinary category $\sSet$. It now suffices to observe that $f$ preserves $\kappa$-filtered colimits, because $\calC$ is $\kappa$-small.

We now complete the proof by showing that $(5) \Rightarrow (1)$. Let $A$ denote the collection
of all $\kappa$-small simplicial subsets $K_{\alpha} \subseteq \calC$, and let $A' \subseteq A$ be the subcollection consisting of indices $\alpha$ such that $K_{\alpha}$ is an $\infty$-category. It is clear that $A$ is a $\kappa$-filtered partially ordered set, and that
$\calC = \bigcup_{ \alpha \in A} K_{\alpha}$. Using the fact that $\kappa > \omega$, it is easy to see that $A'$ is cofinal in $A$, so that $A'$ is also $\kappa$-filtered and $\calC = \bigcup_{ \alpha \in A'} K_{\alpha}$. We may therefore regard $\calC$ as the colimit of a diagram
$P: A' \rightarrow \mSet$ in the ordinary category of fibrant objects of $\mSet$. Since $A'$ is filtered,
we may also regard $\calC$ as a homotopy colimit of $P$. The above argument shows that
$\calC^{\calC} = f \calC$ can be identified with a homotopy colimit of the diagram
$f \circ P: A' \rightarrow \sSet$. In particular, the vertex $\id_{\calC} \in \calC^{\calC}$
must be homotopic to the image of some map $K_{\alpha}^{\calC} \rightarrow \calC^{\calC}$,
for some $\alpha \in A'$. It follows that $\calC$ is a retract of $K_{\alpha}$ in the homotopy category
$\h{\Cat_{\infty}}$. Since $K_{\alpha}$ is $\kappa$-small, we easily deduce that $K_{\alpha}$ satisfies condition $(1)$. Therefore $\calC$, being a retract of $K_{\alpha}$, satisfies condition $(1)$ as well.
\end{proof}

\begin{definition}\index{gen}{essentially $\kappa$-small!$\infty$-category}\index{gen}{$\infty$-category!essentially $\kappa$-small}
An $\infty$-category $\calC$ is {\it essentially $\kappa$-small} if it satisfies the equivalent conditions of Proposition \ref{grapeape}. We will say that $\calC$ is {\it essentially small} if it is essentially $\kappa$-small for some (small) regular cardinal $\kappa$.
\end{definition}

The following criterion for essential smallness is occasionally useful:

\begin{proposition}\label{sumt}
Let $p: \calC \rightarrow \calD$ be a Cartesian fibration of $\infty$-categories and $\kappa$ an uncountable regular cardinal. Suppose that $\calD$ is essentially $\kappa$-small and that, for each
object $D \in \calD$, the fiber $\calC_{D} = \calC \times_{\calD} \{D\}$ is essentially $\kappa$-small. Then $\calC$ is essentially $\kappa$-small.
\end{proposition}

\begin{proof}
We will apply criterion $(1)$ of Proposition \ref{grapeape}. Choose a $\kappa$-small set of representatives $\{ D_{\alpha} \}$ for the equivalence classes of objects of $\calD$. For each $\alpha$, choose a $\kappa$-small set of representatives $\{ C_{\alpha,\beta} \}$ for the equivalence classes of objects of $\calC_{D_{\alpha}}$. The collection of all objects
$C_{\alpha,\beta}$ is $\kappa$-small (since $\kappa$ is regular) and contains representatives for all equivalence classes of objects of $\calC$.

Now suppose that $C$ and $C'$ are objects of $\calC$, having images $D, D' \in \calD$.
Since $\calD$ is essentially $\kappa$-small, the set $\pi_0 \bHom_{\calD}(D,D')$ is $\kappa$-small. Let $f: D \rightarrow D'$ be a morphism, and choose a $p$-Cartesian morphism
$\widetilde{f}: \widetilde{C} \rightarrow D'$ covering $f$. According to Proposition \ref{compspaces}, we have a homotopy fiber sequence
$$ \bHom_{\calC_{D}}( C, \widetilde{C}) \rightarrow \bHom_{\calC}(C,C') \rightarrow
\bHom_{\calD}(D,D')$$
in the homotopy category $\calH$. In particular, we see that $\bHom_{\calC}(C,C')$ contains
fewer than $\kappa$ connected components lying over $f \in \pi_0 \bHom_{\calD}(D,D')$, and therefore fewer than $\kappa$ components in total (since $\kappa$ is regular). Moreover, the long exact sequence of homotopy groups shows that for every $\overline{f}: C \rightarrow C'$ lifting
$f$, the homotopy sets $\pi_{i}( \Hom^r_{\calC}(C,C'), f)$ are $\kappa$-small, as desired. 
\end{proof}

By restricting our attention to {\em Kan complexes}, we obtain an analogue of Proposition \ref{grapeape} for spaces:

\begin{corollary}\label{apegrape}\index{gen}{essentially $\kappa$-small!space}Let $X$ be a Kan complex, and $\kappa$ an {\em uncountable} regular cardinal.
The following conditions are equivalent:
\begin{itemize}

\item[$(1)$] For each vertex $x \in X$ and each $n \geq 0$, the homotopy set
$\pi_n(X,x)$ is $\kappa$-small.

\item[$(2)$] If $X' \subseteq X$ is a minimal model for $X$, then $X'$ is $\kappa$-small.

\item[$(3)$] There exists a $\kappa$-small Kan complex $X'$ and a homotopy equivalence $X' \rightarrow X$.

\item[$(4)$] There exists a $\kappa$-small simplicial set $K$ and a weak homotopy equivalence $K \rightarrow X$.

\item[$(5)$] The $\infty$-category $\calC$ is $\kappa$-compact, when regarded as an object of
$\SSet$.

\item[$(6)$] The Kan complex $X$ is essentially small $($when regarded as an $\infty$-category$)$.

\end{itemize}
\end{corollary}

\begin{proof}
The equivalences $(1) \Leftrightarrow (2) \Leftrightarrow (3) \Leftrightarrow (6)$ follow from Proposition \ref{grapeape}.
The implication $(3) \Rightarrow (4)$ is obvious. We next prove that $(4) \Rightarrow (5)$.
Let $p: K \rightarrow \SSet$ be the constant diagram taking the value $\ast$, let
$\overline{p}: K^{\triangleright} \rightarrow \SSet$ be a colimit of $p$, and let $X' \in \SSet$ be the image under $\overline{p}$ of the cone point of $K^{\triangleright}$. It follows from Proposition \ref{dda} that $\ast$ is a $\kappa$-compact object of $\SSet$. Corollary \ref{tyrmyrr} implies that
$X'$ is a $\kappa$-compact object of $\SSet$. Let $\widetilde{K} \rightarrow K^{\triangleright}$ denote the left fibration associated to $\overline{p}$, and let $X'' \subseteq \widetilde{K}$
denote the fiber lying over the cone point of $K^{\triangleright}$. The inclusion of the cone point in $K^{\triangleright}$ is right anodyne. It follows from Proposition \ref{strokhop} that 
the inclusion $X'' \subseteq \widetilde{K}$ is right anodyne. Since $\overline{p}$ is a colimit diagram, Proposition \ref{charspacecolimit} implies that the inclusion
$K \simeq K \times_{ K^{\triangleright}} \widetilde{K} \subseteq \widetilde{K}$ is a weak homotopy equivalence. We therefore have a chain of weak homotopy equivalences
$$ X \leftarrow K \subseteq \widetilde{K} \leftarrow X'' \leftarrow X',$$
so that $X$ and $X'$ are equivalent objects of $\SSet$. Since $X'$ is $\kappa$-compact, it follows that $X$ is $\kappa$-compact.

To complete the proof, we will show that $(5) \Rightarrow (1)$. 
We employ the argument used in the proof of Proposition \ref{grapeape}. 
Let $F: \SSet \rightarrow \SSet$
be the functor co-represented by $X$. Using Lemma \ref{repco}, we can identify $F$ can be with the simplicial nerve of the functor $f: \Kan \rightarrow \Kan$ given by
$$ Y \mapsto Y^X.$$Let $A$ denote the collection of $\kappa$-small simplicial subsets $X_{\alpha} \subseteq X$ which are Kan complexes. Since $\kappa$ is uncountable,
$A$ is $\kappa$-filtered and $X = \bigcup_{\alpha \in A} K_{\alpha}$. We may regard $X$
as the colimit of a diagram $P: A \rightarrow \sSet$. Since $A$ is filtered, $X$ is also a homotopy colimit of this diagram. Since $F$ preserves $\kappa$-filtered colimits, $f$ preserves $\kappa$-filtered homotopy colimits; therefore $X^X$ is a homotopy colimit of the diagram
$f \circ P$.  In particular, the vertex $\id_{X} \in X^{X}$
must be homotopic to the image of some map $X_{\alpha}^{X} \rightarrow X^{X}$,
for some $\alpha \in A$. It follows that $X$ is a retract of $X_{\alpha}$ in the homotopy category $\calH$. Since $X_{\alpha}$ is $\kappa$-small, we can readily verify that $X_{\alpha}$ satisfies $(1)$. Because $X$ is a retract of $X_{\alpha}$, $X$ satisfies $(1)$ as well.
\end{proof}

\begin{remark}
When $\kappa = \omega$, the situation is quite a bit more complicated. Suppose that
$X$ is a Kan complex representing a compact object of $\SSet$. Then there exists a simplicial set $Y$ with only finitely many nondegenerate simplices, and a map $i: Y \rightarrow X$ which
realizes $X$ as a {\em retract} of $Y$ in the homotopy category $\calH$ of spaces. However, one cannot generally assume that $Y$ is a Kan complex, or that $i$ is a weak homotopy equivalence.
The latter can be achieved if $X$ is connected and simply connected, or more generally if a certain $K$-theoretic invariant of $X$ (the {\em Wall finiteness obstruction} ) vanishes: we refer the reader to \cite{wall} for a discussion.\index{gen}{Wall finiteness obstruction}
\end{remark}

For many applications, it is important to be able to slightly relax the condition that an $\infty$-category be essentiall small.

\begin{proposition}\label{locsm}
Let $\calC$ be an $\infty$-category. The following conditions are equivalent:
\begin{itemize}
\item[$(1)$] For every pair of objects $X,Y \in \calC$, the space
$\bHom_{\calC}(X,Y)$ is essentially small.
\item[$(2)$] For every small collection $S$ of objects of $\calC$, the full subcategory
of $\calC$ spanned by the elements of $S$ is essentially small.
\end{itemize}
\end{proposition}

\begin{proof}
This follows immediately from criterion $(1)$ in Propositions \ref{grapeape} and \ref{apegrape}.
\end{proof}

We will say that an $\infty$-category $\calC$ is {\it locally small} if it satisfies the
equivalent conditions of Proposition \ref{locsm}.\index{gen}{locally small $\infty$-category}\index{gen}{$\infty$-category!locally small}

\begin{example}\label{exlocsm}
Let $\calC$ and $\calD$ be $\infty$-categories. Suppose that $\calC$ is locally small and that $\calD$ is essentially small. Then $\calC^{\calD}$ is essentially small. To prove this, we may assume without loss of generality that $\calC$ and $\calD$ are minimal. Let
$\{ \calC_{\alpha} \}$ denote the collection of all full subcategories of $\calC$, spanned by small collections of objects. Since $\calD$ is small, every finite collection of functors
$\calD \rightarrow \calC$ factors through some small $\calC_{\alpha} \subseteq \calC$. 
It follows that $\Fun(\calD,\calC)$ is the union of it small full subcategories $\Fun(\calD, \calC_{\alpha})$, and is therefore locally small. In particular, for every small $\infty$-category $\calD$, the $\infty$-category $\calP(\calD)$ of presheaves is locally small.
\end{example}

\subsection{Accessibility}\label{accessible}

In this section, we will begin our study of the class of accessible $\infty$-categories.

\begin{definition}\label{kapacc}\index{gen}{accessible!$\infty$-category}\index{gen}{$\kappa$-accessible $\infty$-category}\index{gen}{$\infty$-category!accessible}
Let $\kappa$ be a regular cardinal.
An $\infty$-category $\calC$ is {\it $\kappa$-accessible} if there exists a small
$\infty$-category $\calC^{0}$ and an equivalence $\Ind_{\kappa}(\calC^{0}) \rightarrow \calC$.
We will say that $\calC$ is {\it accessible} if it is $\kappa$-accessible for {\em some} regular cardinal $\kappa$.
\end{definition}

The following result gives a few alternative characterizations of the class of accessible $\infty$-categories.

\begin{proposition}\label{clear}
Let $\calC$ be an $\infty$-category and $\kappa$ a regular cardinal. The
following conditions are equivalent:

\begin{itemize}
\item[$(1)$] The $\infty$-category $\calC$ is $\kappa$-accessible.

\item[$(2)$] The $\infty$-category $\calC$ is locally small, admits $\kappa$-filtered colimits, the
full subcategory $\calC^{\kappa} \subseteq \calC$ of $\kappa$-compact objects is essentially small,  and $\calC^{\kappa}$ generates $\calC$ under small, $\kappa$-filtered colimits.

\item[$(3)$] The $\infty$-category $\calC$ admits small $\kappa$-filtered colimits and contains an essentially small full subcategory $\calC'' \subseteq \calC$ which consists of $\kappa$-compact objects and generates $\calC$ under small $\kappa$-filtered colimits.
\end{itemize}
\end{proposition}

The main obstacle to proving Proposition \ref{clear} is in
verifying that if $\calC_0$ is small, then $\Ind_\kappa(\calC_0)$
has only a bounded number of $\kappa$-compact objects, up to equivalence. It
is tempting to guess that any such object must be equivalent to an
object of $\calC_0$. The following example shows that this is not
necessarily the case.

\begin{example}
Let $R$ be a ring, and let $\calC_0$ denote the (ordinary)
category of finitely generated free $R$-modules. Then $\calC =
\Ind(\calC_0)$ is equivalent to the category of flat $R$-modules
(by Lazard's theorem; see for example the appendix of
\cite{lazard}). The compact objects of $\calC$ are precisely the
finitely generated projective $R$-modules, which need not be free.
\end{example}

Nevertheless, the naive guess is not far off, in virtue of the following result:

\begin{lemma}\label{stylus}\index{gen}{idempotent completion}
Let $\calC$ be a small $\infty$-category, $\kappa$ a regular cardinal, and
$\calC' \subseteq \Ind_{\kappa}(\calC)$ the full subcategory of $\Ind_{\kappa}(\calC)$ spanned by the $\kappa$-compact objects. Then the Yoneda embedding
$j: \calC \rightarrow \calC'$ exhibits $\calC'$ as an idempotent completion of $\calC$. In particular, $\calC'$ is essentially small.
\end{lemma}

\begin{proof}
Corollary \ref{swwe} implies that $\Ind_{\kappa}(\calC)$ is idempotent complete. Since $\calC'$ is stable under retracts in $\Ind_{\kappa}(\calC)$, $\calC'$ is also idempotent complete. Proposition \ref{fulfaith} implies that $j$ is fully faithful. It therefore suffices to prove that every object $C' \in \calC'$ is a retract of $j(C)$, for some $C \in \calC$. 

Let $\calC_{/C'} = \calC \times_{ \Ind_{\kappa}(\calC)} \Ind_{\kappa}(\calC)_{/C'}$. Lemma \ref{longwait0} implies that the diagram
$$ \overline{p}: \calC_{/C'}^{\triangleright} \rightarrow \Ind_{\kappa}(\calC)_{/C'}^{\triangleright} \rightarrow \Ind_{\kappa}(\calC)$$ is a colimit of $p= \overline{p} | \calC_{/C'}$. Let $F: \Ind_{\kappa}(\calC) \rightarrow \SSet$ be the functor co-represented by $C'$; we note that the left fibration associated to $F$ is equivalent to $\Ind_{\kappa}(\calC)_{C'/}$. Since $F$ is $\kappa$-continuous, Proposition \ref{charspacecolimit} implies that the inclusion
$$\calC_{/C'} \times_{ \Ind_{\kappa}(\calC)} \Ind_{\kappa}(\calC)_{C'/}
\subseteq \calC^{\triangleright}_{/C'} \times_{ \Ind_{\kappa}(\calC)} \Ind_{\kappa}(\calC)_{C'/}$$
is a weak homotopy equivalence. The simplicial set on the right has a canonical vertex, corresponding to the identity map $\id_{C'}$. It follows that there exists a vertex on the left hand side belonging to the same path component. Such a vertex classifies a diagram
$$ \xymatrix{ & j(C) \ar[dr] & \\
C' \ar[ur] \ar[rr]^{f} & & C' }$$ where $f$ is homotopic to the identity, which proves that $C'$ is a retract of $j(C)$ in $\Ind_{\kappa}(\calC)$.
\end{proof}

\begin{proof}[Proof of Proposition \ref{clear}]
Suppose that $(1)$ is satisfied. Without loss of generality we may suppose that
$\calC = \Ind_{\kappa} \calC'$, where $\calC'$ is small. Since $\calC$ is a full subcategory of $\calP(\calC')$, it is locally small (see Example \ref{exlocsm}). Proposition \ref{geort}
implies that $\calC$ admits small $\kappa$-filtered colimits. Corollary \ref{indpr} shows that
$\calC$ is generated under $\kappa$-filtered colimits by the essential image of the Yoneda embedding $j: \calC' \rightarrow \calC$, which consists of $\kappa$-compact objects by Proposition \ref{justcut}. Lemma \ref{stylus} implies that full subcategory of $\Ind_{\kappa}(\calC')$ consisting of compact objects is essentially small. We conclude that $(1) \Rightarrow (2)$. 

It is clear that $(2) \Rightarrow (3)$. Suppose that $(3)$ is satisfied. Choose a small $\infty$-category $\calC'$ and an equivalence $i: \calC' \rightarrow \calC''$. Using Proposition \ref{intprop}, we may suppose that $i$ factors as a composition
$$ \calC' \stackrel{j}{\rightarrow} \Ind_{\kappa}(\calC') \stackrel{f}{\rightarrow} \calC$$
where $f$ preserves small $\kappa$-filtered colimits. It follows from Proposition \ref{uterr} that
$f$ is a categorical equivalence. This shows that $(3) \Rightarrow (1)$ and completes the proof.
\end{proof}

\begin{definition}\label{accfun}\index{gen}{accessible!functor}
If $\calC$ is an accessible $\infty$-category, then a functor
$F: \calC \rightarrow \calC'$ is {\it accessible} if it is $\kappa$-continuous for some regular cardinal $\kappa$ (and therefore for all regular cardinals $\tau \geq \kappa$).
\end{definition}

\begin{remark}
Generally we will only speak of the accessibility of a functor $F: \calC \rightarrow \calC'$
in the case where both $\calC$ and $\calC'$ are accessible. However, it is occasionally convenient to use the terminology of Definition \ref{accfun} in the case where $\calC$ is accessible and $\calC'$ is not (or $\calC'$ is not yet known to be accessible).
\end{remark}

\begin{example}\label{spacesareaccessible}
The $\infty$-category $\SSet$ of spaces is accessible. More generally, for any small
$\infty$-category $\calC$, the $\infty$-category $\calP(\calC)$ is accessible: this follows immediately from Proposition \ref{precst}.
\end{example}

If $\calC$ is a $\kappa$-accessible $\infty$-category and $\tau > \kappa$, then $\calC$
is not necessarily $\tau$-accessible. Nevertheless, this is true for many values of $\tau$.

\begin{definition}\label{ineq}\index{not}{kappalltau@$\kappa \ll \tau$}
Let $\kappa$ and $\tau$ be regular cardinals. We write $\tau \ll \kappa$ if the following condition is satisfied: for every $\tau_0 < \tau$ and every $\kappa_0 < \kappa$, we have $\kappa_0^{\tau_0} < \kappa$.
\end{definition}

Note that there exist arbitrarily large regular
cardinals $\kappa'$ with $\kappa' \gg \kappa$: for example, one
may take $\kappa'$ to be the successor of any cardinal having the
form $\tau^{\kappa}$.

\begin{remark}
Every (infinite) regular cardinal $\kappa$ satisfies $\omega \ll \kappa$.
An uncountable regular cardinal $\kappa$ satisfies $\kappa \ll \kappa$ if and only if $\kappa$ is strongly inaccessible.
\end{remark}

\begin{lemma}\label{estimate}
If $\kappa' \gg \kappa$, then any $\kappa'$-filtered partially
ordered set $\calI$ may be written as a union of $\kappa$-filtered
subsets having size $< \kappa'$. Moreover, the family of all such
subsets is $\kappa'$-filtered.
\end{lemma}

\begin{proof}
It will suffice to show that every subset of $S \subseteq \calI$
having cardinality $< \kappa'$ can be included in a larger subset
$S'$, such that $|S'| < \kappa'$, but $S'$ is $\kappa$-filtered.

We define a transfinite sequence of subsets $S_{\alpha} \subseteq
\calI$ by induction. Let $S_{0} = S$, and when $\lambda$ is a
limit ordinal we let $S_{\lambda} = \bigcup_{\alpha < \lambda}
S_{\alpha}$. Finally, we let $S_{\alpha + 1}$ denote a set which
is obtained from $S_{\alpha}$ by adjoining an upper bound for
every subset of $S_{\alpha}$ having size $< \kappa$ (which exists
because $\calI$ is $\kappa'$-filtered). It follows from the
assumption $\kappa' \gg \kappa$ that if $S_{\alpha}$ has size $<
\kappa'$, then so does $S_{\alpha + 1}$. Since $\kappa'$ is
regular, we deduce easily by induction that $|S_{\alpha}| <
\kappa'$ for all $\alpha < \kappa'$. It is easy to check that the
set $S' = S_{\kappa}$ has the desired properties.
\end{proof}

\begin{proposition}\label{enacc}
Let $\calC$ be a $\kappa$-accessible $\infty$-category. Then
$\calC$ is $\kappa'$-accessible for any $\kappa' \gg \kappa$.
\end{proposition}

\begin{proof}
Let $\calC^{\kappa} \subseteq \calC$ denote the full subcategory consisting of $\kappa$-compact objects, and let $\calC' \subseteq \calC$ denote the full subcategory spanned by the colimits of all $\kappa'$-small, $\kappa$-filtered diagrams in $\calC^{\kappa}$. Since $\calC$ is locally small and the collection of all equivalence classes of such diagrams is bounded, we conclude that $\calC'$ is essentially small. Corollary \ref{tyrmyrr} implies that $\calC'$ consists of $\kappa'$-compact objects of $\calC$. According to Proposition \ref{clear}, it will suffice to prove that $\calC'$ generates $\calC$ under small $\kappa'$-filtered colimits. Let $X$ be an object of $\calC$, and let
$p: \calI \rightarrow \calC^{\kappa}$ be a small $\kappa$-filtered diagram with colimit $X$.
Using Proposition \ref{rot}, we may reduce to the case where $\calI$ is the nerve of a $\kappa$-filtered partially ordered set $A$. Lemma \ref{estimate} implies that $A$ can be written as a $\kappa'$-filtered union of $\kappa'$-small, $\kappa$-filtered subsets $\{ A_{\beta} \subseteq A \}_{\beta \in B}$. Using Propositions \ref{extet} and \ref{utl}, we deduce that $X$ can also be obtained as the colimit of a diagram indexed by $\Nerve(B)$, which takes values in $\calC'$.
\end{proof}

\begin{remark}
If $\calC$ is a $\kappa$-accessible $\infty$-category and $\kappa' > \kappa$, then $\calC$ is generally not $\kappa'$-accessible. There are counterexamples even in ordinary category theory: see \cite{adamek}.
\end{remark}

\begin{remark}\label{boundedacc}
Let $\calC$ be an accessible $\infty$-category and $\kappa$ a regular cardinal. Then the full subcategory $\calC^{\kappa} \subseteq \calC$ consisting of $\kappa$-compact objects is essentially small. To prove this, we are free to enlarge $\kappa$ and we may invoke Proposition \ref{enacc} to reduce to the case where $\calC$ is $\kappa$-accessible, in which case the desired result is a consequence of Proposition \ref{clear}.
\end{remark}

\begin{notation}\index{not}{FunAcc@$\aHom(\calC, \calD)$}
If $\calC$ and $\calD$ are accessible $\infty$-categories, we will write
$\aHom(\calC, \calD)$ to denote the full subcategory of $\Fun(\calC, \calD)$
spanned by accessible functors from $\calC$ to $\calD$.
\end{notation}

\begin{remark}
Accessible $\infty$-categories are usually not small. However, they are determined by a ``small'' amount of data: namely, they always have the form $\Ind_{\kappa}(\calC)$ where $\calC$ is a small $\infty$-category. Similarly, an accessible functor $F: \calC \rightarrow \calD$ between accessible categories is determined by a ``small'' amount of data, in the sense that there always exists
a regular cardinal $\kappa$ such that $F$ is $\kappa$-continuous and maps
$\calC^{\kappa}$ into $\calD^{\kappa}$. The restriction $F| \calC^{\kappa}$ then determines
$F$ up to equivalence, by Proposition \ref{intprop}. To prove the existence of $\kappa$, 
we first choose a regular cardinal $\tau$ such that $F$ is $\tau$-continuous. Enlarging $\tau$ if necessary, we may suppose that $\calC$ and $\calD$ are $\tau$-accessible. The collection of equivalence classes of $\tau$-compact objects of $\calC$ is small; consequently, Remark \ref{boundedacc} there exists a (small) regular cardinal $\tau'$ such that
$F$ carries $\calC^{\tau}$ into $\calD^{\tau'}$. We may now choose $\kappa$ to be any regular cardinal such that $\kappa \gg \tau'$. 
\end{remark}

\begin{definition}\index{not}{Acck@$\Acc_{\kappa}$}\index{not}{Acc@$\Acc$}
Let $\kappa$ be a regular cardinal.
We let $\Acc_{\kappa} \subseteq \widehat{\Cat}_{\infty}$ denote the subcategory defined as follows:
\begin{itemize}
\item[$(1)$] The objects of $\Acc_{\kappa}$ are the $\kappa$-accessible $\infty$-categories.
\item[$(2)$] A functor $F: \calC \rightarrow \calD$ between accessible $\infty$-categories
belongs to $\Acc$ if and only if $F$ is $\kappa$-continuous and preserves $\kappa$-compact objects. 
\end{itemize}
Let $\Acc = \bigcup_{\kappa} \Acc_{\kappa}$.
We will refer to $\Acc$ as the {\it $\infty$-category of accessible $\infty$-categories}.
\end{definition}

\begin{proposition}\label{humatch}
Let $\kappa$ be a regular cardinal, and let $\theta: \Acc_{\kappa} \rightarrow \widehat{\Cat}_{\infty}$ be the simplicial nerve of the functor which associates to each $\calC \in \Acc_{\kappa}$ the full subcategory of $\calC$ spanned by the $\kappa$-compact objects. Then:
\begin{itemize}
\item[$(1)$] The functor $\theta$ is fully faithful. 
\item[$(2)$] An $\infty$-category $\calC \in \widehat{\Cat}_{\infty}$ belongs to the essential image of $\theta$ if and only if $\calC$ is essentially small and idempotent complete.
\end{itemize}
\end{proposition}

\begin{proof}
Assertion $(1)$ follows immediately from Proposition \ref{intprop}. If $\calC \in \widehat{\Cat}_{\infty}$ belongs to the essential image of $\theta$, then $\calC$ is
essentially small and idempotent complete (because $\calC$ is stable under retracts in an idempotent complete $\infty$-category). Conversely, suppose that $\calC$ is essentially small and idempotent complete, and choose a minimal model $\calC' \subseteq \calC$. Then $\Ind_{\kappa}(\calC')$ is $\kappa$-accessible. Moreover, the collection of $\kappa$-compact objects of $\Ind_{\kappa}(\calC')$ is an idempotent completion of $\calC'$ (Lemma \ref{stylus}), and therefore equivalent to $\calC$ (since $\calC'$ is already idempotent complete).
\end{proof}

Let $\Cat_{\infty}^{\vee}$ denote the full subcategory of $\Cat_{\infty}$ spanned by the idempotent complete $\infty$-categories.\index{not}{Catinftyvee@$\Cat_{\infty}^{\vee}$}

\begin{proposition}
The inclusion $\Cat_{\infty}^{\vee} \subseteq \Cat_{\infty}$ has a left adjoint.
\end{proposition}

\begin{proof}
Combine Propositions \ref{idmcoo}, \ref{charidemcomp}, and \ref{testreflect}.
\end{proof}

We will refer to a left adjoint to the inclusion $\Cat_{\infty}^{\vee} \subseteq \Cat_{\infty}$ as the
{\it idempotent completion functor}. Proposition \ref{humatch} implies that we have fully faithful embeddings $\Acc_{\kappa} \rightarrow \widehat{\Cat}_{\infty} \hookleftarrow \Cat_{\infty}^{\vee}$ with the same essential image. Consequently, there is a (canonical) equivalence of $\infty$-categories $e: \Cat_{\infty}^{\vee} \simeq \Acc_{\kappa}$, well-defined up to homotopy. We let
$\Ind_{\kappa}: \Cat_{\infty} \rightarrow \Acc_{\kappa}$ denote the composition of $e$ with the idempotent completion functor. In summary:

\begin{proposition}\label{lockap}\index{not}{Indkappa@$\Ind_{\kappa}$}
There is a functor $\Ind_{\kappa}: \Cat_{\infty} \rightarrow \Acc_{\kappa}$ which exhibits
$\Acc_{\kappa}$ as a localization of the $\infty$-category $\Cat_{\infty}$.
\end{proposition}

\begin{remark}
There is a slight danger of confusion with our terminology. The functor $\Ind_{\kappa}: \Cat_{\infty} \rightarrow \Acc_{\kappa}$ is only well-defined up to contractible space of choices, so that
if $\calC$ is an $\infty$-category which admits finite colimits, then the image of $\calC$ under $\Ind_{\kappa}$ is only well-defined up to equivalence. Definition \ref{indsmall} produces a canonical representative for this image.
\end{remark}

\subsection{Accessibility and Idempotent Completeness}\label{accessidem}

Let $\calC$ be an accessible $\infty$-category. Then there exists a regular cardinal
$\kappa$ such that $\calC$ admits $\kappa$-filtered colimits. It follows from Corollary \ref{swwe} that $\calC$ is idempotent complete. Our goal in this section is to prove a converse to this result: if $\calC$ is a small and idempotent complete, then $\calC$ is accessible. 

Let $\calC$ be a small $\infty$-category, and suppose we want to prove that $\calC$ is accessible. 
The main problem is to show that $\calC$ admits $\kappa$-filtered colimits, provided that $\kappa$ is sufficiently large. The idea is that if $\kappa$ is much larger than the size of $\calC$, then any $\kappa$-filtered diagram $\calJ \rightarrow \calC$ is necessarily very ``redundant'' (Proposition \ref{stufenn}). Before we making this precise, we will need a few preliminary results.

\begin{lemma}\label{techycard}
Let $\kappa < \tau$ be uncountable regular cardinals, $A$ a $\tau$-filtered partially ordered set, and $F: A \rightarrow \Kan$ a diagram of Kan complexes indexed by $A$. Suppose that for each $\alpha \in A$, the Kan complex $F(\alpha)$ is essentially $\kappa$-small.
For every $\tau$-small subset $A_0 \subseteq A$, there exists a filtered
$\tau$-small subset $A'_0 \subseteq A$ containing $A_0$, with the property that the map
$$ \colim_{\alpha \in A'_0} F(\alpha) \rightarrow \colim_{\alpha \in A} F(\alpha)$$ is a homotopy equivalence.
\end{lemma}

\begin{proof}
Let $X = \colim_{\alpha \in A} F(\alpha)$. Since $F$ is a filtered diagram, $X$ is also a Kan complex. 
Let $K$ be a simplicial set with only finitely many nondegenerate simplices. Our first claim is that
the set $[K,X]$ of homotopy classes of maps from $K$ into $X$ has cardinality $< \kappa$. For suppose given a collection $\{ g_{\beta}: K \rightarrow X \}$ of pairwise nonhomotopic maps, having cardinality $\kappa$. Since $A$ is $\tau$-filtered, we may suppose that there
is a fixed index $\alpha \in A$ such each $g_{\beta}$ factors as a composition
$$ K \stackrel{g'_{\beta}}{\rightarrow} F(\alpha) \rightarrow X.$$
The maps $g'_{\beta}$ are also pairwise nonhomotopic, which contracts our assumption that
$F(\alpha)$ is weakly homotopy equivalent to a $\kappa$-small simplicial set. 

We now define an increasing sequence $$\alpha_0 \leq \alpha_1 \leq \ldots $$
of elements of $A$. Let $\alpha_0$ be any upper bound for $A_0$. Assuming that $\alpha_i$ has already been selected, choose a representative for every homotopy class of diagrams
$$ \xymatrix{ \bd \Delta^n \ar@{^{(}->}[d] \ar[r] & F(\alpha_i) \ar[d] \\
\Delta^n \ar[r]^{h_{\gamma}} & X. }$$
The argument above proves that we can take the set of all such representatives to be $\kappa$-small, so that there exists $\alpha_{i+1} \geq \alpha_i$ such that each $h_{\gamma}$ factors
as a composition
$$ \Delta^{n} \stackrel{h'_{\gamma}}{\rightarrow} F(\alpha_{i+1}) \rightarrow X$$
and the associated diagram 
$$ \xymatrix{ \bd \Delta^n \ar@{^{(}->}[d] \ar[r] & F(\alpha_i) \ar[d] \\
\Delta^n \ar[r]^-{h'_{\gamma}} & F(\alpha_{i+1}) }$$
is commutative. We now set $A'_0 = A_0 \cup \{\alpha_0, \alpha_1, \ldots \}$; it is easy to check
that this set has the desired properties.
\end{proof}

\begin{lemma}\label{techycardd}
Let $\kappa < \tau$ be uncountable regular cardinals, let $A$ be a $\tau$-filtered partially ordered set, let $\{ F_{\beta} \}_{\beta \in B}$ be a collection of diagrams $A \rightarrow \sSet$ indexed by
a set $B$ of cardinality $< \tau$. Suppose that for each $\alpha \in A$ and each
$\beta \in B$, the Kan complex $F_{\beta}(\alpha)$ is essentially $\kappa$-small.
Then there exists a filtered, $\tau$-small subset
$A' \subseteq A$ such that for each $\beta \in B$, the map
$$ \colim_{A'} F_{\beta}(\alpha) \rightarrow \colim_{A} F_{\beta}(\alpha)$$
is a homotopy equivalence of Kan complexes.
\end{lemma}

\begin{proof}
Without loss of generality, we may suppose that $B = \{ \beta: \beta < \beta_0 \}$ is a set of ordinals.
We will define a sequence of filtered, $\tau$-small subsets $A(n) \subseteq A$ by induction on $n$. For $n = 0$, choose an element $\alpha \in A$ and set $A(0) = \{ \alpha\}$. Suppose next that $A(n)$ has been defined. We define a sequence of enlargements $\{ A(n)_{\beta} \}_{\beta \leq \beta_0}$ by induction on $\beta$. Let $A(n)_{0} = A(n)$, let $A(n)_{\lambda} = \bigcup_{\beta < \lambda} A(n)_{\beta}$ when $\lambda$ is a nonzero limit ordinal, and let
$A(n)_{\beta+1}$ be a $\tau$-small, filtered subset of $A$ such that the map
$$ \colim_{A(n)_{\beta+1} } F_{\beta}(\alpha) \rightarrow \colim_{A} F_{\beta}(\alpha)$$ is a weak homotopy equivalence (such a subset exists in virtue of Lemma 
\ref{techycard}). We now take $A(n+1) = A(n)_{\beta_0}$ and $A' = \bigcup_{n} A(n)$; it is easy to check that $A' \subseteq A$ has the desired properties.
\end{proof}

\begin{lemma}\label{sorens}
Let $\kappa < \tau$ be uncountable regular cardinals. Let $\calC$ be a $\tau$-small $\infty$-category with the property that each of the spaces $\bHom_{\calC}(C,D)$ is essentially $\kappa$-small, and $j: \calC \rightarrow \calP(\calC)$ the Yoneda embedding. Let
$p: \calK \rightarrow \calC$ be a diagram indexed by a $\tau$-filtered $\infty$-category
$\calK$, and $\overline{p}: \calK^{\triangleright} \rightarrow \calP(\calC)$ a colimit of $j \circ p$.
Then there exists a map $i: K \rightarrow \calK$ such that $K$ is $\tau$-small, and the composition
$\overline{p} \circ i^{\triangleright}: K^{\triangleright} \rightarrow \calK^{\triangleright} \rightarrow \calP(\calC)$ is a colimit diagram.
\end{lemma}

\begin{proof}
In view of Proposition \ref{rot}, we may suppose that $\calK$ is the nerve of a $\tau$-filtered partially ordered set $A$. According to Proposition \ref{limiteval}, $\overline{p}$ induces a colimit diagram
$$ \overline{p}_{C}: \calK^{\triangleright} \rightarrow \calP(\calC) \stackrel{e_C}{\rightarrow} \SSet$$
where $e_{C}$ denote the evaluation functor associated to an object $C \in \calC$. We will
identify $\calK^{\triangleright}$ with the nerve of the partially ordered set $A \cup \{\infty\}$. 
Proposition \ref{gumby444} implies that we may replace $\overline{p}_{C}$ with the simplicial nerve
of a functor $F_{C}: A \cup \{ \infty\} \rightarrow \Kan$. Our hypothesis on $\calC$ implies
that $F_{C}|A$ takes values in $\kappa$-small simplicial sets. Applying Theorem \ref{colimcomparee}, we see that the map $\colim_{A} F_{C}(\alpha) \rightarrow F_{C}(\infty)$ is a homotopy equivalence. We now apply Lemma \ref{techycard} to deduce the
existence of a filtered, $\tau$-small subset $A' \subseteq A$ such that each of the maps
$$ \colim_{A'} F_{C}(\alpha) \rightarrow F_{C}(\infty)$$ is a homotopy equivalence.
Let $K = \Nerve(A')$, and let $i: K \rightarrow \calK$ denote the inclusion. Using Theorem \ref{colimcomparee} again, we deduce that the composition
$e_{C} \circ  \overline{p} \circ i^{\triangleright}: K^{\triangleright} \rightarrow \SSet$ is a colimit diagram for each $C \in \calC$. Applying Proposition \ref{limiteval}, we deduce that
$\overline{p} \circ i^{\triangleright}$ is a colimit diagram, as desired.
\end{proof}

\begin{proposition}\label{stufenn}
Let $\kappa < \tau$ be uncountable regular cardinals. Let $\calC$ be an $\infty$-category which is $\tau$-small, such that the morphism spaces $\bHom_{\calC}(C,D)$ are essentially $\kappa$-small.
Let $j: \calC \rightarrow \calP(\calC)$ denote the Yoneda embedding, let
$p: \calK \rightarrow \calC$ be a diagram indexed by a $\tau$-filtered $\infty$-category
$\calK$, and let $X \in \calP(\calC)$ be a colimit of $j \circ p: \calK \rightarrow \calP(\calC)$. 
Then there exists an object $C \in \calC$ such that $X$ is a retract of $j(C)$.
\end{proposition}

\begin{proof}
Let $i: K \rightarrow \calK$ be a map satisfying the conclusions of Lemma \ref{sorens}. Since
$K$ is $\tau$-small and $\calK$ is $\tau$-filtered, there exists an extension
$\overline{i}: K^{\triangleright} \rightarrow \calK$ of $i$. Let $C$ be the image of the cone point
of $K^{\triangleright}$ under $p \circ \overline{i}$, and $\widetilde{C} \in \calC_{p \circ i/}$ the
corresponding lift. Let $\overline{p}: \calK^{\triangleright} \rightarrow \calP(\calC)$
be a colimit of $j \circ p$ carrying the cone point of $\calK^{\triangleright}$ to
$X$. Let $q = j \circ p \circ i: K \rightarrow \calP(\calC)$, $\widetilde{X} \in \calP(\calC)_{q/}$ the corresponding lift of $X$, and $\widetilde{Y} \in \calP(\calC)_{q/}$ a colimit of $q$. 
Since $\widetilde{Y}$ is an initial object of $\calP(\calC)_{q/}$, there is a commutative triangle
$$ \xymatrix{ & j( \widetilde{C}) \ar[dr] & \\
\widetilde{Y} \ar[rr] \ar[ur] & & \widetilde{X} }$$
in the $\infty$-category $\calP(\calC)_{q/}$. Moreover, Lemma \ref{sorens} asserts that
the horizontal map is an equivalence. Thus $\widetilde{X}$ is a retract of
$j(\widetilde{C})$ in the homotopy category of $\calP(\calC)_{q/}$, so that
$X$ is a retract of $j(C)$ in $\calP(\calC)$.
\end{proof}

\begin{corollary}\label{tyrrus}
Let $\kappa < \tau$ be uncountable regular cardinals, and let $\calC$ be a $\tau$-small $\infty$-category whose morphism spaces $\bHom_{\calC}(C,D)$ are essentially $\kappa$-small. 
Then the Yoneda embedding $j: \calC \rightarrow \Ind_{\tau}(\calC)$ exhibits
$\Ind_{\tau}(\calC)$ as an idempotent completion of $\calC$.
\end{corollary}

\begin{proof}
Since $\Ind_{\tau}(\calC)$ admits $\tau$-filtered colimits, it is idempotent complete by Corollary \ref{swwe}. Proposition \ref{stufenn} implies that every object of $\Ind_{\tau}(\calC)$ is a retract of $j(C)$, for some object $C \in \calC$.
\end{proof}

\begin{corollary}\label{sloam}\index{gen}{idempotent complete!and accessibility}
A small $\infty$-category $\calC$ is accessible if and only if it is idempotent complete. Moreover, if these conditions are satisfied and $\calD$ is an any accessible $\infty$-category, then
{\em every} functor $f: \calC \rightarrow \calD$ is accessible.
\end{corollary}

\begin{proof}
The ``only if'' follows from Corollary \ref{swwe}, and the ``if'' direction follows from
Corollary \ref{tyrrus}. Now suppose that $\calC$ is small and accessible, and let
$\calD$ be a $\kappa$-accessible $\infty$-category and $f: \calC \rightarrow \calD$ any functor; we wish to prove that $f$ is accessible. By Proposition \ref{intprop}, we may suppose that
$f = F \circ j$, where $j: \calC \rightarrow \Ind_{\kappa}(\calC)$ is the Yoneda embedding
and $F: \Ind_{\kappa}(\calC) \rightarrow \calD$ is a $\kappa$-continuous functor, and therefore accessible. Enlarging $\kappa$ if necessary, we may suppose that $j$ is an equivalence of $\infty$-categories, so that $f$ is accessible as well.
\end{proof}

%\begin{lemma}\label{zumat}
%Let $X$ and $Y$ be simplicial sets, and let $f: X^{op} \rightarrow \calP(Y)$ be
%a diagram, corresponding to a map $e: X \times Y \rightarrow \SSet^{op}$ which classifies a right fibration $Z \rightarrow X \times Y$.

%Let $M = Z^X \times_{ (X \times Y)^X } Y$ denote the pushforward
%$\pi_{\ast} Z$ along the projection $\pi: X \times Y \rightarrow Y$. Then:

%\begin{itemize}
%\item[$(1)$] The projection $M \rightarorw Y$ is a right fibration.
%\item[$(2)$] If $g: Y^{op} \rightarrow \SSet$ classifies $M$, then $g$ can be identified
%with a limit of the diagram $f$ in $\calP(Y)$. 
%\end{itemize}
%\end{lemma}

%\begin{proof}
%Claim $(1)$ follows immediately from Proposition \ref{smoothbase}, since the projection
%$(X \times Y)^{op} \rightarrow Y^{op}$ is smooth. To prove $(2)$, we first apply Proposition \ref{cofinalcategories} to produce a cofinal map $X' \rightarrow X$, where $X'$ is the nerve of a small category $\calC$. Let $Z' = Z \times_{X} X'$, and let 
%$M' = (Z')^{X'} \times_{ (X' \times Y)^{X'} } Y \simeq Z^{X'} \times_{ (X \times Y)^{X'} }Y$.
%Since $X' \rightarrow X$ is cofinal, the natural map $M \rightarrow M'$ is an equivalence of right fibrations over $Y$. We may therefore replace $X$ by $X'$ and thereby reduce to the case 
%where $X = \Nerve \calC$. 

%Without loss of generality, we may suppose that $Y$ is the nerve of a fibrant simplicial category $\calD$. According to Corollary \ref{strictify}, we may suppose that the map $e$ is the nerve of a simplicial functor $E: \calC^{op} \times \calD^{op} \rightarrow \Kan$. We have a commutative diagram of model categories
%$$ \xymatrix{ (\sSet)^{\calD^{op}} \ar[d]^{F} \ar[r] & (\sSet)_{/Y} \ar[d]^{F'} \\
%(\sSet)^{\calC^{op} \times \calD^{op}} \ar[r] & (\sSet)_{/X \times Y} }$$
%where the vertical maps are given by restriction and the horizontal maps are given by the unstraightening functor (see \S \ref{contrasec}), and are therefore Quillen equivalences (Theorem \ref{struns}). The functors $F$ and $F'$ preserve weak equivalences, and therefore induce
%functors on the homotopy categories
%$$ F: h (\sSet)^{\calD^{op}} \rightarrow h (\sSet)^{\calC^{op} \times \calD^{op}}$$
%$$ F': h (\sSet)_{/Y} \rightarrow h (\sSet)_{/X \times Y}.$$
%The functors $F$ and $F'$ both admit right Quillen adjoints, which we will denote by $G$ and $G'$ (here we regard the diagram categories on the right as endowed with the {\em projective} model structure (see \S \ref{quasilimit3}).
%We note that $M \in (\sSet)_{/Y}$ can be identified with $RG'(Z)$ in $(\sSet)_{/Y}$.
%Using Theorem \ref{colimcompare}, we can identify $G(E)$ with a limit of the diagram $f$.
%The commutativity of the diagram now guarantees that $M$ classifies a limit of $f$, as desired.
%\end{proof}

%\begin{lemma}
%Let $\kappa$ be a regular cardinal, $\calC$ an $\infty$-category which admits
%small, $\kappa$-filtered colimits, $K$ a $\kappa$-small simplicial set, and
%$Z \rightarrow \calC \times K$ a left fibration. Suppose that, for every vertex $k \in K$, 
%the left fibration $Z_k \rightarrow \calC$ classifies a $\kappa$-continuous functor
%$F_{k}: \calC \rightarrow \hat{\SSet}$. Then the associated left fibration
%$$ Z^{K} \times_{ (\calC \times K)^K} \calC \rightarrow \calC$$
%classifies a $\kappa$-continuous functor $F: \calC \rightarrow \hat{\SSet}$.
%\end{lemma}

%\begin{proof}
%According to Lemma \ref{zumat}, we may identify $F$ with a $\kappa$-small limit
%of the functors $F_{k}$ in $\hat{\SSet}^{\calC}$. We now apply Lemma \ref{misst}.
%\end{proof}

%\begin{lemma}
%Let $\calC$ be an $\infty$-category and $K$ a simplicial set.
%Suppose that $M \rightarrow \calC \times \calC^{op}$ is a left fibration which is classified by the Yoneda embedding $\calC \rightarrow \calP(\calC)$. 
%Consider the diagram
%$$ \calC^K \times (\calC^K)^{op} \stackrel{p}{\leftarrow} \calC^K \times (\calC^K)^{op} \times K \times  K^{op} \stackrel{e}{\rightarrow} \calC \times \calC^{op}.$$ 
%Then the left fibration $p_{\ast} e^{\ast} M$ classifies the Yoneda embedding
%$\calC^K \rightarrow \calP(\calC^K)$.
%\end{lemma}

%\begin{proof}
%*** 
%\end{proof}

\subsection{Accessibility of Functor $\infty$-Categories}\label{accessfunk}

Let $\calC$ be an accessible $\infty$-category, and let $K$ be a small simplicial set. Our goal in this section is to prove that $\Fun(K,\calC)$ is accessible (Proposition \ref{horse1}). In \S \ref{accessstable}, we will prove a much more general stability result of this kind (Corollary \ref{storkus1}), but the proof of that result ultimately rests on the ideas presented here. 

Our proof goes roughly as follows. If $\calC$ is accessible, then $\calC$ has a many $\tau$-compact objects, provided that $\tau$ is sufficiently large. Using Proposition \ref{placeabovee}, we deduce the existence of many $\tau$-compact objects in $\Fun(K,\calC)$. Our main problem is to show that these objects generate $\Fun(K,\calC)$ under $\tau$-filtered colimits. To prove this, we will use a rather technical cofinality result (Lemma \ref{kidav} below). We begin with the following prelimiinary observation:

\begin{lemma}\label{pprekidav}
Let $\tau$ be a regular cardinal, and let $q: Y \rightarrow X$ be a coCartesian fibration with the property that for every vertex $x$ of $X$, the fiber $Y_{x} = Y \times_{X} \{x\}$ is $\tau$-filtered.
Then $q$ has the right lifting property with respect to $K \subseteq K^{\triangleright}$, for every
$\tau$-small simplicial set $K$.
\end{lemma}

\begin{proof}
Using Proposition \ref{princex}, we can reduce to the problem of showing that
$q$ has the right lifting property with respect to the inclusion $K \subseteq K \diamond \Delta^0$. 
In other words, we must show that given any edge $e: C \rightarrow D$ in $X^K$,
where $D$ is a constant map, and any vertex $\widetilde{C}$ of $Y^K$ lifting
$C$, there exists an edge $\widetilde{e}: \widetilde{C} \rightarrow \widetilde{D}$
lifting $\widetilde{e}$, where $\widetilde{D}$ is a constant map from $K$ to $Y$.
We first choose an arbitrary edge $\widetilde{e}': \widetilde{C} \rightarrow \widetilde{D}'$ lifting $e$ (since the map $q^K: Y^K \rightarrow X^K$ is a coCartesian fibration, we can even choose $\widetilde{e}'$ to be $q^K$-coCartesian, though we will not need this). Suppose that
$D$ takes the constant value $x: \Delta^0 \rightarrow X$. Since the fiber $Y_{x}$ is $\tau$-filtered,
there exists an edge $\widetilde{e}'': \widetilde{D}' \rightarrow \widetilde{D}$ in
$Y_{x}^K$, where $\widetilde{D}$ is a constant map from $K$ to $Y_{x}$. We now invoke
the fact that $q^K$ is an inner fibration to supply the dotted arrow in the diagram
$$ \xymatrix{ \Lambda^2_1 \ar[rr]^{ (\widetilde{e}', \bigdot, \widetilde{e}'')} \ar@{^{(}->}[d] & & Y^K \ar[d] \\
\Delta^2 \ar@{-->}[urr]^{\sigma} \ar[rr]^{ s_1 e} & & X^K. }$$
We now define $\widetilde{e} = \sigma | \Delta^{ \{0,2\} }$. 
\end{proof}

\begin{lemma}\label{kidav}
Let $\kappa < \tau$ be regular cardinals. Let $q: Y \rightarrow X$ be a map of simplicial sets with the following properties:
\begin{itemize}
\item[$(i)$] The simplicial set $X$ is $\tau$-small.
\item[$(ii)$] The map $q$ is a coCartesian fibration.
\item[$(iii)$] For every vertex $x \in X$, the fiber $Y_{x} = Y \times_{X} \{x\}$ is $\tau$-filtered
and admits $\tau$-small, $\kappa$-filtered colimits.
\item[$(iv)$] For every edge $e: x \rightarrow y$ in $X$, the associated functor
$Y_{x} \rightarrow Y_{y}$ preserves $\tau$-small, $\kappa$-filtered colimits.
\end{itemize}
Then:
\begin{itemize}
\item[$(1)$] The $\infty$-category $\calC = \bHom_{/X}(X,Y)$ of sections of $q$ is $\tau$-filtered.
\item[$(2)$] For each vertex $x$ of $X$, the evaluation map
$e_{x}: \calC \rightarrow Y_{x}$ is cofinal.
\end{itemize}
\end{lemma}

\begin{proof}
Choose a categorical equivalence $X \rightarrow M$, where $M$ is a minimal $\infty$-category. Since $\tau$ is uncountable, Proposition \ref{grapeape} implies that $M$ is $\tau$-small.
According to Corollary \ref{tttroke}, $Y$ is equivalent to the pullback of a coCartesian fibration $Y' \rightarrow M$. We may therefore replace $X$ by $M$ and thereby reduce to the case where
$X$ is a minimal $\infty$-category. For each ordinal $\alpha$, let $(\alpha)= \{ \beta < \alpha\}$. 

Let $K$ be a $\tau$-small simplicial set equipped with a map $f: K \rightarrow Y$.
We define a new object $K'_{X} \in (\sSet)_{/X}$ as follows. For every finite, nonempty, linearly ordered set $J$, a map $\Delta^J \rightarrow K'_X$ is determined by the following data:

\begin{itemize}
\item A map $\chi: \Delta^{J} \rightarrow X$.

\item A map $\Delta^J \rightarrow \Delta^2$, corresponding to a decomposition
$J = J_0 \coprod J_1 \coprod J_2$.

\item A map $\Delta^{J_0} \rightarrow K$.

\item An order-preserving map $m: J_1 \rightarrow (\kappa)$, having the property that
if $m(i) = m(j)$, then $\chi( \Delta^{ \{i,j\} })$ is a degenerate edge of $X$.
\end{itemize} 

We will prove the existence of a dotted arrow $F'_X$ as indicated in the diagram
$$ \xymatrix{ K \ar[d] \ar[r]^{f} & Y \ar[d]^{q} \\
K'_X \ar@{-->}[ur]^{F'_X} \ar[r] & X. }$$
Let $K'' \subseteq K'_X$ be the simplicial subset corresponding to simplices, as above, where
$J_1 = \emptyset$, and let $F'' = F'_X | K''$. Specializing to the case where $K = Z \times X$, $Z$ a $\tau$-small simplicial set, we will deduce that any diagram $Z \rightarrow \calC$ extends to a map $Z^{\triangleright} \rightarrow \calC$ (given by $F''$), which proves $(1)$. Similarly, by specializing to the case $K = (Z \times X) \coprod_{ Z \times \{x\} } (Z^{\triangleleft} \times \{x\} )$, we will deduce that for every object $y \in Y$ with $q(y) = x$, the $\infty$-category
$\calC \times_{ Y_{x} } (Y_{x})_{y/}$ is $\tau$-filtered, and therefore weakly contractible. 
Applying Theorem \ref{hollowtt}, we deduce $(2)$. 

It remains to construct the map $F'_X$. There
is no harm in enlarging $K$. We may therefore apply the small object argument
to replace $K$ by an $\infty$-category (which we may also suppose is $\tau$-small, since $\tau$ is uncountable). We begin by defining, for each $\alpha \leq \kappa$,
a simplicial subset $K(\alpha) \subseteq K'_X$. The definition is as follows: we will say
that a simplex $\Delta^J \rightarrow K'_X$ factors through $K(\alpha)$ if, in the corresponding decomposition $J = J_0 \coprod J_1 \coprod J_2$, we have $J_2 = \emptyset$, and the map
$J_1 \rightarrow (\kappa)$ factors through $(\alpha)$. Our first task is to construct
$F(\alpha) = F'_X | K(\alpha)$, which we do by induction on $\alpha$.
If $\alpha=0$, $K(\alpha) = K$ and we set $F(0) = f$.
When $\alpha$ is a limit ordinal, we have $K(\alpha) = \bigcup_{ \beta < \alpha } K(\beta)$ and we set $F(\alpha) = \bigcup_{\beta < \alpha} F(\beta)$. 
It therefore suffices to construct $F(\alpha+1)$, assuming that $F(\alpha)$ has already been constructed. For each vertex $x$ of $X$, let $\widetilde{x}=(x, \alpha)$ denote the unique vertex of $K(\alpha+1)$ lying over $x$ which does not belong to $K(\alpha)$. Since $X$ is minimal, Proposition \ref{minstrict} implies that we have a pushout diagram
$$ \xymatrix{ \coprod_{x} K(\alpha)_{/\widetilde{x}} \ar@{^{(}->}[r] \ar[d] & \coprod_{x} (K(\alpha)_{/\widetilde{x}})^{\triangleright} \ar[d]  \\
K(\alpha) \ar@{^{(}->}[r] & K(\alpha+1). }$$
Therefore, to construct $f_{\alpha+1}$, it suffices to prove that $q$ has the right lifting property with respect to each inclusion $K(\alpha)_{/\widetilde{x}} \subseteq (K(\alpha)_{/\widetilde{x}})^{\triangleright}$, which follows
from Lemma \ref{pprekidav}.

We now define, for each simplicial subset $X' \subseteq X$, a corresponding simplicial subset
$K'_{X'} \subseteq K'_{X}$. The definition is as follows: let $\sigma: \Delta^J \rightarrow K'_{X}$ be a simplex corresponding to a decomposition $J = J_0 \coprod J_1 \coprod J_2$. Then $\sigma$
factors through $K'_{X'}$ if and only if the induced map $\Delta^{J_2} \rightarrow X$ factors through $X'$. Our next job is to extend the definition of $F'_{X}$ from $K'_{\emptyset} = K(\kappa)$
to $K'_{X}$, by adjoining simplices to $X$ one at a time.

Let $F'_{\emptyset} = F(\kappa)$, and let $x$ be a vertex of $X$. We begin by defining
a map $F'_{\{x\} } : K'_{\{x\} } \rightarrow Y$ which extends $F'_{\emptyset}$. 
Since $X$ is minimal, there is a pushout diagram
$$ \xymatrix{ K(\kappa)_{/x} \ar@{^{(}->}[r] \ar[d] & K(\kappa)_{/x}^{\triangleright} \ar[d] \\
K_{\emptyset} \ar@{^{(}->}[r] &  K_{\{x\} } } $$
where $K(\kappa)_{/x}$ denotes the fiber product $K(\kappa) \times_{X} X_{/x}$. 
Constructing an extension $F'_{ \{x\} }$ of $F'_{\emptyset}$ is therefore equivalent to providing the dotted arrow indicated in the diagram
$$ \xymatrix{ K(\kappa)_{/x} \ar@{^{(}->}[d] \ar[r]^{p_x} & Y \ar[d] \\
K(\kappa)_{/x}^{\triangleright} \ar[r] \ar@{-->}[ur]^{\overline{p}_x} & X }.$$
We will choose $\overline{p}_x$ to be a relative colimit of $p_x$ over $X$ (see \S \ref{relcol}). To prove that
such a relative colimit exists, we consider the inclusion
$i_{x}: \Nerve (\kappa) \subseteq K(\kappa)_{/x} \times_{X_{/x}} \{ \id_{x} \} \subseteq K(\kappa)_{/x}$. Using Proposition \ref{minstrict}, it is not difficult to see that
$K(\kappa)_{/x}$ is an $\infty$-category. For each 
object $y \in K(\kappa)_{/x}$, the minimality of $X$ implies that $\Nerve (\kappa) \times_{ \calK_{/x} } (\calK_{/x})_{y/}$ is isomorphic to $\Nerve(\{ \alpha: \beta < \alpha < \kappa \})$ for some $\beta < \kappa$, and therefore weakly contractible. Theorem \ref{hollowtt} implies that $i_{x}$ is cofinal. Invoking Proposition \ref{relexist}, it will suffice to prove that 
$p_x \circ i_{x}: \Nerve (\kappa) \rightarrow Y$ admits a relative colimit over $X$. Using conditions
$(ii)$, $(iv)$, and Proposition \ref{relcolfibtest}, we may reduce to producing a colimit
of $p_{x} \circ i_{x}$ in the $\infty$-category $Y_{x}$, which is possible in virtue of assumption $(iii)$.

Applying the above argument separately to each vertex of $X$, we may suppose that
$F'_{X^{(0)}}$ has been constructed, where $X^{(0)}$ denotes the $0$-skeleton of $X$. We now consider the collection of all pairs $( X', F'_{X'})$ where $X'$ is a simplicial subset of $X$ containing all vertices of $X$, and $F'_{X'}: K_{X'} \rightarrow Y$ is a map over $X$ whose restriction to
$K_{X^{0}}$ coincides with $F'_{X^{(0)}}$. This collection is partially ordered, if we write
$(X', F'_{X'}) \leq (X'', F'_{X''})$ to mean that $X'\subseteq X''$ and $F'_{X''} | K_{X'} = F'_{X'}$. 
The hypotheses of Zorn's lemma are satisfied, so that there exists a maximal such pair
$(X', F'_{X'})$. To complete the proof, it suffices to show that $X' = X$. If not, we can choose
$X' \subseteq X'' \subseteq X$, where $X''$ is obtained from $X'$ by adjoining a single nondegenerate simplex $\sigma: \Delta^n \rightarrow X$ whose boundary already belongs to $X'$.
Since $X'$ contains $X^{(0)}$, we may suppose that $n > 0$. Let $K(\kappa)_{/\sigma} = K(\kappa) \times_{X} X_{/\sigma}$, and let
$x = \sigma(0)$. Since $X$ is minimal, we have a pushout diagram
$$ \xymatrix{ K(\kappa)_{/\sigma} \star \bd \Delta^n \ar@{^{(}->}[r] \ar[d] & K(\kappa)_{/\sigma} \star \Delta^n \ar[d] \\
K'_{X'} \ar@{^{(}->}[r] & K'_{X''}. }$$
Let $s: K(\kappa)_{/\sigma} \rightarrow Y$ denote the composition of the projection $K(\kappa)_{/\sigma} \rightarrow K'_{X'}$ with
$F'_{X'}$.  We obtain a commutative diagram
$$ \xymatrix{ \bd \Delta^n \ar[r]^{r} \ar@{^{(}->}[d] & Y_{s/} \ar[d] \\
\Delta^n \ar[r] \ar@{-->}[ur] & X_{q \circ s/}, }$$
and supplying the indicated dotted arrow is tantamount to giving a map $F'_{X''}: K_{X''} \rightarrow Y$ over $X$ which extends $F'_{X'}$. To prove the existence of $F'_{X''}$, it suffices to prove
that the map $\overline{s}: {K'}^{\triangleright} \rightarrow Y$ associated to $r(0)$ is a 
a $q$-colimit diagram. We note that $\overline{s}$ is given as a composition
$$ {K'}^{\triangleright} \rightarrow K_{/x}^{\triangleright} \stackrel{\overline{s}'}{\rightarrow} Y,$$
where $\overline{s}'$ is a $q$-colimit diagram by construction. According to Proposition \ref{relexists}, it will suffice to show that the map $K(\kappa)_{/\sigma} \rightarrow K(\kappa)_{/x}$ is cofinal.
We have a pullback diagram
$$ \xymatrix{ K(\kappa)_{/\sigma} \ar[r] \ar[d] & K(\kappa)_{/x} \ar[d] \\
X_{/\sigma} \ar[r] & X_{/x} }$$ where the lower horizontal map is a trivial fibration of simplicial sets.
It follows that the upper horizontal map is a trivial fibration, and in particular cofinal. Consequently, there exists an extension $F_{X''}$ of $F_{X'}$, which contradicts the maximality of $(X', F_{X'})$ and completes the proof.
\end{proof}

\begin{proposition}\label{horse1}\index{gen}{accessible!functor categories}
Let $\calC$ be an accessible $\infty$-category, and let $K$ be a small simplicial set. Then
$\Fun(K,\calC)$ is accessible.
\end{proposition}

\begin{proof}
Without loss of generality, we may suppose that $K$ is an $\infty$-category.
Choose a regular cardinal $\kappa$ such that $\calC$ admits small $\kappa$-filtered colimits, and choose a second regular cardinal $\tau > \kappa$ such that $\calC$ is also $\tau$-accessible and $K$ is $\tau$-small.
We will prove that $\Fun(K,\calC)$ is $\tau$-accessible. Let $\calC' = \Fun(K,\calC^{\tau}) \subseteq \Fun(K,\calC)$. It is clear
that $\calC'$ is essentially small. Proposition \ref{limiteval} implies that $\Fun(K,\calC)$ admits small $\tau$-filtered colimits, and Proposition \ref{placeabovee} asserts that $\calC'$ consists of $\tau$-compact objects of $\Fun(K,\calC)$. According to Proposition \ref{clear}, it will suffice to prove that $\calC'$ generates $\Fun(K,\calC)$ under small, $\tau$-filtered colimits.

Without loss of generality, we may suppose that $\calC = \Ind_{\tau} \calD'$, where $\calD'$ is a small $\infty$-category. Let $\calD \subseteq \calC$ denote the essential image of the Yoneda embedding. Let $F: K \rightarrow \calC$ be an arbitrary object of $\calC^K$, and let $\Fun(K,\calD)^{/F} = \Fun(K,\calD) \times_{\Fun(K,\calC)} \Fun(K,\calC)^{/F}$. 
Consider the composite diagram
$$ \overline{p}: \Fun(K,\calD)^{/F} \diamond \Delta^0 \rightarrow
\Fun(K,\calC)^{/F} \diamond \Delta^0 \rightarrow \Fun(K,\calC).$$
The $\infty$-category $\Fun(K,\calD)^{/F}$ is equivalent to 
$\Fun(K,\calD') \times_{\Fun(K,\calC)} \Fun(K,\calC)^{/F}$, and therefore essentially small.
To complete the proof, it will suffice to show that $\Fun(K,\calD)^{/F}$ is $\tau$-filtered, and that
$\overline{p}$ is a colimit diagram.

We may identify $F$ with a map $f_K: K \rightarrow \calC \times K$ in $(\sSet)_{/K}$. According to 
Proposition \ref{colimfam}, we obtain a coCartesian fibration $q: (\calC \times K)^{/f_K} \rightarrow K$, 
and the $q$-coCartesian morphisms are precisely those which project to equivalences in $\calC$. Let
$X$ denote the full subcategory of $(\calC \times K)^{/f_K}$ consisting of those objects whose projection to $\calC$ belongs to $\calD$. It follows that $q' = q|X: X \rightarrow K$ is a coCartesian fibration. We may identify the fiber of $q'$ over a vertex
$x \in K$ with $\calD^{/F(x)} = \calD \times_{\calC} \calC^{/F(x)}$. It follows that the fibers
of $q'$ are $\tau$-filtered $\infty$-categories; Lemma \ref{kidav} now guarantees that
$\Fun(K,\calD)^{/F} \simeq \bHom_{/K}(K,X)$ is $\tau$-filtered.

According to Proposition \ref{limiteval}, to prove that $\overline{p}$ is a colimit diagram, it will suffice to prove that for every vertex $x$ of $K$, the composition of $\overline{p}$ with the evaluation
map $e_{x}: \Fun(K,\calC) \rightarrow \calC$ is a colimit diagram. The composition
$e_{x} \circ \overline{p}$ admits a factorization
$$ \Fun(K,\calD)^{/F} \diamond \Delta^0 \rightarrow \calD^{/F(x)} \diamond \Delta^0
\rightarrow \calC$$
where $\calD^{/F(x)} = \calD \times_{\calC} \calC^{/F(x)}$ and the second map is a colimit diagram in $\calC$ by Lemma \ref{longwait0}. It will therefore suffice to prove that the map
$g_x: \Fun(K,\calD)^{/F} \rightarrow \calD^{/F(x)}$ is cofinal, which follows from Lemma \ref{kidav}.
\end{proof}

\subsection{Accessibility of Undercategories}\label{accessprime}

Let $\calC$ be an accessible $\infty$-category, and let $p: K \rightarrow \calC$ be a small diagram. Our goal in this section is to prove that the $\infty$-category $\calC_{p/}$ is accessible (Corollary \ref{horsemn}). 

\begin{remark}
The analogous result for the $\infty$-category $\calC_{/p}$ will be proven in \S \ref{accessfiber}, using Propositions \ref{horse1} and \ref{horse2}. It is possible to use the same argument to give a second proof of Corollary \ref{horsemn}; however, we will {\em need} Corollary \ref{horsemn} in our proof of Proposition \ref{horse2}.
\end{remark}

We begin by studying the behavior of colimits with respect to (homotopy) fiber products of $\infty$-categories.

\begin{lemma}\label{bird1}\index{gen}{initial object!in a homotopy fiber product}
Let $$ \xymatrix{ \calX' \ar[r]^{q'} \ar[d]^{p'} & \calX \ar[d]^{p} \\
\calY' \ar[r]^{q} & \calY }$$
be a diagram of $\infty$-categories which is homotopy Cartesian $($with respect to the Joyal model structure$)$. Suppose that $\calX$ and $\calY$ have initial objects, and that $p$ and $q$ preserve initial objects. An object
$X' \in \calX'$ is initial if and only if $p'(X')$ is an initial object of $\calY'$ and
$q'(X')$ is an initial object of $\calX$. Moreover, there exists an initial object of $\calX'$.
\end{lemma}

\begin{proof}
Without loss of generality, we may suppose that $p$ and $q$ are categorical fibrations, and that
$\calX' = \calX \times_{ \calY} \calY'$. Suppose first that $X'$ is an object of $\calX'$ with the property that $X = q'(X')$ and $Y' = p'(X')$ are initial objects of $\calX$ and $\calY'$. Then
$Y = p(X) = q(Y')$ is an initial object of $\calY$. Let $Z$ be another object of $\calX'$.
We have a pullback diagram of Kan complexes
$$ \xymatrix{ \Hom_{\calX'}^{\rght}(X', Z) \ar[r] \ar[d] & \Hom_{\calX}^{\rght}(X, q'(Z)) \ar[d] \\
\Hom_{\calY'}^{\rght}(Y', p'(Z)) \ar[r] & \Hom_{\calY}^{\rght}(Y, (q \circ p')(Z)). }$$
Since the maps $p$ and $q$ are inner fibrations, Lemma \ref{sharpy} implies that this diagram is homotopy Cartesian (with respect to the usual model structure on $\sSet$). Since
$X$, $Y'$, and $Y$ are initial objects, the Kan complexes $\Hom_{\calX}^{\rght}(X, q'(Z))$,
$\Hom_{\calY'}^{\rght}(Y', p'(Z))$, and $\Hom_{\calY}^{\rght}(Y, (q \circ p')(Z))$ are contractible. It follows that $\Hom_{\calX'}^{\rght}(X', Z)$ is contractible as well, so that $X'$ is an initial object of $\calX'$. 

We now prove that there exists an object $X' \in \calX'$ such that $p'(X')$ and $q'(X')$ are initial.
The above argument shows that $X'$ is an initial obejct of $\calX'$. Since all initial objects of $\calX'$ are equivalent, this will prove that for {\em any} initial object $X'' \in \calX'$, the
objects $p'(X'')$ and $q'(X'')$ are initial.

We begin by selecting arbitrary initial objects $X \in \calX$ and $\overline{Y} \in \calY'$.  Then
$p(X)$ and $q(\overline{Y})$ are both initial objects of $\calY$, so there is an equivalence
$e: p(X) \rightarrow q(\overline{Y})$. Since $q$ is a categorical fibration, there exists an equivalence
$\overline{e}: Y' \rightarrow \overline{Y}$ in $\calY$ such that $q(\overline{e}) = e$. It follows that $Y'$ is an initial object of $\calY'$ with $q(Y') = p(X)$, so that the pair $(X,Y')$ can be identified with an object of $\calX'$ which has the desired properties.
\end{proof}

\begin{lemma}\label{airbirdd}
Let $p: \calX \rightarrow \calY$ be a categorical fibration of $\infty$-categories, and
let $f: K \rightarrow \calX$ be a diagram. Then the induced map
$p': \calX_{f/} \rightarrow \calY_{p  f/}$ is a categorical fibration.
\end{lemma}

\begin{proof}
It suffices to show that $p'$ has the right lifting property with respect to every inclusion
$A \subseteq B$ which is a categorical equivalence. Unwinding the definitions, it suffices
to show that $p$ has the right lifting property with respect to $i: K \star A \subseteq K \star B$.
This is immediate, since $p$ is a categorical fibration and $i$ is a categorical equivalence.
\end{proof}

\begin{lemma}\label{airbird}\index{gen}{undercategory!and homotopy fiber products}
Let $$ \xymatrix{ \calX' \ar[r]^{q'} \ar[d]^{p'} & \calX \ar[d]^{p} \\
\calY' \ar[r]^{q} & \calY }$$
be a diagram of $\infty$-categories which is homotopy Cartesian (with respect to the Joyal model structure), and let $f: K \rightarrow \calX'$ be a diagram in $\calX'$. Then the induced
diagram
$$ \xymatrix{ \calX'_{f/} \ar[r] \ar[d] & \calX_{q'  f/} \ar[d] \\
\calY'_{p'  f/} \ar[r] & \calY_{q  p'  f/}}$$
is also homotopy Cartesian.
\end{lemma}

\begin{proof}
Without loss of generality, we may suppose that $p$ and $q$ are categorical fibrations
and that $\calX' = \calX \times_{\calY} \calY'$. Then $\calX'_{f/} \simeq
\calX_{q'  f/} \times_{ \calY_{q  p'  f/} } \calY'_{p'  f/}$, so the result
follows immediately from Lemma \ref{airbirdd}.
\end{proof}

\begin{lemma}\label{bird3}\index{gen}{colimit!and homotopy fiber products}
Let $$ \xymatrix{ \calX' \ar[r]^{q'} \ar[d]^{p'} & \calX \ar[d]^{p} \\
\calY' \ar[r]^{q} & \calY }$$
be a diagram of $\infty$-categories which is homotopy Cartesian (with respect to the Joyal model structure), and let $K$ be a simplicial set. Suppose that $\calX$ and $\calY'$ admit colimits for all diagrams indexed by $K$, and that $p$ and $q$ preserve colimits of diagrams indexed by $K$. 
Then:
\begin{itemize}
\item[$(1)$] A diagram $\overline{f}: K^{\triangleright} \rightarrow \calX'$ is a colimit of $f=\overline{f}|K$ if and only if
$p' \circ \overline{f}$ and $q' \circ \overline{f}$ are colimit diagrams. In particular, 
$p'$ and $q'$ preserve colimits indexed by $K$.
\item[$(2)$] Every diagram $f: K \rightarrow \calX'$ has a colimit in $\calX'$.
\end{itemize}
\end{lemma}

\begin{proof}
Replacing $\calX'$ by $\calX'_{f/}$,
$\calX$ by $\calX_{q'  f/}$, $\calY'$ by $\calY'_{p'  f/}$, and $\calY$ by
$\calY_{q  p'  f/}$, we may apply Lemma \ref{airbird} to reduce to the case $K = \emptyset$. Now apply Lemma \ref{bird1}.
\end{proof}

\begin{lemma}\label{filterprod}
Let $\calC$ be a small filtered category, and let $\calC^{\triangleright}$ be the category
obtained by adjoining a $($new$)$ final object to $\calC$.
Suppose given a homotopy pullback diagram
$$ \xymatrix{ F' \ar[r] \ar[d] & F \ar[d]^{p} \\
G' \ar[r]^{q} & G }$$
in the diagram category $\Set_{\Delta}^{\calC^{\triangleright}}$ $($which we endow with the {\it projective}
model structure$)$. Suppose further that the diagrams $F, G, G': \calC^{\triangleright} \rightarrow \sSet$ are homotopy colimits. Then $F'$ is also a homotopy colimit diagram.
\end{lemma}

\begin{proof}
Without loss of generality, we may suppose that $G$ is fibrant, $p$ and $q$ are fibrations, and that
$F' = F \times_{G} G'$. Let $\ast$ denote the cone point of $\calC^{\triangleright}$, and let
$F(\infty)$, $G(\infty)$, $F'(\infty)$, and $G'(\infty)$ denote the colimits of the diagrams
$F|\calC$, $G|\calC$, $F'| \calC$, and $G'| \calC$. Since fibrations in $\sSet$ are stable under filtered colimits, the pullback diagram
$$ \xymatrix{ F'(\infty) \ar[r] \ar[d] & F(\infty) \ar[d] \\
G'(\infty) \ar[r] & G(\infty) }$$
exhibits $F'(\infty)$ as a homotopy fiber product of $F(\infty)$ with $G'(\infty)$ over
$G(\infty)$ in $\sSet$. 
Since weak homotopy equivalences are stable under filtered colimits, the natural maps $G(\infty) \rightarrow G(\ast)$, $F'(\infty) \rightarrow F'(\ast)$, and $G'(\infty) \rightarrow G'(\ast)$ are weak homotopy equivalences. Consequently, the diagram
$$ \xymatrix{ F'(\infty) \ar[dr]^{f} & & \\
& F'(\ast) \ar[r] \ar[d] & F(\ast) \ar[d] \\
& G'(\ast) \ar[r] & G(\ast) }$$
exhibits both $F'(\infty)$ and $F'(\ast)$ as homotopy fiber products of $F(\ast)$ with
$G'(\ast)$ over $G(\ast)$. It follows that $f$ is a weak homotopy equivalence, so that $F$ is a homotopy colimit diagram as desired.
\end{proof}

\begin{lemma}\label{yoris}
Let $$ \xymatrix{ \calX' \ar[r]^{q'} \ar[d]^{p'} & \calX \ar[d]^{p} \\
\calY' \ar[r]^{q} & \calY }$$
be a diagram of $\infty$-categories which is homotopy Cartesian $($with respect to the Joyal model structure$)$, and let $\kappa$ be a regular cardinal. Suppose that $\calX$ and $\calY'$ admit small
$\kappa$-filtered colimits, and that $p$ and $q$ preserve small $\kappa$-filtered colimits.
Then:
\begin{itemize}
\item[$(1)$] The $\infty$-category $\calX'$ admits small $\kappa$-filtered colimits.
\item[$(2)$] If $X'$ is an object of $\calX'$ such that $Y' = p'(X')$ and $X = q'(X')$,
and $Y = p(X) = q(Y')$ are $\kappa$-compact, then $X'$ is a $\kappa$-compact object of $\calX'$.
\end{itemize}
\end{lemma}

\begin{proof}
Claim $(1)$ follows immediately from Lemma \ref{bird3}. To prove $(2)$, consider a colimit diagram
$\overline{f}: \calI^{\triangleright} \rightarrow \calX'$. We wish to prove that the composition of $\overline{f}$ with the functor $\calX' \rightarrow \hat{\SSet}$
corepresented by $X'$ is also a colimit diagram. Using Proposition \ref{rot}, we may assume without loss of generality that $\calI$ is the nerve of a $\kappa$-filtered partially ordered set $A$. We may further suppose that $p$ and $q$ are categorical fibrations and that
$\calX' = \calX \times_{\calY} \calY'$. Let $\calI^{\triangleright}_{X'/}$ denote the fiber product
$\calI^{\triangleright} \times_{\calX'} \calX_{X'/}$, and define
$\calI^{\triangleright}_{X/}$, $\calI^{\triangleright}_{Y'/}$, and $\calI^{\triangleright}_{Y/}$ similarly. We have a pullback diagram
$$ \xymatrix{ \calI^{\triangleright}_{X'/} \ar[r] \ar[d] & \calI^{\triangleright}_{X/} \ar[d] \\
\calI^{\triangleright}_{Y'/} \ar[r] & \calI^{\triangleright}_{Y/} }$$
of left fibrations over $\calI^{\triangleright}$. Proposition \ref{sharpen} implies that every arrow in this diagram is a left fibration, so that Corollary \ref{ruy} implies that $\calI^{\triangleright}_{X'/}$
is a homotopy fiber product of $\calI^{\triangleright}_{X/}$ with $\calI^{\triangleright}_{Y'/}$ over
$\calI^{\triangleright}_{Y/}$ in the covariant model category $(\sSet)_{/\calI^{\triangleright}}$. 
Let $G: (\sSet)^{A \cup \{ \infty \} } \rightarrow (\sSet)_{\calI^{\triangleright}}$ denote the 
unstraightening functor of \S \ref{contrasec}. Since $G$ is the right Quillen functor of a Quillen equivalence, the above diagram is weakly equivalent to the image under $G$ of a homotopy pullback diagram
$$ \xymatrix{ F_{X'} \ar[r] \ar[d] & F_{X} \ar[d] \\
F_{Y'} \ar[r] & F_{Y} } $$ 
of (weakly) fibrant objects of $(\sSet)^{A \cup \{ \infty \} }$. Moreover, the simplicial nerve of each $F_{Z}$ can be identified with the composition of $\overline{f}$ with the functor corepresented by $Z$. According to Theorem \ref{colimcomparee}, it will suffice to show that $F_{X'}$ is a homotopy colimit diagram. We now observe that $F_{X}$, $F_{Y'}$, and $F_{Y}$ are homotopy colimit diagrams (since $X$, $Y'$, and $Y$ are assumed to be $\kappa$-compact) and conclude by applying Lemma \ref{filterprod}.
\end{proof}

In some of the arguments below, it will be important to be able to replace colimits of
a diagram $\calJ \rightarrow \calC$ by colimits of some composition
$ \calI \stackrel{f}{\rightarrow} \calJ \rightarrow \calC$.
According to Proposition \ref{gute}, this maneuver is justified provided that $f$ is cofinal. Unfortunately, the class of cofinal morphisms is not sufficiently robust for our purposes. We will therefore introduce a property somewhat stronger than cofinality, which has better stability properties.

\begin{definition}\index{gen}{cofinal!weakly}\index{gen}{weakly cofinal}\index{gen}{$\kappa$-cofinal}
Let $f: \calI \rightarrow \calJ$ be a functor between filtered $\infty$-categories. We will say
that $f$ is {\it weakly cofinal} if, for every object $J \in \calJ$, there exists an object
$I \in \calI$ and a morphism $J \rightarrow f(I)$ in $\calJ$. We will say that $f$ is {\it $\kappa$-cofinal} if, for every diagram $p: K \rightarrow \calI$ where $K$ is $\kappa$-small and weakly contractible, the induced functor $\calI_{p/} \rightarrow \calJ_{f  p/}$ is weakly cofinal.
\end{definition}

\begin{example}\label{easex}
Let $\calI$ be a $\tau$-filtered $\infty$-category, and let $p: K \rightarrow \calI$ be a $\tau$-small diagram. Then the projection $\calI_{p/} \rightarrow \calI$ is $\tau$-cofinal. To prove this, consider
a $\tau$-small diagram $K' \rightarrow \calI_{p/}$ where $K'$ is weakly contractible, corresponding to a map $q: K \star K' \rightarrow \calI$. According to Lemma \ref{chotle2}, the inclusion
$K' \subseteq K \star K'$ is right anodyne, so that the map
$\calI_{q/} \rightarrow \calI_{q|K'/}$ is a trivial fibration (and therefore weakly cofinal).
\end{example}

\begin{lemma}\label{storuse}
Let $A$, $B$, and $C$ be simplicial sets, and suppose that $B$ is weakly contractible. Then the inclusion
$$ (A \star B) \coprod_{B} (B \star C) \subseteq A \star B \star C$$
is a categorical equivalence.
\end{lemma}

\begin{proof}
Let $F(A,B,C) = (A \star B) \coprod_{B} (B \star C)$, and let $G(A,B,C) = A \star B \star C$.
We first observe that both $F$ and $G$ preserve filtered colimits and homotopy pushout squares, separately in each argument. Using standard arguments (see, for example, the proof of Proposition \ref{babyy}), we can reduce to the case where $A$ and $C$ are simplices.

Let us say that a simplicial set $B$ is {\it good} if the inclusion
$F(A,B,C) \subseteq G(A,B,C)$ is a categorical equivalence. We now make the following observations:
\begin{itemize}
\item[$(1)$] Every simplex is good. Unwinding the definitions, this is equivalent to the assertion that
for $0 \leq m \leq n \leq p$, the diagram
$$ \xymatrix{ \Delta^{ \{m, \ldots n \} } \ar@{^{(}->}[r] \ar@{^{(}->}[d] & \Delta^{ \{0, \ldots, n\} } 
\ar@{^{(}->}[d] \\
\Delta^{ \{m, \ldots, p\} } \ar@{^{(}->}[r] & \Delta^{ \{0, \ldots, p\} }}$$ is a homotopy pushout square (with respect to the Joyal model structure). It suffices to check that the equivalent subdiagram
$$ \xymatrix{ \Delta^{ \{m, m+1\}} \coprod_{ \{m+1\} }\ldots \coprod_{ \{n-1\} }
\Delta^{ \{n-1, n\} } \ar@{^{(}->}[r] \ar@{^{(}->}[d] & \Delta^{ \{0,1\} } \coprod_{ \{1\} } \ldots \coprod_{ \{n-1\} }
\Delta^{ \{n-1,,n\} } \ar@{^{(}->}[d] \\
\Delta^{ \{ m, m+1 \}} \coprod_{ \{m+1\} } \ldots \coprod_{ \{n-1\} } \Delta^{ \{n-1, n\} } \ar@{^{(}->}[r] &
\Delta^{ \{0,1\} } \coprod_{ \{1\} } \ldots \coprod_{ \{p-1\} } \Delta^{ \{p-1, p\} } }$$
is a homotopy pushout, which is clear.

\item[$(2)$] Given a pushout diagram of simplicial sets
$$ \xymatrix{ B \ar[r] \ar@{^{(}->}[d] & B' \ar@{^{(}->}[d] \\
B'' \ar[r] & B''' }$$
in which the vertical arrows are cofibrations, if $B$, $B'$, and $B''$ are good, then $B'''$ is good. This follows from the compatibility of the functors $F$ and $G$ with homotopy pushouts in $B$.

\item[$(3)$] Every horn $\Lambda^n_i$ is good. This follows by induction on $n$, using $(1)$ and $(2)$. 

\item[$(4)$] The collection of good simplicial sets is stable under filtered colimits; this follows from the compatibility of $F$ and $G$ with filtered colimits, and the stability of categorical equivalences under filtered colimits.

\item[$(5)$] Every retract of a good simplicial set is good (since the collection of categorical equivalences is stable under the formation of retracts).

\item[$(6)$] If $i: B \rightarrow B'$ is an anodyne map of simplicial sets, and $B$ is good, then $B'$ is good. This follows by combining observations $(1)$ through $(5)$.

\item[$(7)$] If $B$ is weakly contractible, then $B$ is good. To see this, choose a vertex
$b$ of $B$. The simplicial set $\{b\} \simeq \Delta^0$ is good (by $(1)$ ), and the inclusion
$\{b\} \subseteq B$ is anodyne. Now apply $(6)$.
\end{itemize}
\end{proof}

\begin{lemma}\label{wolfpup}
Let $\kappa$ and $\tau$ be regular cardinals, let
$f: \calI \rightarrow \calJ$ be a $\kappa$-cofinal functor between $\tau$-filtered $\infty$-categories, and let $p: K \rightarrow \calJ$ be a $\kappa$-small diagram. Then:
\begin{itemize}
\item[$(1)$] The $\infty$-category $\calI_{p/} = \calI \times_{\calJ} \calJ_{p/}$ is $\tau$-filtered.

\item[$(2)$] The induced functor $\calI_{p/} \rightarrow \calJ_{p/}$ is $\kappa$-cofinal.
\end{itemize}

\end{lemma}

\begin{proof}
We first prove $(1)$. Let $\widetilde{q}: K' \rightarrow \calI_{p/}$ be a $\tau$-small diagram, classifying a compatible pair of maps $q: K' \rightarrow \calI$ and $q': K \star K' \rightarrow \calJ$.
Since $\calI$ is $\tau$-filtered, we can find an extension $\overline{q}: (K')^{\triangleright} \rightarrow \calI$ of $q$. To find a compatible extension of $\widetilde{q}$, it suffices to solve the lifting problem
$$ \xymatrix{ (K \star K') \coprod_{ K'} (K')^{\triangleright} \ar@{^{(}->}[d]^{i} \ar[r] & \calJ \\
(K \star K')^{\triangleright}, \ar@{-->}[ur] & } $$
which is possible since $i$ is a categorical equivalence (Lemma \ref{storuse}) and $\calJ$ is an $\infty$-category.

To prove $(2)$, we consider a map $\widetilde{q}: K' \rightarrow \calI_{p/}$ as above, where
now $K$ is $\kappa$-small and weakly contractible. 
We have a pullback diagram
$$ \xymatrix{ (\calI_{p/})_{\overline{q}/} \ar[r] \ar[d] & \calI_{q/} \ar[d] \\
\calJ_{q'/} \ar[r] & \calJ_{q'|K'/}. }$$
Lemma \ref{chotle2} implies that the inclusion $K' \subseteq K \star K'$ is right anodyne, 
so that the lower horizontal map is a trivial fibration. It follows that the upper horizontal map
is also a trivial fibration. Since $f$ is $\kappa$-cofinal, the right vertical map is weakly cofinal, so that the left vertical map is weakly cofinal as well.
\end{proof}

\begin{lemma}\label{cofinalwolf}
Let $\kappa$ be a regular cardinal, and let $f: \calI \rightarrow \calJ$ be an $\kappa$-cofinal map
of filtered $\infty$-categories. Then $f$ is cofinal.
\end{lemma}

\begin{proof}
According to Theorem \ref{hollowtt}, to prove that $f$ is cofinal it suffices to show that for every
object $J \in \calJ$, the fiber product $\calI_{J/} = \calI \times_{\calJ} \calJ_{J/}$ is weakly contractible. Lemma \ref{wolfpup} asserts that $\calI_{J/}$ is $\kappa$-filtered; now apply Lemma \ref{stull2}.
\end{proof}

\begin{lemma}\label{lemmatp}
Let $\kappa$ be a regular cardinal, let $\calC$ be an $\infty$-category which
admits $\kappa$-filtered colimits, let $\overline{p}: K^{\triangleright} \rightarrow \calC^{\tau}$ be a $\kappa$-small diagram in the $\infty$-category of $\kappa$-compact objects of $\calC$, and let
$p = \overline{p} | K$. Then $\overline{p}$ is a $\kappa$-compact object of $\calC_{p/}$. 
\end{lemma}

\begin{proof}
Let $\overline{p}'$ denote the composition
$$ K \diamond \Delta^0 \rightarrow K^{\triangleright} \stackrel{\overline{p}}{\rightarrow} \calC^{\kappa};$$ it will suffice to prove that $\overline{p}'$ is a $\tau$-compact object
of $\calC^{p/}$. Consider the pullback diagram
$$ \xymatrix{ \calC^{p/} \ar[r] \ar[d] & \Fun(K \times \Delta^1, \calC) \ar[d]^{f} \\
\ast \ar[r]^-{p} & \Fun(K \times \{0\}, \calC). } $$
Corollary \ref{tweezegork} implies that the $f$ is a Cartesian fibration, so we can apply Proposition \ref{basechangefunky} to deduce that the diagram is homotopy Cartesian (with respect to the Joyal model structure). Using Proposition \ref{limiteval}, we deduce that $f$ preserves $\kappa$-filtered colimits, and {\em any} functor $\ast \rightarrow \calD$ preserves filtered colimits (since filtered $\infty$-categories are weakly contractible; see \S \ref{quasilimit7}). Consequently, Lemma \ref{yoris} implies that $\overline{p}'$ is a $\kappa$-compact object of $\calC^{p/}$ provided that its images in $\ast$ and $\Fun(K \times \Delta^1, \calC)$ are $\kappa$-compact. The former condition is obvious, and the latter follows from Proposition \ref{placeabovee}.
\end{proof}

\begin{lemma}\label{sturm}
Let $\calC$ be an $\infty$-category which admits small, $\tau$-filtered colimits, and let
$p: K \rightarrow \calC$ be a small diagram. Then $\calC_{p/}$ admits small, $\tau$-filtered colimits.
\end{lemma}

\begin{proof}
Without loss of generality, we may suppose that $K$ is an $\infty$-category. Let $\calI$ be a $\tau$-filtered $\infty$-category and $q_0: \calI \rightarrow \calC_{p/}$ a diagram, corresponding to
a diagram $q: K \star \calI \rightarrow \calC$. We next observe that $K \star \calI$ is small and $\tau$-filtered, so that $q$ admits a colimit $\overline{q}: (K \star \calI)^{\triangleright}
\rightarrow \calC$. The map $\overline{q}$ can also be identified with a colimit of $q_0$.
\end{proof}

\begin{proposition}\label{accessforwardslice}\index{gen}{undercategory!and compact objects}
Let $\tau \gg \kappa$ be regular cardinals, let $\calC$ be a $\tau$-accessible $\infty$-category, and let $p: K \rightarrow \calC^{\tau}$ be a $\kappa$-small diagram. Then
$\calC_{p/}$ is $\tau$-accessible, and an object of $\calC_{p/}$ is $\tau$-compact if and only if
its image in $\calC$ is $\tau$-compact.
\end{proposition}

\begin{proof}
Let $\calD = \calC_{p/} \times_{\calC} \calC^{\tau}$ be the full subcategory of
$\calC_{p/}$ spanned by those objects whose image in $\calC$ is $\tau$-compact.
Since $\calC_{p/}$ is idempotent complete, and the collection of $\tau$-compact objects
of $\calC$ is stable under the formation of retracts, we conclude that $\calD$ is idempotent complete. We also note that $\calD$ is essentially small; replacing $\calC$ by a minimal model if necessary, we may suppose that $\calD$ is actually small. Proposition \ref{intprop} and
Lemma \ref{sturm} imply that there is an (essentially unique) $\tau$-continuous functor
$F: \Ind_{\tau}(\calD) \rightarrow \calC_{p/}$ such that the composition
$\calD \rightarrow \Ind_{\tau}(\calD) \stackrel{F}{\rightarrow} \calC_{p/}$ is equivalent to the inclusion of $\calD$ in $\calC_{p/}$. To complete the proof, it will suffice to show that
$F$ is an equivalence of $\infty$-categories. According to Proposition \ref{uterr}, it will suffice to show that $\calD$ consists of $\tau$-compact objects of $\calC_{p/}$ and generates
$\calC_{p/}$ under $\tau$-filtered colimits. The first assertion follows from Lemma \ref{lemmatp}. 

To complete the proof, choose an object $\overline{p}: K^{\triangleright} \rightarrow \calC$ of
$\calC_{p/}$, and let $C \in \calC$ denote the image under $\overline{p}$ of the cone point
of $K^{\triangleright}$. Then we may identify $\overline{p}$ with a diagram
$\widetilde{p}: K \rightarrow \calC^{\tau}_{/C}$. Since $\calC$ is $\tau$-accessible, the
$\infty$-category $\calE = \calC^{\tau}_{/C}$ is $\tau$-filtered. It follows that
$\calE_{\widetilde{p}/}$ is $\tau$-filtered and essentially small; to complete the proof, it will suffice to show that the associated map
$$ \calE_{\widetilde{p}/}^{\triangleright} \rightarrow \calC_{p/}$$
is a colimit diagram. Equivalently, we must show that the compositition
$$ K \star \calE_{\widetilde{p}/}^{\triangleright} \stackrel{\theta_0^{\triangleright}}{\rightarrow}
\calE^{\triangleright} \stackrel{\theta_1}{\rightarrow} \calC$$
is a colimit diagram. Since $\theta_1$ is a colimit diagram, it suffices to prove that $\theta_0$ is cofinal. For this, we consider the composition
$$ q: \calE_{\widetilde{p}/} \stackrel{i}{\rightarrow} K \star \calE_{\widetilde{p}/} \stackrel{\theta_0}{\rightarrow} \calE.$$
The $\infty$-category $\calE$ is $\tau$-filtered, so that $\calE_{\widetilde{p}/}$ is also
$\tau$-filtered, and therefore weakly contractible (Lemma \ref{stull2}). It follows that 
$i$ is right anodyne (Lemma \ref{chotle2}), and therefore cofinal. Applying Proposition \ref{cofbasic}, we conclude that $\theta_0$ is cofinal if and only if $q$ is cofinal. We now observe that that $q$ is $\tau$-cofinal (Example \ref{easex}) and therefore cofinal (Lemma \ref{cofinalwolf}).
\end{proof}

\begin{corollary}\label{horsemn}\index{gen}{undercategory!accessibility}\index{gen}{accessible!undercategories}
Let $\calC$ be an accessible $\infty$-category, and let $p:K \rightarrow \calC$ be a diagram indexed by a small simplicial set $K$. Then $\calC_{p/}$ is accessible.
\end{corollary}

\begin{proof}
Choose appropriate cardinals $\tau \gg \kappa$ and apply Proposition \ref{accessforwardslice}.
\end{proof}

\subsection{Accessibility of Fiber Products}\label{accessfiber}

Our goal in this section is to prove that the class of accessible $\infty$-categories is stable under (homotopy) fiber products (Proposition \ref{horse2}). The strategy of proof should now be familiar from \S \ref{accessfunk} and \S \ref{accessprime}. Suppose given a homotopy Cartesian diagram
 $$ \xymatrix{ \calX' \ar[r]^{q'} \ar[d]^{p'} & \calX \ar[d]^{p} \\
\calY' \ar[r]^{q} & \calY }$$
of $\infty$-categories, where $\calX$, $\calY'$, and $\calY$ are accessible $\infty$-categories, and the functors $p$ and $q$ are likewise accessible. If $\kappa$ is a sufficiently large regular cardinal, then we can use Lemma \ref{yoris} to produce a good supply of $\kappa$-compact objects of $\calX'$. Our problem is then to prove that these objects generate $\calX'$ under $\kappa$-filtered colimits. This requires some rather delicate cofinality arguments.

\begin{lemma}\label{supwolf}
Let $\tau \gg \kappa$ be regular cardinals, let $f: \calC \rightarrow \calD$ be a $\tau$-continuous
functor between $\tau$-accessible $\infty$-categories which carries $\tau$-compact objects
of $\calC$ to $\tau$-compact objects of $\calD$. Let $C$ be an object of $\calC$, $\calC^{\tau}_{/C}$ the full subcategory of $\calC_{/C}$ spanned by those objects $C' \rightarrow C$ where
$C'$ is $\tau$-compact, and $\calD^{\tau}_{/f(C)}$ the full subcategory spanned by those
objects $D \rightarrow f(C)$ where $D \in \calD$ is $\tau$-compact. Then $f$ induces
a $\kappa$-cofinal functor $f': \calC^{\tau}_{/C} \rightarrow \calD^{\tau}_{/f(C)}$. 
\end{lemma}

\begin{proof}
Let $\widetilde{p}: K \rightarrow \calC^{\tau}_{/C}$ be a diagram indexed by a $\tau$-small, weakly contractible simplicial set $K$, and let $p: K \rightarrow \calC$ be the underlying map.
We need to show that
the induced functor $(\calC^{\tau}_{/C})_{\widetilde{p}/} \rightarrow (\calD^{\tau}_{/f(C)})_{f'  \widetilde{p}/}$ is weakly cofinal. Using Proposition \ref{accessforwardslice}, we may replace $\calC$ by $\calC_{p/}$ and
$\calD$ by $\calD_{f  p/}$, and thereby reduce to the problem of showing that $f$ is weakly cofinal.
Let $\phi: D \rightarrow f(C)$ be an object of $\calD^{\tau}_{/f(C)}$, and let
$F_{D}: \calD \rightarrow \SSet$ be the functor corepresented by $D$. Since $D$ is
$\tau$-compact, the functor $F_{D}$ is $\tau$-continuous, so that $F_{D} \circ f$
is $\tau$-continuous. Consequently, the space $F_{D}(f(C))$ can be obtained as a colimit
of the $\tau$-filtered diagram
$$p: \calC^{\tau}_{/C} \rightarrow \calD^{\tau}_{/f(C)} \rightarrow \calD \stackrel{F_{D}}{\rightarrow} \SSet.$$
In particular, the path component of $F_{D}(f(C))$ containing $\phi$ lies in the image
of $p(\eta)$, for some $\eta: C' \rightarrow C$ as above. It follows that there exists a commutative diagram
$$ \xymatrix{ D \ar[rr]^{\phi} \ar[dr] & & f(C) \\
& f(C') \ar[ur]^{f(\eta)} & }$$
in $\calD$, which can be identified with a morphism in $\calD^{\tau}_{/f(C)}$ having the desired properties.
\end{proof}

\begin{lemma}\label{pup1}
Let $A = A' \cup \{ \infty\}$ be a linearly ordered set containing a largest element $\infty$, and let
$B \subseteq A'$ be a cofinal subset $($in other words, for every $\alpha \in A'$, there
exists $\beta \in B$ such that $\alpha \leq \beta${}$)$. The inclusion
$$\phi: \Nerve(A') \coprod_{ \Nerve(B) } \Nerve ( B \cup \{ \infty\} ) \subseteq \Nerve(A)$$
is a categorical equivalence.
\end{lemma}

\begin{proof}
For each $\beta \in B$, let $\phi_{\beta}$ denote the inclusion
$$ \Nerve(\{ \alpha \in A': \alpha \leq \beta\}) \coprod_{ \Nerve(\{ \alpha \in B: \alpha \leq \beta \}) }
\Nerve(\{ \alpha \in B: \alpha \leq \beta \} \cup \{ \infty \} ) \subseteq 
\Nerve (\{ \alpha \in A': \alpha \leq \beta \} \cup \{\infty\} ).$$
Since $B$ is cofinal in $A'$, $\phi$ is a filtered colimit of the inclusions 
$\phi_{\beta}$. Replacing $A'$ by $\{ \alpha \in A' : \alpha \leq \beta \}$ and
$B$ by $\{ \alpha \in B: \alpha \leq \beta \}$, we may reduce to the case where
$A'$ has a largest element (which we will continue to denote by $\beta$).

We have a categorical equivalence
$$ \Nerve(B) \coprod_{ \{ \beta \} } \Nerve(\{ \beta, \infty \}) \subseteq \Nerve (B \cup \{\infty\} ).$$
Consequently, to prove that $\phi$ is a categorical equivalence, it will suffice to show that the composition
$$ \Nerve(A') \coprod_{ \{\beta \} } \Nerve(\{ \beta, \infty\}) 
\subseteq \Nerve(A') \coprod_{ \Nerve(B) } \Nerve ( B \cup \{ \infty\} ) \subseteq \Nerve(A)$$
is a categorical equivalence, which is clear. 
\end{proof}

\begin{lemma}\label{remuswolf}
Let $\tau > \kappa$ be regular cardinals, and let
$$ \calX \stackrel{p}{\rightarrow} \calY \stackrel{p'}{\leftarrow} \calX'$$
be functors between $\infty$-categories.
Assume that:
\begin{itemize}
\item[$(1)$] The $\infty$-categories $\calX, \calX'$, and $\calY$ are $\kappa$-filtered, and
admit $\tau$-small, $\kappa$-filtered colimits.
\item[$(2)$] The functors $p$ and $p'$ preserve $\tau$-small, $\kappa$-filtered colimits.
\item[$(3)$] The functors $p$ and $p'$ are $\kappa$-cofinal.
\end{itemize}
Then there exist objects $X \in \calX$, $X' \in \calX'$ such that $p(X)$ and $p'(X')$ are equivalent in $\calY$.
\end{lemma}

\begin{proof}
For every ordinal $\alpha$, we let $[\alpha] = \{ \beta : \beta \leq \alpha \}$ and
$(\alpha) = \{ \beta: \beta < \alpha \}$. 
Let us say that an ordinal $\alpha$ is {\em even} if it is of the form $\lambda + n$, where
$\lambda$ is a limit ordinal and $n$ is an even integer; otherwise we will say that $\alpha$ is {\it odd}. Let $A$ denote the set of all even ordinals smaller than $\kappa$, and $A'$ the set of all odd ordinals smaller than $\kappa$. We regard $A$ and $A'$ as subsets of the linearly ordered set $A \cup A' = (\kappa)$. We will construct a commutative diagram
$$ \xymatrix{ \Nerve(A) \ar[r] \ar[d]^{q} & \Nerve (\kappa) \ar[d]^{Q} & \Nerve(A') \ar[d]^{q'} \ar[l] \\
\calX \ar[r]^{p} & \calY & \calX' \ar[l]^{p'}. }$$
Supposing that this is possible, we choose colimits $X \in \calX$, $X' \in \calX'$, and $Y \in \calY$ for $q$, $q'$, and $Q$, respectively. Since the inclusion $\Nerve(A) \subseteq \Nerve (\kappa)$ is cofinal and $p$-preserves $\kappa$-filtered colimits, we conclude that $p(X)$ and $Y$ are equivalent.
Similarly, $p'(X')$ and $Y$ are equivalent, so that $p(X)$ and $p'(X')$ are equivalent, as desired.

The construction of $q$, $q'$, and $Q$ is given by induction. Let $\alpha < \kappa$, and suppose
that $q| \Nerve(\{ \beta \in A: \beta < \alpha \})$, $q' | \Nerve(\{ \beta \in A': \beta < \alpha \})$
and $Q| \Nerve (\alpha)$ have already been constructed. We will show how to extend the definitions of $q$, $q'$, and $Q$ to include the ordinal $\alpha$. We will suppose that $\alpha$ is even; the case where $\alpha$ is odd is similar (but easier). 

Suppose first that $\alpha$ is a limit ordinal. In this case, define
$q | \Nerve(\{ \beta \in A: \beta \leq \alpha \})$ to be an arbitrary extension of
$q | \Nerve(\{ \beta \in A: \beta < \alpha \})$: such an extension exists in virtue of our assumption that $\calX$ is $\kappa$-filtered. In order to define $Q | \Nerve (\alpha)$
it suffices to verify that $\calY$ has the extension property with respect to the inclusion
$$ \Nerve (\alpha) \coprod_{ \Nerve (\{ \beta \in A: \beta < \alpha \}) }
\Nerve (\{ \beta \in A: \beta \leq \alpha \}) \subseteq \Nerve[\alpha]. $$
Since $\calY$ is an $\infty$-category, this follows immediately from Lemma \ref{pup1}.

We now treat the case where $\alpha = \alpha' + 1$ is a successor ordinal. Let
$q_{< \alpha} = q | \{ \beta \in A: \beta < \alpha\}$, and regard
$Q | \Nerve ( \{ \alpha' \} \cup \{ \beta \in A: \beta < \alpha \} )$ as an object of
$\calY_{f  q_{< \alpha} /}$. We now observe that $\Nerve (\{ \beta \in A: \beta < \alpha\})$ is $\kappa$-small and weakly contractible. Since $p$ is $\kappa$-cofinal, we can construct
$q | \{ \beta \in A: \beta \leq \alpha\}$ extending $q_{< \alpha}$ and a compatible map
$Q | \Nerve ( \{ \alpha' \} \cup \{ \beta \in A: \beta \leq \alpha \} )$. To complete the construction of $Q$, it suffices to show that $\calY$ has the extension property with respect to the inclusion
$$ \Nerve (\alpha)  \coprod_{ \Nerve (\{ \beta \in A: \beta < \alpha \} \cup \{ \alpha' \}) }
\Nerve (\{ \beta \in A: \beta \leq \alpha \} \cup \{\alpha' \}) \subseteq \Nerve [\alpha]. $$ Once again, this follows from Lemma \ref{pup1}.
\end{proof}

\begin{lemma}\label{wolfpup2}
Let $\kappa$ and $\tau$ be regular cardinals, let $f: \calI \rightarrow \calJ$ be a $\kappa$-cofinal functor between $\tau$-filtered $\infty$-categories, and let $p: K \rightarrow \calI$ be a diagram indexed by a $\tau$-small simplicial set $K$. Then the induced functor
$$ \calI_{p/} \rightarrow \calJ_{f  p/}$$ is $\kappa$-cofinal.
\end{lemma}

\begin{proof}
Let $K'$ be a simplicial set which is $\kappa$-small and weakly contractible, and let
$q: K \star K' \rightarrow \calI$ be a diagram. We have a commutative diagram
$$ \xymatrix{ \calI_{q/} \ar[r] \ar[d] & \calJ_{f  q/} \ar[d] \\
\calI_{q|K' / } \ar[r] & \calI_{f  q|K' /}. }$$
Lemma \ref{chotle2} implies that $K' \subseteq K \star K'$ is a right anodyne inclusion, so that the vertical maps are trivial fibrations. Since $f$ is $\kappa$-cofinal, the lower horizontal map is
weakly cofinal; it follows that the upper horizontal map is weakly cofinal as well.
\end{proof}

\begin{lemma}\label{rebuswolf}
Let $\tau > \kappa$ be regular cardinals, and let
$$ \xymatrix{ \calI' \ar[r]^{q'} \ar[d]^{p'} & \calI \ar[d]^{p} \\
\calJ' \ar[r]^{q} & \calJ }$$
a diagram of $\infty$-categories which is homotopy Cartesian $($with respect to the Joyal model structure$)$. Suppose that $\calI$, $\calJ$, and $\calJ'$ are $\tau$-filtered $\infty$-categories which admit $\tau$-small, $\kappa$-filtered colimits. Suppose further that $p$ and $q$ are $\kappa$-cofinal functors which preserve $\tau$-small, $\kappa$-filtered colimits. Then $\calI'$ is $\tau$-filtered, and the functors $p'$ and $q'$ are $\kappa$-cofinal.
\end{lemma}

\begin{proof}
Without loss of generality, we may suppose that $p$ and $q$ are categorical fibrations and that
$\calI' = \calI \times_{\calJ} \calJ'$. To prove that $\calI'$ is $\tau$-filtered, we must show that
$\calI'_{f/}$ is nonempty for every diagram $f: K \rightarrow \calI'$ indexed by a $\tau$-small simplicial set $K$. We have a (homotopy) pullback diagram
$$ \xymatrix{ \calI'_{f/} \ar[r] \ar[d] & \calI_{q'  f/} \ar[d]^{g} \\
\calJ'_{p'  f/} \ar[r]^{h} & \calJ_{p  q'  f/}. }$$
Lemma \ref{forfilt} implies that $\calI_{q'  f/}$, $\calJ'_{p'  f/}$, and
$\calJ_{p  q'  f/}$ are $\tau$-filtered, and Lemma \ref{wolfpup2} implies that
$g$ and $h$ are $\kappa$-cofinal. We may therefore apply Lemma \ref{remuswolf} to deduce
that $\calI'_{f/}$ is nonempty, as desired.

We now prove that $q'$ is $\kappa$-cofinal; the analogous assertion for $p'$ is proven by the same argument. We must show that for every diagram $f: K \rightarrow \calI'$, where $K$ is $\kappa$-small and weakly contractible, the induced map
$\calI'_{f/} \rightarrow \calI_{q'  f/}$ is weakly cofinal. Replacing $\calI'$ by $\calI'_{f/}$ as above, we
may reduce to the problem of showing that $q'$ itself is weakly cofinal. Let $I$ be an object of
$\calI$, let $J = p(I) \in \calJ$, and consider the (homotopy) pullback diagram
$$ \xymatrix{ \calI'_{I/} \ar[r] \ar[d] & \calI_{I/} \ar[d]^{u} \\
\calJ'_{J/} \ar[r]^{v} & \calJ_{J/}. }$$
We wish to show that $\calI'_{I/}$ is nonempty. This follows from Lemma \ref{remuswolf}, since
$u$ and $v$ are $\tau$-cofinal by Lemmas \ref{wolfpup2} and \ref{wolfpup}, respectively.
\end{proof}

\begin{proposition}\label{horse2}\index{gen}{accessible!homotopy fiber products}
Let $$ \xymatrix{ \calX' \ar[r]^{q'} \ar[d]^{p'} & \calX \ar[d]^{p} \\
\calY' \ar[r]^{q} & \calY }$$
be a diagram of $\infty$-categories which is homotopy Cartesian (with respect to the Joyal model structure). Suppose further that $\calX$, $\calY$, and $\calY'$ are accessible, and that
$p$ and $q$ are accessible functors. Then $\calX'$ is accessible. Moreover, for any accessible $\infty$-category $\calC$ and any functor $f: \calC \rightarrow \calX$, $f$ is accessible if and only if the compositions $p' \circ f$ and $q' \circ f$ are accessible. In particular $($taking $f = \id_{\calX}${}$)$, the functors $p'$ and $q'$ are accessible.
\end{proposition}

\begin{proof}
Choose a regular cardinal $\kappa$ such that $\calX$, $\calY'$, and $\calY$ are $\kappa$-accessible. Enlarging $\kappa$ if necessary, we may suppose that $p$ and $q$ are $\kappa$-continuous. It follows from Lemma \ref{bird3} that $\calX'$ admits small $\kappa$-filtered
colimits, and that for any $\kappa' > \kappa$, a functor $f: \calC \rightarrow \calX$ is
$\kappa'$-continuous if and only if $p' \circ f$ and $q' \circ f$ are $\kappa'$-continuous.
This proves the second claim; it now suffices to show that $\calX'$ is accessible. For this, we will use characterization $(3)$ of Proposition \ref{clear}. Without loss of generality, we may suppose
that $p$ and $q$ are categorical fibrations, and that $\calX' = \calX \times_{\calY} \calY'$. It then follows easily that $\calX'$ is locally small. It will therefore suffice to show that there exists
a regular cardinal $\tau$ such that $\calX'$ is generated by a small collection of
$\tau$-compact objects under small, $\tau$-filtered colimits.

Since the $\infty$-categories of $\kappa$-compact objects of $\calX$ and $\calY'$ are essentially small, there exists $\tau > \kappa$ such that $p | \calX^{\kappa} \subseteq
\calY^{\tau}$ and $q| {\calY'}^{\kappa} \subseteq \calY^{\tau}$. Enlarging $\tau$ if necessary, we may suppose that $\tau \gg \kappa$. The proof of Proposition \ref{enacc} shows that
every $\tau$-compact object of $\calX$ can be written as a $\tau$-small, $\kappa$-filtered colimit of objects belonging to $\calX^{\kappa}$. Since $p$ is $\kappa$-continuous, it follows that
$p | \calX^{\tau} \subseteq \calY^{\tau}$ and similarly $q| {\calY'}^{\tau} \subseteq \calY^{\tau}$.
Let $\calX'' = \calX^{\tau} \times_{ \calY^{\tau} } {\calY'}^{\tau}$. Then $\calX''$ is an essentially small, full subcategory of $\calX'$. Lemma \ref{yoris} implies that $\calX''$ consists of $\tau$-compact objects of $\calX'$. To complete the proof, it will suffice to show that $\calX''$ generates $\calX'$ under small $\tau$-filtered colimits.

Let $X' = (X,Y')$ be an object of $\calX'$, and let $Y = pX = qY'$. We have a (homotopy) pullback
diagram
$$ \xymatrix{ \calX''_{/X'} \ar[r]^{f'} \ar[d]^{g'} & \calX^{\tau}_{/X} \ar[d]^{g} \\
{\calY'}^{\tau}_{/Y'} \ar[r]^{f} & \calY^{\tau}_{/Y} }$$
of essentially small $\infty$-categories. Lemma \ref{supwolf} asserts that $f$ and $g$ are $\kappa$-cofinal.
We apply Lemma \ref{rebuswolf} to conclude that $\calX''_{/X'}$ is $\tau$-filtered, and that 
$f'$ and $g'$ are $\kappa$-cofinal. Now consider the diagram
$$ \xymatrix{ (\calX''_{/X'})^{\triangleright} \ar[dd] \ar[dr]^{h} \ar[rr] & & (\calX^{\tau}_{/X})^{\triangleright} \ar[d] \\
& \calX' \ar[r]^{q'} \ar[d]^{p'} & \calX \ar[d]^{p} \\
({\calY'}^{\tau}_{/Y})^{\triangleright} \ar[r]^-{q} & \calY' \ar[r] & \calY. }$$
Lemma \ref{cofinalwolf} allows us to conclude that $f'$ and $g'$ are cofinal, so that
$p' \circ h$ and $q' \circ h$ are colimit diagrams. Lemma \ref{bird3} implies that $h$ is a colimit diagram as well,
so that $X'$ is the colimit of an essentially small, $\tau$-filtered diagram taking values in $\calX''$.
\end{proof}

\begin{corollary}\label{horsemuun}\index{gen}{overcategory!accessible}\index{gen}{accessible!overcategories}
Let $\calC$ be an accessible $\infty$-category, and let $p: K \rightarrow \calC$ be a diagram indexed by a small simplicial set $K$. Then the $\infty$-category $\calC_{/p}$ is accessible. 
\end{corollary}

\begin{proof}
Since the map $\calC_{/p} \rightarrow \calC^{/p}$ is a categorical equivalence, it will suffice to prove that $\calC^{/p}$ is accessible. We have a pullback diagram
$$ \xymatrix{ \calC^{/p} \ar[r] \ar[d] & \Fun(K \times \Delta^1, \calC) \ar[d]^{p} \\
\ast \ar[r]^-{q} & \Fun(K \times \{1\}, \calC) }$$ 
of $\infty$-categories. Since $p$ is a coCartesian fibration, Proposition \ref{basechangefunky} implies that this diagram is homotopy Cartesian. According to Proposition \ref{horse1},
the $\infty$-categories $\calC^{K \times \Delta^1}$ and $\calC^{K \times \{1\} }$ are accessible. Using Proposition \ref{limiteval}, we conclude that for every regular cardinal $\kappa$ such
that $\calC$ admits $\kappa$-filtered colimits, $p$ is $\kappa$-continuous; in particular, $p$ is accessible. Corollary \ref{sloam} implies that $\ast$ is accessible and that $q$ is an accessible functor. Applying Proposition \ref{horse2}, we deduce that $\calC^{/p}$ is accessible.
\end{proof}

\subsection{Applications}\label{accessstable}

In \S \ref{accessfunk} through \S \ref{accessfiber}, we established some of the basic stability properties enjoyed by the class of accessible $\infty$-categories. In this section, we will reap some of the rewards for our hard work.

\begin{lemma}\label{simplehorse}\index{gen}{accessible!coproducts}
Let $\{ \calC_{\alpha} \}_{\alpha \in A}$ be a family of $\infty$-categories indexed by
a small set $A$, and let $\calC = \coprod_{ \alpha \in A} \calC_{\alpha}$ be their coproduct. Then
$\calC$ is an accessible if and only if each $\calC_{\alpha}$ is accessible.
\end{lemma}

\begin{proof}
Immediate from the definitions.
\end{proof}

\begin{lemma}\label{complexhorse}\index{gen}{accessible!products}
Let $\{ \calC_{\alpha} \}_{ \alpha \in A}$ be a family of $\infty$-categories indexed by a small
set $A$, and let $\calC = \prod_{\alpha \in A} \calC_{\alpha}$ be their product. If each
$\calC_{\alpha}$ is accessible, then $\calC$ is accessible. Moreover, if $\calD$ is an accessible $\infty$-category, then a functor $\calD \rightarrow \calC$ is accessible if and only if each of the compositions
$$ \calD \rightarrow \calC \rightarrow \calC_{\alpha}$$
is accessible.
\end{lemma}

\begin{proof}
Let $\calD = \coprod_{\alpha \in A} \calC_{\alpha}$. By Lemma \ref{simplehorse}, $\calD$ is accessible. Let $\Nerve(A)$ denote the constant simplicial set with value $A$.
Proposition \ref{horse1} implies that $\Fun(\Nerve(A), \calD)$ is accessible. We now observe that
$\Fun(\Nerve(A), \calD)$ can be written as a disjoint union of $\calC$ with another $\infty$-category; applying Lemma \ref{simplehorse} again, we deduce that $\calC$ is accessible. The second claim follows immediately from the definitions.
\end{proof}

\begin{proposition}\label{accprop}\index{gen}{limit!of accessible $\infty$-categories}
The $\infty$-category $\Acc$ of accessible $\infty$-categories admits small limits, and the inclusion $i: \Acc \subseteq \widehat{ \Cat}_{\infty}$ preserves small limits.
\end{proposition}

\begin{proof}
By Proposition \ref{alllimits}, it suffices to prove that $\Acc$ admits pullbacks and small products, and that $i$ preserves pullbacks and (small) products. Let $\Acc_{\Delta}$ be the (simplicial) subcategory of $\widehat{\sSet}$ defined as follows:
\begin{itemize}
\item[$(1)$] The objects of $\Acc_{\Delta}$ are the accessible $\infty$-categories.
\item[$(2)$] If $\calC$ and $\calD$ are accessible $\infty$-categories, then
$\bHom_{ \Acc_{\Delta} }( \calC, \calD)$ is the subcategory of $\Fun(\calC,\calD)$ whose
objects are accessible functors, and whose morphisms are {\em equivalences} of functors.
\end{itemize}
The $\infty$-category $\Acc$ is isomorphic to the simplicial nerve
$\sNerve( \Acc_{\Delta} )$. 
In view of Theorem \ref{colimcomparee}, it will suffice to prove that the simplicial category $\Acc_{\Delta}$ admits homotopy fiber products
and (small) homotopy products, and that the inclusion $\Acc_{\Delta} \subseteq
( \widehat{\mSet} )^{\degree}$ preserves
homotopy fiber products and homotopy products. The case of homotopy fiber products follows from Proposition \ref{horse2} and the case of (small) homotopy products follows from Lemma \ref{complexhorse}.
\end{proof}

If $\calC$ is an accessible $\infty$-category, then $\calC$ is the union of full subcategories
$\{ \calC^{\tau} \subseteq \calC \}$, where $\tau$ ranges over all (small) regular cardinals. It seems reasonable to expect that if $\tau$ is sufficiently large, then the properties of $\calC$ are mirrored by properties of $\calC^{\tau}$. The following result provides an illustration of this philosophy:

\begin{proposition}\label{tcoherent}\index{gen}{compact object!limits of}
Let $\calC$ be a $\kappa$-accessible $\infty$-category, and let $\tau \gg \kappa$ be an uncountable regular cardinal such that $\calC^{\kappa}$ is essentially $\tau$-small. Then the full subcategory $\calC^{\tau} \subseteq \calC$ is stable under all $\kappa$-small limits which exist in $\calC$.
\end{proposition}

Before giving the proof, we will need to establish a few lemmas.

\begin{lemma}\label{thinman}
Let $\tau \gg \kappa$ be regular cardinals, and assume $\tau$ is uncountable. 
Let $\calC$ be a $\tau$-small $\infty$-category, and let $D$ be an object of $\Ind_{\kappa}(\calC)$. The following are equivalent:
\begin{itemize}
\item[$(1)$] The object $D$ is $\tau$-compact in $\Ind_{\kappa}(\calC)$. 
\item[$(2)$] For every $C \in \calC$, the space
$\bHom_{\Ind_{\kappa}(\calC)}(j(C),D)$ is essentially $\tau$-small, where
$j: \calC \rightarrow \Ind_{\kappa}(\calC)$ denotes the Yoneda embedding.
\end{itemize}
\end{lemma}

\begin{proof}
Suppose first that $(1)$ is satisfied. Using Lemma \ref{longwait0}, we can write $D$
as the colimit of the $\kappa$-filtered diagram
$$ \calC_{/D} = \calC \times_{ \Ind_{\kappa}(\calC)} \Ind_{\kappa}(\calC)_{/D} \rightarrow \Ind_{\kappa}(\calC).$$
Since $\tau \gg \kappa$, we also write $D$ as a small $\tau$-filtered colimit of objects
$\{ D_{\alpha} \}$, where each $D_{\alpha}$ is the colimit of a $\tau$-small, $\kappa$-filtered
diagram $$ \widetilde{\calC} \rightarrow \calC \rightarrow \Ind_{\kappa}(\calC).$$
Since $D$ is $\tau$-compact, we conclude that $D$ is a retract of $D_{\alpha}$. Let
$F: \Ind_{\kappa}(\calC) \rightarrow \SSet$ denote the functor co-represented by $j(C)$.
According to Proposition \ref{justcut}, $F$ is $\kappa$-continuous. It follows that
$F(D)$ is a retract of $F(D_{\alpha})$, which is itself a $\tau$-small colimit of spaces equivalent
to $\bHom_{\Ind_{\kappa}(\calC)}(j(C), j(C')) \simeq \bHom_{\calC}(C,C')$, which is essentially $\tau$-small by assumption, and therefore a $\tau$-compact object of $\SSet$. It follows that $D$ is also $\tau$-compact object of $\SSet$. 

Now assume $(2)$. Once again, we observe that $D$ can be obtained as the colimit of a diagram
$\calC_{/D} \rightarrow \Ind_{\kappa}(\calC)$. By assumption, $\calC$ is $\tau$-small and the fibers of the right fibration $\calC_{/D} \rightarrow \calC$ are essentially $\tau$-small. Proposition \ref{sumt} implies that $\calC_{/D}$ is essentially $\tau$-small, so that $D$ is a
$\tau$-small colimit of $\kappa$-compact objects of $\Ind_{\kappa}(\calC)$ and therefore $\tau$-compact.
\end{proof}

\begin{lemma}\label{paleman}
Let $\tau \gg \kappa$ be regular cardinals such that $\tau$ is uncountable, and let
$\SSet^{\tau}$ be the full subcategory of $\SSet$ consisting of essentially $\tau$-small spaces.
Then $\SSet^{\tau}$ is stable under $\kappa$-small limits in $\SSet$.
\end{lemma}

\begin{proof}
In view of Proposition \ref{alllimits}, it suffices to prove that $\SSet^{\tau}$ is stable under pullbacks and $\kappa$-small products. Using Theorem \ref{colimcomparee}, it will suffice to show that the full subcategory of $\Kan$ spanned by essentially $\tau$-small spaces is stable under $\kappa$-small products and homotopy fiber products. This follows immediately from characterization $(1)$ given in Proposition \ref{apegrape}.
\end{proof}

\begin{proof}[Proof of Proposition \ref{tcoherent}]
Let $K$ be a $\kappa$-small simplicial set and let $p: K \rightarrow \calC^{\tau}$ be a diagram
which admits a limit $X \in \calC$. We wish to show that $X$ is $\tau$-compact. According to
Lemma \ref{thinman}, it suffices to prove that the space $F(X)$ is essentially $\tau$-small, where
$F: \calC \rightarrow \SSet$ denotes the functor co-represented by a $\kappa$-compact object $C \in \calC$. Since $F$ preserves limits, we note that $F(X)$ is a limit of $F \circ p$. Lemma \ref{thinman} implies that the diagram $F \circ p$ takes values in $\SSet^{\tau} \subseteq \SSet$.
We now conclude by applying Lemma \ref{paleman}.
\end{proof}

We note the following useful criterion for establishing that a functor is accessible:

\begin{proposition}\label{adjoints}\index{gen}{accessible!adjoint functors}
Let $G: \calC \rightarrow \calC'$ be a functor between accessible
$\infty$-categories. If $G$ admits a right or a left adjoint, then $G$ is
accessible.
\end{proposition}

\begin{proof}
If $G$ is a left adjoint, then $G$ commutes with all colimits which exist in $\calC$.
Therefore $G$ is $\kappa$-continuous for any cardinal $\kappa$ having the property that
$\calC$ is $\kappa$-accessible. Let us therefore assume that $G$ is a right
adjoint; choose a left adjoint $F$ for $G$.

Choose a regular cardinal $\kappa$ such that $\calC'$ is $\kappa$-accessible. We may suppose without loss of generality that $\calC' = \Ind_{\kappa} \calD$, where $\calD$ is a small $\infty$-category. Consider the composite functor
$$ \calD \stackrel{j}{\rightarrow} \Ind_{\kappa}(\calD) \stackrel{F}{\rightarrow} \calC.$$
Since $\calD$ is small, there exists a regular cardinal $\tau \gg \kappa$ such that
$\calC$ is $\tau$-accessible and the essential image of $F \circ j$ consists of $\tau$-compact
objects of $\calC$. We will show that $G$ is $\tau$-continuous.

Since $\Ind_{\kappa}(\calD) \subseteq \calP(\calD)$ is stable under small $\tau$-filtered colimits, it will suffice to prove that the composition
$$ G': \calC \stackrel{G}{\rightarrow} \Ind_{\kappa}(\calD) \rightarrow \calP(\calD)$$
is $\tau$-continuous. For each object $D \in \calD$, let $G'_{D}: \calC \rightarrow \hat{\SSet}$
denote the composition of $G'$ with the functor given by evaluation at $D$. According to Proposition \ref{limiteval}, it will suffice to show that each $G'_{D}$ is $\tau$-continuous.
Lemma \ref{repco} implies that $G'_{D}$ is equivalent to the composition of $G$ with
the functor $\calC' \rightarrow \hat{\SSet}$ corepresented by $j(D)$. Since $F$ is left adjoint to $G$, we may identify this with the functor corepresented by $F(j(D))$. Since $F(j(D))$ is $\tau$-compact by construction, this functor is $\tau$-continuous.
\end{proof}

\begin{definition}\label{defaccsub}\index{gen}{accessible!subcategory}
Let $\calC$ be an accessible category. A full subcategory $\calD \subseteq \calC$ is
an {\it accessible subcategory} of $\calC$ if $\calD$ is accessible, and the inclusion
of $\calD$ into $\calC$ is an accessible functor.
\end{definition}

\begin{example}\label{colexam}
Let $\calC$ be an accessible $\infty$-category and $K$ a simplicial set. Suppose that
every diagram $K \rightarrow \calC$ has a limit in $\calC$. Let
$\calD \subseteq \Fun(K^{\triangleleft}, \calC) $ be the full subcategory spanned by the
limit diagrams. Then $\calD$ is equivalent to $\Fun(K,\calC)$, and is therefore accessible (Proposition \ref{horse1}). The inclusion $\calD \subseteq \Fun(K^{\triangleleft}, \calC)$ is a right adjoint, and therefore accessible (Proposition \ref{adjoints}). Thus $\calD$ is an accessible subcategory of
$\Fun(K^{\triangleleft}, \calC)$. 
Similarly, if every diagram $K \rightarrow \calC$
has a colimit, then the full subcategory $\calD' \subseteq 
\Fun(K^{\triangleright}, \calC)$ spanned by the colimit diagrams in an accessible subcategory of $\Fun(K^{\triangleright}, \calC)$. 
\end{example}

\begin{proposition}\label{boundint}
Let $\calC$ be an accessible category, and let $\{ \calD_{\alpha} \subseteq \calC \}_{\alpha \in A}$
be a (small) collection of accessible subcategories of $\calC$. Then
$\bigcap_{\alpha \in A} \calD_{\alpha}$ is an accessible subcategory of $\calC$.
\end{proposition}

\begin{proof}
We have a homotopy Cartesian diagram
$$ \xymatrix{ \bigcap_{\alpha \in A} \calD_{\alpha} \ar[r]^-{i'} \ar[d] & \calC \ar[d]^{f} \\
\prod_{\alpha \in A} \calD_{\alpha} \ar[r]^-{i} & \calC^{A}. }$$
Lemma \ref{complexhorse} implies that $\prod_{\alpha \in A} \calD_{\alpha}$ and
$\calC^{A}$ are accessible, and it is easy to see that $f$ and $i$ are accessible functors. Applying Proposition \ref{horse2}, we conclude that $\bigcap_{\alpha \in A} \calD_{\alpha}$ is accessible, and that $i'$ is an accessible functor, as desired.
\end{proof}

We conclude this chapter by establishing a generalization of Proposition \ref{horse1}.

\begin{proposition}\label{prestorkus}
Let $\calC$ be a subcategory of the $\infty$-category $\widehat{\Cat}_{\infty}$ of $($not necessarily small$)$ $\infty$-categories satisfying the following conditions:
\begin{itemize}
\item[$(a)$] The $\infty$-category $\calC$ admits small limits, and the inclusion
$\calC \subseteq \widehat{\Cat}_{\infty}$ preserves small limits.
\item[$(b)$] If $X$ belongs to $\calC$, then $\Fun(\Delta^1,X)$ belongs to $\calC$.
\item[$(c)$] If $X$ and $Y$ belong to $\calC$, then a functor $X \rightarrow \Fun(\Delta^1,Y)$ is a morphism of $\calC$ if and only if, for every vertex $v$ of $\Delta^1$, the composite functor
$X \rightarrow \Fun(\Delta^1,Y) \rightarrow \Fun( \{v\}, Y) \simeq Y$ is a morphism of $\calC$.
\end{itemize}

Let $p: X \rightarrow S$ be a map of simplicial sets, where $S$ is small.
Assume that:
\begin{itemize}
\item[$(i)$] The map $p$ is a categorical fibration and a locally coCartesian fibration.
\item[$(ii)$] For each vertex $s$ in $S$, the fiber $X_{s}$ belongs to $\calC$.
\item[$(iii)$] For each edge $s \rightarrow s'$ in $S$, the associated functor
$X_{s} \rightarrow X_{s'}$ is a morphism in $\calC$.
\end{itemize}
Let $\calE$ be a set of edges of $S$, and let $Y$ be the full subcategory of
$\bHom_{S}(S, X)$ spanned by those sections $f: S \rightarrow X$ of $p$ which
satisfy the following condition:
\begin{itemize}
\item[$(\ast)$] For every edge $e: \Delta^1 \rightarrow S$ belonging to $\calE$,
$f$ carries $e$ to to a $p_{e}$-coCartesian edge of $\Delta^1 \times_{S} X$, where
$p_{e}: \Delta^1 \times_{S} X \rightarrow \Delta^1$ is the projection.
\end{itemize}
Then $Y$ belongs to to $\calC$. Moreover, if $Z \in \calC$, then a functor
$Z \rightarrow Y$ belongs to $\calC$ if and only if, for every vertex $s$ in $S$, the composite map
$Z \rightarrow Y \rightarrow X_{s}$
belongs to $\calC$.
\end{proposition}

\begin{remark}
Hypotheses $(i)$ through $(iii)$ of Proposition \ref{prestorkus} are satisfied, in particular, if $p: X \rightarrow S$ is a coCartesian fibration classified by a functor $S \rightarrow \calC \subseteq \widehat{\Cat}_{\infty}$.
\end{remark}

\begin{remark}
Hypotheses $(a)$, $(b)$, and $(c)$ of Proposition \ref{prestorkus} are satisfied for the following subcategories $\calC \subseteq \widehat{\Cat}_{\infty}$:
\begin{itemize}
\item Fix a class of simplicial sets $\{ K_{\alpha} \}_{\alpha \in A}$. Then we can take $\calC$ be the subcategory of
$\widehat{\Cat}_{\infty}$ whose objects are $\infty$-categories which admit $K_{\alpha}$-indexed (co)limits, for each $\alpha \in A$, and whose morphisms are functors which preserves $K_{\alpha}$-indexed (co)limits, for each $\alpha \in A$.
\item We can take the objects of $\calC$ to be accessible $\infty$-categories, and the morphisms in $\calC$ to be accessible functors (in view of Propositions \ref{horse1} and \ref{accprop}).
\end{itemize}
We will meet some other examples in \S \ref{c5s6}.
\end{remark}

\begin{remark}
In the situation of Proposition \ref{prestorkus}, we can replace ``coCartesian'' by ``Cartesian'' everywhere to obtain a dual result. This follows by applying Proposition \ref{prestorkus} to the map $X^{op} \rightarrow S^{op}$, after replacing $\calC$ by its preimage under the ``opposition'' involution of $\h{ \widehat{\Cat}_{\infty}}$.
\end{remark}

The proof of Proposition \ref{prestorkus} makes use of the following observation:

\begin{lemma}\label{surgem}
Let $p: \calM \rightarrow \Delta^1$ be a coCartesian fibration, classifying a functor
$F: \calC \rightarrow \calD$, where $\calC = p^{-1} \{0\}$ and $\calD = p^{-1} \{1\}$. 
Let $\calX = \bHom_{\Delta^1}( \Delta^1, \calM)$ be the $\infty$-category of sections of $p$.
Then $\calX$ can be identified with a homotopy limit of the diagram
$$ \calC \stackrel{F}{\rightarrow} \Fun( \{0\}, \calD) \leftarrow \Fun(\Delta^1, \calD).$$
\end{lemma}

\begin{proof}
We first replace the diagram in question by a fibrant one. Let $\calC'$ denote the
$\infty$-category of coCartesian sections of $p$. Then the evaluation map
$e: \calC' \rightarrow \calC$ is a trivial fibration of simplicial sets. Moreover, since $F$ is associated to the correspondence $\calM$, the map $e$ admits a section $s$ such that the composition
$$ \calC \stackrel{s}{\rightarrow} \calC' \rightarrow \calD$$
coincides with $F$. It follows that we have a weak equivalence of diagrams
$$ \xymatrix{ \calC \ar[r]^-{F} \ar[d]^{s} & \Fun( \{0\}, \calD) \ar@{=}[d] & \Fun(\Delta^1, \calD) \ar[l] \ar@{=}[d] \\
\calC' \ar[r]^-{F'} & \Fun( \{0\}, \calD) & \Fun(\Delta^1, \calD) \ar[l] }$$
where $F'$ is given by evaluation at $\{1\}$, and is a categorical fibration. Let
$\calX'$ denote the pullback of the lower diagram, which we can identify with 
the full subcategory of $\bHom_{ \Delta^1}( \Lambda^2_1, \calM )$ spanned
by those functors which carry the first edge of $\Lambda^2_1$ to a coCartesian edge of $\calM$.

Regard $\Delta^2$ as an object of $(\sSet)_{/\Delta^1}$ via the unique retraction
$r: \Delta^2 \rightarrow \Delta^1$ onto the
simplicial subset $\Delta^{ \{0,1\} } \subseteq \Delta^{ \{0,1,2\} }$.
Let $\calX''$ denote the full subcategory of $\bHom_{\Delta^1}(\Delta^2, \calM)$
spanned by those maps $\Delta^2 \rightarrow \calM$ which carry the initial edge of $\Delta^2$ to a $p$-coCartesian edge of $\calM$. 

Let $T$ denote the marked simplicial set whose underlying simplicial set is $\Delta^2$, whose sole nondegenerate marked edge is $\Delta^1 \subseteq \Delta^2$, and let $T' = T \times_{ (\Delta^2)^{\sharp} } ( \Lambda^2_1)^{\sharp}$. Since the opposites of the inclusions
$T' \subseteq T$, $( \Delta^{ \{0,2\} } )^{\flat} \subseteq T$ are marked anodyne, we conclude that the evaluation maps
$$ \calX \leftarrow \calX'' \rightarrow \calX'$$ are trivial fibrations of simplicial sets.
It follows that $\calX$ and $\calX'$ are (canonically) homotopy equivalent, as desired.
\end{proof}

\begin{remark}
In the situation of Lemma \ref{surgem}, the full subcategory of $\calX$ spanned by the
{\em coCartesian} sections of $p$ is equivalent (via evaluation at $\{0\}$) to $\calC$.
\end{remark}

\begin{proof}[Proof of Proposition \ref{prestorkus}]
Let us first suppose that $\calE = \emptyset$.
Let $\sk^n S$ denote the $n$-skeleton of $S$. We observe that
$\bHom_{S}(S,X)$ coincides with the (homotopy) inverse limit
$$ \varprojlim \{ \bHom_{ S }(\sk^n S, X) \}. $$ 
In view of assumption $(a)$, it will suffice to prove the result after replacing $S$ by $\sk^n S$.
In other words, we may reduce to the
case where $S$ is $n$-dimensional. 

We now work by induction on $n$, and observe that
there is a homotopy pushout diagram of simplicial sets
$$ \xymatrix{ S_n \times \bd \Delta^n \ar@{^{(}->}[r] \ar[d] & S_n \times \Delta^n \ar[d] \\
\sk^{n-1} S \ar@{^{(}->}[r] & S. }$$
We therefore obtain a homotopy pullback diagram of $\infty$-categories
$$ \xymatrix{ \bHom_{S}( S, X) \ar[r] \ar[d] & \bHom_{S}( \sk^{n-1} S, X) \ar[d] \\
\bHom_{S}( S_n \times \Delta^n,X) \ar[r] & \bHom_{S}( S_n \times \bd \Delta^n, X). }$$
Invoking assumption $(a)$ again, we are reduced to proving the same result after replacing
$S$ by $\sk^{n-1} S$, $S_n \times \bd \Delta^n$, and $S_{n} \times \Delta^n$. The first two cases follow from the inductive hypothesis; we may therefore assume that $S$ is a disjoint union of copies of $\Delta^n$. Applying $(a)$ once more, we can reduce to the case $S = \Delta^n$.

If $n = 0$, there is nothing to prove. If $n > 1$, then we have a trivial fibration
$$ \bHom_{S}(S, X) \rightarrow \bHom_{S}( \Lambda^n_1, X).$$
Since the horn $\Lambda^n_1$ is of dimension $< n$, we may conclude by applying the inductive hypothesis. We are therefore reduced to the case $S = \Delta^1$.

According to Lemma \ref{surgem}, the $\infty$-category $\bHom_{\Delta^1}(\Delta^1, X)$ can be identified with a homotopy limit of the diagram
$$ X_{ \{ 0\} } \stackrel{F}{\rightarrow} X_{ \{1\} } \leftarrow X_{ \{1\} }^{\Delta^1}.$$
In view of $(a)$, it will suffice to prove that all of the $\infty$-categories and functors in the above diagram belong to $\calC$. This follows immediately from $(b)$ and $(c)$.

We now consider the general case where $\calE$ is not required to be empty. For each
edge $e \in \calE$, let $Y(e)$ denote the full subcategory of $\bHom_{S}(S,X)$ spanned by those sections $f: S \rightarrow X$ which satisfy the condition $(\ast)$ for the edge $e$. We wish to prove:
\begin{itemize}
\item[$(1)$] The intersection $\bigcap_{e \in \calE} Y(e)$ belongs to $\calC$.
\item[$(2)$] If $Z \in \calC$, then a functor $Z \rightarrow \bigcap_{e \in \calE} Y(e)$ is a morphism of $\calC$ if and only if the induced map $Z \rightarrow \bHom_{S}(S,X)$ is a morphism of $\calC$.
\end{itemize}
In view of $(a)$, it will suffice to prove the corresponding results where $\bigcap_{e \in \calE} Y(e)$ is replaced by a single subcategory $Y(e) \subseteq \bHom_{S}(S,X)$. 

Let $e: s \rightarrow s'$ be an edge belonging to $\calE$. Lemma \ref{surgem} implies the existence of a homotopy pullback diagram
We now observe that there is a homotopy pullback diagram 
$$ \xymatrix{ Y(e) \ar[r] \ar[d] & \bHom_{S}(S,X) \ar[d] \\
\Fun'( \Delta^1, X_{s'}) \ar[r] & \Fun( \Delta^1, X_{s'} ), }$$
where $\Fun'( \Delta^1, X_{s'} ) \simeq X_{s'}$ is the full subcategory of $\Fun( \Delta^1, X_{s'} )$ spanned by the equivalences. In view of $(a)$, it suffices to prove the following analogues of $(1)$ and $(2)$:
\begin{itemize}
\item[$(1')$] For each vertex $s' \in S$, the $\infty$-categories
$\Fun'(\Delta^1, X_{s'})$ and $\Fun( \Delta^1, X_{s'})$ belong to $\calC$.
\item[$(2')$] Given an object $Z \in \calC$, a functor 
$Z \rightarrow \Fun'( \Delta^1, X_{s'})$ is a morphism in $\calC$ if and only if the
induced map $Z \rightarrow \Fun( \Delta^1, X_{s'})$ is a morphism of $\calC$.
\end{itemize}
These assertions follow immediately from $(b)$ and $(c)$, respectively.
\end{proof}

\begin{corollary}\label{storkus1}\index{gen}{accessible!$\infty$-category of sections}
Let $p: X \rightarrow S$ be a map of simplicial sets which is a coCartesian fibration $($or a Cartesian fibration$)$. Assume that:

\begin{itemize}
\item[$(1)$] The simplicial set $S$ is small.
\item[$(2)$] For each vertex $s$ of $S$, the $\infty$-category $X_{s} = X \times_{S} \{s\}$ is
accessible.
\item[$(3)$] For each edge $e: s \rightarrow s'$ of $S$, the associated functor
$X_{s} \rightarrow X_{s'}$ $($or $X_{s'} \rightarrow X_{s}${}$)$ is accessible. 
\end{itemize}

Then $\bHom_{S}(S,X)$ is an accessible $\infty$-category. Moreover, if $\calC$ is accessible, then
a functor $$\calC \rightarrow \bHom_{S}(S,X)$$ is accessible if and only if, for every vertex $s$ of $S$, the induced map $\calC \rightarrow X_{s}$ is accessible.
\end{corollary}


