
\section{Category Theory}\label{catreview}

\setcounter{theorem}{0}

Familiarity with classical category theory is the main prerequisite for reading this book.
In this section, we will fix some of the notation that we will use when discussing categories, and summarize (generally without proofs) some of the concepts which we will use in the body of the text.\index{gen}{category}

If $\calC$ is a category, we let $\Ob(\calC)$ denote the set of objects of $\calC$.\index{not}{ObC@$\Ob(\calC)$} We will write $X \in \calC$ to
mean that $X$ is an object of $\calC$. For $X,Y \in \calC$, we write $\Hom_{\calC}(X,Y)$ for the set of morphisms from $X$ to $Y$ in $\calC$. We also write
$\id_X$ for the identity automorphism of $X \in \calC$ (regarded as an element of $\Hom_{\calC}(X,X)$).\index{not}{HomC@$\Hom_{\calC}(X,Y)$}

If $Z$ is an object in a category $\calC$, then the {\it overcategory}\index{gen}{overcategory}
$\calC_{/Z}$ of {\it objects over $Z$} is defined as follows: the
objects of $\calC_{/Z}$ are diagrams $X \rightarrow Z$ in $\calC$. A morphism
from $f: X \rightarrow Z$ to $g: Y \rightarrow Z$ is a commutative triangle
$$ \xymatrix{ X \ar[rr] \ar[dr]_{f} & & Y \ar[dl]^{g} \\
& Z.}$$
Dually, we have an {\it undercategory} $\calC_{Z/} = ((\calC^{op})_{Z/})^{op}$ of {\em objects under $Z$}\index{gen}{undercategory}.\index{not}{calC/X@$\calC_{/X}$}\index{not}{calCX/@$\calX_{X/}$}

If $f: X \rightarrow Z$ and $g: Y \rightarrow Z$ are objects in
$\calC_{/Z}$, then we will often write $\Hom_{Z}(X,Y)$ rather than
$\Hom_{\calC_{/Z}}(f,g)$.\index{not}{HomZXY@$\Hom_{Z}(X,Y)$}

We let $\Set$ denote the category of sets, and $\Cat$ the category of (small) categories (where the morphisms are given by functors).\index{not}{Cat@$\Cat$}\index{not}{Set@$\Set$}

If $\kappa$ is a regular cardinal, we will say that a set $S$ is {\it $\kappa$-small}\index{gen}{$\kappa$-small} if it has cardinality less than $\kappa$. We will also use this terminology when discussing mathematical objects other than sets, which are built out of sets. For example, we will say that a category $\calC$ is {\it $\kappa$-small} if the set of all objects of $\calC$ is $\kappa$-small, and the set of all morphisms in $\calC$ is likewise $\kappa$-small.

We will need to discuss categories which are not small. In order to minimize the effort spent dealing with set-theoretic complications, we will adopt the usual device of ``Grothendieck universes''. We fix a strongly inaccessible cardinal $\kappa$, and refer to a mathematical object (such as a set or category) as {\it small} if it is $\kappa$-small, and {\it large} otherwise.
It should be emphasized that this is primarily a linguistic device, and that none of our results depend in an essential way on the existence of a strongly inaccessible cardinal $\kappa$.\index{gen}{small}

Throughout this book, the word ``topos'' will always mean {\em Grothendieck topos}. Strictly speaking, a knowledge of classical topos theory is not required to read this paper: all of the relevant classical concepts will be introduced (though sometimes in a hurried fashion) in the course of our search for suitable $\infty$-categorical analogues. 

\subsection{Compactness and Presentability}

Let $\kappa$ be a regular cardinal.

\begin{definition}\index{gen}{filtered!partially ordered set}
A partially ordered set $\calI$ is {\it $\kappa$-filtered} if, for any subset $\calI_0 \subseteq \calI$ having cardinality $< \kappa$, there exists an upper bound for $\calI_0$ in $\calI$.
\end{definition}

Let $\calC$ be a category which admits (small) colimits, and 
let $X$ be an object of $\calC$.
Suppose given a $\kappa$-filtered partially ordered set $\calI$ and a diagram $\{Y_{\alpha} \}_{\alpha \in \calI}$ in $\calC$, indexed by $\calI$. Let $Y$ denote a colimit of this diagram.
There there is an associated map of sets
$$ \psi: \colim \Hom_{\calC}(X, Y_{\alpha}) \rightarrow \Hom_{\calC}(X,Y).$$
We say that $X$ is {\it $\kappa$-compact} if $\psi$ is bijective, for {\em every} $\kappa$-filtered
partially ordered set $\calI$ and {\em every} diagram $\{ Y_{\alpha} \}$ indexed by $\calI$.
We say that $X$ is {\it small} if it is $\kappa$-compact for some (small) regular cardinal $\kappa$.
In this case, $X$ is $\kappa$-compact for all sufficiently large regular cardinals $\kappa$.\index{gen}{compact!object of a category}\index{gen}{small!object of a category}

\begin{definition}\label{catpor}\index{gen}{presentable!category}
A category $\calC$ is {\it presentable} if it satisfies the following conditions:
\begin{itemize}
\item[$(1)$] The category $\calC$ admits all (small) colimits.

\item[$(2)$] There exists a (small) set $S$ of objects of $\calC$ which generates
$\calC$ under colimits; in other words, every object of $\calC$ may be obtained as the colimit of a (small) diagram taking values in $S$.

\item[$(3)$] Every object in $\calC$ is small. (Assuming $(2)$, this is equivalent to the assertion
that every object which belongs to $S$ is small.)

\item[$(4)$] For any pair of objects $X,Y \in \calC$, the set $\Hom_{\calC}(X,Y)$ is small.
\end{itemize}
\end{definition}

\begin{remark}
In \S \ref{c5s6}, we describe an $\infty$-categorical generalization of Definition \ref{catpor}.
\end{remark}

\begin{remark}
For more details of the theory of presentable categories, we refer the reader to \cite{adamek}. Note that our terminology differs slightly from that of \cite{adamek}, in which our presentable categories are called {\it locally presentable} categories.
\end{remark}

\subsection{Lifting Problems and the Small Object Argument}\label{liftingprobs}

Let $\calC$ be a category, and let $p: A \rightarrow B$ and $q: X
\rightarrow Y$ be morphisms in $\calC$. Recall that $p$ is said to
have the {\it left lifting property} with respect to $q$, and $q$
the {\it right lifting property} with respect to $p$, if given any diagram
$$ \xymatrix{ A \ar[d]^{p} \ar[r] & X \ar[d]^q \\
B \ar@{-->}[ur] \ar[r] & Y \\} $$
there exists a dotted arrow as indicated, rendering the diagram commutative.\index{gen}{left lifting property}\index{gen}{right lifting property}

\begin{remark}
In the case where $Y$ is a final object of $\calC$, we will instead say that
$X$ has the {\it extension property} with respect to $p: A \rightarrow B$.\index{gen}{extension property}
\end{remark}

Let $S$ be any collection of morphisms in $\calC$. We define
$_{\perp} S$ to be the class of all morphisms which have the right
lifting property with respect to all morphisms in $S$, and $S_{\perp}$
to be the class of all morphisms which have the left lifting
property with respect to all morphisms in $S$. We observe that
$$S \subseteq (_{\perp}S)_{\perp}.$$\index{not}{Sperp@$_{\perp}S$}\index{not}{Stop@$S_{\perp}$}

The class of morphisms $(_{\perp} S)_{\perp}$ enjoys several
stability properties which we axiomatize in the following definition.

\begin{definition}\label{saturated}\index{gen}{weakly saturated}\index{gen}{saturated!weakly}
Let $\calC$ be a category with all (small) colimits, and let $S$ be a class of morphisms of $\calC$. We will say that $S$ is {\it weakly saturated} if it has the following properties:

\begin{itemize}
\item[$(1)$] (Closure under the formation of pushouts) Given a pushout diagram
$$ \xymatrix{ C \ar[r]^{f} \ar[d] & D \ar[d] \\
C' \ar[r]^{f'} & D' }$$
such that $f$ belongs to $S$, the morphism $f'$ also belongs to $S$. 

\item[$(2)$] (Closure under transfinite composition) Let $C \in \calC$ be an object, $\alpha$
an ordinal, and let $\{ D_{\beta} \}_{\beta < \alpha} $ be a system of objects of $\calC_{C/}$
indexed by $\alpha$: in other words, for each $\beta < \alpha$, we are supplied with a morphism $C \rightarrow D_{\beta}$, and for each $\gamma \leq \beta < \alpha$ a commutative diagram
$$ \xymatrix{ & D_{\gamma} \ar[dd]^{\phi_{\gamma,\beta}} \\
C \ar[ur] \ar[dr] \\
& D_{\beta} }$$
satisfying $\phi_{\beta,\gamma} \circ \phi_{\gamma, \delta} = \phi_{\beta, \delta}.$
For $\beta \leq \alpha$, we let $D_{<\beta}$ be a colimit of the system
$\{ D_{\gamma} \}_{\gamma < \beta}$, taken in the category $\calC_{C/}$. 

Suppose that, for each $\beta < \alpha$, the natural map $D_{< \beta} \rightarrow D_{\beta}$
belongs to $S$. Then the induced map $C \rightarrow D_{<\alpha}$ belongs to $S$.

\item[$(3)$] (Closure under the formation of retracts) Given a commutative diagram
$$ \xymatrix{ C \ar[d]^{f} \ar[r] & C' \ar[d]^{g} \ar[r] & C \ar[d]^{f} \\
D \ar[r] & D' \ar[r] & D }$$
in which both horizontal compositions are the identity, if $g$ belongs to $S$, then so does $f$.
\end{itemize}
\end{definition}

It is worth noting that saturation has the following consequences:

\begin{proposition}
Let $\calC$ be a category which admits all $($small$)$ colimits, and let $S$ be a weakly saturated class of morphism in $\calC$. Then:
\begin{itemize}
\item[$(1)$] Every isomorphism belongs to $S$.
\item[$(2)$] The class $S$ is stable under composition: if $f: X \rightarrow Y$ and $g: Y \rightarrow Z$ belong to $S$, then so does $g \circ f$.
\end{itemize}
\end{proposition}

\begin{proof}
Assertion $(1)$ is equivalent to the closure of $S$ under transfinite composition, in the special case where $\alpha=0$; $(2)$ is equivalent to the special case where $\alpha=2$.
\end{proof}

\begin{remark}
A reader who is ill-at-ease with the style of the preceding argument should feel free to take
the asserted properties as part of the definition of a weakly saturated class of morphisms.
\end{remark}

The intersection of any collection of weakly saturated classes of morphisms is itself weakly saturated. Consequently, for any category $\calC$ which admits small colimits, and any collection
$A$ of morphisms of $\calC$, there exists a {\em smallest} weakly saturated class of morphisms containing $A$: we will call this the weakly saturated class of morphisms {\it generated} by $A$.
We note that $(_{\perp} A)_{\perp}$ is weakly saturated. Under appropriate set-theoretic assumptions, Quillen's ``small object'' argument can be used to establish that $(_{\perp} A)_{\perp}$ is the weakly saturated class generated by $A$:

\begin{proposition}[Small Object Argument]\label{quillobj}\index{gen}{small object argument}
Let $\calC$ be a presentable category and $A_0 = \{ \phi_i: C_i \rightarrow D_i \}_{i \in I}$ a collection
of morphisms in $\calC$ indexed by a $($small$)$ set $I$. For each $n \geq 0$, let
$\calC^{[n]}$ denote the category of functors from the linearly ordered set $[n] = \{0, \ldots, n\}$ into $\calC$. There exists a functor
$T: \calC^{[1]} \rightarrow \calC^{[2]}$ with the following properties:
\begin{itemize}
\item[$(1)$] The functor $T$ carries a morphism $f: X \rightarrow Z$ to a diagram
$$ \xymatrix{ & Y \ar[dr]^{f''} & \\
X \ar[ur]^{f'} \ar[rr]^{f} & & Z }$$
where $f'$ belongs to the weakly saturated class of morphisms generated by $A_0$ and $f''$
has the right lifting property with respect to each morphism in $A_0$.
\item[$(2)$] If $\kappa$ is a regular cardinal such that each of the objects $C_i$, $D_i$ is $\kappa$-compact, then $T$ commutes with $\kappa$-filtered colimits.
\end{itemize}
\end{proposition}

\begin{proof}
Fix a regular cardinal $\kappa$ as in $(2)$, and fix a morphism $f: X \rightarrow Z$ in $\calC$.
We will give a functorial construction of the desired diagram
$$ \xymatrix{ & Y \ar[dr]^{f''} & \\
X \ar[ur]^{f'} \ar[rr]^{f} & & Z }$$
We define a transfinite sequence of objects
$$ Y_0 \rightarrow Y_1 \rightarrow \ldots $$
in $\calC_{/Z}$, indexed by ordinals smaller than $\kappa$. Let $Y_0 = X$, and let $Y_{\lambda} = \colim_{ \alpha < \lambda} Y_{\alpha}$ when $\lambda$ is a nonzero limit ordinal. For $i \in I$, let
$F_i: \calC_{/Z} \rightarrow \Set$ be the functor
$$(T \rightarrow Z ) \mapsto \Hom_{\calC}( D_i, Z) \times_{ \Hom_{\calC}(C_i, Z)}
\Hom_{\calC}(C_i, T).$$
Supposing that $Y_{\alpha}$ has been defined, we define $Y_{\alpha+1}$ by the following pushout diagram
$$ \xymatrix{ \coprod_{i \in I, \eta \in F_i(Y_{\alpha})} C_i \ar[r] \ar[d] & Y_{\alpha} \ar[d] \\
\coprod_{i \in I, \eta \in F_i(Y_{\alpha})} D_i \ar[r] & Y_{\alpha+1}. }$$
We conclude by defining $Y$ to be $\colim_{\alpha < \kappa} Y_{\alpha}$. It is easy to check that this construction has the desired properties.
\end{proof}

\begin{remark}
If $\calC$ is enriched, tensored and cotensored over another presentable monoidal category
$\bfS$ (see \S \ref{enrichcat}), then a similar construction shows that we can choose
$T$ to be a $\bfS$-enriched functor.
\end{remark}

\begin{corollary}\label{tilobj}
Let $\calC$ be a presentable category, and let $A$ be a {\em set} of morphisms of $\calC$.
Then $(_{\perp} A)_{\perp}$ is the smallest weakly saturated class of morphisms containing $A$.
\end{corollary}

\begin{proof}
Let $\overline{A}$ be the smallest weakly saturated class of morphisms containing $A$, so that
$\overline{A} \subseteq (_{\perp} A)_{\perp}$. To establish the reverse inclusion, 
For the reverse inclusion, let us suppose that $f: X \rightarrow Z$ belongs to
$(_{\perp} A)_{\perp}$. Proposition \ref{quillobj} implies the existence of a factorization
$$ X \stackrel{f'}{\rightarrow} Y \stackrel{f''}{\rightarrow} Z$$
where $f' \in \overline{A}$ and $f''$ belongs to $_{\perp} A$. It follows that $f$ has the left lifting property with respect to $f''$, so that $f$ is a retract of $f'$ and therefore belongs to $\overline{A}$.
\end{proof}

\begin{remark}\label{easyprest}
Let $\calC$ be a presentable category, let $S$ be a (small) set of morphisms in $\calC$, and suppose that $f: X \rightarrow Y$ belongs to the weakly saturated class of morphisms generated by $S$.
The proofs of Proposition \ref{quillobj} and Corollary \ref{tilobj} show that there exists a transfinite sequence 
$$ Y_0 \rightarrow Y_1 \rightarrow \ldots $$
of objects of $\calC_{X/}$, indexed by a set of ordinals $\{ \beta | \beta < \alpha \}$, with the following properties:
\begin{itemize}
\item[$(i)$] For each $\beta < \alpha$, there is a pushout diagram 
$$ \xymatrix{ C \ar[r]^{g} \ar[d] & D \ar[d] \\
\colim_{ \gamma < \beta} Y_{\gamma} \ar[r] & Y_{\beta},}$$
where the colimit is formed in $\calC_{X/}$ and $g \in S$.
\item[$(ii)$] The object $Y$ is a retract of $\colim_{\gamma < \alpha} Y_{\gamma}$
in the category $\calC_{X/}$.
\end{itemize}
\end{remark}

\subsection{Monoidal Categories}\label{monoidaldef}

A {\it monoidal category}\index{gen}{monoidal category}\index{gen}{category!monoidal} is a category $\calC$ equipped with a (coherently) associative ``product''
functor $\otimes: \calC \times \calC \rightarrow \calC$ and a unit object ${\bf 1}$. 
The associativity is expressed by demanding isomorphisms
$$ \eta_{A,B,C}: (A \otimes B) \otimes C \rightarrow A \otimes (B \otimes C),$$
and the requirement that ${\bf 1}$ be unital is expressed by demanding isomorphisms
$$ \alpha_{A}: A \otimes {\bf 1} \rightarrow A$$
$$ \beta_{A}: {\bf 1} \otimes A \rightarrow A.$$
We do not merely require the existence of these isomorphisms: they are part of the structure of a monoidal category. Moreover, these isomorphisms are required to satisfy the following conditions:

\begin{itemize}
\item The isomorphism $\eta_{A,B,C}$ depends {\em functorially} on the triple $(A,B,C)$; in other words, $\eta$ may be regarded as a natural isomorphism between the functors
$$ \calC \times \calC \times \calC \rightarrow \calC.$$
$$ (A,B,C) \mapsto (A \otimes B) \otimes C$$
$$ (A,B,C) \mapsto A \otimes (B \otimes C).$$
Similarly $\alpha_A$ and $\beta_A$ depend functorially on $A$. 

\item Given any quadruple $(A,B,C,D)$ of objects of $\calC$, the {\it MacLane pentagon}\index{gen}{MacLane pentagon}\index{gen}{pentagon axiom}
$$ \xymatrix{ & ((A \otimes B) \otimes C) \otimes D \ar[dl]^{\eta_{A,B,C} \otimes \id_D}
\ar[dr]^{\eta_{ A \otimes B,C,D}} \\
(A \otimes (B \otimes C)) \otimes D \ar[d]^{ \eta_{A, B \otimes C, D}} & & (A \otimes B) \otimes (C \otimes D) \ar[d]^{\eta_{A,B, C \otimes D}} \\
A \otimes (( B \otimes C) \otimes D) \ar[rr]^{\id_A \otimes \eta_{B,C,D}} & & A \otimes (B \otimes (C \otimes D))}$$
is commutative. 

\item For any pair $(A,B)$ of objects of $\calC$, the triangle
$$ \xymatrix{ (A \otimes {\bf 1}) \otimes B \ar[rr]^{ \eta_{A, {\bf 1}, B}} \ar[dr]^{ \alpha_A \otimes \id_B} & & A \otimes ( {\bf 1} \otimes B) \ar[dl]^{\id_A \otimes \beta_{B}} \\
& A \otimes B}$$ is commutative.
\end{itemize}

MacLane's coherence theorem asserts that the commutativity of this pair of diagrams implies the commutativity of {\em all} diagrams that can be written using only the isomorphisms $\eta_{A,B,C}$, $\alpha_A$, and $\beta_A$. More precisely, any monoidal category is equivalent (as a monoidal category) to a {\em strict} monoidal category: that is, a monoidal category in which $\otimes$ is literally associative, ${\bf 1}$ is literally a unit with respect to $\otimes$, and the isomorphisms $\eta_{A,B,C}$, $\alpha_A$, $\beta_A$ are the identity maps.\index{gen}{monoidal category!strict}
\index{gen}{MacLane's coherence theorem}

\begin{example}
Let $\calC$ be a category which admits finite products. Then $\calC$ admits the structure of a monoidal category, where the operation $\otimes$ is given by Cartesian product
$$ A \otimes B \simeq A \times B$$
and the isomorphisms $\eta_{A,B,C}$ are induced from the evident associativity of the Cartesian product. The identity ${\bf 1}$ is defined to be the final object of $\calC$, and the isomorphisms
$\alpha_A$ and $\beta_A$ are determined in the obvious way. We refer to this monoidal structure on $\calC$ as the {\em Cartesian monoidal structure}.

We remark that the Cartesian product $A \times B$ is only well-defined up to (unique) isomorphism (as is the final object ${\bf 1}$), so that strictly speaking the Cartesian monoidal structure on $\calC$ depends on various choices; however, all such choices lead to (canonically) equivalent monoidal categories.\index{gen}{monoidal category!Cartesian}
\end{example}

\begin{remark}
Let $(\calC, \otimes, {\bf 1}, \eta, \alpha, \beta)$ be a monoidal category. We will generally abuse notation by simply saying that $\calC$ is a monoidal category, or that $(\calC, \otimes)$ is a monoidal category, or that $\otimes$ is a {\it monoidal structure} on $\calC$; the other structure is implicitly understood to be present as well.
\end{remark}

\begin{remark}
Let $\calC$ be a category equipped with a monoidal structure $\otimes$. Then we may define a new monoidal structure on $\calC$, by setting $A \otimes^{op} B = B \otimes A$. We refer to this monoidal structure $\otimes^{op}$ as the {\it opposite} of the monoidal structure $\otimes$.
\end{remark}

\begin{definition}\label{tukerdef}
A monoidal category $(\calC,\otimes)$ is said to be {\it left closed} if, for each $A \in \calC$, the functor $$ N \mapsto A \otimes N$$
admits a right adjoint
$$ Y \mapsto {}^A\!Y.$$
We say that $(\calC, \otimes)$ is {\it right-closed} if the opposite monoidal structure $(\calC, \otimes^{op})$ is left-closed; in other words, if every functor
$$ N \mapsto N \otimes A$$ has a right adjoint
$$ Y \mapsto Y^A.$$
Finally, we say that $(\calC, \otimes)$ is {\it closed} if it is both right-closed and left-closed.\index{gen}{monoidal category!left closed}\index{gen}{monoidal category!right closed}\index{gen}{monoidal category!closed}
\end{definition}

In the setting of monoidal categories, it is appropriate to consider only those functors which
are compatible with the monoidal structures in the following sense:

\begin{definition}\index{gen}{functor!lax monoidal}
Let $(\calC, \otimes)$ and $( \calD, \otimes)$ be monoidal categories. A {\it right-lax monoidal functor} from $\calC$ to $\calD$ consists of the following data:
\begin{itemize}
\item A functor $G: \calC \rightarrow \calD$.
\item A natural transformation $\gamma_{A,B}: G(A) \otimes G(B) \rightarrow G(A \otimes B)$ rendering commutative the diagram
$$ \xymatrix{ (G(A) \otimes G(B)) \otimes G(C) \ar[r]^{\gamma_{A,B}}
\ar[d]^{ \eta_{G(A),G(B),G(C)} } 
& G(A \otimes B) \otimes G(C) \ar[r]^{ \gamma_{A \otimes B, C} } & G((A \otimes B) \otimes C)
\ar[d]^{ G(\eta_{A,B,C})} \\
G(A) \otimes( G(B) \otimes G(C) ) \ar[r]^{\gamma_{B,C}}
& G(A) \otimes G(B \otimes C) \ar[r]^{ \gamma_{A, B \otimes C} } & G(A \otimes (B \otimes C)). }$$
\item A map $e: {\bf 1_{\calD}} \rightarrow G( {\bf 1_{\calC}} )$ rendering commutative the diagrams
$$ \xymatrix{ G(A) \otimes {\bf 1_{\calD} } \ar[r]^{ \id \otimes e} \ar[dr]^{\alpha_{G(A)}} & G(A) \otimes G( \bf{1_{\calC}}) \ar[r]^{\gamma_{A, {\bf 1_{\calC}}}} & G(A \otimes {\bf 1_{\calC}}) \ar[dl]^{G(\alpha_A)} \\ 
& G(A) }$$
$$ \xymatrix{  {\bf 1}_{\calD} \otimes G(B) \ar[r]^{ e \otimes \id} 
\ar[dr]^{\beta_{G(B)}} & G( {\bf 1_{\calC}}) \otimes G(B) \ar[r]^{\gamma_{{\bf 1_{\calC}},A}} & G({\bf 1_{\calC}} \otimes B) \ar[dl]^{G(\alpha_B)} \\ 
& G(B) & }.$$
\end{itemize}

A natural transformation between right-lax monoidal functors is {\it monoidal} if it commutes with
the maps $\gamma_{A,B}$, $e$.\index{gen}{functor!monoidal}
\end{definition}

Dually, a {\it left-lax monoidal functor} from $\calC$ to $\calD$ consists of a right-lax monoidal functor from $\calC^{op}$ to $\calD^{op}$; it is determined by giving a functor
$F: \calC \rightarrow \calD$ together with a map $e': F( {\bf 1_{\calC}}) \rightarrow {\bf 1}_{\calD}$
and a natural transformation
$$ \gamma'_{A,B}: F(A \otimes B) \rightarrow F(A) \otimes F(B)$$
satisfying the appropriate analogues of the conditions listed above.

If $F$ is a right-lax monoidal functor via {\em isomorphisms}
$$ e: {\bf 1}_{\calD} \rightarrow F( { \bf 1_{\calC} })$$
$$ \gamma_{A,B}: F(A) \otimes F(B) \rightarrow F(A \otimes B),$$
then $F$ may be regarded as a left-lax monoidal functor by setting $e' = e^{-1}$,
$\gamma'_{A,B} = \gamma_{A,B}^{-1}$. In this case, we simply say that $F$ is a {\em monoidal} functor.

\begin{remark}
Let 
$$ \Adjoint{F}{\calC}{\calD}{G}$$ be an adjunction between categories $\calC$ and $\calD$. Suppose that $\calC$ and $\calD$ are equipped with monoidal structures. Then endowing $G$ with the structure of a right-lax monoidal functor is equivalent to endowing $F$ with the structure of a left-lax monoidal functor.
\end{remark}

\begin{example}
Let $\calC$ and $\calD$ be categories which admit finite products, and let
$F: \calC \rightarrow \calD$ be a functor between them. Then, if we regard $\calC$ and
$\calD$ as endowed with the Cartesian monoidal structure, then $F$ acquires the structure
of a left lax-monoidal functor in a canonical way, via the maps
$F(A \times B) \rightarrow F(A) \times F(B)$ induced from the functoriality of $F$. In this case,
$F$ is a monoidal functor if and only if it commutes with finite products.
\end{example}

\subsection{Enriched Category Theory}\label{enrichcat}

One frequently encounters categories $\calD$ in which the collections of morphisms
$\Hom_{\calD}(X,Y)$ between two objects $X,Y \in \calD$ has additional structure: for example, a topology, or a group structure, or the structure of a vector space. These situations may all be efficiently described using the language of {\it enriched category theory}, which we now introduce.

Let $(\calC, \otimes)$ be a monoidal category. A {\it $\calC$-enriched category} $\calD$ consists of the following data:\index{gen}{enriched category}\index{gen}{category!enriched}

\begin{itemize}
\item[$(1)$] A collection of objects.

\item[$(2)$] For every pair of objects $X,Y \in \calD$, a mapping object
$\bHom_{\calD}(X,Y)$ of $\calC$.

\item[$(3)$] For every triple of objects $X,Y,Z \in \calD$, a composition map
$$ \bHom_{\calD}(Y,Z) \otimes \bHom_{\calD}(X,Y) \rightarrow \bHom_{\calD}(X,Z).$$
Composition is required to be associative in the sense that for any $W,X,Y,Z \in \calC$, the diagram
$$ \xymatrix{ \bHom_{\calD}(Z,Y) \otimes \bHom_{\calD}(Y,X) \otimes \bHom_{\calD}(X,W)
\ar[r] \ar[d] & 
\bHom_{\calD}(Z,X) \otimes \bHom_{\calD}(X,W) \ar[d] \\ 
\bHom_{\calD}(Z,Y) \otimes \bHom_{\calD}(Y,W) \ar[r] & \bHom_{\calD}(Z,W)}$$
is commutative.

\item[$(4)$] For every object $X \in \calD$, a unit map ${\bf 1} \rightarrow \bHom_{\calD}(X,X)$
rendering commutative the diagrams
$$ \xymatrix{ { \bf 1} \otimes \bHom_{\calD}(Y,X) \ar[rr] \ar[dr] & &  \bHom_{\calD}(X,X) \otimes \bHom_{\calD}(Y,X) \ar[dl] \\
& \bHom_{\calD}(Y,X) }$$

$$ \xymatrix{ \bHom_{\calD}(X,Y) \otimes {\bf 1} \ar[rr] \ar[dr] & &  \bHom_{\calD}(X,Y) \otimes \bHom_{\calD}(X,X) \ar[dl] \\
& \bHom_{\calD}(X,Y).}$$
\end{itemize}

\begin{example}
Suppose that $(\calC, \otimes)$ is a {\em right-closed} monoidal category. Then $\calC$ is enriched over itself in a natural way, if one defines $\bHom_{\calC}(X,Y) = Y^{X}$.
\end{example}

\begin{example}
Let $\calC$ be the category of sets, with the Cartesian monoidal structure. Then a $\calC$-enriched category is simply a category in the usual sense. 
\end{example}

\begin{remark}\label{laxcon}
Let $G: \calC \rightarrow \calC'$ be a right-lax monoidal functor between monoidal categories.
Suppose that $\calD$ is a category enriched over $\calC$. We may define a category
$G(\calD)$, enriched over $\calC'$, as follows:

\begin{itemize}
\item[$(1)$] The objects of $G(\calD)$ are the objects of $\calD$.
\item[$(2)$] Given objects $X,Y \in \calD$, we set $$\bHom_{G(\calD)}(X,Y) = G( \bHom_{\calD}(X,Y) ).$$
\item[$(3)$] The composition in $G(\calD)$ is given by the map
$$ G( \bHom_{\calD}(Y,Z) ) \otimes G( \bHom_{\calD}(X,Y) ) \rightarrow
G( \bHom_{\calD}(Y,Z) \otimes \bHom_{\calD}(X,Y) ) \rightarrow G( \bHom_{\calD}(X,Z)).$$
Here the first map is determined by the right-weakly monoidal structure on the functor $G$, and the second is obtained by applying $G$ to the composition law in the category $\calD$.
\item[$(4)$] For every object $X \in \calD$, the associated unit $G(\calD)$ is given by the composition
$$ {\bf 1_{\calC'} } \rightarrow G( { \bf 1_{\calC} } ) \rightarrow G( \bHom_{\calD}(X,X)).$$
\end{itemize}
\end{remark}

\begin{remark}\index{gen}{functor!enriched}
If $\calD$ and $\calD'$ are categories enriched over the same monoidal category $\calC$, then one can define a category of {\em $\calC$-enriched} functors from $\calD$ to $\calD'$ in the evident way. Namely, an enriched functor $F: \calD \rightarrow \calD'$ consists of a map from the objects of $\calD$ to the objects of $\calD'$ and a collection of morphisms
$$ \eta_{X,Y}: \bHom_{\calD}(X,Y) \rightarrow \bHom_{\calD'}(FX, FY)$$ with the following properties:
\begin{itemize}
\item[$(i)$] For each object $X \in \calD$, the composition
$$ {\bf 1_{\calC}} \rightarrow \bHom_{\calD}(X,X) \stackrel{\eta_{X.X}}{\rightarrow}
\bHom_{\calD'}(FX,FX) $$
coincides with the unit map for $FX \in \calD'$.
\item[$(ii)$] For every triple of objects $X, Y, Z \in \calD$, the diagram
$$ \xymatrix{ 
\bHom_{\calD}(X,Y) \otimes \bHom_{\calD}(Y,Z) \ar[r] \ar[d] & \bHom_{\calD}(X,Z) \ar[d] \\
\bHom_{\calD'}(FX,FY) \otimes \bHom_{\calD}(FY,FZ) \ar[r] & \bHom_{\calD}(FX,FZ) }$$
is commutative.
\end{itemize}
If $F$ and $F'$ are enriched functors, an {\em enriched natural transformation $\alpha$} from
$F$ to $F'$ consists of specifying, for each object $X \in \calD$, a morphism
$\alpha_{X} \in \Hom_{\calD'}( FX, F'X)$ which renders commutative the diagram
$$ \xymatrix{ \bHom_{\calD}(X,Y) \ar[r] \ar[d] & \bHom_{\calD'}(FX,FY) \ar[d]^{\alpha_Y} \\
\bHom_{\calD'}(F'X, F'Y) \ar[r]^{\alpha_X} & \bHom_{\calD'}(FX, F'Y). }$$\index{gen}{natural transformation!enriched}
\end{remark}

Suppose that $\calC$ is any monoidal category. Consider the functor $\calC \rightarrow \Set$ given by
$$ X \mapsto \Hom_{\calC}( { \bf 1}, X).$$
This is a right-lax monoidal functor from $(\calC, \otimes)$ to $\Set$, where the latter is equipped with the Cartesian monoidal structure. By the above remarks, we see that we may equip any $\calC$-enriched category $\calD$ with the structure of an ordinary category by setting
$$ \Hom_{\calD}(X,Y) = \Hom_{\calC}( {\bf 1}, \bHom_{\calD}(X,Y) ).$$ 
We will generally not distinguish notationally between $\calD$ as a $\calC$-enriched category
and this (underlying) category having the same objects. However, to avoid confusion, we use different notations for the morphisms: $\bHom_{\calD}(X,Y)$ is an object of $\calC$, while $\Hom_{\calD}(X,Y)$ is a set.

Let $\calC$ be a right-closed monoidal category, and $\calD$ a category enriched over $\calC$.
Fix objects $C \in \calC$, $X \in \calD$, and consider the functor
$$ \calD \rightarrow \calC$$
$$ Y \mapsto \bHom_{\calD}(X,Y)^{C}.$$
This functor may or may not be {\em corepresentable}\index{gen}{corepresentable!functor}, in the sense that there exists an object
$Z \in \calD$ and an isomorphism of functors
$$ \eta: \bHom_{\calD}(X, \bigdot)^{C} \simeq \bHom_{\calD}(Z, \bigdot).$$
If such an object $Z$ exists, we will denote it by $X \otimes C$. The natural isomorphism $\eta$ is determined by specifying a single map $\eta(X): C \rightarrow \bHom_{\calD}(X, X \otimes C)$. By general nonsense, the map $\eta(X)$ determines $X \otimes C$ up to (unique) isomorphism, provided that $X \otimes C$ exists. If the object $X \otimes C$ exists for every $C \in \calC$, $X \in \calD$, then we say that $\calD$ is {\it tensored over $\calC$}.\index{gen}{tensored} In this case, we may regard $$(X,C) \mapsto X \otimes C$$
as determining a functor $\calD \otimes \calC \rightarrow \calD$. Moreover, one has canonical
isomorphisms $$X \otimes (C \otimes D) \simeq (X \otimes C) \otimes D$$
which express the idea that $\calD$ may be regarded as equipped with an ``action'' of $\calC$. Here we imagine $\calC$ as a kind of generalized monoid (via its monoidal structure).

Dually, if $\calC$ is right-closed, then an object of $\calD$ which represents the functor
$$ Y \mapsto ^{C}\!\bHom_{\calD}(Y,X)$$
will be denoted by $^{C}\!X$; the object $^{C}\!X$ (if it exists) is determined up to (unique) isomorphism by a map $C \rightarrow \bHom_{\calD}(^{C}\!X,X)$. 
If this object exists for all $C \in \calC$, $X \in \calD$, then we say that
$\calD$ is {\it cotensored over $\calC$}.\index{gen}{tensored}\index{gen}{cotensored}

\begin{example}
Let $\calC$ be a right-closed monoidal category. Then $\calC$ may be regarded as enriched over itself in a natural way. It is automatically tensored over itself; it is cotensored over itself if and only if it is left-closed.
\end{example}

\subsection{Trees}

Let $\calC$ be a presentable category and $S$ a small collection of morphisms in $\calC$.
According to Remark \ref{easyprest}, the smallest weakly saturated class of morphisms $\overline{S}$
containing $S$ can be obtained from $S$ using pushouts, retracts, and transfinite composition.
It is natural to ask if the formation of retracts is necessary: that is, does the weakly saturated class of morphisms generated by $S$ coincide with the class of morphisms which are given by transfinite compositions of pushouts of morphisms of $S$? Our goal for the remainder of this section is to give an affirmative answer, at least after $S$ has been suitably enlarged (Proposition \ref{easycrust}). This result is of a somewhat technical nature, and will be needed only during our discussion of combinatorial model categories in \S \ref{combimod}.

We begin by introducing a generalization of the notion of a transfinite chain of morphisms.

\begin{definition}\index{gen}{tree}\index{gen}{$S$-tree}\index{gen}{root of an $S$-tree}
Let $\calC$ be a presentable category, and let $S$ be a collection of morphisms in $\calC$. 
An {\it $S$-tree} in $\calC$ consists of the following data:
\begin{itemize}
\item[$(1)$] An object $X \in \calC$, called the {\it root} of the $S$-tree.
\item[$(2)$] A partially ordered set $A$ which is {\it well-founded} (that is, every nonempty subset of $P$ has a minimal element).
\item[$(3)$] A diagram $A \rightarrow \calC_{X/}$, which we will denote by $\alpha \mapsto Y_{\alpha}$.
\item[$(4)$] For each $\alpha \in A$, a pushout diagram
$$ \xymatrix{ C \ar[r]^{f} \ar[d] & D \ar[d] \\
\colim_{\beta < \alpha} X_{\beta} \ar[r] & X_{\alpha}, }$$
where $f \in S$.
\end{itemize}
Let $\kappa$ be a regular cardinal. We will say that an $S$-tree in $\calC$ is {\it $\kappa$-good} if each of the objects $C$ and $D$ appearing above is $\kappa$-compact, and for each
$\alpha \in A$, the set $\{ \beta \in A: \beta < \alpha \}$ is $\kappa$-small.\index{gen}{$S$-tree!$\kappa$-good}\index{gen}{tree!$\kappa$-good}
\end{definition}

\begin{notation}
Let $\calC$ be a presentable category and $S$ a collection of morphisms in $\calC$. We will
indicate an $S$-tree by writing $\{ Y_{\alpha} \}_{\alpha \in A}$. Here the root $X \in \calC$ and the relevant pushout diagrams are understood implicitly to be part of the data.

Suppose given an $S$-tree $\{ Y_{\alpha} \}_{\alpha \in A}$, and a subset $B \subseteq A$ which is closed downwards in the following sense: if $\alpha \in B$ and $\beta \leq \alpha$, then $\beta \in B$. Then $\{ Y_{\alpha} \}_{\alpha \in B}$ is an $S$-tree. We let $Y_{B}$ denote the colimit
$\colim_{\alpha \in B} Y_{\alpha}$, formed in the category $\calC_{X/}$. In particular we have a canonical isomorhism
$Y_{\emptyset} \simeq X$. If $B = \{ \alpha \in A| \alpha \leq \beta \}$, then $Y_{B} \simeq Y_{\alpha}$. 
\end{notation}

\begin{remark}\index{gen}{$S$-tree!associated}\label{asstree}
Let $\calC$ be a presentable category, $S$ a collection of morphisms in $\calC$, and
$\{ Y_{\alpha} \}_{ \alpha in A}$ an $S$-tree in $\calC$ with root $X$. Given a map
$f: X \rightarrow X'$, we can form an {\it associated $S$-tree}
$ \{ Y_{\alpha} \coprod_{X} X' \}_{\alpha \in A}$, having root $X'$.
\end{remark}

\begin{example}
Let $\calC$ be a presentable category, $S$ a collection of morphisms in $\calC$, and
$\{ Y_{\alpha} \}_{\alpha \in A}$ an $S$-tree in $\calC$ with root $X$. If $A$ is linearly ordered, then we may identify $\{ Y_{\alpha} \}_{\alpha \in A}$ with a (possibly transfinite) sequence of morphisms belonging to $S$,
$$ X \rightarrow Y_0 \rightarrow Y_1 \rightarrow \ldots, $$
as in the statement of $(2)$ in Definition \ref{saturated}.
\end{example}

\begin{remark}\label{relci}
Let $\calC$ be a presentable category, $S$ a collection morphisms in $\calC$, and
$\{Y_{\alpha} \}_{\alpha \in A}$ an $S$-tree in $\calC$. Let $B \subseteq A$ be closed downward. 
For $\alpha \in A - B$, let $B_{\alpha} = B \cup \{ \beta \in A: \beta \leq \alpha \}$, and let
$Z_{\alpha} = Y_{B_{\alpha}}$. Then $ \{ Z_{\alpha} \}_{ \alpha \in A-B}$ is an $S$-tree in
$\calC$ with root $Y_{B}$.
\end{remark}

\begin{lemma}\label{uper}
Let $\calC$ be a presentable category and let $S$ be a collection of morphisms
in $\calC$. Let $\{ Y_{\alpha} \}_{\alpha \in A}$ be an $S$-tree in $\calC$, and let
$A'' \subseteq A' \subseteq A$ be subsets which are closed downward in $A$. Then
the induced map
$ Y_{A''} \rightarrow Y_{A'}$ belongs to the weakly saturated class of morphisms generated by $S$.
In particular, the canonical map
$Y_{\emptyset} \rightarrow Y_{A}$ belongs to the weakly saturated class of morphisms generated by $S$.
\end{lemma}

\begin{proof} Using Remarks \ref{relci} and \ref{asstree}, we can assume without loss of enerality that $A'' = \emptyset$ and $A' = A$. Using the assumption that $A$ is well-founded, we can write $A$ as the union of a transfinite sequence (downward closed) subsets $\{ B( \gamma ) \subseteq A \}_{\gamma < \beta }$ with the following property:
\begin{itemize}
\item[$(\ast)$] For each $\gamma < \beta$, the set $B(\gamma)$ is obtained from
$B'(\gamma) = \bigcup_{\gamma' < \gamma} B(\gamma')$ by adjoining a minimal element of
$\alpha_{\gamma}$ of $A - B'(\gamma)$. 
\end{itemize}
For $\gamma < \beta$, let $Z_{\gamma} = Y_{B(\gamma)}$. We now observe that
$Y_{A} \simeq \colim_{\gamma < \beta} Z_{\gamma}$, and that for each
$\gamma < \beta$ there is a pushout diagram
$$ \xymatrix{ \colim_{\alpha < \alpha_{\gamma}} Y_{\alpha} \ar[r] \ar[d] & Y_{\alpha} \ar[d] \\
\colim_{\gamma' < \gamma} Z_{\gamma'} \ar[r]^{f} & Z_{\gamma},  }$$
so that $f$ is the pushout of a morphism belonging to $S$.
\end{proof}

\begin{lemma}\label{humber2}
Let $\calC$ be a presentable category, $\kappa$ a regular cardinal, and let $S = \{ f_{s}: C_{s} \rightarrow D_{s} \}$ be a collection of morphisms in $\calC$, where each of the objects $C_{s}$ and $D_{s}$ is $\kappa$-compact. Suppose that $\{ Y_{\alpha} \}_{\alpha \in A}$ is an $S$-tree in $\calC$, indexed by a partially ordered set $(A, \leq)$. Then there exists the following:
\begin{itemize}
\item[$(1)$] A new ordering $\preceq$ on $A$, which refines $\leq$ in the following sense:
if $\alpha \preceq \beta$, then $\alpha \leq \beta$. Let $A'$ denote the partially ordered set $A$, with this new partial ordering.
\item[$(2)$] A $\kappa$-good $S$-tree $\{ Y'_{\alpha} \}_{\alpha \in A'}$, having the same root
$X$ as $\{ Y_{\alpha} \}_{\alpha \in A}$.
\item[$(3)$] A collection of maps $f_{\alpha}: Y'_{\alpha} \rightarrow Y_{\alpha}$, which form a commutative diagram
$$ \xymatrix{ Y'_{\alpha'} \ar[r] \ar[d]^{f_{\alpha'}} & Y'_{\alpha} \ar[d]^{f_{\alpha}} \\
Y_{\alpha'} \ar[r] & Y_{\alpha} }$$
when $\alpha' \preceq \alpha$.
\item[$(4)$] For every subset $B \subseteq A$ which is closed downwards with respect to $\preceq$, the induced map $f_{B}: Y'_{B} \rightarrow Y_{B}$ is an isomorphism.
\end{itemize}
\end{lemma}

\begin{proof}
Choose a transfinite sequence of downward-closed subsets $\{ A(\gamma) \subseteq A \}_{\gamma \leq \beta}$ so that the following conditions are satisfied:
\begin{itemize}
\item[$(i)$] If $\gamma' \leq \gamma \leq \beta$, then $A(\gamma') \subseteq A(\gamma)$.
\item[$(ii)$] If $\lambda \leq \beta$ is a limit ordinal (possibly zero), then
$A(\lambda) = \bigcup_{ \gamma < \lambda} A(\gamma)$.
\item[$(iii)$] If $\gamma + 1 \leq \beta$, then $A(\gamma+1) = A(\gamma) \cup \{ \alpha_{\gamma} \}$, where $\alpha_{\gamma}$ is a minimal element of $A - A(\gamma)$.
\item[$(iv)$] The subset $A(\beta)$ coincides with $A$.
\end{itemize}

We will construct a compatible family of orderings $A'(\gamma) = (A(\gamma), \preceq)$, $S$-trees $\{ Y'_{\alpha} \}_{ \alpha \in A'(\gamma)} \}$, and collections of morphisms 
$\{ Y'_{\alpha} \rightarrow Y_{\alpha} \}_{\alpha \in A(\gamma)}$ by induction on $\gamma$, so that the analogues of conditions $(1)$ through $(4)$ are satisfied. If $\gamma$ is a limit ordinal, there is nothing to do; let us assume therefore that $\gamma < \beta$ and that the data
$( A'(\gamma), \{ Y'_{\alpha} \}_{\alpha \in A'(\gamma)}, \{ f_{\alpha} \}_{\alpha \in A(\gamma)} )$ has already been constructed. Let $B = \{ \alpha \in A: \alpha < \alpha_{\gamma} \}$, so that we have a pushout diagram
$$ \xymatrix{ C \ar[r]^{f} \ar[d]^{i} & D \ar[d] \\
Y_B \ar[r] & Y_{\alpha} }$$
where $f \in S$. By the inductive hypothesis, we may identify $Y_{B}$ with $Y'_{B}$. 
Since $C$ is $\kappa$-compact, the map $i$ admits a factorization
$$C \stackrel{i'}{\rightarrow} Y'_{B'} \stackrel{i''}{\rightarrow} Y'_{B}$$
where $B'$ is $\kappa$-small. Enlarging $B'$ if necessary, we may suppose that
$B'$ is closed downwards under $\preceq$. We now extend the partial ordering
$\preceq$ to $A'(\gamma+1) = A'(\gamma) \cup \{ \alpha_{\gamma} \}$ by declaring that
$\alpha \leq \alpha_{\gamma}$ if and only if $\alpha \in B'$. We define
$Y'_{\alpha_{\gamma}}$ by forming a pushout diagram
$$ \xymatrix{ C \ar[r]^{f} \ar[d]^{i'} & D \ar[d] \\
Y'_{B'} \ar[r] & Y'_{\alpha_{\gamma}}, }$$
and we define $f_{\alpha_{\gamma}}: Y'_{\alpha_{\gamma}} \rightarrow Y_{\alpha_{\gamma}}$
to be the map induced by $i''$. It is readily verified that these data satisfy the desired conditions.
\end{proof}

\begin{lemma}\label{turkteck}
Let $\calC$ be a presentable category, $\kappa$ an uncountable regular cardinal, and 
$S$ a collection of morphisms in $\calC$. Let $\{ Y_{\alpha} \}_{\alpha \in A}$ be a $\kappa$-good $S$-tree with root $X$, and $T_A: Y_{A} \rightarrow Y_{A}$ an idempotent endomorphism of
$Y_{A}$ in the category $\calC_{X/}$. Let $B_0$ be an arbitrary $\kappa$-small subset of
$A$. Then there exists a $\kappa$-small subset $B \subseteq A$ which is downward closed and contains $B_0$ and an idempotent endomorphism $T_{B}: Y_{B} \rightarrow Y_{B}$
such that the following diagram commutes:
$$ \xymatrix{ X \ar[r] \ar[d]^{=} & Y_B \ar[d]^{T_B} \ar[r] & Y_{A} \ar[d]^{T_A} \ar[d] \\
X \ar[r] & Y_B \ar[r] & Y_A. }$$
\end{lemma}

\begin{proof}
Enlarging $B_0$ if necessary, we may assume that $B_0$ is closed downwards. 
For every pair of downward closed subsets $A'' \subseteq A' \subseteq A$, let
$i_{A'',A'}$ denote the canonical map from $Y_{A''}$ to $Y_{A'}$.
Note that because $\{ Y_{\alpha} \}_{\alpha \in A}$ is a $\kappa$-good $S$-tree, if
$A' \subseteq A$ is closed downward and $\kappa$-small, $Y_{A'}$ is $\kappa$-compact when 
viewed as an object of $\calC_{X/}$. In particular, $Y_{B_0}$ is a $\kappa$-compact object of $\calC_{X/}$. It follows that the composition
$$ Y_{B_0} \stackrel{i_{B_0,A}}{\rightarrow} Y_{A} \stackrel{T_A}{\rightarrow} Y_{A}$$
can also be factored as a composition
$$ Y_{B_0} \stackrel{T_0}{\rightarrow} Y_{B_1} \stackrel{ i_{B_1,A}}{\rightarrow} Y_A,$$
where $B_1 \subseteq A$ is closed downwards and $\kappa$-small. Enlarging $B_1$ if necessary, we may suppose that $B_1$ contains $B_0$.

We now proceed to define a sequence of $\kappa$-small, downward closed subsets
$$ B_0 \subseteq B_1 \subseteq B_2 \subseteq \ldots $$
of $A$, and maps $T_i: Y_{B_i} \rightarrow Y_{B_{i+1}}$. Suppose that $i > 0$, and that
$B_{i}$ and $T_{i-1}$ have already been constructed. By compactness again, we conclude that the composite map
$$ Y_{B_i} \stackrel{ i_{B_i, A}}{\rightarrow} Y_{A} \stackrel{T_A}{\rightarrow} Y_A $$
can be factored as
$$ Y_{B_i} \stackrel{ T_{i} }{\rightarrow} Y_{B_{i+1} } \stackrel{ i_{B_{i+1}, A}}{\rightarrow} Y_A,$$
where $B_{i+1}$ is $\kappa$-small. Enlarging $B_{i+1}$ if necessary, we may assume that
$B_{i+1}$ contains $B_{i}$ and that the following diagrams commute:
$$ \xymatrix{ Y_{B_{i-1}} \ar[r]^{T_{i-1}} \ar[d]^{i_{B_{i-1}, B_{i}}} & Y_{B_{i}} \ar[d]^{i_{B_{i}, B_{i+1}}} \\
Y_{B_i} \ar[r]^{T_i} & Y_{B_{i+1}} }$$
$$ \xymatrix{ Y_{B_{i-1}} \ar[r]^{T_{i-1}} \ar[d]^{T_{i-1}} & Y_{B_i} \ar[d]^{T_{i}} \\
Y_{B_i} \ar[r]^{i_{B_i, B_{i+1}}} & Y_{B_{i+1}}. }$$
Let $B = \bigcup B_i$; then $B$ is $\kappa$-small in virtue of our assumption that $\kappa$ is uncountable. The collection of maps $\{ T_i \}$ assemble to a map $T_{B}: Y_{B} \rightarrow Y_{B}$ with the desired properties.
\end{proof}

\begin{lemma}\label{superturk}
Let $\calC$ be a presentable category, $\kappa$ an uncountable regular cardinal, and 
$S$ a collection of morphisms in $\calC$. Let $\{ Y_{\alpha} \}_{\alpha \in A}$ be a $\kappa$-good $S$-tree with root $X$, let $B \subseteq A$ be downward closed, and suppose given a commutative diagram
$$ \xymatrix{ Y_{B} \ar[r] \ar[d]^{T_B} \ar[r] & Y_A \ar[d]^{T_A} \\
Y_{B} \ar[r] & Y_A }$$
in $\calC_{X/}$, where $T_A$ and $T_B$ are idempotent. Let $C_0 \subseteq A$ be a $\kappa$-small subset. Then there exists a downward closed $\kappa$-small subset $C \subseteq A$ containing $C_0$ and a pair idempotent maps $$T_{C}: Y_{C} \rightarrow Y_{C}$$
$$T_{B \cap C}: Y_{B \cap C} \rightarrow Y_{B \cap C}$$ such that the following diagram
commutes $($in $\calC_{X/}${}$)$:
$$ \xymatrix{ Y_{B} \ar[d]^{T_B} & Y_{B \cap C} \ar[l] \ar[r] \ar[d]^{T_{B \cap C}} & Y_{C} \ar[d]^{T_{C}} \ar[r] & Y_{A} \ar[d]^{T_A} \\
Y_{B} & Y_{B \cap C} \ar[l] \ar[r] & Y_{C} \ar[r] & Y_A.}$$
\end{lemma}

\begin{proof}
Enlarging $C_0$ if necessary, we may suppose that $C_0$ is downward closed.
We will define sequences of $\kappa$-small, downward closed subsets
$$C_0 \subseteq C_1 \subseteq \ldots \subseteq A$$
$$D_1 \subseteq D_2 \subseteq \ldots \subseteq B$$
and idempotent maps $\{ T_{C_i}: Y_{C_i} \rightarrow Y_{C_{i}} \}_{i \geq 1}$,
$\{ T_{D_i}: Y_{D_i} \rightarrow Y_{D_{i}} \}_{i \geq 1}$. Moreover, we will guarantee that the following conditions are satisfied:
\begin{itemize}
\item[$(i)$] For each $i > 0$, the set $D_i$ contains the intersection $B \cap C_{i-1}$.
\item[$(ii)$] For each $i > 0$, the set $C_{i}$ contains $D_i$.
\item[$(iii)$] For each $i > 0$, the diagrams 
$$ \xymatrix{ Y_{D_{i}} \ar[r] \ar[d]^{T_{D_i}} & Y_{B} \ar[d] & Y_{C_{i}} \ar[d]^{T_{C_i}} \ar[r] & Y_{A} \ar[d]^{T_A} \\
Y_{D_i} \ar[r] & Y_{B} & Y_{C_i} \ar[r] & Y_A}$$
are commutative.
\item[$(iv)$] For each $i > 2$, the diagrams
$$ \xymatrix{ Y_{D_{i-2}} \ar[d]^{T_{D_{i-2}}} \ar[r] & Y_{D_{i-1}} \ar[d]^{T_{D_{i-2}}} & Y_{C_{i-2}} \ar[d]^{T_{C_{i-2}}} \ar[r] & Y_{C_{i-1}} \ar[d]^{T_{C_{i-1}}} \\
Y_{D_{i-2}} \ar[d] & Y_{D_{i-1}} \ar[d] & Y_{C_{i-2}} \ar[d] & Y_{C_{i-1}} \ar[d] \\
Y_{D_{i}} \ar[r]^{=} & Y_{D_{i}} & Y_{C_{i}} \ar[r]^{=} & Y_{C_{i}} }$$
commute.
\item[$(v)$] For each $i > 1$, the diagram
$$ \xymatrix{ Y_{D_{i-1}} \ar[r] \ar[d]^{T_{D_{i-1}}} & Y_{C_{i-1}} \ar[d]^{T_{C_{i-1}}} \\
Y_{D_{i-1}} \ar[d] & Y_{C_{i-1}} \ar[d] \\
Y_{D_{i}} \ar[r] & Y_{C_{i}}}$$ is commutative.
\end{itemize}
The construction goes by induction on $i$. Using a compactness argument, we see that conditions $(iv)$ and $(v)$ are satisfied provided that we choose $C_i$ and $D_i$ to be sufficiently large. The existence of the desired idempotent maps satisfying $(iii)$ then follows from Lemma \ref{turkteck}, applied to the roots $\{ Y_{\alpha} \}_{\alpha \in A}$ and $\{ Y_{\alpha} \}_{\alpha \in B}$.
We now take $C = \bigcup C_{i}$. Conditions $(i)$ and $(ii)$ guarantee that
$B \cap C = \bigcup D_{i}$. Using $(iv)$, it follows that the maps $\{ T_{C_{i}} \}$ and
$\{ T_{D_{i}} \}$ glue to give idempotent endomorphisms $T_{C}: Y_{C} \rightarrow Y_{C}$,
$T_{B \cap C}: Y_{B \cap C} \rightarrow Y_{B \cap C}$. Using $(iii)$ and $(v)$, we deduce that all of the desired diagrams are commutative.
\end{proof}

\begin{lemma}\label{tirun}
Let $\calC$ be a presentable category, $\kappa$ a regular cardinal, and suppose that $\calC$ is $\kappa$-accessible: that is, $\calC$ is generated under $\kappa$-filtered colimits by $\kappa$-compact objects $($Definition \ref{kapacc}$)$. Let $f: C \rightarrow D$ be a morphism between $\kappa$-compact object of $\calC$, let
$g: X \rightarrow Y$ be a pushout of $f$ $($so that $Y \simeq X \coprod_{C} D${}$)$, and let
$g': X' \rightarrow Y'$ be a retract of $g$ in the category of morphisms of $\calC$.
Then there exists a morphism $f': C' \rightarrow D'$ with the following properties:
\begin{itemize}
\item[$(1)$] The objects $C', D' \in \calC$ are $\kappa$-compact.
\item[$(2)$] The morphism $g'$ is a pushout of $f'$.
\item[$(3)$] The morphism $f'$ belongs to the weakly saturated class of morphisms generated by $f$.
\end{itemize}
\end{lemma}

\begin{proof}
Since $g'$ is a retract of $g$, there exists a commutative diagram
$$ \xymatrix{ X' \ar[r] \ar[d]^{g'} & X \ar[d]^{g} \ar[r] & X' \ar[d]^{g'} \\
Y' \ar[r] & Y \ar[r] & Y'. }$$
Replacing $g$ by the induced map $X' \rightarrow X' \coprod_{X} Y$, we can reduce to the case
where $X = X'$, and $Y'$ is a retract of $Y$ in $\calC_{X/}$. Then $Y'$ can be identified with
the image of some idempotent $i: Y \rightarrow Y$. 

Since $\calC$ is $\kappa$-accessible, we can write $X$ as the colimit of a $\kappa$-filtered
diagram $\{ X_{\lambda} \}$. The object $C$ is $\kappa$-compact by assumption. Refining our diagram if necessary, we may assume that it takes values in $\calC_{C/}$, and that $Y$ is
given as the colimit of the $\kappa$-filtered diagram $\{ X_{\lambda} \coprod_{C} D \}$. 

Because $D$ is $\kappa$-compact, the composition
$D \rightarrow Y \stackrel{i}{\rightarrow} Y$ admits a factorization
$$ D \stackrel{j}{\rightarrow} X_{\lambda} \coprod_{C} D \rightarrow Y.$$
The $\kappa$-compactness of $C$ implies that, after enlarging $\lambda$ if necessary, we may suppose that the composition $j \circ f$ coincides with the canonical map from $C$
to $X_{\lambda} \coprod_{C} D$. Consequently, $j$ and the $\id_{X_{\lambda}}$ determine
a map $i'$ from $Y_{\lambda} = X_{\lambda} \coprod_{C} D$ to itself. Enlarging $\lambda$ once more, we may suppose that $i'$ is idempotent, and that the diagram
$$ \xymatrix{ Y_{\lambda} \ar[d]^{i'} \ar[r] & Y \ar[d]^{i} \\
Y_{\lambda} \ar[r] & Y }$$
is commutative. Let $Y'_{\lambda}$ be the image of the idempotent $i'$, and let
$f': X_{\lambda} \rightarrow Y'_{\lambda}$ be the canonical map. Then $f'$ is a retract of the map
$X_{\lambda} \rightarrow Y_{\lambda}$, which is a pushout of $f$. This proves $(3)$. The objects
$X_{\lambda}$ and $Y'_{\lambda}$ are $\kappa$-compact by construction, so that $(1)$ is satisfied. We now observe that the diagram
$$ \xymatrix{ X_{\lambda} \ar[r] \ar[d] & Y'_{\lambda} \ar[d] \\
X \ar[r] & Y' }$$
is a retract of the pushout diagram
$$ \xymatrix{ X_{\lambda} \ar[r] \ar[d] & Y_{\lambda} \ar[d] \\
X \ar[r] & Y, }$$
and therefore itself a pushout diagram. This proves $(2)$ and completes the proof. 
\end{proof}

\begin{lemma}\label{tiura}
Let $\calC$ be a presentable category, $\kappa$ a regular cardinal such that $\calC$ is $\kappa$-accessible, and  $S = \{ f_s: C_s \rightarrow D_s \}$ a collection of morphisms $\calC$ such that each $C_s$ is $\kappa$-compact. Let $\{ Y_{\alpha} \}_{ \alpha \in A}$ be an $S$-tree in $\calC$,
with root $X$, and suppose that $A$ is $\kappa$-small. Then there exists a map
$X' \rightarrow X$, where $X$ is $\kappa$-compact, an $S$-tree
$\{ Y'_{\alpha} \}_{\alpha \in A}$ with root $X'$, and an isomorphism of $S$-trees
$$ \{ Y'_{\alpha} \coprod_{X'} X \}_{\alpha \in A} \simeq \{Y_{\alpha} \}_{\alpha \in A}$$
$($see Remark \ref{asstree}$)$. 
\end{lemma}

\begin{proof}
Since $\calC$ is $\kappa$-accessible, we can write $X$ as the colimit of 
diagram $\{ X_{i} \}_{i \in I}$ indexed by a $\kappa$-filtered partially ordered set $I$, where each $X_{i}$ is $\kappa$-compact. 
Choose a transfinite sequence of downward-closed subsets $\{ A(\gamma) \subseteq A \}_{\gamma \leq \beta}$ so that the following conditions are satisfied:
\begin{itemize}
\item[$(i)$] If $\gamma' \leq \gamma \leq \beta$, then $A(\gamma') \subseteq A(\gamma)$.
\item[$(ii)$] If $\lambda \leq \beta$ is a limit ordinal (possibly zero), then
$A(\lambda) = \bigcup_{ \gamma < \lambda} A(\gamma)$.
\item[$(iii)$] If $\gamma + 1 \leq \beta$, then $A(\gamma+1) = A(\gamma) \cup \{ \alpha_{\gamma} \}$, where $\alpha_{\gamma}$ is a minimal element of $A - A(\gamma)$.
\item[$(iv)$] The subset $A(\beta)$ coincides with $A$.
\end{itemize}
Note that, since $A$ is $\kappa$-small, we have $\beta < \kappa$.

We will construct:
\begin{itemize}
\item[$(a)$] A transfinite sequence of elements $\{ i_{\gamma} \in I \}_{\gamma \leq \beta}$, such that $i_{\gamma} \leq i_{\gamma'}$ for $\gamma \leq \gamma'$. 
\item[$(b)$] A sequence of $S$-trees $\{ Y^{\gamma}_{\alpha} \}_{\alpha \in A(\gamma)} \}$, having roots $X_{i_{\gamma}}$. 
\item[$(c)$] A collection of isomorphisms of $S$-trees
$$ \{ Y^{\gamma}_{\alpha} \coprod_{X_{i_{\gamma}}} X_{i_{\gamma'}} \}_{\alpha \in A(\gamma)}
\simeq \{ Y^{\gamma'}_{\alpha} \}_{\alpha \in A(\gamma)}$$
$$ \{ Y^{\gamma}_{\alpha} \coprod_{ X_{i_{\gamma}}} X \}_{\alpha \in A(\gamma)}
\simeq \{ Y_{\alpha} \}_{ \alpha \in A(\gamma) }$$
which are compatible with one another in the obvious sense.
\end{itemize}
If $\gamma$ is a limit ordinal (or zero), we simply choose $i_{\gamma}$ to be
any upper bound for $\{ i_{\gamma'} \}_{\gamma' < \gamma}$ in $I$. The rest of the data is uniquely determined. The existence of such an upper bound is guaranteed by our assumption that $I$ is $\kappa$-filtered, since $\gamma \leq \beta < \kappa$. Let us therefore suppose that the above data has been constructed for all ordinals $\leq \gamma$, and proceed to define $i_{\gamma+1}$.
Let $i = i_{\gamma}$, $\alpha = \alpha_{\gamma}$, and let $B = \{ \beta \in A: \beta < \alpha \}$. Then
we have canonical isomorphisms
$$Y_{B} \simeq Y^{\gamma}_{B} \coprod_{X_i} X
\simeq \colim \{ Y^{\gamma}_{B} \coprod_{X_i} X_j \}_{j \geq i},$$
and a pushout diagram
$$ \xymatrix{ C_{s} \ar[r]^{f_{s}} \ar[d]^{g} & D_{s} \ar[d] \\
Y_B \ar[r] & Y_{\alpha}. }$$
The $\kappa$-compactness of $C_{s}$ implies that $g$ factors as a composition
$$ C_{s} \stackrel{g'}{\rightarrow} Y^{\gamma}_{B} \coprod_{ X_i} X_j $$
for some $j \geq i$. We now define $i_{\gamma+1} = j$, and
$Y^{\gamma+1}_{\alpha}$ by forming a pushout diagram
$$ \xymatrix{ C_{s} \ar[d]^{g_{s}} \ar[r] & D_{s} \ar[d] \\
Y^{\gamma}_{B} \coprod_{X_i} X_j \ar[r] & Y^{\gamma+1}_{\alpha}. }$$
\end{proof}

\begin{proposition}\label{easycrust}
Let $\calC$ be presentable $\infty$-category, $\kappa$ a regular cardinal, $\overline{S}$ a weakly saturated class of morphisms in $\calC$. Let $S \subseteq \overline{S}$ be the subset consisting of those morphisms $f: X \rightarrow Y$ in $\overline{S}$ such that $X$ and $Y$ are $\kappa$-compact. Assume that:
\begin{itemize}
\item[$(i)$] The regular cardinal $\kappa$ is uncountable, and $\calC$ is $\kappa$-accessible.
\item[$(ii)$] The set $S$ generates $\overline{S}$ as a weakly saturated class of morphisms.
\end{itemize}
Then, for every morphism $f: X \rightarrow Y$ belonging to $\overline{S}$, there
exists an transfinite sequence of objects $\{ Y_{\gamma} \}_{\gamma < \beta}$ of
$\calC_{X/}$ with the following properties:
\begin{itemize}
\item[$(1)$] For every ordinal $\gamma < \beta$, the natural map
$\colim_{\gamma' < \gamma} Z_{\gamma'} \rightarrow Z_{\gamma}$ is the pushout of a morphism in  $S$.
\item[$(2)$] The colimit $\colim_{\gamma < \beta} Z_{\gamma}$ is isomorphic to $Y$
(as objects of $\calC_{X/}$). 
\end{itemize}
\end{proposition}

\begin{proof}
Remark \ref{easyprest} implies the existence of a transfinite sequence of objects
$$ Y_0 \rightarrow Y_1 \rightarrow \ldots $$
in $\calC_{X/}$ indexed by a set of ordinals $A = \{ \alpha | \alpha < \lambda \}$, 
satisfying condition $(1)$, such that
$Y$ is a {\em retract} of $\colim_{\alpha < \lambda} Y_{\alpha}$ in $\calC_{X/}$. We may view
the sequence $\{ Y_{\alpha} \}_{ \alpha \in A}$ as an $S$-tree in $\calC$, having root $X$. According to Lemma \ref{humber2}, we can choose a new $S$-tree $\{ Y'_{\alpha} \}_{ \alpha \in A'}$
which is $\kappa$-good, where $Y'_{A'} \simeq Y_A$, so that $Y$ is a retract of
$Y'_{A'}$. Choose an idempotent map $T_{A'}: Y'_{A'} \rightarrow Y'_{A'}$ in $\calC_{X/}$, whose
image is isomorphic to $Y$.

We now define a transfinite sequence
$$ B(0) \subseteq B(1) \subseteq B(2) \subseteq \ldots, $$
indexed by ordinals $\gamma < \beta$, and a compatible system of idempotent maps $T_{B(\gamma)}: Y'_{B_\gamma} \rightarrow Y'_{B_{\gamma}}.$
Fix an ordinal $\gamma$, and suppose that $B(\gamma')$ and $T_{B(\gamma')}$ have been defined for $\gamma' < \gamma$. Let $B'(\gamma) = \bigcup_{ \gamma' < \gamma} B(\gamma')$, and let $T_{B'(\gamma)}$ be the result of amalgamating the maps $\{ T_{B(\gamma')} \}_{\gamma' < \gamma}$. If $B'(\gamma) = A'$, we set $\beta = \gamma$ and conclude the construction; otherwise, choose a minimal element $a \in A' - B'(\gamma)$. Applying Lemma \ref{superturk}, we deduce the existence of a downward closed subset $C(\gamma) \subseteq A'$, and a compatible collection of idempotent maps 
$$ T_{C(\gamma)} : Y'_{C(\gamma)} \rightarrow Y'_{C(\gamma)} $$
$$ T_{C(\gamma) \cap B'(\gamma)}: Y'_{C(\gamma) \cap B'(\gamma)} \rightarrow Y'_{C(\gamma) \cap B'(\gamma)}.$$
We then define $B(\gamma) = B'(\gamma) \cup C(\gamma)$, and
$T_{B(\gamma)}$ to be the result of amalgamating $T_{B'(\gamma)}$ and $T_{C(\gamma)}$. 

We observe that, for every ordinal $\gamma$, there is a $\kappa$-good $S$-tree 
$\{ Y''_{\alpha} \}_{ \alpha \in B(\gamma) - B'(\gamma)}$
with root $Y'_{B(\gamma)}$, such that $Y''_{B(\gamma)-B'(\gamma)} \simeq Y'_{B(\gamma)}$ (Remark \ref{relci}). Combining Lemma \ref{tiura} with the observation that 
$B(\gamma)-B'(\gamma)$ is $\kappa$-small, we deduce that the map
$$ Y'_{B'(\gamma)} \rightarrow Y'_{B(\gamma)}$$ is the pushout of a morphism in $S$.

For each ordinal $\gamma < \beta$, let $Z_{\gamma}$ denote the image of the idempotent map
$T_{B(\gamma)}$. Then $\colim_{\gamma < \beta} Z_{\gamma} \simeq Y$, so that $(2)$ is satisfied. Condition $(1)$ follows from Lemma \ref{tirun}.
\end{proof}

\begin{corollary}\label{unitape}
Under the hypotheses of Proposition \ref{easycrust}, there exists a $\kappa$-good
$S$-tree $\{ Y_{\alpha} \}_{\alpha \in A}$ such that $Y_{A} \simeq Y$ in $\calC_{X/}$.
\end{corollary}

\begin{proof}
Combine Proposition \ref{easycrust} with Lemma \ref{humber2}.
\end{proof}


\section{Model Categories}\label{appmodelcat}

One of the oldest and most successful approaches to the study of
$\infty$-categorical phenomena is Quillen's theory of model
categories. In this book, Quillen's theory will play two (related) roles:

\begin{itemize}
\item[$(1)$] The structures that we use to describe higher categories are naturally organized into model categories. For example, $\infty$-categories are precisely those simplicial sets which are fibrant with respect to the Joyal model structure (Theorem \ref{joyalcharacterization}).
The theory of model categories provides a convenient framework for phrasing certain results and for comparing different models of higher category theory (see, for example, \S \ref{compp3}).

\item[$(2)$] The theory of model categories can itself be regarded as an approach to higher
category theory. If $\bfA$ is a simplicial model category, then the subcategory $\bfA^{\degree} \subseteq \bfA$ of fibrant-cofibrant objects forms a fibrant simplicial category. Proposition \ref{toothy} implies that the simplicial nerve $\sNerve(\bfA^{\degree})$ is an $\infty$-category. We will refer to $\sNerve(\bfA^{\degree})$ as the {\it underlying $\infty$-category} of\index{not}{Adegree@$\bfA^{\degree}$}
$\bfA$. Of course, not every $\infty$-category arises in this way, even up to equivalence: for example, the existence of homotopy limits and homotopy colimits in $\bfA$ implies the existence of various limits and colimits in $\sNerve(\bfA^{\degree})$ (Corollary \ref{limitsinmodel}). Nevertheless, we can often use the theory of model categories to prove theorems about general $\infty$-categories, by reducing to the situation of $\infty$-categories which arise via the above construction (every
$\infty$-category $\calC$ admits a fully faithful embedding into $\sNerve( \bfA^{\degree})$, for
an appropriately chosen simplicial model category $\bfA$). For example, our proof of the $\infty$-categorical Yoneda lemma (Proposition \ref{fulfaith}) uses this strategy.
\end{itemize}

%\begin{remark}
%Using a slightly more complicated construction, one can associate an $\infty$-category to a model category $\bfA$ which is not simplicially enriched. Namely, consider the simplicial set
%$S = \Nerve(\bfA)$ as a {\em marked} simplicial set (see \S \ref{bicat1}) in which the marked edges are those which correspond to weak equivalences in $\bfA$, and choose a marked anodyne map
%$S \rightarrow \calC^{\natural}$, where $\calC$ is an $\infty$-category. Then $\calC$ is canonically determined by $\bfA$ up to equivalence (in fact, up to contractible ambiguity). In the case where
%$\bfA$ is a simplicial model category, $\calC$ is equivalent the simplicial nerve $\sNerve (\bfA^{\degree})$. We will discuss this construction briefly in \S \ref{needles}, but it will not be needed in the main text of this book.
%\end{remark}

The purpose of this section is to review the theory of model categories, with an eye towards the sort of applications described above. Our exposition is somewhat terse and we will omit many proofs.
For a more detailed account, we refer the reader to \cite{hovey} (or any other text on the theory of model categories).

\subsection{The Model Category Axioms}

\begin{definition}\label{modelcatdef}\index{gen}{model category}\index{gen}{category!model}
A {\it model category} is a category $\calC$ which is equipped with three distinguished classes of morphisms in $\calC$, called {\it cofibrations, fibrations,} and {\it weak equivalences}, in which the following axioms are satisfied:
\begin{itemize}
\item[$(1)$] The category $\calC$ admits (small) limits and colimits.
\item[$(2)$] Given a composable pair of maps $X \stackrel{f}{\rightarrow} Y \stackrel{g}{\rightarrow} Z$, if any two of $g \circ f$, $f$, and $g$ are weak equivalences, then so is the third.
\item[$(3)$] Suppose $f: X \rightarrow Y$ is a retract of $g: X' \rightarrow Y'$: that is, suppose
there exists a commutative diagram
$$ \xymatrix{ X \ar[r]^{i} \ar[d]^{f} & X' \ar[d]^{g} \ar[r]^{r} & X \ar[d]^{f} \\
Y \ar[r]^{i'} & Y' \ar[r]^{r'} & Y } $$
where $r \circ i = \id_X$ and $r' \circ i' = \id_Y$. Then
\begin{itemize}
\item[$(i)$] If $g$ is a fibration, so is $f$.\index{gen}{fibration}
\item[$(ii)$] If $g$ is a cofibration, then so is $f$.\index{gen}{cofibration}
\item[$(iii)$] If $g$ is a weak equivalence, then so is $f$.\index{gen}{weak equivalence}
\end{itemize}

\item[$(4)$] Given a diagram
$$ \xymatrix{ A \ar[d]^{i} \ar[r] & X \ar[d]^{p} \\
B \ar[r] \ar@{-->}[ur] & Y,}$$
a dotted arrow can be found rendering the diagram commutative if either
\begin{itemize}
\item[$(i)$] The map $i$ is a cofibration, and the map $p$ is both a fibration and a weak equivalence.
\item[$(ii)$] The map $i$ is both a cofibration and a weak equivalence, and the map $p$ is a fibration.
\end{itemize}
\item[$(5)$] Any map $X \rightarrow Z$ in $\calC$ admits factorizations
$$ X \stackrel{f}{\rightarrow} Y \stackrel{g}{\rightarrow} Z$$
$$ X \stackrel{f'}{\rightarrow} Y' \stackrel{g'}{\rightarrow} Z$$
where $f$ is a cofibration, $g$ is a fibration and a weak equivalence, $f'$ is a cofibration and a weak equivalence, and $g'$ is a fibration.
\end{itemize}
\end{definition}

A map $f$ in a model category $\calC$ is called a {\it trivial cofibration} if it is both a cofibration and a weak equivalence; similarly $f$ is called a {\it trivial fibration} if it is both a fibration and a weak equivalence. By axiom $(1)$, any model category $\calC$ has an initial object $\emptyset$
and a final object $\ast$. An object $X \in \calC$ is said to be {\it fibrant} if the unique map
$X \rightarrow \ast$ is a fibration, and {\it cofibrant} if the unique map $\emptyset \rightarrow X$ is a cofibration.\index{gen}{fibrant}\index{gen}{cofibrant}\index{gen}{trivial!fibration}\index{gen}{trivial!cofibration}

\begin{example}\label{trivmodel}
Let $\calC$ be any category which admits small limits and colimits. Then
$\calC$ can be endowed with the {\em trivial} model structure:
\begin{itemize}
\item[$(W)$] The weak equivalences in $\calC$ are the isomorphisms.
\item[$(C)$] Every morphism in $\calC$ is a cofibration.
\item[$(F)$] Every morphism in $\calC$ is a fibration. 
\end{itemize}
\end{example}

\subsection{The Homotopy Category of a Model Category}

Let $\calC$ be a model category containing an object $X$. A {\it cylinder object} for $X$\index{gen}{cylinder object}\index{gen}{object!cylinder}
is an object $C$ together with a diagram
$ X \coprod X \stackrel{i}{\rightarrow} C \stackrel{j}{\rightarrow} X$
where $i$ is a cofibration and $j$ is a weak equivalence, and the composition $j \circ i$
is the ``fold map'' $X \coprod X \rightarrow X$. 
Dually, a {\it path object} for $Y \in \calC$ is an object $P$ together with a diagram $$Y \stackrel{q}{\rightarrow} P \stackrel{p}{\rightarrow} Y \times Y$$ such that $q$ is a weak equivalence, $p$ is a fibration, and $p \circ q$ is the diagonal map $Y \rightarrow Y \times Y$.
The existence of cylinder and path objects follows from the factorization axiom $(5)$ of Definition \ref{modelcatdef} (factor the ``fold map'' $X \coprod X \rightarrow X$ as a cofibration followed by a trivial fibration and the diagonal map $Y \rightarrow Y \times Y$ as a trivial cofibration followed by a fibration).\index{gen}{path object}\index{gen}{object!path}

\begin{proposition}\label{homotopy}
Let $\calC$ be a model category. Let $X$ be a cofibrant object of $\calC$, $Y$ a fibrant object of $\calC$, and $f,g: X \rightarrow Y$ two maps. The following conditions are equivalent:
\begin{itemize}
\item[$(1)$] For every cylinder object $X \coprod X \stackrel{j}{\rightarrow} C$, there exists a commutative diagram
$$ \xymatrix{ X \coprod X \ar[rr]^{j} \ar[dr]^{(f,g)} & & C \ar[dl] \\
 & Y}$$

\item[$(2)$] There exists a cylinder object $X \coprod X \stackrel{j}{\rightarrow} C$ and a commutative diagram
$$ \xymatrix{ X \coprod X \ar[rr]^{j} \ar[dr]^{(f,g)} & & C \ar[dl] \\
 & Y}$$

\item[$(3)$] For every path object $P \stackrel{p}{\rightarrow} Y \times Y$, there exists a commutative diagram $$ \xymatrix{ X  \ar[rr] \ar[dr]^{(f,g)} & & P \ar[dl]^{p} \\
 & Y \times Y}$$

\item[$(4)$] There exists a path object $P \stackrel{p}{\rightarrow} Y \times Y$ and a commutative diagram $$ \xymatrix{ X  \ar[rr] \ar[dr]^{(f,g)} & & P \ar[dl]^{p} \\
 & Y \times Y}$$
\end{itemize}
\end{proposition}

If $\calC$ is a model category containing a cofibrant object $X$ and a fibrant object $Y$, we say two maps $f,g: X \rightarrow Y$ are {\it homotopic} if the hypotheses of Proposition \ref{homotopy} are satisfied, and write $f \simeq g$. The relation $\simeq$ is an equivalence relation on\index{gen}{homotopy!between morphisms in a model category} $\Hom_{\calC}(X,Y)$. The {\it homotopy category} $\h{\calC}$ may be defined as follows:\index{gen}{homotopy category!of a model category}\index{gen}{category!homotopy}\index{not}{hcalC@$\h{\calC}$}

\begin{itemize}
\item The objects of $\h{\calC}$ are the fibrant-cofibrant objects of $\calC$.
\item For $X,Y \in \h{\calC}$, the set $\Hom_{\h{\calC}}(X,Y)$ is the set of $\simeq$-equivalence classes
of $\Hom_{\calC}(X,Y)$.
\end{itemize}

Composition is well-defined in $\h{\calC}$, in virtue of the fact that if $f \simeq g$, then
$f \circ h \simeq g \circ h$ (this is clear from characterization $(2)$ of Proposition \ref{homotopy}) and $h' \circ f \simeq h' \circ g$ (this is clear from characterization $(4)$ of Proposition \ref{homotopy}), for any maps $h,h'$ such that the compositions are defined in $\calC$.

There is another way of defining $\h{\calC}$ (or at least, a category equivalent to $\h{\calC}$): one begins with all of $\calC$ and formally adjoins inverses to all weak equivalences. Let $H(\calC)$ denote the category so-obtained. If $X \in \calC$ is cofibrant and $Y \in \calC$ is fibrant, then homotopic maps $f,g: X \rightarrow Y$ have the same image in $H(\calC)$; consequently we obtain a functor $\h{\calC} \rightarrow H(\calC)$ which can be shown to be an equivalence. We will generally ignore the distinction between these two categories, employing whichever description is more useful for the problem at hand.

\begin{remark}
Since $\calC$ is (generally) not a small category, it is not immediately clear that $H(\calC)$ has small morphism sets; however, this follows from the equivalence between $H(\calC)$ and $\h{\calC}$.
\end{remark}

\subsection{A Lifting Criterion}

The following basic principle will be used many times throughout this book:

\begin{proposition}\label{princex}
Let $\calC$ be model category containing cofibrant objects $A$ and $B$, and a fibrant object $X$.
Suppose given a cofibration $i: A \rightarrow B$ and any map $f: A \rightarrow X$. Suppose
moreover that there exists a commutative diagram
$$ \xymatrix{ A \ar[dd]^{[i]} \ar[dr]^{[f]} & \\
& X \\
B \ar[ur]^{\overline{g}} } $$
in the homotopy category $h \calC$. Then there exists a commutative diagram
$$ \xymatrix{ A \ar[dd]^{i} \ar[dr]^{f} & \\
& X \\
B \ar[ur]^{g} } $$ in $\calC$, with $[g] = \overline{g}$. $($Here we let $[p]$ denote the homotopy class in $\h{ \calC}$ of a morphism $p$ in $\calC$.$)$ 
\end{proposition}

\begin{proof}
Choose a map $g': B \rightarrow X$ representing the homotopy class $\overline{g}$.
Choose a cylinder object
$$A \coprod A \rightarrow C(A) \rightarrow A,$$
and a factorization
$$ C(A) \coprod_{A \coprod A} (B \coprod B) \rightarrow C(B) \rightarrow B$$
where the first map is a cofibration and the second a trivial fibration. We observe that
$C(B)$ is a cylinder object for $B$.

Since $g' \circ i$ is homotopic to $f$, there exists a map $h_0: C(A) \coprod_{A} B \rightarrow X$
with $h|B = g'$ and $h|A = f$. The inclusion $C(A) \coprod_{A} B \rightarrow C(B)$ is a trivial cofibration, so $h_0$ extends to a map $h: C(B) \rightarrow X$. We may regard $h$ as a homotopy from $g'$ to $g$, where $g \circ i = f$.
\end{proof}

Proposition \ref{princex} will often be applied in the following way. Suppose given a diagram
$$ \xymatrix{ A' \ar[r] \ar[dd] & A \ar[dd]^{i} \ar[dr]^{f} & \\
& & X \\
B' \ar[r] & B \ar@{-->}[ur] } $$
which we would like to extend as indicated by the dotted arrow. If $X$ is fibrant, $i$ is a cofibration between cofibrant objects, and the horizontal arrows are weak equivalences, then it suffices
to solve the (frequently easier) problem of constructing the dotted arrow in the diagram
$$ \xymatrix{ A' \ar[dd] \ar[drr] & \\
& & X \\
B' \ar@{-->}[urr] }.$$

\subsection{Left Properness and Homotopy Pushout Squares}\label{hopush}

\begin{definition}
A model category $\calC$ is {\it left proper} if, for any pushout square\index{gen}{left proper}\index{gen}{model category!left proper}
$$ \xymatrix{ A \ar[r]^{i} \ar[d]^{j} & B \ar[d]^{j'} \\
A' \ar[r]^{i'} & B'}$$
in which $i$ is a cofibration and $j$ is a weak equivalence, the map $j'$ is also a weak equivalence. Dually, $\calC$ is {\it right proper} if, for any pullback square\index{gen}{model category!right proper}\index{gen}{right proper}
$$ \xymatrix{ X' \ar[r]^{p'} \ar[d]^{q'} & Y' \ar[d]^{q} \\
X \ar[r]^{q} & Y}$$
in which $p$ is a fibration and $q$ is a weak equivalence, the map $q'$ is also a weak equivalence.
\end{definition}

In this book, we will deal almost exclusively with left proper model categories. The following provides a useful criterion for establishing left-properness.

\begin{proposition}\label{propob}
Let $\calC$ be a model category in which every object is cofibrant. Then $\calC$ is left proper.
\end{proposition}

Proposition \ref{propob} is an immediate consequence of the following basic lemma:

\begin{lemma}
Let $\calC$ be a model category containing a pushout diagram
$$ \xymatrix{ A \ar[r]^{i} \ar[d]^{j} & B \ar[d]^{j'} \\
A' \ar[r]^{i'} & B'.}$$
Suppose that $A$ and $A'$ are cofibrant, $i$ is a cofibration, and $j$ is a weak equivalence.
Then $j'$ is a weak equivalence.
\end{lemma}

\begin{proof}
We wish to show that $j'$ is an isomorphism in the homotopy category $h\calC$. In other words, we need to show that for every fibrant object
$Z$ of $\calC$, composition with $j'$ induces a bijection $\Hom_{\h{\calC}}(B',Z) \rightarrow \Hom_{\h{ \calC}}(B,Z)$.

We first show that composition with $j'$ is surjective on homotopy classes. Suppose given
a map $f: B \rightarrow Z$. Since $j$ is a weak equivalence, the composition $f \circ i$ is homotopic to $g \circ j$, for some $g: A' \rightarrow B$. According to Proposition \ref{princex}, there
is a map $f': B \rightarrow Z$ such that $f' \circ i = g \circ j$, and such that $f'$ is homotopic to $f$. The amalgamation of $f'$ and $g$ determines a map $B' \rightarrow Z$ which lifts $f'$.

We now show that $j'$ is injective on homotopy classes. Suppose given a pair of maps
$s,s': B' \rightarrow Z$. Let $P$ be a path object for $Z$. If $s \circ j'$ and $s' \circ j'$ are homotopic, then there exists a commutative diagram
$$ \xymatrix{ B \ar[r]^{h} \ar[d]^{j'} & P \ar[d] \\
B' \ar[r]^{s \times s'} & Z \times Z.}$$
We now replace $\calC$ by $\calC_{/Z \times Z}$ and apply the surjectivity statement above
to deduce that there is a map $h': B' \rightarrow P$ such that $h$ is homotopic to $h' \circ j'$. The existence of $h'$ shows that $s$ and $s'$ are homotopic, as desired.
\end{proof}

Suppose given a diagram
$$ A_0 \leftarrow A \rightarrow A_1$$ in a model category $\calC$. In general, the pushout
$ A_0 \coprod_A A_1$ is poorly behaved, in the sense that a map of diagrams
$$ \xymatrix{ A_0 \ar[d] & A \ar[l] \ar[r] \ar[d] & A_1 \ar[d]\\
B_0 & B \ar[l] \ar[r] & B_1 }$$
need not induce a weak equivalence $A_0 \coprod_A A_1 \rightarrow B_0 \coprod_B B_1$, even if
each of the vertical arrows in the diagram is individually a weak equivalence. To correct this difficulty, it is convenient to introduce the left-derived functor of ``pushout''. The {\it homotopy pushout}\index{gen}{homotopy pushout}\index{gen}{pushout!homotopy} of the diagram
$$ \xymatrix{ A_0 & A \ar[l] \ar[r] & A_1 } $$
is defined to be the pushout $A'_0 \coprod_{ A' } A'_1$, where we have chosen a commutative diagram
$$ \xymatrix{ A'_0 \ar[d] & A' \ar[d] \ar[r]^{i} \ar[l]_{j} & A'_1 \ar[d]\\
A_0 & A \ar[l] \ar[r] & A_1 } $$
in which the top row is a {\em cofibrant} diagram, in the sense that $A'$ is cofibrant and the maps
$i$ and $j$ are both cofibrations. One can show that such a diagram exists, and that the pushout $A_0' \coprod_{A'} A_1'$ depends on the choice of diagram only up to weak equivalence. (For a more systematic approach which includes a definition of ``cofibrant'' for more complicated diagrams, we refer the reader to \S \ref{quasilimit3}.) 

More generally, we will say that a diagram
$$ \xymatrix{ & A \ar[dr] \ar[dl] & \\
A_0 \ar[dr] & & A_1 \ar[dl] \\
& M }$$
is a {\it homotopy pushout square} if the composite map
$$ A'_0 \coprod_{A'} A'_1 \rightarrow A_0 \coprod_{A} A_1 \rightarrow M$$
is a weak equivalence. In this case we will also say that $M$ is a {\it homotopy pushout} of
$A_0$ and $A_1$ over $A$. One can show that this condition is independent of the choice of
``cofibrant resolution'' $$ \xymatrix{ A'_0 & A' \ar[l] \ar[r] & A'_1}$$ of the original diagram.
In particular, we note that if the diagram
$$ \xymatrix{ A_0 & A \ar[r] \ar[l] & A_1 }$$
is {\em already} cofibrant, then the ordinary pushout $A_0 \coprod_A A_1$ is a homotopy pushout. However, the condition that the diagram be cofibrant is quite strong; in good situations we can get away with quite a bit less:

\begin{proposition}\label{leftpropsquare}
Let $\calC$ be a model category, and let
$$ \xymatrix{ & A \ar[dl]^{i} \ar[dr]^{j} & \\
A_0 \ar[dr] & & A_1 \ar[dl] \\
& A_0 \coprod_A A_1 } $$
be a pushout square in $\calC$. This diagram is also a homotopy pushout square if
either of the following conditions is satisfied:
\begin{itemize}
\item[$(i)$] The objects $A$ and $A_0$ are cofibrant, and $j$ is a cofibration.
\item[$(ii)$] The map $j$ is a cofibration, and $\calC$ is left proper.
\end{itemize}
\end{proposition}

\begin{remark}
The above discussion of homotopy pushouts can be dualized; one obtains the notion of {\it homotopy pullbacks}, and the analogue of Proposition \ref{leftpropsquare} requires either that $\calC$ be a {\em right} proper model category or that the objects in the diagram be fibrant.\index{gen}{homotopy pullback}\index{gen}{pullback!homotopy}
\end{remark}

\subsection{Quillen Adjunctions and Quillen Equivalences}\label{quilladj}

Let $\calC$ and $\calD$ be model categories, and suppose given a pair of adjoint functors
$$ \Adjoint{F}{\calC}{\calD}{G} $$
(here $F$ is the left adjoint and $G$ is the right adjoint). The following conditions are equivalent:

\begin{itemize}
\item[$(1)$] The functor $F$ preserves cofibrations and trivial cofibrations.

\item[$(2)$] The functor $G$ preserves fibrations and trivial fibrations.

\item[$(3)$] The functor $F$ preserves cofibrations and the functor $G$ preserves fibrations.

\item[$(4)$] The functor $F$ preserves trivial cofibrations and the functor $G$ preserves trivial fibrations.
\end{itemize}

If any of these equivalent conditions is satisfied, then we say that the pair $(F,G)$ is a {\it Quillen adjunction} between $\calC$ and $\calD$. We also say that $F$ is a {\it left Quillen functor} and that $G$ is a {\it right Quillen functor}. In this case, one can show that $F$ preserves weak equivalences between cofibrant objects, and $G$ preserves weak equivalences between fibrant objects.\index{gen}{adjunction!Quillen}\index{gen}{Quillen adjunction}

Suppose that $\Adjoint{F}{\calC}{\calD}{G} $ is a Quillen adjunction.
We may view the homotopy category $\h{\calC}$ as obtained from $\calC$ by first passing to the full subcategory consisting of cofibrant objects, and then inverting all weak equivalences. Applying a similar procedure with $\calD$, we see that because $F$ preserves weak equivalence between cofibrant objects, it induces a functor $\h{\calC} \rightarrow \h{\calD}$; this functor is called the {\it left derived functor of $F$} and denoted $LF$.\index{gen}{left derived functor}\index{gen}{derived functor!left} Similarly, one may define the {\it right derived functor} $RG$ of $G$\index{gen}{right derived functor}\index{gen}{derived functor!right}. One can show that $LF$ and $RG$ determine an adjunction between the homotopy categories $\h{\calC}$ and $\h{\calD}$. 

\begin{proposition}\label{quilleq}
Let $\calC$ and $\calD$ be model categories, and let
$$ \Adjoint{F}{\calC}{\calD}{G} $$
be a Quillen adjunction. The following
are equivalent:
\begin{itemize}
\item[$(1)$] The left derived functor $LF: \h{\calC} \rightarrow \h{\calD}$ is an equivalence of categories.
\item[$(2)$] The right derived functor $RG: \h{\calD} \rightarrow \h{\calC}$ is an equivalence of categories.
\item[$(3)$] For every cofibrant object $C \in \calC$ and every fibrant object $D \in \calD$, a map
$C \rightarrow G(D)$ is a weak equivalence in $\calC$ if and only if the adjoint map $F(C) \rightarrow D$ is a weak equivalence in $\calD$.
\end{itemize}
\end{proposition}

\begin{proof}
Since the derived functors $LF$ and $RG$ are adjoint to one another, it is clear that $(1)$ is equivalent to $(2)$. Moreover, $(1)$ and $(2)$ are equivalent to the assertion that the unit and counit of the adjunction
$$ u: \id_{\calC} \rightarrow RG \circ LF$$
$$ v: LF \circ RG \rightarrow \id_{\calD}$$
are weak equivalences. Let us consider the unit $u$. Choose a fibrant object $C$ of $\calC$.
The composite functor $(RG \circ LF)(C)$ is defined to be $G(D)$, where $F(C) \rightarrow D$ is a weak equivalence in $\calD$, and $D$ is a fibrant object of $\calD$. Thus, $u$ is a weak equivalence when evaluated on $C$ if and only if for any weak equivalence
$F(C) \rightarrow D$, the adjoint map $C \rightarrow G(D)$ is a weak equivalence. Similarly,
the counit $v$ is a weak equivalence if and only if the converse holds. Thus $(1)$ and $(2)$ are equivalent to $(3)$.
\end{proof}

If the equivalent conditions of Proposition \ref{quilleq} are satisfied, then we say that the adjunction $(F,G)$ gives a {\it Quillen equivalence} between the model categories $\calC$ and $\calD$.\index{gen}{Quillen equivalence}

%\subsection{Monoidal Model Categories}\label{monmod}

%\begin{definition}\label{monoidmodel}\index{gen}{model category!monoidal}
%A {\it monoidal model category} is a model category $\calC$, equipped with a closed monoidal structure $\otimes$ satisfying the following conditions:
%\begin{itemize}
%\item[$(1)$] For every pair of cofibrations $i: A \rightarrow A'$, $j: B \rightarrow B'$, the induced map
%$$k: (A \otimes B') \coprod_{ A \otimes B} (A' \otimes B) \rightarrow A' \otimes B'$$
%is a cofibration. Moreover, if either $i$ or $j$ is a weak equivalence, then $k$ is a weak equivalence.

%\item[$(2)$] The unit object $1$ of $\calC$ is cofibrant.
%\end{itemize}
%\end{definition}

%\begin{remark}
%It is customary to demand a weaker form of the second axiom in Definition \ref{monoidmodel}, in order to incorporate certain examples from homotopy theory; however, Definition \ref{monoidmodel} will be sufficiently general to cover all applications in this book.
%\end{remark}

%\begin{remark}
%Given a pair of maps $i: A \rightarrow A'$, $j: B \rightarrow B'$ in a monoidal category $\calC$, one can define their {\it smash $\otimes$-product} to be the map
%$$i \wedge j:  (A \otimes B') \coprod_{ A \otimes B} (A' \otimes B) \rightarrow A' \otimes B'.$$
%The operation $\wedge$ is associative, in the sense that it is possible to identify the source and target of the maps $(i \wedge j) \wedge k$ and $i \wedge (j \wedge k)$ in such a way that the two maps coincide. Definition \ref{monoidmodel} is equivalent to the assertion that if
%$i_1$, $i_2$, $\ldots$, $i_n$ is a finite sequence of cofibrations in $\calC$, then the smash product
%$$ i_1 \wedge \ldots \wedge i_n$$ is a cofibration, which is a weak equivalence if 
%$i_m$ is a weak equivalence for some $1 \leq m \leq n$. Axiom $(2)$ just amounts to the special case $n = 0$.
%\end{remark}

%The following observation is useful for establishing that a monoidal structure and a model structure are compatible:

%\begin{proposition}\label{monoidcompat}
%Let $\calC$ be a model category equipped with a closed monoidal structure $\otimes$. Suppose that:
%\begin{itemize}
%\item[$(1)$] Every object of $\calC$ is cofibrant.
%\item[$(2)$] For every pair of cofibrations $i: A \rightarrow A'$, $j: B \rightarrow B'$, the smash $\otimes$-product $i \wedge j$ is a cofibration.
%\end{itemize}
%Then $\calC$ is a monoidal model category if and only if, for each object $C \in \calC$, the functors
%$$ M \mapsto C \otimes M$$
%$$ N \mapsto N \otimes C$$
%preserve weak equivalences.
%\end{proposition}

%\begin{proof}
%Let $\emptyset$ denote the initial object of $\calC$. Since $\calC$ is a {\em closed} monoidal category, for each $C \in \calC$ the functors 
%$$ M \mapsto C \otimes M$$
%$$ N \mapsto N \otimes C$$
%commute with colimits; in particular, we have $C \otimes \emptyset \simeq \emptyset \simeq \emptyset \otimes C$, where $\emptyset$ denotes the initial object of $\calC$. In particular, we note that if $i_{C}$ denotes the map $\emptyset \rightarrow C$, then for any map $j: M \rightarrow N$, the smash $\otimes$-product $i_C \wedge j$ can be identified with 
%$\id_{C} \otimes j: C \otimes M \rightarrow C \otimes N$. 

%Suppose that $\calC$ is a monoidal model category. The above remarks show that if
%$j: M \rightarrow N$ is a (trivial) cofibration, then $i_{C} \wedge j = \id_{C} \otimes j$ is a (trivial) cofibration. Thus, the functor $M \mapsto C \otimes M$ is a left Quillen functor, and therefore preserves weak equivalences between cofibrant objects. Since every object of $\calC$ is cofibrant, 
%$M \mapsto C \otimes M$ preserves weak equivalences in general. Similarly, the functor
%$N \mapsto N \otimes C$ preserves weak equivalences.

%Now suppose that the functors
%$$ M \mapsto C \otimes M$$
%$$ N \mapsto N \otimes C$$
%preserve weak equivalences, for every object $C$ of $\calC$. Suppose given a pair of cofibrations
%$i: A \rightarrow A'$, $j: B \rightarrow B'$. By assumption $i \wedge j$ is a cofibration; we must show that $i \wedge j$ is a weak equivalence if either $i$ or $j$ is a weak equivalence. We will treat the case where $i$ is a weak equivalence; the other case follows by a dual argument. Consider the diagram
%$$ \xymatrix{ A \otimes B \ar[rr]^{i \otimes \id_{B}} \ar[d] & & A' \otimes B \ar[d] & \\
%A \otimes B' \ar[rr]^{f} & & (A' \otimes B) \coprod_{ A \otimes B} (A \otimes B') \ar[r] & A' \otimes B'.}$$
%By assumption, $i \otimes \id_{B}$ is a weak equivalence. The square in the diagram is a homotopy pushout, so Proposition \ref{propob} implies that $f$ is a weak equivalence as well. 
%The hypothesis implies also that
%$(i \wedge j) \circ f = i \otimes \id_{B'}$ is a weak equivalence. Thus $i \wedge j$ is a weak equivalence, by the two-out-of-three property.
%\end{proof}

%\subsection{Enriched Model Categories}

%\begin{definition}\label{enrichmodel}\index{gen}{model category!enriched}
%Let $(\calC,\otimes)$ be a monoidal model category, and let $\calD$ be a category enriched over
%$\calC$. Suppose that the underlying category of $\calD$ is equipped with a model structure.
%We will say that $\calD$ is a {\it $\calC$-enriched model category} if the following conditions
%are satisfied:

%\begin{itemize}
%\item[$(1)$] As a $\calC$-enriched category, $\calD$ is tensored and cotensored over $\calC$.
%\item[$(2)$] For every cofibration $i: C \rightarrow C'$ in $\calC$ and every cofibration
%$j: D \rightarrow D'$ in $\calD$, the induced map
%$$ k: (D' \otimes C) \coprod_{ D \otimes C} (D \otimes C') \rightarrow D' \otimes C'$$
%is a cofibration. Moreover, if $i$ or $j$ is a weak equivalence, then $k$ is a weak equivalence.
%\end{itemize}
%\end{definition}

%\begin{example}
%If $\calC$ is a monoidal model category, then it may be regarded as a model category enriched over itself (via the same model structure).
%\end{example}

%\begin{remark}
%Any category $\calC$ may be regarded as enriched over the category $\Set$ of sets. However,
%it is not the case that any model category $\calC$ may be regarded as a $\Set$-enriched model category; this is a reflection of the fact that the theory of model categories is an approach to higher category theory, in which morphisms should really be thought of as constituting spaces. There is an analogous assertion for model categories but it is more complicated: any suitably nice model category is equivalent to a model category enriched over the category $\sSet$ of simplicial sets, in an essentially unique way: see \cite{rezk} for a precise statement and a proof.
%\end{remark}

%\subsection{Enriched Quillen Adjunctions}

%In this section, we assume that $\bfS$ is a monoidal model category in which every object is cofibrant.

%Let $\calC$ and $\calD$ be model categories enriched over $\bfS$, and suppose given a Quillen adjunction $$\Adjoint{F}{\calC}{\calD}{G}$$
%between the underlying model categories. We wish to study the situation where $G$ (but not $F$) has the structure of $\bfS$-enriched functor.
%This situation will arise when we study marked model structures in \S \ref{chap4}.

%If $G$ is a $\bfS$-enriched functor, then for each $X \in \calC$, $Y \in \calD$, we obtain a map 
%in $\bfS$
%$$ \bHom_{\calD}(FX,Y) \rightarrow \bHom_{\calC}(GFX, GY).$$
%Thus, for every $S \in \bfS$, we are given a map
%$$ \Hom_{\calD}(F(X) \otimes S,Y) \rightarrow \Hom_{\calC}(G(F(X) \otimes S), G(Y)) = \Hom_{\calD}(F((G \circ F)(X) \otimes S,Y).$$
%As this map is functorial in $Y$, it is induced by composition with a map
%$$b: F((G \circ F)(X) \otimes S) \rightarrow X \otimes S.$$
%Let $\beta_{X,S}$ denote the composition
%$$ F(X \otimes S) \rightarrow F((G\circ F)X \otimes S) \stackrel{b}{\rightarrow} X \otimes S.$$
%The collection of maps $\beta_{X,S}$ is simply another means of encoding the data of $G$
%as a $\bfS$-enriched functor. 

%\begin{proposition}\label{weakcompatequiv}
%Let $\calC$ and $\calD$ be $\bfS$-enriched model categories. Let
%$\Adjoint{F}{\calC}{\calD}{G}$ be a Quillen adjunction between the underlying model categories.
%Suppose that $\beta_{X,S}$ is a weak equivalence for every object $X$ of $\calC$ and every
%$S \in \bfS$, and that every object of $\calC$ is cofibrant. The following are equivalent:
%\begin{itemize}
%\item[$(1)$] The adjunction $(F,G)$ is a Quillen equivalence.
%\item[$(2)$] The restriction of $G$ gives a weak equivalence of $\bfS$-enriched categories
%$\calD^{\degree} \rightarrow \calC^{\degree}$ (see \S \ref{compp4}).
%\end{itemize}
%\end{proposition}

%\begin{remark}
%Strictly speaking, in \S \ref{compp4} we only define weak equivalences between {\em small} $\bfS$-enriched categories; however, the definition extends to large $\bfS$-enriched categories in an obvious way.
%\end{remark}

%\begin{proof}
%Since $G$ preserves fibrant objects, and every object of $\calC$ is cofibrant, it is clear that
%$G$ carries $\calD^{\degree}$ into $\calC^{\degree}$. It is obvious that $(2)$ implies $(1)$.
%Suppose that $(1)$ holds. Then
%$G$ is essentially surjective, since the right derived functor $RG$ is essentially surjective on homotopy categories. It suffices to show that $G$ is fully faithful: in other words, that for every pair of fibrant-cofibrant objects $X,Y \in \calD$, the induced map
%$$ i: \bHom_{\calD}(X,Y) \rightarrow \bHom_{\calC}(G(X),G(Y))$$ 
%is a weak equivalence between fibrant objects of $\bfS$.

%Since the left derived functor $LF$ is essentially surjective, there exists an object $X' \in \calC$
%and a weak equivalence $FX' \rightarrow X$. We may regard $X$ as a ``fibrant replacement'' for $FX'$ in $\calD$; it follows that the adjoint map $X' \rightarrow GX$ may be identified with the
%adjunction $X' \rightarrow (RG \circ LF) X'$, and is therefore a weak equivalence by $(1)$. Thus we have a diagram
%$$ \xymatrix{ \bHom_{\calD}(X,Y) \ar[d] \ar[r]^{i} & \bHom_{\calC}(G(X),G(Y)) \ar[d] \\
%\bHom_{\calD}(F(X'), Y) \ar[r]^{i'} & \bHom_{\calC}(X',G(Y)) }$$
%in which the vertical arrows are homotopy equivalences; thus, to show that $i$ is a weak equivalence, it suffices to show that $i'$ is a weak equivalence. For this, it suffices to show that $i'$ induces a bijection from $[ S, \bHom_{\calD}(F(X'), Y) ]$ to $[S, \bHom_{\calC}(X',G(Y))]$, for every $S \in \bfS$; here $[S,K]$ denotes the set of homotopy classes of maps from $S$ into $K$ in
%the homotopy category $\h{{\bf S}}$. But we may rewrite this map of sets as
%$$i'_S:  \bHom_{\h{\calD}}( F(X') \otimes S, Y) \rightarrow \bHom_{\h{\calC}}( X' \otimes S, G(Y) )
%= \bHom_{\h{\calD}}( F( X' \otimes S), Y),$$
%and it is given by composition with $\beta_{X',S}$. (Here $\h{\calC}$ and $\h{\calD}$ denote the
%homotopy categories of $\calC$ and $\calD$ as {\em model categories}; these are equivalent
%to the corresponding homotopy categories of $\calC^{\degree}$ and $\calD^{\degree}$ as $\bfS$-enriched categories). Since $\beta_{X',S}$ is an isomorphism in the homotopy category $h \calD$, the map $i'_S$ is bijective and $(2)$ holds, as desired.
%\end{proof}

%\begin{corollary}\label{urchug}
%Let $$\Adjoint{F}{\calC}{\calD}{G}$$
%be a Quillen adjunction between simplicial model categories (see \S \ref{simpmod}), and suppose that $G$ is a simplicial functor. Then $G$ induces an equivalence of $\infty$-categories
%$\Nerve( \calD^{\degree}) \rightarrow \Nerve( \calC^{\degree})$.
%\end{corollary}

%\subsection{Perfect Model Categories}
\subsection{Combinatorial Model Categories}\label{combimod}

In this section, we give an overview of Jeff Smith's theory of {\it combinatorial model categories}. Our main goal is to prove Proposition \ref{goot}, which allows us to construct model structures on a  
category $\calC$ by specifying the class of weak equivalences, together with a small amount of additional data.

\begin{definition}[Smith]\index{gen}{combinatorial model category}\index{gen}{model category!combinatorial}\label{sittu}
Let $\bfA$ be model category. We say that $\bfA$ is {\it combinatorial} if the following conditions are satisfied:
\begin{itemize}
\item[$(1)$] The category $\bfA$ is presentable.
\item[$(2)$] There exists a set $I$ of {\it generating cofibrations}, such that the collection of all cofibrations in $\bfA$ is the smallest weakly saturated class of morphisms containing $I$ (see Definition \ref{saturated}). 
\item[$(3)$] There exists a set $J$ of {\it generating trivial cofibrations}, such that the collection of all trivial cofibrations in $\bfA$ is the smallest weakly saturated class of morphisms containing $J$.
\end{itemize}
\end{definition}

If $\calC$ is a combinatorial model category, then the model structure on $\calC$ is uniquely determined by the generating cofibrations and generating trivial cofibrations. However, in practice these generators might be difficult to find. Our goal in this section is to reformulate Definition \ref{sittu} in a manner which puts more emphasis on the category of weak equivalences in $\bfA$. 

In practice, it is often easier to describe the class of {\em all} weak equivalences than it is to describe a class of generating trivial cofibrations. 

\begin{definition}\index{gen}{$\kappa$-accessible subcategory}\index{gen}{accessible!subcategory}\label{kappar}
Let $\calC$ be a presentable category and $\kappa$ a regular cardinal. We will say that a full subcategory $\calC_0 \subseteq \calC$ is an {\it $\kappa$-accessible subcategory} of
$\calC$ if the following conditions are satisfied:
\begin{itemize}
\item[$(1)$] The full subcategory $\calC_0 \subseteq \calC$ is stable under $\kappa$-filtered colimits.
\item[$(2)$] There exists a (small) set of objects of $\calC_0$ which generates $\calC_0$ under $\kappa$-filtered colimits. 
\end{itemize}
We will say that $\calC_0 \subseteq \calC$ is an {\it accessible subcategory} if $\calC_0$ is a $\kappa$-accessible subcategory of $\calC$, for some regular cardinal $\kappa$.
\end{definition}

Condition $(2)$ of Definition \ref{kappar} admits the following reformulation:

\begin{proposition}\label{reefa} Let $\kappa$ be a regular cardinal, let $\calC$ be a presentable category, and let $\calC_0 \subseteq \calC$ be a full subcategory which is stable under $\kappa$-filtered colimits. Then $\calC_0$ satisfies condition $(2)$ of Definition \ref{kappar} if and only if the following condition is satisfied, for all sufficiently large regular cardinals $\tau \gg \kappa$
\begin{itemize}
\item[$(2'_{\tau})$] Let $A$ be a $\tau$-filtered partially ordered set and
$\{ X_{\alpha} \}_{\alpha \in A}$ a diagram of $\tau$-compact objects of $\calC$ indexed by $A$.
For every $\kappa$-filtered subset $B \subseteq A$, we let
$X_B$ denote $(${}$\kappa$-filtered$)$ colimit of the diagram $\{ X_{\alpha} \}_{\alpha \in B}$.
Suppose that $X_{A}$ belongs to $\calC_0$. Then for every $\tau$-small subset $C \subseteq A$,
there exists a $\tau$-small, $\kappa$-filtered subset $B \subseteq A$ which contains $C$, such that
$X_B$ belongs to $\calC_0$. 
\end{itemize}
\end{proposition}

First, we need the following preliminary result:

\begin{lemma}\label{constunt}
Let $\tau \gg \kappa$ be regular cardinals such that $\tau > \kappa$, let $\calD$ be presentable $\infty$-category, let $\{ C_{a} \}_{a \in A}$ and $\{ D_{b} \}_{b \in B}$ be families of $\tau$-compact objects in $\calD$ indexed by $\tau$-filtered partially ordered sets $A$ and $B$, such that
$$ \colim_{a \in A} C_{a} \simeq \colim_{b \in B} D_{b}.$$
Then, for every pair of $\tau$-small subsets $A_0 \subseteq A$, $B_0 \subseteq B$, there
exist $\tau$-small, $\kappa$-filtered subsets $A' \subseteq A$, $B' \subseteq B$ such that
$A_0 \subseteq A'$, $B_0 \subseteq B'$, and $\colim_{a \in A'} C_a \simeq \colim_{b \in B'} D_b$.
\end{lemma}

\begin{proof}
Let $\calA$ be the partially ordered set of all $\tau$-small, $\kappa$-filtered subsets of
$A$ which contain $A_0$, let $\calB$ be the partially ordered set of all $\tau$-small, $\kappa$-filtered subsets of $B$ which contain $B_0$, let $X \in \calD$ be the common colimit
$ \colim_{a \in A} C_{a} \simeq \colim_{b \in B} D_{b}$, and let $\calC$ be the full subcategory
of $\calD_{/X}$ spanned by those morphisms $Y \rightarrow X$ where $Y$ is a $\tau$-compact object of $\calD$. Let $f: \calA \rightarrow \calC$ and $g: \calB \rightarrow \calC$ be the functors described by the formulas
$$f( A' ) = ( \colim_{a \in A'} C_{a} \rightarrow \colim_{a \in A} C_a )$$
$$g( B' ) = ( \colim_{b \in B'} D_{b} \rightarrow \colim_{b \in B} D_b ).$$
The desired result now follows by applying Lemma \ref{remuswolf} to the associated diagram
$$ \Nerve(\calA) \rightarrow \Nerve(\calC) \leftarrow \Nerve(\calB).$$
\end{proof}

\begin{proof}[Proof of Proposition \ref{reefa}]
First suppose that $(2'_{\tau})$ is satisfied for all sufficiently large $\tau \gg \kappa$. Choose
$\tau \gg \kappa$ large enough that $\calC$ is generated under colimits by its full subcategory $\calC^{\tau}$ of $\tau$-compact objects, and such that $(2'_{\tau})$ is satisfied. Let 
$\calD = \calC^{\tau} \cap \calC_0$, so that $\calD$ is essentially small. We will show that
$\calD$ generates $\calC_0$ under $\tau$-filtered colimits.
By assumption, every object $X \in \calC$ can be obtained as a $\tau$-filtered colimit of $\tau$-compact objects $\{ X_{\alpha} \}_{\alpha \in A}$.
Let $A'$ denote the collection of all
$\tau$-small, $\kappa$-filtered subsets $B \subseteq A$ such that $X_{B} \in \calC_0$. 
We regard $A'$ as partially-ordered via inclusions. Invoking condition $(2'_{\tau})$, we deduce that $X_{A}$ is the colimit of the $\tau$-filtered collection of objects $\{ X_{A'} \}_{A' \in B}$. We now observe that each $X_{A'}$ belongs to $\calD$.

Now suppose that condition $(2)$ is satisfied, so that $\calC_0$ is generated under $\kappa$-filtered colimits by a small subcategory $\calD \subseteq \calC_0$. Choose
$\tau \gg \kappa$ large enough that every object of $\calD$ is $\tau$-compact. Enlarging $\tau$ if necessary, we may suppose that $\tau > \kappa$. We claim that $(2'_{\tau})$ is satisfied.
To prove this, we consider any system of morphisms $\{ X \}_{\alpha \in A}$ satisfying the hypotheses of $(5'_{\tau})$. In particular, $X_A$ belongs to
$\calC_0$, so that $X_A$ may be obtained in some {\em other} way as a $\kappa$-filtered colimit of a system $\{ Y_{\beta}  \}_{\beta \in B}$, where each of the objects $Y_{\beta}$ belongs to $\calD$ and is therefore $\tau$-compact. Let $C'$ denote the family of all $\tau$-small, $\kappa$-filtered subsets $B_0 \subseteq B$. Replacing $B$ by $B'$ and the family
$\{ Y_{\beta} \}_{\beta \in B}$ by
$\{ Y_{B_0} \}_{ B_0 \in B'}$, we may assume that $B$ is $\tau$-filtered.

Let $A_0 \subseteq A$ be a $\tau$-small subset. Applying Lemma \ref{constunt} to the diagram category $\calC$, we deduce that $A_0 \subseteq A'$, where $A'$ is a $\tau$-small, $\kappa$-filtered subset of $A$, and there is an isomorphism
$X_{A'} \simeq Y_{B'}$; here $B'$ is a $\kappa$-filtered subset of $B$, so that $Y_{B'} \in \calC_0$ in virtue of our assumption that $\calC_0$ is stable under $\kappa$-filtered colimits.
\end{proof}

\begin{corollary}\label{sundert}
Let $f: \calC \rightarrow \calD$ be a functor between presentable categories which preserves $\kappa$-filtered colimits, and let $\calD_0 \subseteq \calD$ be a $\kappa$-accessible subcategory. Then $f^{-1} \calD_0 \subseteq \calC$ is a $\kappa$-accessible subcategory.
\end{corollary}

\begin{corollary}[Smith]\label{smitty}
Let $\bfA$ be a combinatorial model category, let $\bfA^{[1]}$ be the category of morphisms in $\bfA$, let $W \subseteq \bfA^{[1]}$ be the full subcategory spanned by the weak equivalences, and let
$F \subseteq \bfA^{[1]}$ be the full subcategory spanned by the fibrations. Then
$F$, $W$, and $F \cap W$ are accessible subcategories of $\bfA^{[1]}$. 
\end{corollary}

\begin{proof}
For every morphism $i: A \rightarrow B$, let $F_i: \bfA^{[1]} \rightarrow \Set^{[1]}$ be the functor which carries a morphism $f: X \rightarrow Y$ to the induced map of sets
$$ \Hom_{\bfA}(B, X) \rightarrow \Hom_{\bfA}(B,Y) \times_{ \Hom_{\bfA}(A,Y) } \Hom_{\bfA}(A,X).$$
We observe that if $A$ and $B$ are $\kappa$-compact objects of $\bfA$, then $F_i$ preserves $\kappa$-filtered colimits.

Let $\calC_0$ be the full subcategory of $\Set^{[1]}$ spanned by the collection of {\em surjective} maps between sets. It is easy to see that $\calC_0$ is an accessible category of $\Set^{[1]}$. It follows that the full subcategories $R(i) = F_i^{-1} \calC_0 \subseteq \bfA^{[1]}$ are accessible subcategories of $\bfA^{[1]}$ (Corollary \ref{sundert}). 

Let $I$ be a set of generating cofibrations for $\bfA$, and $J$ a set of generating trivial cofibrations. Then Proposition \ref{boundint} implies that the subcategories
$$ F = \bigcap_{j \in J} R(j)$$
$$ W \cap F = \bigcap_{i \in I} R(i)$$
are accessible subcategories of $\bfA^{[1]}$. 

Applying Proposition \ref{quillobj}, we deduce that there exists a pair of functors
$T', T'': \bfA^{[1]} \rightarrow \bfA^{[1]}$, which carry an arbitrary morphism $f: X \rightarrow Z$ to a factorization
$$ X \stackrel{ T'(f) }{\rightarrow} Y \stackrel{T''(f)}{\rightarrow} Z$$
where $T'(f)$ is a trivial cofibration, and $T''(f)$ is a fibration. Moreover, the functor $T''$ can be chosen to commute with $\kappa$-filtered colimits, for a sufficiently large regular cardinal $\kappa$. We now observe that $W$ is the inverse image of $F \cap W$ under the functor
$T'': \bfA^{[1]} \rightarrow \bfA^{[1]}$, and is therefore an accessible subcategory of
$\bfA^{[1]}$ by Corollary \ref{sundert}.
\end{proof}

Our next goal is to prove a converse to Corollary \ref{smitty}, which will allow us to construct examples of combinatorial model categories. First, we need the following preliminary result.

\begin{lemma}\label{seeva}
Let $\bfA$ be a presentable category. Suppose $W$ and $C$ are collections of morphisms
of $\bfA$ with the following properties:
\begin{itemize}
\item[$(1)$] The collection $C$ is a weakly saturated class of morphisms of $\bfA$, and there exists a $($small$)$ subset $C_0 \subseteq C$ which generates $C$ as a weakly saturated class of morphisms.
\item[$(2)$] The intersection $C \cap W$ is a weakly saturated class of morphisms of $\bfA$.
\item[$(3)$] The full subcategory $W \subseteq \bfA^{[1]}$ is an accessible subcategory
of $\bfA^{[1]}$.
\item[$(4)$] The class $W$ has the two-out-of-three property.
\end{itemize}
Then $C \cap W$ is generated, as a weakly saturated class of morphisms, by a $($small$)$ subset
$S \subseteq C \cap W$.
\end{lemma}

\begin{proof}
Let $\kappa$ be a regular cardinal such that $W$ is $\kappa$-accessible. Choose a regular cardinal $\tau \gg \kappa$ such that $W$ satisfies condition $(2'_{\tau})$ of Proposition \ref{reefa}. Enlarging $\tau$ if necessary, we may assume that $\tau > \kappa$ (so that $\tau$ is uncountable), that $\calC$ is $\tau$-accessible, and that the source and target of every morphism in $C_0$ is 
$\tau$-compact. Enlarging $C_0$ if necessary, we may suppose that $C_0$ consists of {\em all} morphisms $f: X \rightarrow Y$ in $C$ such that $X$ and $Y$ are $\tau$-compact. Let $S = C_0 \cap W$. We will show that $S$ generates $C \cap W$ as a weakly saturated class of morphisms.

Let $\overline{S}$ be the weakly saturated class of morphisms generated by $S$, and let
$f: X \rightarrow Y$ be a morphism which belongs to $C \cap W$. We wish to show that $f \in \overline{S}$. Corollary \ref{unitape} implies that there exists a $\tau$-good $C_0$-tree $\{ Y_{\alpha} \}_{\alpha \in A}$ with root $X$, such that
$Y$ is isomorphic to $Y_{A}$ as objects of $\calC_{X/}$. Let us say that a subset $B \subseteq A$ is {\it good} if it is closed downwards and the canonical map $i: X \rightarrow Y_{B}$ belongs to $W$ (we note that $i$ automatically belongs to $C$, in virtue of Lemma \ref{uper}). 

We now make the following observations:
\begin{itemize}
\item[$(i)$] Given an increasing transfinite sequence of good subsets $\{ A_{\gamma} \}_{\gamma < \beta}$,
the union $\bigcup A_{\gamma}$ is good. This follows from the assumption that $C \cap W$
is weakly saturated.
\item[$(ii)$] Let $B \subseteq A$ be good, and let $B_0 \subseteq B$ be $\tau$-small. Then there exists a $\tau$-small subset $B' \subseteq B$ containing $B_0$. This follows from our assumption
that $W$ satisfies condition $(2'_{\tau})$ of Proposition \ref{reefa}. 
\item[$(iii)$] Suppose that $B,B' \subseteq A$ are such that $B$, $B'$, and $B \cap B'$ are good.
Then $B \cup B'$ is good. To prove this, we consider the pushout diagram
$$ \xymatrix{ Y_{B \cap B'} \ar[r] \ar[d] & Y_{B} \ar[d] \\
Y_{B'} \ar[r] & Y_{B \cup B'}. }$$
Every morphism in this diagram belongs to $C$ (Lemma \ref{uper}), and the upper horizontal map belongs to $W$ in virtue of assumption $(4)$. Since $C \cap W$ is stable under pushouts, we conclude that the lower vertical map belongs to $W$. Assumption $(4)$ now implies that the composite map $X \rightarrow Y_{B'} \rightarrow Y_{B \cup B'}$ belongs to $W$, as desired.
\end{itemize}

The next step is to prove the following claim:
\begin{itemize}
\item[$(\ast)$] Let $A'$ be a good subset of $A$, and let $B_0 \subseteq A$ be $\tau$-small. Then
there exists a $\tau$-small subset $B \subseteq A$ such that $B_0 \subseteq B$, $B$ is good, and $B \cap A'$ is good.
\end{itemize}

To prove $(\ast)$, we begin by setting $B'_0 = A' \cap B_0$. We now define sequences of $\tau$-small subsets
$$ B_0 \subseteq B_1 \subseteq B_2 \subseteq \ldots $$
$$ B'_0 \subseteq B'_1 \subseteq B'_2 \subseteq \ldots$$
as follows. Suppose that $B_i$ and $B'_{i}$ have been defined. Applying $(ii)$, we choose
$B_{i+1}$ to be any $\tau$-small good subset of $A$ which contains $B_i \cup B'_{i}$.
Applying $(ii)$ again, we select $B'_{i+1}$ to be any $\tau$-small good subset of $A'$ which contains $A' \cap B_{i+1}$. Let $B = \bigcup B_{i}$. It follows from $(i)$ that $B$ and $A' \cap B = \bigcup_{i} B'_{i}$ are both good.

We now choose a transfinite sequence of good subsets $\{ A(\gamma) \subseteq A \}_{\gamma < \beta}$. Suppose that $A(\gamma')$ has been defined for $\gamma' < \gamma$, and let
$A'(\gamma) = \bigcup_{\gamma' < \gamma} A(\gamma')$. It follows from $(i)$ that
$A'(\gamma)$ is good. If $A'(\gamma) = A$, we set $\beta = \gamma$ and conclude the construction. Otherwise, choose a minimal element 
$a \in A - A'(\gamma)$. Applying $(\ast)$, we deduce that there exists a $\tau$-small good subset
$B(\gamma) \subseteq A$ containing $a$, such that $A'(\gamma) \cap B(\gamma)$ is good. Let $A(\gamma) = A'(\gamma) \cup B(\gamma)$. It follows from $(iii)$ that $A(\gamma)$ is good. 

We observe that $\{ Y_{A(\gamma)} \}_{\gamma < \beta}$ is a transfinite sequence of objects of
$\calC_{X/}$ having colimit $Y$. To prove that $f: X \rightarrow Y$ belongs to $\overline{S}$, it will suffice to show that for each $\gamma < \beta$, the map $g: Y_{A'(\gamma)} \rightarrow Y_{A(\gamma)}$ belongs to $\overline{S}$. Remark \ref{relci} implies the existence of a $C_0$-tree
$\{ Z_{\alpha} \}_{\alpha \in A(\gamma) - A'(\gamma)}$ with root $Y_{A'(\gamma)}$ and colimit $Y_{A(\gamma)}$. Since $A(\gamma) - A'(\gamma)$ is $\tau$-small, Lemma \ref{tiura} implies the existence of a pushout diagram 
$$ \xymatrix{ M \ar[r] \ar[d] & N \ar[d] \\
Y_{A'(\gamma)} \ar[r] & Y_{A(\gamma)}}$$
where $g \in C_0$. 

Since $\calC$ is $\tau$-accessible, we can write $Y_{A'(\gamma)}$ as the colimit of a family of $\tau$-compact objects $\{ Z_{\lambda} \}_{\lambda \in P}$, indexed by a $\tau$-filtered partially ordered set $P$. Since $M$ is $\tau$-compact, we can assume (reindexing the colimit if necessary) that we have a compatible family of maps $\{ M \rightarrow Z_{\lambda} \}$. For each $\lambda$, let $g_{\lambda}: Z_{\lambda} \rightarrow Z_{\lambda} \coprod_{M} N$ be the induced map. Then $g$ is the filtered colimit of the family $\{ g_{\lambda} \}_{\lambda \in P}$. Since $W$ satisfies condition
$(2'_{\tau})$ of Proposition \ref{reefa}, we conclude that there exists a $\tau$-small, $\kappa$-filtered subset $P_0 \subseteq P$, such that $g' = \colim_{\lambda \in P_0} g_{\lambda}$ belongs to $W$. We now observe that $g' \in S$, and that $g$ is a pushout of $g'$, so that $g \in \overline{S}$ as desired.
\end{proof}

\begin{proposition}\label{bigmaker}
Let $\bfA$ be a presentable category, and let $W$ and $C$ be classes of morphisms
in $\bfA$ with the following properties:
\begin{itemize}
\item[$(1)$] The collection $C$ is a weakly saturated class of morphisms of $\bfA$, and there exists a $($small$)$ subset $C_0 \subseteq C$ which generates $C$ as a weakly saturated class of morphisms.
\item[$(2)$] The intersection $C \cap W$ is a weakly saturated class of morphisms of $\bfA$.
\item[$(3)$] The full subcategory $W \subseteq \bfA^{[1]}$ is an accessible subcategory
of $\bfA^{[1]}$.
\item[$(4)$] The class $W$ has the two-out-of-three property.
\item[$(5)$] If $f$ is a morphism in $\bfA$ which has the right lifting property with respect to each element of $C$, then $f \in W$.
\end{itemize}
Then $\bfA$ admits a combinatorial model structure, which may be described as follows:
\begin{itemize}
\item[$(C)$] The cofibrations in $\bfA$ are the elements of $C$.
\item[$(W)$] The weak equivalences in $\bfA$ are the elements of $W$.
\item[$(F)$] A morphism in $\bfA$ is a fibration if it has the right lifting property with respect to every morphism in $C \cap W$.
\end{itemize}
\end{proposition}

\begin{proof}
The category $\bfA$ has all (small) limits and colimits, since it is presentable. The two-out-three property for $W$ is among our assumptions, and the stability of $W$ under retracts follows from the accessibility of $W \subseteq \bfA^{[1]}$ (Corollary \ref{swwe}). The class of cofibrations is stable under retracts by $(1)$, and the class of fibrations is stable under retracts by definition.
The classes of fibrations and cofibrations are stable under retracts by definition.

We next establish the factorization axioms. By the small object argument, any morphism
$X \rightarrow Z$ admits a factorization
$$X \stackrel{f}{\rightarrow} Y \stackrel{g}{\rightarrow} Z$$
where $f \in C$ and $g$ has the right lifting property with respect to every morphism in $C$.
In particular, $g$ has the right lifting property with respect to every morphism in $C \cap W$, so that $g$ is a fibration; assumption $(5)$ then implies that $g$ is a trivial fibration.
Similarly, using Lemma \ref{seeva} we may choose a factorization as above where $f \in C \cap W$ and $g$ has the right lifting property with respect to $C \cap W$; $g$ is then a fibration by definition.

To complete the proof, it suffices to show that cofibrations have the left lifting property with respect to trivial fibrations, and trivial cofibrations have the left lifting property with respect to fibrations. The second of these statements is clear (it is the definition of a fibration). For the first statement, let us consider an arbitrary trivial fibration $p: X \rightarrow Z$. By the small object argument,
there exists a factorization of $p$
$$ X \stackrel{q}{\rightarrow} Y \stackrel{r}{\rightarrow} Z$$
where $q$ is a cofibration, and $r$ has the right lifting property with respect to all cofibrations.
Then $r$ is a weak equivalence by $(3)$, so that $q$ is a weak equivalence by the two-out-of-three property. Considering the diagram
$$ \xymatrix{ X \ar@{=}[r] \ar[d]^{q} & X \ar[d]^{p} \\
Y \ar[r]^{r} \ar@{-->}[ur] & Z,}$$
we deduce the existence of the dotted arrow from the fact that $p$ is a fibration and $q$ is a trivial
cofibration. It follows that $p$ is a retract of $r$, and therefore $p$ also has the right lifting property with respect to all cofibrations. This completes the proof that $\bfA$ is a model category. The assertion that $\bfA$ is combinatorial follows immediately from $(1)$ and from Lemma \ref{seeva}.
\end{proof}

\begin{corollary}\label{uryt}
Let $\bfA$ be a presentable category equipped with a model structure. Suppose that there exists
a $($small$)$ set which generates the collection of cofibrations in $\bfA$ $($as a weakly saturated class of morphisms$)$. Then the following are equivalent:
\begin{itemize}
\item[$(1)$] The model category $\bfA$ is combinatorial; in other words, there exists a $($small$)$ set which generates the collection of trivial cofibrations in $\bfA$ $($as a weakly saturated class of morphisms$)$.
\item[$(2)$] The collection of weak equivalences in $\bfA$ determines an accessible subcategory
of $\bfA^{[1]}$. 
\end{itemize}
\end{corollary}

\begin{proof}
The implication $(1) \Rightarrow (2)$ follows from Corollary \ref{smitty}, and the reverse implication follows from Proposition \ref{bigmaker}.
\end{proof}

Our next goal is to prove a weaker version of Proposition \ref{bigmaker} which is somewhat easier to apply in practice.

\begin{definition}\label{perfequiv}\index{gen}{perfect!class of morphisms}
Let $\bfA$ be a presentable category. A class $W$ of morphisms in $\calC$ is {\it perfect}
if it satisfies the following conditions:

\begin{itemize}
\item[$(1)$] Every isomorphism belongs to $W$.
\item[$(2)$] Given a pair of composable morphisms $X \stackrel{f}{\rightarrow} Y \stackrel{g}{\rightarrow} Z$, if any two of the morphisms $f$, $g$, and $g \circ f$ belong to $W$, then so does the third.
\item[$(3)$] The class $W$ is stable under filtered colimits. More precisely, suppose given a family of morphisms $\{ f_{\alpha}: X_{\alpha} \rightarrow Y_{\alpha} \}$ which is indexed by a filtered partially ordered set. Let $X$ denote a colimit of $\{ X_{\alpha} \}$ and $Y$ a colimit of
$\{ Y_{\alpha} \}$, and $f: X \rightarrow Y$ the induced map. If each $f_{\alpha}$ belongs to $W$, then so does $f$. 
\item[$(4)$] There exists a (small) subset $W_0 \subseteq W$ such that every morphism belonging to $W$ can be obtained as a filtered colimit of morphisms belonging to $W_0$.
\end{itemize}
\end{definition}

\begin{example}
If $\calC$ is a presentable category, then the class $W$ consisting of all isomorphisms in $\calC$
is perfect.
\end{example}

The following is an immediate consequence of Corollary \ref{sundert}:

\begin{corollary}\label{perfpull}
Let $F: \calC \rightarrow \calC'$ be a functor between presentable categories which preserves filtered colimits, and let $W_{\calC'}$ be a perfect class of morphisms in $\calC'$. Then
$W_{\calC} = F^{-1} W_{\calC'}$ is a perfect class of morphisms in $\calC$.
\end{corollary}

\begin{proposition}\label{goot}
Let $\bfA$ be a presentable category. Suppose given a class $W$ of morphisms of $\calC$, which we will call {\it weak equivalences}, and a $($small$)$ {\em set} $C_0$ of morphisms of $\calC$, which we will call {\it generating cofibrations}. Suppose furthermore that the following assumptions are satisfied:

\begin{itemize}
\item[$(1)$] The class $W$ of weak equivalences is perfect $($Definition \ref{perfequiv}$)$.
\item[$(2)$] For any diagram
$$ \xymatrix{ X \ar[r]^{f} \ar[d] & Y \ar[d] \\
	X' \ar[r] \ar[d]^{g} & Y' \ar[d]^{g'} \\
	X'' \ar[r] & Y'' } $$
in which both squares are coCartesian, $f$ belongs to $C_0$, and $g$ belongs to $W$, 
the map $g'$ also belongs to $W$.
\item[$(3)$] If $g: X \rightarrow Y$ is a morphism in $\bfA$ which has the right lifting property with respect to every morphism in $C_0$, then $g$ belongs to $W$. 
\end{itemize}

Then there exists a left proper, combinatorial model structure on $\calC$ which may be described as follows:

\begin{itemize}
\item[$(C)$] A morphism $f: X \rightarrow Y$ in $\bfA$ is a {\it cofibration} if it belongs to the weakly saturated class of morphisms generated by $C_0$.
\item[$(W)$] A morphism $f: X \rightarrow Y$ in $\calC$ is a {\it weak equivalence} if it belongs to $W$.
\item[$(F)$] A morphism $f: X \rightarrow Y$ in $\calC$ is a {\it fibration} if it has the right lifting property with respect to every map which is both a cofibration and a weak equivalence.
\end{itemize}\index{gen}{perfect!model category}\index{gen}{model category!perfect}
\end{proposition}

\begin{proof}
We first show that the class of weak equivalences is stable under pushouts by cofibrations.
Let $P$ denote the collection of all morphisms $f$ in $\bfA$ with the following property:
for coCartesian diagram
$$ \xymatrix{ X \ar[r]^{f} \ar[d] & Y \ar[d] \\
	X' \ar[r] \ar[d]^{g} & Y' \ar[d]^{g'} \\
	X'' \ar[r] & Y'' } $$
where $g$ belongs to $W$, the map $g'$ also belongs to $W$. By assumption, 
$C_0 \subseteq P$. It is easy to see that $P$ is weakly saturated (using the stability of $W$ under filtered colimits), so that every cofibration belongs to $P$. 

It remains only to show that $\bfA$ is a model category. In view of Proposition \ref{bigmaker}, it will suffice to show that $C \cap W$ is a weakly saturated class of morphisms. It is clear that $C \cap W$ is stable under retracts. It will therefore suffice to verify the stability of $C \cap W$ under pushouts and transfinite composition. The case of transfinite composition is easy: $C$ is stable under transfinite composition because $C$ is weakly saturated, and $W$ is stable under transfinite composition because
it is stable under finite composition and filtered colimits.

It remains to show that $C \cap W$ is stable under pushouts. Suppose given a coCartesian diagram
$$ \xymatrix{ X \ar[d]^{f} \ar[r] & X'' \ar[d]^{f''} \\
Y \ar[r] & Y'' }$$
in which $f$ belongs to $C \cap W$; we wish to show that $f''$ also belongs to $C \cap W$. Since
$C$ is weakly saturated, it will suffice to show that $f''$ belongs to $W$. Using the small object argument, we can factor the top horizontal map to produce a coCartesian rectangle
$$ \xymatrix{ X \ar[d]^{f} \ar[r]^{g} & X' \ar[d]^{f'} \ar[r]^{h} & X'' \ar[d]^{f''} \\
Y \ar[r] & Y' \ar[r]^{h'} & Y'' }$$
in which $g$ is a cofibration and $h$ has the right lifting property with respect to all the morphisms in $C_0$. Since $W$ is stable under the formation of pushouts by cofibrations, we deduce that $f'$
belongs to $W$. Moreover, by assumption $(3)$, $h$ belongs to $W$. Since $h'$ is a pushout
of $h$ by the cofibration $f'$, we deduce that $h'$ belongs to $W$ as well. Applying the two-out-of-three property (twice), we deduce that $f''$ belongs to $W$. 
\end{proof}

%It will be convenient to consider a reformulation of condition $(5)$:
%\begin{proposition}\label{reformm}
%Let $\calC$ be a presentable category, and let $W$ be a class of morphisms of $\calC$
%satisfying condition $(4)$ of Definition \ref{perfequiv}. Then $W$ satisfies condition $(5)$ of Definition \ref{perfequiv} if and only if, for all sufficiently large regular cardinals $\kappa$, the following condition is satisfied:

%\begin{itemize}
%\item[$(5'_{\kappa})$] Let $A$ be a $\kappa$-filtered partially ordered set, 
%$\{ f_{\alpha}: X_{\alpha} \rightarrow Y_{\alpha} \}_{\alpha \in A}$ a family of morphisms
%in $\calC$ indexed by $A$. Suppose that each $X_{\alpha}$ and each $Y_{\alpha}$ is $\kappa$-compact. For each filtered subset $B \subseteq A$, we let
%$X_B$ and $Y_B$ denote (filtered) colimits of the systems $\{ X_{\alpha} \}_{\alpha \in B}$
%and $\{ Y_{\alpha} \}_{\alpha \in B}$, and $f_B: X_B \rightarrow Y_B$ the induced map.
%Suppose that $f_A$ belongs to $W$. Then for any $\kappa$-small subset $C \subseteq A$,
%there exists a filtered $\kappa$-small subset $B \subseteq A$ which contains $C$, such that
%$f_B$ belongs to $W$. 
%\end{itemize}
%\end{proposition}

%\begin{proof}
%First suppose that $(5'_{\kappa})$ is satisfied for all sufficiently large $\kappa$. Choose
%$\kappa$ large enough that $\calC$ is generated under colimits by its full subcategory $\calC^{\kappa}$ of $\kappa$-compact objects, and such that $(5'_{\kappa})$ is satisfied. Let $W_0 \subseteq W$ be a
%set of representatives for all morphisms $f: X \rightarrow Y$ which belong to $W$, such that $X$ and $Y$ are $\kappa$-compact. Since $\calC^{\kappa}$ is essentially small, $W_0$ is small. 
%We note that {\em any} morphism $f: X \rightarrow Y$ may be obtained as a $\kappa$-filtered
%colimit of morphisms $\{ f_{\alpha}: X_{\alpha} \rightarrow Y_{\alpha} \}_{\alpha \in A}$, where $X_{\alpha}$ and $Y_{\alpha}$ are $\kappa$-compact. Let $A'$ denote the collection of all
%$\kappa$-small, filtered subsets $B \subseteq A$, and let $A'_0 \subseteq A'$ denote the subcollection consisting of those $B$ such that $f_B$ belongs to $W$. If $f$ belongs to
%$W$, then condition $(5'_{\kappa})$ implies that $A'_0$ is cofinal in $A$, so that the morphism
%$f$ is a colimit of the (filtered) system $\{ f_B: X_B \rightarrow Y_B \}_{B \in A'_0}$. Since
%each $f_B$ belongs to $W_0$ (or, more precisely, is isomorphic to a morphism which belongs to $W_0$), it follows that $W_0$ generates $W$ under filtered colimits, so that condition $(5)$ is satisfied.

%Now suppose that condition $(5)$ is satisfied for some subset $W_0 \subseteq W$. Choose
%$\kappa$ large enough that the domain and codomain of each morphism of $W_0$ is $\kappa$-compact. Enlarging $\kappa$ if necessary, we may suppose that $\kappa > \omega$. Enlarging $W_0$ if necessary, we may suppose that
%$W_0$ is stable under $\kappa$-small filtered colimits. We claim that $(5'_{\kappa})$ is satisfied. 
%To prove this, we consider any system of morphisms $\{ f_{\alpha}: X_{\alpha} \rightarrow Y_{\alpha} \}_{\alpha \in A}$ satisfying the hypotheses of $(5'_{\kappa})$. In particular, $f_A$ belongs to
%$W$, so that $f_A$ may be obtained in some {\em other} way as a filtered colimit of a system
%$\{ f'_{\alpha}: X'_{\alpha} \rightarrow Y'_{\alpha} \}_{\alpha \in A'}$, where each of the objects
%$X'_{\alpha}$ and $Y'_{\alpha}$ are $\kappa$-compact. Let $A''$ denote the family of all $\kappa$-small, filtered subsets $B \subseteq A'$. Replacing $A''$ by $A'$ and the family
%$\{ f'_{\alpha}: X'_{\alpha} \rightarrow Y'_{\alpha} \}_{\alpha \in A'}$ by
%$\{ f'_{B}: X'_B \rightarrow Y'_B \}_{ B \in A''}$, we may reduce to the case where $A'$ is $\kappa$-filtered.

%Let $C \subseteq A$ be a $\kappa$-small subset. Since $A$ is $\kappa$-filtered, $A$ contains an upper bound $\alpha_0$ for $C$ (not necessarily belonging to $C$). Consider the diagram
%$$ \xymatrix{ X_{\alpha_0} \ar[r]^{f_{\alpha_0}} \ar[d] & Y_{\alpha_{0}} \ar[d] \\
%X_A \ar[r]^{f_A} & Y_A.}$$
%Using the $\kappa$-compactness of $X_{\alpha_0}$ and $Y_{\alpha_0}$, we deduce the existence of a factorization
%$$ \xymatrix{ X_{\alpha_0} \ar[r]^{f_{\alpha_0}} \ar[d]^{g_0} & Y_{\alpha_{0}} \ar[d]^{h_0} \\
%X'_{\beta_0} \ar[r]^{f'_{\beta_0}} \ar[d] & Y'_{\beta_0} \ar[d] \\
%X_A \ar[r]^{f_A} & Y_A}$$
%for some $\beta_0 \in A'$. By the same argument, we may factor the lower square as
%$$ \xymatrix{ X'_{\beta_0} \ar[r]^{f'_{\beta_0}} \ar[d]^{g'_0} & Y_{\beta'_0} \ar[d]^{h'_0} \\
%X_{\alpha_1} \ar[r]^{ f_{\alpha_1} } \ar[d] & Y_{\alpha_1} \ar[d] \\
%X_A \ar[r]^{f_A} & Y_A}$$ 
%Enlarging $\alpha_1$ if necessary, we may suppose that $\alpha_1 > \alpha_0$. The composite
%map $g'_0 \circ g_0$ does not necessarily coincide with the map $\eta_{\alpha_0, \alpha_1}: X_{\alpha_0} \rightarrow X_{\alpha_1}$ given by the filtered system that we began with. In spite of this, the diagram
%$$ \xymatrix{ X_{\alpha_0} \ar[dr] \ar[r]^{g_0} & X'_{\beta_0} \ar[r]^{g'_0}& X_{\alpha_1} \ar[dl]\\
%& X_A }$$
%commutes. Since $X_{\alpha_0}$ is $\kappa$-compact, we may guarantee the equality
%$g'_0 \circ g_0 = \eta_{\alpha_0, \alpha_1}$ holds after enlarging $\alpha_1$. Similarly, after enlarging $\alpha_1$ further if necessary, we may ensure that $h'_0 \circ h_0$ is the defining
%map $Y_{\alpha_0} \rightarrow Y_{\alpha_1}$ of the system that we began with.

%Iterating this argument, we may produce a commutative ladder (in the category of arrows of $\calC$)
%$$ \xymatrix{ f_{\alpha_0} \ar[r] \ar[d] & f'_{\beta_0} \ar[dl] \ar[d] \\
%f_{\alpha_1} \ar[r] \ar[d] & f'_{\beta_1} \ar[dl] \ar[d] \\
%\ldots \ar[r] & \ldots }$$
%We now take $B = C \cup \{ \alpha_0, \alpha_1, \ldots \}$, which is $\kappa$-small in virtue of the assumption that $\kappa > \omega$. We note that the morphism
%$f_B$ may be identified with the colimit of the family of morphisms $\{ f_{\alpha_n} \}_{n \geq 0}$, which coincides with the colimit of the family $\{ f'_{\beta_n} \}_{n \geq 0}$. Since each $f'_{\beta_n}$ belongs to $W$, we deduce that $f_B$ belongs to $W$. 
%\end{proof}

%\begin{definition}\label{perfequiv}\index{gen}{perfect!class of morphisms}
%Let $\calC$ be a presentable category and $\kappa$ a regular cardinal. A class $W$ of morphisms in $\calC$ is {\it $\kappa$-perfect} if it satisfies the following conditions:

%\begin{itemize}
%\item[$(1)$] Every isomorphism belongs to $W$.
%\item[$(2)$] The class $W$ is stable under retracts.
%\item[$(3)$] Given a pair of composable morphisms $X \stackrel{f}{\rightarrow} Y \stackrel{g}{\rightarrow} Z$, if any two of the morphisms $f$, $g$, and $g \circ f$ belong to $W$, then so does the third.
%\item[$(4)$] The class $W$ is stable under $\kappa$-filtered colimits. More precisely, suppose given a family of morphisms $\{ f_{\alpha}: X_{\alpha} \rightarrow Y_{\alpha} \}$ which is indexed by a 
%$\kappa$-filtered partially ordered set. Let $X$ denote a colimit of $\{ X_{\alpha} \}$ and $Y$ a %colimit of
%$\{ Y_{\alpha} \}$, and $f: X \rightarrow Y$ the induced map. If each $f_{\alpha}$ belongs to $W$, then so does $f$. 
%\item[$(5)$] There exists a (small) subset $W_0 \subseteq W$ such that every morphism belonging to $W$ can be obtained as a $\kappa$-filtered colimit of morphisms belonging to $W_0$.
%\end{itemize}
%We will say that $W$ is {\it perfect} if it is $\kappa$-perfect for some regular cardinal $\kappa$.
%\end{definition}

%\begin{example}
%If $\calC$ is a presentable category, then the class $W$ consisting of all isomorphisms in $\calC$
%is perfect.
%\end{example}

%It will be convenient to consider a reformulation of condition $(5)$.


%We first show that $W \cap C$ is saturated. The only nontrivial point
%to verify is that $W \cap C$ is stable under the formation of pushouts. 

\begin{remark} 
Let $\bfA$ be a model category. Then $\bfA$ arises via the construction of Proposition \ref{goot} if and only if it is combinatorial, left proper, and the collection of weak equivalences in $\bfA$ is stable under filtered colimits. 
\end{remark}

\subsection{Simplicial Sets}\label{simpset}

The formalism of simplicial sets plays a prominent role throughout this book. In this section, we will review the definition of a simplicial set, and establish some notation.

For each $n \geq 0$, we let
$[n]$ denote the linearly ordered set $\{ 0, \ldots, n \}$.\index{not}{[n]@$[n]$}
We let $\cDelta$ denote\index{not}{Delta@$\cDelta$}
the category of {\it combinatorial simplices}: the objects of
$\cDelta$ are the linearly ordered sets $[n]$, and morphisms
in $\cDelta$ are given by (nonstrictly) order-preserving maps.

If $\calC$ is any category, a {\it simplicial object} of $\calC$\index{gen}{simplicial object!of a category}
is a functor $\cDelta^{op} \rightarrow \calC$. Dually, a {\it
cosimplicial object} of $\calC$ is a functor $\cDelta \rightarrow
\calC$. A {\it simplicial set} is a simplicial object in the
category of sets. More explicitly, a simplicial set $S$ is
determined by the following data:\index{gen}{simplicial set}

\begin{itemize}
\item A set $S_{n}$ for each $n \geq 0$ (the value of $S$
on the object $[n] \in \cDelta$).

\item A map $p^{\ast}: S_n \rightarrow S_m$ for each
order-preserving map $[m] \rightarrow [n]$, the formation of which is compatible with composition
(including empty composition, so that
$( \id_{[n]} )^{\ast} = \id_{S_n}$).
\end{itemize}

Let recall a bit of standard notation for working with a
simplicial set $S$. For each $0 \leq j \leq n$, the {\it face map}
$d_j: S_n \rightarrow S_{n-1}$ is defined to
be the pullback $p^{\ast}$, where $p: [n-1] \rightarrow [n]$ is given by
$$p(i) =
\begin{cases} i & \text{if } i<j \\
i+1 & \text{if } i \geq j. \end{cases} $$ 
Similarly, the {\it degeneracy map} $s_j: S_{n} \rightarrow S_{n+1}$
is defined to be the pullback $q^{\ast}$, where $q: [n+1] \rightarrow [n]$ is defined by the formula
$$q(i) = \begin{cases} i & \text{if } i \leq j \\
i-1 & \text{if } i > j. \end{cases}$$\index{not}{d_i@$d_i$}\index{not}{s_i@$s_i$}\index{gen}{face map}\index{gen}{degeneracy map}
Because every order-preserving map from 
$[n]$ to $[m]$ can be factored as a composition of face and degeneracy maps, 
the structure of a simplicial set $S$ is completely determined by the sets
$S_n$ for $n \geq 0$, together with the face and degeneracy operations
defined above. These operations are required to satisfy certain identities, which we will not make explicit here.

\begin{remark}
The category $\cDelta$ is equivalent to the (larger) category of all finite, nonempty linearly ordered sets. We will sometimes abuse notation by identifying $\cDelta$ with this larger subcategory, and regarding simplicial sets (or more general simplicial objects) as functors which are defined on all nonempty linearly ordered sets.
\end{remark}

\begin{notation}
The category of simplicial sets will be denoted by $\sSet$.\index{not}{sSet@$\sSet$}
If $J$ is a linearly ordered set, we let $\Delta^J \in \sSet$ denote the
representable functor 
$[n] \mapsto \Hom( [n], J)$, where the morphisms are taken in the category of linearly ordered sets. For each $n \geq 0$, we will write
$\Delta^{n}$ in place of $\Delta^{[n]}$. We observe that, for any simplicial set $S$, there
is a natural identification of sets $S_{n} \simeq \Hom_{\sSet}(\Delta^n, S)$.\index{not}{DeltaJ@$\Delta^J$}\index{not}{Deltan@$\Delta^n$}
\end{notation}

\begin{example}
For $0 \leq j \leq n$, we let $\Lambda^n_j \subset \Delta^n$\index{not}{Lambdanj@$\Lambda^n_{k}$}
denote the ``$j$-th horn''. It is determined by the following
property: an element of $(\Lambda^n_j)_m$ is given by an
order-preserving map $p: [m] \rightarrow [n]$ satisfying the condition that 
$\{j \} \cup p( [m] ) \neq [n]$. Geometrically,
$\Lambda^n_j$ corresponds to the subset of an $n$-simplex
$\Delta^n$ in which the $j$th face and the interior have been
removed.\index{gen}{horn}

More generally, if $J$ is any linearly ordered set containing an element $j$, we let
$\Lambda^J_j$ denote the simplicial subset of $\Delta^J$ obtained by removing the interior
and the ``opposite face'' to the vertex $j$.\index{not}{LambdaJj@$\Lambda^J_{j}$}
\end{example}

The category $\sSet$ of simplicial sets has a (combinatorial, left and right proper) model structure, which we will refer to as the {\it Kan model structure}. It may be described as follows:
\index{gen}{simplicial set!Kan model structure}\index{gen}{Kan!model structure}

\begin{itemize}
\item A map of simplicial sets $f: X \rightarrow Y$ is a {\it cofibration} if it is a monomorphism; that is, if the induced map $X_n \rightarrow Y_n$ is injective for all $n \geq 0$.\index{gen}{cofibration!of simplicial sets}
\item A map of simplicial sets $f: X \rightarrow Y$ is a {\it fibration} if it is a Kan fibration; that is, if for any diagram
$$ \xymatrix{ \Lambda^n_i \ar@{^{(}->}[d] \ar[r] & X \ar[d]^{f} \\
\Delta^n \ar[r] \ar@{-->}[ur] & Y}$$
it is possible to supply the dotted arrow rendering the diagram commutative.\index{gen}{fibration!Kan}\index{gen}{Kan fibration}
\item A map of simplicial sets $f: X \rightarrow Y$ is a {\it weak equivalence} if the induced map of geometric realizations $|X| \rightarrow |Y|$ is a homotopy equivalence of topological spaces.\index{gen}{weak homotopy equivalence!of simplicial sets}
\end{itemize}

To prove this, we observe that the class of all cofibrations is generated by the collection of all inclusions $\bd \Delta^n \subseteq \Delta^n$; it is then easy to see that the conditions of
Proposition \ref{goot} are satisfied. The nontrivial point
is to verify that the fibrations for the resulting model structure are precisely the Kan fibrations, and
that $\sSet$ is right proper; these facts ultimately rely on a delicate analysis due to Quillen (see \cite{goerssjardine}).

\begin{remark}
In \S \ref{compp3}, we introduce another model structure on $\sSet$, the {\it Joyal model structure}. This model structure has the same class of cofibrations, but the fibrations and the weak equivalences differ from those defined in this section. To avoid confusion, we will refer to the fibrations and weak equivalences for the usual model structure on simplicial sets as {\it Kan fibrations} and {\it weak homotopy equivalences}, respectively.
\end{remark}

%\begin{remark}\label{simpcart}
%Since the category $\sSet$ admits finite products, it may be regarded as a monoidal category with respect to the Cartesian monoidal structure. This monoidal structure is closed (in other words, $\sSet$ is a {\em Cartesian closed} category). Moreover, since the geometric realization
%functor $X \mapsto |X|$ commutes with products and detects weak equivalences, we deduce from Proposition \ref{monoidcompat} that the Cartesian monoidal structure on $\sSet$ is compatible with its model structure. In other words, we may regard $\sSet$ as a monoidal model category.
%\end{remark}

\subsection{Diagram Categories and Homotopy (Co)limits}\label{qlim7}

Let $\bfA$ be a combinatorial model category and $\calC$ a small category.
We let $\Fun(\calC, \bfA)$ denote the category of all functors from $\calC$ to $\bfA$.
In this section, we will see that $\Fun(\calC, \bfA)$ again admits the structure of a combinatorial model category: in fact, it admits two such structures. Moreover,
by considering the functoriality of this construction in the category $\calC$, we will obtain
the theory of {\it homotopy limits} and {\it homotopy colimits}.

\begin{definition}\label{injproj}\index{gen}{cofibration!projective}\index{gen}{cofibration!injective}\index{gen}{fibration!injective}\index{gen}{fibration!projective}\index{gen}{injective!cofibration}\index{gen}{injective!fibration}\index{gen}{projective!cofibration}\index{gen}{projective!fibration}\label{cooper}
Let $\calC$ be a small category, and $\bfA$ a model category.
A natural transformation $\alpha: F \rightarrow G$ in $\Fun(\calC, \bfA)$ is a:

\begin{itemize}
\item {\it injective cofibration} if the induced map $F(C) \rightarrow
G(C)$ is a cofibration in $\bfA$, for each $C \in \calC$.

\item {\it projective fibration} if the induced map $F(C) \rightarrow
G(C)$ is a fibration in $\bfA$, for each $C \in \calC$.

\item {\it weak equivalence} if the induced map $F(C) \rightarrow
G(C)$ is a weak equivalence in $\bfA$, for each $C \in \calC$.

\item {\it injective fibration} if it has the right lifting property
with respect to every morphism $\beta$ in $\Fun(\calC, \bfA)$ which is
simultaneously a weak equivalence and a injective cofibration.

\item {\it projective cofibration} if it has the left lifting property
with respect to every morphism $\beta$ in $\Fun(\calC, \bfA)$ which is
simultaneously a weak equivalence and a projective fibration.
\end{itemize}
\end{definition}

\begin{proposition}\label{smurff}\index{gen}{model category!projective}\index{gen}{model category!injective}
Let $\bfA$ be a combinatorial model category and $\calC$
be a small category. Then there exist two combinatorial model structures on $\Fun(\calC, \bfA)$:

\begin{itemize}
\item The {\it projective model structure}, determined by the strong
cofibrations, weak equivalences, and projective fibrations.

\item The {\it injective model structure}, determined by the weak
cofibrations, weak equivalences, and injective fibrations.
\end{itemize}
\end{proposition}

The following is the key step in the proof of Proposition \ref{smurf}:

\begin{lemma}\label{mainerthyme}
Let $\bfA$ be a presentable category, and $\calC$ a small category. Let $S_0$ be a (small) set of morphisms of $\bfA$, and let $\overline{S}_0$ be the weakly saturated class of morphisms generated by $S_0$. Let $\widetilde{S}$ be the collection of all morphisms $F \rightarrow G$ in $\Fun(\calC, \bfA)$ with the following property: for every $C \in \calC$, the map $F(C) \rightarrow G(C)$ belongs to $\overline{S}_0$. Then there exists a $($small$)$ set of morphisms $S$ of $\Fun(\calC, \bfA)$ which generates $\widetilde{S}$ as a weakly saturated class of morphisms.
\end{lemma}

We will prove a generalization of Lemma \ref{mainerthyme} in \S \ref{quasilimit3}
(Lemma \ref{mainertime}).

\begin{proof}[Proof of Proposition \ref{smurff}]
We first treat the case of the projective model structure. For each
For each object $C \in \calC$ and each $A \in \bfA$, we define 
$$\calF^C_A: \calC \rightarrow \bfA$$
by the formula\index{not}{FcalCA@$\calF^{C}_{A}$}
$$\calF^C_A(C') = \coprod_{ \alpha \in \bHom_{\calC}(C,C') } A.$$
We note that if $i: A \rightarrow A'$ is a (trivial) cofibration in $\bfA$, then the induced map
$\calF^C_{A} \rightarrow \calF^C_{A'}$ is a strong (trivial) cofibration in $\Fun(\calC, \bfA)$. 

Let $I_0$ be a set of generating cofibrations $i: A \rightarrow B$ for $\bfA$, and let $I$ be the set of all induced maps $\calF^{C}_{A} \rightarrow \calF^{C}_{B}$ (where $C$ ranges over $\calC$. Let $J_0$ be a set of generating trivial cofibrations for $\bfA$, and define 
$J$ likewise. It follows immediately from the definitions that a morphism in $\Fun(\calC, \bfA)$ is a projective fibration if and only if it has the right lifting property with respect to every morphism in $J$, and a weak trivial fibration if and only if it has the right lifting property with respect to every morphism in $I$. Let $\overline{I}$ and $\overline{J}$ be the weakly saturated classes of morphisms of $\Fun(\calC, \bfA)$ generated by $I$ and $J$, respectively. Using the small object argument, we deduce:
\begin{itemize}
\item[$(i)$] Every morphism $f: X \rightarrow Z$ in $\Fun(\calC, \bfA)$ admits a factorization
$$ X \stackrel{f'}{\rightarrow} Y \stackrel{f''}{\rightarrow} Z$$
where $f' \in \overline{I}$ and $f''$ is a weak trivial fibration.
\item[$(ii)$] Every morphism $f: X \rightarrow Z$ in $\Fun(\calC, \bfA)$ admits a factorization
$$ X \stackrel{f'}{\rightarrow} Y \stackrel{f''}{\rightarrow} Z$$
where $f' \in \overline{J}$ and $f''$ is a projective fibration. 
\item[$(iii)$] The class $\overline{I}$ coincides with the class of projective cofibrations in $\bfA$.
\end{itemize}
Furthermore, since the class of trivial projective cofibrations in $\Fun(\calC, \bfA)$ is weakly saturated and contains $J$, it contains $\overline{J}$. This proves that $\Fun(\calC, \bfA)$ satisfies the factorization axioms. The only other nontrivial point to check is that $\Fun(\calC, \bfA)$ satisfies the lifting axioms. Consider a diagram
$$ \xymatrix{ A \ar[d]^{i} \ar[r] & X \ar[d]^{p} \\
C \ar[r] \ar@{-->}[ur] & Y }$$
in $\Fun(\calC, \bfA)$, where $i$ is a projective cofibration and $p$ is a projective fibration. We wish to show that there exists a dotted arrow as indicated, provided that either $i$ or $p$ is a weak equivalence.
If $p$ is a weak equivalence then this follows immediately from the definition of a injective fibration.
Suppose instead that $i$ is a trivial projective cofibration. We wish to show that $i$ has the left lifting property with respect to every projective fibration. It will suffice to show every trivial injective fibration belongs to $\overline{J}$ (this will also show that $J$ is a set of generating trivial cofibrations for $\Fun(\calC, \bfA)$, which will show that the projective model structure on $\Fun(\calC, \bfA)$ is combinatorial). Suppose then that $i$ is a trivial weak coibration, and choose a factorization
$$ A \stackrel{i'}{\rightarrow} B \stackrel{i''}{\rightarrow} C$$
where $i' \in \overline{J}$ and $i''$ is a projective fibration. Then $i'$ is a weak equivalence, so
that $i''$ is a weak equivalence by the two-out-of-three property. Consider the diagram
$$ \xymatrix{ A \ar[d]^{i} \ar[r]^{i'} & B \ar[d]^{i''} \\
C \ar[r]^{=} \ar@{-->}[ur] & C. }$$
Since $i$ is a cofibration, there exists a dotted arrow as indicated. This proves that $i$ is a retract of $i'$, and therefore belongs to $\overline{J}$ as desired.

We now prove the existence of the injective model structure on $\Fun(\calC, \bfA)$. Here
it is difficult to proceed directly, so we will instead apply Proposition \ref{bigmaker}. It will suffice to check each of the hypotheses in turn:
\begin{itemize}
\item[$(1)$] The collection of injective cofibrations in $\Fun(\calC, \bfA)$ is generated (as a weakly saturated class) by some small set of morphisms. This follows from Lemma \ref{mainertime}.
\item[$(2)$] The collection of trivial injective cofibrations in $\Fun(\calC, \bfA)$ is weakly saturated: this follows immediately from the fact that the class of injective cofibrations in $\bfA$ is weakly saturated.
\item[$(3)$] The collection of weak equivalences in $\Fun(\calC, \bfA)$ is an accessible subcategory
of $\Fun(\calC, \bfA)^{[1]}$: this follows from the proof of Proposition \ref{horse1}, since the collection of weak equivalences in $\bfA$ form an accessible subcategory of $\bfA^{[1]}$.  

\item[$(4)$] The collection of weak equivalences in $\Fun(\calC, \bfA)$ satisfy the two-out-of-three property: this follows immediately from the fact that the weak equivalences in $\bfA$ satisfy the two-out-of-three property.

\item[$(5)$] Let $f: X \rightarrow Y$ be a morphism in $\bfA$ which has the right lifting property with respect to every injective cofibration. In particular, $f$ has the right lifting property with respect to each of the morphisms in the class $I$ defined above, so that $f$ is a trivial projective fibration, and in particular a weak equivalence.
\end{itemize}
\end{proof}

\begin{remark}\label{postsm}
In the situation of Proposition \ref{smurff}, if $\bfA$ is assumed to be right or left proper, then $\Fun(\calC, \bfA)$ is likewise right or left proper (with respect to either the projective or the injective model structures). 
\end{remark}

\begin{remark}\label{postsmurff}
It follows from the proof of Proposition \ref{smurf} that the class of projective cofibrations
is generated (as a weakly saturated class of morphisms) by the maps $j: \calF^{C}_{A} \rightarrow \calF^{C}_{A'}$, where $C \in \calC$ and $A \rightarrow A'$ is a cofibration in $\bfA$. 
We observe that $j$ is a injective cofibration. It follows that every projective cofibration is a injective cofibration; dually, every injective fibration is a projective fibration.
\end{remark}

\begin{remark}\label{twofus}
The construction of Proposition \ref{smurff} is functorial in the following sense:
given a Quillen adjunction of combinatorial model categories
$\Adjoint{F}{\bfA}{\bfB}{G}$ and
a small category $\calC$, composition with $F$ and $G$ determines a Quillen adjunction
$$ \Adjoint{ F^{\calC}}{\Fun(\calC, \bfA)}{\Fun(\calC, \bfB)}{G^{\calC}}$$
(with respect to either the injective or the projective model structures).
Moreover, if $(F,G)$ is a Quillen equivalence, then so is $( F^{\calC}, G^{\calC} )$.
\end{remark}

Because the projective and injective model structures on
$\Fun(\calC, \bfA)$ have the same weak equivalences, the identity
functor $\id_{\Fun(\calC, \bfA)}$ is a Quillen equivalence between them. However, it is
important to keep distinguish these two model structures, because
they have different variance properties as we now explain.

Let $f: \calC \rightarrow \calC'$ be a functor between small categories. Then
composition with $f$ yields a pullback functor $f^{\ast}:
\Fun(\calC', \bfA) \rightarrow \Fun(\calC, \bfA)$. Since $\bfA$ admits small limits and colimits, $f^{\ast}$ has a right adjoint which we shall denote by $f_{\ast}$ and a left
adjoint which we shall denote by $f_{!}$.

\begin{proposition}\label{colbinn}
Let $\bfA$ be a combinatorial model category,
and let $f: \calC \rightarrow \calC'$ be a functor between small categories.
Then:

\begin{itemize}
\item[$(1)$] The pair $( f_{!}, f^{\ast} )$ determines a Quillen
adjunction between the {\em projective} model structures on
$\Fun(\calC, \bfA)$ and $\Fun(\calC', \bfA)$.

\item[$(2)$] The pair $( f^{\ast}, f_{\ast} )$ determines a
Quillen adjunction between the {\em injective} model structures on
$\Fun(\calC, \bfA)$ and $\Fun(\calC', \bfA)$.
\end{itemize}
\end{proposition}

\begin{proof}
This follows immediately from the simple observation that
$f^{\ast}$ preserves weak equivalences, projective fibrations, and weak
cofibrations.
\end{proof}

We now review the theory of homotopy limits and colimits in a combinatorial model category $\bfA$. For simplicity, we will discuss homotopy limits and leave the analogous theory of homotopy colimits
to the reader. Let $\bfA$ be a combinatorial model category, and let 
$f: \calC \rightarrow \calC'$ be functor betweeen (small) categories. 
We wish to consider the right-derived functor $Rf_{\ast}$ of the right Kan extension $f_{\ast}: \Fun(\calC, \bfA) \rightarrow
\Fun(\calC', \bfA)$. This derived functor is called the {\it homotopy right Kan extension} functor.
The usual way of defining it involves choosing a ``fibrant replacement functor'' $Q: \Fun(\calC, \bfA) \rightarrow \Fun(\calC, \bfA)$, and setting $Rf_{\ast} = f_{\ast} \circ Q$. The assumption that $\bfA$ is combinatorial guarantees that such a fibrant replacement functor exists. However, for our purposes it is more convenient to address the indeterminacy in the definition of $Rf_{\ast}$ in another way.

Let $F \in \Fun(\calC, \bfA)$, $G \in \Fun(\calC', \bfA)$, and let $\eta: G \rightarrow f_{\ast} F$ be a map in $\Fun(\calC', \bfA)$. We will say that $\eta$ {\it exhibits $G$ as the homotopy right Kan extension of $F$}
if, for some weak equivalence $F \rightarrow F'$ where $F'$ is injectively fibrant in $\Fun(\calC, \bfA)$, the composite map $G \rightarrow f_{\ast} F \rightarrow f_{\ast} F'$ is a weak equivalence in
$\Fun(\calC', \bfA)$. Since $f_{\ast}$ preserves weak equivalences between injectively fibrant objects, this condition is independent of the choice of $F'$.\index{gen}{Kan extension!homotopy}

\begin{remark}
Given an object $F \in \Fun(\calC, \bfA)$, it is not necessarily the case that there exists a map
$\eta: G \rightarrow f_{\ast} F$ which exhibits $G$ as a homotopy right Kan extension of $F$. 
However, such a map can always be found after replacing $F$ by a weakly equivalent object; for example, if $F$ is injectively fibrant, we may take $G = f_{\ast} F$ and $\eta$ to be the identity map.
\end{remark}

Let $[0]$ denote the final object of $\Cat$: that is, the category with one object and only the identity morphism. For {\em any} category $\calC$, there is a unique
functor $f: \calC \rightarrow [0]$. If $\bfA$ is a combinatorial
model category, $F: \calC \rightarrow \bfA$ a functor, and $A \in \bfA \simeq \Fun( [0], \bfA)$
is an object, then we will say that a natural transformation
$\alpha: f^{\ast} A \rightarrow F$ {\it exhibits $A$ as a homotopy limit of $F$} if it exhibits $A$
as a homotopy right Kan extension of $F$. Note that we can identify
$\alpha$ with a map $A \rightarrow \lim_{C \in \calC} F(C)$ in the model category $\bfA$.

The theory of homotopy right Kan extensions in general can be reduced to the theory
of homotopy limits, in view of the following result:

\begin{proposition}\label{sabke}
Let $\bfA$ be a combinatorial model category, let $f: \calC \rightarrow \calD$ be a functor
between small categories, and let $F: \calC \rightarrow \bfA$ and $G: \calD \rightarrow \bfA$
be diagrams. A natural transformation $\alpha: f^{\ast} G \rightarrow F$ exhibits
$G$ as a homotopy right Kan extension of $F$ if and only if, for each object
$D \in \calD$, $\alpha$ exhibits $G(D)$ as a homotopy limit of the composite diagram
$$ F_{D/}: \calC \times_{ \calD } \calD_{D/} \rightarrow \calC \stackrel{F}{\rightarrow} \bfA. $$
\end{proposition}

To prove Proposition \ref{sabke}, we can immediately reduce to the case where
$F$ is a injectively fibrant diagram. In this case, $\alpha$ exhibits $G$ as a homotopy
right Kan extension of $F$ if and only if it induces a weak homotopy equivalence
$G(D) \rightarrow \lim F_{D/}$, for each $D \in \calD$. It will therefore suffice to prove the following result
(in the case $\calC' = \calC \times_{\calD} \calD_{D/}$):

\begin{lemma}\label{sumtuous}
Let $\bfA$ be a combinatorial model category and $g: \calC' \rightarrow \calC$
a functor which exhibits $\calC'$ as cofibered in sets over $\calC$.
Then the pullback functor $g^{\ast}: \Fun(\calC, \bfA) \rightarrow \Fun( \calC', \bfA)$
preserves injective fibrations.
\end{lemma}

\begin{proof}
It will suffice to show that the left adjoint $g_{!}$ preserves weak trivial cofibrations. 
Let $\alpha: F \rightarrow F'$ be a map in $\Fun( \calC', \bfA)$.
We observe that for each object $C \in \calC$, the map 
$(q_{!} \alpha)(C): (q_{!} F)(C) \rightarrow (q_{!} F')(C)$ can be identified
with the coproduct of the maps $\{ \alpha(C'): F(C') \rightarrow F'(C') \}_{ C' \in g^{-1} \{C\} }$.
If $\alpha$ is a weak trivial cofibration, then each of these maps is a trivial cofibration in $\bfA$, so that
$q_{!} \alpha$ is again a weak trivial cofibration as desired.
\end{proof}

\begin{remark}
In the preceding discussion, we considered injective model structures,
$Rf_{\ast}$, and homotopy limits. An entirely dual discussion may
be carried out with projective model structures and $Lf_{!}$; one obtains a
notion of {\it homotopy colimit} which is the dual of the notion
of homotopy limit.
\end{remark}

\begin{example}
Let $\bfA$ be a combinatorial model category, and consider a diagram
$$ X' \stackrel{f}{\leftarrow} X \stackrel{g}{\rightarrow} X''.$$
This diagram is projectively cofibrant if and only if the object $X$ is cofibrant, and the maps
$f$ and $g$ are both cofibrations. Consequently, the definition of homotopy colimits given above recovers, as a special case, the theory of homotopy pushouts presented in \S \ref{hopush}.
\end{example}

\subsection{Reedy Model Structures}\label{coreed}

Let $\bfA$ be a combinatorial model category and $\calJ$ a small category. In
\S \ref{qlim7}, we saw that the diagram category $\Fun( \calJ, \bfA)$ can again be regarded as a combinatorial model category, via either the projective or injective model structure of
Proposition \ref{smurff}. In the special case where $\calJ$ is a {\em Reedy category}
(see Definition \ref{quellreed}), it is often useful to consider still another model structure on $\Fun(\calJ, \bfA)$: the {\it Reedy model structure}. We will sketch the definition and some of the basic properties of Reedy model categories below; we refer the reader to \cite{hirschhorn} for a more detailed treatment.

\begin{definition}\label{quellreed}\index{gen}{Reedy!category}\index{gen}{category!Reedy}
A {\it Reedy category} is a small category $\calJ$ equipped with a factorization
system $\calJ^{L}, \calJ^{R} \subseteq \calJ$ satisfying the following conditions:
\begin{itemize}
\item[$(1)$] Every isomorphism in $\calJ$ is an identity map.
\item[$(2)$] Given a pair of object $X, Y \in \calJ$, let us write
$X \preceq_0 Y$ if either there exists a morphism $f: X \rightarrow Y$ belonging
to $\calJ^{R}$ or there exists a morphism $g: Y \rightarrow X$ belonging to $\calJ^{L}$.
We will write $X \prec_0 Y$ if $X \preceq_0 Y$ and $X \neq Y$.
Then there are no infinite descending chains $$ \ldots \prec_0 X_2 \prec_0 X_1 \prec_0 X_0.$$
\end{itemize}
\end{definition}

\begin{remark}
Let $\calJ$ be a category equipped with a factorization system $(\calJ^{L}, \calJ^{R})$, and let
$\preceq_0$ be the relation described in Definition \ref{quellreed}. This relation is generally not transitive. We will denote its transitive closure by $\preceq$. Then condition
$(2)$ of Definition \ref{quellreed} guarantees that $\preceq$ is a well-founded partial ordering on the set of objects of $\calJ$. In other words, every nonempty set $S$ of objects
of $\calJ$ contains a $\preceq$-minimal element.
\end{remark}

\begin{remark}
In the situation of Definition \ref{quellreed}, we will often abuse terminology and simply refer to $\calJ$ as a Reedy category, implicitly assuming that a factorization system on $\calJ$ has been specified as well.
\end{remark}

\begin{warning}
Condition $(1)$ of Definition \ref{quellreed} is not stable under equivalence of categories.
Suppose that $\calJ$ is equivalent to a Reedy category. Then $\calJ$ can itself be regarded as a Reedy category if and only if every isomorphism class of objects in $\calJ$ contains
a unique representative. (Definition \ref{quellreed} can easily be modified so as to be invariant under equivalence, but it is slightly more convenient not to do so.)
\end{warning}

\begin{example}\label{onep}
The category $\cDelta$ of combinatorial simplices is a Reedy category with respect to
the factorization system $( \cDelta^{L}, \cDelta^{R} )$; here a morphism
$f: [m] \rightarrow [n]$ belongs to $\cDelta^{L}$ if and only if $f$ is surjective, and
to $\cDelta^{R}$ if and only if $f$ is injective.
\end{example}

\begin{example}\label{twop}
Let $\calJ$ be a Reedy category with respect to the factorization system
$( \calJ^{L}, \calJ^{R})$. Then $\calJ^{op}$ is a Reedy category with respect to the
factorization system $( (\calJ^R)^{op}, (\calJ^{L})^{op} )$. 
\end{example}

\begin{notation}\index{gen}{latching object}\index{gen}{object!latching}\index{not}{LsubJ@$L_{J}(X)$}\label{bugga}
Let $\calJ$ be a Reedy category, $\calC$ a category which admits small limits and colimits, and $X: \calJ \rightarrow \calC$ a functor. For every object $J \in \calJ$, we define the
{\it latching object} $L_{J}(X)$ to be the colimit
$$ \varinjlim_{J' \in \calJ^{R}_{/J}, J' \neq J} X(J').$$
Similarly, we define the {\it matching object}\index{gen}{matching object}\index{gen}{object!matching}
\index{not}{MsubJ@$M_{J}(X)$} to be the limit
$$ \varprojlim_{J' \in \calJ^{L}_{J/}, J' \neq J} X(J').$$
We then have canonical maps
$L_{J}(X) \rightarrow X(J) \rightarrow M_{J}(X).$
\end{notation}

\begin{example}\label{croupus}
Let $X: \cDelta^{op} \rightarrow \Set$ be a simplicial set, and regard
$\cDelta^{op}$ as a Reedy category using Examples \ref{onep} and \ref{twop}.
For every nonnegative integer $n$, the latching object $L_{[n]} X$ can be identified
with the collection of all degenerate simplices of $X$. In particular, the map
$L_{[n]}(X) \rightarrow X([n])$ is always a monomorphism.

More generally, we observe that a map of simplicial sets $f: X \rightarrow Y$ is a monomorphism if and only if, for every $n \geq 0$, the map
$$ L_{[n]}(Y) \coprod_{ L_{[n]}(X) } X([n]) \rightarrow Y([n])$$
is a monomorphism of sets. The ``if'' direction is obvious. For the converse,
let us suppose that $f$ is a monomorphism; we must show that if
$\sigma$ is an $n$-simplex of $X$ such that $f(\sigma)$ is degenerate, then
$\sigma$ is already degenerate. If $f(\sigma)$ is degenerate, then
$f(\sigma) = \alpha^{\ast} f(\sigma) = f( \alpha^{\ast} \sigma)$ where $\alpha: [n] \rightarrow [n]$ is a map of linearly ordered sets other than the identity. Since $f$ is a monomorphism, we deduce that $\sigma = \alpha^{\ast} \sigma$, so that $\sigma$ is degenerate as desired.
\end{example}

\begin{remark}
Let $X: \calJ \rightarrow \calC$ be as in Notation \ref{bugga}. Then the
$J$th matching object $M_{J}(X)$ can be identified with the $J$th latching object
of the induced functor $X^{op}: \calJ^{op} \rightarrow \calC^{op}$. 
\end{remark}

\begin{remark}\label{surpose}
Let $X: \calJ \rightarrow \calC$ be as in Notation \ref{bugga}. Then the
$J$th matching object $M_{J}(X)$ can also be identified with the colimit
$$ \varinjlim_{(f: J' \rightarrow J) \in S} X(J')$$
where $S$ is any full subcategory of $\calJ_{/J}$ with the following properties:
\begin{itemize}
\item[$(1)$] Every morphism $f: J' \rightarrow J$ which belongs to
$\calJ^{R}$ and is not an isomorphism also belongs to $S$.
\item[$(2)$] If $f: J' \rightarrow J$ belongs to $S$, then
$J \npreceq J'$.
\end{itemize}
This follows from a cofinality argument, since every morphism
$f: J' \rightarrow J$ in $S$ admits a canonical factorization
$$ J' \stackrel{f'}{\rightarrow} J'' \stackrel{f''}{\rightarrow} J,$$
where $f'$ belongs to $\calJ^{L}$ and $f''$ belongs to $\calJ^{R}$. Assumption
$(2)$ guarantees that the map $f''$ is not an isomorphism.

Similarly, we can replace the limit $\varprojlim_{ f: J \rightarrow J'} X(J')$
defining the matching object $M_{J}(X)$ by a limit over a slightly larger category, when convenient.
\end{remark}

\begin{notation}
Let $\calJ$ be a Reedy category. A {\it good filtration} of $\calJ$ is a transfinite sequence
$$ \{ \calJ_{\beta} \}_{\beta < \alpha}$$ of full subcategories of $\calJ$ with the following properties:
\begin{itemize}
\item[$(a)$] The filtration is exhaustive in the following sense: every object of
$\calJ$ belongs to $\calJ_{\beta}$ for sufficiently large $\beta < \alpha$.
\item[$(b)$] For each ordinal $\beta < \alpha$, the category $\calJ_{\beta}$ is obtained
from the subcategory $\calJ_{< \beta} = \bigcup_{ \gamma < \beta} \calJ_{\gamma}$
by adjoining a single new object $J_{\beta}$ satisfying the following condition:
if $J \in \calJ$ satisfies $J \prec J_{\beta}$, then $J \in \calJ_{< \beta}$.
\end{itemize}
\end{notation}

\begin{remark}
Let $\calJ$ be a Reedy category. Then there exists a good filtration of $\calJ$. In fact, the
existence of a good filtration is {\em equivalent} to the second assumption of Definition \ref{quellreed}.  
\end{remark}

\begin{remark}
Let $\calJ$ be a Reedy category with respect to the factorization system
$(\calJ^{L}, \calJ^{R})$, and let $\{ \calJ_{\beta} \}_{\beta < \alpha}$ be a good filtration of $\calJ$. Then each $\calJ_{\beta}$ admits a factorization system
$( \calJ^{L} \cap \calJ_{\beta}, \calJ^{R} \cap \calJ_{\beta})$. In other words, if
$f: I \rightarrow K$ is a morphism in $\calJ_{\beta}$ which admits a factorization
$$ I \stackrel{f'}{\rightarrow} J \stackrel{f''}{\rightarrow} K$$
where $f'$ belongs to $\calJ^{L}$ and $f''$ belongs to $\calJ^{R}$, then
the object $J$ also belongs to $\calJ_{\beta}$. This is clear: either
$f''$ is an isomorphism, in which case $J = K \in \calJ_{\beta}$, or
$f''$ is not an isomorphism so that $J \prec K$ implies that $J \in \calJ_{< \beta}$.
\end{remark}

The following result summarizes the essential features of a good filtration:

\begin{proposition}\label{twingood}
Let $\calJ$ be a Reedy category with a good filtration $\{ \calJ_{\beta} \}_{ \beta < \alpha}$,
let $\beta < \alpha$ be an ordinal, so that $\calJ_{\beta}$ is obtained from $\calJ_{< \beta}$ by 
adjoining a single new object $J$. Then we have a homotopy pushout square
(with respect to the Joyal model structure)
$$ \xymatrix{ \Nerve (\calJ_{< \beta})_{/J} \star \Nerve (\calJ_{< \beta})_{J/}
\ar@{^{(}->}[d] \ar[r] & \Nerve(\calJ_{< \beta}) \ar@{^{(}->}[d] \\
\Nerve( \calJ_{< \beta})_{/J} \star \{J\} \star \Nerve( \calJ_{< \beta})_{J/} \ar[r]
& \Nerve( \calJ_{\beta}). }$$
\end{proposition}

\begin{corollary}\label{stapler}
Let $\calJ$ be a Reedy category with a good filtration $\{ \calJ_{\beta} \}_{\beta < \alpha}$,
let $\beta < \alpha$ be an ordinal, so that $\calJ_{\beta}$ is obtained from
$\calJ_{< \beta}$ by adjoining a single new object $J$. Let
$\calC$ be a category which admits small limits and colimits, and let
$X: \calJ_{< \beta} \rightarrow \calC$ be a functor, and let the latching and matching
objects $L_J(X)$ and $M_{J}(X)$ be defined as in Notation \ref{bugga}
(note that this does not require that the functor $X$ be defined on the object $J$), so that we have a canonical map $\alpha: L_{J}(X) \rightarrow M_{J}(X)$. The following data are equivalent:
\begin{itemize}
\item[$(1)$] A functor $\overline{X}: \calJ_{\beta} \rightarrow \calC$ extending $X$.
\item[$(2)$] A commutative diagram
$$ \xymatrix{ & C \ar[dr] & \\
L_{J}(X) \ar[ur] \ar[rr]^{\alpha} & & M_{J}(X) }$$
in the category $\calC$.
\end{itemize}
The equivalence carries a functor $\overline{X}$ to the evident diagram with
$C = \overline{X}(J)$.
\end{corollary}

\begin{proof}
Using Proposition \ref{twingood}, we see that giving an extension $\overline{X}: \calJ_{\beta} \rightarrow \calC$ of $X$ is equivalent to giving an extension $\overline{Y}: ( \calJ_{< \beta})_{/J} \star \{J\} \star ( \calJ_{< \beta})_{J/} \rightarrow \calC$ of the composite functor
$$Y: (\calJ_{< \beta})_{/J} \star (\calJ_{< \beta})_{J/} \rightarrow \calJ_{< \beta} \stackrel{X}{\rightarrow} \calC.$$
This, in turn, is equivalent to giving a commutative diagram
$$ \xymatrix{ & C \ar[dr] & \\
\varinjlim Y| (\calJ_{< \beta})_{/J} \ar[ur] \ar[rr]^{\alpha'} & & \varprojlim Y|(\calJ_{< \beta})_{J/}) }$$
where $\alpha'$ is the map induced by the diagram $Y$. The equivalence of this with the
data $(2)$ follows immediately from Remark \ref{surpose}.
\end{proof}

\begin{remark}\label{stapler2}
The proof of Corollary \ref{stapler} carries over without essential change to the case where
$\calC$ is an $\infty$-category which admits small limits and colimits. In this case, to
extend a functor $X: \Nerve( \calJ_{< \beta}) \rightarrow \calC$ to a functor $\overline{X}$ defined on the whole of $\calJ_{\beta}$, it will suffice to specify the object
$$ \overline{X}(J) \in \calC_{ X | (\calJ_{< \beta})_{/J} / \, X | (\calJ_{< \beta})_{J/}}
\simeq \calC_{ L_{J}(X) / \, / M_J(X) },$$
where the latching and matching objects $L_J(X), M_J(X) \in \calC$ are defined in the obvious way.
\end{remark}

The proof of Proposition \ref{twingood} will require a few preliminaries.

\begin{lemma}\label{gump1}
Let $\calJ$ be a Reedy category equipped with a good filtration
$\{ \calJ_{\beta} \}_{\beta < \alpha}$. Fix $\beta < \alpha$, and let
$\calJ_{\beta}$ be obtained from $\calJ_{< \beta}$ by adjoining the object $J$.
Let $f: J \rightarrow J$ be a map which is not the identity, let
$\calI$ denote the category $(\calJ_{J/})_{/f} \simeq (\calJ_{/J})_{f/}$ of
factorizations of the morphism $f$, and let $\calI_0 = \calI \times_{\calJ} \calJ_{< \beta}$. Then
the nerve $\Nerve \calI_0$ is weakly contractible.
\end{lemma}

\begin{proof}
Let $\calI_1$ denote the full subcategory of $\calI_0$ spanned by those diagrams
$$ \xymatrix{ & I \ar[dr]^{f''} & \\
J \ar[rr]^{f} \ar[ur]^{f'} & & J }$$
where $I \in \calJ_{< \beta}$ and $f''$ is a morphism in $\calJ^{R}$. The inclusion
$\calI_1 \subseteq \calI_0$ admits a left adjoint, so that $\Nerve \calI_1$ is a deformation retract
of $\Nerve \calI_0$. It will therefore suffice to show that $\Nerve \calI_1$ is weakly contractible.
Let $\calI_2$ denote the full subcategory of $\calI_1$ spanned by those diagrams as above
where, in addition, the morphism $f'$ belongs to $\calJ^{L}$. Then the inclusion
$\calI_2 \subseteq \calI_1$ admits a right adjoint, so that $\Nerve \calI_2$ is a deformation retract of $\Nerve \calI_1$. It will therefore suffice to show that $\Nerve \calI_2$ is weakly contractible. This is clear, since the category $\calI_2$ consists of a single object (with no nontrivial endomorphisms).
\end{proof}

\begin{lemma}\label{gump2}
Let $n \geq 1$, and suppose given a sequence of weakly contractible simplicial sets
$\{ A_i \}_{1 \leq i \leq n}$. Let $L$ denote the iterated join
$$ \{ J_0 \} \star A_1 \star \{J_1 \} \star A_2 \star \ldots \star A_n \star \{J_n \},$$
and let $K$ denote the simplicial subset of $L$ spanned by those simplices which do not contain
all of the vertices $\{ J_i \}_{0 \leq i \leq n}$. Then the inclusion $K \subseteq L$ is a categorical equivalence of simplicial sets.
\end{lemma}

\begin{proof}
If $n=1$, then this follows immediately from Lemma \ref{storuse}.
Suppose that $n > 1$. Let $X$ denote the iterated join
$A_1 \star \{J_1 \} \star A_2 \star \ldots \star \{J_{n-1} \} \star A_n$.
For every subset $S \subseteq \{ 1, \ldots, n-1\}$, let
$X(S)$ denote the simplicial subset of $X$ spanned by those simplices
which do not contain any vertex $J_i$ for $i \in S$.
Let $X' = \bigcup_{ S \neq \emptyset} X(S) \subseteq X(\emptyset) = X$.
Then $X'$ is the homotopy colimit of the diagram of simplicial sets
$\{ X(S) \}_{S \neq \emptyset}$. Each $X(S)$ is a join of weakly contractible simplicial sets,
and is therefore weakly contractible. Since $n > 1$, the partially ordered set
$\{ S \subseteq \{1, \ldots, n-1\}: S \neq \emptyset \}$ has a largest element, and is therefore
weakly contractible. It follows that the simplicial set $X'$ is weakly contractible.

The assertion that the inclusion $K \subseteq L$ is a categorical equivalence is
equivalent to the assertion that the diagram
$$ \xymatrix{ (\{J_0\} \star X') \coprod_{X'} ( X' \star \{J_n\}) \ar@{^{(}->}[d] \ar@{^{(}->}[r] & 
( \{J_0\} \star X) \coprod_{ X} (X \star \{J_0\}) \ar@{^{(}->}[d] \\
\{J_0\} \star X' \star \{J_n\} \ar@{^{(}->}[r] & \{J_0\} \star X \star \{J_n \} }$$
is a homotopy pushout square (with respect to the Joyal model structure).
To prove this, it suffices to observe that the vertical maps are both categorical equivalences (Lemma \ref{storuse}).
\end{proof}

\begin{proof}[Proof of Proposition \ref{twingood}]
Let $S$ denote the collection of all composable chains of morphisms
$$\overline{f}: J \stackrel{ f_1}{\rightarrow} J \stackrel{ f_2} \ldots \stackrel{f_n}{\rightarrow} J.$$
where $n \geq 1$ and each $f_{i} \neq \id_{J}$. For every subset $S' \subseteq S$, let
$X(S')$ denote the simplicial subset of $\Nerve( \calJ_{\beta})$ spanned by those simplices
$\sigma$ satisfying the following condition:
\begin{itemize}
\item[$(\ast)$] For every nondegenerate face $\tau$ of $\sigma$ of positive dimension, if every
vertex of $\tau$ coincides with $J$, then $\tau$ belongs to $S'$.
\end{itemize}
Note that $X(S)$ coincides with $\Nerve( \calJ_{\beta})$, while $X( \emptyset)$ coincides with
the pushout 
$$(\Nerve (\calJ_{< \beta})_{/J} \star \{ J \} \star \Nerve (\calJ_{< \beta})_{J/})
\coprod_{ \Nerve( \calJ_{< \beta})_{/J} \star \Nerve( \calJ_{< \beta})_{J/}}
\Nerve( \calJ_{< \beta} ). $$
It will therefore suffice to show that the inclusion $X(\emptyset) \subseteq X(S)$ is a
categorical equivalence of simplicial sets.

Choose a well-ordering
$$ S = \{ \overline{f}_0 < \overline{f}_1 < \overline{f}_2 < \ldots \}$$
with the following property: if $\overline{f}$ has length shorter than $\overline{g}$
(when regarded as a chain of morphisms), then $\overline{f} < \overline{g}$.
For every ordinal $\alpha$, let $S_{\alpha} = \{ \overline{f}_{\beta} \}_{\beta < \alpha}$.
We will prove that for every ordinal $\alpha$, the inclusion
$X( \emptyset) \subseteq X( S_{\alpha})$ is a categorical equivalence.
The proof proceeds by induction on $\alpha$. If $\alpha = 0$ there is nothing to prove, and
if $\alpha$ is a limit ordinal then the desired result follows from the inductive hypothesis and the fact that the class of categorical equivalences is stable under filtered colimits. We may therefore assume that
$\alpha = \beta + 1$ is a successor ordinal. The inductive hypothesis guarantees that
$X(\emptyset) \subseteq X( S_{\beta})$ is a categorical equivalence. It will therefore suffice to show that
the inclusion $j: X( S_{\beta} ) \subseteq X( S_{\alpha})$ is a categorical equivalence.
We may also suppose that $\beta$ is smaller than the order type of $S$, so that
$\overline{f}_{\beta}$ is well-defined (otherwise, the inclusion $j$ is an isomorphism and the result is obvious). 

Let $\overline{f} = \overline{f}_{\beta}$ be the composable chain of morphisms
$$\overline{f}: J \stackrel{ f_1}{\rightarrow} J \stackrel{ f_2}{\rightarrow} \ldots \stackrel{f_n}{\rightarrow} J.$$
For $1 \leq i \leq n$, let $A_i$ denote the nerve of the category
$$\calJ_{< \beta} \times_{\calJ} ( \calJ_{J/})_{/ f_i} \simeq \calJ_{< \beta} \times_{\calJ}
( \calJ_{/J})_{f_i/ }.$$
Let $K$ denote the simplicial subset of
$$ \{ J_0 \} \star A_1 \star \{J_1 \} \star A_2 \star \ldots \star A_n \star \{J_n \}$$
spanned by those simplices which do not contain every vertex $J_{n}$.
We then have a homotopy pushout diagram
$$ \xymatrix{ \Nerve(\calJ_{< \beta})_{/J} \star K \star \Nerve(\calJ_{< \beta})_{J/} \ar@{^{(}->}[d] \ar[r] & 
X( S_{\beta}) \ar[d] \\
\Nerve(\calJ_{< \beta})_{/J} \star \{J_0\} \star A_1 \star \ldots \star A_n \star \{J_n\} 
\star \Nerve(\calJ_{< \beta})_{J/} \ar[r] & X( S_{\alpha}).}$$
It will therefore suffice to prove that the left vertical map is a categorical equivalence.
In view of Corollary \ref{gyyyt}, it will suffice to show that the inclusion
$$ K \subseteq \{ J_0 \} \star A_1 \star \{J_1 \} \star A_2 \star \ldots \star A_n \star \{J_n \}$$
is a categorical equivalence. Since each $A_i$ is weakly contractible
(Lemma \ref{gump1}), this follows immediately from Lemma \ref{gump2}.
\end{proof}

\begin{proposition}\index{gen}{Reedy!model structure}\index{gen}{model category!Reedy}\label{reedmod}
Let $\calJ$ be a Reedy category, and let $\bfA$ be a model category.
Then there exists a model structure on the category of functors $\Fun(\calJ, \bfA)$
with the following properties:
\begin{itemize}\index{gen}{cofibration!Reedy}\index{gen}{Reedy!cofibration}
\item[$(C)$] A morphism $X \rightarrow Y$ in $\Fun(\calJ, \bfA)$ is a 
{\it Reedy cofibration} if and only if, for every object $J \in \calJ$, the induced map
$X(J) \coprod_{ L_{J}(X) } L_{J}(Y) \rightarrow Y(J)$ is a cofibration in $\bfA$.
\item[$(F)$] A morphism $X \rightarrow Y$ in $\Fun(\calJ, \bfA)$ is a\index{gen}{fibration!Reedy}
{\it Reedy fibration} if and only if, for every object $J \in \calJ$, the induced map
$X(J) \rightarrow Y(J) \times_{ M_{J}(Y) } M_{J}(X)$ is a fibration in $\bfA$.\index{gen}{Reedy!fibration}
\item[$(W)$] A morphism $X \rightarrow Y$ in $\Fun(\calJ, \bfA)$ is a weak equivalence if and only if, for every $J \in \calJ$, the map $X(J) \rightarrow Y(J)$ is a weak equivalence.
\end{itemize} 
Moreover, a morphism $f: X \rightarrow Y$ in $\Fun( \calJ, \bfA)$ is a trivial cofibration if
and only if the following condition is satisfied:
\begin{itemize}
\item[$(WC)$] For every object $J \in \calJ$, the map $X(J) \coprod_{ L_{J}(X) } L_{J}(Y) \rightarrow Y(J)$ is a trivial cofibration in $\bfA$.
\end{itemize}
Similarly, $f$ is a fibration if and only if it satisfies the dual condition:
\begin{itemize}
\item[$(WF)$] For every object $J \in \calJ$, the map $X(J) \rightarrow Y(J) \times_{ M_J(Y)} M_{J}(X)$ is a trivial fibration in $\bfA$.
\end{itemize}
\end{proposition}

The model structure of Proposition \ref{reedmod} is called the {\it Reedy model structure} on 
$\Fun(\calJ, \bfA)$. Note that Proposition \ref{reedmod} does not require that the model category $\bfA$ is combinatorial.

\begin{lemma}\label{jackal}
Let $\calJ$ be a Reedy category containing an object $J$, let $\bfA$ be a model category, and let $f: F \rightarrow G$ be a natural transformation in $\Fun(\calJ, \bfA)$.
Let $\calI \subseteq \calJ^{R}_{/J}$ be a sieve: that is, $\calI$ is a full subcategory
of $\calJ^{R}_{/J}$ with the property that if $I \rightarrow I'$ is a morphism in
$\calJ^{R}_{/J}$ such that $I' \in \calI$, then $I \in \calI$. Let $\calI' \subseteq \calI$ be another sieve. Then:
\begin{itemize}
\item[$(a)$] If the map $f$ satisfies condition $(C)$ of Proposition \ref{reedmod} for every object $I \in \calI$, then
the induced map
$$ \chi_{\calI', \calI}: \varinjlim( F | \calI) \coprod_{ \varinjlim(F | \calI') } \varinjlim( G | \calI')
\rightarrow \varinjlim( G | \calI )$$
is a cofibration in $\bfA$.
\item[$(b)$] If the map $f$ satisfies condition $(WC)$ of Proposition \ref{reedmod} for every object $I \in \calI$, then the map $\chi_{\calI', \calI}$ is a trivial cofibration in $\bfA$.
\end{itemize}
\end{lemma}

\begin{proof}
We will prove $(a)$; the proof of $(b)$ is identical. Choose a transfinite sequence of sieves
$\{ \calI_{\beta} \subseteq \calI \}_{ \beta < \alpha }$ with the following properties:
\begin{itemize}
\item[$(i)$] The union $\bigcup_{\beta < \alpha} \calI_{\beta}$ coincides with $\calI$.
\item[$(ii)$] For each $\beta < \alpha$, the sieve $\calI_{\beta}$ is obtained
from $\calI_{< \beta} = \calI' \cup (\bigcup_{ \gamma < \beta} \calI_{\gamma})$
by adjoining a single new object $J_{\beta} \in \calJ^{R}_{/J}$. 
\end{itemize}
For every triple $\delta \leq \gamma \leq \beta \leq \alpha$, let
$\chi_{ \delta, \gamma, \beta}$ denote the induced map
$$ \varinjlim( F| \calI_{< \beta}) \coprod_{ \varinjlim(F | \calI_{< \delta})}
\varinjlim( G| \calI_{< \delta}) \rightarrow
\varinjlim( F | \calI_{< \beta} ) \coprod_{ \varinjlim( F| \calI_{< \gamma}) }
\varinjlim( G| \calI_{< \gamma} ).$$
We wish to prove that $\chi_{0, \alpha, \alpha}$ is a cofibration.
We will prove more generally that $\chi_{\delta, \gamma, \beta}$ is an
equivalence for every $\delta \leq \gamma \leq \beta \leq \alpha$.
The proof uses induction on $\gamma$. If $\gamma$ is a limit ordinal, 
then we can write $\chi_{\delta, \gamma, \beta}$ as the transfinite composition
of the maps $\{ \chi_{ \epsilon, \epsilon + 1, \beta} \}_{\delta \leq \epsilon < \gamma}$, which are cofibrations by the inductive hypothesis. We may therefore assume that
$\gamma = \gamma_0 + 1$ is a successor ordinal. If $\delta = \gamma$, then
$\chi_{\delta, \gamma, \beta}$ is an isomorphism; otherwise, we have
$\delta \leq \gamma_0$. In this case, we have
$$ \chi_{ \delta, \gamma, \beta} = \chi_{ \gamma_0, \gamma, \beta} \circ \chi_{ \delta, \gamma_0, \beta}.$$
Using the inductive hypothesis, we can reduce to the case $\delta = \gamma_0$. The map
$\chi_{\gamma_0, \gamma, \beta}$ is a pushout of the map $\chi_{\gamma_0, \gamma, \gamma}$. We are therefore reduced to proving that $\chi_{\gamma_0, \gamma, \gamma}$
is a cofibration. But $\chi_{\gamma_0, \gamma, \gamma}$ is a pushout of the map
$L_{I}(G) \coprod_{ L_{I}(F)} F(I) \rightarrow G(I)$ for $I = J_{\gamma_0}$.
This map is a cofibration by virtue of our assumption that $f$ satisfies $(C)$.
\end{proof}

\begin{proof}[Proof of Proposition \ref{reedmod}]
Let $f: X \rightarrow Z$ be a morphism in $\Fun(\calJ, \bfA)$. We will prove that
$f$ admits a factorization
$$ X \stackrel{f'}{\rightarrow} Y \stackrel{f''}{\rightarrow} Z$$
where:
\begin{itemize}
\item[$(i)$] The map $f''$ is a fibration, and $f'$ satisfies $(WC)$.
\item[$(ii)$] The map $f'$ is a cofibration, and $f''$ satisfies $(WF)$.
\end{itemize}
By symmetry, it will suffice to consider case $(i)$. Choose a good filtration
$\{ \calJ_{\beta} \}_{ \beta < \alpha}$ of $\calJ$. For $\beta < \alpha$, let
$X_{\beta} = X | \calJ_{\beta}$, $Z_{\beta} = Z | \calJ_{\beta}$, and let
$f_{\beta}: X_{\beta} \rightarrow Z_{\beta}$ be the restriction of $f$. We will construct
a compatible family of factorizations of $f_{\beta}$ as a composition
$$ X_{\beta} \stackrel{f'_{\beta}}{\rightarrow} Y_{\beta} \stackrel{f''_{\beta}}{\rightarrow} Z_{\beta}.$$
Suppose that $\calJ_{\beta}$ is obtained from $\calJ_{< \beta}$ by adjoining a single new object $J$. Assuming that $(f'_{\gamma}, f''_{\gamma})$ has been constructed for all
$\gamma < \beta$, we note that constructing $(f'_{\beta}, f''_{\beta})$ is equivalent
(by virtue of Corollary \ref{stapler}) to giving a commutative diagram
$$ \xymatrix{ L_{J}(X) \ar[d] \ar[r] & L_{J}(Y_{< \beta}) \ar[d] & \\
X(J) \ar[r] & Y_{\beta}(J) \ar[r] \ar[d] & Z(J) \ar[d] \\
& M_J(Y_{< \beta}) \ar[r] & M_{J}(Z). }$$
In other words, we must factor a certain map
$$g: L_{J}( Y_{< \beta}) \coprod_{ L_{J}(X) } X(J) \rightarrow
M_{J}(Y_{< \beta} ) \times_{ M_{J}(Z)} Z(J)$$
as a composition
$$ L_{J}( Y_{< \beta}) \coprod_{ L_{J}(X) } X(J) \stackrel{g'}{\rightarrow}
Y_{\beta}(J) \stackrel{g''}{\rightarrow}
M_{J}(Y_{< \beta} ) \times_{ M_{J}(Z)} Z(J).$$
Using the fact that $\bfA$ is a model category, we can choose a factorization
where $g'$ is a trivial cofibration and $g''$ a fibration. It is readily verified that this construction has the desired properties.

We now prove the following:
\begin{itemize}
\item[$(i')$] A morphism $f: X \rightarrow Y$ in $\Fun(\calJ, \bfA)$ satisfies $(WC)$ if and only if
$f$ is both a fibration and a weak equivalence.
\item[$(ii')$] A morphism $f: X \rightarrow Y$ in $\Fun(\calJ, \bfA)$ satisfies $(WF)$ if and only if $f$ is both a cofibration and a weak equivalence.
\end{itemize}
By symmetry, it will suffice to prove $(i')$. The ``only if'' direction follows from
Lemma \ref{jackal}. For the ``if'' direction, it will suffice to show that for
each $\beta < \alpha$, the induced transformation
$f_{\beta}: X_{\beta} \rightarrow Y_{\beta}$ satisfies $(WC)$ when regarded as a morphism
of $\Fun( \calJ_{\beta}, \bfA)$. Suppose that $\calJ_{\beta}$ is obtained from $\calJ_{< \beta}$ by adjoining a single new element $J$. We have a commutative diagram
$$ \xymatrix{ & L_{J}(Y) \coprod_{ L_{J}(X) } X(J) \ar[dr]^{q} & \\
X(J) \ar[ur]^{p} \ar[rr]^{r} & & Y(J). }$$
We wish to prove that $q$ is a trivial cofibration in $\bfA$. Since $f$ is a cofibration in $\Fun(\calJ, \bfA)$, the map $q$ is a cofibration in $\bfA$. It will therefore suffice to show that
$q$ is a weak equivalence. By the two-out-of-three property, it will suffice to show that
$p$ and $r$ are weak equivalences. For $r$, this follows from our assumption that
$f$ is a weak equivalence in $\Fun(\calJ, \bfA)$. The map $p$ is a pushout of the map of latching objects $L_{J}(X) \rightarrow L_{J}(Y)$, which is a cofibration in $\bfA$ by virtue of the inductive hypothesis and Lemma \ref{jackal}.

Combining $(i)$ and $(i')$ (and the analogous assertions $(ii)$ and $(ii')$), we deduce that
$\Fun(\calJ, \bfA)$ satisfies the factorization axioms for a model category. To complete the proof, it will suffice to verify the lifting axioms:
\begin{itemize}
\item[$(i'')$] Every fibration in $\Fun(\calJ, \bfA)$ has the right lifting property with respect to
morphisms in $\Fun(\calJ, \bfA)$ which satisfy $(WC)$.
\item[$(ii'')$] Every cofibration in $\Fun(\calJ, \bfA)$ has the left lifting property with respect to morphisms in $\Fun(\calJ, \bfA)$ which satisfy $(WF)$.
\end{itemize}
Again, by symmetry it will suffice to prove $(i'')$. Consider a diagram
$$ \xymatrix{ A \ar[r] \ar[d]^{f} & X \ar[d]^{g} \\
B \ar[r] \ar@{-->}[ur]^{h} & Y, }$$
where $f$ satisfies $(WC)$ and $g$ satisfies $(F)$; we wish to prove that there
exists a dotted arrow $h$ as indicated, rendering the diagram commutative.
To prove this, we will construct a compatible family of natural transformations
$\{ h_{\beta}: B| \calJ_{\beta} \rightarrow X| \calJ_{\beta} \}_{ \beta < \alpha}$
which render the diagrams
$$ \xymatrix{ A | \calJ_{\beta} \ar[r] \ar[d] & X| \calJ_{\beta} \ar[d]^{g} \\
B| \calJ_{\beta} \ar[r] \ar[ur]^{h_{\beta}} & Y|\calJ_{\beta} }$$
commutative. Suppose that $\calJ_{\beta}$ is obtained from
$\calJ_{< \beta}$ by adjoining a single new object $J$. Assume that the maps
$\{ h_{\gamma} \}_{\gamma < \beta}$ have already been constructed, and can be amalgamated to a single natural transformation $h_{< \beta}: B | \calJ_{< \beta}
\rightarrow X| \calJ_{< \beta}$. Using Corollary \ref{stapler}, we see that extending
$h_{< \beta}$ to a map $h_{\beta}$ with the desired properties is equivalent to solving a lifting problem of the kind depicted in the following diagram:
$$ \xymatrix{ L_{J}(B) \coprod_{ L_{J}(A)} A(J) \ar[d]^{f'} \ar[r] & X(J) \ar[d]^{g'} \\
B(J) \ar@{-->}[ur] \ar[r] & Y(J) \times_{ M_J(Y)} M_J(X). }$$
Since our assumptions guarantee that $f'$ is a trivial cofibration and that $g'$ is a fibration, 
this lifting problem has a solution as desired.
\end{proof}

\begin{example}\label{tetsu}
Let $\bfA$ be the category of {\em bisimplicial} sets, which we will identify
with $\Fun( \cDelta^{op}, \sSet)$ and endow with the Reedy model structure.
It follows from Example \ref{croupus} that a morphism $f: X \rightarrow Y$ of bisimplicial sets is a Reedy cofibration if and only if it is a monomorphism. Consequently, the
Reedy model structure on $\bfA$ coincides with the injective model structure on $\bfA$.
\end{example}

\begin{example}\label{sued}
Let $\calJ$ be a Reedy category with $\calJ^{L} = \calJ$, and let
$\bfA$ be a model category. Then the weak equivalences and cofibrations of the
Reedy model structure (Proposition \ref{reedmod}) are the injective cofibrations and the
weak equivalences appearing in Definition \ref{cooper}. It follows that the Reedy model structure on $\Fun(\calJ, \bfA)$ coincides with the injective model structure of Proposition \ref{smurff} (in particular, the injective model structure is well defined in this case even without the assumption that $\bfA$ is combinatorial). Similarly, if $\calJ^{R} = \calJ$, then we can identify the Reedy model structure on $\Fun(\calJ, \bfA)$ with the projective model structure of Proposition \ref{smurff}.

In the general case, we can regard the Reedy model structure on $\Fun(\calJ, \bfA)$ as
a mixture of the projective and injective model structures. More precisely, we have the following:
\begin{itemize}
\item[$(i)$] A natural transformation $F \rightarrow G$ in $\Fun(\calJ, \bfA)$ satisfies
condition $(C)$ of Proposition \ref{reedmod} if and only if the induced transformation
$F|\calJ^{R} \rightarrow G| \calJ^{R}$ is a projective cofibration in $\Fun(\calJ^{R}, \bfA)$.
\item[$(ii)$] A natural transformation $F \rightarrow G$ in $\Fun(\calJ, \bfA)$ satisfies
condition $(F)$ of Proposition \ref{reedmod} if and only if the induced transformation
$F|\calJ^{L} \rightarrow G| \calJ^{L}$ is a injective fibration in $\Fun(\calJ^{L}, \bfA)$.
\item[$(iii)$] A natural transformation $F \rightarrow G$ in $\Fun(\calJ, \bfA)$ satisfies
condition $(WC)$ of Proposition \ref{reedmod} if and only if the induced transformation
$F|\calJ^{R} \rightarrow G| \calJ^{R}$ is a trivial projective cofibration in $\Fun(\calJ^{R}, \bfA)$.
\item[$(iv)$] A natural transformation $F \rightarrow G$ in $\Fun(\calJ, \bfA)$ satisfies
condition $(WF)$ of Proposition \ref{reedmod} if and only if the induced transformation
$F|\calJ^{L} \rightarrow G| \calJ^{L}$ is a trivial injective fibration in $\Fun(\calJ^{L}, \bfA)$.
\end{itemize}
\end{example}

\begin{remark}
Let $\calJ$ be a Reedy category and $\bfA$ a combinatorial model category, so that
the injective and projective model structures on $\Fun( \calJ, \bfA)$ are well-defined.
The identity functor from $\Fun(\calJ, \bfA)$ to itself can be regarded as a left Quillen equivalence from the projective model structure to the Reedy model structure, and from the
Reedy model structure to the injective model structure.
\end{remark}

\begin{corollary}\label{dimbu}
Let $\calC$ be a small category. Suppose that there exists a well-ordering
$\leq$ on the collection of objects of $\calC$ satisfying the following condition:
for every pair of objects $X,Y \in \calC$, we have
$$ \Hom_{\calC}(X,Y) = \begin{cases} \emptyset & \text{ if } $X > Y$ \\
\{ \id_X \} & \text{ if } X = Y. \end{cases}$$
Let $\bfA$ be a model category. Then:
\begin{itemize}
\item[$(i)$] A natural transformation $F \rightarrow G$ in $\Fun(\calC, \bfA)$ is a (trivial) projective cofibration if and only if, for every object $C \in \calC$, the induced map
$$ F(C) \coprod_{ \varinjlim_{ D \rightarrow C, D \neq C} F(D) }
\varinjlim_{ D \rightarrow C, D \neq C} G(D) \rightarrow G(C)$$
is a (trivial) cofibration in $\bfA$.
\item[$(ii)$] A natural transformation $F \rightarrow G$ in $\Fun(\calC^{op}, \bfA)$ is a
(trivial) injective fibration if and only if, for every object $C \in \calC$, the induced map
$$ F(C) \rightarrow G(C) \times_{ \varprojlim_{D \rightarrow C, D \neq C} G(D) }
\varprojlim_{D \rightarrow C, D \neq C} F(D) $$
is a (trivial) fibration in $\bfA$.
\end{itemize}
\end{corollary}

\begin{proof}
Combine Example \ref{sued} with Proposition \ref{reedmod}.
\end{proof}

\begin{corollary}\label{jonnyt}
Let $\bfA$ be a model category, let $\alpha$ be an ordinal, and let
$(\alpha)$ denote the linearly ordered set $\{ \beta < \alpha \}$, regarded as a category. Then:
\begin{itemize}
\item[$(1)$] Let $F \rightarrow F'$ be a natural transformation of diagrams
$(\alpha) \rightarrow \bfA$. Suppose that, for each $\beta < \alpha$, the maps
$$ \colim_{\gamma < \beta} F(\gamma) \rightarrow F(\beta)$$
$$ \colim_{\gamma < \beta} F'(\gamma) \rightarrow F'(\beta)$$
are cofibrations, while the map $F(\beta) \rightarrow F'(\beta)$ is a weak equivalence.
Then the induced map $$\colim_{\gamma < \alpha} F(\gamma) \rightarrow \colim_{\gamma < \alpha} F'(\gamma)$$ is a weak equivalence.
\item[$(2)$] Let $G \rightarrow G'$ be a natural transformation of diagrams
$(\alpha)^{op} \rightarrow \bfA$. Suppose that, for each $\beta < \alpha$, the maps
$$ G(\beta) \rightarrow \lim_{\gamma < \beta} G(\gamma)$$
$$ G'(\beta) \rightarrow \lim_{\gamma < \beta} G'(\gamma)$$
are fibrations, while the map $G(\beta) \rightarrow G'(\beta)$ is a weak equivalence.
Then the induced map $$\lim_{\gamma < \alpha} G(\gamma) \rightarrow \lim_{\gamma < \alpha} G'(\gamma)$$ is a weak equivalence.
\end{itemize}
\end{corollary}

\begin{proof}
We will prove $(1)$; $(2)$ follows by the same argument. Let $p: (\alpha) \rightarrow \ast$ be
the unique map, let $p^{\ast}: \bfA \rightarrow \bfA^{(\alpha)}$ be the diagonal map, and let
$p_{!}: \bfA^{(\alpha)} \rightarrow \bfA$ be a left adjoint to $p^{\ast}$. Then $p_{!}$ can be identified with the functor $F \mapsto \colim_{\gamma < \alpha} F(\gamma)$. We observe that $(p_!, p^{\ast})$ is a Quillen adjunction (where
$\bfA^{(\alpha)}$ is endowed with the projective model structure) so that
$p_{!}$ preserves weak equivalence between projectively cofibrant objects. The desired result now follows from Corollary \ref{dimbu}.
\end{proof}

Suppose that we are given a bifunctor
$$ \otimes: \bfA \times \bfB \rightarrow \bfC, $$
where $\bfC$ is a category which admits small limits. For any smallcategory $\calJ$, we define the
{\em coend} functor $\int_{\calJ}: \Fun(\calJ, \bfA) \times \Fun( \calJ^{op}, \bfB) \rightarrow \bfC$
so that the integral $\int_{\calJ}(F,G)$ is defined to be the coequalizer of the diagram
$$\xymatrix{ \coprod_{ J \rightarrow J'} F(J) \otimes G(J') 
\ar@<.4ex>[r] \ar@<-.4ex>[r] & \coprod_{J} F(J) \otimes G(J)}.$$
We then have the following result:

\begin{proposition}\label{intreed}
Let $\otimes: \bfA \times \bfB \rightarrow \bfC$ be a left Quillen bifunctor
(see Proposition \ref{biquill}), and let $\calJ$ be a Reedy category. Then the
coend functor
$$ \int_{\calJ}: \Fun( \calJ, \bfA) \times \Fun( \calJ^{op}, \bfB) \rightarrow \bfC$$
is also a left Quillen bifunctor, where we regard $\Fun(\calJ, \bfA)$ and
$\Fun(\calJ^{op}, \bfB)$ as endowed with the Reedy model structure.
\end{proposition}

\begin{proof}
Let $f: F \rightarrow F'$ be a Reedy cofibration in
$\Fun(\calJ, \bfA)$ and $g: G \rightarrow G'$ a Reedy cofibration in
$\Fun(\calJ^{op}, \bfB)$. Set $C = \int_{\calJ}(F,G') \coprod_{ \int_{\calJ}(F,G) } \int_{\calJ}(F',G) \in \bfC$, and $C' = \int_{\calJ}(F',G')$. We wish to show that the induced map
$C \rightarrow \int_{\calJ}(F',G')$ is a cofibration, which is trivial if either
$f$ or $g$ is trivial. 

Choose a good filtration $\{ \calJ_{\beta} \}_{\beta < \alpha}$ of $\calJ$.
For $\beta \leq \alpha$, we define
$$ C_{\beta} = \int_{ \calJ_{\beta}}( F|\calJ_{<\beta}, G'|\calJ_{<\beta})
\coprod_{ \int{\calJ_{\beta}}(F| \calJ_{<\beta}, G|\calJ_{<beta})} \int_{\calJ_{\beta}}( F'|\calJ_{<\beta}, G| \calJ_{<\beta})$$
$$ C'_{\beta} = \int_{\calJ_{\beta}}( F'| \calJ_{<\beta}, G'| \calJ_{<\beta}).$$
We wish to show that the map 
$$C_{\alpha} \simeq C_{\alpha} \coprod_{ C_0} C'_0 \rightarrow C_{\alpha}
\coprod_{ C_{\alpha} } C'_{\alpha}$$
is a cofibration (which is trivial if either $f$ or $g$ is trivial). We will prove more generally
that for $\delta \leq \gamma \leq \beta \leq \alpha$, the map
$$ \eta_{\delta, \gamma, \beta}: C_{\beta} \coprod_{ C_{\delta} } C'_{\delta} \rightarrow
C_{\beta} \coprod_{ C_{\gamma} } C'_{\gamma}$$
is a cofibration (trivial if either $f$ or $g$ is trivial).
The proof proceeds by induction on $\gamma$. If $\gamma$ is a limit ordinal,
then $\eta_{\delta, \gamma, \beta}$ can be obtained as a transfinite composition of the maps
$\{ \eta_{\epsilon, \epsilon+1, \beta} \}_{ \delta \leq \epsilon < \gamma}$, and the result follows from the inductive hypothesis. We may therefore assume that $\gamma = \gamma_0 +1$ is a successor ordinal. Since $\eta_{\delta, \gamma, \beta} = \eta_{ \gamma_0, \gamma, \beta} \circ \eta_{ \delta, \gamma_0, \beta}$, we can use the inductive hypothesis to reduce to the case where $\delta = \gamma_0$. Since $\eta_{\delta, \gamma, \beta}$ is a pushout of
$\eta_{\delta, \gamma, \gamma}$, we can assume also that $\beta = \gamma$. In other words, we are reduced to proving that the map
$$ h: C_{\gamma_0+1} \coprod_{ C_{\gamma_0} } C'_{\gamma_0} \rightarrow C'_{\gamma_0}$$
is a cofibration, which is trivial if either $f$ or $g$ is trivial. Let $J$ be the object of
$\calJ_{\gamma_0}$ which does not belong to $\calJ_{< \gamma_0}$. We now observe
that $h$ is a pushout of the evident map from
$$ ((F(J) \coprod_{L_J(F)} L_J(F') ) \otimes G'(J))
\coprod_{ (F(J) \coprod_{L_J(F)} L_J(F')) \otimes (G(J) \coprod_{L_J(G)} L_J(G'))}
(F'(J) \otimes (G(J) \coprod_{ L_J(G)} L_J(G'))$$
to $F'(J) \otimes G'(J),$
which is a cofibration (trivial if either $f$ or $g$ is trivial) by virtue of our assumptions on $f$, $g$, and the fact that $\otimes$ is a left Quillen bifunctor.
\end{proof}

\begin{remark}\label{cabler}
Proposition \ref{intreed} has an analogue for the model structures introduced in Proposition \ref{smurff}. That is, suppose that $\bfA$ and $\bfB$ are {\em combinatorial} model categories, and let $\calJ$ be an arbitrary small category. Then any left Quillen bifunctor
$\otimes: \bfA \times \bfB \rightarrow \bfC$ induces a left Quillen bifunctor
$$ \int_{ \calJ}: \Fun( \calJ, \bfA) \times \Fun( \calJ^{op}, \bfB) \rightarrow \bfC,$$
where we regard $\Fun(\calJ, \bfA)$ as endowed with the projective model structure
and $\Fun(\calJ^{op}, \bfB)$ with the injective model structure. To see this, we must show that 
for any projective cofibration $f: F \rightarrow F'$ in $\Fun(\calJ, \bfA)$ and any injective cofibration
$g: G \rightarrow G'$ in $\Fun(\calJ^{op}, \bfB)$, the induced map 
$$ h: \int_{\calJ}(F,G') \coprod_{ \int_{\calJ}(F,G) } \int_{\calJ}(F',G) \rightarrow \int_{\calJ}(F',G')$$
is a cofibration in $\bfC$, which is trivial if either $f$ or $g$ is trivial. Without loss of generality,
we may suppose that $f$ is a generating projective cofibration of the form
$\calF^{J}_{A} \rightarrow \calF^{J}_{A'}$ associated to an object $J \in \calJ$ and a
cofibration $i: A \rightarrow A'$ in $\bfA$, which is trivial if $f$ is trivial (see
the proof of Proposition \ref{smurff} for an explanation of this notation). Unwinding the definitions, we can identify $h$ with the map
$$ (A \otimes G'(J)) \coprod_{ A \otimes G(J) } (A' \otimes G(J)) \rightarrow
A' \otimes G'(J).$$
Since $i$ is a cofibration in $\bfA$ and the map $G(J) \rightarrow G'(J)$
is a cofibration in $\bfB$, we deduce that $h$ is a cofibration
in $\bfC$ (since $\otimes$ is a left Quillen bifunctor) which is trivial if either
$i$ or $h$ is trivial.
\end{remark}

\begin{example}\label{cabletome}
Let $\bfA$ be a simplicial model category, so that we have a left Quillen bifunctor
$$ \otimes: \bfA \times \sSet \rightarrow \bfA.$$
The coend construction determines a left Quillen bifunctor
$$ \int_{\cDelta}: \Fun( \cDelta, \bfA) \times \Fun( \cDelta^{op}, \sSet) \rightarrow \bfA.$$
where $\Fun(\cDelta, \bfA)$ and $\Fun( \cDelta^{op}, \sSet)$ are both endowed with the Reedy model structure. In particular, if we fix a cosimplicial object
$X^{\bigdot} \in \Fun(\cDelta, \bfA)$ which is Reedy cofibrant, then forming the coend
against $X^{\bigdot}$ determines a left Quillen functor from the category
of bisimplicial sets (with the Reedy model structure, which coincides with the
injective model structure by Example \ref{tetsu}) to $\bfA$.
\end{example}

\begin{example}\label{seventwo}
Let $\bfA$ be a simplicial model category, so that we have a left Quillen bifunctor
$$ \otimes: \bfA \times \sSet \rightarrow \bfA,$$
and consider the coend functor
$$ \int_{ \cDelta^{op} } \Fun( \cDelta^{op}, \bfA) \times \Fun( \cDelta, \sSet) \rightarrow \bfA.$$
Let $\Delta^{\bigdot} \in \Fun( \cDelta, \sSet)$ denote the standard simplex
(that is, the functor $[n] \mapsto \Delta^n$), and let ${\bf 1}$ denote the final object
of $\Fun(\cDelta, \sSet)$ (that is, the constant functor given by $[n] \mapsto \Delta^0$).
The unique map $\Delta^{\bigdot} \rightarrow {\bf 1}$ is a weak equivalence, 
and $\Delta^{\bigdot}$ is Reedy cofibrant: we may therefore regard $\Delta^{\bigdot}$ as a cofibrant replacement for the constant functor ${\bf 1}$.

The functor $X_{\bigdot} \mapsto \int_{ \cDelta^{op}}(X_{\bigdot}, {\bf 1})$ can be identified with the colimit functor $\Fun( \cDelta^{op}, \bfA) \rightarrow \bfA$. This is a left Quillen functor if
$\Fun( \cDelta^{op}, \bfA)$ is endowed with the projective model structure, but not the Reedy model structure. However, the {\it geometric realization} functor $X_{\bigdot} \mapsto |X_{\bigdot} | = \int_{ \cDelta^{op}}(X_{\bigdot}, \Delta^{\bigdot})$ is a left Quillen functor with respect to the Reedy model structure. 
\end{example}

\begin{corollary}\label{twinner}
Let $\bfA$ be a combinatorial simplicial model category, and let $X_{\bigdot}$ be a simplicial object
of $\bfA$. There is a canonical map
$$ \gamma: \hocolim X_{\bigdot} \rightarrow | X_{\bigdot} |$$
in the homotopy category of $\bfA$. This map is an equivalence if $X_{\bigdot}$ is Reedy cofibrant.
\end{corollary}

\begin{proof}
Let $\Delta^{\bigdot}$ and $\ast$ be the cosimplicial objects of $\sSet$ described in Example
\ref{seventwo}. Choose a weak equivalence of simplicial objects $X'_{\bigdot} \rightarrow X_{\bigdot}$, where $X'_{\bigdot}$ is projectively cofibrant. We then have a diagram
$$ \hocolim X_{\bigdot} \simeq \colim X'_{\bigdot}
\simeq \int_{ \cDelta^{op}}(X'_{\bigdot}, \ast)
\stackrel{\alpha}{\leftarrow} \int_{ \cDelta^{op}}( X'_{\bigdot}, \Delta^{\bigdot})
\stackrel{\beta}{\rightarrow} \int_{ \cDelta^{op}}( X_{\bigdot}, \Delta^{\bigdot}).$$
Since $X'_{\bigdot}$ is projectively cofibrant, Remark \ref{cabler} implies that the 
coend functor $\int_{\cDelta^{op}}( X'_{\bigdot}, \bigdot)$ preserves weak equivalences between
injectively cofibrant cosimplicial objects of $\sSet$; in particular, $\alpha$ is a weak equivalence in $\bfA$. This gives the desired map $\gamma$. Proposition \ref{intreed} implies that $\int_{ \cDelta^{op}}( \bigdot, \Delta^{\bigdot})$ preserves weak equivalences between Reedy cofibrant simplicial objects of $\bfA$, which proves that $\gamma$ is an isomorphism if $X_{\bigdot}$ is Reedy cofibrant.
\end{proof}

\begin{example}\label{swupt}
If $\bfA$ is the category of simplicial sets, then the map $\gamma$ of Corollary \ref{twinner}
is always an isomorphism; this follows from Example \ref{tetsu}. In other words, if
$X_{\bigdot, \bigdot}$ is a bisimplicial set, then we can identify the diagonal simplicial set
$[n] \mapsto X_{n,n}$ with the homotopy colimit of corresponding diagram
$\cDelta^{op} \rightarrow \sSet$.
\end{example}

\section{Simplicial Categories}\label{techapp}

Among the many different models for higher category theory, the theory of simplicial categories
is perhaps the most rigid. This can be either a curse or a blessing, depending on the situation. 
For the most part, we have chosen to use the less rigid theory of $\infty$-categories (see \S \ref{qqqc}) throughout this book. However, some arguments are substantially easier to carry out in the setting of simplicial categories. For this reason, we have devoted the final section of this appendix to giving a review of the theory of simplicial categories.

There exists a model structure on the category $\sCat$ of (small) simplicial categories, which was constructed by Bergner (\cite{bergner}). In \S \ref{compp4}, we will describe an analogous model structure on the category $\SCat$ of $\bfS$-enriched categories, where $\bfS$ is a suitable model category.
To formulate this generalization, we will need to employ the language of monoidal model categories, which we review in \S \ref{maymy}. Under mild assumptions on $\bfS$, one can show that a $\bfS$-enriched
category $\calC$ is fibrant if and only if each of the mapping objects $\bHom_{\calC}(X,Y)$ is a fibrant object of $\bfS$.\index{not}{CATsubS@$\SCat$}

In \S \ref{quasilimit3}, we will study the category $\bfA^{\calC}$ of diagrams $\calC \rightarrow \bfA$, where
$\calC$ is a small category and $\bfA$ a model category, both enriched over some
fixed model category $\bfS$. In the enriched setting we can again endow $\bfA^{\calC}$
with projective and injective model structures, which can be used to define homotopy limits and
colimits. 

Putting aside set-theoretic technicalities, every $\bfS$-enriched model category $\bfA$
gives rise to a fibrant object of $\SCat$: namely, the full subcategory
$\bfA^{\degree} \subseteq \bfA$ spanned by the fibrant-cofibrant objects. In \S \ref{pathspace}, we
will introduce a path object for $\bfA^{\degree}$, which will enable us to perform some calculations in
the homotopy category of $\SCat$. 

In \S \ref{hoco}, we will consider the problem of constructing homotopy colimits in
the category $\SCat$ of $\bfS$-enriched categories. Our main result, Theorem \ref{tubba}, asserts
that the formation of homotopy colimits in $\SCat$ is compatible with the formation of
(tensor) products in $\SCat$. We will apply this result in \S \ref{camper} to study the homotopy
theory of internal mapping objects in $\SCat$. 

We conclude this section with \S \ref{turka}, where we discuss localizations of (simplicial) model categories.

%For some purposes, the passage from
%$\bfA$ to $\bfA^{\degree}$ is inconvenient; we will therefore introduce in \S \ref{adjutatron} another model category $\wSCat$ which is Quillen equivalent to $\SCat$, which contains an object
%corresponding to $\bfA$ itself.\index{not}{wSCat@$\wSCat$}

\subsection{Enriched and Monoidal Model Categories}\label{maymy}

Many of the model categories which arise naturally are {\em enriched} over the category
of simplicial sets. Our goal in this section to study enrichments of one model category over another.

\begin{definition}\index{gen}{Quillen bifunctor}\index{gen}{left Quillen bifunctor}\label{biquill}
Let $\bfA$, $\bfB$, and $\bfC$ be model categories. We will say that a functor
$F: \bfA \times \bfB \rightarrow \bfC$ is a {\em left Quillen bifunctor} if the following conditions
are satisfied:
\begin{itemize}
\item[$(a)$] Let $i: A \rightarrow A'$ and $j: B \rightarrow B'$ be cofibrations in $\bfA$ and
$\bfB$, respectively. Then the induced map
$$i \wedge j: F(A',B) \coprod_{F(A,B)} F(A,B') \rightarrow F(A',B')$$
is a cofibration in $\bfC$. Moreover, if either $i$ or $j$ is a trivial cofibration, then
$i \wedge j$ is also a trivial cofibration.
\item[$(b)$] The functor $F$ preserves small colimits separately in each variable.
\end{itemize}
\end{definition}

\begin{definition}\index{gen}{monoidal model category}\index{gen}{model category!monoidal}
A {\it monoidal model category} is a monoidal category $\bfS$ equipped with a model structure, which satisfies the following conditions:
\begin{itemize}
\item[$(i)$] The tensor product functor $\otimes: \bfS \times \bfS \rightarrow \bfS$ is a left
Quillen bifunctor.
\item[$(ii)$] The unit object ${\bf 1} \in \bfS$ is cofibrant.
\item[$(iii)$] The monoidal structure on $\bfS$ is closed.
\end{itemize}
\end{definition}

\begin{remark}
Some authors demand only a weakened form of axiom $(ii)$ in the preceding definition.
\end{remark}

\begin{example}\label{shootset}
The category of simplicial sets $\sSet$ is a monoidal model category, with
respect to the Cartesian product and the Kan model structure defined in \S \ref{simpset}.
\end{example}

\begin{definition}\index{gen}{model category!simplicial}\index{gen}{simplicial model category}
\index{gen}{model category!$\bfS$-enriched}
Let $\bfS$ be a monoidal model category. A {\it $\bfS$-enriched model category}
is a $\bfS$-enriched category $\bfA$ equipped with a model structure satisfying the following conditions:
\begin{itemize}
\item[$(1)$] The category $\bfA$ is tensored and cotensored over $\bfS$.
\item[$(2)$] The tensor product $\otimes: \bfA \times \bfS \rightarrow \bfA$ is a left Quillen bifunctor.
\end{itemize}
In the special case where $\bfS$ is the category of simplicial sets (regarded
as a monoidal model category as in Example \ref{shootset}), we will simply refer to
$\bfA$ as a {\it simplicial model category}.
\end{definition}

\begin{remark}\label{cyclor}
An easy formal argument shows that condition $(2)$ is equivalent to either of the following
statements:
\begin{itemize}
\item[$(2')$] Given any cofibration $i: D \rightarrow D'$ in $\bfA$ and any fibration
$j: X \rightarrow Y$ in $\bfA$, the induced map
$$k: \bHom_{\bfA}(D',X) \rightarrow \bHom_{\bfA}(D,X) \times_{ \bHom_{\bfA}(D,Y)} \bHom_{\bfA}(D',Y)$$ is fibration in $\bfS$, which is trivial if either $i$ or $j$ is a weak equivalence.
\item[$(2'')$] Given any cofibration $i: C \rightarrow C'$ in $\bfS$ and any fibration $j: X \rightarrow Y$ in $\bfA$, the induced map $$k: X^{C'} \rightarrow X^C \times_{ Y^C} Y^{C'}$$ is a fibration in $\bfA$, which is trivial if either $i$ or $j$ is trivial.
\end{itemize}
\end{remark}

The following provides a criterion for detecting simplicial model structures:

\begin{proposition}\label{testsimpmodel}
Let $\calC$ be a simplicial category that is equipped with a model structure $($not assumed to be compatible with the simplicial structure on $\calC$$)$. Suppose that every object of $\calC$ is cofibrant  and that the collection of weak equivalences in $\calC$ is stable under filtered colimits. Then $\calC$ is a simplicial model category if and only if the following conditions are satisfied:

\begin{itemize}
\item[$(1)$] As a simplicial category, $\calC$ is both tensored and cotensored over $\sSet$. 
\item[$(2)$] Given a cofibration $i: A \rightarrow B$ and a fibration $p: X \rightarrow Y$ in $\calC$, the induced map of simplicial sets
$$ q: \bHom_{\calC}(B, X) \rightarrow \bHom_{\calC}(A,X) \times_{ \bHom_{\calC}(A,Y)} \bHom_{\calC}(B,Y) $$
is a Kan fibration. 
\item[$(3)$] For every $n \geq 0$ and every object $C$ in $\calC$, the natural map
$$ C \otimes \Delta^n \rightarrow C \otimes \Delta^0 \simeq C$$
is a weak equivalence in $\calC$.
\end{itemize}
\end{proposition}

\begin{proof}
Suppose first that $\calC$ is a simplicial model category. It is clear that $(1)$ and $(2)$ are satisfied. To prove $(3)$, we note that the projection $\Delta^n \rightarrow \Delta^0$ admits a section $s: \Delta^0 \rightarrow \Delta^n$ which is a trivial cofibration. If $\calC$ is a simplicial model category, then since $C$ is cofibrant it follows
that $C \otimes \Delta^0 \rightarrow C \otimes \Delta^n$ is a trivial cofibration, and in particular a weak equivalence. Thus the projection
$C \otimes \Delta^n \rightarrow C \otimes \Delta^0$ is a weak equivalence by the two-out-of-three property.

Now suppose that $(1)$, $(2)$, and $(3)$ are satisfied. We wish to show that $\calC$ is a simplicial model category. 
We first show that the bifunctor
$$ (C,K) \mapsto C \otimes K$$
preserves weak equivalences separately in each variable separately.

Fix the object $C \in \calC$, and suppose that $f: K \rightarrow K'$ is a weak homotopy equivalence of simplicial sets. Choose a cofibration $K \rightarrow K''$, where $K''$ is a contractible Kan complex. Then we may factor $f$ as a composition
$$ K \stackrel{f'}{\rightarrow} K \times K'' \stackrel{f''}{\rightarrow} K.$$
To prove that $\id_{C} \otimes f$ is a weak equivalence, it suffices to prove that
$\id_{C} \otimes f'$ and $\id_{C} \otimes f''$ are weak equivalences. Note that the map
$f''$ has a section $s$, which is a trivial cofibration. Thus, to prove that $\id_{C} \otimes f''$ is a weak equivalence, it suffices to show that $\id_{C} \otimes s$ is a weak equivalence. In other words, we may reduce to the case where $f$ is itself a trivial cofibration of simplicial sets.

Consider the collection $A$ of all monomorphisms $f: K \rightarrow K'$ of simplicial sets having the property that $\id_{C} \otimes f$ is a weak equivalence in $\calC$. It is easy to see that this collection of morphisms is weakly saturated. Thus, to prove that it contains all trivial cofibrations of simplicial sets, it suffices to show that every horn inclusion $\Lambda^n_i \rightarrow \Delta^n$
belongs to $A$. We prove this by induction on $n > 0$. Choose a vertex $v$ belonging to
$\Lambda^n_i$. We note that the inclusion $\{v\} \rightarrow \Lambda^n_i$ is a pushout of horn inclusions in dimensions $< n$; by the inductive hypothesis, this inclusion belongs to $A$.
Thus, it suffices to show that $\{v\} \rightarrow \Delta^n$ belongs to $A$, which is equivalent to assumption $(3)$.

Now let us show that for each simplicial set $K$, the functor
$$ C \mapsto C \otimes K$$ preserves weak equivalences. We will prove this by induction on the (possibly infinite) dimension of $K$. Choose a weak equivalence
$g: C \rightarrow C'$ in $\calC$. Let $S$ denote the collection of all simplicial subsets $L \subseteq K$ such that $g \otimes \id_{L}$ is a weak equivalence. We regard $S$ as a partially ordered set with respect to inclusions of simplicial subsets. Clearly $\emptyset \in S$. Since weak equivalences in $\calC$ are stable under filtered colimits, the supremum of every chain in $S$ belongs to $S$.
By Zorn's lemma, $S$ has a maximal element $L$. We wish to show that $L = K$. If not, 
we may choose some nondegenerate simplex $\sigma$ of $K$ which does not belong to $L$. Choose $\sigma$ of the smallest possible dimension, so that all of the faces of $\sigma$ belong to $L$. Thus, there is an inclusion $L' = L \coprod_{ \bd \sigma} \sigma \subseteq K$. Since
$\calC$ is left proper, assumption $(2)$ implies that the diagram
$$ \xymatrix{ D \otimes \bd \sigma \ar[r] \ar[d] & D \otimes \sigma \ar[d] \\
D \otimes L \ar[r] & D \otimes L' }$$
is a homotopy pushout, for every object $D \in \calC$. We observe that
$g \otimes \id_{L}$ is a weak equivalence by assumption, $g \otimes \id_{\bd \sigma}$ is
a weak equivalence by the inductive hypothesis (since $\bd \sigma$ has dimension smaller than the dimension of $K$), and $g \otimes \id_{\sigma}$ is a weak equivalence in virtue of assumption $(3)$ and the fact that $g$ is a weak equivalence. It follows that $g \otimes \id_{L'}$ is a weak equivalence, which contradicts the maximality of $L$. This completes the proof that the bifunctor
$\otimes: \calC \times \sSet \rightarrow \calC$ preserves weak equivalences separately in each variable.

Now suppose given a cofibration
$i: C \rightarrow C'$ in $\calC$ and another cofibration $j: S \rightarrow S'$ in $\sSet$.
We wish to prove that the induced map 
$$i \wedge j: (C \otimes S') \coprod_{C \otimes S} (C' \otimes S) \rightarrow C' \otimes S'$$
is a cofibration in $\calC$, which is trivial if either $i$ or $j$ is trivial. The first point follows immediately from $(2)$. For the triviality, we will assume that $i$ is a weak equivalence (the case where $j$ is a weak equivalence follows using the same argument).
Consider the diagram
$$ \xymatrix{ C \otimes S \ar[rr]^{i \otimes \id_{S}} \ar[d] & & C' \otimes S \ar[d] & \\
C \otimes S' \ar[rr]^-{f} & & (C' \otimes S) \coprod_{ C \otimes S} (C \otimes S') \ar[r] & C' \otimes S'.}$$
The arguments above show that $i \otimes \id_{S}$ and $i \otimes \id_{S'}$ are weak equivalences.
The square in the diagram is a homotopy pushout, so Proposition \ref{propob} implies that $f$ is a weak equivalence as well. Thus $i \wedge j$ is a weak equivalence, by the two-out-of-three property.
\end{proof}

If $\calC$ is a simplicial model category, then there is automatically a strong relationship between the homotopy theory of the underlying model category and the homotopy theory of the simplicial sets $\bHom_{\calC}(\bigdot, \bigdot)$. For example, we have the following:

\begin{remark}
Let $\calC$ be a simplicial model category, let $X$ be a cofibrant object of $\calC$, and let $Y$ be a fibrant object of $\calC$. The simplicial set $K=\bHom_{\calC}(X,Y)$ is a Kan complex; moreover, 
there is a canonical bijection
$$ \pi_0 K \simeq \Hom_{ \h{ \calC}}(X,Y).$$
\end{remark}

We conclude this section by studying a situation which will arise in \S \ref{chap4}. 
Let $\calC$ and $\calD$ be model categories enriched over another model category $\bfS$, and suppose given a Quillen adjunction $$\Adjoint{F}{\calC}{\calD}{G}$$
between the underlying model categories. We wish to study the situation where $G$ (but not $F$) has the structure of a $\bfS$-enriched functor. Thus, for every triple of objects
$X \in \calC$, $Y \in \calD$, $S \in \bfS$, we have a canonical map
\begin{eqnarray*}
\Hom_{\calC}( S \otimes X, GY) & \simeq & \Hom_{\bfS}( S, \bHom_{\calC}( X, GY) ) \\
& \rightarrow &  \Hom_{\bfS}( S, \bHom_{\calD}(FX, FGY) ) \\
& \simeq & \Hom_{\calD}( S \otimes FX, FGY) \\
& \rightarrow & \Hom_{\calD}( S \otimes FX, Y). \end{eqnarray*}
Taking $Y = F(S \otimes X)$ and applying this map to the unit of the adjunction between
$F$ and $G$, we obtain a map
$S \otimes FX \rightarrow F(S \otimes X)$, which we will denote by $\beta_{X,S}$.
The collection of maps $\beta_{X,S}$ is simply another way of encoding the data of $G$
as a $\bfS$-enriched functor. If the maps $\beta_{X,S}$ are isomorphisms, then
$F$ is again a $\bfS$-enriched functor, and $(F,G)$ is an adjunction between $\bfS$-enriched categories. We wish to study an analogous situation, where the maps $\beta_{X,S}$ are only assumed to be weak equivalences.

\begin{remark}\label{tuccan}
Suppose that $\bfS$ is the category $\sSet$ of simplicial sets, with its usual model structure.
Then the map $\beta_{X,S}$ is automatically a weak equivalence for every cofibrant object
$X \in \calC$. To prove this, we consider the collection $\calK$ of all simplicial sets
$S$ such that $\beta_{S,X}$ is an equivalence. It is not difficult to show that $\calK$ is
closed under weak equivalences, homotopy pushout squares, and coproducts.
Since $\Delta^0 \in \calK$, we conclude that $\calK = \sSet$.
\end{remark}

\begin{proposition}\label{weakcompatequiv}
Let $\calC$ and $\calD$ be $\bfS$-enriched model categories. Let
$\Adjoint{F}{\calC}{\calD}{G}$ be a Quillen adjunction between the underlying model categories.
Assume that every object of $\calC$ is cofibrant,
and that the map $\beta_{X,S}: S \otimes F(X) \rightarrow F(S \otimes X)$ is a weak equivalence for every pair of cofibrant objects $X \in \calC$, $S \in \bfS$. The following are equivalent:
\begin{itemize}
\item[$(1)$] The adjunction $(F,G)$ is a Quillen equivalence.
\item[$(2)$] The restriction of $G$ determines a weak equivalence of $\bfS$-enriched categories
$\calD^{\degree} \rightarrow \calC^{\degree}$ (see \S \ref{compp4}).
\end{itemize}
\end{proposition}

\begin{remark}
Strictly speaking, in \S \ref{compp4} we only define weak equivalences between {\em small} $\bfS$-enriched categories; however, the definition extends to large categories in an obvious way.
\end{remark}

\begin{proof}
Since $G$ preserves fibrant objects, and every object of $\calC$ is cofibrant, it is clear that
$G$ carries $\calD^{\degree}$ into $\calC^{\degree}$. Condition $(1)$
is equivalent to the assertion that for every pair of fibrant-cofibrant objects
$C \in \calC$, $D \in \calD$, a map
$g: C \rightarrow GD$ is a weak equivalence in $\calC$ if and only if the adjoint map
$f: FC \rightarrow D$ is a weak equivalence in $\calD$. Choose a factorization of $f'$ as a composition
$FC \stackrel{f'}{\rightarrow} D' \stackrel{f''}{\rightarrow} D,$
where $f'$ is a trivial cofibration and $f''$ is a fibration. By the two-out-of-three property,
$f$ is a weak equivalence if and only if $f''$ is a weak equivalence. We note that
$g$ admits an analogous factorization as
$$ C \stackrel{g'}{\rightarrow} GD' \stackrel{g''}{\rightarrow} GD.$$
Using $(2)$, we deduce that $f''$ is an equivalence in $\calD^{\degree}$ if and only if
$g''$ is an equivalence in $\calC^{\degree}$. It will therefore suffice to show that
$g'$ is an equivalence in $\calC^{\degree}$. For this, it will suffice to show that
$C$ and $GD'$ corepresent the same functor on the homotopy category $\h{\calC}$.
Invoking $(2)$ again, it will suffice to show that for every fibrant-cofibrant object
$D'' \in \calD$, the induced map
$$\Hom_{ \h{\calC} }( GD', GD'') \rightarrow \Hom_{\h{\calC}}( C, GD'')
\simeq \Hom_{\h{\calD}}( FC, D'')$$ is bijective. Using $(2)$, 
we deduce that map $\Hom_{\h{\calD}}(D',D'') \rightarrow \Hom_{\h{\calD}}( GD', GD'')$
is bijective. The desired result now follows from the fact that $f'$ is a weak equivalence in $\calD$.

We now show that $(1) \Rightarrow (2)$.
The $\bfS$-enriched functor $G^{\degree}: \calD^{\degree} \rightarrow \calC^{\degree}$ is essentially surjective, since the right derived functor $RG$ is essentially surjective on homotopy categories. It suffices to show that $G^{\degree}$ is fully faithful: in other words, that for every pair of fibrant-cofibrant objects $X,Y \in \calD$, the induced map
$$ i: \bHom_{\calD}(X,Y) \rightarrow \bHom_{\calC}(G(X),G(Y))$$ 
is a weak equivalence in $\bfS$.

Since the left derived functor $LF$ is essentially surjective, there exists an object $X' \in \calC$
and a weak equivalence $FX' \rightarrow X$. We may regard $X$ as a fibrant replacement for $FX'$ in $\calD$; it follows that the adjoint map $X' \rightarrow GX$ may be identified with the
adjunction $X' \rightarrow (RG \circ LF) X'$, and is therefore a weak equivalence by $(1)$. Thus we have a diagram
$$ \xymatrix{ \bHom_{\calD}(X,Y) \ar[d] \ar[r]^{i} & \bHom_{\calC}(G(X),G(Y)) \ar[d] \\
\bHom_{\calD}(F(X'), Y) \ar[r]^{i'} & \bHom_{\calC}(X',G(Y)) }$$
in which the vertical arrows are homotopy equivalences; thus, to show that $i$ is a weak equivalence, it suffices to show that $i'$ is a weak equivalence. For this, it suffices to show that $i'$ induces a bijection from $[S, \bHom_{\calD}(F(X'), Y) ]$ to $[S, \bHom_{\calC}(X',G(Y))]$, for every cofibrant object $S \in \bfS$; here $[S,K]$ denotes the set of homotopy classes of maps from $S$ into $K$ in
the homotopy category $\h{\bfS}$. But we may rewrite this map of sets as
$$i'_S:  \bHom_{\h{\calD}}( F(X') \otimes S, Y) \rightarrow \bHom_{\h{\calC}}( X' \otimes S, G(Y) )
= \bHom_{\h{\calD}}( F( X' \otimes S), Y),$$
and it is given by composition with $\beta_{X',S}$. (Here $\h{\calC}$ and $\h{\calD}$ denote the
homotopy categories of $\calC$ and $\calD$ as {\em model categories}; these are equivalent
to the corresponding homotopy categories of $\calC^{\degree}$ and $\calD^{\degree}$ as $\bfS$-enriched categories). Since $\beta_{X',S}$ is an isomorphism in the homotopy category $\h{\calD}$, the map $i'_S$ is bijective and $(2)$ holds, as desired.
\end{proof}

\begin{corollary}\label{urchug}
Let $\Adjoint{F}{\calC}{\calD}{G}$
be a Quillen equivalence between simplicial model categories, where every object of $\calC$ is cofibrant. Suppose that $G$ is a simplicial functor. Then $G$ induces an equivalence of $\infty$-categories $\Nerve( \calD^{\degree}) \rightarrow \Nerve( \calC^{\degree})$.
\end{corollary}

\subsection{The Model Structure on $\bfS$-Enriched Categories}\label{compp4}

Throughout this section, we will fix a symmetric monoidal model category $\bfS$, and
study the category of $\bfS$-enriched categories. The main case of interest to us is that in which $\bfS$ is the category of simplicial sets (with its usual model structure and the Cartesian monoidal structure). However, the treatment of the general case requires little additional effort, and there are a number of other examples which arise naturally in other contexts:

\begin{itemize}
\item[$(i)$] The category $\sSet$ of simplicial sets, equipped with the Cartesian monoidal structure and the {\em Joyal} model structure defined in \S \ref{compp3}.
\item[$(ii)$] The category of complexes
$$ \ldots \rightarrow M_{n} \rightarrow M_{n} \rightarrow M_{n-1} \rightarrow \ldots, $$
of vector spaces over a field $k$, with its usual model structure (in which weak equivalences are quasi-isomorphisms, fibrations are epimorphisms, and cofibrations are monomorphisms) and monoidal structure given by the formation of tensor products of complexes.
\end{itemize}

Let $\bfS$ be an monoidal model category, and let $\SCat$ denote the category of (small) $\bfS$-enriched categories, in which morphisms are
given by $\bfS$-enriched functors. The goal of this section is to describe a model structure on $\SCat$. 
We first note that the monoidal structure on $\bfS$ induces a monoidal structure on its
homotopy category $\h{\bfS}$, which is determined up to (unique) isomorphism by the requirement that there exist a monoidal structure on the functor
$$ \bfS \rightarrow \h{\bfS} $$
given by inverting all weak equivalences. Consequently, 
we note that any $\bfS$-enriched category $\calC$ gives rise to an $\h{\bfS}$-enriched category $\h{\calC}$, having the same objects as $\calC$ and where mapping spaces are given by
$$ \bHom_{\h{\calC}}(X,Y) = [ \bHom_{\calC}(X,Y) ].$$
Here we let $[K]$ denote the image in $\h{\bfS}$ of an object $K \in \bfS$. We will refer to
$\h{\calC}$ as the {\it homotopy category} of $\calC$; the passage from
$\calC$ to $\h{\calC}$ is a special case of Remark \ref{laxcon}.\index{not}{hcalC@$\h{\calC}$}
\index{gen}{homotopy category!of a $\bfS$-enriched category}

\begin{definition}\label{equivdefequiv}\index{gen}{equivalence!of $\bfS$-enriched categories}
Let $\bfS$ be an monoidal model category.
We say that a functor $F: \calC \rightarrow \calC'$ in $\SCat$ is a {\it weak equivalence}
if the induced functor $\h{\calC} \rightarrow \h{\calC'}$ is an equivalence of $\h{\bfS}$-enriched categories. In other words, $F$ is a weak equivalence if and only if:

\begin{itemize}
\item[$(1)$] For every pair of objects $X,Y \in \calC$, the induced map
$$ \bHom_{\calC}(X,Y) \rightarrow \bHom_{\calC'}(F(X),F(Y))$$ is a weak equivalence in $\bfS$.
\item[$(2)$] Every object $Y \in \calC'$ is equivalent to $F(X)$ in the homotopy category $\h{\calC'}$, for some $X \in \calC$.
\end{itemize}
\end{definition}

\begin{remark}
If $\bfS$ is the category $\sSet$ (endowed with the Kan model structure), then Definition \ref{equivdefequiv} reduces to the definition given in \S \ref{stronghcat}.
\end{remark}

\begin{remark}\label{swunk}
Suppose that the collection of  weak equivalences in $\bfS$ is stable under filtered colimits.
Then it is easy to see that the collection of weak equivalences in $\SCat$ is also stable under filtered colimits. If $\bfS$ is also a combinatorial model category, then a bit more effort shows that
the class of weak equivalences in $\SCat$ is perfect, in the sense of Definition \ref{perfequiv}.
\end{remark}

We now introduce a bit of notation for working with $\bfS$-enriched categories. If $A$ is an object of $\bfS$, we will let $[1]_{A}$ denote the $\bfS$-enriched category having two objects $X$ and $Y$, with
$$ \bHom_{[1]_{A}}(Z,Z') = \begin{cases} {\bf 1}_{\bfS} & \text{if } Z=Z'=X \\
{\bf 1}_{\bfS} & \text{if } Z=Z'=Y \\
A & \text{if } Z=X, Z'=Y \\
\emptyset & \text{if } Z=Y, Z'=X.\end{cases}$$
Here $\emptyset$ denotes the initial object of $\bfS$ and ${\bf 1}_{\bfS}$ denotes the
unit object with respect to the monoidal structure on $\bfS$. We will denote
$[1]_{ {\bf 1}_{\bfS} }$ simply by $[1]_{\bfS}$. 
We let $[0]_{\bfS}$ denote the $\bfS$-enriched category having only a single object $X$, with $\bHom_{\ast}(X,X)={\bf 1}_{\bfS}$.\index{not}{1subS@$[1]_{A}$}\index{not}{1bfs@$[1]_{\bfS}$}
\index{not}{0bfs@$[0]_{\bfS}$}

We let $C_0$ denote the collection of all morphisms in $\bfS$ of the following types:
\begin{itemize}
\item[$(i)$] The inclusion $\emptyset \hookrightarrow [0]_{\bfS}$.
\item[$(ii)$] The induced maps $[1]_{S} \rightarrow [1]_{S'}$, where
$S \rightarrow S'$ ranges over a set of generators for the weakly saturated class of
cofibrations in $\bfS$.
\end{itemize}

\begin{proposition}\label{enrichcatper}\index{gen}{model category!of simplicial categories}
Let $\bfS$ be a combinatorial monoidal model category. Assume that every object of
$\bfS$ is cofibrant, and that the collection of weak equivalences in $\bfS$ is stable
under filtered colimits. Then
there exists a left proper, combinatorial model structure on $\SCat$, characterized by the following
conditions:
\begin{itemize}
\item[$(C)$] The class of cofibrations in $\SCat$ is the smallest weakly saturated class of morphisms
containing the set of morphisms $C_0$ appearing above.
\item[$(W)$] The weak equivalences in $\SCat$ are defined as in \S \ref{equivdefequiv}.
\end{itemize}
\end{proposition}

\begin{proof}
It suffices to verify the hypotheses of Proposition \ref{goot}. Condition $(1)$ follows from
Remark \ref{swunk}. For condition $(3)$, we must show that any functor $F: \calC \rightarrow \calC'$ having the right lifting property with respect to all morphisms in $C_0$ is a weak equivalence. Since $F$ has the right lifting property with respect to $i: \emptyset \rightarrow [0]_{\bfS}$, it is surjective on objects and therefore essentially surjective. The assumption that $F$ has the right lifting property with respect to the remaining morphisms of $C_0$ guarantees that for every $X, Y \in \calC$, the induced map
$$\bHom_{\calC}(X,Y) \rightarrow \bHom_{\calC'}(F(X),F(Y))$$ is a trivial fibration in
$\bfS$, and therefore a weak equivalence.

It remains to verify condition $(2)$: namely, that the class of weak equivalences is stable under pushout by the elements of $C_0$. We must show that given any pair of functors
$F: \calC \rightarrow \calD$, $G: \calC \rightarrow \calC'$ with $F$ a weak equivalence and $G$ a pushout of some morphism in $C_0$, the induced map $F': \calC' \rightarrow \calD' = \calD \coprod_{\calC} \calC'$ is a weak equivalence. There are two cases to consider.

First, suppose that $G$ is a pushout of the generating cofibration $i: \emptyset \rightarrow \ast$. In other words, the category $\calC'$ is obtained from $\calC$ by adjoining a new object $X$, which admits no morphisms to or from the objects of $\calC$ (and no endomorphisms other than the identity). The category $\calD'$ is obtained from $\calD$ by adjoining $X$ in the same fashion. It is easy to see that if $F$ is a weak equivalence, then $F'$ is also a weak equivalence.

The other basic case to consider is one in which $G$ is a pushout of one of the generating cofibrations $[1]_{S} \rightarrow [1]_{T}$, where $S \rightarrow T$ is a cofibration in $\bfS$. Let $H: [1]_{S} \rightarrow \calC$
denote the ``attaching map'', so that $H$ is determined by a pair of objects $x= H(X)$ and $y=H(Y)$
and a map of $h: S \rightarrow \bHom_{\calC}(x,y)$. 
By definition, $\calC'$ is universal
with respect to the property that it receives a map from $\calC$, and the map $h$ extends to a map
$\widetilde{h}: T \rightarrow \bHom_{\calC'}(x,y)$. To carry out the proof, we will give an explicit construction of a $\bfS$-enriched category $\calC'$ which has this universal property. 

For the remainder of the proof, we will assume that $\bfS$ is the category of simplicial sets.
This is purely for notational convenience; the same arguments can be employed without
change in the general case.

We begin by declaring that the objects of $\calC'$ are the objects of $\calC$. 
The definition of the morphisms in $\calC'$ is a bit more complicated. Let $w$ and $z$ be objects of $\calC$. We define a sequence of simplicial sets $M_{\calC'}^{k}$ as follows:
$$M_{\calC}^0 = \bHom_{\calC}(w,z)$$
$$M_{\calC}^1 = \bHom_{\calC}(y,z) \times T \times \bHom_{\calC}(w,x)$$
$$M_{\calC}^2 = \bHom_{\calC}(y,z) \times T \times \bHom_{\calC}(y,x) \times
T \times \bHom_{\calC}(w,x)$$ 
and so forth. More specifically, for $k \geq 1$, the $m$-simplices of $M_{\calC}^k$ are finite sequences
$$( \sigma_0, \tau_1, \sigma_1, \tau_2, \ldots, \tau_k, \sigma_k)$$ where 
$\sigma_0 \in \bHom_{\calC}(y,z)_m$, $\sigma_k \in \bHom_{\calC}(w,x)_m$, $\sigma_i \in \bHom_{\calC}(y,x)_m$ for $0 < i < k$, and $\tau_i \in T_m$ for $1 \leq i \leq k$.

We define $\bHom_{\calC'}(w,z)$ to be the quotient of the disjoint union
$ \coprod_{k} M_{\calC}^k$ by the equivalence relation which is generated by making the identification
$$( \sigma_0, \tau_1, \ldots, \sigma_k) \simeq (\sigma_0, \tau_1, \ldots, \tau_{j-1}, \sigma_{j-1} \circ h(\tau_j) \circ \sigma_{j}, \tau_{j+1}, \ldots, \sigma_k )$$ whenever the simplex $\tau_j$ belongs to $S_m \subseteq T_m$.

We equip $\calC'$ with an associative composition law, which is given on the level of simplices
by $$( \sigma_0, \tau_1, \ldots, \sigma_k ) \circ (\sigma'_0, \tau'_1, \ldots, \sigma'_l) =
(\sigma_0, \tau_1, \ldots, \tau_k, \sigma_k \circ \sigma'_0, \tau'_1, \ldots, \sigma'_l).$$
It is easy to verify that this composition law is well-defined (that is, compatible with the equivalence relation introduced above), associative, and that the identification $M_{\calC}^0 = \bHom_{\calC}(w,z)$ gives rise to an inclusion of categories $\calC \subseteq \calC'$. Moreover, the map 
$h: S \rightarrow \bHom_{\calC}(x,y)$ extends to $\widetilde{h}: T \rightarrow \bHom_{\calC'}(x,y)$, given by the composition
$$ T \simeq \{ \id_y \} \times T \times \{ \id_x\} \subseteq \bHom_{\calC}(y,y) \times T \times \bHom_{\calC}(x,x) = M_{\calC}^1 \rightarrow \bHom_{\calC'}(x,y).$$
Moreover, it is not difficult to see that $\calC'$ has the desired universal property.

We observe that, by construction, the simplicial sets $\bHom_{\calC'}(w,z)$ come equipped with a natural filtration. Namely, define $\bHom_{\calC'}(w,z)^k$ to be the image of
$$ \coprod_{0 \leq i \leq k} M_{\calC}^i $$ in $\bHom_{\calC'}(w,z)$. Then we have
$$ \bHom_{\calC}(w,z) = \bHom_{\calC'}(w,z)^0 \subseteq \bHom_{\calC'}(w,z)^1 \subseteq \ldots$$
and $\bigcup_{k} \bHom_{\calC'}(w,z)^k = \bHom_{\calC'}(w,z)$. Moreover, the inclusion
$ \bHom_{\calC'}(w,z)^k \subseteq \bHom_{\calC'}(w,z)^{k+1}$ is a pushout of the inclusion
$N_{\calC}^{k+1} \subseteq M_{\calC}^{k+1}$, where $N^{k+1}$ is the simplicial subset of $M_{\calC}^{k+1}$ whose
$m$-simplices consist of those $(2m+1)$-tuples $(\sigma_0, \tau_1, \ldots, \sigma_m)$ such that
$\tau_i \in S_m$ for at least one value of $i$.

Let us now return to the problem at hand: namely, we wish to prove that $F': \calC' \rightarrow \calD'$ is an equivalence. We note that the construction outlined above may also be employed to produce a model for $\calD'$, and an analogous filtration on its morphism spaces.

Since $G': \calD \rightarrow \calD'$ and $F: \calC \rightarrow \calD$ are essentially surjective, we deduce that $F'$ is essentially surjective. Hence it will suffice to show that, for any objects $w,z \in \calC'$, the induced map $$\phi: \bHom_{\calC'}(w,z) \rightarrow \bHom_{\calD'}(w,z)$$ is a weak homotopy equivalence.
For this, it will suffice to show that for each $i \geq 0$, the induced map $\phi_i: \bHom_{\calC'}(w,z)^i \rightarrow \bHom_{\calD'}(w,z)^i$ is a weak homotopy equivalence; then $\phi$, being a filtered colimit of weak homotopy equivalences $\phi_i$, will itself be a weak homotopy equivalence.

The proof now proceeds by induction on $i$. When $i=0$, $\phi_i$ is a weak homotopy equivalence by assumption (since $F$ is an equivalence of simplicial categories). For the inductive step, we note that
$\phi_{i+1}$ is obtained as a pushout
$$ \bHom_{\calC'}(w,z)^i \coprod_{ N^{i+1}_{\calC}} M^{i+1}_{\calC} \rightarrow \bHom_{\calD'}(w,z)^i \coprod_{ N^{i+1}_{\calD} } M^{i+1}_{\calD}.$$
Since $\bfS$ is left-proper, both of these pushouts are homotopy pushouts. Consequently, to show that $\phi_{i+1}$ is a weak equivalence, it suffices to show that $\phi_i$ is a weak equivalence and the each of the maps
$$ N^{i+1}_{\calC} \rightarrow N^{i+1}_{\calD}$$
$$ M^{i+1}_{\calC} \rightarrow M^{i+1}_{\calD}$$
are weak equivalences. These statements follow easily from the compatibility of the monoidal
structure of $\bfS$ with the model structure, and the assumption that every object of $\bfS$ is cofibrant.
\end{proof}

\begin{remark}
It follows from the proof of Proposition \ref{enrichcatper} that if
$f: \calC \rightarrow \calC'$ is a cofibration of $\bfS$-enriched categories, then
the induced map $\bHom_{\calC}(X,Y) \rightarrow \bHom_{\calC'}(fX,fY)$ is a cofibration
for every pair of objects $X,Y \in \calC$.
\end{remark}

\begin{remark}\label{cuttup}
The model structure of Proposition \ref{enrichcatper} enjoys the following functoriality:
suppose that $f: \bfS \rightarrow \bfS'$ is a monoidal left Quillen functor between
model categories satisfying the hypotheses of Proposition \ref{enrichcatper}, with right adjoint $g: \bfS' \rightarrow \bfS$. Then $f$ and $g$ induce a Quillen adjunction
$$ \Adjoint{ F}{\SCat}{\Cat_{\bfS'},}{G}$$
where $F$ and $G$ are as in Remark \ref{laxcon}. Moreover, if $(f,g)$ is a Quillen equivalence,
then $(F,G)$ is likewise a Quillen equivalence.
\end{remark}

In order for Proposition \ref{enrichcatper} to be useful in practice, we need to understand the fibrations in $\SCat$. For this, we first introduce a few definitions.

\begin{definition}\label{qfibb}
Let $F: \calC \rightarrow \calD$ be a functor between ordinary categories. We will say that
$F$ is a {\it quasi-fibration} if, for every object $C \in \calC$ and every isomorphism
$f: F(X) \rightarrow Y$ in $\calD$, there exists an isomorphism $\overline{f}: X \rightarrow \overline{Y}$ in $\calC$ such that $F( \overline{f} ) = f$.
\end{definition}

\begin{remark}
The relevance of Definition \ref{qfibb} is as follows: the category $\Cat$ admits a model structure
in which the weak equivalences are the equivalences of categories, and the fibrations are the quasi-fibrations. This is a special case of Theorem \ref{staycode}, which we will prove below
(namely, the special case where we take $\bfS = \Set$, endowed with the trivial model
structure of Example \ref{trivmodel}). 
\end{remark}

\begin{definition}\label{twurp}
Let $\bfS$ be a monoidal model category, and let $\calC$ be a
$\bfS$-enriched category. We will say that a morphism $f$
in $\calC$ is an {\it equivalence} if the homotopy class $[f]$
of $f$ is an isomorphism in $\h{\calC}$.\index{gen}{equivalence!in a $\bfS$-enriched category}

We will say that $\calC$ is {\it locally fibrant} if, for every pair of objects
$X,Y \in \calC$, the mapping space $\bHom_{\calC}(X,Y)$ is a fibrant object
of $\bfS$.\index{gen}{fibrant!locally}\index{gen}{locally fibrant}

We will say that a $\bfS$-enriched functor $F: \calC \rightarrow \calC'$ is a
{\it local fibration}\index{gen}{fibration!local} if the following conditions are satisfied:
\begin{itemize}
\item[$(i)$] For every pair of objects
$X,Y \in \calC$, the induced map $\bHom_{\calC}(X,Y) \rightarrow
\bHom_{\calC'}(FX,FY)$ is a fibration in $\bfS$. 
\item[$(ii)$] The induced map $\h{\calC} \rightarrow \h{\calC'}$ is a quasi-fibration
of categories.
\end{itemize}
\end{definition}

\begin{remark}\label{dinty}
Let $F: \calC \rightarrow \calC'$ be a functor between $\bfS$-enriched categories
which satisfies condition $(i)$ of Definition \ref{twurp}. Let $X \in \calC$
and $Y \in \calC'$ be objects. If $\calC'$
is locally fibrant, then every morphism $[f]: F(X) \rightarrow Y$ in
$\h{\calC'}$ can be represented by an equivalence $f: F(X) \rightarrow Y$ in $\calC'$.
Let $\overline{Y}$ be an object of $\calC$ such that $F( \overline{Y} = Y$. Since
${\bf 1}_{\bfS}$ is a cofibrant object of $\bfS$ and the map
$\bHom_{\calC}( X, \overline{Y} ) \rightarrow \bHom_{\calC}( F(X), Y)$
is a fibration, Proposition \ref{princex} implies that $[f]$ can be lifted to
an isomorphism $[\overline{f}]: X \rightarrow \overline{Y}$ in $\h{\calC}$ if
and only if $f$ can be lifted to an equivalence $\overline{f}: X \rightarrow \overline{Y}$
in $\calC$. Consequently, when $\h{\calC}$ is locally fibrant,
condition $(ii)$ is equivalent to the following analogous assertion:
\begin{itemize}
\item[$(ii')$] For every equivalence $f: F(X) \rightarrow Y$ in $\calC'$, there
exists an equivalence $\overline{f}: X \rightarrow \overline{Y}$ in $\calC$
such that $F( \overline{f} ) = f$.
\end{itemize}
\end{remark}

\begin{notation}\index{not}{[1]simS@$[1]^{\sim}_{\bfS}$}
We $[1]^{\sim}_{\bfS}$ denote the $\bfS$-enriched category
containing a pair of objects $X,Y$, with
$$ \bHom_{ [1]^{\sim}_{\bfS} }(Z, Z') = {\bf 1}_{\bfS}$$
for all $Z, Z' \in \{X, Y \}$. 
\end{notation}

\begin{definition}[Invertibility Hypothesis]\index{gen}{invertibility hypothesis}\label{inhyp}
Let $\bfS$ be a monoidal model category satisfying the hypotheses of Proposition
\ref{enrichcatper}. We will say that
$\bfS$ {\em satisfies the invertibility hypothesis} if the following condition is satisfied:
\begin{itemize}
\item[$(\ast)$] Let $i: [1]_{\bfS} \rightarrow \calC$ be a cofibration of $\bfS$-enriched
categories, classifying a morphism $f$ in $\calC$ which is invertible in the homotopy category $\h{\calC}$, and form a pushout diagram
$$ \xymatrix{ [1]_{\bfS} \ar[r]^{i} \ar[d] & \calC \ar[d]^{j} \\
[1]_{\bfS}^{\sim} \ar[r] & \calC\langle f^{-1} \rangle }$$ Then $j$ is an equivalence of $\bfS$-enriched categories.
\end{itemize}
\end{definition}

In other words, the invertibility hypothesis is the assertion that inverting a morphism
$f$ in a $\bfS$-enriched category $\calC$ does not change the homotopy type of $\calC$ when $f$ is already invertible up to homotopy. 

\begin{remark}\label{attaboy}
Let $\bfS$, $f$, and $\calC$ be as in Definition \ref{inhyp}, and choose a trivial cofibration
$F: \calC \rightarrow \calC'$, where $\calC'$ is a fibrant $\bfS$-enriched category.
Since $\SCat$ is left proper, the induced map
$\calC \langle f^{-1} \rangle \rightarrow \calC' \langle F(f)^{-1} \rangle$
is an equivalence of $\bfS$-enriched categories. Consequently,
assertion $(\ast)$ holds for $(\calC, f)$ if and only if it holds for
$( \calC', F(f) )$. In other words, to test whether $\bfS$ satisfies the invertibility
hypothesis, we only need to check $(\ast)$ in the case where $\calC$ is fibrant.
\end{remark}

\begin{remark}\label{uppa}
In Definition \ref{inhyp}, the condition that $i$ be a cofibration guarantees that
the construction $\calC \mapsto \calC \langle f^{-1} \rangle$ is homotopy invariant.
Alternatively, we can guarantee this by choosing a cofibrant replacement for the
map $j: [1]_{\bfS} \rightarrow [1]^{\sim}_{\bfS}$. Namely, choose a factorization for
$j$ as a composition
$$ [1]_{\bfS} \stackrel{j'}{\rightarrow} \calE \stackrel{j''}{\rightarrow} [1]^{\sim}_{\bfS},$$
where $j''$ is a weak equivalence and $j'$ is a cofibration. For every
$\bfS$-enriched category containing a morphism $f$, define
$\calC [f^{-1} ] = \calC \coprod_{ [1]_{\bfS} } \calE$. 
Then we have a canonical map $\calC [f^{-1}] \rightarrow \calC \langle f^{-1} \rangle$,
which is an equivalence whenever the map $[1]_{\bfS} \rightarrow \calC$ classifying $f$ is a cofibration. Moreover, the construction $\calC \mapsto \calC[f^{-1}]$ preserves weak equivalences in $\calC$. Consequently, we may reformulate the invertibility hypothesis as follows:
\begin{itemize}
\item[$(\ast')$] For every $\bfS$-enriched category $\calC$ containing an equivalence
$f$, the map $\calC \rightarrow \calC[f^{-1}]$ is a weak equivalence of $\bfS$-enriched categories.
\end{itemize}
\end{remark}

\begin{remark}\label{cuddle}
Let $\calC$ be a fibrant $\bfS$-enriched category containing an equivalence
$f: X \rightarrow Y$, and let $\calC[ f^{-1} ]$ be defined as in Remark \ref{uppa}. 
The canonical map $\calC \rightarrow \calC[f^{-1}]$ is a trivial cofibration, and
therefore admits a section. This section determines a map of $\bfS$-enriched
categories $h: \calE \rightarrow \calC$. We observe that $\calE$ is a mapping
cylinder for the object $[0]_{\SCat} \in \SCat$, so we can view
$h$ as a homotopy between the maps $[0]_{\SCat} \rightarrow \calC$
classifying the objects $X$ and $Y$.

More generally, the same argument shows that if $F: \calC \rightarrow \calD$
is a fibration of $\bfS$-enriched categories and $f: X \rightarrow Y$ is an equivalence
in $\calC$ such that $F(f) = \id_{D}$ for some object $D \in \calD$, then
the functors $[0]_{\bfS} \rightarrow \calC$ classifying the objects $X$ and $Y$
are homotopic in the model category $( \SCat)_{/ \calD}$. 
\end{remark}

\begin{definition}\index{gen}{excellent!model category}\index{gen}{model category!excellent}\label{modelexcellent}
We will say that a model category $\bfS$ is {\it excellent} if it is equipped with a symmetric
monoidal structure and satisfies the following conditions:
\begin{itemize}
\item[$(A1)$] The model category $\bfS$ is combinatorial.
\item[$(A2)$] Every monomorphism in $\bfS$ is a cofibration, and the collection of cofibrations
is stable under products.
\item[$(A3)$] The collection of weak equivalences in $\bfS$ is stable under filtered colimits.
\item[$(A4)$] The monoidal structure on $\bfS$ is compatible with the model structure.
In other words, the tensor product functor $\otimes: \bfS \times \bfS \rightarrow \bfS$ is a left Quillen bifunctor.
\item[$(A5)$] The monoidal model category $\bfS$ satisfies the invertibility hypothesis.
\end{itemize}
\end{definition}

\begin{remark}\label{dummu}
Axiom $(A2)$ of Definition \ref{modelexcellent} implies that every object of $\bfS$ is cofibrant.
In particular, $\bfS$ is left proper.
\end{remark}

\begin{example}[Dwyer, Kan]\label{cuppata}
The category of simplicial sets is an excellent model category, when endowed with the Kan model structure and the Cartesian product. The only nontrivial point is to show that $\sSet$ satisfies the invertibility hypothesis. This is one of the main theorems of \cite{dwyerkan}.
\end{example}

\begin{example}
Let $\bfS$ be a presentable category equipped with a closed symmetric monoidal structure.
Then $\bfS$ is an excellent model category with respect to the trivial model structure
of Example \ref{trivmodel}.
\end{example}

The following lemma guarantees a good supply of examples of excellent model categories:

\begin{lemma}\label{cuppat}
Suppose given a monoidal left Quillen functor $T: \bfS \rightarrow \bfS'$
between model categories $\bfS$ and $\bfS'$ satisfying axioms
$(A1)$ through $(A4)$ of Definition \ref{modelexcellent}.
If $\bfS$ satisfies axiom $(A5)$, then so does $\bfS'$.
\end{lemma}

\begin{proof}
As indicated in Remark \ref{cuttup}, the functor $T$ determines a Quillen adjunction
$$ \Adjoint{F}{\Cat_{\bfS}}{\Cat_{\bfS'}}{G}.$$
Let $\calC$ be a $\bfS'$-enriched category and $i: [1]_{\bfS'} \rightarrow \calC$
a cofibration classifying an equivalence $f$ in $\calC$. We wish to prove that the
map $\calC \rightarrow \calC \langle f^{-1} \rangle$ is an equivalence of
$\bfS'$-enriched categories. In view of Remark \ref{attaboy}, we may assume
that $\calC$ is fibrant.

Choose a factorization of the map $[1]_{\bfS} \rightarrow [1]_{\bfS}^{\sim}$
as a composition
$$ [1]_{\bfS} \stackrel{j}{\rightarrow} \calE \stackrel{j'}{\rightarrow} [1]_{\bfS}^{\sim}$$
as in Remark \ref{uppa}, so that we have an analogous factorization
$$ [1]_{\bfS'} \rightarrow F(\calE) \rightarrow [1]_{\bfS'}^{\sim}$$
in $\Cat_{\bfS'}$. Using the latter factorization, we can define
$\calC[ f^{-1}]$ as in Remark \ref{uppa}; we wish to show that the map
$h: \calC \rightarrow \calC[f^{-1}]$ is a trivial cofibration. 

Let $f_0$ be the morphism in $G(\calC)$ classified by $f$, and
let $G(\calC)[f_0^{-1}] \in \SCat$ be defined as in Remark \ref{uppa}. 
Using the fact that $\calC$ is locally fibrant (see Theorem \ref{staycode} below),
we conclude that $f_0$ is an equivalence in $G(\calC)$. Since $\bfS$
satisfies the invertibility hypothesis, the map
$h_0: G(\calC) \rightarrow G(\calC)[f_0^{-1}]$ is a trivial cofibration.
We now conclude by observing that $h$ is a pushout of $F(h_0)$.
\end{proof}

\begin{remark}\label{conrem}
Using a similar argument, we can prove a converse to Lemma \ref{cuppat} in the case where
$T$ is a Quillen equivalence.
\end{remark}

\begin{example}\label{supermat}
Let $\bfS$ be the category $\mSet$ of marked simplicial sets, endowed with
the Cartesian model structure defined in \S \ref{twuf}. Then the functor
$X \mapsto X^{\sharp}$ is a monoidal left Quillen functor 
$\sSet \rightarrow \bfS$. Combining Example \ref{cuppata} with Lemma \ref{cuppat}, we
conclude that $\bfS$ satisfies the invertibility hypothesis, so that $\bfS$ is an excellent model category (with respect to the Cartesian product).
\end{example}

\begin{example}
Let $\bfS$ denote the category of simplicial sets, endowed with the Joyal model structure.
The functor $X \mapsto X^{\flat}$ determines a monoidal left Quillen equivalence
$\bfS \rightarrow \mSet$. Using Remark \ref{conrem}, we deduce that $\bfS$ satisfies the invertibility hypothesis, so that $\bfS$ is an excellent model category (with respect to the Cartesian product).
\end{example}

We are now ready to state our main result:

\begin{theorem}\label{staycode}
Let $\bfS$ be an excellent model category.
Then:
\begin{itemize}
\item[$(1)$] An $\bfS$-enriched category $\calC$ is a fibrant object of $\SCat$ if and only if it is {\em locally fibrant}: that is, if and only if the mapping object $\bHom_{\calC}(X,Y) \in \bfS$ is fibrant for
every pair of objects $X,Y \in \calC$. 
\item[$(2)$] Let $F: \calC \rightarrow \calD$ be a $\bfS$-enriched functor, where $\calD$ is a
fibrant object of $\SCat$. Then $F$ is a fibration in $\SCat$ if and only if $F$ is a local fibration.
\end{itemize}
\end{theorem}

\begin{remark}
In the case where $\bfS$ is the category of simplicial sets (with its usual model structure),
Theorem \ref{staycode} is due to Bergner; see \cite{bergner}. Moreover, Bergner
proves a stronger result in this case: assertion $(2)$ holds without the assumption that
$\calD$ is fibrant.
\end{remark}

Before giving the proof of proof of Theorem \ref{staycode}, we need to establish some preliminaries.
Fix an excellent model category $\bfS$. We observe
that $\SCat$ is naturally {\em cotensored} over $\bfS$. That is, for
every $\bfS$-enriched category $\calC$ and every object $K \in \bfS$, we can define a new $\bfS$-enriched category $\calC^{K}$ as follows:

\begin{itemize}
\item[$(i)$] The objects of $\calC^{K}$ are the objects of $\calC$.
\item[$(ii)$] Given a pair of objects $X,Y \in \calC$, we have
$\bHom_{\calC^{K}}(X,Y) = \bHom_{\calC}(X,Y)^{K} \in \bfS$. 
\end{itemize} 

This construction does not endow $\SCat$ with the structure of a $\bfS$-enriched category, because
the construction $\calD \mapsto \calD^{K}$ is not compatible with colimits in $K$. However,
we can remedy the situation as follows. Let $\calC$ and $\calD$ be $\bfS$-enriched categories,
and let $\phi$ be a function from the set of objects of $\calC$ to the set of objects of $\calD$. Then
there exists an object $\bHom_{\SCat}^{\phi}( \calC, \calD) \in \bfS$, which is characterized by the following universal property: for every $K \in \bfS$, there is a natural bijection
$$ \Hom_{\bfS}(K, \bHom_{\SCat}^{u}(\calC, \calD) ) \simeq \Hom^{\phi}_{\SCat}(\calC, \calD^{K} ),$$
where $\Hom^{\phi}_{\SCat}(\calC, \calD^{K} )$ denotes the set of all functors from
$\calC$ to $\calD^{K}$ which is given on objects by the function $\phi$.

\begin{lemma}\label{stuttcat}
Let $\bfS$ be an excellent model category.
Fix a diagram of $\bfS$-enriched categories
$$ \xymatrix{ \calC \ar[d]^{F} \ar[r]^{u} & \calC' \ar[d]^{F'} \\
\calD \ar[r]^{u'} & \calD'. }$$ 
Assume that:
\begin{itemize}
\item[$(a)$] For every pair of objects $X, Y \in \calC$, the diagram
$$ \xymatrix{ \bHom_{\calC}(X,Y) \ar[r] \ar[d] & \bHom_{\calD}( FX, FY) \ar[d] \\
\bHom_{\calC'}(uX, uY) \ar[r] & \bHom_{\calD'}( u'FX, u'FY) }$$
is homotopy pullback square between fibrant objects of $\bfS$, and the vertical arrows are fibrations.
\end{itemize}

Let $G: \calA \rightarrow \calB$ be a functor between $\bfS$-enriched categories
which is a transfinite composition of pushouts of generating cofibrations of the form
$[1]_{S} \rightarrow [1]_{S'}$, where $S \rightarrow S'$ is a cofibration in
$\bfS$, and let $\phi$ be a function from the set of objects of $\calB$ (which
is isomorphic to the set of objects of $\calA$) to $\calC$. 
Then the diagram
$$ \xymatrix{  \bHom^{\phi}_{ \SCat}( \calB, \calC) \ar[r] \ar[d] &
\bHom^{F\phi}_{\SCat}( \calB, \calD) \times_{ \bHom^{F\phi}_{\SCat}(\calA, \calD)}
\bHom^{\phi}_{\SCat}(\calA, \calC) \ar[d] \\
 \bHom^{u\phi}_{\SCat}(\calB, \calC') \ar[r] & 
\bHom^{u'F\phi}_{\SCat}(\calB, \calD') \times_{ \bHom^{u'F\phi}_{\SCat}(\calA, \calD') }
\bHom^{u\phi}_{\SCat}( \calA, \calC') }$$
is a homotopy pullback square between fibrant objects of $\bfS$, and the vertical arrows
are fibrations.
\end{lemma}

\begin{proof}
It is easy to see that the collection of morphisms $G: \calA \rightarrow \calB$ which satisfy the conclusion of the lemma is weakly saturated. It will therefore suffice to show that $G$ contains every
morphism of the form $[1]_{S} \rightarrow [1]_{S'}$, where $S \rightarrow S'$
is a cofibration in $\bfS$. In this case, $\phi$ determines a pair of objects $X,Y \in \calC$, and we can rewrite the diagram of interest as
$$ \xymatrix{ \bHom_{\calC}(X,Y)^{S'} \ar[r] \ar[d] & 
\bHom_{\calC}(X,Y)^{S} \times_{ \bHom_{\calD}(FX,FY)^{S} }
\bHom_{\calD}(FX,FY)^{S'} \ar[d] \\
\bHom_{\calC'}( uX, uY)^{S'} \ar[r] & 
\bHom_{\calC'}(uX,uY)^{S} \times_{ \bHom_{\calD'}(u'FX, u'FY)^{S} }
\bHom_{\calD'}( u' FX, u'FY)^{S'}. }$$
The desired result now follows from $(a)$, since the map $S \rightarrow S'$ is a cofibration
between cofibrant objects of $\bfS$.
\end{proof}

%\begin{warning}
%Definition \ref{qfib} is somewhat unnatural, in that it is not invariant
%under equivalences of categories. It is a notion which 
%exists in the category $\Cat$ of small categories, rather than the $2$-category of small
%categories. One may explain its relevance as follows: the
%$2$-category of small categories is a kind of homotopy theory
%(provided that we ignore non-invertible natural transformations of
%functors), which may be modelled by the category $\Cat$ of small
%categories. In fact, there exists a Quillen model structure on
%$\Cat$ in which the weak equivalences are equivalences of
%categories and the fibrations are the quasi-fibrations of
%categories. For a proof, we refer the reader to \cite{joyalnotpub}.
%\end{warning}

%\begin{remark}
%If $\bfS$ is a monoidal model category, then the monoidal structure on $\bfS$ induces (by taking %left derived functors) a monoidal structure on the homotopy category $h \bfS$. If every object in
%$\bfS$ is cofibrant, then it is not necessary to take the left derived functor of the monoidal structure %$\otimes$, so the functor $\bfS \rightarrow h \bfS$ is monoidal. It follows that any
%$\bfS$-enriched category $\calC$ has an underlying $h \bfS$-enriched category, which we will denote by $h \calC$. In the case where $\bfS$ is the category of simplicial sets (with its usual model structure), this agrees with the definition given in \S \ref{compp1}. We will denote the ordinary category underlying $h \calC$ by $h \calC$.
%\end{remark}


%The following theorem of Bergner gives an explicit characterization of the fibrations in the category $\sCat$ of simplicial categories:

%\begin{theorem}[Bergner \cite{bergner}]\label{berg}\index{gen}{fibration!of simplicial categories}
%Let $F: \calC \rightarrow \calD$ be a functor between $($small$)$ simplicial categories. 
%Then $F$ is a fibration in $\sCat$ if and only if $F$ is a quasi-fibration.
%\end{theorem}

%\begin{corollary}\label{sumtun}
%A simplicial category $\calC$ is fibrant if and only if each of the morphism spaces
%$\bHom_{\calC}(X,Y)$ is a Kan complex.
%\end{corollary}

%\begin{corollary}\label{basetin}
%Let $\calC$ be a fibrant simplicial category containing a morphism
%$f: X \rightarrow Y$, such that the homotopy class $[f]$ is an
%isomorphism in $\h{\calC}$. Then the functor
%$[1] \rightarrow \calC$ classifying $f$ factors as a composition
%$$ [1] \stackrel{i}{\rightarrow} \calD \stackrel{j}{\rightarrow} \calC,$$
%such that the composition $\{0\} \hookrightarrow [1] \stackrel{i}{\rightarrow} \calD$ is a trivial
%cofibration.
%\end{corollary}

%\begin{proof}
%Let $i_0: [0] \rightarrow \calC$ be the inclusion of the object $X$. 
%Choose a factorization of $i_0$ as a composition
%$$ \ast \stackrel{i'_0}{\rightarrow} \calD \stackrel{j}{\rightarrow} \calC,$$
%where $i'_0$ is a trivial cofibration and $j$ is a fibration.
%Theorem \ref{berg} implies that $j$ is a quasi-fibration. It follows that
%the map $f: X \rightarrow Y$ can be lifted to a morphism
%$\overline{f}: \overline{X} \rightarrow \overline{Y}$ in $\calD$, which
%we can identify with a map $i: [1] \rightarrow \calD$. It is easy to see that the maps $i$ and $j$ have the desired properties.
%\end{proof}

\begin{proof}[Proof of Theorem \ref{staycode}]
Assertion $(1)$ is just a special case of $(2)$, where we take $\calD$ to be the final object
of $\SCat$. It will therefore suffice to prove $(2)$. 

We first prove the ``only if'' direction.
If $F$ is a fibration, then $F$ has the right lifting property with respect to every
trivial cofibration of the form $[1]_{S} \rightarrow [1]_{S'}$, where $S \rightarrow S'$ is
a trivial cofibration in $\bfS$. It follows that for every pair of objects $X, Y \in \calC$, 
the induced map $\bHom_{\calC}(X,Y) \rightarrow \bHom_{\calD}(FX, FY)$ is a fibration in $\bfS$.
In particular, $\calC$ is locally fibrant.

To complete the proof that $F$ is a local fibration, we will show that $F$ satisfies condition $(ii')$ of Remark \ref{dinty}. Suppose $X \in \calC$, and that $f: FX \rightarrow Y$ is an equivalence in
$\calD$. We wish to show that we can lift $f$ to an equivalence
$\overline{f}: X \rightarrow \overline{Y}$. Let $\calE$ and $\calD[f^{-1}]$ be defined as
in Remark \ref{uppa}. 
Since $\bfS$ satisfies the invertibility hypothesis, the map $h: \calD \rightarrow \calD[f^{-1}]$
is a trivial cofibration. Because we have assumed $\calD$ to be fibrant, the map $h$
admits a section. This section determines a map $s: \calE \rightarrow \calD$.
We now consider the lifting problem
$$ \xymatrix{ [0]_{\bfS} \ar[r]^{X} \ar[d] & \calC \ar[d]^{F} \\
\calE \ar[r]^{s} \ar@{-->}[ur] & \calD. }$$
Since $F$ is a fibration and the left vertical map is a trivial cofibration, there
exists a solution as indicated. This solution determines a morphism
$\overline{f}: X \rightarrow \overline{Y}$ in $\calC$ lifting $f$.
Moreover, $\overline{f}$ is the image of a morphism in $\calE$.
Since every morphism in $\calE$ is an equivalence, we deduce that
$\overline{f}$ is an equivalence in $\calC$.

Let us now suppose that $F$ is a local fibration. We wish to show that
$F$ is a fibration. Choose a factorization of $F$ as a composition
$$ \calC \stackrel{u}{\rightarrow} \calC' \stackrel{F'}{\rightarrow} \calD,$$
where $u$ is a weak equivalence and $F'$ is a fibration. We will prove the following:

\begin{itemize}
\item[$(\ast)$] Suppose given a commutative diagram of $\bfS$-enriched categories
$$ \xymatrix{ \calA \ar[r]^{v} \ar[d]^{G} & \calC \ar[d]^{F} \\
\calB \ar[r]^{v'} & \calD, }$$
where $G$ is a cofibration. If there exists a functor $\alpha: \calB \rightarrow \calC'$ such
that $\alpha G = u v$ and $F' \alpha = v'$, then there exists a functor $\beta: \calB \rightarrow \calC$
such that $\beta G = v$ and $F \beta = v'$. 
\end{itemize}

Since the map $F'$ has the right lifting property with respect to all trivial cofibrations, 
assertion $(\ast)$ implies that $F$ also has the right lifting property with respect to all trivial cofibrations, so that $F$ is a fibration as desired.

We now prove $(\ast)$. Using the small object argument, we deduce that 
the functor $G$ is a retract of some functor $G': \calA \rightarrow \calB'$, where
$G'$ is a transfinite composition of morphisms obtained as pushouts of 
generating cofibrations. It will therefore suffice to prove $(\ast)$ after replacing $G$ by $G'$.

Reordering the transfinite composition if necessary, we may assume that $G'$ factors as a
composition
$$ \calA \stackrel{G'_0}{\rightarrow} \calB'_0 \stackrel{G'_1}{\rightarrow} \calB',$$
where $\calB'_0$ is obtained from $\calA$ by adjoining a collection of new objects,
$\{B_i \}_{i \in I}$, and $\calB'$ is obtained from $\calB'_0$ by a transfinite sequence of pushouts
by generating cofibrations of the form $\calE_{S} \rightarrow \calE_{S'}$, where
$S \rightarrow S'$ is a cofibration in $\bfS$. Let $C'_i = \alpha(B_i)$ for each $i \in I$.
Since $u$ is an equivalence of $\bfS$-enriched categories, there exists a collection
of objects $\{ C_i \}_{i \in I}$ and equivalences$f_i: u C_i \rightarrow C'_i$. 
Let $g_i$ be the image of $f_i$ in $\calD$. Since $F$ is a local fibration, 
we can lift each $g_i$ to an equivalence
$f'_i: C_i \rightarrow C''_i$ in $\calC$. 
Since the maps $\bHom_{\calC'}( uC''_i, C'_i) \rightarrow \bHom_{\calD}( FC''_i, F'C'_i)$
are fibrations, we can choose morphisms $f''_i: uC''_i \rightarrow C'_i$
in $\calC'$ such that $F'( f'_i)$ is the identity for each $i$, and the diagrams
$$ \xymatrix{ & uC''_i \ar[dr]^{f''_i} & \\
uC_i \ar[rr]^{f_i} \ar[ur]^{f'_i} & & C'_i }$$
commute up to homotopy. Replacing $C_i$ by $C''_i$, we may assume
that each of the maps $f_i$ projects to the identity in $\calD$.

Let $\alpha_0 = \alpha | \calB'_0$, and let $\alpha'_0: \calB'_0 \rightarrow \calC'$ be defined by the formula $$ \alpha'_0(A) = \begin{cases} \alpha_0(A) & \text{if } A \in \calA \\
uC_i & \text{if } A = B_i, i \in I. \end{cases}$$
Remark \ref{cuddle} implies that the maps $\alpha_0$ and
$\alpha'_0$ are homotopic in the model category
$(\SCat)_{\calA/ \, / \calD}$. Applying Proposition \ref{princex}, we deduce the existence of a map
$\alpha': \calB' \rightarrow \calC$ which extends $\alpha_0$ and satisfies
$\alpha' G = u v$ and $F' \alpha' = v'$. We may therefore replace 
$\alpha$ by $\alpha'$, $v$ by $\alpha'_0$, and $\calA$ by $\calB'_0$, and thereby
reduce to the case where the functor $G: \calA \rightarrow \calB$ is a transfinite
composition of generating cofibrations of the form $\calE_{S} \rightarrow \calE_{S'}$, where
$S \rightarrow S'$ is a cofibration in $\bfS$.

Let $\phi$ be the map from the objects of $\calB$ to the objects of $\calC$ determined by $\alpha$.
Applying Lemma \ref{stuttcat}, we obtain a homotopy pullback diagram
$$ \xymatrix{  \bHom^{\phi}_{ \SCat}( \calB, \calC) \ar[r] \ar[d] & \bHom^{u\phi}{\SCat}(\calB, \calC') \ar[d] \\
\bHom^{F\phi}_{\SCat}( \calB, \calD) \times_{ \bHom^{F\phi}_{\SCat}(\calA, \calD)}
\bHom^{\phi}_{\SCat}(\calA, \calC) \ar[r] &
\bHom^{F\phi}_{\SCat}(\calB, \calD) \times_{ \bHom^{F\phi}_{\SCat}(\calA, \calD) }
\bHom^{u\phi}_{\SCat}( \calA, \calC'). }$$
in which the vertical arrows are fibrations. We therefore have a weak
equivalence $$\bHom^{\phi}_{\SCat}( \calB, \calC)
\rightarrow M = \bHom^{u \phi}_{\SCat}(\calB, \calC') \times_{ \bHom^{F \phi}_{\SCat}( \calB, \calD) }
\bHom^{\phi}_{\SCat}(\calA, \calC)$$
of fibrations over $N= \bHom^{F\phi}_{\SCat}(\calB, \calD) \times_{ \bHom^{F\phi}_{\SCat}(\calA, \calD) } \bHom^{u\phi}_{\SCat}( \calA, \calC')$. Moreover, the pair
$(\alpha,v)$ determines a map ${\bf 1}_{\bfS} \rightarrow M$ lifting the map
$(v', uv'): {\bf 1}_{\bfS} \rightarrow N$. Applying Proposition \ref{princex}, we deduce
that $(v,uv'): {\bf 1}_{\bfS} \rightarrow N$ can be lifted to a map
${\bf 1}_{\bfS} \rightarrow \bHom^{\phi}_{\SCat}(\calB, \calC)$, which is equivalent
to the existence of the desired map $\beta$.
\end{proof}

We conclude this section with a few easy results concerning homotopy limits in the
model category $\SCat$.

\begin{proposition}\label{scam}
Let $\bfS$ be an excellent model category, $\calJ$ a small category, and
$\{ \calC_{J} \}_{J \in \calJ}$ a diagram of $\bfS$-enriched categories.
Suppose given a compatible family of functors $\{ f_J: \calC \rightarrow \calC_{J} \}_{J \in \calJ}$ 
which exhibits
$\calC$ as a homotopy limit of the diagram $\{ \calC_{J} \}_{J \in \calJ}$ in $\SCat$. Then
for every pair of objects $X, Y \in \calC$, the maps
$\{ \bHom_{\calC}( X,Y) \rightarrow \bHom_{\calC_{J} }( f_{J} X, f_{J} Y) \}_{J \in \calJ}$
exhibit $\bHom_{\calC}(X,Y)$ as a homotopy limit of the diagram
$\{ \bHom_{ \calC_{J} }( f_J X, f_J Y) \}_{J \in \calJ}$ in $\bfS$.
\end{proposition}

\begin{proof}
Without loss of generality, we may assume that the diagram $\{ \calC_{J} \}_{J \in \calJ}$ is injectively fibrant, and that the maps $f_{J}$ exhibit $\calC$ as a limit of $\{ \calC_{J} \}_{J \in \calJ}$. 
It follows that $\bHom_{\calC}(X,Y)$ is a limit of the diagram $\{ \bHom_{\calC_J}( f_J X, f_J Y) \}_{J \in \calJ}$.
It will therefore suffice to show that the diagram $\{ \bHom_{ \calC_{J}}( f_J X, f_J Y) \}_{J \in \calJ}$
is injectively fibrant. For this, it will suffice to show that $\{ \bHom_{ \calC_{J}}( f_J X, f_J Y) \}_{J \in \calJ}$
has the right lifting property with respect to every weak trivial cofibration 
$\alpha: F \rightarrow F'$ of diagrams $F,F': \calJ \rightarrow \bfS$. Let
$G: \calJ \rightarrow \SCat$ be defined by the formula
$G(J) = [1]_{F(J)}$, and let $G': \calJ \rightarrow \SCat$ be defined likewise. The desired
result now follows from the observe that $\alpha$ induces a weak trivial cofibration
$G \rightarrow G'$ in $\Fun( \calJ, \SCat)$. 
\end{proof}

\begin{corollary}\label{wspin}
Let $\bfS$ be an excellent model category, $\calJ$ a small category, and
$\{ \calC_{J} \}_{J \in \calJ}$ a diagram of $\bfS$-enriched categories.
Suppose given $\bfS$-enriched functors
$$ \calD \stackrel{\beta}{\rightarrow} \calC \stackrel{\alpha}{\rightarrow} \lim \{ \calC_{J} \}_{J \in \calJ} $$
such that $\alpha \circ \beta$ exhibits $\calD$ as a homotopy limit of the diagram
$\{ \calC_{J} \}_{J \in \calJ}$. Then the following conditions are equivalent:
\begin{itemize}
\item[$(1)$] The functor $\alpha$ exhibits $\calC$ as a homotopy limit of the diagram
$\{ \calC_{J} \}_{J \in \calJ}$.
\item[$(2)$] For every pair of objects $X,Y \in \calC$, the functor $\alpha$ exhibits
$\bHom_{\calC}(X,Y)$ as a homotopy limit of the diagram $\{ \bHom_{\calC_{J}}( \alpha_{J} X,
\alpha_{J} Y ) \}_{J \in \calJ}$.
\end{itemize}
\end{corollary}

\begin{proof}
The implication $(1) \Rightarrow (2)$ follows from Proposition \ref{scam}. To prove the
converse, we may assume that the diagram $\{ \calC_{J} \}_{J \in \calJ}$ is injectively fibrant.
In view of $(2)$, Proposition \ref{scam} implies that $\alpha$ induces a fully faithful functor
between $\h{\bfS}$-enriched homotopy categories. It will therefore suffice to show that
$\alpha$ is essentially surjective on homotopy categories, which follows from our assumption that 
$\alpha \circ \beta$ is a weak equivalence. 
\end{proof}

%In the case where $\bfS$ is the category of simplicial sets (with its usual model structure), Theorem \ref{berg} is proven in \cite{bergner}. We will give a proof in this section which works in the general case. Our strategy is taken from \cite{bergner}. Since $\bfS$ is perfect, there exists an uncountable regular cardinal $\kappa$ such that any weak equivalence $1_{\bfS} \rightarrow X$ in $\bfS$ factors as a composition of weak equivalences $1_{\bfS} \rightarrow X' \rightarrow X$, where $X'$ is $\kappa$-compact. Enlarging $\kappa$ if necessary, we may suppose that the collection of $\kappa$-compact objects is stable under the monoidal structure on $\bfS$.

%The main step in the proof of Theorem \ref{berg} is the following rather technical lemma:

%\begin{lemma}\label{sturkem}
%Let $\calC$ be a $\bfS$-enriched category and let $f: X \rightarrow Y$ be a isomorphism in $h \calC$. Then there exists a $\bfS$-enriched category $\calC_0$ and a functor
%$F: \calC_0 \rightarrow \calC$ with the following properties:
%\begin{itemize}
%\item[$(1)$] The $\bfS$-enriched category $\calC_0$ has just two objects, $X_0$ and $Y_0$.
%\item[$(2)$] There exists an isomorphism $f_0: X_0 \rightarrow Y_0$ in $h \calC_0$ such
%that $F(f_0) = f$.
%\item[$(3)$] The map $1_{\bfS} \rightarrow \bHom_{\calC_0}(X_0, X_0)$ is weak equivalence in $\bfS$.
%\item[$(4)$] The functor $\{X_0 \} \rightarrow \calC_0$ is a cofibration (automatically trivial, in virtue of $(2)$ and $(3)$). 
%\item[$(5)$] Each of the mapping objects of $\calC_0$ is $\kappa$-compact (when regarded as an object of $\bfS$).
%\item[$(6)$] The functor $F$ has the left lifting property with respect to every functor which
%satisfies properties $(1)$ and $(2)$ of Theorem \ref{berg}. 
%\end{itemize}
%\end{lemma}

%We will postpone the proof until the end of this section.

%Choose a set of trivial cofibrations $\{ A_{\alpha} \rightarrow B_{\alpha} \}$ in $\bfS$ which
%generates $\bfS$ as a saturated class of morphisms, and let $S_1$ denote the set of
%associated functors $\calE_{A_{\alpha}} \rightarrow \calE_{B_{\alpha}}$.  Let $S_2$ denote set of representatives for the isomorphism classes of cofibrations $\{ X_0 \} \rightarrow \calC_0$, where $\calC_0$ satisfies conditions $(1)$ through $(5)$ of Lemma \ref{sturkem}. Set 
%$S = S_1 \cup S_2$, and let $\overline{S}$ be the saturated class of morphisms of
%$\Cat_{\bfS}$ generated by $S$. Using the small object argument, we immediately deduce the following:

%\begin{lemma}\label{smljk}
%Every functor $F: \calC \rightarrow \calD$ of $\bfS$-enriched categories admits a factorization
%$$ \calC \stackrel{F'}{\rightarrow} \calC' \stackrel{F''}{\rightarrow} \calD$$
%where $F'$ belongs to $\overline{S}$ and $F''$ has the right lifting property with respect to every morphism in $S$.
%\end{lemma}

%The main step in the proof of Theorem \ref{berg} is the following:

%\begin{lemma}\label{whurne}
%Let $F: \calC \rightarrow \calD$ be a functor between $\bfS$-enriched categories.
%Then $F$ satisfies conditions $(1)$ and $(2)$ of Theorem \ref{berg} if and only if $F$ has the right lifting property with respect to every morphism in $S$.
%\end{lemma}

%\begin{proof}
%It follows immediately from the definitions that $F$ satisfies $(1)$ if and only if $F$ has the right lifting property with respect to every element of $S_1$. Moreover, if $F$ satisfies both $(1)$ and $(2)$, then $F$ has the right lifting property with respect to every morphism in $S_2$ 
%by construction (condition $(6)$ of Lemma \ref{sturkem}). To complete the proof, it will suffice to show that if $F$ has the right lifting property with respect to every morphism in $S_2$, then
%$F$ induces a quasi-fibration $h \calC \rightarrow h \calD$.

%Let $X \in \calC$, and let $f: FX \rightarrow \overline{Y}$ be an isomorphism in $h \calD$. 
%Choose a $\bfS$-enriched functor $G: \calD_0 \rightarrow \calD$, as in the statement of Lemma \ref{sturkem}. Consider the diagram
%$$ \xymatrix{ \{X_0\} \ar[r] \ar[d] & \calC \ar[d]^{F} \\
%\calD_0 \ar[r]^{G} \ar@{-->}[ur] & \calD. }$$
%Since the left vertical arrow belongs to $S_2$, there exists a diagonal arrow as indicated, rendering the diagram commutative. It follows that $h \calC$ contains an isomorphism 
%$X \rightarrow Y$ lifting $f$, as desired.
%\end{proof}

%We are now ready to give the proof of Theorem \ref{berg}:

%\begin{proof}
%According to Lemma \ref{whurne}, it will suffice to show that $F$ is a fibration if and only if
%it has the right lifting property with respect to every morphism in $S$. Equivalently, we must
%show that $\overline{S}$ consists of precisely the trivial cofibrations in $\Cat_{\bfS}$. 
%By construction, every morphism in $\overline{S}$ is a trivial cofibration. For the reverse inclusion, we consider an arbitrary trivial cofibration $G: \calE \rightarrow \calE''$, and use Lemma \ref{smljk} to produce a factorization
%$$ \calE \stackrel{G'}{\rightarrow} \calE' \stackrel{G''}{\rightarrow} \calE''$$
%where $G' \in \overline{S}$ and $G''$ has the right lifting property with respect to every
%morphism in $\overline{S}$. By the two-out-of-three property, we conclude that $G''$
%is a weak equivalence. We now claim that $G''$ is a trivial fibration. To verify this, we need only to check two things:

%\begin{itemize}
%\item[$(i)$] The functor $G''$ is surjective on objects. Let $E''$ be an object of $\calE''$. Since
%$G''$ is a weak equivalence, there exists an object $E' \in \calE'$ and an isomorphism
%$f: G'' E' \simeq E''$ in $h \calE''$. Using Lemma \ref{whurne}, we deduce that
%$f$ is the image under $G''$ of an isomorphism $E' \simeq \overline{E}''$ in $h \calE'$, so that
%$E'' = G''( \overline{E}'')$. 

%\item[$(ii)$] The functor $G''$ induces a trivial fibration $\phi: \bHom_{\calE'}(X,Y) \rightarrow
%\bHom_{\calE''}(G'' X, G'' Y)$ for every pair of objects $X,Y \in \calE'$. The fact that
%$\phi$ is a fibration follows from Lemma \ref{whurme}, and the triviality follows from the
%assumption that $G''$ is a weak equivalence.
%\end{itemize}

%We now consider the diagram
%$$ \xymatrix{ \calE \ar[d]^{G} \ar[r]^{G'} & \calE' \ar[d]^{G''} \\
%\calE'' \ar@{=}[r] \ar@{-->}[ur] & \calE''. }
%Since $G$ is a cofibration and $G''$ is a trivial fibration, there exists a dotted arrow as indicated, %rendering the diagram commutative. It follows that $G$ is a retract of $G'$. Since $G' \in \overline{S}$ we conclude that $G \in \overline{S}$, as desired.
%\end{proof}


\subsection{Model Structures on Diagram Categories}\label{quasilimit3}

In this section, we consider enriched analogues of the constructions presented in
\S \ref{qlim7}. Namely, suppose that $\bfS$ is an excellent model category,
$\bfA$ a combinatorial $\bfS$-enriched model category, and $\calC$ a small
$\bfS$-enriched category. Let $\bfA^{\calC}$ denote the category of
$\bfS$-enriched functors from $\calC$ to $\bfA$. In this section, we will study the associated
projective and injective model structures on $\bfA^{\calC}$. The ideas described here will be used in \S \ref{pathspace} to construct certain mapping objects in $\SCat$.

We begin with the analogue of Definition \ref{injproj}.

\begin{definition}\label{projinj}\index{gen}{cofibration!strong}\index{gen}{cofibration!weak}\index{gen}{fibration!strong}\index{gen}{fibration!weak}\index{gen}{strong!cofibration}\index{gen}{strong!fibration}\index{gen}{weak!cofibration}\index{gen}{weak!fibration}
Let $\calC$ be a small $\bfS$-category, and $\bfA$ a combinatorial $\bfS$-enriched model category.
A natural transformation $\alpha: F \rightarrow G$ in $\bfA^{\calC}$ is a:

\begin{itemize}
\item {\it injective cofibration} if the induced map $F(C) \rightarrow
G(C)$ is a cofibration in $\bfA$, for each $C \in \calC$.

\item {\it projective fibration} if the induced map $F(C) \rightarrow
G(C)$ is a fibration in $\bfA$, for each $C \in \calC$.

\item {\it weak equivalence} if the induced map $F(C) \rightarrow
G(C)$ is a weak equivalence in $\bfA$, for each $C \in \calC$.

\item {\it injective fibration} if it has the right lifting property
with respect to every morphism $\beta$ in $\bfA^{\calC}$ which is
simultaneously a weak equivalence and a injective cofibration.

\item {\it projective cofibration} if it has the left lifting property
with respect to every morphism $\beta$ in $\bfA^{\calC}$ which is
simultaneously a weak equivalence and a projective fibration.
\end{itemize}
\end{definition}

\begin{proposition}\label{smurf}\index{gen}{model category!projective}\index{gen}{model category!injective}
Let $\bfS$ be an excellent model category, $\bfA$ be a combinatorial $\bfS$-enriched model category, and let $\calC$ be a small $\bfS$-enriched category. Then there exist two combinatorial model structures on $\bfA^{\calC}$:

\begin{itemize}
\item The {\it projective model structure}, determined by the strong
cofibrations, weak equivalences, and projective fibrations.

\item The {\it injective model structure}, determined by the weak
cofibrations, weak equivalences, and injective fibrations.
\end{itemize}
\end{proposition}

The proof of Proposition \ref{smurf} is identical to that of Proposition \ref{smurff}, 
except that it requires the following more general form of Lemma \ref{mainerthyme}:

\begin{lemma}\label{mainertime}
Let $\bfA$ be a presentable category which is enriched, tensored, and cotensored over another presentable category $\bfS$. Let $S_0$ be a (small) set of morphisms of $\bfA$, and let $\overline{S}_0$ be the weakly saturated class of morphisms generated by $S_0$. Let $\calC$ be a small $\bfS$-enriched category. Let $\widetilde{S}$ be the collection of all morphisms $F \rightarrow G$ in $\bfA^{\calC}$ with the following property: for every $C \in \calC$, the map $F(C) \rightarrow G(C)$ belongs to $\overline{S}_0$. Then there exists a (small) set of morphisms $S$ of $\bfA^{\calC}$ which generates $\widetilde{S}$ (as a weakly saturated class of morphisms).
\end{lemma}

We will defer the proof until the end of this section.

\begin{remark}\label{suboteki}
In the situation of Proposition \ref{smurf}, the category $\bfA^{\calC}$ is again
enriched, tensored and cotensored over $\bfS$. Tensor product with an object $K \in \bfS$ is computed pointwise; in other words, if $\calF \in \bfA^{\calC}$, then we have the formula $$ (K \otimes \calF)(A) = K \otimes \calF(A).$$
Using criterion $(2')$ of Remark \ref{cyclor}, we deduce that $\bfA^{\calC}$ is a $\bfS$-enriched model category with respect to the injective model structure. A dual argument (using criterion $(2'')$ of Remark \ref{cyclor}) shows that $\bfA^{\calC}$ is also a $\bfS$-enriched model category with respect to the projective model structure.
\end{remark}

%\begin{proof}[Proof of Proposition \ref{smurf}]
%We first treat the case of the projective model structure. For each
%For each object $C \in \calC$ and each $A \in \bfA$, we define 
%$$\calF^C_A: \calC \rightarrow \bfA$$
%by the formula
%$$\calF^C_A(C') = A \otimes \bHom_{\calC}(C,C').$$
%We note that if $i: A \rightarrow A'$ is a (trivial) cofibration in $\bfA$, then the induced map
%$\calF^C_{A} \rightarrow \calF^C_{A'}$ is a strong (trivial) cofibration in $\bfA^{\calC}$. 
%
%Let $I_0$ be a set of generating cofibrations $i: A \rightarrow B$ for $\bfA$, and let $I$ be the set of all induced maps $\calF^{C}_{A} \rightarrow \calF^{C}_{B}$ (where $C$ ranges over $\calC$. Let $J_0$ be a set of generating trivial cofibrations for $\bfA$, and define 
%$J$ likewise. It follows immediately from the definitions that a morphism in $\bfA^{\calC}$ is a projective fibration if and only if it has the right lifting property with respect to every morphism in $J$, and a weak trivial fibration if and only if it has the right lifting property with respect to every morphism in $I$. Let $\overline{I}$ and $\overline{J}$ be the saturated classes of morphisms of $\bfA^{\calC}$ generated by $I$ and $J$, respectively. Using the small object argument, we deduce:
%\begin{itemize}
%\item[$(i)$] Every morphism $f: X \rightarrow Z$ in $\bfA^{\calC}$ admits a factorization
%$$ X \stackrel{f'}{\rightarrow} Y \stackrel{f''}{\rightarrow} Z$$
%where $f' \in \overline{I}$ and $f''$ is a weak trivial fibration.
%\item[$(ii)$] Every morphism $f: X \rightarrow Z$ in $\bfA^{\calC}$ admits a factorization
%$$ X \stackrel{f'}{\rightarrow} Y \stackrel{f''}{\rightarrow} Z$$
%where $f' \in \overline{J}$ and $f''$ is a projective fibration. 
%\item[$(iii)$] The class $\overline{I}$ coincides with the class of projective cofibrations in $\bfA$.
%\end{itemize}
%Furthermore, since the class of trivial projective cofibrations in $\bfA^{\calC}$ is saturated and contains $J$, it contains $\overline{J}$. This proves that $\bfA^{\calC}$ satisfies the factorization axioms. The only other nontrivial point to check is that $\bfA^{\calC}$ satisfies the lifting axioms. Consider a diagram
%$$ \xymatrix{ A \ar[d]^{i} \ar[r] & X \ar[d]^{p} \\
%C \ar[r] \ar@{-->}[ur] & Y }$$
%in $\bfA^{\calC}$, where $i$ is a projective cofibration and $p$ is a projective fibration. We wish to show that there exists a dotted arrow as indicated, provided that either $i$ or $p$ is a weak equivalence.
%If $p$ is a weak equivalence then this follows immediately from the definition of a injective fibration.
%Suppose instead that $i$ is a trivial projective cofibration. We wish to show that $i$ has the left lifting property with respect to every projective fibration. It will suffice to show every trivial injective fibration belongs to $\overline{J}$ (this will also show that $J$ is a set of generating trivial cofibrations for $\bfA^{\calC}$, which will show that the projective model structure on $\bfA^{\calC}$ is combinatorial). Suppose then that $i$ is a trivial weak coibration, and choose a factorization
%$$ A \stackrel{i'}{\rightarrow} B \stackrel{i''}{\rightarrow} C$$
%where $i' \in \overline{J}$ and $i''$ is a projective fibration. Then $i'$ is a weak equivalence, so
%that $i''$ is a weak equivalence by the two-out-of-three property. Consider the diagram
%$$ \xymatrix{ A \ar[d]^{i} \ar[r]^{i'} & B \ar[d]^{i''} \\
%C \ar[r]^{=} \ar@{-->}[ur] & C. }$$
%Since $i$ is a cofibration, there exists a dotted arrow as indicated. This proves that $i$ is a retract of $i'$, and therefore belongs to $\overline{J}$ as desired.

%We now treat the case of the injective model structure on $\bfA$. Here
%it is difficult to proceed directly, so we will instead apply Proposition \ref{bigmaker}. It will suffice to check each of the hypotheses in turn:
%\begin{itemize}
%\item[$(1)$] The collection of injective cofibrations in $\bfA^{\calC}$ is generated (as a saturated class) by some small set of morphisms. This follows from Lemma \ref{mainertime}.
%\item[$(2)$] The collection of trivial injective cofibrations in $\bfA^{\calC}$ is saturated: this follows immediately from the fact that the class of injective cofibrations in $\bfA$ is saturated.
%\item[$(3)$] The collection of weak equivalences in $\bfA^{\calC}$ is an accessible subcategory
%of $( \bfA^{\calC})^{[1]}$: this follows from Proposition \ref{horse1}, since the collection of weak equivalences in $\bfA$ form an accessible subcategory of $\bfA^{[1]}$. 
%\item[$(4)$] The collection of weak equivalences in $\bfA^{\calC}$ satisfy the two-out-of-three property: this follows immediately from the fact that the weak equivalences in $\bfA$ satisfy the two-out-of-three property.
%\item[$(5)$] Let $f: X \rightarrow Y$ be a morphism in $\bfA$ which has the right lifting property with respect to every injective cofibration. In particular, $f$ has the right lifting property with respect to each of the morphisms in the class $I$ defined above, so that $f$ is a trivial projective fibration, and in particular a weak equivalence.
%\end{itemize}
%\end{proof}

%\begin{remark}
%In the situation of Proposition \ref{smurf}, if $\bfA$ is assumed to be right or left proper, then $\bfA^{\calC}$ is likewise right or left proper (with respect to either the projective or the injective model structures). 
%\end{remark}

\begin{remark}\label{postsmurf}
For each object $C \in \calC$ and each $A \in \bfA$, let
$\calF^{C}_{A} \in \bfA^{\calC}$ be the functor given by
$$D \mapsto A \otimes \bHom_{\calC}(C,D).$$
As in the proof of Proposition \ref{smurff}, we learn that the
class of projective cofibrations in $\bfA^{\calC}$ is generated by
cofibrations of the form $j: \calF^{C}_{A} \rightarrow \calF^{C}_{A'}$, where
$A \rightarrow A'$ is a cofibration in $\bfA$. It follows that every projective cofibration is a injective cofibration; dually, every injective fibration is a projective fibration.
\end{remark}

As in \S \ref{qlim7}, the construction $(\calC, \bfA) \mapsto \bfA^{\calC}$ is functorial
in both $\calC$ and $\bfA$. We summarize the situation in the following
propositions, whose proofs are left to the reader:

\begin{proposition}
Let $\bfS$ be an excellent model category, $\calC$ a small $\bfS$-enriched model category, and
$\Adjoint{F}{\bfA}{\bfS}{G}$ a $\bfS$-enriched Quillen adjunction between
combinatorial $\bfS$-enriched model categories. The composition with $F$ and $G$
determines another $\bfS$-enriched Quillen adjunction
$$ \Adjoint{ F^{\calC} }{ \bfA^{\calC} }{ \bfB^{\calC} }{G^{\calC} },$$
with respect to either the projective or injective model structures.
Moreover, if $(F,G)$ is a Quillen equivalence, then $(F^{\calC}, G^{\calC})$ is also a Quillen equivalence.
\end{proposition}

Because the projective and injective model structures on
$\bfA^{\calC}$ have the same weak equivalences, the identity
functor $\id_{ \bfA^{\calC}}$ is a Quillen equivalence between them. However, it is
important to keep distinguish these two model structures, because
they have different variance properties as we now explain.

Let $f: \calC \rightarrow \calC'$ be a $\bfS$-enriched functor. Then
composition with $f$ yields a pullback functor $f^{\ast}:
\bfA^{\calC'} \rightarrow \bfA^{\calC}$. Since $\bfA$ has all
$\bfS$-enriched limits and colimits, $f^{\ast}$ has a
right adjoint which we shall denote by $f_{\ast}$ and a left
adjoint which we shall denote by $f_{!}$.

\begin{proposition}\label{colbin}
Let $\bfS$ be an excellent model category, $\bfA$ a combinatorial $\bfS$-enriched
model category, and $f: \calC \rightarrow \calC'$ a $\bfS$-enriched functor between
small $\bfS$-enriched categories. Let
$f^{\ast}: \bfA^{\calC'} \rightarrow \bfA^{\calC}$ be given
by composition with $f$. Then $f^{\ast}$ admits a right adjoint
$f_{\ast}$ and a left adjoint $f_{!}$. Moreover:

\begin{itemize}
\item[$(1)$] The pair $( f_{!}, f^{\ast} )$ determines a Quillen
adjunction between the {\em projective} model structures on
$\bfA^{\calC}$ and $\bfA^{\calC'}$.

\item[$(2)$] The pair $( f^{\ast}, f_{\ast} )$ determines a
Quillen adjunction between the {\em injective} model structures on
$\bfA^{\calC}$ and $\bfA^{\calC'}$.
\end{itemize}
\end{proposition}

We now study some aspects of the theory which are unique to the enriched context.

\begin{proposition}\label{lesstrick}
Let $\bfS$ be an excellent model category,
$\bfA$ a combinatorial $\bfS$-enriched model category, and 
$f: \calC \rightarrow \calC'$ an equivalence of small $\bfS$-enriched categories. 
Then:

\begin{itemize}
\item[$(1)$] The Quillen adjunction $(f_{!}, f^{\ast})$ determines a Quillen equivalence between
the projective model structures on $\bfA^{\calC}$ and $\bfA^{\calC'}$. 

\item[$(2)$] The Quillen adjunction $(f^{\ast}, f_{\ast})$ determines a Quillen equivalence between
the injective model structures on $\bfA^{\calC}$ and $\bfA^{\calC'}$.
\end{itemize}

\end{proposition}

%Before giving the proof, we need a lemma.

%\begin{lemma}\label{style}
%Let $\bfA$ be a left proper simplicial model category, let
%$f: A \rightarrow A'$ be a cofibration in $\bfA$, and let $g: K \rightarrow K'$ be a weak equivalence in $\sSet$. Then the induced map
%$$ \theta_{g}: (A \otimes K') \coprod_{ A \otimes K } (A' \otimes K) \rightarrow A' \otimes K'$$
%is a weak equivalence in $\bfA$.
%\end{lemma}

%\begin{proof}
%If $g$ is a cofibration, then $\theta_{g}$ is a trivial cofibration.
%More generally, suppose that $g$ is given as a composition
%$$ K \stackrel{g'}{\rightarrow} K'' \stackrel{g''}{\rightarrow} K',$$
%where $g'$ is a trivial cofibration. Then $\theta_{g} = \theta_{g''} \circ \theta'_{g'}$, where $\theta'_{g'}$ is a pushout of $\theta_{g'}$, and therefore a trivial cofibration. 
%It follows that $\theta_{g}$ is an equivalence if and only if $\theta_{g''}$ is an equivalence.
%Using this argument, we may reduce to the case where $g$ is a trivial fibration.

%We now observe that $g$ admits a section $s: K' \rightarrow K$, which is a trivial cofibration.
%Applying the above argument to the diagram
%$$ K' \stackrel{s}{\rightarrow} K \stackrel{g}{\rightarrow} K',$$
%we deduce that $\theta_g$ is an equivalence if and only if $\theta_{\id_{K}}$ is an equivalence. 
%But $\theta_{\id_{K}}$ is an isomorphism of simplicial sets.
%\end{proof}

\begin{proof}
We first note that $(1)$ and $(2)$ are equivalent: they are both equivalent to the assertion that
$f^{\ast}$ induces an equivalence on homotopy categories. It therefore suffices to prove $(1)$.
We first prove this under the following additional assumption:
\begin{itemize}
\item[$(\ast)$] For every pair of objects $C, D \in \calC'$, the map
$$ \bHom_{\calC'}( C,D) \rightarrow \bHom_{\calC}( f(C), f(D) )$$
is a cofibration in $\bfS$.
\end{itemize}
Let $Lf_{!}: \bfA^{\calC} \rightarrow \bfA^{\calC'}$ denote the left derived functor of $f_{!}$. We must show that the unit and counit maps
$$ h_F: F \mapsto f^{\ast} Lf_{!} F $$
$$ k_G: Lf_{!} f^{\ast} G \rightarrow G$$
are isomorphisms for all $F \in \h{\bfA^{\calC}}$, $G \in \h{\bfA^{\calC}}$. Since $f$ is essentially surjective on homotopy categories, a natural transformation $K \rightarrow K'$ of $\bfS$-enriched functors $K,K': \calC' \rightarrow \bfA$ is a weak equivalence if and only if $f^{\ast} K \rightarrow f^{\ast} K'$ is a weak equivalence. Consequently, to prove $k_G$ is an isomorphism, it suffices to show that
$h_{f^{\ast} G}$ is an isomorphism. 

Let us say that a map $F \rightarrow F'$ in $\bfA^{\calC}$ is {\it good} if the
induced map $f^{\ast} f_{!} F \coprod_{F} F' \rightarrow f^{\ast} f_{!} F'$
is a weak trivial cofibration. To complete the proof, it will suffice to show that
every projective cofibration is good. Since the collection of good transformations
is weakly saturated, it will suffice to show that each of the generating cofibrations
$\calF^{C}_{A} \rightarrow \calF^{C}_{A'}$ is good, where $C \in \calC'$ and
$j: A \rightarrow A'$ is a cofibration in $\bfA$. Unwinding the definitions, we must show that
for each $D \in \calC'$ the induced map
$$ \theta: (A' \otimes \bHom_{\calC'}(C, D)) \coprod_{ A \otimes \bHom_{\calC'}(C,D)}
(A \otimes \bHom_{\calC}(f(C), f(D))) \rightarrow A' \otimes \bHom_{\calC}( f(C), f(D) )$$
is a trivial cofibration. This follows from the fact that $j$ is a cofibration and our assumption $(\ast)$.

We now treat the general case. First, choose a trivial cofibration $g: \calC \rightarrow \calC''$, where
$\calC''$ is fibrant. Then $g$ satisfies $(\ast)$, so $g_!$ is a Quillen equivalence.
By a two-out-of-three argument, we see that $f_!$ is a Quillen equivalence if and only if
$(g \circ f)_!$ is a Quillen equivalence. Replacing $\calC$ by $\calC''$, we may
reduce to the case where $\calC$ is itself fibrant.

Choose a cofibration $j: \calC \coprod \calC' \rightarrow \calD$, where
$\calD$ is fibrant and equivalent to the final object of $\SCat$. Then $f$ factors as a composition
$$ \calC' \stackrel{f'}{\rightarrow} \calC \times \calD \stackrel{f''}{\rightarrow} \calC.$$
Since $\calC$ and $\calD$ are fibrant, the product $\calC \times \calD$ is equivalent
to $\calC$. Moreover, the map $f''$ admits a section $s: \calC \rightarrow \calC \times \calD$.
Using another two-out-of-three argument, it will suffice to show that $f'_{!}$ and $s_!$ are Quillen equivalences. For this, it will suffice to show that $f'$ and $s$ satisfy $(\ast)$.

We first show that $f'$ satisfies $(\ast)$. Fix a pair of objects $X,Y \in \calC'$.
Then $f'$ induces the composite map
$$ \bHom_{\calC'}(X,Y) \stackrel{u}{\rightarrow} \bHom_{\calC}(fX,fY) \times \bHom_{\calC'}(X,Y)
\stackrel{u'}{\rightarrow} \bHom_{\calC}(fX,fY) \times \bHom_{\calD}(jX,jY) \simeq \bHom_{\calC \times \calD}(f'X, f'Y).$$
The map $u$ is a monomorphism (since it admits a left inverse), and therefore a cofibration
in view of axiom $(A2)$ of Definition \ref{modelexcellent}. 
The map $u'$ is a product of cofibrations, and therefore a cofibration (again by axiom $(A2)$).

The proof that $s$ satisfies $(\ast)$ is similar: for every pair of objects $U,V \in \calC$,
the map $$ \bHom_{\calC}(U,V) \rightarrow \bHom_{ \calC \times \calD}( sU, sV)
\simeq \bHom_{ \calC}( U,V) \times \bHom_{\calD}(jU,jV)$$
is a monomorphism since it admits a left inverse, and is therefore a cofibration.
\end{proof}

In the special case where $f: \calC \rightarrow \calC'$ is a {\em cofibration} between
$\bfS$-enriched categories, we have some additional functoriality:

\begin{proposition}\label{sumner}
Let $\bfS$ be an excellent model category and let $f: \calC \rightarrow \calC'$ be a cofibration
of small $\bfS$-enriched categories. Then:
\begin{itemize}

\item[$(1)$] For every combinatorial $\bfS$-enriched model category $\bfA$, the pullback map
$f^{\ast}: \bfA^{\calC'} \rightarrow \bfA^{\calC}$ preserves projective cofibrations. 

\item[$(2)$] For every projectively cofibrant object $F \in \bfS^{\calC}$, the
unit map $F \rightarrow f^{\ast} f_{!} F$ is a projective cofibration.

\end{itemize}
\end{proposition}

\begin{lemma}\label{pseudopod}
Let $\bfS$ be an excellent model category, and suppose given a pushout diagram
$$ \xymatrix{ [1]_{S} \ar[r] \ar[d]^{i} & [1]_{S'} \ar[d] \\
\calC \ar[r]^{f} & \calC' }$$
of $\bfS$-enriched categories, where $j: S \rightarrow S'$ is a cofibration in $\bfS$.
Let $C$ be an object of $\calC$, and let $F \in \bfS^{\calC}$ be the functor
given by the formula $D \mapsto \bHom_{\calC}(C, D)$. Then
the unit map $F \rightarrow f^{\ast} f_{!} F$ is a projective cofibration in
$\bfS^{\calC}$.
\end{lemma}

\begin{proof}
The map $i$ determines a pair of objects $X,Y \in \calC$, and a map
$S \rightarrow \bHom_{\calC}(X,Y)$. The proof of Proposition \ref{enrichcatper}
shows that the functor $f^{\ast} f_{!} F$ is the colimit of a sequence
$$F = F(0) \stackrel{h_1}{\rightarrow} F(1) \stackrel{h_2}{\rightarrow} F(2) \rightarrow \ldots,$$
where each $h_k$ is a pushout of a map 
$\calF^{Y}_{A} \rightarrow \calF^{Y}_{A'}$ induced by a map
$t: A \rightarrow A'$ in $\bfS$. Moreover, the map
$t$ can be identified with the tensor product 
$$\id_{ \bHom_{\calC}(C, X)} \otimes \id_{ \bHom_{\calC}(Y,X) }^{\otimes k-1}
\otimes \wedge^{k}(j),$$ 
where $\wedge^{k}(j)$ denotes the $k$th pushout power of $j$. 
It follows that $t$ is a cofibration in $\bfS$, so that each $h_k$ is a projective cofibration
in $\bfS^{\calC}$. 
\end{proof}

\begin{proof}[Proof of Proposition \ref{sumner}]
The collection of $\bfS$-enriched functors $f$ which satisfy $(1)$ and $(2)$ is clearly
closed under the formation of retracts. We may there assume without loss of generality
that $f$ is a transfinite composition of pushouts of generating cofibrations
(see the discussion preceding Proposition \ref{enrichcatper}). Reordering
these pushouts if necessary, we can factor $f$ as a composition
$$ \calC \stackrel{f'}{\rightarrow} \overline{\calC} \stackrel{f''}{\rightarrow} \calC'$$
where $\overline{\calC}$ is obtained from $\calC$ by freely adjoining a collection of
new objects, and $f''$ is bijective on objects. Since $f'$ clearly satisfies
$(1)$ and $(2)$, it will suffice to prove that $f''$ satisfies $(1)$ and $(2)$.
Replacing $f$ by $f''$, we may assume that $f$ is bijective on objects.

We now show that $(2) \Rightarrow (1)$. Since the functor $f^{\ast}$ preserves colimits, the collection of morphisms $\alpha$ in $\bfA^{\calC'}$ such that $f^{\ast}$ is a projective cofibration in $\bfA^{\calC}$ is weakly saturated. It will therefore suffice to that for every object $X \in \calC'$ and every cofibration $A \rightarrow A'$ in $\bfA$, if $\alpha: \calF^{X}_{A} \rightarrow \calF^{X}_{A'}$ denotes the corresponding generating projective cofibration, then $f^{\ast}(\alpha)$ is a projective cofibration in $\bfS$.

There is a canonical left Quillen bifunctor
$$ \boxtimes: \bfS^{\calC} \times \bfA \rightarrow \bfA^{\calC}$$
described by the formula $(F \boxtimes A)(C) = F(C) \otimes A$.
(Here we regard $\bfS^{\calC}$ as endowed with the projective model structure.)
We observe that $f^{\ast}(\alpha)$ is the induced map
$(f^{\ast} F) \boxtimes A \rightarrow (f^{\ast} F) \boxtimes A'$, where
$F \in \bfS^{\calC'}$ is given by $F(C') = \bHom_{\calC'}( X, C')$. 
To prove $(1)$, it will suffice to show that $f^{\ast} F$ is projectively cofibrant.

Since $F$ is bijective on objects, we can choose an object $X_0 \in \calC$ such that
$fX_0 = X$. We now observe that $F \simeq f_{!} F_0$, where $F_0 \in \bfS^{\calC}$
is defined by the formula $F_0(C) = \bHom_{\calC}(X_0, C)$. If $(2)$ is satisfied, then
the unit map $F_0 \rightarrow f^{\ast} F$ is a projective cofibration in $\bfS^{\calC}$.
Since $F_0$ is projectively cofibrant, we conclude that $f^{\ast} F$ is projectively cofibrant as well.
This completes the proof that $(2) \Rightarrow (1)$.

To prove $(2)$, let us write $f$ as a transfinite composition of $\bfS$-enriched functors
$$ \calC = \calC_0 \rightarrow \calC_1 \rightarrow \ldots, $$
each of which is a pushout of a generating cofibration of the form $[1]_{S} \rightarrow [1]_{S'}$, where
$S \rightarrow S'$ is a cofibration in $\bfS$. 
For each $\alpha \leq \beta$, let $f^{\beta}_{\alpha}: \calC_{\alpha} \rightarrow \calC_{\beta}$
be the corresponding cofibration. We will prove that the following statement holds, for
every pair of ordinals $\alpha \leq \beta$:
\begin{itemize}
\item[$(2_{\alpha,\beta})$] For every projectively cofibrant object $F \in \bfS^{\calC_{\alpha} }$,
the unit map $u: F \rightarrow (f_{\alpha}^{\beta})^{\ast} (f_{\alpha}^{\beta})_{!} F$
is a projective cofibration.
\end{itemize}

The proof proceeds by induction on $\beta$. We observe that $u$ is a transfinite
composition of maps of the form
$$ u_{\gamma}: (f_{\alpha}^{\gamma})^{\ast} (f_{\alpha}^{\gamma})_{!} F \rightarrow
(f_{\alpha}^{\gamma})^{\ast} (f_{\gamma}^{\gamma+1})^{\ast}
(f_{\gamma}^{\gamma+1})_{!} (f_{\alpha}^{\gamma})_{!} F,$$
where $\gamma < \beta$. It will therefore suffice to show that each $u_{\gamma}$ is a projective cofibration. Our inductive hypothesis therefore guarantees that
$(2_{\alpha, \gamma})$ holds, so the first part of the proof shows that
$(f_{\alpha}^{\gamma})^{\ast}$ preserves trivial cofibrations. We are therefore reduced to proving
assertion $(2_{\gamma, \gamma+1})$. In other words, to prove $(2)$ in general, it will suffice to
treat the case in which $f$ is a pushout of a generating cofibration of the form $[1]_{S} \rightarrow [1]_{S'}$. 

We will in fact prove the following stronger version of $(2)$:
\begin{itemize}
\item[$(3)$] For every projective cofibration $\phi: F' \rightarrow F$ in $\bfS^{\calC}$, the induced map
$\phi': F \coprod_{F'} f^{\ast} f_{!} F' \rightarrow f^{\ast} f_{!} F$ is again a projective cofibration in 
$\bfS^{\calC}$. 
\end{itemize}
Consider the collection of {\em all} morphisms $\phi: F' \rightarrow F$ in $\bfS^{\calC}$ such that
the induced map $\phi': F \coprod_{F'} f^{\ast} f_{!} F' \rightarrow f^{\ast} f_{!} F$ is
a projective cofibration. It is easy to see that this collection is weakly saturated. Consequently, to prove
$(3)$ it suffices to treat the case where $\phi$ is a generating projective cofibration of the form
$\calF^{C}_{A} \rightarrow \calF^{C}_{A'}$, where $A \rightarrow A'$ is a cofibration in $\bfS$. 
In this case, we can identify $\phi'$ with the map 
$$ (F_C \boxtimes A') \coprod_{ F_C \boxtimes A } (f^{\ast} f_{!} F_C \boxtimes A) \rightarrow f^{\ast} f_{!} F_C \boxtimes A',$$
where $F_C \in \bfS^{\calC}$ is the functor $D \mapsto \bHom_{\calC}(C, D)$. Since
$\boxtimes$ is a left Quillen bifunctor, it will suffice to show that the unit map
$f_C \rightarrow f^{\ast} f_{!} F_C$ is a projective cofibration in $\bfS^{\calC}$. This is precisely the content of Lemma \ref{pseudopod}.
\end{proof}

In \S \ref{qlim7}, we introduced the definitions of homotopy limits and colimits in an arbitrary
combinatorial model category $\bfA$. We now discuss an analogous construction in the case where $\bfA$ is enriched over an excellent model category $\bfS$. To simplify the exposition, we will discuss only the case of homotopy limits; the case of homotopy colimits is entirely dual and left to the reader.

Fix an excellent model category $\bfS$ and a combinatorial $\bfS$-enriched model category $\bfA$. Let $f: \calC \rightarrow \calC'$ be a functor between small $\bfS$-enriched categories,
so that we have an induced Quillen adjunction
$$ \Adjoint{f^{\ast}}{\bfA^{\calC'}}{\bfA^{\calC}.}{f_{\ast}}.$$
We will refer to the right derived functor $Rf_{\ast}$ as the {\it homotopy right Kan extension} functor.\index{gen}{Kan extension!homotopy}\index{gen}{homotopy right Kan extension}\index{gen}{right Kan extension!homotopy} If we are given a pair of functors $F \in \bfA^{\calC}$, $G \in \bfA^{\calC'}$, and let $\eta: G \rightarrow f_{\ast} F$ be a map in $\bfA^{\calC'}$. We will say that $\eta$ {\it exhibits $G$ as the homotopy right Kan extension of $F$}
if, for some weak equivalence $F \rightarrow F'$ where $F'$ is injectively fibrant in $\bfA^{\calC}$, the
composite map $G \rightarrow f_{\ast} F \rightarrow f_{\ast} F'$ is a weak equivalence in
$\bfA^{\calC'}$. Since $f_{\ast}$ preserves weak equivalences between injectively fibrant objects, this condition is independent of the choice of $F'$.\index{gen}{Kan extension!homotopy}

\begin{remark}
In \S \ref{qlim7}, we defined homotopy right Kan extensions in the setting of
the diagram categories $\Fun( \calC, \bfA)$, where $\calC$ is an ordinary category.
In fact, this is a special case of the above construction. Namely, there is a unique colimit-preserving monoidal functor $F: \Set \rightarrow \bfS$, given by $F(S) = \coprod_{s \in S} {\bf 1}_{\bfS}$. We can therefore define a $\bfS$-enriched category $\overline{\calC}$ whose objects are the objects of $\calC$, with $\bHom_{ \overline{\calC} }(X,Y) = F \bHom_{\calC}(X,Y)$. We
now observe that we have an identification $\Fun( \calC, \bfA) \simeq \bfA^{\overline{\calC}}$, which is functorial in both $\calC$ and $\bfA$. This identification is compatible with the definition
of the injective model structures on both sides, so that either point of view gives rise to the same theory of homotopy right Kan extensions.
\end{remark}

We now discuss a special feature of the enriched theory of homotopy Kan extensions:
they can be reduced to the theory of homotopy Kan extensions in the model category $\bfS$:

\begin{proposition}\label{usecoinc}
Let $\bfS$ be an excellent model category, $\bfA$ a combinatorial model category
enriched over $\bfS$, and let $f: \calC \rightarrow \calC'$ be a functor between
small $\bfS$-enriched categories. Suppose
given objects $F \in \bfA^{\calC}$, $G \in \bfA^{\calC'}$, and a map $\eta: G \rightarrow f_{\ast} F$. 
Assume that $F$ and $G$ are projectively fibrant.
The following conditions are equivalent:

\begin{itemize}
\item[$(1)$]  The map $\eta$ exhibits $G$ as a homotopy right Kan extension of $F$.

\item[$(2)$] For each cofibrant object $A \in \bfA$, the induced map
$$ \eta_A: G_A \rightarrow f_{\ast} F_A$$
exhibits $G_A$ as a homotopy right Kan extension of $F_A$. Here $F_A \in \bfS^{\calC}$
and $G_A \in \bfS^{\calC'}$ are defined by $F_A(C) = \bHom_{\bfA}(A,F(C)), G_A(C) = \bHom_{\bfA}(A,G(C))$. 

\item[$(3)$] For every fibrant-cofibrant object $A \in \bfA$, the induced map
$$ \eta_A: G_A \rightarrow f_{\ast} F_A$$
exhibits $G_A$ as a homotopy right Kan extension of $F_A$.
\end{itemize}
\end{proposition}

\begin{proof}
Choose an equivalence $F \rightarrow F'$, where $F'$ is injectively fibrant. We note that
the induced maps $F_A \rightarrow F'_A$ are weak equivalences for any
cofibrant $A \in \bfA$, since $\bHom_{\bfA}(A, \bigdot)$ preserves weak equivalences between fibrant objects. Consequently, we may without loss of generality replace $F$ by $F'$ and thereby assume that $F$ is injectively fibrant.

Now suppose that $A$ is any cofibrant object of $\bfA$; we claim that
$F_A$ is injectively fibrant. To show that $F_A$ has
the right lifting property with respect to a trivial weak
cofibration $H \rightarrow H'$ of functors $\calC \rightarrow
\bfS$, one need only observe that $F$ has the right lifting
property with respect to trivial injective cofibration $A \otimes H
\rightarrow A \otimes H'$ in $\bfA^{\calC}$.

Now we note that $(1)$ is equivalent to the assertion that $\eta$ is a weak equivalence, $(2)$
is equivalent to the assertion that $\eta_{A}$ is a weak equivalence for any cofibrant object
$A$, and $(3)$ is equivalent to the assertion that $\eta_A$ is a weak equivalence whenever
$A$ is fibrant-cofibrant. Because $\bHom_{\bfA}(A, \bigdot)$ preserves weak equivalences between fibrant objects, we deduce that $(1) \Rightarrow (2)$. It is
clear that $(2) \Rightarrow (3)$. We will complete the proof by showing that $(3) \Rightarrow (1)$.
Assume that $(3)$ holds; we must show that
$\eta(C'): G(C') \rightarrow f_{\ast} F(C')$ is an isomorphism in the homotopy category $\h{\bfA}$,
for each $C' \in \calC'$.
For this, it suffices to show that $G(C')$ and $f_{\ast} F(C')$ represent the same $\calH$-valued functors on the homotopy category $\h{\bfA}$, which is precisely the content of $(3)$. 
\end{proof}

\begin{remark}\label{curble}
It follows from Proposition \ref{usecoinc} that we can make sense of homotopy right Kan extensions for diagrams
which do not take values in a model category. 
Let $f: \calC \rightarrow \calC'$ be a $\bfS$-enriched functor as in the discussion above, and let $\calA$ be an {\em arbitrary} locally fibrant $\bfS$-enriched category. Suppose given objects $F \in \calA^{\calC}$, $G \in \calA^{\calC'}$, and $\eta: f^{\ast} G \rightarrow F$
we say that $\eta$ {\it exhibits $G$ as a homotopy right
Kan extension of $F$} if, for each object $A \in \calA$, the induced map
$$\eta_A: G_A \rightarrow f_{\ast} F_A$$ exhibits $G_A \in \bfS^{\calC'}$ as a homotopy right Kan extension of $F_A \in \bfS^{\calC}$.\index{gen}{homotopy limit}\index{gen}{limit!homotopy}

Suppose that the monoidal structure on $\bfS$ is given by the Cartesian product, and take
$\calC'$ to be the final object of $\SCat$, so that we can identify $\calA^{\calC'}$ with $\calA$.
In this case, we can identify $G$ with a single object $B \in \calA$, and the map $\eta$ with
a collection of maps $ \{ B \rightarrow F(C) \}_{C \in \calC}$. We will say that $\eta$ 
{\it exhibits $B$ as a homotopy limit of $F$} if it identifies $G$ with a homotopy right Kan extension of $F$. In other words, $\eta$ exhibits $B$ as a homotopy limit of $F$ if, for every object
$A \in \calA$, the induced map
$$ \bHom_{\calA}(A,B) \rightarrow \lim \{ \bHom_{\calA}(A, F(C)) \}_{C \in \calC}$$
exhibits $\bHom_{\calA}(A,B)$ as a homotopy limit of the diagram
$\{ \bHom_{\calA}(A, F(C) \}_{C \in \calC}$ in the model category $\bfS$.

We also have a dual notion of {\em homotopy colimit} in an arbitrary fibrant
$\bfS$-enriched category $\calA$: a compatible family of maps $\{ F(C) \rightarrow B \}_{C \in \calC}$ {\it exhibits $B$ as a homotopy colimit of $F$} if, for every object $A \in \calA$, the
induced maps $\{ \bHom_{\calA}( B, A) \rightarrow \bHom_{\calA}( F(C), A) \}_{C \in \calC}$
exhibit $\bHom_{\calA}(B,A)$ as a homotopy limit of the diagram
$\{ \bHom_{\calA}( F(C), A) \}_{C \in \calC}$ in $\bfS$. \index{gen}{homotopy colimit}\index{gen}{colimit!homotopy}
\end{remark}

\begin{remark}
In view of Proposition \ref{usecoinc}, the terminology introduced in Remark \ref{curble} for general $\calA$ agrees with the terminology introduced for a combinatorial $\bfS$-enriched model category $\bfA$ if we set $\calA = \bfA^{\degree}$. We remark that, in general, the two notions do {\em not} agree if
we take $\calA = \bfA$, so that our terminology is potentially ambiguous; however, we feel that there is little danger of confusion.
\end{remark}

We conclude this section by giving the proof of Lemma \ref{mainertime}. 
Let $\bfA$ be a presentable category which is enriched, tensored, and cotensored over a presentable category $\bfS$. Let $\calC$ be a small $\bfS$-enriched category, and $\overline{S}_0$ a weakly saturated class of morphisms of $\bfA$ generated by a (small) set $S_0$. We regard this data as {\em fixed} for the remainder of this section.

Choose a regular cardinal $\kappa$ satisfying the following conditions:
\begin{itemize}
\item[$(i)$] The cardinal $\kappa$ is uncountable.

\item[$(ii)$] The category $\calC$ has fewer than $\kappa$-objects.

\item[$(iii)$] Let $X,Y \in \calC$, and let $K = \bHom_{\calC}(X,Y)$. Then the functor
from $\bfA$ to itself given by the formula $A \mapsto A^{K}$ preserves $\kappa$-filtered colimits.
This implies, in particular, that the collection of $\kappa$-compact objects of $\bfA$ is stable with respect to the functors $\bigdot \otimes K$.

\item[$(iv)$] The category $\bfA$ is $\kappa$-accessible. It follows also that $\bfA^{\calC}$ is
$\kappa$-accessible, and that an object $F \in \bfA^{\calC}$ is $\kappa$-compact if and only if
each $F(C) \in \bfA$ is $\kappa$-compact. We prove an $\infty$-category generalization of this
statement as Proposition \ref{horse1}. The same proof also works in the setting of ordinary categories.

\item[$(v)$] The source and target of every morphism in $S_0$ is a $\kappa$-compact object of $\bfA$.
\end{itemize}

Enlarging $S_0$ if necessary, we may assume that $S_0$ consists of {\em all} morphisms in
$f \in \overline{S}_0$ such that the source and target of $f$ are $\kappa$-compact.
Let $S$ be the collection of all injective cofibrations between $\kappa$-compact objects of $\bfA$ (in view of $(iv)$, we can equally well define $S$ to be the set of morphisms $F \rightarrow G$ in
$\bfA^{\calC}$ such that each of the induced morphisms $F(C) \rightarrow G(C)$ belongs to $S_0$). Let $\overline{S}$ be the weakly saturated class of morphisms in $\bfA^{\calC}$ generated by $S$, and choose a map $f: F \rightarrow G$ in $\bfA^{\calC}$ such that $f(C) \in \overline{S}_0$ for each $C \in \calC$. We wish to show that
$f \in \overline{S}$. Corollary \ref{unitape} implies that, for each $C \in \calC$, there exists a $\kappa$-good $S_0$-tree $\{ Y(C)_{\alpha} \}_{\alpha \in A(C)}$ with root $F(C)$ and colimit $G(C)$.

Let us define a {\it slice} to be the following data:
\begin{itemize}
\item[$(a)$] For each object $C \in \calC$, a downward-closed subset $B(C) \subseteq A(C)$.
\item[$(b)$] For every object $C \in \calC$, a morphism
$\eta_{C}: \coprod_{C' \in \calC} Y(C')_{B(C')} \otimes \bHom_{\bfA}(C',C) \rightarrow Y(C)_{B(C)}$, rendering the following diagrams commutative:
$$ \xymatrix{ \coprod_{C'',C' \in \calC} Y(C'')_{B(C'')} \otimes \bHom_{\bfA}(C'',C') \otimes \bHom_{\bfA}(C',C) \ar[r] \ar[d]^{\eta_{C'}} &
\coprod_{C'' \in \calC} Y(C'')_{B(C'')} \otimes \bHom_{\bfA}(C'',C) \ar[d]^{\eta_{C''}} \\
\coprod_{C' \in \calC} Y(C')_{B(C')} \otimes \bHom_{\bfA}(C',C) \ar[r]^{\eta_{C}} & Y(C)_{B(C)} }$$

$$ \xymatrix{ \coprod_{C' \in \calC} F(C') \otimes \bHom_{\bfA}(C',C) \ar[r] \ar[d] & F(C) \ar[d] \\
 \coprod_{C' \in \calC} Y(C')_{B(C')} \otimes \bHom_{\bfA}(C',C) \ar[d] \ar[r]^-{\eta_{C}} &
Y(C)_{B(C)} \ar[d] \\
\coprod_{C' \in \calC} G(C')  \otimes \bHom_{\bfA}(C',C) \ar[r] & G(C). }$$
\end{itemize}

We remark that $(b)$ is precisely the data needed to endow $C \mapsto Y(C)_{B(C)}$ with the structure of a $\bfS$-enriched functor $\calC \rightarrow \bfA$, lying between $F$ and $G$ in $\bfA^{\calC}$. 

\begin{lemma}\label{umpin}
Suppose given a collection of $\kappa$-small subsets $\{ B_0(C) \subseteq A(C) \}_{C \in \calC}$. Then there exists a slice $\{ (B(C), \eta_{C} \}_{C \in \calC}$
such that each $B(C)$ is a $\kappa$-small subset of $A(C)$ containing $B_0(C)$.
\end{lemma}

\begin{proof}
Enlarging each $B_0(C)$ if necessary, we may assume that each $B_0(C)$ is closed downwards. 
Note that because each $\{ Y(C)_{\alpha} \}_{\alpha \in A(C)}$ is a $\kappa$-good $S_0$-tree, if
$A' \subseteq A(C)$ is closed downward and $\kappa$-small, $Y(C)_{A'}$ is $\kappa$-compact when viewed as an object of $\bfA_{F(C)/}$. It follows from $(iii)$ that each $Y(C)_{B_0(C)} \otimes \bHom_{\bfA}(C,C')$ is a $\kappa$-compact object of $\bfA_{( F(C) \otimes \bHom_{\bfA}(C,C') )/}$. Consequently, each composition
$$ \coprod_{C' \in \calC} Y(C')_{B_0(C')} \otimes \bHom_{\bfA}(C',C)
{\rightarrow} \coprod_{C' \in \calC} G(C') \otimes \bHom_{\bfA}(C',C) 
\rightarrow G(C)$$
admits another factorization
$$ \coprod_{C' \in \calC} Y(C')_{B_0(C')} \otimes \bHom_{\bfA}(C',C)
\stackrel{\eta^1_{C}}{\rightarrow} Y(C)_{B_1(C)} \rightarrow G(C),$$
where $B_1(C)$ is downward closed and $\kappa$-small, and the diagram
$$ \xymatrix{ \coprod_{C' \in \calC} F(C') \otimes \bHom_{\bfA}(C',C) \ar[d] \ar[r] & 
\coprod_{C' \in \calC} Y(C')_{B_0(C')} \ar[d]^{\eta^1_C} \ar[d] \\
F(C) \ar[r] & Y(C)_{B_1(C)}}$$
commutes. Enlarging $B_1(C)$ if necessary, we may suppose that each $B_1(C)$ contains $B_0(C)$.

We now continue the preceding construction by defining, for each $C \in \calC$, a sequence of $\kappa$-small, downward closed subsets
$$ B_0(C) \subseteq B_1(C) \subseteq B_2(C) \subseteq \ldots $$
of $A(C)$, and maps 
$\eta^{i}_{C}: \coprod_{C' \in \calC} Y(C')_{B_{i-1}(C')} \otimes
\bHom_{\bfA}(C',C) \rightarrow Y(C)_{B_{i}(C)}$. 
Suppose that $i > 1$, and that the sets $B_{j}(C)$ and maps
$\eta^j_{C}$ have been constructed for $j < i$. Using a compactness argument, we conclude that the composition
$$ \coprod_{C' \in \calC} Y(C')_{B_{i-1}(C')} \otimes \bHom_{\bfA}(C',C)
\rightarrow \coprod_{C' \in \calC} G(C') \otimes \bHom_{\bfA}(C',C)
\rightarrow G(C)$$
coincides with
$$ \coprod_{ C' \in \calC} Y(C')_{B_{i-1}(C')} \otimes \bHom_{\bfA}(C',C)
\stackrel{\eta^i_{C}}{\rightarrow} Y(C)_{B_{i}(C)} \rightarrow G(C),$$
where $B_{i}(C)$ is $\kappa$-small and the diagram 
$$ \xymatrix{ \coprod_{C' \in \calC} F(C') \otimes \bHom_{\bfA}(C',C) \ar[d] \ar[r] & 
\coprod_{C' \in \calC} Y(C')_{B_{i-1}(C')} \otimes \bHom_{\bfA}(C',C) \ar[d]^{\eta^i_C} \ar[d] \\
F(C) \ar[r] & Y(C)_{B_i(C)}}$$
commutes. Enlarging $B_{i}(C)$ if necessary, we may suppose that $B_{i}(C)$ contains
$B_{i-1}(C)$ and that the following diagrams commute as well:
$$ \xymatrix{ \coprod_{C', C'' \in \calC} Y(C'')_{ B_{i-2}(C'')} \otimes \bHom_{\bfA}(C'',C')
\otimes \bHom_{\bfA}(C',C) \ar[r] \ar[d] & \coprod_{C'' \in \calC} Y(C'')_{B_{i-1}(C'') }
\otimes \bHom_{\bfA}(C'',C) \ar[d]^{\eta_{C}^{i}} \\
\coprod_{C' \in \calC} Y(C')_{B_{i-1}(C')} \otimes \bHom_{\bfA}(C',C) \ar[r]^{\eta^{i}_{C}} & 
Y(C)_{B_i(C)}}$$
$$ \xymatrix{ \coprod_{C' \in \calC} Y(C')_{B_{i-2}(C')} \otimes \bHom_{\bfA}(C',C) \ar[r] \ar[d]^{\eta^{i-1}_{C}} & \coprod_{C' \in \calC} Y(C')_{B_{i-1}(C')} \otimes \bHom_{\bfA}(C',C) \ar[d]^{\eta^{i}_{C}}
\\ Y(C)_{B_{i-1}(C)} \ar[r] & Y(C)_{B_{i}(C)}. }$$
We now define $B(C) = \bigcup B_i(C)$, and $\eta_{C}$ to be the amalgam of the compositions
$$ \coprod_{C' \in \calC} Y(C')_{B_{i-1}(C')} \otimes \bHom_{\bfA}(C',C)
\stackrel{\eta^i_{C}}{\rightarrow} Y(C)_{B_{i}}(C) \rightarrow Y(C)_{B(C)}.$$
\end{proof}

We now introduce a bit more terminology. Suppose given a pair of slices
$M = \{ ( B(C), \eta_C )\}_{C \in \calC}$, $M' = \{ ( B'(C), \eta'_{C} \}_{C \in \calC} \}$. We will say that $M$ is {\it $\kappa$-small} if each $B(C)$ is $\kappa$-small. We will say that $M'$ {\it extends} $M$
if $B(C) \subseteq B'(C)$ for each $C \in \calC$, and each diagram
$$ \xymatrix{ \coprod_{ C' \in \calC} Y(C')_{B(C')} \otimes \bHom_{\bfA}(C',C) \ar[r] \ar[d]^{\eta_C} &
\coprod_{C' \in \calC} Y(C')_{B'(C')} \otimes \bHom_{\bfA}(C',C) \ar[d]^{\eta'_{C}} \\
Y(C)_{B(C)} \ar[r] & Y(C)_{B'(C)} }$$
is commutative. Equivalently, $M'$ extends $M$ if $B(C) \subseteq B'(C)$ for each $C \in \calC$, and the induced maps $Y(C)_{B(C)} \rightarrow Y(C)_{B(C')}$ constitute a natural transformation of simplicial functors from $\calC$ to $\bfA$.

\begin{lemma}\label{goosebed}
Let $M' = \{ ( A'(C), \theta_{C} ) \}_{C \in \calC}$ be a slice, and let 
$\{ B_0(C) \subseteq A(C) \}_{C \in \calC}$ be a collection of $\kappa$-small subsets of
$A(C)$. Then there exists a pair of slices 
$N = \{ ( B(C), \eta_{C} ) \}_{C \in \calC}$, $N' = \{ (B(C) \cap A'(C), \eta'_{C}) \}$ where
$B(C)$ is $\kappa$-small, and $N'$ is compatible with both $N$ and $M'$.
\end{lemma}

\begin{proof}
Let $B'_0(C) = A'(C) \cap B_0(C)$. For every positive integer $i$, we will construct a pair of slices
$N_i = \{ ( B_i(C), \eta(i)_{C} ) \}$, $N'_{i} = \{ (B'_{i}(C), \eta'(i)_{C} ) \}$ so that the following conditions are satisfied:
\begin{itemize}
\item[$(a)$] Each $B_{i}(C)$ is $\kappa$-small and contains $B_{i-1}(C)$.  
\item[$(b)$] Each $B'_{i}(C)$ is $\kappa$-small, contains
$B'_{i-1}(C)$ and the intersection $B_{i}(C) \cap A'(C)$, and is contained in $A'(C)$.
\item[$(c)$] Each $N'_{i}$ is compatible with $M'$.
\item[$(d)$] If $i > 2$ and $C \in \calC$, then the diagram
$$ \xymatrix{ \coprod_{C' \in \calC} Y(C')_{B_{i-2}(C')} \otimes \bHom_{\bfA}(C',C)
\ar[r] \ar[d]^{ \eta(i-2)_{C} } & \coprod_{C' \in \calC} Y(C')_{B_{i-1}(C') } \otimes \bHom_{\bfA}(C',C) \ar[d]^{\eta(i-1)_{C}} \\
Y(C)_{B_{i-2}(C)} \ar[d] & Y(C)_{B_{i-1}}(C) \ar[d] \\
Y(C)_{B_{i}(C)} \ar@{=}[r] & Y(C)_{B_i(C)} }$$
commutes. 
\item[$(e)$] If $i > 2$ and $C \in \calC$, then the diagram
$$ \xymatrix{ \coprod_{C' \in \calC} Y(C')_{B'_{i-2}(C')} \otimes \bHom_{\bfA}(C',C)
\ar[r] \ar[d]^{ \eta'(i-2)_{C} } & \coprod_{C' \in \calC} Y(C')_{B'_{i-1}(C') } \otimes \bHom_{\bfA}(C',C) \ar[d]^{\eta('i-1)_{C}} \\
Y(C)_{B'_{i-2}(C)} \ar[d] & Y(C)_{B'_{i-1}}(C) \ar[d] \\
Y(C)_{B'_{i}(C)} \ar@{=}[r] & Y(C)_{B'_i(C)} }$$
commutes. 
\item[$(f)$] If $i > 1$ and $C \in \calC$, then the diagram
$$ \xymatrix{ \coprod_{C' \in \calC} Y(C')_{B'_{i-1}(C')} \otimes \bHom_{\bfA}(C',C) \ar[r] 
\ar[d]^{\eta'(i-1)_{C}} & \coprod_{C' \in \calC} Y(C')_{B_{i-1}(C')} \otimes \bHom_{\bfA}(C',C) \ar[d]^{\eta(i-1)_{C}} \\
Y(C)_{B'_{i-1}(C)} \ar[d] & Y(C)_{B_{i-1}(C)} \ar[d] \\
Y(C)_{B'_{i}(C) } \ar[r] & Y(C)_{B'_{i}(C)}  }$$
commutes.
\end{itemize}
The construction is by induction on $i$. 
The existence of $N_{i}$ satisfying $(a)$, $(d)$, and $(f)$ follows from Lemma \ref{umpin} (and a compactness argument). Similarly, the existence of $N'_{i}$ satisfying $(b)$, $(c)$, and $(e)$ follows by applying Lemma \ref{umpin} after replacing $G \in \bfA^{\calC}$ by
the functor $G'$ given by $G'(C) = Y(C)_{A'(C)}$, and the $S_0$-trees
$\{ Y(C)_{\alpha} \}_{\alpha \in A(C)}$ by the smaller trees $\{ Y(C)_{\alpha} \}_{ \alpha \in A'(C) }$.  

We now define $B(C) = \bigcup_{i} B_i(C)$. It follows from $(d)$ that the $\eta(i)_{C}$ assemble to
a map $$\eta_{C}: \coprod_{C' \in \calC} Y(C')_{B(C')} \otimes \bHom_{\bfA}(C',C) \rightarrow Y(C)_{B(C)}.$$ Taken together these maps determine a slice $N = \{ (B(C), \eta_{C} ) \}$. Similarly, $(e)$ implies that
the maps $\eta'(i)_{C}$ assemble to a slice $N' = \{ ( B(C) \cap A'(C), \eta'_{C} ) \}$. The compatibility of $N$ and $N'$ follows from $(f)$, while the compatibility of $M'$ and $N'$ follows from $(c)$.
\end{proof}

We now construct a transfinite sequence of compatible slices $\{ M(\gamma) = \{ (B(\gamma)(C), \eta(\gamma)_{C} ) \}_{C \in \calC} \}_{\gamma < \beta}$. The construction is by recursion. Assume that $M(\gamma')$ has been defined for $\gamma' < \gamma$, and let 
$M'(\gamma) = \{ ( B'(\gamma)(C), \eta'(\gamma)_{C} ) \}_{C \in \calC}$ be the slice
obtained by amalgamating the family $\{ M(\gamma') \}_{\gamma' < \gamma}$. If
$B'(\gamma)(C) = A(C)$ for all $C \in \calC$, we set $\beta = \gamma$ and conclude the construction. Otherwise, choose $C \in \calC$ and $a \in A(C) - B'(\gamma)(C)$. According to 
Lemma \ref{goosebed}, there exists a pair of slices
$N(\gamma) = \{ ( B''(C), \theta_C) \}_{C \in \calC}$, 
$N'(\gamma) = \{ ( B''(C) \cap B'(\gamma)(C), \theta'_{C} \}_{C \in \calC}$ such that
$N'(\gamma)$ is compatible with both $N(\gamma)$ and $M'(\gamma)$. We now define
$M(\gamma)$ to be the slice obtained by amalgamating $M'(\gamma)$ and $N(\gamma)$.

For $\gamma < \beta$, let $G(\gamma): \calC \rightarrow \bfA$ be the simplicial functor corresponding to the slice $M(\gamma)$. Then we have a transfinite sequence of composable morphisms 
$$ G(0) \rightarrow G(1) \rightarrow \ldots$$
in $(\bfA^{\calC})_{F/}$ having colimit $G \simeq \colim_{\gamma < \beta} G(\gamma)$. 
Since $\overline{S}$ is weakly saturated, to prove that the map $F \rightarrow G$ belongs to $\overline{S}$, it will suffice to show that for each $\gamma < \beta$, the map 
$$ f_{\gamma}: \colim_{\gamma' < \gamma} G(\gamma') \rightarrow G(\gamma)$$
belongs to $\overline{S}$. We observe that for each $C in \calC$, the map
$f_{\gamma}(C)$ can be identified with the map
$Y(C)_{ B'(\gamma)(C) } \rightarrow Y(C)_{ B(\gamma)(C) }$. Since $B(\gamma)(C) - B'(\gamma)(C)$ is $\kappa$-small, Remark \ref{relci}, Lemma \ref{tiura} and Lemma \ref{uper} imply that $f_{\gamma}$ is the pushout of a morphism belonging to $S_0$. We now conclude by applying the following result (replacing $G$ by $G(\gamma)$ and $F$ by $\colim_{\gamma' < \gamma} G(\gamma')$:

\begin{lemma}
Suppose that $f: F \rightarrow G$ has the property that, for each $C \in \calC$, there exists a pushout diagram
$$ \xymatrix{ X_{C} \ar[r]^{g_{C}} \ar[d] & Y_{C} \ar[d] \\
F(C) \ar[r]^{f(C)} & G(C) }$$
where $g_{C} \in S_0$. Then $f$ is the pushout of a morphism in $S$.
\end{lemma}

\begin{proof}
In view of $(iv)$, we can write $F$ as the colimit of a diagram $\{ F_{\lambda} \}_{ \lambda \in P}$
indexed by a $\kappa$-filtered partially ordered set $P$, where each $F_{\lambda}$ is a $\kappa$-compact object of $\bfA^{\calC}$, and is therefore a functor whose values are $\kappa$-compact objects of $\bfA$. Since each $X_{C} \in \bfA$ is $\kappa$-compact, the map $X_{C} \rightarrow F(C)$ factors through $F_{\lambda(C)}(C)$ for some sufficiently large $\lambda(C) \in P$. Since $\calC$ has fewer than $\kappa$ objects and $P$ is $\kappa$-filtered, we can choose a single
$\lambda \in P$ which works for every object $C \in \calC$. 

Consider, for each $C \in \calC$, the composite map
$$ \coprod_{C' \in \calC} Y_{C'} \otimes \bHom_{\bfA}(C',C)
\rightarrow \coprod_{C' \in \calC} G(C') \otimes \bHom_{\bfA}(C',C) \rightarrow G(C)
\simeq \colim_{\lambda' \in P} F_{\lambda'}(C) \coprod_{X_C} Y_{C}.$$ 
Using another compactness argument, we deduce that this map is equivalent to a composition
$$ \coprod_{C' \in \calC} Y_{C'} \otimes \bHom_{\bfA}(C',C)
\rightarrow F_{\lambda'(C)}(C) \coprod_{X_C} Y_{C} $$ 
for some sufficiently large $\lambda'(C) \in P$. Once again, because $P$ is $\kappa$-filtered we can choose a single $\lambda' \in P$ which works for all $C$. Enlarging $\lambda$ and $\lambda'$, we can assume $\lambda = \lambda'$. Using another compactness argument, we can (after enlarging $\lambda$ if necessary) assume that
each of the diagrams
$$ \xymatrix{ \coprod_{C' \in \calC} X_{C'} \otimes \bHom_{\bfA}(C',C) \ar[r] \ar[d]  &
F_{\lambda}(C) \ar[d] \\
\coprod_{C' \in \calC} Y_{C'} \otimes \bHom_{\bfA}(C',C) \ar[r] & F_{\lambda}(C) \coprod_{X_C} Y_{C} }$$
$$ \xymatrix{ \coprod_{C', C'' \in \calC} Y_{C''}
\otimes \bHom_{\bfA}(C'',C') \otimes \bHom_{\bfA}(C',C) \ar[r] \ar[d] & \coprod_{C'' \in \calC}
Y_{C''} \otimes \bHom_{\bfA}(C'',C) \ar[d] \\
\coprod_{C' \in \calC} ( F_{\lambda}(C') \coprod_{ X_{C'} } Y_{C'} ) \otimes \bHom_{\bfA}(C',C) \ar[r] & 
F_{\lambda}(C) \coprod_{ X_C} Y_C }$$
is commutative. Then the above maps allow us to define a $\bfS$-enriched functor
$G_{\lambda}: \calC \rightarrow \bfA$ by the formula $G_{\lambda}(C) = F_{\lambda}(C) \coprod_{ X_C} Y_{C}$. We now observe that there is a pushout diagram
$$ \xymatrix{ F_{\lambda} \ar[r]^{f_{\lambda}} \ar[d] & G_{\lambda} \ar[d] \\
F \ar[r]^{f} & G }$$
and that $f_{\lambda} \in S$.
\end{proof}

\subsection{Path Spaces in $\bfS$-Enriched Categories}\label{pathspace}

Let $\bfS$ be a excellent model category. We have seen that there exists a model structure on the category $\SCat$ of $\bfS$-enriched categories, whose fibrant objects are precisely
those categories which are enriched over the full subcategory $\bfS^{\degree}$ of fibrant
objects of $\bfS$. 

The theory of model categories provides a plethora of examples:
for every $\bfS$-enriched model category $\bfA$, the full subcategory $\bfA^{\degree} \subseteq \bfA$ of fibrant-cofibrant objects is a fibrant object of $\SCat$.
In other words, $\bfA^{\degree}$ is suitable to use for computing the homotopy set $[\calC, \bfA^{\degree}] = \Hom_{ \h{\SCat} }( \calC, \bfA^{\degree})$: if $\calC$ is cofibrant, then every map from $\calC$ to $\bfA^{\degree}$ in the homotopy category of $\SCat$ is represented by an actual $\bfS$-enriched functor from $\calC$ to $\bfA^{\degree}$. Moreover, two simplicial functors $F,F': \calC \rightarrow \bfA^{\degree}$ represent the same morphism
in $\h{\SCat}$ if and only if they are homotopic to one another. The relation of homotopy can be described either in terms of a cylinder object for $\calC$ or a path object for $\bfA^{\degree}$. Unfortunately, it is rather difficult to construct a cylinder object for $\calC$ explicitly, since the cofibrations in $\SCat$ are difficult to describe directly even when $\bfS = \sSet$ (the class of cofibrations of simplicial categories is {\em not} stable under products, so the usual procedure of
constructing mapping cylinders via ``product with an interval'' cannot be applied). 
On the other hand, Theorem \ref{staycode} gives a good understanding of the fibrations
in $\SCat$, which will allow us to give a very explicit construction of a path object for $\bfA^{\degree}$.

Let $\bfA$ be a $\bfS$-enriched model category. Our goal in this section is to give a direct construction of a path space object for $\bfA^{\degree}$ in $\SCat$. In other words, we wish to supply
a diagram of $\bfS$-enriched categories $$\bfA^{\degree} \rightarrow P( \bfA )
\rightarrow \bfA^{\degree} \times \bfA^{\degree}$$\index{gen}{path object!in simplicial categories}
where the composite map is the diagonal, the left map is a weak equivalence, and the right map is a fibration. For technical reasons, we will find it convenient to work not with the entire
category $\bfA$, but some (usually small) subcategory thereof. For this reason, we
introduce the following definition:

\begin{definition}\index{gen}{chunk}\label{defchunk}
Let $\bfS$ be an excellent model category, and let $\bfA$ be a
combinatorial $\bfS$-enriched model category. A {\it chunk of $\bfA$}
is a full subcategory $\calU \subseteq \bfA$ with the following properties:
\begin{itemize}
\item[$(a)$] Let $A$ be an object of $\calU$, and let $\{ \phi_i: A \rightarrow B_i \}_{i \in I}$ be a finite
collection of morphisms in $\calU$. Then there exists a factorization
$$ A \stackrel{p}{\rightarrow} \overline{A} \stackrel{q}{\rightarrow} \prod_{i \in I} B_i$$
of the product map $\prod_{i \in I} \phi_i$,
where $p$ is a trivial cofibration, $q$ a fibration, and $\overline{A} \in \calU$.
Moreover, this factorization can be chosen to depend functorially on the collection
$\{ \phi_i \}$, via a $\bfS$-enriched functor.

\item[$(b)$] Let $A$ be an object of $\calU$, and let $\{ \phi_i: B_i \rightarrow A \}_{i \in I}$ be a finite
collection of morphisms in $\calU$. Then there exists a factorization
$$ \coprod_{i \in I} B_i \stackrel{p}{\rightarrow} \overline{A} \stackrel{q}{\rightarrow} A$$
of the coproduct map $\coprod_{i \in I} \phi_i$,
where $p$ is a cofibration, $q$ a trivial fibration, and $\overline{A} \in \calU$.
Moreover, this factorization can be chosen to depend functorially on the collection
$\{ \phi_i \}$, via a $\bfS$-enriched functor.
\end{itemize}

If $\calU$ is a chunk of $\calA$, we let $\calU^{\degree}$\index{not}{Udeg@$\calU^{\degree}$}
denote the full subcategory $\bfA^{\degree} \cap \calU \subseteq \calU$ consisting
of fibrant-cofibrant objects of $\bfA$ which belong to $\calU$.

We will say that two chunks $\calU, \calU' \subseteq \bfA$ are {\it equivalent} if
they have the same essential image in the homotopy category $\h{\bfA}$.
\end{definition}

\begin{remark}
In particular, if $\calU$ is a chunk of $\bfA$, then each object $A \in \calU$ admits (functorial)
fibrant and cofibrant replacements which also belong to $\calU$ (take the set $I$ to be empty in
$(a)$ and $(b)$). 
\end{remark}

\begin{remark}
If $\calU \subseteq \calU' \subseteq \bfA$ are equivalent chunks of $\bfA$, then
the inclusion $\calU^{\degree} \subseteq {\calU'}^{\degree}$ is a weak equivalence
of $\bfS$-enriched categories.
\end{remark}

\begin{example}
Let $\bfS$ be an excellent model category, and $\bfA$ a combinatorial $\bfS$-enriched model category.
Then $\bfA$ is a chunk of itself; this follows from the small object argument.
\end{example}

\begin{example}
Let $\calU \subseteq \bfA$ be a chunk, and let $\{ X_{\alpha} \}$ be a collection of objects in $\bfA$.
Let $\calV \subseteq \calU$ be the full subcategory spanned by those objects $X \in \calU$
such that there exists an isomorphism $[X] \simeq [X_{\alpha}]$ in the homotopy category
$\h{\bfA}$. Then $\calV$ is also a chunk of $\bfA$.
\end{example}

We will prove a general existence theorem for chunks below (see Lemma \ref{exchunk}).

\begin{lemma}\label{tubble}
Let $\bfS$ be an excellent model category, and let $\calC$ be a small $\bfS$-enriched
category. Then there exists a weak equivalence of $\bfS$-enriched categories
$\calC \rightarrow \calU^{\degree}$, where $\calU$ is a chunk of a combinatorial
$\bfS$-enriched category $\bfA$.
\end{lemma}

\begin{proof}
Without loss of generality, we may suppose that $\calC$ is fibrant. Let
$\bfA = \bfS^{ \calC^{op}}$, endowed with the projective model structure.
We can identify $\calC$ with a full subcategory of $\bfA^{\degree}$ via the Yoneda embedding.
Using Lemma \ref{exchunk}, we can enlarge $\calC$ to a chunk in $\bfA$ having the same
image in the homotopy category $\h{\bfA}$.
\end{proof}

\begin{notation}\index{not}{Pbfa@$P(\bfA)$}
Let $\bfS$ be an excellent model category, let $\bfA$ be a combinatorial
$\bfS$-enriched model category, and let $\calU$ be a chunk of $\bfA$. We
define a new category
$P(\calU)$ as follows:
\begin{itemize}
\item[$(i)$] The objects of $P(\calU)$ are fibrations
$\phi: A \rightarrow B \times C$ in $\bfA$, where 
$A, B, C \in \calU^{\degree}$, and the composite maps
$A \rightarrow B$ and $A \rightarrow C$ are weak equivalences.
\item[$(ii)$] Morphisms in $P(\calU)$ are given by
maps of diagrams
$$ \xymatrix{ B \ar[d] & A \ar[l] \ar[r] \ar[d] & C \ar[d] \\
B' & A' \ar[l] \ar[r] & C'. }$$
\end{itemize}

We let $\pi, \pi': P(\calU) \rightarrow \calU^{\degree}$ be the functors described by the formulas
$$ \pi( \phi: A \rightarrow B \times C) = B \quad \quad \pi'( \phi: A \rightarrow B \times C) = C.$$
We observe that both $\pi$ and $\pi'$ have the structure of $\bfS$-enriched functors.
Invoking assumption $(a)$ of Proposition \ref{defchunk}, we deduce the
existence of another $\bfS$-enriched functor $\tau: \calU^{\degree} \rightarrow P(\calU)$, which
carries an object $A \in \bfA^{\degree}$ to the map $q$ appearing in a functorial
factorization
$$ A \stackrel{p}{\rightarrow} \overline{A} \stackrel{q}{\rightarrow} A \times A,$$
of the diagonal, where $p$ is a trivial cofibration and $q$ is a fibration.
\end{notation}

\begin{theorem}\label{catta}
Let $\bfS$ be an excellent model category, let $\bfA$ be a combinatorial $\bfS$-enriched model category, and let $\calU$ be a chunk of $\bfA$. 
Then the morphisms $\pi, \pi': P(\calU) \rightarrow \calU^{\degree}$ and
$\tau: \calU^{\degree} \rightarrow P(\calU)$
furnish $P(\calU)$ with the structure of a path object for
$\calU^{\degree}$ in $\SCat$.
\end{theorem}

\begin{proof}
We first show that $\pi \times \pi'$ is a fibration of $\bfS$-enriched
categories. In view of Theorem \ref{staycode}, it will suffice to show that
$\pi \times \pi'$ is a local fibration. Let $\phi: A \rightarrow B \times C$ and 
$\phi': A' \rightarrow B' \times C'$ be objects of $P(\calU)$.
We must show that the induced map
$$ \bHom_{P(\calU)}(\phi,\phi') \rightarrow \bHom_{\bfA}(B,B') \times
\bHom_{\bfA}(C,C')$$ is a fibration in $\bfS$. 
This map is a base change of 
$$ \bHom_{\bfA}(A,A') \rightarrow \bHom_{\bfA}(A, B' \times C'),$$
which is a fibration in virtue of the assumption that $\phi'$ is a fibration (since
$A$ is assumed to be cofibrant).

To complete the proof that $\pi \times \pi'$ is a quasi-fibration,
we must show that if $\phi: A \rightarrow B \times C$ is an object of
$P( \calU)$ and we are given weak equivalences
$f: B \rightarrow B'$, $g: C \rightarrow C'$, then we can lift
$f$ and $g$ to an equivalence in $P(\calU)$. To do so, we factor the composite map $A \rightarrow B' \times C'$ as a
trivial cofibration $A \rightarrow A'$ followed by a fibration
$\phi': A' \rightarrow B' \times C'$. Since $\calU$ is a chunk of $\bfA$, we may assume that
$A' \in \calU$ so that $\phi' \in P(\calU)$. We have an evident natural transformation
$\alpha: \phi \rightarrow \phi'$. We will show below that $\pi: P(\calU) \rightarrow \calU^{\degree}$
is an equivalence of $\bfS$-enriched categories; since $\pi(\alpha) = f$ is an isomorphism
in $\h{ \calU^{\degree}}$, we conclude that $\alpha$ is an isomorphism in $\h{P(\calU)}$ as
required.

To complete the proof, we must show that $\tau$ is a weak equivalence of $\bfS$-enriched categories.
By the two-out-of-three property, it will suffice to show that $\pi$ is a weak equivalence of
$\bfS$-enriched categories. Since $\tau$ is a section of $\pi$, it is clear that $\pi$ is essentially surjective. It remains only to prove that $\pi$ is fully faithful. Let
$\phi: A \rightarrow B \times C$ and $\phi': A' \rightarrow B' \times C'$ be objects
of $P(\calU)$; we wish to show that the induced map
$p: \bHom_{P(\calU)}(\phi, \phi') \rightarrow \bHom_{\bfA}(B,B')$ is a weak equivalence in
$\bfS$. We have a commutative diagram
$$ \xymatrix{ \bHom_{P(\calU)}(\phi,\phi') \ar[r] \ar[d] & \bHom_{\bfA}(A,A') \ar[d]^{u} \\
\bHom_{\bfA}(B,B') \times \bHom_{\bfA}(C,C') \ar[d] \ar[r] &  \bHom_{\bfA}(A, B' \times C') \ar[d] \\
\bHom_{\bfA}(B,B') \times \bHom_{\bfA}(A,C') \ar[d] \ar[r] & \bHom_{\bfA}(A,B') \times
\bHom_{\bfA}(A,C') \ar[d] \\
\bHom_{\bfA}(B,B') \ar[r] & \bHom_{\bfA}(A,B'). }$$
We note that, since the map $A \rightarrow B$ is a weak equivalence between cofibrant objects,
and $B'$ is fibrant, the bottom horizontal map is a weak equivalence in $\bfS$. Consequently, to
show that the top horizontal map is a weak equivalence in $\bfS$, it will suffice to show that each square
in the diagram is homotopy Cartesian. The bottom square is Cartesian and fibrant, so there is nothing to prove. The middle square is homotopy Cartesian because both of the middle vertical maps are weak equivalences. The upper square is a pullback square between fibrant objects of $\bfS$, and the
map $u$ is a fibration; we now complete the proof by invoking Proposition \ref{leftpropsquare}.
\end{proof}

%We first need to establish a bit of notation for working with
%simplicial categories. We will use the symbol $\ast$ to denote the
%final object both in the category of simplicial sets and in the
%category of simplicial categories: in the latter setting it is the
%simplicial category having a single object $X$, with $\Hom(X,X) =
%\ast$.

%For every simplicial set $S$, we denote by $\calE_S$ the
%simplicial category containing two objects $X$ and $Y$, with
%$$\Hom_{\calE_Z}(X,X) = \Hom_{\calE_Z}(Y,Y) = \ast$$
%$$\Hom_{\calE_Z}(Y,X) = \emptyset$$
%$$\Hom_{\calE_Z}(X,Y) = Z.$$

%We are now ready to describe a model structure on the category
%$\sCat$ of (small) simplicial categories. We begin by considering
%the following class of {\it generating cofibrations}:

%\begin{itemize}
%\item The inclusion $\emptyset \subseteq \ast$ of the empty
%(simplicial) category into the (simplicial) category with a single
%object and a single morphism.

%\item The inclusion $\calE_{ \bd \Delta^n} \subseteq
%\calE_{\Delta^n}$, for each $n \geq 0$.
%\end{itemize}

%We let $C^{\perp}$ denote the class of all morphisms in $\sCat$
%which have the right lifting property with respect to all of these
%generating cofibrations, and we let $C$ denote the class of all
%morphisms which have the left lifting property with respect to all
%morphisms in $C^{\perp}$. By construction, $C$ contains all
%generating cofibrations. By the small object argument, $C$
%consists precisely of those morphisms which are retracts of
%iterated pushouts of generating cofibrations.
%\begin{proof}
%It is easy to see that $\sCat$ has all (small) limits and colimits
%and that the class of equivalences is stable under the formation
%of retracts and satisfies the two-out-of-three property. We note
%that a functor $p: \calC \rightarrow \calC'$ is a fibration and a
%weak equivalence if and only if it is surjective on objects and
%induces trivial fibrations $\Hom_{\calC}(X,Y) \rightarrow
%\Hom_{\calC'}(pX,pY)$ for any $X,Y \in \calC$. It is clear that
%this is equivalent to the requirement that $p$ have the
%right-lifting property with respect to all generating cofibrations
%(therefore all cofibrations).



%As mentioned above, the small object argument shows that any
%functor $p: \calC \rightarrow \calC'$ may be factored as a
%cofibration followed by a morphism in $C^{\perp}$ (which is
%consequently trivial fibration). To complete the proof, it
%suffices to show that we may also factor $p$ as a trivial
%cofibration $p'$ followed by a fibration $p''$.

%Let us first suppose that $p$ induces a quasi-fibration on
%homotopy categories. If we produce any factorization $$ \calC
%\stackrel{p'}{\rightarrow} \calE \stackrel{p''}{\rightarrow}
%\calC'$$ having the property that $p'$ induces a bijection between
%the objects of $\calC$ and the objects of $\calE$, then $p''$ will
%automatically be a quasi-fibration as well. In this case, $p''$
%will be fibration as long as it has the right lifting property
%with respect to all inclusions $\calE_{\Lambda^n_i} \subseteq
%\calE_{\Delta^n}$. Using the small object argument, we see that it
%is possible to construct $p'$ as an iterated pushout of such
%inclusions. This $p'$ is evidently a cofibration which induces a
%bijection on objects, so it suffices to show that $p'$ is an
%equivalence. This follows from Lemma \ref{cof1}, since the class
%of equivalences is stable under filtered colimits.

%We next reduce the general case to the case considered above. It
%will suffice to prove that {\em any} simplicial functor $p: \calC
%\rightarrow \calC'$ admits a factorization $$\calC
%\stackrel{p'}{\rightarrow} \calE \stackrel{p''}{\rightarrow}
%\calC'$$ where $p'$ is a trivial cofibration and $p''$ induces a
%quasi-fibration on homotopy categories. Fix any object $X \in
%\calC$ and any equivalence $h: pX \simeq Y$. We will show how to
%construct a factorization $p=q'' \circ q'$ such that $q'$ is a
%trivial cofibration and $h$ is homotopic to $pq'' \widetilde{h}$
%for some equivalence $\widetilde{h}: q'X \simeq \widetilde{Y}$.
%The desired factorization may be found by iterating this
%construction.

%Let $r: \calC_{(2)} \rightarrow \calC'$ be the functor whose
%existence is guaranteed by applying Lemma \ref{cof3} to the
%equivalence $h$, and let $\ast \subseteq \calD$ be the inclusion
%of the object which maps to $pX$. Altering the choice of $\calD$

% if necessary, we may ensure that the inclusion $\ast \subseteq
%\calD$ is a cofibration. We now define $q'$ to be the inclusion
%$\calC \subseteq \calC \coprod_{\ast} \calD$, and $q''$ the
%obvious functor from the pushout to $\calC'$. Since the inclusion
%$\ast \subseteq \calD$ is a cofibration, so is $p'$. Lemma
%\ref{cof2} now implies that $p'$ is an equivalence, and this
%completes the proof.
%\end{proof}

%\begin{lemma}\label{cof1}
%Let $\calC$ be a simplicial category, $S \subseteq S'$ a trivial
%cofibration of simplicial sets, and $F: \calE_{S} \rightarrow
%\calC$ any functor, and let $\calC'$ be the pushout $\calC
%\coprod_{\calE_{S}} \calE_{S'}$. Then the induced functor $\calC
%\rightarrow \calC'$ is an equivalence.
%\end{lemma}

%\begin{proof}
%We may explicitly construct the pushout $\calC'$ as follows. The
%objects of $\calC'$ are the objects of $\calC$. For $Z,Z' \in
%\calC$, the simplicial set $\Hom_{\calC'}(Z,Z')$ is obtained by
%gluing together the simplicial sets
%$$M_0=\Hom_{\calC}(Z,Z'),$$
%$$M_1=\Hom_{\calC}(Z,FX) \times S' \times \Hom_{\calC}(FY,Z'),$$
%$$M_2=\Hom_{\calC}(Z,FX) \times S' \times \Hom_{\calC}(FY, FX)
%\times S' \times \Hom_{\calC}(FY,Z'),$$ and so on, using the
%obvious identifications. Let $N_i$ denote the union of the images
%of $\{ M_0, \ldots, M_i \}$ in $\calC$, and let $N_{\infty}
%\bigcup_{i} N_i$. Then $\Hom_{\calC}(Z,Z') = N_0 \subseteq
%N_{\infty} = \Hom_{\calC'}(Z,Z')$. To complete the proof, it will
%suffice to show that each inclusion $N_{i-1} \subseteq N_{i}$ is a
%trivial cofibration. This follows from the fact that the map
%$N_{i-1} \rightarrow N_{i}$ is a pushout of an inclusion $ M'_{i}
%\subseteq M_{i}$, which is in turn a (smash) product of inclusions
%of the form
%$$ \emptyset \subseteq \Hom_{\calC}(Z, FX) \times \Hom_{\calC}
%(FY,FX) \times \ldots \times \Hom_{\calC}(FY,FX) \times
%\Hom_{\calC}(FY,Z')$$ and $S \subseteq S'.$ Since $S \subseteq S'$
%is a trivial cofibration, the product inclusion $S' \times A \cup
%S \times B \subseteq S' \times B$ is a trivial cofibration for
%{\em} any inclusion $A \subseteq B$.
%\end{proof}

%\begin{lemma}\label{cof2}
%Let $\calC$ and $\calC'$ be simplicial categories with
%distinguished objects, and let $\calC \coprod_{\ast} \calC'$ be
%the simplicial category obtained by identifying those objects.
%Suppose that $\ast \rightarrow \calC'$ is an equivalence. Then the
%natural functor $F: \calC \rightarrow \calC \coprod_{\ast} \calC'$
%is an equivalence.
%\end{lemma}

%\begin{proof}
%Let us write $X$ for the distinguished object in both $\calC$ and
%$\calC'$. It is clear that $\calC \rightarrow \calC \coprod_{\ast}
%\calC'$ is essentially surjective: its essential image contains
%every object in $\calC$ and the object $X \in \calC'$. Every
%object of $\calC'$ is isomorphic to $X$ in $h \calC'$, and
%consequently in $h (\calC \coprod_{\ast} \calC')$.

%To complete the proof, it suffices to show that $F$ is fully
%faithful. In other words, we must show that for any pair of
%objects $Y,Z \in \calC$, the natural map
%$$ \Hom_{\calC}(Y,Z) \rightarrow \Hom_{\calC \coprod_{\ast}
%\calC'}(FY,FZ)$$ is an equivalence. We note that the latter space
%may be obtained by gluing together the spaces
%$$ M_0 = \Hom_{\calC}(Y,Z),$$
%$$M_1 = \Hom_{\calC}(Y,X) \times \Hom_{\calC'}(X,X) \times
%\Hom_{\calC}(X,Z),$$
%$$M_2 = \Hom_{\calC}(Y,X) \times
%\Hom_{\calC'}(X,X) \times \Hom_{\calC}(X,X) \times
%\Hom_{\calC'}(X,X) \times \Hom_{\calC}(X,Z),$$ and so on, using
%the obvious identifications.

%Let $N_i$ denote the union of the images of $\{ M_j \}_{j \leq i}$
%in $\Hom_{\calC \coprod_{\ast} \calC'}(FY,FZ)$, and let
%$N_{\infty} = \bigcup_{i} N_i$. Then we have $N_0 =
%\Hom_{\calC}(Y,Z)$ and $N_{\infty} = \Hom_{\calC \coprod_{\ast}
%\calC'}(FY,FZ)$. It therefore suffices to show that each inclusion
%$N_{i-1} \subseteq N_i$ is a trivial cofibration. But this
%inclusion is the pushout of an inclusion $M'_i \subseteq M_i$, and
%this inclusion is a (smash) product of inclusions having the form
%$$ \emptyset \subseteq \Hom_{\calC}(Y,X) $$
%$$ \emptyset \subseteq \Hom_{\calC}(X,Z) $$
%$$ \ast \subseteq \Hom_{\calC}(X,X)$$
%$$ \ast \subseteq \Hom_{\calC'}(X,X).$$
%Moreover, this product involves precisely $i$ factors of the last
%type. By assumption, every such factor is a trivial cofibration,
%which implies that $M'_i \subseteq M_i$ is a trivial cofibration.
%\end{proof}

%\begin{lemma}\label{cof3}
%Let $\calC$ be a simplicial category, and let $f: X \rightarrow Y$
%be an equivalence in $\calC$. There exists a simplicial category
%$\calD$ and a functor $r: \calD \rightarrow \calC$ having the
%following properties:

%\begin{itemize}
%\item The simplicial category $\calD$ has just two objects, $X'$
%and $Y'$. Moreover $rX'=X$ and $rY'=Y$.

%\item The simplicial category $\calD$ contains an equivalence $f':
%X' \rightarrow Y'$ satisfying $rf'=f$.

%\item The simplicial set $\Hom_{\calD}(X',X')$ is weakly
%contractible (and therefore so are $\Hom_{\calD}(X',Y')$,
%$\Hom_{\calD}(Y',X')$, and $\Hom_{\calD}(Y',Y')$).
%\end{itemize}
%\end{lemma}

%\begin{proof}
%We first suppose that the simplicial category $\calC$ is fibrant
%(in other words, all of its mapping spaces $\Hom_{\calC}(A,B)$ are
%fibrant simplicial sets). In this case, we construct $\calD$ in
%the following sequence of stages:

%\begin{itemize}
%\item Begin with the discrete simplicial category $\calD_{0}$
%having just two objects, $X'$ and $Y'$, and no non-identity
%morphisms. Set $rX' = X$ and $rY' = Y$.

%\item Form a larger simplicial category $\calD_{1}$ by adjoining
%to $\calD_0$ a single morphism $f'$ from $X'$ to $Y'$. Set $rf' =
%f$.

%\item Since $h$ is an equivalence, there exists a morphism $g: Y
%\rightarrow X$ in $\calC$ which is a homotopy inverse to $f$. Let
%$\calD_2$ be obtained from $\calD_1$ by adjoining a single
%morphism $g': Y' \rightarrow X'$, and set $rg' = g$.

%\item Since $f$ is homotopy inverse to $g$, the composition $g
%\circ f$ lies in the same component of $\Hom_{\calC}(X,X)$ as
%$\id_X$. Since $\calC$ is fibrant, there exists an edge $x \in
%\Hom_{\calC}(X,X)_1$ with $d^1 x = g \circ f$ and $d^0 x = \id_X$.
%Let $\calD_3$ be obtained from $\calD_2$ by adjoining a new edge
%$x' \in \Hom_{\calD_3}(X',X')_1$ with $d^1 x' = g' \circ f'$ and
%$d^0 x' = \id_{X'}$. Set $rx'=x$.

%\item Similarly, there exists an edge $y \in \Hom_{\calC}(Y,Y)$
%with $d^1 y = f \circ g$ and $d^0 y = \id_Y$. Moreover, since
%composition with $f$ induces a homotopy equivalence
%$\Hom_{\calC}(Y,Y) \rightarrow \Hom_{\calC}(X,Y)$, we may choose
%$y$ such that $y \circ s^0 f$ and $f \circ s^0 x$ are homotopic
%paths from $f \circ g \circ f$ to $f$ in $\Hom_{\calC}(X,Y)$. Let
%$\calD_4$ be obtained from $\calD_3$ by adjoining a new edge $y'
%\in \Hom_{\calD_4}(Y',Y')_1$ satisfying $d^1 y' = f' \circ g'$,
%$d^0 y' = \id_{Y'}$. Set $ry'=y$.

%\item Since $y \circ s^0 f$ and $f \circ s^0 x$ are homotopic,
%there exists a $2$-simplex $z \in \Hom_{\calC}(X,Y)_2$ such that
%$d^0 z = s^0 f$, $d^1 z = y \circ s^0 f$, and $d^2 z = f \circ s^0
%x$. Let $\calD_{5}$ be obtained from $\calD_{4}$ by adjoining a
%new $2$-simplex $z' \in \Hom_{\calD_5}(X,Y)_2$ satisfying $d^0 z'
%= s^0 f'$, $d^1 z' = y' \circ s^0 f'$, $d^2 z' = f' \circ s^0 x'$.
%Set $rz'=z$.
%\end{itemize}

%One can show by a simple calculation that $\calD_5$ has the
%desired properties.

%Let us now treat the general case where $\calC$ is not assumed
%fibrant. Let $T$ be the composition of the geometric realization
%functor and the singular complex functor. Then $T$ is a functor
%from simplicial sets to fibrant simplicial sets which commutes
%with products. Moreover, the adjunction map gives rise to a
%natural transformation $S \rightarrow TS$ which is a weak
%equivalence for any simplicial set $S$. We may now construct a
%simplicial category $\calC'$ which is equivalent to $\calC$ as
%follows:

%\begin{itemize}

%\item The objects of $\calC'$ are the objects of $\calC$.

%\item If $X,Y \in \calC$, then $\Hom_{\calC'}(X,Y) = T
%\Hom_{\calC}(X,Y)$.

%\end{itemize}

%There is an obvious functor $p: \calC \rightarrow \calC'$, and
%$\calC'$ is fibrant. Consequently, the first part of the proof
%yields a functor $r': \calD' \rightarrow \calC'$, where $\calD'$
%contains only two objects $X'$ and $Y'$, has weakly contractible
%mapping spaces, and $r'$ carries $X'$ to $pX$ and $Y'$ to $pY$,
%and carries some equivalence $X' \simeq Y'$ to $ph$.

%We claim that it is possible to choose $(\calD', r')$ in such a
%way that $r'$ induces Kan fibrations $\Hom_{ \calD' }(A,B)
%\rightarrow \Hom_{\calC'}(r'A, r'B)$ for any objects $A,B \in
%\calD'$. To achieve this, one iteratively replaces $\calD'$ by
%pushouts $\calD' \coprod_{\calE_{\Lambda^n_i}} \calE_{\Delta^n}$
%(which results in a simplicial category equivalent to $\calD'$ by
%Lemma \ref{cof1}).

%We now define a new simplicial category $\calD$ as follows:

%\begin{itemize}
%\item The objects of $\calD$ are the objects of $\calD'$: namely,
%$X'$ and $Y'$.

%\item For $A,B \in \{X,Y\}$, we have $$\Hom_{\calD}(A',B') =
%\Hom_{\calD'}(A',B') \times_{ \Hom_{\calC'}(pA,pB) }
%\Hom_{\calC}(A,B).$$
%\end{itemize}

%Since the pullback of a weak equivalence of simplicial sets by a
%fibration is a weak equivalence, we deduce that the natural
%functor $\calD \rightarrow \calD'$ is a weak equivalence.
%Moreover, the projection $r: \calD \rightarrow \calC$ clearly has
%the desired properties.
%\end{proof}

%Recall that a model category
%$\calC$ is said to be {\it left proper} if, for any pair of
%morphisms $f: X \rightarrow Y$ and $g: X \rightarrow Z$ having the
%property that $f$ is a cofibration and $g$ is a weak equivalence,
%the induced map $g': Y \rightarrow Y \coprod_X Z$ is also a weak
%equivalence. A model category $\calC$ is {\it right proper} if the dual assertion holds: for
%any pair of maps $f: Y \rightarrow X$, $g: Z \rightarrow X$ such that $f$ is a fibration and
%$g$ is a weak equivalence, the induced map $Y \times_{X} Z \rightarrow Y$ is a weak equivalence.

%\begin{proposition}\label{simpleft}
%With respect to the model structure of Proposition \ref{scatstructure}, $\sCat$ is both left and right proper.
%\end{proposition}

%\begin{proof}
%We first show that $\sCat$ is right proper. Suppose given a pair of simplicial functors
%$F: \calC \rightarrow \calD$, $G: \calD' \rightarrow \calD$. Suppose that $F$ is an equivalence and that $G$ is a fibration. Let $\calC' = \calD' \times_{\calD} \calC$, and let $F': \calC' \rightarrow \calD'$
%and $G': \calC' \rightarrow \calC$ be the projections. We wish to show that $F'$ is an equivalence.

%We first show that $F'$ is fully faithful. Fix objects $X,Y \in \calC'$, and note that
%$$ \bHom_{\calC'}(X,Y) = \bHom_{\calC}(G'X,G'Y) \times_{ \bHom_{\calD}(FG'X,FG'Y) } %\bHom_{\calD'}(F'X,F'Y).$$ Since $\bHom_{\calC}(G'X,G'Y) \rightarrow %\bHom_{\calD}(FG'X,FG'Y)$ is a weak homotopy equivalence and $\bHom_{\calD'}(F'X,F'Y) \rightarrow \bHom_{\calD}(GF'X,GF'Y)$ is a fibration, we deduce that $\bHom_{\calC'}(X,Y) \rightarrow \bHom_{\calD'}(F'X,F'Y)$ is a weak homotopy equivalence, since the usual model structure on $\sSet$ is right-proper.

%Let us now show that $F'$ is essentially surjective. Let $X \in \calD'$ be an object. Since $F$ is essentially surjective, there exists an object $Y \in \calC$ and a homotopy equivalence
%$GX \rightarrow FY$ in $\calD$. Since $G$ is a fibration, this homotopy equivalence lifts to a homotopy equivalence map
%$X' \rightarrow Y$ in $\calD'$. The pair $(X,X')$ is an object of $\calC$, having image $X'$ in $\calD$;
%thus $Y$ lies in the essential image of $F'$.

Fix an excellent model category $\bfS$. The symmetric monoidal structure on $\bfS$
induces a symmetric monoidal structure on $\SCat$:
if $\calC$ and $\calD$ are $\bfS$-enriched categories, then we can define a new
$\bfS$-enriched category $\calC \otimes \calD$ as follows:
\begin{itemize}
\item[$(i)$] The objects of $\calC \otimes \calD$ are pairs $(C,D)$, where
$C \in \calC$ and $D \in \calD$.
\item[$(ii)$] Given a pair of objects $(C,D), (C',D') \in \calC \otimes \calD$, we have
$$ \bHom_{\calC \otimes \calD}( (C,D), (C',D')) = \bHom_{\calC}(C,C') \otimes \bHom_{\calD}(D,D') \in \bfS.$$
\end{itemize}
In the case where the tensor product on $\bfS$ is the Cartesian product, this simply
reduces to the usual product of $\bfS$-enriched categories.

Note that the operation $\otimes: \SCat \times \SCat \rightarrow \SCat$ is {\em not}
a left Quilen bifunctor, even when $\bfS = \sSet$: for example, a product of cofibrant
simplicial categories is generally not cofibrant. Nevertheless, $\otimes$ behaves
much like a left Quillen bifunctor at the level of homotopy categories. 
For example, the operation $\otimes$ respects weak equivalences in each argument, and therefore
induces a functor $\otimes: \h{ \SCat} \times \h{\SCat} \rightarrow \h{\SCat}$, which is
characterized by the existence of natural isomorphisms
$[ \calC \otimes \calD] \simeq [\calC] \otimes [\calD]$.

Our goal for the remainder of this section is to show that the monoidal structure
$\otimes$ on $\SCat$ is {\em closed}: that is, there exist internal mapping objects in
$\h{\SCat}$. This is not completely obvious. It is easy to see that the monoidal structure
$\otimes$ on $\SCat$ is closed: given a pair of $\bfS$-enriched categories $\calC$ and
$\calD$, the category of $\bfS$-enriched functors $\calD^{\calC}$ is itself $\bfS$-enriched,
and possesses the appropriate universal property. However, this is not necessarily
the ``correct'' mapping object, in the sense that the homotopy equivalence class
$[ \calD^{\calC} ]$ does not necessarily coincide with the internal mapping object
$[\calD]^{[ \calC] }$ in $\h{\SCat}$. Roughly speaking, the problem is that
$\calD^{\calC}$ consists of functors which are strictly compatible with composition; 
the correct mapping object should incorporate also functors which preserve composition only
up to (coherent) weak equivalence. However, when $\calD$ is the category of fibrant-cofibrant
objects of a $\bfS$-enriched {\em model} category $\bfA$, then we can proceed more directly.

\begin{definition}\label{cattusi}
Let $\bfS$ be an excellent model category, $\bfA$ a combinatorial $\bfS$-enriched model category,
and $\calC$ a cofibrant $\bfS$-enriched category. We will say that a full subcategory
$\calU \subseteq \bfA$ is a {\it $\calC$-chunk of $\bfA$} if it is a chunk of
$\bfA$ and the subcategory $\calU^{\calC}$ is a chunk of $\bfA^{\calC}$. 
Here we regard $\bfA^{\calC}$ as endowed with the {\em projective}
model structure.
\end{definition}

%\begin{remark}
%In the situation of Definition \ref{cattusi}, if $\calC$ is assumed to be cofibrant, then condition
%$(c)$ is superfluous.
%\end{remark}

\begin{lemma}\label{tuff}
Let $\bfS$ be an excellent model category, $\bfA$ a combinatorial $\bfS$-enriched model category,
$\calC$ a $($small$)$ cofibrant $\bfS$-enriched category, and $\calU \subseteq \bfA$ a $\calC$-chunk.
Let $f,f': \calC \rightarrow \calU^{\degree}$ be a pair of maps. The following conditions are equivalent:
\begin{itemize}
\item[$(1)$] The homotopy classes $[f]$ and $[f']$ coincide in 
$\Hom_{ \h{\SCat}}( \calC, \calU^{\degree})$.
\item[$(2)$] The maps $f$ and $f'$ are weakly equivalent when regarded as objects
of $\bfA^{\calC}$.
\end{itemize}
\end{lemma}

\begin{proof}
%Let $g: \calC' \rightarrow \calC$ be a weak equivalence of $\bfS$-enriched categories.
%Then condition $(1)$ is satisfied for $f$ and $f'$ if and only if it is satisfied for
%$f \circ g$ and $f' \circ g$. In view of Proposition \ref{lesstrick}, the same is true
%for condition $(2)$. Choose $g$ so that $\calC'$ is cofibrant and $\calU \subseteq \bfA$
%is a $\calC'$-chunk. Replacing $\calC$ by $\calC'$, we may reduce to the case where $\calC$ is cofibrant. 

Suppose first that $(1)$ is satisfied. Using Theorem \ref{catta}, we deduce
the existence of a homotopy $h: \calC \rightarrow P(\calU)$ from
$f = \pi \circ h$ to $f' = \pi' \circ h$. The map $h$
determines another simplicial functor $f'': \calC \rightarrow
\calU$, equipped with weak equivalences $f'' \rightarrow f$, $f''
\rightarrow f'$. This proves that $f$ and $f'$ are isomorphic in the homotopy category
of $\bfA^{\calC}$, so that $(2)$ is satisfied.

Now suppose that $(2)$ is satisfied. Since $\calU$ is a $\calC$-chunk, we can find
a projectively cofibrant $f'': \calU \rightarrow \calC$ equipped with a weak equivalence
$\alpha: f'' \rightarrow f$. Using $(2)$, we deduce that there is also a weak equivalence
$\beta: f'' \rightarrow f'$. Using again the assumption that $\calU^{\calC}$ is a chunk of
$\bfA^{\calC}$, we can choose a factorization of $\alpha \times \beta$ as a composition
$$ f'' \stackrel{u}{\rightarrow} f''' \stackrel{v}{\rightarrow} f \times f'$$ 
where $u$ is a trivial projective cofibration, $v$ is a projective fibration, and
$f''' \in \calU^{\calC}$. The map $v$ can be viewed as an object of $\calP(\calU)$, which
determines a right homotopy from $f$ to $f'$. 
\end{proof}

\begin{corollary}\label{suff}
Let $\bfS$ be an excellent model category, and let $f: \calC \rightarrow \calC'$ be a $\bfS$-enriched functor. Suppose that $f$ is fully faithful in the sense that for every pair of objects
$X,Y \in \calC$, the induced map $\bHom_{\calC}( X,Y) \rightarrow \bHom_{\calC'}(fX, fY)$ is a
weak equivalence in $\bfS$. Let $\calD$ be an arbitrary $\bfS$-enriched category. Then:
\begin{itemize}
\item[$(1)$] Composition with $f$ induces an injective map
$\phi: \Hom_{ \h{\SCat}} ( \calD, \calC) \rightarrow \Hom_{ \h{\SCat}}( \calD, \calC' )$.
\item[$(2)$] The image of $\phi$ consists of those maps
$g: \calD \rightarrow \calC'$ in $\h{\SCat}$ such that the essential image of
$[g]$ in $\h{\calC'}$ is contained in the essential image of $[f]$ in $\h{\calC'}$. 
\end{itemize}
\end{corollary}

\begin{proof}
Using Lemma \ref{tubble}, we may assume without loss of generality that
$\calC' = \calU^{\degree}$, where $\calU$ is a chunk of a $\bfA$-enriched model category.
Let $\calV \subseteq \calU$ be the full subcategory spanned by those objects which
are weakly equivalent to an object lying in the image of $f$. Since $f$ is fully faithful,
the induced map $\calC \rightarrow \calV^{\degree}$ is a weak equivalence. We may therefore
assume that $\calC = \calV^{\degree}$. 

Without loss of generality, we may suppose that $\calD$ is cofibrant. 
Enlarging $\calU$ and $\calV$ if necessary (using Lemma \ref{exchunk}), we may assume
that $\calU$ and $\calV$ are $\calD$-chunks. 
The desired results now follow immediately from Lemma \ref{tuff}.
\end{proof} 

%\begin{corollary}
%Let $\bfS$ be an excellent model category, $\bfA$ a combinatorial $\bfS$-enriched
%model category, and $\calC$ a $($small$)$ cofibrant $\bfS$-enriched category. Let
%$\calU \subseteq \calV \subseteq \bfA$ be equivalent $\calC$-chunks of $\bfA$.
%Then $\calU^{\calC} \subseteq \calV^{\calC}$ are equivalent chunks of
%$\bfA^{\calC}$.
%\end{corollary}

%\begin{proof}
%We wish to show that $\calU^{\calC}$ and $\calV^{\calC}$ have the same
%image in the homotopy category $\h{ \bfA^{\calC} }$. In other words, we want
%to prove that each $f \in \calV^{\calC}$ is weakly equivalent (in $\bfA^{\calC}$)
%to an object in $\calU^{\calC}$. Without loss of generality we may assume
%that $f$ is projectively fibrant and cofibrant. Then $f$ defines a map
%$\calC \rightarrow \calV^{\degree}$. Since the inclusion
%$\calU^{\degree} \rightarrow \calV^{\degree}$ is a weak equivalence of fibrant
%$\bfS$-enriched categories, the map $f$ is equivalent in $\h{\SCat}$ to
%some $f': \calC \rightarrow \calU^{\degree}$. Lemma \ref{tuff} now implies
%that $f$ and $f'$ are weakly equivalent in $\bfA^{\calC}$.
%\end{proof}

Let $\pi_0 \bfA^{\calC}$ denote the collection of
weak equivalence classes of objects in $\bfA^{\calC}$. Every
equivalence class contains a fibrant-cofibrant representative, which
determines a $\bfS$-enriched functor $\calC \rightarrow \bfA^{\degree}$. 

\begin{proposition}\label{gumbaa}
Let $\bfS$ be an excellent model category, $\bfA$ a combinatorial
$\bfS$-enriched category, and $\calC$ a $($small$)$ cofibrant $\bfS$-enriched category.
Then the map
$$ \phi: \pi_0 \bfA^{\calC} \rightarrow \Hom_{\h{\SCat}}(\calC, \bfA^{\degree})$$
described above is bijective.
\end{proposition}

\begin{proof}
In view of Proposition \ref{lesstrick}, we may assume that $\calC$ is cofibrant.
Lemma \ref{tuff} shows that $\phi$ is well-defined and injective.
We show that $\phi$ is surjective. Let $[f] \in \Hom_{\h{\SCat}}(
\calC,\calU^{\degree})$. Since $\calC$ is cofibrant and $\bfA^{\degree}$
is fibrant in $\SCat$, we can find a $\bfS$-enriched functor $f: \calC
\rightarrow \bfA^{\degree}$ representing $[f]$. The simplicial
functor $f$ takes values in fibrant-cofibrant objects of $\bfA$,
but is not necessarily fibrant-cofibrant {\em as an object of}
$\bfA^{\calC}$. However, we can choose a 
weak trivial fibration $f' \rightarrow f$, where $f'$ is projectively cofibrant.
Consequently, it will suffice to show that a weak equivalence $u: f' \rightarrow f$
of $\bfS$-enriched functors $\calC \rightarrow \bfA^{\degree}$
guarantees that $[f] = [f'] \in \Hom_{\h{\SCat}}(\calC, \bfA^{\degree})$, which follows
from Lemma \ref{tuff}.
\end{proof}

\begin{proposition}\label{gumbarr}
Let $\bfS$ be an excellent model category, $\bfA$ a combinatorial $\bfS$-enriched model category,
and $\calC$ a small $\bfS$-enriched category.
Then the evaluation map
$e: ( \bfA^{\calC})^{\degree} \otimes \calC \rightarrow \bfA^{\degree}$ has
the following property: for every small $\bfS$-enriched category $\calD$, 
composition with $e$ induces a bijection
$$ \Hom_{ \h{\SCat} }( \calD, (\bfA^{\calC})^{\degree} )
\rightarrow \Hom_{ \h{\SCat} }( \calC \otimes \calD, \bfA^{\degree} ).$$
\end{proposition}

\begin{proof}
Using Proposition \ref{gumbaa}, we can identify both sides with
$\pi_0 \bfA^{ \calD \otimes \calC}$. 
\end{proof}

It is not clear that the conclusion of Proposition \ref{gumbarr} characterizes
$(\bfA^{\calC})^{\degree}$ up to equivalence, since
$(\bfA^{\calC})^{\degree}$ is a {\em large} $\bfS$-enriched category, and the
proof of the Proposition only applies in the case where $\calD$ is small.
To remedy this defect, we establish a more refined version:

\begin{corollary}\label{sniffle}
Let $\bfS$ be an excellent model category, $\bfA$ a combinatorial $\bfS$-enriched model category,
and $\calC$ a small cofibrant $\bfS$-enriched category. Let $\calU$ be a
$\calC$-chunk of $\bfA$. 
Then the evaluation map
$e: ( \calU^{\calC})^{\degree} \otimes \calC \rightarrow \calU^{\degree}$ has
the following property: for every small $\bfS$-enriched category $\calD$, 
composition with $e$ induces a bijection
$$ \Hom_{ \h{\SCat} }( \calD, (\calU^{\calC})^{\degree} )
\rightarrow \Hom_{ \h{\SCat} }( \calC \otimes \calD, \calU^{\degree} ).$$
\end{corollary}

\begin{proof}
Combine Proposition \ref{gumbarr} with Corollary \ref{suff}.
\end{proof}

We conclude this section with a technical result, which ensures the existence of a good supply of chunks of combinatorial model categories. 

\begin{lemma}\label{exchunk}
Let $\bfS$ be an excellent model category, $\bfA$ a combinatorial $\bfS$-enriched model category,
and $\{ \calC_\alpha \}_{ \alpha \in A}$ a $($small$)$ collection of $($small$)$ cofibrant
$\bfS$-enriched categories. Let $\calU$ be a small full subcategory of $\bfA$.
Then there exists a small subcategory $\calV \subseteq \bfA$ containing $\calU$, such
that $\calV$ is a $\calC_{\alpha}$-chunk for each $\alpha \in A$. Moreover, we may arrange
that $\calU$ and $\calV$ have the same essential image in the homotopy category
$\h{\bfA}$. 
\end{lemma}

\begin{proof}
Enlarging $A$ if necessary, we may suppose that the collection $\{ \calC_{\alpha} \}_{\alpha in A}$
includes the unit category $[0]_{\bfS}$. 
For each $\alpha \in A$, we can choose
$\bfS$-enriched functors $$F_{\alpha}: \bfA^{ \calC_{\alpha} \otimes [1]_{\bfS} } \rightarrow
\bfA^{ \calC_{\alpha} \otimes [2]_{\bfS} } \quad \quad
G_{\alpha}: \bfA^{ \calC_{\alpha} \otimes [1]_{\bfS} } \rightarrow
\bfA^{ \calC_{\alpha} \otimes [2]_{\bfS} },$$
such that $F$ carries each morphism $u: f \rightarrow g$ in $\bfA^{\calC}$ to a factorization
$$ f \stackrel{u'}{\rightarrow} f' \stackrel{u''}{\rightarrow} g$$
where $u'$ is a strong trivial cofibration and $u''$ is a projective fibration, and
$G$ carries $u$ to a factorization
$$ f \stackrel{v'}{\rightarrow} g' \stackrel{v''}{\rightarrow} g$$
where $v'$ is a projective cofibration and $v''$ is a weak trivial cofibration.
For $C \in \calC_{\alpha}$, let $F^{C}_{\alpha}$ be the
functor $u \mapsto f'(C)$, and define $G^{C}_{\alpha}$ likewise.

Choose a regular cardinal $\kappa$ such that each $\calC_{\alpha}$ is 
$\kappa$-small. We define a sequence of full subcategories
$\{ \calU_{\alpha} \subseteq \bfA \}_{ \alpha < \kappa }$ as follows:
\begin{itemize}
\item[$(i)$] If $\alpha = 0$, then $\calU_{\alpha} = \calU$.
\item[$(ii)$] If $\alpha$ is a nonzero limit ordinal, then 
$\calU_{\alpha} = \bigcup_{\beta < \alpha} \calU_{\beta}$.
\item[$(iii)$] If $\alpha = \beta + 1$, then
$\calU_{\alpha}$ is the full subcategory of $\bfA$ spanned by the following:
\begin{itemize}
\item[$(a)$] The objects which belong to $\calU_{\beta}$.
\item[$(b)$] The objects $F^{C}_{\alpha}(u) \in \bfA$, where
$\alpha \in A$, $C \in \calC_{\alpha}$, and $u: f \rightarrow g$
is a morphism from an object of $\calU_{\beta}^{\calC_{\alpha}}$ to
a finite product of object in $\calU_{\beta}^{\calC_{\alpha}}$. 
\item[$(c)$] The objects $G^{C}_{\alpha}(u) \in \bfA$, where
$\alpha \in A$, $C \in \calC_{\alpha}$, and $u: f \rightarrow g$
is a morphism from a finite coproduct of objects of $\calU_{\beta}^{\calC_{\alpha}}$ to
an object in $\calU_{\beta}^{\calC_{\alpha}}$. 
\end{itemize}
\end{itemize}

It is readily verified that the subcategory $\calV = \bigcup_{\alpha < \kappa} \calU_{\alpha}$
has the desired properties.
\end{proof}

\subsection{Homotopy Colimits of $\bfS$-Enriched Categories}\label{hoco}

Our goal in this section is to give an explicit construction of (certain) homotopy colimits
in the model category $\SCat$, where $\bfS$ is an excellent model category. We begin
with some general remarks concerning localization.

\begin{notation}\index{not}{calCWinv@$\calC[W^{-1}]$}\index{not}{Inv@$\Inv$}\label{localdef}
Consider the canonical map $\overline{i}: [1]_{\bfS} \rightarrow [1]^{\sim}_{\bfS}$. We
fix once and for all a factorization of $\overline{i}$ as a composition
$$ [1]_{\bfS} \stackrel{i}{\rightarrow} \calE \stackrel{i'}{\rightarrow} [1]_{\bfS}^{\sim},$$
where $i$ is a cofibration and $i'$ is a weak equivalence of $\bfS$-enriched categories.
For every $\bfS$-enriched category $\calC$ and every map of sets
$W \rightarrow \Hom_{\SCat}( [1]_{\bfS}, \calC$, we define a new $\bfS$-enriched
category $\calC[W^{-1}]$ by a pushout diagram
$$ \xymatrix{ \coprod_{w \in W} [1]_{\bfS} \ar[r] \ar[d] & \calC \ar[d] \\
\coprod_{w \in W} \calE \ar[r] & \calC[W^{-1} ]. }$$
\end{notation}

\begin{remark}\label{summat}
Since the model category $\SCat$ is left proper, the construction
$\calC \mapsto \calC[W^{-1}]$ preserves weak equivalences in $\calC$.
\end{remark}

We now characterize $\calC[W^{-1}]$ by a universal property in $\h{\SCat}$.

\begin{lemma}\label{pufft}
Let $\calC$ be a fibrant $\bfS$-enriched category, and 
$f$ be a morphism in $\calC$ classified by a map $j_0: [1]_{\bfS} \rightarrow \calC$.
The following conditions are equivalent:
\begin{itemize}
\item[$(1)$] The map $f$ is an equivalence in $\calC$.
\item[$(2)$] The extension problem depicted in the diagram
$$ \xymatrix{ [1]_{\bfS} \ar[d]^{i} \ar[r]^{j_0} & \calC \\
\calE \ar@{-->}[ur]^{j} & }$$
admits a solution.
\end{itemize}
\end{lemma}

\begin{proof}
The implication $(2) \Rightarrow (1)$ is clear, since every morphism in
$\calE$ is an equivalence. For the converse, we observe that
the desired lifting problem admits a solution if and only if the induced map
$i': \calC \rightarrow \calC \coprod_{[1]_{\bfS} } \calE$ admits a left inverse.
Since $\calC$ is fibrant, it suffices to show that $i'$ is a trivial cofibration.
The map $i'$ is a cofibration since it is a pushout of $i$, and a weak equivalence
because of the invertibility hypothesis. 
\end{proof}

Lemma \ref{pufft} immediately implies the following apparently stronger claim:

\begin{lemma}\label{canner}
Let $f_0: \calC \rightarrow \calD$ be a $\bfS$-enriched functor, where
$\calD$ is a fibrant $\bfS$-enriched category. Let $\psi: W \rightarrow \Hom_{\SCat}([1]_{\bfS},\calC)$ be a map of sets. The following conditions are equivalent:
\begin{itemize}
\item[$(1)$] The map $f_0$ extends to a map $f: \calC[W^{-1}] \rightarrow \calD$.
\item[$(2)$] For each $w \in W$, $f_0$ carries the morphism $\phi(w)$ to an equivalence in $\calD$.
\end{itemize}
\end{lemma}

\begin{proposition}\label{postcan}
Let $\calC$ and $\calD$ be $\bfS$-enriched categories, and let
$\psi: W \rightarrow \Hom_{\SCat}( [1]_{\bfS}, \calC)$ be a map of sets.
Then the induced map
$$ \phi: \Hom_{\h{\SCat}}( \calC[W^{-1}], \calD) \rightarrow \Hom_{\h{\SCat}}(\calC, \calD)$$
is injective, and its image is the subset $\Hom^{W}_{\h{\SCat}}(\calC, \calD)
\subseteq \Hom_{\h{\SCat}}(\calC, \calD)$ consisting of those homotopy classes of maps
which induce functors $\h{\calC} \rightarrow \h{\calD}$ carrying each element of $W$ to an isomorphism in $\h{\calD}$. 
\end{proposition}

\begin{proof}
Without loss of generality, we may suppose that $\calC$ is cofibrant and $\calD$ is fibrant.
The description of the image of $\phi$ follows immediately from Lemma \ref{canner}.
It will therefore suffice to show that $\phi$ is injective. Suppose we are given a pair of maps
$[f], [g] \in \Hom_{ \h{\SCat}}( \calC[W^{-1}], \calD)$ such that $\phi( [f] ) = \phi( [g] )$. Since
$\calC[ W^{-1}]$ is cofibrant, we may assume that $[f]$ and $[g]$ are represented by
actual $\bfS$-enriched functors $f,g: \calC[W^{-1}] \rightarrow \calD$. Moreover, the
condition that $\phi( [f] ) = \phi( [g] )$ guarantees that the restrictions
$f|\calC$ and $g|\calC$ are homotopic. We wish to show that $f$ and $g$ are homotopic.

Invoking Proposition \ref{princex}, we deduce that $g$ is homotopic to a map
$g': \calC[W^{-1}] \rightarrow \calD$ such that $g' | \calC = f | \calC$. Replacing $g$ by
$g'$ if necessary, we may assume that $g| \calC = f| \calC$. It will now suffice to show that
$f$ and $g$ are homotopic in the model category $( \SCat)_{\calC/}$. We
observe that $f$ and $g$ determine a map
$$ h: \calC[ ( W \coprod W)^{-1} ] \simeq \calC[W^{-1}] \coprod_{\calC} \calC[W^{-1}] \rightarrow \calD.$$
Using the invertibility hypothesis, we conclude that
$\calC[ (W \coprod W)^{-1}]$ is a cylinder object for
$\calC[W^{-1}]$ in the model category $( \SCat)_{\calC/}$, so that $h$ is the desired
homotopy from $f$ to $g$. 
\end{proof}

\begin{lemma}\label{swimcase}
Let $f: \calC \rightarrow \calD$ be a $\bfS$-enriched functor, and let
$\calM$ be the {\it categorical mapping cylinder} of $f$, defined as follows:
\begin{itemize}
\item[$(1)$] An object of $\calM$ is either an object of $\calC$ or an object of $\calD$.
\item[$(2)$] Given a pair of objects $X,Y \in \calM$, we have
$$ \bHom_{\calM}(X,Y) = \begin{cases} \bHom_{\calC}(X,Y) & \text{if } X,Y \in \calC \\
\bHom_{\calD}( X,Y) & \text{if } X,Y \in \calD \\
\bHom_{\calD}( fX, Y) & \text{if } X \in \calC, Y \in \calD \\
\emptyset & \text{if } X \in \calD, Y \in \calC. \end{cases} $$
Here $\emptyset$ denotes the initial object of $\bfS$.
\end{itemize}
We observe that there is a canonical retraction $j$ of $\calM$ onto $\calD$, described by the formula $$j(X) = \begin{cases} fX & \text{if } X \in \calC \\
X & \text{if } X \in \calD. \end{cases}$$
Let $W$ be a collection of morphisms in $\calM$ with the following properties:
\begin{itemize}
\item[$(i)$] For each $w \in W$, $j(w)$ is an identity morphism in $\calD$.
\item[$(ii)$] For every object $C \in \calC$, the morphism $C \rightarrow fC$
in $\calM$ classifying the identity map from $fC$ to itself belongs to $W$.
\end{itemize}
Assumption $(i)$ implies that the map $j$ canonically extends to a map
$\overline{j}: \calM[W^{-1}] \rightarrow \calD$. The map $\overline{j}$
is a weak equivalence of $\bfS$-enriched categories.
\end{lemma}

\begin{proof}
It will suffice to show that composition with $\overline{j}$ induces a bijection
$$ \Hom_{ \h{\SCat} }( \calD, \calA) \rightarrow \Hom_{ \h{\SCat}}( \calM[W^{-1}], \calA)$$
for every $\bfS$-enriched category $\calA$. Equivalently, we must show that the map $$t: \Hom_{\h{\SCat}}( \calD, \calA) \rightarrow \Hom^{W}_{\h{\SCat}}(\calM, \calA)$$
is bijective, where $\Hom_{\h{\SCat}}^{W}( \calM, \calA)$ is defined as in Proposition \ref{postcan}.
The map $t$ has a section $t'$, given by composition with the inclusion $\calD \rightarrow \calM$.
It will therefore suffice to show that $t \circ t'$ is the identity on $\Hom_{\h{\SCat}}^{W}(\calM, \calA)$. 

Using Lemma \ref{tubble} and Corollary \ref{suff}, we can reduce to the case where $\calA = \bfA^{\degree}$, where $\bfA$ is a combinatorial $\bfS$-enriched model category. Invoking Proposition \ref{gumbaa}, we deduce that every element $[g] \in \Hom_{\h{\SCat}}( \calM, \calA)$ can be represented
by a diagram $g: \calM \rightarrow \bfA^{\degree}$. We wish to prove that
$g$ and $g \circ i \circ j$ are homotopic. We observe that there is a canonical natural transformation $\alpha: g \rightarrow g \circ i \circ j$. Moreover, if $g$ carries
each element of $W$ to an equivalence in $\bfA^{\degree}$, then assumption
$(ii)$ guarantees that $\alpha$ is a weak equivalence in the model category $\bfA^{\calM}$. We now invoke Proposition \ref{gumbaa} to deduce that $g$ and $g \circ i \circ j$ are homotopic as desired.
\end{proof}

\begin{definition}
Let $A$ be a partially ordered set. An {\it $A$-filtered $\bfS$-enriched category}
is a $\bfS$-enriched category $\calC$ together with a map $r: \Ob(\calC) \rightarrow A$
with the following property: if $C, D \in \calC$ and $r(C) \nleq r(D)$, then
$\bHom_{\calC}(C,D) \simeq \emptyset$, where $\emptyset$ denotes an initial object of $\bfS$.

If $\calC$ is an $A$-filtered $\bfS$-enriched category and $a \in A$, then we let
$\calC_{\leq a}$ denote the full subcategory of $\calC$ spanned by those objects
$C \in \calC$ such that $r(C) \leq a$.
\end{definition}

\begin{remark}\label{sabreton}
Let $\calC$ be an $A$-filtered $\bfS$-enriched category, and let
$\psi: W \rightarrow \Hom_{\SCat}( [1]_{\bfS}, \calC)$ be a map of sets.
For each $a \in A$, we let $W_{a} \subseteq W$ be the subset
consisting of those elements $w \in W$ such that the morphism
$\psi(w)$ belongs to $\calC_{a}$. This data determines a diagram
$\chi^{W}: A \rightarrow \SCat$, described by the formula
$a \mapsto \calC_{\leq a}[ W_{a}^{-1} ]$. Moreover, we have a canonical isomorphism
of $\bfS$-enriched categories $\calC[W^{-1}] \simeq \colim(\chi)$.
\end{remark}

Using the small object argument, we easily deduce the following result:

\begin{lemma}\label{tooter}
Let $A$ be a partially ordered set, and let $\calC$ be an $A$-filtered
$\bfS$-enriched category. Then there exists a $\bfS$-enriched functor
$f: \calC' \rightarrow \calC$ with the following properties:
\begin{itemize}
\item[$(1)$] The functor $f$ is bijective on objects, and for every
pair of objects $C,D \in \calC'$, the map $\bHom_{\calC}(C,D)
\rightarrow \bHom_{\calC}(fC, fD)$ is a trivial fibration in $\bfS$. In particular,
$f$ is a weak equivalence of $\bfS$-enriched categories.

\item[$(2)$] The $A$-filtration on $\calC$ induces an $A$-filtration on $\calC'$. In other
words, if $C, D \in \calC'$ and $r(fC) \nleq r(fD)$, then $\bHom_{\calC'}(C,D)$ is
an initial object of $\bfS$.

\item[$(3)$] The diagram $A \rightarrow \SCat$ described by the formula
$a \mapsto \calC'_{\leq a}$ is projectively cofibrant.
\end{itemize}
\end{lemma}

\begin{proposition}\label{scun}
Let $A$ be a partially ordered set, $\calC$ be an $A$-filtered
$\bfS$-enriched category, and $\psi: W \rightarrow \Hom_{ \SCat}( [1]_{\bfS}, \calC)$
a map of sets. Let $\chi: A \rightarrow \SCat$ be defined as in Remark \ref{sabreton}. Then the
isomorphism $\colim \chi \simeq \calC[W^{-1}]$ exhibits $\calC$ as the homotopy colimit
of the diagram $\chi$.
\end{proposition}

\begin{proof}
Choose a map $\calC' \rightarrow \calC$ as in Lemma \ref{tooter}, and
choose a map $\psi': W \rightarrow \Hom_{\SCat}( [1]_{\bfS}, \calC' )$ lifting
$\psi$, and let $\chi': A \rightarrow \SCat$ be defined as in Remark \ref{sabreton}.
Then we have a canonical map $\chi' \rightarrow \chi$, which is a cofibrant
replacement for $\chi$ in the model category $\Fun( A, \SCat)$. It will therefore suffice to show that the induced map
$\calC'[W^{-1}] \simeq \colim \chi' \rightarrow \colim \chi \simeq \calC[W^{-1}]$ is a weak equivalence of
$\bfS$-enriched categories, which follows immediately from Remark \ref{summat}.
\end{proof}

\begin{definition}\index{gen}{Grothendieck construction}\index{not}{Groth@$\Groth(p)$}\label{swype}
\index{not}{@$W(p)$}
Let $A$ be a partially ordered set, and let $p: A \rightarrow \SCat$ be an $A$-indexed diagram of $\bfS$-enriched categories. Let us denote the image of $a \in A$ under $p$ by $\calC_{a}$.

The {\it Grothendieck construction on $p$} is a category
$\Groth(p)$ defined as follows:
\begin{itemize}
\item[$(1)$] The objects of $\Groth(p)$ are pairs $(a, C)$, where $a \in A$ and
$C \in \calC_{a}$.
\item[$(2)$] Given a pair of objects $(a,C), (a', C')$ in $\Groth(p)$, we set
$$ \bHom_{ \Groth(p) }( (a,C), (a',C') ) = \begin{cases} \bHom_{\calC_{a'}}( p^{a'}_{a} C,
C') & \text{if } a \leq a' \\
\emptyset & \text{otherwise.} \end{cases}$$
Here $p^{a'}_{a}$ denotes the functor $\calC_{a} \rightarrow \calC_{a'}$ determined by
$p$, and $\emptyset$ denotes an initial object of $\bfS$.
\item[$(3)$] Composition in $\Groth(p)$ is defined in the obvious way.
\end{itemize}

We observe that $\Groth(p)$ is $A$-filtered via the map
$r: \Ob( \Groth(p) ) \rightarrow A$ given by the formula
$r(a,C) = a$. We let $W(p)$ denote the collection of all morphisms in
$\Groth(p)$ of the form $\alpha: (a, C) \rightarrow (a', p^{a'}_{a} C)$, where
$a \leq a'$ and $\alpha$ corresponds to the identity in $\calC_{a'}$. 

For each $a \in A$, there is a canonical functor
$\pi_{a}: \Groth(p)_{\leq a} \rightarrow \calC_{a}$, given by the formula
$(C, a') \mapsto p^{a}_{a'}(C)$. We note that $\pi$ carries
each element of $W(p)_{a}$ to an identity map in $\calC_{a}$, so that
$\pi_{a}$ canonically extends to a map $\overline{\pi}_{a}: \Groth(p)_{\leq a}[ W(p)_{a}^{-1}] \rightarrow \calC_{a}$. The maps $\overline{\pi}_{a}$ are functorial in
$a$, and therefore determine a map of diagrams $\chi(p) \rightarrow p$, where
$\chi(p)$ is defined as in Remark \ref{sabreton}.
\end{definition}

\begin{lemma}\label{cutta}
Let $p: A \rightarrow \SCat$ be as in Definition \ref{swype}. Then for each
$a \in A$, the map $\overline{\pi}_{a}: \Groth(p)_{\leq a}[ W(p)_{a}^{-1}] \rightarrow \calC_{a}$
is a weak equivalence of $\bfS$-enriched categories.
\end{lemma}

\begin{proof}
This is a special case of Lemma \ref{swimcase}. 
\end{proof}

\begin{lemma}\label{toopa}
Let $p: A \rightarrow \SCat$ be as in Definition \ref{swype}. Then
there is a canonical isomorphism $\Groth(p)[ W(p)^{-1} ] \simeq \hocolim(p)$
in the homotopy category $\h{\SCat}$.
\end{lemma}

\begin{proof}
Combine Lemma \ref{cutta} with Proposition \ref{scun}.
\end{proof}

\begin{lemma}\label{twoface}
Let $\calC$ and $\calD$ be small $\bfS$-enriched categories.
Let $W$ be a collection of morphisms in $\calC$, and let
$W'$ be the collection of all morphisms in $\calC \otimes \calD$ of the form
$w \otimes \id_{D}$, where $w \in W$ and $D \in \calD$. Then the canonical map
$$ ( \calC \otimes \calD )[ {W'}^{-1} ] \rightarrow \calC[W^{-1}] \otimes \calD$$
is a weak equivalence of $\bfS$-enriched categories.
\end{lemma}

\begin{proof}
It will suffice to show that for every $\bfS$-enriched category $\calA$, 
the induced map
$$ \phi: \Hom_{ \h{\SCat} }( \calC[W^{-1}] \otimes \calD, \calA) \rightarrow
\Hom_{ \h{\SCat} }( ( \calC \otimes \calD)[ {W'}^{-1} ], \calA) $$
is bijective. Using Lemma \ref{tubble} and Corollary \ref{suff}, we can reduce
to the case where $\calA = \bfA^{\degree}$, where $\bfA$ is a combinatorial $\bfS$-enriched
model category. We now invoke Propositions \ref{gumbarr} and \ref{postcan} to get a chain of bijections
\begin{eqnarray*}
\Hom_{ \h{\SCat} }( \calC[W^{-1}] \otimes \calD, \bfA^{\degree}) & \simeq &
\Hom_{\h{\SCat} }( \calC[W^{-1}], ( \bfA^{\calD} )^{\degree} ) \\
& \simeq & \Hom_{\h{\SCat}}^{W}( \calC, ( \bfA^{\calD} )^{\degree} ) \\
& \simeq & \Hom_{\h{\SCat}}^{W'}( \calC \otimes \calD, \bfA^{\degree} ) \\
& \simeq & \Hom_{\h{\SCat}}( (\calC \otimes \calD)[W'], \bfA^{\degree} )
\end{eqnarray*}
whose composition is the map $\phi$.
\end{proof}

\begin{theorem}\label{tubba}
Let $A$ be a partially ordered set, and let $\calD$ be a $\bfS$-enriched category.
Then the functor $\calC \mapsto \calC \otimes \calD$ commutes with $A$-indexed homotopy colimits. In other words, if $p: A \rightarrow \SCat$ is a projectively cofibrant diagram,
and $p': A \rightarrow \SCat$ is defined by $p'(a) = p(a) \otimes \calD$, then the canonical isomorphism $\colim(p') \simeq \colim(p) \otimes \calD$ exhibits $\colim(p) \otimes \calD$
as a homotopy colimit of the diagram $p'$.
\end{theorem}

\begin{proof}
In view of Lemma \ref{toopa}, it will suffice to show that the canonical map
$h: \Groth(p')[ W(p')^{-1} ] \rightarrow \Groth(p)[W(p)^{-1}] \otimes \calD$
is a weak equivalence of $\bfS$-enriched categories. This is a special case of
Lemma \ref{twoface}.
\end{proof}

\subsection{Exponentiation in Model Categories}\label{camper}

Let $\calC$ be a category which admits finite products, containing a pair of objects $X$ and $Y$.
An {\it exponential of $X$ by $Y$} is an object $X^Y \in \calC$ together with a map
$e: X^Y \times Y \rightarrow X$, with the following universal property: for every
object $W \in \calC$, the composition
$$ \Hom_{\calC}( W, X^Y) \rightarrow \Hom_{\calC}( W \times Y, X^Y \times Y)
\stackrel{\circ e}{\rightarrow} \Hom_{\calC}(W \times Y, Z)$$
is bijective.\index{gen}{exponential}

Our goal in this section is to study the existence of exponentials in the homotopy category
of a model category $\bfA$. Suppose given a pair of objects $X, Y \in \bfA$, such that there
exists an exponential of $[X]$ by $[Y]$ in the homotopy category $\h{\bfA}$. 
We can then represent this exponential as $[Z]$, for some object $Z \in \bfA$. 
Without loss of generality, we may assume that $X$, $Y$, and $Z$ are fibrant and cofibrant,
so that we have a canonical identification $[Z] \times [Y] \simeq [Z \times Y]$. However,
we encounter a technical difficulty: the evaluation map $[Z] \times [Y] \rightarrow [X]$ need
not be representable by any morphism from $Z \times Y$ to $X$ in the category $\bfA$, because
$Z \times Y$ need not be cofibrant. We wish to work in certain contexts where this difficulty
does arise (for example, where $\bfA$ is the category of simplicial categories). For this reason we are forced to work with the following somewhat cumbersome definition:

\begin{definition}\index{gen}{weak exponential}\index{gen}{exponential!weak}\label{ki}
Let $\bfA$ be a model category. We will say that a diagram
$$ \xymatrix{ &  P \ar[dl]^{p} \ar[dr] & \\
Z \times Y & & X}$$
{\it exhibits $Z$ as a weak exponential of $X$ by $Y$} if the following conditions are satisfied:
\begin{itemize}
\item[$(1)$] The map $p$ exhibits $P$ as a homotopy product of $Z$ and $Y$; in other words,
the induced map $[p]: [P] \rightarrow [Z] \times [Y]$ is an isomorphism in the homotopy category
$\h{\bfA}$. 
\item[$(2)$] The composition
$[Z] \times [Y] \stackrel{ [p]^{-1} }{\rightarrow} [P] \rightarrow [X]$
exhibits $[Z]$ as an exponential of $[X]$ by $[Y]$ in the homotopy category $\h{\bfA}$.
\end{itemize}

We will say that a map $Z \times Y \rightarrow X$ {\it exhibits $Z$ as an exponential of
$X$ by $Y$} if the diagram
$$ \xymatrix{ & Z \times Y \ar[dl]^{\id} \ar[dr] & \\
Z \times Y & & X }$$
satisfies $(1)$ and $(2)$.
\end{definition}

\begin{remark}\index{gen}{standard diagram}
Suppose given a diagram
$$ \xymatrix{ & P \ar[dl]^{p} \ar[dr] &  \\
Z \times Y & & X}$$
as in Definition \ref{ki}. We will say that this diagram is {\em standard} if
$X, Y, Z \in \bfA$ are fibrant, and the map $p$ is a trivial fibration.

Suppose $X$ and $Y$ are fibrant objects of $\bfA$ such that there exists an exponential
of $[X]$ by $[Y]$ in the homotopy category $\h{\bfA}$. Without loss of generality, this
exponential has the form $[Z]$ where $Z$ is a fibrant object of $\bfA$. We can then
choose a trivial fibration $P \rightarrow Z \times Y$, where $P$ is cofibrant. The evaluation
map $[Z \times Y] \simeq [Z] \times [Y] \rightarrow [X]$ is then representable by
a map $P \rightarrow X$ in $\bfA$, so that we obtain a {\em standard} diagram which exhibits
$Z$ as a weak exponential of $X$ by $Y$.
\end{remark}

\begin{remark}
Suppose given a diagram
$$ \xymatrix{ & P \ar[dl]^{p} \ar[dr] & \\
Z \times Y & & X}$$
in a model category $\bfA$. The condition that this diagram exhibits
$Z$ as a weak exponential of $X$ by $Y$ depends only on the image of this diagram
in the homotopy category $\h{\bfA}$. We are therefore replace the above diagram by a weakly equivalent diagram when testing whether or not the conditions of Definition \ref{ki} are satisfied.
\end{remark}

\begin{remark}\label{toofus}
Let $\Adjoint{F}{\bfA}{\bfB}{G}$ be a Quillen equivalence of model categories.
Suppose given a standard diagram
$$ \xymatrix{&  P \ar[dl]^{p} \ar[dr] &  \\
Z \times Y & & X}$$
in $\calB$. Then this diagram exhibits $Z$ as a weak exponential of $X$ by $Y$ in
$\bfB$ if and only the associated diagram
$$ \xymatrix{ & GP \ar[dl] \ar[dr] & \\
GZ \times GY & & GX}$$
exhibits $GZ$ as a weak exponential of $GX$ by $GY$ in $\bfA$.
\end{remark}

To work effectively with weak exponentials, we need to introduce an additional assumption.

\begin{definition}\label{tumba}\index{gen}{multiplication!and homotopy colimits}
Let $\bfA$ be a combinatorial model category containing a fibrant object
$Y$. We will say that {\it multiplication by $Y$ preserves homotopy colimits}
if the following condition is satisfied:
\begin{itemize}
\item[$(\ast)$] Let $A$ be a (small) partially ordered set, let $F: A \rightarrow \bfA$
be a projectively cofibrant diagram, and let $F': A \rightarrow \bfA$ be another strongly
cofibrant diagram equipped with a natural transformation $F'(a) \rightarrow F(a) \times Y$
which is weak equivalence for each $a \in A$. Then the induced map
$\colim F' \rightarrow (\colim F) \times Y$ exhibits $\colim F'$ as a homotopy product
of $Y$ with $\colim F$ in $\bfA$.
\end{itemize}
We will say that {\it multiplication in $\bfA$ preserves homotopy colimits} if
condition $(\ast)$ is satisfied for every fibrant object $Y \in \bfA$.
\end{definition}

\begin{remark}
Definition \ref{tumba} refers only to homotopy colimits indexed by partially ordered sets.
However, every diagram indexed by an arbitrary category can be replaced by a diagram
indexed by a partially ordered set having the same homotopy colimit. We formulate and prove a precise statement to this effect (in the language of $\infty$-categories) in \S \ref{c4s2}. However, we will not need (or use) any such result in this appendix.
\end{remark}

\begin{remark}\label{canus}
Let $\Adjoint{F}{\bfA}{\bfB}{G}$ be a Quillen equivalence between combinatorial
model categories, and let $Y \in \bfB$ be a fibrant object. Then multiplication by
$Y$ preserves homotopy colimits in $\bfB$ if and only if multiplication by
$G(Y)$ preserves homotopy colimits in $\bfA$. Since the right derived functor $RG$
is essentially surjective on homotopy categories, we see that
multiplication in $\bfB$ preserves homotopy colimits if and only if multiplication in $\bfA$ preserves homotopy colimits.
\end{remark}

\begin{example}\label{canuss}
Let $\bfS$ be an excellent model category with respect to the symmetric monoidal structure
given by Cartesian product in $\bfS$. Then multiplication in $\SCat$ preserves homotopy colimits.
This is precisely the content of Theorem \ref{tubba}.
\end{example}

\begin{lemma}\label{alem}
Let $S$ be a collection of simplicial sets satisfying the following conditions:
\begin{itemize}
\item[$(i)$] The simplicial set $\Delta^0$ belongs to $S$.
\item[$(ii)$] If $f: X \rightarrow Y$ is a weak homotopy equivalence, then
$X \in S$ if and only if $Y \in S$.
\item[$(iii)$] For every small partially ordered set $A$, if
$F: A \rightarrow \sSet$ is a projectively cofibrant diagram such that
each $F(a) \in S$, then $\colim F \in S$.
\end{itemize}
\end{lemma}

\begin{proof}
Using $(ii)$ and $(iii)$, we deduce that if $F: A \rightarrow \sSet$ is
{\em any} diagram such that each $F(a)$ belongs to $S$, then
the homotopy colimit of $F$ belongs to $S$. In particular, $S$ is closed under the formation of coproducts and homotopy pushouts.

We now prove by induction on $n$ that every $n$-dimensional simplicial set $X$ belongs to $S$.
For this, we observe that there is a homotopy pushout diagram
$$ \xymatrix{ B \times \bd \Delta^n \ar[d] \ar[r] & B \times \Delta^n \ar[d] \\
\sk^{n-1} X \ar[r] & X, }$$
where $B$ denotes the set of $n$-simplices of $X$. The simplicial sets
$B \times \bd \Delta^n$ and $\sk^{n-1} X$ belong to $S$ by the inductive hypothesis.
The simplicial set $B \times \Delta^n$ is weakly equivalent to the constant simplicial
set $B$, which belongs to $S$ in view of $(i)$ and the fact that $S$ is stable under coproducts.
Since $S$ is stable under homotopy pushouts, we conclude that $X \in S$ as desired.

An arbitrary simplicial set $X$ can be written as the colimit of a projectively cofibrant diagram
$$ \sk^{0} X \subseteq \sk^{1} X \subseteq \sk^{2} X \subseteq \ldots$$
and therefore belongs to $S$ by assumption $(iii)$.
\end{proof}

%\begin{lemma}\label{alem}
%Let $\bfA$ be a combinatorial simplicial model category containing a fibrant-cofibrant object
%$Y$, and choose a map $f: Y \rightarrow E$, where $E$ is a cofibrant replacement for the final object of $\bfA$. Suppose that multiplication by $Y$ preserves homotopy colimits in $\bfA$.
%Then for every simplicial set $K$, the induced map
%$$\psi_{K}: Y \otimes K \rightarrow Y \times (E \otimes K)$$
%exhibits $Y \otimes K$ as a homotopy product of $Y$ with $E \otimes K$ in $\bfS$.
%\end{lemma}

%\begin{proof}
%Let $S$ denote the collection of all simplicial sets for which $\psi_K$ exhibits
%$Y \otimes K$ as a homotopy product of $Y$ with $E \otimes K$. Since multiplication by $Y$ preserves homotopy colimits, the collection $S$ satisfies the hypotheses of Lemma \ref{rad}, so
%that every simplicial set belongs to $S$.
%\end{proof}

\begin{proposition}\label{scat}
Let $\bfA$ be a combinatorial simplicial model category
containing a standard diagram
$$ \xymatrix{ & P \ar[dl]^{p} \ar[dr] & \\
Z \times Y & & X.}$$
Assume further that multiplication by $Y$ preserves homotopy colimits in $\bfA$.
The following conditions are equivalent:
\begin{itemize}
\item[$(i)$] The above diagram exhibits $Z$ as a weak exponential of 
$X$ by $Y$.

\item[$(ii)$] Let $W$ and $W'$ be cofibrant objects of $\bfA$, and
$W' \rightarrow W \times Y$ a map which exhibits $W'$ as a homotopy product
of $W$ and $Y$. Then the induced map
$$\bHom_{\bfA}(W,Z) \times_{ \bHom_{\bfA}( W', Z \times Y) } \bHom_{\bfA}(W', P)
\rightarrow \bHom_{\bfA}(W', X)$$
is a homotopy equivalence of Kan complexes.
\end{itemize}
\end{proposition}

\begin{remark}\label{kilpot}
In the situation of part $(ii)$ of Proposition \ref{scat}, the projection map
$\bHom_{\bfA}( W', P) \rightarrow \bHom_{\bfA}( W', Z \times Y)$ is a trivial Kan fibration, so
the fiber product
$\bHom_{\bfA}(W,Z) \times_{ \bHom_{\bfA}( W', Z \times Y) } \bHom_{\bfA}(W', P)$ is automatically
a Kan complex which is homotopy equivalent to $\bHom_{\bfA}(W,Z)$.
\end{remark}

\begin{proof}[Proof of Proposition \ref{scat}]
First suppose that $(ii)$ is satisfied. We wish to show that for every object
$[W] \in \h{\bfA}$, the map
$$ \Hom_{\h{\bfA}}( [W], [Z] ) \rightarrow
\Hom_{ \h{\bfA} }( [W] \times [Y], [Z] \times [Y])
\simeq \Hom_{ \h{\bfA}}( [W] \times [Y], [P] ) \rightarrow \Hom_{\h{\bfA}}( [W] \times [Y], [P] )$$
is bijective. Without loss of generality, we may assume that $[W]$ is the homotopy equivalence class
of a fibrant-cofibrant object $W \in \bfA$. Choose a cofibrant replacement $W' \rightarrow W \times Y$.
We observe that the map in question can be identified with the composition
$$ \pi_0 \bHom_{ \bfA}( W, Z) \rightarrow \pi_0 \bHom_{ \bfA}( W', Z \times Y)
\simeq \pi_0 \bHom_{\bfA}( W', P) \rightarrow \pi_0 \bHom_{\bfA}( W',X),$$
which is bijective in view of $(ii)$ and Remark \ref{kilpot}.

We now assume $(i)$ and prove $(ii)$. It will suffice to show that for every simplicial
set $K$, the induced map
$$\Hom_{\h{\sSet}}( K, \bHom_{\bfA}(W,Z) \times_{ \bHom_{\bfA}( W', Z \times Y) } \bHom_{\bfA}(W', P)
) \rightarrow \Hom_{\h{\sSet}}(K, \bHom_{\bfA}(W',X) )$$
is a bijection. Using Remark \ref{kilpot}, we can identify the left side with the
set $\Hom_{\h{\sSet}}(K, \bHom_{\bfA}(W,Z)) \simeq \Hom_{\h{\bfA}}( W \otimes K, Z)$.
Similarly, the right side can be identified with $\Hom_{\h{\bfA}}( W' \otimes K, X)$.
In view of assumption $(i)$, it will suffice to show that the map
$\beta_{K}: W' \otimes K \rightarrow Y \times (W \otimes K)$
exhibits $W' \otimes K$ as a homotopy product of $Y$ with $W \otimes K$.
The collection of simplicial sets $K$ with this property clearly contains
$\Delta^0$ and is stable under weak homotopy equivalence. The assumption
that multiplication by $Y$ preserves homotopy colimits guarantees that the hypotheses of
Lemma \ref{alem} are satisfied, so that the desired conclusion holds for {\em every} simplicial set $K$.
\end{proof}

\begin{lemma}\label{tukka}
Let $\bfA$ be a combinatorial model category and $i: B_0 \rightarrow B$
an inclusion of partially ordered sets. Suppose that there exists a retraction
$r: B \rightarrow B_0$, such that $r(b) \leq b$ for each $b \in B$.
Let $F: B \rightarrow \bfA$ be a diagram. Then a map
$\alpha: X \rightarrow \lim(F)$ in $\bfA$ exhibits $X$ as a homotopy limit of
$F$ if and only if $\alpha$ exhibits $X$ as a homotopy limit of $i^{\ast} F$.
\end{lemma}

\begin{proof}
Without loss of generality, we may assume that $F$ is injectively fibrant.
We have a canonical isomorphism $\lim(F) \simeq \lim( i^{\ast} F )$.
It will therefore suffice to show that the functor $i^{\ast}$ preserves injective fibrations.
It now suffices to observe that $i^{\ast}$ is right adjoint to $r^{\ast}$, and that
the functor $r^{\ast}$ preserves weak trivial cofibrations.
\end{proof}

\begin{lemma}\label{constancy}
Let $\bfA$ be a combinatorial model category, $\calC$ a small category, $F: \calC \rightarrow \bfA$ a diagram, and $\alpha: X \rightarrow \lim(F)$ a morphism in the category $\bfA$.
Suppose that:
\begin{itemize}
\item[$(i)$] For each $C \in \calC$, the induced map $X \rightarrow F(C)$ is a weak equivalence in $\bfA$.
\item[$(ii)$] The category $\calC$ has a final object $C_0$. 
\end{itemize}
Then $\alpha$ exhibits $X$ as a homotopy limit of the diagram $F$.
\end{lemma}

\begin{proof}
Without loss of generality, we may assume that the diagram $F$ is projectively fibrant.
Let $F': \calC \rightarrow \bfA$ be defined by the formula $F'(C) = F(C_0)$.
We observe that, for every $G \in \Fun(\calC, \bfA)$, we have
$\Hom_{ \Fun(\calC, \bfA) }( G, F') = \Hom_{\bfA}( G(C_0), F(C_0) )$. In particular,
we have a canonical map $\beta: F \rightarrow F'$. Condition $(i)$ guarantees that
$\beta$ is a weak equivalence. Since
$F(C_0) \in \bfA$, is fibrant, the diagram $F'$ is injectively fibrant. It therefore suffices to show that
the induced map $X \rightarrow \lim(F') \simeq F(C_0)$ is a weak equivalence, which follows from
$(i)$.
\end{proof}

\begin{lemma}\label{stride}
Let $\bfA$ be a combinatorial model category,
$A$ a partially ordered set, and set $B = \{ (a,b) \in A^{op} \times A: a \geq b \}$,
regarded as a partially ordered subset of $A^{op} \times A$. Let
$\pi: B \rightarrow A^{op}$ denote the projection onto the first factor.

Suppose given diagrams $F: B \rightarrow \bfA$, $G: A \rightarrow \bfA$,
and a natural transformation $\alpha: \pi^{\ast}(G) \rightarrow F$, which induces
weak equivalences $G(a) \rightarrow F(a,b)$ for each $(a,b) \in B$.
Then $\alpha$ exhibits $G$ as a homotopy right Kan extension of $F$.
\end{lemma}

\begin{proof}
In view of Proposition \ref{sabke}, it will suffice to show that
for each $a_0 \in A$, the transformation $\alpha$ exhibits
$G(a_0)$ as a homotopy limit of the diagram
$F | \{ (a,b) \in B: a \leq a_0 \}$. 
Let $B_0 = \{ (a,b) \in B: a = a_0 \}$. In view of Lemma \ref{tukka}, it will
suffice to show that $\alpha$ exhibits $G(a_0)$ as a limit of the diagram
$F_0 = F | B_0$. This follows immediately from Lemma \ref{constancy}.
\end{proof}

\begin{proposition}\label{psygood}
Let $\bfA$ be a combinatorial model category and $A = A_0 \cup \{ \infty \}$
a partially ordered set with a largest element $\infty$. Let 
$B = \{ (a,b) \in A^{op} \times A: a \geq b \}$, regarded as a partially ordered subset
of $A^{op} \times A$. 

Suppose given an object $X \in \bfA$ together with functors
$Y: A \rightarrow \bfA$, $Z: A^{op} \rightarrow \bfA$,
$P: B \rightarrow \bfA$ and diagrams $\sigma_{a,b}$: 
$$ \xymatrix{ & P(a,b) \ar[dl] \ar[dr] & \\
Z(a) \times Y(b) & & X}$$
which depend functorially on $(a,b) \in B$.
Suppose further that:
\begin{itemize}
\item[$(i)$] Each diagram $\sigma_{a,b}$ exhibits $P(a,b)$ as a homotopy product
of $Z(a)$ and $Y(b)$ in $\bfA$.

\item[$(ii)$] The diagrams $\sigma_{a,a}$ exhibit $Z(a)$ as a weak exponential
of $X$ by $Y(a)$.

\item[$(iii)$] Multiplication in $\bfA$ preserves homotopy colimits.

\item[$(iv)$] The diagram $Y$ exhibits $Y(\infty)$ as a homotopy colimit of
$Y_0 = Y|A_0$.
\end{itemize}
Then the diagram $Z$ exhibits $Z(\infty)$ as the homotopy limit of
the diagram $Z_0 = Z | A_0^{op}$. 
\end{proposition}

\begin{proof}
Making fibrant replacements if necessary, we may assume that each diagram
$\sigma_{a,b}$ is standard. 
According to the main result of \cite{combmodel}, there exists a Quillen equivalence
$\Adjoint{F}{\bfA'}{\bfA}{G}$ where $\bfA'$ is a combinatorial {\em simplicial} model category.
In view of Remark \ref{toofus}, we may replace $\bfA$ by $\bfA'$
and thereby reduce to the case where $\bfA$ is a simplicial model category. 

In view of Proposition \ref{usecoinc}, it will suffice to prove the following: for every
fibrant-cofibrant object $C \in \bfA$, if we define $G: A^{op} \rightarrow \sSet$
by the formula $G(a) = \bHom_{\bfA}( C, Z(a) )$, then
$G$ exhibits $G(\infty)$ as a homotopy limit of the diagram $G|A_0^{op}$.

Let $W: A \rightarrow \bfA$ be a cofibrant replacement for the functor
$a \mapsto C \times Y(a)$. Let $G': A^{op} \rightarrow \sSet$ be defined
by the formula $G'(a) = \bHom_{\bfA}( W(a), X)$. 

Define $G'': B \rightarrow \sSet$ by the formula
$$ G''(a,b) = \bHom_{\bfA}( C, Z(a) ) \times_{ \bHom_{\bfA}( W(a), Z(a) \times Y(b) ) }
\bHom_{\bfA}( W(a), P(a,b) ).$$
Let $\pi: B \rightarrow A^{op}$ denote projection onto the first factor, so that
we have natural transformations of diagrams
$$ \pi^{\ast} G \stackrel{\alpha}{\leftarrow} G'' \stackrel{\beta}{\rightarrow} \pi^{\ast} G'.$$
We observe that $\beta$ induces a trivial Kan fibration $G''(a,b) \rightarrow G'(a)$
for all $(a,b) \in B$. In particular, for $a \leq b$ the induced map
$G''(a,a) \rightarrow G''(a,b)$ is a homotopy equivalence.
Condition $(ii)$ guarantees that $\alpha$ induces a
homotopy equivalence $G''(a,b) \rightarrow G(a)$ if $a = b$, and therefore for all
$(a,b) \in B$.

Using Lemma \ref{stride}, we conclude that $\alpha$ and $\beta$ exhibit
$G$ and $G'$ as homotopy right Kan extensions of $G''$ along $\pi$. 
In particular, $G$ and $G'$ are equivalent in the homotopy category
$\h{ \Fun( A^{op}, \bfA)}$. Assumptions $(iii)$ and
$(iv)$ guarantee that $W$ exhibits $W(\infty)$ as the homotopy colimit
of $W | A_0$. Using Proposition \ref{usecoinc}, we deduce that
$G'$ exhibits $G'(\infty)$ as the homotopy limit of $G' | A_0^{op}$. 
It follows that $G$ exhibits $G(\infty)$ as the homotopy limit of
$G| A_0^{op}$, as desired.
\end{proof}

We conclude this section with an application of Proposition \ref{psygood}. 

\begin{proposition}\label{sturb}
Let $\bfS$ be an excellent model category in which the monoidal structure
is given by the Cartesian product. Let $\bfA$ be a combinatorial $\bfS$-enriched model category,
$A = A_0 \cup \{ \infty \}$ a partially ordered set with a largest element $\infty$, and $\{ \calC_{a} \}_{a \in A}$ a diagram of small $\bfS$-enriched categories indexed by $A$. Let
$\calU \subseteq \bfA$ be a chunk. For each $a \in A$, let $\calU^{\calC_{a}}_{f}$ denote the full
subcategory of $\calU^{\calC_a} \subseteq \bfA^{\calC_a}$ spanned by the projectively fibrant diagrams,
and let $W_{a}$ denote the collectinon of weak equivalences in $\calV_{a}$.
Assume that:
\begin{itemize}
\item[$(a)$] For each $a \in A$, the $\bfS$-enriched category $\calC_{a}$ is
cofibrant, and $\calU$ is a $\calC_{a}$-chunk of $\bfA$.

\item[$(b)$] The diagram $\{ \calC_{a} \}_{ a \in A}$ exhibits $\calC_{\infty}$ as a
homotopy colimit of the diagram $\{ \calC_{a} \}_{a \in A_0}$. 

\item[$(c)$] The chunk $\calU$ is small.
\end{itemize}

Then the induced diagram $\{ \calU^{\calC_{a}}_{f}[W_{a}^{-1}] \}_{a \in A}$ exhibits
$ \calU^{\calC_{\infty}}_{f}[ W_{\infty}^{-1} ]$ as a homotopy limit of the diagram
$ \{ \calU^{\calC_{a}}_{f}[ W_a^{-1} ] \}_{a \in A_0}$. 
\end{proposition}

Before proving Proposition \ref{sturb}, we need a simple lemma.

\begin{lemma}\label{kur}
Let $\bfS$ be an excellent model category, $\bfA$ a combinatorial $\bfS$-enriched
model category, and $\calU \subseteq \bfA$ a chunk. Let $\calU_{f}$ denote the
full subcategory of $\calU$ spanned by those objects which are fibrant in $\bfA$, and let
$W$ denote the collection of weak equivalences in $\calU_{f}$. Then the induced map
$\calU^{\degree} \rightarrow \calU_{f}[W^{-1}]$ is a weak equivalence of $\bfS$-enriched categories.
\end{lemma}

\begin{proof}
Let $W_0 = W \cap \calU^{\degree}$. Since every weak equivalence in $\calU^{\degree}$
is actually an equivalence, we conclude that the induced map
$\calU^{\degree} \rightarrow \calU^{\degree}[W_0^{-1}]$ is a weak equivalence.
It will therefore suffice to prove that the map
$i: \calU^{\degree}[W_0^{-1}] \rightarrow \calU_{f}[W^{-1}]$ is a weak equivalence.
Let $F$ be a $\bfS$-enriched fibrant replacement functor which carries $\calU$ to itself, so
that $F$ induces a map $j: \calU_{f}[W^{-1}]$ to $\calU^{\degree}[W_0^{-1}]$. We claim that $j$ is a homotopy inverse to $i$. To prove this, we observe that there is a natural transformation
$\alpha: \id \rightarrow F$, which we can identify with a map
$$\overline{h}: \calU_{f} \otimes [1]_{\bfS} \rightarrow \calU^{\degree}.$$
Let $W'_0$ be the collection of all morphisms in $\calU_{f} \otimes [1]_{\bfS}$
of the form $e \otimes \id$, where $e$ is an equivalence in $\calU_{f}$, and let
$W'_1$ be the collection of all morphisms of $\calU_f \otimes [1]_{\bfS}$ of the form
$\id \otimes g$, where $g: 0 \rightarrow 1$ is the tautological morphism in
$[1]_{\bfS}$. Let $W' = W'_0 \cup W'_1$, so that $\overline{h}$ determines a map
$$ h: ( \calU_{f} \otimes [1]_{\bfS} )[{W'}^{-1} ] \rightarrow \calU^{\degree}[W_0^{-1}].$$
We will prove that $h$ determines a homotopy from the identity to $j \circ i$, so that
$j$ is a left homotopy inverse to $i$. Applying the same argument to the
restriction $\overline{h} | \calU^{\degree} \otimes [1]_{\bfS}$ will show that
$j$ is a right homotopy inverse to $i$. 

To prove that $h$ gives the desired homotopy, it will suffice to show that the inclusions
$\{0\}, \{1\} \hookrightarrow [1]_{\bfS}$ induce weak equivalences
$$ \calU_{f}[W^{-1}] \rightarrow (\calU_{f} \otimes [1]_{\bfS})[{W'}^{-1}].$$
This follows immediately from Corollary \ref{suff} and Lemma \ref{twoface}. 
\end{proof}

\begin{proof}[Proof of Proposition \ref{sturb}]
Let $B = \{ (a,b) \in A^{op} \times A: a \geq b \}$. For
each $(a,b) \in B$, we define $P(a,b) = (\calU^{\calC_{a}}_{f} \times \calC_{b})[V_{a,b}^{-1}]$,
where $V_{a,b}$ is the collection of all morphisms of $\calU^{\calC_{a}}_{f} \times \calC_{b}$
of the form $e \otimes \id_{C}$, where $e \in W_a$ and $C \in \calC_{b}$. 
We have an evident family of diagrams $\sigma(a,b):$
$$ \xymatrix{ & P(a,b) \ar[dl] \ar[dr] & \\
\calU^{\calC_{a}}_{f}[W_{a}^{-1}] \times \calC_{b} & & \calU_{f}[W], }$$
where $\calU_{f}$ denotes the full subcategory of $\calU$ spanned by the fibrant objects,
and $W$ is the collection of weak equivalences in $\calU_{f} \subseteq \bfA$. 
To complete the proof, it will suffice to show that the hypotheses of Proposition
\ref{psygood} are satisfied. Condition $(i)$ follows from Lemma \ref{twoface},
condition $(iii)$ from Theorem \ref{tubba}, and condition $(iv)$ follows from $(b)$.
To prove $(ii)$, we observe that for each $a \in A$, the diagram
$\sigma(a,a)$ is weakly equivalent to the diagram
$$ \xymatrix{ & (\calU^{\calC_{a}})^{\degree} \times \calC_{a} \ar[dl]^{\sim} \ar[dr] & \\
( \calU^{\calC_{a}})^{\degree} \times \calC_{a} & & \calU^{\degree}. }$$
This diagram exhibits $(\calU^{\calC_{a}})^{\degree}$ as a weak exponential of
$\calU^{\degree}$ by $\calC_{a}$ by Corollary \ref{sniffle}. 
\end{proof}

\begin{corollary}\label{uspin}
Let $\bfS$ be an excellent model category in which the monoidal structure
is given by the Cartesian product. Let $\bfA$ be a combinatorial $\bfS$-enriched model category,
$A = A_0 \cup \{ \infty \}$ a partially ordered set with a largest element $\infty$, and $\{ \calC_{a} \}_{a \in A}$ a diagram of small $\bfS$-enriched categories indexed by $A$.

For each $a \in A$, let $\bfA^{\calC_{a}}_{f}$ denote the collection of projectively fibrant objects of
$\bfA^{\calC_a}$, and let $W_{a}$ denote the collection of weak equivalences in
$\bfA^{\calC_a}_{f}$. Assume that the diagram $\{ \calC_{a} \}_{a \in A}$ exhibits
$\calC_{\infty}$ as a homotopy colimit of the diagram $\{ \calC_{a} \}_{a \in A_0}$. Then
the induced diagram $\{ \bfA^{\calC_a}_{f}[W_a^{-1}] \}_{a \in A}$ exhibits
$\bfA^{\calC_{\infty}}[W_{\infty}^{-1}]$ as a homotopy limit of the diagram
$\{ \bfA^{\calC_a}_{f}[W_a^{-1}] \}_{a \in A_0}$.
\end{corollary}

\begin{proof}
%Choose a weak equivalence of diagrams $\{ \calC'_{a} \}_{a \in A} \rightarrow
%\{ \calC_{a} \}_{a \in A}$, where each $\calC'_{a}$ is cofibrant. Using
%Remark \ref{lsstr} and Proposition \ref{lesstrick}, we conclude that the induced map
%$\{ \bfA^{\calC'_{a}}[ {W'}_{a}^{-1}] \}_{a \in A} \rightarrow \{ \bfA^{\calC_a}[W_{a}^{-1}] \}_{a \in A}$
%is a weak equivalence of diagrams. We are therefore free to replace $\{ \calC_{a} \}_{a \in A}$
%by $\{ \calC'_{a} \}_{a \in A}$, and thereby reduce to the case where each $\calC_{a}$ is cofibrant.
Without loss of generality we may suppose that each $\calC_{a}$ is cofibrant.
The proof of Proposition \ref{smurff} shows that there exists a (small) regular cardinal $\kappa$
such that the collection of homotopy limit diagrams in $\Fun( A, \SCat)$ is stable under
$\kappa$-filtered colimits. This cardinal depends only on $A$ and $\bfS$, and remains invariant if we enlarge the universe. Using Lemma \ref{exchunk}, we can write
$\bfA$ as a $\kappa$-filtered union of full subcategories $\calU \subseteq \bfA$,
where $\calU$ is a $\calC_{a}$-chunk for each $a \in A$. We now conclude by
applying Proposition \ref{sturb}.
\end{proof}

\subsection{Localizations of Simplicial Model Categories}\label{turka}

Let $\bfA$ and $\bfA'$ be two model categories with the same underlying category. We say that $\bfA'$ is a {\it $($Bousfield$)$ localization} of $\bfA$ if:\index{gen}{localization!of a model category}\index{gen}{localization!Bousfield}
\begin{itemize}
\item[$(C)$] A morphism $f$ of $\calC$ is a cofibration in $\bfA$ if and only if $f$ is a cofibration in $\bfA'$.
\item[$(W)$] If a morphism $f$ of $\calC$ is a weak equivalence in $\bfA$, then $f$ is a weak equivalence in $\bfA'$.
\end{itemize}
It then follows also that:
\begin{itemize}
\item[$(F)$] If a morphism $f$ of $\calC$ is a fibration in $\bfA'$, then $f$ is a fibration in $\bfA$.
\end{itemize}

Our goal in this section is to study the localizations of a fixed model category $\bfA$, and to relate this to our study of localizations of presentable $\infty$-categories (\S \ref{invloc}). 

Let $\bfA$ be a simplicial model category. Let $\h{\bfA}$ be the homotopy category of $\bfA$, obtained from $\bfA$ by inverting all weak equivalences. Alternatively, we can obtain $\bfA$ by first passing to the full subcategory $\bfA^{\degree} \subseteq \bfA$ spanned by the fibrant-cofibrant objects, and then passing to the homotopy category of the simplicial category $\bfA^{\degree}$. From the second point of view, we see that $\h{\bfA}$ has a natural enrichment over the homotopy category $\calH$: if $X, Y \in \h{\bfA}$ are represented by fibrant-cofibrant objects
$\overline{X}, \overline{Y} \in \bfA$, then we let
$$ \bHom_{ \h{\bfA}}(X,Y) = [ \bHom_{\bfA}(\overline{X}, \overline{Y}) ].$$
Here $[K] \in \calH$ denotes the object of $\calH$ represented by a Kan complex $K$. In fact, this description is accurate if we assume only that $\overline{X}$ is cofibrant and $\overline{Y}$ fibrant.

Let $S$ be a collection of morphisms in $\h{\bfA}$. Then:
\begin{itemize}
\item[$(i)$] We will say that an object $Z \in \h{\bfA}$ is 
{\it $S$-local} if, for every morphism $f: X \rightarrow Y$ in $S$, the induced map
$$ \bHom_{\h{\bfA}}( Y, Z) \rightarrow \bHom_{\h{\bfA}}( X, Z)$$ is a homotopy equivalence.\index{gen}{$S$-local!object} We say that an object $\overline{Z} \in \bfA$ is $S$-local if its image in
$\h{\bfA}$ is $S$-local.
\item[$(ii)$] We will say that a morphism $f: X \rightarrow Y$ of $\h{\bfA}$ is an {\it $S$-equivalence} if, for every $S$-local object $Z \in \h{\bfA}$, the induced map
$$ \bHom_{\h{\bfA}}( Y, Z) \rightarrow \bHom_{\h{\bfA}}( X, Z)$$ is a homotopy equivalence. We say that a morphism $\overline{f}$ in $\bfA$ is an {\it $S$-equivalence} if its image in $\h{\bfA}$ is an $S$-equivalence.\index{gen}{$S$-equivalence}
\end{itemize}
If $\overline{S}$ is a collection of morphisms in $\h{\bfA}$, with image $S$ in $\h{\bfA}$, we will apply the same terminology: an object of $\bfA$ (or $\h{\bfA}$) is said to be {\it $\overline{S}$-local} if it is $S$-local, and morphism of $\bfA$ (or $\h{\bfA}$) will be said to be an {\it $\overline{S}$-equivalence}, if it is an $S$-equivalence.

\begin{lemma}\label{mtomp}
Let $\bfA$ be a left proper simplicial model category, let $S$ be a collection of morphisms
in $\h{\bfA}$, and let $i: A \rightarrow B$ be a cofibration in $\bfA$. The following conditions are equivalent:
\begin{itemize}
\item[$(1)$] The map $i$ is an $S$-equivalence.
\item[$(2)$] For every fibrant object $X \in \bfA$ which is $S$-local, the map
$\bHom_{\bfA}(B, X) \rightarrow \bHom_{\bfA}(A,X)$ is a trivial Kan fibration.
\end{itemize}
\end{lemma}

\begin{proof}
Choose a trivial fibration
$f: A' \rightarrow A$, where $A'$ is cofibrant, and choose a factorization
$$ A' \stackrel{i'}{\rightarrow} B' \stackrel{f'}{\rightarrow} B$$ of
$i \circ f$, where $i'$ is a cofibration and $f'$ is a trivial fibration. We have a commutative diagram
$$ \xymatrix{ A' \ar[r]^{i'} \ar[d]^{f} & B' \ar[d]^{g} \ar[dr]^{f'} & \\
A \ar[r]^{i} & A \coprod_{A'} B' \ar[r]^{j} & B. }$$
Since $f$ is a weak equivalence and $i'$ is a cofibration, the left properness of $\bfA$ guarantees that $g$ is a weak equivalence. It follows from the two-out-of-three property that $j$ is also a weak equivalence.

Suppose first that $(1)$ is satisfied. Let $X$ be an $S$-local fibrant object of $\bfA$. 
The map $p: \bHom_{\bfA}(B,X) \rightarrow \bHom_{\bfA}(A,X)$ is a Kan fibration. We wish to show that $p$ is a trivial Kan fibration. Our assumption that $X$ is $S$-local guarantees that the map
$q': \bHom_{\bfA}(B',X) \rightarrow \bHom_{\bfA}(A', X)$ is a homotopy equivalence, and therefore a trivial fibration (since $i'$ is a cofibration). The map
$$q: \bHom_{\bfA}( A \coprod_{A'} B', X) \rightarrow \bHom_{\bfA}(A)$$ is a pullback of $q'$, and therefore also a trivial fibration. To show that $p$ is a trivial Kan fibration, it will suffice to show that
for every $t: A \rightarrow X$, the fiber $p^{-1} \{ t\}$ is a contractible Kan complex. Since the corresponding fiber $q^{-1} \{t\}$ is contractible, it will suffice to show that composition with $j$ induces a homotopy equivalence $$ p^{-1} \{t\} \rightarrow q^{-1} \{t\}. $$
This is clear, since $j$ is a weak equivalence between cofibrant objects of the simplicial model category $\bfA_{A/}$.

Now suppose that $(2)$ is satisfied. We wish to show that $i$ is an $S$-equivalence. For this, it suffices to show that for every fibrant $S$-local object $X \in \bfA$, the map
$$ q': \bHom_{\bfA}( B', X) \rightarrow \bHom_{\bfA}(A',X)$$ is a trivial Kan fibration.
The preceding argument shows that the fiber of $q'$ over a morphism $t': A' \rightarrow X$ is contractible, provided that $t'$ factors as a composition
$$ A' \stackrel{f}{\rightarrow} A \stackrel{t}{\rightarrow} X.$$
To complete the proof, it suffices to show that the same result holds for an arbitrary vertex
$t'$ of $\bHom_{\bfA}(A', X)$. The map $t'$ factors as a composition
$$ A' \stackrel{u}{\rightarrow} Y \stackrel{v}{\rightarrow} X,$$
where $u$ is a cofibration and $v$ is a trivial fibration. We have a commutative diagram
$$ \xymatrix{ \bHom_{\bfA}(B', Y) \ar[r] \ar[d] & \bHom_{\bfA}(A', Y) \ar[d] \\
\bHom_{\bfA}( B', X) \ar[r] & \bHom_{\bfA}(A', X) }$$
in which the vertical arrows are trivial Kan fibrations. It will therefore suffice to show that the fiber 
$\bHom_{\bfA}(B', Y) \times_{ \bHom_{\bfA}(A',Y) } \{ u \}$ is a contractible.
Choose a trivial cofibration $A' \coprod_{A} Y \rightarrow Z$, where $Z$ is fibrant. We observe that
The map $Y \rightarrow A' \coprod_{A} Y$ is the pushout of a weak equivalence by a cofibration, and therefore a weak equivalence (since $\bfA$ is left proper). It follows that the map
$Y \rightarrow Z$ is a weak equivalence between fibrant objects of $\bfA$. We have a commutative diagram 
$$ \xymatrix{ \bHom_{\bfA}(B', Y) \ar[r] \ar[d] & \bHom_{\bfA}(A', Y) \ar[d] \\
\bHom_{\bfA}( B', Z) \ar[r]^{q''} & \bHom_{\bfA}(A', Z) }$$
in which the vertical maps are homotopy equivalences, and the horizontal maps are Kan fibrations. It will therefore suffice to show that the fiber of $q''$ is contractible, when taken over the composite map $t'': A' \stackrel{u}{\rightarrow} Y \rightarrow Z$. We now observe that $t''$ factors through
$A$, so that the desired result follows from the first part of the proof.
\end{proof}

\begin{corollary}\label{swask}
Let $\bfA$ and $\bfB$ be simplicial model categories, and suppose given a simplicial adjunction
$$ \Adjoint{F}{\bfA}{\bfB.}{G}$$
Assume that $\bfB$ is left proper. The following conditions are equivalent:
\begin{itemize}
\item[$(1)$] The adjunction between $F$ and $G$ is a Quillen adjunction.
\item[$(2)$] The functor $F$ preserves cofibrations and the functor $G$ preserves fibrant objects.
\end{itemize}
\end{corollary}

\begin{proof}
The implication $(1) \Rightarrow (2)$ is obvious. Conversely, suppose that $(2)$ is satisfied.
We wish to prove that $F$ is a left Quillen functor. Since $F$ preserves cofibrations, it will suffice
to show that for every trivial cofibration $u: A \rightarrow A'$ in $\bfA$, the image
$Fu$ is a weak equivalence in $\bfB$. Applying Lemma \ref{mtomp} in the case
$S = \emptyset$, it will suffice to prove the following: for every fibrant object $B \in \bfB$, the
induced map
$$ \bHom_{ \bfB}( FA', B) \rightarrow \bHom_{\bfB}( FA, B)$$
is a trivial Kan fibration. Since $F$ and $G$ are adjoint simplicial functors, this is equivalent
to the requirement that the map $\bHom_{\bfA}( A', GB) \rightarrow \bHom_{\bfA}(A, GB)$ be
a trivial Kan fibration, which follows from our assumption that $u$ is a trivial cofibration in
$\bfA$ and that $GB \in \bfA$ is fibrant.
\end{proof}

\begin{proposition}\label{suritu}
Let $\bfA$ be a left proper combinatorial simplicial model category, and let $S$ be a $($small$)$ set of cofibrations in $\bfA$. 
Let $S^{-1} \bfA$ denote the same category, with the following distinguished classes of morphisms:
\begin{itemize}
\item[$(C)$] A morphism $g$ in $S^{-1} \bfA$ is a {\it cofibration} if it is a cofibration when regarded as a morphism in $\bfA$.
\item[$(W)$] A morphism $g$ in $S^{-1} \bfA$ is a {\it weak equivalence} if it is an $S$-equivalence.
\end{itemize}
Then:
\begin{itemize}
\item[$(1)$] The above definitions endow $S^{-1} \bfA$ with the structure of a combinatorial simplicial model category.
\item[$(2)$] The model category $S^{-1} \bfA$ is left proper.
\item[$(3)$] An object $X \in \bfA$ is fibrant in $S^{-1} \bfA$ if and only if $X$ is $S$-local and fibrant in $\bfA$.
\end{itemize}
\end{proposition}

\begin{proof}
Enlarging $S$ if necessary, we may assume:
\begin{itemize}
\item[$(a)$] For every morphism $f: A \rightarrow B$ in $S$ and every $n \geq 0$, the induced map
$$(A \times \Delta^n) \coprod_{ A \times \bd \Delta^n } (B \times \bd \Delta^n)
\rightarrow B \times \Delta^n$$ belongs to $S$.  
\item[$(b)$] The set $S$ contains a collection of generating trivial cofibrations for $\bfA$.
\end{itemize}
It follows that an object $X \in \bfA$ is fibrant and $S$-local if and only if it has the extension property with respect to every morphism in $S$. Since $S \subseteq C \cap W$, we deduce that every fibrant object of $S^{-1} \bfA$ is $S$-local and fibrant in $\bfA$. The converse follows from Lemma \ref{mtomp}; this proves $(3)$. 

To prove $(1)$, it will suffice to show that the classes $C$ and $W$ satisfy the hypotheses of Proposition \ref{bigmaker} (the compatibility of the simplicial structure on $S^{-1} \bfA$ with its model structure
follows immediately from Proposition \ref{testsimpmodel}). We observe that Lemma \ref{mtomp} implies that $C \cap W$ is a weakly saturated class of morphisms in $\bfA$. The only other nontrivial point is to show that $W$ is an accessible subcategory of $\bfA^{[1]}$.

Proposition \ref{quillobj} implies the existence of a functor $T: \bfA \rightarrow \bfA$, equipped with a natural transformation $\id_{\bfA} \rightarrow T$, with the following properties:
\begin{itemize}
\item[$(i)$] For every $X \in \bfA$, the object $TX \in \bfA$ is fibrant and $S$-local.
\item[$(ii)$] For every $X \in \bfA$, the map $X \rightarrow TX$ belongs to the smallest weakly saturated class of morphisms containing $S$; in particular, it belongs to $W \cap C$ and is therefore an $S$-equivalence.
\item[$(iii)$] There exists a regular cardinal $\kappa$ such that $T$ commutes with $\kappa$-filtered colimits.
\end{itemize}

It follows that a morphism $f: X \rightarrow Y$ in $\bfA$ is an $S$-equivalence if and only if the induced map $Tf: TX \rightarrow TY$ is an $S$-equivalence. Since $TX$ and $TY$ are $S$-local, Yoneda's lemma (in the category $\h{\bfA})$) implies that $Tf$ is an $S$-equivalence if and only if
$Tf$ is a weak equivalence in $\bfA$. It follows that $W$ is the inverse image under $T$ of the collection of weak equivalenes in $\bfA$. Corollaries \ref{sundert} and \ref{smitty} imply that $W$ is an accessible subcategory of $\bfA^{[1]}$, as desired. This completes the proof of $(1)$.

We now prove $(2)$. We need to show that the collection of $S$-equivalences in $\bfA$ is stable under pushouts by cofibrations. We observe that every morphism $f: X \rightarrow Z$ admits a factorization 
$$ X \stackrel{f'}{\rightarrow} Y \stackrel{f''}{\rightarrow} Z$$
where $f'$ is a cofibration and $f''$ is a weak equivalence in $\bfA$ (in fact, we can choose $f''$ to be a trivial fibration in $\bfA$). If $f$ is an $S$-equivalence, then $f'$ is an $S$-equivalence, so that $f' \in C \cap W$. It will therefore suffice to show that $C \cap W$ and the class of weak equivalences in $\bfA$ are stable under pushouts by cofibrations. The first follows from the assertion that $C \cap W$ is weakly saturated, and the second from the assumption that $\bfA$ is left proper.
\end{proof}

\begin{proposition}\label{surito}
Let $\bfA$ be a left proper combinatorial simplicial model category. Then:
\begin{itemize}
\item[$(1)$] Every combinatorial localization of $\bfA$ has the form $S^{-1} \bfA$, where
$S$ is some $($small$)$ set of cofibrations in $\bfA$.
\item[$(2)$] Given two $($small$)$ sets of cofibrations $S$ and $T$, the localizations
$S^{-1} \bfA$ and $T^{-1} \bfA$ coincide if and only if the class of $S$-local objects of
$\h{\bfA}$ coincides with the class of $T$-local objects of $\h{\bfA}$.
\end{itemize}
\end{proposition}

\begin{proof}
The ``if'' direction of $(2)$ is obvious, and the converse follows from the characterization of the fibrant objects of $S^{-1} \bfA$ given in Proposition \ref{suritu}. We now prove $(1)$. Let $\bfB$ be a combinatorial model category which is a localization of $\bfA$, and let $S$ be a set of generating trivial cofibrations for $\bfB$. We claim that $\bfB = S^{-1} \bfA$. The cofibrations of $S^{-1} \bfA$ and $\bfB$ coincide. Moreover, the collection of trivial cofibrations in $S^{-1} \bfA$ is a weakly saturated class of morphisms which contains $S$, and therefore contains every trivial cofibration in $\bfB$. To complete the proof, it will suffice to show that every trivial cofibration $f: X \rightarrow Y$ in $S^{-1} \bfA$ is a trivial cofibration in $\bfB$.

Choose a diagram
$$ \xymatrix{ X' \ar[r]^{f'} \ar[d] & Y' \ar[d] \\
X \ar[r]^{f} & Y }$$
where $X'$ is cofibrant, $f'$ is a cofibration, and the vertical maps are weak equivalences in $\bfA$. Then $f'$ is a trivial cofibration in $S^{-1} \bfA$, and it will suffice to show that $f'$ is a trivial cofibration in $\bfB$. For this, it will suffice to show that for every fibrant object $Z \in \bfB$, the map
$$ \bHom_{\bfB}(Y', Z) \rightarrow \bHom_{\bfB}(X',Z)$$ is a trivial fibration. In view of Lemma \ref{mtomp}, it will suffice to show that $Z$ is $S$-local and fibrant as an object of $\bfA$. The second claim is obvious, and the first follows from the fact that $S$ consists of trivial cofibrations in $\bfB$. 
\end{proof}

\begin{remark}\label{surito2}
In the situation of Proposition \ref{surito}, we may assume that for every cofibration
$f: A \rightarrow B$ in $S$, the objects $A$ and $B$ are themselves cofibrant.
To see this, choose for each cofibration $f: A \rightarrow B$ in $S$ a diagram
$$ \xymatrix{ A' \ar[r]^{g_{f}} \ar[d]^{u} & B' \ar[d]^{v} \ar[dr]^{w} & \\
A \ar[r]^{f'} & A \coprod_{A'} B' \ar[r]^{f''} & B }$$
as in the proof of Lemma \ref{mtomp}, so that $u$ and $w$ are trivial cofibrations,
$f = f'' \circ f'$, and $g_f$ is a cofibration between cofibrant objects. Then $g_{f}$ is a trivial cofibration in $S^{-1} \bfA$. We claim that the localizations $S^{-1} \bfA$ and
$T^{-1} \bfA$ coincide, where $T = \{ g_{f} \}_{f \in S}$. To prove this, it will suffice to show that
for each $f \in S$, every $g_{f}$-local fibrant object $X \in \bfA$ is also $f$-local.

Suppose that $X$ is $g_{f}$-local. We wish to prove that the map
$p: \bHom_{\bfA}(B, X) \rightarrow \bHom_{\bfA}(A,X)$ is a trivial Kan fibration.
Since $p$ is automatically a Kan fibration, it will suffice to show that the fiber
$p^{-1} \{t\}$ is contractible, for every morphism $t: A \rightarrow X$. 
Since $X$ is $g_{f}$-local, we deduce that the fiber $q^{-1} \{t\}$ is contractible, where
$q$ is the projection map
$\bHom_{\bfA}( A \coprod_{A'} B', X) \rightarrow \bHom_{\bfA}(A, X)$. It will therefore
suffice to show that $f''$ induces a homotopy equivalence of fibers
$$ \bHom_{\bfA_{A/}}(B,X) \rightarrow \bHom_{ \bfA_{A/}}( A \coprod_{A'} B', X).$$
This is clear, since $f''$ is a weak equivalence between cofibrant objects
of the simplicial model category $\bfA_{A/}$.
\end{remark}

\begin{proposition}\label{notthereyet}
Let $\calC$ be an $\infty$-category. The following conditions are equivalent:
\begin{itemize}
\item[$(1)$] The $\infty$-category $\calC$ is presentable.
\item[$(2)$] There exists a combinatorial simplicial model category $\bfA$ and an equivalence
$\calC \simeq \Nerve( \bfA^{\degree})$. 
\end{itemize}
\end{proposition}

\begin{proof}
According to Theorem \ref{pretop} and Proposition \ref{local}, $\calC$ is presentable if and only if there exists a small simplicial set $K$, a small set $S$ of morphisms in $\calP(K)$, and an equivalence $\calC \simeq S^{-1} \calP(K)$. Let $\calD$ be the simplicial category $\sCoNerve[K]^{op}$, and let $\bfB$ be the category $\Set_{\Delta}^{\calD}$ of 
simplicial functors $\calD \rightarrow \sSet$, endowed with the injective model structure. Proposition \ref{gumby444} implies that there is an equivalence $\calP(K) \simeq \Nerve( \bfB^{\degree})$. Moreover, Propositions \ref{suritu} and \ref{surito} implies that there is a bijective correspondence between accessible localizations of
$\calP(K)$ (as a presentable $\infty$-category) and combinatorial localizations of $\bfB$ (as a model category). This proves the implication $(1) \Rightarrow (2)$. Moreover, it also shows that $(2) 
\Rightarrow (1)$ in the special case where $\bfA$ is a localization of a category of simplicial presheaves. 

We now complete the proof by invoking the following result, proven in \cite{combmodel}: for every combinatorial model category $\bfA$, there exists a small category $\calD$, a set $S$ of morphisms of $\Set^{\calD^{op}}_{\Delta}$, and a Quillen equivalence of $\bfA$ with $S^{-1} \Set_{\Delta}^{\calD}$.
Moreover, the proof given in \cite{combmodel} shows that when $\bfA$ is a {\em simplicial} model category, then $F$ and $G$ can be chosen to be simplicial functors.
\end{proof}

\begin{remark}
Let $\bfA$ and $\bfB$ be combinatorial simplicial model categories. Then the underlying $\infty$-categories $\Nerve( \bfA^{\degree})$ and $\Nerve( \bfB^{\degree})$ are equivalent if and only if $\bfA$ and $\bfB$ can be joined by a chain of simplicial Quillen equivalences. The ``only if'' assertion follows from Corollary \ref{urchug}, and the ``if'' direction can be proven using the methods described in \cite{combmodel}.
\end{remark}

\begin{proposition}\label{cabber}
Let $\bfA$ be a left proper combinatorial simplicial model category, and let
$\calC = \Nerve( \bfA^{\degree} )$ denote its underlying $\infty$-category.
Suppose that $\calC^{0} \subseteq \calC$ is an accessible localization of $\calC$, and let
$L: \calC \rightarrow \calC^{0}$ denote a left adjoint to the inclusion.

Then there exists a localization $\bfA'$ of $\bfA$ satisfying the following conditions:
\begin{itemize}
\item[$(1)$] An object $X \in \bfA'$ is fibrant if and only if it is fibrant in $\bfA$, and
the associated object of the homotopy category $\h{ \bfA} \simeq \h{ \calC}$ belongs to the full subcategory $\h{ \calC^{0}}$. 
\item[$(2)$] A morphism $f: X \rightarrow Y$ in $\bfA'$ is a weak equivalence if and only if
the functor $L: \h{ \calC} \rightarrow \h{ \calC^{0} }$ carries $f$ to an isomorphism in the homotopy category $\h{ \calC^{0} }$.
\end{itemize}
\end{proposition}

\begin{proof}
According to Proposition \ref{notthereyet}, the $\infty$-category $\calC$ is presentable. 
The results of \S \ref{invloc} imply that we can write $\calC^{0} = S^{-1} \calC$, for some
small collection of morphisms $S$ in $\calC$. We then take $\widetilde{S}$ be a collection of representatives for the elements of $S$ as cofibrations between cofibrant objects of $\bfA$, and let
$\bfA'$ denote the localization $\widetilde{S}^{-1} \bfA$. 
\end{proof}

We conclude this section by establishing a universal property enjoyed by the localization of a combinatorial simplicial model category.

\begin{proposition}\label{stake}
Suppose given a simplicial Quillen adjunction
$$ \Adjoint{F}{\bfA}{\bfB}{G}$$
between left proper combinatorial simplicial model categories, and let $\bfA'$ be a Bousfield localization of $\bfA$. The following conditions are equivalent:
\begin{itemize}
\item[$(1)$] The adjoint functors $F$ and $G$ determine a Quillen adjunction between
$\bfA'$ and $\bfB$.
\item[$(2)$] Let $\alpha$ be a morphism in $\bfA$ which is a weak equivalence in
$\bfA'$. Then the left derived functor $LF: \h{\bfA} \rightarrow \h{\bfB}$ carries
$\alpha$ to an isomorphism in the homotopy category $\h{\bfB}$.
\item[$(3)$] For every fibrant object $X \in \bfB$, the image $GX$ is a fibrant object of $\bfA'$.
\end{itemize}
\end{proposition}

\begin{proof}
The implication $(1) \Rightarrow (2)$ is obvious, and the implication $(3) \Rightarrow (1)$ follows
from Corollary \ref{swask}. We will complete the proof by showing that $(2) \Rightarrow (3)$.
According to Proposition \ref{surito} and Remark \ref{surito2}, we may suppose that $\bfA' = S^{-1} \bfA$, where $S$ is a small collection of cofibrations between cofibrant objects of $\bfA$. Let $X$ be a fibrant object of $\bfB$; we wish to show that $GX$ is a fibrant object of $\bfA'$. Since $GX$ is fibrant in
$\bfA$, it will suffice to show that $GX$ is $S$-local (Proposition \ref{suritu}). In other words, we
must show that if $\alpha: A \rightarrow B$ belongs to $S$, then the induced map
$p: \bHom_{\bfA}( B, GX) \rightarrow \bHom_{\bfA}( A, GX)$ is a weak homotopy equivalence.
Since $F$ and $G$ are simplicial functors, we can identify $p$ with the map
$\bHom_{\bfB}( FB, X) \rightarrow \bHom_{\bfB}( FA, X)$. To prove that $p$ is a weak homotopy equivalence, it will suffice to show that $F(\alpha)$ is a weak equivalence between cofibrant objects
of $\bfB$. This follows immediately from assumption $(2)$ (because $\alpha$ is a cofibration between cofibrant objects of $\bfA$, we can identify $F(\alpha)$ with the left derived functor $LF(\alpha)$.
\end{proof}

\begin{corollary}\label{swinker}
Let $\bfA$ and $\bfB$ be left proper combinatorial simplicial model categories, and suppose given
a simplicial Quillen adjunction
$$ \Adjoint{F}{\bfA}{\bfB.}{G}$$
Then:
\begin{itemize}
\item[$(1)$] There exists a new left proper combinatorial simplicial model structure $\bfA'$ on the category $\bfA$ with the following properties:
\begin{itemize}
\item[$(C)$] A morphism $\alpha$ in $\bfA'$ is a cofibration if and only if it is a cofibration in $\bfA$.
\item[$(W)$] A morphism $\alpha$ in $\bfA'$ is a weak equivalence if and only if the left
derived functor $LF$ carries $\alpha$ to an isomorphism in the homotopy category $\h{\bfB}$.
\item[$(F)$] A morphism $\alpha$ in $\bfA'$ is a fibration if and only if it has the right lifting
property with respect to every morphism in $\bfA'$ satisfying $(C)$ and $(W)$.
\end{itemize}
\item[$(2)$] The functors $F$ and $G$ determine a new simplicial Quillen adjunction
$$ \Adjoint{F'}{\bfA'}{\bfB.}{G'}$$
\item[$(3)$] Suppose that the right derived functor $RG$ is fully faithful. 
Then the adjoint pair $(F',G')$ is a Quillen equivalence.
\end{itemize}
\end{corollary}

\begin{proof}
The functors $F$ and $G$ determine a pair of adjoint functors 
$$ \Adjoint{f}{\Nerve \bfA^{\degree}}{\Nerve \bfB^{\degree}}{g}$$
between the underlying $\infty$-categories (see Proposition \ref{quiladj}), which
are themselves presentable (Proposition \ref{notthereyet}). Let $\overline{S}$
be the collection of all morphisms $u$ in $\Nerve \bfA^{\degree}$ such that
$f(u)$ is an equivalence in $\Nerve \bfB^{\degree}$. Proposition \ref{postbluse}
implies that $\overline{S}$ is generated (as a strongly saturated class of morphisms)
by a small subset $S \subseteq \overline{S}$. Without loss of generality, we may suppose
that the morphisms of $S$ are represented by some (small) collection $T$ of cofibrations between cofibrant objects of $\bfA$. Let $\bfA' = T^{-1} \bfA$. We claim that $\bfA'$ satisfies
the description given in $(1)$. In other words, we claim that a morphism
$\alpha$ in $\bfA$ is a $T$-equivalence if and only if the left derived functor
$LF$ carries $\alpha$ to an isomorphism in $\h{\bfB}$. Without loss of generality, we may suppose
that $\alpha$ is a morphism between fibrant-cofibrant objects of $\bfA$, so that we can
view $\alpha$ as a morphism in the $\infty$-category $\Nerve \bfA^{\degree}$. In this case, both conditions on $\alpha$ are equivalent to the requirement that $\alpha$ belongs to $\overline{S}$.
This completes the proof of $(1)$. Assertion $(2)$ follows immediately from Proposition \ref{stake}.

We now prove $(3)$. Note that the homotopy category $\h{\bfA'}$ can be identified with a full subcategory of the homotopy category $\h{\bfA}$, and that under this identification the left derived functor $LF$ restricts to the left derived functor $LF'$. It follows that for every fibrant object
$X \in \bfB$, the counit map
$$ (LF')(RG')X \simeq (LF')(GX)
\simeq (LF)(GX) \simeq (LF)(RG)X \simeq X$$
is an isomorphism in $\h{\bfB}$ (where the last equivalence follows from our assumption that $RG$
is fully faithful). It follows that the functor $RG'$ is fully faithful. To complete the proof, it will suffice to show that the left derived functor $LF'$ is conservative. In other words, we must show that
if $\alpha: X \rightarrow Y$ is a morphism in $\bfA'$, then $\alpha$ is a weak equivalence if and only if
$LF(\alpha)$ is an isomorphism in $\bfB$; this follows immediately from $(1)$.
\end{proof}

%\begin{corollary}
%Let $\bfA$ be a combinatorial model category.
%Let $A = A_0 \cup \{ \infty \}$ be a partially ordered set with a largest
%element $\infty$. Suppose given diagrams $Z: A^{op} \rightarrow \bfA$,
%$Y: A \rightarrow \bfA$, an object $X \in \bfA$, and a collection of maps
%$\alpha_a: Z(a) \times Y(a) \rightarrow X$ with the following properties:

%\begin{itemize}
%\item[$(i)$] For each $a \in A$, the map $\alpha_a$ exhibits $Z(a)$ as a weak exponential
%of $X$ by $Y(a)$ in $\bfA$. In particular, $Z(a) \times Y(a)$ is a homotopy product
%of $Z(a)$ and $Y(a)$ in $\bfA$.
%\item[$(ii)$] The maps $\{ \alpha_a \}_{a \in A}$ are compatible in the following sense:
%for every pair of elements $a \leq b$, the diagram
%$$ \xymatrix{ Y(a) \times Z(b) \ar[r] \ar[d] & Y(b) \times Z(b) \ar[d]^{\alpha_{b} } \\
%Y(a) \times Y(a) \ar[r]^{\alpha_{a} } & X }$$
%is commutative. (In other words, the maps $\{ \alpha_a \}_{a \in A}$ determine a map
%from the {\em coend} $\int_{A} Y \times Z$ into $X$.)
%\item[$(iii)$] Multiplication in $\bfA$ preserves homotopy colimits.
%\item[$(iv)$] The diagram $Y$ exhibits $Y(\infty)$ as a homotopy colimit of
%$Y_0 = Y|A_0$.
%\end{itemize}
%Then the diagram $Z$ exhibits $Z(\infty)$ as the homotopy limit of
%the diagram $Z_0 = Z | A_0^{op}$. 
%\end{corollary}


%\begin{proposition}\label{surr}
%Let $\bfA$ be a combinatorial model category, $\calC$ a small category, and
%$\alpha: Z \times Y \rightarrow X$ a natural transformation of functors in
%$\Fun(\calC, \bfA)$. Assume that:
%\begin{itemize}

%\item[$(i)$] The objects $X,Y,Z \in \Fun(\calC, \bfA)$ are projectively fibrant, and
%for each $C \in \calC$ the induced map $\alpha_{C}: Z(C) \times Y(C) \rightarrow X(C)$
%exhibits $Z(C)$ as a weak exponential of $X(C)$ by $Y(C)$ in $\bfA$.

%\item[$(ii)$] The functor $Y$ takes some constant value $Y_0 \in \bfA$, and multiplication by
%$Y_0$ preserves homotopy colimits in $\bfA$.
%\end{itemize}

%Then $\alpha$ exhibits $Z$ as a weak exponential of $X$ by $Y$ in $\Fun(\calC, \bfA)$.
%\end{proposition}

%Before we can give the proof of Proposition \ref{surr}, we need to introduce a bit of notation.
%Let $\calC$ be a small category. We let
%$\Ar(\calC)$ denote the {\it arrow category} of $\calC$\index{gen}{arrow category}, which
%is defined as follows:
%\begin{itemize}\index{not}{ArC@$\Ar(\calC)$}
%\item[$(1)$] The objects of $\Ar(\calC)$ are morphisms $f: C \rightarrow D$ in $\calC$.
%\item[$(2)$] Let $f: C \rightarrow D$ and $f': C' \rightarrow D'$ be objects of $\Ar(\calC)$. Then
%$\Hom_{\Ar(\calC)}(f, f')$ is the set of commutative diagrams
%$$ \xymatrix{ C \ar[r]^{f} & D \ar[d] \\
%C' \ar[u] \ar[r]^{f'} & D' }$$
%in $\calC$.
%\item[$(3)$] Composition of morphisms in $\Ar(\calC)$ is given by concatenation of diagrams.
%\end{itemize}

%\begin{lemma}\label{sumtuous}
%Let $\bfA$ be a combinatorial model category, $\calC$ a small category, and $C \in \calC$ an object, and $q: \calC_{C/} \rightarrow \calC$ the forgetful functor. Then the pullback functor
%$q^{\ast}: \Fun( \calC, \bfA) \rightarrow \Fun( \calC_{C/}, \bfA)$ preserves injective fibrations.
%\end{lemma}

%\begin{proof}
%It will suffice to show that the left adjoint $q_{!}$ preserves injective cofibrations. 
%Let $\alpha: F \rightarrow F'$ be a map in $\Fun( \calC_{C/}, \bfA)$.
%We observe that for each object $D \in \calC$, the map 
%$(q_{!} \alpha)(D): (q_{!} F)(D) \rightarrow (q_{!} F')(D)$ can be identified
%with the coproduct of the maps $\{ \alpha(f): F(f) \rightarrow F'(f) \}_{ f \in \Hom_{\calC}(C,D) }$.
%If $\alpha$ is a injective cofibration, then each of these maps is a cofibration in $\bfA$, so that
%$q_{!} \alpha$ is again a injective cofibration as desired.
%\end{proof}

%\begin{lemma}\label{statterr}
%Let $\bfA$ be a combinatorial simplicial model category and $\calC$ a small category.
%Let $G \in \Fun(\calC, \bfA)$ be a injectively fibrant object, and let
%$\alpha: F \rightarrow F'$ be a projective cofibration in $\Fun(\calC, \bfA)$. 
%Let $\overline{F}: \Ar(\calC) \rightarrow \sSet$ be defined by the formula
%$H( f: C \rightarrow D) = \bHom_{\bfA}( F(C), G(D) )$, and let
%$H': \Ar(\calC) \rightarrow \sSet$ be defined similarly. Then
%$\alpha$ induces a injective fibration $\beta: H' \rightarrow H$
%in the diagram category $\Fun( \Ar(\calC), \sSet)$.
%\end{lemma}

%\begin{proof}
%The collection of all morphisms $\alpha$ which satisfy the conclusion of the
%theorem is strongly saturated. It will therefore suffice to prove Lemma \ref{statter} in
%the case where $\alpha$ is a generating projective cofibration of the form
%$\calF^{C}_{A} \rightarrow \calF^{C}_{A'}$, where $C$ is an object of $\calC$ and
%$i: A \rightarrow A'$ is a cofibration in $\bfA$. 

%Let $\overline{G}: \calC \rightarrow \sSet$ be defined by the formula
%$\overline{G}(D) = \bHom_{\bfA}(A, G(D) )$, and let $\overline{G}': \calC \rightarrow \sSet$ be defined similarly. Since $G$ is injectively fibrant and $i$ is a cofibration, the induced map
%$\overline{\alpha}: \overline{G}' \rightarrow \overline{G}$ is a injective fibration in $\Fun( \calC, \sSet)$. 
%Consider the diagram of categories
%$$ \xymatrix{ & \calC_{C/} \ar[dl]^{p} \ar[dr]^{q} & \\
%\Ar(\calC) & & \calC, }$$
%where $q$ is the forgetful functor and $p$ carries an object $D \in \calC_{C/}$ to the diagram
%$(C \rightarrow D) \in \Ar(\calC)$. We observe that
%$\beta$ is the image of $\overline{\alpha}$ under the composite functor
%$p_{\ast} q^{\ast}$. To prove that $\beta$ is a injective fibration, it will suffice to show
%that $p_{\ast}$ and $q^{\ast}$ preserve injective fibrations. For $p_{\ast}$ this is obvious, while
%for $q^{\ast}$ this follows from Lemma \ref{sumtuous}.
%\end{proof}

%\begin{lemma}\label{statter}
%Let $\bfA$ be a combinatorial simplicial model category and $\calC$ a small category.
%Let $F, G \in \Fun(\calC, \bfA)$ be such that $F$ is projectively cofibrant and $G$ is projectively fibrant, and let
%$H: \Ar(\calC) \rightarrow \sSet$ be defined by the formula $H( f: C \rightarrow D) = \bHom_{\bfA}( FC, GD)$. Then the isomorphism of simplicial sets $\bHom_{ \Fun(\calC, \bfA) }(F,G) = \lim H$
%exhibits $\bHom_{ \Fun(\calC, \bfA)}(F,G)$ as a homotopy limit of the diagram $H$.
%\end{lemma}

%\begin{proof}
%Without loss of generality, we may assume that $G$ is injectively fibrant. In this case,
%the diagram $H$ is injectively fibrant by Lemma \ref{statterr}.
%\end{proof}

%\begin{proof}[Proof of Proposition \ref{surr}]
%According to the main result of \cite{combmodel}, there exists a Quillen equivalence
%$\Adjoint{F}{\bfA'}{\bfA}{G}$ where $\bfA'$ is a combinatorial {\em simplicial} model category.
%In view of Remarks \ref{toofus} and \ref{twofus}, we may replace $\bfA$ by $\bfA'$
%and thereby reduce to the case where $\bfA$ is a simplicial model category. 

%Let $W$ and $W'$ be projectively cofibrant, projectively fibrant objects of
%$\Fun(\calC, \bfA)$, and suppose we are given a weak equivalence
%$W' \rightarrow W \times Y$. We wish to show that the composite map
%$$ \Hom_{ \h{\Fun(\calC, \bfA)}}( W, Z) \rightarrow \Hom_{\h{\Fun(\calC,\bfA)}( W \times Y, Z \times Y)
%\rightarrow \Hom_{ \h{\Fun(\calC, \bfA)}}( W', X)$$
%is bijective. We will prove a stronger statement: namely, the map
%$$ \phi: \bHom_{ \Fun(\calC, \bfA) }( W, Z) \rightarrow \Hom_{ \Fun( \calC, \bfA) }( W', X)$$
%is a homotopy equivalence of Kan complexes.

%Let $F,F': \Ar(\calC) \rightarrow \sSet$ be defined by the formulas
%$$F(f: C \rightarrow D) = \bHom_{\bfA}( W(C), Z(D) )$$
%$$F'( f: C \rightarrow D) = \bHom_{\bfA}( W'(C), X(D) ).$$ 
%The map $\alpha$ induces a natural transformation of functors
%$\overline{\phi}: F \rightarrow F'$. Using Proposition \ref{scat} and assumption $(ii)$, we deduce that
%$\overline{\phi}$ is a weak homotopy equivalence in $\Fun( \Ar(\calC), \sSet)$. 
%We observe that $\phi$ is the image of $\overline{\phi}$ under the limit functor
%$\lim: \Fun( \Ar(\calC), \sSet) \rightarrow \sSet$. Applying Lemma \ref{statter}, we deduce that
%$\phi$ is also an equivalence.
%\end{proof}


%\subsection{Homotopy Colimits and the Grothendieck Construction}




%Throughout this section, we will assume that $\bfS$ is an excellent model category
%in which the monoidal structure is given by the Cartesian product.
%Our goal is to describe homotopy colimits in $\SCat$. We begin by reviewing
%the classical {\em Grothendieck construction}, which convert a $\calJ$-indexed diagram of categories into a category (co)fibered over $\calJ$.

%\begin{definition}
%Let $\calJ$ be a small category, and suppose given a functor
%$p: \calJ \rightarrow \Cat$, which carries each object $J \in \calJ$ to
%a category $\calC_{J}$. We define a new category
%$\Groth(p)$ as follows:
%\begin{itemize}
%\item[$(1)$] An object of $\Groth(p)$ is a pair $(J,C)$, where
%$J \in \calJ$ and $C \in \calC_J$.
%\item[$(2)$] Given a pair of objects $(J,C), (J',C') \in \Groth(p)$, a morphism
%from $(J,C)$ to $(J',C')$ is a pair $(\alpha,\beta)$, where
%$\alpha: J \rightarrow J'$ is a morphism in $\calJ$, and
%$\beta: \alpha_{!} C \rightarrow C'$ is a morphism in $\calC_{J'}$, where
%$\alpha_{!}$ denotes the functor $\calC_{J} \rightarrow \calC_{J'}$ induced by
%$\alpha$.
%\item[$(3)$] Composition in $\Groth(p)$ is defined in the obvious way.
%\end{itemize}
%\end{definition}

%\subsection{Another Model Structure on Enriched Categories}

%Throughout this section, we will assume that $\bfS$ is an excellent model category.
%We have seen that the category
%$\SCat$ of $\bfS$-enriched categories admits a model structure, and that
%every $\bfS$-enriched model category $\bfA$ gives rise to a fibrant object
%$\bfA^{\degree} \in \SCat$. However, the construction
%$\bfA \mapsto \bfA^{\degree}$ is inconvenient for many purposes, because
%it has very limited functoriality in $\bfA$. For example, if we have a
%$\bfS$-enriched Quillen adjunction
%$\Adjoint{F}{\bfA}{\bfB}{G}$
%then generally $F$ will not carry $\bfA^{\degree}$ into $\bfB^{\degree}$, and
%$G$ will not carry $\bfB^{\degree}$ into $\bfA^{\degree}$.

%In this section, we will introduce a new model category $\wSCat$ which is Quillen equivalent to $\SCat$, but better suited to certain constructions. 

%\begin{definition}\label{swinner}\index{not}{wSCat@$\wSCat$}
%We define a category $\wSCat$ as follows:
%\begin{itemize}
%\item[$(i)$] An object of $\wSCat$ is a pair $(\calC, W)$, where
%$\calC$ is a (small) $\bfS$-enriched category and $W$ is a set
%equipped with a map $W \rightarrow \Hom_{\SCat}( [1]_{\bfS}, \calC)$;
%here $\Hom_{\SCat}( [1]_{\bfS}, \calC)$ denotes the set of morphisms in $\calC$.
%\item[$(ii)$] A map $(\calC, W) \rightarrow (\calC', W')$ consists of a $\bfS$-enriched
%functor $f: \calC \rightarrow \calC'$ and a map of sets $\alpha: W \rightarrow W'$, such that the diagram of sets
%$$ \xymatrix{ W \ar[d]^{\alpha} \ar[r] & \Hom_{\SCat}( [1]_{\bfS}, \calC) \ar[d]^{f \circ} \\
%W' \ar[r] & \Hom_{\SCat}( [1]_{\bfS}, \calC' ) }$$
%is commutative.
%\end{itemize}
%\end{definition}

%\begin{remark}
%We can informally summarize Definition \ref{swinner} as follows:
%an object of $\wSCat$ is a $\bfS$-enriched category $\calC$ together with
%a set $W$ of morphisms in $\calC$. This description is slightly inaccurate, since we do not require the map $W \rightarrow \Hom_{\SCat}( [1]_{\bfS}, \calC)$ to be injective.
%\end{remark}

%Roughly speaking, the idea is that an object $(\calC, W) \in \wSCat$ corresponds to
%a $\bfS$-enriched category $\calC$ together with a distinguished class of morphisms
%that we would like to invert. We now introduce a construction to invert the elements of $W$.

%\begin{remark}\index{not}{Inv@$\Inv$}
%The construction $( \calC, W) \mapsto \calC[W^{-1}]$ determines a functor
%from $\wSCat$ to $\SCat$, which we will denote by $\Inv$.
%The functor $\Inv: \wSCat \rightarrow \SCat$ admits a right adjoint $\CoInv$, given by the formula
%$\CoInv(\calC) = ( \calC, \Hom_{ \SCat}( \calE, \calC).$\index{not}{Coinv@$\Coinv$}
%\end{remark}

%The main result of this section is the following:

%\begin{proposition}\label{spunner}
%Let $\bfS$ be an excellent model category. Then:

%\begin{itemize}
%\item[$(a)$]
%There exists a left proper combinatorial model structure on $\wSCat$, which may be described as follows:
%\begin{itemize}
%\item[$(C)$] A morphism $(\calC, W) \rightarrow (\calC', W')$ in $\wSCat$ is
%a cofibration if and only if the underlying map $\calC \rightarrow \calC'$ is a cofibration
%in $\SCat$, and the map of sets $W \rightarrow W'$ is injective. 
%\item[$(W)$] A morphism $(\calC, W) \rightarrow (\calC', W')$ in $\wSCat$
%is a weak equivalence if and only if the induced map
%$\calC[ W^{-1}] \rightarrow \calC'[ {W'}^{-1}]$ is a weak equivalence in $\SCat$.
%\end{itemize}

%\item[$(b)$] The functors $\Inv$ and $\CoInv$ determine a Quillen equivalence
%$$ \Adjoint{ \Inv}{\wSCat}{\SCat.}{\CoInv.}$$

%\item[$(c)$] There exists another Quillen equivalence
%$\Adjoint{F}{\SCat}{\wSCat}{G}$, where
%$G$ is the forgetful functor $(\calC, W) \mapsto \calC$, and $F$ 
%its left adjoint $\calC \mapsto (\calC, \emptyset)$.

%\end{itemize}
%\end{proposition}

%\begin{proof}
%To prove $(a)$, we will show that $\wSCat$ satisfies the hypotheses of Proposition \ref{goot}. 
%Since the functor $\Inv: \wSCat \rightarrow \SCat$ commutes with filtered colimits,
%hypothesis $(1)$ follows from the fact that weak equivalences in $\SCat$ are stable under filtered colimits. Since the functor $\Inv$ clearly carries cofibrations in $\wSCat$ to cofibrations
%in $\SCat$, hypothesis $(2)$ follows from the left-properness of $\SCat$. To verify $(3)$,
%we consider a map $\overline{f}: ( \calC, W) \rightarrow (\calC', W')$ in $\wSCat$ which has the right lifting property with respect to all cofibrations. Then $\overline{f}$ is determined by
%a trivial fibration $f: \calC \rightarrow \calC'$ in $\SCat$, together with a surjective map
%$f_0: W \rightarrow W'$. We wish to show that the induced map
%$\calC[ W^{-1} ] \rightarrow \calC'[ {W'}^{-1} ]$ is a weak equivalence in $\SCat$.
%Since $f_0$ is surjective, there exists a decomposition
%$W = W_0 \coprod W_1$, such that $f_0 | W_0$ is a bijection.
%Then we have a commutative diagram
%$$ \xymatrix{ & \calC[W_0^{-1}][W_1^{-1}] \ar[r]^{\sim} & \calC[W^{-1}] \ar[dr] & \\
%\calC[W_0^{-1}] \ar[ur]^{g} \ar[rrr]^{g'} & &  & \calC' [ { W' }^{-1} ]. }$$
%By the two-out-of-three property, it will suffice to show that $g$ and $g'$
%are weak equivalences. For $g$, this follows from the invertibility hypothesis
%(Remark \ref[uppa]). For $g'$, we invoke Remark \ref{summat}. This completes the proof of $(a)$.

%We now prove $(b)$. By construction, $\Inv$ preserves cofibrations and weak equivalences, so that $(\Inv, \CoInv)$ is a Quillen adjunction. To show that this Quillen adjunction is a Quillen equivalence, we must show that if we are given a cofibrant object $( \calC, W) \in \wSCat$
%and a fibrant object $\calD \in \SCat$, then a map
%$f: \calC[W^{-1}] \rightarrow \calD$ is a weak equivalence of $\bfS$-enriched categories
%if and only if the induced map
%$\overline{f}: ( \calC, W) \rightarrow ( \calD, W')$ is a weak equivalence in $\wSCat$, where
%$W' = \Hom_{\bfS}(\calE, \calD)$. In this case we have a commutative diagram
%$$ \xymatrix{ & \calD \ar[dr]^{g} & \\
%\calC[W^{-1}] \ar[ur]^{f} \ar[rr] & & \calD[ {W'}^{-1} ]. }$$
%Using the two-out-of-three property, we are reduced to proving that $g$ is a weak equivalence
%Since $\bfS$ satisfies the invertibility hypothesis, it suffices to show that each element
%of $W'$ maps to an equivalence in $\calD$, which is obvious.
 
%Let us now prove $(c)$. The functor $F$ clearly preserves cofibrations and trivial cofibrations, so that $(F,G)$ is a Quillen adjunction. To show that $F$ induces a Quillen equivalence, it
%suffices to show that the left Quillen functor $\Inv \circ F: \SCat \rightarrow \SCat$ is a Quillen equivalence. This is clear, since $\Inv \circ F \simeq \id_{\SCat}$.
%\end{proof}

%\begin{remark}\label{sapper}
%Suppose that the monoidal structure on $\bfS$ is given by the Cartesian product.
%Then multiplication in $\wSCat$ preserves homotopy colimits. This follows from 
%Proposition \ref{spunner}, Remark \ref{canus}, and Example \ref{canuss}.
%\end{remark}

%\begin{lemma}\label{threeface}
%Let $\calC$ be a $\bfS$-enriched category, let $W'$ be the collection of all
%objects of $\calC$, and let $\psi: W' \rightarrow \Hom_{\SCat}( [1]_{\bfS}, \calC \otimes [1]_{\bfS})$
%carry each object $C \in \calC$ to $\id_{C} \otimes f$, where $f: 0 \rightarrow 1$ is the tautological morphism in $[1]_{\bfS}$. Then the canonical map
%$p: (\calC \otimes [1]_{\bfS}) [{W'}^{-1}] \rightarrow \calC$ is a weak equivalence of $\bfS$-enriched categories.
%\end{lemma}

%\begin{proof}
%We observe that $p$ factors through the map
%$p': (\calC \otimes [1]_{\bfS})[W^{-1}] \rightarrow \calC \otimes [1]_{\bfS}[f^{-1}]
%\simeq \calC \otimes \calE$
%considered in Lemma \ref{twoface}. According to Lemma \ref{twoface}, 
%the map $p'$ is a weak equivalence. It will therefore suffice to show that
%the map $\id_{\calC} \otimes p'': \calC \otimes \calE \rightarrow \calC \otimes [0]_{\bfS} \simeq \calC$
%is a weak equivalence, which follows immediately from the fact that $p'': \calE \rightarrow [0]_{\bfS}$
%is a weak equivalence.
%\end{proof}

%\begin{lemma}\label{fourface}
%Let $(\calC, W)$ be an object of $\wSCat$ with corresponding map
%$\psi: W \rightarrow \Hom_{\SCat}( [1]_{\bfS}, \calC)$, $W'$ the collection of all objects in
%$\calC$, $U = ( W \times \{0,1\}) \coprod W'$, and let
%$\psi': U \rightarrow \Hom_{\SCat}( [1]_{\bfS}, \calC \otimes [1]_{\bfS})$
%be defined 
%$$ \psi(w,j) = \psi(w) \otimes \id_{j}$$
%$$\psi(w') = \id_{w'} \otimes f$$
%where $f: 0 \rightarrow 1$ is the tautological morphism in $[1]_{\bfS}$. Then
%the induced map
%$p: ( \calC \otimes [1]_{\bfS}, U) \rightarrow (\calC, W)$ is a weak equivalence
%in $\wSCat$.
%\end{lemma}

%\begin{proof}
%The map $p$ factors as a composition
%$$ ( \calC \otimes [1]_{\bfS}, U) \stackrel{p'}{\rightarrow} ( \calC, W \times \{0,1\}) \stackrel{p''}{\rightarrow} (\calC, W).$$
%Using the left-properness of $\wSCat$ and Lemma \ref{threeface}, we conclude that $p'$ is a weak equivalence. We now complete the proof by observing that $p''$ is a trivial fibration in $\wSCat$.
%\end{proof}

%\begin{proposition}\label{ws}
%Let $\bfA$ be a combinatorial $\bfS$-enriched model category, and let
%$W$ denote the collection of weak equivalences in $\bfA$. Then
%the inclusion
%$$ ( \bfA^{\degree}, \emptyset) \rightarrow (\bfA, W)$$
%is a weak equivalence in $\wSCat$.
%\end{proposition}

%\begin{proof}
%We observe that $W_0 = W \cap \bfA^{\degree}$ coincides with the collection
%of equivalences in $\bfA^{\degree}$. Since $\bfS$ satisfies the invertibility hypothesis, the
%canonical map $\bfA^{\degree} \rightarrow \bfA^{\degree}[W_0^{-1}]$ is a weak equivalence
%of $\bfS$-enriched categories. It will therefore suffice to show that the inclusion
%$( \bfA^{\degree}, W_0) \rightarrow (\bfA, W)$ is a weak equivalence in $\wSCat$.

%Using the small object argument, we can construct a $\bfS$-enriched
%{\em cofibrant replacement} functor $P: \bfA \rightarrow \bfA$.
%In other words, $P$ is equipped with a $\bfS$-enriched natural
%transformation $\alpha: P \rightarrow \id$ such that for every
%object $A \in \bfA$, the induced map $\alpha_A: P(A) \rightarrow A$ is a trivial
%fibration, and $P(A)$ is cofibrant. Similarly, we can construct
%$\bfS$-enriched {\em fibrant replacement} functor $Q: \bfA \rightarrow \bfA$
%equipped with a natural transformation $\beta: \id \rightarrow Q$ which
%induces a trivial cofibration $A \rightarrow Q(A)$ for each $A \in \bfA$, where
%$Q(A)$ is fibrant.

%The composition $P \circ Q$ carries $(\bfA,W)$ into $(\bfA^{\degree}, W_0)$. To
%complete the proof, it will suffice to show that $P \circ Q$ is a homotopy inverse to
%the inclusion. We will show that $P \circ Q$ is homotopic to the identity on
%$(\bfA, W)$; the same argument will show that the restriction of $P \circ Q$ to
%$( \bfA^{\degree}, W_0)$ is homotopic to the identity.

%By construction, we have natural transformations
%$P \circ Q \stackrel{\alpha}{\rightarrow} Q \stackrel{\beta}{\leftarrow} \id.$
%We will show that $\beta$ determines a homotopy from $Q$ to the identity functor;
%the same argument will show that $\alpha$ determines a homotopy from
%$P \circ Q$ to $Q$, and the proof will be complete.

%The map $\beta$ can be viewed as $\bfS$-enriched functor
%$h: \bfA \otimes [1]_{\bfS} \rightarrow \bfA$. Let $U$ denote
%the collection of objects of $\bfA \otimes [1]_{\bfS}$ defined
%in Lemma \ref{fourface}.
%The map $h$ extends to a map
%$\overline{h}: ( \bfA \otimes [1]_{\bfS}, U) \rightarrow ( \bfA, W)$
%in $\wSCat$, such that $\overline{h} | ( \bfA \otimes \{0\} , W \times \{0\})$ is the identity
%and $\overline{h} | ( \bfA \otimes \{1\}, W \times \{1\})$ coincides with $Q$.
%To show that $\overline{h}$ gives a homotopy between $Q$ and the identity,
%it will suffice to show that each of the inclusions
%$$ ( [0]_{\bfS} \otimes \bfA, W) \rightarrow ( [1]_{\bfS} \otimes \bfA, U)$$
%is a weak equivalence in $\wSCat$. This is an immediate consequence of
%Lemma \ref{fourface}.
%\end{proof}

%\begin{corollary}\label{tanwise}
%Let $\bfS$ be an excellent model category in which the monoidal structure is
%given by the Cartesian product, let $\bfA$ be a combinatorial $\bfS$-enriched
%model category, and let $\calC$ be a small $\bfS$-enriched category. 
%Let $W$ denote the class of weak equivalences in $\bfA$, and
%$W_{\calC}$ the collection of weak equivalences in $\bfA^{\calC}$.
%Then the canonical map
%$$ (\bfA^{\calC}, W_{\calC} ) \times (\calC, \emptyset) \rightarrow (\bfA, W)$$
%exhibits $( \bfA^{\calC}, W_{\calC} )$ as a weak exponential of
%$(\bfA, W)$ by $(\calC, \emptyset)$ in $\widehat{ \wSCat}$.
%\end{corollary}

%\begin{proof}
%Regard $\bfA^{\calC}$ as endowed with the projective model structure. In
%view of Proposition \ref{ws}, it will suffice to show that the induced map
%$$ ( ( \bfA^{\calC})^{\degree}, \emptyset) \times (\calC, \emptyset) \rightarrow
%( \bfA^{\degree}, \emptyset)$$
%exhibits $((\bfA^{\calC})^{\degree}, \emptyset)$ as a homotopy exponential
%of $(\bfA^{\degree}, \emptyset)$ by $(\calC, \emptyset)$, which follows immediately from
%\end{proof}

%Let $\bfA$ be a combinatorial $\bfS$-enriched model category, and let
%$\calC$ be a small $\bfS$-enriched category. We let $W_{\calC}$ denote
%the collection of morphisms in $\bfA^{\calC}$ which are pointwise weak equivalences.
%The construction
%$\calC \mapsto ( \bfA^{\calC}, W_{\calC})$ is contravariantly functorial in $\calC$, and
%therefore defines a functor from the category of small $\bfS$-enriched categories
%to the category $\widehat{ \wSCat}$ 


%\begin{theorem}
%Let $\bfS$ be an excellent model category in which the monoidal structure is given by the Cartesian product, and let $\bfA$ be a combinatorial $\bfS$-enriched model category. 

%Let $A$ be a partially ordered set, and suppose we are given a diagram
%$\{ \calC_{a} \}_{a \in A}$ in the category $\SCat$. Let
%$\calC$ be a $\bfS$-enriched category equipped with a map
%$\colim_{a \in A} \calC_{a} \rightarrow \calC$ which exhibits
%$\calC$ as a homotopy colimit of the diagram $\{ \calC_{a} \}_{a \in A}$. 
%Then the same map exhibits
%$( \bfA^{\calC}, W_{\calC} )$ as the homotopy limit of the diagram
%$\{ ( \bfA^{\calC_a}, W_{\calC_a} ) \}_{a \in A}$ in $\widehat{ \wSCat }$.
%\end{theorem}

%\begin{proof}
%Combine Remark \ref{sapper}, Corollary \ref{tanwise}, and Proposition \ref{psygood}.
%\end{proof}

%The fibrant objects of $\wSCat$ are described by the following analogue of
%\begin{
%\item[$(d)$] Let $
%An object $(\calC,W) \in \wSCat$ is fibrant if and only if
%$\calC$ is fibrant as a $\bfS$-enriched category, and the image of
%the map $W \rightarrow \Hom_{\SCat}( [1]_{\bfS}, \calC)$ consists
%of precisely those morphisms in $\calC$ which induce isomorphisms
%in the homotopy category $\h{\calC}$.

%We now prove $(d)$. First suppose that $(\calC,W)$ is a fibrant object of $\wSCat$.
%Since $G$ is a right Quillen functor, we deduce that $\calC$ is a fibrant $\bfS$-enriched category.
%Suppose that $f$ is an equivalence in $\calC$ and form the pushout
%$\calC' = \calC \coprod_{ [1]_{\bfS} } \calE$. Since $\bfS$ satisfies the invertibility hypothesis, the
%canonical map $j: \calC \rightarrow \calC'$ is a trivial cofibration in $\SCat$. Since
%$\calC$ is fibrant, we conclude that $j$ admits a left inverse $\calC' \rightarrow \calC$.
%This left inverse determines a map $p: \calE \rightarrow \calC$. We now consider the
%extension problem $\wSCat$
%$$ \xymatrix{ (\calE, \emptyset) \ar[d] \ar[r] & (\calC, W) \\
%(\calE, \{f\}). \ar@{-->}[ur] & }$$
%Since $\bfS$ satisfies the invertibility hypothesis, the vertical map is a trivial cofibration.
%Since $(\calC,W)$ is fibrant, there exists a dotted arrow as indicated, which proves that
%$f$ lies in the image of the map $W \rightarrow \Hom_{\SCat}( [1]_{\bfS}, \calC)$.

%Conversely, let us suppose that $(\calC, W)$ satisfies the condition described in $(d)$;
%we wish to show that that $(\calC, W)$ is fibrant. In other words, we must show that it
%is possible to solve every lifting problem of the form
%$$ \xymatrix{ (\calA, S) \ar[d]^{q} \ar[r] & (\calC, W) \\
 
%\end{proof}







%\begin{remark}
%The category $\wSCat$ inherits a tensor structure from the tensor structure on
%$\SCat$: namely, we define
%$$ (\calC,W) \otimes (\calC', W') = (\calC \otimes \calC', W''),$$
%where $W'' = (\Hom_{\SCat}( [0]_{\bfS}, \calC) \times W') \coprod 


%\end{remark}

%\begin{lemma}
%Let $(\calC, W)$ be an object of $\wSCat$.
%\end{lemma}


%\subsection{The Underlying $\infty$-Category of a Model Category}\label{needles}

%Let $\bfA$ be a model category. We would like to extract from $\bfA$ an underlying $\infty$-category $\calC$. Roughly speaking, the idea is that $\calC$ should be obtained from $\bfA$ by formally inverting the weak equivalences. More precisely, we consider the marked simplicial set $( \Nerve(\bfA), W)$, where $W$ is the class of weak equivalences in $\bfA$. This is usually not a fibrant object of $(\mSet)$, but there exists a map
%$( \Nerve(\bfA), W) \rightarrow \calC^{\natural}$, where $\calC$ is an $\infty$-category. 
%Of course, $\calC$ is not uniquely determined, but it is determined up to (canonical) equivalence.
%More precisely, for every $\infty$-category $\calD$, the restriction map
%$$ \Fun(\calC, \calD) \rightarrow \Fun( \Nerve(\bfA), \calD)$$
%is fully faithful, and its essential image consists of those functors $F: \Nerve(\bfA) \rightarrow \calD$
%which carry each weak equivalence in $\bfA$ to an equivalence in $\calD$.

%If $\bfA$ is a {\em simplicial} model category, then there is a much simpler procedure for extracting
%the underlying $\infty$-category of $\bfA$. Let $\bfA^{\degree} \subseteq \bfA$ denote the full subcategory spanned by fibrant-cofibrant objects. Then $\bfA^{\degree}$ is a fibrant simplicial category, and we can define $\calC$ to be the {\em simplicial} nerve $\Nerve( \bfA^{\degree})$. This definition has the advantage of being very direct: $\calC$ is well-defined up to canonical isomorphism, rather than only up to equivalence. However, it is not obvious that this definition has the desired universal property: in fact, it is not obvious that $\calC$ receives a functor from the ordinary nerve of $\bfA$.

%Our goal in this section is to prove that (under mild hypotheses) the two constructions sketched above are equivalent to one another (Corollary \ref{smaj}). The proof will be based on the following more general result:

%\begin{proposition}\label{testur}
%Let $f: \calC \rightarrow \calD$ be a functor between $\infty$-categories, where $\calD$ is locally small. Suppose that for every small category $\calI$, the following conditions are satisfied:
%\begin{itemize}
%\item[$(1)$] The induced functor $$ F: \h{\Fun( \Nerve(\calI), \calC)} \stackrel{\circ f}{\rightarrow} \h{\Fun( \Nerve(\calI), \calD)}$$ is essentially surjective.
%\item[$(2)$] If $X'$ and $X''$ are objects of $\h{\Fun(\Nerve(\calI), \calC)}$ and
%$\eta: F(X') \simeq F(X'')$ is an isomorphism in $\h{\Fun(\Nerve(\calI), \calC)}$, then
%there exist maps $X' \stackrel{\alpha}{\rightarrow} X \stackrel{\beta}{\leftarrow} X''$
%such that $F(\alpha)$ and $F(\beta)$ are isomorphisms in $\h{\Fun( \Nerve(\calI), \calD)}$,
%and $\eta = F(\beta)^{-1} \circ F(\alpha)$.
%\end{itemize}
%Let $\calE$ denote the collection of equivalences in $\calD$. Then the induced map
%$$ ( \calC, f^{-1} \calE) \rightarrow (\calD, \calE)$$
%is an equivalence of marked simplicial sets.
%\end{proposition}

%\begin{corollary}\label{smajj}
%Let $\bfA$ be a combinatorial simplicial model category. Then the marked simplicial sets
%$\Nerve( \bfA^{\degree})^{\natural}$ and $( \Nerve^{\disc}(\bfA), W)$
%are marked equivalent to one another. Here $W$ denotes the class of weak equivalences in $\bfA$, and $\Nerve^{\disc}(\bfA)$ denotes the nerve of $\bfA$ regarded as an ordinary category.
%\end{corollary}

%\begin{proof}
%We have a diagram of marked simplicial sets
%$$ \Nerve( \bfA^{\degree})^{\natural}
%\stackrel{i}{\hookrightarrow} ( \Nerve(\bfA), W) \stackrel{j}{\hookleftarrow} (\Nerve^{\disc}(\bfA), W). $$
%Using a variant on the proof of Proposition \ref{quillobj}, we can construct a {\em simplicial}
%functor $T: \bfA \rightarrow \bfA$, together with a natural weak equivalence
%$\alpha: \id_{\bfA} \rightarrow T$, such that the essential image of $T$ consists of fibrant objects of $\bfA$. This proves that $i$ is a marked equivalence. To prove that $j$ is a marked equivalence,
%it will suffice to show that the hypotheses of Proposition \ref{testur} are satisfied.
%\begin{itemize}
%\item[$(1)$] Let $\calI$ be a small category and $p: \Nerve(\calI) \rightarrow \Nerve(\bfA^{\degree})$ a 
%Proposition \ref{gumby4} implies that $p$ is equivalent to a diagram which
%factors through $\Nerve^{\disc}(\bfA^{\degree})$. 
%\item[$(2)$] Let $q',q'': \calI \rightarrow \bfA^{\degree}$ be such that there is an
%equivalence $\eta: \Nerve(q') \rightarrow \Nerve(q'')$ in $\Fun( \Nerve(\calI), \Nerve( \bfA^{\degree}))$. We wish to show that there exist weak equivalences of diagrams
%$$ q' \stackrel{\alpha}{\rightarrow} q \stackrel{\beta}{\leftarrow} q''$$
%such that $\eta \simeq \Nerve(\beta)^{-1} \circ \Nerve(\alpha)$ in the homotopy category
%$\h \Fun( \Nerve(\calI), \Nerve(\bfA^{\degree}))$. Without loss of generality, we may
%assume that $q'$ and $q''$ are injectively fibrant diagrams. The desired result now follows

%\end{itemize}
%\end{proof}

%The remainder of this section is dedicated to the proof of Proposition \ref{testur}. The main idea is to show that $\infty$-categories can be closely approximated by ordinary categories. More precisely, we will use the following:

%\begin{lemma}\label{sulus}
%Let $(K, \calE)$ be a marked simplicial set. There exists a category $\calI$, a collection $W$ of morphisms of $\calI$ $($which contains every identity morphism$)$, and an equivalence of marked simplicial sets $( \Nerve(\calI), W) \rightarrow (K,\calE)$. Moreover, $\calI$ and $W$ can be chosen to depend functorially in $(K,\calE)$.
%\end{lemma}

%\begin{proof}
%Let $\calI$ be the category of simplices of $K$ and $f: \Nerve(\calI) \rightarrow K$ the map
%constructed in the proof of Proposition \ref{cofinalcategories}. Let $W = f^{-1} \calE$. 
%We observe that the construction
%$$ (K, \calE) \mapsto (\Nerve(\calI), W)$$
%is functorial in $(K,\calE)$, commutes with colimits, and preserves cofibrations. We wish to prove that $f$ induces a marked equivalence $( \Nerve(\calI),W) \rightarrow (K, \calE)$. By a standard argument (as in the proof of Proposition \ref{baby}), we may reduce to the case where
%$K = \Delta^n$. In this case, we can identify $\calI$ with the category $\cDelta_{/[n]}$, whose objects are maps $\sigma: [m] \rightarrow [n]$ of nonempty, finite linearly ordered sets. 
%The map $\Nerve(\calI) \rightarrow \Delta^n$ is induced by a functor from $F: \calI \rightarrow [n]$, which associates to each map $\sigma: [m] \rightarrow [n]$ the element $\sigma(m) \in [n]$.
%This functor has a right adjoint $G$, which carries each $k \in [n]$ to the inclusion
%$\{ 0, \ldots, k\} \subseteq [n]$. Let $g: ( \Delta^n, \calE) \rightarrow ( \Nerve(\calI), W)$
%be the induced map of marked simplicial sets. Then $f \circ g$ is the identity, and the unit map
%$\id_{\calI} \rightarrow G \circ F$ induces a homotopy
%$$ ( \Nerve(\calI), W) \times (\Delta^1)^{\sharp} \rightarrow (\Nerve(\calI),W)$$
%from the identity to $g \circ f$; it follows that $f$ and $g$ are mutually inverse equivalences
%in the homotopy category $\h{\mSet}$.
%\end{proof}

%\begin{notation}
%Let $T: \mSet \rightarrow \mSet$ be a fibrant replacement functor, so that for every marked simplicial set $(X,\calE)$, we have $T(X,\calE) = \calC^{\natural}$ for some $\infty$-category $\calC$.
%We let $\h{(X,\calE)}$ denote the homotopy category $\h{\calC}$, considered as a category enriched over $\calH$. In view of Proposition \ref{markedjoyal}, a map of marked simplicial sets
%$ (X, \calE) \rightarrow (X', \calE')$ is a marked equivalence if and only if the induced map
%$\h{(X,\calE)} \rightarrow \h{(X',\calE')}$ is an equivalence of $\calH$-enriched categories.
%\end{notation}

%\begin{lemma}\label{pretestur}
%Let $f: \calC \rightarrow \calD$ be a functor between $\infty$-categories which satisfies
%the hypotheses of Proposition \ref{testur}, and let $\calE$ denote the collection of all equivalences in $\calD$.
%Suppose given a commutative diagram
%of marked simplicial sets
%$$ \xymatrix{ X \ar[r]^{f'} \ar[d]^{\phi} & Y \ar[d]^{\phi'} \\
%(\calC, f^{-1} \calE) \ar[r]^{f} & \calD^{\natural}, }$$
%where $X$ and $Y$ are small, and consider the induced diagram of $\calH$-enriched homotopy categories
%$$ \xymatrix{ \h{X} \ar[r]^{F'} \ar[d]^{\psi} & \h{Y} \ar[d]^{\psi'} \ar@{-->}[dl]_{\sigma} \\
%\h{(\calC, f^{-1} \calE)} \ar[r]^{F} & \h{\calD}. }$$
%Then there exists a dotted arrow as indicated, together with isomorphisms
%$\beta: \sigma \circ F' \simeq \psi$, $\alpha: F \circ \sigma \simeq \psi'$, with the property
%that the composition
%$$ F \circ \psi \stackrel{\beta}{\simeq} F \circ \sigma \circ F' \stackrel{\alpha}{\simeq}
%\psi' \circ F'$$ is the identity.
%\end{lemma}

%\begin{proof}
%Using Lemma \ref{sulus}, we may assume without loss of generality that
%$X = ( \Nerve(\calI), W)$ and $Y = ( \Nerve(\calI'), W')$, where $\calI$ and $\calI'$ are small
%categories. Assumption $(1)$ of Proposition \ref{testur} guarantees the existence
%of $\widetilde{\sigma}: \Nerve(\calI) \rightarrow \calC$ such that the diagram
%$$ \xymatrix{ & (\Nerve(\calI),W) \ar[d] \ar[dl]_{\widetilde{\sigma}} \\
%(\calC, f^{-1} \calE) \ar[r] & \calD^{\natural} }$$
%commutes up to homotopy $\widetilde{\alpha}: f \circ \widetilde{\sigma} \rightarrow \phi'$.
%Let $\sigma$ denote the induced functor between homotopy categories, and 
%$\alpha: F \circ \sigma \simeq \psi'$ the natural isomorphism induced by the homotopy $\widetilde{\alpha}$. 
%We have an induced homotopy 
%$$f'(\widetilde{\alpha}): f \circ \widetilde{\sigma} \circ f' \rightarrow \phi' \circ f' = f \circ \phi.$$ 
%Using assumption $(2)$ of Proposition \ref{testur}, we deduce the existence of
%natural transformations
%$$ \widetilde{\sigma} \circ f' \stackrel{\widetilde{\beta}'}{\rightarrow} \phi'' \stackrel{\widetilde{\beta}''}{\leftarrow} \phi,$$
%such that $f(\widetilde{\beta}')$ and $f(\widetilde{\beta}'')$ are equivalences,
%and $f'(\widetilde{\alpha})$ is equal to the composition $f( \widetilde{\beta}'')^{-1} \circ f( \widetilde{\beta}')$ in
%the homotopy category $\h{\Fun( \Nerve(\calI), \calD)}$. We observe that
%$\widetilde{\beta}'$ and $\widetilde{\beta}''$ induce isomorphisms $\beta'$
%and $\beta''$ in the category of $\calH$-enriched functors from
%$\h{( \Nerve(\calI),W)}$ to $\h{ (\calC, f^{-1} \calE)}$. We now set $\beta = (\beta'')^{-1} \circ \beta'$.
%It is easy to see that $\sigma$, $\alpha$, and $\beta$ possess the desired properties.
%\end{proof}

%\begin{proof}[Proof of Proposition \ref{testur}]
%The proof is essentially contained in Lemma \ref{pretestur}: we would like to take
%$X = ( \calC, f^{-1} \calE)$ and $Y = \calD^{\natural}$, and use the existence of $\sigma$
%to deduce that $f$ induces an equivalence of $\calH$-enriched homotopy categories
%$\h( \calC, f^{-1} \calE) \rightarrow \h{\calD}$. We encounter here a technicality stemming from the fact that $\calC$ and $\calD$ need not be small (in fact, they are not small in the principal case of interest: Corollary \ref{smaj}). We must therefore use a slightly more indirect argument. To prove that $f$ induces an equivalence of $\calH$-enriched homotopy categories, it will suffice to prove the following assertions:

%\begin{itemize}
%\item[$(i)$] For every object $D \in \calD$, there exists an object $C \in \calC$ and an
%equivalence $D \rightarrow f(C)$.

%\item[$(ii)$] For every pair of objects $C,C' \in \calC$ and every morphism 
%$\overline{\phi}: f(C) \rightarrow f(C')$, there exists a morphism
%$\phi: C \rightarrow C'$ in $\h{ (\calC, f^{-1} \calE)}$ such that
%$f(\phi)$ and $\overline{\phi}$ coincide in the homotopy category $\h{\calD}$. 

%\item[$(iii)$] For every morphism $\phi: C \rightarrow C'$ in 
%$\h{ (\calC, f^{-1} \calE)} $ and every $n \geq 0$, the induced map 
%$$\pi_{n}( \bHom_{\h{( \calC, f^{-1} \calE)}}(C,C'), \phi) \rightarrow
%\pi_{n}(\bHom_{\calD}( f(C), f(C')), f(\phi) )$$ is surjective. 

%\item[$(iv)$] For every morphism $\phi: C \rightarrow C'$ in 
%$\h{ (\calC, f^{-1} \calE)} $ and every $n \geq 0$, the induced map 
%$$\pi_{n}( \bHom_{\h{( \calC, f^{-1} \calE)}}(C,C'), \phi) \rightarrow
%\pi_{n}(\bHom_{\calD}( f(C), f(C')), f(\phi) )$$ is injective. 
%\end{itemize}

%Assertion $(i)$ follows immediately from $(1)$ in the case where $\calI$ consists of a single object. 
%To prove $(ii)$, we let $\calI$ be the be the two-object category $[1] = \{0,1\}$, and 
%$p: \Nerve(\calI) \rightarrow \calD$ the diagram classifying a morphism $\overline{\phi}: f(C) \rightarrow f(C')$. Applying condition $(1)$, we deduce that there is a diagram 
%$\phi': C_0 \rightarrow C'_0$ in $\calC$, and a square
%$$ \xymatrix{ f(C_0) \ar[d]^{\overline{\psi}} \ar[r]^{f(\phi')} & f(C'_0) \ar[d]^{\overline{\psi}'} \\
%f(C) \ar[r]^{\overline{\phi}} & f(C') }$$
%in $\calD$, where $\overline{\psi}$ and $\overline{\psi}'$ are equivalences. Using condition $(2)$, we deduce that there exist isomorphisms $\psi: C_0 \rightarrow C$, $\psi': C'_0 \rightarrow C'$ in the homotopy category $\h{(\calC, f^{-1} \calE)}$ such that $\overline{\psi} \simeq f(\psi)$ and $\overline{\psi}' \simeq f( \psi')$. We now observe that $\overline{\phi} \simeq f(\phi)$, where $\phi = \psi' \circ \phi' \circ \psi^{-1}$ (in the homotopy category $\h{\calD}$).

%We now prove $(iii)$. Without loss of generality, we may suppose that $\calD$ is a minimal
%$\infty$-category. Choose a small simplicial subset $\calC(0) \subseteq \calC$ such that
%$\phi$ belongs to the image of the induced map
%$$\h{ (\calC(0), f^{-1} \calE \cap \calC(0)_{1} )} \rightarrow \h{ (\calC, f^{-1} \calE)},$$ and let
%$\calD(0) \subseteq \calD$ be the essential image of $f| \calC(0)$. Since $\calD$ is minimal and locally %small, $\calD(0)$ is small. We now apply Lemma \ref{pretestur} to the diagram
%$$ \xymatrix{ ( \calC(0), f^{-1} \calE \cap \calC(0)_{1}) \ar[d] \ar[r] & \calD(0)^{\natural} \ar[d] \\
%(\calC, f^{-1} \calE) \ar[r] & \calD^{\natural} }$$
%to deduce the existence of a $\calH$-enriched functor $\sigma: \h{\calD(0)}
%\rightarrow \h{ (\calC, f^{-1} \calE)}$. By construction, $\phi' = (\sigma \circ f)(\phi): B \rightarrow B'$
%is isomorphic to $\phi$ (in the category of arrows of $\h{ (\calC, f^{-1} \calE)}$. It will therefore
%suffice to prove that the map
%$$ \pi_n( \bHom_{ \h{ (\calC, f^{-1} \calE)} }( B,B'), \phi') \rightarrow
%\pi_{n}( \bHom_{ \h{\calD} }( f(B), f(B')), f(\phi') ) \simeq \pi_{n}( \bHom_{\h{\calD}}( f(C), f(C')), \phi)$$
%is surjective. This is clear, since this map admits a section (given by $\sigma$).

%The proof of $(iv)$ is a bit easier. Suppose given a map
%$\eta: S^n \rightarrow \bHom_{ \h{(\calC, f^{-1} \calE)}}(C,C')$ in the homotopy category
%$\calH$, and suppose that $f(\eta): S^n \rightarrow \bHom_{\calD}(f(C), f(C'))$ is nullhomotopic. Choose a map of marked simplicial
%sets $g: X \rightarrow (\calC, f^{-1} \calE)$, where $X$ is small and $\eta$ is the image of a
%map $\widetilde{\eta}: S^n \rightarrow \bHom_{ \h{X}}(B,B')$. Since $(f \circ g)(\eta)$ is nullhomotopic, the map $g \circ f$ factors as a composition
%$$ X \stackrel{ f' }{\rightarrow} Y \stackrel{g'}{\rightarrow} \calD^{\natural},$$
%where $Y$ is small and $f'( \widetilde{\eta})$ is nullhomotopic. We now apply
%Lemma \ref{pretestur} to the diagram
%$$ \xymatrix{ X \ar[d]^{g} \ar[r]^{f'} & Y \ar[d]^{g'} \\
%( \calC, f^{-1} \calE) \ar[r]^{f} & \calD^{\natural} }$$
%to conclude that $\eta$ is itself nullhomotopic, as desired.
%\end{proof}




