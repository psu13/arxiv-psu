 \section{$\infty$-Categories of Inductive Limits}\label{c5s3}

\setcounter{theorem}{0}

Let $\calC$ be a category. An {\it $\Ind$-object} of $\calC$ is a diagram\index{gen}{Ind-object}
$f: \calI \rightarrow \calC$ where $\calI$ is a small filtered category. We will informally denote the $\Ind$-object $f$ by $$ [\colim X_i] $$
where $X_i = f(i)$. The collection of all $\Ind$-objects of $\calC$ forms a category, where
the morphisms are given by the formula
$$ \Hom_{\Ind(\calC)}( [\colim X_i], [\colim Y_j]) =
\varprojlim \varinjlim \Hom_{\calC}(X_i, Y_j).$$
We note that $\calC$ may be identified with a full subcategory of $\Ind(\calC)$, corresponding
to diagrams indexed by the one-point category $\calI = \ast$. 
The idea is that $\Ind(\calC)$ is obtained from $\calC$ by formally adjoining colimits of filtered diagrams. More precisely, $\Ind(\calC)$ may be described by the following universal property:
for any category $\calD$ which admits filtered colimits, and any functor $F: \calC \rightarrow \calD$, there exists a functor $\widetilde{F}: \Ind(\calC) \rightarrow \calD$, whose restriction to $\calC$ is isomorphic to $F$, and which commutes with filtered colimits. Moreover, $\widetilde{F}$ is determined up to (unique) isomorphism.\index{not}{IndC@$\Ind(\calC)$}

\begin{example}
Let $\calC$ denote the category of finitely presented groups. Then $\Ind(\calC)$ is equivalent to the category of groups. (More generally, one could replace ``group'' by any type of mathematical structure described by algebraic operations which are required to satisfy equational axioms.)
\end{example}

Our objective in this section is to generalize the definition of $\Ind(\calC)$ to the case where $\calC$ is an $\infty$-category. If we were to work in the setting of simplicial (or topological) categories, we could apply the definition given above directly. However, this leads to a number of problems:

\begin{itemize}
\item[$(1)$] The construction of $\Ind$-categories does not preserve equivalences between simplicial categories.

\item[$(2)$] The obvious generalization of the right hand side in equation above is given by
$$ \varprojlim \colim \bHom_{\calC}(X_i,Y_j).$$
While the relevant limits and colimits certainly exist in the category of simplicial sets, they
are not necessarily the correct objects: really one should replace the limit by a homotopy limit.

\item[$(3)$] In the higher-categorical setting, we should really allow the indexing diagram $\calI$
to be a higher category as well. While this does not result in any additional generality (Corollary \ref{rot}), the restriction to the diagrams indexed by ordinary categories is a technical inconvenience.
\end{itemize}

Although these difficulties are not insurmountable, it is far more convenient to proceed differently, using the theory of $\infty$-categories. In \S \ref{c5s1}, we showed that if $\calC$ is a $\infty$-category, then $\calP(\calC)$ can be interpreted as an $\infty$-category which is freely generated by $\calC$ under colimits. We might therefore hope to find $\Ind(\calC)$ {\it inside} of $\calP(\calC)$, as a full subcategory. The problem, then, is to characterize this subcategory, and to prove that it has the appropriate universal mapping property.

We will begin in \S \ref{smallfilt}, by introducing the definition of a {\em filtered $\infty$-category}.
Let $\calC$ be a small $\infty$-category. In \S \ref{indlim}, we will define $\Ind(\calC)$ to be the smallest full subcategory of $\calP(\calC)$ which contains all representable presheaves on $\calC$ and is stable under filtered colimits. There is also a more direct characterization of which presheaves $F: \calC \rightarrow \SSet^{op}$ belong to $\Ind(\calC)$: they are precisely the {\em right exact} functors, which we will study in \S \ref{rexex}.

In \S \ref{indlim}, we will define the $\Ind$-categories $\Ind(\calC)$ and study their properties. In particular, we will show that morphism spaces in $\Ind(\calC)$ {\em are} computed by the naive formula
$$ \Hom_{\Ind(\calC)}( [\colim X_i], [\colim Y_j]) =
\varprojlim \varinjlim \Hom_{\calC}(X_i, Y_j).$$
Unwinding the definitions, this amounts to two conditions:
\begin{itemize}
\item[$(1)$] The (Yoneda) embedding of $j: \calC \rightarrow \Ind(\calC)$ is fully faithful (Proposition \ref{fulfaith}).
\item[$(2)$] For each object $C \in \calC$, the corepresentable functor
$$ \Hom_{\Ind(\calC)}( j(C), \bigdot) $$ commutes with filtered colimits.
\end{itemize}
It is useful to translate condition $(2)$ into a definition: an object $D$ of an $\infty$-category
$\calD$ is said to be {\it compact} if the functor $\calD \rightarrow \SSet$ corepresented by $D$ commutes with filtered colimits. We will study this compactness condition in \S \ref{compobj}.

One of our main results asserts that the $\infty$-category $\Ind(\calC)$ is obtained from $\calC$ by freely adjoining colimits of filtered diagrams (Proposition \ref{intprop}). In \S \ref{agileco}, we will describe a similar construction in the case where the class of filtered diagrams has been replaced by {\em any} class of diagrams. We will revisit this idea in \S \ref{stable11}, where we will study the $\infty$-category obtained from $\calC$ by freely adjoining colimits of {\em sifted} diagrams.

\subsection{Filtered $\infty$-Categories}\label{smallfilt}

Recall that a partially ordered\index{gen}{filtered!partially ordered set}
set $A$ is {\it filtered} if every finite subset of $A$ has an
upper bound in $A$. Diagrams indexed by directed partially ordered sets are extremely common in mathematics. For example, if $A$ is the set $$\Z_{\geq 0} = \{0, 1, \ldots \}$$ of natural
numbers, then a diagram indexed by $A$ is a sequence
$$ X_0 \rightarrow X_1 \rightarrow \ldots .$$
The formation of direct limits for such sequences is one of the most basic constructions in mathematics.

In classical category theory, it is convenient to consider not
only diagrams indexed by filtered partially ordered sets, but also
more general diagrams indexed by filtered categories. A category
$\calC$ is said to be {\it filtered} if it satisfies following
conditions:\index{gen}{filtered!category}

\begin{itemize}
\item[$(1)$] For every finite collection $\{ X_i \}$ of objects of $\calC$, there exists an object
$X \in \calC$ equipped with morphisms $\phi_i: X_i \rightarrow X$.

\item[$(2)$]  Given any two morphisms $f,g: X \rightarrow Y$ in $\calC$,
there exists a morphism $h: Y \rightarrow Z$ such that $h \circ f
= h \circ g$.
\end{itemize}

Condition $(1)$ is analogous to the requirement that any
finite part of $\calC$ admits an ``upper bound'', while condition $(2)$ guarantees that the upper bound is unique in some asymptotic sense.

If we wish to extend the above definition to the $\infty$-categorical setting, it is natural to strengthen the second condition.

\begin{definition}\label{topfilt}\index{gen}{filtered!topological category}
Let $\calC$ be a topological category. We will say that $\calC$ is {\it filtered} if it satisfies
the following conditions:
\begin{itemize}
\item[$(1')$] For every finite set $\{X_i\}$ of objects of $\calC$, there exists an object $X \in \calC$
and morphisms $\phi_i: X_i \rightarrow X$.
\item[$(2')$] For every pair $X, Y \in \calC$ of objects of $\calC$, every nonnegative integer $n \geq 0$, and every continuous map $S^n \rightarrow \bHom_{\calC}(X,Y)$, there exists a morphism $Y \rightarrow Z$ such that the induced map $S^n \rightarrow \bHom_{\calC}(X,Z)$ is nullhomotopic.
\end{itemize}
\end{definition}

\begin{remark}
It is easy to see that an ordinary category $\calC$ is filtered in the usual sense if and only if it is filtered when regarded as a topological category with discrete mapping spaces. Conversely, 
if $\calC$ is a filtered topological category, then its homotopy category $\h{\calC}$ is filtered (when viewed as an ordinary category). 
\end{remark}

\begin{remark}
Condition $(2')$ of Definition \ref{topfilt} is a reasonable analogue of condition $(2)$ in the definition of a filtered category. In the special case $n=0$, condition $(2')$ asserts that any pair of morphisms
$f,g: X \rightarrow Y$ become {\em homotopic} after composition with some map $Y \rightarrow Z$.
\end{remark}

\begin{remark}
Topological spheres $S^n$ need not play any distinguished role in the definition of a filtered topological category. Condition $(2')$ is equivalent to the following apparently stronger condition:
\begin{itemize}
\item[$(2'')$] For every pair $X, Y \in \calC$ of objects of $\calC$, every finite cell complex $K$, and every continuous map $K \rightarrow \bHom_{\calC}(X,Y)$, there exists a morphism $Y \rightarrow Z$ such that the induced map $K \rightarrow \bHom_{\calC}(X,Z)$ is nullhomotopic.
\end{itemize}
\end{remark}

\begin{remark}
The condition that a topological category $\calC$ be filtered depends only on the homotopy category $\h{\calC}$, viewed as a $\calH$-enriched category. Consequently if $F: \calC \rightarrow \calC'$ is an equivalence of topological categories, then $\calC$ is filtered if and only if $\calC'$ is filtered.
\end{remark}

\begin{remark}
Definition \ref{topfilt} has an obvious analogue for (fibrant) simplicial categories: one simply replaces the topological $n$-sphere $S^n$ by the simplicial $n$-sphere $\bd \Delta^n$. It is easy to see that a topological category $\calC$ is filtered if and only if the simplicial category $\Sing \calC$ is filtered. Similarly, a (fibrant) simplicial category $\calD$ is filtered if and only if the topological category $|\calD|$ is filtered.
\end{remark}

We now wish to study the analogue of Definition \ref{topfilt} in the setting of $\infty$-categories. It will be convenient to introduce a slightly more general notion:

\begin{definition}\label{filtquas}\index{gen}{filtered!$\infty$-category}\index{gen}{$\kappa$-filtered}
Let $\kappa$ be a regular cardinal, and let $\calC$ be a $\infty$-category. We will say that
$\calC$ is {\it $\kappa$-filtered} if, for every $\kappa$-small simplicial set $K$ and every 
map $f: K \rightarrow \calC$, there exists a map $\overline{f}: K^{\triangleright} \rightarrow \calC$ extending $f$. (In other words, 
$\calC$ is $\kappa$-filtered if it has the extension property with respect to the inclusion
$K \subseteq K^{\triangleright}$, for every $\kappa$-small simplicial set $K$.)

We will say that $\calC$ is {\it filtered} if it is $\omega$-filtered.
\end{definition}

\begin{example}
Let $\calC$ be the nerve of a partially ordered set $A$. Then $\calC$ is $\kappa$-filtered if and only if every $\kappa$-small subset of $A$ has an upper bound in $A$.
\end{example}

\begin{remark}\label{falg}
One may rephrase Definition \ref{filtquas} as follows: an $\infty$-category $\calC$ is $\kappa$-filtered if and only if,
for every diagram $p: K \rightarrow \calC$, where $K$ is $\kappa$-small, the slice $\infty$-category
$\calC_{p/}$ is nonempty. 
Let $q: \calC \rightarrow \calC'$ be a categorical equivalence of $\infty$-categories. Proposition \ref{gorban3} asserts that the induced map $\calC_{p/} \rightarrow \calC'_{q \circ p/}$ is a categorical equivalence. Consequently $\calC_{p/}$ is nonempty if and only if $\calC'_{q \circ p/}$ is nonempty. It follows that $\calC$ is $\kappa$-filtered if and only if $\calC'$ is $\kappa$-filtered.
\end{remark}

\begin{remark}\label{tweeny}
An $\infty$-category $\calC$ is $\kappa$-filtered if and only if, for every
$\kappa$-small partially ordered set $A$, $\calC$ has the right lifting property with respect
to the inclusion $\Nerve(A) \subseteq \Nerve(A)^{\triangleright}  \simeq \Nerve(A \cup \{ \infty\} )$. The ``only if'' direction is obvious. For the converse, we observe that for every $\kappa$-small diagram $p: K \rightarrow \calC$, the $\infty$-category $\calC_{p/}$ is equivalent to
$\calC_{q/}$, where $q$ denotes the composition
$K'' \stackrel{p'}{\rightarrow} K \stackrel{p}{\rightarrow} \calC$. Here $K''$ is the second barycentric subdivision of $K$ and $p'$ is the map described in Variant \ref{baryvar}. We now observe that $K''$ is equivalent to the nerve of a $\kappa$-small partially ordered set. 
\end{remark}

\begin{remark}\index{gen}{filtered!simplicial set}
We will say that an arbitrary simplicial set $S$ is {\it $\kappa$-filtered} if there exists a categorical equivalence $j: S \rightarrow \calC$, where $\calC$ is a $\kappa$-filtered $\infty$-category. In view of the Remark \ref{falg}, this condition is independent of the choice of $j$.
\end{remark}

Our next major goal is to prove Proposition \ref{stook}, which asserts that an $\infty$-category
$\calC$ is filtered if and only if the associated topological category $| \sCoNerve[\calC] |$ is filtered.
First, we need a lemma.

\begin{lemma}\label{goony}
Let $\calC$ be an $\infty$-category. Then $\calC$ is filtered if and only if it has the right extension
property with respect to every inclusion $\bd \Delta^n \subseteq \Lambda^{n+1}_{n+1}$, $n \geq 0$.
\end{lemma}

\begin{proof}
The ``only if'' direction is clear: we simply take $K = \bd \Delta^n$ in Definition \ref{filtquas}.
For the converse, let us suppose that the assumption of Definition \ref{filtquas} is satisfied whenever $K$ is the boundary of a simplex; we must then show that it remains satisfied for {\em any} $K$ which has only finitely many nondegenerate simplices.

We work by induction on the dimension of $K$, and then by the number of nondegenerate simplices of $K$. If $K$ is empty, there is nothing to prove (since it is the boundary of a $0$-simplex). Otherwise, we may write $K = K' \coprod_{ \bd \Delta^n } \Delta^n$, where $n$ is the dimension of $K$. 

Choose a map $p: K \rightarrow \calC$; we wish to show that $p$ may be extended to a map
$\widetilde{p}: K \star \{y\} \rightarrow \calC$. We first consider the restriction $p|K'$; by the inductive hypothesis it admits an extension $q: K' \star \{x\} \rightarrow \calC$. The
restriction $q | \bd \Delta^n \star \{x\}$ together with $p|\Delta^n$ assemble to give a map
$$ r: \bd \Delta^{n+1} \simeq ( \bd \Delta^n \star \{x\} ) \coprod_{ \bd \Delta^n } \Delta^n \rightarrow
\calC.$$
By assumption, the map $r$ admits an extension
$$\widetilde{r}: \bd \Delta^{n+1} \star \{y\} \rightarrow \calC.$$

Let $$s: ( K' \star \{x\} ) \coprod_{ \bd \Delta^{n+1}  } ( \bd \Delta^{n+1} \star \{y\})$$ denote the result of amalgamating $r$ with $\widetilde{p}$. We note that the inclusion
$$ (K' \star \{x\} ) \coprod_{ \bd \Delta^n \star \{x\} } ( \bd \Delta^{n+1} \star \{y\} )
\subseteq (K' \star \{x\} \star \{y\}) \coprod_{ \bd \Delta^n \star \{x\} \star \{y\} } ( \Delta^n \star \{y\} )$$
is a pushout of
$$ (K' \star \{x\}) \coprod_{ \bd \Delta^n \star \{x\} } (\bd \Delta^n \star \{x\} \star \{y\} )
\subseteq K' \star \{x\} \star \{y\},$$ and therefore a categorical equivalence by Lemma \ref{doweneed}. It follows that $s$ admits an extension
$$ \widetilde{s}: (K' \star \{x\} \star \{y\}) \coprod_{ \bd \Delta^n \star \{x\} \star \{y\} } ( \Delta^n \star \{y\} ) \rightarrow \calC, $$ and we may now define $\widetilde{p} = \widetilde{s} | K \star \{y\}$. 
\end{proof}

\begin{proposition}\label{stook}
Let $\calC$ be a topological category. Then $\calC$ is filtered if and only if the $\infty$-category
$\tNerve(\calC)$ is filtered.
\end{proposition}

\begin{proof}
Suppose first that $\tNerve(\calC)$ is filtered. We verify conditions $(1')$ and $(2')$ of Definition \ref{topfilt}:

\begin{itemize}
\item[$(1')$] Let $\{X_i\}_{i \in I}$ be a finite collection of
objects of $\calC$, corresponding to a map $p: I \rightarrow \tNerve(\calC)$, where $I$ is regarded as a discrete simplicial set. If $\tNerve(\calC)$ is filtered, then $p$ extends to a map
$\widetilde{p}: I \star \{x\} \rightarrow \tNerve(\calC)$, corresponding to an object $X = p(x)$
equipped with maps $X_i \rightarrow X$ in $\calC$.

\item[$(2')$] Let $X, Y \in \calC$ be objects, $n \geq 0$, and $S^n \rightarrow \bHom_{\calC}(X,Y)$ a map. We note that this data may be identified with a topological functor $F: | \sCoNerve[K] | \rightarrow \calC$, where $K$ is the simplicial set obtained from $\bd \Delta^{n+2}$ by collapsing the initial face
$\Delta^{n+1}$ to a point. If $\tNerve(\calC)$ is filtered, then $F$ extends to a functor $\widetilde{F}$ defined on $| \sCoNerve[ K \star \{z\}] |$; this gives an object $Z = \widetilde{F}(z)$ and a morphism
$Y \rightarrow Z$ such that the induced map $S^n \rightarrow \bHom_{\calC}(X,Z)$ is nullhomotopic.
\end{itemize}

For the converse, let us suppose that $\calC$ is filtered. We wish to show that $\tNerve(\calC)$ is filtered. By Lemma \ref{goony}, it will suffice to prove that $\tNerve(\calC)$ has the extension property with respect to the inclusion $\bd \Delta^n \subseteq \Lambda^{n+1}_{n+1}$, for each $n \geq 0$. Equivalently, it suffices to show that $\calC$ has the right extension property with
respect to the inclusion $| \sCoNerve[ \bd \Delta^n] | \subseteq | \sCoNerve[ \Lambda^{n+1}_{n+1} ] |$.
If $n = 0$, this is simply the assertion that $\calC$ is nonempty; if $n = 1$, this is the assertion that for any pair of objects $X,Y \in \calC$ there exists an object $Z$ equipped with morphisms
$X \rightarrow Z$, $Y \rightarrow Z$. Both of these conditions follow from part $(1)$ of Definition \ref{topfilt}; we may therefore assume that $n > 1$.

Let $\calA_0 = | \sCoNerve[\bd \Delta^n] |$, $\calA_1 = | \sCoNerve[ \bd \Delta^n \coprod_{ \Lambda^n_n}  \Lambda^n_n \star \{ n+1 \}]|$, $\calA_2 = | \sCoNerve[ \Lambda^{n+1}_{n+1}] |$, and
$\calA_3 = | \sCoNerve[ \Delta^{n+1}] |$, so that we have inclusions of topological categories
$$ \calA_0 \subseteq \calA_1 \subseteq \calA_2 \subseteq \calA_3.$$

We will make use of the description of $\calA_3$ given in Remark \ref{conervexp}: its objects are integers $i$ satisfying $0 \leq i \leq n+1$, with $\bHom_{\calA_3}(i,j)$ given by the cube of all functions $p: \{i, \ldots, j\} \rightarrow [0,1]$ satisfying $p(i)=p(j)=1$ for $i \leq j$, and$\Hom_{\calA_3}(i,j) = \emptyset$ for $j < i$. Composition is given by amalgamation of functions.

We note that $\calA_1$ and $\calA_2$ are subcategories of $\calA_3$ having the same objects, where:
\begin{itemize}
\item $\bHom_{\calA_1}(i,j) = \bHom_{\calA_2}(i,j) = \bHom_{\calA_3}(i,j)$ unless $i=0$ and
$j \in \{n,n+1\}$. 

\item $\bHom_{\calA_1}(0,n) = \bHom_{\calA_2}(0,n)$ is the boundary of the cube $\bHom_{\calA_3}(0,n) = [0,1]^{n-1}$.

\item $\bHom_{\calA_1}(0,n+1)$ consists of all functions $p: [n+1] \rightarrow [0,1]$
satisfying $p(0) = p(n+1) =1$ and $(\exists i) [ (1 \leq i \leq n-1) \wedge p(i) \in \{0,1\} ]$.

\item $\bHom_{\calA_2}(0,n+1)$ is the union of $\bHom_{\calA_1}(0,n+1)$ with the collection
of functions $p: \{0, \ldots, n+1\} \rightarrow [0,1]$ satisfying $p(0) = p(n) = p(n+1) = 1$.
\end{itemize}

Finally, we note that $\calA_0$ is the full subcategory of $\calA_1$ (or $\calA_2$) whose set of objects is $\{0, \ldots, n\}$.

We wish to show that any topological functor $F: \calA_0 \rightarrow \calC$ can be extended to a 
functor $\widetilde{F}: \calA_2 \rightarrow \calC$. Let $X = F(0)$, $Y = F(n)$. Then $F$ induces a map $S^{n-1} \simeq \bHom_{\calA_0}(0,n) \rightarrow \bHom_{\calC}(X,Y)$. Since $\calC$ is filtered, there exists a map $\phi: Y \rightarrow Z$ such that the induced map
$f: S^{n-1} \rightarrow \bHom_{\calC}(X,Z)$ is nullhomotopic.

Now set $\widetilde{F}(n+1) = Z$; for $p \in \bHom_{\calA_1}(i, n+1)$, we set
$\widetilde{F}(p) = \phi \circ F(q)$, where $q \in \bHom_{\calA_1}(i,n)$ is such that
$q| \{i, \ldots, n-1 \} = p | \{i, \ldots, n-1\}$. Finally, we note that the assumption that
$f$ is nullhomotopic allows us to extend $\widetilde{F}$ from $\bHom_{\calA_1}(0,n+1)$ to the
whole of $\bHom_{\calA_2}(0,n+1)$.
\end{proof}

\begin{remark}\label{elisa}
Suppose that $\calC$ is a $\kappa$-filtered $\infty$-category, and let $K$ be a simplicial set which is categorically equivalent to a $\kappa$-small simplicial set. Then $\calC$ has the extension property with respect to the inclusion $K \subseteq K^{\triangleright}$. This follows from Proposition \ref{princex}: to test whether or not a map $K \rightarrow S$ extends over $K^{\triangleright}$, it suffices to check in the homotopy category of $\sSet$ (with respect to the Joyal model structure), where we may replace $K$ by an equivalent $\kappa$-small simplicial set. 
\end{remark}

\begin{proposition}\label{smallity}
Let $\calC$ be a $\infty$-category with a final object. Then $\calC$ is
$\kappa$-filtered for every regular cardinal $\kappa$. Conversely, if $\calC$ is $\kappa$-filtered and there exists a categorical
equivalence $K \rightarrow \calC$, where $K$ is a $\kappa$-small simplicial set, then $\calC$ has a final object.
\end{proposition}

\begin{proof}
We remark that $\calC$ has a final object if and only if there exists
a retraction $r$ of $\calC^{\triangleright}$ onto $\calC$. If $\calC$ is $\kappa$-filtered and categorically equivalent to a $\kappa$-small simplicial set, then the existence of such a retraction follows from Remark \ref{elisa}. On the other hand, if
the retraction $r$ exists, then any map $p: K \rightarrow \calC$
admits an extension $K^{\triangleright} \rightarrow \calC$: one merely
considers the composition $ K^{\triangleright} \rightarrow \calC^{\triangleright} \stackrel{r}{\rightarrow} \calC.$
\end{proof}

A useful observation from classical category theory is that, if we are only interested in using filtered categories to index colimit diagrams, then in fact we do not need the notion of a filtered category at all: we can work instead with diagrams indexed by filtered partially ordered sets. We now prove an $\infty$-categorical analogue of this statement.

\begin{proposition}\label{rot}
Suppose that $\calC$ is a $\kappa$-filtered $\infty$-category. Then there
exists a $\kappa$-filtered partially ordered set $A$ and a cofinal map
$\Nerve(A) \rightarrow \calC$.
\end{proposition}

\begin{proof}
The proof uses the ideas introduced in \S \ref{quasilimit1}, and in particular Proposition \ref{utl}.
Let $X$ be a set of size $\geq \kappa$, and regard $X$ as a
category with a unique isomorphism between any pair of objects. We
note that $\Nerve(X)$ is a contractible Kan complex; consequently the
projection $\calC \times \Nerve(X) \rightarrow \calC$ is cofinal. Hence, it
suffices to produce a cofinal map $\Nerve(A) \rightarrow \calC \times \Nerve(X)$
with the desired properties.

Let $\{ K_{\alpha} \}_{\alpha \in A}$ be the collection of all
simplicial subsets of $K = \calC \times \Nerve(X)$ which are $\kappa$-small and
possess a final vertex. Regard $A$ as a partially ordered by inclusion.
We first claim that $A$ is $\kappa$-filtered and that
$\bigcup_{\alpha \in A} K_{\alpha} = K$. To prove both
of these assertions, it suffices to show that any $\kappa$-small
simplicial subset $L \subseteq K$ is contained in a
$\kappa$-small simplicial subset $L'$ which has a final
vertex.

Since $\calC$ is $\kappa$-filtered, the composition
$$L \rightarrow \calC
\times \Nerve(X) \rightarrow \calC$$ extends to a map $p: L^{\triangleright} \rightarrow \calC$.
Since $X$ has cardinality $\geq \kappa$, there exists an element
$x \in X$ which is not in the image of $L_0 \rightarrow N(X)_0 = X$.
Lift $p$ to a map $\widetilde{p}: L^{\triangleright} \rightarrow K$ which extends the inclusion $L \subseteq K \times
\Nerve(X)$ and carries the cone point to the element $x \in X = N(X)_0$. It
is easy to see that $\widetilde{p}$ is injective, so that we may
regard $L^{\triangleright}$ as a simplicial subset of $K \times
\Nerve(X)$. Moreover, it is clearly $\kappa$-small and has a
final vertex, as desired.

Now regard $A$ as a category, and let $F: A \rightarrow (\sSet)_{/K}$ be the functor
which carries each $\alpha \in A$ to the simplicial set $K_{\alpha}$. For each
$\alpha \in A$, choose a final vertex $x_\alpha$ of $K_{\alpha}$.
Let $K_F$ be defined as in \S \ref{quasilimit1}. We claim next that there 
exists a retraction $r: K_F \rightarrow K$ with
the property that $r(X_{\alpha}) = x_{\alpha}$ for each $I \in \calI$.

The construction of $r$ proceeds as in the proof of Proposition
\ref{extet}. Namely, we well-order the finite linearly ordered subsets $B
\subseteq A$, and define $r|K'_B$ by induction on $B$.
Moreover, we will select $r$ so that it has the property that if
$B$ is nonempty with largest element $\beta$, then $r(K'_B)
\subseteq K_{\beta}$.

If $B$ is empty, then $r|K'_B=r|K$ is the identity map. Otherwise,
$B$ has a least element $\alpha$ and a largest element $\beta$. We are
required to construct a map $K_{\alpha} \star \Delta^B \rightarrow
K_{\beta}$, or a map $r_B: \Delta^B \rightarrow K_{\id|K_{\alpha}/}$, where
the values of this map on $\bd \Delta^B$ have already been
determined. If $B$ is a singleton, we define this map to carry the
vertex $\Delta^B$ to a final object of $K_{\id|K_{\alpha}/}$ lying
over $x_{\beta}$. Otherwise, we are guaranteed that {\em some}
extension exists by the fact that $r_B|\bd \Delta^B$ carries the
final vertex of $\Delta^B$ to a final object of
$K_{\id|K_{\alpha}/}$.

Now let $j: \Nerve(A) \rightarrow K$ denote the restriction
of the retraction of $r$ to $\Nerve(A)$. Using Propositions \ref{extet}
and \ref{utl}, we deduce that $j$ is a cofinal map as desired.
\end{proof}

A similar technique can be used to prove the following characterization of
$\kappa$-filtered $\infty$-categories:

\begin{proposition}\label{charfiltt}
Let $S$ be a simplicial set. The following conditions are equivalent:
\begin{itemize}
\item[$(1)$] The simplicial set $S$ is $\kappa$-filtered.
\item[$(2)$] There exists a diagram of simplicial sets $\{ Y_{\alpha} \}_{ \alpha \in \calI}$ having
colimit $Y$ and a categorical equivalence $S \rightarrow Y$, 
where each $Y_{\alpha}$ is $\kappa$-filtered and the indexing category $\calI$ is $\kappa$-filtered.
\item[$(3)$] There exists a categorical equivalence $S \rightarrow \calC$ where $\calC$ is a $\kappa$-filtered union of simplicial subsets $\calC_{\alpha} \subseteq \calC$ such that each $\calC_{\alpha}$ is an $\infty$-category with a final object.
\end{itemize}
\end{proposition}

\begin{proof}
Let $T: \sSet \rightarrow \sSet$ be the ``fibrant replacement'' functor given by
$$ T(X) = \tNerve( | \sCoNerve[X] |).$$
There is a natural transformation $j_{X}: X \rightarrow T(X)$ which is a categorical equivalence every simplicial set $X$ the map $j_{X}$ is a categorical equivalence. Moreover, each $T(X)$ is an $\infty$-category. Furthermore, the functor $T$ preserves inclusions and commutes with filtered colimits.

It is clear that $(3)$ implies $(2)$. Suppose that $(2)$ is satisfied. Replacing the diagram
$\{ Y_{\alpha} \}_{ \alpha \in \calI}$ by $\{ T(Y_{\alpha}) \}_{\alpha \in \calI}$ if necessary, we may suppose that each $Y_{\alpha}$ is an $\infty$-category. It follows that $Y$ is an $\infty$-category.
If $p: K \rightarrow Y$ is a diagram indexed by a $\kappa$-small simplicial set, then $p$ factors
through a map $p_{\alpha}: K \rightarrow Y_{\alpha}$ for some $\alpha \in \calI$, in virtue of the assumption that $\calI$ is $\kappa$-filtered. Since $Y_{\alpha}$ is a $\kappa$-filtered $\infty$-category, we can find an extension $K^{\triangleright} \rightarrow Y_{\alpha}$ of $p_{\alpha}$, hence an extension $K^{\triangleright} \rightarrow Y$ of $p$.

Now suppose that $(1)$ is satisfied. Replacing $S$ by $T(S)$ if necessary, we may suppose that $S$ is an $\infty$-category. Choose a set $X$ of cardinality $\geq \kappa$, and let $\Nerve(X)$ be defined as in the proof of Proposition \ref{rot}. The proof of Proposition \ref{rot} shows that we may write $S \times \Nerve(X)$ as a $\kappa$-filtered union of simplicial subsets $\{ Y_{\alpha} \}$, where
each $Y_{\alpha}$ has a final vertex. We now take $\calC = T(S \times \Nerve(X) )$, and let
$\calC_{\alpha} = T(Y_{\alpha})$: these choices satisfy $(3)$, which completes the proof.
\end{proof}

By definition, a $\infty$-category $\calC$ is $\kappa$-filtered if any map $p: K \rightarrow \calC$, whose source $K$ is $\kappa$-small, extends over the cone $K^{\triangleright}$. We now consider the possibility of constructing this extension uniformly in $p$. First, we need a few lemmas.

\begin{lemma}\label{stull2}
Let $\calC$ be a filtered $\infty$-category. Then $\calC$ is weakly contractible.
\end{lemma}

\begin{proof}
Since $\calC$ is filtered, it is nonempty. Fix an object $C \in \calC$. Let $| \calC | $ denote the geometric realization of $\calC$ as a simplicial set. We identify $C$ with a point of the topological space $| \calC |$. By Whitehead's theorem, to show that $\calC$ is weakly contractible, it suffices to show that for every $i \geq 0$, the homotopy set $\pi_{i}( |\calC|, C)$ consists of a single point.
If not, we can find a finite simplicial subset $K \subseteq \calC$ containing $C$ such that the map $f: \pi_{i}( |K|, C) \rightarrow \pi_{i}( | \calC|,C)$ has nontrivial image. But $\calC$ is filtered, so the inclusion $K \subseteq \calC$ factors through a map $K^{\triangleright} \rightarrow \calC$.
It follows that $f$ factors through $\pi_{i}( |K^{\triangleright}|, C)$. But this homotopy set is trivial, since $K^{\triangleright}$ is weakly contractible.
\end{proof}

\begin{lemma}\label{forfilt}
Let $\calC$ be a $\kappa$-filtered $\infty$-category, and let $p: K \rightarrow \calC$
be a diagram indexed by a $\kappa$-small simplicial set $K$. Then $\calC_{p/}$ is $\kappa$-filtered.
\end{lemma}

\begin{proof}
Let $K'$ be a $\kappa$-small simplicial set, and $p': K' \rightarrow \calC_{p/}$ a $\kappa$-small diagram. Then we may identify $p'$ with a map $q: K \star K' \rightarrow \calC$, and we get an isomorphism $( \calC_{p/} )_{p'/} \simeq \calC_{q/}$. Since $K \star K'$ is $\kappa$-small,
the $\infty$-category $\calC_{q/}$ is nonempty.
\end{proof}

\begin{proposition}\label{undertruck}
Let $\calC$ be an $\infty$-category and $\kappa$ a regular cardinal. Then $\calC$ is $\kappa$-filtered if and only if, for each $\kappa$-small simplicial set $K$, the diagonal map
$d: \calC \rightarrow \Fun(K,\calC)$ is cofinal.
\end{proposition}

\begin{proof}
Suppose first that the diagonal map $d: \calC \rightarrow \Fun(K,\calC)$ is cofinal, for any $\kappa$-small simplicial set $K$. Choose any map $j: K \rightarrow \calC$; we wish to show that $j$ can be extended to $K^{\triangleright}$. By Proposition \ref{princex}, it suffices to show that $j$ can be extended to the equivalent simplicial set $K \diamond \Delta^0$. In other words, we must produce
an object $C \in \calC$ and a morphism $j \rightarrow d(C)$ in $\Fun(K,\calC)$. It will suffice to prove that
the $\infty$-category $\calD = \calC \times_{ \Fun(K,\calC) } \Fun(K,\calC)_{j/}$ is nonempty. We now invoke Theorem \ref{hollowtt} to deduce that $\calD$ is weakly contractible.

Now suppose that $S$ is $\kappa$-filtered, and that $K$ is a $\kappa$-small simplicial set.
We wish to show that the diagonal map $d: \calC \rightarrow \Fun(K,\calC)$ is cofinal. By Theorem \ref{hollowtt}, it suffices to prove that for every object $X \in \Fun(K,\calC)$, the $\infty$-category
$\Fun(K,\calC)^{X/} \times_{\Fun(K,\calC)} \calC$ is weakly contractible. But if we identify $X$ with a map
$x: K \rightarrow \calC$, then we get a natural identification
$$ \Fun(K,\calC)^{X/} \times_{ \Fun(K,\calC) } \calC \simeq \calC^{x/},$$ which is $\kappa$-filtered
by Lemma \ref{forfilt} and therefore weakly contractible by Lemma \ref{stull2}.
\end{proof}

\subsection{Right Exactness}\label{rexex}

Let $\calA$ and $\calB$ be abelian categories. In classical homological algebra, a functor $F: \calA \rightarrow \calB$ is said to be
{\it right exact} if it is additive, and whenever
$$A' \rightarrow A \rightarrow A'' \rightarrow 0$$
is an exact sequence in $\calA$, the induced sequence
$$F(A') \rightarrow F(A) \rightarrow F(A'') \rightarrow 0$$
is exact in $\calB$.\index{gen}{right exact functor}\index{gen}{functor!right exact}

The notion of right exactness generalizes in a natural way to functors between categories which are not assumed to be abelian. Let $F: \calA \rightarrow \calB$ be a functor between abelian categories, as above. Then $F$ is additive if and only if $F$ preserves finite coproducts. Furthermore, an additive functor $F$ is right exact if and only if it preserves coequalizer diagrams. Since every finite colimit can be built out of finite coproducts and coequalizers, right exactness is equivalent to the requirement that $F$ preserves all finite colimits. This condition makes sense whenever the category $\calA$ admits finite colimits.

It is possible to generalize even further, to the case of a functor between arbitrary categories. To simplify the discussion, let us suppose that $\calB = \Set^{op}$. Then we may regard a functor $F: \calA \rightarrow \calB$ as a presheaf of sets on the category $\calA$. Using this presheaf we can define a new category $\calA_{F}$, whose objects are pairs $(A, \eta)$ where $A \in \calA$ and
$\eta \in F(A)$, and morphisms from $(A, \eta)$ to $(A', \eta')$ are maps $f: A \rightarrow A'$
such that $f^{\ast}(\eta') = \eta$, where $f^{\ast}$ denotes the induced map
$F(A') \rightarrow F(A)$. If $\calA$ admits finite colimits, then the functor $F$ preserves finite colimits if and only if the category $\calA_{F}$ is filtered.

Our goal in this section is to adapt the notion of right-exact functors to the $\infty$-categorical context. We begin with the following:

\begin{definition}\index{gen}{$\kappa$-right exact}\index{gen}{functor!$\kappa$-right exact}\label{spuss}
Let $F: \calA \rightarrow \calB$ be a functor between $\infty$-categories and $\kappa$ a regular cardinal. We will say that
$F$ is {\it $\kappa$-right exact} if, for any right fibration $\calB' \rightarrow \calB$
where $\calB'$ is $\kappa$-filtered, the $\infty$-category $\calA' = \calA \times_{\calB} \calB'$ is also $\kappa$-filtered. We will say that $F$ is {\it right exact} if it is $\omega$-right exact.
\end{definition}

\begin{remark}
We also have an evident dual notion of {\em left exact} functor.\index{gen}{left exact!functor}
\index{gen}{functor!left exact}
\end{remark}

\begin{remark}
If $\calA$ admits finite colimits, then a functor $F: \calA \rightarrow \calB$ is right exact
if and only if $F$ preserves finite colimits (see Proposition \ref{swarmmy} below).
\end{remark}

We note the following basic stability properties of $\kappa$-right exact maps.

\begin{proposition}
Let $\kappa$ be a regular cardinal.
\begin{itemize}
\item[$(1)$] If $F: \calA \rightarrow \calB$ and $G: \calB \rightarrow \calC$ are $\kappa$-right
exact functors between $\infty$-categories, then $G \circ F: \calA \rightarrow \calC$ is $\kappa$-right exact.
\item[$(2)$] Any equivalence of $\infty$-categories is $\kappa$-right exact.
\item[$(3)$] Let $F: \calA \rightarrow \calB$ be a $\kappa$-right exact functor, and let
$F': \calA \rightarrow \calB$ be homotopic to $F$. Then $F'$ is $\kappa$-right exact.
\end{itemize}
\end{proposition}

\begin{proof}
Property $(1)$ is immediate from the definition. We will establish $(2)$ and $(3)$ as a consequence of the following more general assertion: if $F: \calA \rightarrow \calB$ and $G: \calB \rightarrow \calC$ are functors such that $F$ is a categorical equivalence, then $G$ is $\kappa$-right exact if and only if $G \circ F$ is $\kappa$-right exact. To prove this, let $\calC' \rightarrow \calC$ be a right fibration. Proposition \ref{basechangefunky} implies that the induced map
$$ \calA' = \calA \times_{\calC} \calC' \rightarrow \calB \times_{\calC} \calC' = \calB'$$
is a categorical equivalence. Thus $\calA'$ is $\kappa$-filtered if and only if $\calB'$ is $\kappa$-filtered.

We now deduce $(2)$ by specializing to the case where $G$ is the identity map. To prove $(3)$,
we choose a contractible Kan complex $K$ containing a pair of vertices $\{x,y\}$ and a map $g: K \rightarrow \calB^{\calA}$ with $g(x) = F$, $g(y) = F'$. Applying the above argument to the composition
$$ \calA \simeq \calA \times \{x\} \subseteq \calA \times K \stackrel{G}{\rightarrow} \calB,$$
we deduce that $G$ is $\kappa$-right exact. Applying the converse to the diagram
$$ \calA \simeq \calA \times \{y\} \subseteq \calA \times K \stackrel{G}{\rightarrow} \calB$$
we deduce that $F'$ is $\kappa$-right exact.
\end{proof}

The next result shows that the $\kappa$-right exactness of a functor $F: \calA \rightarrow \calB$ can be tested on a very small collection of right fibrations $\calB' \rightarrow \calB$.

\begin{proposition}\label{swarmy}
Let $F: \calA \rightarrow \calB$ be a functor between $\infty$-categories and $\kappa$ a regular cardinal. The following are equivalent:
\begin{itemize}
\item[$(1)$] The functor $F$ is $\kappa$-right exact.
\item[$(2)$] For every object $B$ of $\calB$, the $\infty$-category
$ \calA \times_{\calB} \calB_{/B}$ is $\kappa$-filtered.
\end{itemize}
\end{proposition}

\begin{proof}
We observe that for every object $B \in \calB$, the $\infty$-category $\calB_{/B}$ is right-fibered
over $\calB$ and is $\kappa$-filtered (since it has a final object). Consequently, $(1)$ implies $(2)$. Now suppose that $(2)$ is satisfied. Let $T: (\sSet)_{/\calB} \rightarrow (\sSet)_{/\calB}$
denote the composite functor
$$ (\sSet)_{/\calB} \stackrel{\St_{\calB}}{\rightarrow} (\sSet)^{\sCoNerve[\calB^{op}]}
\stackrel{\Sing |\bigdot|}{\rightarrow} (\sSet)^{\sCoNerve[\calB^{op}]} \stackrel{\Un_{\calB}}{\rightarrow} (\sSet)_{/\calB}.$$
We will use the following properties of $T$:
\begin{itemize}
\item[$(i)$] There is a natural transformation $j_{X}: X \rightarrow T(X)$, where $j_{X}$ is
a contravariant equivalence in $(\sSet)_{/\calB}$ for every $X \in (\sSet)_{/\calB}$.
\item[$(ii)$] For every $X \in (\sSet)_{/\calB}$, the associated map $T(X) \rightarrow \calB$ is
a right fibration.
\item[$(iii)$] The functor $T$ commutes with filtered colimits.
\end{itemize}
We will say that an object $X \in (\sSet)_{/\calB}$ is {\it good} if the $\infty$-category
$T(X) \times_{\calB} \calA$ is $\kappa$-filtered. We now make the following observations:
\begin{itemize}
\item[$(A)$] If $X \rightarrow Y$ is a contravariant equivalence in $(\sSet)_{/\calB}$, then
$X$ is good if and only if if $Y$ is good. This follows from the fact that $T(X) \rightarrow T(Y)$
is an equivalence of right fibrations, so that the induced map $T(X) \times_{\calB} \calA
\rightarrow T(Y) \times_{\calB} \calA$ is an equivalence of right fibrations and consequently a categorical equivalence of $\infty$-categories.
\item[$(B)$] If $X \rightarrow Y$ is a categorical equivalence in $(\sSet)_{/\calB}$, then
$X$ is good if and only if $Y$ is good. This follows $(A)$, since every
categorical equivalence is a contravariant equivalence.
\item[$(C)$] The collection of good objects of $(\sSet)_{\calB}$ is stable under $\kappa$-filtered colimits. This follows from the fact that the functor $X \mapsto T(X) \times_{\calB} \calA$ commutes with $\kappa$-filtered colimits (in fact, with all filtered colimits) and Proposition \ref{charfiltt}.
\item[$(D)$] If $X \in (\sSet)_{/\calB}$ corresponds to a right fibration $X \rightarrow \calB$, then
$X$ is good if and only if $X \times_{ \calB} \calA$ is $\kappa$-filtered.
\item[$(E)$] For every object $B \in \calB$, the overcategory $\calB_{/B}$ is a good
object of $(\sSet)_{/\calB}$. In view of $(D)$, this is equivalent to the assumption $(2)$.
\item[$(F)$] If $X$ consists of a single vertex $x$, then $X$ is good. To see this, let
$B \in \calB$ denote the image of $X$. The natural map $X \rightarrow \calB_{/B}$
can be identified with the inclusion of a final vertex; this map is right anodyne and therefore
a contravariant equivalence. We now conclude by applying $(A)$ and $(E)$.
\item[$(G)$] If $X \in (\sSet)_{/\calB}$ is an $\infty$-category with a final object $x$, then
$X$ is good. To prove this, we note that $\{x\}$ is good by $(F)$ and the inclusion
$\{x\} \subseteq X$ is right anodyne, hence a contravariant equivalence. We conclude by applying $(A)$.
\item[$(H)$] If $X \in (\sSet)_{/\calB}$ is $\kappa$-filtered, then $X$ is good. To prove this, we apply
Proposition \ref{charfiltt} to deduce the existence of a categorical equivalence $i:X \rightarrow \calC$, where $\calC$ is a $\kappa$-filtered union of $\infty$-categories with final objects. Replacing $\calC$ by $\calC \times K$ if necessary, where $K$ is a contractible Kan complex, we may suppose that $i$ is a cofibration. Since $\calB$ is an $\infty$-category, the lifting problem
$$ \xymatrix{ S \ar[r] \ar[d]^{i} & \calB \\
\calC \ar@{-->}[ur] }$$
has a solution. Thus we may regard $\calC$ as an object of $(\sSet)_{/\calB}$.
According to $(B)$, it suffices to show that $\calC$ is good. But $\calC$ is a $\kappa$-filtered
colimit of good objects of $(\sSet)_{\calB}$ (by $(G)$), and is therefore itself good (by $(C)$).
\end{itemize}

Now let $\calB' \rightarrow \calB$ be a right fibration, where $\calB'$ is $\kappa$-filtered.
By $(H)$, $\calB'$ is a good object of $(\sSet)_{/\calB}$. Applying $(D)$, we deduce that
$\calA' = \calB' \times_{\calB} \calA$ is $\kappa$-filtered. This proves $(1)$.
\end{proof}

Our next goal is to prove Proposition \ref{swarmmy}, which gives a very concrete characterization of right exactness under the assumption that there is a sufficient supply of colimits. We first need a few preliminary results.

\begin{lemma}\label{devass}
Let $\calB' \rightarrow \calB$ be a Cartesian fibration. Suppose that $\calB$ has an initial object
$B$ and that $\calB'$ is filtered. Then the fiber $\calB'_{B} = \calB' \times_{\calB} \{B\}$ is a contractible Kan complex.
\end{lemma}

\begin{proof}
Since $B$ is an initial object of $\calB$, the inclusion $\{ B\}^{op} \subseteq \calB^{op}$ is cofinal. Proposition \ref{strokhop} implies that the inclusion $(\calB'_{B})^{op} \subseteq (\calB')^{op}$ is also cofinal, and therefore a weak homotopy equivalence. It now suffices to prove that $\calB'$ is weakly contractible, which follows from Lemma \ref{stull2}.
\end{proof}

\begin{lemma}\label{druv}
Let $f: \calA \rightarrow \calB$ be a right exact functor between $\infty$-categories, and let
$A \in \calA$ be an initial object. Then $f(A)$ is an initial object of $\calB$.
\end{lemma}

\begin{proof}
Let $B$ be an object of $\calB$. Proposition \ref{swarmy} implies that
$\calA' = \calB_{/B} \times_{\calB} \calA$ is filtered. We may identify
$\bHom_{\calB}(f(A), B)$ with the fiber of the right fibration $\calA' \rightarrow \calA$ over the object $A$. We now apply Lemma \ref{devass} to deduce that $\bHom_{\calB}(f(A),B)$ is contractible.
\end{proof}

\begin{lemma}\label{devic}
Let $\kappa$ be a regular cardinal, $f: \calA \rightarrow \calB$ a $\kappa$-right exact functor between $\infty$-categories, and $p: K \rightarrow \calA$ be a diagram indexed by a $\kappa$-small simplicial set $K$.
The induced map $\calA_{p/} \rightarrow \calB_{f p/}$ is $\kappa$-right exact.
\end{lemma}

\begin{proof}
According to Proposition \ref{swarmy}, it suffices to prove that for each object
$\overline{B} \in \calB_{f \circ p/}$, the $\infty$-category
$\calA' = \calA_{p/} \times_{\calB_{f p/}} ( \calB_{f p/} )_{/\overline{B}}$ is $\kappa$-filtered.
Let $B$ denote the image of $\overline{B}$ in $\calB$, and let
$q: K' \rightarrow \calA'$ be a diagram indexed by a $\kappa$-small simplicial set $K'$;
we wish to show that $q$ admits an extension to ${K'}^{\triangleright}$. We may regard $p$ and $q$ together as defining a diagram
$K \star K' \rightarrow \calA \times_{\calB} \calB_{/B}$. Since $f$ is $\kappa$-filtered,
we can extend this to a map
$(K \star K')^{\triangleright} \rightarrow \calA \times_{\calB} \calB_{/B}$, which can be identified with an extension $\overline{q}: {K'}^{\triangleright} \rightarrow \calA'$ of $q$.
\end{proof}

\begin{proposition}\label{swarmmy}\index{gen}{right exact!and colimits}
Let $f: \calA \rightarrow \calB$ be a functor between $\infty$-categories and let $\kappa$ be a regular cardinal.

\begin{itemize}
\item[$(1)$] If $f$ is $\kappa$-right exact, then $f$ preserves all $\kappa$-small colimits
which exist in $\calA$.
\item[$(2)$] Conversely, if $\calA$ admits $\kappa$-small colimits and $f$ preserves $\kappa$-small colimits, then $f$ is right exact.
\end{itemize}
\end{proposition}

\begin{proof}
Suppose first that $f$ is $\kappa$-right exact. Let $K$ be a $\kappa$-small simplicial set, and
let $\overline{p}: K^{\triangleright} \rightarrow \calA$ be a colimit of $p = \overline{p}|K$. 
We wish to show that $f \circ \overline{p}$ is a colimit diagram. Using Lemma \ref{devic}, we may replace $\calA$ by $\calA_{p/}$ and $\calB$ by $\calB_{f p/}$, and thereby reduce to the case $K = \emptyset$. We are then reduced to proving that $f$ preserves initial objects, which follows from Lemma \ref{druv}.

Now suppose that $\calA$ admits $\kappa$-small colimits, and that $f$ preserves $\kappa$-small colimits. We wish to prove that $f$ is $\kappa$-right exact. Let $B$ be an object of $\calB$ and set
$\calA' = \calA \times_{\calB} \calB_{/B}$. We wish to prove that $\calA'$ is $\kappa$-filtered.
Let $p': K \rightarrow \calA'$ be a diagram indexed by a $\kappa$-small simplicial set $K$; we wish to prove that $p'$ extends to a map $\overline{p}': K^{\triangleright} \rightarrow \calA'$. Let $p: K \rightarrow \calA$ be the composition of $p'$ with the projection $\calA' \rightarrow \calA$, and
let $\overline{p}: K^{\triangleright} \rightarrow \calA$ be a colimit of $p$. We may identify
$f \circ \overline{p}$ and $p'$ with objects of $\calB_{f p/}$. Since $f$ preserves $\kappa$-small colimits, $f \circ \overline{p}$ is an initial object of $\calB_{f p/}$, so that there exists a morphism
$\alpha: f \circ \overline{p} \rightarrow p'$ in $\calB_{f \circ p/}$. The morphism $\alpha$
can be identified with the desired extension $\overline{p}': K^{\triangleright} \rightarrow \calA'$.
\end{proof}

\begin{remark}
The results of this section all dualize in an evident way: a functor $G: \calA \rightarrow \calB$
is said to be {\it $\kappa$-left exact} if the induced functor $G^{op}: \calA^{op} \rightarrow \calB^{op}$ is $\kappa$-right exact. In the case where $\calA$ admits $\kappa$-small limits, this is equivalent to the requirement that $G$ preserves $\kappa$-small limits.
\end{remark}

\begin{remark}
Let $\calC$ be an $\infty$-category, and let $F: \calC \rightarrow \SSet^{op}$ be a functor, and let $\widetilde{\calC} \rightarrow \calC$ be the associated right fibration (the pullback of the universal right fibration $\calQ^0 \rightarrow \SSet^{op}$). If $F$ is $\kappa$-right exact, then
$\widetilde{\calC}$ is $\kappa$-filtered (since $\calQ^0$ has a final object). If
$\calC$ admits $\kappa$-small colimits, then the converse holds: if
$\widetilde{\calC}$ is $\kappa$-filtered, then $F$ preserves $\kappa$-small colimits by
Proposition \ref{geort}, and is therefore $\kappa$-right exact by Proposition \ref{swarmy}.
The converse does not hold in general: it is possible to give an example of right fibration
$\widetilde{\calC} \rightarrow \calC$ such that $\widetilde{\calC}$ is filtered, yet the classifying functor $F: \calC \rightarrow \SSet^{op}$ is not right exact.
\end{remark}

\subsection{Filtered Colimits}\label{fcolm}

Filtered categories tend not to be very interesting in themselves. Instead, they are primarily useful for indexing diagrams in other categories. 
This is because the colimits of filtered diagrams enjoy certain exactness properties which are not shared by colimits in general. In this section, we will formulate and prove these exactness properties in the $\infty$-categorical setting. First, we need a few definitions.

\begin{definition}\label{bicard}\index{gen}{$\kappa$-closed}
Let $\kappa$ be a regular cardinal. We will say that an
$\infty$-category $\calC$ is {\it $\kappa$-closed} if every diagram
$p: K \rightarrow \calC$ indexed by a $\kappa$-small simplicial set $K$
admits a colimit $\overline{p}: K^{\triangleright} \rightarrow \calC$.
\end{definition}

In a $\kappa$-closed $\infty$-category, it is possible to construct $\kappa$-small colimits
functorially. More precisely, suppose that $\calC$ is an $\infty$-category and that $K$ is a simplicial set with the property that every diagram $p: K \rightarrow \calC$ has a colimit in $\calC$. Let $\calD$ denote the full subcategory of $\Fun(K^{\triangleright}, \calC)$ spanned by the colimit diagrams. Proposition \ref{lklk} implies that the restriction functor
$\calD \rightarrow \Fun(K,\calC)$ is a trivial fibration. It therefore admits a section $s$ (which is unique up to a contractible ambiguity). Let $e: \Fun(K^{\triangleright}, \calC) \rightarrow \calC$ be the functor given by evaluation at the cone point of $K^{\triangleright}$. We will refer to the composition
$$ \Fun(K,\calC) \stackrel{s}{\rightarrow} \calD \subseteq \Fun(K^{\triangleright}, \calC) \stackrel{e}{\rightarrow} \calC$$
as a {\it colimit} functor; it associates to each diagram $p: K \rightarrow \calC$ a colimit of $p$
in $\calC$. We will generally denote colimit functors by $\colim_{K}: \Fun(K,\calC) \rightarrow \calC$.\index{gen}{colimit!functor}\index{gen}{functor!colimit}\index{not}{colimK@$\colim_{K}$}

\begin{lemma}\label{tractab}
Let $F \in \Fun(K,\SSet)$ be a {\em corepresentable} functor (that is, $F$ lies in the essential image of the Yoneda embedding $K^{op} \rightarrow \Fun(K,\SSet)$), and let $X \in \SSet$ be a colimit of $F$. Then $X$ is contractible.
\end{lemma}

\begin{proof}
Without loss of generality, we may suppose that $K$ is an $\infty$-category. Let $\widetilde{K} \rightarrow K$ be a left fibration classified by $F$. Since $F$ is corepresentable, $\widetilde{K}$ has an initial object and is therefore weakly contractible. Corollary \ref{needka} implies that
there is an isomorphism $\widetilde{K} \simeq X$ in the homotopy category $\calH$, so that $X$ is also contractible.
\end{proof}

\begin{proposition}\label{frent}\index{gen}{filtered colimit!left exactness of}
Let $\kappa$ be a regular cardinal and let $\calI$ be an $\infty$-category. The following conditions are equivalent:
\begin{itemize}
\item[$(1)$] The $\infty$-category $\calI$ is $\kappa$-filtered.
\item[$(2)$] The colimit functor $\colim_{\calI}: \Fun(\calI, \SSet) \rightarrow \SSet$
preserves $\kappa$-small limits.
\end{itemize}
\end{proposition}

\begin{proof}
Suppose that $(1)$ is satisfied. According to Proposition \ref{rot}, there exists a $\kappa$-filtered partially ordered set $A$ and a cofinal map $i: \Nerve(A) \rightarrow \SSet$. Since $i$ is cofinal, the colimit functor for $\calI$ admits a factorization
$$ \Fun(\calI,\SSet) \stackrel{i^{\ast}}{\rightarrow} \Fun(\Nerve(A), \SSet) {\rightarrow} \SSet.$$
Proposition \ref{limiteval} implies that $i^{\ast}$ preserves limits. We may therefore replace
$\calI$ by $\Nerve(A)$ and thereby reduce to the case where $\calI$ is itself the nerve of a $\kappa$-filtered partially ordered set $A$.

We note that the functor $\colim_{\calI}: \Fun(\calI, \SSet) \rightarrow \SSet$ can be characterized as the
left adjoint to the diagonal functor $\delta: \SSet \rightarrow \Fun(\calI,\SSet)$. Let $\bfA$ denote the category
of all functors from $A$ to $\sSet$; we regard $\bfA$ as a simplicial model category with respect to the {\em projective} model structure described in \S \ref{quasilimit3}. Let $\phi^{\ast}: \sSet \rightarrow \bfA$ denote the diagonal functor which associates to each simplicial set $K$ the constant functor $A \rightarrow \sSet$ with value $K$, and let $\phi_{!}$ be a left adjoint of
$\phi^{\ast}$, so that the pair $(\phi^{\ast}, \phi_{!})$ gives a Quillen adjunction between
$\bfA$ and $\sSet$. Proposition \ref{gumby444} implies that there is an equivalence of $\infty$-categories $\sNerve(\bfA^{\degree}) \rightarrow
\Fun(\calI,\SSet)$, and $\delta$ may be identified with the right derived functor of $\phi^{\ast}$. Consequently, the functor $\colim_{\calI}$ may be identified with the left derived functor of $\phi_{!}$. To prove that $\colim_{\calI}$ preserves $\kappa$-small limits, it suffices to prove that $\colim_{\calI}$ preserves fiber products and $\kappa$-small products. According to Theorem \ref{colimcomparee}, it suffices to prove that
$\phi_{!}$ preserves homotopy fiber products and $\kappa$-small homotopy products. For fiber products, this reduces to the classical assertion that if we are given a family of homotopy
Cartesian squares
$$ \xymatrix{ W_{\alpha} \ar[r] \ar[d] & X_{\alpha} \ar[d] \\
Y_{\alpha} \ar[r] & Z_{\alpha} }$$
in the category of Kan complexes, indexed by a filtered partially ordered set $A$, then the colimit square
$$ \xymatrix{ W \ar[r] \ar[d] & X \ar[d] \\
Y \ar[r] & Z }$$
is also homotopy Cartesian. The assertion regarding homotopy products is handled similarly.

Now suppose that $(2)$ is satisfied. Let $K$ be a $\kappa$-small simplicial set and
$p: K \rightarrow \calI^{op}$ a diagram; we wish to prove that $\calI^{op}_{/p}$ is nonempty.
Suppose otherwise. Let $j: \calI^{op} \rightarrow \Fun(\calI,\SSet)$ be the Yoneda embedding, let
$q = j \circ p$, and let $\overline{q}: K^{\triangleleft} \rightarrow \Fun(\calI,\SSet)$ be a limit of
$q$, and let $X \in \Fun(\calI,\SSet)$ be the image of the cone point of $K^{\triangleleft}$
under $\overline{q}$. Since $j$ is fully faithful and $\calI^{op}_{/p}$ is empty, we have
$\bHom_{ \SSet^{\calI} }(j(I), X) = \emptyset$ for each $I \in \calI$. Using Lemma \ref{repco}, we may identify $\bHom_{ \SSet^{\calI} }(j(I),X)$ with $X(I)$ in the homotopy category $\calH$ of spaces. We therefore conclude that $X$ is an initial object of $\Fun(\calI,\SSet)$. Since the functor $\colim_{\calI}: \Fun(\calI,\SSet) \rightarrow \SSet$ is a left adjoint, it preserves initial objects. We conclude that $\colim_{\calI} X$ is an initial object of $\SSet$.
On the other hand, if $\colim_{\calI}$ preserves $\kappa$-small limits, then
$\colim_{\calI} \circ \overline{q}$ exhibits $\colim_{\calI} X$ as the limit of the diagram
$\colim_{\calI} \circ q: K \rightarrow \SSet$. For each vertex $k$ in $K$, Lemmas \ref{repco} and \ref{tractab} imply that $\colim_{\calI} q(k)$ is contractible, and therefore a final object of $\SSet$. It follows that
$\colim_{\calI} X$ is also a final object of $\SSet$. This is a contradiction, since the initial object
of $\SSet$ is not final.
\end{proof}

\subsection{Compact Objects}\label{compobj}

Let $\calC$ be a category which admits filtered colimits. An object $C \in \calC$ is said to be {\it compact} if the corepresentable functor $$ \Hom_{\calC}(C, \bigdot)$$ commutes with filtered colimits.\index{gen}{compact object!of a category}

\begin{example}
Let $\calC = \Set$ be the category of sets. An object $C \in \calC$ is compact if and only if
is finite.
\end{example}

\begin{example}\label{compactgroup}
Let $\calC$ be the category of groups. An object $G$ of $\calC$ is compact if and only if 
it is finitely presented (as a group).
\end{example}

\begin{example}\label{compactspacee}
Let $X$ be a topological space, and let $\calC$ be the category of open sets of $X$ (with morphisms given by inclusions). Then an object $U \in \calC$ is compact if and only if $U$ is compact when viewed as a topological space: that is, every open cover of $U$ admits a finite subcover.
\end{example}

\begin{remark}
Because of Example \ref{compactgroup}, many authors call an object $C$ of a category $\calC$ {\it finitely presented} if $\Hom_{\calC}(C, \bigdot)$ preserves filtered colimits. Our terminology is motivated instead by Example \ref{compactspacee}.
\end{remark}

\begin{definition}\label{kcontdef}\index{gen}{functor!$\kappa$-continuous}\index{gen}{functor!continuous}\index{gen}{continuous functor}\index{gen}{$\kappa$-continuous functor}
Let $\calC$ be an $\infty$-category which admits small, $\kappa$-filtered colimits. We will say a functor $f: \calC \rightarrow \calD$ is {\it $\kappa$-continuous} if it preserves $\kappa$-filtered colimits.

Let $\calC$ be an $\infty$-category containing an object $C$, and let
$j_{C}: \calC \rightarrow \hat{\SSet}$ denote the functor corepresented by $C$.
If $\calC$ admits $\kappa$-filtered colimits, then we will say that $C$ is {\it $\kappa$-compact}
if $j_{C}$ is $\kappa$-continuous. We will say that $C$ is {\it compact} if
it is $\omega$-compact (and $\calC$ admits filtered colimits).\index{gen}{compact object!of an $\infty$-category}\index{gen}{$\kappa$-compact!object}

Let $\kappa$ be a regular cardinal, and let $\calC$ be an $\infty$-category which admits small, $\kappa$-filtered colimits. We will say that a left fibration $\widetilde{\calC} \rightarrow \calC$
is {\it $\kappa$-compact} if it is classified by a $\kappa$-continuous functor
$\calC \rightarrow \hat{\SSet}$.\index{gen}{$\kappa$-compact!left fibration}
\end{definition}

\begin{notation}
Let $\calC$ be an $\infty$-category and $\kappa$ a regular cardinal. We will generally let
$\calC^{\kappa}$ denote the full subcategory spanned by the $\kappa$-compact objects of $\calC$. \index{not}{calCkappa@$\calC^{\kappa}$}
\end{notation}

\begin{lemma}\label{misst}
Let $\calC$ be an $\infty$-category which admits small $\kappa$-filtered colimits,
and let $\calD \subseteq \Fun(\calC, \widehat{\SSet})$ be the full subcategory spanned by the $\kappa$-continuous functors $f: \calC \rightarrow \hat{\SSet}$. Then $\calD$ is stable under $\kappa$-small limits in $\hat{\SSet}^{\calC}$.
\end{lemma}

\begin{proof}
Let $K$ be a $\kappa$-small simplicial set, and let $p: K \rightarrow \Fun(\calC, \widehat{\SSet})$ be a diagram, which we may identify with a map $p': \calC \rightarrow \Fun(K, \widehat{\SSet})$. Using Proposition \ref{limiteval}, we may obtain a limit of the diagram $p$ by composing $p'$ with a limit functor
$$\varprojlim: \Fun(K,\widehat{\SSet}) \rightarrow \widehat{\SSet}$$
(that is, a right adjoint to the diagonal functor $\widehat{\SSet} \rightarrow \Fun(K,\widehat{\SSet})$; see \S \ref{fcolm}). It therefore suffices to show that the functor $\varprojlim$ is $\kappa$-continuous. This is simply a reformulation of Proposition \ref{frent}.
\end{proof}

The basic properties of $\kappa$-compact left fibrations are summarized in the following Lemma::

\begin{lemma}\label{hardstuff0}
Let $\kappa$ be a regular cardinal.

\begin{itemize}
\item[$(1)$] Let $\calC$ be an $\infty$-category which admits small, $\kappa$-filtered colimits, and let $C \in \calC$ be an object. Then $C$ is $\kappa$-compact if and only if the left fibration $\calC_{C/} \rightarrow \calC$ is $\kappa$-compact.

\item[$(2)$] Let $f: \calC \rightarrow \calD$ be a $\kappa$-continuous functor between
$\infty$-categories which admit small, $\kappa$-filtered colimits, and let
$\widetilde{\calD} \rightarrow \calD$ be a $\kappa$-compact left fibration. Then
the associated left fibration $\calC \times_{\calD} \widetilde{\calD} \rightarrow \calC$
is also $\kappa$-compact.

\item[$(3)$] Let $\calC$ be an $\infty$-category which admits small, $\kappa$-filtered colimits, and let $\bfA \subseteq (\sSet)_{/\calC}$ denote the full subcategory spanned by the
$\kappa$-compact left fibrations over $\calC$. Then $\bfA$ is stable under
$\kappa$-small homotopy limits (with respect to the covariant model structure on $(\sSet)_{/\calC}$. 
In particular, $\bfA$ is stable under the formation homotopy pullbacks, $\kappa$-small products, and $($ if $\kappa$ is uncountable $)$ the formation of homotopy inverse limits of towers.
\end{itemize}
\end{lemma}

\begin{proof}
Assertions $(1)$ and $(2)$ are obvious. To prove $(3)$, let us suppose that
$\widetilde{\calC}$ is a $\kappa$-small homotopy limit of $\kappa$-compact
left fibrations $\widetilde{\calC}_{\alpha} \rightarrow \calC$. Let
$\calJ$ be a small, $\kappa$-filtered $\infty$-category, and 
$\overline{p}: \calJ^{\triangleright} \rightarrow \calC$ a colimit diagram.
We wish to prove that the composition of $\overline{p}$ with the functor
$\calC \rightarrow \hat{\SSet}$ classifying $\widetilde{\calC}$ is a colimit diagram.
Applying Proposition \ref{rot}, we may reduce to the case where $\calJ$ is the nerve
of a $\kappa$-filtered partially ordered set $A$. According to Theorem \ref{struns},
it will suffice to show that the collection of homotopy colimit diagrams
$$ A \cup \{\infty \} \rightarrow \Kan$$
is stable under $\kappa$-small homotopy limits in the diagram category $(\sSet)^{A \cup \{\infty\} }$, which follows easily from our assumption that $A$ is $\kappa$-filtered.
\end{proof}

Our next goal is to prove a very useful stability result for $\kappa$-compact objects (Proposition \ref{placeabovee}). We first need to establish a few technical lemmas.

\begin{lemma}\label{hardstuff1}
Let $\kappa$ be a regular cardinal, let $\calC$ be an $\infty$-category which admits small, $\kappa$-filtered colimits, and let $f: C \rightarrow D$ be a morphism in $\calC$. Suppose that $C$ and $D$ are $\kappa$-compact objects of $\calC$. Then $f$ is a $\kappa$-compact object of
$\Fun(\Delta^1, \calC)$.
\end{lemma}

\begin{proof}
Let $X = \Fun(\Delta^1, \calC) \times_{ \Fun(\{1\}, \calC) } \calC_{f/}$,
$Y = \Fun(\Delta^1, \calC_{C/})$, and $Z = \Fun(\Delta^1, \calC) \times_{ \Fun(\{1\}, \calC) } \calC_{C/},$
so that we have a (homotopy) pullback diagram
$$ \xymatrix{ \Fun(\Delta^1, \calC)_{f/} \ar[r] \ar[d] & X \ar[d] \\
Y \ar[r] & Z }$$
of left fibrations over $\Fun(\Delta^1, \calC)$.
According to Lemma \ref{hardstuff0}, it will suffice to show that $X$, $Y$, and $Z$
are $\kappa$-compact left fibrations. To show that $X$ is a $\kappa$-compact left fibration, it suffices to show that $\calC_{f/} \rightarrow \calC$ is a $\kappa$-compact left fibration, which
follows since we have a trivial fibration $\calC_{f/} \rightarrow \calC_{D/}$, where $D$ is $\kappa$-compact by assumption. Similarly, we have a trivial fibration
$Y \rightarrow \Fun(\Delta^1, \calC) \times_{ \calC^{(0)}} \calC_{C/}$, so that the 
$\kappa$-compactness of $C$ implies that $Y$ is a $\kappa$-compact left fibration. Lemma \ref{hardstuff0} and the compactness of $C$ immediately imply that $Z$ is a $\kappa$-compact left fibration, which completes the proof.
\end{proof}

\begin{lemma}\label{hardstuff2}
Let $\kappa$ be a regular cardinal, and let $\{ \calC_{\alpha} \}$ be a $\kappa$-small family of $\infty$-categories having product $\calC$. Suppose that each $\calC$ admits small, $\kappa$-filtered colimits. Then:
\begin{itemize}
\item[$(1)$] The $\infty$-category $\calC$ admits $\kappa$-filtered colimits.

\item[$(2)$] If $C \in \calC$ is an object whose image in each $\calC_{\alpha}$ is $\kappa$-compact, then $C$ is $\kappa$-compact as an object of $\calC$.
\end{itemize}
\end{lemma}

\begin{proof}
The first assertion is obvious, since colimits in a product can be computed pointwise. For the second, choose an object $C \in \calC$ whose images $\{ C_{\alpha} \in \calC_{\alpha} \}$ are $\kappa$-compact. 

The left fibration $\calC_{C/} \rightarrow \calC$ can obtained as a $\kappa$-small product of the left fibrations
$\calC \times_{\calC_{\alpha}} (\calC_{\alpha})_{C_{\alpha}/} \rightarrow \calC$. Lemma \ref{hardstuff0} implies that each factor is $\kappa$-compact, so that the product is also $\kappa$-compact.
\end{proof}

\begin{lemma}\label{showtop}
Let $S$ be a simplicial set, and suppose given a tower
$$ \ldots X(1) \stackrel{f_1}{\rightarrow} X(0) \stackrel{f_0}{\rightarrow} S$$
where each $f_i$ is a left fibration. Then the inverse limit
$X(\infty)$ is a homotopy inverse limit of the tower $\{ X(i) \}$ with respect to the covariant model structure on $(\sSet)_{/S}$.
\end{lemma}

\begin{proof}
Construct a ladder
$$ \xymatrix{ \ldots \ar[r] & X(1) \ar[r]^{f_1} \ar[d] & X(0) \ar[r]^{f_0} \ar[d] & S \ar[d] \\
\ldots \ar[r] & X'(1) \ar[r]^{f'_1} & X'(0) \ar[r]^{f'_0} & S }$$
where the vertical maps are covariant equivalences and the tower
$\{ X'(i) \}$ is fibrant, in the sense that each of the maps $f'_i$ is a covariant fibration.
We wish to show that the induced map on inverse limits $X(\infty) \rightarrow X'(\infty)$
is a covariant equivalence. Since both $X(\infty)$ and $X'(\infty)$ are left-fibered over $S$,
this can be tested by passing to the fibers over each vertex $s$ of $S$. We may therefore reduce to the case where $S$ is a point, in which case the tower $\{ X(i) \}$ is already fibrant (since
a left fibration over a Kan complex is a Kan fibration; see Lemma \ref{toothie2}).
\end{proof}

\begin{lemma}\label{hardstuff3}
Let $\kappa$ be an uncountable regular cardinal, and let
$$ \ldots \rightarrow \calC^2 \stackrel{f_2}{\rightarrow} \calC^1 \stackrel{f_1}{\rightarrow} \calC^0$$ be a tower of $\infty$-categories. Suppose that each $\calC^i$ admits small $\kappa$-filtered colimits, and that each of the functors $f_i$ is a categorical fibration which preserves $\kappa$-filtered colimits.
Let $\calC$ denote the inverse limit of the tower. Then:
\begin{itemize}
\item[$(1)$] The $\infty$-category $\calC$ admits small $\kappa$-filtered colimits, and the projections
$p_n: \calC \rightarrow \calC^n$ are $\kappa$-continuous.

\item[$(2)$] If $C \in \calC$ has $\kappa$-compact image in $\calC^i$ for each $i \geq 0$, then
$C$ is a $\kappa$-compact object of $\calC$.
\end{itemize}
\end{lemma}

\begin{proof}
Let $\overline{q}: K^{\triangleright} \rightarrow \calC$ be a diagram indexed by an arbitrary simplicial set, let $q = \overline{q}|K$, and set $\overline{q}_n = p_n \circ \overline{q}$, 
$q_n = p_n \circ q$. Suppose that each $\overline{q}_n$ is a colimit diagram in $\calC^n$.
Then the map $\calC_{\overline{q}/} \rightarrow \calC_{q/}$ is the inverse limit of a tower of trivial fibrations $\calC^n_{\overline{q}_n/} \rightarrow \calC^n_{q_n/}$, and therefore a a trivial fibration.

To complete the proof of $(1)$, it will suffice to show that if $K$ is a $\kappa$-filtered $\infty$-category, then any diagram $q: K \rightarrow \calC$ can be extended to a map
$\overline{q}: K^{\triangleright} \rightarrow \calC$ with the property described above.
To construct $\overline{q}$, it suffices to construct a compatible family $\overline{q}_n:
K^{\triangleright} \rightarrow \calC^n$. We begin by selecting arbitrary colimit diagrams
$\overline{q}'_n: K^{\triangleright} \rightarrow \calC^n$ which extend $q_n$. 
We now explain how to adjust these choices to make them compatible with one another, using induction on $n$. Set $\overline{q}_0 = \overline{q}'_0$. Suppose next that $n > 0$.
Since $f_n$ preserves $\kappa$-filtered colimits, we may identify
$\overline{q}_{n-1}$ and $f_n \circ \overline{q}'_{n}$ with initial objects of
$\calC^{n-1}_{q_{n-1}/}$. It follows that there exists an equivalence
$e: \overline{q}_{n-1} \rightarrow f_n \circ \overline{q}'_{n}$ in $\calC^{n-1}_{q_{n-1}/}$. The map
$f_n$ induces a categorical fibration $\calC^n_{q_n/} \rightarrow \calC^{n-1}_{q_{n-1}/}$, so that
$e$ lifts to an equivalence $\overline{e}: \overline{q}_n \rightarrow \overline{q}'_n$ in
$\calC^n_{q_n/}$. The existence of the equivalence $\overline{e}$ proves that
$\overline{q}_n$ is a colimit diagram in $\calC^n$, and we have
$\overline{q}_{n-1} = f_n \circ \overline{q}_n$ by construction. This proves $(1)$.

Now suppose that $C \in \calC$ is as in $(2)$, and let $C^n = p_n(C) \in \calC^n$. 
The left fibration $\calC_{/C}$ is the inverse limit of a tower of left fibrations
$$ \ldots \rightarrow \calC^1_{C^1/} \times_{\calC^1} \calC \rightarrow \calC^0_{C^0/} \times_{\calC^0} \calC.$$
Using Lemma \ref{hardstuff0}, we deduce that each term in this tower is a $\kappa$-compact left fibration over $\calC$. Proposition \ref{sharpen} implies that each map in the tower is a left fibration, so that $\calC_{C/}$ is a homotopy inverse limit of a tower of $\kappa$-compact left fibrations, by Lemma \ref{showtop}. We now apply Lemma \ref{hardstuff0} again to deduce that
$\calC_{C/}$ is a $\kappa$-compact left fibration, so that $C \in \calC$ is $\kappa$-compact as desired.
\end{proof}

\begin{proposition}\label{placeabovee}\index{gen}{compact object!of a functor $\infty$-category}
Let $\kappa$ be a regular cardinal, let $\calC$ be an $\infty$-category which admits
small $\kappa$-filtered colimits, and let $f: K \rightarrow \calC$ be a diagram indexed by a $\kappa$-small simplicial set $K$. Suppose that for each vertex $x$ of $K$, $f(x) \in \calC$ is $\kappa$-compact. Then $f$ is a $\kappa$-compact object of $\Fun(K,\calC)$.
\end{proposition}

\begin{proof}
Let us say that a simplicial set $K$ is {\em good} if it satisfies the conclusions of the lemma.
We wish to prove that all $\kappa$-small simplicial sets are good. The proof proceeds in several steps:

\begin{itemize}
\item[$(1)$] Given a pushout square
$$ \xymatrix{ K' \ar[r] \ar[d]^{i} & K \ar[d] \\
L' \ar[r] & L }$$
where $i$ is a cofibration and the simplicial sets $K'$, $K$, and $L'$ are good, the simplicial
set $L$ is also good. To prove this, we observe that the associated diagram of $\infty$-categories
$$ \xymatrix{ \Fun(L,\calC) \ar[r] \ar[d] & \Fun(L',\calC) \ar[d] \\
\Fun(K,\calC) \ar[r] & \Fun(K',\calC) }$$ is homotopy Cartesian, and every arrow in the diagram preserves
$\kappa$-filtered colimits (by Proposition \ref{limiteval}). Now apply Lemma \ref{yoris}.

\item[$(2)$] If $K \rightarrow K'$ is a categorical equivalence and $K$ is good, then $K'$ is good:
the forgetful functor $\Fun(K',\calC) \rightarrow \Fun(K,\calC)$ is an equivalence of $\infty$-categories, and therefore detects $\kappa$-compact objects.

\item[$(3)$] Every simplex $\Delta^n$ is good. To prove this, we observe that the inclusion
$$ \Delta^{ \{0,1\} } \coprod_{ \{1\} } \ldots \coprod_{ \{n-1\} } \Delta^{ \{n-1,n\}} \subseteq \Delta^n$$
is a categorical equivalence. Applying $(1)$ and $(2)$, we can reduce to the case $n \leq 1$.
If $n=0$ there is nothing to prove, and if $n = 1$ we apply Lemma \ref{hardstuff1}.

\item[$(4)$] If $\{ K_{\alpha} \}$ is a $\kappa$-small collection of good simplicial sets having coproduct $K$, then $K$ is also good. To prove this, we observe that
$\Fun(\calC) \simeq \prod_{\alpha} \Fun(K_{\alpha}, \calC)$ and apply Lemma \ref{hardstuff2}.

\item[$(5)$] If $K$ is a $\kappa$-small simplicial set of dimension $\leq n$, then $K$ is good. The proof is by induction on $n$. Let
$K^{(n-1)} \subseteq K$ denote the $(n-1)$-skeleton of $K$, so that we have a pushout diagram
$$ \xymatrix{ \coprod_{ \sigma \in K_n} \bd \Delta^n \ar[r] \ar[d] & K^{(n-1)} \ar[d] \\
\coprod_{\sigma \in K_n} \Delta^n \ar[r] & K.}$$
The inductive hypothesis implies that $\coprod_{\sigma \in K_n} \bd \Delta^n$ and $K^{(n-1)}$
are good. Applying $(3)$ and $(4)$, we deduce that $\coprod_{\sigma \in K_n} \Delta^n$ is good. We now apply $(1)$ to deduce that $K$ is good.

\item[$(6)$] Every $\kappa$-small simplicial set $K$ is good. If $\kappa = \omega$, then this
follows immediately from $(5)$, since every $\kappa$-small simplicial set is finite dimensional.
If $\kappa$ is uncountable, then we have an increasing filtration
$$ K^{(0)} \subseteq K^{(1)} \subseteq \ldots $$
which gives rise to a tower of $\infty$-categories
$$ \ldots \Fun(K^{(1)}, \calC) \rightarrow \Fun(K^{(0)}, \calC) $$
having (homotopy) inverse limit $\Fun(K,\calC)$. Using Proposition \ref{limiteval}, we deduce that the hypotheses of Lemma \ref{hardstuff3} are satisfied, so that $K$ is good.
\end{itemize}
\end{proof}

\begin{corollary}\label{jurman}
Let $\kappa$ be a regular cardinal, and let $\calC$ be an $\infty$-category which admits
small, $\kappa$-filtered colimits. Suppose that $p: K \rightarrow \calC$ is a $\kappa$-small diagram with the property that for every vertex $x$ of $K$, $p(x)$ is a $\kappa$-compact object of $\calC$.
Then the left fibration $\calC_{p/} \rightarrow \calC$ is $\kappa$-compact.
\end{corollary}

\begin{proof}
It will suffice to show that the equivalent left fibration $\calC^{p/ } \rightarrow \calC$
is $\kappa$-compact. Let $P$ be the object of $\Fun(K,\calC)$ corresponding to $p$. Then
we have an isomorphism of simplicial sets $$\calC^{p/} \simeq \calC \times_{ \Fun(K,\calC) } \Fun(K,\calC)^{P/}.$$ Proposition \ref{placeabovee} asserts that $P$ is a $\kappa$-compact object
of $\Fun(K,\calC)$, so that the left fibration $$\Fun(K,\calC)^{P/} \rightarrow \Fun(K,\calC)$$ is $\kappa$-compact.
Proposition \ref{limiteval} implies that the diagonal map $\calC \rightarrow \Fun(K,\calC)$ preserves $\kappa$-filtered colimits, so we can apply part $(2)$ of Lemma \ref{hardstuff0} to deduce that
$\calC^{p/} \rightarrow \calC$ is $\kappa$-compact, as well.
\end{proof}

\begin{corollary}\label{tyrmyrr}
Let $\calC$ be an $\infty$-category which admits small, $\kappa$-filtered colimits, and let
$\calC^{\kappa}$ denote the full subcategory of $\calC$ spanned by the $\kappa$-compact objects. Then $\calC^{\kappa}$ is stable under the formation of all $\kappa$-small colimits which exist in $\calC$.
\end{corollary}

\begin{proof}
Let $K$ be a $\kappa$-small simplicial set, and let $\overline{p}: K^{\triangleright} \rightarrow \calC$ be a colimit diagram. Suppose that, for each vertex $x$ of $K$, the object $\overline{p}(x) \in \calC$ is $\kappa$-compact. We wish to show that $C= \overline{p}(\infty) \in \calC$ is $\kappa$-compact, where $\infty$ denotes the cone point of $K^{\triangleright}$. Let $p = \overline{p}|K$, and consider the maps
$$ \calC_{p/} \leftarrow \calC_{ \overline{p}/ } \rightarrow \calC_{ C/}. $$
Both are trivial fibrations (the first because $\overline{p}$ is a colimit diagram, and the second because the inclusion $\{ \infty\} \subseteq K^{\triangleright}$ is right anodyne). Corollary \ref{jurman} asserts that the left fibration $\calC_{p/} \rightarrow \calC$ is $\kappa$-compact. It follows that the equivalent left fibration $\calC_{C/}$ is $\kappa$-compact, so that $C$ is a $\kappa$-compact object of $\calC$ as desired.
\end{proof}

\begin{remark}\label{trmyr}
Let $\kappa$ be a regular cardinal, and let $\calC$ be an $\infty$-category which admits $\kappa$-filtered colimits. Then the full subcategory $\calC^{\kappa} \subseteq \calC$ of $\kappa$-compact objects is stable under retracts. If $\kappa > \omega$, this follows from 
Proposition \ref{slanger} and Corollary \ref{tyrmyrr} (since every retract can be obtained as a $\kappa$-small colimit). We give an alternative argument that works also in the most important case $\kappa = \omega$. Let $C$ be $\kappa$-compact, and let $D$ be a retract of $C$. Let
$j: \calC^{op} \rightarrow \Fun(\calC, \widehat{\SSet})$ be the Yoneda embedding. Then
$j(D) \in \Fun(\calC, \widehat{\SSet})$ is a retract of $j(C)$. Since $j(C)$ preserves $\kappa$-filtered colimits, then Lemma \ref{compcompcomp} implies that $j(D)$ preserves $\kappa$-filtered colimits, so that $D$ is $\kappa$-compact.
\end{remark}

The following result gives a convenient description of the compact objects of an $\infty$-category of presheaves:

\begin{proposition}\label{charsmallpre}
Let $\calC$ be a small $\infty$-category, $\kappa$ a regular cardinal, and 
$C \in \calP(\calC)$ an object. The following are equivalent:
\begin{itemize}
\item[$(1)$] There exists a diagram $p: K \rightarrow \calC$ indexed by a $\kappa$-small
simplicial set, such that $j \circ p$ has a colimit $D$ in $\calP(\calC)$, and $C$ is a retract of $D$.
\item[$(2)$] The object $C$ is $\kappa$-compact.
\end{itemize}
\end{proposition}

\begin{proof}
Proposition \ref{dda} asserts that for every object $A \in \calC$, $j(A)$ is completely compact, and in particular $\kappa$-compact. According to Corollary \ref{tyrmyrr} and Remark \ref{trmyr}, the collection of $\kappa$-compact objects of $\calP(\calC)$ is stable under $\kappa$-small colimits and retracts. Consequently, $(1) \Rightarrow (2)$.

Now suppose that $(2)$ is satisfied. Let $\calC_{/C} = \calC \times_{\calP(\calC)} \calP(\calC)_{/C}$.
Lemma \ref{longwait0} implies that the composition
$$ \overline{p}: \calC_{/C}^{\triangleright} \rightarrow \calP(S)_{/C}^{\triangleright} \rightarrow \calP(S)$$ is a colimit diagram. As in the proof of Corollary \ref{uterrr}, we can write $C$ as the colimit
of a $\kappa$-filtered diagram $q: \calI \rightarrow \calP(\calC)$, where each object
$q(I)$ is the colimit of $\overline{p} | \calC^{0}$, where $\calC^{0}$ is a $\kappa$-small
simplicial subset of $\calC_{/C}$. Since $C$ is $\kappa$-compact, we may argue as in the proof of Proposition \ref{dda} to deduce that $C$ is a retract of $q(I)$, for some object $I \in \calI$.
This proves $(1)$.
\end{proof}

We close with a result which we will need in \S \ref{c5s6}. First, a bit of notation: if $\calC$ is a small $\infty$-category and $\kappa$ a regular cardinal, we let $\calP^{\kappa}(\calC)$ denote the full subcategory consisting of $\kappa$-compact objects of $\calP(\calC)$.

\begin{proposition}\label{kcolim}
Let $\calC$ be a small, idempotent complete $\infty$-category and $\kappa$ a regular cardinal. The following conditions are equivalent:
\begin{itemize}
\item[$(1)$] The $\infty$-category $\calC$ admits $\kappa$-small colimits.
\item[$(2)$] The Yoneda embedding $j: \calC \rightarrow \calP^{\kappa}(\calC)$ has a left adjoint.
\end{itemize}
\end{proposition}

\begin{proof}
Suppose that $(1)$ is satisfied. For each object
$M \in \calP(\calC)$, let $F_M: \calP(\calC) \rightarrow \hat{\SSet}$ denote the associated corepresentable functor. Let $\calD \subseteq \calP(\calC)$
denote the full subcategory of $\calP(\calC)$ spanned by those objects $M$
such that $F_M \circ j: \calC \rightarrow \hat{\SSet}$ is corepresentable.
According to Proposition \ref{limiteval}, composition with $j$ induces a {\em limit-preserving} functor
$$ \Fun(\calP(\calC), \widehat{\SSet}) \rightarrow \Fun(\calC, \widehat{\SSet}).$$
Applying Proposition \ref{yonedaprop} to $\calC^{op}$, we conclude that the collection
of corepresentable functors on $\calC$ is stable under retracts and $\kappa$-small limits.
A second application of Proposition \ref{yonedaprop} (this time to 
$\calP(\calC)^{op}$) now shows that $\calD$ is stable under retracts
and $\kappa$-small colimits in $\calP(\calC)$. Since $j$ is fully faithful, $\calD$ contains the essential image of $j$. It follows from Proposition \ref{charsmallpre} that $\calD$ contains
$\calP^{\kappa}(\calC)$. We now apply Proposition \ref{adjfuncbaby} to deduce that
$j: \calC \rightarrow \calP^{\kappa}(\calC)$ admits a left adjoint.

Conversely, suppose that $(2)$ is satisfied. Let $L$ denote a left adjoint to the Yoneda embedding, let
$p: K \rightarrow \calC$ be a $\kappa$-small diagram, and let
$q = j \circ p$. Using Corollary \ref{tyrmyrr}, we deduce that $q$ has a colimit
$\overline{q}: K^{\triangleright} \rightarrow \calP^{\kappa}(\calC)$. Since $L$ is a left adjoint,
$L \circ \overline{q}$ is a colimit of $L \circ q$. Since $j$ is fully faithful, the diagram
$p$ is equivalent to $L \circ q$, so that $p$ has a colimit as well.
\end{proof}

\subsection{Ind-Objects}\label{indlim}

Let $S$ be a simplicial set. In \S \ref{presheaf4}, we proved that the $\infty$-category
$\calP(S)$ is {\em freely} generated under small colimits by the image of the Yoneda embedding $j: S \rightarrow \calP(S)$ (Theorem \ref{charpresheaf}). Our goal in this section is to study
the an analogous construction, where we allow only {\em filtered} colimits.

\begin{definition}\label{indsmall}\index{not}{IndC@$\Ind(\calC)$}\index{not}{IndkC@$\Ind_{\kappa}(\calC)$}\index{gen}{$\infty$-category!of $\Ind$-objects}
Let $\calC$ be a small $\infty$-category and let $\kappa$ be a regular cardinal. 
We let $\Ind_{\kappa}(\calC)$ denote the full subcategory of $\calP(\calC)$ spanned by those functors $f: \calC^{op} \rightarrow \SSet$ which classify right fibrations
$\widetilde{\calC} \rightarrow \calC$ where the $\infty$-category
$\widetilde{\calC}$ is $\kappa$-filtered. In the case where $\kappa = \omega$, we will simply write $\Ind(\calC)$ for $\Ind_{\kappa}(\calC)$. We will refer to $\Ind(\calC)$ as the $\infty$-category of {\it $\Ind$-objects} of $\calC$.\index{gen}{Ind-object!of an $\infty$-category}
\end{definition}

\begin{remark}\label{repareind}
Let $\calC$ be a small $\infty$-category and $\kappa$ a regular cardinal. Then the Yoneda embedding $j: \calC \rightarrow \calP(\calC)$ factors through $\Ind_{\kappa}(\calC)$. This follows immediately from Lemma \ref{repco}, since $j(C)$ classifies the right fibration $\calC_{/C} \rightarrow \calC$. The $\infty$-category $\calC_{/C}$ has a final object and is therefore $\kappa$-filtered (Proposition \ref{smallity}).
\end{remark}

\begin{proposition}\label{geort}
Let $\calC$ be a small $\infty$-category, and let $\kappa$ be a regular cardinal.
The full subcategory $\Ind_{\kappa}(\calC) \subseteq \calP(\calC)$ is stable under $\kappa$-filtered colimits.
\end{proposition}

\begin{proof}
Let $\calP'_{\Delta}(\calC)$ denote the full subcategory of $(\sSet)_{/\calC}$ spanned
by the right fibrations $\widetilde{\calC} \rightarrow \calC$. According to Proposition \ref{othermod},
the $\infty$-category $\calP(\calC)$ is equivalent to the simplicial nerve $\sNerve (\calP'_{\Delta}(\calC))$. Let $\Ind'_{\kappa}(\calC)$ denote the full subcategory of
$\calP'_{\Delta}(\calC)$ spanned by right fibrations $\widetilde{\calC} \rightarrow \calC$ where
$\widetilde{\calC}$ is $\kappa$-filtered. It will suffice to prove that for any diagram
$p: \calI \rightarrow \sNerve(\Ind'_{\Delta}(\calC))$ indexed by a small $\kappa$-filtered $\infty$-category $\calI$, the colimit of $p$ in $\sNerve(\calP'_{\Delta}(\calC))$ also belongs to
$\Ind'_{\kappa}(\calC)$. Using Proposition \ref{rot}, we may reduce to the case where $\calI$ is the nerve of a $\kappa$-filtered partially ordered set $A$. Using Proposition \ref{gumby444}, we may further reduce to the case where $p$ is the simplicial nerve of a diagram taking values in the ordinary category $\Ind'_{\kappa}(\calC)$. In virtue of Theorem \ref{colimcomparee}, it will suffice to prove that $\Ind'_{\kappa}(\calC) \subseteq \calP'_{\Delta}(\calC)$ is stable under $\kappa$-filtered homotopy colimits. We may identify $\calP'_{\Delta}$ with the collection of fibrant objects of
$(\sSet)_{/\calC}$ with respect to the contravariant model structure. Since the class of contravariant equivalences is stable under filtered colimits, any $\kappa$-filtered colimit in
$(\sSet)_{/\calC}$ is also a homotopy colimit. Consequently, it will suffice to prove that
$\Ind'_{\kappa}(\calC) \subseteq \calP'_{\Delta}(\calC)$ is stable under $\kappa$-filtered colimits.
This follows immediately from the definition of a $\kappa$-filtered $\infty$-category.
\end{proof}

\begin{corollary}\label{indpr}
Let $\calC$ be a small $\infty$-category, let $\kappa$ be a regular cardinal, and let
$F: \calC^{op} \rightarrow \SSet$ be an object of $\calP(\calC)$. The following conditions are equivalent:

\begin{itemize}
\item[$(1)$] There exists a $($small$)$ $\kappa$-filtered $\infty$-category $\calI$, a diagram
$p: \calI \rightarrow \calC$ such that $F$ is a colimit of the composition
$j \circ p: \calI \rightarrow \calP(\calC)$.

\item[$(2)$] The functor $F$ belongs to $\Ind_{\kappa}(\calC)$.

\end{itemize}

If $\calC$ admits $\kappa$-small colimits, then $(1)$ and $(2)$ are equivalent to
\begin{itemize}

\item[$(3)$] The functor $F$ preserves $\kappa$-small limits.
\end{itemize}
\end{corollary}

\begin{proof}
Lemma \ref{longwait0} implies that $F$ is a colimit of the diagram
$$ \calC_{/F} \rightarrow \calC \stackrel{j}{\rightarrow} \calP(\calC),$$
and Lemma \ref{repco} allows us to identify $\calC_{/F} = \calC \times_{\calP(\calC)} \calP(\calC)_{/F}$ with the right fibration associated to $F$. Thus $(2) \Rightarrow (1)$. The converse follows from Proposition \ref{geort}, since every representable functor belongs to $\Ind_{\kappa}(\calC)$ (Remark \ref{repareind}).

Now suppose that $\calC$ admits $\kappa$-small colimits. If $(3)$ is satisfied, then
$F^{op}: \calC \rightarrow \SSet^{op}$ is $\kappa$-right exact by Proposition \ref{frent}. The right fibration associated to $F$ is the pullback of the universal right fibration by $F^{op}$. Using Corollary \ref{grt}, the universal right fibration over $\SSet^{op}$ is representable by the final object of $\SSet$. Since $F$ is $\kappa$-right exact, the fiber product
$(\SSet^{op})_{/\ast} \times_{\SSet^{op}} \calC$ is $\kappa$-filtered. Thus $(3) \Rightarrow (2)$.

We now complete the proof by showing that $(1) \Rightarrow (3)$. First suppose that
$F$ lies in the essential image of the Yoneda embedding $j: \calC \rightarrow \calP(\calC)$. According to Lemma \ref{repco}, $j(C)$ is equivalent to the composition of the opposite Yoneda embedding $j': \calC^{op} \rightarrow \Fun(\calC, \SSet)$ with the evaluation functor
$e: \Fun(\calC, \SSet) \rightarrow \SSet$ associated to the object $C \in \calC$. Propositions \ref{yonedaprop} and \ref{limiteval} imply that $j'$ and $e$ preserve $\kappa$-small limits, so that $j(C)$ preserves $\kappa$-small limits. To conclude the proof, it will suffice to show that the collection of functors $F: \calC^{op} \rightarrow \SSet$ which satisfy $(3)$ is stable under $\kappa$-filtered colimits: this follows easily from Proposition \ref{frent}.
\end{proof}

\begin{proposition}\label{justcut}
Let $\calC$ be a small $\infty$-category, let $\kappa$ be a regular cardinal, and let
$j: \calC \rightarrow \Ind_{\kappa}(\calC)$ be the Yoneda embedding. For each object $C \in \calC$, $j(C)$ is a $\kappa$-compact object of $\Ind_{\kappa}(\calC)$.
\end{proposition}

\begin{proof}
The functor $\Ind_{\kappa}(\calC) \rightarrow \SSet$ co-represented by $j(C)$ is equivalent to the composition 
$$ \Ind_{\kappa}(\calC) \subseteq \calP(\calC) \rightarrow \SSet$$
where the first map is the canonical inclusion and the second is given by evaluation at
$C$. The second map preserves all colimits (Proposition \ref{limiteval}), and the
first preserves $\kappa$-filtered colimits since $\Ind_{\kappa}(\calC)$ is stable under
$\kappa$-filtered colimits in $\calP(\calC)$ (Proposition \ref{geort}).
\end{proof}

\begin{remark}
Let $\calC$ be a small $\infty$-category and $\kappa$ a regular cardinal. Suppose
that $\calC$ is equivalent to an $n$-category, so that the Yoneda embedding
$j: \calC \rightarrow \calP(\calC)$ factors through $\calP_{\leq n-1}(\calC) = \Fun(\calC^{op}, \tau_{\leq n-1} \SSet)$, where
$\tau_{\leq n-1} \SSet$ denotes the full subcategory of $\SSet$ spanned by the
$(n-1)$-truncated spaces: that is, spaces whose homotopy groups vanish in dimensions $n$ and above. The class of $(n-1)$-truncated spaces is stable under filtered colimits, so that
$\calP_{\leq n-1}(\calC)$ is stable under filtered colimits in $\calP(\calC)$. Corollary \ref{indpr} implies that $\Ind(\calC) \subseteq \calP_{\leq n-1}(\calC)$. In particular, $\Ind(\calC)$ is itself equivalent to an $n$-category. In particular, if $\calC$ is the nerve of an ordinary category $\calI$, then $\Ind(\calC)$ is equivalent to the nerve of an ordinary category $\calJ$, which is uniquely determined up to equivalence. Moreover, $\calJ$ admits filtered colimits, and there is a fully faithful embedding $\calI \rightarrow \calJ$ which generates $\calJ$ under filtered colimits, whose essential image consists of compact objects of $\calJ$. It follows
that $\calJ$ is equivalent to the category of $\Ind$-objects of $\calI$, in the sense of ordinary category theory.
\end{remark}

According to Corollary \ref{indpr}, we may characterize $\Ind_{\kappa}(\calC)$ as the smallest full subcategory of $\calP(\calC)$ which contains the image of the Yoneda embedding $j: \calC \rightarrow \calP(\calC)$ and is stable under $\kappa$-filtered colimits. Our goal is to obtain a more precise characterization of $\Ind_{\kappa}(\calC)$: namely, we will show that it is {\em freely} generated by $\calC$ under $\kappa$-filtered colimits. 

\begin{lemma}\label{diverti}
Let $\calD$ be an $\infty$-category $($not necessarily small$)$. There exists a fully faithful functor
$i: \calD \rightarrow \calD'$ with the following properties:
\begin{itemize}
\item[$(1)$] The $\infty$-category $\calD'$ admits small colimits.
\item[$(2)$] A small diagram $K^{\triangleright} \rightarrow \calD$ is a colimit if and only if the composite map $K^{\triangleright} \rightarrow \calD'$ is a colimit.
\end{itemize}
\end{lemma}

\begin{proof}
Let $\calD' = \Fun( \calD, \widehat{\SSet})^{op}$, and let $i$
be the opposite of the Yoneda embedding. Then $(1)$ follows from Proposition \ref{limiteval} and $(2)$ from Proposition \ref{yonedaprop}.
\end{proof}

We will need the following analogue of Lemma \ref{longwait1}:

\begin{lemma}\label{waitlong1}
Let $\calC$ be a small $\infty$-category, $\kappa$ a regular cardinal,  
$j: \calC \rightarrow \Ind_{\kappa}(\calC)$ the Yoneda embedding, and $\calC' \subseteq \calC$ the essential image of $j$. Let $\calD$ be an $\infty$-category which admits small $\kappa$-filtered colimits. Then:
\begin{itemize}
\item[$(1)$] Every functor $f_0: \calC' \rightarrow \calD$ admits a left Kan extension
$f: \Ind_{\kappa}(\calC) \rightarrow \calD$.

\item[$(2)$] An arbitrary functor $f: \Ind_{\kappa}(\calC) \rightarrow \calD$ is a left Kan extension of $f| \calC'$ if and only if $f$ is $\kappa$-continuous.

\end{itemize}
\end{lemma}

\begin{proof}
Fix an arbitrary functor $f_0: \calC' \rightarrow \calD$. Without loss of generality, we may assume that $\calD$ is a full subcategory of a larger $\infty$-category $\calD'$, satisfying the conclusions of Lemma \ref{diverti}; in particular, $\calD$ is stable under small $\kappa$-filtered colimits in $\calD'$. We may further assume that $\calD$ coincides with its essential image in $\calD'$. Lemma \ref{longwait1} guarantees the existence of a functor $F: \calP(\calC) \rightarrow \calD'$ which is a left Kan extension of $f_0 = F | \calC'$, and such that $F$ preserves small colimits. 
Since $\Ind_{\kappa}(\calC)$ is generated by $\calC'$ under $\kappa$-filtered colimits (Corollary \ref{indpr}), the restriction $f = F | \Ind_{\kappa}(\calC)$ factors through $\calD$. It is then clear that $f: \Ind_{\kappa}(\calC) \rightarrow \calD$ is a left Kan extension of $f_0$, and that $f$ is $\kappa$-continuous. This proves $(1)$ and the ``only if'' direction of $(2)$ (since left Kan extensions of $f_0$ are unique up to equivalence).

We now prove the ``if'' direction of $(2)$. Let $f: \Ind_{\kappa}(\calC) \rightarrow \calD$ be the functor constructed above, and let $f': \Ind_{\kappa}(\calC) \rightarrow \calD$ be an arbitrary
$\kappa$-continuous functor such that $f | \calC'  = f' | \calC'$. We wish to prove that $f'$ is a left Kan extension of $f' | \calC'$. Since $f$ is a left Kan extension of
$f| \calC'$, there exists a natural transformation $\alpha: f \rightarrow f'$ which is an equivalence when restricted to $\calC'$. Let $\calE \subseteq \Ind_{\kappa}(\calC)$ be the full subcategory spanned by those objects $C$ for which the morphism $\alpha_{C}: f(C) \rightarrow f'(C)$ is an equivalence in $\calD$. By hypothesis, $\calC' \subseteq \calE$. Since both $f$ and $f'$ are $\kappa$-continuous, $\calE$ is stable under $\kappa$-filtered colimits in $\Ind_{\kappa}(\calC)$. We now apply Corollary \ref{indpr} to conclude that $\calE = \Ind_{\kappa}(\calC)$. It follows that $f'$ and $f$ are equivalent, so that $f'$ is a left Kan extension of $f' | \calC'$ as desired.
\end{proof}

\begin{remark}\label{poweryoga}
The proof of Lemma \ref{waitlong1} is very robust, and can be used to establish a number of analogous results. Roughly speaking, given any class $S$ of colimits, one can consider the smallest full subcategory $\calC''$ of $\calP(\calC)$ which contains the essential image $\calC'$ of the Yoneda embedding and is stable under colimits of type $S$. Given any functor $f_0: \calC' \rightarrow \calD$, where $\calD$ is an $\infty$-category which admits colimits of type $S$, one can show that there exists a functor $f: \calC'' \rightarrow \calD$ which is a left Kan extension of $f_0 = f| \calC'$, and that $f$ is characterized by the fact that it preserves colimits of type $S$. Taking $S$ to be the class of all small colimits, we recover Lemma \ref{longwait1}. Taking $S$ to be the class of all small $\kappa$-filtered colimits, we recover Lemma \ref{waitlong1}. Other variations are possible as well: we will exploit this idea further in \S \ref{agileco}.
\end{remark}

\begin{proposition}\label{intprop}
Let $\calC$ and $\calD$ be $\infty$-categories, and let $\kappa$ be a regular cardinal.
Suppose that $\calC$ is small and that $\calD$ admits small $\kappa$-filtered colimits.
Then composition with the Yoneda embedding induces an equivalence of $\infty$-categories
$$ \bHom_{\kappa}( \Ind_{\kappa}(\calC), \calD) \rightarrow \Fun(\calC, \calD),$$
where the left hand side denotes the $\infty$-category of all $\kappa$-continuous functors
from $\Ind_{\kappa}(\calC)$ to $\calD$.
\end{proposition}

\begin{proof}
Combine Lemma \ref{waitlong1} with Corollary \ref{leftkanextdef}.
\end{proof}

In other words, if $\calC$ is small and $\calD$ admits $\kappa$-filtered colimits, then any
functor $f: \calC \rightarrow \calD$ determines an essentially unique extension
$F: \Ind_{\kappa}(\calC) \rightarrow \calD$ (such that $f$ is equivalent to $F \circ j$).
We next give a criterion which will allow us to determine when $F$ is an equivalence.

\begin{proposition}\label{uterr}\index{gen}{Ind-object!characterization of}
Let $\calC$ be a small $\infty$-category, $\kappa$ a regular cardinal, and $\calD$ an $\infty$-category which admits $\kappa$-filtered colimits. Let $F: \Ind_{\kappa}(\calC) \rightarrow \calD$
be a $\kappa$-continuous functor, and $f = F \circ j$ its composition with the Yoneda embedding $j: \calC \rightarrow \Ind_{\kappa}(\calC)$. Then:
\begin{itemize}
\item[$(1)$] If $f$ is fully faithful and its essential image consists of $\kappa$-compact objects of $\calD$, then $F$ is fully faithful.
\item[$(2)$] The functor $F$ is an equivalence if and only if the following conditions are satisfied:
\begin{itemize}
\item[$(i)$] The functor $f$ is fully faithful.
\item[$(ii)$] The functor $f$ factors through $\calD^{\kappa}$. 
\item[$(iii)$] The objects $\{ f(C) \}_{C \in \calC}$ generate $\calD$ under $\kappa$-filtered colimits.
\end{itemize}
\end{itemize}
\end{proposition}

\begin{proof}
We first prove $(1)$, using argument of Proposition \ref{trumptow}. Let
$C$ and $D$ be objects of $\Ind_{\kappa}(\calC)$. 
We wish to prove that the map
$$ \eta_{C,D}: \bHom_{\calP(\calC)}(C,D) \rightarrow \bHom_{\calD}(F(C), F(D))$$ is an isomorphism in the homotopy category $\calH$. Suppose first that $C$ belongs to the essential image
of $j$. Let $G: \calP(\calC) \rightarrow \SSet$ be a functor co-represented by $C$, and let
$G': \calD \rightarrow \SSet$ be a functor co-represented by $F(C)$. Then we have a natural transformation of functors $G \rightarrow G' \circ F$. Assumption $(2)$ implies that $G'$ preserves small $\kappa$-filtered colimits, so that $G' \circ F$ preserves small $\kappa$-filtered colimits. 
Proposition \ref{justcut} implies that
$G$ preserves small $\kappa$-filtered colimits. It follows that the collection of objects $D \in \Ind_{\kappa}(\calC)$ such
that $\eta_{C,D}$ is an equivalence is stable under small $\kappa$-filtered colimits colimits. If $D$ belongs to the essential image of $j$, then the assumption that $f$ is fully faithful implies that $\eta_{C,D}$ is a homotopy equivalence. Since the image of the Yoneda embedding generates
$\Ind_{\kappa}(\calC)$ under small $\kappa$-filtered colimits, we conclude that $\eta_{C,D}$ is a homotopy equivalence for every object $D \in \Ind_{\kappa}(\calC)$.

We now drop the assumption that $C$ lies in the essential image of $j$. Fix $D \in \Ind_{\kappa}(\calC)$. Let $H: \Ind_{\kappa}(\calC)^{op} \rightarrow \SSet$
be a functor represented by $D$, and let $H': \calD^{op} \rightarrow \SSet$ be a functor represented by $FD$. Then we have a natural transformation of functors $H \rightarrow H' \circ F^{op}$, which we wish to prove is an equivalence. By assumption, $F^{op}$ preserves small $\kappa$-filtered limits. Proposition \ref{yonedaprop} implies that $H$ and $H'$ preserve small limits. It follows that the collection $P$ of objects $C \in \calP(S)$ such that $\eta_{C,D}$ is an equivalence is stable under small $\kappa$-filtered colimits.
The special case above established that $P$ contains the essential image of the Yoneda embedding. Since $\Ind_{\kappa}(\calC)$ is generated under small $\kappa$-filtered colimits by the image of the Yoneda embedding, we deduce that $\eta_{C,D}$ is an equivalence in general. This completes the proof of $(1)$.

We now prove $(2)$. Suppose first that $F$ is an equivalence. Then
$(i)$ follows from Proposition \ref{fulfaith}, $(ii)$ from Proposition \ref{justcut}, and $(iii)$ from Corollary \ref{indpr}. Conversely, suppose that $(i)$, $(ii)$, and $(iii)$ are satisfied. Using $(1)$, we deduce that $F$ is fully faithful. The essential image of $F$ contains the essential image of $f$ and is stable under small $\kappa$-filtered colimits. Therefore $F$ is essentially surjective, so that $F$ is an equivalence as desired.
\end{proof}

According to Corollary \ref{uterrr}, an $\infty$-category $\calC$ admits small colimits if and only if $\calC$ admits $\kappa$-small colimits and $\kappa$-filtered colimits. Using Proposition \ref{uterr}, we can make a much more precise statement:

\begin{proposition}\label{precst}
Let $\calC$ be a small $\infty$-category and $\kappa$ a regular cardinal. The $\infty$-category
$\calP^{\kappa}(\calC)$ of $\kappa$-compact objects of $\calP(\calC)$ is essentially small: that is, there exists a small $\infty$-category $\calD$ and an equivalence $i: \calD \rightarrow
\calP^{\kappa}(\calC)$. Let $F: \Ind_{\kappa}(\calD) \rightarrow \calP(\calC)$
be a $\kappa$-continuous functor such that the composition of $f$ with the Yoneda embedding
$$ \calD \rightarrow \Ind_{\kappa}(\calD) \rightarrow \calP(\calC)$$
is equivalent to $i$ $($according to Proposition \ref{intprop}, $F$ exists and is unique up
to equivalence$)$. Then $F$ is an equivalence of $\infty$-categories.
\end{proposition}

\begin{proof}
Since $\calP(\calC)$ is locally small, to prove that $\calP^{\kappa}(\calC)$ is small it will suffice to show that the collection of isomorphism classes of objects in the homotopy category
$h \calP^{\kappa}(\calC)$ is small. For this, we invoke Proposition \ref{charsmallpre}:
every $\kappa$-compact object $X$ of $\calP(\calC)$ is a retract of some object $Y$, which is
itself the colimit of some composition
$$ K \stackrel{p}{\rightarrow} \calC \rightarrow \calP(\calC)$$
where $K$ is $\kappa$-small. Since there is a bounded collection of possibilities
for $K$ and $p$ (up to isomorphism in $\sSet$), and a bounded collection of idempotent maps $Y \rightarrow Y$ in $h \calP(\calC)$, there are only a bounded number of possibilities for $X$.

To prove that $F$ is an equivalence, it will suffice to show that $F$ satisfies conditions $(i)$, $(ii)$, and $(iii)$ of Proposition \ref{uterr}. 
Conditions $(i)$ and $(ii)$ are obvious. For $(iii)$, we must prove that every object of
$X \in \calP(\calC)$ can be obtained as a small $\kappa$-filtered colimit of $\kappa$-compact objects of $\calC$. Using Lemma \ref{longwait0}, we can write $X$ as a small colimit
taking values in the essential image of $j: \calC \rightarrow \calP(\calC)$. The proof of Corollary \ref{uterrr} shows that $X$ can be written as a $\kappa$-filtered colimit of a diagram with values
in a full subcategory $\calE \subseteq \calP(\calC)$, where each object of $\calE$ is itself
a $\kappa$-small colimit of some diagram taking values in the essential image of $j$. Using Corollary \ref{tyrmyrr}, we deduce that $\calE \subseteq \calP^{\kappa}(\calC)$, so that $X$ lies in the essential image of $F$ as desired.
\end{proof}

Note that the construction $\calC \mapsto \Ind_{\kappa}(\calC)$ is functorial in $\calC$.
Given a functor $f: \calC \rightarrow \calC'$, Proposition \ref{intprop} implies that the composition of $f$ with the Yoneda embedding $j_{\calC'}: \calC' \rightarrow \Ind_{\kappa} \calC'$ is equivalent to the composition $$ \calC \stackrel{j_{\calC}}{\rightarrow} \Ind_{\kappa} \calC \stackrel{F}{\rightarrow}\Ind_{\kappa} \calC',$$
where $F$ is a $\kappa$-continuous functor. The functor $F$ is well-defined up to equivalence (in fact, up to contractible ambiguity). We will denote $F$ by $\Ind_{\kappa} f$ (though this is perhaps a slight abuse of notation, since $F$ is uniquely determined only up to equivalence).

\begin{proposition}\index{gen}{adjoint functor!between $\Ind$-categories}
Let $f: \calC \rightarrow \calC'$ be a functor between small $\infty$-categories. The following
are equivalent:
\begin{itemize}
\item[$(1)$] The functor $f$ is $\kappa$-right exact.
\item[$(2)$] The map $G: \calP(\calC') \rightarrow \calP(\calC)$ given by composition with $f$ restricts to a functor $g: \Ind_{\kappa}(\calC') \rightarrow \Ind_{\kappa}(\calC)$.
\item[$(3)$] The functor $\Ind_{\kappa} f$ has a right adjoint.
\end{itemize}
Moreover, if these conditions are satisfied, then $g$ is a right adjoint to $\Ind_{\kappa} f$.
\end{proposition}

\begin{proof}
The equivalence $(1) \Leftrightarrow (2)$ is just a reformulation of the definition of $\kappa$-right exactness. Let $\calP(f): \calP(\calC) \rightarrow \calP(\calC')$ be a functor which preserves small colimits such that the diagram of $\infty$-categories
$$ \xymatrix{ \calC \ar[d] \ar[r]^{f} & \calC' \ar[d] \\
\calP(\calC) \ar[r]^{\calP(f)} & \calP(\calC') }$$
is homotopy commutative. Then we may identify $\Ind_{\kappa}(f)$ with the restriction
$\calP(f) | \Ind_{\kappa}(\calC)$. Proposition \ref{adjobs} asserts that $G$ is a right adjoint of
$\calP(f)$. Consequently, if $(2)$ is satisfied, then $g$ is a right adjoint to $\Ind_{\kappa}(f)$. We deduce in particular that $(2) \Rightarrow (3)$. We will complete the proof by showing that $(3)$ implies $(2)$. Suppose that $\Ind_{\kappa}(f)$ admits a right adjoint $g': \Ind_{\kappa}(\calC') \rightarrow \Ind_{\kappa}(\calC)$. Let $X: (\calC')^{op} \rightarrow \SSet$ be an object
of $\Ind_{\kappa}(\calC')$. Then $X^{op}$ is equivalent to the composition
$$ \calC' \stackrel{j}{\rightarrow} \Ind_{\kappa}(\calC') \stackrel{c_X}{\rightarrow} \SSet^{op},$$
where the $c_{X}$ denotes the functor represented by $X$. Since $g'$ is a left adjoint to
$\Ind_{\kappa} f$, the functor $c_{X} \circ \Ind_{\kappa}(f)$ is represented by $g' X$. 
Consequently, we have a homotopy commutative diagram
$$ \xymatrix{ \calC \ar[r]^{j_{\calC}} \ar[d]^{f} & \Ind_{\kappa}(\calC) \ar[r] \ar[d]^{\Ind_{\kappa}(f)} \ar[r]^{c_{g'X}} & \SSet^{op} \ar[d] \\
\calC' \ar[r] & \Ind_{\kappa}(\calC') \ar[r]^{c_{X}} & \SSet^{op} }$$
so that $G(X)^{op} = f \circ X^{op} \simeq c_{g'X} \circ j_{\calC}$, and therefore belongs to
$\Ind_{\kappa}(\calC)$.
\end{proof}

\begin{proposition}\label{turnke}
Let $\calC$ be a small $\infty$-category and $\kappa$ a regular cardinal. 
The Yoneda embedding $j: \calC \rightarrow \Ind_{\kappa}(\calC)$ preserves all
$\kappa$-small colimits which exist in $\calC$.
\end{proposition}

\begin{proof}
Let $K$ be a $\kappa$-small simplicial set, and $\overline{p}: K^{\triangleright} \rightarrow \calC$ a colimit diagram. We wish to show that $j \circ \overline{p}: K^{\triangleright} \rightarrow \Ind_{\kappa}(\calC)$ is also a colimit diagram. Let $C \in \Ind_{\kappa}(\calC)$ be an object, and let
$F: \Ind_{\kappa}(\calC)^{op} \rightarrow \hat{\SSet}$ be the functor represented by $F$. According to Proposition \ref{yonedaprop}, it will suffice to show that 
$F \circ (j \circ \overline{p})^{op}$ is a limit diagram in $\SSet$. We observe that
$F \circ j^{op}$ is equivalent to the object $C \in \Ind_{\kappa}(\calC) \subseteq
\Fun( \calC^{op}, \SSet)$, and therefore $\kappa$-right exact. We now conclude by invoking Proposition \ref{swarmmy}. 
\end{proof}

We conclude this section with a useful result concerning diagrams in $\infty$-categories of $\Ind$-objects:

\begin{proposition}\label{urgh1}
Let $\calC$ be a small $\infty$-category, $\kappa$ a regular cardinal, and 
$j: \calC \rightarrow \Ind_{\kappa}(\calC)$ the Yoneda embedding. Let $A$ be a finite partially ordered set, and let $j': \Fun( \Nerve(A), \calC) \rightarrow \Fun( \Nerve(A), \Ind_{\kappa}(\calC))$ be the induced map. Then $j'$ induces an equivalence
$$ \Ind_{\kappa}( \Fun(\Nerve(A), \calC) ) \rightarrow \Fun( \Nerve(A), \Ind_{\kappa}(\calC)).$$ 
\end{proposition}

In other words, every diagram $\Nerve(A) \rightarrow \Ind_{\kappa}(\calC)$ can be obtained, in
an essentially unique way, as a $\kappa$-filtered colimit of diagrams $\Nerve(A) \rightarrow \calC$.

\begin{warning}
The statement of Proposition \ref{urgh1} fails if we replace $\Nerve(A)$ by an arbitrary finite simplicial set. For example, we may identify the category of abelian groups with the category of $\Ind$-objects of the category of finitely generated abelian groups. If $n > 1$, then the map $q \mapsto \frac{q}{n}$ from the group of rational numbers $\Q$ to itself cannot be obtained as a filtered colimit of endomorphisms finitely generated abelian groups. 
\end{warning}

\begin{proof}[Proof of Proposition \ref{urgh1}]
According to Proposition \ref{uterr}, it will suffice to prove the following:
\begin{itemize}
\item[$(i)$] The functor $j'$ is fully faithful.
\item[$(ii)$] The essential image of $j'$ consists of $\kappa$-compact objects of
$\Fun( \Nerve(A), \Ind_{\kappa}(\calC))$. 
\item[$(iii)$] The essential image of $j'$ generates $\Fun( \Nerve(A), \Ind_{\kappa}(\calC) )$ under small, $\kappa$-filtered colimits. 
\end{itemize}
Since the Yoneda embedding $j: \calC \rightarrow \Ind_{\kappa}(\calC)$ satisfies the analogues of these conditions, $(i)$ is obvious and $(ii)$ follows from 
Proposition \ref{placeabovee}. To prove $(iii)$, we fix an object $F \in \Fun( \Nerve(A),  \Ind_{\kappa}(\calC))$. Let $\calC'$ denote the essential image of $j$, and form a pullback diagram of simplicial sets
$$\xymatrix{ \calD \ar[r] \ar[d] &  \Fun(\Nerve(A), \calC') \ar[d] \\ 
\Fun(\Nerve(A), \Ind_{\kappa}(\calC))_{/F} \ar[r] & \Fun( \Nerve(A), \Ind_{\kappa}(\calC) )}.$$
Since $\calD$ is essentially small, $(iii)$ is a consequence of the following assertions:
\begin{itemize}
\item[$(a)$] The $\infty$-category $\calD$ is $\kappa$-filtered.
\item[$(b)$] The canonical map $\calD^{\triangleright} \rightarrow \Fun( \Nerve(A), \calC)$
is a colimit diagram.
\end{itemize}
To prove $(a)$, it will to show that $\calD$ has the right lifting property with respect
to the inclusion $\Nerve(B) \subseteq \Nerve( B \cup \{\infty\})$, for every $\kappa$-small partially ordered set $B$ (Remark \ref{tweeny}). Regard $B \cup \{ \infty, \infty' \}$ as a partially ordered set with $b < \infty < \infty'$ for each $b \in B$. Unwinding the definitions, we see that
$(a)$ is equivalent to the following assertion:
\begin{itemize}
\item[$(a')$] Let $\overline{F}: \Nerve( A \times ( B \cup \{ \infty' \}) ) \rightarrow \Ind_{\kappa}(\calC)$ 
be such that $\overline{F}| \Nerve(A \times \{ \infty' \}) = F$, and $\overline{F}'| \Nerve(A \times B)$ factors through
$\calC'$. Then there exists a map $\overline{F}': \Nerve(A \times ( B \cup \{ \infty, \infty' \}) ) \rightarrow \Ind_{\kappa}(\calC)$ which extends $\overline{F}$, such that $\overline{F}' | \Nerve(A \times ( B \cup \{\infty\}))$ factors through $\calC'$.
\end{itemize}
To find $\overline{F}'$, we write $A = \{ a_1, \ldots, a_n \}$, where $a_i \leq a_{j}$ implies
$i \leq j$. We will construct a compatible sequence of maps
$$ \overline{F}_{k}: \Nerve( (A \times (B \cup \{ \infty' \})) \cup ( \{ a_1, \ldots, a_k \} \times \{\infty \} )) \rightarrow \calC$$
with $\overline{F}_0 = \overline{F}$ and $\overline{F}_{n} = \overline{F}'$. For each
$a \in A$, we let $ A_{\leq a} = \{ a' \in A: a' \leq a \}$, and we define $A_{< a}$, $A_{\geq a}$, $A_{> a}$ similarly. Supposing that
$\overline{F}_{k-1}$ has been constructed, we observe that constructing $\overline{F}_{k}$
amounts to constructing an object of the $\infty$-category
$$ ( \calC'_{ / F | \Nerve( A_{\geq a_k} )} )_{ \overline{F}_{k-1} | M /},$$
where $M = (A_{\leq a_k} \times B) \cup ( A_{ < a_k } \times \{ \infty \} )$.
The inclusion $\{ a_k \} \subseteq \Nerve( A_{ \geq a_k })$ is left anodyne. It will therefore suffice to construct an object in the equivalent $\infty$-category
$ ( \calC'_{/F(a_k) })_{ \overline{F}_{k-1} |M/ }$. Since $M$ is $\kappa$-small, it suffices to show that
the $\infty$-category $\calC'_{/F(a_k)}$ is $\kappa$-filtered. This is simply a reformulation of the fact that $F(a_k) \in \Ind_{\kappa}(\calC)$. 

We now prove $(b)$. It will suffice to show that for each $a \in A$, the composition
$$ \calD^{\triangleright} \rightarrow \Fun( \Nerve(A), \Ind_{\kappa}(\calC) ) \rightarrow
\Ind_{\kappa}(\calC)$$ is a colimit diagram, where the second map is given by evaluation at $a$. Let $\calD(a) = \calC' \times_{ \Ind_{\kappa}(\calC) } \Ind_{\kappa}(\calC)_{/F(a)}$, so that
$\calD(a)$ is $\kappa$-filtered and the associated map $\calD(a)^{\triangleright} \rightarrow \Ind_\kappa(\calC)$ is a colimit diagram. It will therefore suffice to show that the canonical map
$\calD \rightarrow \calD(a)$ is cofinal. According to Theorem \ref{hollowtt}, it will suffice to show that for each object $D \in \calD(a)$, the fiber product
$\calE = \calD \times_{ \calD(a) } \calD(a)_{D/}$ is weakly contractible. In view of Lemma \ref{stull2}, it will suffice to show that $\calE$ is filtered. This can be established by a minor variation of the argument given above.
\end{proof}

\subsection{Adjoining Colimits to $\infty$-Categories}\label{agileco}

Let $\calC$ be a small $\infty$-category. According to Proposition \ref{intprop}, the 
$\infty$-category $\Ind(\calC)$ enjoys the following properties, which characterize it up to equivalence:
\begin{itemize}
\item[$(1)$] There exists a functor $j: \calC \rightarrow \Ind(\calC)$.
\item[$(2)$] The $\infty$-category $\Ind(\calC)$ admits small filtered colimits.
\item[$(3)$] Let $\calD$ be an $\infty$-category which admits small filtered colimits, and let
$\Fun'( \Ind(\calC), \calD)$ be the full subcategory of $\Fun( \Ind(\calC), \calD)$ spanned by those functors which preserve filtered colimits. Then composition with $j$ induces an equivalence
$\Fun'( \Ind(\calC), \calD) \rightarrow \Fun(\calC, \calD)$.
\end{itemize}
We may informally summarize this characterization as follows: the $\infty$-category $\Ind(\calC)$ is obtained from $\calC$ by freely adjoining the colimits of all small, filtered diagrams. In this section, we will study a generalization of this construction, which allows us to freely adjoin to $\calC$
the colimits of {\em any} collection of diagrams.

\begin{notation}\label{sipser}
Let $\calC$ and $\calD$ be $\infty$-categories, and let $\calR$ be a collection of diagrams
$\{ \overline{p}_{\alpha}: K_{\alpha}^{\triangleright} \rightarrow \calC \}$. We let
$\Fun_{\calR}(\calC, \calD)$ denote the full subcategory of
$\Fun(\calC, \calD)$ spanned by those functors which carry each diagram in $\calR$ to a colimit diagram in $\calD$.\index{not}{FunRCD@$\Fun_{\calR}(\calC, \calD)$}

Let $\calK$ be a collection of simplicial sets. We will say that an $\infty$-category
$\calC$ {\it admits $\calK$-indexed colimits} if it admits $K$-indexed colimits for each $K \in \calK$.
If $f: \calC \rightarrow \calD$ is a functor between $\infty$-categories which admit $\calK$-indexed colimits, then we will say that $f$ {\it preserves $\calK$-indexed colimits} if $f$ preserves $K$-indexed colimits, for each $K \in \calK$. We let $\Fun_{\calK}( \calC, \calD)$ denote the full subcategory of
$\Fun( \calC, \calD)$ spanned by those functors which preserves $\calK$-indexed colimits.\index{not}{FunKCD@$\Fun_{\calK}(\calC,\calD)$}
%More generally, given a collection of $\infty$-categories $\calC_1, \ldots, \calC_n$ and $\calD$
%which admit $\calK$-indexed colimits, we let
%$\Fun_{ \calK \boxtimes \ldots \boxtimes \calK}( \calC_1 \times \ldots \calC_n, \calD)$ denote the full subcategory of $\Fun( \calC_1 \times \ldots \times \calC_n, \calD)$ spanned by those functors
%which preserve $K$-indexed colimits separately in each variable, for each $K \in \calK$.
\end{notation}

\begin{proposition}\label{cupper1}
Let $\calK$ be a collection of simplicial sets, $\calC$ an $\infty$-category, and 
$\calR = \{ \overline{p}_{\alpha}: K^{\triangleright}_{\alpha} \rightarrow \calC \}$ a collection
of diagrams in $\calC$. Assume that each $K_{\alpha}$ belongs to $\calK$. Then there exists
a new $\infty$-category $\calP^{\calK}_{\calR}( \calC)$ and a map $j: \calC \rightarrow \calP^{\calK}_{\calR}(\calC)$ with the following properties:
\begin{itemize}
\item[$(1)$] The $\infty$-category $\calP^{\calK}_{\calR}(\calC)$ admits $\calK$-indexed colimits.
\item[$(2)$] For every $\infty$-category $\calD$ which admits $\calK$-indexed colimits, composition with $j$ induces an equivalence of $\infty$-categories
$$ \Fun_{\calK}( \calP^{\calK}_{\calR}(\calC), \calD) \rightarrow \Fun_{\calR}( \calC, \calD).$$
\end{itemize}
Moreover, if every member of $\calR$ is already a colimit diagram in $\calC$, then we have in addition:
\begin{itemize}
\item[$(3)$] The functor $j$ is fully faithful.
\end{itemize}
\end{proposition}

\begin{remark}
In the situation of Proposition \ref{cupper1}, assertion $(2)$ (applied in the case $\calD = \calP^{\calK}_{\calR}(\calC)$) guarantees that $j$ carries each diagram in $\calR$ to a colimit diagram in $\calP^{\calK}_{\calR}(\calC)$. We can informally summarize conditions $(1)$ and $(2)$ as follows:
the $\infty$-category $\calP^{\calK}_{\calR}(\calC)$ is freely generated by $\calC$ under $\calK$-indexed colimits, subject only to the relations that each diagram in $\calR$ determines a colimit diagram in $\calP^{\calK}_{\calR}(\calC)$. It is clear that this property characterizes $\calP^{\calK}_{\calR}(\calC)$ (and the map $j$) up to equivalence.
\end{remark}

\begin{example}
Suppose that $\calK$ is the collection of all {\em small} simplicial sets, that the $\infty$-category $\calC$ is small, and that the set of diagrams $\calR$ is empty. Then the Yoneda embedding $j: \calC \rightarrow \calP(\calC)$ satisfies the conclusions of Proposition \ref{cupper1}. This is precisely the assertion of Theorem
\ref{charpresheaf} (save for assertion $(3)$, which follows from Proposition \ref{fulfaith}). 
This justifies the notation of Proposition \ref{cupper1}; in the general case we can think of $\calP^{\calK}_{\calR}(\calC)$ as a sort of generalized presheaf category $\calC$, and $j$ as an analogue of the Yoneda embedding.
\end{example}

\begin{proof}[Proof of Proposition \ref{cupper1}:]
We will employ essentially the same argument as in our proof of Proposition \ref{intprop}. 
First, we may enlarge the universe if necessary to reduce to the case where every element of
$\calK$ is a small simplicial set, the $\infty$-category $\calC$ is small, and the collection of diagrams $\calR$ is small. Let $j_0: \calC \rightarrow \calP(\calC)$ denote the Yoneda embedding. For
every diagram $\overline{p}_{\alpha}: K^{\triangleright} \rightarrow \calC$, we let
$p_{\alpha}$ denote the restriction $\overline{p}_{\alpha} | K$, $X_{\alpha} \in \calP(\calC)$ a colimit for
the induced diagram $j \circ p_{\alpha}: K \rightarrow \calP(\calC)$, and
$Y_{\alpha} \in \calC$ the image of the cone point under $\overline{p}_{\alpha}$. The diagram
$j_0 \circ \overline{p}_{\alpha}$ induces a map $s_{\alpha}: X_{\alpha} \rightarrow j_0(Y_{\alpha})$ (well-defined up to homotopy), let $S = \{ s_{\alpha} \}$ be the set of all such morphisms. We let
$S^{-1} \calP(\calC) \subseteq \calP(\calC)$ denote the $\infty$-category of $S$-local objects
of $\calP(\calC)$, and $L: \calP(\calC) \rightarrow S^{-1} \calP(\calC)$ a left adjoint to the inclusion.
We define $\calP^{\calK}_{\calR}(\calC)$ to be the smallest full subcategory of
$S^{-1} \calP(\calC)$ which contains the essential image of the functor $L \circ j_0$ and is closed
under $\calK$-indexed colimits, and let $j = L \circ j_0$ be the induced map. We claim that the
map $j: \calC \rightarrow \calP^{\calK}_{\calR}(\calC)$ has the desired properties.

Assertion $(1)$ is obvious. We now prove $(2)$. Let $\calD$ be an $\infty$-category which admits $\calK$-indexed colimits. In view of Lemma \ref{diverti}, we can assume that there exists a fully faithful inclusion $\calD \subseteq \calD'$, where $\calD'$ admits all small colimits, and $\calD$ is stable under
$\calK$-indexed colimits in $\calD'$. We have a commutative diagram
$$ \xymatrix{ \Fun_{\calK}( \calP^{\calK}_{\calR}(\calC), \calD) \ar[r]^{\phi} \ar[d] & \Fun_{\calR}( \calC, \calD) \ar[d] \\
\Fun_{\calK}( \calP^{\calK}_{\calR}( \calC), \calD') \ar[r]^{\phi'} & \Fun_{\calR}( \calC, \calD'). }$$
We claim that this diagram is a homotopy Cartesian. Unwinding the definitions, this is equivalent to the assertion that a functor $f \in \Fun_{\calK}( \calP^{\calK}_{\calR}(\calC), \calD')$ factors through $\calD$ if and only if $f \circ j: \calC \rightarrow \calD'$ factors through $\calD$. The ``only if'' direction is obvious. Conversely, if $f \circ j$ factors through $\calD$, then $f^{-1} \calD$ is full subcategory of
$\calP^{\calK}_{\calR}(\calC)$ which is stable under $\calK$-indexed limits (since $f$ preserves $\calK$-indexed limits and $\calD$ is stable under $\calK$-indexed limits in $\calD$) and contains the essential image of $j$; by minimality, we conclude that $f^{-1} \calD = \calP^{\calK}_{\calR}(\calC)$. 

Our goal is to prove that the functor $\phi$ is an equivalence of $\infty$-categories. In view of the preceding argument, it will suffice to show that $\phi'$ is an equivalence of $\infty$-categories. In other words, we may replace $\calD$ by $\calD'$ and thereby reduce to the case where
$\calD'$ admit small colimits.

Let $\calE \subseteq \calP(\calC)$ denote the inverse image
$L^{-1} \calP^{\calK}_{\calR}(\calC)$, and let $\overline{S}$ denote the collection of all
morphisms $\alpha$ in $\calE$ such that $L\alpha$ is an equivalence. Composition with
$L$ induces a fully faithful embedding $\Fun( \calP^{\calK}_{\calR}(\calC), \calD) \rightarrow
\Fun( \calE, \calD)$, whose essential image consists of those functors $\calE \rightarrow \calD$
which carry every morphism in $\overline{S}$ to an equivalence in $\calD$. Furthermore, a functor
$f: \calP^{\calK}_{\calR}(\calC) \rightarrow \calD$ preserves $\calK$-indexed colimits if and only if the
composition $f \circ L: \calE \rightarrow \calD$ preserves $\calK$-indexed colimits. The functor $\phi$ factors as a composition
$$ \Fun_{\calK}( \calP^{\calK}_{\calR}(\calC), \calD) \rightarrow \Fun'( \calE, \calD)
\stackrel{\psi}{\rightarrow} \Fun_{\calR}(\calC, \calD),$$
where $\Fun'(\calE, \calD)$ denotes the full subcategory of $\Fun( \calE, \calD)$ spanned by those functors which carry every morphism in $\overline{S}$ to an equivalence and preserve $\calK$-indexed colimits. It will therefore suffice to show that $\psi$ is an equivalence of $\infty$-categories.

In view of Proposition \ref{lklk}, we need only show that if $F: \calE \rightarrow \calD$ is a functor such that $F \circ j_0$ belongs to $\Fun_{\calR}(\calC, \calD)$, then $F$ belongs to
$\Fun'(\calE, \calD)$ if and only if $F$ is a left Kan extension of $F | \calC'$, where
$\calC' \subseteq \calE$ denotes the essential image of the Yoneda embedding $j_0: \calC \rightarrow
\calE$. We first prove the ``if'' direction. Let $F_0 = F | \calC'$. Since $\calD$ admits small colimits, the
functor $F_0$ admits a left Kan extension $\overline{F}: \calP(\calC) \rightarrow \calD$; without loss of generality we may suppose that $F = \overline{F} | \calE$. According to Lemma \ref{longwait1}, 
the functor $\overline{F}$ preserves small colimits. Since $\calE$ is stable under $\calK$-indexed colimits in $\calP(\calC)$, it follows that $\overline{F} | \calE$ preserves $\calK$-indexed colimits.
Furthermore, since $F \circ j_0$ belongs to $\Fun_{\calR}(\calC, \calD)$, the functor
$\overline{F}$ carries each morphism in $S$ to an equivalence in $\calD$. It follows that
$\overline{F}$ factors (up to homotopy) through the localization functor $L$, so that
$\overline{F} | \calE$ carries each morphism in $\overline{S}$ to an equivalence in $\calD$.

For the converse, let us suppose that $F \in \Fun'(\calE, \calD)$; we wish to show that $F$
is a left Kan extension of $F| \calC'$. Let $F'$ denote an arbitrary left Kan extension of
$F | \calC'$, so that the identification $F | \calC' = F' | \calC'$ induces a natural transformation
$\alpha: F' \rightarrow F$. We wish to prove that $\alpha$ is an equivalence. Since $F'$ and
$F$ both carry each morphism in $\overline{S}$ to an equivalence, we may assume without loss of generality that $F = f \circ L$, $F' = f' \circ L$, and $\alpha = \beta \circ L$, where 
$\beta: f' \rightarrow f$ is a natural transformation of functors from $\calP^{\calK}_{\calR}(\calC)$ to
$\calD$. Let $\calX \subseteq \calP^{\calK}_{\calR}(\calC)$ denote the full subcategory spanned by those objects $X$ such that $\beta_{X}: f'(X) \rightarrow f(X)$ is an equivalence. Since both
$f'$ and $f$ preserve $\calK$-indexed colimits, we conclude that $\calX$ is stable under $\calK$-indexed colimits in $\calP^{\calK}_{\calR}(\calC)$. It is clear that $\calX$ contains the essential image
of the functor $j: \calC \rightarrow \calP^{\calK}_{\calR}(\calC)$. It follows by construction that
$\calX = \calP^{\calK}_{\calR}(\calC)$, so that $\beta$ is an equivalence as desired. This completes the proof of $(2)$.

It remains to prove $(3)$. Suppose that every element of $\calR$ is already a colimit diagram in $\calC$. 
We note that the functor $j$ factors as a composition $L \circ j_0$, where the Yoneda embedding $j_0: \calC \rightarrow \calE$ is already known to be fully faithful (Proposition \ref{fulfaith}).
Since the functor $L | S^{-1} \calP(\calC)$ is equivalent to the identity, it will suffice to show that
the essential image of $j_0$ is contained in $S^{-1} \calP(\calC)$. In other words, we must show that
if $s_{\alpha}: X_{\alpha} \rightarrow j_0 Y_{\alpha}$ belongs to $S$, and $C \in \calC$, then the induced map
$$ \bHom_{\calP(\calC)}( j_0 Y_{\alpha}, j_0 C) \rightarrow \bHom_{ \calP(\calC)}( X_{\alpha}, j_0 C)$$
is a homotopy equivalence. Let $\overline{p}: K^{\triangleright}_{\alpha} \rightarrow \calC$
be the corresponding diagram (so that $\overline{p}$ carries the cone point of
$K^{\triangleright}_{\alpha}$ to $Y_{\alpha}$), let $p = \overline{p} | K_{\alpha}$, and let
$\overline{q}: K^{\triangleright}_{\alpha} \rightarrow \calP(\calC)$ be a colimit diagram
extending $q = \overline{q} | K_{\alpha} = j_0 \circ p$. Consider the diagram
$$ \calP(\calC)_{ j_0 Y_{\alpha}/ } \stackrel{g_0}{\leftarrow} \calP(\calC)_{ j_0 \overline{p}/
} \stackrel{g_1}{\rightarrow} \calP(\calC)_{ j_0 p/ } \stackrel{g_2}{\leftarrow} \calP(\calC)_{ \overline{q}/} \stackrel{g_3}{\rightarrow} \calP(\calC)_{X_{\alpha}/ }.$$
The maps $g_0$ and $g_3$ are trivial Kan fibrations (since the inclusion of the cone point into
$K^{\triangleright}_{\alpha}$ is cofinal), and the map $g_2$ is a trivial Kan fibration since
$\overline{q}$ is a colimit diagram. Moreover, for every object $Z \in \calP(\calC)$, the above diagram determines the map $\bHom_{ \calP(\calC)}( j_0 Y_{\alpha}, Z) \rightarrow \bHom_{ \calP(\calC)}( X_{\alpha}, Z)$. Consequently, to prove that this map is an equivalence, it suffices to show that $g_1$
induces a trivial Kan fibration
$$ \calP(\calC)_{ j_0 \overline{p}/ } \times_{ \calP(\calC) } \{ Z \} 
\rightarrow \calP(\calC)_{ j_0 p/} \times_{ \calP(\calC) } \{Z \}.$$ 
Assuming $Z$ belongs to the essential image $\calC'$ of the Yoneda embedding $j_0$, we may
reduce to proving that the induced map
$ \calC'_{ j_0 \overline{p}/ } \rightarrow \calC'_{ j_0 p/}$ is a trivial Kan fibration, which is equivalent to the assertion that $j_0 \circ \overline{p}$ is a colimit diagram in $\calC'$. This is clear, since
$\overline{p}$ is a colimit diagram by assumption and $j_0$ induces an equivalence of
$\infty$-categories from $\calC$ to $\calC'$.
\end{proof}

\begin{definition}\index{not}{PKKC@$\calP^{\calK'}_{\calK}(\calC)$}
Let $\calK \subseteq \calK'$ be collections of simplicial sets, and let $\calC$ be an $\infty$-category which admits $\calK$-indexed limits. We let $\calP^{\calK'}_{\calK}(\calC)$ denote the $\infty$-category $\calP^{\calK'}_{\calR}(\calC)$, where $\calR$ is the set of all colimit diagrams
$\overline{p}: K^{\triangleright} \rightarrow \calC$ such that $K \in \calK$.
\end{definition}

\begin{example}
Let $\calK = \emptyset$, and let $\calK'$ denote the class of {\em all} small simplicial sets.
If $\calC$ is a small $\infty$-category, then we have a canonical equivalence
$\calP^{\calK'}_{\calK}(\calC) \simeq \calP(\calC)$ (Theorem \ref{charpresheaf}).
\end{example}

\begin{example}
Let $\calK = \emptyset$, and let $\calK'$ denote the class of all small $\kappa$-filtered simplicial sets for some regular cardinal $\kappa$. Then for any small $\infty$-category $\calC$, we have a canonical equivalence
$\calP^{\calK'}_{\calK}(\calC) \simeq \Ind_{\kappa}(\calC)$ (Proposition \ref{intprop}).
\end{example}

\begin{example}
Let $\calK$ denote the collection of all $\kappa$-small simplicial sets for some regular cardinal $\kappa$, and let $\calK'$ be the class of all small simplicial sets. Let $\calC$ be a small $\infty$-category which admits $\kappa$-small colimits. Then we have a canonical equivalence
$\calP^{\calK'}_{\calK}(\calC) \simeq \Ind_{\kappa}(\calC)$. This follows from
Theorem \ref{pretop} and Proposition \ref{sumatch}.
\end{example}

\begin{example}
Let $\calK = \emptyset$, and let $\calK' = \{ \Idem \}$, where $\Idem$ is the simplicial set defined
in \S \ref{retrus}. Then, for any $\infty$-category $\calC$, $\calP^{\calK'}_{\calK}(\calC)$ is
an idempotent competion of $\calC$.
\end{example}

\begin{corollary}
Let $\calK \subseteq \calK'$ be classes of simplicial sets. Let $\widehat{\Cat}_{\infty}$ denote the $\infty$-category of $($not necessarily small$)$ $\infty$-categories, let
$\widehat{\Cat}_{\infty}^{\calK}$ denote the subcategory spanned by those $\infty$-categories
which admit $\calK$-indexed colimits and those functors which preserve $\calK$-indexed colimits, and let $\widehat{\Cat}_{\infty}^{\calK'}$ be defined likewise. Then the inclusion
$$ \widehat{ \Cat}_{\infty}^{\calK'} \subseteq \widehat{ \Cat}_{\infty}^{\calK}$$
admits a left adjoint, given by $\calC \mapsto \calP^{\calK'}_{\calK}(\calC)$.
\end{corollary}

\begin{proof}
Combine Proposition \ref{cupper1} with Proposition \ref{sumpytump}.
\end{proof}

We conclude this section by noting the following transitivity property of the construction
$\calC \mapsto \calP^{\calK}_{\calR}(\calC)$:

\begin{proposition}\label{transit1}
Let $\calK \subseteq \calK'$ be collections of simplicial sets and 
$\calC_1, \ldots, \calC_n$ be a sequence of $\infty$-categories. For
$1 \leq i \leq n$, let $\calR_i$ be a collection of diagrams $\{ \overline{p}_{\alpha}: K^{\triangleright}_{\alpha} \rightarrow \calC_i \}$, where each $K_{\alpha}$ belongs to $\calK$, and let
$\calR'_i$ denote the collection of all colimit diagrams $\{ \overline{q}_{\alpha}: K^{\triangleright}_{\alpha} \rightarrow \calP^{\calK}_{\calR_i}( \calC_i)  \}$ such that $K_{\alpha} \in \calK$.
Then the canonical map
$$ \calP^{\calK'}_{ \calR_1 \boxtimes \ldots \boxtimes \calR_n}( \calC_1 \times \ldots \times \calC_n)
\rightarrow \calP^{\calK'}_{ \calR'_1 \boxtimes \ldots \boxtimes \calR'_{n} }( \calP^{\calK}_{\calR_1}( \calC_1) \times \ldots \times \calP^{\calK}_{ \calR_n}(\calC_n) )$$
is an equivalence of $\infty$-categories. Here $\calR_1 \boxtimes \ldots \boxtimes \calR_n$ denotes
the collection of all diagrams of the form
$$ K_{\alpha}^{\triangleright} \stackrel{ \overline{p}_{\alpha}}{\rightarrow}
\calC_i \simeq \{ C_1 \} \times \ldots \times \{C_{i-1} \} \times \calC_i \times \ldots \times \{ C_n \}
\subseteq \calC_1 \times \ldots \times \calC_n$$
where $\overline{p}_{\alpha} \in \calR_i$ and $C_{j}$ is an object of $\calC_j$ for $j \neq i$, and
the collection $\calR'_1 \boxtimes \ldots \boxtimes \calR'_{n}$ is defined likewise.
\end{proposition}

\begin{proof}
Let $\calD$ be an $\infty$-category which admits $\calK'$-indexed colimits. It will suffice to show that
the functor
$$ \Fun_{\calK'} (  \calP^{\calK'}_{ \calR'_1 \boxtimes \ldots \boxtimes \calR'_{n} }( \calP^{\calK}_{\calR_1}( \calC_1) \times \ldots \times \calP^{\calK}_{ \calR_n}(\calC_n) ), \calD) 
\rightarrow \Fun_{\calK'}(  \calP^{\calK'}_{ \calR_1 \boxtimes \ldots \boxtimes \calR_n}( \calC_1 \times \ldots \times \calC_n), \calD)$$ is an equivalence of $\infty$-categories. Unwinding the definitions, we are reduced to proving that the functor
$$ \phi: \Fun_{ \calR'_1 \boxtimes \ldots \boxtimes \calR'_n}(
\calP^{\calK}_{ \calR_1}( \calC_1) \times \ldots \times \calP^{\calK}_{ \calR_n}(\calC_n), \calD)
\rightarrow \Fun_{ \calR \boxtimes \ldots \boxtimes \calR_n}( \calC_1 \times \ldots \times \calC_n, \calD)$$ is an equivalence of $\infty$-categories.
The proof goes by induction
on $n$. If $n = 0$, then both sides are equivalent to $\calD$ and there is nothing to prove.
If $n > 0$, then set $\calD' = \Fun_{\calR_{n}}( \calC_n, \calD)$ and $\calD'' = \Fun_{ \calK}(
\calP^{\calK}_{\calR_1}(\calC_n), \calD)$. Proposition \ref{cupper1} implies that the canonical map
$\calD'' \rightarrow \calD'$ is an equivalence of $\infty$-categories. We can identify $\phi$ with
the functor
$$ 
 \Fun_{\calR'_1 \boxtimes \ldots \boxtimes \calR'_{n-1}} ( \calP^{\calK}_{\calR_1}( \calC)
 \times \ldots \times \calP^{\calK}_{\calR_{n-1}}(\calC), \calD'') \rightarrow
 \Fun_{ \calR_1 \boxtimes \ldots \boxtimes \calR_{n-1}}( \calC_1 \times \ldots \times \calC_{n-1}, \calD').$$
The desired result now follows from the inductive hypothesis.
\end{proof}

