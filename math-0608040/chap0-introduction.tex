\maketitle

\section*{Introduction}\label{intro}

\setcounter{theorem}{0}

Let $X$ be a nice topological space (for example, a CW complex). One goal of algebraic topology is to study the topology of $X$ by means of algebraic invariants, such as the singular cohomology groups
$\HH^{n}(X;G)$ of $X$ with coefficients in an abelian group $G$. These cohomology groups have proven to be an extremely useful tool, due largely to the fact that they enjoy excellent formal properties (which have been axiomatized by Eilenberg and Steenrod, see \cite{eilenbergsteenrod}), and the fact that they tend to be very computable. However, the usual definition of $\HH^{n}(X;G)$ in terms of singular $G$-valued cochains on $X$ is perhaps somewhat unenlightening. This raises the following question: can we understand the cohomology group $\HH^{n}(X;G)$ in more conceptual terms?

As a first step toward answering this question, we observe that $\HH^{n}(X;G)$ is a {\em representable} functor of $X$. That is, there exists an {\it Eilenberg-MacLane space} $K(G,n)$
and a universal cohomology class $\eta \in \HH^{n}( K(G,n); G)$ such that, for any nice topological space $X$, pullback of $\eta$ determines a bijection
$$ [X, K(G,n)] \rightarrow \HH^{n}(X;G).$$
Here $[X, K(G,n)]$ denotes the set of homotopy classes of maps from $X$ to $K(G,n)$.
The space $K(G,n)$ can be characterized up to homotopy equivalence by the above property, or by the the formula
$$ \pi_{k} K(G,n) \simeq \begin{cases} \ast & \text{if } k \neq n \\
G & \text{if } k = n. \end{cases}$$

In the case $n=1$, we can be more concrete. An Eilenberg MacLane space $K(G,1)$
is called a {\it classifying space} for $G$, and is typically denoted by $BG$. The universal cover
of $BG$ is a contractible space $EG$, which carries a free action of the group $G$ by covering transformations. We have a quotient map $\pi: EG \rightarrow BG$.
Each fiber of $\pi$ is a discrete topological space, on which the group $G$ acts simply transitively. We can summarize the situation by saying that $EG$ is a {\it $G$-torsor} over the classifying space $BG$.
For every continuous map $X \rightarrow BG$, the fiber product $\widetilde{X}: EG \times_{BG} X$ has the structure of a $G$-torsor on $X$: that is, it is a space endowed with a free action of $G$ and
a homeomorphism $\widetilde{X} / G \simeq X$. This construction determines a map
from $[X, BG]$ to the set of isomorphism classes of $G$-torsors on $X$. If $X$ is a well-behaved space (such as a CW complex), then this map is a bijection. We therefore have (at least) three different ways of thinking about a cohomology class $\eta \in \HH^{1}(X; G)$:
\begin{itemize}
\item[$(1)$] As a $G$-valued singular cocycle on $X$, which is well-defined up to coboundaries.
\item[$(2)$] As a continuous map $X \rightarrow BG$, which is well-defined up to homotopy.
\item[$(3)$] As a $G$-torsor on $X$, which is well-defined up to isomorphism.
\end{itemize}
These three points of view are equivalent if the space $X$ is sufficiently nice.
However, they are generally quite different from one another. The singular cohomology of a space $X$ is constructed using continuous maps from simplices $\Delta^k$ into $X$. If there are not many maps {\em into} $X$ (for example if every path in $X$ is constant), then we cannot expect singular cohomology to tell us very much about $X$. The second definition
uses maps from $X$ into the classifying space $BG$, which
(ultimately) relies on the existence of continuous real-valued
functions on $X$. If $X$ does not admit many real-valued
functions, then the set of homotopy classes $[X, BG]$ is also not a very useful invariant.
For such spaces, the third approach is the most powerful: there is a good theory of
$G$-torsors on an arbitrary topological space $X$. 

There is another reason for thinking about $\HH^{1}(X;G)$ in the language of $G$-torsors: it continues to make sense in situations where the traditional ideas of topology break down. If $\widetilde{X}$ is a $G$-torsor on a topological space $X$, then the projection map
$\widetilde{X} \rightarrow X$ is a local homeomorphism; we may therefore identify $\widetilde{X}$
with a sheaf of sets $\calF$ on $X$. The action of $G$ on $\widetilde{X}$ determines an action of
$G$ on $\calF$. The sheaf $\calF$ (with its $G$-action) and the space $\widetilde{X}$ (with its $G$-action) determine each other, up to canonical isomorphism. Consequently, we can formulate
the definition of a $G$-torsor in terms of the category $\Shv_{\Set}(X)$ of sheaves of sets
on $X$, without ever mentioning the topological space $X$ itself. The same definition makes
sense in any category which bears a sufficiently strong resemblance to the category of sheaves on a topological space: for example, in any {\em Grothendieck topos}. This observation allows us to construct a theory of torsors in a variety of nonstandard contexts, such as the \'{e}tale topology of algebraic varieties (see \cite{SGA}).

Describing the cohomology of $X$ in terms of the sheaf theory of $X$ has still another advantage, which comes into play even when the space $X$ is assumed to be a CW complex. For a general space
$X$, isomorphism classes of $G$-torsors on $X$ are classified not by the singular cohomology
$\HH^1_{\text{sing}}(X;G)$, but by the sheaf cohomology $\HH^{1}_{\text{sheaf}}(X; \calG)$ of
$X$ with coefficients in the constant sheaf $\calG$ associated to $G$. This sheaf cohomology is defined more generally for {\em any} sheaf of groups $\calG$ on $X$.
Moreover, we have a conceptual interpretation of $\HH^{1}_{\text{sheaf}}(X; \calG)$ in general: it classifies
$\calG$-torsors on $X$ (that is, sheaves $\calF$ on $X$ which carry an action of $\calG$ and locally admit a $\calG$-equivariant isomorphism $\calF \simeq \calG$) up to isomorphism. The general formalism of sheaf cohomology is extremely useful, even if we are interested only in the case where $X$ is a nice topological space: it includes, for example, the theory of cohomology with coefficients in a local system on $X$.

Let us now attempt to obtain a similar interpretation for cohomology classes $\eta \in \HH^{2}(X;G)$. 
What should play the role of a $G$-torsor in this case?
To answer this question, we return to the situation where $X$ is a CW complex, so that
$\eta$ can be identified with a continuous map $X \rightarrow K(G,2)$. 
We can think of
$K(G,2)$ as the classifying space of a group: not the discrete group $G$, but instead the classifying space $BG$ (which, if built in a sufficiently careful way, comes equipped with the structure of a topological abelian group). Namely, we can identify $K(G,2)$ with the quotient
$E/BG$, where $E$ is a contractible space with a free action of $BG$.
Any cohomology class $\eta \in \HH^{2}(X;G)$ determines a map $X \rightarrow K(G,2)$
(which is well-defined up to homotopy), and we can form the pullback $\widetilde{X} = E \times_{BG} X$. We now think of $\widetilde{X}$ as a torsor over $X$: not for the discrete group $G$, but instead for its classifying space $BG$.

To complete the analogy with our analysis in the case $n=1$, we would like to interpret
the fibration $\widetilde{X} \rightarrow X$ as defining some kind of sheaf $\calF$ on the space $X$.
This sheaf $\calF$ should have the property that for each $x \in X$, the stalk
$\calF_x$ can be identified with the fiber $\widetilde{X}_x \simeq BG$. Since the space $BG$ is not discrete (or homotopy equivalent to a discrete space), the situation cannot be adequately described in the usual language of set-valued sheaves. However, the classifying space $BG$ is 
{\em almost} discrete: since the homotopy groups $\pi_i BG$ vanish for $i > 1$, we
can recover $BG$ (up to homotopy equivalence) from its fundamental groupoid.
This suggests that we might try to think about $\calF$ as a ``groupoid-valued sheaf'' on $X$,
or a {\it stack}\index{gen}{stack} (in groupoids) on $X$.

\begin{remark2}\index{gen}{gerbe}
The condition that each stalk $\calF_x$ be equivalent to a classifying space $BG$
can be summarized by saying that $\calF$ is a {\it gerbe} on $X$: more precisely, it is a
gerbe banded by the constant sheaf $\calG$ associated to $G$.
We refer the reader to \cite{giraud} for an explanation of this terminology, and a proof that
such gerbes are indeed classified by the sheaf cohomology group $\HH^{2}_{\text{sheaf}}(X;\calG)$. 
\end{remark2}

For larger values of $n$, even the language of stacks is not sufficient to describe the nature of the sheaf $\calF$ associated to the fibration $\widetilde{X} \rightarrow X$. To address the situation,
Grothendieck proposed (in his infamous letter to Quillen; see \cite{pursuing}) that there
should be a theory of {\it $n$-stacks} on $X$, for every integer $n \geq 0$. Moreover, for every sheaf of abelian groups $\calG$ on $X$, the cohomology group
$\HH^{n+1}_{\text{sheaf}}(X;\calG)$ should have an interpreation as classifying a special type
of $n$-stack: namely, the class of $n$-gerbes banded by $\calG$ (for a discussion in the case $n=2$, we refer the reader to \cite{breen}; we will discuss the general case in \S \ref{chmdim}).
In the special case where the space $X$ is a point (and where we restrict our attention to
$n$-stacks in groupoids), the theory of $n$-stacks on $X$ should recover the classical homotopy theory of {\it $n$-types}: that is, CW complexes $Z$ such that the homotopy groups
$\pi_{i}(Z,z)$ vanish for $i > n$ (and every base point $z \in Z$). More generally, we should think of an $n$-stack (in groupoids) on a general space $X$ as a ``sheaf of $n$-types'' on $X$. 

When $n=0$, an $n$-stack on a topological space $X$ simply means a sheaf of sets on $X$.
The collection of all such sheaves can be organized into a category $\Shv_{\Set}(X)$, and this category is a prototypical example of a {\it Grothendieck topos}. The main goal of this book is to obtain an analogous understanding of the situation for $n > 0$. More precisely, we would like answers to the following questions:
\begin{itemize}
\item[$(Q1)$] Given a topological space $X$, what should we mean by a ``sheaf of $n$-types'' on $X$?

\item[$(Q2)$] Let $\Shv_{\leq n}(X)$ denote the collection of all sheaves of $n$-types on $X$.
What sort of a mathematical object is $\Shv_{\leq n}(X)$?

\item[$(Q3)$] What special features (if any) does $\Shv_{\leq n}(X)$ possess?
\end{itemize}

Our answers to questions $(Q2)$ and $(Q3)$ may be summarized as follows
(our answer to $(Q1)$ is more difficult to summarize, and we will avoid discussing it for the moment):

\begin{itemize}
\item[$(A2)$] The collection $\Shv_{\leq n}(X)$ has the structure of an
{\it $\infty$-category}.
\item[$(A3)$] The $\infty$-category $\Shv_{\leq n}(X)$ is an example of
an {\it $(n+1)$-topos}: that is, an $\infty$-category which satisfies
higher categorical analogues of Giraud's axioms for Grothendieck topoi
(see Theorem \ref{nchar}).\index{gen}{topos!$n$}\index{gen}{$n$-topos}
\end{itemize}

\begin{remark2}
Grothendieck's vision has been realized in various ways, thanks to the work of a number of mathematicians (most notably Brown, Joyal, and Jardine: see for example \cite{jardine}), and their work can also be used to provide answers to questions $(Q1)$ and $(Q2)$ (for more details, we refer the reader to \S \ref{hyperstacks}). Question $(Q3)$ has also been addressed (at least in limiting case $n = \infty$) by To\"{e}n and Vezzosi (see \cite{toen}) and in published work of Rezk. 
\end{remark2}

To provide more complete versions of the answers $(A2)$ and $(A3)$, we will need to develop the language of {\em higher category theory}. This is generally regarded as a technical and forbidding subject, but fortunately we will only need a small fragment of it. More precisely, we will need a theory of
{\it $(\infty,1)$-categories}: higher categories $\calC$ for which the $k$-morphisms of $\calC$ are required to be invertible for $k > 1$. In \S \ref{chap1}, we will present such a theory:
namely, one can define an {\it $\infty$-category}\index{gen}{$\infty$-category} to be a simplicial set
satisfying a weakened version of the Kan extension condition (see Definition \ref{qqcc}; simplicial sets satisfying this condition are also called {\it weak Kan complexes} or {\it quasicategories} in the literature). 
Our intention is that \S \ref{chap1} can be used as a short ``user's guide'' to $\infty$-categories:
it contains many of the basic definitions, and explains how many ideas from classical category theory can be extended to the $\infty$-categorical context. To simplify the exposition, we have deferred many proofs until later chapters, which contain a more thorough account of the theory.
The hope is that \S \ref{chap1} will be useful to readers who want to get the flavor of the subject, without becoming overwhelmed by technical details.

In \S \ref{chap2} we will shift our focus slightly: rather than study individual examples of $\infty$-categories, we consider {\em families} of $\infty$-categories $\{ \calC_{D} \}_{ D \in \calD}$ parametrized by the objects of another $\infty$-category $\calD$. We might expect such a family to be given by a map of $\infty$-categories $p: \calC \rightarrow \calD$: given such a map, we can then define each $\calC_{D}$ to be
the fiber product $\calC \times_{\calD} \{D\}$. This definition behaves poorly in general
(for example, the fibers $\calC_{D}$ need not be $\infty$-categories), but it behaves well if
we make suitable assumptions on the map $p$. Our goal in \S \ref{chap2} is to study some of these assumptions in detail, and show that they lead to a good {\em relative} version of higher category theory.

One motivation for the theory of $\infty$-categories is that it arises naturally in addressing questions like $(Q2)$ above. More precisely, given a collection of mathematical objects $\{ \calF_{\alpha} \}$ whose definition has a homotopy-theoretic flavor (like $n$-stacks on a topological space $X$), we can often organize the collection $\{ \calF_{\alpha} \}$ into an $\infty$-category
(in other words, we can find an $\infty$-category $\calC$ whose vertices correspond to the
objects $\calF_{\alpha}$). Another important example is provided by higher category theory itself: 
the collection of all $\infty$-categories can itself be organized into a (very large) $\infty$-category, which we will denote by $\Cat_{\infty}$. Our goal in \S \ref{chap4} is to study $\Cat_{\infty}$ and to show that it can be characterized by a universal property: namely, functors $\chi: \calD \rightarrow \Cat_{\infty}$ are classified, up to equivalence, by certain kinds of fibrations $\calC \rightarrow \calD$ (see Theorem \ref{straightthm} for a more precise statement). Roughly speaking, this correspondence assigns to a fibration
$\calC \rightarrow \calD$ the functor $\chi$ given by the formula $\chi(D) = \calC \times_{\calD} \{D\}$. 

Classically, category theory is a useful tool not so much because of the light it sheds on any particular
mathematical discipline, but instead because categories are so ubiquitous:
mathematical objects in many different settings (sets, groups, smooth manifolds, etcetera) can be organized into categories. Moreover, many elementary mathematical concepts can be described in purely categorical terms, and therefore make sense in each of these settings. For example, we can form products of sets, groups, and smooth manifolds: each of these notions can simply be described as a Cartesian product in the relevant category. Cartesian products are a special case of the more general notion of {\em limit}\index{gen}{limit}, which plays a central role in classical category theory. In \S \ref{chap3}, we will make a systematic study of limits (and the dual theory of colimits) in the $\infty$-categorical setting. We will also introduce the more general theory of {\em Kan extensions}, in both absolute and relative versions; this theory plays a key technical role throughout the later parts of the book.

In some sense, the material of \S \ref{chap1} through \S \ref{chap3} of this book should be regarded as purely formal. Our main results can be summarized as follows: there exists a reasonable theory of $\infty$-categories, and it behaves in more or less the same way as the theory of ordinary categories. Many of the ideas that we introduce are straightforward generalizations of their ordinary counterparts (though proofs in the $\infty$-categorical setting often require a bit of dexterity in manipulating simplicial sets), which will be familiar to mathematicians who are acquainted with ordinary category theory (as presented, for example, in \cite{maclane}). In \S \ref{chap5}, we introduce $\infty$-categorical analogues of more sophisticated concepts from classical category theory: presheaves, $\Pro$ and $\Ind$-categories, accessible and presentable categories, and localizations. The main theme is that most of the $\infty$-categories which appear ``in nature'' are large, but are determined by small subcategories. Taking careful advantage of this fact will allow us to deduce a number of pleasant results, such as an $\infty$-categorical version of the adjoint functor theorem (Corollary \ref{adjointfunctor}).

In \S \ref{chap6} we come to the heart of the book: the study of {\it $\infty$-topoi}, the $\infty$-categorical analogues of Grothendieck topoi. The theory of $\infty$-topoi is our answer
to the question $(Q3)$ in the limiting case $n = \infty$ (we will also study the analogous notion
for finite values of $n$). Our main result is an analogue of Giraud's theorem, which asserts the equivalence of ``extrinsic'' and ``intrinsic'' approaches to the subject (Theorem \ref{mainchar}). 
Roughly speaking, an $\infty$-topos is an $\infty$-category which ``looks like'' the $\infty$-category of all homotopy types. We will show that this intuition is justified in the sense that it is possible to reconstruct a large portion of classical homotopy theory inside an arbitrary $\infty$-topos. In other words, an $\infty$-topos
is a world in which one can ``do'' homotopy theory (much as an ordinary topos can be regarded as a world in which one can ``do'' other types of mathematics).

%If $X$ is any topological space, then there is an $\infty$-topos $\Shv(X)$ consisting of
%``sheaves of homotopy types'' on $X$. If $X$ is Hausdorff (or more generally if $X$ is
%{\em sober}: see \S \ref{0topoi}), then $X$ can be recovered from $\Shv(X)$ up to homeomorphism.
%This leads to another point of view on the theory of $\infty$-topoi: we can view an $\infty$-topos
%$\calX$ as a kind of generalized topological space. This extra generality is useful from multiple points of view:

%\begin{itemize}
%\item[$(1)$] We sometimes wish to study $\infty$-topoi which are not of the form $\Shv(X)$, for any topological space $X$. For example, the collection of \'{e}tale sheaves (of homotopy types) on an algebraic variety is an $\infty$-topos, but cannot be realized as the $\infty$-topos of
%sheaves on a topological space. Classical homotopy theory can be used to provide another example. The $\infty$-topos $\Shv(X)$ of sheaves on a topological space $X$ contains a subcategory $\Shv_0(X)$, consisting of {\em locally constant} sheaves (of homotopy types) on $X$. If $X$ is a
%sufficiently nice space (such as a CW complex), then $\Shv_0(X)$ is again an $\infty$-topos. 
%We can think of $\Shv_0(X)$ as encoding the homotopy theory of spaces {\em fibered over $X$}
%(see for example \cite{parahomotopy}).

%\item[$(2)$] To a given $\infty$-topos
%There is usually more than one $\infty$-topos associated to a given topological
%space $X$. In addition 
%To a given topological space $X$, one can often associate more than one
%$\infty$-
% \S \ref{chap6sec5}
%\end{itemize}

In \S \ref{chap7} we will discuss the relationship between our theory of $\infty$-topoi and ideas from classical topology. We will show that, if $X$ is a paracompact space, then the $\infty$-topos of ``sheaves of homotopy types'' on $X$ can be interpreted in terms of the classical homotopy theory of spaces {\em over} $X$. Another main theme is that various ideas from geometric topology (such as dimension theory and shape theory) can be described naturally in the language of $\infty$-topoi. We will also formulate and prove ``nonabelian'' generalizations of classical cohomological results, such as Grothendieck's vanishing theorem for the cohomology of Noetherian topological spaces, and the proper base change theorem.

\subsection*{Prerequisites and Suggested Reading}

We have made an effort to keep this book as self-contained as possible. The main prerequisite is familiarity with the classical homotopy theory of simplicial sets (good references include \cite{maysimp} and \cite{goerssjardine}; we have also provided a very brief review in \S \ref{simpset}). The remaining material that we need is either described in the appendix, or developed in the body of the text. However, our exposition of this background material is often somewhat terse, and the reader might benefit from consulting other treatments of the same ideas. Some suggestions for further reading are listed below.

\begin{warning2}
The list of references below is woefully incomplete. We have not attempted, either here or in the body of the text, to give a comprehensive survey of the literature on higher category theory. We have also not attempted to trace all of the ideas presented to their origins, or to present a detailed history of the subject. Many of the topics presented in this book have appeared elsewhere, or belong to the mathematical folklore; it should not be assumed that uncredited results are due to the author.
\end{warning2}

\begin{itemize}
\item {\bf Classical Category Theory:} Large portions of this book are devoted to providing $\infty$-categorical generalizations of the basic notions of category theory. A good reference for many of the concepts we use is MacLane's book \cite{maclane} (see also \cite{adamek} and \cite{makkai} for some of the more advanced material of \S \ref{chap5}).

\item {\bf Classical Topos Theory:} Our main goal in this book is to describe an $\infty$-categorical version of
topos theory. Familiarity with classical topos theory is not strictly necessary (we will define all of the relevant concepts as we need them), but will certainly be helpful. Good references include \cite{SGA} and \cite{where}.

\item {\bf Model Categories:} Quillen's theory of model categories provides a useful tool for studying
specific examples of $\infty$-categories, including the theory of $\infty$-categories itself.
We will summarize the theory of model categories in \S \ref{appmodelcat}; more complete
references include \cite{hovey}, \cite{hirschhorn}, and \cite{goerssjardine}.

\item {\bf Higher Category Theory:} There are many approaches to the theory of higher categories, some of which look quite different from the theory presented in this book. Several other possibilities are presented in the survey article \cite{leinster}. More detailed accounts can be found in \cite{leinster2}, 
\cite{simpson2}, and \cite{tamsamani}. 

In this book, we consider only {\it $(\infty,1)$-categories}: that is, higher categories in which all $k$-morphisms are assumed to be invertible for $k > 1$. There are a number of convenient ways to formalize this idea: via simplicial categories (see for example \cite{dwyerkan} and \cite{bergner}), via
Segal categories (\cite{simpson2}), via complete Segal spaces (\cite{completesegal}), or via
the theory of $\infty$-categories presented in this book (other references include \cite{joyalpub},
\cite{joyalnotpub}, \cite{nichols}, and \cite{quasicat}). The relationship between these various approaches is described in \cite{bergner2}, and an axiomatic framework which encompasses all of them is described in \cite{toenchar}.

\item {\bf Higher Topos Theory:} The idea of studying a topological space $X$ via the theory of
sheaves of $n$-types (or {\it $n$-stacks}) on $X$ goes back at least to Grothendieck
(\cite{pursuing}), and has been taken up a number of times in recent years. For small values of
$n$, we refer the reader to \cite{giraud}, \cite{street}, \cite{breen}, \cite{joyaltierney}, and \cite{polesello}. For the case $n=\infty$, we refer the reader to \cite{brown}, \cite{jardine}, \cite{hirschowitz}, and \cite{toen2}.
A very readable introduction to some of these ideas can be found in \cite{baezshul}.

Higher topos theory itself can be considered an abstraction of this idea: rather than studying
sheaves of $n$-types on a particular topological space $X$, we instead study general $n$-categories with the same formal properties. This idea has been implemented in the work of To\"{e}n and Vezzosi
(see \cite{toen} and \cite{toenvezz}), resulting in a theory which is essentially equivalent to
the one presented in this book. (A rather different variation on this idea in the case $n=2$ can
be also be found in \cite{ditopoi}.) The subject has also been greatly influenced by the unpublished ideas of Charles Rezk.
\end{itemize}


%\subsection*{Related Works}
%*** now discuss references on the $n$-categorical literature
% also, add some acknowledgements for the reviewer and to the PuP.

%Classically, the non-abelian cohomology $\HH^1(X;G)$ of $X$ with
%coefficients in a possibly non-abelian group $G$ can be understood as the set of isomorphism classes of $G$-torsors over $X$. When $X$ is paracompact, such
%torsors can be classified by homotopy classes of maps from $X$
%into an Eilenberg-MacLane space $K(G,1)$. Note that the group $G$
%and the space $K(G,1)$ are essentially the same piece of data: $G$
%determines $K(G,1)$ up to homotopy equivalence, and conversely $G$
%may be recovered as the fundamental group of $K(G,1)$. More canonically, specifying the group $G$ is equivalent to specifying the space $K(G,1)$ {\em together with a base point};
%the space $K(G,1)$ alone only determines $G$ up to inner
%automorphisms. However, inner automorphisms of $G$ act by the identity
%on $\HH^1(X;G)$, so that $\HH^1(X;G)$ really depends only 
%on $K(G,1)$. This suggests the proper
%coefficients for non-abelian cohomology are not groups, but
%``homotopy types'' (which we regard as purely combinatorial
%entities, represented for example by simplicial complexes). We may
%define the non-abelian cohomology $\HH_{\text{rep}}(X;K)$ of $X$
%with coefficients in an arbitrary space $K$ to be the
%collection of homotopy classes of maps from $X$ into $K$. This
%leads to a good theory whenever $X$ is paracompact. Moreover, we
%can learn a great deal by considering the case where $K$ is not an
%Eilenberg-MacLane space. For example, if $K = \BU \times \Z$ is
%the classifying space for complex K-theory and $X$ is a compact
%Hausdorff space, then $\HH_{\text{rep}}(X;K)$ is the usual complex
%K-theory of $X$, defined as the Grothendieck group of the monoid
%of isomorphism classes of complex vector bundles on $X$.

%When $X$ is not paracompact, we are forced to seek a better way of
%defining $\HH(X;K)$. Given the apparent power and flexibility of
%sheaf-theoretic methods, it is natural to look for some
%generalization of sheaf cohomology, using as coefficients
%``sheaves of homotopy types on $X$.'' This is an old idea, laid
%out by Grothendieck in his vision of a theory of {\it higher
%stacks}. This vision has subsequently been realized in the work
%of various authors (most notably Brown, Joyal, and Jardine; see for example \cite{jardine}), who employ various formalisms based on simplicial
%(pre)sheaves on $X$. The resulting theories
%are essentially equivalent, and we shall refer to them collectively as
%the {\it Brown-Joyal-Jardine theory}. According to the philosophy
%of this approach, if $K$ is a simplicial set, then the cohomology
%of $X$ with coefficients in $K$ is given by
%$$\hat{\HH}(X;K) = \pi_0( \calF(X)),$$ where $\calF$ is a fibrant
%replacement for the constant simplicial (pre)sheaf with value $K$
%on $X$. The process of ``fibrant replacement'' should be regarded as a kind of ``sheafification'': the simplicial presheaf $\calF$ is obtained from the constant (pre)sheaf by forcing it to satisfy a descent condition for arbitrary hypercoverings of open subsets of $X$.

%If $K$ is an Eilenberg-MacLane space $K(G,n)$, the Brown-Joyal-Jardine theory
%recovers the classical sheaf-cohomology group (or set, if $n \leq 1$)
%$\HH^n_{\text{sheaf}}(X;G)$. It follows that if $X$ is
%paracompact and $K$ is an Eilenberg-MacLane space, then there is a natural isomorphism $\hat{\HH}(X;K) \simeq \HH_{\text{rep}}(X;K)$. However, it turns out that $\hat{\HH}(X;K) \neq
%\HH_{\text{rep}}(X;K)$ in general, even when $X$ is paracompact.
%In fact, one can give an example of a compact Hausdorff space for%
%which $\hat{\HH}(X; BU \times \Z)$ does not coincide with the complex
%$K$-theory of $X$. We will proceed on the assumption that the classical $K$-group $K(X)$
%is the ``correct'' answer in this case, and give an alternative to
%the Brown-Joyal-Jardine theory which computes this answer. Our
%alternative is distinguished from the Brown-Joyal-Jardine theory
%by the fact that we require our ``sheaves of spaces'' to satisfy a descent condition only for
%ordinary coverings of a space $X$, rather than for arbitrary hypercoverings.
%Aside from this point we can proceed in the same way,
%setting $\HH(X;K)  = \pi_0( \calF'(X))$, where $\calF'$ is the (simplicial)
%sheaf which is obtained by forcing the ``constant presheaf with
%value $K$'' to satisfy this weaker descent condition. In general,
%$\calF'$ will not satisfy descent for hypercoverings, and
%consequently it will not be equivalent to the simplicial presheaf
%$\calF$ used in the definition of $\hat{\HH}$.

%The resulting theory has the following properties:

%\begin{itemize}
%\item[$(1)$] If $X$ is paracompact, $\HH(X;K)$ may be identified with the
%set of homotopy classes from $X$ into $K$.

%\item[$(2)$] There is a canonical map $\theta: \HH(X;K) \rightarrow \widehat{\HH}(X;K)$.

%\item[$(3)$] If $X$ is a paracompact topological space of finite covering dimension (or a Noetherian topological space of finite Krull dimension), then $\theta$ is an isomorphism.

%\item[$(4)$] If $K$ has only finitely many nonvanishing homotopy groups, then $\theta$ is an isomorphism. In particular, taking $K$ to be an Eilenberg-MacLane space $K(G,n)$, then
%$\HH(X; K(G,n))$ is isomorphic to the sheaf cohomology group $\HH^{n}_{\text{sheaf}}(X;G)$.
%\end{itemize}

%Our theory of higher stacks enjoys good formal properties which
%are not always shared by the Brown-Joyal-Jardine theory; we will
%summarize the situation in \S \ref{versus}. However, these good
%properties come with a price attached. The
%essential difference between $\infty$-stacks (sheaves of spaces which are required to satisfy
%descent only for ordinary coverings) and $\infty$-hyperstacks (sheaves of spaces which are
%required to satisfy descent for arbitrary hypercoverings) is that
%the former can fail to satisfy the Whitehead theorem: one can
%have, for example, a pointed stack $(E,\eta)$ for which
%$\pi_i(E,\eta)$ is a trivial sheaf for all $i \geq 0$, and yet
%$E$ is not ``contractible'' (for the definition of these homotopy
%sheaves, see \S \ref{homotopysheaves}).

%In order to make a thorough comparison of our theory of stacks on
%$X$ and the Brown-Joyal-Jardine theory of hyperstacks on $X$, it
%seems desirable to fit both of them into some larger context. The
%proper framework is provided by the theory $\infty$-topoi. Roughly speaking, an $\infty$-topos is an $\infty$-category that ``looks like''
%the $\infty$-category of $\infty$-stacks on a topological space,
%just as an ordinary topos is supposed to be a category that
%``looks like'' the category of sheaves (of sets) on a topological space. For
%every topological space (or topos) $X$, the
%$\infty$-stacks on $X$ constitute an $\infty$-topos, as do the
%$\infty$-hyperstacks on $X$. However, it is the former
%$\infty$-topos which enjoys a more universal position among
%$\infty$-topoi related to $X$.

%The aim of this book is to construct a theory of $\infty$-topoi which will permit us to make sense of the above discussion, and to illustrate some connections between this theory and classical topology. The ideas involved are fundamentally {\em homotopy-theoretic} in nature, and cannot be adequately described in the language of classical category theory. Consequently, most of this book is concerned with the construction of a suitable theory of {\it higher categories}. The language of higher category theory has many other applications, which we will discuss elsewhere (\cite{DAG}, \cite{elliptic}).


\section*{Terminology}

A few comments on some of the terminology which appears in this book:

\begin{itemize}
\item The word {\em topos} will always mean {\em Grothendieck} topos.\index{gen}{topos}

\item We let $\sSet$ denote the category of simplicial sets. If $J$ is a linearly ordered set, we
let $\Delta^{J}$ denote the simplicial set given by the nerve of $J$, so that
the collection of $n$-simplices of $\Delta^{J}$ can be identified with the collection
of all nondecreasing maps $\{0, \ldots, n \} \rightarrow J$. We will frequently apply this notation
when $J$ is a subset of $\{0, \ldots, n \}$; in this case, we can identify
$\Delta^{J}$ with a subsimplex of the standard $n$-simplex $\Delta^{n}$ (at
least if $J \neq \emptyset$; if $J = \emptyset$ then $\Delta^{J}$ is empty).

\item We will refer to a category $\calC$ as {\it accessible} or {\it presentable} if it is {\it locally accessible} or {\it locally presentable} in the terminology of \cite{makkai}.\index{gen}{accessible!category}\index{gen}{presentable!category}

\item Unless otherwise specified, the term {\it $\infty$-category} will be used to indicate a higher category in which all $n$-morphisms are invertible for $n > 1$.\index{gen}{$\infty$-category}

\item We will study higher category theory in Joyal's setting of {\it quasicategories}. However, we do not always follow Joyal's terminology.\index{gen}{quasicategory} In particular, we will use the term {\it $\infty$-category} to refer to what Joyal calls a {\it quasicategory} (which are, in turn, the same as the {\it weak Kan complex}\index{gen}{Kan complex!weak} of Boardman and Vogt); we will use the terms {\it inner fibration} and {\it inner anodyne map} where Joyal uses {\it mid-fibration} and {\it mid-anodyne map}.\index{gen}{weak Kan complex}

\item Let $n \geq 0$. We will say that a homotopy type $X$ (described by either a topological space or a Kan complex) is {\it $n$-truncated} if the homotopy groups $\pi_{i}(X,x)$ vanish for every point $x \in X$
and every $i > n$. By convention, we say that $X$ is $(-1)$-truncated if it is either empty or (weakly) contractible, and $(-2)$-truncated if $X$ is (weakly) contractible.

\item Let $n \geq 0$. We will say that a homotopy type $X$ (described either by a topological space or a Kan complex) is {\it $n$-connective} if $X$ is nonempty and the homotopy groups $\pi_{i}(X,x)$ vanish for every point $x \in X$ and every integer $i < n$. By convention, we will agree that every homotopy type $X$ is $(-1)$-connective. 

\item More generally, we will say that a map of homotopy types $f: X \rightarrow Y$ is
$n$-truncated ($n$-connective) if the homotopy fibers of $f$ are $n$-truncated ($n$-connective).
\end{itemize}

\begin{remark2}
For $n \geq 1$, a homotopy type $X$ is $n$-connective if and only if it is $(n-1)$-connected (in the usual terminology). In particular, $X$ is $1$-connective if and only if it is path connected.
\end{remark2}

\begin{warning2}
In this book, we will often be concerned with sheaves on a topological space $X$ (or some Grothendieck site) that take values in an $\infty$-category $\calC$. The most ``universal'' case is that in which $\calC$ is the $\infty$-category of $\SSet$ of spaces. Consequently, the term ``sheaf on $X$''
without any other qualifiers will typically refer to a sheaf of spaces on $X$, rather than a sheaf of sets on $X$. We will see that the collection of all $\SSet$-valued sheaves on $X$ can be organized into an
$\infty$-category, which we denote by $\Shv(X)$. In particular, $\Shv(X)$ will not denote the ordinary category of set-valued sheaves on $X$; if we need to consider this latter object, we will denote it
by $\Shv_{\Set}(X)$. 
\end{warning2}

\section*{Acknowledgements}

This book would never have come into existence without the advice and encouragement of many people. In particular, I would like to thank Vigleik Angeltveit, Vladimir Drinfeld, Matt Emerton, John Francis, Andre Henriques, James Parson, David Spivak, and James Wallbridge for many suggestions and corrections which have improved the readability of this book; Andre Joyal, who was kind enough to share with me a preliminary version of his work on the theory of quasi-categories; Charles Rezk, for explaining to me a very conceptual reformulation of the axioms for $\infty$-topoi (which we will describe in \S \ref{magnet}); Bertrand To\"{e}n and Gabriele Vezzosi, for many stimulating conversations about their work (which has considerable overlap with the material treated here); Mike Hopkins, for his advice and support throughout my time as a graduate student; Max Lieblich, for offering encouragement during early stages of this project; Josh Nichols-Barrer, for sharing with me some of his ideas about the foundations of higher category theory, and my editors Anna Pierrehumbert and Vickie Kearn at the Princeton Univesity Press, for helping to make this the best book that it can be. Finally, I would like to thank the American Institute of Mathematics for supporting me throughout the (seemingly endless) process of revising this book.
