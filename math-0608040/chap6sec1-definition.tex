\section{$\infty$-Topoi: Definitions and Characterizations}\label{chap6sec1}
 
\setcounter{theorem}{0}


Before we study the $\infty$-categorical version of topos theory, it seems appropriate to briefly review the classical theory. Recall that a {\it topos} is a category $\calC$ which behaves like the category of sets, or (more generally) the category of sheaves of sets on a topological space. There are several (equivalent) ways of making this idea precise. The following result is proved (in a slightly different form) in \cite{SGA}:

\begin{proposition}\label{toposdefined}\index{gen}{topos}
Let $\calC$ be a category. The following conditions are equivalent:
\begin{itemize}
\item[$(A)$] The category $\calC$ is $($equivalent to$)$ the category of
sheaves of sets on some Grothendieck site. \item[$(B)$] The category $\calC$ is
$($equivalent to$)$ a left-exact localization of the category of presheaves of sets
on some small category $\calC_0$. \item[$(C)$]\index{gen}{Giraud's axioms!for ordinary topoi} Giraud's axioms are satisfied: \begin{itemize}
\item[$(i)$] The category $\calC$ is presentable $($that is, $\calC$ has small colimits and a set of small generators$)$.
\item[$(ii)$] Colimits in $\calC$ are universal. 
\item[$(iii)$] Coproducts in $\calC$ are disjoint. 
\item[$(iv)$] Equivalence relations in $\calC$ are effective.
\end{itemize}
\end{itemize}
\end{proposition}

\begin{definition}\label{def1topos}
A category $\calC$ is called a {\it topos} if it satisfies the equivalent conditions of
Proposition \ref{toposdefined}.
\end{definition}

\begin{remark}
A reader who is unfamiliar with some of the terminology used in the statement of Proposition \ref{toposdefined} should not worry: we will review the meaning of each condition in \S \ref{axgir} as we search for $\infty$-categorical generalizations of axioms $(i)$ through $(iv)$.
\end{remark}

Our goal in this section is to introduce the $\infty$-categorical analogue of Definition \ref{def1topos}.
Proposition \ref{toposdefined} suggests several possible approaches. We begin with the simplest of these:

\begin{definition}\label{itoposdef}\index{gen}{$\infty$-topos}
Let $\calX$ be an $\infty$-category. We will say that $\calX$ is an {\it $\infty$-topos} if there
exists a small $\infty$-category $\calC$ and an accessible left exact localization functor
$\calP(\calC) \rightarrow \calX$. 
\end{definition}

\begin{remark}
Definition \ref{itoposdef} involves an accessibility condition which was not
mentioned in Proposition \ref{toposdefined}. This is because every left exact localization of a
category of {\em set}-valued presheaves is automatically accessible (see Proposition \ref{alltoploc}). We do not know if the corresponding result holds for $\SSet$-valued presheaves. However,
it is true under a suitable hypercompleteness assumption:
see \cite{toenvezz}.
\end{remark}

Adopting Definition \ref{itoposdef} amounts to selecting an {\em extrinsic} approach to higher topos theory: the class of $\infty$-topoi is defined to be the smallest collection of $\infty$-categories which contains $\SSet$ and is stable under certain constructions (left exact localizations and the formation of functor categories). The main objective of this section is to give several reformulations of Definition \ref{def1topos} which have a more intrinsic flavor. Our results may be summarized in the following statement (all our our terminology will be explained later in this section):

\begin{theorem}\label{mainchar}\index{gen}{Giraud's theorem!for $\infty$-topoi}
Let $\calX$ be an $\infty$-category. The following conditions are equivalent:

\begin{itemize}
\item[$(1)$] The $\infty$-category $\calX$ is an $\infty$-topos.

\item[$(2)$] The $\infty$-category $\calX$ is presentable, and for every small simplicial
set $K$ and every natural transformation
$\overline{\alpha}: \overline{p} \rightarrow \overline{q}$ of diagrams
$\overline{p}, \overline{q}: K^{\triangleright} \rightarrow \calX$, the following condition is satisfied:
\begin{itemize} 
\item
If $\overline{q}$ is a colimit diagram and $\alpha = \overline{\alpha} | K$ is a Cartesian transformation, then $\overline{p}$ is a colimit diagram if and only if $\overline{\alpha}$ is a Cartesian transformation.
\end{itemize}

\item[$(3)$] The $\infty$-category $\calX$ satisfies the following $\infty$-categorical analogues of Giraud's axioms:
\begin{itemize}
\item[$(i)$] The $\infty$-category $\calX$ is presentable.
\item[$(ii)$] Colimits in $\calX$ are universal.
\item[$(iii)$] Coproducts in $\calX$ are disjoint.
\item[$(iv)$] Every groupoid object of $\calX$ is effective.
\end{itemize}

\end{itemize}
\end{theorem}

We will review the meanings of conditions $(i)$ through $(iv)$ in \S \ref{axgir} and \S \ref{gengroup}.  In \S \ref{magnet} we will give several equivalent formulations of $(2)$, and prove the implications
$(1) \Rightarrow (2) \Rightarrow (3)$. The implication $(3) \Rightarrow (1)$ is the most difficult; we will give the proof in \S \ref{proofgiraud} after establishing a crucial technical lemma in \S \ref{freegroup}. Finally, in \S \ref{rezk2} we will establish yet another characterization of $\infty$-topoi, based on the theory of classifying objects.

\begin{remark}
The characterization of the class of $\infty$-topoi given by part $(2)$ of Theorem \ref{mainchar} is due to Rezk, as are many of the ideas presented in \S \ref{magnet}.
\end{remark}

The equivalence $(1) \Leftrightarrow (3)$ of Theorem \ref{mainchar} can be viewed as an $\infty$-categorical analogue of the equivalence $(B) \Leftrightarrow (C)$ in Proposition \ref{toposdefined}. 
It is natural to ask if there is also some equivalent of the characterization $(A)$. To put the question another way: given a small $\infty$-category $\calC$, does there exist some natural description of the class of all left-exact localizations of $\calC$? Experience with classical topos theory suggests that we might try to characterize such localizations in terms of {\em Grothendieck topologies} on $\calC$. We will introduce a theory of Grothendieck topologies on $\infty$-categories in \S \ref{cough}, and show that every Grothendieck topology on $\calC$ determines a left-exact localization of $\calP(\calC)$. However, it turns out that not every $\infty$-topos arises via this construction. This raises a natural question: is it possible to give an explicit description of {\em all} left-exact localizations of $\calP(\calC)$, perhaps in terms of some more refined theory of Grothendieck topologies? We will give a partial answer to this question in \S \ref{chap6sec5}.

\subsection{Giraud's Axioms in the $\infty$-Categorical Setting}\label{axgir}\index{gen}{Giraud's axioms!for $\infty$-topoi}

Our goal in this section is to formulate higher-categorical analogues of the conditions $(i)$ through $(iv)$ which appear in Proposition \ref{toposdefined}. 
We consider each axiom in turn. In each case, our objective is to find an analogous axiom which makes sense in the setting of $\infty$-categories, and is satisfied by the $\infty$-category $\SSet$
of spaces. 

\begin{itemize}
\item[$(i)$] The category $\calC$ is presentable.
\end{itemize}

The generalization to the case where $\calC$ is a $\infty$-category is obvious: we should merely require $\calC$ to be a presentable $\infty$-category in the sense of Definition \ref{presdef}.
According to Example \ref{spacesarepresentable}, this condition is satisfied when $\calC$ is the $\infty$-category of spaces.

\begin{itemize}\index{gen}{colimit!universal}
\item[$(ii)$] Colimits in $\calC$ are universal.
\end{itemize}

Let us first recall the meaning of this condition in classical category theory. If the axiom $(i)$ is satisfied, then $\calC$ is presentable and therefore admits all (small) limits and colimits. In particular, every diagram
$$ X \rightarrow S \stackrel{f}{\leftarrow} T$$
has a limit $X_{T} = X \times_{S} T$.
This construction determines
a functor
$$ f^{\ast}: \calC_{/S} \rightarrow \calC_{/T}$$
$$X \mapsto X_{T},$$
which is a right adjoint to the functor given by composition with $f$.

We say that colimits in $\calC$ are {\it universal} if the functor $f^{\ast}$ is colimit-preserving, for
every map $f: T \rightarrow S$ in $\calC$. (In other words, colimits are universal in $\calC$ if any colimit in $\calC$ {\em remains} a colimit in $\calC$ after pulling back along a morphism $T \rightarrow S$.)

Let us now attempt to make this notion precise in the setting of an arbitrary $\infty$-category
$\calC$. Let $\calO_{\calC} = \Fun(\Delta^1, \calC)$\index{not}{OC@$\calO_{\calC}$}, and let $p: \calO_{\calC} \rightarrow \calC$ be given by evaluation at
$\{1\} \subseteq \Delta^1$. Corollary \ref{tweezegork} implies that $p$ is a coCartesian fibration.

\begin{lemma}\label{charpull}
Let $\calX$ be an $\infty$-category and let $p: \calO_{\calX} \rightarrow \calX$ be defined as above. Let
$F$ be a morphism in $\calO_{\calX}$, corresponding to a diagram $\sigma: \Delta^1 \times \Delta^1 \simeq (\Lambda^2_2)^{\triangleleft} \rightarrow \calX$, which we will denote by
$$ \xymatrix{ X' \ar[r]^{f'} \ar[d] & Y' \ar[d]^{g} \\
X \ar[r]^{f} & Y}$$
Then $F$ is $p$-Cartesian if and only if the above diagram is a pullback in $\calX$.
In particular, $p$ is a Cartesian fibration if and only if the $\infty$-category $\calX$ admits pullbacks.
\end{lemma}

\begin{proof}
For every simplicial set $K$, let $K^{+}$ denote the full simplicial subset
of $(K \star \{x\} \star \{y\}) \times \Delta^1$ spanned by all of the vertices except
$(x,0)$, and define a simplicial set $\calC$ by setting 
$$\Fun(K, \calC) = \{ m: K^{+} \rightarrow \calX : m| (\{x\} \star \{y\}) \times \{1\} = f,
m| \{y\} \times \Delta^1 = g \}.$$
We observe that we have a commutative diagram
$$ \xymatrix{ \calC \ar[r] \ar[d] & (\calO_{\calX})_{/g} \ar[d] \\
\calX_{/f} \ar[r] & \calX_{/Y'} }$$
which induces a map $q: \calC \rightarrow (\calO_{\calX})_{/g} \times_{ \calX_{/Y'} } \calX_{/f}$.
We first claim that $q$ is a trivial fibration. Unwinding the definitions, we observe that the right lifting property of $q$ with respect to an inclusion $\bd \Delta^n \subseteq \Delta^n$ follows from the extension property of $\calX$ with respect to $\Lambda^{n+2}_{n+1}$, which follows in turn from our assumption that $\calX$ is an $\infty$-category.

The inclusion $K^{+} \subseteq K \times \Delta^1$ induces a projection $q': ( \calO_{\calX})_{/F} \rightarrow \calC$ which fits into a pullback diagram
$$ \xymatrix{ ( \calO_{\calX})_{/F} \ar[r] \ar[d] & \calC \ar[d]^{g} \\
\calX_{/ \sigma } \ar[r]^{q''} & \calX_{/ \sigma | \Lambda^2_2}. }$$
It follows that $q'$ is a right fibration, and that $q'$ is trivial if $\sigma$ is a pullback
diagram. Conversely, we observe that $( \Lambda^2_2)^{\triangleleft}$ is a retract of
$(\Delta^{0})^{+}$, so that the map $g$ is surjective on vertices. Consequently, if
$q'$ is a trivial fibration, then the fibers of $q''$ are contractible, so that $q''$ is a trivial fibration
(Lemma \ref{toothie}) and $\sigma$ is a pullback diagram.

By definition, $F$ is $p$-Cartesian if and only if the composition
$$q \circ q': (\calO_{\calX})_{/F} \rightarrow (\calO_{\calX})_{/g} \times_{ \calX_{/Y'} } \calX_{/f}$$
is a trivial fibration. Since $q$ is a trivial fibration and $q'$ is a right fibration, this is also equivalent to the assertion that $q'$ is a trivial fibration (Lemma \ref{toothie}).
\end{proof}

Now suppose that $\calX$ is an $\infty$-category which admits pullbacks, so that the projection
$p: \calO_{\calX} \rightarrow \calX$ is both a Cartesian fibration and a coCartesian fibration.
Let $f: S \rightarrow T$ be a morphism in $\calX$. Taking the pullback of $p$ along
the corresponding map $\Delta^1 \rightarrow \calX$, we obtain a correspondence from
$p^{-1}(S) = \calX^{/S}$ to $p^{-1}(T) = \calX^{/T}$, associated to a pair of adjoint functors
$$ f_{!}: \calX^{/S} \rightarrow \calX^{/T}$$
$$ f^{\ast}: \calX^{/T} \rightarrow \calX^{/S}.$$
The functors $f_{!}$ and $f^{\ast}$ are well-defined up to homotopy (in fact, up to a contractible space of choices. We may think of $f_{!}$ as the functor given by composition with $f$, and $f^{\ast}$ as the functor given by pullback along $f$ (in view of Lemma \ref{charpull}).\index{gen}{pullback functor} 

We can now formulate the $\infty$-categorical analogue of $(ii)$:

\begin{definition}\label{colu}\index{gen}{colimit!universal}
Let $\calC$ be a presentable $\infty$-category. We will say that {\it colimits in $\calC$
are universal} if, for any morphism $f: T \rightarrow S$ in $\calC$, the associated pullback functor
$$ f^{\ast}: \calC^{/S} \rightarrow \calC^{/T}$$
preserves (small) colimits.
\end{definition}

Assume that $\calC$ is a presentable $\infty$-category, and let $f: T \rightarrow S$ be a morphism in $\calC$. By the adjoint functor theorem, $f^{\ast}: \calC^{/S} \rightarrow \calC^{/T}$ preserves all colimits if and only if it has a right adjoint $f_{\ast}$. Since the existence of adjoint functors can be tested inside the enriched homotopy category, this gives a convenient criterion which allows us to test whether or not colimits in $\calC$ are universal.

\begin{remark}
Let $\calX$ be an $\infty$-category. The assumption that colimits in $\calX$ are universal can be viewed as a kind of distributive law. We have the following table of vague analogies:
$$
\begin{array}{cc}
\text{ Higher Category Theory } & \text{ Algebra } \\
\hline
\\
\text{ $\infty$-Category } & \text{ Set } \\ \\
\text{ Presentable $\infty$-Category } & \text{ Abelian Group } \\ \\
\text{ Colimits } & \text{ Sums } \\ \\
\text{ Limits } & \text{ Products } \\ \\
\colim( X_{\alpha}) \times_{S} T \simeq \colim( X_{\alpha} \times_{S} T) & (x+y)z=xz+yz \\ \\
\text{ $\infty$-Topos } & \text{ Commutative Ring } \\ \\
\end{array}$$
\end{remark}

Definition \ref{colu} has a reformulation in the language of classifying functors (\S \ref{universalfib}):

\begin{proposition}\label{gentur}
Let $\calX$ be an $\infty$-category which admits finite limits. The following conditions are equivalent:
\begin{itemize}
\item[$(1)$] The $\infty$-category $\calX$ is presentable, and colimits in $\calX$ are universal.
\item[$(2)$] The Cartesian fibration $p: \calO_{\calX} \rightarrow \calX$ is classified by a functor
$\calX^{op} \rightarrow \LPres$.
\end{itemize}
\end{proposition}

\begin{proof}
We can restate condition $(2)$ as follows: each fiber $\calX^{/U}$ of $p$ is presentable, and
each of the pullback functors $f^{\ast}: \calX^{/V} \rightarrow \calX^{/U}$ preserves small colimits.
It is clear that $(1) \Rightarrow (2)$, and that $(2)$ implies that colimits in $\calX$ are universal.
Since $\calX$ admits finite limits, it has a final object $1$; condition $(2)$ implies that
$\calX \simeq \calX^{/1}$ is presentable, which proves $(1)$.
\end{proof}

\begin{itemize}\index{gen}{coproduct!disjoint}
\item[$(iii)$] Coproducts in $\calC$ are disjoint.
\end{itemize}

If $\calC$ is an $\infty$-category which admits finite coproducts, then we will say that
{\it coproducts in $\calC$ are disjoint} if every coCartesian diagram
$$ \xymatrix{ & \emptyset \ar[dr] \ar[dl] & \\ X \ar[dr] & & Y \ar[dl] \\ & X \amalg Y & }$$
is also Cartesian, provided that $\emptyset$ is an initial object of $\calC$.
More informally, to say that coproducts are disjoint is to say that the intersection of $X$ and $Y$ inside the union $X \amalg Y$ is empty. 

We now come to the most subtle and interesting of Giraud's axioms:

\begin{itemize}\index{gen}{effective equivalence relation}
\item[$(iv)$] Every equivalence relation in $\calC$ is effective.
\end{itemize}
 
Recall that if $X$ is an object in an (ordinary) category $\calC$,
then an {\it equivalence relation} $R$ on $X$ is an object of
$\calC$ equipped with a map $p: R \rightarrow X \times X$ such
that for any $S$, the induced map $$\Hom_{\calC}(S,R) \rightarrow
\Hom_{\calC}(S,X) \times \Hom_{\calC}(S,X)$$ exhibits
$\Hom_{\calC}(S,R)$ as an equivalence relation on
$\Hom_{\calC}(S,X)$.

If $\calC$ admits finite limits, then it is easy to construct
equivalence relations in $\calC$: given any map $X \rightarrow Y$
in $\calC$, the induced map $X \times_Y X \rightarrow X \times X$
is an equivalence relation on $X$. If the category $\calC$ admits
finite colimits, then one can attempt to invert this process:
given an equivalence relation $R$ on $X$, one can form the
coequalizer of the two projections $R \rightarrow X$ to obtain an
object which we will denote by $X/R$. In the category of sets, one
can recover $R$ as the fiber product $X \times_{X/R} X$. In
general, this need not occur: one always has $R \subseteq X
\times_{X/R} X$, but the inclusion may be strict (as subobjects of
$X \times X$). If equality holds, then $R$ is said to be an {\it
effective equivalence relation}, and the map $X \rightarrow X/R$
is said to be an {\it effective epimorphism}.\index{gen}{effective epimorphism}

\begin{remark}
Recall that a map $f: X \rightarrow Y$ in a category $\calC$ is said to be
a {\it categorical epimorphism} if the natural map $\Hom_{\calC}(Y,Z) \rightarrow \Hom_{\calC}(X,Z)$ is {\em injective} for every object $Z \in \calC$, so that we may identify
$\Hom_{\calC}(Y,Z)$ with a subset of $\Hom_{\calC}(X,Z)$. 
To say that $f$ is an {\em effective} epimorphism is to say that we can characterize this subset: it is the collection of all maps $g: X \rightarrow Z$ such that the diagram
$$ \xymatrix{ & X \ar[dr] \ar[drr]^{g} & & \\
X \times_{Y} X \ar[ur] \ar[dr] & & Y \ar@{-->}[r] & Z \\
& X \ar[ur] \ar[urr]_{g} & & }$$
commutes (which is obviously a necessary condition for the indicated dotted arrow to exist).
\end{remark}

Using the terminology introduced above, we can neatly summarize some of the fundamental properties of the category of sets:

\begin{fact}\label{factoid}
In the category of sets, every equivalence relation is
effective and the effective epimorphisms are precisely the
surjective maps.
\end{fact}

The first assertion of Fact \ref{factoid} remains valid in any
topos, and according to the axiomatic point of view it is one of
the defining features of a topos.

If $\calC$ is a category with finite limits and colimits in which
all equivalence relations are effective, then we obtain a
one-to-one correspondence between equivalence relations on an
object $X$ and {\em quotients} of $X$ (that is, isomorphism classes
of effective epimorphisms $X \rightarrow Y$). This correspondence
is extremely useful because it allows us to make elementary
descent arguments: one can deduce statements about quotients of
$X$ from statements about $X$ and about equivalence relations on
$X$ (which live over $X$). We would like to formulate an $\infty$-categorical analogue of this condition which will allow us to make similar arguments.

In the $\infty$-category $\SSet$ of spaces, the situation is more complicated.
The correct notion of surjection of spaces $X \rightarrow Y$ is a map which
induces a surjection on path components $\pi_0 X \rightarrow \pi_0 Y$.
However, in this case, the (homotopy) fiber product $R= X \times_Y X$ does
not give an equivalence relation on $X$, because the map $R
\rightarrow X \times X$ is not necessarily injective in any
reasonable sense. However, it does retain some of the pleasant
features of an equivalence relation: instead of transitivity, we
have a {\it coherently associative} composition law $R \times_X R
\rightarrow R$ (this is perhaps most familiar in the situation where $X$ is a point: in this case, $R$ can be identified with the based loop space of $Y$, which is endowed with a multiplication given by concatenation of loops). In \S \ref{gengroup} we will make this idea precise, and define {\it groupoid objects} and {\it effective groupoid objects} in an arbitrary $\infty$-category. Granting these notions
for the moment, we have a natural candidate for the $\infty$-categorical generalization
of condition $(iv)$:

\begin{itemize}\index{gen}{groupoid object}\index{gen}{groupoid object!effective}
\item[$(iv)'$] Every groupoid object of $\calC$ is effective.
\end{itemize}

\subsection{Groupoid Objects}\label{gengroup}

Let $\calC$ be a category which admits finite limits. A {\it groupoid object}\index{gen}{groupoid object!of a category} of $\calC$ is a functor $F$ from $\calC$ to the category $\Cat$ of (small) groupoids, which has the following properties:

\begin{itemize}
\item[$(1)$] There exists an object $X_0 \in \calC$ and a (functorial) identification of
$\Hom_{\calC}(C,X_0)$ with the set of objects in the groupoid $F(C)$, for each $C \in \calC$.
\item[$(2)$] There exists an object $X_1 \in \calC$ and a (functorial) identification of
$\Hom_{\calC}(C,X_1)$ with the set of morphisms in groupoid $F(C)$, for each $C \in \calC$.
\end{itemize}

\begin{example}
Let $\calC$ be the category $\Set$ of sets. Then a groupoid object of $\calC$ is simply a (small) groupoid.
\end{example}

Giving a groupoid object of a category $\calC$ is equivalent to giving a pair of objects $X_0 \in \calC$ (the ``object classifier'') and $X_1 \in \calC$ (the ``morphism classifier''), together with a collection of maps which relate $X_0$ to $X_1$ and satisfy appropriate identities, which imitate the usual axiomatics of category theory. These identities can be very efficiently encoded using the formalism of simplicial objects. For every $n \geq 0$, let
$[n]$ denote the category associated to the linearly ordered set $\{0, \ldots, n \}$, and consider
the functor $F_n: \calC \rightarrow \Set$ defined so that
$$F_n(C) = \Hom_{\Cat}( [n], F(C) ).$$
By assumption, $F_0$ and $F_1$ are representable by objects $X_0, X_1 \in \calC$.
Since $\calC$ is stable under finite limits, it follows that 
$$ F_n = F_1 \times_{F_0} \ldots \times_{F_0} F_1$$
is representable by an object $X_n = X_1 \times_{X_0} \ldots \times_{X_0} X_1$.
The objects $X_{n}$ can be assembled into a simplicial object $X_{\bigdot}$ of
$\calC$. We can think of this construction as a generalization of the process which associates
to every groupoid $\calD$ its nerve $\Nerve(\calD)$ (a simplicial set). Moreover, as in the classical case, the association $F \mapsto X_{\bigdot}$ is fully faihtful. In other words, we can identify groupoid objects of $\calC$ with the corresponding simplicial objects. Of course, not every simplicial object $X_{\bigdot}$ of $\calC$ arises via this construction. This is true if and only if certain additional conditions are met: for instance, the diagram
$$ \xymatrix{ X_2 \ar[r]^{d_0} \ar[d]^{d_2} & X_1 \ar[d]^{d_1} \\
X_1 \ar[r]^{d_0}  & X_0 }$$
must be Cartesian. 

The purpose of this section is to generalize the notion of a groupoid object to the setting where $\calC$ is an $\infty$-category. We begin by introducing the class of {\it simplicial objects} of $\calC$; we then define groupoid objects to be simplicial objects which satisfy additional conditions.

\begin{definition}\label{siminf}\index{gen}{simplicial object!of an $\infty$-category}
Let $\cDelta_{+}$\index{not}{Delta+@$\cDelta_{+}$} denote the category of finite (possibly empty) linearly ordered sets.
A {\it simplicial object} of an $\infty$-category $\calC$ is a map of $\infty$-categories
$$U_{\bigdot}: \Nerve(\cDelta)^{op} \rightarrow \calC.$$
An {\it augmented simplicial object}\index{gen}{simplicial object!augmented} of $\calC$ is a map
$$U_{\bigdot}^{+}: \Nerve(\cDelta_{+})^{op} \rightarrow \calC.$$

We let $\calC_{\Delta}$\index{not}{Delta@$\calC_{\Delta}$} denote the $\infty$-category $\Fun( \Nerve(\cDelta)^{op}, \calC)$; we will refer
to $\calC_{\Delta}$ as the {\it $\infty$-category of simplicial objects of $\calC$}. Similarly, we 
will refer to $\Fun(\Nerve(\cDelta_{+})^{op}, \calC)$ 
as the {\it $\infty$-category of augmented simplicial objects of $\calC$} and we will denote it by $\calC_{\Delta_{+}}$\index{not}{DeltaPlus@$\calC_{\Delta_{+}}$}.

If $U_{\bigdot}$ is an (augmented) simplicial object of $\calC$ and $n \geq 0$ ($n \geq -1$), we will write $U_{n}$ for the object $U([n]) \in \calC$. 
\end{definition}

\begin{remark}
In the case where $\calC$ is the nerve of an ordinary category $\calD$, Definition \ref{siminf} recovers the usual notion of a simplicial object of $\calD$. More precisely, the $\infty$-category $\calC_{\Delta}$ of simplicial objects of $\calC$ is naturally isomorphic to the nerve of the category of simplicial objects of $\calD$.
\end{remark}

\begin{lemma}\label{silling}
Let $f: X \rightarrow Y$ be a map of simplicial sets. Suppose that:

\begin{itemize}
\item[$(1)$] The map $f$ induces a bijection $X_0 \rightarrow Y_0$ on vertex sets.
\item[$(2)$] The simplicial set $Y$ is a Kan complex.
\item[$(3)$] The map $f$ has the right lifting property with respect to every horn inclusion
$\Lambda^n_i \subseteq \Delta^n$, for $n \geq 2$.
\item[$(4)$] The map $f$ is a weak homotopy equivalence.
\end{itemize}

Then $f$ is a trivial Kan fibration.
\end{lemma}

\begin{proof}
In view of condition $(4)$, it suffices to prove that $f$ is a Kan fibration. In other words, we must show that $p$ has the right lifting property with respect to every horn inclusion $\Lambda^n_i \subseteq \Delta^n$. If $n > 1$, this follows from $(3)$. We may therefore reduce to the case where $n=1$; by symmetry, we may suppose that $i=0$.

Let $e: y \rightarrow y'$ be an edge of $Y$. Condition $(1)$ implies that there is a (unique) pair of vertices $x,x' \in X_0$ with $y=f(x)$, $y' = f(x')$.
Since $f$ is a homotopy equivalence, there is a path $p$ from $x$ to $x'$ in the topological space $|X|$, such that the induced path $|f| \circ p$ in $|Y|$ is homotopic to $e$ via a homotopy which keeps the endpoints fixed. By cellular approximation we may suppose that this path is contained in the $1$-skeleton of $|X|$. 
Consequently, there is a positive integer $k$, a sequence
of vertices $\{ z_0, \ldots, z_k \}$ with $z_0 = x$, $z_k = x'$ such
each adjacent pair $(z_i, z_{i+1})$ is joined by an edge $p_{i}$ (running in either direction), such that $p$ is homotopic (relative to its boundary) to the path obtained by concatenating the edges $p_i$. Using conditions $(2)$ and $(3)$, we note that $X$ has the extension property with respect to the inclusion $\Lambda^n_i \subseteq \Delta^n$ for each $n \geq 2$.
It follows that we may assume that
$p_i$ runs from $z_i$ to $z_{i+1}$: if it runs in the opposite direction, then we can extend the map $$(p_i, s_0 z_i, \bigdot): \Lambda^2_2 \rightarrow X$$ to a $2$-simplex
$\sigma: \Delta^2 \rightarrow X$, and then replace $p_i$ by $d_2 \sigma$.

Without loss of generality, we may suppose that $k > 0$ is chosen as small as possible.
We claim that $k = 1$. Otherwise, choose an extension $\tau: \Delta^2 \rightarrow X$ of the map
$$ (p_1, \bigdot, p_0 ): \Lambda^2_1 \rightarrow X.$$
We can then replace the initial segment
$$ z_0 \stackrel{p_0}{\rightarrow} z_1 \stackrel{p_1}{\rightarrow} z_2$$
by the edge $d_1(\tau): z_0 \rightarrow z_2$ and obtain a shorter path from $x$ to $x'$, contradicting the minimality of $k$. 

The edges $e$ and $f(p_0)$ are homotopic in $Y$ relative to their endpoints. Using $(3)$, we see that $p_0$ is homotopic (relative to its endpoints) to an edge $\overline{e}$ which satisfies $f(\overline{e}) = e$. This completes the proof that $f$ is a Kan fibration.
\end{proof}

\begin{notation}
Let $K$ be a simplicial set. We let $\cDelta_{/K}$ denote the {\it category of simplices of $K$}\index{gen}{category!of simplices} defined in \S \ref{quasilimit1}. The objects of $\cDelta_{/K}$\index{not}{DeltaK@$\cDelta_{/K}$} are pairs $(J, \eta)$ where $J$ is an object of $\cDelta$ and $\eta \in \Hom_{\sSet}( \Delta^J, K)$. A morphism from $(J, \eta)$ to $(J', \eta')$ is a commutative diagram
$$ \xymatrix{ \Delta^{J} \ar[rr] \ar[dr] & & \Delta^{J'} \ar[dl] \\
& K. & }$$
Equivalently, we can describe $\cDelta_{/K}$ as the fiber product 
$\cDelta \times_{ \sSet } (\sSet)_{/K}$.

If $\calC$ is an $\infty$-category, $U: \Nerve(\cDelta)^{op} \rightarrow \calC$ is a simplicial object of $\calC$, and $K$ is a simplicial set, then we let $U[K]$ denote the composite map
$$ \Nerve(\cDelta_{/K})^{op} \rightarrow \Nerve(\cDelta)^{op} \rightarrow \calC.$$
\end{notation}

\begin{proposition}\label{grpobjdef}\index{gen}{groupoid object!of an $\infty$-category}
Let $\calC$ be an $\infty$-category and $U: \Nerve(\cDelta)^{op} \rightarrow \calC$ a simplicial
object of $\calC$. The following conditions are equivalent:
\begin{itemize}

\item[$(1)$] For every weak homotopy equivalence $f: K \rightarrow K'$ of simplicial
sets which induces a bijection $K_0 \rightarrow K'_0$ on vertices, the induced map
$\calC_{/U[K']} \rightarrow \calC_{/U[K]}$ is a categorical equivalence.

\item[$(2)$] For every cofibration $f: K \rightarrow K'$ of simplicial sets which is a weak homotopy equivalence and bijective on vertices, the induced map
$\calC_{/U[K']} \rightarrow \calC_{/U[K]}$ is a categorical equivalence.

\item[$(2')$] For every cofibration $f: K \rightarrow K'$ of simplicial sets which is a weak homotopy equivalence and bijective on vertices, the induced map
$\calC_{/U[K']} \rightarrow \calC_{/U[K]}$ is a trivial fibration.

\item[$(3)$] For every $n \geq 2$ and every $0 \leq i \leq n$, the induced map
$\calC_{/U[\Delta^n]} \rightarrow \calC_{/U[\Lambda^n_i]}$ is a categorical equivalence.

\item[$(3')$] For every $n \geq 2$ and every $0 \leq i \leq n$, the induced map
$\calC_{/U[\Delta^n]} \rightarrow \calC_{/U[\Lambda^n_i]}$ is a trivial fibration.

\item[$(4)$] For every $n \geq 0$ and every partition $[n] = S \cup S'$ such that
$S \cap S'$ consists of a single element $s$, the induced map
$\calC_{/U[\Delta^n]} \rightarrow \calC_{ /U[K]}$
is a categorical equivalence, where $K = \Delta^{S} \amalg_{ \{s\} } \Delta^{ S'} \subseteq \Delta^n$.

\item[$(4')$] For every $n \geq 0$ and every partition $[n] = S \cup S'$ such that
$S \cap S'$ consists of a single element $s$, the induced map
$\calC_{/U[\Delta^n]} \rightarrow \calC_{ /U[K]}$
is a trivial fibration, where $K = \Delta^{S} \amalg_{ \{s\} } \Delta^{ S'} \subseteq \Delta^n$.

\item[$(4'')$] For every $n \geq 0$ and every partition $[n] = S \cup S'$ such that
$S \cap S'$ consists of a single element $s$, the diagram
$$ \xymatrix{ U([n]) \ar[r] \ar[d] & U(S) \ar[d] \\
U(S') \ar[r] & U( \{s\}) }$$
is a pullback square in the $\infty$-category $\calC$.
\end{itemize}
\end{proposition}

\begin{proof}
The dual of Proposition \ref{sharpen} implies that any monomorphism $K \rightarrow K'$ of simplicial sets induces a right fibration $\calC_{/U[K]} \rightarrow \calC_{/U[K]}$. By Corollary \ref{heath}, a right fibration is a trivial fibration if and only if it is a categorical equivalence.
This proves that $(2) \Leftrightarrow (2')$, $(3) \Leftrightarrow (3')$, and $(4) \Leftrightarrow (4')$. The implications $(1) \Rightarrow (2) \Rightarrow (3)$ are obvious. 

We now prove that $(3)$ implies $(1)$. Let $A$ denote the class of all morphisms $f: K' \rightarrow K$ which induce a categorical equivalence $\calC_{/U[K]} \rightarrow \calC_{/U[K']}$. 
Let $A'$ denote the class of all {\em cofibrations} which have the same property; equivalently, $A'$ is the class of all cofibrations which induce a trivial fibration $\calC_{/U[K]} \rightarrow \calC_{/U[K']}$. From this characterization it is easy to see that $A'$ is weakly saturated. Let $A''$ be the weakly saturated class of morphisms generated by the inclusions $\Lambda^n_i \subseteq \Delta^n$ for $n > 1$.
If we assume $(3)$, then we have the inclusions $A'' \subseteq A' \subseteq A$.

Let $f: K \rightarrow K'$ be an arbitrary morphism of simplicial sets. By Proposition \ref{quillobj}, we can choose a map $h': K' \rightarrow M'$ which belongs to $A''$, where $M'$ has the extension property with respect to $\Lambda^n_i \subseteq \Delta^n$ for $n > 1$ and is therefore a Kan complex. Applying Proposition \ref{quillobj} again, we can construct a commutative diagram
$$ \xymatrix{ K \ar[d]^{f} \ar[r]^{h} & M \ar[d]^{g} \\
K' \ar[r]^{h'} & M' }$$
where the horizontal maps belong to $A''$ and $g$ has the right lifting property with
respect to every morphism in $A''$. If $f$ is a weak homotopy equivalence which is bijective on vertices, then $g$ has the same properties, so that $g$ is a trivial fibration by Lemma \ref{silling}. It follows that $g$ has the right lifting property with respect to the cofibration $g \circ h: K \rightarrow M'$, so that $g \circ h$ is a retract of $h$ and therefore belongs to $A''$. Since $g \circ h = h' \circ f$ and $h'$ belong to $A'' \subseteq A$, it follows
that $f$ belongs to $A$.

It is clear that $(1) \Rightarrow (4)$. We next prove that $(4') \Rightarrow (3)$. We must show
that if $n > 1$, then every inclusion $\Lambda^n_i \subseteq \Delta^n$ belongs to the class $A$ defined above. The proof is by induction on $n$. Replacing $i$ by $n-i$ if necessary, we may suppose that $i < n$. If $(n,i) \neq (2,0)$, we consider the composition
$$ \Delta^{n-1} \amalg_{ \{n-1\} } \Delta^{ \{n-1, n\} } \stackrel{f}{\hookrightarrow} \Lambda^n_i \stackrel{f'}{\hookrightarrow} \Delta^n.$$
Here $f$ belongs to $A'$ by the inductive hypothesis and $f' \circ f$ belongs to $A'$ by virtue of the assumption $(4')$; therefore $f'$ also belongs to $A$. If $n=2$ and $i=0$, then we observe
that the inclusion $\Lambda^2_1 \subseteq \Delta^2$ is of the form
$\Delta^S \amalg_{ \{s\} } \Delta^{S'} \subseteq \Delta^2$, where $S = \{0,1\}$ and
$S' = \{0,2\}$. 

To complete the proof, we show that $(4)$ is equivalent to $(4'')$.  
Fix $n \geq 0$, let $S \cup S' = [n]$ be such that $S \cap S' = \{s\}$, and let
$K = \Delta^{S} \amalg_{ \{s\} } \Delta^{S'} \subseteq \Delta^n$.
Let
$\calI'$ denote the full subcategory of $\cDelta_{/\Delta^n}$ spanned by the objects
$[n]$, $S$, $S'$, and $\{s\}$. Let $\calI \subseteq \calI'$ be the full subcategory obtained by omitting the object $[n]$. Let $p'$ denote the composition
$$ \Nerve(\calI')^{op} \rightarrow \Nerve(\cDelta)^{op} \stackrel{U}{\rightarrow} \calC$$ and let
$p = p' | \Nerve(\calI)^{op}$. 
Consider the diagram
$$ \xymatrix{ \calC_{/U[\Delta^n]} \ar[r] \ar[d]^{u} & \calC_{/U[K]} \ar[d]^{v} \\
\calC_{/p'} \ar[r] & \calC_{/p}. }$$ 
Condition $(4)$ asserts that the upper horizontal map is a categorical equivalence, and condition $(4'')$ asserts that the lower horizontal map is a categorical euivalence. To prove that
$(4) \Leftrightarrow (4'')$, it suffices to show that the vertical maps $u$ and $v$ are categorical equivalences.

We have a commutative diagram
$$ \xymatrix{ \calC_{/U[\Delta^n]} \ar[rr]^{u} \ar[dr] & & \calC_{/p'} \ar[dl] \\
& \calC_{/U[\Delta^n]}.& }$$
Since $\Delta^n$ is a final object of both $\cDelta_{/\Delta^n}$ and $\calI'$, the unlabelled maps are trivial fibrations. It follows that $u$ is a categorical equivalence.

To prove that $v$ is a categorical equivalence, it suffices to show that the inclusion
$g: \calI \subseteq \cDelta_{/K}$ induces a right anodyne map
$$\Nerve(g): \Nerve(\calI) \subseteq \Nerve(\cDelta_{/K})$$ of simplicial sets. We observe that the functor $g$ has a left adjoint $f$, which associates to each simplex $\sigma: \Delta^m \rightarrow
K$ the smallest simplex in $\calI$ which contains the image of $\sigma$. 
The map $\Nerve(g)$ is a section of $\Nerve(f)$, and there is a (fiberwise) simplicial homotopy from
$\id_{ \Nerve(\cDelta_{/K}) }$ to $\Nerve(g) \circ \Nerve(f)$. We now invoke Proposition \ref{trull11} to deduce that $\Nerve(g)$ is right anodyne, as desired.
\end{proof}

\begin{definition}\index{gen}{groupoid object!of an $\infty$-category}
Let $\calC$ be an $\infty$-category. We will often denote simplicial objects of $\calC$ by
$U_{\bigdot}$, and write $U_{n}$ for $U_{\bigdot}([n]) \in \calC$. We will say that a simplicial object $U_{\bigdot} \in \calC_{\Delta}$ is a {\em groupoid object} of $\calC$ if it satisfies the equivalent conditions of Proposition \ref{grpobjdef}. We will let $\Grp(\calC)$\index{not}{Grp@$\Grp(\calC)$} denote the full subcategory of $\calC_{\Delta}$ spanned by the groupoid objects of $\calC$.
\end{definition}

\begin{remark}
It follows from the proof of Proposition \ref{grpobjdef} that to verify that a simplicial object
$X_{\bigdot} \in \calC_{\Delta}$ is a groupoid object, we need only verify condition $(4'')$
in a small class of specific examples, but we will not need this observation.
\end{remark}

\begin{proposition}\label{refle}
Let $\calC$ be a presentable $\infty$-category. The full subcategory $\Grp(\calC) \subseteq \calC_{\Delta}$ is strongly reflective.
\end{proposition}

\begin{proof}
Let $n \geq 0$ and $[n] = S \cup S'$ be as in the statement of $(4'')$ of Proposition \ref{grpobjdef}. 
Let $\calD(S,S') \subseteq \calC_{\Delta}$
be the full subcategory consisting of those simplicial objects $U \in \calC_{\Delta}$
for which the associated diagram
$$ \xymatrix{ U([n]) \ar[r] \ar[d] & U(S) \ar[d] \\
U(S') \ar[r] & U( \{s\}) }$$
is Cartesian. Lemmas \ref{stur1} and \ref{stur2} imply that $\calD(S,S')$ is a strongly reflective subcategory of $\calC_{\Delta}$. 
Let $\calD$ denote the intersection of all these subcategories, taken over all $n \geq 0$ and all such decompositions $[n] = S \cup S'$. Lemma \ref{stur3} implies that
$\calD \subseteq \calC_{\Delta}$ is strongly reflective, and Proposition \ref{grpobjdef}
implies that $\calD = \Grp(\calC)$.
\end{proof}


Our next step is to exhibit a large class of examples of groupoid objects. We first sketch the idea.
Suppose that $\calC$ is an $\infty$-category which admits finite limits, and let $u: U \rightarrow X$ be a morphism in $\calC$. Using this data, we can construct a simplicial object $U_{\bigdot}$ of $\calC$, where $U_{n}$ is given by the $(n+1)$-fold fiber power of $U$ over $X$. In order to describe this construction more precisely, we need to introduce a bit of notation.

\begin{notation}
Let $\cDelta^{\leq n}_{+}$\index{not}{DeltaLeq0Plus@$\cDelta^{\leq n}_{+}$} denote the full subcategory of $\cDelta_{+}$ spanned by the objects $\{ [k] \}_{ -1 \leq k \leq n}$.
\end{notation}

\begin{proposition}\label{strump}
Let $\calC$ be an $\infty$-category, and let $U_{\bigdot}^{+}: \Nerve(\cDelta_{+})^{op} \rightarrow \calC$ be an augmented simplicial object of $\calC$. The following conditions are equivalent:
\begin{itemize}
\item[$(1)$] The augmented simplicial object $U_{\bigdot}^{+}$ is a right Kan extension
of $U_{\bigdot}^{+}| \Nerve(\cDelta^{\leq 0}_{+})^{op}$.
\item[$(2)$] The underlying simplicial object $U_{\bigdot}$ is a groupoid object of $\calC$, and the diagram $U_{\bigdot}^{+} | \Nerve( \cDelta^{\leq 1}_{+})^{op}$ is a pullback square
$$ \xymatrix{ U_{1} \ar[r] \ar[d] & U_0 \ar[d] \\
U_{0} \ar[r] & U_{-1} }$$ 
in the $\infty$-category $\calC$.
\end{itemize}
\end{proposition}

\begin{proof}
Suppose first that $(1)$ is satisfied. It follows immediately from the definition of right Kan extensions that the diagram $$ \xymatrix{ U_{1} \ar[r] \ar[d] & U_0 \ar[d] \\
U_{0} \ar[r] & U_{-1} }$$ 
is a pullback. To prove that $U_{\bigdot}$ is a groupoid, we show that $U_{\bigdot}$ satisfies criterion $(4'')$ of Proposition \ref{grpobjdef}. Let $S$ and $S'$ be sets with union $[n]$ and intersection $S \cap S' = \{s\}$. Let $\calI$ be the nerve of the category
$(\cDelta_{+})_{/\Delta^n}$. For each subset $J \subseteq [n]$, let 
$\calI(J)$ denote the full subcategory of $\calI$ spanned by the initial object
together with the inclusions $\{j\} \rightarrow \Delta^n$, $j \in S$. By assumption,
$U_{\bigdot}^{+}$ exhibits $U_{\bigdot}(S)$ as a limit of
$U_{\bigdot}^{+} | \Nerve(\calI(S))$, $U_{\bigdot}(S')$ as a limit of
$U_{\bigdot}^{+} | \Nerve(\calI(S'))$, $U_{\bigdot}([n])$ as a limit of
$U_{\bigdot}^{+} | \Nerve(\calI( [n] ))$, and $U_{\bigdot}( \{s\} )$ as a limit of
$U_{\bigdot}^{+} | \Nerve(\calI(\{s\}))$. It follows from Corollary \ref{util}
that the diagram
$$ \xymatrix{ U_{\bigdot}( [n] ) \ar[r] \ar[d] & U_{\bigdot}(S) \ar[d] \\
U_{\bigdot}(S') \ar[r] & U_{\bigdot}( \{s\} ) }$$
is a pullback.

We now prove that $(2)$ implies $(1)$. Using the above notation, we must show that for each
$n \geq -1$, $U_{\bigdot}^{+}$ exhibits $U_{\bigdot}^{+}([n])$ as a limit of
$U_{\bigdot}^{+} | \calI( [n] )$. For $n \leq 0$, this is obvious; for $n=1$ it is equivalent to the assumption that
$$ \xymatrix{ U_{1} \ar[r] \ar[d] & U_0 \ar[d] \\
U_{0} \ar[r] & U_{-1} }$$ is a pullback diagram. We prove the general case by induction on $n$.
Using the inductive hypothesis, we conclude that $U_{\bigdot}(\Delta^{S})$ is a limit of
$U_{\bigdot}^{+} | \calI(S)$ for all {\em proper} subsets $S \subset [n]$. Choose
a decomposition $\{0, \ldots, n\} = S \cup S'$, where $S \cap S' = \{s\}$. According to
Proposition \ref{train}, the desired result is equivalent to the assertion that
$$ \xymatrix{ U_{\bigdot}([n]) \ar[r] \ar[d] & U_{\bigdot}(S) \ar[d] \\
U_{\bigdot}(S') \ar[r] & U_{\bigdot}( \{s\} ) }$$
is a pullback diagram, which follows from our assumption that $U_{\bigdot}$ is a groupoid object of $\calC$.
\end{proof}

We will say that augmented simplicial object $U_{\bigdot}^{+}$ in an $\infty$-category $\calC$ is a {\it \Cech nerve} if it satisfies the equivalent conditions of Proposition \ref{strump}. In this case, $U_{\bigdot}^{+}$ is determined up to equivalence by the map $u: U_0 \rightarrow U_{-1}$; we will also say that $U_{\bigdot}^{+}$ is the {\it \Cech nerve of $u$}\index{gen}{Cech nerve@\Cech nerve}. 

\begin{notation}
Let $U_{\bigdot}$ be a simplicial object in an $\infty$-category $\calC$. We may regard
$U_{\bigdot}$ as a diagram in $\calC$ indexed
by $\Nerve(\cDelta)^{op}$. We let $|U_{\bigdot}|: \Nerve(\cDelta_{+})^{op} \rightarrow \calC$
denote a colimit for $U_{\bigdot}$ (if such a colimit exists). We will refer to any such colimit as a {\it geometric realization}\index{gen}{geometric realization} of $U_{\bigdot}$. 
\end{notation}

\begin{remark}
Note that we are regarding $|U_{\bigdot}|$ as a colimit diagram in $\calC$, not as an object of $\calC$. We also note that our notation is somewhat abusive, since $|U_{\bigdot}|$ is not uniquely determined by $U_{\bigdot}$. However, if a colimit of $U_{\bigdot}$ exists, then it is determined up to contractible ambiguity.
\end{remark}

\begin{definition}
Let $U_{\bigdot}$ be a simplicial object of an $\infty$-category $\calC$. We will say that
$U_{\bigdot}$ is an {\it effective groupoid}\index{gen}{groupoid object!effective} if can be extended to a colimit diagram $U^{+}_{\bigdot}: \Nerve( \cDelta_{+})^{op} \rightarrow \calC$ such that
$U^{+}_{\bigdot}$ is a \Cech nerve.
\end{definition}

\begin{remark}
It follows immediately from characterization $(3)$ of Proposition \ref{strump} that any effective groupoid $U_{\bigdot}$ is a groupoid.
\end{remark}

We can now state the $\infty$-categorical counterpart of Fact \ref{factoid}: every groupoid object in $\SSet$ is effective. This statement is somewhat less trivial than its classical analogue. For example, a groupoid object $U_{\bigdot}$ in $\SSet$ with $U_0 = \ast$ can be thought of
as a space $U_{1}$ equipped with a coherently associative multiplication operation. If $U_{\bigdot}$ is effective, then there exists a fiber diagram
$$ \xymatrix{ U_1 \ar[r] \ar[d] & \ast \ar[d] \\
\ast \ar[r] & U_{-1} }$$
so that $U_{1}$ is homotopy equivalent to a loop space.
This is an classical result (see, for example, \cite{stasheff}). We will give a somewhat indirect proof in the next section.

\subsection{$\infty$-Topoi and Descent}\label{magnet}

In this section, we will describe an elegant characterization of the notion of an $\infty$-topos, based on the theory of {\em descent}. We begin by explaining the idea in informal terms.
Let $\calX$ be an $\infty$-category. To each object $U$ of $\calX$ we can associate
the overcategory $\calX^{/U}$. If $\calX$ admits finite limits, then this construction gives a contravariant functor from $\calX$ to the $\infty$-category
$\widehat{ \Cat}_{\infty}$ of (not necessarily small) $\infty$-categories. If $\calX$ is an $\infty$-topos, then this functor carries colimits in $\calX$ to limits of $\infty$-categories. In other words, if an object $X \in \calX$ is obtained as the colimit of some diagram $\{ X_{\alpha} \}$ in $\calX$, then giving a morphism $Y \rightarrow X$ is equivalent to a suitably compatible diagram of morphisms
$\{ Y_{\alpha} \rightarrow X_{\alpha} \}$. Moreover, we will eventually show that this property {\em characterizes} the class of $\infty$-topoi. The ideas presented in this section are due to Charles Rezk.

\begin{definition}\index{gen}{Cartesian transformation}
Let $\calX$ be an $\infty$-category, $K$ a simplicial set, and $p,q: K \rightarrow \calX$
two diagrams. We will say that a natural transformation $\alpha: p \rightarrow q$ is
{\it Cartesian} if, for each edge $\phi: x \rightarrow y$ in $K$, the associated diagram
$$ \xymatrix{ p(x) \ar[r]^{p(\phi)} \ar[d]^{\alpha(x)} & p(y) \ar[d]^{\alpha(y)} \\
q(x) \ar[r]^{q(\phi)} & q(y) }$$
is a pullback in $\calX$. 
\end{definition}

\begin{lemma}\label{ib0}
Let $\calX$ be an $\infty$-category, and 
let $\alpha: p \rightarrow q$ be a natural transformation of diagrams
$p,q: K \diamond \Delta^0 \rightarrow \calX$. Suppose that, for every vertex
$x$ of $K$, the associated transformation
$$ p | \{x\} \diamond \Delta^0 \rightarrow q| \{x\} \diamond \Delta^0 $$
is Cartesian. Then $\alpha$ is Cartesian.
\end{lemma}

\begin{proof}
Let $z$ be the ``cone point'' of $K \diamond \Delta^0$. We note that to every
edge $e: x \rightarrow y$ in $K \diamond \Delta^0$ we can associate a diagram
$$ \xymatrix{ x \ar[d]^{e} \ar[r] \ar[dr]^{g} & z \ar[d]^{\id_{z}} \\
y \ar[r] & z. }$$
The transformation $\alpha$ restricts to a Cartesian transformation on the horizontal edges and the right vertical edge, either by assumption or because they are degenerate. Applying Lemma \ref{transplantt}, we deduce first that $\alpha(g)$ is a Cartesian transformation, then that
$\alpha(e)$ is a Cartesian transformation.
\end{proof}

The condition that an $\infty$-category has universal colimits can be formulated in the language of Cartesian transformations:

\begin{lemma}\label{ib1}
Let $\calX$ be a presentable $\infty$-category. The following conditions are equivalent:
\begin{itemize} 
\item[$(1)$] Colimits in $\calX$ are universal.

\item[$(2)$] Let $p, q: (K^{\triangleright} \diamond \Delta^0)
\rightarrow \calX$ be diagrams which carry $\Delta^0$ to vertices $X, Y \in \calX$, and let
$\alpha: p \rightarrow q$ be a Cartesian transformation. 
If the map $q': K^{\triangleright} \rightarrow \calX^{/Y}$ associated to
$q$ is a colimit diagram, then the map $p': K^{\triangleright} \rightarrow \calX^{/X}$ associated to $p$ is a colimit diagram.

\item[$(3)$] Let $p, q: K \star \Delta^1
\rightarrow \calX$ be diagrams which carry $\{1\}$ to vertices $X, Y \in \calX$, and let
$\alpha: p \rightarrow q$ be a Cartesian transformation. 
If the map $K^{\triangleright} \rightarrow \calX_{/Y}$ associated to
$q$ is a colimit diagram, then the map $K^{\triangleright} \rightarrow \calX_{/X}$ associated to $p$ is a colimit diagram.

\item[$(4)$] Let $p, q: K \star \Delta^1
\rightarrow \calX$ be diagrams which carry $\{1\}$ to vertices $X, Y \in \calX$, and let
$\alpha: p \rightarrow q$ be a Cartesian transformation. 
If $q | K \star \{0\}$ is a colimit diagram, then $p| K \star \{0\}$ is a colimit diagram.

\item[$(5)$] Let $\alpha: p \rightarrow q$ be a Cartesian transformation of diagrams
$K^{\triangleright} \rightarrow \calX$. If $q$ is a colimit diagram, then $p$ is a colimit diagram.
\end{itemize}
\end{lemma}

\begin{proof}
Assume that $(1)$ is satisfied; we will prove $(2)$. The transformation $\alpha$ induces a map $f: X \rightarrow Y$.  Consider the map
$$ \phi: \Fun(K^{\triangleright}, \calO_{\calX}) \rightarrow \Fun(K^{\triangleright}, \calX )$$
given by evaluation at the final vertex of $\Delta^1$.
Let $\delta(f)$ denote the image of $f$ under the diagonal map $\delta: \calX \rightarrow
\Fun(K^{\triangleright}, \calX)$. Then we may identify $\alpha$ with an edge $e$ of
$\Fun(K^{\triangleright}, \calO_{\calX})$ which covers $\delta(f)$. Since $\alpha$ is Cartesian, we can apply Lemma \ref{charpull} and Proposition \ref{doog} to deduce that $e$ is $\phi$-Cartesian.
The composition $f^{\ast} \circ q'$ is the origin of a $\phi$-Cartesian edge $e': f^{\ast} \circ q' \rightarrow q'$ of $\Fun(K^{\triangleright}, \calO_{\calX})$ covering $\delta(f)$, so we conclude that
$f^{\ast} \circ q'$ and $p'$ are equivalent in 
$\Fun(K^{\triangleright}, \calX^{/X})$. Since $q'$ is a colimit diagram and $f^{\ast}$ preserves colimits, $f^{\ast} \circ q'$ is a colimit diagram. It follows that $p'$ is a colimit diagram, as desired.

We now prove that $(2) \Rightarrow (1)$. Let $f: X \rightarrow Y$ be a morphism in $\calX$, and let
$q': K^{\triangleright} \rightarrow \calX^{/Y}$ be a colimit diagram. Choose a $\phi$-Cartesian
edge $e': f^{\ast} \circ q' \rightarrow q'$ as above, corresponding to a natural transformation
$\alpha: p \rightarrow q$ of diagrams $p,q: (K^{\triangleright} \diamond \Delta^0) \rightarrow \calX$.
Since $e$ is $\phi$-Cartesian, we may invoke Proposition \ref{doog} and
Lemma \ref{charpull} to deduce that $\alpha$ restricts to a Cartesian transformation
$p|(\{x\} \diamond \Delta^0) \rightarrow q|( \{x\} \diamond \Delta^0 )$ for every vertex $x$ of
$K^{\triangleright}$. It follows from Lemma \ref{ib0} that $\alpha$ is Cartesian. Invoking $(2)$, we conclude that $f^{\ast} \circ q'$ is a colimit diagram, as desired. 

The equivalence $(2) \Leftrightarrow (3)$ follows from Proposition \ref{rub3}, and the equivalence $(3) \Leftrightarrow (4)$ follows from Proposition \ref{needed17}. The implication $(5) \Rightarrow (4)$ is obvious. The converse implication $(4) \Rightarrow (5)$ follows from the observation that $K \star \{0\}$ is a retract of $K \star \Delta^1$.
\end{proof}

\begin{notation}\label{ugaboo}\index{gen}{stable under pullbacks}
Let $\calX$ be an $\infty$-category which admits pullbacks, and let $S$ be a class of morphisms in $\calX$. We will say that $S$ is {\it stable under pullback} if for any pullback diagram
$$ \xymatrix{ X' \ar[r] \ar[d]^{f'} & X \ar[d]^{f} \\
Y' \ar[r] & Y }$$
in $\calX$ such that $f$ belongs to $S$, $f'$ also belongs to $S$. We let
$\calO_{\calX}^{S}$ denote the full subcategory of $\calO_{\calX}$ spanned by $S$, and
$\calO_{\calX}^{(S)}$ the subcategory of $\calO_{\calX}$ whose objects are 
are elements of $S$, and whose morphisms are pullback diagrams as above.
We observe that evaluation at $\{1\} \subseteq \Delta^1$ induces a Cartesian fibration $\calO_{\calX}^{S} \rightarrow \calX$, which restricts to a right fibration
$\calO_{\calX}^{(S)} \rightarrow \calX$ (Corollary \ref{relativeKan}).\index{not}{OXS@$\calO_{\calX}^{S}$}\index{not}{OX(S)@$\calO_{\calX}^{(S)}$} 
\end{notation}

\begin{lemma}\label{ib2}
Let $\calX$ be a presentable $\infty$-category, and suppose that colimits in $\calX$ are universal. 
Let $S$ be a class of morphisms of $\calX$ which is stable under pullback, $K$ a small simplicial
set, and $\overline{q}: K^{\triangleright} \rightarrow \calX$ a colimit diagram.
The following conditions are equivalent:
\begin{itemize}
\item[$(1)$] The composition $f \circ \overline{q}: K^{\triangleright} \rightarrow \hat{\Cat}_{\infty}^{op}$ is a colimit diagram, where $f: \calX \rightarrow \hat{\Cat}_{\infty}^{op}$
classifies the Cartesian fibration $\calO_{\calX}^{S} \rightarrow \calX$. 

\item[$(2)$] The composition $f' \circ \overline{q}: K^{\triangleright} \rightarrow \hat{\SSet}^{op}$ is a colimit diagram, where $f: \calX \rightarrow \hat{\SSet}^{op}$
classifies the right fibration $\calO_{\calX}^{(S)} \rightarrow \calX$. 

\item[$(3)$] For every natural transformation $\overline{\alpha}: \overline{p} \rightarrow \overline{q}$ of colimit diagrams $K^{\triangleright} \rightarrow \calX$, if $\alpha = \overline{\alpha} | K$ is a Cartesian transformation and $\alpha(x) \in S$ for each vertex $x \in K$, then $\overline{\alpha}$ is a Cartesian transformation and $\overline{\alpha}(\infty) \in S$, where $\infty$ denotes the cone point of $K^{\triangleright}$.
\end{itemize}
\end{lemma}

\begin{proof}
Let $\overline{\calC} = \Fun(K^{\triangleright}, \calX)^{/\overline{q}}$ and
$\calC = \Fun(K,\calX)^{/q}$. Let $\overline{\calC}^{0}$ denote the full subcategory of
$\overline{\calC}$ spanned by {\em Cartesian} natural tranformations $\overline{\alpha}: \overline{p} \rightarrow \overline{q}$ with the property that
$\overline{\alpha}(x)$ belongs to $S$ for each vertex $x \in K^{\triangleright}$, and let $\calC^0$ be defined similarly. Finally, let
$\overline{\calC}^{1}$ denote the full subcategory of $\overline{\calC}$ spanned by those natural transformations $\overline{\alpha}: \overline{p} \rightarrow \overline{q}$ such that $\overline{p}$ is a colimit diagram, $\alpha = \overline{\alpha} | K$ is a Cartesian transformation, and
$\overline{\alpha}(x)$ belongs to $S$ for each vertex $x \in K$. Lemma \ref{ib1} implies that $\overline{\calC}^{0} \subseteq \overline{\calC}^{1}$. 

Let $\calD$ denote the full subcategory of $\Fun(K^{\triangleright}, \calC)$ spanned by the colimit diagrams. Proposition \ref{lklk} asserts that the restriction map $\calD \rightarrow \Fun(K,\calC)$ is a trivial fibration. It follows that the associated map $\calD^{/\overline{q}} \rightarrow \Fun(K,\calC)^{/q}$ is also a trivial fibration, and therefore restricts to a trivial fibration
$\overline{\calC}^{1} \rightarrow \calC^{0}$.

According to Proposition \ref{charcatlimit}, condition $(1)$ is equivalent to the assertion that
the projection $\overline{\calC}^{0} \rightarrow \calC^0$ is an equivalence of $\infty$-categories.
In view of the above argument, this is equivalent to the assertion that the fully faithful inclusion
$\overline{\calC}^{0} \subseteq \overline{\calC}^{1}$ is essentially surjective. Since
$\overline{\calC}^{0}$ is clearly stable under equivalence in $\overline{\calC}$, $(1)$
holds if and only if $\overline{\calC}^0 = \overline{\calC}^1$, which is manifestly equivalent to $(3)$.
The proof that $(2) \Leftrightarrow (3)$ is similar, using Proposition \ref{charspacelimit} in place
of Proposition \ref{charcatlimit} and the maximal Kan complexes contained in
$\overline{\calC}^{0}$, $\overline{\calC}^{1}$, and $\calC^0$.
\end{proof}

\begin{lemma}\label{sumoto}
Let $\calX$ be a presentable category in which colimits are universal.
Let $f: X \rightarrow \emptyset$ be a morphism in $\calX$, where $\emptyset$ is an initial
object of $\calX$. Then $X$ is also initial.
\end{lemma}

\begin{proof}
Observe that $\id_{\emptyset}$ is both an initial object of $\calX^{/ \emptyset}$ (Proposition \ref{needed17}) and a final object of $\calX^{/\emptyset}$. Let $f^{\ast}: \calX^{/\emptyset} \rightarrow \calX^{/X}$ be a pullback functor. Then $f^{\ast}$ preserves limits (since it is a right adjoint) and colimits (since colimits in $\calX$ are universal). Therefore $f^{\ast} \id_{\emptyset}$
is both initial and final in $\calX^{/X}$. It follows that $\id_{X}: X \rightarrow X$, being another final object of $\calX^{/X}$, is also initial. Applying Proposition \ref{needed17}, we deduce that
$X$ is an initial object of $\calX$, as desired.
\end{proof}

\begin{lemma}\label{ib3}
Let $\calX$ be a presentable $\infty$-category in which colimits are universal, and let
$S$ be a class of morphisms in $\calX$ which is stable under pullback. The following conditions are equivalent:
\begin{itemize}
\item[$(1)$] The Cartesian fibration $\calO^{S}_{\calX} \rightarrow \calX$ is classified by a 
colimit-preserving functor $\calX \rightarrow \hat{\Cat}^{op}_{\infty}$. 
\item[$(2)$] The right fibration $\calO^{(S)}_{\calX} \rightarrow \calX$ is classified by a colimit-preserving functor $\calX \rightarrow \hat{\SSet}^{op}$.  
\item[$(3)$] The class $S$ is stable under $($arbitrary$)$ coproducts, and 
for every pushout diagram
$$ \xymatrix{ f \ar[r]^{\alpha} \ar[d]^{\beta} & g \ar[d]^{\beta'} \\
f' \ar[r]^{\alpha'} & g' }$$
in $\calO_{\calX}$, if $\alpha$ and $\beta$ are Cartesian transformations and
$f,f',g \in S$, then $\alpha'$ and $\beta'$ are also Cartesian transformations and $g' \in S$.
\end{itemize}
\end{lemma}

\begin{proof}
The equivalence of $(1)$ and $(2)$ follows easily from Lemma \ref{ib2}.
Let $s: \calX \rightarrow \hat{\Cat}_{\infty}^{op}$ be a functor which classifies $\calO_{\calX}$.
Then $(1)$ is equivalent to the assertion that $s$ preserves small colimits. Supposing that $(1)$ is satisfied, we deduce $(3)$ by applying Lemma \ref{ib2} in the special cases of sums and coproducts. For the converse, let us suppose that $(3)$ is satisfied. Let $\emptyset$ denote an initial object of $\calX$. Since colimits in $\calX$ are universal, Lemma \ref{sumoto} implies that $\calX^{/\emptyset}$ is equivalent to final $\infty$-category $\Delta^0$.
Since the morphism $\id_{\emptyset}$ belongs to $S$ (since $S$ is stable under {\em empty} coproducts), we conclude that $s(\emptyset)$ is a final $\infty$-category, so that
$s$ preserves initial objects. It follows from Corollary \ref{allfinn} that $s$ preserves finite coproducts. According to Proposition \ref{alllimits}, it will suffice to prove that $s$ preserves arbitrary coproducts. To handle the case of infinite coproducts we apply Lemma \ref{ib2} again: we must show that if
$\{ f_{\alpha} \}_{\alpha \in A}$ is a collection of elements of $S$ having a coproduct
$f = \coprod_{\alpha \in A} f_{\alpha}$, then $f \in S$ and each of the maps $f_{\alpha} \rightarrow f$ is a Cartesian transformation. The first condition is true by assumption; for the second we 
let $f'$ be a coproduct of the family $\{ f_{\beta} \}_{\beta \in A, \beta \neq \alpha}$, so that $f \simeq f' \amalg f_{\alpha}$ and $f' \in S$. Applying Lemma \ref{ib2} (and the fact that $s$ preserves finite coproducts), we deduce that $f_{\alpha} \rightarrow f$ is a Cartesian transformation as desired.
\end{proof}

\begin{definition}\label{localitie}\index{gen}{local!class of morphisms}
Let $\calX$ be a presentable $\infty$-category in which colimits are universal, and let $S$ be a class of morphisms in $\calX$. We will say that $S$ is {\it local} if it is stable under pullbacks
and satisfies the equivalent conditions of Lemma \ref{ib3}.
\end{definition}

\begin{theorem}\label{charleschar}
Let $\calX$ be a presentable $\infty$-category. The following conditions are equivalent:
\begin{itemize}
\item[$(1)$] Colimits in $\calX$ are universal, and for every 
pushout diagram
$$ \xymatrix{ f \ar[r]^{\alpha} \ar[d]^{\beta} & g \ar[d]^{\beta'} \\
f' \ar[r]^{\alpha'} & g' }$$
in $\calO_{\calX}$, if $\alpha$ and $\beta$ are Cartesian transformations, then
$\alpha'$ and $\beta'$ are also Cartesian transformations.

\item[$(2)$] Colimits in $\calX$ are universal, and the class of {\em all} morphisms
in $\calX$ is local.

\item[$(3)$] The Cartesian fibration $\calO_{\calX} \rightarrow \calX$ is classified by a limit-preserving functor $\calX^{op} \rightarrow \LPres$.

\item[$(4)$] Let $K$ be a small simplicial set and 
$\overline{\alpha}: \overline{p} \rightarrow \overline{q}$ a natural transformation of diagrams
$\overline{p}, \overline{q}: K^{\triangleright} \rightarrow \calX$. Suppose that
$\overline{q}$ is a colimit diagram, and that $\alpha = \overline{\alpha} | K$ is a Cartesian transformation. Then $\overline{p}$ is a colimit diagram if and only if $\overline{\alpha}$ is 
a Cartesian transformation.
\end{itemize}
\end{theorem}

\begin{proof}
The equivalences $(1) \Leftrightarrow (2) \Leftrightarrow (3)$ follow from Lemma \ref{ib3} and Proposition \ref{gentur}. The equivalence $(3) \Leftrightarrow (4)$ follows from Lemmas \ref{ib1} and \ref{ib2}. 
\end{proof}

We now have most of the tools required to establish the implication $(1) \Rightarrow (2)$ of Theorem \ref{mainchar}. In view of Theorem \ref{charleschar}, it will suffice to prove the following:

\begin{proposition}\label{lemonade}
Let $\calX$ be an $\infty$-topos. Then:
\begin{itemize}
\item[$(1)$] Colimits in $\calX$ are universal.
\item[$(2)$] For every pushout diagram $$ \xymatrix{ f \ar[r]^{\alpha} \ar[d]^{\beta} & g \ar[d]^{\beta'} \\
f' \ar[r]^{\alpha'} & g' }$$
in $\calO_{\calX}$, if $\alpha$ and $\beta$ are Cartesian transformations, then
$\alpha'$ and $\beta'$ are also Cartesian transformations.
\end{itemize}
\end{proposition}

\begin{remark}
Once we have established Theorem \ref{mainchar} in its entirety, it will follow from Theorem \ref{charleschar} that the converse of Proposition \ref{lemonade} is also valid: a presentable $\infty$-category $\calX$ is an $\infty$-topos if and only if it satisfies conditions $(1)$ and $(2)$ as above. Condition $(1)$ is equivalent to the requirement that
for every morphism $f: X \rightarrow Y$ in $\calX$, the pullback functor $f^{\ast}: \calX_{/Y} \rightarrow \calX_{/X}$ has a right adjoint (in the case where $Y$ is a final object of $\calX$, this simply amounts to the requirement that every object $Z \in \calX$ admits an exponential
$Z^X$; in other words, the requirement that $\calX$ be {\it Cartesian closed}), and condition $(2)$ involves only finite diagrams in the $\infty$-category $\calX$. One could conceivably obtain a theory of {\em elementary $\infty$-topoi} by dropping the requirement that $\calX$ be presentable (or replacing it by weaker conditions which are also finite in nature). We will not pursue this idea further.\index{gen}{$\infty$-topos!elementary}
\end{remark}

Before giving the proof of Proposition \ref{lemonade}, we need to establish a few easy lemmas.

\begin{lemma}\label{stugart}
Let 
$$ \xymatrix{ \phi \ar[r]^{p} \ar[d]^{q} & \psi \ar[d]^{q'} \\
\phi' \ar[r]^{p'} & \psi' }$$
be a coCartesian square in the category of arrows of $\sSet$. Suppose that
$p$ and $q$ are homotopy Cartesian and that $q$ is a cofibration. Then:
\begin{itemize}
\item[$(1)$] The maps $p'$ and $q'$ are homotopy Cartesian.
\item[$(2)$] Given any map of arrows $r: \psi' \rightarrow \theta$ such that
$r \circ p'$ and $r \circ q'$ are homotopy Cartesian, the map $r$ is itself homotopy Cartesian.
\end{itemize}
\end{lemma}

\begin{proof}
Let $r: \psi' \rightarrow \theta$ be as in $(2)$. We must show that $r$ is homotopy Cartesian if and only if $r \circ p'$ and $r \circ q'$ are homotopy Cartesian (taking $r = \id_{\psi'}$, we will deduce $(1)$). Without loss of generality, we may replace $\phi$, $\psi$, $\phi'$ and $\theta$ with minimal Kan fibrations. We now observe that $r$, $r \circ p'$, and $r \circ q'$ are homotopy Cartesian if and only if they are Cartesian; the desired result now follows immediately.
\end{proof}

\begin{lemma}\label{tulmand}
Let $\bfA$ be a simplicial model category containing an object $Z$ which is both fibrant and cofibrant, and let $\bfA_{/Z}$ be endowed with the induced model structure. Then
the natural map $\theta: \sNerve ( \bfA_{/Z}^{\degree} ) \rightarrow \sNerve(\bfA^{\degree})_{/Z}$
is an equivalence of $\infty$-categories.
\end{lemma}

\begin{proof}
Let $\phi: Z' \rightarrow Z$ be an object of $\sNerve(\bfA^{\degree})_{/Z}$. Then we can choose a factorization
$$ Z' \stackrel{i}{\rightarrow} Z'' \stackrel{\psi}{\rightarrow} Z$$
where $i$ is a trivial cofibration and $\psi$ is a fibration, corresponding to a fibrant-cofibrant object
of $\bfA_{/Z}$. The above diagram classifies an equivalence between $\phi$ and $\psi$
in $\sNerve(\bfA^{\degree})_{/Z}$, so that $\theta$ is essentially surjective.

Recall that for any simplicial category $\calC$ containing a pair of objects $X$ and $Y$, there is a natural isomorphism of simplicial sets
$$ \Hom^{\rght}_{\sNerve(\calC)}( X, Y ) \simeq \Sing_{Q^{\bigdot}}( \bHom_{\calC}(X,Y) ),$$ 
where $Q^{\bigdot}$ is the cosimplicial object of $\sSet$ introduced in \S \ref{twistt}. The same calculation shows that if $\phi: X \rightarrow Z$, $\psi: Y \rightarrow Z$ are two morphisms
in $\calC$, then
$$ \Hom^{\rght}_{ \sNerve(\calC)_{/Z} }(\phi, \psi) \simeq \Sing_{Q^{\bigdot}}(P),$$
where $P$ denotes the path space 
$$ \bHom_{\calC}(X,Y) \times_{ \bHom_{\calC}(X,Z)^{ \{0\} } }
\bHom_{\calC}(X,Z)^{\Delta^1} \times_{ \bHom_{\calC}(X,Z)^{ \{1\} }} \{ \phi \}.$$
If $\calC$ is fibrant, then we may identify $M$ with the homotopy fiber of the map
$$ f: \bHom_{\calC}(X,Y) \stackrel{\psi}{\rightarrow} \bHom_{\calC}(X,Z)$$
over the vertex $\phi$. Consequently, we may identify the natural map
$$ \Hom^{\rght}_{ \sNerve( \calC_{/Z} )}(\phi, \psi) \rightarrow
\Hom^{\rght}_{ \sNerve(\calC)_{/Z} }(\phi, \psi)$$
with $\Sing_{Q^{\bigdot}}( \theta)$, where $\theta$ denotes the inclusion of
the fiber of $f$ into the homotopy fiber of $f$. Consequently, to show that
$\Sing_{Q^{\bigdot}}(\theta)$ is a homotopy equivalence, it suffices to prove that
$f$ is a Kan fibration. In the special case where $\calC = \bfA^{\degree}$ and
$\psi$ is a fibration, this follows from the definition of a simplicial model category.
\end{proof}

\begin{lemma}\label{sugartime}
Let $\SSet$ denote the $\infty$-category of spaces. Then:
\begin{itemize}
\item[$(1)$] Colimits in $\SSet$ are universal.
\item[$(2)$] For every pushout diagram $$ \xymatrix{ f \ar[r]^{\alpha} \ar[d]^{\beta} & g \ar[d]^{\beta'} \\
f' \ar[r]^{\alpha'} & g' }$$
in $\calO_{\SSet}$, if $\alpha$ and $\beta$ are Cartesian transformations, then
$\alpha'$ and $\beta'$ are also Cartesian transformations.
\end{itemize}
\end{lemma}

\begin{proof}
We first prove $(1)$.
Let $f: X \rightarrow Y$ be a morphism in $\SSet$. Without loss of generality, we may
suppose that $f$ is a Kan fibration. We wish to show that the projection
$$ \SSet_{/f} \rightarrow \SSet_{/Y}$$ has a right adjoint which preserves colimits.
We have a commutative diagram of $\infty$-categories
$$ \xymatrix{ \sNerve( (\sSet)^{\degree}_{/X} ) \ar[r]^{F} \ar[d]^{\phi} & \Nerve((\sSet)^{\degree}_{/Y}) \ar[d]^{\psi} \\
\SSet_{/f} \ar[r] \ar[d]^{\phi'} & \SSet_{/Y} \\
\SSet_{/X} & } $$
Lemma \ref{tulmand} asserts that $\psi$ and $\phi' \circ \phi$ are categorical equivalences, and $\phi'$ is a trivial fibration. It follows that $\phi$ is also a categorical equivalence. Consequently,
it will suffice to show that the functor $F$ has a right adjoint $G$ which preserves colimits.

We observe that $F$ is obtained by restricting the simplicial nerve of the functor $f_{!}: (\sSet)_{/X} \rightarrow (\sSet)_{/Y}$, given by composition with $f$. The functor $f_{!}$ is a left Quillen functor: it has a right adjoint $f^{\ast}$, given by the formula $f^{\ast}(Y') = Y' \times_{Y} X.$
According to Proposition \ref{quiladj}, $F$ admits a right adjoint $G$, which is given by
the restricting the simplicial nerve of the functor $f^{\ast}$. To prove that $G$ preserves colimits, it will suffice to show that $G$ itself admits a right adjoint. Using Proposition \ref{quiladj} again, we are reduced to proving that $f^{\ast}$ is a {\em left} Quillen functor. We observe that
$f^{\ast}$ admits a right adjoint $f_{\ast}$, given by the formula
$f_{\ast}(X') = \bHom_{Y}(X,X')$. It is clear that $f^{\ast}$ preserves cofibrations; it also preserves weak equivalences, since $f$ is a fibration and $\sSet$ is a {\em right proper} model category (with its usual model structure). 

To prove $(2)$, we first apply Proposition \ref{gumby444} to reduce to the case where
the pushout diagram in question arises from a strictly commutative square
$$ \xymatrix{ f \ar[r]^{\alpha} \ar[d]^{\beta} & g \ar[d]^{\beta'} \\
f' \ar[r]^{\alpha'} & g' }$$
of morphisms in the category $\Kan$. We now complete the proof by applying 
Lemma \ref{stugart} and Theorem \ref{colimcomparee}. 
\end{proof}

\begin{lemma}\label{tryme}
Let $\calX$ be a presentable $\infty$-category, and let $L: \calX \rightarrow \calY$ be an accessible left exact localization. If colimits in $\calX$ are universal, then colimits in $\calY$ are universal.
\end{lemma}

\begin{proof}
We will use characterization $(5)$ of Lemma \ref{ib0}. Let $G$ be a right adjoint to $L$, and let
$\alpha: \overline{p} \rightarrow \overline{q}$ be a Cartesian transformation of diagrams $K^{\triangleright} \rightarrow \calY$. Suppose that $\overline{q}$ is a colimit of $q = \overline{q}|K$.
Choose a colimit $\overline{q}'$ of $G \circ q$, so that there exists a morphism
$\overline{q}' \rightarrow G \circ \overline{q}$ in $\calX_{G \circ q/}$ which determines a natural transformation $\beta: \overline{q}' \rightarrow \overline{q}$ in $\Fun(K^{\triangleright}, \calX)$. 
Form a pullback diagram
$$ \xymatrix{ \overline{p}' \ar[r]^{\alpha'} \ar[d] & \overline{q}' \ar[d]^{\beta} \\
G \circ \overline{p} \ar[r]^{G \circ \alpha} & G \circ \overline{q}. }$$
in $\calX^{K^{\triangleright}}$. Since $G$ is left exact, $G \circ \alpha$ is a Cartesian transformation. It follows that $\alpha'$, being a pullback of $G \circ \alpha$, is also a Cartesian transformation. Since colimits in $\calX$ are universal, we conclude that $\overline{p}'$ is a colimit diagram.
Since $L$ is left exact, we obtain a pullback diagram
$$ \xymatrix{ L \circ \overline{p}' \ar[r] \ar[d] & L \circ \overline{q}' \ar[d]^{L \circ \beta} \\
L \circ G \circ \overline{p} \ar[r] & L \circ G \circ \overline{q}. }$$
Since $L$ preserves colimits, $L \circ \overline{q}'$ and $L \circ \overline{p'}$ are colimit diagrams. 
The diagram $L \circ G \circ \overline{q}$ is equivalent to $q$, and therefore also a colimit diagram. We deduce that $L \circ \beta$ is an equivalence. Since the diagram is a pullback, the left vertical arrow is an equivalence as well, so that $L \circ G \circ \overline{p}$ is a colimit diagram. We finally conclude that $\overline{p}$ is a colimit diagram, as desired.
\end{proof}

We are now ready to give the proof of Proposition \ref{lemonade}.

\begin{proof}[Proof of Proposition \ref{lemonade}]
Let us say that a presentable $\infty$-category $\calX$ is {\em good} if it satisfies conditions $(1)$ and $(2)$.
Lemma \ref{sugartime} asserts that $\SSet$ is good. Using Proposition \ref{limiteval}, it is easy to see that if $\calX$ is good then so is $\Fun(K,\calX)$, for every small simplicial set $K$. It follows that
every $\infty$-category $\calP(\calC)$ of presheaves is good. To complete the proof, it will suffice to show that if $\calX$ is good and $L: \calX \rightarrow \calY$ is an accessible left exact localization functor, then $\calY$ is good. Lemma \ref{tryme} shows that colimits in $\calY$ are universal. 
Consider a diagram $\sigma: \Lambda^2_0 \rightarrow \calO_{\calY}$, denoted by 
$$ g \stackrel{\alpha}{\leftarrow} f \stackrel{\beta}{\rightarrow} h$$
where $\alpha$ and $\beta$ are Cartesian transformations. We wish to show that if
$\overline{\sigma}$ is a colimit of $\sigma$ in $\calO_{\calY}$, then $\overline{\sigma}$ carries
each edge to a Cartesian transformation. Without loss of generality, we may suppose that
$\sigma = L \circ \sigma'$ for some $\sigma': \Lambda^2_0 \rightarrow \calO_{\calX}$ which
is equivalent to $G \circ \sigma$. Since $G$ is left exact, $G(\alpha)$ and $G(\beta)$ are
Cartesian transformations. Because $\calX$
satisfies $(2)$, there exists a colimit $\overline{\sigma}'$ of $\sigma'$ which carries each edge to a Cartesian transformation. Then $L \circ \overline{\sigma}'$ is a colimit of $\sigma$. Since $L$ is left exact, $L \circ \overline{\sigma}'$ carries each edge to a Cartesian transformation in 
$\calO_{\calY}$. 
\end{proof}

Our final objective in this section is to prove the implication $(2) \Rightarrow (3)$ of Theorem \ref{mainchar} (Proposition \ref{lemonade2} below). 

\begin{lemma}\label{aclock}
Let $\calX$ be an $\infty$-category and $U^{+}_{\bigdot}: \Nerve(\cDelta_{+})^{op} \rightarrow \calX$
an augmented simplicial object of $\calX$. Let $\cDelta_{\infty}$ denote the category
whose objects are finite, linearly ordered sets $J$, where $\Hom_{\cDelta_{\infty}}(J,J')$ is the collection of all order-preserving maps $J \cup \{\infty\} \rightarrow J' \cup \{\infty\}$ which carry $\infty$ to $\infty$ $($here $\infty$ is regarded as a maximal element of $J \cup \{ \infty \}$ and
$J' \cup \{\infty\}${}$)$. Suppose that $U^{+}_{\bigdot}$ extends to a functor
$F: \Nerve(\cDelta_{\infty})^{op} \rightarrow \calX$. Then $U^{+}_{\bigdot}$ is a colimit diagram in $\calX$.
\end{lemma}

\begin{proof}
Let $\overline{\calC}$ denote the category whose objects are triples $(J, J_{+})$, where $J$ is a finite, linearly ordered set, and $J_{+}$ is an upward-closed subset of $J$. We define
$\Hom_{\overline{\calC}}( (J, J_{+}), (J', J'_{+})$ to be the set of all order-preserving maps
from $J$ into $J'$ that carry $J_{+}$ into $J'_{+}$. Observe that we have a functor
$\overline{\calC} \rightarrow \cDelta_{\infty}$, given by
$$ (J, J_{+}) \mapsto J - J_{+}. $$
Let $F'$ denote the composite functor
$$ \Nerve(\overline{\calC})^{op} \rightarrow \Nerve(\cDelta_{\infty})^{op} \rightarrow \calX.$$

Let $\calC$ be the full subcategory of $\overline{\calC}$ spanned by those pairs $(J, J_+)$ where
$J \neq \emptyset$. Let $\overline{\calC}^{0}$ denote the full subcategory spanned by those pairs $(J, J_{+})$ where $J_{+} = \emptyset$, and let $\calC^{0} = \overline{\calC}^0 \cap \calC$.
We observe that $\overline{\calC}^{0}$ can be identified with $\cDelta_{+}$ and
that $\calC^{0}$ can be identified with $\cDelta$, in such a way that $U^{+}_{\bigdot}$ is identified with $F' | \Nerve (\overline{\calC}^{0})^{op}$.

Our first claim is that the inclusion $\Nerve(\calC^{0})^{op} \subseteq \Nerve(\calC)^{op}$ is cofinal. According to Theorem \ref{hollowtt}, it will suffice
to show that for every object $X = (J, J_{+})$ of $\calC$, the category
$\calC^{0}_{/X}$ has a contractible nerve. This is clear, since the relevant category has a final object: namely, the map $(J, \emptyset) \rightarrow (J, J_{+})$. As a consequence, we conclude that 
$U_{\bigdot}^{+}$ is a colimit diagram if and only if $F'$ is a colimit diagram.

We now define $\calC^{1}$ to be the full subcategory of $\calC$ spanned by those pairs
$(J,J_{+})$ such that $J_{+}$ is nonempty. We claim that $F' | \Nerve(\calC)^{op}$ is a left Kan extension of $F' | \Nerve (\calC^{1})^{op}$. To prove this, we must show that for every
$(J, \emptyset) \in \calC^{0}$, the induced map
$$ (\Nerve (\calC^{1}_{(J, \emptyset)/} )^{op} )^{\triangleright} \rightarrow \Nerve(\calC)^{op}
\rightarrow \calX$$
is a colimit diagram. Let $\calD$ denote the full subcategory of
$\calC^{1}_{(J, \emptyset)/}$ spanned by those morphisms $(J, \emptyset) \rightarrow
(J' , J'_{+} )$ which induce isomorphisms $J \simeq J' - J'_{+}$. We claim that the inclusion
$\Nerve(\calD)^{op} \subseteq \Nerve ( \calC^{1}_{(J, \emptyset)/} )^{op}$ is cofinal.
To prove this, we once again invoke Theorem \ref{hollowtt}, to reduce to the following assertion:
for every morphism $\phi: (J, \emptyset) \rightarrow (J'', J''_{+})$, if $J''_{+} \neq \emptyset$, then the category $\calD_{/\phi}$ of all factorizations
$$ (J, \emptyset) \rightarrow (J', J'_{+}) \rightarrow (J'', J''_{+}) $$ 
such that $J'_{+} \neq \emptyset$ and $J \simeq J' - J'_{+}$, has weakly contractible nerve.
This is clear, since $\calD_{/\phi}$ has a final object $(J \amalg J'''_{+}, J'''_{+})$, where
$J'''_{+} = \{ j \in J''_{+}: (\forall i \in J) [ j \geq \phi(i) ] \}$. Consequently, it will suffice to prove that the induced functor
$$ \Nerve(\calD^{op})^{\triangleright} \rightarrow \calX$$
is a colimit diagram. This diagram can be identified with the constant diagram
$$ \Nerve(\cDelta_{+})^{op} \rightarrow \calX$$ taking the value $U_{\bigdot}(J)$, and
is a colimit diagram because the category $\cDelta$ has weakly contractible nerve (Corollary \ref{silt}).

We now apply Lemma \ref{kan0}, which asserts that $F'$ is a colimit diagram if and only if
$F' | (\Nerve (\calC^{1})^{op})^{\triangleright}$ is a colimit diagram.
Let $\calC^{2} \subseteq \calC^{1}$ be the full subcategory spanned by those objects
$(J, J_{+})$ such that $J = J_{+}$. We claim that the inclusion
$\Nerve(\calC^{2})^{op} \subseteq \Nerve(\calC^{1})^{op}$ is cofinal. According to Theorem
\ref{hollowtt}, it will suffice to show that, for every object $(J, J_+) \in \calC^{1}$, the
category $\calC^{2}_{/(J, J_{+})}$ has weakly contractible nerve. This is clear,
since the map $(J_{+}, J_{+}) \rightarrow (J, J_{+})$ is a final object of 
the category $\calC^{(2)}_{/(J,J_{+})}$. Consequently, to prove that
$F' | (\Nerve (\calC^{1})^{op})^{\triangleright}$ is a colimit diagram, it will suffice to prove that
$F' | ( \Nerve (\calC^{2})^{op})^{\triangleright}$ is a colimit diagram. But this diagram can be identified with the constant map $\Nerve(\cDelta_{+})^{op} \rightarrow \calX$ taking the value
$U_{\bigdot}( \Delta^{-1})$, which is a colimit diagram because the simplicial set
$\Nerve(\cDelta)^{op}$ is weakly contractible (Corollary \ref{silt}). 
\end{proof}

\begin{lemma}\label{bclock}
Let $\calX$ be an $\infty$-category, and let $U_{\bigdot}: \Nerve(\cDelta)^{op} \rightarrow \calX$ be a simplicial object of $\calX$. Let $U'_{\bigdot}$ be the augmented simplicial object
given by composing $U_{\bigdot}$ with the functor
$$ \cDelta_{+} \rightarrow \cDelta$$
$$ J \rightarrow J \amalg \{ \infty \}.$$
Then:
\begin{itemize}
\item[$(1)$] The augmented simplicial object $U'_{\bigdot}$ is a colimit diagram.
\item[$(2)$] If $U_{\bigdot}$ is a groupoid object of $\calX$, then the evident natural transformation of simplicial objects $\alpha: U'_{\bigdot} | \Nerve(\cDelta)^{op} \rightarrow U_{\bigdot}$ is Cartesian.
\end{itemize}
\end{lemma}

\begin{proof}
Assertion $(1)$ follows immediately from Lemma \ref{aclock}. To prove $(2)$, let us consider
the collection $S$ of all morphisms $f: J \rightarrow J'$ in $\cDelta$ such that
$\alpha(f)$ is a pullback square
$$ \xymatrix{ U'_{\bigdot}(J') \ar[r] \ar[d] & U_{\bigdot}(J') \ar[d] \\
U'_{\bigdot}(J) \ar[r] & U_{\bigdot}(J) }$$
in $\calX$. We wish to prove that every morphism of $\cDelta$ belongs to $S$. Using Lemma \ref{transplantt}, we deduce that if $f' \in S$, then $f \in S \Leftrightarrow (f \circ f' \in S)$. Consequently, it will suffice to prove that every inclusion $\{j\} \subseteq J$ belongs to $S$.
Unwinding the definition, this amounts the the requirement that the diagram
$$ \xymatrix{ U_{\bigdot}( J \cup \{\infty\} ) \ar[d]  \ar[r] & U_{\bigdot}( J) \ar[d] \\
U_{\bigdot} ( \{j, \infty\} ) \ar[r] & U_{\bigdot}( \{j\} )}$$
is Cartesian, which follows immediately from Criterion $(4'')$ of Proposition \ref{grpobjdef}.
\end{proof}

\begin{remark}
Assertion $(2)$ of Lemma \ref{bclock} has a converse: if $\alpha$ is a Cartesian transformation, then $U_{\bigdot}$ is a groupoid object of $\calX$. This can be deduced easily by examining the proof of Proposition \ref{grpobjdef}, but we will not have need of it.
\end{remark}

\begin{proposition}\label{lemonade2}
Let $\calX$ be an $\infty$-category satisfying the equivalent conditions of Theorem \ref{charleschar}. Then $\calX$ satisfies the $\infty$-categorical Giraud axioms:
\begin{itemize}
\item[$(i)$] The $\infty$-category $\calX$ is presentable.
\item[$(ii)$] Colimits in $\calX$ are universal.
\item[$(iii)$] Coproducts in $\calX$ are disjoint.
\item[$(iv)$] Every groupoid object of $\calX$ is effective.
\end{itemize}
\end{proposition}

\begin{proof}
Axioms $(i)$ and $(ii)$ are obvious. To prove $(iii)$, let us consider an arbitrary pair of objects
$X, Y \in \calX$, and let $\emptyset$ denote an initial object of $\calX$. Let $f: \emptyset \rightarrow X$ be a morphism (unique up to homotopy, since $\emptyset$ is initial). We observe that
$\id_{\emptyset}$ is an initial object of $\calO_{\calX}$. Form a pushout diagram
$$ \xymatrix{ \id_{\emptyset} \ar[r]^{\alpha} \ar[d]^{\beta} & \id_{Y} \ar[d]^{\beta'}  \\
f \ar[r]^{\alpha'} & g }$$
in $\calO_{\calX}$. It is clear that $\alpha$ is a Cartesian transformation, and Lemma \ref{sumoto} implies that $\beta$ is Cartesian as well. Invoking condition $(2)$ of Theorem \ref{charleschar}, we deduce that $\alpha'$ is
a Cartesian transformation. But $\alpha'$ can be identified with a pushout diagram
$$ \xymatrix{ \emptyset \ar[r] \ar[d] & Y \ar[d] \\
X \ar[r] & X \amalg Y. }$$

It remains to prove that every groupoid object in $\calX$ is effective. Let $U_{\bigdot}$
be a groupoid object of $\calX$, and let $\overline{U}_{\bigdot}  : \Nerve(\cDelta_{+})^{op} \rightarrow \calX$ be a colimit of $U_{\bigdot}$. Let $U'_{\bigdot}: \Nerve(\cDelta_{+})^{op} \rightarrow \calX$
be the result of composing $\overline{U}_{\bigdot}$ with the ``shift'' functor
$$ \cDelta_{+} \rightarrow \cDelta_{+}$$
$$ J \mapsto J \amalg \{ \infty \}.$$
(In other words, $U'_{\bigdot}$ is the shifted simplicial object given by
$U'_{n} = U_{n+1}$.)
Lemma \ref{bclock} implies that $U'_{\bigdot}$ is a colimit diagram in $\calX$.
We have a transformation $\overline{\alpha}: U'_{\bigdot} \rightarrow \overline{U}_{\bigdot}$.
Since $U_{\bigdot}$ is a groupoid, $\alpha = \overline{\alpha} | \Nerve(\cDelta)^{op}$
is a Cartesian transformation (Lemma \ref{bclock} again). Applying $(4)$, we deduce
that $\overline{\alpha}$ is a Cartesian transformation. In particular, we conclude that
$$ \xymatrix{ U'_0 \ar[r] \ar[d] & U'_{-1} \ar[d] \\
\overline{U}_0 \ar[r] & \overline{U}_{-1} }$$
is a pullback diagram in $\calX$. But this diagram can be identified with
$$ \xymatrix{ U_1 \ar[r] \ar[d] & U_0 \ar[d] \\
U_0 \ar[r] & \overline{U}_{-1}, }$$
so that $U_{\bigdot}$ is effective by Proposition \ref{strump}.
\end{proof}

\begin{corollary}\label{alleff}
Every groupoid object of $\SSet$ is effective.
\end{corollary}

\subsection{Free Groupoids}\label{freegroup}

Let $\calX$ be an $\infty$-category which satisfies the $\infty$-categorical Giraud axioms
$(i)-(iv)$ of Theorem \ref{mainchar}. We wish to prove that $\calX$ is an $\infty$-topos. It is clear that any proof will need to make use of the full strength of axioms $(i)$ through $(iv)$; in particular, we will need to apply $(iv)$ to a class of groupoid objects of $\calX$ which are not obviously effective.
The purpose of this section is to describe a construction which will yields nontrivial examples of groupoid objects, and to deduce a consequence (Proposition \ref{storytell})
which we will use in the proof of Theorem \ref{mainchar}.

\begin{definition}\label{swarpy}\index{gen}{left exact!at an object $Z$}
Let $f: \calX \rightarrow \calY$ be a functor between $\infty$-categories which admit finite limits. Let
$Z$ be an object of $\calX$. We will say that $f$ is {\it left exact at $Z$} if, for every pullback square
$$ \xymatrix{ W \ar[r] \ar[d] & Y \ar[d] \\
X \ar[r] & Z }$$
in $\calX$, the induced square
$$ \xymatrix{ f(W) \ar[r] \ar[d] & f(Y) \ar[d] \\
f(X) \ar[r] & f(Z) }$$
is a pullback in $\calY$.
\end{definition}

We can now state the main result of this section:

\begin{proposition}\label{storytell}
Let $\calX$ and $\calY$ be presentable $\infty$-categories, and let $f: \calX \rightarrow \calY$ be a functor which preserves small colimits. Suppose that every groupoid object in either $\calX$ or $\calY$ is effective.
Let
$$ \xymatrix{ U_{1} \ar@<.5ex>[r] \ar@<-.5ex>[r] & U_0 \ar[r]^{s} & U_{-1}. } $$ be a coequalizer diagram in $\calX$, and let
$$ \xymatrix{ X \ar[r] \ar[d] & U_0 \ar[d]^{s} \\
U_0 \ar[r]^{s} & U_{-1} }$$ be a pullback diagram in $\calX$. Suppose that $f$ is left exact
at $U_0$. Then the associated diagram 
$$ \xymatrix{ f(X) \ar[r] \ar[d] & f(U_0) \ar[d]^{s} \\
f(U_0) \ar[r]^{s} & f(U_{-1}) }$$
is a pullback square in $\calY$.
\end{proposition}

Before giving the proof, we must establish some preliminary results.

\begin{lemma}\label{drumb}
Let $\calX$ and $\calY$ be $\infty$-categories which admit finite limits, let
$f: \calX \rightarrow \calY$ be a functor, and let $U_{\bigdot}$ be a groupoid object
of $\calX$. Suppose that $f$ is left exact at $U_0$. Then $f \circ U_{\bigdot}$ is a groupoid
object of $\calY$. 
\end{lemma}

\begin{proof}
This follows immediately from characterization $(4'')$ given in Proposition \ref{grpobjdef}.
\end{proof}

Let $\calX$ be a presentable $\infty$-category. We define a {\it simplicial resolution}\index{gen}{resolution!simplicial} in $\calX$ to be an augmented simplicial object $U_{\bigdot}^{+} : \Nerve(\cDelta_{+})^{op} \rightarrow \calX$
which is a colimit of the underlying simplicial object $U_{\bigdot} = U_{\bigdot}^{+} | \Nerve(\cDelta)^{op}$. We let $\Res(\calX)$\index{not}{ResX@$\Res(\calX)$} denote the full subcategory of $\calX_{\Delta_{+}}$ spanned by the simplicial resolutions. Note that since every simplicial object of $\calX$ has a colimit, 
the restriction functor $\Res(\calX) \rightarrow \calX_{\Delta}$ is a trivial fibration, and therefore an equivalence of $\infty$-categories. We will say that a simplicial resolution $U_{\bigdot}^{+}$
is a {\it groupoid resolution}\index{gen}{resolution!groupoid} if the underlying simplicial object
$U_{\bigdot}$ is a groupoid object of $\calX$.

We will say that a map $f: U_{\bigdot}^{+} \rightarrow V_{\bigdot}^{+}$ of simplicial resolutions {\it exhibits $V_{\bigdot}^{+}$ as the groupoid resolution generated by $U_{\bigdot}^{+}$} if $V_{\bigdot}^{+}$ is a groupoid resolution and the induced map $$ \bHom_{\Res(\calX)}( V_{\bigdot}^{+}, W_{\bigdot}^{+} )
\rightarrow \bHom_{\Res(\calX)}( U_{\bigdot}^{+}, W_{\bigdot}^{+} )$$
is a homotopy equivalence for every groupoid resolution $W_{\bigdot}^{+} \in \Res(\calX)$.

\begin{remark}\label{swather}
Let $\calX$ be a presentable $\infty$-category. Then for every simplicial resolution
$U_{\bigdot}^{+}$ in $\calX$, there is a map $f: U_{\bigdot}^{+} \rightarrow V_{\bigdot}^{+}$ which exhibits $V_{\bigdot}^{+}$ as the groupoid resolution generated by $U_{\bigdot}^{+}$. In view of
the equivalence $\Res(\calX) \rightarrow \calX_{\Delta}$, this is equivalent to the assertion that
$\Grp(\calX)$ is a localization of $\calX_{\Delta}$. This follows from Proposition \ref{grpobjdef} together with Lemmas \ref{stur3} and \ref{stur1}.
\end{remark}

\begin{lemma}\label{step1}
Let $\calX$ be a presentable $\infty$-category and let $f: U_{\bigdot}^{+} \rightarrow V_{\bigdot}^{+}$ be a map of simplicial resolutions which exhibits $V_{\bigdot}^{+}$ as the groupoid resolution generated by $U_{\bigdot}^{+}$. Let $W_{\bigdot}^{+}$ be an augmented simplicial object of $\calX$ such that the underlying simplicial object $W_{\bigdot} \in \calX_{\Delta}$
is a groupoid. Composition with $f$ induces a homotopy equivalence
$$ \bHom_{\calX_{\Delta_{+}}}( V_{\bigdot}^{+}, W_{\bigdot}^{+} )
\rightarrow \bHom_{\calX_{\Delta_{+}}}( U_{\bigdot}^{+}, W_{\bigdot}^{+} ).$$
\end{lemma}

\begin{proof}
Let $|W_{\bigdot}|$ be a colimit of $W_{\bigdot}$. Then we have a commutative diagram
$$ \xymatrix{ \bHom_{\calX_{\Delta_{+}}}( V_{\bigdot}^{+}, |W_{\bigdot}| )
\ar[r] \ar[d] &  \bHom_{\calX_{\Delta_{+}}}( U_{\bigdot}^{+}, |W_{\bigdot}| ) \ar[d] \\
\bHom_{\calX_{\Delta_{+}}}( V_{\bigdot}^{+}, W_{\bigdot}^{+} )
\ar[r] & \bHom_{\calX_{\Delta_{+}}}( U_{\bigdot}^{+}, W_{\bigdot}^{+} )} $$
where the vertical maps are homotopy equivalences (since $U_{\bigdot}^{+}$ and $V_{\bigdot}^{+}$ are resolutions) and the upper horizontal map is a homotopy equivalence (since $|W_{\bigdot}|$ is a groupoid resolution).
\end{proof}

\begin{lemma}\label{step9}
Let $\calX$ be a presentable $\infty$-category. Suppose that $f: U_{\bigdot}^{+} \rightarrow V_{\bigdot}^{+}$ be a map in $\Res(\calX)$ which exhibits $V_{\bigdot}^{+}$ as the groupoid resolution generated by $U_{\bigdot}^{+}$. Then $f$ induces equivalences $U_{-1} \rightarrow V_{-1}$ and
$U_{0} \rightarrow V_{0}$.
\end{lemma}

\begin{proof}
Let $\cDelta^{ \leq 0}_{+}$ be the full subcategory of $\cDelta_{+}$ spanned by the objects $\Delta^{-1}$ and $\Delta^0$. Let $j: \cDelta^{\leq 0}_{+} \rightarrow \cDelta_{+}$ denote the inclusion functor, let
$j^{\ast}: \calX_{\Delta_{+}} \rightarrow \calO_{\calX}$ be the associated restriction functor. We wish to show that $j^{\ast}(f)$ is an equivalence. Equivalently, we show that for every $W \in \calO_{\calX}$, 
composition with $j^{\ast}(f)$ induces a homotopy equivalence
$$ \bHom_{\calO_{\calX}}( j^{\ast} V_{\bigdot}^{+}, W )
\rightarrow \bHom_{\calO_{\calX}}( j^{\ast} U_{\bigdot}^{+}, W ).$$

Let $j_{\ast}$ be a right adjoint to $j^{\ast}$ (a right Kan extension functor). It will suffice to prove that composition with $f$ induces a homotopy equivalence
$$ \bHom_{\calX_{\Delta_{+}}}( V_{\bigdot}^{+}, j_{\ast} W )
\rightarrow \bHom_{\calX_{\Delta_{+}}}( U_{\bigdot}^{+}, j_{\ast} W ).$$
The augmented simplicial object $j_{\ast} W$ is a \Cech nerve, so that the underlying simplicial object of $j_{\ast} W$ is a groupoid by Proposition \ref{strump}. We now conclude by applying Lemma \ref{step1}.
\end{proof}

Let $\calI$ denote the subcategory of $\cDelta_{+}$ spanned by the objects
$\emptyset$, $[0]$, and $[1]$, where the morphisms are given by {\em injective} maps of linearly ordered sets.
This category may be depicted as follows: 
$$ \xymatrix{ \emptyset \ar[r] & [0] \ar@<.5ex>[r] \ar@<-.5ex>[r] & [1]} $$
We let $\calI_0$ denote the full subcategory of $\calI$ spanned by the objects $[0]$ and $[1]$. We will say that a diagram $\Nerve(\calI)^{op} \rightarrow \calX$ is a {\it coequalizer diagram} if it is a colimit of its restriction to $\Nerve(\calI_0)^{op} \rightarrow \calX$.

Let $i$ denote the inclusion $\calI \subseteq \cDelta_{+}$, and let $i^{\ast}$ denote the
restriction functor $\calX_{\Delta_{+}} \rightarrow \Fun( \Nerve(\calI)^{op}, \calX)$. If $\calX$ is a presentable $\infty$-category, then $i^{\ast}$ has a left adjoint $i_{!}$ (a left Kan extension).

\begin{lemma}\label{step7}
Let $\calX$ be a presentable $\infty$-category. The left Kan extension
$i_{!}: \Fun( \Nerve(\calI)^{op}, \calX) \rightarrow \calX_{\Delta_{+}}$ carries
coequalizer diagrams to simplicial resolutions.
\end{lemma}

\begin{proof}
We have a commutative diagram of inclusions of subcategories
$$ \xymatrix{ \calI_0 \ar[r]^{j'} \ar[d]^{i'} & \calI \ar[d]^{i} \\
\cDelta \ar[r]^{j} & \cDelta_{+} }$$
which gives rise to a homotopy commutative diagram of $\infty$-categories
$$ \xymatrix{ \Fun( \Nerve(\calI_0)^{op}, \calX) \ar[r]^{j'_{!}} \ar[d]^{i'_{!}} & 
\Fun( \Nerve(\calI)^{op}, \calX) \ar[d]^{i_{!}} \\
\calX_{\Delta} \ar[r]^{j_{!}} & \calX_{\Delta_{+}} }$$
in which the morphisms are given by left Kan extensions. An object $U \in \Fun( \Nerve(\calI)^{op} \calX)$ is a coequalizer diagram if and only if it lies in the essential image of $j'_{!}$. In
this case, $i_{!} U$ lies in the essential image of $i_{!} \circ j'_{!} \simeq j_{!} \circ i'_{!}$, 
which is contained in the essential image of $j_{!}$: namely, the resolutions. 
\end{proof}

\begin{lemma}\label{step8}
Let $\calX$ be a presentable $\infty$-category and suppose given a diagram
$U: \Nerve(\calI)^{op} \rightarrow \calC$, which we may depict as
$$ \xymatrix{ U_{1} \ar@<.5ex>[r] \ar@<-.5ex>[r] & U_0 \ar[r] & U_{-1}. } $$
Let $V_{\bigdot} = i_{!} U \in \calX_{\Delta_{+}}$ be a left Kan extension of $U$ along
$i: \Nerve(\calJ)^{op} \rightarrow \cDelta_{+}^{op}$. 
Then the augmentation maps $V_0 \rightarrow V_{-1}$ and $U_0 \rightarrow U_{-1}$
are equivalent in the $\infty$-category $\calO_{\calX}$.
\end{lemma}

\begin{proof}
This follows from Proposition \ref{timeless}, since $\Hom_{\calI}(\Delta^i, \bigdot)
\simeq \Hom_{\cDelta_{+}}( \Delta^i, \bigdot )$ for $i \leq 0$.
\end{proof}

\begin{proof}[Proof of Proposition \ref{storytell}]
Let $U: \Nerve(\calI)^{op} \rightarrow \calC$ be a coequalizer diagram in $\calX$, which we denote by
$$ \xymatrix{ U_{1} \ar@<.5ex>[r] \ar@<-.5ex>[r] & U_0 \ar[r]^{s} & U_{-1}, } $$
and form a pullback square
$$ \xymatrix{ X \ar[r] \ar[d] & U_0 \ar[d]^{s} \\
U_0 \ar[r]^{s} & U_{-1}. }$$ 

Let $V_{\bigdot} = i_{!} U \in \calX_{\Delta_{+}}$ be a left Kan extension of $U$. According to Lemma \ref{step7}, $V_{\bigdot}$ is a simplicial resolution. We may therefore choose a map
$V_{\bigdot} \rightarrow W_{\bigdot}$ which exhibits $W_{\bigdot}$ as the groupoid resolution generated by $V_{\bigdot}$ (Remark \ref{swather}). Since every groupoid object in $\calX$ is effective, $W_{\bigdot}$ is a \Cech nerve. It follows from the characterization given in Proposition \ref{strump} that there is a pullback diagram
$$ \xymatrix{ W_{1} \ar[r] \ar[d] & W_{0} \ar[d] \\
W_0 \ar[r] & W_{-1} }$$
in $\calX$. Using Lemma \ref{step8} and Lemma \ref{step9}, we see that this diagram is
equivalent to the pullback diagram $$ \xymatrix{ X \ar[r] \ar[d] & U_0 \ar[d]^{s} \\
U_0 \ar[r]^{s} & U_{-1}. }$$ It therefore suffices to prove that the induced diagram
$$ \xymatrix{ f(W_{1}) \ar[r] \ar[d] & f(W_{0}) \ar[d] \\
f(W_0) \ar[r] & f(W_{-1}) }$$
is a pullback. We make a slightly stronger claim: the augmented simplicial object
$f \circ W_{\bigdot}$ is a \Cech nerve.
Since every groupoid object in $\calY$ is effective, it will suffice to prove that $f \circ W_{\bigdot}$ is a groupoid resolution. Since $f$ preserves colimits, it is clear that $f \circ W_{\bigdot}$ is a simplicial resolution. It follows from Lemma \ref{drumb} that the underlying simplicial object of $f \circ W_{\bigdot}$ is a groupoid.
\end{proof}

\subsection{Giraud's Theorem for $\infty$-Topoi}\label{proofgiraud}

In this section, we will complete the proof of Theorem \ref{mainchar} by showing that every $\infty$-category $\calX$ which satisfies the $\infty$-categorical Giraud axioms $(i)$ through $(iv)$ arises as a left exact localization of an $\infty$-category of presheaves. Our strategy is simple: we choose a small category $\calC$ equipped with a functor $f: \calC \rightarrow \calX$. According to Theorem \ref{charpresheaf}, we obtain a colimit-preserving functor $F: \calP(\calC) \rightarrow \calX$ which extends $f$, up to homotopy. We will apply Proposition \ref{storytell} to show that, under suitable hypotheses, $F$ is a left exact localization functor (Proposition \ref{natash}). 

\begin{lemma}\label{sumdescription}
Let $\calX$ be a presentable $\infty$-category in which colimits are universal and coproducts are disjoint.

Let $\{ \phi_i: Z_i \rightarrow Z\}_{i \in I}$ be a family of morphisms in $\calX$ which exhibit $Z$ as a coproduct of the family of objects $\{Z_i\}_{i \in I}$. Let
$$ \xymatrix{ W \ar[r]^{\alpha} \ar[d] & Z_i \ar[d]^{\phi_i} \\
Z_j \ar[r]^{\phi_j} & Z}$$ be a square diagram in $\calX$. Then:
\begin{itemize}
\item[$(1)$] If $i \neq j$, then the diagram is a pullback square if and only if $W$ is an initial object of $\calX$.
\item[$(2)$] If $i = j$, then the diagram is a pullback square if and only if $\alpha$
is an equivalence.
\end{itemize}
\end{lemma}

\begin{proof}
Let $Z_{i}^{\vee}$ be a coproduct for the objects $\{ Z_{k} \}_{k \in I, k \neq i}$, and
let $\psi: Z_{i}^{\vee} \rightarrow Z$ be a morphism such that each of the compositions
$$ Z_{k} \rightarrow Z_{i}^{\vee} \stackrel{\psi}{\rightarrow} Z$$
is equivalent to $Z$. Then there is a pushout square
$$ \xymatrix{ \emptyset \ar[r]^{\beta} \ar[d] & Z_i \ar[d]^{\phi_i} \\
Z_i^{\vee} \ar[r]^{\psi} & Z }$$
where $\emptyset$ denotes an initial object of $\calX$.
Since coproducts in $\calX$ are disjoint, this pushout square is also a pullback.

Let $\phi_{i}^{\ast}: \calX^{/Z} \rightarrow \calX^{/Z_i}$ denote a pullback functor. The above
argument shows that $\phi_{i}^{\ast}(\psi)$ is an initial object of $\calX^{/Z_i}$.
If $j \neq i$, then there is a map of arrows $\phi_j \rightarrow \psi$ in $\calX^{/Z}$, and therefore a map $\phi_i^{\ast}(\phi_j) \rightarrow \phi_i^{\ast}(\psi)$ in $\calX^{/Z_i}$. Consequently,
if $\alpha \simeq \phi_i^{\ast}(\phi_j)$, then $W$ admits a map to an initial object of $\calX$, and is therefore itself initial by Lemma \ref{sumoto}. This proves the ``only if'' direction of $(1)$. The converse follows from the uniqueness of initial objects.

Now suppose that $i = j$. We observe that $\id_{/Z}$ is a coproduct of $\phi_i$ and $\psi$ in the
$\infty$-category $\calX^{/Z}$. Since $\phi_i^{\ast}$ preserves coproducts, we deduce
that $\id_{Z_i}$ is a coproduct of $\phi^{\ast}(\phi_j): X \rightarrow Z_i$ and $\beta: \emptyset \rightarrow Z_i$ in $\calX^{/Z_i}$. Since $\beta$ is an initial object of $\calX^{/Z_i}$, we see that
$\phi^{\ast}(\phi_j)$ is an equivalence. The natural map $\gamma: \alpha \rightarrow \phi_i^{\ast}(\phi_i)$
corresponds to a commutative diagram
$$ \xymatrix{ W \ar[r]^{\alpha} \ar[d]^{\gamma_0} & Z_i \ar[d]^{\id_{Z_i} } \ar[d] \\
X \ar[r]^{ \phi_i^{\ast}(\phi_i)} &  Z_i }$$
in the $\infty$-category $\calX$. Consequently, $\alpha$ is an equivalence if and only if
$\gamma_0$ is an equivalence, if and only if $\gamma$ is an equivalence in
$\calX^{/Z_i}$. This proves $(2)$.
\end{proof}

\begin{proposition}\label{natash}
Let $\calC$ be a small $\infty$-category which admits finite limits, and let
$\calX$ be an $\infty$-category which satisfies the $\infty$-categorical Giraud axioms
$(i)-(iv)$ of Theorem \ref{mainchar}. Let $F: \calP(\calC) \rightarrow \calX$ be
a colimit-preserving functor. Suppose that the composition
$F \circ j: \calC \rightarrow \calX$ is left exact, where $j: \calC \rightarrow \calP(\calC)$ denotes the Yoneda embedding. Then $F$ is left exact.
\end{proposition}

\begin{proof}
According to Corollary \ref{allfinn}, to prove that $F$ is left exact, it will suffice to prove that $F$ preserves pullbacks and final objects. Since all final objects are equivalent, to prove that $F$ preserves final objects it suffices to exhibit a single final object $Z$ of $\calP(\calC)$ such that $FY \in \calX$ is final. Let $z$ be a final object of $\calC$ (which exists in virtue of our assumption that $\calC$ admits finite limits). Then $Z = j(z)$ is a final object of $\calP(\calC)$, since $j$ preserves limits by Proposition \ref{yonedaprop}. Consequently $F(Z) = f(z)$ is final, since $f$ is left-exact.

Let $\alpha: Y \rightarrow Z$ be a morphism in $\calP(\calC)$. We will say that $\alpha$ is {\it good} if for every pullback square
$$ \xymatrix{ W \ar[r] \ar[d] & Y \ar[d]^{\alpha} \\
X \ar[r] & Z}$$ 
in $\calP(\calC)$, the induced square
$$ \xymatrix{ F(W) \ar[r] \ar[d] & F(Y) \ar[d]^{F(\alpha)} \\
F(X) \ar[r]^{\beta} & F(Z)}$$ 
is a pullback in $\calX$. Note that Lemma \ref{transplantt} implies that the class of good morphisms in $\calP(\calC)$ is stable under composition.

We rephrase this condition that a morphism $\alpha$ be good in terms of the pullback functors
$\alpha^{\ast}: \calP(\calC)^{/Z} \rightarrow \calP(\calC)^{/Y}$, $F(\alpha)^{\ast}: \calX^{/F(Z)} \rightarrow \calP(\calC)^{/F(Y)}$. Application of the functor $F$ gives a map
$$ t: F \circ \alpha^{\ast} \rightarrow F(\alpha)^{\ast} \circ F$$ in the $\infty$-category
of functors from $\calP(\calC)^{/Z}$ to $\calX^{/F(Z)}$, and $\alpha$ is good if and only if $t$ is an equivalence. Note that $t$ is a natural transformation of colimit-preserving functors. Since the image of the Yoneda embedding $j: \calC \rightarrow \calP(\calC)$ generates $\calP(\calC)$ under colimits, it will suffice to prove that $t$ is an equivalence when evaluated on objects
of the form $\beta: j(x) \rightarrow Z$, where $x$ is an object of $\calC$.

Let us say that an object $Z \in \calP(\calC)$ is {\it good} if every morphism $\alpha: Y \rightarrow Z$ is good. In other words, an object $Z \in \calP(\calC)$ is good if $F$
is left exact at $Z$ in the sense of Definition \ref{swarpy}.
By repeating the above argument, we deduce that $Z$ is good if and only if every morphism of the form $\alpha: j(y) \rightarrow Z$ is good for $y \in \calC$. 

We next claim that for every object $z \in \calC$, the Yoneda image $j(z) \in \calP(\calC)$ is good.
In other words, we must show that for every pullback square
$$ \xymatrix{ W \ar[r] \ar[d] & j(y) \ar[d]^{\alpha} \\
j(x) \ar[r]^{\beta} & j(z) }$$
in $\calP(\calC)$, the induced square
$$ \xymatrix{ F(W) \ar[r] \ar[d] & f(x) \ar[d] \\
f(y) \ar[r] & f(z)} $$
is a pullback in $\calX$. Since the Yoneda embedding is fully faithful, we may suppose
that $\alpha$ and $\beta$ are the Yoneda images of morphisms $x \rightarrow z$,
$y \rightarrow z$. Since $j$ preserves limits, we may reduce to the case where the first
diagram is the Yoneda image of a pullback diagram in $\calC$. The desired result then follows from the assumption that $f$ is left exact.

To complete the proof that $F$ is left exact, it will suffice to prove that every object
of $\calP(\calC)$ is good. Because the Yoneda embedding $j: \calC \rightarrow \calP(\calC)$
generates $\calP(\calC)$ under colimits, it will suffice to prove that the collection of good objects of $\calP(\calC)$ is stable under colimits. According to Proposition \ref{appendicites}, it will suffice to prove that the collection of good objects of $\calP(\calC)$ is stable under coequalizers and small coproducts. 

We first consider the case of coproducts. Let $\{ Z_{i} \}_{i \in I}$ be a family of good objects
of $\calP(\calC)$ indexed by a (small) set $I$, and $\{ \phi_i: Z_i \rightarrow Z\}_{i \in I}$ be a family
of morphisms which exhibit $Z$ as a coproduct of the family $\{ Z_i \}_{i \in I}$.
Suppose given a pullback diagram
$$ \xymatrix{ W \ar[r] \ar[d] & j(y) \ar[d]^{\alpha} \\
j(x) \ar[r]^{\beta} & Z }$$
in $\calP(\calC)$. According to Proposition \ref{limiteval}, evaluation at the object
$y$ induces a colimit-preserving functor $\calP(\calC) \rightarrow \SSet$. Consequently,
we have a homotopy equivalence
$$ \bHom_{\calP(\calC)}( j(y), Z ) \simeq 
\coprod_{i \in I} \bHom_{\calP(\calC)}( j(y), Z_i) $$
in the homotopy category $\calH$. Therefore we may assume that
$\alpha$ factors as a composition
$$ j(y) \stackrel{\alpha'}{\rightarrow} Z_i \stackrel{\phi_i}{\rightarrow} Z$$
for some $i \in I$. By assumption, the morphism $\alpha'$ is good; it therefore suffices
to prove that $\phi_i$ is good. By a similar argument, we can replace
$\beta$ by a map $\phi_j: Z_j \rightarrow Z$, for some $j \in I$. We are now required to show
that if $$ \xymatrix{ W' \ar[r] \ar[d] & Z_i \ar[d]^{\phi_i} \\
Z_j \ar[r]^{\phi_j} & Z }$$ is a pullback diagram in $\calP(\calC)$, then
$$ \xymatrix{ F(W') \ar[r] \ar[d] & F(Z_i) \ar[d]^{\phi_i} \\
Z_j \ar[r]^{\phi_j} & Z }$$
is a pullback diagram in $\calX$. Since $F$ preserves initial objects, this follows immediately
from Lemma \ref{sumdescription}.

We now complete the proof by showing that the collection of good objects of $\calP(\calC)$ is stable under the formation of coequalizers. Let
$$ \xymatrix{ Z_{1} \ar@<.5ex>[r] \ar@<-.5ex>[r] & Z_0 \ar[r]^{s} & Z_{-1}. } $$ 
be a coequalizer diagram in $\calP(\calC)$, and suppose that $Z_0$ and $Z_1$ are good. We must show that any pullback diagram
$$ \xymatrix{ W \ar[r] \ar[d] & j(y) \ar[d]^{\alpha} \\
j(x) \ar[r]^{\beta} & Z_{-1} }$$
remains a pullback diagram after applying the functor $F$. The functor
$$ \calP(\calC) \rightarrow \Nerve(\Set)$$
$$ T \mapsto \Hom_{\h{\calP(\calC)}}( j(x), T)$$
can be written as a composition
$$ \calP(\calC) \rightarrow \SSet \stackrel{\pi_0}{\rightarrow} \Nerve(\Set)$$
where the first functor is given by evaluation at $x$. Both of these functors commute
with colimits. Consequently, we have a coequalizer diagram
$$ \xymatrix{ \Hom_{\h{\calP(\calC)}}(j(x), Z_{1}) \ar@<.5ex>[r] \ar@<-.5ex>[r] & 
\Hom_{\h{\calP(\calC)}}(j(x),Z_0) \ar[r] & \Hom_{\h{\calP(\calC)}}(j(x),Z_{-1}). } $$ 
in the category of sets. In particular, the map $\beta$ factors as a composition
$$ j(x) \stackrel{\beta'}{\rightarrow} Z_0 \stackrel{s}{\rightarrow} Z_{-1}. $$
Since we have already assumed that $\beta'$ is good, we can replace
$\beta$ by the map $s: Z_0 \rightarrow Z_{-1}$ in the above diagram. By a similar argument, we can replace $\alpha: Y \rightarrow Z_{-1}$ by the map $s: Z_0 \rightarrow Z_{-1}$. We now obtain the desired result by applying Proposition \ref{storytell}.
\end{proof}

We are now ready to complete the proof of Theorem \ref{mainchar}:

\begin{proposition}\label{precisechar}
Let $\calX$ be an $\infty$-category. Suppose that $\calX$ satisfies the $\infty$-categorical
Giraud axioms:
\begin{itemize}
\item[$(i)$] The $\infty$-category $\calX$ is presentable.
\item[$(ii)$] Colimits in $\calX$ are universal.
\item[$(iii)$] Coproducts in $\calX$ are disjoint.
\item[$(iv)$] Every groupoid object of $\calX$ is effective.
\end{itemize}
Then there exists a small $\infty$-category $\calC$ which admits finite limits, and an accessible left exact localization functor $\calP(\calC) \rightarrow \calX$. In particular, $\calX$ is an $\infty$-topos.
\end{proposition}

\begin{proof}
Let $\calX$ be an $\infty$-topos. According to Proposition \ref{tcoherent}, there exists a regular cardinal $\tau$ such that $\calX$ is $\tau$-accessible, and the full subcategory $\calX^{\tau}$ spanned by the $\tau$-compact objects of $\calX$ is stable under finite limits.
Let $\calC$ be a minimal model for $\calX^{\tau}$, so that there is an equivalence $\Ind_{\tau}(\calC) \rightarrow \calX$. The proof of Theorem \ref{pretop} shows that 
the inclusion $\Ind_{\tau}(\calC) \subseteq \calP(\calC)$ has a left adjoint $L$. 
The composition of $L$ with the Yoneda embedding $\calC \rightarrow \calP(\calC)$ can be identified with the Yoneda embedding $\calC \rightarrow \Ind_{\tau}(\calC)$, therefore preserves all limits which exist in $\calC$ (Proposition \ref{yonedaprop}). Applying Proposition \ref{natash}, we deduce that $L$ is left exact, so that $\Ind_{\tau}(\calC)$ is a left exact localization (automatically accessible) of $\calP(\calC)$. Since $\calX$ is equivalent to $\Ind_{\tau}(\calC)$, we conclude that
$\calX$ is also an accessible left exact localization of $\calP(\calC)$.
\end{proof}

\subsection{$\infty$-Topoi and Classifying Objects}\label{rezk2}

Let $\calX$ be an ordinary category, and let $X$ be an object of $\calX$. Let $\Sub(X)$\index{not}{SubX@$\Sub(X)$}
denote the partially ordered collection of {\it subobjects} of $X$: an object of $\Sub(X)$ is an equivalence class of monomorphisms $Y \rightarrow X$\index{gen}{subobject}. If $\calC$ is accessible, then $\Sub(X)$ is actually a set. If $\calX$ admits finite limits, then $\Sub(X)$ is contravariantly functorial in $X$:
given a subobject $Y \rightarrow X$ and any map $X' \rightarrow X$, the fiber product
$Y' = X' \times_{X} Y$ is a subobject of $X'$. A {\it subobject classifier}\index{gen}{classifying map!for subobjects} is an object $\Omega$ of $\calX$ which {\em represents} the functor $\Sub$. In other words, $\Omega$ has a universal subobject 
$\Omega_0 \subseteq \Omega$ such that every monomorphism $Y \rightarrow X$ fits into a {\em unique} Cartesian diagram
$$ \xymatrix{ Y \ar[r] \ar@{^{(}->}[d] & \Omega_0 \ar@{^{(}->}[d] \\
X \ar[r] & \Omega.}$$
(In this case, $\Omega_0$ is automatically a final object of $\calC$.)

Every topos has a subobject classifier. In fact, in the theory of {\it elementary topoi}, the existence of a subobject classifier is taken as one of the axioms. Thus, the existence of a subobject classifier is one of the defining characteristics of a topos. We would like to discuss the appropriate $\infty$-categorical generalization of the theory of subobject classifiers. The ideas presented here are due to Charles Rezk.

\begin{definition}\label{ugatoo}
Let $\calX$ be an $\infty$-category which admits pullbacks, and $S$ a collection of morphisms
of $\calX$ which is stable under pullback. We will say that a morphism $f: X \rightarrow Y$ {\it classifies $S$} if it is a final object of $\calO_{\calX}^{(S)}$ (see Notation \ref{ugaboo}). 
In this situation, we will also say that the object $Y \in \calX$ {\it classifies $S$}\index{gen}{classifying map!for a collection of morphisms}. 
A {\it subobject classifier} for $\calX$ is an object which classifies the collection of all monomorphisms in $\calX$.
\end{definition}

\begin{example}
The $\infty$-category $\SSet$ of spaces has a subobject classifier: namely, the discrete space $\{ 0, 1\}$ with two elements.
\end{example}

The following result provides a necessary and sufficient condition for the existence of a classifying object for $S$: 

\begin{proposition}\label{classexist}
Let $\calX$ be a presentable $\infty$-category in which colimits are universal, and let $S$
be a class of morphisms in $\calX$ which is stable under pullbacks. There exists a classifying object for $S$ if and only if the following conditions are satisfied:
\begin{itemize}
\item[$(1)$] The class $S$ is local $($Definition \ref{localitie}$)$.
\item[$(2)$] For every object $X \in \calX$, the full subcategory of
$\calX_{/X}$ spanned by the elements of $S$ is essentially small.
\end{itemize}
\end{proposition}

\begin{proof}
Let $s: \calX^{op} \rightarrow \hat{\SSet}$ be a functor which classifies the right fibration
$\calO_{\calX}^{(S)} \rightarrow \calX$. Then $S$ has a classifying object if and only if
$s$ is a representable functor. According to the representability criterion of Proposition \ref{representable}, this is equivalent to the assertion that $s$ preserves small limits, and the essential image of $s$ consists of essentially small spaces. According to Lemma \ref{ib3}, $s$ preserves small limits if and only if $(1)$ is satisfied. It now suffices to observe that for each
$X \in \calX$, the space $s(X)$ is essentially small if and only if the full subcategory of
$\calX_{/X}$ spanned by $S$ is essentially small. 
\end{proof}

Using Proposition \ref{classexist}, one can show that every $\infty$-topos has a subobject classifier. However, in the $\infty$-categorical context, the emphasis on {\em subobjects} misses the point. To see why, let us return to considering an ordinary category $\calX$ with a subobject classifier $\Omega$. By definition, for every object $X \in \calX$, we may identify maps $X \rightarrow \Omega$ with subobjects of $X$: that is, isomorphism classes of maps $Y \rightarrow X$ which happen to be monomorphisms. Even better would be an {\it object classifier}\index{gen}{classifying map!for objects}: that is, an object $\widetilde{\Omega}$ such that $\Hom_{\calX}(X, \widetilde{\Omega})$ could be identified with {\em arbitrary} maps $Y \rightarrow X$. But this is an unreasonable demand: if $Y \rightarrow X$ is not an monomorphism, then there may be automorphisms of $Y$ as an object of $\calX_{/X}$. It would be unnatural to ignore these automorphisms. However, it is also not possible to take them into account, since $\Hom_{\calX}(X, \widetilde{\Omega})$ must be a set rather than a groupoid.

If we allow $\calX$ to be an $\infty$-category, this objection loses its force. Informally speaking, we can consider the functor which associates to each $X \in \calX$ the maximal $\infty$-groupoid contained in $\calX_{/X}$ (this is contravariantly functorial in $X$, provided that $\calX$ has finite limits). We might hope that this functor is representable by some $\Omega_{\infty} \in \calX$, which we would then call an {\it object classifier}.

Unfortunately, a new problem arises: it is generally unreasonable to ask for the collection of {\em all} morphisms in $\calX$ to be classified by an object of $\calX$, since this would require each slice $\calX_{/X}$ to be essentially small (Proposition \ref{classexist}). This is essentially a technical difficulty, which we will circumvent by introducing a cardinality bound.

\begin{definition}\index{gen}{relatively $\kappa$-compact morphism}
Let $\calX$ be a presentable $\infty$-category. We will say that a morphism
$f: X \rightarrow Y$ is {\it relatively $\kappa$-compact} if, for every pullback diagram
$$ \xymatrix{ X' \ar[r] \ar[d]^{f'} & X \ar[d]^{f} \\
Y' \ar[r] & Y }$$
such that $Y'$ is $\kappa$-compact, $X'$ is also $\kappa$-compact.
\end{definition}

\begin{lemma}\label{sumarus}
Let $\calX$ be a presentable $\infty$-category, $\kappa$ a regular cardinal, $\calJ$ a $\kappa$-filtered $\infty$-category, and $\overline{p}: \calJ^{\triangleright} \rightarrow \calX$ a colimit diagram. Let $f: X \rightarrow Y$ be a morphism in $\calX$, where $Y$ is the image
under $\overline{p}$ of the cone point of $\calJ^{\triangleright}$. For each $\alpha$ in $\calJ$, let $Y_{\alpha} = \overline{p}(\alpha)$ and form a pullback diagram
$$ \xymatrix{ X_{\alpha} \ar[r] \ar[d]^{f_{\alpha}} & X \ar[d]^{f} \\
Y_{\alpha} \ar[r]^{g_{\alpha}} & Y. }$$
Suppose that each $f_{\alpha}$ is relatively $\kappa$-compact. Then $f$ is relatively $\kappa$-compact.
\end{lemma}

\begin{proof}
Let $Z$ be a $\kappa$-compact object of $\calX$, and $g: Z \rightarrow Y$ a morphism.
Since $Z$ is $\kappa$-compact and $\calJ$ is $\kappa$-filtered, there exists a $2$-simplex of $\calX$, corresponding to a diagram
$$ \xymatrix{ & Y_{\alpha} \ar[dr]^{g_{\alpha}} & \\
Z \ar[ur] \ar[rr]^{g} & & Y. }$$
Form a Cartesian rectangle $\Delta^2 \times \Delta^1 \rightarrow \calX$, which we will depict
as
$$ \xymatrix{ Z' \ar[r] \ar[d]^{f'} & X_{\alpha} \ar[r] \ar[d]^{f_{\alpha}} & X \ar[d]^{f} \\
Z \ar[r] & Y_{\alpha} \ar[r] & Y. }$$
Since $f'$ is a pullback of $f_{\alpha}$, we conclude that $Z'$ is $\kappa$-compact. 
Lemma \ref{transplantt} implies that $f'$ is also a pullback of $f$ along $g$, so that $f$ is relatively $\kappa$-compact as desired.
\end{proof}

\begin{lemma}\label{sumaris}
Let $\calX$ be a presentable $\infty$-category in which colimits are universal. Let $\tau > \kappa$ be regular cardinals such that
$\calX$ is $\kappa$-accessible and the full subcategory $\calX^{\tau}$ consisting of $\tau$-compact objects of $\calX$ is stable under pullbacks in $\calX$. Let $\alpha: \sigma \rightarrow \sigma'$ be a Cartesian transformation between pushout squares $\sigma, \sigma': \Delta^1 \times \Delta^1 \rightarrow \calX$, which we may view as a pushout square
$$ \xymatrix{ f \ar[r]^{\alpha} \ar[d]^{\beta} & g \ar[d]^{\beta'} \\
f' \ar[r]^{\alpha'} & g' }$$
in $\Fun(\Delta^1,\calX)$. Suppose that $f$, $g$, and $f'$ are relatively $\tau$-compact. Then $g'$ is relatively $\tau$-compact. 
\end{lemma}

\begin{proof}
Let $\calC$ denote the full subcategory of $\Fun(\Delta^1 \times \Delta^1, \calX)$ spanned by the pushout squares, and let $\calC^{\tau} = \calC \cap \Fun( \Delta^1 \times \Delta^1, \calX^{\tau})$.
Since the class of $\tau$-compact objects of $\calX$ is stable under pushouts (Corollary \ref{tyrmyrr}), we have a commutative diagram
$$ \xymatrix{ \calC^{\tau} \ar[r] \ar[d] & \Fun( \Lambda^2_0, \calX^{\tau})  \ar[d] \\
\calC \ar[r] & \Fun(\Lambda^2_0, \calX) }$$
where the horizontal arrows are trivial fibrations (Proposition \ref{lklk}). The proof of Proposition
\ref{horse1} shows that every object of $\Fun(\Lambda^2_0, \calX)$ can be written as the colimit of a $\tau$-filtered diagram in $\Fun(\Lambda^2_0, \calX^{\tau})$. It follows that $\sigma' \in \calC$
can be obtained as the colimit of a $\tau$-filtered diagram in $\calC^{\tau}$. Since
colimits in $\calX$ are universal, we conclude that the natural transformation $\alpha$
can be obtained as a $\tau$-filtered colimit of natural transformations
$ \alpha_{i}: \sigma_{i} \rightarrow \sigma'_{i}$ in $\calC^{\tau}$. Lemma \ref{limitscommute} implies that the inclusion $\calC \subseteq \Fun(\Delta^1 \times \Delta^1, \calX)$ is colimit-preserving. Consequently, we deduce that $g'$ can be written as a $\tau$-filtered colimit of morphisms
$\{ g'_{i} \}$ determined by restricting $\{ \alpha_i \}$. According to Lemma \ref{sumaris}, it will suffice to prove that each morphism $g'_{i}$ is relatively $\tau$-compact. In other words, we may replace $\sigma'$ by $\sigma'_i$ and thereby reduce to the case where $\sigma'$ belongs to $\calC^{\tau}$. Since $f,g,$ and $f'$ are relatively $\tau$-compact, we conclude that
$\sigma | \Lambda^2_0$ takes values in $\calX^{\tau}$. Since $\sigma$ is a pushout diagram, Corollary \ref{tyrmyrr} implies that $\sigma$ takes values in $\calX^{\tau}$. Now we observe that $g'$ is a morphism between $\tau$-compact objects of $\calX$, and therefore automatically relatively $\tau$-compact in virtue of our assumption that $\calX^{\tau}$ is stable under pullbacks in $\calX$.
\end{proof}


\begin{proposition}\label{cardyp}  
Let $\calX$ be a presentable $\infty$-category in which colimits are universal, and let
$S$ be a local class of morphisms in $\calX$. For each regular cardinal $\kappa$, 
let $S_{\kappa}$ denote the collection of all morphisms $f$ which belong to $S$ and are relatively $\kappa$-compact. If $\kappa$ is sufficiently large, then $S_{\kappa}$ has a classifying object.
\end{proposition}

\begin{proof}
Choose $\kappa'$ such that $\calX$ is $\kappa'$-accessible.
The restriction functor $r: 
\Fun( (\Lambda^2_2)^{\triangleleft}, \calX) \rightarrow \Fun( \Lambda^2_2, \calX)$
is accessible: in fact, it preserves all colimits (Proposition \ref{limiteval}). Let $g$ be a right adjoint to $r$ (a limit functor); Proposition \ref{adjoints} implies that $g$ is also accessible. Choose a regular cardinal $\kappa'' > \kappa'$ such that $g$ is $\kappa''$-continuous, and choose $\kappa \geq \kappa''$ such that $g$ carries $\kappa''$-compact objects of 
$\Fun(\Lambda^2_2, \calX)$ into $\Fun( (\Lambda^2_2)^{\triangleleft}, \calX^{\kappa})$. It follows that the class of $\kappa$-compact objects of $\calX$ is stable under pullbacks. We will show that $S_{\kappa}$ has a classifying object.

We will verify the hypotheses of Proposition \ref{classexist}. First, we must show that $S_{\kappa}$ is local. For this, we will verify condition $(3)$ of Lemma \ref{ib3}. We begin by showing that
$S_{\kappa}$ is stable under small coproducts. Let $\{ f_{\alpha}: X_{\alpha} \rightarrow Y_{\alpha} \}_{\alpha \in A}$ be a small collection of morphisms belonging to $S_{\kappa}$, and let $f: X \rightarrow Y$
be a coproduct $\coprod_{\alpha \in A} f_{\alpha}$ in $\Fun(\Delta^1,\calX)$. We wish to show that $f \in S_{\kappa}$. Since $S$ is local, we conclude that $f \in S$ (using Lemma \ref{ib3}). It therefore suffices to show that $f$ is relatively $\kappa$-compact. Suppose given a $\kappa$-compact
object $Z \in \calX$ and a morphism $g: Z \rightarrow Y$. Using Proposition \ref{extet} and Corollary \ref{util}, we conclude that $Y$ can be obtained as a $\kappa$-filtered colimit of objects
$Y_{A_0} = \coprod_{ \alpha \in A_0} Y_{\alpha}$, where $A_0$ ranges over the $\kappa$-small subsets of $A$. Since $Z$ is $\kappa$-compact, we conclude that there exists a factorization
$$ Z \stackrel{g'}{\rightarrow} Y_{A_0} \stackrel{g''}{\rightarrow} Y$$
of $g$. Form a Cartesian rectangle $\Delta^2 \times \Delta^1 \rightarrow \calX$,
$$ \xymatrix{ Z' \ar[r] \ar[d] & X_{A_0} \ar[r] \ar[d] & X \ar[d] \\
Z \ar[r] & Y_{A_0} \ar[r] & Y. }$$
Since $S$ is local, we can identify $X_{A_0}$ with the coproduct $\coprod_{\alpha \in A_0} X_{\alpha}$. Since colimits are universal, we conclude that $Z'$ is a coproduct of
objects $Z'_{\alpha} = X_{\alpha} \times_{Y_{\alpha} } Z$, where $\alpha$ ranges over
$A_0$. Since each $f_{\alpha}$ is relatively $\kappa$-compact, we conclude that each $Z'_{\alpha}$ is $\kappa$-compact. Thus $Z'$, as a $\kappa$-small colimit of $\kappa$-compact objects, is also $\kappa$-compact (Corollary \ref{tyrmyrr}). 

We must now show that for every pushout diagram
$$ \xymatrix{ f \ar[r]^{\alpha} \ar[d]^{\beta} & g \ar[d]^{\beta'} \\
f' \ar[r]^{\alpha'} & g' }$$
in $\calO_{\calX}$, if $\alpha$ and $\beta$ are Cartesian transformations and
$f,f',g \in S_{\kappa}$, then $\alpha'$ and $\beta'$ are also Cartesian transformations and $g' \in S_{\kappa}$. The first assertion follows immediately from Lemma \ref{ib3} (since $S$ is local), and we deduce also that $g' \in S$. It therefore suffices to show that $g$ is relatively $\kappa$-compact, which follows from Lemma \ref{sumaris}.

It remains to show that, for each $X \in \calX$, the full subcategory of $\calX_{/X}$ spanned by
the elements of $S$ is essentially small. Equivalently, we must show that the right fibration
$p: \calO_{\calX}^{(S)} \rightarrow \calX$ has essentially small fibers. Let
$F: \calX^{op} \rightarrow \hat{\SSet}$ classify $p$. Since $S$ is local, $F$ preserves limits.
The full subcategory of $\hat{\SSet}$ spanned by the essentially small Kan complexes is stable under small limits, and $\calX$ is generated by $\calX^{\kappa}$ under small ($\kappa$-filtered) colimits. Consequently, it will suffice to show that $F(X)$ is essentially small, when $X$ is $\kappa$-compact. In other words, we must show that there are only a bounded number equivalence classes of morphisms $f: Y \rightarrow X$ such that $f \in S_{\kappa}$. We now observe that if
$f \in S_{\kappa}$, then $f$ is relatively $\kappa$-compact, so that $Y$ also belongs to $\calX^{\kappa}$. We now conclude by observing that the $\infty$-category $\calX^{\kappa}$ is essentially small.
\end{proof}

We now give a characterization of $\infty$-topoi based on the existence of object classifiers.

\begin{theorem}[Rezk]\label{colimsurt}\index{gen}{classifying map!for relatively $\kappa$-compact morphisms}
Let $\calX$ be a presentable $\infty$-category. Then
$\calX$ is an $\infty$-topos if and only if the following conditions are satisfied:
\begin{itemize}
\item[$(1)$] Colimits in $\calX$ are universal.
\item[$(2)$] For all sufficiently large regular cardinals $\kappa$, there exists
a classifying object for the class of all relatively $\kappa$-compact morphisms
in $\calX$.
\end{itemize}
\end{theorem}

\begin{proof}
Assume that colimits in $\calX$ are universal. According to Theorems \ref{mainchar} and \ref{charleschar}, $\calX$ is an $\infty$-topos if and only if the class $S$ consisting of all morphisms of $\calX$ is local. This clearly implies $(2)$, in view of Proposition \ref{cardyp}. Conversely,
suppose that $(2)$ is satisfied, and let $S_{\kappa}$ be defined as in the statement of Proposition \ref{cardyp}. Proposition \ref{classexist} ensures that $S_{\kappa}$ is local for all sufficiently large regular cardinals $\kappa$. We note that $S = \bigcup S_{\kappa}$. It follows from Criterion $(3)$ of Lemma \ref{ib3} that $S$ is also local, so that $\calX$ is an $\infty$-topos.
\end{proof}