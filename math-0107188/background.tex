
\section*{Background}		\label{p:background}


\concept{Category Theory}

Here is a summary of the categorical background and terminology needed in order
to read the entire paper.  The reader who isn't familiar with everything below
shouldn't be put off: each individual Definition only uses some of it.

I assume familiarity with \demph{categories}, \demph{functors},
\demph{natural transformations}, \demph{adjunctions}, \demph{limits}, and
\demph{monads} and their \demph{algebras}.  Limits include \demph{products},
\demph{pullbacks} (with the pullback of a diagram $X \go Z \og Y$ sometimes
written $X \times_Z Y$), and \demph{terminal objects} (written $1$,
especially for the terminal set $\{ * \}$); we also use
\demph{initial objects}.  A monad $(T,\eta,\mu)$ is often abbreviated to $T$.

I make no mention of the difference between sets and classes (`small
and large collections').  All the Definitions are really of \emph{small}
weak $n$-category.

Let \cat{C} be a category.  $X\in \cat{C}$ means that $X$ is an object of
$\cat{C}$, and $\cat{C}(X,Y)$ is the set of morphisms (or \demph{maps}, or
\demph{arrows}) from $X$ to $Y$ in \cat{C}.  If $f\in \cat{C}(X,Y)$ then $X$
is the \demph{domain} or \demph{source} of $f$, and $Y$ the \demph{codomain}
or \demph{target}.

\Set\ is the category (sets $+$ functions), and \Cat\ is (categories $+$
functors).  A set is just a \demph{discrete category} (one in which the only
maps are the identities).

$\cat{C}^\op$ is the \demph{opposite} or \demph{dual} of a category
\cat{C}.  $\ftrcat{\cat{C}}{\cat{D}}$ is the category of functors from
$\cat{C}$ to $\cat{D}$ and natural transformations between them.  Any object
$X$ of $\cat{C}$ induces a functor $\cat{C}(X, \dashbk): \cat{C} \go \Set$,
and a natural transformation from $\cat{C}(X, \dashbk)$ to
$F: \cat{C} \go \Set$ is the same thing as an element of $FX$ (the
\demph{Yoneda Lemma}); dually for $\cat{C}(\dashbk,X): \cat{C}^\op \go \Set$.

A functor $F: \cat{C} \go \cat{D}$ is an \demph{equivalence} if these
equivalent conditions hold: (i) $F$ is full, faithful and essentially
surjective on objects; (ii) there exist a functor $G: \cat{D} \go \cat{C}$ (a
\demph{pseudo-inverse} to $F$) and natural isomorphisms $\eta: 1 \go GF$,
$\epsln: FG \go 1$ ; (iii) as~(ii), but with $(F,G,\eta,\epsln)$ also being
an adjunction.

Any set $\cat{D}_0$ of objects of a category \cat{C} determines a \demph{full
subcategory} \cat{D} of \cat{C}, with object-set $\cat{D}_0$ and
$\cat{D}(X,Y) = \cat{C}(X,Y)$.  Every category \cat{C} has a
\demph{skeleton}: a subcategory whose inclusion into \cat{C} is an
equivalence and in which no two distinct objects are isomorphic.  If $F, G:
\cat{C} \go \Set$, $GX \sub FX$ for each $X \in \cat{C}$, and $F$ and $G$
agree on morphisms of \cat{C}, then $G$ is a
\demph{subfunctor} of $F$.

A \demph{total order} on a set $I$ is a reflexive transitive
relation $\leq$ such that if $i \neq j$ then exactly one of $i\leq j$ and
$j\leq i$ holds.  
$(I,\leq)$ can be seen as a category with object-set $I$
in which each hom-set has at most one element.
An \demph{order-preserving map} $(I,\leq) \go (I',\leq')$
is a function $f$ such that $i \leq j \implies f(i) \leq' f(j)$.

Let \Del\ be the category with objects $[k]=\{0,\ldots,k\}$ for $k\geq 0$,
and order-preserving functions as maps.  A \demph{simplicial set} is a
functor $\Delop \go \Set$.  Every category \cat{C} has a \demph{nerve} (the
simplicial set $N\cat{C}: [k] \goesto \Cat([k],\cat{C})$), giving a full and
faithful functor $N: \Cat \go \ftrcat{\Delop}{\Set}$.  So \Cat\ is equivalent
to the full subcategory of \ftrcat{\Delop}{\Set} with objects $\{ X \such X
\iso N\cat{C} \textrm{ for some } \cat{C} \}$; there are various
characterizations of such $X$, but we come to that in the main text.

Leftovers: a \demph{monoid} is a set (or more generally, an object of a
monoidal category) with an associative binary operation and a two-sided unit.
\Cat\ is monadic over the category of directed graphs.  The \demph{natural
numbers} start at $0$.


\clearpage



\concept{Strict $n$-Categories}


If \cat{V} is a category with finite products then there is a category
$\cat{V}\hyph\Cat$ of \cat{V}-enriched categories and \cat{V}-enriched
functors, and this itself has finite products.  (A \demph{\cat{V}-enriched
category} is just like an ordinary category, except that the `hom-sets' are
now objects of \cat{V}.)  Let $0\hyph\Cat = \Set$ and, for $n\geq 0$,
$(n+1)\hyph\Cat = (n\hyph\Cat)\hyph\Cat$; a \demph{strict $n$-category} is an
object of $n\hyph\Cat$.  Note that $1\hyph\Cat = \Cat$.

Any finite-product-preserving functor $U: \cat{V} \go \cat{W}$
induces a finite-product-preserving functor $U_*: \cat{V}\hyph\Cat \go
\cat{W}\hyph\Cat$, so we can define functors $U_n: (n+1)\hyph\Cat \go
n\hyph\Cat$ by taking $U_0$ to be the objects functor and $U_{n+1} =
(U_n)_*$.  The category $\omega\hyph\Cat$ of \demph{strict
$\omega$-categories} is the limit of the diagram
\[
\cdots 
\goby{U_{n+1}} (n+1)\hyph\Cat  \goby{U_n} 
\cdots
\goby{U_1} 1\hyph\Cat = \Cat
\goby{U_0} 0\hyph\Cat = \Set.
\]

Alternatively: a \demph{globular set} (or \demph{$\omega$-graph}) $A$
consists of sets and functions
\[
\cdots 
\parpair{s}{t} A_m  \parpair{s}{t} A_{m-1} \parpair{s}{t} 
\cdots 
\parpair{s}{t} A_0
\]
such that for $m\geq 2$ and $\alpha\in A_m$, $ss(\alpha) = st(\alpha)$ and
$ts(\alpha) = tt(\alpha)$.  An element of $A_m$ is called an
\demph{$m$-cell}, and we draw a $0$-cell $a$ as $\gzeros{a}$, a $1$-cell $f$
as $\gfsts{a}\gones{f}\glsts{b}$ (where $a=s(f), b=t(f)$), a 2-cell $\alpha$
as $\gfsts{a}\gtwos{f}{g}{\alpha}\glsts{b}$, etc.  For $m > p \geq 0$, write
$ A_m \times_{A_p} A_m = \{ (\alpha',\alpha) \in A_m \times A_m \such
s^{m-p}(\alpha') = t^{m-p}(\alpha) \}$.

A \demph{strict $\omega$-category} is a globular set $A$ together with a
function $\ofdim{p}: A_m \times_{A_p} A_m \go A_m$ (\demph{composition}) for
each $m > p \geq 0$ and a function $i: A_m \go A_{m+1}$ (\demph{identities},
usually written $i(\alpha) = 1_\alpha$) for each $m\geq 0$, such that
%
\begin{enumerate}
\item 	\label{part:strict-n:source-comp}
if $m > p \geq 0$ and
$(\alpha',\alpha) \in A_m \times_{A_p} A_m$ then
\[
\begin{array}{llll}
s(\alpha' \ofdim{p} \alpha) = 
s(\alpha) 			&	
\textrm{and}			&
t(\alpha' \ofdim{p} \alpha) = 
t(\alpha') 			&
\textrm{for }
m=p+1	\\
s(\alpha' \ofdim{p} \alpha) = 
s(\alpha') \ofdim{p} s(\alpha)	&
\textrm{and}			&
t(\alpha' \ofdim{p} \alpha) = 
t(\alpha') \ofdim{p} t(\alpha)	&
\textrm{for }
m\geq p+2	
\end{array}
\]
\item  	\label{part:strict-n:source-id}
if $m\geq 0$ and $\alpha\in A_m$ then $s(i(\alpha)) = \alpha =
t(i(\alpha))$ 
\item \label{part:strict-n:ass-and-id} 
if $m > p \geq 0$ and $\alpha \in A_m$ then $i^{m-p}(t^{m-p}(\alpha))
\ofdim{p} \alpha = \alpha = \alpha \ofdim{p}$\linebreak
$i^{m-p}(s^{m-p}(\alpha))$; if also $\alpha', \alpha''$ are such that
$(\alpha'', \alpha'), (\alpha', \alpha) \in A_m \times_{A_p} A_m$, then
$(\alpha'' \ofdim{p} \alpha') \ofdim{p} \alpha = \alpha'' \ofdim{p} (\alpha'
\ofdim{p} \alpha)$
\item  	\label{part:strict-n:int}
if $p>q\geq 0$ and $(f',f) \in A_p \times_{A_q} A_p$ then
$i(f') \ofdim{q} i(f) = i(f' \ofdim{q} f)$; 
if also $m>p$ and $\alpha,\alpha',\beta,\beta'$ are such that
$(\beta',\beta), (\alpha', \alpha) \in A_m \times_{A_p} A_m$, 
$(\beta',\alpha'), (\beta, \alpha) \in A_m \times_{A_q} A_m$,
then 
$(\beta' \ofdim{p} \beta) \ofdim{q} (\alpha' \ofdim{p} \alpha) 
= 
(\beta' \ofdim{q} \alpha') \ofdim{p} (\beta \ofdim{q} \alpha)$.
\end{enumerate}

The composition $\ofdim{p}$ is `composition of $m$-cells by gluing along
$p$-cells'.  Pictures for $(m,p) = (2,1), (1,0), (2,0)$ are in the
Bicategories section below. 

\demph{Strict $n$-categories} are defined similarly, but with the globular
set only going up to $A_n$.  \demph{Strict $n$- and $\omega$-functors} are
maps of globular sets preserving composition and identities; the categories
$n\hyph\Cat$ and $\omega\hyph\Cat$ thus defined are equivalent to the ones
defined above.  The comments below on the two alternative definitions of
bicategory give an impression of how this equivalence works. 

\clearpage




\concept{Bicategories}

Bicategories are the traditional and best-known formulation of `weak
2-category'.  

A \demph{bicategory} $B$ consists of
%
\begin{itemize}
\item a set $B_0$, whose elements $a$ are called \demph{0-cells} or
\demph{objects} of $B$ and drawn
$\gzeros{a}$
\item for each $a,b \in B_0$, a category $B(a,b)$, whose objects $f$ are
called \demph{1-cells} and drawn $\gfsts{a} \gones{f} \glsts{b}$, whose
arrows $\alpha: f \go g$ are called \demph{2-cells} and drawn $\gfsts{a}
\gtwos{f}{g}{\alpha} \glsts{b}$, and whose composition $\gfsts{a}
\gthrees{f}{g}{h}{\alpha}{\beta} \glsts{b} \goesto \gfsts{a}
\gtwos{f}{h}{\!\!\!\!\!\! \beta \sof \alpha} \glsts{b}$ is called
\demph{vertical composition} of 2-cells
\item for each $a \in B_0$, an object $1_a \in B(a,a)$ (the \demph{identity}
on $a$); and for each $a,b,c \in B_0$, a functor $B(b,c) \times B(a,b) \go
B(a,c)$, which on objects is called \emph{1-cell composition},
$\gfsts{a}\gones{f}\gblws{b}\gones{g}\glsts{c} \goesto 
\gfsts{a}\gones{g\sof f}\glsts{c}$, and on arrows is called \demph{horizontal
composition} of 2-cells, $\gfsts{a} \gtwos{f}{g}{\alpha} \gfbws{a'}
\gtwos{f'}{g'}{\alpha'} \glsts{a''} \goesto \gfsts{a} \gtwos{f' \sof f}{g' \sof
g}{\!\!\!\!\!\! \alpha' * \alpha} \glsts{a''}$ 
\item \demph{coherence 2-cells}: for each $f \in B(a,b), g \in B(b,c), h \in
B(c,d)$, an \demph{associativity isomorphism} $\xi_{h,g,f}: (h\of g)\of f
\go h\of (g\of f)$; and for each $f \in B(a,b)$, \demph{unit isomorphisms}
$\lambda_f: 1_b \of f \go f$ and $\rho_f: f \of 1_a \go f$
\end{itemize}
%
satisfying the following \demph{coherence axioms}:
%
\begin{itemize}
\item $\xi_{h,g,f}$ is natural in $h$, $g$ and $f$, and $\lambda_f$ and
$\rho_f$ are natural in $f$
\item if $f \in B(a,b), g \in B(b,c), h \in B(c,d), k
\in B(d,e)$, then 
$
\xi_{k,h,g\sof f} \,\of\, \xi_{k\sof h, g, f} = 
(1_k * \xi_{h,g,f}) \,\of\, \xi_{k,h\sof g,f} \,\of\, (\xi_{k,h,g} * 1_f)
$
(the \demph{pentagon axiom});
and if $f \in B(a,b), g \in B(b,c)$, then 
$
\rho_g * 1_f =
(1_g * \lambda_f) \,\of\, \xi_{g,1_b,f}
$
(the \demph{triangle axiom}).
\end{itemize}

An alternative definition is that a bicategory consists of sets and functions
$B_2 \parpair{s}{t} B_1 \parpair{s}{t} B_0$ satisfying $ss=st$ and $ts=tt$,
together with functions determining composition, identities and coherence
cells (in the style of the second definition of strict $\omega$-category
above).  The idea is that $B_m$ is the set of $m$-cells and that $s$ and $t$
give the source and target of a cell.  Strict 2-categories can be identified
with bicategories in which the coherence 2-cells are all identities.

A 1-cell $\gfsts{a}\gones{f}\glsts{b}$ in a bicategory $B$ is called an
\demph{equivalence} if there exists a 1-cell $\gfsts{b}\gones{g}\glsts{a}$
such that $g\of f \iso 1_a$ and $f\of g \iso 1_b$.  

A \demph{monoidal category} can be defined as a bicategory with only one
0-cell: for if the 0-cell is called $\star$ then the bicategory just consists
of a category $B(\star,\star)$ equipped with an object $I$, a functor
$\otimes: B(\star,\star)^2 \go B(\star,\star)$, and associativity and unit
isomorphisms satisfying coherence axioms.

We can consider \demph{strict functors} of bicategories, in which composition
etc is preserved strictly; more interesting are \demph{weak functors} $F$, in
which there are isomorphisms $Fg \of Ff \go F(g \of f)$, $1_{Fa} \go
F(1_a)$ satisfying coherence axioms.
