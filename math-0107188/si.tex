
\defnheading{Si}		\label{p:si}


\concept{Simplicial Objects}

\paragraph{The Simplicial Category}

Let \Del\ be a skeleton of the category of nonempty finite totally ordered
sets: that is, \Del\ has objects $[k]=\{0,\ldots,k\}$ for $k\geq 0$, and maps
are order-preserving functions (with respect to the usual ordering $\leq$).

\paragraph{Some Maps in \Del}

Let $\sigma, \tau: [0] \go [1]$ be the maps in \Del\ with respective values
$0$ and $1$.  Given $k\geq 0$, let $\iota_1, \ldots, \iota_k: [1] \go [k]$
denote the `embeddings' of $[1]$ into $[k]$, defined by
$\iota_j(0) = j-1$ and $\iota_j(1) = j$.

\paragraph{The Segal Maps}

Let $k\geq 0$.  Then the following diagram in \Del\ commutes:
%
\begin{diagram}[width=2em,height=2em]
	&	&	&	&	&	&	&[k]	&	&
	&	&	&	&	&	\\
	&	&	&	&	&	&\ruTo(7,2)<{\iota_1}	
						 \ruTo(3,2)<{\iota_2}
							&	&\ldots\luTo(7,2)>{\iota_k}	&
	&	&	&	&	&	\\
[1]	&	&	&	&[1]	&	&	&	&	&
\ldots	&	&	&	&	&[1]	\\
	&\luTo<\tau&	&\ruTo>\sigma&	&\luTo<\tau&	&\ruTo>\sigma&	&
	&	&\luTo<\tau&	&\ruTo>\sigma&	\\
	&	&[0]	&	&	&	&[0]	&	&	&
\cdots	&	&	&[0].	&	&	\\
\end{diagram}
%
Let $X: \Delop \go \cat{E}$ be a functor into a category \cat{E} possessing
finite limits, and write $ X[1] \times_{X[0]} \cdots \times_{X[0]} X[1] $
(with $k$ occurrences of $X[1]$) for the limit of the diagram
%
\begin{diagram}[width=2em,height=2em]
X[1]	&	&	&	&X[1]	&	&	&	&	&
\ldots	&	&	&	&	&X[1]	\\
	&\rdTo<{X\tau}&	&\ldTo>{X\sigma}&	&\rdTo<{X\tau}&	&\ldTo>{X\sigma}&	&
	&	&\rdTo<{X\tau}&	&\ldTo>{X\sigma}&	\\
	&	&X[0]	&	&	&	&X[0]	&	&	&
\cdots	&	&	&X[0]	&	&	\\
\end{diagram}
%
(with, again, $k$ occurrences of $X[1]$) in \cat{E}.  Then by commutativity
of the first diagram, there is an induced map in \cat{E}---a \demph{Segal
map}---
%
\begin{equation}	\label{eq:Segal-Si}
X[k] \go X[1] \times_{X[0]} \cdots \times_{X[0]} X[1].
\end{equation}


\concept{Contractibility}

\paragraph{Sources and Targets}

If $0\leq p\leq r$, write $I_p$ for the object $ (\underbrace{[1], \ldots,
[1]}_p , \underbrace{[0], \ldots, [0]}_{r-p}) $ of $\Deln{r}$.  Let $X:
\Delnop{r} \go \Set$, $0 \leq p \leq r$, and $x, x' \in X(I_p)$.  Then
$x, x'$ are \demph{parallel} if $p=0$ or if $p\geq 1$ and $s(x) =
s(x')$ and $t(x) = t(x')$; here $s$ and $t$ are the maps
% 
\begin{diagram}
X(I_p)	&
\pile{\rTo^{X(\id, \ldots, \id, X\sigma, \id, \ldots, \id)}\\
\rTo_{X(\id, \ldots, \id, X\tau, \id, \ldots, \id)}}	&
X(I_{p-1}).\\
\end{diagram}


\paragraph{Contractible Maps}

Let $r\geq 1$ and let $X,Y: \Delnop{r} \go \Set$.  A natural transformation
$\phi: X \go Y$ is \demph{contractible} if
%
\begin{itemize}
\item 	
the function $\phi_{I_0}: X(I_0) \go Y(I_0)$ is surjective
\item  	
given $p\in \{0, \ldots, r-1\}$, parallel $x,x'\in X(I_p)$, and $h\in
Y(I_{p+1})$ satisfying
\[
s(h)=\phi_{I_p}(x), \diagspace
t(h)= \phi_{I_p}(x'), \diagspace
\]
%
\marginpar{\centering
\fbox{%
\begin{diagram}[width=2em,height=1.3em]
x	&\rGet^g	&x'	\\
	&		&	\\
	&\dGoesto>\phi	&	\\
	&		&	\\
\phi(x)	&\rTo^h		&\phi(x')\\
\end{diagram}}}
%
there exists $g\in X(I_{p+1})$ satisfying 
\[
s(g)=x, \diagspace
t(g)=x',\diagspace
\phi_{I_{p+1}}(g)=h
\]
\item  	
given parallel $x,x'\in X(I_r)$ satisfying $\phi_{I_r}(x) = \phi_{I_r}(x')$,
then $x=x'$. 
\end{itemize}

If $r=0$ then $X$ and $Y$ are just sets and $\phi$ is just a function $X\go
Y$; call $\phi$ \demph{contractible} if it is bijective.



\concept{The Definition}

Let $n\geq 0$.  A \demph{weak $n$-category} is a functor $A: \Delnop{n} \go
\Set$ such that for each $m\in \{0, \ldots, n-1\}$ and $K=([k_1], \ldots,
[k_m]) \in \Deln{m}$,
%
\begin{enumerate}
\item 	\label{part:defn:degen-Si}
the functor
$
A(K, [0], \dashbk): \Delnop{n-m-1} \go \Set
$ 
is constant, and
\item 	\label{part:defn:main-Si} 
for each $[k]\in\Del$, the Segal map
\[
A(K, [k], \dashbk) \go 
A(K, [1],\dashbk) \times_{A(K, [0],\dashbk)} \cdots 
\times_{A(K, [0],\dashbk)} A(K, [1],\dashbk)
\]
is contractible.  (We are taking
$\cat{E}=\ftrcat{\Delnop{n-m-1}}{\Set}$ and $X[j]=A(K,[j],\dashbk)$ in the
definition of Segal map.)
\end{enumerate}
%




\clearpage

\lowdimsheading{Si}


\concept{$n=0$}

Parts~\bref{part:defn:degen-Si} and~\bref{part:defn:main-Si} of the
definition are vacuous, so a weak $0$-category is just a functor
$\Delnop{0}\go\Set$, that is, a set.


\concept{$n=1$}

A weak $1$-category is a functor $A: \Delop \go \Set$ (that is, a simplicial
set) satisfying~\bref{part:defn:degen-Si} and~\bref{part:defn:main-Si}.
Part~\bref{part:defn:degen-Si} is always true, and~\bref{part:defn:main-Si}
says that for each $k\geq 0$ the Segal map~\bref{eq:Segal-Si} (with $X=A$) is
a bijection---in other words, that $A$ is a \demph{nerve}.  The category of
nerves and natural transformations between them is equivalent to \Cat, where
a nerve $A$ corresponds to a certain category with object-set $A[0]$ and
morphism-set $A[1]$.  So a weak $1$-category is essentially just a category.


\concept{$n=2$}

A weak $2$-category is a
functor $A: \Delnop{2} \go \Set$ such that
%
\begin{enumerate}
\item the functor $A([0],\dashbk): \Delop \go \Set$ is constant
\item for each $k\geq 0$, the Segal map 
\[
A([k], \dashbk) \go 
A([1],\dashbk) \times_{A([0],\dashbk)} \cdots 
\times_{A([0],\dashbk)} A([1],\dashbk)
\]
is contractible, and for each $k_1, k\geq 0$, the Segal map
\[
A([k_1],[k]) \go 
A([k_1],[1]) \times_{A([k_1],[0])} \cdots 
\times_{A([k_1],[0])} A([k_1],[1])
\]
is a bijection.
\end{enumerate}
%
The second half of~\bref{part:defn:main-Si} says that $A([k_1],\dashbk)$ is a
nerve for each $k_1$, so we can regard $A$ as a functor $\Delop \go \Cat$.
Note that if $X$ and $Y$ are nerves then a natural transformation $\phi: X
\go Y$ is the same thing as a functor between the corresponding categories,
and that $\phi$ is contractible if and only if this functor is full, faithful
and surjective on objects.  So a weak $2$-category is a functor $A: \Delop
\go \Cat$ such that
%
\begin{enumerate}
\item $A[0]$ is a discrete category (i.e.\ the only morphisms are the
identities) 
\item for each $k\geq 0$, the Segal functor 
\[
A[k] \go A[1] \times_{A[0]} \cdots \times_{A[0]} A[1]
\]
is full, faithful and surjective on objects.  
\end{enumerate}

I will now argue that a weak $2$-category is essentially the same thing as a
bicategory.

First take a weak $2$-category $A: \Delop \go \Cat$, and let us construct a
bicategory $B$.  The object-set of $B$ is $A[0]$.  The two functors
$s,t: A[1] \go A[0]$ express the category $A[1]$ as a disjoint union
$\coprod_{a,b\in A[0]}B(a,b)$ of categories; the $1$-cells from $a$ to $b$
are the objects of $B(a,b)$, and the $2$-cells are the morphisms.

Vertical composition of $2$-cells in $B$ is composition in each
$B(a,b)$.  To define horizontal composition of $1$-~and $2$-cells,
first choose for each $k$ a pseudo-inverse
\[
A[1] \times_{A[0]} \cdots \times_{A[0]} A[1] \goby{\psi_k} A[k]
\]
to the Segal functor $\phi_k$ (an equivalence of categories), and
natural isomorphisms $\eta_k: 1 \go \psi_k\,\of\,\phi_k$, $\epsln_k:
\phi_k\,\of\,\psi_k \go 1$.  Horizontal composition is given by
\[
A[1] \times_{A[0]} A[1] \goby{\psi_2} A[2] \goby{A\delta} A[1],
\]
where $\delta:[1] \go [2]$ is the injection whose image omits $1\in [2]$.
The associativity isomorphisms are built up from $\eta_k$'s and $\epsln_k$'s,
and the pentagon commutes just as long as the equivalence
$(\phi_k,\psi_k,\eta_k,\epsln_k)$ was chosen to be an adjunction too (which
is always possible).  Identities work similarly.

Conversely, take a bicategory $B$ and construct a weak 2-category $A:
\Delnop{2}$\linebreak$\go \Set$ (its `2-nerve') as follows.  An element of
$A([j],[k])$ is a quadruple
%
\renewcommand{\arraystretch}{0.8}	
%
\[
((a_u)_{0\leq u\leq j},
(f_{uv}^z)_{\begin{array}{cc} \scriptstyle
	0\leq u < v\leq j\\ \scriptstyle
	0\leq z \leq k
	\end{array}},
(\alpha_{uv}^z)_{\begin{array}{cc} \scriptstyle
	0\leq u < v \leq j\\ \scriptstyle
	1\leq z \leq k
	\end{array}},
(\iota_{uvw}^z)_{\begin{array}{cc} \scriptstyle
	0\leq u < v < w \leq j\\ \scriptstyle
	0 \leq z \leq k
	\end{array}})
\]
%
\renewcommand{\arraystretch}{1}		
%
where
%
\begin{itemize}
\item $a_u$ is an object of $B$
\item $f_{uv}^z: a_u \go a_v$ is a 1-cell of $B$
\item $\alpha_{uv}^z: f_{uv}^{z-1} \go f_{uv}^z$ is a
2-cell of $B$
\item $\iota_{uvw}^z: f_{vw}^z \of f_{uv}^z 
\goiso f_{uw}^z$ is an invertible 2-cell of $B$ 
\end{itemize}
such that
\begin{itemize}
\item $\iota_{uvw}^z \,\of\, (\alpha_{vw}^z * \alpha_{uv}^z) =
\alpha_{uw}^z \,\of\, \iota_{uvw}^{z-1}$ 
whenever $0\leq u < v < w \leq j$, $1\leq z \leq k$
\item $\iota_{uwx}^z \,\of\, (1_{f_{wx}^z} * \iota_{uvw}^z)
\,\of\, (\textrm{associativity isomorphism})
=
\iota_{uvx}^z \,\of\, (\iota_{vwx}^z * 1_{f_{uv}^z})$
whenever $0\leq u < v < w < x \leq j$, $0\leq z \leq k$.
\end{itemize}
%
This defines the functor $A$ on objects of $\Deln{2}$; it is defined on maps
by a combination of inserting identities and forgetting data.

To get a rough picture of $A$, consider the analogous construction for
strict 2-categories, in which we insist that the isomorphisms
$\iota_{uvw}^z$ are actually equalities.  Then an element of
\marginpar{\centering\fbox{%
$
\begin{array}{c}
\gfstsu\gfoursu\gzersu\gfoursu\glstsu\\
j=2,\, k=3
\end{array}
$
}}
$A([j],[k])$ is a grid of $jk$ $2$-cells, of width $j$ and height $k$.  (When
$j=0$ this is just a single object of $B$, regardless of $k$.)  The
bicategorical version is a suitable weakening of this construction.

Finally, passing from a bicategory to a weak 2-category and back again gives
a bicategory isomorphic (in the category of bicategories and weak functors)
to the original one.  Passing from a weak 2-category to a bicategory and back
again gives a weak 2-category which is `equivalent' to the original one in a
sense which we do not have quite enough vocabulary to make precise here.




\clearpage












