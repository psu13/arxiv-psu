
\section*{Introduction}		\label{p:intro}

\begin{quote}
\textit{%
L\'evy \ldots\ once remarked to me that reading other mathematicians' research
gave him actual physical pain.}
\end{quote}
---J. L. Doob on the probabilist Paul L\'evy, \emph{Statistical Science}
   \textbf{1}, no.~1, 1986. 

\begin{quote}
\textit{%
Hell is other people.}
\end{quote}
---Jean-Paul Sartre, \emph{Huis Clos}.

\paragraph*{}

The last five years have seen a vast increase in the literature on
higher-dimensional categories.  Yet one question of central concern remains
resolutely unanswered: what exactly is a weak $n$-category?  There have,
notoriously, been many proposed definitions, but there seems to be a general
perception that most of these definitions are obscure, difficult and long.  
I hope that the present work will persuade the reader that this is not the
case, or at least does not \emph{need} to be: that while no existing
approach is without its mysteries, it is quite possible to state the
definitions in a concise and straightforward way.


\subsection*{What's in here, and what's not}

The sole purpose of this paper is to state several possible definitions of
weak $n$-category.  In particular, I have made no attempt to compare the
proposed definitions with one another (although certainly I hope that this
work will help with the task of comparison).  So the definitions of weak
$n$-category that follow may or may not be `equivalent'; I make no comment.
Moreover, I have not included any notions of weak functor or equivalence
between weak $n$-categories, which would almost certainly be required before
one could make any statement such as `Professor Yin's definition of weak
$n$-category is equivalent to Professor Yang's'.

I have also omitted any kind of motivational or introductory material.  The
`Further Reading' section lists various texts which attempt to explain the
relevance of $n$-categories and other higher categorical structures to
mathematics at large (and to physics and computer science).  I will just
mention two points here for those new to the area.  Firstly, it is easy to
define \emph{strict} $n$-categories (see `Preliminaries'), and it is true
that every weak $2$-category is equivalent to a strict $2$-category, but the
analogous statement fails for $n$-categories when $n>2$: so the difference
between the theories of weak and strict $n$-categories is nontrivial.
Secondly, the issue of comparing definitions of weak $n$-category is a
slippery one, as it is hard to say what it even \emph{means} for two such
definitions to be equivalent.  For instance, suppose you and I each have in
mind a definition of algebraic variety and of morphism of varieties; then we
might reasonably say that our definitions of variety are `equivalent' if your
category of varieties is equivalent to mine.  This makes sense because the
structure formed by varieties and their morphisms is a category.  It is
widely held that the structure formed by weak $n$-categories and the
functors, transformations, \ldots\ between them should be a weak
$(n+1)$-category; and if this is the case then the question is whether your
weak $(n+1)$-category of weak $n$-categories is equivalent to mine---but
whose definition of weak $(n+1)$-category are we using here\ldots?

This paper gives primary importance to $n$-categories, with other higher
categorical structures only mentioned where they have to be.  In writing it
this way I do not mean to imply that $n$-categories are the only interesting
structures in higher-dimensional category theory: on the contrary, I see the
subject as including a whole range of interesting structures, such as operads
and multicategories in their various forms, double and $n$-tuple categories,
computads and string diagrams, homotopy-algebras, $n$-vector spaces, and
structures appropriate for the study of braids, knots, graphs, cobordisms,
proof nets, flowcharts, circuit diagrams, \ldots.  Moreover, consideration of
$n$-categories seems inevitably to lead into consideration of some of these
other structures, as is borne out by the definitions below.  However,
$n$-categories are here allowed to upstage the other structures in what is
probably an unfair way.

Finally, I do not claim to have included \emph{all} the definitions of weak
$n$-category that have been proposed by people; in fact, I am aware that I
have omitted a few.  They are omitted purely because I am not familiar with
them.  More information can be found under `Further Reading'.


\subsection*{Layout}

The first section is `Background'.  This is mainly for reference, and it is
not really recommended that anyone starts reading here.  It begins with a
page on ordinary category theory, recalling those concepts that will be used
in the main text and fixing some terminology.  Everything here is completely
standard, and almost all of it can be found in any introductory book or
course on the subject; but only a small portion of it is used in each
definition of weak $n$-category.  There is then a page each on strict
$n$-categories and bicategories, again recalling widely-known material.

Next come the ten definitions of weak $n$-category.  They are absolutely
independent and self-contained, and can be read in any order.  No
significance should be attached to the order in which they are presented; I
tried to arrange them so that definitions with common themes were grouped
together in the sequence, but that is all.  (Some structures just don't fit
naturally into a single dimension.)

Each definition of weak $n$-category is given in two pages, so that if this
is printed double-sided then the whole definition will be visible on a
double-page spread.  This is followed, again in two pages, by an explanation
of the cases $n=0,1,2$.  We expect weak $0$-categories to be sets, weak
$1$-categories to be categories, and weak $2$-categories to be
bicategories---or at least, to resemble them to some reasonable degree---and
this is indeed the case for all of the definitions as long as we interpret
the word `reasonable' generously.  Each main definition is given in a formal,
minimal style, but the analysis of $n\leq 2$ is less formal and more
explanatory; partly the analysis of $n\leq 2$ is to show that the proposed
definition of $n$-category is a reasonable one, but partly it is for
illustrative purposes.  The reader who gets stuck on a definition might
therefore be helped by looking at $n\leq 2$.

Taking a definition of weak $n$-category and performing a rigorous comparison
between the case $n=2$ and bicategories is typically a long and tedious
process.  For this reason, I have not checked all the details in the $n\leq
2$ sections.  The extent to which I feel confident in my assertions can be
judged from the number of occurrences of phrases such as `probably' and `it
appears that', and by the presence or absence of references under `Further
Reading'.

There are a few exceptions to this overall scheme.  The section labelled
\ds{B} consists, in fact, of \emph{two} definitions of weak $n$-category, but
they are so similar in their presentation that it seemed wasteful to give
them two different sections.  (The reason for the name \ds{B} is explained
below.)  The same goes for definition \ds{L}, so we have definitions of weak
$n$-category called \ds{B1}, \ds{B2}, \ds{L1} and \ds{L2}.  A variant for
definition \ds{St} is also given (in the $n\leq 2$ section), but this goes
nameless.  However, definition \ds{X} is not strictly speaking a mathematical
definition at all: I was unable to find a way to present it in two pages, so
instead I have given an informal version, with one sub-definition (opetopic
set) done by example only.  The cases $n\leq 2$ are clear enough to be
analysed precisely.

Another complicating factor comes from those definitions which include a
notion of weak $\omega$-category ($=$ weak $\infty$-category).  There, the
pattern is very often to define weak $\omega$-category and then to define a
weak $n$-category as a weak $\omega$-category with only trivial cells in
dimensions $>n$.  This presents a problem when one comes to attempt a precise
analysis of $n\leq 2$, as even to determine what a weak $0$-category is
involves considering an infinite-dimensional structure.  For this reason it
is more convenient to redefine weak $n$-category in a way which never
mentions cells of dimension $>n$, by imitating the original definition of
weak $\omega$-category.  Of course, one then has to show that the two
different notions of weak $n$-category are equivalent, and again I have not
always done this with full rigour (and there is certainly not the space to
give proofs here).  So, this paper actually contains significantly more than
ten possible definitions of weak $n$-category.

`Further Reading' is the final section.  To keep the definitions of
$n$-category brief and self-contained, there are no citations at all in the
main text; so this section is a combination of reference list, historical
notes, and general comments, together with a few pointers to literature in
related areas.


\subsection*{Overview of the definitions}

Table~\ref{table:defns} shows some of the main features of the definitions of
weak $n$-category.  
%
\begin{table}
\centering
\begin{tabular}{lllll}
\emph{Definition}	&
\emph{Author(s)}	&
\emph{Shapes used}	&
\emph{A/the}		&
\emph{$\omega$?}	\\
\\
\ds{Tr}			&
Trimble			&
path parametrizations	&
the			&
\crossmark		\\
\ds{P}			&
Penon			&
globular		&
the			&
\checkmark		\\
\ds{B}			&
Batanin			&
globular		&
the (\ds{B1}), a (\ds{B2})&
\checkmark		\\
\ds{L}			&
Leinster		&
globular		&
the (\ds{L1}), a (\ds{L2})&
\checkmark		\\
\ds{\lp}		&
Leinster		&
globular		&
a			&
\checkmark		\\
\ds{Si}			&
Simpson			&
simplicial/globular	&
a			&
\crossmark		\\
\ds{Ta}			&
Tamsamani		&
simplicial/globular	&
a			&
\crossmark		\\
\ds{J}			&
Joyal			&
globular/simplicial	&
a			&
\checkmark		\\
\ds{St}			&
Street			&
simplicial		&
a			&
\checkmark		\\
\ds{X}			&
see text		&
opetopic		&
a			&
\crossmark		\\
\end{tabular}
\caption{Some features of the definitions}
\label{table:defns}
\end{table}
%
Each definition is given a name such as \ds{A} or \ds{Z}, according to the
name of the author from whom the definition is derived.  (Definition \ds{X}
is a combination of the work of many people, principally Baez, Dolan,
Hermida, Makkai and Power.)  The point of these abbreviations is to put some
distance between the definitions as proposed by those authors and the
definitions as stated below.  At the most basic level, I have in all cases
changed some notation and terminology.  Moreover, taking what is often a long
paper and turning it into a two-page definition has seldom been just a matter
of leaving out words; sometimes it has required a serious reshaping of the
concepts involved.  Whether the end result (the definition of weak
$n$-category) is mathematically the same as that of the original author is
not something I always know: on various occasions there have been passages in
the source paper that have been opaque to me, so I have guessed at the
author's intended meaning.  Finally, in several cases only a definition of
weak $\omega$-category was explicitly given, leaving me to supply the
definition of weak $n$-category for finite $n$.  In summary, then, I do
believe that I have given ten reasonable definitions of weak $n$-category,
but I do not guarantee that they are the same as those of the authors listed
in Table~\ref{table:defns}; ultimately, the responsibility for them is mine.

The column headed `shapes used' refers to the different shapes of $m$-cell
(or `$m$-arrow', or `$m$-morphism') employed in the definitions.  These are
shown in Figure~\ref{fig:shapes}.
%
\begin{figure}
\centering
\begin{tabular}{ccccccc}%
\gfstsu\gtwosu\glstsu%	
&&%
\raisebox{-3.4mm}{%
\setlength{\unitlength}{0.75mm}%
\begin{picture}(12,11)(-6,-2)
\cell{-6}{0}{c}{\zmark}
\cell{0}{9}{c}{\zmark}
\cell{6}{0}{c}{\zmark}
\put(-6,0){\vector(2,3){6}}
\put(0,9){\vector(2,-3){6}}
\put(-6,0){\vector(1,0){12}}
\put(0,6){\vector(0,-1){5}}
\end{picture}}%
&&%
\raisebox{-3.4mm}{%
\setlength{\unitlength}{0.75mm}%
\begin{picture}(16,11)(-8,-2)
\cell{-8}{0}{c}{\zmark}
\cell{-6}{6}{c}{\zmark}
\cell{0}{9}{c}{\zmark}
\cell{6}{6}{c}{\zmark}
\cell{8}{0}{c}{\zmark}
\put(-8,0){\vector(1,3){2}}
\put(-6,6){\vector(2,1){6}}
\put(0,9){\vector(2,-1){6}}
\put(6,6){\vector(1,-3){2}}
\put(-8,0){\vector(1,0){16}}
\put(0,6){\vector(0,-1){5}}
\end{picture}}%
&&%
\raisebox{-3.4mm}{%
\setlength{\unitlength}{0.75mm}%
\begin{picture}(24,11)(-12,-2)
\cell{-3}{9}{c}{\zmark}
\cell{3}{9}{c}{\zmark}
\cell{-12}{0}{c}{\zmark}
\cell{-6}{0}{c}{\zmark}
\cell{0}{0}{c}{\zmark}
\cell{6}{0}{c}{\zmark}
\cell{12}{0}{c}{\zmark}
\put(-3,9){\line(1,0){6}}
\put(-12,0){\line(1,0){24}}
\qbezier(-3.5,7.5)(-5,4)(-10.5,1)
\qbezier(3.5,7.5)(5,4)(10.5,1)
\put(0,7.5){\line(0,-1){6}}
\cell{-7}{3.5}{bl}{\scriptstyle\llcorner}
\cell{7.1}{3.5}{br}{\scriptstyle\lrcorner}
\cell{0.1}{3}{b}{\scriptscriptstyle\vee}
\end{picture}}%
\\
globular	&&
simplicial	&&
opetopic	&&
path parametrizations\\
\end{tabular}
\caption{Shapes used in the definitions}
\label{fig:shapes}
\end{figure}

It has widely been observed that the various definitions of $n$-category fall
into two groups, according to the attitude one takes to the status of
composition.  This distinction can be explained by analogy with products.
Given two sets $A$ and $B$, one can define \emph{a product} of $A$ and $B$ to
be a triple $(P,p_1,p_2)$ where $P$ is a set and $p_1: P \go A$, $p_2: P \go
B$ are functions with the usual universal property.  This is of course the
standard thing to do in category theory, and in this context one can strictly
speaking never refer to \emph{the} product of $A$ and $B$.  On the other
hand, one could define \emph{the product} of $A$ and $B$ to be the set
$A\times B$ of ordered pairs $(a,b) = \{ \{a\}, \{a,b\} \}$ with $a\in A$ and
$b\in B$; this has the virtue of being definite and allowing one to speak of
\emph{the} product in the customary way, but involves a wholly artificial
construction.  Similarly, in some of the proposed definitions of weak
$n$-category, one can never speak of \emph{the} composite of morphisms $g$
and $f$, only of \emph{a} composite (of which there may be many, all equally
valid); but in some of the definitions one does have definite composites
$g\of f$, \emph{the} composite of $g$ and $f$.  (The use of the word `the' is
not meant to imply strictness, e.g.\ the three-fold composite $h\of (g\of f)$
will in general be different from the three-fold composite $(h\of g) \of f$.)
So this is the meaning of the column headed `a/the'; it might also have been
headed `indefinite/definite', `relational/functional', `universal/coherent',
or even `geometric/algebraic'.

All of the sections include a definition of weak $n$-category for
natural numbers $n$, but some also include a definition of weak
$\omega$-category (in which there are $m$-cells for all natural $m$).  This
is shown in the last column.

Finally, I warn the reader that the words `contractible' and `contraction'
occur in many of the definitions, but mean different things from
definition to definition.  This is simply to save having to invent new words
for concepts which are similar but not identical, and to draw attention to
the common idea.  










\subsection*{Acknowledgements}

I first want to thank Eugenia Cheng and Martin Hyland.  Their involvement in
this project has both made it much more pleasurable for me and provided a
powerful motivating force.  Without them, I suspect it would still not be
done.

I prepared for writing this by giving a series of seminars (one definition
per week) in Cambridge in spring 2001, and am grateful to the participants:
the two just mentioned, Mario C\'accamo, Marcelo Fiore, and Joe Templeton.  I
would also like to thank those who have contributed over the years to the
many Cambridge Category Theory seminars on the subject of $n$-categories,
especially Jeff Egger (who introduced me to Tamsamani's definition), Peter
Johnstone, and Craig Snydal (with whom I have also had countless interesting
conversations on the subject).

Todd Trimble was generous enough to let me publish his definition for the
first time, and to cast his eye over a draft of what appears below as
definition \ds{Tr}---though all errors, naturally, are mine.

I am also grateful to the other people with whom I have had helpful
communications, including Michael Batanin (who told me about Penon's
definition), David Carlton, Jack Duskin, Anders Kock, Peter May, Carlos
Simpson, Ross Street, Bertrand Toen, Dominic Verity, Marek Zawadowski (who
told me about Joyal's definition), and surely others, whose names I apologize
for omitting.

Many of the diagrams were drawn using Paul Taylor's commutative diagrams
package.

It is a pleasure to thank St John's College, Cambridge, where I hold the
Laurence Goddard Fellowship, for their support. 

\clearpage
