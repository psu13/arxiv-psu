
\defnheading{Ta}		\label{p:ta}


\concept{Simplicial Objects}

\paragraph{The Simplicial Category}

Let \Del\ be a skeleton of the category of nonempty finite totally ordered
sets: that is, \Del\ has objects $[k]=\{0,\ldots,k\}$ for $k\geq 0$, and maps
are order-preserving functions (with respect to the usual ordering $\leq$).

\paragraph{Some Objects and Morphisms}

Given $k\geq 0$, let $\iota_1, \ldots, \iota_k: [1] \go [k]$ denote the
`embeddings' of $[1]$ into $[k]$, defined by $\iota_j(0) = j-1$ and
$\iota_j(1) = j$.  Let $\sigma, \tau: [0] \go [1]$ be the maps in \Del\ with
respective values $0$ and $1$.  Given $p\geq 0$, write $0^p = ([0], \ldots,
[0]) \in \Deln{p}$ and $1^p = ([1], \ldots, [1]) \in \Deln{p}$.
%
Let $X: \Delnop{r} \go \Set$, $0 \leq p \leq r$, and $x, x' \in X(1^p,
0^{r-p})$.  Then $x, x'$ are \demph{parallel} if $p=0$ or if $p\geq 1$ and
$s(x) = s(x')$ and $t(x) = t(x')$; here $s$ and $t$ are the maps
%
\begin{diagram}
X(1^p,0^{r-p})	&
\pile{\rTo^{X(\id, \ldots, \id, X\sigma, \id, \ldots, \id)}\\
\rTo_{X(\id, \ldots, \id, X\tau, \id, \ldots, \id)}}	&
X(1^{p-1},0^{r-p+1}).\\
\end{diagram}

\paragraph{The Segal Maps}

Let $k\geq 0$.  Then the following diagram in \Del\ commutes:
%
\begin{diagram}[width=2em,height=2em]
	&	&	&	&	&	&	&[k]	&	&
	&	&	&	&	&	\\
	&	&	&	&	&	&\ruTo(7,2)<{\iota_1}	
						 \ruTo(3,2)<{\iota_2}
							&	&\ldots\luTo(7,2)>{\iota_k}	&
	&	&	&	&	&	\\
[1]	&	&	&	&[1]	&	&	&	&	&
\ldots	&	&	&	&	&[1]	\\
	&\luTo<\tau&	&\ruTo>\sigma&	&\luTo<\tau&	&\ruTo>\sigma&	&
	&	&\luTo<\tau&	&\ruTo>\sigma&	\\
	&	&[0]	&	&	&	&[0]	&	&	&
\cdots	&	&	&[0].	&	&	\\
\end{diagram}
%
Let $X: \Delop \go \cat{E}$ be a functor into a category
\cat{E} possessing finite limits, and write
$
X[1] \times_{X[0]} \cdots \times_{X[0]} X[1]
$
(with $k$ occurrences of $X[1]$) for the limit of the diagram  
%
\begin{diagram}[width=2em,height=2em]
X[1]	&	&	&	&X[1]	&	&	&	&	&
\ldots	&	&	&	&	&X[1]	\\
	&\rdTo<{X\tau}&	&\ldTo>{X\sigma}&	&\rdTo<{X\tau}&	&\ldTo>{X\sigma}&	&
	&	&\rdTo<{X\tau}&	&\ldTo>{X\sigma}&	\\
	&	&X[0]	&	&	&	&X[0]	&	&	&
\cdots	&	&	&X[0]	&	&	\\
\end{diagram}
%
(with, again, $k$ occurrences of $X[1]$) in \cat{E}.  Then by commutativity
of the first diagram, there is an induced map in \cat{E}---a \demph{Segal
map}---
%
\begin{equation}	\label{eq:Segal}
X[k] \go X[1] \times_{X[0]} \cdots \times_{X[0]} X[1].
\end{equation}

\paragraph{Nerves}

Call $X: \Delop \go \Set$ a \demph{nerve} if for each $k\geq
0$, the Segal map~\bref{eq:Segal} is a bijection.  The category of nerves and
natural transformations is equivalent to \Cat, where a nerve $X$ corresponds
to a category with object-set $X[0]$ and morphism-set $X[1]$.   
Let $QX$ be the set of isomorphism classes of objects of the category
corresponding to $X$, and let $\pi_X: X[0] \go QX$ be the quotient map.



\concept{Truncatability}

For each $r\geq 0$, we define what it means for $X: \Delnop{r} \go
\Set$ to be \demph{truncatable}, writing $\Trunc{r}$ for the
category of truncatable functors $\Delnop{r} \go \Set$ and natural
transformations between them.  We also define functors $\mr{ob}^{(r)}$,
$Q^{(r)}: \Trunc{r} \go \Set$ and a natural transformation
$\pi^{(r)}: \mr{ob}^{(r)} \go Q^{(r)}$.

The functor $\mr{ob}^{(r)}$ is given by $\mr{ob}^{(r)}X = X(0^r)$.  All
functors $\Delnop{0} \go \Set$ are truncatable, and $Q^{(0)}$ and
$\pi^{(0)}$ are identities.  Inductively, when $r\geq 1$, a functor $X:
\Delnop{r} \go \Set$ is truncatable if 
%
\begin{itemize}
\item for each $k\geq 0$, the functor $X([k],\dashbk): \Delnop{r-1} \go
\Set$ is truncatable 
\item the functor $\widehat{X}: \Delop \go \Set$ defined by $[k] \goesto
Q^{(r-1)}(X([k],\dashbk))$ is a nerve.
\end{itemize}
%
If $X$ is truncatable then we define $Q^{(r)}(X)=Q(\widehat{X})$ and
$\pi^{(r)}_X = \pi_{\widehat{X}} \,\of\, \pi^{(r-1)}_{X([0],\dashbk)}$.



\concept{Equivalence}

\paragraph{Internal Equivalence}

Let $0\leq p\leq r$, let $X: \Delnop{r} \go \Set$ be truncatable, and let
$x_1,x_2 \in X(1^p, 0^{r-p})$.  We call $x_1$ and $x_2$ \demph{equivalent},
and write $x_1\inteqv x_2$, if $x_1$ and $x_2$ are parallel and
$\pi^{(r-p)}_{X(1^p,\dashbk)}(x_1) = \pi^{(r-p)}_{X(1^p,\dashbk)}(x_2)$.

\paragraph{External Equivalence}

Let $r\geq 0$.  A natural transformation $\phi: X \go Y$ of truncatable
functors $X, Y: \Delnop{r} \go \Set$ is called an \demph{equivalence} if
%
\begin{itemize}
\item 
for each $y\in Y(0^r)$ there exists $x\in X(0^r)$ with $\phi_{0^r}(x)
\inteqv y$, and this $x$ is unique up to equivalence
\item
for all $0 \leq p\leq r-1$, parallel $x, x' \in X(1^p,0^{r-p})$, and $h\in
Y(1^{p+1},0^{r-p-1})$ satisfying
\[
s(h)=\phi_{(1^p,0^{r-p})}(x),	\diagspace 
t(h)=\phi_{(1^p,0^{r-p})}(x'),
\]
there is an element $g\in X(1^{p+1},0^{r-p-1})$, unique up to equivalence,
satisfying 
%
\marginpar{\centering
\fbox{%
\begin{diagram}[width=2em,height=1.3em]
x	&\rGet^g	&x'	\\
	&		&	\\
	&\dGoesto>\phi	&	\\
	&		&	\\
\phi(x)	&\rTo^h		&\phi(x')\\
\end{diagram}}}
%
\[
s(g)=x, 		\diagspace
t(g)=x', 		\diagspace
\phi_{(1^{p+1},0^{r-p-1})}(g) \inteqv h.
\]
\end{itemize}



\concept{The Definition}

Let $n\geq 0$.  A \demph{weak $n$-category} is a truncatable functor $A:
\Delnop{n} \go \Set$ such that for each $m\in \{0, \ldots, n-1\}$ and
$K=([k_1], \ldots, [k_m]) \in \Deln{m}$,
%
\begin{enumerate}
\item 	\label{part:defn:degen}
the functor
$
A(K, [0], \dashbk): \Delnop{n-m-1} \go \Set
$ 
is constant, and
\item 	\label{part:defn:main} 
for each $[k]\in\Del$, the Segal map
%
\begin{equation}	\label{eq:defining}
A(K, [k], \dashbk) \go 
A(K, [1],\dashbk) \times_{A(K, [0],\dashbk)} \cdots 
\times_{A(K, [0],\dashbk)} A(K, [1],\dashbk)
\end{equation}
%
is an equivalence.  (We are taking $\cat{E}=\ftrcat{\Delnop{n-m-1}}{\Set}$
and $X[j]=A(K,[j],\dashbk)$ in the definition of Segal map, and we can check
that both the domain and the codomain of~\bref{eq:defining} are truncatable.)
\end{enumerate}





\clearpage


\lowdimsheading{Ta}


\concept{$n=0$}

Parts~\bref{part:defn:degen} and~\bref{part:defn:main} of the definition are
vacuous, and truncatability is automatic, so a weak $0$-category is just a
functor $\Delnop{0}\go\Set$, that is, a set.



\concept{$n=1$}

Note that a functor $A: \Delop \go \Set$ is truncatable exactly when it is a
nerve; that a functor $X: \Delnop{0} \go \Set$ is merely a set, and two
elements of $X$ are equivalent just when they are equal; and that a map
$\phi: X \go Y$ of functors $X, Y: \Delnop{0} \go \Set$ is an equivalence
just when it is a bijection.  A weak $1$-category is a truncatable functor
$A: \Delop \go \Set$ satisfying~\bref{part:defn:degen}
and~\bref{part:defn:main}.  Part~\bref{part:defn:degen} is trivially true,
and both truncatability and~\bref{part:defn:main} say that $A$ is a nerve.
So a weak $1$-category is just a nerve---that is, essentially just a
category.


\concept{$n=2$}

First note that if $X: \Delop \go \Set$ is a nerve then two elements of
$X[0]$ are equivalent just when they are isomorphic (as objects of
the category corresponding to $X$), and two elements of $X[1]$ are equivalent
just when they are equal.  Note also that a map $\phi: X \go Y$ of nerves is
an equivalence if and only if (regarded as a functor between the
corresponding categories) it is full, faithful and essentially surjective on
objects---that is, an equivalence of categories.

A weak $2$-category is a truncatable functor $A: \Delnop{2} \go \Set$ such
that
%
\begin{enumerate}
\item the functor $A([0],\dashbk): \Delop \go \Set$ is constant
\item for each $k\geq 0$, the Segal map 
\[
A([k], \dashbk) \go 
A([1],\dashbk) \times_{A([0],\dashbk)} \cdots 
\times_{A([0],\dashbk)} A([1],\dashbk)
\]
is an equivalence, and for each $k_1, k\geq 0$, the Segal map
\[
A([k_1],[k]) \go 
A([k_1],[1]) \times_{A([k_1],[0])} \cdots 
\times_{A([k_1],[0])} A([k_1],[1])
\]
is a bijection.
\end{enumerate}
% 
The second half of~\bref{part:defn:main} says that $A([k_1],\dashbk)$ is a
nerve for each $k_1$, so we can regard $A$ as a functor $A: \Delop \go \Cat$;
then the first half of~\bref{part:defn:main} says that the Segal
map~\bref{eq:Segal} (with $X=A$)
is an equivalence of categories.  Truncatability of $A$ says that the functor
$\Delop \go \Set$ given by
$
[k] \goesto \{
$%
isomorphism classes of objects of
$
A[k] \}
$
is a nerve, which follows anyway from the other conditions.  So a weak
$2$-category is a functor $A: \Delop \go \Cat$ such that
%
\begin{enumerate}
\item $A[0]$ is a discrete category (i.e.\ the only morphisms are the
identities) 
\item for each $k\geq 0$, the Segal functor 
$
A[k] \go A[1] \times_{A[0]} \cdots \times_{A[0]} A[1]
$
is an equivalence of categories. 
\end{enumerate}

It seems that a weak $2$-category is essentially just a bicategory.

First take a weak $2$-category $A: \Delop \go \Cat$, and let us construct a
bicategory $B$.  The object-set of $B$ is $A[0]$.  The two functors
$s,t: A[1] \go A[0]$ express the category $A[1]$ as a disjoint union
$\coprod_{a,b\in A[0]}B(a,b)$ of categories; the $1$-cells from $a$ to $b$
are the objects of $B(a,b)$, and the $2$-cells are the morphisms.

Vertical composition of $2$-cells in $B$ is composition in each
$B(a,b)$.  To define horizontal composition of $1$-~and $2$-cells,
first choose for each $k$ a pseudo-inverse
\[
A[1] \times_{A[0]} \cdots \times_{A[0]} A[1] \goby{\psi_k} A[k]
\]
to the Segal functor $\phi_k$, and choose natural isomorphisms $\eta_k: 1 \go
\psi_k\,\of\,\phi_k$, $\epsln_k: \phi_k\,\of\,\psi_k \go 1$.  Horizontal
composition is then given as
\[
A[1] \times_{A[0]} A[1] \goby{\psi_2} A[2] \goby{A\delta} A[1],
\]
where $\delta:[1] \go [2]$ is the injection whose image omits $1\in [2]$.
The associativity isomorphisms are built up from $\eta_k$'s and $\epsln_k$'s,
and the pentagon commutes just as long as the equivalence
$(\phi_k,\psi_k,\eta_k,\epsln_k)$ was chosen to be an adjunction too (which
is always possible).  Identities work similarly.

Conversely, take a bicategory $B$ and construct a weak 2-category $A:
\Delnop{2}$\linebreak$\go \Set$ (its `2-nerve') as follows.  An element of
$A([j],[k])$ is a quadruple
%
\renewcommand{\arraystretch}{0.8}	% See Lamport p 207
%
\[
((a_u)_{0\leq u\leq j},
(f_{uv}^z)_{\begin{array}{cc} \scriptstyle
	0\leq u < v\leq j\\ \scriptstyle
	0\leq z \leq k
	\end{array}},
(\alpha_{uv}^z)_{\begin{array}{cc} \scriptstyle
	0\leq u < v \leq j\\ \scriptstyle
	1\leq z \leq k
	\end{array}},
(\iota_{uvw}^z)_{\begin{array}{cc} \scriptstyle
	0\leq u < v < w \leq j\\ \scriptstyle
	0 \leq z \leq k
	\end{array}})
\]
%
\renewcommand{\arraystretch}{1}		
%
where
%
\begin{itemize}
\item $a_u$ is an object of $B$
\item $f_{uv}^z: a_u \go a_v$ is a 1-cell of $B$
\item $\alpha_{uv}^z: f_{uv}^{z-1} \go f_{uv}^z$ is a
2-cell of $B$
\item $\iota_{uvw}^z: f_{vw}^z \of f_{uv}^z 
\goiso f_{uw}^z$ is an invertible 2-cell of $B$ 
\end{itemize}
such that
\begin{itemize}
\item $\iota_{uvw}^z \,\of\, (\alpha_{vw}^z * \alpha_{uv}^z) =
\alpha_{uw}^z \,\of\, \iota_{uvw}^{z-1}$ 
whenever $0\leq u < v < w \leq j$, $1\leq z \leq k$
\item $\iota_{uwx}^z \,\of\, (1_{f_{wx}^z} * \iota_{uvw}^z)
\,\of\, (\textrm{associativity isomorphism})
=
\iota_{uvx}^z \,\of\, (\iota_{vwx}^z * 1_{f_{uv}^z})$
whenever $0\leq u < v < w < x \leq j$, $0\leq z \leq k$.
\end{itemize}
%
This defines the functor $A$ on objects of $\Deln{2}$; it is defined on maps
by a combination of inserting identities and forgetting data.

To get a rough picture of $A$, consider the analogous construction for
strict 2-categories, in which we insist that the isomorphisms
$\iota_{uvw}^z$ are actually equalities.  Then an element of
\marginpar{\centering\fbox{%
$
\begin{array}{c}
\gfstsu\gfoursu\gzersu\gfoursu\glstsu\\
j=2,\, k=3
\end{array}
$
}}
$A([j],[k])$ is a grid of $jk$ $2$-cells, of width $j$ and height $k$.  (When
$j=0$ this is just a single object of $B$, regardless of $k$.)  The
bicategorical version is a suitable weakening of this construction.

Finally, it appears that passing from a bicategory to a weak 2-category and
back again gives a bicategory isomorphic (by weak functors) to the original
one, and that passing from a weak 2-category to a bicategory and back again
gives a weak 2-category equivalent to the original one.


\clearpage









