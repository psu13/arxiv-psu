
\defnheading{P}		\label{p:p}


\concept{Some Globular Structures}

\paragraph{Reflexive Globular Sets}

Let \scat{R} be the category whose objects are the natural numbers
$0,1,\ldots$, and whose arrows are generated by
\vspace{-4ex}
\[
\cdots \diagspace
m+1 \,
\splitcoeqlhs{\sigma_{m+1}}{\tau_{m+1}}{\iota_{m+1}} 
\, m \, 
\splitcoeqlhs{\sigma_m}{\tau_m}{\iota_m}
\diagspace \cdots \diagspace
\splitcoeqlhs{\sigma_1}{\tau_1}{\iota_1} 
0
\]
subject to the equations
\[
\sigma_m \of \sigma_{m+1} = \sigma_m \of \tau_{m+1},
\diagspace
\tau_m \of \sigma_{m+1} = \tau_m \of \tau_{m+1},
\diagspace
\sigma_m \of \iota_m = 1 = \tau_m \of \iota_m
\]
($m\geq 1$).  A functor $A: \scat{R} \go \Set$ is called a \demph{reflexive
globular set}.  I will write $s$ for $A(\sigma_m)$, and $t$ for $A(\tau_m)$,
and $1_a$ for $(A(\iota_m))(a)$ when $a \in A(m-1)$. 

\paragraph{Strict $\omega$-Categories, and $\omega$-Magmas}

A \demph{strict $\omega$-category} is a reflexive globular set $S$ together
with a function (\demph{composition}) $\ofdim{p}: S(m) \times_{S(p)} S(m) \go
S(m)$ for 
each $m > p \geq 0$, satisfying
%
\begin{itemize}
\item axioms determining the source and target of a composite
(part~\bref{part:strict-n:source-comp} in the Preliminary section `Strict
$n$-Categories')
\item 	
strict associativity, unit and interchange axioms (parts
~\bref{part:strict-n:ass-and-id} and~\bref{part:strict-n:int}). 
\end{itemize}

An \demph{$\omega$-magma} is like a strict $\omega$-category, but only
satisfying the first group of axioms~(\bref{part:strict-n:source-comp}) and
not necessarily the second
(\bref{part:strict-n:ass-and-id},~\bref{part:strict-n:int}).
A \demph{map of $\omega$-magmas} is a map of reflexive globular sets which
commutes with all the composition operations.  (A strict $\omega$-functor
between strict $\omega$-categories is, therefore, just a map of the
underlying $\omega$-magmas.)



\concept{Contractions}

Let $\phi: A \go B$ be a map of reflexive globular sets.  For $m\geq 1$,
define
\[
V_\phi(m) =
\{ (f_0,f_1) \in A(m) \times A(m) \such
s(f_0) = s(f_1), t(f_0) = t(f_1), \phi(f_0) = \phi(f_1) \},
\]
and define
\[
V_\phi(0) = 
\{ (f_0, f_1) \in A(0) \times A(0) \such \phi(f_0) = \phi(f_1) \}.
\]
A \demph{contraction} $\gamma$ on $\phi$ is a family of functions
\[
(\gamma_m : V_\phi(m) \go A(m+1))_{m\geq 0}
\]
such that for all $m\geq 0$ and $(f_0,f_1) \in V_\phi (m)$,
\[
s(\gamma_m(f_0,f_1)) = f_0, 
\ \ \ 
t(\gamma_m(f_0,f_1)) = f_1,
\ \ \ 
\phi(\gamma_m(f_0,f_1)) = 1_{\phi(f_0)} (= 1_{\phi(f_1)}),
\]
and for all $m\geq 0$ and $f \in A(m)$,
\[
\gamma_m(f,f) = 1_f.
\]


\concept{The Mysterious Category \cat{Q}}

\paragraph{Objects}

An object of \cat{Q} (see Fig.~\ref{fig:object}) is a quadruple
$(M,S,\pi,\gamma)$ in which
%
\begin{figure}
\[
%
\begin{diagram}[width=2.5em,height=2.7em,tight]
M		&\lLabelling	&\parbox{5em}{\raggedright
				$\omega$-magma}			\\
\dTo>{\pi}	&\lLabelling	&\parbox{5em}{\raggedright 
				map of $\omega$-magmas}		\\
S		&\lLabelling	&\parbox{5em}{\raggedright
				strict $\omega$-\\category}	\\
\end{diagram}
%
\mbox{\hspace{5em}}
%
\begin{array}{ccc}
\begin{diagram}[width=2.5em]
a&\pile{\rTo^{f_0}\\ \rTo_{f_1}}&b
\end{diagram} 				
&
\mbox{\hspace{1.6em}} \raisebox{-3ex}{\goesto} \mbox{\hspace{1.3em}}
&
a \ctwocentre{f_0}{f_1}{\,\scriptstyle \gamma_1(f_0,f_1)} b	\\
\pi(f_0) = \pi(f_1)			
&
&
\pi(\gamma_1(f_0,f_1)) = 1_{\pi(f_0)} 		
\end{array}
\]
\caption{An object of $\cat{Q}$, with $\gamma$ shown for $m=1$} 
\label{fig:object}
\end{figure}
%
\begin{itemize}
\item $M$ is an $\omega$-magma
\item $S$ is a strict $\omega$-category
\item $\pi$ is a map of $\omega$-magmas from $M$ to (the underlying
$\omega$-magma of) $S$
\item $\gamma$ is a contraction on $\pi$.
\end{itemize}


\paragraph{Maps}

A map $(M,S,\pi,\gamma) \go (M',S',\pi',\gamma')$ in \cat{Q} is a pair $(M
\goby{\chi} M',$ $S \goby{\zeta} S')$ commuting with everything in sight.
That is, $\chi$ is a map of $\omega$-magmas, $\zeta$ is a strict
$\omega$-functor, $\pi' \of \chi = \zeta \of \pi$, and $\gamma'_m(\chi(f_0),
\chi(f_1)) = \chi (\gamma_m(f_0, f_1))$ for all $(f_0, f_1) \in V_M(m)$.

\paragraph{Composition and Identities}

These are defined in the obvious way.


\concept{The Definition}

\paragraph{An Adjunction}

Let $U: \cat{Q} \go \ftrcat{\scat{R}}{\Set}$ be the functor sending
$(M,S,\pi,\gamma)$ to the underlying reflexive globular set of the
$\omega$-magma $M$.  It can be shown that $U$ has a left adjoint: so
there is an induced monad $T$ on \ftrcat{\scat{R}}{\Set}. 

\paragraph{Weak $\omega$-Categories}

A \demph{weak $\omega$-category} is a $T$-algebra.

\paragraph{Weak $n$-Categories}

Let $n\geq 0$.  A reflexive globular set $A$ is \demph{$n$-dimensional} if
for all $m \geq n$, the map $A(\iota_{m+1}): A(m) \go A(m+1)$ is an
isomorphism (and so $s = t = (A(\iota_{m+1}))^{-1}$).  A
\demph{weak $n$-category} is a weak $\omega$-category whose underlying
reflexive globular set is $n$-dimensional. 




\clearpage


\lowdimsheading{P}



\concept{Direct Interpretation}

\paragraph{The Left Adjoint in Low Dimensions}

Here is a description of what the left adjoint $F$ to $U$ does in dimensions
$\leq 2$.  It is perhaps not obvious that $F$ as described does form the left
adjoint; we come to that later.

For a reflexive globular set $A$, write
\[
F(A) = 
\left(
\begin{diagram}[height=1.5em]
A^\#		\\
\dTo>{\pi_A}	\\
A^*		\\
\end{diagram},\ 
\gamma_A
\right).
\]
$A^*$ is, in fact, relatively easy to describe: it is the free strict
$\omega$-category on $A$, in which an $m$-cell is a formal pasting-together
of cells of $A$ of dimension $\leq m$.


\begin{description}
\item[Dimension $0$] We have $A^\#(0) = A^*(0) = A(0)$  and $(\pi_A)_0
= \id$.
%
\item[Dimension $1$] Next, $A^*(1)$ is the set of formal paths of 1-cells in
$A$, where we identify each identity cell $1_a$ with the identity path on
$a$.  The set $A^\#(1)$ and the functions $s,t: A^\#(1) \go A(0)$ are generated
by the following recursive clauses:
\begin{itemize}
\item if $a_0 \goby{f} a_1$ is a 1-cell in $A$ then $A^\#(1)$ contains an
element celled $f$, with $s(f)=a_0$ and $t(f)=a_1$
\item if $w,w' \in A^\#(1)$ with $t(w)=s(w')$ then $A^\#(1)$ contains an
element called $(w' \bofdim{0} w)$, with $s(w' \bofdim{0} w)=s(w)$ and $t(w'
\bofdim{0} w)=t(w')$. 
\end{itemize}
The identity map $A(0) \go A^\#(1)$ sends $a$ to $1_a \in A(1) \sub A^\#(1)$,
the map $\pi_A$ removes parentheses and sends $\bofdim{0}$ to $\ofdim{0}$,
and the contraction $\gamma_A$ is given by $\gamma_A(a,a) = 1_a$ (for $a\in
A(0)$).
%
\item[Dimension $2$] $A^*(1)$ is the set of formal pastings of
2-cells in $A$, again respecting the identities.  $A^\#(2)$ and $s, t:
A^\#(2) \go A^\#(1)$ are generated by:
\begin{itemize}
\item if $\alpha$ is a 2-cell in $A$ then $A^\#(2)$ has an element
called $\alpha$, with the evident source and target
\item if $a \parpair{w_0}{w_1} b$ in $A^\#(1)$ with $\pi_A(w_0)=\pi_A(w_1)$
then $A^\#(2)$ has an element called $\gamma_A(w_0,w_1)$, with source $w_0$ and
target $w_1$
\item if $x,x' \in A^\#(2)$ with $t(x)=s(x')$ then $A^\#(2)$ has an
element called $(x' \bofdim{1} x)$, with source $s(x)$ and target $t(x')$
\item if $x,x' \in A^\#(2)$ with $tt(x)=ss(x')$ then $A^\#(2)$ has an element
called $(x' \bofdim{0} x)$, with source $s(x')\bofdim{0} s(x)$ and target
$t(x')\bofdim{0} t(x)$;
\end{itemize}
furthermore, if $f \in A(1)$ then $1_f$ (from the first clause) is to be
identified with $\gamma_A(f,f)$ (from the second).  The identity map $A^\#(1)
\go A^\#(2)$ sends $w$ to $\gamma_A(w,w)$.  The map $\pi_A$ sends cells of
the form $\gamma_A(w_0,w_1)$ to identity cells, and otherwise acts as in
dimension 1.  The contraction $\gamma_A$ is defined in the way suggested by
the notation.

\end{description}


\paragraph{Adjointness}

We now have to see that this $F$ is indeed left adjoint to $U$.  First
observe that there is a natural embedding of $A(m)$ into $A^\#(m)$ (for
$m\leq 2$); this gives the unit of the adjunction.  Adjointness then says:
given $(M,S,\pi,\gamma) \in \cat{Q}$ and a map $A \goby{\phi} M$
of reflexive globular sets, there's a unique map
\[
(\chi,\zeta): \vslob{A^\#}{\pi_A}{A^*} \go \vslob{M}{\pi}{S}
\]
in \cat{Q} such that $\chi$ extends $\phi$.  This can be seen from the
description above.

\paragraph{Weak $2$-Categories}

A weak $2$-category consists of a $2$-dimensional reflexive globular set $A$
together with:
%
\begin{itemize}
\item (a map $A^\#(0) \go A(0)$ obeying axioms---which force it to be the
identity)
\item a map $A^\#(1) \go A(1)$ obeying axioms, which amounts to a binary
composition on the 1-cells of $A$ (\emph{not} obeying any axioms)
\item similarly, vertical and horizontal binary compositions of 2-cells, not
obeying any axioms `yet'
\item for each string $\cdot \goby{f_1} \cdots \goby{f_k} \cdot$ of 1-cells,
and each pair $\tau,\tau'$ of $k$-leafed binary trees, a 2-cell
$
\omega_{\tau,\tau'}: 
\of_{\tau}(f_1, \ldots, f_k) \go \of_{\tau'}(f_1, \ldots, f_k), 
$
where $\of_{\tau}$ indicates the iterated composition dictated by the shape
of $\tau$
\item amongst other things in dimension $3$: whenever we have some $2$-cells
$(\alpha_i)$, and two different ways of composing all the $\alpha_i$'s and
some $\omega_{\tau,\tau'}$'s to obtain new 2-cells $\beta$ and $\beta'$
respectively, and these satisfy $s(\beta)=s(\beta')$ and
$t(\beta)=t(\beta')$, then there is assigned a 3-cell $\beta\go\beta'$.
\end{itemize}
%
Since `the only $3$-cells of $A$ are equalities', we get $\beta=\beta'$ in
the last item.  Analysing this precisely, we find that the category of weak
2-categories is equivalent to the category of bicategories and strict
functors.  And more easily, a weak 1-category is just a category and a weak
0-category is just a set.



\concept{Indirect Interpretation}

An alternative way of handling weak $n$-categories is to work only with
$n$-dimensional (not infinite-dimensional) structures throughout: e.g.\
reflexive globular sets $A$ in which $A(m)$ is only defined for $m\leq n$.
We then only speak of contractions on a map $\phi$ if $(f_0,f_1) \in
V_\phi(n) \implies f_0=f_1$ (and in particular, the map $\pi$ must satisfy
this condition in order for $(M,S,\pi,\gamma)$ to qualify as an object of
\cat{Q}).  Our new category of weak $n$-categories appears to be equivalent
to the old one, taking algebra maps as the morphisms in both cases.

The analysis of $n=2$ is easier now: we can write down the left adjoint $F$
explicitly, and so get an explicit description of the monad $T$ on the
category of `reflexive 2-globular sets'.  This monad is presumably the free
bicategory monad.

\clearpage



