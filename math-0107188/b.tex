
\defnsheading{B}		\label{p:b}



\concept{Globular Operads and their Algebras}

\paragraph{Globular Sets}
 
Let \scat{G} be the category whose objects are the natural numbers
$0,1,\ldots$, and whose arrows are generated by
$
\sigma_m, \tau_m: m \go m-1
$
for each $m\geq 1$, subject to equations
\[
\sigma_{m-1} \of \sigma_m = \sigma_{m-1} \of \tau_m,
\diagspace
\tau_{m-1} \of \sigma_m = \tau_{m-1} \of \tau_m
\]
($m\geq 2$).  A functor $A: \scat{G} \go \Set$ is called a \demph{globular
set}; I will write $s$ for $A(\sigma_m)$, and $t$ for $A(\tau_m)$. 

\paragraph{The Free Strict $\omega$-Category Monad}

Any (small) strict $\omega$-category has an underlying globular set $A$, in
which $A(m)$ is the set of $m$-cells and $s$ and $t$ are the source and
target maps.  We thus obtain a forgetful functor $U$ from the category of
strict $\omega$-categories and strict $\omega$-functors to the category
\ftrcat{\scat{G}}{\Set} of globular sets.  $U$ has a left adjoint, so there
is an induced monad $(T, \id \goby{\eta} T, T^2 \goby{\mu} T)$ on
\ftrcat{\scat{G}}{\Set}.

\paragraph{Collections}

We define a monoidal category \fcat{Coll} of collections.  Let $1$ be the
terminal globular set.  A \demph{(globular) collection} is a map $C \goby{d}
T1$ into $T1$ in \ftrcat{\scat{G}}{\Set}; a \demph{map of collections} is a
commutative triangle.  The \demph{tensor product} of collections $C \goby{d}
T1$, $C' \goby{d'} T1$ is the composite along the top row of
%
\begin{diagram}[size=2em]
\SEpbk C \otimes C'&\rTo&TC'	&\rTo^{Td'}	&T^2 1	&\rTo^{\mu_1}&T1\\
\dTo		&	&\dTo>{T!}&		&	&	&	\\
C		&\rTo^d	&T1,	&		&	&	&	\\
\end{diagram}
%
where the right-angle symbol means that the square containing it is a
pullback, and $!$ denotes the unique map to $1$.  The \demph{unit} for the
tensor is $1 \goby{\eta_1} T1$.

\paragraph{Globular Operads}

A \demph{(globular) operad} is a monoid in the monoidal category \fcat{Coll};
a \demph{map of operads} is a map of monoids.  

\paragraph{Algebras}

Any operad $C$ induces a monad $C\cdot\dashbk$ on \ftrcat{\scat{G}}{\Set}.
For an object $A$ of \ftrcat{\scat{G}}{\Set}, this is defined by pullback: 
%
\begin{diagram}[size=2em]
\SEpbk C\cdot A &\rTo		&TA		\\
\dTo		&		&\dTo>{T!}	\\
C		&\rTo^d		&T1.		\\
\end{diagram}
%
The multiplication and unit of the monad come from the multiplication and
unit of the operad.  A $C$-\emph{algebra} is an algebra for the monad
$C\cdot\dashbk$.  Note that every $C$-algebra has an underlying globular
set.


\concept{Contractions and Systems of Composition}

\paragraph{Contractions}

Let $C \goby{d} T1$
be a collection.  For $m\geq 0$ and $\nu \in (T1)(m)$, write
$C(\nu) = \{ \theta \in C(m) \such d(\theta) = \nu \}$.  For $m\geq 1$ and
$\nu \in (T1)(m)$, define
\[
Q_C (\nu) =	\{ (\theta_0, \theta_1) \in C(\nu) \times C(\nu) \such
	s(\theta_0) = s(\theta_1) \mbox{ and } t(\theta_0) = t(\theta_1)\},
\]
and for $\nu \in (T1)(0)$, define $Q_C(\nu) = C(\nu) \times C(\nu)$.  Part of
the strict $\omega$-category structure on $T1$ is that each element $\nu \in
(T1)(m)$ gives rise to an element $1_\nu \in (T1)(m+1)$.  A
\demph{contraction} on $C$ is a family of functions
\[
(\gamma_\nu: Q_C (\nu) \go C(1_\nu))_{m\geq 0, \nu\in (T1)(m)} 
\]
satisfying
\[
s(\gamma_\nu \pr{\theta_0}{\theta_1} ) = \theta_0,
\diagspace
t(\gamma_\nu \pr{\theta_0}{\theta_1} ) = \theta_1
\]
for every $m\geq 0$, $\nu\in (T1)(m)$ and $\pr{\theta_0}{\theta_1} \in
Q_C (\nu)$. 

\paragraph{Systems of Compositions}

The map $\eta_1: 1 \go T1$ picks out an element $\eta_{1,m}$ of $(T1)(m)$
for each $m\geq 0$.  The strict $\omega$-category structure on $T1$ then
gives an element
\[
\beta_p^m = \eta_{1,m} \ofdim{p} \eta_{1,m} \in (T1)(m)
\]
for each $m > p \geq 0$; also put $\beta_m^m = \eta_{1,m}$.  Defining
$B(m) = \{ \beta_p^m \such m \geq p \geq 0 \} \sub (T1)(m)$, we obtain a
collection $B \rIncl T1$.

Also, the elements $\beta_m^m = \eta_{1,m} \in (T1)(m)$ determine a map $1
\go B$.

A \demph{system of compositions} in an operad $C$ is a map $B \go C$ of
collections such that the composite $1 \go B \go C$ is the unit of the
operad $C$.

\paragraph{Initial Object}  

Let \fcat{OCS} be the category in which an object is an operad equipped with
both a contraction and a system of compositions, and in which a map is a map
of operads preserving both the specified contraction and the specified system
of compositions.  Then \fcat{OCS} can be shown to have an initial object,
whose underlying operad will be written $K$.


\concept{The Definitions}

\paragraph{Definition \ds{B1}} 

A \demph{weak $\omega$-category} is a $K$-algebra.

\paragraph{Definition \ds{B2}} 

A \demph{weak $\omega$-category} is a pair $(C,A)$, where $C$ is an operad
satisfying $C(0) \iso 1$ and on which there exist a contraction and a system
of compositions, and $A$ is a $C$-algebra.

\paragraph{Weak $n$-Categories}

Let $n\geq 0$.  A globular set $A$ is \emph{$n$-dimensional} if for all
$m\geq n$,
\[
s=t: A(m+1) \go A(m)
\]
and this map is an isomorphism.  A \demph{weak $n$-category} is a weak
$\omega$-category whose underlying globular set is $n$-dimensional.  This can
be interpreted according to either \ds{B1} or \ds{B2}.

\clearpage



\lowsdimsheading{B}

\concept{Definition \ds{B1}}

An alternative way of handling weak $n$-categories is to work with only $n$-
(not infinite-) dimensional structures throughout.  So we replace $\scat{G}$
by its full subcategory $\scat{G}_n$ with objects $0, \ldots, n$, and $T$ by
the free strict $n$-category monad $T_n$, to obtain definitions of
\demph{$n$-collection}, \demph{$n$-operads}, and their \demph{algebras}.
\demph{Contractions} are defined as before, except that we only speak of
contractions on $C$ if
%
\begin{equation}	\label{eq:n-contr-b}
\forall \nu \in (T_n 1)(n), (\theta_0, \theta_1) \in Q_C(\nu)
\implies 
\theta_0 = \theta_1.
\end{equation}
%
There is an initial $n$-operad $K_n$ equipped with a contraction and a system
of compositions, and the category of weak $n$-categories turns out to be
equivalent to the category of $K_n$-algebras.  The latter is easier to
analyse.

\paragraph{$n=0$} 

We have $\ftrcat{\scat{G}_0}{\Set} \iso \Set$, $T_0 = \id$, and
$0\hyph\fcat{Coll} \iso \Set$; a 0-operad $C$ is a monoid, and a $C$-algebra
is a set with a $C$-action.  By~\bref{eq:n-contr-b}, the only 0-operad with a
contraction is the one-element monoid, so a weak $0$-category is just a set.

\paragraph{$n=1$}

$\ftrcat{\scat{G}_1}{\Set}$ is the category of directed graphs and $T_1$ is
the free category monad.  $K_1$ is the terminal 1-operad, by arguments
similar to those under `$n=2$' below.  It follows that the induced monad $K_1
\cdot \dashbk$ is just $T_1$, and so a weak $1$-category is just a
$T_1$-algebra, that is, a category.

\paragraph{$n=2$} 

A functor $A: \scat{G}_2 \go \Set$ consists of a set of $0$-cells (drawn
$\gzeros{a}$), a set of $1$-cells ($\gfsts{a}\gones{f}\glsts{b}$), and a set
of $2$-cells ($\gfsts{a}\gtwos{f}{g}{\alpha}\glsts{b}$).  A 2-collection $C$
consists of a set $C(0)$, a set $C(\nu_k)$ for each $k\geq 0$ (where $\nu_k$
indicates the `1-pasting diagram' $\gfstsu\gonesu\ \ldots\ \gonesu\glstsu$
with $k$ arrows), and a set $C(\pi)$ for each `2-pasting diagram' $\pi$ such
as the $\pi_i$ in Fig.~\ref{fig:op-comp-l}, together with source and target
functions.

A 2-operad is a 2-collection $C$ together with `composition' functions such as 
\[
\begin{array}{ccc}
C(\nu_3) \times [C(\nu_2) \times_{C(0)} C(\nu_1) \times_{C(0)} C(\nu_2)] 
&\go &
C(\nu_5), \\
C(\pi_1) \times [C(\pi_2) \times_{C(\nu_2)} C(\pi_3)] &\go & C(\pi_4).
\end{array}
\]
In the first, the point is that there are $3$ terms $2,1,2$ and their sum is
$5$.  This makes sense if an element of $C(\nu_k)$ is regarded as an
operation which takes a string of $k$ 1-cells and turns it into a single
$1$-cell.  (The $\times_{C(0)}$'s denote pullbacks.)  Similarly for the
second; see Fig.~\ref{fig:op-comp-b}.  There are also identities for the
compositions.
%
\begin{figure}
\piccy{compoppic.ps}
\caption{Composition of operations in a globular operad}
\label{fig:op-comp-b}
\end{figure}
%
A $C$-algebra is a functor $A: \scat{G}_2 \go \Set$ together with functions 
\[
\begin{array}{l}
\ovln{\psi}:	A(0) \go A(0) 
\textrm{ for each } \psi \in C(0),
\\
\ovln{\phi}: 	\{ \textrm{diagrams } \gfsts{a_0}\gones{f_1}\ \cdots \
\gones{f_k} \glsts{a_k} \textrm{ in } A \} \go A(1)			
\textrm{ for each } \phi \in C(\nu_k),
\\
\ovln{\theta}:	\{ \textrm{diagrams }
\gfsts{a}%
\gthrees{f}{g}{h}{\alpha}{\beta}%
\gfbws{\ b}\gtwos{l}{m}{\!\gamma}\glsts{c} 
\textrm{ in } A \} \go A(2)
\textrm{ for each } \theta \in C(\gfstsu\gthreesu\gzersu\gtwosu\glstsu)
\end{array}
\]
(etc), all compatible with the source, target, composition and identities in
$C$.

$K_2$ is generated from the empty collection by adding in the minimal amount
to obtain a 2-operad with contraction and system of compositions.  We have
the identity $1\in K_2(0)$.  Then, contraction gives an element (`1-cell
identities') of $K_2(\nu_0)$, the system of compositions gives an element
(`1-cell composition') of $K_2(\nu_2)$, and composition in $K_2$ gives one
element of $K_2(\nu_k)$ for each $k$-leafed tree in which every vertex has 0
or 2 edges coming up out of it.  Contraction at the next level gives
associativity and unit isomorphisms and identity 2-cells; the system of
compositions gives vertical and horizontal 2-cell composition.
Condition~\bref{eq:n-contr-b} gives coherence axioms.  Thus a weak 2-category
is exactly a bicategory.



\concept{Definition \ds{B2}}

Definition \ds{B2} of weak $n$-category refers to
infinite-dimensional globular operads.  So in order to do a concrete analysis
of $n\leq 2$, we redefine a \demph{weak $n$-category} as a pair $(C,A)$ where
$C$ is a $n$-operad admitting a contraction and a system of compositions and
with $C(0) \iso 1$, and $A$ is a $C$-algebra.  (Temporarily, call such a $C$
\demph{good}.)  I do not know to what extent this is equivalent to \ds{B2},
but the spirit, at least, is the same.


\paragraph{$n=0$} 

From `$n=0$' above we see that a weak $0$-category is just a set.


\paragraph{$n=1$}

The only good $1$-operad is the terminal $1$-operad, so by `$n=1$' above, a
weak $1$-category is just a category.


\paragraph{$n=2$}

A \demph{(non-symmetric) classical operad} $D$ is a sequence $(D(k))_{k\geq
0}$ of sets together with an element (the \demph{identity}) of $D(1)$ and for
each $k, r_1, \ldots, r_k \geq 0$ a map $ D(k) \times D(r_1) \times \cdots
\times D(r_k) \go D(r_1 + \cdots + r_k) $ (\demph{composition}), obeying unit
and associativity laws.  It turns out that good $2$-operads $C$ correspond
one-to-one with classical operads $D$ such that $D(k) \neq \emptyset$ for
each $k$, via $D(k) = C(\nu_k)$.  A $C$-algebra is then something like a
2-category or bicategory, with one way of composing a string of $k$ $1$-cells
for each element of $D(k)$, and all the appropriate coherence $2$-cells.
E.g.\ if $D=1$ then a $C$-algebra is a 2-category; if $D(k)$ is the set of
$k$-leafed trees in which each vertex has either $0$ or $2$ edges coming up
out of it then a $C$-algebra is a bicategory.  $C$ can, therefore, be
regarded as a theory of (more or less weak) 2-categories, and $A$ as a model
for such a theory.

