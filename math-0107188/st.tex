

\defnheading{St}		\label{p:st}



\concept{Simplicial Sets}

\paragraph{The Simplicial Category}

Let \Del\ be a skeleton of the category of nonempty finite totally ordered
sets: that is, \Del\ has objects $[m]=\{0,\ldots,m\}$ for $m\geq 0$, and maps
are order-preserving functions (with respect to the usual ordering $\leq$).
A \demph{simplicial set} is a functor $\Delop \go \Set$.

\paragraph{Maps in \Del}

Let $m\geq 1$: then there are injections $\delta_0, \ldots, \delta_m: [m-1]
\go [m]$ in \Del, determined by saying that the image of $\delta_i$ is $[m]
\without \{ i \}$.  

Let $A: \Delop \go \Set$ and $m\geq 0$.  An element $a\in A[m]$ is called
\demph{degenerate} if there exist a natural number $m'<m$, a surjection
$\sigma: [m] \go [m']$, and an element $a' \in A[m']$ such that $a =
(A\sigma)a'$.

\paragraph{Horns}

Given $0\leq k\leq m$, we define the \demph{horn} $\Lambda^k_m: \Delop \go
\Set$ by
\[
\Lambda^k_m[m'] = \{ \psi \in \Del([m'],[m]) \such \textrm{image}(\psi)
\not\supseteq [m]\without\{k\} \}.
\]
That is, $\Lambda^k_m[m']$ is the set of all maps $\psi: [m'] \go [m]$ in
\Del\ except for the surjections and the maps with image $ \{ 0, \ldots, k-1,
k+1, \ldots, m \}$.  So for each $m'$ we have an inclusion $\Lambda^k_m[m']
\rIncl \Del([m'],[m])$, and $\Lambda^k_m$ is thus a subfunctor of
$\Del(\dashbk,[m])$.  Write $i^k_m: \Lambda^k_m \rIncl \Del(\dashbk,[m])$ for
the inclusion.

Let $A$ be a simplicial set.  A \demph{horn in $A$} is a natural
transformation $h: \Lambda^k_m \go A$, for some $0\leq k\leq m$.  A
\demph{filler} for the horn $h$ is a natural transformation $\ovln{h}:
\Del(\dashbk,[m]) \go A$ making the following diagram commute:
%
\begin{diagram}[height=2em]
\Lambda^k_m	&\rIncl^{i^k_m}	&\Del(\dashbk,[m])	\\
		&\rdTo<h	&\dTo>{\ovln{h}}	\\
		&		&A.			\\
\end{diagram}



\concept{Orientation}

\paragraph{Alternating Sets} 

A set of natural numbers is \demph{alternating} if its elements,
when written in ascending order, alternate in parity.

Let $0\leq k\leq m$, and write $k^\pm = \{k-1,k,k+1\} \cap [m]$.  A subset $S
\sub [m]$ is \demph{$k$-alternating} if
%
\begin{itemize}
\item $k^\pm \sub S$
\item the set $k^\pm \cup ([m] \without S)$ is alternating.
\end{itemize}

\paragraph{Admissible Horns}

A \demph{simplicial set with hollowness} is a simplicial set $A$ together
with a subset $H_m\sub A[m]$ for each $m\geq 1$, whose elements are called
the \demph{hollow} elements of $A[m]$ (and may also be thought of as `thin'
or `universal').

Let $(A,H)$ be a simplicial set with hollowness, and $0\leq k\leq m$.  A horn
$h: \Lambda^k_m \go A$ is \demph{admissible} if for every $m'\geq 1$ and
$\psi\in\Lambda^k_m[m']$,
\[
\textrm{image}(\psi) \textrm{ is a } k\textrm{-alternating subset of }[m]
\ \implies\ 
h_{[m']}(\psi) \textrm{ is hollow.}	
\]



\concept{The Definition}

\paragraph{Weak $\omega$-Categories}

A \demph{weak $\omega$-category} is a simplicial set with hollowness $(A,H)$
such that
%
\begin{enumerate}
\item  	\label{part:degen} 
for $m\geq 1$, $H_m \supseteq \{ \textrm{degenerate elements of }
A[m] \}$
\item 	\label{part:filler}
for $m\geq 1$ and $0\leq k\leq m$, every admissible horn $h:
\Lambda^k_m \go A$ has a filler $\ovln{h}$ satisfying $\ovln{h}_{[m]}(1_{[m]})
\in H_m$ (`every admissible horn has a hollow filler')
\item  	\label{part:comp}
for $m\geq 2$ and $0\leq k\leq m$, if $a\in H_m$ has the property that
$(A\delta_i)a \in H_{m-1}$ for each $i \in [m]\without \{k\}$ then also
$(A\delta_k)a \in H_{m-1}$. 
\end{enumerate}


\paragraph{Weak $n$-Categories}

Let $n\geq 0$.  A \demph{weak $n$-category} is a weak $\omega$-category
$(A,H)$ such that 
%
\renewcommand{\theenumi}{\roman{enumi}$'$}
%
\begin{enumerate}
\item 	\label{part:high-dims}
for $m > n$, $H_m = A[m]$
\item 	\label{part:unique}
in condition~\bref{part:filler} above, when $m > n$ there is a
\emph{unique} filler $\ovln{h}$ for $h$ (which necessarily satisfies
$\ovln{h}_{[m]}(1_{[m]}) \in H_m$). 
\end{enumerate}
%
\renewcommand{\theenumi}{\roman{enumi}}




\clearpage



\lowdimsheading{St}


Let $m\geq 0$ and let $S$ be a nonempty subset of $[m]$; then in \Del\ there
is a unique injection $\phi$ into $[m]$ with image $S$.  Given a simplicial
set $A$ and an element $a\in A[m]$, the \demph{$S$-face} of $a$ is the
element $(A\phi)a$ of $A[l]$, where $l+1$ is the cardinality of $S$.
Similarly, the \demph{$S$-face} of a horn $h: \Lambda^k_m \go A$ is
$h_{[l]}(\phi) \in A[l]$ (which makes sense as long as $S \not\supseteq
[m]\without \{k\}$).

To compare weak $1$-($2$-)categories with (bi)categories, we
need to interpret elements $a\in A[m]$ as arrows pointing in some direction.
Our convention is: if $S$ is an $m$-element subset of the
$(m+1)$-element set $[m]$ and the missing element is odd, then we regard the
$S$-face of $a$ as a source; if even, a target.
See Fig.~\ref{fig:simplices}.

Suppose $(A,H)$ is a simplicial set with hollowness
satisfying~\bref{part:degen}, and let $h: \Lambda^k_m \go A$ be a horn
satisfying the defining condition for admissibility for just the
\emph{injective} $\psi\in\Lambda^k_m[m']$.  Then, in fact, $h$ is admissible.
So $h$ is admissible if and only if: \emph{for every $k$-alternating subset
$S$ of $[m]$, the $S$-face of $h$ is hollow}.  Table~\ref{table:alt} shows the
$k$-alternating subsets of $[m]$ in the cases we need.
%
\begin{table}[b]
\centering
\begin{tabular}{c|c}
$k$		&$k$-alternating subsets of $[m]$ of cardinality $\leq 3$\\
\hline
$0$		&$\{0,1\}, \{0,1,m\}$					\\
$1, \ldots, m-1$&$\{k-1,k,k+1\}$					\\
$m$		&$\{m-1,m\}, \{0,m-1,m\}$				
\end{tabular}
%
\caption{$k$-alternating subsets of $[m]$ of cardinality $\leq 3$, for $m\geq
1$} 
\label{table:alt}
\end{table}


\concept{$n=0$}

A weak $0$-category is a simplicial set $A$ in which every horn has a unique
filler---including those of shape $\Lambda^k_1$.  It follows that the functor
$A: \Delop \go \Set$ is constant, so a weak $0$-category is just a set.


\concept{$n=1$}

A horn of shape $\Lambda_m^k$ is called \emph{inner} if $0<k<m$; a simplicial
set is the nerve of a category if and only if every inner horn has a unique
filler.  If $(A,H)$ is a simplicial set with hollowness
satisfying~\bref{part:high-dims} for $n=1$ then every inner horn is
admissible, hence, if~\bref{part:filler} and~\bref{part:unique} also hold,
has a unique filler: so $A$ is (the nerve of) a category.  Working out the
other conditions, we find that a weak $1$-category is a category equipped
with a set $H_1$ of isomorphisms containing all the identity maps and closed
under composition and inverses.  So given a weak $1$-category we obtain a
category by forgetting $H$; conversely, given a category we can take $H_1 =
\{ \textrm{all isomorphisms} \}$ (or $\{ \textrm{all identities} \}$) to
obtain a weak $1$-category.


\concept{$n=2$}

\begin{figure}[t]
\setlength{\unitlength}{1mm}
\begin{picture}(122,35) 
%
\cell{12}{20}{c}{\textrm{(a)}}
\put(0,23){%
\begin{picture}(24,12)
\cell{4}{1}{c}{\zmark}
\cell{20}{1}{c}{\zmark}
\put(4,1){\vector(1,0){15.5}}
\cell{3}{1}{tr}{\scriptstyle a_0}
\cell{21}{1}{tl}{\scriptstyle a_1}
\cell{12}{2}{b}{\scriptstyle f_{01}}
\end{picture}
}
%
\cell{45}{20}{c}{\textrm{(b)}}
\put(33,23){%
\begin{picture}(24,12)
\cell{4}{1}{c}{\zmark}
\cell{20}{1}{c}{\zmark}
\cell{12}{11}{c}{\zmark}
\put(4,1){\vector(1,0){15.5}}
\put(4,1){\line(4,5){7.8}}
\put(11.8,10.8){\vector(1,1){0}}
\put(12,11){\line(4,-5){7.8}}
\put(19.8,1.2){\vector(1,-1){0}}
\put(12,3){\vector(0,1){5}}
\cell{3}{1}{tr}{\scriptstyle a_0}
\cell{13}{11}{l}{\scriptstyle a_1}
\cell{21}{1}{tl}{\scriptstyle a_2}
\cell{8}{6}{br}{\scriptstyle f_{01}}
\cell{16}{6}{bl}{\scriptstyle f_{12}}
\cell{12}{1.5}{t}{\scriptstyle f_{02}}
\cell{12}{5}{c}{\scriptstyle \alpha_{012}}
\end{picture}
}
%
\cell{77}{20}{c}{\textrm{(c)}}
\put(65,23){%
\begin{picture}(24,12)
\cell{4}{1}{c}{\zmark}
\cell{20}{1}{c}{\zmark}
\cell{12}{11}{c}{\zmark}
\qbezier[16](4,1)(11.8,1)(19.6,1)
\put(19.5,1){\vector(1,0){0}}
\put(4,1){\line(4,5){7.8}}
\put(11.8,10.8){\vector(1,1){0}}
\put(12,11){\line(4,-5){7.8}}
\put(19.8,1.2){\vector(1,-1){0}}
\qbezier[5](12,3)(12,5.2)(12,7.4)
\put(12,8){\vector(0,1){0}}
\cell{3}{1}{tr}{\scriptstyle a_0}
\cell{13}{11}{l}{\scriptstyle a_1}
\cell{21}{1}{tl}{\scriptstyle a_2}
\cell{8}{6}{br}{\scriptstyle f_{01}}
\cell{16}{6}{bl}{\scriptstyle f_{12}}
\end{picture}
}
%
\cell{110}{20}{c}{\textrm{(d)}}
\put(98,23){%
\begin{picture}(24,12)
\cell{4}{1}{c}{\zmark}
\cell{20}{1}{c}{\zmark}
\cell{12}{11}{c}{\zmark}
\put(4,1){\vector(1,0){15.5}}
\put(4,1){\line(4,5){7.8}}
\put(11.8,10.8){\vector(1,1){0}}
\qbezier[13](12,11)(15.9,6.6)(19.8,1.2)
\put(19.8,1.2){\vector(1,-1){0}}
\qbezier[5](12,3)(12,5.2)(12,7.4)
\put(12,8){\vector(0,1){0}}
\cell{3}{1}{tr}{\scriptstyle a_0}
\cell{13}{11}{l}{\scriptstyle a_1}
\cell{21}{1}{tl}{\scriptstyle a_2}
\cell{9}{7}{br}{\scriptstyle f_{01}}
\cell{12}{1.5}{t}{\scriptstyle f_{02}}
\cell{5.5}{5}{c}{\sim}
\end{picture}
}
%
\cell{21}{2}{c}{\textrm{(e)}}
\put(-11,5){%
\begin{picture}(64,10)
% Left-hand bit
\cell{0}{0}{c}{\zmark}
\cell{26}{0}{c}{\zmark}
\cell{6}{10}{c}{\zmark}
\cell{20}{10}{c}{\zmark}
\put(0,0){\vector(1,0){25.8}}
\put(0,0){\line(3,5){5.8}}
\put(5.9,9.8){\vector(2,3){0}}
\put(6,10){\vector(1,0){13.8}}
\put(20,10){\line(3,-5){5.8}}
\put(25.9,0.2){\vector(2,-3){0}}
\put(0,0){\vector(2,1){19.8}}
\put(18,2){\vector(0,1){5}}
\put(11,7){\vector(-2,1){4}}
\cell{1.5}{5}{c}{\sim}
\cell{-0.5}{0}{br}{\scriptstyle a_0}
\cell{5}{10}{tr}{\scriptstyle a_1}
\cell{21}{10}{tl}{\scriptstyle a_2}
\cell{26.5}{0}{bl}{\scriptstyle a_3}
% Middle arrow
\qbezier[10](27,6)(31.8,6)(36.6,6)
\put(37,6){\vector(1,0){0}}
% Right-hand bit
\cell{38}{0}{c}{\zmark}
\cell{64}{0}{c}{\zmark}
\cell{44}{10}{c}{\zmark}
\cell{58}{10}{c}{\zmark}
\put(38,0){\vector(1,0){25.8}}
\put(38,0){\line(3,5){5.8}}
\put(43.9,9.8){\vector(2,3){0}}
\put(44,10){\vector(1,0){13.8}}
\put(58,10){\line(3,-5){5.8}}
\put(63.9,0.2){\vector(2,-3){0}}
\put(44,10){\vector(2,-1){19.8}}
\put(46,2){\vector(0,1){5}}
\put(57,9){\vector(2,1){0}}
\qbezier[4](53,7)(54.6,7.8)(56.2,8.6)
\cell{39.5}{5}{c}{\sim}
\cell{48}{4}{c}{\sim}
\cell{37.5}{0}{br}{\scriptstyle a_0}
\cell{43}{10}{tr}{\scriptstyle a_1}
\cell{59}{10}{tl}{\scriptstyle a_2}
\cell{64.5}{0}{bl}{\scriptstyle a_3}
\end{picture}
}
%
\cell{101}{2}{c}{\textrm{(f)}}
\put(69,5){%
\begin{picture}(64,12)
% Left-hand bit
\cell{0}{0}{c}{\zmark}
\cell{26}{0}{c}{\zmark}
\cell{6}{10}{c}{\zmark}
\cell{20}{10}{c}{\zmark}
\put(0,0){\vector(1,0){25.8}}
\put(0,0){\line(3,5){5.8}}
\put(5.9,9.8){\vector(2,3){0}}
\put(6,10){\vector(1,0){13.8}}
\put(20,10){\line(3,-5){5.8}}
\put(25.9,0.2){\vector(2,-3){0}}
\put(0,0){\vector(2,1){19.8}}
\put(18,7){\vector(0,1){0}}
\qbezier[5](18,2)(18,4.2)(18,6.4)
\put(11,7){\vector(-2,1){4}}
\cell{8}{6.5}{c}{\sim}
\cell{-0.5}{0}{br}{\scriptstyle a_0}
\cell{5}{10}{tr}{\scriptstyle a_1}
\cell{21}{10}{tl}{\scriptstyle a_2}
\cell{26.5}{0}{bl}{\scriptstyle a_3}
% Middle arrow
\qbezier[10](27,6)(31.8,6)(36.6,6)
\put(37,6){\vector(1,0){0}}
% Right-hand bit
\cell{38}{0}{c}{\zmark}
\cell{64}{0}{c}{\zmark}
\cell{44}{10}{c}{\zmark}
\cell{58}{10}{c}{\zmark}
\put(38,0){\vector(1,0){25.8}}
\put(38,0){\line(3,5){5.8}}
\put(43.9,9.8){\vector(2,3){0}}
\put(44,10){\vector(1,0){13.8}}
\put(58,10){\line(3,-5){5.8}}
\put(63.9,0.2){\vector(2,-3){0}}
\put(44,10){\vector(2,-1){19.8}}
\put(46,2){\vector(0,1){5}}
\put(53,7){\vector(2,1){4}}
\cell{37.5}{0}{br}{\scriptstyle a_0}
\cell{43}{10}{tr}{\scriptstyle a_1}
\cell{59}{10}{tl}{\scriptstyle a_2}
\cell{64.5}{0}{bl}{\scriptstyle a_3}
\end{picture}
}
\end{picture}
%
\caption{(a)~Element of $A[1]$, with, for instance, the $\{0\}$-face
labelled $a_0$; 
(b)~element of $A[2]$; 
(c)~(admissible) horn $\Lambda^1_2 \protect\go A$;
(d)~admissible horn $\Lambda^0_2 \protect\go A$, with $\diso$ indicating a
hollow face; 
(e)~admissible horn $\Lambda^0_3 \protect\go A$, with labels $f_{ij},
\alpha_{ijk}$ omitted; 
(f)~as (e), but for $\Lambda^1_3$.}
\label{fig:simplices}
\end{figure}

A weak $2$-category is a simplicial set $A$ equipped with subsets $H_1\sub
A[1]$ and $H_2 \sub A[2]$, satisfying certain axioms.  It appears that this
is the same as a bicategory equipped with a set $H_1$ of $1$-cells which are
equivalences and a set $H_2$ of 2-cells which are isomorphisms, satisfying
closure conditions similar to those under `$n=1$' above.

So, let $(A,H)$ be a weak $2$-category.  We construct a bicategory whose $0$-
and $1$-cells are the elements of $A[0]$ and $A[1]$; a 2-cell $f\go g$ is an
element of $A[2]$ of the form
%
\raisebox{-5mm}{%
\setlength{\unitlength}{1mm}
\begin{picture}(24,10)
\cell{4}{1}{c}{\zmark}
\cell{20}{1}{c}{\zmark}
\cell{12}{9}{c}{\zmark}
\put(4,1){\vector(1,0){15.5}}
\put(4,1){\vector(1,1){7.8}}
\put(12,9){\vector(1,-1){7.8}}
\put(12,3){\vector(0,1){4}}
\cell{3}{2}{tr}{\scriptstyle a}
\cell{13}{9}{l}{\scriptstyle a}
\cell{21}{2}{tl}{\scriptstyle b}
\cell{8}{5}{br}{\scriptstyle 1_a}
\cell{16}{5}{bl}{\scriptstyle g}
\cell{11}{1.5}{c}{\scriptstyle f}
\end{picture}},
%
where $1_a$ indicates a degenerate $1$-cell.  Composition of $1$-cells is
defined by making a random choice of hollow filler for each horn of shape
$\Lambda^1_2$; composition of $2$-cells is defined by filling in
$3$-dimensional horns $\Lambda^k_3$; identities are got from degeneracies.
See Fig.~\ref{fig:simplices}.  The associativity and unit isomorphisms are
certain hollow cells, and the coherence axioms hold because of the uniqueness
of certain fillers.

Conversely, let $B$ be a bicategory, and construct a weak $2$-category
$(A,H)$ as follows.  $A[0]$ and $A[1]$ are the sets of $0$- and $1$-cells of
$B$; an element of $A[2]$ as in Fig~\ref{fig:simplices}(b) is a $2$-cell
$\alpha_{012}: f_{02} \go f_{12}\of f_{01}$ in $B$.  (In general, an element
of $A[m]$ is a `strictly unitary colax morphism' $[m] \go B$, where $[m]$ is
regarded as a $2$-category whose only $2$-cells are identities.)  $H_1 \sub
A[1]$ is the set of $1$-cells which are equivalences, and $H_2$ is the set of
$2$-cells which are isomorphisms.  Then all the axioms for a weak 2-category
are satisfied.



\concept{Variant}

We could add to conditions~\bref{part:degen}--\bref{part:comp} on $(A,H)$ the
further condition that $H$ is maximal with respect
to~\bref{part:degen}--\bref{part:comp}: that is, if $(A,H')$ also
satisfies~\bref{part:degen}--\bref{part:comp} and $H'_m \supseteq H_m$ for
all $m$ then $H'=H$.  (Compare the issue of maximal atlases in the definition
of smooth manifold.)  With this addition, a weak $1$-category is essentially
just a category, and a weak $2$-category a bicategory.  This is in contrast
to the original \ds{St} (as analysed above), where the flexibility in the
choice of $H$ means that the correspondence between weak $1$-($2$-)categories
and (bi)categories is inexact.



\clearpage







