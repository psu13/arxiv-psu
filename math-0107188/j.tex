
\defnheading{J}		\label{p:j}



\concept{Disks}

\paragraph{Disks}

A \demph{disk} $D$ is a diagram of sets and functions
\vspace*{-3ex}
\[
\cdots\diagspace
D_m \bundleint{p_m}{u_m}{v_m} D_{m-1} 
\diagspace\cdots\diagspace
\bundleint{p_2}{u_2}{v_2} D_1
\bundleint{p_1}{u_1}{v_1} D_0 = 1
\]
equipped with a total order on the fibre $p_m^{-1}(d)$ for each $m\geq 1$
and $d\in D_{m-1}$, such that for each $m\geq 1$ and $d\in D_{m-1}$,
%
\begin{itemize}
\item $u_m(d)$ and $v_m(d)$ are respectively the least and greatest elements
of $p_m^{-1}(d)$
\item $u_m(d)=v_m(d) \iff d \in \textrm{image}(u_{m-1}) \cup
\textrm{image}(v_{m-1})$.
\end{itemize}
% 
When $m=1$, the second condition is to be interpreted as saying that $u_1\neq
v_1$ (or equivalently, that $D_1$ has at least two elements).

A \demph{map $D\goby{\psi} D'$ of disks} is a family of functions $(D_m
\goby{\psi_m} D'_m)_{m\geq 0}$ commuting with the $p$'s, $u$'s and $v$'s and
preserving the order in each fibre.  (The last condition means that if $d\in
D_{m-1}$ and $b,c\in p_m^{-1}(d)$ with $b\leq c$, then $\psi_m(b)\leq
\psi_m(c) \in p'^{-1}_m(\psi_{m-1}(d))$.)  Call $\psi$ a \demph{surjection}
if each $\psi_m$ is a surjection.

\paragraph{Interiors, Volume, Dimension}

Let $D$ be a disk.  For $m\geq 1$, define
\[
\iota D_m = D_m \backslash (\textrm{image}(u_m) \cup 
\textrm{image}(v_m)),
\]
(the \demph{interior} of $D_m$), and define $\iota D_0 = D_0$.  If the set
$\coprod_{m\geq 1} \iota D_m$ is finite then we call $D$ \demph{finite} and
define the \demph{volume} $|D|$ of $D$ to be its cardinality.  In this case
we may also define the \demph{dimension} of $D$ to be the largest $m\geq 0$
for which $\iota D_m \neq \emptyset$.

\paragraph{Finite Disks}

Write \scat{D} for a skeleton of the category of finite disks and maps
between them.  In other words, take the category of all finite disks and
choose one object in each isomorphism class; the objects of $\scat{D}$ are
all these chosen objects, and the morphisms in $\scat{D}$ are all disk maps
between them.  Thus \scat{D} is equivalent to the category of finite disks
and no two distinct objects of \scat{D} are isomorphic.


\concept{Faces and Horns, Cofaces and Cohorns}

\paragraph{Cofaces}

Let $D \in \scat{D}$.  A \demph{(covolume $1$) coface} of $D$ is a
surjection $D \goby{\phi} E$ in \scat{D} where $|E| = |D|-1$.  We call $\phi$
an \demph{inner coface} of $D$ if $\phi_m(\iota D_m) \sub \iota E_m$ for all
$m\geq 0$.


\paragraph{Cohorns}

For each $D \in \scat{D}$ and coface $D \goby{\phi} E$ of \scat{D}, define
the \demph{cohorn} 
\[
\Lambda^D_\phi: \scat{D} \go \Set
\]
by
\[
\Lambda^D_\phi(C) = \{ \psi \in \scat{D}(D,C) \such
\psi \textrm{ factors through some coface of }D
\textrm{ other than }\phi \}.
\]
That is, a map $\psi: D \go C$ is a member of $\Lambda^D_\phi(C)$ if and
only if there is a coface $(D \goby{\phi'}E') \neq (D \goby{\phi}E)$ of $D$
and a map $\chi: E' \go C$ such that
%
\begin{diagram}[height=2em]
D	&\rTo^{\phi'}	&E'			\\
	&\rdTo<\psi	&\dTo>{\chi}		\\
	&		&C			\\
\end{diagram}
%
commutes.  There is an inclusion $\Lambda^D_\phi(C) \rIncl \scat{D}(D,C)$ for
each $C$, and $\Lambda^D_\phi$ is thus a subfunctor of $\scat{D}(D,\dashbk)$.
Write $i^D_\phi: \Lambda^D_\phi \rIncl \scat{D}(D, \dashbk)$ for the
inclusion.


\paragraph{Fillers}

Let $A: \scat{D} \go \Set$, let $D\in\scat{D}$, and let $\phi$ be a coface of
$D$.  A \demph{$(D,\phi)$-cohorn in $A$} is a natural transformation
$h:\Lambda^D_\phi \go A$; if $\phi$ is an inner coface then $h$ is an
\demph{inner cohorn in $A$}.

A \demph{filler} for a $(D,\phi)$-cohorn $h$ in $A$ is a natural
transformation $\ovln{h}: \scat{D}(D,\dashbk)$ $\go A$ making the following
diagram commute:
%
\begin{diagram}[height=2em]
\Lambda^D_\phi	&\rIncl^{i^D_\phi}	&\scat{D}(D,\dashbk)		\\
		&\rdTo<h		&\dTo>{\ovln{h}}		\\
		&			&A.				\\
\end{diagram}


\concept{The Definition}


\paragraph{Weak $\omega$-Categories}

A \demph{weak $\omega$-category} is a functor $A: \scat{D} \go \Set$ such
that there exists a filler for every inner cohorn in $A$.

\paragraph{Weak $n$-Categories}

Let $n\geq 0$.  A functor $A: \scat{D} \go \Set$ is \demph{$n$-dimensional}
if, whenever $\psi: D \go E$ is a map in \scat{D} such that
%
\begin{itemize}
\item $D$ has dimension $n$ 
\item $\psi_m$ is a bijection for every $m\leq n$,
\end{itemize}
%
then $A(\psi)$ is a bijection.  A \demph{weak $n$-category} is an
$n$-dimensional weak $\omega$-category.







\clearpage



\lowdimsheading{J}

Let $n\geq 0$.  An \demph{$n$-disk} is defined in the same way as a disk,
except that $D_m$, $p_m$, $u_m$ and $v_m$ are now only defined for $m\leq n$:
so an $n$-disk is essentially the same thing as a disk of dimension $\leq n$.
Write $\scat{D}_n$ for a skeleton of the category of finite $n$-disks.  An
$n$-dimensional functor $\scat{D} \go \Set$ is determined by its effect on
disks of dimension $\leq n$, and conversely any functor $\scat{D}_n \go \Set$
extends uniquely to become an $n$-dimensional functor $\scat{D} \go \Set$.
So the category of $n$-dimensional functors $\scat{D}\go \Set$ is equivalent
to $\ftrcat{\scat{D}_n}{\Set}$.

Take an $n$-dimensional functor $A: \scat{D} \go \Set$ and its restriction
$\twid{A}: \scat{D}_n$ $\go \Set$.  Then there is automatically a unique
filler for every cohorn of dimension $\geq n+2$ in $A$ (that is, cohorn
$\Lambda^D_\phi \go A$ where $D$ has dimension $\geq n+2$).  Moreover, there
exists a filler for every inner cohorn of dimension $n+1$ in $A$ if and only
if there is at most one filler for every inner cohorn of dimension $n$ in
$\twid{A}$.  (We do not prove this, but the idea of the method is in the last
sentence of `$n=2$'.)  So: a weak $n$-category is a functor $\scat{D}_n \go
\Set$ such that every inner cohorn has a filler, unique when the cohorn is of
dimension $n$.



\concept{$n=0$}

$\scat{D}_0$ is the terminal category \One.  The unique $0$-disk has no
cofaces, so a weak $0$-category is merely a functor $\One \go \Set$, that is,
a set.



\concept{$n=1$}

An \demph{interval} is a totally ordered set with a least and a greatest
element, and is called \demph{strict} if these elements are distinct.
$\scat{D}_1$ is (a skeleton of) the category of finite strict intervals, so
we can take its objects to be the intervals $\intvl{k} = \{0, \ldots, k+1\} $
for $k\geq 0$ and its morphisms to be the interval maps.

The cofaces of \intvl{k} are the surjections $\intvl{k}\go\intvl{k-1}$
(assuming $k\geq 1$; if $k=0$ then there are none).  They are $\phi_0,
\ldots, \phi_k$, where $\phi_i$ identifies $i$ and $i+1$; of these, $\phi_1,
\ldots, \phi_{k-1}$ are inner.  The cohorn $\Lambda^{\intvl{k}}_{\phi_i}:
\scat{D}_1 \go \Set$ sends \intvl{l} to
$
\{ \psi: \intvl{k}\go\intvl{l} \such
\psi \textrm{ factors through }\phi_{i'} \textrm{ for some } 
i'\in\{0, \ldots, i-1, i+1, \ldots, k\} \}.
$

Now, let $\Del$ be a skeleton of the category of nonempty finite totally
ordered sets, with objects $[k] = \{0, \ldots, k\}$ ($k\geq 0$).  Then
$\scat{D}_1 \iso \Del^\op$, with $\intvl{k}$ corresponding to $[k]$, the
cofaces $\phi_i: \intvl{k}\go\intvl{k-1}$ to the usual face maps $[k-1] \go
[k]$, and the inner cofaces to the inner faces (i.e.\ all but the first and
last).  Trivially, cohorns $\Lambda^{\intvl{k}}_{\phi_i}$ correspond to horns
in the standard sense, and fillers to fillers.  Hence a weak $1$-category is
a functor $A: \Delop \go \Set$ in which every inner horn has a unique
filler---exactly the condition that $A$ is the nerve of a category.  So a
weak $1$-category is just a category.



\concept{$n=2$}

Again we use a duality.  Given natural numbers $l_1, \ldots, l_k$, let
$T_{l_1, \ldots, l_k}$ be the strict $2$-category generated by objects $x_0,
\ldots, x_k$, $1$-cells $p_i^j: x_{i-1} \go x_i$ ($1\leq i\leq k$, $0\leq j
\leq l_i$), and $2$-cells $\xi_i^j: p_i^{j-1} \go p_i^j$ ($1\leq i\leq k$,
$1\leq j \leq l_i$).  (E.g.\ the lower half of Fig.~\ref{fig:disks}(a) shows
$T_{1,0}$.)  Let $\Delta_2$ be the category whose objects are sequences
$(l_1, \ldots, l_k)$ with $k,l_i\geq 0$, and whose maps $(l_1, \ldots, l_k)
\go (l'_1, \ldots, l'_{k'})$ are the strict $2$-functors $T_{l_1, \ldots,
l_k} \go T_{l'_1, \ldots, l'_{k'}}$.  Then $\scat{D}_2 \iso \Delta_2^\op$.
On objects, this says that a finite 2-disk is just a finite sequence of
numbers, e.g.\ $(1,0)$ in Fig.~\ref{fig:disks}(a).
%
\begin{figure}
\[
\begin{array}{ccccc}
% Upper row
% (a)
\begin{tree}
\enode		&\enode		&\lanode{w}	&\enode		&	&
\enode		&		&\enode		&\enode		\\
\dn		&		&\rt{1}\dn\lt{1}&		&	&
		&\rt{1}\lt{1}	&		&\dn		\\
\enode		&		&\lnode{u}	&		&	&
		&\lnode{v}	&		&\enode		\\
		&\rt{4}		&		&\rt{2}		&	&
\lt{2}		&		&\lt{4}		&		\\
		&		&		&		&\node	&
		&		&		&		\\
\end{tree}
\ &
% (b)
\begin{tree}
\enode		&\enode		&		&\laenode{w}	&	&
\enode		&		&\enode		&\enode		\\
\dn		&		&\rt{1}\lt{1}	&		&	&
		&\rt{1}\lt{1}	&		&\dn		\\
\enode		&		&\lnode{u}	&		&	&
		&\lnode{v}	&		&\enode		\\
		&\rt{4}		&		&\rt{2}		&	&
\lt{2}		&		&\lt{4}		&		\\
		&		&		&		&\node	&
		&		&		&		\\
\end{tree}
\ &
% (c)
\begin{tree}
\enode		&\laenode{w}	&		&\enode		&	&
\enode		&		&\enode		&\enode		\\
\dn		&		&\rt{1}\lt{1}	&		&	&
		&\rt{1}\lt{1}	&		&\dn		\\
\enode		&		&\lnode{u}	&		&	&
		&\lnode{v}	&		&\enode		\\
		&\rt{4}		&		&\rt{2}		&	&
\lt{2}		&		&\lt{4}		&		\\
		&		&		&		&\node	&
		&		&		&		\\
\end{tree}
\ &
% (d)
\begin{tree}
\enode		&\enode		&\lanode{w}	&\enode		&\enode	\\
\dn		&		&\rt{1}\dn\lt{1}&		&\dn	\\
\enode		&		&\lnode{u}	&		&\lenode{v}\\
		&\rt{2}		&\dn		&\lt{2}		&	\\
		&		&\node		&		&	\\
\end{tree}
\ &
% (e)
\begin{tree}
\enode		&\enode		&\lanode{w}	&\enode		&\enode	\\
\dn		&		&\rt{1}\dn\lt{1}&		&\dn	\\
\enode		&		&\lnode{\!\!\!\!\!\!\!u\ v}&	&\enode	\\
		&\rt{2}		&\dn		&\lt{2}		&	\\
		&		&\node		&		&	\\
\end{tree}
\\
% Lower row
% (a)
\gzersu\gtwosu\gzersu\gonesu\gzersu	
&	
% (b)
\gzersu\gunhappysu\gzersu\gonesu\gzersu
&	
% (c)
\gzersu\ghappysu\gzersu\gonesu\gzersu
&
% (d)
\gzersu\gtwosu\gzersu\ghole
&
% (e)
\gzersu\gtwowidesu\gzersu
\\
\textrm{(a)}	&\textrm{(b)}	&\textrm{(c)}	&
\textrm{(d)}	&\textrm{(e)}	
\end{array}
\]
\caption{The duality.  In the upper row, $\node$ denotes an interior element
and $\enode$ an endpoint of a fibre, and the labels $u, v, w$ show what the
coface maps `$\phi$' do}
\label{fig:disks}
\end{figure}

Any bicategory $B$ has a `nerve' $A: \Delta_2^\op \go \Set$, where $A(l_1,
\ldots, l_k) = \{$weak functors $T_{l_1, \ldots, l_k} \go B$ strictly
preserving identities$\}$.  We can recover $B$ from $A$, so weak
$2$-categories are the same as bicategories just as long as the definition
gives the right conditions on functors $\scat{D}_2 \iso \Delta_2^\op \go
\Set$.  I do not have a full proof that this is so, hence there are gaps in
what follows.

Defining \demph{faces} of an object of $\Delta_2$ as cofaces of the
corresponding object of $\scat{D}_2$, and similarly \demph{horns}, a weak
$2$-category is a functor $\Delta_2^\op \go \Set$ in which every 1-
(respectively, 2-) dimensional horn has a filler (respectively, unique
filler).  Faces are certain subcategories: e.g.\ Fig.~\ref{fig:disks}(b)--(e)
shows the 4 cofaces of a disk and correspondingly the 4 faces of $T_{1,0}$,
of which only (e) is inner.

For the converse of the nerve construction, we take a weak $2$-category $A$
and form a bicategory $B$.  Its graph $B_2 \parpairu B_1 \parpairu B_0$ is
the image under $A$ of the diagram $\gfstsu\gtwosu\glstsu \pile{\lIncl \\
\lIncl} \gfstsu\gonesu\glstsu \pile{\lIncl \\ \lIncl} \gzeros{}$ in
$\Delta_2$.  A diagram $\gfsts{a}\gones{f}\gblws{b}\gones{g}\glsts{c}$ in $B$
is a horn in $A$ for the unique inner face of $T_{0,0} =
\gfstsu\gonesu\gzersu\gonesu\glstsu$; choose a filler $K_{f,g}$ and write
$g\of f$ for its third face.  This gives 1-cell composition; vertical 2-cell
composition works similarly but without choice.  Next, a diagram
$\gfsts{a}\gtwos{f}{f'}{\alpha}\gfbws{b}\gones{g}\glsts{c}$, with $K_{f,g}$
and $K_{f',g}$, forms a horn for the unique inner face of $T_{1,0}$
(Fig.~\ref{fig:disks}), so has a unique filler $K_{\alpha,g}$; write $g\of
\alpha : g\of f \go g\of f'$ for face~(e) of $K_{\alpha,g}$.  Horizontal
2-cell composition is defined via this construction, its dual, and vertical
composition.  Next, $\gfsts{a} \gones{f} \gblws{b} \gones{g} \gblws{c}
\gones{h} \glsts{d}$, together with $K_{f,g}, K_{g,h}, K_{g\sof f,h}$, gives
an inner horn for $T_{0,0,0}$.  There's a (unique?)  filler, whose final face
$L_{f,g,h}$ is itself a filler of $\gfsts{a} \gones{f} \gblws{b} \gones{h\sof
g} \glsts{d}$ with third face $h\of (g\of f)$.  Considering
$\gfsts{a}\gtwos{f}{f}{1} \gfbws{b} \gones{h\sof g}\glsts{d}$ with
$K_{f,h\sof g}$ and $L_{f,g,h}$ gives an invertible 2-cell $(h\of g)\of f \go
h\of (g\of f)$.














