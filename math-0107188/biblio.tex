
\section*{Further Reading}	\label{p:biblio}


This section contains the references and historical notes missing in the
main text.  It is not meant to be a survey of the literature.  Where I have
omitted relevant references it is almost certainly a result of my own
ignorance, and I hope that the authors will forgive me.

First are some references to introductory and general material, and a very
brief account of the history of higher-dimensional category theory.  Then
there are references for each of the sections in turn: `Background',
followed by the ten definitions.  Finally there are references to some
proposed definitions of $n$-category which I didn't include, and a very few
references to areas of mathematics related to $n$-categories. 

Citations such as \url{math\dt CT\slsh 9810058} and \url{alg-geom\slsh
9708010} refer to the electronic mathematics archive at
\url{http:\dblslsh arXiv\dt org}.
Readers unfamiliar with the archive may find it easiest to go straight to the
address of the form \url{http:\dblslsh arXiv\dt org\slsh abs\slsh math\dt
CT\slsh 9810058} .


\subsection*{Introductory Texts}

Introductions to $n$-categories come slanted towards various different
audiences.  One for theoretical computer scientists and logicians is 
% 
\bibentry{Pow}{%
A. J. Power, 
Why tricategories?, 
\jnl{Information and Computation}{120}{1995}{no.~2, 251--262}; 
also 
LFCS report ECS-LFCS-94-289, April 1994, 
\web{http:\dblslsh www\dt lfcs\dt informatics\dt ed\dt ac\dt uk}}%
% 
and another with a logical slant, but this time with foundational concerns,
is 
% 
\bibentry{MakTCF}{%
M. Makkai,
Towards a categorical foundation of mathematics,
\contrib{Logic Colloquium '95 (Haifa)}
{Lecture Notes in Logic~\textbf{11}}
{Springer}{1998}{153--190}.}%
%

Moving to introductions for those more interested in topology, geometry and
physics, one which starts at a very basic level (the definition of category)
is 
%
\bibentry{BaezTNC}{%
John C. Baez, 
A tale of $n$-categories,
\web{http:\dblslsh math\dt ucr\dt edu\slsh home\slsh baez\slsh week73\dt
html}, 1996--97.}% 
%
With similar themes but at a more advanced level, there are
% 
\bibentry{BaezINC}{%
John C. Baez,
An introduction to $n$-categories,
\contrib{Category theory and computer science (Santa Margherita Ligure,
1997)}
{Lecture Notes in Computer Science}{1290}
{Springer}{1997}{1--33};
also
\epr{q-alg\slsh 9705009}{1997}}%
%
(especially sections~1--3) and
%
\bibentry{THDCTRI}{%
Tom Leinster, 
Topology and higher-dimensional category theory: the rough idea,
\eprint{math\dt CT\slsh 0106240}{2001}{15}.}%
%
The ambitious might, if they can find a copy, like to look at the highly
discursive 600-page letter of Grothendieck to Quillen,
% 
\bibentry{Gro}{%
A. Grothendieck, 
Pursuing stacks,
manuscript, 1983,}%
% 
in which (amongst many other things) the idea is put that tame topology is
really the study of weak $\omega$-groupoids.  A more accessible discussion of
what higher-dimensional algebra might `do', especially in the context of
topology, is
%
\bibentry{Bro}{%
Ronald Brown,
Higher dimensional group theory,
\web{http:\dblslsh www\dt bangor\dt ac\dt uk\slsh $\sim$mas010}.}%



\subsection*{General Comments and History}


The easiest way to begin a history of $n$-categories is as follows.

\begin{sloppypar}
$0$-categories---sets or classes---came into the mathematical consciousness
around the end of the 19th century.  $1$-categories---categories---arrived in
the middle of the 20th century.  Strict $2$-categories and, implicitly,
strict $n$-categories, made their presence felt around the late 1950s and
early 1960s, with the work of Ehresmann \cyte{Ehr}.  Weak $2$-categories were
first introduced by B\'enabou \cyte{Ben} in 1967, under the name of
bicategories, and thereafter the question was in the air: `what might a weak
$n$-category be?'  The first precise proposal for a definition was given by
Street \cyte{StrAOS} in 1987.  This was followed by three more proposals
around 1995: Baez and Dolan's \cyte{BDHDA3}, Batanin's \cyte{BatMGC}, and
Tamsamani's \cyte{TamNNC}.  A constant stream of further proposed definitions
has issued forth since then, and will doubtless continue for a while.
Work on low values of $n$ was also going on at the same time: an
axiomatic definition of weak $3$-category was proposed in
% 
\bibentry{GPS}{%
R. Gordon, A. J. Power, Ross Street, 
\emph{Coherence for Tricategories}, 
Memoirs of the American Mathematical Society~\textbf{117}, no.~558,
1995,}%
% 
and a proposal in similar vein for $n=4$ was made in
% 
\bibentry{TriDT}{%
Todd Trimble, 
The definition of tetracategory, 
manuscript, 1995.}%
% 
Crucially, it was shown in~\cyte{GPS} that not every tricategory is
equivalent to a strict $3$-category (in contrast to the situation for $n=2$),
from which it follows that the theory of weak $n$-categories is genuinely
different from that of strict ones. 
\end{sloppypar}

But this is far too simplistic.  A realistic history must take account of
categorical structures other than $n$-categories \emph{per se}: for instance,
the various kinds of monoidal category (plain, symmetric, braided,
tortile/ribbon, \ldots), and of monoidal 2-categories and monoidal
bicategories.  The direct importance of these is that a monoidal category is
a bicategory with only one $0$-cell, and similarly a braided monoidal
category is a tricategory with only one $0$-cell and one $1$-cell.  The basic
reference for braided monoidal categories is
% 
\bibentry{JS}{%
Andr\'e Joyal, Ross Street,
Braided tensor categories, 
\jnl{Advances in Mathematics}{102}{1993}{no. 1, 20--78},}%
% 
and they can also be found in the new edition of Mac Lane's book
\cyte{MacCWM}.

Moreover, around the same time as the theory of $n$-categories was starting
to develop, another theory was emerging with which it was later to converge:
the theory of multicategories and operads.  Multicategories first appeared in
% 
\bibentry{Lam}{%
Joachim Lambek, 
Deductive systems and categories II: standard constructions
and closed categories,
\contribau{Category Theory, Homology Theory and their Applications, I
(Battelle Institute Conference, Seattle, 1968, Vol. One)}
{ed.\ Hilton}{Lecture Notes in Mathematics~\textbf{86}}
{Springer}{1969}{76--122}.}%
% 
(The definition is on page~103.)  A multicategory is like a category, but
each arrow has as its source or input a \emph{sequence} of objects (and, as
usual, as its target or output a single object).  An operad is basically just
a multicategory with only one object.  For this reason, multicategories are
sometimes called `coloured operads', and the objects are then named after
colours (black, white, etc.).  The development of operads is generally
attributed to Boardman, Vogt and May:
% 
\bibentry{BV}{%
J. M. Boardman, R. M. Vogt, 
\emph{Homotopy Invariant Algebraic Structures on
Topological Spaces},
Lecture Notes in Mathematics~\textbf{347},
Springer, 1973,}%
% 
% 
\bibentry{MayGIL}{%
J. P. May, 
\emph{The Geometry of Iterated Loop Spaces},
Lectures Notes in Mathematics~\textbf{271}, Springer, 1972,}%
% 
although I am told that essentially the same idea was the subject of 
% 
\bibentry{Laz}{%
Michel Lazard,
Lois de groupes et analyseurs,
\jnl{Annales Scientifiques de l'\'Ecole Normale Sup\'erieure (3)}
{72}{1955}{299--400}}%
% 
(where operads go by the name of `analyseurs').

It seems to have taken a long time before it was realized that operads and
multicategories were so closely related; I do not know of any pre-1995 text
which mentions both Lambek and Boardman-Vogt or May in its bibliography.
This can perhaps be explained by the different fields in which they were
being studied: multicategories were introduced in the context of logic and
found application in linguistics, whereas operads were used for the theory of
loop spaces.  Moreover, if one uses the terms in their original senses then
it is not strictly true that an operad is the same thing as a one-object
multicategory; operads are also equipped with a symmetric structure, and the
`hom-sets' (sets of operations) are topological spaces rather than just sets.
(It is also very natural to consider multicategories with both these pieces
of extra structure, but historically this is beside the point.)

Many short introductions to operads have appeared as section~1 of papers by
topologists and quantum algebraists.  The interested reader may also find
useful the following texts dedicated to the subject:
% 
\bibentry{MayDOA}{%
J. P. May, 
Definitions: operads, algebras and modules,
\contrib{Operads: Proceedings of Renaissance Conferences (Hartford,
CT\slsh Luminy, 1995)} 
{Contemporary Mathematics~\textbf{202}}
{AMS}{1997}{1--7};
also
\web{http:\dblslsh www\dt math\dt uchicago\dt edu\slsh $\sim$may},}%
% 
% 
\bibentry{MayOAM}{%
J. P. May, 
Operads, algebras and modules,
\contrib{Operads: Proceedings of Renaissance Conferences (Hartford,
CT\slsh Luminy, 1995)} 
{Contemporary Mathematics~\textbf{202}}
{AMS}{1997}{15--31};
also
\web{http:\dblslsh www\dt math\dt uchicago\dt edu\slsh $\sim$may},}%
% 
% 
\bibentry{MSS}{%
Martin Markl, Steve Shnider, Jim Stasheff, 
\emph{Operads in Algebra, Topology and Physics}, 
book in preparation.}%
%

A glimpse of the role of operads and multicategories in higher-dimensional
category theory can be seen in the definitions of weak $n$-category above.
Often the `operads' and `multicategories' used are not the original kinds,
but more general kinds adapted for the different shapes and dimensions which
occur in the subject; for more references, see `Definitions \ds{B} and
\ds{L}' below. 



\subsection*{Background}

\paragraph*{Category Theory}

Almost any book on the subject will provide the necessary %category-theoretic
background.  The second edition of the classic book by Mac Lane,
% 
\bibentry{MacCWM}{%
Saunders Mac Lane,
\emph{Categories for the Working Mathematician},
second edition, Graduate Texts in Mathematics~\textbf{5}, Springer, 1998,}%
% 
is especially useful, containing as it does two new chapters on such
topics as bicategories and nerves of categories.  

\paragraph*{Strict $n$-Categories}

I do not know of any good text introducing strict $n$-categories.
Ehresmann's original book
% 
\bibentry{Ehr}{%
Charles Ehresmann, 
\emph{Cat\'egories et Structures},
Dunod, Paris, 1965}%
% 
could be consulted, but is generally regarded as a very demanding read.  
Probably more useful is 
%
\bibentry{KS}{%
G. M. Kelly, Ross Street, 
Review of the elements of $2$-categories, 
\contrib{Category Seminar (Sydney, 1972\slsh 1973)}
{Lecture Notes in Mathematics~\textbf{420}}
{Springer}{1974}{75--103},}%
%
which only covers strict $2$-categories (traditionally just called
`2-categories') but should give a good idea of strict $n$-categories for
general $n$.  This could usefully be supplemented by
%
\bibentry{EK}{%
Samuel Eilenberg, G. Max Kelly, 
Closed categories,
\contb{Proceedings of Conference on Categorical Algebra (La Jolla,
California, 1965)}
{Springer}{1966}{421--562}}%
%
(see e.g.\ page~552), which also covers enrichment.  For another reference on
enriched categories, see chapter~6 of
%
\bibentry{Borx2}{%
Francis Borceux, 
\emph{Handbook of Categorical Algebra 2: Categories and
Structures},
Encyclopedia of Mathematics and its Applications~\textbf{51},
Cambridge University Press, 1994.}%

\paragraph*{Bicategories}

Bicategories were first explained by B\'enabou:
%
\bibentry{Ben}{%
Jean B\'enabou, 
Introduction to bicategories,
\contribau{Reports of the Midwest Category Seminar}
{ed.\ B\'enabou et al}
{Lecture Notes in Mathematics~\textbf{47}}{Springer}{1967}{1--77},}%
%
and further important work on them is in
% 
\bibentry{Gray}{%
John W. Gray,
\emph{Formal Category Theory: Adjointness for 2-Categories}, 
Lecture Notes in Mathematics~\textbf{391}, Springer, 1974.}%
%
(At least) two texts contain summaries of the `basic theory' of bicategories:
that is, the definitions of bicategory and of weak functor (homomorphism),
transformation and modification between bicategories, together with the
result that any bicategory is in a suitable sense equivalent to a strict
2-category.  These are section 9 of
%
\bibentry{StrCS}{%
Ross Street,
Categorical structures, 
\contbau{Handbook of Algebra~\textbf{1}}
{ed.\ M. Hazewinkel}
{North-Holland}{1996}{529--577}}%
%
and the whole of
%
\bibentry{BB}{%
Tom Leinster,
Basic bicategories,
\eprint{math\dt CT\slsh 9810017}{1998}{11}.}%




\subsection*{Definition \ds{Tr}}

The definition was given in a talk,
% 
\bibentry{TriWFN}{%
Todd Trimble, 
What are `fundamental $n$-groupoids'?,
seminar at \mbox{DPMMS}, Cambridge, 
24 August 1999,}%
% 
and has not been written up previously.  Trimble used the term `flabby
$n$-category' rather than `weak $n$-category'.  

As the title of the talk suggests, the idea was not to develop the weakest
possible notion of $n$-category, but to provide (in his words) `a sensible
niche for discussing fundamental $n$-groupoids'.  In a world where all the
definitions have been settled, it may be that fundamental $n$-groupoids of
topological spaces have certain special features (other than the
invertibility of their cells) not shared by all weak $n$-categories.  Thus it
may be that the word `weak' is less appropriate for definition \ds{Tr} than
the other definitions.

Evidence that this is the case comes from two directions.  Firstly, the maps
$\gamma_{a_0, \ldots, a_k}$ describing composition of hom-$(n-1)$-categories
in an $n$-category are \emph{strict} $(n-1)$-functors.  This corresponds to
having strict interchange laws.  It therefore seems likely that a precise
analysis of $n=3$ would show that every weak $3$-category gives rise to a
tricategory (in a similar manner to $n=2$) but that not every tricategory is
triequivalent to one arising from a weak $3$-category.  Secondly, forget
\ds{Tr} for the moment and consider in naive terms what the fundamental
$n$-groupoid of a space $X$ might look like when $n\geq 3$.  $0$-cells would
be points of $X$, $1$-cells could very reasonably be maps $[0,1] \go X$, and
similarly $2$-cells could be maps $[0,1]^2 \go X$ satisfying suitable
boundary conditions.  Composition of $1$-cells could be defined by travelling
each path at double speed, in the fashion customary to homotopy theorists,
and similarly for vertical and horizontal composition of $2$-cells.  The
point now is that although none of these compositions is strictly associative
or unital, the interchange law between horizontal and vertical $2$-cell
composition \emph{is} obeyed strictly.  This provides the kind of `special
feature' of fundamental $n$-groupoids referred to above.

Prospects for comparing \ds{Tr} with \ds{B} and \ds{L} look bright: it seems
very likely that weak $n$-categories according to \ds{Tr} are just the
algebras for a certain globular $n$-operad (in the sense of \ds{B} or
\ds{L}).

Other ideas on fundamental $n$-groupoids, $n$-categories, and how they tie
together can be found in Grothendieck's letter \cyte{Gro}.  More practical
material on fundamental 1- and 2-groupoids is in
% 
\bibentry{KP}{%
K. H. Kamps, T. Porter, 
\emph{Abstract Homotopy and Simple Homotopy Theory},
World Scientific Publishing Co., 1997.}%
% 

For more on operads, see the references under `General Comments and History'
above.  The name of the operad $E$ was not only chosen to stand for
`endpoint-preserving', but also because it comes after $D$ for `disk'---the
idea being that $E$ is something like the little disks operad $D$ (crucial
in the theory of loop spaces).  A touch more precisely, $E$ seems to play the
same kind of role for paths as $D$ does for closed loops.

More about how bicategories comes from the operad of trees can be found in
Appendix~A (and Chapter~1) of my thesis, \cyte{OHDCT}, and in my \cyte{WMC}.  




\subsection*{Definition \ds{P}}

Definition \ds{P} of weak $\omega$-category is in
% 
\bibentry{Pen}{%
Jacques Penon, 
Approche polygraphique des $\infty$-categories non strictes, 
\jnl{Cahiers de Topologie et G\'eom\'etrie Diff\'erentielle}
{40}{1999}{no.~1, 31--80}.}%
% 
His chosen term for weak $\omega$-category is `prolixe', whose closest
English translation is perhaps `waffle'.  As far as I can see there is no 
actual definition of weak $n$-category or $n$-dimensional prolixe in the
paper, although he clearly has one in mind on page~48:
%
\begin{quote}
Les prolixes de dimension $\leq 2$ s'identifient exactement aux
bicat\'egories [\ldots] la preuve de ce r\'esultat sera montr\'e dans un
article ult\'erieur
\end{quote}
%
(`waffles of dimension $\leq 2$ correspond exactly to bicategories [\ldots]
the proof of this result will be given in a forthcoming paper').

Other translations: my category $\ftrcat{\scat{R}}{\Set}$ of reflexive
globular sets is his category $\infty\hyph\mathbb{G}rr$ of reflexive
$\infty$-graphs; my $s$ and $t$ are his $s$ and $b$; my strict
$\omega$-categories are his $\infty$-categories; my category \cat{Q} is
called by him $\mathbb{E}tC$, the category of \'etirements cat\'egoriques
(`categorical stretchings'); my contractions $\gamma$ are written
$[\dashbk,\dashbk]$ (with the arguments reversed: $\gamma_m(f_0,f_1) =
[f_1,f_0]$); and my adjunction $F\ladj U$ is called $\hat{\mathcal{E}} \ladj
\hat{V}$.

The word `magma' is borrowed from Bourbaki, who used it to mean a set
equipped with a binary operation.  It is a slightly inaccurate borrowing, in
that $\omega$-magmas are equipped with (nominal) identities as well as binary
compositions; put another way, it would have been more suitable if Bourbaki
had used the word to mean a set equipped with a binary operation and a
distinguished basepoint.

It seems plausible that Penon's construction can be generalized to provide
weak versions of structures other than $\omega$-categories (e.g.\
up-to-homotopy topological monoids).  Batanin has done something
precise along the lines of generalizing Penon's definition and comparing it
to his own:
% 
\bibentry{BatPMW}{%
M. A. Batanin,  
On the Penon method of weakening algebraic structures, 
to appear in \emph{Journal of Pure and Applied Algebra};
also
\webprint{http:\dblslsh www\dt math\dt mq\dt edu\dt au\slsh
$\sim$mbatanin\slsh papers\dt html}{2001}{25}.}% 
%  




\subsection*{Definitions \ds{B} and \ds{L}}

Batanin gave his definition, together with an examination of $n=2$, in 
% 
\bibentry{BatMGC}{%
M. A. Batanin,  
Monoidal globular categories as a natural environment for the theory of weak
$n$-categories, 
\jnl{Advances in Mathematics}{136}{1998}{no.~1, 39--103};
also 
\web{http:\dblslsh www\dt math\dt mq\dt edu\dt au\slsh $\sim$mbatanin\slsh
papers\dt html}.}% 
% 
Another account of it is
% 
\bibentry{StrRMB}{%
Ross Street,
The role of Michael Batanin's monoidal globular categories,
\contrib{Higher category theory (Evanston, IL, 1997)}
{Contemporary Mathematics~\textbf{230}}
{AMS}{1998}{99--116};
also
\web{http:\dblslsh www\dt math\dt mq\dt edu\dt au\slsh $\sim$street}.}%
% 
The definition of weak $n$-category which appears as~8.7 in \cyte{BatMGC} is
(I believe) what is here called definition \ds{B1}.  More precisely, let
\cat{O} be the category whose objects are (globular) operads on which
\emph{there exist} a contraction and a system of compositions, and whose maps
are just maps of operads.  What Batanin does is to construct an operad $K$
which is weakly initial in \cat{O}.  `Weakly initial' means that there is at
least one map from $K$ to any other object of \cat{O}, so this does not
determine $K$ up to isomorphism; one needs some further information.  But in
Remark~2 just before Definition~8.6, Batanin suggests that, once given the
appropriate extra structure, $K$ is initial in the category \fcat{OCS} of
operads \emph{equipped with} a contraction and a system of compositions,
which does determine $K$.  This is the approach taken in \ds{B1}.

A weak $n$-category according to \ds{B2} is (I believe) almost exactly what
Batanin calls a `weak $n$-categorical object in \emph{Span}' in his
Definition~8.6.  The only difference is my extra condition that the operad
$C$ is (in his terminology) \demph{normalized}: $C(0) \iso 1$.  Now $C(0)$ is
the set of operations in the operad which take a $0$-cell of an algebra and
turn it into another $0$-cell, so normality means that there are no such
operations except, trivially, the identity.  This seems reasonable in the
context of $n$-categories, since one expects to have operations for composing
$m$-cells only when $m\geq 1$.  The lack of normality in Batanin's version
ought to be harmless, since the contraction means that all the operations on
$0$-cells are in some sense equivalent to the identity operation, but it does
make the analysis of $n\leq 2$ a good deal messier.  (Note that the operad
$K$ is normalized, so any weak $\omega$-/$n$-category in the sense of \ds{B1}
is also one in the sense of \ds{B2}; the same goes for \ds{L1} and \ds{L2}.) 

My modification \ds{L1} of Batanin's definition first appeared in
% 
\bibentry{SHDCT}{%
Tom Leinster,
Structures in higher-dimensional category theory,
\webprint{http:\dblslsh www\dt dpmms\dt cam\dt ac\dt uk\slsh
$\sim$leinster}{1998}{80},}%
% 
but a more comprehensive and, I think, comprehensible account is in
% 
\bibentry{OHDCT}{%
Tom Leinster,
Operads in higher-dimensional category theory,
Ph.D. thesis, University of Cambridge, 2000;
also
\eprint{math\dt CT\slsh 0011106}{2000}{viii + 127}.}%
% 
(There, $C\otimes C'$ is called $C\of C'$, $C\cdot \dashbk$ is $T_C$, and
$P_C(\pi)$ is $P_\pi(C)$.)  \cyte{OHDCT} also contains a precise analysis of
\ds{L1} for $n\leq 2$, including proofs of~(a) the equivalence of the two
different categories of weak $n$-categories (for \emph{finite} $n$)
mentioned at the start of the analysis of $n\leq 2$ above, and~(b) the
equivalence of the category of unbiased bicategories and weak functors with
that of (classical) bicategories and weak functors.  Definition \ds{L2} has not
appeared before, and has just been added here for symmetry. 

The globular operads in \ds{B} and \ds{L} are called `$\omega$-operads in
\emph{Span}' by Batanin in \cyte{BatMGC}.  They are a special case of
\demph{generalized operads}, a family of higher-dimensional categorical
structures which are perhaps as interesting and applicable as $n$-categories
themselves.  Briefly, the theory goes as follows.  Given a monad $T$ on a
category $\cat{E}$, satisfying some natural conditions, one can define a
category of \demph{$T$-multicategories}.  For example, when $T$ is the
identity monad on the category \cat{E} of sets, a $T$-multicategory is just a
category, and when $T$ is the free-monoid monad on \Set, a $T$-multicategory
is just an ordinary multicategory (see `General Comments and History' above).
A \demph{$T$-operad} is a one-object $T$-multicategory, so in the first of
these examples it is a monoid and in the second it is an operad in the
original sense (but without symmetric or topological structure).  Now take
$T$ to be the free strict $\omega$-category monad on the category \cat{E} of
globular sets, as in \ds{B} and \ds{L}: a $T$-operad is then exactly a
globular operad.  Algebras for $T$-multicategories can be defined in the
general context, and again this notion specializes to the one in \ds{B} and
\ds{L}.

Generalized (operads and) multicategories were first put forward in
% 
\bibentry{Bur}{%
Albert Burroni,  
$T$-cat\'egories (cat\'egories dans un triple),
\jnl{Cahiers de Topologie et G\'eom\'etrie
Diff\'erentielle}{12}{1971}{215--321}}%
% 
and were twice rediscovered independently:
% 
\bibentry{HerRM}{%
Claudio Hermida,  
Representable multicategories,
\jnl{Advances in Mathematics}{151}{2000}{no.~2, 164--225};
also 
\web{http:\dblslsh www\dt cs\dt math\dt ist\dt utl\dt pt\slsh s84\dt www\slsh
cs\slsh claudio\dt html},}%
% 
% 
\bibentry{GOM}{%
Tom Leinster,
General operads and multicategories,
\eprint{math\dt CT\slsh 9810053}{1997}{35}.}%
% 
As far as I know, the notion of algebra for a $T$-multicategory only appears
in the third of these.  (\cyte{GOM} also appears, more or less, as Chapter~I
of \cyte{SHDCT} and Chapter~2 of \cyte{OHDCT}.) 

The difference between definitions \ds{L1} and \ds{B1} can be summarized by
saying that \ds{L1} takes \ds{B1}, dispenses with the notions of system of
compositions and \ds{B}-style contraction, and merges them into a single more
powerful notion of contraction.  A few more words on the difference are in
section~4.5 of \cyte{OHDCT}.  The operad $L$ canonically carries a
\ds{B}-style contraction and a system of compositions, so there is a
canonical map $K \go L$ of operads, and this induces a functor in the
opposite direction on the categories of algebras.  Hence every weak
$\omega$-/$n$-category in the sense of \ds{L1} gives rise canonically to one
in the sense of \ds{B1}.





\subsection*{Definition \lp}

This is the first time in print for definition \lp.  Once we have the
language of generalized multicategories (described in the previous section)
and the theory of free strict $\omega$-categories, it is very quickly stated.
My papers \cyte{OHDCT} and \cyte{SHDCT} (and to some extent \cyte{GOM}) cover
generalized multicategories and globular operads, but not specifically
globular multicategories.  The $1$-dimensional case, $1$-globular
multicategories, are the `$\mathbf{fc}$-multicategories' described briefly in
% 
\bibentry{FCM}{%
Tom Leinster,
$\mathbf{fc}$-multicategories,
\eprint{math\dt CT\slsh 9903004}{1999}{8},}%
% 
at a little more length in
% 
\bibentry{GEC}{%
Tom Leinster,
Generalized enrichment of categories,
to appear in \emph{Journal of Pure and Applied Algebra},}%
% 
and in detail in
% 
\bibentry{GECM}{%
Tom Leinster,
Generalized enrichment for categories and multicategories,
\eprint{math\dt CT\slsh 9901139}{1999}{79}.}%
% 

Logicians might like to view \lp\ through proof-theoretic spectacles,
substituting the word `proof' for `reason'.  They (and others) might also be
interested to read
% 
\bibentry{MakAAC}{%
M. Makkai,
Avoiding the axiom of choice in general category theory,
\jnl{Journal of Pure and Applied Algebra}{108}{1996}{no.~2, 109--173};
also 
\web{http:\dblslsh www\dt math\dt mcgill\dt ca\slsh makkai}}%
% 
in which Makkai defines anafunctors and anabicategories and discusses the
philosophical viewpoint which led him to them.  In the same vein, see also
Makkai's \cyte{MakTCF} and the remarks on `a composite' \emph{vs.}\ `the
composite' towards the end of the Introduction to the present paper. 

A weak $\omega$-/$n$-category in the sense of \ds{L2} (and so \ds{L1} too)
gives rise to one in the sense of \lp.  For just as `algebras' for a category
$C$ (functors $C \go \Set$) correspond one-to-one with discrete opfibrations
over $C$, via the so-called Gro\-th\-en\-dieck construction, so the same is
true in a suitable sense for globular multicategories.  This generalization
is explained in section~4.2 of \cyte{GOM}, section~I.3 of \cyte{SHDCT}, and
section~3.4 of \cyte{OHDCT} (any one of which will do, but they are listed in
increasing order of clarity).  What this means is that an algebra for a
globular operad gives rise to a globular multicategory (the domain of the
opfibration), and if the operad admits a contraction in the sense of \ds{L}
then the resulting multicategory is a weak $\omega$-category in the sense of
\lp.

Midway between \lp\ and \ds{J} is another possible definition of weak
$\omega$-category, which for various reasons I have not included here.  It
was presented in a talk,
% 
\bibentry{LeiNQJ}{%
Tom Leinster,
Not quite Joyal's definition of $n$-category (a.k.a.~`algebraic nerves'),
seminar at \mbox{DPMMS}, Cambridge,
22 February 2001,}%
% 
notes from which, in the $(2+2)$-page format of this paper, are available on
request.  The idea behind it can be traced back to Segal's formalization of
the notion of up-to-homotopy topological commutative monoid, \demph{special
$\Gamma$-spaces} (and their non-commutative counterparts, \demph{special
$\Delta$-spaces}).  The analogy is that just as Segal took the theory of
honest topological commutative monoids and did something to it to obtain an
up-to-homotopy version, so we take the theory of strict $n$-categories and do
something similar to obtain a weak version.  Segal's original paper is
% 
\bibentry{SegCCT}{%
Graeme Segal, 
Categories and cohomology theories,
\jnl{Topology}{13}{1974}{293--312}.}%
% 
A different generalization of his idea defines up-to-homotopy algebras for
any (classical) operad.  This is done at length in my
paper~\cyte{HAO}; or a much briefer explanation of the idea is
% 
\bibentry{UTHM}{%
Tom Leinster,
Up-to-homotopy monoids,
\eprint{math\dt QA\slsh 9912084}{1999}{8}.}%
% 




\subsection*{Definitions \ds{Si} and \ds{Ta}}

Tamsamani's original definition appeared in
% 
\bibentry{TamNNC}{%
Zouhair Tamsamani,
Sur des notions de $n$-cat\'egorie et $n$-groupo\"{\i}de non strictes via des
ensembles multi-simpliciaux,  
\jnl{$K$-Theory}{16}{1999}{no.~1, 51--99};  
also
\epr{alg-geom\slsh 9512006}{1995}.}%
% 
What I have called truncatability of a functor $\Delnop{r} \go \Set$ is
called `$r$-troncabilit\'e' by Tamsamani.  It is not immediately obvious that
the two conditions are equivalent, but a thoroughly mundane induction shows
that they are.  Other translations: my $1^p$ is his $I_p$, my $s$ and $t$ are
his $s$ and $b$, my $Q^{(m)}$ is his $T^m$, my $\pi^{(m)}$ is his $T^m$, my
internal equivalence of cells $x_1, x_2$ (as in the text of \ds{Ta}) is his
$(r-p)$-\'equivalence int\'erieure, and my external equivalence of functors
$\Delnop{r} \go \Set$ is his $r$-\'equivalence ext\'erieure.  His term for a
weak $n$-category is `$n$-nerf' or `$n$-cat\'egorie large'.  (`Large' has
nothing to do with large and small categories: it means broad or generous,
and can perhaps be translated here as `lax'; compare the English word
`largesse'.)

Tamsamani also offers a proof that his weak $2$-categories are essentially
the same as bicategories, but I believe that it is slightly flawed, in that
he has omitted a necessary axiom for the $2$-nerve of a bicategory (the last
bulleted item in `Definition \ds{Ta} for $n\leq 2$', starting
`$\iota_{uwx}^z$').  Without this, the constructed functor $\Delnop{2} \go
\Set$ will not necessarily be a weak $2$-category in the sense of \ds{Ta}.
(In this context, my $a$'s are his $x$'s, my $\alpha$'s are his $\lambda$'s,
and my $\iota$'s are his $\epsln$'s.)

Working with him in Toulouse, Simpson produced a simplified version of
Tamsamani's definition, which first appeared in
% 
\bibentry{SimCMS}{%
Carlos Simpson,
A closed model structure for $n$-categories, internal
\textit{\underline{Hom}}, $n$-stacks and generalized Seifert-Van Kampen,  
\eprint{alg-geom\slsh 9704006}{1997}{69}.}%
% 
He used the term `easy $n$-category' for his weak $n$-categories, and `easy
equivalence' for what is called a contractible map in \ds{Si}.

The simplification lies in the treatment of equivalences.  Weak
$1$-categories according to either \ds{Ta} or \ds{Si} are just categories,
but whereas a \ds{Ta}-style equivalence of weak $1$-categories is a functor
which is full, faithful and essentially surjective on objects (that is, an
ordinary equivalence of categories), an easy equivalence is a functor which
is full, faithful and \emph{genuinely} surjective on objects.  The latter
property of functors is expressible at a significantly more primitive
conceptual level than the former, since it is purely in terms of the
underlying directed graphs and has nothing to do with the actual category
structure.  For this reason, \ds{Si} is much shorter than \ds{Ta}.  (But to
develop the theory of weak $n$-categories we still need Tamsamani's more
general notion of equivalence; this is, for instance, the missing piece of
vocabulary referred to at the very end of `Definition \ds{Si} for $n\leq
2$'.)

As one would expect from this description, any easy equivalence (contractible
map) is an equivalence in the sense of \ds{Ta}.  So as long as it is true
that any weak $n$-category $\Delnop{n} \go \Set$ in the sense of \ds{Si} is
truncatable (which I cannot claim to have proved), it follows that any weak
$n$-category in the sense of \ds{Si} is also one in the sense of \ds{Ta}.

Following on from his definition, Tamsamani investigated homotopy
$n$-gr\-ou\-poids of spaces:
% 
\bibentry{TamETH}{%
Zouhair Tamsamani,
Equivalence de la th\'eorie homotopique des $n$-groupo\"{\i}des
et celle des espaces topologiques $n$-tronqu\'es,
\eprint{alg-geom\slsh 9607010}{1996}{24}.}%
% 
Numerous papers by Simpson, using a mixture of his definition and Tamsamani's
and largely in the language of Quillen model categories, push the theory of
weak $n$-categories further along:
% 
\bibentry{SimLNC}{%
Carlos Simpson,
Limits in $n$-categories,
\eprint{alg-geom\slsh 9708010}{1997}{92},}%
% 
% 
\bibentry{SimHTS}{%
Carlos Simpson,
Homotopy types of strict 3-groupoids,
\eprint{math\dt CT\slsh 9810059}{1998}{29},}%
% 
% 
\bibentry{SimBBD}{%
Carlos Simpson,
On the Breen-Baez-Dolan stabilization hypothesis for Tamsamani's weak
$n$-categories,  
\eprint{math\dt CT\slsh 9810058}{1998}{36},}%
% 
% 
\bibentry{SimCMB}{%
Carlos Simpson,
Calculating maps between $n$-categories,
\eprint{math\dt CT\slsh 0009107}{2000}{13}.}%
% 
Toen has also applied Tamsamani's definition, as in
% 
\bibentry{ToenDTS1}{%
B. Toen,
Dualit\'e de Tannaka sup\'erieure I: structures monoidales,
Max-Planck-Institut preprint MPI-2000-57, 
\web{http:\dblslsh www\dt mpim-bonn\dt mpg\dt de},
2000, 71 pages}%
% 
and
% 
\bibentry{ToenNHC}{%
B. Toen,
Notes on higher categorical structures in topological quantum field theory,
\webprint{http:\dblslsh guests\dt mpim-bonn\dt mpg\dt de\slsh rosellen\slsh
etqft00\dt html}{2000}{14},}%
% 
and the theory finds its way into some very grown-up mathematics in
% 
\bibentry{SimAAH}{%
Carlos Simpson,
Algebraic aspects of higher nonabelian Hodge theory,
\eprint{math\dt AG\slsh 9902067}{1999}{186}.}%
% 

The connection between categories and their nerves is covered briefly in one
of the new chapters of Mac Lane's book \cyte{MacCWM}; the more-or-less
original source is
% 
\bibentry{SegCSS}{%
Graeme Segal, 
Classifying spaces and spectral sequences, 
\jnl{Institut des Hautes \'Etudes Scientifiques Publications Math\'ematiques}
{34}{1968}{105--112}.}%
% 
Presumably the `Segal maps' are so named because of the prominent role they
play in Segal's paper \cyte{SegCCT} on loop spaces and homotopy-algebraic
structures.

The basic method by which a Simpson or Tamsamani weak 2-category gives rise
to a bicategory seems implicit in Segal's \cyte{SegCCT}, is made explicit in
section~3 of my \cyte{UTHM}, and is done in even more detail in section~3.3
of
% 
\bibentry{HAO}{%
Tom Leinster,
Homotopy algebras for operads,
\eprint{math\dt QA\slsh 0002180}{2000}{101}.}%
% 
(Actually, these last two papers only describe the method for monoidal
categories rather than bicategories in general, but there is no substantial
difference.)  There is also a discussion of the converse process in
section~4.4 of \cyte{HAO}, and the idea behind this is once more implicit in
the work of Segal.




\subsection*{Definition \ds{J}}

Joyal gave his definition in an unpublished note,
% 
\bibentry{Joy}{%
A. Joyal,
Disks, duality and $\Theta$-categories,
preprint, \emph{c}.~1997, 6 pages.}%
% 
There he defined a notion of weak $\omega$-category, which he called
`$\theta$-category'.  He also wrote a few informal words about structures
called $\theta^n$-categories, and how one could derive from them a definition
of weak $n$-category; but I was unable to interpret his meaning, and
consequently definition \ds{J} of weak $n$-category might not be what
he envisaged.

The term `disk' comes from the case where, in the notation of \ds{J}, $D_m$
is the closed $m$-dimensional unit disk ($=$ ball) in $\mathbb{R}^m$, $p_m$
is projection onto the first $(m-1)$ coordinates, and the order on the fibres
is given by the usual order on the real numbers.  The second bulleted
condition in the paragraph headed `Disks' holds at a point $d$ of $D_m$ if
and only if $d$ is on the boundary of $D_m$.  From another point of view,
this condition can be regarded as a form of exactness.

The handling of faces in \ds{J} is not necessarily equivalent to that in
\cyte{Joy}; again, I had trouble understanding the intended meaning and made
my own path.  In fact, Joyal works the duality discussed under $n\leq 2$ into
the definition itself, putting $\Theta = \scat{D}^\op$ and calling $\Theta$
the category of `Batanin cells' (for reasons suggested by
Figures~\ref{fig:op-comp-b} and~\ref{fig:disks}).  So he does not speak of
cofaces and cohorns in \scat{D}, but rather of faces and horns in $\Theta$.

Of the analyses of $n\leq 2$ for the ten definitions, that for \ds{J} is
probably the furthest from complete.  It appears to be the case that in a
weak $n$-category $A: \scat{D}_n \go \Set$, any cohorn $\Lambda^D_\phi \go A$
where $D$ has volume $>n$ has a unique filler.  (We know that this is true
when the dimension of $D$ is $n$.)  If this conjecture holds then we can
complete the proof (sketched in `$n\leq 2$') that any weak $2$-category gives
rise to a bicategory; for instance, applied to $T_{0,0,0}$ it tells us that
there is a canonical choice of associativity isomorphism, and applied to
$T_{0,0,0,0}$ it gives us the pentagon axiom.  However, I have not been able
to find a proof (or counterexample).

Introductory material on simplicial sets and horns can be found in, for
instance, Kamps and Porter's book \cyte{KP}.

The duality between the skeletal category $\Del$ of nonempty finite totally
ordered sets and the skeletal category \scat{I} of finite strict intervals
has been well-known for a long time.  Nevertheless, I have been unable to
trace the original reference, or even a text where it is explained
directly---except for Joyal's preprint \cyte{Joy}, which the reader may have
trouble obtaining.  Put briefly, the duality comes from mapping into the
2-element ordered set: if $k$ is a natural number then the set
$\Del([k],[1])$ naturally has the structure of an interval (isomorphic to
$\langle k\rangle$) and the set $\scat{I}(\langle k\rangle, \langle
0\rangle)$ naturally has the structure of a totally ordered set (isomorphic
to $[k]$).  This provides functors $\Del(\dashbk,[1]): \Delop \go \scat{I}$
and $\scat{I}(\dashbk, \langle 0\rangle): \scat{I}^\op \go \Del$ which are
mutually inverse, so $\Delop \iso \scat{I}$.

The higher duality has been the subject of detailed investigation by Makkai
and Zawadowski:
% 
\bibentry{MZ}{%
Mihaly Makkai, Marek Zawadowski, 
Duality for simple $\omega$-categories and disks,
\jnl{Theory and Applications of Categories}{8}{2001}{114--243},}%
% 
% 
\bibentry{ZawDBD}{%
Marek Zawadowski,
Duality between disks and simple categories,
talk at 70th Peripatetic Seminar on Sheaves and Logic, Cambridge, 1999,}%
% 
% 
\bibentry{ZawDTD}{%
Marek Zawadowski,
A duality theorem on disks and simple $\omega$-categories, with applications to
weak higher-dimensional categories, 
talk at CT2000, Como, Italy, 2000.}%
%
(Slides and notes from Zawadowski's talks have the virtue of containing some
pictures absent in the published version.)  More on this duality and on the
relationship between definitions \ds{J} and \ds{B} is in
% 
\bibentry{Ber}{%
Clemens Berger,
A cellular nerve for higher categories,
Universit\'e de Nice---Sophia Antipolis Pr\'epublication 602 (2000), 50 pages;
also 
\web{http:\dblslsh math\dt unice\dt fr\slsh $\sim$cberger}}%
% 
(where a closed model category structure on $\ftrcat{\scat{D}}{\Set}$ is
also discussed) and in
% 
\bibentry{BS}{%
Michael Batanin, Ross Street,
The universal property of the multitude of
trees, 
\jnl{Journal of Pure and Applied Algebra}{154}{2000}{no.~1-3, 3--13};
also
\web{http:\dblslsh www\dt math\dt mq\dt edu\dt au\slsh $\sim$mbatanin\slsh
papers\dt html}.}% 
% 
As mentioned above, there is another way to define weak $n$-category which
has strong connections to both \ds{J} and \ds{L}: \cyte{LeiNQJ}.



\subsection*{Definition \ds{St}}

Street proposed his definition of weak $\omega$-category in a very tentative
manner, in the final sentence of
% 
\bibentry{StrAOS}{%
Ross Street,
The algebra of oriented simplexes,
\jnl{Journal of Pure and Applied Algebra}{49}{1987}{no.~3, 283--335}.}%
% 
He did not explicitly formulate a notion of weak $n$-category for finite
$n$; this small addition is mine, as is the Variant at the end of the
section on $n\leq 2$.  

There is one minor but material difference, and a small number of cosmetic
differences, between Street's definition and \ds{St}.  The material
difference is that in a weak $\omega$-category as proposed in \cyte{StrAOS},
the only hollow $1$-cells are the degenerate ones.  One terminological
difference is that a pair $(A,H)$ is called a `simplicial set with
hollowness' in \cyte{StrAOS} only when~\bref{part:degen} and the
aforementioned condition on $1$-cells hold: so the term has a narrower
meaning there than here.  (Street informs me that he and Verity have used the
term `stratified simplicial set' for the same purpose, either with the two
conditions or without.)  Another is that he uses `$\omega$-category' in a
wider sense: his potentially have infinite-dimensional cells, and the
category of strict $\omega$-categories in the sense of the present paper is
denoted $\omega$-Cath.  Further translations: I say that a subset $S \sub
[m]$ is $k$-alternating where Street says that the set $[m]\without S$ is
`$k$-divided', and he calles a map $[l] \go [m]$ `$k$-monic' if its image is
a $k$-alternating subset of $[m]$.

The focus of \cyte{StrAOS} is actually on \emph{strict} $n$- and
$\omega$-categories.  To this end he considers the condition on $1$-cells
mentioned above, and conditions \bref{part:degen}--\bref{part:comp} of
\ds{St} with `unique' inserted before the word `filler'
in~\bref{part:filler}.  Having spent much of the paper constructing the nerve
of a strict $\omega$-category (this being a simplicial set with hollowness),
he conjectures that a given simplicial set with hollowness is the nerve of
some strict $\omega$-category if and only if all the conditions just
mentioned hold.  (The conjecture was, I believe, a result of joint work with
John Roberts.)  The necessity of these conditions was proved soon afterwards
in
% 
\bibentry{StrFN}{%
Ross Street,
Fillers for nerves,
\contrib{Categorical algebra and its applications (Louvain-La-Neuve, 1987)}
{Lecture Notes in Mathematics~\textbf{1348}}
{Springer}{1988}{337--341}.}%
% 
A proof of their sufficiency was supplied by Dominic Verity; this has not
appeared in print, but was presented at various seminars in Berkeley, Bangor
and Sydney around 1993.

It is entirely possible that most of the detailed work for $n\leq 2$ has
already been done by Duskin.  A short account of his work on this was
presented as
% 
\bibentry{DusComo}{%
John W. Duskin, 
A simplicial-matrix approach to higher dimensional category theory,  
talk at CT2000, Como, Italy, 2000,}%
% 
and a full-length version is in preparation:
% 
\bibentry{DusSMA}{%
John W. Duskin, 
A simplicial-matrix approach to higher dimensional category theory I: 
nerves of bicategories,
preprint, 2001, 82 pages.}%
% 
What Duskin does is to construct the nerve of any bicategory (this being a
simplicial set) and to give exact conditions saying which simplicial sets
arise in this way.  He moreover shows how to recover a bicategory from its
nerve.  Duskin does not deal explicitly with Street's conditions or his
notion of hollowness (although he does mention them); indeed, the results
just mentioned suggest that for $n=2$, the hollow structure on the nerve of a
bicategory is superfluous.  

The word `thin' has been used for the same purpose as `hollow', hence the
name \emph{T-complex}, as discussed in~III.2.26 and onwards in Kamps and
Porter's book \cyte{KP} (which also contains basic information on simplicial
sets and horn-filling). The original definition of T-complex was given by
M. K. Dakin in his 1975 Ph.D.\ thesis, published as
% 
\bibentry{DakKCM}{%
M. K. Dakin, 
Kan complexes and multiple groupoid structures,
\jnl{Mathematical sketches (Esquisses Math\'ematiques)}{32}{1983}{xi+92 pages},
University of Amiens.}%
% 
T-complexes are simplicial sets with hollowness satisfying
conditions~\bref{part:degen}--\bref{part:comp}, but with `admissible' dropped
and `filler' changed to `unique filler' in~\bref{part:filler}.  The dropping
of `admissible' means that the delicate orientation considerations of
Street's paper are ignored and any direction is as good as any
other---everything can be run backwards.  Thus, T-complexes are meant to be
like strict $\omega$-groupoids rather than strict $\omega$-categories.




\subsection*{Definition \ds{X}}

The story of \ds{X} is complicated.  Essentially it is a combination of the
ideas of Baez, Dolan, Hermida, Makkai and Power.  Baez and Dolan proposed a
definition of weak $n$-category, drawing on that of Street, in
%
% 
\bibentry{BDHDA3}{%
John C. Baez, James Dolan,
Higher-dimensional algebra III: $n$-categories and the algebra of opetopes,
\jnl{Advances in Mathematics}{135}{1998}{no.~2, 145--206};
also
\epr{q-alg\slsh 9702014}{1997}.}%
% 
An informal account is in section~4 of Baez's \cyte{BaezINC}.  In turn,
Hermida, Makkai and Power drew on the work of Baez and Dolan, producing a
modified version of Baez and Dolan's opetopic sets, which they called
multitopic sets.  (My use in \ds{X} of the former term rather than the
latter should not be interpreted as significant.)  Their original preprint
still seems to be available somewhere on the web:
% 
\bibentry{HMPWHDpre}{%
Claudio Hermida, Michael Makkai, John Power,
On weak higher-dimensional categories, 
\webprint{http:\dblslsh fcs\dt math\dt sci\dt hokudai\dt ac\dt jp\slsh
doc\slsh info\slsh ncat\dt html}{1997}{104}}%
% 
and is currently enjoying a journal serialization:
% 
\bibentry{HMPWHD1}{%
Claudio Hermida, Michael Makkai, John Power,
On weak higher dimensional categories I: Part 1, 
\jnl{Journal of Pure and Applied Algebra}{154}{2000}{no.~1-3, 221--246},}%
% 
% 
\bibentry{HMPWHD2}{%
Claudio Hermida, Michael Makkai, John Power,
On weak higher-dimensional categories I---2, 
\jnl{Journal of Pure and Applied Algebra}{157}{2001}{no.~2-3, 247--277},}%
% 
% 
\bibentry{HMPWHD3}{%
Claudio Hermida, Michael Makkai, John Power,
On weak higher-dimensional categories I: third part,
to appear in 
\emph{Journal of Pure and Applied Algebra}.}%
% 
A related paper with a somewhat different slant and in a much more elementary
style is 
% 
\bibentry{HMPHDM}{%
Claudio Hermida, Michael Makkai, John Power,
Higher-dimensional multigraphs,
\contb{Thirteenth Annual IEEE Symposium on Logic in Computer Science
(Indianapolis, IN, 1998)}
{IEEE Computer Society, Los Alamitos, CA}{1998}{199--206}.}%
% 
Hermida, Makkai and Power's original work did not go as far as an alternative
definition of weak $n$-category, although see the description below of
\cyte{MakMOC}.

I learned something near to definition \ds{X} from
% 
\bibentry{Hy}{%
Martin Hyland,
Definition of lax $n$-category,
seminar at \mbox{DPMMS}, Cambridge, based on a conversation with John Power, 
18 June 1997.}%
% 
Whether this is closer to the approach of Baez and Dolan or of Hermida,
Makkai and Power is hard to say.  The Baez-Dolan definition falls into two
parts: the definition of opetopic set, then the definition of universality.
Certainly the universality in \ds{X} is Baez and Dolan's, but the sketch of
the definition of opetopic set is very elementary, in contrast to the highly
involved definitions of opetopic/multitopic set given by both these groups of
authors.

Opetopic sets are, it is claimed in \cyte{BDHDA3}, just presheaves on a
certain category, the category of \emph{opetopes}.  (The situation can be
compared with that of simplicial sets, which are just presheaves on the
category $\Delta$.)  Multitopic sets are shown in \cyte{HMPWHDpre} to be
presheaves on a category of multitopes.  A third notion of opetope, going
(perhaps reprehensibly) by the same name, is given briefly in section~4.1 of
my \cyte{GOM}, and is laid out in more detail in Chapter~IV of my
\cyte{SHDCT}.  Roughly speaking, it is shown that all three notions are
equivalent in
% 
\bibentry{CheROM}{%
Eugenia Cheng,
The relationship between the opetopic and multitopic approaches to weak
$n$-categories,
\webprint{http:\dblslsh www\dt dpmms\dt cam\dt ac\dt uk\slsh
$\sim$elgc2}{2000}{36}}%
% 
(which compares Baez-Dolan's notion with Hermida-Makkai-Power's) and
% 
\bibentry{CheEAT}{%
Eugenia Cheng,
Equivalence between approaches to the theory of opetopes,
\webprint{http:\dblslsh www\dt dpmms\dt cam\dt ac\dt uk\slsh
$\sim$elgc2}{2000}{36}}%
% 
(which adds in my own).  More accurately, Cheng begins \cyte{CheROM} by
modifying Baez and Dolan's notion of operad; the effect of this is that the
symmetries present in Baez and Dolan's account are now handled much more
cleanly and naturally, especially when it comes to the crucial process of
`slicing'.  So this means that the Baez-Dolan opetopes are not necessarily
the same as the three equivalent kinds of opetope involved in Cheng's
result, and it remains to be seen whether they fit in.

Let us now turn from opetopic sets to universality.  The notion of liminality
does not appear in Baez and Dolan's paper, and is in some sense a substitute
for their notion of `balanced puncture niche'.  I made this change in order
to shorten the inductive definitions; it is just a rephrasing and has no
effect on the definition of universal cell.  The price to
be paid is that in isolation, liminality is probably a less meaningful
concept than that of balanced punctured niche.  

More on the formulation of universality can be found in
% 
\bibentry{CheNUO}{%
Eugenia Cheng,
A notion of universality in the opetopic theory of $n$-categories,
\webprint{http:\dblslsh www\dt dpmms\dt cam\dt ac\dt uk\slsh
$\sim$elgc2}{2001}{12}.}%
% 
Makkai appears to have hit upon a notion of `$\omega$-dimensional
universal properties', and thereby developed the definition of multitopic set
into a new definition of weak $\omega$-category:
% 
\bibentry{MakMOC}{%
M. Makkai,
The multitopic $\omega$-category of all multitopic $\omega$-categories,
\webprint{http:\dblslsh mystic\dt biomed\dt mcgill\dt ca\slsh
M\_Makkai}{1999}{67}.}%
% 
I do not, unfortunately, know enough about this to include an account here.
Nor have I included the definition of
% 
\bibentry{LeiBMB}{%
Tom Leinster,
Batanin meets Baez and Dolan: yet more ways to define weak $n$-category,
seminar at \mbox{DPMMS}, Cambridge,
6 February 2001,}%
% 
which uses opetopic shapes but an algebraic approach like that of \ds{L}.
This definition can be repeated for various other shapes, such as globular
(giving exactly~\ds{L}) and computads (which are like opetopes but with many
outputs as well as many inputs), and perhaps simplicial and even cubical.

Finally, the analysis of $n\leq 2$ has been done in a very precise way, in
% 
\bibentry{CheEOC}{%
Eugenia Cheng,
Equivalence between the opetopic and classical approaches to bicategories,
\webprint{http:\dblslsh www\dt dpmms\dt cam\dt ac\dt uk\slsh
$\sim$elgc2}{2000}{68}.}%
% 
This uses the notion of opetopic/multitopic set given by Cheng's modification
of Baez and Dolan, or by my opetopes, or by Hermida, Makkai and Power (for by
her equivalence result, all three notions are the same), together with the
Baez-Dolan notion of universality.  I am fairly confident that this gives the
same definition of weak $2$-category as is described in \ds{X} above.



\subsection*{Other Definitions of $n$-Category}


I have already mentioned several proposed definitions of weak $n$-category
which are not presented here.  My own \cyte{LeiNQJ} and \cyte{LeiBMB} are
missing.  The opetopic definitions---those related to definition \ds{X}---are
under-represented, as I have not given \emph{any} such definition in precise
terms; in particular, there is no exact presentation of Baez-Dolan's
definition~\cyte{BDHDA3}, of Cheng's modification of Baez-Dolan's definition
(\cyte{CheROM}, \cyte{CheEAT}), or of Makkai's definition \cyte{MakMOC}.

In the final stages of writing this I received a preprint,
% 
\bibentry{MayOCA}{%
J. P. May, 
Operadic categories, $A_\infty$-categories and $n$-categories,
notes of a talk given at Morelia, Mexico on 25 May 2001,
10 pages,}%
% 
containing another definition of weak $n$-category.  I have not had time to
assimilate this; nor have I yet digested the approach to weak $n$-categories
in
% 
\bibentry{MiTs}{%
Hiroyuki Miyoshi, Toru Tsujishita,
Weak $\omega$-categories as $\omega$-hypergraphs,
\eprint{math\dt CT\slsh 0003137}{2000}{26},}%
% 
% 
\bibentry{HMT}{%
Akira Higuchi, Hiroyuki Miyoshi, Toru Tsujishita,
Higher dimensional hypercategories,
\eprint{math\dt CT\slsh 9907150}{1999}{25}.}%
% 


\subsection*{Comparing Definitions}

It seems that not a great deal of rigorous work has been done on comparing the
proposed definitions, although there are plenty of informal ideas floating
about.  The papers that I know of are listed above under the appropriate
definitions.  

The `$n\leq 2$' sections show that there are many reasonable notions even of
weak $2$-category.  This is not diminished by restricting to one-object weak
$2$-categories, that is, monoidal categories.  So by examining and trying to
compare various possible notions of monoidal category, one can hope to get
some idea of what things will be like for weak $n$-categories in general.  A
proof of the equivalence of various `algebraic' or `definite' notions of
monoidal category is in
% 
\bibentry{WMC}{%
Tom Leinster,
What's a monoidal category?,
poster at CT2000, Como, Italy, 2000,}%
% 
and a similar but less general result is in Chapter~1 of my \cyte{OHDCT}
(actually stated for bicategories).  Hermida compares the indefinite with the
definite in his paper \cyte{HerRM} on representable multicategories, and a
different definite/indefinite comparison is in section~3 of my \cyte{UTHM} or
section~3.3 of my \cyte{HAO}.  (I use the terms `definite' and `algebraic' in
the sense of the Introduction.)

No-one who has seen the definition of tricategory given by Gordon, Power and
Street in \cyte{GPS} will take lightly the prospect of analysing the case
$n=3$.  However, it is worth pointing out an aspect of this definition less
well-known than its complexity: that it is not quite algebraic.  

In precise terms, what I mean by this is that the category whose objects are
tricategories and whose maps are strict maps of tricategories is not monadic
over the category of $3$-globular sets.  ($3$-globular sets are globular sets
as in the `Strict $n$-Categories' section of `Background', but with $m$ only
running from $0$ up to $3$.  So the graph structure of a tricategory is a
$3$-globular set.)  For whereas most of the definition of tricategory
consists of some data subject to some equations, a small part does not: in
items~(TD5) and~(TD6), it is stipulated that certain transformations of
bicategories are equivalences.  This is not an algebraic axiom; to make it
into one, we would have to add in as data a pseudo-inverse for each of these
equivalences, together with two invertible modifications witnessing the fact
that it is a pseudo-inverse, and then we would want to add more coherence
axioms (saying, amongst other things, that this data forms an \emph{adjoint}
equivalence).  The impact is that there is little chance of proving that the
category of weak $3$-categories (and strict maps) according to \ds{P},
\ds{B1}, \ds{L1}, or any other algebraic definition is equivalent to the
category of tricategories and strict maps.  This is in contrast to the
situation for $n=2$.



\subsection*{Related Areas}

I will be extremely brief here; as stated above, this is not meant to be a
survey of the literature.  However, there are two areas I feel it would be
inappropriate to omit.  Most of the references that follow are meant to
function as `meta-references', and are chosen for their comprehensive
bibliographies.

The first area is the Australian school of 2-dimensional algebra, a
representative of which is
% 
\bibentry{BKP}{%
R. Blackwell, G. M. Kelly, A. J. Power,
Two-dimensional monad theory,
\jnl{Journal of Pure and Applied Algebra}{59}{1989}{no.~1, 1--41}.}%
% 
The issues arising there merge into questions of coherence, one starting point
for which is the paper `On braidings, syllepses and symmetries' by
Sjoerd Crans:
% 
\bibentry{CraBSS}{%
Sjoed Crans, 
On braidings, syllapses and symmetries,
\jnl{Cahiers de Topologie et G\'eom\'etrie Diff\'erentielle}{41}
{2000}{no.~1, 2--74};
also
\web{http:\dblslsh math\dt unice\dt fr\slsh $\sim$crans},}%
% 
% 
\bibentry{CraBSSerr}{%
S. Crans,  
Erratum: `On braidings, syllapses and symmetries', 
\jnl{Cahiers de Topologie et G\'eom\'etrie Diff\'erentielle}{41}
{2000}{no.~2, 156}.}%
% 
More references for work in this area are to be found in Street's \cyte{StrCS}.

The second area is from algebraic topology: where higher-dimensional category
theorists want to take strict algebraic structures and weaken them, stable
homotopy theorists like to take strict topological-algebraic structures and do
them up to homotopy (in a more sensitive way than one might at first
imagine).  The two have much in common.  Various systematic ways of doing the
latter have been proposed, and some of these are listed on the last page of
text in my \cyte{HAO}.  Missing from that list is the method of
% 
\bibentry{BatHCC}{%
Mikhail A. Batanin,
Homotopy coherent category theory and $A_\infty$-structures in monoidal
categories,
\jnl{Journal of Pure and Applied Algebra}{123}{1998}{no.~1-3, 67--103};
also
\web{http:\dblslsh www\dt math\dt mq\dt edu\dt au\slsh $\sim$mbatanin\slsh
papers\dt html}.}% 
% 
Another connection with homotopy theory and loop spaces is in 
% 
\bibentry{BFSV}{%
C. Balteanu, Z. Fiedorowicz, R. Schw\"anzl, R. Vogt,
Iterated monoidal categories,
\eprint{math\dt AT\slsh 9808082}{1998}{55}.}%

Further references for these two areas and more can be found in the
`Introductory Texts' listed above. 











