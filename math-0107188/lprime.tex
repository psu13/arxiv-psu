
\defnheading{\lp}	\label{p:lprime}

% (Contractible multicategories)



\concept{Globular Multicategories}

\paragraph{Globular Sets}
 
Let \scat{G} be the category whose objects are the natural numbers
$0,1,\ldots$, and whose arrows are generated by
$
\sigma_m, \tau_m: m \go m-1
$
for each $m\geq 1$, subject to equations
\[
\sigma_{m-1} \of \sigma_m = \sigma_{m-1} \of \tau_m,
\diagspace
\tau_{m-1} \of \sigma_m = \tau_{m-1} \of \tau_m
\]
($m\geq 2$).  A functor $A: \scat{G} \go \Set$ is called a \demph{globular
set}; I will write $s$ for $A(\sigma_m)$, and $t$ for $A(\tau_m)$. 

\paragraph{The Free Strict $\omega$-Category Monad}

Any (small) strict $\omega$-category has an underlying globular set $A$, in
which $A(m)$ is the set of $m$-cells and $s$ and $t$ are the source and
target maps.  We thus obtain a forgetful functor $U$ from the category of
strict $\omega$-categories and strict $\omega$-functors to the category
\ftrcat{\scat{G}}{\Set} of globular sets.  $U$ has a left adjoint, so there
is an induced monad $(T, \id \goby{\eta} T, T^2 \goby{\mu} T)$ on
\ftrcat{\scat{G}}{\Set}.

\paragraph{Globular Graphs}

For each globular set $A$, we define a monoidal category $\Graph_A$.  An
object of $\Graph_A$ is a \demph{(globular) graph on $A$}: that is, a globular
set $R$ together with maps of globular sets
%
\begin{diagram}[size=2em]
	&		&R	&		&	\\
	&\ldTo<\dom	&	&\rdTo>\cod	&	\\
TA	&		&	&		&A.	\\
\end{diagram}
%
A \demph{map} $(R,\dom,\cod) \go (R',\dom',\cod')$ of graphs on $A$ is a map $R
\go R'$ making the evident triangles commute.  The \demph{tensor product} of
graphs $(R,\dom,\cod)$, $(R',\dom',\cod')$ is given by composing along the
upper edges of the following diagram, in which the right-angle symbol means that the square containing it is a
pullback:
%
\begin{diagram}[size=2em]
&&	&	&	&	&R\otimes R'\Spbk&&	&	&	\\
&&	&	&	&\ldTo	&	&\rdTo	&	&	&	\\
&&	&	&TR'	&	&	&	&R	&	&	\\
&&	&\ldTo<{T\dom'}&&\rdTo<{T\cod'}&&\ldTo>\dom&	&\rdTo>\cod&	\\
&&T^2A	&	&	&	&TA	&	&	&	&A.	\\
&\ldTo<{\mu_A}&&&	&	&	&	&	&	&	\\
TA&&	&	&	&	&	&	&	&	&	\\
\end{diagram}
%
The \demph{unit} for the tensor is the graph
%
\begin{diagram}[size=2em]
	&		&A	&		&	\\
	&\ldTo<{\eta_A}	&	&\rdTo>1	&	\\
TA	&		&	&		&A.	\\
\end{diagram}

\paragraph{Globular Multicategories}  

A \demph{globular multicategory} is a globular set $A$ together with a
monoid in $\Graph_A$.  A globular multicategory $\cat{A}$ therefore consists of a
globular set $A$, a graph $(R,\dom,\cod)$ on $A$, and maps $\comp: R \otimes
R \go R$ and $\ids: A \go R$ compatible with \dom\ and \cod\ and obeying associativity and identity laws.


\concept{Contractible Maps}

A map $d: R \go S$ of globular sets is \demph{contractible}
(Figure~\ref{fig:contr}) if
%
\begin{enumerate}
\item the function $d_0: R(0) \go S(0)$ is bijective, and
\item 	\label{part:major}
for every 
\begin{description}
\item $m\geq 0$,
\item $r_0,r_1 \in R(m)$ with $s(r_0)=s(r_1)$ and $t(r_0)=t(r_1)$,
\item $\phi \in S(m+1)$ with $s(\phi)=d_m(r_0)$ and $t(\phi)=d_m(r_1)$,
\end{description}
there exists $\rho \in R(m+1)$ with $s(\rho)=r_0$, $t(\rho)=r_1$, and
$d_{m+1}(\rho)=\phi$.  In the case $m=0$ we drop the (nonsensical) conditions
that $s(r_0)=s(r_1)$ and $t(r_0)=t(r_1)$.
\end{enumerate}

\begin{figure}
\begin{center}
$
\gfst{q_0}\gtwodotty{r_0}{r_1}{\exists\rho}\glst{q_1}
\mbox{\hspace{2.5em}}
\stackrel{d}{\goesto}
\mbox{\hspace{2.5em}}
\gfst{d(q_0)}\gtwo{d(r_0)}{d(r_1)}{\phi}\glst{d(q_1)}
$
\end{center}
\vspace*{-1em}
\caption{Part~\bref{part:major} of the definition of contractibility, shown
for $m=1$}
\label{fig:contr}
\end{figure}

\concept{The Definition}

\paragraph{Weak $\omega$-Categories}

A \demph{weak $\omega$-category} is a globular multicategory 
$\cat{A} = (A,R,\dom,\cod,\comp,\ids)$ such that $\dom: R \go TA$ is
contractible. 

\paragraph{Weak $n$-Categories}

Let $n\geq 0$.  A globular set $A$ is \demph{$n$-dimensional} if for all
$m\geq n$,
\[
s=t: A(m+1) \go A(m)
\]
and this map is an isomorphism.  A \demph{weak $n$-category} is a weak
$\omega$-category $\cat{A}$ such that the globular sets $A$ and $R$ are
$n$-dimensional. 



\clearpage


\lowdimsheading{\lp}

An alternative way of handling weak $n$-categories is to work with only
$n$-dimensional (not infinite-dimensional) structures throughout.  Thus we
replace $\scat{G}$ by its full subcategory $\scat{G}_n$ with objects $0,
\ldots, n$, replace $T$ by the free strict $n$-category monad $T_n$, and so
obtain a definition of \demph{$n$-globular multicategory}.  We also modify
part~\bref{part:major} of the definition of contractibility by changing
`$m\geq 0$' to `$n-1\geq m\geq 0$', and `there exists $\rho$' to `there
exists a unique $\rho$' in the case $m=n-1$.  From these ingredients we get a
new definition of weak $n$-category.

The new and old definitions give two different, but equivalent, categories of
weak $n$-categories (with maps of multicategories as the morphisms); the
analysis of $n\leq 2$ is more convenient with the new definition.


\concept{$n=0$}

We have $\ftrcat{\scat{G}_0}{\Set} \iso \Set$ and $T_0=\id$, and the
contractible maps are the bijections.  So a weak $0$-category is a category
whose domain map is a bijection; that is, a discrete category; that is,
a set.


\concept{$n=1$}

$\ftrcat{\scat{G}_1}{\Set}$ is the category of directed graphs, $T_1$ is the
free category monad on it, and a map of graphs is contractible if and only if
it is an isomorphism.  So a weak $1$-category is essentially a 1-globular
multicategory whose underlying 1-globular graph looks like $T_1 A \ogby{1}
T_1 A \goby{\cod} A$.  Such a graph has at most one multicategory structure,
and it has one if and only if $\cod$ is a $T_1$-algebra structure on $A$.  So
a weak $1$-category is just a $T_1$-algebra, i.e., a category.


\concept{$n=2$}

The free 2-category $T_2 A$ on a 2-globular set $A \in
\ftrcat{\scat{G}_2}{\Set}$ has the same 0-cells as $A$; 1-cells of $T_2 A$
are formal paths $\psi$ in $A$ as in Fig.~\ref{fig:reasons}(a); and a typical
2-cell of $T_2 A$ is the diagram $\phi$ in Fig.~\ref{fig:reasons}(b).

Next, what is a 2-globular multicategory
$\cat{A}=(A,R,\dom,\cod,\comp,\ids)$?  Since we ultimately want to consider
just those \cat{A} in which \dom\ is contractible, let us assume immediately
that $R(0)=A(0)$.  Then \cat{A} consists of:
%
\begin{itemize}
\item a 2-globular set $A \in \ftrcat{\scat{G}_2}{\Set}$
\item for each $\psi$ and $f$ as in Fig.~\ref{fig:reasons}(a), a set of cells
$r: \psi \reason f$; such an $r$ is a 1-cell of $R$, and can be
regarded as a `reason why $f$ is a composite of $\psi$'
\item for each $\phi$ and $g_0,g_1,\alpha$ as in Fig.~\ref{fig:reasons}(b), a
set of cells $\rho: \phi\Reason\alpha$; such a $\rho$ is a
2-cell of $R$, and can be regarded as a `reason why $\alpha$ is a
composite of $\phi$'
\item source and target functions $R(2) \parpairu R(1)$, which, for instance,
assign to $\rho$ a reason $s(\rho)$ why $g_0$ is a composite of
$\gfsts{a_0}\gones{f_1}\gblws{a_1}\gones{f_5}
\gblws{a_2}\gones{f_6}\glsts{a_3}$
\item composition and identities: given $r$ as in Fig.~\ref{fig:reasons}(a)
and similarly $r_i: (f_i^1, \ldots, f_i^{p_i}) \reason f_i$ for each $i=1,
\ldots, k$, there is a composite $r\,\of\,(r_1, \ldots, r_k): (f_1^1, \ldots,
f_k^{p_k}) \reason f$; and similarly for 2-cells and for identities,
\end{itemize}
%
such that the composition and identities satisfy associativity and identity
axioms and are compatible with source and target.

\begin{figure}
\centering
\setlength{\unitlength}{1mm}
\begin{picture}(40,40)
\cell{0}{30}{l}{\psi = 
	\gfsts{a_0} \gones{f_1} \ \ldots \ \gones{f_k} \glsts{a_k}}
\cell{23.5}{18}{c}{\Downarrow}
\cell{25}{18}{l}{r}
\cell{23.5}{10}{c}{\gfsts{a_0} \gones{f} \glsts{a_k}}
\cell{20}{2}{t}{\textrm{(a)}}
\end{picture}
%
\hspace{15mm}
%
\begin{picture}(40,40)
\cell{0}{30}{l}{\phi = 
	\gfsts{a_0} \gfours{f_1}{f_2}{f_3}{f_4}{\alpha_1}{\alpha_2}{\alpha_3}
	\grgts{a_1} \gones{f_5}
	\glfts{a_2} \gthrees{f_6}{f_7}{f_8}{\alpha_4}{\alpha_5}
	\glsts{a_3}}
\cell{24.5}{20.8}{c}{\Ddownarrow}
\cell{25.5}{18}{l}{\rho}
\cell{23.5}{10}{c}{\gfsts{a_0} \gtwos{g_0}{g_1}{\alpha} \glsts{a_3}}
\cell{20}{2}{t}{\textrm{(b)}}
\end{picture}
\caption{(a) A 1-cell, and (b) a typical 2-cell, of $R$.  Here $a_i, f_i, f,
g_i, \alpha_i$ and $\alpha$ are all cells of $A$}
\label{fig:reasons}
\end{figure}


Contractibility says that for each $\psi$ as in
Fig.~\ref{fig:reasons}(a) there is at least one pair $(r,f)$ as in
Fig.~\ref{fig:reasons}(a), and that for each $\phi$ as in
Fig.~\ref{fig:reasons}(b) and each $r_0: (f_1,f_5,f_6) \reason g_0$ and $r_1:
(f_4,f_5,f_8) \reason g_1$, there is exactly one pair $(\rho,\alpha)$ as in
Fig.~\ref{fig:reasons}(b) satisfying $s(\rho)=r_0$ and $t(\rho)=r_1$.  That
is: every diagram of 1-cells has at least one composite, and every diagram of
2-cells has exactly one composite once a way of composing the 1-cells
along its boundary has been chosen.

When $\phi = \gfsts{a_0} \gthrees{f_0}{f_1}{f_2}{\alpha_1}{\alpha_2}
\glsts{a_1}$, the identity reasons for $f_0$ and $f_2$ give via
contractibility a composite $\alpha_2 \of \alpha_1: f_0 \go f_2$, and in this
way the 1- and 2-cells between $a_0$ and $a_1$ form a category
$\cat{A}(a_0,a_1)$.  Now suppose that $\psi$ is as in
Fig.~\ref{fig:reasons}(a) and $r:\psi\reason f$, $r':\psi\reason f'$.
Applying contractibility to the degenerate 2-cell diagram $\phi$ which looks
exactly like $\psi$, we obtain a 2-cell
$\gfsts{a_0}\gtwos{f}{f'}{}\glsts{a_k}$; and similarly the other way round;
so by the uniqueness property of the $\rho$'s, $f\iso f'$ in
$\cat{A}(a_0,a_k)$.  Thus any two composites of a string of 1-cells are
canonically isomorphic.

A weak 2-category is essentially what is known as an `anabicategory'.  To see
how one of these gives rise to a bicategory, choose for each
$\gfsts{a_0}\gones{f}\gblws{a_1}\gones{g}\glsts{a_2}$ in $A$ a reason
$r_{f,g}: (f,g) \reason h$ and write $h=(g\of f)$; and similarly for
identities.  Then, for instance, the horizontal composite of 2-cells $
\gfsts{a_0} \gtwos{f_0}{f_1}{} \gfbws{a_1} \gtwos{g_0}{g_1}{} \glsts{a_2} $
comes via contractibility from $r_{f_0,g_0}$ and $r_{f_1,g_1}$, the
associativity cells arise from the degenerate 2-cell diagram $ \phi =
\gfsts{a_0} \gones{f_1}\gblws{a_1} \gones{f_2}\gblws{a_2}
\gones{f_3}\glsts{a_3} $, and the coherence axioms come from the uniqueness
of the $\rho$'s.





