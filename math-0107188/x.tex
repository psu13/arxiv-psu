
\defnheading{X}		\label{p:x}

This definition is not intended to be rigorous, although it can be made so.


\concept{Opetopic Sets}

An \demph{opetopic set} $A$ is a commutative diagram of sets and functions
%
\begin{diagram}[width=1.7em,tight,alignlabels]
\cdots		&\ 	&\rTo^s_{\raisebox{-.5ex}{$t$}}	&&		&
		A'_2	&\rTo^s_{\raisebox{-.5ex}{$t$}}	&&		&
		A'_1	&\rTo^s_{\raisebox{-.5ex}{$t$}}	&&		&
		A'_0=A_0\\
\cdots 		&\ 	&\rdTo(4,2)			&&\ruTo(4,2)	&
			&\rdTo(4,2)			&&\ruTo(4,2)	&
			&\rdTo(4,2)			&&\ruTo(4,2)	&
			\\
\cdots		&\ 	&\rTo_t^{\raisebox{.7ex}{$s$}}	&&		&
		A_2	&\rTo_t^{\raisebox{.7ex}{$s$}}	&&		&
		A_1	&\rTo_t^{\raisebox{.7ex}{$s$}}	&&		&
		A_0	\\
\end{diagram}
%
where for each $m\geq 1$, the set $A'_m$ and the functions $s: A'_m \go
A'_{m-1}$ and $t: A'_m \go A_{m-1}$ are defined from the sets $A_m, A'_{m-1},
A_{m-1}, \ldots, A'_1, A_1, A_0$ and the functions $s,t$ between them in the
following way.

An element $a \in A_0$ is regarded as a $0$-cell, and drawn \gzeros{a}.  An
element $f \in A_1$ is regarded as a $1$-cell $\gfsts{a} \gones{f}
\glsts{b}$, where $a=s(f)$ and $b=t(f)$.  $A'_1$ is the set of `1-pasting
diagrams' in $A$, that is, diagrams of 1-cells pasted together, that is,
paths $\gfsts{a_0} \gones{f_1} \gblws{a_1} \gones{f_2} \ldots \gones{f_k}
\glsts{a_k}$ ($k\geq 0$) in $A$.  An element $\alpha \in A_2$ has a source
$s(\alpha)$ of this form and a target $t(\alpha)$ of the form $\gfsts{a_0}
\gones{g} \glsts{a_k}$, and is drawn as
%
\begin{equation}	\label{eq:two-ope}
\raisebox{-6.5mm}{%
\setlength{\unitlength}{1mm}
\begin{picture}(36,13)(0,-2)
% Zero-cell marks
\cell{0}{0}{c}{\zmark}
\cell{6}{8}{c}{\zmark}
\cell{36}{0}{c}{\zmark}
% One-cell arrows
\put(0,0){\vector(3,4){6}}
\put(6,8){\vector(3,1){9}}
\put(30,8){\vector(3,-4){6}}
\put(0,0){\vector(1,0){36}}
% Two-cell arrow
\put(18,7){\vector(0,-1){5}}
% Dot-dot-dot
\cell{22}{9.5}{c}{\cdots}
% Labels
\cell{-2.5}{0}{c}{a_0}
\cell{4}{9}{c}{a_1}
\cell{39}{0}{c}{a_k}
\cell{1}{5}{c}{f_1}
\cell{10}{11.5}{c}{f_2}
\cell{35.5}{5}{c}{f_k}
\cell{18}{-1.5}{c}{g}
\cell{20}{4.5}{c}{\alpha}
\end{picture}}
\end{equation}
%
Next, $A'_2$ is the set of `2-pasting diagrams', that is, diagrams of cells
of the form~\bref{eq:two-ope} pasted together, such as
%
\begin{equation}	\label{eq:two-pd}
\raisebox{-10.5mm}{%
\setlength{\unitlength}{1mm}
\begin{picture}(50,21)(-5,-3)
% Zero-cell marks
\cell{0}{0}{c}{\zmark}
\cell{-5}{7.5}{c}{\zmark}
\cell{0}{12.5}{c}{\zmark}
\cell{7.5}{10}{c}{\zmark}
\cell{22.5}{10}{c}{\zmark}
\cell{26.25}{15}{c}{\zmark}
\cell{36.25}{12.5}{c}{\zmark}
\cell{43.75}{7.5}{c}{\zmark}
\cell{36.25}{2.5}{c}{\zmark}
\cell{30}{0}{c}{\zmark}
% 1-cell arrows
\put(0,0){\vector(-2,3){5}}
\put(-5,7.5){\vector(1,1){5}}
\put(0,12.5){\vector(3,-1){7.5}}
\put(7.5,10){\vector(1,0){15}}
\put(22.5,10){\vector(3,4){3.75}}
\put(26.25,15){\vector(4,-1){10}}
\put(36.25,12.5){\vector(3,-2){7.5}}
\put(43.75,7.5){\vector(-3,-2){7.5}}
\put(36.25,2.5){\line(-5,-2){6.25}}
\put(30,0){\vector(-3,-1){0}}
\put(36.25,12.5){\vector(0,-1){10}}
\put(0,0){\vector(3,4){7.5}}
\put(22.5,10){\vector(3,-4){7.5}}
\put(0,0){\vector(1,0){30}}
% 2-cell arrows
\put(15,7.5){\vector(0,-1){5}}
\put(-2,9){\vector(1,-1){4}}
\put(32,10.5){\vector(-1,-1){4}}
\put(41.25,7){\vector(-1,0){4}}
% Labels
\cell{-1}{-1}{c}{\scriptstyle a_0}
\cell{-7}{7.5}{c}{\scriptstyle a_1}
\cell{0}{14}{c}{\scriptstyle a_2}
\cell{8.5}{11.5}{c}{\scriptstyle a_3}
\cell{21}{11}{c}{\scriptstyle a_4}
\cell{27}{16.5}{c}{\scriptstyle a_5}
\cell{37}{14}{c}{\scriptstyle a_6}
\cell{46}{7.5}{c}{\scriptstyle a_7}
\cell{38}{1.5}{c}{\scriptstyle a_8}
\cell{30}{-1.5}{c}{\scriptstyle a_9}
\cell{-3.5}{3}{c}{\scriptstyle f_1}
\cell{-3.5}{11}{c}{\scriptstyle f_2}
\cell{4}{12.5}{c}{\scriptstyle f_3}
\cell{15}{11.5}{c}{\scriptstyle f_4}
\cell{23.5}{13.5}{c}{\scriptstyle f_5}
\cell{31.5}{15}{c}{\scriptstyle f_6}
\cell{41.5}{11}{c}{\scriptstyle f_7}
\cell{41.5}{4}{c}{\scriptstyle f_8}
\cell{34}{0.25}{c}{\scriptstyle f_9}
\cell{34.5}{6.5}{c}{\scriptstyle f_{10}}
\cell{25}{4}{c}{\scriptstyle f_{11}}
\cell{5}{4}{c}{\scriptstyle f_{12}}
\cell{15}{-1.5}{c}{\scriptstyle f_{13}}
\cell{17}{5}{c}{\scriptstyle \alpha_1}
\cell{1}{8}{c}{\scriptstyle \alpha_2}
\cell{28.5}{9.5}{c}{\scriptstyle \alpha_3}
\cell{40}{8}{c}{\scriptstyle \alpha_4}
\end{picture}}
\end{equation}
%
Note that the arrows go in compatible directions: e.g.\ the target or
`output' edge $f_{11}$ of $\alpha_3$ is a source or `input' edge of
$\alpha_1$.  The source of this element of $A'_2$ is $\gfsts{a_0} \gones{f_1}
\cdots \gones{f_9} \glsts{a_9} \in A'_1$, and the target is $f_{13} \in A_1$.
Next, if $\gamma \in A_3$ and $s(\gamma)$ is~\bref{eq:two-pd} then
$t(\gamma)$ is of the form~\bref{eq:two-ope} with $k=9$ and $g = f_{13}$, and
we picture $\gamma$ as a $3$-dimensional cell with a flat bottom face
(labelled $\alpha$) and four curved faces on top (labelled $\alpha_1,
\alpha_2, \alpha_3, \alpha_4$).  Carrying on, $A'_3$ is the set of
$3$-pasting diagrams, $A_4$ is the set of $4$-cells, etc.

We need some terminology concerning cells.  Let $\Phi \goby{\alpha} g$ be an
$m$-cell: that is, let $\alpha \in A_m$ with $s(\alpha) = \Phi \in A'_{m-1}$
and $t(\alpha) = g \in A_{m-1}$.

For any $p$-cell $e$, there is a $p$-pasting diagram $\langle e \rangle$
consisting of $e$ alone.  If $\langle g \rangle \goby{\beta} h$ then
$\alpha$ and $\beta$ can be pasted to obtain $(\Phi \goby{\beta_*(\alpha)} h)
\in A'_m$.

The \demph{faces} of $\Phi$ are the $(m-1)$-cells which have been pasted
together to form it, e.g.\ if $\alpha$ is as in~\bref{eq:two-ope} then $\Phi$
has faces $f_1, \ldots, f_k$, and the $\Phi$ of~\bref{eq:two-pd} has faces
$\alpha_1, \alpha_2, \alpha_3, \alpha_4$.  If $f$ is a face of $\Phi$ and $e$
is a cell \demph{parallel} to $f$ (i.e.\ $e\in A_{m-1}$ with $s(e)=s(f)$,
$t(e)=t(f)$) then we obtain a new pasting diagram $\Phi(e/f) \in A_{m-1}$ by
replacing $f$ with $e$ in $\Phi$.  (Read $\Phi(e/f)$ as `$\Phi$ with $e$
replacing $f$'.)  If also $\langle e \rangle \goby{\beta} f$ then $\alpha$
and $\beta$ can be pasted to obtain $(\Phi(e/f) \goby{\beta^*(\alpha)} g) \in
A'_m$.


\concept{Universal Cells}

Let $A$ be an opetopic set and fix $n\geq 0$.  We define what it means for a
cell $\Phi \goby{\epsln} g$ of $A$ to be `universal', and, when $f$ is a face
of $\Phi$, what it means for $f$ to be `liminal' in the cell.  The
definitions depend on $n$, so I should really say `$n$-universal' rather than
just `universal', and similarly `$n$-liminal'; but I will drop the `$n$'
since it is regarded as fixed.  

The two definitions are given inductively in an interdependent way.

\paragraph{Universality} 

Let $m\geq n+1$.  A cell $(\Phi \goby{\epsln} g) \in A_m$ is
\demph{universal} if whenever $(\Phi \goby{\epsln'} g')\in A_m$, then
$\epsln'=\epsln$.

Let $1\leq m\leq n$.  A cell $(\Phi \goby{\epsln} g) \in A_m$ is
\demph{universal} if
%
\begin{enumerate}
\item 	\label{part:univ-univ}
for every $\alpha: \Phi \go h$, there exist $\ovln{\alpha}: \langle
g\rangle \go h$ and a universal cell $U: \ovln{\alpha}_*(\epsln) \go \alpha$,
and 
\item 	\label{part:univ-liminal}
for every $\alpha: \Phi \go h$, $\ovln{\alpha}: \langle g\rangle \go h$
and universal $U: \ovln{\alpha}_*(\epsln) \go \alpha$, $\ovln{\alpha}$ is
liminal in $U$.
\end{enumerate}

\paragraph{Liminality}

Let $m\geq 1$, let $(\Phi \goby{\epsln} g) \in A_m$, and let $f$ be a face of
$\Phi$.  Then $f$ is \demph{liminal in $\epsln$} if $m\geq n+2$ or
%
\begin{enumerate}
\item 	\label{part:liminal-univ}
for every cell $e$ parallel to $f$ and $\beta: \Phi(e/f) \go g$, there
exist $\ovln{\beta}: \langle e\rangle \go f$ and a universal cell $U:
\ovln{\beta}^*(\epsln) \go \beta$, and
\item 	\label{part:liminal-liminal}
for every $e$ parallel to $f$, $\beta: \Phi(e/f) \go g$, $\ovln{\beta}:
\langle e\rangle \go f$  and universal $U: \ovln{\beta}^*(\epsln) \go \beta$,
$\ovln{\beta}$ is liminal in $U$.
\end{enumerate}


\concept{The Definition}

Let $n\geq 0$.  A \demph{weak $n$-category} is an opetopic set $A$ satisfying
%
\begin{description}
\item[existence of universal fillers:] for every $m\geq 0$ and $\Phi \in
A'_m$, there exists a universal cell of the form $\Phi \goby{\epsln} g$
\item[closure of universals under composition:] if $m\geq 2$, $(\Phi
\goby{\epsln} g) \in A_m$, each face of $\Phi$ is universal, and $\epsln$ is
universal, then $g$ is universal.
\end{description}



\clearpage

\lowdimsheading{X}


First consider high-dimensional cells in an $n$-category $A$.  For every
$m\geq n+1$ and $\Phi \in A'_{m-1}$, there is a unique $\epsln \in A_m$ whose
source is $\Phi$: in other words, $s: A_m \go A'_{m-1}$ is a bijection.  This
means that the entire opetopic set is determined by the part of dimension
$\leq n$ and the map $t: A_{n+1} \go A_n$.  Identifying $A_{n+1}$ with
$A'_n$, this map $t$ assigns to each $n$-pasting diagram $\Phi$ the target
$g$ of the unique $(n+1)$-cell with source $\Phi$, and we regard $g$ as the
composite of $\Phi$.  (In general, we may regard the target of a universal
cell as a composite of its source; note that all cells of dimension $>n$ are
universal.)  This composition of $n$-cells is strictly associative and
unital.  A weak $n$-category therefore consists of a commutative diagram 
%
\begin{equation}	\label{eq:trunc-ope}
\begin{diagram}[width=1.7em,tight,alignlabels]
A'_n	&\rTo^s_{\raisebox{-.5ex}{$t$}}	&&		&
\ &\cdots&\	
	&\rTo^s_{\raisebox{-.5ex}{$t$}}	&&		&
A'_1	&\rTo^s_{\raisebox{-.5ex}{$t$}}	&&		&
A'_0=A_0\\
\dTo<\of &\rdTo(4,2)			&&\ruTo(4,2)	&
&&	
	&\rdTo(4,2)			&&\ruTo(4,2)	&
	&\rdTo(4,2)			&&\ruTo(4,2)	&
	\\
A_n 	&\rTo_t^{\raisebox{.7ex}{$s$}}	&&		&
\ &\cdots&\	
	&\rTo_t^{\raisebox{.7ex}{$s$}}	&&		&
A_1	&\rTo_t^{\raisebox{.7ex}{$s$}}	&&		&
A_0	\\
\end{diagram}
\end{equation}
%
of sets and functions such that $n$-dimensional composition $\of$ obeys
strict laws and the defining conditions on existence and closure of
universals hold in lower dimensions.

\concept{$n=0$}

A weak $0$-category is a set $A_0$ together with a function $\of: A_0 \go
A_0$ obeying strict laws---which say that $\of$ is the
identity.  So a weak $0$-category is just a set.


\concept{$n=1$}

When $n=1$, diagram~\bref{eq:trunc-ope} is a directed graph $A_1
\parpair{s}{t} A_0$ together with a map $\of: A'_1 \go A_1$ compatible with
the source and target maps: in other words, assigning to each string of edges
\[
a_0 \goby{f_1} a_1 \goby{f_2} \ \cdots \ \goby{f_k} a_k
\]
in $A$ a new edge $a_0 \goby{(f_k \sof\cdots\sof f_1)} a_k$.  The axioms on
$\of$ say that
\[
((f_k^{r_k} \of\cdots\of f_k^1) \of\cdots\of (f_1^{r_1} \of\cdots\of f_1^1))
= (f_k^{r_k} \of\cdots\of f_1^1),
\diagspace
(f) = f
\]
---in other words, that $A$ forms a category.  A weak $1$-category is
therefore a category satisfying the extra conditions that every object is the
domain of some universal morphism and that the composite of universal
morphisms is universal.  I claim that the universal morphisms are the
isomorphisms, so that these conditions hold automatically and a weak
$1$-category is just a category. 

So: a morphism $\epsln: a \go b$ is universal if~\bref{part:univ-univ} every
morphism $\alpha: a \go c$ factors as $\alpha = \ovln{\alpha} \of \epsln$,
and~\bref{part:univ-liminal} such an $\ovln{\alpha}$ is always liminal in the
unique 2-cell
%
\raisebox{-6mm}{%
\setlength{\unitlength}{1mm}
\begin{picture}(24,12)(0,-2)
\cell{4}{1}{c}{\zmark}
\cell{20}{1}{c}{\zmark}
\cell{12}{9}{c}{\zmark}
\put(4,1){\vector(1,0){15.5}}
\put(4,1){\vector(1,1){7.8}}
\put(12,9){\vector(1,-1){7.8}}
\put(12,6){\vector(0,-1){4}}
\cell{3}{2}{tr}{\scriptstyle a}
\cell{13}{9}{l}{\scriptstyle b}
\cell{21}{2}{tl}{\scriptstyle c}
\cell{8}{5}{br}{\scriptstyle \epsln}
\cell{16}{5}{bl}{\scriptstyle \ovln{\alpha}}
\cell{12}{-0.5}{c}{\scriptstyle \alpha}
\cell{13}{4}{l}{\scriptstyle U}
\end{picture}}.
%
Liminality of $\ovln{\alpha}$ in $U$ means, in turn, that if $\twid{\alpha}:
b \go c$ satisfies $\twid{\alpha} \of \epsln = \alpha$ then $\twid{\alpha} =
\ovln{\alpha}$.  (This is part~\bref{part:liminal-univ} of the definition of
liminality; part~\bref{part:liminal-liminal} holds trivially.)  So $\epsln: a
\go b$ is universal if and only if every morphism out of $a$ factors uniquely
through $\epsln$, which holds if and only if $\epsln$ is an isomorphism.


\concept{$n=2$}

A weak $2$-category is essentially the same thing as a bicategory.  More
precisely, the category of bicategories and weak functors is equivalent to
the category whose objects are weak $2$-categories and whose morphisms are
those maps of opetopic sets which send universal cells to universal cells.  
The equivalence works as follows.

Given a bicategory $B$, define a weak $2$-category $A$ by taking $A_0$ and
$A_1$ to be the sets of $0$- and $1$-cells in $B$, and a
$2$-cell~\bref{eq:two-ope} in $A$ to be a $2$-cell $(f_k \of \cdots \of f_1)
\go g$ in $B$.  Here $(f_k \of \cdots \of f_1)$ is defined inductively as
$f_k \of (f_{k-1} \of\cdots\of f_1)$ if $k\geq 1$, or as $1$ if $k=0$; any
other iterated composite would do just as well.  Composition $\of: A'_2 \go
A_2$ is pasting of $2$-cells in $B$.  Then a 1-cell (respectively, 2-cell) in
$A$ turns out to be universal if and only if the corresponding 1-cell
(2-cell) in $B$ is an equivalence (isomorphism), and it follows that $A$ is a
weak $2$-category.

Conversely, take a weak $2$-category $A$ and construct a bicategory $B$ as
follows.  The 0- and 1-cells of $B$ are just those of $A$, and a 2-cell of
$B$ is an element of $A_2$ of the form $\topeavar{\scriptstyle
a}{\scriptstyle b}{\scriptstyle f}{\scriptstyle g}{\scriptstyle \alpha}$
(i.e.\ $\alpha: \langle f \rangle \go g$).  For each diagram $\gfsts{a}
\gones{f} \gblws{b} \gones{g} \glsts{c}$ of 1-cells, choose at random a
universal filler
%
\raisebox{-6mm}{%
\setlength{\unitlength}{1mm}
\begin{picture}(24,12)(0,-2)
\cell{4}{1}{c}{\zmark}
\cell{20}{1}{c}{\zmark}
\cell{12}{9}{c}{\zmark}
\put(4,1){\vector(1,0){15.5}}
\put(4,1){\vector(1,1){7.8}}
\put(12,9){\vector(1,-1){7.8}}
\put(11,6){\vector(0,-1){4}}
\cell{3}{2}{tr}{\scriptstyle a}
\cell{13}{9}{l}{\scriptstyle b}
\cell{21}{2}{tl}{\scriptstyle c}
\cell{8}{5}{br}{\scriptstyle f}
\cell{16}{5}{bl}{\scriptstyle g}
\cell{12}{-0.5}{c}{\scriptstyle g\sof f}
\cell{11.5}{4}{l}{\scriptstyle \epsln_{f,g}}
\end{picture}},
%
where by definition $g\of f = t(\epsln_{f,g})$; this defines composition of
1-cells.  Vertical composition of 2-cells comes from $\of: A'_2 \go A_2$.  To
define the horizontal composite of $\gfsts{a} \gtwos{f}{g}{\alpha} \gfbws{a'}
\gtwos{f'}{g'}{\alpha'} \glsts{a''}$, consider pasting $\epsln_{g,g'}$ to
$\alpha$ and $\alpha'$, and then use the universality of $\epsln_{f,f'}$.
Next observe that given two universal fillers $\Phi \goby{\epsln} g$, $\Phi
\goby{\epsln'} g'$ for a 1-pasting diagram $\Phi = (\gfsts{a_0} \gones{f_1}
\ldots \gones{f_k} \glsts{a_k})$, there is a unique 2-cell $\langle g \rangle
\goby{\delta} g'$ such that the composite $\of (\delta_* (\epsln))$ is
$\epsln'$.  Applying this observation to a certain pair of universal fillers
for $(\gfsts{} \gones{f} \gblws{} \gones{g} \gblws{} \gones{h} \glsts{})$
gives the associativity isomorphism, and the word `unique' in the observation
gives the pentagon axiom.  Identities work similarly, where this time we
choose a random universal filler for each degenerate 1-pasting diagram
\gzeros{a}.

 






