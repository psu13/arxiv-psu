
\defnheading{Tr}	\label{p:tr}


\concept{Topological Background}

\paragraph{Spaces}
Let \Top\ be the category of topological spaces and continuous maps.  Recall
that compact spaces are exponentiable in \Top: that is, if $K$ is compact
then the set $Z^K$ of continuous maps from $K$ to a space
$Z$ can be given a topology (namely, the compact-open topology) in such a way
that there is an isomorphism $\Top(Y, Z^K) \iso \Top(K \times Y, Z)$
natural in $Y,Z \in \Top$.

\paragraph{Operads}
A \demph{(non-symmetric, topological) operad} $D$ is a sequence
$(D(k))_{k\geq 0}$ of spaces together with an element (the
\demph{identity}) of $D(1)$ and for each $k, r_1, \ldots, r_k \geq 0$ a map
\[
D(k) \times D(r_1) \times \cdots \times D(r_k) 	
\go						
D(r_1 + \cdots + r_k)
\]
(\demph{composition}), obeying unit and associativity laws.  (Example: fix an
object $M$ of a monoidal category \cat{M}, and define $D(k) =
\cat{M}(M^{\otimes k},M)$.)

\paragraph{The All-Important Operad}
There is an operad $E$ in which $E(k)$ is the space of continuous
endpoint-preserving maps from $[0,1]$ to $[0,k]$.  (`Endpoint-preserving'
means that $0$ maps to $0$ and $1$ to $k$.)  The identity element of $E(1)$
is the identity map, and composition in the operad is by substitution.

\paragraph{Path Spaces}
For any space $X$ and $x,x'\in X$, a \demph{path} from $x$ to $x'$ in $X$ is
a map $p: [0,1] \go X$ satisfying $p(0) = x$ and $p(1) = x'$.  There is a
space $X(x,x')$ of paths from $x$ to $x'$, a subspace of the exponential
$X^{[0,1]}$.

\paragraph{Operad Action on Path Spaces}
Fix a space $X$. For any $k\geq 0$ and $x_0, \ldots, x_k \in X$, there is a
canonical map
\[
\act{x_0,\ldots,x_k}: 
E(k) \times X(x_0,x_1) \times\cdots\times X(x_{k-1},x_k)
\go
X(x_0,x_k).
\]
These maps are compatible with the composition and identity of the operad
$E$,
and the construction is functorial in $X$.

\paragraph{Path-Components} Let $\Pi_0: \Top \go \Set$ be the functor
assigning to each space its set of path-components, and note that $\Pi_0$
preserves finite products.


\concept{The Definition}

We will define inductively, for each $n\geq 0$, a category \TCat{n} with
finite products and a functor $\Pi_n: \Top \go \TCat{n}$ preserving finite
products.  A \demph{weak $n$-category} is an object of \TCat{n}.
(Maps in \TCat{n} are to be thought of as \emph{strict}
$n$-functors.) 


\paragraph{Base Case}

$\TCat{0} = \Set$, and $\Pi_0: \Top \go \Set$ is as above.

\paragraph{Objects of \TCat{(n+1)}} 

Inductively, a \demph{weak (n+1)-category} \pr{A}{\gamma} consists of
%
\begin{itemize}
\item a set $A_0$
\item a family $(A(a,a'))_{a,a'\in A_0}$ of weak $n$-categories
\item for each $k\geq 0$ and $a_0, \ldots, a_k \in A_0$, a map
\[
\gamma_{a_0, \ldots, a_k}:
\Pi_n(E(k)) \times A(a_0,a_1) \times\cdots\times A(a_{k-1},a_k)
\go
A(a_0,a_k)
\]
in \TCat{n},
\end{itemize}
%
such that the $\gamma_{a_0, \ldots, a_k}$'s satisfy compatibility axioms of
the same form as those satisfied by the $\act{x_0,\ldots,x_k}$'s.  (All this
makes sense because $\Pi_n$ preserves finite products and \TCat{n} has them.)

\paragraph{Maps in \TCat{(n+1)}} 

A \demph{map $\pr{A}{\gamma}\go\pr{B}{\delta}$} in \TCat{(n+1)} consists of
%
\begin{itemize}
\item a function $F_0: A_0 \go B_0$
\item for each $a,a'\in A_0$, a map $F_{a,a'}: A(a,a') \go B(F_0 a, F_0 a')$ of
weak $n$-categories,
\end{itemize}
%
satisfying the axiom
\[
F_{a_0,a_k} \,\of\, \gamma_{a_0, \ldots, a_k}
=
\delta_{F_0 a_0, \ldots, F_0 a_k} \,\of\, 
(1_{\Pi_n(E(k))} \times F_{a_0,a_1} \times\cdots\times F_{a_{k-1},a_k})
\]
for all $k\geq 0$ and $a_0, \ldots, a_k \in A_0$.

\paragraph{Composition and Identities in \TCat{(n+1)}}

Obvious.

\paragraph{$\Pi_{n+1}$ on Objects}

For a space $X$ we define $\Pi_{n+1}(X) = (A,\gamma)$, where 
%
\begin{itemize}
\item $A_0$ is the underlying set of $X$
\item $A(x,x') = \Pi_n(X(x,x'))$
\item for $x_0, \ldots, x_k \in X$, the map $\gamma_{x_0, \ldots, x_k}$ is the
composite 
%
\begin{eqnarray*}
\lefteqn{\Pi_n(E(k)) \times \Pi_n(X(x_0,x_1)) \times\cdots\times
\Pi_n(X(x_{k-1},x_k))} 							\\
	&\goiso	&\Pi_n(E(k) \times X(x_0,x_1) \times\cdots\times
		 X(x_{k-1},x_k))					\\
	&\goby{\Pi_n(\act{x_0,\ldots,x_k})}
		&\Pi_n(X(x_0,x_k)).
\end{eqnarray*}
\end{itemize}

\paragraph{$\Pi_{n+1}$ on Maps}

The functor $\Pi_{n+1}$ is defined on maps in the obvious way.

\paragraph{Finite Products Behave}

It is easy to show that \TCat{(n+1)} has finite products and that
$\Pi_{n+1}$ preserves finite products: so the inductive definition goes
through. 



\clearpage



\lowdimsheading{Tr}

First observe that the space $E(k)$ is contractible for each $k$ (being, in a
suitable sense, convex).  In particular this tells us that $E(k)$ is
path-connected, and that the path space $E(k)(\theta,\theta')$ is
path-connected for every $\theta, \theta' \in E(k)$.


\concept{$n=0$}

By definition, $\TCat{0}=\Set$ and $\Pi_0: \Top\go\Set$ is the
path-components functor.


\concept{$n=1$}

\paragraph{The Category \TCat{1}}

A weak 1-category \pr{A}{\gamma} consists of
%
\begin{itemize}
\item a set $A_0$
\item a set $A(a,a')$ for each $a, a' \in A_0$
\item for each $k\geq 0$ and $a_0, \ldots, a_k \in A_0$, a function
\[
\gamma_{a_0, \ldots, a_k}:
\Pi_0(E(k)) \times A(a_0,a_1) \times\cdots\times A(a_{k-1},a_k)
\go
A(a_0,a_k)
\]
\end{itemize}
%
such that these functions satisfy certain axioms.  So a weak 1-category looks
something like a category: $A_0$ is the set of objects, $A(a,a')$ is
the set of maps from $a$ to $a'$, and $\gamma$ provides some kind of
composition.  Since $E(k)$ is path-connected, we may strike out $\Pi_0(E(k))$
from the product above; and then we may suggestively write
\[
(f_k \of\cdots\of f_1) = \gamma_{a_0, \ldots, a_k}(f_1, \ldots, f_k).
\]
The axioms on these `$k$-fold composition functions' mean that a weak
1-category is, in fact, exactly a category.  Maps in \TCat{1} are just
functors, and so $\TCat{1}$ is equivalent to $\Cat$.

\paragraph{The Functor $\Pi_1$} 

For a space $X$, the (weak 1-)category
$\Pi_1(X) = (A,\gamma)$ is given by
%
\begin{itemize}
\item $A_0$ is the underlying set of $X$
\item $A(x,x')$ is the set of path-components of the path-space $X(x,x')$:
that is, the set of homotopy classes of paths from $x$ to $x'$
\item Let $x_0 \goby{p_1} \ \cdots\ \goby{p_k} x_k$ be a sequence of paths in
$X$, and write $[p]$ for the homotopy class of a path $p$.  Then
\[
([p_k] \of \cdots \of [p_1]) = 
[\act{x_0,\ldots,x_k}(\theta, p_1, \ldots p_k)]
\]
where $\theta$ is any member of $E(k)$---it doesn't matter which.  In other
words, composition of paths is by laying them end to end.
\end{itemize}
%
Hence $\Pi_1(X)$ is the usual fundamental groupoid of $X$, and indeed $\Pi_1:
\Top \go$ $\Cat$ is the usual fundamental groupoid functor.




\concept{$n=2$} 

A weak 2-category \pr{A}{\gamma} consists of
%
\begin{itemize}
\item a set $A_0$
\item a category $A(a,a')$ for each $a, a' \in A_0$
\item for each $k\geq 0$ and $a_0, \ldots, a_k \in A_0$, a functor
\[
\gamma_{a_0, \ldots, a_k}:
\Pi_1(E(k)) \times A(a_0,a_1) \times\cdots\times A(a_{k-1},a_k)
\go
A(a_0,a_k)
\]
\end{itemize}
%
such that these functors satisfy axioms expressing compatibility with the
composition and identity of the operad $E$. 

By the description of $\Pi_1$ and the initial observations of this section,
the category $\Pi_1(E(k))$ is indiscrete (i.e.\ all hom-sets have one
element) and its objects are the elements of $E(k)$.  So $\gamma$ assigns to
each $\theta \in E(k)$ and $a_i\in A_0$ a functor
\[
\ovln{\theta}: 
A(a_0,a_1) \times\cdots\times A(a_{k-1},a_k)
\go
A(a_0,a_k),
\] 
and to each $\theta, \theta' \in E(k)$ and $a_i\in A_0$ a natural isomorphism
\[
\omega_{\theta,\theta'}: \ovln{\theta} \goiso \ovln{\theta'}.
\]
(Really we should add `$a_0, \ldots, a_k$' as a subscript to $\ovln{\theta}$
and to $\omega_{\theta,\theta'}$.)  Functoriality of $\gamma_{a_0, \ldots,
a_k}$ says that
\[
\omega_{\theta,\theta} = 1, \diagspace 
\omega_{\theta, \theta''} = 
\omega_{\theta',\theta''} \,\of\, \omega_{\theta,\theta'}.
\]
The `certain axioms' say firstly that 
\[
\ovln{\theta \,\of\, (\theta_1, \ldots, \theta_k)} =
\ovln{\theta} \,\of\, (\ovln{\theta_1} \times \cdots \times \ovln{\theta_k}),
\diagspace
\ovln{1} = 1
\]
for $\theta\in E(k)$ and $\theta_i\in E(r_i)$, where the left-hand sides of
the two equations refer respectively to composition and identity in the
operad $E$; and secondly that the natural isomorphisms
$\omega_{\theta,\theta'}$ fit together in a coherent way.

So a weak 2-category is probably not a structure with which we are already
familiar.  However, it nearly is.  For define $\tr(k)$ to be the set of
$k$-leafed rooted trees which are `unitrivalent' (each vertex has either 0 or
2 edges coming up out of it); and suppose we replaced $\Pi_1(E(k))$ by the
indiscrete category with object-set $\tr(k)$, so that the $\theta$'s above
would be trees.  A weak 2-category would then be exactly a bicategory: e.g.\
if $\theta=\littletree$ then $\ovln{\theta}$ is binary composition, and if
$(\theta,\theta') = (\lefttree,\righttree)$ then $\omega_{\theta,\theta'}$
is the associativity isomorphism.  And in some sense, a $k$-leafed tree might
be thought of as a discrete version of an endpoint-preserving map $[0,1] \go
[0,k]$.

With this in mind, any weak $2$-category $(A,\gamma)$ gives rise to a
bicategory $B$ (although the converse process seems less straightforward).
First pick at random an element $\theta_2$ of $E(2)$, and let $\theta_0$ be
the unique element of $E(0)$.  Then take $B_0 = A_0$, $B(a,a')=A(a,a')$,
binary composition to be $\ovln{\theta_2}$, identities to be
$\ovln{\theta_0}$, the associativity isomorphism to be $\omega_{\theta_2\sof
(1,\theta_2), \theta_2\sof (\theta_2,1)}$, and similarly units.  The coherence
axioms on $B$ follow from the coherence axioms on $\omega$: and so we have
a bicategory.


\clearpage
