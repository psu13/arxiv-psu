% Basic Category Theory
% Tom Leinster <Tom.Leinster@ed.ac.uk>
% 
% Copyright (c) Tom Leinster 2014-2016
% 
% Front matter.  The abstract and keywords are not actually used in the
% text, although a version of the abstract appears on the back cover of the
% print edition. 
% 

\title{Basic Category Theory}

\author{Tom Leinster\\[1ex]
\emph{University of Edinburgh}}

\bookabstract{This short introduction to category theory is for readers
with relatively little mathematical background.  At its heart is the
concept of a universal property, important throughout mathematics.  After
a chapter giving the basic definitions, the three main chapters present
three ways of expressing universal properties: via adjoint functors,
representable functors, and limits.  A final chapter ties the three
together.

For each new categorical concept, a generous supply of examples is
provided, taken from different parts of mathematics.  At points where the
leap in abstraction is particularly great (such as the Yoneda lemma), the
reader will find careful and extensive explanations.}

\bookkeywords{Category, functor, adjoint, limit, universal property.  MSC
  2010: 18A (primary), 03E (secondary).}

\tableofcontents

% The next command makes the contents fit onto a single page.  

\addtocontents{toc}{\vspace{-4\baselineskip}} 
\clearpage 
\thispagestyle{empty}
\ 
