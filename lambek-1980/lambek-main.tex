% !TEX root = lambek-1980-pal.tex
%\usepackage[dotinlabels]{titletoc}
%\titlelabel{{\thetitle}.\quad}
%\usepackage{titletoc}
\usepackage[small]{titlesec}

\titleformat{\section}[block]
  {\fillast\medskip}
  {\bfseries{\thesection. }}
  {1ex minus .1ex}
  {\bfseries}
 
\titleformat*{\subsection}{\itshape}
\titleformat*{\subsubsection}{\itshape}

\setcounter{tocdepth}{2}

\titlecontents{section}
              [2.3em] 
              {\bigskip}
              {{\contentslabel{2.3em}}}
              {\hspace*{-2.3em}}
              {\titlerule*[1pc]{}\contentspage}
              
\titlecontents{subsection}
              [4.7em] 
              {}
              {{\contentslabel{2.3em}}}
              {\hspace*{-2.3em}}
              {\titlerule*[.5pc]{}\contentspage}

% hopefully not used.           
\titlecontents{subsubsection}
              [7.9em]
              {}
              {{\contentslabel{3.3em}}}
              {\hspace*{-3.3em}}
              {\titlerule*[.5pc]{}\contentspage}
%\makeatletter
\renewcommand\tableofcontents{%
    \section*{\contentsname
        \@mkboth{%
           \MakeLowercase\contentsname}{\MakeLowercase\contentsname}}%
    \@starttoc{toc}%
    }
\def\@oddhead{{\scshape\rightmark}\hfil{\small\scshape\thepage}}%
\def\sectionmark#1{%
      \markright{\MakeLowercase{%
        \ifnum \c@secnumdepth >\m@ne
          \thesection\quad
        \fi
        #1}}}
        
\makeatother

%\makeatletter

 \def\small{%
  \@setfontsize\small\@xipt{13pt}%
  \abovedisplayskip 8\p@ \@plus3\p@ \@minus6\p@
  \belowdisplayskip \abovedisplayskip
  \abovedisplayshortskip \z@ \@plus3\p@
  \belowdisplayshortskip 6.5\p@ \@plus3.5\p@ \@minus3\p@
  \def\@listi{%
    \leftmargin\leftmargini
    \topsep 9\p@ \@plus3\p@ \@minus5\p@
    \parsep 4.5\p@ \@plus2\p@ \@minus\p@
    \itemsep \parsep
  }%
}%
 \def\footnotesize{%
  \@setfontsize\footnotesize\@xpt{12pt}%
  \abovedisplayskip 10\p@ \@plus2\p@ \@minus5\p@
  \belowdisplayskip \abovedisplayskip
  \abovedisplayshortskip \z@ \@plus3\p@
  \belowdisplayshortskip 6\p@ \@plus3\p@ \@minus3\p@
  \def\@listi{%
    \leftmargin\leftmargini
    \topsep 6\p@ \@plus2\p@ \@minus2\p@
    \parsep 3\p@ \@plus2\p@ \@minus\p@
    \itemsep \parsep
  }%
}%
\def\open@column@one#1{%
 \ltxgrid@info@sw{\class@info{\string\open@column@one\string#1}}{}%
 \unvbox\pagesofar
 \@ifvoid{\footsofar}{}{%
  \insert\footins\bgroup\unvbox\footsofar\egroup
  \penalty\z@
 }%
 \gdef\thepagegrid{one}%
 \global\pagegrid@col#1%
 \global\pagegrid@cur\@ne
 \global\count\footins\@m
 \set@column@hsize\pagegrid@col
 \set@colht
}%

\def\frontmatter@abstractheading{%
\bigskip
 \begingroup
  \centering\large
  \abstractname
  \par\bigskip
 \endgroup
}%

\makeatother

%\DeclareSymbolFont{CMlargesymbols}{OMX}{cmex}{m}{n}
%\DeclareMathSymbol{\sum}{\mathop}{CMlargesymbols}{"50}

\usepackage[papersize={6.6in, 10.0in}, left=.5in, right=.5in, top=.6in, bottom=.9in]{geometry}
\linespread{1.05}
%\sloppy
\raggedbottom
\usepackage[leqno]{amsmath}

\pagestyle{plain}
\usepackage{mathpartir}
\usepackage{stmaryrd}
\usepackage{mathtools}
\usepackage{tikz-cd}
\usepackage{microtype}
\usepackage{amssymb}
\usepackage{enumitem}

%\usepackage{fdsymbol}

% these include amsmath and that can cause trouble in older docs.
\makeatletter
\@ifpackageloaded{amsmath}{}{\RequirePackage{amsmath}}

\DeclareFontFamily{U}  {cmex}{}
\DeclareSymbolFont{Csymbols}       {U}  {cmex}{m}{n}
\DeclareFontShape{U}{cmex}{m}{n}{
    <-6>  cmex5
   <6-7>  cmex6
   <7-8>  cmex6
   <8-9>  cmex7
   <9-10> cmex8
  <10-12> cmex9
  <12->   cmex10}{}

\def\Set@Mn@Sym#1{\@tempcnta #1\relax}
\def\Next@Mn@Sym{\advance\@tempcnta 1\relax}
\def\Prev@Mn@Sym{\advance\@tempcnta-1\relax}
\def\@Decl@Mn@Sym#1#2#3#4{\DeclareMathSymbol{#2}{#3}{#4}{#1}}
\def\Decl@Mn@Sym#1#2#3{%
  \if\relax\noexpand#1%
    \let#1\undefined
  \fi
  \expandafter\@Decl@Mn@Sym\expandafter{\the\@tempcnta}{#1}{#3}{#2}%
  \Next@Mn@Sym}
\def\Decl@Mn@Alias#1#2#3{\Prev@Mn@Sym\Decl@Mn@Sym{#1}{#2}{#3}}
\let\Decl@Mn@Char\Decl@Mn@Sym
\def\Decl@Mn@Op#1#2#3{\def#1{\DOTSB#3\slimits@}}
\def\Decl@Mn@Int#1#2#3{\def#1{\DOTSI#3\ilimits@}}

\let\sum\undefined
\DeclareMathSymbol{\tsum}{\mathop}{Csymbols}{"50}
\DeclareMathSymbol{\dsum}{\mathop}{Csymbols}{"51}

\Decl@Mn@Op\sum\dsum\tsum

\makeatother

\makeatletter
\@ifpackageloaded{amsmath}{}{\RequirePackage{amsmath}}

\DeclareFontFamily{OMX}{MnSymbolE}{}
\DeclareSymbolFont{largesymbolsX}{OMX}{MnSymbolE}{m}{n}
\DeclareFontShape{OMX}{MnSymbolE}{m}{n}{
    <-6>  MnSymbolE5
   <6-7>  MnSymbolE6
   <7-8>  MnSymbolE7
   <8-9>  MnSymbolE8
   <9-10> MnSymbolE9
  <10-12> MnSymbolE10
  <12->   MnSymbolE12}{}

\DeclareMathSymbol{\downbrace}    {\mathord}{largesymbolsX}{'251}
\DeclareMathSymbol{\downbraceg}   {\mathord}{largesymbolsX}{'252}
\DeclareMathSymbol{\downbracegg}  {\mathord}{largesymbolsX}{'253}
\DeclareMathSymbol{\downbraceggg} {\mathord}{largesymbolsX}{'254}
\DeclareMathSymbol{\downbracegggg}{\mathord}{largesymbolsX}{'255}
\DeclareMathSymbol{\upbrace}      {\mathord}{largesymbolsX}{'256}
\DeclareMathSymbol{\upbraceg}     {\mathord}{largesymbolsX}{'257}
\DeclareMathSymbol{\upbracegg}    {\mathord}{largesymbolsX}{'260}
\DeclareMathSymbol{\upbraceggg}   {\mathord}{largesymbolsX}{'261}
\DeclareMathSymbol{\upbracegggg}  {\mathord}{largesymbolsX}{'262}
\DeclareMathSymbol{\braceld}      {\mathord}{largesymbolsX}{'263}
\DeclareMathSymbol{\bracelu}      {\mathord}{largesymbolsX}{'264}
\DeclareMathSymbol{\bracerd}      {\mathord}{largesymbolsX}{'265}
\DeclareMathSymbol{\braceru}      {\mathord}{largesymbolsX}{'266}
\DeclareMathSymbol{\bracemd}      {\mathord}{largesymbolsX}{'267}
\DeclareMathSymbol{\bracemu}      {\mathord}{largesymbolsX}{'270}
\DeclareMathSymbol{\bracemid}     {\mathord}{largesymbolsX}{'271}

\def\horiz@expandable#1#2#3#4#5#6#7#8{%
  \@mathmeasure\z@#7{#8}%
  \@tempdima=\wd\z@
  \@mathmeasure\z@#7{#1}%
  \ifdim\noexpand\wd\z@>\@tempdima
    $\m@th#7#1$%
  \else
    \@mathmeasure\z@#7{#2}%
    \ifdim\noexpand\wd\z@>\@tempdima
      $\m@th#7#2$%
    \else
      \@mathmeasure\z@#7{#3}%
      \ifdim\noexpand\wd\z@>\@tempdima
        $\m@th#7#3$%
      \else
        \@mathmeasure\z@#7{#4}%
        \ifdim\noexpand\wd\z@>\@tempdima
          $\m@th#7#4$%
        \else
          \@mathmeasure\z@#7{#5}%
          \ifdim\noexpand\wd\z@>\@tempdima
            $\m@th#7#5$%
          \else
           #6#7%
          \fi
        \fi
      \fi
    \fi
  \fi}

\def\overbrace@expandable#1#2#3{\vbox{\m@th\ialign{##\crcr
  #1#2{#3}\crcr\noalign{\kern2\p@\nointerlineskip}%
  $\m@th\hfil#2#3\hfil$\crcr}}}
\def\underbrace@expandable#1#2#3{\vtop{\m@th\ialign{##\crcr
  $\m@th\hfil#2#3\hfil$\crcr
  \noalign{\kern2\p@\nointerlineskip}%
  #1#2{#3}\crcr}}}

\def\overbrace@#1#2#3{\vbox{\m@th\ialign{##\crcr
  #1#2\crcr\noalign{\kern2\p@\nointerlineskip}%
  $\m@th\hfil#2#3\hfil$\crcr}}}
\def\underbrace@#1#2#3{\vtop{\m@th\ialign{##\crcr
  $\m@th\hfil#2#3\hfil$\crcr
  \noalign{\kern2\p@\nointerlineskip}%
  #1#2\crcr}}}

\def\bracefill@#1#2#3#4#5{$\m@th#5#1\leaders\hbox{$#4$}\hfill#2\leaders\hbox{$#4$}\hfill#3$}

\def\downbracefill@{\bracefill@\braceld\bracemd\bracerd\bracemid}
\def\upbracefill@{\bracefill@\bracelu\bracemu\braceru\bracemid}

\DeclareRobustCommand{\downbracefill}{\downbracefill@\textstyle}
\DeclareRobustCommand{\upbracefill}{\upbracefill@\textstyle}

\def\upbrace@expandable{%
  \horiz@expandable
    \upbrace
    \upbraceg
    \upbracegg
    \upbraceggg
    \upbracegggg
    \upbracefill@}
\def\downbrace@expandable{%
  \horiz@expandable
    \downbrace
    \downbraceg
    \downbracegg
    \downbraceggg
    \downbracegggg
    \downbracefill@}

\DeclareRobustCommand{\overbrace}[1]{\mathop{\mathpalette{\overbrace@expandable\downbrace@expandable}{#1}}\limits}
\DeclareRobustCommand{\underbrace}[1]{\mathop{\mathpalette{\underbrace@expandable\upbrace@expandable}{#1}}\limits}

\makeatother


\usepackage[small]{titlesec}
\usepackage{cite}

% make sure there is enough TOC for reasonable pdf bookmarks.
\setcounter{tocdepth}{3}
\usepackage{amsthm}

\newtheorem{theorem}{Theorem}
\newtheorem{prop}[theorem]{Proposition}


\usepackage[colorlinks=true
,breaklinks=true
,urlcolor=blue
,anchorcolor=blue
,citecolor=blue
,filecolor=blue
,linkcolor=blue
,menucolor=blue
,linktocpage=true]{hyperref}
\hypersetup{
bookmarksopen=true,
bookmarksnumbered=true,
bookmarksopenlevel=10,
}

\date{}
\def\to{\longrightarrow}
\def\imp{\shortrightarrow}
\def\iff{\leftrightarrow}
\def\union{\cup}
\def\inc{\subseteq}
\def\dom{\mathop{\rm dom}}
\def\cod{\mathop{\rm cod}}
\def\id{{\mathrm 1}}
\def\res{\!\upharpoonleft\!}
\def\ffam{\varphi}
\def\comp{\circ}
\def\bbone{\mathbb 1}
\def\zeromap{0}
\def\bbzero{{\mathbb O}}
\def\ccc{{c.c.c.}}
\def\ev{\varepsilon}
\def\ebc{\varepsilon_{BC}}
\def\L{\Lambda}
\def\l{\lambda}
\def\lx{\lambda_x}
\def\ly{\lambda_y}
\def\lu{\lambda_u}
\def\lv{\lambda_v}
\def\lz{\lambda_z}
\def\lm#1.#2{\lambda#1.\, #2}
\def\br#1{[\, #1 \, ]}
\def\V{V}
\def\U{U}
\def\D{D}
\def\C{\mathcal C}
\def\S{\mathcal S}
\def\lxy{\l x\, \l y . \,}
\def\lmm#1#2.#3{\l #1\, \l #2 . \, #3}
\def\sss{(*\!*\!*)}
\def\ss{(**)}
\def\ssn{(**_n)}
\def\scop{\S^{\C^{op}}}
\def\pr{^\prime}

\def\PU{\mathcal P U}
\def\P{\mathcal P}
\def\UU{(U\to U)}
\def\BA{B \to A}
\def\AB{A \to B}
\def\calA{{\cal A}}
\def\cI{{I}}
\def\cS{{S}}
\def\cK{{K}}
\def\app{\mathop{{}^\wr}\kern-.8pt}
\def\schon{Sch\"onfinkel}
\newcommand{\be}{\begin{equation}}
\newcommand{\ee}{\end{equation}}
\newcommand{\bes}{\begin{equation*}}
\newcommand{\ees}{\end{equation*}}

% makes "=" with "x" under it
\makeatletter
\DeclareRobustCommand{\eqx}{\mathrel{\mathpalette\eq@{x}}}
\DeclareRobustCommand{\eqy}{\mathrel{\mathpalette\eq@{y}}}
\newcommand{\eq@}[2]{%
  \vtop{\offinterlineskip
    \ialign{\hfil##\hfil\cr
      $\m@th#1=$\cr % top
      \noalign{\sbox\z@{$\m@th#1\mkern0mu$}\kern-\wd\z@}
      $\m@th\alexey@demote{#1}#2$\cr
    }%
  }%
}
\newcommand{\alexey@demote}[1]{%
  \ifx#1\displaystyle\scriptstyle\else
  \ifx#1\textstyle\scriptstyle\else
  \scriptscriptstyle\fi\fi
}

\newcommand*\dotop{\mathpalette\bigcdot@{.6}}
\newcommand*\bigcdot@[2]{\mathbin{\vcenter{\hbox{\scalebox{#2}{$\m@th#1\bullet$}}}}}
\makeatother

\title{\large From $\l$-Calculus to Cartesian Closed Categories}
\author{\normalsize J. Lambek \\
{\small\it Mathematics Department, McGill University} \\
{\small\it Montreal, P.Q. HSA 2K63 Canada.}}
\begin{document}
\pdfbookmark{Introduction}{intro}
\maketitle
{\centerline
{\small\it Dedicated to Professor H. B. Curry on the occasion of his 80th Birthday
%
\footnote{ This is a remake of the paper {\it From $\l$-Calculus to Cartesian Closed
Categories} originally published in R. Hindley and J. Seldin, editors, {\it To H.B. Curry:
Essays in Combinatory Logic, Lambda Calculus and Formalisms}. Academic Press, 1980. This
file was created in March 2023.}}} \bigskip\bigskip

\noindent
Haskell Curry may be surprised to hear that he has spent a lifetime doing fundamental work
in category theory. The purpose of this account is to convince categorists that Cartesian
closed categories (Eilenberg and Kelly, 1966) have been anticipated by logicians (Curry,
1930) by many years and, conversely, to per­ suade logicians that combinatory logic may
benefit from being phrased in categorical language.

I have attempted to tell this story twice before (1972, 1974), but am not entirely
satisfied with these earlier accounts. The present exposition is essentially my
unscheduled talk at the 1977 Durham Symposium on applications of sheaf theory to logic,
algebra and analysis.

I regret that limitations of space do not permit a discus­ sion of illative combinatory
logic (Curry and Feys, 1958) or combinatory type theory (Church, 1940) and applications
thereof to the construction of free toposes.

Let me confess at once that I am not a historical scholar and that I have taken some
liberties with the original material. Thus, I have taken the opportunity to present the
early discov­eries of combinatory logic in the language of universal algebra.

Our story begins in 1924, when \schon\ studied what would now be called an algebra $A =
(|A|, \app, \cI, \cS, \cK)$ consisting of a set $|A|$ equipped with a binary operation
$\app$ and constants $\cI$, $\cS$ and $\cK$. These were to satisfy the following
identities:
\be
\cI \app a = a,
\ee
\be
(\cK \app a) \app b = a,
\ee
\be
((\cS \app f) \app g) \app c = (f \app c) \app\, (g \app c), 
%
\footnote{Editor's note: Lambek uses this squiggle $\app$ to denote a binary operation for
function application. It appears here and in his later book about this same subject. I
made a best guess about how to translate this into \LaTeX.}
\ee

\noindent
for all elements $a$, $b$, $c$, $f$ and $g$ of $|A|$.
Actually, \schon\ did not employ the language of universal algebra, and he defined $\cI$ in terms of $\cK$ and $\cS$, but of this we shall speak later. His main result would now be stated as follows.

\pdfbookmark{Schonfinkel Algebra}{prop1}
\begin{prop}
Every polynomial $\varphi(x)$ over a \schon\ algebra $A$ can be written in the form $f \app x$, where $f \in |A|$.
\end{prop}
\noindent
Polynomials are of course formed as words in an indeterminate $x$ and are subject to the same three identities. 
More precisely, equality $\eqx$ between polynomials is the smallest equivalence relation $\equiv$ between words in $x$ which has the substitution property
\bes
\inferrule {\varphi(x) \equiv \psi(x) \qquad \alpha(x) \equiv \beta(x)}{\varphi(x) \app \alpha(x) \equiv \psi(x) \app \beta(x)}
\tag{$0_{X}$}
\ees
and which satisfies
\bes
\cI \app \alpha(x) \equiv \alpha(x),
\tag{$1_X$}
\ees
\bes
\tag{$2_X$}
(K \app \alpha(x))\app\,\beta(x) \equiv \alpha(x)
\ees
\bes
((S\app \varphi(x))\app \psi(x))\app \gamma(x) \equiv (\varphi(x) \app \gamma(x)) \app \, (\psi(x) \app \gamma(x))
\tag{$3_X$}
\ees
Alternatively, one may regard the polynomial $\varphi(x)$ as an element of the \schon\
algebra $A[x]$, which comes equipped with an element $x$ and with a homomorphism $h_x :A
\to A[x]$ with the usual universal property: for every algebra $B$ every homomorphism $f:
A \to B$ and every element $b\in |B|$, there exists a unique homomorphism $f\pr : A[x] \to
B$ such that $f\pr h_x = f$ and $f\pr (x) = b$. In the special case when $B = A$ and $f$
is the identity homomorphism $A \to A$, $f\pr$ is the {\it substitution} homomorphism
which allows us to replace $x$ in any polynomial by $b$. Similarly, when $B=A[x]$ and
$f=h_x$, $f\pr$ allows us to replace $x$ by $a$ polynomial $\beta(x)$.

The proof of Proposition 1 is remarkably simple. It proceeds by induction on the length of the word $\varphi(x)$,
which must be either $x$, or a constant or of the form $\psi(x)\app \chi(x)$ not constant,
in which last case we may assume by inductional assumption that $\psi(x) \eqx g \app x$ and
$\chi(x) \eqx h \app x$. In the three cases we have
\begin{align*}
\varphi(x) &\eqx \cI \app x,\\
\varphi(x) &\eqx (\cK \app a) \app x,\\
\varphi(x) &\eqx (g \app x)\app (h \app x) \eqx ((\cS \app g) \app h) \app x,
\label{foo}
\end{align*}
respectively.

This proof also yields an algorithm for converting every polynomial into the form $f \app x$.
For example, one easily calculates
$$
x \app x \eqx ((\cS \app \cI) \app \cI) \app x.
$$
In 1930, \schon's result was rediscovered by Curry. However, Curry was interested in imposing an
additional requirement:
\be
\hbox{\rm If } f\app x \eqx g \app x \hbox{\rm\ in\ } A[x] \hbox{\rm\ then } f= g\hbox{\rm\ in\ } A. \tag{4}
\label{4}
\ee
For example, from
$$
((\cS \app \cK)\app I) \app x \eqx (\cK\app x)\app (\cI \app x) \eqx x \eqx \cI\app x
$$
one could use (4) to obtain $(\cS\app \cK) \app \cI = I$, which equation cannot be derived
from (1) to (3) alone. In the same way, one could deduce that $(\cS\app \cK) \app \cK =
\cI$. In fact, \schon\ originally defined $\cI$ by this equation. For reasons that will
become clear later, we shall not follow him in this application of Occam's razor.

While (4) does not have the form of an identity, by which I mean an equation prefixed by
universal quantifiers, Curry discovered that it could be replaced by a finite number of
equations. These were later simplified, and Rosenbloom lists four, one of which reads:
$$
(\cS \app ((\cS \app (\cK \app \cS)) \app \cK))\app\, (\cK\app \cI) = \cI.
$$
The reader will forgive us for not copying out the other three!

\pdfbookmark{Curry Algebra}{prop2}
By a {\it Curry algebra} we shall mean a \schon\ algebra subject to certain additional
equations or identities whose conjunction is equivalent to (4). Curry's result may then be
formulated thus:
\begin{prop}
Over a Curry algebra $A$ every polynomial $\varphi(x)$ may be uniquely written in the form
$f\app x$ with $f \in |A|$.
\end{prop}

\noindent
This property of Curry algebras is called {\it functional completeness}. It is an
immediate consequence of the equivalence of (4) with the conjunction of the above
mentioned four equations. For a proof we could refer the reader to the book by Rosenbloom.
However, we prefer to give another proof, in the course of which we shall discover five
equations whose conjunction is equivalent to (4).

\begin{proof}
Let $\lx\varphi(x)$ be defined by induction on the length of the word $\varphi(x)$ thus:
\begin{enumerate}
\item[(i)] $\lx x = \cI$,
\item[(ii)] $\lx a = \cK\app a$, when $a$ is a constant;
\item[(iii)] $\lx(\psi(x)\app\chi(x)) = (S\app\lx\psi(x))\app\lx \chi(x)$ when 
$\psi(x)$ and $\chi(x)$ are not both constant.
\end{enumerate}
We shall prove below that the restriction on (iii) is not necessary.
In view of the above proof of Proposition 1, we have
$$
\varphi(x) = \lx \varphi(x) \app x
$$
so the existence of $f$ with $\varphi(x) \eqx f\app x$ is assured.
It remains to prove its uniqueness. First, we claim that
\bes
\varphi(x) \eqx \psi(x) \quad \hbox{\rm implies} \quad \lx \varphi(x) = \lx \psi(x),
\tag{*}
\ees
so that $\lx \varphi(x)$ depends not just on the word $\varphi(x)$
but on the polynomial $\varphi(x)$, that is, the word modulo the equivalence
relation $\eqx$.

To prove (*), we write $\varphi(x) \equiv \psi(x)$ for $\lx \varphi(x) = \lx \psi(x)$.
It is easily seen that $\equiv$ is an equivalence relation between words
which satisfies ($0_X$), that is, a congruence relation. If we make sure that $\equiv$
also satisfies ($1_X$) to ($3_X$) it will follow that $\equiv$ contains $\eqx$,
and this is what (*) asserts.

To say that = satisfies ($1_X$) means that 
$$
\lx(\cI \app\alpha(x)) = \lx \alpha(x).
$$

Writing $a$ for $\lx \alpha(x)$X we may rewrite this, in view of the unrestricted (iii), as
$$
(\cS\app (\cK\app \cI))\app a = a,
$$
an easy consequence of
\bes
\cS\app(\cK\app\cI) = \cI,
\tag{4.1}
\ees
which itself has already been derived from (4).

In the same manner we may obtain consequences (4.2) and (4.3)
of (4) which imply ($2_X$) and ($3_X$) respectively.
We shall not bother to spell them out.

We now turn to the uniqueness of $f$ in Proposition 2.
Suppose also $g\app x \eqx \varphi(x)$, we claim that $g = \lx \varphi(x)$.
Now, by (*) $\lx (g\app x) = \lx \varphi(x)$, so it suffices to
prove that $\lx (g\app x) = g$.

Now a small calculation shows that $\lx (g\app x) = (\cS\app(\cK \app g))\app \cI$,
so we require the identity
$$
(\cS\app (\cK\app g)) \app \cI = g
$$
for all $g$, which is an easy consequence of (4).

This identity must remain valid if we adjoin an indeterminate
$y$ to the algebra, so we have
$$
(\cS\app (\cK\app y)) \app \cI \eqy y.
$$
We may therefore replace the required identity by the equation
\bes
\ly (\cS\app (\cK\app y)) \app \cI = y.
\tag{4.4}
\ees
Of course the A may be eliminated from this using (i) to (iii).

It remains to show the validity of (iii) when $\psi(x) \app \chi(x)$ is
constant, say $b\app c$. So we want to show that
$$
\lx (b\app c) = (\cS\app(\cK \app b))\app (\cK \app c).
$$
But, by (ii), $\lx(b\app c) = \cK \app (b\app c)$, so we are led to
stipulate the identity
$$
(\cS\app (\cK \app b))\app (\cK \app c) = \cK \app (b\app c)
$$
for all $b$ and $c$. gain, this is an easy consequence of (4).
By the same argument as above, we may replace the stipulated identity
by the equation
\bes
\lx\ly ((\cS\app (\cK\app x))\app (\cK\app y))= \lx \ly (\cK\app (x\app y)).
\tag{4.5}
\ees
The proof of Proposition 2 is now complete, provided we adopt
(4.1) to (4.5) as the five equations which a Curry algebra must
satisfy in addition to the identities (1) to (3).
\end{proof}
\pdfbookmark{Lambda Calculus}{prop3}
From now on we shall write $\lx \varphi(x)$ for the unique $f$ corresponding
to $\varphi(x)$by Proposition 2, as we did in the proof. The properties of
the new symbol $\lx$ are embodied in the $\lambda$-calculus of Church (1932).
The equivalence of the systems of Curry and Church ($\lambda K$-calculus, 1941)
are summed up as follows.

\begin{prop}The identities and equations of Curry algebras imply and are implied by the following:
\begin{enumerate}
\item[{\rm (1)}]$\cI = \lx x$,
\item[{\rm (2)}]$\cK = \lx \ly x$,
\item[{\rm (3)}]$\cS = \lu\lv\lz ((u\app z)\app (v\app z))$,
\item[{\rm (4)}]$\lx (f\app x)= f$,
\item[{\rm (5)}]$(\lx \varphi(x))\app a = \varphi(a)$.
\end{enumerate}
\end{prop}
The proof is almost straightforward. We shall only explain
why (5) holds for Curry algebras. By Proposition 2, we have
$f \app x \eqx \varphi(x)$ and we want to deduce from this that
$f \app a =  \varphi(a)$. by the universal property of $A[x]$, there
exists a unique homomorphism $h\pr: A[x] \to A$ such that
$h\pr = h_x$ and $h\pr(x) = a$. This is of course the substitution
homomorphism which replaces $x$ by $a$, hence yields the required equation from
$f \app x \eqx \varphi(x)$.

\pdfbookmark{Combinatory Logic}{combinator}
To recapture the traditional terminology, let us mention that the theory of
Curry algebras is called {\it combinatory logic}.
Proposition 2 may then be compressed into the slogan:
$$
\hbox{\rm\ combinatory logic } = \,\lambda\hbox{\rm -calculus.}
$$
Incidentally, {\it combinators} are the canonical elements of Curry
(or Schonfinkel) algebras, that is, the elements of the free
Curry algebra generated by the empty set.

Both \schon\ and Curry had intended to use combinatory logic for
the foundations of mathematics. An obstacle arose in the following result.

\begin{thebibliography}{888}


\end{thebibliography}
