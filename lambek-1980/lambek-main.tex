%!TEX root = lambek-1980-pal.tex
%\usepackage[dotinlabels]{titletoc}
%\titlelabel{{\thetitle}.\quad}
%\usepackage{titletoc}
\usepackage[small]{titlesec}

\titleformat{\section}[block]
  {\fillast\medskip}
  {\bfseries{\thesection. }}
  {1ex minus .1ex}
  {\bfseries}
 
\titleformat*{\subsection}{\itshape}
\titleformat*{\subsubsection}{\itshape}

\setcounter{tocdepth}{2}

\titlecontents{section}
              [2.3em] 
              {\bigskip}
              {{\contentslabel{2.3em}}}
              {\hspace*{-2.3em}}
              {\titlerule*[1pc]{}\contentspage}
              
\titlecontents{subsection}
              [4.7em] 
              {}
              {{\contentslabel{2.3em}}}
              {\hspace*{-2.3em}}
              {\titlerule*[.5pc]{}\contentspage}

% hopefully not used.           
\titlecontents{subsubsection}
              [7.9em]
              {}
              {{\contentslabel{3.3em}}}
              {\hspace*{-3.3em}}
              {\titlerule*[.5pc]{}\contentspage}
%\makeatletter
\renewcommand\tableofcontents{%
    \section*{\contentsname
        \@mkboth{%
           \MakeLowercase\contentsname}{\MakeLowercase\contentsname}}%
    \@starttoc{toc}%
    }
\def\@oddhead{{\scshape\rightmark}\hfil{\small\scshape\thepage}}%
\def\sectionmark#1{%
      \markright{\MakeLowercase{%
        \ifnum \c@secnumdepth >\m@ne
          \thesection\quad
        \fi
        #1}}}
        
\makeatother

%\makeatletter

 \def\small{%
  \@setfontsize\small\@xipt{13pt}%
  \abovedisplayskip 8\p@ \@plus3\p@ \@minus6\p@
  \belowdisplayskip \abovedisplayskip
  \abovedisplayshortskip \z@ \@plus3\p@
  \belowdisplayshortskip 6.5\p@ \@plus3.5\p@ \@minus3\p@
  \def\@listi{%
    \leftmargin\leftmargini
    \topsep 9\p@ \@plus3\p@ \@minus5\p@
    \parsep 4.5\p@ \@plus2\p@ \@minus\p@
    \itemsep \parsep
  }%
}%
 \def\footnotesize{%
  \@setfontsize\footnotesize\@xpt{12pt}%
  \abovedisplayskip 10\p@ \@plus2\p@ \@minus5\p@
  \belowdisplayskip \abovedisplayskip
  \abovedisplayshortskip \z@ \@plus3\p@
  \belowdisplayshortskip 6\p@ \@plus3\p@ \@minus3\p@
  \def\@listi{%
    \leftmargin\leftmargini
    \topsep 6\p@ \@plus2\p@ \@minus2\p@
    \parsep 3\p@ \@plus2\p@ \@minus\p@
    \itemsep \parsep
  }%
}%
\def\open@column@one#1{%
 \ltxgrid@info@sw{\class@info{\string\open@column@one\string#1}}{}%
 \unvbox\pagesofar
 \@ifvoid{\footsofar}{}{%
  \insert\footins\bgroup\unvbox\footsofar\egroup
  \penalty\z@
 }%
 \gdef\thepagegrid{one}%
 \global\pagegrid@col#1%
 \global\pagegrid@cur\@ne
 \global\count\footins\@m
 \set@column@hsize\pagegrid@col
 \set@colht
}%

\def\frontmatter@abstractheading{%
\bigskip
 \begingroup
  \centering\large
  \abstractname
  \par\bigskip
 \endgroup
}%

\makeatother

%\DeclareSymbolFont{CMlargesymbols}{OMX}{cmex}{m}{n}
%\DeclareMathSymbol{\sum}{\mathop}{CMlargesymbols}{"50}

\usepackage[papersize={6.6in, 10.0in}, left=.5in, right=.5in, top=.6in, bottom=.9in]{geometry}
\linespread{1.05}
%\sloppy
%\raggedbottom
\usepackage[leqno]{amsmath}

\pagestyle{plain}
\usepackage{mathpartir}
\usepackage{stmaryrd}
\usepackage{mathtools}
\usepackage{tikz-cd}
\usepackage{microtype}
\usepackage{amssymb}
\usepackage{enumitem}

\newcommand{\mprime}{\ensuremath{^\prime}}

%\usepackage{fdsymbol}

% these include amsmath and that can cause trouble in older docs.
\makeatletter
\@ifpackageloaded{amsmath}{}{\RequirePackage{amsmath}}

\DeclareFontFamily{U}  {cmex}{}
\DeclareSymbolFont{Csymbols}       {U}  {cmex}{m}{n}
\DeclareFontShape{U}{cmex}{m}{n}{
    <-6>  cmex5
   <6-7>  cmex6
   <7-8>  cmex6
   <8-9>  cmex7
   <9-10> cmex8
  <10-12> cmex9
  <12->   cmex10}{}

\def\Set@Mn@Sym#1{\@tempcnta #1\relax}
\def\Next@Mn@Sym{\advance\@tempcnta 1\relax}
\def\Prev@Mn@Sym{\advance\@tempcnta-1\relax}
\def\@Decl@Mn@Sym#1#2#3#4{\DeclareMathSymbol{#2}{#3}{#4}{#1}}
\def\Decl@Mn@Sym#1#2#3{%
  \if\relax\noexpand#1%
    \let#1\undefined
  \fi
  \expandafter\@Decl@Mn@Sym\expandafter{\the\@tempcnta}{#1}{#3}{#2}%
  \Next@Mn@Sym}
\def\Decl@Mn@Alias#1#2#3{\Prev@Mn@Sym\Decl@Mn@Sym{#1}{#2}{#3}}
\let\Decl@Mn@Char\Decl@Mn@Sym
\def\Decl@Mn@Op#1#2#3{\def#1{\DOTSB#3\slimits@}}
\def\Decl@Mn@Int#1#2#3{\def#1{\DOTSI#3\ilimits@}}

\let\sum\undefined
\DeclareMathSymbol{\tsum}{\mathop}{Csymbols}{"50}
\DeclareMathSymbol{\dsum}{\mathop}{Csymbols}{"51}

\Decl@Mn@Op\sum\dsum\tsum

\makeatother

\makeatletter
\@ifpackageloaded{amsmath}{}{\RequirePackage{amsmath}}

\DeclareFontFamily{OMX}{MnSymbolE}{}
\DeclareSymbolFont{largesymbolsX}{OMX}{MnSymbolE}{m}{n}
\DeclareFontShape{OMX}{MnSymbolE}{m}{n}{
    <-6>  MnSymbolE5
   <6-7>  MnSymbolE6
   <7-8>  MnSymbolE7
   <8-9>  MnSymbolE8
   <9-10> MnSymbolE9
  <10-12> MnSymbolE10
  <12->   MnSymbolE12}{}

\DeclareMathSymbol{\downbrace}    {\mathord}{largesymbolsX}{'251}
\DeclareMathSymbol{\downbraceg}   {\mathord}{largesymbolsX}{'252}
\DeclareMathSymbol{\downbracegg}  {\mathord}{largesymbolsX}{'253}
\DeclareMathSymbol{\downbraceggg} {\mathord}{largesymbolsX}{'254}
\DeclareMathSymbol{\downbracegggg}{\mathord}{largesymbolsX}{'255}
\DeclareMathSymbol{\upbrace}      {\mathord}{largesymbolsX}{'256}
\DeclareMathSymbol{\upbraceg}     {\mathord}{largesymbolsX}{'257}
\DeclareMathSymbol{\upbracegg}    {\mathord}{largesymbolsX}{'260}
\DeclareMathSymbol{\upbraceggg}   {\mathord}{largesymbolsX}{'261}
\DeclareMathSymbol{\upbracegggg}  {\mathord}{largesymbolsX}{'262}
\DeclareMathSymbol{\braceld}      {\mathord}{largesymbolsX}{'263}
\DeclareMathSymbol{\bracelu}      {\mathord}{largesymbolsX}{'264}
\DeclareMathSymbol{\bracerd}      {\mathord}{largesymbolsX}{'265}
\DeclareMathSymbol{\braceru}      {\mathord}{largesymbolsX}{'266}
\DeclareMathSymbol{\bracemd}      {\mathord}{largesymbolsX}{'267}
\DeclareMathSymbol{\bracemu}      {\mathord}{largesymbolsX}{'270}
\DeclareMathSymbol{\bracemid}     {\mathord}{largesymbolsX}{'271}

\def\horiz@expandable#1#2#3#4#5#6#7#8{%
  \@mathmeasure\z@#7{#8}%
  \@tempdima=\wd\z@
  \@mathmeasure\z@#7{#1}%
  \ifdim\noexpand\wd\z@>\@tempdima
    $\m@th#7#1$%
  \else
    \@mathmeasure\z@#7{#2}%
    \ifdim\noexpand\wd\z@>\@tempdima
      $\m@th#7#2$%
    \else
      \@mathmeasure\z@#7{#3}%
      \ifdim\noexpand\wd\z@>\@tempdima
        $\m@th#7#3$%
      \else
        \@mathmeasure\z@#7{#4}%
        \ifdim\noexpand\wd\z@>\@tempdima
          $\m@th#7#4$%
        \else
          \@mathmeasure\z@#7{#5}%
          \ifdim\noexpand\wd\z@>\@tempdima
            $\m@th#7#5$%
          \else
           #6#7%
          \fi
        \fi
      \fi
    \fi
  \fi}

\def\overbrace@expandable#1#2#3{\vbox{\m@th\ialign{##\crcr
  #1#2{#3}\crcr\noalign{\kern2\p@\nointerlineskip}%
  $\m@th\hfil#2#3\hfil$\crcr}}}
\def\underbrace@expandable#1#2#3{\vtop{\m@th\ialign{##\crcr
  $\m@th\hfil#2#3\hfil$\crcr
  \noalign{\kern2\p@\nointerlineskip}%
  #1#2{#3}\crcr}}}

\def\overbrace@#1#2#3{\vbox{\m@th\ialign{##\crcr
  #1#2\crcr\noalign{\kern2\p@\nointerlineskip}%
  $\m@th\hfil#2#3\hfil$\crcr}}}
\def\underbrace@#1#2#3{\vtop{\m@th\ialign{##\crcr
  $\m@th\hfil#2#3\hfil$\crcr
  \noalign{\kern2\p@\nointerlineskip}%
  #1#2\crcr}}}

\def\bracefill@#1#2#3#4#5{$\m@th#5#1\leaders\hbox{$#4$}\hfill#2\leaders\hbox{$#4$}\hfill#3$}

\def\downbracefill@{\bracefill@\braceld\bracemd\bracerd\bracemid}
\def\upbracefill@{\bracefill@\bracelu\bracemu\braceru\bracemid}

\DeclareRobustCommand{\downbracefill}{\downbracefill@\textstyle}
\DeclareRobustCommand{\upbracefill}{\upbracefill@\textstyle}

\def\upbrace@expandable{%
  \horiz@expandable
    \upbrace
    \upbraceg
    \upbracegg
    \upbraceggg
    \upbracegggg
    \upbracefill@}
\def\downbrace@expandable{%
  \horiz@expandable
    \downbrace
    \downbraceg
    \downbracegg
    \downbraceggg
    \downbracegggg
    \downbracefill@}

\DeclareRobustCommand{\overbrace}[1]{\mathop{\mathpalette{\overbrace@expandable\downbrace@expandable}{#1}}\limits}
\DeclareRobustCommand{\underbrace}[1]{\mathop{\mathpalette{\underbrace@expandable\upbrace@expandable}{#1}}\limits}

\makeatother


% some nicer symbols
\makeatletter
\DeclareFontFamily{U}{matha}{\hyphenchar\font45}
\DeclareFontShape{U}{matha}{m}{n}{
      <5> <6> <7> <8> <9> <10> gen * matha
      <10.95> matha10 <12> <14.4> <17.28> <20.74> <24.88> matha12
      }{}
\DeclareSymbolFont{matha}{U}{matha}{m}{n}
\DeclareFontSubstitution{U}{matha}{m}{n}

\def\mathabx@aliases#1#2{\@mathabx@aliases#1#2?\@end}
\def\@mathabx@aliases#1#2#3\@end{\ifx#2?\else
	\let#2=#1\@mathabx@aliases#1#3\@end\fi}%
\DeclareMathSymbol{\leftarrow}             {3}{matha}{"D0}
	\mathabx@aliases\leftarrow\gets
\DeclareMathSymbol{\rightarrow}            {3}{matha}{"D1}
	\mathabx@aliases\rightarrow\to
\DeclareMathSymbol{\wedge}         {2}{matha}{"5E}
	\mathabx@aliases\wedge\land
\DeclareMathSymbol{\vee}           {2}{matha}{"5F}
	\mathabx@aliases\vee\lor
\DeclareMathSymbol{\vdash}         {3}{matha}{"24}
\DeclareMathSymbol{\dashv}         {3}{matha}{"25}
\DeclareMathSymbol{\nvdash}        {3}{matha}{"26}
\DeclareMathSymbol{\ndashv}        {3}{matha}{"27}
\DeclareMathSymbol{\vDash}         {3}{matha}{"28}
\DeclareMathSymbol{\Dashv}         {3}{matha}{"29}
\DeclareMathSymbol{\nvDash}        {3}{matha}{"2A}
\DeclareMathSymbol{\nDashv}        {3}{matha}{"2B}
\DeclareMathSymbol{\Vdash}         {3}{matha}{"2C}
\DeclareMathSymbol{\dashV}         {3}{matha}{"2D}
\DeclareMathSymbol{\nVdash}        {3}{matha}{"2E}
\DeclareMathSymbol{\ndashV}        {3}{matha}{"2F}
\makeatother

\usepackage[small]{titlesec}
\usepackage{cite}

% make sure there is enough TOC for reasonable pdf bookmarks.
\setcounter{tocdepth}{3}
\usepackage{amsthm}

\newtheorem{theorem}{Theorem}
\newtheorem{prop}{Proposition}
\newtheorem*{corollary}{Corollary}

\theoremstyle{remark}
\newtheorem{remark}{Remark}

\usepackage[colorlinks=true
,breaklinks=true
,urlcolor=blue
,anchorcolor=blue
,citecolor=blue
,filecolor=blue
,linkcolor=blue
,menucolor=blue
,linktocpage=true]{hyperref}
\hypersetup{
bookmarksopen=true,
bookmarksnumbered=true,
bookmarksopenlevel=10,
}

\date{}
\def\mm{\Vdash}
\def\kxa{\kappa_{x \in A}}
\def\prAB{\pi_{A,B}}
\def\prpAB{\pi'_{A,B}}
\def\pr#1{\pi_{#1}}
\def\prp#1{\pi'_{#1}}
\def\to{\longrightarrow}
\def\xto#1{\xrightarrow{\kern.6em #1 \kern.6em}}
\def\ent{\longrightarrow}
\def\imp{\shortrightarrow}
\def\iff{\leftrightarrow}
\def\from{\Leftarrow}
\def\union{\cup}
\def\inc{\subseteq}
\def\dom{\mathop{\rm dom}}
\def\cod{\mathop{\rm cod}}
\def\id{{\mathrm 1}}
\def\res{\!\upharpoonleft\!}
\def\ffam{\varphi}
\def\comp{\circ}
\def\bbone{\mathbb 1}
\def\zeromap{0}
\def\bbzero{{\mathbb O}}
\def\ccc{{c.c.c.}}
\def\ev{\varepsilon}
\def\ebc{\varepsilon_{BC}}
\def\L{\Lambda}
\def\l{\lambda}
\def\lx{\lambda_x}
\def\ly{\lambda_y}
\def\lu{\lambda_u}
\def\lv{\lambda_v}
\def\lz{\lambda_z}
\def\subX{\ensuremath{_X}}
\def\lm#1.#2{\lambda#1.\, #2}
\def\br#1{[\, #1 \, ]}
\def\V{V}
\def\U{U}
\def\D{D}
\def\C{\mathcal C}
\def\S{\mathcal S}
\def\lxy{\l x\, \l y . \,}
\def\lmm#1#2.#3{\l #1\, \l #2 . \, #3}
\def\sss{(*\!*\!*)}
\def\ss{(**)}
\def\ssn{(**_n)}
\def\scop{\S^{\C^{op}}}
\def\sland{\wedge}
\def\PU{\mathcal P U}
\def\P{\mathcal P}
\def\UU{(U\to U)}
\def\BA{B \to A}
\def\AB{A \to B}
\def\calA{{\cal A}}
\def\calB{{\cal B}}
\def\cI{{I}}
\def\cS{{S}}
\def\cK{{K}}
\def\cIA{\cI_{A}}
\def\cKAB{\cK_{A,B}}
\def\cSABC{\cS_{A,B,C}}
\def\la{\langle}
\def\ra{\rangle}
\def\bracket#1{\la #1 \ra}
\def\app{\mathop{{}^\wr}\kern-.8pt}
\def\schon{Sch\"onfinkel}
\newcommand{\be}{\begin{equation}}
\newcommand{\ee}{\end{equation}}
\newcommand{\bes}{\begin{equation*}}
\newcommand{\ees}{\end{equation*}}

% makes "=" with "x" under it
\makeatletter
\DeclareRobustCommand{\eqx}{\mathrel{\mathpalette\eq@{x}}}
\DeclareRobustCommand{\eqy}{\mathrel{\mathpalette\eq@{y}}}
\newcommand{\eq@}[2]{%
  \vtop{\offinterlineskip
    \ialign{\hfil##\hfil\cr
      $\m@th#1=$\cr % top
      \noalign{\sbox\z@{$\m@th#1\mkern0mu$}\kern-\wd\z@}
      $\m@th\alexey@demote{#1}#2$\cr
    }%
  }%
}
\newcommand{\alexey@demote}[1]{%
  \ifx#1\displaystyle\scriptstyle\else
  \ifx#1\textstyle\scriptstyle\else
  \scriptscriptstyle\fi\fi
}
\makeatother

% footnote tricks
\usepackage[symbol]{footmisc}
\renewcommand{\thefootnote}{\fnsymbol{footnote}}

\makeatletter
\let\original@footnotemark\footnotemark
\newcommand{\align@footnotemark}{%
  \ifmeasuring@
    \chardef\@tempfn=\value{footnote}%
    \original@footnotemark
    \setcounter{footnote}{\@tempfn}%
  \else
    \iffirstchoice@
      \original@footnotemark
    \fi
  \fi}
\pretocmd{\start@align}{\let\footnotemark\align@footnotemark}{}{}
\makeatother

\makeatletter
\newcommand*\dotop{\mathpalette\bigcdot@{.6}}
\newcommand*\bigcdot@[2]{\mathbin{\vcenter{\hbox{\scalebox{#2}{$\m@th#1\bullet$}}}}}
\makeatother

\title{\large From $\l$-Calculus to Cartesian Closed Categories}
\author{\normalsize J. Lambek \\
{\small\it Mathematics Department, McGill University} \\
{\small\it Montreal, P.Q. HSA 2K63 Canada.}}
\begin{document}
\beginhook
\pdfbookmark{Introduction}{intro}
\maketitle
{\centerline
{\small\it Dedicated to Professor H. B. Curry on the occasion of his 80th Birthday
%
\footnote{ This is a remake of the paper {\it From $\l$-Calculus to Cartesian Closed
Categories} originally published in R. Hindley and J. Seldin, editors, {\it To H.B. Curry:
Essays in Combinatory Logic, Lambda Calculus and Formalisms}. Academic Press, 1980. This
file was created in April 2023.}}} \bigskip\bigskip

\noindent
Haskell Curry may be surprised to hear that he has spent a lifetime doing fundamental work
in category theory. The purpose of this account is to convince categorists that Cartesian
closed categories (Eilenberg and Kelly, 1966) have been anticipated by logicians (Curry,
1930) by many years and, conversely, to persuade logicians that combinatory logic may
benefit from being phrased in categorical language.

I have attempted to tell this story twice before (1972, 1974), but am not entirely
satisfied with these earlier accounts. The present exposition is essentially my
unscheduled talk at the 1977 Durham Symposium on applications of sheaf theory to logic,
algebra and analysis.

I regret that limitations of space do not permit a discus­sion of illative combinatory
logic (Curry and Feys, 1958) or combinatory type theory (Church, 1940) and applications
thereof to the construction of free toposes.

Let me confess at once that I am not a historical scholar and that I have taken some
liberties with the original material. Thus, I have taken the opportunity to present the
early discoveries of combinatory logic in the language of universal algebra.

Our story begins in 1924, when \schon\ studied what would now be called an algebra $\calA =
(|\calA|, \app, \cI, \cS, \cK)$ consisting of a set $|\calA|$ equipped with a binary operation
$\app$ and constants $\cI$, $\cS$ and $\cK$. These were to satisfy the following
identities:%
\footnote{Editor's note: Lambek uses this squiggle $\app$ to denote a binary operation for
function application. It appears here and in his later book about this same subject. I
made a best guess about how to translate this into \LaTeX.}
\begin{align*}
\cI \app a &= a,\tag{1}\\
(\cK \app a) \app b &= a,\tag{2}\\
((\cS \app f) \app g) \app c &= (f \app c) \app\, (g \app c),\tag{3}
\end{align*}
\noindent
for all elements $a$, $b$, $c$, $f$ and $g$ of $|\calA|$.
Actually, \schon\ did not employ the language of universal algebra, and he defined $\cI$ in
terms of $\cK$ and $\cS$, but of this we shall speak later. 
\pdfbookmark{Schonfinkel Algebra}{prop1}%
His main result would now be stated as follows.
\begin{prop}
Every polynomial $\varphi(x)$ over a \schon\ algebra $\calA$ can be written in the form $f \app x$, where $f \in |\calA|$.
\end{prop}
\noindent
Polynomials are of course formed as words in an indeterminate $x$ and are subject to the same three identities. 
More precisely, equality $\eqx$ between polynomials is the smallest equivalence relation $\equiv$ between words
in $x$ which has the substitution property
\bes
\inferrule {\varphi(x) \equiv \psi(x) \qquad \alpha(x) \equiv \beta(x)}{\varphi(x) \app \alpha(x) \equiv \psi(x) \app \beta(x)}
\tag{0\subX}
\ees
and which satisfies
\begin{align*}
\cI \app \alpha(x) &\equiv \alpha(x),
\tag{1\subX}\\
(K \app \alpha(x))\app\,\beta(x) &\equiv \alpha(x)
\tag{2\subX}\\
((S\app \varphi(x))\app \psi(x))\app \gamma(x) &\equiv (\varphi(x) \app \gamma(x)) \app \, (\psi(x) \app \gamma(x))
\tag{3\subX}
\end{align*}
Alternatively, one may regard the polynomial $\varphi(x)$ as an element of the
Sch\"on\-finkel algebra $\calA[x]$, which comes equipped with an element $x$ and with a homomorphism $h_x : \calA
\to \calA[x]$ with the usual universal property: for every algebra $\cal B$, every homomorphism $f:
\calA \to \calB$, and every element $b\in |\calB|$, there exists a unique homomorphism $f' : \calA[x] \to
\calB$ such that $f' h_x = f$ and $f' (x) = b$. In the special case when $\calB = \calA$ and $f$
is the identity homomorphism $\calA \to \calA$, and $f'$ is the {\it substitution} homomorphism
which allows us to replace $x$ in any polynomial by $b$. Similarly, when $\calB=\calA[x]$ and
$f=h_x$, $f'$ allows us to replace $x$ by $a$ polynomial $\beta(x)$.

The proof of Proposition 1 is remarkably simple. It proceeds by induction on the length of the word $\varphi(x)$,
which must be either $x$, or a constant or of the form $\psi(x)\app \chi(x)$ not constant,
in which last case we may assume by inductional assumption that $\psi(x) \eqx g \app x$ and
$\chi(x) \eqx h \app x$. In the three cases we have
\begin{align*}
\varphi(x) &\eqx \cI \app x,\\
\varphi(x) &\eqx (\cK \app a) \app x,\\
\varphi(x) &\eqx (g \app x)\app (h \app x) \eqx ((\cS \app g) \app h) \app x,
\label{foo}
\end{align*}
respectively.

This proof also yields an algorithm for converting every polynomial into the form $f \app x$.
For example, one easily calculates
$$
x \app x \eqx ((\cS \app \cI) \app \cI) \app x.
$$
In 1930, \schon's result was rediscovered by Curry. However, Curry was interested in imposing an
additional requirement:
\be
\hbox{\rm If } f\app x \eqx g \app x \hbox{\rm\ in\ } \calA[x] \hbox{\rm\ then } f= g\hbox{\rm\ in\ } \calA. \tag{4}
\label{4}
\ee
For example, from
$$
((\cS \app \cK)\app I) \app x \eqx (\cK\app x)\app (\cI \app x) \eqx x \eqx \cI\app x
$$
one could use (4) to obtain $(\cS\app \cK) \app \cI = I$, which equation cannot be derived
from (1) to (3) alone. In the same way, one could deduce that $(\cS\app \cK) \app \cK =
\cI$. In fact, \schon\ originally defined $\cI$ by this equation. For reasons that will
become clear later, we shall not follow him in this application of Occam's razor.

While (4) does not have the form of an identity, by which I mean an equation prefixed by
universal quantifiers, Curry discovered that it could be replaced by a finite number of
equations. These were later simplified, and Rosenbloom lists four, one of which reads:
$$
(\cS \app ((\cS \app (\cK \app \cS)) \app \cK))\app\, (\cK\app \cI) = \cI.
$$
The reader will forgive us for not copying out the other three!

\pdfbookmark{Curry Algebra}{prop2}
By a {\it Curry algebra} we shall mean a \schon\ algebra subject to certain additional
equations or identities whose conjunction is equivalent to (4). Curry's result may then be
formulated thus:
\begin{prop}
Over a Curry algebra $\calA$ every polynomial $\varphi(x)$ may be uniquely written in the form
$f\app x$ with $f \in |\calA|$.
\end{prop}

\noindent
This property of Curry algebras is called {\it functional completeness}. It is an
immediate consequence of the equivalence of (4) with the conjunction of the above
mentioned four equations. For a proof we could refer the reader to the book by Rosenbloom.
However, we prefer to give another proof, in the course of which we shall discover five
equations whose conjunction is equivalent to (4).

\begin{proof}
Let $\lx\varphi(x)$ be defined by induction on the length of the word $\varphi(x)$ thus:
\begin{enumerate}
\item[(i)] $\lx x = \cI$,
\item[(ii)] $\lx a = \cK\app a$, when $a$ is a constant;
\item[(iii)] $\lx(\psi(x)\app\chi(x)) = (S\app\lx\psi(x))\app\lx \chi(x)$ when 
$\psi(x)$ and $\chi(x)$ are not both constant.
\end{enumerate}
We shall prove below that the restriction on (iii) is not necessary.
In view of the above proof of Proposition 1, we have
$$
\varphi(x) \eqx \lx \varphi(x) \app x
$$
so the existence of $f$ with $\varphi(x) \eqx f\app x$ is assured.
It remains to prove its uniqueness. First, we claim that
\bes
\varphi(x) \eqx \psi(x) \quad \hbox{\rm implies} \quad \lx \varphi(x) = \lx \psi(x),
\tag{*}
\ees
so that $\lx \varphi(x)$ depends not just on the word $\varphi(x)$
but on the polynomial $\varphi(x)$, that is, the word modulo the equivalence
relation $\eqx$.

To prove (*), we write $\varphi(x) \equiv \psi(x)$ for $\lx \varphi(x) = \lx \psi(x)$.
It is easily seen that $\equiv$ is an equivalence relation between words
which satisfies (0\subX), that is, a congruence relation. If we make sure that $\equiv$
also satisfies (1\subX) to (3\subX) it will follow that $\equiv$ contains $\eqx$,
and this is what (*) asserts.

To say that $\equiv$ satisfies (1\subX) means that 
$$
\lx(\cI \app\alpha(x)) = \lx \alpha(x).
$$

Writing $a$ for $\lx \alpha(x)$ we may rewrite this, in view of the unrestricted (iii), as
$$
(\cS\app (\cK\app \cI))\app a = a,
$$
an easy consequence of
\bes
\cS\app(\cK\app\cI) = \cI,
\tag{4.1}
\ees
which itself has already been derived from (4).

In the same manner we may obtain consequences (4.2) and (4.3)
of (4) which imply (2\subX) and (3\subX) respectively.
We shall not bother to spell them out.%
\footnote{Editor's note: It's too bad they did not spell out the details here.}
We now turn to the uniqueness of $f$ in Proposition 2.
Suppose also $g\app x \eqx \varphi(x)$, we claim that $g = \lx \varphi(x)$.
Now, by (*), $\lx (g\app x) = \lx \varphi(x)$, so it suffices to
prove that $\lx (g\app x) = g$.

Now a small calculation shows that $\lx (g\app x) = (\cS\app(\cK \app g))\app \cI$,
so we require the identity
$$
(\cS\app (\cK\app g)) \app \cI = g
$$
for all $g$, which is an easy consequence of (4).

This identity must remain valid if we adjoin an indeterminate
$y$ to the algebra, so we have
$$
(\cS\app (\cK\app y)) \app \cI \eqy y.
$$
We may therefore replace the required identity by the equation
\bes
\ly (\cS\app (\cK\app y)) \app \cI = y.
\tag{4.4}
\ees
Of course the $\l$ may be eliminated from this using (i) to (iii).

It remains to show the validity of (iii) when $\psi(x) \app \chi(x)$ is
constant, say $b\app c$. So we want to show that
$$
\lx (b\app c) = (\cS\app(\cK \app b))\app (\cK \app c).
$$
But, by (ii), $\lx(b\app c) = \cK \app (b\app c)$, so we are led to
stipulate the identity
$$
(\cS\app (\cK \app b))\app (\cK \app c) = \cK \app (b\app c)
$$
for all $b$ and $c$. Again, this is an easy consequence of (4).
By the same argument as above, we may replace the stipulated identity
by the equation
\bes
\lx\ly ((\cS\app (\cK\app x))\app (\cK\app y))= \lx \ly (\cK\app (x\app y)).
\tag{4.5}
\ees
The proof of Proposition 2 is now complete, provided we adopt
(4.1) to (4.5)
as the five equations which a Curry algebra must satisfy
in addition to the identities (1) to (3).%
\end{proof}
\pdfbookmark{Lambda Calculus}{prop3}
From now on we shall write $\lx \varphi(x)$ for the unique $f$ corresponding
to $\varphi(x)$ by Proposition 2, as we did in the proof. The properties of
the new symbol $\lx$ are embodied in the $\lambda$-calculus of Church (1932).
The equivalence of the systems of Curry and Church ($\lambda K$-calculus, 1941)
are summed up as follows.
\renewcommand*{\thefootnote}{\arabic{footnote}}%
\setcounter{footnote}{0}%
% the start of the proposition here happens near the bottom
% of a page. these skips will force a page break if needed
% but if we were close enough to the bottom that TeX would
% break the page anyway they'll be ignored.
\bigskip
\bigskip
\bigskip
\bigskip
\bigskip
\bigskip
\begin{prop}The identities and equations of Curry algebras imply and are implied by the following:%
\footnote{Here $=$ denotes equality in $\calA$, $\calA[x]$, $\calA[x,y]$, etc. simultaneously,
subscripts having been suppressed. It is an equivalence relation subject to the following rules:
$$
\inferrule{f=g \qquad a = b}{f\app a = g\app b} \,\, , \,\, \inferrule{\varphi(x) = \psi(x)}{\lx \varphi(x) = \lx \psi(x)}
$$
}
\begin{enumerate}
\item[{\rm (1)}]$\cI = \lx x$,
\item[{\rm (2)}]$\cK = \lx \ly x$,
\item[{\rm (3)}]$\cS = \lu\lv\lz ((u\app z)\app (v\app z))$,
\item[{\rm (4)}]$\lx (f\app x)= f$,
\item[{\rm (5)}]$(\lx \varphi(x))\app a = \varphi(a)$.
\end{enumerate}
\end{prop}
The proof is almost straightforward. We shall only explain
why (5) holds for Curry algebras. By Proposition 2, we have
$f \app x \eqx \varphi(x)$ and we want to deduce from this that
$f \app a =  \varphi(a)$. By the universal property of $\calA[x]$, there
exists a unique homomorphism $h': \calA[x] \to \calA$ such that
$h' = h_x$ and $h'(x) = a$. This is of course the substitution
homomorphism which replaces $x$ by $a$, hence yields the required equation from
$f \app x \eqx \varphi(x)$.

\pdfbookmark{Combinatory Logic}{combinator}
To recapture the traditional terminology, let us mention that the theory of
Curry algebras is called {\it combinatory logic}.
Proposition 2 may then be compressed into the slogan:
$$
\hbox{\rm\ combinatory logic } = \,\lambda\hbox{\rm -calculus.}
$$
Incidentally, {\it combinators} are the canonical elements of Curry
(or \schon) algebras, that is, the elements of the free
Curry algebra generated by the empty set.

\pdfbookmark{Fixed Point}{prop4}
Both \schon\ and Curry had intended to use combinatory logic for
the foundations of mathematics. An obstacle arose in the following result.
\begin{prop}
In a Curry algebra every element $f$ has a ``fixpoint'' a such that $f\app a = a$.
\end{prop}
\begin{proof}
Let $b = \lx (f\app (x\app x))$ and put $a = b \app b$. Then
$$
f\app a = f \app (b \app b) = b \app b = a.
$$
\end{proof}
If $f$ is negation, usually denoted by $\lnot$, we have $\lnot a = a$,
so $a$ cannot be a proposition. We must therefore distinguish between propositions
and other entities; but even this distinction does not prevent
Russell's paradox from raising its head (e.g. Curry and Feys, 1958).

\pdfbookmark{Types and Proofs}{types}
It is of course well-known that Russell's paradox may be
avoided by introducing types. In the following exposition of
typed combinatory logic I shall follow Curry and Feys
in principle, even though I shall reject their permissiveness
in allowing symbols with ambiguous types.

First of all we replace a Curry algebra by a many-sorted algebra, whose elements,
which we call entities, may belong to different sorts, which we call types.
If $A$ and $B$ are types then so is $B^A$, the type of all ``functions'' from $A$ to $B$. For
typographical reasons, we write $B \from A$ in place of $B^A$ . The
binary operation symbol $\app$ may not be placed between entities indiscriminately,
but is subject to the following rule:
\bes
\inferrule {a \in A \qquad f \in B \from A}{f \app a \in B},
\tag{0}
\ees
meaning that, if $a$ is of type $A$ and $f$ of type $B \from A$ then
$f\app a$ is of type $B$.%
\footnote{Actually, the application symbol $\app$ should carry subscripts $A$ and $B$;
but we omit these whenever they are clear from the context, as they are throughout this paper.}

In place of the three constants $\cI$, $\cK$ and $\cS$ we adopt three families of constants:
\begin{enumerate}[align=left]
\item[(1)]$\qquad \cIA \in A \from A$ such that $\cIA \app a = a$
\item[(2)]$\qquad \cKAB \in (B \from A) \from A$ such that $(\cKAB\app a)\app b = a$
\item[(3)]$\qquad \cSABC \in ((A \from C) \from (B \from C)) \from ((A \from B) \from C))$ such that
$((\cSABC \app f) \app g) \app c = (f \app c) \app (g \app c).$
\end{enumerate}
It is assumed here that $a\in A$, $b \in B$, $c\in C$, $f \in (A \from B) \from C$ and $g \in B \from C$.

A many-sorted algebra of the kind constructed above
may be called a {\it typed \schon\ algebra}, or a {\it typed Curry algebra} if the appropriate
equations are postulated.

Curry and Feys have pointed out that, if $\from$ is read as ``if'',
the types of $\cIA$, $\cKAB$, and $\cSABC$ are precisely the axioms of the
intuitionist implicational calculus, while (0) bears an obvious relation to
the usual rule of modus ponens.

Two remarks are in order. The same axioms also appear in the classical
propositional calculus accompanied by an additional axiom involving negation.
Nonetheless, the negationless formula $A \from (A \from (B \from A))$ is a
theorem classically but not in the system without negation.
This is why we call the present system intuitionistic.

Secondly, it should be pointed out that $A \from A$ is usually not
taken as an axiom, but is deduced from the other two axioms. Nevertheless,
we prefer to regard it as an axiom here.

Implicit in the definition of a typed \schon\ algebra
is a way of regarding each entity of type $A$ as a proof of the
formula $A$.

For example, $\cS_{A,B,A}$ is by definition a proof of the axiom
$((A \from A)\from(B\from A))\from((A\from B)\from A)$
and $\cKAB$ is by definition a proof of the axiom $(A \from B) \from A$.
Hence $\cS_{A,B,A} \app \cKAB$ is a proof of the theorem $(A\from A) \from (B\from A)$
by modus ponens. Again, $\cK_{A,C}$ is a proof of the axiom
$(A \from C) \from A$. Take $B = (A \from C)$
then $(\cS_{A,B,A}\app \cKAB)\app \cK_{A,C}$ is a proof of the theorem $A \from A$.
Since we regard $A\from A$ as axiom, another proof is $\cIA$. Incidentally,
we see here that the derivation of $A \from A$ the other axioms of the
implicational calculus is nothing else then \schon's
definition of $\cI$ as $(\cS \app \cK)\app \cK$.

The association of entities with proofs becomes even more
striking when we compare the free typed \schon\ algebra
(generated by a set of letters) with pure
intuitionistic implicational logic. Then
$$
\hbox{\rm\ combinators } = \hbox{\rm\ proofs.}
$$
The reader should note that throughout we distinguish between an axiom
such as $(A \from B) \from A$ and its proof $\cKAB$, a pedantic but necessary
distinction.

Let us now look at Proposition 1 for typed \schon\ algebras. The reader will easily
convince himself that the proposition and its proof remain valid, provided $x$
is an indeterminate of type $A$, where $A$ is any type of $\calA$.
It asserts that, if $\varphi(x)$ is a polynomial of type $B$ in the indeterminate $x$
of type $A$, then there is a constant $f$ of type
$B \from A$ such that $\varphi(x) \eqx f \app x$.


In the proof-theoretic interpretation we should regard $x$ as an assumption
that $A$ holds. Proposition 1 then becomes the usual deduction theorem with an
extra punch at the end: if $\varphi(x)$ is a proof of $B$ from the assumption $x$
that $A$ holds, then there is a proof $f$ of $B \from A$ without any assumption such
that $\varphi(x) \eqx f \app x$.

The extra bit at the end asserts the ``equality'' of two proofs.
Perhaps it would have been better to speak only of ``equivalence'' of proofs.


As far as I know, this association between combinators and proofs is due to Curry and Feys (1958).
It was developed further by Howard (see Stenlund, 1972) and the author (1969, 1972).%
\footnote{For related ideas and further references see also the thought provoking paper by Scott (1970).}%

Proposition 2 will also remain valid for typed Curry algebras.
It asserts that the proof $f$ of $B \from A$, whose existence has
been established in Proposition 1, is unique up to equivalence of
proofs (which we called equality here).

Proposition 3 will remain valid for typed Curry algebras
provided we use the typed $\lambda$-calculus and write
$f = \lambda_{x \in A} \varphi(x)$.

Proposition 4 will not remain valid for typed Curry algebras without
very strong restrictions. After all, the whole purpose of introducing types was
to render expressions such as $b \app b$ meaningless in general. If $b \in B$, it does have
a meaning only if $B = B \from B$. Proposition 4 remains valid for typed algebras in the
following sense: if $A$ is a type such that $A \from A = A$, then every entity $f$ of type $A$
has a fixpoint a such that $f\app a= a$.

\pdfbookmark{Positive Logic}{logic}
The fragment of propositional logic investigated up to now studiously avoids
conjunction, which classically was usually defined in terms of implication and negation.
Let us now turn to a fragment of logic, the positive intuitionist propositional calculus,
which deals with implication $\from$, conjuntion $\sland$, and truth $\top$.
We shall present this as a deductive system, making use of the symbol $\ent$ for entailment,
in the spirit of Gentzen. Here are our axioms and rules of inference, suitably labelled:
\bes
A \xto{1_A} A, \qquad \inferrule{A \xto{f} B\quad B \xto{g} C}{A \xto{gf} C};
\tag{1}
\ees
\bes
A \xto{0_A} \top;
\tag{2}
\ees
\bes
A \sland B \xto{\prAB} A,\,\,A \sland B \xto{\prpAB} B, \quad
\inferrule{C \xto{f} A\qquad C \xto{g} B}{C \xto{\bracket{f, g}} A \sland B};
\tag{3}
\ees
\bes
(A\from B) \sland B \xto{\ev_{A,B}} A, \qquad \inferrule{C\sland B \xto{h} A}{C \xto{h^*} (A \from B)}
\tag{4}
\ees
The labels are useful in naming proofs. For example, the commutative law is proved thus:
$$
\inferrule{A \sland B \xto{\prpAB} B\quad A \sland B \xto{\prAB} A}{A \sland B \xto{\bracket{\prpAB,\prAB}} B\sland A}
$$
The label $\bracket{\prpAB,\prAB}$ appearing in the last line may be used
to denote the whole proof.

We are also interested in an equivalence relation between proofs,
which we may as well denote by the equality symbol $=$. We do not bother to
write down the usual reflexive, symmetric, transitive and substitution laws for
equality. However, we list the following equations, where in $f = g$ it is
assumed that $f$ and $g$ have the same source and target.
\begin{enumerate}[align=left]
\item[(1\mprime)]$f1_A = f$, $1_Bf = f$, and $(hg)f = g(hf)$,
 for all $f:A \to B$, $g:B \to C$, $h: C\to D$;
\item[(2\mprime)] $f= 0_A$, for all $f: A \to 1$;
\item[(3\mprime)] $\prAB\bracket{f,g} = f$, $\prpAB\bracket{f,g}= g$, and
$\la\prAB h, \prpAB h\ra = h$, for all $f:C \to A$, $g:C \to B$, and $h: C\to A\sland B$;
\item[(4\mprime)] $\ev_{A,B}\bracket{h^*\pr{C,B}, \prp{C,B}} = h$, and
$(\ev_{A,B}\bracket{k\,\pr{C,B}, \prp{C,B}})^* = k$,
for all $h: C\sland A \to A$, and $k: C \to A \from B$.
\end{enumerate}
\pdfbookmark{Cartesian closed categories}{ccc}
The reader will recognize that (1) and (1\mprime) define a category.
Moreover, (2) and (2\mprime) assert that $\top$
is a terminal object, usually denoted by $1$.
Furthermore, (3) and (3\mprime) assert that $A \sland B$ is the Cartesian product
of A and B, usually written $A \times B$.
Finally, (4) and (4\mprime) express the fact that $(-) \from B$ is right adjoint to
$(-) \sland B: \calA \to \calA$ , which makes $\from$ an internal hom-functor, the
usual notation for $A \from B$ being $A^B$.
(1) to (3) and (1\mprime) to (3\mprime) define a {\it Cartesian category},
(1) to (4) and (1\mprime) to (4\mprime) define a {\it Cartesian closed category}.

Cartesian closed categories were introduced under this name
by Eilenberg and Kelly (1966). Lawvere (1969) also pointed out that $\times$
is right adjoint to the diagonal $\calA \to \calA \times \calA$ and emphasized the analogy
with propositional logic. Our definition of Cartesian closed categories is slightly different,
in as much as $1$, $\times$, and exponentiation are not only said to exist, but are regarded
as part of the structure. In fact, our notion of Cartesian closed category is algebraic over
the category of categories, or the category of graphs for that matter, in much the same way
that the notion of group is algebraic over the category of sets.

For future reference, we also note the following consequence of conditions (1\mprime) to (4\mprime):
\bes
\bracket{f,g}h = \bracket{fh,gh}
\tag{5\mprime}
\ees
This is proved thus: let $k = \bracket{f,g} h$, then, omitting subscripts, we have
$$
\pi k=fh, \,\pi' k=gh
$$
by (3\mprime) and hence
$$
\bracket{fh,gh} = \bracket{\pi k,\pi' k} = k,
$$
by (3\mprime) again.

In the author's opinion, it is a tour de force to present
propositional logic without conjunction. Curiously, the same tour de force
is found in the paper by Eilenberg and Kelly, who
went to some trouble to eliminate the Cartesian product from Cartesian closed categories.
One could argue that closed categories without products are essentially the same
as typed Curry algebras. It therefore seems reasonable to proceed
from the study of typed Curry algebras to the study of Cartesian closed categories.

I had proved in 1972 that functional completeness holds for Cartesian closed categories,
provided that these satisfied a finite set of equations, like those due to Curry,
and conjec­tured that these equations are already a consequence of (1\mprime) to (4\mprime) above,
that is, they hold in any Cartesian closed category. I proved this later (1974)
and shall give another exposition of the proof here. The new versions of Propositions
1 and 2 will be called Theorems 1 and 2.

\pdfbookmark{Deduction Theorem}{deduction}
First, we must explain what it means to adjoin an indeterminate morphism $x: 1 \to A$
to a Cartesian closed category $\calA$ with $A$ being an object of $\calA$, which we also
regard as a formula. In the same spirit, we may regard $x$ as an assumption that $\top$ entails $A$.
The objects of $\calA[x]$ are the same as those of $\calA$, but the morphisms
$\varphi(x): B \to C$ in $\calA[x]$ may be regarded as proofs that $B$ entails $C$ on
the assumption $x$. Equality between proofs or polynomials is determined by (1\mprime) to (4\mprime) and is
denoted by $\eqx$. One must check of course that $\calA[x]$ thus constructed has 
the expected universal property.

\begin{theorem}[Deduction theorem] With every proof $\varphi(x): B \to C$ from the
assumption $x: 1 \to A$ there is associated a proof $f: A \sland B \to C$ not depending
on this assumption. Moreover $f\bracket{x0_B, 1_B} \eqx \varphi(x)$.

\end{theorem}

It should be pointed out that we have presented the positive
intuitionist propositional calculus as a deductive system, so
that the usual deduction theorem becomes absorbed in the rules governing
the deduction symbol $\to$, thus:
$$
\inferrule{A \sland B \to C}{A \to C \from B}
$$
However, at a higher level the horizontal bar functions as a deduction symbol and
Theorem 1 plays the role of a new deduction theorem.
\begin{proof}
Clearly, every polynomial $\varphi(x)$ must have one of the five forms:
$k$, $x$, $\bracket{\psi(x), \chi(x)}$, $\chi(x)\psi(x)$, and $\psi(x)^*$, where $k$
is constant and where $\psi(x)$ and $\chi(x)$ are shorter polynomials.
By inductional assumption we have, omitting subscripts:
$$
\psi(x) \eqx g\bracket{x0,1}, \qquad \chi(x) \eqx h\bracket{x 0,1}.
$$
The result now follows by verifying the following equations:
\begin{align*}
k \prpAB \bracket{x0_B, 1_B} &\eqx k, \\
\pr{A,1}\bracket{x0_1,1_1} &\eqx x,\\
\bracket{g,h}\bracket{x0_B, 1_B} &\eqx \bracket{\psi(x),\chi(x)},\\
h\bracket{\prAB,g}\bracket{x0_B,1_B} &\eqx \chi(x)\psi(x),\\
(g\alpha_{A,B,D})^*\bracket{x0_B,1_B} &\eqx \psi(x)^*,
\end{align*}
where $\alpha_{A,B,D}:(A \sland B)\land D \to A\sland(B \sland D)$ is given by
$$
\alpha_{A,B,D} = \bracket{\prAB\, \pr{A\sland B,D}\, , \bracket{\prpAB\pr{A\,\sland B,D}\,,
\prp{A\sland B,D}}}
$$
The last equation is proved by showing (omitting subscripts)
that:
$$
\ev\bracket{(g\alpha)^*\bracket{x0,1}\pi,\pi'} \eqx g\bracket{x0,1}
$$
which follows from a routine calculation, as in (Lambek 1974, p. 272).
\end{proof}

\pdfbookmark{Functional}{completeness}
\begin{theorem}[Functional completeness of Cartesian closed categories]
For every polynomial $\varphi(x):B \to C$ in an indeterminate $x: 1 \to A$
there is a unique constant $f: A \times B \to C$ such that $f\bracket{x0_B,1_B} = \varphi(x)$.
\end{theorem}
\begin{proof}
Let us write $\kxa \varphi(x)$ for the constant $f$ which is assigned to the proof $\varphi(x)$
in the proof of Theorem 1. Thus, when $k$ is a constant, we have:
\begin{align*}
\kxa k &= k\prpAB; \tag{i}\\
\kxa x &= \pr{A,1}; \tag{ii}\\
\kxa \bracket{\psi(x),\chi(x)} &= \bracket{\kxa\psi(x),\kxa\chi(x)}; \tag{iii}\\
\kxa (\psi(x) \chi(x))&= \kxa\chi(x)\bracket{\prAB, \kxa\psi(x)};\tag{iv}\\
\kxa (\psi(x)^*) &= (\kxa\psi(x)\alpha_{A,B,D})^*, \tag{v}
\end{align*}
where (iii), (iv) and (v) are subject to the restriction that $\varphi(x)$ is not constant.
Note, in particular, the following special case of (iv), in view of (i):
\bes
\kxa(k\,\psi(x)) = k \,\kxa \psi(x)
\tag{vi}
\ees
where again it is assumed that $\psi(x)$ is not constant.

We first show that the above restrictions on (iii) to (v),
and therefore on (vi), may be removed. For example, to take the
most difficult case, let us check that (v) holds when $\psi(x)$ is
a constant $g$. Then
\begin{align*}
(\kxa g \,\alpha)^* & = (g\, \pi' \alpha)^* & \hbox to 2em { by \hfil} & \textrm{(i)} & \\
& = (g\, \beta)^* & \hbox to 2em{ if \hfil}  & \beta = \bracket{\pi'\pi, \pi'} &\\
& = (\ev\bracket{g^*\pi,\pi'}\beta)^* & \hbox to 2em { by \hfil} & \textrm{(4\mprime)} & \\
& = (\ev\bracket{g^*\pi\,\beta,\pi'\,\beta})^* & \hbox to 2em { by \hfil} & \textrm{(4\mprime)} & \\
& = (\ev\bracket{g^*\pi'\pi,\pi'})^* & \hbox to 2em { by \hfil} & \textrm{(3\mprime)} & \\
& = g^*\pi' & \hbox to 2em { by \hfil} & \textrm{(4\mprime)} & \\
& = \kxa (g^*) & \hbox to 2em { by \hfil} & \textrm{(i)} &
\end{align*}
We next show that $\kxa \varphi(x)$ depends only on the polynomial
$\varphi(x)$, which may be regarded as the proof $\varphi(x)$ modulo the
equivalence relation $\eqx$. Let us write $\varphi(x)\equiv \psi(x)$ for
$\kxa \varphi(x) = \kxa \psi(x)$
Then it is easily checked that $\equiv$ has the
substitution property and satisfies all the conditions which
equality in $\calA[x]$ must satisfy (see the sample calculation below).
Since $\eqx$ is by definition the smallest such equivalence relation, it follows that
$\eqx$ is contained in $\equiv$, that is,
\bes
\varphi(x) \eqx \psi(x) \text{ implies } \kxa \varphi(x) = \kxa \psi(x),
\tag{*}
\ees
as claimed.
%
\renewcommand{\thefootnote}{\fnsymbol{footnote}}
%
As promised, we shall prove, for example, that
$\ev\bracket{\chi(x)^*\pi,\pi'} \equiv\chi(x)$ to take the worst case. Indeed, writing
$\kxa\chi(x) = h$ we have:
\begin{align*}
\kxa(\ev\bracket{\chi(x)^*\pi,\pi'}) & = \ev\kxa \bracket{\chi(x)^*\pi,\pi'}  & \hbox to 2em { by \hfil} & \textrm{(vi)}\\
& = \ev\bracket{\kxa(\chi(x)^*\pi),\kxa\pi'}  & \hbox to 2em { by \hfil} & \textrm{(iii)}\\
& = \ev\bracket{\kxa\chi(x)^*\bracket{\pi,\kxa\pi},\pi'\pi'}  & \hbox to 2em { by \hfil} & \textrm{(iv) and (i)}\\
& = \ev\bracket{(h\alpha)^* \bracket{\pi,\pi\pi'},\pi'\pi'} & \hbox to 2em { by \hfil} & \textrm{(v) and (i)}
\end{align*}
Now the associativity morphism $\alpha = \bracket{\pi\pi,\bracket{\pi\pi',\pi'}}$  clearly has an inverse $\alpha^{-1}$
so the above
\begin{align*}
\ev\bracket{(h\alpha)^* \bracket{\pi,\pi\pi'},\pi'\pi'}
&= \ev\bracket{(h\alpha)^*\bracket{\pi , \pi\pi'},\pi'\pi'}\alpha\alpha^{-1} \\
&= \ev\bracket{(h\alpha)^*\bracket{\pi \alpha, \pi\pi'\alpha},\pi'\pi'\alpha}\alpha^{-1} & \hbox { by } & \textrm{(4\mprime)}\\
&= \ev\bracket{(h\alpha)^*\bracket{\pi \pi, \pi'\pi},\pi'}\alpha^{-1} & \hbox { by } & \textrm{(3\mprime)}\footnotemark\\
&= \ev\bracket{(h\alpha)^*\pi, \pi'}\alpha^{-1} & \hbox { by } & \textrm{(3\mprime)}\\
&= h\alpha\alpha' = h & \hbox { by } & \textrm{(4\mprime)}.
\end{align*}
\footnotetext{Editor's note:
This third step was a bit mysterious to me. But remember that
$\pi\alpha = \pi\bracket{\pi\pi,\bracket{\pi\pi',\pi'}} = {\pi\pi}$ and
$\pi'\alpha = \pi'\bracket{\pi\pi,\bracket{\pi\pi',\pi'}} = \bracket{\pi\pi',\pi'}$. Then
push all the projection operators through and you get the right answer.}%
We are finally ready to prove the uniqueness of $f$.
Suppose $g\bracket{x0_B, 1_B} \eqx \varphi(x)$, then
\begin{align*}
\kxa	\varphi(x) &= \kxa (g\bracket{x0_B, 1_B}) & \hbox to 2em { by \hfil} & \textrm{(*)}\\
&= g \kxa (\bracket{x0_B, 1_B}) & \hbox to 2em { by \hfil} & \textrm{(vi)}\\
&= g  \bracket{\kxa (x0_B), \kxa (1_B)} & \hbox to 2em { by \hfil} & \textrm{(iii)}\\
&= g  \bracket{\kxa x\bracket{\prAB, \kxa 0_B},  1_B \prpAB} & \hbox to 2em { by \hfil} & \textrm{(iv) and (i)}\\
&= g  \bracket{\pr{A,1}\bracket{\prAB, 0_B\prpAB},\prpAB}& \hbox to 2em { by \hfil} & \textrm{(ii) and (i)}\\
&= g  \bracket{\prAB, \prpAB} & \hbox to 2em { by \hfil} & \textrm{(3\mprime)}\\
&= g 1_{A \times B} & \hbox to 2em { by \hfil} & \textrm{(3\mprime)}\\
&= g  & \hbox to 2em { by \hfil} & \textrm{(1\mprime)}
\end{align*}
\end{proof}
\begin{remark}
Theorems 1 and 2 remain valid for Cartesian categories, that is, if exponentiation is absent.
\end{remark}
\begin{remark}[For categorists]
Theorem 2 may be interpreted
as saying that $\calA[x]$ is the Kleisli category of the obvious
cotriple on $\calA$ associated with the function $A \times (-): \calA \to \calA$
or of the triple on $\calA$ associated with the functor $(-)^{A}$, see (Lambek 1974) for details.
It is also not difficult to prove directly that the Kleisli category has the universal
property of $\calA[x]$. This alternative approach (Lambek 1974, section 5)
bypasses the historical development discussed above, but also loses the algorithm for calculating
$\kxa \varphi(x)$. Such a direct approach, without mention of Kleisli categories by that name,
was also used in (Volger, 1975).

As a special case of Theorem 2 we obtain the following.
\end{remark}
\pdfbookmark{Lambda Coda}{lambda}
\begin{corollary}
For every polynomial $\varphi(x): 1 \to C$ in an indeterminate $x: 1 \to A$ there is
a unique constant $g: A \to C$ such that $gx \eqx \varphi(x)$ and, equivalently, a unique
constant $h: 1 \to C^A$ such that $\ev_{C,A}\bracket{h,x} \eqx \varphi(x)$.
\end{corollary}
\begin{proof}
To obtain this from Theorem 2, we merely put
\begin{align*}
g &= \kxa\varphi(x)\bracket{1_A,0_A}\\
h &= (g\prp{1,A})^* = (\kxa\varphi(x)\bracket{\prp{1,A},\pr{1,A}})^*
\end{align*}
\end{proof}
Actually, the corollary is no weaker than the theorem, since
polynomials $B\to C$ are in one-to-one correspondence with polynomials
$1\to C^B$. To compare the corollary with the $\l$-calculus, we write
$$
h\app = \ev_{C,A}\bracket{h\, 0_A, 1_A},
$$
so that
$$
h\app x = \ev_{C,A}\bracket{h,x}\eqx \varphi(x),
$$
and define
$$
\l_{x\in A}\varphi(x) =  (\kxa \varphi(x)\bracket{\prp{1,A},\pr{1,A}})^* = h.
$$
There are two traditional applications of the $\l$-calculus.
The first is to arithmetic, in particular, the theory of recursive functions.
The second is to the foundations of mathematics via Curry's illative combinatory logic
or the type theory of Church. We shall briefly discuss the first application and regret
that space does not permit discussion of the second and the use which has recently been
made of it in the construction of free toposes.

Writing
$$
f \comp g = \lx(f \app (g \app x)) ,
$$
one may introduce natural numbers
$$
\text{$0 = \lx I$, $1 = \lx x$, $2 = \lx (x \circ x)$, $\cdots$}
$$
and successor, addition, multiplication and exponentiation by
\begin{align*}
S \app n &= \ly(y\circ(n\ \app y)), \\
m + n &= \ly ((m\app y)\circ (n \app y)),\\
mn &= m \circ n,\\
m^n &= n \app m.
\end{align*}
These are recursive functions, and Kleene proved in 1936 that recursive (or Turing computable)
functions are precisely those recursive functions which are definable in the type-free
$\l$-calculus.

Unfortunately there are difficulties when one introduces
types. If $a$ has type $A$, then $f$ and $g$ in $f\circ g \app a$ have types $A^A=B$, say.
For $n\app f$ to make sense, $n$ will then have to be of type $B^B$, and for $n\app m$ to make sense,
$m$ will have to be of type $B$. If m and n are both natural numbers, we
are led to $B^B = B$, and this would be a consequence of $A^A = A$.

Now, it is certainly possible to postulate a type A such
that $A^A = A$ (Scott, 1972). However, it seems more natural to postulate a type $N$
with entities $0 \in N$, $S \in N \from N$, and $R_A \in ((A \from A) \from (A \from A)) \from N$
satisfying certain equations. In the language of Cartesian closed categories, we thus require an object
$N$ and morphisms $0: 1 \to N$, $\sigma: N\to N$ and $\rho_A: N \to (A^A)^{(A^A)}$
satisfying certain identities, to wit:
\begin{align*}
(\rho_A 0)\app f &= I\\
(\rho_A (\sigma\, n))\app f &= f\circ((\rho_A n)\app f)
\end{align*}
for all $n:1 \to N$ and $f:1 \to A^A$. 

These identities are related to the Peano--Lawvere axiom,
which deals with a morphism $A \times A \to A^N$ instead.
They imply the existence (but not the uniqueness) of a morphism
$N\to A$ such that the following diagram commutes for given morphisms $1 \to A$ and $A \to A$:
\[
\begin{tikzcd}[row sep=.75in, column sep = .75in]
    1 \arrow[rd] \arrow{r}{0} &N \arrow[dotted]{d}{} \arrow{r}{\sigma}  & N \arrow[dotted]{d}{} \\
    {} & N  \arrow{r}{} & N
\end{tikzcd}
\]
The Peano-Lawvere axiom also requires uniqueness.

Marie--France Thibault called a Cartesian closed category with
$N$, $0$, $\sigma$, $\rho$ satisfying the above identities a {\it prerecursive} category. She proved the following:
\begin{enumerate}
\item[(a)] The set of primitive recursive functions is properly contained in the set $\cal F$
of functions represented by morphisms $N \to N$ in the free prerecursive category
generated by the empty category.
\item[(b)] $\cal F$ is properly contained in the set of all recursive.
\item[(c)] $\cal F$ coincides with the set of type-recursive functions
discussed by Grzegorczyk.
\end{enumerate}
It seems clear from (Hindley, Lercher and Seldin, 1972, Chapter
11) that $\cal F$ is also essentially the same as the set of G\"odel's functions of finite type.

\section*{Acknowledgements}

The author wishes to acknowledge support from the Natural Sciences and Engineering Research Council
of Canada and partial support from the Quebec Department of Education. He is in debted to W.S. Hatcher
for drawing his attention to an error in the original manuscript.

\pdfbookmark{References}{bib}
\def\bb#1{\bibitem[#1]{#1}}

\begin{thebibliography}{888}

\bb{1} Church, A. (1940). {\it A formulation of the simple theory of types}, J. Symbolic Logic {\bf 5}, 56--58.

\bb{2} Church, A. (1941). {\it The calculi of lambda conversion}, Princeton University Press, Princeton.

\bb{3} Curry, H.B. (1930). {\it Grundlagen der Kombinatorischen Logik}, Amer J. Math. {\bf 52}, 789--834.

\bb{4} Curry, H.B. (1932). {\it Some additions to the theory of combinators}, Amer. J. Math. {\bf 54}, 551--558.

\bb{5} Curry, H.B. (1942).{\it The inconsistency of certain formal logics}, J. Symbolic Logic {\bf 7}, 115--117.

\bb{6} Curry, H.B. (1968). {\it Recent advances in combinatory logic}, Bull. Soc. Math. Belg. {\bf 20}, 288--298.

\bb{7} Curry, H.B. and Feys, R. (1958). {\it Combinatory logic}, Vol. 1, North-Holland, Amsterdam.

\bb{8} Curry, H.B., Hindley, J.R. and Seldin, J.P. (1972). {\it Combinatory logic}, Vol. 2, North-Holland, Amsterdam.

\bb{9} Eilenberg, S. and Kelly, G.M. (1966). {\it Closed categories}. In Proc. Conf. on Categorical Algebra, LaJolla 1965,
 (eds. S. Eilenberg {\it et. al.}), 421--562.

\bb{10} Grzegorczyk, A. (1964). {\it Recursive objects in all finite types}, Fundamenta Mathematicae \textbf{54}, 73--93.

\bb{11} Hindley, J.R., Lercher, B. and Seldin, J.P. (1972). {\it Combinatory logic}, Cambridge University Press, Cambridge.

\bb{12} Kleene, S.C. (1936). {\it $\l$-defInability and recursiveness}, Duke Math. J. \textbf{2}, 340--353.

\bb{13} Kleene, S.C. and Rosser, J.B. (1935). {\it The inconsistency of certain formal logics}, Ann. Math. \textbf{36}, 630--636.

\bb{14} Kleisli, H. (1965). {\it Every standard construction is induced by a pair of adjoint functors}, 
Proc. Amer. Math. Soc. \textbf{16}, 544--546.

\bb{15} Lambek, J. (1969). {\it Deductive systems and categories II}, Lecture Notes in Mathematics {\bf 86}, 76--122.

\bb{16} Lambek, J. (1972). {\it Deductive systems and categories III}, Lecture Notes in Mathematics {\bf 274}, 57--82.

\bb{17} Lambek, J. (1974). {\it Functional completeness of Cartesian categories}, Annals of Math. Logic {\bf 6}, 259--292.

\bb{18} Lambek, J. (1980). {\it From types to sets}, Advances in Mathematics {\bf 35}, (to appear).

\bb{19} Lawvere, F.W. (1969). {\it Adjointness in foundations}, Dialectica {\bf 23}, 281--296.

\bb{20} MacLane,S.(1971).{\it Categories for the working mathematician}, Springer, New York.

\bb{21} Rosenbloom, P.C. (1950). {\it The elements of mathematical logic}, Dover, New York.

\bb{22} Rosser, J.B. (1942). {\it New sets of postulates for combinatory logic}, J. Symbolic Logic {\bf 7}, 18--27.

\bb{23} \schon, M. (1924). {\it \"Uber die Bausteine der mathematischen Logik}, Math. Ann. {\bf 92}, 305--316. 
Translated in {\it From Frege to Godel} (Ed. J. Van Heijenoort).

\bb{24} Scott, D. (1970). {\it Constructive validity}, Lecture Notes in Mathematics {\bf 125}, 237--275.

\bb{25} Scott, D. (1972). {\it Continuous lattices}, Lecture Notes in Mathematics {\bf 274}, 97--136.

\bb{26} Stenlund, S. (1972). {\it Combinators, A-terms and proof theory}, Reidel, Dordrecht-Holland.

\bb{27} Szabo, M.E. (1974). {\it A categorical equivalence of proofs}, Notre Dame J. Formal Logic {\bf 15}, 177--191.

\bb{28} Szabo, M.E. (1978). {\it Algebra of proofs}, North-Holland, Amsterdam.

\bb{29} Thibault, M.-F. (1977). {\it Representations des fonctions recursives dans les categories}, 
Thesis, McGill University, Montreal.

\bb{30} Thibault, M.-F. (1978). {\it Prerecursive categories}, Manuscript.

\bb{31} van Heijenoort, J. (1967). {\it From Frege to Godel}, Harvard University Press, Cambridge, Massachusetts.

\bb{32} Volger, H. (1975). {\it Completeness theorem for logical categories}, Lecture Notes in Mathematics {\bf 445}, 51--86.


\end{thebibliography}
