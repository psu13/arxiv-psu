\section{Conclusion}
\label{sec:conclusion}

We have exhibited an einstein, the first topological disk that tiles
aperiodically with no additional constraints or matching conditions.
The hat polykite is in fact a member of a continuous family of aperiodic
monotiles that admit combinatorially equivalent tilings.
The hat forces
tilings with hierarchical structure, as is the case for many aperiodic
sets of tiles in the plane, but a new method introduced in
\secref{sec:coupling} also suffices to show the lack of periodic
tilings without needing that hierarchical structure, beyond
demonstrating the existence of a tiling.

The hat is a $13$-sided non-convex polygon.  A convex polygon cannot
be an aperiodic monotile, and all non-convex quadrilaterals can 
easily be seen to tile periodically.  Therefore, in terms of number of 
sides, the ``simplest'' aperiodic $n$-gon must have $5\le n \le 13$.
Subsequent research could chip away at this range, by finding aperiodic
$n$-gons for $n<13$ or ruling them out for $n\ge 5$.

Finding such a monotile pushes the boundaries of complexity known to
be achievable by the tiling behaviour of a single closed topological
disk.  It does not, however, settle various other unresolved
questions about that complexity.  For example, all of the following
questions remain open.

\begin{itemize}
  \item Are Heesch numbers unbounded?  That is, does there exist, for
  	every positive integer $n$, a topological disk that does not tile the 
	plane and has Heesch number at least $n$?  We conjecture that there is
	no bound on Heesch numbers.

  \item Are isohedral numbers unbounded?  That is, does there exist, for
  	every positive integer $n$, a topological disk that tiles the plane 
	periodically
	but only admits tilings with at least $n$ transitivity classes?  Again,
	we conjecture that no bound exists.  If the requirement of periodicity
	is omitted here, then the hat polykite requires infinitely many 
	transitivity classes in any tiling. 
	Socolar~\cite{Socolar} showed that if the tile is not required to be
    a closed topological disk, then tiles exist with every positive
	isohedral number.

  \item Is it undecidable whether a single closed topological disk (or
    indeed a more general single tile in the plane) admits a tiling?
    It would again be reasonable to conjecture yes, which would also
    imply unbounded Heesch numbers.  Greenfeld and
    Tao~\cite{GT1} demonstrated undecidability in a more general
    context.  For sets of tiles in the plane, Ollinger~\cite{Ollinger}
    proved  undecidability for sets of five polyominoes.

  \item Is it undecidable whether a single closed topological disk (or
    indeed a more general single tile in the plane) admits a periodic
    tiling?  It would again be reasonable to conjecture yes.  Such
    an answer would imply unbounded isohedral numbers.
\end{itemize}

Although we have provided a complete classification of tilings by the
hat polykite and related tiles described here (all such tilings are
given by the substitution system of \secref{sec:subst}, as applied to
the clusters of tiles from \secref{sec:clusters}), there are various
informal observations in \secref{sec:discussion} that have not been
fully explored or given a precise statement.  Those observations could
provide starting points for possible future investigation of the tiles
described here and their tilings, the metatiles used in classifying
tilings by the hat polykite, and other related substitution tilings.
It is not clear which ideas from this work will be most promising for
future work, so we have generally erred on the side of including 
observations that might be of use, rather than making the paper focus
more narrowly on a single proof of a single main result.

We believe the approach of \secref{sec:coupling}, of coupling two
separate tilings to show that a third tiling cannot be periodic, is a
new approach for proving such a result about plane tilings.  It would
be worth investigating whether it can be applied to other tiles and
tilings.  In particular, polykites (and, more generally,
poly-$[4.6.12]$-tiles, a subset of the shapes known as
\emph{polydrafters}) may be unusually well-suited to this method of
proof, because their edges can naturally be split into those falling
in two sets of lines, with each set of lines forming a regular
triangular tiling. It might also be applicable to some
poly-$[4.8.8]$-tiles (a subset of the
\emph{polyaboloes}).\footnote{Note that if Lemma~\ref{lemma:align} is
applied to poly-$[4.6.12]$-tiles or poly-$[4.8.8]$-tiles, the
conclusion is weaker than that of Lemma~\ref{lemma:polykitealign} for
polykites, so tilings may need to be considered that are only aligned
in such a weaker sense.}  This style of proof might help explain how
small polykites
proved to be aperiodic when polyominoes, polyiamonds and polyhexes up
to high orders yielded no aperiodic monotiles.  However, as noted in
\secref{sec:family}, searches of polykites have not found other
aperiodic examples outside the family described in this paper.
