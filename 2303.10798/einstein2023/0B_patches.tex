\section{Case analysis for $1$-patches}
\label{sec:patches}

We present here details of a computer-generated but human-verifiable
case analysis, based on consideration of $1$-patches rather than
$2$-patches.  This analysis can be used to complete a variant of the
proof in \secref{sec:clusters} that,
when tiles in a tiling by the hat polykite
are assigned labels following the rules given there, (a)~the labels
assigned do induce a division into the clusters shown, and (b)~the
clusters adjoin other clusters in accordance with the matching rules.
As is justified in Appendix~\ref{sec:align}, we only consider tilings
where all tiles are aligned with an underlying $[3.4.6.4]$ Laves
tiling.

\subsection{Enumeration of neighbours}

First we produce a list of possible neighbours of the hat polykite in a
tiling.  There are $58$~possible neighbours when we only require such
a neighbour not to intersect the original polykite; these are shown in
Figure~\ref{fig:nbr}, with that original polykite shaded.  The first
$41$ of these neighbours remain in consideration for the enumeration
of $1$-patches.  The final~$17$ are immediately eliminated (in the
order shown) because they cannot be extended to a tiling: either there
is no possible neighbour that can contain the shaded kite (without
resulting in an intersection, or a pair of tiles that were previously
eliminated as possible neighbours), or we eliminated $Y$ as a
neighbour of~$X$ and so can also eliminate $X$ as a neighbour of~$Y$.

% Automatically generated figures.
\begin{figure}[htp!]
\renewcommand\thesubfigure{\arabic{subfigure}}%
\captionsetup{margin=0pt,justification=raggedright}%
\begin{center}
\subfloat[Possible neighbour~$1$]{%
\begin{minipage}[b]{6cm}
\begin{center}
\begin{tikzpicture}[x=3mm,y=3mm]
  \ftileA{0}{0}{0}{};
  \tileA{0}{-2}{1}{};
\end{tikzpicture}%
\end{center}%
\end{minipage}%
} \qquad \subfloat[Possible neighbour~$2$]{%
\begin{minipage}[b]{6cm}
\begin{center}
\begin{tikzpicture}[x=3mm,y=3mm]
  \ftileA{0}{0}{0}{};
  \tileA{60}{-1}{-1}{};
\end{tikzpicture}%
\end{center}%
\end{minipage}%
}%
\end{center}
\caption{Possible neighbours (part 1)}
\label{fig:nbr}
\end{figure}
\begin{figure}[htp!]
\renewcommand\thesubfigure{\arabic{subfigure}}%
\ContinuedFloat
\captionsetup{margin=0pt,justification=raggedright}%
\begin{center}
\subfloat[Possible neighbour~$3$]{%
\begin{minipage}[b]{6cm}
\begin{center}
\begin{tikzpicture}[x=3mm,y=3mm]
  \ftileA{0}{0}{0}{};
  \tileA{60}{-1}{0}{};
\end{tikzpicture}%
\end{center}%
\end{minipage}%
} \qquad \subfloat[Possible neighbour~$4$]{%
\begin{minipage}[b]{6cm}
\begin{center}
\begin{tikzpicture}[x=3mm,y=3mm]
  \ftileA{0}{0}{0}{};
  \tileA{120}{-1}{0}{};
\end{tikzpicture}%
\end{center}%
\end{minipage}%
} \\ \subfloat[Possible neighbour~$5$]{%
\begin{minipage}[b]{6cm}
\begin{center}
\begin{tikzpicture}[x=3mm,y=3mm]
  \ftileA{0}{0}{0}{};
  \tileA{180}{-1}{0}{};
\end{tikzpicture}%
\end{center}%
\end{minipage}%
} \qquad \subfloat[Possible neighbour~$6$]{%
\begin{minipage}[b]{6cm}
\begin{center}
\begin{tikzpicture}[x=3mm,y=3mm]
  \ftileA{0}{0}{0}{};
  \tileA{300}{-1}{0}{};
\end{tikzpicture}%
\end{center}%
\end{minipage}%
} \\ \subfloat[Possible neighbour~$7$]{%
\begin{minipage}[b]{6cm}
\begin{center}
\begin{tikzpicture}[x=3mm,y=3mm]
  \ftileA{0}{0}{0}{};
  \tileAr{60}{-1}{0}{};
\end{tikzpicture}%
\end{center}%
\end{minipage}%
} \qquad \subfloat[Possible neighbour~$8$]{%
\begin{minipage}[b]{6cm}
\begin{center}
\begin{tikzpicture}[x=3mm,y=3mm]
  \ftileA{0}{0}{0}{};
  \tileA{240}{-1}{1}{};
\end{tikzpicture}%
\end{center}%
\end{minipage}%
} \\ \subfloat[Possible neighbour~$9$]{%
\begin{minipage}[b]{6cm}
\begin{center}
\begin{tikzpicture}[x=3mm,y=3mm]
  \ftileA{0}{0}{0}{};
  \tileA{300}{-1}{1}{};
\end{tikzpicture}%
\end{center}%
\end{minipage}%
} \qquad \subfloat[Possible neighbour~$10$]{%
\begin{minipage}[b]{6cm}
\begin{center}
\begin{tikzpicture}[x=3mm,y=3mm]
  \ftileA{0}{0}{0}{};
  \tileAr{0}{-1}{1}{};
\end{tikzpicture}%
\end{center}%
\end{minipage}%
} \\ \subfloat[Possible neighbour~$11$]{%
\begin{minipage}[b]{6cm}
\begin{center}
\begin{tikzpicture}[x=3mm,y=3mm]
  \ftileA{0}{0}{0}{};
  \tileAr{300}{-1}{1}{};
\end{tikzpicture}%
\end{center}%
\end{minipage}%
} \qquad \subfloat[Possible neighbour~$12$]{%
\begin{minipage}[b]{6cm}
\begin{center}
\begin{tikzpicture}[x=3mm,y=3mm]
  \ftileA{0}{0}{0}{};
  \tileAr{240}{-1}{2}{};
\end{tikzpicture}%
\end{center}%
\end{minipage}%
}%
\end{center}
\caption{Possible neighbours (part 2)}
\label{fig:nbr:2}
\end{figure}
\begin{figure}[htp!]
\renewcommand\thesubfigure{\arabic{subfigure}}%
\ContinuedFloat
\captionsetup{margin=0pt,justification=raggedright}%
\begin{center}
\subfloat[Possible neighbour~$13$]{%
\begin{minipage}[b]{6cm}
\begin{center}
\begin{tikzpicture}[x=3mm,y=3mm]
  \ftileA{0}{0}{0}{};
  \tileA{0}{0}{-1}{};
\end{tikzpicture}%
\end{center}%
\end{minipage}%
} \qquad \subfloat[Possible neighbour~$14$]{%
\begin{minipage}[b]{6cm}
\begin{center}
\begin{tikzpicture}[x=3mm,y=3mm]
  \ftileA{0}{0}{0}{};
  \tileA{180}{0}{-1}{};
\end{tikzpicture}%
\end{center}%
\end{minipage}%
} \\ \subfloat[Possible neighbour~$15$]{%
\begin{minipage}[b]{6cm}
\begin{center}
\begin{tikzpicture}[x=3mm,y=3mm]
  \ftileA{0}{0}{0}{};
  \tileA{240}{0}{-1}{};
\end{tikzpicture}%
\end{center}%
\end{minipage}%
} \qquad \subfloat[Possible neighbour~$16$]{%
\begin{minipage}[b]{6cm}
\begin{center}
\begin{tikzpicture}[x=3mm,y=3mm]
  \ftileA{0}{0}{0}{};
  \tileAr{120}{0}{-1}{};
\end{tikzpicture}%
\end{center}%
\end{minipage}%
} \\ \subfloat[Possible neighbour~$17$]{%
\begin{minipage}[b]{6cm}
\begin{center}
\begin{tikzpicture}[x=3mm,y=3mm]
  \ftileA{0}{0}{0}{};
  \tileA{0}{0}{1}{};
\end{tikzpicture}%
\end{center}%
\end{minipage}%
} \qquad \subfloat[Possible neighbour~$18$]{%
\begin{minipage}[b]{6cm}
\begin{center}
\begin{tikzpicture}[x=3mm,y=3mm]
  \ftileA{0}{0}{0}{};
  \tileA{60}{0}{1}{};
\end{tikzpicture}%
\end{center}%
\end{minipage}%
} \\ \subfloat[Possible neighbour~$19$]{%
\begin{minipage}[b]{6cm}
\begin{center}
\begin{tikzpicture}[x=3mm,y=3mm]
  \ftileA{0}{0}{0}{};
  \tileA{120}{0}{1}{};
\end{tikzpicture}%
\end{center}%
\end{minipage}%
} \qquad \subfloat[Possible neighbour~$20$]{%
\begin{minipage}[b]{6cm}
\begin{center}
\begin{tikzpicture}[x=3mm,y=3mm]
  \ftileA{0}{0}{0}{};
  \tileA{180}{0}{1}{};
\end{tikzpicture}%
\end{center}%
\end{minipage}%
} \\ \subfloat[Possible neighbour~$21$]{%
\begin{minipage}[b]{6cm}
\begin{center}
\begin{tikzpicture}[x=3mm,y=3mm]
  \ftileA{0}{0}{0}{};
  \tileAr{0}{0}{1}{};
\end{tikzpicture}%
\end{center}%
\end{minipage}%
} \qquad \subfloat[Possible neighbour~$22$]{%
\begin{minipage}[b]{6cm}
\begin{center}
\begin{tikzpicture}[x=3mm,y=3mm]
  \ftileA{0}{0}{0}{};
  \tileAr{60}{0}{1}{};
\end{tikzpicture}%
\end{center}%
\end{minipage}%
}%
\end{center}
\caption{Possible neighbours (part 3)}
\label{fig:nbr:3}
\end{figure}
\begin{figure}[htp!]
\renewcommand\thesubfigure{\arabic{subfigure}}%
\ContinuedFloat
\captionsetup{margin=0pt,justification=raggedright}%
\begin{center}
\subfloat[Possible neighbour~$23$]{%
\begin{minipage}[b]{6cm}
\begin{center}
\begin{tikzpicture}[x=3mm,y=3mm]
  \ftileA{0}{0}{0}{};
  \tileAr{120}{0}{1}{};
\end{tikzpicture}%
\end{center}%
\end{minipage}%
} \qquad \subfloat[Possible neighbour~$24$]{%
\begin{minipage}[b]{6cm}
\begin{center}
\begin{tikzpicture}[x=3mm,y=3mm]
  \ftileA{0}{0}{0}{};
  \tileAr{180}{0}{1}{};
\end{tikzpicture}%
\end{center}%
\end{minipage}%
} \\ \subfloat[Possible neighbour~$25$]{%
\begin{minipage}[b]{6cm}
\begin{center}
\begin{tikzpicture}[x=3mm,y=3mm]
  \ftileA{0}{0}{0}{};
  \tileA{300}{0}{2}{};
\end{tikzpicture}%
\end{center}%
\end{minipage}%
} \qquad \subfloat[Possible neighbour~$26$]{%
\begin{minipage}[b]{6cm}
\begin{center}
\begin{tikzpicture}[x=3mm,y=3mm]
  \ftileA{0}{0}{0}{};
  \tileA{120}{1}{-2}{};
\end{tikzpicture}%
\end{center}%
\end{minipage}%
} \\ \subfloat[Possible neighbour~$27$]{%
\begin{minipage}[b]{6cm}
\begin{center}
\begin{tikzpicture}[x=3mm,y=3mm]
  \ftileA{0}{0}{0}{};
  \tileA{300}{1}{-1}{};
\end{tikzpicture}%
\end{center}%
\end{minipage}%
} \qquad \subfloat[Possible neighbour~$28$]{%
\begin{minipage}[b]{6cm}
\begin{center}
\begin{tikzpicture}[x=3mm,y=3mm]
  \ftileA{0}{0}{0}{};
  \tileAr{180}{1}{-1}{};
\end{tikzpicture}%
\end{center}%
\end{minipage}%
} \\ \subfloat[Possible neighbour~$29$]{%
\begin{minipage}[b]{6cm}
\begin{center}
\begin{tikzpicture}[x=3mm,y=3mm]
  \ftileA{0}{0}{0}{};
  \tileA{60}{1}{0}{};
\end{tikzpicture}%
\end{center}%
\end{minipage}%
} \qquad \subfloat[Possible neighbour~$30$]{%
\begin{minipage}[b]{6cm}
\begin{center}
\begin{tikzpicture}[x=3mm,y=3mm]
  \ftileA{0}{0}{0}{};
  \tileAr{300}{1}{0}{};
\end{tikzpicture}%
\end{center}%
\end{minipage}%
} \\ \subfloat[Possible neighbour~$31$]{%
\begin{minipage}[b]{6cm}
\begin{center}
\begin{tikzpicture}[x=3mm,y=3mm]
  \ftileA{0}{0}{0}{};
  \tileA{180}{1}{1}{};
\end{tikzpicture}%
\end{center}%
\end{minipage}%
} \qquad \subfloat[Possible neighbour~$32$]{%
\begin{minipage}[b]{6cm}
\begin{center}
\begin{tikzpicture}[x=3mm,y=3mm]
  \ftileA{0}{0}{0}{};
  \tileA{240}{1}{1}{};
\end{tikzpicture}%
\end{center}%
\end{minipage}%
}%
\end{center}
\caption{Possible neighbours (part 4)}
\label{fig:nbr:4}
\end{figure}
\begin{figure}[htp!]
\renewcommand\thesubfigure{\arabic{subfigure}}%
\ContinuedFloat
\captionsetup{margin=0pt,justification=raggedright}%
\begin{center}
\subfloat[Possible neighbour~$33$]{%
\begin{minipage}[b]{6cm}
\begin{center}
\begin{tikzpicture}[x=3mm,y=3mm]
  \ftileA{0}{0}{0}{};
  \tileA{60}{2}{-2}{};
\end{tikzpicture}%
\end{center}%
\end{minipage}%
} \qquad \subfloat[Possible neighbour~$34$]{%
\begin{minipage}[b]{6cm}
\begin{center}
\begin{tikzpicture}[x=3mm,y=3mm]
  \ftileA{0}{0}{0}{};
  \tileA{120}{2}{-2}{};
\end{tikzpicture}%
\end{center}%
\end{minipage}%
} \\ \subfloat[Possible neighbour~$35$]{%
\begin{minipage}[b]{6cm}
\begin{center}
\begin{tikzpicture}[x=3mm,y=3mm]
  \ftileA{0}{0}{0}{};
  \tileA{0}{2}{-1}{};
\end{tikzpicture}%
\end{center}%
\end{minipage}%
} \qquad \subfloat[Possible neighbour~$36$]{%
\begin{minipage}[b]{6cm}
\begin{center}
\begin{tikzpicture}[x=3mm,y=3mm]
  \ftileA{0}{0}{0}{};
  \tileA{120}{2}{-1}{};
\end{tikzpicture}%
\end{center}%
\end{minipage}%
} \\ \subfloat[Possible neighbour~$37$]{%
\begin{minipage}[b]{6cm}
\begin{center}
\begin{tikzpicture}[x=3mm,y=3mm]
  \ftileA{0}{0}{0}{};
  \tileA{240}{2}{-1}{};
\end{tikzpicture}%
\end{center}%
\end{minipage}%
} \qquad \subfloat[Possible neighbour~$38$]{%
\begin{minipage}[b]{6cm}
\begin{center}
\begin{tikzpicture}[x=3mm,y=3mm]
  \ftileA{0}{0}{0}{};
  \tileA{300}{2}{-1}{};
\end{tikzpicture}%
\end{center}%
\end{minipage}%
} \\ \subfloat[Possible neighbour~$39$]{%
\begin{minipage}[b]{6cm}
\begin{center}
\begin{tikzpicture}[x=3mm,y=3mm]
  \ftileA{0}{0}{0}{};
  \tileAr{240}{2}{-1}{};
\end{tikzpicture}%
\end{center}%
\end{minipage}%
} \qquad \subfloat[Possible neighbour~$40$]{%
\begin{minipage}[b]{6cm}
\begin{center}
\begin{tikzpicture}[x=3mm,y=3mm]
  \ftileA{0}{0}{0}{};
  \tileA{240}{2}{0}{};
\end{tikzpicture}%
\end{center}%
\end{minipage}%
} \\ \subfloat[Possible neighbour~$41$]{%
\begin{minipage}[b]{6cm}
\begin{center}
\begin{tikzpicture}[x=3mm,y=3mm]
  \ftileA{0}{0}{0}{};
  \tileA{180}{3}{-2}{};
\end{tikzpicture}%
\end{center}%
\end{minipage}%
} \qquad \subfloat[Possible neighbour~$42$ (eliminated by considering neighbours containing the shaded kite)]{%
\begin{minipage}[b]{6cm}
\begin{center}
\begin{tikzpicture}[x=3mm,y=3mm]
  \ffkite{120}{1}{-1};
  \ftileA{0}{0}{0}{};
  \tileAr{300}{0}{-1}{};
\end{tikzpicture}%
\end{center}%
\end{minipage}%
}%
\end{center}
\caption{Possible neighbours (part 5)}
\label{fig:nbr:5}
\end{figure}
\begin{figure}[htp!]
\renewcommand\thesubfigure{\arabic{subfigure}}%
\ContinuedFloat
\captionsetup{margin=0pt,justification=raggedright}%
\begin{center}
\subfloat[Possible neighbour~$43$ (eliminated by considering neighbours containing the shaded kite)]{%
\begin{minipage}[b]{6cm}
\begin{center}
\begin{tikzpicture}[x=3mm,y=3mm]
  \ffkite{120}{1}{-1};
  \ftileA{0}{0}{0}{};
  \tileA{60}{1}{-2}{};
\end{tikzpicture}%
\end{center}%
\end{minipage}%
} \qquad \subfloat[Possible neighbour~$44$ (eliminated together with possible neighbour~$43$)]{%
\begin{minipage}[b]{6cm}
\begin{center}
\begin{tikzpicture}[x=3mm,y=3mm]
  \ftileA{0}{0}{0}{};
  \tileA{300}{1}{1}{};
\end{tikzpicture}%
\end{center}%
\end{minipage}%
} \\ \subfloat[Possible neighbour~$45$ (eliminated by considering neighbours containing the shaded kite)]{%
\begin{minipage}[b]{6cm}
\begin{center}
\begin{tikzpicture}[x=3mm,y=3mm]
  \ffkite{0}{1}{-1};
  \ftileA{0}{0}{0}{};
  \tileAr{0}{1}{-2}{};
\end{tikzpicture}%
\end{center}%
\end{minipage}%
} \qquad \subfloat[Possible neighbour~$46$ (eliminated by considering neighbours containing the shaded kite)]{%
\begin{minipage}[b]{6cm}
\begin{center}
\begin{tikzpicture}[x=3mm,y=3mm]
  \ffkite{120}{1}{-1};
  \ftileA{0}{0}{0}{};
  \tileAr{60}{1}{-2}{};
\end{tikzpicture}%
\end{center}%
\end{minipage}%
} \\ \subfloat[Possible neighbour~$47$ (eliminated together with possible neighbour~$46$)]{%
\begin{minipage}[b]{6cm}
\begin{center}
\begin{tikzpicture}[x=3mm,y=3mm]
  \ftileA{0}{0}{0}{};
  \tileAr{60}{2}{-1}{};
\end{tikzpicture}%
\end{center}%
\end{minipage}%
} \qquad \subfloat[Possible neighbour~$48$ (eliminated by considering neighbours containing the shaded kite)]{%
\begin{minipage}[b]{6cm}
\begin{center}
\begin{tikzpicture}[x=3mm,y=3mm]
  \ffkite{60}{1}{0};
  \ftileA{0}{0}{0}{};
  \tileAr{240}{1}{1}{};
\end{tikzpicture}%
\end{center}%
\end{minipage}%
} \\ \subfloat[Possible neighbour~$49$ (eliminated by considering neighbours containing the shaded kite)]{%
\begin{minipage}[b]{6cm}
\begin{center}
\begin{tikzpicture}[x=3mm,y=3mm]
  \ffkite{120}{1}{-1};
  \ftileA{0}{0}{0}{};
  \tileA{180}{2}{-2}{};
\end{tikzpicture}%
\end{center}%
\end{minipage}%
} \qquad \subfloat[Possible neighbour~$50$ (eliminated by considering neighbours containing the shaded kite)]{%
\begin{minipage}[b]{6cm}
\begin{center}
\begin{tikzpicture}[x=3mm,y=3mm]
  \ffkite{0}{1}{-1};
  \ftileA{0}{0}{0}{};
  \tileAr{120}{2}{-2}{};
\end{tikzpicture}%
\end{center}%
\end{minipage}%
}%
\end{center}
\caption{Possible neighbours (part 6)}
\label{fig:nbr:6}
\end{figure}
\begin{figure}[htp!]
\renewcommand\thesubfigure{\arabic{subfigure}}%
\ContinuedFloat
\captionsetup{margin=0pt,justification=raggedright}%
\begin{center}
\subfloat[Possible neighbour~$51$ (eliminated together with possible neighbour~$50$)]{%
\begin{minipage}[b]{6cm}
\begin{center}
\begin{tikzpicture}[x=3mm,y=3mm]
  \ftileA{0}{0}{0}{};
  \tileAr{120}{2}{0}{};
\end{tikzpicture}%
\end{center}%
\end{minipage}%
} \qquad \subfloat[Possible neighbour~$52$ (eliminated by considering neighbours containing the shaded kite)]{%
\begin{minipage}[b]{6cm}
\begin{center}
\begin{tikzpicture}[x=3mm,y=3mm]
  \ffkite{60}{1}{0};
  \ftileA{0}{0}{0}{};
  \tileAr{0}{2}{-1}{};
\end{tikzpicture}%
\end{center}%
\end{minipage}%
} \\ \subfloat[Possible neighbour~$53$ (eliminated together with possible neighbour~$52$)]{%
\begin{minipage}[b]{6cm}
\begin{center}
\begin{tikzpicture}[x=3mm,y=3mm]
  \ftileA{0}{0}{0}{};
  \tileAr{0}{-1}{-1}{};
\end{tikzpicture}%
\end{center}%
\end{minipage}%
} \qquad \subfloat[Possible neighbour~$54$ (eliminated by considering neighbours containing the shaded kite)]{%
\begin{minipage}[b]{6cm}
\begin{center}
\begin{tikzpicture}[x=3mm,y=3mm]
  \ffkite{60}{1}{0};
  \ftileA{0}{0}{0}{};
  \tileA{180}{2}{0}{};
\end{tikzpicture}%
\end{center}%
\end{minipage}%
} \\ \subfloat[Possible neighbour~$55$ (eliminated by considering neighbours containing the shaded kite)]{%
\begin{minipage}[b]{6cm}
\begin{center}
\begin{tikzpicture}[x=3mm,y=3mm]
  \ffkite{60}{1}{0};
  \ftileA{0}{0}{0}{};
  \tileAr{180}{2}{0}{};
\end{tikzpicture}%
\end{center}%
\end{minipage}%
} \qquad \subfloat[Possible neighbour~$56$ (eliminated by considering neighbours containing the shaded kite)]{%
\begin{minipage}[b]{6cm}
\begin{center}
\begin{tikzpicture}[x=3mm,y=3mm]
  \ffkite{240}{0}{-1};
  \ftileA{0}{0}{0}{};
  \tileAr{180}{-1}{0}{};
\end{tikzpicture}%
\end{center}%
\end{minipage}%
} \\ \subfloat[Possible neighbour~$57$ (eliminated by considering neighbours containing the shaded kite)]{%
\begin{minipage}[b]{6cm}
\begin{center}
\begin{tikzpicture}[x=3mm,y=3mm]
  \ffkite{180}{0}{-1};
  \ftileA{0}{0}{0}{};
  \tileAr{240}{-1}{0}{};
\end{tikzpicture}%
\end{center}%
\end{minipage}%
} \qquad \subfloat[Possible neighbour~$58$ (eliminated together with possible neighbour~$57$)]{%
\begin{minipage}[b]{6cm}
\begin{center}
\begin{tikzpicture}[x=3mm,y=3mm]
  \ftileA{0}{0}{0}{};
  \tileAr{240}{0}{-1}{};
\end{tikzpicture}%
\end{center}%
\end{minipage}%
}%
\end{center}
\caption{Possible neighbours (part 7)}
\label{fig:nbr:7}
\end{figure}


\FloatBarrier

\subsection{Enumeration of $1$-patches}

Having produced a list of possible neighbours, we now proceed to
enumerating possible $1$-patches.  When we have a partial $1$-patch
(some number of neighbours for the original, shaded polykite), we pick
some kite neighbouring that original polykite and enumerate the
possible neighbouring polykites containing that kite, excluding any
that would result in the patch containing two polykites that either
intersect or form a pair of neighbours previously ruled out; the kite
we use is chosen so that the number of choices for the neighbour added
is minimal.  This process results in $37$~possible $1$-patches; the
partial patches from the search process are shown in
Figure~\ref{fig:partpatch} and the $1$-patches are shown in
Figures~\ref{fig:patch}.

Some of the $1$-patches found can be immediately eliminated at this
point, by identifying a tile in the $1$-patch that cannot itself be
surrounded by any of the $1$-patches (that has not yet been
eliminated) without resulting in either an intersection or a pair of
neighbours that were previously ruled out.  In the $12$ cases 
implicated here, the tile that cannot be surrounded is shaded, and they are
eliminated in the order shown, leaving $25$~remaining $1$-patches.
For each of those remaining $1$-patches, the classification of the
central tile by the
rules in \secref{sec:clusters} is shown.

% Automatically generated figures.
\begin{figure}[htp!]
\renewcommand\thesubfigure{\arabic{subfigure}}%
\captionsetup{margin=0pt,justification=raggedright}%
\begin{center}
\subfloat[Partial patch 1 (extends to partial patches 2--4)]{%
\begin{minipage}[b]{6cm}
\begin{center}
\begin{tikzpicture}[x=3mm,y=3mm]
  \ffkite{180}{0}{-1};
  \ftileA{0}{0}{0}{};
\end{tikzpicture}%
\end{center}%
\end{minipage}%
} \qquad \subfloat[Partial patch 2 (extends to partial patches 5--6)]{%
\begin{minipage}[b]{6cm}
\begin{center}
\begin{tikzpicture}[x=3mm,y=3mm]
  \ffkite{240}{0}{-1};
  \tileA{0}{0}{-1}{};
  \ftileA{0}{0}{0}{};
\end{tikzpicture}%
\end{center}%
\end{minipage}%
} \\ \subfloat[Partial patch 3 (extends to partial patches 7--8)]{%
\begin{minipage}[b]{6cm}
\begin{center}
\begin{tikzpicture}[x=3mm,y=3mm]
  \ffkite{120}{-1}{0};
  \tileA{180}{0}{-1}{};
  \ftileA{0}{0}{0}{};
\end{tikzpicture}%
\end{center}%
\end{minipage}%
} \qquad \subfloat[Partial patch 4 (extends to partial patches 9--10)]{%
\begin{minipage}[b]{6cm}
\begin{center}
\begin{tikzpicture}[x=3mm,y=3mm]
  \ffkite{240}{0}{-1};
  \ftileA{0}{0}{0}{};
  \tileAr{180}{1}{-1}{};
\end{tikzpicture}%
\end{center}%
\end{minipage}%
} \\ \subfloat[Partial patch 5 (extends to partial patches 11--12)]{%
\begin{minipage}[b]{6cm}
\begin{center}
\begin{tikzpicture}[x=3mm,y=3mm]
  \ffkite{120}{-1}{0};
  \tileA{60}{-1}{-1}{};
  \tileA{0}{0}{-1}{};
  \ftileA{0}{0}{0}{};
\end{tikzpicture}%
\end{center}%
\end{minipage}%
} \qquad \subfloat[Partial patch 6 (extends to partial patches 13--14)]{%
\begin{minipage}[b]{6cm}
\begin{center}
\begin{tikzpicture}[x=3mm,y=3mm]
  \ffkite{180}{-1}{0};
  \tileA{300}{-1}{0}{};
  \tileA{0}{0}{-1}{};
  \ftileA{0}{0}{0}{};
\end{tikzpicture}%
\end{center}%
\end{minipage}%
}%
\end{center}
\caption{Partial patches (part 1)}
\label{fig:partpatch}
\end{figure}
\begin{figure}[htp!]
\renewcommand\thesubfigure{\arabic{subfigure}}%
\ContinuedFloat
\captionsetup{margin=0pt,justification=raggedright}%
\begin{center}
\subfloat[Partial patch 7 (extends to partial patch 15)]{%
\begin{minipage}[b]{6cm}
\begin{center}
\begin{tikzpicture}[x=3mm,y=3mm]
  \ffkite{240}{0}{0};
  \tileA{120}{-1}{0}{};
  \tileA{180}{0}{-1}{};
  \ftileA{0}{0}{0}{};
\end{tikzpicture}%
\end{center}%
\end{minipage}%
} \qquad \subfloat[Partial patch 8 (extends to partial patches 16--17)]{%
\begin{minipage}[b]{6cm}
\begin{center}
\begin{tikzpicture}[x=3mm,y=3mm]
  \ffkite{0}{1}{-1};
  \tileA{300}{-1}{1}{};
  \tileA{180}{0}{-1}{};
  \ftileA{0}{0}{0}{};
\end{tikzpicture}%
\end{center}%
\end{minipage}%
} \\ \subfloat[Partial patch 9 (extends to partial patches 18--19)]{%
\begin{minipage}[b]{6cm}
\begin{center}
\begin{tikzpicture}[x=3mm,y=3mm]
  \ffkite{0}{2}{-1};
  \tileA{240}{0}{-1}{};
  \ftileA{0}{0}{0}{};
  \tileAr{180}{1}{-1}{};
\end{tikzpicture}%
\end{center}%
\end{minipage}%
} \qquad \subfloat[Partial patch 10 (extends to partial patches 20--21)]{%
\begin{minipage}[b]{6cm}
\begin{center}
\begin{tikzpicture}[x=3mm,y=3mm]
  \ffkite{180}{-1}{0};
  \tileAr{120}{0}{-1}{};
  \ftileA{0}{0}{0}{};
  \tileAr{180}{1}{-1}{};
\end{tikzpicture}%
\end{center}%
\end{minipage}%
} \\ \subfloat[Partial patch 11 (extends to partial patch 22)]{%
\begin{minipage}[b]{6cm}
\begin{center}
\begin{tikzpicture}[x=3mm,y=3mm]
  \ffkite{240}{0}{0};
  \tileA{60}{-1}{-1}{};
  \tileA{120}{-1}{0}{};
  \tileA{0}{0}{-1}{};
  \ftileA{0}{0}{0}{};
\end{tikzpicture}%
\end{center}%
\end{minipage}%
} \qquad \subfloat[Partial patch 12 (extends to partial patches 23--24)]{%
\begin{minipage}[b]{6cm}
\begin{center}
\begin{tikzpicture}[x=3mm,y=3mm]
  \ffkite{60}{1}{-1};
  \tileA{60}{-1}{-1}{};
  \tileA{300}{-1}{1}{};
  \tileA{0}{0}{-1}{};
  \ftileA{0}{0}{0}{};
\end{tikzpicture}%
\end{center}%
\end{minipage}%
} \\ \subfloat[Partial patch 13 (extends to partial patch 25)]{%
\begin{minipage}[b]{6cm}
\begin{center}
\begin{tikzpicture}[x=3mm,y=3mm]
  \ffkite{240}{0}{0};
  \tileA{0}{-2}{1}{};
  \tileA{300}{-1}{0}{};
  \tileA{0}{0}{-1}{};
  \ftileA{0}{0}{0}{};
\end{tikzpicture}%
\end{center}%
\end{minipage}%
} \qquad \subfloat[Partial patch 14 (extends to partial patch 26)]{%
\begin{minipage}[b]{6cm}
\begin{center}
\begin{tikzpicture}[x=3mm,y=3mm]
  \ffkite{240}{0}{0};
  \tileA{300}{-1}{0}{};
  \tileA{240}{-1}{1}{};
  \tileA{0}{0}{-1}{};
  \ftileA{0}{0}{0}{};
\end{tikzpicture}%
\end{center}%
\end{minipage}%
}%
\end{center}
\caption{Partial patches (part 2)}
\label{fig:partpatch:2}
\end{figure}
\begin{figure}[htp!]
\renewcommand\thesubfigure{\arabic{subfigure}}%
\ContinuedFloat
\captionsetup{margin=0pt,justification=raggedright}%
\begin{center}
\subfloat[Partial patch 15 (extends to partial patches 27--28)]{%
\begin{minipage}[b]{6cm}
\begin{center}
\begin{tikzpicture}[x=3mm,y=3mm]
  \ffkite{120}{0}{1};
  \tileA{120}{-1}{0}{};
  \tileAr{300}{-1}{1}{};
  \tileA{180}{0}{-1}{};
  \ftileA{0}{0}{0}{};
\end{tikzpicture}%
\end{center}%
\end{minipage}%
} \qquad \subfloat[Partial patch 16 (extends to partial patches 29--30)]{%
\begin{minipage}[b]{6cm}
\begin{center}
\begin{tikzpicture}[x=3mm,y=3mm]
  \ffkite{60}{1}{-1};
  \tileA{300}{-1}{1}{};
  \tileA{180}{0}{-1}{};
  \ftileA{0}{0}{0}{};
  \tileA{120}{1}{-2}{};
\end{tikzpicture}%
\end{center}%
\end{minipage}%
} \\ \subfloat[Partial patch 17 (extends to partial patches 31--33)]{%
\begin{minipage}[b]{6cm}
\begin{center}
\begin{tikzpicture}[x=3mm,y=3mm]
  \ffkite{0}{2}{-1};
  \tileA{300}{-1}{1}{};
  \tileA{180}{0}{-1}{};
  \ftileA{0}{0}{0}{};
  \tileA{300}{1}{-1}{};
\end{tikzpicture}%
\end{center}%
\end{minipage}%
} \qquad \subfloat[Partial patch 18 (extends to partial patch 34)]{%
\begin{minipage}[b]{6cm}
\begin{center}
\begin{tikzpicture}[x=3mm,y=3mm]
  \ffkite{300}{2}{-1};
  \tileA{240}{0}{-1}{};
  \ftileA{0}{0}{0}{};
  \tileAr{180}{1}{-1}{};
  \tileA{60}{2}{-2}{};
\end{tikzpicture}%
\end{center}%
\end{minipage}%
} \\ \subfloat[Partial patch 19 (extends to partial patch 35)]{%
\begin{minipage}[b]{6cm}
\begin{center}
\begin{tikzpicture}[x=3mm,y=3mm]
  \ffkite{300}{2}{-1};
  \tileA{240}{0}{-1}{};
  \ftileA{0}{0}{0}{};
  \tileAr{180}{1}{-1}{};
  \tileAr{240}{2}{-1}{};
\end{tikzpicture}%
\end{center}%
\end{minipage}%
} \qquad \subfloat[Partial patch 20 (extends to partial patch 36)]{%
\begin{minipage}[b]{6cm}
\begin{center}
\begin{tikzpicture}[x=3mm,y=3mm]
  \ffkite{240}{0}{0};
  \tileA{180}{-1}{0}{};
  \tileAr{120}{0}{-1}{};
  \ftileA{0}{0}{0}{};
  \tileAr{180}{1}{-1}{};
\end{tikzpicture}%
\end{center}%
\end{minipage}%
} \\ \subfloat[Partial patch 21 (extends to partial patch 37)]{%
\begin{minipage}[b]{6cm}
\begin{center}
\begin{tikzpicture}[x=3mm,y=3mm]
  \ffkite{240}{0}{0};
  \tileAr{60}{-1}{0}{};
  \tileAr{120}{0}{-1}{};
  \ftileA{0}{0}{0}{};
  \tileAr{180}{1}{-1}{};
\end{tikzpicture}%
\end{center}%
\end{minipage}%
} \qquad \subfloat[Partial patch 22 (extends to partial patches 38--39)]{%
\begin{minipage}[b]{6cm}
\begin{center}
\begin{tikzpicture}[x=3mm,y=3mm]
  \ffkite{120}{0}{1};
  \tileA{60}{-1}{-1}{};
  \tileA{120}{-1}{0}{};
  \tileAr{300}{-1}{1}{};
  \tileA{0}{0}{-1}{};
  \ftileA{0}{0}{0}{};
\end{tikzpicture}%
\end{center}%
\end{minipage}%
}%
\end{center}
\caption{Partial patches (part 3)}
\label{fig:partpatch:3}
\end{figure}
\begin{figure}[htp!]
\renewcommand\thesubfigure{\arabic{subfigure}}%
\ContinuedFloat
\captionsetup{margin=0pt,justification=raggedright}%
\begin{center}
\subfloat[Partial patch 23 (extends to partial patches 40--41)]{%
\begin{minipage}[b]{6cm}
\begin{center}
\begin{tikzpicture}[x=3mm,y=3mm]
  \ffkite{60}{1}{0};
  \tileA{60}{-1}{-1}{};
  \tileA{300}{-1}{1}{};
  \tileA{0}{0}{-1}{};
  \ftileA{0}{0}{0}{};
  \tileA{120}{2}{-2}{};
\end{tikzpicture}%
\end{center}%
\end{minipage}%
} \qquad \subfloat[Partial patch 24 (extends to partial patch 42)]{%
\begin{minipage}[b]{6cm}
\begin{center}
\begin{tikzpicture}[x=3mm,y=3mm]
  \ffkite{60}{1}{0};
  \tileA{60}{-1}{-1}{};
  \tileA{300}{-1}{1}{};
  \tileA{0}{0}{-1}{};
  \ftileA{0}{0}{0}{};
  \tileA{240}{2}{-1}{};
\end{tikzpicture}%
\end{center}%
\end{minipage}%
} \\ \subfloat[Partial patch 25 (extends to partial patches 43--44)]{%
\begin{minipage}[b]{6cm}
\begin{center}
\begin{tikzpicture}[x=3mm,y=3mm]
  \ffkite{120}{0}{1};
  \tileA{0}{-2}{1}{};
  \tileA{300}{-1}{0}{};
  \tileAr{300}{-1}{1}{};
  \tileA{0}{0}{-1}{};
  \ftileA{0}{0}{0}{};
\end{tikzpicture}%
\end{center}%
\end{minipage}%
} \qquad \subfloat[Partial patch 26 (extends to partial patches 45--46)]{%
\begin{minipage}[b]{6cm}
\begin{center}
\begin{tikzpicture}[x=3mm,y=3mm]
  \ffkite{60}{1}{-1};
  \tileA{300}{-1}{0}{};
  \tileA{240}{-1}{1}{};
  \tileA{0}{0}{-1}{};
  \ftileA{0}{0}{0}{};
  \tileAr{180}{0}{1}{};
\end{tikzpicture}%
\end{center}%
\end{minipage}%
} \\ \subfloat[Partial patch 27 (extends to partial patch 47)]{%
\begin{minipage}[b]{6cm}
\begin{center}
\begin{tikzpicture}[x=3mm,y=3mm]
  \ffkite{180}{1}{0};
  \tileA{120}{-1}{0}{};
  \tileAr{300}{-1}{1}{};
  \tileA{180}{0}{-1}{};
  \ftileA{0}{0}{0}{};
  \tileA{120}{0}{1}{};
\end{tikzpicture}%
\end{center}%
\end{minipage}%
} \qquad \subfloat[Partial patch 28 (extends to partial patch 48)]{%
\begin{minipage}[b]{6cm}
\begin{center}
\begin{tikzpicture}[x=3mm,y=3mm]
  \ffkite{180}{1}{0};
  \tileA{120}{-1}{0}{};
  \tileAr{300}{-1}{1}{};
  \tileA{180}{0}{-1}{};
  \ftileA{0}{0}{0}{};
  \tileAr{0}{0}{1}{};
\end{tikzpicture}%
\end{center}%
\end{minipage}%
} \\ \subfloat[Partial patch 29 (extends to partial patches 49--50)]{%
\begin{minipage}[b]{6cm}
\begin{center}
\begin{tikzpicture}[x=3mm,y=3mm]
  \ffkite{60}{1}{0};
  \tileA{300}{-1}{1}{};
  \tileA{180}{0}{-1}{};
  \ftileA{0}{0}{0}{};
  \tileA{120}{1}{-2}{};
  \tileA{120}{2}{-2}{};
\end{tikzpicture}%
\end{center}%
\end{minipage}%
} \qquad \subfloat[Partial patch 30 (extends to partial patch 51)]{%
\begin{minipage}[b]{6cm}
\begin{center}
\begin{tikzpicture}[x=3mm,y=3mm]
  \ffkite{60}{1}{0};
  \tileA{300}{-1}{1}{};
  \tileA{180}{0}{-1}{};
  \ftileA{0}{0}{0}{};
  \tileA{120}{1}{-2}{};
  \tileA{240}{2}{-1}{};
\end{tikzpicture}%
\end{center}%
\end{minipage}%
}%
\end{center}
\caption{Partial patches (part 4)}
\label{fig:partpatch:4}
\end{figure}
\begin{figure}[htp!]
\renewcommand\thesubfigure{\arabic{subfigure}}%
\ContinuedFloat
\captionsetup{margin=0pt,justification=raggedright}%
\begin{center}
\subfloat[Partial patch 31 (extends to partial patch 52)]{%
\begin{minipage}[b]{6cm}
\begin{center}
\begin{tikzpicture}[x=3mm,y=3mm]
  \ffkite{300}{2}{-1};
  \tileA{300}{-1}{1}{};
  \tileA{180}{0}{-1}{};
  \ftileA{0}{0}{0}{};
  \tileA{300}{1}{-1}{};
  \tileA{0}{2}{-1}{};
\end{tikzpicture}%
\end{center}%
\end{minipage}%
} \qquad \subfloat[Partial patch 32 (extends to partial patches 53--54)]{%
\begin{minipage}[b]{6cm}
\begin{center}
\begin{tikzpicture}[x=3mm,y=3mm]
  \ffkite{60}{1}{0};
  \tileA{300}{-1}{1}{};
  \tileA{180}{0}{-1}{};
  \ftileA{0}{0}{0}{};
  \tileA{300}{1}{-1}{};
  \tileA{300}{2}{-1}{};
\end{tikzpicture}%
\end{center}%
\end{minipage}%
} \\ \subfloat[Partial patch 33 (extends to partial patch 55)]{%
\begin{minipage}[b]{6cm}
\begin{center}
\begin{tikzpicture}[x=3mm,y=3mm]
  \ffkite{300}{2}{-1};
  \tileA{300}{-1}{1}{};
  \tileA{180}{0}{-1}{};
  \ftileA{0}{0}{0}{};
  \tileA{300}{1}{-1}{};
  \tileA{180}{3}{-2}{};
\end{tikzpicture}%
\end{center}%
\end{minipage}%
} \qquad \subfloat[Partial patch 34 (extends to partial patches 56--57)]{%
\begin{minipage}[b]{6cm}
\begin{center}
\begin{tikzpicture}[x=3mm,y=3mm]
  \ffkite{180}{1}{0};
  \tileA{240}{0}{-1}{};
  \ftileA{0}{0}{0}{};
  \tileAr{180}{1}{-1}{};
  \tileA{60}{2}{-2}{};
  \tileA{120}{2}{-1}{};
\end{tikzpicture}%
\end{center}%
\end{minipage}%
} \\ \subfloat[Partial patch 35 (extends to partial patches 58--60)]{%
\begin{minipage}[b]{6cm}
\begin{center}
\begin{tikzpicture}[x=3mm,y=3mm]
  \ffkite{120}{-1}{0};
  \tileA{240}{0}{-1}{};
  \ftileA{0}{0}{0}{};
  \tileAr{180}{1}{-1}{};
  \tileAr{300}{1}{0}{};
  \tileAr{240}{2}{-1}{};
\end{tikzpicture}%
\end{center}%
\end{minipage}%
} \qquad \subfloat[Partial patch 36 (extends to partial patches 61--62)]{%
\begin{minipage}[b]{6cm}
\begin{center}
\begin{tikzpicture}[x=3mm,y=3mm]
  \ffkite{120}{0}{1};
  \tileA{180}{-1}{0}{};
  \tileAr{300}{-1}{1}{};
  \tileAr{120}{0}{-1}{};
  \ftileA{0}{0}{0}{};
  \tileAr{180}{1}{-1}{};
\end{tikzpicture}%
\end{center}%
\end{minipage}%
} \\ \subfloat[Partial patch 37 (extends to partial patches 63--64)]{%
\begin{minipage}[b]{6cm}
\begin{center}
\begin{tikzpicture}[x=3mm,y=3mm]
  \ffkite{180}{1}{0};
  \tileAr{60}{-1}{0}{};
  \tileAr{120}{0}{-1}{};
  \ftileA{0}{0}{0}{};
  \tileAr{180}{0}{1}{};
  \tileAr{180}{1}{-1}{};
\end{tikzpicture}%
\end{center}%
\end{minipage}%
} \qquad \subfloat[Partial patch 38 (extends to partial patch 65)]{%
\begin{minipage}[b]{6cm}
\begin{center}
\begin{tikzpicture}[x=3mm,y=3mm]
  \ffkite{180}{1}{0};
  \tileA{60}{-1}{-1}{};
  \tileA{120}{-1}{0}{};
  \tileAr{300}{-1}{1}{};
  \tileA{0}{0}{-1}{};
  \ftileA{0}{0}{0}{};
  \tileA{120}{0}{1}{};
\end{tikzpicture}%
\end{center}%
\end{minipage}%
}%
\end{center}
\caption{Partial patches (part 5)}
\label{fig:partpatch:5}
\end{figure}
\begin{figure}[htp!]
\renewcommand\thesubfigure{\arabic{subfigure}}%
\ContinuedFloat
\captionsetup{margin=0pt,justification=raggedright}%
\begin{center}
\subfloat[Partial patch 39 (extends to partial patch 66)]{%
\begin{minipage}[b]{6cm}
\begin{center}
\begin{tikzpicture}[x=3mm,y=3mm]
  \ffkite{180}{1}{0};
  \tileA{60}{-1}{-1}{};
  \tileA{120}{-1}{0}{};
  \tileAr{300}{-1}{1}{};
  \tileA{0}{0}{-1}{};
  \ftileA{0}{0}{0}{};
  \tileAr{0}{0}{1}{};
\end{tikzpicture}%
\end{center}%
\end{minipage}%
} \qquad \subfloat[Partial patch 40 (extends to partial patches 67--68)]{%
\begin{minipage}[b]{6cm}
\begin{center}
\begin{tikzpicture}[x=3mm,y=3mm]
  \ffkite{60}{0}{1};
  \tileA{60}{-1}{-1}{};
  \tileA{300}{-1}{1}{};
  \tileA{0}{0}{-1}{};
  \ftileA{0}{0}{0}{};
  \tileA{60}{1}{0}{};
  \tileA{120}{2}{-2}{};
\end{tikzpicture}%
\end{center}%
\end{minipage}%
} \\ \subfloat[Partial patch 41 (extends to partial patch 69, $1$-patch 1)]{%
\begin{minipage}[b]{6cm}
\begin{center}
\begin{tikzpicture}[x=3mm,y=3mm]
  \ffkite{180}{1}{0};
  \tileA{60}{-1}{-1}{};
  \tileA{300}{-1}{1}{};
  \tileA{0}{0}{-1}{};
  \ftileA{0}{0}{0}{};
  \tileA{120}{2}{-2}{};
  \tileA{240}{2}{0}{};
\end{tikzpicture}%
\end{center}%
\end{minipage}%
} \qquad \subfloat[Partial patch 42 (extends to partial patches 70--71)]{%
\begin{minipage}[b]{6cm}
\begin{center}
\begin{tikzpicture}[x=3mm,y=3mm]
  \ffkite{60}{0}{1};
  \tileA{60}{-1}{-1}{};
  \tileA{300}{-1}{1}{};
  \tileA{0}{0}{-1}{};
  \ftileA{0}{0}{0}{};
  \tileA{60}{1}{0}{};
  \tileA{240}{2}{-1}{};
\end{tikzpicture}%
\end{center}%
\end{minipage}%
} \\ \subfloat[Partial patch 43 (extends to partial patch 72)]{%
\begin{minipage}[b]{6cm}
\begin{center}
\begin{tikzpicture}[x=3mm,y=3mm]
  \ffkite{180}{1}{0};
  \tileA{0}{-2}{1}{};
  \tileA{300}{-1}{0}{};
  \tileAr{300}{-1}{1}{};
  \tileA{0}{0}{-1}{};
  \ftileA{0}{0}{0}{};
  \tileA{120}{0}{1}{};
\end{tikzpicture}%
\end{center}%
\end{minipage}%
} \qquad \subfloat[Partial patch 44 (extends to partial patch 73)]{%
\begin{minipage}[b]{6cm}
\begin{center}
\begin{tikzpicture}[x=3mm,y=3mm]
  \ffkite{180}{1}{0};
  \tileA{0}{-2}{1}{};
  \tileA{300}{-1}{0}{};
  \tileAr{300}{-1}{1}{};
  \tileA{0}{0}{-1}{};
  \ftileA{0}{0}{0}{};
  \tileAr{0}{0}{1}{};
\end{tikzpicture}%
\end{center}%
\end{minipage}%
} \\ \subfloat[Partial patch 45 (extends to $1$-patch 2)]{%
\begin{minipage}[b]{6cm}
\begin{center}
\begin{tikzpicture}[x=3mm,y=3mm]
  \ffkite{180}{1}{0};
  \tileA{300}{-1}{0}{};
  \tileA{240}{-1}{1}{};
  \tileA{0}{0}{-1}{};
  \ftileA{0}{0}{0}{};
  \tileAr{180}{0}{1}{};
  \tileA{120}{2}{-2}{};
\end{tikzpicture}%
\end{center}%
\end{minipage}%
} \qquad \subfloat[Partial patch 46 (extends to $1$-patch 3)]{%
\begin{minipage}[b]{6cm}
\begin{center}
\begin{tikzpicture}[x=3mm,y=3mm]
  \ffkite{60}{1}{0};
  \tileA{300}{-1}{0}{};
  \tileA{240}{-1}{1}{};
  \tileA{0}{0}{-1}{};
  \ftileA{0}{0}{0}{};
  \tileAr{180}{0}{1}{};
  \tileA{240}{2}{-1}{};
\end{tikzpicture}%
\end{center}%
\end{minipage}%
}%
\end{center}
\caption{Partial patches (part 6)}
\label{fig:partpatch:6}
\end{figure}
\begin{figure}[htp!]
\renewcommand\thesubfigure{\arabic{subfigure}}%
\ContinuedFloat
\captionsetup{margin=0pt,justification=raggedright}%
\begin{center}
\subfloat[Partial patch 47 (extends to partial patches 74--75)]{%
\begin{minipage}[b]{6cm}
\begin{center}
\begin{tikzpicture}[x=3mm,y=3mm]
  \ffkite{0}{1}{-1};
  \tileA{120}{-1}{0}{};
  \tileAr{300}{-1}{1}{};
  \tileA{180}{0}{-1}{};
  \ftileA{0}{0}{0}{};
  \tileA{120}{0}{1}{};
  \tileA{60}{1}{0}{};
\end{tikzpicture}%
\end{center}%
\end{minipage}%
} \qquad \subfloat[Partial patch 48 (extends to partial patch 76)]{%
\begin{minipage}[b]{6cm}
\begin{center}
\begin{tikzpicture}[x=3mm,y=3mm]
  \ffkite{60}{1}{-1};
  \tileA{120}{-1}{0}{};
  \tileAr{300}{-1}{1}{};
  \tileA{180}{0}{-1}{};
  \ftileA{0}{0}{0}{};
  \tileAr{0}{0}{1}{};
  \tileAr{300}{1}{0}{};
\end{tikzpicture}%
\end{center}%
\end{minipage}%
} \\ \subfloat[Partial patch 49 (extends to partial patches 77--78)]{%
\begin{minipage}[b]{6cm}
\begin{center}
\begin{tikzpicture}[x=3mm,y=3mm]
  \ffkite{60}{0}{1};
  \tileA{300}{-1}{1}{};
  \tileA{180}{0}{-1}{};
  \ftileA{0}{0}{0}{};
  \tileA{120}{1}{-2}{};
  \tileA{60}{1}{0}{};
  \tileA{120}{2}{-2}{};
\end{tikzpicture}%
\end{center}%
\end{minipage}%
} \qquad \subfloat[Partial patch 50 (extends to partial patch 79, $1$-patch 4)]{%
\begin{minipage}[b]{6cm}
\begin{center}
\begin{tikzpicture}[x=3mm,y=3mm]
  \ffkite{180}{1}{0};
  \tileA{300}{-1}{1}{};
  \tileA{180}{0}{-1}{};
  \ftileA{0}{0}{0}{};
  \tileA{120}{1}{-2}{};
  \tileA{120}{2}{-2}{};
  \tileA{240}{2}{0}{};
\end{tikzpicture}%
\end{center}%
\end{minipage}%
} \\ \subfloat[Partial patch 51 (extends to partial patches 80--81)]{%
\begin{minipage}[b]{6cm}
\begin{center}
\begin{tikzpicture}[x=3mm,y=3mm]
  \ffkite{60}{0}{1};
  \tileA{300}{-1}{1}{};
  \tileA{180}{0}{-1}{};
  \ftileA{0}{0}{0}{};
  \tileA{120}{1}{-2}{};
  \tileA{60}{1}{0}{};
  \tileA{240}{2}{-1}{};
\end{tikzpicture}%
\end{center}%
\end{minipage}%
} \qquad \subfloat[Partial patch 52 (extends to partial patch 82)]{%
\begin{minipage}[b]{6cm}
\begin{center}
\begin{tikzpicture}[x=3mm,y=3mm]
  \ffkite{60}{0}{1};
  \tileA{300}{-1}{1}{};
  \tileA{180}{0}{-1}{};
  \ftileA{0}{0}{0}{};
  \tileA{300}{1}{-1}{};
  \tileAr{300}{1}{0}{};
  \tileA{0}{2}{-1}{};
\end{tikzpicture}%
\end{center}%
\end{minipage}%
} \\ \subfloat[Partial patch 53 (extends to partial patches 83--84)]{%
\begin{minipage}[b]{6cm}
\begin{center}
\begin{tikzpicture}[x=3mm,y=3mm]
  \ffkite{60}{0}{1};
  \tileA{300}{-1}{1}{};
  \tileA{180}{0}{-1}{};
  \ftileA{0}{0}{0}{};
  \tileA{300}{1}{-1}{};
  \tileA{60}{1}{0}{};
  \tileA{300}{2}{-1}{};
\end{tikzpicture}%
\end{center}%
\end{minipage}%
} \qquad \subfloat[Partial patch 54 (extends to partial patch 85, $1$-patch 5)]{%
\begin{minipage}[b]{6cm}
\begin{center}
\begin{tikzpicture}[x=3mm,y=3mm]
  \ffkite{180}{1}{0};
  \tileA{300}{-1}{1}{};
  \tileA{180}{0}{-1}{};
  \ftileA{0}{0}{0}{};
  \tileA{300}{1}{-1}{};
  \tileA{300}{2}{-1}{};
  \tileA{240}{2}{0}{};
\end{tikzpicture}%
\end{center}%
\end{minipage}%
}%
\end{center}
\caption{Partial patches (part 7)}
\label{fig:partpatch:7}
\end{figure}
\begin{figure}[htp!]
\renewcommand\thesubfigure{\arabic{subfigure}}%
\ContinuedFloat
\captionsetup{margin=0pt,justification=raggedright}%
\begin{center}
\subfloat[Partial patch 55 (extends to partial patch 86, $1$-patch 6)]{%
\begin{minipage}[b]{6cm}
\begin{center}
\begin{tikzpicture}[x=3mm,y=3mm]
  \ffkite{180}{1}{0};
  \tileA{300}{-1}{1}{};
  \tileA{180}{0}{-1}{};
  \ftileA{0}{0}{0}{};
  \tileA{300}{1}{-1}{};
  \tileA{120}{2}{-1}{};
  \tileA{180}{3}{-2}{};
\end{tikzpicture}%
\end{center}%
\end{minipage}%
} \qquad \subfloat[Partial patch 56 (extends to $1$-patches 7--8)]{%
\begin{minipage}[b]{6cm}
\begin{center}
\begin{tikzpicture}[x=3mm,y=3mm]
  \ffkite{240}{0}{0};
  \tileA{240}{0}{-1}{};
  \ftileA{0}{0}{0}{};
  \tileA{0}{0}{1}{};
  \tileAr{180}{1}{-1}{};
  \tileA{60}{2}{-2}{};
  \tileA{120}{2}{-1}{};
\end{tikzpicture}%
\end{center}%
\end{minipage}%
} \\ \subfloat[Partial patch 57 (extends to partial patches 87--88)]{%
\begin{minipage}[b]{6cm}
\begin{center}
\begin{tikzpicture}[x=3mm,y=3mm]
  \ffkite{240}{0}{0};
  \tileA{240}{0}{-1}{};
  \ftileA{0}{0}{0}{};
  \tileAr{180}{1}{-1}{};
  \tileA{240}{1}{1}{};
  \tileA{60}{2}{-2}{};
  \tileA{120}{2}{-1}{};
\end{tikzpicture}%
\end{center}%
\end{minipage}%
} \qquad \subfloat[Partial patch 58 (no extensions)]{%
\begin{minipage}[b]{6cm}
\begin{center}
\begin{tikzpicture}[x=3mm,y=3mm]
  \ffkite{60}{0}{1};
  \tileA{60}{-1}{0}{};
  \tileA{240}{0}{-1}{};
  \ftileA{0}{0}{0}{};
  \tileAr{180}{1}{-1}{};
  \tileAr{300}{1}{0}{};
  \tileAr{240}{2}{-1}{};
\end{tikzpicture}%
\end{center}%
\end{minipage}%
} \\ \subfloat[Partial patch 59 (extends to partial patch 89)]{%
\begin{minipage}[b]{6cm}
\begin{center}
\begin{tikzpicture}[x=3mm,y=3mm]
  \ffkite{240}{0}{0};
  \tileA{120}{-1}{0}{};
  \tileA{240}{0}{-1}{};
  \ftileA{0}{0}{0}{};
  \tileAr{180}{1}{-1}{};
  \tileAr{300}{1}{0}{};
  \tileAr{240}{2}{-1}{};
\end{tikzpicture}%
\end{center}%
\end{minipage}%
} \qquad \subfloat[Partial patch 60 (extends to partial patch 90)]{%
\begin{minipage}[b]{6cm}
\begin{center}
\begin{tikzpicture}[x=3mm,y=3mm]
  \ffkite{60}{0}{1};
  \tileA{300}{-1}{1}{};
  \tileA{240}{0}{-1}{};
  \ftileA{0}{0}{0}{};
  \tileAr{180}{1}{-1}{};
  \tileAr{300}{1}{0}{};
  \tileAr{240}{2}{-1}{};
\end{tikzpicture}%
\end{center}%
\end{minipage}%
}%
\end{center}
\caption{Partial patches (part 8)}
\label{fig:partpatch:8}
\end{figure}
\begin{figure}[htp!]
\renewcommand\thesubfigure{\arabic{subfigure}}%
\ContinuedFloat
\captionsetup{margin=0pt,justification=raggedright}%
\begin{center}
\subfloat[Partial patch 61 (extends to partial patch 91)]{%
\begin{minipage}[b]{6cm}
\begin{center}
\begin{tikzpicture}[x=3mm,y=3mm]
  \ffkite{180}{1}{0};
  \tileA{180}{-1}{0}{};
  \tileAr{300}{-1}{1}{};
  \tileAr{120}{0}{-1}{};
  \ftileA{0}{0}{0}{};
  \tileA{120}{0}{1}{};
  \tileAr{180}{1}{-1}{};
\end{tikzpicture}%
\end{center}%
\end{minipage}%
} \qquad \subfloat[Partial patch 62 (extends to partial patch 92)]{%
\begin{minipage}[b]{6cm}
\begin{center}
\begin{tikzpicture}[x=3mm,y=3mm]
  \ffkite{180}{1}{0};
  \tileA{180}{-1}{0}{};
  \tileAr{300}{-1}{1}{};
  \tileAr{120}{0}{-1}{};
  \ftileA{0}{0}{0}{};
  \tileAr{0}{0}{1}{};
  \tileAr{180}{1}{-1}{};
\end{tikzpicture}%
\end{center}%
\end{minipage}%
} \\ \subfloat[Partial patch 63 (no extensions)]{%
\begin{minipage}[b]{6cm}
\begin{center}
\begin{tikzpicture}[x=3mm,y=3mm]
  \ffkite{300}{2}{-1};
  \tileAr{60}{-1}{0}{};
  \tileAr{120}{0}{-1}{};
  \ftileA{0}{0}{0}{};
  \tileAr{180}{0}{1}{};
  \tileAr{180}{1}{-1}{};
  \tileA{60}{1}{0}{};
\end{tikzpicture}%
\end{center}%
\end{minipage}%
} \qquad \subfloat[Partial patch 64 (extends to $1$-patch 9)]{%
\begin{minipage}[b]{6cm}
\begin{center}
\begin{tikzpicture}[x=3mm,y=3mm]
  \ffkite{0}{2}{-1};
  \tileAr{60}{-1}{0}{};
  \tileAr{120}{0}{-1}{};
  \ftileA{0}{0}{0}{};
  \tileAr{180}{0}{1}{};
  \tileAr{180}{1}{-1}{};
  \tileAr{300}{1}{0}{};
\end{tikzpicture}%
\end{center}%
\end{minipage}%
} \\ \subfloat[Partial patch 65 (extends to $1$-patches 10--11)]{%
\begin{minipage}[b]{6cm}
\begin{center}
\begin{tikzpicture}[x=3mm,y=3mm]
  \ffkite{60}{1}{-1};
  \tileA{60}{-1}{-1}{};
  \tileA{120}{-1}{0}{};
  \tileAr{300}{-1}{1}{};
  \tileA{0}{0}{-1}{};
  \ftileA{0}{0}{0}{};
  \tileA{120}{0}{1}{};
  \tileA{60}{1}{0}{};
\end{tikzpicture}%
\end{center}%
\end{minipage}%
} \qquad \subfloat[Partial patch 66 (no extensions)]{%
\begin{minipage}[b]{6cm}
\begin{center}
\begin{tikzpicture}[x=3mm,y=3mm]
  \ffkite{60}{1}{-1};
  \tileA{60}{-1}{-1}{};
  \tileA{120}{-1}{0}{};
  \tileAr{300}{-1}{1}{};
  \tileA{0}{0}{-1}{};
  \ftileA{0}{0}{0}{};
  \tileAr{0}{0}{1}{};
  \tileAr{300}{1}{0}{};
\end{tikzpicture}%
\end{center}%
\end{minipage}%
} \\ \subfloat[Partial patch 67 (extends to $1$-patch 12)]{%
\begin{minipage}[b]{6cm}
\begin{center}
\begin{tikzpicture}[x=3mm,y=3mm]
  \ffkite{0}{0}{1};
  \tileA{60}{-1}{-1}{};
  \tileA{300}{-1}{1}{};
  \tileA{0}{0}{-1}{};
  \ftileA{0}{0}{0}{};
  \tileA{60}{0}{1}{};
  \tileA{60}{1}{0}{};
  \tileA{120}{2}{-2}{};
\end{tikzpicture}%
\end{center}%
\end{minipage}%
} \qquad \subfloat[Partial patch 68 (extends to $1$-patch 13)]{%
\begin{minipage}[b]{6cm}
\begin{center}
\begin{tikzpicture}[x=3mm,y=3mm]
  \ffkite{120}{0}{1};
  \tileA{60}{-1}{-1}{};
  \tileA{300}{-1}{1}{};
  \tileA{0}{0}{-1}{};
  \ftileA{0}{0}{0}{};
  \tileAr{120}{0}{1}{};
  \tileA{60}{1}{0}{};
  \tileA{120}{2}{-2}{};
\end{tikzpicture}%
\end{center}%
\end{minipage}%
}%
\end{center}
\caption{Partial patches (part 9)}
\label{fig:partpatch:9}
\end{figure}
\begin{figure}[htp!]
\renewcommand\thesubfigure{\arabic{subfigure}}%
\ContinuedFloat
\captionsetup{margin=0pt,justification=raggedright}%
\begin{center}
\subfloat[Partial patch 69 (extends to $1$-patch 14)]{%
\begin{minipage}[b]{6cm}
\begin{center}
\begin{tikzpicture}[x=3mm,y=3mm]
  \ffkite{0}{0}{1};
  \tileA{60}{-1}{-1}{};
  \tileA{300}{-1}{1}{};
  \tileA{0}{0}{-1}{};
  \ftileA{0}{0}{0}{};
  \tileA{240}{1}{1}{};
  \tileA{120}{2}{-2}{};
  \tileA{240}{2}{0}{};
\end{tikzpicture}%
\end{center}%
\end{minipage}%
} \qquad \subfloat[Partial patch 70 (extends to $1$-patch 15)]{%
\begin{minipage}[b]{6cm}
\begin{center}
\begin{tikzpicture}[x=3mm,y=3mm]
  \ffkite{0}{0}{1};
  \tileA{60}{-1}{-1}{};
  \tileA{300}{-1}{1}{};
  \tileA{0}{0}{-1}{};
  \ftileA{0}{0}{0}{};
  \tileA{60}{0}{1}{};
  \tileA{60}{1}{0}{};
  \tileA{240}{2}{-1}{};
\end{tikzpicture}%
\end{center}%
\end{minipage}%
} \\ \subfloat[Partial patch 71 (extends to $1$-patch 16)]{%
\begin{minipage}[b]{6cm}
\begin{center}
\begin{tikzpicture}[x=3mm,y=3mm]
  \ffkite{120}{0}{1};
  \tileA{60}{-1}{-1}{};
  \tileA{300}{-1}{1}{};
  \tileA{0}{0}{-1}{};
  \ftileA{0}{0}{0}{};
  \tileAr{120}{0}{1}{};
  \tileA{60}{1}{0}{};
  \tileA{240}{2}{-1}{};
\end{tikzpicture}%
\end{center}%
\end{minipage}%
} \qquad \subfloat[Partial patch 72 (extends to $1$-patches 17--18)]{%
\begin{minipage}[b]{6cm}
\begin{center}
\begin{tikzpicture}[x=3mm,y=3mm]
  \ffkite{60}{1}{-1};
  \tileA{0}{-2}{1}{};
  \tileA{300}{-1}{0}{};
  \tileAr{300}{-1}{1}{};
  \tileA{0}{0}{-1}{};
  \ftileA{0}{0}{0}{};
  \tileA{120}{0}{1}{};
  \tileA{60}{1}{0}{};
\end{tikzpicture}%
\end{center}%
\end{minipage}%
} \\ \subfloat[Partial patch 73 (no extensions)]{%
\begin{minipage}[b]{6cm}
\begin{center}
\begin{tikzpicture}[x=3mm,y=3mm]
  \ffkite{60}{1}{-1};
  \tileA{0}{-2}{1}{};
  \tileA{300}{-1}{0}{};
  \tileAr{300}{-1}{1}{};
  \tileA{0}{0}{-1}{};
  \ftileA{0}{0}{0}{};
  \tileAr{0}{0}{1}{};
  \tileAr{300}{1}{0}{};
\end{tikzpicture}%
\end{center}%
\end{minipage}%
} \qquad \subfloat[Partial patch 74 (extends to $1$-patches 19--20)]{%
\begin{minipage}[b]{6cm}
\begin{center}
\begin{tikzpicture}[x=3mm,y=3mm]
  \ffkite{60}{1}{-1};
  \tileA{120}{-1}{0}{};
  \tileAr{300}{-1}{1}{};
  \tileA{180}{0}{-1}{};
  \ftileA{0}{0}{0}{};
  \tileA{120}{0}{1}{};
  \tileA{120}{1}{-2}{};
  \tileA{60}{1}{0}{};
\end{tikzpicture}%
\end{center}%
\end{minipage}%
}%
\end{center}
\caption{Partial patches (part 10)}
\label{fig:partpatch:10}
\end{figure}
\begin{figure}[htp!]
\renewcommand\thesubfigure{\arabic{subfigure}}%
\ContinuedFloat
\captionsetup{margin=0pt,justification=raggedright}%
\begin{center}
\subfloat[Partial patch 75 (extends to $1$-patch 21)]{%
\begin{minipage}[b]{6cm}
\begin{center}
\begin{tikzpicture}[x=3mm,y=3mm]
  \ffkite{300}{2}{-1};
  \tileA{120}{-1}{0}{};
  \tileAr{300}{-1}{1}{};
  \tileA{180}{0}{-1}{};
  \ftileA{0}{0}{0}{};
  \tileA{120}{0}{1}{};
  \tileA{300}{1}{-1}{};
  \tileA{60}{1}{0}{};
\end{tikzpicture}%
\end{center}%
\end{minipage}%
} \qquad \subfloat[Partial patch 76 (extends to $1$-patch 22)]{%
\begin{minipage}[b]{6cm}
\begin{center}
\begin{tikzpicture}[x=3mm,y=3mm]
  \ffkite{0}{2}{-1};
  \tileA{120}{-1}{0}{};
  \tileAr{300}{-1}{1}{};
  \tileA{180}{0}{-1}{};
  \ftileA{0}{0}{0}{};
  \tileAr{0}{0}{1}{};
  \tileA{300}{1}{-1}{};
  \tileAr{300}{1}{0}{};
\end{tikzpicture}%
\end{center}%
\end{minipage}%
} \\ \subfloat[Partial patch 77 (extends to $1$-patch 23)]{%
\begin{minipage}[b]{6cm}
\begin{center}
\begin{tikzpicture}[x=3mm,y=3mm]
  \ffkite{0}{0}{1};
  \tileA{300}{-1}{1}{};
  \tileA{180}{0}{-1}{};
  \ftileA{0}{0}{0}{};
  \tileA{60}{0}{1}{};
  \tileA{120}{1}{-2}{};
  \tileA{60}{1}{0}{};
  \tileA{120}{2}{-2}{};
\end{tikzpicture}%
\end{center}%
\end{minipage}%
} \qquad \subfloat[Partial patch 78 (extends to $1$-patch 24)]{%
\begin{minipage}[b]{6cm}
\begin{center}
\begin{tikzpicture}[x=3mm,y=3mm]
  \ffkite{120}{0}{1};
  \tileA{300}{-1}{1}{};
  \tileA{180}{0}{-1}{};
  \ftileA{0}{0}{0}{};
  \tileAr{120}{0}{1}{};
  \tileA{120}{1}{-2}{};
  \tileA{60}{1}{0}{};
  \tileA{120}{2}{-2}{};
\end{tikzpicture}%
\end{center}%
\end{minipage}%
} \\ \subfloat[Partial patch 79 (extends to $1$-patch 25)]{%
\begin{minipage}[b]{6cm}
\begin{center}
\begin{tikzpicture}[x=3mm,y=3mm]
  \ffkite{0}{0}{1};
  \tileA{300}{-1}{1}{};
  \tileA{180}{0}{-1}{};
  \ftileA{0}{0}{0}{};
  \tileA{120}{1}{-2}{};
  \tileA{240}{1}{1}{};
  \tileA{120}{2}{-2}{};
  \tileA{240}{2}{0}{};
\end{tikzpicture}%
\end{center}%
\end{minipage}%
} \qquad \subfloat[Partial patch 80 (extends to $1$-patch 26)]{%
\begin{minipage}[b]{6cm}
\begin{center}
\begin{tikzpicture}[x=3mm,y=3mm]
  \ffkite{0}{0}{1};
  \tileA{300}{-1}{1}{};
  \tileA{180}{0}{-1}{};
  \ftileA{0}{0}{0}{};
  \tileA{60}{0}{1}{};
  \tileA{120}{1}{-2}{};
  \tileA{60}{1}{0}{};
  \tileA{240}{2}{-1}{};
\end{tikzpicture}%
\end{center}%
\end{minipage}%
}%
\end{center}
\caption{Partial patches (part 11)}
\label{fig:partpatch:11}
\end{figure}
\begin{figure}[htp!]
\renewcommand\thesubfigure{\arabic{subfigure}}%
\ContinuedFloat
\captionsetup{margin=0pt,justification=raggedright}%
\begin{center}
\subfloat[Partial patch 81 (extends to $1$-patch 27)]{%
\begin{minipage}[b]{6cm}
\begin{center}
\begin{tikzpicture}[x=3mm,y=3mm]
  \ffkite{120}{0}{1};
  \tileA{300}{-1}{1}{};
  \tileA{180}{0}{-1}{};
  \ftileA{0}{0}{0}{};
  \tileAr{120}{0}{1}{};
  \tileA{120}{1}{-2}{};
  \tileA{60}{1}{0}{};
  \tileA{240}{2}{-1}{};
\end{tikzpicture}%
\end{center}%
\end{minipage}%
} \qquad \subfloat[Partial patch 82 (extends to $1$-patch 28)]{%
\begin{minipage}[b]{6cm}
\begin{center}
\begin{tikzpicture}[x=3mm,y=3mm]
  \ffkite{120}{0}{1};
  \tileA{300}{-1}{1}{};
  \tileA{180}{0}{-1}{};
  \ftileA{0}{0}{0}{};
  \tileAr{120}{0}{1}{};
  \tileA{300}{1}{-1}{};
  \tileAr{300}{1}{0}{};
  \tileA{0}{2}{-1}{};
\end{tikzpicture}%
\end{center}%
\end{minipage}%
} \\ \subfloat[Partial patch 83 (extends to $1$-patch 29)]{%
\begin{minipage}[b]{6cm}
\begin{center}
\begin{tikzpicture}[x=3mm,y=3mm]
  \ffkite{0}{0}{1};
  \tileA{300}{-1}{1}{};
  \tileA{180}{0}{-1}{};
  \ftileA{0}{0}{0}{};
  \tileA{60}{0}{1}{};
  \tileA{300}{1}{-1}{};
  \tileA{60}{1}{0}{};
  \tileA{300}{2}{-1}{};
\end{tikzpicture}%
\end{center}%
\end{minipage}%
} \qquad \subfloat[Partial patch 84 (extends to $1$-patch 30)]{%
\begin{minipage}[b]{6cm}
\begin{center}
\begin{tikzpicture}[x=3mm,y=3mm]
  \ffkite{120}{0}{1};
  \tileA{300}{-1}{1}{};
  \tileA{180}{0}{-1}{};
  \ftileA{0}{0}{0}{};
  \tileAr{120}{0}{1}{};
  \tileA{300}{1}{-1}{};
  \tileA{60}{1}{0}{};
  \tileA{300}{2}{-1}{};
\end{tikzpicture}%
\end{center}%
\end{minipage}%
} \\ \subfloat[Partial patch 85 (extends to $1$-patch 31)]{%
\begin{minipage}[b]{6cm}
\begin{center}
\begin{tikzpicture}[x=3mm,y=3mm]
  \ffkite{0}{0}{1};
  \tileA{300}{-1}{1}{};
  \tileA{180}{0}{-1}{};
  \ftileA{0}{0}{0}{};
  \tileA{300}{1}{-1}{};
  \tileA{240}{1}{1}{};
  \tileA{300}{2}{-1}{};
  \tileA{240}{2}{0}{};
\end{tikzpicture}%
\end{center}%
\end{minipage}%
} \qquad \subfloat[Partial patch 86 (extends to $1$-patch 32)]{%
\begin{minipage}[b]{6cm}
\begin{center}
\begin{tikzpicture}[x=3mm,y=3mm]
  \ffkite{0}{0}{1};
  \tileA{300}{-1}{1}{};
  \tileA{180}{0}{-1}{};
  \ftileA{0}{0}{0}{};
  \tileA{300}{1}{-1}{};
  \tileA{240}{1}{1}{};
  \tileA{120}{2}{-1}{};
  \tileA{180}{3}{-2}{};
\end{tikzpicture}%
\end{center}%
\end{minipage}%
}%
\end{center}
\caption{Partial patches (part 12)}
\label{fig:partpatch:12}
\end{figure}
\begin{figure}[htp!]
\renewcommand\thesubfigure{\arabic{subfigure}}%
\ContinuedFloat
\captionsetup{margin=0pt,justification=raggedright}%
\begin{center}
\subfloat[Partial patch 87 (extends to $1$-patch 33)]{%
\begin{minipage}[b]{6cm}
\begin{center}
\begin{tikzpicture}[x=3mm,y=3mm]
  \ffkite{0}{0}{1};
  \tileA{60}{-1}{0}{};
  \tileA{240}{0}{-1}{};
  \ftileA{0}{0}{0}{};
  \tileAr{180}{1}{-1}{};
  \tileA{240}{1}{1}{};
  \tileA{60}{2}{-2}{};
  \tileA{120}{2}{-1}{};
\end{tikzpicture}%
\end{center}%
\end{minipage}%
} \qquad \subfloat[Partial patch 88 (extends to $1$-patch 34)]{%
\begin{minipage}[b]{6cm}
\begin{center}
\begin{tikzpicture}[x=3mm,y=3mm]
  \ffkite{0}{0}{1};
  \tileA{300}{-1}{1}{};
  \tileA{240}{0}{-1}{};
  \ftileA{0}{0}{0}{};
  \tileAr{180}{1}{-1}{};
  \tileA{240}{1}{1}{};
  \tileA{60}{2}{-2}{};
  \tileA{120}{2}{-1}{};
\end{tikzpicture}%
\end{center}%
\end{minipage}%
} \\ \subfloat[Partial patch 89 (extends to $1$-patch 35)]{%
\begin{minipage}[b]{6cm}
\begin{center}
\begin{tikzpicture}[x=3mm,y=3mm]
  \ffkite{120}{0}{1};
  \tileA{120}{-1}{0}{};
  \tileAr{300}{-1}{1}{};
  \tileA{240}{0}{-1}{};
  \ftileA{0}{0}{0}{};
  \tileAr{180}{1}{-1}{};
  \tileAr{300}{1}{0}{};
  \tileAr{240}{2}{-1}{};
\end{tikzpicture}%
\end{center}%
\end{minipage}%
} \qquad \subfloat[Partial patch 90 (extends to $1$-patch 36)]{%
\begin{minipage}[b]{6cm}
\begin{center}
\begin{tikzpicture}[x=3mm,y=3mm]
  \ffkite{120}{0}{1};
  \tileA{300}{-1}{1}{};
  \tileA{240}{0}{-1}{};
  \ftileA{0}{0}{0}{};
  \tileAr{120}{0}{1}{};
  \tileAr{180}{1}{-1}{};
  \tileAr{300}{1}{0}{};
  \tileAr{240}{2}{-1}{};
\end{tikzpicture}%
\end{center}%
\end{minipage}%
} \\ \subfloat[Partial patch 91 (no extensions)]{%
\begin{minipage}[b]{6cm}
\begin{center}
\begin{tikzpicture}[x=3mm,y=3mm]
  \ffkite{300}{2}{-1};
  \tileA{180}{-1}{0}{};
  \tileAr{300}{-1}{1}{};
  \tileAr{120}{0}{-1}{};
  \ftileA{0}{0}{0}{};
  \tileA{120}{0}{1}{};
  \tileAr{180}{1}{-1}{};
  \tileA{60}{1}{0}{};
\end{tikzpicture}%
\end{center}%
\end{minipage}%
} \qquad \subfloat[Partial patch 92 (extends to $1$-patch 37)]{%
\begin{minipage}[b]{6cm}
\begin{center}
\begin{tikzpicture}[x=3mm,y=3mm]
  \ffkite{0}{2}{-1};
  \tileA{180}{-1}{0}{};
  \tileAr{300}{-1}{1}{};
  \tileAr{120}{0}{-1}{};
  \ftileA{0}{0}{0}{};
  \tileAr{0}{0}{1}{};
  \tileAr{180}{1}{-1}{};
  \tileAr{300}{1}{0}{};
\end{tikzpicture}%
\end{center}%
\end{minipage}%
}%
\end{center}
\caption{Partial patches (part 13)}
\label{fig:partpatch:13}
\end{figure}
\begin{figure}[htp!]
\renewcommand\thesubfigure{\arabic{subfigure}}%
\captionsetup{margin=0pt,justification=raggedright}%
\begin{center}
\subfloat[$1$-patch 1 (central tile class $F_2$)]{%
\begin{minipage}[b]{6cm}
\begin{center}
\begin{tikzpicture}[x=3mm,y=3mm]
  \tileA{60}{-1}{-1}{};
  \tileA{300}{-1}{1}{};
  \tileA{0}{0}{-1}{};
  \tileA{0}{0}{0}{};
  \tileA{0}{0}{1}{};
  \tileA{120}{2}{-2}{};
  \tileA{240}{2}{0}{};
\end{tikzpicture}%
\end{center}%
\end{minipage}%
} \qquad \subfloat[$1$-patch 2 (central tile class $H_3$)]{%
\begin{minipage}[b]{6cm}
\begin{center}
\begin{tikzpicture}[x=3mm,y=3mm]
  \tileA{300}{-1}{0}{};
  \tileA{240}{-1}{1}{};
  \tileA{0}{0}{-1}{};
  \tileA{0}{0}{0}{};
  \tileAr{180}{0}{1}{};
  \tileA{60}{1}{0}{};
  \tileA{120}{2}{-2}{};
\end{tikzpicture}%
\end{center}%
\end{minipage}%
} \\ \subfloat[$1$-patch 3 (central tile class $H_3$)]{%
\begin{minipage}[b]{6cm}
\begin{center}
\begin{tikzpicture}[x=3mm,y=3mm]
  \tileA{300}{-1}{0}{};
  \tileA{240}{-1}{1}{};
  \tileA{0}{0}{-1}{};
  \tileA{0}{0}{0}{};
  \tileAr{180}{0}{1}{};
  \tileA{60}{1}{0}{};
  \tileA{240}{2}{-1}{};
\end{tikzpicture}%
\end{center}%
\end{minipage}%
} \qquad \subfloat[$1$-patch 4 (central tile class $F_2$)]{%
\begin{minipage}[b]{6cm}
\begin{center}
\begin{tikzpicture}[x=3mm,y=3mm]
  \tileA{300}{-1}{1}{};
  \tileA{180}{0}{-1}{};
  \tileA{0}{0}{0}{};
  \tileA{0}{0}{1}{};
  \tileA{120}{1}{-2}{};
  \tileA{120}{2}{-2}{};
  \tileA{240}{2}{0}{};
\end{tikzpicture}%
\end{center}%
\end{minipage}%
} \\ \subfloat[$1$-patch 5 (central tile class $P_2$)]{%
\begin{minipage}[b]{6cm}
\begin{center}
\begin{tikzpicture}[x=3mm,y=3mm]
  \tileA{300}{-1}{1}{};
  \tileA{180}{0}{-1}{};
  \tileA{0}{0}{0}{};
  \tileA{0}{0}{1}{};
  \tileA{300}{1}{-1}{};
  \tileA{300}{2}{-1}{};
  \tileA{240}{2}{0}{};
\end{tikzpicture}%
\end{center}%
\end{minipage}%
} \qquad \subfloat[$1$-patch 6 (eliminated by trying to surround shaded tile)]{%
\begin{minipage}[b]{6cm}
\begin{center}
\begin{tikzpicture}[x=3mm,y=3mm]
  \tileA{300}{-1}{1}{};
  \tileA{180}{0}{-1}{};
  \tileA{0}{0}{0}{};
  \tileA{0}{0}{1}{};
  \tileA{300}{1}{-1}{};
  \fftileA{120}{2}{-1}{};
  \tileA{180}{3}{-2}{};
\end{tikzpicture}%
\end{center}%
\end{minipage}%
}%
\end{center}
\caption{$1$-patches (part 1)}
\label{fig:patch}
\end{figure}
\begin{figure}[htp!]
\renewcommand\thesubfigure{\arabic{subfigure}}%
\ContinuedFloat
\captionsetup{margin=0pt,justification=raggedright}%
\begin{center}
\subfloat[$1$-patch 7 (central tile class $H_2$)]{%
\begin{minipage}[b]{6cm}
\begin{center}
\begin{tikzpicture}[x=3mm,y=3mm]
  \tileA{60}{-1}{0}{};
  \tileA{240}{0}{-1}{};
  \tileA{0}{0}{0}{};
  \tileA{0}{0}{1}{};
  \tileAr{180}{1}{-1}{};
  \tileA{60}{2}{-2}{};
  \tileA{120}{2}{-1}{};
\end{tikzpicture}%
\end{center}%
\end{minipage}%
} \qquad \subfloat[$1$-patch 8 (central tile class $H_2$)]{%
\begin{minipage}[b]{6cm}
\begin{center}
\begin{tikzpicture}[x=3mm,y=3mm]
  \tileA{300}{-1}{1}{};
  \tileA{240}{0}{-1}{};
  \tileA{0}{0}{0}{};
  \tileA{0}{0}{1}{};
  \tileAr{180}{1}{-1}{};
  \tileA{60}{2}{-2}{};
  \tileA{120}{2}{-1}{};
\end{tikzpicture}%
\end{center}%
\end{minipage}%
} \\ \subfloat[$1$-patch 9 (central tile class $H_1$)]{%
\begin{minipage}[b]{6cm}
\begin{center}
\begin{tikzpicture}[x=3mm,y=3mm]
  \tileAr{60}{-1}{0}{};
  \tileAr{120}{0}{-1}{};
  \tileA{0}{0}{0}{};
  \tileAr{180}{0}{1}{};
  \tileAr{180}{1}{-1}{};
  \tileAr{300}{1}{0}{};
  \tileAr{240}{2}{-1}{};
\end{tikzpicture}%
\end{center}%
\end{minipage}%
} \qquad \subfloat[$1$-patch 10 (central tile class $H_4$)]{%
\begin{minipage}[b]{6cm}
\begin{center}
\begin{tikzpicture}[x=3mm,y=3mm]
  \tileA{60}{-1}{-1}{};
  \tileA{120}{-1}{0}{};
  \tileAr{300}{-1}{1}{};
  \tileA{0}{0}{-1}{};
  \tileA{0}{0}{0}{};
  \tileA{120}{0}{1}{};
  \tileA{60}{1}{0}{};
  \tileA{120}{2}{-2}{};
\end{tikzpicture}%
\end{center}%
\end{minipage}%
} \\ \subfloat[$1$-patch 11 (eliminated by trying to surround shaded tile)]{%
\begin{minipage}[b]{6cm}
\begin{center}
\begin{tikzpicture}[x=3mm,y=3mm]
  \tileA{60}{-1}{-1}{};
  \tileA{120}{-1}{0}{};
  \tileAr{300}{-1}{1}{};
  \fftileA{0}{0}{-1}{};
  \tileA{0}{0}{0}{};
  \tileA{120}{0}{1}{};
  \tileA{60}{1}{0}{};
  \tileA{240}{2}{-1}{};
\end{tikzpicture}%
\end{center}%
\end{minipage}%
} \qquad \subfloat[$1$-patch 12 (central tile class $FP_1$)]{%
\begin{minipage}[b]{6cm}
\begin{center}
\begin{tikzpicture}[x=3mm,y=3mm]
  \tileA{60}{-1}{-1}{};
  \tileA{300}{-1}{1}{};
  \tileAr{240}{-1}{2}{};
  \tileA{0}{0}{-1}{};
  \tileA{0}{0}{0}{};
  \tileA{60}{0}{1}{};
  \tileA{60}{1}{0}{};
  \tileA{120}{2}{-2}{};
\end{tikzpicture}%
\end{center}%
\end{minipage}%
}%
\end{center}
\caption{$1$-patches (part 2)}
\label{fig:patch:2}
\end{figure}
\begin{figure}[htp!]
\renewcommand\thesubfigure{\arabic{subfigure}}%
\ContinuedFloat
\captionsetup{margin=0pt,justification=raggedright}%
\begin{center}
\subfloat[$1$-patch 13 (central tile class $FP_1$)]{%
\begin{minipage}[b]{6cm}
\begin{center}
\begin{tikzpicture}[x=3mm,y=3mm]
  \tileA{60}{-1}{-1}{};
  \tileA{300}{-1}{1}{};
  \tileA{0}{0}{-1}{};
  \tileA{0}{0}{0}{};
  \tileAr{120}{0}{1}{};
  \tileA{300}{0}{2}{};
  \tileA{60}{1}{0}{};
  \tileA{120}{2}{-2}{};
\end{tikzpicture}%
\end{center}%
\end{minipage}%
} \qquad \subfloat[$1$-patch 14 (central tile class $F_2$)]{%
\begin{minipage}[b]{6cm}
\begin{center}
\begin{tikzpicture}[x=3mm,y=3mm]
  \tileA{60}{-1}{-1}{};
  \tileA{300}{-1}{1}{};
  \tileA{0}{0}{-1}{};
  \tileA{0}{0}{0}{};
  \tileAr{60}{0}{1}{};
  \tileA{240}{1}{1}{};
  \tileA{120}{2}{-2}{};
  \tileA{240}{2}{0}{};
\end{tikzpicture}%
\end{center}%
\end{minipage}%
} \\ \subfloat[$1$-patch 15 (eliminated by trying to surround shaded tile)]{%
\begin{minipage}[b]{6cm}
\begin{center}
\begin{tikzpicture}[x=3mm,y=3mm]
  \tileA{60}{-1}{-1}{};
  \tileA{300}{-1}{1}{};
  \tileAr{240}{-1}{2}{};
  \fftileA{0}{0}{-1}{};
  \tileA{0}{0}{0}{};
  \tileA{60}{0}{1}{};
  \tileA{60}{1}{0}{};
  \tileA{240}{2}{-1}{};
\end{tikzpicture}%
\end{center}%
\end{minipage}%
} \qquad \subfloat[$1$-patch 16 (eliminated by trying to surround shaded tile)]{%
\begin{minipage}[b]{6cm}
\begin{center}
\begin{tikzpicture}[x=3mm,y=3mm]
  \tileA{60}{-1}{-1}{};
  \tileA{300}{-1}{1}{};
  \fftileA{0}{0}{-1}{};
  \tileA{0}{0}{0}{};
  \tileAr{120}{0}{1}{};
  \tileA{300}{0}{2}{};
  \tileA{60}{1}{0}{};
  \tileA{240}{2}{-1}{};
\end{tikzpicture}%
\end{center}%
\end{minipage}%
} \\ \subfloat[$1$-patch 17 (eliminated by trying to surround shaded tile)]{%
\begin{minipage}[b]{6cm}
\begin{center}
\begin{tikzpicture}[x=3mm,y=3mm]
  \tileA{0}{-2}{1}{};
  \fftileA{300}{-1}{0}{};
  \tileAr{300}{-1}{1}{};
  \tileA{0}{0}{-1}{};
  \tileA{0}{0}{0}{};
  \tileA{120}{0}{1}{};
  \tileA{60}{1}{0}{};
  \tileA{120}{2}{-2}{};
\end{tikzpicture}%
\end{center}%
\end{minipage}%
} \qquad \subfloat[$1$-patch 18 (eliminated by trying to surround shaded tile)]{%
\begin{minipage}[b]{6cm}
\begin{center}
\begin{tikzpicture}[x=3mm,y=3mm]
  \tileA{0}{-2}{1}{};
  \fftileA{300}{-1}{0}{};
  \tileAr{300}{-1}{1}{};
  \tileA{0}{0}{-1}{};
  \tileA{0}{0}{0}{};
  \tileA{120}{0}{1}{};
  \tileA{60}{1}{0}{};
  \tileA{240}{2}{-1}{};
\end{tikzpicture}%
\end{center}%
\end{minipage}%
}%
\end{center}
\caption{$1$-patches (part 3)}
\label{fig:patch:3}
\end{figure}
\begin{figure}[htp!]
\renewcommand\thesubfigure{\arabic{subfigure}}%
\ContinuedFloat
\captionsetup{margin=0pt,justification=raggedright}%
\begin{center}
\subfloat[$1$-patch 19 (central tile class $H_4$)]{%
\begin{minipage}[b]{6cm}
\begin{center}
\begin{tikzpicture}[x=3mm,y=3mm]
  \tileA{120}{-1}{0}{};
  \tileAr{300}{-1}{1}{};
  \tileA{180}{0}{-1}{};
  \tileA{0}{0}{0}{};
  \tileA{120}{0}{1}{};
  \tileA{120}{1}{-2}{};
  \tileA{60}{1}{0}{};
  \tileA{120}{2}{-2}{};
\end{tikzpicture}%
\end{center}%
\end{minipage}%
} \qquad \subfloat[$1$-patch 20 (central tile class $H_4$)]{%
\begin{minipage}[b]{6cm}
\begin{center}
\begin{tikzpicture}[x=3mm,y=3mm]
  \tileA{120}{-1}{0}{};
  \tileAr{300}{-1}{1}{};
  \tileA{180}{0}{-1}{};
  \tileA{0}{0}{0}{};
  \tileA{120}{0}{1}{};
  \tileA{120}{1}{-2}{};
  \tileA{60}{1}{0}{};
  \tileA{240}{2}{-1}{};
\end{tikzpicture}%
\end{center}%
\end{minipage}%
} \\ \subfloat[$1$-patch 21 (eliminated by trying to surround shaded tile)]{%
\begin{minipage}[b]{6cm}
\begin{center}
\begin{tikzpicture}[x=3mm,y=3mm]
  \tileA{120}{-1}{0}{};
  \tileAr{300}{-1}{1}{};
  \tileA{180}{0}{-1}{};
  \tileA{0}{0}{0}{};
  \tileA{120}{0}{1}{};
  \tileA{300}{1}{-1}{};
  \fftileA{60}{1}{0}{};
  \tileA{300}{2}{-1}{};
\end{tikzpicture}%
\end{center}%
\end{minipage}%
} \qquad \subfloat[$1$-patch 22 (eliminated by trying to surround shaded tile)]{%
\begin{minipage}[b]{6cm}
\begin{center}
\begin{tikzpicture}[x=3mm,y=3mm]
  \tileA{120}{-1}{0}{};
  \fftileAr{300}{-1}{1}{};
  \tileA{180}{0}{-1}{};
  \tileA{0}{0}{0}{};
  \tileAr{0}{0}{1}{};
  \tileA{300}{1}{-1}{};
  \tileAr{300}{1}{0}{};
  \tileA{0}{2}{-1}{};
\end{tikzpicture}%
\end{center}%
\end{minipage}%
} \\ \subfloat[$1$-patch 23 (central tile class $FP_1$)]{%
\begin{minipage}[b]{6cm}
\begin{center}
\begin{tikzpicture}[x=3mm,y=3mm]
  \tileA{300}{-1}{1}{};
  \tileAr{240}{-1}{2}{};
  \tileA{180}{0}{-1}{};
  \tileA{0}{0}{0}{};
  \tileA{60}{0}{1}{};
  \tileA{120}{1}{-2}{};
  \tileA{60}{1}{0}{};
  \tileA{120}{2}{-2}{};
\end{tikzpicture}%
\end{center}%
\end{minipage}%
} \qquad \subfloat[$1$-patch 24 (central tile class $FP_1$)]{%
\begin{minipage}[b]{6cm}
\begin{center}
\begin{tikzpicture}[x=3mm,y=3mm]
  \tileA{300}{-1}{1}{};
  \tileA{180}{0}{-1}{};
  \tileA{0}{0}{0}{};
  \tileAr{120}{0}{1}{};
  \tileA{300}{0}{2}{};
  \tileA{120}{1}{-2}{};
  \tileA{60}{1}{0}{};
  \tileA{120}{2}{-2}{};
\end{tikzpicture}%
\end{center}%
\end{minipage}%
}%
\end{center}
\caption{$1$-patches (part 4)}
\label{fig:patch:4}
\end{figure}
\begin{figure}[htp!]
\renewcommand\thesubfigure{\arabic{subfigure}}%
\ContinuedFloat
\captionsetup{margin=0pt,justification=raggedright}%
\begin{center}
\subfloat[$1$-patch 25 (central tile class $F_2$)]{%
\begin{minipage}[b]{6cm}
\begin{center}
\begin{tikzpicture}[x=3mm,y=3mm]
  \tileA{300}{-1}{1}{};
  \tileA{180}{0}{-1}{};
  \tileA{0}{0}{0}{};
  \tileAr{60}{0}{1}{};
  \tileA{120}{1}{-2}{};
  \tileA{240}{1}{1}{};
  \tileA{120}{2}{-2}{};
  \tileA{240}{2}{0}{};
\end{tikzpicture}%
\end{center}%
\end{minipage}%
} \qquad \subfloat[$1$-patch 26 (central tile class $FP_1$)]{%
\begin{minipage}[b]{6cm}
\begin{center}
\begin{tikzpicture}[x=3mm,y=3mm]
  \tileA{300}{-1}{1}{};
  \tileAr{240}{-1}{2}{};
  \tileA{180}{0}{-1}{};
  \tileA{0}{0}{0}{};
  \tileA{60}{0}{1}{};
  \tileA{120}{1}{-2}{};
  \tileA{60}{1}{0}{};
  \tileA{240}{2}{-1}{};
\end{tikzpicture}%
\end{center}%
\end{minipage}%
} \\ \subfloat[$1$-patch 27 (central tile class $FP_1$)]{%
\begin{minipage}[b]{6cm}
\begin{center}
\begin{tikzpicture}[x=3mm,y=3mm]
  \tileA{300}{-1}{1}{};
  \tileA{180}{0}{-1}{};
  \tileA{0}{0}{0}{};
  \tileAr{120}{0}{1}{};
  \tileA{300}{0}{2}{};
  \tileA{120}{1}{-2}{};
  \tileA{60}{1}{0}{};
  \tileA{240}{2}{-1}{};
\end{tikzpicture}%
\end{center}%
\end{minipage}%
} \qquad \subfloat[$1$-patch 28 (eliminated by trying to surround shaded tile)]{%
\begin{minipage}[b]{6cm}
\begin{center}
\begin{tikzpicture}[x=3mm,y=3mm]
  \tileA{300}{-1}{1}{};
  \tileA{180}{0}{-1}{};
  \tileA{0}{0}{0}{};
  \tileAr{120}{0}{1}{};
  \fftileA{300}{1}{-1}{};
  \tileAr{300}{1}{0}{};
  \tileA{180}{1}{1}{};
  \tileA{0}{2}{-1}{};
\end{tikzpicture}%
\end{center}%
\end{minipage}%
} \\ \subfloat[$1$-patch 29 (central tile class $T_1$)]{%
\begin{minipage}[b]{6cm}
\begin{center}
\begin{tikzpicture}[x=3mm,y=3mm]
  \tileA{300}{-1}{1}{};
  \tileAr{240}{-1}{2}{};
  \tileA{180}{0}{-1}{};
  \tileA{0}{0}{0}{};
  \tileA{60}{0}{1}{};
  \tileA{300}{1}{-1}{};
  \tileA{60}{1}{0}{};
  \tileA{300}{2}{-1}{};
\end{tikzpicture}%
\end{center}%
\end{minipage}%
} \qquad \subfloat[$1$-patch 30 (central tile class $T_1$)]{%
\begin{minipage}[b]{6cm}
\begin{center}
\begin{tikzpicture}[x=3mm,y=3mm]
  \tileA{300}{-1}{1}{};
  \tileA{180}{0}{-1}{};
  \tileA{0}{0}{0}{};
  \tileAr{120}{0}{1}{};
  \tileA{300}{0}{2}{};
  \tileA{300}{1}{-1}{};
  \tileA{60}{1}{0}{};
  \tileA{300}{2}{-1}{};
\end{tikzpicture}%
\end{center}%
\end{minipage}%
}%
\end{center}
\caption{$1$-patches (part 5)}
\label{fig:patch:5}
\end{figure}
\begin{figure}[htp!]
\renewcommand\thesubfigure{\arabic{subfigure}}%
\ContinuedFloat
\captionsetup{margin=0pt,justification=raggedright}%
\begin{center}
\subfloat[$1$-patch 31 (central tile class $P_2$)]{%
\begin{minipage}[b]{6cm}
\begin{center}
\begin{tikzpicture}[x=3mm,y=3mm]
  \tileA{300}{-1}{1}{};
  \tileA{180}{0}{-1}{};
  \tileA{0}{0}{0}{};
  \tileAr{60}{0}{1}{};
  \tileA{300}{1}{-1}{};
  \tileA{240}{1}{1}{};
  \tileA{300}{2}{-1}{};
  \tileA{240}{2}{0}{};
\end{tikzpicture}%
\end{center}%
\end{minipage}%
} \qquad \subfloat[$1$-patch 32 (central tile class $P_2$)]{%
\begin{minipage}[b]{6cm}
\begin{center}
\begin{tikzpicture}[x=3mm,y=3mm]
  \tileA{300}{-1}{1}{};
  \tileA{180}{0}{-1}{};
  \tileA{0}{0}{0}{};
  \tileAr{60}{0}{1}{};
  \tileA{300}{1}{-1}{};
  \tileA{240}{1}{1}{};
  \tileA{120}{2}{-1}{};
  \tileA{180}{3}{-2}{};
\end{tikzpicture}%
\end{center}%
\end{minipage}%
} \\ \subfloat[$1$-patch 33 (central tile class $H_2$)]{%
\begin{minipage}[b]{6cm}
\begin{center}
\begin{tikzpicture}[x=3mm,y=3mm]
  \tileA{60}{-1}{0}{};
  \tileA{240}{0}{-1}{};
  \tileA{0}{0}{0}{};
  \tileA{180}{0}{1}{};
  \tileAr{180}{1}{-1}{};
  \tileA{240}{1}{1}{};
  \tileA{60}{2}{-2}{};
  \tileA{120}{2}{-1}{};
\end{tikzpicture}%
\end{center}%
\end{minipage}%
} \qquad \subfloat[$1$-patch 34 (central tile class $H_2$)]{%
\begin{minipage}[b]{6cm}
\begin{center}
\begin{tikzpicture}[x=3mm,y=3mm]
  \tileA{300}{-1}{1}{};
  \tileA{240}{0}{-1}{};
  \tileA{0}{0}{0}{};
  \tileAr{60}{0}{1}{};
  \tileAr{180}{1}{-1}{};
  \tileA{240}{1}{1}{};
  \tileA{60}{2}{-2}{};
  \tileA{120}{2}{-1}{};
\end{tikzpicture}%
\end{center}%
\end{minipage}%
} \\ \subfloat[$1$-patch 35 (eliminated by trying to surround shaded tile)]{%
\begin{minipage}[b]{6cm}
\begin{center}
\begin{tikzpicture}[x=3mm,y=3mm]
  \tileA{120}{-1}{0}{};
  \fftileAr{300}{-1}{1}{};
  \tileA{240}{0}{-1}{};
  \tileA{0}{0}{0}{};
  \tileAr{0}{0}{1}{};
  \tileAr{180}{1}{-1}{};
  \tileAr{300}{1}{0}{};
  \tileAr{240}{2}{-1}{};
\end{tikzpicture}%
\end{center}%
\end{minipage}%
} \qquad \subfloat[$1$-patch 36 (eliminated by trying to surround shaded tile)]{%
\begin{minipage}[b]{6cm}
\begin{center}
\begin{tikzpicture}[x=3mm,y=3mm]
  \fftileA{300}{-1}{1}{};
  \tileA{240}{0}{-1}{};
  \tileA{0}{0}{0}{};
  \tileAr{120}{0}{1}{};
  \tileAr{180}{1}{-1}{};
  \tileAr{300}{1}{0}{};
  \tileA{180}{1}{1}{};
  \tileAr{240}{2}{-1}{};
\end{tikzpicture}%
\end{center}%
\end{minipage}%
}%
\end{center}
\caption{$1$-patches (part 6)}
\label{fig:patch:6}
\end{figure}
\begin{figure}[htp!]
\renewcommand\thesubfigure{\arabic{subfigure}}%
\ContinuedFloat
\captionsetup{margin=0pt,justification=raggedright}%
\begin{center}
\subfloat[$1$-patch 37 (eliminated by trying to surround shaded tile)]{%
\begin{minipage}[b]{6cm}
\begin{center}
\begin{tikzpicture}[x=3mm,y=3mm]
  \tileA{180}{-1}{0}{};
  \fftileAr{300}{-1}{1}{};
  \tileAr{120}{0}{-1}{};
  \tileA{0}{0}{0}{};
  \tileAr{0}{0}{1}{};
  \tileAr{180}{1}{-1}{};
  \tileAr{300}{1}{0}{};
  \tileAr{240}{2}{-1}{};
\end{tikzpicture}%
\end{center}%
\end{minipage}%
}%
\end{center}
\caption{$1$-patches (part 7)}
\label{fig:patch:7}
\end{figure}


\FloatBarrier

\subsection{Classification of outer tiles}

For each of the possible neighbours that actually occurs in some of
the remaining $1$-patches, we can now list the possible
classifications of a central tile that has such a neighbour; see
Table~\ref{table:nbrclass}.

For each of the outer tiles in a $1$-patch, we have some but not all
of its neighbours, and can take the intersection of the sets from
Table~\ref{table:nbrclass} to produce a set of possible classes for
that outer tile.  Although this is not a single class, it can still be
used for the within-cluster and between-cluster checks.  In each case,
it turns out that the set of possible classes for a neighbour
appearing in one of those checks is a subset of the classes permitted
by that check, and so we have a complete proof of the within-cluster
and between-cluster matching properties that depends only on the
enumeration of $1$-patches presented here and not on a larger
enumeration of $2$-patches; the lists of checks and corresponding sets
of classes appear below.

% Automatically generated table and list.
\begin{table}[htp!]
\begin{center}
\begin{tabular}{|c|c|}
\textbf{Possible neighbour} & \textbf{Possible classes for central tile}\\
$2$ & $\{FP_1, F_2, H_4\}$\\
$3$ & $\{H_2\}$\\
$4$ & $\{H_4\}$\\
$6$ & $\{H_3\}$\\
$7$ & $\{H_1\}$\\
$8$ & $\{H_3\}$\\
$9$ & $\{FP_1, F_2, H_2, P_2, T_1\}$\\
$11$ & $\{H_4\}$\\
$12$ & $\{FP_1, T_1\}$\\
$13$ & $\{FP_1, F_2, H_3, H_4\}$\\
$14$ & $\{FP_1, F_2, H_4, P_2, T_1\}$\\
$15$ & $\{H_2\}$\\
$16$ & $\{H_1\}$\\
$17$ & $\{F_2, H_2, P_2\}$\\
$18$ & $\{FP_1, T_1\}$\\
$19$ & $\{H_4\}$\\
$20$ & $\{H_2\}$\\
$22$ & $\{F_2, H_2, P_2\}$\\
$23$ & $\{FP_1, T_1\}$\\
$24$ & $\{H_1, H_3\}$\\
$25$ & $\{FP_1, T_1\}$\\
$26$ & $\{FP_1, F_2, H_4\}$\\
$27$ & $\{P_2, T_1\}$\\
$28$ & $\{H_1, H_2\}$\\
$29$ & $\{FP_1, H_3, H_4, T_1\}$\\
$30$ & $\{H_1\}$\\
$32$ & $\{F_2, H_2, P_2\}$\\
$33$ & $\{H_2\}$\\
$34$ & $\{FP_1, F_2, H_3, H_4\}$\\
$36$ & $\{H_2, P_2\}$\\
$37$ & $\{FP_1, H_3, H_4\}$\\
$38$ & $\{P_2, T_1\}$\\
$39$ & $\{H_1\}$\\
$40$ & $\{F_2, P_2\}$\\
$41$ & $\{P_2\}$\\
\end{tabular}
\caption{}
\label{table:nbrclass}
\end{center}
\end{table}
\begin{itemize}
\item $1$-patch 1 (class $F_2$)
\begin{itemize}
\item $P_2$ or $F_2$ neighbour $FP_1$ OK: $\{FP_1\} \subseteq \{FP_1\}$
\item $F$ edge $F^+$ OK: $\{F_2\} \subseteq \{F_2\}$
\item $F$ edge $F^-$ OK: $\{F_2\} \subseteq \{F_2\}$
\item $X^+$ edge at top of polykite OK: $\{H_3\} \subseteq \{F_2, FP_1, H_3, H_4\}$
\item $X^-$ edge at bottom of polykite OK: $\{H_2\} \subseteq \{F_2, H_2, P_2\}$
\item $L$ edge at bottom of polykite OK: $\{P_2\} \subseteq \{P_2\}$
\end{itemize}
\item $1$-patch 2 (class $H_3$)
\begin{itemize}
\item $H_3$ neighbour $H_1$ OK: $\{H_1\} \subseteq \{H_1\}$
\item $H$ lower edge $B^-$ OK: $\{FP_1, T_1\} \subseteq \{FP_1, T_1\}$
\item $X^+$ edge at right of polykite OK: $\{P_2\} \subseteq \{H_2, P_2\}$
\item $X^-$ edge at bottom of polykite OK: $\{H_2, P_2\} \subseteq \{F_2, H_2, P_2\}$
\end{itemize}
\item $1$-patch 3 (class $H_3$)
\begin{itemize}
\item $H_3$ neighbour $H_1$ OK: $\{H_1\} \subseteq \{H_1\}$
\item $H$ lower edge $B^-$ OK: $\{FP_1, T_1\} \subseteq \{FP_1, T_1\}$
\item $X^+$ edge at right of polykite OK: $\{F_2, FP_1, H_4\} \subseteq \{F_2, FP_1, H_3, H_4\}$
\item $X^-$ edge at bottom of polykite OK: $\{F_2, P_2\} \subseteq \{F_2, H_2, P_2\}$
\end{itemize}
\item $1$-patch 4 (class $F_2$)
\begin{itemize}
\item $P_2$ or $F_2$ neighbour $FP_1$ OK: $\{FP_1\} \subseteq \{FP_1\}$
\item $F$ edge $F^+$ OK: $\{F_2\} \subseteq \{F_2\}$
\item $F$ edge $F^-$ OK: $\{F_2\} \subseteq \{F_2\}$
\item $X^+$ edge at top of polykite OK: $\{H_3\} \subseteq \{F_2, FP_1, H_3, H_4\}$
\item $X^-$ edge at bottom of polykite OK: $\{FP_1, H_3, H_4\} \subseteq \{FP_1, H_3, H_4\}$
\item $L$ edge at bottom of polykite OK: $\{F_2, FP_1\} \subseteq \{F_2, FP_1\}$
\end{itemize}
\item $1$-patch 5 (class $P_2$)
\begin{itemize}
\item $P_2$ or $F_2$ neighbour $FP_1$ OK: $\{FP_1\} \subseteq \{FP_1\}$
\item $T$ or $P$ lower edge $A^-$ OK: $\{H_2\} \subseteq \{H_2\}$
\item $X^+$ edge at top of polykite OK: $\{H_3\} \subseteq \{F_2, FP_1, H_3, H_4\}$
\item $X^-$ edge at right of polykite OK: $\{FP_1, H_4\} \subseteq \{FP_1, H_3, H_4\}$
\item $L$ edge at right of polykite OK: $\{F_2, FP_1\} \subseteq \{F_2, FP_1\}$
\end{itemize}
\item $1$-patch 7 (class $H_2$)
\begin{itemize}
\item $H_2$ neighbour $H_1$ OK: $\{H_1\} \subseteq \{H_1\}$
\item $H$ edge $A^+$ OK: $\{P_2, T_1\} \subseteq \{P_2, T_1\}$
\item $X^+$ edge at top of polykite OK: $\{F_2, FP_1, H_4\} \subseteq \{F_2, FP_1, H_3, H_4\}$
\item $X^-$ edge at right of polykite OK: $\{F_2, P_2\} \subseteq \{F_2, H_2, P_2\}$
\end{itemize}
\item $1$-patch 8 (class $H_2$)
\begin{itemize}
\item $H_2$ neighbour $H_1$ OK: $\{H_1\} \subseteq \{H_1\}$
\item $H$ edge $A^+$ OK: $\{T_1\} \subseteq \{T_1\}$
\item $X^+$ edge at top of polykite OK: $\{H_3\} \subseteq \{F_2, FP_1, H_3, H_4\}$
\item $X^-$ edge at right of polykite OK: $\{F_2, P_2\} \subseteq \{F_2, H_2, P_2\}$
\end{itemize}
\item $1$-patch 9 (class $H_1$)
\begin{itemize}
\item $H_1$ neighbour $H_2$ OK: $\{H_2\} \subseteq \{H_2\}$
\item $H_1$ neighbour $H_3$ OK: $\{H_3\} \subseteq \{H_3\}$
\item $H_1$ neighbour $H_4$ OK: $\{H_4\} \subseteq \{H_4\}$
\item $H$ upper edge $B^-$ OK: $\{FP_1, T_1\} \subseteq \{FP_1, T_1\}$
\end{itemize}
\item $1$-patch 10 (class $H_4$)
\begin{itemize}
\item $H_4$ neighbour $H_1$ OK: $\{H_1\} \subseteq \{H_1\}$
\item $X^+$ edge at right of polykite OK: $\{P_2\} \subseteq \{H_2, P_2\}$
\item $X^-$ edge at bottom of polykite OK: $\{H_2\} \subseteq \{F_2, H_2, P_2\}$
\end{itemize}
\item $1$-patch 12 (class $FP_1$)
\begin{itemize}
\item $FP_1$ neighbour $P_2$ or $F_2$ OK: $\{F_2\} \subseteq \{F_2, P_2\}$
\item $T$, $P$ or $F$ edge $B^+$ OK: $\{H_4\} \subseteq \{H_3, H_4\}$
\item $X^+$ edge at right of polykite OK: $\{P_2\} \subseteq \{H_2, P_2\}$
\item $X^-$ edge at bottom of polykite OK: $\{H_2\} \subseteq \{F_2, H_2, P_2\}$
\item $L$ edge at bottom of polykite OK: $\{P_2\} \subseteq \{P_2\}$
\end{itemize}
\item $1$-patch 13 (class $FP_1$)
\begin{itemize}
\item $FP_1$ neighbour $P_2$ or $F_2$ OK: $\{F_2\} \subseteq \{F_2, P_2\}$
\item $T$, $P$ or $F$ edge $B^+$ OK: $\{H_3\} \subseteq \{H_3, H_4\}$
\item $X^+$ edge at right of polykite OK: $\{P_2\} \subseteq \{H_2, P_2\}$
\item $X^-$ edge at bottom of polykite OK: $\{H_2\} \subseteq \{F_2, H_2, P_2\}$
\item $L$ edge at bottom of polykite OK: $\{P_2\} \subseteq \{P_2\}$
\end{itemize}
\item $1$-patch 14 (class $F_2$)
\begin{itemize}
\item $P_2$ or $F_2$ neighbour $FP_1$ OK: $\{FP_1\} \subseteq \{FP_1\}$
\item $F$ edge $F^+$ OK: $\{F_2\} \subseteq \{F_2\}$
\item $F$ edge $F^-$ OK: $\{F_2\} \subseteq \{F_2\}$
\item $X^+$ edge at top of polykite OK: $\{H_2\} \subseteq \{H_2, P_2\}$
\item $X^-$ edge at bottom of polykite OK: $\{H_2\} \subseteq \{F_2, H_2, P_2\}$
\item $L$ edge at bottom of polykite OK: $\{P_2\} \subseteq \{P_2\}$
\end{itemize}
\item $1$-patch 19 (class $H_4$)
\begin{itemize}
\item $H_4$ neighbour $H_1$ OK: $\{H_1\} \subseteq \{H_1\}$
\item $X^+$ edge at right of polykite OK: $\{P_2\} \subseteq \{H_2, P_2\}$
\item $X^-$ edge at bottom of polykite OK: $\{FP_1, H_3, H_4\} \subseteq \{FP_1, H_3, H_4\}$
\end{itemize}
\item $1$-patch 20 (class $H_4$)
\begin{itemize}
\item $H_4$ neighbour $H_1$ OK: $\{H_1\} \subseteq \{H_1\}$
\item $X^+$ edge at right of polykite OK: $\{FP_1, H_4\} \subseteq \{F_2, FP_1, H_3, H_4\}$
\item $X^-$ edge at bottom of polykite OK: $\{FP_1, H_4\} \subseteq \{FP_1, H_3, H_4\}$
\end{itemize}
\item $1$-patch 23 (class $FP_1$)
\begin{itemize}
\item $FP_1$ neighbour $P_2$ or $F_2$ OK: $\{F_2\} \subseteq \{F_2, P_2\}$
\item $T$, $P$ or $F$ edge $B^+$ OK: $\{H_4\} \subseteq \{H_3, H_4\}$
\item $X^+$ edge at right of polykite OK: $\{P_2\} \subseteq \{H_2, P_2\}$
\item $X^-$ edge at bottom of polykite OK: $\{FP_1, H_3, H_4\} \subseteq \{FP_1, H_3, H_4\}$
\item $L$ edge at bottom of polykite OK: $\{F_2, FP_1\} \subseteq \{F_2, FP_1\}$
\end{itemize}
\item $1$-patch 24 (class $FP_1$)
\begin{itemize}
\item $FP_1$ neighbour $P_2$ or $F_2$ OK: $\{F_2\} \subseteq \{F_2, P_2\}$
\item $T$, $P$ or $F$ edge $B^+$ OK: $\{H_3\} \subseteq \{H_3, H_4\}$
\item $X^+$ edge at right of polykite OK: $\{P_2\} \subseteq \{H_2, P_2\}$
\item $X^-$ edge at bottom of polykite OK: $\{FP_1, H_3, H_4\} \subseteq \{FP_1, H_3, H_4\}$
\item $L$ edge at bottom of polykite OK: $\{F_2, FP_1\} \subseteq \{F_2, FP_1\}$
\end{itemize}
\item $1$-patch 25 (class $F_2$)
\begin{itemize}
\item $P_2$ or $F_2$ neighbour $FP_1$ OK: $\{FP_1\} \subseteq \{FP_1\}$
\item $F$ edge $F^+$ OK: $\{F_2\} \subseteq \{F_2\}$
\item $F$ edge $F^-$ OK: $\{F_2\} \subseteq \{F_2\}$
\item $X^+$ edge at top of polykite OK: $\{H_2\} \subseteq \{H_2, P_2\}$
\item $X^-$ edge at bottom of polykite OK: $\{FP_1, H_3, H_4\} \subseteq \{FP_1, H_3, H_4\}$
\item $L$ edge at bottom of polykite OK: $\{F_2, FP_1\} \subseteq \{F_2, FP_1\}$
\end{itemize}
\item $1$-patch 26 (class $FP_1$)
\begin{itemize}
\item $FP_1$ neighbour $P_2$ or $F_2$ OK: $\{F_2, P_2\} \subseteq \{F_2, P_2\}$
\item $T$, $P$ or $F$ edge $B^+$ OK: $\{H_4\} \subseteq \{H_3, H_4\}$
\item $X^+$ edge at right of polykite OK: $\{FP_1, H_4\} \subseteq \{F_2, FP_1, H_3, H_4\}$
\item $X^-$ edge at bottom of polykite OK: $\{FP_1, H_4\} \subseteq \{FP_1, H_3, H_4\}$
\item $L$ edge at bottom of polykite OK: $\{F_2, FP_1\} \subseteq \{F_2, FP_1\}$
\end{itemize}
\item $1$-patch 27 (class $FP_1$)
\begin{itemize}
\item $FP_1$ neighbour $P_2$ or $F_2$ OK: $\{F_2, P_2\} \subseteq \{F_2, P_2\}$
\item $T$, $P$ or $F$ edge $B^+$ OK: $\{H_3\} \subseteq \{H_3, H_4\}$
\item $X^+$ edge at right of polykite OK: $\{FP_1, H_4\} \subseteq \{F_2, FP_1, H_3, H_4\}$
\item $X^-$ edge at bottom of polykite OK: $\{FP_1, H_4\} \subseteq \{FP_1, H_3, H_4\}$
\item $L$ edge at bottom of polykite OK: $\{F_2, FP_1\} \subseteq \{F_2, FP_1\}$
\end{itemize}
\item $1$-patch 29 (class $T_1$)
\begin{itemize}
\item $T$ upper edge $A^-$ OK: $\{H_2\} \subseteq \{H_2\}$
\item $T$ or $P$ lower edge $A^-$ OK: $\{H_2\} \subseteq \{H_2\}$
\item $T$, $P$ or $F$ edge $B^+$ OK: $\{H_4\} \subseteq \{H_3, H_4\}$
\end{itemize}
\item $1$-patch 30 (class $T_1$)
\begin{itemize}
\item $T$ upper edge $A^-$ OK: $\{H_2\} \subseteq \{H_2\}$
\item $T$ or $P$ lower edge $A^-$ OK: $\{H_2\} \subseteq \{H_2\}$
\item $T$, $P$ or $F$ edge $B^+$ OK: $\{H_3\} \subseteq \{H_3, H_4\}$
\end{itemize}
\item $1$-patch 31 (class $P_2$)
\begin{itemize}
\item $P_2$ or $F_2$ neighbour $FP_1$ OK: $\{FP_1\} \subseteq \{FP_1\}$
\item $T$ or $P$ lower edge $A^-$ OK: $\{H_2\} \subseteq \{H_2\}$
\item $X^+$ edge at top of polykite OK: $\{H_2\} \subseteq \{H_2, P_2\}$
\item $X^-$ edge at right of polykite OK: $\{FP_1, H_3, H_4\} \subseteq \{FP_1, H_3, H_4\}$
\item $L$ edge at right of polykite OK: $\{F_2, FP_1\} \subseteq \{F_2, FP_1\}$
\end{itemize}
\item $1$-patch 32 (class $P_2$)
\begin{itemize}
\item $P_2$ or $F_2$ neighbour $FP_1$ OK: $\{FP_1\} \subseteq \{FP_1\}$
\item $T$ or $P$ lower edge $A^-$ OK: $\{H_2\} \subseteq \{H_2\}$
\item $X^+$ edge at top of polykite OK: $\{H_2\} \subseteq \{H_2, P_2\}$
\item $X^-$ edge at right of polykite OK: $\{H_2\} \subseteq \{F_2, H_2, P_2\}$
\item $L$ edge at right of polykite OK: $\{P_2\} \subseteq \{P_2\}$
\end{itemize}
\item $1$-patch 33 (class $H_2$)
\begin{itemize}
\item $H_2$ neighbour $H_1$ OK: $\{H_1\} \subseteq \{H_1\}$
\item $H$ edge $A^+$ OK: $\{P_2\} \subseteq \{P_2, T_1\}$
\item $X^+$ edge at top of polykite OK: $\{P_2\} \subseteq \{H_2, P_2\}$
\item $X^-$ edge at right of polykite OK: $\{H_2, P_2\} \subseteq \{F_2, H_2, P_2\}$
\end{itemize}
\item $1$-patch 34 (class $H_2$)
\begin{itemize}
\item $H_2$ neighbour $H_1$ OK: $\{H_1\} \subseteq \{H_1\}$
\item $H$ edge $A^+$ OK: $\{T_1\} \subseteq \{T_1\}$
\item $X^+$ edge at top of polykite OK: $\{H_2\} \subseteq \{H_2, P_2\}$
\item $X^-$ edge at right of polykite OK: $\{H_2, P_2\} \subseteq \{F_2, H_2, P_2\}$
\end{itemize}
\end{itemize}


\FloatBarrier
