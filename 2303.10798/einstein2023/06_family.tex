\section{A family of aperiodic monotiles}
\label{sec:family}

In the previous sections, we showed that the hat polykite
is an aperiodic monotile.  This polykite is
formed of $8$~kites from the $[3.4.6.4]$ Laves tiling.  Another small
polykite, formed of $10$~kites and shown in Figure~\ref{fig:tileb}, is
also aperiodic.  We have verified via a computer search that there are no 
other aperiodic $n$-kites for $n \le 21$.

\begin{figure}[htp!]
\begin{center}
\begin{tikzpicture}[x=5mm,y=5mm]
  \tileB{0}{0}{0}{};
\end{tikzpicture}
\end{center}
\caption{An aperiodic $10$-kite}
\label{fig:tileb}
\end{figure}

These two aperiodic polykites are two examples of a family of
aperiodic monotiles, all of which have combinatorially equivalent sets
of tilings, and which are determined by the choice of two side
lengths.

The hat polykite has sides of lengths $1$, $2$, and~$\sqrt{3}$; for the
purposes of this section, we consider the side of length~$2$ as two
consecutive sides of length~$1$ with a $180^\circ$~angle between them.
The tile of Figure~\ref{fig:tileb} has the same angles, but with the
side lengths of $1$ and~$\sqrt{3}$ swapped.

Let $a$ and~$b$ be nonnegative reals, not both zero, and if $a\ne 0$
write $r = b / a$.
Define $\mathrm{Tile}(a, b)$ to be the polygon resulting from replacing the sides of
length~$1$ in the hat polykite with sides of length~$a$ (we
refer to the resulting sides as \emph{$1$-sides}) and replacing the
sides of length~$\sqrt{3}$ in the hat polykite with sides of
length~$b$ (we refer to the resulting sides as \emph{$r$-sides}).
This process results in a closed curve (because the vectors of the
$1$-sides add up to~$0$, as do those of the $r$-sides), and it is also
straightforward to see that this curve is not self-intersecting; it is
a $13$-gon (or one with a smaller number of sides if $a$ or~$b$ is
zero), but considered as a $14$-gon for the purposes of this section.
This tile has area~$\sqrt{3}(2a^2+\sqrt{3}ab+b^2)$.  The hat polykite
is then $\mathrm{Tile}(1, \sqrt{3})$ in this notation, while the tile of
Figure~\ref{fig:tileb} is $\mathrm{Tile}(\sqrt{3}, 1)$.

For nonzero~$a$, the value 
of~$r$ determines the tile up to similarity.  In acknowledgment of 
these similarity classes, we write $\mathrm{Tile}(r)$ as a
shorthand for $\mathrm{Tile}(1,r)$.  We will show this tile is
aperiodic for any positive $r \ne 1$.
In fact, $\mathrm{Tile}(1, k\sqrt{3})$ and $\mathrm{Tile}(k\sqrt{3},
1)$ are polykites for all odd positive integers~$k$, so this continuum
of aperiodic monotiles also contains a countably infinite family of
aperiodic polykites.

\begin{figure}[htp!]
\begin{center}
\begin{tikzpicture}[x=5mm,y=5mm]
  \tiler{1}{0}{0}{0}{0};
  \tiler{1}{-1}{2}{3}{0};
  \tiler{1}{-2}{4}{6}{0};
  \tiler{1}{-3}{6}{9}{0};
  \tilerr{1}{3}{-2}{1}{-2};
  \tilerr{1}{2}{0}{4}{-2};
  \tilerr{1}{1}{2}{7}{-2};
  \tilerr{1}{0}{4}{10}{-2};
  \tiler{1}{4}{-2}{6}{-6};
  \tiler{1}{3}{0}{9}{-6};
  \tiler{1}{2}{2}{12}{-6};
  \tiler{1}{1}{4}{15}{-6};
  \tilerr{1}{7}{-4}{7}{-8};
  \tilerr{1}{6}{-2}{10}{-8};
  \tilerr{1}{5}{0}{13}{-8};
  \tilerr{1}{4}{2}{16}{-8};
\end{tikzpicture}
\end{center}
\caption{Periodic tiling by $\mathrm{Tile}(1, 1)$}
\label{fig:tile1periodic}
\end{figure}

We define a notion of combinatorial equivalence between tilings of
these tiles, for two positive values of~$r$, as follows: two tilings
are combinatorially equivalent if there exists a bijection between
their tiles, and a bijection between the maximal line segments in the
unions of the boundaries of the tiles, such that corresponding tiles
and line segments in the two tilings are in the same orientation,
corresponding tiles adjoin corresponding line segments, on the same
side of those line segments, in the two tilings, and corresponding
tiles on the corresponding sides of corresponding line segments appear
in the same order along those segments.  All the interior angles of
the tile are at least~$90^\circ$, and no two $90^\circ$~angles appear
consecutively, so any maximal line segment has at most two sides of
tiles on each side of the line segment (and in particular is finite).

We now prove the following result:

\begin{theorem}
\label{thm:tilercomb}
Suppose $r \ne 1$ and $r'\ne 1$ are positive.  Then there is a
bijection between combinatorially equivalent tilings for $\mathrm{Tile}(r)$ and
$\mathrm{Tile}(r')$, given by changing the lengths of all $r$-sides from $r$
to~$r'$, while preserving angles, orientations and adjacency to
maximal line segments.
\end{theorem}

Suppose first that $r$ is irrational. 
If a maximal line segment in the union of the
boundaries of the tiles has $p$~$1$-sides and $q$~$r$-sides on one
side of the line segment, it also has $p$~$1$-sides and $q$~$r$-sides
on the other side of the line segment. Because a maximal line segment
has at most two sides of tiles on each side of the segment, the same
argument also applies for any rational~$r$ except possibly $\frac{1}{2}$, $1$,
and~$2$.

If $r = 2$, there is the additional possibility that two $1$-sides
align with one $r$-side.  When there are two consecutive $1$-sides on
one side of a line, with $90^\circ$~corners of the two tiles between
those two sides (or the $180^\circ$~corner of a single tile), the
other ends of those sides have corners with angles $120^\circ$
or~$240^\circ$.  But for every $r$-side, one corner has angle
$90^\circ$ or~$270^\circ$, and the angles of the tile do not permit
$120^\circ$ or $240^\circ$ at the same vertex of a tiling as
$90^\circ$ or~$270^\circ$.

Similarly, in the case $r = \frac{1}{2}$, the only additional
possibility is that two $r$-sides align with one $1$-side.  The outer
corners of the two $r$-sides have angles $120^\circ$ or~$240^\circ$,
one corner of every $1$-side has angle $90^\circ$, $180^\circ$
or~$270^\circ$, and those cannot appear at the same vertex.

Thus for any positive $r\ne 1$, we have shown that if a maximal line segment 
in the union of the boundaries of the tiles has $p$~$1$-sides and
$q$~$r$-sides on one side of the line segment, it also has
$p$~$1$-sides and $q$~$r$-sides on the other side of the line segment.
We can now construct the required bijection.  Because side vectors
around any tile add up to zero, and
the sides of tiles on both sides of a maximal line segment add up to
the same length, the specified process converts a tiling by $\mathrm{Tile}(r)$ into one
by $\mathrm{Tile}(r')$ that is combinatorially equivalent~\cite[Lemma~1.1]{regprod}.

(This argument relies on the fact that the plane is simply connected;
a tiling by $\mathrm{Tile}(r)$ of a region with a hole that cannot be filled
with tiles might not convert to a tiling by $\mathrm{Tile}(r')$ of a region with
a combinatorially equivalent hole, and indeed the side vectors of the
hole might be such that no combinatorially equivalent hole exists, if
the vectors of the $1$-sides among the sides of the hole do not add up
to~$0$.)

As shown in Lemma~\ref{lemma:tileaalign}, all tilings by
$\mathrm{Tile}(\sqrt{3})$ are aligned to an underlying $[3.4.6.4]$ Laves tiling,
so in fact each maximal line segment is made up only of $1$-sides or
only of $r$-sides.

Finally, $\mathrm{Tile}(1)$, or more generally $\mathrm{Tile}(a, a)$, is not aperiodic, as shown by the periodic tiling in
Figure~\ref{fig:tile1periodic}.  The polyiamonds $\mathrm{Tile}(a, 0)$ and
$\mathrm{Tile}(0, b)$ are also not aperiodic.  A tiling by $\mathrm{Tile}(r)$ for
positive $r\ne 1$ can still be mapped to a corresponding tiling by
$\mathrm{Tile}(a, a)$, $\mathrm{Tile}(a, 0)$ or $\mathrm{Tile}(0, b)$ following the process
described above, but the map is not a bijection.
