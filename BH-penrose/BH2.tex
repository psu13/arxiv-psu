\documentclass[12pt,border=3pt,tikz]{article}
\usepackage{tikz}
\usepackage{amsmath} % for \text
\usepackage{mathrsfs} % for \mathscr -> scri
\usepackage{xfp} % for fpeval -> floating point expression


\usepackage[noBBpl,sc]{mathpazo}
\usepackage[papersize={6.8in, 10.0in}, left=.5in, right=.5in, top=.6in, bottom=.9in]{geometry}
\linespread{1.1}
\sloppy
\raggedbottom
\pagestyle{plain}

% these include amsmath and that can cause trouble in older docs.
\makeatletter
\@ifpackageloaded{amsmath}{}{\RequirePackage{amsmath}}

\DeclareFontFamily{U}  {cmex}{}
\DeclareSymbolFont{Csymbols}       {U}  {cmex}{m}{n}
\DeclareFontShape{U}{cmex}{m}{n}{
    <-6>  cmex5
   <6-7>  cmex6
   <7-8>  cmex6
   <8-9>  cmex7
   <9-10> cmex8
  <10-12> cmex9
  <12->   cmex10}{}

\def\Set@Mn@Sym#1{\@tempcnta #1\relax}
\def\Next@Mn@Sym{\advance\@tempcnta 1\relax}
\def\Prev@Mn@Sym{\advance\@tempcnta-1\relax}
\def\@Decl@Mn@Sym#1#2#3#4{\DeclareMathSymbol{#2}{#3}{#4}{#1}}
\def\Decl@Mn@Sym#1#2#3{%
  \if\relax\noexpand#1%
    \let#1\undefined
  \fi
  \expandafter\@Decl@Mn@Sym\expandafter{\the\@tempcnta}{#1}{#3}{#2}%
  \Next@Mn@Sym}
\def\Decl@Mn@Alias#1#2#3{\Prev@Mn@Sym\Decl@Mn@Sym{#1}{#2}{#3}}
\let\Decl@Mn@Char\Decl@Mn@Sym
\def\Decl@Mn@Op#1#2#3{\def#1{\DOTSB#3\slimits@}}
\def\Decl@Mn@Int#1#2#3{\def#1{\DOTSI#3\ilimits@}}

\let\sum\undefined
\DeclareMathSymbol{\tsum}{\mathop}{Csymbols}{"50}
\DeclareMathSymbol{\dsum}{\mathop}{Csymbols}{"51}

\Decl@Mn@Op\sum\dsum\tsum

\makeatother

\makeatletter
\@ifpackageloaded{amsmath}{}{\RequirePackage{amsmath}}

\DeclareFontFamily{OMX}{MnSymbolE}{}
\DeclareSymbolFont{largesymbolsX}{OMX}{MnSymbolE}{m}{n}
\DeclareFontShape{OMX}{MnSymbolE}{m}{n}{
    <-6>  MnSymbolE5
   <6-7>  MnSymbolE6
   <7-8>  MnSymbolE7
   <8-9>  MnSymbolE8
   <9-10> MnSymbolE9
  <10-12> MnSymbolE10
  <12->   MnSymbolE12}{}

\DeclareMathSymbol{\downbrace}    {\mathord}{largesymbolsX}{'251}
\DeclareMathSymbol{\downbraceg}   {\mathord}{largesymbolsX}{'252}
\DeclareMathSymbol{\downbracegg}  {\mathord}{largesymbolsX}{'253}
\DeclareMathSymbol{\downbraceggg} {\mathord}{largesymbolsX}{'254}
\DeclareMathSymbol{\downbracegggg}{\mathord}{largesymbolsX}{'255}
\DeclareMathSymbol{\upbrace}      {\mathord}{largesymbolsX}{'256}
\DeclareMathSymbol{\upbraceg}     {\mathord}{largesymbolsX}{'257}
\DeclareMathSymbol{\upbracegg}    {\mathord}{largesymbolsX}{'260}
\DeclareMathSymbol{\upbraceggg}   {\mathord}{largesymbolsX}{'261}
\DeclareMathSymbol{\upbracegggg}  {\mathord}{largesymbolsX}{'262}
\DeclareMathSymbol{\braceld}      {\mathord}{largesymbolsX}{'263}
\DeclareMathSymbol{\bracelu}      {\mathord}{largesymbolsX}{'264}
\DeclareMathSymbol{\bracerd}      {\mathord}{largesymbolsX}{'265}
\DeclareMathSymbol{\braceru}      {\mathord}{largesymbolsX}{'266}
\DeclareMathSymbol{\bracemd}      {\mathord}{largesymbolsX}{'267}
\DeclareMathSymbol{\bracemu}      {\mathord}{largesymbolsX}{'270}
\DeclareMathSymbol{\bracemid}     {\mathord}{largesymbolsX}{'271}

\def\horiz@expandable#1#2#3#4#5#6#7#8{%
  \@mathmeasure\z@#7{#8}%
  \@tempdima=\wd\z@
  \@mathmeasure\z@#7{#1}%
  \ifdim\noexpand\wd\z@>\@tempdima
    $\m@th#7#1$%
  \else
    \@mathmeasure\z@#7{#2}%
    \ifdim\noexpand\wd\z@>\@tempdima
      $\m@th#7#2$%
    \else
      \@mathmeasure\z@#7{#3}%
      \ifdim\noexpand\wd\z@>\@tempdima
        $\m@th#7#3$%
      \else
        \@mathmeasure\z@#7{#4}%
        \ifdim\noexpand\wd\z@>\@tempdima
          $\m@th#7#4$%
        \else
          \@mathmeasure\z@#7{#5}%
          \ifdim\noexpand\wd\z@>\@tempdima
            $\m@th#7#5$%
          \else
           #6#7%
          \fi
        \fi
      \fi
    \fi
  \fi}

\def\overbrace@expandable#1#2#3{\vbox{\m@th\ialign{##\crcr
  #1#2{#3}\crcr\noalign{\kern2\p@\nointerlineskip}%
  $\m@th\hfil#2#3\hfil$\crcr}}}
\def\underbrace@expandable#1#2#3{\vtop{\m@th\ialign{##\crcr
  $\m@th\hfil#2#3\hfil$\crcr
  \noalign{\kern2\p@\nointerlineskip}%
  #1#2{#3}\crcr}}}

\def\overbrace@#1#2#3{\vbox{\m@th\ialign{##\crcr
  #1#2\crcr\noalign{\kern2\p@\nointerlineskip}%
  $\m@th\hfil#2#3\hfil$\crcr}}}
\def\underbrace@#1#2#3{\vtop{\m@th\ialign{##\crcr
  $\m@th\hfil#2#3\hfil$\crcr
  \noalign{\kern2\p@\nointerlineskip}%
  #1#2\crcr}}}

\def\bracefill@#1#2#3#4#5{$\m@th#5#1\leaders\hbox{$#4$}\hfill#2\leaders\hbox{$#4$}\hfill#3$}

\def\downbracefill@{\bracefill@\braceld\bracemd\bracerd\bracemid}
\def\upbracefill@{\bracefill@\bracelu\bracemu\braceru\bracemid}

\DeclareRobustCommand{\downbracefill}{\downbracefill@\textstyle}
\DeclareRobustCommand{\upbracefill}{\upbracefill@\textstyle}

\def\upbrace@expandable{%
  \horiz@expandable
    \upbrace
    \upbraceg
    \upbracegg
    \upbraceggg
    \upbracegggg
    \upbracefill@}
\def\downbrace@expandable{%
  \horiz@expandable
    \downbrace
    \downbraceg
    \downbracegg
    \downbraceggg
    \downbracegggg
    \downbracefill@}

\DeclareRobustCommand{\overbrace}[1]{\mathop{\mathpalette{\overbrace@expandable\downbrace@expandable}{#1}}\limits}
\DeclareRobustCommand{\underbrace}[1]{\mathop{\mathpalette{\underbrace@expandable\upbrace@expandable}{#1}}\limits}

\makeatother


\usepackage[small]{titlesec}
\titlelabel{\thetitle.\quad}

\usepackage{cite}
\usepackage{microtype}

% hyperref last because otherwise some things go wrong.
\usepackage{hyperref}
\hypersetup{colorlinks=true
,breaklinks=true
,urlcolor=blue
,anchorcolor=blue
,citecolor=blue
,filecolor=blue
,linkcolor=blue
,menucolor=blue
,linktocpage=true}
\hypersetup{
bookmarksopen=true,
bookmarksnumbered=true,
bookmarksopenlevel=10
}


% make sure there is enough TOC for reasonable pdf bookmarks.
\setcounter{tocdepth}{3}

%\usepackage[dotinlabels]{titletoc}
%\titlelabel{{\thetitle}.\quad}
%\usepackage{titletoc}
\usepackage[small]{titlesec}

\titleformat{\section}[block]
  {\fillast\medskip}
  {\bfseries{\thesection. }}
  {1ex minus .1ex}
  {\bfseries}
 
\titleformat*{\subsection}{\itshape}
\titleformat*{\subsubsection}{\itshape}

\setcounter{tocdepth}{2}

\titlecontents{section}
              [2.3em] 
              {\bigskip}
              {{\contentslabel{2.3em}}}
              {\hspace*{-2.3em}}
              {\titlerule*[1pc]{}\contentspage}
              
\titlecontents{subsection}
              [4.7em] 
              {}
              {{\contentslabel{2.3em}}}
              {\hspace*{-2.3em}}
              {\titlerule*[.5pc]{}\contentspage}

% hopefully not used.           
\titlecontents{subsubsection}
              [7.9em]
              {}
              {{\contentslabel{3.3em}}}
              {\hspace*{-3.3em}}
              {\titlerule*[.5pc]{}\contentspage}
%\makeatletter
\renewcommand\tableofcontents{%
    \section*{\contentsname
        \@mkboth{%
           \MakeLowercase\contentsname}{\MakeLowercase\contentsname}}%
    \@starttoc{toc}%
    }
\def\@oddhead{{\scshape\rightmark}\hfil{\small\scshape\thepage}}%
\def\sectionmark#1{%
      \markright{\MakeLowercase{%
        \ifnum \c@secnumdepth >\m@ne
          \thesection\quad
        \fi
        #1}}}
        
\makeatother

%\makeatletter

 \def\small{%
  \@setfontsize\small\@xipt{13pt}%
  \abovedisplayskip 8\p@ \@plus3\p@ \@minus6\p@
  \belowdisplayskip \abovedisplayskip
  \abovedisplayshortskip \z@ \@plus3\p@
  \belowdisplayshortskip 6.5\p@ \@plus3.5\p@ \@minus3\p@
  \def\@listi{%
    \leftmargin\leftmargini
    \topsep 9\p@ \@plus3\p@ \@minus5\p@
    \parsep 4.5\p@ \@plus2\p@ \@minus\p@
    \itemsep \parsep
  }%
}%
 \def\footnotesize{%
  \@setfontsize\footnotesize\@xpt{12pt}%
  \abovedisplayskip 10\p@ \@plus2\p@ \@minus5\p@
  \belowdisplayskip \abovedisplayskip
  \abovedisplayshortskip \z@ \@plus3\p@
  \belowdisplayshortskip 6\p@ \@plus3\p@ \@minus3\p@
  \def\@listi{%
    \leftmargin\leftmargini
    \topsep 6\p@ \@plus2\p@ \@minus2\p@
    \parsep 3\p@ \@plus2\p@ \@minus\p@
    \itemsep \parsep
  }%
}%
\def\open@column@one#1{%
 \ltxgrid@info@sw{\class@info{\string\open@column@one\string#1}}{}%
 \unvbox\pagesofar
 \@ifvoid{\footsofar}{}{%
  \insert\footins\bgroup\unvbox\footsofar\egroup
  \penalty\z@
 }%
 \gdef\thepagegrid{one}%
 \global\pagegrid@col#1%
 \global\pagegrid@cur\@ne
 \global\count\footins\@m
 \set@column@hsize\pagegrid@col
 \set@colht
}%

\def\frontmatter@abstractheading{%
\bigskip
 \begingroup
  \centering\large
  \abstractname
  \par\bigskip
 \endgroup
}%

\makeatother

%\DeclareSymbolFont{CMlargesymbols}{OMX}{cmex}{m}{n}
%\DeclareMathSymbol{\sum}{\mathop}{CMlargesymbols}{"50}
%\pdfbookmark[1]{Introduction}{Introduction}
\usetikzlibrary{decorations.pathmorphing,decorations.pathreplacing,decorations.markings}

\def\scri{\mathscr I}
\def\H{\mathscr H}
\def\ip{i^+}
\def\im{i^-}
\def\iz{i^0}

\begin{document}

Here is a Penrose diagram (or Carter-Penrose diagram, or Conformal diagram) for empty space.

\begin{center}
\begin{tikzpicture}[scale=5]
        
    % BOUNDING DIAGRAM
    \coordinate (O) at ( 0, 0); % center: origin (r,t) = (0,0)
    \coordinate (S) at ( 0,-1); % south: t=-infty, i-
    \coordinate (N) at ( 0, 1); % north: t=+infty, i+
    \coordinate (E) at ( 1, 0); % east:  r=+infty, i0
    \coordinate (NE) at ( .5, .5); 
    \coordinate (SE) at ( .5, -.5);
    \coordinate (SI) at (.6,.7);
    \draw[thick] (NE) -- (E);
    \draw[thick] (SE) -- (E);
    \draw[dashed] (SE) -- (NE);

    % LABELS
    \node[above right] at (E) {\small $\iz$};
    \node[above] at (NE) {\small $\ip$};
    \node[below] at (SE) {\small $\im$};
    \node[above right] at (0.75,0.25) {\small $\scri^+$};
    \node[below ] at (0.75,-0.3) {\small $\scri^-$};

\end{tikzpicture}
\end{center}

What this picture does is collapse all of space into the $x$-axis and time into the $y$-axis.
The idea is then that you can summarize all of the possible causal relationships in
all of infinite space and all of time in one small finite picture. Objects moving at the speed
of light then move along lines that have a slope of exactly 45 degrees
($\pi/4$). These are also called ``null'' rays or geodesics (e.g. shortest paths through spacetime).

For convenience we stretched the diagram in the $x$ direction a bit so we have a bit more room for labels
and such. 


Here are what the labels mean:

\begin{itemize}
\item $i^+$ is where time is at $+$ infinity, also called ``timelike'' infinity.
\item $i^-$ is where time is at $-$ infinity.
\item $i^0$ is $+$ infinity for space, , also called ``spacelike'' infinity. The zero here is because
time is zero on this line?
\item $\scri^+$ is ``future null infinity''. That is, it's where null rays go to in the future.
\item $\scri^-$ is ``past null infinity''. This is where null rays come from in the past.
\end{itemize}

Now here is a more complicated diagram of a star collapsing to become a black hole. This diagram is not
square anymore because I wanted more room for the extra labels and things. So it's a bit wider than it is
tall for convenience. That's why $i^0$ seems like it's in the wrong place.

\begin{center}
\begin{tikzpicture}[scale=5]
        
    % BOUNDING DIAGRAM
    \coordinate (O) at ( 0, 0); % center: origin (r,t) = (0,0)
    \coordinate (S) at ( 0,-1); % south: t=-infty, i-
    \coordinate (N) at ( 0.5, 1); % north: t=+infty, i+
    \coordinate (E) at ( 1.25, 0.25); % east:  r=+infty, i0
    \coordinate (NE) at ( .75, .75); 
    \coordinate (SM) at ( .5, -.5);
    \coordinate (NM) at ( 1, .5);
    \coordinate (mid) at (.5,.5);
    \draw[thick] (E) -- (NE);
    \draw[thick] (S) -- (E);
    \draw[dashed] (S) -- (0,.75);
    \draw[thick] (NE) -- (O);
	\draw[thick] (SM) -- (O);
    % LABELS
    	
    \node[above right] at (E) {\small $\iz$};
    \node[above] at (NE) {\small $\ip$};
    \node[below] at (S) {\small $\im$};
    \node[rotate=90] at (-.06,0) {\small$r=0$};
    \node[above right] at (1,.5) {\small$\scri^+$};
    \node[below right] at (.8,-.15) {\small$\scri^-$};
    \node[above, rotate=45] at (0.4,.4) {\small ${\H}^+$};

    % Coordinates
    \node[below, rotate=45] at (0.4,0.4) {\small $u = \infty$};
    \node[above, rotate=45] at (0.86,-0.15) {\small $u = -\infty$};
    
    \node[above, rotate=-45] at (0.25,0.-0.25) {\small $v = v_0$};
    \node[below, rotate=-45] at (1,.5) {\small $v = \infty$};
    
    % singularity
  	\draw[thick, decorate,decoration={zigzag,aspect=0,amplitude=2.3pt,segment length=3.5pt}] (0,0.75) -- (NE);

    % Collapsing shell
    \path [draw=blue] (S) .. controls (.2,0) .. (.01,.75);
    
    % Text labels
    \coordinate (SI) at (.5,1);
    \node [anchor=north](C) at (.55,-.75) {\footnotesize Collapsing star};
    \node[anchor=south] at (SI) {\footnotesize Singularity};

    \draw[-latex]  (C.north) -- (0.11, -.5);
    \draw[-latex]  (SI)  -- (0.3, .78);

\end{tikzpicture}
\end{center}

Here are what the new labels mean:

\begin{itemize}
\item $r=0$ is the origin of a polar coordinate system if we wanted to use one. Along this line the
radius of the start is $0$.
\item ${\mathcal H}^+$ is position of the event horizon at plus infinity.
\item The wavy line is the the singularity inside the black hole.
\item The blue line is the star collapsing and forming the black hole.
\item The coordinate system $(u,v)$ moves along the null rays (45 degree lines). The event
horizon of the black hole is that $u=\infty$, and the future of everything is that $v=\infty$.
\item The line at $v=v_0$  passes through the star and forms the
event horizon. Thus any ray with $v > v_0$ falls into the black hole, and $v < v_0$ escpes. 
We can think of $v_0$ as the $0$ point in the $v$ direction.
\end{itemize}

\end{document}
