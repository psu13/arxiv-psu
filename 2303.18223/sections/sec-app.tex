\section{Applications}\label{sec-application}


As LLMs are pre-trained on a mixture of source corpora, they can capture rich knowledge from large-scale pre-training data, thus having the potential to serve as domain experts or specialists for specific areas. 
In this section, we briefly review the recent progress on the applications of LLMs on several representative domains, including healthcare, education, law, finance, and scientific research. 

\paratitle{Healthcare} is a vital application field closely related to human life. Ever since the advent of ChatGPT, a number of studies have applied ChatGPT or other LLMs to the medical domain. 
It has been shown that LLMs are capable of handling  %
{a variety of healthcare tasks, \eg biology information extraction~\cite{tang-arxiv-2023-does}, medical advice consultation~\cite{Nov-arxiv-2023-Medical}, mental health analysis~\cite{Yang-arxiv-2023-mental}, and report simplification~\cite{Jeblick-arxiv-2023-Medicine}}. 
{As the major technical approach, researchers typically design specific prompts or instructions to guide LLMs to perform a wide range of medical tasks. }
To further harness the power of LLMs in the healthcare domain, researchers propose to develop healthcare-related  LLMs.
Specifically, the Med-PaLM models~\cite{singhal-arxiv-2022-large,Singhal-2023-arxiv-Towards} achieves  expert-level performance  on the {United States Medical Licensing Examination (USMLE)}, and earns greater approval from physicians in answering consumer's medical questions.
However, LLMs may fabricate medical misinformation~\cite{Jeblick-arxiv-2023-Medicine,Chen-medrxiv-2023-cancer}, \eg misinterpreting  medical terms and suggesting advice inconsistent  with medical guidelines.  In addition, it would also raise privacy concerns to upload the health information of patients~\cite{tang-arxiv-2023-does} into a commercial server that support the LLM.

{
\paratitle{Education}
is also an important application domain where LLMs potentially exert significant influence.  %
Existing work has found that LLMs can achieve student-level performance on standardized tests~\cite{OpenAI-OpenAI-2023-GPT-4} in a variety of  subjects of mathematics (\eg physics, computer science) on both multiple-choice and free-response problems.
In addition, empirical studies have shown that LLMs can  serve as writing or reading assistant  for education~\cite{Malinka-arxiv-2023-Education,Susnjak-arxiv-2022-Education}.
A recent study~\cite{Susnjak-arxiv-2022-Education} reveals that 
ChatGPT is capable of generating logically consistent answers across disciplines, balancing both depth and breadth.
Another quantitative analysis~\cite{Malinka-arxiv-2023-Education} shows that {students utilizing ChatGPT (either keeping or refining the results from LLMs as their own answers) perform better than average students in some courses from the computer security field. 
{Recently, several perspective papers~\cite{Tan-arxiv-2023-towards,Kamalov-2023-arxiv-A} also explore various application scenarios of LLMs in classroom teaching, such as teacher-student collaboration, personalized learning, and assessment automation.}
{However, the application of LLMs in education may lead to a series of practical issues, \eg  plagiarism, potential bias in AI-generated content, overreliance  on LLMs, and inequitable access for non-English speaking individuals~\cite{Kasneci-learning-2023-chatgpt}.}
}


\paratitle{Law}
is a specialized domain that is built on professional domain knowledge. 
{Recently, a number of studies have applied LLMs} to solve various legal tasks, \eg legal document analysis~\cite{Stanek-arxiv-2023-Can}, legal judgment prediction~\cite{Trautmann-arxiv-2022-Legal}, and legal document writing~\cite{Choi-SSRN-2023-Chatgpt}. A recent study~\cite{Nay-arxiv-2022-Law} has found that 
{LLMs exhibit powerful abilities of legal interpretation and reasoning.} 
Moreover, the latest GPT-4 model achieves a top 10\% score in a simulated bar exam compared with human test-takers~\cite{OpenAI-OpenAI-2023-GPT-4}. 
{
To further improve the performance of LLMs in the law domain,  specially designed legal prompt engineering are employed to  yield advanced performance in long legal document comprehension and complex legal reasoning~\cite{Yu-2022-arxiv-Legal,Trautmann-2022-arxiv-Legal}.
To summarize the progress, LLMs can act as helpful assistants to legal profession.  
Despite the progress, the use of LLMs in law raises concerns about legal challenges, including copyright issues~\cite{Tamkin-arxiv-2021-Understanding}, personal information leakage~\cite{Sun-arxiv-2023-A}, or bias and discrimination~\cite{Abid-AIES-2021-Persistent}.
}


\paratitle{Finance}
is an important field where LLMs have promising application prospects. 
LLMs have been employed on various finance related tasks, such as numerical claim detection~\cite{Shah-arxiv-2023-Zero}, financial sentiment analysis~\cite{Araci-arxiv-2023-FinBERT}, financial named entity recognition~\cite{Alvarado-ALTA-2015-Domain}, and financial reasoning~\cite{Son-arxiv-2023-Beyond}.
Despite the competitive zero-shot performance exhibited by general-purpose LLMs in the finance tasks, they still underperform domain-specific PLMs containing million-scale  parameters~\cite{Shah-arxiv-2023-Zero}.
To leverage the scaling effect of LLMs, researchers collect large-scale finance corpora for continually pre-training LLMs (\eg BloombergGPT~\cite{wu-arxiv-2023-bloomberggpt}, XuanYuan 2.0~\cite{zhang-arxiv-2023-xuanyuan}, and FinGPT~\cite{Yang-2023-arxiv-FinGPT}).
BloombergGPT has demonstrated remarkable performance across a diverse range of financial tasks while maintaining competitive performance in general-purpose tasks~\cite{wu-arxiv-2023-bloomberggpt}.
Nevertheless, it is imperative to consider the potential risks in the application of LLMs in finance, as the generation of inaccurate or harmful content by LLMs could have significant adverse implications for financial markets~\cite{wu-arxiv-2023-bloomberggpt}.
Therefore, it needs more strict reviewing  and monitoring on the use of LLMs in the financial field. 


\paratitle{Scientific research} is another 
promising field that LLMs can empower the development progress. %
Prior research demonstrates the effectiveness of LLMs in handling knowledge-intensive scientific tasks (\eg PubMedQA~\cite{Jin-emnlp-2019-PubMedQA}, BioASQ~\cite{Anastasia-blog-2022-BioASQ}), especially for LLMs that are pre-trained on scientific-related corpora (\eg Galactica~\cite{Taylor-arxiv-2022-Galactica}, Minerva~\cite{Lewkowycz-arxiv-2022-Solving}).
Given the excellent general abilities and broad scientific knowledge, LLMs hold significant potential as helpful assistants across various stages of the scientific research pipeline~\cite{Zhang-arxiv-2023-One}. 
First, during the literature survey stage, LLMs can help conduct a comprehensive overview of the  progress in a specific research  field~\cite{Haman-2023-air-Using,Aydn-2022-ssrn-OpenAI}.
Second, during the research idea generation stage, LLMs demonstrate the ability to generate intriguing scientific hypotheses~\cite{Part-2023-arxiv-Can}.
Third, during the data analysis stage, LLMs can be employed to conduct automatic approaches to analyzing the data characteristics,  including data exploration, visualization, and deriving analytical conclusions~\cite{Hasaan-2023-arxiv-ChatGPT,Cheng-2023-arxiv-Is}.
Fourth, during the paper writing stage,  researchers can also benefit from the assistance of LLMs in scientific writing~\cite{Alkaissi-pubmed-2023-Artificial,Azaria-2023-arxiv-ChatGPT}, in which  {LLMs can offer valuable support for scientific writing through diverse means, such as summarizing the existing  content and polishing the writing~\cite{Buruk-2023-arxiv-Academic}. } 
In addition, LLMs can aid in the automated paper review process, encompassing tasks such as error detection, checklist verification, and candidate ranking~\cite{Liu-2023-arxiv-ReviewerGPT}.
Despite these advances, 
there is much room for improving the capacities of LLMs to serve as helpful, trustworthy scientific assistants, to both increase the quality of the generated scientific content  and reduce the harmful hallucinations. 
%

\emph{Summary}.
In addition to the aforementioned work, 
the applications of LLMs have been also discussed in several other domains.  
For instance, in the psychologic domain, some recent work has studied the human-like characteristics of LLMs, such as self-awareness, theory of mind~(ToM), and affective computing~\cite{Kosinski-arxiv-2023-tom,Amin-arxiv-2023-affective}.
In particular, %
{an  empirical evaluation of ToM conducted on two classic false-belief tasks} speculates that LLMs may have ToM-like abilities since the model in the GPT-3.5 series achieves comparable performance with nine-year-old children in ToM task~\cite{Kosinski-arxiv-2023-tom}.
In addition, another line of work has investigated applying LLMs into the software development domain, \eg code suggestion~\cite{Sridhara-2023-arxiv-ChatGPT}, code summarization~\cite{Sun-2023-arxiv-Automatic}, and automated program repair~\cite{Xia-2023-arxiv-Conversational}.
To summarize, to assist humans by LLMs in real-world tasks has become a significant area of research.
However, it also presents challenges. Ensuring the accuracy of LLM-generated content, addressing biases, and maintaining user privacy and data security are crucial considerations when applying LLMs to real-world scenarios.

