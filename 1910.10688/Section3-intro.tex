%!TEX root =  ./JanasJanssenCuffaro-August2019.tex

%SECTION 3 -- INTRO
%\section{Generalization to the singlet state of two particles with higher spin} \label{2}
%Three settings and more than two outcomes per setting: 
%Generalization to pairs of higher-spin particles entangled in the singlet state


In Section \ref{1} we studied pairs of bananas that behave like pairs of spin-$\frac12$ particles in the singlet state. This is just one type of banana in Bananaworld. There are many more. In Section \ref{1.2}, for instance, we came across Popescu-Rohrlich bananas. \citet[Ch.\ 6]{Bub 2016} has made extensive study of other exotic species, such as Aravind-Mermin bananas and Klyachko bananas. One could add to this taxonomy by introducing pairs of bananas that behave like pairs of particles of arbitrary integer or half-integer spin $s \ge 1$ in the singlet state. For $s=1$, bananas would taste yummy ($+$), nasty ($-$), or meh ($0$). As we move to higher spin, we would have to invent more refined taste palettes. The banana imagery thus starts to feel forced for higher spin and we will largely dispense with it in the remainder of this paper. Instead we will phrase our analysis directly in terms of spin. So rather than have Alice and Bob peel and taste pairs of bananas, we imagine them sending pairs of particles through Dubois magnets and measuring a component of their spin, choosing between three different directions, represented, as before, by unit vectors $(\vec{e}_a, \vec{e}_b, \vec{e}_c)$, corresponding to settings $(\hat{a}, \hat{b}, \hat{c})$. In Section \ref{2.1} we present the quantum-mechanical analysis of the correlations found in these measurements, extending our discussion of the spin-$\frac12$ case in Section \ref{1.5}. In Section \ref{2.2} we design raffles like the ones we introduced in Section \ref{1.4} to simulate these correlations. 

For the quantum-mechanical analysis we rely on the standard treatment of rotation in quantum mechanics (see, e.g., Messiah 1962, Vol.\ 2, Appendix C, or Baym 1969, Ch.\ 17). This will lead us to the so-called Wigner d-matrices (see Eq.\ (\ref{wigner})) with which we can readily compute the probabilities entering into the correlation arrays for measurements on the singlet state of two particles with arbitrary (half-) integer spin $s$. After showing how the results we found in Section \ref{1.4} for the spin-$\frac12$ case are recovered (and justified) in this more general formalism, we use it to find the entries of a typical cell in the correlation array for the spin-1 case (see Figure \ref{CA-cell-spin1-chi}). We then prove that the correlation arrays for higher-spin cases share some key properties with those for the spin-$\frac12$ and spin-1 cases (Sections \ref{2.1.1}--\ref{2.1.3}). 

In Section \ref{2.1.4}, we show that all such correlation arrays have uniform marginals and are therefore non-signaling. In Section \ref{2.1.5}, we show that they can still be parametrized by the anti-correlation coefficients for three of their off-diagonal cells\footnote{In Section \ref{2}, we referred cells on and off the diagonal of various correlation arrays. From now on, we will use the more economical designations `diagonal cells' and `off-diagonal cells'.\label{diag cells economy}} and that these coefficients are still given by the cosines of the angles between measuring directions,
\begin{equation}
\chi_{ab} = \cos{\varphi_{ab}}, \quad \chi_{ac} = \cos{\varphi_{ac}},  \quad \chi_{bc} = \cos{\varphi_{bc}},
\label{intro sec 2a}
\end{equation} 
and subject to the same constraint we found in the spin-$\frac12$ case  in Section \ref{1.4} (see Eq.\ (\ref{QM14})): 
\begin{equation}
1 - \chi_{ab}^2 - \chi_{ac}^2 - \chi_{bc}^2 + 2 \, \chi_{ab} \, \chi_{ac} \, \chi_{bc} \ge 0.
\label{intro sec 2b}
\end{equation}
It follows that the class of correlations allowed by quantum mechanics in measurements on the singlet state of two particles with  (half-)integer spin $s$ can be represented by the elliptope in Figure \ref{elliptope} regardless of the value of $s$.

In Section \ref{2.1.6}, we turn to a property of the correlation arrays for these higher-spin cases we did not pay much attention to in the spin-$\frac12$ case: the symmetries of their cells. We show that, for arbitrary (half)-integer $s$, any cell in the correlation array for measurements on the singlet state of particles with (half-)integer spin $s$ is \emph{centrosymmetric}, \emph{symmetric} and \emph{persymmetric}, i.e., it is unchanged under reflection about its center, across its main diagonal and across its main anti-diagonal (see Eq.\ (\ref{Prob sym}) for what this means in terms of the probabilities that form the entries of such cells). 

In Section \ref{2.2}, we design raffles to simulate these quantum correlations. In preparation for this, we formalize the description and analysis of the raffles we used in Section \ref{1.4} for the spin-$\frac12$ case (Section \ref{2.2.1}). In Sections \ref{2.2.2}--\ref{2.2.4}, we adapt the formalism developed for this case to design raffles that simulate---to the extent that this at all possible---the correlation arrays for measurements on the singlet state of two particles with higher spin. In doing so, we run into four main complications compared to the spin-$\frac12$ case. 

First, as we already noted in Section \ref{1.6}, we can no longer admit single-ticket raffles. Our raffle tickets only have two outcomes per setting. If there are more than two possible outcomes (and for spin $s$ there are $2s+1$), it is therefore impossible to have diagonal cells of the form shown in Figure \ref{diag-cell-sxs}. Such single-ticket raffles not only fail to reproduce the quantum correlations; without identical diagonal cells, the anti-correlation coefficients for the off-diagonal cells no longer suffice to characterize the correlations generated by these raffles. To get around this problem we need to restrict ourselves to mixed raffles that give uniform marginals. These consist of combinations of tickets such that all $2s+1$ outcomes occur with the same frequency. This will guarantee that the diagonal cells in the resulting correlation arrays all have the form of the cell in Figure \ref{diag-cell-sxs}. Since our raffles are non-signaling by construction, this ensures uniform marginals for the entire correlation array.   

The second complication has to do with the relationship between probabilities and expectation values (or anti-correlation coefficients). In the spin-$\frac12$ case, the probabilities 
in any cell of the correlation array could be parametrized by the anti-correlation coefficient for that cell (see Figures \ref{CA-2set2out-cell} and \ref{CA-3set2out-non-signaling-chis} in Section \ref{1.3}). In the quantum correlation arrays, as noted above, this remains true for arbitrary (half-) integer spin $s$. As soon as there are more than two possible outcomes, however, it fails for our raffles. This severely limits our ability to simulate the quantum correlation arrays for higher-spin cases. We can design mixed raffles that simulate the diagonal cells of the quantum correlation arrays and that give the correct values for the \emph{anti-correlation coefficients} of the off-diagonal cells; yet these raffles will, in general, still not give the right values for the \emph{probabilities} in the off-diagonal cells. In Section \ref{2.2.2}, we will encounter a striking example of this complication. We will construct a raffle for the spin-1 case for which the sum of the anti-correlation coefficients is $-\sfrac32$, the Tsirelson bound for this setup (see Eq.\ (\ref{Mermin CHSH integer spin}) in Section \ref{1.6}). Yet the off-diagonal cells of the correlation array for this raffle are different from those in the quantum correlation array it was meant to simulate (see Eq.\ (\ref{off diag cell quantum v raffle})). 

The third (less serious) complication has to do with the symmetries of the off-diagonal cells in the correlation array. The design of our raffles guarantees that all cells in their correlation arrays are centrosymmetric. The condition for centrosymmetry of such cells, say the  $\hat{a} \, \hat{b}$ one,  is (see Eq.\ (\ref{Prob sym})): 
\begin{equation}
\mathrm{Pr}(m_1 m_2| \hat{a} \,\hat{b}) = \mathrm{Pr}(-m_1 -\!m_2| \hat{a} \,\hat{b}).
\label{raffle cell centro-symmetry} 
\end{equation}
This condition is automatically satisfied by our raffles: it simply expresses that Alice and Bob are as likely to get one side of the ticket as the other. The entries in cells of correlation arrays in the spin-$\frac12$ case form $2 \times 2$ matrices. In that case, centrosymmetry trivially implies both symmetry and persymmetry. For the $(2s +1) \times (2s +1)$ matrices formed by the entries in cells of correlation arrays in the spin-$s$ case with $s \ge 1$, this is no longer true (although any two of these symmetries still imply the third). Cells in the quantum correlation arrays, as noted above, have all three symmetries, regardless of the spin of the particles in the singlet state on which Alice and Bob perform their measurements. To correctly simulate this feature of the quantum correlations we thus need to impose additional symmetry conditions on our raffles. Fortunately, this can be done without too much trouble.
%\footnote{It does have the unfortunate consequence, however, that we cannot formally prove that the polyhedra we will construct for the higher-spin cases asymptotically approach the elliptope in the limit that $s$ goes to infinity, as strongly suggested by looking at these polyhedra (see note \ref{no-convergence-proof}).} 

The fourth complication is perhaps the most obvious one. As the number of outcomes increases, so does the number of different ticket types in our raffles. Figure \ref{raffle-tickets-3set3out-i-xiv} shows the $(3^3 + 1)/2 = 14$ different ticket types for raffles in the spin-1 case. Figures \ref{raffles-spin32-tickets-mu} and \ref{raffles-spin32-tickets-nu} show some of the $4^3/2 = 32$ different ticket types for the spin-$\frac32$ case. In dealing with these higher-spin cases, we therefore turned to the computer for guidance.

As in the spin-$\frac12$ case, we will represent the class of triplets of anti-correlation coefficients $(\chi_{ab}, \chi_{ac}, \chi_{bc})$ for \emph{admissible raffles} in the spin-$s$ case (i.e., raffles that give uniform marginals and meet the symmetry requirements) by a polyhedron in the same 3-dimensional non-signaling cube as before. Henceforth, we will call this  the \emph{anti-correlation polyhedron}. In the spin-$\frac12$ case, the anti-correlation polyhedron doubles as the local polytope (see Figures \ref{LQP} and \ref{elliptope-LQPslice}). In the spin-$s$ case (with $s \ge 1$), the anti-correlation polyhedron is a particular (highly informative) projection of (a restricted version of) a now higher-dimensional local polytope (restricted by our admissibility conditions) to three dimensions (cf.\ the discussion above about complications in the relationship between probabilities and anti-correlation coefficients). The flowchart in Figure \ref{flowchart} shows how we get from the local polytope to the anti-correlation polyhedron in the spin-1 case. With considerable help from the computer, we were able to construct anti-correlation polyhedra for $s = 1, \sfrac32,  2, \sfrac52$ (see Figures \ref{polytope-spin1}, \ref{SpinThreeHalfFace} and \ref{FacetsSpin2Spin52}). 

We pay special attention to admissible raffles for which the sum $\chi_{ab} + \chi_{ac} + \chi_{bc}$ takes on its minimum value (see Eqs.\ (\ref{Mermin CHSH half-integer spin}) and (\ref{Mermin CHSH integer spin}) in Section \ref{1.6.3}). In constructing these raffles, we take advantage of the insight that they will involve tickets for which the sum of the outcomes on both sides is either zero (for integer spin) or $\sfrac12$ (for half-integer spin) (see Figures \ref{admissible-raffles-spin1}, \ref{raffles-spin32-tickets-mu}, \ref{raffles-spin32-tickets-nu} and Table \ref{Spin2TicketGroups} for the tickets in such raffles). Comparing the anti-correlation polyhedra for higher spin values $s$  to the tetrahedron for $s = \sfrac12$, we see that these polyhedra get closer and closer to the elliptope as the number of outcomes $2s + 1$ increases (see Figure \ref{polytopevolume}).\footnote{We have not been able to construct a formal proof of this convergence  (see note \ref{no-convergence-proof}).} This is just what we would expect given what we learned in Section \ref{1.6.3}, viz.\ that Eq.\ (\ref{intro sec 2b}) for the elliptope determines the broadest conceivable class of triplets of (anti-)correlation coefficients.
%Slogan: there is no physics beyond the elliptope. 
