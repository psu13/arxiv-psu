%!TEX root =  ./JanasJanssenCuffaro-August2019.tex

%SECTION 6 -- Conclusion

We noted in Section \ref{0} that for Heisenberg, quantum mechanics' significance lay in its provision of a new framework for doing physics, one that was sorely needed in light of the persistent failures of classical mechanics and the old quantum theory of Bohr and Sommerfeld to deal with the puzzling (mostly spectroscopic) experimental data it was confronted with in the first two decades of the last century \citep{Duncan and Janssen 2019}. Heisenberg's core insight into quantum mechanics' significance is one that we and the others close to us on the phylogenetic tree of interpretations share. In the body of this paper we saw a number of concrete examples vividly illustrating the essential differences between the quantum and the classical kinematical framework, how those differences are manifested in the correlations between and in the dynamics of quantum systems, and finally how the quantum-kinematical framework enables us to learn about the specifics of particular systems through measurement. In this final section we present our view in a nutshell.

Quantum mechanics is about probabilities. The kinematical framework of the theory is probabilistic in the sense that the state specification of a given system yields, in general, only the probability that a selected observable will take on a particular value when we query the system concerning it. Quantum mechanics' kinematical framework is also non-Boolean: The Boolean algebras corresponding to the individual observables associated with a given system cannot be embedded into a global Boolean algebra comprising them all, and thus the values of these observables cannot (at least not straightforwardly) be taken to represent the properties possessed by that system in advance of their determination through measurement. It is in this latter---non-Boolean---aspect of the probabilistic quantum-kinematical framework that its departure from classicality can most essentially be located.

Despite this character, we have seen above how the quantum-mechanical framework provides a recipe\footnote{Cf. \citet[]{Wallace 2019}, who argues that the Everett interpretation provides a general ``recipe'' for interpreting quantum theory (see also note \ref{wallace-recipe} in Section \ref{0}).} through which one can acquire information concerning particular systems by classical means. Given an ensemble of quantum systems either prepared uniformly in a particular state $| \psi \rangle$ or as a mixture of states $| \psi \rangle_i$ (described by the density operators $\hat{\rho} = | \psi \rangle\langle \psi |$ and $\hat{\rho} = \sum_i| \psi \rangle_{\!i} \, _{i\!}\langle \psi |$, respectively), and conditional upon a particular classically describable assessment of one of the parameters of the systems in that ensemble---conditional, that is, upon a particular Boolean frame that we impose on those systems---the information we obtain from our assessment can always be (re)described as having arisen from an ensemble of classical systems (like the raffles in our examples) with a certain distribution of values for the parameter in question. Further, the particular distribution observed can be predicted from the quantum state.

This recipe does not solve the \emph{profound} problem of measurement; i.e., the problem to account for how it is that only some of the classical probability distributions implicit in the quantum state description are actualized in the context of a given measurement. But even without providing an answer to this question, we see how the kinematical core of quantum mechanics provides us with all of the tools we need to give an account of the dynamics of a particular measurement interaction, and through this explain why a particular classical probability distribution can be used to characterize the statistics observed within that measurement context, despite the non-classical nature of the quantum state description.

It may be objected that the world we experience does not consist in probability distributions. Its objects include this table, that banana and the other dynamical objects we observe and interact with, both in the kitchens of the world and outside of them, every day. These objects will not be found within the quantum-kinematical framework, nor will the recipe just mentioned yield them up in and of itself. Conditional upon a given measurement, however, that recipe will allow one to transition from the quantum description of a system to the classical description of the observations which ensue. And from there we already know how to use classical theory to construct, from these observations, the familiar objects of our world.

As our examples have demonstrated, quantum theory is successful where classical theory falls short in its description of physical phenomena, and its advent has uncovered aspects of our world that were before then veiled in darkness. But besides these particular lessons there is a wider moral that we can glean from the new kinematical framework of quantum theory, and in particular by considering how it differs from classical theory. The logical framework of classical physics is a globally Boolean structure. Through it a global noncontextual assignment of values to the observables associated with physical systems becomes possible. Because of this, these value assignments may unproblematically be thought of as the underlying properties of the physical systems they have been assigned to. This allows us to speak of a world that exists in a particular way irrespective of our particular interactions with it. Quantum mechanics, however, shows us that this classical description is valid only up to a certain point, and that the logical structure of the world \emph{as it presents itself to us} is globally non-Boolean. Whatever else we may discover in the course of the future development of physical theory, this is a non-trivial fact that we have discovered about the world. Moreover it is a fact that will remain with us \citep[cf.][p. 98]{Pitowsky 1994}. It is, further, a non-trivial fact that we can learn about our world, despite this non-Boolean character, through classical means \citep[cf.][p. 293]{Bohr 1937}.\footnote{Bohr writes: ``the proper r\^ole of the indeterminacy relations consists in assuring quantitatively the logical compatibility of apparently contradictory laws which appear when we use two different experimental arrangements, of which only one permits an unambiguous use of the concept of position, while only the other permits the application of the concept of momentum.''}

It will be objected that what we have just called ``facts about the world'' are really only relational facts about our connection to the world \citep[cf.][]{Healey 2016}. This is entirely correct. But that, we maintain, is how it should be. For we are entangled with the world, and our concepts both of the world and of ourselves are only marginals of that true entangled description. That description, along with its many seemingly incompatible aspects, arises out of and is made possible through the non-Boolean probabilistic structure of the quantum-mechanical kinematical core.

Quantum theory provides us with an \emph{objective} description of a given system. This description is valid \emph{irrespective} of one's particular choices and irrespective of one's particular interests in making those choices. At the same time the description that quantum theory provides to us of a given system's dynamical state is unlike the corresponding description given to us by classical theory. In quantum theory, what is exhibited to us through the quantum state description is not the set of dynamical properties, in the classical sense, of the system of interest. What is exhibited, rather, is the structure of, interrelations between, and interdependencies among the possible perspectives one can take on that system. In this way quantum theory informs us regarding the structure of the world---a world that \emph{includes} ourselves---and of our place within that structure.
