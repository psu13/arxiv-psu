%!TEX root =  ./JanasJanssenCuffaro-August2019.tex

%SECTION 1 -- OVERALL INTRO
%\section{Introduction}\label{0}

This paper is a brief for a specific take on the general framework of quantum mechanics.\footnote{This paper deals with \emph{philosophy}, \emph{pedagogy} and \emph{polytopes}. In this introduction, we will explain how these three components are connected, both to each other and to \emph{Bananaworld} \citep{Bub 2016}. Cuffaro's main interest is in philosophy, Janssen's in pedagogy and Janas's in polytopes. Though all three of us made substantial contributions to all six sections of the paper, Janssen had final responsibility for Sections \ref{0}--\ref{1}, Janas for Sections \ref{2}--\ref{3} and Cuffaro for Sections \ref{4}--\ref{5}.} In terms of the usual partisan labels, it is an \emph{information-theoretic} interpretation in which the status of the state vector is \emph{epistemic} rather than \emph{ontic}. On the ontic view, state vectors represent what is ultimately real in the quantum world; on the epistemic view, they are auxiliary quantities for assigning definite values to observables in a world in which it is no longer possible to do so for all observables. Such labels, however, are of limited use for a taxonomy of interpretations of quantum mechanics. A more promising approach might be to construct a genealogy of such interpretations.\footnote{The contemporary literature on quantum foundations has muddied the waters in regards to the classification of interpretations of quantum mechanics, and it is partly for this reason that we prefer to give a genealogy rather than a taxonomy of interpretations. Ours is \emph{not} an epistemic interpretation of quantum mechanics in the sense compatible with the ontological models framework of \citet[]{Harrigan and Spekkens 2010}. In particular it is not among our assumptions that a quantum system has, at any time, a well-defined ontic state. Actually we take one of the lessons of quantum mechanics to be that this view is untenable (see Section \ref{4.3} below). For more on the differences between a view such as ours and the kind of epistemic interpretation explicated in \citet[]{Harrigan and Spekkens 2010}, and for more on why the no-go theorem proved by \citet{PBR} places restrictions on the latter kind of epistemic interpretation but is not relevant to ours, see \citet[]{Ben-Menahem 2017}.}
As this is not a historical paper, however, a rough characterization of the relevant phylogenetic tree must suffice here.\footnote{One of us is working on a two-volume book on the genesis of quantum mechanics, the first of which has recently come out \citep{Duncan and Janssen 2019}.} The main thing to note then is that the mathematical equivalence of wave and matrix mechanics papers over a key difference in what its originators thought their big discovery was. These big discoveries are certainly compatible with one another but there is at least a striking difference in emphasis.\footnote{We will return to this point in Section \ref{4.3a}.} For Erwin Schr\"odinger the big discovery was that a wave phenomenon underlies the particle behavior of matter, just as physicists in the 19th century had discovered that a wave phenomenon underlies geometrical optics \citep{Joas and Lehner 2009}. For Werner Heisenberg it was that the problems facing atomic physics in the 1920s called for a new framework to represent physical quantities just as electrodynamics had called for a new framework to represent their spatio-temporal relations two decades earlier \citep[pp.\ 134--142]{Duncan and Janssen 2007, Janssen 2009}. What are now labeled ontic interpretations---e.g., Everett's many-worlds interpretation, De Broglie-Bohm pilot-wave theory and the spontaneous-collapse theory of Ghirardi, Rimini and Weber (GRW)---can be seen as descendants of wave mechanics; what are now labeled epistemic interpretations---e.g., the much maligned Copenhagen interpretation and Quantum Bayesianism or QBism---as descendants of matrix mechanics.\footnote{David \citet{Wallace 2019} provides an example from the quantum foundations literature showing that the ``big discoveries'' of matrix and wave mechanics are not mutually exclusive. He argues that the Everett interpretation should be seen as a general new framework for physics while endorsing the view that vectors in Hilbert space represent what is real in the quantum world. Wallace and other Oxford Everettians derive the Born rule for probabilities in quantum mechanics from decision-theoretic considerations instead of taking it to be given by the Hilbert space formalism the way von Neumann showed one could (see below). For Berlin Everettians (i.e., at least some of the Christoph Lehners  in their multiverse) state vectors are both ontic and epistemic. They help themselves to the Born rule \emph{\`a la} von Neumann but also use state vectors to represent physical reality (Christoph Lehner, private communication).\label{wallace-recipe}}

The interpretation for which we will advocate in this paper can, more specifically, be seen as a descendant of the (statistical) transformation theory of Pascual \citet{Jordan 1927a, Jordan 1927b} and Paul \citet[amplified in his famous book, \citealt{Dirac 1930}]{Dirac 1927} and of the ``probability-theoretic construction'' (\emph{Wahrscheinlichkeitstheoretischer Aufbau}) of quantum mechanics in the second installment of the trilogy of papers by John \citet{von Neumann 1927a, von Neumann 1927b, von Neumann 1927c}  that would form the backbone of \emph{his} famous book \citep{von Neumann 1932}. While incorporating the wave functions of wave mechanics, both Jordan's and Dirac's version of transformation theory grew out of matrix mechanics. More strongly than Dirac, Jordan emphasized the statistical aspect. The ``new foundation'' (\emph{Neue Begr\"undung}) of quantum mechanics announced in the titles of Jordan's 1927 papers consisted of some postulates about the probability of finding a value for one quantum variable given the value of another. Von Neumann belongs to that same lineage. Although he proved the mathematical equivalence of wave and matrix mechanics in the process (by showing that they correspond to two different instantiations of Hilbert space), he wrote his 1927 trilogy in direct response to Jordan's version of transformation theory. His \emph{Wahrscheinlichkeitstheoretischer Aufbau} grew out of his dissatisfaction with Jordan's treatment of probabilities. Drawing on work on probability theory by Richard von Mises \citep[soon to be published in book form;][]{von Mises 1928}, he introduced the now familiar density operators characterizing (pure and mixed state) ensembles of quantum systems.\footnote{For historical analysis of these developments, focusing on Jordan and von Neumann, see \citet{Duncan and Janssen 2013} and, for a summary aimed at a broader audience, \citet[pp.\ 142--161]{Janssen 2019}.} He showed that what came to be known as the Born rule for probabilities in quantum mechanics can be derived from the Hilbert space formalism and some seemingly innocuous assumptions about properties of the function giving expectation values \citep[sec.\ 6, pp.\ 246--251]{Duncan and Janssen 2013}. This derivation was later re-purposed for the infamous von Neumann no-hidden variables proof, in which case the assumptions, entirely appropriate in the context of the Hilbert space formalism for quantum mechanics, become highly questionable \citep{Bub 2010, Dieks 2017}. 

A branch on the phylogenetic tree of interpretations of quantum mechanics closer to our own is the one with Jeffrey Bub and Itamar Pitowsky's (2010) ``Two dogmas of quantum mechanics,''  a play on W.\ V.\ O.\ Quine's (1951) celebrated ``Two dogmas of empiricism.'' Bub and Pitowsky presented their paper in the Everettians' lion's den at the 2007 conference in Oxford marking the 50th anniversary of the Everett interpretation.\footnote{The video of their talk can still be watched at 
\textless\url{users.ox.ac.uk/~everett/videobub.htm}\textgreater} It appears in the proceedings of this conference. Enlisting the help of his daughter Tanya, a graphic artist, Bub has since made two valiant attempts to bring his and Pitowsky's take on quantum mechanics to the masses. Despite its title and lavish illustrations, \emph{Bananaworld: Quantum Mechanics for Primates} \citep{Bub 2016} is not really a popular book. Its sequel, however, the graphic novel \emph{Totally Random} \citep{Bub and Bub 2018}, triumphantly succeeds where  \emph{Bananaworld} came up short.\footnote{See, e.g., the review in \emph{Physics World} by Minnesota physicist Jim \citet{Kakalios 2018}, well-known for his use of comic books to explain physics \citep{Kakalios 2009}, and the review in \emph{Physics Today} by philosopher of quantum mechanics Richard \citet{Healey 2019}.} The interpretation promoted overtly in \emph{Bananaworld} and covertly in \emph{Totally Random} has been dubbed \emph{Bubism} by Robert Rynasiewicz (private communication).\footnote{In an essay review of \citet{Ball 2018}, \citet{Becker 2018} and \citet{Freire 2015}, \citet{Bub 2019} gives a concise characterization of his views and places them explicitly in the lineage of Heisenberg sketched above.} 
Like QBism, Bubism is an information-theoretic interpretation but for a Bubist quantum probabilities are objective chances whereas for a QBist they are subjective degrees of belief. Our defense of Bubism builds on the Bubs' two books and on ``Two dogmas \ldots'' as well as on earlier work by (Jeff) Bub and Pitowsky, especially the latter's lecture notes \emph{Quantum Probability---Quantum Logic} and his paper on George Boole's ``conditions of possible experience'' \citep{Pitowsky 1989a, Pitowsky 1994}. We will rely heavily on tools developed by these two authors, Bub's correlation arrays and Pitowsky's correlation polytopes. A third musketeer on whose insights we drew for this paper is William Demopoulos (see, e.g., Demopoulos, 2010, and, especially, Demopoulos, 2018, a monograph he completed shortly before he died, which we fervently hope will be published soon).\footnote{We dedicate our paper to Bill and Itamar. See \citet{Bub and Demopoulos 2010} for a moving obituary of Itamar.} 

In the spirit of \emph{Bananaworld}, \emph{Totally Random} and Louisa Gilder's (2008) lovely \emph{The Age of Entanglement}, we wrote the first part of our paper (i.e., most of Section \ref{1}) with a general audience in mind. We will frame our argument in this part of the paper in terms of a variation of Bub's peeling and tasting of quantum bananas scheme (see Figures \ref{AliceBob-Mermin} and \ref{AliceBob-tasting}). This is not just a gimmick adopted for pedagogical purposes. It is also intended to remind the reader that, on a Bubist view, inspired by Heisenberg rather than Schr\"odinger, quantum mechanics provides a new framework for dealing with arbitrary physical systems, be they waves, particles, or various species of fictitious quantum bananas. The peeling and tasting of bananas also makes for an apt metaphor for the (projective) measurements we will be considering throughout \citep{Popescu 2016}. 

As the title of our paper makes clear, however, we follow Jordan rather than Bub in arguing that quantum mechanics is essentially a new framework for handling \emph{probability} rather than \emph{information}. We are under no illusion that this substitution will help us steer clear of two knee-jerk objections to information-theoretic approaches to the foundations of quantum mechanics: \emph{parochialism} and \emph{instrumentalism} (or \emph{anti-realism}).

What invites complaints of parochialism is the slogan ``Quantum mechanics is all about information,'' which conjures up the unflattering image of a quantum-computing engineer, who, like the proverbial carpenter, only has a hammer and therefore sees every problem as a nail. It famously led John \citet[p.\ 34]{Bell 1990} to object: ``\emph{Whose} information? Information about \emph{what}?'' In \emph{Bananaworld}, \citet[p.\ 7]{Bub 2016} counters that ``we don't ask these questions about a USB flash drive. A 64 GB drive is an information storage device with a certain capacity, and whose information or information about what is irrelevant.'' A computer analogy, however, is probably not the most effective way to combat the lingering impression of parochialism. We can think of two better responses to the parochialism charge.

The first is an analogy with meter rather than memory sticks. Consider the slogan ``Special relativity is all about space-time'' or ``Special relativity is all about spatio-temporal relations." These slogans, we suspect, would not provoke the hostile reactions routinely elicited by the slogan ``Quantum mechanics is all about information.'' Yet, one could ask, parroting Bell: ``spatio-temporal relations of \emph{what}?'' The rejoinder in this case would simply be that \emph{what} could be any physical system allowed by the theory; and that, to qualify as such, it suffices that \emph{what} can consistently be described in terms of mathematical quantities that transform as scalars, vectors, tensors or spinors under Lorentz transformations. When we say that a moving meter stick contracts by such-and-such a factor, we only have to specify its velocity with respect to the inertial frame of interest, not what it is made of. Special relativity imposes certain kinematical constraints on any physical systems allowed by theory. Those constraints are codified in the geometry of Minkowski space-time. There is no need to reify Minkowski space-time. We can think of it in relational rather than substantival  terms \citep{Janssen 2009}. The slogan ``Quantum mechanics is all about information/probability'' can be unpacked in a similar way. Quantum mechanics imposes a kinematical constraint on allowed values and combinations of values of observables. Which observables? Any observable that can be represented by a Hermitian operator on Hilbert space. As in the case of Minkowski space-time, there is no need to reify Hilbert space. So, yes, quantum mechanics is obviously about more than just information, just as special relativity is obviously about more than just space-time. Yet the slogans that special relativity is all about space-time and that quantum mechanics is all about information (or probability) do capture---the way slogans do---what is distinctive about these theories and what sets them apart from the theories they superseded.

In Section \ref{4}, we will revisit this comparison between quantum mechanics and special relativity. We should warn the reader upfront though that the kinematical take on special relativity underlying this comparison, while in line with the majority view among physicists, is not without its detractors. In fact, the defense of the kinematical view by one of us \citep{Janssen 2009} was mounted in response to an alternative dynamical interpretation of special relativity articulated and defended most forcefully by Harvey \citet{Brown 2005}.\footnote{See \citet{Acuna 2014} for an enlightening discussion of the debate over whether special relativity is best understood kinematically or dynamically.} Both \citet[p.\ 228]{Bub 2016} in \emph{Bananaworld} and \citet[p.\ 439]{Bub and Pitowsky 2010} in ``Two dogmas \ldots''  invoked analogies with special relativity to defend their information-theoretic interpretation of quantum mechanics. \citet{Brown and Timpson 2006} have disputed the cogency of these analogies \citep[see also][]{Timpson 2010}.\footnote{What complicates matters here is that the distinction between kinematics and dynamics tends to get conflated with the distinction between constructive and principle theories \citep[p.\ 38; see Section \ref{4.1} below for further discussion]{Janssen 2009}.}

Our second response to the parochialism charge is that the quantum formalism for dealing with intrinsic angular momentum, i.e., spin, laid out in Section \ref{2.1} and used throughout in our analysis of an experimental setup to test the Bell inequalities, is key to spectroscopy and other areas of physics as well. These two responses are not unrelated. In Sections \ref{1.6.2} and \ref{4.3a}, drawing on work on the history of quantum physics by one of us,
%\footnote{See \citet{Duncan and Janssen 2008, Duncan and Janssen 2014, Duncan and Janssen 2015} and \citet{Midwinter and Janssen 2013}.} 
we will give a few examples of puzzles for the old quantum theory that physicists resolved not by altering the dynamical equations but by using key features of the kinematical core of the new quantum mechanics.  

What about the other charge against information-theoretic interpretation of quantum mechanics, \emph{instrumentalism} or \emph{anti-realism}? What invites complaints on this score in the case of Bub and Pitowsky  is their identification of the second of the two dogmas they want to reject: ``the quantum state is a representation of physical reality'' \citep[p.\ 433]{Bub and Pitowsky 2010}. This statement of the purported dogma is offered as shorthand for a more elaborate one: ``[T]he quantum state has an ontological significance analogous to the significance of the classical state as the `truthmaker' for propositions about the occurrence or non-occurrence of events'' (ibid.). Of course, denying that state vectors in Hilbert space represent physical reality in and of itself does not make one an anti-realist. We can still be realists as long as we can point to other elements of the theory's formalism that represent physical reality. The sentence we just quoted from ``Two dogmas \ldots'' suggests that for Bub and Pitowsky ``events'' fit that bill. 

That same sentence also points to an important difference between the role of points in classical phase space and vectors in Hilbert space when it comes to identifying what represents physical reality in classical and quantum mechanics. In fact, their notion of a ``truthmaker'' is particularly useful not just for pinpointing how quantum and classical mechanics differ when it comes to representing physical reality but also---even though this may not have been Bub and Pitowsky's intention---for articulating how they are similar. In both classical and quantum mechanics, reality is ultimately represented by values of observable quantities posited by the theory. How we get from catalogs of values of observable quantities to the notion of some object or system possessing the properties represented by those quantities is a separate issue. Physicists may want to leave that for philosophers to ponder, especially since this is not, we believe, what separates quantum physics from classical physics. In both cases, it seems, catalogs of values of observable quantities are primary and objects carrying properties (be it swarms of particles, fields, bananas, tables and chairs or lions and tigers) are somehow constructed out of those.\footnote{Everettians face the same issue as part of the task of explaining how the seemingly classical (Boolean) world we find ourselves in emerges from their multiverse. Bubists could piggy-back on whatever scheme the Everettians come up with to handle this issue.}  

Where quantum and classical mechanics differ is in how values are assigned to observable quantities. In classical mechanics, observable quantities are represented by functions on phase space. Picking a point in phase space fixes the values of all of these. It is in this sense that points in phase space are ``truthmakers''. In quantum mechanics, observable quantities are represented by Hermitian operators on Hilbert space. The possible values of these quantities are given by the eigenvalues of these operators. Picking a vector in Hilbert space, however, does not fix the value of any observable quantity. It fails to do so in two ways. First, the observable(s) being measured must be selected. Only those selected will be assigned definite values. Quantum mechanics tells us that, once this has happened, it is impossible for any observable represented by an operator that does not commute with those representing the selected ones to be assigned a definite value as well. Second, even after this selection has been made, the state vector will in general only give a probability distribution over the various eigenvalues of the operators for the selected observables. Which of those values is found upon measurement of the observable is a matter of chance. Vectors in Hilbert space thus doubly fail to be ``truthmakers''. \emph{Pace} Bub and Pitowsky, however, it does not follow that classical and quantum states have a different ``ontological significance''. One can maintain that neither vectors in Hilbert space \emph{nor points in phase space} represent physical reality; both can be seen as mathematical auxiliaries for assigning definite values (albeit in radically different ways) to quantities that do.\footnote{In \emph{Wahrscheinlichkeitstheoretischer Aufbau}, von Neumann also resisted the idea that vectors in Hilbert space ultimately represent (our knowledge of) physical reality. He wrote: ``our knowledge  of a system $\mathfrak{S}'$, i.e., of the structure of a statistical ensemble $\{ \mathfrak{S}'_1, \mathfrak{S}'_2,$ $\ldots \}$, is never described by the specification of a state---or even by the corresponding $\varphi$ [i.e., the vector $| \varphi \rangle$]; but usually by the result of measurements performed on the system'' \citep[p.\ 260]{von Neumann 1927b}. He thus wanted to represent ``our knowledge  of a system'' by the values of a set of observables corresponding to a complete set of commuting operators \citep[pp.\ 251--251]{Duncan and Janssen 2013}.} 

This quite naturally leads us to the first dogma \citet[p.\ 433]{Bub and Pitowsky 2010} want to reject: Measurement outcomes should be accounted for in terms of the dynamical interaction between the system being measured and a measuring device. As we will argue in Section \ref{4.4}, rejection of this dogma does not amount to black-boxing measurements. On Bub and Pitowsky's view, any measurement can be analyzed in as much detail as on any other view of quantum mechanics. It does mean, however, that one accepts that there comes a point where no meaningful further analysis can be given of why a measurement gives one particular outcome rather than another. Instead it becomes a matter of irreducible randomness---the ultimate crapshoot.\footnote{Paraphrasing what E.\ M.\ \citet[p.\ 28]{Forster 1942} once said about Virginia Woolf (``[S]he pushed the light of the English language a little further against the darkness''), one might say that quantum mechanics pushes physics right up to the point where total randomness takes over.} 

In the opening sentence of their paper, \citet[p.\ 433]{Bub and Pitowsky 2010} announce that rejection of the two dogmas they identified will expose ``the intractable part of the measurement problem''---which they, with thick irony, call the ``big'' measurement problem---as a pseudo-problem. We agree with Bub and Pitowsky that rejecting the first dogma trivially solves the measurement problem \emph{in its traditional form} of having two different dynamics side-by-side, unitary Schr\"odinger evolution as long as we do not make any measurement, state vector collapse when we do. If one accepts that ultimately measurements do not call for a dynamical account (in the sense just mentioned), the problem \emph{in this particular form} evaporates.

By our reckoning, however, the real problem is still with us, just under a different guise. That the quantum state vector is not a ``truthmaker'' in the two senses explained above raises two questions. First, how does one set of observables rather than another get selected to be assigned definite values? Second, why does an observable, once selected, take on one value rather than another? Rejection of the first dogma makes it respectable to resist the call for a dynamical account to deal with the second question and endorse the ``totally random'' response instead.\footnote{We realize that it is easier to swallow this ``totally random'' response for the observables considered in this paper (where the spin of some particle can be up or down or a banana can taste yummy or nasty) than for others, such as, notably, position (where a particle can be here or on the other side of the universe).}
Though arguments from authority will not carry much weight in these matters, we note that a prominent member of the Copenhagen camp did endorse this very answer. In an essay originally published in 1954, Wolfgang Pauli wrote: ``Like an ultimate fact without any cause, the individual outcome of a measurement is \ldots\  in general not comprehended by laws'' \citep[p.\ 32, quoted by Gilder, 2008, p.\ 169]{Pauli 1994}. This then solves Bub and Pitowsky's ``big'' measurement problem. However, it does not address the first question and thus fails to solve what they, again ironically, call the ``small'' measurement problem, which is closely related to the problem posed by this first question.\footnote{See Section \ref{4.4} for careful discussion of how our profound measurement problem differs from their ``small'' one.}

We will accordingly call  their ``big'' problem the \emph{minor} or \emph{superficial} problem and the problem closely related to their ``small'' one the \emph{major} or \emph{profound} one. The profound problem cannot be solved by a stroke of the pen---crossing out this or that alleged dogma in some quantum catechism. What it would seem to require is some account of the conditions under which one set of observables rather than another acquire (or appear to acquire) definite values (regardless of \emph{which} values). The reader will search our paper in vain for such an account. Instead, we will argue that even in the absence of a solution to the profound problem there are strong indications that Bub and Pitowsky were right to reject the two dogmas they identified (and thereby the Everettian solution to both the profound and the superficial problem).

These indications will come from our analysis---in terms of Bub's correlation arrays and Pitowsky's correlation polytopes---of correlations found in measurements on a special but informative quantum state in a simple experimental setup to test a Bell inequality due to David \citet[see Figure \ref{CA-3set2out-Mermin} for the correlation array for Mermin's example of a violation of this inequality]{Mermin 1981, Mermin 1988}. 

We introduce special raffles to determine which of these quantum correlations can be simulated by local hidden-variable theories (see Figure \ref{raffles-spin32-tickets-mu} for an example of tickets for such raffles and Figures \ref{CA-3set2out-raffles-i-thru-iv} and \ref{CA-3set2out-raffle-mix} for examples of the correlation arrays that raffles with different mixes of these tickets give rise to). These raffles will serve as our models of local hidden-variable theories. They are both easy to visualize and tolerably tractable mathematically (see Section \ref{2.2}). They also make for a natural classical counterpart to the quantum ensembles central to von Neumann's \emph{Wahrscheinlichkeitstheoretischer Aufbau}, which were themselves inspired by von Mises's classical statistical ensembles. Finally, they provide simple examples of theories suffering from the superficial but not the profound measurement problem (see note \ref{minor/major} in Section \ref{1.6.2}). 

The quantum state we will focus on is that of two particles of spin $s$ entangled in the so-called singlet state (with zero overall spin). For most of our argument it suffices to consider entangled pairs of spin-$\frac12$ particles. In Section \ref{1} we will almost exclusively consider this case. Our analysis of this case, however, is informed (and justified) at several junctures by our analysis in Section \ref{2} of cases with larger integer or half-integer values of $s$. In Section \ref{2.1}, we analyze the quantum correlations for these larger spin values; in Section \ref{2.2} we analyze the raffles designed to simulate as many features of these quantum correlations as possible. 

In Section \ref{3} we show how our analysis in Sections \ref{1} and \ref{2} can be adapted to the more common experimental setup used to test the Clauser-Horne-Shimony-Holt (CHSH) inequality. The advantage of the Mermin setup, as we will see in Section \ref{1}, is that in that case the classes of correlations allowed by quantum mechanics and by local hidden-variable theories can be pictured in ordinary three-dimensional space. The corresponding picture for the setup to test the CHSH inequality is four-dimensional. The class of all correlations in the Mermin setup that cannot be used for sending signals faster than light can be represented by an ordinary three-dimensional cube, the so-called \emph{non-signaling cube} for this setup; the class of correlations allowed by quantum mechanics by an elliptope contained within this cube; those allowed by classical mechanics by a tetrahedron contained within this elliptope (see Figures \ref{tetrahedron} and \ref{elliptope}). This provides a concrete example of the way in which Pitowsky and others have used nested polytopes to represent the convex sets formed by these classes and subclasses of correlations (compare the cross-section of the non-signaling cube, the tetrahedron and the elliptope in Figure \ref{elliptope-LQPslice} to the usual Vitruvian-man-like cartoon in Figure \ref{LQP}). Such polytopes completely characterize these classes of correlations whereas the familiar Bell inequalities in the case of local hidden-variable theories or Tsirelson bounds in the case of quantum mechanics only provide partial characterizations. 

As Pitowsky pointed out in the preface of  \emph{Quantum Probability---Quantum Logic}:  
\begin{quote}
The possible range of values of classical correlations is constrained by linear inequalities which can be represented as facets of polytopes, which I call ``classical correlation polytopes.'' These constraints have been the subject of investigation by probability theorists and statisticians at least since the 1930s, though the context of investigation was far removed from physics %The linear constraints in question include Bell's inequalities, Clauser-Horne inequalities and their generalizations 
\citep[p.\ IV]{Pitowsky 1989a}.
\end{quote}
The non-linear constraint represented by the elliptope has likewise been investigated by probability theorists and statisticians before in contexts far removed from physics. As we will see in Section \ref{1.6}, it can be found in a paper by Udny \citet{Yule 1896} on what are now called Pearson correlation coefficients as well as in papers by Ronald A.\ \citet{Fisher 1924} and Bruno de Finetti (1937).
%\citet{De Finetti 1937}. 
Yule, like Pearson, was especially interested in applications in evolutionary biology (see notes \ref{biometrist} and \ref{mendel}). We illustrate the results of these statisticians with a simple example from physics, involving a balance beam with three pans containing different weights (see Figure \ref{3M-balance} in Section \ref{1.6.4}). These antecedents in probability theory and statistics provide us with our strongest argument for the thesis that the Hilbert space formalism of quantum mechanics is best understood as a general framework for handling probabilities in a world in which only some observables can take on definite values. 

In Section \ref{1.5} we show that it follows directly from the geometry of Hilbert space that the correlations found in our simple quantum system are constrained by the elliptope and do not 
 saturate the non-signaling cube. This  derivation of the equation for the elliptope is thus a derivation \emph{from within} quantum mechanics.

\citet{Popescu and Rohrlich 1994} and others have raised the question why quantum mechanics does not allow \emph{all} non-signaling correlations. They introduced an imaginary device, now called a PR box, that exhibits non-signaling correlations stronger than those allowed by quantum mechanics.\footnote{See Figure \ref{CA-PRbox} for the correlation array for a PR box. Figure \ref{raffle-ticket-PRbox} shows that it is impossible to design tickets for a raffle that could simulate the correlations generated by a PR box.} Several authors have looked for information-theoretic principles that would reduce the class of all non-signaling correlations to those allowed by quantum mechanics (see, e.g., Clifton, Bub, and Halvorson, 2003, Bub 2016, Ch. 9, Cuffaro, 2018). Such principles would allow us to derive the elliptope \emph{from without}.\footnote{We took the within/without terminology from the chorus of ``Quinn the Eskimo,'' a song from Bob Dylan's 1967 \emph{Basement Tapes}: ``Come all without, come all within. You'll not see nothing like the mighty Quinn.'' Could ``the mighty Quinn'' be an oblique but prescient reference to a quantum computer?\label{Dylan}} 

What the result of Yule and others shows is that the elliptope expresses a general constraint on the possible correlations between three arbitrary random variables. It has nothing to do with quantum mechanics per se. As such it provides an instructive example of a kinematical constraint encoded in the geometrical structure of Hilbert space, just as time dilation and length contraction provide instructive examples of kinematic constraints encoded in the geometry of Minkowski space-time.  In Sections \ref{4.2}--\ref{4.3}, we return to this and other analogies between quantum mechanics and special relativity. In this context, we take a closer look at the interplay between \emph{from within} and \emph{from without} approaches to understanding fundamental features of quantum mechanics. 

We want to make one more observation before we get down to business. As we just saw, it is not surprising that the correlations found in measurements on a pair of particles of (half-)integer spin $s$ in the singlet state do not saturate the non-signaling cube. No such correlations between three random variables could. What is surprising (see Section \ref{1.6}) is that, even in the spin-$\frac12$ case, these correlations do saturate the elliptope. This is in striking contrast to the correlations that can be generated with the raffles designed to simulate the quantum correlations. In the spin-$\frac12$ case, the correlations allowed by our raffles are all represented by points inside the tetrahedron inscribed in the elliptope. As we will see in Section \ref{1.6}, this is because there are only two possible outcomes in the spin-$\frac12$ case, $\pm \sfrac12$. In the spin-$s$ case, there are $2s+1$ possible outcomes: $-s, -s+1, \ldots, s-1, s$. With considerable help from the computer (see the flowchart in Figure \ref{flowchart} in Section \ref{2.2.2} and the discussion of its limitations in Section \ref{2.2.4}), we generated figures showing that, with increasing $s$, the correlations allowed by the raffles designed to simulate the quantum correlations are represented by polytopes that get closer and closer to the elliptope (see Figures \ref{polytope-spin1}, \ref{SpinThreeHalfFace} and \ref{FacetsSpin2Spin52} for $s = 1, \sfrac32, 2, \sfrac52$). That the quantum correlations already fully saturate the elliptope in the spin-$\frac12$ case is due to a remarkable feature of quantum mechanics: it allows a sum to have a definite value even if the individual terms in this sum do not.


