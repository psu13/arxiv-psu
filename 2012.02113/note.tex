\chapter*{Note to the reader}


This book began life as a seminar course on functional equations at the
University of Edinburgh in 2017, motivated by recent research on the
quantification of biological diversity.  The course attracted not only
mathematicians in subjects from stochastic analysis to algebraic topology,
but also participants from physics and biology.  In response, I did
everything I could to minimize the mathematical prerequisites.

I have tried here to retain the broad accessibility of the course.  At the
same time, I have not censored myself from including the many fruitful
connections with more advanced parts of mathematics.

These two opposing forces have been reconciled by confining the more
advanced material to separate chapters or sections that can easily be
omitted.  Chapter~\ref{ch:prob} requires some probability theory,
Chapter~\ref{ch:p} some abstract algebra, and Chapter~\ref{ch:cat} some
category theory, while Sections~\ref{sec:rel-misc}, \ref{sec:mag}
and~\ref{sec:mag-geom} also call on parts of geometry, analysis and
statistics.
% 
However, the core narrative thread requires no more mathematics than a
first course in rigorous ($\epsln$-$\delta$) analysis.  Readers with this
background are promised that they are equipped to follow all the main
ideas and results.  The parts just listed, and any remarks that refer to
more specialized knowledge, can safely be omitted.

Moreover, those who regard themselves as wholly `pure' mathematicians will
find no barriers here.  Although much of this book is about the diversity
of ecological systems, no knowledge of ecology is needed.  Similarly, the
information theory that we use is introduced from the ground up.

In the middle parts of the book, many conditions on means and diversity
measures are defined: homogeneity, consistency, symmetry, etc.
Appendix~\ref{app:condns} contains a summary of this terminology for easy
reference.  There is also an index of notation.

