 

\chapter{Appendix: Klein-Gordon and Dirac equations in curved spacetimes}
\label{appB}

In this brief appendix we introduce the Klein-Gordon and Dirac equations in the background of curved spacetimes. We also present the basis to understand what is a Grassmann field as a 1+1 Dirac field.

\section{Klein Gordon equation in curved spacetimes}

In flat spacetime the Klein-gordon equation has the well-known form
\begin{equation}
(\Box-m^2) \phi =0.
\end{equation}
where the D'Alembert operator is defined as $\Box=\partial_\mu\partial^{\mu}$. 

To extend this equation to general spacetimes the first step would be promoting the partial derivatives in the D'Alembert operator to covariant derivatives, we define such a general version of this operator as
\begin{equation}
\Box_g=\nabla^\mu \nabla_\mu=g^{\mu\nu}\nabla_\mu\nabla_\nu=\frac{1}{\sqrt{|g|}}\partial_\mu\left(\sqrt{|g|}g^{\mu\nu}\partial_\nu\right)
\end{equation}
where $g$ is the determinant of the metric tensor.

One can consider a free scalar field minimally coupled (one which does not transform under change of coordinates: $\phi'(x')=\phi(x)$ and therefore the field equation can be simply be written as
\begin{equation}
(\Box_g-m^2) \phi =0.
\end{equation}

However, this is not the most general kind of field one could have considered. This equation is a particular case of the Euler-Lagrange equations coming from the more general Klein-Gordon Langrangian density 
\begin{equation}
\mathcal{L}=\frac12\sqrt{\left|g\right|}\left(g^{\mu\nu}\partial_\mu\phi \partial_\nu \phi +m^2\phi^2-\xi R\phi^2\right),
\end{equation}
where the dimensionless constant $\xi$ couples the field to the scalar curvature. The more general Klein-Gordon equation is of the form 
\begin{equation}
(\Box_g-m^2+\xi R) \phi =0.
\end{equation}
The inclusion of this extra term, coupling the field with the curvature, is often included in the Lagrangian as a counter-term necessary to renormalise the theory when we include interaction terms such as $\sqrt{|g|}\lambda\phi^4$. In principle there is no physical reason to include such a term in the free case. Notice, as a curiosity, that for 4 dimensions and when $m=0$ if we chose $\xi=1/6$ (conformal coupling) the field equations are invariant under conformal transformations\footnote{The conformally related actions differ only by a surface term}. The case $\xi=0$ is known as `minimal coupling', and it is the case we adhere to. 

Notice also that there are physical reasons to induce that the constant $\xi$ cannot be large because if $\xi\neq0$ then the Lagrangian being proportional to $R\phi^2$  would cause the effective gravitational constant to vary with time and position as a result of the variations in $\phi$. In any case, the introduction of such term would act as an effective mass term (although dependent on spacetime). To describe all the casuistics we study here this is irrelevant.

\section{Dirac equation in curved spacetimes}

Let us define the flat spacetime Dirac matrices $\{\gamma^0,\gamma^1,\gamma^2,\gamma^3\}$
which have the following properties
\begin{equation}
\{\gamma^a,\gamma^b\}=2\eta^{ab},
\end{equation}
where $\eta^{ab}$ is the usual flat Minkowskian metric. Then the Dirac equation in Minkowski spacetime can be written as
\begin{equation}\label{Dirac3}
(i\gamma^{a}\partial_{a}+m)\psi=0,
\end{equation}

To write the Dirac equation in curved spacetimes we need to introduce the vierbein. Namely, an orthonormal set of four vector fields that serve as a local reference frame of the tangent Lorentzian manifold in each point of spacetime such that
\begin{equation}
g^{\mu\nu}=e^\mu_ae^\nu_b\eta^{ab}
\end{equation}
The vierbein enables us to convert local Lorentz indices to general indices.

With the help of the vierbein we can write the Dirac matrices $\gamma^\mu$ in a general spacetime as a function of the local gamma matrices
\begin{equation}
\gamma^\mu=e^\mu_a\gamma^a.
\end{equation}

 This curved spacetime gamma matrices fulfil
 \begin{equation}
\gamma^{\mu}\gamma^{\nu}+\gamma^{\nu}\gamma^{\mu}=2g^{\mu\nu}.
 \end{equation}

Now we have to be careful when defining the covariant derivative: we have a spinor bundle defined over the spacetime manifold. The spin connection can be expressed in terms of the Levi-Civita connection $\hat\Gamma^\nu_{\sigma\mu}$ as
\begin{equation}
\omega_\mu^{ab}=e^a_\nu\partial_\mu e^{\nu b}+ e^{a}_\nu e^{\sigma b}\hat\Gamma^{\nu}_{\sigma\mu}
\end{equation}
Now we want to define the covariant derivative that satisfies
\begin{equation}
D_{[\mu}e^a_{\nu]}=\partial_{[\mu}e^a_{\nu]}+\omega^a_{b[\mu}e^{b}_{\nu]}=0
\end{equation}
with that covariant derivative we write the Dirac equation as
\begin{equation}(i\gamma^{\mu}D_\mu+m)\psi=0\end{equation}
or explicitly
\begin{equation}
[i\gamma^{\mu}\left(\partial_\mu+\Gamma_\mu\right)+m]\psi=0
\end{equation}
where
\begin{equation}
\Gamma_\mu=\frac14\omega^{ab}_\mu[\gamma_{a},\gamma_b].
\end{equation}

\section{Grassmann fields}

The so-called Grassmann scalar field  has been very often considered in relativistic quantum information literature . This is a fermionic field
with no internal degrees of freedom that captures the essence of fermionic entanglement behaviour in general relativistic scenarios. This kind of field has been extremely useful to study the general features of entanglement in fermionic fields. This is because it is the simplest field that capture all the characteristics of the entanglement behaviour of fermionic field. However, as seen in this thesis, there is a universality principle that guarantees that all the results for Grassmann fields are exportable to Dirac fields which Grassmannian analogs.

However, such field is not free from problems when trying to trace it back to a Lorentz-covariant field theory. This is so because the only covariant equation of motion for scalar fields is the Klein-Gordon equation which, when quantised, results in bosonic statistics. This fact might result somehow confusing when one reflects about its use and physical meaning in relativistic quantum information settings.

There are, however, some physical scenarios in which Grassmann fields have complete physical meaning. Maybe the most obvious is considering quantisation of a Dirac field in one spatial dimension. The Lorentz group representation $SO(1,1)$ consists of only one boost and have no rotations. In this context  there are no internal degrees of freedom and the resulting theory is a Grassmann scalar field. This is particularly relevant for analog
gravity and quantum simulation implementations on experimental setups such as trapped ions \cite{SimulJuan}.


Indeed, a Dirac field in 1+1 dimensions  is naturally spinless, a possible gamma matrices representation in this case takes the form
\begin{equation}
\gamma^0=\sigma_x\qquad\gamma^1=-i\sigma_y
\end{equation}
where $\sigma_i$ are the usual Pauli matrices and the Dirac equation takes the usual form
\begin{equation}\label{Dirac3}
(i\gamma^{a}\partial_{a}+m)\psi=0,
\end{equation}
where now $\psi$ is a two component spinor, one for the positive and one for negative energy solutions branch. 

There are some other scenarios in which the use of Grassmann fields can be mathematically acceptable and a useful tool: we could regard the Grassmann scalar field as a Dirac field with a fixed spin-z component. Thomas precession prevents this setup from being Lorentz-covariant,
but if we choose the acceleration to be in the spin quantisation direction this phenomenon does not occur.




\cleardoublepage