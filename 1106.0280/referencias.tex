\cleardoublepage\addcontentsline{toc}{chapter}{Bibliografía}
%\chapter*{Referencias}

%============================================================
\begin{thebibliography}{999}
%============================================================
%References
%\newpage
%-------------------------------------------------------------

%referencias introducción

%GR

\bibitem{Eins1} A. Einstein, Die Feldgleichungen der Gravitation (The Field Equations of
Gravitation), Sitzungsber. Preuss. Akad. Wiss. Berlin (Math.Phys.), 844 (1915).

\bibitem{Eins2} A. Einstein, Die Grundlage der
Allgemeinen Relativitätstheorie (The Foundation of the General Theory of Relativity), Annalen der
Physik {\bf49}, 769 (1916).

\bibitem{HawEll} S. W. Hawking y G. F. R. Ellis, {\it The Large Scale Structure of Space-Time}
(Cambridge University Press, Cambridge, Inglaterra, 1973).

\bibitem{carlip} Véase, p. ej., S. Carlip, Quantum gravity: A Progress Report,
Rept. Prog. Phys. {\bf64}, 885 (2001).

%loop quantum gravity general
\bibitem{lqg1} T. Thiemann, {\it{Modern Canonical Quantum General Relativity}} (Cambridge University
Press, Cambridge, Inglaterra, 2007).

\bibitem{lqg2} C. Rovelli, {\it{Quantum Gravity}} (Cambridge University Press, Cambridge,
Inglaterra,
2004).

\bibitem{lqg3} A. Ashtekar y J. Lewandowski, Background Independent Quantum Gravity: A Status
Report, Classical Quantum Gravity {\bf 21}, R53 (2004).

\bibitem{lqg4} H. Sahlmann, Loop Quantum Gravity - A Short Review, \texttt{arXiv:1001.4188}. 

\bibitem{dirac} P. A. M. Dirac, {\it Lectures on Quantum Mechanics} (Belfer
Graduate School of Science, Yeshiva University, Nueva York, 1964).

\bibitem{adm1} R. Arnowitt, S. Deser y  C. W. Misner, The Dynamics of General Relativity,
en {\it Gravitation: An Introduction to Current Research}, editado por L. Witten (Wiley, Nueva
York, 1962).

\bibitem{wald} R. M. Wald, {\it General Relativity} (University of Chicago Press, Chicago, 1984).

\bibitem{kuc-ish1} C. J. Isham y K. V. Kucha\v{r}, Representations of Spacetime Diffeomorphisms. I.
Canonical Parametrized Field Theories, Ann. Phys. {\bf164}, 288 (1985).

\bibitem{kuc-ish2} C. J. Isham y K. V. Kucha\v{r}, Representations of Spacetime Diffeomorphisms. II.
Canonical Geometrodynamics, Ann. Phys. {\bf164}, 316 (1985).
%loop quantum cosmology general
\bibitem{lqc1} M. Bojowald, Loop Quantum Cosmology, Living Rev. Rel. {\bf 11}, 4 (2008).

\bibitem{lqc2a} A. Ashtekar, An Introduction to Loop Quantum Gravity through
Cosmology, Nuovo Cim. {\bf122B}, 135 (2007).

\bibitem{lqc2b} A. Ashtekar, Loop Quantum Cosmology: An Overview,
Gen. Rel. Grav. {\bf41}, 707 (2009).

\bibitem{lqc3} G. A. Mena Marug\'an, A Brief Introduction to Loop Quantum Cosmology, AIP Conf.
Proceedings {\bf 1130}, 89 (2009).

%wdw 
\bibitem{witt} B. S. DeWitt, Quantum Theory of Gravity I. The Canonical Theory, Phys. Rev.
{\bf160}, 1113 (1967).

\bibitem{mis-a} C. W. Misner, Quantum Cosmology I, Phys. Rev. {\bf 186}, 1319 (1969).
\bibitem{mis-b} C. W. Misner, Mixmaster Universe, Phys. Rev. Lett. {\bf22}, 1071 (1969).
\bibitem{mis-c} C. W. Misner, Minisuperspace, in {\it Magic Without Magic: John Archibald Wheeler}
(W. H. Freeman, San Francisco, 1972).

\bibitem{hartle} J. B. Hartle, Quantum Cosmology, en {\it High Energy Physics 1985: Proceedings of
the Yale Summer School}, editado por M. J. Bowick y F. Gursey (World Scientific, Singapur, 1985).

\bibitem{hall} Para una guía de la bibliografía sobre cuantización canónica de minisuperespacios,
véase J. J. Halliwell, A Bibliography of Papers on Quantum Cosmology, Int. J. Mod. Phys. A {\bf5},
2473 (1990).

\bibitem{whe} J. A. Wheeler, Superspace and the Nature of Quantum Geometrodynamics, in {\it Batelle
Rencontres}, editado por C. M. DeWitt y J. A. Wheeler (W. A. Benjamin, Nueva York, 1972).

%lqc pionera bojowald
\bibitem{boj1a} M. Bojowald, Loop Quantum Cosmology I: Kinematics, Classical Quantum Gravity
{\bf17}, 1489 (2000).

\bibitem{boj1b} M. Bojowald, Loop Quantum Cosmology II: Volume Operators, Classical Quantum Gravity
{\bf17}, 1509 (2000).

\bibitem{boj1c} M. Bojowald, Loop Quantum Cosmology III: Wheeler-DeWitt Operators; Classical Quantum
Gravity {\bf18} (2001).

\bibitem{boj1d} M. Bojowald,Loop Quantum Cosmology IV: Discrete Time Evolution; Classical Quantum
Gravity {\bf18}, 1701 (2001). 

\bibitem{boj2} M. Bojowald, Absence of Singularity in Loop Quantum Cosmology,
Phys. Rev. Lett. {\bf86}, 5227 (2001).

%lqc isotropa principal
\bibitem{abl} A. Ashtekar, M. Bojowald y J. Lewandowski,  Mathematical Structure of Loop Quantum
Cosmology, Adv. Theor. Math. Phys. {\bf 7}, 233 (2003).

\bibitem{aps1a} A. Ashtekar, T. Paw{\l}owski y P. Singh, Quantum Nature of the Big Bang, Phys.
Rev. Lett. {\bf 96}, 141301 (2006).

\bibitem{aps1b}  A. Ashtekar, T. Paw{\l}owski y P. Singh, Quantum Nature of the Big Bang: An
Analytical and Numerical
Investigation. I, Phys. Rev. D {\bf 73}, 124038 (2006).

\bibitem{aps3} A. Ashtekar, T. Paw{\l}owski y P. Singh, Quantum Nature of the Big Bang: Improved
Dynamics, Phys. Rev. D {\bf 74}, 84003 (2006).

\bibitem{apsv} A. Ashtekar, T. Paw{\l}owski, P. Singh y K. Vandersloot, Loop Quantum Cosmology of
k=1 FRW Models, Phys. Rev. D {\bf 75}, 024035 (2007).

\bibitem{skl} L. Szulc, W. Kami\'nski y J. Lewandowski, Closed FRW Model in Loop Quantum
Cosmology, Classical Quantum Gravity {\bf 24}, 2621 (2007).

\bibitem{vand} K. Vandersloot, Loop Quantum Cosmology and the k = - 1 RW Model, Phys. Rev. D {\bf
75}, 23523 (2007).

\bibitem{luc} L. Szulc, Open FRW Model in Loop Quantum Cosmology, Classical Quantum Gravity {\bf24},
6191
(2007).

\bibitem{tom} E. Bentivegna y T. Paw{\l}owski,  Anti-deSitter Universe Dynamics in LQC, 
Phys. Rev. D {\bf 77}, 124025 (2008).

\bibitem{tom2} W. Kami\'nski y T. Paw{\l}owski, The LQC Evolution Operator of FRW Universe with
Positive
Cosmological Constant, Phys. Rev. D {\bf81}, 024014 (2010).

\bibitem{victor}
V.~Taveras, Corrections to the Friedmann Equations from LQG for a Universe with a Free Scalar Field,
Phys. Rev. D {\bf 78},064072 (2008).

\bibitem{sv-eff}
P.~Singh y K.~Vandersloot, Semi-classical States, Effective Dynamics and Classical Emergence in
Loop Quantum Cosmology, Phys. Rev. D {\bf 72}, 084004 (2005).

\bibitem{cop1} E. J. Copeland, D. J. Mulryne, N. J. Nunes y M. Shaeri, Super-Inflation in Loop
Quantum Cosmology, Phys. Rev. D {\bf77}, 023510 (2008).


\bibitem{cop2} E. J. Copeland, D. J. Mulryne, N. J. Nunes y M. Shaeri, The Gravitational Wave
Background from Super-Inflation in Loop Quantum Cosmology, Phys. Rev. D {\bf79}, 023508 (2009).


\bibitem{grain1} J.~Grain y A.~Barrau, Cosmological Footprints of Loop Quantum
Gravity, Phys. Rev. Lett. {\bf 102}, 081301 (2009).

\bibitem{grain2} J.~Grain, Loop Quantum Cosmology Corrections on Gravity Waves Produced during
Primordial Inflation, \texttt{arXiv:0911.1625}.


%gowdy principal

\bibitem{kuchar2} K. Kucha\v{r}, Canonical
Quantization of Cylindrical Gravitational Waves, Phys. Rev. D {\bf}4, 955 (1971).

\bibitem{gowdy1} R. H.~Gowdy, {Gravitational Waves in Closed Universes},
Phys. Rev. Lett. \textbf{27}, 826 (1971).

\bibitem{gowdy2} R. H. Gowdy, Vacuum Spacetimes with Two-parameter Spacelike Isometry Groups and
Compact Invariant Hypersurfaces: Topologies and Boundary Conditions, Ann. Phys. {\bf83}, 203 (1974).

\bibitem{mon1} V. Moncrief, Global Properties of Gowdy Spacetimes with $T^3\times R$ Topology, Ann.
Phys. {\bf 132}, 87 (1981).

\bibitem{mon2} V. Moncrief, Infinite-dimensional Family of Vacuum Cosmological Models with Taub-NUT
(Newman-Unti-Tamburino)-Type Extensions, Phys. Rev. D {\bf23}, 312 (1981).

\bibitem{ise} J. Isenberg y V. Moncrief, Asymptotic Behavior of the Gravitational Field and the
Nature of Singularities in Gowdy Spacetimes, Ann. Phys. {\bf 199}, 84 (1990).

\bibitem{men1a}
A. Corichi, J. Cortez y G. A. Mena Marug\'{a}n, Unitary Evolution in Gowdy
Cosmology, Phys. Rev. D {\bf73}, 041502 (2006).

\bibitem{men1b}
A. Corichi, J. Cortez y G. A. Mena Marug\'{a}n, Quantum Gowdy $T^3$ Model: A Unitary Description,
Phys. Rev. D {\bf73}, 084020 (2006).

\bibitem{men2} A. Corichi, J. Cortez, G. A. Mena Marug\'{a}n y J. M. Velhinho, Quantum Gowdy
$T^3$ Model: A Uniqueness Result, Classical Quantum Gravity {\bf23}, 6301 (2006).

\bibitem{men3} J. Cortez, G. A. Mena Marug\'{a}n y J. M. Velhinho, Uniqueness of the Fock
Quantization of the Gowdy $T^3$ Model,  Phys. Rev. D {\bf75}, 084027 (2007).

%bianchi I principal y anisótropos y agujeros negros

\bibitem{boj} M. Bojowald, Homogeneous Loop Quantum Cosmology, Classical Quantum Gravity {\bf 20},
2595 (2003).

\bibitem{chio} D. W. Chiou, Loop Quantum Cosmology in Bianchi Type I Models: Analytical
Investigation, Phys. Rev. D {\bf 75}, 024029 (2007).

\bibitem{awe} A. Ashtekar y E. Wilson-Ewing, Loop Quantum Cosmology of Bianchi Type I Models,
Phys. Rev. D {\bf 79}, 083535 (2009).

\bibitem{bdv} M.~Bojowald, G.~Date y K.~Vandersloot, {Homogeneous
Loop Quantum Cosmology: The Role of the Spin Connection}, Classical
Quantum Gravity \textbf{21}, 1253 (2004).

\bibitem{awe2} A.~Ashtekar y E.~Wilson-Ewing, {Loop Quantum
Cosmology of Bianchi Type II Models}, Phys. Rev. D \textbf{80},
123532 (2009).

\bibitem{bdh} M.~Bojowald, G.~Date y G.~M.~Hossain, {The Bianchi
IX Model in Loop Quantum Cosmology}, Classical
Quantum Gravity \textbf{21}, 3541 (2004).

\bibitem{we} E.~Wilson-Ewing, {Loop Quantum Cosmology of Bianchi Type
IX Models}, \texttt{arXiv:1005.5565}.

\bibitem{boj-bh1} M. Bojowald, Spherically Symmetric Quantum Geometry: States and Basic
Operators, Classical Quantum Gravity {\bf21}, 3733 (2004). 

\bibitem{boj-bh2} M. Bojowald y R. Swiderski, The Volume Operator in Spherically Symmetric Quantum
Geometry, Classical Quantum Gravity {\bf 21}, 4881 (2004).

\bibitem{boj-bh3} M. Bojowald, Nonsingular Black Holes and Degrees of Freedom in Quantum
Gravity, Phys. Rev. Lett. {\bf95}, 061301 (2005).

\bibitem{boj-bh4} A. Ashtekar y M. Bojowald, Black hole evaporation: A paradigm,
Classical Quantum Gravity {\bf22}, 3349 (2005). 

\bibitem{boj-bh5} A. Ashtekar y M. Bojowald, Quantum geometry and the Schwarzschild
singularity, Classical Quantum Gravity {\bf23}, 391 (2006).

\bibitem{boj-bh6} M. Bojowald y R. Swiderski, Spherically Symmetric Quantum Geometry: Hamiltonian
Constraint, Classical Quantum Gravity {\bf 23}, 2129 (2006).

\bibitem{pullin1} M. Campiglia, R. Gambini y J. Pullin, Loop Quantization of Spherically Symmetric
Midi-Superspaces, Classical Quantum Gravity {\bf24}, 3649 (2007).

\bibitem{pullin2} R. Gambini y J. Pullin, Black Holes in Loop Quantum Gravity: the Complete
Space-Time, Phys. Rev. Lett. {\bf101}, 161301 (2008).



%%%%%%%%%%%%%%%%%%%%%%%%%%%%%%%%%%%%
%%%%%%%%%%%%%%%%%%%%%%%%%%%%%%%%%%%%
% referencias 01-introLQC

\bibitem{adm2} C. W. Misner, K. S. Thorne y J. A. Wheeler, {\it Gravitation} (W. H. Freeman, San
Francisco, 1973).

\bibitem{lqg0aa}  A. Ashtekar, New Variables for Classical and Quantum Gravity, Phys. Rev. Lett.
{\bf57}, 2244 (1986).

\bibitem{lqg0ab}  A. Ashtekar, New Hamiltonian Formulation of General Relativity,  Phys. Rev. D
{\bf36}, 1587 (1987).

\bibitem{lqg0ba} C. Rovelli y L. Smolin, Knot Theory and Quantum Gravity, Phys. Rev. Lett. {\bf61},
1155 (1988).

\bibitem{lqg0bb} C. Rovelli y L. Smolin, Loop Representation for Quantum General Relativity, Nucl.
Phys. B {\bf331}, 80 (1990).

\bibitem{lqg0c} A. Ashtekar, {\it Lectures on Non-Perturbative Canonical Gravity} (World Scientific,
Singapur, 1991).

\bibitem{spin} G. A. Mena Marug\'{a}n, Reality Conditions for Lorentzian and Euclidean Gravity in
the Ashtekar Formulation, Int. J. Mod. Phys. D {\bf3}, 513 (1994).

\bibitem{bar} J. F. Barbero G.,  Real Ashtekar Variables for Lorentzian Signature Space Times,
Phys. Rev. D {\bf51}, 5507 (1995). 

\bibitem{gior1} G. Immirzi, Quantum Gravity and Regge Calculus, Nucl. Phys. B (Proc. Suppl.)
{\bf 57}, 65 (1997).

\bibitem{gior2} G. Immirzi, Real and Complex Connections for Canonical Gravity, Classical Quantum
Gravity {\bf14}, L177 (1997).


\bibitem{ALmeasure1} A. Ashtekar y C.J. Isham, Representations of the Holonomy Algebras of Gravity
and Non-Abelian Gauge Theories, Classical Quantum Gravity {\bf9}, 1433 (1992).

\bibitem{ALmeasure2} A. Ashtekar y J. Lewandowski, Representation Theory of Analytic Holonomy
$C^*$-algebras, en {\it Knots and Quantum Gravity}, editado por J.C Baez (Oxford University Press,
Oxford, 1994). 

\bibitem{baez} J. C. Baez, Generalized Measures in Gauge Theory, Lett. Math. Phys. {\bf31}, 213
(1994).

\bibitem{ALmeasure3} A. Ashtekar y J. Lewandowski, Projective Techniques and Functional Integration
for Gauge Theories, J. Math. Phys. {\bf36}, 2170 (1995).

\bibitem{fleis} C. Fleischhack, Proof of a Conjecture by Lewandowski and Thiemann, Commun. Math.
Phys. {\bf249}, 331 (2004).

\bibitem{lost1} H. Sahlmann y T. Thiemann, Irreducibility of the Ashtekar-Isham-Lewandowski
Representation, Classical Quantum Gravity {\bf23}, 4453 (2006). 

\bibitem{lost2} J. Lewandowski, A. Okolow, H. Sahlmann y T. Thiemann,
Uniqueness of Diffeomorphism Invariant States on Holonomy-flux Algebras,
Commun. Math. Phys. {\bf267}, 703 (2006).

\bibitem{Vel} J. M. Velhinho, The Quantum Configuration Space of Loop Quantum Cosmology, Classical
Quantum Gravity \textbf{24}, 3745 (2007).

\bibitem{SvN1} M. H. Stone,  Linear Transformations in Hilbert Space, III:
Operational Methods and Group Theory, Proc. Nat. Acad. Sci. U.S.A. {\bf16}, 172
(1930).

\bibitem{SvN2} J. von Neumann, Die Eindeutigkeit der Schr\"odingerschen
Operatoren (The Uniqueness of the Schr\"odinger Operators), Math. Ann. {\bf104}, 570 (1931).

\bibitem{area1} C. Rovelli y L. Smolin, Discreteness of Area and Volume in Quantum Gravity, Nucl.
phys. B {\bf442}, 593 (1995); Erratum, Nucl. Phys. B {\bf456}, 753 (1995).

\bibitem{area2} A. Ashtekar y J. Lewandowski, Quantum Theory of Geometry. 1: Area Operators,
Classical Quantum Gravity {\bf14}, A55 (1997).

\bibitem{cs} A. Corichi y P. Singh, Is Loop Quantization in Cosmology Unique?, Phys. Rev.
D {\bf 78}, 024034 (2008).

\bibitem{inv-vol-lqg1} T. Thiemann, Anomaly-Free Formulation of Non-Perturbative Four-Dimensional
Lorentzian Quantum Gravity, Phys. Lett. {\bf B380}, 257 (1996).

\bibitem{inv-vol-lqg2} T. Thiemann, Quantum Spin Dynamics (QSD),
Classical Quantum Gravity {\bf15:}, 839 (1998).

\bibitem{gave1a} D. Marolf, Refined Algebraic Quantization: Systems with a Single Constraint,
\texttt{arXiv:gr-qc/9508015}.

\bibitem{gave1b} D. Marolf, Quantum Observables and Recollapsing Dynamics,
Classical Quantum Gravity {\bf 12}, 1199 (1995).

\bibitem{gave1c} D. Marolf, Observables and a Hilbert Space for Bianchi IX,
Classical Quantum Gravity {\bf 12}, 1441 (1995).

\bibitem{gave1d} D. Marolf, Almost Ideal Clocks in Quantum Cosmology: A Brief
Derivation of Time, Classical Quantum Gravity {\bf 12}, 2469, (1995).

\bibitem{gave2} A. Ashtekar, J. Lewandowski, D. Marolf, J. Mour\~ao y T. Thiemann, Quantization
of Diffeomorphism Invariant Theories of Connections with Local Degrees of Freedom, J. Math. Phys.
{\bf 36}, 6456 (1995).

\bibitem{kale} W. Kami\'nski y J. Lewandowski, The Flat FRW Model in LQC: The Self-Adjointness,
Classical Quantum Gravity {\bf 25}, 35001 (2008).



%%%%%%%%%%%%%%%%%%%%%5
%%%%%%%%%%%%%%%%%%%%%
%referencias 02-introGowdy

%ondas de Einstein-Rosen

%\bibitem{ERwaves} A. Einstein y N. Rosen, On Gravitational waves, J. Franklin Inst.
%{\bf223}, 43 (1937). 

\bibitem{misner} C. W. Misner, A Minisuperspace Example: The Gowdy $T^3$ Cosmology, Phys. Rev. D
{\bf8}, 3271 (1973)

\bibitem{berger1} B. K. Berger, Quantum Graviton Creation in a Model Universe, Ann. Phys. {\bf83},
458 (1974).

\bibitem{berger2} B. K. Berger, Quantum Cosmology: Exact Solution for the Gowdy $T^3$ Model, Phys.
Rev. D {\bf 11}, 2770 (1975).

\bibitem{berger3} B. K. Berger, Quantum Effects in the Gowdy $T^3$ Cosmology, Ann. Phys. {\bf 156},
155 (1984).

\bibitem{pierri} M. Pierri, Probing Quantum General Relativity through Exactly
Soluble Midi-Superspaces. II: Polarized Gowdy models, Int. J. Mod. Phys. D {\bf11}, 135
(2002).

\bibitem{ccq} A. Corichi, J. Cortez y H. Quevedo, On Unitary Time Evolution in Gowdy $T^3$
Cosmologies, Int. J. Mod. Phys. D {\bf11}, 1451 (2002). 

\bibitem{torre} C. G. Torre, Quantum Dynamics of the Polarized Gowdy $T^3$ Model, Phys. Rev. D
{\bf66}, 084017 (2002).

\bibitem{men0}  J. Cortez y G. A. Mena Marug\'{a}n, Feasibility of a Unitary Quantum Dynamics in
the Gowdy $T^3$ Cosmological Model, Phys. Rev. D {\bf72}, 064020 (2005).

\bibitem{dani1} J. Fernando Barbero G., D. Gómez Vergel, E. J. S. Villaseñor, Evolution
Operators for Linearly Polarized Two-Killing Cosmological Models, Phys. Rev. D {\bf74}, 024003
(2006).

\bibitem{men4} A. Corichi, J. Cortez, G. A. Mena Marug\'{a}n y J. M. Velhinho, Quantum Gowdy $T^3$
Model: Schr\"odinger Representation with Unitary Dynamics, Classical Quantum Gravity {\bf23}, 6301
(2006).

\bibitem{men5} J. Cortez, G. A. Mena Marugán y J. M. Velhinho,
Uniqueness of the Fock Representation of the Gowdy $S^1\times S^2$ and $S^3$ Models,
Classical Quantum Gravity {\bf25}, 105005 (2008).

\bibitem{dani2} J. F. Barbero G., D. Gómez Vergel y E. J. S. Villaseñor, Quantum Unitary Evolution
of Linearly Polarized $S^1\times S^2$ and $S^3$ Gowdy Models Coupled to Massless Scalar Fields,
Classical Quantum Gravity {\bf25}, 085002 (2008).

\bibitem{men6} J. Cortez, G. A. Mena Marugán, R. Serôdio y J. M. Velhinho,
Uniqueness of the Fock Quantization of a Free Scalar Field on $S^1$ with Time Dependent Mass,
Phys. Rev. D {\bf79}, 084040 (2009).

\bibitem{men7} J. Cortez, G. A. Mena Marugán y J. M. Velhinho,
Fock Quantization of a Scalar Field with Time Dependent Mass on the Three-Sphere: Unitarity and
Uniqueness, Phys. Rev. D {\bf81}, 044037 (2010).

\bibitem{men8} J. Cortez, G. A. Mena Marugan, J. Olmedo y J. M. Velhinho,
A Unique Fock Quantization for Fields in Non-Stationary Spacetimes, \texttt{arXiv:1004.5320}.


\bibitem{man} G. A. Mena Marug\'{a}n y M. Montejo, Quantization of Pure Gravitational Plane
Waves, Phys. Rev. D {\bf58}, 104017 (1998).

\bibitem{wald2} R. M. Wald, {\it Quantum Field Theory in Curved Spacetime and Black Hole
Thermodynamics} (University of Chicago Press, Chicago, 1994). 

%%%%%%%%%%%%%%%%%%%%%%%%%%%%%%%%5
%%%%%%%%%%%%%%%%%%%%%%%%%%%%%%%%
%referencias 3-FRW

\bibitem{Pol} W. Kami\'nski, J. Lewandowski y T. Paw{\l}owski, Physical Time and other Conceptual
Issues of QG on the Example of LQC, Classical Quantum Gravity {\bf 26},
035012 (2009).


\bibitem{mmo} M. Mart\'{\i}n-Benito, G. A. Mena Marug\'{a}n y J. Olmedo, 
Further Improvements in the Understanding of Isotropic Loop Quantum Cosmology, Phys. Rev. D {\bf
80}, 104015 (2009).

\bibitem{acs}A. Ashtekar, A. Corichi y P. Singh, On the Robustness of Key Features of Loop
Quantum Cosmology, Phys. Rev. D {\bf 77}, 024046 (2008).

\bibitem{chinos} Y. Ding, Y. Ma y J. Yang, Effective Scenario of Loop Quantum Cosmology, Phys. Rev.
Lett {\bf 102} 051301 (2009).

\bibitem{red1} A. D. Rendall, Unique Determination of an Inner Product by Adjointness Relations in
the Algebra of Quantum Observables, Classical Quantum Gravity {\bf 10}, 2261 (1993).

\bibitem{red2} A. D. Rendall, Adjointness Relations as a Criterion for Choosing an Inner Product, 
\texttt{arXiv:gr-qc/9403001}.

\bibitem{kp-posL} W. Kami\'nski y T. Paw{\l}owski, Cosmic Recall and the Scattering
Picture of Loop Quantum Cosmology, Phys. Rev. D. {\bf81}, 084027 (2010).

\bibitem{haw1} S. W. Hawking, The Boundary Conditions of the Universe, Pont. Acad. Sci. Varia
{\bf48}, 563 (1982).

\bibitem{haw2} J. B. Hartle y S. W. Hawking, Wave Function of the Universe, Phys. Rev. D {\bf
28}, 2960 (1983).

\bibitem{haw3} S. W. Hawking, The Quantum State of the Universe, Nucl. Phys. B
{\bf239}, 257 (1984).

\bibitem{cs2} A. Corichi y P. Singh, Quantum Bounce and Cosmic Recall, Phys. Rev. Lett. {\bf
100}, 161302 (2008).


%%%%%%%%%%%%%%%%%%%%
%%%%%%%%%%%%%%%%%%%%
%referencia 04-Bianchi-kin

\bibitem{bianchi} L. Bianchi, Sugli Spazii a Tre Dimensioni che Ammettono un Gruppo Continuo di
Movimenti (On the Spaces of Three Dimensions that Admit a Continuous Group of Movements), Soc.
Ital. Sci. Mem. di Mat. {\bf11}, 267 (1897).

\bibitem{kramer} D. Kramer, H. Stephani, M. MacCallum y E. Herlt, {\it Exact Solutions of Einstein
Field Equations} (Cambridge University Press, Cambridge, Inglaterra, 1980).

\bibitem{aspu} A. Ashtekar y J. Pullin, Bianchi Cosmologies: A New Description, Ann. Israel
Phys. Soc. {\bf 9}, 65 (1990).

\bibitem{neg2} N. Manojlovi\'c  y A. Mikovic, Canonical Analysis of the Bianchi Models in an
Ashtekar Formulation, Classical Quantum Gravity {\bf10}, 559 (1993).

\bibitem{neg1} N. Manojlovi\'{c} y G. A. Mena Marug\'{a}n, Nonperturbative Canonical Quantization of
Minisuperspace Models: Bianchi types I and II, Phys. Rev. D {\bf 48}, 3704 (1993).


\bibitem{chi1} D. W. Chiou y K. Vandersloot, The Behavior of Non-Linear Anisotropies in Bouncing
Bianchi I Models of Loop Quantum Cosmology, Phys. Rev. D {\bf76}, 084015 (2007).

\bibitem{chi2}
D. W. Chiou, Effective Dynamics, Big Bounces and Scaling Symmetry in Bianchi Type I Loop Quantum
Cosmology, Phys. Rev. D {\bf 76}, 124037 (2007).


\bibitem{mmp} M. Mart\'{\i}n-Benito, G. A. Mena Marug\'{a}n y T. Paw{\l}owski, Loop Quantization of
Vacuum Bianchi I Cosmology, Phys. Rev. D {\bf 78}, 064008 (2008).

\bibitem{luc2} L. Szulc, Loop Quantum Cosmology of Diagonal Bianchi Type I model: Simplifications
and Scaling Problems, Phys. Rev. D {\bf78}, 064035 (2008).

\bibitem{mmw} M. Mart\'{\i}n-Benito, G. A. Mena Marug\'{a}n y E. Wilson-Ewing, Hybrid Quantization:
from Bianchi I to the Gowdy Model, \texttt{arXiv:1006.2369}.

\bibitem{kasner} E. Kasner, Geometrical Theorems on Einstein's Cosmological Equations, Amer. J.
Math. {\bf43}, 217 (1921).

%\bibitem{solutions} D. Kramer, H. Stephani, E. Herlt y M. MacCallum, {\it{Exact solutions of
%Einstein's Field Equations}} (Cambridge University Press, Cambridge, Inglaterra, 1980).

\bibitem{cs3} A. Corichi y P. Singh, A Geometric Perspective on Singularity Resolution and
Uniqueness in Loop Quantum Cosmology, Phys. Rev. D {\bf80},044024 (2009).

\bibitem{Vel2} J. M. Velhinho, Comments on the Kinematical Structure of Loop Quantum Cosmology, 
Classical Quantum Gravity {\bf21}, L109 (2004). 


%%%%%%%%%%%%%%%%%%%
%%%%%%%%%%%%%%%%%%
%Referencias evolution

\bibitem{front} S. W. Hawking y C. J. Hunter, The Gravitational Hamiltonian in the Presence of
Non-Orthogonal Boundaries, Classical Quantum Gravity {\bf13}, 2735 (1996).

\bibitem{rovelli-obs1} C. Rovelli, What is Observable in Classical and Quantum Gravity?, Classical
Quantum Gravity {\bf8}, 297 (1991).

\bibitem{rovelli-obs2} C. Rovelli, Partial Observables, Phys. Rev. D {\bf65}, 124013
(2002).

\bibitem{rovelli-obs3} C. Rovelli, Forget time, \texttt{arXiv:0903.3832}.

\bibitem{bianca-obs1} B. Dittrich, Partial and Complete Observables for Canonical General
Relativity,
Classical Quantum Gravity {\bf23}, 6155 (2006).

\bibitem{bianca-obs2} B. Dittrich, Partial and Complete Observables for Hamiltonian
Constrained Systems, Gen. Relativ. Gravit. {\bf39}, 1891 (2007).

 \bibitem{thiemann-obs} T. Thiemann, Reduced Phase Space Quantization and Dirac Observables,
Classical Quantum Gravity {\bf23}, 1163 (2006).

\bibitem{thiemann-obs2} K. Giesel y T. Thiemann, Algebraic Quantum Gravity (AQG) IV. Reduced Phase
Space Quantisation of Loop Quantum Gravity, \texttt{arXiv:0711.0119}.

\bibitem{david1} D. Brizuela, J. M. Martin-Garcia y G. A. Mena Marug\'an, High-Order
Gauge-Invariant Perturbations of a Spherical Spacetime, Phys. Rev. D {\bf76}, 024004 (2007).

\bibitem{david2} D. Brizuela, J. M. Martin-Garcia y M. Tiglio, A Complete Gauge-Invariant
Formalism for Arbitrary Second-Order Perturbations of a Schwarzschild Black Hole, Phys.
Rev. D {\bf80}, 024021 (2009).

\bibitem{mmp2} M. Mart\'{\i}n-Benito, G. A. Mena Marug\'{a}n y T. Paw{\l}owski, Physical Evolution
in Loop Quantum Cosmology: The Example of Vacuum Bianchi I,
Phys. Rev. D {\bf80}, 084038 (2009).

\bibitem{trap-rule} E. W. Weisstein, Trapezoidal Rule. From MathWorld---A Wolfram Web
Resource. http://mathworld.wolfram.com/TrapezoidalRule.html 

\bibitem{FFT} W. H. Press, B. P Flannery, S. A. Teukolsky y W. T. Vetterling, {\it{Numerical Recipes
in
C: The Art of Scientific Computing}} (Cambridge University Press, Cambridge, Inglaterra, 2007).

%%%%%%%%
%gowdy
\bibitem{gow-let} M. Mart\'{\i}n-Benito, L. J. Garay y G. A. Mena Marug\'{a}n, Hybrid Quantum Gowdy
Cosmology: Combining Loop and Fock Quantizations, Phys. Rev. D \textbf{78}, 083516 (2008).

\bibitem{gow-ijmp} G. A. Mena Marug\'{a}n y M. Mart\'{\i}n-Benito,  Hybrid Quantum Cosmology:
Combining Loop and Fock Quantizations, Int. J. Mod. Phys. A {\bf24}, 2820
(2009).

\bibitem{gow-B} L. J. Garay, M. Mart\'{\i}n-Benito y G. A. Mena Marug\'{a}n, Inhomogeneous Loop
Quantum Cosmology: Hybrid Quantization of the Gowdy Model, \texttt{arXiv:1005.5654}.

\bibitem{guille-gow} G. A. Mena Marug\'{a}n, Canonical Quantization of the Gowdy Model, Phys. Rev. 
D {\bf56}, 908 (1997).

\bibitem{date1} K. Banerjee y G. Date, Loop Quantization of Polarized Gowdy Model on $T^3$:
Classical Theory, Classical Quantum Gravity {\bf25}, 105014 (2008).

\bibitem{date2} K. Banerjee y G. Date, Loop Quantization of Polarized Gowdy Model on $T^3$:
Kinematical States and Constraint Operators, Classical Quantum Gravity {\bf25}, 145004 (2008).

\bibitem{rv} C. Rovelli y F. Vidotto, Stepping out of Homogeneity in
Loop Quantum Cosmology, Classical Quantum Gravity {\bf 25}, 225024 (2008).

\bibitem{bhks} M. Bojowald, G. M. Hossain, M. Kagan y S.~Shankaranarayanan, Anomaly Freedom
in Perturbative Loop Quantum Gravity, Phys. Rev. D {\bf 78}, 063547 (2008).

\bibitem{eff} D. Brizuela, G. A. Mena Marug\'an y T. Paw{\l}owski, Big Bounce and Inhomogeneities,
Classical Quantum Gravity {\bf27}, 052001 (2010).

\bibitem{mpw} G. A. Mena Marug\'an, T. Paw{\l}owski y E. Wilson-Ewing, An Effective Analysis of the
Hybrid Quantization of the Gowdy Model ({\it{en preparación}}).
%resolucion singularidades

%\bibitem{sing1} M. Bojowald, Singularities and quantum gravity,

%\bibitem{sing2} A. Ashtekar, Singularity Resolution in Loop Quantum Cosmology: A Brief Overview,
%J. Phys. Conf. Ser. {\bf189}, 012003 (2009). 

%%%%%%%%%%%%%%%%%%

%apendice operadores

\bibitem{functional1} M. Reed y B. Simon, {\it Methods of Modern Mathematical Physics
I: Functional Analysis} (Academic Press, San Diego, 1980).

\bibitem{functional2} A. Galindo y P. Pascual, {\it {Mecánica
Cuántica I}} (Eudema Universidad, Madrid, 1989); {\it{Quantum
Mechanics I}} (Springer-Verlag, Berlin, 1990).

\bibitem{kato} T. Kato, {\it Perturbation Theory for Linear
Operators} (Springer-Verlag, Berlín, 1980).

\bibitem{functional3} L. Abellanas y A. Galindo, {\it {Espacios de Hilbert}} (Eudema
Universidad, Madrid, 1987).


%%%%%%%%%%%%%%%
%%%%%%%%%%%%%%%%
%nuestra publicaciones




%============================================================
\end{thebibliography}
%============================================================
