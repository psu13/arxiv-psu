
%%%%%%%%%%%%%%%%%%%%%%%%%%%%%%%%%%%%%
{\renewcommand{\thechapter}{}\renewcommand{\chaptername}{}
\addtocounter{chapter}{0}
\chapter*{Introduction}\markboth{\sl Introduction}{\sl Introduction}}
%%%%%%%%%%%%%%%%%%%%%%%%%%%%%%%%%%%%%
\addcontentsline{toc}{chapter}{Introduction} 


\section*{Introduction and objectives}
\addcontentsline{toc}{section}{Introduction and objectives} 

General relativity is the theory that describes gravity which is currently accepted in the frame of modern physics. It fits all the phenomenology previously observed and successfully predicted a plethora of experimental results. Not only has it been experimentally tested a number of times but its application has been instrumental in leading to day-to-day modern technology as, for instance, the Global Positioning System (GPS). The theory basically consists of  a geometric description of gravity: mass and energy move in a curved spacetime and the spacetime is curved by the presence of mass and energy. However, it is far from being complete. General relativity allows ill-defined objects such as singularities, and in the presence of a singularity it loses its predictive power. These problems are strongly related with the classicality of the theory: general relativity is classical and close to a singularity the energies and distances involved reach the Planck scale. A quantum description of gravity is still to appear, being one of the most important (if not the most) challenges of modern theoretical physics. In the absence of a full quantum theory for gravity, quantum field theory in curved spacetimes, which describe the interaction of quantum fields with this classical (but relativistic) gravity, is the most complete theory so far.

Quantum information theory, on the other hand, deals with problems in information theory when the information is stored in and managed with quantum systems. Quantum mechanics allows us to carry out tasks that were considered impossible in the classical world: we can use quantum simulators to find solutions to quantum dynamical problems that would take too long for classical computers; we can store a large amount of information in quantum memories taking advantage of the superposition principle; we can implement completely secure communication using quantum key distribution protocols (quantum cryptography) and much more. Arguably, the most important of these achievements is being able to construct and implement quantum algorithms that transform quantum mechanical systems into quantum computers that can, for instance, factorise prime numbers in a time that grows polynomially with their lengths \cite{Shor} or find elements in a non-indexed list in a time that grows as the square root of the number of elements \cite{Grover}. Again, this is one of the challenges of modern physics: to tame the laws of quantum mechanics and use this new quantum physics knowledge to build new technology and solve problems which are practically unsolvable otherwise.

Despite their apparently separated application areas, general relativity and quantum information are not disjoint research fields. On the contrary, following the pioneering work of Alsing and Milburn \cite{Alsingtelep} a wealth of works considered different situations in which entanglement was studied in a general relativistic setting, for instance, quantum information tasks influenced by black holes  \cite{TeraUedaSpd,PanBlackHoles,ManSchullBlack,Adeschul}, entanglement in an expanding universe \cite{caball,Steeg} and entanglement with non-inertial partners \cite{Alicefalls,AlsingSchul,TeraUeda2,KBr}.

Even though many of
the systems used in the implementation of quantum information
involve relativistic systems such as photons, the vast majority of
investigations on entanglement assume that the Universe is flat and
non-relativistic. Understanding entanglement in general spacetimes is
ultimately necessary because the world is fundamentally
relativistic. Moreover, entanglement  plays a prominent
role in black hole thermodynamics \cite{bombelli,Canent,Terashima,Emparan,levay,Cadoni,Hu2,NavarroSalas} and in the
information loss problem \cite{Maldacena,Preskill,Lloyd2,Ahn1,ManSchullBlack}.


Entanglement behaviour in non-inertial frames was first considered in \cite{Alsingtelep} where the fidelity of teleportation between relative accelerated partners was analysed. After this, occupation number entanglement degradation of scalar \cite{Alicefalls} and Dirac \cite{AlsingSchul} fields due to Unruh effect was shown. 

 In particular, the Unruh effect \cite{DaviesUnr,Unruh0,Takagi,Crispino} --which consists in the emergence of noise when an accelerated observer is describing Minkowski vacuum from his proper frame-- affects the possible entanglement that an accelerated observer Rob would share with an inertial observer Alice. 

To analyse quantum correlations in non-inertial settings it is necessary to combine knowledge from different branches of physics; quantum field theory in curved spacetimes and quantum information theory. This combination of disciplines became known as relativistic quantum information, which is developing at an accelerated pace. It also provides novel tools for the analysis of the Unruh and Hawking effects \cite{DaviesUnr,Unruh,Takagi,Crispino,Hawking} allowing us to study the behaviour of the correlations shared between non-inertial observers.

Recently, there has been increased interest in understanding entanglement and
quantum communication in black hole spacetimes
\cite{Xian,Pan2,Ahntropez} and in using quantum information techniques
to address questions in gravity \cite{Ternada,Ternada2}. Studies on
relativistic entanglement show the emergence of conceptually important
qualitative differences to a non-relativistic treatment. For
instance, entanglement was found to be an observer-dependent
property that is degraded from the perspective of accelerated
observers moving in flat spacetime \cite{Alicefalls,AlsingSchul,Adeschul,Villalba}.
These results show that entanglement in curved spacetime might
not be an invariant concept. Relativisitic quantum information theory uses well-known tools coming from quantum information and quantum optics to study quantum effects provoked by gravity to learn information about the spacetime. We can take advantage of our knowledge about quantum correlations and effects produced by the gravitational interaction to set the basis for experimental proposals ultimately aiming at finding corrections due to quantum gravity effects, too mild to be directly observed.


The differences found between bosonic \cite{Alicefalls} and fermionic \cite{AlsingSchul} entanglement leave an open question about the origin of this distinct behaviour. How can it be possible that bosonic entanglement quickly dies as the acceleration of a non-inertial observer increases while some amount of fermionic entanglement survives even in the limit of infinite acceleration?.

First answers given in the literature by the pioneers who discovered the phenomenon pointed at the difference in the dimension of each system Hilbert space as a possible responsible for these discrepancies, but the question remained open. In this thesis we will demonstrate the strong relationship between statistics and entanglement in non-inertial frames. We will prove that the huge differences between bosonic and fermionic non-inertial entanglement behaviour are related to the counting statistics of the field and have little to do with the Hilbert space dimension for each field mode. This result banishes previous ideas, that were extended in the literature, about the origin of those differences.

Entanglement behaviour in the presence of black holes had not been thoroughly analysed previously. The few studies about entanglement degradation focused on the asymptotically flat region of Schwarzschild spacetime. It would be much more interesting to have results about  entanglement behaviour in the proximities of the event horizon. In this thesis we will present the way to export the results obtained in the frame of uniformly accelerated observers to proper curved space times and black holes scenarios with event horizons. We will also develop a formalism to account for the behaviour of entanglement as a function of the observer's distance to the event horizon of a black hole,  going beyond the analysis in the asymptotically flat region of Schwarzschild spacetime made in previous literature. Here we will provide a rigorous study about what happens when entangled pairs are at small distances from the event horizon.

Almost all the previous work on field entanglement in non-inertial settings made use of what is known as `single mode approximation'. This approximation has allowed pioneering studies of correlations for non inertial observers, but it is based on misleading assumptions about the change of basis between inertial and uniformly accelerated observers and it is partially flawed. In this thesis, we will discuss how this approximation has been misinterpreted since its inception \cite{Alsingtelep,AlsingMcmhMil} and thereafter in all the subsequent works.  We will see the proper physical meaning of such an approximation and will learn to what extent it is valid and how to relax it. We will show that going beyond the single mode approximation will allows us to reach a better understanding of the phenomenon of fermionic entanglement survival in the limit of infinite acceleration \cite{AlsingSchul} and find that the Unruh effect can amplify entanglement and not only destroy it as it was thought before.

There are very few works on field entanglement in general relativistic scenarios for non-stationary spacetimes. Only for bosonic fields and expanding universes some work exists \cite{caball}. As a part of this thesis we will analyse the behaviour of entanglement in non-stationary scenarios. The objective is to prove that the gravitational interaction induces non-classical effects in quantum fields that can be useful in a dual sense: account for quantum effects of the gravitational interaction and provide a basis to obtain information about the nature of gravity in real and analog gravity systems. In simple words, we will analyse how the vacuum state of a field evolves --under the gravitational interaction-- to states that present quantum entanglement. Once again we will see that  huge differences between fermions and bosons appear in a very relevant way in this context. We will prove that fermions are more useful in order to experimentally account for this entanglement and suggest how one can take advantage of these differences to extract information about the underlying background geometry in analog gravity experiments or in cosmology.

Last but not least, using the knowledge gained from other disciplines (in particular tools coming from quantum optics and solid state physics) we will confront the problem of directly measuring the Unruh effect. Experimental detection of the Unruh effect  \cite{ChenTaj,Crispino} required accelerations of order $10^{25}g$ where $g$ is the surface gravity of the Earth. We prove that a detector moving in a flat spacetime acquires a global geometric phase, which is the same for any inertial detector but differs, due to the Unruh effect, for accelerated ones. Taking advantage of this phenomenon we will propose a general experimental setting to detect this effect where the accelerations needed are $10^9$ times smaller than previous proposals, sustained only for a few nanoseconds' time.


\section*{Structure of the thesis}
\phantomsection\addcontentsline{toc}{section}{Structure of the thesis dissertation} 
\begin{list}{\labelitemi}{\leftmargin=1em}
 \item The first section of this thesis (Preliminaries) intends to serve as a brief and notational introduction to the formalism of quantum information theory (chapter \ref{QIi}) and quantum field theory in curved spacetimes (chapter \ref{INTROU}). In these two chapters we present the basic concepts that serve as building blocks for the rest of the original content presented in this thesis. We will also present in this section the problem of the single mode approximation used in previous literature.\end{list}
 
\noindent The research presented here is structured in three blocks that conform the three parts of this thesis:

\begin{list}{\labelitemi}{\leftmargin=1em}
\item  Part \ref{part1}: The relationship between statistics and entanglement in non-inertial frames is studied, disproving the previous idea that the dimensionality of the Hilbert  space controls entanglement behaviour and  obtaining universal laws (only dependent on statistics) for non-inertial entanglement. This part consists of a brief discussion about previous results and the following 7 chapters:
\begin{list}{\labelitemii}{\leftmargin=1em}
\item In chapter \ref{onehalf} we investigate the Unruh effect on entanglement taking into account the spin degree of freedom of the Dirac field. Previous works only explored spinless fermionic fields\footnote{See Grassmann scalar fields in Appendix \ref{appB}}. We go beyond earlier results and we also  analyse spin Bell states, obtaining their entanglement dependence on the acceleration of one of the partners. Then, we consider simple analogs to the occupation number entangled state $\left|00\right\rangle+\left|11\right\rangle$ but with spin quantum numbers for $\left|11\right\rangle$. We show that entanglement degradation in terms of the acceleration happens to be the same for both cases and, furthermore, it coincides with that of the spinless fermionic field despite the different Hilbert space dimension in each case. This is a  first hint against the idea that dimension rules entanglement behaviour. We also introduce a procedure to consistently erase the spin information from our setting, being able to account for correlations present only in the occupation number degree of freedom.  
\item In Chapter \ref{multimode} we introduce an explicitly multimode formalism considering an arbitrary number of accessible modes when analysing bipartite entanglement degradation due to Unruh effect.  A single frequency mode of a fermion field only has a few accessible levels due to Pauli exclusion principle, conversely to bosonic fields which had an infinite number of excitable levels. This was argued to justify fermionic entanglement survival in the infinite acceleration limit. Here we consider entangled states that mix different frequency modes. Hence, the dimension of the Hilbert space in the accelerated observer basis can grow unboundedly, even for a fermion field. We will prove that, despite this analogy with the bosonic case, entanglement loss is limited. We will show that this comes from fermionic statistics through the characteristic structure it imposes on the system's density matrix regardless of its dimension. The surviving entanglement is shown to be independent of the specific maximally entangled state chosen, the kind of fermionic field analysed, and the number of accessible modes considered.
\item In Chapter \ref{etanthrough} we  disclose the behaviour of quantum and classical correlations among all the different spatial-temporal regions of a spacetime with apparent horizons, comparing fermionic with bosonic fields. We show the emergence of conservation laws for entanglement and classical correlations, pointing out the crucial role that statistics plays in the information exchange (and more specifically, the entanglement tradeoff) across the horizon. 
\item In Chapter \ref{boundedpop} we analyse the effect of bounding the occupation number of bosonic field modes on the correlations among inertial and non-inertial observers in a spacetime with apparent horizons. We show that the behaviour of finite-dimensional bosonic fields is qualitatively similar to standard bosonic fields and not to fermionic fields. This completely banishes the notion that dimension rules entanglement behaviour. We show that the main differences between bosonic fields and fermionic fields are still there even if we impose the same dimension for both: for bosonic fields no entanglement is created in the physical subsystems whatever the values of the dimension bound and the acceleration. Moreover, entanglement is very quickly lost as acceleration increases for both finite and infinite dimension. We study in detail the mutual information conservation law found before for bosons and fermions. We will show that for bosons this law stems from classical correlations while for fermions it has a quantum origin. Finally, we will also discuss the entanglement across the causally disconnected regions comparing the fermionic cases with their finite occupation number bosonic analogs. 
\item In Chapter \ref{blackhole1} we analyse the entanglement degradation provoked by the Hawking effect in a bipartite system Alice-Rob when Rob is in the proximities of a Schwarzschild black hole while Alice is free-falling into it. As a result, we will be able to determine the degree of entanglement as a function of the distance of Rob to the event horizon, the mass of the black hole, and the frequency of Rob's entangled modes. By means of this analysis we will show that all the interesting phenomena occur in the vicinity of the event horizon and that, in fact,  Rob has to be very close to the the black hole to see appreciable effects. The universality of the phenomenon is presented: there are not fundamental differences for different masses when working in the natural unit system adapted to each black hole. We also discuss some aspects of the localization of Alice and Rob states. 
\end{list}
\item Part \ref{part2}: We explore the so-called single mode approximation, finding its appropriate physical interpretation and correcting previous statements and uses of such an approximation in the literature. We will see how one can go beyond it, obtaining striking results: on the one hand, we will gain a  deeper understanding about the strange entanglement behaviour of fermionic fields in the infinite acceleration limit and on the other hand we will see how to implement techniques to amplify entanglement by means of the Unruh and Hawking effects. This part consists of the following 3 chapters:
\begin{list}{\labelitemii}{\leftmargin=1em}
\item In Chapter \ref{sma} we address the validity of the single-mode approximation that is commonly invoked in the analysis of entanglement in non-inertial frames and in other relativistic quantum information scenarios. We show that the single-mode approximation is not valid for arbitrary states, finding corrections to previous studies beyond such an approximation in the bosonic and fermionic cases. We also exhibit a class of wave packets for which the single-mode approximation is justified subject to the peaking constraints set by an appropriate Fourier transform. This will give us the proper physical frame of such an approximation.
\item In Chapter \ref{parantpar} we show that going beyond the single mode approximation allows us to analyse the entanglement tradeoff between particle and anti-particle modes of a Dirac field from the perspective of inertial and uniformly accelerated observers. Our results show that a redistribution of entanglement between particle and anti-particle modes plays a key role in the survival of fermionic field entanglement in the infinite acceleration limit.
\item In Chapter \ref{generatio}  going beyond the single mode approximation we show that the Unruh effect can create net quantum entanglement between inertial and accelerated observers, depending on the choice of the inertial state. For the first time, it is shown that the Unruh effect not only destroys entanglement, but may also create it. This opens a new and unexpected resource for finding experimental evidence of the Unruh and Hawking effects.
\end{list}
\item  Part \ref{part3}:   In this last part we study entanglement creation due to the gravitational interaction in two dynamical physically interesting scenarios: the formation of a black hole due to stellar collapse and the expansion of the Universe. We end this thesis presenting a proposal of dectection of the Unruh and Hawking effect by means of the geometric phase acquired by moving detectors. This part consists of the following 3 chapters:
\begin{list}{\labelitemii}{\leftmargin=1em}
\item Chapter \ref{stellarcollapse} shows that a field in the vacuum state, which is  in principle  separable, can evolve to an entangled state  induced by a  gravitational collapse. We will study, quantify, and discuss the origin of this entanglement, showing that it could even reach the maximal entanglement limit for low frequencies or very small black holes, with consequences in micro-black hole formation and the final stages of evaporating black holes. This entanglement provides quantum information resources between the modes that escape to the asymptotic future (thermal Hawking radiation) and those which fall into the event horizon. We will also show that  fermions are more sensitive than bosons to this quantum entanglement generation. This fact could be helpful in finding experimental evidence of the genuine quantum Hawking effect in analog models.
\item In chapter \ref{expandingU}  we study the entanglement generated between Dirac modes in a 2-dimensional conformally flat Robertson-Walker universe showing that inflation-like expansion generates quantum entanglement.  We find radical qualitative differences between the bosonic and fermionic entanglement generated by the expansion. The particular way in which fermionic fields become entangled encodes more information about the underlying spacetime than in the bosonic case, thereby allowing us to reconstruct the history of the expansion. This highlights, once again, the importance of bosonic/fermionic statistics to account for relativistic effects on the entanglement of quantum fields.
\item In chapter \ref{BerryPh} we show that a detector acquires a Berry phase due to its motion in spacetime. The phase is different for the inertial and accelerated detectors as a direct consequence of the Unruh effect. We exploit this fact to design a novel method to measure the Unruh effect.  Surprisingly, the effect is detectable for accelerations $10^9$ times smaller than previous proposals, sustained only for times of nanoseconds.
\end{list}
\end{list}

 The main results of this thesis are summarised in the conclusions section.

In appendix \ref{appB} we present the standard formalism of Klein-Gordon and Dirac equation in curved spacetimes. 


\cleardoublepage





