
%\chapter*{A brief introduction to quantum field theory in curved spacetimes: Introduction to the Unruh and Hawking effects}[Intro to the Unruh and Hawking effects]
%\label{chap:introUnruh}

\chapter{Multimode analysis of fermionic non-inertial entanglement\footnote{E. Mart\'in-Mart\'inez, J. Le\'on, Phys. Rev. A, 80, 042318 (2009)}}\label{multimode}

We have seen in the previous chapter that fermionic maximally entangled states seem to degrade the same no matter the kind of fermionic field considered (Dirac or Grassmann) and the state analysed (Spin Bell or occupation number). This prompts suspicions about the relationship between dimension of the Hilbert space and the entanglement survival in the limit of infinite acceleration for fermions. 

In this chapter we will present a manifestly multimode formalism equivalent to that presented in the chapters before but that will reveal useful to study multimode entangled states. We will show that when we consider states that mix different frequency modes a larger number of modes can become excited by the Unruh effect even for fermion fields, and so, the argument about the Hilbert space dimension playing a role in the degradation phenomenon looks lees plausible. Here a fundamental question arises; does fermionic statistics protect the entanglement in these different frequencies entangled states? In this chapter we shall show that such entanglement survival is fundamentally inherent in the Fermi-Dirac statistics, and that it is independent of the number of modes considered, of the maximally entangled state we start from, and even of the spin of the fermion field studied.

We will begin revisiting the derivation of the entanglement for the states computed in chapter \ref{onehalf} and in \cite{AlsingSchul} with this multimode formalism that will prove to be useful to handle multimode scenarios. After that we will study a state that has no analog with any state studied before: an entangled state of two different spins and frequencies. This state dwells in a Hilbert space of higher dimension than the previous ones and, in principle, there is no guarantee that entanglement is going to behave in the same way as the others.

In section \ref{s2} we will revisit the field vacuum and one particle state in the basis of an accelerated observer and for two different kinds of fermionic fields (a Dirac field and a Grassmann\footnote{See appendix \ref{appB}} ``spinless'' fermion field) in the context of this new formalism. After that, we will analyse entanglement degradation for two different kinds of maximally entangled states that were already analysed with the single mode formalism: the case of vacuum entangled with one particle state studied in chapter \ref{onehalf} and a maximally entangled state of a ``spinless fermion'' field (considered with a single mode formalism in \cite{AlsingSchul}). Studying these known cases where we will learn about the multimode formalism, we    analyse the case of a maximally entangled state of two different frequency modes. We will see that even for the radically different final states obtained in each case, after non-trivial computations entanglement degradation ends up being the same for all of them and the dependence of entanglement on $a$ turns out to be exactly the same as in the other fermionic cases analysed.

\section{Vacuum and 1-Particle states in the multimode scenario}\label{s2}

In this section we shall build the vacuum state and the 1-particle excited state for two very different kinds of fermionic fields in the explicitly multimode formalism: First a Dirac field and then a spinless fermion field. Both kinds of fields were analysed before in a pure single mode scenario (the spinless case in \cite{AlsingSchul} and the Dirac field in chapter \ref{onehalf}).

To begin with, let us consider that a discrete number $n$ of different modes of a Dirac field $\omega_1,\dots,\omega_n$ is relevant.  We label with $s_i$ the spin degree of freedom of each mode. We will rederive a expression for the Minkowski vacuum in a way that takes explicitly into account all the relevant frequency modes that will be useful for further considerations. 

As seen before, the Minkowski multimode vacuum should be expressed in the Rindler basis as a squeezed state, which is an arbitrary superposition of spins and frequencies as it is discussed in chapter \ref{onehalf}
\begin{equation}
\label{vacuumCOMP}\ket{0}=\sum_{m=0}^{2n}\sum_{\substack{s_1,\dots,s_{m}\\\omega_1,\dots,\omega_{m}}}\!\!\!\!C^{m}_{s_1,\dots,s_{m},\omega_1,\dots,\omega_{m}}
\xi_{s_1,\dots,s_{m}}^{\omega_1,\dots,\omega_{m}} \biket{\tilde{m}}{\tilde{m}},
\end{equation}
where, the notation is
\begin{equation}\label{notationmod}
|{\tilde i}\rangle_\text{I}|{\tilde i}\rangle_\text{II}=\biket{s_1,\!\omega_1;\dots;\!s_i,\!\omega_i}{-s_1,\omega_1;\dots;-s_i,\omega_i},
\end{equation}
with
\begin{equation}
\ket{\omega_1,s_1;\dots;\omega_m,s_m}_\text{I}=c^\dagger_{I,\omega_m,s_m}\dots c^\dagger_{I,\omega_1,s_1}\ket{0}_\text{I}.
\end{equation}
The label outside the kets notates Rindler spacetime region, and the symbol $\xi$ is 0 if $\{\omega_i,s_i\}=\{\omega_j,s_j\}$ for any $i\neq j$, and it is $1$ otherwise, imposing Pauli exclusion principle constraints on the state (quantum numbers of fermions cannot coincide).

Due to the anticommutation relations of the fermionic operators, terms with different orderings are not independent.
So, without loss of generality, we could choose not to write all the possible orderings in \eqref{vacuumCOMP}, selecting one of them instead. In this fashion we will write the elements \eqref{notationmod} with the following ordering criterion:
\begin{eqnarray}\label{ordering}
 \nonumber & \omega_i\le \omega_{i+1}, &\\
 &\omega_i=\omega_{i+1}\Rightarrow s_i=\uparrow,s_{i+1}=\downarrow.&
 \end{eqnarray}
The coefficients $C^m$ are constrained because the Minkowski vacuum should satisfy 
$a_{\omega,s}\ket0=0$, $\forall \omega,s$. 
As the elements \eqref{notationmod} form an orthogonal set, this implies that all the terms proportional to different elements of the set should be zero simultaneously, which gives the following conditions on the coefficients
\begin{itemize}
\item $C^1_{s,\omega}$ as a function of $C^0$\\[-9mm]
\end{itemize}
\begin{eqnarray}
\label{01} C^1_{\uparrow,\omega}\cos r-C^0\sin r&=&0,\\*
C^1_{\downarrow,\omega}\cos r-C^0\sin r&=&0. \label{02}
\end{eqnarray}
Since equations \eqref{01},\eqref{02} should be satisfied $\forall \omega$, we obtain that $C^1_{\uparrow,\omega}=C^1_{\downarrow,\omega}=\text{const.}$ because $C^0$ does not depend on $\omega$ or $s$. We will denote $C^1_{s,\omega}\equiv C^1$.
\begin{itemize}
\item $C^2_{s_1,s_2,\omega_1,\omega_2}$ as a function of $C^1$\\[-9mm]
\end{itemize}
\begin{eqnarray}
\label{03}  C^1\sin r- C^2_{ss',\omega_1,\omega_2}\cos r&=&0,\\*
\label{04}  C^1\sin r- C^2_{ss',\omega_1,\omega_2}\cos r&=&0,
\end{eqnarray}
where we consider\footnote{Our interest here is to show that the increase on the Hilbert space dimension does not play a role in entanglement behaviour. With this motivation, we consider that all the $N$ frequencies $\{\omega^j_\text{R}\}_{j=1}^N$ are close enough to roughly approximate $r^1\approx r^2\approx\cdots r^N\equiv r$. This will show the result we want to prove more transparently. } $r_1\approx r_2\equiv r$. Since equations \eqref{03}, \eqref{04} should be satisfied $\forall \omega_0$, we obtain that $C^2_{s_1,s_2,\omega_1,\omega_2}=C^2$ where $C^2$ does not depend on spins or frequencies since $C^1$ does not depend on $\omega$ or $s$, the only dependence of the coefficients \eqref{vacuumCOMP} with $\omega_i$ and $s_i$ is given by Pauli exclusion principle. This dependence comes through the symbol $\xi$.

In fact it is very easy to show inductively that all the coefficients are independent of $s_i$ and $\omega_i$ --apart from the Pauli exclusion principle constraint--. Using the fact that $C^0$ does not depend on $s_i$ and $\omega_i$ and noticing that by applying the annihilator on the vacuum state and equalling it to zero we will always obtain the linear relationship between $C^{n}$ and $C^{n-1}$ given below.
\begin{itemize}
\item $C^m$ as a function of $C^{m-1}$\\[-9mm]
\end{itemize}
\begin{eqnarray}
\label{05}  C^{m-1}\sin r- C^m\cos r&=&0,\\*
\label{06}  C^{m-1}\sin r- C^m\cos r&=&0.
\end{eqnarray}
We finally obtain that $C^m$ is a constant which can be expressed as a function of $C^0$ as
\begin{equation}\label{coeff2}
C^m=C^0 \tan^m r,
\end{equation}
where $\tan r=\exp\left(-\pi \omega c/a\right)$. $C^m$ is independent of $s_i$ and $\omega$. Therefore, we obtain the vacuum state by substituting \eqref{coeff2} in \eqref{vacuumCOMP} and factoring the coefficients out of the summation.
\begin{equation}\label{vacuumCOMP2}
\ket{0}=C^0\sum_{m=0}^{2n}\tan^m r\sum_{\substack{s_1,\dots,s_m\\\omega_1,\dots,\omega_m}}\!\!\!\!\xi_{s_1,\dots,s_m}^{\omega_1,\dots,\omega_m} \biket{\tilde m}{\tilde m}.\\
\end{equation}
The only parameter not yet fixed is $C^0$. To derive $C^0$ except for a global phase, we impose the normalisation of the vacuum state in the Rindler basis $\braket00=1$, from \eqref{vacuumCOMP2}, we see that this means that
\begin{equation}\label{normalispinap}C^0=\left[\sum_{m=0}^n\Upsilon_m\tan^{2m}r+\sum_{m=n+1}^{2n}\Upsilon_{2n-m}\tan^{2m}r\right]^{-1/2},\end{equation}
where
\begin{equation}\label{upsilon}\Upsilon_m=\sum_{\substack{s_1,\dots,s_m\\\omega_1,\dots,\omega_m}}\!\!\!\!\xi_{s_1,\dots,s_m}^{\omega_1,\dots,\omega_m}.
\end{equation}
Now, we are going to show that \eqref{upsilon}, has the form
\begin{equation}\label{upsilonap}\Upsilon_m=\sum_{p=0}^{\lfloor \frac{m}
{2}\rfloor}\binom{n-p}{m-2p}\binom{n}{p}2^{m-2p}.
\end{equation}
To see how this expression comes from Pauli exclusion principle, we have to read $p$ as an index that represents the number of possible spin pairs ($\omega_i=\omega_{i+1},s_i=\uparrow,s_{i+1}=\downarrow$) which can be formed, and goes from $0$ to the integer part of $m/2$, and then
\begin{itemize}
\item The combinatory number $\binom{n-p}{m-2p}$ represents the possible number of combinations of modes that can be formed taking into account that $p$ different frequencies $\omega_i$ are not available since they are already occupied by the $p$ pairs. Hence, it is given by the combinations of the $n-p$ available frequencies taken $m-2p$ at time, since $m-2p$ is the number of free momentum `slots' (the total number of different frequencies $m$ minus the number of positions taken by pairs $2p$).
\item The combinatory factor $\binom{n}{p}$ represents the different possible combinations for the configuration of the $p$ pairs, which have $n$ possible different frequencies to be combined among them without repetition and in a particular order.
\item The factor $2^{m-2p}$ represents the possible combination for the spin degree of freedom of each mode. As a spin pair only admits one spin configuration, only the unpaired modes will give different spin contributions, so the factor is $(2S+1)^{m-2p}$ giving the formula \eqref{upsilon}
\end{itemize}

After some lengthy but elementary algebra we can see that
\begin{equation}\label{upsilon22}\Upsilon_m=\binom{2n}{m},
\end{equation}
and using the property $\binom{a}{a-b}=\binom{a}{b}$, 
we can express \eqref{normalispinap} as
\begin{equation}\label{normalispin}
C^0=\left[\sum_{m=0}^{2n}\binom{2n}{m}\tan^{2m}r\right]^{-1/2}=\cos^{2n}r 
\end{equation}
and, therefore, rewrite the vacuum \eqref{vacuumCOMP} as
\begin{equation}\label{vacuumCOMP2b}
\ket{0}=\cos^{2n}r\sum_{m=0}^{2n}\tan^m r\sum_{\substack{s_1,\dots,s_m\\\omega_1,\dots,\omega_m}}\!\!\!\!\xi_{s_1,\dots,s_m}^{\omega_1,\dots,\omega_m} \biket{\tilde m}{\tilde m}.\\
\end{equation}

Next, the 1-particle state can be worked out translating the Minkowski one particle Unruh state $\ket{\omega,s}=a^\dagger_{\omega,s}\ket0$ into the Rindler basis
\begin{equation}\label{onepart23m}
\ket{k,s}_\text{U}=\sum_{m=0}^{2n-1}A^m\sum_{\substack{s_1,\dots,s_m\\\omega_1,\dots,\omega_m}}\!\!\!\!\xi_{s_1,\dots,s_m,s}^{\omega_1,\dots,\omega_m,\omega} \biket{\tilde m;\omega,s}{\tilde m},
\end{equation}
where
\begin{equation}\label{Am}
A^m=(C^m\cos r+C^{m+1}\sin r)
\end{equation}
and the notation $\ket{\tilde m;\omega,s}_\text{I}$, consequently with \eqref{notationmod}, means the ordered version of 
$\ket{s_1,\!\omega_1;\dots;\!s_n,\!\omega_n;\omega,s}_\text{I}$. 

Another different kind of field that we are going to consider appears by neglecting spin while keeping the fermionic statistics (Grassmann scalar 
fields). We will analyse Unruh decoherence in this multimode formalism. The Minkowski multimode vacuum state would be expressed as
\begin{equation}\label{vacuzero}
\ket{0}=\sum_{m=0}^n\sum_{\omega_1,\dots,\omega_m}\xi_{\omega_1,\dots,\omega_m}\hat C^m_{\omega_1,\dots,\omega_m}\ket{\tilde m}_\text{I}\ket{\tilde m}_\text{II},
\end{equation}
where, in this occasion $\ket{\tilde m}_\text{I}\ket{\tilde m}_\text{II}=\ket{\omega_1,\dots,\omega_m}_\text{I}$ $\ket{\omega_1,\dots,\omega_m}_\text{II}$. Using the same procedures as for the spin $1/2$ 
case \eqref{vacuumCOMP} we can prove that all the coefficients are independent of $\omega_i$ and can be related to $\hat C^0$ as in \eqref{coeff2}, $\hat C^m=\hat C^0 \tan^m r$. We can now fix $\hat C^0$ imposing the normalisation relation $\braket00=1$ giving
\begin{equation}\label{normalizeropre}
\hat C^0=\left[\sum_{m=0}^n\chi_m\tan^{2m}r\right]^{-1/2}.
\end{equation}
For the spinless fermion field we have
\begin{equation}\label{XIIap}
\chi_m\equiv\sum_{\omega_1,\dots,\omega_m}\xi_{\omega_1,\dots,\omega_m}=\binom{n}{m},
\end{equation}
corresponding to the possible combinations of m values of $\omega_i$ imposing that $\omega_i\neq \omega_j$ if $j\neq i$ (which is the translation of Pauli exclusion principle to spinless modes). This expression can be readily obtained taking into account that the $n$ possible values of $\omega_i$ should be combined without repetition in a particular ordering of the $m$ modes, so the possible combinations are simply the combinatory number $\binom{n}{m}$.

Therefore, \eqref{normalizeropre} can be simplified to
\begin{equation}\label{normalizero}
\hat C^0=\left[\sum_{m=0}^n\binom{n}{m}\tan^{2m}r\right]^{-1/2}=\cos^n r.
\end{equation}
 
Finally, the Grassmann one particle state $a_\omega^\dagger\ket0$ is
\begin{equation}\label{vacuuno}
\ket{\omega}_\text{U}=\sum_{m=0}^{n-1}\hat A^m\!\!\sum_{\omega_1,\dots,\omega_m}\!\!\xi_{\omega_1,\dots,\omega_m,k}\biket{\tilde m,k}{\tilde m},
\end{equation}
where $\hat A^m$ has the expression \eqref{Am} but substituting $C^m$ by $\hat C^m$.

\section{Entanglement degradation for a Dirac field}\label{caso1}

In the following we will analyse Unruh entanglement degradation in various \mbox{settings} corresponding to different maximally entangled states of fermion fields. First we consider a state that was already analysed in chapter \ref{onehalf} but computing entanglement with this new formalism. This will be useful as a pedagogical example of operation of this explicitly multimode formalism and to compare with the results obtained when we go beyond the cases studied in previous chapters. Let us consider the state 
\begin{equation}\label{minkowstate1}\ket\Psi=\frac{1}{\sqrt2}\big(\ket{0}\ket{0}+\ket{\omega_A,s_A}_\text{U}\ket{\omega_\text{R},s_\text{R}}_\text{U}\big).\end{equation}
The density matrix for the accelerated observer Rob is obtained after expressing Rob's state in the Rindler basis --which means using \eqref{vacuumCOMP} and \eqref{onepart23m} in Rob's part of \eqref{minkowstate1}-- and then, tracing over Rindler's region II since Rob is causally disconnected from it and he is not to extract any information from beyond the horizon. Following this procedure we obtain the density matrix
\begin{align}\label{densmat1}
\nonumber\rho&=\frac{1}{2}\Big[\sum_{m=0}^{2n}\Big(D_{0}^m\!\!\!\sum_{\substack{s_1,\dots,s_m\\\omega_1,\dots,\omega_m}}\!\!\!
\xi_{s_1,\dots,s_m}^{\omega_1,\dots,\omega_m}\ket{0}_{\text{U}}\ket{\tilde m}_\text{I}\bra{0}_{\text{U}}\bra{\tilde m}_\text{I}\Big)\\*
\nonumber&+\sum_{m=0}^{2n-1}\Big(D_{1}^m\!\!\!\sum_{\substack{s_1,\dots,s_m\\\omega_1,\dots,\omega_m}}\!\xi_{s_1,\dots,s_m,s_\text{R}}^{\omega_1,\dots,\omega_m,\omega_\text{R}}
\ket{0}_{\text{U}}\ket{\tilde m}_\text{I}\bra{\omega_A,s_A}_{\text{U}}\bra{\tilde m;\omega_\text{R},s_\text{R}}_\text{I}\!\!\Big)\\*
&+\sum_{m=0}^{2n-1}\Big(D_{2}^m\!\!\!\!\sum_{\substack{s_1,\dots,s_m\\\omega_1,\dots,\omega_m}}\!\!\!\!
\xi_{s_1,\dots,s_m,s_\text{R}}^{\omega_1,\dots,\omega_m,\omega_\text{R}}\ket{\omega_A,s_A}_{\text{U}}\ket{\tilde m;\omega_\text{R},s_\text{R}}_\text{I}\bra{\omega_A,s_A}_{\text{U}}\bra{\tilde m;\omega_\text{R},s_\text{R}}_\text{I}\Big)\Big]+(\text{H.c.})_{_{\substack{\text{non-}\\\text{diag.}}}},
\end{align}
where $(\text{H.c.})_{\text{non-diag.}}$ means Hermitian conjugate of only the non-diagonal terms and
\begin{equation}\label{Des}
D_i^m=|C^0|^2\frac{\tan^{2m}r}{\cos^i r}
\end{equation}
with $i=0,1,2$. The derivation of \eqref{densmat1} can be found in the appendix to this chapter (section \ref{ap2}). 

Notice that as Rob accelerates, the state becomes mixed, showing all the available modes $(\omega_1,\dots,\omega_n)$ excitations explicitly. 

As in the previous chapter we will compute the negativity as a function of $a$ as a measure of the state entanglement.

The partial transpose of \eqref{densmat1} has a $2\times2$ and $1\times1$ blocks structure. Each eigenvalue in the $1\times1$ blocks is non-negative (since 
$D_i^m\ge0$), so we are interested in the $2\times2$ which are the ones that may have negative eigenvalues. These $2\times2$ blocks expressed in the basis
\begin{equation}\label{blocksbasis}
\Big\{\ket0_{A}\ket{\tilde m;\omega_\text{R},s_\text{R}}_\text{I},\ket{s_A,\omega_A}_\text{U}\ket{\tilde m}_\text{I}\Big\}_{m=0}^{2n-1}
\end{equation}
are of the form
\begin{equation}\label{blocks3m}
\frac12
\left(\begin{array}{cc}
D^{m+1}_0 & \pm D_1^m\\
\pm D_1^m & 0
\end{array}\right).
\end{equation}
There is no matrix element proportional to $D_2^m$ because it would correspond to $\ket{\omega_A,s_A}_{\text{U}}\ket{\tilde m;\omega_\text{R},s_\text{R}}_\text{I}\bra{\omega_A,s_A}_{\text{U}}\bra{\tilde m;\omega_\text{R},s_\text{R}}_\text{I}$ 
which cannot have any element within this block as Pauli exclusion principle imposes $\omega_\text{R},s_\text{R}\not\in\left\{\omega_i,s_i\right\}_{i=1,\dots,m}$.

Each $2\times2$ block of \eqref{blocks3m} appears a number of times $B_m$. Taking a look at the basis in which those blocks are expressed \eqref{blocksbasis}, we can see that the expression for $B_m$ is given by two terms:
\begin{itemize}
\item  The number of possible combinations of $m$ modes with $n$ possible different frequencies $\omega_i$ and two possible spins $s_i$ according to Pauli exclusion principle as in \eqref{upsilonap}.
\item A negative contribution which comes from excluding those combinations in which $\{\omega_\text{R},s_\text{R}\}$ coincides with any $\{\omega_i,s_i\}$, which means excluding the number of combinations in \eqref{upsilonap} which have one of their values fixed to $\{\omega_i,s_i\}=\{\omega_\text{R},s_\text{R}\}$. This number is given by the combinatory number $\binom{2n-1}{m-1}$ provided that $m>0$ and it is zero if $m=0$.
\end{itemize}

To see where this negative contribution comes from let us assume that $\{\omega_i,s_i\}$ is the mode which coincides with $\{\omega_\text{R},s_\text{R}\}$ we will have $2n-1$ possible choices for each $\{\omega_{j\neq i},s_{j\neq i}\}$ ($2$ values for $s$ and $n$ for $\omega$ excepting $\omega_i,s_i$ due to Pauli exclusion principle). This happens for all the combinations of all the possible values $\{\omega_j,s_j\}$ with $j\neq i$. Hence, as there are $m$ modes and one of them is fixed $\omega_i=\omega_\text{R}, s_i=s_\text{R}$, we have to consider the combinations of $2n-1$ elements taken $m-1$ at time.

If $m>n$ the situation is equivalent to having $m'=2n-m$. Since having more modes $m$ than possible values of $\omega_i$ we are forced to have $n-m$ pairs and we lose freedom to combine the available modes.

Now if we compute
\begin{equation}\label{Nblocksap}B_m=\Upsilon_m-\binom{2n-1}{m-1}=\binom{2n}{m}-\binom{2n-1}{m-1},
\end{equation}
after some basic algebra we obtain
\begin{equation}\label{Nblocks3m}B_m=\binom{2n-1}{m}\end{equation}

Using \eqref{Des}, the negative eigenvalue of each block can be expressed
\begin{equation}\label{neigenb}
|\lambda^-_m|=\frac12|C^0|^2\,\tan^{2m} r,
\end{equation}
where $C^0$ is given by \eqref{normalispin}. Therefore, the negativity is expressed as the sum of the negative eigenvalue of each block $|\lambda_m^-|$ multiplied by the number of times $B_m$ that that block appears in the partially transposed density matrix. The 
summation of the series is
\begin{equation}\label{negativitypre1}
\mathcal{N}=\sum_{m=0}^{2n-1} B_m|\lambda^-_m|=\frac{\cos^{4n}r}{2}\sum_{m=0}^{2n-1}\binom{2n-1}{m}\tan^{2m}r ,
\end{equation}
but this result can be easily simplified to
\begin{equation}\label{negativitypre}
\mathcal{N}=\frac12\cos^2 r,
\end{equation}
which is independent of the number of modes that we have considered. This is the same result obtained in chapter \ref{onehalf}. This expected result shows that this multimode formalism is valid to analyse the entanglement degradation due to Unruh effect. This also emphasises a somewhat non trivial result: despite the fact that all the available modes are excited when Rob accelerates \eqref{densmat1}, the quantum correlations behave as if we were considering only one possible mode for the field. This is nothing but a consequence of the tensor product structure of the Hilbert space showed in the previous chapter.

\section{Entanglement degradation for a spinless fermion field}\label{caso3}

We can also revisit the results on the literature \cite{AlsingSchul} and consider a spinless field on which we have imposed the fermionic statistics. We will re-obtain with the explicitly multimode formalism the entanglement degradation for the maximally entangled state with vacuum and one particle components
\begin{equation}\label{minkowstate3}
\ket\Psi=\frac{1}{\sqrt{2}}\Big(\ket{0}_{\text{U}}\ket{0}_\text{R}+\ket{\omega_A}_{\text{U}}\ket{\omega_\text{R}}_\text{R}\Big).
\end{equation}
As it is shown in appendix \ref{ap2}, this leads to the following density matrix for the accelerated observer Rob after using expressions \eqref{vacuzero} and \eqref{vacuuno} and after tracing over Rindler's region II 
\begin{align}\label{densmat3}
\nonumber\rho&=\frac{1}{2}\Big[\sum_{m=0}^{n}\!\hat D_{0}^m\!\!\!\sum_{\omega_1,\dots,\omega_m}\!\!\!\xi_{\omega_1,\dots,\omega_m}\ket{0}_{\text{U}}\ket{\tilde m}_\text{I}\bra{0}_{\text{U}}\bra{\tilde m}_\text{I}\\*
\nonumber&+
\sum_{m=0}^{n-1}\Big(\hat D_{1}^m\!\!\!\sum_{\omega_1,\dots,\omega_m}\!\xi_{\omega_1,\dots,\omega_m,\omega_\text{R}}\ket{0}_{\text{U}}\ket{\tilde m}_\text{I}\bra{\omega_A}_{\text{U}}\bra{\tilde m;\omega_\text{R}}_\text{I}\\
 &+\hat D_{2}^m\!\!\!\sum_{\omega_1,\dots,\omega_m}\xi_{\omega_1,\dots,\omega_m,\omega_\text{R}}\ket{\omega_A}_{\text{U}}\ket{\tilde m;\omega_\text{R}}_\text{I}\bra{\omega_A}_{\text{U}}\bra{\tilde m;\omega_\text{R}}_\text{I}\Big)\Big]+(\text{H.c.})_{_{\substack{\text{non-}\\\text{diag.}}}},
\end{align}
where $\hat D^,_i$ is given by the expression \eqref{Des} but substituting $C^0$ by $\hat C^0$ (given in equation \eqref{normalizero}). 

Analogously to \eqref{densmat1}, the partial transpose of \eqref{densmat3} has a $2\times2$ and $1\times1$ blocks structure. The 
$2\times2$ blocks expressed in the basis
\begin{equation}\label{blocksbasis3}
\Big\{\ket{0}_\text{U}\ket{\tilde m;\omega_\text{R}}_\text{I},\ket{\omega_A}_\text{U}\ket{\tilde m}_\text{I}\Big\}_{m=0}^{n-1}
\end{equation}
would have the form 
\begin{equation}\label{blockszero}
\frac12
\left(\begin{array}{cc}
\hat D^{m+1}_0 & \pm \hat D_1^m\\
\pm \hat D_1^m & 0
\end{array}\right).
\end{equation}
The main difference with \eqref{blocks3m} is that $\hat C^0$ is given by \eqref{normalizero} (instead of $C^0$ given by \eqref{normalispin}). Here, $\hat D_2^m$ does not appear because Pauli exclusion principle imposes that $\omega_\text{R}\not\in\left\{\omega_i\right\}_{i=1,\dots,m}$. Now, each $2\times2$ block multiplicity is $W_m$.

$W_m$ can be easily obtained taking into account that the number of $2\times2$ blocks \eqref{blocksbasis3} is given by the number of mode combinations allowed by Pauli principle \eqref{XIIap}, subtracting the terms having $\omega_i=\omega_\text{R}$. The number of possible $\omega_j$ values allowed for the rest $m-1$ modes having fixed $\omega_i=\omega_\text{R}$ is $n-1$, so the number of combinations we must subtract is the combinatory number $\binom{n-1}{m-1}$, obtaining
 \begin{equation}\label{Nblocks3ap}W_m=\binom{n}{m}-\binom{n-1}{m-1}=\binom{n-1}{m}.\end{equation}

The negative eigenvalue of each block is given by the same expression \eqref{neigenb} but $C^0$ is now given by 
\eqref{normalizero}, which is to say
\begin{equation}\label{neigenb0}
|\lambda^-_m|=\frac12|\hat C^0|^2\,\tan^{2n} r=\frac12\cos^{2m}\,\tan^{2m} r.
\end{equation}
 The negativity yields
\begin{equation}\label{negazeropre}
\mathcal{N}=\sum_{m=0}^{n-1} W_m|\lambda^-_m|=\frac12\cos^{2n} r\sum_{m=0}^{n-1}\binom{n-1}{m}\tan^{2m}r.
\end{equation}
At this point, the reader might not be surprised by the resulting negativity after straightforward simplification
\begin{equation}\label{negazero}
\mathcal{N}=\frac{1}{2}\cos^2 r
\end{equation}
which is the same result as in the cases \eqref{minkowstate1} and 
\eqref{minkowstate2}. Again, entanglement degradation due to Unruh effect is 
the same as considering one mode of a Dirac field.

Although we have seen that the derivation here does not add anything new from the standard mode-by-mode expressions for the vacuum and one-particle states, it will be useful to study entangled states of a discrete number of different frequency modes.



\section[Entanglement degradation between different modes]{Entanglement degradation between different frequency modes}\label{caso2}

In this section we will go beyond the states analysed in previous sections of this thesis and the published literature. We will analyse a state that in the Minkowskian basis is a maximally entangled superposition of different (but very close) frequency modes with arbitrary spin components. In principle each mode would suffer its own decoherence induced by the Unruh noise and the naive expectation would be that, even though the frequencies are close, the state presents a qualitative different entanglement behaviour than the other single-mode states analysed previously.

If instead of \eqref{minkowstate1} we start from a Bell momentum-spin state in the Minkowskian basis
\begin{equation}\label{minkowstate2}\ket\Psi=\frac{1}{\sqrt2}\Big(\ket{\omega_A^1,s_A^1}_{\text{U}}\ket{\omega^1_\text{R},s^1_\text{R}}_\text{U}+\ket{\omega^2_A,s^2_A}_{\text{U}}\ket{\omega^2_\text{R},s^2_\text{R}}_\text{U}\Big).
\end{equation}
As it can be seen in the appendix \ref{ap2} the density matrix for Rob takes the form 
\begin{align}\label{densmat2}
\rho&=\nonumber\sum_{m=0}^{2n-1}\frac{D_{2}^m}{2}\!\!\!\sum_{\substack{s_1,\dots,s_m\\\omega_1,\dots,\omega_m}}\!\!\!\!
\Big(\xi_{s_1,\dots,s_m,s_\text{R}^1}^{\omega_1,\dots,\omega_m,\omega_\text{R}^1}\!\ket{\omega^1_A,s^1_A}_\text{U}\!\ket{\tilde m;\omega^1_\text{R},s^1_\text{R}}_\text{I}\bra{\omega^1_A,s^1_A}_{\text{U}}\bra{\tilde m,\omega^1_\text{R},s^1_\text{R}}_\text{I}\\*
&\nonumber+\xi_{s_1,\dots,s_m,s_\text{R}^2}^{\omega_1,\dots,\omega_m,\omega_\text{R}^2}\ket{\omega^2_A,s^2_A}_{\text{U}}\ket{\tilde m;\omega^2_\text{R},s^2_\text{R}}\bra{\omega_A^2,s_A^2}_{\text{U}}\bra{\tilde m;\omega^2_\text{R},s_\text{R}^2}_\text{I}\\*
&+\xi_{s_1,\dots,s_m,s_\text{R}^1,s_\text{R}^2}^{\omega_1,\dots,\omega_m,\omega_\text{R}^1,\omega_\text{R}^2}\ket{\omega^1_A,s^1_A}_{\text{U}}\ket{\tilde m;\omega_\text{R}^1,s_\text{R}^1}_\text{I}\bra{\omega^2_A,s^2_A}_{\text{U}}\bra{\tilde m;\omega^2_\text{R},s^2_\text{R}}_\text{I}\Big)+(\text{H.c.})_{_{\substack{\text{non-}\\\text{diag.}}}}.
\end{align}
Analogously to \eqref{densmat1}, the partial transpose of \eqref{densmat2} has a $2\times2$ and $1\times1$ blocks structure. Again, we are interested in the 
$2\times2$ blocks --the ones that may have negative eigenvalues--. These blocks expressed in the basis
\begin{equation}\label{blocksbasis2}
\Big\{\ket{\omega^1_A,s^1_A}_\text{U}\ket{\tilde m;\omega^2_\text{R},s^2_\text{R}}_\text{I},\ket{s^2_A,\omega^2_A}_\text{U}\ket{\tilde m,\omega^1_\text{R},s^1_\text{R}}_\text{I}\Big\}_{m=0}^{2n-2}
\end{equation}
are of the form
\begin{equation}\label{blocks2}
\frac12
\left(\begin{array}{cc}
0 & \pm D_2^m\\
\pm D_2^m & 0
\end{array}\right).
\end{equation}
Notice that there is no diagonal elements in the block because the terms that would go in the diagonal are forbidden by Pauli exclusion principle,  which imposes 
that $\omega^1_\text{R},s^1_\text{R};\omega^2_\text{R},s^2_\text{R}\not\in\left\{\omega_i,s_i\right\}_{i=1,\dots,m}$. This time, each $2\times2$ block of the form \eqref{blocks2} appears a number of  times $B'_m$. The derivation of $B'_m$ is quite straightforward considering the derivation of $B_m$. Looking at the basis of the $2\times2$ blocks \eqref{blocksbasis2} we can see that this case would be exactly the same as the previous one but now $\{\omega_i,s_i\}$ cannot coincide neither with $\{\omega_\text{R}^1,s_\text{R}^1\}$ nor $\{\omega_\text{R}^2,s_\text{R}^2\}$. Repeating the same reasoning as before we have to do three operations as follows
\begin{itemize}
\item Discounting the combinations which have a coincidence $\{\omega_i,s_i\}=\{\omega_\text{R}^1,s_\text{R}^1\}$ from the total number \eqref{upsilonap} and obtain the expression \eqref{Nblocksap}
\item Subtracting the combinations with coincidences $\{\omega_j,s_j\}=\{\omega_\text{R}^2,s_\text{R}^2\}$
\item Taking into account that we have subtracted twice the cases in which we have double coincidences, we need to add the number of double coincidences once to compensate it.
\end{itemize}
 The number of cases with double coincidences (which require $m>1$) is the combinatory number $\binom{2n-2}{m-2}$, as we have $2n$ possible spins and frequencies minus the two fixed possibilities ($\{\omega_i,s_i\}=\{\omega_\text{R}^1,s_\text{R}^1\}$ and $\{\omega_j,s_j\}=\{\omega_\text{R}^2,s_\text{R}^2\}$) and $m$ modes being 2 of them fixed. Taking this into account
\begin{equation}\label{apend2}
B'_m=\Upsilon_m -\binom{2n-1}{m-1}-\binom{2n-1}{m-1} + \binom{2n-2}{m-2}.
\end{equation}
This expression can be simplified to
\begin{equation}\label{Nblocks2ap}B'_m=B_m-\binom{2n-2}{m-1}=\binom{2n-2}{m}.
\end{equation}

The negative eigenvalue of each block is
\begin{equation}\label{neigen2}
|\lambda^-_m|=\frac{D_2^m}{2}=\frac{\cos^{4n-2}r}{2}\tan^{2m}r,
\end{equation}
where $C^0$ has been substituted by \eqref{normalispin}. Therefore, the negativity results
\begin{equation}\label{negativity2pre}
\mathcal{N}=\sum_{m=0}^{2n-2} B'_m|\lambda^-_m| =\frac{\cos^{4n-2}r}{2}\sum_{m=0}^{2n-2}\binom{2n-2}{m}  \tan^{2m}r.
\end{equation}
This can be readily simplified to
\begin{equation}\label{negativity2pre}
\mathcal{N}= \frac12\cos^2 r.
\end{equation}
Strikingly we run into the same simple result as above\footnote{It can be proved that if we relax the approximation $r_1\approx r_2$ the negativity is the geometric mean of each mode negativity $\mathcal{N}=\frac12\cos r_1\cos r_2$.} \eqref{negativitypre}. Even starting from a spin Bell state, the entanglement is degraded by Unruh effect in the same way as in the previous case.


\section{Discussion}

Let us summarise our results so far. We have studied entanglement degradation by Unruh effect due to Rob's acceleration for  three different Minkowskian maximally entangled states: 1) Vacuum-vacuum plus one-particle-one-particle maximally entangled state of a Dirac field, 2) Vacuum-vacuum plus one-particle-one-particle maximally entangled state of a spinless fermion field 3) Multimode Bell state for a Dirac field. In spite of the essential differences among these states and the very different dimension of the Hilbert space for the three cases, the negativity degrades in exactly the same way for any acceleration. This result may look surprising but this is an outcome of fermionic statistics.

In the bosonic case acceleration excites an infinite number of modes being the Hilbert subspace that contains the state of higher dimension, and this completely degrades the entanglement in the limit $a\rightarrow\infty$. If the Hilbert space dimension were determinant for this phenomenon one could expect a similar behaviour here when we increase the number of modes involved in a fermionic entangled state, but our results show that  when two different frequency modes are involved, entanglement behaves in the same way as for the single mode entangled state. 

This striking result can be traced back to the fanciful block structure of Rob density matrix, which produces the same negativity even when the characteristics of the entangled states (and even the field) change. The culprit of this structure is fermionic statistics, (as we have discussed after \eqref{blocks3m}, \eqref{blocks2}, \eqref{blockszero}) which is responsible for the identical, and somewhat unforeseen, negativity behaviour. This is a global feature of maximally entangled states of fermion fields and not a consequence of the specific cases chosen and the number of modes considered. 

So, $\mathcal{N}\rightarrow 
1/4$ when $a\rightarrow\infty$, and this happens independently of the number of modes of the field that we are considering, of the starting maximally entangled state, and even of the spin of the field which we study. What all the cases have in common is the fermionic statistics itself, so, widening the margin for Unruh degradation for fermionic fields will not affect entanglement degradation.

Notice that a very different scenario would come from a setting in which we erase partial information for the state as Rob accelerates (e.g. angular momentum). In that case, it was shown in the previous chapter that entanglement degradation is greater than in the cases where all the information is taken into account, but this has more to do with this erasure of information than which the fermionic nature of the states.

One question immediately arises from these results; are the remaining correlations purely statistical? As all the states undergo the same degradation,  the quantum correlations which survive the infinite acceleration limit may only contain the information about the fermionic nature of the system and nothing else. We will understand better this question when we analyse fermionic entanglement beyond the single mode approximation in chapter \ref{parantpar}. 


\section[Appendix to the Chapter: Density matrix construction]{Appendix to Chapter \ref{multimode}: Density matrix construction}\label{ap2}

In this appendix we will derive expressions \eqref{densmat1}, \eqref{densmat2}, \eqref{densmat3} for the density matrix of the system Alice-Rob.

Using expression \eqref{vacuumCOMP2} we see that the Alice-Rob Minkowskian operator $P_{00}\equiv\proj{0_\text{A}0_\text{R}}{0_\text{A}0_\text{R}}$ when Rob is accelerating translates into
\begin{equation}\label{0000}
 P_{00}=\sum_{m=0}^{2n}\sum_{l=0}^{2n}C^m (C^{l})^*\!\!\!\sum_{\substack{s_1,\dots,s_m\\\omega_1,\dots,\omega_m}}\!\xi_{s_1,\dots,s_m}^{\omega_1,\dots,\omega_m}\sum_{\substack{s'_1,\dots,s'_{l}\\\omega_1,\dots,\omega'_{l}}}\!\xi_{s'_1,\dots,s'_l}^{\omega'_1,\dots,\omega'_l}\ket{0}_{\text{U}}\biket{\tilde m}{\tilde m}\langle{\tilde l}|_{\text{II}}\langle{\tilde l}|_{\text{I}}\bra{0}_{\text{U}},
\end{equation}
where
\begin{align}
\nonumber\langle{\tilde l}|_{\text{I}}&=\bra{\omega'_1,s'_1;\dots; \omega'_l,s'_l}_\text{I}=\bra0_I c_{I,\omega'_1,s'_1}\dots c_{I,\omega'_m,s'_m}\\*
\langle{\tilde l}|_{\text{II}}&=\bra{\omega'_1,-s'_1;\dots; \omega'_l,-s'_l}_\text{I}=\bra0_{\text{II}} c_{II,\omega'_1,-s'_1}\dots c_{II,\omega'_m,-s'_m}.
\end{align}

Using expression \eqref{onepart23m} we can compute $P_{11}^{ij}\equiv | \omega^i_\text{A},s^i_\text{A};\omega^i_\text{R},s^i_\text{R}\rangle_\text{U}\langle \omega^j_\text{A},s^j_\text{A};\omega^j_\text{R},s^j_\text{R}|_\text{U}$ in the Rindler basis for Rob
\begin{align}\label{1111}
 \nonumber P_{11}^{ij}&=\sum_{m=0}^{2n-1}\sum_{l=0}^{2n-1}A^m(A^l)^*\sum_{\substack{s_1,\dots,s_m\\\omega_1,\dots,\omega_m}}\!\xi_{s_1,\dots,s_m,s^i_\text{R}}^{\omega_1,\dots,\omega_m,\omega^i_\text{R}}\sum_{\substack{s'_1,\dots,s'_{l}\\\omega_1,\dots,\omega'_{l}}}\!\xi_{s'_1,\dots,s'_l,s^j_\text{R}}^{\omega'_1,\dots,\omega'_l,\omega^j_\text{R}} \ket{\omega^i_\text{A},s^i_\text{A}}_{\text{U}}\biket{\tilde m;\omega^i_\text{R},s^i_\text{R}}{\tilde m}\\*
 &\times\langle \tilde l|_{\text{II}}\langle \tilde l;\omega^j_\text{R},s^j_\text{R}|_\text{I}\langle{\omega^j_\text{A},s^j_\text{A}}|_\text{A},
\end{align}
where $A^m$ is given by \eqref{Am}.

Notice that the objects $\ket{\tilde m;\omega^i_\text{R},s^i_\text{R}}_\text{I}$ represent the appropriate ordering of the elements inside with its sign, taking the criterion \eqref{ordering} into account.

Now we can use expressions \eqref{vacuumCOMP2} and \eqref{onepart23m} to obtain the operator $P_{01}\equiv\proj{00}{\omega_A,s_A;\omega_\text{R},s_\text{R}}_\text{U}$ in the Rindler basis for Rob.
\begin{equation}\label{0011}
 P_{01}=\sum_{m=0}^{2n}\sum_{l=0}^{2n-1}C^m (A^{l})^*\!\!\!\sum_{\substack{s_1,\dots,s_m\\\omega_1,\dots,\omega_m}}\!\xi_{s_1,\dots,s_m}^{\omega_1,\dots,\omega_m}\sum_{\substack{s'_1,\dots,s'_{l}\\\omega_1,\dots,\omega'_{l}}}\!\xi_{s'_1,\dots,s'_l,s_\text{R}}^{\omega'_1,\dots,\omega'_l,\omega_\text{R}}\ket{0}_{\text{U}}\biket{\tilde m}{\tilde m}\langle{\tilde l}|_{\text{II}}\langle{\tilde l;\omega_\text{R},s_\text{R}}|_{\text{I}}\bra{0}_{\text{U}}.
\end{equation}

After obtaining the expressions for the operators $P_{00},P_{11},P_{01}$ we can write the density matrix associated with the state \eqref{minkowstate1} in the Rindler basis for Rob, 
\begin{equation}\label{roimp1}
\rho=\frac12\left(P_{00}+P_{01}+P_{01}^\dagger+P^{ii}_{11}\right)
\end{equation}
Where for $P_{11}^{ii}$ we are considering $\{\omega^i_\text{R},s^i_\text{R}\}=\{\omega^j_\text{R},s^j_\text{R}\}\equiv \{\omega_\text{R},s_\text{R}\}$ and $\{\omega^i_\text{A},s^i_\text{A}\}=\{\omega^j_\text{A},s^j_\text{A}\}\equiv \{\omega_A,s_A\}$.

We can do the same to obtain the density matrix associated with \eqref{minkowstate2} in the Rindler basis for Rob
\begin{equation}\label{roimp2}
\rho=\frac12\left(P^{11}_{11}+P^{22}_{11}+P^{12}_{11}+(P^{12}_{11})^\dagger\right).
\end{equation}

Now, we must consider that, as Rob is causally disconnected from Ridler's region II, we should trace over that region to obtain Rob's density matrix. Hence,  we need to compute the trace over II for each of the previous operators \eqref{0000}, \eqref{1111}, \eqref{0011}.

Taking this trace is actually quite straightforward taking into account the orthonormality of our basis once we have chosen one particular ordering criterion \eqref{ordering}, 
\begin{equation}\label{products}
\braket{\tilde m}{\tilde m'}_\text{II}=\delta_{mm'}\left(\delta_{s_1,s'_1}\delta_{\omega_1,\omega'_1}\dots\delta_{s_m,s'_m}\delta_{\omega_m,\omega'_m}\right).
\end{equation} 
Hence,
\begin{equation}\label{traza00}
\tr_{\text{II}} P_{00}=\sum_{m'=0}^{2n}\bra{\tilde m'}_\text{II}P_{00}\ket{\tilde m'}_\text{II}.
\end{equation}
Using \eqref{products} only the diagonal elements in region II survive and \eqref{traza00} turns out to be
\begin{equation}\label{t00pre}
\tr_{\text{II}} P_{00}=\sum_{m=0}^{2n}|C^m|^2 \!\!\!\sum_{\substack{s_1,\dots,s_m\\\omega_1,\dots,\omega_m}}\!\!\!\!\xi_{s_1,\dots,s_m}^{\omega_1,\dots,\omega_m}\ket{0}_{\text{U}}\ket{\tilde m}_{\text{I}}\langle{\tilde m}|_{\text{I}}\bra{0}_{\text{U}},
\end{equation}
which, substituting $C^m$ as a function of $C^0$ using \eqref{coeff2} and then \eqref{Des}, is expressed as
\begin{equation}\label{t00}
\tr_{\text{II}} P_{00}=\sum_{m=0}^{2n}D_0^m \!\!\!\sum_{\substack{s_1,\dots,s_m\\\omega_1,\dots,\omega_m}}\!\!\!\!\xi_{s_1,\dots,s_m}^{\omega_1,\dots,\omega_m}\ket{0}_{\text{U}}\ket{\tilde m}_{\text{I}}\langle{\tilde m}|_{\text{I}}\bra{0}_{\text{U}}.
\end{equation}

Now, let us compute the trace of $P^{ij}_{11}$ over region II:
\begin{equation}\label{traza11}
\tr_{\text{II}} P^{ij}_{11}=\sum_{m'=0}^{2n}\bra{\tilde m'}_\text{II}P^{ij}_{11}\ket{\tilde m'}_\text{II},
\end{equation}
\begin{equation}\label{t11pre}
 \tr_{\text{II}} P_{11}^{ij}=\sum_{m=0}^{2n-1}|A^m|^2 \!\!\!\sum_{\substack{s_1,\dots,s_m\\\omega_1,\dots,\omega_m}}\xi_{s_1,\dots,s_m,s^j_\text{R}}^{\omega_1,\dots,\omega_m,\omega^j_\text{R}}\ket{\omega_A,s_A}_{\text{U}}\ket{\tilde m;\omega_\text{R},s_\text{R}}_{\text{I}}\langle{\tilde m;\omega_\text{R},s_\text{R}}|_{\text{I}}\bra{\omega_A,s_A}_{\text{U}}.
\end{equation}
Substituting $C^m$ as a function of $C^0$ (combining \eqref{Am} and \eqref{coeff2}) we can express
\begin{equation}
|A^m|^2=|C^0|^2 \tan^{2m}r\left(\cos r+\frac{\sin^2 r}{\cos r}\right)^2=|C^0|^2 \frac{\tan^{2m}r}{\cos^2 r}=D^m_2
\end{equation}
such that we obtain
\begin{equation}\label{t11}
\tr_{\text{II}} P^{ii}_{11}=\sum_{m=0}^{2n-1}D^m_2 \!\!\!\sum_{\substack{s_1,\dots,s_m\\\omega_1,\dots,\omega_m}}\!\xi_{s_1,\dots,s_m,s_\text{R}}^{\omega_1,\dots,\omega_m,\omega_\text{R}}\ket{\omega^i_\text{A},s^i_\text{A}}_{\text{U}}\ket{\tilde m;\omega^i_\text{R},s^i_\text{R}}_{\text{I}}\langle{\tilde m;\omega^j_\text{R},s^j_\text{R}}|_{\text{I}}\langle{\omega^j_\text{A},s^j_\text{A}}|_{\text{U}}.
\end{equation}
When $\{\omega^i_\text{R},s^i_\text{R}\}=\{\omega^j_\text{R},s^j_\text{R}\}$ $\equiv$ $\{\omega_\text{R},s_\text{R}\}$, $\{\omega^i_\text{A},s^i_\text{A}\}=\{\omega^j_\text{A},s^j_\text{A}\}\equiv \{\omega_A,s_A\}$, and
\begin{equation}\label{t112}
\tr_{\text{II}} P^{ij}_{11}=\sum_{m=0}^{2n-1}D^m_2 \!\!\!\sum_{\substack{s_1,\dots,s_m\\\omega_1,\dots,\omega_m}}\!\xi_{s_1,\dots,s_m,s^i_\text{R},s^j_\text{R}}^{\omega_1,\dots,\omega_m,\omega^i_\text{R},\omega^j_\text{R}}\ket{\omega^i_\text{A},s^i_\text{A}}_{\text{U}}\ket{\tilde m;\omega^i_\text{R},s^i_\text{R}}_{\text{I}}\langle{\tilde m;\omega^j_\text{R},s^j_\text{R}}|_{\text{I}}\langle{\omega^j_\text{A},s^j_\text{A}}|_{\text{U}}
\end{equation}
for $i\neq j$.

Now, let us compute the trace
\begin{equation}\label{traza01pre}
\tr_{\text{II}} P_{01}=\sum_{m'=0}^{2n}\bra{\tilde m'}_\text{II}P_{01}\ket{\tilde m'}_\text{II},
\end{equation}
\begin{equation}\label{traza01pre2}
\tr_{\text{II}} P_{01}=\sum_{m=0}^{2n-1}C^m (A^{m})^*\!\!\!\sum_{\substack{s_1,\dots,s_{m}\\\omega_1,\dots,\omega_{m}}}\!\xi_{s_1,\dots,s_m,s_\text{R}}^{\omega_1,\dots,\omega_m,\omega_\text{R}}\ket{0}_{\text{U}}\ket{\tilde m}_\text{I}\langle{\tilde l;\omega_\text{R},s_\text{R}}|_{\text{I}}\bra{0}_{\text{U}}.
\end{equation}
From \eqref{Am} and \eqref{coeff2} we see that the product $C^m(A^m)^*$ is real and has the expression
\begin{equation}
C^m(A^m)^*=|C^0|^2\tan^{2m}r\left(\cos r+\frac{\sin^2 r}{\cos r}\right)=|C^0|^2\frac{\tan^{2m}r}{\cos r}=D^m_1
\end{equation}
so that
\begin{equation}\label{traza01}
 \tr_{\text{II}} P_{01}=\sum_{m=0}^{2n-1}D^m_1\!\!\!\sum_{\substack{s_1,\dots,s_{m}\\\omega_1,\dots,\omega_{m}}}\!\xi_{s_1,\dots,s_m,s_\text{R}}^{\omega_1,\dots,\omega_m,\omega_\text{R}}\ket{0}_{\text{U}}\ket{\tilde m}_\text{I}\langle{\tilde l;\omega_\text{R},s_\text{R}}|_{\text{I}}\bra{0}_{\text{U}}.
\end{equation}

Now we can compute Rob's density matrices for each case tracing over II in expressions \eqref{roimp1} and \eqref{roimp2}. First the matrix \eqref{roimp1} is, after tracing over II, 
\begin{equation}
\tr_{\text{II}}\rho=\frac12\left(\tr_{\text{II}}P_{00}+\tr_{\text{II}}P_{01}+\tr_{\text{II}}P_{01}^\dagger+\tr_{\text{II}}P^{ii}_{11}\right).
\end{equation}
Substituting expressions \eqref{t00}, \eqref{t112}, \eqref{traza01} we get expression \eqref{densmat1}.

Now, concerning \eqref{roimp2}
\begin{equation}
\tr_{\text{II}} \rho=\frac12\tr_{\text{II}}\left( P^{11}_{11}+P^{22}_{11}+P^{12}_{11}+(P^{12}_{11})^\dagger\right).
\end{equation}
Substituting expressions \eqref{t11} and \eqref{t112} we obtain expression \eqref{densmat2}.

The derivation of \eqref{densmat3} is completely analogous to \eqref{densmat1}, taking now into account that we have $\hat C^m$ and $\hat D^m$ instead of $C^m$ and $D^m$ and that we have no spin degree of freedom. Notice that, even though the structure of \eqref{densmat3} is completely analogous to the structure of \eqref{densmat1}, and therefore, repeating the derivation will add nothing to this appendix,  these density matrices are completely different due to the different dimensions, the different values of $\hat C^0$ and $C^0$ and the number of $2\times 2$ blocks which give negative eigenvalues. 



\cleardoublepage
