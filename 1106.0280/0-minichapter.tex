
\chapter{Quantum entanglement and mutual information}\label{QIi}

This chapter is intended as a very brief presentation (almost merely notational) to the concept of quantum entanglement. The motivation for this section is to introduce the concept and to present two state functionals that we will use to measure the `amount' of entanglement and correlations of a bipartite quantum system. Nevertheless, the matter of entanglement is not a simple topic. Its study is itself  a huge, and still open, discipline. For a more detailed view there are many other sources where a much more thorough study can be found, for instance \cite{Nichuang}. 

\section{Quantum entanglement and entanglement measures}\label{entangsec}

Quantum entanglement is a feature of some multipartite quantum systems which is strongly related with non-locality. Basically, to describe entangled $k$-partite systems in quantum mechanics it is not enough with the description of the $k$ individual quantum states for each subsystem, even if the subsystems are spatially separated. This means that carrying out measurements on one of the subsystems we can gather information about the result of future measurements on any of the rest of the subsystems without directly acting on them beyond the limits imposed by classical physics \cite{Bellprime}.

 Quantum entanglement was central in the debate about the non-locality and completeness of quantum mechanics \cite{EPR0} which ended up with the banishing of local hidden-variable theories \cite{EPRtest}. More important, quantum entanglement is the principal resource for quantum information tasks such as quantum teleportation \cite{telep1} and quantum computing \cite{Nichuang} and, as we will discuss in this thesis, can be used to obtain information about quantum effects provoked by gravity.

In the case of pure states, if we have two quantum systems A and B and the Hilbert spaces for the states of these systems are $\mathcal{H}_\text{A}$ and $\mathcal{H}_\text{B}$ respectively, the Hilbert space of the composite system is the tensor product $\mathcal{H}_\text{A}\otimes\mathcal{H}_\text{B}$.  A bipartite state $\ket{\Psi}_{\text{AB}}$ is entangled when it is not possible to express $\ket{\Psi}_{\text{AR}}$ as the tensor product of states for the individual subsystems
\begin{equation}
\ket{\Psi}_{\text{AB}}\neq\ket{\phi}_\text{A}\otimes\ket\phi_{\text{B}}\Leftrightarrow \ket{\Psi}_{\text{AB}}\, \text{ Entangled.}
\end{equation}
In other words, if $\{\ket{j}_\text{A}\}$ and $\{\ket{k}_\text{B}\}$ are respectively bases of  $\mathcal{H}_\text{A}$ and $\mathcal{H}_\text{B}$ the most general bipartite state in  $\mathcal{H}_\text{A}\otimes\mathcal{H}_\text{B}$ has the form
\begin{equation}
\ket{\Psi}_{\text{AB}}=\sum_{j,k}c_{jk}\ket{j}_\text{A}\otimes\ket{k}_\text{B}.
\end{equation}
The state is separable when
 \begin{equation}\label{consep}
 c_{jk}=c_j^\text{A}c_k^\text{B},  
\end{equation}
yielding
\b
\left.
\begin{array}{l}
\displaystyle{\ket{\phi}_\text{A}=\sum_jc_j^\text{A}\ket{j}_\text{A}}\\[5mm]
\displaystyle{\ket{\phi}_\text{B}=\sum_kc_k^\text{B}\ket{k}_\text{B}}
\end{array}\right\}
\Rightarrow \ket{\Psi}_{\text{AB}}=\ket{\phi}_\text{A}\otimes\ket\phi_{\text{B}}.\e
If condition \eqref{consep} does not hold the state is entangled.


For mixed states the general definition is slightly more complicated. A general state is entangled if, and only if, it cannot be expressed as a probability distribution of the uncorrelated individual states. In other words, given a set of positive numbers $\{p_i\}$ such that $\sum_{i}p_i=1$ then
\begin{equation}
\rho_{AB}\neq \sum_{i}p_i\,\rho^A_i\otimes\rho^B_i  \Leftrightarrow \rho_{AR}\, \text{ Entangled.}
\end{equation}

Although determining if a state is entangled or not is conceptually simple, computationally speaking is a very hard problem for general states of arbitrary dimension. Actually there is no such thing as a unique measure of entanglement. Instead, a measure of entanglement is any positive function of the state $E(\rho)$ which satisfy the following axioms
\begin{itemize}
\item Must be maximum for maximally entangled states (Bell states)
\item Must be zero for separable states.
\item Must be non-zero for all non-separable states.
\item Must not grow under LOCC (Local Operations + Classical Communication)
\end{itemize}

For pure states the entanglement entropy (entropy of the reduced states of $A$ or $B$) is a natural measure of entanglement which have also a well understood physical interpretation, but it does not fulfill the previous axioms for non-pure states.

To account for the entanglement of general states let us introduce the partial transpose density matrix. For a general density matrix of a bipartite system $AB$
\begin{equation}
\rho_{AB}=\sum_{ijkl}\rho_{ijkl}\ket{i}_A\ket{j}_B\bra{k}_A\bra{l}_B,
\end{equation}
the partial transpose is defined as
\begin{equation}\label{ptranspdef}
\rho_{AB}^{p{T_B}}=\sum_{ijkl}\rho_{ijkl}\ket{i}_A\ket{l}_B\bra{k}_A\bra{j}_B
\end{equation}
or, equivalently for our purposes, as 
\begin{equation}
\rho_{AB}^{p{T_A}}=\sum_{ijkl}\rho_{ijkl}\ket{k}_A\ket{j}_B\bra{i}_A\bra{l}_B.
\end{equation}

There is a theorem for the lower dimensional cases, for bipartite systems of dimension $2\times 2$ (two-qubit states) and $3\times2$ (qutrit-qubit states) the well-known Peres criterion \cite{PeresCriterion} guarantees that a state is non-separable (and therefore, entangled) if, and only if, the partial transposed density matrix has, at least, one negative eigenvalue.

Unfortunately, for higher dimension the condition is no longer necessary and sufficient, but only sufficient due to the existence of bound entanglement: there are states which are entangled, but no pure entangled states can be obtained from them by means of local operations and classical communication (LOCC). Such states are called bound entangled states \cite{Bound} and its entanglement is of no utility to quantum information tasks. Peres criterion only accounts for the existence of entanglement that can be distilled and therefore useful to perform quantum information tasks. In this thesis we will only be interested in distillable entanglement so in principle we will not need to worry about the existence or not of bound entanglement.

Based on Peres criterion a number of entanglement measures have been introduced. In this thesis we have used negativity ($\mathcal{N}$) to account for the quantum correlations between the different bipartitions of the system \cite{Negat}. It is an entanglement monotone sensitive to distillable entanglement defined as the sum of the negative eigenvalues of the partial transpose density matrix, in other words, if $\sigma_i$ are the eigenvalues of any $\rho^{pT}_{AB}$ then
\begin{equation}\label{negativitydef}
\mathcal{N}_{AB}=\frac12\sum_{ i}(|\sigma_i|-\sigma_i)=-\sum_{\sigma_i<0}\sigma_i.
\end{equation}
The minimum value of negativity is zero (for states with no distillable entanglement) and its maximum (reached for maximally entangled states) depends on the dimension of the maximally entangled state. Specifically, for qubits $\mathcal{N}_{AB}^{\text{max}}=1/2$.

\section{Mutual information}\label{mutusec}

The mutual information of two random variables $(X,Y)$ is a function of these two variables that measures how much uncertainty about one of the variables is reduced by our knowledge about the other. It accounts for the correlations between the two variables.

Given two random variables $(X,Y)$ the mutual information $I_{XY}$ is defined as
\begin{equation}
I(X,Y)=H(X)+H(Y)-H(X,Y),
\end{equation}
where $H(X),H(Y)$ are the marginal Shannon entropies and $H(X,Y)$ the joint entropy defined as
\begin{align}
H(X,Y)&=-\sum_{x,y}P(x,y)\log_2\left[P(x,y)\right],\\*
H(X)&=-\sum_{x}P(x)\log_2\left[P(x)\right],\\*
H(Y)&=-\sum_{y}P(y)\log_2\left[P(y)\right],
\end{align}
where $P(x,y)$ is the joint probability distribution of the random variables $X,Y$ and
\begin{equation}
p(x)=\sum_yP(x,y),\qquad p(y)=\sum_xP(x,y)
\end{equation}
are the marginal probability distributions for $X$ and $Y$.

For a quantum bipartite system of density matrix $\rho_{\text{AB}}$ the quantum mutual information is expressed in terms of the Von Neumann Entropy
\begin{equation}
I_{\text{AB}}=S_\text{A}+S_\text{B}-S_{\text{AB}},
\end{equation}
where the Von Neumann entropies are\footnote{The $\log_2$ is chosen to be base 2 because in quantum information it is common to work with qubits, but any other basis can be chosen instead.}
\begin{align}
S_{\text{AB}}&=-\tr_{\text{AB}}\left(\rho_{\text{AB}}\log_2\rho_{\text{AB}}\right),\\*
S_{\text{A}}&=-\tr_{\text{A}}\left(\rho_{\text{A}}\log_2\rho_{\text{A}}\right),\\*
S_{\text{B}}&=-\tr_{\text{B}}\left(\rho_{\text{B}}\log_2\rho_{\text{B}}\right),
\end{align}
and the partial systems are $\rho_\text{A}=\tr_\text{B}\left(\rho_\text{AB}\right)$, $\rho_\text{B}=\tr_\text{A}\left(\rho_\text{AB}\right)$.

Mutual information accounts for both, classical and quantum correlations, so that it can be used together with an entanglement measure to distinguish the behaviour of classical correlations: in a system which has no quantum correlations, mutual information accounts exclusively for classical correlations.


