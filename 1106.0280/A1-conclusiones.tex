%%%%%%%%%%%%%%%%%%%%%%%%%%%%%%%%%%%%%
{\renewcommand{\thechapter}{}\renewcommand{\chaptername}{}
\addtocounter{chapter}{0}
\chapter*{Conclusiones}\markboth{\sl CONCLUSIONES}{\sl CONCLUSIONES}}
%%%%%%%%%%%%%%%%%%%%%%%%%%%%%%%%%%%%%
\addcontentsline{toc}{part}{Conclusiones}

La motivación principal de esta tesis es extender la aplicabilidad de la cosmología
cuántica de lazos
a situaciones con complejidad suficiente como para permitir la extracción de predicciones físicas
realistas. Con el fin de progresar en esta dirección, hemos analizado la cuantización de diversos
modelos con un orden jerárquico de complejidad, proporcionado por la inclusión de anisotropías e
inhomogeneidades.

\section*{Resultados específicos}

\begin{list}{\labelitemi}{\leftmargin=1em}
 \item Hemos propuesto una nueva prescripción de simetrización y de densitización para el operador
ligadura hamiltoniana, tal que en todos los modelos estudiados:
\begin{itemize}
\item[(i)] El operador resultante desacopla los estados de volumen (homogéneo) nulo. Gracias a
ello, el problema de la singularidad cosmológica inicial queda resuelto cuánticamente incluso sin
restringirse a estados físicos.
\item[(ii)] Dicho operador deja invariantes espacios de Hilbert simples, que son generados
por autoestados de las componentes (homogéneas) de la tríada densitizada, cada una de ellas con una
orientación fija. Como consecuencia, los estados que contienen la información acerca de la
discretización de la geometría verifican una descripción de tipo {ausencia de frontera}, ya
que surgen en una sola sección de volumen (homogéneo) mínimo no nulo. Dichos espacios de
Hilbert proporcionan sectores de superselección que a su vez son simples y tienen propiedades
físicas óptimas.
\end{itemize}
\item En el modelo de FRW acoplado a un campo escalar sin masa, hemos demostrado que los estados
que encierran la información sobre la geometría cuántica convergen a una onda estacionaria
\emph{exacta} en el límite de volumen grande. Este comportamiento, junto con la descripción de tipo
{ausencia de frontera}, muestra la existencia de un rebote cuántico genérico
que reemplaza dinámicamente la singularidad clásica. Por tanto, en este contexto, hemos
demostrado que el mecanismo de resolución de singularidades
mediante un rebote cuántico es una consecuencia directa de los efectos discretos de la geometría que
subyacen en cosmología cuántica de lazos y, por consiguiente, que es una propiedad fundamental de la
teoría.
\item En el modelo de Bianchi I en vacío y en el denominado esquema A para la dinámica mejorada
(esquema factorizable),
el estudio se reduce esencialmente a considerar tres copias del sector gravitacional del modelo de
FRW. Como
consecuencia, los estados que codifican aquí la información sobre la geometría cuántica poseen el
mismo tipo de comportamiento que los análogos del modelo de FRW: descripción de tipo {ausencia
de frontera} y límite de WDW de tipo onda estacionaria. Por tanto, una vez más, existen
rebotes cuánticos genéricos que curan las singularidades.
\item En el modelo de Bianchi I en vacío y en el esquema B (no factorizable), hemos demostrado que
no solo el volumen
toma valores en conjuntos discretos, sino también las variables que representan las anisotropías.
Éstas, sin embargo, no poseen realmente un mínimo, aunque sí un ínfimo nulo. Más aún, en nuestro
análisis,
restringido al sector en el que los autovalores de los operadores
que representan todas las componentes de la tríada densitizada son estrictamente positivos, el
rango de valores de las anisotropías cubre densamente la semirrecta real positiva.
\item Para este modelo, hemos argumentado que los datos iniciales de la sección de volumen mínimo
(no nulo) determinan completamente las soluciones de la ligadura hamiltoniana, y hemos
caracterizado el espacio de Hilbert físico como el espacio de Hilbert de los datos iniciales, cuya
estructura, a su vez, ha sido determinada imponiendo
condiciones de realidad en un conjunto (super) completo de observables físicos.
\item Usando el modelo de Bianchi I en vacío como ejemplo, hemos sido pioneros en analizar y
desarrollar metodológicamente la imagen de evolución física en cosmología cuántica de lazos usando
como tiempo interno alguno de los grados de libertad de la geometría. En particular, hemos llevado a
cabo dos construcciones distintas de observables relacionales o completos, en las que la variable
del espacio de fases que desempeña la función de tiempo interno es, respectivamente: $(i)$ un
parámetro afin asociado a uno de los coeficientes de la tríada densitizada, $(ii)$ su momento
canónicamente conjugado.
\item En ambas construcciones, hemos llevado a cabo un análisis numérico de los valores esperados y
de las dispersiones de los observables, evaluados en estados semiclásicos con perfil gaussiano, que
revela las siguientes predicciones:
\begin{itemize}
 \item[(a)] Los estados gaussianos permanecen picados a lo largo de toda la evolución, en el sentido
de que las dispersiones de los observables se mantienen acotadas.
\item[(b)] En los regímenes de volumen grande, las trayectorias definidas por los valores esperados
de los observables coinciden con las predichas por la relatividad general.
\item[(c)] A medida que las trayectorias se acercan a la ubicación de la singularidad, la dinámica
cuántica se desvía con respecto a la clásica y aparecen rebotes cuánticos que evitan la
singularidad. 
\item[(d)] Además, el límite de WDW de tipo onda estacionaria de las autofunciones de la
cosmología cuántica de lazos nos ha permitido analizar la región asintótica que el cálculo
numérico no permite explorar. Hemos demostrado que el comportamiento semiclásico de estados físicos
muy generales (no solo correspondientes a perfiles gaussianos) se conserva a través del rebote
cuántico.
\end{itemize}
Por el contrario, el mismo tipo de análisis aplicado a la cuantización de WDW del modelo
muestra que esta teoría no cura las singularidades.
\item En el contexto particular del modelo de Gowdy $T^3$, hemos propuesto un procedimiento híbrido,
que combina las cuantizaciones de lazos y de Fock, para tratar la cuantización de cosmologías
inhomogéneas. La cuantización de lazos del sector homogéneo es suficiente para curar la singularidad
cosmológica inicial incluso en el espacio de Hilbert cinemático.
\item Hemos obtenido un operador ligadura hamiltoniana bien definido en un dominio
denso y que conserva los sectores de superselección obtenidos en el modelo de Bianchi I, que, por
tanto, proporcionan en este sistema el sector homogéneo de espacios de Hilbert separables que
también están superseleccionados.
\item Como en modelos anteriores, la ligadura hamiltoniana da lugar a una ecuación en
diferencias en un parámetro interno discreto que tiene un valor mínimo estrictamente positivo.
Hemos demostrado que las soluciones de la ligadura están unívocamente determinadas por los datos
proporcionados en la sección inicial de tal parámetro. Identificando datos iniciales con soluciones,
hemos podido caracterizar el espacio de Hilbert físico como el espacio de Hilbert de los datos
iniciales, cuyo producto interno queda fijado al imponer condiciones de realidad en un conjunto
adecuado de observables.
\item El espacio de Hilbert físico obtenido es el producto tensorial del espacio de Hilbert físico
del modelo de Bianchi I (el correspondiente a cada esquema, A o B) por un espacio de Fock
unitariamente equivalente al obtenido en la cuantización estándar del modelo deparametrizado.
Por tanto, recuperamos la teoría cuántica de campos estándar para las inhomogeneidades, que pueden
enterderse como grados de libertad que se propagan sobre un fondo de tipo Bianchi I.
\item En definitiva, hemos sido capaces de mejorar las técnicas de la cosmología cuántica de lazos
ya antes desarrolladas en contextos homogéneos y de completar la cuantización de cosmologías
inhomogéneas por primera vez dentro del marco de la teoría de lazos.
\end{list}

 
\cleardoublepage


