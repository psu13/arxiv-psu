%%%%%%%%%%%%%%%%%%%%%%%%%%%%%%%%%%%%%
{\renewcommand{\thechapter}{}\renewcommand{\chaptername}{}
\addtocounter{chapter}{0}
\chapter*{Abstract}\markboth{\sl ABSTRACT}{\sl ABSTRACT}}
%%%%%%%%%%%%%%%%%%%%%%%%%%%%%%%%%%%%%
\addcontentsline{toc}{chapter}{Abstract}

In spite of the impressive progress that cosmology has experienced in recent years, we are still
missing a consistent explanation of the origin of the Universe and the formation of structures,
which should be deduced entirely from a fundamental theory. General Relativity breaks down in the
very initial instants of the history of the Universe, leading to a cosmological singularity of the
big bang type. In this regime General Relativity cannot be trusted, and the very own predictability
of the laws of physics is lost. One expects instead that the physics of the Early Universe
belongs to the realm of Quantum Gravity, namely, a theory of the gravitational field which
incorporates the quantum behavior of nature. One of the most promising candidates for such a theory
is Loop Quantum Gravity. At present, important efforts are being made in order to adapt the
techniques of Loop Quantum Gravity to much simpler settings than those of the complete theory, which
on the other hand remains to be concluded. This is the case of a series of homogeneous cosmological
models obtained from General Relativity by symmetry reduction. The resulting field of research is
known under the general name of Loop Quantum Cosmology (LQC).

As a necessary step towards the extraction of realistic results from LQC, we should consider the
inclusion of inhomogeneities, which play a central role in current cosmology. The main goal of this
thesis is to progress in this direction. With this aim we have studied one of the simplest
inhomogeneous cosmological models, namely the linearly polarized Gowdy $T^3$ model. This model is a
natural test bed to incorporate inhomogeneities in LQC. On the one hand, its quantization by means
of standard techniques has been discussed in detail, and a successful Fock quantization has
already been achieved. On the other hand the subset of its homogeneous solutions describes the
Bianchi I model in vacuo with three-torus topology, and the Bianchi I model has been already
considered within LQC.

We have attained a thorough quantization of this Gowdy model, in which the cosmological singularity
is resolved, by means of a hybrid quantization. This combines the polymeric quantization
characteristic of LQC applied to the homogeneous sector of the (partially reduced) phase space,
which is formed by the set of degrees of freedom that describe the homogeneous Bianchi I
solutions, with a Fock quantization for the inhomogeneities. This approach investigates the effects
of quantum geometry underlying LQC only on the homogeneous sector, while disregards the discreteness
of the geometry encoded by the inhomogeneities. A most natural treatment for the inhomogeneities is
then the Fock quantization. Indeed, one would expect that a quantum field theory for the
inhomogeneities, which can be regarded as a field living on a homogeneous (Bianchi~I) background, be
approximately valid on the polymerically quantized background. 

In order to carry out the loop quantization of the homogeneous sector in the Gowdy model as
rigorously
as possible, we have needed to revisit the very foundations of LQC. We have first reviewed the flat,
homogeneous, and isotropic Friedmann-Robertson-Walker (FRW) universe with a massless scalar field,
which is
a paradigmatic model in LQC. In spite of the prominent role that the model has played in the
development of LQC, there still remained some aspects of its quantization which deserved a more
detailed discussion. These aspects included the kinematical resolution of the cosmological
singularity, the precise relation between the solutions of the densitized and non-densitized
versions of the quantum Hamiltonian constraint, the possibility of identifying superselection
sectors which are as simple as possible, and a clear comprehension of the Wheeler-DeWitt (WDW) limit
associated with
the theory in those sectors. We have revised its quantization by proposing an alternative
prescription when representing the Hamiltonian constraint. Our proposal leads to a
symmetric Hamiltonian constraint operator, which is specially suitable to deal
with all these issues in a detailed and satisfactory way. In particular, with our constraint
operator, the singularity decouples in the kinematical Hilbert space and can be removed already at
this level. Thanks to this fact, we can densitize the quantum Hamiltonian constraint in a
well-controlled manner. Besides, together with the physical observables, this constraint
superselects simple sectors for the universe volume, with a discrete support contained in a single
semiaxis of the real line and for which the basic functions that encode the information about the
geometry possess optimal physical properties: They provide a no-boundary description around
the cosmological singularity and admit a well-defined WDW limit in terms of standing waves.
Both properties explain the presence of a generic quantum bounce replacing the classical singularity
at a fundamental level, in contrast with previous studies where the bounce was proved in concrete
regimes.

Following our program to quantize the Gowdy $T^3$ model, and since the homogeneous sector of this
system coincides with the Bianchi~I model in vacuo with spatial sections of three-torus topology, we
have also faced the quantization of these anisotropic cosmologies in LQC. The implementation of the
so-called \emph{improved dynamics} prescription, which was successfully established for isotropic
situations, gives rise to two different schemes in the presence of anisotropies. The original and
simplest one is a naive adaptation of the isotropic scheme employed for FRW. In turn, the most
recently proposed scheme leads to a more complicated quantum theory. We have investigated the loop
quantization of the Bianchi I model within both
schemes. When representing the Hamiltonian constraint operator, we have followed the same kind of
prescription as in the isotropic FRW model. As a consequence, once again the Hamiltonian constraint
operator leaves suitable subspaces of the kinematical Hilbert space invariant, whose
supports are each contained in a single octant of $\mathbb{R}^3$. This has allowed us to
complete the quantization within both schemes, providing the corresponding physical Hilbert space
and a complete set of physical observables. In both cases, the physical Hilbert space is
superselected in separable sectors. We have analyzed the structure of the resulting superselection
sectors, being those of the second scheme richer and more interesting.

On the other hand, in the vacuum Bianchi I model, and for the simplest scheme of the improved
dynamics, we have explicitly determined the form of the physical solutions to the Hamiltonian
constraint in
terms of elements of the kinematical Hilbert space used to carry out the quantization. This
knowledge makes this model a most appropriate arena to investigate the concept of physical
evolution in LQC in the absence of the massless scalar field which has been used so far in the
literature as an internal time. In order to retrieve the system dynamics when no such a suitable
clock field is present, we have explored different constructions of families of unitarily related
partial observables. These observables are essentially parameterized, respectively, by: $(i)$ one of
the components of the densitized triad, and $(ii)$ its conjugate momentum; each of them
playing the role of an evolution parameter. In order to describe the construction in a simpler
setting, and also for comparison, we have first carried out the analysis in the WDW
quantization of the model, and shown that singularities persist in this quantum approach. In the
loop quantized model the construction is more involved owing to the polymeric nature of the
geometry. Exploiting the properties of the considered example, we have investigated in detail the
domains of applicability of each construction. In both cases the observables possess a neat physical
interpretation only in an approximate sense. However, whereas in case $(i)$ such interpretation is
reasonably  accurate only for a portion of the evolution of the universe, in case $(ii)$ it remains
so during all the evolution (at least in the physically interesting cases). The constructed families
of observables have been next used to describe the evolution of the Bianchi I universe. The
performed
analysis confirms the robustness of the bounces, also in absence of matter fields, as well as the
preservation of the semiclassicality through them. The concept of evolution studied here and the
presented construction of observables are applicable to a wide class of models in LQC, including
the quantization of the Bianchi I model obtained with the other scheme for the improved
dynamics.

Finally, we have implemented the loop quantization of the Bianchi I model corresponding to
each of the two schemes for the improved dynamics in the representation of the homogeneous sector
of the hybrid Gowdy model. For both schemes, the resulting Hamiltonian constraint preserves the
sectors superselected in the Bianchi I model. Thanks to the nice features of these superselection
sectors, we have been able to complete the quantization also in this inhomogeneous system, even
though the homogeneous and inhomogeneous sectors are coupled in a non-trivial way, and now we are
dealing with an infinite number of degrees of freedom. The polymeric quantization of the homogeneous
sector suffices to cure quantum-mechanically the cosmological singularity.
The resulting physical Hilbert space has the
tensor product structure of the physical Hilbert space of the Bianchi I model times a Fock space
which is
unitarily equivalent to the physical space of the conventional Fock quantization of the
deparametrized system. Therefore, we indeed recover the standard quantum field theory for the
inhomogeneities.

In conclusion, we have studied a series of models in the framework of LQC with increasing
complexity, provided by the inclusion of anisotropies and inhomogeneities. We have been able
to improve the LQC techniques that were already developed in homogeneous settings and
complete the quantization of inhomogeneous cosmologies for the first time in the context of LQC.
\cleardoublepage