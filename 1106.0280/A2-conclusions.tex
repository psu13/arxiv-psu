%%%%%%%%%%%%%%%%%%%%%%%%%%%%%%%%%%%%%
{\renewcommand{\thechapter}{}\renewcommand{\chaptername}{}
\addtocounter{chapter}{0}
\chapter*{Conclusions}\markboth{\sl CONCLUSIONS}{\sl CONCLUSIONS}}
%%%%%%%%%%%%%%%%%%%%%%%%%%%%%%%%%%%%%
\addcontentsline{toc}{part}{Conclusions}


This thesis is centred in the study of entanglement and quantum information problems in the background of general relativistic settings. In our exploration  of this brand new field called relativistic quantum information  we have obtained results in three different categories:

\begin{itemize}
\item[--]On the fundamental side, analysing the impact of statistics (fermionic or bosonic) on the behaviour of field entanglement in non-inertial frames; building the formalism to deal with entanglement of different degrees of freedom (spin, occupation number, ...) and studying how correlations behave in the proximities of an event horizon. 
\item[--]On the applied side, showing how entanglement can be useful to study the expansion of the Universe, the process of stellar collapse or to serve as a witness of the Unruh and Hawking effect which have not been detected yet.
\item[--]On the experimental proposals side, using the knowledge gained in quantum information to suggest experiments to detect quantum effects provoked by gravity: in analog gravity experiments, proposing ways to use entanglement to distinguish between quantum and classical Hawking effect; or taking advantage of the geometric phases acquired by moving detectors to directly measure the Unruh effect for accelerations much smaller than previous proposals.
\end{itemize}


\section*{Specific outcomes}

\begin{list}{\labelitemi}{\leftmargin=1em}
 \item We have shown the relationship between statistics and entanglement behaviour in non-inertial frames. We have studied fermionic cases beyond those in the literature (which only focused on spinless Grassmannian fields). We have formulated questions about the differences between fermionic and bosonic entanglement that helped us understand the origin of such differences. As a result of these studies we have disproved previous beliefs concerning the reason for the differences between bosonic and fermionic entanglement in non-inertial frames, showing that the Hilbert space dimension of the system has nothing to do with those differences.
\begin{itemize}
\item[--] We have extended the study of entanglement behaviour in non-inertial frames to spin 1/2 fields, analysing entanglement of spin Bell states from the perspective of non-inertial observers and comparing it with occupation number entanglement.
\item[--] A method to consistently erase the spin information from field states has been presented. With this method occupation number entanglement can be studied independently of the spin of the field considered.
\item[--] We have analysed entanglement behaviour in different kinds of states of fermionic fields, different spins and different dimension of the Hilbert space. We obtained universal laws for entanglement and mutual information that show that the Hilbert space dimension does not play any role in the  entanglement behaviour between inertial and accelerated observers.
\item[--] A comparative study between fermions and bosons in non-inertial frames has been presented, clearly exposing the differences between these two statistics not only for entanglement but also for the rest of correlations, classical and quantum, investigated by means of the mutual information. We payed special attention to all the possible bipartite systems that emerge from a spacetime with apparent horizons.
\item[--] Non-inertial bosonic field entanglement has been analysed in a scenario where we impose a bound on occupation number. We have seen that limiting the Hilbert space dimension has no qualitative effect on bosonic entanglement. We have compared a limited dimension bosonic field entangled state with its fermionic analog, showing that statistics (which imposes certain structure in the density matrix for the fermionic case via Pauli exclusion principle) is responsible for the differences between fermions and bosons.
\end{itemize}
\item We introduced a formalism to rigorously analyse entanglement behaviour between free-falling observers and observers resisting at a finite distance from the event horizon of a Schwarzschild black hole. We have shown at what distance from the event horizon of a typical solar mass black hole the Hawking effect starts to seriously disturb our ability to perform quantum information tasks.
\item The problems of the so-called `single mode approximation', used for years in all the literature on relativistic quantum information, have been exposed.  We have proved that it is not valid in general and that its meaning was misunderstood in those cases in which it is valid. We have shown and discussed the appropriate physical interpretation of such an approximation.  
 \item The first non-trivial results beyond the single mode approximation have been presented:
 \begin{itemize}
 \item[--] We have shown that for fermionic field entangled states there is an entanglement tradeoff between the particle and antiparticle sector of the different regions of the spacetime which is crucial to understand the phenomenon of fermionic entanglement survival in the limit of infinite Unruh temperature.
 \item[--] Contrary to the extended belief in the whole field of relativistic quantum information, we have shown that beyond the single mode approximation and for certain states of both fermionic and bosonic fields entanglement can be amplified instead of degraded.
 \end{itemize}
 \item We have analysed two dynamical scenarios in which the gravitational interaction generates quantum entanglement in the fields dwelling in the background of some interesting non-stationary spacetimes:
 \begin{itemize}
 \item[--] It has been shown how a stellar collapse generates entanglement. We have analysed the differences between fermionic and bosonic fields, showing that the former are better candidates to serve as a tool to detect genuine Hawking radiation in analog gravity experiments. We have also shown that for extremal black holes (microblackholes or the final stages of an evaporating black hole) the Hawking radiation emitted by the event horizon is in a maximally entangled state with the radiation falling into the black hole.
 \item[--] The fact that the expansion of the Universe generates entanglement in quantum fields has been analysed. Specifically we have analysed the differences between the bosonic and the fermionic case, showing that the entanglement created in fermionic fields contains more information about the history of the expansion and it is more reliable for a hypothetical experiment to obtain information about the parameters of the expansion. We have analysed a specific solvable model of inflationary-type expansion showing a protocol to extract complete information about the volume and rapidity of the expansion by means of fermionic entanglement. 
 \end{itemize}
 \item We have proved that a single mode detector moving through spacetime (even if it is at rest) acquires a geometric phase. We have shown that the phase is the same for any inertial detector but, due to the Unruh effect, it depends on the acceleration of the detector. As a consequence of this result we propose a generic way to use this geometric phase to detect the Unruh effect for accelerations as small as $10^{-9}$ times previous proposals. We conclude presenting a concrete experimental setup based on atomic interferometry using present-day technology. 
 \end{list}
\cleardoublepage


