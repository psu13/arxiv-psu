\documentclass[journal]{IEEEtran}

%\usepackage{iccv}
\usepackage{times}
\usepackage{epsfig}
\usepackage{graphicx}
\usepackage{amsmath}
\usepackage{bm}
\usepackage{amssymb}
\usepackage{mathtools, cuted}
\usepackage{mdframed}
\usepackage[]{subfigure}
\usepackage{multirow}
\usepackage{colortbl}
\usepackage{mathrsfs,amsmath}
\usepackage{dsfont}
\usepackage[normalem]{ulem}
\usepackage[table]{xcolor}% http://ctan.org/pkg/xcolor
\newcommand{\argmin}{\arg\!\min}
\newcommand{\argmax}{\arg\!\max}
%\newcommand*{\Scale}[2][4]{\scalebox{#1}{\ensuremath{#2}}}%
\usepackage{boldline}
\usepackage{algorithm2e}
\usepackage{array}
\usepackage{booktabs}
\usepackage{makecell}

\usepackage{setspace}


\newcommand{\specialcell}[2][c]{\begin{tabular}[#1]{@{}c@{}}#2\end{tabular}}
\makeatletter
\newcommand{\thickhline}{%
	\noalign {\ifnum 0=`}\fi \hrule height 5pt
	\futurelet \reserved@a \@xhline
}
\newcolumntype{"}{@{\hskip\tabcolsep\vrule width 5pt\hskip\tabcolsep}}
\makeatother



\def\aj{AJ} % Astronomical Journal 
\def\araa{ARA\&A}% Annual Review of Astron and Astrophys 
\def\apj{ApJ}%          % Astrophysical Journal 
\def\apjl{ApJL}%           % Astrophysical Journal, Letters 
\def\apjs{ApJS}%           % Astrophysical Journal, Supplement 
\def\ao{Appl.~Opt.}%           % Applied Optics 
\def\apss{Ap\&SS}%           % Astrophysics and Space Science 
\def\aap{A\&A}%           % Astronomy and Astrophysics 
\def\aapr{A\&A~Rev.}%           % Astronomy and Astrophysics Reviews 
\def\aaps{A\&AS}%          % Astronomy and Astrophysics, Supplement 
\def\azh{AZh}%          % Astronomicheskii Zhurnal 
\def\baas{BAAS}%          % Bulletin of the AAS 
\def\jrasc{JRASC}%           % Journal of the RAS of Canada 
\def\memras{MmRAS}%           % Memoirs of the RAS 
\def\mnras{MNRAS}%           % Monthly Notices of the RAS 
\def\pra{Phys.~Rev.~A}%           % Physical Review A: General Physics 
\def\prb{Phys.~Rev.~B}%           % Physical Review B: Solid State 
\def\prc{Phys.~Rev.~C}%           % Physical Review C 
\def\prd{Phys.~Rev.~D}%           % Physical Review D 
\def\pre{Phys.~Rev.~E}%           % Physical Review E 
\def\prl{Phys.~Rev.~Lett.}%           % Physical Review Letters 
\def\pasp{PASP}%           % Publications of the ASP 
\def\pasj{PASJ}%           % Publications of the ASJ 
\def\qjras{QJRAS}% 
\def\aplett{Astrophys.~Lett.}
\def\pasa{PASA}
%          % Quarterly Journal of the RAS 
%\newcommand\skytel{\ref@jnl{S\&T}}% 
%          % Sky and Telescope 
%\newcommand\solphys{\ref@jnl{Sol.~Phys.}}% 
%          % Solar Physics 
%\newcommand\sovast{\ref@jnl{Soviet~Ast.}}% 
%          % Soviet Astronomy 
%\newcommand\ssr{\ref@jnl{Space~Sci.~Rev.}}% 
%          % Space Science Reviews 
%\newcommand\zap{\ref@jnl{ZAp}}% 
%          % Zeitschrift fuer Astrophysik 
\def\nat{Nature}% 



\newif\ifdebugon
%\debugontrue
\debugonfalse

\definecolor{MyDarkBlue}{rgb}{0,0.08,0.5}
\definecolor{MyDarkRed}{rgb}{0.7,0.02,0.02}
\definecolor{MyDarkGreen}{rgb}{0.02,0.30,0.02}
\definecolor{MyDarkOrange}{rgb}{0.40,0.2,0.02}
\definecolor{MyRed}{rgb}{1.0,0.0,0.0}

\ifdebugon
\newcommand{\katie}[1]{\textcolor{MyDarkRed}{[Katie: #1]}}
\else
\newcommand{\katie}[1]{}
\fi

\ifdebugon
\newcommand{\michael}[1]{\textcolor{MyDarkBlue}{[Michael: #1]}}
\else
\newcommand{\michael}[1]{}
\fi

\ifdebugon
\newcommand{\delete}[1]{\textred{\sout{#1}}}
\else
\newcommand{\delete}[1]{}
\fi

\ifdebugon
\newcommand{\add}[1]{\textcolor{blue}{#1}}
\else
\newcommand{\add}[1]{#1}
\fi


\makeatletter
\newcommand{\Spvek}[2][r]{%
	\gdef\@VORNE{1}
	\left[\hskip-\arraycolsep%
	\begin{array}{#1}\vekSp@lten{#2}\end{array}%
	\hskip-\arraycolsep\right]}

\def\vekSp@lten#1{\xvekSp@lten#1;vekL@stLine;}
\def\vekL@stLine{vekL@stLine}
\def\xvekSp@lten#1;{\def\temp{#1}%
	\ifx\temp\vekL@stLine
	\else
	\ifnum\@VORNE=1\gdef\@VORNE{0}
	\else\@arraycr\fi%
	#1%
	\expandafter\xvekSp@lten
	\fi}
\makeatother


\newcommand{\xpos}{\varrho}
\newcommand{\ypos}{\delta}
\newcommand{\FTmtx}{\bm{F}}
\newcommand{\vecFTmtx}{\mathbf{f} }
\newcommand{\vis}{\Gamma}
\newcommand{\bLambda}{\bm{\Lambda} }
\newcommand{\bR}{\bm{R} }
\newcommand{\evolve}{\bm{A} }
\newcommand{\bQ}{\bm{Q} }
\newcommand{\bmu}{\bm{\mu} }
\newcommand{\nmeas}{K}
\newcommand{\ntime}{N}
\newcommand{\meas}{ \bm{y}}
\newcommand{\im}{ \bm{x}}
\newcommand{\npix}{M}
\newcommand{\ntimes}{N}
\newcommand{\ntele}{P}
\newcommand{\bispec}{B}


\newcommand{\HSa}{39}
\newcommand{\HSb}{41}
\newcommand{\HSc}{43}
\newcommand{\HSd}{45}
\newcommand{\HSe}{47}
\newcommand{\HSf}{49}






% Include other packages here, before hyperref.

% If you comment hyperref and then uncomment it, you should delete
% egpaper.aux before re-running latex.  (Or just hit 'q' on the first latex
% run, let it finish, and you should be clear).
\usepackage[pagebackref=true,breaklinks=true,letterpaper=true,colorlinks,bookmarks=false]{hyperref}

%% \iccvfinalcopy % *** Uncomment this line for the final submission
%
%\def\iccvPaperID{2371} % *** Enter the ICCV Paper ID here
%\def\httilde{\mbox{\tt\raisebox{-.5ex}{\symbol{126}}}}
%
%% Pages are numbered in submission mode, and unnumbered in camera-ready
%\ificcvfinal\pagestyle{empty}\fi
%\begin{document}
%\abovedisplayskip=0pt
%\belowdisplayskip=0pt

%%%%%%%%% TITLE
\begin{document}

\title{Reconstructing Video from Interferometric \\ Measurements of Time-Varying Sources}



\author{Katherine L.\ Bouman$^{1,2}$, 
	Michael D.\ Johnson$^2$, 
	Adrian V.\ Dalca$^{1,3}$,
	Andrew A.\ Chael$^2$,  \\
	%Lindy Blackburn$^2$, 
	Freek Roelofs$^{4}$, 
	Sheperd S.\ Doeleman$^2$,
	William T.\ Freeman$^{1,5}$\\ \vspace{0.1in}
    {\small
	$^1${Massachusetts Institute of Technology, CSAIL}
	$^2${Harvard-Smithsonian Center for Astrophysics} \\
    $^3${Massachusetts General Hospital, HMS}
    $^4${Radboud University}
    $^5${Google Research} \vspace{-0.3in}
    }
}


\maketitle
%\thispagestyle{empty}


	

\begin{abstract}	
	Very long baseline interferometry (VLBI) makes it possible to recover images of astronomical sources with extremely high angular resolution. 
	Most recently, the Event Horizon Telescope (EHT) has extended VLBI to short millimeter wavelengths with a goal of achieving angular resolution sufficient for imaging the event horizons of nearby supermassive black holes.
	%Most recently, the Event Horizon Telescope (EHT) has begun observing sources with 20 $\mu$-arcsecond resolution at millimeter wavelengths. 
	%Being able to observe sources at this resolution makes it possible to 
	VLBI provides measurements related to the underlying source image through a sparse set spatial frequencies. 
	An image can then be recovered from these measurements by making assumptions about the underlying image. 
	One of the most important assumptions made by conventional imaging methods is that over the course of a night's observation the image is static. 
	However, for quickly evolving sources, such as the galactic center's supermassive black hole (SgrA*) targeted by the EHT, this assumption is violated and these conventional imaging approaches fail.  
	In this work we propose a new way to model VLBI measurements that allows us to recover both the appearance and dynamics of an evolving source by reconstructing a video rather than a static image.  
    %By modeling VLBI measurements using a Gaussian Markov Model, we are able to propagate information in order to reconstruct a video that combines observations taken over time while simultaneously learning the dynamics of the system.
By modeling VLBI measurements using a Gaussian Markov Model, we are able to propagate information across observations in time to reconstruct a video,
%while simultaneously solving for parameterized dynamical flows present in the source. 
while simultaneously learning about the dynamics of the source's emission region.
    %substantially improves results compared to the state-of-the-art. 
	We demonstrate our proposed Expectation-Maximization (EM) algorithm, StarWarps, on realistic synthetic observations of black holes, and show how it substantially improves results compared to conventional imaging algorithms. Additionally, we demonstrate StarWarps on real VLBI data of the M87 Jet from the VLBA. 
\end{abstract}


\section{Introduction \label{sec:introduction}}

When probed at very short wavelengths, QCD is essentially a theory of
free \index{Partons}`partons' --- quarks and gluons --- which only
scatter off one another through relatively small quantum corrections,
that can be systematically calculated. 
But at longer wavelengths, of order the size of the proton $\sim
1\mathrm{fm} = 10^{-15}\mathrm{m}$,  
we see strongly bound towers of hadron resonances emerge, with string-like
potentials building up if we try to separate their partonic
constituents. Due to our
inability to perform analytic calculations in 
strongly coupled field theories, QCD is therefore 
still only partially solved. Nonetheless,  all its features, across all
distance scales, are believed to be encoded in a single one-line
formula of alluring simplicity; the
\index{QCD!Lagrangian}%
Lagrangian\footnote{Throughout these notes we let it be implicit that
  ``Lagrangian'' really refers to Lagrangian density, ${\cal L}$, the
  four-dimensional space-time integral of which is the action.} of QCD.

The consequence for collider physics is that some parts of QCD can be
calculated in terms of the fundamental parameters of the Lagrangian,
whereas others must be expressed through models or functions whose effective 
parameters are not a priori calculable but which can be constrained
by fits to data. 
However, even in the absence of a
perturbative expansion, there are still several strong theorems which
hold, and which can be used to give relations between seemingly
different processes. (This is, e.g., the reason it makes sense to 
measure the partonic substructure of the proton in $ep$ collisions and
then re-use the same parametrisations for $pp$
collisions.) Thus, in the chapters 
dealing with phenomenological models we shall emphasise that the loss
of a factorised perturbative expansion is not equivalent to a total
loss of predictivity.   

An alternative approach would be to give up on calculating QCD 
and use leptons instead. Formally, this amounts to summing inclusively over
strong-interaction phenomena, when such are present. While such a
strategy might succeed in replacing what we do know about QCD by
``unity'', however, even the most adamant chromophobe would acknowledge
the following basic facts of collider physics for the next decade(s): 
1) At the LHC, the initial states are
hadrons, and hence, at
the very least, well-understood and precise parton distribution
functions (PDFs) will be required; 2) high precision will mandate
 calculations to higher orders in perturbation theory, 
which in turn will involve more QCD; 3) the requirement of lepton
\emph{isolation} makes the very definition of a lepton
 depend implicitly on QCD and 4) 
 the rate of jets that are misreconstructed as leptons in
 the experiment depends explicitly on it. 
And, 5) though many new-physics signals \emph{do} give observable
signals in the lepton sector, this is far from guaranteed, nor is it
exclusive when it occurs. 
 It would therefore be  unwise not to attempt to solve QCD to the best
 of our ability, the better to prepare ourselves for both the largest
 possible discovery reach and the highest attainable subsequent
 precision. 

Furthermore, QCD is the richest gauge theory we have so far
 encountered. Its emergent phenomena, unitarity properties, colour structure, 
 non-perturbative dynamics, quantum vs.\ classical limits, 
interplay between scale-invariant and
 scale-dependent properties, and its wide
 range of phenomenological applications, are still very much topics of
 active investigation, about which we continue to learn.  

In addition, or perhaps as a consequence, the field of QCD is
currently experiencing something of a revolution. On the perturbative
side, new methods to compute scattering amplitudes with very high
particle multiplicities are being developed, together with advanced
techniques for combining such amplitudes with all-orders resummation
frameworks. On the non-perturbative side, the wealth of data on
soft-physics processes from the LHC is
forcing us to reconsider the reliability of the standard fragmentation
models, and heavy-ion collisions are providing new insights into
the collective behavior of hadronic matter. The
study of cosmic rays impinging on the Earth's
atmosphere challenges our ability to extrapolate fragmentation models
from collider energy scales to the region of ultra-high energy cosmic
rays. And finally, dark-matter annihilation processes in space  may produce 
hadrons, whose spectra are sensitive to the modeling 
of fragmentation.

In the following, we shall focus on QCD for mainstream 
collider physics. This includes the basics of SU(3), colour factors, the running
of $\alpha_s$, factorisation, 
hard processes, infrared safety, parton showers and matching, event generators, hadronisation, and the so-called underlying event. 
While not covering everything, hopefully these topics can also serve
at least as stepping stones to more specialised
issues that have been left out, such as twistor-inspired techniques, 
heavy flavours, polarisation, or forward physics, or to topics more tangential to
other fields, such as axions, lattice QCD, or heavy-ion physics.  

\subsection{A First Hint of Colour}
Looking for new physics, as we do now at the LHC, it is instructive to 
consider the story of the discovery of colour. The first hint was
arguably the $\Delta^{++}$ \index{Baryons}baryon, discovered in 
1951~\cite{Brueckner:1952zz}. The title and part of the abstract from this
historical paper are reproduced in \figRef{fig:Delta}.
\begin{figure}[t]
\begin{center}
\begin{tabular}{c}
\colorbox{gray}{\includegraphics*[scale=0.75]{DeltaTitle.pdf}}\\[5mm]
\hspace*{2mm}\begin{minipage}{0.88\textwidth}
\small\sl  ``[...] It is concluded that the apparently anomalous features of the
scattering can be interpreted to be an indication of a resonant
meson-nucleon interaction corresponding to a nucleon isobar with spin
$\frac32$, isotopic spin $\frac32$, and with an excitation energy of
$277\,$MeV.''\\[1mm]
\end{minipage}
\end{tabular}
\caption{The title and part of the abstract of the 1951 paper
  \cite{Brueckner:1952zz} (published in 1952) in which the existence 
  of the $\Delta^{++}$ baryon was deduced, based on data from Sachs and
  Steinberger at Columbia~\cite{Chedester:1951sc}  and from Anderson,
  Fermi, Nagle, et al.~at Chicago~\cite{Fermi:1952zz}. Further studies 
  at Chicago were quickly performed
  in~\cite{Anderson:1952nw,Anderson:1952zza}. See also the memoir by
  Nagle~\cite{nagle1984delta}. 
\label{fig:Delta}}  
\end{center}
\end{figure}
In the context of the \index{Quarks}quark model --- which first
had to be developed, successively joining together the notions of 
spin, isospin, strangeness, and 
the \index{Eightfold way}eightfold way\footnote{In physics, the ``eightfold way''
refers to the classification of the lowest-lying pseudoscalar
\index{Mesons}mesons and 
\index{SU(3)!Of Flavour}%
spin-1/2 \index{Baryons}baryons within \index{Octet}octets in SU(3)-flavour space ($u,d,s$). The
$\Delta^{++}$ is part of a spin-3/2 baryon \index{Decuplet}decuplet, a ``tenfold way'' in this
terminology.} 
--- the \index{Flavour}flavour and spin content of the $\Delta^{++}$
baryon is: 
\begin{equation}
\left\vert \Delta^{++} \right> = \left\vert
\,u_\uparrow\ u_\uparrow\ u_\uparrow \right>~,
\end{equation} 
clearly a highly symmetric configuration. However, since 
the $\Delta^{++}$ is a fermion, it must have an overall
antisymmetric wave function. In 1965, fourteen years after its
discovery, this was finally understood by the introduction of colour
\index{SU(3)}%
\index{SU(3)!Of Colour}%
as a new quantum number associated with the group SU(3)
\cite{Greenberg:1964pe,Han:1965pf}. The $\Delta^{++}$ wave function can now be made
antisymmetric by arranging its three quarks antisymmetrically 
in this new degree of freedom, 
\begin{equation}
\left\vert \Delta^{++} \right> = \epsilon^{ijk} \left\vert
\,u_{i\uparrow}\ u_{j\uparrow}\ u_{k\uparrow}\right>~,
\end{equation} 
hence solving the mystery.

More direct experimental tests of the number of colours were provided first by
measurements of the decay width of $\pi^0\to \gamma\gamma$ decays, which 
is proportional to $N_C^2$, 
and later by the famous ``R'' ratio in
$e^+e^-$ collisions ($R=\sigma(e^+e^-\to q\bar{q})/\sigma(e^+e^-\to
\mu^+\mu^-)$), which is proportional to $N_C$, see
e.g.~\cite{Dissertori:2003pj}. 
Below, in \SecRef{sec:L} we shall see how to
calculate such colour factors. 

\subsection{The Lagrangian of QCD \label{sec:L}}
\index{QCD!Lagrangian}%
Quantum Chromodynamics is based on the gauge group
\index{SU(3)}$\mrm{SU(3)}$, the 
Special Unitary group in 3 (complex) dimensions, whose elements 
are the set of unitary $3\times 3$ matrices with determinant one. 
\index{Fundamental representation}%
\index{SU(3)!Fundamental representation}%
Since there are 9 linearly independent unitary complex
matrices\footnote{A complex $N\times N$ matrix has $2N^2$ degrees of
  freedom, on which unitarity provides $N^2$ constraints.}, one of
which has determinant $-1$, there are a total of 8
independent directions in this matrix space, corresponding to eight
different generators as compared
with the single one of QED. In the context of QCD, we normally
represent this group using the 
so-called \emph{fundamental}, or \emph{defining}, representation, in
which the generators of $\mrm{SU(3)}$ appear as a set of eight traceless and
hermitean matrices, to which we return below.  
We shall refer to indices enumerating
the rows and columns of these matrices  (from 1 to 3) as
\emph{fundamental} indices, and we use the letters $i$,
$j$, $k$, \ldots, to denote them.
\index{Adjoint representation}%
\index{SU(3)!Adjoint representation}%
We refer to indices enumerating the generators (from 1 to 8),
as \emph{adjoint} 
indices\footnote{The dimension of the \emph{adjoint}, or
  \emph{vector}, representation is equal to the number of generators,
  $N^2-1=8$ for $\mrm{SU(3)}$, while the  
\index{Fundamental representation}%
\index{SU(3)!Fundamental representation}%
dimension of the fundamental representation is
  the degree of the group, $N=3$ for $\mrm{SU(3)}$.}, and we use the first
letters of the alphabet ($a$, $b$, $c$, \ldots) to denote them. 
These matrices can operate both on each other (representing
combinations of successive gauge transformations) and on a set of
$3$-vectors, the latter of 
which represent \index{Quarks}quarks in colour 
space; the quarks are \emph{triplets} under $\mrm{SU(3)}$. The matrices can be
thought of as representing gluons in colour 
space (or, more precisely, the gauge transformations carried out by
gluons), hence there are
eight different gluons; the gluons are \emph{octets} under $\mrm{SU(3)}$. 

\index{QCD!Lagrangian}%
The Lagrangian density of QCD is 
\begin{equation}
{\cal L} = \bar{\psi}_q^i(i\gamma^\mu)(D_\mu)_{ij}\psi_q^j - m_q
\bar{\psi}_q^i\psi_{qi} - \frac14 F^a_{\mu\nu}F^{a\mu\nu}~,\label{eq:L}
\end{equation}
where $\psi_q^i$ denotes a quark field with
(fundamental) colour index $i$, 
$\psi_q = ({\textcolor{red}{\psi_{qR}}},{\color{green}\psi_{qG}}, 
{\color{blue}\psi_{qB}})^T$, 
$\gamma^\mu$ is a Dirac matrix that expresses the
vector nature of the strong interaction, with $\mu$ being a Lorentz
vector index, $m_q$ allows for the
possibility of non-zero \index{Quarks}quark masses (induced by the
standard Higgs 
mechanism or similar), $F^a_{\mu\nu}$ is the gluon field strength 
tensor for a gluon\footnote{The definition of the gluon field strength
  tensor will be given below in \eqRef{eq:F}.} with (adjoint) 
colour index $a$ (i.e., $a\in[1,\ldots,8]$), 
and $D_\mu$ is the covariant derivative in QCD,
\begin{equation}
(D_{\mu})_{ij} = \delta_{ij}\partial_\mu - i g_s t_{ij}^a A_\mu^a~,\label{eq:D}
\end{equation}
\index{QCD!Coupling}
with $g_s$ the \index{alphaS@$\alpha_s$}strong coupling (related to
$\alpha_s$ by $g_s^2 = 4\pi 
\alpha_s$; we return to the strong coupling in more detail below), 
$A^a_\mu$  the gluon field with 
colour index $a$, and $t_{ij}^a$ proportional to the hermitean and
traceless \index{Gell-Mann matrices|see{SU(3)}}Gell-Mann matrices of $\mrm{SU(3)}$, 
\index{SU(3)!Generators}%
\begin{equation}
\mbox{\includegraphics*[scale=1.0]{gell-mann}}~.
\end{equation}
These generators are just the $\mrm{SU(3)}$ analogs of the
Pauli matrices in 
$\mrm{SU(2)}$. 
By convention, the constant of proportionality is normally
taken to 
be 
\begin{equation}
t^a_{ij} = \frac12 \lambda^a_{ij}~. \label{eq:t}
\end{equation}
\index{QCD!Coupling}
This choice in turn determines the normalisation of the coupling
$g_s$, via \eqRef{eq:D}, and
fixes the values of the $\mrm{SU(3)}$ \index{Casimirs}Casimirs and structure constants, to which we return below. 

An example of the colour flow for a
quark-gluon interaction in colour 
space is given in \figRef{fig:qg}.
\begin{figure}[t]
\begin{center}
\begin{minipage}[h]{4.6cm}
\begin{center}
$A^1_\mu$\\
\includegraphics*[scale=0.75]{qgv.pdf}\\[-3mm]
$\psi_{q\textcolor{green}{G}}$\hfill$\psi_{q\textcolor{red}{R}}$
\end{center}
\end{minipage}~~~
\parbox{0.4\textwidth}{
$
\begin{array}{ccccc}
\propto & - \frac{i}{2} g_s & \bar{\psi}_{q\color{red}R}  & \lambda^{1} & \psi_{q\color{green}G} 
\\[2mm]
= & -\frac{i}{2}g_s & \left(\begin{array}{ccc} \textcolor{red}{1} & \color{green} 0 &
  \color{blue} 0 
\end{array}\right) & 
\left(\begin{array}{ccc}
0 & 1 & 0  \\
1 & 0 & 0 \\
0 & 0 & 0
\end{array}\right) & 
 \left(\begin{array}{c}
\textcolor{red}{0} \\
\color{green}1 \\
\color{blue}0
\end{array}\right) \end{array}
$}
\caption{Illustration of a 
\index{Quarks}\index{Gluons}$qqg$ vertex in QCD, before
  summing/averaging over colours: a gluon in a state represented by $\lambda^1$
  interacts with quarks in the states $\psi_{qR}$ and
  $\psi_{qG}$. \label{fig:qg}}
\end{center}
\end{figure}
Normally, of course, we sum over all the colour indices, so this
example merely gives a pictorial representation of what one particular
(non-zero) term in the colour sum looks like.


\subsection{Colour Factors}
\index{QCD!Colour factors}
\index{Colour factors}%
\index{Colour-space indices|see{Colour connections}}%
\index{Matrix elements}%
Typically, we do not measure colour in the final state ---
instead we average over all possible incoming colours and sum over all
possible outgoing ones, wherefore QCD scattering amplitudes (squared) in
practice always contain sums over quark fields contracted with
\index{SU(3)!Generators}Gell-Mann matrices. These contractions in turn
produce traces  
which yield the \index{Colour factors}\emph{colour factors} that are associated to each QCD
process, and which basically count the number of ``paths through
colour space'' that the process at hand can take\footnote{The
  convention choice represented by \eqRef{eq:t} introduces a
  ``spurious'' factor of 2 for each power of the coupling $\alpha_s$. 
Although one could in principle absorb that factor into a redefinition
of the coupling, effectively redefining the normalisation of ``unit
colour charge'', the standard definition of $\alpha_s$ is now so
entrenched that alternative choices would be counter-productive, at
least in the context of a pedagogical review.}.

A very simple example of a colour factor is given by the decay process $Z\to
q\bar{q}$. This vertex contains a simple $\delta_{ij}$ in colour
space; the outgoing quark and antiquark must have identical 
(anti-)col\-ours. Squaring the corresponding matrix element and summing over
final-state colours yields a colour factor of
\begin{equation}
e^+e^-\to Z \to q\bar{q}~~~:~~~\sum_{\mrm{colours}}|M|^2 \propto
\delta_{ij}\delta_{ji} = \mrm{Tr}\{\delta\} = N_C = 3~,
\end{equation}
since $i$ and $j$ are quark (i.e., 3-dimensional
fundamental) indices. This factor corresponds directly to the 3 different
``paths through colour space'' that the process at hand can take; the
produced quarks can be red, green, or blue. 

A next-to-simplest example is given by $q\bar{q}\to
\gamma^*/Z\to\ell^+\ell^-$ (usually referred to as the
\index{Drell-Yan}Drell-Yan 
process~\cite{Drell:1970wh}),  
which is just a crossing of the previous one. By crossing
symmetry, the squared matrix element, including the colour factor, is
exactly the same as before, but since the quarks are here incoming, we
must \emph{average} rather than sum over their colours, leading to
\begin{equation}
q\bar{q}\to Z\to e^+e^-~~~:~~~\frac{1}{9}\sum_{\mrm{colours}}|M|^2 \propto \frac19\delta_{ij}\delta_{ji} = \frac19 \mrm{Tr}\{\delta\} = \frac13~,
\end{equation}
where the colour factor now expresses a \emph{suppression} which can
be interpreted as due to the fact that only quarks of matching colours
are able to collide and produce a $Z$ boson. The chance that a quark
and an antiquark picked at random from the colliding hadrons have 
matching colours is $1/N_C$. 
\begin{figure}[t]
\end{figure}

Similarly, $\ell q \to
\ell q$ via $t$-channel photon exchange (usually called Deep
Inelastic Scattering --- \index{DIS}\index{Deep inelastic scattering|see{DIS}}DIS --- with ``deep'' referring to a 
large virtuality of the exchanged photon), constitutes yet another
crossing of the same basic process, 
see \figRef{fig:Zcrossings}. \index{Colour factors}The colour factor in this case 
comes out as unity. 
\begin{figure}[t]
\centering\vspace*{-8mm}
\begin{tabular}{ccc}
\rotatebox{360}{\includegraphics*[scale=0.93]{ee2qq}} \ \ 
& \ \ \includegraphics*[scale=0.93,angle=180,origin=c]{ee2qq}
\ \ & \ \ \includegraphics*[scale=0.9,angle=297,origin=c]{ee2qq}\\
Hadronic $Z$ decay & \index{Drell-Yan}Drell-Yan & \index{DIS}DIS \\[1mm]
$e^-e^+ \to \gamma^*/Z^0 \to q\bar{q}$ &
$q\bar{q} \to \gamma^*/Z^0 \to \ell^+\ell^-$ &
$\ell \bar{q} \stackrel{\gamma^*/Z^*}{\to} \ell \bar{q}$
\\[2mm] 
$\propto N_C$ & $\propto 1/N_C$ & $\propto 1$
\end{tabular}
\caption{Illustration of the three crossings of the interaction of a
  lepton current (black) with a \index{Quarks}quark current (red) 
  via an intermediate photon or
  $Z$ boson, with corresponding colour factors. \label{fig:Zcrossings}}
\end{figure}

To illustrate what happens when we insert (and sum over)
quark-gluon
vertices, such as the one depicted in \figRef{fig:qg}, we take
the process $Z\to3\,$jets. \index{Colour factors}The colour factor for
this process can be 
computed as follows, with the accompanying illustration showing a
corresponding diagram (squared) with explicit colour-space indices on
each vertex:\\
\index{Colour connections}
\begin{equation}
\mbox{
\begin{tabular}{cc}
\parbox{5.2cm}{
$Z \to qg\bar{q}$~~~:~~~\\
\[
\begin{array}{rcl}
\displaystyle\sum_{\mrm{colours}}|M|^2 & \propto & \displaystyle
\delta_{ij}t_{jk}^a t_{k\ell
    }^a\delta_{\ell i} \\
& = & \displaystyle
\mrm{Tr}\{t^at^a\}\\[4mm] & = & \displaystyle
  \frac12\mrm{Tr}\{\delta\} = 4~,
\end{array}
\]}
&
\parbox{8.5cm}{\includegraphics*[scale=0.6]{colFacZ3.pdf}
}
\end{tabular}}
\end{equation}
where the last $\mrm{Tr}\{\delta\} = 8$, since the trace runs over
the 8-dimensional adjoint indices. If we
want to ``count the paths through colour space'', we should leave out
the factor $\frac12$ which comes from the normalisation convention for
the $t$ matrices, \eqRef{eq:t}, hence this process can take 8
different paths through colour space, one for each gluon basis state.

The tedious task of taking traces over $t$
matrices can be greatly alleviated by use of the relations given in
\TabRef{tab:lambda}.  
\index{Traces in SU(3)|see{SU(3)}}%
\index{SU(3)!Trace relations}%
\index{QCD!Trace relations|see{SU(3)}}%
\begin{table}
\begin{center}
\scalebox{1.04}{\begin{tabular}{ccc}
\toprule
\index{SU(3)!Trace relations}Trace Relation & Indices & Occurs in Diagram Squared
\\
\midrule
$\mrm{Tr}\{t^at^b\} = T_R\, \delta^{ab}$ & $a,b\in[1,\ldots,8]$
& \parbox[c]{4cm}{\includegraphics*[scale=0.5]{traces1}}\\
$\sum_a t^a_{ij}t^a_{jk} = C_F\, \delta_{ik}$ &%
\parbox[c]{3cm}{\begin{center}
$a\in[1,\ldots,8]$\\
$i,j,k\in[1,\ldots,3]$\end{center}}
& \parbox[c]{4cm}{\includegraphics*[scale=0.5]{traces2}}\\
$\sum_{c,d} f^{acd} f^{bcd} = C_A\, \delta^{ab}$ & $a,b,c,d\in[1,\ldots,8]$
& \parbox[c]{4cm}{\includegraphics*[scale=0.5]{traces3}}\\
$ t^a_{ij}t^a_{k\ell} = T_R \left(\delta_{jk}\delta_{i\ell}
- \frac{1}{N_C}\delta_{ij}\delta_{k\ell}\right)$ & $i,j,k,\ell\in[1,\ldots,3]$
& \parbox[c]{4cm}{\includegraphics*[scale=0.5]{traces4}}\hspace*{-0.2cm}(Fierz)\\
\bottomrule
\end{tabular}}
\caption{Trace relations for $t$ matrices (convention-independent). 
 More relations
  can be found in \cite[Section 1.2]{Ellis:1991qj} and in 
  \cite[Appendix A.3]{Peskin:1995ev}.
\label{tab:lambda}}
\end{center}
\end{table}
In the standard normalisation convention for the \index{SU(3)}$\mrm{SU(3)}$ generators,
\eqRef{eq:t}, the \index{Casimirs}Casimirs of $\mrm{SU(3)}$ appearing in
\TabRef{tab:lambda} are\footnote{See, e.g., \cite[Appendix
    A.3]{Peskin:1995ev} for how to obtain the Casimirs in other
  normalisation conventions. As an example, choosing $t^a_{ij} = \lambda_{ij}^a/\sqrt{2}$ would yield $T_R=1$, $C_F=T_R(N_C^2-1)/N_C=8/3$, $C_A=3$.} 
\index{Casimirs}\index{TR@$T_R$}\index{CA@$C_A$}\index{CF@$C_F$}
\begin{equation}
T_R = \frac12 \hspace*{2cm} C_F = \frac43 \hspace*{2cm} C_A = N_C = 3~.
\end{equation}
In addition, the gluon self-coupling on the third line in
\TabRef{tab:lambda} involves factors of $f^{abc}$. These
\index{QCD!Structure constants|see{SU(3)}}%
are called the \index{SU(3)!Structure constants}\emph{structure constants} of QCD and they enter via 
the non-Abelian term in the \index{Gluons}gluon field strength tensor appearing in
\eqRef{eq:L}, 
\begin{equation}
F^a_{\mu\nu} = \underbrace{\partial_\mu A_\nu^a - \partial_\nu
  A^a_\mu}_{\mathrm{Abelian}} +
\underbrace{ g_s f^{abc} A_\mu^b A_\nu^c}_{\mathrm{non-Abelian}}~. \label{eq:F}
\end{equation}

\noindent\begin{minipage}[t]{0.46\textwidth}
The structure constants of $\mrm{SU(3)}$ are listed in the table to the
right. They define the \emph{adjoint}, or \emph{vector}, representation of $\mrm{SU(3)}$
and are related to the fundamental-representation generators via the
commutator relations
\begin{equation}
t^at^b - t^bt^a = [t^a,t^b] = i f^{abc} t_c~,
\end{equation} 
or equivalently,
\begin{equation}
if^{abc}~=~2\mrm{Tr}\{t^c[t^a,t^b]\}~.
\end{equation}
Thus, it is a matter of choice whether one prefers to express colour
space on a basis of fundamental-representation $t$ matrices, or via
the structure constants $f$, and one can go back and forth between the
two.
\end{minipage}%
\hfill%
\colorbox{darkgray}{%
\colorbox{lightgray}{%
\begin{minipage}[t]{0.46\textwidth}
\vspace*{3mm}\begin{center}
\textbf{Structure Constants of SU(3)}
\begin{equation}
f_{123} = 1
\end{equation}
\begin{equation}
f_{147} = f_{246} = f_{257} = f_{345} = \frac12
\end{equation}
\begin{equation}
f_{156} = f_{367} = -\frac12
\end{equation}
\begin{equation}
f_{458} = f_{678} = \frac{\sqrt{3}}{2}
\end{equation}
Antisymmetric in all indices\\[3mm]
All other $f_{abc}=0$\vspace*{3mm}\\
\end{center}
\end{minipage}%
}}\vskip1mm

\begin{figure}[t]
\begin{center}
\begin{minipage}[h]{4.6cm}
\begin{center}
$A_\nu^4(k_2)$\\
\includegraphics*[scale=0.75]{ggv.pdf}\\[-3mm]
$A^6_\rho(k_1)$\hfill$A_\mu^2(k_3)$
\end{center}
\end{minipage}~~~
\parbox{0.35\textwidth}{
$
\begin{array}{cccc}
\propto & - g_s \ f^{246} \!\! & \!\! [ (k_3 - k_2)^\rho g^{\mu\nu}  \\ 
& & +(k_2 - k_1)^\mu g^{\nu\rho} \\ 
& &+(k_1 - k_3)^\nu g^{\rho\mu}]
\end{array}
$}\vspace*{1mm}
\caption{Illustration of a \index{Gluons}$ggg$ vertex in QCD, before
  summing/averaging over colours: interaction between gluons in the 
  states $\lambda^2$, $\lambda^4$, and $\lambda^6$ is represented by
  the structure constant $f^{246}$. 
\label{fig:gg}}
\end{center}
\end{figure}
 Expanding the $F_{\mu\nu}F^{\mu\nu}$ term of the
Lagrangian using \eqRef{eq:F}, we see that there is a 3-gluon and a
4-gluon vertex that involve $f^{abc}$, the latter of which has two
powers of $f$ and two powers of the coupling. 

Finally, the last line of \TabRef{tab:lambda} is not really a trace
relation but instead a useful so-called Fierz transformation, which
expresses products of $t$ matrices in terms of Kronecker $\delta$ functions. 
It is often used, for instance, in shower Monte Carlo
applications, to assist in mapping between colour flows in $N_C = 3$,
in which cross sections and splitting probabilities are calculated, 
and those in $N_C\to\infty$ (``leading colour''), used to represent colour flow in
the MC ``event record''.

A \index{Gluons}gluon self-interaction vertex is
illustrated in \figRef{fig:gg}, to be compared with the quark-gluon
one in \figRef{fig:qg}. We remind the reader that gauge boson
self-interactions are a hallmark of non-Abelian theories and that their
presence leads to some of the main differences between QED and
QCD. One should also keep in mind 
that the \index{Colour factors}colour factor for the vertex in \figRef{fig:gg}, \index{CA@$C_A$}$C_A$, 
is roughly twice as large as that for a quark, \index{CF@$C_F$}$C_F$.

\subsection{The Strong Coupling \label{sec:coupling}}
\index{QCD!Coupling}
\index{Jets}
\index{alphaS@$\alpha_s$}To first approximation, QCD is 
\index{QCD!Scale invariance}\emph{scale invariant}. That is, if one
``zooms in'' on a QCD jet, one will find a repeated self-similar 
pattern of jets within jets within jets, reminiscent of
fractals. 
In the context of QCD, this property was originally 
called \index{Lightcone scaling|see{QCD Scale invariance}}light-cone scaling, or 
\index{Bjorken scaling|see{QCD Scale invariance}}Bj{\o}rken scaling. 
This type of scaling is closely related to the class of
angle-preserving symmetries, called \index{Conformal
invariance}\emph{conformal} symmetries. In physics 
today, the terms ``conformal'' and ``scale invariant'' are used 
interchangeably\footnote{Strictly speaking, conformal symmetry is more
restrictive than just scale invariance, but examples of
scale-invariant field theories that are not conformal are rare.}.
Conformal invariance is a mathematical property of several
QCD-``like'' theories which are now being studied (such as $N=4$
supersymmetric relatives of QCD). It is also 
related to the physics of so-called ``unparticles'', though that is a
relation that goes beyond the scope of these lectures.

Regardless of the labelling, 
if the  \index{alphaS@$\alpha_s$}strong coupling did not run (we shall
return to the running 
of the coupling below), Bj{\o}rken scaling would be absolutely true. QCD
would be a theory with a fixed coupling, the same at all scales. 
This simplified picture already captures some of the most important
properties of QCD, as we shall discuss presently.  

\index{QCD!Scale invariance}%
In the limit of exact Bj{\o}rken scaling --- QCD at fixed coupling
--- properties of high-energy interactions are determined 
only by \emph{dimensionless} kinematic quantities, such as scattering
angles (pseudorapidities) and ratios of energy
scales\footnote{Originally, the observed approximate agreement with
this was used as a powerful argument
for pointlike substructure in hadrons; since measurements at different
energies are sensitive to different resolution scales, independence of the absolute
energy scale is indicative of the absence of other fundamental
scales in the problem and hence of pointlike constituents.}.
For applications of QCD to high-energy collider physics, an important
consequence of Bj{\o}rken scaling is thus that the rate of 
\index{Parton showers}%
\index{Bremsstrahlung|see{Parton showers}}
bremsstrahlung
jets, with a given transverse momentum, scales in direct proportion to
the hardness 
of the fundamental partonic scattering process they are produced in
association with. This agrees well with our intuition about accelerated
charges; the harder you ``kick'' them, the harder the radiation they
produce.  

For instance, in the limit of exact scaling, a
measurement of the rate of 10-GeV jets produced in association with an
ordinary $Z$ 
boson could be used as a direct prediction of the rate of 100-GeV jets
that would be 
produced in association with a 900-GeV $Z'$ boson, and so 
forth. Our intuition about how many bremsstrahlung jets a given type of
process is likely to have should therefore be governed first and
foremost by the \emph{ratios} of scales that appear in that particular
process, as has been  highlighted in a number of studies focusing on
the mass and $p_\perp$ scales appearing, e.g., in
Beyond-the-Standard-Model (BSM) 
physics processes
\cite{Plehn:2005cq,Alwall:2008qv,Papaefstathiou:2009hp,Krohn:2011zp}. 
\index{QCD!Scale invariance}Bj{\o}rken scaling 
\index{Scale invariance|see{QCD}}
is also fundamental to the understanding of jet substructure in QCD, see, e.g.,
\cite{Vermilion:2011nm,Altheimer:2012mn}.  

\index{alphaS@$\alpha_s$!Running coupling}%
On top of the underlying scaling behavior, the running coupling will
introduce a dependence on the absolute scale, implying more radiation
at low scales than at high ones. The running is logarithmic with
\index{alphaS@$\alpha_s$!beta function}%
energy, and is governed by the so-called \emph{beta function}, 
\index{alphaS@$\alpha_s$}
\begin{equation}
Q^2 \frac{\partial \alpha_s}{\partial Q^2} = \frac{\partial
  \alpha_s}{\partial \ln Q^2} =
\beta(\alpha_s)~, \label{eq:running}
\end{equation}
where the function driving the energy dependence, the \index{Beta function}{beta
  function}, is defined as
\begin{equation}
\beta(\alpha_s) = -\alpha_s^2(b_0 +
b_1\alpha_s + b_2\alpha_s^2 + \ldots)~,\label{eq:beta}
\end{equation}
with LO (1-loop) and NLO (2-loop) coefficients
\begin{eqnarray}
b_0 & = & \frac{11C_A - 4 T_R n_f}{12\pi}~,\\[3mm]
b_1 & = & \frac{17C_A^2 - 10 T_R C_A n_f - 6 T_R C_F n_f}{24\pi^2} ~=~
\frac{153-19 n_f}{24\pi^2}~.\label{eq:b}
\end{eqnarray}
In the $b_0$ coefficient, the first term is due to
\index{Gluons!Contribution to beta function}gluon loops while the
second is due to \index{Quarks!Contribution to beta function}quark
ones. Similarly, the first 
term of the $b_1$ coefficient arises from double gluon loops,
while the second and third represent mixed quark-gluon ones. 
At higher loop orders, the $b_i$ coefficients depend explicitly on the
renormalisation scheme that is used. A brief discussion can be found in the
PDG review on QCD~\cite{pdg2012}, with more elaborate ones
contained in \cite{Dissertori:2003pj,Ellis:1991qj}. 
Note that, if there are additional coloured particles beyond the
Standard-Model ones, loops involving those particles enter
 at energy scales above the masses of the
new particles, thus modifying the  \index{alphaS@$\alpha_s$}running of the coupling at high scales. 
This is discussed, e.g., for supersymmetric models in
\cite{Martin:1997ns}. For the running of other SM couplings, see
e.g.,~\cite{Langacker:2010zza}. 

\index{alphaS@$\alpha_s$!Running coupling}%
Numerically, the value of the  \index{alphaS@$\alpha_s$}strong coupling is usually specified by
giving its value at the specific 
reference scale $Q^2=M^2_Z$, from which we can obtain its
value at any other scale by solving \eqRef{eq:running}, 
\begin{equation}
\alpha_s(Q^2) = \alpha_s(M_Z^2) \frac{1}{1+b_0
  \alpha_s(M_Z^2)\ln\frac{Q^2}{M_Z^2} + {\cal O}(\alpha_s^2)}~,
\label{eq:alphaq2}
\end{equation}
with relations including the ${\cal O}(\alpha_s^2)$ terms 
available, e.g., in \cite{Ellis:1991qj}. 
Relations between scales 
not involving $M_Z^2$ can obviously be obtained by just replacing $M_Z^2$
by some other scale $Q'^2$ everywhere in \eqRef{eq:alphaq2}. A
comparison of running at one- and two-loop order, in both cases starting from
$\alpha_s(M_Z)=0.12$, is given in \figRef{fig:asRun}.
\begin{figure}[t]
\centering
\includegraphics*[scale=0.45]{vc-alphaS.pdf}
\caption{Illustration of the running of
 $\alpha_s$ at 1- (open 
  circles) and 2-loop
  order (filled circles), 
starting from the same value of $\alpha_s(M_Z)=0.12$. 
\label{fig:asRun}}
\end{figure}
As is evident from the figure, the 2-loop running is somewhat faster
than the 1-loop one.

\index{alphaS@$\alpha_s$!Running coupling}%
As an application, let us prove that the 
logarithmic running of the coupling implies that an intrinsically 
multi-scale problem can be converted to a single-scale one, up to
corrections suppressed by two powers of $\alpha_s$, 
by taking the geometric mean of the scales involved. This follows from
expanding an arbitrary product of individual  \index{alphaS@$\alpha_s$}$\alpha_s$ factors around an
arbitrary scale $\mu$, using \eqRef{eq:alphaq2}, 
\begin{eqnarray}
\alpha_s(\mu_1)\alpha_s(\mu_2)\cdots\alpha_s(\mu_n) & = &
\prod_{i=1}^{n} \alpha_s(\mu) \left(1 +
b_0\,\alpha_s\ln\left(\frac{\mu^2}{\mu_i^2}\right) + {\cal O}(\alpha_s^2)\right)
\nonumber\\[2mm]
& = & \alpha_s^n(\mu) \left(1 + b_0\, \alpha_s \ln \left(
 \frac{\mu^{2n}}{\mu_1^2\mu_2^2\cdots\mu_n^2}\right) +  {\cal
   O}(\alpha_s^2) \right)~,
\end{eqnarray}
whereby the specific single-scale choice $\mu^n =
\mu_1\mu_2\cdots\mu_n$ (the geometric mean) can
be seen to push the difference between the two sides of the equation one order higher
than would be the case for any other combination of scales\footnote{In
  a fixed-order calculation, the individual scales $\mu_i$,
would correspond, e.g., to the $n$ hardest scales appearing in an infrared
safe sequential clustering algorithm applied to the given momentum
configuration.}. 

The appearance of the number of \index{Flavour}flavours, $n_f$, in $b_0$ implies that the
slope of the running depends on the number of contributing
\index{Flavour}flavours. Since full QCD is best approximated by $n_f=3$
below the charm threshold, by $n_f=4$ and $5$ from there to the $b$
and $t$ thresholds, respectively, and then by $n_f=6$ at scales
higher than $m_t$, it is therefore important to be aware that 
the running changes slope across quark \index{Flavour}flavour
thresholds. Likewise, it would change across the threshold for any coloured
new-physics particles that might exist, with a magnitude depending on
the particles' colour and spin quantum numbers.

\index{alphaS@$\alpha_s$!Running coupling}%
\index{alphaS@$\alpha_s$}
The negative overall sign of \eqRef{eq:beta}, combined with the fact
that $b_0 > 0$ (for $n_f \le 16$), leads to the famous
result\footnote{
Perhaps the highest pinnacle of fame for \eqRef{eq:beta} was reached
when the sign of it featured in an episode of the TV series ``Big Bang
Theory''.} 
that the QCD coupling effectively \emph{decreases} with
 energy, called \index{Asymptotic freedom}asymptotic 
freedom, for the discovery of which the \index{Nobel prize}Nobel prize in physics was
awarded to D.~Gross, H.~Politzer, and F.~Wilczek in 2004. An extract
of the prize announcement runs as follows:
\begin{center}
\begin{minipage}{0.84\textwidth}
\sl  What this year's Laureates discovered was something that, at
first sight, seemed completely contradictory. The interpretation of
their mathematical result was that the closer the quarks are to each
other, the \emph{weaker} is the ``colour charge''. When the quarks are
really close to each other, the force is so weak that they behave
almost as free particles\footnote{More correctly, it is the coupling
  rather than the  
  force which becomes weak as the distance decreases. 
  The $1/r^2$ Coulomb singularity of the force is only dampened, not removed, 
  by the diminishing coupling.}. 
This phenomenon is called ``asymptotic
freedom''. The converse is true when the quarks move apart: the force
becomes stronger when the distance increases\footnote{More correctly,
 it is the potential which grows, linearly, while the force becomes
 constant.}. 
\end{minipage}
\end{center}

\index{Running coupling|see{alphaS@$\alpha_s$}}%
\index{alphaS@$\alpha_s$!Running coupling}%
Among the consequences of \index{Asymptotic freedom}asymptotic freedom is that perturbation
theory becomes better behaved at higher absolute energies, due to the
effectively decreasing coupling. Perturbative calculations for our
900-GeV $Z'$ boson from before should therefore be slightly faster
converging than equivalent calculations for the 90-GeV one. 
Furthermore, since the running of  \index{alphaS@$\alpha_s$}$\alpha_s$ explicitly
breaks Bj{\o}rken scaling, we also expect to see small changes in jet
shapes and in jet production ratios as we vary the energy. For
instance, since high-$p_\perp$ jets
start out with a smaller effective coupling, their intrinsic shape
(irrespective of boost effects) is
somewhat narrower than for low-$p_\perp$ jets, an issue which can be
important for jet calibration. Our current understanding of the
running of the QCD coupling is summarised by the plot in
\figRef{fig:alphas}, taken from a recent comprehensive review by S.\ Bethke
\cite{pdg2012,Bethke:2012jm}. A complementary up-to-date overview of
$\alpha_s$ determinations can be found in~\cite{d'Enterria:2015toz}. 

\index{alphaS@$\alpha_s$!Running coupling}%
As a final remark on \index{Asymptotic freedom}asymptotic freedom, note
that the decreasing 
value of the  \index{alphaS@$\alpha_s$}strong coupling with energy must eventually cause it to
become comparable to the electromagnetic and weak ones, at some energy
scale. Beyond that point, which may lie at energies of order
$10^{15}-10^{17}\,$GeV (though it may be lower if as yet undiscovered
particles generate large corrections to the running), 
we do not know  what the further evolution of the combined theory will 
actually look like, or whether it will continue to exhibit
\index{Asymptotic freedom}asymptotic
freedom. 

\index{alphaS@$\alpha_s$}%
\index{alphaS@$\alpha_s$!Running coupling}%
\index{alphaS@$\alpha_s$!LambdaQCD@$\Lambda_{\mathrm{QCD}}$}%
Now consider what happens when we run the coupling in the other
direction, towards smaller energies. 
\begin{figure}[t]
\begin{center}\hspace*{-0.25cm}
\parbox[c]{3.1cm}{\includegraphics*[scale=0.65]{arr-ir.pdf}}
\parbox[c]{8cm}{\includegraphics*[scale=0.5]{asq-2011.pdf}}\hspace*{-1mm}
\parbox[c]{3.1cm}{\includegraphics*[scale=0.65]{arr-uv.pdf}}
\caption{Illustration of the running of $\alpha_s$ in a theoretical
  calculation (band) and in physical processes at
  different characteristic scales, from
  \cite{pdg2012,Bethke:2012jm}. The little kinks at $Q=m_{c}$ and
  $Q=m_b$ are
  caused by discontinuities in the running across the flavour
  thresholds.\label{fig:alphas}}  
\end{center}           
\end{figure}
Taken at face value, the numerical value of the coupling diverges
rapidly at scales below 1 GeV, as illustrated by the curves
disappearing off the left-hand edge of the plot in
\figRef{fig:alphas}. To make this divergence
explicit, one can rewrite
\eqRef{eq:alphaq2} in the following form, 
 \index{alphaS@$\alpha_s$}
\begin{equation}
\alpha_s(Q^2) = \frac{1}{b_0 \ln \frac{Q^2}{\Lambda^2}}~,\label{eq:alphasLam}
\end{equation}
where 
\begin{equation}
\Lambda \sim 200\, \mbox{MeV}
\end{equation}
\index{alphaS@$\alpha_s$!LambdaQCD@$\Lambda_{\mathrm{QCD}}$}%
\index{alphaS@$\alpha_s$!Landau Pole|see{$\Lambda_{\mathrm{QCD}}$}}%
\index{LambdaQCD@$\Lambda_{\mathrm{QCD}}$|see{alphaS@$\alpha_s$}}%
specifies the energy scale at which the perturbative coupling would nominally become
infinite, called the Landau pole. (Note, however, that this only
parametrises the purely \emph{perturbative} result, which is not
reliable at \index{Strong coupling}strong coupling, so \eqRef{eq:alphasLam} should 
not be taken to imply that the physical behavior of full QCD should
exhibit a divergence for $Q\to \Lambda$.) 

\index{alphaS@$\alpha_s$}%
\index{alphaS@$\alpha_s$!Running coupling}%
\index{alphaS@$\alpha_s$!LambdaQCD@$\Lambda_{\mathrm{QCD}}$}%
Finally, one should be aware that there is a multitude of different
ways of defining both $\Lambda$ and $\alpha_s(M_Z)$. At the very
least, the numerical value one obtains depends both on the
renormalisation scheme used (with the dimensional-regularisation-based
``modified minimal subtraction'' scheme, $\overline{\mbox{MS}}$, being the
most common one) and on the perturbative order of the calculations 
used to extract them. As a rule of thumb, fits to experimental data typically yield 
smaller values for $\alpha_s(M_Z)$ the higher the order of the
calculation used to extract it (see, e.g.,
\cite{Bethke:2009jm,Dissertori:2009ik,Bethke:2012jm,pdg2012}), with  $
\alpha_s(M_Z)\vert_{\mrm{LO}} \gsim \alpha_s(M_Z)\vert_{\mrm{NLO}}
\gsim \alpha_s(M_Z)\vert_{\mrm{NNLO}}$. 
Further, since the number of \index{Flavour}flavours changes the slope
of the running, the location of the Landau pole for fixed
$\alpha_s(M_Z)$ depends explicitly on the number of \index{Flavour}flavours used in
the running. Thus each value of $n_f$ is associated with its own
value of $\Lambda$, with the following matching relations across
thresholds guaranteeing continuity of the coupling at one loop,
\index{LambdaQCD@$\Lambda_{\mathrm{QCD}}$|see{$\alpha_s$}}
\index{alphaS@$\alpha_s$!LambdaQCD@$\Lambda_{\mathrm{QCD}}$}%
\begin{eqnarray}
n_f = 5 \leftrightarrow 6 ~~~:~~~~~~\Lambda_6 = \Lambda_5
  \left(\frac{\Lambda_5}{m_t}\right)^{\frac{2}{21}} & & 
\Lambda_5 = \Lambda_6
  \left(\frac{m_t}{\Lambda_6}\right)^{\frac{2}{23}} ~, \\[2mm]
n_f = 4 \leftrightarrow 5 ~~~:~~~~~~\Lambda_5 = \Lambda_4
  \left(\frac{\Lambda_4}{m_b}\right)^{\frac{2}{23}} & & 
\Lambda_4 = \Lambda_5
  \left(\frac{m_b}{\Lambda_5}\right)^{\frac{2}{25}} ~, \\[2mm]
n_f = 3 \leftrightarrow 4 ~~~:~~~~~~\Lambda_4 = \Lambda_3 
  \left(\frac{\Lambda_3}{m_c}\right)^{\frac{2}{25}} & &
\Lambda_3 = \Lambda_4 
  \left(\frac{m_c}{\Lambda_4}\right)^{\frac{2}{27}} ~.
\end{eqnarray}

\index{alphaS@$\alpha_s$}%
\index{alphaS@$\alpha_s$!Running coupling}%
It is sometimes stated that QCD only has a single free
parameter, the  \index{alphaS@$\alpha_s$}strong coupling. 
However, even in the perturbative
region, the beta function depends explicitly on the number of
quark \index{Flavour}flavours, as we have seen, and thereby also on the quark
masses. Furthermore, in the non-perturbative region around or below
$\Lambda_{\mrm{QCD}}$, the value of the 
perturbative coupling, as obtained, e.g., from \eqRef{eq:alphasLam},
gives little or no insight into the behavior of the full theory. 
Instead, universal functions (such as parton densities, form factors,
fragmentation functions, etc), effective theories (such as the
Operator Product Expansion, Chiral Perturbation Theory, or Heavy Quark
Effective Theory), or phenomenological models (such as Regge Theory or
the String and Cluster Hadronisation Models) must be used, which in
turn depend on additional non-perturbative parameters whose relation to, e.g.,
$\alpha_s(M_Z)$, is not a priori known. 

\index{Lattice QCD}
For some of these questions,
such as hadron masses, lattice QCD can furnish important
additional insight, but for multi-scale and/or time-evolution
problems, the applicability of lattice methods is still severely
restricted; the lattice formulation of QCD requires 
  a Wick rotation to
  Euclidean space. The time-coordinate can then be treated on an
  equal footing with the other dimensions, but intrinsically
  Minkowskian problems, such as the time evolution of a system, are
   inaccessible. The limited size of current lattices
  also severely constrain the scale hierarchies that it is possible to
  ``fit'' between the lattice spacing and the lattice size. 

\index{Landau pole|see{$\alpha_s$}}%
\index{QCD!Landau Pole|see{$\alpha_s$}}%
\index{Renormalisation|see{$\alpha_s$}}%
\index{QCD!Renormalisation|see{$\alpha_s$}}%

\subsection{Colour States}
\index{Coherence}%
A final example of the application of the underlying $\mrm{SU(3)}$ group
theory to QCD is given by considering which colour states we can
obtain by combinations of quarks and gluons. The simplest example of
this is the combination of a quark and antiquark. We can form a total
of nine different colour-anticolour combinations, which fall into two
irreducible representations of $\mrm{SU(3)}$:
\begin{equation}
3 \otimes \overline{3} = 8 \oplus 1~.\label{eq:33bar}
\end{equation}
The singlet corresponds to the symmetric wave function 
$\frac{1}{\sqrt{3}}\left(\left|R\bar{R}\right>+\left|G\bar{G}\right>+\left|B\bar{B}\right>\right)$, 
which is invariant under $\mrm{SU(3)}$ transformations (the definition of a
singlet). The other eight linearly independent 
combinations (which can be represented by one for each Gell-Mann
matrix, with the singlet corresponding to the identity matrix) transform
into each other under $\mrm{SU(3)}$. Thus, although we sometimes talk about
colour-singlet states as 
being made up, e.g., of ``red-antired'', that is not quite precise
language. The actual state $\left|R\bar{R}\right>$ is \emph{not} a
pure colour singlet.  Although it does
have a non-zero \emph{projection} onto the singlet wave function
above, it also has non-zero projections onto the two members of
the octet that correspond to the diagonal Gell-Mann
matrices. Intuitively, one can also easily realise this by noting that
an $\mrm{SU(3)}$ rotation of $\left|R\bar{R}\right>$ would in general turn it into a
different state, say $\left|B\bar{B}\right>$, whereas a true colour singlet
would be invariant. 
Finally, we can also realise from \eqRef{eq:33bar} that a random
(colour-uncorrelated) quark-antiquark pair has a $1/N^2=1/9$ 
chance to be in an overall colour-singlet state; otherwise it is in
an octet. 

Similarly, there are also nine possible quark-quark (or
antiquark-antiquark) combinations, six of which are symmetric
under interchange of the two quarks and three of which are antisymmetric:
\index{Sextet}%
\begin{equation}
6 ~=~ \left(\begin{array}{c}
\left|RR\right>\\
\left|GG\right>\\
\left|BB\right>\\
\frac{1}{\sqrt{2}}\left(\left|RG\right> + \left|GR\right>\right)\\
\frac{1}{\sqrt{2}}\left(\left|GB\right> + \left|BG\right>\right)\\
\frac{1}{\sqrt{2}}\left(\left|BR\right> + \left|RB\right>\right)
\end{array}\right)
~~~~~~~~~
\bar{3} = \left(\begin{array}{c}
\frac{1}{\sqrt{2}}\left(\left|RG\right> - \left|GR\right>\right)\\
\frac{1}{\sqrt{2}}\left(\left|GB\right> - \left|BG\right>\right)\\
\frac{1}{\sqrt{2}}\left(\left|BR\right> - \left|RB\right>\right)
\end{array}\right)~.
\end{equation}
The members of the sextet transform into (linear combinations of) 
each other under $\mrm{SU(3)}$ transformations, and similarly for the
members of the antitriplet, hence neither of these can be reduced
further. The breakdown into
irreducible $\mrm{SU(3)}$ multiplets is therefore
\begin{equation}
3 \otimes 3 = 6 \oplus \overline{3}~.
\end{equation}
Thus, an uncorrelated pair of quarks has a $1/3$ chance to add to an overall
anti-triplet state (corresponding to coherent
superpositions like ``red + green = antiblue''\footnote{In the context of
  hadronisation models, 
  this coherent superposition of two quarks in an overall antitriplet
  state is sometimes called a
  \index{Diquarks}``diquark'' (at low $m_{qq}$)
  \index{String junctions}or a ``string junction'' (at high $m_{qq}$), see
  \secRef{sec:stringModel}; it corresponds to the antisymmatric ``red
  + green = antiblue'' combination needed to create a baryon
  wavefunction. }); otherwise it is in an overall 
sextet state. 

Note that the emphasis on
the quark-(anti)quark pair being \emph{uncorrelated} is important;
production processes that correlate the produced partons, like $Z\to q\bar{q}$ or $g\to q\bar{q}$, will
project out specific components (here the singlet and octet,
respectively). 
Note also that, if the quark
and (anti)quark are on opposite sides of the universe (i.e., living in
two different hadrons), the QCD \emph{dynamics} will not care what
overall colour state they 
are in, so for the formation of multi-partonic states in QCD, obviously the
spatial part of the wave functions (causality at the very least) 
will also play a role. Here, we are considering \emph{only} the colour part
of the wave functions. 
Some additional examples are 
\begin{eqnarray}
8\otimes 8 & = & 27 \oplus 10 \oplus \overline{10} \oplus 8 \oplus 8
\oplus 1 ~,\\ 
3 \otimes 8 & = & 15 \oplus 6 \oplus 3~,\\
3 \otimes 6 & = & 10 \oplus 8~,\\
3\otimes3\otimes3 & = & (6 \oplus \overline{3}) \otimes 3 = 10 \oplus 8
\oplus 8 \oplus 1 ~.
\end{eqnarray}
Physically, the 27 in the first line corresponds to a completely
incoherent addition of the colour charges of two gluons;
\index{Decuplet}the decuplets are slightly more coherent (with a lower
total colour charge), the octets
yet more, and the singlet corresponds to the combination of two gluons
that have precisely equal and opposite colour charges, so that their
total charge is zero. 
Further extensions and generalisations of these combination rules can
\index{Young tableaux}be obtained, e.g., using the method of Young
tableaux~\cite{young1901,youngSagan}.  



\section{A Brief Introduction to VLBI}
\label{sec:vlbi}

We briefly describe VLBI to provide the necessary background for building an accurate likelihood model. Our goal is to provide intuition; for additional details 
%and derivations 
we recommend~\cite{thompson2008interferometry}.
%due to the inverse relationship between angular resolution and telescope diameter. 
As with cameras, a single-dish telescope is diffraction limited. 
%\delete{Diffraction imposes a fundamental limit on the minimum recoverable angular resolution of a $D$-diameter telescope observing at $\lambda$-wavelength to be $ \theta_{rad} \approx \lambda / D $.} 
However, simultaneously collecting data from an array of telescopes, called an interferometer, allows us to
%to decouple angular resolution from telescope diameter to 
overcome the single-dish diffraction limit. 
%The minimum angular resolution of an interferometer is inversely related to the maximum distance between telescopes. 



Figure~\ref{fig:introinterferometry} provides a simplified explanation of a two telescope interferometer. 
%Sinusoidal 
Electromagnetic radiation travels from a point source to the telescopes. However, because the telescopes are separated by a distance $B$, they will not receive the signal concurrently. 
%\footnote{This approximation is valid when the celestial emission being imaged is very far away as compared to the telescope baseline.}
For spatially incoherent extended emissions, the time-averaged correlation of the received signals is equivalent to the projection of this sinusoidal variation on the emission's intensity distribution.
%\footnote{ {\color{red}  Noise in the local atomic-clocks and baseline measurements are eliminated through a process called fringe searching that requires the use of a super-computer. }}


%Suppose we are observing a celestial point-source using two telescopes separated by a baseline vector $B$. Since the source is very far away, {\it both} telescopes are pointed at the source in the direction of the unit vector $\hat{s}$ and receive the same {\it plane-wave} signal. However, because the telescopes are separated by $B$ they will not receive the signals at the same time. 
%In Figure~\ref{fig:introinterferometry} we show geometrically that we can approximate the difference in the distance traveled to the telescopes as $ \Delta d  = B \cdot \hat{s} \label{eq:chdist} $. 
%Time-averaging the correlation of these phase-shifted signals results in a value that varies sinsoidally with the time delay. 
%For incoherant extended sources, it can be easily shown that the time-averaged correlation will be equivalent to the projection of this $\frac{B \cdot \hat{s}}{\lambda}$-frequency sinusoid on the source's spatial distribution. \katie{check this}


\begin{figure}[tb!]
	\centering
	{\includegraphics[width=\linewidth]
		{interferometryfig.pdf}}
	\caption{ \footnotesize{{\bf Simplified Interferometry Diagram:} 
			Light is emitted from a distant source and arrives at the telescopes as a plane wave in the direction $\hat{s}$. An additional distance of $B \cdot \hat{s}$ is necessary for the light to travel to the farther telescope, introducing  a time delay between the received signals that varies depending on the source's location in the sky. The time-averaged correlation of these signals is a sinusoidal function related to the location of the source. This insight is generalized to extended emissions in the van Cittert-Zernike Thm. and used to relate the time-averaged correlation to a Fourier component of the emission image in the direction $\hat{s}$.
			%Light of a single frequency is emitted from a point source and travels towards the telescopes. Since the point source is very far away, the telescopes point in the same direction, $\hat{s}$, and are hit by a plane wave. An additional distance of $B \cdot \hat{s}$ is necessary for the light to travel to the farther telescope. This extra distance introduces a time delay between the received signals. The time averaged correlation of these signals is then used to extract information about this delay. The time delay required for an emitted signal to reach the telescopes varies depending on the source location. We can use this insight to reconstruct an image of the emission.
			\vspace{-.2in} }}
	\label{fig:introinterferometry}
\end{figure}


%For incoherant extended sources, it can be shown that the time-averaged correlation of the signal from 2 telescopes is equivalent to the projection of a sinusoid on the source's spatial distribution. 
This phenomenon is formally described by the \textit{van Cittert-Zernike Theorem}. The theorem states that, for ideal sensors, the time-averaged correlation of the measured signals from two telescopes, $i$ and $j$, 
%seperated by a baseline, $B$, 
for a single wavelength, $\lambda$, can be approximated as:

\vspace{-.2in}
\begin{equation}  \Gamma_{i,j}(u,v) \approx \int_\ell{\int_{m} {e^{-i 2 \pi  (u\ell + vm) }} I_{\lambda}(\ell,m) dl} dm  \label{eq:visibility} \vspace{-.03in} \end{equation} 


\noindent{where $ I_{\lambda}(\ell,m)$ is the emission of wavelength $\lambda$ traveling from the direction  $\hat{s} = (\ell, m, \sqrt{1 - \ell^2 - m^2} )$. 
The dimensionless coordinates $(u,v)$ (measured in wavelengths) are the projected baseline, $B$, orthogonal to the line of sight.\footnote{The change in elevation between telescopes can be neglected due to corrections made in pre-processing. Additionally, for small FOVs wide-field effects are negligible.} Notice that Eq.~\ref{eq:visibility} is just the Fourier transform of the source emission image, $I_{\lambda}(\ell,m)$. Thus, $\Gamma_{i,j}(u,v)$
%\delete{the time-averaged correlation of the measured signals from two telescopes} 
provides a single complex Fourier component of $I_{\lambda}$ at position $(u,v)$ on the 2D spatial frequency plane. {\it We refer to these measurements, $\Gamma_{i,j}$, as visibilities.}
Since the spatial frequency, $(u,v)$, is proportional to the baseline, $B$, increasing the distance between telescopes increases the resolving power of the interferometer, allowing it to distinguish finer details. 


\vspace{-0.15in}
\paragraph{Earth Rotation Synthesis} At a single time, for an $N$ telescope array,
%, for every wavelength $\lambda$, 
%at a given time 
we obtain $ \frac{N(N-1)}{2} $ visibility measurements corresponding to each pair of telescopes. 
%we obtain measurements for $ \frac{N(N-1)}{2} $ constraints corresponding to the spatial frequency $(u,v)$ for each pair of telescopes. 
As the Earth rotates, the direction that the telescopes point towards the source ($\hat{s}$) changes. 
%Assuming a static source, this results in measuring different visibilities along elliptical paths in the $(u,v)$ frequency plane 
Assuming a static source, this yields measurements of spatial frequency components (visibilities) of the desired image along elliptical paths in the $(u,v)$ frequency plane (see Fig.~\ref{fig:uvcov}b). 
%Sampling $K$ times over the course of a day increases the number of frequency constraints $K$-fold.

\vspace{-0.15in}
\paragraph{Phase Closure} All equations thus far assumed that light travels from the source to a telescope through a vacuum. However, inhomogeneities in the atmosphere cause the light to travel at different velocities towards each telescope. These delays have a significant effect on the phase of measurements, and renders the phase unusable for image reconstructions at wavelengths less than 3 mm~\cite{monnier2013radio}. 

Although absolute phase measurements cannot be used, a clever observation - termed phase closure - allows us to still recover some information from the phases. 
The atmosphere affects an ideal visibility (spatial frequency measurement) by 
%The error due to the atmosphere for a single visibility measurement, 
introducing an additional phase term: $\Gamma_{i,j}^{\mbox{\tiny{meas}}} = e^{i(\phi_i - \phi_j)}\Gamma_{i,j}^{\mbox{\tiny{ideal}}}$,
%\begin{equation} \Gamma_{i,j}^{\mbox{\tiny{meas}}} = e^{i(\phi_i - \phi_j)}\Gamma_{i,j}^{\mbox{\tiny{ideal}}} \end{equation}
\noindent{where $\phi_i$ and $\phi_j$ are the phase delays introduced in the path to telescopes $i$ and $j$ respectively. By multiplying the visibilities from three different telescopes, we obtain an expression that is invariant to the atmosphere, as the unknown phase offsets cancel, see Eq.~\ref{eq:phaseclosure}~\cite{felli1989very}.  }

\vspace{-.2in}
\begin{align}  
\notag \Gamma^{\mbox{\tiny{meas}}}_{i,j}\Gamma^{\mbox{\tiny{meas}}}_{j,k}\Gamma^{\mbox{\tiny{meas}}}_{k,i} &= e^{i(\phi_i-\phi_j)}\Gamma^{\mbox{\tiny{ideal}}}_{i,j}e^{i(\phi_j-\phi_k)}\Gamma^{\mbox{\tiny{ideal}}}_{j,k}e^{i(\phi_k-\phi_i)}\Gamma^{\mbox{\tiny{ideal}}}_{k,i} \\
%= e^{i(\phi_i-\phi_j+\phi_j-\phi_k + \phi_k-\phi_i)}\Gamma^{\mbox{\tiny{true}}}_{i,j}\Gamma^{\mbox{\tiny{true}}}_{j,k}\Gamma^{\mbox{\tiny{true}}}_{k,i}  
&=\Gamma^{\mbox{\tiny{ideal}}}_{i,j}\Gamma^{\mbox{\tiny{ideal}}}_{j,k}\Gamma^{\mbox{\tiny{ideal}}}_{k,i} 
\label{eq:phaseclosure} 
 \end{align}
 


{\it We refer to this triple product of visibilities as the bispectrum}. The bispectrum is invariant to atmospheric noise; however, in exchange, it reduces the number of constraints that can be used in image reconstruction. Although the number of triple pairs in an $N$ telescope array is ${N\choose 3}$, the number of independent values is only $\frac{(N-1)(N-2)}{2}$.
%As $N$ becomes large, the ratio between the number of independent bispectrum values and the number of visibility measurements, $\frac{N(N-1)}{2}$, approaches unity.
%However, 
For small telescope arrays, such as the EHT, this effect is large. 
For instance, in an eight telescope array, using the bispectrum rather than visibilities results in 25\% fewer independent 
%bispectrum 
constraints 
%than visibilities
~\cite{felli1989very}.


%%\delete{Note how the phase delays perfectly cancel in the bispectrum. }
%{\it We refer to this triple product of visibilities as the bispectrum}. The bispectrum is invariant to a large amount of noise; 
%%on the other hand, 
%however, in exchange,
%%\katie{dont know if i like this change} 
%it reduces the number of constraints that can be used in image reconstruction. Although the number of triple pairs in an $N$ telescope array is ${N\choose 3}$, the number of independent values is only $\frac{(N-1)(N-2)}{2}$.
%As $N$ becomes large, the ratio between the number of independent bispectrum values and the number of visibility measurements, $\frac{N(N-1)}{2}$, approaches unity.
%However, for small telescope arrays, such as the EHT, the effect is large. 
%For instance, in an eight telescope array, this results in 25\% fewer independent 
%%bispectrum 
%constraints 
%%than visibilities
%~\cite{felli1989very}.


%However, for small telescope arrays, the effect is large. For instance, in a seven telescope array, there are only $15$ independent triple product measurements at a given time - this results in 28\% fewer independent bispectrum constraints when compared to the use of visibilities~\cite{felli1989very}.
 
\vspace{-.05in}
\section{Related Work}
\label{sec:related}
\vspace{-.05in}


 
%More BLAH however is the fact that absolute phase of VLBI Fourier measurements (visibilities) are lost due to atmospheric path delays. In SAR the Fourier samples are all coherantly related and the absolute phase can generally be recovered. However, 


%Note that this problem differs from traditional sparse spectral reconstruction methods (e.g MRI, CT) due to the large atmospheric phase errors. 

 %However, unlike in VLBI, 
 
 %Additionally, unlike in VLBI, traditional SAR and MRI are generally not plagued by large corruption of the signal's phase. High-resolution spaceborne SAR, on the other hand, does suffer from atmospheric path delays~\cite{atmosphereSAR}, which could be partially alleviated by modifying the system and incorporating phase closure in imaging, as is done in this work.

%multiple platforms into one super sar

%SAR sampling pattern is determined by trajectory of the pattern 
%typically measurements are made along a straight line

%SAR typically has uniform sampling of 
%In SAR resolves objects in range andn cross range where range is distance from antenna and range i perpendicular
%using conventuall sar with repeating chip (ie linear freq modulation ) waveforms you get a uniform sampling grid in spatial frequency corresponding to the range and cross range directions. 

%but more importantly those samples in the fourier domain are all coherantly related and the absolute phase can generally be recovered

%In there is an application with SAR there is no phase 

%the measurments in the Fourier space are all relatively coherant so you can do inverse 




%VLBI image reconstruction shares a lot in common with other sparse spectral image reconstruction problems, such as Synthetic Aperture Radar (SAR) and Magnetic Resonance Imaging (MRI)~\cite{bracewell2004fourier, compressedSAR}. In particular, although very different in practice, SAR is based on many of the same core ideas. 
%However, VLBI faces unique challenges that are not sufficiently addressed in these other fields.  %explored/
%For instance, the sampling of an image's Fourier plane is limited by the number and location of viable telescopes, whereas SAR and MRI both have much more flexibility in densely and uniformly sampling the plane. 
%Additionally, unlike in VLBI, traditional SAR and MRI are generally not plagued by large corruption of the signal's phase.  
%High-resolution spaceborne SAR on the other hand does suffer from atmospheric path delays~\cite{atmosphereSAR} that could be partially alleviated by modifying the system and incorporating phase closure in imaging, as is done in this work.


%However, the correction of atmospheric path delays in spaceborn SAR has become increasingly important 



%http://www.geo.uzh.ch/microsite/rsl-documents/research/publications/peer-reviewed-articles/sensors-08-08479-0152375552/sensors-08-08479.pdf
%http://www.geo.uzh.ch/microsite/rsl-documents/research/SARlab/Publications/PDF/JFS+04_poster.pdf

%http://download.springer.com/static/pdf/100/chp%253A10.1007%252F978-3-642-38398-4_13.pdf?originUrl=http%3A%2F%2Flink.springer.com%2Fchapter%2F10.1007%2F978-3-642-38398-4_13&token2=exp=1460032536~acl=%2Fstatic%2Fpdf%2F100%2Fchp%25253A10.1007%25252F978-3-642-38398-4_13.pdf%3ForiginUrl%3Dhttp%253A%252F%252Flink.springer.com%252Fchapter%252F10.1007%252F978-3-642-38398-4_13*~hmac=ad1ca5fad5ffee145885e5a423b3c4d472973f49766b40eb8fdcceff802d26b0

%The correction of atmospheric path delays in high-resolution spaceborne synthetic aperture radar systems has become increasingly important with continuing improvements to the resolution of SAR systems surveying the Earth. Atmospheric path delays must be taken into account in order to achieve geolocation accuracies better than 1 meter. These effects are mainly due to ionospheric and tropospheric influences. 

We summarize a few significant algorithms from the astronomical interferometry imaging literature.

%Image reconstruction from VLBI data has been studied for decades. We summarize a few significant algorithms.
%Reconstructing images from VLBI data has been an active field of research for some time. %decades. 
%While it is out of this paper's scope to provide a full survey of current literature, we summarize a few significant algorithms.
%present a brief summary of a few representative algorithms.


%\delete{ most applicable to VLBI image reconstruction.}

\vspace{-.15in}
\paragraph{CLEAN}

CLEAN is the de-facto standard method used for VLBI image reconstruction. It assumes that the image is made up of a number of bright point sources. From an initialization image, CLEAN iteratively looks for the brightest point in the image and ``deconvolves" around that location by removing side lobes that occur due to sparse sampling in the $(u,v)$ frequency plane. After many iterations, the final image of point sources is blurred~\cite{hogbom1974aperture}. Since CLEAN assumes a distribution of point sources, it often struggles with reconstructing images of extended emissions~\cite{taylor1999synthesis}.


For mm/sub-mm wavelength VLBI, reconstruction is complicated by corruption of the visibility phases. CLEAN is not inherently capable of handling this problem; however, self-calibration methods have been developed to greedily recover the phases during imaging. Self-calibration requires manual input from a knowledgeable user and often fails when the SNR is too low or the source is complex~\cite{taylor1999synthesis}. 


% CLEAN is the de-facto standard method used for VLBI image reconstruction. It works under the assumption that the image is made up of a number of bright point sources. Starting from an initialization, CLEAN iteratively looks for the brightest point in the image and ``deconvolves" around that location by removing side lobes that occur due to sparse sampling in the $(u,v)$ frequency plane. After many iterations, the final image is then blurred to de-emphasize spurious high frequencies~\cite{hogbom1974aperture}. Since CLEAN assumes a distribution of bright point sources, it often struggles with reconstructing images containing extended emissions~\cite{taylor1999synthesis}.

%For short-wavelength VLBI, reconstruction is complicated by the corruption of phase in visibility measurements. CLEAN is not inherently capable of handling this problem; however, self-calibration methods have been developed to greedily iterate back and forth between CLEANing the image and solving for closure constraints. Self-calibration requires manual input from a knowledgeable user and often fails when the SNR is too low or the source is complex~\cite{taylor1999synthesis}. 

%Although CLEAN is over 35 years old, theoretical understanding of the algorithm is limited. Success using CLEAN generally requires the use of many tricks and manual parameter tuning - especially in the case of self-calibration. Additionally, since it assumes a distribution of bright point sources, it struggles with reconstructing images containing extended emissions

 

\vspace{-.15in}
\paragraph{Optical Interferometry} 
Interferometry at visible wavelengths faces the same phase-corruption challenges as mm/sub-mm VLBI. %Therefore, in recent years a number of algorithms have been developed to try to reconstruct images under these conditions. 
Although historically the optical and radio interferometry communities have been separate, fundamentally the resulting measurements and imaging process are very similar~\cite{monnier2013radio}. 
We have selected two optical interferometry reconstruction algorithms representative of the field to discuss and compare to in this work~\cite{rusenimaging}. Both algorithms take a regularized maximum likelihood approach and can use the bispectrum, rather than visibilities, for reconstruction~\cite{baron2010novel, buscher1994direct}. Recent methods based on compressed sensing have been proposed, but have yet to demonstrate superior results~\cite{compressedsensing, rusenimaging}. 
%\footnote{Recent methods based on compressed sensing have yet to demonstrate superior results. CITATION}

%Interferometry at visible wavelengths faces the same phase-corruption challenges as mm-wavelength VLBI. %Therefore, in recent years a number of algorithms have been developed to try to reconstruct images under these conditions. 
%Although historically the optical and radio interferometry communities have been separate due to large differences in the specifics of data collection, fundamentally the resulting measurements and imaging process are the same~\cite{monnier2013radio}. 
%We have selected two optical interferometry reconstruction algorithms representative of the field to discuss and compare to in this work~\cite{rusenimaging}. Both algorithms take a regularized maximum likelihood approach and use the bispectrum, rather than visibilities, for reconstruction~\cite{buscher1994direct, baron2010novel}. Recent methods based on compressed sensing have been proposed, but have yet to demonstrate superior results~\cite{rusenimaging, compressedsensing}. 
%%\footnote{Recent methods based on compressed sensing have yet to demonstrate superior results. CITATION}

 BSMEM (BiSpectrum Maximum Entropy Method) takes a Bayesian approach to image reconstruction~\cite{buscher1994direct}. Gradient descent optimization~\cite{skilling1990quantified} using a maximum entropy prior is used to find an optimal reconstruction of the image. 
 %While there are many choices for the entropy function, BSMEM uses the Gull and Skilling entropy measure~\cite{skilling1990quantified}, generally with a flat or Gaussian model prior. 
 %As the winner of the last BLAH optical interferomic beauty contents, BSMEM has been the most successful method in recent years. 
 Under a flat image prior BSMEM is often able to achieve impressive super-resolution results on simple celestial images. However, in Section~\ref{section:results} we demonstrate how it often struggles on complex, extended emissions.
 %, which are believed to be present around black holes~\cite{fish2014imaging}.
 %extended emissions such as one expects to see around black holes.


% BSMEM (BiSpectrum Maximum Entropy Method), like our proposed algorithm, takes a Bayesian approach to image reconstruction~\cite{buscher1994direct}. Gradient descent optimization~\cite{skilling1990quantified} using a maximum entropy prior is used to find an optimal reconstruction of the image. While there are many choices for the entropy function, BSMEM uses the Gull and Skilling entropy measure~\cite{skilling1990quantified}, generally with a flat or Gaussian model prior. 
% %As the winner of the last BLAH optical interferomic beauty contents, BSMEM has been the most successful method in recent years. 
% Under a flat image prior BSMEM is often able to achieve impressive super-resolution results on simple celestial images. However, in Section~\ref{section:results} we demonstrate how it often struggles on complex, extended emissions, which are believed to be present around black holes~\cite{fish2014imaging}.
% %extended emissions such as one expects to see around black holes. 


SQUEEZE takes a Markov chain Monte Carlo (MCMC) approach to sample images from a posterior distribution~\cite{baron2010novel}. To obtain a sample image, SQUEEZE moves a set of point sources around the field of view (FOV). The final image is then calculated as the average of a number of sample images. 
Contrary to gradient descent methods, SQUEEZE is not limited in its choice of regularizers or constraints~\cite{rusenimaging}. 
%Additionally, under a specified set of parameters, given enough time, the algorithm is guaranteed to find the distribution's expected image. 
However, this freedom comes at the cost of a large number of parameter choices that may be hard for an unknowledgeable user to select and tune. 

%SQUEEZE takes a Markov chain Monte Carlo (MCMC) approach to sample images from a posterior probability distribution~\cite{baron2010novel}. To obtain a sample image, SQUEEZE moves a set of point sources around the field of view (FOV) until a maximum number of iterations has been reached. The final image is then calculated as the mean of a chosen number of sample images. 
%Contrary to gradient descent methods, SQUEEZE is not limited in its choice of regularizers or constraints~\cite{rusenimaging}. Additionally, under a specified set of parameters, given enough time, the algorithm is guaranteed to find the distribution's expected image. However, this freedom comes at the cost of a large number of parameter choices that may be hard for an unknowledgable user to select and tune. 

\subsection{Spectral Image Reconstruction}

VLBI image reconstruction has similarities with other spectral image reconstruction problems, such as Synthetic Aperture Radar (SAR), Magnetic Resonance Imaging (MRI), and Computed Tomography (CT)~\cite{bracewell2004fourier, lustig2007sparse, 1456966, thibault2007three}. %, although a detailed comparison is beyond the scope of this work
However, VLBI faces a number of challenges that are typically not relevant in these other fields. % (e.g. sampling limits, SNR). 
%For instance, in VLBI the sampling of an image's Fourier plane is limited by the number and location of viable telescopes, whereas other methods typically have more flexibility in densely and uniformly sampling the Fourier plane. 
For instance, SAR, MRI, and CT are generally not plagued by large corruption of the signal's phase, as is the case due to atmospheric differences in mm/sub-mm VLBI. In SAR the Fourier samples are all coherently related and the absolute phase can generally be recovered, even under atmospheric changes~\cite{533208, 6504845}. However, although fully understanding the connection remains an open problem, incorporating ideas of phase closure, as is done in this work, may open the potential to push SAR techniques~\cite{atmosphereSAR} past their current limits.


\section*{Background}		\label{p:background}


\concept{Category Theory}

Here is a summary of the categorical background and terminology needed in order
to read the entire paper.  The reader who isn't familiar with everything below
shouldn't be put off: each individual Definition only uses some of it.

I assume familiarity with \demph{categories}, \demph{functors},
\demph{natural transformations}, \demph{adjunctions}, \demph{limits}, and
\demph{monads} and their \demph{algebras}.  Limits include \demph{products},
\demph{pullbacks} (with the pullback of a diagram $X \go Z \og Y$ sometimes
written $X \times_Z Y$), and \demph{terminal objects} (written $1$,
especially for the terminal set $\{ * \}$); we also use
\demph{initial objects}.  A monad $(T,\eta,\mu)$ is often abbreviated to $T$.

I make no mention of the difference between sets and classes (`small
and large collections').  All the Definitions are really of \emph{small}
weak $n$-category.

Let \cat{C} be a category.  $X\in \cat{C}$ means that $X$ is an object of
$\cat{C}$, and $\cat{C}(X,Y)$ is the set of morphisms (or \demph{maps}, or
\demph{arrows}) from $X$ to $Y$ in \cat{C}.  If $f\in \cat{C}(X,Y)$ then $X$
is the \demph{domain} or \demph{source} of $f$, and $Y$ the \demph{codomain}
or \demph{target}.

\Set\ is the category (sets $+$ functions), and \Cat\ is (categories $+$
functors).  A set is just a \demph{discrete category} (one in which the only
maps are the identities).

$\cat{C}^\op$ is the \demph{opposite} or \demph{dual} of a category
\cat{C}.  $\ftrcat{\cat{C}}{\cat{D}}$ is the category of functors from
$\cat{C}$ to $\cat{D}$ and natural transformations between them.  Any object
$X$ of $\cat{C}$ induces a functor $\cat{C}(X, \dashbk): \cat{C} \go \Set$,
and a natural transformation from $\cat{C}(X, \dashbk)$ to
$F: \cat{C} \go \Set$ is the same thing as an element of $FX$ (the
\demph{Yoneda Lemma}); dually for $\cat{C}(\dashbk,X): \cat{C}^\op \go \Set$.

A functor $F: \cat{C} \go \cat{D}$ is an \demph{equivalence} if these
equivalent conditions hold: (i) $F$ is full, faithful and essentially
surjective on objects; (ii) there exist a functor $G: \cat{D} \go \cat{C}$ (a
\demph{pseudo-inverse} to $F$) and natural isomorphisms $\eta: 1 \go GF$,
$\epsln: FG \go 1$ ; (iii) as~(ii), but with $(F,G,\eta,\epsln)$ also being
an adjunction.

Any set $\cat{D}_0$ of objects of a category \cat{C} determines a \demph{full
subcategory} \cat{D} of \cat{C}, with object-set $\cat{D}_0$ and
$\cat{D}(X,Y) = \cat{C}(X,Y)$.  Every category \cat{C} has a
\demph{skeleton}: a subcategory whose inclusion into \cat{C} is an
equivalence and in which no two distinct objects are isomorphic.  If $F, G:
\cat{C} \go \Set$, $GX \sub FX$ for each $X \in \cat{C}$, and $F$ and $G$
agree on morphisms of \cat{C}, then $G$ is a
\demph{subfunctor} of $F$.

A \demph{total order} on a set $I$ is a reflexive transitive
relation $\leq$ such that if $i \neq j$ then exactly one of $i\leq j$ and
$j\leq i$ holds.  
$(I,\leq)$ can be seen as a category with object-set $I$
in which each hom-set has at most one element.
An \demph{order-preserving map} $(I,\leq) \go (I',\leq')$
is a function $f$ such that $i \leq j \implies f(i) \leq' f(j)$.

Let \Del\ be the category with objects $[k]=\{0,\ldots,k\}$ for $k\geq 0$,
and order-preserving functions as maps.  A \demph{simplicial set} is a
functor $\Delop \go \Set$.  Every category \cat{C} has a \demph{nerve} (the
simplicial set $N\cat{C}: [k] \goesto \Cat([k],\cat{C})$), giving a full and
faithful functor $N: \Cat \go \ftrcat{\Delop}{\Set}$.  So \Cat\ is equivalent
to the full subcategory of \ftrcat{\Delop}{\Set} with objects $\{ X \such X
\iso N\cat{C} \textrm{ for some } \cat{C} \}$; there are various
characterizations of such $X$, but we come to that in the main text.

Leftovers: a \demph{monoid} is a set (or more generally, an object of a
monoidal category) with an associative binary operation and a two-sided unit.
\Cat\ is monadic over the category of directed graphs.  The \demph{natural
numbers} start at $0$.


\clearpage



\concept{Strict $n$-Categories}


If \cat{V} is a category with finite products then there is a category
$\cat{V}\hyph\Cat$ of \cat{V}-enriched categories and \cat{V}-enriched
functors, and this itself has finite products.  (A \demph{\cat{V}-enriched
category} is just like an ordinary category, except that the `hom-sets' are
now objects of \cat{V}.)  Let $0\hyph\Cat = \Set$ and, for $n\geq 0$,
$(n+1)\hyph\Cat = (n\hyph\Cat)\hyph\Cat$; a \demph{strict $n$-category} is an
object of $n\hyph\Cat$.  Note that $1\hyph\Cat = \Cat$.

Any finite-product-preserving functor $U: \cat{V} \go \cat{W}$
induces a finite-product-preserving functor $U_*: \cat{V}\hyph\Cat \go
\cat{W}\hyph\Cat$, so we can define functors $U_n: (n+1)\hyph\Cat \go
n\hyph\Cat$ by taking $U_0$ to be the objects functor and $U_{n+1} =
(U_n)_*$.  The category $\omega\hyph\Cat$ of \demph{strict
$\omega$-categories} is the limit of the diagram
\[
\cdots 
\goby{U_{n+1}} (n+1)\hyph\Cat  \goby{U_n} 
\cdots
\goby{U_1} 1\hyph\Cat = \Cat
\goby{U_0} 0\hyph\Cat = \Set.
\]

Alternatively: a \demph{globular set} (or \demph{$\omega$-graph}) $A$
consists of sets and functions
\[
\cdots 
\parpair{s}{t} A_m  \parpair{s}{t} A_{m-1} \parpair{s}{t} 
\cdots 
\parpair{s}{t} A_0
\]
such that for $m\geq 2$ and $\alpha\in A_m$, $ss(\alpha) = st(\alpha)$ and
$ts(\alpha) = tt(\alpha)$.  An element of $A_m$ is called an
\demph{$m$-cell}, and we draw a $0$-cell $a$ as $\gzeros{a}$, a $1$-cell $f$
as $\gfsts{a}\gones{f}\glsts{b}$ (where $a=s(f), b=t(f)$), a 2-cell $\alpha$
as $\gfsts{a}\gtwos{f}{g}{\alpha}\glsts{b}$, etc.  For $m > p \geq 0$, write
$ A_m \times_{A_p} A_m = \{ (\alpha',\alpha) \in A_m \times A_m \such
s^{m-p}(\alpha') = t^{m-p}(\alpha) \}$.

A \demph{strict $\omega$-category} is a globular set $A$ together with a
function $\ofdim{p}: A_m \times_{A_p} A_m \go A_m$ (\demph{composition}) for
each $m > p \geq 0$ and a function $i: A_m \go A_{m+1}$ (\demph{identities},
usually written $i(\alpha) = 1_\alpha$) for each $m\geq 0$, such that
%
\begin{enumerate}
\item 	\label{part:strict-n:source-comp}
if $m > p \geq 0$ and
$(\alpha',\alpha) \in A_m \times_{A_p} A_m$ then
\[
\begin{array}{llll}
s(\alpha' \ofdim{p} \alpha) = 
s(\alpha) 			&	
\textrm{and}			&
t(\alpha' \ofdim{p} \alpha) = 
t(\alpha') 			&
\textrm{for }
m=p+1	\\
s(\alpha' \ofdim{p} \alpha) = 
s(\alpha') \ofdim{p} s(\alpha)	&
\textrm{and}			&
t(\alpha' \ofdim{p} \alpha) = 
t(\alpha') \ofdim{p} t(\alpha)	&
\textrm{for }
m\geq p+2	
\end{array}
\]
\item  	\label{part:strict-n:source-id}
if $m\geq 0$ and $\alpha\in A_m$ then $s(i(\alpha)) = \alpha =
t(i(\alpha))$ 
\item \label{part:strict-n:ass-and-id} 
if $m > p \geq 0$ and $\alpha \in A_m$ then $i^{m-p}(t^{m-p}(\alpha))
\ofdim{p} \alpha = \alpha = \alpha \ofdim{p}$\linebreak
$i^{m-p}(s^{m-p}(\alpha))$; if also $\alpha', \alpha''$ are such that
$(\alpha'', \alpha'), (\alpha', \alpha) \in A_m \times_{A_p} A_m$, then
$(\alpha'' \ofdim{p} \alpha') \ofdim{p} \alpha = \alpha'' \ofdim{p} (\alpha'
\ofdim{p} \alpha)$
\item  	\label{part:strict-n:int}
if $p>q\geq 0$ and $(f',f) \in A_p \times_{A_q} A_p$ then
$i(f') \ofdim{q} i(f) = i(f' \ofdim{q} f)$; 
if also $m>p$ and $\alpha,\alpha',\beta,\beta'$ are such that
$(\beta',\beta), (\alpha', \alpha) \in A_m \times_{A_p} A_m$, 
$(\beta',\alpha'), (\beta, \alpha) \in A_m \times_{A_q} A_m$,
then 
$(\beta' \ofdim{p} \beta) \ofdim{q} (\alpha' \ofdim{p} \alpha) 
= 
(\beta' \ofdim{q} \alpha') \ofdim{p} (\beta \ofdim{q} \alpha)$.
\end{enumerate}

The composition $\ofdim{p}$ is `composition of $m$-cells by gluing along
$p$-cells'.  Pictures for $(m,p) = (2,1), (1,0), (2,0)$ are in the
Bicategories section below. 

\demph{Strict $n$-categories} are defined similarly, but with the globular
set only going up to $A_n$.  \demph{Strict $n$- and $\omega$-functors} are
maps of globular sets preserving composition and identities; the categories
$n\hyph\Cat$ and $\omega\hyph\Cat$ thus defined are equivalent to the ones
defined above.  The comments below on the two alternative definitions of
bicategory give an impression of how this equivalence works. 

\clearpage




\concept{Bicategories}

Bicategories are the traditional and best-known formulation of `weak
2-category'.  

A \demph{bicategory} $B$ consists of
%
\begin{itemize}
\item a set $B_0$, whose elements $a$ are called \demph{0-cells} or
\demph{objects} of $B$ and drawn
$\gzeros{a}$
\item for each $a,b \in B_0$, a category $B(a,b)$, whose objects $f$ are
called \demph{1-cells} and drawn $\gfsts{a} \gones{f} \glsts{b}$, whose
arrows $\alpha: f \go g$ are called \demph{2-cells} and drawn $\gfsts{a}
\gtwos{f}{g}{\alpha} \glsts{b}$, and whose composition $\gfsts{a}
\gthrees{f}{g}{h}{\alpha}{\beta} \glsts{b} \goesto \gfsts{a}
\gtwos{f}{h}{\!\!\!\!\!\! \beta \sof \alpha} \glsts{b}$ is called
\demph{vertical composition} of 2-cells
\item for each $a \in B_0$, an object $1_a \in B(a,a)$ (the \demph{identity}
on $a$); and for each $a,b,c \in B_0$, a functor $B(b,c) \times B(a,b) \go
B(a,c)$, which on objects is called \emph{1-cell composition},
$\gfsts{a}\gones{f}\gblws{b}\gones{g}\glsts{c} \goesto 
\gfsts{a}\gones{g\sof f}\glsts{c}$, and on arrows is called \demph{horizontal
composition} of 2-cells, $\gfsts{a} \gtwos{f}{g}{\alpha} \gfbws{a'}
\gtwos{f'}{g'}{\alpha'} \glsts{a''} \goesto \gfsts{a} \gtwos{f' \sof f}{g' \sof
g}{\!\!\!\!\!\! \alpha' * \alpha} \glsts{a''}$ 
\item \demph{coherence 2-cells}: for each $f \in B(a,b), g \in B(b,c), h \in
B(c,d)$, an \demph{associativity isomorphism} $\xi_{h,g,f}: (h\of g)\of f
\go h\of (g\of f)$; and for each $f \in B(a,b)$, \demph{unit isomorphisms}
$\lambda_f: 1_b \of f \go f$ and $\rho_f: f \of 1_a \go f$
\end{itemize}
%
satisfying the following \demph{coherence axioms}:
%
\begin{itemize}
\item $\xi_{h,g,f}$ is natural in $h$, $g$ and $f$, and $\lambda_f$ and
$\rho_f$ are natural in $f$
\item if $f \in B(a,b), g \in B(b,c), h \in B(c,d), k
\in B(d,e)$, then 
$
\xi_{k,h,g\sof f} \,\of\, \xi_{k\sof h, g, f} = 
(1_k * \xi_{h,g,f}) \,\of\, \xi_{k,h\sof g,f} \,\of\, (\xi_{k,h,g} * 1_f)
$
(the \demph{pentagon axiom});
and if $f \in B(a,b), g \in B(b,c)$, then 
$
\rho_g * 1_f =
(1_g * \lambda_f) \,\of\, \xi_{g,1_b,f}
$
(the \demph{triangle axiom}).
\end{itemize}

An alternative definition is that a bicategory consists of sets and functions
$B_2 \parpair{s}{t} B_1 \parpair{s}{t} B_0$ satisfying $ss=st$ and $ts=tt$,
together with functions determining composition, identities and coherence
cells (in the style of the second definition of strict $\omega$-category
above).  The idea is that $B_m$ is the set of $m$-cells and that $s$ and $t$
give the source and target of a cell.  Strict 2-categories can be identified
with bicategories in which the coherence 2-cells are all identities.

A 1-cell $\gfsts{a}\gones{f}\glsts{b}$ in a bicategory $B$ is called an
\demph{equivalence} if there exists a 1-cell $\gfsts{b}\gones{g}\glsts{a}$
such that $g\of f \iso 1_a$ and $f\of g \iso 1_b$.  

A \demph{monoidal category} can be defined as a bicategory with only one
0-cell: for if the 0-cell is called $\star$ then the bicategory just consists
of a category $B(\star,\star)$ equipped with an object $I$, a functor
$\otimes: B(\star,\star)^2 \go B(\star,\star)$, and associativity and unit
isomorphisms satisfying coherence axioms.

We can consider \demph{strict functors} of bicategories, in which composition
etc is preserved strictly; more interesting are \demph{weak functors} $F$, in
which there are isomorphisms $Fg \of Ff \go F(g \of f)$, $1_{Fa} \go
F(1_a)$ satisfying coherence axioms.

\section{Static Model \& Inference}
\label{sec:static}

%We review interferometric imaging for a static emission in the case of a multivariate Gaussian image prior. The techniques and insights obtained in this section will lead to 


Before discussing our proposed approach to dynamic imaging for time-varying sources, we first review %the basics of 
interferometric imaging for a static source and discuss a simple, yet instructive, approach using a multivariate Gaussian image prior. 
The intent of this section is {\it not} to present a novel and competitive static imaging method, but instead to set up the tools necessary to easily understand dynamic imaging in Sections~\ref{sec:dynamic_model} and~\ref{sec:dynamic_inference}.
%For simplicity, in this section we assume measurements are a vector of complex visibilities, $y$, with Gaussian noise (i.e. no atmospheric error), of variance $\sigma^2$. 

We measure 
a vector of real values $\meas$ 
%(e.g $y = [\Re (\bm{\vis}), \Im( \bm{\vis})]^T$), 
that are generated by observing a static source's emission region image, $I(\xpos, \ypos)$. These measurements are extremely sparse and noisy, and thus do not fully characterize the underlying image. 
For example, a simple PCA analysis on rows of the DTFT matrix $\FTmtx$ for the EHT 2017 campaign (see the uv-coverage of Fig.~\ref{fig:staticimaging}) shows that 95\% of the variance can be described using only 1624 measurement sub-functions, $g(\im)$, for a 10000 pixel image; essentially $\FTmtx$ constrains only 16\% of the unknowns. 
To solve this problem, we impose a prior distribution on $\im$ and seek a maximum a posteriori (MAP) estimate of the underlying image given these sparse observations. We adopt the model presented in~\cite{bouman2016computational} to represent $I(\xpos, \ypos)$ as vectorized coefficients, $\im$. Using this representation, we define our observation model as:
\begin{align}
\meas & \sim \mathcal{N}_{\meas}(f(\im), \bR), \\
\im & \sim \mathcal{N}_{\im}(\bmu, \bLambda),
\end{align}
\noindent{where $\mathcal{N}_{z}(m, \Sigma)$ is the multivariate normal distribution of $z$ with mean $m$ and covariance $\Sigma$. In the chosen model, both the data likelihood, $p(\meas|\im)$, and the underlying image prior, $p(\im)$, are multivariate normal distributions. The posterior probability is written in terms of these two terms: % the data likelihood, $p(\meas|\im)$, and the image prior, $p(\im)$. 
}
\begin{align} 
\label{eq:bayes}
p(\im|\meas) & \propto p(\meas|\im) p(\im) \\
\notag & = \mathcal{N}_{\meas}(f(\im), \bR) \mathcal{N}_{\im}(\bmu, \bLambda).
\end{align}

%In this model we assume that both the observed data products, $\meas$, and the true underlying image, $\im$, are samples from multivariate Gaussian distributions. 
%Note that Equation BLAH is only valid if every linear combination of $\meas$ results in a univariate Gaussian distribution. 
Unfortunately, $p(\meas|\im)$ is not truly Gaussian when $\meas$ is composed of bispectra or closure phases, as each visibility is used to compute multiple terms. However, as discussed in Section~\ref{sec:bispec}, we assume that each term of $\meas$ is independent and can be described with a Gaussian noise model. This approximation has been shown to be a good approximation in practice (see the supplemental material)~\cite{TMS,bouman2016computational}.

%Each vector of measured data products, $\meas$, is assumed to be a function of an underlying image, $\im$, (refer to Section BLAH) with added Gaussian noise. The true underlying image, $\im$, is assumed to be a sample from a multivariate Gaussian distribution. 

\begin{figure}[tb]
                       	\begin{center}
                       		\begin{tabular}{  c | c | c   }
                       			%\hline
                       			&\large{\textsf{$\bLambda$}}   &\large{\textsf{SAMPLES }} \hspace{.55in}   \\ \hline
                     
&\vspace{-.1in}& \\
                       			\multirow{1}{*}[.6in]{ \rotatebox[origin=t]{90}{\large{\textsf{a = 2}} }}
                                &
{{ \includegraphics[width=0.2\linewidth]{figures/prior/outfile_drop2_cropped.pdf}} } &
                       			{ \includegraphics[width=0.2\linewidth]{figures/prior/newfiles/sampfig_drop2_1_scale2.pdf}} \includegraphics[width=0.2\linewidth]{figures/prior/newfiles/sampfig_drop2_4.pdf} 
                       			\multirow{3}{*}[.6in]{ \includegraphics[width=0.155\linewidth]{figures/prior/newfiles/sampfig_drop2_1_cbar.pdf} }
                                \\
                       			&\vspace{-.1in}&\\
                       			\multirow{1}{*}[.6in]{ \rotatebox[origin=t]{90}{\large{\textsf{a = 3}} }} & 
                       			{{ \includegraphics[height=0.2\linewidth]{figures/prior/outfile_drop3_cropped.pdf}} } &
                       			{ \includegraphics[height=0.2\linewidth]{figures/prior/newfiles/sampfig_drop3_2.pdf}} \includegraphics[height=0.2\linewidth]{figures/prior/newfiles/sampfig_drop3_3.pdf}  
                       			\hspace{.65in}
                       			%\multirow{3}{*}[.6in]{ \includegraphics[width=0.18\linewidth]{figures/prior/newfiles/placeholder.pdf} }
                       			\\
                       			&\vspace{-.1in}& \\
                       			\multirow{1}{*}[.6in]{ \rotatebox[origin=t]{90}{\large{\textsf{a = 4}} }} & 
                       			{{ \includegraphics[height=0.2\linewidth]{figures/prior/outfile_drop4_cropped.pdf}} } &
                       			{ \includegraphics[height=0.2\linewidth]{figures/prior/newfiles/sampfig_drop4_1.pdf}} \includegraphics[height=0.2\linewidth]{figures/prior/newfiles/sampfig_drop4_2.pdf}                        			
                       			\hspace{.65in}
                       			%\multirow{3}{*}[.6in]{ \includegraphics[width=0.18\linewidth]{figures/prior/newfiles/placeholder.pdf} } 
                       			\\            	
                       		\end{tabular}
                       		\caption{\footnotesize{{\bf Gaussian Image Prior:} The covariance matrix constructed for $a=2,3,4$ along with image samples from the prior  $\mathcal{N}_x(\mu, \Lambda)$. The image samples have a field of view of 160 $\mu$-arcseconds. Notice that as $a$ increases, the sampled images appear smoother (i.e., the prior encourages smoother structure). In these examples $\mu$ is a 2D Gaussian image with standard deviation of 75 $\mu$-arcseconds. and $c=0.5$. 
                       			}}
                       			\label{fig:priorsamples}
\end{center}
\vspace{-.2in}
\end{figure}

\begin{figure*}[h!]
	\vspace{-.0in}
	\setlength{\tabcolsep}{1pt}
	\begin{center}
		\begin{tabular}{ c  c  | c  c  c c  }
			%\hline
			 \hspace*{-1.0cm}  
             \multirow{4}{*}[-0in]{ \rotatebox[origin=t]{0}{ {\vspace*{1in} \includegraphics[trim=0cm 0 0 -4cm,height=.43\linewidth]{figures/uvcoverage/uv_eht2017_2.pdf}}
			 		\qquad  }}
			 &  \hspace{-0.7cm}  \large{\textsf{Truth}}   & &  \large{\textsf{a = 2}} & \large{\textsf{a = 5}}  &  \large{\textsf{a = 10}}    \\
			&  \hspace{-0.5cm} {{\includegraphics[height=.13\linewidth]{figures/singleimage/visibilities/hotakaframe3.pdf}} } &
			\multirow{1}{*}[0.82in]{ \rotatebox[origin=t]{90}{ \small{\textsf{Gauss. Recon.}} }}
			&
			{\includegraphics[height=.13\linewidth]{figures/singleimage/visibilities/img_powerdropoff_2.pdf}} &
			{\includegraphics[height=.13\linewidth]{figures/singleimage/visibilities/img_powerdropoff_5.pdf}} 
			&
			{\includegraphics[height=.13\linewidth]{figures/singleimage/visibilities/img_powerdropoff_10.pdf}} 
			\\
			& \vspace{-.0in}  \hspace{-0.8cm} \large{\textsf{MEM \& TV}}  & &  \multicolumn{3}{c}{ \includegraphics[width=.25\linewidth]{figures/cbar/horizontal_cbar_-2to4_r2.pdf} }
			\\
			%& \vspace{-.15in}  \hspace{-0.8cm} \large{\textsf{MEM \& TV}}  & & & &       \\
			&\hspace{-0.5cm} {{\includegraphics[height=.13\linewidth]{figures/singleimage/visibilities/img_maxen.pdf}} } &
			\multirow{1}{*}[0.82in]{ \rotatebox[origin=t]{90}{ \small{\textsf{Clipped Recon.}} }}
			&
			\includegraphics[height=.13\linewidth]{figures/singleimage/visibilities/imgClip_powerdropoff_2.pdf} &
			\includegraphics[height=.13\linewidth]{figures/singleimage/visibilities/imgClip_powerdropoff_5.pdf} &
			\includegraphics[height=.13\linewidth]{figures/singleimage/visibilities/imgClip_powerdropoff_10.pdf} 
			\\
			& \vspace{-.0in} \hspace{-.8cm} \large{\textsf{CHIRP}}  & &  \multicolumn{3}{c}{ \includegraphics[width=.25\linewidth]{figures/cbar/horizontal_cbar_0to4_r2.pdf} }
			\\ 
			%& \vspace{-.15in} \hspace{-.8cm} \large{\textsf{CHIRP}}   & & & &     \\
			&\hspace{-0.5cm} {{\includegraphics[width=.13\linewidth,trim=0.0cm -1.5cm 0.0cm 0.0cm]{figures/singleimage/visibilities/hotakaframe_chirp.pdf}} \vspace{0.04cm} } &
			\multirow{1}{*}[1.05in]{ \rotatebox[origin=t]{90}{ \small{\textsf{Diagonal Std. Dev. }} }}
			&
			\hspace{-.1in} \includegraphics[width=.138\linewidth]{figures/singleimage/visibilities/covImg_powerdropoff_2_r2.pdf} &
			\includegraphics[width=.148\linewidth]{figures/singleimage/visibilities/covImg_powerdropoff_5_r2.pdf} 
			&\hspace{-.1in}
			\includegraphics[height=.145\linewidth]{figures/singleimage/visibilities/covImg_powerdropoff_10_r2.pdf} 
		\end{tabular}
		\caption{{\bf Static Imaging Comparison:} Results of static imaging using a multivariate Gaussian prior ( \textsf{a} = 2, 5, 10) compared to state-of-the-art reconstruction methods using MEM \& TV regularizers~\cite{andrew} as well as patch-based regularizers (CHIRP)~\cite{bouman2016computational}. All images are shown with a field of view of 160 $\mu$-arcseconds. Data is generated using a static image with the uv-coverage of the EHT2017 array shown on the left (see Section~\ref{sec:results}). The uv-coverage is colored by time, as indicated by the colorbar in Figure~\ref{fig:uvcov2}. Note however that in this static imaging case the time of measurements is not relevant. Although the previous algorithms (MEM \& TV and CHIRP) both produce better results, the Gaussian reconstruction is able to correctly get the broad structure of the underlying image. Since we do not impose positivity, negative values are reconstructed. However, by clipping the resulting image we can see that the result aligns well with the true static image. The Gaussian prior model also allows us to easily estimate our reconstructed image uncertainty. We visualize the diagonal entries of the posterior covariance matrix as the reshaped standard deviation image. Note that as the smoothness parameter \textsf{a} is increased, the per-pixel standard deviation becomes smaller, but the structure of the standard deviation deviates from what was specified in the prior (recall $\bLambda$ is scaled by $\bmu$, which we have specified as a 2D Gaussian in this work). For large \textsf{a} the uncertainty is shown to be primarily in the diagonal north-west to south-east direction, due to the lack of spatial frequencies sampled by the telescope array in this direction. To avoid approximations and best show the recovered posterior covariance matrices, atmospheric error has not been included in the data used to recover these images. The scaling of the colormaps is in mili-Jansky per squared $\mu$-arcsecond. } 
		\label{fig:staticimaging}
		%-.2, .4
		%0, .4
	\end{center}
	\vspace{-.2in}
\end{figure*}



\vspace{-.2in}
\subsection{Multivariate Gaussian Image Prior}
\label{sec:gauss_prior}

A prior distribution on $\im$ constrains the space of possible solutions during inference, and can be defined in a variety of ways.
%There are many ways that a prior distribution on $\im$ can be defined. 
For instance, maximum entropy, sparsity, and patch priors have been all used previously for VLBI imaging~\cite{andrew,kazu,bouman2016computational, rusenimaging}.
%Priors that have been used previously in VLBI imaging include maximum entropy, sparsity, and patch priors. 
In this work we instead choose to define the underlying image, $\im$, as being a sample from the distribution $\mathcal{N}_{\im}(\bmu, \bLambda)$. This choice leads to less sharp image reconstructions compared to richer priors, %is less powerful 
%expressive
%than some of the other image priors in reconstructing a sharp image, 
%but its simple expression 
%comes with the advantage of allowing 
%allows for a better theoretical understanding of our solutions, which proves especially valuable when propagating uncertainties in dynamical imaging (refer to Section~\ref{sec:dynamic_inference}). 
but its simplicity allows for a cleaner understanding of our solutions. This proves especially valuable in propagating uncertainties during dynamic imaging (refer to Section~\ref{sec:dynamic_inference}). 
%is less expressive
%allows for a better theoretical understanding of our solutions that BLAH BLAH BLAH.  

%Image regularizers that enforce spatial smoothness can often be described with a multivariate Gaussian image prior. 
%For instance, the common squared total variation regularizer can be expressed by writing the image covariance, $\bLambda$, in terms of  the $ 2 \npix^4 \times \npix^4$ gradient matrix, $\bm{G}$: $\bLambda \propto \left[ \bm{G}^T \bm{G} \right]^{-1}$.

Studies have shown that the average power spectrum of an image often falls with the inverse of spatial frequency in the form $1/(u^2 + v^2)^{a/2}$, where $a$ is a value that specifies the smoothness of the image~\cite{torralba2003statistics}. 
%$1/f^a$. %(Burton and Moorhead 1987, Field 1987, 1994, Tolhurst et al 1992) http://web.mit.edu/torralba/www/ne3302.pdf
As the amplitude of a spatial frequency is linearly related to the image itself, this statistical property can also be enforced by specifying the covariance in a prior distribution. Specifically, 
%\begin{align}
%\bLambda'  =  &  \bm{W}^{*T}  \mbox{diag} \left[ \frac{1}{ %(\bm{u}^2 + \bm{v}^2)^{a/2} } \right]  \bm{W} 
%\end{align}
\begin{align}
& \hspace{.5in} \bLambda'  =    \bm{W}^{*T}  \mbox{diag} \left[ \bm{b}  \right]  \bm{W}  \\ 
b[i] = & \begin{cases} 
({u[i]}^2 + {v[i]}^2)^{-a/2}  & {u[i]}^2 + {v[i]}^2 > 0 \\
\epsilon & {u[i]}^2 + {v[i]}^2  = 0 \\
\end{cases}
\end{align}
\noindent{for DFT matrix $\bm{W}$ of size $M^2 \times M^2$ %and vector $\bm{b}$ of size $M^2$ 
for an $M\times M$ pixel image and a small positive value, $\epsilon$. Each row of $\bm{W}$ and $\bm{b}$ %is comprised of $\vecFTmtx(u,v)$ (see Section BLAH), and 
	corresponds to a $(u,v)$ coordinate in the 2D grid of frequencies, \{ $S \times S$ \}, for }
\begin{align}
%UV &= S \times S = \{ (u,v) | u \in S, v \in S \} \\
S &= \left\{ \frac{m-M/2}{FOV} \right\}, m\in \mathbb{Z}: m \in [0,M-1],
\end{align}
where $FOV$ is the image's field of view in radians.
To specify the variance of each pixel and help encourage positivity, we modify the amplitude of the covariance by left and right multiplying by ${c \cdot \mbox{diag}[\bmu ]} $:
\begin{align}
\bLambda = c^2 \mbox{diag}[\bmu ]^T \bLambda' \hspace{0.01in} \mbox{diag}[\bmu ] 
\end{align}
\noindent{A $c$ value of 1/3 implies that 99\% of flux values sampled from $\mathcal{N}_x(\bmu, \bLambda)$ will be positive. In this work we have chosen $\mu$ to be a circular Gaussian with a standard deviation of 40-50\% of the reconstructed FOV to encourage most of the flux to stay near the center of the image and away from the edges. Figure~\ref{fig:priorsamples} shows the covariance matrix constructed for $a=2,3,4$ along with images sampled from the prior  $\mathcal{N}_x(\bmu, \bLambda)$. Notice that as $a$ increases, the sampled images are smoother. Thus, $a$ provides the ability to tune the desired smoothness of the inferred images. 
}

%THIS DOESNT BELONG HERE!!!
%In Section BLAH we explain how this image prior can be introduced into the estimation of each $x_t$. Introducing the prior into the estimate of each image helps to further constrain the images in this very ill-posed setting. 


\subsection{Inference}
\label{sec:static_inf}

%As a Gaussian distribution is a conjugate prior for a Gaussian likelihood, the posterior distribution 

Our goal is to find the most likely image, $\im$, that describes the data products we have observed, $\meas$. A maximum a posteriori (MAP) solution is found by maximizing the log-posterior from Equation~\ref{eq:bayes}:
\begin{align}
\hat{\im}  = \argmax_{\im} & \log p(\im|\meas) \\
\notag =  \argmin_{\im}  &\left[ (f(\im)-\meas)^T \bR^{-1} (f(\im)-\meas) \right. \\
& \left. + (\im-\bmu)^T \bLambda^{-1} (\im-\bmu) \right] .
\end{align}

Note the similarities of this equation's structure to that of previous static imaging methods in Equation~\ref{eq:setup} and~\ref{eq:chi2}. 
Although the hyperparameter $\gamma$ is no longer explicit, the scaling of $\bLambda$ acts like this hyperparameter and balances influence of the measured data with influence of the prior.
%used in many VLBI imaging frameworks.

%\subsection{Data Likelihood - change from likelihood}


\vspace{0.1in}
\subsubsection{Linear Measurements}


As explained in Section~\ref{sec:dataproducts}, $f(\im)$ is linear when $\meas$ is composed solely of calibrated complex visibilities with no %remaining
atmospheric error.
In this case -- when $f(\im) = \FTmtx \im$ --
%with corresponding Gaussian noise of variance, $\bm{\sigma}^2$  (i.e. no atmospheric error), 
a closed-form solution of $\hat{\im}$ can be found %. 
%In the case that $f(\im) = \FTmtx \im$ is a linear function of $\im$,a closed-form solution of $\hat{x}$ can be found. 
%In this situation, $\im$ can be solved 
through traditional Weiner filtering. Specifically, we can compute the most likely estimate of each $\im$ as: 
% Equation~\ref{eq:map}:
\begin{align}
\hat{\im} &=  \bmu  + \bLambda {\bf F}^T ( \bR + \FTmtx \bLambda \FTmtx^{T} )^{-1} (  \meas -  \FTmtx \bmu ) .
\label{eq:map}
\end{align}
\noindent{
	%Refer to the supplemental material for details of this derivation. 
	In the limit of having no prior information about the underlying image $\im$, e.g.,  $\bLambda = \lim_{\lambda \to\infty} \lambda \mathds{1}$ for identity matrix $\mathds{1}$, this MAP solution reduces to $\hat{\im} =  \FTmtx^{+} \meas$. In other words, in the absence of prior image assumptions, the noise on each measurement, $\bR$, is no longer relevant and the reconstructed image is simply obtained by inverting $\meas = \FTmtx \im$.
	This is very similar to reconstructing the ``dirty image''~\cite{taylor1999synthesis}.
	
%When $\meas$ are sparse visibilities, although $\FTmtx^{-1}$ is undefined, since $\FTmtx \FTmtx^{*T} = \mathds{1}$, $\hat{\im} =  \FTmtx^{-1} \meas$ is very similar to reconstructing the ``dirty image''~\cite{taylor1999synthesis}.

%evaluating, inspecting

The same solution can also be obtained by evaluating %and inspecting 
the posterior distribution. 
With a linear measurement function, $f(\im)$, the proposed Gaussian formulation leads to a closed-form expression for the posterior. In particular, 
\begin{align}
p(\im|\meas) & = \mathcal{N}_{\im} (\hat{\im}, \bm{C} ). 
\end{align}
for covariance matrix
%\noindent{The mean of this posterior distribution is equivalent to the MAP estimate obtained through Weiner Filtering.}
%Our proposed Gaussian formulation not only facilitates easily computing the MAP estimate, but also the uncertainty in the estimated $\hat{\im}$:
\begin{align}
\bm{C} = \bLambda - \bLambda \FTmtx^T ( \bR + \FTmtx \bLambda \FTmtx^T )^{-1} \FTmtx \bLambda .
\end{align}
\noindent Estimating uncertainty with the covariance matrix is useful in understanding what regions of the reconstructed image we trust. This becomes especially helpful when propagating information in dynamical imaging, as will be demonstrated in Figure~\ref{fig:propinfo}. 




\vspace{0.1in}
\subsubsection{Non-linear Measurements}
%When the data products in $\meas$ are invariant to atmospheric noise, $f(\im)$ is a non-linear function of $\im$ and a closed-form solution does not exist. 
When $f(\im)$ is a non-linear function of $\im$, as is the case when the data products in $\meas$ are invariant to atmospheric noise, a closed-form solution does not exist. 
%of atmospheric noise, large phase errors are added to the complex visibility measurements. However, these phase errors are incorporated in a way that preserves closure phase (refer to Section BLAH). In order to handle this additional phase error, without having to explicitly model the errors as latent variables, we can change the measurement function, $f(.)$, to one that is invariant to atmospheric inhomogeneity.
%A measurement function, $f(.)$, which extracts the bispectrum, visibility amplitude, or closure phase would be invariant to this atmospheric inhomogeneity. However, it comes at the expense of being a non-linear function of the image, $x$. 
However, to solve for the optimal $\im$ we linearize $f(\im)$ to obtain an approximate solution, $\hat{\im}$. Using a first order Taylor series expansion approximation around $\tilde{\im}$, we approximate the data likelihood as
\begin{align}
p(\meas|\im) = \mathcal{N}_{\meas}(f(\im),\bR) \approx \mathcal{N}_{\meas} \left( f( \tilde{\im} ) +  \dot{\FTmtx} (\im - \tilde{\im} )  , \bR \right) , 
\end{align}
\noindent{ for $\dot{\FTmtx} = \left. \frac{df(\im)}{d \im} \right| _{\tilde{\im} }$. Using this approximation, % we solve for 
	the optimal $\hat{\im}$ is}
\begin{align}
\hat{\im} &=  \bmu  + \bLambda \dot{\FTmtx}^T ( \bR + \dot{\FTmtx} \bLambda \dot{\FTmtx}^{T} )^{-1} (  y  - f(\tilde{\im}) +  \dot{\FTmtx} (\tilde{\im}-\bmu) ) .
\label{eq:approxoptimal}
\end{align}

To further improve the solution, we solve Equation~\ref{eq:approxoptimal} iteratively by updating  $\hat{\im}$ and setting $\tilde{\im} = \hat{\im} $ until convergence.
%By iteratively solving for Equation~\ref{eq:approxoptimal} and setting $\tilde{\im} = \hat{\im} $, we can improve our estimated image reconstruction, $\hat{\im}$. 
Note that in the case that $f(\im)$ is linear, $\dot{\FTmtx} = \FTmtx$ and Equation~\ref{eq:approxoptimal} reduces to Equation~\ref{eq:map}. We compare results of this reconstruction method to other state-of-the-art methods for a static source in Figure~\ref{fig:staticimaging}. Figure~\ref{fig:staticimaging} demonstrates that, although this approach does not outperform other state-of-the-art static imaging methods, reasonable results are achieved despite a simpler image regularizer and optimization procedure. This simpler approach will become useful in developing a dynamic imaging approach. 


\section{Dynamic Model}
\label{sec:dynamic_model}

Earth rotation synthesis inherently assumes that the source being imaged is static over the course of an observation~\cite{taylor1999synthesis}. 
%Generally, this makes it possible to collect multiple measurements that inform us about the same underlying image. 
If this assumption holds, it is possible to collect more than $\ntele (\ntele-1)/2$ measurements that inform us about the underlying source through earth rotation synthesis
However, in the case of an evolving source, as is predicted to be the case for SgrA*, this assumption is violated -- measurements taken at different times throughout the observation correspond to different underlying source images. 

At each time $t=1,...,\ntime$ we measure a vector of data products $\meas_t$, that are observed
%generated 
from an evolving source image, $\im_t$. Our goal is to reconstruct the $\ntime$ instantaneous images $ {\bf X} = \{\im_1, ..., \im_\ntime \}$ using the set of sparse observations $ {\bf Y} = \{\meas_1, ..., \meas_\ntime \}$. We define a dynamic imaging %generative (?)
model for this observed data as potentials ($\varphi$) of an undirected tree graph (see Figure~\ref{fig:model}):
\begin{align}
\varphi_{\meas_t | \im_t} &=  \mathcal{N}_{\meas_t} ( f_t ( \im_{t} ) , \bR_t ), \\
\varphi_{\im_t} &=  \mathcal{N}_{\im_1} ( \bmu_t, \bLambda_t ), \\
\varphi_{\im_t|\im_{t-1}} &=  \mathcal{N}_{\im_t} ( \evolve \im_{t-1}, \bQ ), 
\label{eq:evolution_potential}
\end{align}
for $\bLambda_t = \mathrm{diag}[\bmu_t]^T \bLambda'\mathrm{diag}[\bmu_t]$. 


Similar to the static imaging model, each set of observed data $\meas_t$ taken at time $t$ is related to the underlying instantaneous source image, $\im_t$, through the functional relationship, $f_t(\im_t)$, and $\im_t$ is encouraged to be a sample from a multivariate Gaussian distribution. However, new to this dynamic imaging model is the addition of (\ref{eq:evolution_potential}) that describes how images evolve over time. 
%The final set of terms involving the evolution matrix, $A$, define how the underlying image evolves over time. 
If we assume that there is no evolution between neighboring images in time ($\evolve=\mathds{1}, \bQ = \bm{0}$), this dynamic model reduces to that of static imaging. 
Using the Hammersley-Clifford Theorem~\cite{hammersley1971markov}, the joint distribution of this dynamic model can be written as a product of its potential functions: 
\begin{align}
p({\bf X}, {\bf Y} ; \evolve )  \propto \prod_{t=1}^{\ntime} \varphi_{\meas_t |\im_t} \prod_{t=1}^{N} \varphi_{\im_t}   \prod_{t=2}^{\ntime} \varphi_{\im_t|\im_{t-1}} .
\label{eq:likelihood}
\end{align}

% \begin{align}
%  & p({\bf X}, {\bf Y} | \evolve ) =  p(\im_1, ..., \im_N, \meas_1, ..., \meas_\ntime | \evolve )    \\
% \notag & \propto \prod_{t=1}^{\ntime} \mathcal{N}_{\im_1} ( \bmu_t, \bLambda_t ) \prod_{t=1}^{N}  \mathcal{N}_{\meas_t} ( \FTmtx_t \im_{t} , \bR_t ) \prod_{t=2}^{\ntime}  \mathcal{N}_{\im_t} ( \evolve \im_{t-1}, \bQ )
% \end{align}

% \begin{align}
%  & p({\bf X}, {\bf Y} | \evolve ) =  p(\im_1, ..., \im_N, \meas_1, ..., \meas_\ntime | \evolve )    \\
% \notag & = \mathcal{N}_{\im_1} ( \bmu, \bLambda ) \prod_{t=1}^{N}  \mathcal{N}_{\meas_t} ( \FTmtx_t \im_{t} , \bR_t ) \prod_{t=2}^{\ntime}  \mathcal{N}_{\im_t} ( \evolve \im_{t-1}, \bQ )
% \end{align}


% \begin{align}
%  p({\bf X}, {\bf Y} | \theta ) &= \mathcal{\ntime}_{\im_1} ( \bmu, \bLambda ) \prod_{t=1}^{\ntime}  \mathcal{N}_{\meas_t} ( \FTmtx_t \im_{t} , \bR_t )  \\
%  & s.t. \hspace{0.1in} \im_1 = \im_2 = ... = \im_\ntime
% \end{align}

% By vertically stacking each matrix $\FTmtx_t$ into a matrix $\FTmtx$ and placing $\bR_t$ along the diagonal of matrix $\bR$, we can calculate the most likely estimate of each $\im_t$ using Weiner filtering (refer to Equation BLAH).

% \begin{align}
% \hat{x}_t &=  \bmu  + \bLambda \FTmtx^T ( \bR + {\bf F} \bLambda \FTmtx^T )^{-1} (  y -  \FTmtx \bmu ) \hspace{.12in} \forall t \in 1,...,N
% \end{align}




%\subsection{Observation Model} 



%\paragraph{Linear Observations} As mentioned in Section BLAH, the visibilities measured by an interferometer are related to the underlying image via a Fourier transform. Thus, for simplicity, we first assume that we are able to measure complex visibilities with Gaussian noise.  relationship is linear and each measurement can be expressed as $y_t + \mbox{noise} = f_t(x_t) = F_t x_t$ for DTFT matrix $F_t$. 


%\paragraph{Non-linear Observations} However, as mentioned in Section BLAH the complex visibilities often have large atmospheric noise that make them   
%we often do not have access to complex vi
%For simplicity, we first present derivations assuming a linear observation model. In Section BLAH we specify how to handle non-linear relationships, such as must be used in the case of constraining phase-closure.


\begin{figure}
	\centering
	\includegraphics[width=0.8\linewidth]{figures/graphicalmodel_2.pdf}
\caption{{\bf Graphical Representation of our Dynamic Imaging Model}: At each time $t$ we observe a vector of data products $\meas_t$ corresponding to the instantaneous source image $\im_t$. 
	%To solve for the set of latent images $\bm{X} = \{ \im_t\}_{t=1}^N$ 
	We assume each image $\im_t$ is related to its adjacent neighbors in time, $\im_{t-1}$ and $\im_{t+1}$, and is also related to a multivariate Gaussian distribution specified by mean $\bmu_t$ and covariance $\bLambda$. The persistent global evolution of the source images over time is specified by $\evolve$, which is further parameterized by $\theta$. Additional intensity perturbations in time are constrained by the covariance matrix $\bQ$. In this diagram, squares indicate parameters, circles are variables, and shaded circles indicate the variable is observed. }
\label{fig:model}
\vspace{-.2in}
\end{figure}









\vspace{-.2in}
\subsection{Evolution Model}
\label{sec:evolution}

Each image $\im_t$ is related to the previous image $\im_{t-1}$ through a linear relationship: $\im_t \approx \evolve \im_{t-1}$.  Matrix $\evolve$ (size $\npix^2 \times \npix^2$) %and $B$ (size $M^2 \times 1$)
defines the evolution of the source's emission region between time steps.
 For instance, $\evolve=\mathds{1}$ indicates that, on average, the source image does not change, whereas $\evolve=2 \mathds{1}$ indicates that the image's brightness doubles at each time step. Since the evolution matrix $\evolve$ is not time dependent, the underlying source image evolves similarly over the entire observation. However, at each time the image can deviate slightly from this persistent evolution. 
 %The amount of allowed deviation 
 The amount of allowed intensity deviation is expressed in the time-invariant covariance matrix $\bQ$.  
%We refer to this property as persistent flow.

We assume that the evolution of the emission region over time is primarily described by small perturbations on top of a persistent 2D projected flow of material that preserves total flux.
We treat each source image like a 2D array of light pulses originating at locations $( \bm{\xpos}, \bm{\ypos})$. These pulses can shift around, causing motion in the image. %resemble
As described by the Shift Theorem, the shift of a pulse by $\Delta$ will change the phase of its Fourier Transform by $2 \pi f \Delta$ for each frequency $f$. % can be related to its phase in the frequency domain.
%As described by the Shift Theorem, the position of a pulse can be related to its phase in the frequency domain
Thus, under small motions, we can write $\evolve$ in terms of the image's full $M^2 \times M^2$ DFT matrix $\bm{W}$ (see Section~\ref{sec:gauss_prior}), and a column-vector of $M$ pixel shifts: $\bm{s} = (\bm{s}_{ \bm{\xpos}}, \bm{s}_{\bm{\ypos}})$:
%As the position of a pulse can be related to its phase in the frequency domain, under small motions we write $\evolve$ in terms of the image's full DTFT matrix $\bm{W}$, and a column-vector of pixel shifts: $\bm{s} = (\bm{s}_{ \bm{\xpos}}, \bm{s}_{\bm{\ypos}})$. \katie{check if there should be a negative in the exponent}
\begin{align}
\evolve = \Re \left[ \bm{W}^{*T} \hspace{0.01in}  \left(  \exp \left[ -i 2 \pi ( \bm{u} \bm{s}_{\xpos}^T + \bm{v} \bm{s}_{\ypos}^T ) \right]  \odot  \bm{W}  \right) \right] . 
\end{align}
Applying $\evolve$ to a vectorized image $\im_t$ of light pulses results in a new image, $\im_{t+1}$, where the pulses have been shifted according to $\bm{s}$ and re-interpolated on the 2D DFT grid. Note that in the case $\bm{s} = \bm{0}$ then $\evolve = \bm{W}^{*T} \bm{W} = \mathds{1}$.

%Note that in the case $\bm{s} = \bm{0}$ then $\evolve = \bm{W}^{*T} \bm{W} = \mathds{1}$, and that the same 2D DFT grid locations  image $\im_{t+1}$

%Note that in the case $\bm{s} = \bm{0}$ then $\evolve = \bm{W}^{*T} \bm{W} = \mathds{1}$, and that this $\evolve$ matrix results in new light pulses being defined at the same 2D DFT grid locations after the old pulses have been shifted. 

%\vspace{0.1in}
%\subsubsection{Low-Dimensional Motion Basis}

The above parameterization of evolution matrix $\evolve$ in terms of $\bm{s}$ allows for independent, arbitrary shifts of each pulse of light, resulting in $2 \npix^2$ shift parameters. However, as neighboring material generally moves together, the pixel shifts should have a much lower intrinsic dimensionality. 
%BLAH BLAH BLAH. 
%Additionally, during inference solving for the $2 M^2$ parameters of $\bm{s}$. 
%However, solving this problem is very ill-posed because the number of unknowns, $s$, (number of pixels) far exceeds the number of measurements, $\vis$, for a single time step. 
To address this, and simultaneously reduce the number of free parameters, we instead describe motion $\evolve$ using a low-dimensional subspace, parameterized by $\theta$. The length of $\theta$, $D$, is much smaller than the number of unconstrained shift parameters, $2 \npix^2$. %: $\evolve(\theta)$. 
We define a motion basis $\mathcal{M} = \Spvek{\mathcal{M}_{\xpos}, \mathcal{M}_\ypos}^T$ of size $2\npix^2 \times D+1$, and restrict the motion at every time step to be a linear function of this motion:
\begin{align}
\Spvek{\bm{\xpos}_{t+1}; \bm{\ypos}_{t+1}} = \Spvek{\bm{\xpos}_t; \bm{\ypos}_t} + \Spvek{\bm{s}_{\xpos_t}; \bm{s}_{\ypos_t}} =  \Spvek{\mathcal{M}_{\xpos_t}; \mathcal{M}_{\ypos_t}} \Spvek{1; \theta} .
\end{align}
%where $\bm{\xpos}_t$ and $\bm{\ypos}_t$ indicate the two-dimensional position of each pulse at time $t$. Note that the $\evolve$ matrix defined above results in new light pulses being defined at the same 2D DFT grid locations after the old pulses have been warped. 

This parameterization allows us to describe a wide variety of motion (or warp) fields. 
Generic ``smooth" warp fields can be described by using a truncated Discrete Cosine Transform (DCT) basis as $\mathcal{M}$.
However, more compressed motion bases can also be used~\cite{lowdim14, erikmiller}. 
In this work, results are shown using an affine transformation parametrized with a four-dimensional $\theta$ that captures rotation, shear, and scaling. As an affine transformation, $\theta$, acting on a pulse at location $(\xpos_t, \ypos_t)$ results in moving the pulse to location
%fit into this framework. 
    \begin{align}
    \Spvek{{\xpos}_{t+1}; {\ypos}_{t+1}} = \Spvek{\theta_1 \hspace{.1in}\theta_2 ; 
    	\theta_3 \hspace{.1in}\theta_4  } \Spvek{{\xpos}_{t}; {\ypos}_{t}} = \Spvek{ \xpos_t \hspace{.1in} \ypos_t \hspace{.1in}0 \hspace{.1in} 0 ; 
    	0 \hspace{.1in}0 \hspace{.1in} \xpos_t \hspace{.1in} \ypos_t } \Spvek{\theta_1; \theta_2; \theta_3; \theta_4},
    \end{align}
 in this work we define, 
   \begin{align}
  \Spvek{\mathcal{M}_{\xpos_t}; \mathcal{M}_{\ypos_t}}  =  \Spvek{\bm{0} \hspace{.1in} \bm{\xpos}_t \hspace{.1in} \bm{\ypos}_t \hspace{.1in} \bm{0} \hspace{.1in} \bm{0} ; \bm{0} \hspace{.1in}
   	\bm{0} \hspace{.1in} \bm{0} \hspace{.1in} \bm{\xpos}_t \hspace{.1in} \bm{\ypos}_t } .
   \end{align}
For example, using this motion basis with $\theta_1 = \cos \phi$, $\theta_2 = \sin \phi$, $\theta_3 = -\sin \phi$, and $\theta_4 = \cos \phi$ would specify that every time step the image is rotated by $\phi$ radians. 


%This parameterization allows us to describe a wide variety of motion (or warp) fields. For instance, affine transformation can be parametrized with a six-dimensional $\theta$ that captures rotation, shear, translation, and scaling.
%fit into this framework. 
%More general warp fields can also be described by using a truncated Discrete Cosine Transform (DCT) basis as $\mathcal{M}$. 


% \paragraph{Affine Motion Basis}

% An affine transformation is able to capture rotation, shear, translation, and scaling. It can be written in terms of a six dimensional vector $\theta$, where $\theta = \Spvek{1, 0, 0, 0, 1, 0}^T$ would be equivalent to no transformation. 

%   \begin{align}
%   \Spvek{\xpos_{t+1}; \ypos_{t+1}} = \Spvek{\xpos_t \hspace{.1in}\ypos_t \hspace{.1in} 1 \hspace{.1in}0 \hspace{.1in} 0 \hspace{.1in} 0; 
%   	0 \hspace{.1in}0 \hspace{.1in} 0 \hspace{.1in} \xpos_t \hspace{.1in} \ypos_t \hspace{.1in} 1} \theta
%   \end{align}

% \paragraph{Smooth Motion Basis} $\bm{D'} = (\FTmtx[ \sqrt{\bm{u}^2 + \bm{v}^2} <\tau,:])^T$


%    \begin{align}
%   \Spvek{\bm{\xpos}_{t+1}; \bm{\ypos}_{t+1}} = \Spvek{ \Re (\bm{D'}) \Im( \bm{D'} ); \Re (\bm{D'}) \Im( \bm{D'} ) } \theta
%   \end{align}



\section{Dynamic Imaging Inference \& Learning}
\label{sec:dynamic_inference}


%The dynamical observation model presented in Section BLAH can be thought of as an undirected graphical model. Refer to Figure BLAH. 


We solve for the best set of $\ntime$ images $ \bm{X} $ constrained by the $\ntime$ vectors of sparse observations $\bm{Y}$. In general, we assume that $f_t(.)$, $\bR_t$, $\bmu_t$, $\bLambda_t$, $\bQ$ are known/specified model parameters. However, $\evolve$, which defines how the source evolves, is not necessarily known ahead of time. If there is reason to believe that only small perturbations exist in the source image over time, then a reasonable assumption is to set $\evolve = \mathds{1}$. However, in the case of large persistent motion this may fail to give informative results. 

We begin in Section~\ref{sec:dynamic_inference_known} by discussing how to solve for $\bm{X}$ when $\evolve$ is known. In this case, the model contains no unspecified parameters and the goal is to simply solve for the latent images. 
In Section~\ref{sec:dynamic_inference_unknown}, we forgo this assumption and no longer assume that $\evolve$ is known. In this case, we jointly solve for $\evolve$ and $\bm{X}$ by first learning $\evolve$'s parameters $\theta$ using an Expectation-Maximization (EM) algorithm before solving for the latent images, $\bm{X}$. We refer to our proposed method as StarWarps. 
%{\bf MDJ: We might want to change the language from "the model contains no unknown parameters" to "the model contains no unspecified parameters". Similar to Kazu's point, we should emphasize what parameters the user must specify in an ad hoc fashion.}













\subsection{Known Evolution}
\label{sec:dynamic_inference_known}

Given all of the model parameters and observed data, our goal is to estimate the optimal set of latent images, $\bm{X}$. 
In static imaging we set up an optimization problem that allowed us to easily solve for the most likely latent image, $\im$, given the observed data, $\meas$. 
In the proposed dynamic model, a similar closed-form solution exists in the case of a linear $f(\im)$ and diagonal $\bR_t$ and $\bQ$ matrices~\cite{fessler}. However, this requires us to invert a large $\npix^4 \times \ntime^2$ non-block-diagonal matrix.   
Thus, instead of the MAP estimate, we compute the most likely instantaneous image at each time, $t$, given all of the observed data $\bm{Y}$.
In particular, we estimate the marginal distribution of each $\im_t$, $p(\im_t | \bm{Y})$, by integrating out the other latent images in time, and set $\hat{\im}_t$ equal to the mean of each distribution. 

%Given all of the model parameters and observed data, our goal is to estimate the optimal set of latent images, $\bm{X}$. 
%In static imaging we set up an optimization problem that allowed us to easily solve for the most likely latent image, $\im$, given the observed data, $\meas$. 
%However, in the proposed dynamic model, solving for a MAP estimate -- the set of all images that maximizes Equation~\ref{eq:likelihood} -- is much more difficult, as a closed-form solution does not exist. 
%Instead of the MAP estimate, we compute the most likely instantaneous image at each time, $t$, given all of the observed data $\bm{Y}$.
%In particular, we estimate the marginal distribution of each $\im_t$, $p(\im_t | \bm{Y})$, by integrating out the other latent images in time, and set $\hat{\im}_t$ equal to the mean (and mode) of each distribution. 


%Instead, we take the approach of solving for the marginal distribution of each $\im_t$ and setting $\hat{\im}_t$ equal to the mean (and mode) of each distribution. Although this does not necessarily produce the MAP estimate under our model, it provides the most likely instantaneous image at each time, $t$, given all of the observed data $\bm{Y}$.


Since we have defined our dynamic model in terms of Gaussian distributions, we can efficiently solve for $p(\im_t | \meas_1,...,\meas_\ntime)$ by marginalizing out the latent images $ \{ \im_1,...,\im_{t-1},\im_{t+1},...,\im_{\ntime} \}$ using the Elimination Algorithm~\cite{graphicalmodels}. Specifically, we derive a function proportional to the marginal distributions. This function is evaluated using a two-pass algorithm, which consists of a forward pass and a backward pass. Each pass, outlined in Algorithms~\ref{alg:forward} and~\ref{alg:backward}, propagates information using recursive updates that compute distributions proportional to 
%$p(\im_t | \meas_1,...,\meas_{t-1})$ and $p(\im_t | \meas_{t},...,\meas_{\ntime})$ 
$p(\im_t, \meas_1,...,\meas_{t-1})$ and $p( \meas_{t},...,\meas_{\ntime} | \im_t )$ 
for each $\im_t$ in the forward and backward pass, respectively. By combining these terms we obtain
\begin{align}
\label{eq:marg1}
p(\im_t|\bm{Y}) & = \mathcal{N}_{x_t}(\hat{\im}_t, \bm{C}_t ) \\
\notag & \propto \mathcal{N}_{x_t}({\bf z}^{\alpha}_{t|t-1}, {\bf P}^{\alpha}_{t|t-1})  \mathcal{N}_{x_{t}}( {\bf z}^{\beta}_{t|t} , {\bf P}^{\beta}_{t|t} ),
\end{align}
which, as shown in the supplemental material, has mean $\hat{\im}_t$ and covariance $\bf{C}_t$:
{\footnotesize
	\begin{align}
	\notag    \hat{\im}_t & = {\bf P}^{\beta}_{t|t} ( {\bf P}^{\alpha}_{t|t-1} + {\bf P}^{\beta}_{t|t})^{-1} {\bf z}^{\alpha}_{t|t-1}  +   {\bf P}^{\alpha}_{t|t-1}( {\bf P}^{\alpha}_{t|t-1} + {\bf P}^{\beta}_{t|t})^{-1} {\bf z}^{\beta}_{t|t} \\
	\bf{C}_t & =  {\bf P}^{\alpha}_{t|t-1}( {\bf P}^{\alpha}_{t|t-1} + {\bf P}^{\beta}_{t|t})^{-1} {\bf P}^{\beta}_{t|t} , 
	\label{eq:marg2}
	\end{align}
}where ${\bf z}_{t|\tau}^{\alpha}$, ${\bf P}_{t|\tau}^{\alpha}$ are the estimates of the mean and covariance of $\im_t$ using observations at time steps $1$ through $\tau$. Similarly, ${\bf z}_{t|\tau}^{\beta}$, ${\bf P}_{t|\tau}^{\beta}$ are the estimates of the mean and covariance of $\im_t$ using observations $\tau$ through $\ntimes$.



For generality, we have listed the forward and backward algorithms in terms of non-linear measurement functions, $f_t(\im_t)$ with derivative $\dot{\FTmtx}$. In this case, similar to our static model inference in Section~\ref{sec:static_inf}, we linearize the solution around $\tilde{\im}_t$ to get an approximate estimate. To improve the solution of the forward and backward terms, each step in the forward pass can be iteratively re-solved, updating $\tilde{\im}_t$ at each iteration. The values of $\tilde{\im}_t$ are then fixed for the backwards pass. Recall %from Section BLAH 
that when $f_t(\im)$ is linear in $\im$ then $f_t(\im) = \FTmtx_t \im  = \dot{\FTmtx}_t \im$, and the $\hat{\im}$ will converge to the optimal solution in a single update.

The above inference algorithm is similar to %Kalman filtering and smoothing for Linear Dynamical Systems. 
the forward-backward algorithm used for Gaussian Hidden Markov Models~\cite{graphicalmodels}. 
In fact, removing the $\varphi_{\im_t}$ term for $t>1$ in Equation~\ref{eq:likelihood} yields the familiar form of a Gaussian Hidden Markov Model. In this case, inference reduces to the traditional Kalman filtering and smoothing (extended Kalman filtering in the case of non-linear $f_t(\im)$)~\cite{anderson1979optimal}. Although this simpler formulation can sometimes produce acceptable results, in our typical scenario of especially sparse or noisy data keeping the additional potential terms helps to further constrain the problem, and results in better reconstructions. 
%For derivations in the case of a Hidden Markov Model, see the supplemental material. 

\RestyleAlgo{boxruled}
\begin{algorithm}[t]
	\caption{Forward Updates {\footnotesize $t = 1 \rightarrow 2 \rightarrow ... \rightarrow \ntime$ } \label{alg:forward} }
	
	{\bf Predict:}
	
	{ \footnotesize
		\begin{align}
		\notag {\bf z}^{\alpha}_{t|t-1} &= \evolve {\bf z}^{\alpha}_{t-1|t-1} \\
		\notag {\bf P}^{\alpha}_{t|t-1} &= \bQ  + \evolve {\bf P}^{\alpha}_{t-1|t-1} \evolve^T
		\end{align}
		\begin{align}
		\notag {\bf z}^{\alpha*}_{t|t-1} &= \bLambda_t ( \bLambda_t + {\bf P}^{\alpha}_{t|t-1} )^{-1} {\bf z}^{\alpha}_{t|t-1} + {\bf P}^{\alpha}_{t|t-1} ( \bLambda_t + {\bf P}^{\alpha}_{t|t-1} )^{-1} \bmu_t \\
		\notag {\bf P}^{\alpha*}_{t|t-1} & = \bLambda_t ( \bLambda_t + {\bf P}^{\alpha}_{t|t-1} )^{-1} {\bf P}^{\alpha}_{t|t-1} 
		\end{align}
	}
	
	{\bf Update:}
	
	{\footnotesize
		\begin{align}
		\notag \meas_\Delta &= (  \meas_t + \dot{\FTmtx} \tilde{\im}_t - f(\tilde{\im}_t) -  \dot{\FTmtx} {\bf z}^{\alpha*}_{t|t-1} ) \\
		\notag {\bf z}^{\alpha}_{t|t} &  =  {\bf z}^{\alpha*}_{t|t-1}  + {\bf P}^{\alpha*}_{t|t-1} \dot{\FTmtx}_t^T ( \bR_t +  \dot{\FTmtx}_t {\bf P}^{\alpha*}_{t|t-1} \dot{\FTmtx}_t^T )^{-1} \meas_\Delta  \\
		\notag {\bf P}^{\alpha}_{t|t} & = {\bf P}^{\alpha*}_{t|t-1} - {\bf P}^{\alpha*}_{t|t-1}\dot{\FTmtx}_t^T ( \bR_t + \dot{\FTmtx}_t {\bf P}^{\alpha*}_{t|t-1} \dot{\FTmtx}_t^T )^{-1} \dot{\FTmtx}_t {\bf P}^{\alpha*}_{t|t-1} 
		\end{align}
	}
	
	
	{\bf Initialization:}
	
	{\footnotesize
		\begin{align}
		\notag {\bf z}^{\alpha*}_{1|0} = \bmu_1 \mbox{   ,   } {\bf P}^{\alpha}_{1,0}   = \bLambda_1
		\end{align}
	}
	
	
\end{algorithm}


\RestyleAlgo{boxruled}
\begin{algorithm}[b]
	\caption{Backward Updates: {\footnotesize $t = \ntime \rightarrow \ntime-1 \rightarrow ... \rightarrow 1 \hspace{-.1in}$ }  \label{alg:backward}}
	
	{\bf Predict:}
	{\footnotesize
		\begin{align}
		\notag {\bf z}^{\beta*}_{t|t+1}  & \hspace{-.02in} = \hspace{-.02in} \bmu_t  + \bLambda_t  \evolve^T ( \bQ + {\bf P}^{\beta}_{t+1|t+1} +  \evolve \bLambda_t  \evolve^T )^{-1} (  {\bf z}^{\beta}_{t+1|t+1} \hspace{-.13in} -   \evolve \bmu_t ) \\
		\notag  {\bf P}^{\beta*}_{t|t+1} &  \hspace{-.02in}= \hspace{-.02in} \bLambda_t - \bLambda_t \evolve^T (  \bQ + {\bf P}^{\beta}_{t+1|t+1} + \evolve \bLambda_t  \evolve^T )^{-1}  \evolve \bLambda_t 
		\end{align}
	}
	
	
	{\bf Update:}
	
	{\footnotesize
		\begin{align}
		\notag \meas_\Delta &= (  \meas_t + \dot{\FTmtx} \tilde{\im_t} - f(\tilde{\im}_t) -  \dot{\FTmtx}_t {\bf z}^{\beta*}_{t|t+1} ) \\
		\notag  {\bf z}^{\beta}_{t|t} &  =  {\bf z}^{\beta*}_{t|t+1}  + {\bf P}^{\beta*}_{t|t+1} \dot{\FTmtx}_t^T ( \bR_t +  \dot{\FTmtx}_t {\bf P}^{\beta*}_{t|t+1} \dot{\FTmtx}_t^T )^{-1} \meas_\Delta \\
		\notag  {\bf P}^{\beta}_{t|t} & = {\bf P}^{\beta*}_{t|t+1} - {\bf P}^{\beta*}_{t|t+1}\dot{\FTmtx}_t^T ( \bR_t + \dot{\FTmtx}_t {\bf P}^{\beta*}_{t|t+1} \dot{\FTmtx}_t^T )^{-1} \dot{\FTmtx}_t {\bf P}^{\beta*}_{t|t+1} 
		\end{align}
	}
	
	
	{\bf Initialization:}
	
	{\footnotesize
		\begin{align}
		\notag  {\bf z}^{\beta*}_{\ntime|\ntime+1} = \bmu_{\ntime} \mbox{   ,   } {\bf P}^{\beta*}_{\ntime|\ntime+1}   = \bLambda_{\ntime}
		\end{align}
	}
	
	
\end{algorithm}


\vspace{-.1in}

\subsection{Unknown Evolution}
\label{sec:dynamic_inference_unknown}

% \subsection{Inference With Known Evolution}

% In general, we do not know how an emission evolves over time. T
% In this model, we assume that $f_t, \bR_t, \bQ, \mu$, and $\bLambda$ are known, but matrix $A$ is unknown.

% \subsection{Inference Without Known Evolution}

If the evolution matrix $\evolve$ is unknown, it is not possible to simply solve for $\bm{X}$ in the way outlined in Section~\ref{sec:dynamic_inference_known}. 
%In theory, although it is possible to compute a gradient of Equation~\ref{eq:likelihood} with respect to each $\bm{X}$ and $\evolve$ and perform gradient ascent, this would be very computationally intensive and prone to local maxima. This becomes even more pronounced when solving for the evolution parameters $\theta$ (see Section~\ref{sec:evolution}) rather than $\evolve$ itself. 
Instead we choose to use an Expectation-Maximization (EM) algorithm to recover $\evolve$ (parameterized by $\theta$), and then subsequently use the procedure presented in Section~\ref{sec:dynamic_inference_known} to recover $\bm{X}$. 
%Instead we choose to jointly solve for latent images $\bm{X}$ and parameters $\evolve$ (parameterized by $\theta$) using the Expectation-Maximization (EM) algorithm. 
%The EM algorithm defines an iterative process that allows us to maximize the likelihood function of a parametric model in the case in which some variables of the model are "latent" or unknown. 

The EM algorithm defines an iterative process that
solves for 
%helps us to find 
the evolution parameters $\theta$ that maximize the complete likelihood in Equation~\ref{eq:likelihood} when all of the underlying images, $\bm{X}$, are unknown (latent). 
Each iteration of EM improves the log-likelihood of the data under the defined objective function and is especially useful when the likelihood is from an exponential family, as is the case in our proposed model. In particular the EM algorithm consists of the following two iterative steps:

\begin{itemize}
	\item Expectation step (E step): Calculate the expected value of the log likelihood function (see Equation~\ref{eq:likelihood}), with respect to the conditional distribution of $\bm{X}$ given $\bm{Y}$ under the current estimate of the $\theta$ parameters, $\theta^{(i)}$: %$Q(\theta|\theta^{(i)} = E_{X|Y, \theta^{(i}} [ \log L (\theta, Y, X) ]$.
	\begin{align}
	Q(\theta|\theta^{(i)}) = E_{ {\bf X}| {\bf Y}, \theta^{(i)}} [ \log p ({\bf X}, {\bf Y} | \theta) ]
	\label{eq:qfunction}
	\end{align}
	\item Maximization step (M step): Find the parameter that maximizes: %$\theta^{(i+1)} = \argmax_{\theta} Q(\theta | \theta^{(i)} )$.
	\begin{align}
	\theta^{(i+1)} = \argmax_{\theta} Q(\theta | \theta^{(i)} ) .
	\label{eq:argmaxem}
	\end{align}
\end{itemize}

We solve for $\theta$ using gradient ascent. As $\evolve$ is a function of $\theta$, we must %solve for the optimal $\theta$ we must first 
compute the derivative of $Q(\theta|\theta^{(i)})$ using the chain rule. We compute this derivative with respect to each element $j$ in $\theta$: 
\begin{align}
\frac{d}{d \theta[j] } Q(\theta|\theta^{(i)} ) &=  \sum_p \sum_q  \frac{d Q(\theta|\theta^{(i)} )  }{d \evolve[p,q]} \frac{d \evolve[p,q]}{d \theta[j] } 
\end{align}
Using the low-dimensional subspace evolution model proposed in Section~\ref{sec:evolution}, the derivative $\frac{d \evolve[p,q]}{d \theta[j] }$ can be computed as \katie{I took out taking just the real portion to make it nicer looking, should probably resolve that}
\begin{align}
\frac{d \evolve }{d \theta[j]} = -i 2 \pi \theta[j] \evolve  \left( \bm{u} \mathcal{M}_{\xpos}[:,j+1]^T  + \bm{v} \mathcal{M}_{\ypos}[:,j+1]^T  \right).
%\left[ \bm{W}^{*T} \hspace{0.05in} \exp \left[ -i 2 \pi \left( \bm{u} \left[ \mathcal{M}_{\xpos} \Spvek{1; \theta} \right]^T \bm{s}_{\xpos}^T + \bm{v} \left[ \mathcal{M}_{\ypos} \Spvek{1; \theta} \right]^T  \right) \right] \right]
\end{align}
%What remains to be calculated is the derivative of $Q(\theta|\theta^{(i)})$ with respect to $\evolve$. 
By expanding and taking the derivative of the log-likelihood from Equation~\ref{eq:likelihood} with respect to $\evolve$, we obtain the expression 
%which is a function of the sufficient statistics of $\im$. 
\begin{align}
& \hspace{-.1in}  \frac{d}{d \evolve} Q(\theta|\theta^{(i)}) %&=  \frac{d}{d \theta} E_{ \bm{X}| \bm{Y}, \theta} [ \log p ({\bf X}, {\bf Y} | \theta) ]\\
%&= \frac{d}{d \theta} E_{ \bm{X}| \bm{Y}, \theta} \left[ \frac{-1}{2} \sum_{t=2}^{N} \left[( \im_t - \evolve_{\theta} \im_{t-1} )^T Q^{-1} ( \im_t - \evolve_{\theta} \im_{t-1} ) \right] + \mathcal{G}(x_{1:N}, y_{1:N}) \right] \\
% \notag  & = \frac{d}{d \theta} \frac{1}{2} \sum_{t=2}^{N} \left[     E_{ \bm{X}| \bm{Y}, \theta^{(i)}} \left[ \im_t^T \bQ^{-1} \evolve_{\theta} \im_{t-1} \right] + E_{ \bm{X}| \bm{Y}, \theta^{(i)}} \left[ \im_{t-1}^T \evolve_{\theta}^T \bQ^{-1} \im_t \right] \right. \\
% \notag & \hspace{.5in}  \left. - E_{ \bm{X}| \bm{Y}, \theta^{(i)} } \left[ \im_{t-1}^T \evolve_{\theta}^T \bQ^{-1} \evolve_{\theta} \im_{t-1} \right]  \right] \\
%=   \frac{1}{2} \sum_{t=2}^{N} \left[    \bQ^{-1} E_{ \bm{X}| \bm{Y}, \theta^{(i)} }  \left[ \im_t \im_{t-1}^T \right] \right. \\
%\notag & \left. +  \bQ^{-1} E_{ \bm{X}| \bm{Y}, \theta^{(i)} } \left[ \im_{t-1} \im_t^T \right] - 2 \bQ^{-1} \evolve_{\theta}  E_{ \bm{X}| \bm{Y}, \theta^{(i)} } \left[ \im_{t-1}   \im_{t-1}^T\right]  \right] \\
=   -\frac{1}{2} \sum_{t=2}^{N} \left[ 2 \bQ^{-1} \evolve  E_{ \bm{X}| \bm{Y}, \theta^{(i)} } \left[ \im_{t-1}   \im_{t-1}^T\right]   \right. \\
\notag & \hspace{0.2in} \left. - \bQ^{-1} E_{ \bm{X}| \bm{Y}, \theta^{(i)} }  \left[ \im_t \im_{t-1}^T \right] -  \bQ^{-1} E_{ \bm{X}| \bm{Y}, \theta^{(i)} } \left[ \im_{t-1} \im_t^T \right]   \right].
\end{align}
By inspecting this expression we can see that the sufficient statistics we require to maximize the log-likelihood are the expected value of $\im_t \im_t^T$ and $\im_{t-1} \im_t^T$ under the distribution $p(\bm{X} | \bm{Y}; \theta^{(i)} ) $. Conveniently, these sufficient statistics can be computed using from the set of ${\bf z}$'s and $\bf{P}$'s computed in Section~\ref{sec:dynamic_inference_known}.  From the marginal distributions (Equations~\ref{eq:marg1} and~\ref{eq:marg2}) derived in Section~\ref{sec:dynamic_inference_known}, we obtain
\begin{align}
E_{ \bm{X}| \bm{Y}, \theta^{(i)} } \left[ \im_{t}   \im_{t}^T\right] = \hat{\im}_t  \hat{\im}_t ^T + {\bf C}_{t}.
\end{align}
The sufficient statistic $E_{ \bm{X}| \bm{Y}, \theta^{(i)} } \left[ \im_{t-1} \im_t^T \right]$ is a bit trickier to obtain, but can also be calculated using the same forward-backward terms, as shown in the supplemental material. Mathematically,	
\begin{align}
E_{ \bm{X}| \bm{Y}, \theta^{(i)} } \left[ \im_{t-1} \im_t^T \right] = \hat{\im}_{t-1} \hat{\im}_t^T  + \bm{\xi}_3 \bm{\xi}_1^{T -1}, 
\end{align}	
where $\bm{\xi}_1$ and $\bm{\xi}_3$ are defined according to
    \begin{align}
    & p(\im_t, \im_{t-1}|\bm{Y})  = \mathcal{N}_{\im_{t-1}}(\bm{\xi}_1\im_t + \xi_2, \bm{\xi}_3) \\
    \notag & \propto \mathcal{N}_{\im_{t-1}} ({\bf z}^{\alpha}_{t-1|t-1}, {\bf P}^{\alpha}_{t-1|t-1} ) \mathcal{N}_{\im_t} ( \evolve \im_{t-1}, \bQ )  \mathcal{N}_{\im_{t}} ({\bf z}^{\beta}_{t|t}, {\bf P}^{\beta}_{t|t} ).
    % \notag & \propto p(\im_{t-1}, \meas_1, ..., \meas_{t-1}) p( \im_t | \im_{t-1}) p(\meas_t, ..., \meas_\ntime | \im_t )
    \end{align}
	
	%Mathematically, 
	
	% p(x|y) =  p(y|x)p(x)/p(y) ... p(y) = int_x p(y|x)p(x) dx = int_x c N(m,S) = c int_x N(m,S) = c 
	
	%EM is especially useful when the likelihood is an exponential family: the E step becomes the sum of expectations of sufficient statistics, and the M step involves maximizing a linear function. In such a case, it is usually possible to derive closed-form expression updates for each step, using the Sundberg formula (published by Rolf Sundberg using unpublished results of Per Martin-Löf and Anders Martin-Löf).[3][4][7][8][9][10][11]
	
	
	
	%Given a set of observations and model parameters $\theta$, we would like to estimate the marginal distribution for each of the $N$ latent images. 
	
	% \begin{align}
	% E_{ \bm{X}| \bm{Y}, \theta^{(i)} } \left[ \im_{t}   \im_{t}^T\right] & = {\bf z}_t {\bf z}_t^T + {\bf P}_t = M_{t} \\
	% %E_{ \bm{X}| \bm{Y}, \theta^{(i)} } \left[ \im_{t-1}   \im_{t}^T\right] & = K_t \left[  m_{t-1}  {\bf z}^{\beta}_{t|t}^T (Q + {\bf P}^{\beta}_{t|t})^{-1 T} Q^T + [m_{t-1} m_{t-1}^T + C_{t-1}] A^T (Q + {\bf P}^{\beta}_{t|t})^{-1 T} {\bf P}^{\beta}_{t|t}^T \right]  %\overleftarrow{A}_{t-1}{\bf P}_{t}  + {\bf z}_{t-1} {\bf z}_{t}^T = M_{t,t-1}  
	% \end{align}
	
	
	To learn the parameters $\theta$, we iterate between computing sufficient statistics of $\bm{X}$ given the current estimate of parameters, $\theta^{(i)}$, % $\theta^{(i)}$, 
	and solving for new parameters that maximize the updated log-likelihood, $\theta^{(i+1)}$, under those statistics. 
	Once convergence has been reached, we return the parameters $\hat{\theta}$ and the optimal set of instantaneous images under that transformation $E[\bm{X}] =  \{ \hat{\im}_t \}_{t=1}^{\ntimes} $.
	
	%This process allows us to search for the $\theta$ parameters of $\evolve$ that maximize Equation~\ref{eq:likelihood}. 
	Note that since our EM method's maximization step requires solving a non-convex problem we likely will only find a local-maximum of $\theta$ at each step. Nonetheless, the log-likelihood is guaranteed to increase for a $\theta$ that increases $Q(\theta|\theta^{(i)})$ (Equation~\ref{eq:qfunction})~\cite{little2014statistical}. This class of algorithms, which do not necessarily find the optimal $\theta$ at each iteration, are more rigorously referred to as ``Generalized EM"~\cite{little2014statistical}. In the case of a linear $f(\im)$ this EM procedure is exact and the log-likelihood increases at every iteration. In the case of a non-linear $f(\im)$ the forward-backward algorithm in Section~\ref{sec:dynamic_inference_known} provides only an approximation of the true sufficient statistics. Nonetheless, we empirically find that, when we fix each latent image's linearization point, the log-likelihood consistently improves. 
	
	
	%Once the algorithm converges, we use the resulting $\hat{\theta}$ to compute $\evolve$ and infer the best set of instantaneous images, $\bm{X}$ using the two-pass method described in Section~\ref{sec:dynamic_inference_known}.
	
	%numerically stable 
	
	
	
	% Although the covariance matrix is simply an approximation we demonstrate later in Figure BLAH how important it is to not ignore the uncertainty when using a small number of samples. 
	
	%It seems unclear whether uncertainties that are propagated from a	(likely) rather idealized gaussian prior are very "valuable" in practice. 
	





\begin{figure*}[h!]
	\begin{center}
		\hspace*{-.4in}
		\vspace*{-0.3in}
		\begin{tabular}{   c | c  c  c  c  }
			%\hline
			& \large{\textsf{VIDEO 1 }} &\large{\textsf{VIDEO 2}}   &\large{\textsf{VIDEO 3}} &\large{\textsf{VIDEO 4}}      \\ 
			&\vspace{-.1in}&&&\\
			& \large{\textsf{Pure Rotation}} &\large{\textsf{Rotating Hotspot}}   &\large{\textsf{Face-on Disk}} &\large{\textsf{Edge-on Disk}}      \\ \hline
			&\vspace{-.1in}&&&\\
			\multirow{1}{*}[0.9in]{ \rotatebox[origin=t]{90}{\small{\textsf{SINGLE FRAME}} }}
			&
			{{\includegraphics[height=.15\linewidth]{figures/starwarps_results/rotation30/gt/frames/gt_15_colorbar.pdf}} } 
			\multirow{2}{*}[.9in]{ \includegraphics[width=0.06\linewidth]{figures/cbar/vertical_cbar_0to12.pdf} }
			&
			\includegraphics[height=.15\linewidth]{figures/starwarps_results/hotspot100sR2/gt/frames/gt_115_colorbar.pdf} 
			\multirow{2}{*}[.9in]{ \includegraphics[width=0.06\linewidth]{figures/cbar/vertical_cbar_0to16.pdf} }
			&
			\includegraphics[height=.15\linewidth]{figures/starwarps_results/hotakamovie_02/gt/frames/gt_85_colorbar.pdf} 
			\multirow{2}{*}[.9in]{ \includegraphics[width=0.0668\linewidth]{figures/cbar/vertical_cbar_0to4.pdf} }
			&
			\includegraphics[height=.15\linewidth]{figures/starwarps_results/hotakamovie_45/gt/frames/gt_63_colorbar.pdf} 
			\multirow{2}{*}[.9in]{ \includegraphics[width=0.06\linewidth]{figures/cbar/vertical_cbar_0to8.pdf} }
			\\
			\multirow{1}{*}[0.7in]{ \rotatebox[origin=t]{90}{\small{\textsf{MEAN}} }}
			&
			{{\includegraphics[height=.15\linewidth]{figures/starwarps_results/rotation30/gt/pavgimg_colorbar.pdf}} } \hspace{.55in} &
			\includegraphics[height=.15\linewidth]{figures/starwarps_results/hotspot100sR2/gt/pavgimg_colorbar.pdf} \hspace{.55in} &
			\includegraphics[height=.15\linewidth]{figures/starwarps_results/hotakamovie_02/gt/pavgimg_colorbar.pdf} \hspace{.55in} &
			\includegraphics[height=.15\linewidth]{figures/starwarps_results/hotakamovie_45/gt/pavgimg_colorbar.pdf} \hspace{.55in}
			\\ \hline
			&\vspace{-.1in}&&&\\
			\multirow{1}{*}[0.9in]{ \rotatebox[origin=t]{90}{\small{\textsf{STD. DEVIATION}} }}
			& \hspace{0.01in}
			{{\includegraphics[height=.147\linewidth]{figures/starwarps_results/rotation30/gt/stdevimg_noaxis_r2.pdf}} } \hspace{.59in} & 
			
			\hspace{0.01in}
			 \includegraphics[height=.147\linewidth]{figures/starwarps_results/hotspot100sR2/gt/stdevimg_noaxis_r2.pdf}  \hspace{.59in} &
			
			\hspace{-0.01in} \includegraphics[height=.147\linewidth]{figures/starwarps_results/hotakamovie_02/gt/stdevimg_noaxis_r2.pdf}  \hspace{.55in} & 
			\hspace{-0.01in}
			\includegraphics[height=.147\linewidth]{figures/starwarps_results/hotakamovie_45/gt/stdevimg_noaxis_r2.pdf} \hspace{.55in} 
			\\
		\end{tabular}
		\vspace{.1in}
		\caption{{\bf Ground truth videos:} The four ground truth sequences used to demonstrate results. We show a single frame from each sequence, the mean frame, and the spatial standard deviation of flux density. Video 1 consists of a $160 \mu$-arcsecond image~\cite{avery} that rotates $180^{\circ}$ over the course of a 12 hour observation (24 hour rotational period). Video 2 is a $120 \mu$-arcsecond view of an edge-on black hole disk with a rotating ``hot spot" predicted by~\cite{Broderick_Loeb_2006} with a rotational period of 2.78 hours. Video 3 and 4 are generated using a model of a black hole observed face on and at a $45^{\circ}$ inclination with a $160 \mu$-arcsecond field of view~\cite{Shiokawa_2013}. They assume a spin of 0.9375 with an Innermost Stable Circular Orbit (ISCO) rotational period of 8.96 minutes. The specified FOV and colormaps for the single frame and mean images are used for each corresponding video throughout the remainder of the paper. }
		%video 1: 1.2 max val
		%video 2: 1.6 maxval
		%video 3: .4 maxval
		%video 4: .8 maxval
		\label{fig:groundtruth}
	\end{center}
		\vspace{-.2in}
\end{figure*}

\vspace{-.07in}
\section{Dynamic Imaging Results}
\label{sec:results}

As data from the EHT 2017 campaign is yet to be released, in this section we demonstrate our method on synthetic EHT data and real data from the Very Long Baseline Array (VLBA). Additional results can be seen in the supplemental document and video. The StarWarps algorithm has been implemented as part of the publicly available python \texttt{eht-imaging}\footnote{\url{https://github.com/achael/eht-imaging}} library~\cite{andrew}.

\vspace{-.2in}
\subsection{Synthetic Data Generation} 

We demonstrate our algorithm on synthetic data generated from four different sequences of time-varying sources. These sequences include two realistic fluid simulations of a black hole accretion disk for different observing orientations~\cite{Shiokawa_2013}, a realistic sequence of a ``hot spot" rotating around a black hole~\cite{Broderick_Loeb_2006}, and a toy sequence evolving with pure rotation. The field of view of each sequence ranges from 120 to 160 $\mu$-arcseconds. A still frame from each sequence is shown in Figure~\ref{fig:groundtruth}. To help give a sense of the variation in each sequence, the figure also displays the mean and standard deviation of flux density. We refer to these sequences by their video number, indicated in the figure. 



In order to demonstrate the quality of results under various observing conditions, VLBI observations of SgrA* at 1.3 mm (230 GHz) are simulated assuming
 three different telescope arrays.  The first array, EHT2017, consists of the 8 telescopes at 6 distinct locations that were used to collect measurements for the Event Horizon Telescope in the spring of 2017. The uv-coverage for this array can be seen in Figure~\ref{fig:staticimaging}. The second array, EHT2017+, augments the EHT2017 array with 3 potential additions to the EHT: Plateau de Bure (PDB), Haystack (HAY), and Kitt Peak (KP) Observatory. 
Details on telescopes used in the EHT2017 and EHT2017+ array are shown in Table~\ref{tab:telescopes}.  
The third array, FUTURE, consists of 9 additional telescopes. The uv-coverage of these latter two arrays, along with a colorbar indicating the time of each measurement, is shown in Figures~\ref{fig:uvcov2}. 
%Details on telescopes used in these arrays are shown in the supplemental material.


Visibility measurements are generated using the python \texttt{eht-imaging} library~\cite{andrew}. %assuming a 4 GHz bandwidth with a 100 second integration time. 
Realistic thermal noise, resulting from a bandwidth ($\Delta \nu$) of 4 GHz and a 100 second integration time ($t_{\rm int}$), is introduced on each visibility. The standard deviation of thermal noise is given by
\begin{align}
\sigma = \frac{1}{0.88} \sqrt{\frac{\mbox{SEFD}_1 \times \mbox{SEFD}_2 }{2 \times \Delta \nu \times t_{\rm int} } },
\label{eq:thermal}
\end{align}

\noindent{for System Equivalent Flux Density (SEFD) of the two telescopes corresponding to each visibility\footnote{The factor of 1/0.88 is due to information loss due to recording 2-bit quantized data-streams at each telescope~\cite{TMS}.}~\cite{taylor1999synthesis}.  %, and using bandwidth $\Delta \nu$ and integration time $t_{int}$.}
% in the absense of atmospheric error
%Atmospheric phase error is also introduced into measurements using the \texttt{eht-imaging} library. 
Random station-based atmospheric phases drawn from a uniform distribution at each time step are introduced into measurements using the \texttt{eht-imaging} library. 
In Videos 2-4 a set of measurements is sampled every 5 minutes over a roughly 14 hour duration, resulting in 173 time steps. In Video 1 only 30 time steps are measured over a 12 hour duration. 
}


\begin{figure}[h!]
	\centering
	%{\includegraphics[height=.28\linewidth]{figures/uvcoverage/uv_eht2017.pdf}}
	{\includegraphics[height=.38\linewidth]{figures/uvcoverage/uv_ehtfuture2_2.pdf}}
	{\includegraphics[height=.38\linewidth]{figures/uvcoverage/uv_ehtfuture1_2.pdf}}
	\caption{{\bf Time-varying uv-coverage:} The uv-coverage for EHT2017+ and FUTURE arrays when observing SgrA∗. The uv-coverage for the EHT2017 array can be seen in Figure~\ref{fig:staticimaging}. 
		Baselines are colored by the time of each observation relative to the start time, indicated by the colorbar to the right.}
	\label{fig:uvcov2}

\end{figure}


\begin{table}[!t]
	\begin{center}
		\caption{EHT 2017 Station Parameters}
		\label{tab::eht_station}
		\begin{tabular}{ccc}
 Telescope & Location & SEFD (Jy) \\ \hline
 ALMA & Chile & 110 \\ 
 APEX & Chile & 22000 \\ 
 LMT & Mexico & 560 \\
 SMT & Arizona & 11900 \\ 
 SMA & Hawaii & 4900 \\
 JCMT & Hawaii & 4700 \\ 
 PV & Spain & 2900 \\ 
 SPT & South Pole & 1600 \\ \hline
 PDB* & France & 1600 \\ 
 HAY* & Massachusetts & 2500 \\ 
 KP* & Arizona & 2500 \\ \hline
		\end{tabular}\\
	\end{center}
	\bigskip
	\footnotesize{The location and SEFD of each telescope in the EHT2017 and EHT2017+ arrays. These parameters and locations were used to generate the uv-trajectories in frequency space shown in Figure~\ref{fig:staticimaging} and~\ref{fig:uvcov2}. Telescope names followed by a star (*) were not included in the EHT2017 array.}
	\label{tab:telescopes}
	\vspace{-0.3in}
\end{table}



%\begin{table}[h!]
%\begin{center}
%\begin{tabular}{ c c c }
% Telescope & Location & SEFD (Jy) \\ \hline
% ALMA & Chile & 110 \\ 
% APEX & Chile & 22000 \\ 
% LMT & Mexico & 560 \\
% SMT & Arizona & 11900 \\ 
%SMA & Hawaii & 4900 \\
%JCMT & Hawaii & 4700 \\ 
% PV & Spain & 2900 \\ 
% SPT & South Pole & 1600 \\ \hline
% PDB* & France & 1600 \\ 
% HAY* & Massachusetts & 2500 \\ 
%  KP* & Arizona & 2500 \\ \hline
%\end{tabular}
%\end{center}
%\caption{ The location and SEFD of each telescope in the EHT2017 and EHT2017+ arrays. These parameters and locations were used to generate the uv-trajectories in frequency space shown in Figure~\ref{fig:staticimaging} and~\ref{fig:uvcov2}. Telescope names followed by a star (*) were not included in the EHT2017 array. }
%\label{tab:telescopes}
%\vspace{-0.3in}
%\end{table}

\vspace{-.1in}
\subsection{Static Evolution Model (No Warp)}
\label{sec:nomotionresults}

We first demonstrate results of our method under a static evolution model. In this case, we fix parameters $\theta$ such that $A=\mathds{1}$. This assumes that there is no global motion under a persistent warp field, but only perturbations around a fairly static scene. Despite this incorrect assumption (especially in Videos 1 and 2), this simple model results in reconstructions that surpass the state-of-the-art methods, and recovers distinctive structures that appear in the underlying source images. 


\subsubsection{Synthetic Data Result Comparison}

\begin{figure*}
	\begin{center}
	%	\vspace{-0.5in}
		%\hspace*{-2.3cm}
		\setlength{\tabcolsep}{3pt}
		
		\hspace{-0.5in}\normalsize{\textsf{BLURRED TRUE MEAN}}  \hspace{5.5cm} \normalsize{\textsf{NORMALIZED RMSE}} 
		\vspace{0.1in}
		
		
%	\begin{tabular}{ c " c}
%		VIDEO 1& VIDEO 2 \\ \thickhline
%		& \vspace{-.1in} \\
%		VIDEO 3 & VIDEO 4
%	\end{tabular}
%	\qquad
%		\begin{tabular}{ c " c}
%		{{\includegraphics[height=.1\linewidth]{figures/starwarps_results/rotation30/gt/frames/gt_15_blurredbeam75_noaxis.pdf}} } & {{\includegraphics[height=.1\linewidth]{figures/starwarps_results/hotspot100sR2/gt/frames/gt_115_blurredbeam75_noaxis.pdf}} } \\ \thickhline
%		& \vspace{-.1in} \\
%	{{\includegraphics[height=.1\linewidth]{figures/starwarps_results/hotakamovie_02/gt/frames/gt_85_blurredbeam75_noaxis.pdf}} } & {{\includegraphics[height=.1\linewidth]{figures/starwarps_results/hotakamovie_45/gt/frames/gt_63_blurredbeam75_noaxis.pdf}} }
%		\end{tabular}
						\begin{tabular}{ c " c}
							\hspace{-.06in} \textsf{Video 1} & \hspace{-.06in} \textsf{Video 2} \\
							{{\includegraphics[height=.1\linewidth]{figures/starwarps_results/rotation30/gt/pavgImg_blurredbeam75_noaxis.pdf}} } & {{\includegraphics[height=.1\linewidth]{figures/starwarps_results/hotspot100sR2/gt/pavgImg_blurredbeam75_noaxis.pdf}} } \\ \thickhline
							& \vspace{-.1in} \\
							\hspace{-.06in} \textsf{Video 3} & \hspace{-.06in} \textsf{Video 4} \\
							{{\includegraphics[height=.1\linewidth]{figures/starwarps_results/hotakamovie_02/gt/pavgImg_blurredbeam75_noaxis.pdf}} } & {{\includegraphics[height=.1\linewidth]{figures/starwarps_results/hotakamovie_45/gt/pavgImg_blurredbeam75_noaxis.pdf}} }
						\end{tabular}	
	%	\qquad	
		\qquad
		\begin{tabular}{ r | c | c | c | c " c | c | c | c}
			& \rotatebox[origin=t]{90}{\small{\textsf{EHT 2017}} } \rotatebox[origin=t]{90}{\small{\textsf{NO ATM.}} }  & \rotatebox[origin=t]{90}{\small{\textsf{EHT 2017}} } \rotatebox[origin=t]{90}{\small{\textsf{ATM.}} } & \rotatebox[origin=t]{90}{\small{\textsf{EHT 2017+}} } \rotatebox[origin=t]{90}{\small{\textsf{ATM.}} } &\rotatebox[origin=t]{90}{\small{\textsf{FUTURE}} } \rotatebox[origin=t]{90}{\small{\textsf{ATM.}} } & \rotatebox[origin=t]{90}{\small{\textsf{EHT 2017}} } \rotatebox[origin=t]{90}{\small{\textsf{NO ATM.}} }  & \rotatebox[origin=t]{90}{\small{\textsf{EHT 2017}} } \rotatebox[origin=t]{90}{\small{\textsf{ATM.}} } & \rotatebox[origin=t]{90}{\small{\textsf{EHT 2017+}} } \rotatebox[origin=t]{90}{\small{\textsf{ATM.}} } &\rotatebox[origin=t]{90}{\small{\textsf{FUTURE}} } \rotatebox[origin=t]{90}{\small{\textsf{ATM.}} }  \\ \hline
			{\small{\textsf{StarWarps Mean} } } & {\bf {0.67} }& {\bf {0.65}} & {\bf {0.55} }&  {\bf {0.55}}& {\bf {0.74}} & {\bf {0.73} }& {\bf {0.73}} &  0.71 \\ 
				\cite{freek} & 1.05  & 0.82 & 0.73  & 0.68 &  0.98 & 1.21 & 0.99 & 0.35 \\ 
				\cite{andrew} & 0.79 & 0.80 & 0.73 & 0.63 & 0.83& 1.05 & 0.81 & {\bf 0.23}  \\  \thickhline
				\small{\textsf{StarWarps Mean} } & {\bf 0.32} & {\bf 0.55} & 0.67 & {\bf 0.36} & {\bf 0.34} & {\bf 0.44} & {\bf 0.23} & {\bf 0.31} \\
				\cite{freek} & 0.93 & 0.90 & {\bf 0.53} & 0.51 & 0.84 & 0.75 & 1.02 & 0.61 \\
				\cite{andrew} &0.84 & 0.85 & 0.71 &  0.39  & 0.60 & 0.50 & 0.60 & 0.47
			\end{tabular}

		\vspace{0.4in}
		
		\hspace*{-1.3cm}
		\begin{tabular}{  c c | c  c  c  c "  c  c  c  c  }
			%\hline
			& \small{\textsf{Array:}} &\small{\textsf{EHT 2017}}   &\small{\textsf{EHT 2017 }} &\small{\textsf{EHT2017+}}    &\small{\textsf{FUTURE}}    &\small{\textsf{EHT 2017}}   &\small{\textsf{EHT 2017 }} &\small{\textsf{EHT2017+}}    &\small{\textsf{FUTURE}}     \\ 
			&\vspace{-.1in} & & & & & & & &\\
			& \small{\textsf{Error:}} &\small{\textsf{NO ATM.}}   &\small{\textsf{ATM.}} &\small{\textsf{ATM.}}    &\small{\textsf{ATM.}}  &\small{\textsf{NO ATM.}}   &\small{\textsf{ATM.}} &\small{\textsf{ATM.}}    &\small{\textsf{ATM.}}   \\ \hline
%			&\vspace{-.1in} & & & & & & & &\\
%			\multirow{2}{*}[0.2in]{ \rotatebox[origin=t]{90}{\small{\textsf{StarWarps}} }}   \hspace{-0.3in} & \multirow{1}{*}[0.5in]{ \rotatebox[origin=t]{90}{\small{\textsf{FRAME}} }}
%			&
%			{{\includegraphics[height=.1\linewidth]{figures/starwarps_results/rotation30/eht2017_100_visibility/nomotion/frames/mean_noaxis_15.pdf}} } &
%			\includegraphics[height=.1\linewidth]{figures/starwarps_results/rotation30/eht2017_100_amp-bispectrum/nomotion/frames/mean_noaxis_15.pdf} &
%			\includegraphics[height=.1\linewidth]{figures/starwarps_results/rotation30/ehtfuture2_100_amp-bispectrum/nomotion/frames/mean_noaxis_15.pdf} &
%			\includegraphics[height=.1\linewidth]{figures/starwarps_results/rotation30/ehtfuture1_100_amp-bispectrum/nomotion/frames/mean_noaxis_15.pdf}  
%			&
%			{{\includegraphics[height=.1\linewidth]{figures/starwarps_results/hotspot100sR2/eht2017_100_visibility/nomotion/frames/mean_noaxis_115.pdf}} } &
%			\includegraphics[height=.1\linewidth]{figures/starwarps_results/hotspot100sR2/eht2017_100_amp-bispectrum/nomotion/frames/mean_noaxis_115.pdf} &
%			\includegraphics[height=.1\linewidth]{figures/starwarps_results/hotspot100sR2/ehtfuture2_100_amp-bispectrum/nomotion/frames/mean_noaxis_115.pdf} &
%			\includegraphics[height=.1\linewidth]{figures/starwarps_results/hotspot100sR2/ehtfuture1_100_amp-bispectrum/nomotion/frames/mean_noaxis_115.pdf} \\
			&\vspace{-.1in} & & & & & & & &\\
			 \multirow{2}{*}[0.6in]{ \rotatebox[origin=t]{90}{\small{\textsf{StarWarps}} }}   \hspace{-0.3in} &	\multirow{1}{*}[0.45in]{ \rotatebox[origin=t]{90}{\small{\textsf{Mean}} }}
			&
			{{\includegraphics[height=.1\linewidth]{figures/starwarps_results/rotation30/eht2017_100_visibility/nomotion/pavgimg_noaxis.pdf}} } &
			\includegraphics[height=.1\linewidth]{figures/starwarps_results/rotation30/eht2017_100_amp-bispectrum/nomotion/pavgimg_noaxis.pdf} &
			\includegraphics[height=.1\linewidth]{figures/starwarps_results/rotation30/ehtfuture2_100_amp-bispectrum/nomotion/pavgimg_noaxis.pdf} &
			\includegraphics[height=.1\linewidth]{figures/starwarps_results/rotation30/ehtfuture1_100_amp-bispectrum/nomotion/pavgimg_noaxis.pdf} 
			&
			{{\includegraphics[height=.1\linewidth]{figures/starwarps_results/hotspot100sR2/eht2017_100_visibility/nomotion/pavgimg_noaxis.pdf}} } &
			\includegraphics[height=.1\linewidth]{figures/starwarps_results/hotspot100sR2/eht2017_100_amp-bispectrum/nomotion/pavgimg_noaxis.pdf} &
			\includegraphics[height=.1\linewidth]{figures/starwarps_results/hotspot100sR2/ehtfuture2_100_amp-bispectrum/nomotion/pavgimg_noaxis.pdf} &
			\includegraphics[height=.1\linewidth]{figures/starwarps_results/hotspot100sR2/ehtfuture1_100_amp-bispectrum/nomotion/pavgimg_noaxis.pdf} 
			\\   \hline
			&\vspace{-.1in} & & & & & & & &\\
			\multirow{2}{*}[0.6in]{ \rotatebox[origin=t]{90}{\small{\textsf{MEM \& TV Regularization}} }}  \hspace{-0.3in} & \multirow{1}{*}[0.4in]{ \rotatebox[origin=t]{90}{\small{\textsf{\cite{freek}}} }}
			&
			{{\includegraphics[height=.1\linewidth]{figures/freeksmoothingresults/im_vis_rotation30_tint100_eht2017_directim_maxit100_it0.pdf}} } &
			\includegraphics[height=.1\linewidth]{figures/freeksmoothingresults/im_apar_rotation30_tint100_eht2017_directim_maxit100_it0.pdf} &
			\includegraphics[height=.1\linewidth]{figures/freeksmoothingresults/im_apar_rotation30_tint100_ehtfuture2_directim_maxit100_it0.pdf} &
			\includegraphics[height=.1\linewidth]{figures/freeksmoothingresults/im_apar_rotation30_tint100_ehtfuture1_directim_maxit100_it0.pdf} 
			&
			{{\includegraphics[height=.1\linewidth]{figures/freeksmoothingresults/im_vis_hotspot100sR2_tint100_eht2017_directim_maxit100_it0.pdf}} } &
			\includegraphics[height=.1\linewidth]{figures/freeksmoothingresults/im_apar_hotspot100sR2_tint100_eht2017_directim_maxit100_it0.pdf} &
			\includegraphics[height=.1\linewidth]{figures/freeksmoothingresults/im_apar_hotspot100sR2_tint100_ehtfuture2_directim_maxit100_it0.pdf} &
			\includegraphics[height=.1\linewidth]{figures/freeksmoothingresults/im_apar_hotspot100sR2_tint100_ehtfuture1_directim_maxit100_it0.pdf} 
			\\
			&\vspace{-.1in} & & & & & & & &\\
			&	\multirow{1}{*}[0.4in]{ \rotatebox[origin=t]{90}{\small{\textsf{\cite{andrew}}} }}
			&
			{{\includegraphics[height=.1\linewidth]{figures/starwarps_results/rotation30/eht2017_100_compare/none_vis_blur025.pdf}} } &
			\includegraphics[height=.1\linewidth]{figures/starwarps_results/rotation30/eht2017_100_compare/none_amp-clphase_blur025.pdf} &
			\includegraphics[height=.1\linewidth]{figures/starwarps_results/rotation30/ehtfuture2_100_compare/none_amp-clphase_blur025.pdf} &
			\includegraphics[height=.1\linewidth]{figures/starwarps_results/rotation30/ehtfuture1_100_compare/none_amp-clphase_blur025.pdf} 
			&
			{{\includegraphics[height=.1\linewidth]{figures/starwarps_results/hotspot100sR2/eht2017_100_compare/none_vis_blur025.pdf}} } &
			\includegraphics[height=.1\linewidth]{figures/starwarps_results/hotspot100sR2/eht2017_100_compare/none_amp-clphase_blur025.pdf} &
			\includegraphics[height=.1\linewidth]{figures/starwarps_results/hotspot100sR2/ehtfuture2_100_compare/none_amp-clphase_blur025.pdf} &
			\includegraphics[height=.1\linewidth]{figures/starwarps_results/hotspot100sR2/ehtfuture1_100_compare/none_amp-clphase_blur025.pdf} 
			\\   \thickhline
			
%			&\vspace{-.1in} & & & & & & & &\\
%			\multirow{2}{*}[0.2in]{ \rotatebox[origin=t]{90}{\small{\textsf{StarWarps}} }}  \hspace{-0.3in} & \multirow{1}{*}[0.5in]{ \rotatebox[origin=t]{90}{\small{\textsf{FRAME}} }}
%			&
%			{{\includegraphics[height=.1\linewidth]{figures/starwarps_results/hotakamovie_02/eht2017_100_visibility/nomotion/frames/mean_noaxis_85.pdf}} } &
%			\includegraphics[height=.1\linewidth]{figures/starwarps_results/hotakamovie_02/eht2017_100_amp-bispectrum/nomotion/frames/mean_noaxis_85.pdf} 
%			%\includegraphics[height=.1\linewidth]{figures/starwarps_results/starwarps_hotaka02_2/nomotion/mean_85.png} 
%			&
%			\includegraphics[height=.1\linewidth]{figures/starwarps_results/hotakamovie_02/ehtfuture2_100_amp-bispectrum/nomotion/frames/mean_noaxis_85.pdf} &
%			\includegraphics[height=.1\linewidth]{figures/starwarps_results/hotakamovie_02/ehtfuture1_100_amp-bispectrum/nomotion/frames/mean_noaxis_85.pdf} 
%			&
%			{{\includegraphics[height=.1\linewidth]{figures/starwarps_results/hotakamovie_45/eht2017_100_visibility/nomotion/frames/mean_noaxis_63.pdf}} } &
%			\includegraphics[height=.1\linewidth]{figures/starwarps_results/hotakamovie_45/eht2017_100_amp-bispectrum/nomotion/frames/mean_noaxis_63.pdf} &
%			\includegraphics[height=.1\linewidth]{figures/starwarps_results/hotakamovie_45/ehtfuture2_100_amp-bispectrum/nomotion/frames/mean_noaxis_63.pdf} &
%			\includegraphics[height=.1\linewidth]{figures/starwarps_results/hotakamovie_45/ehtfuture1_100_amp-bispectrum/nomotion/frames/mean_noaxis_63.pdf} \\
			&\vspace{-.1in} & & & & & & & &\\
			 \multirow{2}{*}[0.6in]{ \rotatebox[origin=t]{90}{\small{\textsf{StarWarps}} }}   \hspace{-0.3in} &	\multirow{1}{*}[0.45in]{ \rotatebox[origin=t]{90}{\small{\textsf{Mean}} }}
			&
			{{\includegraphics[height=.1\linewidth]{figures/starwarps_results/hotakamovie_02/eht2017_100_visibility/nomotion/pavgimg_noaxis.pdf}} } 
			&
			\includegraphics[height=.1\linewidth]{figures/starwarps_results/hotakamovie_02/eht2017_100_amp-bispectrum/nomotion/pavgimg_noaxis.pdf} 
			%\includegraphics[height=.1\linewidth]{figures/starwarps_results/starwarps_hotaka02_2/avgImg.pdf} 
			&
			\includegraphics[height=.1\linewidth]{figures/starwarps_results/hotakamovie_02/ehtfuture2_100_amp-bispectrum/nomotion/pavgimg_noaxis.pdf} &
			\includegraphics[height=.1\linewidth]{figures/starwarps_results/hotakamovie_02/ehtfuture1_100_amp-bispectrum/nomotion/pavgimg_noaxis.pdf} 
			&
			{{\includegraphics[height=.1\linewidth]{figures/starwarps_results/hotakamovie_45/eht2017_100_visibility/nomotion/pavgimg_noaxis.pdf}} } &
			\includegraphics[height=.1\linewidth]{figures/starwarps_results/hotakamovie_45/eht2017_100_amp-bispectrum/nomotion/pavgimg_noaxis.pdf} &
			\includegraphics[height=.1\linewidth]{figures/starwarps_results/hotakamovie_45/ehtfuture2_100_amp-bispectrum/nomotion/pavgimg_noaxis.pdf} &
			\includegraphics[height=.1\linewidth]{figures/starwarps_results/hotakamovie_45/ehtfuture1_100_amp-bispectrum/nomotion/pavgimg_noaxis.pdf} 
			\\   \hline
			&\vspace{-.1in} & & & & & & & &\\
			\multirow{2}{*}[0.6in]{ \rotatebox[origin=t]{90}{\small{\textsf{MEM \& TV Regularization}} }}  \hspace{-0.3in} & 	\multirow{1}{*}[0.4in]{ \rotatebox[origin=t]{90}{\small{\textsf{\cite{freek}}} }}
			&
			{{\includegraphics[height=.1\linewidth]{figures/freeksmoothingresults/im_vis_hotakamovie_02_16300_20999_tint100_eht2017_directim_maxit100_it0.pdf}} } &
			\includegraphics[height=.1\linewidth]{figures/freeksmoothingresults/im_apar_hotakamovie_02_16300_20999_tint100_eht2017_directim_maxit100_it0.pdf} &
			\includegraphics[height=.1\linewidth]{figures/freeksmoothingresults/im_apar_hotakamovie_02_16300_20999_tint100_ehtfuture2_directim_maxit100_it0.pdf} &
			\includegraphics[height=.1\linewidth]{figures/freeksmoothingresults/im_apar_hotakamovie_02_16300_20999_tint100_ehtfuture1_directim_maxit100_it0.pdf} 
			&
			{{\includegraphics[height=.1\linewidth]{figures/freeksmoothingresults/im_vis_hotakamovie_45_16300_20999_tint100_eht2017_directim_maxit100_it0.pdf}} } &
			\includegraphics[height=.1\linewidth]{figures/freeksmoothingresults/im_apar_hotakamovie_45_16300_20999_tint100_eht2017_directim_maxit100_it0.pdf} &
			\includegraphics[height=.1\linewidth]{figures/freeksmoothingresults/im_apar_hotakamovie_45_16300_20999_tint100_ehtfuture2_directim_maxit100_it0.pdf} &
			\includegraphics[height=.1\linewidth]{figures/freeksmoothingresults/im_apar_hotakamovie_45_16300_20999_tint100_ehtfuture1_directim_maxit100_it0.pdf} 
			\\
			&\vspace{-.1in} & & & & & & & &\\
			&	\multirow{1}{*}[0.4in]{ \rotatebox[origin=t]{90}{\small{\textsf{\cite{andrew}}} }}
			&
			{{\includegraphics[height=.1\linewidth]{figures/starwarps_results/hotakamovie_02/eht2017_100_compare/none_vis_blur025.pdf}} } &
			\includegraphics[height=.1\linewidth]{figures/starwarps_results/hotakamovie_02/eht2017_100_compare/none_amp-clphase_blur025.pdf} 
			&
			\includegraphics[height=.1\linewidth]{figures/starwarps_results/hotakamovie_02/ehtfuture2_100_compare/none_amp-clphase_blur025.pdf} &
			\includegraphics[height=.1\linewidth]{figures/starwarps_results/hotakamovie_02/ehtfuture1_100_compare/none_amp-clphase_blur025.pdf} 
			&
			{{\includegraphics[height=.1\linewidth]{figures/starwarps_results/hotakamovie_45/eht2017_100_compare/none_vis_blur025.pdf}} } &
			\includegraphics[height=.1\linewidth]{figures/starwarps_results/hotakamovie_45/eht2017_100_compare/none_amp-clphase_blur025.pdf} &
			\includegraphics[height=.1\linewidth]{figures/starwarps_results/hotakamovie_45/ehtfuture2_100_compare/none_amp-clphase_blur025.pdf} &
			\includegraphics[height=.1\linewidth]{figures/starwarps_results/hotakamovie_45/ehtfuture1_100_compare/none_amp-clphase_blur025.pdf} 
			\\ 
		\end{tabular}
		\caption{{\bf Static evolution model:} Results obtained using data simulated from each of the 4 video sequences (see Figure~\ref{fig:groundtruth}) under different telescope arrays (see Figures~\ref{fig:staticimaging} and~\ref{fig:uvcov2}) and noise conditions. The main portion of the figure is broken up into 4 quadrants corresponding to Videos 1-4 when moving from left to right, top to bottom. The true mean image from the ground truth videos, blurred to 3/4 the nominal resolution of the array, is shown on the top. We compare results of our proposed method, StarWarps, to that of the single imaging methods presented in~\cite{freek} and~\cite{andrew}. In particular, we compare the mean image obtained using StarWarps video reconstruction. The error type NO ATM. indicates reconstructing using visibilities on data with no atmospheric error, while the error type ATM. indicates using the visibility amplitudes and bispectrum on data where atmospheric phase errors have been introduced. The quality of each result, compared to the ground truth mean image, is indicated in the table of normalized root mean squared errors (Normalized RMSE). To account for the loss of absolute position in the presence of atmospheric phase error, images were rigidly aligned to the true mean before computing the error. The FOV and colorbar used for each reconstruction can be seen in Figure~\ref{fig:groundtruth}. }
		\label{fig:staticevolutionresults}
	\end{center}
\end{figure*}





























































Figure~\ref{fig:staticevolutionresults} shows example reconstructions, and corresponding measured error (NRMSE), for combinations of the 4 source videos observed under the 3 telescope arrays. For these results we have set $a=2$, $c=1/2$, and $\bQ = 10^{\text{-}7} \mathds{1}$. The main portion of this figure is broken up into 4 quadrants, each containing results for one video. From left to right, up to down, each quadrant corresponds to Video 1-4 respectively. The ground truth mean image for each video is shown in the upper table. These images correspond to those shown in Figure~\ref{fig:groundtruth}, but are smoothed to 3/4 the nominal resolution of the interferometer to help illustrate the level of resolution we aim to recover. 

%\vspace{0.05in}
Horizontally within each quadrant we present results obtained using data with varying degrees of difficulty. As the number of telescopes in the array increases, so does the spatial frequency coverage. Therefore, reconstructing an accurate video with the FUTURE array is a much easier task than with the EHT2017 array.
%Additionally, having a linear measurement function, $f(\im)=\FTmtx \im $, results in a convex inference method. 
Additionally, using complex visibilities that are not subject to atmospheric errors is much easier than having to recover images from phase corrupted measurements. 
In the case where there are atmospheric phase errors (ATM.), we constrain the reconstruction problem using a combination of visibility amplitude and bispectrum data products. This results in a non-convex problem (that we approximate with series of linearizations) that is much more difficult to solve than when using complex visibilities when there is no atmospheric phase error (NO ATM.). We demonstrate results on the EHT2017 array for both cases, and the EHT2017+ and FUTURE arrays in the case of atmospheric error. 



%\vspace{0.05in}
Vertically within each quadrant we illustrate the results of our method, StarWarps, by displaying the average frame reconstructed. 
We compare our method to two state-of-the-art Bayesian-style methods.~\cite{andrew} solves for a single image by imposing a combination of MEM and TV priors. This method performs well in the case of a static source (see Figure~\ref{fig:staticimaging}), however, in the case of an evolving source it often results in artifact-heavy reconstructions that are difficult to interpret. In~\cite{freek} the authors attempt to mitigate this problem by first smoothing the time-varying data products before imaging.
%\footnote{In the case of no atmospheric error the visibilities are smoothed. In the case of atmospheric error the bispectrum and visibility amplitudes are smoothed.}. 
This approach was originally designed to work on mutli-epoch data; we find it is unable to accurately recover the source structure from a single day (epoch) observation. 
%Although this approach can sometimes work well, we found that it often requires manual tuning and does not always help to improve reconstructions. 
Results of~\cite{freek} are reconstructed by an author of the method. 







\begin{figure}[tb]
	\begin{center}
		\vspace{-.2in}
		
		\begin{tabular}{   c c | c  c  c   }
			%\hline
			& & \large{\textsf{23 GST}} &\large{\textsf{2 GST}}   &\large{\textsf{6 GST}}    \\ 
			&\vspace{-.1in} & & & \\ \hline
			&\vspace{-.1in} & & & \\
			& \multirow{1}{*}[0.45in]{ \rotatebox[origin=t]{90}{\small{\textsf{Truth}} }} & 
			{{\includegraphics[height=.2\linewidth]{figures/propcmp/gt/gt_74.pdf}} } &
			\includegraphics[height=.2\linewidth]{figures/propcmp/gt/gt_111.pdf} &
			\includegraphics[height=.2\linewidth]{figures/propcmp/gt/gt_159.pdf}  
			\\ \hline
			&\vspace{-.1in} & & & \\
			\multirow{1}{*}[0.6in]{ \rotatebox[origin=t]{90}{\small{\textsf{NO ATM. \& }} }} \hspace{-0.25in} &
			\multirow{1}{*}[0.55in]{ \rotatebox[origin=t]{90}{\small{\textsf{ NO PROP.}} }} &
			{{\includegraphics[height=.2\linewidth]{figures/propcmp/nomotion_NOPROP_hotoka02_vis_eht2017/mean_74.pdf}} } &
			\includegraphics[height=.2\linewidth]{figures/propcmp/nomotion_NOPROP_hotoka02_vis_eht2017/mean_111.pdf} &
			\includegraphics[height=.2\linewidth]{figures/propcmp/nomotion_NOPROP_hotoka02_vis_eht2017/mean_159.pdf} 
			\\
			\multirow{1}{*}[0.55in]{ \rotatebox[origin=t]{90}{\small{\textsf{NO ATM. }} }} \hspace{-0.25in} &
			\multirow{1}{*}[0.5in]{ \rotatebox[origin=t]{90}{\small{\textsf{ \& PROP.}} }} &
			{{\includegraphics[height=.2\linewidth]{figures/propcmp/nomotion_ORIG_hotoka02_vis_eht2017/mean_74.pdf}} } &
			\includegraphics[height=.2\linewidth]{figures/propcmp/nomotion_ORIG_hotoka02_vis_eht2017/mean_111.pdf} &
			\includegraphics[height=.2\linewidth]{figures/propcmp/nomotion_ORIG_hotoka02_vis_eht2017/mean_159.pdf} 
			\\ \hline
			&\vspace{-.1in} & & & \\
			\multirow{1}{*}[0.45in]{ \rotatebox[origin=t]{90}{\small{\textsf{ATM. \&}} }} \hspace{-0.25in} &
			\multirow{1}{*}[0.55in]{ \rotatebox[origin=t]{90}{\small{\textsf{NO PROP.}} }}
			&
			{{\includegraphics[height=.2\linewidth]{figures/propcmp/nomotion_NOPROP_hotoka02_ampbis_eht2017/mean_74.pdf}} } &
			\includegraphics[height=.2\linewidth]{figures/propcmp/nomotion_NOPROP_hotoka02_ampbis_eht2017/mean_111.pdf} &
			\includegraphics[height=.2\linewidth]{figures/propcmp/nomotion_NOPROP_hotoka02_ampbis_eht2017/mean_159.pdf} 
			\\ 
			\multirow{1}{*}[0.48in]{ \rotatebox[origin=t]{90}{\small{\textsf{ATM. \&}} }} \hspace{-0.25in} &
			\multirow{1}{*}[0.45in]{ \rotatebox[origin=t]{90}{\small{\textsf{PROP.}} }} &
			{{\includegraphics[height=.2\linewidth]{figures/propcmp/nomotion_ORIG_hotoka02_ampbis_eht2017/mean_74.pdf}} } &
			\includegraphics[height=.2\linewidth]{figures/propcmp/nomotion_ORIG_hotoka02_ampbis_eht2017/mean_111.pdf} &
			\includegraphics[height=.2\linewidth]{figures/propcmp/nomotion_ORIG_hotoka02_ampbis_eht2017/mean_159.pdf} 
			\\ \hline
		\end{tabular}
		\caption{{\bf Propagating Uncertainty:} During inference, StarWarps approximates each image's covariance matrix in order to propagate its uncertainty to neighboring frames in time. Propagating this information is crucial when using very few measurements. We show frames resulting from the same EHT2017 data of Video 3 when the covariance is propagated (PROP.) as described in this paper, versus not propagated (NO PROP.). Note that propagating the covariance results in significantly improved results. This is true even in the case of using non-linear measurements when atmospheric error is present (ATM.). In this non-linear case the covariance matrix is simply a crude approximation of the uncertainty, but proves critical in obtaining a result that captures the ring structure of the underlying source.   }
		\label{fig:propinfo}
	\end{center}
	\vspace{-.3in}
\end{figure}

%
%\begin{figure*}
%	\begin{center}
%		%	\vspace{-0.5in}
%		%\hspace*{-2.3cm}
%		\setlength{\tabcolsep}{3pt}
%		
%		\hspace{-0.5in}\normalsize{\textsf{BLURRED TRUE MEAN}}  \hspace{5.5cm} \normalsize{\textsf{NORMALIZED RMSE}} 
%		\vspace{0.1in}
%		
%		
%		
%		
%		\hspace*{-1.3cm}
%		\begin{tabular}{  c c | c c  c  c  c  c  c  c  c  }
%			%\hline
%			& & \small{\textsf{Error:}} &\small{\textsf{NO ATM.}}   &\small{\textsf{ATM.}} &\small{\textsf{ATM.}}    &\small{\textsf{ATM.}}  &\small{\textsf{NO ATM.}}   &\small{\textsf{ATM.}} &\small{\textsf{ATM.}}    &\small{\textsf{ATM.}}   \\ \hline
%			&\vspace{-.1in} & & & & & & & & &\\
%			
%			\multirow{2}{*}[0.6in]{ \rotatebox[origin=t]{90}{\small{\textsf{StarWarps}} }}   \hspace{-0.3in} &	\multirow{1}{*}[0.45in]{ \rotatebox[origin=t]{90}{\small{\textsf{Mean}} }}
%			&
%			{{\includegraphics[height=.1\linewidth]{figures/dynamicimagingcmp/gt/gt_104.pdf}} } &
%			{{\includegraphics[height=.1\linewidth]{figures/dynamicimagingcmp/gt/gt_106.pdf}} } &
%			\includegraphics[height=.1\linewidth]{figures/dynamicimagingcmp/gt/gt_108.pdf} &
%			\includegraphics[height=.1\linewidth]{figures/dynamicimagingcmp/gt/gt_110.pdf} &
%			\includegraphics[height=.1\linewidth]{figures/dynamicimagingcmp/gt/gt_112.pdf} 
%			&
%			{{\includegraphics[height=.1\linewidth]{figures/dynamicimagingcmp/gt/gt_114.pdf}} } &
%			\includegraphics[height=.1\linewidth]{figures/dynamicimagingcmp/gt/gt_116.pdf} &
%			\includegraphics[height=.1\linewidth]{figures/dynamicimagingcmp/gt/gt_118.pdf} &
%			\includegraphics[height=.1\linewidth]{figures/dynamicimagingcmp/gt/gt_120.pdf} 
%			\\   \hline
%			&\vspace{-.1in} & & & & & & & & &\\
%			\multirow{2}{*}[0.6in]{ \rotatebox[origin=t]{90}{\small{\textsf{StarWarps}} }}   \hspace{-0.3in} &	\multirow{1}{*}[0.45in]{ \rotatebox[origin=t]{90}{\small{\textsf{Mean}} }}
%			&
%			{{\includegraphics[height=.1\linewidth]{figures/dynamicimagingcmp/snapshot/mean_rescale_104.pdf}} } &
%			{{\includegraphics[height=.1\linewidth]{figures/dynamicimagingcmp/snapshot/mean_rescale_106.pdf}} } &
%			\includegraphics[height=.1\linewidth]{figures/dynamicimagingcmp/snapshot/mean_rescale_108.pdf} &
%			\includegraphics[height=.1\linewidth]{figures/dynamicimagingcmp/snapshot/mean_rescale_110.pdf} &
%			\includegraphics[height=.1\linewidth]{figures/dynamicimagingcmp/snapshot/mean_rescale_112.pdf} 
%			&
%			{{\includegraphics[height=.1\linewidth]{figures/dynamicimagingcmp/snapshot/mean_rescale_114.pdf}} } &
%			\includegraphics[height=.1\linewidth]{figures/dynamicimagingcmp/snapshot/mean_rescale_116.pdf} &
%			\includegraphics[height=.1\linewidth]{figures/dynamicimagingcmp/snapshot/mean_rescale_118.pdf} &
%			\includegraphics[height=.1\linewidth]{figures/dynamicimagingcmp/snapshot/mean_rescale_120.pdf} 
%			\\   \hline
%			&\vspace{-.1in} & & & & & & & & &\\
%			\multirow{2}{*}[0.6in]{ \rotatebox[origin=t]{90}{\small{\textsf{StarWarps}} }}   \hspace{-0.3in} &	\multirow{1}{*}[0.45in]{ \rotatebox[origin=t]{90}{\small{\textsf{Mean}} }}
%			&
%			{{\includegraphics[height=.1\linewidth]{figures/dynamicimagingcmp/di/mean_rescale_104.pdf}} } &
%			{{\includegraphics[height=.1\linewidth]{figures/dynamicimagingcmp/di/mean_rescale_106.pdf}} } &
%			\includegraphics[height=.1\linewidth]{figures/dynamicimagingcmp/di/mean_rescale_108.pdf} &
%			\includegraphics[height=.1\linewidth]{figures/dynamicimagingcmp/di/mean_rescale_110.pdf} &
%			\includegraphics[height=.1\linewidth]{figures/dynamicimagingcmp/di/mean_rescale_112.pdf} 
%			&
%			{{\includegraphics[height=.1\linewidth]{figures/dynamicimagingcmp/di/mean_rescale_114.pdf}} } &
%			\includegraphics[height=.1\linewidth]{figures/dynamicimagingcmp/di/mean_rescale_116.pdf} &
%			\includegraphics[height=.1\linewidth]{figures/dynamicimagingcmp/di/mean_rescale_118.pdf} &
%			\includegraphics[height=.1\linewidth]{figures/dynamicimagingcmp/di/mean_rescale_120.pdf} 
%			\\   \hline
%			&\vspace{-.1in} & & & & & & & & &\\
%			\multirow{2}{*}[0.6in]{ \rotatebox[origin=t]{90}{\small{\textsf{StarWarps}} }}   \hspace{-0.3in} &	\multirow{1}{*}[0.45in]{ \rotatebox[origin=t]{90}{\small{\textsf{Mean}} }}
%			&
%			{{\includegraphics[height=.1\linewidth]{figures/dynamicimagingcmp/nomotion/mean_rescale_104.pdf}} } &
%			{{\includegraphics[height=.1\linewidth]{figures/dynamicimagingcmp/nomotion/mean_rescale_106.pdf}} } &
%			\includegraphics[height=.1\linewidth]{figures/dynamicimagingcmp/nomotion/mean_rescale_108.pdf} &
%			\includegraphics[height=.1\linewidth]{figures/dynamicimagingcmp/nomotion/mean_rescale_110.pdf} &
%			\includegraphics[height=.1\linewidth]{figures/dynamicimagingcmp/nomotion/mean_rescale_112.pdf} 
%			&
%			{{\includegraphics[height=.1\linewidth]{figures/dynamicimagingcmp/nomotion/mean_rescale_114.pdf}} } &
%			\includegraphics[height=.1\linewidth]{figures/dynamicimagingcmp/nomotion/mean_rescale_116.pdf} &
%			\includegraphics[height=.1\linewidth]{figures/dynamicimagingcmp/nomotion/mean_rescale_118.pdf} &
%			\includegraphics[height=.1\linewidth]{figures/dynamicimagingcmp/nomotion/mean_rescale_120.pdf} 
%			\\   \hline
%			&\vspace{-.1in} & & & & & & & & &\\
%			\multirow{2}{*}[0.6in]{ \rotatebox[origin=t]{90}{\small{\textsf{StarWarps}} }}   \hspace{-0.3in} &	\multirow{1}{*}[0.45in]{ \rotatebox[origin=t]{90}{\small{\textsf{Mean}} }}
%			&
%			{{\includegraphics[height=.1\linewidth]{figures/dynamicimagingcmp/sw_di/mean_rescale_104.pdf}} } &
%			{{\includegraphics[height=.1\linewidth]{figures/dynamicimagingcmp/sw_di/mean_rescale_106.pdf}} } &
%			\includegraphics[height=.1\linewidth]{figures/dynamicimagingcmp/sw_di/mean_rescale_108.pdf} &
%			\includegraphics[height=.1\linewidth]{figures/dynamicimagingcmp/sw_di/mean_rescale_110.pdf} &
%			\includegraphics[height=.1\linewidth]{figures/dynamicimagingcmp/sw_di/mean_rescale_112.pdf} 
%			&
%			{{\includegraphics[height=.1\linewidth]{figures/dynamicimagingcmp/sw_di/mean_rescale_114.pdf}} } &
%			\includegraphics[height=.1\linewidth]{figures/dynamicimagingcmp/sw_di/mean_rescale_116.pdf} &
%			\includegraphics[height=.1\linewidth]{figures/dynamicimagingcmp/sw_di/mean_rescale_118.pdf} &
%			\includegraphics[height=.1\linewidth]{figures/dynamicimagingcmp/sw_di/mean_rescale_120.pdf} 
%			\\   \hline
%			&\vspace{-.1in} & & & & & & & & &\\
%		\end{tabular}
%		\caption{{\bf Static evolution model:} }
%		\label{fig:staticevolutionresults}
%	\end{center}
%\end{figure*}


















\begin{figure}
	\begin{center}
			\vspace{-0.2in}
		%\hspace*{-2.3cm}
		\setlength{\tabcolsep}{3pt}
		
		\hspace*{-.3cm}
		\begin{tabular}{  c c | ccccc  }
			%\hline
			& \small{\textsf{GST:}} &\small{\textsf{1:13}}   &\small{\textsf{1:33}} &\small{\textsf{1:53}}    &\small{\textsf{2:13}}  &\small{\textsf{2:33}}   \\ \hline
		&	&\vspace{-.1in} & & & &\\
			
	&	\multirow{1}{*}[0.33in]{ \rotatebox[origin=t]{90}{\small{\textsf{Truth}} }}
			&
			{{\includegraphics[height=.15\linewidth]{figures/dynamicimagingcmp/gt/gt_102.pdf}} } &
			{{\includegraphics[height=.15\linewidth]{figures/dynamicimagingcmp/gt/gt_106.pdf}} } &
			\includegraphics[height=.15\linewidth]{figures/dynamicimagingcmp/gt/gt_110.pdf} &
			\includegraphics[height=.15\linewidth]{figures/dynamicimagingcmp/gt/gt_114.pdf} &
			\includegraphics[height=.15\linewidth]{figures/dynamicimagingcmp/gt/gt_118.pdf} 
			\\   \hline
		&	&\vspace{-.1in} & & & &\\
			 & 	\multirow{1}{*}[0.45in]{ \rotatebox[origin=t]{90}{\small{\textsf{Snapshot}} }}
			&
			{{\includegraphics[height=.15\linewidth]{figures/dynamicimagingcmp/snapshot/mean_rescale_102.pdf}} } &
			{{\includegraphics[height=.15\linewidth]{figures/dynamicimagingcmp/snapshot/mean_rescale_106.pdf}} } &
			\includegraphics[height=.15\linewidth]{figures/dynamicimagingcmp/snapshot/mean_rescale_110.pdf} &
			\includegraphics[height=.15\linewidth]{figures/dynamicimagingcmp/snapshot/mean_rescale_114.pdf} &
			\includegraphics[height=.15\linewidth]{figures/dynamicimagingcmp/snapshot/mean_rescale_118.pdf}  
			\\   \hline
		&	&\vspace{-.1in} & & & &\\
		&	 	\multirow{1}{*}[0.3in]{ \rotatebox[origin=t]{90}{\small{\textsf{~\cite{Johnson_dynamical}}} }}
			&
			{{\includegraphics[height=.15\linewidth]{figures/dynamicimagingcmp/di/mean_rescale_102.pdf}} } &
			{{\includegraphics[height=.15\linewidth]{figures/dynamicimagingcmp/di/mean_rescale_106.pdf}} } &
			\includegraphics[height=.15\linewidth]{figures/dynamicimagingcmp/di/mean_rescale_110.pdf} &
			\includegraphics[height=.15\linewidth]{figures/dynamicimagingcmp/di/mean_rescale_114.pdf} &
			\includegraphics[height=.15\linewidth]{figures/dynamicimagingcmp/di/mean_rescale_118.pdf} 
			\\   \hline
			& &\vspace{-.1in} & & & &\\
		 	& \multirow{1}{*}[0.48in]{ \rotatebox[origin=t]{90}{\small{\textsf{StarWarps}} }}
			&
			{{\includegraphics[height=.15\linewidth]{figures/dynamicimagingcmp/nomotion/mean_rescale_102.pdf}} } &
			{{\includegraphics[height=.15\linewidth]{figures/dynamicimagingcmp/nomotion/mean_rescale_106.pdf}} } &
			\includegraphics[height=.15\linewidth]{figures/dynamicimagingcmp/nomotion/mean_rescale_110.pdf} &
			\includegraphics[height=.15\linewidth]{figures/dynamicimagingcmp/nomotion/mean_rescale_114.pdf} & 
			\includegraphics[height=.15\linewidth]{figures/dynamicimagingcmp/nomotion/mean_rescale_118.pdf} 
			\\   \hline
			& &\vspace{-.1in} & & & &\\
		 		\multirow{2}{*}[0.52in]{ \rotatebox[origin=t]{90}{\small{\textsf{StarWarps }} }}   \hspace{-0.25in} &\multirow{1}{*}[0.33in]{ \rotatebox[origin=t]{90}{\small{\textsf{+~\cite{Johnson_dynamical}}} }}
			&
			{{\includegraphics[height=.15\linewidth]{figures/dynamicimagingcmp/sw_di/mean_rescale_102.pdf}} } &
			{{\includegraphics[height=.15\linewidth]{figures/dynamicimagingcmp/sw_di/mean_rescale_106.pdf}} } &
			\includegraphics[height=.15\linewidth]{figures/dynamicimagingcmp/sw_di/mean_rescale_110.pdf} &
			\includegraphics[height=.15\linewidth]{figures/dynamicimagingcmp/sw_di/mean_rescale_114.pdf} &
			\includegraphics[height=.15\linewidth]{figures/dynamicimagingcmp/sw_di/mean_rescale_118.pdf} 
			\\   \hline
		\end{tabular}
		\caption{{\bf Comparison to~\cite{Johnson_dynamical} and Snapshot Imaging:} A comparison of results obtained using the proposed StarWarps method to Snapshot imaging and a method presented in~\cite{Johnson_dynamical}. The same simulated EHT2017 data of Video 3 was used for each result, and contained atmospheric noise. Snapshot imaging, which independently reconstructs each frame, is unable to produce reasonable results, and has no continuity through time due to the loss of absolute location information when using atmosphere corrupted measurements. The more flexible framework of~\cite{Johnson_dynamical} often makes it possible to obtain sharper and cleaner images, however struggles when working with very few measurements, as is the case for the EHT2017 array. Although results are consistent through time,~\cite{Johnson_dynamical} fails to recover the true ring structure of the source. StarWarps is able to begin recovering this ring structure, but contains a number of artifacts spurring from the main ring structure. Initializing~\cite{Johnson_dynamical} with the result of StarWarps produces a cleaner and sharper result.  
			 }
		\label{fig:dynamicimagingcmp}
	\end{center}
\end{figure}














%\begin{figure}
%	\begin{center}
%		%	\vspace{-0.5in}
%		%\hspace*{-2.3cm}
%		\setlength{\tabcolsep}{3pt}
%		
%		\hspace*{-.3cm}
%		\begin{tabular}{  c c | c  c  c   c   c  }
%			%\hline
%			& \small{\textsf{GST:}} &\small{\textsf{1:13}}   &\small{\textsf{1:33}} &\small{\textsf{1:53}}    &\small{\textsf{2:13}}  &\small{\textsf{2:33}}   \\ \hline
%			&	&\vspace{-.1in} & & & &\\
%			
%			&	\multirow{1}{*}[0.33in]{ \rotatebox[origin=t]{90}{\small{\textsf{Truth}} }}
%			&
%			{{\includegraphics[height=.15\linewidth]{figures/dynamicimagingcmp_hotspot/gt/gt_crop_104.pdf}} } &
%			{{\includegraphics[height=.15\linewidth]{figures/dynamicimagingcmp_hotspot/gt/gt_crop_108.pdf}} } &
%			\includegraphics[height=.15\linewidth]{figures/dynamicimagingcmp_hotspot/gt/gt_crop_112.pdf} &
%			\includegraphics[height=.15\linewidth]{figures/dynamicimagingcmp_hotspot/gt/gt_crop_116.pdf} &
%			\includegraphics[height=.15\linewidth]{figures/dynamicimagingcmp_hotspot/gt/gt_crop_120.pdf} 
%			\\   \hline
%			&	&\vspace{-.1in} & & & &\\
%			& 	\multirow{1}{*}[0.45in]{ \rotatebox[origin=t]{90}{\small{\textsf{Snapshot}} }}
%			&
%			{{\includegraphics[height=.15\linewidth]{figures/dynamicimagingcmp_hotspot/snapshot/mean_rescale_104.pdf}} } &
%			{{\includegraphics[height=.15\linewidth]{figures/dynamicimagingcmp_hotspot/snapshot/mean_rescale_108.pdf}} } &
%			\includegraphics[height=.15\linewidth]{figures/dynamicimagingcmp_hotspot/snapshot/mean_rescale_112.pdf} &
%			\includegraphics[height=.15\linewidth]{figures/dynamicimagingcmp_hotspot/snapshot/mean_rescale_116.pdf} &
%			\includegraphics[height=.15\linewidth]{figures/dynamicimagingcmp_hotspot/snapshot/mean_rescale_120.pdf}  
%			\\   \hline
%			&	&\vspace{-.1in} & & & &\\
%			&	 	\multirow{1}{*}[0.3in]{ \rotatebox[origin=t]{90}{\small{\textsf{~\cite{Johnson_dynamical}}} }}
%			&
%			{{\includegraphics[height=.15\linewidth]{figures/dynamicimagingcmp_hotspot/di/mean_rescale_104.pdf}} } &
%			{{\includegraphics[height=.15\linewidth]{figures/dynamicimagingcmp_hotspot/di/mean_rescale_108.pdf}} } &
%			\includegraphics[height=.15\linewidth]{figures/dynamicimagingcmp_hotspot/di/mean_rescale_112.pdf} &
%			\includegraphics[height=.15\linewidth]{figures/dynamicimagingcmp_hotspot/di/mean_rescale_116.pdf} &
%			\includegraphics[height=.15\linewidth]{figures/dynamicimagingcmp_hotspot/di/mean_rescale_120.pdf} 
%			\\   \hline
%			& &\vspace{-.1in} & & & &\\
%			& \multirow{1}{*}[0.48in]{ \rotatebox[origin=t]{90}{\small{\textsf{StarWarps}} }}
%			&
%			{{\includegraphics[height=.15\linewidth]{figures/dynamicimagingcmp_hotspot/nomotion/mean_rescale_104.pdf}} } &
%			{{\includegraphics[height=.15\linewidth]{figures/dynamicimagingcmp_hotspot/nomotion/mean_rescale_108.pdf}} } &
%			\includegraphics[height=.15\linewidth]{figures/dynamicimagingcmp_hotspot/nomotion/mean_rescale_112.pdf} &
%			\includegraphics[height=.15\linewidth]{figures/dynamicimagingcmp_hotspot/nomotion/mean_rescale_116.pdf} & 
%			\includegraphics[height=.15\linewidth]{figures/dynamicimagingcmp_hotspot/nomotion/mean_rescale_120.pdf} 
%			\\   \hline
%			& &\vspace{-.1in} & & & &\\
%			\multirow{2}{*}[0.52in]{ \rotatebox[origin=t]{90}{\small{\textsf{StarWarps }} }}   \hspace{-0.25in} &\multirow{1}{*}[0.33in]{ \rotatebox[origin=t]{90}{\small{\textsf{+~\cite{Johnson_dynamical}}} }}
%			&
%			{{\includegraphics[height=.15\linewidth]{figures/dynamicimagingcmp_hotspot/sw_di/mean_rescale_104.pdf}} } &
%			{{\includegraphics[height=.15\linewidth]{figures/dynamicimagingcmp_hotspot/sw_di/mean_rescale_108.pdf}} } &
%			\includegraphics[height=.15\linewidth]{figures/dynamicimagingcmp_hotspot/sw_di/mean_rescale_112.pdf} &
%			\includegraphics[height=.15\linewidth]{figures/dynamicimagingcmp_hotspot/sw_di/mean_rescale_116.pdf} &
%			\includegraphics[height=.15\linewidth]{figures/dynamicimagingcmp_hotspot/sw_di/mean_rescale_120.pdf} 
%			\\   \hline
%			&\vspace{-.1in} & & & &\\
%		\end{tabular}
%		\caption{{\bf Static evolution model:} }
%		\label{fig:staticevolutionresults}
%	\end{center}
%\end{figure}
%
%







\begin{figure}[h!]
	\vspace*{-.3in}
	\centering
	%{\includegraphics[height=.28\linewidth]{figures/uvcoverage/uv_eht2017.pdf}}
	\subfigure[Fig.~\ref{fig:dynamicimagingcmp} frame uv-coverage  ]{\includegraphics[height=.43\linewidth]{figures/dynamicimagingcmp/uvcovarage/hotoka_uvcov.pdf}}
	\subfigure[Fig.~\ref{fig:m87} frame uv-coverage]{\includegraphics[height=.43\linewidth]{figures/dynamicimagingcmp/uvcovarage/m87_uvcov.pdf}}
	\vspace{-.1in}
	\caption{{\bf Single frame uv-coverage:} (a) The uv-coverage for the first frame shown in Figure~\ref{fig:dynamicimagingcmp} contains 21 measurements while (b) the uv-coverage for the first frame shown in Figure~\ref{fig:m87} contains 1736. When the measurements provided are very sparse, as in (a), StarWarps significantly outperforms~\cite{Johnson_dynamical}. However, in the case of many measurements, as in (b),~\cite{Johnson_dynamical} achieves better results with a higher dynamic range.  }
	\label{fig:uvcov3}
	\vspace{-.25in}
	
\end{figure}









\begin{figure*}
	\begin{center}

			\vspace*{-.35in}

	\begin{tabular}{  c c c  }
		%\hline
					\multicolumn{3}{c}{{\includegraphics[width=1\linewidth]{figures/m87/m87_figure_r2.pdf}} } 
					\\
					\vspace{-.25in} && \\
%	\hspace{.8in}	\normalsize{\textsf{\cite{Johnson_dynamical} }}& \hspace{2.1in} \normalsize{\textsf{StarWarps }}&\hspace{1.5in} 	 \normalsize{\textsf{Space $\times$ Time Image }}\\
	\hspace{.8in}	\normalsize{\textsf{\cite{Johnson_dynamical} }}& \hspace{2.1in} \normalsize{\textsf{StarWarps }}&\hspace{1.5in} 	 \\
		\vspace{.1in} && 
\end{tabular}

\vspace*{-.27in}
		\caption{{\bf Video Reconstruction of Real Observations:} A StarWarps movie reconstruction obtained using real VLBI data taken of the M87 jet over the course of a year. Frames are shown with a gamma correction of $\gamma={1}/{3}$ to highlight weak emission. Data was collected in 2007 using the Very Long Baseline Array (VLBA) at 43 GHz~\cite{walker2016observations}. As the source structure does not evolve over the course of a night, traditional imaging approaches can be used to reconstruct `snapshot' images from this data. 
			The forked structure appearing in the StarWarps reconstructions also appears in images reconstructed using the CLEAN static imaging approach~\cite{walker2016observations} and the dynamic imaging approach presented in~\cite{Johnson_dynamical} (see left-most image). 
StarWarps produces a video that allows us to easily visualize the moving arms of the jet; the reconstructed video appears to contain outward motion, with a brighter region propagating down the arms. By visualizing the same slice of each frame (indicated by the cyan line) it becomes easier to see this motion as a static image (see the 3 Space $\times$ Time images on the far right). Note the diagonal `line' shown in the top 2 of these Space $\times$ Time images indicates a bright region moves down the arm, towards the right of the image. The intensity of these slices has been increased by $70 \%$ to highlight the evolving, weak emission. By controlling the amount of temporal regularization, $\bQ$, we control the amount of motion that appears in the reconstructed video. Increasing the temporal regularization, by decreasing $\bQ$, results in a Space $\times$ Time slice that varies less with Time. Individual frames shown were generated using $\bQ = 10^{\text{-}6} \mathds{1}$. }
\vspace{-.3in}
		\label{fig:m87}
	\end{center}
\end{figure*}






\subsubsection{The Importance in Propagating Uncertainty}

StarWarps uses a multivariate Gaussian regularizer for imaging, which leads to a straightforward optimization method that propagates information through time. The uncertainty of each reconstructed image is encompassed in its approximated covariance matrix (${\bf P}_{t|t}$), which informs the reconstruction of each neighboring latent image. Although this covariance matrix is sometimes a crude estimate of the true uncertainty, it is still crucial in reconstructing faithful images when measurements are especially sparse.

The importance of propagating uncertainty through the covariance matrix is demonstrated in Figure~\ref{fig:propinfo}. This figure shows the effect of turning on and off the covariance propagation. Covariance propagation can be easily turned off by setting ${\bf P}_{t|t}=0$ at each forward and backward update. Results in the figure are shown on simulated data from the EHT2017 array on Video 3, with and without atmospheric error. Note that in both cases, propagating the covariance matrix helps to substantially improve results. This is true even even in the case of atmospheric error, when the measurement function $f(\im)$ is non-linear and the covariance matrix is only a rough approximation of the true uncertainty. 





\subsubsection{Dynamical Imaging Comparison}

%StarWarps uses a multivariate Gaussian imaging regularizer for interferometric imaging, which enables a straightforward optimization method that propagates information through time.  
%In~\cite{Johnson_dynamical} (in prep), we develop alternative methods for reconstructing video from interferometric data. These methods allow for greater flexibility to incorporate a variety of imaging assumptions. However, they are prone to local minima, and thus come at the expense of much more difficulty in converging to the true underlying structure.  
%These approaches have significantly different strengths and may ultimately lead to hybrid approaches for video reconstruction that produce higher quality results, even with noisy and sparse data. 


As discussed in Section~\ref{sec:setup}, the Dynamical Imaging method presented in~\cite{Johnson_dynamical} was developed simultaneously, and shares many similarities to the work presented in this paper: they both aim to solve for a video rather than a static image. However, they have significant differences, leading to different strengths and weaknesses. The framework of~\cite{Johnson_dynamical} allows for more sophisticated image and temporal regularization, at the cost of a difficult optimization problem that does not propagate uncertainty. This results in sharper and cleaner videos when there is sufficient data, but can lead to poor results when there are very few measurements. 
Conversely, StarWarps' use of very simple Gaussian image and temporal regularization results in blurry results, but allows us to propagate an approximation of uncertainty (through the covariance matrix) and produce better results when very few measurements are available.

A comparison of results from~\cite{Johnson_dynamical} and StarWarps on EHT2017 simulated data can be seen in Figure~\ref{fig:dynamicimagingcmp}. Results of~\cite{Johnson_dynamical} were produced using $\mathcal{R}_{\Delta I}$ and KL $\mathcal{R}_{\Delta t}$ temporal regularization, and Maximum Entropy and Total Variation Squared image regularization. Note that for this especially sparse data,~\cite{Johnson_dynamical} on its own does not faithfully reconstruct the ring structure of the underlying source. StarWarps is able to produce a ring, but with a number of blurry artifacts. Initializing~\cite{Johnson_dynamical} with the output of StarWarps produces the cleanest result. 
Although the StarWarps method runs faster on this example than~\cite{Johnson_dynamical} (84 seconds vs 204 seconds in Python on a 2.8 GHz Intel Core i7), StarWarps is memory intensive and its computational complexity scales poorly with increasing image size compared to~\cite{Johnson_dynamical}.  
%However, in most relevant cases StarWarps can still be easily run on a personal computer. 
To help solve these issues, in the future ideas from Ensemble Kalman Filters could be adapted in order to avoid StarWarp's costly matrix inversions and reduce the method's memory footprint~\cite{evensen2003ensemble}.  

An additional result comparing the two methods can be seen in Figure~\ref{fig:m87}, which is discussed in the next section. In this example there is sufficient data to reconstruct each frame independently, and~\cite{Johnson_dynamical} is able to produce a cleaner image with a higher dynamic range than StarWarps. 

Figure~\ref{fig:uvcov3} compares the uv-coverage of a single frame for Figures~\ref{fig:dynamicimagingcmp} and Figure~\ref{fig:m87}, highlighting that StarWarps is comparatively strongest  in the case of sparse data, as will be available for the EHT. These examples demonstrate that StarWarps and~\cite{Johnson_dynamical} are complementary methods, and may ultimately lead to hybrid approaches for video reconstruction that produce higher quality results. 


%even with noisy and sparse data. 

%these methods are complementary to one another and can be used together to produce better results. 

 
%only a few measurements are available. 


\subsubsection{Application to Real VLBI Data}


Although StarWarps was developed with the considerations of the EHT in mind, it can be applied to VLBI data taken from other sources and telescope arrays. For instance, galactic relativistic jet sources (``microquasars") often show variability over the course of a single observation~\cite{timedeprecon}. 
However, due to physical constraints, most VLBI telescope networks observe sources that do not evolve this quickly, such as distant jets from the cores of Active Galactic Nuclei. In these cases, traditional static imaging approaches can be applied to each night of data to produce faithful reconstructions. Yet, by jointly processing the data taken over a larger span of time, we are able to make movies of long-term source evolution that preserve continuity of features through time, thus reducing the flickering that occurs when independently reconstructing each frame. 

In Figure~\ref{fig:m87} we demonstrate StarWarps on archival data taken of the M87 jet. This data was taken using the Very Long Baseline Array (VLBA) as part of the M87 Movie Project~\cite{walker2016observations}. Ten epochs of data between the beginning of January and end of August in 2007 were processed simultaneously. Images were reconstructed with a 10 m-arcsecond field of view with $\npix=70$ pixels. 
%Note this field of view is an order of magnitude larger than is expected for EHT observations. 

Unlike as expected in EHT observations, the dynamic range of the M87 Jet is very high. In order to faithfully reconstruct a high dynamic range image using the simple Gaussian prior, we have incorporated gamma correction into our measurement function. Rather than reconstruct a video containing linear-scale images, we instead reconstruct gamma-corrected images. To do this we replace the measurement function $f(\im)$ with $f(\im^{\frac{1}{\gamma}})$. During reconstruction of this M87 Jet video we have used $\gamma=1/2$. Although these images still do not have the same dynamic range that is achieved through other imaging methods~\cite{walker2016observations,Johnson_dynamical}, StarWarps is still able to recover the faint arms of the jet. 

The reconstructed movie produced by StarWarps shows outward motion along the jet. While this motion is hard to see in Figure~\ref{fig:m87}'s static frames, by visualizing a slice of each image (indicated by the cyan line) through time the motion becomes more apparent. The resulting Space $\times$ Time image shows a brighter region of emission moving along the arm of the jet. We show the same Space $\times$ Time reconstruction for different weightings of temporal regularization, $\bQ$. Note that as temporal regularization increases, by decreasing $\bQ$, the Space $\times$ Time image becomes more uniform in time. 
%See the supplemental material for reconstructed videos of this source. 
 
\vspace{-.2in}
\subsection{Unknown Evolution  Model (Learn Warp)}

\begin{figure}[tb]
\vspace{-.35in}
\hspace*{-.5in}
\centering
	\begin{center}
		\setlength{\tabcolsep}{1pt}
		%\hspace*{-1.5cm}
		\begin{tabular}{  c | c | c }
\hspace*{-.1in} & \large{\textsf{NO ATM. ERROR}}   &\large{\textsf{ATM. ERROR}}      \\  \hline


& \vspace{-0.05in} & \\

\hspace*{-.1in} \multirow{1}{*}[0.9in]{ \rotatebox[origin=t]{90}{  \large{\textsf{Video 1}} }} & \hspace{0.05in} {{\includegraphics[height=0.35\linewidth]{figures/recov_flowfields/rot30_vis/flow_crop.pdf}} } \hspace{0.005in}  & \hspace{0.05in}
{{\includegraphics[height=0.35\linewidth]{figures/recov_flowfields/rot30_bis/flow_crop.pdf}} } \\ 

& \vspace{-0.09in} & \\

\hline

& \vspace{-0.05in} & \\

\hspace*{-.1in} \multirow{1}{*}[0.9in]{ \rotatebox[origin=t]{90}{ \large{\textsf{Video 2}} }} \hspace{0.06in}  &  \hspace{0.05in} {{\includegraphics[height=0.35\linewidth]{figures/recov_flowfields/hotspot100sR2_vis/flow_crop.pdf}} }  \hspace{0.005in}& \hspace{0.05in}
{{\includegraphics[height=0.35\linewidth]{figures/recov_flowfields/hotspot100sR2_bis/flow_crop.pdf}} }
\end{tabular}
\end{center}
\caption{{\bf Recovering Warp Field:} By solving for the parameters of a persistent warp field using the proposed EM algorithm, we are able to recover a low-dimensional representation of the source dynamics. Results are shown using the EHT2017+ array with and without atmospheric error (ATM. and NO ATM. ERROR, respectively). Arrows showing the direction of recovered motion are overlaid on the mean image for a recovered video. Refer to the supplemental video for a visualization of the true underlying and recovered videos. In Video 1 the true underlying motion can be described by a clockwise rotation. The proposed method is able to recover Video 1's motion from the observed data. Video 2 contains a `hot spot' rotating counter-clockwise around a static emission. Video 2 cannot be described using a single persistent flow field. Yet, despite this, the proposed method is still able to recover the general direction of counter-clockwise motion. 
}
\label{fig:warpfield}
\vspace{-.15in}
\end{figure}
\begin{figure}[tb]
	\begin{center}
		\vspace{-.35in}
		\hspace*{-0.5cm}
		\begin{tabular}{   c | c  c   }
			%\hline
			&\vspace{-.1in}&\\
			& \small{\textsf{Single Frame}} &\small{\textsf{ Unwrapped }}       \\ 
			&\vspace{-.1in}&\\
			& \small{\textsf{with Overlaid Circle}} &\small{\textsf{Space $\times$ Time Image }}       \\ \hline
			&\vspace{-.1in}&\\
			\multirow{1}{*}[0.5in]{ \rotatebox[origin=t]{90}{\small{\textsf{ Truth }} }}
				&
				{{\includegraphics[height=.2\linewidth]{figures/hotspotcircleinterp/groundtruth.pdf}} } &
				\includegraphics[height=.2\linewidth]{figures/hotspotcircleinterp/labels.pdf} 
				\\
				&\vspace{-.1in}&\\
				\multirow{1}{*}[0.65in]{ \rotatebox[origin=t]{90}{  \specialcell{ \small{\textsf{Blurred Truth}} \\  \tiny{\textsf{ $\frac{3}{4}$ Nominal Beam}}}  }}
			&
			{{\includegraphics[height=.2\linewidth]{figures/hotspotcircleinterp/groundtruth_blur.pdf}} } &
			\includegraphics[height=.2\linewidth]{figures/hotspotcircleinterp/groundtruth_blur.png} 
			\\
			&\vspace{-.1in}&\\
			\multirow{1}{*}[0.6in]{ \rotatebox[origin=t]{90}{\small{\textsf{ Snapshot }} }}
			&
			{{\includegraphics[height=.2\linewidth]{figures/hotspotcircleinterp/snapshot.pdf}} } &
			\includegraphics[height=.2\linewidth]{figures/hotspotcircleinterp/snapshot.png} 
			\\
				&\vspace{-.1in}&\\
					\multirow{1}{*}[0.65in]{ \rotatebox[origin=t]{90}{  \specialcell{ \small{\textsf{StarWarps:}} \\  \small{\textsf{No Warp}}}  }}
				&
				{{\includegraphics[height=.2\linewidth]{figures/hotspotcircleinterp/nomotion.pdf}} } &
				\includegraphics[height=.2\linewidth]{figures/hotspotcircleinterp/nomotion.png} 
				\\
				&\vspace{-.1in}&\\
				\multirow{1}{*}[0.65in]{ \rotatebox[origin=t]{90}{  \specialcell{ \small{\textsf{StarWarps:}} \\  \small{\textsf{Learn Warp}}}  }}
				&
				{{\includegraphics[height=.2\linewidth]{figures/hotspotcircleinterp/best.pdf}} } &
				\includegraphics[height=.2\linewidth]{figures/hotspotcircleinterp/best.png} 
				\\
		\end{tabular}
		\caption{{\bf Visualizing Recovered Motion:} We visualize the recovered motion in Video 2 by displaying the change in intensity around a circle in the image over time. After fitting a circle of constant radius to each video,  the intensities around the circle in each image are unwrapped and placed in a single column in the unwrapped space $\times$ time image. As the hot spot rotates around the black hole a distinctive line appears in the true angle $\times$ time image. These lines also appear in the StarWarps angle $\times$ time images, but are harder to discern among the other artifacts in the snapshot imaging result. Results were obtained using the EHT2017+ array with added atmospheric noise, and correspond to results shown in Figure 2 of the supplemental document. As the absolute position of the source is lost when using the closure phase or bispectrum, the position of the recovered black hole moves slightly over the course of the video. This causes the fluctuation in the intensity of the bright horizontal line in the StarWarps recovered angle $\times$ time images, as we do not shift the position of the fitted circle.  }
		\label{fig:motion}
	\end{center}
	\vspace{-.35in}
\end{figure}


In Section~\ref{sec:nomotionresults} we showed that a static model can often substantially improve results over the state-of-the-art methods, even when there is significant global motion. However, when a source's emission region evolves in a similar way over time, we are able to further improve results by simultaneously estimating a persistent warp field along with the video frames.
%erratic  
%can be modeled by a persistnat change 
We demonstrate the StarWarps EM approach proposed in Section~\ref{sec:dynamic_inference_unknown}, on Videos 1 and 2. In results presented, we have assumed an affine motion basis with no translation ($\theta$ consists of 4 parameters), and have allowed the method to converge over 30 EM iterations. % in Figures BLAH and BLAH for Videos 1 and 2 respectively. 

%The recovered warp fields for videos 1 and 2 is visualized in Figure~\ref{fig:warpfield}. 
Figure~\ref{fig:warpfield} shows the warp field recovered by our EM algorithm. 
Results were obtained from data with and without atmospheric error. In Video 1 the true underlying motion of the emission region can be perfectly captured by the affine model we assume. This allows us to freely recover a very similar warp field. 
However, in the ``hot spot" video (Video 2), there does not exist a persistent warp field that fits the data, let alone an affine warp field. Although the true motion cannot be described by our model, we still recover an accurate estimate indicating the direction of motion. 
%The recovered motion fields for videos 1 and 2 is visualized in Figure~\ref{fig:warpfield}. 


Figure~\ref{fig:motion} helps to further visualize the recovered motion in the ``hot spot" video by showing how the intensities of a region evolve over time. Results of our method are compared to that of a simple baseline method that we refer to as `snapshot imaging'. In snapshot imaging each frame is independently reconstructed using only the small number of measurements taken at that time step.  In particular, we use the MEM \& TV method shown in Figure~\ref{fig:staticimaging} to reconstruct each snapshot. 
Our results using StarWarps show substantial improvement over snapshot imaging, especially in the case of data containing atmospheric phase error.


We expand upon these results in the supplemental material's video and document. 
%Figures~\ref{fig:rotation_example1} and~\ref{fig:rotation_example2} 
Figures 1 and 2 in the supplemental material document compare results obtained when we assume no global motion ($A=\mathds{1}$) to those when we allow the method to search for a persistent warp field. %Results are shown in two settings: when data is generated using the EHT2017+ array assuming no atmospheric phase error (VIS), as well as when phase errors are introduced (AMP \& BISP) into the measurements. At each time, only a small number of measurements are observed (indicated by the corresponding uv-coverage). However, by propagating information across the video we are able to reconstruct good quality images at each time step. 
In the case of large global motion, most of the reconstructed motion is suppressed when we assume $A=\mathds{1}$. However, by solving for the low dimensional parameters of the warp field, $\theta$, we can learn about the underlying dynamics and sometimes produce higher quality videos. %, while also inferring the underlying dynamics of the source. 










%In the case of using complex visibilities, both our method and snapshot imaging produce meaningful results. Although distinctive features of the true underlying image are recovered by both methods, the quality of our StarWarps reconstructions is higher. However, in the case of data containing atmospheric phase errors our method shows substantial improvement over snapshot imaging. As the closure phase and bispectrum are invariant to the absolute position of the source, each snapshot reconstruction produces an image that is shifted by a different amount. This makes it challenging to align the snapshot frames to pull out meaningful structure in the reconstructed video when there is sparse uv coverage. For this reason our method substantially outperforms snapshot imaging. 










%\input{paperparts/figures/rotation_example}







\vspace{-.15in}
\section{Conclusion}
\label{sec:conclusion}

Traditional interferometric imaging methods are designed under the assumption that the target source is static over the course of an observation~\cite{TMS}. However, as we continue to push instruments to recover finer angular resolution, this assumption may no longer be valid. For instance, the innermost orbital periods around the Milky Way's supermassive black hole, Sgr A*, are just minutes~\cite{orbitalperiod}. In these cases, we have demonstrated that traditional imaging methods often break down. 

In this work, we propose a way to model VLBI measurements that allows us to recover both the appearance and dynamics of a rapidly evolving source. Our proposed approach, StarWarps, reconstructs a video rather than a static image. By propagating information across time, it produces significant improvements over conventional approaches to create static images or a series of snapshot images in time.

Our technique will hopefully soon allow for video reconstruction of sources that change on timescales of minutes, allowing a real-time view of the most energetic and explosive events in the universe. 

\vspace{-.2in}




%Methods such as this will make it possible to use imaging to study quickly-changing astronomical source and open up many scientific opportunities once barred from this form of study. This paper represents a first step into this undoubtedly rich field of astronomical dynamical imaging.


%In this work we propose a new way to model VLBI measurements that allows us to recover both the appearance and dynamics of an evolving emission. In past work, imaging methods were designed under the assumption that the source is static over the course of an observation. However, as we continue to push our instruments to be able to recover finer angular resolution, this assumption is no longer valid. For instance, the immediate environment around the Milky Way's galactic black hole, SgrA*, and micro-quasar jets evolve over the course of just minutes. In these cases, we have shown that traditional imaging methods often break down. Our proposed approach, StarWarps, reconstructs a video rather than a static image and propagates information across time in an attempt to retain as much image detail as possible. 
 
%This paper represents a first step into this undoubtedly rich field of astronomical dynamical imaging. However, we believe there is much more than can be done in the future to improve these methods. 
%In this paper, we make the choice to use a simple multivariate Gaussian imaging regularizer. This allows us to better understand the system and derive a straight-forward, simple optimization method.
%However, just as more sophisticated regularizer have propelled the single-image imaging methods, we believe that these can be built on top of our method to further improve results. Methods such as this will make it possible to use imaging to study quickly-changing astronomical source and open up many scientific opportunities once barred from this form of study. 

 




%\clearpage
%\footnotesize{
\begingroup
\setstretch{0.79}
\bibliographystyle{splncs}
\bibliography{egbib}
\endgroup
%}

%\begin{partbacktext}
\part{Appendices}
\end{partbacktext}

\appendix

\chapter{Collections}
\label{appendix:collections}

\abstract{This appendix gives a description of the vector collections used in experiments
throughout this monograph. These collections demonstrate different operating points in
a typical use-case. For example, some consist of dense vectors, others of sparse vectors;
some have few dimensions and others are in much higher dimensions; some are relatively small
while others contain a large number of points.}

\bigskip

Table~\ref{table:appendix:collections:dense} gives a description of the dense vector collections
used throughout this monograph and summarizes their key statistics.

\begin{table*}[ht]
\caption{Dense collections used in this monograph along with select statistics.}
\scriptsize
\label{table:appendix:collections:dense}
\begin{center}
\begin{sc}
\begin{tabular}{p{5cm}|ccc}
\toprule
Collection & Vector Count & Query Count & Dimensions \\
\midrule
\textsc{GloVe}-$25$~\citep{pennington-etal-2014-glove} & $1.18$M & $10{,}000$ & $25$ \\
\textsc{GloVe}-$50$ & $1.18$M & $10{,}000$ & $50$ \\
\textsc{GloVe}-$100$ & $1.18$M & $10{,}000$ & $100$ \\
\textsc{GloVe}-$200$ & $1.18$M & $10{,}000$ & $200$ \\
\textsc{Deep1b}~\citep{deep1b} & $9.99$M & $10{,}000$ & $96$ \\
\textsc{MS Turing}~\citep{msturingDataset} & $10$M & $100{,}000$ & $100$ \\
\textsc{Sift}~\citep{Lowe2004DistinctiveIF} & $1$M & $10{,}000$ & $128$ \\
\textsc{Gist}~\citep{Oliva2001ModelingTS} & $1$M & $1{,}000$ & $960$ \\
\bottomrule
\end{tabular}
\end{sc}
\end{center}
\end{table*}

In addition to the vector collections above, we convert a few text collections
into vectors using various embedding models. These collections are described in
Table~\ref{table:appendix:collections:text}. Please see~\citep{nguyen2016msmarco} for
a complete description of the MS MARCO v1 collection and~\citep{thakur2021beir} for the others.

\begin{table*}[ht]
\caption{Text collections along with key statistics.
The rightmost two columns report the average number of non-zero
entries in data points and, in parentheses, queries for sparse vector
representations of the collections.}
\scriptsize
\label{table:appendix:collections:text}
\begin{center}
\begin{sc}
\begin{tabular}{c|cc|cc}
\toprule
Collection & Vector Count & Query Count & \splade{} & \esplade{}\\
\midrule
\textsc{MS Marco} Passage& $8.8$M & $6{,}980$ & 127 (49) & 185 (5.9) \\
NQ & $2.68$M & $3{,}452$ & 153 (51) & 212 (8) \\
\textsc{Quora} & $523$K & $10{,}000$ & 68 (65) & 68 (8.9) \\
\textsc{HotpotQA} & $5.23$M & $7{,}405$ & 131 (59) & 125 (13) \\
\textsc{Fever} & $5.42$M & $6{,}666$ & 145 (67) & 140 (8.6) \\
\textsc{DBPedia} & $4.63$M & $400$ & 134 (49) & 131 (5.9) \\
\bottomrule
\end{tabular}
\end{sc}
\end{center}
\end{table*}

When transforming the text collections of Table~\ref{table:appendix:collections:text}
into vectors, we use the following embedding models:
\begin{itemize}
    \item \textsc{AllMiniLM-l6-v2}:\footnote{Available at \url{https://huggingface.co/sentence-transformers/all-MiniLM-L6-v2}}
    Projects text documents into $384$-dimensional dense vectors for retrieval with angular distance.

    \item \textsc{Tas-B}~\citep{tas-b}: A bi-encoder model that was trained using supervision from a cross-encoder and a ColBERT~\citep{colbert2020khattab} model,
    and produces $768$-dimensional dense vectors that are meant for MIPS.
    The checkpoint used in this work is available on HuggingFace.\footnote{Available at \url{https://huggingface.co/sentence-transformers/msmarco-distilbert-base-tas-b}}

    \item \splade{}~\citep{formal2022splade}:\footnote{Pre-trained checkpoint from HuggingFace available at \url{https://huggingface.co/naver/splade-cocondenser-ensembledistil}}
    Produces sparse representations for text.
    The vectors have roughly $30{,}000$ dimensions, where each dimension corresponds
    to a term in the BERT~\citep{devlin2019bert} WordPiece~\citep{wordpiece} vocabulary.
    Non-zero entries in a vector reflect learnt term importance weights.

    \item \esplade{}~\citep{lassance2022sigir}:\footnote{Pre-trained checkpoints for document and
    query encoders were obtained from \url{https://huggingface.co/naver/efficient-splade-V-large-doc} and \url{https://huggingface.co/naver/efficient-splade-V-large-query},
    respectively.}
    This model produces queries that have far fewer non-zero entries than the original
    \splade{} model, but documents that may have a larger number of non-zero entries.
\end{itemize}

\bibliographystyle{abbrvnat}
\bibliography{biblio}


\chapter{Probability Review}
\label{appendix:probability}

\abstract{We briefly review key concepts in probability in this appendix.}

\section{Probability}
We identify a \emph{probability space} denoted by $(\Omega, \mathcal{F}, \probability)$
with an \emph{outcome space}, an \emph{events} set, and a \emph{probability measure}.
The outcome space, $\Omega$, is the set of all
possible outcomes. For example, when flipping a two-sided coin, the outcome
space is simply $\{0, 1\}$. When rolling a six-sided die, it is instead
the set $[6] = \{ 1, 2, \ldots, 6\}$.

The events set $\mathcal{F}$ is a set of subsets of $\Omega$ that
includes $\Omega$ as a member and is closed under complementation and
countable unions. That is, if $E \in \mathcal{F}$,
then we must have that $E^\complement \mathcal{F}$.
Furthermore, the union of countably many events $E_i$'s
in $\mathcal{F}$ is itself in $\mathcal{F}$: $\cup_i E_i \in \mathcal{F}$.
A set $\mathcal{F}$ that satisfies these properties is called a $\sigma$-algebra.

Finally, a function $\probability: \mathcal{F} \rightarrow \mathbb{R}$ is
a probability measure if it satisfies the following conditions: $\probability[\Omega] = 1$;
$\probability[E] \geq 0$ for any event $E \in \mathcal{F}$;
$\probability[E^\complement] = 1 - \probability[E]$; and, finally,
for countably many disjoint events $E_i$'s:
$\probability[\cup_i E_i] = \sum_i \probability[E_i]$.

We should note that, $\probability$ is also known as a ``probability distribution''
or simply a ``distribution.'' The pair $(\Omega, \mathcal{F})$ is called
a \emph{measurable space}, and the elements of $\mathcal{F}$ are
known as a \emph{measurable sets}. The reason they are called ``measurable''
is because they can be ``measured'' with $\probability$: The function
$\probability$ assigns values to them.

In many of the discussions throughout this monograph, we omit the outcome space
and events set because that information is generally clear from context.
However, a more formal treatment of our arguments requires a complete
definition of the probability space.

\section{Random Variables}
A random variable on a measurable space $(\Omega, \mathcal{F})$ is
a measurable function $X: \Omega \rightarrow \mathbb{R}$.
It is measurable in the sense that the \emph{preimage} of any Borel set $B \in \mathcal{B}$
is an event: $X^{-1}(B) = \{ \omega \in \Omega \;|\; X(\omega) \in B \} \in \mathcal{F}$.

A random variable $X$ generates a $\sigma$-algebra that comprises of the preimage
of all Borel sets. It is denoted by $\sigma(X)$
and formally defined as $\sigma(X) = \{ X^{-1}(B) \;|\; B \in \mathcal{B} \}$.

\bigskip

Random variables are typically categorized as discrete or continuous.
$X$ is \emph{discrete} when it maps $\Omega$ to a discrete set.
In that case, its \emph{probability mass function} is defined as $\probability[X = x]$
for some $x$ in its range.
A \emph{continuous} random variable is often associated with a
probability \emph{density} function, $f_X$, such that:
\begin{equation*}
    \probability[a \leq X \leq b] = \int_a^b f_X(x) dx.
\end{equation*}

Consider, for instance, the following probability density function over the real line for
parameters $\mu \in \mathbb{R}$ and $\sigma > 0$:
\begin{equation*}
    f(x) = \frac{1}{\sqrt{2 \pi \sigma^2}} e^{- \frac{(x - \mu)^2}{2\sigma^2}}.
\end{equation*}
A random variable with the density function above is said to follow a Gaussian
distribution with mean $\mu$ and variance $\sigma^2$, denoted by $X \sim \mathcal{N}(\mu, \sigma^2)$.
When $\mu = 0$ and $\sigma^2 = 1$, the resulting distribution is called the standard
Normal distribution.

Gaussian random variables have attractive properties.
For example, the sum of two independent Gaussian random variables is itself a Gaussian variable.
Concretely, $X_1 \sim \mathcal{N}(\mu_1, \sigma_1^2)$ and $X_2 \sim \mathcal{N}(\mu_2, \sigma_2^2)$,
then $X_1 + X_2 \sim \mathcal{N}(\mu_1 + \mu_2, \sigma_1^2 + \sigma_2^2)$.
The sum of the squares of $m$ independent Gaussian random variables, on the other hand,
follows a $\chi^2$-distribution with $m$ degrees of freedom.

\section{Conditional Probability}
Conditional probabilities give us a way to model how the probability of an event changes
in the presence of extra information, such as partial knowledge about a random outcome.
Concretely, if $(\Omega, \mathcal{F}, \probability)$ is a probability space and
$A, B \in \mathcal{F}$ such that $\probability[B] > 0$, then the \emph{conditional
probability} of $A$ given the event $B$ is denoted by $\probability[A \;\lvert\; B]$ and
defined as follows:
\begin{equation*}
    \probability[A \;\lvert\; B] = \frac{\probability[A \cap B]}{\probability[B]}.
\end{equation*}

We use a number of helpful results concerning conditional probabilities
in proofs throughout the monograph. One particularly useful inequality
is what is known as the \emph{union bound} and is stated as follows:
\begin{equation*}
    \probability[\cup_i A_i] \leq \sum_i \probability[A_i].
\end{equation*}

Another fundamental property is the law of total probability.
It states that, for mutually disjoint events $A_i$'s such that
$\Omega = \cup A_i$, the probability of any event $B$ can be expanded
as follows:
\begin{equation*}
    \probability[B] = \sum_i \probability[B \;\lvert\; A_i] \probability[A_i].
\end{equation*}
This is easy to verify: the summand is by definition equal to $\probability[B \cap A_i]$
and, considering the events $(B \cap A_i)$'s are mutually disjoint, their sum
is equal to $\probability[B \cap (\cup A_i)] = \probability[B]$.


\section{Independence}
Another tool that reflects the effect (or lack thereof) of additional knowledge on probabilities
is the concept of \emph{independence}. Two events $A$ and $B$ are said to be
\emph{independent} if $\probability[A \cap B] = \probability[A] \times \probability[B]$.
Equivalently, we say that $A$ is independent of $B$ if and only if
$\probability[A \;\lvert\; B] = \probability[A]$ when $\probability[B] > 0$.

\bigskip

Independence between two random variables is defined similarly but requires a bit more care.
If $X$ and $Y$ are two random variables and $\sigma(X)$ and $\sigma(Y)$ denote
the $\sigma$-algebras generated by them, then $X$ is independent of $Y$ if
all events $A \in \sigma(X)$ and $B \in \sigma(Y)$ are independent.

When a sequence of random variables are \emph{mutually} independent and are drawn
from the same distribution (i.e., have the same probability density function),
we say the random variables are drawn \emph{iid}: independent and identically-distributed.
We stress that \emph{mutual} independence is a stronger restriction than
\emph{pairwise} independence: $m$ events $\{ E_i \}_{i=1}^m$ are mutually independent if
$\probability[\cap_i E_i] = \prod_i \probability[E_i]$.

We typically assume that data and query points are drawn \emph{iid} from some
(unknown) distribution. This is a standard and often necessary assumption
that eases analysis.

\section{Expectation and Variance}

The \emph{expected value} of a discrete random variable $X$ is denoted by $\ev[X]$
and defined as follows:
\begin{equation*}
    \ev[X] = \sum_x x \probability[X = x].
\end{equation*}
When $X$ is continuous, its expected value is based on the following Lebesgue integral:
\begin{equation*}
    \ev[X] = \int_{\Omega} X d \probability.
\end{equation*}
So when a random variable has probability density function $f_X$, its expected value
becomes:
\begin{equation*}
    \ev[X] = \int x f_X(x) dx.
\end{equation*}

For a \emph{nonnegative} random variable $X$, it is sometimes more convenient to
unpack $\ev{X}$ as follows instead:
\begin{equation*}
    \ev[X] = \int_0^\infty \probability[X > x] dx.
\end{equation*}

A fundamental property of expectation is that it is a linear operator.
Formally, $\ev[X + Y] = \ev[X] + \ev[Y]$ for two random variables $X$ and $Y$.
We use this property often in proofs.

We state another important property for independent random variables
that is easy to prove.
If $X$ and $Y$ are independent, then $\ev[XY] = \ev[X]\ev[Y]$.

\bigskip

The \emph{variance} of a random variable is defined as follows:
\begin{equation*}
    \var[X] = \ev\Big[ (X - \ev[X])^2 \Big] = \ev[X]^2 - \ev[X^2].
\end{equation*}
Unlike expectation, variance is not linear unless the random variables involved
are independent. It is also easy to see that $\var[aX] = a^2 \var[X]$ for a
constant $a$.

\section{Central Limit Theorem}
The result known as the Central Limit Theorem is one of the most
useful tools in probability. Informally, it states that the average of \emph{iid}
random variables with finite mean and variance converges to a Gaussian distribution.
There are several variants of this result that extend the claim to, for example,
independent but not identically distributed variables. Below we repeat the formal
result for the \emph{iid} case.

\begin{theorem}
    Let $X_i$'s be a sequence of $n$ \emph{iid} random variables with finite mean $\mu$
    and variance $\sigma^2$. Then, for any $x \in \mathbb{R}$:
    \begin{equation*}
        \lim_{n \rightarrow \infty} \probability \Big[
            \underbrace{\frac{(1/n \sum_{i=1}^n X_i) - \mu}{\sigma^2/n}}_Z \leq x
        \Big] = \int_{-\infty}^x \frac{1}{\sqrt{2 \pi}} e^{-\frac{t^2}{2}} dt,
    \end{equation*}
    implying that $Z \sim \mathcal{N}(0, 1)$.
\end{theorem}

\chapter{Concentration of Measure}
\label{appendix:measure}

\abstract{
By the strong law of large numbers, we know that the average of a sequence
of $m$ \emph{iid} random variables with mean $\mu$ converges to $\mu$ with
probability $1$ as $m$ tends to infinity. But how far is that average from
$\mu$ when $m$ is finite? Concentration inequalities helps us answer that question
quantitatively. This appendix reviews important inequalities that are used
in the proofs and arguments throughout this monograph.
}

\section{Markov's Inequality}

\begin{lemma}
    \label{lemma:appendix:concentration:markov}
    For a nonnegative random variable $X$ and a nonnegative constant $a \geq 0$:
    \begin{equation*}
        \probability[X \geq a] \leq \frac{\ev[X]}{a}.
    \end{equation*}
\end{lemma}
\begin{proof}
    Recall that the expectation of a nonnegative random variable $X$ can be written
    as:
    \begin{equation*}
        \ev[X] = \int_0^\infty \probability[X \geq x] dx.
    \end{equation*}
    Because $\probability[X \geq x]$ is monotonically nonincreasing, we can expand
    the above as follows to complete the proof:
    \begin{equation*}
        \ev[X] \geq \int_0^a \probability[X \geq x] dx \geq \int_0^a \probability[X \geq a] dx = a \probability[X \geq a].
    \end{equation*}
\end{proof}

\section{Chebyshev's Inequality}

\begin{lemma}
    \label{lemma:appendix:concentration:chebyshev}
    For a random variable $X$ and a constant $a > 0$:
    \begin{equation*}
        \probability \Big[ \big\lvert X - \ev[X] \big\rvert \geq a \Big] \leq \frac{\var[X]}{a^2}.
    \end{equation*}
\end{lemma}
\begin{proof}
    \begin{equation*}
        \probability \Big[ \big\lvert X - \ev[X] \big\rvert \geq a \Big] =
        \probability \Big[ \big( X - \ev[X] \big)^2 \geq a^2 \Big] \leq \frac{\var[X]}{a^2},
    \end{equation*}
    where the last step follows by the application of Markov's inequality.
\end{proof}

\begin{lemma}
    Let $\{ X_i \}_{i=1}^n$ be a sequence of iid random variables
    with mean $\mu < \infty$ and variance $\sigma^2 < \infty$. For $\delta \in (0, 1)$,
    with probability $1 - \delta$:
    \begin{equation*}
        \Big\lvert \frac{1}{n} \sum_{i = 1}^n X_i - \mu \Big\rvert \leq \sqrt{\frac{\sigma^2}{\delta n}}.
    \end{equation*}
\end{lemma}
\begin{proof}
    By Lemma~\ref{lemma:appendix:concentration:chebyshev}, for any $a > 0$:
    \begin{equation*}
        \probability \Bigg[ \Big\lvert \frac{1}{n}\sum_{i=1}^n X_i - \mu \Big\rvert \geq a \Bigg]
        \leq \frac{\sigma^2/n}{a^2}.
    \end{equation*}
    Setting the right-hand-side to $\delta$, we obtain:
    \begin{equation*}
        \frac{\sigma^2}{n a^2} = \delta \implies a = \sqrt{\frac{\sigma^2}{\delta n}},
    \end{equation*}
    which completes the proof.
\end{proof}

\section{Chernoff Bounds}

\begin{lemma}
    Let $\{ X_i \}_{i=1}^n$ be independent Bernoulli variables with success probability $p_i$.
    Define $X = \sum_i X_i$ and $\mu = \ev[X] = \sum_i p_i$. Then:
    \begin{equation*}
        \probability \Big[ X > (1 + \delta) \mu \Big] \leq e^{-h(\delta) \mu},
    \end{equation*}
    where,
    \begin{equation*}
        h(t) = (1 + t) \log(1 + t) - t.
    \end{equation*}
\end{lemma}
\begin{proof}
    Using Markov's inequality of Lemma~\ref{lemma:appendix:concentration:markov}
    we can write the following for any $t > 0$:
    \begin{equation*}
        \probability\Big[ X > (1 + \delta)\mu \Big] =
            \probability\Big[ e^{tX} > e^{t(1 + \delta)\mu} \Big] \leq
            \frac{\ev\big[ e^{tX} \big]}{e^{t (1 + \delta) \mu}}.
    \end{equation*}
    Expanding the expectation, we obtain:
    \begin{align*}
        \ev\big[e^{tX}\big] &= \ev\Big[ e^{t \sum_i X_i} \Big] = \ev\Big[ \prod_i e^{tX_i} \Big]
        = \prod_i \ev[e^{tX_i}] \\
        &= \prod_i \Big( p_i e^t + (1 - p_i) \Big) \\
        &= \prod_i \big( 1 + p_i (e^t - 1) \big) \\
        &\leq \prod_i e^{p_i(e^t - 1)} = e^{(e^t - 1)\mu}. && \text{by $(1 + t \leq e^t)$} \\
    \end{align*}
    Putting all this together gives us:
    \begin{equation}
        \label{equation:appendix:concentration:chernoff:proof}
        \probability\Big[ X > (1 + \delta)\mu \Big] \leq 
        \frac{e^{(e^t - 1) \mu}}{e^{t (1 + \delta) \mu}}.
    \end{equation}
    This bound holds for any value $t > 0$, and in particular a value of $t$ that
    minimizes the right-hand-side. To find such a $t$, we may differentiate
    the right-hand-side, set it to $0$, and solve for $t$ to obtain:
    \begin{align*}
        \frac{\mu e^t e^{(e^t - 1) \mu}}{e^{t (1 + \delta) \mu}} &-
        \mu ( 1 + \delta ) \frac{e^{(e^t - 1) \mu}}{e^{t (1 + \delta) \mu}} = 0 \\
        &\implies \mu e^t = \mu (1 + \delta) \\
        &\implies t = \log(1 + \delta).
    \end{align*}
    Substituting $t$ into Equation~(\ref{equation:appendix:concentration:chernoff:proof})
    gives the desired result.
\end{proof}

\section{Hoeffding's Inequality}

We need the following result, known as Hoeffding's Lemma, to present
Hoeffding's inequality.

\begin{lemma}
    \label{lemma:appendix:concentration:hoeffding-lemma}
    Let $X$ be a zero-mean random variable that takes values in $[a, b]$.
    For any $t > 0$:
    \begin{equation*}
        \ev\big[ e^{tX} \big] \leq \exp\Big( \frac{t^2 (b - a)^2}{8} \Big).
    \end{equation*}
\end{lemma}
\begin{proof}
    By convexity of $e^{tx}$ and given $x \in [a, b]$ we have that:
    \begin{equation*}
        e^{tx} \leq \frac{b - x}{b - a} e^{ta} +
            \frac{x - a}{b - a} e^{tb}.
    \end{equation*}
    Taking the expectation of both sides, we arrive at:
    \begin{equation*}
        \ev\Big[e^{tx}\Big] \leq
            \frac{b}{b - a} e^{ta} - \frac{a}{b - a} e^{tb}.
    \end{equation*}
    To conclude the proof, we first write the right-hand-side as
    $\exp(h(t(b - a)))$ where:
    \begin{equation*}
        h(x) = \frac{a}{b - a} x + \log \Big( \frac{b}{b - a} - \frac{a}{b - a} e^x \Big).
    \end{equation*}
    By expanding $h(x)$ using Taylor's theorem, it can be shown that
    $h(x) \leq x^2/8$. That completes the proof.
\end{proof}

We are ready to present Hoeffding's inequality.

\begin{lemma}
    Let $\{ X_i \}_{i=1}^n$ be a sequence of iid random variables
    with finite mean $\mu$ and suppose $X_i \in [a, b]$ almost surely.
    For all $\epsilon > 0$:
    \begin{equation*}
        \probability\Bigg[ \Big\lvert \frac{1}{n} \sum_{i=1}^n X_i - \mu \Big\rvert > \epsilon \Bigg] \leq 2 \exp\Big({-\frac{2n \epsilon^2}{(b - a)^2}}\Big).
    \end{equation*}
\end{lemma}
\begin{proof}
    Let $X = 1/n \sum_i X_i - \mu$. Observe by Markov's inequality that:
    \begin{equation*}
        \probability[X \geq \epsilon] = \probability\Big[ e^{tX} \geq e^{t\epsilon} \Big]
        \leq e^{-t\epsilon} \ev[e^{tX}].
    \end{equation*}
    By independence of $X_i$'s and
    the application of Lemma~\ref{lemma:appendix:concentration:hoeffding-lemma}:
    \begin{align*}
        \ev[e^{tX}] &= \ev \Bigg[ \prod_i e^\frac{t(X_i - \mu)}{n} \Bigg] \\
        &= \prod_i \ev \Big[ e^{\frac{t(X_i-\mu)}{n}} \Big] \\
        &\leq \prod_i \exp\Big( \frac{t^2 (b - a)^2}{8 n^2} \Big) \\
        &= \exp\Big( \frac{t^2 (b - a)^2}{8 n} \Big).
    \end{align*}
    We have shown that:
    \begin{equation*}
        \probability[X \geq \epsilon] \leq \exp\Big( -t \epsilon + \frac{t^2 (b - a)^2}{8 n} \Big).
    \end{equation*}
    That statement holds for all values of $t$ and in particular one that minimizes
    the right-hand-side. Solving for that value of $t$ gives us
    $t = 4n\epsilon / (b - a^2)$, which implies:
    \begin{equation*}
        \probability[X \geq \epsilon] \leq e^{-\frac{2n \epsilon^2}{(b - a)^2}}.
    \end{equation*}
    By a symmetric argument we can bound $\probability[X \leq -\epsilon]$. The claim
    follows by the union bound over the two cases.
\end{proof}

\section{Bennet's Inequality}

\begin{lemma}
    Let $\{ X_i \}_{i=1}^n$ be a sequence of independent random variables with zero mean
    and finite variance $\sigma_i^2$. Assume that $\lvert X_i \rvert \leq a$ almost surely for all $i$. Then:
    \begin{equation*}
        \probability\Big[\sum_i X_i \geq t \Big] \leq 
        \exp \Bigg( -\frac{\sigma^2}{a^2} h\Big( \frac{a t}{\sigma^2} \Big) \Bigg),
    \end{equation*}
    where $h(x) = (1 + x) \log(1 + x) - x$ and $\sigma^2 = \sum_i \sigma_i^2$.
\end{lemma}
\begin{proof}
    As usual, we take advantage of Markov's inequality to write:
    \begin{align*}
        \probability\Big[\sum_i X_i \geq t \Big] &\leq
            e^{-\lambda t} \ev \Big[ e^{\lambda \sum_i X_i} \Big] \\
        &= e^{-\lambda t} \ev \Big[ \prod_i e^{\lambda X_i} \Big] \\
        &= e^{-\lambda t} \prod_i \ev \Big[ e^{\lambda X_i} \Big] \\
    \end{align*}
    Using the Taylor expansion of $e^x$, we obtain:
    \begin{align*}
        \ev \Big[ e^{\lambda X_i} \Big] &= \ev \Big[ \sum_{k=0}^\infty \frac{\lambda^k X_i^k}{k!} \Big] \\
        &= 1 + \sum_{k=2}^\infty \frac{\lambda^k \ev[X_i^2 X_i^{k - 2}]}{k!} \\
        &\leq 1 + \sum_{k=2}^\infty \frac{\lambda^k \sigma_i^2 a^{k-2}}{k!} \\
        &= 1 + \frac{\sigma_i^2}{a^2} \sum_{k=2}^\infty \frac{\lambda^k a^k}{k!} \\
        &= 1 + \frac{\sigma_i^2}{a^2} \big( e^{\lambda a} - 1 - \lambda a \big) \\
        &\leq \exp\Big( \frac{\sigma_i^2}{a^2} \big( e^{\lambda a} - 1 - \lambda a \big) \Big).
    \end{align*}
    Putting it all together:
    \begin{align*}
        \probability\Big[\sum_i X_i \geq t \Big] &\leq
            e^{-\lambda t} \prod_i \exp\Big( \frac{\sigma_i^2}{a^2} \big( e^{\lambda a} - 1 - \lambda a \big) \Big) \\
        &= e^{-\lambda t} \exp\Big( \frac{\sigma^2}{a^2} \big( e^{\lambda a} - 1 - \lambda a \big) \Big).
    \end{align*}
    This inequality holds for all values of $\lambda$, and in particular one that minimizes the
    right-hand-side. Setting the derivative of the right-hand-side to $0$ and solving for $\lambda$
    leads to the desired result.
\end{proof}

\chapter{Linear Algebra Review}
\label{appendix:linear-algebra}

\abstract{
This appendix reviews basic concepts from Linear Algebra that are useful
in digesting the material in this monograph.
}

\section{Inner Product}

Denote by $\mathbb{H}$ a vector space.
An inner product $\langle \cdot, \cdot \rangle: \mathbb{H} \times \mathbb{H} \rightarrow \mathbb{R}$
is a function with the following properties:
\begin{itemize}
    \item $\forall \; u \in \mathbb{H},\; \langle u, u \rangle \geq 0$;
    \item $\forall \; u \in \mathbb{H},\; \langle u, u \rangle = 0 \Leftrightarrow u = 0$;
    \item $\forall \; u, v \in \mathbb{H},\; \langle u, v \rangle = \langle v, u \rangle$; and,
    \item $\forall \; u, v, w \in \mathbb{H}, \textit{ and } \alpha, \beta \in \mathbb{R},\; 
    \langle \alpha u + \beta v, w \rangle = \alpha \langle u, w \rangle + \beta \langle v, w \rangle$.
\end{itemize}

We call $\mathbb{H}$ together with the inner product $\langle \cdot, \cdot \rangle$
an \emph{inner product space}.
As an example, when $\mathbb{H} = \mathbb{R}^d$, given two vectors
$u = \sum_{i=1}^d u_i e_i$ and $v = \sum_{i=1}^d v_i e_i$, where $e_i$'s
are the standard basis vectors, the following is an inner product:
\begin{equation*}
    \langle u, v \rangle = \sum_{i = 1}^d u_i v_i.
\end{equation*}

We say two vectors $u$ and $v$ in an inner product space are \emph{orthogonal}
if their inner product is $0$: $\langle u, v \rangle = 0$.

\section{Norms}

A function $\Phi: \mathbb{H} \rightarrow \mathbb{R}_+$ is a norm on
$\mathbb{H}$ if it has the following properties:
\begin{itemize}
    \item Definiteness: For all $u \in \mathbb{H}$, $\Phi(u) = 0 \Leftrightarrow u = 0$;
    \item Homogeneity: For all $u \in \mathbb{H}$ and $\alpha \in \mathbb{R}$,
        $\Phi(\alpha u) = \lvert \alpha \rvert \Phi(u)$; and,
    \item Triangle inequality: $\forall \; u, v \in \mathbb{H}, \; \Phi(u + v) \leq \Phi(u) + \Phi(v)$.
\end{itemize}

Examples include the absolute value on $\mathbb{R}$,
and the $L_p$ norm (for $p \geq 1$) on $\mathbb{R}^d$ denoted by $\lVert \cdot \rVert_p$
and defined as:
\begin{equation*}
    \lVert u \rVert_p = \Big( \sum_{i=1}^d \lvert u_i \rvert^p \Big)^{\frac{1}{p}}.
\end{equation*}
Instances of $L_p$ include the commonly used $L_1$, $L_2$ (Euclidean),
and $L_\infty$ norms, where $\lVert u \rVert_\infty = \max_i \lvert u_i \rvert$.

Note that, when $\mathbb{H}$ is an inner product space, then
the function $\lVert u \rVert = \sqrt{\langle u, u \rangle}$ is a norm.

\section{Distance}
A norm on a vector space induces a notion of distance between two vectors.
Concretely, if $\mathbb{H}$ is a normed space equipped with $\lVert \cdot \rVert$,
then we define the distance between two vectors $u, v \in \mathbb{H}$ as follows:
\begin{equation*}
    \delta(u, v) = \lVert u - v \rVert.
\end{equation*}

\section{Orthogonal Projection}

\begin{lemma}
    Let $\mathbb{H}$ be an inner product space and suppose $u \in \mathbb{H}$ and $u \neq 0$.
    Any vector $v \in \mathbb{H}$ can be uniquely decomposed along $u$ as:
    \begin{equation*}
        v = v_{\perp} + v_{\parallel},
    \end{equation*}
    such that $\langle v_\perp, v_\parallel \rangle = 0$. Additionally:
    \begin{equation*}
        v_\parallel = \frac{\langle u, v \rangle}{\langle u, u \rangle} u,
    \end{equation*}
    and $v_\perp = v - v_\parallel$.
\end{lemma}
\begin{proof}
    Let $v_\parallel = \alpha u$ and $v_\perp = v - v_\parallel$.
    Because $v_\parallel$ and $v_\perp$ are orthogonal, we deduce that:
    \begin{align*}
        \langle v_\parallel, v_\perp \rangle = 0 \implies
            \langle \alpha u, v_\perp \rangle = 0 \implies
            \langle u, v_\perp \rangle = 0.
    \end{align*}
    That implies:
    \begin{align*}
        \langle v, u \rangle = \alpha \langle u, u \rangle \implies
        \alpha = \frac{\langle u, v \rangle}{\langle u, u \rangle},
    \end{align*}
    so that:
    \begin{equation*}
        v_\parallel = \frac{\langle u, v \rangle}{\langle u, u \rangle} u.
    \end{equation*}

    We prove the uniqueness of the decomposition by contradiction.
    Suppose there exists another decomposition of $v$ to $v_\parallel^\prime + v_\perp^\prime$.
    Then:
    \begin{align*}
        v_\parallel + v_\perp = v_\parallel^\prime + v_\perp^\prime &\implies
        \langle u, v_\parallel + v_\perp \rangle = \langle u,  v_\parallel^\prime + v_\perp^\prime\rangle \\
        &\implies \langle u, v_\parallel \rangle = \langle u,  v_\parallel^\prime \rangle \\
        &\implies \langle u, \alpha u \rangle = \langle u, \beta u \rangle \\
        &\implies \alpha = \beta.
    \end{align*}
    We must therefore also have that $v_\perp = v_\perp^\prime$.
\end{proof}


\end{document}