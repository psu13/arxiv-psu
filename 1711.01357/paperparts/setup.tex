

\section{Image and Data Models}
\label{sec:setup}

Before discussing how to reconstruct images from the data products discussed in Section~\ref{sec:meas}, it is first important to show how to generate samples of these data products from a known emission image.
%underlying image %or video 
%of the emission. 

%Constructing a generative model of the samples is important in 

\subsection{Image}

An image of instantaneous emission, $I(\xpos,\ypos)$, is defined over the continuous space of angular coordinates $\xpos$ and $\ypos$. However, to make computation tractable,
we represent this continuous image as a square $M \times M$ array of values that represent the emission image's flux over a specified field of view (FOV). 

Following the image parameterization convention of BLAH, we represent the continuous image, $I$, as the sum of a discrete number of scaled and shifted triangle pulses of height $\im$ centered around $(\xpos, \ypos) \in L \times L$, where

\begin{align}
L &=  \left \{ k \Delta + \frac{\Delta}{2}  -\frac{FOV}{2} \right \} \hspace{0.26in} \mbox{for} \hspace{0.1 in} k = 0,...,M-1
\label{eq:discrete1}
\end{align} 


\noindent{for $\Delta = \frac{FOV}{\npix}$. Using this representation, we are able to describe the 2D continuous image as a vector, $\im$, of length $M^2$. When applicable, we denote a single image in a sequence of images as $\im_t$, where $t$ indexes the time of the instantaneous image.}

%By representing $I(\xpos, \ypos)$ as its vectorized coefficients, $x$, we are able to represent a continuous image using a discrete number of terms. 

\subsection{Data Products}
\label{sec:dataproducts}

In order to evaluate how consistent a proposed image, $\im$, is with the measured VLBI data, $\meas$, we construct a function $f(\im)$ that returns the values we would expect to see in $\meas$ if $\im$ were the true underlying image, and no additional noise was added. This measurement function, $f(\im)$, is composed of a set of sub-functions, $g_k(\im)$, that each simulate an ideally observed data product: 

\begin{align}
f(\im) = \Spvek{g_1(\im); g_2(\im); ...; g_{\nmeas}(\im)} : \Re^{ M^2 \times 1} \rightarrow \Re^{ \nmeas \times 1 }
\end{align}

Each $g_k(\im)$ produces a real-value corresponding to element $k$ of the vector $\meas$. For example, $g_k(\im)$ may return the imaginary portion of $\im$'s spatial frequency, corresponding to the visibility obtained between a pair of telescopes.

Depending on the data products selected for $\meas$, the measurement function, $f(\im)$, may be a linear or non-linear function of $\im$. Although this choice does not affect the quality of the values produced by $f(\im)$, it does affect the optimization process during inference (refer to Section~\ref{sec:static}). % and~\ref{sec:dynamic_inference}). %Thus, we list the cases when the measurement function can be expressed linearly or must require a non-linear function.  
Below we list the cases when a data product can be expressed linearly or requires a non-linear function. In the supplemental material we provide the exact measurement function used for each data product and their derivatives. 

\vspace{.1in}
\subsubsection{Linear Measurement Functions}
\label{sec:dataproducts_lin}
%The question remains how to define each $g(x)$. 
Since each measured complex visibility is approximated as the Fourier transform of $I(\xpos,\ypos)$, we can extract an image's ideal visibilities by performing a linear matrix operation similar to a Discrete Time Fourier Transform (DTFT): $f(\im) = \FTmtx \im$.
%When complex visibilities can be used for image reconstruction, $f(\im)$ is expressed simply as a linear matrix operation: $f(\im) = \FTmtx \im$. 
In particular, BLAH shows that by plugging the parameterized image, $\im$, into the van Cittert-Zernike theorem (Equation~\ref{eq:vcz}), we can express a single visibility as $\im$ multiplied by a complex row-vector $\vecFTmtx(u,v)$ of size $1 \times \npix^2$.  $\FTmtx$ is then constructed by stacking the real and imaginary components of these row vectors for each of sampled $(u,v)$ point into a matrix of size $\nmeas \times \npix^2$. 
%Additionally, blurring via convolution to simulate interstellar scattering is also a linear operation and can be represented in an 


%\begin{align}
%f(\im) = \FTmtx \im = \Spvek{\Re \left[ \vecFTmtx(u,v) \right]; \Im \left[ \vecFTmtx(u,v) \right]} \im 
%\end{align}

\vspace{.1in}
\subsubsection{Non-Linear Measurement Functions}
In the case of atmospheric noise, uniformly random phase errors are included in each complex visibility measurements. 
%However, these phase errors are incorporated in a way that preserves closure phase (refer to Section BLAH). 
In order to handle this additional phase error, without having to explicitly model the errors as latent variables, we populate $\meas$ with measured data products that are invariant to atmospheric inhomogeneity. 
The bispectrum, closure phase, and visibility amplitude data products are invariant to this error, but come at the cost of requiring a non-linear measurement function, $f(\im)$ to \katie{BLAH BLAH express them(?).} 



