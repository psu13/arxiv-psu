We start this part of the review with a brief overview of the steps by
which event generators build up the structure of a hadron-hadron
collision involving a hard process of interest -- that is, a process
in which heavy objects are created or a large momentum transfer
occurs. As outlined already in \SecRef{sec:general-introduction},
there are several
basic phases of the process that need to be simulated: a primary hard
subprocess, parton showers associated with the incoming and outgoing
coloured participants in the subprocess, non-perturbative interactions
that convert the showers into outgoing hadrons and connect them to the
incoming beam hadrons, secondary interactions that give rise to the
underlying event, and the decays of unstable particles that do not
escape from the detector. There are corresponding steps in the event generation.

Of course, not all these
steps are relevant in all processes.  In particular, the majority of
events that make up the total hadron-hadron cross section are of
soft QCD type and rely more on phenomenological models.  At the
other extreme the simulation of new-physics events such as
supersymmetric particle production and decay, and the SM backgrounds to
them, rely on essentially all of the components.

We also briefly introduce two issues that affect all areas of the
simulation: the jet structure of the final state and a widely used
approximation to QCD~-- the large-$\Nc$ limit.

In most applications of event generators, one is interested in events of
a particular type.  Rather than simulating typical events and waiting
for one of them to be of the required type, which can be as rare as
1~in~$10^{15}$ in some applications, the simulation is built around the
hard subprocess.  The user selects hard subprocesses of given types and
partonic events are generated according to their matrix elements
and phase space, as described in \SecRef{sec:subprocesses} and in more
detail in \AppRef{sec:app_mcs}.  These are typically of LO for the given
process selected (which could be relatively high order in the QCD
coupling, for example for $Z$+4~partons) and calculated with the
tree-level matrix elements.  There has however been important progress
in including loop corrections into hard process generation, as described
in \SecRef{sec:me-nlo-matching}.

Since the particles entering the hard subprocess, and some of those
leaving it, are typically
QCD partons, they can radiate gluons.  These gluons can radiate
others, and also produce
quark--antiquark pairs, generating showers of outgoing partons.
This process is simulated with a step-wise Markov
chain, choosing probabilistically to add one more parton to the final
state at a time, called a parton shower algorithm, described in
\SecRef{sec:parton-showers}.  It is formulated as an evolution in some
momentum-transfer-like variable downwards from a scale defined by the
hard process, and as both a forwards evolution of the outgoing partons
and a backwards evolution of the incoming partons progressively towards
the incoming hadrons.

The incoming hadrons are complex bound states of strongly-interacting
partons and it is possible that, in a given hadron-hadron collision,
more than one pair of partons may interact with each other.  These
multiple interactions go on to produce additional partons throughout the
event, which may contribute to any observable, in addition to those from
the hard process and associated parton showers that we are primarily
interested in.  We therefore describe this part of the event structure
as the underlying event.  As described in
\SecRef{sec:minim-bias-underly}, it can also be formulated as a downward
evolution in a momentum-transfer-like variable.

As the event is evolved downwards in momentum scales it ultimately
reaches the region, at scales of order 1~GeV, in which QCD becomes
strongly interacting and perturbation theory breaks down.  Therefore at
this scale the perturbative evolution must be terminated and replaced by
a non-perturbative hadronization model that describes the confinement of the system of
coloured partons into colourless hadrons. A key feature of these
models, described in \SecRef{sec:hadronization},  is that individual
partons do not hadronize independently, but rather colour-connected
systems of partons hadronize collectively.  These models are not
derived directly from QCD and consequently have more free parameters
than the preceding components.  However, to a good approximation
they are universal~-- the hadronization of a given coloured system is
independent of how that system was produced, so that once tuned on one
data set the models are predictive for new collision types or energies.

Finally, many of the hadrons that are produced during hadronization are
unstable resonances.  Sophisticated models are used to simulate their
decay to the lighter hadrons that are long-lived enough to be considered
stable on the time-scales of particle physics detectors,
\SecRef{sec:hadron-decays}.  Since many of the particles involved with
all stages of the simulation are charged, QED radiation effects can also
be inserted into the event chain at various stages,
\SecRef{sec:qed-radiation}.

\mcsubsection{Jets and jet algorithms}

The final states of many subprocesses of interest include hard partons.
Radiation from the incoming partons is a source of additional partons in
the final state.  The parton shower evolution is dominated by the
emission of additional partons that are either collinear with the
outgoing partons or are soft.  The final state of the parton shower
therefore predominantly has a structure in which most of the energy is
carried by localized collinear bundles of partons, called jets.  The
hadronization mechanism is such that this jet structure is preserved and
it is experimentally observed that the final state of
high-momentum-transfer hadronic events is
dominated by jets of hadrons.  The distributions of the total momentum
of hadrons in jets are approximately described by perturbative
calculations of partons with the same total momentum.

Although jets are a prominant feature of hadronic events, they are not
fundamental objects that are defined by the theory.  In order to
classify the jet final state of a collision, define which hadrons belong
to which jet and reconstruct their total momentum, we need a precise
algorithmic jet definition, or jet algorithm.  There has been much
progress on the properties that such algorithms must satisfy in order to
be convenient theoretically and experimentally.  We are not able to
review this work here (for a recent thorough review, see
\cite{Salam:2009jx} for example), but we mention one important property
that we require of a jet algorithm.  One of the applications we will use
them for is the matching of perturbative calculations at different
orders and with different jet structures and in order for this to be
well-defined we must use an algorithm for which jet cross sections can
be calculated on the parton level to arbitrarily high order of
perturbation theory.  This is only true of jet algorithms that are
\emph{collinear and infrared safe}.  That is, for
any partonic configuration, replacing any parton with a collinear set of
partons with the same total momentum, or adding any number of infinitely
soft partons in any directions, should produce the identical
result.  One can show that, provided this property is satisfied, jet
cross sections are finite at any perturbative order and have
non-perturbative corrections that are suppressed by powers of the jet
momenta, so that at high momentum transfers the jet structure of the
hadronic final state of
a collision is very well described by a parton-level calculation.

\mcsubsection{The large-$\Nc$ limit}
\label{sec:large-nc-limit}

It is of course well established that QCD is an SU(3) gauge theory.
Nevertheless it is frequently useful to consider the generalization to
a theory with $\Nc$ colours, SU(\Nc). 
  We will see that various aspects of event simulation
simplify in the limit of large \Nc.  For any \Nc, one can combine a
fundamental colour with a fundamental anticolour to produce an adjoint
colour and a colour singlet, $\Nc\otimes\bar{\Nc}=(\Nc^2-1)\oplus1$.
Conversely, we can think of the colour of a gluon as being that of a
quark and an antiquark, up to corrections from the fact that the gluon
does not have a singlet component.  One can decompose the colour
structure of each of the Feynman rules, and hence of any Feynman diagram,
into a set of delta-functions between external fundamental colours.  We
call this the colour flow of the diagram.  In the limit of large \Nc,
only diagrams whose colour flow is planar, \ie for which the fundamental
colour connections can be drawn in a single plane, contribute.  Each
colour connection that needs to come out of the plane results in a
suppression of $1/\Nc^2$.  This connection between the topology of a
diagram and its colour flow is an extremely powerful organizing
principle, which we will see comes into several different aspects of
event modelling.  One should bear in mind that whenever we use the
large-$\Nc$ limit, corrections to it are expected to be suppressed by at
least $1/\Nc^2\sim10$\% and in practice, because of the connection with
the topology, are often further dynamically suppressed.

% Local Variables:
% mode: LaTeX
% TeX-master: "../mcreview"
% End:
