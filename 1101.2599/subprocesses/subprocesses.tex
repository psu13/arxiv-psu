Many LHC processes of interest involve large momentum transfers, for
example to produce heavy particles or jets with high transverse momenta.
Thus the simulation of subprocesses with large invariant momentum
transfer is at the core of any simulation of collider events in
contemporary experiments through Monte Carlo event generators. As QCD
quanta are asymptotically free, such reactions can be described by
perturbation theory, thus making it possible to compute many features
of the subprocess in question by, for example, using Feynman diagrams.

%%%%%%%%%%%%%%%%%%%%%%%%%%%%%%%%%%%%%%%%%%%%%%%%%%%%%%%%%%%%%%%%%%%%%%
\mcsubsection{Factorization formula for QCD cross sections}
%%%%%%%%%%%%%%%%%%%%%%%%%%%%%%%%%%%%%%%%%%%%%%%%%%%%%%%%%%%%%%%%%%%%%%
Cross sections for a scattering subprocess $ab\to n$ at hadron colliders 
can be computed in collinear factorization through~\cite{Ellis:1991qj}
\begin{eqnarray}
\label{Eq::Master_For_XSec}
\sigma%_{h_1h_2\to n} 
&=&
\sum\limits_{a,b}\,\int\limits_{0}^{1}\done x_a\done x_b\,\int\,
\pdf[h_1]{a}(x_a,\mu_F)\pdf[h_2]{b}(x_b,\mu_F)\,
\done\hat\sigma_{ab\to n}(\mu_F,\mu_R)\\
&=&
\sum\limits_{a,b}\,\int\limits_{0}^{1}\done x_a\done x_b\,\int\done\Phi_n\,
\pdf[h_1]{a}(x_a,\mu_F)\pdf[h_2]{b}(x_b,\mu_F)\nonumber\\
&&\qquad\times\frac{1}{2\hat s}|{\cal M}_{ab\to n}|^2(\Phi_n;\mu_F,\mu_R)\,,
\nonumber
\end{eqnarray}
where
\begin{itemize}
\item $\pdf[h]{a}(x,\mu)$ are the parton distribution functions
      (PDFs), which depend on the light-cone momentum fraction $x$ of parton $a$ 
      with respect to its parent hadron $h$, and on the factorization 
      scale\footnote{One could imagine to have two factorization scales,
      one for each hadron.  This may be relevant for certain processes such as 
      the fusion of electroweak bosons into a Higgs boson, where, at leading 
      order, the two hadrons do not interact through the exchange of colour.}
      $\mu_F$;
\item $\hat\sigma_{ab\to n}$ denotes the parton-level cross section for the 
      production of the final state $n$ through the initial partons $a$ and $b$.
      It depends on the momenta given by the final-state phase space $\Phi_n$, 
      on the factorization scale and on the renormalization scale $\mu_R$.
      The fully differential parton-level cross section is given by the product 
      of the corresponding matrix element squared, averaged over initial-state 
      spin and colour degrees of freedom, $|{\cal M}_{ab\to n}|^2$, and the 
      parton flux $1/(2\hat s) = 1/(2x_ax_bs)$, where $s$ is the hadronic 
      centre-of-mass energy squared.
\item The matrix element squared $|{\cal M}_{ab\to n}|^2(\Phi_n;\mu_F,\mu_R)$ can
      be evaluated in different ways.  In \AppRef{Sec:TL_ME} we discuss
      some of the technology used for tree-level matrix elements.  Here it 
      should suffice to say that the matrix element can be written as a sum
      over Feynman diagrams, 
      \begin{equation}
      {\cal M}_{ab\to n} = \sum\limits_{i}\,{\cal F}^{(i)}_{ab\to n}\,. 
      \end{equation}
      However, any summation over quantum numbers can be moved outside the 
      square, allowing one to sum over helicity and colour orderings
      such that
      \begin{equation}
      |{\cal M}_{ab\to n}|^2(\Phi_n;\mu_F,\mu_R) = 
      \sum\limits_{h_i;\,c_j}\,|{\cal M}^{\{ij\}}_{ab\to n}|^2
      (\Phi_n,\{h_i\},\{c_j\};\mu_F,\mu_R)\,. 
      \end{equation}
      In the computation of cross sections, this allows one to Monte
      Carlo sample not only over the phase space, but also over the
      helicities and colour configurations.  Picking one of the latter
      in fact defines the starting conditions for the subsequent
      parton showering, as discussed in more detail in
      \SecRef{sec:parton-showers}.
\item $\done\Phi_n$ denotes the differential phase space element over the $n$ 
      final-state particles,
      \begin{equation}
      \done\Phi_n = \prod\limits_{i=1}^n\frac{{\rm d}^3p_i}{(2\pi)^32E_i}
      \cdot(2\pi)^4\delta^{(4)}(p_a+p_b-\sum\limits_{i=1}^n p_i)\,,
      \end{equation}
      where $p_a$ and $p_b$ are the initial-state momenta.  For
      hadronic collisions, they are given by $x_aP_a$ and $x_bP_b$,
      where the Bjorken variables, $x_a$ and $x_b$, are also
      integrated over, and $P_a$ and $P_b$ are the fixed hadron
      momenta.
\end{itemize}

This equation holds to all orders in perturbation theory.   However,
when the subprocess cross section is computed beyond leading order
there are subtleties, which will be discussed later, and therefore for
the moment we consider only the use of leading-order (LO) subprocess
matrix elements.

It should be noted that the integration over the phase space may contain
cuts, for two reasons. First of all there are cuts reflecting the
geometry and acceptance of detectors, which are relevant for
the comparison with measured cross sections and other
related quantities.  On top of that there are other cuts, which, although
their details may be dominated by similar considerations, reflect
a physical necessity.  These are, for instance, cuts on the transverse
momentum of particles produced in $t$-channel processes, which exhibit
the analogue of the Coulomb singularity in classical electron scattering
and are related to internal particles going on their mass shell.  In a
similar way, especially for QCD processes, the notion of jets defined
by suitable algorithms (see \SecRef{sec:event-structure}) shields the 
calculation of the cross section of a process from unphysical soft and/or
collinear divergences.  
At leading order, these correspond simply to a set of cuts on parton momenta, 
preventing them from becoming soft or collinear.
%%%%%%%%%%%%%%%%%%%%%%%%%%%%%%%%%%%%%%%%%%%%%%%%%%%%%%%%%%%%%%%%%%%%%%
\mcsubsection{Leading-order matrix-element generators}
%%%%%%%%%%%%%%%%%%%%%%%%%%%%%%%%%%%%%%%%%%%%%%%%%%%%%%%%%%%%%%%%%%%%%%
All multi-purpose event generators provide a comprehensive list of LO 
matrix elements and the corresponding phase-space parameterizations for 
$2\to 1$, $2\to 2$ and some $2\to 3$ production channels in the 
framework of the Standard Model and some of its new physics extensions. 
For higher-multiplicity final states they employ dedicated matrix-element
and phase-space generators, such as \Alpgen~\cite{Mangano:2002ea}, 
\Amegic~\cite{Krauss:2001iv}, \Comix~\cite{Gleisberg:2008fv}, 
\Helac/\Phegas~\cite{Kanaki:2000ey,Papadopoulos:2000tt}, 
\Madgraph/\Madevent~\cite{Stelzer:1994ta,Maltoni:2002qb} and
\Whizard/\OmegaCode~\cite{Moretti:2001zz,Kilian:2007gr}, which are 
either interfaced (see \AppRef{sec:MEinterfaces}) or built-in 
as for the case of \Sherpa.  These codes specialize in the efficient 
generation and evaluation of tree-level matrix elements for multi-particle 
processes, see \AppRef{Sec:TL_ME} and \ref{Sec:PS_ME}.  

In doing so they have to overcome a number of obstacles. First of all, the 
number of Feynman diagrams used to construct the matrix elements increases
roughly factorially with the number of final-state particles.  This
typically renders textbook methods based on the squaring of amplitudes through
completeness relations inappropriate for final-state multiplicities of four 
or larger.  Processes with multiplicities larger than six are even more
cumbersome to compute and usually accessible through recursive relations only.
Secondly, the phase space of final-state particles in such reactions
necessitates the construction of dedicated integration algorithms, based on
the multi-channel method.  This, and other integration techniques, will be
discussed in more detail in \AppRef{sec:mc-methods} and \ref{Sec:PS_ME}.

%%%%%%%%%%%%%%%%%%%%%%%%%%%%%%%%%%%%%%%%%%%%%%%%%%%%%%%%%%%%%%%%%%%%%%
\mcsubsection{Choices for renormalization and factorization scales}
%%%%%%%%%%%%%%%%%%%%%%%%%%%%%%%%%%%%%%%%%%%%%%%%%%%%%%%%%%%%%%%%%%%%%%
The cross section defined by \EqRef{Eq::Master_For_XSec} is fully
specified only for a given PDF set and a certain choice for the
unphysical factorization and renormalization scales. There exists no
first principle defining what are the \emph{correct} $\mu_F$ and
$\mu_R$. However, our knowledge of the logarithmic structure of QCD
for different classes of hard scattering processes limits the range
of reasonable values. This knowledge is used as a guide when setting
the default choices in the various generators. Considering the class
of $2\to 1$ and $2\to 2$ processes, typically one hard scale $Q^2$ is
identified such that $\mu_F=\mu_R=Q^2$. Examples thereof are the
production of an $s$-channel resonance of mass $M$, where $Q^2=M^2$ or
the production of a pair of massless particles with transverse
momentum $p_T$, where typically $Q^2=p_T^2$. In general-purpose event
generators the hard scale $Q^2$ has the further meaning of a starting
scale for subsequent initial- and final-state parton showers\footnote{
  The precise phase-space limits of course depend on the relation
  between the generator's shower evolution scale and $Q^2$.}.
Accordingly when choosing $\mu_F$ and $\mu_R$ for processes with
final-state multiplicity larger than two, care has to be taken not to
introduce any double counting between the matrix-element calculation
and the parton-shower simulation, see \SecRef{sec:me-nlo-matching}.

%%%%%%%%%%%%%%%%%%%%%%%%%%%%%%%%%%%%%%%%%%%%%%%%%%%%%%%%%%%%%%%%%%%%%%
\mcsubsection{Choices for PDFs}
%%%%%%%%%%%%%%%%%%%%%%%%%%%%%%%%%%%%%%%%%%%%%%%%%%%%%%%%%%%%%%%%%%%%%%
Regarding the PDF, one is in principle free to choose any parameterization
that matches the formal accuracy of the cross section calculation, see 
\EqRef{Eq::Master_For_XSec}. All generators provide access to commonly
used PDF sets via the LHAPDF interface~\cite{Whalley:2005nh}. However, each generator
uses a default PDF set and the predictions of certain tunes of parton shower, 
hadronization and underlying event model parameters might be altered when 
changing the default PDF set, see \SecRef{sec:rivet-professor}. For a 
detailed discussion on PDF issues in Monte Carlo event generators see 
\SecRef{sec:pdfs-event-gener}.

%%%%%%%%%%%%%%%%%%%%%%%%%%%%%%%%%%%%%%%%%%%%%%%%%%%%%%%%%%%%%%%%%%%%%%
\mcsubsection{Anatomy of NLO cross section calculations}
\label{sec:subprocesses:NLOcross_sections}
%%%%%%%%%%%%%%%%%%%%%%%%%%%%%%%%%%%%%%%%%%%%%%%%%%%%%%%%%%%%%%%%%%%%%%
Most of the current multi-purpose event generators currently employ
leading-order (LO) matrix elements to drive the simulation. This means
that the results are only reliable for the shape of distributions,
while the absolute normalization is often badly described, due to
large higher-order corrections. One therefore often introduces a
so-called $K$-factor when comparing results from event generators with
experimental data. This factor is normally just that, a single factor
multiplying the LO cross section, typically obtained by the ratio of
the total NLO cross section to the LO one for the relevant
process. However, in this report we use the concept in a broader
sense, where the $K$-factor can depend on the underlying kinematics of
the LO process.

However, in striving for a higher accuracy and a better control
of theoretical uncertainties, some processes have been made accessible
at next-to-leading order accuracy and have been included in the
complete simulation chain, properly matched to the subsequent parton
showers.  This motivates the introduction of some formalism here,
which will be used in \SecRef{sec:me-nlo-matching}, where NLO event
generation will be discussed in some detail.

A cross section calculated at NLO accuracy is composed of three parts,
the LO or Born-level part, and two corrections, the virtual and the 
real-emission one.  Schematically,
\begin{equation}
\done\sigma^{\rm NLO} =
\done\tilde\Phi_n\left[{\cal B}(\tilde\Phi_n) + \alphaS{\cal V}(\tilde\Phi_n)\right] +
\done\tilde\Phi_{n+1}\alphaS{\cal R}(\tilde\Phi_{n+1})\,,
\end{equation}
where the tildes over the phase space elements $\done\tilde\Phi_{n}$ denote integrals over
the $n$-particle final state {\em and} the Bjorken variables, and
include the incoming partonic flux, and where the terms ${\cal B}$,
${\cal V}$, and ${\cal R}$ denote the Born, virtual and real emission
parts.  They in turn include the PDFs, and the summation over flavours
is implicit.

An obstacle in calculating these parts is the occurrence of
ultraviolet and infrared divergences.  The former are treated in a
straightforward manner, by firstly regularizing them, usually in
dimensional regularization, before the theory is renormalized.  The
infrared divergences, on the other hand, are a bit more cumbersome to
deal with.  This is due to the fact that they show up both in the
virtual contributions, which lead to the same $n$-particle final
state, and in the real corrections, leading to an $n+1$-particle final
state.  According to the Bloch-Nordsieck~\cite{Bloch:1937pw} and
Kinoshita-Lee-Nauenberg theorems~\cite{Kinoshita:1962ur,Lee:1964is},
for sensible, \ie infrared-safe, observables these divergences must
mutually cancel.  This presents some difficulty, since they are
related to phase spaces of different dimensionality.  In order to cure
the problem several strategies have been devised, which broadly fall
into two categories: phase-space slicing methods, pioneered in
\cite{Giele:1991vf,Giele:1993dj}, and infrared subtraction
algorithms~\cite{Catani:1996vz,Catani:2002hc,
  Kosower:1997zr,Kosower:2003bh,GehrmannDeRidder:2005cm,Daleo:2006xa,
  Frixione:1995ms,Frixione:1997np}. Current NLO calculations usually
use the latter.  They are based on the observation that the soft and
collinear divergences in the real-emission correction $\cal R$ exhibit
a universal structure.  This structure can be described by the
convolution of (finite) Born-level matrix elements, ${\cal B}$, with
suitably chosen, universal splitting kernels, ${\cal S}$, which in
turn encode the divergent structure.  Therefore, the ``subtracted
real-emission term'' $[{\cal R}-{\cal B}\otimes{\cal S}]$ is infrared finite and can
be integrated over the full phase space $\Phi_{n+1}$ of the
real-emission correction in four space-time dimensions.  The
subtraction terms ${\cal B}\otimes{\cal S}$
are added back in and combined with the virtual
term, ${\cal V}$, after they have been integrated over the radiative
phase space.  This integration is typically achieved in $D$
dimensions, such that the divergences emerge as poles in $4-D$.
Taking everything together, the parton-level cross section at NLO
accuracy reads, schematically,
\begin{eqnarray}
\label{Eq::NLO_XSec_Subtracted}
\sigma^{\rm NLO} =
\int\limits_n\done\tilde\Phi_n^{(4)}\,{\cal B} &+&
\alphaS\int\limits_{n+1}\done\tilde\Phi_{n+1}^{(4)}
        \left[\vphantom{\int\limits_{n}}
        {\cal R}-{\cal B}\otimes{\cal S}\right]\nonumber\\
&+&\alphaS\int\limits_n\done\tilde\Phi_n^{(D)}\,
        \left[\tilde{\cal V}+
              {\cal B}\otimes
              \int\limits_1\done\Phi_1^{(D)}{\cal S}\right]\,,
\end{eqnarray}
where the dimensions of the phase space elements and the number of
final-state particles have been made explicit and where collinear counter-terms
have been absorbed into the modified virtual contribution, $\tilde{\cal V}$.

It is worth noting that the task of evaluating the above equation can be
compartmentalized in a straightforward way.  A natural division is between
specialized codes, so-called one-loop providers (OLPs), that provide the 
virtual part, ${\cal V}$, and generic tree-level matrix element generators
which will take care of the rest, including phase space integration.
For details see \AppRef{Sec:Inter_ME}.  In the long run this will allow
for an automated inclusion of NLO accuracy into the multi-purpose
event generators; first steps in this direction have been made
in \cite{Binoth:2010xt,Alioli:2010xd,Hoeche:2010pf}.

In order to go to even higher accuracy, \ie to the NNLO level,
the above equation would become even more cumbersome, with more
contributions to trace.  This, however, will most likely remain far beyond 
the anticipated accuracy reach of the multi-purpose event generators for a 
long time to come.

%%%%%%%%%%%%%%%%%%%%%%%%%%%%%%%%%%%%%%%%%%%%%%%%%%%%%%%%%%%%%%%%%%%%%%
\mcsubsection{Summary}
%%%%%%%%%%%%%%%%%%%%%%%%%%%%%%%%%%%%%%%%%%%%%%%%%%%%%%%%%%%%%%%%%%%%%%
\begin{itemize}
\item The factorization formula in \EqRef{Eq::Master_For_XSec}
  is employed to calculate cross sections at hadron colliders.  The
  necessary ingredients are the parton-level matrix element, the
  parton distribution functions and the integration over the
  corresponding phase space.
\item At leading order, \ie for tree-level processes, there is a plethora
      of fully automated tools, constructing and evaluating the matrix elements
      with different methods.  They typically do not rely on textbook methods
      but on the helicity method or recursion relations.  
\item Due to the complexity of the processes, the phase space integration
      is a complicated task, which is usually performed using Monte
      Carlo sampling methods, which extend to include also treatment of
      the sum over polarizations and, more recently, even colours.  
\item The choice of the renormalization and factorization scales is not
      fixed by first principles, but rather by experience.  Combining the
      matrix elements with the subsequent parton shower defines, to some 
      extent, which choices are consistent and therefore ``allowed''.
\item Higher-order calculations, \ie including loop effects, are not yet
      fully automated.  They consist of more than just one matrix element
      with a fixed number of final-state particles, but they include 
      terms with extra particles in loops and/or legs.  These extra emissions
      introduce infrared divergences, which must cancel between the
      various terms.  This also makes the combination with the parton shower
      more cumbersome.  
\end{itemize}

% Local Variables: 
% mode: LaTeX
% TeX-master: "../mcreview"
% End: 
