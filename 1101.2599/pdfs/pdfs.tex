
Parton Distribution Functions play a central role in event
generators, for the simulation of hard processes, parton showers 
and multiple parton interactions. The choice of PDF set therefore will 
influence both cross sections and event shapes.

To lowest order the function $f_i(x, \mu_F)$ describes the probability
to find a parton of species $i$ with a momentum fraction $x$ when a 
proton is probed at a scale $\mu_F$. This distribution cannot be predicted
from first principles, since it depends on the non-perturbative physics 
of the proton wave function. With an ansatz for the initial 
distributions at some low scale $\mu_{F0}$, the evolution towards larger 
scales is predicted by the DGLAP equations 
\cite{Gribov:1972ri,Dokshitzer:1977sg,Altarelli:1977zs}, however.
Different tunes have been made, by comparing an evolved ansatz with 
relevant data, \eg from Deeply Inelastic Scattering. Over the years
many such tunes have been presented, with increasing accuracy as newer
data have been added. Also the theoretical framework has seen some 
improvements. The CTEQ \cite{Lai:2010vv} and MRST/MSTW 
\cite{Martin:2009iq} collaborations have been especially diligent in 
regularly presenting updated tunes. These and others are available 
in the LHAPDF library \cite{Whalley:2005nh}.

Precision tests of QCD today normally involve comparisons with NLO 
matrix elements convoluted with NLO PDFs (\EqRef{Eq::Master_For_XSec}) 
both defined in the $\overline{\mathrm{MS}}$ renormalization scheme. In general 
this allows a greatly improved description of data, relative to LO 
results. But NLO expressions are not guaranteed to be positive definite, 
and do not have a simple probabilistic interpretation.
For PDFs, specifically, it is well known that the NLO gluon has a 
tendency to start out negative at small $x$ for small scales, and 
only turn positive by the QCD evolution towards larger $\mu_F$. In the 
MRST/MSTW sets the ansatz allows negative gluons, while the CTEQ ansatz
is constructed to be positive definite. Either way, the NLO gluon PDF
is very different from the LO one, also at larger scales. 
Specifically, it remains smaller in NLO than in LO at small $x$, which 
then typically is compensated by the NLO MEs being larger than the LO 
ones, by the presence of $\ln(1/x)$-enhanced terms.

In recent years the emphasis has been put on NLO PDFs, as offering the
best description of hard-process data. The problem is that generators
are largely of an LO character. That is, while several hard processes are 
implemented to NLO accuracy in some generators, notably in the MC@NLO 
and POWHEG frameworks, see \SecRef{sec:me-nlo-matching}, most
are still only 
provided at LO, as are all the standard PS and MPI models, \eg with 
LO splitting kernels in the showers. 
The latter two components additionally have most of their activity in 
the low-$p_{\perp}$ region, and the last one especially with low-$x$ 
gluons. A usage of NLO PDFs with LO MEs here would be strongly 
questionable, since this combination typically undershoots the 
interaction rate obtained both in LO and NLO calculations.

Therefore the norm is for LO generators to use the few LO PDF fits
that are still produced. In cases where the shape is more important
than the absolute normalization --- such as backwards evolution of ISR,
where it is ratios of PDFs that sets the evolution rate --- this may
be perfectly adequate. But for absolute cross sections NLO calculations 
usually lie above LO ones, and this enhancement is needed 
to obtain a good description of data. This introduces a tension in LO 
PDF fits: a data set that is mainly sensitive to a particular $x$ range
prefers to have more of the total momentum of the proton located in
that $x$ range, at the expense of other $x$ ranges. To address this 
issue, new Monte Carlo adapted PDFs have been presented, wherein the 
momentum sum rule $\sum_i \int_0^1 \xpdf{i}(x, \mu_F) \, \mathrm{d}x = 1$
is relaxed: MRST LO* and LO** \cite{Sherstnev:2007nd}, 
and CT09 MC1, MC2 and MCS \cite{Lai:2009ne}. By allowing the 
normalization to float, one typically finds a value 10 -- 15\% above 
unity, and the whole $x$ range 
can be enhanced without tension. This should certainly not be viewed 
as a breaking of momentum conservation, but as a way to include an 
approximate $K$ factor that can depend on the parton flavours and
momenta but not on the specific process.

Another trick introduced is to make use of pseudo-data, obtained by 
generating several different processes to NLO, but then fitting them
to PDFs within an LO context \cite{Lai:2009ne}. That way it is possible 
to obtain a more uniform coverage over a large $x$ range, and for 
different flavours. Additionally, of the above five sets, CT09 MCS 
does not relax the momentum sum rule, but instead optimizes
renormalization and factorization scales process by process.

Note that much of the case for the modified LO sets is based
on the LO PDF behaviour at small $x$ and $\mu_F$. In the opposite region,
large $x$ and $\mu_F$, differences are not  as clear-cut. Often the NLO
corrections to the MEs there correspond to rescaling by an approximately
constant factor. Then, if NLO PDFs provide a better shape than LO PDFs
in that region, the combination of LO MEs and NLO PDFs makes sense.
It is therefore possible to use one PDF set for the hard processes and 
another for the softer parton showers and multiple interactions. 

To conclude, the new PDFs offer the hope of improved descriptions 
of data within generators, and have already started to be used, both
by generator authors and by the experimental collaborations, as an
alternative to normal LO ones. The outcome is still not clear; studies
suggest that you can find distributions where the new sets do better 
than the traditional ones, and distributions where they do worse
\cite{Kasemets:2010sg}. There can be no doubt, however, that these
sets have put new tools in the hands of generator authors, and
opened the way for further developments of PDFs especially well suited
for generator applications.

\begin{itemize}
\item PDFs are used in generators for the description of hard processes,
showers and MPIs.
\item While NLO PDFs are appropriate for studies of hard processes,
especially when combined with NLO MEs, they are not well suited for
current LO shower and MPI models.
\item New tricks have been introduced to improve the usefulness of  
LO PDFs in generators, and new sets have been presented, but currently
there is no obvious winner.
\end{itemize}

% Local Variables: 
% mode: LaTeX
% TeX-master: "../mcreview"
% End: 
