\mcsubsection{Introduction}%

The \Ariadne program \cite{Lonnblad:1992tz} was the first parton
shower generator to implement a dipole cascade. It uses the colour
dipole model by Gustafson \lletal
\cite{Gustafson:1986db,Gustafson:1987rq,Andersson:1988gp}, where gluon
emissions are modelled as coherent radiation from two colour-connected
partons.

For final-state radiation the \Ariadne cascade is rather similar to
any other dipole-based cascade, such as the ones described in
\SecsRef{sec:pythia8} and~\ref{sec:sherpa}. In \llee annihilation
into quarks, the first gluon emission is given by a dipole splitting
function identical to the exact differential cross section for
$e^+e^-\to q\bar{q}g$. Hence the matrix element matching
  described in \SecRef{sec:matching-first-ps} is automatically
  included.  The next emission will then either come from the dipole
between the quark and the gluon or from the dipole between the gluon
and the antiquark, with a trivial generalization for subsequent
emissions. The only difference in the subsequent emissions is the
colour factors and the non-singular behaviour of the splitting
functions, which in the soft and collinear limits coincide with the
standard Altarelli--Parisi splitting functions in \EqRef{DGLAP}.

The recoils in the emissions are taken by both emitting partons in the
radiating dipole. In this way all partons can be put on-shell in each
step of the cascade. The way the recoils in the transverse direction
are distributed between the radiating partons differs somewhat between
different types of dipoles. For example in quark--gluon dipoles, the quark
takes the full transverse recoil, so that the neighbouring dipole on
the gluon side is minimally disturbed, while for a gluon--gluon dipole
the transverse recoil is distributed so as to minimize the sum of the
squared transverse momenta of the emitters. Apart from these recoils,
all dipoles are treated independently.

The emissions are ordered (using appropriate Sudakov form factors) in
a Lorentz-invariant transverse momentum defined as
\begin{equation}
  \label{eq:ariadne-pt}
  \llipt^2=\llsdip(1-x_1)(1-x_2),
\end{equation}
where \llsdip is the squared invariant mass of the dipole, and
$x_i=2E_i/\sqrt{\llsdip}$ are the scaled energies of the partons after
the emission in the dipole rest frame. The transverse momentum is also
used as the scale in \alphaS. With an invariant definition of rapidity
\begin{equation}
  \label{eq:ariadne-y}
  \llirap=\frac{1}{2}\ln\frac{1-x_1}{1-x_2}
\end{equation}
it can easily be shown that the dipole splitting function is well
approximated by $\llD(\llipt,\llirap)\propto d\llirap\, d\ln\llipt$,
so apart from the running of \alphaS, the emission probability is
essentially flat in the $(\ln\llipt,\llirap)$ plane, where the
available phase space is given as an approximately triangular region,
$\ln(\llipti{\max}/\llipt)\lesssim|\llirap|$.

For final-state emissions, \Ariadne also includes the
$\llg\to\llqrk\llqbar$ splitting, by simply dividing the normal
Altarelli--Parisi splitting function between the two dipoles to which
the gluon is connected \cite{Andersson:1989ki}.

\mcsubsection{Hadronic collisions}%
What makes \Ariadne truly unique is the handling of radiation in
collisions where there are incoming hadrons. In a normal parton shower
one would then apply a backwards evolution of initial-state
splittings, and in more recent dipole shower implementations such as
those in \pythiaeight and \sherpa (see \SecsRef{sec:pythia8} and
\ref{sec:sherpa}) dipoles are defined between \eg incoming and
outgoing partons in the hard interaction. The \Ariadne program, in
contrast, uses the so-called Soft Radiation Model
\cite{Andersson:1988gp}, where there are dipoles between the hadron
remnants and the partons from the hard interactions.

Consider the process of deeply inelastic \llep scattering. We can view
it as a quark being kicked out of the proton by the virtual
photon. The quark carries colour, while the corresponding anti-colour
is continuing with the proton remnant down the beam pipe. From a
semi-classical viewpoint we then would have a large dipole spanned
between the struck quark and the proton remnant, and we could argue
that this dipole would radiate gluons in the same way as a dipole
between a \llqrk and \llqbar in an \llee-annihilation.

The difference between \llee and \llep is that in the former case the
emitting \llqqbar-pair is essentially point-like, while the proton
remnant in the \llep case is an extended object with about the same
size as the proton itself. And, just as the emission of
small-wavelength photons from an extended electric dipole antenna is
suppressed, one can argue that high-\llipt emission of gluons in the
proton direction should be suppressed \cite{Andersson:1988gp}. In
\Ariadne this is implemented by assuming that in any emission from a
dipole connected to a hadron remnant, only a fraction
\begin{equation}
  \label{eq:ariadne-softsup}
  a=\left(\frac{\mu}{\llipt}\right)^\alpha
\end{equation}
of the remnant energy is available. Here $\mu$ is the inverse
(transverse) size of the remnant (typically around 1~GeV), and
$\alpha$ is a parameter related to the dimensionality of the remnant
(1 would correspond to a string-like remnant, and 2 to a disc --- the
default value is 1). This gives a sharp cutoff in the phase space
allowed for gluon radiation, but optionally also some emission outside
this region is allowed with a power suppressed tail (in \llipt).

One can relate this suppression to the ratios of parton densities
which enter the initial-state splittings in a conventional backward
evolution shower (\cf \EqRef{eq:matching-sudnoem});
 however, especially in the remnant direction at
small-$x$, the suppression in the \Ariadne case is much less
severe. This shows up very distinctly in the case of forward jet rates
at HERA (see \eg \cite{Aktas:2005up}), where \Ariadne gives a much
higher jet rate than conventional cascades, in better agreement with
data. Comparing the Sudakov form factors one can see that the dipole
shower in \Ariadne resums some large logarithms of $1/x$, similarly to
what is done in BFKL evolution \cite{Rathsman:1996jc}.

There is one additional peculiarity in \Ariadne related to the proton
remnant. As only a part of the remnant takes part in the emission,
only that part of it will receive a recoil. This means that there will
be an extra gluon produced which is given some transverse
momentum. This gluon will, however, also be suppressed by an
additional Sudakov form factor corresponding to the probability that
no standard emission would produce a higher \llipt.

In \eg \llW boson production in hadronic collisions, where there are no
final-state coloured partons in the hard subprocess, the initial
dipole will be spanned between the two remnants. In this case the
recoil from emissions must also be shared with the \llW boson, which
is done in a way described in \cite{Lonnblad:1995ex}.

As in the final-state cascade, the initial-state $\llg\to\llqqbar$
splitting is not naturally described in terms of dipole
radiation. Instead this is included as a standard backward-evolution
step in a conventional initial-state shower. Also, the initial-state
emission of a quark in a $\llqrk\to\llg\llqrk$ splitting may be
included in the same way.

\mcsubsection{The \Ariadne program and the LHC}

The \Ariadne program was initially written in Fortran, to be run
together with the Fortran version of \Pythia, simply replacing the
\Pythia parton shower with the dipole cascade. In principle it can be
used to generate \lhc events, but some care must be taken when
generating processes with incoming gluons, such as Higgs production,
since the initial-state emission of quarks in the $\llqrk\to\llg\llqrk$
splitting mentioned above was not implemented in the Fortran version.

The \Ariadne program is currently being rewritten in \Cpp using the
\thepeg framework \cite{Lonnblad:2006pt} (see section
\SecRef{sec:thepeg}) and should soon be publicly available for
generating dipole cascades for any Standard Model process. The aim is
that it then will also be easily merged with matrix element generators
using the CKKW-L algorithm (see
\SecRef{sec:matching-at-tree}). Possibly it will also include
NLO-merging (see \SecRef{sec:nlo-merging}), but this is conditional on
whether the concept of recoil gluons, described briefly above, can be
reformulated in a way that is compatible with a proper \alphaS
expansion of the dipole emissions.

Finally it should be noted that the \Cpp version of \Ariadne is also
used in conjunction with a new initial-state evolution model
\cite{Avsar:2005iz,Avsar:2006jy,Avsar:2007xg} based on Mueller's
dipole evolution \cite{Mueller:1993rr,Mueller:1994jq,Mueller:1994gb}
formulated in impact-parameter space. The new program, called \llDIPSY
\cite{Flensburg:2010xx} is mainly intended to generate soft QCD
events, and can do so for both hadron collisions and heavy ion
collisions.

% Local Variables: 
% mode: LaTeX
% TeX-master: "../mcreview"
% End: 
