
\mcsubsection{Introduction}
\label{sec:matching:introduction}

In the previous sections we have described how to simulate
partonic final states with matrix elements and with parton showers,
and it should be clear to the reader that these approaches have
different merits and shortcomings. While fixed-order matrix elements
are excellent when simulating well separated, hard partons, they have
problems when trying to describe collinear and soft partons, due to the
occurrence of large logarithms. 
Also, obtaining the correct matrix element becomes very cumbersome when
we have more than a handful of partons. With parton showers it is the
other way around; hard, wide-angle emissions are poorly approximated,
while soft and collinear parton emissions are well described even for
very many partons. Clearly it would be desirable to combine the matrix
element and parton shower approaches to get a good description of any
partonic state. In particular, we note that a good description of soft
and collinear multi-parton states is necessary for hadronization
models such as string and cluster fragmentation (see
\SecRef{sec:hadronization}) to work properly.

To combine fixed-order matrix elements with parton showers is,
however, not a trivial task. The na\"ive procedure of simply adding a
parton shower to an event generated with a matrix element generator
does not work. One problem is related to the fact that tree-level
matrix elements are \textit{inclusive}, in that they give the
probability of having \textit{at least} $n$ partons in a state
calculated exactly to lowest order in \alphaS, while the
corresponding state generated by a parton shower is
\textit{exclusive}, given by the probability that there are
\textit{exactly} $n$ partons calculated approximately to all orders in
\alphaS. Another problem is that care must be taken not to double
count some regions of phase space or, conversely, to undercount other
regions.

\begin{figure}
\begin{center}
\hfil\llfiginc{matching/blobsNLL}\hfil\llfiginc{matching/blobsME4}\hfil
  \caption{Pictorial view of the terms in the \alphaS-expansion that
  enter into a jet cross section in \lleetoj. For each order in \alphaS,
  there are a number of large logarithms of the form
  $L^m=\log({Q\llsub{cm}/Q\llsub{jet}})^m$ (vertical axis). For
  $\alphaS^n$ the largest such logarithmic term is proportional to $L^{2n}$.
  For \eg a 4-jet observable we want to correctly include all
  coefficients from $\alphaS^2$ and onwards. In (a) we see the terms
  that would be correctly included in a NLL parton shower (filled
  blobs), while in (b) we see the terms correctly included in a
  tree-level matrix element.\label{fig:matching-ordersPS}}
\end{center}
\end{figure}
This problem can be understood on a more pictorial level in
\FigRef{fig:matching-ordersPS}.  There we have, for the process
\lleetoj, depicted the orders in the coupling constant $\alphaS$ on
the horizontal axis vs.\ the number of potentially occurring large
logarithms of the type $L^m=\log(Q\llsub{cm}/Q\llsub{jet})^m$ on the
vertical axis.  Here, $Q\llsub{cm}$ is an energy scale of the order of
the invariant mass of the produced system and $Q\llsub{jet}$ is
related to the resolution scale of a given jet algorithm (see
\SecRef{sec:event-structure}). The Born process is of order
$\alphaS^0$ and typically associated with the production of two
jets, the quark and antiquark in $e^+e^-\to q\bar q$.  Clearly, for each
additional emission, another factor of $\alphaS$ is necessary, such
that four jet production is of order $\alphaS^2$ and so on.  Also,
each emission can be related to at most two large logarithms,
associated with the soft and collinear divergences, see the previous
section on parton showers.

Now, the parton shower takes into account the exact leading and maybe
even next-to-leading logarithms, \ie\ it correctly takes into
account all real emissions and virtual corrections at all orders of
the type $\alphaS^nL^{2n}$ and $\alphaS^nL^{2n-1}$, while lower
powers of $L$ are treated approximately or
completely omitted. The leading $\alphaS^nL^{2n}$ term is easily
obtained by an \alphaS-expansion of the Sudakov form factor in
\EqRef{Sudakov}, while the next-to-leading term is obtained from the
hard collinear emission and from coherent treatment of soft emissions
in \SecRef{parton-showers:soft-gluons}. The treatment of these
logarithms will not impact on the total hadronic cross section, which
is still given by the Born-level value, due to the probabilistic
structure of the parton shower as discussed in
\SecRef{sec:parton-showers}.

On the other hand, differential distributions and observables
sensitive to the pattern of additional QCD radiation will be defined
by these logarithms.  Stated in other words: the parton shower will
not change the norm, but it will describe the shape of
radiation-sensitive distributions.

\begin{figure}
\begin{center}
\hfil\llfiginc{matching/blobsME1}\hfil\llfiginc{matching/blobsNLO}\hfil\llfiginc{matching/blobsME5}\hfil
\caption{Pictorial view analogous to
  \FigRef{fig:matching-ordersPS}. (a) The terms included in tree-level
  matching of the first emission. Note that the $\alphaS^1L^0$ blob
  is only half filled, as it is correctly taken into account only in
  the real-emission contributions, not in the virtual ones, which
  means that the shapes of distributions will be correct but not their
  normalization. (b) The terms included in NLO matching of the first
  emission. (c) The terms included in tree-level CKKW-like merging up
  to 5-jet, where the half-filled blobs are only correctly taken into
  account for real-emission contributions above the merging
  scale.\label{fig:matching-ordersNLO}}
\end{center}
\end{figure}

Taken together, coherent parton showers will correctly include all
filled blobs in \FigRef[a]{fig:matching-ordersPS} (equivalent to the
terms $\alphaS^nL^{2n}$ and $\alphaS^nL^{2n-1})$.  The natural
question thus arises of how to include more terms into the picture in
an exact way.  To this end, a number of different procedures have been
devised in the past two decades, and will be discussed is some detail
in subsequent subsections:
\begin{itemize}
\item \underline{Tree-level matching:}\\
  The first procedure for matching matrix elements with parton showers
  was invented decades ago by Bengtsson and
  Sjöstrand\cite{Bengtsson:1986hr}.  Similar techniques were later
  used also in
  \cite{Gustafson:1987rq,Seymour:1994we,Seymour:1994df,Miu:1998ju,Lonnblad:1995ex}
  and concentrated on correcting the first or hardest parton shower
  emission (\SecRef{sec:matching-first-ps}).  In our pictorial
  language, this amounts to including one extra blob as in
  \FigRef[a]{fig:matching-ordersNLO}, of order $\alphaS^1 L^0$, but
  \emph{only on the shape}, while the norm for the inclusive process (its
  cross section) was still given at the LO Born-level.
\item \underline{NLO matching:}\\
  In order to include also the NLO correction to the total cross
  section of the inclusive sample, one could naively apply a constant
  $K$-factor, as discussed in
  \SecRef{sec:subprocesses:NLOcross_sections}.  This, however, will
  not necessarily be good enough for precision studies, since these
  higher-order corrections may also influence the kinematics of the
  Born-level configuration.  Therefore, in the past decade much effort
  has been put into correcting parton showers also with exact
  next-to-leading-order (NLO) matrix elements, allowing to take into
  account the full effect of the $\alphaS^1 L^0$-term (see
  \FigRef[b]{fig:matching-ordersNLO}).  Here the
  \MCatNLO\cite{Frixione:2002ik,Frixione:2006gn}
  (\SecRef{sec:matching-mcatnlo}) and \POWHEG\cite{Nason:2004rx}
  (\SecRef{sec:powheg}) procedures have been very successful for the
  correction of the first parton shower emission and for including the
  effect on the cross section.  These ideas have been implemented in
  various programs, and we refer to the later parts of this section
  and to the descriptions of the individual event generators in
  \PartRef{sec:spec-revi-main} for the corresponding references.
\item \underline{Multi-jet merging at LO:}\\
  Another active area of developments in the last decade has been the
  treatment of multi-jet topologies, starting with the merging
  algorithm by Catani, Kuhn, Krauss and Webber (CKKW)
  \cite{Catani:2001cc} and a similar procedure developed in parallel
  by L\"onnblad (later coined CKKW-L) \cite{Lonnblad:2001iq}. In these procedures
  also the second and higher emissions in the parton shower were
  corrected to the corresponding tree-level matrix element, but at the
  price of introducing a technical merging scale above which the
  corrections are made (\SecRef{sec:matching-at-tree}). So, the aim of
  these procedures is to simulate each jet multiplicity (with jets
  above the cut) with the corresponding tree-level matrix element,
  dressed with the parton shower. Pictorially it corresponds to taking
  into account all $\alphaS^nL^{2n}$ and $\alphaS^nL^{2n-1}$ blobs
  as in a normal PS, but also the impact \emph{on the shape} of all
  the other blobs up to $\alphaS^n$ for the $n$ first real emissions
  above the cut, as indicated in \FigRef[c]{fig:matching-ordersNLO}.

  The CKKW and CKKW-L approaches can be shown to maintain the
  logarithmic accuracy of the parton shower, at least in $e^+e^-$
  annihilations, without any double-counting of contributions.  In
  addition, more pragmatic versions of the CKKW(-L) merging have
  been introduced, with less focus on the formal accuracy; the
  MLM approach by Mangano \cite{Mangano:2010xx} (see also
  \cite{Alwall:2007fs}) and the Pseudo-Shower algorithm by
  Mrenna and Richardson\cite{Mrenna:2003if}.  By now, the field has
  matured quite a lot, and in some recent publications the emphasis
  shifted to a more careful discussion of formal accuracy and its
  preservation, as in \cite{Hoeche:2009rj,Hamilton:2009ne}.
\item \underline{Multi-jet merging at NLO:}\\
  For higher parton multiplicities, there have also been some
  suggestions for NLO corrections (\SecRef{sec:nlo-merging}). The
  \MENLOPS procedure \cite{Hamilton:2010wh} simply combines \POWHEG
  with CKKW so that the Born-level differential cross section becomes
  correct to NLO.  The NL$^3$ procedure \cite{Lavesson:2008ah} and the
  procedure being implemented in the \Vincia
  program\cite{Giele:2007di,Skands:2010xx} go further and try to
  also correct the higher jet cross sections to NLO, but so far no
  easily accessible implementation exists.
\end{itemize}

Clearly there are several different strategies for combining matrix
elements and parton showers. As indicated above, it is useful to
distinguish two groups of approaches.
% for combining matrix elements and parton showers.
With \textit{matching} we refer to those approaches
in which high-order corrections to an inclusive process are
integrated with the parton shower.  The other strategy involves a
\textit{merging} scale, usually defined in terms of a jet resolution
scale, where any parton produced above that scale is generated with a
corresponding higher-order matrix element and, conversely, any parton
produced below is generated by the shower.


\mcsubsection{Correcting the first emission}
\label{sec:matching-first}

We start out by looking at the first emission in the parton shower. In
a transverse-momentum-ordered parton shower, this is the hardest
emission and will determine the main structure of the final
state. Hence, it is important to get this right also far away from the
soft and collinear regions where the parton shower is a good
approximation.

\mcsubsubsection{The NLO cross section}

To guide us we refer to the typical inclusive NLO cross section for a
process with a given Born-level state, and we rewrite the schematic
formula in \EqRef{Eq::NLO_XSec_Subtracted} from
\SecRef{sec:subprocesses:NLOcross_sections} in a more explicit form,
%
\bea
\lld\llxsec\llsup{NLO} &=&
\lld\llPS_{0}\left[B(\llPS_{0})+\alphaS V_1(\llPS_{0})
  + \alphaS\int\lld\llPS_{1|0} S_1(\llPS_{1})\right]\nnb\\
&+& \lld\llPS_{1}
\left[\alphaS R_1(\llPS_{1})
  - \alphaS S_1(\llPS_{1})\vphantom{\int}\right]\,.
\label{eq:matching:NLO1}
\eea
%
Here we identify\footnote{Note that implicitly these terms also
  contain symmetry factors and parton luminosities.} the Born-level
and the real-emission phase space $\llPS_{0}$ and $\llPS_{1}$, together
with the corresponding tree-level matrix elements $B(\llPS_{0})$ and
$\alphaS R_1(\llPS_{1})$. We also have the virtual or loop
contribution $\alphaS V_1(\llPS_{0})$ and the subtraction term
$\alphaS S_1(\llPS_{1})$ which, when integrated over the
one-particle phase-space element, $\llPS_{1|0}$, renders the first
bracket finite, and also regularizes the real-emission term, making
everything finite.

Typically, using the universality of soft and collinear divergencies,
we can write the subtraction term in a factorized form as
\begin{equation}
  \label{eq:matching:factorized-subtaction}
  S_1(\llPS_{1}) = B(\llPS_0)\otimes\tilde{S}(\llPS_{1|0}),
\end{equation}
where $\tilde{S}(\llPS_{1|0})$ are the universal subtraction kernels
with analytically known integrals.  At this point it is useful to
split the real-emission correction into one part containing all
singularities, $R_1\llsup{s}$, and one non-singular part,
$R_1\llsup{ns}$:
\begin{equation}
  \label{eq:matching:factorized-subtaction-ns}
R_1(\llPS_1) = R_1\llsup{s}(\llPS_1) + R_1\llsup{ns}(\llPS_1)\,.
\end{equation}
The splitting is quite arbitrary, as long as $R_1\llsup{s}$ contains
all singularities, but will later be used for illustrating differences
in some of the matching algorithms.

We can now write the inclusive NLO cross section
%
\bea
\lld\llxsec\llsup{NLO} &=&
\lld\llPS_{0}\left[B(\llPS_{0})+\alphaS V_1(\llPS_{0})
  + \alphaS\int\lld\llPS_{1|0} S_1(\llPS_{1})\right]\nnb\\
&+& \lld\llPS_{1}
\alphaS\left[R_1\llsup{s}(\llPS_{1})
  - S_1(\llPS_{1})\vphantom{\int}\right]
+ \lld\llPS_{1}\alphaS R_1\llsup{ns}(\llPS_{1})\,,
\label{eq:matching:NLO2}
\eea
%
where, again, all individual terms in the sum are finite. We can now
absorb the second bracket into the first and we can define the
\emph{NLO-weighted Born contribution}, $\bar{B}$, by integrating out
the singular terms,
%
\bea
\bar{B}(\llPS_{0}) &=&
B(\llPS_{0})+\alphaS V_1(\llPS_{0})
  + \alphaS\int\lld\llPS_{1|0} S_1(\llPS_{1})\nnb\\
&+& \alphaS\int\lld\llPS_{1|0} \left[R_1\llsup{s}(\llPS_{1})
  - S_1(\llPS_{1})\vphantom{\int}\right].
\label{eq:matching:NLOBorn}
\eea
%
Here we stress again that the $S_1$ can be written as a convolution of
a Born term and a universal subtraction term, and in
\begin{equation}
  \label{eq:matching:factorized-subtaction-ps}
  \int\lld\llPS_{1|0}S_1(\llPS_{1}) =
  B(\llPS_0)\otimes\int\lld\llPS_{1|0}\tilde{S}(\llPS_{1|0}),
\end{equation}
the integral over $\tilde{S}$ can be calculated analytically using
dimensional regularization, allowing us to add it to the virtual part
and thus explicitly cancel the divergences there. The integral over
$R_1\llsup{s} - S_1$, in contrast, must typically be evaluated
numerically through Monte Carlo integration.

We now arrive at a form of the inclusive NLO cross section,
\begin{equation}
  \label{eq:matching:NLO3}
  \lld\llxsec\llsup{NLO}=
  \lld\llPS_0\bar{B}(\llPS_{0})+\lld\llPS_1\alphaS R_1\llsup{ns}(\llPS_1),
\end{equation}
which will serve as a starting point for our discussion about
higher-order corrections to the parton shower approach.

\mcsubsubsection{The first emission in a parton shower}
\label{sec:matching-first-ps}

We now look at the inclusive cross section as given by the first
emission in a parton shower:
\begin{equation}
  \label{eq:matching:PSfirst}
  \lld\llxsec\llsup{PS}=
  \lld\llPS_0B(\llPS_0)\left[\llsud{0}(\llordmax,\llordcut)+
    \int_{\llordcut}\frac{\lld\llord_1}{\llord_1}
    \int\!\!\lld\llaux_1\,\frac{\alphaS}{2\pi}\llsplitP(\llaux_1)
    \llsud{0}(\llordmax,\llord_1)\right],
\end{equation}
where we have denoted the ordering variable \llord (with $\llordmax$
giving the starting scale, and $\llordcut$ the cutoff scale for the
shower). We note again that, performing the integral, the bracket is
unity, reflecting the unitary nature of the parton shower.

In principle higher-order corrections will have two effects in a
parton shower. They alter
\begin{enumerate}
\item the shape of distributions related to the first, hardest emission;
\item the norm --- the total cross section --- of the produced sample;
\end{enumerate}
and there are methods that focus on one or the other, or both.
%It is thus not surprising that they will be introduced in various ways.

The first modification, which was also historically the first matching
procedure \cite{Bengtsson:1986hr}, is simply to replace the splitting
function above with the singular part of the real emission matrix
element,
\begin{equation}
  \label{eq:matching:first}
  \frac{\lld\llord_1}{\llord_1}
  \lld\llaux_1\frac{\alphaS}{2\pi}\llsplitP(\llaux_1)\rightarrow
  \lld\llPS_{1|0}\frac{R_1\llsup{s}(\llPS_1)}{B(\llPS_0)}
\end{equation}
Two things are worth noting here:
\begin{itemize}
\item In the soft and collinear limits of the real-emission matrix
  elements, the effect of the extra emission factorizes into universal
  terms, which exhibit {\em exactly the same singularity structure as
    the splitting kernels employed in parton shower Monte Carlos}.
  These singularities, through the cut in the emission phase space,
  given by $\llordcut$, lead to large logarithms, which in turn are
  resummed by the parton shower.  The same large logarithms are, of
  course, then also encoded in $R_1\llsup{s}/B$, and thus the
  logarithmic structure of the parton shower is preserved.
\item In this transition, the one-particle phase space element of the
  parton shower is replaced by $\lld\llPS_{1|0}$, the phase space
  element that, starting from a Born configuration, produces a
  real-emission phase space configuration.  Clearly, if the
  one-particle emission phase space is not completely covered by the
  parton shower, the replacement above on its own will not be
  sufficient, since parts of the true available phase space will be
  left out.  In this case, a hard matrix element correction,
  essentially through the non-singular term $R_1\llsup{ns}$, is
  mandatory.  This effect of not covering the full phase space may
  happen for two reasons:
  \begin{enumerate}
  \item the parton shower does not completely cover the emission phase
    space.  This is true, \eg, for angular-ordered parton showers;
  \item the Born configuration exhibits zeros, due to polarization
    effects or similar, that are not present after the first emission.
    This happens, \eg, for the zero in the lepton pseudo-rapidity distribution from
    hadronic $W$-boson production, as discussed in
    \cite{Alioli:2008gx}.
%for the radiation zero in hadronic di-boson production
%    \cite{do-we-have-a-reference-for-radiation-zeros-in-W-production}.
  \end{enumerate}
\end{itemize}
Note that the replacement above can fairly easily be carried over to
the Sudakov form factor
\begin{equation}
  \label{eq:matching:sudcorr}
  \llsudb{0}(\llordmax,\llord)=
  \exp\left[-\int\lld\llPScut{1|0}{\llord}\alphaS
    \frac{R_1\llsup{s}(\llPS_1)}{B(\llPS_0)}\right],
\end{equation}
using the so-called veto algorithm (see \AppRef{mcmethods:veto}).

We now get the inclusive cross section
\begin{eqnarray}
  \label{eq:matching:PScorrfirst}
  \lld\llxsec\llsup{PScorr}&=&
  \lld\llPS_0B(\llPS_0)\left[\llsudb{0}(\llordmax,\llordcut)+
    \int\lld\llPScut{1|0}{\llordcut}\alphaS
    \frac{R_1\llsup{s}(\llPS_1)}{B(\llPS_0)}
    \llsudb{0}(\llordmax,\llord_1)\right]\nnb\\
  &+&\lld\llPS_1\alphaS  R_1\llsup{ns}(\llPS_1)\,,
%  \left[R_1(\llPS_1)-R_1\llsup{s}(\llPS_1)\vphantom{\int}\right],
\end{eqnarray}
and by undoing the integral over the real emission we get the first
emission of the parton cascade, properly weighted by the Sudakov form
factor to give the higher-order $\alphaS^nL^{2n}$ and
$\alphaS^nL^{2n-1}$ blobs in \FigRef[a]{fig:matching-ordersPS}, from
which we can now continue with the subsequent emissions in the parton
shower to get fully exclusive partonic final states.

The next logical step is to also achieve ${\cal O}(\alphaS)$
accuracy at the cross section level.  There are two ways to do this,
which go under the names of \POWHEG and \MCatNLO, respectively.

\mcsubsubsection{\POWHEG}\label{sec:powheg}

The \POWHEG approach, effectively, is an advanced matrix element
reweighting procedure, where the Born-level term in front of the first
square bracket in \EqRef{eq:matching:PScorrfirst} is replaced by the
NLO weighted Born-level term.  Furthermore, the whole first-order
real-emission term $R_1$ is used for $R_1\llsup{s}$, so that
$R_1\llsup{ns}=0$. Thus
\begin{equation}
  \label{eq:matching:powheg}
  \lld\llxsec\llsup{POWHEG} =
  \lld\llPS_0\bar{B}(\llPS_0)\left[\llsudb{0}(\llordmax,\llordcut)+
    \int\lld\llPScut{1|0}{\llordcut}
    \alphaS\frac{R_1(\llPS_1)}{B(\llPS_0)}
    \llsudb{0}(\llordmax,\llord_1)\right],
%  &+&\lld\llPS_1\alphaS
%  \left[R_1(\llPS_1)-R_1\llsup{s}(\llPS_1)\vphantom{\int}\right],
\end{equation}
and parton showering will give rise to similar emissions as the first
term in \EqRef{eq:matching:PScorrfirst} but with a global NLO-reweighting,
$\bar{B}/B$, acting as a local $K$-factor.

\paragraph{Truncated and vetoed parton showers}
\label{sec:matching:truncated}

Here we digress a bit to consider the problems arising if a parton
shower is not ordered in hardness or \kt. In the \POWHEG approach,
the first emission is supposed to be the hardest one, and if the parton
shower is not ordered in hardness one cannot simply add it to the
states generated by \POWHEG, as that would not ensure that the
subsequent emissions would be less hard than the first one.

The simplest solution is to start the shower at its maximum possible
ordering scale, but veto any emission that is harder than the first
\POWHEG one. However, as pointed out in \cite{Nason:2004rx}, this
means that the colour structure and kinematics of the parton shower
would be altered, as the emission with the highest ordering scale
would be emitted from the $+1$-parton state rather than from the Born
state, as it would have been in the normal shower.

The solution is to first reconstruct the parton shower variables
$(\llord_1,\llaux_1)$ for the emission given by \POWHEG. Then the
shower is started from the corresponding Born-state with the maximum
ordering variable $\llordmax$ and is allowed to evolve down to
$\llord_1$, vetoing any emission harder than the first emission. Then
the $(\llord_1,\llaux_1)$ emission is inserted, and the shower can
continue evolving, still vetoing any emission harder than the first
emission. This procedure is called a truncated, vetoed
shower\cite{Nason:2004rx}.

\mcsubsubsection{\MCatNLO}\label{sec:matching-mcatnlo}

Having at hand the formula for the cross section in the \POWHEG
formalism allows us to discuss the alternative \MCatNLO approach on
the same footing.  The main idea in \MCatNLO is that the singular terms
$R_1\llsup{s}$ are taken to be {\em identical} to the subtraction
terms $S_1$.  They in turn are given by the convolution $S_1 =
B\otimes \llsplitP$, \ie by additionally identifying the universal
subtraction terms with the parton shower splitting kernels.  Therefore
\begin{eqnarray}
  \label{eq:matching:MC@NLO}
  \lld\llxsec\llsup{MC@NLO}&=&\lld\llPS_0
  \left[B(\llPS_0)+\alphaS V_1(\llPS_0)+\alphaS
    B(\llPS_0)\otimes\int\lld\llPS_{1|0}\llsplitP(\llPS_{1|0})\right]\nnb\\
  &&\quad\times\left[\llsud{0}(\llordmax,\llordcut)+
    \int_{\llordcut}\frac{\lld\llord_1}{\llord_1}
    \int\!\!\lld\llaux_1\,\frac{\alphaS}{2\pi}\llsplitP(\llaux_1)
    \llsud{0}(\llordmax,\llord_1)\right]\nnb\\
  &+&\lld\llPS_1\alphaS\left[R_1(\llPS_1)-
    B(\llPS_0)\otimes \llsplitP(\llPS_{1|0})\vphantom{\int}\right].
\end{eqnarray}
Again, undoing the integral in the second bracket, we obtain the first
parton shower splitting, and we can continue the shower as before. In
practice, two sets of events are given to the parton shower. The first
set contain Born-level states given by the first bracket in
\EqRef{eq:matching:MC@NLO}, where the integral in the second bracket
is simply undone by running the parton shower. The second set contains
events with one extra parton, where the parton shower is added using
suitable starting conditions. In this way the ${\cal O}(\alphaS)$
contribution of the parton shower is removed and is instead replaced
by the exact NLO result.

To ${\cal O}(\alphaS)$, this is equivalent to \POWHEG, however, we
note that the real emission matrix element is not exponentiated in the
Sudakov form factor (which is \llsud{} rather than \llsudb{}). Also, contrary
to \POWHEG, it is not guaranteed that the weights of the generated
states are positive definite, as one can easily imagine having
splitting functions that overestimate the real emission matrix
element, rendering the last bracket negative.

\mcsubsection{Tree-level multi-jet merging and CKKW}
\label{sec:matching-at-tree}

An alternative way of choosing the $R_1\llsup{s}$ term is to introduce
a cutoff, which we shall call the merging scale, $\llordms$ such that
$R_1\llsup{s}(\llPS_1)=R_1(\llPS_1)\times\Theta(\llordms-\llord(\llPS_1))$.
We then get a modified Born term compared to \EqRef{eq:matching:NLOBorn}
\begin{equation}
  \label{eq:matching:exclusiveborn}
  \tilde{B}(\llPS_{0}) =
  B(\llPS_{0})+\alphaS V_1(\llPS_{0})
  + \alphaS\int\lld\llPSmax{1|0}{\llordms} R_1(\llPS_{1}),
\end{equation}
which we immediately identify as the NLO expression for the exclusive
cross section with no partons above the scale \llordms, in the full
inclusive NLO cross section,
\begin{equation}
  \label{eq:matching:NLO4}
  \lld\llxsec\llsup{NLO}=
  \lld\llPS_0\tilde{B}(\llPS_{0})+
  \lld\llPS_1\alphaS R_1(\llPS_1)\Theta(\llord(\llPS_1)-\llordms).
\end{equation}
We also see that the second term is exactly what we would get from a
standard tree-level matrix element generator for the one-parton cross
section with \llordms as cutoff.

\mcsubsubsection{Merging for the first emission}

Ignoring the NLO-reweighting of the Born term for the moment, we can
now obtain a parton shower where the first emission is corrected to
the tree-level matrix element if it is above \llordms:
\begin{eqnarray}
  \label{eq:matching:CKKWfirst}
  \lld\llxsec\llsup{CKKW}&=&
  \lld\llPS_0B(\llPS_0)\left[\llsud{0}(\llordmax,\llordcut)
    \vphantom{\int\frac{R_1}{B}}\right.\nnb\\
    &&+\int_{\llordcut}\frac{\lld\llord_1}{\llord_1}
    \int\!\!\lld\llaux_1\,\frac{\alphaS}{2\pi}\llsplitP(\llaux_1)
    \Theta(\llordms-\llord_1)
    \llsud{0}(\llordmax,\llord_1)\nnb\\
    &&\left.+\int\lld\llPS_{1|0}\alphaS
    \frac{R_1(\llPS_1)}{B(\llPS_0)}\Theta(\llord(\llPS_{1|0})-\llordms)
    \llsud{0}(\llordmax,\llord_1)\right].
\end{eqnarray}
We see that the matrix element will fill the phase space above the
merging scale (jet production) and the parton shower will fill the
phase space below (jet evolution), effectively amounting to a phase
space slicing into two disjoint regimes. This is the basis of the
CKKW-based merging algorithms\cite{Catani:2001cc,Lonnblad:2001iq}.

A number of things are worth noting here:
\begin{enumerate}
\item The expression just exhibits the inclusive cross section for the
  first, hardest emission of a given Born configuration.  The effect
  of the phase space slicing is made manifest, and leads to the two
  emission terms.  The second of these emission terms, with
  $\llord>\llordms$, is in fact generated differently from how the
  equation above suggests: In practice this term gives rise to a
  separate term contributing to the inclusive sample, but with a
  Born-level matrix element for a one-parton emission process as seed
  rather than the initial Born matrix element.
\item The equation above exhibits a new feature, namely a violation of
  unitarity, \ie the square bracket does not integrate to one any more.
  As long as a reasonable range for the value of \llordms is chosen, this 
  does not have a big impact, though, and the total cross section of all
  contributions will be relatively stable with respect to changes in \llordms.  
\item The equation above explicitly shows that the logarithmic accuracy of
  the parton shower is preserved in the merging.
\end{enumerate}

\mcsubsubsection{Multi-jet merging}

If we again undo the integrals in \EqRef{eq:matching:CKKWfirst}, we
can as in \EqRef{eq:matching:PScorrfirst} continue the cascade below
$\llord_1$. However, as noted above, the second integral can then be
thought of as an additional Born-level contribution for a one-parton
emission process, for which we again can correct the first splitting. This
would give us
\begin{eqnarray}
  \label{eq:matching:CKKWsecond}
  \lld\llxsec_1\llsup{CKKW}&=&
  \lld\llPS_1\alphaS R_1(\llPS_1)\Theta(\llord(\llPS_{1|0})-\llordms)
  \llsud{0}(\llordmax,\llord_1)\times\nnb\\
  &&\left[\llsud{0}(\llord_1,\llordcut)
    \vphantom{\int\frac{R_1}{B}}\right.\nnb\\
    &&+\int_{\llordcut}\frac{\lld\llord_2}{\llord_2}
    \int\!\!\lld\llaux_2\,\frac{\alphaS}{2\pi}\llsplitP(\llaux_2)
    \Theta(\llordms-\llord_2)
    \llsud{0}(\llord_1,\llord_2)\nnb\\
    &&\left.+\int\lld\llPS_{2|1}\alphaS
    \frac{R_2(\llPS_2)}{R_1(\llPS_1)}\Theta(\llord(\llPS_{2|1})-\llordms)
    \llsud{0}(\llord_1,\llord_2)\right].
\end{eqnarray}
Clearly if we also have higher-order tree-level matrix elements,
$R_3,R_4,\ldots$, available we can now continue to correct also higher
parton multiplicities.

Further notes are in order.
\begin{itemize}
\item The Sudakov form factors can either be calculated analytically,
  as in the original CKKW scheme, or they can be generated by the
  shower itself as in CKKW-L, where the form factor is interpreted
  strictly as a no-emission probability.
\item If interpreted as no-emission probabilities, the Sudakov form
  factors are always below unity, and the reweighting can be
  implemented as a simple vetoing procedure (see
  \AppRef{mcmethods:veto}).
\item In the MLM and Pseudo-Shower schemes, the Sudakov form factors
  are approximated by allowing the shower to radiate all the way down
  to its cutoff, $\llordcut$, and this partonic state is then clustered
  with a jet algorithm with \llordms as resolution scale. The
  probability that these partonic jets are close to the original
  partons is then an approximation of the no-emission probability. The
  results are similar to those of CKKW(-L)\cite{Alwall:2007fs}, but it
  is not entirely clear how far the formal accuracy of the parton
  shower can be maintained.\footnote{See \eg the discussion in
    \cite{Lavesson:2007uu}.}
\item Equating the Sudakov form factors in \EqRef{Sudakov} with
  no-emission probabilities is only correct in final-state
  radiation. For initial-state emissions we must use the no-emission
  probability in \EqRef{ISRsudakov}, and it can be shown that this is
  related to the Sudakov form factor needed in the merging by a simple
  ratio of parton density function\cite{Krauss:2002up,Lavesson:2005xu}
  (see also \cite{Ellis:1991qj}), such that
  \begin{equation}
    \label{eq:matching-sudnoem}
    \llsud{i}(\llord_i,\llord_{i+1})=
    \frac{\pdf{}(x,\llord_{i})}{\pdf{}(x,\llord_{i+1})}\times
    \llsud{i}(\llord_i,\llord_{i+1};x).
  \end{equation}
  Hence, the procedure above needs to be amended with an extra
  reweighting with this PDF-ratio in the case of hadronic collisions.
\item We have not considered the running of \alphaS in the parton
  shower. In practice, the matrix-element generators will use a fixed
  \alphaS, and in the CKKW-based algorithms, an additional weight,
  $\prod\alphaS\llsup{PS}(\llord_i)/\alphaS$, is introduced.
\item If the parton shower is not ordered in hardness, or \kt, we
  cannot simply add a shower below the merging scale. Instead we must
  use the truncated, vetoed shower, described in
  \SecRef{sec:matching:truncated}, generalized to several hard
  emissions.
\end{itemize}

\mcsubsection{Multi-jet NLO merging}
\label{sec:nlo-merging}

If we look at \EqRef{eq:matching:CKKWfirst}, it is easy to see how we
can reintroduce the NLO corrections from \POWHEG in the CKKW-(L)
matching schemes. What is needed is to replace the splitting function
in the first integral with the same ratio $R_1/B$ as in the second,
and also reintroduce the corrected Sudakov form factors
$\llsud{0}\to\llsudb{0}$. If we then also reweight all states with the
dynamic $K$-factor $\bar{B}(\llPS_0)/B(\llPS_0)$, we arrive at the
so-called \MENLOPS procedure \cite{Hamilton:2010wh}, where the inclusive
cross section is correct to NLO, and the hardest emissions are
corrected with tree-level matrix elements.

Lately, there has also been some progress in combining parton showers
and matrix elements in such a way that also the cross sections for
multi-jet final states becomes correct to next-to-leading order.  The
scheme called NL$^3$, suggested in \cite{Lavesson:2008ah}, relies on
being able to generate Born states with $n$ extra partons above some
merging scale exactly according to the exclusive NLO cross section, in
the same way as in \EqRef{eq:matching:exclusiveborn}. From these
states a shower is then allowed to evolve below the merging scale. To
these event samples one can then add event samples generated according to
the standard CKKW-L procedure, but these are reweighted such that the
two first orders in \alphaS (corresponding to those in the NLO
samples) are subtracted, by carefully expanding out the Sudakov form
factors and the running of \alphaS.

The procedure is technically complicated and will not be described in
detail here, and so far it has only been implemented for \lleetoj. In
the end (returning to the language of \FigsRef{fig:matching-ordersPS} and
\ref{fig:matching-ordersNLO}) the procedure corresponds to correctly
taking into account the impact, both on shape \emph{and} cross
section, of all the blobs up to $\alphaS^n$ together with all the
$\alphaS^kL^{2k}$ and $\alphaS^kL^{2k-1}$ ones ($k>n$), for the
$n$ first real emissions above the cut, while all other emissions are
only correct to $\alphaS^nL^{2n}$ and $\alphaS^nL^{2n-1}$
(corresponding to filling all the half-filled blobs in
\FigRef[c]{fig:matching-ordersNLO}).

\mcsubsection{Summary}
\label{sec:matching-outlook}


\begin{itemize}
\item Fixed-order matrix elements and parton showers have different
  merits and shortcomings. They should be combined to get the best of
  both, for an optimal description of multi-parton states.
\item Tree-level matrix elements cannot be blindly combined with
  a parton shower. The former are inclusive in nature, while the latter
  produces exclusive final states.
\item Matrix elements must be supplemented with Sudakov form factors
  to give exclusive final states that can be combined with a parton
  shower.
\item Special care must be taken if the parton shower is not ordered
  in hardness. When adding such a parton shower to a
  matrix-element-generated state it must therefore be properly
  truncated and vetoed.
\item Combining next-to-leading-order matrix elements with parton
  showers for the first emission is now state of the art. For
  multi-leg matching and merging, the state of the art is still
  tree-level matrix elements.
\end{itemize}

Combining fixed-order matrix elements with parton showers is a very
active research topic, and is important for giving reliable precision
predictions for jet production from QCD. The last word is surely not
said yet, and we are looking forward to many new ideas and
improvements in the near future. In the long run it does not seem
inconceivable that we will have generators producing results that are
correct to next-to-next-to-leading order.

% Local Variables: 
% mode: LaTeX
% TeX-master: "../mcreview"
% End: 
