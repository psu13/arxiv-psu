\label{sec:MEinterfaces}
%%%%%%%%%%%%%%%%%%%%%%%%%%%%%%%%%%%%%%%%%%%%%%%%%%%%%%%%%%%%%%%%%%%%%%
\paragraph{Les Houches Event Files}
%%%%%%%%%%%%%%%%%%%%%%%%%%%%%%%%%%%%%%%%%%%%%%%%%%%%%%%%%%%%%%%%%%%%%%
The Les Houches Event File (LHEF) format offers a simple structure for 
transferring parton-level events to general purpose event generators that
subsequently accomplish parton showering and hadronization. While the 
original version proposed in \cite{Boos:2001cv} was based on exchanging two
Fortran common blocks, in the recent version the information is 
embedded in a minimal XML-style file structure \cite{Alwall:2006yp}. 

All information specifying the actual run that 
produced the events, \eg the incoming beams, their 
energies and the PDF set used, is collected in a header structure.
This is supplemented by information on the 
cross sections and the event weighting strategy.

For each parton-level event all necessary information is stored in a 
separate structure, listing, amongst other things, the incoming and 
outgoing particle momenta, their flavour and potential mother--daughter 
relations as well as the event's weight. Most importantly for subsequent 
showering, each event carries a definite colour flow, 
determined according to some algorithm by the matrix-element generator 
code.

The LHEF format to output parton-level events is supported by all the 
major matrix-element generator programs and has proved to be a robust tool
for interfacing them with general-purpose event generators,
greatly boosting the set of available processes for the latter. 

%%%%%%%%%%%%%%%%%%%%%%%%%%%%%%%%%%%%%%%%%%%%%%%%%%%%%%%%%%%%%%%%%%%%%%
\paragraph{Binoth Les Houches Accord} 
%%%%%%%%%%%%%%%%%%%%%%%%%%%%%%%%%%%%%%%%%%%%%%%%%%%%%%%%%%%%%%%%%%%%%%
It is apparent from \EqRef{Eq::NLO_XSec_Subtracted}
that calculating a cross section at NLO is a very modular 
task. This is exploited by the Binoth Les Houches Accord, see 
Ref.~\cite{Binoth:2010xt}. It defines
a standard for passing the virtual times Born contribution of an one-loop 
calculation to a tree-level MC program that deals with the 
generation of the corresponding Born and real-emission processes, as 
well as the differential and integrated subtraction terms. 

In an initialization stage the one-loop provider (OLP) and the MC 
program exchange information on the calculational scheme. Then, for a given 
set of Born level momenta, the OLP returns the coefficients of the 
$1/\epsilon^2$ and $1/\epsilon$ poles and the finite term. This is sufficient 
to compose the full cross section calculation within a tree-level generator 
that implements the necessary subtraction terms \cite{Gleisberg:2007md,Frederix:2008hu,Frederix:2009yq,Czakon:2009ss}. This interface 
structure was used recently to calculate the NLO corrections to $W+3$ jets 
\cite{Berger:2009ep} and $t\bar{t}+2$ jets \cite{Bevilacqua:2010ve}.


%%%%%%%%%%%%%%%%%%%%%%%%%%%%%%%%%%%%%%%%%%%%%%%%%%%%%%%%%%%%%%%%%%%%%%
\paragraph{Implementing your own ME into MCs}
%%%%%%%%%%%%%%%%%%%%%%%%%%%%%%%%%%%%%%%%%%%%%%%%%%%%%%%%%%%%%%%%%%%%%%

Of course, the various event generators also support, to varying degrees,
implementations of matrix elements by their users.  This option is particularly
interesting for models with unorthodox particle content or to study small
fragments of larger models.  
