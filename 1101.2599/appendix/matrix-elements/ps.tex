%%%%%%%%%%%%%%%%%%%%%%%%%%%%%%%%%%%%%%%%%%%%%%%%%%%%%%%%%%%%%%%%%%%%%%
\paragraph{MC integration and sampling methods}
%%%%%%%%%%%%%%%%%%%%%%%%%%%%%%%%%%%%%%%%%%%%%%%%%%%%%%%%%%%%%%%%%%%%%%
Once the matrix element for a given process has been constructed, one is left with 
the task of performing the related integral over the phase space of the 
initial- and final-state particles. Taking account of the unknown
momentum fractions of the initial-state partons, the dimension of this phase space 
is $(3n-4)+2$ at hadron colliders, with the integrand (the matrix element 
squared) typically exhibiting a challenging structure with pronounced peaks.  
While the large dimensionality renders traditional quadrature methods useless 
and enforces the usage of MC techniques, the difficult structures 
pose a serious threat to the convergence of the integration.  
This means that highly advanced sampling algorithms need to be introduced,
which are dubbed phase-space integrators or ``integration channels''.
From the formal point of view, all integration channels dealing with the same 
final state are equal, as they must finally yield exactly the same value 
of the MC integral. However, they usually differ greatly in the
rate of convergence and in practice one would obviously prefer to use 
the channel leading to the smallest error in the shortest time.

%%%%%%%%%%%%%%%%%%%%%%%%%%%%%%%%%%%%%%%%%%%%%%%%%%%%%%%%%%%%%%%%%%%%%%
\paragraph{Multi-channel integration}
%%%%%%%%%%%%%%%%%%%%%%%%%%%%%%%%%%%%%%%%%%%%%%%%%%%%%%%%%%%%%%%%%%%%%%
To generate an adequate phase-space integrator for realistic 
$2\to n$-particle processes, several existing channels can be combined 
using the multi-channel method~\cite{Kleiss:1994qy}. 
Symbolically one can write a single channel as a map $X$ from uniformly 
distributed random numbers $\vec x\in[0,1]^{3n-4}$
to the four-momenta $\vec p=(p_1,\ldots,p_n)$ of final-state particles,
The corresponding MC weight $g$ is then given by
\begin{equation}
  \frac{1}{g}=\frac{\done\Phi_n(X(\vec x))}{\done\vec x}
\end{equation}
where $\Phi_n$ represents the $n$-particle phase space.
The multi-channel method now combines several maps $X_i$ into a new map
$\rm X$ as follows:
\begin{equation}\label{eq:gen_point_mc}
  {\rm X}(\vec x,\tilde\alpha)=X_k(\vec x)\;,\quad\text{for}\quad
  \sum_{l=1}^{k-1}\alpha_l<\tilde\alpha<\sum_{l=1}^{k}\alpha_l\;,
\end{equation}
requiring an additional random number $\tilde\alpha$ and arbitrary
coefficients $\alpha_k$ with $\alpha_k>0$ and $\sum_k \alpha_k=1$. 
The corresponding MC weight is given by
\begin{equation}\label{eq:gen_weight_mc}
  G=\sum_k \alpha_k\; g_k\;.
\end{equation}
The coefficients $\alpha_k$ can be adapted to minimize the variance of the 
phase-space integral.

%%%%%%%%%%%%%%%%%%%%%%%%%%%%%%%%%%%%%%%%%%%%%%%%%%%%%%%%%%%%%%%%%%%%%%
\paragraph{Brief review of phase-space factorization}
%%%%%%%%%%%%%%%%%%%%%%%%%%%%%%%%%%%%%%%%%%%%%%%%%%%%%%%%%%%%%%%%%%%%%%
Consider the $2\to n$ scattering process in \EqRef{Eq::Master_For_XSec}, 
where we label incoming particles by $a$ and $b$ and outgoing particles by 
$1\ldots n$.  The corresponding partonic $n$-particle differential phase 
space element reads
\begin{equation}
  \begin{split}
  {\rm d}\Phi_n(a,b;1,\ldots,n)=&
    \left[\,\prod\limits_{i=1}^n\frac{{\rm d}^4 p_i}{(2\pi)^3}\,
    \delta(p_i^2-m_i^2)\Theta(p_{i0})\,\right]\,\\
    &\qquad\times(2\pi)^4\delta^{(4)}\left(p_a+p_b-\sum_{i=1}^n p_i\right)\;,
  \end{split}
\end{equation}
where $m_i$ are the on-shell masses of the outgoing particles.  Following 
Ref.~\cite{James:1968gu}, the $n$-particle phase space can be factorized as
\begin{equation}\label{eq:split_ps}
  \begin{split}
  {\rm d}\Phi_n(a,b;1,\ldots,n)=&\;
    {\rm d}\Phi_{n-m+1}(a,b;\pi,m+1,\ldots,n)\,\\
    &\qquad\times \frac{{\rm d} s_\pi}{2\pi}\,
    {\rm d}\Phi_m(\pi;1,\ldots,m)\;,
  \end{split}
\end{equation}
where $\pi=\{1\ldots m\}$ indicates an $s$-channel virtual particle.
\EqRef{eq:split_ps} allows one to decompose 
the complete phase space into building blocks corresponding to
$s$- and $t$-channel two-body decay processes of the form 
${\rm d}\Phi_{2}(\{12\};1,2)$ and ${\rm d}\Phi_{2}(a,b;1,2)$.
We refer to these objects as phase-space 
vertices, while the integral ${\rm d} s_\pi/2\pi$, introduced in 
\EqRef{eq:split_ps}, is called a phase-space propagator. 
There is a close correspondence between matrix element computation and 
phase-space generation, justifying this notation.
Even though $s$- and $t$-channel decay seem identical, since 
both represent a solid angle integration, in practice one would use 
different sampling strategies~\cite{Byckling:1969sx}. 

%%%%%%%%%%%%%%%%%%%%%%%%%%%%%%%%%%%%%%%%%%%%%%%%%%%%%%%%%%%%%%%%%%%%%%
\paragraph{Sequential algorithm for phase-space integration}
%%%%%%%%%%%%%%%%%%%%%%%%%%%%%%%%%%%%%%%%%%%%%%%%%%%%%%%%%%%%%%%%%%%%%%
\label{sec:sequential_phasespace}
One of the most efficient general approaches to sampling the phase space 
of multi-particle processes is to employ a sequential algorithm, constructing
the full phase space based on the pole structure of one of the Feynman diagrams
that contribute to the matrix element. This technique was suggested 
very early on in the history of MC programs~\cite{Byckling:1969sx}. 
It is then also possible to construct a separate integrator for each possible 
graph and employ multi-channel methods to optimize the integration~\cite{Kleiss:1994qy}. 
The method provides a general way to adapt to the assumed pole structure of 
arbitrarily complicated matrix elements. It is nowadays widely used 
by the most advanced general-purpose phase-space generators~\cite{
  Kanaki:2000ey,Papadopoulos:2000tt,Cafarella:2007pc,
  Krauss:2001iv,Maltoni:2002qb,Alwall:2007st,Gleisberg:2008fv}. 
The core algorithm can be formulated as recursive relations in terms 
of the phase-space propagators and vertices.

The difference between the various phase-space generators available
today is usually only {\it how} these recursive equations are employed.
If the basic building blocks are 
used to build ``phase-space diagrams'', we obtain an integrator which is
suitable for combination with a diagram-based matrix-element generator.
If the recursion is implemented as is, the resulting phase-space generator
is best combined with a recursive method to compute the matrix elements.
\if{false}
In the latter case it is important to note that while momenta 
labelled by $\alpha$ in \EqRef{eq:ps_building_blocks} 
may correspond to an off-shell internal particle, $b$ always indicates 
a fixed external incoming particle. This eventually allows the reuse of 
MC weights, just as currents are reused in recursive matrix-element 
computations. 
\fi

%%%%%%%%%%%%%%%%%%%%%%%%%%%%%%%%%%%%%%%%%%%%%%%%%%%%%%%%%%%%%%%%%%%%%%
\paragraph{Other algorithms}
%%%%%%%%%%%%%%%%%%%%%%%%%%%%%%%%%%%%%%%%%%%%%%%%%%%%%%%%%%%%%%%%%%%%%%
Other algorithms for phase-space integration exist, which are 
often less general, but potentially more efficient for their purpose.
One of them is the HAAG method~\cite{vanHameren:2002tc},
which is designed to produce momenta distributed approximately according to 
a QCD antenna function for an $n$-particle process, which reads
\begin{equation}\label{eq:antenna}
  A_{n}(p_0,p_1,...,p_{n-1})=\frac{1}{
    (p_0 p_1)(p_1 p_2)...(p_{n-2} p_{n-1})(p_{n-1} p_0)}.
\end{equation}
Different antennae can be obtained from permutations of the momenta $\{p_i\}$.
Generally, like the sequential phase-space integrator described above,
HAAG relies on phase-space factorization over time-like intermediate 
momenta.  The main difference lies in the sequence of factorization 
and in the sampling technique for the basic vertices,
which resembles the phase-space sampling in a dipole shower.

An important, simple but universally applicable phase-space integrator is 
Rambo~\cite{Kleiss:1985gy}. It is widely used because the underlying
algorithm requires no information about the integrand. This makes
Rambo the preferred default choice if no time is to be spent on the 
construction of a dedicated integration channel for the process in question.
Rambo assumes an unconstrained phase space, \ie a phase space where 
four-momentum conservation does not hold, to generate initial particle
momenta. These momenta are in turn boosted and rescaled to arrive at 
a physically meaningful phase-space point. The conformal transformation 
thus applied is associated with an additional weight. Rambo can be used
for massless and massive particles alike, where massive particles simply
require an additional step in the algorithm.
