Several distinct physics and modelling issues come under the 
heading of ``soft QCD'' and ``underlying event''. In \SecRef{sec:primkt}  
we discuss ``primordial \kT'', a
topic that lies on the intersection between parton showers and
soft QCD.
The rest of this section is  devoted to a discussion of
the different physics (sub-)processes that  contribute to the total
observed activity in hadron-hadron collisions. 
Thus, in \SecRef{sec:mbtypes}, we give a brief introduction to --- and
dictionary of --- the different 
QCD processes that form the dominant part of the total hadron-hadron
cross section, and to the origin of the so-called ``underlying event''
and the associated ``pedestal effect''.
In \SecRef{sec:mbmpi}, we then 
take a closer look at models based on multiple parton interactions
(MPI). \SecRef{sec:colrec} focuses on the particular issue of colour
reconnections. 
Finally, \SecRef{s_ue:pomerons} gives a very brief introduction to models
of diffraction, in particular models based on pomerons. 

For more specific details on the current implementations in each of
the main generators, see the individual descriptions in
\PartRef{sec:spec-revi-main}. 

\mcsubsection{Primordial \kT \label{sec:primkt}} 

In Monte Carlo models, the term ``primordial \kT'' is used to refer
collectively to any transverse momentum given to initial-state
partons \emph{beyond} that generated by the normal ISR shower
evolution described in \SecRef{sec:isr}.  

Physically, such additional momentum could come from several
sources, as follows:
\begin{enumerate}
\item Fermi motion of confined
partons inside their parent hadron, with a magnitude of order the
inverse hadron radius $\sim \LambdaQCD$.  
\item ``Unresolved'' ISR shower activity, coming from scales 
below the infrared cutoff.
\item Activity not accounted for, or incorrectly accounted for, 
by the particular shower model in question. In particular, this may be
relevant for parton evolution involving 
low parton momentum fractions $x$, as was briefly
discussed in \SecRef{sec:isr}.
\end{enumerate}
Of these, only Fermi motion is relatively straightforward.
It is also the only component that is genuinely
``primordial'', \ie intrinsic to the incoming hadron.
The others depend sensitively on issues that are
inherently ambiguous in the shower description: whether the
low-$\pT{}$ divergences in the parton shower are regulated by a sharp
cutoff or by a smooth 
suppression (and in what variable), how $\alphaS$ is treated close to
the cutoff, how the shower radiation functions and recoil effects 
behave, and whether any non-trivial low-$x$ effects are included. 

In the simplest possible treatment one lumps 
all these ``unresolved'' effects, regardless of origin, into 
a single number, called ``primordial \kT''. In each event, the
beam-collinear partons extracted from each of the original hadrons
can then be given a \pt\ of this magnitude, typically distributed
according to a Gaussian or similar distribution. (Since the beam
remnant must necessarily take up the recoil from this kick, 
an upper cutoff is usually also enforced, limiting the amount by
which the beam remnant is allowed to be kicked off axis by 
tails of this effect.)

However, even in this simplified case, the above discussion should serve
to illustrate that the value of the ``primordial \kT'' should not really be
perceived of as a universal constant. At the very least, the shower evolution
equations imply that it must have some  implicit dependence on the ISR cutoff --- 
if we increase the shower cutoff, for instance, 
the primordial \kT should increase slightly as well, to compensate for the now
missing shower activity in the region that has been cut
away. In current models, this scaling does not happen automatically,
but must be taken care of by retuning the primordial \kT if the ISR
cutoff (or any other parameter affecting the infrared regularization
of the initial-state shower) is changed. More troublingly, perhaps, 
since a lot of possible process-dependent physics effects have been
``swept under the rug'', there is also no strong reason why the same 
value of this primordial \kT should work equally well for all
processes or even in different phase space regions. 
Attempting to extract a value for it in several different, mutually
complementary, processes and regions, 
could therefore be a valuable input to guide future modelling. 

A next-to-simplest iteration can be obtained by letting 
the value of primordial \kT scale with the $Q^2$ of the hard
interaction \cite{Sjostrand:2004pf}. This generates a minimum of
process-dependence (\eg partons entering a soft QCD scattering can 
now be given a smaller primordial \kT than ones producing a $Z$
boson), but still does not really address the underlying physics. 

A first stab at a more physical
model was made in \cite{Gieseke:2007ad}, by including a
non-perturbative function that depends explicitly on the phase space
available for unresolved initial-state radiation, in
addition to a smaller and more
universal component to be modelled by a Gaussian. 
Although the data available at the time 
could not clearly differentiate this from the conventional models, 
the expectation is that this should generate a more realistic 
process- and collision-energy-dependence of the effective primordial
\kT.

A secondary modelling issue, relevant to the MPI models discussed in
the next subsection, is how much primordial \kT is assigned to
partons initiating multiple parton interactions, and 
how the associated recoil effects are distributed
among those initiators and the remnant. Typically, MPI initiators are
only assigned a primordial \kT of the order of Fermi motion, 
although this is a model-dependent statement that
may of course change, as models improve.

\begin{figure}[tp]
  \centering
    \label{fig:inline:drell-yan-intrinsic-kt-tvt}
    \includegraphics*[scale=0.55]{mc-plots/CDF_2000_S4155203-inline/CDF_2000_S4155203_d01-x01-y01}
    \includegraphics*[scale=0.55]{mc-plots/CDF_2000_S4155203-inline/CDF_2000_S4155203_MOD_d01-x01-y01}
  \caption{The low-\pt\ peak of the \pt\ distribution of lepton pairs
  in Drell-Yan events at the Tevatron, compared to CDF
  data \cite{Affolder:1999jh}. 
  A Monte Carlo model
  (\herwigpp) is shown with four different choices for the
  ``primordial \kT''. a) $p\bar p$ at 1.8 TeV b) $pp$ at
  14 TeV.
\label{fig:primkt}}
\end{figure}
Empirically, the most important distribution for constraining the
magnitude of this effect is the \pt\ distribution of lepton pairs
in Drell-Yan events. The peak of this distribution is extremely
sensitive to infrared effects. In \FigRef[a]{fig:primkt}, we
compare the distribution measured by the CDF
experiment \cite{Affolder:1999jh} 
to  a Monte Carlo model (\herwigpp) with four different primordial-\kT
settings: $0$ GeV (off), 
          $1.9$ GeV (the default in \herwigpp), 
          $3.8$ GeV (twice the default), 
          and the IR-augemented shower
          model \cite{Gieseke:2007ad}. 
To illustrate how these predictions scale with  collider centre-of-mass energy,
keeping the $Q^2$ of the hard interaction fixed, we also include a
plot showing the \pt\ of Drell-Yan pairs in $pp$ collisions at 14 TeV
in \FigRef[b]{fig:primkt}; the distributions becomes broader, but the
peak position stays relatively constant. A comparison of different
generators on this distribution can be found
in \FigRef{fig:cmp:intrinsic-kt} in the comparisons section of the
review (\SecRef{sec:physics-areas-where}). 

It is also worth noting that, depending on the model, details of how
the transverse momenta generated by the initial-state parton shower
and the primordial component are combined, the latter can also have a
significant effect well above the peak region. In the more primitive
models, there is also the (probably artificial) possibility of a
``double-peak'' structure emerging at high energies, with a
higher-\pt\ peak generated by the perturbative shower and a low-\pt\
one by primordial \kT. 

As mentioned above, it is important to consider also
complementary distributions, involving different scales or $x$
values, to fully constrain this ambiguous component of Monte Carlo
models. Good examples here would be the Drell-Yan process at different $Q^2$
values or at different rapidities. A systematic comparison to
extractions in DIS could also be fruitful. 

\mcsubsection{Soft QCD processes\label{sec:mbtypes}}

\paragraph{Elastic and inelastic} Elastic scattering 
consists of  all reactions of the type 
\begin{equation}
A(p_A)B(p_B)\to A(p_A')B(p_B')~,
\end{equation}
where $A$ and $B$ are particles
carrying momenta $p_A$ and $p_B$, respectively. Specifically, 
the only exchanged quantity is momentum; all quantum numbers and
masses remain unaltered, and no new particles are produced. 
Inelastic scattering covers everything else, \ie  
\begin{equation}
 AB\to X \ne AB~,
\end{equation} 
where $X\ne AB$ signifies that one 
or more quantum numbers are changed, or more particles are
produced. The distinction between elastic and inelastic scattering is
physically observable and is therefore quantum mechanically
meaningful (see \SecRef{sec:phys-phil-behind} for further discussion
of this point). 
Thus, we divide the total hadron-hadron cross
section into two physically distinguishable components, 
\begin{equation}
\sigma_{\mathrm{tot}}(s) = 
\sigma_{\mathrm{el}}(s) +
\sigma_{\mathrm{inel}}(s)~, 
\end{equation}
where $s=(p_A+p_B)^2$ is the beam-beam centre-of-mass energy squared. 

\paragraph{Diffractive and non-diffractive}
If $A$ or $B$ are not elementary the inelastic final states may be
further divided into ``diffractive'' and ``non-diffractive''
topologies. This is a qualitative classification, usually based on
whether the final state looks like
the decay of an excitation of the beam particles 
(diffractive), or not (non-diffractive), or upon the presence of a
large rapidity gap somewhere in the final state which would separate such excitations.
There  are two schools of thought on how to specify this distinction more precisely: 
\begin{enumerate}
\item Use a theoretical model, whose 
different physics subprocesses can each be uniquely assigned as 
diffractive or non-diffractive. 
However, different models produce different final-state spectra,
and hence such a classification necessarily depends on
the model used to make it. Furthermore, if the 
model allows for events of both diffractive and non-diffractive origin
to populate the same phase space points, the
interference terms between them have no unique assignments and hence 
the classification cannot be made quantum mechanically meaningful, see
\SecRef{sec:phys-phil-behind} for a more general 
discussion of this issue. 
\item
Use one or more physical observables, which
guarantees that the definition will also be valid at the quantum level. 
In this case, the arbitrariness is instead reflected
in the fact that one has to choose what one means by a ``diffractive
topology'', at the level of a final-state observable, and this 
choice is without a unique ``correct'' answer. In general, one defines
diffractive topologies as 
events that contain large rapidity gaps in the activity,
consistent with (possibly multiple) decays of excited states, 
with ``large'' often taken to be 
somewhere in the range of 3--5 units of rapidity.
\end{enumerate}

\paragraph{Types of diffraction}
Given that an event has been labelled as diffractive either by a
theoretical model or by a final-state observable, we may distinguish
between three different classes of diffractive topologies, which it is
possible to distinguish between physically, at least in principle. 
In double-diffractive dissociation (DD) events, both of the beam particles are
diffractively excited and hence neither of them survive the collision
intact. In single-diffractive dissociation (SD) events, only one of the beam
particles gets excited and the other survives intact. The last 
diffractive topology is  central diffraction (CD),  in which 
both of the beam particles survive intact, leaving an excited system
in the central region between them\footnote{This latter topology also includes
so-called ``central exclusive production''~\cite{Khoze:2001xm}.}.
That is, 
\begin{equation}
\sigma_{\mathrm{inel}}(s) = 
\sigma_{\mathrm{SD} }(s)
+
\sigma_{\mathrm{DD}}(s) +
\sigma_{\mathrm{CD}}(s) + 
\sigma_{\mathrm{ND}}(s) ~, \label{eq:diff}
\end{equation}
where ``ND'' (non-diffractive, here understood not to include elastic
scattering) contains no gaps in the event
consistent with the chosen definition of diffraction. Further, 
each of the diffractively excited systems in the events labelled SD,
DD, and CD, respectively, may in principle consist of
several subsystems with gaps between them. \EqRef{eq:diff} may 
thus be defined to be exact, within a specific definition of
diffraction, even in the presence of multi-gap events. 
Note, however, that different
theoretical models almost always use different (model-dependent) definitions of
diffraction, and therefore the individual components in one model are
in general not directly comparable to those of another. It is
therefore important that data be presented at the level of physical
observables if unambiguous conclusions are to be drawn from them, see
\SecRef{sec:phys-phil-behind} for a more detailed discussion of this issue.
Monte Carlo models of diffraction will be discussed briefly in
\SecRef{s_ue:pomerons} below.

\paragraph{Minimum bias and soft inclusive physics}
The term ``minimum bias'' is an experimental term, used to define a
certain class of events that are selected with the minimum
possible selection bias, to ensure they are as inclusive as possible. This will
be discussed in more detail in \SecRef{sec:phys-phil-behind}.
In theoretical contexts 
the term ``minimum bias'' is often used with a slightly different
meaning: to denote specific (classes of) inclusive soft QCD
subprocesses in a given model. 
Since these two usages are not exactly identical, in this review we have chosen to 
reserve the term ``minimum bias'' to pertain strictly to
definitions of experimental measurements, and instead use  
the term ``soft inclusive physics'' as a generic descriptor for the
class of processes which generally dominate the various experimental
minimum bias measurements in theoretical models.

\paragraph{Underlying event and jet pedestals} 
In events containing a hard parton-parton interaction, 
the underlying event represents the additional
activity which is not directly associated with that interaction. 
There is some ambiguity in how one defines what is
``associated'' with the hard interaction, and what is not. Here, we
shall define the underlying event to represent the additional
activity \emph{after} all bremsstrahlung off the hard interaction
has already been taken into account. Specifically, initial-state
radiation off the hard interaction is \emph{not} included in our
definition of the underlying event. Note also that the underlying
event is usually much more active, with larger fluctuations, than
 soft-inclusive collisions at the same energy. 
This is called the ``jet pedestal'' effect (hard jets sit on top of a
higher-than-average ``pedestal'' of underlying activity), and is
interpreted as follows. When two hadrons collide at non-zero impact
parameter, high-$p_\perp$ interactions can only take place  
inside the overlapping region. 
Imposing a hard selection cut therefore statistically
biases the event sample toward more central collisions, which will also
have more underlying activity. The size of the pedestal, as a function
of leading track \pt, is illustrated in
\FigsRef{fig:cmp:mpi-ue-atlas-1}--\ref{fig:cmp:mpi-ue-atlas-3} in the
comparisons section of the review (\SecRef{sec:physics-areas-where}). 

\paragraph{Multiple interactions}
In a hadron-hadron collision more than one pair of partons may
interact, leading to the possibility of multiple interactions.
In Monte Carlo modelling contexts, the most striking and easily
identifiable consequence of multiple interactions is arguably
the possibility of observing several hard parton-parton 
interactions in one and the same hadron-hadron event\footnote{
Additional jet pairs produced in this way are sometimes
referred to as ``minijets'', and theoretically belong to a class of
perturbative  corrections called ``higher twist'', but in the interest
of maintaining a compact terminology, we shall here just call them MPI
jets.}. The main distinguishing feature of such jets is that they tend to form 
back-to-back pairs, with little total \pt. For comparison, jets from 
bremsstrahlung tend to be aligned with the direction of their
``parent'' partons. The fraction of multiple interactions that give
rise to additional reconstructible jets is, however, quite small (how
small depends on the exact jet definition used). 
Additional soft interactions, below the
jet cutoff, are much more plentiful, and can give significant
corrections to the colour flow and total scattered energy 
of the event. 
This affects the final-state activity in a more global way, increasing the
multiplicity and summed transverse energy, and contributing to the
break-up of the beam remnant in the forward direction. 

\begin{figure}[t]
\center
\includegraphics*[scale=0.75]{mc-plots/ATLAS_2010_CONF_2010_046-inline/ATLAS_2010_CONF_2010_046_d03-x02-y01}
\caption{Models with and without MPI and parton showers, compared to
the charged-particle multiplicity measured by the ATLAS
experiment \cite{Atlas:2010xx}, 
% The non-preliminary version: 
%experiment \cite{Collaboration:2010i}, 
for particles with $\pt>100$~MeV, $|\eta|<2.5$, and $c\tau > 10$~mm,
in events that contain at least two such particles. \label{fig:mbnch}}
\end{figure}
To illustrate this we include in \FigRef{fig:mbnch} a comparison
between an ATLAS minimum bias measurement of the charged-track
multiplicity at 7~TeV to a Monte Carlo model with and without MPI
switched on (curves labelled as ``default'' and ``no MPI'',
respectively). Clearly, the predicted multiplicity distribution
without MPI is far too narrow, regardless of whether parton showers
are included or not (curve labelled ``no MPI, no shower''). This by
itself is one of the strongest arguments that MPI must be included in
realistic models of soft-inclusive physics.

The possibility of multiple interactions has also been implicit
or explicit in many calculations of the total hadron-hadron cross
section.  Two recent and representative examples can be found in  
 \cite{Avsar:2008dn,Grau:2009qx}.  
The increase of the parton-parton cross section with CM energy is here
directly driving an increase also of $\sigma_{\mathrm{tot}}(s)$.   

The first detailed Monte Carlo model for 
perturbative MPI was proposed by Sj\"ostrand and van Zijl in 
\cite{Sjostrand:1987su}, and most modern implementations employ a
similar physical picture. Below, in section \SecRef{sec:mbmpi}, 
we therefore first summarize the main points of this basic framework,  
pointing out the differences between the currently existing models 
as we go along. 
Some useful additional references to the history and development of
 the subject of  multiple interactions also outside the Monte Carlo context 
can be found in the Perugia MPI workshop
 proceedings \cite{Bartalini:2010su} and in the mini-reviews contained
 in \cite{Sjostrand:2004pf,Gustafson:2007sb}.

\mcsubsection{Models based on multiple parton interactions (MPI) \label{sec:mbmpi}}
\mcsubsubsection{Basics of MPI \label{sec:mpibasics}}

Consider first the cross section for a \emph{single} parton-parton
scattering, \eg by $t$-channel gluon exchange (Rutherford scattering).
This process, and simple variations of it, make up the vast
majority of the total scattering
processes occurring between coloured particles, and it is thus on
this basic process that perturbative models of both soft inclusive and
underlying event physics are currently built.

An intuitive way of arriving at the idea of multiple interactions
is to view hadrons simply as ``bunches'' of incoming
partons. No physical law then  prevents several distinct pairs of partons
from undergoing scattering processes within one and the same hadron-hadron
collision.  The other key idea to bear in mind is that the exchanged QCD
particles are coloured, and hence such multiple interactions, even when soft, can cause non-trivial changes to the
colour topology of the colliding system as a whole, with potentially
major consequences for the particle multiplicity in the final state.

In the soft QCD region, the $t$-channel gluon propagator almost goes
on shell (reminiscent of the case of bremsstrahlung, described in detail in
\SecRef{sec:parton-showers}), causing the subprocess differential cross
section to become very large, behaving roughly as:
\begin{equation}
\mr{d}\hat{\sigma}_{2j} \propto \
\frac{\mr{d}t }{t^2} \ \sim \
 \frac{\drm\pPerp{}^2}{\pPerp{}^4}~, \label{eq:dpt4}
\end{equation}

\begin{figure}
\begin{center}
\vspace*{-30mm}\includegraphics*[scale=0.35]{minim-bias/qcd2to2}
\caption{The inclusive jet cross section calculated at LO for three
different proton PDFs, compared to various extrapolations
of the non-perturbative fits to the total pp
cross section at 14~TeV centre-of-mass
energy. From \cite{Bahr:2008wk}. \label{fig:sigma2to2}}
\end{center}
\end{figure}
An integration of this cross section from a lower cutoff
$\pPerp{\mr{min}}$ to $\sqrt{s}$, \
using the full (leading-order) QCD $2\to 2$ matrix elements
folded with some recent parton-density sets, is
shown in \FigRef{fig:sigma2to2}, for $pp$ collisions at 14~TeV
\cite{Bahr:2008wk}.
The solid curves, representing the calculated
cross sections as functions of $\pPerp{\mr{min}}$, are
compared to a few of the Donnachie-Landshoff (DL)
predictions \cite{Donnachie:1992ny,Donnachie:2004pi}
for the total $pp$ cross section $\sigma_{\mr{tot}}$,
shown as horizontal lines with  different dashing styles on the same
plot. Physically, the jet cross section can of course not
exceed the total $pp$ one, yet this is what appears to be happening
at scales of order 4--5~GeV in \FigRef{fig:sigma2to2}.
How to interpret this behaviour?

Recall that the interaction cross section is an inclusive
number. Thus, an event with two parton-parton interactions will count
twice in $\sigma_{2j}$ but only once in $\sigma_{\mr{tot}}$,
and so on for higher multiplicities. In the limit that all the
individual parton-parton interactions are independent and equivalent
(to be improved on below), we have
\begin{equation}
\sigma_{2j}(\pPerp{\mr{min}}) = \langle n \rangle(\pPerp{\mr{min}})
\ \sigma_{\mr{tot}} ~,
\end{equation}
with $ \langle n \rangle(\pPerp{\mr{min}})$ giving the
average of the distribution in the number of parton-parton interactions
above $\pPerp{\mr{min}}$ per hadron-hadron collision,
and that number may well be above unity.
This simple argument in fact expresses unitarity;
instead of the total interaction cross section diverging as $\pPerp{\mr{min}}\to 0$ (which would violate
unitarity), we have restated the problem so
that it is now the \emph{number of interactions per collision} that
diverges.

Two important ingredients remain to be introduced in order to fully
regulate the remaining divergence. The first is the correlation due to
energy-momentum conservation --- the interactions cannot use up more
momentum than is available in the parent hadron ---
which will suppress the large-$n$ tail of the na\"ive estimate above.
This is handled slightly differently in the various models on the market.
In the \pythia and \Sherpa models, the
multiple interactions are ordered in $\pPerp{}$, and the parton
distributions for each successive
interaction are explicitly constructed so
that the sum of $x$ fractions can never be greater than unity. In
the \herwigpp model, instead the uncorrelated estimate of
$\langle n \rangle$ above is used directly as an initial guess, but the actual
generation of interactions stop once the energy-momentum conservation
limit is exceeded (with the last ``offending'' interaction also
removed from consideration).

Even with this suppression taken into account, however, the number of
multiple interactions still grows uncomfortably fast as $\pPerp{\mr{min}}\to
0$. A second ingredient suppressing the number of interactions,
at low-\pPerp{} and $x$, is colour screening /
saturation. Screening and saturation both roughly correspond to
partons being unable to resolve each other as independent particles
at low scales, but the underlying physics pictures are slightly
different, as follows.

Screening is interpreted as an effect
of the wavelength $\sim$ $1/\pt$ of the exchanged particle becoming
larger than a typical colour-anticolour separation distance; it will
then only couple to an average colour charge that vanishes in the limit
$p_{\perp} \to 0$, hence leading to suppressed interactions.
This screening effectively provides an infrared cutoff for
MPI similar to that provided by the hadronization
scale for parton showers.
Saturation instead invokes explicit parton
recombination effects to reduce the growth of the parton
densities at low $x$. In either case, the product of cross section and parton densities is reduced. However, an
important modelling distinction is therefore that the reduction takes place
mainly as a function of \pt, for screening, whereas the $x$ variable
plays a central role in saturation arguments. As usual, the truth is
likely to be a combination of both. Since most of the models considered in
this review employ a screening-like cutoff, in \pt, we devote
a bit more space to this possibility here.

A first estimate of an effective lower cutoff due to colour screening would be
the proton size
\begin{equation}
\pPerp{\mr{min}} \simeq \frac{\hbar}{r_{p}} \approx
\frac{0.2~\mr{GeV}\cdot\mr{fm}}{0.7~\mr{fm}} \approx
0.3~\mr{GeV} \simeq \LambdaQCD\,, \label{eq:cutoff}
\end{equation}
but empirically this appears to be too low. In current models, one
replaces the proton radius $r_{p}$ in the above formula by a ``typical
colour screening distance'' $d$, \ie an average size of a region within
which the net compensation of a given colour
charge occurs. This number is not known from first principles, so
effectively this is simply a cutoff parameter, which can then just as
well be put in transverse momentum space.
The simplest choice is to introduce a step function
$\Theta(\pPerp{} - \pPerp{\mr{min}})$, such that the perturbative cross section
completely vanishes below the $\pPerp{\mr{min}}$ scale.
This is the procedure followed in the original \jimmy 
model \cite{Butterworth:1996zw} (an add-on to the Fortran
\herwig generator), whose predictions therefore have
a large dependence on the value of this parameter. The
\herwigpp model \cite{Bahr:2008dy} builds on this, and
improves it by adding a set of
``soft'' scatterings below $\pPerp{\mr{min}}$, with a $\pPerp{\mr{min}}$-dependent
cross section defined such that the two components add up to the
total (inelastic non-diffractive)
cross section \cite{Bahr:2008wk}, which should reduce the explicit
dependence on \pPerp{\mr{min}}.

Alternatively, one may note that the jet cross section is divergent
like $\alphaS^2(\pPerp{}^2)/\pPerp{}^4$, cf.~\EqRef{eq:dpt4},
and that therefore a factor
\begin{equation}
\frac{\alphaS^2(\pPerp{0}^2 + \pPerp{}^2)}{\alphaS^2(\pPerp{}^2)} \,
\frac{\pPerp{}^4}{(\pPerp{0}^2 + \pPerp{}^2)^2}
\label{eq:ptzerodampen}
\end{equation}
would smoothly regularize the divergences, now with $\pPerp{0}$ as the
free parameter to be tuned to data. This is the default in
the current \pythia and \Sherpa models. Note that,
since this merely represents a ``smoothed-out''
variant of the $\Theta$ function cutoff above, there is still a strong
dependence on the value of $\pPerp{0}$. This is thus one of the main ``tuning''
parameters in such models.

The parameters do not have to be energy-independent, however.
Higher energies imply that parton densities can be probed at smaller
$x$ values, where the number of partons rapidly increases. Partons
then become closer packed and the colour screening distance $d$
decreases. Just as the small-$x$ rise varies like some power of $x$
one could therefore expect the energy dependence of $\pPerp{\mr{min}}$ and
$\pPerp{0}$ to vary like some power of the centre-of-mass energy. Explicit toy simulations
\cite{Dischler:2000pk} lend some credence to such an ansatz, although with
large uncertainties. One could also let the cutoff increase
with decreasing $x$; this would lead to a similar phenomenology since
larger energies probe smaller $x$ values. Especially for models with
strong dependence on this cutoff, the uncertainty on this
energy and $x$ scaling of the cutoff is a major concern when
extrapolating between different collider energies.

As an alternative we therefore
note that the introduction of so-called unintegrated parton
densities, as used in the BFKL \cite{Kuraev:1977fs,Balitsky:1978ic}, CCFM
\cite{Ciafaloni:1987ur,Catani:1989sg}, and LDC
\cite{Andersson:1995ju,Andersson:1997bx,Kharraziha:1997dn}
approaches to initial-state radiation allows the possibility of
replacing the $\pPerp{\mr{min}}$ or $\pPerp{0}$ cutoff by parton densities that
explicitly vanish in the $\pPerp{} \to 0$ limit
\cite{Gustafson:2002jy}. This allows the possibility of an
alternative implementation of multiple interactions
\cite{Gustafson:2002kz} and may be a useful ingredient in future
phenomenological models, possibly in combination with more explicit
physical modelling of saturation effects.

\mcsubsubsection{Impact parameter dependence}
As mentioned in \SecRef{sec:mbtypes}, the so-called ``pedestal
effect'' (see also
\FigsRef{fig:cmp:mpi-ue-atlas-1}--\ref{fig:cmp:mpi-ue-atlas-3})
is partly driven by impact parameter dependence; in peripheral
collisions, only a small fraction of events contain any
high-\pt\ activity, whereas central collisions are more
likely to contain at least one hard scattering; a sample with a
high-\pt\ selection cut will therefore be biased towards small impact parameters.
The ability of a model to describe the shape of the pedestal (\eg to
describe both minimum bias data and underlying event distributions
simultaneously) is therefore related to its modelling of the
impact parameter dependence.
A related effect is that, also for a fixed selected \pt, events at
comparatively higher impact parameters should exhibit relatively less underlying
event and vice versa.

All the models discussed here contain an explicit treatment of
impact parameter, but we note that there are still
substantial simplifications made. Most importantly, the
impact parameter dependence is so far still assumed to be factorized from
the $x$ dependence, $\pdf{}(x,b)=\pdf{}(x)g(b)$, where $b$ denotes impact
parameter, a simplifying assumption that by no means should be treated
as inviolate, see \eg
\cite{Hagler:2007xi,Treleani:2007gi,Blok:2010ge}. Also, the
hadron-hadron impact parameter only enters in an averaged global
sense, not as a vector, and the individual MPI are not assigned
individual ``locations'' in transverse space.

In order to quantify the concept of hadronic matter overlap, one may
assume a spherically symmetric distribution of matter inside a
hadron at rest, $\rho(\mb{x}) \, \drm^3 x = \rho(r) \, \drm^3 x$.
The form of $\rho$ is a matter of some uncertainty, with various
more or less phenomenologically motivated choices available in
models. The options range from simple parameteric forms in
\pythia-based models, such as Gaussians, double Gaussians,
exponentials \cite{Sjostrand:1987su}, and forms interpolating between
them  \cite{Sjostrand:2004pf}, to a form based on the  electromagnetic
form factor in the \herwig-based ones
\cite{Forshaw:1991gd}. A possibility for future model refinements
thus lies in the input of more detailed information on the flavour-
or $x$-dependence of the transverse structure of the proton, \eg
obtained from sum rules, from analytic fits beyond the EM form factor,
or from lattice studies.

For a collision with impact parameter $b$, the time-integrated
overlap $\mathcal{O}(b)$ between the matter distributions of the
colliding hadrons is given by
\begin{equation}
\mathcal{O}(b) \propto \int \drm t \int \drm^3 x \, \,
\rho(x,y,z) \, \rho(x+b,y,z+t)     ~,
\end{equation}
where the necessity to use boosted $\rho(\mb{x})$ distributions
has been circumvented by a suitable scale transformation of the $z$
and $t$ coordinates, see \cite{Sjostrand:1987su}. The overlap function
$\mathcal{O}(b)$ is identical
to $A(b)$ in ``\jimmy notation''
\cite{Butterworth:1996zw,Bahr:2008dy}. It is
closely  related to the $\Omega(b)$ of eikonal models
(see, for example, \cite{Bourrely:2002wr,Borozan:2002fk,Treleani:2007gi}),
but is somewhat simpler in spirit.

The larger the overlap $\mathcal{O}(b)$ is, the more likely it is to
have interactions between partons in the two colliding hadrons.
In fact, to first approximation, there should be a linear relationship
\begin{equation}
\langle \tilde{n}(b) \rangle = k \mathcal{O}(b) ~,
\label{eq:bdepend}
\end{equation}
where $\tilde{n} = 0, 1, 2, \ldots$ counts the number of interactions
when two hadrons pass each other with an impact parameter $b$ and
$k$ is an undefined constant of proportionality, to be specified
below.

For each impact parameter, $b$, the number of interactions $\tilde{n}$
can be assumed to be distributed according to a Poissonian,
modulo momentum conservation, with the mean value of the Poisson
distribution depending on impact
parameter, $\langle \tilde{n}(b)\rangle$. If the matter
distribution has a tail to infinity (as, e.g., Gaussians do),
one may nominally obtain events with arbitrarily large $b$ values.
In order to obtain finite total cross sections, it is therefore
necessary to give a separate interpretation to the ``zero bin'' of the
Poisson distribution, which corresponds to ``no-interaction'' events.

In the \jimmy \cite{Butterworth:1996zw} and \herwigpp
\cite{Bahr:2008dy} models the part of the $pp$ cross section
containing hard scatters is calculated from the area
overlap function, the parton densities and the partonic cross section;
the ``no-interaction'' possibility is then accounted for as a
reduction of this cross section with respect to its value without
allowing for MPI. The \jimmy model stops here, considering only hard events,
and so it can only be applied to underlying event. As mentioned above,
the \herwigpp model also permits the possibility of soft scatters
(see also \SecRef{sec:herwigmpi}) and so can also be used to
simulate soft-inclusive physics.

In the framework of \cite{Sjostrand:1987su}, used by \pythia and
\Sherpa, the restriction to at least one perturbative scattering for soft
inclusive scatters implies that
the probability that two hadrons, passing each other
with an impact parameter $b$, will produce a real event is given by
\begin{equation}
\mathcal{P}_{\mr{int}}(b) =
\sum_{\tilde{n} = 1}^{\infty} \mathcal{P}_{\tilde{n}}(b) =
1 - \mathcal{P}_0(b) =
1 - \exp ( - \langle \tilde{n}(b) \rangle )
= 1 - \exp (- k \mathcal{O}(b) ) ~,
\end{equation}
according to Poisson statistics. The average number of
interactions per event at impact parameter $b$ is now
$\langle n(b) \rangle = \langle \tilde{n}(b) \rangle /
\mathcal{P}_{\mr{int}}(b)$, where the denominator comes from the
removal of hadron pairs that pass without interaction, \ie which do
not produce any events. While the removal of $\tilde{n}=0$ from the potential
event sample gives a narrower-than-Poisson interaction distribution
at each fixed $b$, the variation of $\langle n(b) \rangle$ with $b$
gives a $b$-integrated broader-than-Poisson interaction multiplicity
distribution.

Averaged over all $b$ the relationship $\langle n \rangle =
\sigma_{2j}/\sigma_{\mr{nd}}$ should still hold. Here, as before,
$\sigma_{2j}$ is the integrated interaction cross section for a given
regularization prescription at small \pt, while the inelastic
non-diffractive cross section $\sigma_{\mr{nd}}$ is taken from
parameterization \cite{Donnachie:1992ny,Donnachie:2004pi,Schuler:1993wr}.
This relation can be used to solve for the
proportionality factor $k$ in \EqRef{eq:bdepend}. Note that, since
now each event has to have at least one interaction, $\langle n
\rangle > 1$, one must ensure that $\sigma_{2j} >
\sigma_{\mr{nd}}$. The $\pPerp{0}$ parameter has to be chosen
accordingly small.

\mcsubsubsection{Perturbative corrections beyond MPI}
There are essentially two perturbative modelling aspects which go
beyond the introduction of MPI themselves. In particular, this concerns
\begin{enumerate}
\item Parton showers off the MPI.
\item Perturbative parton-rescattering effects.
\end{enumerate}

Without showers, MPI models would generate very sharp peaks for back-to-back
MPI jets, caused by unshowered partons passed directly to the hadronization
model. However, with the exception of the oldest \pythia~6 model
\cite{Sjostrand:1987su}, all of the models discussed in this
review do include such showers, and hence should exhibit more
realistic (\ie  broader and
more decorrelated) MPI jets --- although not much can be said concerning
their expected formal level of precision of course. A secondary effect is that a
showered interaction also generates a larger hadronic
multiplicity than an unshowered one. Therefore, a smaller total number
of MPI is needed when tuning models incorporating such showers.
More discussion of this tuning effect can be found in
\cite{Buttar:2008jx,Skands:2010ak}.
On the initial state side of the MPI shower issue the main questions
are whether and how correlated multi-parton densities
are taken into account (for a recent treatment of this issue see \eg
\cite{Gaunt:2009re} and references therein), and, as discussed
previously, how the showers are regulated at low \pt (or low $x$).
Although none of the Monte Carlo models currently
impose a rigorous correlated multi-parton evolution, all of them
include some elementary aspects. The most significant for
parton-level results is arguably momentum conservation, which is
enforced explicitly in all the models, although in slightly different
ways, as was discussed briefly above. The so-called ``interleaved'' models
\cite{Sjostrand:2004pf,Sjostrand:2004ef} attempt to go a step
further, generating an explicitly correlated multi-parton evolution
in which flavour sum rules can be imposed to conserve \eg
the total numbers of valence and sea quarks across interaction chains.

Perturbative rescattering in the final state can occur if
partons are allowed to undergo several distinct interactions, with
showering activity possibly taking place in between. This has so far
not been studied extensively, but a first
fairly complete model and exploratory study has been presented in the
context of \pythia~8 \cite{Corke:2009tk}.
The net effect there is a slight increase in the mean $p_\perp$ of the
partonic final states, but more dramatic signatures have not yet been
identified. In the initial state, parton rescattering effects may lead
to saturation (discussed briefly in \SecRef{sec:mpibasics}),
or to the incoming partons carrying enhanced
``primordial $k_\perp$'' values (discussed in
\SecRef{sec:primkt}). It could also produce  correlations
between different MPI initiators, in particular in colour
space. At the \emph{parton level}, however, such
colour correlations probably play a rather minor role.
This is suggested both by an exploratory study of ``intertwined''
multiple interactions \cite{Sjostrand:2004ef} (that is,
letting several partons from the same shower chain undergo
perturbative interactions, thus letting the perturbative shower
evolution generate their colour correlations)
which numerically found only very small effects, and by a
more heuristic argument; that the multiple interactions are likely to
be taking place slightly displaced from each other in
space-time. Their ``perturbative cross-talk'' should therefore be
suppressed by wave-function overlap
factors, which mean they should only be able to emit coherently at
rather small $p_\perp$ anyway.
Compared to the usual perturbative subleading-colour ambiguities
associated with parton shower Monte Carlos, this particular source of
colour space ambiguity should therefore not represent any significant
additional ambiguity \emph{at the parton level}. The interleaved
rescattering model of \cite{Corke:2009tk} is currently the only one to
address emissions inside colour dipoles spanned
\emph{between} different MPI subsystems.

\mcsubsubsection{Non-perturbative aspects\label{sec:mbmpi-np}}

Consider a hadron-hadron collision at a
resolution scale of about 1~GeV. The system of coloured partons emerging
from the short-distance phase (primary parton-parton interaction plus
parton-level underlying event plus beam-remnant partons)
must now undergo the transition to colourless hadrons.

In this context, it is useful to consider what happens to
infrared safe, and infrared sensitive, observables separately.
For infrared safe observables, such as
energy flow and jet observables, the parton flow in phase space
already gives quite a good
approximation, and hadronization only gives small corrections. (For
precision studies, these must of course still be taken into account, see \eg
\cite{Bhatti:2005ai,Dasgupta:2007wa}.)

Infrared sensitive observables, on the other hand, such as individual
hadron multiplicities and spectra are crucially dependent on the
parton-parton correlations in colour space, and on the properties and
parameters of the hadronization model used, pedagogical
descriptions of which can be found in \SecRef{sec:hadronization}.
Here, we concentrate on the specific issues connected with the
structure  of the underlying event in hadron collisions.

Keeping the short-distance parts unchanged, the colour structure
\emph{inside} each of the MPI systems is normally still described using
just the ordinary leading-colour matrix-element and parton-shower
machinery described in \SecsRef{sec:subprocesses} and
\ref{sec:parton-showers}. The crucial question, in the context of MPI,
is then how colour is neutralized \emph{between} different MPI
systems, including also the remnants. Since these systems can lie at
very different rapidities (the extreme case being the two opposite
beam remnants), the strings or clusters spanned between them can have
very large invariant masses (though normally low~\pt),
and give rise to large amounts of (soft)
particle production. Indeed, in the context of soft-inclusive physics,
it is precisely these ``inter-system'' clusters/strings which furnish the
dominant particle production mechanism (cf.\ again
\FigRef{fig:mbnch}), and hence their modelling
is an essential part of the infrared physics description.
For more on
the physics of the string and cluster hadronization models, see
\SecRef{sec:hadronization}.

\begin{figure}
\begin{center}
\includegraphics*[scale=0.7]{minim-bias/mb-col1}\\
\caption{Left: example of colour assignments in a $p\bar{p}$ collision with two
  interactions. Explicit colour labels are shown on each propagator
  line. Right: The string/cluster topology resulting from the colour
  topology on the left, with horizontal and vertical axes
  illustrating transverse momentum and rapidity, respectively. On the
  right, the two undisturbed valence quarks in each of the beam
  remnant are represented as ``diquarks'' carrying the (anti-)baryon
  number of the original beam particles. \label{fig:justring}}
\end{center}
\end{figure}

On the left-hand side of
\FigRef{fig:justring} we give a simple sketch of what a $p\bar{p}$
collision containing two distinct parton-parton interactions might
look like in (planar) colour space. The additional complications of parton
showers and further MPI have been suppressed, so
that we can focus entirely on the correlations \emph{between} the two
scatterings. Tracing the colour lines in this diagram and connecting
each colour-connected parton pair in the final state by a dashed line
results in the sketch shown to the right of the colour-flow
diagram. In current hadronization models, each of these dashed lines
would represent a string piece, or a cluster, as appropriate.
The vertical axis roughly represents rapidity, with the
beam-remnant partons at either extreme. The horizontal axis is
intended to illustrate $p_\perp$. Thus, the $q'\bar{q}'$ pair
from the primary parton-parton interaction
(furthest to the left in the colour-flow
diagram) are depicted at high $p_\perp$ and central rapidity, while the partons
from the secondary interaction, $\bar{q}_{\mr{val}1}g$
are depicted at larger rapidities, with
smaller transverse momenta.

\begin{figure}
\begin{center}\vspace*{2mm}%
\hspace*{-1mm}\includegraphics*[scale=0.7]{minim-bias/mb-col2}\\
\caption{The same momentum and perturbative colour-flow
configuration as in \FigRef{fig:justring}, but with the second gluon
extracted from the proton now attached to a different quark line than
that of the first gluon. The corresponding string topology reflects
the change and now has become more complicated.
 The baryon number of the proton beam has been ``liberated'', as represented by
  the $\Delta$ symbol towards the middle of the diagram.
\label{fig:justring2}}
\end{center}
\end{figure}

The ambiguity in the colour correlations can be illustrated by
comparing with \FigRef{fig:justring2} which shows exactly the same
perturbative momentum and colour-space configuration, but now with the
second gluon extracted from the proton attached to a different
beam-remnant quark line than that of the first gluon. The
corresponding string topology reflects the change and now has become
more complicated.
The baryon number of the proton beam has been ``liberated'', as represented by
the $\Delta$ symbol towards the middle of the graph. As an aside, this
also illustrates that the baryon number can migrate to the central
region despite all the valence quarks remaining in the beam
remnant; see \cite{Sjostrand:2004pf} for an explicit Monte Carlo model of this effect.
When adding more perturbative parton-parton interactions
this ambiguity grows, and there appears to be very little one can say
about it from perturbation theory.

In the \herwig-based models the colour correlations are set up
by first forcing the initiator of the primary parton-parton
interaction to be a valence quark at low $Q^2$. The initiators of the
subsequent MPI are then forced to be gluons,
which are attached at random to the primary-interaction valence quark
initiator. Since only one quark line is ``disturbed'' in this process,
the beam baryon number remains in the remnant, with the two
undisturbed valence quarks forming a ``diquark'' system
(see \SecRef{sec:hadronization}).

In the old \pythia model the colour flow of the primary
interaction is likewise taken as the basic skeleton onto which the
underlying event interactions are added. In that model, the MPI
final states can either be $q\bar{q}$ pairs, which form isolated
single string pieces, or $gg$ pairs, which either form a closed colour
loop or are inserted to give kinks on the existing primary-interaction
colour topology in a way that minimizes the total string length
(equivalent to minimizing a measure
of the classical potential energy). Again, since at most one quark in
the beam remnant is directly ``disturbed'', the beam baryon number
remains in the remnant. Empirically, the Tevatron data
appear to prefer almost exclusively $gg$ pairs that
minimize the string length. This is indicative of very strong colour
correlations between the MPI final states, although the model does not
address the physical origin of them. (We return to this point
below, under colour reconnections.)

In the new \pythia framework, described in \cite{Sjostrand:2004pf},
a more elaborate modelling of beam remnants was introduced, that
allowed the tracing of colour flow in more detail and also allowed
the beam baryon number to migrate away from the remnant. However,
also in this modelling context, empirical comparisons to the Tevatron
minimum bias data appear to require stronger colour correlations
between the MPI final states than those naively generated by the
model.

This brings us from colour connections, to colour
\emph{reconnections}, which we shall discuss in the next section.

\mcsubsection{Colour reconnections \label{sec:colrec}}
In a first study of colour rearrangements,
Gustafson, Pettersson, and Zerwas (GPZ) \cite{Gustafson:1988fs}
observed that, \eg in hadronic $WW$ events at LEP, colour interference
effects and gluon exchanges may cause `crosstalk' between the two
$W$ systems, leading \eg to uncertainties in the $W$ mass
determination.
In the GPZ picture, the corresponding changes occurred
already at the perturbative QCD level, leading to predictions
of quite large effects. Sj\"ostrand and Khoze (SK)
\cite{Sjostrand:1993rb,Sjostrand:1993hi}
subsequently argued against large perturbative effects
and instead considered a non-perturbative scenario in which
QCD strings can fuse or cut each other up (see \eg
\cite{Artru:1979ye}). These models resulted in
effects much smaller than for GPZ, leading to a predicted total uncertainty
on the $W$ mass from this source of $\sigma_{M_{W}}<40~\mathrm{MeV}$.

Subsequently, several alternative models have been proposed,
most notably by the Lund group, based on QCD dipoles
\cite{Gustafson:1994cd,Lonnblad:1995yk,Friberg:1996xc}, and by Webber based
on clusters \cite{Webber:1997iw}. Apart from $WW$ physics,
colour reconnections have also been proposed to model rapidity gaps
\cite{Buchmuller:1995qa,Edin:1995gi,Rathsman:1998tp,Enberg:2001vq}
and quarkonium production \cite{Edin:1997zb}.

Experimental investigations of colour reconnections at LEP
 \cite{Abbiendi:1998jb,Abbiendi:2005es,Schael:2006ns,
Abdallah:2006ve} were able to exclude at least the
most dramatic scenarios, such as GPZ and
extreme versions of SK with the recoupling strength parameter close to
unity, but more moderate scenarios have not been excluded.
Furthermore, in hadron collisions the initial state contains soft
colour fields with wavelengths of
order the confinement scale. The presence of such fields,
unconstrained by LEP measurements, could impact in a
non-trivial way the process of colour  neutralization
\cite{Buchmuller:1995qa,Edin:1995gi}. And finally,
the MPI produce an additional amount of displaced colour charges,
translating to a larger density of hadronizing systems. It is not
known to what extent the collective hadronization of such a system
differs from a simple sum of independent systems, as will also be briefly
mentioned in \SecRef{sec:hadronization} on hadronization.

A new generation of colour-reconnection toy models have therefore been
developed specifically with soft-inclusive and underlying event
physics in mind \cite{Sandhoff:2005jh,Skands:2007zg,Skands:2010ak},
and also the cluster-based \cite{Webber:1997iw} and Generalized-Areal-Law
\cite{Rathsman:1998tp} models have been revisited in that context.
Although still quite crude, these models do appear to be able to
describe significant features of the Tevatron and LHC data, such as
the distribution of the mean \pt\  of charged particles vs. the number of charged particles, $\langle \pt \rangle(N_{\mr{ch}})$.
\begin{figure}[t]
\center
\includegraphics*[scale=0.75]{mc-plots/ATLAS_2010_CONF_2010_046-inline/ATLAS_2010_CONF_2010_046_d03-x04-y01}
\caption{Models with and without colour reconnections compared to
the $\langle \pt \rangle (N_{\mr{ch}})$ distribution measured by the ATLAS
experiment \cite{Atlas:2010xx}, 
% The non-preliminary version: 
%experiment \cite{Collaboration:2010i}, 
for particles with $\pt>100$~MeV, $|\eta|<2.5$, and $c\tau > 10$~mm,
in events that contain at least two such particles. \label{fig:mbptofnch}}
\end{figure}
To illustrate this, we include in \FigRef{fig:mbptofnch} a comparison
between an ATLAS minimum bias measurement of the $\langle \pt \rangle
(N_{\mr{ch}})$ distribution at 7~TeV and two Monte Carlo models,
with and without colour reconnections switched on (curves labelled as
``default'' and ``no CR'', respectively). Without colour
reconnections, the predicted  $\langle \pt \rangle
(N_{\mr{ch}})$ distributions appear to rise too slowly with $N_{\mr{ch}}$.

It is nonetheless clear
that the details of the full fragmentation process in hadron-hadron
collisions are still far from completely understood.

\mcsubsection{Diffraction and models based on pomerons \label{s_ue:pomerons}}

Essentially, the MPI-based models discussed above start from
 perturbatively calculable cross sections and attempt to extend these
 to low $p_\perp$ by a combination of resummations of soft
 perturbative effects and  explicit modelling of
 non-perturbative effects.

It is also possible to start from a non-perturbative standpoint,
using unitarity to relate elastic and inelastic scattering processes
 through the optical theorem. In this language, the total cross
 section is driven by the exchange of ``reggeons'' and ``pomerons''
 --- colour-singlet fluctuations with leading
$f\bar{f}$ and $gg$ contents, respectively ---
 with the latter dominating at high energies.

In this picture diffractive events originate from collisions between
an effective flux of such (virtual) colour-singlet objects within the beam
particles, leading to a characteristic spectrum that varies roughly
like $dM^2/M^2$, with $M$ the invariant mass of the diffractively
excited system. (For
multiple diffraction, the behaviour is a product of such factors.)
An important modelling aspect
is whether the pomerons are considered to have a substructure
themselves. If so, partonic collisions between pomeron
fluctuations can generate high-mass diffractive processes
such as diffractive dijets and other hard central exclusive
processes. If not, only the  $dM^2/M^2$ spectrum is present.

Inelastic events are understood in terms of cut pomerons, which
furnishes a relation between diffractive and non-diffractive
scattering that is absent in the MPI-based models discussed
above. The relation between MPI, diffraction, and pomerons is usefully
discussed in \cite{Treleani:2007gi}.

Translated into the terminology used in this review,
each cut pomeron corresponds
to the exchange of a soft gluon, which results in two `strings'
being drawn between the two beam remnants. Uncut pomerons give
virtual corrections that help preserve unitarity. A variable number
of cut pomerons are allowed, which furnishes the equivalent of
MPI in this language. However,
note that cut pomerons were originally viewed as purely soft objects,
and so generated only transverse momenta of order
$\LambdaQCD$, unlike the multiple interactions considered
above. In \textsc{Dtujet} \cite{Aurenche:1994ev},
\textsc{Phojet} \cite{Engel:1994vs,Engel:1995yda} and \textsc{Dpmjet}
\cite{Ranft:1994fd,Roesler:2000he}, however,
also hard interactions have been included, so
that the picture now is one of both hard and soft pomerons, ideally
with a smooth transition between the two.
The three programs all make use of the Lund string hadronization
description, however, and hence share the fundamental properties and
ambiguities of this part of the
modelling with the string-based MPI models discussed
above.

\mcsubsection{Summary}
\begin{itemize}
\item Several parton-parton interactions can occur within
  a single hadron-hadron collision. This is called multiple
  parton interactions (MPI).
\item The hard perturbative tail of MPI is approximately proportional
  to $\drm\pt^2/\pt^4$. This tail produces additional observable
  (pairs of) jets  in the  underlying event. The oft repeated mantra
  that the underlying event is non-perturbative is thus a misconception.
\item Most MPI \emph{are} relatively soft, however, and do not lead to easily
  identifiable additional jets. Instead, they contribute to building
  up the total amount of scattered energy and cause colour exchanges
  between the remnants, thereby increasing the number of particles
  produced in the hadronization stage.
\item Hadron-hadron collisions at small impact parameter
  have a higher number of MPI than peripheral ones. What the enhancement
  factor is depends on the shape of the hadron transverse mass
  distribution.
\item A hadron-hadron collision with a large number of MPI has a
  higher probability of containing a hard jet than one with few
  MPI. This produces the ``pedestal effect''.
\item The number of MPI is regulated by colour screening and
  saturation effects. The detailed behaviour of this regularization,
  including its dependence on collider centre-of-mass energy, is
  poorly known and represents one of the main uncertain / tunable
  aspects of the models.
\item In addition to the MPI $2\to 2$ scatterings,
  realistic models must also incorporate showers off the MPI, to
  describe the broadening and decorrelation of MPI jets.
\item It is also possible to include perturbative rescattering
  effects, but this is so far not available in all models.
\item At the non-perturbative level, the assumed structure of the beam
  remnant can be important, \eg affecting the event structure at large
  rapidities and migration of the beam baryon number in rapidity.
\item There are significant ambiguities concerning colour-space
  correlations, in particular \emph{between} the various MPI
  systems. In current models some amount of \emph{colour
    reconnections} appear to be necessary to properly describe
  minimum bias and underlying event data.
  This is probably the most poorly understood part
  of the modelling, however, and is associated with significant
  uncertainties.
\item The distinction between the  diffractive and non-diffractive
  components of the total inelastic hadron-hadron cross section is
  fundamentally ambiguous and must be interpreted with care, as discussed
  in \SecRef{sec:mbtypes}. This issue is also discussed in
  \PartRef{sec:comp-gener-with}, there from the point of view of
  measurements.
\item Diffractive processes are typically modelled as a separate class
  of processes driven by the exchange of so-called pomerons. These can
  be viewed either as purely soft objects or as having an internal
  partonic substructure. The latter provides a mechanism for
  generating high-mass diffractive processes such as diffractive dijets.
\end{itemize}
% Local Variables:
% mode: LaTeX
% TeX-master: "../mcreview"
% End:

