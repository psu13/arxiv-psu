\mcsubsection{\gensectionintro}

Historically, \herwigpp is based on the event generator \Herwig 
(Hadron Emission Reactions With Interfering
Gluons), which was first published in 1986 \cite{Marchesini:1987cf} and was
developed throughout the era of LEP, with the latest major release
version 6.5.10 \cite{Corcella:2000bw,Corcella:2002jc} in
2005\footnote{Version 6.5.20, released in 2010, contains bug fixes.}. From the
beginning it has featured angular ordered parton showers to take colour
coherence effects into account.  The cluster hadronization model it uses
(\SecRef{sec:cluster-model}) was developed at the same time.

\herwig was written  in \fortran, but
with the advent of the LHC it was decided to freeze its development and
develop a new generator, with the same strengths as the old program, in
\cpp.  The idea was to not just rewrite the generator but to introduce
physics improvements whenever they seemed necessary and feasible.  The
new generator, \Herwigpp, was first released only for $e^+e^-$
annihilation in 2003 \cite{Gieseke:2003hm}.  Since then it was
further developed into a complete event generator for collider
physics \cite{Bahr:2008tf}, with the current version 2.4.2 released in 2009.
The code and its physics features are fully documented in the main
reference \cite{Bahr:2008pv}, which will be updated as the code
develops continually. Some distinctive physics features of \Herwigpp are:
\begin{itemize}
\item Automatic generation of hard processes and decays with full spin
  correlations for many BSM models.
\item Matching of many hard processes at NLO with the POWHEG method
  built in. 
\item Angular ordered parton showers. 
\item Cluster hadronization.
\item Sophisticated hadronic decay models, particularly for bottom
  hadrons and $\tau$ leptons. 
\item Hard and soft multiple partonic interactions to  model the
  underlying event and soft inclusive interactions. 
\end{itemize}
We will describe the most important details of the physics models in the
remainder of this section.

\mcsubsection{ThePEG}
\label{sec:thepeg}

\Herwigpp is distributed as a comprehensive collection of plugin
modules to \thepeg, the Toolkit for High
Energy Physics Event Generation \cite{Lonnblad:2006pt}. 
\thepeg provides all the infrastructure that
is necessary to construct an event generator, handling \eg random
number generation, the event record, tuneable parameter settings, and
most importantly, a mechanism to plug in physics implementations for
all steps of event generation. \Herwigpp provides such a set of
plugins and comes with several complete generator setups for $e^+e^-$,
$ep$ and hadron-hadron collisions. 

\thepeg's core component is the Repository, which holds the relations
between all the different modules involved in a generator run and
their tuneable parameter settings. It can be controlled through a
simple command language in plain text, which is used to set up the
modules involved in a generator run.  Using such files at run time,
the user can override any of the default parameters that \Herwigpp
comes with; no recompilations are necessary to change parameters, or to
switch between physics models, different matrix elements or analyses.

\thepeg provides a reader for the Les Houches Accord event format
\cite{Alwall:2006yp} to read in parton-level events for further processing, an
output module for HepMC events \cite{Dobbs:2001ck}, as well as a native
interface to Rivet \cite{Buckley:2010ar}, which avoids the overhead of having
to pipe events through text files. Additionally, \thepeg can be linked
to LHAPDF \cite{Whalley:2005nh} to get direct access to any PDF sets that are
available there.

The repository plugin structure allows for easy inclusion of user-defined
modules. Any \cpp object that inherits from the respective base
classes in \thepeg can be used transparently in addition to, or instead
of, one of the default plugins. Any user code can be loaded at
runtime as dynamically linked libraries. This allows modification
of the program's behaviour without having to recompile the 
main program or needing to edit the core libraries.
They can therefore always be installed centrally, possibly as part of a larger framework.


\mcsubsection{\gensectionhard} 
Three main mechanisms for
simulating hard processes are available 
in \herwigpp. First, there is a large set of
hand-coded matrix elements for the most common subprocesses for hadron,
lepton and DIS collisions. They are written using a reimplementation
of the HELAS helicity amplitude formalism \cite{Murayama:1992gi}, which allows
the spin correlations to be carried forward to the remaining event
simulation consistently. Second, \Herwigpp also contains a generic
matrix element calculator for $2 \to 2$ processes, mainly used for BSM
physics, which automatically determines the permitted diagrams for a
set of given external legs from a list of active vertices. The third
source of hard subprocesses is \thepeg's Les Houches reader, which
allows parton-level events with any number of legs to be read from
external sources.

%%% \paragraph{diffraction}

For several processes, \Herwigpp incorporates the full NLO corrections in
the parton shower \cite{Hamilton:2008pd,Hamilton:2009za,Hamilton:2010mb}
using the POWHEG formalism (\SecRef{sec:powheg}). An implementation of
the CKKW merging scheme for tree-level multi-jet events
(\SecRef{sec:matching-at-tree}) will be included in an upcoming
release \cite{Hamilton:2009ne}.

\mcsubsection{BSM physics}\label{sec:hwbsm}
The simulation of BSM physics in \herwigpp
\cite{Gigg:2007cr,Gigg:2008yc} makes extensive use of \thepeg's plugin
architecture. Each model is implemented in a model class, which
holds the relevant new parameters, and a list of Feynman rules for its
vertices. Based on this information, all possible production and decay
matrix elements with up to four external legs are constructed and can
be selected in the text-based input files. 
\Herwigpp currently provides models to simulate processes in the MSSM
\cite{Haber:1984rc,Gunion:1984yn} and NMSSM \cite{Ellwanger:2009dp}
scenarios with an SLHA \cite{Skands:2003cj,Allanach:2008qq}
 file reader to provide the relevant parameters,
a model for universal extra dimensions
\cite{Cheng:2002iz,Appelquist:2000nn},
an implementation of
Randall-Sundrum \cite{Randall:1999ee}
and ADD-type gravitons \cite{ArkaniHamed:1998rs,Antoniadis:1998ig},
 as well as a model of transplanckian
scattering \cite{Giudice:2001ce}.

The production and decay matrix elements are all calculated using
helicity amplitude techniques so that spin correlations between the
production and decay of unstable particles can be generated using the
approach of \cite{Richardson:2001df}, as described in
\SecRef{sec:bsm-general-purpose}.  This ensures that \Herwigpp can
generate the spin correlations for individual decay chains in a
computationally efficient way, while still allowing the simulation of
inclusive BSM signals. The efficiency comes at the expense of
neglecting interference effects with other decay chains leading to the
same final state.  Off-shell effects -- including the suppression of
decay modes close to threshold -- are simulated using the approach of
\cite{Gigg:2008yc}, which includes the running width of the unstable
particle in the denominator of the Breit-Wigner propagator and in the
calculation of the production matrix element for the particle.


\mcsubsection{\gensectionshower}

The parton shower in \Herwigpp is based on a new evolution variable
$\tilde q$ \cite{Gieseke:2003rz}, motivated from the branching of gluons
off heavy quarks \cite{Catani:2002hc}.  This is one of the possible
choices in \EqRef{eq:evolchoices}.  As in \Herwig, the evolution in this
variable ensures the angular ordering of parton shower emissions,
to take colour coherence effects into account 
(see \SecRef{parton-shower:initial-conditions}). In addition to the treatment
of mass effects in the splitting functions and the showering of
coloured particles in BSM models,
 the shower differs from \Herwig's implementation in the way it
 fills the
available phase space for emissions.  Considering only the first gluon
emission from a $q\bar q$ pair, the new
variable fills the soft gluon emission region of phase space 
without any overlap between the parton showers. 

A so-called dead region is still present in the phase space, as in
\Herwig, but is filled by either a hard matrix element correction or by
higher order emissions.  Potential discontinuities in the emission phase
space at the transition from the parton shower to the hard emission
region are avoided by applying a so-called soft matrix element
correction: the emission rates in the parton shower overestimate the
rates one would obtain from a full matrix element calculation. For each
parton shower emission, the overestimated rate is then corrected down to
the matrix element by a veto (see \AppRef{mcmethods:veto}) 
which reflects the relative emission
probability between parton shower and matrix element, respectively.  In
\Herwigpp there are matrix element corrections for $e^+e^-\to q\bar q$,
Drell-Yan production of vector bosons in hadronic collisions, $gg\to
h^0$ and for top decays.  In order to fill the phase space smoothly, it
should be noted that the starting scales of the parton shower are
adjusted to the values that are given by the requirement of
colour coherence, see
\EqRef{eq:psinitialconditions}.  In \herwigpp, these
initial conditions cannot be
altered, \eg by raising the initial evolution scale. 
Before the parton shower generates emissions, all heavy unstable
particles in the partonic
final state, which typically have a very narrow width, are decayed.  
All intermediate coloured lines are then also showered.
These decays are done for the Higgs particle,
electroweak gauge bosons, top quarks, and also \eg
supersymmetric particles (see \SecRef{sec:hwbsm}). 


\mcsubsection{\gensectionMPI}
\label{sec:herwigmpi} 

The default model for the simulation of
underlying
event physics is a model for multiple partonic interactions
\cite{Bahr:2008dy}.  Both hard and soft multiple partonic interactions
are taken into account.  The hard interactions are modelled as hard QCD
$2\to 2$ scatters with a transverse momentum above the cutoff value
$\pt^{\rm min}$.  The hard scattering centres are thought to be
spatially distributed within the proton similarly to the charge,
as measured in elastic electron--proton scattering, leading to the
dipole form factor.  However, a different width for the
distribution of colour charges, quarks and gluons, parameterized by the inverse
radius $\mu^2$, is allowed. 
%
In the region $0<\pt<\pt^{\rm min}$, soft scatters are generated
with a transverse momentum distribution that has a Gaussian form, has an
integral given by the soft parton-parton cross section, and is
continuously matched with the perturbative distribution at $\pt=\pt^{\rm min}$
\cite{Borozan:2002fk}. These constraints are sufficient to uniquely specify
this distribution. The soft partons' longitudinal momentum
distribution is taken to be like that of the low-energy sea,
\ie $\xpdf{}(x) \approx \text{flat}$.
The spatial distribution of soft colour charges, given by a
parameter $\mu_{\rm soft}^2$, is allowed to be different from the hard ones as
otherwise it was not possible to obtain a model that is consistent with Tevatron
data on multiple scattering \cite{Bahr:2008wk}.  The soft cross
section and $\mu_{\rm soft}^2$ are fixed from measurements of the total
cross section and the elastic slope parameter, if available, or
parameterizations of them otherwise.

After initial studies, good agreement with Tevatron underlying event
measurements from Run I and II were found \cite{Bahr:2008dy}.  With
the availability of first LHC data, e.g\ \cite{Aad:2010rd,Atlas:2010xx}
it became clear that the model suffered from the lack of a colour
reconnection mechanism, which will be included in an upcoming release.
It gives a very satisfactory description of these hard underlying event
data.

\mcsubsection{\gensectionhadronize}

\Herwigpp uses the cluster hadronization model, described in
\SecRef{sec:cluster-model}.  Its first step is a non-perturbative gluon
splitting, where each gluon splits
isotropically in its rest frame into a $q\bar q$ pair of
one of the three lightest flavours. We stress that at this stage all
partons are treated as non-perturbative objects and acquire a
constituent mass.  The value of the gluon mass in particular is one of
the important model parameters.  After cluster formation we are left
with a small number of heavier clusters of mass $M$, that will fission
in binary sequential decays, whenever the condition
\begin{equation}
  \label{eq:clfission}
  M^p \geq M_{\rm max}^p +(m_1+m_2)^p
\end{equation}
is fulfilled, where $m_{1, 2}$ are the masses of the constituent partons
of the cluster and $M_{\rm max}$ and $p$ are the main parameters of the
cluster hadronization model, chosen independently for light,
charmed and bottom clusters.  Once a cluster is split, a new
particle-antiparticle pair of quarks or diquarks is taken out of the
vacuum, chosen with adjustable weights.  The kinematics of the new clusters
preserve the original directions of the constituent particles and depend
on whether they contain a perturbative parton or a beam remnant.  Once
clusters fall below the limit of \EqRef{eq:clfission}, they
decay isotropically in their rest frames 
into pairs of hadrons.  The hadron species
are determined according to available phase space and phenomenological
weights for flavour multiplets.  As heavier baryons tend to be
suppressed in this approach \cite{Kupco:1998fx}, the choice between a
baryonic or non-baryonic decay is made before the hadron
species are selected.  In some cases clusters will turn out to be too
light to decay into a pair of hadrons; they will decay into a
single light particle instead and share some momentum with a 
cluster close by in spacetime.
For any cluster that contains a parton from the original hard
process, \eg a bottom quark, the resulting heavy meson retains the
original parton direction in the cluster rest frame, up to some Gaussian
smearing.

In addition to the hadronization of partonic final states, the
implementation of the model in \Herwigpp can also handle stable coloured
particles or baryon number violating vertices, which both occur in BSM models.

As with all tuneable parameters, a detailed list and description can be found
in the manual \cite{Bahr:2008pv}.


\mcsubsection{\gensectiondecay}

The decays of both fundamental particles and unstable hadrons in
\Herwigpp are modelled in the same framework, using either a general
matrix element based on the spin structure of the decay, or with a
specific matrix element for important decay modes, with a particular
emphasis on baryon decays. This allows for a
sophisticated treatment of off-shell effects, the treatment of excited
baryonic multiplets, and for example the easy
integration of the semileptonic $\tau$ lepton
decays~\cite{Grellscheid:2007tt}. Spin correlation effects are
included fully for the decays of all unstable
particles~\cite{Richardson:2001df} and are
consistent with the preceding stages of event generation all the way
back to the production matrix element. QED radiation in decays is
simulated using the YFS formalism
\cite{Hamilton:2006xz} (see \SecRef{sec:qed-radiation}). 
All the decay matrix elements have been extensively tested against
external packages where available, and are in full agreement. 

The particle properties such as masses, widths, lifetimes, decay modes
and branching ratios that are used in \Herwigpp can be found
in the online interface to its database of particle properties at
\begin{center}
\href{http://www.ippp.dur.ac.uk/~richardn/particles/}{\tt
  http://www.ippp.dur.ac.uk/$\sim$richardn/particles/}
\end{center}

\mcsubsection{Outlook}

\Herwigpp in its current version (2.4.2) is 
superior to its \fortran predecessor in almost all aspects of
physics simulation and is the recommended version for new
studies. Once this is true without any exceptions the version 
number will move to 3.0.  The program package, together
with \thepeg can be found at 
\begin{center} 
\texttt{http://projects.hepforge.org/herwig/}
\end{center}


% Local Variables: 
% mode: LaTeX
% TeX-master: "../mcreview"
% End: 
