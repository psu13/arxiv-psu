  We do not know what kind of physics beyond the Standard Model may be encountered
  at the LHC;  if any is found, a variety of new physics models will need to be considered in 
  order to determine its exact nature. Despite the large number of
  models, they can be split into two broad 
  classes:\footnote{There are some scenarios such as Little 
                    Higgs~\cite{ArkaniHamed:2002qx,ArkaniHamed:2002qy}
                    and Leptoquark models which are intermediate 
                    between the two cases with only a small number of
                    additional particles.}
\begin{enumerate}
\item models that contain either new effective operators which 
      modify the cross sections and distributions for Standard Model
      processes, or only a few new particles which are generally
      produced as resonances, \eg the 
      ADD~\cite{ArkaniHamed:1998rs,Antoniadis:1998ig} or
      Randall-Sundrum~\cite{Randall:1999ee} extra-dimensional models;
\item models that contain a large number of new particles, often new partners
      for each Standard Model particle, which can be produced in a variety of
      ways at the LHC and then decay, for example the Minimal Supersymmetric
      Standard Model~(MSSM), Universal Extra Dimension
      models~(UED)~\cite{Appelquist:2000nn,Cheng:2002ab} or
      Little Higgs models with T-parity~\cite{Low:2004xc,Hubisz:2004ft}.
\end{enumerate}
  In general the first class of models are relatively simple to simulate with only
  minor changes to the Standard Model production processes that are
  present in all general purpose event generators. The simulation of the second
  class of models is more complicated.
  There are two approaches that 
  have been adopted to simulate these models:
\begin{enumerate}
\item the production of the new heavy particles is simulated first, usually using a
      leading-order $2\to2$ scattering process, followed by the subsequent decay of
      the heavy particles, which often leads to long decay chains as the 
      heavier BSM particles cascade decay into lighter ones;
\item a high multiplicity matrix element including all the final-state partons
      is used to simulate the process including the decays of any unstable
      heavy particles.
\end{enumerate}
  The first approach has the advantage of both computational simplicity
  and being able to easily simulate the fully inclusive BSM signal. However,
  while the second approach is more computationally expensive it has the advantage
  of correctly treating unstable intermediate particles and any correlation effects.

  Methods have therefore been developed to allow all the correlation 
  effects to be retained, in the approximation that only resonant diagrams
  are included and all interferences are neglected, while still simulating
  the production and decay of heavy particles 
  separately~\cite{Collins:1987cp,Knowles:1987cu,Knowles:1988hu,Knowles:1988vs,Richardson:2001df}. Generally
  when using such methods the masses of the heavy particles
  are smeared using the Breit-Wigner distribution, although more sophisticated
  techniques have been developed~\cite{Gigg:2008yc}. Currently \herwigpp and \pythiaeight use
  the first of the above approaches, with \herwigpp using the methods of
  Refs.~\cite{Richardson:2001df,Gigg:2007cr,Gigg:2008yc} 
  to include spin correlations and off-shell effects.
  This also has the advantage that QCD radiation from new coloured
  particles can be simulated more easily.
  As \sherpa includes
  a sophisticated matrix element generator, it currently uses the second approach.

  Historically models of new physics were implemented directly in the Monte Carlo
  event generators by hard coding the production and decay matrix elements.
  In recent years this has changed, with both \sherpa and \herwigpp using a method
  where the production processes and decays are automatically calculated from
  the Feynman rules, for arbitrary processes in \sherpa and for $2\to2$ scattering
  processes and $1\to2$ or 3 decays in \herwigpp. It has also become increasingly
  common to use an external matrix element generator interfaced via the Les
  Houches Accord~\cite{Boos:2001cv,Alwall:2006yp}
  to simulate the hard scattering process. The most recent 
  development is the \FeynRules~\cite{Christensen:2008py} package which
  can automatically calculate the Feynman rules in a given model from the Lagrangian
  in a form that can be used by a matrix element generator. \sherpa already uses
  this approach to allow a large range of models to be simulated and work is
  in progress to use it with \herwigpp.

  While in most cases models of new physics only require the simulation of
  the hard process, any subsequent decays, and the QCD radiation from the heavy particles,
  recently a number of more exotic models have been proposed where
  the nature of the new physics leads to changes in other parts of the Monte
  Carlo simulation. In general this occurs when the new model involves colour 
  structures which do not occur in the Standard Model. 
  Three situations have arisen.

  Firstly, in R-parity violating SUSY models baryon number can be violated
  by a new operator which 
  couples three particles in the fundamental, or anti-fundamental, representation
  of $SU(3)_C$ via the total antisymmetric tensor $\epsilon^{ijk}$. This
  can be considered as a junction where three
  colour lines meet. This presents a problem both in the selection of 
  the colour partners for the parton shower evolution and later in the hadronization
  stage. The simulation of these models, with the angular-ordered parton
  shower and cluster hadronization model~\cite{Gibbs:1994cw,Dreiner:1999qz} and
  in the string model~\cite{Sjostrand:2002ip}~(see \SecRef{sec:string-model}),
  has been studied in detail with
  the selection of the colour partners for the perturbative radiation being done 
  at random from among the potential partners, and after the shower a 
  special treatment of three partons colour connected to the 
  junction.
  
  Secondly, in hidden valley
  models~\cite{Strassler:2006im,Strassler:2006qa} there are particles
  that are charged under a new strongly interacting gauge group.
  In these models, radiation of the gauge bosons of the new strong force
  by the new particle must be simulated both in the parton shower phase,
  as in~\cite{Carloni:2010tw}, and in the subsequent hadronization.
  
  Finally, some models have recently been proposed in which there
  are particles in representations of the $SU(3)_C$ group of the strong
  force other than those we know how to simulate (the fundamental and adjoint),
  for example particles in the sextet 
  representation~\cite{Chen:2008hh,Han:2009ya,Berger:2010fy}. The simulation
  of these particles is not currently possible in any of the general-purpose
  event generators.

  In summary:
\begin{itemize}
\item most BSM models can be simulated either by incorporating changes to 
      Standard Model production processes or by adding the production
      and decay of the new particles in the specific model;
\item in general the production and decay of new particles are simulated
      separately in order to generate exclusive production processes;
\item if new colour structures are present the parton shower and
      hadronization phases must also be modified.
\end{itemize}
 
% Local Variables: 
% mode: LaTeX
% TeX-master: "../mcreview"
% End: 
