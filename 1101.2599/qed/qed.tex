The simulation of electromagnetic radiation in general-purpose event generators
uses one of two approaches. The most common is to use the same parton
shower algorithm that was used for the simulation of QCD radiation. Indeed this 
is the preferred option for processes where the emission of
both QCD and QED radiation is possible. The simulation
of QED radiation proceeds in a similar way as for QCD radiation, with the evolution
partner selected according to the charge, rather than the colour flow. This
can cause problems in some processes where there are destructive contributions
that would be suppressed by $1/\Nc^2$ in QCD, but which are leading in QED.
Despite these problems this is the most common approach in Monte Carlo simulations
as both QED and QCD radiation can be generated at the same time.
This interleaving of both types of radiation in one shower gives
interesting phase space competition effects and could be used
to shed light on the parton shower ordering
variable\cite{Seymour:1994bx}.

An alternative to the parton shower is the Yennie-Frautschi-Suura (YFS) formalism~\cite{Yennie:1961ad}
which proceeds by exponentiating the full eikonal distribution for
soft photon emission, below a cut-off,
together with the corresponding virtual corrections, given by the YFS form factor.
In this approach, starting with the production of
$n$ particles, the cross section with the radiation of
an additional $n_\gamma$ photons can be written as 
\begin{eqnarray}
\sigma  
&=& 
\displaystyle{\frac{(2\pi)^4}{2\hat{s}}
\prod_{i=1}^{n} \frac{{\rm d}^3p_i}{(2\pi)^32p_i^0}
|\overline{\mathcal{M}}|^2
\delta^4\left(l_1+l_2-\sum_{i=1}^np_i-\sum_{i=1}^{n_\gamma}k_i\right)}
\label{eqn:YFSmaster} \\
&&\displaystyle{\times
\sum^\infty_{n_\gamma=0}\frac1{n_\gamma!}\prod_{j=1}^{n_\gamma}\int 
\frac{{\rm d}^3k_j}{k_j^0}\tilde{S}_{\rm total}(k_j)}
 e^{Y_{\rm total}(\Omega)},
\nonumber
\end{eqnarray}
where $p_i$ are the momenta of the outgoing particles, $k_i$ are those
of the outgoing photons, $l_i$ those of the incoming partons and
 $|\overline{\mathcal{M}}|^2$ is the 
spin summed/averaged matrix element for the leading-order process.
  The total dipole radiation function is
\begin{equation}
\tilde{S}_{\rm total}(k) = \sum_{i=0}^n\sum_{j=1,j>i}^n \frac{\alpha Z_i\theta_iZ_j\theta_j}{4\pi^2}
\left(\frac{p_i}{p_i\cdot k}-\frac{p_j}{p_j\cdot k}\right)^2,
\end{equation}
  where $Z_{i,j}$ is the charge of the $i,j^{\rm th}$ particle in units 
  of the positron charge and \mbox{$\theta_{i,j}=+1(-1)$} if the $i,j^{\rm th}$ particle is
  outgoing~(incoming).

 The total YFS form factor~\cite{Yennie:1961ad}, $Y_{\rm total}(\Omega)$,
  is a sum of contributions from pairs of charged particles:
\begin{equation}
Y_{\rm total}(\Omega) = \sum_{i=0}^n\sum^n_{j>i} Y_{ij}(p_i,p_j,\Omega),
\end{equation}
  where $\Omega$ is used to symbolically indicate the dependence
  on the infrared cutoff on the photon energy.
 The YFS form factor for a pair of charged pairs is given by
\begin{equation}
Y_{ij}(p_i,p_j,\Omega) = 2\alpha
\left(\mathcal{R}e B_{ij}(p_i,p_j)+\tilde{B}_{ij}(p_i,p_j,\Omega)\right).
\end{equation}
  The real emission piece, $\tilde{B}_{ij}$, is
\begin{equation}
\tilde{B}_{ij}(p_i,p_j,\Omega) =
\frac{Z_i\theta_iZ_j\theta_j}{8\pi^2}\int_0^{\left|\mathbf{k}\right|<\omega}
\frac{{\rm {d}}^3k}{\left|\mathbf{k}\right|}
\left(\frac{p_i}{k\cdot p_i}-\frac{p_j}{k\cdot p_j}\right)^2,
\end{equation}
  where $\omega$ is the upper limit on the photon energy.
  The virtual piece does not depend on the cutoff and is given by
\begin{equation}
B_{ij}(p_i,p_j) = -\frac{iZ_i\theta_iZ_j\theta_j}{8\pi^3}
\int {\rm d}^4k \frac1{k^2}
\left(
\frac{2p_i\theta_i-k}{k^2-2k\cdot p_i\theta_i}+
\frac{2p_j\theta_j+k}{k^2+2k\cdot p_j\theta_j}\right)^2.
\end{equation}

The standard technique to generating photons according
to \EqRef{eqn:YFSmaster} works in two stages.
First, the distribution is generated according to the leading-order result 
in which each photon is produced independently.
A correction weight is then applied in order to give exactly the
distribution in \EqRef{eqn:YFSmaster}.
The major advantage of this technique is that because the distribution
used to generate the additional photons is known analytically, higher
order corrections can be included exactly. It is this feature which allowed the construction of high precision 
Monte Carlo simulations for LEP 
physics~\cite{Jadach:1988gb,Jadach:2000ir,Jadach:1999vf,Placzek:2003zg,Jadach:2001mp,Jadach:2001uu}
and is included in \sherpa for initial-state photon radiation in lepton collisions~\cite{Schalicke:2002ck}.


In the general-purpose event generators the parton shower
approach is used in the majority of perturbative processes, where
both QCD and QED radiation must be generated. However, both
\herwigpp~\cite{Hamilton:2006xz} and \sherpa~\cite{Schonherr:2008av}
include the simulation of QED radiation using the YFS formalism
in cases where no QCD radiation is possible, \ie
for the leptonic decays of $W^\pm$ and $Z^0$ bosons, hadron and tau decays. 
In particular, the latter two applications simplify the decay tables
considerably since many decay modes are produced by adding photons to
simpler modes.
In the previous generation of Monte Carlo simulations the production
of QED radiation in particle decays was normally simulated using an interface to the PHOTOS 
program~\cite{Barberio:1993qi,Barberio:1990ms,Golonka:2005pn}. This program
is based on the collinear approximation for the radiation of 
photons together with corrections to reproduce the correct result in the soft
limit~\cite{Barberio:1993qi,Barberio:1990ms}. Recently it has been improved 
to include the full next-to-leading order QED corrections for certain 
processes~\cite{Golonka:2005pn}. However, given the inherent problems
with interfacing to external programs, the superior accuracy of the YFS
formalism and the ability to systematically improve it, 
in \herwigpp and \sherpa the YFS approach is preferred.

In summary:
\begin{itemize}
\item QED radiation can be simulated using either a parton shower or
      YFS based approach;
\item historically the parton shower approach has been more common in
      general-purpose event generators and is still used when both QED and
      QCD radiation is possible;
\item both \herwigpp and
      \sherpa now use the YFS formalism for the simulation of
      QED radiation in particle decays.
\end{itemize}
% Local Variables: 
% mode: LaTeX
