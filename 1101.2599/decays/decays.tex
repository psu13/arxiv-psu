
 Following the hadronization phase of event generation a number of unstable
 hadrons are produced, which must be decayed into particles that are
 stable on collider timescales.
 This is an important part of the event
 simulation, because the observed final-state hadrons result
 from a convolution of hadronization and decay, so that a particular set
 of tuned hadronization parameters is applicable only in combination
 with a particular decay package.

 Simulating hadron decays involves non-trivial modelling.
 At first glance it might seem that all the information
 needed to simulate these decays is readily available in the Particle Data Group~(PDG)'s
 Review of Particle Physics~\cite{Amsler:2008zzb}, but the information
 on particle properties in the PDG is often insufficient and numerous choices have to be made.
 This is particularly true for members of excited meson multiplets, excited mesons containing
 heavy~(bottom and charm) quarks, and baryons containing heavy quarks.
 The number of choices which have to be made
 increases as more excited meson and baryon multiplets
 are added to the simulation.

 The first choice that must be made is which hadrons to include in the simulation.
 This choice is generator specific and is closely connected with
 the tuning of hadronization parameters.  In the cluster model in particular it
 is important that all the light members\footnote{The hadrons containing only up, down and strange quarks.}
 of a multiplet are included, as the absence of members can lead to isospin or $SU(3)$ flavour
 violation at an unphysical rate. All the general-purpose event
 generators include the lightest pseudoscalar, vector, scalar, even and odd charge conjugation
 pseudovector, and tensor multiplets of light mesons. In addition, some excited
 vector  multiplets of light mesons are often included. Usually the mesons containing
 a single heavy quark, or heavy quarks with different flavours,
 from the same multiplets are included, although particularly
 for bottom mesons the properties of the excited mesons are taken from theoretical models
 rather than the PDG. A large number of states containing $c\bar c$ or $b\bar b$ have been
 observed and usually most of these states are included in the simulation, with the
 exception of some recently discovered particles for which the quark model interpretation
 is unclear. While a large number of mesons are normally included, usually only the
 lightest octet, decuplet and singlet baryons are present, although both \herwigpp and
 \Sherpa now include some heavier baryon multiplets. Although a number of
 baryons containing two heavy quarks have been observed these are not generally
 included in the standard generators as their production is rare.

 Having selected the hadrons to use in the simulation, the choice of which decay modes
 to include and how to simulate them is closely related. Consider the example
 of the $a_1$ meson, which decays to three pions, $a_1\to\pi\pi\pi$, where
 the dominant contribution takes place via an intermediate rho meson. If
 we choose
 to use a simple simulation without matrix element or off-shell effects, then
 this decay is best simulated as $a_1\to\rho\pi$ followed by the decay of the
 rho meson to two pions. However, if a matrix
 element for the decay is included it is better to generate the three-body
 decay including
 the effect of the intermediate rho, and other suppressed contributions,
 in the matrix element for the decay.

 Historically, the standard generators included few matrix elements for
 hadron decays and at best used a na\"ive Breit-Wigner smearing of the masses
 of the particles. More sophisticated simulation of hadronic decays
 was then performed using specialized external packages such as
 \evtgen~\cite{Lange:2001uf} for hadron decays and
 \tauola~\cite{Golonka:2003xt,Jadach:1993hs,Jadach:1990mz} for tau decays.
 This still holds true for \pythiaeight, while \herwigpp and \sherpa
 now include much better
 simulation of hadronic, and particularly tau lepton, decays. This was primarily
 motivated by the need to provide a better description of spin effects in tau decays.
 The perturbative production mechanism of the tau can have observable
 effects on its decay properties, which can be used to probe the properties of
 the Higgs boson and particles in BSM models. This is facilitated by using
 the same approach for both the perturbative and non-perturbative
 decays.

  The decays of different types of hadron, and the tau lepton, are simulated
  in a variety of different ways.
  The light mesons and baryons which decay via the weak
  interaction typically have long lifetimes and therefore these decays do not
  need to be simulated in high energy collisions.
  The remaining strong and electromagnetic decays of the
  light mesons are normally simulated using simple matrix elements based
  on parity and charge conjugation invariance.  
  It is important that modes with relatively low branching
  ratios, for example pion Dalitz decay $\pi^0\to e^+e^-\gamma$, are included
  as although they rarely occur for a single particle they can
  contribute significantly
  given the large rate for the production of light mesons.

  The simulation of light baryon decays is often the most primitive part of the simulation,
  particularly in the external decay packages, as these were originally developed
  to simulate events at the B-factories, where baryons are rarely produced. 
  The new hadron decay models  have
  significant improvements for the simulation of baryon decays, typically using
  simple matrix elements based on the relevant conservation laws in the same way as for
  the light mesons.

  While not a hadron, due to its mass the tau lepton primarily decays
  semi-leptonically to a tau neutrino and a small number of light
  mesons.\footnote{The semi-leptonic branching ratio of the tau lepton
  is approximately 65\% with the remaining 35\% being fully leptonic decays.}
  This can be simulated as the decay of the tau lepton to its associated
  neutrino and a virtual $W$ boson. The matrix element can be written as
\begin{equation}
\mathcal{M} = \frac{G_F}{\sqrt{2}}\,L_\mu\,J^\mu,\qquad
L_\mu       = \bar{u}(p_{\nu_\tau})\,\gamma_\mu(1-\gamma_5)\,
        u(p_{\tau}),
\label{eqn:taudecay}
\end{equation}
  where $p_\tau$ is the momentum of the $\tau$ and $p_{\nu_\tau}$ is the momentum of the
  neutrino produced in the decay. The information on the decay products of the
  virtual $W$ boson is contained in the hadronic current, $J^\mu$.
  These currents are calculated either for low-energy effective theories or
  fits to experimental data. The currents for a large number of
  decays, from both modern theoretical models and experimental fits, are included
  in the most recent simulations of
  tau decay~\cite{Golonka:2003xt,Jadach:1993hs,Jadach:1990mz,Grellscheid:2007tt,Gleisberg:2008ta}.
  In some hadron decay models these currents are also used to simulate
  the weak decay of heavy mesons and baryons in the na\"ive factorization
  approximation~\cite{Wirbel:1985ji,Bauer:1986bm}.

  In recent years there has been a lot of interest in the decays of charm and, especially,
  bottom mesons motivated by the study of the CKM matrix and CP violation at the B-factories
  and the Tevatron. This has led to the development of detailed simulations, in particular the
  \evtgen package~\cite{Lange:2001uf}, for these decays. However, this 
  package mainly concentrates on the simulation of $B^0$--$\overline{B}^0$ mixing and rare
  $B$-meson decays that are of interest for the study of CP violating phenomena. 

  While a large number of inclusive decay modes of the weakly decaying mesons containing a 
  single charm or bottom quark have been observed, the branching ratios for these modes
  are insufficient to account for all the decays. The simulation of these decays
  therefore uses a combination of:
\begin{itemize}
\item a number of inclusive, generally low multiplicity, decays
      simulated using either a phase-space distribution, or matrix elements based on 
      na\"ive factorization or experimental fits;
\item partonic decays of the heavy quark, for example $b\to c\ell^-\bar\nu_\ell$, followed
      by the hadronization of the partonic final state including the spectator quark
      using the hadronization models described in
      \SecRef{sec:hadronization} to simulate the remaining observed decay modes.
\end{itemize}
  This approach for the simulation of heavy meson decays 
  is sufficient in most collider physics applications.
  However, the simulation of the oscillations of $B^0_d$ and $B_s^0$ mesons
  and CP violation in $B$-meson mixing and decays are needed for both
  $B$-physics studies and some other applications that are sensitive to 
  mixing phenomena.
  
 \herwigpp and \pythiaeight include the oscillation of neutral
  $B$-mesons using the probability for the meson to oscillate into its antiparticle
  before it decays.
  \sherpa and \evtgen use a more sophisticated simulation including
  CP-violating effects and, for common decay modes of the neutral meson
  and its antiparticle, the interference between the direct decay and
  oscillation followed by decay.

  While a number of decay modes of the weakly decaying charm baryons are known,
  very few weak decays of bottom baryons have been observed and, with the exception
  of the $\Lambda_c^+$, only ratios of the branching ratios are known. The simulation
  of the decays of the weakly decaying heavy baryons therefore uses a 
  very small number of inclusive modes together with partonic decays for
  the majority of the decays.

  A number of excited charm, and in the recent years, bottom mesons
  have been observed, although the properties of a number of the excited
  bottom mesons are uncertain.
  The strong and electromagnetic decays of excited bottom
  and charm mesons are normally
  treated in the same way as the decays of the light mesons, \ie
  using simple matrix elements based on the relevant conservation laws. 

  While a number of charm baryons which decay via either electromagnetic or
  strong interactions have been observed, only the $\Sigma_b$ and $\Sigma_b^{*}$ bottom
  baryons, decaying via the electromagnetic and strong
  interactions respectively, have been observed.
  In general the baryons containing a single heavy quark required to 
  complete the octet and decuplet baryon multiplets are included, although
  for many of the strongly decaying particles the masses and decay modes are based
  on theoretical models or the properties of the corresponding charmed baryons,
  rather than experimental results.

  The decay rates of bottom- and charm-onium resonances to $\ell^+\ell^-$ and various
  partonic final states can be computed in terms of the quarkonium wavefunction, which 
  is calculated in various models. As knowledge of the wavefunction is only needed to
  compute the width, which is taken from experimental results in event generators, the
  matrix elements can be used to simulate the decays of quarkonium states. In practice
  the simulation of the exclusive decays of these resonances is usually supplemented
  with the inclusion of a number of observed low multiplicity decay modes in
  a similar way as for weakly decaying charm and bottom hadrons.

  \evtgen, \herwigpp and \sherpa include spin correlations between different decays
  in all hadron decays
  where matrix elements are used to calculate the distributions of the decay
  products, whereas \pythiaeight only includes correlations in certain decay chains.
  All the simulations include at least the generation of the masses of
  unstable particles according to the Breit-Wigner distribution, with improvements
  in some simulations for particles where new decay modes become kinematically
  accessible close to the particle's mass.

In summary:
\begin{itemize}
\item the simulation of hadron decays is based on a combination of
      experimental results and theoretically motivated assumptions which
      are required in order to generate exclusive events;
\item the modern simulations of hadron decay are sophisticated, including matrix elements
      for many modes and spin correlations;
\item given the close relationship between the hadron decay and hadronization models
      care should be taken when changing the hadron decay model, unless the
      hadronization parameters are retuned.
\end{itemize}

% Local Variables:
% mode: LaTeX
% TeX-master: "../mcreview"
% End:
