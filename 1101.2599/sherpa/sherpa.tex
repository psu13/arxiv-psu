
\mcsubsection{\gensectionintro}
\label{Sec:intro_sherpa}
\Sherpa is a general-purpose event generator, capable of simulating
the physics of lepton-lepton, lepton-hadron, and hadron-hadron collisions 
as well as photon induced processes. Unlike the programs
 \Ariadne, \Herwig and \Pythia, it was constructed from the beginning in 
\cpp, and in contrast to the \cpp versions of those programs some of the
physics modules (such as the old parton shower, encoded in 
\Apacic, or the matrix element generator \Amegic) were established
before the actual event generation framework.  The construction paradigm
of the \Sherpa framework can be summarized as follows:
\begin{itemize}
\item {\em emphasis on strict modularity of physics modules}\\
  In fact the organization is such that physics modules are only
  connected through relatively unspecific event phase handlers, which in turn
  call interfaces to the underlying physics modules.  These interfaces
  are constructed such that they can connect to various independent
  modules performing the same tasks.  A prime example in \Sherpa is
  the treatment of hard matrix elements, where various ME generators
  (see below) are available to \Sherpa, but all of them accessible 
  through one and the same {\tt Matrix\_Element\_Handler}.  This allows 
  a comparably simple replacement of outdated modules, for instance 
  the old \Apacic parton shower.  
\item {\em bottom-up approach}\\
  The event organization within \Sherpa is kept as simple as possible.
  In particular, there are no abstract overheads for possible 
  event phases when there exists no corresponding physics module yet.
\end{itemize}
Traditionally \Sherpa's main focus is on the perturbative event phase; 
\Sherpa is a frontrunner in the automated generation 
of tree-level matrix elements and hosts two fully-fledged ME generators with
highly advanced phase-space integration methods.  In recent years, the scope
there has widened to also include infrastructure to support the calculation
of cross sections at NLO accuracy, by providing automated subtraction methods.
In addition, the cornerstone of \Sherpa's event simulation, from the beginning,
was the multijet merging described in \SecRef{sec:matching-at-tree}.
Only quite recently the description of parton showering in \Sherpa has been 
improved by the inclusion of a parton shower based on Catani-Seymour
subtraction \cite{Schumann:2007mg} and the development of a true dipole 
shower \cite{Winter:2007ye}, the latter still awaiting full incorporation 
into the framework.  Similarly, in the beginning hadronization in \Sherpa
was performed through an interface to the Fortran version of \Pythia and
only in recent years a new independent implementation of the cluster
hadronization idea \cite{Winter:2003tt}, see \SecRef{sec:cluster-model}, 
has been added. Other additions include a complete model of hadron and 
$\tau$ decays and QED final-state radiation \cite{Schonherr:2008av} and 
a simulation of the underlying event based on the multiple-parton 
scattering ideas of \cite{Sjostrand:1987su}.

\mcsubsection{\gensectionhard}
\label{Sec:hard_process_sherpa}
\paragraph{Tree-level matrix-element generators}

Processes simulated by \Sherpa are selected by defining initial and final 
states of the hard subprocess, generated by the matrix element generator chosen, 
see below.  These initial and final state also define the particles that 
will actually appear in the event record, following the philosophy outlined 
in \SecRef{sec:phys-obj-mc-truth}.\footnote{
  It should be noted that in \Sherpa projections on intermediate states
  in various schemes (narrow width or propagator, both with full spin
  correlations) are also available; these intermediate states, however, will
  typically {\em not} appear in the event record.
}
To generate the cross sections for the hard subprocesses, \Sherpa provides 
two built-in matrix-element generators, \Amegic~\cite{Krauss:2001iv} and 
\Comix~\cite{Gleisberg:2008fv}, as well as facilities for hard-coded matrix 
elements, see \AppRef{sec:app_mcs}.

\Amegic is a Feynman diagram based generator that constructs tree-level 
amplitudes and suitable phase-space mappings from given sets 
of interaction vertices. The Feynman diagrams then get translated into 
helicity amplitudes using an algorithm similar to the one described in 
\cite{Kleiss:1985yh,Ballestrero:1992dv} and extended to include also 
spin-two particles in \cite{Gleisberg:2003ue}. The list of supported 
physics models covers:
\begin{itemize}
  \item the complete Standard Model,
  \item extension of the SM by a general set of anomalous triple and
    quartic gauge couplings \cite{Appelquist:1980vg,Appelquist:1993ka},
  \item extension of the SM by a single complex scalar
	\cite{Dedes:2008bf},
  \item extension of the SM by a fourth generation,
  \item extension of the SM by an axigluon
	\cite{Pati:1975ze,Hall:1985wz,Frampton:1987dn,Frampton:1987ut,Bagger:1987fz},
  \item Two-Higgs-Doublet Model,
  \item Minimal Supersymmetric Standard Model \cite{Hagiwara:2005wg}, 
  \item ADD model of large extra dimensions
    \cite{ArkaniHamed:1998rs,Antoniadis:1998ig}.
\end{itemize}
Other new physics models can easily be invoked by providing model 
representations generated with the \FeynRules 
program~\cite{Christensen:2008py,Christensen:2009jx}. Based on the 
information of all Feynman diagrams contributing to a given process,
\Amegic automatically constructs suitable phase-space mappings. For the
actual integration all contributing channels are combined in a self-adaptive 
multi-channel integrator, see \AppRef{Sec:PS_ME}, which automatically
adjusts to the relative  importance of the single phase-space maps to
minimize the variance. The efficiency of the integrator is further 
improved by applying the self-adaptive \Vegas \cite{Lepage:1980dq} 
algorithm on single phase-space maps.

\Comix is especially suited for the simulation of highest-multiplicity 
processes. This generator is based on an extension of the colour-dressed 
Berends-Giele recursive relations 
to the full Standard Model, see \AppRef{Sec:TL_ME}. Within \Comix any 
four-particle vertex of the Standard Model is decomposed into 
three-particle vertices. This leads to a significantly improved 
performance for large final-state multiplicities, compared to \Amegic. 
The summation (averaging) over colours in QCD and QCD-associated processes 
is performed in a Monte Carlo fashion and colour-ordered amplitudes can 
therefore be computed. Following the reasoning of~\cite{Duhr:2006iq}, 
the colour-flow basis is employed throughout the code. As discussed 
in~\cite{Maltoni:2002mq}, this yields a certain correspondence between 
the large-$\Nc$ limit employed in parton-shower simulations and full QCD 
results, which is especially useful in the context of a merging with the 
parton shower, see \SecRef{sec:me-nlo-matching}. 


\paragraph{Next-to-leading order event generation}
\label{Sec:NLO_ME_Sherpa}
The \Amegic matrix-element generator has the further functionality to 
construct dipole-subtraction terms and their integrals over the one-parton 
emission phase space in the Catani--Seymour formalism \cite{Catani:1996vz}
for arbitrary Standard Model processes \cite{Gleisberg:2007md}. 
When supplemented with corresponding one-loop amplitudes, using the 
Binoth-Les-Houches-Accord \cite{Binoth:2010xt} interface structure, \Sherpa 
is capable of generating parton-level events at next-to-leading order
precision, see \SecRef{sec:subprocesses:NLOcross_sections}. This framework
has for example been used to evaluate the QCD NLO corrections to $W/Z+3$jets
\cite{Berger:2009zg,Berger:2009ep,Berger:2010vm} and $W+4$jets 
\cite{Berger:2010zx} production, with the loop amplitudes obtained from 
{\sc BlackHat} \cite{Berger:2008sj}. In \cite{Binoth:2009wk} 
the NLO corrections to $ZZ+$jet production have been calculated for the 
first time, relying on {\sc Golem} \cite{Binoth:2005ff} for the 
generation of the loop-amplitude expressions. 

Besides the implementation of the Catani--Seymour dipole subtraction 
method, facilitating parton-level event generation at NLO, \Sherpa also 
provides the possibility to generate hadron-level events at NLO accuracy 
using the \POWHEG algorithm to combine NLO matrix elements with the \Sherpa 
parton shower. This is achieved in a completely process-independent way, 
using a reformulation of the original \POWHEG method, which was presented 
in~\cite{Hoeche:2010pf}. The respective \POWHEG generator is based 
on the same principles as the internal parton-shower module described in 
the following section.

\mcsubsection{\gensectionshower}
\label{Sec:shower_sherpa}
\Sherpa's default parton-shower algorithm, first presented in 
\cite{Schumann:2007mg}, is based on the Catani--Seymour 
dipole factorization formalism \cite{Catani:1996vz,Catani:2002hc}.
The underlying key idea is to derive the corresponding shower splitting 
operators from the four-dimensional unintegrated dipole-subtraction terms 
by performing the large-$\Nc$ limit and summing and averaging over all 
spin degrees of freedom. Accordingly, one arrives at a completely factorized 
approximation for the real-emission process in terms of the underlying Born 
channel times a sum of suitable splitting operators that correctly account
for (quasi-)collinear and soft emissions. 

The emerging shower picture corresponds to sequential splittings of dipoles 
where, in the Catani--Seymour formulation, a dipole is made up of the actual 
parton that is supposed to split and a well-defined spectator parton that 
is colour-connected to the emitter. Four dipole configurations have to be 
considered, classified by the emitter/spectator being either in the final 
(F) or initial (I) state; FF, FI, IF and II. All dipole configuration are 
treated on an equal footing and as a consequence there is no formal distinction 
between initial- and final-state parton showers. Successive emissions are 
ordered in terms of the invariant transverse momentum between final-state 
splitting products or with respect to the emitting beam particle. At present 
the Catani--Seymour dipole shower in \Sherpa implements all QCD splittings 
in the Standard Model and the MSSM as well as QED photon emissions 
\cite{Hoeche:2009xc}.

In its original formulation presented in \cite{Schumann:2007mg}, the recoil 
strategy for the various types of dipole splittings closely followed the 
choice of the Catani--Seymour formalism \cite{Catani:1996vz,Catani:2002hc}.
However, when considering initial-state splittings this can lead to the
situation that only the first splitting of an initial--initial dipole 
transfers transverse momentum to the rest of the event. As an intuitive 
example, consider the shower evolution of a Drell-Yan event, which starts from just initial-initial dipoles. In the extreme case the gauge boson 
would get a finite recoil from the first splitting only, clearly at odds 
with the resummation of associated large logarithms. In 
\cite{Platzer:2009jq,Hoeche:2009xc} alternative, crossing symmetric, 
recoil strategies were presented that avoid this peculiar feature. 
For the \Sherpa implementation, \cite{Carli:2010cg} studied the impact 
of different recoil strategies in the context of deep-inelastic lepton 
scattering events.

The shower formulation based on Catani--Seymour dipole factorization 
offers two substantial advantages with respect to traditional parton showers,
which help to facilitate the merging with fixed-order matrix-element 
calculations:
\begin{itemize}
\item Due to the notion of specific spectator partons, four-momentum 
conservation is maintained locally, while only a single external
particle, the spectator, takes the recoil when the splitting parton
goes off-shell. This is important for the construction of a backward 
clustering algorithm based on the parton shower in the spirit of 
\cite{Hoeche:2009rj}.
\item The parton-shower model inherently respects QCD soft-colour coherence. 
By construction in Catani--Seymour factorization, the eikonal factor associated 
with soft gluon emission off a colour dipole, used to derive the angular ordering 
constrained in conventional parton showers, is exactly mapped onto two CS dipoles,
which only differ by the role of emitter and spectator.
\end{itemize}   
 

\mcsubsection{Matrix-element parton-shower merging}
\label{Sec:meps_sherpa}
One of the key features of \Sherpa is a generic implementation of the 
technique for combining tree-level matrix elements with parton showers
that was presented in~\cite{Hoeche:2009rj}, see \SecRef{sec:me-nlo-matching}. 
The method was extensively tested and validated for multijet production in 
$e^+e^-$ and hadron-hadron 
collisions, as well as deep-inelastic scattering processes, a scenario where 
event generators based on collinear factorization assumptions 
are unreliable due to a lack of matrix elements with sufficiently 
high final-state multiplicity~\cite{Carli:2010cg}.
An extension of the merging algorithm, which simulates hard QED radiation 
in a democratic approach, \ie on the same footing as QCD radiation, 
was implemented in \Sherpa and reported in~\cite{Hoeche:2009xc}. 
It yields excellent agreement with existing experimental data on prompt
photon production at both $e^+e^-$ and hadron colliders.
Although the novel merging technique implemented in recent versions of \Sherpa
has yielded significant improvements over the original CKKW algorithm,
in the sense that results are more accurate and stable, the CKKW approach
itself was already employed in former versions of \Sherpa with 
great success~\cite{Krauss:2004bs,Krauss:2005nu,Gleisberg:2005qq,Alwall:2007fs}.

In order to realize the ME+PS merging, \Sherpa makes use of its two 
internal tree-level matrix-element generators \Amegic and \Comix. Soft and 
collinear parton radiation is simulated by means of the internal parton 
shower. It should be noted that \Sherpa implements its matrix-element 
parton-shower merging in a modular way, distributing only necessary tasks to 
the matrix-element and parton-shower generators and handling all cross-module
interaction in the overall framework. This means in particular that the
matrix-element generator is only used to identify possible parton-shower
histories in the matrix elements by testing for respective subamplitudes
in the Feynman diagrams. The parton shower supplies information about 
the weight associated with a backward clustering that would reduce the 
actual partonic final state to the respective subamplitude. If external 
parton showers or matrix-element generators are provided by the user,
they must be capable of performing these operations. If so, they can in turn
be employed for automatic matrix-element parton-shower merging without 
any further adjustments of the \Sherpa framework, 
see also \SecRef{Sec:interfaces_extensions_sherpa}.

The \MENLOPS algorithm for merging lowest-multiplicity NLO matrix-elements
with higher-order tree-level contributions as presented independently 
in~\cite{Hamilton:2010wh} and~\cite{Hoeche:2010kg} is fully implemented
in the \Sherpa generator~\cite{Hoeche:2010kg}. It relies on the internal
generic \POWHEG generator described in \SecRef{Sec:NLO_ME_Sherpa}, 
which drives the lowest-multiplicity simulation and interfaces to the 
Catani--Seymour dipole shower to generate additional parton radiation.

\mcsubsection{\gensectionMPI}
\label{Sec:mpi_sherpa}
The multiple-interactions model used in \Sherpa closely follows the original 
ideas of~\cite{Sjostrand:1987su}. There are however important details where
the approach deviates from the formalism in \Pythia. Secondary interactions 
undergo parton-shower corrections in \Sherpa, but the evolution does not 
interleave parton showers and additional hard scatterings.
Care must then be taken when combining ME+PS merging with the modelling of 
multiple interactions. It is vital that the parton showers related to secondary 
collisions do not alter the initial jet spectra of the hard process. This can 
be achieved by a special jet veto procedure, which is described in some detail
in~\cite{Gleisberg:2008ta}.

The modelling of beam remnants in \Sherpa is realized in such a way that only 
a minimal set of particles (quarks and diquarks, the latter as carriers of 
baryon number) is produced in order to reconstruct the constituent 
flavour configuration of an incoming hadron. The distribution of colour in the 
remnants is guided by the idea of minimizing the relative transverse momentum 
of colour dipoles spanning the outgoing partons.  When including multiple 
parton interactions in the simulation, it is not always possible to accomplish 
free colour selection in the hard process and minimization of relative transverse 
momenta simultaneously.  In such cases the colour configurations of the 
matrix elements are kept but the configuration of the beam remnants is shuffled 
at random until a suitable solution is found.

In addition to the issues related to colour neutralization with the
beam remnants, all shower initiators and beam partons obtain a
primordial~\kt, see \SecRef{sec:primkt}. When tuning this
distribution, for example by using the data shown in \FigRef{fig:cmp:intrinsic-kt}, 
the mean and width parameter values obtained are typically rather
small (about $0.5-1.0$~GeV).

\mcsubsection{\gensectionhadronize}
\label{Sec:hadronization_sherpa}
The idea underlying \Ahadic, \Sherpa's module dealing with hadronization, 
is to take the interpretation of clusters as excited hadrons very literally, 
to compose clusters out of all possible flavours including diquarks 
and to have a flavour-dependent transition scale between clusters and 
hadrons.   This results in converting only the very lightest clusters 
directly into hadrons, whereas slightly heavier clusters experience a 
competition between either being converted into heavy hadrons or decaying 
into lighter clusters. For all decays, QCD-inspired, dipole-like 
kinematics are chosen. In somewhat more detail, in \Ahadic, the hadronization 
of quarks and gluons proceeds as follows:
\begin{itemize}
\item Firstly, all gluons are forced to decay into quark or diquark-pairs,
  $q\bar q$ or $d\bar d$, and all remaining partons are brought on 
  constituent mass shells.  Recoils are compensated mainly through 
  colour-connected particles.
\item Subsequent decays of heavy clusters are modelled by first emitting 
  a gluon from the $q\bar q$ pair and then splitting this gluon again.
\item In all non-perturbative decays ($g\to q\bar q$ and cluster decays) the 
  transverse momentum is limited to be smaller than a parameter 
  $\pt^{\rm max}$, typically of the order of the parton-shower cutoff 
  scale, and $\pt^2$ is chosen according to 
  $\alphaS(\pt^2+p_0^2)/(\pt^2+p_0^2)$, invoking a second parameter
  $p_0$ \footnote{There is also the option to use a non-perturbative 
    $\alphaS$ coupling, agreeing with a measurement from the GDH sum rule
    \cite{Deur:2008rf}.}.
  In this picture lighter-flavour pairs are preferentially produced due to 
  available phase space; this is supplemented by weight parameters.
\item The decays of clusters into hadrons are determined by various weights 
  including flavour wave functions, phase-space factors, flavour and 
  hadron-multiplet weights, and other dynamical measures.  
\end{itemize}

\mcsubsection{\gensectiondecay}
\label{Sec:decays_sherpa}

\Sherpa's hadron decay module is quite exhaustive, with approximately 
200 decay tables (one for each particle) consisting of more than 2500 
decay channels. Each of them is modelled by isotropic decay
and the branching ratio, but on top of that, spin-dependent
matrix elements and even form factors can be included.  This leaves \Sherpa
in a situation where for some decays various form factor models
are available\footnote{There is a plethora of sources and models, 
  for instance HQET \cite{Neubert:1993mb,Caprini:1997mu,Richman:1995wm}, 
  quark-model predictions~\cite{Isgur:1988gb,Scora:1995ty,Goity:1994xn} or
  QCD sum rules\cite{Ball:2004ye,Ball:2004rg,Ball:2007hb,Aliev:2007uu},
  all for heavy meson decays.  In addition, form factor models for $\tau$
  decays based on the Kuhn-Santamaria parameterization \cite{Kuhn:1990ad},
  or on Resonance Chiral Theory 
  \cite{Weinberg:1978kz,Gasser:1983yg,Gasser:1984gg,Ecker:1988te}
  have also been included.}, 
while for others even the branching ratios are not well known and have to be 
estimated from symmetry principles and phase-space arguments.  

In addition to the simulation of individual decays, non-trivial quantum 
effects are also modelled in \Sherpa, including spin correlations in 
sequential decays and CP violation introduced by mixing phenomena or their 
interplay with direct CP violation in decays.  For the latter, \Sherpa allows 
the user to include separate decay tables for particles and antiparticles.  

For QED FSR, the \Photons module \cite{Schonherr:2008av} is invoked, 
which employs the YFS formalism (see \SecRef{sec:qed-radiation} )
allowing for a systematic improvement of the 
eikonal approximation order-by-order in the QED coupling constant. 
${\cal O}(\alpha)$ corrections are
included for a number of processes, among them decays of vector particles
into leptons, leptonic $\tau$ decays and some $B$ decays.  
At present the module is only capable of handling single-particle initial 
states, \ie particle decays possibly including QED FSR off the hard 
process.  In contrast to some other implementations, however, it can deal with
decays involving more than two charged particles.  

\mcsubsection{Interfaces and extensions}
\label{Sec:interfaces_extensions_sherpa}
\paragraph{Interfaces provided}
\Sherpa supports most of the commonly used standard interfaces for information input or output:
\begin{itemize}
\item parameters and interactions of new physics models can be 
incorporated through \FeynRules generated input files \cite{Christensen:2009jx},
\item the spectra of supersymmetric models can be provided in the form of 
SLHA files \cite{Skands:2003cj,Allanach:2008qq},
\item to link external parton densities the LHAPDF package is supported,
\item hard-process configurations generated with either \Amegic or \Comix 
can be output in the Les-Houches-Event-File format,
\item one-loop amplitudes can be invoked using the Binoth-Les-Houches-Accord 
outlined in \cite{Binoth:2010xt},
\item fully showered and hadronized events can be output in the HepMC 
\cite{Dobbs:2001ck} or HepEvt format. 
\end{itemize}

\paragraph{Extending \Sherpa}
Extensions of \Sherpa can be provided in various ways. 
The easiest would certainly be to enhance the functionality of
an existing module of the program, thus providing the code with
the capability to, for example, simulate reactions in a new physics
scenario. The most challenging, but nevertheless available option
would be to supply a complete new module to the event-generation
framework, encoding for example an alternative underlying event 
model. Considerable modifications of the core framework of \Sherpa
will only be necessary if an extension of the program requires 
cross-module interaction that has not been foreseen and therefore
has not been implemented yet. In such cases, users are strongly 
encouraged to coordinate their efforts for implementing extensions
with the authors of \Sherpa. In most cases, however, existing 
structures will suffice to satisfy the needs for possible enhancements.

\Sherpa provides the option of loading most possible extensions of the
program package at runtime, using dynamically linked libraries.
This mechanism is especially convenient to use in bigger software
frameworks, where the \Sherpa core library itself is just part of
a larger event generation and analysis framework. It also allows
the user to install \Sherpa in a predefined location and to provide an 
extension of the program without altering its core modules.

For most of its extensions \Sherpa employs a so-called ``getter''
mechanism to identify possibly available external sources.
This means that an extension module is registered with the \Sherpa
instance at load time of its shared library, using a predefined
protocol associated with the physics task of the extension module.
An example would be an externally supplied parton shower,
which registers using its name (``Apacic'', for example) using the 
parton-shower identification protocol. At runtime, users can then 
specify this name in the input card to enable the respective shower
model in event generation.

The following extensions can currently be supplied to \Sherpa using 
``getter'' methods:
\begin{itemize}
\item {\em Analysis programs}\\
Both the Rivet library and the HZTool library are interfaced 
using such extensions of \Sherpa. These interfaces are distributed 
with the \Sherpa package itself.
\item {\em Parton distribution functions}\\
Despite most PDFs being available nowadays within the LHAPDF library, 
it might, in some cases, become necessary to interface the code for
a dedicated PDF. \Sherpa provides this option and supplies, for example,
in-house interfaces to photon PDFs.
\item {\em Matrix elements}\\
While there is little need to extend \Sherpa with tree-level matrix 
elements, this option is nevertheless provided to allow implementation 
of special matrix elements, for example for upsilon production.
Additionally, \Sherpa provides the option of including external NLO virtual
matrix elements, which can then be combined with automatically generated
Born-level, real-emission and subtraction terms.
\item {\em New physics models}\\
Even though the usage of the \FeynRules program package and its interface 
to \Sherpa is strongly encouraged, \Sherpa provides the option of implementing
a new-physics scenario directly. The corresponding vertices will then be
available for both internal matrix-element generators, \Amegic and \Comix.
\item {\em Helicity-amplitude building blocks}\\
New helicity-amplitude building blocks might become necessary for exotic BSM scenarios.
They can be provided for both internal matrix-element generators, 
\Amegic and \Comix. Of course they will have a different underlying 
structure in each case.
\item {\em Matrix-element generators}\\
If necessary, a complete external matrix-element generator can be supplied.
Note, however, that it must also satisfy the requirements imposed by the
possibility of merging matrix-element level events with parton showers.
This means in particular that it must provide a clustering algorithm
that identifies allowed parton-shower histories in tree-level matrix 
elements.
\item {\em Parton-shower generators}\\
If necessary, a complete external parton-shower model can be supplied.
Such a parton shower must, however, comply with \Sherpa's rules for 
matrix-element parton-shower merging, \ie it must provide a related
algorithm for computing the branching probability leading to tree-level 
matrix-element final states.
\item {\em Hadron decayers}\\
As already hinted at above, \Sherpa already provides quite an extensive
library for hadron decays.  They can be further extended by providing
form factors or ``skeleton'' matrix elements in the \Hadrons package.
\end{itemize}

\mcsubsection{\gensectionoutlook}
As with any other event generator, it is hard to conceive that \Sherpa will 
ever reach a state of ``perfection'', where nothing is left to be done.  
However, for the near future, a number of enhancements are foreseen:
\begin{itemize}
\item Work on the automated implementation of the 
  \POWHEG algorithm, including non-trivial colour configurations and 
  new-physics processes, will be finalized.  Equipped with this tool, the path towards a 
  multijet merging at NLO seems to be viable.  
\item A second, independent parton-shower formulation is ready to be fully
  included into the framework.  This will allow systematic comparison of
  parton-shower effects with two independent modules in the same framework,
  a huge step forward.
\item A new model for the simulation of soft inclusive physics and the
  underlying event, based on the multichannel eikonal approach of
  \cite{Ryskin:2009tj} is under way.  It will supplement or replace the
  old model, based on MPI.  
\end{itemize}

The current and future releases of the \Sherpa package as well as the most recent documentation can be found at
\begin{center} 
\href{http://projects.hepforge.org/sherpa/}{\tt http://projects.hepforge.org/sherpa/}
\end{center}

% Local Variables: 
% mode: LaTeX
% TeX-master: "../mcreview"
% End: 
