
\mcsubsection{\gensectionintro}

\pythia is a general-purpose event generator. It has been used 
extensively for $e^+e^-$, $ep$ and $pp/p\overline{p}$ physics, 
\eg at LEP, HERA and the Tevatron, and during the last 20 years 
has probably been the most used generator for LHC physics studies.
As a building block it has also been used in heavy-ion
physics and cosmic-ray physics.
 
The history of the Lund family of event generators began with 
\jetset \cite{Sjostrand:1982fn,Sjostrand:1982am,Sjostrand:1985ys,
Sjostrand:1986hx} in 1978, which later was merged into 
\pythia \cite{Bengtsson:1982jr,Bengtsson:1984yx,Bengtsson:1987kr,
Sjostrand:1993yb,Sjostrand:2000wi,Sjostrand:2006za}. Over the years 
many new physics models, especially for perturbative and 
nonperturbative QCD, have been developed and tested in parallel 
with the respective code. Thus the \pythiasix generator is the 
product of over thirty years of progress, but some of the code 
has not been touched in a very long time. New options have been 
added, but old ones seldom removed. The basic structure has been 
expanded in different directions, well beyond what it was once 
intended for, making it rather cumbersome by now.

{}From the outset, all code was written in Fortran~77. For the
LHC era, the experimental community made the decision to discontinue
Fortran and move heavy computing to \cpp. Therefore it was logical
also to migrate \pythia to \cpp, and in the process clean up and 
modernize various aspects. A first attempt in this direction was 
the \pythiaseven project \cite{Lonnblad:1998cq,Bertini:2000uh}, 
however, early on this was redirected to become a generic 
administrative structure, and renamed \thepeg (see \SecRef{sec:thepeg}).

\pythiaeight is a clean new start, to provide a successor to \pythiasix. 
It is a completely standalone generator, but several optional hooks 
for links to other programs are provided. Work on it began in 2004, 
and the first fully operational version (8.100) was released in 2007 
\cite{Sjostrand:2007gs}. It is not yet as well tested and tuned as 
\pythiasix, and therefore not as much used, although a slow shift is 
underway. Since priority has been to be ready in time for LHC startup,
some topics have not yet been addressed. Other parts of the \pythiasix 
were deemed obsolete and are permanently dropped.

Here follows a very brief summary of the current \pythia~8.1 program.
Much of the physics is the same as documented in the \pythia~6.4 manual 
\cite{Sjostrand:2006za} and the literature quoted there, with some relevant 
later updates \cite{Corke:2009tk,Corke:2010zj,Navin:2010kk,
Kasemets:2010sg,Carloni:2010tw,Corke:2010yf}. 
A complete manual also comes with the code distribution.
 
The physics summary below is split into core processes, the further 
perturbative evolution, and hadronization. An introduction to the 
code structure completes this outline.  

\mcsubsection{\gensectionhard}

Currently the program is only set up to handle collisions either between 
hadrons, such as $p$, $\overline{p}$, $\pi^{\pm}$ and $\pi^0$, or between 
same-generation leptons. That is, $pp$, $p\overline{p}$ and $e^+e^-$ 
beam combinations can be used,  but currently not $ep$, 
$\gamma\mathrm{p}$ or $\gamma\gamma$.

\pythia contains an extensive list of hardcoded subprocesses, over 200, 
that can be switched on individually. These are mainly $2 \to 1$ and 
$2 \to 2$, some $2 \to 3$, but no multiplicities higher than that. 
Consecutive resonance decays may of course lead to more final-state 
particles, as will parton showers. A brief summary of the main sets 
of subprocesses is as follows:
\begin{itemize}
\item hard  QCD processes, giving two high-$p_{\perp}$ partons;
\item $t$-channel exchange of a $\gamma^*/Z^0$ or $W^{\pm}$, also giving
  two high-\pt partons;
\item prompt-photon production with one or two photons in the final state;
\item a single $\gamma^*/Z^0$ or $W^{\pm}$ gauge boson, a pair of gauge 
bosons, or a gauge boson together with a parton; 
\item charmonium and bottomonium in the colour singlet and octet models;
\item top and a hypothetical heavy fourth generation;
\item Higgses within and beyond the Standard Model;
\item Supersymmetry (in progress); and
\item other exotic physics with new gauge bosons, left--right symmetry,
leptoquarks, excited fermions, hidden valleys, or extra dimensions. 
\end{itemize}

The subprocess cross sections have to be convoluted with PDFs to obtain
the event rates. Several proton PDFs are hardcoded in \pythia for ease 
of use and speed. These include
\begin{itemize}
\item traditional LO sets such as GRV 94L, CTEQ 5L, 6L and 6L1, 
and MSTW 2008 LO; 
\item the newer-style Monte-Carlo-adapted modified LO sets MRST LO* and 
LO**, and CT09 MC1, MC2 and MCS; and 
\item two central members of NLO sets, namely MSTW 2008 NLO and
CTEQ 66.00; 
\end{itemize}
see \SecRef{sec:pdfs-event-gener}.  
Further sets are available through an interface to the LHAPDF library.
It is possible to use separate PDF sets for the hard interaction, on one
hand, and for the subsequent showers and MPIs, on the other. Specifically, 
NLO sets are only intended to be used for hard subprocesses.

Obviously this list is far from complete, in terms of what will be
required at the LHC. Furthermore \pythia does not have automatic code 
generation for new processes, unlike some other generators. 
The intention is that \pythia should be open to external input to the 
largest extent possible, however. That way specialists from many areas 
can contribute hard subprocesses, which thereafter are handled further 
by the normal \pythia machinery.
\begin{itemize}
\item If an external program can generate a Les Houches Event File
\cite{Alwall:2006yp}, this can easily be read in by \pythia. 
A large number of programs can do just that. This includes 
general-purpose matrix-element programs, such as MadGraph or 
CompHep/CalcHep, and ready-made collections of processes, such as 
ALPGEN or AcerMC, see \SecRef{sec:subprocesses}. It also includes 
several processes implemented in the POWHEG approach to NLO calculations,
see \SecRef{sec:me-nlo-matching}. 
\item It is also possible to have a runtime link to \cpp or Fortran
programs, using the Les Houches Accord \cite{Boos:2001cv} structure 
to transfer information between the programs.
\item You can implement your own hard process inside a class derived from 
a \pythia base class, send in a pointer to it, and then let \pythia
handle the generation exactly as if it were an internal process. 
Notably MadGraph~5 will provide a facility whereby the complete code 
for such a class can be written automatically, ready to be linked.  
\end{itemize}  
\pythia is also open to input from other sources, such as the 
SUSY Les Houches Accord \cite{Skands:2003cj,Allanach:2008qq}.


\mcsubsection{Soft processes}

The so-called soft processes are elastic, single and double diffractive, 
and nondiffractive, see \SecRef{sec:mbtypes}. Together they are intended to 
offer an inclusive description of the total $pp$ cross section, with the 
exception of some of the rare (and even hypothetical) processes that
are better simulated separately. Thus the inelastic event sample includes 
high-$p_{\perp}$ physics as a tail of the low-$p_{\perp}$ one, in a 
consistent way, as provided by the MPI framework. To be precise, 
``soft'' events contain an inclusive production of standard QCD 
$2 \to 2$ processes, prompt photons, charmonia and bottomonia, low-mass 
Drell-Yan pairs, and $t$-channel $\gamma^*/\mathrm{Z}^0/\mathrm{W}^{\pm}$ 
exchange, in their expected proportions, with the MPI approach ensuring
that several of them can occur in the same event. One can alternatively use 
the same list as exclusive ``hard'' processes, if one is only interested in 
the high-$p_{\perp}$ tail, where a generation of the complete cross section 
would be inefficient.

Nondiffractive events provide the bulk of the inelastic cross section,
\ie what is observed in central detectors. Inside \pythia it is also 
referred to as the {\it minbias} component, but it does not have a one-to-one
overlap with the experimental definition of minimum bias,
see the warning in \SecRef{sec:mbtypes}. The nondiffractive
component is expected to provide an even bigger fraction of the events
that contain a hard process. Therefore, if the user requests an exclusive
hard process, currently \pythia would always simulate the underlying 
event as being of the nondiffractive type.

Diffractive events are handled in the Ingelman--Schlein picture
\cite{Ingelman:1984ns}, wherein single diffraction is viewed
as the emission of a pomeron pseudoparticle from one incoming proton,
leaving that proton intact but with reduced momentum, followed by the 
subsequent collision between this pomeron and the other proton. The
pomeron is, to first approximation, viewed as a glueball state with
the quantum numbers of the vacuum, but by QCD interactions it will also
have a quark content. The pomeron--proton collision can then be handled 
as a normal hadron--hadron nondiffractive event, displaying the same
structure with MPI, ISR, FSR and the rest. Double diffractive
events contain two pomeron--proton collisions. 

Elastic scattering by default only includes strong interactions
\cite{Schuler:1993wr}, but it is possible to switch on the QED 
Coulomb term and interference as well \cite{Bernard:1987vq}.  

\mcsubsection{The perturbative evolution}

The \pythiaeight showers are ordered in transverse momentum 
\cite{Sjostrand:2004ef}, both for ISR and for FSR. Also MPIs are ordered 
in $p_{\perp}$ \cite{Sjostrand:1987su}.
This allows a picture where MPI, ISR and FSR are interleaved in one 
common sequence of decreasing $p_{\perp}$ values. 
This is most important for MPI and ISR, since they are in direct 
competition for momentum from the beams, while FSR mainly 
redistributes momenta between already kicked-out partons.
The interleaving implies that there is one combined evolution equation
\begin{eqnarray}
\frac{\mathrm{d} \mathcal{P}}{\mathrm{d} p_{\perp}}&=& 
\left( \frac{\vphantom{\left(\right)} \mathrm{d}\mathcal{P}_{\mathrm{MPI}}}%
{\mathrm{d} p_{\perp}}  + 
\sum   \frac{\vphantom{\left(\right)} \mathrm{d}\mathcal{P}_{\mathrm{ISR}}}%
{\mathrm{d} p_{\perp}}  +
\sum   \frac{\vphantom{\left(\right)} \mathrm{d}\mathcal{P}_{\mathrm{FSR}}}%
{\mathrm{d} p_{\perp}} \right)
\nonumber \\ 
 & \times & \exp \left( - \int_{p_{\perp}}^{p_{\perp\mathrm{max}}} 
\left( \frac{\vphantom{\left(\right)} \mathrm{d}\mathcal{P}_{\mathrm{MPI}}}%
{\mathrm{d} p_{\perp}'}  + 
\sum   \frac{\vphantom{\left(\right)} \mathrm{d}\mathcal{P}_{\mathrm{ISR}}}%
{\mathrm{d} p_{\perp}'}  +
\sum   \frac{\vphantom{\left(\right)} \mathrm{d}\mathcal{P}_{\mathrm{FSR}}}%
{\mathrm{d} p_{\perp}'} 
\right) \mathrm{d} p_{\perp}' \right) 
\label{eq:pythiacombinedevol}
\end{eqnarray}
that probabilistically determines what the next step will be.
Here the ISR sum runs over all incoming partons, two per
already produced MPI (including the hard process), the FSR sum runs over 
all outgoing partons, and $p_{\perp\mathrm{max}}$ is the $p_{\perp}$ of the 
previous step.

Starting from a large $p_{\perp}$ scale of the hard process, the 
decreasing $p_{\perp}$ of \EqRef{eq:pythiacombinedevol} can be viewed 
as an evolution towards
increasing resolution; given that the event has a particular structure
when activity above some $p_{\perp}$ scale is resolved, how might that 
picture change when the resolution cutoff is reduced by some infinitesimal
$\mathrm{d} p_{\perp}$? That is, let the ``harder'' features of the event 
set the pattern to which ``softer'' features have to adapt.
It does not have a simple interpretation in absolute time;
all the MPIs occur essentially simultaneously (in a simpleminded
picture where the protons have been Lorentz contracted to pancakes),
while ISR stretches backwards in time and FSR forwards in time. 

\mcsubsection{\gensectionshower}

The initial- and final-state algorithms are partly based on a 
dipole-type approach to recoils, see \SecRef{sec:dipoles}, but with 
some modifications. 

In the simplest case, consider a colour dipole stretched between 
two final-state partons. The emission off such a dipole can be 
associated with either of the two ends, approximately in proportion 
to the respective $1/Q^2$ propagator, which gives a smooth transition 
across phase space for having it associated with either end, 
say $(1 \pm \cos\theta)/2$ in the soft-gluon limit.
By this classification the radiator end is the one that branches 
into two, which implies changed kinematics, to be compensated by 
the recoiler end. 

If the colour dipole is stretched between a final and an initial 
parton, radiation off  the final end has to be compensated by a 
changed momentum for this incoming parton. The subdivision of 
radiation from the two dipole ends is also somewhat more delicate 
in this case, and necessitates the introduction of a special damping 
factor on emission from the final end. 
 
ISR is described by backwards evolution \cite{Sjostrand:1985xi}, 
wherein branchings are constructed from the hard subprocess, back to 
the shower initiators. In each step the whole previously generated
partonic system takes the recoil of the newly emitted parton.
This applies whether the radiating initial dipole end has a 
colour partner in the initial or in the final state. In the latter 
case the pure dipole picture dictates that only this one partner
should take the recoil, but such a picture would not give the correct 
resummation behaviour \eg for $Z^0$ production, see \SecRef{sec:dipoles},  
and is therefore rejected. Emissions are allowed partially to line up 
in azimuthal angle by colour flow, however, which retains some memory 
of the dipole structure.

The evolution variable is closely related to the transverse momemtum
of a branching, but is not identical with it. Instead the 
lightcone-motivated relationships $p_{\perp\mathrm{evol}}^2 = (1-z)Q^2$ 
for ISR and $p_{\perp\mathrm{evol}}^2 = z(1-z)Q^2$ for FSR are used to 
define the space- or time-like virtuality $Q^2$ of the off-shell 
intermediate parton, given the chosen $p_{\perp\mathrm{evol}}$ 
and $z$. When kinematics are actually reconstructed, Lorentz
invariant expressions for $z$ are being used, based on ratios of 
invariant masses, which leads to a kinematical 
$p_{\perp\mathrm{kin}} \leq p_{\perp\mathrm{evol}}$.
Specifically, as a function of emission angle, $p_{\perp\mathrm{kin}}$
peaks at $90^{\circ}$, whereas $Q^2$ and hence $p_{\perp\mathrm{evol}}$ 
keeps on rising with angle.

Both QCD and QED emissions are allowed, and fully interleaved. 
Currently allowed branchings in the shower are $q \to qg$,
$g \to gg$, $g \to q\overline{q}$, $q \to q \gamma$,
$\ell \to \ell \gamma$ and, for FSR only, $\gamma \to \ell^+ \ell^-$
and $\gamma \to q\overline{q}$. 

Many resonance decays involve full matching to NLO QCD matrix 
elements. For production, however, all internally implemented 
processes are LO only. Production of $\gamma^*/Z^0/W^{\pm}$ is
matched to the real-emission corrections, so as to obtain the 
NLO $p_{\perp}$ spectrum, but without the NLO $K$-factor.
Showers have been constructed so that they, by default, have a 
sensible behaviour over the full phase space, all the way up to 
the kinematical limit, for a wide range of processes, but they 
will not be perfect. 

Final-state showers have a sharp lower cutoff, that should define 
the transition to hadronization. For ISR it is also possible to
use a sharp cutoff, but a valid alternative is a smooth turn-off
related to what is done for MPI below.

\mcsubsection{\gensectionMPI}

MPI modelling has traditionally been a hallmark of \pythia.
The framework is extensively described in \SecRef{sec:mbmpi},
and here only the basic principles are recapitulated, to put
them in context.

The perturbative $2 \to 2$ QCD cross section, which is dominated by 
$t$-channel gluon exchange, diverges roughly like 
$\mathrm{d} p_{\perp}^2 / p_{\perp}^4$. But this is based on the 
assumption of free incoming states, which is not the case when 
partons are confined in colour-singlet hadrons. Screening by nearby 
opposite colour charges will dampen the interaction of gluons with a 
large transverse wavelength. This is introduced by reweighting the 
interaction cross section by the factor in \EqRef{eq:ptzerodampen},
where $p_{\perp 0}$ is a free parameter in the model. To be more precise, 
it is the physical cross section $\mathrm{d} \sigma / \mathrm{d} p_{\perp}^2$ 
that needs to be regularized, \ie  the convolution of
$\mathrm{d} \hat{\sigma} / \mathrm{d} p_{\perp}^2$ with the two 
parton densities. One is thus at liberty to associate the screening 
factor with the incoming hadrons, half for each of them, instead of 
with the interaction. Such an association also gives a recipe to 
regularize the ISR divergence, as already noted.

The $p_{\perp 0}$ parameter can be energy-dependent,
since higher energies probe partons at smaller $x$, where the 
parton density increases and thereby the colour screening distance
decreases. An ansatz $p_{\perp 0} \propto E_{\mathrm{CM}}^{\epsilon}$
is therefore assumed, with some small power $\epsilon$. 

The spatial shape of the proton determines the balance between 
peripheral and central collisions, as reflected for example in the width 
of the multiplicity distribution. Several different shapes are available,
starting with a simple Gaussian ansatz.

Rescattering has been implemented, \ie the possibility of one parton 
scattering several times. So far no good experimental signals have been 
found for it, and it is off by default.

For dedicated studies of two low-rate processes in coincidence, two
hard interactions can be set in the same event, by a somewhat 
simplified duplication of the normal hard-process selection machinery. 
There are no Sudakov factors included for these two interactions, 
similarly to normal events with one hard interaction.

Rescaled parton densities are defined after each interaction, 
that take into account the nature of the previous partons extracted
from the hadron. This guarantees energy--momentum--flavour conservation.

Currently there is only one scenario for colour reconnection in the 
final state, see \SecRef{sec:mbmpi-np}, in which there is a certain 
probability for the partons of two subscatterings to have their 
colours interarranged in a way that reduces the total string length. 
(This is intermediate in character between the original strategy 
\cite{Sjostrand:1987su} and the more recent ones \cite{Skands:2010ak}.) 

At the end of the perturbative stage, a number of leftover partons
are found in the proton beam remnants, with colour connections to
the scattered partons, see \SecRef{sec:mbmpi-np}. Primordial $k_{\perp}$'s 
are introduced both for the scattering subsystems and the remnants, 
colours are assigned to connect the subsystems and the remnants with 
each other, and leftover longitudinal momentum is split between the 
remnant partons. When necessary, the junction approach is used to 
keep track of the baryon number, see \SecRef{sec:string-model}.

\mcsubsection{\gensectionhadronize}

Hadronization is based solely on the Lund string fragmentation
framework \cite{Andersson:1983ia,Sjostrand:1984ic}, 
\SecRef{sec:string-model}, which is at the origin of the \jetset 
program and thus of \pythia.

Particle data have been updated in agreement with the 2006 PDG
tables \cite{Yao:2006px}.  Some updated charm and bottom decay 
tables have been obtained from the DELPHI and LHCb collaborations.

The BE$_{32}$ model for Bose--Einstein effects \cite{Lonnblad:1997kk} 
has been implemented, but is not on by default. It does a reasonable
job with $e^+e^-$ data but not so well for hadronic collisions.   

\mcsubsection{Program structure and usage}

The \pythiaeight homepage is at
\begin{center} 
\texttt{http://www.thep.lu.se/}$\sim$\texttt{torbjorn/Pythia.html}
\end{center}
and from there you can download the most recent version as a gzipped
tar file, which also includes documentation as well as several example
main programs illustrating different ways in which \pythiaeight can be
used and linked. The documentation can also be accessed directly from the
\pythiaeight homepage. 

It is possible to perform analyses of the event record inside the 
main program. Alternatively events can be output to the HepMC
format, from which they can be studied further, or sent on to
detector simulation programs like Geant.

\mcsubsection{\gensectionoutlook}

\pythiaeight by now offers a complete replacement of \pythiasix
for essentially all aspects related to LHC physics studies, and in 
many respects contains improved physics models and new features. 
While development of \pythiasix has stopped, and new subversions
will only be prompted by bug fixes, \pythiaeight is being further
improved and extended in several directions. Experimental usage is 
still lagging behind, but interest is picking up, so one should 
expect a gradual phaseover during the next few years. 

% Local Variables: 
% mode: LaTeX
% TeX-master: "../mcreview"
% End: 
