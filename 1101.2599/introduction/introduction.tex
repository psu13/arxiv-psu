Understanding the final states of high energy particle
collisions such as those at the Large Hadron Collider (LHC) is an extremely challenging
theoretical problem.  Typically hundreds of particles are produced,
and in most processes of interest their momenta range over
many orders of magnitude.  All the particle species of the Standard
Model (SM), and maybe some beyond, are involved.  The relevant matrix
elements are too laborious to compute beyond the first few orders of
perturbation theory, and in the case of QCD processes they involve the
intrinsically non-perturbative and unsolved problem of confinement.
Once these matrix elements have been computed within some
approximation scheme, there remains the problem of dealing with their
many divergences and/or near-divergences.  Finally they must be
integrated over a final-state phase space of huge and variable
dimension in order to obtain predictions of experimental observables.

Over the past thirty years an armoury of techniques has been developed
to tackle these seemingly intractable problems.  The crucial tool
of factorization allows us to separate the treatment of many processes
of interest into different regimes, according to the scales of
momentum transfer involved.  At the highest scales, the constituent
partons of the incoming beams interact to produce a relatively small
number of energetic outgoing partons, leptons or gauge bosons.  The matrix
elements of these hard subprocesses are perturbatively computable.
At the very lowest scales, of the order of 1~GeV, incoming partons are
confined in the beams and outgoing partons interact non-perturbatively
to form the observed final-state hadrons.  These soft processes cannot
yet be calculated from first principles but have to be modelled.
The hard and soft regimes are distinct but connected
by an evolutionary process that can be calculated in principle from
perturbative QCD.  One consequence of this scale evolution is the
production of many additional partons in the form of initial- and
final-state parton showers, which eventually participate in the
low-scale process of hadron formation.

All three regimes of this highly successful picture of hard collisions
are eminently suited to computer simulation using Monte Carlo
techniques.  The large and variable dimension of the phase space,
$3n-4$ dimensions\footnote{Three components of momentum per produced
  particle,  minus four constraints of overall energy-momentum
  conservation.}
plus flavour and spin labels for an $n$-particle final state,  makes Monte
Carlo the integration method of choice: its accuracy improves inversely as the
square root of the number of integration points, irrespective of the dimension.
The evolution of scales that leads to parton showering is a
Markov process that can be simulated efficiently with Monte Carlo
techniques, and the available hadronization models are
formulated as Monte Carlo processes from the outset. 
Furthermore the factorized nature of the
problem means that the treatment of each regime can be improved
systematically as more precise perturbative calculations or more
sophisticated hadronization models become available.   

Putting all
these elements together, one has a Monte Carlo event generator capable
of simulating a wide range of the most interesting processes that are
expected at the LHC, which can be used for several distinct purposes in
particle physics experiments.  Event generators are usually required to extract a
signal of new physics from the background of SM
processes. Comparisons of their predictions to the data can be used to
perform measurements of SM parameters. They also provide realistic
input for the design of new experiments, or for new selection or
reconstruction procedures within an existing experiment. 

Historically, the development of event generators 
began shortly after the discovery of the partonic structure of
hadrons and of QCD as the theory of strong interactions.\footnote{For an early
review, see~\cite{Webber:1986mc}.}  Some important features of hard
  processes, such as deep inelastic scattering and hadroproduction of jets and lepton
  pairs, could be understood simply in terms of parton interactions.
  To describe final states in more detail, at first
simple models were used to fragment the primary partons
directly into hadrons, but this could not account for the transverse
broadening of jets and lepton pair distributions with increasing
hardness of the interaction.  It was soon appreciated that the primary
partons, being coloured, would emit gluons in the same way that
scattered charged particles emit photons, and that these gluons,
unlike photons, could themselves radiate, leading to a parton cascade
or shower that might account for the broadening.  It was then evident
 that hadron formation would occur
naturally as the endpoint of parton showering, when the typical scale
of momentum transfers is low and the corresponding value of the QCD
running coupling is large.  However, this very fact renders the
hadronization process non-perturbative, so hadronization models,
inspired by QCD but not so far derivable from it, were developed with
tunable parameters to describe the hadron-level properties of final states.

Although most of the signal processes of interest at the LHC fall into
the category of hard interactions that can be treated by the above
methods, the vast majority of collisions are soft, leading to
diffractive scattering or multiparticle production with low transverse
momenta.  These soft processes also need to be simulated
but, as in the case of hadronization, their non-perturbative nature
means that we have to resort to models with tunable parameters to
describe the data.  A related phenomenon is the component of the final
state in hard interactions that is not associated with the primary hard process
-- the so-called ``underlying event''.  There is convincing evidence
that this is due to secondary interactions of the other constituent
partons of the colliding hadrons.  The hard tails of these
interactions are described by perturbative QCD, but again the soft
component has to be modelled.  The same multiple-parton interaction
model can serve for the simulation of soft collisions,
provided there is no conflict between the parameter values needed to
describe the two phenomena. 

The main purpose of this review is to provide a survey of how all
the above components are implemented in the general-purpose event
generators that are currently available for the simulation of LHC
proton-proton collisions.  The authors are members of
MCnet,\footnote{http://www.montecarlonet.org/} a European Union funded
Marie Curie Research Training Network dedicated to developing the next
  generation of Monte Carlo event generators and providing training of
  its user base; the review seeks to contribute to those objectives.

Our discussion is aimed at phenomenologists wishing to understand better
the simulation of hadron-level events as well as experimentalists
wanting a deeper insight into the tools available for signal and
background simulation at the LHC.  We have tried to start at a level
that does not assume expertise beyond graduate particle physics
courses.  However, some sections dealing with current developments,
such as the matching of matrix elements and parton showers, are
necessarily more technical.  In those cases the treatment is less
pedagogical but we provide references to further discussion and proofs. 
Each section ends with a set of bullet points summarizing the main
points. In many cases we illustrate points by reference to plots of event
generator output, and compare with experimental data where available.

We begin in \PartRef{sec:revi-phys-behind} with a more detailed
discussion of the physics involved in event generators, starting with
an overview in \SecRef{sec:event-structure} of the structure of an event
and the steps by which it is generated.  We then describe the hard
subprocess in \SecRef{sec:subprocesses} before going on to the
parton showers in \SecRef{sec:parton-showers}. The precision of these
perturbative components of the simulation has been improved in recent
years by various schemes to include higher-order QCD corrections
without double counting, which we review in
\SecRef{sec:me-nlo-matching}.

Next we turn to the non-perturbative aspects of event generation,
starting in \SecRef{sec:pdfs-event-gener} with the parton distribution
functions of the incoming hadrons, which are used not only to compute
the hard subprocess cross section but also for the generation of
initial-state parton showers.  We go on to discuss the modelling of
soft collisions, the underlying event and diffraction in
\SecRef{sec:minim-bias-underly}, and then in \SecRef{sec:hadronization}
we describe the principal hadronization models used in present-day
event generators.

It is well established that a large fraction of produced particles
come from the decays of unstable hadronic resonances, and therefore the
accurate simulation of these decays, together with electroweak decays
that occur before particles have exited a typical beampipe or detector, is an essential
part of event generation, reviewed in \SecRef{sec:hadron-decays}.  Next we
describe the available techniques for simulating QED
radiation. \PartRef{sec:revi-phys-behind} ends with a discussion of
the simulation of physics beyond the Standard Model.

\PartRef{sec:spec-revi-main} contains brief reviews of the individual
event generators that were developed as part of the MCnet Network,
referring back to \PartRef{sec:revi-phys-behind} for the physics
involved and the modelling options that are implemented.  Then
in \PartRef{sec:comp-gener-with} we discuss issues involved in the use
of event generators, their validation and tuning, and the tools that
have been developed for these purposes. In particular, guidelines for
making experimental measurements that are optimally useful for Monte
Carlo validation and tuning are given.

\PartRef{sec:comp-gener-with} ends with some illustrative plots of
results from the MCnet event generators for a wide range of processes.  It
should be emphasised that these results are only ``snapshots'' of the
current state of the generators, which have not yet been thoroughly
tuned for use at the LHC.  For up-to-date comparisons with LHC data
one should consult the repository of plots at {\tt mcplots.cern.ch}.

A number of Appendices deal with important technical points in more
detail.  \AppRef{sec:mc-methods} gives a brief survey of the basic
Monte Carlo methods employed in event generators, while
\AppRef{sec:app_mcs} discusses methods for evaluating hard subprocess
matrix elements and phase space integration.
A particularly important Standard Model parameter is the top quark mass, and
we devote \AppRef{sec:top-quark-masses} to the meaning of this quantity as determined
by tuning the corresponding event generator parameters.

  As space is limited, and the emphasis of MCnet has been on
  general-purpose event generation for proton colliders, some topics
  relevant to the LHC programme, notably  heavy ion collisions, are
  not included.  We also do not cover specialized generators for
  specific processes, or programs that operate only at parton level
  and do not generate complete hadron-level final states. In
  most cases the latter can be interfaced to the MCnet generators
  through standard file formats,  as outlined in
  \AppRef{sec:MEinterfaces},  although care must be taken to avoid
  double counting, as discussed in \SecRef{sec:me-nlo-matching}.

For reference and to avoid repetition, we have collected in
\TabRef{tab:abbrev} the common abbreviations used throughout the review.

\begin{table}
\begin{center}
\begin{tabular}{|l|l|}
\hline
BSM & Beyond Standard Model \\ 
DIS & Deep inelastic (lepton) scattering \\ 
FSR & Final-state (QCD) radiation \\  
ISR & Initial-state (QCD) radiation \\
LL & Leading logarithm(ic) \\
LO & Leading order \\ 
MC & Monte Carlo \\ 
ME & Matrix element \\
MPI & Multiple parton interations \\
NLL & Next-to-leading logarithm(ic) \\
NLO & Next-to-leading order \\
PDF & Parton distribution function \\
PS & Parton shower \\
SM & Standard Model \\ 
UE & Underlying event \\
\hline
\end{tabular}
\caption{ Abbreviations used in this review.\label{tab:abbrev}}
\end{center}
\end{table}

%\mcintrosubsection{Physical observables and Monte Carlo truth}
\label{sec:phys-obj-mc-truth}

When simulating a process with a Monte Carlo event generator it is
important to make the distinction between ``Monte Carlo truth'' and
``physical observables'' (see also the discussion of Monte Carlo truth
contained in the contribution by Buckley et al.\ in \cite{Butterworth:2010ym}).

It is often desired to specify a process in terms of intermediate
objects as well as initial and final states, for example lepton
pair hadroproduction via production and decay of a $Z^0$ boson. However,
the intermediate objects are not physical observables,
and in practice it is not always possible to classify the process in
this way.
In particular one must keep in mind that such a classification is only exact in the
limit that all quantum interference effects can be neglected.
Thus, although it may be convenient to model a
double-slit experiment by shooting particles either through slit S
(signal) or slit B (background), that distinction, as it stands, is not  quantum
mechanically meaningful when both slits are open.\footnote{``Background''
here refers only to fundamentally {\it irreducible} background, which
can produce the same final states as the signal.} Likewise, soft
bremsstrahlung in particular depends strongly on interference effects
(coherence, see \SecRef{sec:parton-showers} on parton showers),
and hence the assignment of radiation as coming off this or that
parton is inherently ambiguous.
The one fail-safe way to make sure a distinction
is quantum mechanically meaningful to all orders is well known: to
classify an event according to the values of specific physical
observables (such as where the photon struck the
actual screen, in the case of the double-slit experiment).

The \Sherpa event generator (see \SecRef{sec:sherpa}) goes so far as to insist that a process is
defined in terms of initial and final states, such that it is
not possible for a user to access any intermediate objects.
All possible contributing subprocesses, as well as any interference terms between
them, are then included in the calculation.
While other event generators do allow the user to specify the process
of interest in more detail, users should be aware of the possible limitations.
In addition, when an experimental measurement is  performed it should  be
presented in an unambiguous way, in terms of physical observables.











% Local Variables: 
% mode: LaTeX
% TeX-master: "../mcreview"
% End: 
