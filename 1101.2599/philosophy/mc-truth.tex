\mcintrosubsection{Physical observables and Monte Carlo truth}
\label{sec:phys-obj-mc-truth}

When simulating a process with a Monte Carlo event generator it is
important to make the distinction between ``Monte Carlo truth'' and
``physical observables'' (see also the discussion of Monte Carlo truth
contained in the contribution by Buckley et al.\ in \cite{Butterworth:2010ym}).

It is often desired to specify a process in terms of intermediate
objects as well as initial and final states, for example lepton
pair hadroproduction via production and decay of a $Z^0$ boson. However,
the intermediate objects are not physical observables,
and in practice it is not always possible to classify the process in
this way.
In particular one must keep in mind that such a classification is only exact in the
limit that all quantum interference effects can be neglected.
Thus, although it may be convenient to model a
double-slit experiment by shooting particles either through slit S
(signal) or slit B (background), that distinction, as it stands, is not  quantum
mechanically meaningful when both slits are open.\footnote{``Background''
here refers only to fundamentally {\it irreducible} background, which
can produce the same final states as the signal.} Likewise, soft
bremsstrahlung in particular depends strongly on interference effects
(coherence, see \SecRef{sec:parton-showers} on parton showers),
and hence the assignment of radiation as coming off this or that
parton is inherently ambiguous.
The one fail-safe way to make sure a distinction
is quantum mechanically meaningful to all orders is well known: to
classify an event according to the values of specific physical
observables (such as where the photon struck the
actual screen, in the case of the double-slit experiment).

The \Sherpa event generator (see \SecRef{sec:sherpa}) goes so far as to insist that a process is
defined in terms of initial and final states, such that it is
not possible for a user to access any intermediate objects.
All possible contributing subprocesses, as well as any interference terms between
them, are then included in the calculation.
While other event generators do allow the user to specify the process
of interest in more detail, users should be aware of the possible limitations.
In addition, when an experimental measurement is  performed it should  be
presented in an unambiguous way, in terms of physical observables.









