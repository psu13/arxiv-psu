% !TEX root = runners-in-action.tex

\section{Implementation}
\label{sect:implementation}

We accompany the theoretical development with two implementations of~$\lambdacoop$: a
prototype language \pl{Coop}~\cite{bauer19:Coop}, and a \pl{Haskell} library
\pl{Haskell-Coop}~\cite{ahman19:HaskellCoop}.

\pl{Coop}, implemented in \pl{OCaml}, demonstrates what a more fully-featured
language based on $\lambdacoop$ might look like.
%
It implements a bi-directional variant of $\lambdacoop$'s type system, extended
with type definitions and algebraic datatypes, to provide algorithmic typechecking and
type inference. The operational semantics is based on the computation rules of the
equational theory from \cref{sect:eqtheory}, but extended with general
recursion, pairing of runners from \cref{ex:pairing}, and an interface to the 
\pl{OCaml} runtime called \emph{containers}---these are essentially top-level runners
defined directly in \pl{OCaml}. They are a modular and systematic way of
offering several possible top-level runtime environments to the programmer.

The \pl{Haskell-Coop} library is a shallow embedding of $\lambdacoop$ in 
\pl{Haskell}. The implementation closely follows the denotational
semantics of~$\lambdacoop$. For instance, user and kernel monads are
implemented as corresponding \pl{Haskell} monads. Internally, the library
uses the \pl{Freer} monad of Kiselyov~\cite{Kiselyov:Freer}
to implement free model monads for given signatures of operations.
The library also provides a means to run user code via \pl{Haskell}'s top-level monads.
For instance, code that performs input-output operations may be run in \pl{Haskell}'s $\tm{IO}$ monad.

\pl{Haskell}'s advanced features
make it possible to use \pl{Haskell-Coop} to implement several 
extensions to examples from \cref{sect:examples}.
For instance, we implement ML-style state that allow references
holding arbitrary values (of different types), and state 
that uses \pl{Haskell}'s type system to track
which references are alive.
%
The library also provides pairing of runners from \cref{ex:pairing}, e.g., 
to combine state and input-output.
%
We also use the library to demonstrate that
\emph{ambient functions} from the \pl{Koka} language~\cite{Leijen:Ambients}
can be implemented with runners by treating their binding
and application as co-operations.
(These are functions that are bound dynamically but evaluated
in the lexical scope of their binding.)

%%% Local Variables:
%%% mode: latex
%%% TeX-master: "runners-in-action"
%%% End:
