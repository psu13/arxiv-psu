% !TEX root = runners-in-action.tex

\section{Denotational semantics}
\label{sec:denotational-semantics}

We provide a coherent denotational semantics for~$\lambdacoop$, and 
prove it sound with respect to the equational theory given
in \cref{sect:eqtheory}. Having eschewed all forms of recursion,
we may afford to work simply over the category of sets and functions, while noting that
there is no obstacle to incorporating recursion at all levels and switching to domain theory,
similarly to the treatment of effect handlers in~\cite{Bauer:EffectSystem}.

\subsection{Semantics of types}
\label{sec:semantics-types}

The meaning of terms is most naturally defined by structural induction on their typing
derivations, which however are not unique in~$\lambdacoop$ due to subsumption rules.
Thus we must worry about devising a \emph{coherent} semantics, i.e., one in which all derivations
of a judgement get the same meaning.
%
We follow prior work on the semantics of effect systems for
handlers~\cite{Bauer:EffectSystem}, and proceed by first giving a \emph{skeletal}
semantics of~$\lambdacoop$ in which derivations are manifestly unique because the effect
information is unrefined. We then use the skeletal semantics as the frame upon which rests
a refinement-style coherent semantics of the effectful types of~$\lambdacoop$.

The \emph{skeletal} types are like $\lambdacoop$'s types, but with all effect information
erased. 
In particular, the ground types $A$, and hence the kernel state types $C$, do not change
as they contain no effect information. The skeletal value types are
%
\begin{equation*}
  P, Q
   \bnfis A
   \mid \tyunit
   \mid \tyempty
   \mid \typrod{P}{Q}
   \mid \tysum{P}{Q}
   \mid \tyfunskel{P}{\tyuserskel{Q}}
   \mid \tyfunKskel{P}{\tykernelskel{Q}{C}}
   \mid \tyrunnerskel{C}.
\end{equation*}
%
The skeletal versions of the user and kernel types are $\tyuserskel{P}$ and
$\tykernelskel{P}{C}$, respectively. It is best to think of the skeletal types as ML-style
types which implicitly over-approximate effect information by ``any effect is possible'',
an idea which is mathematically expressed by their semantics, as explained below.

First of all, the semantics of ground types is straightforward. One only needs to provide
sets denoting the base types $\tybase$, after which the ground types receive the standard
set-theoretic meaning, as given in \cref{fig:semantics-ground-skeletal-types}.

Recall that $\Ops$, $\Sigs$, and $\Excs$ are the sets of all operations, signals, 
and exceptions, and that each $\op \in \Ops$ has a signature
$\op : \tysigop{A_\op}{B_\op}{E_\op}$.
%
Let us additionally assume that there is a distinguished operation
$\rip \in \Ops$ with signature $\rip : \tysigop{\One}{\Zero}{\Zero}$ (otherwise
we adjoin it to $\Ops$). It ensures that the denotations of skeletal
user and kernel types are \emph{pointed} sets, while operationally~$\rip$ indicates
a \emph{runtime error}.

Next, we define the \emph{skeletal user and kernel monads} as
%
\begin{align*}
  \UUskel X &\defeq \UU{\Ops}{\Excs} X= \Tree{\Ops}{X + \Excs}, \\
  \KKskel{C} X &\defeq \KK{C}{\Ops}{\Excs}{\Sigs} X = (C \expto \Tree{\Ops}{(X + \Excs) \times C + \Sigs}),
\end{align*}
%
and $\RunnerSkel{C}$ as the set of all \emph{skeletal runners $\R$ (with state $C$)}, 
which are families of co-operations
%
$
  \{\coop_\R : \skelsem{A_\op} \to \KK{C}{\Ops}{E_\op}{\Sigs} \skelsem{B_\op} \}_{\op \in \Ops}
$.
%
%
Note that $\KK{C}{\Ops}{E_\op}{\Sigs}$ is a coproduct~\cite{Hyland:CombiningEffects} of monads
$C \expto \Tree{\Ops}{{-} \times C + \Sigs}$ and~$\Exc{E_\op}$, and thus the skeletal
runners are the effectful runners for the former monad, so long as we read the effectful
signatures $\op : \tysigop{A_\op}{B_\op}{E_\op}$ as ordinary algebraic ones $\op : A_\op \leadsto B_\op + E_\op$.
%
While there is no semantic difference between the two readings, there is one of intention:
$\KK{C}{\Ops}{E_\op}{\Sigs} \skelsem{B_\op}$ is a kernel computation that (apart from
using state and sending signals) returns values of type $B_\op$ and raises
exceptions $E_\op$, whereas $C \expto \Tree{\Ops}{(\skelsem{B_\op} + E_\op) \times C + \Sigs}$ returns values of type
$B_\op + E_\op$ and raises no exceptions. We prefer the former, as it reflects our treatment of exceptions as a
control mechanism rather than exceptional values.

These ingredients suffice for the denotation of skeletal types as sets, as given in
\cref{fig:semantics-ground-skeletal-types}. The user and kernel skeletal types
are interpreted using the respective skeletal monads, and hence the two function types as Kleisli 
exponentials.

\begin{figure}[ht]
  \centering
  \small
  \textbf{Ground types}
  %
  \begin{gather*}
    \skelsem{\tybase} \defeq \cdots
    \qquad\qquad
    \skelsem{\tyunit} \defeq \One
    \qquad\qquad
    \skelsem{\tyempty} \defeq \Zero
    \\
    \skelsem{\typrod{A}{B}} \defeq \skelsem{A} \times \skelsem{B}
    \qquad\qquad
    \skelsem{\tysum{A}{B}} \defeq \skelsem{A} + \skelsem{B}
  \end{gather*}
  %
  \textbf{Skeletal types}
  %
  \begin{gather*}
    \begin{aligned}
      \skelsem{\typrod{P}{Q}} &\defeq \skelsem{P} \times \skelsem{Q}
      &
      \qquad
      \skelsem{\tyfun{P}{\tyuser{Q}{}}} &\defeq \skelsem{P} \expto \skelsem{\tyuser{Q}{}}
      \\
      \skelsem{\tysum{P}{Q}} &\defeq \skelsem{P} + \skelsem{Q}
      &
      \skelsem{\tyfunKskel{P}{\tykernelskel{Q}{C}}} &\defeq \skelsem{P} \expto \skelsem{\tykernelskel{Q}{C}}
    \end{aligned}
    \\
    \skelsem{\tyrunnerskel{C}} \defeq \RunnerSkel{\skelsem{C}}
    \qquad\quad
    \skelsem{\tyuserskel{P}} \defeq \UUskel \skelsem{P}
    \qquad\quad
    \skelsem{\tykernelskel{P}{C}} \defeq \KKskel{\skelsem{C}} \skelsem{P}
    \\
    \skelsem{x_1 : P_1, \ldots, x_n : P_n} \defeq \skelsem{P_1} \times \cdots \times \skelsem{P_n}
  \end{gather*}
  %
  \caption{Denotations of ground and skeletal types.}
  \label{fig:semantics-ground-skeletal-types}
\end{figure}

We proceed with the semantics of effectful types. The \emph{skeleton} of a value
type~$X$ is the skeletal type $\skeleton{X}$ obtained by removing all effect
information, and similarly for user and kernel types, see \cref{fig:skeletons-and-denotations}.
%
% Had to put this one in a figure too, because of page breaks. 
%
\begin{figure}[ht]
  \centering
  \small
  %
  \textbf{Skeletons}
  %
  \begin{gather*}
  \skeleton{A} \defeq A
  \qquad
    \skeleton{(\tyrunner{\sig}{\sig'}{S}{C})} \defeq \tyrunnerskel{C}
  \qquad
    \skeleton{(\typrod{X}{Y})} \defeq \typrod{\skeleton{X}}{\skeleton{Y}}
  \\
  \skeleton{(\tyfun{X}{\tyuser{Y}{\Ueff}})} \defeq \tyfunskel{\skeleton{X}}{\skeleton{(\tyuser{Y}{\Ueff})}}
  \qquad
    \skeleton{(\tysum{X}{Y})} \defeq \tysum{\skeleton{X}}{\skeleton{Y}}
  \\
  \skeleton{(\tyfunK{X}{\tykernel{Y}{\Keff}})} \defeq
                                                     \tyfunKskel{\skeleton{X}}{\skeleton{(\tykernel{Y}{\Keff})}}
  \qquad
    \skeleton{(\tyuser{X}{\Ueff})} \defeq \tyuserskel{\skeleton{X}}
  \\
  \skeleton{(x_1 : X_1, \ldots, x_n : X_n)} \defeq (x_1 : \skeleton{X}_1, \ldots, x_n : \skeleton{X}_n)
  \qquad
    \skeleton{(\tykernel{X}{(\Sigma, E, S, C)})} \defeq \tykernelskel{\skeleton{X}}{C}
  \end{gather*}
  %
  %
  \abovedisplayskip=4pt % to keep the definition of $\coop$ on page 19
  \textbf{Denotations}
  %
  \begin{gather*}
    \begin{aligned}
    \sem{A} &\defeq \skelsem{A}
    &
    \sem{\typrod{X}{Y}} &\defeq \sem{X} \times \sem{X}
    \\
    \sem{\tyrunner{\sig}{\sig'}{S}{C}} &\defeq \Runner{\sig}{\sig'}{S}{\sem{C}}
    &
    \qquad\qquad
    \sem{\tysum{X}{Y}} &\defeq \sem{X} + \sem{X}
    \end{aligned}
    \\
    \begin{aligned}
      \sem{\tyfun{X}{\tyuser{Y}{\Ueff}}} &\defeq
      (\sem{X}, \skelsem{\skeleton{X}}) \subexpto (\sem{\tyuser{Y}{\Ueff}}, \skelsem{\skeleton{(\tyuser{Y}{\Ueff})}})
    \\
      \sem{\tyfunK{X}{\tykernel{Y}{\Keff}}} &\defeq
      (\sem{X}, \skelsem{\skeleton{X}}) \subexpto (\sem{\tykernel{Y}{\Keff}}, \skelsem{\skeleton{(\tykernel{Y}{\Keff})}})
    \end{aligned}
    \\
    \begin{aligned}
      \sem{\tyuser{X}{(\sig,E)}} &\defeq \UU{\sig}{E} \sem{X}
      &
      \sem{\tykernel{X}{(\sig, E, S, C)}} &\defeq \KK{\skelsem{C}}{\sig}{E}{S} \sem{X}
    \end{aligned}
    \\
    \sem{x_1 : X_1, \ldots, x_n : X_n} \defeq \sem{X_1} \times \cdots \times \sem{X_n}
  \end{gather*}
  %
  %
  \caption{Skeletons and denotations of types.}
  \label{fig:skeletons-and-denotations}
\end{figure}
%
% NOTE: no paragraph here, please.
%
We interpret a value type~$X$ as a subset
$\sem{X} \subseteq \skelsem{\skeleton{X}}$ of the denotation of its skeleton,
and similarly for user and computation types. In other words, we treat the
effectful types as \emph{refinements} of their skeletons. For this,
we define the operation $(X_0, X_1) \subexpto (Y_0, Y_1)$, for any  
$X_0 \subseteq X_1$ and $Y_0 \subseteq Y_1$, as the set of maps $X_1 \to Y_1$
restricted to $X_0 \to Y_0$: 
%
\begin{equation*}
  (X_0, X_1) \subexpto (Y_0, Y_1) \defeq
  \{ f : X_1 \to Y_1 \mid \forall x \in X_0 \,.\, f(x) \in Y_0 \}.
\end{equation*}
%
Next, observe that the user and the kernel monads preserve subset inclusions, in
the sense that $\UU{\sig}{E} X \subseteq \UU{\sig'}{E'} X'$ and
$\KK{C}{\sig}{E}{S} X \subseteq \KK{C}{\sig'}{E'}{S'} X'$ if
$\sig \subseteq \sig'$, $E \subseteq E'$, $S \subseteq S'$, and
$X \subseteq X'$. In particular, we always have
$\UU{\sig}{E} X \subseteq \UUskel X$ and
$\KK{C}{\sig}{E}{S} X \subseteq \KKskel{C} X$.
%
Finally, let $\Runner{\sig}{\sig'}{S}{C} \subseteq \RunnerSkel{C}$ be the subset of those
runners~$\R$ whose co-operations for $\sig$ factor through $\KK{C}{\sig'}{E_\op}{S}$,
i.e.,
%
$
  \coop_\R : \skelsem{A_\op} \to \KK{C}{\sig'}{E_\op}{S} \skelsem{B_\op}
  \subseteq \KK{C}{\Ops}{E_\op}{\Sigs} \skelsem{B_\op}, 
$
%
for each $\op \in \sig$.

Semantics of effectful types is given in \cref{fig:skeletons-and-denotations}.
%
From a category-theoretic viewpoint, it assigns meaning in the category $\Sub(\Set)$
whose objects are subset inclusions $X_0 \subseteq X_1$ and morphisms from
$X_0 \subseteq X_1$ to $Y_0 \subseteq Y_1$ those maps $X_1 \to Y_1$ that restrict to
$X_0 \to Y_0$. The interpretations of products, sums, and function types are precisely the
corresponding category-theoretic notions $\times$, $+$, and $\subexpto$ in $\Sub(\Set)$.
Even better, the pairs of submonads $\UU{\sig}{E} \subseteq \UUskel$ and
$\KK{C}{\sig}{E}{S} \subseteq \KKskel{C}$ are the ``$\Sub(\Set)$-variants'' of the user
and kernel monads.
%
Such an abstract point of view drives the interpretation of terms, given
below, and it additionally suggests how our semantics can be set up on top of a
category other than~$\Set$. For example, if we replace $\Set$ with the category $\Cpo$ of
$\omega$-complete partial orders, we obtain the domain-theoretic semantics
of effect handlers from~\cite{Bauer:EffectSystem} that models recursion and operations whose
signatures contain arbitrary types.

\subsection{Semantics of values and computations}
\label{sec:semant-valu-comp}

To give semantics to $\lambdacoop$'s terms, we introduce \emph{skeletal
  typing} judgements
%
\begin{align*}
  &\Gamma \typesskel V : P,
& &\Gamma \typesskel M : \tyuserskel{P},
& &\Gamma \typesskel K : \tykernelskel{P}{C}, 
\end{align*}
%
which assign skeletal types to values and computations.
%
In these judgements, $\Gamma$ is a \emph{skeletal context} which assigns skeletal types to variables.

The rules for these judgements are
obtained from $\lambdacoop$'s typing rules, by \emph{excluding} subsumption rules and by relaxing restrictions on effects. For example, 
the skeletal versions of the rules \textsc{TyValue-Runner} and \textsc{TyKernel-Kill} are
%
\begin{equation*}
  \infer{
    \big(
      \Gamma, x \of A_\op \typesskel K_\op : \tykernelskel{B_\op}{C}
    \big)_{\op \in \sig}
  }{
    \Gamma \typesskel
    \tmrunner{(\tm{op}\,x \mapsto K_{\tm{op}})_{\tm{op} \in \Sigma}}{C} :
    \tyrunnerskel{C}
  }
  %
  \qquad\qquad
  %
  \infer{
     s \in \Sigs
  }{
     \Gamma \typesskel \tmkill[X \at C]{s} : \tykernelskel{\skeleton X}{C}
  }
\end{equation*}
The relationship between effectful and skeletal typing is summarised as follows:

\begin{proposition}
\label{prop:skeletaltypes}
(1) Skeletal typing derivations are unique.
(2) If $\subty X Y$, then $\skel X = \skel Y$, and analogously for subtyping of user and kernel types.
(3) If ${\Gamma \types V : X}$, then ${\skeleton{\Gamma} \typesskel V : \skeleton{X}}$, and
analogously for user and kernel computations.
\end{proposition}

\begin{proof}
  We prove (1) by induction on skeletal typing derivations, and
  (2) by induction on subtyping derivations. 
  %
  For (1), we further use the occasional type annotations, and the 
  absence of skeletal subsumption rules.
  %
  For proving (3), suppose that $\mathcal{D}$ is a derivation of
  $\Gamma \types V : X$. We
  may translate $\mathcal{D}$ to its \emph{skeleton} $\skeleton{\mathcal{D}}$ deriving
  $\skeleton{\Gamma} \typesskel V : \skeleton{X}$ by replacing typing rules
  with matching skeletal ones, skipping subsumption rules due to (2).
  Computations are treated similarly. \qed
\end{proof}

To ensure semantic coherence, we first define the \emph{skeletal semantics} of skeletal
typing judgements, 
%
$\skelsem{\Gamma \typesskel V : P} : \skelsem{\Gamma} \to \skelsem{P}$,
%
$\skelsem{\Gamma \typesskel M : \tyuserskel{P}} : \skelsem{\Gamma} \to \skelsem{\tyuserskel{P}}$,
and
$\skelsem{\Gamma \typesskel K : \tykernelskel{P}{C}} : \skelsem{\Gamma} \to \skelsem{\tykernelskel{P}{C}}$, 
%
by induction on their (unique) derivations.

Provided maps $\skelsem{A_1} \times \cdots \times \skelsem{A_n} \to \skelsem{B}$
denoting ground constants $\tmconst{f}$, 
values are interpreted in a standard
way, using the bi-cartesian closed structure of sets, except for a runner
$\tmrunner{(\op\, x \mapsto K_\op)_{\op \in \sig}}{C}$, which is interpreted at
an environment $\gamma \in \skelsem{\Gamma}$ as the skeletal runner
$\{\coop : \skelsem{A_\op} \to \KK{\skelsem{C}}{\Ops}{E_\op}{\Sigs} \skelsem{B_\op}\}_{\op
  \in \Ops}$, given by
%
\begin{equation*}
  \coop\,a \defeq
  (\cond
        {\op \in \sig}
        {\rho(\skelsem{\Gamma, x : A_\op \typesskel K_\op : \tykernelskel{B_\op}{C}}(\gamma, a))}
        {\rip}).
\end{equation*}
%
Here the map 
$\rho : \KKskel{\skelsem{C}} \skelsem{B_\op} \to \KK{\skelsem{C}}{\Ops}{E_\op}{\Sigs}
\skelsem{B_\op}$ is the skeletal kernel theory homomorphism characterised by the equations
%
\begin{gather*}
  \rho (\retTree \, b) = \retTree \, b, 
  \qquad
  \rho (\op'(a', \kappa, (\nu_e)_{e \in E_{\op'}})) =
     \op'(a', \rho \circ \kappa, (\rho(\nu_e))_{e \in E_{\op'}}), 
  \\
  \begin{aligned}
    \rho (\siggetenv\,\kappa) &= \siggetenv (\rho \circ \kappa), 
    &
    \rho (\sigraise\,e) &= (\cond {e \in E_\op} {\sigraise\, e} {\rip}), 
    \\
    \rho (\sigsetenv (c, \kappa)) &= \siggetenv (c, \rho \circ \kappa), 
    &
    \rho (\sigkill\,s) &= \sigkill\,s.
  \end{aligned}
\end{gather*}
%
The purpose of $\rip$ in the definition of $\coop$ is to model a runtime error
when the runner is asked to handle an unexpected operation, while~$\rho$ makes sure
that~$\coop$ raises at most the exceptions~$E_\op$, as prescribed by the 
signature of $\op$.

User and kernel computations are interpreted as elements of the corresponding skeletal
user and kernel monads. Again, most constructs are interpreted in a standard way:
$\tmkw{return}$s as the units of the monads; the operations $\tmkw{raise}$,
$\tmkw{kill}$, $\tmkw{getenv}$, $\tmkw{setenv}$, and $\tm{op}$s as the corresponding
algebraic operations; and $\tmkw{match}$ statements as the corresponding semantic
elimination forms. The interpretation of exception handling offers no surprises, e.g., as
in~\cite{Plotkin:HandlingEffects}, as long as we follow the strategy of treating
unexpected situations with the runtime error~$\rip$.

The most interesting part of the interpretation is the semantics of
%
\begin{align}
  \label{eq:semantics-run}
  & {\Gamma \typesskel (\tmrun{V}{W}{M}{F}) : \tyuserskel{Q}}, 
\end{align}
where $ F \defeq \{ \tmreturn{x} \at c \mapsto N, 
             (\tmraise{e} \at c \mapsto N_e)_{e \in E},
             (\tmkill{s} \mapsto N_s)_{s \in S} \}$.
%
At an environment $\gamma \in \skelsem{\Gamma}$, $V$ is interpreted as a skeletal runner
with state $\skelsem{C}$, which induces a monad morphism
$\runh : \Tree{\Ops}{-} \to (\skelsem{C} \expto \Tree{\Ops}{{-} \times \skelsem{C} + \Sigs})$, 
as in the proof of \cref{prop:monadmorphism}. 
Let
$f : \KKskel{\skelsem{C}} \skelsem{P} \to (\skelsem{C} \expto \UUskel \skelsem{Q})$ be the skeletal
kernel theory homomorphism characterised by the equations
%
\begin{gather}
  \label{eq:finally-map}
  \begin{aligned}
    f(\retTree{p}) &= \lam{c} \skelsem{\Gamma, x \of P, c \of C \typesskel N : Q}(\gamma, p, c), 
    \\
    f(\op(a, \kappa, (\nu_e)_{e \in E_\op})) &= \lam{c} \op(a, \lam{b} f (\kappa \, b) \, c, (f(\nu_e)\, c)_{e \in E_\op}), 
    \\
    f(\sigraise\,e) &= \lam{c}
      (\cond
        {e \in E}
        {\skelsem{\Gamma, c : C \typesskel N_e : Q}(\gamma, c)}
        {\rip}), 
    \\
    f(\sigkill\,s) &= \lam{c}
      (\cond
        {s \in S}
        {\skelsem{\Gamma \typesskel N_s : Q}\,\gamma}
        {\rip}), 
  \end{aligned} 
  \\
 \notag
  f(\siggetenv\,\kappa) = \lam{c} f (\kappa \, c) \, c, 
  \qquad\qquad
  f(\sigsetenv(c', \kappa)) = \lam{c} f \, \kappa \, c'.
\end{gather}
%
The interpretation of~\eqref{eq:semantics-run} at $\gamma$ is
%
$
  f(\runh_{\skelsem{P} + \Excs} (\skelsem{\Gamma \typesskel M : \tyuserskel{P}} \, \gamma))
  \, (\skelsem{\Gamma \typesskel W : C} \, \gamma)
$,
%
which reads: map the interpretation of $M$ at~$\gamma$ from the skeletal user
monad to the skeletal kernel monad using~$\runh$ (which models the operations of $M$ by the
cooperations of~$V$), and from there using~$f$ to a map
$\skelsem{C} \expto \UUskel \skelsem{Q}$, that is then applied to the initial kernel state, namely, the
interpretation of $W$ at~$\gamma$.

We interpret the context switch $\Gamma \typesskel \tmkernel{K}{W}{F} : \tyuserskel{Q}$
at an environment $\gamma \in \skelsem{\Gamma}$
as $f(\skelsem{\Gamma \typesskel K : \tykernelskel{P}{C}} \, \gamma)\,
  (\skelsem{\Gamma \typesskel W : C} \, \gamma)$, where $f$ is the map~\eqref{eq:finally-map}. 
Finally, $\tmkw{user}$ context switch is interpreted much like exception handling.

We now define coherent semantics of $\lambdacoop$'s typing derivations by passing
through the skeletal semantics. Given a derivation $\mathcal{D}$ of $\Gamma \types V : X$,
its skeleton $\skeleton{\mathcal{D}}$ derives
$\skeleton{\Gamma} \typesskel V : \skeleton{X}$.
We identify the denotation of~$V$ with the skeletal one,
%
\begin{equation*}
  \sem{\Gamma \types V : X} \defeq \skelsem{\skeleton{\Gamma} \typesskel V : \skeleton{X}} :
  \skelsem{\skeleton{\Gamma}} \to \skelsem{\skeleton{X}}.
\end{equation*}
%
All that remains is to check that $\sem{\Gamma \types V : X}$ restricts to
$\sem{\Gamma} \to \sem{X}$. This is accomplished by induction on~$\mathcal{D}$. 
The only interesting step is subsumption, which relies on a further
observation that $\subty{X}{Y}$ implies $\sem{X} \subseteq \sem{Y}$. Typing derivations for user and kernel
computations are treated analogously.

\subsection{Coherence, soundness, and finalisation theorems}
\label{sec:finalisation-theorem}

We are now ready to prove a theorem that guarantees execution of finalisation code. But
first, let us record the fact that the semantics is coherent and sound.

\begin{theorem}[Coherence and soundness]
  \label{thm:soundness-coherence}%
  The denotational semantics of $\lambdacoop$ is coherent, and it is sound for the equational
  theory of~$\lambdacoop$ from \cref{sect:eqtheory}.
\end{theorem}

\begin{proof}
  Coherence is established by construction: any two derivations of the same
  typing judgement have the same denotation because they are both
  (the same) restriction of skeletal semantics.
  %
  For proving soundness, one just needs to unfold
  the denotations of the left- and right-hand sides of equations from \cref{sect:eqtheory},
  and compare them, where some cases rely on suitable substitution lemmas. \qed
\end{proof}

To set the stage for the finalisation theorem, let us 
consider the 
computation $\tmrun{V}{W}{M}{F}$, well-typed by the rule \textsc{TyUser-Run} from
\cref{fig:typing-selected}. At an environment $\gamma \in \sem{\Gamma}$, the finalisation
clauses~$F$ are captured semantically by the \emph{finalisation map}
$\phi_\gamma : (\sem{X} + E) \times \sem{C} + S \to \sem{\tyuser{Y}{(\sig',E')}}$, given by
%
\begin{align*}
    \phi_\gamma(\iota_1(\iota_1\, x, c)) &\defeq \sem{\Gamma, x \of X, c \of C \types N : \tyuser{Y}{(\sig',E')}}(\gamma, x, c),  \\
    \phi_\gamma(\iota_1(\iota_2\, e, c)) &\defeq \sem{\Gamma, c \of C \types N_e : \tyuser{Y}{(\sig',E')}}(\gamma, c),  \\
    \phi_\gamma(\iota_2(s)) &\defeq \sem{\Gamma \types N_s : \tyuser{Y}{(\sig',E')}} \, \gamma.
\end{align*}
%
With~$\phi$ in hand, we may formulate the finalisation theorem for $\lambdacoop$,
stating that the semantics of~$\tmrun{V}{W}{M}{F}$ is a computation tree all of
whose branches end with finalisation clauses from~$F$. Thus, unless some enveloping
runner sends a signal, finalisation with $F$ is guaranteed to take place.

\begin{theorem}[Finalisation]
  \label{thm:finalisation}%
  A well-typed $\tmkw{run}$ factors through finalisation:
  %
  \begin{equation*}
    \sem{\Gamma \types (\tmrun{V}{W}{M}{F}) : \tyuser{Y}{(\sig',E')}}\,\gamma = \lift{\phi_\gamma}\,t, 
  \end{equation*}
  %
  for some $t \in \Tree{\sig'}{(\sem{X} + E) \times \sem{C} + S}$.
\end{theorem}

\begin{proof}
  We first prove that $f \, u \, c = \lift{\phi_\gamma}(u\,c)$ holds for all
  $u \in \KK{\sem{C}}{\sig'}{E}{S} \sem{X}$ and $c \in \sem{C}$, where $f$ is the
  map~\eqref{eq:finally-map}. The proof proceeds by computational induction
  on~$u$~\cite{Plotkin:Logic}. The finalisation statement is then just the special case with
  $u \defeq \runh_{\sem{X} + E} (\sem{\Gamma \types M : \tyuser{X}{(\sig,E)}} \, \gamma)$ and
  $c \defeq \sem{\Gamma \types W : C} \, \gamma$. \qed
\end{proof}

%%% Local Variables:
%%% mode: latex
%%% TeX-master: "runners-in-action"
%%% End:
