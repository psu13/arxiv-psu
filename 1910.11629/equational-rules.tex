% !TEX root = runners-in-action.tex

\subsection{Equational theory}
\label{sect:eqtheory}

We present $\lambdacoop$ as an \emph{equational calculus}, i.e., the interactions between
its components are described by equations. Such a presentation makes it easy to reason
about program equivalence.
%
There are three equality judgements
%
\begin{equation*}
\Gamma \types V \equiv W : X, 
\qquad
\Gamma \types M \equiv N : \tyuser{X}{\Ueff}, 
\qquad
\Gamma \types K \equiv L  : \tyuser{X}{\Keff}.
\end{equation*}
%
It is presupposed that we only compare well-typed expressions with the indicated types.
For the most part, the context and the type annotation on judgements will play no significant role,
and so we shall drop them whenever possible.

We comment on the computational equations for constructs characteristic
of~$\lambdacoop$, and refer the reader to the online appendix for other equations.
%
When read left-to-right, these equations explain the operational meaning of programs.

Of the three equations for $\tmkw{run}$, the first two specify that returned values and
raised exceptions are handled by the corresponding clauses,
%
\begin{align*}
  \tmrun{V}{W}{(\tmreturn{V'})}{F} &\equiv N[V'/x, W/c], 
  \\
  \tmrun{V}{W}{(\tmraise[X]{e})}{F} &\equiv N_{e}[W/c],
\end{align*}
%
where
%
$F \defeq \{\tmreturn{x} \at c \mapsto N,
   (\tmraise{e} \at c \mapsto N_e)_{e \in E},
   (\tmkill{s} \mapsto N_s)_{s \in S}
\}$.
%
The third equation below relates running an operation $\op$ with executing the corresponding co-operation~$K_\op$, 
where $R$ stands for the runner
%
$\tmrunner{(\tm{op}\,x \mapsto K_{\tm{op}})_{\tm{op} \in \sig}}{C}$:
%
\begin{multline*}
  \tmrun{R}{W}{(
    \tmop{op}{X}{V}{\tmcont x M}{\tmexccont {N'} {e'} {E_\op}}
    )}{F} \equiv {}
  \\
  \begin{aligned}[t]
     &\tmkernel{K_\op[V/x]}{W}{} \\
     &\qquad\big\{
         \begin{aligned}[t]
           &\tmreturn{x} \at c' \mapsto (\tmrun{R}{c'}{M}{F}), \\
           &\left(
               \tmraise{e'} \at c' \mapsto (\tmrun{R}{c'}{N'_{e'}}{F})
             \right)_{e' \in E_\op},\\
           &\left(
               \tmkill{s} \mapsto N_s
             \right)_{s \in S} \big\}
         \end{aligned}
 \end{aligned}
\end{multline*}
%
Because $K_\op$ is kernel code, it is executed in kernel mode, whose
$\tmkw{finally}$ clauses specify what happens afterwards: if $K_\op$ returns a value, or
raises an exception, execution continues with a suitable continuation, with~$R$
wrapped around it; and if $K_\op$ sends a signal, the corresponding finalisation code from $F$ is
evaluated.

The next bundle describes how kernel code is executed within user code:
%
\begin{align*}
  \tmkernel{(\tmreturn[C]{V})}{W}{F} &\equiv N[V/x, W/c], \\
  \tmkernel{(\tmraise[X \at C]{e})}{W}{F} &\equiv N_{e}[W/c], \\
  \tmkernel{(\tmkill[X \at C]{s})}{W}{F} &\equiv N_{s}, \\
  \tmkernel{(\tmgetenv[C]{\tmcont c K})}{W}{F} &\equiv \tmkernel{K[W/c]}{W}{F}, \\
  \tmkernel{(\tmsetenv{V}{K})}{W}{F} &\equiv \tmkernel{K}{V}{F}.
\end{align*}
%
We also have an equation stating that an operation called in kernel mode propagates out to
user mode, with its continuations wrapped in kernel mode:
%
\begin{multline*}
  \tmkernel{\tmop{op}{X}{V}{\tmcont x K}{\tmexccont L {e'} E}}{W}{F}
  \equiv {} \\
  \tmop{op}{X}{V}{\tmcont x {\tmkernel{K}{W}{F}}}{
    \left(
      \tmkernel{L_{e'}}{W}{F}
    \right)_{e' \in E}}.
\end{multline*}
%
Similar equations govern execution of user computations in kernel mode.

The remaining equations include standard $\beta\eta$-equations for
exception handling~\cite{Benton:ExceptionalSyntax}, deconstruction of products and sums,
algebraicity equations for operations~\cite{Pretnar:Thesis}, and the equations of kernel theory from \cref{sec:user-kernel-monads}, describing how $\tmkw{getenv}$ and $\tmkw{setenv}$ work, and how they interact with signals and other operations.

%%% Local Variables:
%%% mode: latex
%%% TeX-master: "runners-in-action"
%%% End:
