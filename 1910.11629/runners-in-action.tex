\documentclass[12pt,envcountsame]{llncs}

\pagestyle{plain}

% ESOP 2020 submission - 25.10.2019
% ESOP 2020 final version - 21.02.2020

%% BIBLIOGRAPHY STYLE
\bibliographystyle{splncs04}

\usepackage{charter,eulervm}

%% PACKAGES
\usepackage[T1]{fontenc}
\usepackage[utf8]{inputenc}

\usepackage{etoolbox}
\makeatletter
\let\llncs@addcontentsline\addcontentsline
\patchcmd{\maketitle}{\addcontentsline}{\llncs@addcontentsline}{}{}
\patchcmd{\maketitle}{\addcontentsline}{\llncs@addcontentsline}{}{}
\patchcmd{\maketitle}{\addcontentsline}{\llncs@addcontentsline}{}{}
\setcounter{tocdepth}{4}
\makeatother

\usepackage{amsmath}
\usepackage{amssymb}
\usepackage{bbold} % alternative blackboard letters (amssymb already provides a set of them)
\usepackage{graphicx}
\usepackage{color}
\usepackage{listings}
\usepackage{mathabx}
\usepackage{mathpartir} % inference rules
\usepackage{stmaryrd}
\usepackage{mathtools} % to typeset equations in the appendix
\usepackage{upquote}
\usepackage[dvipsnames]{xcolor}
\usepackage[all]{xy}

% Fix the definition environment (whoever had the bright idea to make it cursive?).
\let\definition\relax % undefine the environment
\spnewtheorem{definition}{Definition}{\bfseries}{\rmfamily}

%\usepackage{amsmath,amsthm,amscd,amssymb}
\usepackage[colorlinks=true
,breaklinks=true
,urlcolor=blue
,anchorcolor=blue
,citecolor=blue
,filecolor=blue
,linkcolor=blue
,menucolor=blue
,linktocpage=true]{hyperref}
\hypersetup{
bookmarksopen=true,
bookmarksnumbered=true,
bookmarksopenlevel=10
}
\usepackage[capitalise,nameinlink]{cleveref} % customisable references

%\usepackage[noBBpl,sc]{mathpazo}
%\linespread{1.05}
%\usepackage[scaled=.92]{helvet}
%\DeclareMathAlphabet      {\mathsf}{OT1}{phv}{m}{n}

\usepackage[papersize={6.9in, 10.0in}, left=.5in, right=.5in, top=1in, bottom=.9in]{geometry}
\tolerance=2000
\hbadness=2000

% these include amsmath and that can cause trouble in older docs.
\input{../helpers/cmrsum}
\makeatletter

\DeclareFontFamily{OMX}{MnSymbolE}{}
\DeclareSymbolFont{largesymbolsX}{OMX}{MnSymbolE}{m}{n}
\DeclareFontShape{OMX}{MnSymbolE}{m}{n}{
    <-6>  MnSymbolE5
   <6-7>  MnSymbolE6
   <7-8>  MnSymbolE7
   <8-9>  MnSymbolE8
   <9-10> MnSymbolE9
  <10-12> MnSymbolE10
  <12->   MnSymbolE12}{}

\DeclareMathSymbol{\downbrace}    {\mathord}{largesymbolsX}{'251}
\DeclareMathSymbol{\downbraceg}   {\mathord}{largesymbolsX}{'252}
\DeclareMathSymbol{\downbracegg}  {\mathord}{largesymbolsX}{'253}
\DeclareMathSymbol{\downbraceggg} {\mathord}{largesymbolsX}{'254}
\DeclareMathSymbol{\downbracegggg}{\mathord}{largesymbolsX}{'255}
\DeclareMathSymbol{\upbrace}      {\mathord}{largesymbolsX}{'256}
\DeclareMathSymbol{\upbraceg}     {\mathord}{largesymbolsX}{'257}
\DeclareMathSymbol{\upbracegg}    {\mathord}{largesymbolsX}{'260}
\DeclareMathSymbol{\upbraceggg}   {\mathord}{largesymbolsX}{'261}
\DeclareMathSymbol{\upbracegggg}  {\mathord}{largesymbolsX}{'262}
\DeclareMathSymbol{\braceld}      {\mathord}{largesymbolsX}{'263}
\DeclareMathSymbol{\bracelu}      {\mathord}{largesymbolsX}{'264}
\DeclareMathSymbol{\bracerd}      {\mathord}{largesymbolsX}{'265}
\DeclareMathSymbol{\braceru}      {\mathord}{largesymbolsX}{'266}
\DeclareMathSymbol{\bracemd}      {\mathord}{largesymbolsX}{'267}
\DeclareMathSymbol{\bracemu}      {\mathord}{largesymbolsX}{'270}
\DeclareMathSymbol{\bracemid}     {\mathord}{largesymbolsX}{'271}

\def\horiz@expandable#1#2#3#4#5#6#7#8{%
  \@mathmeasure\z@#7{#8}%
  \@tempdima=\wd\z@
  \@mathmeasure\z@#7{#1}%
  \ifdim\noexpand\wd\z@>\@tempdima
    $\m@th#7#1$%
  \else
    \@mathmeasure\z@#7{#2}%
    \ifdim\noexpand\wd\z@>\@tempdima
      $\m@th#7#2$%
    \else
      \@mathmeasure\z@#7{#3}%
      \ifdim\noexpand\wd\z@>\@tempdima
        $\m@th#7#3$%
      \else
        \@mathmeasure\z@#7{#4}%
        \ifdim\noexpand\wd\z@>\@tempdima
          $\m@th#7#4$%
        \else
          \@mathmeasure\z@#7{#5}%
          \ifdim\noexpand\wd\z@>\@tempdima
            $\m@th#7#5$%
          \else
           #6#7%
          \fi
        \fi
      \fi
    \fi
  \fi}

\def\overbrace@expandable#1#2#3{\vbox{\m@th\ialign{##\crcr
  #1#2{#3}\crcr\noalign{\kern2\p@\nointerlineskip}%
  $\m@th\hfil#2#3\hfil$\crcr}}}
\def\underbrace@expandable#1#2#3{\vtop{\m@th\ialign{##\crcr
  $\m@th\hfil#2#3\hfil$\crcr
  \noalign{\kern2\p@\nointerlineskip}%
  #1#2{#3}\crcr}}}

\def\overbrace@#1#2#3{\vbox{\m@th\ialign{##\crcr
  #1#2\crcr\noalign{\kern2\p@\nointerlineskip}%
  $\m@th\hfil#2#3\hfil$\crcr}}}
\def\underbrace@#1#2#3{\vtop{\m@th\ialign{##\crcr
  $\m@th\hfil#2#3\hfil$\crcr
  \noalign{\kern2\p@\nointerlineskip}%
  #1#2\crcr}}}

\def\bracefill@#1#2#3#4#5{$\m@th#5#1\leaders\hbox{$#4$}\hfill#2\leaders\hbox{$#4$}\hfill#3$}

\def\downbracefill@{\bracefill@\braceld\bracemd\bracerd\bracemid}
\def\upbracefill@{\bracefill@\bracelu\bracemu\braceru\bracemid}

\DeclareRobustCommand{\downbracefill}{\downbracefill@\textstyle}
\DeclareRobustCommand{\upbracefill}{\upbracefill@\textstyle}

\def\upbrace@expandable{%
  \horiz@expandable
    \upbrace
    \upbraceg
    \upbracegg
    \upbraceggg
    \upbracegggg
    \upbracefill@}
\def\downbrace@expandable{%
  \horiz@expandable
    \downbrace
    \downbraceg
    \downbracegg
    \downbraceggg
    \downbracegggg
    \downbracefill@}

\DeclareRobustCommand{\overbrace}[1]{\mathop{\mathpalette{\overbrace@expandable\downbrace@expandable}{#1}}\limits}
\DeclareRobustCommand{\underbrace}[1]{\mathop{\mathpalette{\underbrace@expandable\upbrace@expandable}{#1}}\limits}

\makeatother


%\usepackage[dotinlabels]{titletoc}
%\titlelabel{{\thetitle}.\quad}
%\titleformat{\section}[block]
  {\fillast\medskip}
  {{\thesection. }}
  {1ex minus .1ex}
  {\scshape}
 
\titleformat*{\subsection}{\itshape}
\titleformat*{\subsubsection}{\itshape}

\setcounter{tocdepth}{2}

\titlecontents{section}
              [2.3em] 
              {\bigskip}
              {{\contentslabel{2.3em}}\large\scshape}
              {\hspace*{-2.3em}}
              {\titlerule*[1pc]{}\contentspage}
              
\titlecontents{subsection}
              [4.7em] 
              {}
              {{\contentslabel{2.3em}}}
              {\hspace*{-2.3em}}
              {\titlerule*[.5pc]{}\contentspage}

% hopefully not used.           
\titlecontents{subsubsection}
              [7.9em]
              {}
              {{\contentslabel{3.3em}}}
              {\hspace*{-3.3em}}
              {\titlerule*[.5pc]{}\contentspage}
%\makeatletter
\renewcommand\tableofcontents{%
    \section*{\contentsname
        \@mkboth{%
           \MakeLowercase\contentsname}{\MakeLowercase\contentsname}}%
    \@starttoc{toc}%
    }
\def\@oddhead{{\scshape\rightmark}\hfil{\small\scshape\thepage}}%
\def\sectionmark#1{%
      \markright{\MakeLowercase{%
        \ifnum \c@secnumdepth >\m@ne
          \thesection\quad
        \fi
        #1}}}
        
\makeatother



%\makeatletter

 \def\small{%
  \@setfontsize\small\@xipt{13pt}%
  \abovedisplayskip 8\p@ \@plus3\p@ \@minus6\p@
  \belowdisplayskip \abovedisplayskip
  \abovedisplayshortskip \z@ \@plus3\p@
  \belowdisplayshortskip 6.5\p@ \@plus3.5\p@ \@minus3\p@
  \def\@listi{%
    \leftmargin\leftmargini
    \topsep 9\p@ \@plus3\p@ \@minus5\p@
    \parsep 4.5\p@ \@plus2\p@ \@minus\p@
    \itemsep \parsep
  }%
}%
 \def\footnotesize{%
  \@setfontsize\footnotesize\@xpt{12pt}%
  \abovedisplayskip 10\p@ \@plus2\p@ \@minus5\p@
  \belowdisplayskip \abovedisplayskip
  \abovedisplayshortskip \z@ \@plus3\p@
  \belowdisplayshortskip 6\p@ \@plus3\p@ \@minus3\p@
  \def\@listi{%
    \leftmargin\leftmargini
    \topsep 6\p@ \@plus2\p@ \@minus2\p@
    \parsep 3\p@ \@plus2\p@ \@minus\p@
    \itemsep \parsep
  }%
}%
\def\open@column@one#1{%
 \ltxgrid@info@sw{\class@info{\string\open@column@one\string#1}}{}%
 \unvbox\pagesofar
  \gdef\thepagegrid{one}%
 \global\pagegrid@col#1%
 \global\pagegrid@cur\@ne
 \global\count\footins\@m
 \set@column@hsize\pagegrid@col
 \set@colht
}%

\def\frontmatter@abstractheading{%
\bigskip
 \begingroup
  \centering\large
  \abstractname
  \par\bigskip
 \endgroup
}%

\makeatother

%\DeclareSymbolFont{CMlargesymbols}{OMX}{cmex}{m}{n}
%\DeclareMathSymbol{\sum}{\mathop}{CMlargesymbols}{"50}


\begin{document}

%!TEX root=all.tex
% ***************************************************************
% ** Title:            Dom's Standard Macros
% ** Author:           Dominic Verity.
% ** Commenced:        9/7/2009
% ***************************************************************

% A useful conditional construct.
\newcommand{\ifundef}[1]{\expandafter\ifx\csname#1\endcsname\relax}

% Font fiddles

% import these fonts by hand here to avoid clashes with usual blackboard bold usage.
%\pdfmapfile{+bbold.map}
\newcommand{\bbefamily}{\fontencoding{U}\fontfamily{bbold}\selectfont}
\newcommand{\textbbe}[1]{{\bbefamily #1}}
\DeclareMathAlphabet{\mathbbe}{U}{bbold}{m}{n}

\makeatletter

\def\re@DeclareMathSymbol#1#2#3#4{%
    \let#1=\undefined
    \DeclareMathSymbol{#1}{#2}{#3}{#4}}

% Top and Bottom stolen from txsymb
\ifundef{Top}
  \DeclareSymbolFont{tcSyC}{U}{txsyc}{m}{n}
  \SetSymbolFont{tcSyC}{bold}{U}{txsyc}{bx}{n}
  \DeclareFontSubstitution{U}{txsyc}{m}{n}

  \re@DeclareMathSymbol{\Top}{\mathord}{tcSyC}{120}
  \re@DeclareMathSymbol{\Bot}{\mathord}{tcSyC}{121}
\fi

% Symbols for pushout and pullback diagram shapes stolen
% from MnSymbol
\ifundef{righthalfcup}
  \DeclareFontFamily{U}{MnSymbolC}{}
  \DeclareSymbolFont{mnSyC}{U}{MnSymbolC}{m}{n}
  \SetSymbolFont{mnSyC}{bold}{U}{MnSymbolC}{b}{n}
  \DeclareFontShape{U}{MnSymbolC}{m}{n}{
      <-6>  MnSymbolC5
     <6-7>  MnSymbolC6
     <7-8>  MnSymbolC7
     <8-9>  MnSymbolC8
     <9-10> MnSymbolC9
    <10-12> MnSymbolC10
    <12->   MnSymbolC12}{}
  \DeclareFontShape{U}{MnSymbolC}{b}{n}{
      <-6>  MnSymbolC-Bold5
     <6-7>  MnSymbolC-Bold6
     <7-8>  MnSymbolC-Bold7
     <8-9>  MnSymbolC-Bold8
     <9-10> MnSymbolC-Bold9
    <10-12> MnSymbolC-Bold10
    <12->   MnSymbolC-Bold12}{}
  
  \re@DeclareMathSymbol{\righthalfcup}{\mathord}{mnSyC}{184}
  \re@DeclareMathSymbol{\lefthalfcap}{\mathord}{mnSyC}{185}
\fi

\DeclareFontFamily{U}{MnSymbolA}{}
\DeclareSymbolFont{mnSyA}{U}{MnSymbolA}{m}{n}
\SetSymbolFont{mnSyA}{bold}{U}{MnSymbolA}{b}{n}
\DeclareFontShape{U}{MnSymbolA}{m}{n}{
    <-6>  MnSymbolA5
   <6-7>  MnSymbolA6
   <7-8>  MnSymbolA7
   <8-9>  MnSymbolA8
   <9-10> MnSymbolA9
  <10-12> MnSymbolA10
  <12->   MnSymbolA12}{}
\DeclareFontShape{U}{MnSymbolA}{b}{n}{
    <-6>  MnSymbolA-Bold5
   <6-7>  MnSymbolA-Bold6
   <7-8>  MnSymbolA-Bold7
   <8-9>  MnSymbolA-Bold8
   <9-10> MnSymbolA-Bold9
  <10-12> MnSymbolA-Bold10
  <12->   MnSymbolA-Bold12}{}

\re@DeclareMathSymbol{\twoheadedswarrow}{\mathord}{mnSyA}{30}

\makeatother

\def\sdagger{{\!\text{\mdseries\textdagger}}}

% ***************************************************************
% ** Description:      Miscellaneous bits and pieces.            
% ***************************************************************

% *** Now some general definitions ***

% *** Fiddling with boxes, depths etc. ***

\newcommand{\mlaux}[3]{\setbox0=\hbox{$\mathsurround=0pt #2{#3}$}%
  \dimen0=\dp0\advance\dimen0 by \ht0\lower#1\dimen0\box0}
\newcommand{\mlower}[2]{\mathpalette{\mlaux{#1}}{#2}}

\newcommand{\makellapm}[2]{\hbox to 0pt{\hss$\mathsurround=0pt #1{#2}$}}
\newcommand{\llapm}{\relax\mathpalette\makellapm}
\newcommand{\makerlapm}[2]{\hbox to 0pt{$\mathsurround=0pt #1{#2}$\hss}}
\newcommand{\rlapm}{\relax\mathpalette\makerlapm}
\newcommand{\makelapm}[2]{\hbox to 0pt{\hss$\mathsurround=0pt #1{#2}$\hss}}
\newcommand{\lapm}{\relax\mathpalette\makelapm}

\newcommand{\makeushort}[3]{%
	\setbox0=\hbox{$\mathsurround=0pt #2{#3}$}%
	\hbox to 1\wd0{\hss\underbar{\hbox to #1\wd0{\hss\box0\hss}}\hss}}
\newcommand{\ushort}[1]{\relax\mathpalette{\makeushort{#1}}}

% Macro to typeset part of a math formula at a bigger size
\def\makebigger#1#2#3{\scalebox{#1}{$\mathsurround=0pt #2{#3}$}}
\def\bigger#1#2{{\relax\mathpalette{\makebigger{#1}}{#2}}}

\def\scaleuphalf{1.0954}
\def\scaleupone{1.2}
\def\scaleuptwo{1.44}

% Duals / Superscripted postfix ops

\newcommand{\dual}{^\circ}
\newcommand{\oth}{^{\mathord{\text{th}}}}
\newcommand{\ost}{^{\mathord{\text{st}}}}
\newcommand{\ond}{^{\mathord{\text{nd}}}}
\newcommand{\op}{^{\mathord{\text{\rm op}}}}
\newcommand{\co}{^{\mathord{\text{\rm co}}}}
\newcommand{\coop}{^{\mathord{\text{\rm coop}}}}
\newcommand{\vop}{^{\mathord{\text{\rm vop}}}}
\newcommand{\hop}{^{\mathord{\text{\rm hop}}}}
\newcommand{\hvop}{^{\mathord{\text{\rm hvop}}}}
\newcommand{\refld}{^{\mathord{\text{\rm r}}}}
\newcommand{\rev}{^{\mathord{\text{\rm rev}}}}
\newcommand{\rhv}{^r}
\newcommand{\lhv}{^l}
\newcommand{\eqv}{^e}
\newcommand{\tr}{^{\text{\rm t}}}
\newcommand{\mapcat}{^\cattwo}
\newcommand{\fp}{_{\mathord{\text{\em fp}}}}

% ordinal stuff

\newcommand{\st}{^{\text{st}}}
\newcommand{\nd}{^{\text{nd}}}
\newcommand{\rd}{^{\text{rd}}}
\renewcommand{\th}{^{\text{th}}}

% General mathematical connectives etc.

\newcommand{\orelse}{\mathrel{\text{or}}}
\newcommand{\also}{\mathrel{\text{and}}}
\newcommand{\defeq}{\mathrel{:=}}
\newcommand{\eqdef}{\mathrel{=:}}

% ***************************************************************
% ** Description:      Some useful mathematical operators.            
% ***************************************************************

\makeatletter

\def\newmop{\@ifstar{\@newmop m}{\@newmop o}}
\def\@newmop#1{\@ifnextchar[{\@@newmop #1}{\@@@newmop #1}}
\def\@@newmop#1[#2]{\@declmathop #1#2}
\def\@@@newmop#1#2{\expandafter\@declmathop\expandafter #1\csname #2\endcsname{#2}}

\makeatother

%new xypic tails
%\newdir{ >}{{}*!/-10pt/@{>}}
\newdir{ |}{{}*!/-5pt/@{|}}

% General operations on maps etc.
\newmop{im}
\newmop{coim}
\newmop{dom}
\newmop{cod}
\newmop{id}
\newmop{Map}

\newmop{obj}
\newmop{arr}
\newmop{sq}
\newmop{norm}

\newmop{el}
\def\card{\#}

\newmop{ev}

%Misc

\newmop{Ext}
\newmop{icon}
\newmop{pbk}

% (Weak) factorisation systems.
\newmop{cell}
\newmop{cof}
\newmop{fib}

% Bisimplicial sets
\newmop{diag}
\def\Wbar{\overline{W}}

% Colimits and limits
\newmop*{colim}
\newmop*{holim}

% Kan extension
\newmop[\lan]{lan}
\newmop[\ran]{ran}

% ***************************************************************
% ** Description:      General categorical notations
% ***************************************************************

\newcommand{\comma}{\mathbin{\downarrow}}

\newcommand{\unit}{\eta}
\newcommand{\counit}{\varepsilon}

\makeatletter
\newcommand{\rotatemath}[2]{\rotatebox[origin=c]{180}{$\m@th #1{#2}$}}
\makeatother

\newcommand{\Coproj}{\mathpalette\rotatemath\Pi}
\newcommand{\cprod}{\mathbin{\mathpalette\rotatemath\Pi}}
\newcommand{\ip}{\mathord{\mathpalette\rotatemath\pi}}

\newcommand{\yonmap}[1]{\ulcorner{#1}\urcorner}
\newcommand{\kancon}{\tilde}

\newcommand{\yoneda}{\mathscr{Y}\!}

% projections for products, pullbacks and comma objects
\newmop[\pr]{pr}
\newmop[\dm]{d} 

% inclusions for coproducts and pushouts
\newmop[\cpr]{in}

% Pushout and Pullback ``corners'' for xypic diagrams.

\newcommand{\pocorner}{\hbox to 8pt{{\vrule height8pt depth0pt width0.5pt}%
    \vbox to 8pt{{\hrule height0.5pt width7.5pt depth0pt}\vfill}}}
\newcommand{\poexcursion}{\save[]-<15pt,-15pt>*{\pocorner}\restore}
\newcommand{\pbcorner}{\vbox to 0pt{\kern 4pt\hbox to 0pt{\kern 4pt%
      \vbox{{\hrule height0.5pt width7.5pt depth0pt}}%
      {\vrule height8pt depth0pt width0.5pt}\hss}\vss}}
\newcommand{\pbexcursion}{\save[]+<5pt,-5pt>*{\pbcorner}\restore}
\newcommand{\pbdiamond}{\save[]+<0pt,-5pt>*{\rotatebox{-45}{$\pbcorner$}}\restore}


% ***************************************************************
% ** Description:      Tensors, actions etc.            
% ***************************************************************

\newcommand{\hact}{\cdot_h}
\newcommand{\vact}{\cdot_v}
\newcommand{\act}[1]{\cdot_{#1}}

\newcommand{\pwr}{\pitchfork}
\newcommand{\tns}{\ast}
\newcommand{\wcolim}{\circledast}
\newcommand{\leibwcolim}{\leib\wcolim}
\newcommand{\wlim}[2]{\{#1,#2\}}
\newcommand{\leibwlim}[2]{\{#1,#2\}^{\wedge}}

\newmop{cls}

\newcommand{\leib}[1]{\mathbin{\widehat{#1}}}
\newcommand{\wleib}{\widehat}

\newcommand{\etimes}{\mathbin{\ushort{0.5}{\mathord\times}}}

\newcommand{\pbtimes}[1]{\mathbin{\mathop{\times}_{#1}}}
\newcommand{\pbotimes}[1]{\mathbin{\mathop{\otimes}_{#1}}}

% ***************************************************************
% ** Description:      Standard categories. 
% ***************************************************************

% Macros for typesetting the names of different kinds of category

\newcommand{\category}[1]{\underline{\smash[b]{\text{\rm{#1}}}}}
\newcommand{\bicat}[1]{\underline{\smash[b]{\mathcal{#1}}}}
\newcommand{\twocat}[1]{\bicat{\text{\rm\em #1}}}
\newcommand{\tcat}{\lcat}
\newcommand{\bifun}{\underline}
\newcommand{\trans}{\underline}
\newcommand{\icat}{\mathbb}
\newcommand{\lcat}{\mathcal}
\newcommand{\scat}{\mathbf}
\newcommand{\stcat}{\scat}
\newcommand{\sspan}[2]{{}_{#1}\scat{sSet}_{#2}}
\newcommand{\qspan}[2]{{}_{#1}\scat{qCat}_{#2}}

\newcommand{\catthree}{{\bigger{1.12}{\mathbbe{3}}}}
\newcommand{\cattwo}{{\bigger{1.12}{\mathbbe{2}}}}
\newcommand{\catone}{{\bigger{1.16}{\mathbbe{1}}}}
\newcommand{\iso}{{\bigger\scaleuphalf{\mathbb{I}}}}

% Magic with Delta

\makeatletter

\def\Del@Sym{{\bigger\scaleuphalf{\mathbbe{\Delta}}}}

\def\del@fn{\futurelet\del@next}
\def\del@dn{\def\del@next}

\def\parsedel@{%
  \ifx +\del@next \del@dn+{\Del@Sym_{\mathord{+}}}%
  \else \del@dn {\del@fn\parsedel@@}%
  \fi\del@next}

\def\parsedel@@{%
  \ifx\space@\del@next \expandafter\del@dn\space{\del@fn\parsedel@@}%
  \else\ifx [\del@next \del@dn[{\del@fn\parsedel@@@}%
  \else\ifx _\del@next \del@dn{\Delta}%
  \else\ifx ^\del@next \del@dn{\Delta}%
  \else \del@dn{\Del@Sym}%
  \fi\fi\fi\fi\del@next}

\def\parsedel@@@{%
  \ifx\space@\del@next \expandafter\del@dn\space{\del@fn\parsedel@@@}%
  \else\ifx t\del@next \del@dn t{\Del@Sym_\infty\del@fn\parsedel@@@@}%
  \else\ifx b\del@next \del@dn b{\Del@Sym_{-\infty}\del@fn\parsedel@@@@}%
  \else \del@dn{\errmessage{unexpected modifier}}%
  \fi\fi\fi\del@next}

\def\parsedel@@@@{%
  \ifx\space@\del@next \expandafter\del@dn\space{\del@fn\parsedel@@@@}%
  \else\ifx ]\del@next \del@dn]{}%
  \else \del@dn{\errmessage{expecting close of option block}}%
  \fi\fi\del@next}

\def\Del{\del@fn\parsedel@}

\def\DelTop{\Del[t]}
\def\DelBot{\Del[b]}

\makeatother

\newcommand{\Horn}{\Lambda}

% Standard categories

\newcommand{\aDelta}{\Del+}

\newcommand{\Set}{\category{Set}}
\newcommand{\Cat}{\category{Cat}}

\newcommand{\fact}{\category{fact}}

\newcommand{\eCat}[1]{#1\text{-}\Cat} % Enriched categories

\newcommand{\Gph}{\category{Gph}}
\newcommand{\sCat}{\sSet\text{-}\Cat}
\newcommand{\sSet}{\category{sSet}}
\newcommand{\ssSet}{\category{ssSet}}%bisimplicial sets
\newcommand{\qCat}{\category{qCat}}
\newcommand{\Adj}{\category{Adj}}
\newcommand{\Mnd}{\category{Mnd}}
\newcommand{\Cmd}{\category{Cmd}}
\newcommand{\asSet}{\sSet_{\mathord{+}}}
\newcommand{\msSet}{\category{msSet}} % marked simplicial sets = strat
\newcommand{\amsSet}{\msSet_{\mathord{+}}}

\newcommand{\twoCat}{\eCat{2}}

% legacy names for a few things
\newcommand{\Simp}{\sSet}
\newcommand{\aSimp}{\asSet}
\newcommand{\sSimp}{\category{sSimp}}
\newcommand{\Strat}{\sSimp}
\newcommand{\asSimp}{\sSimp_{\mathord{+}}}

\newcommand{\Span}[1]{\category{Span}(#1)}
\newcommand{\Mod}[2]{{}_{#1}\category{Mod}_{#2}}
\newcommand{\qMod}[2]{{}_{#1}\category{Mod}_{#2}'}

\newcommand{\BiSimp}{\category{BiSimp}}

\newcommand{\genericarr}{\{\bullet\to\bullet\}}

% Notation for Adj related structures
\newcommand{\Atom}{\text{Atom}}
\newcommand{\Fillable}{\text{Fill}}

% Shapes of diagrams for pullbacks and pushouts
\newcommand{\pbshape}{{\mathord{\bigger\scaleupone\righthalfcup}}}
\newcommand{\poshape}{{\mathord{\bigger\scaleupone\lefthalfcap}}}

% ***************************************************************
% ** Description:      Simplicial Set / Model Category notation.            
% ***************************************************************

% Elementary operators in the theory of (stratified) simplicial sets.

\newcommand{\face}{\delta}
\newcommand{\vertex}{\nu}
\newcommand{\degen}{\sigma}
\newcommand{\aug}{\iota}
\newcommand{\tdegen}{\varsigma}
\newcommand{\thop}{\varrho}
\newcommand{\genop}[1]{\{{#1}\}}

\newcommand{\faceact}{\mathrm{d}}
\newcommand{\degenact}{\mathrm{s}}
\newcommand{\vertexact}{\mathrm{v}}
\newcommand{\tdegenact}{\mathrm{e}}
\newcommand{\augact}{\mathrm{i}}

\newcommand{\ssub}{\subseteq_s}

% face by vertices (fbv)
\newcommand{\sembl}{\mathopen{\mathord[\mkern-3mu\mathord[}}
\newcommand{\sembr}{\mathclose{\mathord]\mkern-3mu\mathord]}}
\newcommand{\fbv}[1]{\{{#1}\}}

% Partition operators.

\newcommand{\partinj}{\Bot}
\newcommand{\partproj}{\Top}

% other simplicial notation

\newcommand{\join}{\star}
\newcommand{\fatjoin}{\mathbin\diamond}
\newmop{dec}
\newmop{fatdec}
\newmop{slc}
\newmop{fatslc}
\newcommand{\fatslice}{\mathbin{\mkern-1mu{/}\mkern-5mu{/}\mkern-1mu}}
\newcommand{\fatslicel}[2]{\vphantom{#2}^{{#1}\fatslice{}}\mkern-2mu{#2}}
\newcommand{\fatslicer}[2]{{#1\mkern-1mu}_{{}\fatslice{#2}}}
\newcommand{\slice}{/}
\newcommand{\slicel}[2]{\vphantom{#2}^{{#1}\slice{}}\mkern-2mu{#2}}
\newcommand{\slicer}[2]{{#1\mkern-1mu}_{{}\slice{#2}}}
\newcommand{\slicelr}[3]{\vphantom{#2}^{{#1}\slice{}}\mkern-2mu{#2}_{{}\slice{#3}}}
\newcommand{\underlie}{\overline}
\newmop{ir} % the interval representation functor
\newmop{incl}

% nerves etc
\newcommand{\nrv}{N}
\newcommand{\ho}{h}
\newcommand{\kan}{k}

\newcommand{\nrvhc}{\nrv}
\newcommand{\gC}{\mathfrak{C}}%left adjoint to the homotopy coherent nerve

% Notation associated with Reedy categories

\newcommand{\Reedy}{\category{Reedy}}

\newcommand{\boundary}{\partial}
\newcommand{\coboundary}{\dot\partial}
\newcommand{\direct}{\overrightarrow}
\newcommand{\inverse}{\overleftarrow}
\newcommand{\reedycat}[1]{(\scat{#1},\direct{\scat{#1}},\inverse{\scat{#1}})}
\newcommand{\reedyclass}[3]{#1^{#2}_#3}
\newcommand{\twar}[1]{\scat{#1}\wr\scat{#1}}

\newcommand{\latch}{L}
\newcommand{\match}{M}

\newcommand{\bdrymap}{b}
\newcommand{\hornmap}{h}

\newcommand{\cofibrep}{L}
\newcommand{\fibrep}{R}
\newcommand{\coresol}{\Psi}
\newcommand{\resol}{\Phi}

\newcommand{\cfr}{_{\text{\rm c}}}
\newcommand{\fbr}{_{\text{\rm f}}}
\newcommand{\cfbr}{_{\text{\rm fc}}}
\newcommand{\srs}{_{\text{\rm sr}}}
\newcommand{\crs}{_{\text{\rm cr}}}

\def\reedyfilt#1_#2{#1_{\leq #2}}
\newmop{sk}
\newmop{cosk}
\newmop{res}

%decoration
\newcommand{\cart}{{\mathrm{cart}}}

\newcommand{\homsp}[1]{\hom_{\lcat{#1}}}

% General model category notation

\newcommand{\mclass}{\mathcal}
\newcommand{\lp}{\mathrel\square}
\newcommand{\lpclass}{{}^\square}

\newcommand{\Mono}{\mclass{M}\mkern-2mu\text{\it ono}}
\newcommand{\Epi}{\mclass{E}\mkern-2mu\text{\it pi}}

\newmop{map}
\newmop{Ho}

\newcommand{\cpts}{\pi_0}

\newmop[\pth]{path}
\newmop{cyl}

% Standard notations for operations on (iterated) functor categories

\newmop[\const]{c}
  
% Some standard model categories

\newcommand{\Kan}{\category{Kan}}
\newcommand{\Quasi}{\category{Quasi}}
\newcommand{\Diag}{\category{Diag}}

\newcommand{\dcsQuasi}{\category{dcsQuasi}}

% Augmentation
\def\iaug{^\bot}
\def\taug{^\top}

% collage and its right adjoint 
% (NB: this is probably in the wrong place)

\newmop{coll}
\newmop{wgt}

% spaces of homotopy coherent adjunctions

\newcommand{\counits}{\mathrm{counit}}
\newcommand{\cohadjs}{\mathrm{cohadj}}
\newcommand{\leftadjs}{\mathrm{leftadj}}
%\newmop{counits}
%\newmop{cohadjs}
%\newmop{leftadjs}

% ***************************************************************
% ** Description:      In-line arrows.            
% ***************************************************************

\newdir{ >}{{}*!/-7pt/@{>}}
\newdir{u(}{{}*!/-4pt/@^{(}}
\newdir{d(}{{}*!/-4pt/@_{(}}
\newdir{|>}{%
  !/4.5pt/@{|}*:(1,-.2)@^{>}*:(1,+.2)@_{>}*+@{}}

% New style, simpler, inline arrows. To match these in xypic diagrams
% use cm arrow heads there.

\makeatletter
\def\makeslashed#1#2#3#4#5{#1{\mathpalette{\sla@{#2}{#3}{#4}}{#5}}}

\def\@mathlower#1#2#3{\setbox0=\hbox{$\m@th#2#3$}\lower#1\ht0\box0}
\def\mathlower#1#2{\mathpalette{\@mathlower{#1}}{#2}}
\makeatother


\newcommand{\epi}{\twoheadrightarrow}
\newcommand{\inc}{\hookrightarrow}
\newcommand{\mono}{\rightarrowtail}
\newcommand{\tcof}{\hookrightarrow}
\newcommand{\tfib}{\twoheadrightarrow}
\newcommand{\fibt}{\twoheadleftarrow}

\newcommand{\xtfib}[1]{\xtwoheadrightarrow{#1}}
\newcommand{\xfibt}[1]{\xtwoheadleftarrow{#1}}

\newcommand{\prof}{\makeslashed\mathbin\shortmid{-0.1}{0.16}\to}
\newcommand{\mat}{\makeslashed\mathbin\circ{-0.1}{0.1}\to}

\newcommand{\we}{\xrightarrow{\mkern10mu{\smash{\mathlower{0.6}{\sim}}}\mkern10mu}}
\newcommand{\weleft}{\xleftarrow{\mkern10mu{\smash{\mathlower{0.6}{\sim}}}\mkern10mu}}
\newcommand{\trvfib}{\xtwoheadrightarrow{\smash{\mathlower{1.2}{\sim}}}}
\newcommand{\trvcof}{\xhookrightarrow{\mkern8mu{\smash{\mathlower{1}{\sim}}}\mkern12mu}}

% 2-cells

\newcommand{\To}{\Rightarrow}

% ***************************************************************
% ** Description:      Macros to support mixed variance tensor 
% **                   style sub/super-script notation            
% ***************************************************************

% Note - the \tn macro is not "re-entrant".

\makeatletter

\def\tens@fn{\futurelet\tens@next}
\def\tens@dn{\def\tens@nextcont}
\newtoks\tens@toks
\def\addtotens@toks#1{\tens@toks=\expandafter{\the\tens@toks#1}}

\def\parsetens@@{%
    \ifx\space@\tens@next \expandafter\tens@dn\space{\tens@fn\parsetens@@}%
    \else\ifx ^\tens@next \tens@dn ^##1{\parsetens@procsep^\addtotens@toks{##1}%
      \tens@fn\parsetens@@}%
    \else\ifx _\tens@next \tens@dn _##1{\parsetens@procsep_\addtotens@toks{##1}%
      \tens@fn\parsetens@@}%
    \else\tens@dn{\ifx *\tens@last \else\addtotens@toks\egroup\fi\the\tens@toks}%
    \fi\fi\fi\tens@nextcont}

\def\parsetens@procsep#1{%
  \ifx *\tens@last \addtotens@toks{#1}\addtotens@toks\bgroup%
  \else\ifx \tens@last\tens@next \addtotens@toks,%
  \else \addtotens@toks\egroup\addtotens@toks\bgroup%
    \addtotens@toks\egroup\addtotens@toks{#1}\addtotens@toks\bgroup% 
  \fi\fi\let\tens@last\tens@next}

\newcommand{\tn}[1]{\let\tens@last=*\tens@toks={#1}\tens@fn\parsetens@@}

\makeatother

\newcommand{\tlambda}{\tn\lambda}

% ***************************************************************
% ** Description:      Some standard xypic diagrams            
% ***************************************************************

\def\adjdisplay#1-|#2:#3->#4.{{%
    \xymatrix@R=0em@!C=2.5em{%
      *+[l]{#3} \ar@/_0.55pc/[rr]_-{#2} & {\bot} &
      *+[r]{#4}\ar@/_0.55pc/[ll]_-{#1}}}}

\def\adjdisplaytwo#1-|#2:#3->#4.{{% 
\xymatrix@=1.2em{
      {#3}\ar@/_1.5ex/[rr]_-{#2}^-{}="one"
      & & {#4}
      \ar@/_1.5ex/[ll]_-{#1}^-{}="two" 
      \ar@{}"one";"two"|{\bot}
    }}}

\def\tripleadjdisplay#1-|#2-|#3:#4->#5.{{%
\xymatrix@=2.4em{ 
{#4}\ar[r]|{#2} &
{#5} \ar@/_3ex/[l]_{#1}^{\bot} \ar@/^3ex/[l]_{\bot}^{#3}}
}}

\def\adjinline#1-|#2:#3->#4.{{#1}\dashv{#2}:#3\to #4}

\newcommand{\pent}[1]{
  \xybox{
    \POS (0,-15)*+{\a}="0", 
         (-14,-5)*+{\b}="1", 
         (-9,12)*+{\c}="2", 
         (9,12)*+{\d}="3", 
         (14,-5)*+{\e}="4"
    \POS"0" \ar "1"^{\labelstyle \ab}|{}="01"
    \POS"1" \ar "2"^{\labelstyle \bc}|{}="12"
    \POS"2" \ar "3"^{\labelstyle \cd}|{}="23"
    \POS"3" \ar "4"^{\labelstyle \de}|{}="34"
    \POS"0" \ar "4"_{\labelstyle \ae}|{}="04"
    \ifcase #1
    \POS"0" \ar "2"|{\labelstyle \ac}="02"
    \POS"0" \ar "3"|{\labelstyle \ad}="03"
    \POS"02";"1"**{}, ?(0.3) \ar@{=>} ?(0.7)^{\labelstyle \abc}
    \POS"03";"2"**{}, ?(0.25) \ar@{=>} ?(0.5)_{\labelstyle \acd}
    \POS"04";"3"**{}, ?(0.2) \ar@{=>} ?(0.4)_{\labelstyle \ade}
    \or
    \POS"1" \ar "3"|{\labelstyle \bd}="13"
    \POS"1" \ar "4"|{\labelstyle \be}="14"
    \POS"13";"2"**{}, ?(0.3) \ar@{=>} ?(0.7)_{\labelstyle \bcd}
    \POS"14";"3"**{}, ?(0.25) \ar@{=>} ?(0.5)_{\labelstyle \bde}
    \POS"04";"1"**{}, ?(0.25) \ar@{=>} ?(0.5)_{\labelstyle \abe}
    \or
    \POS"2" \ar "4"|{\labelstyle \ce}="24"
    \POS"0" \ar "2"|{\labelstyle \ac}="02"
    \POS"02";"1"**{}, ?(0.3) \ar@{=>} ?(0.7)^{\labelstyle \abc}
    \POS"04";"2"**{}, ?(0.2) \ar@{=>} ?(0.35)_{\labelstyle \ace}
    \POS"24";"3"**{}, ?(0.2) \ar@{=>} ?(0.6)^{\labelstyle \cde}
    \or
    \POS"1" \ar "3"|{\labelstyle \bd}="13"
    \POS"0" \ar "3"|{\labelstyle \ad}="03"
    \POS"04";"3"**{}, ?(0.2) \ar@{=>} ?(0.4)_{\labelstyle \ade}
    \POS"13";"2"**{}, ?(0.3) \ar@{=>} ?(0.7)_{\labelstyle \bcd}
    \POS"03";"1"**{}, ?(0.25) \ar@{=>} ?(0.5)^{\labelstyle \abd}
    \or
    \POS"2" \ar "4"|{\labelstyle \ce}="24"
    \POS"1" \ar "4"|{\labelstyle \be}="14"
    \POS"24";"3"**{}, ?(0.2) \ar@{=>} ?(0.6)^{\labelstyle \cde}
    \POS"04";"1"**{}, ?(0.25) \ar@{=>} ?(0.5)_{\labelstyle \abe}
    \POS"14";"2"**{}, ?(0.25) \ar@{=>} ?(0.5)^{\labelstyle \bce}
    \else\fi
  }
}

\newcommand{\pentofpent}[1]{
  \def\baselen{#1}
  \begin{xy}
    0;<\baselen,0mm>:
    *{\xybox{
        \POS(0,-4)*[o]{\pent 0}="zero"
        \POS(16,40)*[o]{\pent 3}="three"
        \POS(72,40)*[o]{\pent 1}="one"
        \POS(88,-4)*[o]{\pent 4}="four"
        \POS(44,-36)*[o]{\pent 2}="two"
        \ar@<1ex>"zero";"three"^-{\objectstyle\abcd}
        \ar@<1ex>"three";"one"^-{\objectstyle\abde}
        \ar@<1ex>"one";"four"^-{\objectstyle\bcde}
        \ar@<-1ex>"zero";"two"_-{\objectstyle\acde}
        \ar@<-1ex>"two";"four"_-{\objectstyle\abce}
        \ar@{=>}(44,-5);(44,+15)^{\objectstyle\abcde}
     }}
  \end{xy}
}

% Local Variables:
% mode: LaTeX
% TeX-master: "main.tex"
% TeX-PDF-mode: t
% TeX-parse-self: t
% TeX-auto-save: t
% End: 


%% TITLE
\title{Runners in action}

%% AUTHORS
\author{Danel Ahman \and Andrej Bauer}
\institute{%
  Faculty of Mathematics and Physics\\
  University of Ljubljana, Slovenia}

\maketitle

%% ABSTRACT
\begin{abstract}

Runners of algebraic effects, also known as comodels, provide a mathematical
model of resource management. We show that they also give rise to a programming
concept that models top-level external resources, as well as allows programmers
to modularly define their own intermediate ``virtual machines''.
We capture the core ideas of programming with runners in an equational calculus $\lambdacoop$, which we equip  
with a sound and coherent denotational semantics that guarantees the linear use of
resources and execution of finalisation code. We accompany $\lambdacoop$ with
examples of runners in action, provide a prototype language implementation in \pl{OCaml},
as well as a \pl{Haskell} library based on $\lambdacoop$.
\keywords{Runners, comodels, algebraic effects, resources, finalisation.}
\end{abstract}

%%%%%%%%%%%%%%%%%%%%%%%%%%%%%%%%%%%%%%%%%%%%%%%%%%%%

%% CODE SNIPPETS TYPESETTING

\definecolor{codegreen}{rgb}{0,0.6,0}
\definecolor{codegray}{rgb}{0.5,0.5,0.5}
\definecolor{codepurple}{rgb}{0.58,0,0.82}
\definecolor{backcolour}{rgb}{0.95,0.95,0.92}

\colorlet{keywordColor}{NavyBlue} % the color of language keywords
\colorlet{rulenameColor}{Gray} % the color of rule names

\def\lstlanguagefiles{coop.tex}
\lstset{language=coop,upquote=true}
\let\ls\lstinline

%%%%%%%%%%%%%%%%%%%%%%%%%%%%%%%%%%%%%%%%%%%%%%%%%%%%

%% PAPER CONTENTS

% Customize the display of references to sections, subsections, subsubsections, theorems, and propositions.
\crefformat{section}{\S#2#1#3}
\Crefformat{section}{\S#2#1#3}

\crefformat{subsection}{\S#2#1#3}
\Crefformat{subsection}{\S#2#1#3}

\crefformat{subsubsection}{\S#2#1#3}
\Crefformat{subsubsection}{\S#2#1#3}

\crefformat{theorem}{Thm.~#2#1#3}
\Crefformat{theorem}{Thm.~#2#1#3}

\crefformat{proposition}{Prop.~#2#1#3}
\Crefformat{proposition}{Prop.~#2#1#3}

\crefformat{figure}{Fig.~#2#1#3}
\Crefformat{figure}{Fig.~#2#1#3}


\section{Introduction \label{sec:introduction}}

When probed at very short wavelengths, QCD is essentially a theory of
free \index{Partons}`partons' --- quarks and gluons --- which only
scatter off one another through relatively small quantum corrections,
that can be systematically calculated. 
But at longer wavelengths, of order the size of the proton $\sim
1\mathrm{fm} = 10^{-15}\mathrm{m}$,  
we see strongly bound towers of hadron resonances emerge, with string-like
potentials building up if we try to separate their partonic
constituents. Due to our
inability to perform analytic calculations in 
strongly coupled field theories, QCD is therefore 
still only partially solved. Nonetheless,  all its features, across all
distance scales, are believed to be encoded in a single one-line
formula of alluring simplicity; the
\index{QCD!Lagrangian}%
Lagrangian\footnote{Throughout these notes we let it be implicit that
  ``Lagrangian'' really refers to Lagrangian density, ${\cal L}$, the
  four-dimensional space-time integral of which is the action.} of QCD.

The consequence for collider physics is that some parts of QCD can be
calculated in terms of the fundamental parameters of the Lagrangian,
whereas others must be expressed through models or functions whose effective 
parameters are not a priori calculable but which can be constrained
by fits to data. 
However, even in the absence of a
perturbative expansion, there are still several strong theorems which
hold, and which can be used to give relations between seemingly
different processes. (This is, e.g., the reason it makes sense to 
measure the partonic substructure of the proton in $ep$ collisions and
then re-use the same parametrisations for $pp$
collisions.) Thus, in the chapters 
dealing with phenomenological models we shall emphasise that the loss
of a factorised perturbative expansion is not equivalent to a total
loss of predictivity.   

An alternative approach would be to give up on calculating QCD 
and use leptons instead. Formally, this amounts to summing inclusively over
strong-interaction phenomena, when such are present. While such a
strategy might succeed in replacing what we do know about QCD by
``unity'', however, even the most adamant chromophobe would acknowledge
the following basic facts of collider physics for the next decade(s): 
1) At the LHC, the initial states are
hadrons, and hence, at
the very least, well-understood and precise parton distribution
functions (PDFs) will be required; 2) high precision will mandate
 calculations to higher orders in perturbation theory, 
which in turn will involve more QCD; 3) the requirement of lepton
\emph{isolation} makes the very definition of a lepton
 depend implicitly on QCD and 4) 
 the rate of jets that are misreconstructed as leptons in
 the experiment depends explicitly on it. 
And, 5) though many new-physics signals \emph{do} give observable
signals in the lepton sector, this is far from guaranteed, nor is it
exclusive when it occurs. 
 It would therefore be  unwise not to attempt to solve QCD to the best
 of our ability, the better to prepare ourselves for both the largest
 possible discovery reach and the highest attainable subsequent
 precision. 

Furthermore, QCD is the richest gauge theory we have so far
 encountered. Its emergent phenomena, unitarity properties, colour structure, 
 non-perturbative dynamics, quantum vs.\ classical limits, 
interplay between scale-invariant and
 scale-dependent properties, and its wide
 range of phenomenological applications, are still very much topics of
 active investigation, about which we continue to learn.  

In addition, or perhaps as a consequence, the field of QCD is
currently experiencing something of a revolution. On the perturbative
side, new methods to compute scattering amplitudes with very high
particle multiplicities are being developed, together with advanced
techniques for combining such amplitudes with all-orders resummation
frameworks. On the non-perturbative side, the wealth of data on
soft-physics processes from the LHC is
forcing us to reconsider the reliability of the standard fragmentation
models, and heavy-ion collisions are providing new insights into
the collective behavior of hadronic matter. The
study of cosmic rays impinging on the Earth's
atmosphere challenges our ability to extrapolate fragmentation models
from collider energy scales to the region of ultra-high energy cosmic
rays. And finally, dark-matter annihilation processes in space  may produce 
hadrons, whose spectra are sensitive to the modeling 
of fragmentation.

In the following, we shall focus on QCD for mainstream 
collider physics. This includes the basics of SU(3), colour factors, the running
of $\alpha_s$, factorisation, 
hard processes, infrared safety, parton showers and matching, event generators, hadronisation, and the so-called underlying event. 
While not covering everything, hopefully these topics can also serve
at least as stepping stones to more specialised
issues that have been left out, such as twistor-inspired techniques, 
heavy flavours, polarisation, or forward physics, or to topics more tangential to
other fields, such as axions, lattice QCD, or heavy-ion physics.  

\subsection{A First Hint of Colour}
Looking for new physics, as we do now at the LHC, it is instructive to 
consider the story of the discovery of colour. The first hint was
arguably the $\Delta^{++}$ \index{Baryons}baryon, discovered in 
1951~\cite{Brueckner:1952zz}. The title and part of the abstract from this
historical paper are reproduced in \figRef{fig:Delta}.
\begin{figure}[t]
\begin{center}
\begin{tabular}{c}
\colorbox{gray}{\includegraphics*[scale=0.75]{DeltaTitle.pdf}}\\[5mm]
\hspace*{2mm}\begin{minipage}{0.88\textwidth}
\small\sl  ``[...] It is concluded that the apparently anomalous features of the
scattering can be interpreted to be an indication of a resonant
meson-nucleon interaction corresponding to a nucleon isobar with spin
$\frac32$, isotopic spin $\frac32$, and with an excitation energy of
$277\,$MeV.''\\[1mm]
\end{minipage}
\end{tabular}
\caption{The title and part of the abstract of the 1951 paper
  \cite{Brueckner:1952zz} (published in 1952) in which the existence 
  of the $\Delta^{++}$ baryon was deduced, based on data from Sachs and
  Steinberger at Columbia~\cite{Chedester:1951sc}  and from Anderson,
  Fermi, Nagle, et al.~at Chicago~\cite{Fermi:1952zz}. Further studies 
  at Chicago were quickly performed
  in~\cite{Anderson:1952nw,Anderson:1952zza}. See also the memoir by
  Nagle~\cite{nagle1984delta}. 
\label{fig:Delta}}  
\end{center}
\end{figure}
In the context of the \index{Quarks}quark model --- which first
had to be developed, successively joining together the notions of 
spin, isospin, strangeness, and 
the \index{Eightfold way}eightfold way\footnote{In physics, the ``eightfold way''
refers to the classification of the lowest-lying pseudoscalar
\index{Mesons}mesons and 
\index{SU(3)!Of Flavour}%
spin-1/2 \index{Baryons}baryons within \index{Octet}octets in SU(3)-flavour space ($u,d,s$). The
$\Delta^{++}$ is part of a spin-3/2 baryon \index{Decuplet}decuplet, a ``tenfold way'' in this
terminology.} 
--- the \index{Flavour}flavour and spin content of the $\Delta^{++}$
baryon is: 
\begin{equation}
\left\vert \Delta^{++} \right> = \left\vert
\,u_\uparrow\ u_\uparrow\ u_\uparrow \right>~,
\end{equation} 
clearly a highly symmetric configuration. However, since 
the $\Delta^{++}$ is a fermion, it must have an overall
antisymmetric wave function. In 1965, fourteen years after its
discovery, this was finally understood by the introduction of colour
\index{SU(3)}%
\index{SU(3)!Of Colour}%
as a new quantum number associated with the group SU(3)
\cite{Greenberg:1964pe,Han:1965pf}. The $\Delta^{++}$ wave function can now be made
antisymmetric by arranging its three quarks antisymmetrically 
in this new degree of freedom, 
\begin{equation}
\left\vert \Delta^{++} \right> = \epsilon^{ijk} \left\vert
\,u_{i\uparrow}\ u_{j\uparrow}\ u_{k\uparrow}\right>~,
\end{equation} 
hence solving the mystery.

More direct experimental tests of the number of colours were provided first by
measurements of the decay width of $\pi^0\to \gamma\gamma$ decays, which 
is proportional to $N_C^2$, 
and later by the famous ``R'' ratio in
$e^+e^-$ collisions ($R=\sigma(e^+e^-\to q\bar{q})/\sigma(e^+e^-\to
\mu^+\mu^-)$), which is proportional to $N_C$, see
e.g.~\cite{Dissertori:2003pj}. 
Below, in \SecRef{sec:L} we shall see how to
calculate such colour factors. 

\subsection{The Lagrangian of QCD \label{sec:L}}
\index{QCD!Lagrangian}%
Quantum Chromodynamics is based on the gauge group
\index{SU(3)}$\mrm{SU(3)}$, the 
Special Unitary group in 3 (complex) dimensions, whose elements 
are the set of unitary $3\times 3$ matrices with determinant one. 
\index{Fundamental representation}%
\index{SU(3)!Fundamental representation}%
Since there are 9 linearly independent unitary complex
matrices\footnote{A complex $N\times N$ matrix has $2N^2$ degrees of
  freedom, on which unitarity provides $N^2$ constraints.}, one of
which has determinant $-1$, there are a total of 8
independent directions in this matrix space, corresponding to eight
different generators as compared
with the single one of QED. In the context of QCD, we normally
represent this group using the 
so-called \emph{fundamental}, or \emph{defining}, representation, in
which the generators of $\mrm{SU(3)}$ appear as a set of eight traceless and
hermitean matrices, to which we return below.  
We shall refer to indices enumerating
the rows and columns of these matrices  (from 1 to 3) as
\emph{fundamental} indices, and we use the letters $i$,
$j$, $k$, \ldots, to denote them.
\index{Adjoint representation}%
\index{SU(3)!Adjoint representation}%
We refer to indices enumerating the generators (from 1 to 8),
as \emph{adjoint} 
indices\footnote{The dimension of the \emph{adjoint}, or
  \emph{vector}, representation is equal to the number of generators,
  $N^2-1=8$ for $\mrm{SU(3)}$, while the  
\index{Fundamental representation}%
\index{SU(3)!Fundamental representation}%
dimension of the fundamental representation is
  the degree of the group, $N=3$ for $\mrm{SU(3)}$.}, and we use the first
letters of the alphabet ($a$, $b$, $c$, \ldots) to denote them. 
These matrices can operate both on each other (representing
combinations of successive gauge transformations) and on a set of
$3$-vectors, the latter of 
which represent \index{Quarks}quarks in colour 
space; the quarks are \emph{triplets} under $\mrm{SU(3)}$. The matrices can be
thought of as representing gluons in colour 
space (or, more precisely, the gauge transformations carried out by
gluons), hence there are
eight different gluons; the gluons are \emph{octets} under $\mrm{SU(3)}$. 

\index{QCD!Lagrangian}%
The Lagrangian density of QCD is 
\begin{equation}
{\cal L} = \bar{\psi}_q^i(i\gamma^\mu)(D_\mu)_{ij}\psi_q^j - m_q
\bar{\psi}_q^i\psi_{qi} - \frac14 F^a_{\mu\nu}F^{a\mu\nu}~,\label{eq:L}
\end{equation}
where $\psi_q^i$ denotes a quark field with
(fundamental) colour index $i$, 
$\psi_q = ({\textcolor{red}{\psi_{qR}}},{\color{green}\psi_{qG}}, 
{\color{blue}\psi_{qB}})^T$, 
$\gamma^\mu$ is a Dirac matrix that expresses the
vector nature of the strong interaction, with $\mu$ being a Lorentz
vector index, $m_q$ allows for the
possibility of non-zero \index{Quarks}quark masses (induced by the
standard Higgs 
mechanism or similar), $F^a_{\mu\nu}$ is the gluon field strength 
tensor for a gluon\footnote{The definition of the gluon field strength
  tensor will be given below in \eqRef{eq:F}.} with (adjoint) 
colour index $a$ (i.e., $a\in[1,\ldots,8]$), 
and $D_\mu$ is the covariant derivative in QCD,
\begin{equation}
(D_{\mu})_{ij} = \delta_{ij}\partial_\mu - i g_s t_{ij}^a A_\mu^a~,\label{eq:D}
\end{equation}
\index{QCD!Coupling}
with $g_s$ the \index{alphaS@$\alpha_s$}strong coupling (related to
$\alpha_s$ by $g_s^2 = 4\pi 
\alpha_s$; we return to the strong coupling in more detail below), 
$A^a_\mu$  the gluon field with 
colour index $a$, and $t_{ij}^a$ proportional to the hermitean and
traceless \index{Gell-Mann matrices|see{SU(3)}}Gell-Mann matrices of $\mrm{SU(3)}$, 
\index{SU(3)!Generators}%
\begin{equation}
\mbox{\includegraphics*[scale=1.0]{gell-mann}}~.
\end{equation}
These generators are just the $\mrm{SU(3)}$ analogs of the
Pauli matrices in 
$\mrm{SU(2)}$. 
By convention, the constant of proportionality is normally
taken to 
be 
\begin{equation}
t^a_{ij} = \frac12 \lambda^a_{ij}~. \label{eq:t}
\end{equation}
\index{QCD!Coupling}
This choice in turn determines the normalisation of the coupling
$g_s$, via \eqRef{eq:D}, and
fixes the values of the $\mrm{SU(3)}$ \index{Casimirs}Casimirs and structure constants, to which we return below. 

An example of the colour flow for a
quark-gluon interaction in colour 
space is given in \figRef{fig:qg}.
\begin{figure}[t]
\begin{center}
\begin{minipage}[h]{4.6cm}
\begin{center}
$A^1_\mu$\\
\includegraphics*[scale=0.75]{qgv.pdf}\\[-3mm]
$\psi_{q\textcolor{green}{G}}$\hfill$\psi_{q\textcolor{red}{R}}$
\end{center}
\end{minipage}~~~
\parbox{0.4\textwidth}{
$
\begin{array}{ccccc}
\propto & - \frac{i}{2} g_s & \bar{\psi}_{q\color{red}R}  & \lambda^{1} & \psi_{q\color{green}G} 
\\[2mm]
= & -\frac{i}{2}g_s & \left(\begin{array}{ccc} \textcolor{red}{1} & \color{green} 0 &
  \color{blue} 0 
\end{array}\right) & 
\left(\begin{array}{ccc}
0 & 1 & 0  \\
1 & 0 & 0 \\
0 & 0 & 0
\end{array}\right) & 
 \left(\begin{array}{c}
\textcolor{red}{0} \\
\color{green}1 \\
\color{blue}0
\end{array}\right) \end{array}
$}
\caption{Illustration of a 
\index{Quarks}\index{Gluons}$qqg$ vertex in QCD, before
  summing/averaging over colours: a gluon in a state represented by $\lambda^1$
  interacts with quarks in the states $\psi_{qR}$ and
  $\psi_{qG}$. \label{fig:qg}}
\end{center}
\end{figure}
Normally, of course, we sum over all the colour indices, so this
example merely gives a pictorial representation of what one particular
(non-zero) term in the colour sum looks like.


\subsection{Colour Factors}
\index{QCD!Colour factors}
\index{Colour factors}%
\index{Colour-space indices|see{Colour connections}}%
\index{Matrix elements}%
Typically, we do not measure colour in the final state ---
instead we average over all possible incoming colours and sum over all
possible outgoing ones, wherefore QCD scattering amplitudes (squared) in
practice always contain sums over quark fields contracted with
\index{SU(3)!Generators}Gell-Mann matrices. These contractions in turn
produce traces  
which yield the \index{Colour factors}\emph{colour factors} that are associated to each QCD
process, and which basically count the number of ``paths through
colour space'' that the process at hand can take\footnote{The
  convention choice represented by \eqRef{eq:t} introduces a
  ``spurious'' factor of 2 for each power of the coupling $\alpha_s$. 
Although one could in principle absorb that factor into a redefinition
of the coupling, effectively redefining the normalisation of ``unit
colour charge'', the standard definition of $\alpha_s$ is now so
entrenched that alternative choices would be counter-productive, at
least in the context of a pedagogical review.}.

A very simple example of a colour factor is given by the decay process $Z\to
q\bar{q}$. This vertex contains a simple $\delta_{ij}$ in colour
space; the outgoing quark and antiquark must have identical 
(anti-)col\-ours. Squaring the corresponding matrix element and summing over
final-state colours yields a colour factor of
\begin{equation}
e^+e^-\to Z \to q\bar{q}~~~:~~~\sum_{\mrm{colours}}|M|^2 \propto
\delta_{ij}\delta_{ji} = \mrm{Tr}\{\delta\} = N_C = 3~,
\end{equation}
since $i$ and $j$ are quark (i.e., 3-dimensional
fundamental) indices. This factor corresponds directly to the 3 different
``paths through colour space'' that the process at hand can take; the
produced quarks can be red, green, or blue. 

A next-to-simplest example is given by $q\bar{q}\to
\gamma^*/Z\to\ell^+\ell^-$ (usually referred to as the
\index{Drell-Yan}Drell-Yan 
process~\cite{Drell:1970wh}),  
which is just a crossing of the previous one. By crossing
symmetry, the squared matrix element, including the colour factor, is
exactly the same as before, but since the quarks are here incoming, we
must \emph{average} rather than sum over their colours, leading to
\begin{equation}
q\bar{q}\to Z\to e^+e^-~~~:~~~\frac{1}{9}\sum_{\mrm{colours}}|M|^2 \propto \frac19\delta_{ij}\delta_{ji} = \frac19 \mrm{Tr}\{\delta\} = \frac13~,
\end{equation}
where the colour factor now expresses a \emph{suppression} which can
be interpreted as due to the fact that only quarks of matching colours
are able to collide and produce a $Z$ boson. The chance that a quark
and an antiquark picked at random from the colliding hadrons have 
matching colours is $1/N_C$. 
\begin{figure}[t]
\end{figure}

Similarly, $\ell q \to
\ell q$ via $t$-channel photon exchange (usually called Deep
Inelastic Scattering --- \index{DIS}\index{Deep inelastic scattering|see{DIS}}DIS --- with ``deep'' referring to a 
large virtuality of the exchanged photon), constitutes yet another
crossing of the same basic process, 
see \figRef{fig:Zcrossings}. \index{Colour factors}The colour factor in this case 
comes out as unity. 
\begin{figure}[t]
\centering\vspace*{-8mm}
\begin{tabular}{ccc}
\rotatebox{360}{\includegraphics*[scale=0.93]{ee2qq}} \ \ 
& \ \ \includegraphics*[scale=0.93,angle=180,origin=c]{ee2qq}
\ \ & \ \ \includegraphics*[scale=0.9,angle=297,origin=c]{ee2qq}\\
Hadronic $Z$ decay & \index{Drell-Yan}Drell-Yan & \index{DIS}DIS \\[1mm]
$e^-e^+ \to \gamma^*/Z^0 \to q\bar{q}$ &
$q\bar{q} \to \gamma^*/Z^0 \to \ell^+\ell^-$ &
$\ell \bar{q} \stackrel{\gamma^*/Z^*}{\to} \ell \bar{q}$
\\[2mm] 
$\propto N_C$ & $\propto 1/N_C$ & $\propto 1$
\end{tabular}
\caption{Illustration of the three crossings of the interaction of a
  lepton current (black) with a \index{Quarks}quark current (red) 
  via an intermediate photon or
  $Z$ boson, with corresponding colour factors. \label{fig:Zcrossings}}
\end{figure}

To illustrate what happens when we insert (and sum over)
quark-gluon
vertices, such as the one depicted in \figRef{fig:qg}, we take
the process $Z\to3\,$jets. \index{Colour factors}The colour factor for
this process can be 
computed as follows, with the accompanying illustration showing a
corresponding diagram (squared) with explicit colour-space indices on
each vertex:\\
\index{Colour connections}
\begin{equation}
\mbox{
\begin{tabular}{cc}
\parbox{5.2cm}{
$Z \to qg\bar{q}$~~~:~~~\\
\[
\begin{array}{rcl}
\displaystyle\sum_{\mrm{colours}}|M|^2 & \propto & \displaystyle
\delta_{ij}t_{jk}^a t_{k\ell
    }^a\delta_{\ell i} \\
& = & \displaystyle
\mrm{Tr}\{t^at^a\}\\[4mm] & = & \displaystyle
  \frac12\mrm{Tr}\{\delta\} = 4~,
\end{array}
\]}
&
\parbox{8.5cm}{\includegraphics*[scale=0.6]{colFacZ3.pdf}
}
\end{tabular}}
\end{equation}
where the last $\mrm{Tr}\{\delta\} = 8$, since the trace runs over
the 8-dimensional adjoint indices. If we
want to ``count the paths through colour space'', we should leave out
the factor $\frac12$ which comes from the normalisation convention for
the $t$ matrices, \eqRef{eq:t}, hence this process can take 8
different paths through colour space, one for each gluon basis state.

The tedious task of taking traces over $t$
matrices can be greatly alleviated by use of the relations given in
\TabRef{tab:lambda}.  
\index{Traces in SU(3)|see{SU(3)}}%
\index{SU(3)!Trace relations}%
\index{QCD!Trace relations|see{SU(3)}}%
\begin{table}
\begin{center}
\scalebox{1.04}{\begin{tabular}{ccc}
\toprule
\index{SU(3)!Trace relations}Trace Relation & Indices & Occurs in Diagram Squared
\\
\midrule
$\mrm{Tr}\{t^at^b\} = T_R\, \delta^{ab}$ & $a,b\in[1,\ldots,8]$
& \parbox[c]{4cm}{\includegraphics*[scale=0.5]{traces1}}\\
$\sum_a t^a_{ij}t^a_{jk} = C_F\, \delta_{ik}$ &%
\parbox[c]{3cm}{\begin{center}
$a\in[1,\ldots,8]$\\
$i,j,k\in[1,\ldots,3]$\end{center}}
& \parbox[c]{4cm}{\includegraphics*[scale=0.5]{traces2}}\\
$\sum_{c,d} f^{acd} f^{bcd} = C_A\, \delta^{ab}$ & $a,b,c,d\in[1,\ldots,8]$
& \parbox[c]{4cm}{\includegraphics*[scale=0.5]{traces3}}\\
$ t^a_{ij}t^a_{k\ell} = T_R \left(\delta_{jk}\delta_{i\ell}
- \frac{1}{N_C}\delta_{ij}\delta_{k\ell}\right)$ & $i,j,k,\ell\in[1,\ldots,3]$
& \parbox[c]{4cm}{\includegraphics*[scale=0.5]{traces4}}\hspace*{-0.2cm}(Fierz)\\
\bottomrule
\end{tabular}}
\caption{Trace relations for $t$ matrices (convention-independent). 
 More relations
  can be found in \cite[Section 1.2]{Ellis:1991qj} and in 
  \cite[Appendix A.3]{Peskin:1995ev}.
\label{tab:lambda}}
\end{center}
\end{table}
In the standard normalisation convention for the \index{SU(3)}$\mrm{SU(3)}$ generators,
\eqRef{eq:t}, the \index{Casimirs}Casimirs of $\mrm{SU(3)}$ appearing in
\TabRef{tab:lambda} are\footnote{See, e.g., \cite[Appendix
    A.3]{Peskin:1995ev} for how to obtain the Casimirs in other
  normalisation conventions. As an example, choosing $t^a_{ij} = \lambda_{ij}^a/\sqrt{2}$ would yield $T_R=1$, $C_F=T_R(N_C^2-1)/N_C=8/3$, $C_A=3$.} 
\index{Casimirs}\index{TR@$T_R$}\index{CA@$C_A$}\index{CF@$C_F$}
\begin{equation}
T_R = \frac12 \hspace*{2cm} C_F = \frac43 \hspace*{2cm} C_A = N_C = 3~.
\end{equation}
In addition, the gluon self-coupling on the third line in
\TabRef{tab:lambda} involves factors of $f^{abc}$. These
\index{QCD!Structure constants|see{SU(3)}}%
are called the \index{SU(3)!Structure constants}\emph{structure constants} of QCD and they enter via 
the non-Abelian term in the \index{Gluons}gluon field strength tensor appearing in
\eqRef{eq:L}, 
\begin{equation}
F^a_{\mu\nu} = \underbrace{\partial_\mu A_\nu^a - \partial_\nu
  A^a_\mu}_{\mathrm{Abelian}} +
\underbrace{ g_s f^{abc} A_\mu^b A_\nu^c}_{\mathrm{non-Abelian}}~. \label{eq:F}
\end{equation}

\noindent\begin{minipage}[t]{0.46\textwidth}
The structure constants of $\mrm{SU(3)}$ are listed in the table to the
right. They define the \emph{adjoint}, or \emph{vector}, representation of $\mrm{SU(3)}$
and are related to the fundamental-representation generators via the
commutator relations
\begin{equation}
t^at^b - t^bt^a = [t^a,t^b] = i f^{abc} t_c~,
\end{equation} 
or equivalently,
\begin{equation}
if^{abc}~=~2\mrm{Tr}\{t^c[t^a,t^b]\}~.
\end{equation}
Thus, it is a matter of choice whether one prefers to express colour
space on a basis of fundamental-representation $t$ matrices, or via
the structure constants $f$, and one can go back and forth between the
two.
\end{minipage}%
\hfill%
\colorbox{darkgray}{%
\colorbox{lightgray}{%
\begin{minipage}[t]{0.46\textwidth}
\vspace*{3mm}\begin{center}
\textbf{Structure Constants of SU(3)}
\begin{equation}
f_{123} = 1
\end{equation}
\begin{equation}
f_{147} = f_{246} = f_{257} = f_{345} = \frac12
\end{equation}
\begin{equation}
f_{156} = f_{367} = -\frac12
\end{equation}
\begin{equation}
f_{458} = f_{678} = \frac{\sqrt{3}}{2}
\end{equation}
Antisymmetric in all indices\\[3mm]
All other $f_{abc}=0$\vspace*{3mm}\\
\end{center}
\end{minipage}%
}}\vskip1mm

\begin{figure}[t]
\begin{center}
\begin{minipage}[h]{4.6cm}
\begin{center}
$A_\nu^4(k_2)$\\
\includegraphics*[scale=0.75]{ggv.pdf}\\[-3mm]
$A^6_\rho(k_1)$\hfill$A_\mu^2(k_3)$
\end{center}
\end{minipage}~~~
\parbox{0.35\textwidth}{
$
\begin{array}{cccc}
\propto & - g_s \ f^{246} \!\! & \!\! [ (k_3 - k_2)^\rho g^{\mu\nu}  \\ 
& & +(k_2 - k_1)^\mu g^{\nu\rho} \\ 
& &+(k_1 - k_3)^\nu g^{\rho\mu}]
\end{array}
$}\vspace*{1mm}
\caption{Illustration of a \index{Gluons}$ggg$ vertex in QCD, before
  summing/averaging over colours: interaction between gluons in the 
  states $\lambda^2$, $\lambda^4$, and $\lambda^6$ is represented by
  the structure constant $f^{246}$. 
\label{fig:gg}}
\end{center}
\end{figure}
 Expanding the $F_{\mu\nu}F^{\mu\nu}$ term of the
Lagrangian using \eqRef{eq:F}, we see that there is a 3-gluon and a
4-gluon vertex that involve $f^{abc}$, the latter of which has two
powers of $f$ and two powers of the coupling. 

Finally, the last line of \TabRef{tab:lambda} is not really a trace
relation but instead a useful so-called Fierz transformation, which
expresses products of $t$ matrices in terms of Kronecker $\delta$ functions. 
It is often used, for instance, in shower Monte Carlo
applications, to assist in mapping between colour flows in $N_C = 3$,
in which cross sections and splitting probabilities are calculated, 
and those in $N_C\to\infty$ (``leading colour''), used to represent colour flow in
the MC ``event record''.

A \index{Gluons}gluon self-interaction vertex is
illustrated in \figRef{fig:gg}, to be compared with the quark-gluon
one in \figRef{fig:qg}. We remind the reader that gauge boson
self-interactions are a hallmark of non-Abelian theories and that their
presence leads to some of the main differences between QED and
QCD. One should also keep in mind 
that the \index{Colour factors}colour factor for the vertex in \figRef{fig:gg}, \index{CA@$C_A$}$C_A$, 
is roughly twice as large as that for a quark, \index{CF@$C_F$}$C_F$.

\subsection{The Strong Coupling \label{sec:coupling}}
\index{QCD!Coupling}
\index{Jets}
\index{alphaS@$\alpha_s$}To first approximation, QCD is 
\index{QCD!Scale invariance}\emph{scale invariant}. That is, if one
``zooms in'' on a QCD jet, one will find a repeated self-similar 
pattern of jets within jets within jets, reminiscent of
fractals. 
In the context of QCD, this property was originally 
called \index{Lightcone scaling|see{QCD Scale invariance}}light-cone scaling, or 
\index{Bjorken scaling|see{QCD Scale invariance}}Bj{\o}rken scaling. 
This type of scaling is closely related to the class of
angle-preserving symmetries, called \index{Conformal
invariance}\emph{conformal} symmetries. In physics 
today, the terms ``conformal'' and ``scale invariant'' are used 
interchangeably\footnote{Strictly speaking, conformal symmetry is more
restrictive than just scale invariance, but examples of
scale-invariant field theories that are not conformal are rare.}.
Conformal invariance is a mathematical property of several
QCD-``like'' theories which are now being studied (such as $N=4$
supersymmetric relatives of QCD). It is also 
related to the physics of so-called ``unparticles'', though that is a
relation that goes beyond the scope of these lectures.

Regardless of the labelling, 
if the  \index{alphaS@$\alpha_s$}strong coupling did not run (we shall
return to the running 
of the coupling below), Bj{\o}rken scaling would be absolutely true. QCD
would be a theory with a fixed coupling, the same at all scales. 
This simplified picture already captures some of the most important
properties of QCD, as we shall discuss presently.  

\index{QCD!Scale invariance}%
In the limit of exact Bj{\o}rken scaling --- QCD at fixed coupling
--- properties of high-energy interactions are determined 
only by \emph{dimensionless} kinematic quantities, such as scattering
angles (pseudorapidities) and ratios of energy
scales\footnote{Originally, the observed approximate agreement with
this was used as a powerful argument
for pointlike substructure in hadrons; since measurements at different
energies are sensitive to different resolution scales, independence of the absolute
energy scale is indicative of the absence of other fundamental
scales in the problem and hence of pointlike constituents.}.
For applications of QCD to high-energy collider physics, an important
consequence of Bj{\o}rken scaling is thus that the rate of 
\index{Parton showers}%
\index{Bremsstrahlung|see{Parton showers}}
bremsstrahlung
jets, with a given transverse momentum, scales in direct proportion to
the hardness 
of the fundamental partonic scattering process they are produced in
association with. This agrees well with our intuition about accelerated
charges; the harder you ``kick'' them, the harder the radiation they
produce.  

For instance, in the limit of exact scaling, a
measurement of the rate of 10-GeV jets produced in association with an
ordinary $Z$ 
boson could be used as a direct prediction of the rate of 100-GeV jets
that would be 
produced in association with a 900-GeV $Z'$ boson, and so 
forth. Our intuition about how many bremsstrahlung jets a given type of
process is likely to have should therefore be governed first and
foremost by the \emph{ratios} of scales that appear in that particular
process, as has been  highlighted in a number of studies focusing on
the mass and $p_\perp$ scales appearing, e.g., in
Beyond-the-Standard-Model (BSM) 
physics processes
\cite{Plehn:2005cq,Alwall:2008qv,Papaefstathiou:2009hp,Krohn:2011zp}. 
\index{QCD!Scale invariance}Bj{\o}rken scaling 
\index{Scale invariance|see{QCD}}
is also fundamental to the understanding of jet substructure in QCD, see, e.g.,
\cite{Vermilion:2011nm,Altheimer:2012mn}.  

\index{alphaS@$\alpha_s$!Running coupling}%
On top of the underlying scaling behavior, the running coupling will
introduce a dependence on the absolute scale, implying more radiation
at low scales than at high ones. The running is logarithmic with
\index{alphaS@$\alpha_s$!beta function}%
energy, and is governed by the so-called \emph{beta function}, 
\index{alphaS@$\alpha_s$}
\begin{equation}
Q^2 \frac{\partial \alpha_s}{\partial Q^2} = \frac{\partial
  \alpha_s}{\partial \ln Q^2} =
\beta(\alpha_s)~, \label{eq:running}
\end{equation}
where the function driving the energy dependence, the \index{Beta function}{beta
  function}, is defined as
\begin{equation}
\beta(\alpha_s) = -\alpha_s^2(b_0 +
b_1\alpha_s + b_2\alpha_s^2 + \ldots)~,\label{eq:beta}
\end{equation}
with LO (1-loop) and NLO (2-loop) coefficients
\begin{eqnarray}
b_0 & = & \frac{11C_A - 4 T_R n_f}{12\pi}~,\\[3mm]
b_1 & = & \frac{17C_A^2 - 10 T_R C_A n_f - 6 T_R C_F n_f}{24\pi^2} ~=~
\frac{153-19 n_f}{24\pi^2}~.\label{eq:b}
\end{eqnarray}
In the $b_0$ coefficient, the first term is due to
\index{Gluons!Contribution to beta function}gluon loops while the
second is due to \index{Quarks!Contribution to beta function}quark
ones. Similarly, the first 
term of the $b_1$ coefficient arises from double gluon loops,
while the second and third represent mixed quark-gluon ones. 
At higher loop orders, the $b_i$ coefficients depend explicitly on the
renormalisation scheme that is used. A brief discussion can be found in the
PDG review on QCD~\cite{pdg2012}, with more elaborate ones
contained in \cite{Dissertori:2003pj,Ellis:1991qj}. 
Note that, if there are additional coloured particles beyond the
Standard-Model ones, loops involving those particles enter
 at energy scales above the masses of the
new particles, thus modifying the  \index{alphaS@$\alpha_s$}running of the coupling at high scales. 
This is discussed, e.g., for supersymmetric models in
\cite{Martin:1997ns}. For the running of other SM couplings, see
e.g.,~\cite{Langacker:2010zza}. 

\index{alphaS@$\alpha_s$!Running coupling}%
Numerically, the value of the  \index{alphaS@$\alpha_s$}strong coupling is usually specified by
giving its value at the specific 
reference scale $Q^2=M^2_Z$, from which we can obtain its
value at any other scale by solving \eqRef{eq:running}, 
\begin{equation}
\alpha_s(Q^2) = \alpha_s(M_Z^2) \frac{1}{1+b_0
  \alpha_s(M_Z^2)\ln\frac{Q^2}{M_Z^2} + {\cal O}(\alpha_s^2)}~,
\label{eq:alphaq2}
\end{equation}
with relations including the ${\cal O}(\alpha_s^2)$ terms 
available, e.g., in \cite{Ellis:1991qj}. 
Relations between scales 
not involving $M_Z^2$ can obviously be obtained by just replacing $M_Z^2$
by some other scale $Q'^2$ everywhere in \eqRef{eq:alphaq2}. A
comparison of running at one- and two-loop order, in both cases starting from
$\alpha_s(M_Z)=0.12$, is given in \figRef{fig:asRun}.
\begin{figure}[t]
\centering
\includegraphics*[scale=0.45]{vc-alphaS.pdf}
\caption{Illustration of the running of
 $\alpha_s$ at 1- (open 
  circles) and 2-loop
  order (filled circles), 
starting from the same value of $\alpha_s(M_Z)=0.12$. 
\label{fig:asRun}}
\end{figure}
As is evident from the figure, the 2-loop running is somewhat faster
than the 1-loop one.

\index{alphaS@$\alpha_s$!Running coupling}%
As an application, let us prove that the 
logarithmic running of the coupling implies that an intrinsically 
multi-scale problem can be converted to a single-scale one, up to
corrections suppressed by two powers of $\alpha_s$, 
by taking the geometric mean of the scales involved. This follows from
expanding an arbitrary product of individual  \index{alphaS@$\alpha_s$}$\alpha_s$ factors around an
arbitrary scale $\mu$, using \eqRef{eq:alphaq2}, 
\begin{eqnarray}
\alpha_s(\mu_1)\alpha_s(\mu_2)\cdots\alpha_s(\mu_n) & = &
\prod_{i=1}^{n} \alpha_s(\mu) \left(1 +
b_0\,\alpha_s\ln\left(\frac{\mu^2}{\mu_i^2}\right) + {\cal O}(\alpha_s^2)\right)
\nonumber\\[2mm]
& = & \alpha_s^n(\mu) \left(1 + b_0\, \alpha_s \ln \left(
 \frac{\mu^{2n}}{\mu_1^2\mu_2^2\cdots\mu_n^2}\right) +  {\cal
   O}(\alpha_s^2) \right)~,
\end{eqnarray}
whereby the specific single-scale choice $\mu^n =
\mu_1\mu_2\cdots\mu_n$ (the geometric mean) can
be seen to push the difference between the two sides of the equation one order higher
than would be the case for any other combination of scales\footnote{In
  a fixed-order calculation, the individual scales $\mu_i$,
would correspond, e.g., to the $n$ hardest scales appearing in an infrared
safe sequential clustering algorithm applied to the given momentum
configuration.}. 

The appearance of the number of \index{Flavour}flavours, $n_f$, in $b_0$ implies that the
slope of the running depends on the number of contributing
\index{Flavour}flavours. Since full QCD is best approximated by $n_f=3$
below the charm threshold, by $n_f=4$ and $5$ from there to the $b$
and $t$ thresholds, respectively, and then by $n_f=6$ at scales
higher than $m_t$, it is therefore important to be aware that 
the running changes slope across quark \index{Flavour}flavour
thresholds. Likewise, it would change across the threshold for any coloured
new-physics particles that might exist, with a magnitude depending on
the particles' colour and spin quantum numbers.

\index{alphaS@$\alpha_s$!Running coupling}%
\index{alphaS@$\alpha_s$}
The negative overall sign of \eqRef{eq:beta}, combined with the fact
that $b_0 > 0$ (for $n_f \le 16$), leads to the famous
result\footnote{
Perhaps the highest pinnacle of fame for \eqRef{eq:beta} was reached
when the sign of it featured in an episode of the TV series ``Big Bang
Theory''.} 
that the QCD coupling effectively \emph{decreases} with
 energy, called \index{Asymptotic freedom}asymptotic 
freedom, for the discovery of which the \index{Nobel prize}Nobel prize in physics was
awarded to D.~Gross, H.~Politzer, and F.~Wilczek in 2004. An extract
of the prize announcement runs as follows:
\begin{center}
\begin{minipage}{0.84\textwidth}
\sl  What this year's Laureates discovered was something that, at
first sight, seemed completely contradictory. The interpretation of
their mathematical result was that the closer the quarks are to each
other, the \emph{weaker} is the ``colour charge''. When the quarks are
really close to each other, the force is so weak that they behave
almost as free particles\footnote{More correctly, it is the coupling
  rather than the  
  force which becomes weak as the distance decreases. 
  The $1/r^2$ Coulomb singularity of the force is only dampened, not removed, 
  by the diminishing coupling.}. 
This phenomenon is called ``asymptotic
freedom''. The converse is true when the quarks move apart: the force
becomes stronger when the distance increases\footnote{More correctly,
 it is the potential which grows, linearly, while the force becomes
 constant.}. 
\end{minipage}
\end{center}

\index{Running coupling|see{alphaS@$\alpha_s$}}%
\index{alphaS@$\alpha_s$!Running coupling}%
Among the consequences of \index{Asymptotic freedom}asymptotic freedom is that perturbation
theory becomes better behaved at higher absolute energies, due to the
effectively decreasing coupling. Perturbative calculations for our
900-GeV $Z'$ boson from before should therefore be slightly faster
converging than equivalent calculations for the 90-GeV one. 
Furthermore, since the running of  \index{alphaS@$\alpha_s$}$\alpha_s$ explicitly
breaks Bj{\o}rken scaling, we also expect to see small changes in jet
shapes and in jet production ratios as we vary the energy. For
instance, since high-$p_\perp$ jets
start out with a smaller effective coupling, their intrinsic shape
(irrespective of boost effects) is
somewhat narrower than for low-$p_\perp$ jets, an issue which can be
important for jet calibration. Our current understanding of the
running of the QCD coupling is summarised by the plot in
\figRef{fig:alphas}, taken from a recent comprehensive review by S.\ Bethke
\cite{pdg2012,Bethke:2012jm}. A complementary up-to-date overview of
$\alpha_s$ determinations can be found in~\cite{d'Enterria:2015toz}. 

\index{alphaS@$\alpha_s$!Running coupling}%
As a final remark on \index{Asymptotic freedom}asymptotic freedom, note
that the decreasing 
value of the  \index{alphaS@$\alpha_s$}strong coupling with energy must eventually cause it to
become comparable to the electromagnetic and weak ones, at some energy
scale. Beyond that point, which may lie at energies of order
$10^{15}-10^{17}\,$GeV (though it may be lower if as yet undiscovered
particles generate large corrections to the running), 
we do not know  what the further evolution of the combined theory will 
actually look like, or whether it will continue to exhibit
\index{Asymptotic freedom}asymptotic
freedom. 

\index{alphaS@$\alpha_s$}%
\index{alphaS@$\alpha_s$!Running coupling}%
\index{alphaS@$\alpha_s$!LambdaQCD@$\Lambda_{\mathrm{QCD}}$}%
Now consider what happens when we run the coupling in the other
direction, towards smaller energies. 
\begin{figure}[t]
\begin{center}\hspace*{-0.25cm}
\parbox[c]{3.1cm}{\includegraphics*[scale=0.65]{arr-ir.pdf}}
\parbox[c]{8cm}{\includegraphics*[scale=0.5]{asq-2011.pdf}}\hspace*{-1mm}
\parbox[c]{3.1cm}{\includegraphics*[scale=0.65]{arr-uv.pdf}}
\caption{Illustration of the running of $\alpha_s$ in a theoretical
  calculation (band) and in physical processes at
  different characteristic scales, from
  \cite{pdg2012,Bethke:2012jm}. The little kinks at $Q=m_{c}$ and
  $Q=m_b$ are
  caused by discontinuities in the running across the flavour
  thresholds.\label{fig:alphas}}  
\end{center}           
\end{figure}
Taken at face value, the numerical value of the coupling diverges
rapidly at scales below 1 GeV, as illustrated by the curves
disappearing off the left-hand edge of the plot in
\figRef{fig:alphas}. To make this divergence
explicit, one can rewrite
\eqRef{eq:alphaq2} in the following form, 
 \index{alphaS@$\alpha_s$}
\begin{equation}
\alpha_s(Q^2) = \frac{1}{b_0 \ln \frac{Q^2}{\Lambda^2}}~,\label{eq:alphasLam}
\end{equation}
where 
\begin{equation}
\Lambda \sim 200\, \mbox{MeV}
\end{equation}
\index{alphaS@$\alpha_s$!LambdaQCD@$\Lambda_{\mathrm{QCD}}$}%
\index{alphaS@$\alpha_s$!Landau Pole|see{$\Lambda_{\mathrm{QCD}}$}}%
\index{LambdaQCD@$\Lambda_{\mathrm{QCD}}$|see{alphaS@$\alpha_s$}}%
specifies the energy scale at which the perturbative coupling would nominally become
infinite, called the Landau pole. (Note, however, that this only
parametrises the purely \emph{perturbative} result, which is not
reliable at \index{Strong coupling}strong coupling, so \eqRef{eq:alphasLam} should 
not be taken to imply that the physical behavior of full QCD should
exhibit a divergence for $Q\to \Lambda$.) 

\index{alphaS@$\alpha_s$}%
\index{alphaS@$\alpha_s$!Running coupling}%
\index{alphaS@$\alpha_s$!LambdaQCD@$\Lambda_{\mathrm{QCD}}$}%
Finally, one should be aware that there is a multitude of different
ways of defining both $\Lambda$ and $\alpha_s(M_Z)$. At the very
least, the numerical value one obtains depends both on the
renormalisation scheme used (with the dimensional-regularisation-based
``modified minimal subtraction'' scheme, $\overline{\mbox{MS}}$, being the
most common one) and on the perturbative order of the calculations 
used to extract them. As a rule of thumb, fits to experimental data typically yield 
smaller values for $\alpha_s(M_Z)$ the higher the order of the
calculation used to extract it (see, e.g.,
\cite{Bethke:2009jm,Dissertori:2009ik,Bethke:2012jm,pdg2012}), with  $
\alpha_s(M_Z)\vert_{\mrm{LO}} \gsim \alpha_s(M_Z)\vert_{\mrm{NLO}}
\gsim \alpha_s(M_Z)\vert_{\mrm{NNLO}}$. 
Further, since the number of \index{Flavour}flavours changes the slope
of the running, the location of the Landau pole for fixed
$\alpha_s(M_Z)$ depends explicitly on the number of \index{Flavour}flavours used in
the running. Thus each value of $n_f$ is associated with its own
value of $\Lambda$, with the following matching relations across
thresholds guaranteeing continuity of the coupling at one loop,
\index{LambdaQCD@$\Lambda_{\mathrm{QCD}}$|see{$\alpha_s$}}
\index{alphaS@$\alpha_s$!LambdaQCD@$\Lambda_{\mathrm{QCD}}$}%
\begin{eqnarray}
n_f = 5 \leftrightarrow 6 ~~~:~~~~~~\Lambda_6 = \Lambda_5
  \left(\frac{\Lambda_5}{m_t}\right)^{\frac{2}{21}} & & 
\Lambda_5 = \Lambda_6
  \left(\frac{m_t}{\Lambda_6}\right)^{\frac{2}{23}} ~, \\[2mm]
n_f = 4 \leftrightarrow 5 ~~~:~~~~~~\Lambda_5 = \Lambda_4
  \left(\frac{\Lambda_4}{m_b}\right)^{\frac{2}{23}} & & 
\Lambda_4 = \Lambda_5
  \left(\frac{m_b}{\Lambda_5}\right)^{\frac{2}{25}} ~, \\[2mm]
n_f = 3 \leftrightarrow 4 ~~~:~~~~~~\Lambda_4 = \Lambda_3 
  \left(\frac{\Lambda_3}{m_c}\right)^{\frac{2}{25}} & &
\Lambda_3 = \Lambda_4 
  \left(\frac{m_c}{\Lambda_4}\right)^{\frac{2}{27}} ~.
\end{eqnarray}

\index{alphaS@$\alpha_s$}%
\index{alphaS@$\alpha_s$!Running coupling}%
It is sometimes stated that QCD only has a single free
parameter, the  \index{alphaS@$\alpha_s$}strong coupling. 
However, even in the perturbative
region, the beta function depends explicitly on the number of
quark \index{Flavour}flavours, as we have seen, and thereby also on the quark
masses. Furthermore, in the non-perturbative region around or below
$\Lambda_{\mrm{QCD}}$, the value of the 
perturbative coupling, as obtained, e.g., from \eqRef{eq:alphasLam},
gives little or no insight into the behavior of the full theory. 
Instead, universal functions (such as parton densities, form factors,
fragmentation functions, etc), effective theories (such as the
Operator Product Expansion, Chiral Perturbation Theory, or Heavy Quark
Effective Theory), or phenomenological models (such as Regge Theory or
the String and Cluster Hadronisation Models) must be used, which in
turn depend on additional non-perturbative parameters whose relation to, e.g.,
$\alpha_s(M_Z)$, is not a priori known. 

\index{Lattice QCD}
For some of these questions,
such as hadron masses, lattice QCD can furnish important
additional insight, but for multi-scale and/or time-evolution
problems, the applicability of lattice methods is still severely
restricted; the lattice formulation of QCD requires 
  a Wick rotation to
  Euclidean space. The time-coordinate can then be treated on an
  equal footing with the other dimensions, but intrinsically
  Minkowskian problems, such as the time evolution of a system, are
   inaccessible. The limited size of current lattices
  also severely constrain the scale hierarchies that it is possible to
  ``fit'' between the lattice spacing and the lattice size. 

\index{Landau pole|see{$\alpha_s$}}%
\index{QCD!Landau Pole|see{$\alpha_s$}}%
\index{Renormalisation|see{$\alpha_s$}}%
\index{QCD!Renormalisation|see{$\alpha_s$}}%

\subsection{Colour States}
\index{Coherence}%
A final example of the application of the underlying $\mrm{SU(3)}$ group
theory to QCD is given by considering which colour states we can
obtain by combinations of quarks and gluons. The simplest example of
this is the combination of a quark and antiquark. We can form a total
of nine different colour-anticolour combinations, which fall into two
irreducible representations of $\mrm{SU(3)}$:
\begin{equation}
3 \otimes \overline{3} = 8 \oplus 1~.\label{eq:33bar}
\end{equation}
The singlet corresponds to the symmetric wave function 
$\frac{1}{\sqrt{3}}\left(\left|R\bar{R}\right>+\left|G\bar{G}\right>+\left|B\bar{B}\right>\right)$, 
which is invariant under $\mrm{SU(3)}$ transformations (the definition of a
singlet). The other eight linearly independent 
combinations (which can be represented by one for each Gell-Mann
matrix, with the singlet corresponding to the identity matrix) transform
into each other under $\mrm{SU(3)}$. Thus, although we sometimes talk about
colour-singlet states as 
being made up, e.g., of ``red-antired'', that is not quite precise
language. The actual state $\left|R\bar{R}\right>$ is \emph{not} a
pure colour singlet.  Although it does
have a non-zero \emph{projection} onto the singlet wave function
above, it also has non-zero projections onto the two members of
the octet that correspond to the diagonal Gell-Mann
matrices. Intuitively, one can also easily realise this by noting that
an $\mrm{SU(3)}$ rotation of $\left|R\bar{R}\right>$ would in general turn it into a
different state, say $\left|B\bar{B}\right>$, whereas a true colour singlet
would be invariant. 
Finally, we can also realise from \eqRef{eq:33bar} that a random
(colour-uncorrelated) quark-antiquark pair has a $1/N^2=1/9$ 
chance to be in an overall colour-singlet state; otherwise it is in
an octet. 

Similarly, there are also nine possible quark-quark (or
antiquark-antiquark) combinations, six of which are symmetric
under interchange of the two quarks and three of which are antisymmetric:
\index{Sextet}%
\begin{equation}
6 ~=~ \left(\begin{array}{c}
\left|RR\right>\\
\left|GG\right>\\
\left|BB\right>\\
\frac{1}{\sqrt{2}}\left(\left|RG\right> + \left|GR\right>\right)\\
\frac{1}{\sqrt{2}}\left(\left|GB\right> + \left|BG\right>\right)\\
\frac{1}{\sqrt{2}}\left(\left|BR\right> + \left|RB\right>\right)
\end{array}\right)
~~~~~~~~~
\bar{3} = \left(\begin{array}{c}
\frac{1}{\sqrt{2}}\left(\left|RG\right> - \left|GR\right>\right)\\
\frac{1}{\sqrt{2}}\left(\left|GB\right> - \left|BG\right>\right)\\
\frac{1}{\sqrt{2}}\left(\left|BR\right> - \left|RB\right>\right)
\end{array}\right)~.
\end{equation}
The members of the sextet transform into (linear combinations of) 
each other under $\mrm{SU(3)}$ transformations, and similarly for the
members of the antitriplet, hence neither of these can be reduced
further. The breakdown into
irreducible $\mrm{SU(3)}$ multiplets is therefore
\begin{equation}
3 \otimes 3 = 6 \oplus \overline{3}~.
\end{equation}
Thus, an uncorrelated pair of quarks has a $1/3$ chance to add to an overall
anti-triplet state (corresponding to coherent
superpositions like ``red + green = antiblue''\footnote{In the context of
  hadronisation models, 
  this coherent superposition of two quarks in an overall antitriplet
  state is sometimes called a
  \index{Diquarks}``diquark'' (at low $m_{qq}$)
  \index{String junctions}or a ``string junction'' (at high $m_{qq}$), see
  \secRef{sec:stringModel}; it corresponds to the antisymmatric ``red
  + green = antiblue'' combination needed to create a baryon
  wavefunction. }); otherwise it is in an overall 
sextet state. 

Note that the emphasis on
the quark-(anti)quark pair being \emph{uncorrelated} is important;
production processes that correlate the produced partons, like $Z\to q\bar{q}$ or $g\to q\bar{q}$, will
project out specific components (here the singlet and octet,
respectively). 
Note also that, if the quark
and (anti)quark are on opposite sides of the universe (i.e., living in
two different hadrons), the QCD \emph{dynamics} will not care what
overall colour state they 
are in, so for the formation of multi-partonic states in QCD, obviously the
spatial part of the wave functions (causality at the very least) 
will also play a role. Here, we are considering \emph{only} the colour part
of the wave functions. 
Some additional examples are 
\begin{eqnarray}
8\otimes 8 & = & 27 \oplus 10 \oplus \overline{10} \oplus 8 \oplus 8
\oplus 1 ~,\\ 
3 \otimes 8 & = & 15 \oplus 6 \oplus 3~,\\
3 \otimes 6 & = & 10 \oplus 8~,\\
3\otimes3\otimes3 & = & (6 \oplus \overline{3}) \otimes 3 = 10 \oplus 8
\oplus 8 \oplus 1 ~.
\end{eqnarray}
Physically, the 27 in the first line corresponds to a completely
incoherent addition of the colour charges of two gluons;
\index{Decuplet}the decuplets are slightly more coherent (with a lower
total colour charge), the octets
yet more, and the singlet corresponds to the combination of two gluons
that have precisely equal and opposite colour charges, so that their
total charge is zero. 
Further extensions and generalisations of these combination rules can
\index{Young tableaux}be obtained, e.g., using the method of Young
tableaux~\cite{young1901,youngSagan}.  


% !TEX root = runners-in-action.tex

\section{Algebraic effects, handlers, and runners}
\label{sect:runnersbyexample}

We begin with a short overview of the theory of algebraic effects and handlers, as well as
runners. To keep focus on how runners give rise to a programming
concept, we work naively in set theory. Nevertheless, we use
category-theoretic language as appropriate, to make it clear that there are no essential
obstacles to extending our work to other settings (we return to this point in \cref{sec:semantics-types}).

\subsection{Algebraic effects and handlers}
\label{sect:algebraiceffects}

There is by now no lack of material on the algebraic approach to structuring computational effects.
For an introductory treatment we refer to~\cite{Bauer:WhatIsAlgebraic}, while of course
also recommend the seminal papers by Plotkin and
Power~\cite{Plotkin:SemanticsForAlgOperations,Plotkin:NotionsOfComputation}. The brief
summary given here only recalls the essentials and introduces notation.

An \emph{(algebraic) signature} is given by a set $\sig$ of \emph{operation symbols},
and for each $\op \in \sig$ its \emph{operation signature}
%
$
  \op : A_\op \opto B_\op
$,
%
where  $A_\op$ and $B_\op$ are called the \emph{parameter} and \emph{arity} set.
A \emph{$\sig$-structure} $\M$ is given by a carrier set $\Mcarrier$, and for each
operation symbol $\op \in \sig$, a map $\op_{\M} : A_\op \times (B_\op \expto \Mcarrier) \to \Mcarrier$,
where~$\expto$ is set exponentiation. The \emph{free $\sig$-structure~$\Tree{\sig}{X}$}
over a set~$X$ is the set of well-founded trees generated inductively by
%
\begin{itemize}
\item $\retTree{x} \in \Tree{\sig}{X}$, for every $x \in X$, and
\item $\op(a, \kappa) \in \Tree{\sig}{X}$, for every $\op \in \sig$, $a \in A_\op$, and $\kappa : B_\op \to \Tree{\sig}{X}$.
\end{itemize}
%
We are abusing notation in a slight but standard way, by using $\op$ both as the
name of an operation and a tree-forming constructor.
%
The elements of $\Tree{\sig}{X}$ are called \emph{computation trees}: a leaf
$\retTree{x}$ represents a pure computation returning a value $x$, while
$\op(a, \kappa)$ represents an effectful computation that calls $\op$ with
parameter~$a$ and continuation~$\kappa$, which expects a result from~$B_\op$.

An \emph{algebraic theory $\Th = (\Thsig, \Theq)$} is given by a \emph{signature~$\Thsig$} and
a set of \emph{equations~$\Theq$}.
%
The equations $\Theq$ express computational behaviour via interactions between
operations, and
are written in a suitable formalism, e.g.,~\cite{Plotkin:HandlingEffects}. We
explain these by way of examples, as the precise details do not matter for our purposes.
Let $\Zero = \{\,\}$ be the empty set and $\One = \{\star\}$ the standard singleton.

\begin{example}
  \label{ex:state}
  Given a set $C$ of possible states, the theory of \emph{$C$-valued state} has two operations, whose
  somewhat unusual naming will become clear later on,
  %
  \begin{equation*}
    \siggetenv : \One \opto C,
    \qquad\qquad
    \sigsetenv : C \opto \One
  \end{equation*}
  %
  and the equations (where we elide appearances of $\star$):
  %
  \begin{gather*}
    \siggetenv (\lam{c} \sigsetenv(c, \kappa)) = \kappa,
    \qquad
    \sigsetenv (c, \siggetenv\, \kappa) = \sigsetenv(c, \kappa\, c), \\
    \sigsetenv (c, \sigsetenv(c', \kappa)) = \sigsetenv(c', \kappa).
  \end{gather*}
  %
  For example,
  the second equation states that reading state right after setting it to~$c$ gives precisely~$c$.
  The third equation states that $\sigsetenv$ overwrites the state.
\end{example}

\begin{example}
  \label{ex:exceptions}
  Given a set of exceptions $E$, the algebraic theory of \emph{$E$-many exceptions} is given by
  a single operation $\sigraise : E \opto \Zero$, and no equations.
\end{example}

A \emph{$\Th$-model}, also called a \emph{$\Th$-algebra}, is a $\Thsig$-structure which
satisfies the equations in $\Theq$. The \emph{free $\Th$-model} over a set~$X$ is constructed
as the quotient
%
\begin{equation*}
  \FreeAlg{\Th}{X} = \Tree{\Thsig}{X}/{\sim}
\end{equation*}
%
by the $\Thsig$-congruence $\sim$ generated by $\Theq$. Each $\op \in \Thsig$
is interpreted in the free model as the map
$(a, \kappa) \mapsto [\op(a, \kappa)]$, where $[{-}]$ is the $\sim$-equivalence class.

$\FreeAlg{\Th}{-}$ is the functor part of a \emph{monad} on sets, whose \emph{unit} at a
set~$X$ is
%
\begin{equation*}
  \xymatrix@C=3em@R=2em@M=0.5em{
    {X} \ar[r]^(0.35){\retTree{}}
    &
    {\Tree{\Thsig}{X}} \ar@{->>}[r]^{[{-}]}
    &
    {\FreeAlg{\Th}{X}.}
  }
\end{equation*}
%
The \emph{Kleisli extension} for this monad is then the operation which lifts any map \linebreak
$f : X \to \Tree{\Thsig}{Y}$ to the map $\lift{f} : \FreeAlg{\Thsig}{X} \to \FreeAlg{\Thsig}{Y}$,
given by
%
\begin{equation*}
  \lift{f}\,[\retTree{x}] \defeq f \, x,
  \qquad\qquad
  \lift{f}\,[\op(a, \kappa)] \defeq [\op(a, \lift{f} \circ \kappa)].
\end{equation*}
%
That is, $\lift{f}$ traverses a computation tree and replaces each leaf $\retTree{x}$
with $f\,x$.

The preceding construction of free models and the monad may be retro-fitted to an
algebraic signature~$\sig$, if we construe~$\sig$ as an algebraic theory with no
equations. In this case~$\sim$ is just equality, and so we may omit the quotient and the
pesky equivalence classes. Thus the carrier of the free $\sig$-model is the set of
well-founded trees $\Tree{\sig}{X}$, with the evident monad structure.

A fundamental insight of Plotkin and
Power~\cite{Plotkin:SemanticsForAlgOperations,Plotkin:NotionsOfComputation} was that many
computational effects may be adequately described by algebraic theories, with
the elements of free models corresponding to effectful computations. For example, the monads
induced by the theories from \cref{ex:state,ex:exceptions} are respectively isomorphic to the usual \emph{state monad}
$\St{C}\,X \defeq (C \Rightarrow X \times C)$ and the \emph{exceptions monad}
$\Exc{E}\,X \defeq X + E$.

Plotkin and Pretnar~\cite{Plotkin:HandlingEffects} further
observed that the universal property of free models may be used to model a
programming concept known as \emph{handlers}. Given a $\Th$-model $\M$ and a map
$f : X \to \Mcarrier$, the universal property of the free $\Th$-model gives us a
unique $\Th$-homomorphism $\freelift{f} : \FreeAlg{\Th}{X} \to \Mcarrier$ satisfying
%
\begin{equation*}
  \freelift{f} \, [\retTree{x}] = f\,x,
  \qquad\qquad
  \freelift{f} \, [\op(a, \kappa)] = \op_\M(a, \freelift{f} \circ \kappa).
\end{equation*}

A handler for a theory $\Th$ in a language such as \pl{Eff} amounts to a model 
$\M$ whose carrier $\Mcarrier$
is the carrier $\FreeAlg{\Th'}{Y}$ of the free model
for some other theory~$\Th'$, while the associated handling
construct is the induced $\Th$-homomorphism $\FreeAlg{\Th}{X} \to \FreeAlg{\Th'}{Y}$.
Thus handling transforms computations with effects~$\Th$ to computations with
effects~$\Th'$. There is however no restriction on how a handler implements an
operation, in particular, it may use its continuation in an arbitrary
fashion.
%
We shall put the universal property of free models to good use as well, while
making sure that the continuations are always used affinely.

\subsection{Runners}
\label{sect:purerunners}

Much like monads, handlers are useful for simulating computational effects, because they
allow us to transform $\Th$-computations to $\Th'$-computations. However, eventually there
has to be a ``top level'' where such transformations cease and actual computational
effects happen. For these we need another concept, known as
\emph{runners}~\cite{Uustalu:Runners}.
%
Runners are equivalent to the concept of
\emph{comodels}~\cite{Plotkin:TensorsOfModels,Power:Comodels}, which are ``just
models in the opposite category'', although one has to apply the motto
correctly by using powers and co-powers where seemingly exponentials and products would
do. 
Without getting into the intricacies, let us spell out the definition.

\begin{definition}
  A \emph{runner} $\R$ for a signature $\sig$ is given by a carrier set~$\Rcarrier$ together with, for
  each $\op \in \sig$, a
  \emph{co-operation~$\coop_{\R} : A_\op \to (\Rcarrier \expto B_\op \times \Rcarrier)$.}
\end{definition}

%
Runners are usually defined to have co-operations in the equivalent uncurried form
$\coop_\R : A_\op \times \Rcarrier \to B_\op \times \Rcarrier$, but that is less convenient for our purposes.

Runners may be defined more generally for theories $\Th$, rather than just signatures,
by requiring that the co-operations satisfy $\Theq$. We shall have no use for these,
although we expect no obstacles in incorporating them into our work.

A runner tells us what to do when an effectful computation reaches the top-level runtime
environment. Think of~$\Rcarrier$ as the set of configurations of the runtime environment. Given
the current configuration $c \in \Rcarrier$, the operation $\op(a, \kappa)$ is executed as the
corresponding co-operation $\coop_\R\,a\,c$ whose result $(b, c') \in B_\op \times \Rcarrier$ gives
the result of the operation $b$ and 
the next runtime configuration $c'$. The continuation $\kappa\,b$
then proceeds in runtime configuration~$c'$.

It is not too difficult to turn this idea into a mathematical model. For any
$X$, the co-operations induce a $\sig$-structure $\M$ with
$\Mcarrier \defeq \St{\Rcarrier} X = (\Rcarrier \expto X \times \Rcarrier)$ 
and operations $\op_\M : A_\op \times (B_\op \expto \St{\Rcarrier} X) \to \St{\Rcarrier} X$
given by
%
\begin{equation*}
  \op_\M (a, \kappa) \defeq \lam{c} \kappa\, (\pi_1 (\coop_\R\,a\,c))\, (\pi_2 (\coop_\R\,a\,c)).
\end{equation*}
%
We may then use the universal property of the free $\sig$-model to obtain a $\sig$-homomorphism
$\runh_X : \Tree{\sig}{X} \to \St{\Rcarrier} X$ satisfying the equations
%
\[
  \runh_X(\retTree{x}) = \lam{c} (x, c),
  \qquad\qquad
  \runh_X(\op(a, \kappa)) = \op_\M(a, \runh_X \circ \kappa).
\]
%
The map $\runh_X$ precisely captures the idea that a runner
\emph{runs computations} by transforming (static) computation trees into
state-passing maps. Note how in the above definition of $\op_\M$, the
continuation~$\kappa$ is used in a controlled way, as it appears precisely once
as the head of the outermost application. In terms of programming, this
corresponds to linear use in a tail-call position.

Runners are less ad-hoc than they may seem. First, notice that $\op_\M$ is just the
composition of the co-operation $\coop_\R$ with
the state monad's Kleisli extension of the continuation $\kappa$, and so is
the standard way of turning \emph{generic effects} into $\sig$-structures~\cite{Plotkin:AlgOperations}.
%
Second, the map $\runh_X$ is the component at $X$ of a monad morphism
$\runh : \Tree{\sig}{{-}} \to \St{\Rcarrier}$. Møgelberg \& Staton~\cite{Mogelberg:LinearUsageOfState}, as
well as Uustalu~\cite{Uustalu:Runners}, showed that the passage from a runner~$\R$ to the
corresponding monad morphism~$\runh$ forms a one-to-one correspondence between the former and the
latter.

As defined, runners are too restrictive a model of top-level computation, because the only
effect available to co-operations is state, but in practice the runtime
environment may also signal errors and perform other effects, by calling its own runtime
environment. We are led to the following generalisation.

\begin{definition}
  For a signature $\sig$ and monad $\T$, a \emph{$\T$-runner $\R$} for~$\sig$, 
  or just an \emph{effectful runner}, 
  is given by, for each $\op \in \sig$, a \emph{co-operation}
  $\coop_\R : A_\op \to \T B_\op$.
\end{definition}

The correspondence between runners and monad morphisms still holds.

\begin{proposition}
  \label{prop:monadmorphism}
  For a signature $\sig$ and a monad $\T$, the monad morphisms $\Tree{\sig}{{-}} \to \T$
  are in one-to-one correspondence with $\T$-runners for~$\sig$.
\end{proposition}

\begin{proof}
  This is an easy generalisation of the correspondence for 
  ordinary runners. Let us fix a signature $\sig$, and a monad $\T$ 
  with unit $\eta$ and Kleisli extension $\lift{{-}}$.

  Let $\R$ be a $\T$-runner for $\sig$. For any set $X$, $\R$ induces a $\sig$-structure
  $\M$ with $\Mcarrier \defeq \T X$ and 
  $\op_\M : A_\op \times (B_\op \expto \T X) \to \T X$ defined as
  %
  $
    \op_\M (a, \kappa) \defeq \lift{\kappa} (\coop_R\,a)
  $.
  %
  As before, the universal property of the free model $\Tree{\sig}{X}$ provides a unique
  $\sig$-homomorphism $\runh_X : \Tree{\sig}{X} \to \T X$, satisfying the equations
  %
  \begin{equation*}
    \runh_X (\retTree{x}) = \eta_X(x),
    \qquad\qquad
    \runh_X (\op(a, \kappa)) = \op_\M (a, \runh_X \circ \kappa).
  \end{equation*}
  %
  The maps $\runh_X$ collectively give us the desired monad morphism $\runh$ induced by $\R$.
  
  Conversely, given a monad morphism $\theta : \Tree{\sig}{{-}} \to \T$, we may recover a
  $\T$-runner~$\R$ for $\sig$ by defining the co-operations as
  $
    \coop_\R \, a \defeq \theta_{B_\op} (\op (a, \lam{b} \retTree{b}))
  $.
  It is not hard to check that we have described a one-to-one correspondence.
  \qed
\end{proof}



%%% Local Variables:
%%% mode: latex
%%% TeX-master: "runners-in-action"
%%% End:

% !TEX root = runners-in-action.tex

\section{Programming with runners}
\label{sec:programming-with-runners}

If ordinary runners are not general enough, the effectful ones are too general: 
parameterised by arbitrary monads $\T$, they do
not combine easily and they lack a clear notion of resource management. Thus,
we now engineer more specific monads whose associated runners can be turned into a
programming concept.
%
While we give up complete generality, the monads presented below are still quite
versatile, as they are parameterised by arbitrary algebraic signatures $\Sigma$,
and so are extensible and support various combinations of effects.

\subsection{The user and kernel monads}
\label{sec:user-kernel-monads}

Effectful source code running inside a runtime environment is just one example of a more
general phenomenon in which effectful computations are enveloped by a layer that provides
a supervised access to external resources: a user process is controlled by a kernel, a web
page by a browser, an operating system by hardware, or a virtual machine, etc. We shall
adopt the parlance of software systems, and refer to the two layers generically as the
\emph{user} and \emph{kernel} code.
%
Since the two kinds of code need not, and will not, use the same effects, each
will be described by its own algebraic theory and compute in its own monad.

We first address the kernel theory. 
Specifically, we look for an algebraic theory such that effectful runners for the induced monad
satisfy the following desiderata:
%
\begin{enumerate}
\item Runners support management and controlled finalisation of resources.
\item Runners may use further external resources.
\item Runners may signal failure caused by unavoidable circumstances.
\end{enumerate}

The totality of external resources
available to user code appears as a stateful external environment, even though it
has no direct access to it. Thus, kernel computations should carry state. We
achieve this by incorporating into the kernel theory the operations $\siggetenv$
and $\sigsetenv$, and equations for state from \cref{ex:state}.

Apart from managing state, kernel code should have access to further
effects, which may be true external effects, or some outer
layer of runners. In either case, we should allow the kernel code to call
operations from a given signature~$\sig$.

Because kernel computations ought to be able to signal failure, we should
include an exception mechanism. In practice, many programming languages and
systems have two flavours of exceptions, variously called recoverable and fatal,
checked and unchecked, exceptions and errors, etc. One kind, which we call just
\emph{exceptions}, is raised by kernel code when a situation requires special
attention by user code. The other kind, which we call \emph{signals},
indicates an unrecoverable condition that prevents normal execution of user
code. These correspond precisely to the two standard ways of combining
exceptions with state, namely the coproduct and the tensor of algebraic
theories~\cite{Hyland:CombiningEffects}. The coproduct simply adjoins exceptions
$\sigraise : E \leadsto \Zero$ from \cref{ex:exceptions} to the theory of
state, while the tensor extends the theory of state with signals
$\sigkill : S \leadsto \Zero$, together with equations
%
\begin{equation}
  \label{eq:kill-state}%
  \siggetenv(\lam{c} \sigkill\,s) = \sigkill\,s,
  \qquad\qquad
  \sigsetenv(c, \sigkill\,s) = \sigkill\,s.
\end{equation}
%
These equations say that a signal discards state, which makes it unrecoverable.

To summarise, the \emph{kernel theory} $\ThKK{C}{\sig}{E}{S}$ contains 
operations from a signature $\sig$, as well as state operations
$\siggetenv : \One \opto C$, $\sigsetenv : C \opto \One$, exceptions
$\sigraise : E \opto \Zero$, and signals $\sigkill : S \opto \Zero$, with equations for state
from \cref{ex:state}, equations~\eqref{eq:kill-state} relating state and
signals, and for each operation $\op \in \sig$, equations
%
\begin{align*}
  \siggetenv(\lam{c} \op(a, \kappa\,c)) &= \op(a, \lam{b} \siggetenv (\lam{c} \kappa\,c\,b)),\\
  \sigsetenv(c, \op(a, \kappa)) &= \op(a, \lam{b} \sigsetenv(c, \kappa\,b)),
\end{align*}
%
expressing that external operations do not interact with kernel state. 
It is not difficult to see that $\ThKK{C}{\sig}{E}{S}$ induces, up to
isomorphism, the \emph{kernel monad}
%
\begin{equation*}
  \KK{C}{\sig}{E}{S} X \quad\defeq\quad C \expto \Tree{\sig}{((X + E) \times C) + S}.
\end{equation*}

How about user code? It can of course call operations from a 
signature~$\sig$ (not necessarily the same as the kernel code), and because we
intend it to handle exceptions, it might as well have the ability to raise them.
However, user code knows nothing about signals and kernel state. Thus, we choose the \emph{user theory
  $\ThUU{\sig}{E}$} to be the algebraic theory with operations $\sig$, exceptions
$\sigraise : E \opto \Zero$, and no equations. This theory induces the \emph{user
  monad} $\UU{\sig}{E} X \defeq \Tree{\sig}{X + E}$.

\subsection{Runners as a programming construct}
\label{sec:runn-as-progr}

In this section, we turn the ideas presented so far into programming constructs.
We strive for a realistic result,
but when faced with several design options, we prefer simplicity and semantic
clarity. We focus here on translating the central concepts, and postpone
various details to \cref{sect:corecalculus}, where we present a full calculus.

We codify the idea of user and kernel computations by having syntactic
categories for each of them, as well as one for values. We use letters $M$, 
$N$ to indicate user computations, $K$, $L$ for kernel computations, 
and $V$, $W$ for values.

User and kernel code raise exceptions with operation $\tmkw{raise}$, and catch
them with exception handlers based on Benton and Kennedy's \emph{exceptional
  syntax}~\cite{Benton:ExceptionalSyntax},
%
\begin{equation*}
  \tmtry{M}{\{
    \tmreturn{x} \mapsto N,
    \ldots, \tmraise{e} \mapsto N_e, \ldots
  \}}, 
\end{equation*}
%
and analogously for kernel code. The familiar binding construct
%
$\tmlet{x}{M}{N}$
%
is simply shorthand for
%
$\tmtry{M}{\{\tmreturn{x} \mapsto N, \ldots, \tmraise{e} \mapsto \tmraise{e}, \ldots\}}$.

As a programming concept, a runner $R$ takes the form
%
\begin{equation*}
  \tmrunner{(\tm{op}\,x \mapsto K_{\tm{op}})_{\tm{op} \in \sig}}{C}, 
\end{equation*}
%
where each $K_\op$ is a kernel computation, with the variable $x$ bound in $K_{\tm{op}}$, so that
each clause $\tm{op} \, x \mapsto K_{\tm{op}}$ determines a co-operation for the
kernel monad. The subscript $C$ indicates the type of the state used by  
the kernel code $K_\op$.

The corresponding elimination form is a handling-like construct
%
\begin{equation}
  \label{eq:using}
  \tmrun{R}{V}{M}{F}, 
\end{equation}
%
which uses the co-operations of runner $R$ ``at'' initial kernel state~$V$ to
run user code~$M$, and finalises its return value, exceptions, and signals
with~$F$, see~\eqref{eq:finally-clause} below.
%
When user code $M$ calls an operation $\op$, the enveloping $\tmkw{run}$ construct runs the
corresponding co-operation $K_\op$ of $R$. While doing so, $K_\op$ might raise 
exceptions. But not every exception makes sense for every operation, and so
we assign to each operation $\op$ a set of exceptions $E_\op$ which the
co-operations implementing it may raise, by augmenting its operation signature with $E_\op$, as 
%
\begin{equation*}
  \op : \tysigop{A_\op}{B_\op}{E_\op}.
\end{equation*}
%
An exception raised by the co-operation $K_\op$ propagates back to the operation call in
the user code. Therefore, an operation call should have not only a continuation $x\,.\,M$
receiving a result, but also continuations $N_e$, one for each $e \in E_\op$,
%
\begin{equation*}
  \tmop{op}{}{V}{\tmcont x M}{\tmexccont N e {E_\op}}.
\end{equation*}
%
If $K_\op$ returns a value $b \in B_\op$, the execution proceeds 
as $M[b/x]$, and as $N_e$ if $K_\op$ raises an exception 
$e \in E_\op$. In examples, we use the generic versions of 
operations~\cite{Plotkin:AlgOperations}, written $\tmgeneff{op}{V}$,
which pass on return values and re-raise exceptions.

One can pass exceptions back to operation calls also in 
a language with handlers, such as \pl{Eff}, by 
changing the signatures of operations to
$A_\op \opto B_\op + E_\op$, and implementing 
the exception mechanism by hand, so that every
operation call is followed by a case distinction on $B_\op + E_\op$.
One is reminded of how operating system calls communicate 
errors back to user code as exceptional values.

A co-operation $K_\op$ may also send a signal, in which case the rest of the user code $M$
is skipped and the control proceeds directly to the corresponding case of the finalisation part~$F$ of the
$\tmkw{run}$ construct~\eqref{eq:using}, whose syntactic form is
%
\begin{equation}
  \label{eq:finally-clause}%
  \{ \tmreturn{x} \at c \mapsto N,
     \ldots, \tmraise{e} \at c \mapsto N_e,
     \ldots, \tmkill{s} \mapsto N_s,
     \ldots
  \}.
\end{equation}
%
Specifically, if $M$ returns a value $v$, then $N$ is evaluated with $x$ bound
to $v$ and $c$ to the final kernel state; if~$M$ raises an
exception~$e$ (either directly or indirectly via a co-operation of $R$), 
then $N_e$ is executed, again with $c$ bound to the final kernel state; and 
if a co-operation of $R$ sends a signal $s$, then $N_s$ is executed.

\begin{example}
  \label{ex:file-IO}%
  In anticipation of setting up the complete calculus we show how one can work with files.
  %
  The language implementors can provide an operation $\tm{open}$ which opens a
  file for writing and returns its file handle, an operation $\tm{close}$ which closes a
  file handle, and a runner $\mathsf{fileIO}$ that implements
  writing.
  Let us further suppose that $\mathsf{fileIO}$ may raise
  an exception $\mathsf{QuotaExceeded}$ if a write exceeds the user disk quota,
  and send a signal $\mathsf{IOError}$ 
  if an unrecoverable external error occurs.
  %
  The following code illustrates how to guarantee proper closing of the file:
  %
\begin{lstlisting}
using fileIO @ (open "hello.txt") run
  write "Hello, world."
finally {
  return x @ fh -> close fh,
  raise QuotaExceeded @ fh -> close fh,
  kill IOError -> return () }
\end{lstlisting}
  %
  Notice that the user code does not have direct access to the file handle.
  Instead, the runner holds it in its state, where it is available to the
  co-operation that implements $\tm{write}$. The finalisation block gets access to
  the file handle upon successful completion and raised exception, so it can close
  the file, but when a signal happens the finalisation cannot close the file,
  nor should it attempt to do so.

  We also mention that the code ``cheats'' by placing the call to $\tm{open}$ in a
  position where a value is expected. We should have $\tmkw{let}$-bound the file handle
  returned by $\tm{open}$ outside the $\tmkw{run}$ construct, which would make it clear that
  opening the file happens \emph{before} this construct (and that $\tm{open}$ is
  \emph{not} handled by the finalisation), but would also expose the file handle. Since
  there are clear advantages to keeping the file handle inaccessible, a realistic
  language should accept the above code and hoist computations from value positions
  automatically.
\end{example}

%%% Local Variables:
%%% mode: latex
%%% TeX-master: "runners-in-action"
%%% End:

% !TEX root = runners-in-action.tex


\newcommand{\tmcoop}[3]{\mathtt{\overline{#1}}~#2 \mapsto #3}

\section{A calculus for programming with runners}
\label{sect:corecalculus}

Inspired by the semantic notion of runners and the ideas of the previous
section, we now present a calculus for programming with co-operations and
runners, called $\lambdacoop$. It is a low-level fine-grain call-by-value
calculus~\cite{Levy:CBPV}, and as such could inspire an intermediate language
that a high-level language is compiled to.

\subsection{Types}
\label{sect:types}

\begin{figure}[tb]
  \parbox{\textwidth}{
  \centering
  \small
  \begin{align*}
  \text{Ground type $A$, $B$, $C$}
  \bnfis& \tybase      & &\text{base type} \\
  \bnfor& \tyunit       & &\text{unit type} \\
  \bnfor& \tyempty      & &\text{empty type} \\
  \bnfor& \typrod{A}{B} & &\text{product type} \\
  \bnfor& \tysum{A}{B}  & &\text{sum type}
  \\[1ex]
  \text{Constant signature:}
  \phantom{\bnfis}& \tmconst{f} : (A_1,\ldots,A_n) \to B
  \\[1ex]
  \text{Signature $\sig$}
  \bnfis& \{\op_1, \op_2, \ldots, \op_n\} \subset \Ops
  \\
  \text{Exception set $E$}
  \bnfis& \{e_1, e_2, \ldots, e_n\} \subset \Excs
  \\
  \text{Signal set $S$}
  \bnfis& \{s_1, s_2, \ldots, s_n\} \subset \Sigs
  \\[1ex]
  \text{Operation signature:}
  \phantom{\bnfis}& \op : \tysigop{A_\op}{B_\op}{E_\op}
  \\[1ex]
  \text{Value type $X$, $Y$, $Z$}
  \bnfis& A                                       & &\text{ground type} \\
  \bnfor& \typrod{X}{Y}                           & &\text{product type} \\
  \bnfor& \tysum{X}{Y}                            & &\text{sum type} \\
  \bnfor& \tyfun{X}{\tyuser{Y}{\Ueff}}           & &\text{user function type} \\
  \bnfor& \tyfunK{X}{\tykernel{Y}{\Keff}}         & &\text{kernel function type} \\
  \bnfor& \tyrunner{\sig}{\sig'}{S}{C}      & &\text{runner type}
  \\[1ex]
  \text{User (computation) type:}
  \phantom{\bnfor} &\tyuser{X}{\Ueff} \quad \text{where $\Ueff = (\sig, E)$}
  \\
  \text{Kernel (computation) type:}
  \phantom{\bnfor}& \tykernel{X}{\Keff} \quad \text{where $\Keff = (\sig, E, S, C)$}
  \end{align*}
  } 
  \caption{The types of $\lambdacoop$.}
  \label{fig:lambdacoop-types}
\end{figure}

The types of $\lambdacoop$ are shown in \cref{fig:lambdacoop-types}.
%
The \emph{ground types} contain \emph{base types}, and are closed under finite sums and
products. These are used in operation signatures and as types of kernel state. (Allowing
arbitrary types in either of these entails substantial complications that can be dealt
with but are tangential to our goals.) Ground types can also come with corresponding 
constant symbols~$\tmconst{f}$, each associated with a fixed \emph{constant signature}
$\tmconst{f} : (A_1,\ldots,A_n) \to B$.

We assume a supply of operation symbols $\Ops$, exception names $\Excs$, and
signal names $\Sigs$. Each operation symbol~$\op \in \Ops$ is equipped with an
\emph{operation signature} $\tysigop{A_\op}{B_\op}{E_\op}$, which specifies its
parameter type~$A_\op$ and arity type~$B_\op$, and the exceptions~$E_\op$ 
that the corresponding co-operations may raise in runners.

The \emph{value types} extend ground types with two 
function types, and a type of runners.
%
The \emph{user function type $\tyfun{X}{\tyuser{Y}{(\sig, E)}}$} classifies
functions taking arguments of type~$X$ to computations classified by the \emph{user
  (computation) type}~${\tyuser{Y}{(\sig, E)}}$, i.e., those that return values of
type~$Y$, and may call operations~$\sig$ and raise exceptions~$E$.
%
Similarly, the \emph{kernel function type
  $\tyfunK{X}{\tykernel{Y}{(\sig, E, S, C)}}$} classifies functions taking
arguments of type~$X$ to computations classified by the \emph{kernel
  (computation) type~$\tykernel{Y}{(\sig, E, S, C)}$}, i.e., those that return
values of type~$Y$, and may call operations~$\sig$, raise exceptions~$E$, send
signals~$S$, and use state of type~$C$. We note that the ingredients for user and kernel types
correspond precisely to the parameters of the user monad $\UU{\sig}{E}$ and the
kernel monad $\KK{C}{\sig}{E}{S}$ from \cref{sec:user-kernel-monads}.
%
Finally, the \emph{runner type $\tyrunner{\sig}{\sig'}{S}{C}$} classifies runners that
implement co-operations for the operations~$\sig$ as kernel computations which use 
operations~$\sig'$, send signals~$S$, and use state of type~$C$.


\subsection{Values and computations}
\label{sec:values-computations}


\begin{figure}[tp]
  \parbox{\textwidth}{
  \centering
  \small
  \abovedisplayskip=0pt
  \begin{align*}
  \intertext{\textbf{Values}}
  V, W
  \bnfis& x                                       & &\text{variable} \\
  \bnfor& \tmconst{f}(V_1,\ldots,V_n)                                       & &\text{ground constant} \\
  \bnfor& \tmunit                                 & &\text{unit} \\
  \bnfor& \tmpair{V}{W}                           & &\text{pair} \\
  \bnfor& \tminl[X,Y]{V} \bnfor \tminr[X,Y]{V}    & &\text{injection} \\
  \bnfor& \tmfun{x : X}{M}                        & &\text{user function} \\
  \bnfor& \tmfunK{x : X}{K}                       & &\text{kernel function} \\
  \bnfor& \tmrunner{(\tm{op}\,x \mapsto K_{\tm{op}})_{\tm{op} \in \sig}}{C}
                                                  & &\text{runner}
  \\[1ex]
  \intertext{\textbf{User computations}}
  M, N
  \bnfis& \tmreturn{V}                            & &\text{value} \\
  \bnfor& V\,W                                    & &\text{application} \\
  \bnfor& \tmtry{M}{
          \{ \tmreturn{x} \mapsto N,
             (\tmraise{e} \mapsto N_e)_{e \in E} \}
          }
                                                  & &\text{exception handler} \\
  \bnfor& \tmmatch{V}{\tmpair{x}{y} \mapsto M}    & &\text{product elimination} \\
  \bnfor& \tmmatch[X]{V}{}                        & &\text{empty elimination} \\
  \bnfor& \tmmatch{V}{\tminl{x} \mapsto M, \tminr{y} \mapsto N}
                                                  & &\text{sum elimination} \\
  \bnfor& \tmop{op}{X}{V}{\tmcont x M}{\tmexccont N e {E_\op}}
                                                  & &\text{operation call} \\
  \bnfor& \tmraise[X]{e}                          & &\text{raise exception} \\
  \bnfor& \tmrun{V}{W}{M}{F} 
                                                  & &\text{running user code} \\
  \bnfor& 
            \tmkernel{K}{W}{F}
                                                  & &\text{switch to kernel mode}
  \\[2ex]
  F \bnfis & \omit \rlap{$\{ \tmreturn{x} \at c \mapsto N, 
                    (\tmraise{e} \at c \mapsto N_e)_{e \in E},
                    (\tmkill{s} \mapsto N_s)_{s \in S} \}$}
  \\[1ex]
  \intertext{\textbf{Kernel computations}}
  K, L
  \bnfis& \tmreturn[C]{V}                         & &\text{value} \\
  \bnfor& V\,W                                    & &\text{application} \\
  \bnfor& \tmtry{K}{
          \{ \tmreturn{x} \mapsto L,
             (\tmraise{e} \mapsto L_e)_{e \in E} \}
          }
                                                  & &\text{exception handler} \\
  \bnfor& \tmmatch{V}{\tmpair{x}{y} \mapsto K}    & &\text{product elimination} \\
  \bnfor& \tmmatch[X \at C]{V}{}                  & &\text{empty elimination} \\
  \bnfor& \tmmatch{V}{\tminl{x} \mapsto K, \tminr{y} \mapsto L}
                                                  & &\text{sum elimination} \\
  \bnfor& \tmop{op}{X}{V}{\tmcont x K}{\tmexccont L e {E_\op}}
                                                  & &\text{operation call} \\
  \bnfor& \tmraise[X \at C]{e}                    & &\text{raise exception} \\
  \bnfor& \tmkill[X \at C]{s}                     & &\text{send signal} \\
  \bnfor& \tmgetenv[C]{\tmcont c K}               & &\text{get kernel state} \\
  \bnfor& \tmsetenv{V}{K}                         & &\text{set kernel state} \\
  \bnfor& \tmuser{M}{
          \begin{aligned}[t]
          \{ &\tmreturn{x} \mapsto K,
             (\tmraise{e} \mapsto L_e)_{e \in E} \}
          \end{aligned}
          }
                                                  & &\text{switch to user mode}
  \end{align*}
  } 
  \caption{Values, user computations, and kernel computations of $\lambdacoop$.}
  \label{fig:lambdacoop-terms}
\end{figure}

The syntax of terms is shown in \cref{fig:lambdacoop-terms}. The
usual fine-grain call-by-value stratification of terms into pure values and effectful
computations is present, except that we further distinguish between \emph{user} and
\emph{kernel} computations.

\subsubsection{Values}
\label{sec:values}

Among the values are variables, constants for ground types, and constructors 
for sums and products. There are two kinds of functions, for abstracting over user and
kernel computations. A \emph{runner} is a value of the form
%
\begin{equation*}
  \tmrunner{(\tm{op}\,x \mapsto K_{\tm{op}})_{\tm{op} \in \sig}}{C}.
\end{equation*}
%
It implements co-operations for operations $\tm{\op}$ as kernel
computations~$K_\op$, with $x$ bound in~$K_\op$. The type annotation~$C$
specifies the type of the state that~$K_\op$ uses.
Note that $C$ ranges over ground types, a restriction that allows us to define a naive
set-theoretic semantics.
%
We sometimes omit these type annotations.

\subsubsection{User and kernel computations}

The user and kernel computations both have pure computations, function application,
exception raising and handling, standard elimination forms, and operation calls.
Note that the typing annotations on some of these differ according to their mode.
For instance, a user operation call is annotated
with the result type~$X$, whereas the annotation $X \at C$ on a kernel operation call
also specifies the kernel state type~$C$.

The binding construct $\tmlet[X ! E]{x}{M}{N}$ is not part of the syntax,
but is an abbreviation for
%
$
  \tmtry{M}{
  \{ \tmreturn{x} \mapsto N,
     (\tmraise{e} \mapsto \tmraise[X]{e})_{e \in E}
   \}}
$,
%
and there is an analogous one for kernel computations. We often drop the 
annotation $X ! E$.

Some computations are specific to one or the other mode. Only the kernel mode
may send a signal with $\tmkill{\!}$, and manipulate state with
$\tmkw{getenv}$ and $\tmkw{setenv}$, but only the user mode has the 
$\tmkw{run}$ construct from \cref{sec:runn-as-progr}.
%
Finally, each mode has the ability to ``context switch'' to the other one.
The kernel computation
%
\begin{equation*}
\tmuser{M}{
   \{
     \tmreturn{x} \mapsto K,
     (\tmraise{e} \mapsto L_e)_{e \in E}
   \}
}
\end{equation*}
%
runs a user computation $M$ and handles the returned value and leftover
exceptions with kernel computations $K$ and $L_e$.
Conversely, the user computation
%
\begin{equation*}
\tmkernel{K}{W}{
  \{x \at c \mapsto M,
    (\tmraise{e} \at c \mapsto N_e)_{e \in E},
    (\tmkill{s} \mapsto N_s)_{s \in S}
  \}
}
\end{equation*}
%
runs kernel computation $K$ with initial state $W$, and handles the returned value,
and leftover exceptions and signals with user computations $M$, $N_e$, and $N_s$.

\subsection{Type system}
\label{sec:typesystem}

We equip $\lambdacoop$ with a type system akin to type and effect systems for
algebraic effects and handlers~\cite{Bauer:EffectSystem,Benton:ExceptionalSyntax,Kammar:Handlers}.
We are experimenting with resource control, so it makes sense for the type system
to tightly control resources. Consequently, our effect system
does not allow effects to be implicitly propagated outwards.

In \cref{sect:types}, we assumed that each operation $\op \in \Ops$ 
is equipped with some fixed operation signature
%
$
  \op : \tysigop{A_\op}{B_\op}{E_\op}
$.
%
We also assumed a fixed constant signature $\tmconst{f} : (A_1, \ldots, A_n) \to B$
for each ground constant $\tmconst{f}$.
%
We consider this information to be part of the type system and say no more about it.

Values, user computations, and kernel computations each have a corresponding
\emph{typing judgement} form and a \emph{subtyping relation}, given by 
%
\begin{align*}
  &\Gamma \types V : X,
& &\Gamma \types M : \tyuser{X}{\Ueff},
& &\Gamma \types K : \tykernel{X}{\Keff},\\
  &X \sub Y,
& &\tyuser{X}{\Ueff} \sub \tyuser{Y}{\Veff},
& &\tykernel{X}{\Keff} \sub \tykernel{Y}{\Leff}, 
\end{align*}
%
where $\Gamma$ is a \emph{typing context} $x_1 : X_1, \ldots, x_n : X_n$.
%
The effect information is an over-approximation, i.e., $M$ and $K$ employ \emph{at
  most} the effects described by $\Ueff$ and $\Keff$.
%
The complete rules for these judgements are given in the online appendix. %\cref{sec:typing-rules}. 
We comment here
only on the rules that are peculiar to~$\lambdacoop$, see \cref{fig:typing-selected}.

\begin{figure}[tp]
  \centering
  \small
  \begin{mathpar}
    %%% NOTE: these should be kept in sync with the rules given in
    %%%       tpying-rules.tex. The rules in that file are the official version,
    %%%       the ones here are the copies. If you delete a rule here, make sure
    %%%       to note so in typing-rules.tex.
    %%%
    %%%       Do not place a rule here without commenting on it in the text,
    %%%       that's bad manners.
    \coopinfer{Sub-Ground}{ }{A \sub A}
    
    \coopinfer{Sub-Runner}{
      \sig_1' \subseteq \sig_1 \\
      \sig_2 \subseteq \sig_2' \\
      S \subseteq S' \\
      C \equiv C'
    }{
      \tyrunner{\sig_1}{\sig_2}{S}{C} \sub \tyrunner{\sig_1'}{\sig_2'}{S'}{C'}
    }

    \coopinfer{Sub-Kernel}{
      X \sub X' \\
      \sig \subseteq \sig' \\
      E \subseteq E' \\
      S \subseteq S' \\
      C \equiv C'
    }{
      \tykernel{X}{(\sig, E, S, C)} \sub \tykernel{X'}{(\sig', E', S', C')}
    }

  \coopinfer{TyUser-Try}{
    \Gamma \types M : \tyuser{X}{(\sig,E)}
    \\
    \Gamma, x \of X \types N : \tyuser{Y}{(\sig,E')}
    \\
    \big(
      \Gamma \types N_e : \tyuser{Y}{(\sig,E')}
    \big)_{e \in E}
  }{
    \Gamma \types
    \tmtry{M}{
        \{ \tmreturn{x} \mapsto N,
           (\tmraise{e} \mapsto N_e)_{e \in E} \}
        }
    : \tyuser{Y}{(\sig,E')}
  }

  \coopinfer{TyUser-Run}{
    F \equiv
    \{ \tmreturn{x} \at c \mapsto N,
       (\tmraise{e} \at c \mapsto N_e)_{e \in E},
       (\tmkill{s} \mapsto N_s)_{s \in S}
    \}
    \\\\
    \Gamma \types V : \tyrunner{\sig}{\sig'}{S}{C} \\
    \Gamma \types W : C \\\\
    \Gamma \types M : \tyuser{X}{(\sig, E)} \\
    \Gamma, x \of X, c \of C \types N : \tyuser{Y}{(\sig', E')} \\
    \big(
       \Gamma, c \of C \types N_e : \tyuser{Y}{(\sig', E')}
    \big)_{e \in E} \\
    \big(
       \Gamma \types N_s : \tyuser{Y}{(\sig', E')}
    \big)_{s \in S} \\
  }{
    \Gamma \types \tmrun{V}{W}{M}{F} : \tyuser{Y}{(\sig', E')}
  }

  \coopinfer{TyUser-Op}{
    \Ueff \equiv (\sig,E) \\
    \op \in \sig \\
    \Gamma \types V : A_\op \\\\
    \Gamma, x \of B_\op \types M : \tyuser{X}{\Ueff} \\
    \big(
      \Gamma \vdash N_e : \tyuser{X}{\Ueff}
    \big)_{e \in E_\op}
  }{
    \Gamma \types \tmop{op}{X}{V}{\tmcont x M}{\tmexccont N e {E_\op}} : \tyuser{X}{\Ueff}
  }

  \coopinfer{TyKernel-Op}{
    \Keff \equiv (\sig, E, S, C) \\
    \op \in \sig \\
    \Gamma \types V : A_\op \\\\
    \Gamma, x \of B_\op \types K : \tykernel{X}{\Keff} \\
    \big(
      \Gamma \vdash L_e : \tykernel{X}{\Keff}
    \big)_{e \in E_\op}
  }{
    \Gamma \types \tmop{op}{X}{V}{\tmcont x K}{\tmexccont L e {E_\op}} : \tykernel{X}{\Keff}
  }

  \coopinfer{TyUser-Kernel}{
    F \equiv
    \{ \tmreturn{x} \at c \mapsto N,
       (\tmraise{e} \at c \mapsto N_e)_{e \in E},
       (\tmkill{s} \mapsto N_s)_{s \in S}
    \}
    \\\\
    \Gamma \types K : \tykernel{X}{(\sig, E, S, C)} \\
    \Gamma \types W : C \\
    \Gamma, x \of X, c \of C \types N : \tyuser{Y}{(\sig, E')} \\
    \big(
      \Gamma, c \of C \types N_e : \tyuser{Y}{(\sig, E')}
    \big)_{e \in E} \\
    \big(
      \Gamma \types N_s : \tyuser{Y}{(\sig, E')}
    \big)_{s \in S} \\
  }{
    \Gamma \types \tmkernel{K}{W}{F} : \tyuser{Y}{(\sig, E')}
  }

  \coopinfer{TyKernel-User}{
   \Keff \equiv (\sig, E', S, C) \\
   \Gamma \types M : \tyuser{X}{(\sig, E)} \\\\
   \Gamma, x \of X \types K : \tykernel{Y}{\Keff} \\
   \big(
     \Gamma \types L_e : \tykernel{Y}{\Keff}
   \big)_{e \in E}
  }{
    \Gamma \types
    \tmuser{M}{
      \{ \tmreturn{x} \mapsto K,
         (\tmraise{e} \mapsto L_e)_{e \in E}
      \}
    }
    : \tykernel{Y}{\Keff}
  }
  \end{mathpar}
  \caption{Selected typing and subtyping rules.}
  \label{fig:typing-selected}
\end{figure}

Subtyping of ground types \textsc{Sub-Ground} is trivial, as it relates only equal types.
Subtyping of runners \textsc{Sub-Runner} and kernel computations
\textsc{Sub-Kernel} requires equality of the kernel state types~$C$ and~$C'$
because state is used invariantly in the kernel monad.
We leave it for future work to replace ${C \equiv C'}$ with a
\emph{lens}~\cite{Foster:Lenses} from~$C'$ to~$C$, i.e., maps $C' \to C$ and ${C' \times C \to C'}$
satisfying state equations analogous to \cref{ex:state}. It has been
observed~\cite{OConnor:Lens,Power:Comodels} that such a lens in fact amounts to
an ordinary runner for $C$-valued state.

The rules \textsc{TyUser-Op} and \textsc{TyKernel-Op} govern operation calls, where 
we have a success continuation which receives a value returned by a 
co-operation, and exceptional continuations which receive exceptions raised by co-operations.

The rule \textsc{TyUser-Run} requires that the runner $V$ implements \emph{all} the
operations $M$ can use, meaning that operations are \emph{not} implicitly propagated outside a $\tmkw{run}$ block (which is different from how handlers are sometimes implemented). Of course, the co-operations of the runner may call further external operations, as recorded by the signature~$\sig'$. Similarly, we require the finally block~$F$ to intercept all exceptions and signals that might be produced by the co-operations of $V$ or the user code $M$.
%
Such strict control is exercised throughout. For example, in 
\textsc{TyUser-Run}, \textsc{TyUser-Kernel}, and \textsc{TyKernel-User} we catch all the exceptions and signals that the code might produce.
%
One should judiciously relax these requirements in a language that is presented to
the programmer, and allow re-raising and re-sending clauses to be automatically inserted.

%%% Local Variables:
%%% mode: latex
%%% TeX-master: "runners-in-action"
%%% End:

% !TEX root = runners-in-action.tex

\subsection{Equational theory}
\label{sect:eqtheory}

We present $\lambdacoop$ as an \emph{equational calculus}, i.e., the interactions between
its components are described by equations. Such a presentation makes it easy to reason
about program equivalence.
%
There are three equality judgements
%
\begin{equation*}
\Gamma \types V \equiv W : X, 
\qquad
\Gamma \types M \equiv N : \tyuser{X}{\Ueff}, 
\qquad
\Gamma \types K \equiv L  : \tyuser{X}{\Keff}.
\end{equation*}
%
It is presupposed that we only compare well-typed expressions with the indicated types.
For the most part, the context and the type annotation on judgements will play no significant role,
and so we shall drop them whenever possible.

We comment on the computational equations for constructs characteristic
of~$\lambdacoop$, and refer the reader to the online appendix for other equations.
%
When read left-to-right, these equations explain the operational meaning of programs.

Of the three equations for $\tmkw{run}$, the first two specify that returned values and
raised exceptions are handled by the corresponding clauses,
%
\begin{align*}
  \tmrun{V}{W}{(\tmreturn{V'})}{F} &\equiv N[V'/x, W/c], 
  \\
  \tmrun{V}{W}{(\tmraise[X]{e})}{F} &\equiv N_{e}[W/c],
\end{align*}
%
where
%
$F \defeq \{\tmreturn{x} \at c \mapsto N,
   (\tmraise{e} \at c \mapsto N_e)_{e \in E},
   (\tmkill{s} \mapsto N_s)_{s \in S}
\}$.
%
The third equation below relates running an operation $\op$ with executing the corresponding co-operation~$K_\op$, 
where $R$ stands for the runner
%
$\tmrunner{(\tm{op}\,x \mapsto K_{\tm{op}})_{\tm{op} \in \sig}}{C}$:
%
\begin{multline*}
  \tmrun{R}{W}{(
    \tmop{op}{X}{V}{\tmcont x M}{\tmexccont {N'} {e'} {E_\op}}
    )}{F} \equiv {}
  \\
  \begin{aligned}[t]
     &\tmkernel{K_\op[V/x]}{W}{} \\
     &\qquad\big\{
         \begin{aligned}[t]
           &\tmreturn{x} \at c' \mapsto (\tmrun{R}{c'}{M}{F}), \\
           &\left(
               \tmraise{e'} \at c' \mapsto (\tmrun{R}{c'}{N'_{e'}}{F})
             \right)_{e' \in E_\op},\\
           &\left(
               \tmkill{s} \mapsto N_s
             \right)_{s \in S} \big\}
         \end{aligned}
 \end{aligned}
\end{multline*}
%
Because $K_\op$ is kernel code, it is executed in kernel mode, whose
$\tmkw{finally}$ clauses specify what happens afterwards: if $K_\op$ returns a value, or
raises an exception, execution continues with a suitable continuation, with~$R$
wrapped around it; and if $K_\op$ sends a signal, the corresponding finalisation code from $F$ is
evaluated.

The next bundle describes how kernel code is executed within user code:
%
\begin{align*}
  \tmkernel{(\tmreturn[C]{V})}{W}{F} &\equiv N[V/x, W/c], \\
  \tmkernel{(\tmraise[X \at C]{e})}{W}{F} &\equiv N_{e}[W/c], \\
  \tmkernel{(\tmkill[X \at C]{s})}{W}{F} &\equiv N_{s}, \\
  \tmkernel{(\tmgetenv[C]{\tmcont c K})}{W}{F} &\equiv \tmkernel{K[W/c]}{W}{F}, \\
  \tmkernel{(\tmsetenv{V}{K})}{W}{F} &\equiv \tmkernel{K}{V}{F}.
\end{align*}
%
We also have an equation stating that an operation called in kernel mode propagates out to
user mode, with its continuations wrapped in kernel mode:
%
\begin{multline*}
  \tmkernel{\tmop{op}{X}{V}{\tmcont x K}{\tmexccont L {e'} E}}{W}{F}
  \equiv {} \\
  \tmop{op}{X}{V}{\tmcont x {\tmkernel{K}{W}{F}}}{
    \left(
      \tmkernel{L_{e'}}{W}{F}
    \right)_{e' \in E}}.
\end{multline*}
%
Similar equations govern execution of user computations in kernel mode.

The remaining equations include standard $\beta\eta$-equations for
exception handling~\cite{Benton:ExceptionalSyntax}, deconstruction of products and sums,
algebraicity equations for operations~\cite{Pretnar:Thesis}, and the equations of kernel theory from \cref{sec:user-kernel-monads}, describing how $\tmkw{getenv}$ and $\tmkw{setenv}$ work, and how they interact with signals and other operations.

%%% Local Variables:
%%% mode: latex
%%% TeX-master: "runners-in-action"
%%% End:

\section{Denotational semantics}
\label{sec:denotation}

In this section we assign mathematical meaning to the constituent parts of Clerical.
%
The types are interpreted by  the expected sets:
%
\begin{align*}
\sem{\dZ} &\defeq \IZ &
\sem{\dB} &\defeq \{\semff, \semtt\} &
\sem{\dR} &\defeq \IR &
\sem{\dU} &\defeq \{\star\} \enspace .
\end{align*}
%
Typing contexts are interpreted by cartesian product:
%
\begin{align*}
  \sem{x_1 \of \tau_1, \ldots, x_m \of \tau_m} &\defeq
  \sem{\tau_1} \times \cdots \times \sem{\tau_m} \enspace .
%  &
%  \sem{\Gamma; \Delta} = \sem{\Gamma} \times \sem{\Delta}
\end{align*}
%
The denotation $\sem{\Gamma}$ of a read-only context $\Gamma$ is thought of as an \defemph{environment} specifying values of variables, whereas the denotation of a read-write context $\Gamma ; \Delta$ has two components, the environment $\sem{\Gamma}$ and the \defemph{state} $\sem{\Delta}$.

The meaning of a well-typed pure expression $\Gamma \rotypes e : \tau$ and
general expression $\Gamma; \Delta \rwtypes c : \tau$ 
 will be maps:
%
\begin{align*}
  \sem{\Gamma  \rotypes e : \tau}  & : \sem{\Gamma} \to \PP{\sem{\tau}}, \\
  \sem{\Gamma; \Delta \rwtypes c : \tau}  & : \sem{\Gamma} \to (\sem{\Delta} \to \PP{\sem{\Delta} \times \sem{\tau}}),
\end{align*}
%
where $\PP{S}$ is  a \defemph{powerdomain}, a collection of sets representing the possible sets of outcomes of nondeterministic computations that return values from $S$. Due to the distinction between error and non-termination, already mentioned in Section~\ref{sec:overview}, 
Clerical requires a rather specific form of powerdomain. In order to motivate it, we discuss the distinction between
error and non-termination in more detail.

Mathematically, we would like to  consider a well-behaved (deterministic) expression $e$ of type $\dR$ as defining a real number value $r$. Computationally, however, the best that $e$ will be able to do is to produce, on demand, an approximation of $r$ to within any specified precision. In the ideal case, given precision $\epsilon > 0$, the evaluation of $e$ will determine in finite time some rational approximation $q$ such that $|q - r| < \epsilon$. In Clerical, not every  deterministic expression of type $\dR$ achieves this ideal.

Some expressions simply give rise to non-terminating computations that never provide any approximating information. Others, may appear to provide approximating information, but do so in a way that is either incomplete or inconsistent.\footnote{Incompleteness may  arise, for example, if approximating values are only computed for $\epsilon$ that are not too small. Similarly, inconsistency can occur if two different $\epsilon_1, \epsilon_2$ result in putative approximations $q_1,q_2$ with $|q_1 - q_2| \geq \epsilon_1 + \epsilon_2$.} Our semantics of Clerical distinguishes between these two eventualities. Expressions that produce incomplete or inconsistent approximating information are considered \defemph{erroneous}, and the semantics for Clerical will ensure that no such expression is ever executed within a program with valid semantics. The motivation for this is to avoid any situation in which faulty approximations can provide misleading information. In contrast, \defemph{non-terminating} expressions are considered harmless in the sense that they cannot be a source of incorrect information. As is standard in denotational semantics, such expressions are assigned the special denotation $\bot$. 
These need to be distinguished from erroneous ones, because, as discussed in Section~\ref{sec:overview},  non-terminating expressions 
have an essential role to play  when programming in Clerical.


For any set~$S$,  define $\liftnoerr{S} \defeq S + \{\bot\}$, where~$\bot$ represents non-termination, as discussed above. 
Although $S + \{\bot\}$ is strictly speaking a coproduct (sum) of two sets, in practice we shall only use it
in instances in which $\bot \notin S$. This allows us to represent $\liftnoerr{S}$ as the (disjoint) union 
$S \cup \{\bot\}$.
Define:
\begin{equation*}
  \PP{S} \defeq
  \{ X \subseteq \liftnoerr{S} \such
       \text{$X$ infinite} \Rightarrow \bot \in X
  \},
\end{equation*}
%
A set $X \in  \PP{S}$ represents the results of a nondeterministic Clerical computation in the following way. Firstly, $X = \emptyset$ in the case that the computation is \emph{erroneous} (see the discussion above), in which case no result value is relevant. 
%
The case $\bot \in X$ applies if the nondeterministic computation has at least one non-terminating branch, in which case $X \setminus \{\bot\}$ is the set of all result values returned by terminating nondeterministic branches. If instead $\bot \notin X$ then $X$ represents the set of possible results of a necessarily terminating nondeterministic computation. Since Clerical has only  finite nondeterministic branching, such a set is necessarily finite.

We shall implicitly make use of  the fact that powerdomain $\Pstar$ carries the structure of a monad on the category of sets. For the purposes of this paper, we never need to directly refer to the abstract structure of the monad. However, the following maps associated with this structure will be useful. Firstly, for any $S$, there is a map $x \mapsto \pure{x} : S \to \PP{S}$, that maps any $x \in S$ to the singleton $\pure{x}$ representing the deterministic computation that returns $x$ as its result. 
Secondly, for any function $f : S \to  \PP{T}$ we define $\lift{f} : \PP{S} \to  \PP{T}$ by:
\[
\lift{f}(X) = 
\begin{cases} 
\emptyset & \text{if $\some{x \in X}\ f(x)= \emptyset$,}\\
\textstyle
\{\bot \such \bot \in X\} \cup \bigcup_{x \in X \setminus \{\bot\}} f(x) & \text{otherwise.}
\end{cases}
\]
It is easily checked that this indeed defines a set in $\PP{T}$. The idea behind the definition is that 
$\lift{f}(X)$ models a sequencing of nondeterministic computations: first execute the nondeterministic computation whose result is represented by $X$, then for each potential value $x \in X$
run the nondeterministic computation modelled by $f(x)$ to obtain potential return values.
This idea motivates the alternative notation
\[
\PPlet{x}{X}\, f(x)
\]
for  $\lift{f}(X)$, which we shall often use.

The reason behind the first clause in the definition of 
$\lift{f}(X)$ is that a computation is considered illegitimate if an error occurs along any possible nondeterministic branch. In such a case the entire computation is given the error denotation $\emptyset$.

\begin{figure}
  \begin{mdframed}
  \centering
  \small
\begin{align*}
  \sem{\Gamma \rotypes c : \tau} \, \gamma = {}&
     \PPlet
        {(v, ())}
        {\sem{\Gamma; \emptyctx \rwtypes c : \tau} \, \gamma \, ()}
        {\pure{v}}
  \\
  \sem{x_1 \of \tau_1; \ldots, x_n \of \tau_n \rotypes x_i : \tau_i} \, \gamma
  = {}& \pure{\gamma_i}
  \\
  \sem{\Gamma \rotypes \cfalse : \dB} \, \gamma = {}& \pure{\semff} \\
  \sem{\Gamma \rotypes \ctrue : \dB} \, \gamma = {}& \pure{\semtt} \\
  \sem{\Gamma \rotypes \numeral{k} : \dZ} \, \gamma = {}& \pure{k} \\
  \sem{\Gamma \rotypes \cskip : \dU} \, \gamma = {}& \pure{\semuu} \\
  \sem{\Gamma \rotypes \ccoerce{e} : \dR} \, \gamma = {}& \PPlet
        {z}
        {\sem{\Gamma \rotypes e : \dZ}\,\gamma}
        {\pure{\inclZ{z}}}
  \\
  \sem{\Gamma \rotypes e_1 \iop e_2 : \dZ} \, \gamma = {}&
    \PPlet
    {x}
    {\sem{\Gamma \rotypes e_1 : \dZ}\,\gamma} 
 \\
 & \PPlet
    {y}
    {\sem{\Gamma \rotypes e_2 : \dZ}\,\gamma}
    \pure{x \op y}
 \\
 \sem{\Gamma \rotypes e_1 \rop e_2 : \dR} \, \gamma = {}& 
   \PPlet
    {x}
    {\sem{\Gamma \rotypes e_1 : \dR}\,\gamma}
\\
& \PPlet
    {y}
    {\sem{\Gamma \rotypes e_2 : \dR}\,\gamma}
    \pure{x \op y}
 \\
 \sem{\Gamma \rotypes \cinv{e} : \dR} \, \gamma = {}&
 \PPlet
    {x}
    {\sem{\Gamma \rotypes e : \dR}\,\gamma}
    \begin{cases} 
    \pure{x^{-1}} & \text{if $x \neq 0$} \\
    \PPbot  & \text{if $x=0$}
    \end{cases}
 \\
  \sem{\Gamma \rotypes e_1 = e_2 : \dZ} \, \gamma = {}&
    \PPlet
    {x}
    {\sem{\Gamma \rotypes e_1 : \dZ}\,\gamma}
\\
& 
\PPlet
    {y}
    {\sem{\Gamma \rotypes e_2 : \dZ}\,\gamma}
    \begin{cases} 
    \pure{\semtt} & \text{if $x = y$} \\
    \pure{\semff} & \text{if $x \neq y$}
    \end{cases}
\\
  \sem{\Gamma \rotypes e_1 < e_2 : \dZ} \, \gamma = {}&
    \PPlet
    {x}
    {\sem{\Gamma \rotypes e_1 : \dZ}\,\gamma}
\\
& \PPlet
    {y}
    {\sem{\Gamma \rotypes e_2 : \dZ}\,\gamma}
    \begin{cases} 
    \pure{\semtt} & \text{if $x < y$} \\
    \pure{\semff} & \text{if $x \geq y$}
    \end{cases}
\\
   \sem{\Gamma \rotypes e_1 \rlt e_2 : \dR} \, \gamma = {}&
    \PPlet
    {x}
    {\sem{\Gamma \rotypes e_1 : \dR}\,\gamma} \\
    &\PPlet
    {y}
    {\sem{\Gamma \rotypes e_2 : \dR}\,\gamma}
    \begin{cases}
    \pure{\semtt} & \text{if $x < y$} \\
    \pure{\semff} & \text{if $x > y$} \\
    \PPbot &  \text{if $x = y$} 
    \end{cases}
  \\
  \sem{\Gamma \rotypes (\clim{x}{e}) : \dR}\,\gamma = {}&
  \begin{cases}
    \pure{t} &
      \begin{aligned}[t]
      &\text{if $t \in \IR$ and $\all{k \in \IZ}$} \\
      &\text{  $\sem{\Gamma, x \of \dZ \rotypes e : \dR} (\gamma, k) \subseteq \IR$ and} \\
      &\text{  $\all{u \in \sem{\Gamma, x \of \dZ \rotypes e : \dR} (\gamma, k)} |u - t| < 2^{-k}$,}
      \end{aligned}
    \\
    \PPerr   & \text{if no such $t \in \IR$ exists.}
  \end{cases}
\end{align*}
\end{mdframed}
\caption{Denotational semantics of pure expressions}
\label{figure:ro-denotations}
\end{figure}


\begin{figure}
  \begin{mdframed}
  \centering
  \small
\begin{align*}
\sem{\Gamma; \Delta \rwtypes e : \tau} \, \gamma \, \delta &=
  \begin{aligned}[t]
   &\PPlet
     {v}
     {\sem{\Gamma, \Delta \rotypes e : \tau} \, (\gamma, \delta)} \\
             &
        \quad
     \pure{(\delta, v)}
    \end{aligned}
\\
\sem{\Gamma; \Delta \rwtypes c_1 ; c_2 : \tau} \, \gamma \, \delta &=
   \begin{aligned}[t]
   &\PPlet
     {(\delta', \semuu)}
     {\sem{\Gamma; \Delta \rwtypes c_1 : \dU} \, \gamma \, \delta} 
   \\
   &  
     \quad\sem{\Gamma; \Delta \rwtypes c_2 : \tau} \, \gamma \, \delta'
       \end{aligned}
\\
\sem{\Gamma ; \Delta \rwtypes (\cvar{x}{e} c) : \tau} \, \gamma \, \delta &=
  \begin{aligned}[t]
   &\PPlet
     {v}
     {\sem{\Gamma, \Delta \rotypes e : \sigma} \, (\gamma, \delta)}
   \\
   &
    \PPletx
     {((\delta', v'), v'')}
     \\
     &
     \qquad\PPinx{\sem{\Gamma; \Delta, x \of \sigma \rwtypes c : \tau} \, \gamma \, (\delta,v)}
   \\
    &
     \quad\pure{(\delta', v'')}
  \end{aligned}
\\
\sem{\Gamma ; \Delta \rwtypes (\clet{x}{e}) : \dU} \, \gamma \, \delta &=
  \begin{aligned}[t]
  &\PPlet
    {v}
    {\sem{\Gamma, \Delta \rotypes e : \tau} \, (\gamma, \delta)}
        \\
        &
        \quad
    \pure{(\delta[x :=v]), \semuu)}
    \end{aligned}
\\
\sem{\Gamma ; \Delta \rwtypes (\cif e \cthen c_1 \celse c_2 \cend) : \tau} \, \gamma \, \delta &=
  \begin{aligned}[t]
  &\PPlet
    {b}
    {\sem{\Gamma, \Delta \rotypes e : \dU} \, (\gamma, \delta)}
    \\
    & \quad \begin{cases}
    \sem{\Gamma; \Delta \rwtypes c_1 : \tau} \, \gamma \, \delta & \text{if $b = \semtt$} \\
     \sem{\Gamma; \Delta \rwtypes c_2 : \tau} \, \gamma \, \delta & \text{if $b = \semff$}
     \end{cases}
  \end{aligned}
\end{align*}
%
\end{mdframed}
\caption{Denotational semantics of general expressions, excluding $\mathtt{case}$ and $\mathtt{while}$}
\label{figure:rw-denotations}
\end{figure}

\Cref{figure:ro-denotations} assigns denotational semantics to pure expressions.
One point that deserves explanation is the denotation of $\cinv{e}$, when~$e$ is a real expression representing~$0$. Since there is no appropriate real value to be given, the denotation could be chosen to be 
either $\PPerr$ or $\PPbot$. We choose the latter, as it reflects the fact that an algorithm for calculating reciprocal will run forever, given a representation of the real number~$0$, without ever returning any erroneous approximation to a result value.
Similarly, $\PPbot$ is given as the denotation of  $e_1 \rlt e_2$ when $e_1,e_2$ are two real expressions representing equal numbers, reflecting the fact that an algorithm trying to distinguish between the two numbers will run forever when given equal inputs.

The most complex definition in \cref{figure:ro-denotations} is the semantics of the  limit operation $\clim{x}{e}$.
%
In this definition, note that there is at most one $t \in \IR$ satisfying the first condition.
Its existence places strong requirements on the expression $e$, which must represent a sequence of sets of real numbers, such that every real number~$u$ in the $k$-th set lies within a distance of $2^{-k}$ of~$t$.
That is, every choice of a real number from every set furnishes a Cauchy sequence rapidly converging to a common limit~$t$.
The use of a sequence of sets allows the behaviour of~$e$ to be nondeterministic, but this nondeterminism is highly constrained by the common limit requirement. Furthermore, $e$ is neither allowed to diverge nor be erroneous. If any of the conditions required for~$t$ to exist fails, then the $\clim{x}{e}$ computation is declared erroneous.
This is appropriate because, in an algorithmic implementation of the limit operation, the source of error may occur deep in the computation (e.g., only at some high value for the integer~$k$) meaning that the algorithm may, before the error transpires, return erroneous information in the form of approximating values to a non-existent limit.

\Cref{figure:rw-denotations} assigns semantics to
several of the general expression  constructors: sequencing $c_1 ; c_2$,
local variable declarations $\cvar{x}{e} c$,
assignments $\clet{x_{i}}{e}$,
conditionals $\cif e \cthen c_1 \celse c_2 \cend$, and expressions $e$ qua commands.

Next, let us define the semantics of guarded choice
%
\begin{equation*}
  \Gamma ; \Delta \rwtypes (\ccase e_1 \To c_1 \such \cdots \such e_n \To c_n \cend) : \tau.
\end{equation*}
%
Using the abbreviations
%
$\sem{e_i} = \sem{\Gamma, \Delta \rotypes e_i : \dB}$
and
$\sem{c_i} = \sem{\Gamma; \Delta \rwtypes c_i : \tau}$
%
we set
%
\begin{align*}
  &\sem{\Gamma ; \Delta \rwtypes (\ccase e_1 \To c_1 \such \cdots \such e_n \To c_n \cend) : \tau} \, \gamma \, \delta = 
  \\
  &\quad
  \begin{cases}
    \emptyset \qquad\text{if $\some{i} \sem{e_i} \, (\gamma, \delta) = \emptyset \lor (\semtt \in \sem{e_i} \, (\gamma, \delta) \land \sem{c_i} \, \gamma \, \delta = \emptyset)$}
    \\
    S
    \cup \{ \bot \such \all{i} \sem{e_i} (\gamma, \delta) \neq \pure{\semtt} \}
    \qquad\text{otherwise}
  \end{cases}
  \\
  &\quad\text{where $S = \bigcup \left\{\sem{c_i} \, \gamma \, \delta \such
                 1 \leq i \leq n \land
                \semtt \in \sem{e_i} (\gamma, \delta) \right\}.$}
\end{align*}
The idea behind this definition is as follows. 
The $n$ (potentially nondeterministic) guard expressions $e_1, \dots, e_n$ are evaluated in parallel. 
Ignoring, for the moment, the possibility that one of these expressions might be erroneous, suppose
that one of them, 
$e_i$ say, evaluates to $\semtt$. If this occurs, then the parallel evaluation of the other guards is terminated and the continuation
$c_i$ is executed. Note that the choice of $i$ here is potentially nondeterministic.
If instead none of the guards evaluates to  $\semtt$, then none of the continuations is triggered and we consider this as amounting to nontermination (no error is caused), so we include 
$\bot$ in the denotation of the case expression. Note that this $\bot$ can arise as a result of \emph{bona fide} nontermination
(none of the guards terminates) or of deadlock (all guards terminate with $\semff$).
In the case that any of the $e_i$ guards causes an error (i.e., has denotation $\emptyset$), or if some $e_i$ has a possible evaluation to $\semtt$ that triggers the execution of an erroneous continuation command $c_i$, then the case statement is itself given the error denotation $\PPerr$.
Some of the subtleties of guarded choice are illustrated through the examples in
Section~\ref{sec:boolean-ops} below. 



It remains to define the semantics of while loops.
As usual, the meaning of a while loop is required to be invariant under unrolling; i.e.,
\begin{align*}
& \sem{\Gamma;\Delta \rwtypes \cwhile e \cdo c \cend :\dU}\,\gamma\,\delta  = 
\\
& \qquad \sem{\Gamma;\Delta \rwtypes\cif e \cthen (c\, ; \, \cwhile e \cdo c \cend) \celse \cskip \cend :\dU}\,\gamma\,\delta
\end{align*}
That is, we want the value $\sem{\Gamma ; \Delta \rwtypes (\cwhile e \cdo c \cend) : \dU} \, \gamma $ to be a fixed point of the map 
%
\begin{equation}
  \label{eq:def-W}
  \begin{aligned}
    W_{\gamma} &:  \PP{\sem{\Delta} \times \{\semuu\}}^{\sem{\Delta}}
    \to \PP{\sem{\Delta} \times \{\semuu\}}^{\sem{\Delta}}
    \\
    W_\gamma&(h) \, \delta 
    \defeq \PPlet{b}{\sem{e}\,(\gamma,\delta)}
      \begin{cases}
        \lift{(h \circ \pi_1)}(\sem{\Gamma; \Delta \rwtypes c : \dU}\,\gamma\,\delta) & \text{if $b = \semtt$} \\
        \pure{(\delta, \semuu)} & \text{if $b = \semff$}
      \end{cases}
  \end{aligned}
\end{equation}
%
where $\pi_1$ is the projecting isomorphism from $\sem{\Delta} \times \{\semuu\}$ to $\sem{\Delta}$.

As is standard, we find the appropriate fixed point  by showing that $\PP{S}$ carries the structure of a domain ($\omega$-complete partial order with least element) and that $W_\gamma$ is an $\omega$-continuous function with respect to the induced pointwise order on the function space 
$\PP{\sem{\Delta} \times \{\semuu\}}^{\sem{\Delta}}$.
This allows the definition of $\sem{\Gamma ; \Delta \rwtypes (\cwhile e \cdo c \cend) : \dU} \, \gamma $ as the least fixed point $\mathsf{LFP}(W_\gamma)$ of $W_\gamma$:
%
\begin{equation}
\label{equation:lfp}
  \sem{\Gamma ; \Delta \rwtypes (\cwhile e \cdo c \cend) : \dU} \, \gamma  \defeq
  \mathsf{LFP}(W_\gamma) \enspace .
\end{equation}
%


The required partial order on $\PP{S}$ is that of the Plotkin powerdomain~\cite{plotkin1976powerdomain} modified to take account of our use of the error set $\PPerr$:
%
\begin{equation*}
  X \PPleq Y ~ \defiff~
  X = Y  ~\lor ~
  (\bot \in X \, \land\, (Y = \PPerr \lor (X \!\setminus\! \PPbot \subseteq Y))).
\end{equation*}
%
For $X, Y$ other than the error set $\PPerr$, the above coincides with the usual Egli-Milner order of the Plotkin powerdomain.
The positioning of~$\PPerr$ within the partial order is motivated by the following considerations.
%
The denotational semantics of $\cwhile e \cdo c \cend$ in environment~$\gamma$ is approximated by
applying $W_\gamma$ iteratively to the constantly-bottom function  $\delta \mapsto \PPbot$, yielding an $\omega$-chain
%
\begin{equation*}
  (\delta \mapsto \PPbot) \PPleq W_\gamma(\delta \mapsto \PPbot) \PPleq W_\gamma^2(\delta \mapsto \PPbot) \PPleq \dots
\end{equation*}
%
with each new approximation corresponding to one further level of unrolling of the loop. 
The presence of~$\bot$ in any $W_\gamma^n(\delta \mapsto \PPbot)\,\delta_0$ can indicate that
further unfoldings are needed to determine
$\sem{\Gamma ; \Delta \rwtypes (\cwhile e \cdo c \cend) : \dU} \, \gamma \, \delta_0$.
It is possible that some such further unfolding will result in an erroneous computation, at which point the denotational semantics will assume the value~$\PPerr$. For this reason it is necessary to have $X \PPleq \PPerr$, whenever $\bot \in X$.
In the case that $\PPbot \notin W_\gamma^n(\delta \mapsto \PPbot)\,\delta_0$, it holds that 
$\sem{\Gamma ; \Delta \rwtypes (\cwhile e \cdo c \cend) : \dU} \, \gamma \, \delta_0 = 
W_\gamma^n(\delta \mapsto \PPbot)\,\delta_0$, i.e., the semantics is fully determined at iteration $n$, and does not change under further iterations.
Thus nonempty sets $X$ with $\bot \notin X$ do not approximate $\PPerr$, i.e., $X \not\PPleq \PPerr$.

\begin{proposition} 
\label{prop:domain}
For any set $S$, it holds that $(\PP{S}, \PPleq)$ is an $\omega$-complete partial order with least element.
\end{proposition}

\begin{proof}
The least element is $\PPbot$.
%
For $\omega$-completeness, suppose that
%
\begin{equation*}
  X_0 \PPleq X_1 \PPleq X_2 \PPleq \dots
\end{equation*}
%
is an $\omega$-chain.
In the case that every $X_n$ contains $\bot$, it is easy to check that the supremum is $ \dsup{i} X_i \defeq \bigcup_i X_i$.
If instead there exists $n$ such that $\bot \notin X_n$ then necessarily $X_m = X_n$ for all $m \geq n$, so 
the supremum is $\dsup{i} X_i \defeq X_n$. (In the case that no $X_n$ is~$\PPerr$, the above coincides with the Plotkin powerdomain.)
\end{proof}

In the proof of the following proposition we use the \defemph{strict union} operation on $\PP{S}$, which models nondeterministic choice:
\[
X \uplus Y = \begin{cases}
\emptyset & \text{if $X = \emptyset$ or $Y = \emptyset$,} \\
X \cup Y & \text{otherwise.}
\end{cases}
\]

\begin{proposition} 
\label{prop:continuity}
The function $W_\gamma$ defined by \eqref{eq:def-W} is continuous with respect to the
pointwise partial order on $\PP{\sem{\Delta} \times \{\semuu\}}^{\sem{\Delta}}$.
\end{proposition}
%

\begin{proof}
%
The function $W_\gamma$ is the composition of several maps, two of which need their continuity checked.
%
The first one is monadic evaluation
%
\begin{align*}
   \PP{T}^S \times \PP{S} &\to \PP{T}  \\
  (g, X) &\mapsto \lift{g}(X)
\end{align*}
%
Monotonicity is straightforward.
%
To establish continuity in~$X$, consider an $\omega$-chain $X_0 \PPleq X_1 \PPleq X_2 \PPleq \dots$.
If every $X_n$ contains $\bot$, then 
\begin{multline*}
\textstyle
\lift{g}(\dsup{i}X_i) =
\lift{g}\left(\bigcup_{i}X_i\right) =
\{\bot\} \cup \bigcup \left\{g(x) \such x \in \left(\bigcup_i X_i\right) - \{\bot\}\right\}
\\
\textstyle
= \{\bot\} \cup \bigcup_i \{g(x) \such x \in X_i - \{\bot\}\}
= \dsup{i}  \lift{g}(X_i).
\end{multline*}
%
Otherwise the chain is eventually constant and~$g$ preserves its supremum because it is monotone.

To establish continuity in $g$, consider an $\omega$-chain $g_0 \PPleq g_1 \PPleq g_2 \PPleq \dots$ with respect to the pointwise order on $\PP{T}^S$.
%
If there are $k \in \IN$ and $x \in X$ such that $g_k(x) = \PPerr$ then $\lift{(\dsup{i} g_i)}(X) = \PPerr = \dsup{i} \lift{g_i}(X)$.
Thus it remains to prove that the sets
%
\begin{equation*}
  \textstyle
  \lift{(\dsup{i} g_i)}(X) =
      \{\bot \such \bot \in X\} \cup 
      \bigcup_{x \in X \setminus \{\bot\}} \dsup{i} g_i(x) \eqdef A
\end{equation*}
%
and
%
\begin{equation*}
  \textstyle
  \dsup{i} \lift{g_i}(X) =
  \dsup{i} \left(
      \{\bot \such \bot \in X\} \cup 
      \bigcup_{x \in X \setminus \{\bot\}} g_i(x)
   \right) \eqdef B.
\end{equation*}
%
are equal, assuming $g_i(x) \neq \PPerr$ for all $i \in \IN$ and $x \in X$.
%
Clearly, $\bot \in A \liff \bot \in B$.
%
To show that $A \setminus \{\bot\} = B \setminus \{\bot\}$, we first note that, for all $k \in \IN$, $x \in X$ and $y \neq \bot$,
%
\begin{equation*}
  y \in \dsup{i} g_i(x)
  \iff
  \some{i \in \IN} y \in g_i(x) \setminus \{\bot\},
\end{equation*}
%
thanks to the standing assumption that~$\PPerr$ does not appear in the supremum.
%
Now, if $y \in A \setminus \{\bot\}$ then $y \in \dsup{i} g_i(x)$ for some $x \in X \setminus \{\bot\}$,
hence $y \in g_j(x)$ for some $j \in \IN$, and so $y \in B$.
%
Conversely, if $y \in B \setminus \{\bot\}$ then $y \in g_j(x)$ for some $j \in \IN$ and $x \in X \setminus \{\bot\}$, hence $y \in \dsup{i} g_i(x)$, and so $y \in A$.

The other function partaking in $W_\gamma$ is
\begin{align*}
  C &: \PP{\{\semff, \semtt\}} \times \PP{S} \times \PP{S} \to \PP{S}
  \\
  C &: (X,Y,Z) \mapsto \PPlet{b}{X} 
      \begin{cases} 
        Y & \text{if $b = \semtt$} \\
        Z & \text{if $b = \semff$} 
      \end{cases}
\end{align*}
%
We only verify continuity in $Y$, which is done by case analysis on~$X$:
%
\begin{itemize}
\item If $\semtt \notin X$ then, $C$ is constant in~$Y$.
\item If $X = \pure{\semtt}$ then $C(\pure{\semtt},Y,Z) = Y$ is a projection.
\item If $X = \{\semtt, \semff\}$ then $C(Y) = Y \uplus Z$.
\item If $X = \{\semtt,\bot\}$ then $C(Y) = Y\cup \{\bot \such Y \neq \PPerr\}$.
\item If $X = \{\semtt, \semff,\bot\}$ then  $C(Y) = (Y \uplus Z) \cup \{\bot \such (Y \uplus Z) \neq \PPerr\}$.\end{itemize}
In each case, continuity in $Y$ is easy to show.
\end{proof}

It follows from \cref{prop:domain,prop:continuity} that the semantic definition of while commands in~\eqref{equation:lfp} is well-defined.


\subsection{Semantics of first-order functions}
\label{sec:semant-first-order}

We briefly address the denotational semantics of first-order functions from \cref{sec:first-order-func}.
The denotation of a function
%
\begin{equation*}
  \cfunction f {x_1 \of \tau_1, \ldots, x_n \of \tau_n} {\sigma} {e}
\end{equation*}
%
is just the denotation of its body,
%
\begin{equation*}
  \sem{f} \defeq
  \sem{x_1 \of \tau_1, \ldots, x_n \of \tau_n \rotypes e : \sigma} :
  \sem{\tau_1} \times \cdots \times \sem{\tau_n} \to \PP{\sem{\sigma}}.
\end{equation*}
%
We need to be a bit careful about the denotation of a function call $f(e_1, \ldots, e_n)$ because the arguments
$e_1, \ldots, e_n$ may diverge or yield nondeterministic values, so it matters if and when they are evaluated.
%
We opt for the call-by-value evaluation strategy that is most commonly seen in imperative languages.

Given sets~$S$ and~$T$, define the \defemph{monadic pairing}
%
$\PPtuple{{-}, {-}} : \PP{S} \times \PP{T} \to \PP{S \times T}$
%
by
%
\begin{equation*}
  \PPtuple{X, Y} \defeq (\PPlet{x}{X} \PPlet{y}{Y} \{(x, y)\}).
\end{equation*}
%
Note that the binding order does not matter, i.e., $(\PPlet{x}{X} \PPlet{y}{Y} \cdots) = (\PPlet{y}{Y} \PPlet{x}{X} \cdots)\})$ because $\Pstar$ is a commutative monad.
%
Monadic pairing is easily extended from pairs to $n$-tuples for arbitrary~$n$.

The denotation of a function call is application adorned with the monad structure:
%
\begin{align*}
  \sem{\Gamma \rotypes f(e_1, \ldots, e_n) : \sigma} \gamma \defeq &
  \lift{\sem{f}} \PPtuple{\sem{e_1} \gamma, \ldots, \sem{e_n} \gamma}.
\end{align*}


\section{Nondeterminism and parallelism}
\label{sec:boolean-ops}

The guarded case construct of Clerical requires parallel evaluation of the guards
leading to potential nondeterminism.  As a result, several basic operations involving nondeterminism and parallel evaluation are definable in Clerical.

A simple binary nondeterministic choice between two pure expressions
$\Gamma \rotypes e_1 : \tau$ and $\Gamma \rotypes e_2 : \tau$ is implemented
by
\[\Gamma \rotypes (\ccase \ctrue \To e_1 \such \ctrue \To e_2 \cend) : \tau \]
This has the derived semantics
\[
\sem{\Gamma \rotypes (\ccase \ctrue \To e_1 \such \ctrue \To e_2 \cend) : \tau }\, \gamma
~ = ~ \sem{\Gamma \rotypes e_1 : \tau } \, \gamma\,
\uplus \,
\sem{\Gamma \rotypes e_2  : \tau }\,\gamma
\]
using the strict union operation defined above Proposition~\ref{prop:continuity}.

It is also possible to implement McCarthy's \defemph{ambiguous choice} between 
$\Gamma \rotypes e_1 : \tau$ and $\Gamma \rotypes e_2 : \tau$, by:
\[\Gamma \rotypes (\ccase (\cvar{x}{e_1} \ctrue) \To e_1 \such 
(\cvar{x}{e_2} \ctrue) \To e_2 \cend) : \tau \]
Writing $\Gamma \rotypes e_1 \,\mathtt{amb}\, e_2 : \tau$ for the above, 
we have
%
\begin{multline*}
\sem{\Gamma \rotypes  e_1 \,\mathtt{amb}\, e_2 : \tau }\, \gamma = \\
  (\sem{\Gamma \rotypes e_1 : \tau } \, \gamma\, \uplus \, \sem{\Gamma \rotypes e_2  : \tau }\,\gamma)
  \setminus
  \{ \bot \mid \bot \notin (\sem{\Gamma \rotypes e_1  : \tau }\,\gamma 
\,\cap \,\sem{\Gamma \rotypes e_2  : \tau}\,\gamma)\}\;.
\end{multline*}

A well-known issue with ambiguous choice is that it is not monotonic with respect to any reasonable powerdomain partial ordering \cite{LEVY2007221}, meaning that it does not have a domain-theoretic semantics. Indeed, such a failure of monotonicity holds with respect to 
the ordering $\PPleq$ we have defined on our powerdomain $\PP{-}$. It follows that the denotational semantics of Clerical expressions does not, in general, act monotonically in the semantics of subexpressions. We present a  simple example of this phenomen.


Consider the expression below, which is parametric in the subexpression
$\rotypes e  : \dB$:
%
\begin{lstlisting}
case
| e => while true do skip end
| true => skip
end
\end{lstlisting}
%
In the case that $\sem{e} =  \pure{\bot}$, the denotation of the whole expression is $\{\semuu\} $. If instead $\sem{e} =  \pure{\semtt}$, then the denotation of the expression is $\{\semuu, \bot\}$. This breaks monotonicity because
$\pure{\bot} \PPleq  \pure{\semtt}$ in $\PP{\{\semff, \semtt\}}$, but $\{\semuu\} \not \PPleq \{\semuu, \bot\}$  in $\PP{\{\semuu\}}$.
Similarly, considering the case in which $e$ is an erroneous expression (i.e., $\sem{e} = \PPerr$),
we have $\pure{\bot} \PPleq  \PPerr$, but $\{\semuu\} \not \PPleq \PPerr$.


Given the non-monotonicity properties illustrated above, it is perhaps surprising that it is possible to give Clerical a denotational semantics involving a domain-theoretic fixed-point argument to establish the semantics of while loops. The reason for this is that the operator $W_\gamma$, used in defining the semantics of while loops, is defined purely as a combination of conditional statements and sequencing, and does not involve the problematic non-monotonic aspects of Clerical. Indeed, as Proposition~\ref{prop:continuity} establishes,  the particular operator $W_\gamma$ is always continuous, hence \emph{a fortiori}  monotone. 

Clerical, as we have defined it, does not include any primitive operator for manipulating booleans. This does not limit expressivity since logical operations are definable using the conditional construct.
For example, negation of a boolean expression $b$ is defined by
\[
\neg b \defeq (\cif b \cthen \cfalse \celse  \ctrue  \cend)
\]
The most concise way of defining the disjunction of two boolean typed expressions $b_1$ and $b_2$ is by $\cif b_1 \cthen \ctrue \celse  b_2 \cend$. This is asymmetric: if 
$\sem{b_1} = \pure{\semtt}$ and $\sem{b_2} = \pure {\bot}$ then the disjunction has denotation $\pure{\semtt}$, whereas if
$\sem{b_1} = \pure{\bot}$ and $\sem{b_2} = \pure {\semtt}$ then 
the resulting denotation is $\pure{\bot}$. It is similarly possible to define a symmetric strict disjunction by 
\[\cif b_1 \cthen (\cif b_2 \cthen \ctrue \celse \ctrue \cend) \celse  b_2 \cend\]
More interestingly, the guarded case construct can be used to define Plotkin's \defemph{parallel or} \cite{plotkin1977lcf}, another symmetric version of  disjunction which requires parallel evaluation, by
\begin{equation*}
b_1 \cdisj b_2 \defeq
(\ccase
      b_1 \To \ctrue
\such b_2 \To \ctrue
\such \neg b_1 \To \neg b_2
\cend)
\end{equation*}
From a logical perspective, when applied to deterministic expressions, $b_1 \cdisj b_2$ computes the disjunction of $b_1$ and $b_2$ from Kleene (and Priest) 3-valued logic.

Since Clerical is a nondeterministic language, the defined logical operators all have an induced effect on nondeterministic and erroneous expressions. For example, the derived full semantics for parallel or is:
\begin{align*}
\sem{\Gamma \rotypes b_1 \cdisj b_2 : \dB }(\gamma) &= \bigcup_{\substack{v_1 \in \sem{\Gamma \rotypes b_1:\dB}(\gamma)\\v_2 \in \sem{\Gamma \rotypes b_2 : \dB}(\gamma)}}\begin{cases}
\{\semtt\}&\text{if }v_1 = \semtt \lor v_2 = \semtt,\\
\{\semff\}&\text{if }v_1 = \semff \land v_2 = \semff,\\
\{\bot\}&\text{otherwise}.
\end{cases}
\end{align*}



%%% Local Variables:
%%% mode: latex
%%% TeX-master: "clerical"
%%% End:

\chapter{Example Sheets}

%% TJWP removed plain tex commands
%\magnification\magstephalf
%\font\bb=msbm10 scaled 1200
%\font\small=cmr8
%\font\little=cmmi7
%%%%%%%%%%%%%%%%%%%%%%%

%% TJWP include page numbers
%\nopagenumbers
\def \pr{\partial}
\def\pmb#1{\setbox0=\hbox{#1}%
 \kern-.025em\copy0\kern-\wd0
 \kern.05em\copy0\kern-\wd0
 \kern-.025em\raise.0433em\box0 }
\def \bE{{\pmb {${\cal E}$}}}
\noindent

%\centerline{{\bf Example Sheet 1}}
%\vskip 10pt
\section{Example Sheet 1}

\noindent
{\bf 1.} Explain why 
\vskip 0.3cm
(i) GR effects are important for neutron stars but not for white
dwarfs
\vskip 0.3cm
(ii) inverse beta-decay becomes energetically favourable for
densities higher than those in white dwarfs.
\vskip 10 pt
\noindent
{\bf 2.} Use Newtonian theory to derive the Newtonian pressure
support equation
$$
P'(r) \equiv {dP\over dr} = -{Gm\rho\over r^2}\ ,
$$
where
$$
m=4\pi \int_0^r \tilde r^2 \rho(\tilde r) d\tilde r\ ,
$$
for a spherically-symmetric and static star with pressure $P(r)$ and 
density $\rho(r)$. Show that
%% TJWP no eqalign in latex  \eqalign{
\bean
\int_0^r P(\tilde r)\tilde r^3 d\tilde r & = & 
{P(r)r^4\over 4} -{1\over 4}\int_0^r P'(\tilde r) \tilde r^4 d\tilde 
r \\
 & = & {Gm^2(r)\over 32\pi} + {P(r)r^4\over 4}\ .
\eean
Assuming that $P'\le 0$, with $P=0$ at the star's surface, show that
$$
{d\over dr}\left[\left(\int_0^r P(\tilde r)\tilde r^3 d\tilde
r\right)^{3/4}\right] \le {3\sqrt{2}\over 4} P^{3/4} r^2\ .
$$
Assuming the bound
$$
P\ {\buildrel <\over\sim}\ (\hbar c) n_e^{4/3}\ ,
$$
where $n_e(r)$ is the electron number density, show that the total
mass, $M$, of the star satisfies
$$
M\ {\buildrel <\over\sim}\ \left({hc\over
G}\right)^{3/2}\left({\mu_e\over m_N}\right)^2 
$$
where $m_N$ is the nucleon mass and $\mu_e$ is the number of
electrons per nucleon. Why is it reasonable to bound the pressure as
you have done? Compare your bound with Chandresekhar's limit.

\vskip 10 pt
\noindent
{\bf 3.} A particle orbits a Schwarzschild black hole with
non-zero angular momentum per unit mass $h$. Given that $\sigma=0$ for
a massless particle and $\sigma=1$ for a massive particle, show that
the orbit satisfies  
$$ 
{d^2 u\over d\phi^2} + u = {M\sigma \over
h^2} + 3Mu^2 
$$
where $u=1/r$ and $\phi$ is the azimuthal angle. Verify that this
equation is solved by
$$
u= {1\over 6M} + {2\omega^2\over 3M} -{2\omega^2\over M\cosh ^2
(\omega\phi)}\ ,
$$
where $\omega$ is given by
$$
4\omega^2 = \pm\sqrt{\left(1 - {12M^2\sigma\over h^2}\right)}\ .
$$
where $\sigma=1$ for a massive particle and $\sigma=0$ for a massless
particle. Interpret these orbits in terms of the effective potential.
Comment on the cases $\omega^2=1/4$, $\omega^2=1/8$ and $\omega^2=0$.

\vskip 10 pt
\noindent
{\bf 4.} A photon is emitted outward from a point P outside
a Schwarzschild black hole with radial coordinate r in the range
$2M<r<3M$. Show that if the photon is to reach infinity the angle its
initial direction makes with the radial direction (as determined by 
a stationary observer at P) cannot exceed
$$
{\rm arcsin} \sqrt{{27M^2\over r^2}\left( 1-{2M\over r}\right)}\ .
$$

\vskip 10 pt
\noindent
{\bf 5.} Show that in region II of the Kruskal manifold one may
regard $r$ as a time coordinate and introduce a new spatial
coordinate $x$ such that
$$
ds^2 = -{dr^2\over \left({2M\over r} -1\right)} +\left({2M\over
r}-1\right)dx^2 + r^2d\Omega^2\ .
$$
Hence show that {\it every} timelike
curve in region II intersects the singularity at $r=0$ within a proper
time no greater than $\pi M$. For what curves is this bound attained?
Compare your result with the time taken for the collapse of a ball of
pressure free matter of the same gravitational mass $M$. Calculate
the binding energy of such a ball of dust as a fraction of its
(conserved) rest mass.
\vskip 10 pt
\noindent
{\bf 6.} Using the map
$$
(t,x,y,z) \mapsto X= \begin{pmatrix}t+z & x+iy\\\ x-iy & t-z\end{pmatrix}\ ,
$$
show that Minkowski spacetime may be identified with the space of
Hermitian $2\times 2$ matrices $X$ with metric
$$
ds^2 = -\det (dX)\ .
$$
Using the Cayley map $X\mapsto U={1+iX\over1-iX}$, show further that
Minkowski spacetime may be identified with the space of unitary
$2\times 2$ matrices $U$ for which $\det (1+U)\ne0$. Now show that any
$2\times 2$ unitary matrix $U$ may be expressed uniquely in terms of
a real number $\tau$ and two complex numbers $\alpha$, $\beta$, as 
$$
U=e^{i\tau}\begin{pmatrix}\alpha & \beta\cr -\bar\beta & \bar\alpha\end{pmatrix}
$$
where the parameters $(\tau,\alpha,\beta)$ satisfy $|\alpha|^2
+|\beta|^2 =1$, and are subject to the identification 
$$
(\tau,\alpha,\beta) \sim (\tau +\pi, -\alpha, -\beta)\ .
$$
Using the relation
$$
(1+U)dX = -2i dU(1+U)^{-1}\ ,
$$
deduce that 
$$
ds^2 = {1\over (\cos \tau + {\cal R}e\,\alpha)^2}\left(-d\tau^2
+|d\alpha|^2 + |d\beta|^2\right) 
$$
is the metric on Minkowski spacetime and hence conclude that the
conformal compactification of Minkowski spacetime may be identified
with the space of unitary $2\times 2$ matrices, i.e the group
$U(2)$. Explain how $U(2)$ may be identified with a portion of the
Einstein static universe $S^3\times {\bb R}$.

\vfill\eject

%\centerline {{\bf Example Sheet 2}}
%\vskip 10 pt
\section{Example Sheet 2}

\noindent
{\bf 1.} Let $\zeta$ be a Killing vector field. Prove that
$$
D_\sigma D_\mu \zeta_\nu = R_{\nu\mu\sigma}{}^\lambda \zeta_\lambda\ ,
$$
where $R_{\nu\mu\sigma\lambda}$ is the Riemann tensor, defined by
$[D_\mu,D_\nu]\, v_\rho = R_{\mu\nu\rho}{}^\sigma v_\sigma$ for
arbitrary vector field $v$.

\vskip 10 pt
\noindent
{\bf 2.}  A conformal Killing vector is one for which  
$$
({\cal L}_\xi g)_{\mu\nu} = \Omega^2 g_{\mu\nu}\ .
$$
for some non-zero function $\Omega$.
Given that $\xi$ is a Killing vector of $ds^2$, show that it is a
conformal Killing vector of the conformally-equivalent
metric $\Lambda^2 ds^2$ for arbitrary (non-vanishing) conformal factor
$\Lambda$.
 
Show that the action for a {\it massless} particle,
$$ S[x,e]={1\over2}\int d\lambda\, e^{-1}\dot x^\mu\dot x^\nu
g_{\mu\nu}(x)\ , $$ is invariant, to first order in the constant
$\alpha$, under the transformation 
$$
x^\mu \rightarrow x^\mu + \alpha \xi^\mu(x) \qquad \qquad
e \rightarrow e + {1\over4}\alpha\, e g^{\mu\nu}({\cal L}_\xi
g)_{\mu\nu}
$$
if $\xi =\xi^\mu \partial_\mu$ is a conformal Killing vector. Show
that $\xi$ is the operator corresponding to the conserved charge
implied by Noether's theorem.



\vskip 10 pt
\noindent
{\bf 3.} Show that the extreme RN metric in isotropic coordinates is
$$
ds^2 = -\left(1+{M\over\rho}\right)^{-2}dt^2 + 
\left(1+{M\over\rho}\right)^{2}\left(d\rho^2 + \rho^2 d\Omega^2\right)
\qquad (\dagger)
$$
Verify that $\rho=0$ is at infinite proper distance from any finite
$\rho$ along any curve of constant $t$. Verify also that
$|t|\rightarrow \infty$ as $\rho\rightarrow 0$ along any timelike or
null curve but that a timelike or null ingoing radial geodesic reaches
$\rho=0$ for {\it finite} affine parameter. By introducing a null
coordinate to replace $\rho$ show that $\rho=0$ is merely a coordinate
singularity and hence that the metric ($\dagger$) is geodesically
incomplete. What happens to the particles that reach $\rho=0$?
Illustrate your answers using a Penrose diagram.

\vskip 10 pt
 \noindent
{\bf 4.} The action for a particle of mass $m$ and charge $q$ is
$$
S[x,e] =\int d\lambda\,\left[{1\over2} e^{-1}\dot x^\mu\dot x^\nu
g_{\mu\nu}(x) -{1\over2}m^2 e -q\, \dot x^\mu A_\mu(x)\right]\qquad
\qquad (*) 
$$
where $A_\mu$ is the electromagnetic 4-potential. Show
that if
$$
({\cal L}_\xi A)_\mu \equiv \xi^\nu\partial_\nu A_\mu +
(\partial_\mu \xi^\nu) A_\nu =0
$$
for Killing vector $\xi$, then $S$ is invariant, to first-order in
$\xi$, under the transformation $x^\mu\rightarrow x^\mu
+\alpha\xi^\mu(x)$. Verify that the corresponding Noether charge
$$
-\xi^\mu \left(m u_\mu -q A_\mu\right)\ ,
$$ 
where $u^\mu$ is the particle's 4-velocity, is a constant of the 
motion. Verify for the Reissner-Nordstrom solution of the vacuum
Einstein-Maxwell equations, with mass $M$ and charge $Q$, that ${\cal
L}_k A =0$ for $k={\partial\over \partial t}$ and hence deduce, for
$m\ne 0$, that  
$$
\left(1-{2M\over r} + {Q^2\over r^2}\right) {dt\over d\tau} =
\varepsilon -{q\over m}{Q\over r}\ ,
$$
where $\tau$ is the particle's proper time and $\varepsilon$ is the
energy per unit mass. Show that the trajectories $r(t)$ of
massive particles with zero angular momentum satisfy
$$
({dr\over d\tau})^2 = (\varepsilon^2 -1) + \left(1-\varepsilon
{qQ\over mM}\right){2M\over r} +\left(\left({q\over m}\right)^2
-1\right){Q^2\over r^2}\ .
$$
Give a physical interpretation of this result for the special case
for which $q^2=m^2$, $qQ=mM$, and $\varepsilon=1$.


\vskip 10 pt
\noindent
{\bf 5.} Show that the action
$$
S[p,x,e]=\int d\lambda \big\{ p_\mu \dot x^\mu
-{1\over2}e\, \big[g^{\mu\nu}(x)p_\mu p_\nu + m^2\big]\big\}
$$
for a point particle of mass $m$ is equivalent, for $q=0$, to the action
of Q.4. Show that $S$ is invariant to
first order in $\alpha$ under the transformation
$$
\delta x^\mu =\alpha K^{\mu\nu}p_\nu \qquad \delta p_\mu
=-{1\over2}\alpha\, p_\rho p_\sigma \partial_\mu K^{\rho\sigma}
$$
for any symmetric tensor $K_{\mu\nu}$ obeying the {\it Killing
tensor} condition
$$
D_{(\rho} K_{\mu\nu)}=0\ .
$$
Show that the corresponding Noether charge is proportional to
$K^{\mu\nu}p_\mu p_\nu$ and verify that it is a constant of the
motion. A trivial example is $K_{\mu\nu}=g_{\mu\nu}$; what is
the corresponding constant of the motion? Show that
$\xi_\mu\xi_\nu$ is a Killing tensor if $\xi$ is a Killing vector. [A
Killing tensor that cannot be constructed from the metric and Killing
vectors is said to be irreducible. In a general axisymmetric
metric there are no such tensors, and so only three constants of the
motion, but for geodesics of the Kerr-Newman metric there is a 
`fourth constant' of the motion corresponding to an
irreducible Killing tensor.]
%$$
%\eqalign{
%K_{\mu\nu}dx^\mu dx^\nu = - &{\Sigma a^2\cos^2\theta\over \Delta}dr^2
%+ {\Delta a^2 \cos^2\theta\over \Sigma}(dt -a\sin^2\theta\,
%d\phi)^2\cr & +{r^2\sin^2\theta\over\Sigma}[adt - (r^2 + a^2)d\phi]^2
%+ r^2\Sigma
% d\theta^2\ .}
%$$

\vskip 10 pt 
\noindent 
{\bf 6.} By replacing the time coordinate $t$
by one of the radial null coordinates
$$
u= t+ {M\over \lambda} \qquad v= t- {M\over \lambda}
$$
show that the singularity at $\lambda=0$ of the Robinson-Bertotti (RB)
metric
$$
ds^2 = -\lambda^2 dt^2 + M^2 \left({d\lambda\over \lambda}\right)^2 +
M^2 d\Omega^2 
$$
is merely a coordinate singularity.
Show also that $\lambda=0$ is a degenerate Killing Horizon with respect
to $\partial\over \partial t$. By
introducing the new coordinates $(U,V)$, defined by
$$
u= \tan \left({U\over 2}\right)\qquad v=-\cot \left({V\over2}\right)
$$
obtain the maximal analytic extension of the RB metric and deduce its
Penrose diagram.

\vfill\eject

%\centerline {{\bf Example Sheet 3}}
%\vskip 10pt
\section{Example Sheet 3}

\noindent 
{\bf 1.} Let $\varepsilon$ and $h$ be the energy and and angular
momentum per unit mass of a zero charge particle in free fall
within the equatorial plane, i.e on a timelike ($\sigma=1$) or null
($\sigma=0$) geodesic with $\theta=\pi/2$, of a Kerr-Newman black hole.
Show that the particle's Boyer-Lindquist radial coordinate $r$
satisfies   
$$ 
\left({dr\over d\lambda}\right)^2 =\varepsilon^2 - V_{\mathit{eff}}(r)\ ,
$$ 
where $\lambda$ is an affine parameter, and the effective potential
$V_{\mathit{eff}}$ is given by $$ 
V_{\mathit{eff}} =
\left(1-{2M\over r}+{e^2\over r^2}\right)\left(\sigma + {h^2\over
r^2}\right) + {2a\varepsilon h\over r^3}\left( 2M -{e^2\over r}\right)
+{a^2\over r^2}\left[\sigma-\varepsilon^2\left(1+{2M\over r}-{e^2\over
r^2}\right)\right] \ .
$$

\vskip 10 pt 
\noindent 
{\bf 2.} Show that the surface gravity of the event horizon of a Kerr
black hole of mass $M$ and angular momentum $J$ is given by
$$
\kappa = {\sqrt{M^4 -J^2}\over 2M( M^2 + \sqrt{M^4 -J^2})}\ .
$$

\vskip 10 pt 
\noindent 
{\bf 3.} A particle at fixed $r$ and $\theta$ in a stationary
spacetime, with metric $ds^2= g_{\mu\nu}(r,\theta)dx^\mu dx^\nu$, has
angular velocity $\Omega= {d\phi\over dt}$ with respect to infinity.
Show that $\Omega(r,\theta)$ must satisfy
$$
g_{tt} + 2g_{t\phi}\Omega + g_{\phi\phi}\Omega^2 \le 0
$$
and hence deduce that
$$
{\cal D}\equiv g_{t\phi}^2 - g_{tt}g_{\phi\phi} \ge 0
$$
Show that ${\cal D}=\Delta (r) \sin^2\theta$ for the Kerr-Newman 
metric in Boyer-Lindquist coordinates, where $\Delta= r^2-2Mr+a^2+e^2$.
What happens if $(r,\theta)$ are such that ${\cal D}<0$? For what
values of $(r,\theta)$ can $\Omega$ vanish? Given that $r_{\pm}$ are the 
roots of $\Delta$, show that when ${\cal D}=0$  
$$ 
\Omega = {a\over r_{\pm}^2 +a^2} \ .
$$
\vskip 10pt
\noindent
{\bf 4.} Show that the area of the event horizon of a Kerr-Newman
black hole is
$$
A= 8\pi\big[ M^2 - {e^2\over2} + \sqrt{ M^4 -e^2 M^2 -J^2 }\,\big]\ .
$$

\vskip 10 pt
\noindent
{\bf 5.} A perfect fluid has stress tensor
$$
T_{\mu\nu} = (\rho + P)u_\mu u_\nu + P g_{\mu\nu}\ ,
$$
where $\rho$ is the density and $P(\rho)$ the pressure. State the
dominant energy condition for $T_{\mu\nu}$ and show that
for a perfect fluid in Minkowski spacetime this condition
is equivalent to  
$$ 
\rho\ge |P|\ .
$$
Show that the same condition arises from the requirement of
causality, i.e. that the speed of sound, $\sqrt{|dP/d\rho|}$, not
exceed that of light, together with the fact that the pressure 
vanishes in the vacuum.
\vskip 10pt
\noindent
{\bf 6.} The vacuum Einstein-Maxwell equations are
$$
G_{\mu\nu}= 8\pi T_{\mu\nu}(F) \qquad D_\mu F^{\mu\nu}=0
$$
where $F_{\mu\nu}= \partial_{\mu}A_{\nu}- \partial_\nu A_\mu$, and
$$
T_{\mu\nu}(F)= {1\over 4\pi}\big(F_{\mu}{}^\lambda F_{\nu\lambda}
-{1\over4}g_{\mu\nu}F^{\alpha\beta}F_{\alpha\beta}\big)\ .
$$
Asymptotically-flat solutions are stationary and axisymmetric if
the metric admits Killing vectors $k$ and $m$ that can be taken to be
$k={\partial\over\partial t}$ and $m={\partial\over\partial \phi}$ near
infinity, and if (for some choice of electromagnetic gauge)
$$
{\cal L}_k A={\cal L}_m A=0\ ,
$$
where the Lie derivative of $A$ with respect to a vector $\xi$, 
${\cal L}_\xi A$, is as defined in Q.4 of Example Sheet 2. The event
horizon of such a solution is necessarily a Killing horizon of $\xi =
k+\Omega_H m$, for some constant $\Omega_H$. What is the physical
interpretation of $\Omega_H$? What is its value for the Kerr-Newman
solution? The co-rotating electric potential is defined by 
$$ 
\Phi = \xi^\mu A_\mu \ .
$$
Use the fact that $R_{\mu\nu}\xi^\mu\xi^\nu=0$ on a Killing horizon
to show that $\Phi$ is constant on the horizon. In particular, show
that for a choice of the electromagnetic gauge for which $\Phi=0$ at
infinity, 
$$
\Phi_H= {Qr_+\over r_+^2 +a^2}
$$
for a charged rotating black hole, where $r_+= M+\sqrt{M^2-Q^2-a^2}$.

\vskip 10pt
\noindent
{\bf 7.} Let $({\cal M},g,A)$ be an asymptotically flat, stationary,
axisymmetric, solution of the Einstein-Maxwell equations of Q.6 and
let $\Sigma$ be a spacelike hypersurface with one boundary at spatial
infinity and an internal boundary, $H$, at the event horizon of a black
hole of charge $Q$. Show that    
$$ 
-2\int_\Sigma dS_\mu T^\mu{}_\nu(F)\xi^\nu = \Phi_H Q  
$$
where $\Phi_H$ is the co-rotating electric potential on the horizon. 
Use this result to deduce that the mass $M$ of a charged rotating black
hole is given by 
$$
M= {\kappa A\over 4\pi} + 2\Omega_H J + \Phi_H Q\ .
$$
where $J$ is the total angular momentum.
Use this formula for $M$ to deduce the first law of black hole
mechanics for charged rotating black holes: 
$$
dM= {\kappa \over 8\pi}dA + \Omega_H dJ + \Phi_H dQ\ .
$$
[Hint: ${\cal L}_\xi (F^{\mu\nu}A_\nu)=0$ ]

\vfill\eject


%\centerline{{\bf Example Sheet 4}}
%\vskip 10pt
\section{Example Sheet 4}

\noindent
{\bf 1.} Use the Komar integral,
$$
J= {1\over 16\pi G}\oint_\infty dS_{\mu\nu}D^\mu m^\nu\ ,
$$
for the total angular momentum of an asymptotically-flat axisymmetric
spacetime (with Killing vector $m$) to verify that $J=Ma$ for the
Kerr-Newman solution with parameter $a$.

\vskip 10pt
\noindent
{\bf 2.} Let $l$ and $n$ be two linearly independent vectors and
$\hat B$ a second rank tensor such that
$$
\hat B_\mu{}^\nu l_\nu =\hat B_\mu{}^\nu n_\nu =0\ .
$$
Given that $\eta^{(i)}$ $(i=1,2)$ are two further linearly
independent vectors, show that
$$
\varepsilon^{\mu\nu\rho\sigma}l_\mu n_\nu \hat B_\rho{}^\lambda
\big(\eta^{(1)}_\lambda\eta^{(2)}_\sigma - 
\eta^{(1)}_\sigma\eta^{(2)}_\lambda\big) =  \theta\, 
\varepsilon^{\mu\nu\rho\sigma} l_\mu n_\nu
\eta_\rho^{(1)}\eta_\sigma^{(2)}\ .
$$
where $\theta= \hat B_\alpha{}^\alpha$.

\vskip 10pt
\noindent
{\bf 3.} Let ${\cal N}$ be a Killing horizon of a Killing vector field
$\xi$, with surface gravity $\kappa$. Explain why, for any third-rank
totally-antisymmetric tensor $A$, the scalar 
$\Psi = A^{\mu\nu\rho}(\xi_\mu D_\nu\xi_\rho)$ vanishes on ${\cal N}$.
Use this to show that
$$
(\xi_{[\rho}D_{\sigma]} \xi_\nu)(D^\nu\xi^\mu) =\kappa
\xi_{[\rho}D_{\sigma]}\xi^\mu \qquad ({\rm on}\ {\cal N})\ ,\qquad (*)
$$
where the square brackets indicate antisymmetrization on the enclosed
indices.

From the fact that $\Psi$ vanishes on ${\cal N}$ it follows that its
derivative on ${\cal N}$ is normal to ${\cal N}$, and hence that
$\xi_{[\mu}\partial_{\nu]}\Psi=0$ on ${\cal N}$. Use this fact and the
Killing vector lemma of Q.II.1 to deduce that, on ${\cal N}$,
$$
(\xi_\nu R_{\sigma\rho[\beta}{}^\lambda\xi_{\alpha]}
+\xi_\rho R_{\nu\sigma[\beta}{}^\lambda\xi_{\alpha]}
+\xi_\sigma R_{\rho\nu[\beta}{}^\lambda\xi_{\alpha]})\xi_\lambda\ .
$$
Contract on $\rho$ and $\alpha$ and use the fact that $\xi^2=0$ on
${\cal N}$ to show that
$$
\xi^\nu\xi_{[\rho}R_{\sigma]\nu\mu}{}^\lambda\xi_\lambda =
-\xi_\mu\xi_{[\rho}R_{\sigma]}{}^\lambda\xi_\lambda
\qquad ({\rm on}\ {\cal N})\ ,\qquad (\dagger)
$$
where $R_{\mu\nu}$ is the Ricci tensor.

For any vector $v$ the scalar $\Phi=(\xi\cdot D\xi -\kappa\xi)\cdot v$
vanishes on ${\cal N}$. It follows that
$\xi_{[\mu}\partial_{\nu]}\Phi|_{\cal N} =0$. Show that this fact, the
result (*) derived above and the Killing vector lemma imply
that, on ${\cal N}$, 
%% $$ \eqalign{
\bean
\xi^\mu\xi_{[\rho}\partial_{\sigma]}\kappa & = & \xi^\nu
R_{\mu\nu[\sigma}{}^\lambda\xi_{\rho]}\xi_\lambda \\
& = &\xi^\nu\xi_{[\rho}R_{\sigma]\nu\mu}{}^\lambda\xi_\lambda\ ,
\eean
%% } $$
where the second line is a consequence of the cyclic identity
satisfied by the Riemann tensor. Now use $(\dagger)$ to show that, on
${\cal N}$,
%$$ \eqalign{
\bea
\xi^\mu\xi_{[\rho}\partial_{\sigma]}\kappa & = &
\xi_{[\sigma}R_{\rho]}{}^\lambda\xi_\lambda \\
&= & 8\pi G\, \xi_{[\sigma}T_{\rho]}{}^\lambda\xi_\lambda\ ,
\eea
%} $$
where the second line follows on using the Einstein equations. Hence
deduce the zeroth law of black hole mechanics: that, provided the
matter stress tensor satisfies the dominant energy condition, the
surface gravity of any Killing vector field $\xi$ is constant on
each connected component of its Killing horizon (in particular, on the
event horizon of a stationary spacetime). 
% TJWP removed \vfill\eject

\vskip 10pt
\noindent
{\bf 4.} A scalar field $\Phi$ in the Kruskal spacetime satisfies
the Klein-Gordon equation
$$
D^2\Phi -m^2\Phi =0\ .
$$
Given that, in static Schwarzshild coordinates, $\Phi$ takes the form
$$
\Phi = R_\ell(r) e^{-i\omega t} Y_{\ell}(\theta,\phi)
$$
where $Y_{\ell\, m}$ is a spherical harmonic, find the radial equation
satisfied by $R_\ell(r)$. Show that near the horizon at $r=2M$,
$\Phi\sim e^{\pm i\omega r^*}$, where $r^*$ is the Regge-Wheeler radial
coordinate. Verify that ingoing waves are analytic, in Kruskal
coordinates, on the future horizon, ${\cal H}^+$, but not, in general,
on the past horizon, ${\cal H}^-$, and conversely for outgoing waves.

Given that both $m$ and $\omega$ vanish, show that
$$
R_\ell = A_\ell P_\ell(z) + B_\ell Q_\ell(z)
$$
for constants $A_\ell,\, B_\ell$, where $z=(r-M)/M$, $P_\ell(z)$ is a
Legendre Polynomial and $Q_\ell(z)$ a linearly-independent solution.
Hence show that there are no {\it non-constant} solutions that are both
regular on the horizon, ${\cal H}= {\cal H}^+ \cup {\cal H}^-$, and
bounded at infinity.

\vskip 10 pt
\noindent
{\bf 5.} Use the fact that a Schwarzschild black hole radiates at the
Hawking temperature
$$
T_H ={1\over 8\pi M}
$$
(in units for which $\hbar$, $G$, $c$, and Bolzmann's constant all
equal $1$) to show that the thermal equilibrium of a black hole with an
infinite reservoir of radiation at temperature $T_H$ is unstable.

A finite reservoir of radiation of volume $V$ at temperature $T$ has
an energy, $E_{res}$ and entropy, $S_{res}$ given by
$$
E_{res} = \sigma VT^4 \qquad S_{res} ={4\over3}\sigma VT^3
$$
where $\sigma$ is a constant. A Schwarzschild black hole of mass $M$ is
placed in the reservoir. Assuming that the black hole has entropy
$$
S_{BH} =4\pi M^2\ ,
$$
show that the total entropy $S= S_{BH}+S_{res}$ is extremized 
for fixed total energy $E= M+E_{res}$, when $T=T_H$, Show that the
extremum is a maximum if and only if $V<V_c$, where the critical value
of $V$ is $$
V_c = {2^{20}\pi^4E^5\over 5^5\sigma }
$$
What happens as $V$ passes from $V<V_c$ to $V>V_c$, or
vice-versa?


\vskip 10pt
\noindent
{\bf 6.} The specific heat of a charged black hole of mass $M$, at
fixed charge $Q$, is 
$$
C\equiv T_H {\partial S_{BH}\over \partial
T_H}\bigg|_Q \ ,
$$
where $T_H$ is its Hawking temperature and $S_{BH}$ its
entropy. Assuming that the entropy of a black hole is given by $S_{BH}=
{1\over4}A$, where $A$ is the area of the event horizon, show that the
specific heat of a Reissner-Nordstrom black hole is
$$
C= {2S_{BH}\sqrt{M^2-Q^2}\over (M-2\sqrt{M^2-Q^2})}\ .
$$
Hence show that $C^{-1}$ changes sign when $M$ passes through
${2|Q|\over\sqrt{3}}$. 

Repeat Q.5 for a Reissner-Nordstrom black hole.
Specifically, show that the critical reservoir volume, $V_c$, is
infinite for $|Q|\le M \le {2|Q|\over\sqrt{3}}$. Why is this result
to be expected from your previous result for $C$? 



%\end










\section{Implementation}
\label{sec:implementation}

We turn attention to how Clerical, or a language based on it, might be implemented in practice.
A sensible implementation ought to work in such a way that an error-free well-typed \defemph{program} (a closed expression) $e$ of type~$\tau$ evaluates to one of its denotations, i.e., if $\sem{\emptyctx \rotypes e : \tau}() \neq \emptyset$ then $e$ evaluates to any $v \in \sem{\emptyctx \rotypes e : \tau}()$. More precisely, $e$ ought to evaluate to a \emph{representation} of~$v$, with the proviso that~$\bot$ corresponds to a non-terminating evaluation.

An implementation is certainly possible in principle. To see this, we can follow the approach of~\cite{brausse2016semantics} to show that Clerical programs are computable in the sense of Type Two Effectivity~\cite{w00}. We first endow each datatype~$\tau$ with a standard Baire space representation. In particular, reals are encoded by rapidly converging sequences of (encoded) rationals. We claim that whenever $\Gamma \rotypes e : \tau$ there is a type two Turing machine~$M$ which takes as input a representation of $\gamma \in \sem{\Gamma}$ and
%
\begin{itemize}
\item either $\sem{\Gamma \rotypes e : \tau} \, \gamma = \emptyset$, or
\item $M$ diverges and $\bot \in \sem{\Gamma \rotypes e : \tau} \, \gamma$, or
\item $M$ outputs a representation of an element of $\sem{\Gamma \rotypes e : \tau} \, \gamma$.
\end{itemize}
%
The construction of~$M$ proceeds recursively on the structure of~$e$. The most interesting is the $\ccase$ statement, which is implemented by combining the machines that compute the guards into a single machine that interleaves the guard computations and proceeds with the case whose guard first evaluates to~$\ctrue$.

More interesting and relevant is the question of an actual implementation of Clerical.
We implemented a proof-of-concept interpreter in OCaml, available at~\cite{clerical_ocaml}.
The connoisseurs will recognize the strong influence of the iRRAM package for exact-real arithmetic~\cite{muller2000irram}, which we gladly acknowledge.
%

We use the GNU MPFR library~\cite{mpfr} to compute with large integers and multiple-precision floating-point numbers.
%
During evaluation, real numbers are approximated by intervals whose endpoints are represented by multiple-precision floating-point numbers, rounded at the current \defemph{working precision}.
As the computation progresses, rounding, interval arithmetic, and limit approximations contribute to making the intervals ever wider.
If the intervals approximating $x$ and $y$ in a comparison $x \rlt y$ overlap, its Boolean value cannot be computed and evaluation is aborted due to \defemph{loss of precision}. The control returns to the top level, where the entire computation is restarted with higher working precision. The computation of a reciprocal $x^{-1}$ behaves analogously in case the interval approximating~$x$ overlaps with~$0$.
%
If the top level needs to output the result of a computation at a given precision, it keeps restarting it with ever higher working precision until the desired output precision is achieved.

The most interesting part of the interpreter is the implementation of guarded case
%
\begin{equation*}
  \ccase e_1 \To c_1 \mid \cdots \mid e_n \To c_n \cend.
\end{equation*}
%
Our semantics demands that one of the branches~$c_j$ must evaluate so long as one of the guards necessarily evaluates to $\ctrue$. Therefore, we cannot evaluate the guards $e_1, \ldots, e_n$ sequentially one after the other, lest the evaluation get stuck in a non-terminating guard.
%
We employed algebraic effects and handlers~\cite{plotkin09:_handl_algeb_effec,bauer15:_progr}, which are supported in OCaml~5, to implement cooperative threads that interleave the computations of the guards. The threads periodically yield control to the main scheduler, which enforces a round-robing evaluation strategy.
%
As soon as one of the threads~$e_i$ evaluates to~$\ctrue$, the other ones are aborted and the computation proceeds with the corresponding case~$c_i$.
%
We are looking forward to future experimentation with parallel execution of guards, which can be implemented using OCaml~5 multi-core features~\cite{sivaramakrishnan22:_retrof_concur}.

%%% Local Variables:
%%% mode: latex
%%% TeX-master: "clerical"
%%% End:

% !TEX root = runners-in-action.tex

\section{Related work}
\label{sect:relatedwork}

Comodels and (ordinary) runners have been used as a natural
model of stateful top-level behaviour.
%
For instance, Plotkin and Power~\cite{Plotkin:TensorsOfModels} have given a treatment of operational 
semantics using the tensor product of a model and a comodel.
%
Recently, Katsumata, Rivas, and Uustalu have generalised this interaction of models and comodels
to monads and comonads~\cite{Katsumata:InteractionLaws}.
%
An early version of \pl{Eff}~\cite{Bauer:AlgebraicEffects} implemented \emph{resources},
which were a kind of stateful runners, although they lacked satisfactory theory.
%
Uustalu~\cite{Uustalu:Runners} has pointed out that runners are the additional
structure that one has to impose on state
to run algebraic effects statefully.
%
Møgelberg and Staton's~\cite{Mogelberg:LinearUsageOfState} linear-use state-passing
translation also relies on equipping the state with a comodel
structure for the effects at hand. Our runners arise
when their setup is specialised to a certain Kleisli adjunction.

Our use of kernel state is analogous to the use
of parameters in parameter-passing 
handlers~\cite{Plotkin:HandlingEffects}: their 
$\tmkw{return}$ clause also provides a form of finalisation, as the 
final value of the parameter is available.
There is however no guarantee of
finalisation happening because handlers need not use the
continuation linearly.

The need to tame the excessive generality of handlers, and willingness to
give it up in exchange for efficiency and predictability, has recently
been recognised by \pl{Multicore OCaml}'s implementors, who
have observed that in practice most handlers resume 
continuations precisely once~\cite{Dolan:MulticoreOCaml}. In exchange for impressive efficiency, they require
continuations to be used linearly by default, whereas discarding and copying
must be done explicitly, incurring additional cost.
%
Leijen~\cite{Leijen:Finalisation} has extended 
handlers in \pl{Koka} with a $\tmkw{finally}$ clause, whose 
semantics ensures that finalisation happens whenever a handler
discards its continuation.
Leijen also added an $\tmkw{initially}$ clause to
parameter-passing handlers, which is used to compute the initial value of the parameter before handling, but that 
gets executed again every time the handler resumes its continuation.


%%% Local Variables:
%%% mode: latex
%%% TeX-master: "runners-in-action"
%%% End:

\section{Future work}
\label{sec:future-work}

In conclusion we discuss several directions for further work.

One is to explore how Clerical could be extended to include higher-order functions and general recursion.
%
Incorporating higher-order function without recursion should be straightforward from a language-design viewpoint. However, generalising the denotational semantics to cover higher-order functions may not be so straightforward, since the powerdomain $\PP{S}$ from 
Section~\ref{sec:denotation} will have to be generalised to allow $S$ to range over denotations of arbitrary types. 
%
With regards to general recursion, the non-monotonicity phenomena associated with the guarded case construct are likely to make it very challenging to define denotational semantics; see~\cite{LEVY2007221} for a discussion of related issues.

Second, in this paper we have not presented any formal operational semantics for Clerical.
Having one would provide an alternative and direct
account of the computability of the language, as well as a
framework within which implementation-relevant information, such as
the scope for parallelism in the execution strategy, could be studied in a mathematical setting. Also, a formally specified operational semantics
could guide implementations of Clerical and Clerical-like languages, and help estaliblish their correctness.

Third, we could further experiment with our implementation, which is good enough to evaluate the~$\pi$ program but cannot compete with the mature libraries for exact-real numbers.
%
To speed it up, we should at least implement parallel execution of threads, which is supported by the latest OCaml version.
%
A more substantive improvements would explore better evaluation strategies for nondeterminism, and compilation to a more efficient low-level language.

Fourth, there is significant room for improvement in the Coq formalization. By implementing better automation and tactics for proving correctness assertions, we would obtain a workable environment for formal verification of exact real computation, supported by the formidable machinery of Coq.


%% BIBLIOGRAPHY
\bibliography{references}



%% CC-BY PARAGRAPH REQUIRED BY THE PUBLISHER
\vfill

{\small\medskip\noindent{\bf Open Access} This chapter is licensed under the terms of the Creative Commons\break Attribution 4.0 International License (\url{http://creativecommons.org/licenses/by/4.0/}), which permits use, sharing, adaptation, distribution and reproduction in any medium or format, as long as you give appropriate credit to the original author(s) and the source, provide a link to the Creative Commons license and indicate if changes were made.}

{\small \spaceskip .28em plus .1em minus .1em The images or other third party material in this chapter are included in the chapter's Creative Commons license, unless indicated otherwise in a credit line to the material.~If material is not included in the chapter's Creative Commons license and your intended\break use is not permitted by statutory regulation or exceeds the permitted use, you will need to obtain permission directly from the copyright holder.}

\medskip\noindent\includegraphics{cc_by_4-0.eps}



% APPENDIX
\appendix
\makeatletter\def\@seccntformat#1{\appendixname~\csname the#1\endcsname: }\makeatother
% !TEX root = runners-in-action.tex

\section{Typing rules of $\lambdacoop$}
\label{sec:typing-rules}

In this appendix we give the complete typing rules for $\lambdacoop$. We refer to
\cref{fig:lambdacoop-types,fig:lambdacoop-terms} for the syntax of types, values, 
and user and kernel computations.
%
For each operation symbol $\op \in \Ops$, we assume a given and fixed operation signature
%
\begin{equation*}
  \op : \tysigop{A_\op}{B_\op}{E_\op},
\end{equation*}
%
and for each ground constant $\tmconst{f}$, we assume 
a signature $\tmconst{f} : (A_1,\ldots,A_n) \to B$, 
both of which the typing rules refer to without further ado.
%
Values, and user and kernel computations each have a typing and a subtyping judgement of the
form
%
\begin{align*}
  &\Gamma \types V : X,
& &\Gamma \types M : \tyuser{X}{\Ueff},
& &\Gamma \types K : \tykernel{X}{\Keff},\\
  &X \sub Y,
& &\tyuser{X}{\Ueff} \sub \tyuser{Y}{\Veff},
& &\tykernel{X}{\Keff} \sub \tykernel{Y}{\Leff}.
\end{align*}
%
where $\Gamma$ is the customary typing context assigning value types to variables. 
The subtyping rules are given in \cref{fig:subtyping}, and the 
typing rules in \cref{fig:typing-values,fig:typing-user,fig:typing-kernel}.


%%%%%%%%%%%%%%%%%%%%%% SUBTYPING RULES %%%%%%%%%%%%%%%%%%%%%%%%%%%%%%%%%%
\begin{figure}[p]
  \small
  \begin{mathpar}
    \coopinfer{Sub-Ground}{ }{A \sub A}

    \coopinfer{Sub-Product}{
      X \sub X' \\
      Y \sub Y'
    }{
      X \times Y \sub X' \times Y'
    }

    \coopinfer{Sub-Sum}{
      X \sub X' \\
      Y \sub Y'
    }{
      X + Y \sub X' + Y'
    }

    \coopinfer{Sub-UserFun}{
      X' \sub X \\
      \tyuser{Y}{\Ueff} \sub \tyuser{Y'}{\Ueff'}
    }{
      \tyfun{X}{\tyuser{Y}{\Ueff}} \sub \tyfun{X'}{\tyuser{Y'}{\Ueff'}}
    }

    \coopinfer{Sub-KernelFun}{
      X' \sub X \\
      \tykernel{Y}{\Keff} \sub \tykernel{Y'}{\Keff'}
    }{
      \tyfunK{X}{\tykernel{Y}{\Keff}} \sub \tyfunK{X'}{\tykernel{Y'}{\Keff'}}
    }

    %%% SELECTED RULE, KEEP IN SYNC WITH fig:typing-selected
    \coopinfer{Sub-Runner}{
      \sig_1' \subseteq \sig_1 \\
      \sig_2 \subseteq \sig_2' \\
      S \subseteq S' \\
      C \equiv C'
    }{
      \tyrunner{\sig_1}{\sig_2}{S}{C} \sub \tyrunner{\sig_1'}{\sig_2'}{S'}{C'}
    }

    \coopinfer{Sub-User}{
      X \sub X' \\
      \sig \subseteq \sig' \\
      E \subseteq E'
    }{
      \tyuser{X}{(\sig, E)} \sub \tyuser{X'}{(\sig', E')}
    }

    %%% SELECTED RULE, KEEP IN SYNC WITH fig:typing-selected
    \coopinfer{Sub-Kernel}{
      X \sub X' \\
      \sig \subseteq \sig' \\
      E \subseteq E' \\\\
      S \subseteq S' \\
      C \equiv C'
    }{
      \tykernel{X}{(\sig, E, S, C)} \sub \tykernel{X'}{(\sig', E', S', C')}
    }

    \coopinfer{Subsume-Value}{
      \Gamma \types V : X \\
      X \sub X'
    }{
      \Gamma \types V : X'
    }

    \coopinfer{Subsume-User}{
      \Gamma \types M : \tyuser{X}{\Ueff} \\
      \tyuser{X}{\Ueff} \sub \tyuser{X'}{\Ueff'}
    }{
      \Gamma \types M : \tyuser{X'}{\Ueff'}
    }

    \coopinfer{Subsume-Kernel}{
      \Gamma \types K : \tykernel{X}{\Keff} \\
      \tykernel{X}{\Keff} \sub \tykernel{X'}{\Keff'}
    }{
      \Gamma \types M : \tykernel{X'}{\Keff'}
    }
  \end{mathpar}
  \caption{Subtyping and subsumption rules.}
  \label{fig:subtyping}
\end{figure}

%%%%%%%%%%%%%%%%% VALUE RULES %%%%%%%%%%%%%%%%%%%%
\begin{figure}[p]
  \centering
  \small
\begin{mathpar}

  \coopinfer{TyValue-Var}{
    \Gamma(x) \equiv X
  }{
    \Gamma \types x : X
  }
  
  \coopinfer{TyValue-Const}{
    (\Gamma \types V_i : A_i)_{1 \leq i \leq n}
  }{
    \Gamma \types \tmconst{f}(V_1, \ldots , V_n) : B
  }

  \coopinfer{TyValue-Unit}{
  }{
    \Gamma \types \tmunit : \tyunit
  }

  \coopinfer{TyValue-Pair}{
    \Gamma \types V : X \\
    \Gamma \types W : Y
  }{
    \Gamma \types \tmpair{V}{W} : \typrod{X}{Y}
  }

  \coopinfer{TyValue-Inl}{
    \Gamma \types V : X
  }{
    \Gamma \types \tminl[X,Y]{V} : X + Y
  }

  \coopinfer{TyValue-Inr}{
    \Gamma \types W : Y
  }{
    \Gamma \types \tminr[X,Y]{W} : X + Y
  }

  \coopinfer{TyValue-UserFun}{
    \Gamma, x \of X \types M : \tyuser{Y}{\Ueff}
  }{
    \Gamma \types \tmfun{x : X}{M} : \tyfun{X}{\tyuser{Y}{\Ueff}}
  }

  \coopinfer{TyValue-KernelFun}{
    \Gamma, x \of X \types K : \tykernel{Y}{\Keff}
  }{
    \Gamma \types \tmfunK{x : X}{K} : \tyfunK{X}{\tykernel{Y}{\Keff}}
  }

  \coopinfer{TyValue-Runner}{
    \big(
      \Gamma, x \of A_\op \types K_\op : \tykernel{B_\op}{(\sig', E_\op, S, C)}
    \big)_{\op \in \sig}
  }{
    \Gamma \types
    \tmrunner{(\tm{op}\,x \mapsto K_{\tm{op}})_{\tm{op} \in \Sigma}}{C} :
    \tyrunner{\sig}{\sig'}{S}{C}
  }
\end{mathpar}
  \caption{Value typing rules.}
  \label{fig:typing-values}
\end{figure}

%%%%%%%%%%%%%%%%% USER RULES %%%%%%%%%%%%%%%%%%%%
\begin{figure}[tp]
  \centering
  \small
\begin{mathpar}
  \coopinfer{TyUser-Return}{
    \Gamma \types V : X
  }{
    \Gamma \types \tmreturn{V} : \tyuser{X}{\Ueff}
  }

  \coopinfer{TyUser-Apply}{
    \Gamma \types V : \tyfun{X}{\tyuser{Y}{\Ueff}} \\
    \Gamma \types W : X
  }{
    \Gamma \types \tmapp{V}{W} : \tyuser{Y}{\Ueff}
  }

  %%% SELECTED RULE, KEEP IN SYNC WITH fig:typing-selected
  \coopinfer{TyUser-Try}{
    \Gamma \types M : \tyuser{X}{(\sig,E)}
    \\
    \Gamma, x \of X \types N : \tyuser{Y}{(\sig,E')}
    \\
    \big(
      \Gamma \types N_e : \tyuser{Y}{(\sig,E')}
    \big)_{e \in E}
  }{
    \Gamma \types
    \tmtry{M}{
        \{ \tmreturn{x} \mapsto N,
           (\tmraise{e} \mapsto N_e)_{e \in E} \}
        }
    : \tyuser{Y}{(\sig,E')}
  }

  \coopinfer{TyUser-MatchPair}{
    \Gamma \types V : \typrod{X}{Y} \\
    \Gamma, x \of X, y \of Y \types M : \tyuser{Z}{\Ueff}
  }{
    \Gamma \types \tmmatch{V}{\tmpair{x}{y} \mapsto M} : \tyuser{Z}{\Ueff}
  }

  \coopinfer{TyUser-MatchEmpty}{
    \Gamma \types V : \tyempty
  }{
    \Gamma \types \tmmatch[Z]{V}{} : \tyuser{Z}{\Ueff}
  }

  \coopinfer{TyUser-MatchSum}{
    \Gamma \types V : X + Y \\
    \Gamma, x \of X \types M : \tyuser{Z}{\Ueff} \\
    \Gamma, y \of Y \types N : \tyuser{Z}{\Ueff} \\
  }{
    \Gamma \types \tmmatch{V}{\tminl{x} \mapsto M, \tminr{y} \mapsto N} : \tyuser{Z}{\Ueff}
  }

  %%% SELECTED RULE, KEEP IN SYNC WITH fig:typing-selected
  \coopinfer{TyUser-Op}{
    \Ueff \equiv (\sig,E) \\
    \op \in \sig \\
    \Gamma \types V : A_\op \\\\
    \Gamma, x \of B_\op \types M : \tyuser{X}{\Ueff} \\
    \big(
      \Gamma \vdash N_e : \tyuser{X}{\Ueff}
    \big)_{e \in E_\op}
  }{
    \Gamma \types \tmop{op}{X}{V}{\tmcont x M}{\tmexccont N e {E_\op}} : \tyuser{X}{\Ueff}
  }

  \coopinfer{TyUser-Raise}{
     e \in E
  }{
     \Gamma \types \tmraise[X]{e} : \tyuser{X}{(\sig, E)}
  }

  %%% SELECTED RULE, KEEP IN SYNC WITH fig:typing-selected
  \coopinfer{TyUser-Run}{
    F \equiv
    \{ \tmreturn{x} \at c \mapsto N,
       (\tmraise{e} \at c \mapsto N_e)_{e \in E},
       (\tmkill{s} \mapsto N_s)_{s \in S}
    \}
    \\\\
    \Gamma \types V : \tyrunner{\sig}{\sig'}{S}{C} \\
    \Gamma \types W : C \\\\
    \Gamma \types M : \tyuser{X}{(\sig, E)} \\
    \Gamma, x \of X, c \of C \types N : \tyuser{Y}{(\sig', E')} \\
    \big(
       \Gamma, c \of C \types N_e : \tyuser{Y}{(\sig', E')}
    \big)_{e \in E} \\
    \big(
       \Gamma \types N_s : \tyuser{Y}{(\sig', E')}
    \big)_{s \in S} \\
  }{
    \Gamma \types \tmrun{V}{W}{M}{F} : \tyuser{Y}{(\sig', E')}
  }

  %%% SELECTED RULE, KEEP IN SYNC WITH fig:typing-selected
  \coopinfer{TyUser-Kernel}{
    F \equiv
    \{ \tmreturn{x} \at c \mapsto N,
       (\tmraise{e} \at c \mapsto N_e)_{e \in E},
       (\tmkill{s} \mapsto N_s)_{s \in S}
    \}
    \\\\
    \Gamma \types K : \tykernel{X}{(\sig, E, S, C)} \\
    \Gamma \types W : C \\
    \Gamma, x \of X, c \of C \types N : \tyuser{Y}{(\sig, E')} \\
    \big(
      \Gamma, c \of C \types N_e : \tyuser{Y}{(\sig, E')}
    \big)_{e \in E} \\
    \big(
      \Gamma \types N_s : \tyuser{Y}{(\sig, E')}
    \big)_{s \in S} \\
  }{
    \Gamma \types \tmkernel{K}{W}{F} : \tyuser{Y}{(\sig, E')}
  }
\end{mathpar}
  \caption{User typing rules.}
  \label{fig:typing-user}
\end{figure}


%%%%%%%%%%%%%%%%% KERNEL RULES %%%%%%%%%%%%%%%%%%%%
\begin{figure}[tp]
  \centering
  \small
\begin{mathpar}

  \coopinfer{TyKernel-Return}{
    \Gamma \types V : X
  }{
    \Gamma \types \tmreturn[C]{V} : \tykernel{X}{(\sig, E, S, C)}
  }

  \coopinfer{TyKernel-Apply}{
    \Gamma \types V : \tyfun{X}{\tykernel{Y}{\Keff}} \\
    \Gamma \types W : X
  }{
    \Gamma \types \tmapp{V}{W} : \tykernel{Y}{\Keff}
  }

  \coopinfer{TyKernel-Try}{
    \Gamma \types K : \tykernel{X}{(\sig, E, S, C)}
    \\
    \Gamma, x \of X \types L : \tykernel{Y}{(\sig, E', S, C)}
    \\
    \big(
      \Gamma \types L_e : \tykernel{Y}{(\sig, E', S, C)}
    \big)_{e \in E}
  }{
    \Gamma \types
    \tmtry{K}{
        \{ \tmreturn{x} \mapsto L,
           (\tmraise{e} \mapsto L_e)_{e \in E} \}
        }
    : \tykernel{Y}{(\sig, E', S, C)}
  }

  \coopinfer{TyKernel-MatchPair}{
    \Gamma \types V : \typrod{X}{Y} \\
    \Gamma, x \of X, y \of Y \types K : \tykernel{Z}{\Keff}
  }{
    \Gamma \types \tmmatch{V}{\tmpair{x}{y} \mapsto K} : \tykernel{Z}{\Keff}
  }

  \coopinfer{TyKernel-MatchEmpty}{
    \Gamma \types V : \tyempty
  }{
    \Gamma \types \tmmatch[Z \at C]{V}{} : \tykernel{Z}{(\sig, E, S, C)}
  }

  \coopinfer{TyKernel-MatchSum}{
    \Gamma \types V : X + Y \\
    \Gamma, x \of X \types K : \tykernel{Z}{\Keff} \\
    \Gamma, y \of Y \types L : \tykernel{Z}{\Keff} \\
  }{
    \Gamma \types \tmmatch{V}{\tminl{x} \mapsto K, \tminr{y} \mapsto L} : \tykernel{Z}{\Keff}
  }

  %%% SELECTED RULE, KEEP IN SYNC WITH fig:typing-selected
  \coopinfer{TyKernel-Op}{
    \Keff \equiv (\sig, E, S, C) \\
    \op \in \sig \\
    \Gamma \types V : A_\op \\\\
    \Gamma, x \of B_\op \types K : \tykernel{X}{\Keff} \\
    \big(
      \Gamma \vdash L_e : \tykernel{X}{\Keff}
    \big)_{e \in E_\op}
  }{
    \Gamma \types \tmop{op}{X}{V}{\tmcont x K}{\tmexccont L e {E_\op}} : \tykernel{X}{\Keff}
  }

  \coopinfer{TyKernel-Raise}{
     e \in E
  }{
     \Gamma \types \tmraise[X \at C]{e} : \tykernel{X}{(\sig, E, S, C)}
  }

  \coopinfer{TyKernel-Kill}{
     s \in S
  }{
     \Gamma \types \tmkill[X \at C]{s} : \tykernel{X}{(\sig, E, S, C)}
  }

  \coopinfer{TyKernel-Getenv}{
    \Gamma, c \of C \types K : \tykernel{X}{(\sig, E, S, C)}
  }{
    \Gamma \types \tmgetenv[C]{\tmcont c K} : \tykernel{X}{(\sig, E, S, C)}
  }

  \coopinfer{TyKernel-Setenv}{
    \Gamma \types V : C \\
    \Gamma \types K : \tykernel{X}{(\sig, E, S, C)}
  }{
    \Gamma \types \tmsetenv{V}{K} : \tykernel{X}{(\sig, E, S, C)}
  }

  %%% SELECTED RULE, KEEP IN SYNC WITH fig:typing-selected
  \coopinfer{TyKernel-User}{
   \Keff \equiv (\sig, E', S, C) \\
   \Gamma \types M : \tyuser{X}{(\sig, E)} \\\\
   \Gamma, x \of X \types K : \tykernel{Y}{\Keff} \\
   \big(
     \Gamma \types L_e : \tykernel{Y}{\Keff}
   \big)_{e \in E}
  }{
    \Gamma \types
    \tmuser{M}{
      \{ \tmreturn{x} \mapsto K,
         (\tmraise{e} \mapsto L_e)_{e \in E}
      \}
    }
    : \tykernel{Y}{\Keff}
  }
\end{mathpar}
  \caption{Kernel typing rules.}
  \label{fig:typing-kernel}
\end{figure}


%%% Local Variables:
%%% mode: latex
%%% TeX-master: "runners-in-action"
%%% End:

% !TEX root = runners-in-action.tex

\section{Equational theory of $\lambdacoop$}
\label{sec:appendix-equational-rules}

Values, user and kernel computations each have an equality judgement
\begin{equation*}
\Gamma \types V \equiv W : X
\qquad
\Gamma \types M \equiv N : \tyuser{X}{\Ueff}
\qquad
\Gamma \types K \equiv L  : \tyuser{X}{\Keff}.
\end{equation*}
%
It is presupposed that we only compare well-typed expressions with the indicated types.
For the most part, the context and the type annotation will play no part in the equation,
and so we shall drop them when no confusion can arise.

The \emph{computational equations} are displayed in \cref{fig:computational-equations-user,fig:computational-equations-kernel}.
These can be read left-to-right as evaluation rules that explain the operational meaning
of computations. The remaining equations are displayed in \cref{fig:other-equations}.
%
We omit standard equations which specify how substitution is performed, as well
as equations stating that equality is a congruence with respect to all the term
formers.


%%%% User computation rules
\begin{figure}[tbp]
  \centering
  \parbox{\textwidth}{
  \small
  %
  \mathtoolsset{original-shortintertext=false,below-shortintertext-sep=0pt,above-shortintertext-sep=0pt}
  \begin{align*}
    \tmapp{(\tmfun{x \of X}{M})}{V} &\equiv M[V/x]
    \\[1ex]
    \tmtry{(\tmreturn{V})}{H} &\equiv N[V/x]
    \\
    \tmtry{(\tmraise[X]{e})}{H} &\equiv N_{e}
    \\
    \tmtry{(
      \tmop{op}{X}{V}{\tmcont x M}{\tmexccont {N'} {e'} {E_\op}}
    )}{H} &\equiv \\
    \tmop{op}{X}{V}{\tmcont x {\tmtry{&M}{H}}}{\left(\tmtry{N'_{e'}}{H}\right)_{e' \in E_\op}}
    \\[1ex]
    \tmmatch{\tmpair{V}{W}}{\tmpair{x}{y} \mapsto M} &\equiv
    M[V/x, W/y]
    \\
    \tmmatch[X]{V}{} &\equiv
    N
    \\
    \mathllap{
      \tmmatch{(\tminl[X,Y]{V})}{\tminl{x} \mapsto M, \tminr{y} \mapsto N} 
    } &\equiv
    M[V/x]
    \\
    \mathllap{
      \tmmatch{(\tminr[X,Y]{W})}{\tminl{x} \mapsto M, \tminr{y} \mapsto N}
    } &\equiv
    N[W/y]
    \\[1ex]
    \tmrun{V}{W}{(\tmreturn{V'})}{F} &\equiv N[V'/x, W/c]
    \\
    \tmrun{V}{W}{(\tmraise[X]{e})}{F} &\equiv N_{e}[W/c]
    \\
    \omit\rlap{$
    \tmkw{using}\; R \at W \;\tmkw{run}\;
      \tmop{op}{X}{V}{\tmcont x M}{\tmexccont {N'} {e'} {E_\op}}
      \;\tmkw{finally}\; F \equiv $}
    \\
    &
    \quad\tmkernel{K_\op[V/x]}{W}{F'}
    \\
    \shortintertext{\hfil \parbox{0.6\textwidth}{
    \abovedisplayskip=0pt
    \belowdisplayskip=0pt
    \begin{equation*}
      \text{where\ } F' \defeq 
      \begin{aligned}[t]
        \{
        &\tmreturn{x} \at c' \mapsto (\tmrun{R}{c'}{M}{F}), \\
        &\left(
            \tmraise{e'} \at c' \mapsto (\tmrun{R}{c'}{N'_{e'}}{F})
         \right)_{e' \in E_\op},\\
        &\left(
            \tmkill{s} \mapsto N_s
          \right)_{s \in S}
      \}
      \end{aligned}
    \end{equation*}
    }}
  \\
    \tmkernel{(\tmreturn[C]{V})}{W}{F} &\equiv N[V/x, W/c] \\
    \tmkernel{(\tmraise[X \at C]{e})}{W}{F} &\equiv N_{e}[W/c] \\
    \tmkernel{(\tmkill[X \at C]{s})}{W}{F} &\equiv N_{s} \\
    \tmkernel{(\tmgetenv[C]{\tmcont c K})}{W}{F} &\equiv \tmkernel{K[W/c]}{W}{F} \\
    \tmkernel{(\tmsetenv{V}{K})}{W}{F} &\equiv \tmkernel{K}{V}{F}
    \\
    \omit\rlap{$
       \tmkernel{\tmop{op}{X}{V}{\tmcont x K}{\tmexccont L {e'} {E_\op}}}{W}{F}
      \equiv $}
    \\
      \tmop{op}{X}{V}{\tmcont x {\tmkernel{K}{W}{&F}}}{
      \left(
         \tmkernel{L_{e'}}{W}{F}
      \right)_{e' \in E_\op}}
  \end{align*}
  %
  Abbreviations:
  %
  \begin{align*}
    F &\defeq
       \{ \tmreturn{x} \at c \mapsto N,
       (\tmraise{e} \at c \mapsto N_e)_{e \in E},
       (\tmkill{s} \mapsto N_s)_{s \in S}
       \}
    \\
    H &\defeq
       \{ \tmreturn{x} \mapsto N,
          (\tmraise{e} \mapsto N_e)_{e \in E}
       \}
    \\
    R &\defeq \tmrunner{(\tm{op}\,x \mapsto K_{\tm{op}})_{\tm{op} \in \sig}}{C}
  \end{align*}
  } % parbox
  \caption{Computational equations (user mode).}
  \label{fig:computational-equations-user}
\end{figure}

%%%% Kernel computation rules
\begin{figure}[tb]
  \centering
  \parbox{\textwidth}{
  \small
  %
  \begin{align*}
    \tmapp{(\tmfunK{x \of X}{K})}{V} &\equiv K[V/x]
    \\[1ex]
    \tmtry{(\tmreturn{V})}{G} &\equiv L[V/x]
    \\
    \tmtry{(\tmraise[X \at C]{e})}{G} &\equiv L_{e}
    \\
    \tmtry{(\tmkill[X \at C]{s})}{G} &\equiv \tmkill[X \at C]{s}
    \\
    \tmtry{(
      \tmop{op}{X}{V}{\tmcont x K}{\tmexccont {L'} {e'} {E_\op}}
    )}{G} &\equiv \\
    \tmop{op}{X}{V}{\tmcont x {\tmtry{&K}{G}}}{\left(\tmtry{L'_{e'}}{G}\right)_{e' \in E_\op}}
    \\
    \tmtry{(\tmgetenv[C]{\tmcont c K})}{G} &\equiv \tmgetenv[C]{\tmcont c {\tmtry{K}{G}}}
    \\
    \tmtry{(\tmsetenv{V}{K})}{G} &\equiv \tmsetenv{V}{\tmtry{K}{G}}
    \\[1ex]
    \tmmatch{\tmpair{V}{W}}{\tmpair{x}{y} \mapsto K} &\equiv
    K[V/x, W/y]
    \\
    \tmmatch[X \at C]{V}{} &\equiv
    K
    \\
    \tmmatch{(\tminl[X,Y]{V})}{\tminl{x} \mapsto K, \tminr{y} \mapsto L} &\equiv
    K[V/x]
    \\
    \tmmatch{(\tminr[X,Y]{W})}{\tminl{x} \mapsto K, \tminr{y} \mapsto L} &\equiv
    L[W/y]
    \\[1ex]
    \tmuser{(\tmreturn{V})}{G} &\equiv L[V/x]
    \\
    \tmuser{(\tmraise[X]{e})}{G} &\equiv L_e
    \\
    \tmuser{(
      \tmop{op}{X}{V}{\tmcont x M}{\tmexccont {N'} {e'} {E_\op}}
    )}{G} &\equiv \\
    \tmop{op}{X}{V}{\tmcont x {\tmuser{&M}{G}}}{\left(\tmuser{N'_{e'}}{G} \right)_{e' \in E_\op}}
  \end{align*}
  %
  Abbreviation: $G \defeq \{ \tmreturn{x} \mapsto L, (\tmraise{e} \mapsto L_e)_{e \in E} \}$
  } % parbox
  \caption{Computational equations (kernel mode).}
  \label{fig:computational-equations-kernel}
\end{figure}


\begin{figure}[tb]
  \centering
  \parbox{\textwidth}{
  \small
  \begin{gather*}
    V \equiv \tmunit : \tyunit \qquad\qquad
    \tmfun{x \of A}{\tmapp{V}{x}} \equiv V \qquad\qquad
    \tmfunK{x \of A}{\tmapp{V}{x}} \equiv V
    \\
    \tmtry{M}{
       \{ \tmreturn{x} \mapsto \tmreturn{x},
          (\tmraise{e} \mapsto \tmraise[X]{e})_{e \in E}
       \}
    }
    \equiv M
    \\
    \tmtry{K}{
       \{ \tmreturn{x} \mapsto \tmreturn{x},
          (\tmraise{e} \mapsto \tmraise[X \at C]{e})_{e \in E}
       \}
    }
    \equiv K
    \\[1ex]
  \begin{aligned}
  \tmgetenv[C]{\tmcont c {\tmsetenv{c}{K}}} &\equiv K
  \\
  \tmsetenv{V}{\tmgetenv[C]{\tmcont c K}} &\equiv
  \tmsetenv{V}{K[V/c]}
  \\
  \tmsetenv{V}{\tmsetenv{W}{K}} &\equiv \tmsetenv{W}{K}
  \\
  \tmgetenv[C]{\tmcont c {\tmkill[X \at C]{s}}} &\equiv
  \tmkill[X \at C]{s}
  \\
  \tmsetenv{V}{\tmkill[X \at C]{s}} &\equiv
  \tmkill[X \at C]{s}
  \\
  \tmgetenv[C]{\tmcont c {\tmop{op}{X}{V}{\tmcont x K}{\tmexccont L e {E_\op}}}}
  &\equiv \\
  \tmop{op}{X}{V}{\tmcont x {\tmgetenv[C]{&\tmcont c K}}}{\left(\tmgetenv[C]{\tmcont c {L_e}}\right)_{e \in E_\op}}
  \\
  \tmsetenv{V}{\tmop{op}{X}{V}{\tmcont x K}{\tmexccont L e {E_\op}}}
  &\equiv \\
  \tmop{op}{X}{&V}{\tmcont x {\tmsetenv{V}{K}}}{\left(\tmsetenv{V}{L_e}\right)_{e \in E_\op}}
  \end{aligned}
  \end{gather*}
  } % parbox
  \caption{Other equations (for $\eta$-expansion and the kernel theory from \cref{sec:user-kernel-monads}).}
  \label{fig:other-equations}
\end{figure}

%%% Local Variables:
%%% mode: latex
%%% TeX-master: "runners-in-action"
%%% End:


\end{document}


