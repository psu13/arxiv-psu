% !TEX root = runners-in-action.tex


\newcommand{\tmcoop}[3]{\mathtt{\overline{#1}}~#2 \mapsto #3}

\section{A calculus for programming with runners}
\label{sect:corecalculus}

Inspired by the semantic notion of runners and the ideas of the previous
section, we now present a calculus for programming with co-operations and
runners, called $\lambdacoop$. It is a low-level fine-grain call-by-value
calculus~\cite{Levy:CBPV}, and as such could inspire an intermediate language
that a high-level language is compiled to.

\subsection{Types}
\label{sect:types}

\begin{figure}[tb]
  \parbox{\textwidth}{
  \centering
  \small
  \begin{align*}
  \text{Ground type $A$, $B$, $C$}
  \bnfis& \tybase      & &\text{base type} \\
  \bnfor& \tyunit       & &\text{unit type} \\
  \bnfor& \tyempty      & &\text{empty type} \\
  \bnfor& \typrod{A}{B} & &\text{product type} \\
  \bnfor& \tysum{A}{B}  & &\text{sum type}
  \\[1ex]
  \text{Constant signature:}
  \phantom{\bnfis}& \tmconst{f} : (A_1,\ldots,A_n) \to B
  \\[1ex]
  \text{Signature $\sig$}
  \bnfis& \{\op_1, \op_2, \ldots, \op_n\} \subset \Ops
  \\
  \text{Exception set $E$}
  \bnfis& \{e_1, e_2, \ldots, e_n\} \subset \Excs
  \\
  \text{Signal set $S$}
  \bnfis& \{s_1, s_2, \ldots, s_n\} \subset \Sigs
  \\[1ex]
  \text{Operation signature:}
  \phantom{\bnfis}& \op : \tysigop{A_\op}{B_\op}{E_\op}
  \\[1ex]
  \text{Value type $X$, $Y$, $Z$}
  \bnfis& A                                       & &\text{ground type} \\
  \bnfor& \typrod{X}{Y}                           & &\text{product type} \\
  \bnfor& \tysum{X}{Y}                            & &\text{sum type} \\
  \bnfor& \tyfun{X}{\tyuser{Y}{\Ueff}}           & &\text{user function type} \\
  \bnfor& \tyfunK{X}{\tykernel{Y}{\Keff}}         & &\text{kernel function type} \\
  \bnfor& \tyrunner{\sig}{\sig'}{S}{C}      & &\text{runner type}
  \\[1ex]
  \text{User (computation) type:}
  \phantom{\bnfor} &\tyuser{X}{\Ueff} \quad \text{where $\Ueff = (\sig, E)$}
  \\
  \text{Kernel (computation) type:}
  \phantom{\bnfor}& \tykernel{X}{\Keff} \quad \text{where $\Keff = (\sig, E, S, C)$}
  \end{align*}
  } 
  \caption{The types of $\lambdacoop$.}
  \label{fig:lambdacoop-types}
\end{figure}

The types of $\lambdacoop$ are shown in \cref{fig:lambdacoop-types}.
%
The \emph{ground types} contain \emph{base types}, and are closed under finite sums and
products. These are used in operation signatures and as types of kernel state. (Allowing
arbitrary types in either of these entails substantial complications that can be dealt
with but are tangential to our goals.) Ground types can also come with corresponding 
constant symbols~$\tmconst{f}$, each associated with a fixed \emph{constant signature}
$\tmconst{f} : (A_1,\ldots,A_n) \to B$.

We assume a supply of operation symbols $\Ops$, exception names $\Excs$, and
signal names $\Sigs$. Each operation symbol~$\op \in \Ops$ is equipped with an
\emph{operation signature} $\tysigop{A_\op}{B_\op}{E_\op}$, which specifies its
parameter type~$A_\op$ and arity type~$B_\op$, and the exceptions~$E_\op$ 
that the corresponding co-operations may raise in runners.

The \emph{value types} extend ground types with two 
function types, and a type of runners.
%
The \emph{user function type $\tyfun{X}{\tyuser{Y}{(\sig, E)}}$} classifies
functions taking arguments of type~$X$ to computations classified by the \emph{user
  (computation) type}~${\tyuser{Y}{(\sig, E)}}$, i.e., those that return values of
type~$Y$, and may call operations~$\sig$ and raise exceptions~$E$.
%
Similarly, the \emph{kernel function type
  $\tyfunK{X}{\tykernel{Y}{(\sig, E, S, C)}}$} classifies functions taking
arguments of type~$X$ to computations classified by the \emph{kernel
  (computation) type~$\tykernel{Y}{(\sig, E, S, C)}$}, i.e., those that return
values of type~$Y$, and may call operations~$\sig$, raise exceptions~$E$, send
signals~$S$, and use state of type~$C$. We note that the ingredients for user and kernel types
correspond precisely to the parameters of the user monad $\UU{\sig}{E}$ and the
kernel monad $\KK{C}{\sig}{E}{S}$ from \cref{sec:user-kernel-monads}.
%
Finally, the \emph{runner type $\tyrunner{\sig}{\sig'}{S}{C}$} classifies runners that
implement co-operations for the operations~$\sig$ as kernel computations which use 
operations~$\sig'$, send signals~$S$, and use state of type~$C$.


\subsection{Values and computations}
\label{sec:values-computations}


\begin{figure}[tp]
  \parbox{\textwidth}{
  \centering
  \small
  \abovedisplayskip=0pt
  \begin{align*}
  \intertext{\textbf{Values}}
  V, W
  \bnfis& x                                       & &\text{variable} \\
  \bnfor& \tmconst{f}(V_1,\ldots,V_n)                                       & &\text{ground constant} \\
  \bnfor& \tmunit                                 & &\text{unit} \\
  \bnfor& \tmpair{V}{W}                           & &\text{pair} \\
  \bnfor& \tminl[X,Y]{V} \bnfor \tminr[X,Y]{V}    & &\text{injection} \\
  \bnfor& \tmfun{x : X}{M}                        & &\text{user function} \\
  \bnfor& \tmfunK{x : X}{K}                       & &\text{kernel function} \\
  \bnfor& \tmrunner{(\tm{op}\,x \mapsto K_{\tm{op}})_{\tm{op} \in \sig}}{C}
                                                  & &\text{runner}
  \\[1ex]
  \intertext{\textbf{User computations}}
  M, N
  \bnfis& \tmreturn{V}                            & &\text{value} \\
  \bnfor& V\,W                                    & &\text{application} \\
  \bnfor& \tmtry{M}{
          \{ \tmreturn{x} \mapsto N,
             (\tmraise{e} \mapsto N_e)_{e \in E} \}
          }
                                                  & &\text{exception handler} \\
  \bnfor& \tmmatch{V}{\tmpair{x}{y} \mapsto M}    & &\text{product elimination} \\
  \bnfor& \tmmatch[X]{V}{}                        & &\text{empty elimination} \\
  \bnfor& \tmmatch{V}{\tminl{x} \mapsto M, \tminr{y} \mapsto N}
                                                  & &\text{sum elimination} \\
  \bnfor& \tmop{op}{X}{V}{\tmcont x M}{\tmexccont N e {E_\op}}
                                                  & &\text{operation call} \\
  \bnfor& \tmraise[X]{e}                          & &\text{raise exception} \\
  \bnfor& \tmrun{V}{W}{M}{F} 
                                                  & &\text{running user code} \\
  \bnfor& 
            \tmkernel{K}{W}{F}
                                                  & &\text{switch to kernel mode}
  \\[2ex]
  F \bnfis & \omit \rlap{$\{ \tmreturn{x} \at c \mapsto N, 
                    (\tmraise{e} \at c \mapsto N_e)_{e \in E},
                    (\tmkill{s} \mapsto N_s)_{s \in S} \}$}
  \\[1ex]
  \intertext{\textbf{Kernel computations}}
  K, L
  \bnfis& \tmreturn[C]{V}                         & &\text{value} \\
  \bnfor& V\,W                                    & &\text{application} \\
  \bnfor& \tmtry{K}{
          \{ \tmreturn{x} \mapsto L,
             (\tmraise{e} \mapsto L_e)_{e \in E} \}
          }
                                                  & &\text{exception handler} \\
  \bnfor& \tmmatch{V}{\tmpair{x}{y} \mapsto K}    & &\text{product elimination} \\
  \bnfor& \tmmatch[X \at C]{V}{}                  & &\text{empty elimination} \\
  \bnfor& \tmmatch{V}{\tminl{x} \mapsto K, \tminr{y} \mapsto L}
                                                  & &\text{sum elimination} \\
  \bnfor& \tmop{op}{X}{V}{\tmcont x K}{\tmexccont L e {E_\op}}
                                                  & &\text{operation call} \\
  \bnfor& \tmraise[X \at C]{e}                    & &\text{raise exception} \\
  \bnfor& \tmkill[X \at C]{s}                     & &\text{send signal} \\
  \bnfor& \tmgetenv[C]{\tmcont c K}               & &\text{get kernel state} \\
  \bnfor& \tmsetenv{V}{K}                         & &\text{set kernel state} \\
  \bnfor& \tmuser{M}{
          \begin{aligned}[t]
          \{ &\tmreturn{x} \mapsto K,
             (\tmraise{e} \mapsto L_e)_{e \in E} \}
          \end{aligned}
          }
                                                  & &\text{switch to user mode}
  \end{align*}
  } 
  \caption{Values, user computations, and kernel computations of $\lambdacoop$.}
  \label{fig:lambdacoop-terms}
\end{figure}

The syntax of terms is shown in \cref{fig:lambdacoop-terms}. The
usual fine-grain call-by-value stratification of terms into pure values and effectful
computations is present, except that we further distinguish between \emph{user} and
\emph{kernel} computations.

\subsubsection{Values}
\label{sec:values}

Among the values are variables, constants for ground types, and constructors 
for sums and products. There are two kinds of functions, for abstracting over user and
kernel computations. A \emph{runner} is a value of the form
%
\begin{equation*}
  \tmrunner{(\tm{op}\,x \mapsto K_{\tm{op}})_{\tm{op} \in \sig}}{C}.
\end{equation*}
%
It implements co-operations for operations $\tm{\op}$ as kernel
computations~$K_\op$, with $x$ bound in~$K_\op$. The type annotation~$C$
specifies the type of the state that~$K_\op$ uses.
Note that $C$ ranges over ground types, a restriction that allows us to define a naive
set-theoretic semantics.
%
We sometimes omit these type annotations.

\subsubsection{User and kernel computations}

The user and kernel computations both have pure computations, function application,
exception raising and handling, standard elimination forms, and operation calls.
Note that the typing annotations on some of these differ according to their mode.
For instance, a user operation call is annotated
with the result type~$X$, whereas the annotation $X \at C$ on a kernel operation call
also specifies the kernel state type~$C$.

The binding construct $\tmlet[X ! E]{x}{M}{N}$ is not part of the syntax,
but is an abbreviation for
%
$
  \tmtry{M}{
  \{ \tmreturn{x} \mapsto N,
     (\tmraise{e} \mapsto \tmraise[X]{e})_{e \in E}
   \}}
$,
%
and there is an analogous one for kernel computations. We often drop the 
annotation $X ! E$.

Some computations are specific to one or the other mode. Only the kernel mode
may send a signal with $\tmkill{\!}$, and manipulate state with
$\tmkw{getenv}$ and $\tmkw{setenv}$, but only the user mode has the 
$\tmkw{run}$ construct from \cref{sec:runn-as-progr}.
%
Finally, each mode has the ability to ``context switch'' to the other one.
The kernel computation
%
\begin{equation*}
\tmuser{M}{
   \{
     \tmreturn{x} \mapsto K,
     (\tmraise{e} \mapsto L_e)_{e \in E}
   \}
}
\end{equation*}
%
runs a user computation $M$ and handles the returned value and leftover
exceptions with kernel computations $K$ and $L_e$.
Conversely, the user computation
%
\begin{equation*}
\tmkernel{K}{W}{
  \{x \at c \mapsto M,
    (\tmraise{e} \at c \mapsto N_e)_{e \in E},
    (\tmkill{s} \mapsto N_s)_{s \in S}
  \}
}
\end{equation*}
%
runs kernel computation $K$ with initial state $W$, and handles the returned value,
and leftover exceptions and signals with user computations $M$, $N_e$, and $N_s$.

\subsection{Type system}
\label{sec:typesystem}

We equip $\lambdacoop$ with a type system akin to type and effect systems for
algebraic effects and handlers~\cite{Bauer:EffectSystem,Benton:ExceptionalSyntax,Kammar:Handlers}.
We are experimenting with resource control, so it makes sense for the type system
to tightly control resources. Consequently, our effect system
does not allow effects to be implicitly propagated outwards.

In \cref{sect:types}, we assumed that each operation $\op \in \Ops$ 
is equipped with some fixed operation signature
%
$
  \op : \tysigop{A_\op}{B_\op}{E_\op}
$.
%
We also assumed a fixed constant signature $\tmconst{f} : (A_1, \ldots, A_n) \to B$
for each ground constant $\tmconst{f}$.
%
We consider this information to be part of the type system and say no more about it.

Values, user computations, and kernel computations each have a corresponding
\emph{typing judgement} form and a \emph{subtyping relation}, given by 
%
\begin{align*}
  &\Gamma \types V : X,
& &\Gamma \types M : \tyuser{X}{\Ueff},
& &\Gamma \types K : \tykernel{X}{\Keff},\\
  &X \sub Y,
& &\tyuser{X}{\Ueff} \sub \tyuser{Y}{\Veff},
& &\tykernel{X}{\Keff} \sub \tykernel{Y}{\Leff}, 
\end{align*}
%
where $\Gamma$ is a \emph{typing context} $x_1 : X_1, \ldots, x_n : X_n$.
%
The effect information is an over-approximation, i.e., $M$ and $K$ employ \emph{at
  most} the effects described by $\Ueff$ and $\Keff$.
%
The complete rules for these judgements are given in the online appendix. %\cref{sec:typing-rules}. 
We comment here
only on the rules that are peculiar to~$\lambdacoop$, see \cref{fig:typing-selected}.

\begin{figure}[tp]
  \centering
  \small
  \begin{mathpar}
    %%% NOTE: these should be kept in sync with the rules given in
    %%%       tpying-rules.tex. The rules in that file are the official version,
    %%%       the ones here are the copies. If you delete a rule here, make sure
    %%%       to note so in typing-rules.tex.
    %%%
    %%%       Do not place a rule here without commenting on it in the text,
    %%%       that's bad manners.
    \coopinfer{Sub-Ground}{ }{A \sub A}
    
    \coopinfer{Sub-Runner}{
      \sig_1' \subseteq \sig_1 \\
      \sig_2 \subseteq \sig_2' \\
      S \subseteq S' \\
      C \equiv C'
    }{
      \tyrunner{\sig_1}{\sig_2}{S}{C} \sub \tyrunner{\sig_1'}{\sig_2'}{S'}{C'}
    }

    \coopinfer{Sub-Kernel}{
      X \sub X' \\
      \sig \subseteq \sig' \\
      E \subseteq E' \\
      S \subseteq S' \\
      C \equiv C'
    }{
      \tykernel{X}{(\sig, E, S, C)} \sub \tykernel{X'}{(\sig', E', S', C')}
    }

  \coopinfer{TyUser-Try}{
    \Gamma \types M : \tyuser{X}{(\sig,E)}
    \\
    \Gamma, x \of X \types N : \tyuser{Y}{(\sig,E')}
    \\
    \big(
      \Gamma \types N_e : \tyuser{Y}{(\sig,E')}
    \big)_{e \in E}
  }{
    \Gamma \types
    \tmtry{M}{
        \{ \tmreturn{x} \mapsto N,
           (\tmraise{e} \mapsto N_e)_{e \in E} \}
        }
    : \tyuser{Y}{(\sig,E')}
  }

  \coopinfer{TyUser-Run}{
    F \equiv
    \{ \tmreturn{x} \at c \mapsto N,
       (\tmraise{e} \at c \mapsto N_e)_{e \in E},
       (\tmkill{s} \mapsto N_s)_{s \in S}
    \}
    \\\\
    \Gamma \types V : \tyrunner{\sig}{\sig'}{S}{C} \\
    \Gamma \types W : C \\\\
    \Gamma \types M : \tyuser{X}{(\sig, E)} \\
    \Gamma, x \of X, c \of C \types N : \tyuser{Y}{(\sig', E')} \\
    \big(
       \Gamma, c \of C \types N_e : \tyuser{Y}{(\sig', E')}
    \big)_{e \in E} \\
    \big(
       \Gamma \types N_s : \tyuser{Y}{(\sig', E')}
    \big)_{s \in S} \\
  }{
    \Gamma \types \tmrun{V}{W}{M}{F} : \tyuser{Y}{(\sig', E')}
  }

  \coopinfer{TyUser-Op}{
    \Ueff \equiv (\sig,E) \\
    \op \in \sig \\
    \Gamma \types V : A_\op \\\\
    \Gamma, x \of B_\op \types M : \tyuser{X}{\Ueff} \\
    \big(
      \Gamma \vdash N_e : \tyuser{X}{\Ueff}
    \big)_{e \in E_\op}
  }{
    \Gamma \types \tmop{op}{X}{V}{\tmcont x M}{\tmexccont N e {E_\op}} : \tyuser{X}{\Ueff}
  }

  \coopinfer{TyKernel-Op}{
    \Keff \equiv (\sig, E, S, C) \\
    \op \in \sig \\
    \Gamma \types V : A_\op \\\\
    \Gamma, x \of B_\op \types K : \tykernel{X}{\Keff} \\
    \big(
      \Gamma \vdash L_e : \tykernel{X}{\Keff}
    \big)_{e \in E_\op}
  }{
    \Gamma \types \tmop{op}{X}{V}{\tmcont x K}{\tmexccont L e {E_\op}} : \tykernel{X}{\Keff}
  }

  \coopinfer{TyUser-Kernel}{
    F \equiv
    \{ \tmreturn{x} \at c \mapsto N,
       (\tmraise{e} \at c \mapsto N_e)_{e \in E},
       (\tmkill{s} \mapsto N_s)_{s \in S}
    \}
    \\\\
    \Gamma \types K : \tykernel{X}{(\sig, E, S, C)} \\
    \Gamma \types W : C \\
    \Gamma, x \of X, c \of C \types N : \tyuser{Y}{(\sig, E')} \\
    \big(
      \Gamma, c \of C \types N_e : \tyuser{Y}{(\sig, E')}
    \big)_{e \in E} \\
    \big(
      \Gamma \types N_s : \tyuser{Y}{(\sig, E')}
    \big)_{s \in S} \\
  }{
    \Gamma \types \tmkernel{K}{W}{F} : \tyuser{Y}{(\sig, E')}
  }

  \coopinfer{TyKernel-User}{
   \Keff \equiv (\sig, E', S, C) \\
   \Gamma \types M : \tyuser{X}{(\sig, E)} \\\\
   \Gamma, x \of X \types K : \tykernel{Y}{\Keff} \\
   \big(
     \Gamma \types L_e : \tykernel{Y}{\Keff}
   \big)_{e \in E}
  }{
    \Gamma \types
    \tmuser{M}{
      \{ \tmreturn{x} \mapsto K,
         (\tmraise{e} \mapsto L_e)_{e \in E}
      \}
    }
    : \tykernel{Y}{\Keff}
  }
  \end{mathpar}
  \caption{Selected typing and subtyping rules.}
  \label{fig:typing-selected}
\end{figure}

Subtyping of ground types \textsc{Sub-Ground} is trivial, as it relates only equal types.
Subtyping of runners \textsc{Sub-Runner} and kernel computations
\textsc{Sub-Kernel} requires equality of the kernel state types~$C$ and~$C'$
because state is used invariantly in the kernel monad.
We leave it for future work to replace ${C \equiv C'}$ with a
\emph{lens}~\cite{Foster:Lenses} from~$C'$ to~$C$, i.e., maps $C' \to C$ and ${C' \times C \to C'}$
satisfying state equations analogous to \cref{ex:state}. It has been
observed~\cite{OConnor:Lens,Power:Comodels} that such a lens in fact amounts to
an ordinary runner for $C$-valued state.

The rules \textsc{TyUser-Op} and \textsc{TyKernel-Op} govern operation calls, where 
we have a success continuation which receives a value returned by a 
co-operation, and exceptional continuations which receive exceptions raised by co-operations.

The rule \textsc{TyUser-Run} requires that the runner $V$ implements \emph{all} the
operations $M$ can use, meaning that operations are \emph{not} implicitly propagated outside a $\tmkw{run}$ block (which is different from how handlers are sometimes implemented). Of course, the co-operations of the runner may call further external operations, as recorded by the signature~$\sig'$. Similarly, we require the finally block~$F$ to intercept all exceptions and signals that might be produced by the co-operations of $V$ or the user code $M$.
%
Such strict control is exercised throughout. For example, in 
\textsc{TyUser-Run}, \textsc{TyUser-Kernel}, and \textsc{TyKernel-User} we catch all the exceptions and signals that the code might produce.
%
One should judiciously relax these requirements in a language that is presented to
the programmer, and allow re-raising and re-sending clauses to be automatically inserted.

%%% Local Variables:
%%% mode: latex
%%% TeX-master: "runners-in-action"
%%% End:
