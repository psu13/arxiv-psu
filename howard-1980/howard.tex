\documentclass[12pt]{article}

%\usepackage{amsmath,amsthm,amscd,amssymb}
\usepackage[colorlinks=true
,breaklinks=true
,urlcolor=blue
,anchorcolor=blue
,citecolor=blue
,filecolor=blue
,linkcolor=blue
,menucolor=blue
,linktocpage=true]{hyperref}
\hypersetup{
bookmarksopen=true,
bookmarksnumbered=true,
bookmarksopenlevel=10
}
\usepackage[noBBpl,sc]{mathpazo}
%\usepackage[full]{textcomp}
%\usepackage[osf]{XCharter}% lining figures in math, osf in text
%\usepackage[scaled=1.04,varqu,varl]{inconsolata}% inconsolata typewriter
\usepackage[bb=pazo,cal=cm,scr=esstix]{mathalpha}
%\usepackage{eulervm}

% make a ||- symbol
% the right amount to kern backwards is font dependent here. so it's here instead of
% in the main file.
\def\mm{\mathrel{||}\mathrel{\mkern-4.4mu}\relbar}

\usepackage[papersize={6.8in, 9.0in}, left=.5in, right=.5in, top=1in, bottom=.9in]{geometry}
\linespread{1.05}
\sloppy
\raggedbottom
\pagestyle{plain}
%\usepackage{eulervm}
\usepackage{enumitem}
\usepackage{mathpartir}
\usepackage{stmaryrd}
\usepackage{mathtools}
\usepackage{tikz-cd}
\usepackage{microtype}
\usepackage{amssymb}
%\usepackage{fdsymbol}

% these include amsmath and that can cause trouble in older docs.
\input{../helpers/cmrsum}
\makeatletter

\DeclareFontFamily{OMX}{MnSymbolE}{}
\DeclareSymbolFont{largesymbolsX}{OMX}{MnSymbolE}{m}{n}
\DeclareFontShape{OMX}{MnSymbolE}{m}{n}{
    <-6>  MnSymbolE5
   <6-7>  MnSymbolE6
   <7-8>  MnSymbolE7
   <8-9>  MnSymbolE8
   <9-10> MnSymbolE9
  <10-12> MnSymbolE10
  <12->   MnSymbolE12}{}

\DeclareMathSymbol{\downbrace}    {\mathord}{largesymbolsX}{'251}
\DeclareMathSymbol{\downbraceg}   {\mathord}{largesymbolsX}{'252}
\DeclareMathSymbol{\downbracegg}  {\mathord}{largesymbolsX}{'253}
\DeclareMathSymbol{\downbraceggg} {\mathord}{largesymbolsX}{'254}
\DeclareMathSymbol{\downbracegggg}{\mathord}{largesymbolsX}{'255}
\DeclareMathSymbol{\upbrace}      {\mathord}{largesymbolsX}{'256}
\DeclareMathSymbol{\upbraceg}     {\mathord}{largesymbolsX}{'257}
\DeclareMathSymbol{\upbracegg}    {\mathord}{largesymbolsX}{'260}
\DeclareMathSymbol{\upbraceggg}   {\mathord}{largesymbolsX}{'261}
\DeclareMathSymbol{\upbracegggg}  {\mathord}{largesymbolsX}{'262}
\DeclareMathSymbol{\braceld}      {\mathord}{largesymbolsX}{'263}
\DeclareMathSymbol{\bracelu}      {\mathord}{largesymbolsX}{'264}
\DeclareMathSymbol{\bracerd}      {\mathord}{largesymbolsX}{'265}
\DeclareMathSymbol{\braceru}      {\mathord}{largesymbolsX}{'266}
\DeclareMathSymbol{\bracemd}      {\mathord}{largesymbolsX}{'267}
\DeclareMathSymbol{\bracemu}      {\mathord}{largesymbolsX}{'270}
\DeclareMathSymbol{\bracemid}     {\mathord}{largesymbolsX}{'271}

\def\horiz@expandable#1#2#3#4#5#6#7#8{%
  \@mathmeasure\z@#7{#8}%
  \@tempdima=\wd\z@
  \@mathmeasure\z@#7{#1}%
  \ifdim\noexpand\wd\z@>\@tempdima
    $\m@th#7#1$%
  \else
    \@mathmeasure\z@#7{#2}%
    \ifdim\noexpand\wd\z@>\@tempdima
      $\m@th#7#2$%
    \else
      \@mathmeasure\z@#7{#3}%
      \ifdim\noexpand\wd\z@>\@tempdima
        $\m@th#7#3$%
      \else
        \@mathmeasure\z@#7{#4}%
        \ifdim\noexpand\wd\z@>\@tempdima
          $\m@th#7#4$%
        \else
          \@mathmeasure\z@#7{#5}%
          \ifdim\noexpand\wd\z@>\@tempdima
            $\m@th#7#5$%
          \else
           #6#7%
          \fi
        \fi
      \fi
    \fi
  \fi}

\def\overbrace@expandable#1#2#3{\vbox{\m@th\ialign{##\crcr
  #1#2{#3}\crcr\noalign{\kern2\p@\nointerlineskip}%
  $\m@th\hfil#2#3\hfil$\crcr}}}
\def\underbrace@expandable#1#2#3{\vtop{\m@th\ialign{##\crcr
  $\m@th\hfil#2#3\hfil$\crcr
  \noalign{\kern2\p@\nointerlineskip}%
  #1#2{#3}\crcr}}}

\def\overbrace@#1#2#3{\vbox{\m@th\ialign{##\crcr
  #1#2\crcr\noalign{\kern2\p@\nointerlineskip}%
  $\m@th\hfil#2#3\hfil$\crcr}}}
\def\underbrace@#1#2#3{\vtop{\m@th\ialign{##\crcr
  $\m@th\hfil#2#3\hfil$\crcr
  \noalign{\kern2\p@\nointerlineskip}%
  #1#2\crcr}}}

\def\bracefill@#1#2#3#4#5{$\m@th#5#1\leaders\hbox{$#4$}\hfill#2\leaders\hbox{$#4$}\hfill#3$}

\def\downbracefill@{\bracefill@\braceld\bracemd\bracerd\bracemid}
\def\upbracefill@{\bracefill@\bracelu\bracemu\braceru\bracemid}

\DeclareRobustCommand{\downbracefill}{\downbracefill@\textstyle}
\DeclareRobustCommand{\upbracefill}{\upbracefill@\textstyle}

\def\upbrace@expandable{%
  \horiz@expandable
    \upbrace
    \upbraceg
    \upbracegg
    \upbraceggg
    \upbracegggg
    \upbracefill@}
\def\downbrace@expandable{%
  \horiz@expandable
    \downbrace
    \downbraceg
    \downbracegg
    \downbraceggg
    \downbracegggg
    \downbracefill@}

\DeclareRobustCommand{\overbrace}[1]{\mathop{\mathpalette{\overbrace@expandable\downbrace@expandable}{#1}}\limits}
\DeclareRobustCommand{\underbrace}[1]{\mathop{\mathpalette{\underbrace@expandable\upbrace@expandable}{#1}}\limits}

\makeatother


\usepackage[small]{titlesec}
\usepackage{cite}

% make sure there is enough TOC for reasonable pdf bookmarks.
\setcounter{tocdepth}{3}

\date{}
\def\to{\rightarrow}
\def\imp{\shortrightarrow}
\def\iff{\leftrightarrow}
\def\union{\cup}
\def\inc{\subseteq}
\def\dom{\mathop{\rm dom}}
\def\cod{\mathop{\rm cod}}
\def\id{{\mathrm 1}}
\def\res{\!\upharpoonleft\!}
\def\ffam{\varphi}
\def\comp{\circ}
\def\bbone{\mathbb 1}
\def\zeromap{0}
\def\bbzero{{\mathbb O}}
\def\ccc{{c.c.c.}}
\def\ev{\varepsilon}
\def\ebc{\varepsilon_{BC}}
\def\L{\Lambda}
\def\l{\lambda}
\def\lm#1.#2{\lambda#1.\, #2}
\def\br#1{[\, #1 \, ]}
\def\V{V}
\def\U{U}
\def\D{D}
\def\C{\mathcal C}
\def\S{\mathcal S}
\def\lxy{\l x\, \l y . \,}
\def\lmm#1#2.#3{\l #1\, \l #2 . \, #3}
\def\sss{(*\!*\!*)}
\def\ss{(**)}
\def\ssn{(**_n)}
\def\scop{\S^{\C^{op}}}
\def\PU{\mathcal P U}
\def\P{\mathcal P}
\def\UU{(U\to U)}
\def\BA{B \to A}
\def\AB{A \to B}
\def\limp{\supset}
\def\PI{P(\limp)}
\def\G{\Gamma}
\def\a{\alpha}
\def\b{\beta}
\def\D{\Delta}
\def\ab{\a \imp \b}

\newtheorem{theorem}{Theorem}

\makeatletter
\newcommand*\dotop{\mathpalette\bigcdot@{.6}}
\newcommand*\bigcdot@[2]{\mathbin{\vcenter{\hbox{\scalebox{#2}{$\m@th#1\bullet$}}}}}
\makeatother

\title{The formulae-as-types notion of construction}
\author{W. H. Howard}
\date{1980}
\begin{document}
\maketitle
\bigskip
{\centerline
{\small\it Dedicated to Professor H. B. Curry on the occasion of his 80th Birthday}
\bigskip

\medskip
\noindent
The following consists of notes which were privately circulated in 1969. Since they have been referred to a few times in the literature, it seems worth while to publish them. They have been rearranged for easier reading, and some inessential corrections have been made.

The ultimate goal was to develop a notion of construction suitable for the interpretation of intuitionistic mathematics. The notion of construction developed in the notes is certainly too crude for that, so the use of the word construction is not very appropriate. However, the terminology has been kept in order to preserve the original title and also to preserve the character of the notes. The title has a second defect; namely, a type should be regarded as a abstract object whereas a formula is the name of a type.

In Part I the ideas are illustrated for the intuitionistic propositional calculus and in Part II (page 6) they are applied to Heyting arithmetic.

\section*{I. Intuitionistic propositional calculus}

H. Curry (1958) has observed that there is a close correspondence between axioms of positive implicational propositional logic, on the one hand, and basic combinators on the other hand. For example, the combinator $K = \l X.\,\l Y.\,X$ corresponds to the axiom $\a \limp (\b \limp \a)$.

The following notion of construction, for positive implicational propositional logic, was motivated by Curry’s observation. More precisely, Curry’s observation provided half the motivation. The other half was provided by W. Tait’s discovery of the close correspondence between cut elimination and reduction of $\l$-terms (W. W. Tait, 1965). It is convenient to use $\l$-terms rather than combinators. This corresponds to the sequent formulation of propositional logic.


\subsection*{\it 1. Formulation of the sequent calculus}

Let $\PI$ denote positive implicational propositional logic. The prime formulae of $\PI$ are propositional variables. If $\a$ and $\b
$ are formulae, so is $\a \limp \b$. A sequent has the form $\G \rightarrow \b$, where $\G$ is a (possibly empty) finite sequence of formulae and $\b$ is a formula. The axioms and rules of inference of $\PI$ are as follows.

\begin{quote}
\begin{description}[font=\normalfont]

\item[(1.1)] Axioms: all sequents of the form $\a \imp \a$,

\item[(1.2)] $\inferrule {\G, \ab}{\G \imp \a \limp \b}$,

\item[(1.3)] $\inferrule {\G \imp \a, \, \D \imp \a \limp \b}{\G, \D \imp \b}$,

\item[(1.4)] Thinning, permutation and contraction rules.
\end{description}
\end{quote}

\subsection*{\it 2. Type symbols, terms and constructions}

By a type symbol is meant a formula of $\PI$. We will consider a $\l$-formalism in which each term has a type symbol $\a$ as a superscript (which we may not always write); the term is said to be of type $\a$. The rules of term formation are as follows.

\begin{quote}
\begin{description}[font=\normalfont]

\item[(2.1)] Variables $X^\a, Y^\b, ... $ are terms.

\item[(2.2)] $\l$-abstraction: from $F$ get $(\l X^\a .\, F^\b )^{\a \limp \b}$.

\item[(2.3)] Application: from $G^{\a \limp b}$ and $H^\a$ get $(G^{\a \limp b} H^\a )^\b$ .
\end{description}
\end{quote}
By a {\it construction} of a sequent $\G \imp \b$ is meant a term $F^\b$ of type $\b$ such that for
every free variable $X^\a$ occurring in $F^\b$ there is a corresponding occurrence of $\a$
in $\G$ (it being understood that the existence of k distinct free variables of the
same type $\a$ in $F^\b$ is reflected by at least $k$ occurrences of $\a$ in $\G$). Thus $X^\a$ is
a construction of $\a \imp \a$ (but $X^\a$ is also a construction of $\b, \a \imp \a)$. Another 
example: $\l X^\a .\, \l Y^\b .\, X^\a$ is a construction of $\imp \a \limp (\b \limp \a)$.

\subsection*{\it 3. Correspondence between derivation and terms}

Clearly the axioms and rules of inference (1.1)–(1.3) of $\PI$ correspond exactly to the rules 
(2.1)–(2.3) of term formation. A construction of $\G \limp \b$ is clearly also a construction of $\G$, $\ab$
(thinning); similarly for a permutation of $\G$; and the contraction
$$
\inferrule{\G, \a, \ab}{\G, \ab}
$$
corresponds to replacing two distinct variables of type $\a$
by one variable of type $\a$ in the corresponding construction. Hence:
\begin{theorem}
Given any derivation of $\G \imp \b$ in $\PI$ we can find a construction of $\G \imp \b$ and conversely.
\end{theorem}

\subsection*{\it 4. Interpretation of terms}

\def\fxg{[F^\a/X^\a]G^\b}

For an interpretation in ordinary set theory let each propositional variable (i.e., prime type symbol) denote a specific set of basic
objects. Then every type symbol can be taken to denote a set of things according to the rule: $\a \limp b$ 
denotes the set of all functions whose domain is a superset of $\a$ and whose range is a subset of $\b$. (According as to whether the superset depends on the function in question, or whether it just depends on $\a$, we get somewhat differing interpretations). 
The variables of type $\a$ are interpreted as ranging over the set $\a$.

It is now clear, by induction on the rules (2.1)–(2.3) of term formation, how each term is to be interpreted as a function of the objects over which its free variables range. Thus the closed terms can be interpreted as a perfectly concrete set of functionals of finite type over the basic objects. This interpretation is used by H. Laüchli in the paper he read at the Summer Conference at Buffalo in 1968: see (Laüchli, 1970) pp. 227–229.

Of course a constructivist would be interested in other interpretations; for example, interpretations related to the calculation of terms (i.e., reduction to irreducible form). It is easy to prove that the terms given above can he reduced to normal 
(i.e., irreducible) form by $\l$-contractions. The relation between this and cut elimination will now be discussed briefly.

\subsection*{\it 5. Normalization of terms and cut elimination}

Clearly the cut rule for $\PI$ corresponds to the following rule of term formation: from $F^\a$ and $G^\b$ get $\fxg$ (the result of substituting $F\a$ for the free variable $X^\a$ in $G^\b$, where no free variable in $F^\a$ becomes hound in $\fxg$. Though we did not include substitution in our rules of term formation, the rule (2.3) (application) is just about as bad --- from the viewpoint of obtaining irreducible terms. Professor Curry is fond of pointing out how to get irreducible terms: simply replace the rule (2.3) by:
\begin{align*}
&{\text{from a variable }} X \text{ of type } \a_1 \limp (\a_2 \limp ( ... ( \a_n \limp \b)... ))\\
\tag{5.1}&\text{and terms } F_1, ... F_n \text{ of types }  \a_1, ... \a_n, \text{ respectively,} \\ 
&\text{get the term }  XF_1 ... F_n\text{ of type } \b
\end{align*}
Correspondingly, replace the rule (1.3) of $\PI$ by the $n$-premise rule
\begin{equation}
\inferrule{\G_1 \imp \a_1, \G_2 \imp \a_2 ... \G_n \imp \a_n}{\G_1, \G_2, ... \G_n,\,\,  \a_1 \limp (\a_2 \limp ( ... ( \a_n \limp \b)... )) \imp \b}.
\tag{5.2}
\end{equation}
Of course (5.2) can be obtained by n applications of one of the Gentzen rules
\begin{equation}
\inferrule{\G \imp \a, \quad \b, \D \imp \gamma}{\a \limp \b, \,\G, \,\,\D \imp \gamma}
\tag{5.3}
\end{equation}
with $\gamma$ equal to $\b$, and use of $\b,\, \D \imp \b$. We could replace (5.1) by a rule of term formation corresponding to (5.3), but (5.1) seems more natural. As a modification of Theorem 1 we have:

\def\PIstar{P^\star(\limp)}
\def\gimpb{\G \imp \b}

\begin{theorem}
Let $\PIstar$ be $\PI$ with the rule {\rm (1.3)} replaced by {\rm (5.2)}. Then given any derivation of $\G \imp \b$
in  $\PIstar$ we can find an irreducible construction of $\G \imp \b$ and conversely.
\end{theorem}
Cut {\it elimination} can be taken to mean: the transformation of a proof of $\gimpb$ in $\PI$ into a proof of $\gimpb$ in $\PIstar$. Thus cut elimination can be obtained as a consequence of the reduction of terms to normal form. As mentioned in section 4, such reduction is easy to prove for the terms under discussion. Results following from cut elimination in $\PI$ (e.g.) the nonderivability of Peirce’s Law $(\a \limp \b. \limp \a) \limp \a$ seem to be obtainable at least as easily from the normalizability of constructions.

\def\fimpa{f \limp a}
\def\contr{\mathop{\sf contr}}

\subsection*{\it 6. Addition of $\lnot$, $\land$ and $\lor$ to $\PI$}

Corresponding to each of these connectives we add certain closed prime terms to our supply of terms.

\medskip
\noindent
(i) For $\lnot$ add a new prime formula $f$ to $\PI$. Then, for each formula $\a$, introduce a term
$A^{\fimpa}$. As an exercise the reader may wish to prove --- for the resulting system ---
that there are no closed terms of type $f$. (By normalizability it is sufficient to prove this for irreducible terms). 
There are open terms of type $f$; for example, the variable $X^f$ — which is a construction of $f \imp f$.
\medskip

\noindent
(ii) For $\land$ add the terms: $B_1^{\a \limp (\b \limp (\a \land \b))}$, $B_2^{\a \land \b \limp \a}$, and $B_3^{\b \land \a \limp \b}$.These are just pairing and projection functionals ($\a \land \b$ is the type of a pair of
terms of types $\a$ and $b$). We do not need to add a term of type $\b \limp (\a \limp \a \land \b)$
because such a term can be defined as $\l Y^\b . \, \l X ^\a . \, B_1 X^\a Y^\b$.

In connection with the theory of reducibility of constructions it is useful to postulate the contraction
\begin{align*}
&B_2(B_1 FG) \contr F \text{ and } \\
&B_3(B_1 FG) \contr G.
\end{align*}
Then we get the following theorem.

\begin{theorem}
Theorem 3 Every closed irreducible term of type $\a \land \b$ has the form $B_1 F^\a G^\b $ where $B_1$ is as above.
\end{theorem}
\end{document}
