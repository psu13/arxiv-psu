\documentclass[12pt]{article}

%\usepackage{amsmath,amsthm,amscd,amssymb}
\usepackage[full]{textcomp}
\usepackage[osf]{XCharter}% lining figures in math, osf in text
\usepackage[scaled=1.04,varqu,varl]{inconsolata}% inconsolata typewriter
\usepackage[bb=pazo,cal=cm,scr=esstix]{mathalpha}

\usepackage{amsmath}
\makeatletter
\newcommand{\leqnos}{\tagsleft@true\let\veqno\@@leqno}
\newcommand{\reqnos}{\tagsleft@false\let\veqno\@@eqno}
\leqnos
\makeatother

% make a ||- symbol
% the right amount to kern backwards is font dependent here. so it's here instead of
% in the main file.
\def\mm{\mathrel{||}\mathrel{\mkern-4.4mu}\relbar}

\usepackage[papersize={6.8in, 9.0in}, left=.5in, right=.5in, top=.7in, bottom=.9in]{geometry}
\linespread{1.1}
\sloppy
\raggedbottom
\pagestyle{plain}
\usepackage[euler-digits]{eulervm}
\usepackage{enumitem}
\usepackage{mathpartir}
\usepackage{stmaryrd}
\usepackage{mathtools}
\usepackage{tikz-cd}
\usepackage{microtype}
\usepackage{amssymb}
%\usepackage{fdsymbol}

% these include amsmath and that can cause trouble in older docs.
\input{../helpers/cmrsum}
\makeatletter

\DeclareFontFamily{OMX}{MnSymbolE}{}
\DeclareSymbolFont{largesymbolsX}{OMX}{MnSymbolE}{m}{n}
\DeclareFontShape{OMX}{MnSymbolE}{m}{n}{
    <-6>  MnSymbolE5
   <6-7>  MnSymbolE6
   <7-8>  MnSymbolE7
   <8-9>  MnSymbolE8
   <9-10> MnSymbolE9
  <10-12> MnSymbolE10
  <12->   MnSymbolE12}{}

\DeclareMathSymbol{\downbrace}    {\mathord}{largesymbolsX}{'251}
\DeclareMathSymbol{\downbraceg}   {\mathord}{largesymbolsX}{'252}
\DeclareMathSymbol{\downbracegg}  {\mathord}{largesymbolsX}{'253}
\DeclareMathSymbol{\downbraceggg} {\mathord}{largesymbolsX}{'254}
\DeclareMathSymbol{\downbracegggg}{\mathord}{largesymbolsX}{'255}
\DeclareMathSymbol{\upbrace}      {\mathord}{largesymbolsX}{'256}
\DeclareMathSymbol{\upbraceg}     {\mathord}{largesymbolsX}{'257}
\DeclareMathSymbol{\upbracegg}    {\mathord}{largesymbolsX}{'260}
\DeclareMathSymbol{\upbraceggg}   {\mathord}{largesymbolsX}{'261}
\DeclareMathSymbol{\upbracegggg}  {\mathord}{largesymbolsX}{'262}
\DeclareMathSymbol{\braceld}      {\mathord}{largesymbolsX}{'263}
\DeclareMathSymbol{\bracelu}      {\mathord}{largesymbolsX}{'264}
\DeclareMathSymbol{\bracerd}      {\mathord}{largesymbolsX}{'265}
\DeclareMathSymbol{\braceru}      {\mathord}{largesymbolsX}{'266}
\DeclareMathSymbol{\bracemd}      {\mathord}{largesymbolsX}{'267}
\DeclareMathSymbol{\bracemu}      {\mathord}{largesymbolsX}{'270}
\DeclareMathSymbol{\bracemid}     {\mathord}{largesymbolsX}{'271}

\def\horiz@expandable#1#2#3#4#5#6#7#8{%
  \@mathmeasure\z@#7{#8}%
  \@tempdima=\wd\z@
  \@mathmeasure\z@#7{#1}%
  \ifdim\noexpand\wd\z@>\@tempdima
    $\m@th#7#1$%
  \else
    \@mathmeasure\z@#7{#2}%
    \ifdim\noexpand\wd\z@>\@tempdima
      $\m@th#7#2$%
    \else
      \@mathmeasure\z@#7{#3}%
      \ifdim\noexpand\wd\z@>\@tempdima
        $\m@th#7#3$%
      \else
        \@mathmeasure\z@#7{#4}%
        \ifdim\noexpand\wd\z@>\@tempdima
          $\m@th#7#4$%
        \else
          \@mathmeasure\z@#7{#5}%
          \ifdim\noexpand\wd\z@>\@tempdima
            $\m@th#7#5$%
          \else
           #6#7%
          \fi
        \fi
      \fi
    \fi
  \fi}

\def\overbrace@expandable#1#2#3{\vbox{\m@th\ialign{##\crcr
  #1#2{#3}\crcr\noalign{\kern2\p@\nointerlineskip}%
  $\m@th\hfil#2#3\hfil$\crcr}}}
\def\underbrace@expandable#1#2#3{\vtop{\m@th\ialign{##\crcr
  $\m@th\hfil#2#3\hfil$\crcr
  \noalign{\kern2\p@\nointerlineskip}%
  #1#2{#3}\crcr}}}

\def\overbrace@#1#2#3{\vbox{\m@th\ialign{##\crcr
  #1#2\crcr\noalign{\kern2\p@\nointerlineskip}%
  $\m@th\hfil#2#3\hfil$\crcr}}}
\def\underbrace@#1#2#3{\vtop{\m@th\ialign{##\crcr
  $\m@th\hfil#2#3\hfil$\crcr
  \noalign{\kern2\p@\nointerlineskip}%
  #1#2\crcr}}}

\def\bracefill@#1#2#3#4#5{$\m@th#5#1\leaders\hbox{$#4$}\hfill#2\leaders\hbox{$#4$}\hfill#3$}

\def\downbracefill@{\bracefill@\braceld\bracemd\bracerd\bracemid}
\def\upbracefill@{\bracefill@\bracelu\bracemu\braceru\bracemid}

\DeclareRobustCommand{\downbracefill}{\downbracefill@\textstyle}
\DeclareRobustCommand{\upbracefill}{\upbracefill@\textstyle}

\def\upbrace@expandable{%
  \horiz@expandable
    \upbrace
    \upbraceg
    \upbracegg
    \upbraceggg
    \upbracegggg
    \upbracefill@}
\def\downbrace@expandable{%
  \horiz@expandable
    \downbrace
    \downbraceg
    \downbracegg
    \downbraceggg
    \downbracegggg
    \downbracefill@}

\DeclareRobustCommand{\overbrace}[1]{\mathop{\mathpalette{\overbrace@expandable\downbrace@expandable}{#1}}\limits}
\DeclareRobustCommand{\underbrace}[1]{\mathop{\mathpalette{\underbrace@expandable\upbrace@expandable}{#1}}\limits}

\makeatother


\usepackage[small]{titlesec}
\usepackage{cite}

% make sure there is enough TOC for reasonable pdf bookmarks.
\setcounter{tocdepth}{3}
\setcounter{secnumdepth}{0}

\usepackage[colorlinks=true
,breaklinks=true
,urlcolor=blue
,anchorcolor=blue
,citecolor=blue
,filecolor=blue
,linkcolor=blue
,menucolor=blue
,linktocpage=true]{hyperref}
\hypersetup{
bookmarksopen=true,
bookmarksnumbered=true,
bookmarksopenlevel=10
}

\date{}
\def\to{\longrightarrow}
\def\imp{\rightarrow}
\def\iff{\leftrightarrow}
\def\union{\cup}
\def\inc{\subseteq}
\def\dom{\mathop{\rm dom}}
\def\cod{\mathop{\rm cod}}
\def\id{{\mathrm 1}}
\def\res{\!\upharpoonleft\!}
\def\ffam{\varphi}
\def\comp{\circ}
\def\bbone{\mathbb 1}
\def\zeromap{0}
\def\bbzero{{\mathbb O}}
\def\ccc{{c.c.c.}}
\def\ev{\varepsilon}
\def\ebc{\varepsilon_{BC}}
\def\L{\Lambda}
\def\l{\lambda}
\def\lm#1.#2{\lambda#1.\, #2}
\def\br#1{[\, #1 \, ]}
\def\V{V}
\def\U{U}
\def\D{D}
\def\C{\mathcal C}
\def\S{\mathcal S}
\def\lxy{\l x\, \l y . \,}
\def\lmm#1#2.#3{\l #1\, \l #2 . \, #3}
\def\sss{(*\!*\!*)}
\def\ss{(**)}
\def\ssn{(**_n)}
\def\scop{\S^{\C^{op}}}
\def\PU{\mathcal P U}
\def\P{\mathcal P}
\def\UU{(U\to U)}
\def\BA{B \to A}
\def\AB{A \to B}
\def\limp{\supset}
\def\PI{P(\limp)}
\def\G{\Gamma}
\def\a{\alpha}
\def\b{\beta}
\def\D{\Delta}
\def\ab{\a \imp \b}
\def\PIstar{P^\star(\limp)}
\def\gimpb{\G \imp \b}
\def\gimpa{\G \imp \a}
\def\rbnn{R_n^{\b(n)}}
\def\ey{\exists y \, \a(y)}
\def\HIMP{H(\limp, \lor, \forall)}
\def\HIMPE{H(\limp, \lor, \forall, \exists)}

\newtheorem{theorem}{Theorem}

\makeatletter
\newcommand*\dotop{\mathpalette\bigcdot@{.6}}
\newcommand*\bigcdot@[2]{\mathbin{\vcenter{\hbox{\scalebox{#2}{$\m@th#1\bullet$}}}}}
\makeatother


\title{The Formulae-as-types Notion of Construction}
\author{W. H. Howard}
\date{1980}

\begin{document}

\maketitle
\begin{center}
{\small\it Dedicated to Professor H. B. Curry on the occasion of his 80th Birthday}
\end{center}
\begin{center}
{\small\it Transcribed into \TeX\ by psu in 2021 based on another transcription \\ by Armando B. Matos from 2017}
\end{center}
\bigskip

\medskip
\noindent
The following consists of notes which were privately circulated in 1969. Since they have been referred to a few times in the literature, it seems worthwhile to publish them. They have been rearranged for easier reading, and some inessential corrections have been made.

The ultimate goal was to develop a notion of construction suitable for the interpretation of intuitionistic mathematics. The notion of construction developed in the notes is certainly too crude for that, so the use of the word {\it construction} is not very appropriate. However, the terminology has been kept in order to preserve the original title and also to preserve the character of the notes. The title has a second defect; namely, a {\it type} should be regarded as an abstract object whereas a {\it formula} is the name of a type.

In Part I the ideas are illustrated for the intuitionistic propositional calculus and in Part II (page 7) they are applied to Heyting arithmetic.

\section{I. Intuitionistic propositional calculus}

H. Curry (1958) has observed that there is a close correspondence between axioms of positive implicational propositional logic, on the one hand, and basic combinators on the other hand. For example, the combinator $K = \l X.\,\l Y.\,X$ corresponds to the axiom $\a \limp (\b \limp \a)$.

The following notion of construction, for positive implicational propositional logic, was motivated by Curry's observation. More precisely, Curry's observation provided half the motivation. The other half was provided by W. Tait's discovery of the close correspondence between cut elimination and reduction of $\l$-terms (W. W. Tait, 1965). It is convenient to use $\l$-terms rather than combinators. This corresponds to the sequent formulation of propositional logic.


\subsection{\it 1. Formulation of the sequent calculus}

Let $\PI$ denote positive implicational propositional logic. The prime formulae of $\PI$ are propositional variables. If $\a$ and $\b
$ are formulae, so is $\a \limp \b$. A sequent has the form $\G \imp \b$\footnote{Editor's note: In most modern exposition on this subject one would use the notation $\vdash$ where this paper uses $\imp$ in these deduction rules. So don't get confused and think that those arrows mean implication and/or a function arrow.}, where $\G$ is a (possibly empty) finite sequence of formulae and $\b$ is a formula. The axioms and rules of inference of $\PI$ are as follows.

\begin{quote}
\begin{description}[font=\normalfont]

\item[(1.1)] Axioms: all sequents of the form $\a \imp \a$,

\item[(1.2)] $\inferrule {\G, \ab}{\G \imp \a \limp \b}$,

\item[(1.3)] $\inferrule {\G \imp \a, \, \D \imp \a \limp \b}{\G, \D \imp \b}$,

\item[(1.4)] Thinning, permutation and contraction rules.
\end{description}
\end{quote}

\subsection{\it 2. Type symbols, terms and constructions}

By a type symbol is meant a formula of $\PI$. We will consider a $\l$-formalism in which each term has a type symbol $\a$ as a superscript (which we may not always write); the term is said to be of type $\a$. The rules of term formation are as follows.

\begin{quote}
\begin{description}[font=\normalfont]

\item[(2.1)] Variables $X^\a, Y^\b, ... $ are terms.

\item[(2.2)] $\l$-abstraction: from $F^\b$ get $(\l X^\a .\, F^\b )^{\a \limp \b}$.

\item[(2.3)] Application: from $G^{\a \limp b}$ and $H^\a$ get $(G^{\a \limp b} H^\a )^\b$ .
\end{description}
\end{quote}
By a {\it construction} of a sequent $\G \imp \b$ is meant a term $F^\b$ of type $\b$ such that for
every free variable $X^\a$ occurring in $F^\b$ there is a corresponding occurrence of $\a$
in $\G$ (it being understood that the existence of $k$ distinct free variables of the
same type $\a$ in $F^\b$ is reflected by at least $k$ occurrences of $\a$ in $\G$). Thus $X^\a$ is
a construction of $\a \imp \a$ (but $X^\a$ is also a construction of $\b, \a \imp \a)$. Another 
example: $\l X^\a .\, \l Y^\b .\, X^\a$ is a construction of $\imp \a \limp (\b \limp \a)$.

\subsection{\it 3. Correspondence between derivation and terms}

Clearly the axioms and rules of inference (1.1)--(1.3) of $\PI$ correspond exactly to the rules 
(2.1)--(2.3) of term formation. A construction of $\G \limp \b$ is clearly also a construction of $\G$, $\ab$
(thinning); similarly for a permutation of $\G$; and the contraction
$$
\inferrule{\G, \a, \ab}{\G, \ab}
$$
corresponds to replacing two distinct variables of type $\a$
by one variable of type $\a$ in the corresponding construction. Hence:
\begin{theorem}
Given any derivation of $\G \imp \b$ in $\PI$ we can find a construction of $\G \imp \b$ and conversely.
\end{theorem}

\subsection{\it 4. Interpretation of terms}

\def\fxg{[F^\a/X^\a]G^\b}

For an interpretation in ordinary set theory let each propositional variable (i.e., prime type symbol) denote a specific set of basic
objects. Then every type symbol can be taken to denote a set of things according to the rule: $\a \limp b$ 
denotes the set of all functions whose domain is a superset of $\a$ and whose range is a subset of $\b$. (According as to whether the superset depends on the function in question, or whether it just depends on $\a$, we get somewhat differing interpretations). 
The variables of type $\a$ are interpreted as ranging over the set $\a$. It is now clear, by induction on the rules (2.1)--(2.3) of term formation, how each term is to be interpreted as a function of the objects over which its free variables range. Thus the closed terms can be interpreted as a perfectly concrete set of functionals of finite type over the basic objects. This interpretation is used by H. Laüchli in the paper he read at the Summer Conference at Buffalo in 1968: see (Laüchli, 1970) pp. 227–-229.

Of course a constructivist would be interested in other interpretations; for example, interpretations related to the {\it calculation}
of terms (i.e., reduction to irreducible form). It is easy to prove that the terms given above can he reduced to normal 
(i.e., irreducible) form by $\l$-contractions. The relation between this and cut elimination will now be discussed briefly.

\subsection{\it 5. Normalization of terms and cut elimination}

Clearly the cut rule for $\PI$ corresponds to the following rule of term formation: from $F^\a$ and $G^\b$ get $\fxg$ (the result of substituting $F^\a$ for the free variable $X^\a$ in $G^\b$, where no free variable in $F^\a$ becomes hound in $\fxg$. Though we did not include substitution in our rules of term formation, the rule (2.3) (application) is just about as bad --- from the viewpoint of obtaining irreducible terms. Professor Curry is fond of pointing out how to get irreducible terms: simply replace the rule (2.3) by:
\begin{align*}
&{\text{from a variable }} X \text{ of type } \a_1 \limp (\a_2 \limp ( ... ( \a_n \limp \b)... ))\\
\tag{5.1}&\text{and terms } F_1, ... F_n \text{ of types }  \a_1, ... \a_n, \text{ respectively,} \\ 
&\text{get the term }  XF_1 ... F_n\text{ of type } \b
\end{align*}
Correspondingly, replace the rule (1.3) of $\PI$ by the $n$-premise rule
\begin{equation}
\inferrule{\G_1 \imp \a_1, \G_2 \imp \a_2 ... \G_n \imp \a_n}{\G_1, \G_2, ... \G_n,\,\,  \a_1 \limp (\a_2 \limp ( ... ( \a_n \limp \b)... )) \imp \b}.
\tag{5.2}
\end{equation}
Of course (5.2) can be obtained by $n$ applications of one of the Gentzen rules
\begin{equation}
\inferrule{\G \imp \a, \quad \b, \D \imp \gamma}{\a \limp \b, \,\G, \,\,\D \imp \gamma}
\tag{5.3}
\end{equation}
with $\gamma$ equal to $\b$, and use of $\b,\, \D \imp \b$. We could replace (5.1) by a rule of term formation corresponding to (5.3), but (5.1) seems more natural. As a modification of Theorem 1 we have:

\begin{theorem}
Let $\PIstar$ be $\PI$ with the rule {\rm (1.3)} replaced by {\rm (5.2)}. Then given any derivation of $\G \imp \b$
in  $\PIstar$ we can find an irreducible construction of $\G \imp \b$ and conversely.
\end{theorem}
Cut {\it elimination} can be taken to mean: the transformation of a proof of $\gimpb$ in $\PI$ into a proof of $\gimpb$ in $\PIstar$. Thus cut elimination can be obtained as a consequence of the reduction of terms to normal form. As mentioned in section 4, such reduction is easy to prove for the terms under discussion. Results following from cut elimination in $\PI$ (e.g.) the nonderivability of Peirce's Law $((\a \limp \b) \limp \a) \limp \a$ seem to be obtainable at least as easily from the normalizability of constructions.

\def\fimpa{f \limp a}
\def\contr{\mathop{\sf contr}}

\subsection{\it 6. Addition of $\lnot$, $\land$ and $\lor$ to $\PI$}

Corresponding to each of these connectives we add certain closed prime terms to our supply of terms.

\medskip
\noindent
(i) For $\lnot$ add a new prime formula $f$ to $\PI$. Then, for each formula $\a$, introduce a term
$A^{\fimpa}$. As an exercise the reader may wish to prove --- for the resulting system ---
that there are no closed terms of type $f$. (By normalizability it is sufficient to prove this for irreducible terms). 
There are open terms of type $f$; for example, the variable $X^f$ — which is a construction of $f \imp f$.

\medskip
\noindent
(ii) For $\land$ add the terms: $B_1^{\a \limp (\b \limp (\a \land \b))}$, $B_2^{\a \land \b \limp \a}$, and $B_3^{\a \land \b \limp \b}$.These are just pairing and projection functionals ($\a \land \b$ is the type of a pair of
terms of types $\a$ and $b$). We do not need to add a term of type $\b \limp (\a \limp \a \land \b)$
because such a term can be defined as $\l Y^\b . \, \l X ^\a . \, B_1 X^\a Y^\b$.

In connection with the theory of reducibility of constructions it is useful to postulate the contraction
$B_2(B_1 FG) \contr F $ and $B_3(B_1 FG) \contr G$.
Then we get the following theorem.

\begin{theorem}
Every closed irreducible term of type $\a \land \b$ has the form $B_1 F^\a G^\b $ where $B_1$ is as above.
\end{theorem}
[{\it Note added 1979}. This treatment of $\lnot$ and $\land$ does not seem to be appropriate for cut elimination in Gentzen's sequent calculus. 
As P. Martin-Löf soon pointed out to me, it is appropriate for D. Prawitz's theory of Gentzen's system of natural deduction.
See Prawitz (1965). The terms $A^{f \limp \a}$ and $B_1^{\a \limp (\b \limp (\a \land \b))}$ correspond
to the inference rules
$$
\inferrule{f}{\a} \text{ and } \inferrule{\a \b}{\a \land \b}, \text{ respectively}.
$$
while $B_2^{\a \land \b \limp \a}$, and $B_3^{\a \land \b \limp \b}$ correspond to
$$
\inferrule{\a \land \b}{\a} \text{ and } \inferrule{\a \land \b}{\b}, \text{ respectively}. ]
$$

\medskip
\noindent
(iii) For $\lor$: there are two possibilities, corresponding to the discussion of weak existence and strong existence in section 12, below.
Corresponding to the case of weak existence we add terms $C_1^{\a \limp (\a \lor \b)}$, $C_2^{\b \limp (\a \lor \b)}$ and 
$C_3^{\a \lor \b \limp (\a \limp \gamma. \, \limp (\b \limp \gamma. \, \limp \gamma))}$. It is useful to postulate the contractions
$C_3(C_1M) FG \contr FM$ and $C_3(C_2N)FG \contr GN$
for all terms $M$,$N$,$F$,$G$ of types $\a$, $\b$, $a \limp \b$ and $\b \limp \gamma$ respectively. Then we get the following theorem.

\begin{theorem}
Every closed irreducible term of type $\a \lor \b$ has the form $C_1F^\a$ or $C_2 G^\b$, where $C_1$ and $C_2$ are as above.
\end{theorem}

\section{II. Heyting Arithmetic}

We will be concerned mainly with the subsystem $H(\limp, \lor, \forall)$ of Heyting arithmetic obtained by omitting $\lor$ and $\exists$. As is well-known, $\lnot\a$ can be defined as $\a \limp 0 = 1$ In section 12 we will make some remarks about the question of including $\exists$. Of course $\lor$ can be defined by use of $\exists$. The variables belonging to $H(\limp, \lor, \forall)$ will be called {\it number variables}.

\subsection{\it 7. Constructions}

Our constructions will be terms built up from prime terms by means of rules of term formation as indicated in (i)--(iv), below. Every term is supplied with a unique type symbol. The numerical terms — namely, the terms belonging to $\HIMP$ --— have type $0$.

\begin{enumerate}[label=(\roman*)]

\item {\it Type Symbols.} The prime type symbols are: $0$ and every equation of $\HIMP$. From these we generate all type symbols by the following two rules:
\begin{enumerate}
\item From $\a$ and $\b$ get $\a \limp b$ and $\a \land \b$.
\item From $\a$ and a number variable $x$ get $\forall x\, \a$
\end{enumerate}
\item {\it Prime terms.} These are:
\begin{enumerate}
\item Number variables $x, y, ...$; constants $0$ and $1$; function symbols for plus and times,
\item Variables $X^\a, Y^\b, ...$,
\item Certain special terms, mentioned in section 8, below, corresponding to axioms and rules of inference of $\HIMP$.
\end{enumerate}
\item {\it $\l$-abstraction.}
\begin{enumerate}
\item From $F^\b$ get $(\l X^\a .\, F^\b )^{\a \limp \b}$ as in section 2.
\item If $x$ does not occur free in the type symbol of any free variable of $F$, form $(\l x. \, F^\b)^{\forall x \, \b}$.
\end{enumerate}

\item {\it Application.}
\begin{enumerate}
\item From $G^{\a \limp b}$ and $F^\a$ form $(G^{\a \limp b} F^\a )^\b$ as in section 2.
\item From $G^{\forall x \, \a(x)}$ and $t$ of type $0$ form $G(t)^{\a(t)}$.
\end{enumerate}

\end{enumerate}

\subsection{\it 8. Special terms}

\begin{enumerate}[label=(\roman*)]
\item Terms of types $x+0=x$ and $x+(y+1)=(x+y)+1$,
\item Terms of types $x \cdot 0=0$ and $x\cdot(y+1)=x \cdot y + x$,
\item A term of type $x = x$,
\item A term of type $x=y \limp t(x)=t(y)$ for each term $t(x)$ of type $0$,
\item The terms $B_1$, $B_2$, and $B_3$ discussed in section 6(ii),
\item A term $R^{\forall y \, \b(y)}$ for each $\b(y)$ of the form
$$
\a(0) \limp [\forall x \, (\a(x) \limp \a(x + 1)) \limp \a(y)],
$$
also a term $R^{\b(n)}$ for each numeral $n$.
\end{enumerate}

\subsection{\it 9. Constructions and derivations in $\HIMP$}

Define constructions as in section 2. As in the case of $\PI$ —-- see section 3 —-- the axioms of $\HIMP$ correspond to the existence of
certain terms, and the rules of inference correspond to rules of term formation. In particular $\forall x \a(x) \imp \a(t)$ has the
construction $Y^{\forall x  \a(x) \imp \a(t)}$. If $F$ is a construction of $\gimpa$, then $\l x. \, F$ is a construction of $\G \imp \forall x \, \a$. If $G(X^{\a(t)})$ is a construction of $\G,\a(y) \imp \b$, then $G(Y^{\forall x \, \a (x)} t)$ is a construction of 
$\G, \, \forall x \a (x) \imp \b$. Thus, we obtain:

\begin{theorem}
Given any derivation of $\gimpb$ in $\HIMP$ we can find a construction of $\gimpb$ and conversely.
\end{theorem}

\subsection{\it 10. Interpretation of terms of $\HIMP$}

We extend the discussion of section 4 as follows.

\begin{enumerate}[label=(\roman*)]

\item We interpret each closed formula $\a$ as a set $\a\star$ in the following manner. If $\a$
is a closed equation, then $\a^\star$ is the singleton set $\{1\}$ if the equation is true and
the set $\{0\}$ if the equation is false. If $\a(n)^\star$ has been defined for each numeral $n$,
then define $(\forall x\, \a(x))^\star$ as the set of all functions $f$ such $f(n) \in a(n)^\star$ for every $n$.
Define $(\a\limp \b)^\star$ in terms of $\a^\star$ and $\b^\star$ as in section 4. Define $(\a \land \b)^\star$ as the Cartesian
product of $\a^\star$ and $\b^\star$.

\item To each term we associate an object $F^\star$ by use of the following clauses. It
should be noted that when $F$ is closed, then the type symbol of $F$ is a closed
formula $\a$; and $F$ is an element of $\a^\star$.

\begin{enumerate}

\item Suppose $F$ has the form $G(x_1, ... ,x_k)$, where $x_l, ... , x_k$ are the free number
variables of $F$. Assuming that $G(n_1, ... , n_k)^\star$ has been defined for all numerals 
$n_1, ... , n_k$, we define $F^\star$ to be the mapping which sends each $k$-tuple $n_1, ... , n_k$ into
$G(n_1, ... , n_k)^\star$

\item Suppose $F$ has no free number variables. If $F$ has the form $\forall x\, G(x)$, then
define $F$ as the mapping which takes each numeral $n$ into $G(n)$. If $F$ does not
have this form, then the free variables of F will have types $\b_1, ... ,\b_k$ which have
been interpreted as sets $\b_1^\star,...,\b_k^\star$ in (i), above, and $F$ will be a mapping from
the Cartesian product of $\b_1^\star,...,\b_k^\star$ into $\a^\star$.%
\medskip

In particular, if $F$ has the form $\l Y^\b .G$, then $F^\star$ is defined in terms of $G^\star$ by the usual interpretation of $\l$-abstraction. To be systematic we must allow free variables, here and in (a), to occur vacuously.

\item If $F$ has the form $\rbnn$ as in section 8(vi), then $F^\star$ is defined by primitive recursion
on $n$. If $F$ is $R^{\forall y \, \b(y)}$, then $F$ is the mapping which takes each numeral $n$ into $(\rbnn)^\star$.

\end{enumerate}

\end{enumerate}

\subsection{\it 11. Normalization of terms}

\leqnos

For the theory of reducibility of terms we postulate the following contraction
schemes:

\begin{equation}
\begin{gathered}
(\l X. \, F(X))^{\a \limp \b} G \quad \contr \quad F(G)^\b \\
(\l X. \, F(X))^{\forall x \a(x)} t \quad \contr\quad  F(t)^{\a(t)}
\end{gathered}
\tag{i}
\end{equation}

\begin{equation}
\begin{aligned}
B_2 (B_1 FG) & \quad \contr \quad F\\
B_3 (B_1 FG) & \quad \contr\quad  G\\
\end{aligned}
\tag{ii}
\end{equation}

\begin{equation}
\begin{aligned}
&R_0^{\b(0)} FG & \contr & \quad F\\
&R_{n+1}^{\b(n+1)} FG & \contr & \quad GnR_n^{\b(n)}\\
&R^{\forall x \, \b(x)} n & \contr & \quad R_n^\b(n)
\end{aligned}
\tag{iii}
\end{equation}

Also we consider the steps in the calculation of closed numerical terms to be
reduction steps. 

By the method of Tait (1967) it is easy to prove:

\begin{theorem}
Every term can be reduced to irreducible form.
\end{theorem}

Theorem 3 of section 6 extends to the present terms. By use of Theorem 3 it is easy to show that there is no irreducible construction of a sequent of the form $\imp m = n$ with $m \neq n$. Thus by finitistic reasoning it can be proved that Theorem 6 implies the consistency of $\HIMP$.

\subsection{12. Introduction of $\exists$}

How are we to handle the existential quantifier in our theory of constructions? This appears to be a nontrivial question. The following two alternatives suggest themselves.

\bigskip
\noindent
(i) {\it Weak existence}. In the formulation of $\HIMPE$ take the sequents
\begin{align*}
&\imp \a (t) \limp \exists y\, \a(y) \\
&\imp \exists y\, \a(y) \limp ( \forall x \, (\a(x) \limp \b) \limp \b)
\end{align*}
%
($x$ not free in $\b$) to be axioms and introduce new prime terms $C_1$ , $C_2$ of types $\a(t) \limp \exists y \, \a(y)$ and  $\exists y\, \a(y) \limp ( \forall x \, (\a(x) \limp \b) \limp \b)$, respectively. Postulate the contraction $C_2 ( C_1 F ) G \contr G t F$
for all terms $F$ and $G$ of types $\a( t )$ and $\forall x \, (\a(x) \limp \b) \limp \b)$, respectively. Theorems 3, 5 and 6 extend to $\HIMPE$.

Corresponding to Theorem 4 we have:

\begin{theorem}
Every closed irreducible term of type $\exists y \, \a(y)$ has the form $C_1^{ \a(n) \limp \exists y \, \a(y)} F ^{\a (n)}$.
\end{theorem}
%
Thus from an irreducible construction of $\imp  \exists y \, \a(y)$ we get a numeral $n$ and an irreducible construction of $\imp \a( n )$ (assuming $\exists y \, \a(y)$ closed).

\bigskip
\noindent
(ii) {\it Strong existence (choice operators)}. It is natural to interpret an object of type $\ey$ as a pair $\langle t, F^{\a(t)}\rangle$ . Thus, in section 10, $(\ey)^\star$ would be defined as the set of all pairs $\langle n, Z\rangle$ such that $Z \in \a(n)^\star$. Hence introduce projection operators $P_1$ and $P_2$ which give the required components $t$ and $F^{\a(t)}$ when applied to the pair $\langle t, F^{\a(t)}\rangle$ regarded as an object of type $\ey$. The operators $P_1$ and $P_2$ can be considered to have types $\ey \limp 0$ and $X^{\ey}\a(P_1 X^{\ey})$, respectively. This takes us outside the formalism of $\HIMPE$: the type symbols of $P_1$ and $P_2$ are not formulae of $\HIMPE$. Nevertheless, it is clear what meaning we are to attach to $P_1$ and $P_2$. Namely, $P_1$ applies to an object $F$ of type $\ey$ and an object $P_1 F$ of type $0$; and $P_2$ when applied to $F$ yields an object of type $\a ( P_1 F )$.

To illustrate the use of $P_1$ as a choice operator, let $\a(x,y)$ be a formula of $\HIMPE$ with free variables $x$ and $y$ , and let $F$
 be a construction of type $\ey$. Then $\l x . \, P_1 F$ is a term $\varphi$ satisfying $\forall x \, \a (x, \varphi(x))$

In general a type symbol of the form $\forall X^ \a\b$ arises by abstraction from a term $F^\b$, where the variable $X^\a$ occurs in $\b$ (as well as in other locations in $F$). A theory of such terms would extend the theory developed in the preceding sections. In the latter theory the only variables occurring in the type symbols have type $0$ (they are, namely, numerical variables of $\HIMPE$).


It might be of interest to know the answer to the following question. Let $G$ be a closed term obtained by extending our supply of constructions in the manner just described. Suppose $G$ has type $\a$ where $\a$ is a formula of $\HIMPE$. Must there be a derivation of $\imp \a$ in $\HIMPE$?


\subsection{\it 13. Infinite constructions}

As is well-known, we do not have cut elimination for $\HIMP$ unless the axiom of mathematical induction is replaced by an $\omega$-rule.
There is no difficulty in developing a theory of constructions for the $\omega$-rule version of $\HIMP$. 
In fact, if one uses Tait's notion of infinite terms (Tait, 1967), one gets a very simple theory (modulo the question of handling infinite terms constructively).

\newpage

\renewcommand{\refname}{}

\subsection{\rm References}

\vskip-20pt

\begin{thebibliography}{9}


\bibitem{1} Curry, H. B. and Feys, R. (1958). Combinatory Logic, North-Holland, Amsterdam.

\bibitem{2} Lauchli, H. (1970). An abstract notion of realizability for which intuitionistic predicate calculus is complete. In ``Intuitionism and Proof Theory. Proceedings of the Summer Conference at Buffalo N.Y.'' (Eds. G. Myhill, A. Kino, R. E. Vesley) pp. 227–234. North-Holland, Amsterdam.

\bibitem{3} Prawitz, D. (1965). Natural Deduction, Almqvist and Wiksell, Stockholm.

\bibitem{4} Tait, W. W. (1965). Infinitely long terms of transfinite type. In ``Formal Systems and Recursive Functions'' (Eds. J. N. Crossley and M. A. E. Dummett), North-Holland, Amsterdam.

\bibitem{5} Tait, W. W. (1967). Intensional interpretations of functionals of finite type, Journal of Symbolic Logic 32, 198–212.

\end{thebibliography}

\end{document}
