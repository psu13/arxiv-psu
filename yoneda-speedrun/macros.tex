% Basic Category Theory
% Tom Leinster <Tom.Leinster@ed.ac.uk>
% 
% Copyright (c) Tom Leinster 2014-2016
% 
% Macros
% 


% PACKAGES

\usepackage{latexsym}
\usepackage{amssymb,amsmath}
\usepackage{bbm}
\usepackage{graphicx}
\usepackage{color}
\usepackage{multicol}
\usepackage[cal=cm,scr=euler]{mathalpha}
\usepackage{url}
\urlstyle{same}


% MATHEMATICAL EXPRESSIONS TAKING ARGUMENTS

\newcommand{\h}{H}                      % Homs (e.g. \h_A means Hom(-, A))
\newcommand{\op}{\mathrm{op}}           % Opposite category
\newcommand{\true}{\texttt{true}}       % Truth value "true"    
\newcommand{\false}{\texttt{false}}   
\newcommand{\ftrcat}[2]{[#1,#2]}                % Functor category
\newcommand{\pshf}[1]{\ftrcat{#1^\op}{\Set}}    % Presheaf category
\newcommand{\psh}[1]{\hat{#1}}                  % Alternative pshf cat

\newcommand{\epsln}{\varepsilon}        % \epsilon is not used anywhere

\newcommand{\nat}{{\mathbb{N}}}           % Natural numbers
\newcommand{\integers}{{\mathbb Z}}
\newcommand{\rationals}{{\mathbb{Q}}}
\newcommand{\reals}{{\mathbb{R}}}
\newcommand{\complexes}{{\mathbb{C}}}

\newcommand{\cat}[1]{\mathbf{#1}}      % Arbitrary category
\newcommand{\scat}[1]{{\mathbf {#1}}}     % Arbitrary small category
\newcommand{\fcat}[1]{{\mathbf {#1}}}    % Typeface for fixed categories
\newcommand{\id}{\mathrm{id}} % identity
\newcommand{\CC}{\cat{C}}
\newcommand{\DD}{\cat{D}}
\DeclareMathOperator{\Arrows}{\mathit{Arrows}}
\DeclareMathOperator{\Objects}{\mathit{Objects}}
\DeclareMathOperator{\Hom}{\mathit{Hom}}
\DeclareMathOperator{\Nat}{\mathit{Nat}}
\def\objc{\Objects(\cat{C})}
\newcommand{\Set}{\fcat{Set}}           % Sets
\newcommand{\iso}{\cong}                % Isomorphism
\newcommand{\eqv}{\simeq}               % Equivalence
\newcommand{\sub}{\subseteq}            % Subset (possibly not proper)