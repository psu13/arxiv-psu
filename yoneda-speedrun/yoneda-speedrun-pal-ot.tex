\documentclass[12pt]{article}

\usepackage{newpxtext}
\usepackage{mathtools}
\usepackage{euler-math}


\usepackage[small]{titlesec}
\usepackage{cite}
\usepackage{tabularx}

%\usepackage{amsmath,amsthm,amscd,amssymb}
\usepackage[colorlinks=true,
breaklinks=true,
urlcolor=blue,
anchorcolor=blue,
citecolor=blue,
filecolor=blue,
linkcolor=blue,
menucolor=blue,linktocpage=true]{hyperref}
\hypersetup{bookmarksopen=true, bookmarksnumbered=true, bookmarksopenlevel=10}
%\setsansfont{Optima}
%\setmathfont{Asana Math}
%\usepackage[scale=.9]{tgheros}   %% Option 'sfdefault' only if
%\usepackage[scaled]{FiraSans}
\linespread{1.05}
\usepackage[papersize={6.9in, 10in}, left=.5in, right=.5in, top=.7in, bottom=.9in]{geometry}
\sloppy
\raggedbottom
\pagestyle{plain}
\usepackage{tikz}
\usepackage{tikz-cd}

% make sure there is enough TOC for reasonable pdf bookmarks.
\setcounter{tocdepth}{3}

\usepackage{amsthm}
\theoremstyle{definition}

\newtheorem{thm}{Theorem}[]
\newtheorem{lemma}[thm]{Lemma}
\newtheorem{cor}[thm]{Corollary}

\theoremstyle{definition}
\newtheorem{defn}{Definition}[]
\newtheorem{example}{Example}[]

\theoremstyle{definition}
\newtheorem{remark}{Remark}[]
\newtheorem{note}{Note}[]

\numberwithin{equation}{section}

\titleformat{\section}[block]
  {\fillast} {\bfseries{\thesection }.} {1ex minus .1ex} {\bfseries}
  
\titleformat{\subsection}[block]
  {\fillast} {{\itshape \thesection }.} {1ex minus .1ex} {\itshape}

\titleformat*{\subsubsection}{\scshape}

% PACKAGES
\usepackage{url}
\urlstyle{same}

\newcommand{\op}{\mathrm{op}}           % Opposite category
\newcommand{\true}{\texttt{true}}       % Truth value "true"    
\newcommand{\false}{\texttt{false}}   
\newcommand{\ftrcat}[2]{[#1,#2]}                % Functor category
\newcommand{\pshf}[1]{\ftrcat{#1^\op}{\Set}}    % Presheaf category
\newcommand{\psh}[1]{\hat{#1}}                  % Alternative pshf cat

\newcommand{\epsln}{\varepsilon}        % \epsilon is not used anywhere

\newcommand{\nat}{{\mathbb{N}}}           % Natural numbers
\newcommand{\integers}{{\mathbb Z}} 
\newcommand{\rationals}{{\mathbb{Q}}}
\newcommand{\reals}{{\mathbb{R}}} 
\newcommand{\complexes}{{\mathbb{C}}}


\newcommand{\Set}{\fcat{Sets}}           % Sets
\newcommand{\iso}{\cong}                % Isomorphism
\newcommand{\eqv}{\simeq}               % Equivalence
\newcommand{\sub}{\subseteq}            % Subset (possibly not proper)

%%%%%%% YONEDA %%%%%%%%%%%%
\newcommand{\yo}{\text{\usefont{U}{min}{m}{n}\symbol{'210}}}
\DeclareFontFamily{U}{min}{}
\DeclareFontShape{U}{min}{m}{n}{<-> udmj30}{}
\newcommand{\Yo}{\mathop{Y\kern-.5pt o}}

\newcommand{\gr}{\fcat{Group}}
 \newcommand{\mat}{\fcat{Matrix}}
\newcommand{\mon}{\fcat{Monoid}}

\newcommand{\fto}{\Rightarrow}

\let\setName=\mathrm

\newcommand{\cat}[1]{\mathbf{#1}}      % Arbitrary category
\newcommand{\scat}[1]{{\mathbf {#1}}}     % Arbitrary small category
\newcommand{\fcat}[1]{{\mathbf {#1}}}    % Typeface for fixed categories
\newcommand{\id}{\mathrm{id}} % identity
\newcommand{\CC}{\cat{C}} 
\newcommand{\CCop}{\cat{C}^{\mathrm op}}
\newcommand{\DD}{\cat{D}}
\newcommand{\DDop}{\cat{D}^{\mathrm op}}
\DeclareMathOperator{\Arrows}{\setName{Arrows}}
\DeclareMathOperator{\Objects}{\setName{Objects}}
 \DeclareMathOperator{\Hom}{\setName{Hom}}
\DeclareMathOperator{\ihom}{\setName{hom}}
 \DeclareMathOperator{\Nat}{\setName{Natural}}
\DeclareMathOperator{\Fun}{\setName{Functors}}
 \def\objc{\Objects(\cat{C})}
\newcommand{\h}{{H}}                      % Homs (e.g. \h_A means Hom(-, A))
\newcommand{\shortNat}{\setName{Nat}}
\newcommand{\shortFun}{\setName{Fun}}
\newcommand{\func}{\setName{Func}}
\newcommand{\funct}{\setName{Funct}}
\newcommand{\littleh}{{h}}

\def\pg{\bigskip\goodbreak\noindent}

\input yoneda-main

\end{document}
