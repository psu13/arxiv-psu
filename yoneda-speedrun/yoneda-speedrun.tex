\documentclass[12pt]{article}

\usepackage{amsthm}

%\usepackage{amsmath,amsthm,amscd,amssymb}
\usepackage[colorlinks=true
,breaklinks=true
,urlcolor=blue
,anchorcolor=blue
,citecolor=blue
,filecolor=blue
,linkcolor=blue
,menucolor=blue
,linktocpage=true]{hyperref}
\hypersetup{
bookmarksopen=true,
bookmarksnumbered=true,
bookmarksopenlevel=10
}
\usepackage[noBBpl,sc]{mathpazo}
\usepackage{eulervm}
\linespread{1.05}
\usepackage[papersize={6.9in, 10.0in}, left=.5in, right=.5in, top=1in, bottom=.9in]{geometry}
\sloppy
\raggedbottom
\pagestyle{plain}
\usepackage{tikz}
\usepackage{tikz-cd}


% these include amsmath and that can cause trouble in older docs.
\makeatletter
\@ifpackageloaded{amsmath}{}{\RequirePackage{amsmath}}

\DeclareFontFamily{U}  {cmex}{}
\DeclareSymbolFont{Csymbols}       {U}  {cmex}{m}{n}
\DeclareFontShape{U}{cmex}{m}{n}{
    <-6>  cmex5
   <6-7>  cmex6
   <7-8>  cmex6
   <8-9>  cmex7
   <9-10> cmex8
  <10-12> cmex9
  <12->   cmex10}{}

\def\Set@Mn@Sym#1{\@tempcnta #1\relax}
\def\Next@Mn@Sym{\advance\@tempcnta 1\relax}
\def\Prev@Mn@Sym{\advance\@tempcnta-1\relax}
\def\@Decl@Mn@Sym#1#2#3#4{\DeclareMathSymbol{#2}{#3}{#4}{#1}}
\def\Decl@Mn@Sym#1#2#3{%
  \if\relax\noexpand#1%
    \let#1\undefined
  \fi
  \expandafter\@Decl@Mn@Sym\expandafter{\the\@tempcnta}{#1}{#3}{#2}%
  \Next@Mn@Sym}
\def\Decl@Mn@Alias#1#2#3{\Prev@Mn@Sym\Decl@Mn@Sym{#1}{#2}{#3}}
\let\Decl@Mn@Char\Decl@Mn@Sym
\def\Decl@Mn@Op#1#2#3{\def#1{\DOTSB#3\slimits@}}
\def\Decl@Mn@Int#1#2#3{\def#1{\DOTSI#3\ilimits@}}

\let\sum\undefined
\DeclareMathSymbol{\tsum}{\mathop}{Csymbols}{"50}
\DeclareMathSymbol{\dsum}{\mathop}{Csymbols}{"51}

\Decl@Mn@Op\sum\dsum\tsum

\makeatother

\makeatletter
\@ifpackageloaded{amsmath}{}{\RequirePackage{amsmath}}

\DeclareFontFamily{OMX}{MnSymbolE}{}
\DeclareSymbolFont{largesymbolsX}{OMX}{MnSymbolE}{m}{n}
\DeclareFontShape{OMX}{MnSymbolE}{m}{n}{
    <-6>  MnSymbolE5
   <6-7>  MnSymbolE6
   <7-8>  MnSymbolE7
   <8-9>  MnSymbolE8
   <9-10> MnSymbolE9
  <10-12> MnSymbolE10
  <12->   MnSymbolE12}{}

\DeclareMathSymbol{\downbrace}    {\mathord}{largesymbolsX}{'251}
\DeclareMathSymbol{\downbraceg}   {\mathord}{largesymbolsX}{'252}
\DeclareMathSymbol{\downbracegg}  {\mathord}{largesymbolsX}{'253}
\DeclareMathSymbol{\downbraceggg} {\mathord}{largesymbolsX}{'254}
\DeclareMathSymbol{\downbracegggg}{\mathord}{largesymbolsX}{'255}
\DeclareMathSymbol{\upbrace}      {\mathord}{largesymbolsX}{'256}
\DeclareMathSymbol{\upbraceg}     {\mathord}{largesymbolsX}{'257}
\DeclareMathSymbol{\upbracegg}    {\mathord}{largesymbolsX}{'260}
\DeclareMathSymbol{\upbraceggg}   {\mathord}{largesymbolsX}{'261}
\DeclareMathSymbol{\upbracegggg}  {\mathord}{largesymbolsX}{'262}
\DeclareMathSymbol{\braceld}      {\mathord}{largesymbolsX}{'263}
\DeclareMathSymbol{\bracelu}      {\mathord}{largesymbolsX}{'264}
\DeclareMathSymbol{\bracerd}      {\mathord}{largesymbolsX}{'265}
\DeclareMathSymbol{\braceru}      {\mathord}{largesymbolsX}{'266}
\DeclareMathSymbol{\bracemd}      {\mathord}{largesymbolsX}{'267}
\DeclareMathSymbol{\bracemu}      {\mathord}{largesymbolsX}{'270}
\DeclareMathSymbol{\bracemid}     {\mathord}{largesymbolsX}{'271}

\def\horiz@expandable#1#2#3#4#5#6#7#8{%
  \@mathmeasure\z@#7{#8}%
  \@tempdima=\wd\z@
  \@mathmeasure\z@#7{#1}%
  \ifdim\noexpand\wd\z@>\@tempdima
    $\m@th#7#1$%
  \else
    \@mathmeasure\z@#7{#2}%
    \ifdim\noexpand\wd\z@>\@tempdima
      $\m@th#7#2$%
    \else
      \@mathmeasure\z@#7{#3}%
      \ifdim\noexpand\wd\z@>\@tempdima
        $\m@th#7#3$%
      \else
        \@mathmeasure\z@#7{#4}%
        \ifdim\noexpand\wd\z@>\@tempdima
          $\m@th#7#4$%
        \else
          \@mathmeasure\z@#7{#5}%
          \ifdim\noexpand\wd\z@>\@tempdima
            $\m@th#7#5$%
          \else
           #6#7%
          \fi
        \fi
      \fi
    \fi
  \fi}

\def\overbrace@expandable#1#2#3{\vbox{\m@th\ialign{##\crcr
  #1#2{#3}\crcr\noalign{\kern2\p@\nointerlineskip}%
  $\m@th\hfil#2#3\hfil$\crcr}}}
\def\underbrace@expandable#1#2#3{\vtop{\m@th\ialign{##\crcr
  $\m@th\hfil#2#3\hfil$\crcr
  \noalign{\kern2\p@\nointerlineskip}%
  #1#2{#3}\crcr}}}

\def\overbrace@#1#2#3{\vbox{\m@th\ialign{##\crcr
  #1#2\crcr\noalign{\kern2\p@\nointerlineskip}%
  $\m@th\hfil#2#3\hfil$\crcr}}}
\def\underbrace@#1#2#3{\vtop{\m@th\ialign{##\crcr
  $\m@th\hfil#2#3\hfil$\crcr
  \noalign{\kern2\p@\nointerlineskip}%
  #1#2\crcr}}}

\def\bracefill@#1#2#3#4#5{$\m@th#5#1\leaders\hbox{$#4$}\hfill#2\leaders\hbox{$#4$}\hfill#3$}

\def\downbracefill@{\bracefill@\braceld\bracemd\bracerd\bracemid}
\def\upbracefill@{\bracefill@\bracelu\bracemu\braceru\bracemid}

\DeclareRobustCommand{\downbracefill}{\downbracefill@\textstyle}
\DeclareRobustCommand{\upbracefill}{\upbracefill@\textstyle}

\def\upbrace@expandable{%
  \horiz@expandable
    \upbrace
    \upbraceg
    \upbracegg
    \upbraceggg
    \upbracegggg
    \upbracefill@}
\def\downbrace@expandable{%
  \horiz@expandable
    \downbrace
    \downbraceg
    \downbracegg
    \downbraceggg
    \downbracegggg
    \downbracefill@}

\DeclareRobustCommand{\overbrace}[1]{\mathop{\mathpalette{\overbrace@expandable\downbrace@expandable}{#1}}\limits}
\DeclareRobustCommand{\underbrace}[1]{\mathop{\mathpalette{\underbrace@expandable\upbrace@expandable}{#1}}\limits}

\makeatother


\usepackage[small]{titlesec}
\usepackage{cite}

% make sure there is enough TOC for reasonable pdf bookmarks.
\setcounter{tocdepth}{3}

\theoremstyle{definition}

\newtheorem{thm}{Theorem}[]
\newtheorem{lemma}[thm]{Lemma}

\theoremstyle{definition}
\newtheorem{defn}{Definition}[]
\newtheorem{example}{Example}[]

\theoremstyle{definition}
\newtheorem{remark}{Remark}[]
\newtheorem{note}{Note}[]

\numberwithin{equation}{section}


%    Blank box placeholder for figures (to avoid requiring any
%    particular graphics capabilities for printing this document).
\newcommand{\blankbox}[2]{%
  \parbox{\columnwidth}{\centering
%    Set fboxsep to 0 so that the actual size of the box will match the
%    given measurements more closely.
    \setlength{\fboxsep}{0pt}%
    \fbox{\raisebox{0pt}[#2]{\hspace{#1}}}%
  }%
}

\titleformat{\section}[block]
  {\fillast}
  {\bfseries{\thesection }.}
  {1ex minus .1ex}
  {\bfseries}
  
\titleformat{\subsection}[block]
  {\fillast}
  {{\it \thesection }.}
  {1ex minus .1ex}
  {\it}

\titleformat*{\subsubsection}{\scshape}

% PACKAGES

\usepackage{latexsym}
\usepackage{amssymb,amsmath}
\usepackage{bbm}
\usepackage[cal=cm,scr=euler]{mathalpha}
\usepackage{url}
\urlstyle{same}

\newcommand{\h}{H}                      % Homs (e.g. \h_A means Hom(-, A))
\newcommand{\op}{\mathrm{op}}           % Opposite category
\newcommand{\true}{\texttt{true}}       % Truth value "true"    
\newcommand{\false}{\texttt{false}}   
\newcommand{\ftrcat}[2]{[#1,#2]}                % Functor category
\newcommand{\pshf}[1]{\ftrcat{#1^\op}{\Set}}    % Presheaf category
\newcommand{\psh}[1]{\hat{#1}}                  % Alternative pshf cat

\newcommand{\epsln}{\varepsilon}        % \epsilon is not used anywhere

\newcommand{\nat}{{\mathbb{N}}}           % Natural numbers
\newcommand{\integers}{{\mathbb Z}}
\newcommand{\rationals}{{\mathbb{Q}}}
\newcommand{\reals}{{\mathbb{R}}}
\newcommand{\complexes}{{\mathbb{C}}}

\newcommand{\cat}[1]{\mathbf{#1}}      % Arbitrary category
\newcommand{\scat}[1]{{\mathbf {#1}}}     % Arbitrary small category
\newcommand{\fcat}[1]{{\mathbf {#1}}}    % Typeface for fixed categories
\newcommand{\id}{\mathrm{id}} % identity
\newcommand{\CC}{\cat{C}}
\newcommand{\CCop}{\cat{C}^{\mathrm op}}
\newcommand{\DD}{\cat{D}}
\newcommand{\DDop}{\cat{D}^{\mathrm op}}
\DeclareMathOperator{\Arrows}{\mathit{Arrows}}
\DeclareMathOperator{\Objects}{\mathit{Objects}}
\DeclareMathOperator{\Hom}{\mathit{Hom}}
\DeclareMathOperator{\ihom}{\mathit{hom}}
\DeclareMathOperator{\Nat}{\mathit{Natural}}
\DeclareMathOperator{\Fun}{\mathit{Functor}}
\def\objc{\Objects(\cat{C})}
\newcommand{\Set}{\fcat{Set}}           % Sets
\newcommand{\iso}{\cong}                % Isomorphism
\newcommand{\eqv}{\simeq}               % Equivalence
\newcommand{\sub}{\subseteq}            % Subset (possibly not proper)

\newcommand{\gr}{\fcat{Group}}
\newcommand{\mat}{\fcat{Matrix}}
\newcommand{\mon}{\fcat{Monoid}}

\newcommand{\fto}{\Rightarrow}

\usepackage{microtype}

\begin{document}

\title{\Large A Speedrun to the Yoneda Lemma}
\author{\large Pete Su}
\date{\large 31 October 2021}

\maketitle

%!TEX root = yoneda-speedrun.tex

\section{The Big Picture}

The Yoneda Lemma is a basic and beloved result in category theory. Its statement is
deceivingly simple because its content is buried inside layers of abstraction and
notation. For me the hard part of understanding this result was unwrapping the abstraction
ladder.

I am going to do the following dumb thing: I am going to state the result at the top to
give us the target. Then, in the spirit of video game speedruns, we will only
write down what we absolutely need to to understand the Lemma, running past a lot of
interesting material that is not absolutely necessary to reach our goal.

Note that I am not a mathematician or a category theory expert. I'm just wrote this
down trying to figure out the language.
So everything in this document is probably wrong.

\section{Statement of the Lemma}

The Lemma is usually stated in some form similar this:

\begin{lemma}[Yoneda]\label{yoneda} Let $\CC$ be a locally small category. Let $X$ be an
object of $\CC$, and let $F: \CC \to \Set$ be a functor from $\CC$ to the category $\Set$.
Then there is an invertible mapping between the set of natural transformations from
$\Arrows(X, -)$ to $F$ and the elements of $FX$ and this mapping is natural in both $F$
and $X$.
\end{lemma}
\goodbreak
\noindent
To make sense of this we need to look into the various parts of the statement:
\begin{itemize}
\item Categories
\item Functors
\item Natural transformations
\item Functor categories
\item Catgories and sets.
\end{itemize}
\noindent
So that's the path that we will take, and when we get to the end we'll state the result
again in various other ways.

\section{Categories}

Categories have a deliciously multi-part definition.

\begin{defn}
\label{category}
A {\it category} $\CC$ consists of:
%
\begin{itemize}
\item 
A collection of {\it objects} that we will denote with upper case letters $X, Y, Z, ...$,
and so on.  
We call this collection $\objc$. Traditionally people write just $\CC$ to mean $\objc$
when the context makes clear what is going on.
\item
A collection of {\it arrows} denoted with lower case letters $f, g, h, ...$, and so on.
Other names for {\it arrows} include {\it mappings} or {\it functions} or {\it morphims}.
We will call this collection $\Arrows(\CC)$. \end{itemize}%
The objects and arrows of a category satisfy the following conditions:
\begin{itemize}
\item
Each arrow $f$ maps one object $A \in \objc$ to another object $B \in \objc$ and we denote
this by writing $f: A \to B$. $A$ is called the {\it domain} of $f$ and $B$ the {\it
codomain}.
\item
For each pair of arrows $f:A \to B$ and $g : B \to C$ we can form a new arrow $g \circ f:
A \to C$ called the {\it composition} of $f$ and $g$. This is also sometimes written $gf$.
\item
For each $A \in \objc$ there is a function $1_A: A \to A$, called the {\it identity} at
$A$ that maps $A$ to itself. Sometimes this object is also written as $\id_A$.
\end{itemize}
\goodbreak\noindent Finally, we have the last two rules:

\begin{itemize}
\item For any $f: A \to B$ we have that $1_B \circ f$ and $f \circ 1_A$ are both equal to
$f$. 
\item Given $f: A \to B$, $g: B \to C$, $h: C\to D$ we have that $(h \circ g) \circ f = h
\circ (g \circ f)$, or alternatively $(hg)f$ = $h(gf)$. What this also means is that we
can always just write $hgf$ if we want. \end{itemize}%
\end{defn}%
\noindent
We will call the collection of all arrows from $A$ to $B$ $\Arrows_{\CC}(A, B)$. We will
usually write $\Arrows(A,B)$ when it's clear what category $A$ and $B$ come from. People
also write $\Hom(A, B)$ or $\Hom_{\CC}(A,B)$, or $\ihom(A, B)$ or just $\CC(A,B)$ to mean
$\Arrows(A,B)$. Here ``$\Hom$'' stands for homomorphism, which is a standard word for
mappings that preserve some kind of structure. Category theory, and the Yoneda Lemma, it
it turns out, is mostly about the arrows.

At this point every category theory book will list a few dozen examples of categories. 
These will have strangely truncated names like
$\cat{Measu}$ or $\cat{Htpy}$ or $\cat{Matr}$.
{The general fear of readable names in the mathematical literature is fascinating
to me, having spent most of my life trying to think up readable names in program source
code. In the ``modern'' world of \LaTeX\ there is no reason to limit names to being only four
or five random letters in length. Thus, I have done the unthinkable and written many names
out in full.}
For these short notes I think the only specific
category that we will run into is $\cat{Set}$, where the objects are sets and the arrows
are mappings between sets.

Speaking of sets, the definition of categories we were careful about not calling anything
a {\it set}. This is because some categories involve collections of things that are too
``large'' to be called sets and not get into set theory trouble. Here are two more short
definitions about this that we will need.

\begin{defn}
A category $\CC$ is called {\it small} if $\Arrows(\CC)$ is a set.
\end{defn}

\begin{defn}
A category $\CC$ is called {\it locally small} if $\Arrows_{\CC}(A,B)$ is a set for every
$A, B \in \CC$. \end{defn}%
\noindent
For the rest of this note we will only deal with locally small categories, since in the
the setup for the Lemma, we are given a category $\CC$ that is locally small.

Finally, one more notion that we'll need later is the idea of an {\it isomorphism}.

\begin{defn}
An arrow $f: X \to Y$ in a category $\CC$ is an {\it isomorphism} if there exists an arrow
$g: B \to A$ such that $gf = 1_X$ and $fg = 1_Y$. We say that the objects $X$ and $Y$ are
{\it isomorphic} to each other whenever there exists an isomorphism between them. If two
objects in a category are isomorphic to each other we write $X \iso Y$.
\end{defn}
\noindent
Note that in the category $\cat{Set}$ the isomorphisms are exactly the invertible mappings
between sets.

\section{Functors}

As we navigate our way from basic categories up to the statement of the lemma we will
travel through multiple layers conceptual abstraction. Functors are the first step up this
ladder.

Functors are the {\it arrows between categories}. That is, if you were to define the
category where the objects were all categories of some kind then the arrows would be
functors.

\goodbreak
\begin{defn}
Given two categories $\CC$ and $\DD$ a {\it functor} $F : \CC \to \DD$ is defined by two
sets of parallel rules. First:
\begin{itemize}
\item For each object $X \in \CC$ we assign an object $F(X) \in \DD$.
\item For each arrow $f: X \to Y$ in $\CC$ we assign an arrow $F(f): F(X) \to F(Y)$ in
$\DD$.
\end{itemize}
\noindent
So $F$ maps objects in $\CC$ to objects in $\DD$ and also arrows in $\CC$ to arrows in
$\DD$ such that the domains and codomains match up the right way. That is, the domain of
$F(f)$ is $F$ applied to the domain of $f$, and the codomain of $F(f)$ is $F$ applied to
the codomain of $f$. In addition the following must be true:
\begin{itemize}
\item If $f:X \to Y$ and $g: Y \to Z$ are arrows in $\CC$ then $F(g \circ f) = F(g) \circ
F(f)$ (or $F(gf) = F(g)F(f)$).
\item For every $X \in \CC$ it is the case that $F(1_X) = 1_{F(X)}$.
\end{itemize}

\end{defn}
\noindent
So, a functor consists of two mappings, one on objects and one on arrows. And, these
mappings preserve all of the structure of a category, namely domains and codomains,
composition, and identities.

If $F: \CC \to \DD$ is a functor from a category $\CC$ to another category $\DD$ and an
object $X \in \CC$ we may write $F X$ to mean $F(X)$. This is analogous to the more
compact notation for composition of arrows above.

Functors are notationally confusing because we are using one letter to denote two
mappings. So if $F: \CC \to \DD$ and $X \in \CC$ then $F(X)$ is the functor applied to the
object, which will be an object in $\DD$. On the other hand, if $f : A \to B$ is an arrow
in $\CC$ then $F(f)$ is an arrow in $\DD$. This seems obvious from the definition but in
proofs and calculations the notations will often shift back and forth without enough
context and can be very confusing.

\section{Natural Transformations}

Natural transformations are the next step up the ladder. If functors are arrows between
categories, then natural transformations are arrows between functors.
\begin{defn}
Let $\CC$ and $\DD$ be categories, and let $F$ and $G$ be functors $\CC \to \DD$. To
define a \emph{natural transformation} $\alpha$ from $F$ to $G$, we assign to each object
$X$ of $\CC$, an arrow $\alpha_X:FX\to GX$ in $\DD$, called the \emph{component} of
$\alpha$ at $X$. In addition, for each arrow $f:X\to Y$ of $\CC$, the following diagram
has to commute:
  $$
  \begin{tikzcd}
   FX \ar{r}{Ff} \ar{d}{\alpha_X} & FY \ar{d}{\alpha_{Y}} \\
   GX \ar{r}{Gf} & GY
  \end{tikzcd}
  $$
\end{defn}
\noindent
This is the first commutative diagram that I've tossed up. There is no magic here. The
idea is that you get the same result no matter which way you travel through the diagram.
So here $\alpha_Y \circ F$ and $G \circ \alpha_X$ must be equal.

We denote natural transformations as double arrows, $\alpha: F \fto G$, to distinguish
them in diagrams from functors (which are denoted by single arrows):
 $$
 \begin{tikzcd}[column sep=large]
  \CC \ar[bend left,""{below,name=F}]{r}{F} \ar[bend right,""{above,name=G}]{r}[swap]{G} & \DD
  \ar[Rightarrow,from=F,to=G,"\,\alpha"]
 \end{tikzcd}
 $$

You might wonder to yourself: what makes natural transformations ``natural''? The answer
appears to be related to the fact that you can construct them from {\it only} what is
given to you in the categories at hand. The natural transformation takes the action of $F$
on $\CC$ and lines it up exactly with the action of $G$ on $\CC$. No other assumptions or
conditions are needed. In this sense they define a relationship between functors that is
just sitting there in the world no matter what, and thus ``natural''. Another apt way of
putting this is that natural transformations give a canonical way of mapping between the
images of two functors.

As with arrows, it will be useful to define what an isomorphism means in the context of
natural transformations:

\begin{defn}
A {\it natural isomorphism} is a natural transformation $\alpha: F \fto G$ in which every
component $\alpha_X$ is an isomorphism. In this case, the natural isomorphism may be
depicted as $\alpha: F \iso G$.
\end{defn}

\section{Functor Categories}

We are almost there, but there are two more steps up the abstraction ladder. We have in
our one hand objects called functors, and we have in our other hand the natural
transformations. So the next obvious thing is to make a category out of them.

\begin{defn}
 Let $\CC$ and $\DD$ be categories. The \emph{functor category} between $\CC$ and $\DD$ is
 constructed as follows:
 \begin{itemize}
  \item Objects are functors $F: \CC \to \DD$;
  \item Morphisms are natural transformations $\alpha:F\fto G$.
 \end{itemize}
\end{defn}
\noindent
Right now you should be wondering to yourself: ``wait, does this definition actually
work?'' I have brazenly claimed without any justification that the it's OK to use the
natural transformations as arrows. Luckily it's fairly clear that this works out if you
just do everything component-wise. So if we have all of these things: 
\begin{itemize}
\item Three functors, $F: \CC \to \DD$ and $G: \CC \to \DD$ and $H:\CC \to \DD$.

\item Two natural transformations $\alpha: F \fto G$ and $\beta: G \fto H$

\item One object $X \in \CC$.
\end{itemize}
\noindent
Then you can define $(\beta \circ \alpha)(X) = \beta(X) \circ \alpha(X)$ and you get the
right behavior. Similarly, the identity transformation $1_F$ can be defined
component-wise: $(1_F)(X) = 1_{F(X)}$.

There are a lot of standard notations for this object, none of which I really like. The
most popular seems to be $[\CC, \DD]$, but you also see $\DD^{\CC}$, and various  
abbreviations like $\mathop{\mathit{Fun}}(\CC,\DD)$ or $\mathop{\mathit{Func}}(\CC,\DD)$,
or $\mathop{\mathit{Funct}}(\CC,\DD)$. I think we should just spell it out and use
$\Fun(\CC,\DD)$. So there.

Now we can define this notation:

\begin{defn}
Let $\CC$ and $\DD$ be categories, and let $F, G \in \Fun(\CC, \DD)$. Then
we'll write $\Nat(F, G)$ for the set of all natural transformations from $F$ to $G$, which
in this context is the same as the arrows from $F$ to $G$ in the functor category.
\end{defn}
\noindent
You will also see people write $\Hom(F, G)$ or $\CC(F, G)$ for this, which overloads
$\Hom$ to work on both categories and functor categories.

\section{Represented Functors}

The last conceptual step that we need is a way to construct {\it functors} from {\it
objects}. The following definition is a natural way to do this.

\begin{defn}
Given a locally small category $\CC$ and an object $X \in \CC$ we define a functor
$$
\Arrows(X,-) : \CC \to \Set
$$
using the following assignments:
\begin{itemize}
\item A mapping from $\CC \to \Set$ that assigns to each $Y \in \Objects(\CC)$ the set
$\Arrows(X,Y)$
\item A mapping from $\Arrows(\CC) \to \Arrows(\Set) $ that assigns to each arrow $f: A
\to B$ the function defined by mapping each $g: X \to A$ the arrow $f\circ g$.
\end{itemize}
\noindent
Some texts call a functor defined this way the functor {\it represented} by $X$, which
will make sense when we get back to the Lemma.
\end{defn}
\noindent
The definition for objects here is pretty clear. But the one for arrows maybe needs some
thought. Given an arrow $f: A \to B$, it should be the case that $\Arrows(X,-)$ applied to
$f$ is an arrow from $\Arrows(X,A) \to \Arrows(X,B)$. Since $\CC$ is locally small $f \in
\Arrows(\Set)$. Now, if $g: X \to A$ we have that $(f \circ g): X \to B$ is the arrow we
want. This mapping on $f$ is called the {\it post-composition} mapping, since it composes
$f$ {\it after} $g$. As an overloaded abuse of notation we will also write $\Arrows(X,
-)(f) = \Arrows(X,f) = f \circ -$. Note how the placeholder symbols mean completely
different things on each side of this formula. Some people write $f_*$ for $f \circ -$,
but that doesn't seem as fun.

Other notations for this functor include just $\CC(X,-)$, $\Hom(X, -)$, $\Hom_{\CC}(X,
-)$, $H^X$ and $h^X$. In my notational convention we probably should have written this as
$\Arrows_{\CC}(X, -)$. Some people also call this kind of functor a {\it hom-functor}.

Finally, we can use the above definition to characterize an important relationship between
objects and functors:

\begin{defn}
 Let $\cat{C}$ be a category. A functor $F:\cat{C}\to\Set$ is called \emph{representable}
 if it is naturally isomorphic to the functor $\Arrows_\cat{C}(X,-):\cat{C}\to\Set$ for some
 object $X$ of $\cat{C}$. In that case we call $X$ the \emph{representing object}. 
\end{defn}
\noindent
Using objects of one kind to represent objects of another kind is the bread and butter of many
different kinds of mathematical inquiry. Often it allows  you to study something simple in place
of something complicated (e.g. a single object rather than a whole functor).
While this definition is not used directly in our discussion of the Lemma, it's closely
related to what the Lemma ultimately says about functors.

We can actually return to the statement of the Yoneda Lemma now, but 
I'm going to take one tangent first to define one more handy piece of
the conceptual framework of category theory.

\section{Opposites and Duals}

This section is the one optional side quest that I'm including because it comes up in
comparing different versions of the Lemma. Duality in mathematics comes up in a lot of
different ways. Covering it all is way beyond the scope of these notes. But the following
definition is a basic part of category theory so it's worth including.

\begin{defn}
Let $\CC$ be a category. Then we write $\CCop$ for the {\it opposite} or {\it dual}
category of $\CC$, and define it as follows:
\begin{itemize}
\item The objects of $\CCop$ are the same as the objects of $\CC$.
\item $\Arrows(\CCop)$ is defined by taking each arrow $f :X \to Y$ in $\Arrows(\CC)$ and
flipping their direction, so we put $f': Y \to X$ into $\Arrows(\CCop)$. In particular for
$X, Y \in \Objects(\CC)$ we have $\Arrows_{\CC}(A, B) = \Arrows_{\CCop}(B, A)$ (or $\CC(A,
B) = \CCop(B, A)$.
\item Composition of arrows is the same, but with the arguments reversed.
\end{itemize}
\end{defn}
\noindent
The {\it principle of duality} then says, informally, that every categorical definition,
theorem and proof has a dual, obtained by reversing all the arrows.

Duality also applies to functors.

\begin{defn}
Given categories $\CC$ and $\DD$ a {\it contravariant} functor from $\CC$ to $\DD$ is a
functor $F: \CCop \to \DD$ where:
\begin{itemize}
\item $F(X) \in \Objects(\DD)$ for each $X \in \objc$.
\item For each arrow $f \in \Arrows(\CC)$ an arrow $F(X): Y \to X$.
\end{itemize}
\goodbreak
\noindent
In addition
\begin{itemize}
\item For any two arrows $f, g \in \Arrows(\CC)$ where $g \circ f$ is defined we have
$F(f) \circ F(g) = F(g \circ f)$.
\item For each $X \in \Objects(\CC)$ we have $1_{F(X)} = F(1_X)$
\end{itemize}

\end{defn}
\noindent
Note how the arrows go backwards when they need to. With this terminology in mind, we call
regular functors from $\CC \to \DD$ {\it covariant}.

Now we have all the language we need to look at the statement of the Lemma again.

\section{The Lemma Returns}

So, here is what we wrote down before, now that we know what all the words mean:

\begin{lemma}[Yoneda]\label{yoneda} Let $\CC$ be a locally small category and $F:\CC \to \Set$ a functor.
Let $X$ be an object of $\CC$.
Then there is an invertible mapping between
$\Nat(\Arrows(X, -),F)$
and the elements of $FX$. In addition this
mapping is natural in both $F$ and $X$.
\end{lemma}
\noindent
So now we can break it down:
\begin{itemize}
\item A locally small category $\CC$ means that all of the collections of arrows are sets.
So the functor $\Arrows(X,-)$ maps objects in $\CC$ to sets of arrows.
\item $\Nat(\Arrows(X, -),F)$ are the natural transformations from $\Arrows(X,-)$ to $F$.
\item In principle $\Nat(\Arrows(X, -),F)$ could be a giant complicated thing.
\item But actually it can only be as large as $FX$.
\item In other words, every natural transformation is the same as
an element of the set $FX$. You can see how this is similar to 
the idea of a representable functor, except for natural transformations.
\item Which is pretty amazing.
\end{itemize}
\noindent
The recipe for achieving this is also easy as pie. You can define the mapping you need just by
assigning each transformation its value at the identity. So we map each $\alpha: \Arrows(X, -) \fto F$ 
to $\alpha_X(1_X) \in FX$.

To write this in the covariant language, you just change $\Arrows(X, -)$ to $\Arrows(-, X)$, which
switches the direction of all the arrows.

Finally, here are some other ways people write the result, and how their statements translate to 
my notational scheme.

\goodbreak\noindent
This statement is due to Tom Leinster, and uses the contravariant language.

\begin{lemma}[Yoneda]   
\label{yoneda-leinster}
Let $\CC$ be a locally small category.  Then
%
$$
\pshf{\CC}(\h_X, F)
\iso
F(X)
$$
%
naturally in $X \in \CC$ and $F \in \pshf{\CC}$.  
\end{lemma}
\noindent
Here $[\CCop, \Set]$ is the category of functors from $\CCop$ to $\Set$ and $\h_X$ is the corresponding
arrow functor. The notation $\pshf{\CC}(\h_X, F)$ denotes the arrows in the category $\pshf{\CC}$ between 
$\h_X$ and $F$, so it's the same as $\Nat(\h_X, F)$

\bigskip\noindent
Emily Rhiel writes it down like this:

\begin{lemma}[Yoneda]\label{yoneda-rhiel} Let $\CC$ be a locally small category and $X \in
\CC$. Then for any functor $F : \CC \to \Set$ there is a bijection
$$
\Hom(\CC(X,-), F) \iso FX
$$
that associates each natural transformation $\alpha:\CC(X,-) \fto F$ with the element
$\alpha_X(1_X) \in FX$. Moreover, this correspondence is natural in both $X$ and $F$.
\end{lemma}
\noindent
Here $\Hom(\CC(X,-), F)$ means $\Nat(\Arrows(X,-), F)$.

\bigskip\noindent
Peter Smith does this

\begin{lemma}[Yoneda]\label{yoneda-smith} For any locally small category $\CC$, object $X \in
\CC$, and functor $F:\CC \to \Set$ we have  $\Nat(\CC(X,-),F) \iso FX$ both naturally in
$X \in \CC$ and $F \in [\CC, \Set]$
\end{lemma}

\bigskip\noindent
Paolo Perrone has has the contravariant version, and uses the standard term "presheaf" for a functor from
$\CCop$ to $\Set$. 
\begin{lemma}[Yoneda]\label{yoneda-perrone} Let $\cat{C}$ be a category, let $X$ be an object of
 $\cat{C}$, and let $F:\cat{C}^\op\to\Set$ be a presheaf on $\cat{C}$. Consider the map
 from
 $$
 \Hom_{[\cat{C}^\op,\Set]} \bigl(\Hom_\cat{C} (-,X) , F \bigr) \to FX
 $$
 assigning to a natural transformation $\alpha:\Hom_\cat{C} (-,X)\fto F$ the element
 $\alpha_X(\id_X)\in FX$, which is the value of the component $\alpha_X$ of $\alpha$ on
 the identity at $X$. 

 This assignment is a bijection, and it is natural both in $X$ and in $F$.
\end{lemma}
\noindent
Here he writes $\Hom_{[\cat{C}^\op,\Set]}$ to mean the arrows in the category $[\cat{C}^\op,\Set]$,
which are the natural transformations.

\bigskip\noindent
Finally, Peter Johnstone has my favorite, relatively concrete statement:

\begin{lemma}[Yoneda]\label{yoneda-johnstone} Let $\CC$ be a locally small category, let $X$ be an object of $\CC$
and let $F:\CC \to \Set$ a functor. Then

(i)  there is a bijection between natural transformations $\CC(X, -) \fto F$

(ii) the bijection in (i) is natural in both $F$ and $X$.
\end{lemma}
\section{Who I Stole From}

\small
\begin{itemize}
\item \url{https://arxiv.org/abs/1912.10642}

\item \url{https://math.jhu.edu/~eriehl/context/}

\item \url{https://arxiv.org/abs/1612.09375}

\item \url{https://www.logicmatters.net/2018/01/29/category-theory-a-gentle-introduction/}

\item \url{http://pi.math.cornell.edu/~dmehrle/notes/partiii/cattheory_partiii_notes.pdf}

\item \url{http://www.julia-goedecke.de/pdf/CategoryTheoryNotes.pdf}

\item \url{https://www.youtube.com/watch?v=SsgEvrDFJsM}

\end{itemize}

And of course, a bit from

\url{https://en.wikipedia.org/wiki/Categories_for_the_Working_Mathematician}
\end{document}
