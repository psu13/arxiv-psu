%!TEX root = yoneda-speedrun-pal-eu.tex
% these include amsmath and that can cause trouble in older docs.

\usepackage{microtype}
\usepackage{datetime2}

\begin{document}

\title{\Large A Yoneda Speedrun}
\author{\large Pete Su}
\date{\normalsize 2021-10-30 \\ {\footnotesize (Updated 2023-10-02)}}

\maketitle

\section{The Big Picture}

The Yoneda Lemma is a basic and beloved result in category theory. Even though it is
called a ``lemma'', a word usually used to describe a minor result that you prove on the
way to the main event, the Yoneda lemma {\it is} a main event. It is a result that
expresses one of the main goals of category theory: it characterizes universal facts about
general abstract constructs.

Its statement is deceivingly simple \cite{Riehl2016}:

\pg
Let $\CC$ be a locally small category. Let $X$ be an object of $\CC$, and let $F: \CC \to
\Set$ be a functor from $\CC$ to the category $\Set$. Then there is an invertible mapping
$$
\Hom(\CC(X, -),F) \iso FX
$$
that associates each natural transformation $\alpha:\CC(X,-) \fto F$ with the element
$\alpha_X(1_X) \in FX$. Moreover, this correspondence is natural in both $X$ and $F$.

\pg
But as Sean Carroll famously wrote about general relativity, ``\dots, these statements are
incomprehensible unless you sling the lingo'' \cite{carroll}.

I am going to do the following dumb thing: having stated a version of the lemma above I'm
going to define only the parts of the category theory needed to explain what the lingo
means. There are five or six layers of abstraction that I will 
try to explain. As for the larger meaning of the result itself, you are on your own. 

In the spirit of video game speedruns \cite{lobos}, we will skip entire interesting areas
of category theory in the name of getting to the end of our ``game'' as fast
as possible. Clearly this will be no substitute for really learning the subject. Any of
the references listed at the end will be a good place to start to better understand the
whole game.

\pg
{\bf Note}: I am not a mathematician or a category theory expert. I just wrote this down
trying to figure out the language. So everything in this document is probably wrong.

\goodbreak
\section{Categories}

Categories have a deliciously chewy multi-part definition.

\begin{defn}
\label{category}
A {\it category} $\CC$ consists of:
%
\begin{itemize}
\item 
A collection of {\it objects} that we will denote with upper case letters $X, Y, Z, ...$,
and so on.  
We call this collection $\objc$. Traditionally people write just $\CC$ to mean $\objc$
when the context makes clear what is going on.
\item
A collection of {\it arrows} denoted with lower case letters $f, g, h, ...$, and so on.
Other names for {\it arrows} include {\it mappings} or {\it functions} or {\it morphisms}.
We will call this collection $\Arrows(\CC)$. \end{itemize}%
The objects and arrows of a category satisfy the following conditions:
\begin{itemize}
\item
Each arrow $f$ connects one object $A \in \objc$ to another object $B \in \objc$ and we denote
this by writing $f: A \to B$. $A$ is called the {\it source} (or {\it domain}) of $f$ and $B$ the {\it target} (or 
{\it codomain}). Source and target are somewhat more intuitive terms, but domain and codomain connect the language
to functions in other areas of mathematics.
\item
For each pair of arrows $f:A \to B$ and $g : B \to C$ we can form a new arrow $g \circ f:
A \to C$ called the {\it composition} of $f$ and $g$. This is also sometimes written $gf$.
\item
For each $A \in \objc$ there is an arrow $1_A: A \to A$, called the {\it identity} at
$A$ that maps $A$ to itself. Sometimes this object is also written as $\id_A$.
\end{itemize}
\goodbreak\noindent Finally, we have the last two rules:

\begin{itemize}
\item For any $f: A \to B$ we have that $1_B \circ f$ and $f \circ 1_A$ are both equal to
$f$. 
\item Given $f: A \to B$, $g: B \to C$, $h: C\to D$ we have that $(h \circ g) \circ f = h
\circ (g \circ f)$, or alternatively $(hg)f$ = $h(gf)$. What this also means is that we
can always just write $hgf$ if we want. \end{itemize}%
\end{defn}%
\noindent
Given a category $\CC$ and objects $A, B$ in $\CC$ we write 
$\Arrows_{\CC}(A, B)$ to mean the collection of all arrows 
from $A$ to $B$ in $\CC$ if we are being maximally careful. In practice
we will usually write $\Arrows(A,B)$ because the $\CC$ subscript is tedious and
it's clear what category $A$ and $B$ come from. People
also write $\Hom(A, B)$ or $\Hom_{\CC}(A,B)$, or $\ihom(A, B)$ or just $\CC(A,B)$ to mean
$\Arrows(A,B)$. Here ``$\Hom$'' stands for homomorphism, which is a standard word for
mappings that preserve some kind of structure. Category theory, and the Yoneda lemma,
it turns out, is mostly about the arrows.

\goodbreak
I have broken with well established tradition in mathematical writing and mostly
spelled out names for clarity rather than engaging in the strange and random abbreviations
that I see in most category theory texts. The general fear of readable names in the
mathematical literature is fascinating to me, having spent most of my life trying to think
up readable names in program source code. Life is too short to deal with names like
$\mathop{\mathit{ob}}$, or $\cat{Htpy}$, or $\cat{Matr}$. Luckily, for this note the only
specific category that we will run into is the straightforwardly named $\Set$, 
where the objects are sets and the
arrows are mappings between sets.

Speaking of sets, in the definition of categories we were careful about not calling
anything a {\it set}. This is because some categories involve collections of things that
are too ``large'' to be called sets and not get into set theory trouble. Here are two more
short definitions about this that we will need.

\begin{defn}
A category $\CC$ is called {\it small} if $\Arrows(\CC)$ is a set.
\end{defn}

\begin{defn}
A category $\CC$ is called {\it locally small} if $\Arrows_{\CC}(A,B)$ is a set for every
$A, B \in \CC$. \end{defn}%
\noindent
For the rest of this note we will only deal with locally small categories, since in the
setup for the lemma, we are given a category $\CC$ that is locally small.

Finally, one more notion that we'll need later is the idea of an {\it isomorphism}.

\begin{defn}
An arrow $f: X \to Y$ in a category $\CC$ is an {\it isomorphism} if there exists an arrow
$g: Y \to X$ such that $gf = 1_X$ and $fg = 1_Y$. We say that the objects $X$ and $Y$ are
{\it isomorphic} to each other whenever there exists an isomorphism between them. If two
objects in a category are isomorphic to each other we write $X \iso Y$.
\end{defn}
\noindent
Note that in the category $\Set$ the isomorphisms are exactly the invertible mappings
between sets. An invertible mapping is also called a {\it bijection} (because it's
injective and surjective, you see), so you will see that word sometimes.

\section{Functors}

As we navigate our way from basic categories up to the statement of the lemma we will
travel through multiple layers conceptual abstraction. At the base of this ladder
are the categories which themselves are already an abstraction of the many  ways
that we express ``mathematical structures''. But we have much higher to climb.
Functors are the first step up.

Functors are the {\it arrows between categories}. That is, if you were to define a
category where the objects were all categories of some kind then the arrows would be
functors.

\goodbreak
\begin{defn}
Given two categories $\CC$ and $\DD$ a {\it functor} $F : \CC \to \DD$ is defined by two
sets of parallel rules. First:
\begin{itemize}
\item For each object $X \in \CC$ we assign an object $F(X) \in \DD$.
\item For each arrow $f: X \to Y$ in $\CC$ we assign an arrow $F(f): F(X) \to F(Y)$ in
$\DD$.
\end{itemize}
\noindent
So $F$ maps objects in $\CC$ to objects in $\DD$ and also arrows in $\CC$ to arrows in
$\DD$ such that the sources and targets match up the right way. That is, the source of
$F(f)$ is $F$ applied to the source of $f$, and the target of $F(f)$ is $F$ applied to
the target of $f$. In addition the following must be true:
\begin{itemize}
\item If $f:X \to Y$ and $g: Y \to Z$ are arrows in $\CC$ then $F(g \circ f) = F(g) \circ
F(f)$ (or $F(gf) = F(g)F(f)$).
\item For every $X \in \CC$ it is the case that $F(1_X) = 1_{F(X)}$.
\end{itemize}

\end{defn}
\noindent
Thus, the mappings that make up a functor preserve all of the structure of the source category
in its target, namely the sources and targets of arrows, composition, and the identities.

If $F: \CC \to \DD$ is a functor from a category $\CC$ to another category $\DD$,
$X$ and $Y$ are objects in $\CC$, and $f: X \to Y$ is an arrow in $\CC$ we may write 
$F X$ to mean $F(X)$ and $Ff$ to mean $F(f)$.
This is analogous to the more compact notation for composition of arrows above.

Functors can be notationally confusing because we are using one name to denote two
mappings. So if $F: \CC \to \DD$ and $X \in \CC$ then $F(X)$ is the functor applied to the
object, which will be an object in $\DD$. On the other hand, if $f : A \to B$ is an arrow
in $\CC$ then we also write $F(f) \in \DD$ for the functor applied to the arrow. This makes
sense but can be a little weird. Sometimes in proofs and calculations the notations will 
shift back and forth without enough context and can be disorienting.

\section{Natural Transformations}

Natural transformations are the next step up the ladder. If functors are arrows between
categories, then natural transformations are arrows between functors.
\begin{defn}
Let $\CC$ and $\DD$ be categories, and let $F$ and $G$ be functors $\CC \to \DD$. To
define a \emph{natural transformation} $\alpha$ from $F$ to $G$, we assign to each object
$X$ of $\CC$, an arrow $\alpha_X:FX\to GX$ in $\DD$, called the \emph{component} of
$\alpha$ at $X$. 

In addition, for each arrow $f:X\to Y$ of $\CC$, the following diagram
has to commute:
  $$
  \begin{tikzcd}
   FX \ar{r}{Ff} \ar{d}{\alpha_X} & FY \ar{d}{\alpha_{Y}} \\
   GX \ar{r}{Gf} & GY
  \end{tikzcd}
  $$
\end{defn}
\noindent
This is the first commutative diagram that I've tossed up. There is no magic here. The
idea is that you get the same result no matter which way you travel through the diagram.
So here $\alpha_Y \circ F$ and $G \circ \alpha_X$ must be equal.

We write natural transformations with double arrows, $\alpha: F \fto G$, to distinguish
them in diagrams from functors (which are written with single arrows):
 $$
 \begin{tikzcd}[column sep=large]
  \CC \ar[bend left,""{below,name=F}]{r}{F} \ar[bend right,""{above,name=G}]{r}[swap]{G} & \DD
  \ar[Rightarrow,from=F,to=G,"\,\alpha"]
 \end{tikzcd}
 $$

You might wonder to yourself: what makes natural transformations ``natural''? The answer
appears to be related to the fact that you can construct them from {\it only} what is
given to you in the categories at hand. The natural transformation takes the action of $F$
on $\CC$ and lines it up exactly with the action of $G$ on $\CC$. No other assumptions or
conditions are needed. In this sense they define a relationship between functors that is
just sitting there in the world no matter what, and thus ``natural''. Another apt way of
putting this is that natural transformations give a canonical way of moving between the
images of two functors \cite{Goedecke}.

As with arrows, it will be useful to define what an isomorphism means in the context of
natural transformations:

\begin{defn}
A {\it natural isomorphism} is a natural transformation $\alpha: F \fto G$ in which every
component $\alpha_X$ is an isomorphism. In this case, the natural isomorphism may be
depicted as $\alpha: F \iso G$.
\end{defn}

\section{Functor Categories}

In the last two sections we have defined functors, and then the natural
transformations. Given that functors and natural transformations look a lot like objects and arrows,
the next obvious thing is to use them to make a new kind of category.

\goodbreak
\begin{defn}
 Let $\CC$ and $\DD$ be categories. The \emph{functor category} from $\CC$ to $\DD$ is
 constructed as follows:
 \begin{itemize}
  \item The objects are functors $F: \CC \to \DD$;
  \item The arrows are natural transformations $\alpha:F\fto G$.
 \end{itemize}
\end{defn}
\noindent
Right now you should be wondering to yourself: ``wait, does this definition actually
work?'' I have brazenly claimed without any justification that the it's OK to use the
natural transformations as arrows. Luckily it's fairly clear that this works out if you
just do everything component-wise. So if we have all of these things: 
\begin{itemize}
\item Three functors, $F: \CC \to \DD$ and $G: \CC \to \DD$ and $H:\CC \to \DD$.

\item Two natural transformations $\alpha: F \fto G$ and $\beta: G \fto H$

\item One object $X \in \CC$.
\end{itemize}
\noindent
Then you can define $(\beta \circ \alpha)(X) = \beta(X) \circ \alpha(X)$ and you get the
right behavior. Similarly, the identity transformation $1_F$ can be defined
component-wise: $(1_F)(X) = 1_{F(X)}$.

There are a lot of standard notations for the functor category, none of which I really
like. The most popular seems to be $[\CC, \DD]$, but you also see $\DD^{\CC}$, and various
abbreviations like $\shortFun(\CC,\DD)$ or $\func(\CC,\DD)$,
or $\funct(\CC,\DD)$. I think we should just spell it out and use
$\Fun(\CC,\DD)$. So there.

Now we can define this notation:

\begin{defn}
Let $\CC$ and $\DD$ be categories, and let $F, G \in \Fun(\CC, \DD)$. Then we'll write
$\Nat(F, G)$ for the set of all natural transformations from $F$ to $G$, which in this
context is the same as the arrows from $F$ to $G$ in the functor category.
\end{defn}
\noindent
You will also see people write $\Hom(F, G)$, $\Hom_{[\CC,\DD]}(F,G)$, or 
$[\CC,\DD](F,G)$ for this. Or, if $\cat{K}$ is a functor category then 
people will write $\Hom_{\cat{K}}(F,G)$ or
$\cat{K}(F,G)$ for this.

\section{Representing Functors}

The next conceptual step that we need is a way to relate {\it functors} to {\it objects}.
The following definition is a natural way to do this once you see how it works but is also
probably the most confusing definition in these notes.

\begin{defn}
Given a locally small category $\CC$ and an object $X \in \CC$ we define the functor
$$
\Arrows(X,-) : \CC \to \Set
$$
using the following assignments:
\begin{itemize}
\item A mapping from $\CC \to \Set$ that assigns to each $A \in \Objects(\CC)$ the set
$\Arrows(X,A)$
\item A mapping from $\Arrows(\CC) \to \Arrows(\Set) $ that assigns to each arrow $f: A
\to B$ a mapping $f_*$ defined by $f_*(g) = f\circ g$ where $g$ is an arrow from $X$ to $A$.
\end{itemize}
\end{defn}
\goodbreak\noindent
Here is a good picture of what is going on  \cite{Riehl2016}:%
\footnote{Added this in 2023 because I did not notice it before. Sigh.}
\begin{center}
\begin{tikzpicture}[commutative diagrams/every diagram]
\matrix[matrix of math nodes, name=m, commutative diagrams/every cell,row sep=.4em, column sep=1em] {
\CC & & \Set\\
A & \mapsto &  \Arrows(X,A)\\ 
& \mapsto & \\
B & \mapsto & \Arrows(X,B)\\};
\path[commutative diagrams/.cd, every arrow, every label]
(m-1-1) edge node {$\Arrows(X,-)$}  (m-1-3)
(m-2-1) edge node[swap] {$f$} (m-4-1)
(m-2-3) edge node {$f_*$}(m-4-3);
\end{tikzpicture}
\end{center}
\vskip-.5em
\noindent
Check over this picture in your head, and note that there are {\it two} mappings,
and two different kinds of placeholders: one for objects and one for arrows.

The notation $\Arrows(X,-)$ is a bit strange. The idea is that we have
defined a mapping with two arguments, but then fixed the object $X$. Then we use the
``$-$'' symbol as a placeholder for the second argument. So $\Arrows(X,-)$ operates on
all the other objects in $\CC$, named by the dummy variable $A$. This is
a bit of an abuse of notation since we are apparently using the symbol ``$\Arrows$'' to mean
two different things (one is a set, the other a functor). Oh well.

The mapping for the arrows works the same way. Given $A,B \in
\CC$ and an arrow $f: A \to B$, it should be the case that $\Arrows(X,-)$ applied to $f$
is an arrow that maps $\Arrows(X,A) \to \Arrows(X,B)$. We will call this arrow $f_*$. If
$g: X \to A$ is in $\Arrows(X,A)$ then the value that we want for $f_*$ at $g$ is $f_*(g)
= (f \circ g): X \to B$. This mapping is called the {\it post-composition} map of $f$
since we apply $f$ {\it after} $g$. You also see it written as $f \circ -$. The {\it
pre-composition} map is then $f^*$ or $- \circ f$.

Other notations for this functor include $\Hom(X, -)$, $\Hom_\CC(X, -)$, $\h^X$, ${\littleh}^X$, and
just plain $\CC(X,-)$. In my notation we should have written this as $\Arrows_{\CC}(X,
-)$, but I'm lazy. This kind of functor is also called a {\it hom-functor}.
\goodbreak
Finally, we can give two more important definitions.
\begin{defn}
\label{represented}
Given an object $X \in \CC$ we call the functor $\Arrows(X,-)$ defined above the functor
{\it represented} by $X$.
\end{defn}
\noindent
In addition, we can characterize another important relationship between objects and
functors:

\begin{defn}
 Let $\cat{C}$ be a category. A functor $F:\cat{C}\to\Set$ is called \emph{representable}
 if it is naturally isomorphic to the functor $\Arrows_\cat{C}(X,-):\cat{C}\to\Set$ for
 some object $X$ of $\cat{C}$. In that case we call $X$ the \emph{representing object}. 
\end{defn}

\section{Opposites and Duals}

Next we move a bit sideways. Duality in mathematics comes up in
a lot of different ways. Covering it all is way beyond the scope of these notes. But the
following definition is a basic part of category theory so it's worth including.

\begin{defn}
Let $\CC$ be a category. Then we write $\CCop$ for the {\it opposite} or {\it dual}
category of $\CC$, and define it as follows:
\begin{itemize}
\item The objects of $\CCop$ are the same as the objects of $\CC$.
\item $\Arrows(\CCop)$ is defined by taking each arrow $f :A \to B$ in $\Arrows(\CC)$ and
flipping its direction, so we put $f': B \to A$ into $\Arrows(\CCop)$. In particular for
$A, B \in \Objects(\CC)$ we have $\Arrows_{\CC}(A, B) = \Arrows_{\CCop}(B, A)$ (or $\CC(A,
B) = \CCop(B, A)$).
\item Composition of arrows is the same, but with the arguments reversed.
\end{itemize}
\end{defn}
\noindent
The {\it principle of duality} then says, informally, that every categorical definition,
theorem and proof has a dual, obtained by reversing all the arrows.

Duality also applies to functors.

\begin{defn}
Given categories $\CC$ and $\DD$ a {\it contravariant} functor from $\CC$ to $\DD$ is a
functor $F: \CCop \to \DD$ where:
\begin{itemize}
\item We have an object $F(A) \in \Objects(\DD)$ for each $A \in \objc$.
\item For each arrow $f : A \to B \in \Arrows(\CC)$ we have an arrow $F(f): FB \to FA$ in $\Arrows(\DD)$.
\end{itemize}
\noindent
In addition:
\begin{itemize}
\item For any two arrows $f, g \in \Arrows(\CC)$ where $g \circ f$ is defined we have
$F(f) \circ F(g) = F(g \circ f)$.
\item For each $A \in \Objects(\CC)$ we have $1_{F(A)} = F(1_A)$
\end{itemize}

\end{defn}
\noindent
Note how the arrows and composition go backwards when they need to. With this terminology
in mind, we call regular functors from $\CC \to \DD$ {\it covariant}.

\section{Yoneda Again}

Now we have all the language we need to look at the statement of the lemma again. So, here
is what we wrote down before, more verbosely, and in my notation.

\begin{lemma}[Yoneda]\label{yoneda} Let $\CC$ be a locally small category, $F:\CC \to
\Set$ a functor, and $X \in \objc$. We can define a mapping from $\Nat(\Arrows(X, -),F)
\to FX$ by assigning each transformation $\alpha: \Arrows(X, -) \fto F$ the value
$\alpha_X(1_X) \in FX$. This mapping is invertible and is natural in both $F$ and $X$.
\end{lemma}
\noindent
So now we can break it down:
\begin{itemize}
\item In principle the natural transformations from $\Arrows(X, -) \fto F$ could be a
giant complicated thing.
\item But actually it can only be as large as $FX$. The fact that this mapping is
invertible implies that $\Nat(\Arrows(X, -),F)$ and  $FX$ are isomorphic (that is,
$\Nat(\Arrows(X, -),F) \iso FX$).
\item In other words, every natural transformation from $\Arrows(X, -)$ to $F$ is the same
as an element of the set $FX$. In particular, all we need to know is how $\alpha_X(1_X)$
is defined to know how any of the natural transformations are defined.
\item Which is pretty amazing.
\end{itemize}
To write this in the dual language, you just change $\Arrows(X, -)$ to $\Arrows(-, X)$,
which switches the direction of all the arrows and the order of composition in the
composition maps.

So with that, here are some other ways people write the result, and how their lingo
translates to my notational scheme. As one last bit of terminology, in some of the
definitions below the word {\it bijection} is used to mean an invertible mapping.

\pg
This statement is due to Tom Leinster \cite{Leinster}, and uses the contravariant
language.

\begin{lemma}[Yoneda]   
\label{yoneda-leinster}
Let $\CC$ be a locally small category.  Then
%
$$
\pshf{\CC}(\h_X, F)
\iso
F(X)
$$
%
naturally in $X \in \CC$ and $F \in \pshf{\CC}$.  
\end{lemma}
\noindent
Here $[\CCop, \Set]$ is the category of functors from $\CCop$ to $\Set$ and 
$\h_X$ means $\Arrows(-,X)$.
The notation $\pshf{\CC}(\h_X, F)$ denotes the arrows in the functor
category $\pshf{\CC}$ between $\h_X$ and $F$, so it's the same as $\Nat(\h_X, F)$.

\pg
Emily Riehl's \cite{Riehl2016} version is what I used at the top:

\begin{lemma}[Yoneda]\label{yoneda-Riehl} Let $\CC$ be a locally small category and $X \in
\CC$. Then for any functor $F : \CC \to \Set$ there is a bijection
$$
\Hom(\CC(X,-), F) \iso FX
$$
that associates each natural transformation $\alpha:\CC(X,-) \fto F$ with the element
$\alpha_X(1_X) \in FX$. Moreover, this correspondence is natural in both $X$ and $F$.
\end{lemma}
\noindent
Here $\Hom(\CC(X,-), F)$ means $\Nat(\Arrows(X,-), F)$. I think this is my favorite
``standard'' way of writing this.

\pg
Peter Smith \cite{Smith} does this:

\begin{lemma}[Yoneda]\label{yoneda-smith} For any locally small category $\CC$, object $X
\in \CC$, and functor $F:\CC \to \Set$ we have  $\mathop{\shortNat}(\CC(X,-),F) \iso
FX$ both naturally in $X \in \CC$ and $F \in [\CC, \Set]$.
\end{lemma}
\noindent
He uses the $[\CC, \Set]$ notation for the functor category, and ``$\mathop{\shortNat}$''
where we use ``$\Nat$''.

\pg
Paolo Perrone \cite{Perrone} writes the dual version, and uses the standard term
"presheaf" to mean a functor from $\CCop$ to $\Set$.

\begin{lemma}[Yoneda]\label{yoneda-perrone} Let $\cat{C}$ be a category, let $X$ be an
 object of $\cat{C}$, and let $F:\cat{C}^\op\to\Set$ be a presheaf on $\cat{C}$. Consider
 the map from
 $$
 \Hom_{[\cat{C}^\op,\Set]} \bigl(\Hom_\cat{C} (-,X) , F \bigr) \to FX
 $$
 assigning to a natural transformation $\alpha:\Hom_\cat{C} (-,X)\fto F$ the element
 $\alpha_X(\id_X)\in FX$, which is the value of the component $\alpha_X$ of $\alpha$ on
 the identity at $X$. 

This assignment is a bijection, and it is natural both in $X$ and in $F$.
\end{lemma}
\noindent
Here he writes $\Hom_\CC$ for $\Arrows_\CC$ and $\Hom_{[\cat{C}^\op,\Set]}$ to mean the
arrows in the functor category $[\cat{C}^\op,\Set]$, which are the natural
transformations.

\pg
Finally, Peter Johnstone \cite{Johnstone} has my favorite, relatively concrete statement:

\begin{lemma}[Yoneda]\label{yoneda-johnstone} Let $\CC$ be a locally small category, let
$X$ be an object of $\CC$ and let $F:\CC \to \Set$ be a functor. Then
\begin{enumerate}
\item[(i)]  there is a bijection between natural transformations $\CC(X, -) \fto F$ and the 
elements of $FX$; and

\item[(ii)] the bijection in (i) is natural in both $F$ and $X$.
\end{enumerate}

\end{lemma}

\section{One More Thing}

Now your reward for having climbed all the way up this abstraction ladder with me is yet
another abstraction!

Suppose we have $\CC$ and $X$ as given in the statement of the lemma, and then we are given another object $Y \in \CC$. 
We can apply the lemma by substituting
$\Arrows(Y,-)$ for the functor $F$. Then
$$
\Nat(\Arrows(X, -),\Arrows(Y,-)) \iso \Arrows(Y,-)(X) = \Arrows(Y,X)
$$
Note the order of the arguments! We can also write:
$$
\Arrows(X,Y) \iso \Nat(\Arrows(-, X),\Arrows(-,Y))
$$
Each of the functors $\Arrows(-, X)$ maps from $\CC^\op \to \Set$ because that's how we
defined the represented functors. So now let's jump up one more level of abstraction. We
define a functor that maps objects to the functors that they represent, and arrows to the
natural transformations between those functors. Given an object $Y\in\CC$ define the functor
$\Yo:\CC \to \Fun(\CC^\op, \Set)$ by
$$
\Yo(Y) = \Arrows(-, Y) : \CC^\op \to \Set
$$
and given an arrow $f: A \to B$ with $A,B \in \CC$ define
$$
\Yo(f) = f_* = (f \circ -) : \Arrows(-,A) \fto \Arrows(-,B)
$$
This definition has the same ``shape'' as the one for represented functors, but we have
abstracted over all the objects and arrows. Also note that we could have also defined this
as $\Yo:\CC^\op \to \Fun(\CC, \Set)$ using duality. All that changes is the 
order of the arguments in the functors.

The Yoneda lemma can now be used to prove that these mappings are invertible, so
$\Yo$ is what is called an {\it embedding} of the category $\CC$ inside the functor
category $\Fun(\CC^\op, \Set)$. Thus $\Yo$ is called the {\it Yoneda embedding}, and you
can read about the rest of the details in the references.

This construction tells us why people say things like, ``Every object in a category can be
understood by understanding the maps into (or out of) it.''  This statement can be made precise:
\begin{cor}
Let $\CC$, $X$, and $Y$ be given as above.
\begin{itemize}
\item $X$  and $Y$ are isomorphic if and only if for every object $A \in \CC$, the sets
$\Arrows(X, A)$ and $\Arrows(Y, A)$ are naturally isomorphic.
\item $X$ and $Y$ are isomorphic if and only if the functors that they represent are
naturally isomorphic. In particular, if X and Y represent the same functor then they must
be isomorphic.
\end{itemize}
\end{cor}

\section{Final Thoughts}

To close, a few final thoughts, and no more abstraction.

First, the modern internet is something of an endless treasure trove for the amateur category theory nerd.
I have listed my favorite references at the end of this note, and it's amazing that you can download
almost them all for free, and sometimes with source code! When trying to understand something that is as 
deep an abstraction stack as this result it is very useful to be able to look at it from many different
points of view.\footnote{Which, of course, is very much in the spirit of the subject matter.}
So, I am grateful for all of the sources. 

Second, I wish I could have thought of a better notation for the represented functor
than $\Arrows(X,-)$ with all that placeholder nonsense. I don't like how the placeholders
can stand in for anything you want and how their meaning can shift and change in different contexts.
But, even with those problems it's better than hiding the definition behind yet another
layer of naming (e.g. $\h_X$), which is the only other obvious choice. 

Third, you might have found my use of $\Yo$ for the Yoneda embedding to be frivolous, and perhaps
childish. And I would have agreed. But then I read in multiple sources that
the Yoneda embedding is sometimes denoted by \!\!\yo, the hiragana kana for ``Yo''.
Given this, how could I resist?

Finally, I need to shout out the 
\href{https://www.youtube.com/watch?v=SsgEvrDFJsM}{excellent tutorial video by Emily Riehl} 
that demonstrates how this result works
the specific category of matrices \cite{Riehl2020}. The whole Yoneda picture suddenly
became more clear while I was watching this talk the second time.
Her book, \href{https://emilyriehl.github.io/files/context.pdf}{\it Category
Theory in Context}, is also excellent \cite{Riehl2016}. Recommended.

\newpage

\section{Cheat Sheet}
\small
\renewcommand{\arraystretch}{1.1}
\begin{tabularx}{.9\linewidth}{l X}
$\CC$, $\CC^\op$ & Categories and opposite categoies. \\
$\Objects$ & Objects in a category category $\CC$. Often just written $\CC$.\\
$\Arrows(\CC)$ & Arrows in a category. \\
$\Arrows_\CC(X,Y)$ & Arrows between two objects.  Also written $\Arrows(X,Y)$ or
$\Hom(X,Y)$ or $\Hom_\CC(A, B)$ or just $\CC(X,Y)$. \\
$f: X \to Y$ & An arrow from $X$ to $Y$.\\
$g \circ f$, $gf$ & Composition of arrows.\\
$X \iso Y$ & Isomorphism.\\
$F:\CC \to\DD$ & A functor from $\CC$ to $\DD$.\\
$\alpha: F \fto G$ & Natural transformation.\\
$\Fun(\CC, \DD)$ & Functor category between $\CC$ and $\DD$. Also $[\CC,\DD]$ or
$\DD^\CC$.\\
$\Nat(F, G)$ & The collection of natural transformations from $F$ to $G$. Also written
$[\CC,\DD](F,G)$, or~$\shortNat(F,G)$ or just~$\Hom(F,G)$. \\
$\Arrows(X, -)$ & The represented or ``arrow'' functor for $X$. Also called the ``hom''
functor and written $\CC(X,-)$, $\h^X$, $\littleh^X$, $\ihom_X$, or $\Hom(X,-)$. \\
$f \circ - $, $- \circ f$ & Pre- and post-composition maps. Also written $f_*$ and
$f^*$.\\
$\Yo$ & Yoneda Embedding.\\
\end{tabularx}

\normalsize

\begin{thebibliography}{100}
\parskip 0pt
\small
\bibitem{carroll} Sean Carroll,
\href{https://preposterousuniverse.com/wp-content/uploads/2015/08/grtinypdf.pdf}{\it A
No-Nonsense Introduction to General Relativity},  2001.

\bibitem{Cheng} Eugenia Cheng, \href{https://www.cambridge.org/core/books/joy-of-abstraction/00D9AFD3046A406CB85D1AFF5450E657}
{\it The Joy Of Abstraction}, 2022.

\bibitem{Goedecke} Julia Goedecke,
\href{http://www.julia-goedecke.de/pdf/CategoryTheoryNotes.pdf}{\it Category Theory
Notes}, 2013.

\bibitem{Johnstone} Peter Johnstone,
\href{http://pi.math.cornell.edu/~dmehrle/notes/partiii/cattheory_partiii_notes.pdf}{\it
Category Theory}, notes written by David Mehrle, 2015.

\bibitem{Leinster} Tom Leinster, \href{https://arxiv.org/abs/1612.09375}{\it Basic Category
Theory}, 2016.

\bibitem{lobos} LobosJr, \href{https://www.youtube.com/watch?v=ImMOdTxtf-s}{\it Dark Souls
1 Speedrun, Personal Best}, 2013.

\bibitem{ML} Saunders Mac Lane,
\href{https://link.springer.com/book/10.1007%2F978-1-4757-4721-8}{\it Categories for the
Working Mathematician}, Second Edition, Springer, 1978.

\bibitem{Perrone} Paolo Perrone, \href{https://arxiv.org/abs/1912.10642}{\it Notes on
Category Theory with examples from basic mathematics}, 2019.

\bibitem{Riehl2016} Emily Riehl, \href{https://math.jhu.edu/~eriehl/context/}{\it Category
Theory in Context}, Dover, 2016.

\bibitem{Riehl2020} Emily Riehl, \href{https://www.youtube.com/watch?v=SsgEvrDFJsM}{ACT
2020 Tutorial: {\it The Yoneda lemma in the category of matrices}}.

\bibitem{Smith} Peter Smith,
\href{https://www.logicmatters.net/2018/01/29/category-theory-a-gentle-introduction/}{\it
Category Theory: A Gentle Introduction}, 2019.

\end{thebibliography}