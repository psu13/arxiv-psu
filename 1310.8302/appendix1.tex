% !TEX root = epistemic.tex
\section{APPENDIX 1}\label{sec:appendix1}
We derive the inequality in Eq.~\eqref{eq:initial_inequality_main} as follows.

\subsection{The measure space}
For an ontic state space $\Lambda$, let us define the set ${\Gamma:=\Lambda \times \mathbb{R}^+}$. A positive function $f(\lambda) \ge 0$ that is integrable on $\Lambda$ 
(i.e. $\int_\Lambda f(\lambda)\,\mathrm{d}\lambda  < \infty $) defines a region $F$ of $\Gamma$ (i.e.~the area under $f$), where:
 \begin{equation}\label{eq:function_region}
 F:=\{(\lambda,x) \,|\, 0 \le x \le f(\lambda)\}
 \end{equation}
The union and intersection of two such regions are
\begin{eqnarray*}
F \cup G&=&\{(\lambda, x) \,|\, 0 \le x \le \max(f(\lambda),g(\lambda))\} \\
F \cap G&=&\{(\lambda, x) \,|\, 0 \le x \le \min(f(\lambda),g(\lambda))\}
\end{eqnarray*}

The volume measure on the space is $\mathrm{d}\gamma=\mathrm{d}\lambda \times \mathrm{d}x$, where $\textrm{d}x$ is the Lebesgue measure.  This gives
\[
\nu(F)=\int_F \,\mathrm{d}\gamma=\int_\Lambda \,\mathrm{d}\lambda  \int_0^{f(\lambda)} \,\mathrm{d}x =\int_\Lambda f(\lambda)
\,\mathrm{d}\lambda 
\]
and
\ba\label{eq:measure}
\nu(F \cup G )&=& \int \max(f(\lambda),g(\lambda))\,\mathrm{d}\lambda  \nonumber\\
\nu(F \cap G) &=& \int \min(f(\lambda),g(\lambda))\,\mathrm{d}\lambda 
\ea

For quantum states, a region $\Phi$ is defined by the corresponding epistemic states $\mu_\phi$. Eq.~\eqref{eq:function_region} becomes:
\be\label{eq:epistemicregion}
\Phi=\{(\lambda, x) \,|\,0\le x\le\mu_\phi(\lambda)\}.
\ee

\subsection{Bonferroni Inequality}
Now, using the first Bonferroni inequality \cite{Rohatgi2011}, on any measure space with measure $\nu$ we have:
\be \label{eq:Bonferroni}
\nu\left( \bigcup_k A_k \right) \ge \sum_{k} \nu\left( A_k \right) \\
-\sum_{k<k'} \nu\left(A_k\cap A_{k'} \right)
\ee
Consider a family of states $\{\ket{e^\alpha_i}\}_{i,\alpha}$, such that $\alpha$ labels a basis, and $i$ a basis element. Using Eq.~\eqref{eq:epistemicregion} each such state defines a region $E^\alpha_i$. Let $\ket{c}$ be a fixed state, with region $C$. We shall use Eq.~\eqref{eq:Bonferroni} with $A_k=A_{\alpha,i}=C \cap E^\alpha_i$, i.e.~we replace the index $k$ with the pair of indices $(\alpha,i)$.
\begin{multline} \label{eq:Bonferroni2}
\nu\left( \bigcup_{\alpha,i} C \cap E^\alpha_i  \right) \ge \sum_{\alpha,i} \nu\left( C \cap E^\alpha_i \right) \\
- \sum_{\substack{\alpha<\beta\\i,j}} \nu\left(C \cap E^\alpha_i \cap E^\beta_j \right)  - \sum_{\substack{\alpha=\beta\\i<j}} \nu\left(C \cap E^\alpha_i \cap E^\beta_j \right) 
\end{multline}
Note that from the normalization of $\mu_c$ we have
\be
\nu\left(\bigcup_{\alpha,i} C \cap E^\alpha_i\right)\le\nu(C)=1
\ee
Eq.~\eqref{eq:Bonferroni2} then becomes:
\begin{multline}
\sum_{\alpha,i} \nu\left(C \cap E^\alpha_i\right) \le 1 + \sum_{\substack{\alpha<\beta\\ i,j}} \nu\left(C \cap E^\alpha_i \cap E^\beta_j  \right) \\
+\sum_{\substack{\alpha=\beta\\i<j}} \nu\left(C \cap E^\alpha_i \cap E^\beta_j  \right)
\end{multline}
It will be useful to substitute:
\be
\nu\left(C \cap E^\alpha_i \cap E^\beta_j  \right) \le \nu\left(E^\alpha_i \cap E^\beta_j  \right)
\ee
for the cases where $\alpha=\beta$. Using Eq.~\eqref{eq:measure}, we then have:
\begin{multline}\label{eq:initial_inequality}
\sum_{\substack{\alpha,i}}  \int_\Lambda \min \left(\mu_c(\lambda),\mu_{e^\alpha_i}(\lambda) \right)\mathrm{d}\lambda \le   \\
1 +  \sum_{\substack{\alpha<\beta\\i,j}}\int_\Lambda \min\left(\mu_c(\lambda),\mu_{e^\alpha_i}(\lambda), \mu_{e^\beta_j}(\lambda)\right)\mathrm{d}\lambda    \\
  +  \sum_{\substack{\alpha\\i<j}}\int_\Lambda \min\left(\mu_{e^\alpha_i}(\lambda),\mu_{e^\alpha_j}(\lambda)\right)\mathrm{d}\lambda
\end{multline}


\subsection{Noise}
Consider a set of $n$ quantum states $\{\psi_1, \dots,\psi_n\}$ and a measurement $M$ with outcomes $\{f_1, \dots, f_n\}$. In an ontological model, the experimentally observed frequencies $R[f_i | \psi_j]$ are obtained by using the corresponding response functions $\xi_M(f_i|\lambda)$ and epistemic states $\mu_{\psi_j}$:
\be
\int_\Lambda \xi_M(f_i|\lambda)\mu_{\psi_j}(\lambda) \,\mathrm{d}\lambda = R[f_i | \psi_j].
\ee
Now, since $\forall i, \, \min_j \left(\mu_{\psi_j}(\lambda)\right) \le  \mu_{\psi_i}(\lambda)$,
\be
\int_\Lambda \xi_M(f_i|\lambda)\min_j \left(\mu_{\psi_j}(\lambda)\right) \,\mathrm{d}\lambda \le  R[f_i | \psi_i] \,.
\ee
From the normalization constraint ${\sum_{i=1}^n \xi_M(f_i|\lambda)=1}$, we then have:
\be\label{eq:bound} 
\int_\Lambda \min_j \left(\mu_{\psi_j}(\lambda)\right) \,\mathrm{d}\lambda \le \sum_{i=1}^n R[f_i | \psi_i].
\ee

Using this, along with Eqs.~\eqref{eq:epsilonbound1}, \eqref{eq:epsilonbound2} and \eqref{eq:cond_prob_def} in Eq.~\eqref{eq:initial_inequality}, we obtain
\begin{multline}
\label{eq:initial_inequality2}
\sum_{\substack{\alpha,i}}  \int_\Lambda \min \left(\mu_c(\lambda),\mu_{e^\alpha_i}(\lambda) \right)\mathrm{d}\lambda \le \\ 
1+   3 \sum_{\substack{\alpha<\beta \\ i,j}} \epsilon(c, e^\alpha_i,e^\beta_j) +  2 \sum_{\substack{\alpha \\ i < j }} \epsilon(e^\alpha_i,e^\alpha_j) \,.
\end{multline}
Then suppose, as in Theorem \ref{maintheorem}, that the ontological model satisfies $\omega_C(\mu_\psi,\mu_\phi) \geq k \,\omega_Q(\psi,\phi)$ for all pairs of states of a system of dimension $d\geq 4$, for some constant $k$. This implies that
\begin{equation}\label{overlapboundbykd}
\int_\Lambda \min \left(\mu_c(\lambda),\mu_{e^\alpha_i}(\lambda) \right)\mathrm{d}\lambda \ge k \left( 1 - \sqrt{1 - \frac1d}\right).
\end{equation}
If $d$ is power prime, then $d+1$ mutually unbiased bases can be found, as in the proof of Theorem~\ref{maintheorem}. In this case, substituting Eq.~(\ref{overlapboundbykd}) in Eq.~\eqref{eq:initial_inequality2} yields Eq.~\eqref{eq:initial_inequality_main} as required.