% Useful packages
\usepackage{amsmath, amssymb}
\usepackage{titlesec}
\usepackage{soul}
\usepackage[dvipsnames]{xcolor}
\usepackage{graphicx}
\usepackage{needspace} %this ensures that my section titles remain as one block and are not broken into 2 bits even at the end of a page
\usepackage{listings}
\usepackage{longtable} %multipage tabular
\usepackage{lettrine}
\usepackage{ebgaramond}
\usepackage{subcaption}
\usepackage{pdfpages}
\usepackage{wrapfig}

\usepackage{lmodern}

\renewcommand{\baselinestretch}{1.125}      % Allow paragraphs to breathe by spreading the lines further

% \newcommand{\fakesection}[1]{%
%   % \par\refstepcounter{section}% Increase section counter
%   \sectionmark{#1}% Add section mark (header)
%   % \setcounter{secnumdepth}{0}
%   \addcontentsline{toc}{section}{\protect\numberline{\thesection}#1}% Add section to ToC
%   % Add more content here, if needed.
% }
% %https://tex.stackexchange.com/questions/129978/how-to-remove-section-subsection-titles


\usepackage[titles]{tocloft}

\renewcommand{\cftsecleader}{\cftdotfill{\cftdotsep}}

% Set page size and margins
% Replace `letterpaper' with `a4paper' for UK/EU standard size
% \usepackage[letterpaper,top=2cm,bottom=2cm,left=3cm,right=3cm,marginparwidth=1.75cm]{geometry}
\usepackage[total={5.5in, 8.5in},top=3cm,bottom=4cm,left=3cm,right=3cm]{geometry}

\usepackage{array}
% to allow wrapping of lines in tables

\usepackage{fancybox}

\usepackage[round]{natbib}

\sloppy %calms down all those hbox warnings

%%%%%%%%%%%%%%%%%%%%%%%%%%%%%%%%%%%%%%%%%%%
%%%%%%%%%%%%%   S T Y L E   %%%%%%%%%%%%%%%
%%%%%%%%%%%%%%%%%%%%%%%%%%%%%%%%%%%%%%%%%%% 

% Change bullet style
\renewcommand{\labelitemi}{|}


%===============[ SECTIONS ]===============

% TITLE %
\makeatletter
\renewcommand\maketitle
  {\begin{center}
   {
   \Large\scshape{\@title}
   }
   \end{center}
  }
\makeatother

% SECTION
\titleformat{\section}[display]
    {\clearpage\flushright}
    {\fontsize{96}{50}\selectfont\thesection.}
    {-5pt}
    {\Huge}
    [\vspace{-1.5ex} \hspace{1.3ex} \rule{0.8\textwidth}{0.2pt} ]
\titlespacing*{\section}{0pt}{0pt}{\baselineskip}


% SUBSECTION 

%\counterwithout{subsection}{section}

\titleformat{\subsection}[block]
    {\Large\scshape}
    {}
    {0pt}
    {\needspace{5\baselineskip}\rule{0.8\textwidth}{0.2pt} \\ \vspace{0.5ex} \thesubsection. \quad }
    [\vspace{-1.5ex} \rule{0.8\textwidth}{0.2pt} ] 
\titlespacing{\subsection}{-1cm}{\baselineskip}{\baselineskip}

% SUBSUBSECTION
\renewcommand{\thesubsubsection}{\arabic{subsubsection}}
\setcounter{secnumdepth}{3}

\titleformat{\subsubsection}{\scshape}{}{0.2ex}
{\thesubsubsection. \quad}
[ \vspace{-2ex} \rule{\textwidth}{0.15pt} ]

% PARAGRAPHS
%\titleformat{\paragraph}{\bfseries}{}{0.2ex}
%{\theparagraph. }
%\renewcommand{\theparagraph}{\arabic{paragraph}}
%\setcounter{secnumdepth}{4}


\setlength{\parindent}{0pt}
\setlength{\parskip}{0.5\baselineskip plus 2pt}


% TABLE OF CONTENTS

\renewcommand\contentsname{Table of contents}

% CAPTIONS

% Label format
\DeclareCaptionLabelFormat{custom}
{%
      #1 \textbf{(#2)}
}
% Separator style
\DeclareCaptionLabelSeparator{custom}{--}
% Caption format    
\DeclareCaptionFormat{custom}
{%
    #1#2\small #3
}
\captionsetup
{
    format=custom,%
    labelformat=custom,%
    labelsep=custom,
    width=0.7\textwidth
}


% COMMENTS

\newcommand{\key}[1]{{\bfseries\color{LimeGreen}{\textit{[Keywords: #1]}}}}
\newcommand{\todo}[1]{{\bfseries\color{Red}{[#1]}}}

\newcommand{\jlm}[1]{{\color{LimeGreen}{\textit{#1}}}}
\newcommand{\syn}[1]{\colorbox{White!50!Peach}{[#1]}}

\newcommand{\question}[1]{
\setlength{\fboxrule}{1.5em}
\setlength{\fboxsep}{1em} 
\cornersize{2}
\begin{center}
    \ovalbox{#1}
\end{center}
}

\newcommand{\boxoftruth}[1]{%
    \setlength{\fboxsep}{10pt}% Adjust the spacing as desired
    %\setlength{\fboxrule}{0.01pt} %bro doesn't want me to have a thinner line :(
    \begin{center}
        \boxed{#1}
    \end{center}
}


\newcommand{\quothauthor}[1]{
    \begin{flushright} 
    \vspace{-1ex} \rule{0.4\textwidth}{0.1pt}\\
    --- \textit{\cite{#1}}
    \end{flushright}}



\newenvironment{quoth}
{\begin{center}
\begin{longtable}{|p{14cm}}}
{\end{longtable}\end{center}}


\newenvironment{sectionauthor}
{\begin{flushright}\it}
{\end{flushright}}




%Aquamarine,Peach




%================[ SYNTAX ]================


\definecolor{codegreen}{rgb}{0,0.6,0}
\definecolor{codegray}{rgb}{0.5,0.5,0.5}
\definecolor{codepurple}{rgb}{0.58,0,0.82}
\definecolor{codekeyword}{rgb}{0.5, 0.5, 0.2}
\definecolor{Orchid}{rgb}{0.686,0.447,0.690}

\definecolor{CLIprompt}{rgb}{0.4,0.6,0}
\definecolor{xspeckeyword}{rgb}{0.6, 0.4, 0.3}


\lstdefinestyle{terminal}{    
    commentstyle=\it\color{MidnightBlue},
    keywordstyle=\color{MidnightBlue},
    %backgroundcolor=\color{Black!40!Periwinkle!30},
    numberstyle=\tiny\color{codegray},
    stringstyle=\bfseries\color{codepurple},
    basicstyle=\ttfamily\footnotesize,
    breakatwhitespace=false,         
    breaklines=true,                 
    captionpos=b,                    
    keepspaces=true,                 
    numbers=none,                                      
    showspaces=false,                
    showstringspaces=false,
    showtabs=false,                  
    tabsize=2,
    otherkeywords = {user@here:\$, XSPEC12>,PLT>, PyXspec>},
    keywordstyle={\color{CLIprompt}\bfseries},
    xleftmargin=.25in,
    xrightmargin=.25in
}

\lstdefinestyle{file}{    
    commentstyle=\color{codegreen},
    keywordstyle=\color{codekeyword},
    numberstyle=\tiny\color{codegray},
    stringstyle=\bfseries\color{codepurple},
    basicstyle=\ttfamily\footnotesize,
    breakatwhitespace=false,         
    breaklines=true,                 
    captionpos=b,                    
    keepspaces=true,                 
    numbers=left,                    
    numbersep=10pt,                  
    showspaces=false,                
    showstringspaces=false,
    showtabs=false,                  
    tabsize=2,
    xrightmargin=.3in
}
 
\lstdefinelanguage{xspec}{
    basicstyle=\ttfamily\small,
    columns=fullflexible,
    morecomment=[s][\color{Orchid}\bfseries]{[}{]},
    morecomment=[l]{\#},
    morecomment=[l]{;},
    commentstyle=\color{gray}\ttfamily,
    %
    morekeywords = [2]{lmod, cpd, setplot, dummyrsp, dummy, fakeit, model, Model},
    keywordstyle = [2]{\color{xspeckeyword}\bfseries},
    %
    otherkeywords = {XSPEC12>,PLT>},
    keywordstyle = {\color{CLIprompt}\bfseries}
}

\makeatletter \def\lst@gkeywords@sty{\color{CLIprompt}\bfseries} % here to fix what seems to be a bug with lstlisting where I give a straightforward definition to whatever internal function it's missing

%%%%%%%%%%%%%%%%%%%%%%%%%%%%%%%%%%%%%%%%%%%
%%%%%%%%%   S H O R T H A N D   %%%%%%%%%%%
%%%%%%%%%%%%%%%%%%%%%%%%%%%%%%%%%%%%%%%%%%% 

% Code
\newcommand{\relxill}{\texttt{relxill}}
\newcommand{\xspec}{\texttt{Xspec}}
\newcommand{\pyxspec}{\texttt{PyXspec}}

\newcommand{\PLT}{\texttt{PLT}}

% Astrophysics symbols


\newcommand{\feka}{FeK$\alpha$ }

\newcommand{\Msun}{{\ensuremath\rm M_{\odot}}}
\newcommand{\Lsun}{\ensuremath{{\rm L_{\odot}}}}


% Observatories & missions
\newcommand{\lisa}{\textsc{lisa}}
\newcommand{\strobex}{\textsc{strobe-x}}
\newcommand{\athena}{\textsc{athena}}
\newcommand{\xrism}{\textsc{xrism}}
\newcommand{\axis}{\textsc{axis}}
\newcommand{\lynx}{\textsc{lynx}}
\newcommand{\pta}{\textsc{pta}}

\newcommand{\rin}[1][]{%
  \ifx\relax#1\relax
    \ensuremath{r_{\rm in}}%
  \else
    \ensuremath{r_{{\rm in}, #1}}%
  \fi
}

\newcommand{\rout}[1][]{%
  \ifx\relax#1\relax
    \ensuremath{r_{\rm out}}%
  \else
    \ensuremath{r_{{\rm out}, #1}}%
  \fi
}

\newcommand{\rg}[1][]{%
  \ifx\relax#1\relax
    \ensuremath{r_{g}}%
  \else
    \ensuremath{r_{g, #1}}%
  \fi
}

\newcommand{\risco}[1][]{%
  \ifx\relax#1\relax
    \ensuremath{r_{\rm ISCO}}%
  \else
    \ensuremath{r_{#1{\rm (ISCO)}}}%
  \fi
}

\newcommand{\rl}[1][]{%
  \ifx\relax#1\relax
    \ensuremath{r_{\rm L}}%
  \else
    \ensuremath{r_{#1{\rm (L)}}}%
  \fi
}

\newcommand{\rms}[1][]{%
  \ifx\relax#1\relax
    \ensuremath{r_{\rm MS}}%
  \else
    \ensuremath{r_{#1{\rm (MS)}}}%
  \fi
}



\newcommand{\dotdotdot}{
\medskip
\begin{center}
    [...]
\end{center}
\medskip
}



\newcommand{\hly}[2]{\colorbox{#1}{\parbox{\textwidth}{#2}}}

\usepackage{hyperref}
\hypersetup{colorlinks=true
,breaklinks=true
,urlcolor=blue
,anchorcolor=blue
,citecolor=blue
,filecolor=blue
,linkcolor=blue
,menucolor=blue
,linktocpage=true,
bookmarksopen=true,
bookmarksnumbered=true,
bookmarksopenlevel=10
}

% make sure there is enough TOC for reasonable pdf bookmarks.
\setcounter{tocdepth}{3}

