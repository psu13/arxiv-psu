\section{Machine Learning \& Artificial Intelligence}
\begin{sectionauthor}
    Yasmeen Asali (Yale University)
\end{sectionauthor}
\vspace{20pt}


\noindent Imagine you have a cat and a dog, and you want to create a simple rule for telling them apart. You might say, ``Cats have pointy ears, and dogs have floppy ears." This is your basic \textit{algorithm}. An algorithm just means a set of clear and specific instructions. So for this example, your algorithm might be: look at the ears, and if they're pointy, you say it's a cat; if they're floppy, you say it's a dog.

But what if you come across a cat with floppy ears or a dog with pointy ears? Your simple rule might not work very well in those cases. That's where machine learning comes to the rescue! Now, instead of relying on a simple rule, you gather lots and lots of pictures of cats and dogs. You show these pictures to a computer and say, ``This is a cat," or ``This is a dog". Machine learning refers to how your computer can learn from looking at lots of examples and get better at recognizing things.

\subsection{How do Computers Learn?}

Let's continue with our cat and dog analogy. As we feed cat and dog pictures to the computer, it will start to come up with rules on its own to decide whether the animal in a picture is a cat or a dog. Here's how it works:
\begin{enumerate}
    \item \textbf{Seeing Examples:} You feed the computer thousands of cat and dog pictures. It doesn't know anything about cats or dogs at first, but as it looks at all these pictures, it starts to notice things on its own. It might see that cats often have whiskers, and dogs have wet noses.
    \item \textbf{Learning Patterns:} The computer starts to learn from these examples. It learns to recognize patterns, like pointy ears, whiskers, and wet noses. It figures out that these are important clues to tell cats and dogs apart.
    \item \textbf{Decision-Making:} Now, when you show the computer a new picture, it looks for these clues. If it sees pointy ears and whiskers, it might confidently say, ``This is a cat." If it sees floppy ears and a wet nose, it might say, ``This is a dog."
\end{enumerate}

Importantly, the computer has come up with its set of rules on its own! It can be really hard for a human to sit down and write a set of rules that can differentiate every single type of dog from every single type of cat. With so many different breeds, the rules need to get increasingly complex to account for all the possible combinations of features. The really special thing about machine learning is that the computer can learn to recognize those patterns on its own, the same way that a baby human will learn to recognize the difference between cats and dogs without having to think through a checklist of features. 

\vspace{15pt}

\textsc{Training}\\

\vspace{-1cm}

\strut\hrulefill

\vspace{5pt}

The process of seeing examples and learning how to make decisions about them is called \textit{training}. Once a machine learning algorithm is trained, we can use it on new examples it's never seen! There are two main ways to train a ML algorithm: \textit{supervised} learning and \textit{unsupervised} learning. The basic difference between these two methods is that supervised learning uses labeled data, while unsupervised using does not require labeled data.  

\textbf{Supervised Learning}: Let's think about our cat and dog sorting example. When the computer is still learning, we are labeling the pictures as cats or dogs so it can figure out the difference between them. This is called supervised learning, because we are providing the computer with labels that inform its learning. Then, for each subsequent or new image we show the computer, we can ask it to label the name image either cat or dog. 

\textbf{Unsupervised Learning}: Now, think of sorting cat and dog pictures without any labels. You put similar images together without someone telling you which group is which. You just notice the similarities and differences. In unsupervised learning, the computer works similarly. It looks for patterns or groups in the data without being given specific labels. It's like exploring and discovering on its own. The computer might group similar pictures without being told what's in each picture, so it might not necessarily create just two groups. Maybe the computer finds patterns based on breed and it creates more than two groups separating out types of cats and types of dogs! Unsupervised learning is really useful for data that we don't already have human labels for. 

\subsection{Machine Learning in Astrophysics}

Astronomy generates an enormous amount of data from telescopes and space missions. Machine learning can help scientists sift through this data, identify patterns, and make sense of it all. For instance, ML algorithms can detect and classify celestial objects, like stars and galaxies, by analyzing the light we detect. They can also recognize transient events, such as supernovae, gravitational wave signals, or exoplanet transits, by picking out patterns in the data that may be otherwise hard to identify. Here are a few examples of the ways in which machine learning is used to aid astronomical research:

\begin{itemize}
    \item \textbf{Galaxy Classification with Galaxy Zoo}: Galaxy Zoo is a citizen science project that uses the collective power of human pattern recognition to generate large labeled training sets for a machine learning classification algorithm. Participants are usually given a set of questions about the observed galaxies. These questions might include whether the galaxy is spiral or elliptical, the presence of distinctive features like bars or spiral arms, and other characteristics relevant to astronomers studying galaxy morphology. The classifications provided by the citizen scientists are aggregated and used by astronomers to further their understanding of galaxy evolution, structure, and other properties. The large-scale classification efforts facilitated by Galaxy Zoo allow researchers to process big datasets much more quickly than would be possible with automated methods alone! 
    \item \textbf{Predicting Solar Flares with the Solar Dynamics Observatory}: Machine learning is used to study day-to-day changes on our Sun based on data from the space-based Solar Dynamics Observatory (SDO). The SDO generates a vast amount of data, including images and videos of the Sun in different wavelengths. Machine learning models can analyze historical data from the SDO to identify patterns and precursors associated with solar flares. These models can then be used for solar flare prediction, providing valuable information for space weather forecasting and mitigating potential impacts on satellite communications and power grids.
    \item \textbf{Hunting for Exoplanets}: The Kepler/K2 observatory has collected an immense amount of data on more than 150,000 stars over several years. Machine learning accelerates the analysis process, allowing us to discover exoplanets that might be difficult for traditional methods to detect. For instance, some exoplanet systems host multiple planets orbiting the same star. The signals from these systems can overlap and become intricate, making them challenging for human observers to differentiate. Machine learning excels in recognizing complex patterns within data, enabling the identification of multiple-planet systems with greater efficiency and accuracy.

\end{itemize}


These examples represent just a glimpse of the many ways ML is transforming astronomy. As ML applications continue to evolve, it's increasingly clear that this powerful tool not only revolutionizes scientific research but also quietly influences various facets of our everyday lives. ML is used by social media companies to sort recommended posts, it's used in the facial recognition software that can unlock your phone, even the auto-complete suggestions in text messaging apps leverage ML. Keep an eye out for the subtle yet impactful presence of machine learning in your own life! 