\section{FAQ}
%\addtocontents{toc}{\protect\setcounter{tocdepth}{0}}

\vspace{-5pt}

\subsection{So what is the difference between an astronomer and an astrophysicist?}

There is none! (At least not anymore ... traditionally, the distinction was whether or not spectroscopy was being used in research.) Now, when someone uses the term \emph{astronomer} or \emph{astrophysicist}, they will use them interchangeably. Colloquially within the field, some people will reserve the term \emph{astronomer} for observers and use \emph{astrophysicist} as the general term regardless of focus on instrumentation/observation/theory. In truth, everyone working in the field is doing astrophysics (i.e., studying the physical mechanisms that govern the universe) and almost no one is doing astronomy in the classical sense of the term, so \emph{astrophysicist} often fits the field better.

A different perspective comes from the ``guidance'' you'll sometimes hear that if asked what you do, \textit{astronomer} will continue the conversation and \textit{astrophysicist} will end it. From experience, people hear these terms the same way, so even this is not a foolproof way to decide which term to use. 

There are also some subfields of astrophysics that like to further differentiate themselves with their own names. Cosmologists are astrophysicists who work on cosmology and, though it is less pervasive as a term, exoplaneteers work on exoplanets/planetary science. Rarely, if ever, will cosmologists introduce themselves as \emph{astrophysicists} -- generally they will say that they are a \emph{cosmologist} or a \emph{physicist}.

\subsection{What is the difference between a physicist and an astrophysicist?}

An astrophysicist is just a type of physicist, so the distinction is one of specificity. Astrophysics is a subfield of physics along with atomic/molecular/optical, condensed matter, high energy, quantum, ... so saying that someone is an astrophysicist is like saying that someone is a condensed matter physicist.

Some confusion arises from the fact that astrophysics is housed in a separate department at some institutions. This is generally due to the historic significance of astronomy (in many universities in the United States, for instance, astronomy was one of the first departments established after the college was founded) and the fact that it is a large and well-funded subfield. To some extent, this is no different than a biophysicist working in a biophysics institute, but it does mean that astronomy departments have more educational autonomy (like running their own classes), which is an advantage that is often not afforded to institutes.

\pagebreak
\strut
\vspace{-3cm}

\subsection{What kind of education/training is necessary to become an astrophysicist?}

Long-term research jobs in astrophysics almost always require a PhD. To work at a national lab or a university as a scientist (or professor in the latter case), the usual track is:
\begin{itemize}
    \item[-]$\sim4$ years for a bachelor's degree (generally in physics, potentially in astronomy/astrophysics)
    \item[-]\textit{Possibly $\sim2$ years for a master's degree -- though not required for PhD programs in the United States, it is increasingly common for PhD applicants to hold a master's}
    \item[-]$\sim 6$ years for a PhD -- in STEM, PhD programs are fully funded, meaning that you will not have to pay tuition and will actually receive a livable (though generally non-competitive compared to industry) wage for the teaching/research you do  
    \item[-] Generally 3 - 6 years in postdoctoral research positions
\end{itemize} 
After those 10 - 15 years, you will be competitive for \textit{permanent} positions like tenure-track professorships at universities or scientist roles at nationally funded facilities.

There are other positions adjacent to astronomy that require much less in the way of preparation. For instance, telescope operators are absolutely integral to the science astronomers do, and may only have a bachelor's degree (ideally in physics/astronomy). Science writing positions are generally open to those with only an undergraduate degree, but may require more of a background in journalism.

Because graduate school admissions are increasingly competitive, it is becoming more and more usual for prospective applicants to do post-baccalaureate research at a university (frequently their undergraduate alma mater) in order to get full-time research experience. These positions often last one or two years and are paid comparably to graduate research positions. Other students may choose to do a master's degree before (re-)applying for PhD programs. Depending upon the institution, these programs may or may not charge tuition and/or pay a stipend.

\subsection{Should I study physics or astrophysics?}

At the undergraduate level, the answer is generally physics, given the expectation that students entering graduate programs will have the same knowledge of advanced math, classical mechanics, statistical mechanics, electrodynamics and electromagnetism, and quantum mechanics that a physics undergraduate affords. If there is the option, you may be able to major in physics with a concentration in astronomy/astrophysics, so that your elective courses are cross-listed grad/undergrad courses in astro, or you may be able to double major in astronomy/astrophysics. Many standalone astronomy majors were established for students with an interest in the science, but no aspirations to pursue it professionally. There are exceptions, though, and you should confirm on an institution-specific level. 

Some people will also apply to astro grad programs from geology/planetary science, math, or computer science majors, but this is a non-standard path and, since graduate admissions are incredibly competitive, there is no guarantee of success in getting into a PhD program this way (or, frankly, even with a physics background).

In graduate school, deciding between a PhD in physics and astronomy/astrophysics is much more dependent on the institution and supervisor with whom you would like to work. Standard advice is that astro-specific programs are generally better resourced than the astro portion of a physics department, and you will be able to focus more heavily on research-relevant coursework, but even within a university with both departments, you may prefer the feel and requirements (for coursework, teaching, etc.) of the physics department. The other deciding factor may be that you prefer to take a physics vs. astronomy qualifying exam (or vice versa), but this is something to think about/decide much further down the line.

\subsection{How do I get involved with undergraduate research?}

There are both funded and unfunded opportunities for undergraduate research. The simplest positions to get are unfunded research assistantships, which are generally with a professor in your department. The tried-and-true approach is to cold contact professors who do research that interests you and ask if they have any opportunities for undergraduates to participate. This may be somewhat easier with faculty you know and have impressed -- i.e., you've already taken a class with them and performed very well -- but not all research groups have roles for undergraduates, so you may have to reach out to people you don't know.

These unfunded positions may become funded after you've proven yourself; there are usually also university-wide undergraduate research grants you can apply for to support work you're doing in a professor's group.

The other avenue for getting a funded position is to apply for more prestigious undergraduate research fellowships, like the NSF REU (only open to US citizens) or the DAAD RISE, which fund summer research on specific projects that are external to your undergraduate institution. These are very competitive, so you will generally need existing research experience for a successful application, though there is some priority given to students from liberal arts colleges or similar schools where research is generally de-prioritized and less available.

\subsection{How do I know what research opportunities exist at a school?}

This is a consideration at both the undergraduate and graduate level. For undergraduates, though, it is somewhat less pressing -- you are not locked in by the research you did in college and can (usually) change your focus in graduate school. (It serves to remember that astrophysics is already a sub-field, so further narrowing is generally not crucial until you begin a PhD.) 

It may be important to you to have lots of options for research direction/modality. In that case, it's best to consider this before you select your undergraduate institution. Schools with a large astro presence (generally due to having a dedicated department and/or an institute of some kind) will often have more diversity in terms of the questions being covered. At the same time, smaller institutions may offer more research opportunities to undergraduates. Certainly, liberal arts colleges and predominantly undergraduate institutions will often have substantially less breadth in available astrophysics research, but \textit{undergraduate} research positions may be more plentiful and come with more responsibility.

If you know in advance that you are going to pursue (astro)physics, check the department website for universities you are considering attending for college. They will often list research areas on which their faculty focus and may discuss specific projects on which students have worked. Some departments will also tout their commitment to undergraduate research and list fellowship opportunities or open positions. Some universities require an undergraduate thesis in order to finish a bachelor's degree. Theses generally signal a department commitment to involving college students in research early, so this can be another positive indicator.

It also does not hurt to ask -- once admitted and at the point of deciding between schools, find the contact information for your prospective department's director of undergraduate studies (or equivalent) and ask explicitly about undergraduate research opportunities. (Note that a non-response is not a reason to worry. Academics are overworked and often receive a deluge of emails -- if you don't hear back, your message was likely buried, not ignored.)

\subsection{What skills will make me more competitive for undergraduate research positions?}

At this point, coding is critical for most aspects of astrophysical research. There are a number of languages that people use -- commonly IDL, Julia, C/C++, and Python, though Python is by far the most used in astro research. Completing free beginner courses from DataCamp or Codeacademy will get you up and running with these skills, but the best way to learn is to write code for a specific application. You can also refer to something like Python for Astronomers (\href{https://prappleizer.github.io/}{prappleizer.github.io}) as a reference or look at other guidebooks/tutorials (some listed at \href{https://astroteaching.github.io/computational/}{astroteaching.github.io/computational}).

Additionally, learning version control and collaborative work with git, via GitHub, Bitbucket, or similar, will give you a leg up. Version control is critical when you are developing software or even individual scripts, and the collaborative aspects of git allow you to contribute to larger projects where you are not necessarily the primary developer, but have reason to access, and potentially edit, source code.

Learning LaTeX will also help -- both for typesetting problem sets/homework assignments (quick way to become teacher's pet!) and working on paper drafts. Instead of Word, Pages, or free/open source alternatives, almost all drafting in (astro)physics -- including for this booklet! -- is done in LaTeX . There are local installations + editors like TeXShop, but if you have LaTeX installed on your computer, you can also use any text editor (some favorites for coding and other applications are Sublime Text and VSCode) to draft and compile your text. Collaborative work is often done in Overleaf, which is a cloud-based option -- even if you're not at the point of collaborating, it can still be very useful to access your documents from multiple devices without worrying about syncing.

For anyone just starting a project, it is also really important to understand the existing literature on the topic. Your supervisor may give you papers to read to start out, but finding others that are relevant (particularly as they are released) is a great way to show interest and initiative \textit{and} learn about where the field is headed.

\vspace{-2cm}

\subsection{Where do I find research papers to read?}

Every astro-relevant paper, at least for the last decades, is indexed in NASA/ADS (\href{https://ui.adsabs.harvard.edu/}{ui.adsabs.harvard.edu}); this is a primary resource for anyone doing astrophysics research. Pre-prints are also ingested by ADS. Those can be accessed directly (as they come out) on arXiv (\href{https://arxiv.org/}{arxiv.org}); pre-prints are released $\sim 9$pm ET (Sunday through Thursday) -- to see what's new, you can click on ``new'' next to astro-ph.

Some pre-prints are summarized in astrobites (\href{https://astrobites.org/}{astrobites.org}), which aims to make the literature accessible to undergraduates. Especially as you're starting out, it can be good practice to read both the paper and the astrobites summary to better digest the content.

\subsection{How do I get actively involved in my university's (astro)physics department?}

In addition to getting involved with research as an undergraduate, which is often critical to full integration in your department, attending seminars and colloquia is a fantastic way to be exposed to new science and meet your future colleagues. Even if you don't understand what is being discussed initially, the more talks you attend (and the further you get in your undergraduate study in general), the more you will take away from each talk. Eventually, you will be giving talks like that yourself.

Additionally, your department may have a chapter of the Society of Physics Students or affinity organizations like Women in Physics. These are wonderful opportunities to take more initiative and be fully involved in pre-professional activities.

\vspace{-2cm}


\subsection{Outside of class/research, what should I do to prepare myself for a career in astronomy?}

Building personal/professional connections and having a network of mentors is really important in the early stages of your career. (That's one of the goals of SIRIUS B's VERGE program!) Involving yourself in pre-professional activities like those listed above does a lot in this respect, and being visible in your department (as from going to talks) can help as well.

Reading broadly and understanding the common language of astrophysics (some of which is subfield specific) will also make for an easier transition to doing science professionally.

\subsection{What can I do now to prepare for a career in astrophysics?}

The best thing you can do to prepare for a career in astrophysics is gain solid footing in mathematics. There are also some formal programs through which you can get advanced pre-college training in astrophysics, namely the Yale Summer Program in Astrophysics (YSPA; \href{https://yspa.yale.edu}{yspa.yale.edu}) and Summer Science Program (SSP; \href{https://summerscience.org}{summerscience.org}). Some universities will also run daytime summer research programming for local high school students -- this will usually be less intensive than YSPA or SSP, but will still amount to an enormous leg up for participating students. 

\vspace{-1cm}

\subsection{I want to play with data now. How do I do that?}

The good news in astronomy is that so much of our software and data (both from observations and simulations) are free and open source by default! 

Observational data is generally stored in FITS -- Flexible Image Transport System -- files. These can be easily opened and perused with Python if you're comfortable coding or you can use SAOImageDS9 (DS9 for short; \href{https://sites.google.com/cfa.harvard.edu/saoimageds9}{sites.google.com/cfa.harvard.edu/saoimageds9}) or similar to go through a graphical user interface.

You can find data of all kinds on different survey and data access websites, for example:
\begin{itemize}
    \item Barbara A. Mikulski Archive for Space Telescopes (MAST) -- \\\href{https://mast.stsci.edu/portal/Mashup/Clients/Mast/Portal.html}{mast.stsci.edu/portal/Mashup/Clients/Mast/Portal.html}
    \item Dark Energy Spectroscopic Instrument (DESI) Legacy Imaging Survey -- \href{https://www.legacysurvey.org}{legacysurvey.org}
    \item Hyper Suprime-Cam Subaru Strategic Program -- \\\href{https://hsc-release.mtk.nao.ac.jp/doc/index.php/tools/}{hsc-release.mtk.nao.ac.jp/doc/index.php/tools}
    \item Sloan Digital Sky Survey -- \href{https://www.sdss4.org/dr17/data_access/}{sdss4.org/dr17/data\_access}
\end{itemize}

If you are comfortable coding, you have even more options -- for instance, you could play with other types of data like those from:
\begin{itemize}
    \item Arecibo Legacy Fast ALFA Survey -- \href{http://egg.astro.cornell.edu/alfalfa/data/}{egg.astro.cornell.edu/alfalfa/data}
    \item Gaia stellar information -- \href{https://gea.esac.esa.int/archive/}{gea.esac.esa.int/archive}
    \item IllustrisTNG (cosmological simulation) -- \href{https://www.illustris-project.org/data/}{illustris-project.org/data}
    \item NASA Exoplanet Archive -- \href{https://exoplanetarchive.ipac.caltech.edu}{exoplanetarchive.ipac.caltech.edu}
    \item X-ray transient light curves -- \href{https://github.com/avapolzin/X-rayLCs}{github.com/avapolzin/X-rayLCs}
\end{itemize}

You can generally find documentation by poking around these pages, but if that isn't sufficiently helpful, don't worry -- these are resources intended for professionals and are not made specifically accessible to the public.

There are also websites where you can engage with data via the site including:
\begin{itemize}
    \item JADES interactive viewer -- \href{https://jades.idies.jhu.edu}{jades.idies.jhu.edu}
    \item Legacy Viewer -- \href{https://www.legacysurvey.org/viewer}{legacysurvey.org/viewer}
    \item Magnetar Outburst Online Catalog analysis tab -- \\\href{http://magnetars.ice.csic.es/#/outbursts}{magnetars.ice.csic.es/\#/outbursts}
    \item NASA Exoplanet Archive confirmed planet plotting tool -- \\\href{https://exoplanetarchive.ipac.caltech.edu/cgi-bin/IcePlotter/nph-icePlotInit?mode=demo&set=confirmed}{exoplanetarchive.ipac.caltech.edu/cgi-bin/IcePlotter/nph-icePlotInit?mode=demo\&set=confirmed}
    
\end{itemize}

If you're looking for a more sanitized experience where you can engage in citizen science, the Zooniverse (\href{https://www.zooniverse.org}{zooniverse.org}) has a number of different projects in astronomy among other topics. This can be a really fun way to spend some time that also contributes to active science!

If you have a telescope, you can always take your own data that can be analysed! Assuming you have the equipment, you may want to contribute to the monitoring of time-domain events (like exoplanet transits and stellar variability); you can then add to the American Association of Variable Star Observers (AAVSO; \href{https://www.aavso.org/}{aavso.org}) repositories. This will require that any data you take are properly calibrated, which necessitates some additional work, but is a terrific way for an excited amateur to get involved in observational astronomy.

\vspace{-1cm}

\subsection{How do I connect with other women in astrophysics?}

Universities often have a number of organizations that connect women in STEM and specific scientific fields. Most universities will have some version of a Society of Women in Physics, which can be a supportive discipline-specific community even for undergraduates. There are often also Women in Science groups (sometimes dominated by one field or another) that you can get involved with at that level. At the undergraduate level, the American Physical Society holds the Conference for Undergraduate Women in Physics (CUWiP), which is hosted simultaneously at a number of regional hubs throughout the United States. CUWiP generally focuses as much on pre-professional development as it does on science, so this is an opportunity to talk about your research \textit{and} learn best practices for navigating graduate applications, ignoring/mitigating impostor syndrome, engaging in self-advocacy, etc.

There are initiatives that look to aggregate lists of women in astrophysics in order to showcase their work and generate a professional network. For instance 1400 Degrees (\href{https://1400degrees.org}{1400degrees.org}) is a directory of women and gender minorities in (astro)physics. There is also If/Then (\href{https://www.ifthenshecan.org}{ifthenshecan.org}), which showcases a somewhat smaller cohort across all of STEM.

Additionally, as you progress in your career, you will find that individual mentorship -- both informal and formal -- will play a role. Most, if not all, of the women who contributed to this guide, can point to female scientists who shaped their path. These mentoring situations happen organically over time, so there isn't a framework for approaching them; just trust that you, too, will end up with scientists in your extended network who will advise and support you, and that you too will eventually be able to reach back and supported the interested girls behind you.