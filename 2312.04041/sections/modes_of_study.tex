\section{Modes of Study}

As with anything, there are multiple ways to approach science. In (astro)physics, we usually break this down into two to three categories. In the binary classification, there are experimentalists, who use purpose-built instruments to study the universe, and there are theorists, who rely more heavily on physical first principles to make predictions about the universe. In the ternary classification, there are instrumentationalists, who build, maintain, and operate telescopes and other instruments; there are observers, who use existing telescopes to study the universe with collected data; and there are theorists.

\subsection{Instrumentation}

Scientists who are focused on instrumentation spend much, if not all, of their time, working on the construction, commissioning, and maintenance of an instrument or several. In astronomy, these instruments are generally telescopes, though some folks who work closer to the intersection of biology or geology (primarily planetary scientists and exoplanet-focused astronomers) may work on robotic probes or other projects.

Instrumentationalists (sometimes called experimentalists) will shepherd an instrument from its initial conception through its construction and commissioning, and then remain working on maintaining/running it. Given the long timescales of large-scale astrophysical projects, some people will only ever work on a few of these different phases, but there is still a tremendous variety of activities that fall under the experimentalist umbrella. Ground-based instruments at the commissioning phase, for instance, may require travel to the site -- often in remote locations like the Atacama Desert or the South Pole.

For students who are interested in engineering and in astrophysics, this may be a means of combining those passions. 

\subsection{Observation}

Observers are like the explorers of modern astrophysics -- they use telescopes to answer questions about how the universe operates. Some of these questions are specific and were posed to a telescope time allocation committee to earn dedicated time on that instrument; others come out of looking at data (often from large surveys, but sometimes from more targeted observations) and seeing something warrants follow-up.

With increasingly large datasets available to astronomers, this is sometimes referred to as data-driven astronomy. In the case of archival data, scientists will not have to observe but simply download and analyze the data. On the other hand, when a specific question is asked and bespoke observations are taken, it can go one of two ways. There are queue-based instruments, where observations are taken automatically when they're scheduled and data need only be reduced (i.e., made usable) and analyzed, and there are more classical instruments, like the Keck telescopes, which require that astronomers actually oversee and run observations themselves. Sometimes this involves travelling to the telescope, but, especially after the COVID-19 pandemic, there has been a move toward remote, ``pajama mode'' observations. This mode also have the benefit of helping to lessen our field's carbon footprint by limiting truly unnecessary travel. Those data will also need to be reduced and analyzed after observations are completed.

Because the reduction and analysis of data of variable quality is so necessary to this science, there is a fair bit of creativity and scientific software development that goes into observational research. For students who are interested in such things and/or are night owls, observational astrophysics may be for you.

\subsection{Theory}

There are two types of theorists -- your classic pen-and-paper types and computational scientists who use simulations.

In general, there are fewer pen-and-paper theorists now than there were historically. In some parts of cosmology and particle astrophysics, there are people who still work this way, but most astrophysics theory is done with either numerical simulations or semi-analytic modeling, which take known physical principles and apply them together in a consistent way to make predictions about the world/universe around us. 

Computational astrophysics can accurately reproduce observations and use the tunable model parameters to offer insights into why things are the way they are or the simulations may probe scales/environments that are not accessible with current observations, making predictions that will later be substantiated (or not!) by observations.

Pen-and-paper theorists use pure math and physics, while computational astrophysicists also have to have strong backgrounds in coding/computing. For the would-be-mathematicians who are interested in astrophysics, the former may be most engaging, while would-be-computer scientists may find the work of computational astrophysics satisfies interests in both physics and computing.