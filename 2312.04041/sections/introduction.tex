\section{Introduction}

One of the hardest parts of starting a career in astrophysics, as in most fields, is getting a foot in the door. The goal of this booklet is to demystify some aspects of research careers in astronomy/astrophysics and offer a jumping off point for further independent investigation of topics that might spark interest. Though it is not possible to comprehensively cover a science tasked with understanding something as large as the universe, we attempt here to offer a primer on many broad (and hopefully representative) topics in astronomy and provide some insights into the current frontiers in those sub-fields. For clarity, each sub-field is presented in its own section, but these research areas are not islands -- instead, much of the knowledge, and many of the questions, presented here are interconnected.

At the same time, modern astrophysics is, by its very nature, interdisciplinary. On top of having a vast working knowledge of physics, astrophysicists generally have strong foundations in mathematics, statistics, chemistry, and computer science. In fact, scientists in general wear many hats. Not only are scientists often charged with engineering and construction, software development, science communication, and more in the course of their science, they are also frequently charged with lobbying and informing matters of public policy. Though there is substantial freedom to study what you want (past a certain career stage), it comes with the institutional and intellectual responsibility to share what you have learned and use your science for social good.

In astronomy, we are very fortunate; the breathtaking pictures produced by ground- and space-based telescopes (across almost the entire electromagnetic spectrum) speak to people of all backgrounds and interests. There is something innately human about wanting to know where we came from, what our place is in the universe, and where we are going, and the larger questions we ask connect to this desire. The women who've contributed here fundamentally believe in the shared human pursuit of these questions.