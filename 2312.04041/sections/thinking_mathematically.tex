\section{Thinking Mathematically}

It's really common to approach equations as intangible, but, in science, they are a shorthand for describing the world around us. There is no part of astrophysics where you can avoid understanding the way math is used to describe physical concepts. In this section, you will walk through two (perhaps already obvious) points -- 1) these physical descriptions can be broken down to be more easily digested and retained and 2) using units to build physical intuition (i.e., \textit{dimensional analysis}).

\subsection{Interpreting Mathematical Expressions}

Math, often from the way it's taught in school, can feel somewhat intimidating. Especially when you're encountering concepts for the first time, it can be a challenge to break down what you're seeing and develop an intuition.

One of the things that can make math more accessible in physical contexts is recognizing that physical intuition applies. If we take a simple example $v = \frac{\Delta x}{\Delta t}$, where $v$ is velocity (or speed in this case, since we can ignore direction), $\Delta x$ represents a change in position, and $\Delta t$ represents a change in time. (\emph{$\Delta$ generally represents a change in whatever variable follows.})

Since $v$ is proportional to $\Delta x$, if Object 1 goes farther than Object 2 in the same amount of time, the velocity of Object 1 must be higher than the velocity of Object 2. Similarly, since $v$ is inversely proportional to $\Delta t$, if Object 1 takes less time to go the same distance as Object 2, then Object 1 must once again be moving faster than Object 2.

This sort of thinking can be extended to much more complicated systems and equations. In essence, it is a matter of ensuring that you engage critically with the math rather than just ``plugging and chugging'', where you use equations to an end without considering what they're telling you. It takes practice to become comfortable reading the math this way, but once you begin doing it regularly, it will substantially deepen your understanding of what you are doing.

Looking toward the next section, we might consider that a change in position is measured in units of length, and a change in time is measured in units of time, which means that velocity must carry units of length/time. Because these units are different, we can multiply or divide them, but cannot add or subtract them. The following expression is fine:
\begin{equation}
   v_1 =  v_0 + \frac{\Delta x}{\Delta t} \notag
\end{equation}
while $v_1 = v_0 + \Delta x$ makes no physical or mathematical sense. Additionally, the units on either side of an expression must be equivalent so the rearranged version, $\Delta x = v_0 + v_1$, would similarly be non-physical and incorrect. Another way to think about this is that units essentially act algebraically -- this is why it is always important to know what dimensions and units your variables carry!

(An example from calculus, using the same expression for velocity and position, is that $x_1 = x_0 + \int v \, dt$ -- here $dt$ already confers dimensions of time, so that this integral is valid.)

Knowing the physical dimensions that numbers carry is really powerful. You can gain a tremendous amount of insight into a problem by balancing these dimensions in a practice called dimensional analysis.

Don't worry if this isn't totally clear yet, it takes time and repeated exposure. In an undergraduate physics major, you will generally be required to take formal courses in differential, integral, vector, and multivariate (or multivariable) calculus, linear algebra, differential equations, and Fourier analysis. (Real and complex analysis are often deemed useful, but are usually not required.) Take it one step at a time, really internalize the math, and eventually you will gain intuition for the physical meaning of expressions employing even these more advanced mathematics.


\subsection{Dimensional Analysis}

Everyone's favorite example of dimensional analysis is the analytic (using only algebra and not plugging all variables in) calculation of the extent of a black hole's event horizon. Say that you're given that exercise without further information or equations, only the value of certain constants: all you need to know is that the event horizon (a term we will use loosely here) is the distance from the black hole at which the escape velocity becomes greater than the speed of light. Because the event horizon is the location at which $v_\mathrm{esc}$ becomes larger than the speed of light, $c$, we can consider the boundary case where $v_\mathrm{esc} = c$. The escape velocity is, physically, the velocity at which the kinetic energy ($K$) of some object is greater than the gravitational potential energy ($P$) of the body from which it is trying to escape, which gives us another boundary $|K| = |P|$ (we take the absolute value here to avoid worrying about the orientation of our coordinate systems, since we only care about the relative strength of $K$ vs. $P$).

We can see that both of these boundaries make sense dimensionally -- we are simply stating that one velocity is equivalent to another and one energy is equivalent to another at this boundary. But what of the actual expressions for $K$ and $P$? Energy has dimensions of mass length$^2$ time$^{-2}$. In the case of kinetic energy, we often see it as [mass] $\times$ [length/time]$^{2}$, which, even without knowing the constant at the beginning, will get you within a factor of two of $K = \frac{1}{2}mv^2$. Given that $G$ (the gravitational constant) carries dimensions of mass$^{-1}$ length$^3$ time$^{-2}$, without worrying about the value of $G$, can you figure out the expression for potential energy knowing that it has to depend on radius (or else there wouldn't be a way to infer the radius of the event horizon!), so that $P$ decreases as $r$ increases? Similarly, $P$ increases with increasing mass of the body $M$ and increasing mass of the object $m$.  Remember that the final expression for $P$, should have units of energy (mass length$^2$ time$^{-2}$).\\
\\
\vfill
% \noindent [\textit{Here's some blank space so that you can try this exercise without seeing the answer, which begins on the next page.}]

\pagebreak

Let's think about this in parts. We know that we want units of mass length$^2$ time$^{-2}$ and already have mass$^{-1}$ length$^3$ time$^{-2}$ from $G$. $M$ and $m$ will each have units of mass by default, and $r$ has units of length.

We already know that $P \propto G$ gives us time$^{-2}$, so this is a good first step. $G$ does give us an extra factor of length, though. To fix this, we can take $P \propto \frac{G}{r}$, which has units of mass$^{-1}$ length$^2$ time$^{-2}$. Now we need two powers of mass and, knowing that the gravitational potential depends on the masses of both the body and the object, we can assume that this comes from $M\times m$, so our gravitational potential energy is $\propto \frac{GMm}{r}$. It turns out there's a factor of $-1$ here relative to $K$, but this can also be set by the orientation of $r$, so $P = -\frac{GMm}{r}$ which we largely recover just by thinking about the units!

Now for the rest of the question -- we are trying to find the radius of the event horizon, so we can set up the following equation:
\begin{equation}
    \frac{1}{2} m v_\mathrm{esc}^2 = \frac{G M_\mathrm{BH} m}{r} \notag
\end{equation}
Immediately, we can see that the mass of the object trying to escape ($m$) does not matter because that term cancels (and good thing, since in the case where $v_\mathrm{esc} = c$, we are discussing light, and photons are massless). This means that we are left with:
\begin{equation}
    \frac{1}{2} v_\mathrm{esc}^2 = \frac{GM_\mathrm{BH}}{r} \notag
\end{equation}
Rearranging things and taking $v_\mathrm{esc} = c$ as in the definition of the event horizon, we find that:
\begin{equation}
    r = \frac{2 G M_\mathrm{BH}}{c^2}
\end{equation}
This tells us that the extent of the event horizon \emph{only} depends on the mass of the black hole, since $G$ and $c$ are constants. In fact, it relates linearly ($r \propto M_\mathrm{BH}$), which means that if there are two black holes with $M_\mathrm{BH,1} = 10 M_\mathrm{BH,2}$, then $r_1 = 10\,r_2$.