\section*{Introduction}
\label{sec:introduction}

\Eff is a programming language based on the algebraic approach to effects, in
which computational effects are modelled as operations of a suitably chosen
algebraic theory~\cite{plotkin03algebraic}. Common computational effects such as
input, output, state, exceptions, and non-determinism, are of this kind.
Continuations are not algebraic~\cite{hyland07combining}, but they turn out to
be naturally supported by \eff as well. Effect handlers are a related
notion~\cite{plotkin09handlers,pretnar10:_logic_handl_algeb_effec} which
encompasses exception handlers, stream redirection, transactions, backtracking,
and many others. These are modelled as homomorphisms induced by the universal
property of free algebras.

Because an algebraic theory gives rise to a monad~\cite{plotkin02notions},
algebraic effects are subsumed by the monadic approach to computational
effects~\cite{benton00monads}. They have their own virtues, though. Effects are
combined more easily than monads~\cite{hyland06combining}, and the interaction
between effects and handlers offers new ways of programming. An experiment in
the design of a programming language based on the algebraic approach therefore
seems warranted.

Philip Wadler once opined~\cite{wadler95monads} that monads as a programming
concept would not have been discovered without their category-theoretic
counterparts, but once they were, programmers could live in blissful ignorance of
their origin. Because the same holds for algebraic effects and handlers, we
streamlined the paper for the benefit of programmers, trusting that connoisseurs
will recognize the connections with the underlying mathematical theory.

The paper is organized as follows. Section~\ref{sec:syntax} describes the syntax
of \eff, Section~\ref{sec:eff-specific} informally introduces constructs
specific to \eff, Section~\ref{sec:type-checking} is devoted to type checking,
in Section~\ref{sec:semantics} we give a domain-theoretic semantics of~\eff, and
in Section~\ref{sec:implementation} we briefly discuss our prototype
implementation. The examples in Section~\ref{sec:examples} demonstrate how
effects and handlers can be used to produce standard computational effects, such
as exceptions, state, input and output, as well as their variations and
combinations. Further examples show how \eff's delimited control capabilities
are used for nondeterministic and probabilistic choice, backtracking, selection
functionals, and cooperative multithreading. We conclude with thoughts about the
future work.

The implementation of \eff is freely available at \url{http://math.andrej.com/eff/}.

\paragraph{Acknowledgements}
% Added s to fix a line break
\label{sec:acknowledgment}

We thank Ohad Kammar, Gordon Plotkin, Alex Simpson, and Chris
Stone for helpful discussions and suggestions. Ohad Kammar contributed parts of
the type inference code in our implementation of \eff. The cooperative
multithreading example from Section~\ref{sec:cooperative-multithreading} was
written together with Chris Stone.



%%% Local Variables: 
%%% mode: latex
%%% TeX-master: "eff"
%%% End: 
