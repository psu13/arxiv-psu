\section{Implementation}
\label{sec:implementation}

To experiment with \eff we implemented a prototype interpreter whose main evaluation loop
is essentially the same as the denotational semantics described in
Section~\ref{sec:semantics}. Apart from inessential improvements, such as recursive type
definitions, inclusion of $\kord{for}$ and $\kord{while}$ loops, and pattern matching, the
implemented language differs from the one presented here in two ways that make it usable:
we implemented Hindley-Milner style type inference with parametric
polymorphism~\cite{milner78atheory}, and in the concrete syntax we hid the distinction
between expressions and computations. We briefly discuss each.

There are no surprises in the passage from monomorphic type checking to parametric
polymorphism and type inference. The infamous value
restriction~\cite{wright95simpleimperative} is straightforward because the distinction
between expressions and computations corresponds exactly to the distinction between
nonexpansive and expansive terms. In fact, it may be worth investing in an effect system
that would relax the value restriction on those computations
that can safely be presumed pure. Because $\new E$ is a computation, effect instances are
\emph{not} polymorphic, which is in agreement with ML-style references being
non-polymorphic.
%, see Section~\ref{sec:state}.
% There is nothing in this section about value restriction

A syntactic division between pure expressions and possibly effectful computations is
annoying because even something as simple as $f \, x \, y$ has to be written as
$\letin{g = f\, x} g \, y$, and having to insert $\kord{val}$ in the right places is no
fun either. Therefore, the concrete syntax allows the programmer to arbitrarily mix expressions
and computations, and a desugaring phase appropriately separates the two.

The desugaring process is fairly simple. It inserts a $\kord{val}$ when an expression
stands where a computation is expected. And if a computation stands where an expression is expected, the computation is hoisted to an enclosing $\kord{let}$
statement. Because several computations may be hoisted from a single expression, the
question arises how to order the corresponding $\kord{let}$ statements. For example, $(f\, x, g\, y)$ can be desugared as
%
\begin{equation*}
  \begin{aligned}
    &\letin{a = f \, x} \\ &\quad \letin{b = g \, y} (a, b)
  \end{aligned}
  \qquad\text{or}\qquad
  \begin{aligned}
    &\letin{b = g \, y} \\ &\quad \letin{a = f \, x} (a, b)
  \end{aligned}
\end{equation*}
%
The order of $f \, x$ and $g \, y$ matters when both computations cause computational
effects. The desugaring phase avoids making a decision by using the \emph{simultaneous}
$\kord{let}$ statement
%
\begin{equation*}
  \kpre{let} a = f \, x \kop{and} b = g \, y \kop{in} (a, b)
\end{equation*}
%
which leaves open the possibility of various compiler optimizations. The prototype simply
evaluates simultaneous bindings in the order they are given, and a command-line
option enables sequencing warnings about possible unexpected order of effects.
%
It could be argued that the warnings should actually be errors, but we allow some slack until
we have an effect system that can detects harmless instances of simultaneous binding.

For one-off handlers, \eff provides an inline syntax so that one can write

\begin{center}
\begin{tabular}{ccc}
\begin{source}
handle
  $c$
with
  $\cdots$
\end{source}&
\quad instead of \quad \hbox{}&
\begin{source}
with
  handler
    $\cdots$
handle
  $c$
\end{source}
\end{tabular}
\end{center}
%
Additionally, the \inline{val} and \inline{finally} clauses may be omitted, in which case
they are assumed to be the identities.

%%% Local Variables: 
%%% mode: latex
%%% TeX-master: "eff"
%%% End: 
