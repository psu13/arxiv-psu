\section{Denotational semantics}
\label{sec:semantics}

Our aim is to describe a denotational semantics which explains how programs in \eff are evaluated.
%
Since the implemented runtime has no type information, we give Curry-style semantics
in which terms are interpreted without being typed.
%
See the exposition by John Reynolds~\cite{reynolds00themeaning} on how such semantics can
be related to Church-style semantics in which types and typing judgements receive meanings.

We give interpretations of expressions and computations in domains of \emph{values} $V$
and \emph{results} $R$, respectively. We follow Reynolds by avoiding a particular choice
of $V$ and $R$, and instead require properties of $V$ and $R$ that ensure the semantics
works out. The requirements can be met in a number of ways, for example by solving
suitable domain equations or by taking $V$ and $R$ to be sufficiently large universal
domains.

The domain $V$ has to contain integers, booleans, functions, etc.
In particular, we require that $V$ contains the following retracts, where $\II$ is a 
set of effect instances, and $\oplus$ is coalesced sum:
%
\begin{align*}
  &\retract{\ZZ_\bot}{V}{\iota_{\intty}}{\rho_{\intty}}
  &
  &\retract{\set{0, 1}_\bot}{V}{\iota_{\boolty}}{\rho_{\boolty}}
  &
  &\retract{\set{\star}_\bot}{V}{\iota_{\unitty}}{\rho_{\unitty}}
  \\
  &\retract{\II_\bot}{V}{\iota_{\kord{effect}}}{\rho_{\kord{effect}}}
  &
  &\retract{V \times V}{V}{\iota_{\times}}{\rho_{\times}}
  &
  &\retract{V \oplus V}{V}{\iota_{+}}{\rho_{+}}
  \\
  &\retract{R^V}{V}{\iota_{\to}}{\rho_{\to}}
  &
  &\retract{R^R}{V}{\iota_{\hto}}{\rho_{\hto}}
\end{align*}
%
As expressions are terminating, the bottom element of $V$ is never used to denote
divergence, but we do use it to indicate ill-formed values and runtime errors.

\newcommand{\operationDomain}{\II \times \OO \times V \times R^V}

The domain
%
\begin{equation}
  \label{eq:resultDomain}
  (V + \operationDomain)_\bot
\end{equation}
%
embodies the idea that a terminating computation is either a value or an operation applied to a parameter and a continuation. There are canonical retractions from~\eqref{eq:resultDomain} onto the two summands,
%
\begin{equation}
  \label{eq:resultDomain-retraction}
  \xymatrix{
     *!R{V\;} \ar@<0.25em>[rr]^(0.3){\iota_{\kord{val}}}
     & &
     {(V + \operationDomain)_\bot}
     \ar@<0.25em>[ll]^(0.7){\rho_{\kord{val}}}
     \ar@<-0.25em>[r]_(0.7){\rho_{\kord{oper}}}
     &
     *!L{\;(\operationDomain)_\bot}
     \ar@<-0.25em>[l]_(0.3){\iota_\kord{oper}}
  }
\end{equation}
%
A typical element of~\eqref{eq:resultDomain} is either $\bot$, of the form $\iota_{\kord{val}}(v)$ for a unique $v \in V$, or of the form $\iota_{\kord{oper}}(n, \op, v, \kappa)$ for unique $n \in \II$, $\op \in \OO$, $v \in V$, and $\kappa \in R^V$. We require that $R$ contains~\eqref{eq:resultDomain} as a retract:
%
\begin{equation}
  \label{eq:resultDomain-in-R}
  \retract{(V + \operationDomain)_\bot}{R}{\iota_{\kord{res}}}{\rho_{\kord{res}}}.
\end{equation}
%
We may define a strict map from~\eqref{eq:resultDomain} by cases, with one case specifying how to map $\iota_{\kord{val}}(v)$ and the other how to map $\iota_{\kord{oper}}(n, \op, v, \kappa)$. For example, given a map $f : V \to R$, there is a unique strict map $\lift{f} : (V + \operationDomain)_\bot \to R$, called the \emph{lifting} of~$f$, which depends on $f$ continuously and satisfies the recursive equations
%
\begin{align*}
  \lift{f}(\iota_{\kord{val}}(v)) &= f(v),
  \\
  \lift{f}(\iota_{\kord{oper}}(n,\op,v,\kappa)) &=
  \iota_{\kord{oper}}(n, \op, v, \lift{f} \circ \rho_{\kord{res}} \circ \kappa).
\end{align*}

An \emph{environment $\eta$} is a map from variable names to values. We denote by
$\extend{\eta}{x}{v}$ the environment which assigns $v$ to $x$ and otherwise behaves as
$\eta$. An expression is interpreted as a map from environments to values. The standard
cases are as follows:
%
\begin{align*}
  \sem{x}{\eta} &= \eta(x)
  \\
  \sem{n}{\eta} &= \iota_\intty(\overline{n})
  \\
  \sem{\fls}{\eta} &= \iota_\boolty(0)
  \\
  \sem{\tru}{\eta} &= \iota_\boolty(1)
  \\
  \sem{\unt}{\eta} &= \iota_\unitty(\star)
  \\
  \sem{\pair{e_1, e_2}}{\eta} &= \iota_{\times}(\sem{e_1}{\eta}, \sem{e_2}{\eta})
  \\
  \sem{\Left e}{\eta} &= \iota_{+}(\iota_0(\sem{e}{\eta}))
  \\
  \sem{\Right e}{\eta} &= \iota_{+}(\iota_1(\sem{e}{\eta}))
  \\
  \sem{\fun{x \T A} c}{\eta} &= \iota_{\to}(\lambda v : V \,.\, \sem{c}{\extend{\eta}{x}{v}})  
\end{align*}
%
Of course, we need to provide the semantics of other built-in constants, too.
The interpretation of $\hash{e}{\op}$ make sense only when $e$ evaluates to an instance, so we define
% 
\begin{equation*}
  \sem{\hash{e}{\op}}{\eta} = 
  \begin{cases}
    \iota_\to(\lambda v : V \,.\,
       \iota_{\kord{res}}(\iota_{\kord{oper}}(n, \op, v, \iota_\kord{res} \circ \iota_\kord{val}))) &
    \text{if $\rho_{\kord{effect}}(\sem{e}{\eta}) = n \in \II$,}\\
    \iota_\to(\lambda v : V \,.\, \bot) &
    \text{if $\rho_{\kord{effect}}(\sem{e}{\eta}) = \bot$.}
  \end{cases}
\end{equation*}
%
The interpretation of a handler is
%
\begin{multline*}
  \sem{
    \handler
    (\hash{e_i}{\op_i} \, x \, k \mapsto c_i)_i \case
    \val x \mapsto c_v \case
    \fin x \mapsto c_f}{\eta} = \\ 
  \iota_\hto(\lift{f} \circ \rho_{\kord{res}} \circ  h \circ \rho_{\kord{res}})
\end{multline*}
%
where $f : V \to R$ is $f(v) = \sem{c_f}{\extend{\eta}{x}{v}}$ and $h : (V + \operationDomain)_\bot \to R$ is characterized as follows:
%
if one of the $\rho_{\kord{effect}}(\sem{e_i}{\eta})$ is $\bot$ we set $h = \lambda x
\,.\, \bot$, otherwise $\rho_{\kord{effect}}(\sem{e_i}{\eta}) = n_i \in \II$ for all $i$
and then we take the $h$ defined by cases as
%
\begin{align*}
  h(\iota_\kord{val}(v)) &= \sem{c_v}{\extend{\eta}{x}{v}}
  \\
  h(\iota_{\kord{oper}}(n_i, \op_i, v, \kappa)) &=
  \sem{c_i}{\extends{\eta}{x \sto v, k \sto \kappa}} \qquad \text{for all $i$,}
  \\
  h(\iota_{\kord{oper}}(n, \op, v, \kappa)) &=
  \begin{aligned}[t]
  &\iota_{\kord{res}}(\iota_{\kord{oper}}(n, \op, v, h \circ \rho_{\kord{res}} \circ \kappa))\\
  &\qquad \text{if $(n, \op) \neq (n_i, \op_i)$ for all $i$.}
  \end{aligned}
\end{align*}

We proceed to the meaning of computations, which are interpreted as maps from
environments to results. Promotion of expressions is interpreted in the obvious way as
%
\begin{equation*}
  \sem{\val e}{\eta} = \iota_{\kord{res}}(\iota_\kord{val}(\sem{e}{\eta}))
\end{equation*}
%
The $\kord{let}$ statement corresponds to monadic-style binding:
%
\begin{equation*}
  \sem{\letin{x = c_1}{c_2}}{\eta} =
  \lift{(\lambda v : V \,.\, \sem{c_2}{\extend{\eta}{x}{v}})}
  (\rho_{\kord{res}}(\sem{c_1}{\eta})),
\end{equation*}
%
A recursive function definition is interpreted as
%
\begin{equation*}
  \sem{\letrecin{f \, x = c_1} c_2}{\eta} = \sem{c_2}{\extend{\eta}{f}{\iota_\to(t)}}
\end{equation*}
%
where $t : V \to R$ is the least fixed point of the map
%
\begin{equation*}
  t \mapsto (\lambda v : V \,.\, \sem{c_1}{\extends{\eta}{f \sto \iota_\to(t), x \sto v}}).
\end{equation*}
%
The elimination forms are interpreted in the usual way as:
\begin{align*}
  \sem{\ifthenelse{e}{c_1}{c_2}}{\eta} &=
  \begin{cases}
    \sem{c_1}{\eta} & \text{if $\rho_{\boolty}{\sem{e}{\eta}} = 1$} \\
    \sem{c_2}{\eta} & \text{if $\rho_{\boolty}{\sem{e}{\eta}} = 0$} \\
    \bot & \text{otherwise}
  \end{cases}
  \\
  \sem{\absurd e}{\eta} &= \bot
  \\
  \sem{\matchpair{e}{x,y}{c}}{\eta} &=
  \begin{aligned}[t]
  &\sem{c}{\extends{\eta}{x \sto v_0, y \sto v_1}} \\
  &\qquad \text{where $(v_0, v_1) = \rho_\times (\sem{e}{\eta})$}
  \end{aligned}
  \\
  \sem{\matchsum{e}{x}{c_1}{y}{c_2}}{\eta} &=
  \begin{cases}
    \sem{c_1}{\extend{\eta}{x}{v}} & \text{if $\rho_{+}(\sem{e}{\eta}) = \iota_0(v)$} \\
    \sem{c_2}{\extend{\eta}{y}{v}} & \text{if $\rho_{+}(\sem{e}{\eta}) = \iota_1(v)$} \\
    \bot & \text{otherwise}
  \end{cases}
  \\
  \sem{e_1 \, e_2}{\eta} &= \rho_\to(\sem{e_1}{\eta}) (\sem{e_2}{\eta})
\end{align*}  
%
For the interpretation of $\kord{new}$ we need a way of generating fresh names so that we
may sensibly interpret
%
\begin{equation*}
  \sem{\new E}{\eta} = \iota_{\kord{res}}(\iota_\kord{val}(\iota_{\kord{effect}}(n)))
  \quad \text{where $n \in \II$ is fresh.}
\end{equation*}
%
The implementation simply uses a local counter, but a satisfactory semantic solution needs
a model of names, such as the permutation models of Pitts and
Gabbay~\cite{gabbay01a-new-approach}, with $\II$ then being the set of atoms.

Finally, the handling construct is just an application
%
\begin{equation*}
  \sem{\handle{c}{e}}{\eta} = \rho_{\hto}(\sem{e}{\eta}) (\sem{c}{\eta}).
\end{equation*}


\subsection{Semantics of resources}
\label{sec:semantics-resources}

To model resources, the denotational semantics has to support the mutable nature of resource state, for example by explicitly threading it through the evaluation.
But we prefer not to burden ourselves and the readers with the ensuing technicalities.
Instead, we assume a mutable store $\sigma$ indexed by effect instances which supports the lookup and update operations.
That is, $\lookup{\sigma}{n}$ gives the current contents at location $n \in \II$, while
$\change{\sigma}{n}{v}{x}$ sets the contents at location $n$ to $v \in V$ and yields $x$.

A resource describes the behaviour of operations, i.e., it is a map $\OO \times V \times V \to V \times V$ which computes a value and the new state from a given operation symbol, parameter, and the current state. Thus an effect instance is not just an element of $\II$ anymore, but an element of $\JJ = \II \times (V \times V)^{\OO \times V \times V}$. Consequently in the semantics we replace $\II$ with $\JJ$ throughout and adapt the semantics of $\kord{new}$ so that it sets the initial resource state and gives an element of $\JJ$:
%
\begin{multline*}
  \sem{\newwith{E}{e}{(\kpre{operation} \op_i \, x \, @ \, y \mapsto c_i)_i}}{\eta} = \\
    \change{\sigma}{n}{\sem{e}{\eta}}
           {\iota_\kord{res}(\iota_\kord{val}(\iota_\kord{effect}(n,r)))}
    \quad \text{where $n \in \II$ is fresh}
\end{multline*}
%
where $r : \OO \times V \times V \to V \times V$ is defined by
%
\begin{equation*}
  r(\op, v, s) =
  \begin{cases}
    \rho_{\times}(\rho_{\kord{val}}(\rho_\kord{res}(\sem{c_i}{\extends{\eta}{x \sto v, y \sto s}}))) &
      \text{if $\op = \op_i$ for some $i$,} \\
    \bot & \text{otherwise.}
  \end{cases}
\end{equation*}
%
If no resource is provided, we use a trivial one:
%
\begin{equation*}
  \sem{\new E}{\eta} =
  \iota_\kord{res}(\iota_\kord{val}(\iota_\kord{effect}(n, \bot)))
  \quad \quad \text{where $n \in \II$ is fresh.}
\end{equation*}

Finally, to model evaluation at the toplevel, we define $\Evalsym : (V + \operationDomain)_\bot \to V$ by cases:
%
\begin{align*}
  \Eval{\iota_{\val}(v)} &= v \\
  \Eval{\iota_{\kord{oper}}((n, r), \op, v, \kappa)} &=
  \begin{aligned}[t]
  &\change{\sigma}{n}{s}{\Eval{\rho_\kord{res}(\kappa \, v')}} \\
  &\qquad \text{where $(v', s) = r(\op, v, \lookup{\sigma}{n})$}.
  \end{aligned}
\end{align*}
%
The meaning of a computation $c$ at the toplevel in the environment $\eta$ (and an
implicit resource state $\sigma$) is $\Eval{\rho_\kord{res}(\sem{c}{\eta})}$.


%%% Local Variables:
%%% mode: latex
%%% TeX-master: "eff"
%%% End:
