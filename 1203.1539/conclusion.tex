\section{Discussion}
\label{sec:conclusion}
 
Our purpose was to design a programming language based on the algebraic approach to computational effects and their handlers. We feel that we succeeded and that our experiment holds many promises.

First, we already pointed out several times that \eff would benefit from an effect system that provided a static analysis of computational effects. However, for a useful result we need to find a good balance between expressivity and complexity.

Next, it is worth investigating how to best reason about programs in \eff. Because the language has been inspired by an algebraic point of view, it seems clear that we should look into equational reasoning. The general theory has been investigated in some detail~\cite{plotkin08a-logic, pretnar10:_logic_handl_algeb_effec}, but the addition of effect instances may complicate matters.

Finally, continuations are the canonical example of a non-algebraic computational effect, so
it is a bit surprising that \eff provides a flexible and clean form of delimited control, especially since continuations were not at all on our design agenda. What then can we learn from \eff about control operators in an effectful setting?


%%% Local Variables:
%%% mode: latex
%%% TeX-master: "eff"
%%% End:
