\documentclass[10pt]{article}

\usepackage{times}
\usepackage{amsmath, amssymb}
\usepackage{xspace}
\usepackage{mathpartir}
\usepackage[only, mapsfrom, llbracket, rrbracket]{stmaryrd}
\usepackage{xypic}
\usepackage{listings}
% Match listings font to whatever font is given for tt in times package
\renewcommand{\ttdefault}{txtt}
%%%% MACROS FOR NOTATION %%%%
% Use these for any notation where there are multiple options.

%%% Notes and exercise sections
\makeatletter
\newcommand{\sectionNotes}{\phantomsection\section*{Notes}\addcontentsline{toc}{section}{Notes}\markright{\textsc{\@chapapp{} \thechapter{} Notes}}}
\newcommand{\sectionExercises}[1]{\ifdef{\OPTexerciseperpage}{\newpage}{}\phantomsection\section*{Exercises}\addcontentsline{toc}{section}{Exercises}\markright{\textsc{\@chapapp{} \thechapter{} Exercises}}}
\makeatother

%%% Definitional equality (used infix) %%%
\newcommand{\jdeq}{\equiv}      % An equality judgment
\let\judgeq\jdeq
%\newcommand{\defeq}{\coloneqq}  % An equality currently being defined
\newcommand{\defeq}{\vcentcolon\equiv}  % A judgmental equality currently being defined

%%% Term being defined
\newcommand{\define}[1]{\textbf{#1}}

%%% Vec (for example)

\newcommand{\Vect}{\ensuremath{\mathsf{Vec}}}
\newcommand{\Fin}{\ensuremath{\mathsf{Fin}}}
\newcommand{\fmax}{\ensuremath{\mathsf{fmax}}}
\newcommand{\seq}[1]{\langle #1\rangle}

%%% Dependent products %%%
\def\prdsym{\textstyle\prod}
%% Call the macro like \prd{x,y:A}{p:x=y} with any number of
%% arguments.  Make sure that whatever comes *after* the call doesn't
%% begin with an open-brace, or it will be parsed as another argument.
\makeatletter
% Currently the macro is configured to produce
%     {\textstyle\prod}(x:A) \; {\textstyle\prod}(y:B),{\ }
% in display-math mode, and
%     \prod_{(x:A)} \prod_{y:B}
% in text-math mode.
% \def\prd#1{\@ifnextchar\bgroup{\prd@parens{#1}}{%
%     \@ifnextchar\sm{\prd@parens{#1}\@eatsm}{%
%         \prd@noparens{#1}}}}
\def\prd#1{\@ifnextchar\bgroup{\prd@parens{#1}}{%
    \@ifnextchar\sm{\prd@parens{#1}\@eatsm}{%
    \@ifnextchar\prd{\prd@parens{#1}\@eatprd}{%
    \@ifnextchar\;{\prd@parens{#1}\@eatsemicolonspace}{%
    \@ifnextchar\\{\prd@parens{#1}\@eatlinebreak}{%
    \@ifnextchar\narrowbreak{\prd@parens{#1}\@eatnarrowbreak}{%
      \prd@noparens{#1}}}}}}}}
\def\prd@parens#1{\@ifnextchar\bgroup%
  {\mathchoice{\@dprd{#1}}{\@tprd{#1}}{\@tprd{#1}}{\@tprd{#1}}\prd@parens}%
  {\@ifnextchar\sm%
    {\mathchoice{\@dprd{#1}}{\@tprd{#1}}{\@tprd{#1}}{\@tprd{#1}}\@eatsm}%
    {\mathchoice{\@dprd{#1}}{\@tprd{#1}}{\@tprd{#1}}{\@tprd{#1}}}}}
\def\@eatsm\sm{\sm@parens}
\def\prd@noparens#1{\mathchoice{\@dprd@noparens{#1}}{\@tprd{#1}}{\@tprd{#1}}{\@tprd{#1}}}
% Helper macros for three styles
\def\lprd#1{\@ifnextchar\bgroup{\@lprd{#1}\lprd}{\@@lprd{#1}}}
\def\@lprd#1{\mathchoice{{\textstyle\prod}}{\prod}{\prod}{\prod}({\textstyle #1})\;}
\def\@@lprd#1{\mathchoice{{\textstyle\prod}}{\prod}{\prod}{\prod}({\textstyle #1}),\ }
\def\tprd#1{\@tprd{#1}\@ifnextchar\bgroup{\tprd}{}}
\def\@tprd#1{\mathchoice{{\textstyle\prod_{(#1)}}}{\prod_{(#1)}}{\prod_{(#1)}}{\prod_{(#1)}}}
\def\dprd#1{\@dprd{#1}\@ifnextchar\bgroup{\dprd}{}}
\def\@dprd#1{\prod_{(#1)}\,}
\def\@dprd@noparens#1{\prod_{#1}\,}

% Look through spaces and linebreaks
\def\@eatnarrowbreak\narrowbreak{%
  \@ifnextchar\prd{\narrowbreak\@eatprd}{%
    \@ifnextchar\sm{\narrowbreak\@eatsm}{%
      \narrowbreak}}}
\def\@eatlinebreak\\{%
  \@ifnextchar\prd{\\\@eatprd}{%
    \@ifnextchar\sm{\\\@eatsm}{%
      \\}}}
\def\@eatsemicolonspace\;{%
  \@ifnextchar\prd{\;\@eatprd}{%
    \@ifnextchar\sm{\;\@eatsm}{%
      \;}}}

%%% Lambda abstractions.
% Each variable being abstracted over is a separate argument.  If
% there is more than one such argument, they *must* be enclosed in
% braces.  Arguments can be untyped, as in \lam{x}{y}, or typed with a
% colon, as in \lam{x:A}{y:B}. In the latter case, the colons are
% automatically noticed and (with current implementation) the space
% around the colon is reduced.  You can even give more than one variable
% the same type, as in \lam{x,y:A}.
\def\lam#1{{\lambda}\@lamarg#1:\@endlamarg\@ifnextchar\bgroup{.\,\lam}{.\,}}
\def\@lamarg#1:#2\@endlamarg{\if\relax\detokenize{#2}\relax #1\else\@lamvar{\@lameatcolon#2},#1\@endlamvar\fi}
\def\@lamvar#1,#2\@endlamvar{(#2\,{:}\,#1)}
% \def\@lamvar#1,#2{{#2}^{#1}\@ifnextchar,{.\,{\lambda}\@lamvar{#1}}{\let\@endlamvar\relax}}
\def\@lameatcolon#1:{#1}
\let\lamt\lam
% This version silently eats any typing annotation.
\def\lamu#1{{\lambda}\@lamuarg#1:\@endlamuarg\@ifnextchar\bgroup{.\,\lamu}{.\,}}
\def\@lamuarg#1:#2\@endlamuarg{#1}

%%% Dependent products written with \forall, in the same style
\def\fall#1{\forall (#1)\@ifnextchar\bgroup{.\,\fall}{.\,}}

%%% Existential quantifier %%%
\def\exis#1{\exists (#1)\@ifnextchar\bgroup{.\,\exis}{.\,}}

%%% Dependent sums %%%
\def\smsym{\textstyle\sum}
% Use in the same way as \prd
\def\sm#1{\@ifnextchar\bgroup{\sm@parens{#1}}{%
    \@ifnextchar\prd{\sm@parens{#1}\@eatprd}{%
    \@ifnextchar\sm{\sm@parens{#1}\@eatsm}{%
    \@ifnextchar\;{\sm@parens{#1}\@eatsemicolonspace}{%
    \@ifnextchar\\{\sm@parens{#1}\@eatlinebreak}{%
    \@ifnextchar\narrowbreak{\sm@parens{#1}\@eatnarrowbreak}{%
        \sm@noparens{#1}}}}}}}}
\def\sm@parens#1{\@ifnextchar\bgroup%
  {\mathchoice{\@dsm{#1}}{\@tsm{#1}}{\@tsm{#1}}{\@tsm{#1}}\sm@parens}%
  {\@ifnextchar\prd%
    {\mathchoice{\@dsm{#1}}{\@tsm{#1}}{\@tsm{#1}}{\@tsm{#1}}\@eatprd}%
    {\mathchoice{\@dsm{#1}}{\@tsm{#1}}{\@tsm{#1}}{\@tsm{#1}}}}}
\def\@eatprd\prd{\prd@parens}
\def\sm@noparens#1{\mathchoice{\@dsm@noparens{#1}}{\@tsm{#1}}{\@tsm{#1}}{\@tsm{#1}}}
\def\lsm#1{\@ifnextchar\bgroup{\@lsm{#1}\lsm}{\@@lsm{#1}}}
\def\@lsm#1{\mathchoice{{\textstyle\sum}}{\sum}{\sum}{\sum}({\textstyle #1})\;}
\def\@@lsm#1{\mathchoice{{\textstyle\sum}}{\sum}{\sum}{\sum}({\textstyle #1}),\ }
\def\tsm#1{\@tsm{#1}\@ifnextchar\bgroup{\tsm}{}}
\def\@tsm#1{\mathchoice{{\textstyle\sum_{(#1)}}}{\sum_{(#1)}}{\sum_{(#1)}}{\sum_{(#1)}}}
\def\dsm#1{\@dsm{#1}\@ifnextchar\bgroup{\dsm}{}}
\def\@dsm#1{\sum_{(#1)}\,}
\def\@dsm@noparens#1{\sum_{#1}\,}

%%% W-types
\def\wtypesym{{\mathsf{W}}}
\def\wtype#1{\@ifnextchar\bgroup%
  {\mathchoice{\@twtype{#1}}{\@twtype{#1}}{\@twtype{#1}}{\@twtype{#1}}\wtype}%
  {\mathchoice{\@twtype{#1}}{\@twtype{#1}}{\@twtype{#1}}{\@twtype{#1}}}}
\def\lwtype#1{\@ifnextchar\bgroup{\@lwtype{#1}\lwtype}{\@@lwtype{#1}}}
\def\@lwtype#1{\mathchoice{{\textstyle\mathsf{W}}}{\mathsf{W}}{\mathsf{W}}{\mathsf{W}}({\textstyle #1})\;}
\def\@@lwtype#1{\mathchoice{{\textstyle\mathsf{W}}}{\mathsf{W}}{\mathsf{W}}{\mathsf{W}}({\textstyle #1}),\ }
\def\twtype#1{\@twtype{#1}\@ifnextchar\bgroup{\twtype}{}}
\def\@twtype#1{\mathchoice{{\textstyle\mathsf{W}_{(#1)}}}{\mathsf{W}_{(#1)}}{\mathsf{W}_{(#1)}}{\mathsf{W}_{(#1)}}}
\def\dwtype#1{\@dwtype{#1}\@ifnextchar\bgroup{\dwtype}{}}
\def\@dwtype#1{\mathsf{W}_{(#1)}\,}

\newcommand{\suppsym}{{\mathsf{sup}}}
\newcommand{\supp}{\ensuremath\suppsym\xspace}

\def\wtypeh#1{\@ifnextchar\bgroup%
  {\mathchoice{\@lwtypeh{#1}}{\@twtypeh{#1}}{\@twtypeh{#1}}{\@twtypeh{#1}}\wtypeh}%
  {\mathchoice{\@@lwtypeh{#1}}{\@twtypeh{#1}}{\@twtypeh{#1}}{\@twtypeh{#1}}}}
\def\lwtypeh#1{\@ifnextchar\bgroup{\@lwtypeh{#1}\lwtypeh}{\@@lwtypeh{#1}}}
\def\@lwtypeh#1{\mathchoice{{\textstyle\mathsf{W}^h}}{\mathsf{W}^h}{\mathsf{W}^h}{\mathsf{W}^h}({\textstyle #1})\;}
\def\@@lwtypeh#1{\mathchoice{{\textstyle\mathsf{W}^h}}{\mathsf{W}^h}{\mathsf{W}^h}{\mathsf{W}^h}({\textstyle #1}),\ }
\def\twtypeh#1{\@twtypeh{#1}\@ifnextchar\bgroup{\twtypeh}{}}
\def\@twtypeh#1{\mathchoice{{\textstyle\mathsf{W}^h_{(#1)}}}{\mathsf{W}^h_{(#1)}}{\mathsf{W}^h_{(#1)}}{\mathsf{W}^h_{(#1)}}}
\def\dwtypeh#1{\@dwtypeh{#1}\@ifnextchar\bgroup{\dwtypeh}{}}
\def\@dwtypeh#1{\mathsf{W}^h_{(#1)}\,}


\makeatother

% Other notations related to dependent sums
\let\setof\Set    % from package 'braket', write \setof{ x:A | P(x) }.
\newcommand{\pair}{\ensuremath{\mathsf{pair}}\xspace}
\newcommand{\tup}[2]{(#1,#2)}
\newcommand{\proj}[1]{\ensuremath{\mathsf{pr}_{#1}}\xspace}
\newcommand{\fst}{\ensuremath{\proj1}\xspace}
\newcommand{\snd}{\ensuremath{\proj2}\xspace}
\newcommand{\ac}{\ensuremath{\mathsf{ac}}\xspace} % not needed in symbol index

%%% recursor and induction
\newcommand{\rec}[1]{\mathsf{rec}_{#1}}
\newcommand{\ind}[1]{\mathsf{ind}_{#1}}
\newcommand{\indid}[1]{\ind{=_{#1}}} % (Martin-Lof) path induction principle for identity types
\newcommand{\indidb}[1]{\ind{=_{#1}}'} % (Paulin-Mohring) based path induction principle for identity types

%%% Uniqueness principles
\newcommand{\uniq}[1]{\mathsf{uniq}_{#1}}

% Paths in pairs
\newcommand{\pairpath}{\ensuremath{\mathsf{pair}^{\mathord{=}}}\xspace}
% \newcommand{\projpath}[1]{\proj{#1}^{\mathord{=}}}
\newcommand{\projpath}[1]{\ensuremath{\apfunc{\proj{#1}}}\xspace}
\newcommand{\pairct}{\ensuremath{\mathsf{pair}^{\mathord{\ct}}}\xspace}

%%% For quotients %%%
%\newcommand{\pairr}[1]{{\langle #1\rangle}}
\newcommand{\pairr}[1]{{\mathopen{}(#1)\mathclose{}}}
\newcommand{\Pairr}[1]{{\mathopen{}\left(#1\right)\mathclose{}}}

% \newcommand{\type}{\ensuremath{\mathsf{Type}}} % this command is overridden below, so it's commented out
\newcommand{\im}{\ensuremath{\mathsf{im}}} % the image

%%% 2D path operations
\newcommand{\leftwhisker}{\mathbin{{\ct}_{\mathsf{l}}}}  % was \ell
\newcommand{\rightwhisker}{\mathbin{{\ct}_{\mathsf{r}}}} % was r
\newcommand{\hct}{\star}

%%% modalities %%%
\newcommand{\modal}{\ensuremath{\ocircle}}
\let\reflect\modal
\newcommand{\modaltype}{\ensuremath{\type_\modal}}
% \newcommand{\ism}[1]{\ensuremath{\mathsf{is}_{#1}}}
% \newcommand{\ismodal}{\ism{\modal}}
% \newcommand{\existsmodal}{\ensuremath{{\exists}_{\modal}}}
% \newcommand{\existsmodalunique}{\ensuremath{{\exists!}_{\modal}}}
% \newcommand{\modalfunc}{\textsf{\modal-fun}}
% \newcommand{\Ecirc}{\ensuremath{\mathsf{E}_\modal}}
% \newcommand{\Mcirc}{\ensuremath{\mathsf{M}_\modal}}
\newcommand{\mreturn}{\ensuremath{\eta}}
\let\project\mreturn
%\newcommand{\mbind}[1]{\ensuremath{\hat{#1}}}
\newcommand{\ext}{\mathsf{ext}}
%\newcommand{\mmap}[1]{\ensuremath{\bar{#1}}}
%\newcommand{\mjoin}{\ensuremath{\mreturn^{-1}}}
% Subuniverse
\renewcommand{\P}{\ensuremath{\type_{P}}\xspace}

%%% Localizations
% \newcommand{\islocal}[1]{\ensuremath{\mathsf{islocal}_{#1}}\xspace}
% \newcommand{\loc}[1]{\ensuremath{\mathcal{L}_{#1}}\xspace}

%%% Identity types %%%
\newcommand{\idsym}{{=}}
\newcommand{\id}[3][]{\ensuremath{#2 =_{#1} #3}\xspace}
\newcommand{\idtype}[3][]{\ensuremath{\mathsf{Id}_{#1}(#2,#3)}\xspace}
\newcommand{\idtypevar}[1]{\ensuremath{\mathsf{Id}_{#1}}\xspace}
% A propositional equality currently being defined
\newcommand{\defid}{\coloneqq}

%%% Dependent paths
\newcommand{\dpath}[4]{#3 =^{#1}_{#2} #4}

%%% singleton
% \newcommand{\sgl}{\ensuremath{\mathsf{sgl}}\xspace}
% \newcommand{\sctr}{\ensuremath{\mathsf{sctr}}\xspace}

%%% Reflexivity terms %%%
% \newcommand{\reflsym}{{\mathsf{refl}}}
\newcommand{\refl}[1]{\ensuremath{\mathsf{refl}_{#1}}\xspace}

%%% Path concatenation (used infix, in diagrammatic order) %%%
\newcommand{\ct}{%
  \mathchoice{\mathbin{\raisebox{0.5ex}{$\displaystyle\centerdot$}}}%
             {\mathbin{\raisebox{0.5ex}{$\centerdot$}}}%
             {\mathbin{\raisebox{0.25ex}{$\scriptstyle\,\centerdot\,$}}}%
             {\mathbin{\raisebox{0.1ex}{$\scriptscriptstyle\,\centerdot\,$}}}
}

%%% Path reversal %%%
\newcommand{\opp}[1]{\mathord{{#1}^{-1}}}
\let\rev\opp

%%% Coherence paths %%%
\newcommand{\ctassoc}{\mathsf{assoc}} % associativity law

%%% Transport (covariant) %%%
\newcommand{\trans}[2]{\ensuremath{{#1}_{*}\mathopen{}\left({#2}\right)\mathclose{}}\xspace}
\let\Trans\trans
%\newcommand{\Trans}[2]{\ensuremath{{#1}_{*}\left({#2}\right)}\xspace}
\newcommand{\transf}[1]{\ensuremath{{#1}_{*}}\xspace} % Without argument
%\newcommand{\transport}[2]{\ensuremath{\mathsf{transport}_{*} \: {#2}\xspace}}
\newcommand{\transfib}[3]{\ensuremath{\mathsf{transport}^{#1}(#2,#3)\xspace}}
\newcommand{\Transfib}[3]{\ensuremath{\mathsf{transport}^{#1}\Big(#2,\, #3\Big)\xspace}}
\newcommand{\transfibf}[1]{\ensuremath{\mathsf{transport}^{#1}\xspace}}

%%% 2D transport
\newcommand{\transtwo}[2]{\ensuremath{\mathsf{transport}^2\mathopen{}\left({#1},{#2}\right)\mathclose{}}\xspace}

%%% Constant transport
\newcommand{\transconst}[3]{\ensuremath{\mathsf{transportconst}}^{#1}_{#2}(#3)\xspace}
\newcommand{\transconstf}{\ensuremath{\mathsf{transportconst}}\xspace}

%%% Map on paths %%%
\newcommand{\mapfunc}[1]{\ensuremath{\mathsf{ap}_{#1}}\xspace} % Without argument
\newcommand{\map}[2]{\ensuremath{{#1}\mathopen{}\left({#2}\right)\mathclose{}}\xspace}
\let\Ap\map
%\newcommand{\Ap}[2]{\ensuremath{{#1}\left({#2}\right)}\xspace}
\newcommand{\mapdepfunc}[1]{\ensuremath{\mathsf{apd}_{#1}}\xspace} % Without argument
% \newcommand{\mapdep}[2]{\ensuremath{{#1}\llparenthesis{#2}\rrparenthesis}\xspace}
\newcommand{\mapdep}[2]{\ensuremath{\mapdepfunc{#1}\mathopen{}\left(#2\right)\mathclose{}}\xspace}
\let\apfunc\mapfunc
\let\ap\map
\let\apdfunc\mapdepfunc
\let\apd\mapdep

%%% 2D map on paths
\newcommand{\aptwofunc}[1]{\ensuremath{\mathsf{ap}^2_{#1}}\xspace}
\newcommand{\aptwo}[2]{\ensuremath{\aptwofunc{#1}\mathopen{}\left({#2}\right)\mathclose{}}\xspace}
\newcommand{\apdtwofunc}[1]{\ensuremath{\mathsf{apd}^2_{#1}}\xspace}
\newcommand{\apdtwo}[2]{\ensuremath{\apdtwofunc{#1}\mathopen{}\left(#2\right)\mathclose{}}\xspace}

%%% Identity functions %%%
\newcommand{\idfunc}[1][]{\ensuremath{\mathsf{id}_{#1}}\xspace}

%%% Homotopies (written infix) %%%
\newcommand{\htpy}{\sim}

%%% Other meanings of \sim
\newcommand{\bisim}{\sim}       % bisimulation
\newcommand{\eqr}{\sim}         % an equivalence relation

%%% Equivalence types %%%
\newcommand{\eqv}[2]{\ensuremath{#1 \simeq #2}\xspace}
\newcommand{\eqvspaced}[2]{\ensuremath{#1 \;\simeq\; #2}\xspace}
\newcommand{\eqvsym}{\simeq}    % infix symbol
\newcommand{\texteqv}[2]{\ensuremath{\mathsf{Equiv}(#1,#2)}\xspace}
\newcommand{\isequiv}{\ensuremath{\mathsf{isequiv}}}
\newcommand{\qinv}{\ensuremath{\mathsf{qinv}}}
\newcommand{\ishae}{\ensuremath{\mathsf{ishae}}}
\newcommand{\linv}{\ensuremath{\mathsf{linv}}}
\newcommand{\rinv}{\ensuremath{\mathsf{rinv}}}
\newcommand{\biinv}{\ensuremath{\mathsf{biinv}}}
\newcommand{\lcoh}[3]{\mathsf{lcoh}_{#1}(#2,#3)}
\newcommand{\rcoh}[3]{\mathsf{rcoh}_{#1}(#2,#3)}
\newcommand{\hfib}[2]{{\mathsf{fib}}_{#1}(#2)}

%%% Map on total spaces %%%
\newcommand{\total}[1]{\ensuremath{\mathsf{total}(#1)}}

%%% Universe types %%%
%\newcommand{\type}{\ensuremath{\mathsf{Type}}\xspace}
\newcommand{\UU}{\ensuremath{\mathcal{U}}\xspace}
\let\bbU\UU
\let\type\UU
% Universes of truncated types
\newcommand{\typele}[1]{\ensuremath{{#1}\text-\mathsf{Type}}\xspace}
\newcommand{\typeleU}[1]{\ensuremath{{#1}\text-\mathsf{Type}_\UU}\xspace}
\newcommand{\typelep}[1]{\ensuremath{{(#1)}\text-\mathsf{Type}}\xspace}
\newcommand{\typelepU}[1]{\ensuremath{{(#1)}\text-\mathsf{Type}_\UU}\xspace}
\let\ntype\typele
\let\ntypeU\typeleU
\let\ntypep\typelep
\let\ntypepU\typelepU
\renewcommand{\set}{\ensuremath{\mathsf{Set}}\xspace}
\newcommand{\setU}{\ensuremath{\mathsf{Set}_\UU}\xspace}
\newcommand{\prop}{\ensuremath{\mathsf{Prop}}\xspace}
\newcommand{\propU}{\ensuremath{\mathsf{Prop}_\UU}\xspace}
%Pointed types
\newcommand{\pointed}[1]{\ensuremath{#1_\bullet}}

%%% Ordinals and cardinals
\newcommand{\card}{\ensuremath{\mathsf{Card}}\xspace}
\newcommand{\ord}{\ensuremath{\mathsf{Ord}}\xspace}
\newcommand{\ordsl}[2]{{#1}_{/#2}}

%%% Univalence
\newcommand{\ua}{\ensuremath{\mathsf{ua}}\xspace} % the inverse of idtoeqv
\newcommand{\idtoeqv}{\ensuremath{\mathsf{idtoeqv}}\xspace}
\newcommand{\univalence}{\ensuremath{\mathsf{univalence}}\xspace} % the full axiom

%%% Truncation levels
\newcommand{\iscontr}{\ensuremath{\mathsf{isContr}}}
\newcommand{\contr}{\ensuremath{\mathsf{contr}}} % The path to the center of contraction
\newcommand{\isset}{\ensuremath{\mathsf{isSet}}}
\newcommand{\isprop}{\ensuremath{\mathsf{isProp}}}
% h-propositions
% \newcommand{\anhprop}{a mere proposition\xspace}
% \newcommand{\hprops}{mere propositions\xspace}

%%% Homotopy fibers %%%
%\newcommand{\hfiber}[2]{\ensuremath{\mathsf{hFiber}(#1,#2)}\xspace}
\let\hfiber\hfib

%%% Bracket/squash/truncation types %%%
% \newcommand{\brck}[1]{\textsf{mere}(#1)}
% \newcommand{\Brck}[1]{\textsf{mere}\Big(#1\Big)}
% \newcommand{\trunc}[2]{\tau_{#1}(#2)}
% \newcommand{\Trunc}[2]{\tau_{#1}\Big(#2\Big)}
% \newcommand{\truncf}[1]{\tau_{#1}}
%\newcommand{\trunc}[2]{\Vert #2\Vert_{#1}}
\newcommand{\trunc}[2]{\mathopen{}\left\Vert #2\right\Vert_{#1}\mathclose{}}
\newcommand{\ttrunc}[2]{\bigl\Vert #2\bigr\Vert_{#1}}
\newcommand{\Trunc}[2]{\Bigl\Vert #2\Bigr\Vert_{#1}}
\newcommand{\truncf}[1]{\Vert \blank \Vert_{#1}}
\newcommand{\tproj}[3][]{\mathopen{}\left|#3\right|_{#2}^{#1}\mathclose{}}
\newcommand{\tprojf}[2][]{|\blank|_{#2}^{#1}}
\def\pizero{\trunc0}
%\newcommand{\brck}[1]{\trunc{-1}{#1}}
%\newcommand{\Brck}[1]{\Trunc{-1}{#1}}
%\newcommand{\bproj}[1]{\tproj{-1}{#1}}
%\newcommand{\bprojf}{\tprojf{-1}}

\newcommand{\brck}[1]{\trunc{}{#1}}
\newcommand{\bbrck}[1]{\ttrunc{}{#1}}
\newcommand{\Brck}[1]{\Trunc{}{#1}}
\newcommand{\bproj}[1]{\tproj{}{#1}}
\newcommand{\bprojf}{\tprojf{}}

% Big parentheses
\newcommand{\Parens}[1]{\Bigl(#1\Bigr)}

% Projection and extension for truncations
\let\extendsmb\ext
\newcommand{\extend}[1]{\extendsmb(#1)}

%
%%% The empty type
\newcommand{\emptyt}{\ensuremath{\mathbf{0}}\xspace}

%%% The unit type
\newcommand{\unit}{\ensuremath{\mathbf{1}}\xspace}
\newcommand{\ttt}{\ensuremath{\star}\xspace}

%%% The two-element type
\newcommand{\bool}{\ensuremath{\mathbf{2}}\xspace}
\newcommand{\btrue}{{1_{\bool}}}
\newcommand{\bfalse}{{0_{\bool}}}

%%% Injections into binary sums and pushouts
\newcommand{\inlsym}{{\mathsf{inl}}}
\newcommand{\inrsym}{{\mathsf{inr}}}
\newcommand{\inl}{\ensuremath\inlsym\xspace}
\newcommand{\inr}{\ensuremath\inrsym\xspace}

%%% The segment of the interval
\newcommand{\seg}{\ensuremath{\mathsf{seg}}\xspace}

%%% Free groups
\newcommand{\freegroup}[1]{F(#1)}
\newcommand{\freegroupx}[1]{F'(#1)} % the "other" free group

%%% Glue of a pushout
\newcommand{\glue}{\mathsf{glue}}

%%% Colimits
\newcommand{\colim}{\mathsf{colim}}
\newcommand{\inc}{\mathsf{inc}}
\newcommand{\cmp}{\mathsf{cmp}}

%%% Circles and spheres
\newcommand{\Sn}{\mathbb{S}}
\newcommand{\base}{\ensuremath{\mathsf{base}}\xspace}
\newcommand{\lloop}{\ensuremath{\mathsf{loop}}\xspace}
\newcommand{\surf}{\ensuremath{\mathsf{surf}}\xspace}

%%% Suspension
\newcommand{\susp}{\Sigma}
\newcommand{\north}{\mathsf{N}}
\newcommand{\south}{\mathsf{S}}
\newcommand{\merid}{\mathsf{merid}}

%%% Blanks (shorthand for lambda abstractions)
\newcommand{\blank}{\mathord{\hspace{1pt}\text{--}\hspace{1pt}}}

%%% Nameless objects
\newcommand{\nameless}{\mathord{\hspace{1pt}\underline{\hspace{1ex}}\hspace{1pt}}}

%%% Some decorations
%\newcommand{\bbU}{\ensuremath{\mathbb{U}}\xspace}
% \newcommand{\bbB}{\ensuremath{\mathbb{B}}\xspace}
\newcommand{\bbP}{\ensuremath{\mathbb{P}}\xspace}

%%% Some categories
\newcommand{\uset}{\ensuremath{\mathcal{S}et}\xspace}
\newcommand{\ucat}{\ensuremath{{\mathcal{C}at}}\xspace}
\newcommand{\urel}{\ensuremath{\mathcal{R}el}\xspace}
\newcommand{\uhilb}{\ensuremath{\mathcal{H}ilb}\xspace}
\newcommand{\utype}{\ensuremath{\mathcal{T}\!ype}\xspace}

% Pullback corner
\newbox\pbbox
\setbox\pbbox=\hbox{\xy \POS(65,0)\ar@{-} (0,0) \ar@{-} (65,65)\endxy}
\def\pb{\save[]+<3.5mm,-3.5mm>*{\copy\pbbox} \restore}

% Macros for the categories chapter
\newcommand{\inv}[1]{{#1}^{-1}}
\newcommand{\idtoiso}{\ensuremath{\mathsf{idtoiso}}\xspace}
\newcommand{\isotoid}{\ensuremath{\mathsf{isotoid}}\xspace}
\newcommand{\op}{^{\mathrm{op}}}
\newcommand{\y}{\ensuremath{\mathbf{y}}\xspace}
\newcommand{\dgr}[1]{{#1}^{\dagger}}
\newcommand{\unitaryiso}{\mathrel{\cong^\dagger}}
\newcommand{\cteqv}[2]{\ensuremath{#1 \simeq #2}\xspace}
\newcommand{\cteqvsym}{\simeq}     % Symbol for equivalence of categories

%%% Natural numbers
\newcommand{\N}{\ensuremath{\mathbb{N}}\xspace}
%\newcommand{\N}{\textbf{N}}
\let\nat\N
\newcommand{\natp}{\ensuremath{\nat'}\xspace} % alternative nat in induction chapter

\newcommand{\zerop}{\ensuremath{0'}\xspace}   % alternative zero in induction chapter
\newcommand{\suc}{\mathsf{succ}}
\newcommand{\sucp}{\ensuremath{\suc'}\xspace} % alternative suc in induction chapter
\newcommand{\add}{\mathsf{add}}
\newcommand{\ack}{\mathsf{ack}}
\newcommand{\ite}{\mathsf{iter}}
\newcommand{\assoc}{\mathsf{assoc}}
\newcommand{\dbl}{\ensuremath{\mathsf{double}}}
\newcommand{\dblp}{\ensuremath{\dbl'}\xspace} % alternative double in induction chapter


%%% Lists
\newcommand{\lst}[1]{\mathsf{List}(#1)}
\newcommand{\nil}{\mathsf{nil}}
\newcommand{\cons}{\mathsf{cons}}
\newcommand{\lost}[1]{\mathsf{Lost}(#1)}

%%% Vectors of given length, used in induction chapter
\newcommand{\vect}[2]{\ensuremath{\mathsf{Vec}_{#1}(#2)}\xspace}

%%% Integers
\newcommand{\Z}{\ensuremath{\mathbb{Z}}\xspace}
\newcommand{\Zsuc}{\mathsf{succ}}
\newcommand{\Zpred}{\mathsf{pred}}

%%% Rationals
\newcommand{\Q}{\ensuremath{\mathbb{Q}}\xspace}

%%% Function extensionality
\newcommand{\funext}{\mathsf{funext}}
\newcommand{\happly}{\mathsf{happly}}

%%% A naturality lemma
\newcommand{\com}[3]{\mathsf{swap}_{#1,#2}(#3)}

%%% Code/encode/decode
\newcommand{\code}{\ensuremath{\mathsf{code}}\xspace}
\newcommand{\encode}{\ensuremath{\mathsf{encode}}\xspace}
\newcommand{\decode}{\ensuremath{\mathsf{decode}}\xspace}

% Function definition with domain and codomain
\newcommand{\function}[4]{\left\{\begin{array}{rcl}#1 &
      \longrightarrow & #2 \\ #3 & \longmapsto & #4 \end{array}\right.}

%%% Cones and cocones
\newcommand{\cone}[2]{\mathsf{cone}_{#1}(#2)}
\newcommand{\cocone}[2]{\mathsf{cocone}_{#1}(#2)}
% Apply a function to a cocone
\newcommand{\composecocone}[2]{#1\circ#2}
\newcommand{\composecone}[2]{#2\circ#1}
%%% Diagrams
\newcommand{\Ddiag}{\mathscr{D}}

%%% (pointed) mapping spaces
\newcommand{\Map}{\mathsf{Map}}

%%% The interval
\newcommand{\interval}{\ensuremath{I}\xspace}
\newcommand{\izero}{\ensuremath{0_{\interval}}\xspace}
\newcommand{\ione}{\ensuremath{1_{\interval}}\xspace}

%%% Arrows
\newcommand{\epi}{\ensuremath{\twoheadrightarrow}}
\newcommand{\mono}{\ensuremath{\rightarrowtail}}

%%% Sets
\newcommand{\bin}{\ensuremath{\mathrel{\widetilde{\in}}}}

%%% Semigroup structure
\newcommand{\semigroupstrsym}{\ensuremath{\mathsf{SemigroupStr}}}
\newcommand{\semigroupstr}[1]{\ensuremath{\mathsf{SemigroupStr}}(#1)}
\newcommand{\semigroup}[0]{\ensuremath{\mathsf{Semigroup}}}

%%% Macros for the formal type theory
\newcommand{\emptyctx}{\ensuremath{\cdot}}
\newcommand{\production}{\vcentcolon\vcentcolon=}
\newcommand{\conv}{\downarrow}
\newcommand{\ctx}{\ensuremath{\mathsf{ctx}}}
\newcommand{\wfctx}[1]{#1\ \ctx}
\newcommand{\oftp}[3]{#1 \vdash #2 : #3}
\newcommand{\jdeqtp}[4]{#1 \vdash #2 \jdeq #3 : #4}
\newcommand{\judg}[2]{#1 \vdash #2}
\newcommand{\tmtp}[2]{#1 \mathord{:} #2}

% rule names
\newcommand{\rform}{\textsc{form}}
\newcommand{\rintro}{\textsc{intro}}
\newcommand{\relim}{\textsc{elim}}
\newcommand{\rcomp}{\textsc{comp}}
\newcommand{\runiq}{\textsc{uniq}}
\newcommand{\Weak}{\mathsf{Wkg}}
\newcommand{\Vble}{\mathsf{Vble}}
\newcommand{\Exch}{\mathsf{Exch}}
\newcommand{\Subst}{\mathsf{Subst}}

%%% Macros for HITs
\newcommand{\cc}{\mathsf{c}}
\newcommand{\pp}{\mathsf{p}}
\newcommand{\cct}{\widetilde{\mathsf{c}}}
\newcommand{\ppt}{\widetilde{\mathsf{p}}}
\newcommand{\Wtil}{\ensuremath{\widetilde{W}}\xspace}

%%% Macros for n-types
\newcommand{\istype}[1]{\mathsf{is}\mbox{-}{#1}\mbox{-}\mathsf{type}}
\newcommand{\nplusone}{\ensuremath{(n+1)}}
\newcommand{\nminusone}{\ensuremath{(n-1)}}
\newcommand{\fact}{\mathsf{fact}}

%%% Macros for homotopy
\newcommand{\kbar}{\overline{k}} % Used in van Kampen's theorem

%%% Macros for induction
\newcommand{\natw}{\ensuremath{\mathbf{N^w}}\xspace}
\newcommand{\zerow}{\ensuremath{0^\mathbf{w}}\xspace}
\newcommand{\sucw}{\ensuremath{\mathsf{succ}^{\mathbf{w}}}\xspace}
\newcommand{\nalg}{\nat\mathsf{Alg}}
\newcommand{\nhom}{\nat\mathsf{Hom}}
\newcommand{\ishinitw}{\mathsf{isHinit}_{\mathsf{W}}}
\newcommand{\ishinitn}{\mathsf{isHinit}_\nat}
\newcommand{\w}{\mathsf{W}}
\newcommand{\walg}{\w\mathsf{Alg}}
\newcommand{\whom}{\w\mathsf{Hom}}

%%% Macros for real numbers
\newcommand{\RC}{\ensuremath{\mathbb{R}_\mathsf{c}}\xspace} % Cauchy
\newcommand{\RD}{\ensuremath{\mathbb{R}_\mathsf{d}}\xspace} % Dedekind
\newcommand{\R}{\ensuremath{\mathbb{R}}\xspace}           % Either
\newcommand{\barRD}{\ensuremath{\bar{\mathbb{R}}_\mathsf{d}}\xspace} % Dedekind completion of Dedekind

\newcommand{\close}[1]{\sim_{#1}} % Relation of closeness
\newcommand{\closesym}{\mathord\sim}
\newcommand{\rclim}{\mathsf{lim}} % HIT constructor for Cauchy reals
\newcommand{\rcrat}{\mathsf{rat}} % Embedding of rationals into Cauchy reals
\newcommand{\rceq}{\mathsf{eq}_{\RC}} % HIT path constructor
\newcommand{\CAP}{\mathcal{C}}    % The type of Cauchy approximations
\newcommand{\Qp}{\Q_{+}}
\newcommand{\apart}{\mathrel{\#}}  % apartness
\newcommand{\dcut}{\mathsf{isCut}}  % Dedekind cut
\newcommand{\cover}{\triangleleft} % inductive cover
\newcommand{\intfam}[3]{(#2, \lam{#1} #3)} % family of rational intervals

% Macros for the Cauchy reals construction
\newcommand{\bsim}{\frown}
\newcommand{\bbsim}{\smile}

\newcommand{\hapx}{\diamondsuit\approx}
\newcommand{\hapname}{\diamondsuit}
\newcommand{\hapxb}{\heartsuit\approx}
\newcommand{\hapbname}{\heartsuit}
\newcommand{\tap}[1]{\bullet\approx_{#1}\triangle}
\newcommand{\tapname}{\triangle}
\newcommand{\tapb}[1]{\bullet\approx_{#1}\square}
\newcommand{\tapbname}{\square}

%%% Macros for surreals
\newcommand{\NO}{\ensuremath{\mathsf{No}}\xspace}
\newcommand{\surr}[2]{\{\,#1\,\big|\,#2\,\}}
\newcommand{\LL}{\mathcal{L}}
\newcommand{\RR}{\mathcal{R}}
\newcommand{\noeq}{\mathsf{eq}_{\NO}} % HIT path constructor

\newcommand{\ble}{\trianglelefteqslant}
\newcommand{\blt}{\vartriangleleft}
\newcommand{\bble}{\sqsubseteq}
\newcommand{\bblt}{\sqsubset}

\newcommand{\hle}{\diamondsuit\preceq}
\newcommand{\hlt}{\diamondsuit\prec}
\newcommand{\hlname}{\diamondsuit}
\newcommand{\hleb}{\heartsuit\preceq}
\newcommand{\hltb}{\heartsuit\prec}
\newcommand{\hlbname}{\heartsuit}
% \newcommand{\tle}{(\bullet\preceq\triangle)}
% \newcommand{\tlt}{(\bullet\prec\triangle)}
\newcommand{\tle}{\triangle\preceq}
\newcommand{\tlt}{\triangle\prec}
\newcommand{\tlname}{\triangle}
% \newcommand{\tleb}{(\bullet\preceq\square)}
% \newcommand{\tltb}{(\bullet\prec\square)}
\newcommand{\tleb}{\square\preceq}
\newcommand{\tltb}{\square\prec}
\newcommand{\tlbname}{\square}

%%% Macros for set theory
\newcommand{\vset}{\mathsf{set}}  % point constructor for cummulative hierarchy V
\def\cd{\tproj0}
\newcommand{\inj}{\ensuremath{\mathsf{inj}}} % type of injections
\newcommand{\acc}{\ensuremath{\mathsf{acc}}} % accessibility

\newcommand{\atMostOne}{\mathsf{atMostOne}}

\newcommand{\power}[1]{\mathcal{P}(#1)} % power set
\newcommand{\powerp}[1]{\mathcal{P}_+(#1)} % inhabited power set

%%%% THEOREM ENVIRONMENTS %%%%

% The cleveref package provides \cref{...} which is like \ref{...}
% except that it automatically inserts the type of the thing you're
% referring to, e.g. it produces "Theorem 3.8" instead of just "3.8"
% (and hyperref makes the whole thing a hyperlink).  This saves a slight amount
% of typing, but more importantly it means that if you decide later on
% that 3.8 should be a Lemma or a Definition instead of a Theorem, you
% don't have to change the name in all the places you referred to it.

% The following hack improves on this by using the same counter for
% all theorem-type environments, so that after Theorem 1.1 comes
% Corollary 1.2 rather than Corollary 1.1.  This makes it much easier
% for the reader to find a particular theorem when flipping through
% the document.
\makeatletter
\def\defthm#1#2#3{%
  %% Ensure all theorem types are numbered with the same counter
  \newaliascnt{#1}{thm}
  \newtheorem{#1}[#1]{#2}
  \aliascntresetthe{#1}
  %% This command tells cleveref's \cref what to call things
  \crefname{#1}{#2}{#3}% following brace must be on separate line to support poorman cleveref sed file
}

% Now define a bunch of theorem-type environments.
\newtheorem{thm}{Theorem}[section]
\crefname{thm}{Theorem}{Theorems}
%\defthm{prop}{Proposition}   % Probably we shouldn't use "Proposition" in this way
\defthm{cor}{Corollary}{Corollaries}
\defthm{lem}{Lemma}{Lemmas}
\defthm{axiom}{Axiom}{Axioms}
% Since definitions and theorems in type theory are synonymous, should
% we actually use the same theoremstyle for them?
\theoremstyle{definition}
\defthm{defn}{Definition}{Definitions}
\theoremstyle{remark}
\defthm{rmk}{Remark}{Remarks}
\defthm{eg}{Example}{Examples}
\defthm{egs}{Examples}{Examples}
\defthm{notes}{Notes}{Notes}
% Number exercises within chapters, with their own counter.
\newtheorem{ex}{Exercise}[chapter]
\crefname{ex}{Exercise}{Exercises}

% Display format for sections
\crefformat{section}{\S#2#1#3}
\Crefformat{section}{Section~#2#1#3}
\crefrangeformat{section}{\S\S#3#1#4--#5#2#6}
\Crefrangeformat{section}{Sections~#3#1#4--#5#2#6}
\crefmultiformat{section}{\S\S#2#1#3}{ and~#2#1#3}{, #2#1#3}{ and~#2#1#3}
\Crefmultiformat{section}{Sections~#2#1#3}{ and~#2#1#3}{, #2#1#3}{ and~#2#1#3}
\crefrangemultiformat{section}{\S\S#3#1#4--#5#2#6}{ and~#3#1#4--#5#2#6}{, #3#1#4--#5#2#6}{ and~#3#1#4--#5#2#6}
\Crefrangemultiformat{section}{Sections~#3#1#4--#5#2#6}{ and~#3#1#4--#5#2#6}{, #3#1#4--#5#2#6}{ and~#3#1#4--#5#2#6}

% Display format for appendices
\crefformat{appendix}{Appendix~#2#1#3}
\Crefformat{appendix}{Appendix~#2#1#3}
\crefrangeformat{appendix}{Appendices~#3#1#4--#5#2#6}
\Crefrangeformat{appendix}{Appendices~#3#1#4--#5#2#6}
\crefmultiformat{appendix}{Appendices~#2#1#3}{ and~#2#1#3}{, #2#1#3}{ and~#2#1#3}
\Crefmultiformat{appendix}{Appendices~#2#1#3}{ and~#2#1#3}{, #2#1#3}{ and~#2#1#3}
\crefrangemultiformat{appendix}{Appendices~#3#1#4--#5#2#6}{ and~#3#1#4--#5#2#6}{, #3#1#4--#5#2#6}{ and~#3#1#4--#5#2#6}
\Crefrangemultiformat{appendix}{Appendices~#3#1#4--#5#2#6}{ and~#3#1#4--#5#2#6}{, #3#1#4--#5#2#6}{ and~#3#1#4--#5#2#6}

\crefname{part}{Part}{Parts}

% Number subsubsections
\setcounter{secnumdepth}{5}

% Display format for figures
\crefname{figure}{Figure}{Figures}

%%%% EQUATION NUMBERING %%%%

% The following hack uses the single theorem counter to number
% equations as well, so that we don't have both Theorem 1.1 and
% equation (1.1).
\let\c@equation\c@thm
\numberwithin{equation}{section}


%%%% ENUMERATE NUMBERING %%%%

% Number the first level of enumerates as (i), (ii), ...
\renewcommand{\theenumi}{(\roman{enumi})}
\renewcommand{\labelenumi}{\theenumi}


%%%% MARGINS %%%%

% This is a matter of personal preference, but I think the left
% margins on enumerates and itemizes are too wide.
\setitemize[1]{leftmargin=2em}
\setenumerate[1]{leftmargin=*}

% Likewise that they are too spaced out.
\setitemize[1]{itemsep=-0.2em}
\setenumerate[1]{itemsep=-0.2em}

%%% Notes %%%
\def\noteson{%
\gdef\note##1{\mbox{}\marginpar{\color{blue}\textasteriskcentered\ ##1}}}
\gdef\notesoff{\gdef\note##1{\null}}
\noteson

\newcommand{\Coq}{\textsc{Coq}\xspace}
\newcommand{\Agda}{\textsc{Agda}\xspace}
\newcommand{\NuPRL}{\textsc{NuPRL}\xspace}

%%%% CITATIONS %%%%

% \let \cite \citep

%%%% INDEX %%%%

\newcommand{\footstyle}[1]{{\hyperpage{#1}}n} % If you index something that is in a footnote
\newcommand{\defstyle}[1]{\textbf{\hyperpage{#1}}}  % Style for pageref to a definition

\newcommand{\indexdef}[1]{\index{#1|defstyle}}   % Index a definition
\newcommand{\indexfoot}[1]{\index{#1|footstyle}} % Index a term in a footnote
\newcommand{\indexsee}[2]{\index{#1|see{#2}}}    % Index "see also"


%%%% Standard phrasing or spelling of common phrases %%%%

\newcommand{\ZF}{Zermelo--Fraenkel}
\newcommand{\CZF}{Constructive \ZF{} Set Theory}

\newcommand{\LEM}[1]{\ensuremath{\mathsf{LEM}_{#1}}\xspace}
\newcommand{\choice}[1]{\ensuremath{\mathsf{AC}_{#1}}\xspace}

%%%% MISC %%%%

\newcommand{\mentalpause}{\medskip} % Use for "mental" pause, instead of \smallskip or \medskip

%% Use \symlabel instead of \label to mark a pageref that you need in the index of symbols
\newcounter{symindex}
\newcommand{\symlabel}[1]{\refstepcounter{symindex}\label{#1}}

% Local Variables:
% mode: latex
% TeX-master: "hott-online"
% End:

\lstset{%
language=Caml,
moredelim=*[is][\itshape]{/@}{@/},
numbers=none,mathescape=true,showstringspaces=false,
morekeywords={handle,handler,with,new,effect,operation,type,end,val,finally,match,true,false},
keywordstyle=\relax,
xleftmargin=1em,basicstyle=\ttfamily}
\lstnewenvironment{source}{\lstset{
basicstyle=\ttfamily\small,
% Uncomment to enable bold keywords in source environments
% keywordstyle=\bfseries
}}{}
\let\inline\lstinline

\usepackage[
  pdfauthor={Andrej Bauer and Matija Pretnar},
  pdftitle={Programming with Algebraic Effects and Handlers}
]{hyperref}

\begin{document}

\title{Programming with \\ Algebraic Effects and Handlers}

\author{
Andrej Bauer\\\texttt{\small andrej@andrej.com}
\and
Matija Pretnar\\\texttt{\small matija@pretnar.info}
}
\date{\small Department of Mathematics and Physics\\
University of Ljubljana, Slovenia}

\maketitle

%\keywords{algebraic effects, handlers, programming language}
%\subjclass{D3.3, F3.3}

\begin{abstract}
  \Eff is a programming language based on the algebraic approach to computational effects,
  in which effects are viewed as algebraic operations and effect handlers as homomorphisms
  from free algebras. \Eff supports first-class effects and handlers through which we may
  easily define new computational effects, seamlessly combine existing ones, and handle
  them in novel ways. We give a denotational semantics of \eff and discuss a prototype
  implementation based on it. Through examples we demonstrate how the standard effects
  are treated in \eff, and how \eff supports programming techniques that use various forms
  of delimited continuations, such as backtracking, breadth-first search, selection
  functionals, cooperative multi-threading, and others.
\end{abstract}

\chapter*{Introduction}
\markboth{\textsc{Introduction}}{}
\addcontentsline{toc}{chapter}{Introduction}
\setcounter{page}{1}
\pagenumbering{arabic}


\emph{Homotopy type theory} is a new branch of mathematics that combines aspects of several different fields in a surprising way. It is based on a recently discovered connection between \emph{homotopy theory} and \emph{type theory}.
Homotopy theory is an outgrowth of algebraic topology and homological algebra, with relationships to higher category theory; while type theory is a branch of mathematical logic and theoretical computer science.
Although the connections between the two are currently the focus of intense investigation, it is increasingly clear that they are just the beginning of a subject that will take more time and more hard work to fully understand.
It touches on topics as seemingly distant as the homotopy groups of spheres, the algorithms for type checking, and the definition of weak $\infty$-groupoids.

Homotopy type theory also brings new ideas into the very foundation of mathematics.
\index{foundations, univalent}%
On the one hand, there is Voevodsky's subtle and beautiful \emph{univalence axiom}. 
\index{univalence axiom}%
The univalence axiom implies, in particular, that isomorphic structures can be identified, a principle that mathematicians have been happily using on workdays, despite its incompatibility with the ``official'' doctrines of conventional foundations.
On the other hand, we have \emph{higher inductive types}, which provide direct, logical descriptions of some of the basic spaces and constructions of homotopy theory: spheres, cylinders, truncations, localizations, etc.
Both ideas are impossible to capture directly in classical set-theoretic foundations, but when combined in homotopy type theory, they permit an entirely new kind of ``logic of homotopy types''.
\index{foundations}%

This suggests a new conception of foundations of mathematics, with intrinsic homotopical content, an ``invariant'' conception of the objects of mathematics --- and convenient machine implementations, which can serve as a practical aid to the working mathematician.
This is the \emph{Univalent Foundations} program.
The present book is intended as a first systematic exposition of the basics of univalent foundations, and a collection of examples of this new style of reasoning --- but without requiring the reader to know or learn any formal logic, or to use any computer proof assistant.

% This enlarges the page by one line in letter format. Use sparringly.
\OPTwidow

We emphasize that homotopy type theory is a young field, and univalent foundations is very much a work in progress. 
This book should be regarded as a ``snapshot'' of just one portion of the field, taken at the time it was written, rather than a polished exposition of a completed edifice. 
As we will discuss briefly later, there are many aspects of homotopy type theory that are not yet fully understood --- and some that are not even touched upon here. 
The ultimate theory will almost certainly not look exactly like the one described in this book, but it will surely be \emph{at least} as capable and powerful; we therefore believe that univalent foundations will eventually become a viable alternative to set theory as the ``implicit foundation'' for the unformalized mathematics done by most mathematicians.

\subsection*{Type theory}

Type theory was originally invented by Bertrand Russell \cite{Russell:1908},\index{Russell, Bertrand} as a device for blocking the paradoxes in the logical foundations of mathematics  that were under investigation at the time.
It was developed further by many people over the next few decades, particularly Church~\cite{Church:1940tu,Church:1941tc} who combined it with his \textit{$\lambda$-calculus}.
Although it is not generally regarded as the foundation for classical mathematics, set theory being more customary, type theory still has numerous applications, especially in computer science and the theory of programming languages~\cite{Pierce-TAPL}.
\index{programming}%
\index{type theory}%
\index{lambda-calculus@$\lambda$-calculus}%
Per Martin-L\"{o}f \cite{Martin-Lof-1972,Martin-Lof-1973,Martin-Lof-1979,martin-lof:bibliopolis}, among others,
developed a ``predicative'' modification of Church's type system, which is now usually called dependent, constructive, intuitionistic, or simply \emph{Martin\--L\"of type theory}. This is the basis of the system that we consider here; it was originally intended as a rigorous framework for the formalization of constructive mathematics.  In what follows, we will often use ``type theory'' to refer specifically to this system and similar ones, although type theory as a subject is much broader (see \cite{somma,kamar} for the history of type theory).

In type theory, unlike set theory, objects are classified using a primitive notion of \emph{type}, similar to the data-types used in programming languages.  These elaborately structured types can be used to express detailed specifications of the objects classified, giving rise to principles of reasoning about these objects.  To take a very simple example, the objects of a product type $A\times B$ are known to be of the form $\pairr{a,b}$, and so one automatically knows how to construct them and how to decompose them. Similarly, an object of function type $A\to B$ can be acquired from an object of type $B$ parametrized by objects of type $A$, and can be evaluated at an argument of type $A$.  This rigidly predictable behavior of all objects (as opposed to set theory's more liberal formation principles, allowing inhomogeneous sets) is one aspect of type theory that has led to its extensive use in verifying the correctness of computer programs.  The clear reasoning principles associated with the construction of types also form the basis of modern \emph{computer proof assistants},%
\index{proof!assistant}%
\indexsee{computer proof assistant}{proof assistant}
\index{mathematics!formalized}%
which are used for formalizing mathematics and verifying the correctness of formalized proofs.  We return to this aspect of type theory below.  

One problem in understanding type theory from a mathematical point of view, however, has always been that the basic concept of \emph{type} is unlike that of \emph{set} in ways that have been hard to make precise.  We believe that the new idea of regarding types, not as strange sets (perhaps constructed without using classical logic), but as spaces, viewed from the perspective of homotopy theory, is a significant step forward.  In particular, it solves the problem of understanding how the notion of equality of elements of a type differs from that of elements of a set.

In homotopy theory one is concerned with spaces
\index{topological!space}%
and continuous mappings between them, 
\index{function!continuous!in classical homotopy theory}%
up to homotopy.  A \emph{homotopy}
\index{homotopy!topological}%
between a pair of continuous maps $f : X \to Y$
and  $g : X\to Y$ is 
a continuous map $H : X \times [0, 1] \to Y$ satisfying
$H(x, 0) = f (x)$  and $H(x, 1) = g(x)$. The homotopy $H$ may be thought of as a ``continuous deformation'' of $f$ into $g$. The spaces $X$ and $Y$ are said to be \emph{homotopy equivalent},
\index{homotopy!equivalence!topological}%
$\eqv X Y$, if there are continuous maps going back and forth, the composites of which are homotopical to the respective identity mappings, i.e., if they are isomorphic ``up to homotopy''.  Homotopy equivalent spaces have the same algebraic invariants (e.g., homology, or the fundamental group), and are said to have the same \emph{homotopy type}.

\subsection*{Homotopy type theory}

Homotopy type theory (HoTT) interprets type theory from a homotopical perspective.
In homotopy type theory, we regard the types as ``spaces'' (as studied in homotopy theory) or higher groupoids, and the logical constructions (such as the product $A\times B$) as homotopy-invariant constructions on these spaces.
In this way, we are able to manipulate spaces directly without first having to develop point-set topology (or any combinatorial replacement for it, such as the theory of simplicial sets).
To briefly explain this perspective, consider first the basic concept of type theory, namely that
the \emph{term} $a$ is of \emph{type} $A$, which is written:
\[ a:A. \]
This expression is traditionally thought of as akin to:
\begin{center}
``$a$ is an element of the set $A$''.
\end{center}
However, in homotopy type theory we think of it instead as:
\begin{center}
``$a$ is a point of the space $A$''.
\end{center}
\index{continuity of functions in type theory@``continuity'' of functions in type theory}%
Similarly, every function $f : A\to B$ in type theory is regarded as a continuous map from the space $A$ to the space $B$.

We should stress that these ``spaces'' are treated purely homotopically, not topologically.
For instance, there is no notion of ``open subset'' of a type or of ``convergence'' of a sequence of elements of a type.
We only have ``homotopical'' notions, such as paths between points and homotopies between paths, which also make sense in other models of homotopy theory (such as simplicial sets).
Thus, it would be more accurate to say that we treat types as \emph{$\infty$-groupoids}\index{.infinity-groupoid@$\infty$-groupoid}; this is a name for the ``invariant objects'' of homotopy theory which can be presented by topological spaces,
\index{topological!space}%
simplicial sets, or any other model for homotopy theory.
However, it is convenient to sometimes use topological words such as ``space'' and ``path'', as long as we remember that other topological concepts are not applicable.

(It is tempting to also use the phrase \emph{homotopy type}
\index{homotopy!type}%
for these objects, suggesting the dual interpretation of ``a type (as in type theory) viewed homotopically'' and ``a space considered from the point of view of homotopy theory''.
The latter is a bit different from the classical meaning of ``homotopy type'' as an \emph{equivalence class} of spaces modulo homotopy equivalence, although it does preserve the meaning of phrases such as ``these two spaces have the same homotopy type''.)

The idea of interpreting types as structured objects, rather than sets, has a long pedigree, and is known to clarify various mysterious aspects of type theory.
For instance, interpreting types as sheaves helps explain the intuitionistic nature of type-theoretic logic, while interpreting them as partial equivalence relations or ``domains'' helps explain its computational aspects.
It also implies that we can use type-theoretic reasoning to study the structured objects, leading to the rich field of categorical logic.
The homotopical interpretation fits this same pattern: it clarifies the nature of \emph{identity} (or equality) in type theory, and allows us to use type-theoretic reasoning in the study of homotopy theory.

The key new idea of the homotopy interpretation is that the logical notion of identity $a = b$ of two objects $a, b: A$ of the same type $A$ can be understood as the existence of a path $p : a \leadsto b$ from point $a$ to point $b$ in the space $A$.
This also means that two functions $f, g: A\to B$ can be identified if they are homotopic, since a homotopy is just a (continuous) family of paths $p_x: f(x) \leadsto g(x)$ in $B$, one for each $x:A$.
In type theory, for every type $A$ there is a (formerly somewhat mysterious) type $\idtypevar{A}$ of identifications of two objects of $A$; in homotopy type theory, this is just the \emph{path space} $A^I$ of all continuous maps $I\to A$ from the unit interval.
\index{unit!interval}%
\index{interval!topological unit}%
\index{path!topological}%
\index{topological!path}%
In this way, a term $p : \idtype[A]{a}{b}$ represents a path $p : a \leadsto b$ in $A$. 

The idea of homotopy type theory arose around 2006 in independent work by Awodey and Warren~\cite{AW} and Voevodsky~\cite{VV}, but it was inspired by 
Hofmann and Streicher's earlier groupoid interpretation~\cite{hs:gpd-typethy}.
Indeed, higher-dimensional category theory (particularly the theory of weak $\infty$-groupoids) is now known to be intimately connected to homotopy theory, as proposed by Grothendieck and now being studied intensely by mathematicians of both sorts.
The original semantic models of Awodey--Warren and Voevodsky use well-known notions and techniques from homotopy theory which are now also in use in higher category theory, such as  Quillen model categories and Kan\index{Kan complex} simplicial sets\index{simplicial!sets}.
\index{Quillen model category}%
\index{model category}%

In particular, Voevodsky constructed an interpretation of type theory in Kan simplicial sets, and recognized that this interpretation satisfied a further crucial property which he dubbed \emph{univalence}.
This had not previously been considered in type theory (although Church's principle of extensionality for propositions turns out to be a very special case of it, and Hofmann and Streicher had considered another special case under the name ``universe extensionality'').
Adding univalence to type theory in the form of a new axiom has far-reaching consequences, many of which are natural, simplifying and compelling.
The univalence axiom also further strengthens the homotopical view of type theory, since it holds in the simplicial model and other related models, while failing under the view of types as sets.

\subsection*{Univalent foundations}

Very briefly, the basic idea of the univalence axiom can be explained as follows.
In type theory, one can have a universe $\UU$, the terms of which are themselves types, $A : \UU$, etc.
Those types that are terms of $\UU$ are commonly called \emph{small} types.
\index{type!small}%
\index{small!type}%
Like any type, $\UU$ has an identity type $\idtypevar{\UU}$, which expresses the identity relation $A = B$ between small types.
Thinking of types as spaces, $\UU$ is a space, the points of which are spaces; to understand its identity type, we must ask, what is a path $p : A \leadsto B$ between spaces in $\UU$?
The univalence axiom says that such paths correspond to homotopy equivalences $\eqv A B$, (roughly) as explained above.
A bit more precisely, given any (small) types $A$ and $B$, in addition to the primitive type $\idtype[\UU]AB$ of identifications of $A$ with $B$, there is the defined type $\texteqv AB$ of equivalences from $A$ to $B$.
Since the identity map on any object is an equivalence, there is a canonical map,
\[\idtype[\UU]AB\to\texteqv AB.\]
The univalence axiom states that this map is itself an equivalence.
At the risk of oversimplifying, we can state this succinctly as follows:

\begin{description}\index{univalence axiom}%
\item[Univalence Axiom:]  $\eqvspaced{(A = B)}{(\eqv A B)}$.
\end{description}
%
In other words, identity is equivalent to equivalence. \index{identity}% 
In particular, one may say that ``equivalent types are identical''.
However, this phrase is somewhat misleading, since it may sound like a sort of ``skeletality'' condition which \emph{collapses} the notion of equivalence to coincide with identity, whereas in fact univalence is about \emph{expanding} the notion of identity so as to coincide with the (unchanged) notion of equivalence.

From the homotopical point of view, univalence implies that spaces of the same homotopy type are connected by a path in the universe $\UU$, in accord with the intuition of a classifying space for (small) spaces.
From the logical point of view, however, it is a radically new idea: it says that isomorphic things can be identified!  Mathematicians are of course used to identifying isomorphic structures in practice, but they generally do so by ``abuse of notation''\index{abuse!of notation}, or some other informal device, knowing that the objects involved are not ``really'' identical.  But in this new foundational scheme, such structures can be formally identified, in the logical sense that every property or construction involving one also applies to the other. Indeed, the identification is now made explicit, and properties and constructions can be systematically transported along it.  Moreover, the different ways in which such identifications may be made themselves form a structure that one can (and should!)\ take into account.

Thus in sum, for points $A$ and $B$ of the universe $\UU$ (i.e., small types), the univalence axiom identifies the following three notions:
\begin{itemize}
\item (logical) an identification $p:A=B$ of $A$ and $B$
\item (topological) a path $p:A \leadsto B$ from $A$ to $B$ in $\UU$
\item (homotopical) an equivalence $p:\eqv A B$ between $A$ and $B$.
\end{itemize}

\subsection*{Higher inductive types}\index{type!higher inductive}%

One of the classical advantages of type theory is its simple and effective techniques for working with inductively defined structures.
The simplest nontrivial inductively defined structure is the natural numbers, which is inductively generated by zero and the successor function.
From this statement one can algorithmically\index{algorithm} extract the principle of mathematical induction, which characterizes the natural numbers.
More general inductive definitions encompass lists and well-founded trees of all sorts, each of which is characterized by a corresponding ``induction principle''.
This includes most data structures used in certain programming languages; hence the usefulness of type theory in formal reasoning about the latter.
If conceived in a very general sense, inductive definitions also include examples such as a disjoint union $A+B$, which may be regarded as ``inductively'' generated by the two injections $A\to A+B$ and $B\to A+B$.
The ``induction principle'' in this case is ``proof by case analysis'', which characterizes the disjoint union.

In homotopy theory, it is natural to consider also ``inductively defined spaces'' which are generated not merely by a collection of \emph{points}, but also by collections of \emph{paths} and higher paths.
Classically, such spaces are called \emph{CW complexes}.
\index{CW complex}%
For instance, the circle $S^1$ is generated by a single point and a single path from that point to itself.
Similarly, the 2-sphere $S^2$ is generated by a single point $b$ and a single two-dimensional path from the constant path at $b$ to itself, while the torus $T^2$ is generated by a single point, two paths $p$ and $q$ from that point to itself, and a two-dimensional path from $p\ct q$ to $q\ct p$.

By using the identification of paths with identities in homotopy type theory, these sort of ``inductively defined spaces'' can be characterized in type theory by ``induction principles'', entirely analogously to classical examples such as the natural numbers and the disjoint union.
The resulting \emph{higher inductive types}
\index{type!higher inductive}%
give a direct ``logical'' way to reason about familiar spaces such as spheres, which (in combination with univalence) can be used to perform familiar arguments from homotopy theory, such as calculating homotopy groups of spheres, in a purely formal way.
The resulting proofs are a marriage of classical homotopy-theoretic ideas with classical type-theoretic ones, yielding new insight into both disciplines.

Moreover, this is only the tip of the iceberg: many abstract constructions from homotopy theory, such as homotopy colimits, suspensions, Postnikov towers, localization, completion, and spectrification, can also be expressed as higher inductive types.
Many of these are classically constructed using Quillen's ``small object argument'', which can be regarded as a finite way of algorithmically describing an infinite CW complex presentation\index{presentation!of a space as a CW complex} of a space, just as ``zero and successor'' is a finite algorithmic\index{algorithm} description of the infinite set of natural numbers.
Spaces produced by the small object argument are infamously complicated and difficult to understand; the type-theoretic approach is potentially much simpler, bypassing the need for any explicit construction by giving direct access to the appropriate ``induction principle''.
Thus, the combination of univalence and higher inductive types suggests the possibility of a revolution, of sorts, in the practice of homotopy theory.


\subsection*{Sets in univalent foundations}

\index{set|(}%

We have claimed that univalent foundations can eventually serve as a foundation for ``all'' of mathematics, but so far we have discussed 
only homotopy theory.  Of course, there are many specific examples of the use of type theory without the new homotopy type theory features to formalize mathematics,
\index{mathematics!formalized}%
\index{theorem!Feit--Thompson}%
\index{theorem!odd-order}%
\index{Feit--Thompson theorem}%
\index{odd-order theorem}%
such as the recent formalization of the Feit--Thompson odd-order theorem in \Coq~\cite{gonthier}.

But the traditional view is that mathematics is founded on set theory, in the sense that all mathematical objects and constructions can be coded into a theory such as Zermelo--Fraenkel set theory (ZF).
\index{set theory!Zermelo--Fraenkel}%
\indexsee{Zermelo-Fraenkel set theory}{set theory}%
\indexsee{ZF}{set theory}%
\indexsee{ZFC}{set theory}%
However, it is well-established by now that for most mathematics outside of set theory proper, the intricate hierarchical membership structure of sets in ZF is really unnecessary: a more ``structural'' theory, such as Lawvere's\index{Lawvere} Elementary Theory of the Category of Sets~\cite{lawvere:etcs-long}, suffices.
\index{Elementary Theory of the Category of Sets}%

In univalent foundations, the basic objects are ``homotopy types'' rather than sets, but we can \emph{define} a class of types which behave like sets.
Homotopically, these can be thought of as spaces in which every connected component is contractible, i.e.\ those which are homotopy equivalent to a discrete space.
\index{discrete!space}%
It is a theorem  that the category of such ``sets'' satisfies Lawvere's\index{Lawvere} axioms (or related ones, depending on the details of the theory).
Thus, any sort of mathematics that can be represented in an ETCS-like theory (which, experience suggests, is essentially all of mathematics) can equally well be represented in univalent foundations.  

This supports the claim that univalent foundations is at least as good as existing foundations of mathematics.
A mathematician working in univalent foundations can build structures out of sets in a familiar way, with more general homotopy types waiting in the foundational background until there is need of them.
For this reason, most of the applications in this book have been chosen to be areas where univalent foundations has something \emph{new} to contribute that distinguishes it from existing foundational systems.

Unsurprisingly, homotopy theory and category theory are two of these, but perhaps less obvious is that univalent foundations has something new and interesting to offer even in subjects such as set theory and real analysis.
For instance, the univalence axiom allows us to identify isomorphic structures, while higher inductive types allow direct descriptions of objects by their universal properties.
Thus we can generally avoid resorting to arbitrarily chosen representatives or transfinite iterative constructions.
In fact, even the objects of study in ZF set theory can be characterized, inside the sets of univalent foundations, by such an inductive universal property.

\index{set|)}%


\subsection*{Informal type theory}

\index{mathematics!formalized|(defstyle}%
\index{informal type theory|(defstyle}%
\index{type theory!informal|(defstyle}%
\index{type theory!formal|(}%
One difficulty often encountered by the classical mathematician when faced with learning about type theory is that it is usually presented as a fully or partially formalized deductive system.
This style, which is very useful for proof-theoretic investigations, is not particularly convenient for use in applied, informal reasoning.
Nor is it even familiar to most working mathematicians, even those who might be interested in foundations of mathematics.
One objective of the present work is to develop an informal style of doing mathematics in univalent foundations that is at once rigorous and precise, but is also closer to the language and style of presentation of everyday mathematics.

In present-day mathematics, one usually constructs and reasons about mathematical objects in a way that could in principle, one presumes, be formalized in a system of elementary set theory, such as ZFC --- at least given enough ingenuity and patience.
For the most part, one does not even need to be aware of this possibility, since it largely coincides with the condition that a proof be ``fully rigorous'' (in the sense that all mathematicians have come to understand intuitively through education and experience).
But one does need to learn to be careful about a few aspects of ``informal set theory'': the use of collections too large or inchoate to be sets; the axiom of choice and its equivalents; even (for undergraduates) the method of proof by contradiction; and so on.
Adopting a new foundational system such as homotopy type theory as the \emph{implicit formal basis} of informal reasoning will require adjusting some of one's instincts and practices.
The present text is intended to serve as an example of this ``new kind of mathematics'', which is still informal, but could now in principle be formalized in homotopy type theory, rather than ZFC, again given enough ingenuity and patience.

It is worth emphasizing that, in this new system, such formalization can have real practical benefits.
The formal system of type theory is suited to computer systems and has been implemented in existing proof assistants.
\index{proof!assistant}%
A proof assistant is a computer program which guides the user in construction of a fully formal proof, only allowing valid steps of reasoning.
It also provides some degree of automation, can search libraries for existing theorems, and can even extract numerical algorithms\index{algorithm} \index{extraction of algorithms} from the resulting (constructive) proofs.

We believe that this aspect of the univalent foundations program distinguishes it from other approaches to foundations, potentially providing a new practical utility for the working mathematician.
Indeed, proof assistants based on older type theories have already been used to formalize substantial mathematical proofs, such as the four-color theorem\index{theorem!four-color} \index{four-color theorem} and the Feit--Thompson theorem.
Computer implementations of univalent foundations are presently works in progress (like the theory itself).
\index{proof!assistant}%
However, even its currently available implementations (which are mostly small modifications to existing proof assistants such as \Coq and 
\Agda) have already demonstrated their worth, not only in the formalization of known proofs, but in the discovery of new ones.
Indeed, many of the proofs described in this book were actually \emph{first} done in a fully formalized form in a proof assistant, and are only now being ``unformalized'' for the first time --- a reversal of the usual relation between formal and informal mathematics.

One can imagine a not-too-distant future when it will be possible for mathematicians to verify the correctness of their own papers by working within the system of univalent foundations, formalized in a proof assistant, and that doing so will become as natural as typesetting their own papers in \TeX.
%(Whether this proves to be the publishers' dream or their nightmare remains to be seen.) 
In principle, this could be equally true for any other foundational system, but we believe it to be more practically attainable using univalent foundations, as witnessed by the present work and its formal counterpart.

\index{type theory!formal|)}%
\index{informal type theory|)}%
\index{type theory!informal|)}%
\index{mathematics!formalized|)}%

\subsection*{Constructivity} 

\index{mathematics!constructive|(}%

One of the most striking differences between classical\index{mathematics!classical} foundations and type theory is the idea of \emph{proof relevance}, according to which mathematical statements, and even their proofs, become first-class mathematical objects.
In type theory, we represent mathematical statements by types, which can be regarded simultaneously as both mathematical constructions and mathematical assertions, a conception also known as \emph{propositions as types}.
\index{proposition!as types}%
Accordingly, we can regard a term $a : A$ as both an element of the type $A$ (or in homotopy type theory, a point of the space $A$), and at the same time, a proof of the proposition $A$.
To take an example, suppose we have sets $A$ and $B$ (discrete spaces),
\index{discrete!space}%
and consider the statement ``$A$ is isomorphic to $B$''.
In type theory, this can be rendered as:
\begin{narrowmultline*}
  \mathsf{Iso}(A,B) \defeq \narrowbreak
  \sm{f : A\to B}{g : B\to A}\Big(\big(\tprd{x:A} g(f(x)) = x\big) \times \big(\tprd{y:B}\, f(g(y)) = y\big)\Big).
\end{narrowmultline*}
%
Reading the type constructors $\Sigma, \Pi, \times$  here  as ``there exists'', ``for all'', and ``and'' respectively yields the usual formulation of ``$A$ and $B$ are isomorphic''; on the other hand, reading them as sums and products yields the \emph{type of all isomorphisms} between $A$ and $B$!  To prove that $A$ and $B$ are isomorphic, one  constructs a proof $p : \mathsf{Iso}(A,B)$, which is therefore the same  as constructing an isomorphism between $A$ and $B$, i.e., exhibiting a pair of functions $f, g$ together with \emph{proofs} that their composites are the respective identity maps.  The latter proofs, in turn, are nothing but homotopies of the appropriate sorts.  In this way, \emph{proving a proposition is the same as constructing an element of some particular type.}
In particular, to prove a statement of the form ``$A$ and $B$'' is just to prove $A$ and to prove $B$, i.e., to give an element of the type $A\times B$.
And to prove that $A$ implies $B$ is just to find an element of $A\to B$, i.e.\ a function from $A$ to $B$ (determining a mapping of proofs of $A$ to proofs of $B$).

The logic of propositions-as-types is flexible and supports many variations, such as using only a subclass of types to represent propositions.
In homotopy type theory, there are natural such subclasses arising from the fact that the system of all types, just like spaces in classical homotopy theory, is ``stratified'' according to the dimensions in which their higher homotopy structure exists or collapses.
In particular, Voevodsky has found a purely type-theoretic definition of \emph{homotopy $n$-types}, corresponding to spaces with no nontrivial homotopy information above dimension $n$.
(The $0$-types are the ``sets'' mentioned previously as satisfying Lawvere's axioms\index{Lawvere}.)
Moreover, with higher inductive types, we can universally ``truncate'' a type into an $n$-type; in classical homotopy theory this would be its $n^{\mathrm{th}}$ Postnikov\index{Postnikov tower} section.\index{n-type@$n$-type}
Particularly important for logic is the case of homotopy $(-1)$-types, which we call \emph{mere propositions}.
Classically, every $(-1)$-type is empty or contractible; we interpret these possibilities as the truth values ``false'' and ``true'' respectively.

Using all types as propositions yields a very ``constructive'' conception of logic; for more on this, see~\cite{kolmogorov,TroelstraI,TroelstraII}.
For instance, every proof that something exists carries with it enough information to actually find such an object; and every proof that ``$A$ or $B$'' holds is either a proof that $A$ holds or a proof that $B$ holds.
Thus, from every proof we can automatically extract an algorithm;\index{algorithm} \index{extraction of algorithms} this can be very useful in applications to computer programming.

On the other hand, however, this logic does diverge from the traditional understanding of existence proofs in mathematics.
In particular, it does not faithfully represent certain important classical principles of reasoning, such as the axiom of choice and the law of excluded middle.
For these we need to use the ``$(-1)$-truncated'' logic, in which only the homotopy $(-1)$-types represent propositions.

\index{axiom!of choice}%
More specifically, consider on one hand the \emph{axiom of choice}: ``if for every $x: A$ there exists a $y:B$ such that $R(x,y)$, there is a function $f : A\to B$ such that for all $x:A$ we have $R(x, f(x))$.''
The pure propositions-as-types notion of ``there exists'' is strong enough to make this statement simply provable --- yet it does not have all the consequences of the usual axiom of choice.
However, in $(-1)$-truncated logic, this statement is not automatically true, but is a strong assumption with the same sorts of consequences as its counterpart in classical\index{mathematics!classical} set theory.

\index{excluded middle}%
\index{univalence axiom}%
On the other hand, consider the \emph{law of excluded middle}: ``for all $A$, either $A$ or not $A$.''
Interpreting this in the pure propositions-as-types logic yields a statement that is inconsistent with the univalence axiom.
For since proving ``$A$'' means exhibiting an element of it, this assumption would give a uniform way of selecting an element from every nonempty type --- a sort of Hilbertian choice operator.
Univalence implies that the element of $A$ selected by such a choice operator must be invariant under all self-equivalences of $A$, since these are identified with self-identities and every operation must respect identity; but clearly some types have automorphisms with no fixed points, e.g.\ we can swap the elements of a two-element type.
\index{automorphism!fixed-point-free}%
However, the ``$(-1)$-truncated law of excluded middle'', though also not automatically true, may consistently be assumed with most of the same consequences as in classical mathematics.

In other words, while the pure propositions-as-types logic is ``constructive'' in the strong algorithmic sense mentioned above, the default $(-1)$-truncated logic is ``constructive'' in a different sense (namely, that of the logic formalized by Heyting under the name ``intuitionistic''); and to the latter we may freely add the axioms of choice and excluded middle to obtain a logic that may be called ``classical''.
Thus, homotopy type theory is compatible with both constructive and classical conceptions of logic, and many more besides.
\index{logic!constructive vs classical}%
Indeed, the homotopical perspective reveals that classical and constructive logic can coexist, as endpoints of a spectrum of different systems, with an infinite number of possibilities in between (the homotopy $n$-types for $-1 < n < \infty$).
We may speak of ``\LEM{n}'' and ``\choice{n}'', with $\choice{\infty}$ being provable and \LEM{\infty} inconsistent with univalence, while $\choice{-1}$ and $\LEM{-1}$ are the versions familiar to classical mathematicians (hence in most cases it is appropriate to assume the subscript $(-1)$ when none is given).  Indeed, one can even have useful systems in which only \emph{certain} types satisfy such further ``classical'' principles, while types in general remain ``constructive''.\index{excluded middle}\index{axiom!of choice}%%

It is worth emphasizing that univalent foundations does not \emph{require} the use of constructive or intuitionistic logic.\index{logic!intuitionistic}\index{logic!constructive} %
Most of classical mathematics which depends on the law of excluded middle and the axiom of choice can be performed in univalent foundations, simply by assuming that these two principles hold (in their proper, $(-1)$-truncated, form).
However, type theory does encourage avoiding these principles when they are unnecessary, for several reasons.

First of all, every mathematician knows that a theorem is more powerful when proven using fewer assumptions, since it applies to more examples.
The situation with \choice{} and \LEM{} is no different:
type theory admits many interesting ``nonstandard'' models, such as in sheaf toposes,\index{topos} where classicality principles such as \choice{} and \LEM{} tend to fail.
Homotopy type theory admits similar models in higher toposes, such as are studied in~\cite{ToenVezzosi02,Rezk05,lurie:higher-topoi}.
Thus, if we avoid using these principles, the theorems we prove will be valid internally to all such models.

Secondly, one of the additional virtues of type theory is its computable character.
In addition to being a foundation for mathematics, type theory is a formal theory of computation, and can be treated as a powerful programming language.
\index{programming}%
From this perspective, the rules of the system cannot be chosen arbitrarily the way set-theoretic axioms can: there must be a harmony between them which allows all proofs to be ``executed'' as programs.
We do not yet fully understand the new principles introduced by homotopy type theory, such as univalence and higher inductive types, from
this point of view, but the basic outlines are emerging; see, for example,~\cite{lh:canonicity}.
It has been known for a long time, however, that principles such as \choice{} and \LEM{} are fundamentally antithetical to computability, since they assert baldly that certain things exist without giving any way to compute them.
Thus, avoiding them is necessary to maintain the character of type theory as a theory of computation.

Fortunately, constructive reasoning is not as hard as it may seem.
In some cases, simply by rephrasing some definitions, a theorem can be made constructive and its proof more elegant.
Moreover, in univalent foundations this seems to happen more often.
For instance:
\begin{enumerate}
\item In set-theoretic foundations, at various points in homotopy theory and category theory one needs the axiom of choice to perform transfinite constructions.
  But with higher inductive types, we can encode these constructions directly and constructively.
  In particular, none of the ``synthetic'' homotopy theory in \cref{cha:homotopy} requires \LEM{} or \choice{}.
\item In set-theoretic foundations, the statement ``every fully faithful and essentially surjective functor is an equivalence of categories'' is equiv\-a\-lent to the axiom of choice.
  But with the univalence axiom, it is just \emph{true}; see \cref{cha:category-theory}.
\item In set theory, various circumlocutions are required to obtain notions of ``cardinal number'' and ``ordinal number'' which canonically represent isomorphism classes of sets and well-ordered sets, respectively --- possibly involving the axiom of choice or the axiom of foundation.
  But with univalence and higher inductive types, we can obtain such representatives directly by truncating the universe; see \cref{cha:set-math}.
\item In set-theoretic foundations, the definition of the real numbers as equivalence classes of Cauchy sequences requires either the law of excluded middle or the axiom of (countable) choice to be well-behaved.
  But with higher inductive types, we can give a version of this definition which is well-behaved and avoids any choice principles; see \cref{cha:real-numbers}.
\end{enumerate}
Of course, these simplifications could as well be taken as evidence that the new methods will not, ultimately, prove to be really constructive.  However, we emphasize again that the reader does not have to care, or worry, about constructivity in order to read this book.  The point is that in all of the above examples, the version of the theory we give has independent advantages, whether or not \LEM{} and \choice{} are assumed to be available.  Constructivity, if attained, will be an added bonus.\index{constructivity}%

Given this discussion of adding new principles such as univalence, higher inductive types, \choice{}, and \LEM{}, one may wonder whether the resulting system remains consistent.
(One of the original virtues of type theory, relative to set theory, was that it can be seen to be consistent by proof-theoretic means).
As with any foundational system, consistency\index{consistency} is a relative question: ``consistent with respect to what?''
The short answer is that all of the constructions and axioms considered in this book have a model in the category of Kan\index{Kan complex} complexes, due to Voevodsky~\cite{klv:ssetmodel} (see~\cite{ls:hits} for higher inductive types).
Thus, they are known to be consistent relative to ZFC (with as many inaccessible cardinals
\index{inaccessible cardinal}\index{consistency}%
as we need nested univalent universes).
Giving a more traditionally type-theoretic account of this consistency is work in progress (see,
e.g.,~\cite{lh:canonicity,coquand2012constructive}).

We summarize the different points of view of the type-theoretic operations in \cref{tab:pov}.

\begin{table}[htb]
  \centering
  \OPTsmalltable
 \begin{tabular}{lllll}
    \toprule
       Types && Logic & Sets & Homotopy\\ \addlinespace[2pt]
    \midrule
       $A$ && proposition & set & space\\ \addlinespace[2pt]
       $a:A$ && proof & element & point \\ \addlinespace[2pt]
       $B(x)$ && predicate & family of sets & fibration \\ \addlinespace[2pt]
       $b(x) : B(x)$ && conditional proof & family of elements & section\\ \addlinespace[2pt]
       $\emptyt, \unit$ && $\bot, \top$ & $\emptyset, \{ \emptyset \}$ & $\emptyset, *$\\ \addlinespace[2pt]
       $A + B$ && $A\vee B$ & disjoint union & coproduct\\ \addlinespace[2pt]
       $A\times B$ && $A\wedge B$ & set of pairs & product space\\ \addlinespace[2pt]
       $A\to B$ && $A\Rightarrow B$ & set of functions & function space\\ \addlinespace[2pt]
       $\sm{x:A}B(x)$ &&  $\exists_{x:A}B(x)$ & disjoint sum & total space\\ \addlinespace[2pt]
       $\prd{x:A}B(x)$ &&  $\forall_{x:A}B(x)$ & product & space of sections\\ \addlinespace[2pt]
       $\mathsf{Id}_{A}$ && equality $=$ & $\setof{\pairr{x,x} | x\in A}$ & path space $A^I$ \\ \addlinespace[2pt]
    \bottomrule
  \end{tabular}
  \caption{Comparing points of view on type-theoretic operations}\label{tab:pov}
\end{table}

\index{mathematics!constructive|)}%

\subsection*{Open problems} 

\index{open!problem|(}%

For those interested in contributing to this new branch of mathematics, it may be encouraging to know that there are many interesting open questions.

\index{univalence axiom!constructivity of}%
Perhaps the most pressing of them is the ``constructivity'' of the Univalence Axiom, posed by Voevodsky in \cite{Universe-poly}.
The basic system of type theory follows the structure of Gentzen's natural deduction. Logical connectives are defined by their introduction rules, and have elimination rules justified by computation rules. Following this pattern, and using Tait's computability method, originally designed to analyse G\"odel's Dialectica interpretation, one can show the property of \emph{normalization} for type theory. This in turn implies important properties such as decidability of type-checking (a crucial property since type-checking corresponds to proof-checking, and one can argue that we should be able to ``recognize a proof when we see one''), and the so-called ``canonicity\index{canonicity} property'' that any closed term of the type of natural numbers reduces to a numeral. This last property, and the uniform structure of introduction/elimination rules, are lost when one extends type theory with an axiom, such as the axiom of function extensionality, or the univalence axiom. Voevodsky has formulated a precise mathematical conjecture connected to this question of canonicity for type theory extended with the axiom of Univalence: given a closed term of the type of natural numbers, is it always possible to find a numeral and a proof that this term is equal to this numeral, where this proof of equality may itself use the univalence axiom? More generally, an important issue is whether it is possible to provide a constructive justification of the univalence axiom.
What about if one adds other homotopically motivated constructions, like higher inductive types?
These questions remain open at the present time, although methods are currently being developed to try to find answers.

Another basic issue is the difficulty of working with types, such as the natural numbers, that are essentially sets (i.e., discrete spaces),
\index{discrete!space}%
containing only trivial paths.
At present, homotopy type theory can really only characterize spaces up to homotopy equivalence, which means that these ``discrete spaces'' may only be \emph{homotopy equivalent} to discrete spaces.
Type-theoretically, this means there are many paths that are equal to reflexivity, but not \emph{judgmentally} equal to it (see \cref{sec:types-vs-sets} for the meaning of ``judgmentally'').
While this homotopy-invariance has advantages, these ``meaningless'' identity terms do introduce needless complications into arguments and constructions, so it would be convenient to have a systematic way of eliminating or collapsing them.
% In some cases, the proliferation of such superfluous identity terms makes it very difficult or impossible to formulate what should be a straightforward concept, such as the definition of a (semi-)simplicial type.

A more specialized, but no less important, problem is the relation between homotopy type theory and the research on \emph{higher toposes}%
\index{.infinity1-topos@$(\infty,1)$-topos}
currently happening at the intersection of higher category theory and homotopy theory.
There is a growing conviction among those familiar with both subjects that they are intimately connected.
For instance, the notion of a univalent universe should coincide with that of an object classifier, while higher inductive types should be an ``elementary'' reflection of local presentability.
More generally, homotopy type theory should be the ``internal language'' of $(\infty,1)$-toposes, just as intuitionistic higher-order logic is the internal language of ordinary 1-toposes.
Despite this general consensus, however, details remain to be worked out --- in particular, questions of coherence and strictness remain to be addressed  --- and doing so will undoubtedly lead to further insights into both concepts.

\index{mathematics!formalized}%
But by far the largest field of work to be done is in the ongoing formalization of everyday mathematics in this new system.
Recent successes in formalizing some facts from basic homotopy theory and category theory have been encouraging; some of these are described in \cref{cha:homotopy,cha:category-theory}.
Obviously, however, much work remains to be done.

\index{open!problem|)}%

The homotopy type theory community maintains a web site and group blog at \url{http://homotopytypetheory.org}, as well as a discussion email list.
Newcomers are always welcome!


\subsection*{How to read this book}

This book is divided into two parts.
\cref{part:foundations}, ``Foundations'', develops the fundamental concepts of homotopy type theory.
This is the mathematical foundation on which the development of specific subjects is built, and which is required for the understanding of the univalent foundations approach. To a programmer, this is ``library code''.
Since univalent foundations is a new and different kind of mathematics, its basic notions take some getting used to; thus \cref{part:foundations} is fairly extensive.

\cref{part:mathematics}, ``Mathematics'', consists of four chapters that build on the basic notions of \cref{part:foundations} to exhibit some of the new things we can do with univalent foundations in four different areas of mathematics: homotopy theory (\cref{cha:homotopy}), category theory (\cref{cha:category-theory}), set theory (\cref{cha:set-math}), and real analysis (\cref{cha:real-numbers}).
The chapters in \cref{part:mathematics} are more or less independent of each other, although occasionally one will use a lemma proven in another.

A reader who wants to seriously understand univalent foundations, and be able to work in it, will eventually have to read and understand most of \cref{part:foundations}.
However, a reader who just wants to get a taste of univalent foundations and what it can do may understandably balk at having to work through over 200 pages before getting to the ``meat'' in \cref{part:mathematics}.
Fortunately, not all of \cref{part:foundations} is necessary in order to read the chapters in \cref{part:mathematics}.
Each chapter in \cref{part:mathematics} begins with a brief overview of its subject, what univalent foundations has to contribute to it, and the necessary background from \cref{part:foundations}, so the courageous reader can turn immediately to the appropriate chapter for their favorite subject.
For those who want to understand one or more chapters in \cref{part:mathematics} more deeply than this, but are not ready to read all of \cref{part:foundations}, we provide here a brief summary of \cref{part:foundations}, with remarks about which parts are necessary for which chapters in \cref{part:mathematics}.

\cref{cha:typetheory} is about the basic notions of type theory, prior to any homotopical interpretation.
A reader who is familiar with Martin-L\"of type theory can quickly skim it to pick up the particulars of the theory we are using.
However, readers without experience in type theory will need to read \cref{cha:typetheory}, as there are many subtle differences between type theory and other foundations such as set theory.

\cref{cha:basics} introduces the homotopical viewpoint on type theory, along with the basic notions supporting this view, and describes the homotopical behavior of each component of the type theory from \cref{cha:typetheory}.
It also introduces the \emph{univalence axiom} (\cref{sec:compute-universe}) --- the first of the two basic innovations of homotopy type theory.
Thus, it is quite basic and we encourage everyone to read it, especially \crefrange{sec:equality}{sec:basics-equivalences}.

\cref{cha:logic} describes how we represent logic in homotopy type theory, and its connection to classical logic as well as to constructive and intuitionistic logic.
Here we define the law of excluded middle, the axiom of choice, and the axiom of propositional resizing (although, for the most part, we do not need to assume any of these in the rest of the book), as well as the \emph{propositional truncation} which is essential for representing traditional logic.
This chapter is essential background for \cref{cha:set-math,cha:real-numbers}, less important for \cref{cha:category-theory}, and not so necessary for \cref{cha:homotopy}.

\cref{cha:equivalences,cha:induction} study two special topics in detail: equivalences (and related notions) and generalized inductive definitions.
While these are important subjects in their own rights and provide a deeper understanding of homotopy type theory, for the most part they are not necessary for \cref{part:mathematics}.
Only a few lemmas from \cref{cha:equivalences} are used here and there, while the general discussions in \cref{sec:bool-nat,sec:strictly-positive,sec:generalizations} are helpful for providing the intuition required for \cref{cha:hits}.
The generalized sorts of inductive definition discussed in \cref{sec:generalizations} are also used in a few places in \cref{cha:set-math,cha:real-numbers}.

\cref{cha:hits} introduces the second basic innovation of homotopy type theory --- \emph{higher inductive types} --- with many examples.
Higher inductive types are the primary object of study in \cref{cha:homotopy}, and some particular ones play important roles in \cref{cha:set-math,cha:real-numbers}.
They are not so necessary for \cref{cha:category-theory}, although one example is used in \cref{sec:rezk}.

Finally, \cref{cha:hlevels} discusses homotopy $n$-types and related notions such as $n$-connected types.
These notions are important for \cref{cha:homotopy}, but not so important in the rest of \cref{part:mathematics}, although the case $n=-1$ of some of the lemmas are used in \cref{sec:piw-pretopos}.

This completes \cref{part:foundations}.
As mentioned above, \cref{part:mathematics} consists of four largely unrelated chapters, each describing what univalent foundations has to offer to a particular subject.

Of the chapters in \cref{part:mathematics}, \cref{cha:homotopy} (Homotopy theory) is perhaps the most radical.
Univalent foundations has a very different ``synthetic'' approach to homotopy theory in which homotopy types are the basic objects (namely, the types) rather than being constructed using topological spaces or some other set-theoretic model.
This enables new styles of proof for classical theorems in algebraic topology, of which we present a sampling, from $\pi_1(\Sn^1)=\Z$ to the Freudenthal suspension theorem.

In \cref{cha:category-theory} (Category theory), we develop some basic (1-)category theory, adhering to the principle of the univalence axiom that \emph{equality is isomorphism}.
This has the pleasant effect of ensuring that all definitions and constructions are automatically invariant under equivalence of categories: indeed, equivalent categories are equal just as equivalent types are equal.
(It also has connections to higher category theory and higher topos theory.)

\cref{cha:set-math} (Set theory) studies sets in univalent foundations.
The category of sets has its usual properties, hence provides a foundation for any mathematics that doesn't need homotopical or higher-categorical structures.
We also observe that univalence makes cardinal and ordinal numbers a bit more pleasant, and that higher inductive types yield a cumulative hierarchy satisfying the usual axioms of Zermelo--Fraenkel set theory.

In \cref{cha:real-numbers} (Real numbers), we summarize the construction of Dedekind real numbers, and then observe that higher inductive types allow a definition of Cauchy real numbers that avoids some associated problems in constructive mathematics.
Then we sketch a similar approach to Conway's surreal numbers.

Each chapter in this book ends with a Notes section, which collects historical comments, references to the literature, and attributions of results, to the extent possible.
We have also included Exercises at the end of each chapter, to assist the reader in gaining familiarity with doing mathematics in univalent foundations.

Finally, recall that this book was written as a massively collaborative effort by a large number of people.
We have done our best to achieve consistency in terminology and notation, and to put the mathematics in a linear sequence that flows logically, but it is very likely that some imperfections remain.
We ask the reader's forgiveness for any such infelicities, and welcome suggestions for improvement of the next edition.


% Local Variables:
% TeX-master: "hott-online"
% End:

\section{Syntax}
\label{sec:syntax}

\Eff is a statically typed language with parametric polymorphism and type
inference. Its types include products, sums, records, and recursive type
definitions. To keep to the point, we focus on a core language with
monomorphic types and type checking. The concrete syntax follows that of OCaml~\cite{OCaml}, and except for new constructs, we discuss it only briefly.

\subsection{Types}
\label{sec:types}

Apart from the standard types, \eff has \emph{effect types $E$} and \emph{handler
  types $A \hto B$}:
%
\begin{align*}
  \tag{type}
  A, B, C \bnfis {}
    &\intty \bnfor
    \boolty \bnfor
    \unitty \bnfor
    \emptyty \bnfor
    \\
    &A \times B \bnfor
    A + B \bnfor
    A \to B \bnfor
    E \bnfor
    A \hto B,\\
  \tag{effect type}
  E \bnfis {}
    &\kpre{effect} (\kpre{operation} \op_i \T A_i \to B_i)_i \kpost{end}.
\end{align*}
%
In the rule for effect types and elsewhere below $(\cdots)_i$ indicates that
$\cdots$ may be repeated a finite number of times. We include the empty type
as we need it to describe exceptions, see Section~\ref{sec:exceptions}.
%
An effect type describes a collection of related operation symbols, for example those for
writing to and reading from a communication channel. We write $\op \T A \to B \in E$ or
just $\op \in E$ to indicate that the effect type $E$ contains an operation $\op$ with parameters of
type $A$ and results of type $B$.
%
The handler type $A \hto B$ should be understood as the type of handlers acting on
computations of type~$A$ and yielding computations of type~$B$.

\subsection{Expressions and computations}

\Eff distinguishes between \emph{expressions} and \emph{computations}, which are
similar to values and producers of fine-grain call-by-value~\cite{levy03modelling}. The former are inert and free from computational
effects, including divergence, while the latter may diverge or cause
computational effects. As discussed in Section~\ref{sec:implementation}, the
concrete syntax of \eff hides the distinction and allows the programmer to
freely mix expressions and computations.

Beware that we use two kinds of vertical bars below: the tall~$\bnfor$ separates
grammatical alternatives, and the short~$\case$ separates cases in handlers and
match statements. The expressions are
%
\begin{align*}
  \tag{expression}
  e \bnfis {}
    &x \bnfor
    n \bnfor
    c \bnfor
    \tru \bnfor
    \fls \bnfor
    \unt \bnfor
    \pair{e_1, e_2} \bnfor \\
    &\Left{e} \bnfor \Right{e} \bnfor
    \fun{x \T A} c \bnfor
    \hash{e}{\op} \bnfor
    h, \\
  \tag{handler}
  h \bnfis {}
    &\handler
    (\hash{e_i}{\op_i} \, x \, k \mapsto c_i)_i \case
    \val x \mapsto c_v \case
    \fin x \mapsto c_f,
\end{align*}
%
where $x$ signifies a variable, $n$ an integer constant, and $c$ other built-in constants.
The expressions $\unt$, $\pair{e_1, e_2}$, $\Left{e}$, $\Right{e}$, and $\fun{x \T A} c$
are introduction forms for the unit, product, sum, and function types, respectively.
Operations $\hash{e}{\op}$ and handlers $h$ are discussed in
Section~\ref{sec:eff-specific}.

The computations are
%
\begin{align*}
  \tag{computation}
  c \bnfis {}
    &\val e \bnfor
    \letin{x = c_1} c_2 \bnfor
    \letrecin{f \, x = c_1} c_2 \bnfor \\
    &\ifthenelse{e}{c_1}{c_2} \bnfor
    \absurd{e} \bnfor
    \matchpair{e}{x, y}{c} \bnfor \\
    &\matchsum{e}{x}{c_1}{y}{c_2} \bnfor
    e_1 \, e_2 \bnfor \\
    &\new E \bnfor 
    \newwith{E}{e}{
      (\kpre{operation} \op_i \, x \, @ \, y \mapsto c_i)_i
    } \bnfor \\
    &\handle{c}{e}.
\end{align*}
%
An expression $e$ is promoted to a computation with $\val e$, but in the concrete syntax
$\kord{val}$ is omitted, as there is no distinction between expressions and computations.
%
The statement $\letin{x = c_1}{c_2}$ binds $x$ in $c_2$, and $\letrecin{f \, x =
  c_1}{c_2}$ defines a recursive function $f$ in $c_2$. The conditional statement and the
variations of $\kord{match}$ are elimination forms for booleans, the empty type, products,
and sums, respectively. Instance creation and the handling construct are discussed in Section~\ref{sec:eff-specific}.

Arithmetical expressions such as $e_1 + e_2$ count as computations because the arithmetical
operators are defined as built-in constants, so that $e_1 + e_2$ is parsed as a double
application. This allows us to uniformly treat all operations, irrespective
of whether they are pure or effectful (division by zero).

\section{Constructs specific to \eff}
\label{sec:eff-specific}

We explain the intuitive meaning of notions that are specific to \eff, namely
instances, operations, handlers, and resources.
We allow ourselves some slack in distinguishing syntax from semantics, which is treated in detail in Section~\ref{sec:semantics}.
It is helpful to think of a terminating computation as evaluating either to a value or an operation applied to a parameter.

\subsection{Instances and operations}
\label{sec:instances-operations}

The computation $\new E$ generates a fresh \emph{effect instance} of effect type~$E$.
For example, $\new \effect{ref}$ generates a new reference, $\new \effect{channel}$ a new communication channel, etc.
%
The extended form of $\kord{new}$ creates an effect instance with an associated \emph{resource},
which determines the default behaviour of operations and is explained separately in
Section~\ref{sec:resources}.

For each effect instance $e$ of effect type $E$ and an operation symbol $\op \in E$ there
is an \emph{operation} $\hash{e}{\op}$, also known as a \emph{generic
  effect}~\cite{plotkin03algebraic}. By itself, an operation is a value, and hence
effect-free, but an applied operation $\hash{e}{\op}\,e'$ is a computational effect
whose ramifications are determined by enveloping handlers and the resource associated with~$e$.

\subsection{Handlers}
\label{sec:handlers}

A handler
%
\begin{equation*}
 h =
 \handler
 (\hash{e_i}{\op_i} \, x \, k \mapsto c_i)_i \case \val x \mapsto c_v \case \fin x \mapsto c_f
\end{equation*}
%
may be applied to a computation $c$ with the handling construct
%
\begin{equation}
  \label{eq:handling}
   \handle{c}{h},
\end{equation}
%
which works as follows (we ignore the $\fin$ clause for the moment):
%
\begin{enumerate}
\item If $c$ evaluates to $\val e$, \eqref{eq:handling} evaluates to $c_v$ with $x$ bound to $e$.
\item If the evaluation of $c$ encounters an operation $\hash{e_i}{\op_i} \, e$,
  \eqref{eq:handling} evaluates to $c_i$ with $x$ bound to $e$ and $k$ bound to the
  continuation of $\hash{e_i}{\op_i} \, e$, i.e., whatever remains to be computed after the
  operation. The continuation is delimited by~\eqref{eq:handling} and is handled by~$h$ as
  well.
\end{enumerate}
%
The $\kord{finally}$ clause can be thought of as an outer wrapper which performs an
additional transformation, so that \eqref{eq:handling} is equivalent to
%
\begin{equation*}
  \letin{x = (\handle{c}{h'})}{c_f}  
\end{equation*}
%
where $h'$ is like $h$ without the $\kord{finally}$ clause. Such a wrapper is useful
because we often perform the same transformation every time a given handler is applied.
For example, the handler for state handles a computation by transforming it to a function
accepting the state, and $\kord{finally}$ applies the function to the initial state, see
Section~\ref{sec:state}.

If the evaluation of $c$ encounters an operation $\hash{e}{\op} \, e'$ that is not listed
in $h$, the control propagates to outer handling constructs, and eventually to the
toplevel, where the behaviour is determined by the resource associated with $e$.

\subsection{Resources}
\label{sec:resources}

The construct
%
\begin{equation*}
  \newwith{E}{e}{(\kpre{operation} \op_i \, x \, @ \, y \mapsto c_i)_i}
\end{equation*}
%
creates an instance $n$ with an associated \emph{resource}, inspired by coalgebraic
semantics of computational
effects~\cite{power04from,plotkin08:_tensor_comod_model_operat_seman}. A resource carries
a state and prescribes the default behaviour of the operations $\hash{n}{\op_i}$.
%
The paradigmatic case of resources is the definition of ML-style references, see Section~\ref{sec:state}.

The initial resource state for $n$ is set to $e$.
When the toplevel evaluation encounters an operation
$\hash{n}{\op_i} \, e'$, it evaluates $c_i$ with $x$ bound to $e'$
and $y$ bound to the current resource state. The result must be a pair of values, the first of
which is passed to the continuation, and the second of which is the new resource state.
%
If $c_i$ evaluates to an operation rather than a pair of values, a runtime error is
reported, as there is no reasonable way of handling it.

In \eff the interaction with the real world is accomplished through built-in resources.
For example, there is a predefined channel instance $\kord{std}$ with operations
$\hash{\kord{std}}{\kord{read}}$ and $\hash{\kord{std}}{\kord{write}}$ whose associated
resource performs actual interaction with the standard input and the standard output.

\section{Type checking}
\label{sec:type-checking}

Types in \eff are like those of ML~\cite{milner97the-definition} in the sense that they do not capture any
information about computational effects.
%
There are two typing judgements, $\ctx \ente e \T A$ states that expression $e$ has type
$A$ in context $\ctx$, and $\ctx \entc c \T A$ does so for a computation~$c$.
As usual, a context $\Gamma$ is a list of
distinct variables with associated types. The standard typing rules for expressions are:
%%
\begin{mathpar}
  \infer
    {x \T A \in \ctx}
    {\ctx \ente x \T A}

  \infer
    {}
    {\ctx \ente n \T \intty}

  \infer
    {}
    {\ctx \ente \tru \T \boolty}

  \infer
    {}
    {\ctx \ente \fls \T \boolty}

  \infer
    {}
    {\ctx \ente \unt \T \unitty}

  \infer
    {\Gamma \ente e_1 : A \\
     \Gamma \ente e_2 : B}
    {\ctx \ente \pair{e_1, e_2} \T A \times B}

  \infer
    {\Gamma \ente e : A}
    {\Gamma \ente \Left{e} : A + B}

  \infer
    {\Gamma \ente e : B}
    {\Gamma \ente \Right{e} : A + B}

  \infer
    {\ctx, x \T A \entc c \T B}
    {\ctx \ente \fun{x \T A} c \T A \to B}
\end{mathpar}
%
We also have to include judgements that assign types to other built-in constants.
An operation receives a function type
%
\begin{mathpar}
  \infer
  {\ctx \ente e \T E \\
    \op \T A \to B \in E}
  {\ctx \ente \hash{e}{\op} \T A \to B}
\end{mathpar}
%
while a handler is typed by the somewhat complicated rule
%
\begin{equation*}
  \infer
  {\infer{\ctx \ente e_i \T E_i \\
    \op_i \T A_i \to B_i \in E_i \\\\
    \ctx, x \T A_i, k \T B_i \to B \entc c_i \T B}{} \\
     \ctx, x \T A \entc c_v \T B \\
     \ctx, x \T B \entc c_f \T C}
   {\ctx \ente (\handler
     (\hash{e_i}{\op_i} \, x \, k \mapsto c_i)_i \mid
     \val x \mapsto c_v \mid
     \fin x \mapsto c_f) \T A \hto C}
\end{equation*}
%
which states that a handler first handles a computation of type $A$ into
a computation of type $B$ according to the $\kord{val}$ and operation clauses,
after which the $\kord{finally}$ clause transforms it further into a computation of type $C$.

The typing rules for computations are familiar or expected. Promotions of expressions and $\kord{let}$ statements are typed by
%
\begin{mathpar}
  \infer
    {\ctx \ente e \T A}
    {\ctx \entc \val e \T A}

  \infer
    {\ctx \entc c_1 \T A \\
     \ctx, x \T A \entc c_2 \T B}
    {\ctx \entc \letin{x = c_1} c_2 \T B}

  \infer
    {\ctx, f \T A \to B, x \T A \entc c_1 \T B \\
     \ctx, f \T A \to B \entc c_2 \T C}
    {\ctx \entc \letrecin{f\,x = c_1} c_2 \T C}
\end{mathpar}
%
and various elimination forms are typed by
%
\begin{mathpar}
  \infer
    {\ctx \ente e \T \boolty \\
     \ctx \entc c_1 \T A \\
     \ctx \entc c_2 \T A}
    {\ctx \entc \ifthenelse{e}{c_1}{c_2} \T A}

  \infer
    {\ctx \ente e \T \emptyty}
    {\ctx \entc \absurd{e} \T A}
      
  \infer
    {\ctx \ente e \T A \times B \\
     \ctx, x \T A, y \T B \entc c \T C}
    {\ctx \entc \matchpair{e}{x,y}{c} \T C}

  \infer
    {\ctx \ente e \T A + B \\
     \ctx, x \T A \entc c_1 \T C \\
     \ctx, y \T B \entc c_2 \T C}
    {\ctx \entc \matchsum{e}{x}{c_1}{y}{c_2} \T C}

  \infer
    {\ctx \ente e_1 \T A \to B \\
     \ctx \ente e_2 \T A}
    {\ctx \entc e_1 \, e_2 \T B}
\end{mathpar}
%
The instance creation is typed by the rules
%
\begin{mathpar}
  \infer
    {}
    {\ctx \entc \new E \T E}

  \infer
    {\ctx \ente e : C \\
     \op_i \T A_i \to B_i \in E \\
     \ctx, x \T A_i, y : C \entc c_i : B_i \times C}
    {\ctx \entc \newwith{E}{e}{(\kpre{operation} \op_i \, x \, @ \, y \mapsto c_i)_i} \T E}
\end{mathpar}
%
The rule for the simple form is obvious, while the one for the extended form checks that
the initial state $e$ has type $C$ and that, for each operation $\op_i \T A_i \to B_i \in
E$, the corresponding computation $c_i$ evaluates to a pair of type $B_i \times C$.

Finally, the rule for handling expresses the fact that handlers are like functions:
%
\begin{mathpar}
    \infer
    {\ctx \entc c \T A \\
     \ctx \ente e \T A \hto B}
    {\ctx \entc \handle{c}{e} \T B}
\end{mathpar}


%%% Local Variables:
%%% mode: latex
%%% TeX-master: "eff"
%%% End:

\section{Denotational semantics}
\label{sec:semantics}

Our aim is to describe a denotational semantics which explains how programs in \eff are evaluated.
%
Since the implemented runtime has no type information, we give Curry-style semantics
in which terms are interpreted without being typed.
%
See the exposition by John Reynolds~\cite{reynolds00themeaning} on how such semantics can
be related to Church-style semantics in which types and typing judgements receive meanings.

We give interpretations of expressions and computations in domains of \emph{values} $V$
and \emph{results} $R$, respectively. We follow Reynolds by avoiding a particular choice
of $V$ and $R$, and instead require properties of $V$ and $R$ that ensure the semantics
works out. The requirements can be met in a number of ways, for example by solving
suitable domain equations or by taking $V$ and $R$ to be sufficiently large universal
domains.

The domain $V$ has to contain integers, booleans, functions, etc.
In particular, we require that $V$ contains the following retracts, where $\II$ is a 
set of effect instances, and $\oplus$ is coalesced sum:
%
\begin{align*}
  &\retract{\ZZ_\bot}{V}{\iota_{\intty}}{\rho_{\intty}}
  &
  &\retract{\set{0, 1}_\bot}{V}{\iota_{\boolty}}{\rho_{\boolty}}
  &
  &\retract{\set{\star}_\bot}{V}{\iota_{\unitty}}{\rho_{\unitty}}
  \\
  &\retract{\II_\bot}{V}{\iota_{\kord{effect}}}{\rho_{\kord{effect}}}
  &
  &\retract{V \times V}{V}{\iota_{\times}}{\rho_{\times}}
  &
  &\retract{V \oplus V}{V}{\iota_{+}}{\rho_{+}}
  \\
  &\retract{R^V}{V}{\iota_{\to}}{\rho_{\to}}
  &
  &\retract{R^R}{V}{\iota_{\hto}}{\rho_{\hto}}
\end{align*}
%
As expressions are terminating, the bottom element of $V$ is never used to denote
divergence, but we do use it to indicate ill-formed values and runtime errors.

\newcommand{\operationDomain}{\II \times \OO \times V \times R^V}

The domain
%
\begin{equation}
  \label{eq:resultDomain}
  (V + \operationDomain)_\bot
\end{equation}
%
embodies the idea that a terminating computation is either a value or an operation applied to a parameter and a continuation. There are canonical retractions from~\eqref{eq:resultDomain} onto the two summands,
%
\begin{equation}
  \label{eq:resultDomain-retraction}
  \xymatrix{
     *!R{V\;} \ar@<0.25em>[rr]^(0.3){\iota_{\kord{val}}}
     & &
     {(V + \operationDomain)_\bot}
     \ar@<0.25em>[ll]^(0.7){\rho_{\kord{val}}}
     \ar@<-0.25em>[r]_(0.7){\rho_{\kord{oper}}}
     &
     *!L{\;(\operationDomain)_\bot}
     \ar@<-0.25em>[l]_(0.3){\iota_\kord{oper}}
  }
\end{equation}
%
A typical element of~\eqref{eq:resultDomain} is either $\bot$, of the form $\iota_{\kord{val}}(v)$ for a unique $v \in V$, or of the form $\iota_{\kord{oper}}(n, \op, v, \kappa)$ for unique $n \in \II$, $\op \in \OO$, $v \in V$, and $\kappa \in R^V$. We require that $R$ contains~\eqref{eq:resultDomain} as a retract:
%
\begin{equation}
  \label{eq:resultDomain-in-R}
  \retract{(V + \operationDomain)_\bot}{R}{\iota_{\kord{res}}}{\rho_{\kord{res}}}.
\end{equation}
%
We may define a strict map from~\eqref{eq:resultDomain} by cases, with one case specifying how to map $\iota_{\kord{val}}(v)$ and the other how to map $\iota_{\kord{oper}}(n, \op, v, \kappa)$. For example, given a map $f : V \to R$, there is a unique strict map $\lift{f} : (V + \operationDomain)_\bot \to R$, called the \emph{lifting} of~$f$, which depends on $f$ continuously and satisfies the recursive equations
%
\begin{align*}
  \lift{f}(\iota_{\kord{val}}(v)) &= f(v),
  \\
  \lift{f}(\iota_{\kord{oper}}(n,\op,v,\kappa)) &=
  \iota_{\kord{oper}}(n, \op, v, \lift{f} \circ \rho_{\kord{res}} \circ \kappa).
\end{align*}

An \emph{environment $\eta$} is a map from variable names to values. We denote by
$\extend{\eta}{x}{v}$ the environment which assigns $v$ to $x$ and otherwise behaves as
$\eta$. An expression is interpreted as a map from environments to values. The standard
cases are as follows:
%
\begin{align*}
  \sem{x}{\eta} &= \eta(x)
  \\
  \sem{n}{\eta} &= \iota_\intty(\overline{n})
  \\
  \sem{\fls}{\eta} &= \iota_\boolty(0)
  \\
  \sem{\tru}{\eta} &= \iota_\boolty(1)
  \\
  \sem{\unt}{\eta} &= \iota_\unitty(\star)
  \\
  \sem{\pair{e_1, e_2}}{\eta} &= \iota_{\times}(\sem{e_1}{\eta}, \sem{e_2}{\eta})
  \\
  \sem{\Left e}{\eta} &= \iota_{+}(\iota_0(\sem{e}{\eta}))
  \\
  \sem{\Right e}{\eta} &= \iota_{+}(\iota_1(\sem{e}{\eta}))
  \\
  \sem{\fun{x \T A} c}{\eta} &= \iota_{\to}(\lambda v : V \,.\, \sem{c}{\extend{\eta}{x}{v}})  
\end{align*}
%
Of course, we need to provide the semantics of other built-in constants, too.
The interpretation of $\hash{e}{\op}$ make sense only when $e$ evaluates to an instance, so we define
% 
\begin{equation*}
  \sem{\hash{e}{\op}}{\eta} = 
  \begin{cases}
    \iota_\to(\lambda v : V \,.\,
       \iota_{\kord{res}}(\iota_{\kord{oper}}(n, \op, v, \iota_\kord{res} \circ \iota_\kord{val}))) &
    \text{if $\rho_{\kord{effect}}(\sem{e}{\eta}) = n \in \II$,}\\
    \iota_\to(\lambda v : V \,.\, \bot) &
    \text{if $\rho_{\kord{effect}}(\sem{e}{\eta}) = \bot$.}
  \end{cases}
\end{equation*}
%
The interpretation of a handler is
%
\begin{multline*}
  \sem{
    \handler
    (\hash{e_i}{\op_i} \, x \, k \mapsto c_i)_i \case
    \val x \mapsto c_v \case
    \fin x \mapsto c_f}{\eta} = \\ 
  \iota_\hto(\lift{f} \circ \rho_{\kord{res}} \circ  h \circ \rho_{\kord{res}})
\end{multline*}
%
where $f : V \to R$ is $f(v) = \sem{c_f}{\extend{\eta}{x}{v}}$ and $h : (V + \operationDomain)_\bot \to R$ is characterized as follows:
%
if one of the $\rho_{\kord{effect}}(\sem{e_i}{\eta})$ is $\bot$ we set $h = \lambda x
\,.\, \bot$, otherwise $\rho_{\kord{effect}}(\sem{e_i}{\eta}) = n_i \in \II$ for all $i$
and then we take the $h$ defined by cases as
%
\begin{align*}
  h(\iota_\kord{val}(v)) &= \sem{c_v}{\extend{\eta}{x}{v}}
  \\
  h(\iota_{\kord{oper}}(n_i, \op_i, v, \kappa)) &=
  \sem{c_i}{\extends{\eta}{x \sto v, k \sto \kappa}} \qquad \text{for all $i$,}
  \\
  h(\iota_{\kord{oper}}(n, \op, v, \kappa)) &=
  \begin{aligned}[t]
  &\iota_{\kord{res}}(\iota_{\kord{oper}}(n, \op, v, h \circ \rho_{\kord{res}} \circ \kappa))\\
  &\qquad \text{if $(n, \op) \neq (n_i, \op_i)$ for all $i$.}
  \end{aligned}
\end{align*}

We proceed to the meaning of computations, which are interpreted as maps from
environments to results. Promotion of expressions is interpreted in the obvious way as
%
\begin{equation*}
  \sem{\val e}{\eta} = \iota_{\kord{res}}(\iota_\kord{val}(\sem{e}{\eta}))
\end{equation*}
%
The $\kord{let}$ statement corresponds to monadic-style binding:
%
\begin{equation*}
  \sem{\letin{x = c_1}{c_2}}{\eta} =
  \lift{(\lambda v : V \,.\, \sem{c_2}{\extend{\eta}{x}{v}})}
  (\rho_{\kord{res}}(\sem{c_1}{\eta})),
\end{equation*}
%
A recursive function definition is interpreted as
%
\begin{equation*}
  \sem{\letrecin{f \, x = c_1} c_2}{\eta} = \sem{c_2}{\extend{\eta}{f}{\iota_\to(t)}}
\end{equation*}
%
where $t : V \to R$ is the least fixed point of the map
%
\begin{equation*}
  t \mapsto (\lambda v : V \,.\, \sem{c_1}{\extends{\eta}{f \sto \iota_\to(t), x \sto v}}).
\end{equation*}
%
The elimination forms are interpreted in the usual way as:
\begin{align*}
  \sem{\ifthenelse{e}{c_1}{c_2}}{\eta} &=
  \begin{cases}
    \sem{c_1}{\eta} & \text{if $\rho_{\boolty}{\sem{e}{\eta}} = 1$} \\
    \sem{c_2}{\eta} & \text{if $\rho_{\boolty}{\sem{e}{\eta}} = 0$} \\
    \bot & \text{otherwise}
  \end{cases}
  \\
  \sem{\absurd e}{\eta} &= \bot
  \\
  \sem{\matchpair{e}{x,y}{c}}{\eta} &=
  \begin{aligned}[t]
  &\sem{c}{\extends{\eta}{x \sto v_0, y \sto v_1}} \\
  &\qquad \text{where $(v_0, v_1) = \rho_\times (\sem{e}{\eta})$}
  \end{aligned}
  \\
  \sem{\matchsum{e}{x}{c_1}{y}{c_2}}{\eta} &=
  \begin{cases}
    \sem{c_1}{\extend{\eta}{x}{v}} & \text{if $\rho_{+}(\sem{e}{\eta}) = \iota_0(v)$} \\
    \sem{c_2}{\extend{\eta}{y}{v}} & \text{if $\rho_{+}(\sem{e}{\eta}) = \iota_1(v)$} \\
    \bot & \text{otherwise}
  \end{cases}
  \\
  \sem{e_1 \, e_2}{\eta} &= \rho_\to(\sem{e_1}{\eta}) (\sem{e_2}{\eta})
\end{align*}  
%
For the interpretation of $\kord{new}$ we need a way of generating fresh names so that we
may sensibly interpret
%
\begin{equation*}
  \sem{\new E}{\eta} = \iota_{\kord{res}}(\iota_\kord{val}(\iota_{\kord{effect}}(n)))
  \quad \text{where $n \in \II$ is fresh.}
\end{equation*}
%
The implementation simply uses a local counter, but a satisfactory semantic solution needs
a model of names, such as the permutation models of Pitts and
Gabbay~\cite{gabbay01a-new-approach}, with $\II$ then being the set of atoms.

Finally, the handling construct is just an application
%
\begin{equation*}
  \sem{\handle{c}{e}}{\eta} = \rho_{\hto}(\sem{e}{\eta}) (\sem{c}{\eta}).
\end{equation*}


\subsection{Semantics of resources}
\label{sec:semantics-resources}

To model resources, the denotational semantics has to support the mutable nature of resource state, for example by explicitly threading it through the evaluation.
But we prefer not to burden ourselves and the readers with the ensuing technicalities.
Instead, we assume a mutable store $\sigma$ indexed by effect instances which supports the lookup and update operations.
That is, $\lookup{\sigma}{n}$ gives the current contents at location $n \in \II$, while
$\change{\sigma}{n}{v}{x}$ sets the contents at location $n$ to $v \in V$ and yields $x$.

A resource describes the behaviour of operations, i.e., it is a map $\OO \times V \times V \to V \times V$ which computes a value and the new state from a given operation symbol, parameter, and the current state. Thus an effect instance is not just an element of $\II$ anymore, but an element of $\JJ = \II \times (V \times V)^{\OO \times V \times V}$. Consequently in the semantics we replace $\II$ with $\JJ$ throughout and adapt the semantics of $\kord{new}$ so that it sets the initial resource state and gives an element of $\JJ$:
%
\begin{multline*}
  \sem{\newwith{E}{e}{(\kpre{operation} \op_i \, x \, @ \, y \mapsto c_i)_i}}{\eta} = \\
    \change{\sigma}{n}{\sem{e}{\eta}}
           {\iota_\kord{res}(\iota_\kord{val}(\iota_\kord{effect}(n,r)))}
    \quad \text{where $n \in \II$ is fresh}
\end{multline*}
%
where $r : \OO \times V \times V \to V \times V$ is defined by
%
\begin{equation*}
  r(\op, v, s) =
  \begin{cases}
    \rho_{\times}(\rho_{\kord{val}}(\rho_\kord{res}(\sem{c_i}{\extends{\eta}{x \sto v, y \sto s}}))) &
      \text{if $\op = \op_i$ for some $i$,} \\
    \bot & \text{otherwise.}
  \end{cases}
\end{equation*}
%
If no resource is provided, we use a trivial one:
%
\begin{equation*}
  \sem{\new E}{\eta} =
  \iota_\kord{res}(\iota_\kord{val}(\iota_\kord{effect}(n, \bot)))
  \quad \quad \text{where $n \in \II$ is fresh.}
\end{equation*}

Finally, to model evaluation at the toplevel, we define $\Evalsym : (V + \operationDomain)_\bot \to V$ by cases:
%
\begin{align*}
  \Eval{\iota_{\val}(v)} &= v \\
  \Eval{\iota_{\kord{oper}}((n, r), \op, v, \kappa)} &=
  \begin{aligned}[t]
  &\change{\sigma}{n}{s}{\Eval{\rho_\kord{res}(\kappa \, v')}} \\
  &\qquad \text{where $(v', s) = r(\op, v, \lookup{\sigma}{n})$}.
  \end{aligned}
\end{align*}
%
The meaning of a computation $c$ at the toplevel in the environment $\eta$ (and an
implicit resource state $\sigma$) is $\Eval{\rho_\kord{res}(\sem{c}{\eta})}$.


%%% Local Variables:
%%% mode: latex
%%% TeX-master: "eff"
%%% End:

\section{Implementation}
\label{sec:implementation}

To experiment with \eff we implemented a prototype interpreter whose main evaluation loop
is essentially the same as the denotational semantics described in
Section~\ref{sec:semantics}. Apart from inessential improvements, such as recursive type
definitions, inclusion of $\kord{for}$ and $\kord{while}$ loops, and pattern matching, the
implemented language differs from the one presented here in two ways that make it usable:
we implemented Hindley-Milner style type inference with parametric
polymorphism~\cite{milner78atheory}, and in the concrete syntax we hid the distinction
between expressions and computations. We briefly discuss each.

There are no surprises in the passage from monomorphic type checking to parametric
polymorphism and type inference. The infamous value
restriction~\cite{wright95simpleimperative} is straightforward because the distinction
between expressions and computations corresponds exactly to the distinction between
nonexpansive and expansive terms. In fact, it may be worth investing in an effect system
that would relax the value restriction on those computations
that can safely be presumed pure. Because $\new E$ is a computation, effect instances are
\emph{not} polymorphic, which is in agreement with ML-style references being
non-polymorphic.
%, see Section~\ref{sec:state}.
% There is nothing in this section about value restriction

A syntactic division between pure expressions and possibly effectful computations is
annoying because even something as simple as $f \, x \, y$ has to be written as
$\letin{g = f\, x} g \, y$, and having to insert $\kord{val}$ in the right places is no
fun either. Therefore, the concrete syntax allows the programmer to arbitrarily mix expressions
and computations, and a desugaring phase appropriately separates the two.

The desugaring process is fairly simple. It inserts a $\kord{val}$ when an expression
stands where a computation is expected. And if a computation stands where an expression is expected, the computation is hoisted to an enclosing $\kord{let}$
statement. Because several computations may be hoisted from a single expression, the
question arises how to order the corresponding $\kord{let}$ statements. For example, $(f\, x, g\, y)$ can be desugared as
%
\begin{equation*}
  \begin{aligned}
    &\letin{a = f \, x} \\ &\quad \letin{b = g \, y} (a, b)
  \end{aligned}
  \qquad\text{or}\qquad
  \begin{aligned}
    &\letin{b = g \, y} \\ &\quad \letin{a = f \, x} (a, b)
  \end{aligned}
\end{equation*}
%
The order of $f \, x$ and $g \, y$ matters when both computations cause computational
effects. The desugaring phase avoids making a decision by using the \emph{simultaneous}
$\kord{let}$ statement
%
\begin{equation*}
  \kpre{let} a = f \, x \kop{and} b = g \, y \kop{in} (a, b)
\end{equation*}
%
which leaves open the possibility of various compiler optimizations. The prototype simply
evaluates simultaneous bindings in the order they are given, and a command-line
option enables sequencing warnings about possible unexpected order of effects.
%
It could be argued that the warnings should actually be errors, but we allow some slack until
we have an effect system that can detects harmless instances of simultaneous binding.

For one-off handlers, \eff provides an inline syntax so that one can write

\begin{center}
\begin{tabular}{ccc}
\begin{source}
handle
  $c$
with
  $\cdots$
\end{source}&
\quad instead of \quad \hbox{}&
\begin{source}
with
  handler
    $\cdots$
handle
  $c$
\end{source}
\end{tabular}
\end{center}
%
Additionally, the \inline{val} and \inline{finally} clauses may be omitted, in which case
they are assumed to be the identities.

%%% Local Variables: 
%%% mode: latex
%%% TeX-master: "eff"
%%% End: 

\chapter{Example Sheets}

%% TJWP removed plain tex commands
%\magnification\magstephalf
%\font\bb=msbm10 scaled 1200
%\font\small=cmr8
%\font\little=cmmi7
%%%%%%%%%%%%%%%%%%%%%%%

%% TJWP include page numbers
%\nopagenumbers
\def \pr{\partial}
\def\pmb#1{\setbox0=\hbox{#1}%
 \kern-.025em\copy0\kern-\wd0
 \kern.05em\copy0\kern-\wd0
 \kern-.025em\raise.0433em\box0 }
\def \bE{{\pmb {${\cal E}$}}}
\noindent

%\centerline{{\bf Example Sheet 1}}
%\vskip 10pt
\section{Example Sheet 1}

\noindent
{\bf 1.} Explain why 
\vskip 0.3cm
(i) GR effects are important for neutron stars but not for white
dwarfs
\vskip 0.3cm
(ii) inverse beta-decay becomes energetically favourable for
densities higher than those in white dwarfs.
\vskip 10 pt
\noindent
{\bf 2.} Use Newtonian theory to derive the Newtonian pressure
support equation
$$
P'(r) \equiv {dP\over dr} = -{Gm\rho\over r^2}\ ,
$$
where
$$
m=4\pi \int_0^r \tilde r^2 \rho(\tilde r) d\tilde r\ ,
$$
for a spherically-symmetric and static star with pressure $P(r)$ and 
density $\rho(r)$. Show that
%% TJWP no eqalign in latex  \eqalign{
\bean
\int_0^r P(\tilde r)\tilde r^3 d\tilde r & = & 
{P(r)r^4\over 4} -{1\over 4}\int_0^r P'(\tilde r) \tilde r^4 d\tilde 
r \\
 & = & {Gm^2(r)\over 32\pi} + {P(r)r^4\over 4}\ .
\eean
Assuming that $P'\le 0$, with $P=0$ at the star's surface, show that
$$
{d\over dr}\left[\left(\int_0^r P(\tilde r)\tilde r^3 d\tilde
r\right)^{3/4}\right] \le {3\sqrt{2}\over 4} P^{3/4} r^2\ .
$$
Assuming the bound
$$
P\ {\buildrel <\over\sim}\ (\hbar c) n_e^{4/3}\ ,
$$
where $n_e(r)$ is the electron number density, show that the total
mass, $M$, of the star satisfies
$$
M\ {\buildrel <\over\sim}\ \left({hc\over
G}\right)^{3/2}\left({\mu_e\over m_N}\right)^2 
$$
where $m_N$ is the nucleon mass and $\mu_e$ is the number of
electrons per nucleon. Why is it reasonable to bound the pressure as
you have done? Compare your bound with Chandresekhar's limit.

\vskip 10 pt
\noindent
{\bf 3.} A particle orbits a Schwarzschild black hole with
non-zero angular momentum per unit mass $h$. Given that $\sigma=0$ for
a massless particle and $\sigma=1$ for a massive particle, show that
the orbit satisfies  
$$ 
{d^2 u\over d\phi^2} + u = {M\sigma \over
h^2} + 3Mu^2 
$$
where $u=1/r$ and $\phi$ is the azimuthal angle. Verify that this
equation is solved by
$$
u= {1\over 6M} + {2\omega^2\over 3M} -{2\omega^2\over M\cosh ^2
(\omega\phi)}\ ,
$$
where $\omega$ is given by
$$
4\omega^2 = \pm\sqrt{\left(1 - {12M^2\sigma\over h^2}\right)}\ .
$$
where $\sigma=1$ for a massive particle and $\sigma=0$ for a massless
particle. Interpret these orbits in terms of the effective potential.
Comment on the cases $\omega^2=1/4$, $\omega^2=1/8$ and $\omega^2=0$.

\vskip 10 pt
\noindent
{\bf 4.} A photon is emitted outward from a point P outside
a Schwarzschild black hole with radial coordinate r in the range
$2M<r<3M$. Show that if the photon is to reach infinity the angle its
initial direction makes with the radial direction (as determined by 
a stationary observer at P) cannot exceed
$$
{\rm arcsin} \sqrt{{27M^2\over r^2}\left( 1-{2M\over r}\right)}\ .
$$

\vskip 10 pt
\noindent
{\bf 5.} Show that in region II of the Kruskal manifold one may
regard $r$ as a time coordinate and introduce a new spatial
coordinate $x$ such that
$$
ds^2 = -{dr^2\over \left({2M\over r} -1\right)} +\left({2M\over
r}-1\right)dx^2 + r^2d\Omega^2\ .
$$
Hence show that {\it every} timelike
curve in region II intersects the singularity at $r=0$ within a proper
time no greater than $\pi M$. For what curves is this bound attained?
Compare your result with the time taken for the collapse of a ball of
pressure free matter of the same gravitational mass $M$. Calculate
the binding energy of such a ball of dust as a fraction of its
(conserved) rest mass.
\vskip 10 pt
\noindent
{\bf 6.} Using the map
$$
(t,x,y,z) \mapsto X= \pmatrix{t+z & x+iy\cr x-iy & t-z}\ ,
$$
show that Minkowski spacetime may be identified with the space of
Hermitian $2\times 2$ matrices $X$ with metric
$$
ds^2 = -\det (dX)\ .
$$
Using the Cayley map $X\mapsto U={1+iX\over1-iX}$, show further that
Minkowski spacetime may be identified with the space of unitary
$2\times 2$ matrices $U$ for which $\det (1+U)\ne0$. Now show that any
$2\times 2$ unitary matrix $U$ may be expressed uniquely in terms of
a real number $\tau$ and two complex numbers $\alpha$, $\beta$, as 
$$
U=e^{i\tau}\pmatrix{\alpha & \beta\cr -\bar\beta & \bar\alpha}
$$
where the parameters $(\tau,\alpha,\beta)$ satisfy $|\alpha|^2
+|\beta|^2 =1$, and are subject to the identification 
$$
(\tau,\alpha,\beta) \sim (\tau +\pi, -\alpha, -\beta)\ .
$$
Using the relation
$$
(1+U)dX = -2i dU(1+U)^{-1}\ ,
$$
deduce that 
$$
ds^2 = {1\over (\cos \tau + {\cal R}e\,\alpha)^2}\left(-d\tau^2
+|d\alpha|^2 + |d\beta|^2\right) 
$$
is the metric on Minkowski spacetime and hence conclude that the
conformal compactification of Minkowski spacetime may be identified
with the space of unitary $2\times 2$ matrices, i.e the group
$U(2)$. Explain how $U(2)$ may be identified with a portion of the
Einstein static universe $S^3\times {\bb R}$.

\vfill\eject

%\centerline {{\bf Example Sheet 2}}
%\vskip 10 pt
\section{Example Sheet 2}

\noindent
{\bf 1.} Let $\zeta$ be a Killing vector field. Prove that
$$
D_\sigma D_\mu \zeta_\nu = R_{\nu\mu\sigma}{}^\lambda \zeta_\lambda\ ,
$$
where $R_{\nu\mu\sigma\lambda}$ is the Riemann tensor, defined by
$[D_\mu,D_\nu]\, v_\rho = R_{\mu\nu\rho}{}^\sigma v_\sigma$ for
arbitrary vector field $v$.

\vskip 10 pt
\noindent
{\bf 2.}  A conformal Killing vector is one for which  
$$
({\cal L}_\xi g)_{\mu\nu} = \Omega^2 g_{\mu\nu}\ .
$$
for some non-zero function $\Omega$.
Given that $\xi$ is a Killing vector of $ds^2$, show that it is a
conformal Killing vector of the conformally-equivalent
metric $\Lambda^2 ds^2$ for arbitrary (non-vanishing) conformal factor
$\Lambda$.
 
Show that the action for a {\it massless} particle,
$$ S[x,e]={1\over2}\int d\lambda\, e^{-1}\dot x^\mu\dot x^\nu
g_{\mu\nu}(x)\ , $$ is invariant, to first order in the constant
$\alpha$, under the transformation 
$$
x^\mu \rightarrow x^\mu + \alpha \xi^\mu(x) \qquad \qquad
e \rightarrow e + {1\over4}\alpha\, e g^{\mu\nu}({\cal L}_\xi
g)_{\mu\nu}
$$
if $\xi =\xi^\mu \partial_\mu$ is a conformal Killing vector. Show
that $\xi$ is the operator corresponding to the conserved charge
implied by Noether's theorem.



\vskip 10 pt
\noindent
{\bf 3.} Show that the extreme RN metric in isotropic coordinates is
$$
ds^2 = -\left(1+{M\over\rho}\right)^{-2}dt^2 + 
\left(1+{M\over\rho}\right)^{2}\left(d\rho^2 + \rho^2 d\Omega^2\right)
\qquad (\dagger)
$$
Verify that $\rho=0$ is at infinite proper distance from any finite
$\rho$ along any curve of constant $t$. Verify also that
$|t|\rightarrow \infty$ as $\rho\rightarrow 0$ along any timelike or
null curve but that a timelike or null ingoing radial geodesic reaches
$\rho=0$ for {\it finite} affine parameter. By introducing a null
coordinate to replace $\rho$ show that $\rho=0$ is merely a coordinate
singularity and hence that the metric ($\dagger$) is geodesically
incomplete. What happens to the particles that reach $\rho=0$?
Illustrate your answers using a Penrose diagram.

\vskip 10 pt
 \noindent
{\bf 4.} The action for a particle of mass $m$ and charge $q$ is
$$
S[x,e] =\int d\lambda\,\left[{1\over2} e^{-1}\dot x^\mu\dot x^\nu
g_{\mu\nu}(x) -{1\over2}m^2 e -q\, \dot x^\mu A_\mu(x)\right]\qquad
\qquad (*) 
$$
where $A_\mu$ is the electromagnetic 4-potential. Show
that if
$$
({\cal L}_\xi A)_\mu \equiv \xi^\nu\partial_\nu A_\mu +
(\partial_\mu \xi^\nu) A_\nu =0
$$
for Killing vector $\xi$, then $S$ is invariant, to first-order in
$\xi$, under the transformation $x^\mu\rightarrow x^\mu
+\alpha\xi^\mu(x)$. Verify that the corresponding Noether charge
$$
-\xi^\mu \left(m u_\mu -q A_\mu\right)\ ,
$$ 
where $u^\mu$ is the particle's 4-velocity, is a constant of the 
motion. Verify for the Reissner-Nordstrom solution of the vacuum
Einstein-Maxwell equations, with mass $M$ and charge $Q$, that ${\cal
L}_k A =0$ for $k={\partial\over \partial t}$ and hence deduce, for
$m\ne 0$, that  
$$
\left(1-{2M\over r} + {Q^2\over r^2}\right) {dt\over d\tau} =
\varepsilon -{q\over m}{Q\over r}\ ,
$$
where $\tau$ is the particle's proper time and $\varepsilon$ is the
energy per unit mass. Show that the trajectories $r(t)$ of
massive particles with zero angular momentum satisfy
$$
({dr\over d\tau})^2 = (\varepsilon^2 -1) + \left(1-\varepsilon
{qQ\over mM}\right){2M\over r} +\left(\left({q\over m}\right)^2
-1\right){Q^2\over r^2}\ .
$$
Give a physical interpretation of this result for the special case
for which $q^2=m^2$, $qQ=mM$, and $\varepsilon=1$.


\vskip 10 pt
\noindent
{\bf 5.} Show that the action
$$
S[p,x,e]=\int d\lambda \big\{ p_\mu \dot x^\mu
-{1\over2}e\, \big[g^{\mu\nu}(x)p_\mu p_\nu + m^2\big]\big\}
$$
for a point particle of mass $m$ is equivalent, for $q=0$, to the action
of Q.4. Show that $S$ is invariant to
first order in $\alpha$ under the transformation
$$
\delta x^\mu =\alpha K^{\mu\nu}p_\nu \qquad \delta p_\mu
=-{1\over2}\alpha\, p_\rho p_\sigma \partial_\mu K^{\rho\sigma}
$$
for any symmetric tensor $K_{\mu\nu}$ obeying the {\it Killing
tensor} condition
$$
D_{(\rho} K_{\mu\nu)}=0\ .
$$
Show that the corresponding Noether charge is proportional to
$K^{\mu\nu}p_\mu p_\nu$ and verify that it is a constant of the
motion. A trivial example is $K_{\mu\nu}=g_{\mu\nu}$; what is
the corresponding constant of the motion? Show that
$\xi_\mu\xi_\nu$ is a Killing tensor if $\xi$ is a Killing vector. [A
Killing tensor that cannot be constructed from the metric and Killing
vectors is said to be irreducible. In a general axisymmetric
metric there are no such tensors, and so only three constants of the
motion, but for geodesics of the Kerr-Newman metric there is a 
`fourth constant' of the motion corresponding to an
irreducible Killing tensor.]
%$$
%\eqalign{
%K_{\mu\nu}dx^\mu dx^\nu = - &{\Sigma a^2\cos^2\theta\over \Delta}dr^2
%+ {\Delta a^2 \cos^2\theta\over \Sigma}(dt -a\sin^2\theta\,
%d\phi)^2\cr & +{r^2\sin^2\theta\over\Sigma}[adt - (r^2 + a^2)d\phi]^2
%+ r^2\Sigma
% d\theta^2\ .}
%$$

\vskip 10 pt 
\noindent 
{\bf 6.} By replacing the time coordinate $t$
by one of the radial null coordinates
$$
u= t+ {M\over \lambda} \qquad v= t- {M\over \lambda}
$$
show that the singularity at $\lambda=0$ of the Robinson-Bertotti (RB)
metric
$$
ds^2 = -\lambda^2 dt^2 + M^2 \left({d\lambda\over \lambda}\right)^2 +
M^2 d\Omega^2 
$$
is merely a coordinate singularity.
Show also that $\lambda=0$ is a degenerate Killing Horizon with respect
to $\partial\over \partial t$. By
introducing the new coordinates $(U,V)$, defined by
$$
u= \tan \left({U\over 2}\right)\qquad v=-\cot \left({V\over2}\right)
$$
obtain the maximal analytic extension of the RB metric and deduce its
Penrose diagram.

\vfill\eject

%\centerline {{\bf Example Sheet 3}}
%\vskip 10pt
\section{Example Sheet 3}

\noindent 
{\bf 1.} Let $\varepsilon$ and $h$ be the energy and and angular
momentum per unit mass of a zero charge particle in free fall
within the equatorial plane, i.e on a timelike ($\sigma=1$) or null
($\sigma=0$) geodesic with $\theta=\pi/2$, of a Kerr-Newman black hole.
Show that the particle's Boyer-Lindquist radial coordinate $r$
satisfies   
$$ 
\left({dr\over d\lambda}\right)^2 =\varepsilon^2 - V_{eff}(r)\ ,
$$ 
where $\lambda$ is an affine parameter, and the effective potential
$V_{eff}$ is given by $$ 
V_{eff} =
\left(1-{2M\over r}+{e^2\over r^2}\right)\left(\sigma + {h^2\over
r^2}\right) + {2a\varepsilon h\over r^3}\left( 2M -{e^2\over r}\right)
+{a^2\over r^2}\left[\sigma-\varepsilon^2\left(1+{2M\over r}-{e^2\over
r^2}\right)\right] \ .
$$

\vskip 10 pt 
\noindent 
{\bf 2.} Show that the surface gravity of the event horizon of a Kerr
black hole of mass $M$ and angular momentum $J$ is given by
$$
\kappa = {\sqrt{M^4 -J^2}\over 2M( M^2 + \sqrt{M^4 -J^2})}\ .
$$

\vskip 10 pt 
\noindent 
{\bf 3.} A particle at fixed $r$ and $\theta$ in a stationary
spacetime, with metric $ds^2= g_{\mu\nu}(r,\theta)dx^\mu dx^\nu$, has
angular velocity $\Omega= {d\phi\over dt}$ with respect to infinity.
Show that $\Omega(r,\theta)$ must satisfy
$$
g_{tt} + 2g_{t\phi}\Omega + g_{\phi\phi}\Omega^2 \le 0
$$
and hence deduce that
$$
{\cal D}\equiv g_{t\phi}^2 - g_{tt}g_{\phi\phi} \ge 0
$$
Show that ${\cal D}=\Delta (r) \sin^2\theta$ for the Kerr-Newman 
metric in Boyer-Lindquist coordinates, where $\Delta= r^2-2Mr+a^2+e^2$.
What happens if $(r,\theta)$ are such that ${\cal D}<0$? For what
values of $(r,\theta)$ can $\Omega$ vanish? Given that $r_{\pm}$ are the 
roots of $\Delta$, show that when ${\cal D}=0$  
$$ 
\Omega = {a\over r_{\pm}^2 +a^2} \ .
$$
\vskip 10pt
\noindent
{\bf 4.} Show that the area of the event horizon of a Kerr-Newman
black hole is
$$
A= 8\pi\big[ M^2 - {e^2\over2} + \sqrt{ M^4 -e^2 M^2 -J^2 }\,\big]\ .
$$

\vskip 10 pt
\noindent
{\bf 5.} A perfect fluid has stress tensor
$$
T_{\mu\nu} = (\rho + P)u_\mu u_\nu + P g_{\mu\nu}\ ,
$$
where $\rho$ is the density and $P(\rho)$ the pressure. State the
dominant energy condition for $T_{\mu\nu}$ and show that
for a perfect fluid in Minkowski spacetime this condition
is equivalent to  
$$ 
\rho\ge |P|\ .
$$
Show that the same condition arises from the requirement of
causality, i.e. that the speed of sound, $\sqrt{|dP/d\rho|}$, not
exceed that of light, together with the fact that the pressure 
vanishes in the vacuum.
\vskip 10pt
\noindent
{\bf 6.} The vacuum Einstein-Maxwell equations are
$$
G_{\mu\nu}= 8\pi T_{\mu\nu}(F) \qquad D_\mu F^{\mu\nu}=0
$$
where $F_{\mu\nu}= \partial_{\mu}A_{\nu}- \partial_\nu A_\mu$, and
$$
T_{\mu\nu}(F)= {1\over 4\pi}\big(F_{\mu}{}^\lambda F_{\nu\lambda}
-{1\over4}g_{\mu\nu}F^{\alpha\beta}F_{\alpha\beta}\big)\ .
$$
Asymptotically-flat solutions are stationary and axisymmetric if
the metric admits Killing vectors $k$ and $m$ that can be taken to be
$k={\partial\over\partial t}$ and $m={\partial\over\partial \phi}$ near
infinity, and if (for some choice of electromagnetic gauge)
$$
{\cal L}_k A={\cal L}_m A=0\ ,
$$
where the Lie derivative of $A$ with respect to a vector $\xi$, 
${\cal L}_\xi A$, is as defined in Q.4 of Example Sheet 2. The event
horizon of such a solution is necessarily a Killing horizon of $\xi =
k+\Omega_H m$, for some constant $\Omega_H$. What is the physical
interpretation of $\Omega_H$? What is its value for the Kerr-Newman
solution? The co-rotating electric potential is defined by 
$$ 
\Phi = \xi^\mu A_\mu \ .
$$
Use the fact that $R_{\mu\nu}\xi^\mu\xi^\nu=0$ on a Killing horizon
to show that $\Phi$ is constant on the horizon. In particular, show
that for a choice of the electromagnetic gauge for which $\Phi=0$ at
infinity, 
$$
\Phi_H= {Qr_+\over r_+^2 +a^2}
$$
for a charged rotating black hole, where $r_+= M+\sqrt{M^2-Q^2-a^2}$.

\vskip 10pt
\noindent
{\bf 7.} Let $({\cal M},g,A)$ be an asymptotically flat, stationary,
axisymmetric, solution of the Einstein-Maxwell equations of Q.6 and
let $\Sigma$ be a spacelike hypersurface with one boundary at spatial
infinity and an internal boundary, $H$, at the event horizon of a black
hole of charge $Q$. Show that    
$$ 
-2\int_\Sigma dS_\mu T^\mu{}_\nu(F)\xi^\nu = \Phi_H Q  
$$
where $\Phi_H$ is the co-rotating electric potential on the horizon. 
Use this result to deduce that the mass $M$ of a charged rotating black
hole is given by 
$$
M= {\kappa A\over 4\pi} + 2\Omega_H J + \Phi_H Q\ .
$$
where $J$ is the total angular momentum.
Use this formula for $M$ to deduce the first law of black hole
mechanics for charged rotating black holes: 
$$
dM= {\kappa \over 8\pi}dA + \Omega_H dJ + \Phi_H dQ\ .
$$
[Hint: ${\cal L}_\xi (F^{\mu\nu}A_\nu)=0$ ]

\vfill\eject


%\centerline{{\bf Example Sheet 4}}
%\vskip 10pt
\section{Example Sheet 4}

\noindent
{\bf 1.} Use the Komar integral,
$$
J= {1\over 16\pi G}\oint_\infty dS_{\mu\nu}D^\mu m^\nu\ ,
$$
for the total angular momentum of an asymptotically-flat axisymmetric
spacetime (with Killing vector $m$) to verify that $J=Ma$ for the
Kerr-Newman solution with parameter $a$.

\vskip 10pt
\noindent
{\bf 2.} Let $l$ and $n$ be two linearly independent vectors and
$\hat B$ a second rank tensor such that
$$
\hat B_\mu{}^\nu l_\nu =\hat B_\mu{}^\nu n_\nu =0\ .
$$
Given that $\eta^{(i)}$ $(i=1,2)$ are two further linearly
independent vectors, show that
$$
\varepsilon^{\mu\nu\rho\sigma}l_\mu n_\nu \hat B_\rho{}^\lambda
\big(\eta^{(1)}_\lambda\eta^{(2)}_\sigma - 
\eta^{(1)}_\sigma\eta^{(2)}_\lambda\big) =  \theta\, 
\varepsilon^{\mu\nu\rho\sigma} l_\mu n_\nu
\eta_\rho^{(1)}\eta_\sigma^{(2)}\ .
$$
where $\theta= \hat B_\alpha{}^\alpha$.

\vskip 10pt
\noindent
{\bf 3.} Let ${\cal N}$ be a Killing horizon of a Killing vector field
$\xi$, with surface gravity $\kappa$. Explain why, for any third-rank
totally-antisymmetric tensor $A$, the scalar 
$\Psi = A^{\mu\nu\rho}(\xi_\mu D_\nu\xi_\rho)$ vanishes on ${\cal N}$.
Use this to show that
$$
(\xi_{[\rho}D_{\sigma]} \xi_\nu)(D^\nu\xi^\mu) =\kappa
\xi_{[\rho}D_{\sigma]}\xi^\mu \qquad ({\rm on}\ {\cal N})\ ,\qquad (*)
$$
where the square brackets indicate antisymmetrization on the enclosed
indices.

>From the fact that $\Psi$ vanishes on ${\cal N}$ it follows that its
derivative on ${\cal N}$ is normal to ${\cal N}$, and hence that
$\xi_{[\mu}\partial_{\nu]}\Psi=0$ on ${\cal N}$. Use this fact and the
Killing vector lemma of Q.II.1 to deduce that, on ${\cal N}$,
$$
(\xi_\nu R_{\sigma\rho[\beta}{}^\lambda\xi_{\alpha]}
+\xi_\rho R_{\nu\sigma[\beta}{}^\lambda\xi_{\alpha]}
+\xi_\sigma R_{\rho\nu[\beta}{}^\lambda\xi_{\alpha]})\xi_\lambda\ .
$$
Contract on $\rho$ and $\alpha$ and use the fact that $\xi^2=0$ on
${\cal N}$ to show that
$$
\xi^\nu\xi_{[\rho}R_{\sigma]\nu\mu}{}^\lambda\xi_\lambda =
-\xi_\mu\xi_{[\rho}R_{\sigma]}{}^\lambda\xi_\lambda
\qquad ({\rm on}\ {\cal N})\ ,\qquad (\dagger)
$$
where $R_{\mu\nu}$ is the Ricci tensor.

For any vector $v$ the scalar $\Phi=(\xi\cdot D\xi -\kappa\xi)\cdot v$
vanishes on ${\cal N}$. It follows that
$\xi_{[\mu}\partial_{\nu]}\Phi|_{\cal N} =0$. Show that this fact, the
result (*) derived above and the Killing vector lemma imply
that, on ${\cal N}$, 
%% $$ \eqalign{
\bean
\xi^\mu\xi_{[\rho}\partial_{\sigma]}\kappa & = & \xi^\nu
R_{\mu\nu[\sigma}{}^\lambda\xi_{\rho]}\xi_\lambda \\
& = &\xi^\nu\xi_{[\rho}R_{\sigma]\nu\mu}{}^\lambda\xi_\lambda\ ,
\eean
%% } $$
where the second line is a consequence of the cyclic identity
satisfied by the Riemann tensor. Now use $(\dagger)$ to show that, on
${\cal N}$,
%$$ \eqalign{
\bea
\xi^\mu\xi_{[\rho}\partial_{\sigma]}\kappa & = &
\xi_{[\sigma}R_{\rho]}{}^\lambda\xi_\lambda \\
&= & 8\pi G\, \xi_{[\sigma}T_{\rho]}{}^\lambda\xi_\lambda\ ,
\eea
%} $$
where the second line follows on using the Einstein equations. Hence
deduce the zeroth law of black hole mechanics: that, provided the
matter stress tensor satisfies the dominant energy condition, the
surface gravity of any Killing vector field $\xi$ is constant on
each connected component of its Killing horizon (in particular, on the
event horizon of a stationary spacetime). 
% TJWP removed \vfill\eject

\vskip 10pt
\noindent
{\bf 4.} A scalar field $\Phi$ in the Kruskal spacetime satisfies
the Klein-Gordon equation
$$
D^2\Phi -m^2\Phi =0\ .
$$
Given that, in static Schwarzshild coordinates, $\Phi$ takes the form
$$
\Phi = R_\ell(r) e^{-i\omega t} Y_{\ell}(\theta,\phi)
$$
where $Y_{\ell\, m}$ is a spherical harmonic, find the radial equation
satisfied by $R_\ell(r)$. Show that near the horizon at $r=2M$,
$\Phi\sim e^{\pm i\omega r^*}$, where $r^*$ is the Regge-Wheeler radial
coordinate. Verify that ingoing waves are analytic, in Kruskal
coordinates, on the future horizon, ${\cal H}^+$, but not, in general,
on the past horizon, ${\cal H}^-$, and conversely for outgoing waves.

Given that both $m$ and $\omega$ vanish, show that
$$
R_\ell = A_\ell P_\ell(z) + B_\ell Q_\ell(z)
$$
for constants $A_\ell,\, B_\ell$, where $z=(r-M)/M$, $P_\ell(z)$ is a
Legendre Polynomial and $Q_\ell(z)$ a linearly-independent solution.
Hence show that there are no {\it non-constant} solutions that are both
regular on the horizon, ${\cal H}= {\cal H}^+ \cup {\cal H}^-$, and
bounded at infinity.

\vskip 10 pt
\noindent
{\bf 5.} Use the fact that a Schwarzschild black hole radiates at the
Hawking temperature
$$
T_H ={1\over 8\pi M}
$$
(in units for which $\hbar$, $G$, $c$, and Bolzmann's constant all
equal $1$) to show that the thermal equilibrium of a black hole with an
infinite reservoir of radiation at temperature $T_H$ is unstable.

A finite reservoir of radiation of volume $V$ at temperature $T$ has
an energy, $E_{res}$ and entropy, $S_{res}$ given by
$$
E_{res} = \sigma VT^4 \qquad S_{res} ={4\over3}\sigma VT^3
$$
where $\sigma$ is a constant. A Schwarzschild black hole of mass $M$ is
placed in the reservoir. Assuming that the black hole has entropy
$$
S_{BH} =4\pi M^2\ ,
$$
show that the total entropy $S= S_{BH}+S_{res}$ is extremized 
for fixed total energy $E= M+E_{res}$, when $T=T_H$, Show that the
extremum is a maximum if and only if $V<V_c$, where the critical value
of $V$ is $$
V_c = {2^{20}\pi^4E^5\over 5^5\sigma }
$$
What happens as $V$ passes from $V<V_c$ to $V>V_c$, or
vice-versa?


\vskip 10pt
\noindent
{\bf 6.} The specific heat of a charged black hole of mass $M$, at
fixed charge $Q$, is 
$$
C\equiv T_H {\partial S_{BH}\over \partial
T_H}\bigg|_Q \ ,
$$
where $T_H$ is its Hawking temperature and $S_{BH}$ its
entropy. Assuming that the entropy of a black hole is given by $S_{BH}=
{1\over4}A$, where $A$ is the area of the event horizon, show that the
specific heat of a Reissner-Nordstrom black hole is
$$
C= {2S_{BH}\sqrt{M^2-Q^2}\over (M-2\sqrt{M^2-Q^2})}\ .
$$
Hence show that $C^{-1}$ changes sign when $M$ passes through
${2|Q|\over\sqrt{3}}$. 

Repeat Q.5 for a Reissner-Nordstrom black hole.
Specifically, show that the critical reservoir volume, $V_c$, is
infinite for $|Q|\le M \le {2|Q|\over\sqrt{3}}$. Why is this result
to be expected from your previous result for $C$? 



%\end










\section{Directions and Conclusions}
\label{sec:conclusions}

We have presented our initial exploration of \DV\ across a wide range of tasks and domains, providing supporting evidence to the claim that \DV's abilities are comparable to human-level for many of them. This conclusion is consistent with the findings by OpenAI presented in \cite{gpt4}. A primary goal of our experiments is to give a preliminary assessment of \DV's {\em intelligence}, which is an arduous task given the lack of formal definition for this concept, especially for artificial systems. We hope that our exploration provides a useful and necessary first step to appreciate the remarkable capabilities and challenges of {\DV}, and that it opens up new opportunities for developing more formal and comprehensive methods for testing and analyzing future AI systems with such broad intelligence. The capabilities of the model, which have been demonstrated above, both in terms of depth and generality, suggest that the machine learning community needs to move beyond classical benchmarking via structured datasets and tasks, and that the evaluation of the capabilities and cognitive abilities of those new models have become much closer in essence to the task of evaluating those of a human rather than those of a narrow AI model. We hope our investigation stimulates further research on {\DV} and similar systems, both in terms of exploring new applications and domains, and in terms of understanding the mechanisms and principles that underlie their intelligence.
\newline

The central claim of our work is that \DV\ attains a form of \emph{general} intelligence, indeed showing {\em sparks of artificial general intelligence}. This is demonstrated by its core mental capabilities (such as reasoning, creativity, and deduction), its range of topics on which it has gained expertise (such as literature, medicine, and coding), and the variety of tasks it is able to perform (e.g., playing games, using tools, explaining itself, ...). A lot remains to be done to create a system that could qualify as a complete AGI. We conclude this paper by discussing several immediate next steps, regarding defining AGI itself, building some of missing components in LLMs for AGI, as well as gaining better understanding into the origin of the intelligence displayed by the recent LLMs.

\subsection{Definitions of intelligence, AI, and AGI} \label{sec:otherdefinitions}
In this paper, we have used the 1994 definition of intelligence by a group of psychologists \cite{gottfredson1997mainstream} as a guiding framework to explore \DV's artificial intelligence. This definition captures some important aspects of intelligence, such as reasoning, problem-solving, and abstraction, but it is also vague and incomplete. It does not specify how to measure or compare these abilities. Moreover, it may not reflect the specific challenges and opportunities of artificial systems, which may have different goals and constraints than natural ones. Therefore, we acknowledge that this definition is not the final word on intelligence, but rather a useful starting point for our investigation. There is a rich and ongoing literature that attempts to propose more formal and comprehensive definitions of intelligence, artificial intelligence, and artificial general intelligence \cite{goertzel2014artificial, chollet2019measure}, but none of them is without problems or controversies.
For instance, Legg and Hutter \cite{legg2008machine} propose a goal-oriented definition of artificial general intelligence: Intelligence measures an agent’s ability to achieve goals in a wide range of environments. However, this definition does not necessarily capture the full spectrum of intelligence, as it excludes passive or reactive systems that can perform complex tasks or answer questions without any intrinsic motivation or goal. One could imagine as an artificial general intelligence, a brilliant oracle, for example, that has no agency or preferences, but can provide accurate and useful information on any topic or domain. Moreover, the definition around achieving goals in a wide range of environments also implies a certain degree of universality or optimality, which may not be realistic (certainly human intelligence is in no way universal or optimal). The need to recognize the importance of priors (as opposed to {\em universality}) was emphasized in the definition put forward by Chollet in \cite{chollet2019measure} which centers intelligence around skill-acquisition efficiency, or in other words puts the emphasis on a single component of the 1994 definition: learning from experience (which also happens to be one of the key weaknesses of LLMs). Another candidate definition of artificial general intelligence from Legg and Hutter \cite{legg2007universal} is: a system that can do anything a human can do. However, this definition is also problematic, as it assumes that there is a single standard or measure of human intelligence or ability, which is clearly not the case. Humans have different skills, talents, preferences, and limitations, and there is no human that can do everything that any other human can do. Furthermore, this definition also implies a certain anthropocentric bias, which may not be appropriate or relevant for artificial systems. While we do not adopt any of those definitions in the paper, we recognize that they provide important angles on intelligence. For example, whether intelligence can be achieved without any agency or intrinsic motivation is an important philosophical question. Equipping LLMs with agency and intrinsic motivation is a fascinating and important direction for future work. With this direction of work, great care would have to be taken on alignment and safety per a system's abilities to take autonomous actions in the world and to perform autonomous self-improvement via cycles of learning. We discuss a few other crucial missing components of LLMs next.

\subsection{On the path to more general artificial intelligence}
%We have provided evidence supporting that claim that {\DV} performance on a wide range of tasks is comparable to human-level abilities. We have argued that the model attains a form of \emph{general} intelligence in terms of core mental capabilities (such as reasoning, creativity, and deduction), in terms of the range of topics on which is has gained expertise (such as literature, medicine, and coding), and in terms of the variety of tasks it is able to perform (e.g., playing games, using tools, explaining itself, ...). We have also shown that {\DV} can generate and understand content that combines different topics, skills, and modalities, demonstrating its flexibility and creativity and that, despite being trained purely on text, it demonstrates remarkable capabilities in a variety of modalities such as vision. We have compared {\DV}'s performance to those of previous large language models (LLMs), most notably ChatGPT \cite{gpt3}, and we have found that {\DV} is far superior in terms of generality, creativity, and closeness to human-level intelligence. 

%As we allude to in the title of the paper, this work explores a ``first contact" with {\DV} and its potential descendants, rather than a comprehensive evaluation of the model's intelligence. We hope that our exploration provides a useful and necessary first step to appreciate the remarkable capabilities and challenges of {\DV}, and that it opens up new opportunities for developing more formal and comprehensive methods for testing and analyzing future AGI systems. The capabilities of the model, which have been demonstrated above, both in terms of depth and generality, suggest that the machine learning community needs to move beyond classical benchmarking via structured datasets and tasks, and that the evaluation of the capabilities and cognitive abilities of those new models have become much closer in essence to the task of evaluating those of a human rather than those of a narrow AI model. We hope our investigation stimulates further research on {\DV} and similar systems, both in terms of exploring new applications and domains, and in terms of understanding the mechanisms and principles that underlie their intelligence.

%We have also identified some of the main drawbacks of \DV, and we have discussed how they might be addressed in future work. These drawbacks include:
Some of the areas where \DV\ (and LLMs more generally) should be improved to achieve more general intelligence include (note that many of them are interconnected):
\begin{itemize}
    \item \textbf{Confidence calibration:} The model has trouble knowing when it should be confident and when it is just guessing. It both makes up facts that have not appeared in its training data, and also exhibits inconsistencies between the generated content and the prompt, which we referred to as {\em open-domain} and {\em closed-domain} hallucination in Figure \ref{fig:hallucination}. These hallucinations can be stated in a confident and persuasive manner that can be difficult to detect. Thus, such generations can lead to errors, and also to confusion and mistrust. While hallucination is a good thing when generating creative content, reliance on factual claims made by a model with hallucinations can be costly, especially for uses in high-stakes domains such as healthcare. There are several complementary ways to attempt to address hallucinations. One way is to improve the calibration of the model (either via prompting or fine-tuning) so that it either abstains from answering when it is unlikely to be correct or provides some other indicator of confidence that can be used downstream. Another approach, that is suitable for mitigating open-domain hallucination, is to insert information that the model lacks into the prompt, for example by allowing the model to make calls to external sources of information, such as a search engine as in Section \ref{sec:affordances}. For closed-domain hallucination the use of additional model computation through post-hoc checks is also promising, see Figure \ref{fig:hallucination} for an example. Finally, building the user experience of an application with the possibility of hallucinations in mind can also be part of an effective mitigation strategy. %Other directions include developing and refining mechanisms that endow systems with well-calibrated likelihoods that its generations are grounded, or, more directly, the likelihood that it is hallucinating versus relying upon and communicating content that it has learned from its training data. 
    \item \textbf{Long-term memory:} The model's context is very limited (currently 8000 tokens, but not scalable in terms of computation), it operates in a ``stateless" fashion and there is no obvious way to teach the model new facts. In fact, it is not even clear whether the model is able to perform tasks which require an evolving memory and context, such as reading a book, with the task of following the plot and understanding references to prior chapters over the course of reading.
    \item \textbf{Continual learning:} The model lacks the ability to update itself or adapt to a changing environment. The model is fixed once it is trained, and there is no mechanism for incorporating new information or feedback from the user or the world. One can fine-tune the model on new data, but this can cause degradation of performance or overfitting. Given the potential lag between cycles of training, the system will often be out of date when it comes to events, information, and knowledge that came into being after the latest cycle of training.
    \item \textbf{Personalization:} Some of the applications require the model to be tailored to a specific organization or end user. The system may need to acquire knowledge about the workings of an organization or the preferences of an individual. And in many cases, the system would need to adapt in a personalized manner over periods of time with specific changes linked to the dynamics of people and organizations. For example, in an educational setting, there would be an expectation of the need for the system to understand particular learning styles as well as to adapt over time to a student's progress with comprehension and prowess. The model does not have any way to incorporate such personalized information into its responses, except by using meta-prompts, which are both limited and inefficient. 
    \item \textbf{Planning and conceptual leaps:} As suggested by the examples in Section \ref{sec:limitations}, the model exhibits difficulties in performing tasks that require planning ahead or that require a ``Eureka idea" constituting a discontinuous conceptual leap in the progress towards completing a task. In other words, the model does not perform well on tasks that require the sort of conceptual leaps of the form that often typifies human genius.  
    \item \textbf{Transparency, interpretability and consistency:} Not only does the model hallucinate, make up facts and produce inconsistent content, but it seems that the model has no way of verifying whether or not the content that it produces is consistent with the training data, or whether it's self-consistent. While the model is often able to provide high-quality post-hoc explanations for its decisions (as demonstrated in Section \ref{sec:explainability}), using explanations to verify the process that led to a certain decision or conclusion only works when that process is accurately modeled and a sufficiently powerful explanation process is also accurately modeled (Section \ref{sec:explainability}). Both of these conditions are hard to verify, and when they fail there are inconsistencies between the model's decisions and its explanations. Since the model does not have a clear sense of its own limitations it makes it hard to establish trust or collaboration with the user without extensive experimentation in a narrow domain.
    \item \textbf{Cognitive fallacies and irrationality:} The model seems to exhibit some of some of the limitations of human knowledge and reasoning, such as cognitive biases and irrationality (such as biases of confirmation, anchoring, and base-rate neglect) and statistical fallacies. The model may inherit some of the biases, prejudices, or errors that are present in its training data, which may reflect the distribution of opinions or perspectives linked to subsets of the population or  larger common views and assessments. 
     \item \textbf{Challenges with sensitivity to inputs:} The model's responses can be very sensitive to details of the framing or wording of prompts and their sequencing in a session. Such non-robustness suggests that significant effort and experimentation is often required with engineering prompts and their sequencing and that uses in the absence of such investments of time and effort by people can lead to suboptimal and non-aligned inferences and results. 
\end{itemize}

A limitation of our exploration is the absence of a clear distinction between drawbacks founded in the way that the reinforcement learning step (RLHF) was carried out, versus drawbacks which are fundamentally inherent in the larger architecture and methodology. For example, it is not clear to what extent the hallucination problem can be addressed via a refined reinforcement learning step or via a focused effort to introduce new forms of calibration about the likelihoods of the veracity of alternative inferences that the system can compute and consider in its generations (see also \cite{gpt4} for more discussion on this). To draw an analogy to humans, cognitive biases and irrational thinking may be based in artifacts of our culture as well as to limitations in our cognitive capabilities. Pursuing better understandings of the sources and potential solutions to challenges of hallucination in \DV, will benefit from studies that compare several versions of the RL stage over the same architecture.
\newline

A broader question on the identified limitations is: which of the aforementioned drawbacks can be mitigated within the scope of next word prediction? Is it simply the case that a bigger model and more data will fix those issues, or does the architecture need to be modified, extended, or reformulated? Potential extensions to next word prediction include the following:

\begin{itemize}
    \item External calls by the model to components and tools such as a calculator, a database search or code execution, as suggested in Section \ref{sec:affordances}. 
    \item A richer, more complex ``slow-thinking" deeper mechanism that oversees the ``fast-thinking" mechanism of next word prediction. Such an approach could allow the model to perform long-term planning, exploration, or verification, and to maintain a working memory or a plan of action. The slow-thinking mechanism would use the next word prediction model as a subroutine, but it would also have access to external sources of information or feedback, and it would be able to revise or correct the outputs of the fast-thinking mechanism.
    \item Integration of long-term memory as an inherent part of the architecture, perhaps in the sense that both the input and output of the model will include, in addition to the tokens representing the text, a vector which represents the context.
    \item Going beyond single-word prediction: Replacing the sequence of tokens by a hierarchical structure, where higher-level parts of the text such as sentences, paragraphs or ideas are represented in the embedding and where the content is generated in a top-down manner. It is unclear whether richer predictions about the sequencing and interdependency of such higher-level concepts might emerge from large-scale compute and data centered on a next-word--prediction paradigm.
\end{itemize}

%In conclusion, we have worked to demonstrate that {\DV} has remarkable capabilities that challenge many of the recent assumptions and expectations within the AI community. We have also shown that {\DV} is by no means a perfect or complete AGI system, and that it has many limitations and biases that need to be addressed and understood. We hope that our exploration will inspire and inform further research on {\DV} and similar systems, both in terms of exploring their potential applications and domains, and in terms of understanding foundational mechanisms and potentially with identifying principles of intelligence. We believe that {\DV} represents a paradigm shift in the field of computer science and beyond, and that the model and its capabilities frame new questions, possibilities, and horizons for the field and for the advancement of human capabilities and well-being.

\subsection{What is actually happening?} \label{sec:whatsgoingon}
Our study of {\DV} is entirely phenomenological: We have focused on the surprising things that {\DV} can do, but we do not address the fundamental questions of why and how it achieves such remarkable intelligence. How does it reason, plan, and create? Why does it exhibit such general and flexible intelligence when it is at its core merely the combination of simple algorithmic components---gradient descent and large-scale transformers with extremely large amounts of data? These questions are part of the mystery and fascination of LLMs, which challenge our understanding of learning and cognition, fuel our curiosity, and motivate deeper research. Key directions include ongoing research on the phenomenon of emergence in LLMs (see \cite{wei2022emergent} for a recent survey). Yet, despite intense interest in questions about the capabilities of LLMs, progress to date has been quite limited with only toy models where some phenomenon of emergence is proved \cite{barak2022hidden, ahn2022learning,jelassi2022vision}. One general hypothesis \cite{olah2020zoom} is that the large amount of data (especially the  diversity of the content) forces neural networks to learn generic and useful ``neural circuits'', such as the ones discovered in \cite{olsson2022context, zhang2022unveiling, liu2022transformers}, while the large size of models provide enough redundancy and diversity for the neural circuits to specialize and fine-tune to specific tasks. 
%The Mixture of Experts (MoE) layers in modern LLMs can also contribute to the generality of the model~\cite{chen2022towards}. 
Proving these hypotheses for large-scale models remains a challenge, and, moreover, it is all but certain that the conjecture is only part of the answer. On another direction of thinking, the huge size of the model could have several other benefits, such as making gradient descent more effective by connecting different minima \cite{venturi2019spurious} or by simply enabling smooth fitting of high-dimensional data \cite{pmlr-v49-eldan16, NEURIPS2021_f197002b}. Overall, elucidating the nature and mechanisms of AI systems such as {\DV} is a formidable challenge that has suddenly become important and urgent.
\newline

\paragraph{Acknowledgements.} We thank OpenAI for creating such a marvelous tool and giving us early access to experience it. We also thank Miles Brundage at OpenAI, and the numerous people at Microsoft, who have provided thoughtful feedback on this work.

\bibliographystyle{plain}
%\bibliography{bibliography}

% --- including .bbl file generated by Bibtex -----

\begin{thebibliography}{10}

\bibitem{benton00monads}
Nick Benton, John Hughes, and Eugenio Moggi.
\newblock Monads and effects.
\newblock In {\em APPSEM 2000}, volume 2395 of {\em Lecture Notes in Computer
  Science}, pages 42--122, 2000.

\bibitem{danvy92representing}
Oliver Danvy and Andrzej Filinski.
\newblock Representing control: a study of the {CPS} transformation.
\newblock {\em Mathematical Structures in Computer Science}, 2(4):361--391,
  1992.

\bibitem{escardo10selection}
Mart{\'\i}n Escard{\'o} and Paulo Oliva.
\newblock Selection functions, bar recursion and backward induction.
\newblock {\em Mathematical Structures in Computer Science}, 20:127--168, 2010.

\bibitem{hyland07combining}
Martin Hyland, Paul~Blain Levy, Gordon Plotkin, and John Power.
\newblock Combining algebraic effects with continuations.
\newblock {\em Theoretical Computer Science}, 375(1--3):20--40, 2007.

\bibitem{hyland06combining}
Martin Hyland, Gordon Plotkin, and John Power.
\newblock Combining effects: Sum and tensor.
\newblock {\em Theoretical Computer Science}, 357(1--3):70--99, 2006.

\bibitem{OCaml}
Xavier Leroy, Damien Doligez, Alain Frisch, Jacques Garrigue, Didier R\'emy,
  and J\'er\^ome Vouillon.
\newblock {\em The OCaml system (release 3.12): Documentation and user's
  manual}.
\newblock Institut National de Recherche en Informatique et en Automatique,
  2011.

\bibitem{levy03modelling}
Paul~Blain Levy, John Power, and Hayo Thielecke.
\newblock Modelling environments in call-by-value programming languages.
\newblock {\em Information and Computation}, 185:182--210, September 2003.

\bibitem{milner78atheory}
Robin Milner.
\newblock A theory of type polymorphism in programming.
\newblock {\em Journal of Computer and System Sciences}, 17:348--375, 1978.

\bibitem{milner97the-definition}
Robin Milner, Mads Tofte, Robert Harper, and David MacQueen.
\newblock {\em The Definition of {S}tandard {ML}}.
\newblock MIT Press, 1997.

\bibitem{gabbay01a-new-approach}
Gabbay~J. Murdoch and Andrew~M. Pitts.
\newblock A new approach to abstract syntax with variable binding.
\newblock {\em Formal Aspects of Computing}, 13(3--5):341--363, July 2001.

\bibitem{plotkin02notions}
Gordon Plotkin and John Power.
\newblock Notions of computation determine monads.
\newblock In {\em 5th International Conference on Foundations of Software
  Science and Computation Structures}, volume 2303 of {\em Lecture Notes in
  Computer Science}, pages 342--356, 2002.

\bibitem{plotkin03algebraic}
Gordon Plotkin and John Power.
\newblock Algebraic operations and generic effects.
\newblock {\em Applied Categorical Structures}, 11(1):69--94, 2003.

\bibitem{plotkin08:_tensor_comod_model_operat_seman}
Gordon Plotkin and John Power.
\newblock Tensors of comodels and models for operational semantics.
\newblock In Andrej Bauer and Michael Mislove, editors, {\em Proceedings of the
  24th Conference on the Mathematical Foundations of Programming Semantics
  (MFPS XXIV)}, volume 218 of {\em Electronic Notes in Theoretical Computer
  Science}, pages 295--311, 2008.

\bibitem{plotkin09handlers}
Gordon Plotkin and Matija Pretnar.
\newblock Handlers of algebraic effects.
\newblock In {\em ESOP 2009}, volume 5502 of {\em Lecture Notes in Computer
  Science}, pages 80--94, 2009.

\bibitem{plotkin08a-logic}
Gordon~David Plotkin and Matija Pretnar.
\newblock A logic for algebraic effects.
\newblock In {\em 23rd Symposium on Logic in Computer Science}, pages 118--129,
  2008.

\bibitem{power04from}
John Power and Olha Shkaravska.
\newblock From comodels to coalgebras: State and arrays.
\newblock {\em Electronic Notes in Theoretical Computer Science}, 106:297--314,
  2004.

\bibitem{pretnar10:_logic_handl_algeb_effec}
Matija Pretnar.
\newblock {\em The Logic and Handling of Algebraic Effects}.
\newblock PhD thesis, School of Informatics, University of Edinburgh, 2010.

\bibitem{reynolds00themeaning}
John Reynolds.
\newblock The meaning of types---from intrinsic to extrinsic semantics.
\newblock Technical report, Department of Computer Science, University of
  Aarhus, 2000.

\bibitem{wadler95monads}
Philip Wadler.
\newblock Monads for functional programming.
\newblock In {\em Advanced Functional Programming, First International Spring
  School on Advanced Functional Programming Techniques-Tutorial Text}, pages
  24--52, London, UK, 1995. Springer-Verlag.

\bibitem{wright95simpleimperative}
Andrew Wright.
\newblock Simple imperative polymorphism.
\newblock In {\em LISP and Symbolic Computation}, pages 343--356, 1995.

\end{thebibliography}


% ----- end of included file -----


\end{document}


%%% Local Variables:
%%% mode: latex
%%% TeX-master: t
%%% End: