\section{Syntax}
\label{sec:syntax}

\Eff is a statically typed language with parametric polymorphism and type
inference. Its types include products, sums, records, and recursive type
definitions. To keep to the point, we focus on a core language with
monomorphic types and type checking. The concrete syntax follows that of OCaml~\cite{OCaml}, and except for new constructs, we discuss it only briefly.

\subsection{Types}
\label{sec:types}

Apart from the standard types, \eff has \emph{effect types $E$} and \emph{handler
  types $A \hto B$}:
%
\begin{align*}
  \tag{type}
  A, B, C \bnfis {}
    &\intty \bnfor
    \boolty \bnfor
    \unitty \bnfor
    \emptyty \bnfor
    \\
    &A \times B \bnfor
    A + B \bnfor
    A \to B \bnfor
    E \bnfor
    A \hto B,\\
  \tag{effect type}
  E \bnfis {}
    &\kpre{effect} (\kpre{operation} \op_i \T A_i \to B_i)_i \kpost{end}.
\end{align*}
%
In the rule for effect types and elsewhere below $(\cdots)_i$ indicates that
$\cdots$ may be repeated a finite number of times. We include the empty type
as we need it to describe exceptions, see Section~\ref{sec:exceptions}.
%
An effect type describes a collection of related operation symbols, for example those for
writing to and reading from a communication channel. We write $\op \T A \to B \in E$ or
just $\op \in E$ to indicate that the effect type $E$ contains an operation $\op$ with parameters of
type $A$ and results of type $B$.
%
The handler type $A \hto B$ should be understood as the type of handlers acting on
computations of type~$A$ and yielding computations of type~$B$.

\subsection{Expressions and computations}

\Eff distinguishes between \emph{expressions} and \emph{computations}, which are
similar to values and producers of fine-grain call-by-value~\cite{levy03modelling}. The former are inert and free from computational
effects, including divergence, while the latter may diverge or cause
computational effects. As discussed in Section~\ref{sec:implementation}, the
concrete syntax of \eff hides the distinction and allows the programmer to
freely mix expressions and computations.

Beware that we use two kinds of vertical bars below: the tall~$\bnfor$ separates
grammatical alternatives, and the short~$\case$ separates cases in handlers and
match statements. The expressions are
%
\begin{align*}
  \tag{expression}
  e \bnfis {}
    &x \bnfor
    n \bnfor
    c \bnfor
    \tru \bnfor
    \fls \bnfor
    \unt \bnfor
    \pair{e_1, e_2} \bnfor \\
    &\Left{e} \bnfor \Right{e} \bnfor
    \fun{x \T A} c \bnfor
    \hash{e}{\op} \bnfor
    h, \\
  \tag{handler}
  h \bnfis {}
    &\handler
    (\hash{e_i}{\op_i} \, x \, k \mapsto c_i)_i \case
    \val x \mapsto c_v \case
    \fin x \mapsto c_f,
\end{align*}
%
where $x$ signifies a variable, $n$ an integer constant, and $c$ other built-in constants.
The expressions $\unt$, $\pair{e_1, e_2}$, $\Left{e}$, $\Right{e}$, and $\fun{x \T A} c$
are introduction forms for the unit, product, sum, and function types, respectively.
Operations $\hash{e}{\op}$ and handlers $h$ are discussed in
Section~\ref{sec:eff-specific}.

The computations are
%
\begin{align*}
  \tag{computation}
  c \bnfis {}
    &\val e \bnfor
    \letin{x = c_1} c_2 \bnfor
    \letrecin{f \, x = c_1} c_2 \bnfor \\
    &\ifthenelse{e}{c_1}{c_2} \bnfor
    \absurd{e} \bnfor
    \matchpair{e}{x, y}{c} \bnfor \\
    &\matchsum{e}{x}{c_1}{y}{c_2} \bnfor
    e_1 \, e_2 \bnfor \\
    &\new E \bnfor 
    \newwith{E}{e}{
      (\kpre{operation} \op_i \, x \, @ \, y \mapsto c_i)_i
    } \bnfor \\
    &\handle{c}{e}.
\end{align*}
%
An expression $e$ is promoted to a computation with $\val e$, but in the concrete syntax
$\kord{val}$ is omitted, as there is no distinction between expressions and computations.
%
The statement $\letin{x = c_1}{c_2}$ binds $x$ in $c_2$, and $\letrecin{f \, x =
  c_1}{c_2}$ defines a recursive function $f$ in $c_2$. The conditional statement and the
variations of $\kord{match}$ are elimination forms for booleans, the empty type, products,
and sums, respectively. Instance creation and the handling construct are discussed in Section~\ref{sec:eff-specific}.

Arithmetical expressions such as $e_1 + e_2$ count as computations because the arithmetical
operators are defined as built-in constants, so that $e_1 + e_2$ is parsed as a double
application. This allows us to uniformly treat all operations, irrespective
of whether they are pure or effectful (division by zero).

\section{Constructs specific to \eff}
\label{sec:eff-specific}

We explain the intuitive meaning of notions that are specific to \eff, namely
instances, operations, handlers, and resources.
We allow ourselves some slack in distinguishing syntax from semantics, which is treated in detail in Section~\ref{sec:semantics}.
It is helpful to think of a terminating computation as evaluating either to a value or an operation applied to a parameter.

\subsection{Instances and operations}
\label{sec:instances-operations}

The computation $\new E$ generates a fresh \emph{effect instance} of effect type~$E$.
For example, $\new \effect{ref}$ generates a new reference, $\new \effect{channel}$ a new communication channel, etc.
%
The extended form of $\kord{new}$ creates an effect instance with an associated \emph{resource},
which determines the default behaviour of operations and is explained separately in
Section~\ref{sec:resources}.

For each effect instance $e$ of effect type $E$ and an operation symbol $\op \in E$ there
is an \emph{operation} $\hash{e}{\op}$, also known as a \emph{generic
  effect}~\cite{plotkin03algebraic}. By itself, an operation is a value, and hence
effect-free, but an applied operation $\hash{e}{\op}\,e'$ is a computational effect
whose ramifications are determined by enveloping handlers and the resource associated with~$e$.

\subsection{Handlers}
\label{sec:handlers}

A handler
%
\begin{equation*}
 h =
 \handler
 (\hash{e_i}{\op_i} \, x \, k \mapsto c_i)_i \case \val x \mapsto c_v \case \fin x \mapsto c_f
\end{equation*}
%
may be applied to a computation $c$ with the handling construct
%
\begin{equation}
  \label{eq:handling}
   \handle{c}{h},
\end{equation}
%
which works as follows (we ignore the $\fin$ clause for the moment):
%
\begin{enumerate}
\item If $c$ evaluates to $\val e$, \eqref{eq:handling} evaluates to $c_v$ with $x$ bound to $e$.
\item If the evaluation of $c$ encounters an operation $\hash{e_i}{\op_i} \, e$,
  \eqref{eq:handling} evaluates to $c_i$ with $x$ bound to $e$ and $k$ bound to the
  continuation of $\hash{e_i}{\op_i} \, e$, i.e., whatever remains to be computed after the
  operation. The continuation is delimited by~\eqref{eq:handling} and is handled by~$h$ as
  well.
\end{enumerate}
%
The $\kord{finally}$ clause can be thought of as an outer wrapper which performs an
additional transformation, so that \eqref{eq:handling} is equivalent to
%
\begin{equation*}
  \letin{x = (\handle{c}{h'})}{c_f}  
\end{equation*}
%
where $h'$ is like $h$ without the $\kord{finally}$ clause. Such a wrapper is useful
because we often perform the same transformation every time a given handler is applied.
For example, the handler for state handles a computation by transforming it to a function
accepting the state, and $\kord{finally}$ applies the function to the initial state, see
Section~\ref{sec:state}.

If the evaluation of $c$ encounters an operation $\hash{e}{\op} \, e'$ that is not listed
in $h$, the control propagates to outer handling constructs, and eventually to the
toplevel, where the behaviour is determined by the resource associated with $e$.

\subsection{Resources}
\label{sec:resources}

The construct
%
\begin{equation*}
  \newwith{E}{e}{(\kpre{operation} \op_i \, x \, @ \, y \mapsto c_i)_i}
\end{equation*}
%
creates an instance $n$ with an associated \emph{resource}, inspired by coalgebraic
semantics of computational
effects~\cite{power04from,plotkin08:_tensor_comod_model_operat_seman}. A resource carries
a state and prescribes the default behaviour of the operations $\hash{n}{\op_i}$.
%
The paradigmatic case of resources is the definition of ML-style references, see Section~\ref{sec:state}.

The initial resource state for $n$ is set to $e$.
When the toplevel evaluation encounters an operation
$\hash{n}{\op_i} \, e'$, it evaluates $c_i$ with $x$ bound to $e'$
and $y$ bound to the current resource state. The result must be a pair of values, the first of
which is passed to the continuation, and the second of which is the new resource state.
%
If $c_i$ evaluates to an operation rather than a pair of values, a runtime error is
reported, as there is no reasonable way of handling it.

In \eff the interaction with the real world is accomplished through built-in resources.
For example, there is a predefined channel instance $\kord{std}$ with operations
$\hash{\kord{std}}{\kord{read}}$ and $\hash{\kord{std}}{\kord{write}}$ whose associated
resource performs actual interaction with the standard input and the standard output.

\section{Type checking}
\label{sec:type-checking}

Types in \eff are like those of ML~\cite{milner97the-definition} in the sense that they do not capture any
information about computational effects.
%
There are two typing judgements, $\ctx \ente e \T A$ states that expression $e$ has type
$A$ in context $\ctx$, and $\ctx \entc c \T A$ does so for a computation~$c$.
As usual, a context $\Gamma$ is a list of
distinct variables with associated types. The standard typing rules for expressions are:
%%
\begin{mathpar}
  \infer
    {x \T A \in \ctx}
    {\ctx \ente x \T A}

  \infer
    {}
    {\ctx \ente n \T \intty}

  \infer
    {}
    {\ctx \ente \tru \T \boolty}

  \infer
    {}
    {\ctx \ente \fls \T \boolty}

  \infer
    {}
    {\ctx \ente \unt \T \unitty}

  \infer
    {\Gamma \ente e_1 : A \\
     \Gamma \ente e_2 : B}
    {\ctx \ente \pair{e_1, e_2} \T A \times B}

  \infer
    {\Gamma \ente e : A}
    {\Gamma \ente \Left{e} : A + B}

  \infer
    {\Gamma \ente e : B}
    {\Gamma \ente \Right{e} : A + B}

  \infer
    {\ctx, x \T A \entc c \T B}
    {\ctx \ente \fun{x \T A} c \T A \to B}
\end{mathpar}
%
We also have to include judgements that assign types to other built-in constants.
An operation receives a function type
%
\begin{mathpar}
  \infer
  {\ctx \ente e \T E \\
    \op \T A \to B \in E}
  {\ctx \ente \hash{e}{\op} \T A \to B}
\end{mathpar}
%
while a handler is typed by the somewhat complicated rule
%
\begin{equation*}
  \infer
  {\infer{\ctx \ente e_i \T E_i \\
    \op_i \T A_i \to B_i \in E_i \\\\
    \ctx, x \T A_i, k \T B_i \to B \entc c_i \T B}{} \\
     \ctx, x \T A \entc c_v \T B \\
     \ctx, x \T B \entc c_f \T C}
   {\ctx \ente (\handler
     (\hash{e_i}{\op_i} \, x \, k \mapsto c_i)_i \mid
     \val x \mapsto c_v \mid
     \fin x \mapsto c_f) \T A \hto C}
\end{equation*}
%
which states that a handler first handles a computation of type $A$ into
a computation of type $B$ according to the $\kord{val}$ and operation clauses,
after which the $\kord{finally}$ clause transforms it further into a computation of type $C$.

The typing rules for computations are familiar or expected. Promotions of expressions and $\kord{let}$ statements are typed by
%
\begin{mathpar}
  \infer
    {\ctx \ente e \T A}
    {\ctx \entc \val e \T A}

  \infer
    {\ctx \entc c_1 \T A \\
     \ctx, x \T A \entc c_2 \T B}
    {\ctx \entc \letin{x = c_1} c_2 \T B}

  \infer
    {\ctx, f \T A \to B, x \T A \entc c_1 \T B \\
     \ctx, f \T A \to B \entc c_2 \T C}
    {\ctx \entc \letrecin{f\,x = c_1} c_2 \T C}
\end{mathpar}
%
and various elimination forms are typed by
%
\begin{mathpar}
  \infer
    {\ctx \ente e \T \boolty \\
     \ctx \entc c_1 \T A \\
     \ctx \entc c_2 \T A}
    {\ctx \entc \ifthenelse{e}{c_1}{c_2} \T A}

  \infer
    {\ctx \ente e \T \emptyty}
    {\ctx \entc \absurd{e} \T A}
      
  \infer
    {\ctx \ente e \T A \times B \\
     \ctx, x \T A, y \T B \entc c \T C}
    {\ctx \entc \matchpair{e}{x,y}{c} \T C}

  \infer
    {\ctx \ente e \T A + B \\
     \ctx, x \T A \entc c_1 \T C \\
     \ctx, y \T B \entc c_2 \T C}
    {\ctx \entc \matchsum{e}{x}{c_1}{y}{c_2} \T C}

  \infer
    {\ctx \ente e_1 \T A \to B \\
     \ctx \ente e_2 \T A}
    {\ctx \entc e_1 \, e_2 \T B}
\end{mathpar}
%
The instance creation is typed by the rules
%
\begin{mathpar}
  \infer
    {}
    {\ctx \entc \new E \T E}

  \infer
    {\ctx \ente e : C \\
     \op_i \T A_i \to B_i \in E \\
     \ctx, x \T A_i, y : C \entc c_i : B_i \times C}
    {\ctx \entc \newwith{E}{e}{(\kpre{operation} \op_i \, x \, @ \, y \mapsto c_i)_i} \T E}
\end{mathpar}
%
The rule for the simple form is obvious, while the one for the extended form checks that
the initial state $e$ has type $C$ and that, for each operation $\op_i \T A_i \to B_i \in
E$, the corresponding computation $c_i$ evaluates to a pair of type $B_i \times C$.

Finally, the rule for handling expresses the fact that handlers are like functions:
%
\begin{mathpar}
    \infer
    {\ctx \entc c \T A \\
     \ctx \ente e \T A \hto B}
    {\ctx \entc \handle{c}{e} \T B}
\end{mathpar}


%%% Local Variables:
%%% mode: latex
%%% TeX-master: "eff"
%%% End:
