\haddockmoduleheading{Foreign.C.Error}
\label{module:Foreign.C.Error}
\haddockbeginheader
{\haddockverb\begin{verbatim}
module Foreign.C.Error (
    Errno(Errno),  eOK,  e2BIG,  eACCES,  eADDRINUSE,  eADDRNOTAVAIL,  eADV, 
    eAFNOSUPPORT,  eAGAIN,  eALREADY,  eBADF,  eBADMSG,  eBADRPC,  eBUSY, 
    eCHILD,  eCOMM,  eCONNABORTED,  eCONNREFUSED,  eCONNRESET,  eDEADLK, 
    eDESTADDRREQ,  eDIRTY,  eDOM,  eDQUOT,  eEXIST,  eFAULT,  eFBIG,  eFTYPE, 
    eHOSTDOWN,  eHOSTUNREACH,  eIDRM,  eILSEQ,  eINPROGRESS,  eINTR,  eINVAL, 
    eIO,  eISCONN,  eISDIR,  eLOOP,  eMFILE,  eMLINK,  eMSGSIZE,  eMULTIHOP, 
    eNAMETOOLONG,  eNETDOWN,  eNETRESET,  eNETUNREACH,  eNFILE,  eNOBUFS, 
    eNODATA,  eNODEV,  eNOENT,  eNOEXEC,  eNOLCK,  eNOLINK,  eNOMEM,  eNOMSG, 
    eNONET,  eNOPROTOOPT,  eNOSPC,  eNOSR,  eNOSTR,  eNOSYS,  eNOTBLK, 
    eNOTCONN,  eNOTDIR,  eNOTEMPTY,  eNOTSOCK,  eNOTTY,  eNXIO,  eOPNOTSUPP, 
    ePERM,  ePFNOSUPPORT,  ePIPE,  ePROCLIM,  ePROCUNAVAIL,  ePROGMISMATCH, 
    ePROGUNAVAIL,  ePROTO,  ePROTONOSUPPORT,  ePROTOTYPE,  eRANGE,  eREMCHG, 
    eREMOTE,  eROFS,  eRPCMISMATCH,  eRREMOTE,  eSHUTDOWN,  eSOCKTNOSUPPORT, 
    eSPIPE,  eSRCH,  eSRMNT,  eSTALE,  eTIME,  eTIMEDOUT,  eTOOMANYREFS, 
    eTXTBSY,  eUSERS,  eWOULDBLOCK,  eXDEV,  isValidErrno,  getErrno, 
    resetErrno,  errnoToIOError,  throwErrno,  throwErrnoIf,  throwErrnoIf_, 
    throwErrnoIfRetry,  throwErrnoIfRetry_,  throwErrnoIfMinus1, 
    throwErrnoIfMinus1_,  throwErrnoIfMinus1Retry,  throwErrnoIfMinus1Retry_, 
    throwErrnoIfNull,  throwErrnoIfNullRetry,  throwErrnoIfRetryMayBlock, 
    throwErrnoIfRetryMayBlock_,  throwErrnoIfMinus1RetryMayBlock, 
    throwErrnoIfMinus1RetryMayBlock_,  throwErrnoIfNullRetryMayBlock, 
    throwErrnoPath,  throwErrnoPathIf,  throwErrnoPathIf_, 
    throwErrnoPathIfNull,  throwErrnoPathIfMinus1,  throwErrnoPathIfMinus1_
  ) where\end{verbatim}}
\haddockendheader

The module \haddocktt{Foreign.C.Error} facilitates C-specific error
 handling of \haddocktt{errno}.
\par

\section{Haskell representations of \haddocktt{errno} values
}
\begin{haddockdesc}
\item[\begin{tabular}{@{}l}
newtype\ Errno
\end{tabular}]\haddockbegindoc
\haddockbeginconstrs
\haddockdecltt{=} & \haddockdecltt{Errno CInt} & \\
\end{tabulary}\par
Haskell representation for \haddocktt{errno} values.
 The implementation is deliberately exposed, to allow users to add
 their own definitions of \haddockid{Errno} values.
\par

\end{haddockdesc}
\begin{haddockdesc}
\item[\begin{tabular}{@{}l}
instance\ Eq\ Errno
\end{tabular}]
\end{haddockdesc}
\subsection{Common \haddocktt{errno} symbols
}
Different operating systems and/or C libraries often support
 different values of \haddocktt{errno}.  This module defines the common values,
 but due to the open definition of \haddockid{Errno} users may add definitions
 which are not predefined.
\par

\begin{haddockdesc}
\item[
eOK\ ::\ Errno
]
\item[
e2BIG\ ::\ Errno
]
\item[
eACCES\ ::\ Errno
]
\item[
eADDRINUSE\ ::\ Errno
]
\item[
eADDRNOTAVAIL\ ::\ Errno
]
\item[
eADV\ ::\ Errno
]
\item[
eAFNOSUPPORT\ ::\ Errno
]
\item[
eAGAIN\ ::\ Errno
]
\item[
eALREADY\ ::\ Errno
]
\item[
eBADF\ ::\ Errno
]
\item[
eBADMSG\ ::\ Errno
]
\item[
eBADRPC\ ::\ Errno
]
\item[
eBUSY\ ::\ Errno
]
\item[
eCHILD\ ::\ Errno
]
\item[
eCOMM\ ::\ Errno
]
\item[
eCONNABORTED\ ::\ Errno
]
\item[
eCONNREFUSED\ ::\ Errno
]
\item[
eCONNRESET\ ::\ Errno
]
\item[
eDEADLK\ ::\ Errno
]
\item[
eDESTADDRREQ\ ::\ Errno
]
\item[
eDIRTY\ ::\ Errno
]
\item[
eDOM\ ::\ Errno
]
\item[
eDQUOT\ ::\ Errno
]
\item[
eEXIST\ ::\ Errno
]
\item[
eFAULT\ ::\ Errno
]
\item[
eFBIG\ ::\ Errno
]
\item[
eFTYPE\ ::\ Errno
]
\item[
eHOSTDOWN\ ::\ Errno
]
\item[
eHOSTUNREACH\ ::\ Errno
]
\item[
eIDRM\ ::\ Errno
]
\item[
eILSEQ\ ::\ Errno
]
\item[
eINPROGRESS\ ::\ Errno
]
\item[
eINTR\ ::\ Errno
]
\item[
eINVAL\ ::\ Errno
]
\item[
eIO\ ::\ Errno
]
\item[
eISCONN\ ::\ Errno
]
\item[
eISDIR\ ::\ Errno
]
\item[
eLOOP\ ::\ Errno
]
\item[
eMFILE\ ::\ Errno
]
\item[
eMLINK\ ::\ Errno
]
\item[
eMSGSIZE\ ::\ Errno
]
\item[
eMULTIHOP\ ::\ Errno
]
\item[
eNAMETOOLONG\ ::\ Errno
]
\item[
eNETDOWN\ ::\ Errno
]
\item[
eNETRESET\ ::\ Errno
]
\item[
eNETUNREACH\ ::\ Errno
]
\item[
eNFILE\ ::\ Errno
]
\item[
eNOBUFS\ ::\ Errno
]
\item[
eNODATA\ ::\ Errno
]
\item[
eNODEV\ ::\ Errno
]
\item[
eNOENT\ ::\ Errno
]
\item[
eNOEXEC\ ::\ Errno
]
\item[
eNOLCK\ ::\ Errno
]
\item[
eNOLINK\ ::\ Errno
]
\item[
eNOMEM\ ::\ Errno
]
\item[
eNOMSG\ ::\ Errno
]
\item[
eNONET\ ::\ Errno
]
\item[
eNOPROTOOPT\ ::\ Errno
]
\item[
eNOSPC\ ::\ Errno
]
\item[
eNOSR\ ::\ Errno
]
\item[
eNOSTR\ ::\ Errno
]
\item[
eNOSYS\ ::\ Errno
]
\item[
eNOTBLK\ ::\ Errno
]
\item[
eNOTCONN\ ::\ Errno
]
\item[
eNOTDIR\ ::\ Errno
]
\item[
eNOTEMPTY\ ::\ Errno
]
\item[
eNOTSOCK\ ::\ Errno
]
\item[
eNOTTY\ ::\ Errno
]
\item[
eNXIO\ ::\ Errno
]
\item[
eOPNOTSUPP\ ::\ Errno
]
\item[
ePERM\ ::\ Errno
]
\item[
ePFNOSUPPORT\ ::\ Errno
]
\item[
ePIPE\ ::\ Errno
]
\item[
ePROCLIM\ ::\ Errno
]
\item[
ePROCUNAVAIL\ ::\ Errno
]
\item[
ePROGMISMATCH\ ::\ Errno
]
\item[
ePROGUNAVAIL\ ::\ Errno
]
\item[
ePROTO\ ::\ Errno
]
\item[
ePROTONOSUPPORT\ ::\ Errno
]
\item[
ePROTOTYPE\ ::\ Errno
]
\item[
eRANGE\ ::\ Errno
]
\item[
eREMCHG\ ::\ Errno
]
\item[
eREMOTE\ ::\ Errno
]
\item[
eROFS\ ::\ Errno
]
\item[
eRPCMISMATCH\ ::\ Errno
]
\item[
eRREMOTE\ ::\ Errno
]
\item[
eSHUTDOWN\ ::\ Errno
]
\item[
eSOCKTNOSUPPORT\ ::\ Errno
]
\item[
eSPIPE\ ::\ Errno
]
\item[
eSRCH\ ::\ Errno
]
\item[
eSRMNT\ ::\ Errno
]
\item[
eSTALE\ ::\ Errno
]
\item[
eTIME\ ::\ Errno
]
\item[
eTIMEDOUT\ ::\ Errno
]
\item[
eTOOMANYREFS\ ::\ Errno
]
\item[
eTXTBSY\ ::\ Errno
]
\item[
eUSERS\ ::\ Errno
]
\item[
eWOULDBLOCK\ ::\ Errno
]
\item[
eXDEV\ ::\ Errno
]
\end{haddockdesc}
\subsection{\haddockid{Errno} functions
}
\begin{haddockdesc}
\item[\begin{tabular}{@{}l}
isValidErrno\ ::\ Errno\ ->\ Bool
\end{tabular}]\haddockbegindoc
Yield \haddockid{True} if the given \haddockid{Errno} value is valid on the system.
 This implies that the \haddockid{Eq} instance of \haddockid{Errno} is also system dependent
 as it is only defined for valid values of \haddockid{Errno}.
\par

\end{haddockdesc}
\begin{haddockdesc}
\item[\begin{tabular}{@{}l}
getErrno\ ::\ IO\ Errno
\end{tabular}]\haddockbegindoc
Get the current value of \haddocktt{errno} in the current thread.
\par

\end{haddockdesc}
\begin{haddockdesc}
\item[\begin{tabular}{@{}l}
resetErrno\ ::\ IO\ ()
\end{tabular}]\haddockbegindoc
Reset the current thread's \haddocktt{errno} value to \haddockid{eOK}.
\par

\end{haddockdesc}
\begin{haddockdesc}
\item[\begin{tabular}{@{}l}
errnoToIOError
\end{tabular}]\haddockbegindoc
\haddockbeginargs
\haddockdecltt{::} & \haddockdecltt{String} & the location where the error occurred
 \\
                                              \haddockdecltt{->} & \haddockdecltt{Errno} & the error number
 \\
                                                                                           \haddockdecltt{->} & \haddockdecltt{Maybe Handle} & optional handle associated with the error
 \\
                                                                                                                                               \haddockdecltt{->} & \haddockdecltt{Maybe String} & optional filename associated with the error
 \\
                                                                                                                                                                                                   \haddockdecltt{->} & \haddockdecltt{IOError} & \\
\end{tabulary}\par
Construct an \haddockid{IOError} based on the given \haddockid{Errno} value.
 The optional information can be used to improve the accuracy of
 error messages.
\par

\end{haddockdesc}
\begin{haddockdesc}
\item[\begin{tabular}{@{}l}
throwErrno
\end{tabular}]\haddockbegindoc
\haddockbeginargs
\haddockdecltt{::} & \haddockdecltt{String} & textual description of the error location
 \\
                                              \haddockdecltt{->} & \haddockdecltt{IO a} & \\
\end{tabulary}\par
Throw an \haddockid{IOError} corresponding to the current value of \haddockid{getErrno}.
\par

\end{haddockdesc}
\subsection{Guards for IO operations that may fail
}
\begin{haddockdesc}
\item[\begin{tabular}{@{}l}
throwErrnoIf
\end{tabular}]\haddockbegindoc
\haddockbeginargs
\haddockdecltt{::} & \haddockdecltt{(a
                                     -> Bool)} & predicate to apply to the result value
 of the \haddockid{IO} operation
 \\
                                                 \haddockdecltt{->} & \haddockdecltt{String} & textual description of the location
 \\
                                                                                               \haddockdecltt{->} & \haddockdecltt{IO a} & the \haddockid{IO} operation to be executed
 \\
                                                                                                                                           \haddockdecltt{->} & \haddockdecltt{IO a} & \\
\end{tabulary}\par
Throw an \haddockid{IOError} corresponding to the current value of \haddockid{getErrno}
 if the result value of the \haddockid{IO} action meets the given predicate.
\par

\end{haddockdesc}
\begin{haddockdesc}
\item[\begin{tabular}{@{}l}
throwErrnoIf{\char '137}\ ::\ (a\ ->\ Bool)\ ->\ String\ ->\ IO\ a\ ->\ IO\ ()
\end{tabular}]\haddockbegindoc
as \haddockid{throwErrnoIf}, but discards the result of the \haddockid{IO} action after
 error handling.
\par

\end{haddockdesc}
\begin{haddockdesc}
\item[\begin{tabular}{@{}l}
throwErrnoIfRetry\ ::\ (a\ ->\ Bool)\ ->\ String\ ->\ IO\ a\ ->\ IO\ a
\end{tabular}]\haddockbegindoc
as \haddockid{throwErrnoIf}, but retry the \haddockid{IO} action when it yields the
 error code \haddockid{eINTR} - this amounts to the standard retry loop for
 interrupted POSIX system calls.
\par

\end{haddockdesc}
\begin{haddockdesc}
\item[\begin{tabular}{@{}l}
throwErrnoIfRetry{\char '137}\ ::\ (a\ ->\ Bool)\ ->\ String\ ->\ IO\ a\ ->\ IO\ ()
\end{tabular}]\haddockbegindoc
as \haddockid{throwErrnoIfRetry}, but discards the result.
\par

\end{haddockdesc}
\begin{haddockdesc}
\item[\begin{tabular}{@{}l}
throwErrnoIfMinus1\ ::\ Num\ a\ =>\ String\ ->\ IO\ a\ ->\ IO\ a
\end{tabular}]\haddockbegindoc
Throw an \haddockid{IOError} corresponding to the current value of \haddockid{getErrno}
 if the \haddockid{IO} action returns a result of \haddocktt{-1}.
\par

\end{haddockdesc}
\begin{haddockdesc}
\item[\begin{tabular}{@{}l}
throwErrnoIfMinus1{\char '137}\ ::\ Num\ a\ =>\ String\ ->\ IO\ a\ ->\ IO\ ()
\end{tabular}]\haddockbegindoc
as \haddockid{throwErrnoIfMinus1}, but discards the result.
\par

\end{haddockdesc}
\begin{haddockdesc}
\item[\begin{tabular}{@{}l}
throwErrnoIfMinus1Retry\ ::\ Num\ a\ =>\ String\ ->\ IO\ a\ ->\ IO\ a
\end{tabular}]\haddockbegindoc
Throw an \haddockid{IOError} corresponding to the current value of \haddockid{getErrno}
 if the \haddockid{IO} action returns a result of \haddocktt{-1}, but retries in case of
 an interrupted operation.
\par

\end{haddockdesc}
\begin{haddockdesc}
\item[\begin{tabular}{@{}l}
throwErrnoIfMinus1Retry{\char '137}\ ::\ Num\ a\ =>\ String\ ->\ IO\ a\ ->\ IO\ ()
\end{tabular}]\haddockbegindoc
as \haddockid{throwErrnoIfMinus1}, but discards the result.
\par

\end{haddockdesc}
\begin{haddockdesc}
\item[\begin{tabular}{@{}l}
throwErrnoIfNull\ ::\ String\ ->\ IO\ (Ptr\ a)\ ->\ IO\ (Ptr\ a)
\end{tabular}]\haddockbegindoc
Throw an \haddockid{IOError} corresponding to the current value of \haddockid{getErrno}
 if the \haddockid{IO} action returns \haddockid{nullPtr}.
\par

\end{haddockdesc}
\begin{haddockdesc}
\item[\begin{tabular}{@{}l}
throwErrnoIfNullRetry\ ::\ String\ ->\ IO\ (Ptr\ a)\ ->\ IO\ (Ptr\ a)
\end{tabular}]\haddockbegindoc
Throw an \haddockid{IOError} corresponding to the current value of \haddockid{getErrno}
 if the \haddockid{IO} action returns \haddockid{nullPtr},
 but retry in case of an interrupted operation.
\par

\end{haddockdesc}
\begin{haddockdesc}
\item[\begin{tabular}{@{}l}
throwErrnoIfRetryMayBlock
\end{tabular}]\haddockbegindoc
\haddockbeginargs
\haddockdecltt{::} & \haddockdecltt{(a
                                     -> Bool)} & predicate to apply to the result value
 of the \haddockid{IO} operation
 \\
                                                 \haddockdecltt{->} & \haddockdecltt{String} & textual description of the location
 \\
                                                                                               \haddockdecltt{->} & \haddockdecltt{IO a} & the \haddockid{IO} operation to be executed
 \\
                                                                                                                                           \haddockdecltt{->} & \haddockdecltt{IO b} & action to execute before retrying if
 an immediate retry would block
 \\
                                                                                                                                                                                       \haddockdecltt{->} & \haddockdecltt{IO a} & \\
\end{tabulary}\par
as \haddockid{throwErrnoIfRetry}, but additionally if the operation 
 yields the error code \haddockid{eAGAIN} or \haddockid{eWOULDBLOCK}, an alternative
 action is executed before retrying.
\par

\end{haddockdesc}
\begin{haddockdesc}
\item[\begin{tabular}{@{}l}
throwErrnoIfRetryMayBlock{\char '137}\ ::\ (a\ ->\ Bool)\\\ \ \ \ \ \ \ \ \ \ \ \ \ \ \ \ \ \ \ \ \ \ \ \ \ \ \ \ \ \ ->\ String\ ->\ IO\ a\ ->\ IO\ b\ ->\ IO\ ()
\end{tabular}]\haddockbegindoc
as \haddockid{throwErrnoIfRetryMayBlock}, but discards the result.
\par

\end{haddockdesc}
\begin{haddockdesc}
\item[\begin{tabular}{@{}l}
throwErrnoIfMinus1RetryMayBlock\ ::\ Num\ a\ =>\ String\\\ \ \ \ \ \ \ \ \ \ \ \ \ \ \ \ \ \ \ \ \ \ \ \ \ \ \ \ \ \ \ \ \ \ \ \ \ \ \ \ \ \ \ \ ->\ IO\ a\ ->\ IO\ b\ ->\ IO\ a
\end{tabular}]\haddockbegindoc
as \haddockid{throwErrnoIfMinus1Retry}, but checks for operations that would block.
\par

\end{haddockdesc}
\begin{haddockdesc}
\item[\begin{tabular}{@{}l}
throwErrnoIfMinus1RetryMayBlock{\char '137}\ ::\ Num\ a\ =>\ String\\\ \ \ \ \ \ \ \ \ \ \ \ \ \ \ \ \ \ \ \ \ \ \ \ \ \ \ \ \ \ \ \ \ \ \ \ \ \ \ \ \ \ \ \ \ ->\ IO\ a\ ->\ IO\ b\ ->\ IO\ ()
\end{tabular}]\haddockbegindoc
as \haddockid{throwErrnoIfMinus1RetryMayBlock}, but discards the result.
\par

\end{haddockdesc}
\begin{haddockdesc}
\item[\begin{tabular}{@{}l}
throwErrnoIfNullRetryMayBlock\ ::\ String\\\ \ \ \ \ \ \ \ \ \ \ \ \ \ \ \ \ \ \ \ \ \ \ \ \ \ \ \ \ \ \ \ \ ->\ IO\ (Ptr\ a)\ ->\ IO\ b\ ->\ IO\ (Ptr\ a)
\end{tabular}]\haddockbegindoc
as \haddockid{throwErrnoIfNullRetry}, but checks for operations that would block.
\par

\end{haddockdesc}
\begin{haddockdesc}
\item[\begin{tabular}{@{}l}
throwErrnoPath\ ::\ String\ ->\ FilePath\ ->\ IO\ a
\end{tabular}]\haddockbegindoc
as \haddockid{throwErrno}, but exceptions include the given path when appropriate.
\par

\end{haddockdesc}
\begin{haddockdesc}
\item[\begin{tabular}{@{}l}
throwErrnoPathIf\ ::\ (a\ ->\ Bool)\\\ \ \ \ \ \ \ \ \ \ \ \ \ \ \ \ \ \ \ \ ->\ String\ ->\ FilePath\ ->\ IO\ a\ ->\ IO\ a
\end{tabular}]\haddockbegindoc
as \haddockid{throwErrnoIf}, but exceptions include the given path when
   appropriate.
\par

\end{haddockdesc}
\begin{haddockdesc}
\item[\begin{tabular}{@{}l}
throwErrnoPathIf{\char '137}\ ::\ (a\ ->\ Bool)\\\ \ \ \ \ \ \ \ \ \ \ \ \ \ \ \ \ \ \ \ \ ->\ String\ ->\ FilePath\ ->\ IO\ a\ ->\ IO\ ()
\end{tabular}]\haddockbegindoc
as \haddockid{throwErrnoIf{\char '137}}, but exceptions include the given path when
   appropriate.
\par

\end{haddockdesc}
\begin{haddockdesc}
\item[\begin{tabular}{@{}l}
throwErrnoPathIfNull\ ::\ String\\\ \ \ \ \ \ \ \ \ \ \ \ \ \ \ \ \ \ \ \ \ \ \ \ ->\ FilePath\ ->\ IO\ (Ptr\ a)\ ->\ IO\ (Ptr\ a)
\end{tabular}]\haddockbegindoc
as \haddockid{throwErrnoIfNull}, but exceptions include the given path when
   appropriate.
\par

\end{haddockdesc}
\begin{haddockdesc}
\item[\begin{tabular}{@{}l}
throwErrnoPathIfMinus1\ ::\ Num\ a\ =>\ String\\\ \ \ \ \ \ \ \ \ \ \ \ \ \ \ \ \ \ \ \ \ \ \ \ \ \ \ \ \ \ \ \ \ \ \ ->\ FilePath\ ->\ IO\ a\ ->\ IO\ a
\end{tabular}]\haddockbegindoc
as \haddockid{throwErrnoIfMinus1}, but exceptions include the given path when
   appropriate.
\par

\end{haddockdesc}
\begin{haddockdesc}
\item[\begin{tabular}{@{}l}
throwErrnoPathIfMinus1{\char '137}\ ::\ Num\ a\ =>\ String\\\ \ \ \ \ \ \ \ \ \ \ \ \ \ \ \ \ \ \ \ \ \ \ \ \ \ \ \ \ \ \ \ \ \ \ \ ->\ FilePath\ ->\ IO\ a\ ->\ IO\ ()
\end{tabular}]\haddockbegindoc
as \haddockid{throwErrnoIfMinus1{\char '137}}, but exceptions include the given path when
   appropriate.
\par

\end{haddockdesc}