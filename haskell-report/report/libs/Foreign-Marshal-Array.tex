\haddockmoduleheading{Foreign.Marshal.Array}
\label{module:Foreign.Marshal.Array}
\haddockbeginheader
{\haddockverb\begin{verbatim}
module Foreign.Marshal.Array (
    mallocArray,  mallocArray0,  allocaArray,  allocaArray0,  reallocArray, 
    reallocArray0,  peekArray,  peekArray0,  pokeArray,  pokeArray0,  newArray, 
    newArray0,  withArray,  withArray0,  withArrayLen,  withArrayLen0, 
    copyArray,  moveArray,  lengthArray0,  advancePtr
  ) where\end{verbatim}}
\haddockendheader

The module \haddocktt{Foreign.Marshal.Array} provides operations for marshalling Haskell
lists into monolithic arrays and vice versa.  Most functions come in two
flavours: one for arrays terminated by a special termination element and one
where an explicit length parameter is used to determine the extent of an
array.  The typical example for the former case are C's NUL terminated
strings.  However, please note that C strings should usually be marshalled
using the functions provided by \haddocktt{Foreign.C.String} as
the Unicode encoding has to be taken into account.  All functions specifically
operating on arrays that are terminated by a special termination element have
a name ending on \haddocktt{0}---e.g., \haddockid{mallocArray} allocates space for an
array of the given size, whereas \haddockid{mallocArray0} allocates space for one
more element to ensure that there is room for the terminator.
\par

\section{Marshalling arrays
}
\subsection{Allocation
}
\begin{haddockdesc}
\item[\begin{tabular}{@{}l}
mallocArray\ ::\ Storable\ a\ =>\ Int\ ->\ IO\ (Ptr\ a)
\end{tabular}]\haddockbegindoc
Allocate storage for the given number of elements of a storable type
 (like \haddocktt{Foreign.Marshal.Alloc.malloc}, but for multiple elements).
\par

\end{haddockdesc}
\begin{haddockdesc}
\item[\begin{tabular}{@{}l}
mallocArray0\ ::\ Storable\ a\ =>\ Int\ ->\ IO\ (Ptr\ a)
\end{tabular}]\haddockbegindoc
Like \haddockid{mallocArray}, but add an extra position to hold a special
 termination element.
\par

\end{haddockdesc}
\begin{haddockdesc}
\item[\begin{tabular}{@{}l}
allocaArray\ ::\ Storable\ a\ =>\ Int\ ->\ (Ptr\ a\ ->\ IO\ b)\ ->\ IO\ b
\end{tabular}]\haddockbegindoc
Temporarily allocate space for the given number of elements
 (like \haddocktt{Foreign.Marshal.Alloc.alloca}, but for multiple elements).
\par

\end{haddockdesc}
\begin{haddockdesc}
\item[\begin{tabular}{@{}l}
allocaArray0\ ::\ Storable\ a\ =>\ Int\ ->\ (Ptr\ a\ ->\ IO\ b)\ ->\ IO\ b
\end{tabular}]\haddockbegindoc
Like \haddockid{allocaArray}, but add an extra position to hold a special
 termination element.
\par

\end{haddockdesc}
\begin{haddockdesc}
\item[\begin{tabular}{@{}l}
reallocArray\ ::\ Storable\ a\ =>\ Ptr\ a\ ->\ Int\ ->\ IO\ (Ptr\ a)
\end{tabular}]\haddockbegindoc
Adjust the size of an array
\par

\end{haddockdesc}
\begin{haddockdesc}
\item[\begin{tabular}{@{}l}
reallocArray0\ ::\ Storable\ a\ =>\ Ptr\ a\ ->\ Int\ ->\ IO\ (Ptr\ a)
\end{tabular}]\haddockbegindoc
Adjust the size of an array including an extra position for the end marker.
\par

\end{haddockdesc}
\subsection{Marshalling
}
\begin{haddockdesc}
\item[\begin{tabular}{@{}l}
peekArray\ ::\ Storable\ a\ =>\ Int\ ->\ Ptr\ a\ ->\ IO\ {\char 91}a{\char 93}
\end{tabular}]\haddockbegindoc
Convert an array of given length into a Haskell list.
\par

\end{haddockdesc}
\begin{haddockdesc}
\item[\begin{tabular}{@{}l}
peekArray0\ ::\ (Storable\ a,\ Eq\ a)\ =>\ a\ ->\ Ptr\ a\ ->\ IO\ {\char 91}a{\char 93}
\end{tabular}]\haddockbegindoc
Convert an array terminated by the given end marker into a Haskell list
\par

\end{haddockdesc}
\begin{haddockdesc}
\item[\begin{tabular}{@{}l}
pokeArray\ ::\ Storable\ a\ =>\ Ptr\ a\ ->\ {\char 91}a{\char 93}\ ->\ IO\ ()
\end{tabular}]\haddockbegindoc
Write the list elements consecutive into memory
\par

\end{haddockdesc}
\begin{haddockdesc}
\item[\begin{tabular}{@{}l}
pokeArray0\ ::\ Storable\ a\ =>\ a\ ->\ Ptr\ a\ ->\ {\char 91}a{\char 93}\ ->\ IO\ ()
\end{tabular}]\haddockbegindoc
Write the list elements consecutive into memory and terminate them with the
 given marker element
\par

\end{haddockdesc}
\subsection{Combined allocation and marshalling
}
\begin{haddockdesc}
\item[\begin{tabular}{@{}l}
newArray\ ::\ Storable\ a\ =>\ {\char 91}a{\char 93}\ ->\ IO\ (Ptr\ a)
\end{tabular}]\haddockbegindoc
Write a list of storable elements into a newly allocated, consecutive
 sequence of storable values
 (like \haddocktt{Foreign.Marshal.Utils.new}, but for multiple elements).
\par

\end{haddockdesc}
\begin{haddockdesc}
\item[\begin{tabular}{@{}l}
newArray0\ ::\ Storable\ a\ =>\ a\ ->\ {\char 91}a{\char 93}\ ->\ IO\ (Ptr\ a)
\end{tabular}]\haddockbegindoc
Write a list of storable elements into a newly allocated, consecutive
 sequence of storable values, where the end is fixed by the given end marker
\par

\end{haddockdesc}
\begin{haddockdesc}
\item[\begin{tabular}{@{}l}
withArray\ ::\ Storable\ a\ =>\ {\char 91}a{\char 93}\ ->\ (Ptr\ a\ ->\ IO\ b)\ ->\ IO\ b
\end{tabular}]\haddockbegindoc
Temporarily store a list of storable values in memory
 (like \haddocktt{Foreign.Marshal.Utils.with}, but for multiple elements).
\par

\end{haddockdesc}
\begin{haddockdesc}
\item[\begin{tabular}{@{}l}
withArray0\ ::\ Storable\ a\ =>\ a\ ->\ {\char 91}a{\char 93}\ ->\ (Ptr\ a\ ->\ IO\ b)\ ->\ IO\ b
\end{tabular}]\haddockbegindoc
Like \haddockid{withArray}, but a terminator indicates where the array ends
\par

\end{haddockdesc}
\begin{haddockdesc}
\item[\begin{tabular}{@{}l}
withArrayLen\ ::\ Storable\ a\ =>\ {\char 91}a{\char 93}\ ->\ (Int\ ->\ Ptr\ a\ ->\ IO\ b)\ ->\ IO\ b
\end{tabular}]\haddockbegindoc
Like \haddockid{withArray}, but the action gets the number of values
 as an additional parameter
\par

\end{haddockdesc}
\begin{haddockdesc}
\item[\begin{tabular}{@{}l}
withArrayLen0\ ::\ Storable\ a\ =>\ a\\\ \ \ \ \ \ \ \ \ \ \ \ \ \ \ \ \ \ \ \ \ \ \ \ \ \ \ \ \ \ \ ->\ {\char 91}a{\char 93}\ ->\ (Int\ ->\ Ptr\ a\ ->\ IO\ b)\ ->\ IO\ b
\end{tabular}]\haddockbegindoc
Like \haddockid{withArrayLen}, but a terminator indicates where the array ends
\par

\end{haddockdesc}
\subsection{Copying
}
(argument order: destination, source)
\par

\begin{haddockdesc}
\item[\begin{tabular}{@{}l}
copyArray\ ::\ Storable\ a\ =>\ Ptr\ a\ ->\ Ptr\ a\ ->\ Int\ ->\ IO\ ()
\end{tabular}]\haddockbegindoc
Copy the given number of elements from the second array (source) into the
 first array (destination); the copied areas may \emph{not} overlap
\par

\end{haddockdesc}
\begin{haddockdesc}
\item[\begin{tabular}{@{}l}
moveArray\ ::\ Storable\ a\ =>\ Ptr\ a\ ->\ Ptr\ a\ ->\ Int\ ->\ IO\ ()
\end{tabular}]\haddockbegindoc
Copy the given number of elements from the second array (source) into the
 first array (destination); the copied areas \emph{may} overlap
\par

\end{haddockdesc}
\subsection{Finding the length
}
\begin{haddockdesc}
\item[\begin{tabular}{@{}l}
lengthArray0\ ::\ (Storable\ a,\ Eq\ a)\ =>\ a\ ->\ Ptr\ a\ ->\ IO\ Int
\end{tabular}]\haddockbegindoc
Return the number of elements in an array, excluding the terminator
\par

\end{haddockdesc}
\subsection{Indexing
}
\begin{haddockdesc}
\item[\begin{tabular}{@{}l}
advancePtr\ ::\ Storable\ a\ =>\ Ptr\ a\ ->\ Int\ ->\ Ptr\ a
\end{tabular}]\haddockbegindoc
Advance a pointer into an array by the given number of elements
\par

\end{haddockdesc}