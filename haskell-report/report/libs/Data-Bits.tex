\haddockmoduleheading{Data.Bits}
\label{module:Data.Bits}
\haddockbeginheader
{\haddockverb\begin{verbatim}
module Data.Bits (
    Bits((.&.),
         (.|.),
         xor,
         complement,
         shift,
         rotate,
         bit,
         setBit,
         clearBit,
         complementBit,
         testBit,
         bitSize,
         isSigned,
         shiftL,
         shiftR,
         rotateL,
         rotateR)
  ) where\end{verbatim}}
\haddockendheader

This module defines bitwise operations for signed and unsigned
 integers.
\par

\begin{haddockdesc}
\item[\begin{tabular}{@{}l}
class\ Num\ a\ =>\ Bits\ a\ where
\end{tabular}]\haddockbegindoc
The \haddockid{Bits} class defines bitwise operations over integral types.
\par
\begin{itemize}
\item
 Bits are numbered from 0 with bit 0 being the least
  significant bit.
\par

\end{itemize}
Minimal complete definition: \haddockid{.{\char '46}.}, \haddockid{.|.}, \haddockid{xor}, \haddockid{complement},
(\haddockid{shift} or (\haddockid{shiftL} and \haddockid{shiftR})), (\haddockid{rotate} or (\haddockid{rotateL} and \haddockid{rotateR})),
\haddockid{bitSize} and \haddockid{isSigned}.
\par

\haddockpremethods{}\textbf{Methods}
\begin{haddockdesc}
\item[\begin{tabular}{@{}l}
(.{\char '46}.)\ ::\ a\ ->\ a\ ->\ a
\end{tabular}]\haddockbegindoc
Bitwise "and"
\par

\end{haddockdesc}
\begin{haddockdesc}
\item[\begin{tabular}{@{}l}
(.|.)\ ::\ a\ ->\ a\ ->\ a
\end{tabular}]\haddockbegindoc
Bitwise "or"
\par

\end{haddockdesc}
\begin{haddockdesc}
\item[\begin{tabular}{@{}l}
xor\ ::\ a\ ->\ a\ ->\ a
\end{tabular}]\haddockbegindoc
Bitwise "xor"
\par

\end{haddockdesc}
\begin{haddockdesc}
\item[\begin{tabular}{@{}l}
complement\ ::\ a\ ->\ a
\end{tabular}]\haddockbegindoc
Reverse all the bits in the argument 
\par

\end{haddockdesc}
\begin{haddockdesc}
\item[\begin{tabular}{@{}l}
shift\ ::\ a\ ->\ Int\ ->\ a
\end{tabular}]\haddockbegindoc
\haddocktt{shift\ x\ i} shifts \haddocktt{x} left by \haddocktt{i} bits if \haddocktt{i} is positive,
        or right by \haddocktt{-i} bits otherwise.
        Right shifts perform sign extension on signed number types;
        i.e. they fill the top bits with 1 if the \haddocktt{x} is negative
        and with 0 otherwise.
\par
An instance can define either this unified \haddockid{shift} or \haddockid{shiftL} and
        \haddockid{shiftR}, depending on which is more convenient for the type in
        question. 
\par

\end{haddockdesc}
\begin{haddockdesc}
\item[\begin{tabular}{@{}l}
rotate\ ::\ a\ ->\ Int\ ->\ a
\end{tabular}]\haddockbegindoc
\haddocktt{rotate\ x\ i} rotates \haddocktt{x} left by \haddocktt{i} bits if \haddocktt{i} is positive,
        or right by \haddocktt{-i} bits otherwise.
\par
For unbounded types like \haddockid{Integer}, \haddockid{rotate} is equivalent to \haddockid{shift}.
\par
An instance can define either this unified \haddockid{rotate} or \haddockid{rotateL} and
        \haddockid{rotateR}, depending on which is more convenient for the type in
        question. 
\par

\end{haddockdesc}
\begin{haddockdesc}
\item[\begin{tabular}{@{}l}
bit\ ::\ Int\ ->\ a
\end{tabular}]\haddockbegindoc
\haddocktt{bit\ i} is a value with the \haddocktt{i}th bit set and all other bits clear
\par

\end{haddockdesc}
\begin{haddockdesc}
\item[\begin{tabular}{@{}l}
setBit\ ::\ a\ ->\ Int\ ->\ a
\end{tabular}]\haddockbegindoc
\haddocktt{x\ `setBit`\ i} is the same as \haddocktt{x\ .|.\ bit\ i}
\par

\end{haddockdesc}
\begin{haddockdesc}
\item[\begin{tabular}{@{}l}
clearBit\ ::\ a\ ->\ Int\ ->\ a
\end{tabular}]\haddockbegindoc
\haddocktt{x\ `clearBit`\ i} is the same as \haddocktt{x\ .{\char '46}.\ complement\ (bit\ i)}
\par

\end{haddockdesc}
\begin{haddockdesc}
\item[\begin{tabular}{@{}l}
complementBit\ ::\ a\ ->\ Int\ ->\ a
\end{tabular}]\haddockbegindoc
\haddocktt{x\ `complementBit`\ i} is the same as \haddocktt{x\ `xor`\ bit\ i}
\par

\end{haddockdesc}
\begin{haddockdesc}
\item[\begin{tabular}{@{}l}
testBit\ ::\ a\ ->\ Int\ ->\ Bool
\end{tabular}]\haddockbegindoc
Return \haddockid{True} if the \haddocktt{n}th bit of the argument is 1
\par

\end{haddockdesc}
\begin{haddockdesc}
\item[\begin{tabular}{@{}l}
bitSize\ ::\ a\ ->\ Int
\end{tabular}]\haddockbegindoc
Return the number of bits in the type of the argument.  The actual
        value of the argument is ignored.  The function \haddockid{bitSize} is
        undefined for types that do not have a fixed bitsize, like \haddockid{Integer}.
\par

\end{haddockdesc}
\begin{haddockdesc}
\item[\begin{tabular}{@{}l}
isSigned\ ::\ a\ ->\ Bool
\end{tabular}]\haddockbegindoc
Return \haddockid{True} if the argument is a signed type.  The actual
        value of the argument is ignored 
\par

\end{haddockdesc}
\begin{haddockdesc}
\item[\begin{tabular}{@{}l}
shiftL\ ::\ a\ ->\ Int\ ->\ a
\end{tabular}]\haddockbegindoc
Shift the argument left by the specified number of bits
        (which must be non-negative).
\par
An instance can define either this and \haddockid{shiftR} or the unified
        \haddockid{shift}, depending on which is more convenient for the type in
        question. 
\par

\end{haddockdesc}
\begin{haddockdesc}
\item[\begin{tabular}{@{}l}
shiftR\ ::\ a\ ->\ Int\ ->\ a
\end{tabular}]\haddockbegindoc
Shift the first argument right by the specified number of bits
        (which must be non-negative).
        Right shifts perform sign extension on signed number types;
        i.e. they fill the top bits with 1 if the \haddocktt{x} is negative
        and with 0 otherwise.
\par
An instance can define either this and \haddockid{shiftL} or the unified
        \haddockid{shift}, depending on which is more convenient for the type in
        question. 
\par

\end{haddockdesc}
\begin{haddockdesc}
\item[\begin{tabular}{@{}l}
rotateL\ ::\ a\ ->\ Int\ ->\ a
\end{tabular}]\haddockbegindoc
Rotate the argument left by the specified number of bits
        (which must be non-negative).
\par
An instance can define either this and \haddockid{rotateR} or the unified
        \haddockid{rotate}, depending on which is more convenient for the type in
        question. 
\par

\end{haddockdesc}
\begin{haddockdesc}
\item[\begin{tabular}{@{}l}
rotateR\ ::\ a\ ->\ Int\ ->\ a
\end{tabular}]\haddockbegindoc
Rotate the argument right by the specified number of bits
        (which must be non-negative).
\par
An instance can define either this and \haddockid{rotateL} or the unified
        \haddockid{rotate}, depending on which is more convenient for the type in
        question. 
\par

\end{haddockdesc}
\end{haddockdesc}
\begin{haddockdesc}
\item[\begin{tabular}{@{}l}
instance\ Bits\ Int\\instance\ Bits\ Int8\\instance\ Bits\ Int16\\instance\ Bits\ Int32\\instance\ Bits\ Int64\\instance\ Bits\ Integer\\instance\ Bits\ Word\\instance\ Bits\ Word8\\instance\ Bits\ Word16\\instance\ Bits\ Word32\\instance\ Bits\ Word64\\instance\ Bits\ WordPtr\\instance\ Bits\ IntPtr\\instance\ Bits\ CChar\\instance\ Bits\ CSChar\\instance\ Bits\ CUChar\\instance\ Bits\ CShort\\instance\ Bits\ CUShort\\instance\ Bits\ CInt\\instance\ Bits\ CUInt\\instance\ Bits\ CLong\\instance\ Bits\ CULong\\instance\ Bits\ CLLong\\instance\ Bits\ CULLong\\instance\ Bits\ CPtrdiff\\instance\ Bits\ CSize\\instance\ Bits\ CWchar\\instance\ Bits\ CSigAtomic\\instance\ Bits\ CIntPtr\\instance\ Bits\ CUIntPtr\\instance\ Bits\ CIntMax\\instance\ Bits\ CUIntMax
\end{tabular}]
\end{haddockdesc}