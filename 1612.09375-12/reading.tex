% Basic Category Theory
% Tom Leinster <Tom.Leinster@ed.ac.uk>
% 
% Copyright (c) Tom Leinster 2014-2016
% 
% Further reading
% 

\chapter*{Further reading}


This book is intentionally short.  Even some topics that are included in
most introductions to category theory are omitted here.  I will indicate
some of the topics that lie beyond the scope of this book, and suggest
where you might read about them.  Since there is far more written on
category theory than anyone could read in a lifetime, these recommendations
are necessarily subjective.

The towering presence among category theory books is the classic by
one of its founders:
% 
\begin{citedsource}
Saunders Mac~Lane,
\emph{Categories for the Working Mathematician}.\linebreak
Springer, 
1971;
second edition with two new chapters, 
1998.
\end{citedsource}
% 
It is so well-written that more than forty years on, it is still the most
popular introduction to the subject.  It addresses a more mature readership
than this text, and covers many topics omitted here, including monads (one
formalization of the idea of algebraic theory), monoidal categories
(categories equipped with a tensor product), 2-categories (mentioned at the
end of our Chapter~\ref{ch:cfnt}), abelian categories (categories of
modules), ends (an elegant generalization of the notion of limit), and Kan
extensions (which provide the tongue-in-cheek title of the book's final
section: `All concepts are Kan extensions').

Another well-liked book, longer than the one you hold in your hands but
written for a similar readership, is:
% 
\begin{citedsource}
Steve Awodey,
\emph{Category Theory}.
Oxford University Press, 
2010.
\end{citedsource}
% 
Awodey's book covers less than Mac~Lane's, but is particularly strong on
connections between category theory and other parts of logic.  It has a
full chapter on cartesian closed categories, and also covers the theory of
monads.

Those who prefer lectures to books might try this library of 75 ten-minute
introductory category theory videos:
% 
\begin{citedsource}
Eugenia Cheng and Simon Willerton,
The Catsters.
Available at\linebreak
\href{https://www.youtube.com/user/TheCatsters}{\url{https://www.youtube.com/user/TheCatsters}}, 
2007--2010.
\end{citedsource}
% 
Other than the topics treated here, they cover monads, enriched categories,
internal groups (and other internal algebraic structures), string diagrams
(which we touched on in Remark~\ref{rmk:triangle-string}), and several more
sophisticated topics.

For inspiration as much as instruction, here are two further
recommendations.
% 
\begin{citedsource}
Saunders Mac~Lane,
\emph{Mathematics: Form and Function}.
Springer,
1986.
\end{citedsource}
% 
\begin{citedsource}
F. William Lawvere and Stephen H. Schanuel,
\emph{Conceptual Mathematics: A First Introduction to Categories}.
Cambridge University Press,
1997.
\end{citedsource}
% 
\emph{Mathematics: Form and Function} is a tour through much of pure and
applied mathematics, written from a categorical perspective.  Its declared
purpose is to present the author's philosophy of mathematics, but it can
also be enjoyed for its many excellent vignettes of exposition.  (Beware of
the numerous small errors.)  \emph{Conceptual Mathematics} is a
thought-provoking text and an intriguing experiment: category theory for
high-school students, complete with classroom dialogues.

For categorical topics beyond the scope of this book, two good general
references are:
% 
\begin{citedsource}
Francis Borceux,
\emph{Handbook of Categorical Algebra, Volumes 1--3}.
Cambridge University Press, 
1994.
\end{citedsource}
% 
\begin{citedsource}
Various authors,
\emph{The $n$Lab}.
Available at \href{https://ncatlab.org}{\url{https://ncatlab.org}}, 2008--\linebreak
present.
\end{citedsource}
% 
Borceux's encyclopaedic work often takes a different point of view from the
present text, but covers many, many more topics.  Apart from those just
mentioned in connection with other books, some of the more important ones
are fibrations, bimodules (also called profunctors or distributors),
Lawvere theories, Cauchy completeness, Morita equivalence, absolute
colimits, and flatness.

The $n$Lab is an ever-growing online resource for mathematics, focusing on
category theory and operating on similar principles to Wikipedia.
Individual entries can be idiosyncratic, but it has become a very useful
reference for advanced categorical topics.

Vigorous research in category theory continues to be done.  The sources
listed above provide ample onward references for anyone wishing to explore.

\minihead{Other texts cited}

\begin{citedsource}
Timothy Gowers,
\emph{Mathematics: A Very Short Introduction}.
Oxford University Press,
2002.
\end{citedsource}

\begin{citedsource}
G. M. Kelly,
\emph{Basic Concepts of Enriched Category Theory}.
Cambridge University Press,
1982. 
Also 
\emph{Reprints in Theory and Applications of Categories}
10 (2005), 1--136,
available at 
\href{http://www.tac.mta.ca/tac/reprints}{\url{http://www.tac.mta.ca/tac/reprints}}.
\end{citedsource}

\begin{citedsource}
F. William Lawvere and Robert Rosebrugh,
\emph{Sets for Mathematics}.
Cambridge University Press,
2003.
\end{citedsource}

\begin{citedsource}
Tom Leinster,
Rethinking set theory.
\emph{American Mathematical Mon\-thly} 121 (2014), no.~5, 403--415.
Also available at 
\href{https://arxiv.org/abs/1212.6543}{\url{https://arxiv.org/abs/1212.6543}}.
\end{citedsource}

