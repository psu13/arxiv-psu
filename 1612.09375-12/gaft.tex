% Basic Category Theory
% Tom Leinster <Tom.Leinster@ed.ac.uk>
% 
% Copyright (c) Tom Leinster 2014-2016
% 
% Appendix: Proof of the general adjoint functor theorem
% 

\chapter{Proof of the general adjoint functor theorem}

%
\index{adjoint functor theorems!general|(}%
\index{general adjoint functor theorem (GAFT)|(}
%

Here we prove the general adjoint functor theorem, which for convenience is
restated below.  The left-to-right implication follows immediately from
Theorem~\ref{thm:adjts-cts}; it is the right-to-left implication that we
have to prove.

\paragraph*{\normalfont\scshape Theorem~\ref{thm:gaft} (General adjoint functor theorem)} 
{\itshape
Let $\cat{A}$ be a category, $\cat{B}$ a complete category, and $G\from
\cat{B} \to \cat{A}$ a functor.  Suppose that $\cat{B}$ is locally small and
that for each $A \in \cat{A}$, the category $\comma{A}{G}$ has a weakly%
%
\index{set!weakly initial|(}%
\index{weakly initial|(}
%
initial set.  Then
\[
G \text{ has a left adjoint}
\iff
G \text{ preserves limits}.
\]
}

The heart of the proof is the case $\cat{A} = \One$, where GAFT asserts
that a complete locally small category with a weakly initial set has an
initial object.  We prove this first.

The proof of this special case is illuminated by considering the even more
special case where $\cat{A} = \One$ and the category $\cat{B}$ is a poset
$\scat{B}$.  We saw in Example~\ref{eg:oaft-least} that the initial object
(least%
%
\index{least element}
%
element) of a complete poset $\scat{B}$ can be constructed as the meet of
all its elements.  Otherwise put, it is the limit of the identity functor
$1_\scat{B}\from \scat{B} \to \scat{B}$.

One might try to extend this result to arbitrary categories $\cat{B}$ by
proving that the limit of the identity%
%
\index{functor!identity!limit of}%
\index{object!initial!limit of identity@as limit of identity}%
\index{limit!identity@of identity}
%
functor $1_\cat{B}\from \cat{B} \to \cat{B}$ is (if it exists) an initial
object.  This is indeed true (Exercise~\ref{ex:lim-of-id} below).  However,
it is unhelpful: for if $\cat{B}$ is large then the limit of $1_\cat{B}$
is a large%
%
\index{limit!large} 
%
limit, but we are only given that $\cat{B}$ has small limits.

We seem to be at an impasse~-- but this is where the clever idea behind GAFT
comes in.  In order to construct the least element of a complete poset, it
is not necessary to take the meet of \emph{all} the elements.  More
economically, we could just take the meet of the elements of some weakly
initial subset (Exercise~\ref{ex:wi-poset}).  In general, for an arbitrary
complete category, the limit of any weakly initial set is an initial
object.  We prove this now.

\begin{alemma}  
\label{lemma:wi-init}
Let $\cat{C}$ be a complete locally small category with a weakly initial
set.  Then $\cat{C}$ has an initial object.
\end{alemma}

\begin{pf}
Let $\scat{S}$ be a weakly initial set in $\cat{C}$.  Regard $\scat{S}$ as
a full subcategory of $\cat{C}$; then $\scat{S}$ is small, since $\cat{C}$
is locally small.  We may therefore take a limit cone
% 
\begin{equation}        
\label{eq:lim-wi}
\Bigl( 0 \toby{p_S} S \Bigr)_{S \in \scat{S}}
\end{equation}
% 
of the inclusion $\scat{S} \incl \cat{C}$.  We prove that $0$ is initial.

Let $C \in \cat{C}$.  We have to show that there is exactly one map $0 \to
C$.  Certainly there is at least one, since we may choose some $S \in
\scat{S}$ and map $j \from S \to C$, and we then have the composite $j p_S
\from 0 \to C$.  To prove uniqueness, let $f, g \from 0 \to C$.  Form the
equalizer
\[
\xymatrix@1{
E \ar[r]^i      &0 \ar@<.5ex>[r]^f \ar@<-.5ex>[r]_g       &C.
}
\]
Since $\scat{S}$ is weakly initial, we may choose $S \in \scat{S}$ and
$h\from S \to E$.  We then have maps
\[
\xymatrix@1{
0 \ar[r]^{p_S} &S \ar[r]^h &E \ar[r]^i &0 \\
}
\]
with the property that for all $S' \in \scat{S}$,
\[
p_{S'}(ihp_S) = (p_{S'}ih)p_S = p_{S'} = p_{S'} 1_0
\]
(where the second equality follows from~\eqref{eq:lim-wi} being a cone).
But~\eqref{eq:lim-wi} is a \emph{limit} cone, so $ihp_S = 1_0$ by
Exercise~\ref{ex:jointly-monic}\bref{part:j-m-main}.  Hence
\[
f = fihp_S = gihp_S = g,
\]
as required.
\end{pf}

We have now proved GAFT in the special case $\cat{A} = \One$.  The rest of
the proof is comparatively routine.

\begin{alemma}  
\label{lemma:gaft-creates}
%
\index{limit!creation of}%
\index{creation of limits}%
\index{comma category!limits in}
%
Let $\cat{A}$ and $\cat{B}$ be categories.  Let $G \from \cat{B} \to
\cat{A}$ be a functor that preserves limits.  Then the projection functor
$P_A \from \comma{A}{G} \to \cat{B}$ of~\eqref{eq:pjn-aft} creates limits,
for each $A \in \cat{A}$.  In particular, if $\cat{B}$ is complete then so
is each comma category $\comma{A}{G}$.
\end{alemma}

\begin{pf}
The first statement is
Exercise~\ref{ex:gaft-creates}\bref{part:gaft-creates-creates}, and the
second follows from Lemma~\ref{lemma:creates-preserves}.
\end{pf}

We now prove GAFT.  By Corollary~\ref{cor:pre-AFT}, it is enough to show
that $\comma{A}{G}$ has an initial object for each $A \in \cat{A}$.  Let $A
\in \cat{A}$.  By Lemma~\ref{lemma:gaft-creates}, $\comma{A}{G}$ is
complete, and by hypothesis, it has a weakly initial set.  It is also
locally small, since $\cat{B}$ is.  Hence by Lemma~\ref{lemma:wi-init}, it
has an initial object, as required.


\exs


\begin{aquestion}        
\label{ex:lim-of-id}
In this exercise, we suspend the convention (made implicitly in
Definition~\ref{defn:lim}) that we only speak of the limit of a functor
$\scat{I} \to \cat{C}$ when $\scat{I}$ is small.%
%
\index{limit!small}%
\index{limit!large}
%
Let $\cat{B}$ be a
category, possibly large.  The aim is to prove that a limit%
%
\index{functor!identity!limit of}%
\index{object!initial!limit of identity@as limit of identity}%
\index{limit!identity@of identity}
% 
of the identity functor on $\cat{B}$ is exactly an initial object of
$\cat{B}$.
% 
\begin{enumerate}[(b)]
\item 
Let $0$ be an initial object of $\cat{B}$.  Show that the cone $(0 \to
B)_{B \in \cat{B}}$ on the identity functor $1_\cat{B}$ is a limit cone.

\item
Now let $\Bigl(L \toby{p_B} B\Bigr)_{B \in \cat{B}}$ be a limit cone on
$1_\cat{B}$.  Prove that $p_L$ is the identity on $L$, and deduce that $L$
is initial.  
\end{enumerate}
\end{aquestion}


\begin{aquestion}        
\label{ex:wi-poset}
Here you will prove the special case of Lemma~\ref{lemma:wi-init} in which
the category concerned is a poset.  Let $C$ be a poset and $S \sub C$.
% 
\begin{enumerate}[(b)]
\item 
What does it mean, in purely order-theoretic terms, for $S$ to be a weakly
initial set in $C$?

\item 
Prove directly that if $S$ is weakly initial and the meet $\Meet_{s \in S}
s$ exists then $\Meet_{s \in S} s$ is a least%
%
\index{least element}
%
element of $C$.%
%
\index{set!weakly initial|)}%
\index{weakly initial|)}
%
\end{enumerate}
\end{aquestion}


\begin{aquestion}       
\label{ex:gaft-creates}
Let $G \from \cat{B} \to \cat{A}$ be a limit-preserving functor, and let $A
\in \cat{A}$.
% 
\begin{enumerate}[(b)]
\item 
Show that for any small category $\scat{I}$, a diagram of shape $\scat{I}$
in $\comma{A}{G}$ amounts to a diagram $E$ of shape $\scat{I}$ in $\cat{B}$
together with a cone on $G \of E$ with vertex $A$.

\item
\label{part:gaft-creates-creates}
Prove that the projection functor $P_A \from \comma{A}{G} \to \cat{B}$
of~\eqref{eq:pjn-aft} creates limits.%
%
\index{adjoint functor theorems!general|)}%
\index{general adjoint functor theorem (GAFT)|)}
%
\end{enumerate}
\end{aquestion}
