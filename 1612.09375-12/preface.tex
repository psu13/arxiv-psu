% Basic Category Theory
% Tom Leinster <Tom.Leinster@ed.ac.uk>
% 
% Copyright (c) Tom Leinster 2014-2016
% 
% Note to the reader
% 

\chapter*{Note to the reader}
\label{ch:preface}


This is not a sophisticated text.  In writing it, I have assumed no more
mathematical knowledge than might be acquired from an undergraduate degree
at an ordinary British university, and I have not assumed that you are used
to learning mathematics by reading a book rather than attending lectures.
Furthermore, the list of topics covered is deliberately short, omitting all
but the most fundamental parts of category theory.  A `further reading'
section points to suitable follow-on texts.

There are two things that every reader should know about this book.  One
concerns the examples, and the other is about the exercises.

Each new concept is illustrated with a generous supply of examples, but it
is not necessary to understand them all.  In courses I have taught based on
earlier versions of this text, probably no student has had the background to
understand every example.  All that matters is to understand enough
examples that you can connect the new concepts with mathematics that you
already know.

As for the exercises, I join every other textbook author in exhorting you
to do them; but there is a further important point.  In subjects such as
number theory and combinatorics, some questions are simple to state but
extremely hard to answer.  Basic category theory is not like that.  To
understand the question is very nearly to know the answer.  In most of the
exercises, there is only one possible way to proceed.  So, if you are stuck
on an exercise, a likely remedy is to go back through each term in the
question and make sure that you understand it \emph{in full}.  Take your
time.  Understanding, rather than problem solving, is the main challenge of
learning category theory.

Citations such as \citeCWM\ refer to the sources listed in `Further
reading'.

This book developed out of master's-level courses taught several times at
the University of Glasgow and, before that, at the University of Cambridge.
In turn, the Cambridge version was based on Part~III courses taught for
many years by Martin Hyland and Peter Johnstone.  Although this text is
significantly different from any of their courses, I am conscious that
certain exercises, lines of development and even turns of phrase have
persisted through that long evolution.  I would like to record my
indebtedness to them, as well as my thanks to Fran\c{c}ois Petit, my past
students, the anonymous reviewers, and the staff of Cambridge University
Press.




