\documentclass[12pt]{article}
%\documentclass[landscape,twocolumn,A4]{article}
\usepackage[dvips]{graphicx}
\usepackage{latexsym}
\usepackage[affil-it]{authblk}
\usepackage{mathptm}
\usepackage{amsmath}
\usepackage{amssymb}
%\usepackage{color}
\usepackage{floatrow}
\usepackage{units}
\usepackage{setspace}
\usepackage[font=small,labelfont=bf]{caption}
%\usepackage{fullpage}
%\textwidth 14cm
%\usepackage{amsmath,amsthm,amscd,amssymb}
\usepackage[colorlinks=true
,breaklinks=true
,urlcolor=blue
,anchorcolor=blue
,citecolor=blue
,filecolor=blue
,linkcolor=blue
,menucolor=blue
,linktocpage=true]{hyperref}
\hypersetup{
bookmarksopen=true,
bookmarksnumbered=true,
bookmarksopenlevel=10
}
\usepackage[noBBpl,sc]{mathpazo}
\usepackage[papersize={7.0in, 10.0in}, left=.5in, right=.5in, top=1in, bottom=.9in]{geometry}
\linespread{1.05}
\sloppy
\raggedbottom
\pagestyle{plain}

% these include amsmath and that can cause trouble in older docs.
\makeatletter
\@ifpackageloaded{amsmath}{}{\RequirePackage{amsmath}}

\DeclareFontFamily{U}  {cmex}{}
\DeclareSymbolFont{Csymbols}       {U}  {cmex}{m}{n}
\DeclareFontShape{U}{cmex}{m}{n}{
    <-6>  cmex5
   <6-7>  cmex6
   <7-8>  cmex6
   <8-9>  cmex7
   <9-10> cmex8
  <10-12> cmex9
  <12->   cmex10}{}

\def\Set@Mn@Sym#1{\@tempcnta #1\relax}
\def\Next@Mn@Sym{\advance\@tempcnta 1\relax}
\def\Prev@Mn@Sym{\advance\@tempcnta-1\relax}
\def\@Decl@Mn@Sym#1#2#3#4{\DeclareMathSymbol{#2}{#3}{#4}{#1}}
\def\Decl@Mn@Sym#1#2#3{%
  \if\relax\noexpand#1%
    \let#1\undefined
  \fi
  \expandafter\@Decl@Mn@Sym\expandafter{\the\@tempcnta}{#1}{#3}{#2}%
  \Next@Mn@Sym}
\def\Decl@Mn@Alias#1#2#3{\Prev@Mn@Sym\Decl@Mn@Sym{#1}{#2}{#3}}
\let\Decl@Mn@Char\Decl@Mn@Sym
\def\Decl@Mn@Op#1#2#3{\def#1{\DOTSB#3\slimits@}}
\def\Decl@Mn@Int#1#2#3{\def#1{\DOTSI#3\ilimits@}}

\let\sum\undefined
\DeclareMathSymbol{\tsum}{\mathop}{Csymbols}{"50}
\DeclareMathSymbol{\dsum}{\mathop}{Csymbols}{"51}

\Decl@Mn@Op\sum\dsum\tsum

\makeatother

\makeatletter
\@ifpackageloaded{amsmath}{}{\RequirePackage{amsmath}}

\DeclareFontFamily{OMX}{MnSymbolE}{}
\DeclareSymbolFont{largesymbolsX}{OMX}{MnSymbolE}{m}{n}
\DeclareFontShape{OMX}{MnSymbolE}{m}{n}{
    <-6>  MnSymbolE5
   <6-7>  MnSymbolE6
   <7-8>  MnSymbolE7
   <8-9>  MnSymbolE8
   <9-10> MnSymbolE9
  <10-12> MnSymbolE10
  <12->   MnSymbolE12}{}

\DeclareMathSymbol{\downbrace}    {\mathord}{largesymbolsX}{'251}
\DeclareMathSymbol{\downbraceg}   {\mathord}{largesymbolsX}{'252}
\DeclareMathSymbol{\downbracegg}  {\mathord}{largesymbolsX}{'253}
\DeclareMathSymbol{\downbraceggg} {\mathord}{largesymbolsX}{'254}
\DeclareMathSymbol{\downbracegggg}{\mathord}{largesymbolsX}{'255}
\DeclareMathSymbol{\upbrace}      {\mathord}{largesymbolsX}{'256}
\DeclareMathSymbol{\upbraceg}     {\mathord}{largesymbolsX}{'257}
\DeclareMathSymbol{\upbracegg}    {\mathord}{largesymbolsX}{'260}
\DeclareMathSymbol{\upbraceggg}   {\mathord}{largesymbolsX}{'261}
\DeclareMathSymbol{\upbracegggg}  {\mathord}{largesymbolsX}{'262}
\DeclareMathSymbol{\braceld}      {\mathord}{largesymbolsX}{'263}
\DeclareMathSymbol{\bracelu}      {\mathord}{largesymbolsX}{'264}
\DeclareMathSymbol{\bracerd}      {\mathord}{largesymbolsX}{'265}
\DeclareMathSymbol{\braceru}      {\mathord}{largesymbolsX}{'266}
\DeclareMathSymbol{\bracemd}      {\mathord}{largesymbolsX}{'267}
\DeclareMathSymbol{\bracemu}      {\mathord}{largesymbolsX}{'270}
\DeclareMathSymbol{\bracemid}     {\mathord}{largesymbolsX}{'271}

\def\horiz@expandable#1#2#3#4#5#6#7#8{%
  \@mathmeasure\z@#7{#8}%
  \@tempdima=\wd\z@
  \@mathmeasure\z@#7{#1}%
  \ifdim\noexpand\wd\z@>\@tempdima
    $\m@th#7#1$%
  \else
    \@mathmeasure\z@#7{#2}%
    \ifdim\noexpand\wd\z@>\@tempdima
      $\m@th#7#2$%
    \else
      \@mathmeasure\z@#7{#3}%
      \ifdim\noexpand\wd\z@>\@tempdima
        $\m@th#7#3$%
      \else
        \@mathmeasure\z@#7{#4}%
        \ifdim\noexpand\wd\z@>\@tempdima
          $\m@th#7#4$%
        \else
          \@mathmeasure\z@#7{#5}%
          \ifdim\noexpand\wd\z@>\@tempdima
            $\m@th#7#5$%
          \else
           #6#7%
          \fi
        \fi
      \fi
    \fi
  \fi}

\def\overbrace@expandable#1#2#3{\vbox{\m@th\ialign{##\crcr
  #1#2{#3}\crcr\noalign{\kern2\p@\nointerlineskip}%
  $\m@th\hfil#2#3\hfil$\crcr}}}
\def\underbrace@expandable#1#2#3{\vtop{\m@th\ialign{##\crcr
  $\m@th\hfil#2#3\hfil$\crcr
  \noalign{\kern2\p@\nointerlineskip}%
  #1#2{#3}\crcr}}}

\def\overbrace@#1#2#3{\vbox{\m@th\ialign{##\crcr
  #1#2\crcr\noalign{\kern2\p@\nointerlineskip}%
  $\m@th\hfil#2#3\hfil$\crcr}}}
\def\underbrace@#1#2#3{\vtop{\m@th\ialign{##\crcr
  $\m@th\hfil#2#3\hfil$\crcr
  \noalign{\kern2\p@\nointerlineskip}%
  #1#2\crcr}}}

\def\bracefill@#1#2#3#4#5{$\m@th#5#1\leaders\hbox{$#4$}\hfill#2\leaders\hbox{$#4$}\hfill#3$}

\def\downbracefill@{\bracefill@\braceld\bracemd\bracerd\bracemid}
\def\upbracefill@{\bracefill@\bracelu\bracemu\braceru\bracemid}

\DeclareRobustCommand{\downbracefill}{\downbracefill@\textstyle}
\DeclareRobustCommand{\upbracefill}{\upbracefill@\textstyle}

\def\upbrace@expandable{%
  \horiz@expandable
    \upbrace
    \upbraceg
    \upbracegg
    \upbraceggg
    \upbracegggg
    \upbracefill@}
\def\downbrace@expandable{%
  \horiz@expandable
    \downbrace
    \downbraceg
    \downbracegg
    \downbraceggg
    \downbracegggg
    \downbracefill@}

\DeclareRobustCommand{\overbrace}[1]{\mathop{\mathpalette{\overbrace@expandable\downbrace@expandable}{#1}}\limits}
\DeclareRobustCommand{\underbrace}[1]{\mathop{\mathpalette{\underbrace@expandable\upbrace@expandable}{#1}}\limits}

\makeatother


\usepackage[small]{titlesec}
\usepackage{cite}

% make sure there is enough TOC for reasonable pdf bookmarks.
\setcounter{tocdepth}{3}

%\usepackage[dotinlabels]{titletoc}
%\titlelabel{{\thetitle}.\quad}
%\usepackage{titletoc}
\usepackage[small]{titlesec}

\titleformat{\section}[block]
  {\fillast\medskip}
  {\bfseries{\thesection. }}
  {1ex minus .1ex}
  {\bfseries}
 
\titleformat*{\subsection}{\itshape}
\titleformat*{\subsubsection}{\itshape}

\setcounter{tocdepth}{2}

\titlecontents{section}
              [2.3em] 
              {\bigskip}
              {{\contentslabel{2.3em}}}
              {\hspace*{-2.3em}}
              {\titlerule*[1pc]{}\contentspage}
              
\titlecontents{subsection}
              [4.7em] 
              {}
              {{\contentslabel{2.3em}}}
              {\hspace*{-2.3em}}
              {\titlerule*[.5pc]{}\contentspage}

% hopefully not used.           
\titlecontents{subsubsection}
              [7.9em]
              {}
              {{\contentslabel{3.3em}}}
              {\hspace*{-3.3em}}
              {\titlerule*[.5pc]{}\contentspage}
%\makeatletter
\renewcommand\tableofcontents{%
    \section*{\contentsname
        \@mkboth{%
           \MakeLowercase\contentsname}{\MakeLowercase\contentsname}}%
    \@starttoc{toc}%
    }
\def\@oddhead{{\scshape\rightmark}\hfil{\small\scshape\thepage}}%
\def\sectionmark#1{%
      \markright{\MakeLowercase{%
        \ifnum \c@secnumdepth >\m@ne
          \thesection\quad
        \fi
        #1}}}
        
\makeatother

%\makeatletter

 \def\small{%
  \@setfontsize\small\@xipt{13pt}%
  \abovedisplayskip 8\p@ \@plus3\p@ \@minus6\p@
  \belowdisplayskip \abovedisplayskip
  \abovedisplayshortskip \z@ \@plus3\p@
  \belowdisplayshortskip 6.5\p@ \@plus3.5\p@ \@minus3\p@
  \def\@listi{%
    \leftmargin\leftmargini
    \topsep 9\p@ \@plus3\p@ \@minus5\p@
    \parsep 4.5\p@ \@plus2\p@ \@minus\p@
    \itemsep \parsep
  }%
}%
 \def\footnotesize{%
  \@setfontsize\footnotesize\@xpt{12pt}%
  \abovedisplayskip 10\p@ \@plus2\p@ \@minus5\p@
  \belowdisplayskip \abovedisplayskip
  \abovedisplayshortskip \z@ \@plus3\p@
  \belowdisplayshortskip 6\p@ \@plus3\p@ \@minus3\p@
  \def\@listi{%
    \leftmargin\leftmargini
    \topsep 6\p@ \@plus2\p@ \@minus2\p@
    \parsep 3\p@ \@plus2\p@ \@minus\p@
    \itemsep \parsep
  }%
}%
\def\open@column@one#1{%
 \ltxgrid@info@sw{\class@info{\string\open@column@one\string#1}}{}%
 \unvbox\pagesofar
 \@ifvoid{\footsofar}{}{%
  \insert\footins\bgroup\unvbox\footsofar\egroup
  \penalty\z@
 }%
 \gdef\thepagegrid{one}%
 \global\pagegrid@col#1%
 \global\pagegrid@cur\@ne
 \global\count\footins\@m
 \set@column@hsize\pagegrid@col
 \set@colht
}%

\def\frontmatter@abstractheading{%
\bigskip
 \begingroup
  \centering\large
  \abstractname
  \par\bigskip
 \endgroup
}%

\makeatother

%\DeclareSymbolFont{CMlargesymbols}{OMX}{cmex}{m}{n}
%\DeclareMathSymbol{\sum}{\mathop}{CMlargesymbols}{"50}
%\fancyhead[RO,RE]{Born's Rule}
%\fancyhead[CO,CE]{Notes \today}
%\fancyhead[LO,LE]{{\sl Hossenfelder}}
%\fancyfoot[RO, LE] {\thepage}

\def\beq{\begin{equation}}
\def\eeq{\end{equation}}
\def\beqn{\begin{eqnarray}}
\def\eeqn{\end{eqnarray}}
\def\slash{/\hspace*{-2.0mm}}
\newcommand{\intsum}{
\hspace*{1mm}{\mbox{\Large$\Sigma$}}\hspace*{-4.2mm}\int}

%\thispagestyle{plain}
\setcounter{page}{1}
\begin{document}

\title{Born's rule from almost nothing}
\author{Sabine Hossenfelder}
\affil{Frankfurt Institute for Advanced Studies\\
Ruth-Moufang-Str. 1,
D-60438 Frankfurt am Main, Germany
}
\date{}
\maketitle
%\centerline{ \today}
\vspace*{-1cm}

\begin{abstract}
We here put forward a simple argument for Born's rule based on the requirement that the probability distribution should not be a function of the number of degrees of freedom.
\end{abstract} 

\section{Introduction}

Quantum mechanics does not make definite predictions but only predicts probabilities for measurement outcomes. One calculates these probabilities
from the wave-function using Born's rule. In axiomatic formulations of quantum mechanics, Born's rule is usually added as an
axiom on its own right. However, it seems the kind of assumption that should not require a postulate, but that should instead follow from
the physical properties of the theory. 

Since the early days of quantum mechanics, there have thus been many attempts to derive 
Born's rule from other assumptions: non-contextuality \cite{Gleason}, decision theory \cite{Deutsch}, non-triviality \cite{Aaronson}, and the number of degrees of freedom in composite systems \cite{Hardy,Masanes:2010tt,Brukner}. The argument discussed here is most similar to the ones for many worlds presented in \cite{Vaidman, Carroll:2014mea} and the one using environment-assisted invariance \cite{Zurek}. 

However, as will become clear shortly, it is unnecessary to refer to many worlds or environment-assisted invariance (“envariance”) in particular to argue that Born’s rule is the only consistent way to put a probability distribution on a theory like quantum mechanics. For the here presented argument, one does not need to refer to the observer’s knowledge (or absence thereof) or consider the coupling of a system to its environment (as one does for envariance) or to the branching of the wave-function.

The derivation presented here is particularly relevant for hidden variables ap- proaches that violate the assumptions of Gleason’s theorem \cite{Gleason}, such as superde- terminism. Such approaches have been criticized for requiring finetuning to arrive at Born’s rule \cite{wood, sen1, sen2}. A similar criticism has been leveled at Bohmian Mechanics \cite{colin}. Symmetry requirements are a well-trodden path to explaining why certain features of a theory appear to be finetuned, but are not, which is why we hope the proof provided below may be useful to the end of settling the finetuning debate.
It must be empasized, though, that the purpose of the argument put forward here is merely to elucidate how much it takes to get Born’s rule if one already knows one is dealing with a probability distribution in a theory that is unitarily invariant. That is, we here do not aim to derive quantum mechanics, but focus on the much more tractable problem: If one has a unitary theory which gives rise to well-defined probabilities, then what else does one need to make sure the theory reproduces quantum mechanics. The answer, as we will see in the following is: not much.

\section{Argument}

To get started, let $|\Psi \rangle$ and $|\Phi \rangle$ be elements of a vector space over the complex numbers which denote points on the complex sphere of dimension $N \in {\mathbb{N}}^+$, i.e. $|\langle \Psi | \Psi \rangle| =1$ and $|\langle \Phi | \Phi \rangle| =1$.
\bigskip

{\bf Claim:} The only well-defined and consistent distribution for transition probabilities $P_N(|\Psi\rangle \to |\Phi\rangle)$ on the complex sphere of dimension $N$ which is continuous, independent of $N$, and invariant under unitary operations is $P_N(| \Psi \rangle \to |\Phi \rangle) = |\langle \Psi | \Phi \rangle|^2$. The continuity assumption is unnecessary if one restricts the original space to states of norm $K/N$ or, correspondingly, to rational-valued probabilities as a frequentist interpretation would suggest. 
\bigskip

\noindent By $P_N$ being a well-defined probability distribution we mean
\beqn
P_N(|\Psi\rangle \to |\Phi\rangle) \in \mathbb{R} ~\wedge~ 0 \leq P_N(|\Psi\rangle \to |\Phi\rangle) \leq 1 ~~~~\forall~ |\Psi \rangle, |\Phi \rangle ~. \label{welldef}
\eeqn
By $P_N$ being consistent, we mean
\beqn
&& \sum_{i=1}^N P_N(|\Psi\rangle \to |\Psi^i\rangle) = 1 ~~~~ \forall~ |\Psi \rangle~,~ \langle \Psi^i |\Psi^j\rangle = \delta^{ij} ~, \label{norm} \\
&& P_N(|\Psi^i\rangle \to |\Psi^j\rangle) = \delta^{ij} ~~~~ {\mbox{for}}~~~~ \langle \Psi^i |\Psi^j\rangle = \delta^{ij} \label{orth}~.
\eeqn

The physical motivation for these assumptions is that $P_N$ should allow us to calculate the probability for measurement outcomes. To make physical sense, this probability should not depend on the choice of basis in the Hilbert-space, which is arbitrary. Therefore, we assume that the probabilites are invariant unter unitary transformations. There is no physical motivation for the independence on N, by which we mean concretely that $P_I(\cdot) = P_J(\cdot)$ for all integers $I,J$. Instead, the proof of the above claim demonstrates that starting with this assumption one arrives at a mathematical relation (Born’s rule) which agrees with observation. That is to say, the physical relevance of this assumption follows from the proof below.

\bigskip

{\bf Proof}: Since $P_N$ is invariant under unitary operations, transition probabilities can only be functions of scalar products, ie 
$P_N(|\Psi \rangle \to |\Phi \rangle)  = P_N(\langle \Psi |\Phi \rangle) ~ \forall ~ |\Psi \rangle, |\Phi \rangle$. This means that $P_N$ is actually a map from the complex unit sphere of dimension 1 to the interval $[ 0,1] \in {\mathbb{R}}$. It follows from (\ref{orth}) that $P(\langle \Psi | \Phi \rangle) = 0$ whenever $\langle \Psi | \Phi \rangle =0$, because if $|\Psi \rangle$ and $|\Phi \rangle$ are orthogonal, one could construct a basis of which they are elements. This means $P(0) = 0$. That $P_N$ is independent of $N$ then just means that this function is the same for all $N$, so we will from now on omit the index $N$. 

Let $|\Psi^i \rangle$, $i \in \{1... N \}$ be an arbitrary orthonormal basis and
\beqn
|\Psi^* \rangle := e^{{\rm i} \theta} \sqrt{\frac{1}{N}} \sum_{i=1}^N |\Psi^i \rangle~,
\eeqn
where $\theta$ is a real number in $[0,2 \pi [$. Because of (\ref{norm}) we then have
\beqn
\sum_{j=1}^N P (\langle \Psi^j | \Psi^* \rangle)  = N  P( e^{{\rm i} \theta}/\sqrt{N}) =  1~,
\eeqn
which means
\beqn
P( e^{{\rm i} \theta} /\sqrt{N}) = 1/N ~. \label{1/N}
\eeqn

Next we use a new basis, $|\widetilde \Psi^i \rangle$ with $i \in \{1..N \}$. With $K \in {\mathbb{N}}^+$, $K < N$ we define
\beqn
| \widetilde \Psi^j \rangle &:=& \sqrt{\frac{1}{K}} \sum_{l=1}^K \exp \left( -2 \pi {\rm i} \frac{(l-1) (j-1)}{K} \right) |\Psi^l \rangle  ~~~~{\mbox{for}}~1 \leq j \leq K~, \nonumber\\
| \widetilde \Psi^j \rangle &:=& |\Psi^j \rangle ~~~~{\mbox{for}}~K+1 \leq j \leq N~. \label{tildebasis}
\eeqn
The first $K$ basis-vectors are obviously orthogonal to the last $N-K$ and we already know that the last $N-K$ are orthonormal, so it remains to show that the first $K$ basis vectors are orthogonal. For this we note that for $j,m \leq K$
\beqn
\langle \widetilde \Psi^j | \widetilde \Psi^m \rangle 
%&=& \frac{1}{K} \sum_{l=1}^K \sum_{r=1}^K        
%\exp \left( -2 \pi {\rm i} \frac{r (m-j)}{K} \right) \langle \Psi^r |\Psi^l \rangle  \\
= \frac{1}{K} \sum_{l=1}^ K        
\exp \left( -2 \pi {\rm i} \frac{m-j}{K} \right)^{l-1} ~.
\eeqn
For $m=j$ the sum gives $=K$, so the norm is $=1$ as it should be. For $m \neq j$ the sum is a geometric series and we have
\beqn
\langle \widetilde \Psi^j | \widetilde \Psi^m \rangle = \frac{ 1- \exp \left( -2 \pi {\rm i} (m-j)\right)} {1- \exp \left( -2 \pi {\rm i} (m-j) /K \right)  } =0 ~.
\eeqn

The first basis vector $|\widetilde \Psi^1 \rangle$ is paralell to $|\Psi^* \rangle$ in the subspace spanned by the first $K$ basisvectors $|\Psi^i \rangle$, therefore the $|\widetilde \Psi^j \rangle$'s for $2\leq j \leq K$ are also orthogonal to $|\Psi^*\rangle$. With this, and using $P(0) = 0$, we get
\beqn
\sum_{j=1}^N P (\langle \widetilde \Psi^j | \Psi^* \rangle) = P( e^{{\rm i} \theta}  \sqrt{K/N} ) + \sum_{j=K+1}^N P (  e^{{\rm i} \theta}  /\sqrt{N}) = 1~.
\eeqn
And by using (\ref{1/N}) we arrive at
\beqn
P ( e^{{\rm i} \theta} \sqrt{K/N} ) = K/N ~~~~\forall~ K,N \in {\mathbb{N^+}}, K \leq N, \theta \in [0, 2 \pi [~. \label{this}
\eeqn
If one was dealing with probabilitites that only took on rational values, then one can stop here. For real-valued probabilities one merely notes that
the only continuous function on the complex unit sphere with property (\ref{this}) is $P(x) = |x|^2$, hence
\beqn
P_N ( |\Psi \rangle \to |\Phi \rangle) = |\langle \Psi | \Phi \rangle|^2 ~~~~~~~~~~ \Box
\eeqn


\section{Discussion}

The verbal summary of this argument is that since the probability distribution has to be invariant under unitary operations, we can permute all basis elements, which means that the probabilities of a symmetric state like $|\Psi^*\rangle$ have to be evenly distributed, which fixes them. Then we can combine any $K$-dimensional subset of these basis elements to one whose probability is just the sum of probabilities assigned to the previous $K$ basis elements. Again we can do that in arbitary permutations and this fixes the probabilities for all $K/N$. The assumption that the distribution is independent of $N$ is essential; it is what allows us to fill in the sphere densely. 

\section*{Acknowledgements}

I thank Scott Aaronson, Sandro Donadi, and Tim Palmer for helpful feedback. This research was supported by the German Research Foundation (DFG) and the Franklin Fetzer Fund. 

\begin{thebibliography}{99}
\small{

\bibitem{Gleason}
A.~M.~Gleason, {\sl ``Measures on the closed subspaces of a Hilbert space,''} J. Math. Mech. 6:885–893, 1957.

\bibitem{Deutsch} D.~Deutsch, {\sl ``Quantum theory of probability and decisions,''} Proc. Royal Soc. A455:3129–3137 (1999)
[arXiv:quant-ph/9906015].



\bibitem{Aaronson}
S.~Aaronson, {\sl ``Is Quantum Mechanics An Island In Theoryspace?'',} {\url{https://www.scottaaronson.com/papers/island.pdf}}, retrieved June 24, 2020. 

\bibitem{Hardy} L.~Hardy, {\sl ``Quantum theory from five reasonable axioms,''} arXiv:quant-ph/0101012 (2001). 

%\cite{Masanes:2010tt}
\bibitem{Masanes:2010tt}
L.~Masanes and M.~P.~Muller,
{\sl ``A Derivation of quantum theory from physical requirements,''}
New J. Phys. \textbf{13}, 063001 (2011)
%doi:10.1088/1367-2630/13/6/063001
[arXiv:1004.1483 [quant-ph]].
%59 citations counted in INSPIRE as of 25 Jun 2020

\bibitem{Brukner}
Dakic B.~ and Brukner C.~, 
{\sl ``Quantum Theory and Beyond: Is Entanglement Special?,''} in Halvorson, H.~ (ed) {\sl ``Deep Beauty: Understanding the Quantum World through Mathematical Innovation,''} Cambridge University Press, 365-392 (2011) [arXiv:0911.0695 [quant-ph]].

%\cite{Carroll:2014mea}
\bibitem{Zurek} W.~H.~Zurek, {\sl ``Environment-assisted invariance, causality, and probabilities in quantum physics,''} Phys.
Rev.\ Lett.\ 90:120404 (2003) [arXiv:quant-ph/0211037].


\bibitem{Vaidman} L.~Vaidman, {\sl ``Probability in the Many-Worlds Interpretation of Quantum Mechanics,''}
in Y.~Ben-Menahem and M.~Hemmo (eds), {\sl ``The Probable and the Improbable: Understanding Probability in Physics, Essays in Memory of Itamar Pitowsky,''} Springer (2011) {\url{http://philsci-archive.pitt.edu/8558/}}.

\bibitem{Carroll:2014mea}
S.~M.~Carroll and C.~T.~Sebens,
{\sl ``Many Worlds, the Born Rule, and Self-Locating Uncertainty,''}  in: D.~Struppa and J.~Tollaksen (eds) {\sl ``Quantum Theory: A Two-Time Success Story''}, Springer (2014)
%doi:10.1007/978-88-470-5217-8_10
[arXiv:1405.7907 [gr-qc]].
%3 citations counted in INSPIRE as of 24 Jun 2020

\bibitem{wood}
C. J. Wood, R. W. Spekkens, {\sl ``The lesson of causal discovery algorithms for quantum correlations: Causal explanations of Bell-inequality violations require fine- tuning''}, New J. Phys. 17, 033002 (2015) [arXiv:1208.4119 [quant-ph]].

\bibitem{sen1}
I. Sen, A.Valentini, {\sl ``Superdeterministichidden-variablesmodelsI:nonequilibrium and signaling''} arXiv:2003.11989 [quant-ph]

\bibitem{sen2} 
I. Sen, A. Valentini, {\sl ``Superdeterministic hidden-variables models II: conspiracy''}, arXiv:2003.12195 [quant-ph].

\bibitem{colin} 
S. Colin and A. Valentini, {\sl ``Instability of quantum equilibrium in Bohm’s dynamics''}, Proc. Roy. Soc. Lond. A 470, 20140288 (2014) [arXiv:1306.1576 [quant-ph]].

}
\end{thebibliography}
\end{document}


