
\chapter{Initial Operad-with-Contraction}
\lbl{app:initial}%
%
\index{globular operad!contraction@with contraction|(}
%

\chapterquote{%
There existed another ending to the story of O.}{%
R\'eage~\cite{Rea}}


\noindent
We prove Proposition~\ref{propn:OC-initial}: the category $\fcat{OC}$ of
operads-with-contraction has an initial object.  This was needed in
Chapter~\ref{ch:a-defn} for the definition of weak $\omega$-category.

The explanation in~\ref{sec:contr} suggests an explicit construction of the
initial operad-with-contraction: ascend through the dimensions, at each
stage freely adding in elements obtained by contraction and then freely
adding in elements obtained by operadic composition.  Here we take a
different approach, exploiting a known existence theorem.


\section{The proof}
\lbl{sec:initial-proof}

The following result appears to be due to Kelly~\cite[27.1]{KelUTT}.%
%
\index{Kelly, Max}
%
%
\begin{thm}	\lbl{thm:comb-mon}
Let
%
\begin{diagram}[size=2em]
\cat{D}	&\rTo	&\cat{C}	\\
\dTo	&	&\dTo>V		\\
\cat{B}&\rTo_U	&\cat{A}	\\
\end{diagram}
%
be a (strict) pullback diagram in \fcat{CAT}. If $\cat{A}$ is locally
finitely presentable and each of $U$ and $V$ is finitary and monadic then
the functor $\cat{D} \go \cat{A}$ is also monadic.
\done
\end{thm}
%
Actually, all we need is:
%
\begin{cor}
In the situation of Theorem~\ref{thm:comb-mon}, \cat{D} has an initial object.
\end{cor}
%
\begin{proof}
A locally finitely presentable category is by definition cocomplete, so
\cat{A} has an initial object. The functor $\cat{D} \go \cat{A}$ has a left
adjoint (being monadic), which applied to the initial object of \cat{A} gives
an initial object of \cat{D}. 
\done
\end{proof}



Let $T$ be the free strict $\omega$-category monad on the category
$\ftrcat{\scat{G}^\op}{\Set}$ of globular sets, as in
Chapters~\ref{ch:globular} and~\ref{ch:a-defn}.  Write $\fcat{Coll}$ for
the category $\ftrcat{\scat{G}^\op}{\Set}/\pd$ of collections
(p.~\pageref{p:defn-collection}), $\fcat{CC}$ for the category of
collections-with-contraction%
%
\index{collection!contraction@with contraction}
%
(defined by replacing `operad' by `collection'
throughout Definition~\ref{defn:OC}), and $\fcat{Operad}$ for the category
of globular operads.  Then there is a (strict) pullback diagram
%
\begin{diagram}[size=2em]
\fcat{OC}	&\rTo	&\fcat{Operad}	\\
\dTo		&	&\dTo>V		\\
\fcat{CC}	&\rTo_U	&\fcat{Coll}	\\
\end{diagram}
%
in \fcat{CAT}, made up of forgetful functors.

To prove that $\fcat{OC}$ has an initial object, we verify the hypotheses
of Theorem~\ref{thm:comb-mon}.  The only non-routine part is showing that
we can freely add a contraction%
%
\index{contraction!free}
%
to any collection.

\minihead{\fcat{Coll} is locally finitely presentable} 

Since $\fcat{Coll}$ is a slice of a presheaf category, it is itself a
presheaf category~(\ref{propn:pshf-slice}) and so locally finitely
presentable (Borceux~\cite[Example 5.2.2(b)]{Borx2}).

\minihead{$U$ is finitary and monadic} 

It is straightforward to calculate that $U$ creates filtered colimits; and
since \fcat{Coll} possesses all filtered colimits, $U$ preserves them too.
It is also easy to calculate that $U$ creates coequalizers for $U$-split
coequalizer pairs.  So we have only to show that $U$ has a left adjoint.

Let $P$ be a collection.  We construct a new collection $FP$, a contraction
$\kappa^P$ on $FP$, and a map $\alpha_P: P \go FP$, together having the
appropriate universal property; so the functor $P \goesto (FP, \kappa^P)$
is left adjoint to $U$, with $\alpha$ as unit.  The definitions of $FP$ and
$\alpha_P$ are by induction on dimension:
% 
\begin{itemize}
\item if $\pi$ is the unique element of $\pd(0)$ then $(FP)(\pi) = P(\pi)$
\item if $n\geq 1$ and $\pi\in\pd(n)$ then
$
% \begin{equation}	\label{eq:FP}
(FP)(\pi)
=
P(\pi) + 
\mr{Par}_{FP}(\pi)
$
\item $\alpha_{P,\pi}: P(\pi) \rIncl (FP)(\pi)$ is inclusion as the first
summand, for all $\pi$
\item if $n\geq 1$ and $\pi\in\pd(n)$ then the source map $s: (FP)(\pi) \go
(FP)(\bdry\pi)$ is defined on the first summand of $(FP)(\pi)$ as the
composite
\[
P(\pi) \goby{s} P(\bdry\pi) \goby{\alpha_{P,\bdry\pi}} (FP)(\bdry\pi)
\]
and on the second summand as first projection; the target map is defined
dually.
\end{itemize}
%
The globularity equations hold, so $FP$ forms a collection.  Clearly
$\alpha_P: P \go FP$ is a map of collections.  The contraction $\kappa^P$
on $FP$ is defined by taking 
\[
\kappa^P_\pi: \mr{Par}_{FP}(\pi) \go (FP)(\pi)
\]
($n\geq 1$, $\pi\in\pd(n)$) to be inclusion as the second summand.  It is
easy to check that $FP$, $\kappa^P$ and $\alpha^P$ have the requisite
universal property: so $U$ has a left adjoint.


\minihead{$V$ is finitary and monadic}

The functor $T$ is finitary, by~\ref{thm:omega-forgetful-properties}.  This
implies by~\ref{thm:free-gen} that the monad $T$ is suitable, and so
by~\ref{thm:free-fixed} that $V$ is monadic.  It also implies by the
`moreover' of~\ref{propn:free-refined} that the monad induced by $V$ and
its left adjoint is finitary.  If a category has colimits of a certain
shape and a monad on it preserves colimits of that shape, then so too does
the forgetful algebra functor; hence $V$ is finitary, as required.  

\begin{notes}

I thank Steve Lack and John Power for telling me that the result I
needed,~\ref{thm:comb-mon}, was in Kelly~\cite{KelUTT}, and Sjoerd Crans
for telling me exactly where.%
%
\index{globular operad!contraction@with contraction|)}
%




\end{notes}
