
\chapter{Globular Operads}
\lbl{ch:globular}


\chapterquote{%
Once Theofilos was [\ldots] painting a mural in a Mytilene baker's shop
[\ldots]  As was his habit, he had depicted the loaves of bread upright in
their trays, like heraldic emblems on an out-thrust shield---so that no one
could be in any doubt that they were loaves of bread and very fine ones,
too.  The irate baker pointed out that in real life loaves thus placed
would have fallen to the floor.  `No,' replied Theofilos---surely with that
calm, implacable self-certainty which carried him throughout what most
people would call a miserable life---`only real loaves fall down.  Painted
ones stay where you put them.'}
{\emph{The Athenian}~\cite{Athenian}}


\noindent
In the next two chapters we explore one possible definition of weak
$\omega$-category.  Its formal shape is very simple: we take
the category \Eee\ of globular sets and the free strict $\omega$-category
monad $T$ on \Eee, construct a certain $T$-operad $L$, and define a weak
$\omega$-category to be an $L$-algebra. 

In order to see that this is a \emph{reasonable} definition, and to get a
feel for the concepts involved, we proceed at a leisurely pace.  The
present chapter is devoted to contemplation of the monad
$T$~(\ref{sec:free-strict}), of $T$-operads ($=$~globular
operads,~\ref{sec:glob-opds}), and of their algebras~(\ref{sec:glob-algs}).
In Chapter~\ref{ch:a-defn} we define the particular globular operad $L$ and
look at its algebras---that is, at weak $\omega$-categories.  To keep the
explanation in this chapter uncluttered, the discussion of the
finite-dimensional case ($n$-categories) is also deferred to
Chapter~\ref{ch:a-defn}; we stick to $\omega$-categories here.

Globular operads are an absolutely typical example of generalized operads.
Pictorially, they are typical in that $T1$ is a family of shapes (globular
pasting diagrams), and an operation in a $T$-operad is naturally drawn as
an arrow with data fitting one of these shapes as its input.  (Compare
plain operads, where the `shapes' are mere finite sequences.)  Technically,
globular operads are typical in that $T$ is, in the terminology of
Appendix~\ref{app:special-cart}, a finitary familially representable monad
on a presheaf category.  Hence the explanations contained in this chapter
can easily be adapted to many other species of generalized operad.




\section{The free strict $\omega$-category monad}
\lbl{sec:free-strict}


In~\ref{sec:cl-strict} we defined a strict $\omega$-category as a globular
set equipped with extra structure, and a strict $\omega$-functor as a map
of globular sets preserving that structure.  There is consequently a
forgetful functor from the category \strcat{\omega} of strict
$\omega$-categories and strict $\omega$-functors to the category
\ftrcat{\scat{G}^\op}{\Set} of globular sets.  In
Appendix~\ref{app:free-strict} it is shown that this forgetful functor has
a left adjoint, that the adjunction is monadic, and that the induced monad
$(T, \mu, \eta)$%
% 
\glo{strommon}%
%
\index{omega-category@$\omega$-category!strict!free}
%
% 
on \ftrcat{\scat{G}^\op}{\Set} is cartesian.  

To understand this in pictorial terms, we start by considering $T1$, the
free strict $\omega$-category on the terminal globular set
\[
1 = (\cdots \pile{\rTo \\ \rTo} 1 \pile{\rTo \\ \rTo} \cdots
\pile{\rTo \\ \rTo} 1).	\\
\] 
The free strict $\omega$-category functor takes a globular set and creates
all possible formal composites in it.  A typical element of $(T1)(2)$ looks
like
%
\begin{equation}	\label{eq:typical-glob-pd}
\gfstsu
\gfoursu
\gzersu
\gonesu
\gzersu
\gtwosu
\glstsu,
\end{equation}
% 
where each $k$-cell drawn represents the unique member of $1(k)$.  Note
that although this picture contains 4 dots representing 0-cells of $1$,
they actually all represent the same 0-cell; of course, $1$ only \emph{has}
one 0-cell.  The same goes for the 1- and 2-cells.  So we have drawn a
flattened-out version of the true, twisted, picture.  We call an element of
$(T1)(m)$ a \demph{(globular) $m$-pasting%
%
\index{pasting diagram!globular}
%
diagram} and write $T1=\pd$.%
% 
\glo{pd}
% 

Since the theory of strict $\omega$-categories includes identities, there
is for each $m\geq 2$ an element of $\pd(m)$ looking
like~\bref{eq:typical-glob-pd}.  Although the pictures look the same, they
are regarded as \emph{different}%
%
\lbl{p:degen-pds}%
%
\index{pasting diagram!globular!degenerate}
%
%
pasting diagrams for different values of $m$; the sets $\pd(m)$ and
$\pd(m')$ are considered disjoint when $m \neq m'$.  When it comes to
understanding the definition of weak $n$-category, this point will be
crucial.  

In the globular set $1$, all cells are endomorphisms---in other words, the
source and target maps are equal.  It follows that the same is true in
the globular set $\pd$.  We write $\bdry: \pd(m+1) \go \pd(m)$%
% 
\glo{bdry}
% 
instead of
$s$ or $t$, and call $\bdry$ the \demph{boundary}%
%
\index{boundary!pasting diagram@of pasting diagram}
%
operator.  For instance,
the boundary of the 2-pasting diagram~\bref{eq:typical-glob-pd} is the
1-pasting diagram
\[
\gfstsu
\gonesu
\gzersu
\gonesu
\gzersu
\gonesu
\glstsu.
\]

It is easy to describe%
%
\index{pasting diagram!globular!free monoid@via free monoids}
%
the globular set $\pd$ explicitly: writing
$\blank^*$%
% 
\glo{starfreemon}
% 
for the free monoid functor on $\Set$, we have $\pd(0)=1$ and
%
\lbl{p:pd-description}%
$\pd(m+1) = \pd(m)^*$.  That is, an $(m+1)$-pasting diagram is a
sequence of $m$-pasting diagrams.  For example, the $2$-pasting diagram
depicted in~\bref{eq:typical-glob-pd} is the sequence
\[
(\gfstsu\gonesu\gzersu\gonesu\gzersu\gonesu\glstsu, \ 
\glstsu, \ 
\gfstsu\gonesu\glstsu)
\]
of $1$-pasting diagrams, so if we write the unique element of $\pd(0)$ as
$\blob$ then~\bref{eq:typical-glob-pd} is the double sequence
\[
((\blob, \blob, \blob), (), (\blob)) 
\in \pd(2).
\]
The boundary map $\bdry: \pd(m+1) \go \pd(m)$ is defined inductively by
\[
\left(
\pd(m+1) \goby{\bdry} \pd(m)
\right) 
= 
\left(
\pd(m) \goby{\bdry} \pd(m-1)
\right)^*  
\]
($m\geq 1$).  The correctness of this description of $\pd$ follows from the
results of Appendix~\ref{app:free-strict}.

Having described \pd\ as a globular set, we turn to its strict
$\omega$-category structure: how pasting diagrams may be composed.

Typical binary compositions are
% 
\begin{equation}	\label{eq:bin-1-comp}
\left.
\begin{array}{c}
\gfstsu\gthreesu\gzersu\gonesu\gzersu\gtwosu\glstsu	\\
\ofdim{1}	\\
\gfstsu\gonesu\gzersu\gonesu\gzersu\gthreesu\glstsu
\end{array}
\right.
\ =\ 
\gfstsu\gthreesu\gzersu\gonesu\gzersu\gfoursu\glstsu,
\end{equation}
% 
illustrating the composition function 
$
\ofdim{1}: \pd(2) \times_{\pd(1)} \pd(2) \go \pd(2),
$
and
% 
\begin{equation}	\label{eq:bin-0-comp}
\gfst{}\gspecialtwo\glst{}
\ofdim{0}
\gfst{}\gthree{}{}{}{}{}\glst{}
\ =\ 
\gfst{}\gspecialtwo\gfbw{}\gthree{}{}{}{}{}\glst{}\!,
\end{equation}
%
illustrating the composition function $ \ofdim{0}: \pd(3) \times_{\pd(0)}
\pd(3) \go \pd(3).  $ These compositions are possible because the
boundaries match: in~\bref{eq:bin-1-comp}, the 1-dimensional boundaries of
the two pasting diagrams on the left-hand side are equal, and similarly for
the 0-dimensional boundaries~\bref{eq:bin-0-comp}---indeed, this is
inevitable as there is only one 0-pasting diagram.

(The arguments on the left-hand side of~\bref{eq:bin-1-comp} are stacked
vertically rather than horizontally just to make the picture more
compelling; strictly speaking we should have written
\[
\gfstsu\gonesu\gzersu\gonesu\gzersu\gthreesu\glstsu
\ \ofdim{1} \ 
\gfstsu\gthreesu\gzersu\gonesu\gtwosu\glstsu
\]
instead.  The same applies to~\bref{eq:bin-0-comp}.)

A typical nullary%
%
\index{nullary!composite}
%
composition (identity) is
%
\begin{equation}	\label{eq:typical-id}
\gfstsu\gonesu\gzersu\gonesu\gzersu\gonesu\glstsu
\diagspace \goesto \diagspace 	
\gfstsu\gonesu\gzersu\gonesu\gzersu\gonesu\glstsu,
\end{equation}
% 
illustrating the identity function $i: \pd(1) \go \pd(2)$ in the strict
$\omega$-category $\pd$.  (Recall the remarks above on degenerate%
%
\index{pasting diagram!globular!degenerate}
%
pasting
diagrams.)  So the left-hand side of~\bref{eq:typical-id} is a 1-pasting
diagram $\pi\in\pd(1)$, and the right-hand side is 2-pasting diagram $1_\pi
= i(\pi) \in \pd(2)$.  We have $\bdry(1_\pi) = \pi$.%
%
\lbl{p:bdry-degen-pd}

We will need to consider not just binary and nullary composition in \pd,
but composition `indexed' by arbitrary shapes, in the sense now explained.
The first binary composition~\bref{eq:bin-1-comp} above is indexed by the
2-pasting diagram $\gfstsu\gthreesu\glstsu$, in that we were composing one
2-cell with another by joining along their bounding 1-cells.  The
composition can be represented as
% 
\begin{equation}	%\label{pic:bin-1-comp-expanded}
\setlength{\unitlength}{1cm}
\begin{array}{c}
\begin{picture}(10,3.6)%
\put(1,3){\gzersu\gthreesu\gzersu\gonesu\gzersu\gtwosu\gzersu}%
\put(1,0.4){\gzersu\gonesu\gzersu\gonesu\gzersu\gthreesu\gzersu}%
\put(5.5,1.8){\gzersu\gthreesu\gzersu}%
\put(9,1.8){\gzersu\gtwosu\gzersu}%
\cell{5.8}{2.15}{b}{\triangledown}
\cell{5.8}{1.55}{t}{\vartriangle}
\cell{9.3}{2.0}{b}{\triangledown}
\qbezier(4,3.4)(5.8,4.4)(5.8,2.35)
\qbezier(4,0)(5.8,-1)(5.8,1.35)
\qbezier(6.6,2.2)(9.3,4)(9.3,2.2)
\end{picture}
\end{array}
.
% \absentpiccy{pain35b.ps}.
\end{equation}
% 
In general, the ways of composing pasting diagrams are indexed by pasting
diagrams themselves: for instance,
%
\begin{equation}	\label{pic:gen-comp}
\setlength{\unitlength}{1cm}
\begin{array}{c}
\begin{picture}(10,4.2)%
\put(4,2.1){\gzersu\gthreesu\gzersu\gtwosu\gzersu\gonesu\gzersu}%
\put(0,3){\gzersu\gonesu\gzersu\gonesu\gzersu}%
\put(2.5,3.7){\gzersu\gonesu\gzersu\gthreesu\gzersu}%
\put(5,4){\gzersu\gonesu\gzersu\gonesu\gzersu\gonesu\gzersu}%
\put(1,0.5){\gzersu\gtwosu\gzersu\gfoursu\gzersu}%
\put(9,2.1){\gzersu\gtwosu\gzersu}%
\cell{4.3}{2.45}{b}{\triangledown}
\cell{4.3}{1.85}{t}{\vartriangle}
\cell{5.2}{2.3}{b}{\triangledown}
\cell{6.2}{2.2}{b}{\triangledown}
\cell{9.3}{2.3}{b}{\triangledown}
\qbezier(1.9,3.1)(4.3,3.1)(4.3,2.65)
\qbezier(4.4,3.5)(5.2,3.1)(5.2,2.5)
\qbezier(5.8,3.8)(6.2,3.1)(6.2,2.4)
\qbezier(2.9,0.3)(4.3,0.6)(4.3,1.65)
\qbezier(6.9,2.4)(9.3,3.5)(9.3,2.5)
\end{picture}
\end{array}
% \absentpiccy{pain36.ps}
\end{equation}
%
represents the composition
% 
\renewcommand{\qbeziermax}{80}%
\begin{eqnarray*}
&
\left(\begin{array}{c}
	\gfstsu\gonesu\gzersu\gonesu\glstsu	\\
	\ofdim{1}	\\
	\gfstsu\gtwosu\gzersu\gfoursu\glstsu
\end{array}\right)
\ \ofdim{0}\ 
\gfstsu\gonesu\gzersu\gthreesu\glstsu
\ \ofdim{0}\ 
\gfstsu\gonesu\gzersu\gonesu\gzersu\gonesu\glstsu
\\
= &
\gfstsu\gtwosu\gzersu\gfoursu\gzersu\gonesu\gzersu%
\gthreesu\gzersu\gonesu\gzersu\gonesu\gzersu\gonesu\glstsu
\end{eqnarray*}%
\renewcommand{\qbeziermax}{150}%
% 
(with the same pictorial convention on positioning the arguments as
previously).

\paragraph*{}

This describes the free strict $\omega$-category $\pd$ on the terminal
globular set.  Before progressing to free strict $\omega$-categories in
general, let us pause to consider an alternative way of representing
pasting diagrams, due to Batanin~\cite{BatMGC}:%
%
\index{Batanin, Michael!globular operads@on globular operads}
%
as
trees.


First a warning:%
%
\index{tree!different types of}
%
these are not the same trees as appear elsewhere in this
text (\ref{sec:trees},~for instance).  Not only is there a formal
difference, but also the two kinds of trees play very different roles.  The
exact connection remains unclear, but for the purposes of understanding
what is written here they can be regarded as entirely different species.

The idea is that, for instance, the 2-pasting diagram
\[
\gfstsu%
\gfoursu%
\gzersu%
\gonesu%
\gzersu%
\gtwosu%
\glstsu
\]
can be portrayed as the tree 
%
\begin{equation}	\label{diag:Batanin-tree}
\begin{tree}
\node &\node            &\node &      &      &\node \\
      &\rt{1} \dn \lt{1}&      &      &      & \dn  \\
      &\node            &      &\node &      &\node \\
      &                 &\rt{2}&\dn   &\lt{2}&      \\
      &                 &      &\node &      &      \\
\end{tree}
\end{equation}
%
according to the following method.  The pasting diagram is 3 1-cells
long, so the tree begins life as
\[
\begin{tree}
\node	&	&\node	&	&\node	\\
	&\rt{2}	&\dn	&\lt{2}	&	\\
	&	&\node\makebox[0em]{\ \ .}	&	&	\\
\end{tree}
\]
Then the first column is 3 2-cells high, the second 0, and the third 1, so
it grows to~\bref{diag:Batanin-tree}.  Finally, there are no 3-cells so it
stops there.

Formally, an \demph{$m$-stage level%
%
\index{tree!level}
%
tree} ($m\in\nat$) is a
diagram
\[
\tau(m)\go\tau(m-1)\go\cdots\go\tau(1)\go\tau(0)=1
\]
in the skeletal category $\scat{D}$ of finite (possibly empty) totally
ordered sets~(\ref{eg:str-mon-D}); we write $\fcat{lt}(m)$ for the set of
all $m$-stage level trees.  The element of $\fcat{lt}(2)$
in~\bref{diag:Batanin-tree} corresponds to a certain diagram $4\go 3\go 1$
in $\scat{D}$, for example.  (Note that if $\tau$ is an $m$-stage tree with
$\tau(m)=0$ then the height of the picture of $\tau$ will be less than
$m$.)  The \demph{boundary}%
%
\index{boundary!level tree@of level tree}
%
$\bdry\tau$ of an $m$-stage tree $\tau$ is the
$(m-1)$-stage tree obtained by removing all the nodes at height $m$, or
formally, truncating
\[
\tau(m)\go\tau(m-1)\go\cdots\go\tau(1)\go\tau(0)
\]
to
\[
\mbox{\hspace{5.2em}}
\tau(m-1)\go\cdots\go\tau(1)\go\tau(0).
\]
This defines a diagram
%
\begin{equation}	%\label{eq:lt}
\cdots 
\goby{\bdry} 
\fcat{lt}(m) 
\goby{\bdry} 
\fcat{lt}(m-1) 
\goby{\bdry}
\cdots 
\goby{\bdry} 
\fcat{lt}(0)
\end{equation}
%
of sets and functions, hence a globular set $\fcat{lt}$ with $s = t =
\bdry$.    

\begin{propn}
There is an isomorphism of globular sets $\pd \iso \fcat{lt}$.
\end{propn}
%
\begin{proof}
There is one 0-stage tree, and informally it is clear that an $(m+1)$-stage
tree amounts to a finite sequence of $m$-stage trees (placed side by side
and with a root node adjoined).  Formally, take an $(m+1)$-stage tree
$\tau$, write $\tau(1) = \{1, \ldots, r\}$, and for $1\leq i \leq r$ and $0
\leq p \leq m$, let
\[
\tau_i(p)
=
\{ j \in \tau(p+1)
\such
\bdry^p (j) = i \}.
\]
Then we have a finite sequence $(\tau_1, \ldots, \tau_r)$ of $m$-stage
trees and so, inductively, a finite sequence of $m$-pasting diagrams, which
is an $(m+1)$-pasting diagram.  It is easy to check that this defines an
isomorphism.  
\done
\end{proof}

Composition and identities in the strict $\omega$-category $\pd$ can also
be expressed in the pictorial language of trees, in a simple way: see
Batanin~\cite{BatMGC} or Leinster~\cite[Ch.~II]{SHDCT}.

\paragraph*{}

Let us now see what $T$ does to an arbitrary globular set $X$.  An $m$-cell
of $TX$ is a formal pasting-together of cells of $X$ of dimension at most
$m$: for instance, a typical element of $(TX)(2)$ looks like%
%
\index{pasting diagram!globular!labelled}
%
% 
\begin{equation}	\label{eq:labelled-glob-pd}
\gfst{a}%
\gfour{f}{f'}{f''}{f'''}{\alpha}{\alpha'}{\alpha''}%
\grgt{b}%
\gone{g}%
\glft{c}%
\gtwo{h}{h'}{\gamma}%
\glst{d},
% \absentpiccy{pain38.ps}
\end{equation}
% 
where $a,b,c,d \in X(0)$, $f,f',f'',f''',g,h,h' \in X(1)$,
$\alpha,\alpha',\alpha'',\gamma \in X(2)$, and $s(\alpha)=f$, $t(\alpha)=f'$,
and so on. 

We can describe the functor $T$ explicitly; in the terminology of
Appendix~\ref{app:special-cart}, we give a `familial%
%
\index{familial representability!free strict omega-category functor@of free strict $\omega$-category functor}
%
representation'.  

First we associate to each pasting diagram $\pi$ the globular set
$\rep{\pi}$%
% 
\glo{reppi}%
%
\index{pasting diagram!globular!globular set from}
%
% 
that `looks like $\pi$'.  If $\pi$ is the unique $0$-pasting
diagram then
\[
\rep{\pi} 
= 
(
\ 
\cdots
\parpairu
\emptyset 
\parpairu
\emptyset 
\parpairu 
1
).
\]
Inductively, suppose that $m\geq 0$ and $\pi\in\pd(m+1)$: then $\pi =
(\pi_1, \ldots, \pi_r)$ for some $r\in\nat$ and $\pi_1, \ldots, \pi_r \in
\pd(m)$, and we put 
%
\begin{equation}	\label{eqn:rep-constr}
\rep{\pi} = (\ \cdots \parpairu \coprod_{i=1}^{r} \rep{\pi_i}(1) 
		\parpairu \coprod_{i=1}^{r} \rep{\pi_i}(0)
		\parpairu \{0, 1, \ldots, r\} ).
\end{equation}
%
The source and target maps in all but the bottom dimension are the evident
disjoint unions, and in the bottom dimension they are defined at
$x\in\rep{\pi_i}(0)$ by 
\[
s(x) = i-1, \diagspace t(x) = i.
\]
For example, if $\pi$ is the $2$-pasting diagram~\bref{eq:typical-glob-pd}
then $\rep{\pi}$ is of the form
\[
\cdots 
\parpairu \emptyset 
\parpairu 4
\parpairu 7
\parpairu 4
\]
where `4' means a 4-element set, etc.  This reflects the fact that the
picture~\bref{eq:typical-glob-pd} contains 4 0-cells, 7 1-cells, 4 2-cells,
and no higher cells.

The case of degenerate%
%
\index{pasting diagram!globular!degenerate}
%
pasting diagrams deserves attention.  If
$\pi\in\pd(m)$ then $1_\pi \in \pd(m+1)$ is represented by the same picture
as $\pi$; formally, $\rep{1_\pi} = \rep{\pi}$.%
%   
\lbl{p:degen-rep}
% 
In fact, if $\sigma \in \pd(m+1)$ then $\rep{\sigma}(m+1) = \emptyset$ if
and only if $\sigma = 1_{\bdry\sigma}$.

An $m$-cell of $TX$ is meant to be an
`$m$-pasting diagram labelled%
%
\index{pasting diagram!globular!labelled}
%
by cells of $X$', that is, an $m$-pasting
diagram $\pi$ together with a map $\rep{\pi} \go X$ of globular sets, which
suggests that there is an isomorphism
%
\begin{equation}	\label{eq:pd-rep-of-T}
(TX)(m) \iso \coprod_{\pi\in\pd(m)} 
\ftrcat{\scat{G}^\op}{\Set}(\rep{\pi}, X).
\end{equation}
%
This is proved as Proposition~\ref{propn:pds-formula}.  

For each $m$-pasting diagram $\pi$ ($m\geq 1$) there are source%
%
\index{pasting diagram!globular!source and target inclusions}
%
and target
inclusions $\rep{\bdry\pi} \parpairu \rep{\pi}$.%  
%
\lbl{p:rep-source-target}
%
For instance, when $\pi$ is the 2-pasting diagram
of~\bref{eq:typical-glob-pd}, these embed
$\rep{\bdry\pi}$ (a string of 3 1-cells) as the top and bottom edges of
$\rep{\pi}$.  The formal definition is straightforward and left as an
exercise.  Given a globular set $X$, these embeddings induce functions
$(TX)(m) \parpairu (TX)(m-1)$ for each $m$, so that $TX$ becomes a globular
set.  So we now have the desired explicit description of the free strict
$\omega$-category functor $T$.

Finally, $T$ is not just a functor but a monad.  The multiplication turns a
pasting diagram of pasting diagrams of cells of some globular set $X$ into
a single pasting diagram of cells of $X$ by `erasing the joins';
compare~\bref{pic:gen-comp}.  The unit realizes a single cell of $X$ as a
(trivial) pasting diagram of cells of $X$.

We could also try to describe the multiplication and unit explicitly in
terms of the family $(\rep{\pi})_{m\in\nat, \pi\in\pd(m)}$ `representing'
$T$.  This can be done, but takes appreciable effort and seems to be both
very complicated and not especially illuminating; as discussed in the
Notes to Appendix~\ref{app:special-cart}, the full theory of familially
representable monads on presheaf categories is currently beyond us.  But
for the purposes of this chapter, we have all the description of $T$ that
we need. 



\section{Globular operads}
\lbl{sec:glob-opds}

A \demph{globular%
%
\index{globular operad}
%
operad} is a $T$-operad.  The purpose of this section is
to describe globular operads pictorially.  The more general
$T$-multicategories are not mentioned until the end of
Chapter~\ref{ch:other-defns}, and there only briefly.

A globular operad $P$ is a $T$-graph 
%
\begin{equation}	\label{diag:collection}
\begin{slopeydiag}
	&	&P	&	&	\\
	&\ldTo<d&	&\rdTo	&	\\
\pd = T1&	&	&	&1
\end{slopeydiag}
\end{equation}
%
equipped with composition and identity operations satisfying associativity
and identity axioms.  (In a standard abuse of language, we use $P$ to mean
either the whole operad or just the globular set at the apex of the
diagram.)  We consider each part of this description in turn.

A $T$-graph~\bref{diag:collection} whose object-of-objects is $1$ will be
called a \demph{collection}.%
%
\lbl{p:defn-collection}%
%
\index{collection}
%
So a collection is merely a globular set over $\pd$, and consists of a set
$P(\pi)$ for each $m\in\nat$ and $m$-pasting diagram $\pi$, together with a
pair of functions $P(\pi) \parpair{s}{t} P(\bdry\pi)$ (when $m\geq 1$)
satisfying the usual globularity equations.
(Formally,~\ref{propn:pshf-slice} tells us that a presheaf on $\scat{G}$
over $\pd$ is the same thing as a presheaf on the category of elements of
$\pd$, whose objects are pasting diagrams and whose arrows are generated by
those of the form $\bdry\pi \parpair{\sigma}{\tau} \pi$ subject to the
duals of the globularity equations.)

If we were discussing plain%
%
\index{globular operad!plain operad@\vs.\ plain operad}
%
rather than globular operads then $\pd =
T1$ would be replaced by $\nat$, and a collection would be a sequence
$(P(k))_{k\in\nat}$ of sets.  In that context we think of an element
$\theta$ of $P(k)$ as an operation of arity $k$ (even though it does not
actually act on anything before a $P$-algebra is specified) and draw it as
\[
k
\left\{
\mbox{\rule[-4ex]{0em}{8ex}}
\right. 
% \!\!\!\!\!\!
\begin{centredpic}
\begin{picture}(6,4)(-1,-2)
\cell{0}{0}{l}{\tusual{\theta}}
\cell{0}{0}{r}{\tinputsslft{}{}{}}
\cell{4}{0}{l}{\toutputrgt{}}
\end{picture}
\end{centredpic}
.
\]
Similarly, if $P$ is a (globular) collection and $\pi$ an $m$-pasting
diagram for some $m\in\nat$, we think of an element of $P(\pi)$ as an
`operation of arity $\pi$' and draw it as an arrow whose input is (a
picture of) $\pi$ and whose output is a single $m$-cell.  For instance, if
$m=2$ and
\[
\pi = \gfstsu\gthreesu\gzersu\gtwosu\glstsu
\]
then $\theta\in P(\pi)$ is drawn as
\[
\begin{diagram}[width=4em]
\gfstsu\gthreesu\gzersu\gtwosu\glstsu	&
\rGlobopd^\theta			&
\gfstsu\gtwosu\glstsu .			% \makebox[0em]{\ .}\\
\end{diagram}
\]
So $\theta$ is thought of as an operation capable of taking data shaped
like the pasting diagram $\pi$ as input and producing a single 2-cell as
output.  This is a figurative description but, as we shall see, becomes
literal when an algebra for $P$ is present.

Composition%
%
\index{globular operad!composition in}
%
in a globular operad $P$ is a map $\comp: P\of P \go P$ of
collections.  The collection $P\of P$ is the composite down the left-hand
diagonal of the diagram
\[
\begin{slopeydiag}
   &       &   &       &   &       &P\of P\Spbk&  &   \\
   &       &   &       &   &\ldTo  &      &\rdTo  &   \\
   &       &   &       &TP &       &      &       &P, \\
   &       &   &\ldTo<{Td}&&\rdTo<{T!}&   &\ldTo>d&   \\
   &       &T\pd&      &   &       &\pd   &       &   \\
   &\ldTo<{\mu_1}&&    &   &       &      &       &   \\
\pd&       &   &       &   &       &      &       &   \\
\end{slopeydiag}
\]
and a typical element of $(P\of P)(2)$ is depicted as
%
\begin{equation}	\label{pic:compn-in-operad}
\setlength{\unitlength}{1cm}
\begin{array}{c}
\begin{picture}(9.8,5)(-0.8,0)%
\put(3.5,2.5){\gzersu\gthreesu\gzersu\gtwosu\gzersu}%
\put(0,4){\gzersu\gfoursu\gzersu\gonesu\gzersu\gtwosu\gzersu}%
\put(0,1.2){\gzersu\gonesu\gzersu\gonesu\gzersu\gthreesu\gzersu}%
\put(3.5,0.4){\gzersu\gtwosu\gzersu\gtwosu\gzersu}%
\put(8,2.5){\gzersu\gtwosu\glstsu .}%
\cell{3.8}{2.85}{b}{\triangledown}
\cell{3.8}{2.25}{t}{\vartriangle}
\cell{4.6}{2.45}{t}{\vartriangle}
\cell{8.3}{2.7}{b}{\triangledown}
\cell{-0.1}{4}{r}{\pi_{1}=}
\cell{-0.1}{1.2}{r}{\pi_{2}=}
\cell{3.4}{0.4}{r}{\pi_{3}=}
\cell{3.4}{2.5}{r}{\pi=}
\qbezier(2.8,4.1)(3.8,3.8)(3.8,3.05)
\qbezier(2.8,1.2)(3.8,1.4)(3.8,2.05)
\qbezier(4.2,0.8)(4.6,1.6)(4.6,2.25)
\qbezier(5.4,2.7)(8.3,3.5)(8.3,2.9)
\cell{3.9}{3.7}{c}{\theta_1}
\cell{3.7}{1.3}{c}{\theta_2}
\cell{4.7}{1.4}{c}{\theta_3}
\cell{7.3}{3.3}{c}{\theta}
\end{picture}
\end{array}
% 
% \absentpiccy{pain42.ps}.
\end{equation}
% 
Here $\theta_1 \in P(\pi_1)$, $\theta_2 \in P(\pi_2)$, $\theta_3 \in
P(\pi_3)$, $\theta \in P(\pi)$, and it is meant to be implicit that
$\theta_1$, $\theta_2$, and $\theta_3$ match on their sources and targets:
$t\theta_1 = s\theta_2$ and $tt\theta_1 = ss\theta_3$.  The
left-hand half of the diagram (containing the $\theta_i$'s) is an element
of the fibre over $\pi$ of the map $T!: (TP)(2) \go \pd(2)$, and the
right-hand half ($\theta$) is an element of the fibre over $\pi$ of the map
$d: P(2) \go \pd(2)$ (that is, an element of $P(\pi)$), so the whole
diagram is a 2-cell of $P\of P$.  More precisely, it is an element of $(P\of
P)(\pi\of(\pi_1, \pi_2, \pi_3))$, where 
% 
\[
\pi\of(\pi_{1},\pi_{2},\pi_{3}) = 
\gzersu\gfoursu\gzersu\gonesu\gzersu%
\gfoursu\gzersu\gtwosu\gzersu\gtwosu\gzersu
\]
is the composite of $\pi$ with $\pi_1$, $\pi_2$, $\pi_3$ in the
$\omega$-category \pd.  So, the composition function $\comp$ of the
globular operad $P$ sends the data assembled in~\bref{pic:compn-in-operad}
to an element $\theta \of (\theta_1, \theta_2, \theta_3) \in
P(\pi\of(\pi_{1},\pi_{2},\pi_{3}))$, which may be drawn as
\[
\setlength{\unitlength}{1cm}
\begin{picture}(10,3)%
\put(2,1.5){\gzersu\gfoursu\gzersu\gonesu\gzersu%
\gfoursu\gzersu\gtwosu\gzersu\gtwosu\gzersu}%
\put(9,1.5){\gzersu\gtwosu\glstsu.}%
\cell{9.3}{1.7}{b}{\triangledown}%
\cell{1.8}{1.5}{r}{\pi\of(\pi_{1},\pi_{2},\pi_{3})=}
\cell{8.3}{2.1}{br}{\theta\of (\theta_{1},\theta_{2},\theta_{3})}
\qbezier(6.6,1.7)(9.3,2.4)(9.3,1.9)
\end{picture}	
% 
% \absentpiccy{pain43.ps}.
\]

(The `linear' notation $\pi\of(\pi_{1},\pi_{2},\pi_{3})$ and
$\theta\of(\theta_{1},\theta_{2},\theta_{3})$ should not be taken too
seriously: there is evidently no canonical order in which to put the
$\pi_i$'s.)

That $\comp: P\of P \go P$ is a map of globular sets says that composition
is compatible with source and target.  In the example above,
%
\begin{eqnarray*}
s(\theta\of(\theta_{1},\theta_{2},\theta_{3}))	&
=	& 
s\theta \of (s\theta_1, s\theta_3),	\\
% \diagspace
t(\theta\of(\theta_{1},\theta_{2},\theta_{3}))	&
=	& 
t\theta \of (t\theta_2, t\theta_3),
\end{eqnarray*}
%
where the composite 
\[
s\theta \of (s\theta_1, s\theta_3) \in 
P(\gfstsu\gonesu\gzersu\gonesu
\gzersu\gonesu\gzersu\gonesu
\gzersu\gonesu\glstsu)
\]
is as shown:
%
\[
\setlength{\unitlength}{1cm}
\begin{picture}(9.8,3.8)(-1.1,0.4)%
\put(3.5,2.5){\gzersu\gonesu\gzersu\gonesu\gzersu}%
\put(0,3.95){\gzersu\gonesu\gzersu\gonesu\gzersu\gonesu\gzersu}%
\put(2.9,0.5){\gzersu\gonesu\gzersu\gonesu\gzersu}%
\put(7.9,2.5){\gzersu\gonesu\gzersu}%
\cell{3.9}{2.75}{b}{\triangledown}
\cell{4.8}{2.35}{t}{\vartriangle}
\cell{8.3}{2.7}{b}{\triangledown}
\cell{3.35}{2.55}{r}{\bdry\pi=}
\cell{-0.2}{4}{r}{\bdry\pi_1=}
\cell{2.7}{0.55}{r}{\bdry\pi_3=}
\qbezier(2.9,4)(3.8,3.8)(3.9,2.95)
\qbezier(3.9,0.8)(4.8,1.5)(4.8,2.15)
\qbezier(5.4,2.7)(8.3,3.5)(8.3,2.9)
\cell{4.0}{3.7}{c}{s\theta_1}
\cell{4.85}{1.4}{c}{s\theta_3}
\cell{7.2}{3.3}{c}{s\theta}
\end{picture}
\]
and $t\theta \of (t\theta_2, t\theta_3)$ similarly.

To see what identities%
%
\index{globular operad!identities in}
%
in a globular operad $P$ are, let $\iota_{m} \in
\pd(m)$%
% 
\glo{iotasingle}%
%
\index{pasting diagram!globular!unit}
% 
be the $m$-pasting diagram
% 
\lbl{p:looking-single}%
% 
looking like a single $m$-cell.  (Formally, $\iota_0$ is the unique element
of $\pd(0)$ and
\[
\iota_{m+1} = (\iota_m) \in (\pd(m))^* = \pd(m+1),
\]
using the description of \pd\ on p.~\pageref{p:pd-description}).  Then
the map $1 \goby{\eta_1} \pd$ sends the unique $m$-cell of $1$ to
$\iota_m$, so the identities function $\ids: 1 \go P$ consists of an
element $1_m \in P(\iota_m)$%
% 
\glo{idglobopd}
% 
for each $m\in\nat$.  For instance, the
2-dimensional identity operation $1_2$ of $P$ is drawn as 
\[
\begin{diagram}[size=4em]
\gfstsu\gtwosu\glstsu			&
\rGlobopd^{1_2}				&
\gfstsu\gtwosu\glstsu .			\\
\end{diagram}
\]
That $\ids$ is a map of globular sets says that $s(1_m) = 1_{m-1} = t(1_m)$
for all $m\geq 1$.

Finally, the composition and identities in $P$ are required to obey
associativity and identity laws.  Together these say that there is only
one way of composing any `tree' of operations of the operad: for instance,
if
\[
\begin{diagram}[height=1.5em]
\star	&	&	&	&	&	&	\\
	&\rdTo>{\theta_{11}}&&	&	&	&	\\
\star	&\rTo^{\theta_{12}}&\star&&	&	&	\\
	&\ruTo>{\theta_{13}}&&\rdTo(2,3)>{\theta_1}&&&	\\
\star	&	&	&	&	&	&	\\
	&	&	&	&\star	&\rTo^{\theta}&\gfstsu\gtwosu\glstsu\\
	&	&	&\ruTo>{\theta_2}&&	&	\\
	&	&\star	&	&	&	&	\\
\end{diagram}
\]
is a diagram of the same general kind as~(\ref{pic:compn-in-operad}), with
each $\star$ representing a 2-pasting diagram, then
\[
\theta \of 
(\theta_1 \of (\theta_{11},\theta_{12},\theta_{13}),\theta_2) 
=
(\theta\of(\theta_1,\theta_2)) 
\of 
(\theta_{11},\theta_{12},\theta_{13},1_2).
\]

We have now unwound all of the data and axioms for a globular operad.
Although it may seem complicated on first reading, it is summed up simply:
a globular operad is a collection of operations together with a unique
composite for any family of operations that might plausibly be composed.


\section{Algebras for globular operads}
\lbl{sec:glob-algs}%
%
\index{globular operad!algebra for|(}%
%
\index{algebra!globular operad@for globular operad|(}
%

We now confirm what was suggested implicitly in the previous section: that
if $P$ is a globular operad then a $P$-algebra structure on a globular set
$X$ consists of a function
\[
\ovln{\theta}:
\{ \textrm{labellings of } \pi \textrm{ by cells of } X \}
\go
X(m)
\]%
%
\index{pasting diagram!globular!labelled|(}%
%
for each number $m$, $m$-pasting diagram $\pi$, and operation $\theta\in
P(\pi)$, satisfying sensible axioms.

So, fix a globular operad $P$.  According to the general definition, an
algebra for $P$ is an algebra for the monad $T_P$ on the category of
globular sets, which is defined on objects $X \in
\ftrcat{\scat{G}^\op}{\Set}$ by
\[
\begin{slopeydiag}
	&		&T_P X\Spbk	&		&	\\
	&\ldTo		&		&\rdTo		&	\\
TX	&		&		&		&P	\\
	&\rdTo<{T!}	&		&\ldTo>{d}	&	\\
	&		&\pd,		&		&	\\
\end{slopeydiag}
\]
that is, by
\[
(T_P X)(m) \iso
\coprod_{\pi\in\pd(m)}
P(\pi) \times \ftrcat{\scat{G}^\op}{\Set}(\rep{\pi},X).
\]
So a $P$-algebra is a globular set $X$ together with a function
\[
h_\pi: P(\pi) \times \ftrcat{\scat{G}^\op}{\Set}(\rep{\pi},X)
\go
X(m)
\]
for $m\in\nat$ and $\pi\in P(m)$, satisfying axioms.  Writing
$h_\pi(\theta, \dashbk)$ as $\ovln{\theta}$%
% 
\glo{actionglobopd}
% 
and recalling that a map
$\rep{\pi} \go X$ of globular sets is a `labelling of $\pi$ by cells of
$X$', we see that this is exactly the description above.
% 
For example, suppose that 
% 
\begin{equation}	\label{eq:lute}
\pi = 
\gfst{}\gfour{}{}{}{}{}{}{}\grgt{}\gone{}\glst{}
\in \pd(2),
\end{equation}
% 
that $\theta\in P(\pi)$, and that
% 
\begin{equation}	\label{eq:labelled-lute}
\mathbf{a} =
\gfst{a}%
\gfour{f}{f'}{f''}{f'''}{\alpha}{\alpha'}{\alpha''}%
\grgt{b}%
\gone{g}%
\glst{c}
\end{equation}
% 
is a diagram of cells in $X$: then $\ovln{\theta}$ assigns to this diagram a
2-cell $\ovln{\theta}(\mathbf{a})$ of $X$.%
%
\index{pasting diagram!globular!labelled|)}
%

What are the axioms?  First, $h: T_P X \go X$ must be a map of globular
sets, which says that $\ovln{s(\theta)} = s \of \ovln\theta$ and
$\ovln{t(\theta)} = t \of \ovln\theta$.  So in our example,
$\ovln{\theta}(\mathbf{a})$ is a 2-cell of the form
\[
\gfst{d}\gtwo{k}{k'}{}\glst{e}
\]
where 
% 
\begin{eqnarray*}
k	&= 	
	&\ovln{s(\theta)}(\gfsts{a}\gones{f}\gblws{b}\gones{g}\glsts{c}),\\
k'	&= 	
	&\ovln{t(\theta)}(\gfsts{a}\gones{f''}\gblws{b}\gones{g}\glsts{c}),\\
d	&=	&\ovln{ss(\theta)}(\gzeros{a}),			\\
e	&=	&\ovln{tt(\theta)}(\gzeros{c}).
\end{eqnarray*}

Second, $h: T_P X \go X$ must obey the usual axioms for an algebra for a
monad.  These say that composition in the operad is interpreted in the
model (algebra) as ordinary composition of functions, and identities
similarly.

An example for composition:%
%
\index{globular operad!composition in}
%
take 2-pasting diagrams
\[
\begin{array}{c}
\pi_1 = \gfstsu\gfoursu\gzersu\gonesu\glstsu,
\diagspace
\pi_2 = \gfstsu\gthreesu\glstsu,		\\
\pi = \gfstsu\gtwosu\gzersu\gtwosu\glstsu
\end{array}
\]
and write
\[
\pi \of (\pi_1, \pi_2) = 
\gfstsu\gfoursu\gzersu\gonesu\gzersu\gthreesu\glstsu.
\]
Let
\[
\begin{array}{c}
\theta_1 \in P(\pi_1), 
\diagspace
\theta_2 \in P(\pi_2),				\\
\theta \in P(\pi)
\end{array}
\]
be operations of $P$ satisfying $tt(\theta_1) = ss(\theta_2) \in
P(\gzeros{})$.  Let
\[
\begin{array}{c}
\mathbf{a}_1 = 
\gfsts{a}
\gfours{}{}{}{}{\alpha}{\alpha'}{\alpha''}
% \gfours{f}{f'}{f''}{f'''}{\alpha}{\alpha'}{\alpha''}
\grgts{b}
\gones{}
% \gones{g}
\glsts{c},
\diagspace
\mathbf{a}_2 = 
\gfsts{c}
\gthrees{}{}{}{\gamma}{\gamma'}
% \gthrees{h}{h'}{h''}{\gamma}{\gamma'}
\glsts{d},					\\
\mathbf{a} = 
\gfsts{a}
\gfours{}{}{}{}{\alpha}{\alpha'}{\alpha''}
% \gfours{f}{f'}{f''}{f'''}{\alpha}{\alpha'}{\alpha''}
\grgts{b}
\gones{}
% \gones{g}
\glfts{c}
\gthrees{}{}{}{\gamma}{\gamma'}
% \gthrees{h}{h'}{h''}{\gamma}{\gamma'}
\glsts{d}
\end{array}
\]
be diagrams of cells in $X$ (from which the 1-cell labels have been
omitted).  Then there is a composite operation $\theta \of (\theta_1,
\theta_2) \in P(\pi \of (\pi_1, \pi_2))$, and the composition-compatibility
axiom on the algebra $X$ says that
\[
\ovln{\theta \of (\theta_1, \theta_2)} (\mathbf{a}) 
=
\ovln{\theta} 
\left(
\gfst{}
\gtwo{}{}{\!\!\!\!\!\!\!\ovln{\theta_1}(\mathbf{a_1})}
\gblw{}
\gtwo{}{}{\!\!\!\!\!\!\!\ovln{\theta_2}(\mathbf{a_2})}
\glst{}
\right).
\]

An example for identities:%
%
\index{globular operad!identities in}
%
if $\alpha$ is a 2-cell of $X$ then
$\ovln{1_2}(\alpha) = \alpha$.  In general, the $m$-pasting diagram
$\iota_m$ (defined on p.~\pageref{p:looking-single}) satisfies
$\rep{\iota_m} \iso \scat{G}(\dashbk, m)$, and the identity axiom says that
\[
\ovln{1_m}: \ftrcat{\scat{G}^\op}{\Set}(\rep{\iota_m},X) \go X(m)
\]
is the canonical (Yoneda) isomorphism.  

We meet a non-trivial example of a globular operad in the next chapter.
Its algebras are, by definition, the weak $\omega$-categories.  As a
trivial example for now, the terminal globular operad $P$ is characterized
by $P(\pi)$ having exactly one element for each pasting diagram $\pi$, and
for the general reasons given in~\ref{eg:alg-terminal}, a $P$-algebra is
exactly a $T$-algebra, that is, a \emph{strict}%
%
\index{omega-category@$\omega$-category!strict!operad for}
%
$\omega$-category.%
%
\index{globular operad!algebra for|)}%
%
\index{algebra!globular operad@for globular operad|)}
%



\begin{notes}

Globular operads were introduced by Batanin~\cite{BatMGC}.%
%
\index{Batanin, Michael!globular operads@on globular operads}
%
 He studied them
in the wider context of `monoidal%
%
\index{monoidal globular category}
%
globular categories', and considered `operads in \cat{C}' for any monoidal
globular category \cat{C}.  There is a particular monoidal globular
category $\mathit{Span}$ such that operads in $\mathit{Span}$ are exactly
the globular operads of this chapter, which are the only kind of operads we
need in order to define weak $\omega$-categories.

The realization that Batanin's operads in $\mathit{Span}$ are just
$T$-operads, for $T$ the free strict $\omega$-category monad on globular
sets, was first recorded in my paper of~\cite{GOM}, and subsequently
explained in more detail in my~\cite{SHDCT} and~\cite{OHDCT}.

\end{notes}






